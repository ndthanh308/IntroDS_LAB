\documentclass[letterpaper,11pt]{article}

\usepackage[square,numbers]{natbib}

\usepackage[utf8]{inputenc}   
\usepackage[T1]{fontenc}  

\usepackage{pgf,tikz}
\usetikzlibrary{arrows}
\usepackage{multicol}
\usepackage{amsthm}
\usepackage{amsmath}
\usepackage{amssymb}
\usepackage{amsfonts}
\usepackage{stmaryrd}
\usepackage{mathabx}
%\usepackage{mn­sym­bol}
\usepackage{latexsym}
\usepackage{color}
\usepackage{graphics,graphicx,graphpap}
\usepackage{multirow}
\usepackage{rotating}
\usepackage[new]{old-arrows}
\usepackage[all]{xy}
\usepackage[letterpaper]{geometry}
\usepackage{subfigure}
\usepackage{hyperref}

\theoremstyle{plain}

\usepackage{setspace}
\onehalfspacing

\usepackage{tocbibind}

\DeclareMathAlphabet{\mathpzc}{OT1}{pzc}{m}{it}
\newtheorem{theorem}{Theorem}
\newtheorem{prop}[theorem]{Proposition}
\newtheorem{cor}[theorem]{Corollary}
\newtheorem{lem}[theorem]{Lemma}
\newtheorem{conj}[theorem]{Conjecture}
\newtheorem{que}[theorem]{Question}
\newcommand{\aste}[1]{\stackrel{*}{#1}}
\newcommand{\hocolim}{\operatornamewithlimits{\mathrm{hocolim}}}
\newcommand{\colim}{\operatornamewithlimits{\mathrm{colim}}}

\title{Polyhedral joins and graph complexes}
\author{Andrés Carnero Bravo}

\begin{document}
\maketitle
\begin{abstract}
    We give the homotopy type of the suspension of a polyhedral join in terms of the polyhedral smash product for the same
    family of pairs and show that the polyhedral join of pairs $\left(\bigvee\mathbb{S}^0,\emptyset\right)$ 
    over a skeleton of $\Delta^n$ has the homotopy type of some wedge of spheres. We use these results to study 
    the homotopy type of the forest filtration for some lexicographic products of graphs. 
\end{abstract}

\tableofcontents
\section{Introduction}
Given two graphs $G$ and $H$ one can define their lexicographic product as the graph obtained by taking a copy of $H$ for each vertex of $G$ 
and adding all the possible edges between two copies if the corresponding vertices are adjacent in $G$. This construction seems natural 
and one can ask for analogous constructions for simplicial complexes; the polyhedral join is a natural generalization. This construction and 
the lexicographic product of graphs have a connection for some graph complexes. 
Given a graph $G$ and any $d \in \mathbb{N} \cup \{\infty\}$, we associate a simplicial complex
$\mathcal{F}_d(G)$ whose vertices are the same as those of $G$ and where a subset $S$ is a simplex if the induced subgraph
on $S$ is a forest with maximal degree at most $d$ (for $d = \infty$ all degrees are allowed).
In this paper we will prove some general results about polyhedral joins, and calculate the homotopy type of some of these complexes for 
some lexicographic products using the results for polyhedral joins.

%\begin{itemize}
 %   \item We calculate the homotopy type of $\mathcal{F}_0\left(P_n\circ H\right)$ for all $H$ in terms of $\mathcal{F}_0(P_n)$ and 
%$\mathcal{F}_0(G\circ H_1)\simeq \mathcal{F}_0(G\circ H_2)$. We also give examples showing this is not true for
 % $\mathcal{F}_d$ with $d>0$.
  %  \item We prove that $\displaystyle\Sigma \mathcal{F}_0(G\circ H)\simeq\Sigma \mathcal{F}_0(G)\vee\bigvee_{\sigma\in \mathcal{F}_0(G)}\sum\left(\mathcal{F}_0\left(G-\bigcup_{v\in\sigma}N[v]\right)*\mathcal{F}_0(H)^{*|\sigma|}\right)$.
   % \item We calculate the homotopy type of $\mathcal{F}_d\left(K_{1,n}\circ K_r\right)$ for all $d$.
   % \item We calculate the homotopy type of $\Sigma\mathcal{F}_d\left(K_{n_1,\dots,n_k}\circ K_r\right)$ for all $k\geq2$ and all $d$.
%\end{itemize}

%For this, we need the following results about polyhedral joins:
%\begin{itemize}
 %   \item For a family $(\underline{X},\underline{A})$ of pairs of CW-complexes,
  %  $\Sigma\aste{Z}_K(\underline{X},\underline{A})\simeq\hat{Z}_K\left(\underline{\Sigma X},\underline{\Sigma A}\right)$.
   % \item The homotopy type of $\aste{Z}_{\Delta^n}(sk_0\Delta^{r-1},\emptyset)$.
%\end{itemize}



\section{Preliminaries}
The graphs will be simple, no loops nor multiedges. For a graph $G$ its vertex set will be $V(G)$ and $E(G)$ will be its edge set. 
$\Delta(G)$ is maximal degree of the graph. For a vertex $v$, $N_G(v)=\{u\in V(G):\;uv\in E(G)\}$ is its open neighborhood 
and $N_G[v]=N_G(G)\cup\{v\}$ its closed neighborhood, 
we omit the subindex $G$ if there is no risk of confusion.
For $S\subseteq V(G)$, $G[S]$ is the graph induced by the set $S$. A forest is a graph 
without cycles. For all the graph definitions not stated here we follow \citep{graphsanddigraphs}.
The \textit{lexicographic product} of the graphs $G$ and $H$ is the graph $G\circ H$ 
with vertex set $V(G)\times V(H)$ and edge set 
$$\{\{(u,v_1),(u,v_2)\}:\;\{v_1,v_2\}\in E(H)\}\cup\{\{(u_1,v_1),(u_2,v_2)\}:\;\{u_1,u_2\}\in E(G)\}$$

Given a graph $G$ and a non-negative integer $d$, its \textit{$d$-forest complex} is the complex
$$\mathcal{F}_d(G)=\{\sigma\subseteq V(G):\:G[\sigma] \mbox{ is a forest such that }\Delta(G[\sigma])\leq d\};$$
for $d=\infty$ we take 
$$\mathcal{F}_\infty(G)=\{\sigma\subseteq V(G):\:G[\sigma] \mbox{ is a forest}\}.$$
For $d=0$, $\mathcal{F}_0(G)$ is the independence complex of $G$ and for $d=1$ is also called the $2$-independence complex.


All product spaces will be taken with the compactly generated topology. Given a topological space $X$, $X^{\wedge n}$ will be the smash product
of $n$ copies of $X$ and $X^{\ast n}$ will be the join of $n$ copies of $X$.

Taking $\underline{n}=\{1,\dots,n\}$ and $\mathcal{P}_1(\underline{n})=\mathcal{P}(\underline{n})-\{\underline{n}\}$, 
a punctured $n$-cube $\mathcal{X}$ consists of:
\begin{itemize}
    \item a topological space $\mathcal{X}(S)$ for each $S$ in $\mathcal{P}_1(\underline{n})$, and
    \item a continuous function $f_{S\subseteq T}:\mathcal{X}(S)\longrightarrow\mathcal{X}(T)$ for each $S\subseteq T$,
\end{itemize}
such that $f_{S\subseteq S}=1_{\mathcal{X}(S)}$ and for any $R\subseteq S\subseteq T$ the following diagram commutes:
\begin{equation*}
    \xymatrix{
    \mathcal{X}(R) \ar@{->}[r]^{f_{R\subseteq S}} \ar@{->}[dr]_{f_{R\subseteq T}}& \mathcal{X}(S) \ar@{->}[d]^{f_{S\subseteq T}}\\
     & \mathcal{X}(T).
    }
\end{equation*}  
A punctured $n$-cube of interest for a given topological space $X$ is the constant punctured cube $\mathcal{C}_X$, 
where $\mathcal{C}_X(S)=X$ for any set $S$ and all the functions are $1_X$.
The colimit of a punctured $n$-cube is the space
$$\colim(\mathcal{X})=\bigsqcup_{S\in\mathcal{P}_1(\underline{n})}\mathcal{X}(S)/\sim,$$
where $\sim$ is the equivalence relation generated by $f_{S\subseteq T_1}(x_S)\sim f_{S\subseteq T_2}(x_S)$ for $T_1,T_2$ and $S\subseteq T_1,T_2$. From the definition
is clear that $\colim(\mathcal{C}_X)\cong X$ for any $X$.

For any $n\geq1$ and $S$ in $\mathcal{P}_1(\underline{n})$ we take:
$$\Delta(S)=\left\lbrace(t_1,t_2,\dots,t_n)\in \mathbb{R}^n:\;\sum_{i=1}^nt_i=1\mbox{ and }t_i=0\mbox{ for all }i\in S\right\rbrace,$$
and $d_{S\subseteq T}:\Delta(T)\longrightarrow\Delta(S)$ the corresponding inclusion. Now, for a punctured 
$n$-cube $\mathcal{X}$, the homotopy colimit is 
$$\hocolim(\mathcal{X})=\bigsqcup_{S\in\mathcal{P}_1(\underline{n})}\mathcal{X}(S)\times\Delta(S)/\sim,$$
where $(x_S,d_{S\subseteq T}(t))\sim(f_{S\subseteq T}(x_S),t)$. When $n=2$, we will specify a punctured $2$-cube via a diagram
\begin{equation*}
    \xymatrix{
    \mathcal{D}: & X\ar@{<-}[r]^{f} & Z \ar@{->}[r]^{g} & Y,
    }
\end{equation*}
and its homotopy colimit is called the homotopy pushout. 

Given a punctured $n$-cube $\mathcal{X}$ for $n\geq2$ and defining the punctured $(n-1)$-cubes
$\mathcal{X}_1(S)=\mathcal{X}(S)$ and $\mathcal{X}_2(S)=\mathcal{X}(S\cup\{n\})$,
we have that (Lemma 5.7.6 \citep{cubicalhomotopy})
$$\hocolim(\mathcal{X})\cong \hocolim\left(\mathcal{X}\left(\underline{n-1}\right)\longleftarrow \hocolim(\mathcal{X}_1)\longrightarrow \hocolim(\mathcal{X}_2)\right).$$

If for all  $S\subsetneq[n]$ the map 
$$\colim_{T\subsetneq S} X_T\longrightarrow X_S$$
is a cofibration, we call the punctured cube cofibrant.  
If we have CW-complexes $X_1,\dots,X_n$ such that their intersections are subcomplexes and we take the punctured cube given by the 
intersections and the inclusions, then the punctured cube is cofibrant and 
$\hocolim(\mathcal{X})\simeq \colim(\mathcal{X})$ (see Proposition 5.8.25 \citep{cubicalhomotopy}).

In particular combining the last two observations, we see that we can compute the homotopy type of a union of the CW complexes $X,Y,Z$ that intersect in subcomplexes, by means of three homotopy pushouts, as shown in the following diagram whose top and bottom squares, as well as the rightmost vertical square are homotopy pushouts and where $R \simeq X \cup Y \cup Z$:

\begin{equation*}
\xymatrix{
X \cap Y \cap Z \ar@{->}[rr]^{\cong} \ar@{->}[dr] \ar@{->}[dd] & & Y \cap Z \ar@{->}[dd] \ar@{->}[rd] & & \\
 & X \cap Z \ar@{->}[rr]^{\cong} \ar@{->}[dd] & & P \ar@{->}[r] \ar@{->}[dd] & Z \ar@{->}[dd] \\
X \cap Y \ar@{->}[rr] \ar@{->}[dr] & & Y \ar@{->}[dr] &  & \\
  & X \ar@{->}[rr] &  & Q \ar@{->}[r] & R
}
\end{equation*}

\begin{lem}\label{homocolimpegado}
Let $X,Y,Z$ be spaces with maps $f:Z\longrightarrow X$ and $g:Z\longrightarrow Y$ such that both maps are null-homotopic. Then
$$\hocolim\left(\mathcal{S}\right)\simeq X\vee Y\vee \Sigma Z$$
where 
\begin{equation*}
    \xymatrix{
    \mathcal{S}: & Y \ar@{<-}[r]^{g} & Z \ar@{->}[r]^{f} & X
    }
\end{equation*}
\end{lem}

\section{Polyhedral products}
Given a family of pointed pairs of CW-complexes $(\underline{X},\underline{A})=\{(X_i,A_i)\}_{i=1}^n$ and $K$ a simplicial complex on 
$n$ vertices, we define the \textit{polyhedral smash product} determined by $(\underline{X},\underline{A})$ and $K$ as the space
$$\hat{Z}_K(\underline{X},\underline{A})=\hat{D}(\emptyset)\cup\bigcup_{\sigma\in K}\hat{D}(\sigma)$$
with
$$\hat{D}(\sigma)=\bigwedge_{i\in\underline{n}}Y_i,\;\;\;\;\mbox{where }Y_i=\left\lbrace\begin{array}{cc}
    X_i & \mbox{ if }i\in\sigma \\
    A_i & \mbox{ if }i\notin\sigma.
\end{array}\right.$$

\begin{theorem}\label{polysmash}\citep{gitler}
Let $K$ be a complex with $n$ vertices and $(\underline{X},\underline{A})$ a family of pointed pairs of CW-complexes such that 
$A_i\longhookrightarrow X_i$ is null-homotopic. Then 
$$\hat{Z}_K(\underline{X},\underline{A})\simeq\left(K*\hat{D}(\emptyset)\right)\vee\bigvee_{\sigma\in K}lk(\sigma)*\hat{D}(\sigma)$$
\end{theorem}

Given a complex $K$ with vertices $\underline{n}$ and $(\underline{X},\underline{A})=\{(X_1,A_1),\dots,(X_n,A_n)\}$ a family of 
pairs of CW-complexes, we define the \textit{polyhedral join} as the space
$$\aste{Z}_{K}(\underline{X},\underline{A})=\bigcup_{\sigma\in K}J(\sigma)$$
with 
$$J(\sigma)=\bigast_{i\in\underline{n}}Y_i,\;\;\;\;\mbox{where }Y_i=\left\lbrace\begin{array}{cc}
    X_i & \mbox{ if }i\in\sigma \\
    A_i & \mbox{ if }i\notin\sigma
\end{array}\right.$$

There is also a definition of polyhedral product similar to the polyhedral smash product or polyhedral join given above, but using the Cartesian product of spaces. In \citep{gitler} the homotopy type of the suspension of such a polyhedral product is given in terms of the polyhedral smash product; here we show that 
the suspension of a polyhedral join has the same homotopy type as an polyhedral smash product ---this 
is probably known to experts, but we could not find a published reference.
\begin{theorem}\label{teosmashjoinpol}
If $(\underline{X},\underline{A})=\{(X_1,A_1),\dots,(X_n,A_n)\}$ is a family of  pairs of CW-complexes and 
$(\underline{\Sigma X},\underline{\Sigma A})=\{(\Sigma X_1,\Sigma A_1),\dots,(\Sigma X_n,\Sigma A_n)\}$ the family of their suspensions as pointed pairs, then
$$\Sigma\aste{Z}_K(\underline{X},\underline{A})\simeq\hat{Z}_K\left(\underline{\Sigma X},\underline{\Sigma A}\right).$$
\end{theorem}
\begin{proof}
If $\sigma_1,\dots,\sigma_r$ are the maximal simplices of $K$, we take the punctured $r$-cube
$$\mathcal{X}(S)=\bigcap_{i\in S^c}J(\sigma_i)$$
with the inclusions as maps. Then $\aste{Z}_{K}(\underline{X},\underline{A})\simeq \hocolim(\mathcal{X})$ and 
we have a weak-homotopy equivalence $\Sigma \hocolim(\mathcal{X})\simeq \hocolim(\Sigma\mathcal{X})$ 
(see \citep{cubicalhomotopy} Corollary 5.8.10). Now, for any non-empty 
CW-complexes $Z_1,\dots,Z_l$ with base points $z_1,\dots,z_l$
$$\sum\left(\bigast_{i\in \underline{l}}Z_i\right)\cong\left.\left(\sum\left(\bigast_{i\in \underline{l}}Z_i\right)\right)\middle/\sim\right.,$$ 
where $x \sim y$ if $x, y \in \sum\left(\bigcup_{j=1}^lz_j*\left(\bigast_{i\in\underline{l}-\{j\}}Z_i\right)\right)$; that last space 
is contractible 
by the Nerve Theorem, as its nerve is the $(l-1)$-simplex. We take the the punctured $r$-cube
$$\tilde{\mathcal{X}}(S)=\left.\left(\sum\left(\bigcap_{i\in S^c}J(\sigma_i)\right)\right)\middle/\sim\right..$$
Now the quotient maps give us an homotopy equivalence of cubes, therefore 
$\hocolim(\Sigma\mathcal{X})\simeq \hocolim(\tilde{\mathcal{X}})$.

Now we take the punctured $r$-cube $\mathcal{Y}$ given by:
$$\mathcal{Y}(S)=\bigcap_{i\in S^c}\hat{D}(\sigma_i),$$
for $(\underline{\Sigma X},\underline{\Sigma A})$ with the inclusions as maps. 
Therefore $\hat{Z}_K(\underline{\Sigma X},\underline{\Sigma A})\simeq \hocolim(\mathcal{Y})$.

Defining $\displaystyle \rho(S)=\left\lbrace j\notin\bigcap_{i\in S^c}\sigma_i:\;A_j=\emptyset\right\rbrace$, if we take 
$$\tilde{\mathcal{Y}}(S)=\left\lbrace\begin{array}{cc}
\displaystyle\left.\left(\bigwedge_{j\in \rho(S)^c}Y_j\right)\middle/\sim\right.  & \mbox{ if } n-|\rho(S)|=l>0\\
  & \\
\mathbb{S}^0 &  \mbox{ if } \rho(S)=\underline{n},
\end{array}\right.$$
where the quotient is by the contractible subspace $\displaystyle\bigwedge_{j\in \rho(S)^c}\Sigma (*_j)$, we have that
$\hocolim(\mathcal{Y}_0)\simeq \hocolim(\tilde{\mathcal{Y}})$, where $\mathcal{Y}_0(S)$ is $\tilde{\mathcal{Y}}(S)$ without doing the 
quotient. Taking the inclusions of $\mathcal{Y}_0(S)$ in $\mathcal{Y}(S)$, we have that 
$\hocolim(\mathcal{Y}_0)\simeq \hocolim(\mathcal{Y})$.

Now, we take the punctured cube
$$\mathcal{Z}(S)=\left\lbrace\begin{array}{cc}
\left.\left(I^l\times\displaystyle\prod_{i\in \rho(S)^c}B_i\right)\middle/\sim\right.  & \mbox{ if } n-|\rho(S)|=l>0\\
  & \\
\mathbb{S}^0 &  \mbox{ if } \rho(S)=\underline{n}
\end{array}\right.,$$
where the quotient is by the subspace 
$$\partial I^l\times\prod_{i\notin f(S)}B_i\bigcup I^l\times W(B_1,\dots,B_l),$$ 
with $B_i=X_i$ if $\displaystyle i\in\bigcap_{j\in S^c}\sigma_j$ and $B_i=A_i$ if $\displaystyle i\notin\bigcap_{j\in S^c}\sigma_j$, and 
for $S\subseteq T$, the map $f_{S\subseteq T}$ is the inclusion if $\rho(S)=\rho(T)$ and the constant maps to the base point in other case. 
We will see that $\hocolim(\tilde{\mathcal{X}})\simeq \hocolim(\mathcal{Z})\simeq \hocolim(\tilde{\mathcal{Y}})$.

If $\rho(S)\neq\underline{n}$, we take the following composition of quotient maps
\begin{equation*}
    \xymatrix{
    I^l\times\displaystyle\prod_{i\notin \rho(S)}B_i \ar@{->}[r] & \displaystyle\prod_{i\notin \rho(S)}\Sigma B_i \ar@{->}[r] & \displaystyle\bigwedge_{i\notin \rho(S)}\Sigma B_i \ar@{->}[r] & \tilde{\mathcal{Y}}(S)
    }
\end{equation*}
where the first map sends $\left((t_1,\dots,t_l),(x_i)_{i\notin \rho(S)}\right)$ to $\left([t_i,x_i]\right)_{i\notin \rho(S)}$. Therefore 
$\mathcal{Z}(S)\cong\tilde{\mathcal{Y}}(S)$. In other case both spaces are $\mathbb{S}^0$. If $\rho(S)=\rho(T)$ for $S\subset T$, then 
the maps of $\mathcal{Z}$ are inclusions and we have the following commutative diagram
\begin{equation*}
    \xymatrix{
    \mathcal{Z}(S) \ar@{->}[r] \ar@{->}[d] & \mathcal{Z}(T) \ar@{->}[d]\\
    \mathcal{\tilde{Y}}(S) \ar@{->}[r] & \mathcal{\tilde{Y}}(T)
    }
\end{equation*}
In other case, the maps for both cubes are the constant map to the base point. Form this we have that 
$\hocolim(\mathcal{Z})\simeq \hocolim(\tilde{\mathcal{Y}})$.

If $\rho(S)\neq\underline{n}$ and $\rho(S)^c=\{i_1,\dots,i_l\}$ with $i_{j}<i_{j+1}$, we take the following composition of quotient maps
\begin{equation*}
    \resizebox{\textwidth}{!}{\xymatrix{
    I^l\times\displaystyle\prod_{j=1}^lB_{i_j} \ar@{->}[r] & I^{l-1}\times\displaystyle\left(B_{i_1}*\prod_{j=2}^lB_{i_j}\right) \ar@{->}[r] & \cdots \ar@{->}[r] & I\times\displaystyle\bigast_{j=1}^lB_{i_j} \ar@{->}[r]& \displaystyle\sum\bigast_{j=1}^lB_{i_j} \ar@{->}[r] & \tilde{\mathcal{X}}(S)
    }}
\end{equation*}
from which we see that $\mathcal{Z}(S)\cong\tilde{\mathcal{X}}(S)$. Otherwise both spaces are $\mathbb{S}^0$. As before, these maps 
induce a homotopy equivalence and therefore $\hocolim(\tilde{\mathcal{X}})\simeq \hocolim(\mathcal{Z})$.
\end{proof}

From the last theorem and Theorem \ref{polysmash} we get the following corollary:

\begin{cor}
If $(\underline{X},\underline{A})=\{(X_1,A_1),\dots,(X_n,A_n)\}$ is a family of  pairs of CW-complexes such that the inclusion 
$\Sigma A_i\longhookrightarrow \Sigma X_i$ is null-homotopic for all $i$, then 
$$\Sigma\aste{Z}_K(\underline{X},\underline{A})\simeq\Sigma\left(K*\aste{D}(\emptyset)\right)\vee\bigvee_{\sigma\in K}\Sigma lk(\sigma)*\aste{D}(\sigma).$$
\end{cor}

\begin{prop}
For $d\leq n$,
$$\aste{\mathcal{Z}}_{_{sk_d\Delta^n}}\left(\bigvee_{r-1}\mathbb{S}^0,\emptyset\right)\simeq\bigvee_{f_d(r,n)}\mathbb{S}^d,$$
where
$$f_d(r,n)=\sum_{i=0}^{d+1}(-1)^{d+1-i}\binom{n+1}{i}r^i.$$
\end{prop}
\begin{proof}
We will set $X=\displaystyle\bigvee_{r-1}\mathbb{S}^0$. Now, for $d=n$, 
$$\aste{Z}_{_{sk_d\Delta^n}}\left(X,\emptyset\right)=\aste{Z}_{_{\Delta^n}}\left(X,\emptyset\right)=\bigast_{i=1}^{n+1}\left(\bigvee_{r-1}\mathbb{S}^0\right)\simeq\bigvee_{(r-1)^{n+1}}\mathbb{S}^n.$$
We will use induction on $d$ and for each $d$ induction on $n$. For $d=1$, $\aste{Z}_{_{sk_1\Delta^n}}\left(X,\emptyset\right)$
is the complete $(n+1)$-partite graph with $r$ vertices in each partition. Therefore:
$$\aste{\mathcal{Z}}_{_{sk_d\Delta^n}}\left(X,\emptyset\right)\simeq\bigvee_{\binom{n+1}{2}r^2-(n+1)r+1}\mathbb{S}^1.$$
Now, assume it is true for $d-1$ and any $n$ and also for $(d,n-1)$; and consider the case $(d,n)$. By case analysis on the first vertex of $\Delta^n$, we obtain:
$$\aste{\mathcal{Z}}_{_{sk_d\Delta^n}}\left(X,\emptyset\right)=\left[\left(\bigvee_{r-1}\mathbb{S}^0\right)*\aste{\mathcal{Z}}_{_{sk_{d-1}\Delta^{n-1}}}\left(X,\emptyset\right)\right]\bigcup\aste{\mathcal{Z}}_{_{sk_d\Delta^{n-1}}}\left(X,\emptyset\right).$$
Since the intersection of those two subcomplexes is $\aste{\mathcal{Z}}_{_{sk_{d-1}\Delta^{k-1}}}\left(X,\emptyset\right)$, we can conclude that $\aste{\mathcal{Z}}_{_{sk_d\Delta^n}}\left(X,\emptyset\right)$ is homotopy equivalent to the homotopy pushout of 
\begin{equation*}
    \xymatrix{
    \displaystyle\bigvee_{(r-1)f_d(r,n)}\mathbb{S}^d \ar@{<-^)}[r]& \displaystyle\bigvee_{f_d(r,n)}\mathbb{S}^{d-1} \ar@{^(->}[r]& \displaystyle\bigvee_{f_{d+1}(r,n)}\mathbb{S}^d.
    }
\end{equation*}
Both inclusions in that diagram must be null-homotopic, so Lemma \ref{homocolimpegado} applies, and we obtain the desired homotopy type: a wedge of $f_d(r,n)$
copies of $\mathbb{S}^d$, where
$$f_{d}(r,n):=r f_{d-1}(r,n-1)+f_d(r,n-1).$$

Now we need only prove the stated formula for $f_d(r,n)$. For $d=1$ or $d=n$ we know it is true. Assume the formula is true for $d-1$ and any $n$ and also for $(d,n-1)$; and consider the case of $(d,n)$. Now,
$$f_d(r,n)=\sum_{i=0}^{d}(-1)^{d-i}\binom{n}{i}r^{i+1}+\sum_{i=0}^{d+1}(-1)^{d+1-i}\binom{n}{i}r^i$$
Reindexing the first sum from $i=1$ to $d+1$, and using a standard identity for binomial coefficients, we obtain the desired formula.
\end{proof}

\section{Complexes of $G\circ H$}
Remember that the lexicographic product $G\circ H$ is the graph obtained by taking a copy of $H$ for each vertex of $G$ and all the possible 
edges between two copies if the corresponding vertices are adjacent in $G$. First we will see that the independence complex of a 
lexicographic product has the homotopy type of a homotopy colimit.

\begin{prop}
Let $G$ and $H$ be two graphs. Then $\mathcal{F}_0(G\circ H)\simeq\hocolim{\mathcal{X}}$, 
with $\mathcal{X}$ a punctured $n$-cube, where $\mathcal{F}_0(G)$ has $n$ maximal simplices $\sigma_1,\dots,\sigma_n$,
$$\mathcal{X}(S)\cong\mathcal{F}_0(H)^{*n_s}$$
and $n_s=\displaystyle\left|\bigcap_{i\notin S}\sigma_i\right|$
\end{prop}
\begin{proof}
By definition, the maximal simplices of $\mathcal{F}_0(G\circ H)$ are given by taking a maximal simplex $\sigma$ 
of $\mathcal{F}_0(G)$ and for each vertex of $\sigma$ taking a maximal simplex in the corresponding copy of $\mathcal{F}_0(H)$. Fixing 
$\sigma$ and taking all the possible combinations of maximal simplices in the copies of $\mathcal{F}_0(H)$, we get the simplicial 
complex 
$$X_\sigma=\bigast_{u\in \sigma}\mathcal{F}_0(H_u)$$
where $H_u$ is the copy of $H$ corresponding to the vertex $u$. Thus, if $\sigma_1,\dots,\sigma_n$ are the maximal simplices 
of $\mathcal{F}_0(G)$, then 
$$\mathcal{F}_0(G\circ H)=\bigcup_{i=1}^nX_{\sigma_i}.$$
Taking the punctured cube $\mathcal{X}$ given by the intersections we get a cofibrant punctured cube satisfying 
$\mathcal{F}_0(G\circ H)\simeq\hocolim{\mathcal{X}}$.
\end{proof}

Now we will see that for the second factor only the 
homotopy type of its independence complex matters. 
\begin{theorem}
Let $H_1$ and $H_2$ be graphs such that $\mathcal{F}_0(H_1)\simeq \mathcal{F}_0(H_2)$, then $\mathcal{F}_0(G\circ H_1)\simeq \mathcal{F}_0(G\circ H_2)$.
\end{theorem}
\begin{proof}
If $\sigma_1,\dots,\sigma_k$ are the maximal simplices of $\mathcal{F}_0(G)$, taking $G_i=G[\sigma_i]$, $X_i=\mathcal{F}_0(G_i\circ H_1)$ and $Y_i=\mathcal{F}_0(G_i\circ H_2)$, we 
have that $X_i\cong \mathcal{F}_0(H_1)^{*|\sigma_i|}$, $Y_i\cong \mathcal{F}_0(H_2)^{*|\sigma_i|}$. 

From this, $\mathcal{F}_0(G\circ H_1)=X_1\cup\cdots\cup X_k$ and $\mathcal{F}_0(G\circ H_2)=Y_1\cup\cdots\cup Y_k$. We take the 
punctured $k$-cubes 
$$\mathcal{X}(S)=\bigcap_{i\in S^c}X_i \;\;\;\;\mbox{and}\;\;\;\;\mathcal{Y}(S)=\bigcap_{i\in S^c}Y_i$$
with the inclusions as the maps. If $f:\mathcal{F}_0(H_1)\longrightarrow \mathcal{F}_0(H_2)$ is a homotopy equivalence, taking 
$f_S:\mathcal{X}(S)\longrightarrow\mathcal{Y}(S)$ the corresponding induced homotopy equivalence if 
$\displaystyle\bigcap_{i\in S^c}\sigma_i\neq\emptyset$, we have that the collection of maps $\{f_S:S\in\mathcal{P}_1(\underline{k})\}$ is 
an homotopy equivalence between the punctured cubes.
\end{proof}

Now, the homotopy type of $\mathcal{F}_0(G\circ H)$ does depend on finer details of $G$ than just the homotopy type of its independence complex: for example the independence complexes of $P_5$ and $P_6$ have the same 
homotopy type \citep{kozlovdire}  but the ones for the 
corresponding lexicographic products do not have to agree. In \citep{okura} the homotopy type of $\mathcal{F}_0(P_{n}\circ H)$ is given when 
$\mathcal{F}_0(H)$ is homotopy equivalent to a wedge of spheres and in \citep{okurajoinbosq} for any graph $H$, here we will 
determine the homotopy type of $\mathcal{F}_0(P_n\circ H)$ for any graph $H$ and all $n$ with a different proof. 
For this we need the following polynomials:
$$a_0(x,y)=0,\;\;\;\;b_0(x,y)=y,\;\;\;\;c_0(x,y)=2y$$
and for $r\geq1$,
$$a_r(x,y)=xyb_{r-1}(x,y)+(x+xy)a_{r-1}(x,y)+x^{r-1}y,$$
$$b_r(x,y)=xyc_{r-1}(x,y)+(x+xy)b_{r-1}(x,y)+x^ry,$$
$$c_r(x,y)=xya_r(x,y)+(x+xy)c_{r-1}(x,y)+2x^ry.$$
\begin{theorem}\label{teopnh}
For any graph $H$
$$\mathcal{F}_0(P_{n}\circ H)\simeq\left\lbrace\begin{array}{cc}
    \mathcal{F}_0(H) &  \mbox{ if }n=1\\
    \mathcal{F}_0(H)\sqcup \mathcal{F}_0(H) & \mbox{ if } n=2\\
    \displaystyle\Sigma\left(\mathcal{F}_0(H)^{\wedge2}\right)\sqcup \mathcal{F}_0(H)& \mbox{ if } n=3\\
    \displaystyle\mathbb{S}^{r-1}\vee\bigvee_{a_r(x,y)}\bigvee_{a_{ij}}\left(\Sigma^i\mathcal{F}_0(H)^{\wedge j}\right)& \mbox{ if } n=3r\geq6\\
    \displaystyle\bigvee_{b_r(x,y)}\bigvee_{b_{ij}}\left(\Sigma^i\mathcal{F}_0(H)^{\wedge j}\right)& \mbox{ if } n=3r+1\geq4\\
    \displaystyle\mathbb{S}^{r}\vee\bigvee_{c_r(x,y)}\bigvee_{c_{ij}}\left(\Sigma^i\mathcal{F}_0(H)^{\wedge j}\right)& \mbox{ if } n=3r+2\geq5
\end{array}
\right.$$
where $z_{ij}$ is the coefficient of the term $x^iy^j$ of the corresponding polynomial.
\end{theorem}
\begin{proof}
For $n=1,2,3$ the theorem is clear. For $n=4$, $\mathcal{F}_0(P_{4}\circ H)$ is the union of three complexes $X_1,X_2,X_3$ 
isomorphic to $\mathcal{F}_0(H)^{*2}$ corresponding to the edges of $P_4^c$. We compute the homotopy type of the union via pushouts as explained in the preliminaries:
\begin{equation*}
\xymatrix{
\emptyset \ar@{->}[rr]^{\cong} \ar@{->}[dr] \ar@{->}[dd] & & \emptyset \ar@{->}[dd] \ar@{->}[rd] & & \\
 & \mathcal{F}_0(H) \ar@{->}[rr]^{\cong} \ar@{->}[dd] & & \mathcal{F}_0(H) \ar@{->}[r] \ar@{->}[dd] & \mathcal{F}_0(H)^{\ast 2} \ar@{->}[dd] \\
\mathcal{F}_0(H) \ar@{->}[rr] \ar@{->}[dr] & & \mathcal{F}_0(H)^{\ast 2} \ar@{->}[dr] &  & \\
  & \mathcal{F}_0(H)^{\ast 2} \ar@{->}[rr] &  & \displaystyle\bigvee_2\mathcal{F}_0(H)^{\ast 2}\vee\Sigma \mathcal{F}_0(H) \ar@{->}[r] & \displaystyle\bigvee_3\mathcal{F}_0(H)^{\ast 2}\bigvee_2\Sigma \mathcal{F}_0(H)
}
\end{equation*}
For $n=5$, $\mathcal{F}_0(P_{5}\circ H)$ is the union of two complexes $X_1\cong \mathcal{F}_0(H)^{*3}$ and $X_2\cong \mathcal{F}_0(P_{4}\circ H)$, 
where $\displaystyle X_1\cap X_2\cong \bigsqcup_{2}\mathcal{F}_0(H)$. Therefore, by Lemma \ref{homocolimpegado}, 
$$\mathcal{F}_0(P_{5}\circ H)\simeq \mathcal{F}_0(H)^{*3}\vee\bigvee_3\mathcal{F}_0(H)^{*2}\vee\bigvee_4\Sigma \mathcal{F}_0(H)\vee\mathbb{S}^1.$$
For $n=6$, $\mathcal{F}_0(P_{6}\circ H)$ is the union of two complexes $X_1\cong \mathcal{F}_0(H)*\mathcal{F}_0(P_{4}\circ H)$ and 
$X_2\cong \mathcal{F}_0(H)*\mathcal{F}_0(P_{3}\circ H)$, where $X_1\cap X_2\cong \mathcal{F}_0(P_{3}\circ H)$. By Lemma \ref{homocolimpegado},
$$\mathcal{F}_0(P_{6}\circ H)\simeq \mathcal{F}_0(H)*\mathcal{F}_0(P_{4}\circ H)\vee \mathcal{F}_0(H)*\mathcal{F}_0(P_{3}\circ H)\vee\Sigma \mathcal{F}_0(P_{3}\circ H)$$
$$\simeq \bigvee_{4}\mathcal{F}_0(H)^{*3}\vee \mathcal{F}_0(H)^{*2}\vee\bigvee_{3}\Sigma \mathcal{F}_0(H)^{*2}\vee\bigvee_{2}\Sigma \mathcal{F}_0(H)\vee\mathbb{S}^1.$$
For $n\geq7$,
$$\mathcal{F}_0(P_{n}\circ H)\cong\left(\mathcal{F}_0(H)*\mathcal{F}_0(P_{n-2}\circ H)\right)\cup\left(\mathcal{F}_0(H)*\mathcal{F}_0(P_{n-3}\circ H)\right).$$
Therefore $\mathcal{F}_0(P_{n}\circ H)$ has the homotopy type of the homotopy pushout of 
\begin{equation*}
    \xymatrix{
    \mathcal{F}_0(H)*\mathcal{F}_0(P_{n-2}\circ H) \ar@{<-^)}[r] & \mathcal{F}_0(P_{n-3}\circ H) \ar@{^(->}[r] & \mathcal{F}_0(H)* \mathcal{F}_0(P_{n-3}\circ H)
    }
\end{equation*}
and from this, by Lemma \ref{homocolimpegado},
$$\mathcal{F}_0(P_{n}\circ H)\simeq \mathcal{F}_0(H)*\mathcal{F}_0(P_{n-2}\circ H)\vee\Sigma \mathcal{F}_0(P_{n-3}\circ H)\vee \mathcal{F}_0(P_{n-3}\circ H)$$
The rest follows by inductive hypothesis.
\end{proof}

Now we give the generating functions for the polynomials of the last Theorem. Taking
$$F(t)=\sum_{r\geq0}a_r(x,y)t^r,\;\;\;\;G(t)=\sum_{r\geq0}b_r(x,y)t^r,\;\;\;\;H(t)=\sum_{r\geq0}c_r(x,y)t^r,$$
we have that:
\begin{align*}
F(t)&=xytG(t)+(x+xy)tF(t)+\sum_{r\geq1}x^{r-1}yt^r\\
G(t)&=xytH(t)+(x+xy)tG(t)+y+\sum_{r\geq1}x^ryt^r\\
H(t)&=xyF(t)+(x+xy)tH(t)+2y+2\sum_{r\geq1}x^ryt^r\\
\end{align*}
Taking $\displaystyle K(t)=\sum_{r\geq1}x^{r-1}yt^r$ we see that 
$$xK(t)=y\left(\frac{1}{1-xt}-1\right)=\frac{xyt}{1-xt}.$$
Therefore $\displaystyle K(t)=\frac{yt}{1-xt}$ and
\begin{align*}
F(t)&=xytG(t)+(x+xy)tF(t)+\frac{yt}{1-xt}\\
G(t)&=xytH(t)+(x+xy)tG(t)+y+\frac{xyt}{1-xt}\\
H(t)&=xyF(t)+(x+xy)tH(t)+2y+\frac{2xyt}{1-xt}\\
\end{align*}
From this we obtain:
\begin{align*}
F(t)&=\frac{xyt}{1-(x+xy)t}G(t)+\frac{yt}{(1-(x+xy)t)(1-xt)}\\
G(t)&=\frac{xyt}{1-(x+xy)t}H(t)+\frac{y}{(1-(x+xy)t)(1-xt)}\\
H(t)&=\frac{xy}{1-(x+xy)t}F(t)+\frac{2y}{(1-(x+xy)t)(1-xt)}\\
\end{align*}
Solving these equations we arrive at:
$$F(t)=\frac{-(x^2y^3+x^2y^2)t^3+(x^2y^3+x^2-x^2y^2y-2xy^2-xy)t^2+(xy^2+y-xy)t}{(1-xt)\left[(1-(x+xy)t)^3-x^3y^3t^2\right]},$$
and from this the other generating functions can be easily obtained.

We now give a formula for the homotopy type of the suspension of $\mathcal{F}_0$ for any lexicographic product in terms of the 
$\mathcal{F}_0$'s of the factors and induced subgraphs of the first factor. For this, notice that the independence complex of a lexicographic product is a polyhedral join, as has been pointed out in \citep{okurapoljoin}.

\begin{theorem}\label{teosuspf0}
For any graphs $G$ and $H$,
$$\Sigma \mathcal{F}_0(G\circ H)\simeq\Sigma \mathcal{F}_0(G)\vee\bigvee_{\sigma\in \mathcal{F}_0(G)}\sum\left(\mathcal{F}_0\left(G-\bigcup_{v\in\sigma}N[v]\right)*\mathcal{F}_0(H)^{*|\sigma|}\right).$$
\end{theorem}
\begin{proof}
By definition, $\mathcal{F}_0(G\circ H)=\aste{Z}_{\mathcal{F}_0(G)}(\mathcal{F}_0(H),\emptyset)$. Then, by Theorem \ref{teosmashjoinpol}, we have that
$$\Sigma \mathcal{F}_0(G\circ H)\simeq\hat{Z}_{\mathcal{F}_0(G)}(\Sigma \mathcal{F}_0(H),\mathbb{S}^0),$$
and by Theorem \ref{polysmash},
$$\hat{Z}_{\mathcal{F}_0(G)}(\Sigma\mathcal{F}_0(H),\mathbb{S}^0)\simeq\Sigma \mathcal{F}_0(G)\vee\bigvee_{\sigma\in \mathcal{F}_0(G)}\sum\left(\mathcal{F}_0\left(G-\bigcup_{v\in\sigma}N[v]\right)*\mathcal{F}_0(H)^{*|\sigma|}\right)$$
\end{proof}
As an immediate corollary we have the following:
\begin{cor}
For any graph $G$ such that $\mathcal{F}_0(G)$ is connected and any graph $W$,
$$\tilde{H}_q(\mathcal{F}_0(G\circ W))\cong\tilde{H}_q(\mathcal{F}_0(G))\oplus\bigoplus_{\sigma\in \mathcal{F}_0(G)}\tilde{H}_q\left(\mathcal{F}_0\left(G-\bigcup_{v\in\sigma}N[v]\right)*\mathcal{F}_0(W)^{*|\sigma|}\right).$$
\end{cor}

The last theorem gives us an equivalence between the suspensions of two spaces, so it is natural to ask if the formula is true without 
suspending, for some $G$.
For example, for $n=5,6$ is not hard to see that
$$\mathcal{F}_0(C_n\circ H)\simeq\mathcal{F}_0(C_n)\vee\bigvee_{\sigma\in\mathcal{F}_0(C_n)}lk(\sigma)*\mathcal{F}_0(H)^{*|\sigma|}.$$
\begin{que}
Is the above homotopy equivalence valid for other positive integers $n$?
\end{que}
Also, using the same proof of Theorem \ref{teopnh} it can be show that for $n\geq4$, the homotopy type of 
$\mathcal{F}_0(P_n\circ H)$ follows the formula of Theorem \ref{teosuspf0}, but without the suspension. So another question is the following:

\begin{que}
For which graphs $G$, with $\mathcal{F}_0(G)$ connected, is it true that
$$\mathcal{F}_0(G\circ H)\simeq\mathcal{F}_0(G)\vee\bigvee_{\sigma\in \mathcal{F}_0(G)}\left(\mathcal{F}_0\left(G-\bigcup_{v\in\sigma}N[v]\right)*\mathcal{F}_0(H)^{*|\sigma|}\right)$$
for all $H$?
\end{que}

Now for any graph $G$, the graph $K_2\circ G$ is also graph join of two copies of $G$, in \citep{forestfilt} is given a formula 
in general for the join of two graphs for $d\geq1$, here we state this result for the particular case of $K_2\circ G$.

\begin{lem}\citep{forestfilt}\label{lemjoingraf}
Let $G$ be a graph of order $n$. Then:
\begin{enumerate}
    \item $\displaystyle\mathcal{F}_1(K_2\circ G)\simeq\bigvee_2\mathcal{F}_1(G)\vee\bigvee_{n^2-1}\mathbb{S}^1$.
    \item If $\mathcal{F}_0(G)$ is connected, then, for all $d\geq2$,
$$\mathcal{F}_d(K_2\circ G)\simeq\bigvee_{2n-2}\Sigma sk_{_{d-1}}\mathcal{F}_0(G)\vee\bigvee_{(n-1)^2}\mathbb{S}^2\vee\bigvee_2A$$
where $A=\mathcal{F}_d(G)\cup C(sk_{_{d-1}} \mathcal{F}_0(G))$.
\end{enumerate}
\end{lem}

Now we can see that in contrast with $\mathcal{F}_0$, the homotopy type of the second factor does not determine the homotopy type of 
$\mathcal{F}_d(G\circ\underline{\;\;\;\;})$ for $d\geq1$. It is known that $\mathcal{F}_0(P_5)\simeq\mathcal{F}_0(P_6)\simeq\mathbb{S}^1$ 
and $\mathcal{F}_0(P_4)\simeq*$ (see \citep{kozlovdire}), and also that $\mathcal{F}_1(P_5)\simeq\mathcal{F}_1(P_6)\simeq*$ (see \citep{salvetti2018}),
and it is not hard to see that $sk_1(\mathcal{F}_0(P_5))\simeq\mathbb{S}^1\vee\mathbb{S}^1$ and $sk_1(\mathcal{F}_0(P_4))\simeq*$. From 
all this and Lemma \ref{lemjoingraf} we have that
$$\mathcal{F}_1(K_2\circ P_5)\simeq\bigvee_{24}\mathbb{S}^1\not\simeq\bigvee_{35}\mathbb{S}^1\simeq\mathcal{F}_1(K_2\circ P_6),$$
$$\mathcal{F}_2(K_2\circ P_4)\simeq\bigvee_{9}\mathbb{S}^2\not\simeq\bigvee_{36}\mathbb{S}^2\simeq\mathcal{F}_2(K_2\circ P_5),$$
and for $d\geq3$,
$$\mathcal{F}_d(K_2\circ P_4)\simeq\bigvee_{9}\mathbb{S}^2\not\simeq\bigvee_{26}\mathbb{S}^2\simeq\mathcal{F}_d(K_2\circ P_5).$$

Until now we only have worked with $\mathcal{F}_0$; for $d\geq1$, sadly $\mathcal{F}_d(G\circ H)$ is not a polyhedral join but 
$\aste{Z}_{\mathcal{F}_d(G)}\left(\Delta^{V(H)},\emptyset\right)$ is a subcomplex. Now, for $H=K_n$ we will make calculations for some 
graphs $G$.

\begin{prop}
For any $r$ and $n$,
$$\mathcal{F}_1\left(K_{1,n}\circ K_r\right)\simeq\bigvee_{\binom{r-1}{2}^n}\mathbb{S}^{2n-1}\vee\bigvee_{(nr^2-1)+\binom{r-1}{2}}\mathbb{S}^1;$$
for $2\leq d\leq n-1$,
$$\mathcal{F}_d\left(K_{1,n}\circ K_r\right)\simeq\bigvee_{\binom{r-1}{2}^n}\mathbb{S}^{2n-1}\vee\bigvee_{rf_{_{d-1}}(r,n-1)}\mathbb{S}^d\vee\bigvee_{\binom{r}{2}}\mathbb{S}^1;$$
and for $d=\infty$,
$$\mathcal{F}_\infty\left(K_{1,n}\circ K_r\right)\simeq\bigvee_{\binom{r-1}{2}^n}\mathbb{S}^{2n-1}\vee\bigvee_{r(r-1)^n}\mathbb{S}^n\vee\bigvee_{\binom{r}{2}}\mathbb{S}^1.$$
\end{prop}
\begin{proof}
For $d=1$ the result follows from Lema \ref{lemjoingraf}.

We take $0,1,\dots,n$ as the vertices of $K_{1,n}$ with $0$ the vertex of degree $n$ and $K_r^i$ the copy of $K_r$ corresponding to the vertex $i$.

For $2\leq d\leq n-1$, $\mathcal{F}_d\left(K_{1,n}\circ K_r\right)=X\cup Y\cup Z$ where
$$X=V\left(K_r^0\right)*\aste{\mathcal{Z}}_{_{sk_{d-1}\Delta^{n-1}}}(sk_0K_r,\emptyset),
\;\;\;Y=\mathcal{F}_d\left(K_r^0\right)\simeq\bigvee_{\binom{r-1}{2}}\mathbb{S}^1,$$
$$Z=\bigast_{i=1}^n\mathcal{F}_d(K_r^i)\simeq\bigvee_{\binom{r-1}{2}^n}\mathbb{S}^{2n-1}.$$
We have that $Y\cap Z=\emptyset$, $X\cap Y=sk_0Y$ and 
$$X\cap Z=\aste{\mathcal{Z}}_{_{sk_{d-1}\Delta^{n-1}}}(sk_0K_r,\emptyset)\simeq\bigvee_{f_{_{d-1}}(r,n-1)}\mathbb{S}^{d-1}$$
Once again we compute the homotopy type of union via homotopy pushouts as explained at the end of the preliminaries:
\begin{equation*}
\xymatrix{
\emptyset \ar@{->}[rr]^{\cong} \ar@{->}[dr] \ar@{->}[dd] & & \emptyset \ar@{->}[dd] \ar@{->}[rd] & & \\
 & \displaystyle\bigvee_{r-1}\mathbb{S}^0 \ar@{->}[rr]^{\simeq} \ar@{->}[dd] & & \displaystyle\bigvee_{r-1}\mathbb{S}^0 \ar@{->}[r] \ar@{->}[dd] & \displaystyle\bigvee_{\binom{r-1}{2}}\mathbb{S}^1 \ar@{->}[dd] \\
\displaystyle\bigvee_{f_{_{d-1}}(r,n-1)}\mathbb{S}^{d-1} \ar@{->}[rr] \ar@{->}[dr] & & \displaystyle\bigvee_{\binom{r-1}{2}^n}\mathbb{S}^{2n-1} \ar@{->}[dr] &  & \\
  & \displaystyle\bigvee_{(r-1)f_{_{d-1}}(r,n-1)}\mathbb{S}^{d} \ar@{->}[rr] &  & \hocolim(\mathcal{S}') \ar@{->}[r] & \hocolim(\mathcal{S})
}
\end{equation*}
where $\mathcal{S}'$ is the diagram of the bottom of the cube. Then
$$\hocolim(\mathcal{S}')\simeq\bigvee_{\binom{r-1}{2}^n}\mathbb{S}^{2n-1}\vee\bigvee_{rf_{_{d-1}}(r,n-1)}\mathbb{S}^{d}$$
and the rest follows from this.

Now, for $d=\infty$, $\mathcal{F}_\infty\left(K_{1,n}\circ K_r\right)=X\cup Y\cup Z$ where $Y$ and $Z$ are as before, and
$$X=\aste{\mathcal{Z}}_{_{\Delta^n}}(sk_0K_r,\emptyset)\simeq\bigvee_{(r-1)^{n+1}}\mathbb{S}^n.$$
As before, $Y\cap Z=\emptyset$, $X\cap Y=sk_0Y$ and 
$$X\cap Z=\aste{\mathcal{Z}}_{_{\Delta^{n-1}}}(sk_0K_r,\emptyset)\simeq\bigvee_{(r-1)^n}\mathbb{S}^{n-1}.$$
Again we use the technique we've been using to compute the homotopy type of the union via homotopy pushouts:
\begin{equation*}
\xymatrix{
\emptyset \ar@{->}[rr]^{\cong} \ar@{->}[dr] \ar@{->}[dd] & & \emptyset \ar@{->}[dd] \ar@{->}[rd] & & \\
 & \displaystyle\bigvee_{r-1}\mathbb{S}^0 \ar@{->}[rr]^{\simeq} \ar@{->}[dd] & & \displaystyle\bigvee_{r-1}\mathbb{S}^0 \ar@{->}[r] \ar@{->}[dd] & \displaystyle\bigvee_{\binom{r-1}{2}}\mathbb{S}^1 \ar@{->}[dd] \\
\displaystyle\bigvee_{(r-1)^n}\mathbb{S}^{n-1} \ar@{->}[rr] \ar@{->}[dr] & & \displaystyle\bigvee_{\binom{r-1}{2}^n}\mathbb{S}^{2n-1} \ar@{->}[dr] &  & \\
  & \displaystyle\bigvee_{(r-1)^{n+1}}\mathbb{S}^n \ar@{->}[rr] &  & \hocolim(\mathcal{S}') \ar@{->}[r] & \hocolim(\mathcal{S})
}
\end{equation*}
where $\mathcal{S}'$ again is the diagram of the bottom of the cube. Then 
$$\hocolim(\mathcal{S}')\simeq\bigvee_{\binom{r-1}{2}^n}\mathbb{S}^{2n-1}\vee\bigvee_{r(r-1)^n}\mathbb{S}^n.$$
The result follows.
\end{proof}

\begin{prop}
For any integers $n,m,r\geq2$,
%$$\mathcal{F}_\infty(K_{n,m}\circ K_r)\simeq\bigvee_{\binom{r-1}{2}^n}\mathbb{S}^{2n-1}\vee\bigvee_{\binom{r-1}{2}^m}\mathbb{S}^{2m-1}\vee\bigvee_{(r-1)^n}\mathbb{S}^n\vee\bigvee_{(r-1)^m}\mathbb{S}^m\vee\aste{\mathcal{Z}}_{_{\mathcal{F}_\infty(K_{n,m})}}\left(sk_0K_r,\emptyset\right)$$
%and
$$\Sigma\mathcal{F}_\infty(K_{n,m}\circ K_r)\simeq\bigvee_{\binom{r-1}{2}^n}\mathbb{S}^{2n}\vee\bigvee_{\binom{r-1}{2}^m}\mathbb{S}^{2m}\vee\bigvee_{a}\mathbb{S}^{n+1}\vee\bigvee_{b}\mathbb{S}^{m+1}\vee\bigvee_{c}\mathbb{S}^3,$$
where $a=m(r-1)^n+m^2(r-1)^m$, $b=n(r-1)^m+n^2(r-1)^n$ and $c=(rn-1)(rm-1)$.
\end{prop}
\begin{proof}
Assume $U=\{u_1,\dots,u_n\}$ and $V=\{v_1,\dots,v_m\}$ are the partition of the vertices of $K_{n,m}$. Taking
$$X=\aste{\mathcal{Z}}_{_{\mathcal{F}_\infty(K_{n,m})}}\left(sk_0K_r,\emptyset\right),\;Y=\aste{\mathcal{Z}}_{_{\Delta^U}}\left(K_r,\emptyset\right),\;W=\aste{\mathcal{Z}}_{_{\Delta^V}}\left(K_r,\emptyset\right),$$
we have that $\mathcal{F}_\infty(K_{n,m}\circ K_r)=X\cup Y\cup W$. Now, $Y\cap W=X\cap Y\cap W=\emptyset$ and
$$X\cap Y=\aste{\mathcal{Z}}_{_{\Delta^U}}\left(sk_0K_r,\emptyset\right), \;X\cap W=\aste{\mathcal{Z}}_{_{\Delta^V}}\left(sk_0K_r,\emptyset\right).$$
Taking any vertices $u_i\in U$ and $v_j\in V$, we can factor the inclusions to $X$ as
\begin{equation*}
    \xymatrix{
    X\cap Y \ar@{^(->}[r] & \aste{\mathcal{Z}}_{_{\Delta^{U\cup\{v_j\}}}}\left(V(K_r),\emptyset\right) \ar@{^(->}[r] & X\\
    X\cap W \ar@{^(->}[r] & \aste{\mathcal{Z}}_{_{\Delta^{V\cup\{u_i\}}}}\left(V(K_r),\emptyset\right) \ar@{^(->}[r] & X
    }
\end{equation*}
where the first inclusions are null-homotopic. Therefore 
$$\hocolim\left(X\cup Y \longhookleftarrow X\cap W\longhookrightarrow W\right)\simeq X\cup Y\vee W\vee\Sigma(X\cap W)$$
and 
$$\hocolim\left(X\longhookleftarrow X\cap Y\longhookrightarrow Y\right)\simeq X\vee Y\vee\Sigma(X\cap Y)$$
From where we obtain that 
$$\mathcal{F}_\infty(K_{n,m}\circ K_r)\simeq X\vee Y\vee W\vee\Sigma(X\cap W)\vee\Sigma(X\cap Y)$$
Now, for $\Sigma\mathcal{F}_\infty(K_{n,m}\circ K_r)$ we only need to determine the homotopy type of 
$\Sigma\aste{\mathcal{Z}}_{_{\mathcal{F}_\infty(K_{n,m})}}\left(sk_0K_r,\emptyset\right)$. Now, 
$$\Sigma\aste{\mathcal{Z}}_{_{\mathcal{F}_\infty(K_{n,m})}}\left(sk_0K_r,\emptyset\right)\simeq\hat{\mathcal{Z}}_{_{\mathcal{F}_\infty(K_{n,m})}}\left(\bigvee_{r-1}\mathbb{S}^1,\mathbb{S}^0\right)$$
Because the inclusion of $\mathbb{S}^0$ in a wedge of copies of $\mathbb{S}^1$ is null-homotopic, we have that 
$$\hat{\mathcal{Z}}_{_{\mathcal{F}_\infty(K_{n,m})}}\left(\bigvee_{r-1}\mathbb{S}^1,\mathbb{S}^0\right)\simeq\Sigma\mathcal{F}_\infty(K_{n,m})\vee\bigvee_{\sigma\in\mathcal{F}_\infty(K_{n,m})}lk(\sigma)*\hat{D}(\sigma).$$
We have $\displaystyle\hat{D}=\bigwedge_{|\sigma|}\bigvee_{r-1}\mathbb{S}^1\simeq\bigvee_{(r-1)^{|\sigma|}}\mathbb{S}^{|\sigma|}$, and thus
$$\hat{\mathcal{Z}}_{_{\mathcal{F}_\infty(K_{n,m})}}\left(\bigvee_{r-1}\mathbb{S}^1,\mathbb{S}^0\right)\simeq\Sigma\mathcal{F}_\infty(K_{n,m})\vee\bigvee_{\sigma\in\mathcal{F}_\infty(K_{n,m})}\left(\bigvee_{(r-1)^{|\sigma|}}\Sigma^{|\sigma|+1}lk(\sigma)\right).$$
If we take any two vertices from $U$ and any two from $V$, we get a cycle. Therefore $|\sigma\cap U|\leq1$ or $|\sigma\cap V|\leq1$ for 
any simplex $\sigma$. Take $\sigma\in\mathcal{F}_\infty(K_{n,m})$. There are two possibilities:
\begin{itemize}
    \item $\sigma$ is totally contained in $U$ or $V$. Assume $\sigma\subseteq U$. There are two cases:
    \begin{enumerate}
        \item If $|\sigma|=1$, then 
        $$lk(\sigma)=\left(sk_0\Delta^V*\Delta^{U-\sigma}\right)\cup\Delta^V$$
        and
        $$lk(\sigma)\simeq \hocolim\left(*\longleftarrow sk_0\Delta^V\longrightarrow*\right)\simeq\bigvee_{m-1}\mathbb{S}^1.$$
        \item If $|\sigma|>1$, then
        $$lk(\sigma)\cong sk_0\Delta^V*\Delta^{U-\sigma}\simeq\left\lbrace\begin{array}{cc}
            * &  \mbox{ if }|\sigma|<n\\
            \displaystyle\bigvee_{m-1}\mathbb{S}^0 &   \mbox{ if }|\sigma|=n.
        \end{array}\right.$$
    \end{enumerate}
    \item $\sigma\cap U\neq\emptyset\neq\sigma\cap V$. Assume $|\sigma\cap U|=1$. There are three cases:
    \begin{enumerate}
        \item If $2=|\sigma|$, then $lk(\sigma)=\Delta^{V-\sigma}\sqcup\Delta^{U-\sigma}$ and thus is homotopy equivalent to 
        $\mathbb{S}^0$.
        \item If $2<|\sigma|<m+1$, then $lk(\sigma)=\Delta^{V-\sigma}$ and therefore is contractible.
        \item If $|\sigma|=m+1$, then $\sigma$ is a maximal simplex and $lk(\sigma)=\emptyset$.
    \end{enumerate}
\end{itemize}
Therefore:
$$\hat{\mathcal{Z}}_{_{\mathcal{F}_\infty(K_{n,m})}}\left(\bigvee_{r-1}\mathbb{S}^1,\mathbb{S}^0\right)\simeq\bigvee_{c}\mathbb{S}^3\vee\bigvee_{a'}\mathbb{S}^{n+1}\vee\bigvee_{b'}\mathbb{S}^{m+1},$$
where $a'=(m-1)(r-1)^{n}+m(r-1)^{n+1}$, $b'=(n-1)(r-1)^{m}+n(r-1)^{m+1}$ and $c=nm(r-1)^2+n(m-1)(r-1)+m(n-1)(r-1)+(n-1)(m-1)=(rn-1)(rm-1)$.
\end{proof}

\begin{theorem}
For any positive integers $r,n_1,\dots,n_k\geq2$, with $k\geq3$ and $G=K_{n_1,\dots,n_k}\circ K_r$ we have that
%$$\mathcal{F}_\infty(G)\simeq\bigvee_{i=1}^k\left(\bigvee_{\binom{r-1}{2}^{n_i}}\mathbb{S}^{2n_i-1}\vee\bigvee_{(r-1)^{n_i}}\mathbb{S}^{n_i}\right)\vee\aste{\mathcal{Z}}_{_{\mathcal{F}_\infty(K_{n_1,\dots,n_k})}}\left(sk_0K_r,\emptyset\right)$$
%and
$$\Sigma\mathcal{F}_\infty(G)\simeq\bigvee_{i=1}^k\left(\bigvee_{\binom{r-1}{2}^{n_i}}\mathbb{S}^{2n_i}\vee\bigvee_{a_i}\mathbb{S}^{n_i+1}\right)\vee\bigvee_b\mathbb{S}^3\vee\bigvee_{\binom{k-1}{2}}\mathbb{S}^2$$
where 
$$a_i=(r-1)^{n_i}+(t_i+1)(r-1)^{n_i+1}+t_i(r-1)^{n_i},$$
$$b=\sum_{i<j}(n_i-1)(n_j-1)+\sum_{i<j}n_in_j(r-1)^2+\sum_{i=1}^kt_in_i(r-1), \text{ and}$$
$$t_i=\sum_{j\neq i}n_j\;-1.$$
\end{theorem}
\begin{proof}
$$\mathcal{F}_\infty(G)=\aste{\mathcal{Z}}_{_{\mathcal{F}_\infty(K_{n_1,\dots,n_k})}}\left(sk_0K_r,\emptyset\right)\cup\bigsqcup_{i=1}^k\aste{\mathcal{Z}}_{_{\Delta^{V_i}}}\left(K_r,\emptyset\right)$$
For all $i$, 
$$\aste{\mathcal{Z}}_{_{\mathcal{F}_\infty(K_{n_1,\dots,n_k})}}\left(sk_0K_r,\emptyset\right)\cap \aste{\mathcal{Z}}_{_{\Delta^{V_i}}}\left(K_r,\emptyset\right)=\aste{\mathcal{Z}}_{_{\Delta^{V_i}}}\left(sk_0K_r,\emptyset\right).$$
As in the proposition before, we take $v\in V_j$ with $j\neq i$. Then the inclusion factors as
$$\aste{\mathcal{Z}}_{_{\Delta^{V_i}}}\left(sk_0K_r,\emptyset\right)\longhookrightarrow \aste{\mathcal{Z}}_{_{\Delta^{V_i\cup\{v\}}}}\left(sk_0K_r,\emptyset\right)\longhookrightarrow\aste{\mathcal{Z}}_{_{\mathcal{F}_\infty(K_{n_1,\dots,n_k})}}\left(sk_0K_r,\emptyset\right),$$
and is thus null-homotopic. Therefore,
$$\mathcal{F}_\infty(G)\simeq\aste{\mathcal{Z}}_{_{\mathcal{F}_\infty(K_{n_1,\dots,n_k})}}\left(sk_0K_r,\emptyset\right)\vee\bigvee_{i=1}^k\left(\aste{\mathcal{Z}}_{_{\Delta^{V_i}}}\left(K_r,\emptyset\right)\vee\Sigma\aste{\mathcal{Z}}_{_{\Delta^{V_i}}}\left(sk_0K_r,\emptyset\right)\right).$$
For the suspension, as in the last proposition, we have that
$$\Sigma\aste{\mathcal{Z}}_{_{\mathcal{F}_\infty(K_{n_1,\dots,n_k})}}\left(sk_0K_r,\emptyset\right)\simeq\Sigma\mathcal{F}_\infty(K_{n_1,\dots,n_k})\vee\bigvee_{\sigma\in\mathcal{F}_\infty(K_{n_1,\dots,n_k})}lk(\sigma)*\aste{D}(\sigma)$$
$$\simeq\Sigma\mathcal{F}_\infty(K_{n_1,\dots,n_k})\vee\bigvee_{\sigma\in\mathcal{F}_\infty(K_{n_1,\dots,n_k})}\bigvee_{(r-1)^{|\sigma|}}\Sigma^{|\sigma|+1}lk(\sigma).$$
Now, 
$$\Sigma\mathcal{F}_\infty(K_{n_1,\dots,n_k})\simeq\bigvee_{\binom{k-1}{2}}\mathbb{S}^2\vee\bigvee_{i<j}\Sigma\mathcal{F}_\infty(K_{n_i,n_j}).  $$
Because any three vertices $v_i\in V_i$, $v_j\in V_j$ and $v_l\in V_l$, with $i<j<l$, form a cycle, the simplices with vertices in 
two $V_i,V_j$, with $i\neq j$, have the link as in the corresponding bipartite graph. Therefore, if $\sigma$ is a simplex such that
$\sigma\cap V_i\neq\emptyset\neq\sigma\cap V_j$, for $i\neq j$, and $|\sigma\cap V_i|=1$, there are three cases:
    \begin{enumerate}
        \item If $2=|\sigma|$, then $lk(\sigma)=\Delta^{V_j-\sigma}\sqcup\Delta^{V_j-\sigma}$ and thus is homotopy equivalent to 
        $\mathbb{S}^0$.
        \item If $2<|\sigma|<n_j+1$, then $lk(\sigma)=\Delta^{V_j-\sigma}$ and therefore is contractible.
        \item If $|\sigma|=n_j+1$, then $\sigma$ is a maximal simplex and $lk(\sigma)=\emptyset$.
    \end{enumerate}
Now if $\sigma$ is a simplex such that $\sigma\subseteq V_i$, then 
\begin{enumerate}
        \item If $|\sigma|=1$, then
        $$lk(\sigma)\cong\left(\left(\bigsqcup_{j\neq i}sk_0\Delta^{V_j}\right)*\Delta^{V_i-\sigma}\right)\cup\left(\bigsqcup_{j\neq i}\Delta^{V_j}\right)$$
        and
        $$lk(\sigma)\simeq \hocolim\left(*\longleftarrow \bigsqcup_{j\neq i}sk_0\Delta^{V_j} \longrightarrow*\right)\simeq\bigvee_{t_i}\mathbb{S}^1.$$
        \item If $|\sigma|>1$, then
        $$lk(\sigma)\cong\left(\bigsqcup_{j\neq i}sk_0\Delta^{V_j}\right)*\Delta^{V_i-\sigma}\simeq\left\lbrace\begin{array}{cc}
            * &  \mbox{ if }|\sigma|<n\\
            \displaystyle\bigvee_{t_i}\mathbb{S}^0 &   \mbox{ if }|\sigma|=n_i
        \end{array}\right.$$
    \end{enumerate}
\end{proof}

\begin{theorem}
For $1\leq d\leq\min\{n-1,m-1\}$, 
$$\Sigma\mathcal{F}_d(K_{n,m}\circ K_r)\simeq\bigvee_{a_d}\mathbb{S}^2\vee\bigvee_{b_d}\mathbb{S}^3\vee\bigvee_{c_d}\mathbb{S}^{d+1}\vee\bigvee_{\binom{r-1}{2}^n}\mathbb{S}^{2n}\vee\bigvee_{\binom{r-1}{2}^m}\mathbb{S}^{2m},$$
where $a_1=r^2nm-1$, $b_1=c_1=0$ and, for $d\geq2$, $a_d=(n+m)(r-1)$, $b_d=nm(r-1)^2+(m-1)(n-1)$, and
$$c_d=n\binom{m-1}{d}+\left[m\binom{n-1}{d}+n\binom{m}{d}+m\binom{n}{d}\right](r-1)^{d+1}+mn\left[\binom{n-2}{d-1}+\binom{m-2}{d-1}\right](r-1)^2$$
$$+\sum_{i=2}^d\left[m\binom{n}{i}\binom{n-i-1}{d-i}+n\binom{m}{i}\binom{m-i-1}{d-i}\right](r-1)^i$$
$$+\sum_{i=1}^d\left[m\binom{n}{i-1}\binom{n-i}{d-i}+n\binom{m}{i-1}\binom{m-i}{d-i}\right](r-1)^i$$
\end{theorem}
\begin{proof}
Assume $U=\{u_1,\dots,u_n\}$ and $V=\{v_1,\dots,v_m\}$ are the partition of the vertices of $K_{n,m}$. Taking
$$X=\aste{\mathcal{Z}}_{_{\mathcal{F}_d(K_{n,m})}}\left(sk_0K_r,\emptyset\right),\;Y=\aste{\mathcal{Z}}_{_{\Delta^U}}\left(K_r,\emptyset\right),\;W=\aste{\mathcal{Z}}_{_{\Delta^V}}\left(K_r,\emptyset\right),$$
we have that $\mathcal{F}_d(K_{n,m}\circ K_r)=X\cup Y\cup W$. Now, $Y\cap W=X\cap Y\cap W=\emptyset$ and
$$X\cap Y=\aste{\mathcal{Z}}_{_{\Delta^U}}\left(sk_0K_r,\emptyset\right), \;X\cap W=\aste{\mathcal{Z}}_{_{\Delta^V}}\left(V(K_r),\emptyset\right).$$
The inclusions $X\cap Y\longhookrightarrow Y$ and $X\cap Z\longhookrightarrow Z$ are null-homotopic, therefore 
$$\mathcal{F}_d(K_{n,m}\circ K_r)\simeq\bigvee_{\binom{r-1}{2}^n}\mathbb{S}^{2n-1}\vee\bigvee_{\binom{r-1}{2}^m}\mathbb{S}^{2m-1}\vee \hocolim(\mathcal{S}),$$
where
\begin{equation*}
    \xymatrix{
    \mathcal{S}: & \ast\sqcup\ast \ar@{<-}[r] & X\cap Y\sqcup X\cap W \ar@{^(->}[r] & X
    }
\end{equation*}
Now, if we define a new complex $K$ from $\mathcal{F}_d(K_{n,m})$ by gluing 
two new simplices $\Delta^U*\{u_0\}$ and $\Delta^V*\{v_0\}$, with $u_0,v_0$ new vertices, we have that $K\simeq\mathcal{F}_d(K_{n,m})$
and 
$$\hocolim(\mathcal{S})\cong\aste{\mathcal{Z}}_{_{K}}\left(\underline{L},\underline{\emptyset}\right),$$
where $L_{u_i}=sk_0K_r=L_{v_j}$ for $i,j>0$ and $L_{u_0}=pt=L_{v_0}$. Now, 
$$\Sigma\aste{\mathcal{Z}}_{_{K}}\left(\underline{L},\underline{\emptyset}\right)\simeq\Sigma\mathcal{F}_d(K_{n,m})\vee\bigvee_{\sigma\in K}\bigvee_{(r-1)^{|\sigma|}}\Sigma^{|\sigma|+1}lk(\sigma).$$
For any $\sigma$ which contains $u_0$ or $v_0$, $\hat{D}(\sigma)\simeq*$, therefore we only need to know the link for simplices without
those vertices. Take $U'=U\cup\{u_0\}$ and $V'=V\cup\{v_0\}$
\begin{itemize}
    \item If $\sigma\subseteq U$, there are three possibilities:
    \begin{enumerate}
        \item If $|\sigma|=1$, then $\displaystyle lk(\sigma)=\Delta^{U'-\{u_i\}}\sqcup sk_{d-1}\Delta^{V'}\simeq\bigvee_{\binom{m-1}{d}}\mathbb{S}^{d-1}\vee\mathbb{S}^0$.
        \item If $2\leq|\sigma|\leq d$, then
        $$lk(\sigma)=\left(\bigvee_{m-1}\mathbb{S}^0*sk_{d-|\sigma|-1}\Delta^{U'-\sigma}\right)\cup\Delta^{U'-\sigma}\simeq\bigvee_{m\binom{n-|\sigma|-1}{d-|\sigma|}}\mathbb{S}^{d-|\sigma|}.$$
        \item If $|\sigma|\geq d+1$, then $lk(\sigma)=\Delta^{U'-\sigma}\simeq*$.
    \end{enumerate}
    \item Assume $|\sigma\cap U|=1$ and $|\sigma|\geq2$.
    \begin{enumerate}
        \item If $|\sigma|=2$, for $d\geq2$
        $$lk(\sigma)=sk_{d-2}\Delta^{U'-\sigma}\sqcup sk_{d-2}\Delta^{V'-\sigma}\simeq\bigvee_{\binom{n-2}{d-1}}\mathbb{S}^{d-2}\sqcup\bigvee_{\binom{m-2}{d-1}}\mathbb{S}^{d-2}.$$
        \item If $3\leq|\sigma|\leq d$, then 
        $$lk(\sigma)=sk_{d-|\sigma|}\Delta^{V'-\sigma}\simeq\bigvee_{\binom{m-|\sigma|}{d-|\sigma|}}\mathbb{S}^{d-|\sigma|}.$$
        \item If $|\sigma|=d+1$, then $lk(\sigma)=\emptyset$.
    \end{enumerate}
\end{itemize}
\end{proof}

\begin{theorem}
For any positive integers $r,n_1,\dots,n_k\geq2$, with $k\geq3$ and $G=K_{n_1,\dots,n_k}\circ K_r$ we have for 
$1\leq d\leq\min\{n_1-1,\dots,n_k-1\}$ that
$$\Sigma\mathcal{F}_d(G)\simeq\bigvee_{a_d}\mathbb{S}^2\vee\bigvee_{b_d}\mathbb{S}^3\vee\bigvee_{c_d}\mathbb{S}^{d+1}\vee\bigvee_{i=1}^k\left(\bigvee_{\binom{r-1}{2}^{n_i}}\mathbb{S}^{2n_i}\right),$$
where $b_1=c_1=0$, 
$$a_1=\sum_{\{i,j\}\in\binom{\underline{k}}{2}}(r^2-2r+2)n_in_j\;+\sum_{i=1}^kn_i(t_i+1)(r-1)\;-k+1,$$
and for $d\geq2$ $a_d=\frac{(k-1)(k-2)}{2}$,
$$b_d=\sum_{i=1}^k(r-1)n_i\left(\sum_{l=2}^d(t_i-l+2)\right)+\sum_{\{i,j\}\in\binom{\underline{k}}{2}}(n_i^2n_j^2-n_i^2n_j-n_in_j^2+rn_in_j)$$
taking 
$$t_i=\sum_{j\neq i}n_j\;-1,\;\;\;\;p_i=\sum_{j\neq i}\binom{n_j-2}{d}$$
$$c_d=(r-1)\left(\sum_{i=1}^k\left[\sum_{l=2}^d(t_i+1)\binom{n_i}{l}\binom{n_i-l-1}{d-l}+n_i(t_i+1)\binom{n_i-2}{d-1}+n_ip_i\right]\right)$$
$$+\sum_{\{i,j\}\in\binom{\underline{k}}{2}}(r-1)\left[n_in_j\left(\binom{n_i-1}{d-2}+\binom{n_j-1}{d-2}+n_i\binom{n_j}{d}\right)\right]$$
$$+\sum_{\{i,j\}\in\binom{\underline{k}}{2}}n_in_j\left[n_j\binom{n_i-1}{d}+n_i\binom{n_j-1}{d}+n_j\binom{n_i}{d}\right]$$
$$+(r-1)\left(\sum_{\{i,j\}\in\binom{\underline{k}}{2}}n_in_j\sum_{l=2}^d\left[n_i\binom{n_j}{l}\binom{n_j-1}{d-k-1}+n_j\binom{n_i}{l}\binom{n_i-1}{d-k-1}\right]\right)$$
\end{theorem}
\begin{proof}
Assume $V_1,\dots,V_k$ are the partition of the vertices of $K_{n_1,\dots,n_k}$. We have
$$\mathcal{F}_d(G)=\aste{\mathcal{Z}}_{_{\mathcal{F}_d(K_{n_1,\dots,n_k})}}\left(sk_0K_r,\emptyset\right)\cup\bigsqcup_{i=1}^k\left(\bigast_{n_i}K_r\right).$$
For all $1\leq i\leq k$,
$$\aste{\mathcal{Z}}_{_{\mathcal{F}_d(K_{n_1,\dots,n_k})}}\left(sk_0K_r,\emptyset\right)\cap\bigast_{n_i}K_r=\aste{\mathcal{Z}}_{_{\Delta^{V_i}}}\left(sk_0K_r,\emptyset\right),$$
and the inclusion $\displaystyle\aste{\mathcal{Z}}_{_{\Delta^{V_i}}}\left(sk_0K_r,\emptyset\right)\longhookrightarrow\bigast_{n_i}K_r$ is 
null-homotopic. Therefore 
$$\Sigma\mathcal{F}_d(G)=\Sigma\aste{\mathcal{Z}}_{_{K}}\left(\underline{L},\underline{\emptyset}\right)\vee\bigvee_{i=1}^k\left(\bigvee_{\binom{r-1}{2}^{n_i}}\mathbb{S}^{2n_i}\right),$$
where:
\begin{itemize}
    \item $K$ is the complex obtain from $\mathcal{F}_d(K_{n_1,\dots,n_k})$ by ading the simplexes $\Delta^{V_i'}$, 
    where $V_i'=V_i\cup\{v_i^0\}$ with $v_i^0$ a new vertice.
    \item $L_u=sk_0K_r$ for any $u\in V(G)$.
    \item $L_{v_i^0}=pt$ for all $i$.
\end{itemize}
As before,
$$\Sigma\aste{\mathcal{Z}}_{_{K}}\left(\underline{L},\underline{\emptyset}\right)\simeq\Sigma K\vee\bigvee_{\sigma\in K}\bigvee_{(r-1)^{|\sigma|}}\Sigma^{|\sigma|+1}lk(\sigma).$$
Since any three vertices form three different sets of the vertex partition give a cycle and for any $\sigma$ which contains a $v_i^0$ 
we have that $\hat{D}(\sigma)\simeq*$, 
the only links we need to determine are those of simplices contain in one or two sets and that do not contain a vertex $v_i^0$. 

Let $\sigma$ be a simplex such that $\sigma\subseteq V_i$ for some $i$.
\begin{enumerate}
        \item If $|\sigma|=1$, for $d=1$, 
    $\displaystyle lk(\sigma)=\left(\bigsqcup_{j\neq i}sk_0\Delta^{V_j}\right)\cup\Delta^{V_i-\sigma}\simeq\bigvee_{t_i+1}\mathbb{S}^0$
        and for $d\geq2$
        $$lk(\sigma)=\left(\left(\bigsqcup_{j\neq i}sk_0\Delta^{V_j}\right)*sk_{d-2}\Delta^{V_i-\sigma}\right)\cup\left(\bigsqcup_{j\neq i}sk_{d-1}\Delta^{V_j}\right)\cup\Delta^{V_i'-\sigma}.$$
\begin{equation*}
\xymatrix{
\emptyset \ar@{->}[rr]^{\cong} \ar@{->}[dr] \ar@{->}[dd] & & \emptyset \ar@{->}[dd] \ar@{->}[rd] & & \\
 & \displaystyle\bigvee_{t_i}\mathbb{S}^0 \ar@{->}[rr]^{\simeq} \ar@{->}[dd] & & \displaystyle\bigvee_{t_i}\mathbb{S}^0 \ar@{->}[r] \ar@{->}[dd] & \displaystyle\bigsqcup_{j\neq i}\bigvee_{\binom{n_j-2}{d}}\mathbb{S}^{d-1} \ar@{->}[dd] \\
\displaystyle\bigvee_{\binom{n_i-2}{d-1}}\mathbb{S}^{d-2} \ar@{->}[rr] \ar@{->}[dr] & & \ast \ar@{->}[dr] &  & \\
  & \displaystyle\bigvee_{t_i\binom{n_i-2}{d-1}}\mathbb{S}^{d-1} \ar@{->}[rr] &  & \bigvee\mathbb{S}^{d-1} \ar@{->}[r] & \displaystyle\bigvee\mathbb{S}^{d-1}\vee\bigvee_{t_i-k+2}\mathbb{S}^{1}
}
\end{equation*}
        \item If $2\leq|\sigma|\leq d$, then
        $$lk(\sigma)=\Delta^{V_i'-\sigma}\cup\left(\left(\bigsqcup_{j\neq i}sk_0\Delta^{V_j}\right)*sk_{d-|\sigma|-1}\Delta^{V_i-\sigma}\right)\simeq\bigvee_{(t_i+1)\binom{n_i-|\sigma|-1}{d-|\sigma|}}\mathbb{S}^{d-|\sigma|}.$$
        \item If $d+1\leq|\sigma|\leq n_i$, then $lk(\sigma)=\Delta^{V_i'-\sigma}\simeq*$.
    \end{enumerate}
Let $\sigma$ be a simplex such that $|\sigma\cap V_i|\geq1$ and $|\sigma\cap V_j|=1$ for $i\neq j$.
\begin{enumerate}
    \item If $|\sigma\cap V_i|=1$, for $d=1$ $lk(\sigma)=\emptyset$ and for $d\geq2$,
    $$lk(\sigma)=sk_{d-2}\Delta^{V_i}\sqcup sk_{d-2}\Delta^{V_j}\simeq\bigvee_{\binom{n_i-1}{d-2}}\mathbb{S}^{d-2}\sqcup\bigvee_{\binom{n_j-1}{d-2}}\mathbb{S}^{d-2}.$$
    \item If $|\sigma\cap V_i|=l$ with $2\leq l\leq d-1$, then
    $$lk(\sigma)=sk_{d-l-1}\Delta^{V_i}\simeq\bigvee_{\binom{n_i-1}{d-l-1}}\mathbb{S}^{d-l-1},$$
    \item If $|\sigma\cap V_i|=d$, then $lk(\sigma)=\emptyset$.
\end{enumerate}
\end{proof}

\textbf{Acknowledgments.} The author wishes to thank Omar Antolín-Camarena for his comments and suggestions which improved this paper. 

\bibliographystyle{acm}
\bibliography{tipohomotopias}

\vspace{1cm}

\hspace{0cm}Andrés Carnero Bravo

\hspace{0cm}Instituto de Matemáticas, UNAM, Mexico City, Mexico

\hspace{0cm}\textit{E-mail address:} \href{mailto:acarnerobravo@gmail.com}{acarnerobravo@gmail.com}

\end{document}
