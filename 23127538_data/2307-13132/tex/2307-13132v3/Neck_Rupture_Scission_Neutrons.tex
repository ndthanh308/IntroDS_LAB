%% ****** Start of file apstemplate.tex ****** %
%%
%%
%%   This file is part of the APS files in the REVTeX 4 distribution.
%%   Version 4.1r of REVTeX, August 2010
%%
%%
%%   Copyright (c) 2001, 2009, 2010 The American Physical Society.
%%
%%   See the REVTeX 4 README file for restrictions and more information.
%% 
%
% This is a template for producing manuscripts for use with REVTEX 4.0
% Copy this file to another name and then work on that file.
% That way, you always have this original template file to use.
%
% Group addresses by affiliation; use superscriptaddress for long 
% author lists, or if there are many overlapping affiliations.
% For Phys. Rev. appearance, change preprint to twocolumn.
% Choose pra, prb, prc, prd, pre, prl, prstab, prstper, or rmp for journal
%  Add 'draft' option to mark overfull boxes with black boxes
%  Add 'showpacs' option to make PACS codes appear
%  Add 'showkeys' option to make keywords appear
\documentclass[10pt, aps,prc,twocolumn,superscriptaddress,preprintnumbers,
amsmath, 
floatfix,
longbibliography,
nofootinbib
]{revtex4-1}
\usepackage[T1]{fontenc}
%\usepackage[utf8x]{inputenc} 
\usepackage[utf8]{inputenc}
\usepackage{adjustbox}          % Used to adjust figure frames
\usepackage[caption=false]{subfig}
\usepackage{url}
\usepackage{color}
\usepackage{float}
\usepackage[pdftex,colorlinks=true, linkcolor = blue, citecolor=blue,urlcolor=blue, bookmarksnumbered=true, bookmarksopen=true]{hyperref}
\usepackage{longtable}
\usepackage{amsfonts}
\usepackage{dsfont}
\usepackage{wrapfig,bm} 
\usepackage[normalem]{ulem}
\usepackage{MnSymbol}
\usepackage{float}
\usepackage{color,soul}
\setlength{\belowcaptionskip}{1pt} 
%\usepackage{mcite}

%\documentclass[aps,prl,preprint,superscriptaddress]{revtex4-1} 
%\documentclass[aps,prl,reprint,groupedaddress]{revtex4-1}

\newcommand{\etal}{{\bfet al.,\;}} 
\newcommand{\eqn}[1]{Eq.~(\ref{#1})}
\newcommand{\fig}[1]{Fig.~\ref{#1}}
%\newcommand{\baa}{\begin{align}}
%\newcommand{\eaa}{\end{align}}
\newcommand{\beq}{\begin{equation}}
\newcommand{\eeq}{\end{equation}}
\newcommand{\bea}{\begin{eqnarray}}
\newcommand{\eea}{\end{eqnarray}}
\newcommand{\Tr}{\textrm{Tr}} 
\newcommand{\RE}{\textrm{Re}}
\newcommand{\IM}{\textrm{Im}}


\begin{document}
\title{Neck Rupture and Scission Neutrons in Nuclear Fission}

%\title{The most non-equilibrium stages of nuclear fission: the neck rupture and the emission of scission nucleons}
  

\author{{Ibrahim Abdurrahman}} 
\affiliation{Theoretical Division, Los Alamos National Laboratory, Los Alamos, NM 87545, USA}   
\author{{Matthew Kafker}}
\affiliation{Department of Physics, University of Washington, Seattle, WA 98195--1560, USA}
\author{{Aurel Bulgac}}
\affiliation{Department of Physics, University of Washington, Seattle, WA 98195--1560, USA}
\author{{Ionel Stetcu}}
\affiliation{Theoretical Division, Los Alamos National Laboratory, Los Alamos, NM 87545, USA}   

%\author{\textcolor{red}{Kyle Godbey}}
%\affiliation{FRIB, Michigan State University, East Lansing, MI 48824, USA}

%\author{\textcolor{red}{Guillaume Scamps}}
%\affiliation{Department of Physics, University of Washington, Seattle, WA 98195--1560, USA}
   
\date{\today}

 
\begin{abstract}

Just before a nucleus undergoes fission, a neck is formed between the emerging fission fragments. It is widely accepted that this neck undergoes a rather violent rupture, despite the absence of unambiguous experimental evidence. 
The main difficulty in addressing the neck rupture and saddle-to-scission stages of fission is that both are highly non-equilibrium processes. Here, we present the first fully microscopic characterization of the scission mechanism, along with the spectrum and the spatial distribution of scission neutrons (SNs), and some upper limit estimates for the emission of charged particles. The spectrum of SNs has a distinct angular distribution, with neutrons emitted in roughly equal numbers in the equatorial plane and along the fission axis.  They carry an average energy around $ 3 \pm 0.5 $ MeV for the fission of $^{236}$U, $^{240}$Pu and $^{252}$Cf, and a maximum of 16 - 18 MeV. We estimate a conservative lower bound of $9-14$ \% of the total emitted neutrons are produced at scission.  


\end{abstract}  

\preprint{NT@UW-23-08,LA-UR-23-25884}

\maketitle   


%\section{Introduction}




Nuclear fission was experimentally discovered by \textcite{Hahn:1939} in 1939.  Later in 1939, it was named and its main mechanism was explained by \textcite{Meitner:1939}.  It is a quantum many-body process of extreme complexity, with various parts of the process occurring at vastly different timescales.
The total time it takes, from the moment a neutron initiates the formation of a compound nucleus 
until all final fission products have attained their equilibrium state after $\beta$-decay, 
 can be on the order of billions of years \cite{Gonnenwein:2014}, and is greater by enormous orders of magnitude relative to the time it takes a nucleon to cross a nucleus, ${\cal O}(10^{-22})$ sec.

The compound system, formed by a
low-energy neutron~\cite{Hahn:1939} interacting with a target nucleus, evolves through many distinct stages. The first stage is a relatively slow quasi-equilibrium evolution, that lasts until the compound 
system~\cite{Bohr:1936} 
reaches the outer saddle-point at $\approx 10^{-14}$ sec.~\cite{Gonnenwein:2014}. During this stage, the nucleus, with an initial prolate intrinsic shape and axial symmetry, evolves into a nucleus with triaxial shape, and eventually into a reflection asymmetric and axially symmetric elongated shape near the outer fission barrier~\cite{Ryssens:2015}.
The second stage is a highly non-equilibrium evolution from 
saddle-to-scission~\cite{Bulgac:2016,Bulgac:2019c,Bulgac:2020}, when the primordial fission fragments (FFs) properties are defined within a duration of $\approx 5\times10^{-21}$ sec.~\cite{Gonnenwein:2014}. Even though this second stage is much faster than the first stage, it corresponds to rather slow dynamics, relative to the third stage (scission).
In this stage, the compound nucleus undergoes a relatively rapid separation into two FFs, lasting $\approx 10^{-22}$ sec. This stage is also known in the literature as the  neck rupture.  
This is followed by a fourth stage, the  FFs Coulomb acceleration during an interval of time of ${\cal O}(10^{-18})$ sec., when at the end the FFs achieve 
a shape equilibration.
While the initial compound nucleus is a relatively cold system with a very small spin, the primordial FFs are very hot and have relatively large spins as well \cite{Bulgac:2019c,Bulgac:2021}. 
These highly excited FFs emit prompt neutrons for an interval of time up to about ${\cal O}(10^{-14})$ sec., followed by the emission of the majority of
prompt $\gamma$-rays  for an interval of time until about ${\cal O}(10^{-3})$ sec., which is further followed by much slower $\beta$-decays. 
Other processes are also possible, such as delayed neutron or $\gamma$ emission after $\beta$-decay. 

With the exception of the saddle-to-scission configuration and the neck rupture, all the other stages of fission are relatively slow quasi-equilibrium processes. 
The fission dynamics after the compound nucleus reaches the outer saddle point has been typically
described in terms of the potential energy surface of the nucleus, determined by its shape~\cite{Gonnenwein:2014,Ring:2004,Pomorski:2012,Schunck:2016},
not compatible with recent microscopic studies~\cite{Bulgac:2016,Bulgac:2019c,Bulgac:2020, Bender:2020}, and agreement with experimental data typically requires adjustment of  many parameters \cite{Sierk:2017}. On the other hand, many different approaches to FF mass and charge distributions lead to agreement with experiment~\cite{Verriere:2021,Mumpower:2020,Sierk:2017,Ivanyuk:2024,Sadhukhan:2016,Sadhukhan:2017}, even though they rely on clearly contradicting physics assumptions, which simply demonstrates that these distributions are not very sensitive measures of the fission dynamics.  
Approximately at the top of the outer saddle, the nucleus starts forming a barely seen ``wrinkle'', where eventually the neck between the two FFs is formed. 
This ``wrinkle'' tends to appear when the fissioning nucleus is at a very early stage during the descent to scission, and its position hardly changes in time. A significant change in this position would require a large amount of energy for displacement that would not be available from fluctuations \cite{Hill:1953,Griffin:1957,Reinhard:1983,Bulgac:2019d,Bulgac:2022}. At the top of the outer saddle the nucleus starts a relatively slow dissipative evolution 
towards scission~\cite{Bulgac:2016,Bulgac:2019c,Bulgac:2020}.  During this period, the fissioning nucleus 
gets more elongated and the neck becomes more and more pronounced. 
The nuclear fluid behaves as nuclear molasses, with a very small collective velocity~\cite{Bulgac:2016,Bulgac:2019c,Bulgac:2020},  while at the same time the
intrinsic temperature of the system gradually increases. The bond between the two fission partners slowly weakens until the neck, which was still keeping them together, reaches 
a critical small diameter of approximately 3 fm and ruptures, exactly where the initial ``wrinkle'' formed much earlier at the top of the outer saddle. 
This dramatic separation of the two emerging FFs is  a rather short-time event. 
For \textcite{Brosa:1990} scission was the defining stage of fission, where the total kinetic energy (TKE) of the FFs is defined along with the average FF properties. 
The Brosa model assumes that the nucleus is a very viscous fluid, with a long neck that ruptures 
at a random position, and is widely invoked today in many phenomenological models \cite{Oberstedt:1999,Vogt:2009,Becker:2013,Talou:2021,Litaize:2012,Lovell2021,Fujio:2023,Najumunnisa:2023}, even though it has no microscopic justification and its claimed grounding in experimental data does not necessarily support a unique interpretation. Additionally, the Brosa random neck rupture model contradicts the theoretical assumptions of 
other popular approaches, such as the scission-point model of \textcite{Wilkins:1976}, 
where the FF formation is based on statistical equilibrium~\cite{Lemaitre:2015,Lemaitre:2021}, and Brownian motion or Langevin models~\cite{Sierk:2017,Randrup:2011,Albertsson:2020,Verriere:2021a,Ivanyuk:2024}. 
The drama of scission is followed by unavoidable debris characteristic of such dramatic separations, 
the scission neutrons (SNs), envisioned as early as 1939 by \textcite{Bohr:1939}.  Potentially other heavier fragments,
usually termed as ternary fission products~\cite{Vandenbosch:1973, Rose:1984, Wagemans:1991}, are created as well.  We relegate a  brief review of the history of SNs as an online supplementary material \cite{supplement}, with additional references~\cite{Debenedetti:1948,Fraser:1952,Fraser:1954,Stavinsky:1959,Halpern:1959,Vorobyev:2010,Brosa:1992,Rizea:2008,Stetcu:2011,Tanimura:2017,Ren:2022,Scamps:2012,Bulgac:2023,Demers:1946,Farwell:1947,Wollan:1947,Marshall:1949,Titterton:1951,Allen:1950,Feather:1947,Nobles:1962,Fraenkel:1967,Albenesius:1959,Watson:1961,Marshall:1966,Goward:1949,Whetstone:1965,Dakowski:1967,Cosper:1967,Krogulski:1969,Chwaszczewska:1967,Vorobiev:1969,Vandenbosch:1973,Rose:1984,Poenaru:1986,Dalfovo:1999,Pethick:2008,Giorgini:2008,Fowler:1928,Uehling:1933,Bertsch:1988,Bulgac:2023b,Tully:1971,Tully:1990,Hammes:1994,Frobrich:1998}, where we also present many more details of our study. 

% Time series figure.
% Figure environment removed


% Neck dynamics
% Figure environment removed
%\vspace{0.5cm}

%% Kinetic energy figure.
% Figure environment removed
%

% Number of SNs figure.
% Figure environment removed
%



%\section{Neck Formation and Decay Dynamics } 

In these simulations, we started by placing the initial compound nucleus 
near the top of the outer barrier in a very large simulation volume, in order to allow to the emitted nucleons enough time to decouple from the FFs 
after the neck rupture. 
We have performed a range of simulations for $^{235}$U(n$_{th}$,f), $^{239}$Pu(n$_{th}$,f), and $^{252}$Cf(sf),
using the nuclear energy density functional (NEDF)  SeaLL1~\cite{Bulgac:2018} in simulation volumes $48^2\times120$  and $48^2\times100$ fm$^3$, 
with a lattice constant of 1 fm, for further technical details see Ref.~\cite{Shi:2020}. 
The SeaLL1 NEDF is defined by only 8 basic nuclear parameters, each related to specific nuclear properties known for decades, and contains the smallest number of phenomenological parameters of any NEDF to date ~\cite{Bulgac:2018,Bulgac:2022c}.
We started the simulations at various deformations $Q_{20}$ and $Q_{30}$, as listed in Ref.~\cite{supplement}, near 
the outer fission barrier rim and see Refs.~\cite{Bulgac:2016,Bulgac:2019c,Bulgac:2020}, where 
one can find more details about how the FF properties vary with the choice of initial conditions. 
Our simulation volume of $48^2\times120$ fm$^3$ required the use of the entire supercomputer Summit (27,648 GPUs), corresponding to 442 TBs of total GPU memory, with further details provided in Ref.~\cite{supplement}. Despite this, we still could not follow the emission of nucleons for a long time, 
since the emitted nucleons are reflected back at the boundary relatively rapidly, see the lowest two rows of Fig.~\ref{fig:tseries}, where interference patterns emerge. In the transversal direction the reflection from the boundaries occurs 
earlier than along the fission axis, and that has effected some of the properties of the nucleons emitted perpendicular to the fission axis. However the effect is minor, see Ref.~\cite{supplement}.  



From here, we will concentrate on the dynamics of the neck formation and rupture, followed by the emission of nucleons,   all treated   
within the time-dependent density functional theory extended to superfluid fermionic systems~\cite{Bulgac:2019}. The integrated neck density, shown in Fig.~\ref{fig:neck}, is defined as
\begin{align}
 n_{\mathrm{neck},\tau}(t) = \int \!\!dxdy \, n_\tau(x,y,z_\mathrm{neck},t), \quad \tau = n, \, p,
\end{align}
separately for neutrons and protons, where the $z_{neck}$ is the position along the fission axis $Oz$ where the neck has the smallest radius.  The neck decays relatively slowly at scission, until its diameter reaches about 3 fm, after-which it undergoes a very rapid decay.  Different curves illustrated in the lower panel correspond to trajectories started at various initial conditions for the deformations $Q_{20}, Q_{30}$ close to the outer fission barrier~\cite{supplement}. The time to reach scission can vary significantly, depending on the initial values of the deformations $Q_{20}, Q_{30}$ and on the NEDF used, typically ranging from 1,000 to 3,000 fm/c.


These microscopic results illustrate several points, which were unknown until now, due to the absence 
of any detailed fully microscopic quantum many-body simulations of fission dynamics.  First, the ``wrinkle'' in the nuclear density, where the neck is eventually formed and where the nucleus eventually scissions, is determined a long time before the nucleus reaches scission. Within the TDDFT framework the position of the neck rupture is not random,
unlike in the Brosa model~\cite{Brosa:1990,supplement}. At the time when the neck reaches a critical diameter of $\approx 3 $ fm, the nuclear surface tension and the shape of the compound around the neck region, can no longer counteract the strong Coulomb repulsion between the preformed FFs, causing the system to violently ``snap''. 
One should keep in mind that as the intrinsic temperature of the compound nucleus increases, the surface tension also decreases.    
The geometry of the nuclear shape changes dramatically at this stage, from exhibiting a neck region where the Gaussian curvature is negative, to two separated FFs with 
surfaces characterized by predominantly positive Gaussian curvatures. 

Second, the proton neck completes its rupture earlier than the neutron neck does, see lower panel in
 Fig.~\ref{fig:neck}, resulting in the neck being mostly sustained by the neutrons just before the full rupture.  This is similar to the neutron density in the neutron skin of heavy nuclei. In this time interval, the number of neutrons per unit area at the neck varies by an order of magnitude.
The protons in the emerging FFs separate about 50-100 fm/c before the neutrons neck ruptures. 
Additionally, the integrated neutron and proton densities at $z_{\mathrm{neck}}$ asymptotically reach almost equilibrium values, after the neck ruptures.

 Third, the rupture is unarguably the fastest stage of the fission dynamics, starting from the capture of the incident neutron and formation of the compound nucleus, until all fission products have been emitted. The decay times are 15 fm/c and 35 fm/c for proton and neutron necks respectively, which are significantly faster processes than the time it takes the fastest nucleon to communicate any information or facilitate any kind of equilibrium between the two preformed FFs, which is at minimum $ \sim 160$ fm/c. 

 Fourth, the neck decay dynamics displays a clear universality (for asymmetric fission) irrespective of the initial conditions, as shown in the lower panel of Fig.~\ref{fig:neck}, with the proton neck rupturing well ahead of the neutron neck, due to the presence of a well-defined neutron skin.  The time to full mass, charge, excitation energies and TKE definition all vary~\cite{Bulgac:2019c,Bulgac:2020}, while the neck dynamics are essentially unchanged for both proton and neutron components.

Last, the scission mechanism emerging from a fully microscopic treatment of the fission dynamics is totally at odds with previous models, including the Brosa random rupture model and the scission-point models. TDDFT extended to superfluid systems is the only theoretical microscopic framework so far in the literature in which scission is treated without any unchecked assumptions or fitting parameters, which produces results that are in agreement with data~\cite{Bulgac:2016,Bulgac:2019c,Bulgac:2020,Bulgac:2019}. 
 



%\section{Emission of Scission Nucleons}

%Consider the case of a long neck described in the Brosa model, whose existence was argued for in order to describe both TKE and FF yields distributions. If a long neck was to randomly rupture somewhere in the middle, the surface of the emerging FFs would initially have at least one rather long ``nose,'' where the Gaussian  curvature is negative. These noses would eventually have to be swallowed into the FFs, leading to a particle current lasting a noticeable time interval.  This involves a larger number of particles, thus more scission nucleons,  longer equilibration times, and would also cause the FFs deformations to relax at a later time. Before particle equilibration can be reached however, the presence of a long remnant neck in the FFs and its re-absorption would likely lead to a significant number of nucleons being emitted in the direction of motion of each fragment, an aspect which will discuss in more detail below.

The neck rupture is a very fast ``healing'' process of the nuclear surface in the neck region.  Unlike a gas in a punctured balloon, which would rapidly escape the enclosure, 
due to the presence of the nuclear ``skin'' and strong surface tension, the nucleus behaves as a fluid. 
The  surface tension quickly ``heals'' the ``wound,'' however a small fraction of matter manages to escape like a gas, with no droplet formation,
see Fig.~\ref{fig:tseries}. The potential condensation of this emitted gas into light charged particles cannot be described within the present framework, which includes at most two particle correlations. This is not to be confused with ternary fission of a preformed fragment, where further discussion is provided in Ref.~\cite{supplement}.  In Fig.~\ref{fig:tseries}, and more in Ref.~\cite{supplement}, we show several 
representative frames for neutrons and protons of the neck formation and emission of nucleons. 

The most remarkable features of this process are the following.
 As visible in Fig.~\ref{fig:neck} the proton neck completes its rupture before the neutron neck, however in two stages.  
 Immediately at scission, which in Fig.~\ref{fig:tseries} is identified with the rupture of the proton neck, 
 a number of nucleons are emitted in the plane perpendicular to the fission axis (see second time frame).  After a sufficient time for nucleons to propagate from the neck to the nose of each FF, scission nucleons also appear propagating in front of each FF.  In 1984, it was suggested by \textcite{Madler:1985}, that the reabsorption of the neck stumps by the FFs, being a relatively rapid process, could act as a ``catapult'', which is more appropriately described as a slingshot, and ``push'' nucleons out of the front of the FFs.  It is also important to note, the formation of three neutron clouds, two in front of the FFs and one transverse ring perpendicular to the fission axis, appears across all considered trajectories and nuclei thus far.  

 Remembering that the TKE is roughly 171-186 MeV, at the end of the full Coulomb acceleration the light and heavy FFs will have an average kinetic energy per 
 nucleon of about 1 MeV and 0.5 MeV respectively, which is significantly lower than the average kinetic energy of the SNs, see Fig.~\ref{fig:ke}, given by 3.51 $\pm$ 0.25 MeV, 3.42 $\pm$ 0.27 MeV, and 2.67 $\pm$ 0.24 MeV for $^{236}$U, $^{240}$Pu, and $^{252}$Cf respectively.  As noted by R. Capote~\cite{Capote:2023}, our results, which are consistent with high-energy neutrons observed via dosimetry measurements~\cite{Schulc:2023}, point to an unmistakable need to include SNs in the analysis of prompt neutron spectra~\cite{Brown:2018}.
As a result, the FFs will never have a chance to catch up with them.  Additionally, the SNs are essentially free, since their total interaction energy, estimated using the 
neutron equation of state~\cite{Bulgac:2018},
\begin{equation}
  E_n^{\mathrm{int}} =\int dV[ a_n n_n^{5/3} + b_n n_n^2 + c_n n_n^{7/3}] \ll E_n^{\mathrm{kin}},
 \end{equation}
comprises less than 1 percent of their kinetic energy.  
The total number of emitted neutrons, shown in Fig.~\ref{fig:nsci}, is about $0.30\pm 0.05$, $0.26 \pm 0.05 $, and $0.55\pm0.02$ per fission event for $^{236}$U, $^{240}$Pu, 
 and $^{252}$Cf respectively, which is a considerable portion (roughly $9-14$ \%) of the total emitted prompt neutrons, see
 Refs.~\cite{Vandenbosch:1973,Wagemans:1991}. These are somewhat conservative estimates, see the  discussion in Ref.~\cite{supplement}, and these numbers can be likely enhanced by at least a factor of 1.25.  This is clear in Fig.~\ref{fig:nsci} where neither the emission of neutrons or protons has flattened. In comparison, Carjan \textit{et al.}~\cite{ Rizea:2008,Carjan:2010,Carjan:2015,Capote:2016,Carjan:2019} estimated an upper bound of $25 - 50$ \% of prompt fission neutrons are emitted during scission. At the same time the number of emitted protons is about 
 two orders of magnitude lower, see Fig.~\ref{fig:nsci}.  
The nucleons are emitted in roughly equal numbers both transverse to the scission axis and in front of the FFs.


%\section{Conclusions} 

  In summary, we have clarified several aspects of the most non-equilibrium and fastest stage of nuclear fission dynamics.  Within TDDFT, the neck 
rupture is not a random process, as previously argued in various phenomenological models.  Additionally, it appears that the neck rupture has similar  dynamics for a large class of asymmetric fission events, irrespective of nucleus considered or the initial conditions, beyond the top of the outer fission barrier.  This universality carries over to the emission of SNs, whose signal always appears as three distinct clouds, one transverse to the fission axis and two in front of each FF, in almost equal proportions.  
The aspects of the neck dynamics discussed above, 
can serve as a theoretical input for any semi-phenomenological approach to study FF properties~\cite{Becker:2013,Talou:2021,Vogt:2009,Litaize:2012}. 

 
 The idea of SNs, proposed by \textcite{Bohr:1939}, is almost as old as nuclear fission itself.  The existence of SNs has been debated over the years~\cite{Wilson:1947, Debenedetti:1948, Stavinsky:1959, Fuller:1962, Bowman:1962, Kapoor:1963, Skarsvaag:1963, Gavron:1974, Boneh:1974, Pringle:1975,Franklyn:1978, Franklyn:1986, Jorgensen:1988,  Milek:1988,Samant:1995,Hwang:1999,Carjan:2007, Rizea:2008,Carjan:2010,Carjan:2015,Capote:2016b,Capote:2016,Guseva:2018, Vorobyev:2018, Carjan:2019}, see also \textit{Historical Note} in Ref.~\cite{Wagemans:1991}, and their experimental confirmation is still an open question. While neutron properties in earlier studies using simplified models~\cite{Rizea:2008,Capote:2016,Carjan:2019} have some features somewhat similar to what we find, they are missing the major component of emission perpendicular to the fission axis. The small fraction of SNs carry noticeable kinetic energy, and what is more surprising is that their energy spectrum ranges up to almost 18 MeV, as in shown in Fig.~\ref{fig:ke}, similar to other observations~\cite{Capote:2023,Brown:2018}. Along with SNs, a very small fraction of protons are emitted as well, and their fraction suggests a theoretical estimate for the emission of $\alpha$-particles 
and other charged nuclei.
   
   \vspace{0.5cm}         
{\bf Acknowledgements} \\

We thank Kyle Godbey and Guillaume Scamps for many useful discussions, and to Roberto Capote for his feedback regarding the impact of this work on the description of prompt fission neutron spectra.  The work of I.A. and I.S. was supported by the U.S. Department of Energy through the Los
Alamos National Laboratory. The Los Alamos National
Laboratory is operated by Triad National Security, LLC,
for the National Nuclear Security Administration of the U.S.
Department of Energy Contract No. 89233218CNA000001.
I.A. and I.S. gratefully acknowledge partial support and computational
resources provided by the Advanced Simulation and
Computing (ASC) Program.  The work of M.K. and A.B. was supported by the US DOE, Office of Science, Grant No. DE-FG02-97ER41014 and
also partially by NNSA cooperative Agreement
DE-NA0003841, and is greatly appreciated. 
This research used resources of the Oak Ridge
Leadership Computing Facility, which is a U.S. DOE Office of Science
User Facility supported under Contract No. DE-AC05-00OR22725. 

%%%%%%%%%%%%%%%%%%%%%%%%%%%%%%%%%%%%%%%%%%%%%

% These are needed to avoid a babel error.
\providecommand{\selectlanguage}[1]{}
\renewcommand{\selectlanguage}[1]{}

\bibliography{local_fission}%,SciNeuIntroBib}

\end{document}
