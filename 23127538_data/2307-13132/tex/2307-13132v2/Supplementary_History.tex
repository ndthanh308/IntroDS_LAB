%% ****** Start of file apstemplate.tex ****** %
%%
%%
%%   This file is part of the APS files in the REVTeX 4 distribution.
%%   Version 4.1r of REVTeX, August 2010
%%
%%
%%   Copyright (c) 2001, 2009, 2010 The American Physical Society.
%%
%%   See the REVTeX 4 README file for restrictions and more information.
%% 
%
% This is a template for producing manuscripts for use with REVTEX 4.0
% Copy this file to another name and then work on that file.
% That way, you always have this original template file to use.
%
% Group addresses by affiliation; use superscriptaddress for long 
% author lists, or if there are many overlapping affiliations.
% For Phys. Rev. appearance, change preprint to twocolumn.
% Choose pra, prb, prc, prd, pre, prl, prstab, prstper, or rmp for journal
%  Add 'draft' option to mark overfull boxes with black boxes
%  Add 'showpacs' option to make PACS codes appear
%  Add 'showkeys' option to make keywords appear
\documentclass[10pt, aps,prc,twocolumn,superscriptaddress,preprintnumbers,
amsmath, 
floatfix,
longbibliography,
nofootinbib
]{revtex4-1}
\usepackage[T1]{fontenc}
%\usepackage[utf8x]{inputenc} 
\usepackage[utf8]{inputenc}
\usepackage{adjustbox}          % Used to adjust figure frames
\usepackage[caption=false]{subfig}
\usepackage{url}
\usepackage{color}
\usepackage{float}
\usepackage[pdftex,colorlinks=true, linkcolor = blue, citecolor=blue,urlcolor=blue, bookmarksnumbered=true, bookmarksopen=true]{hyperref}
\usepackage{longtable}
\usepackage{amsfonts}
\usepackage{dsfont}
\usepackage{wrapfig,bm} 
\usepackage[normalem]{ulem}
\usepackage{MnSymbol}
\usepackage{float}

%\documentclass[aps,prl,preprint,superscriptaddress]{revtex4-1} 
%\documentclass[aps,prl,reprint,groupedaddress]{revtex4-1}

\newcommand{\etal}{{\bfet al.,\;}} 
\newcommand{\eqn}[1]{Eq.~(\ref{#1})}
\newcommand{\fig}[1]{Fig.~\ref{#1}}
%\newcommand{\baa}{\begin{align}}
%\newcommand{\eaa}{\end{align}}
\newcommand{\beq}{\begin{equation}}
\newcommand{\eeq}{\end{equation}}
\newcommand{\bea}{\begin{eqnarray}}
\newcommand{\eea}{\end{eqnarray}}
\newcommand{\Tr}{\textrm{Tr}} 
\newcommand{\RE}{\textrm{Re}}
\newcommand{\IM}{\textrm{Im}}

\begin{document}
%\title{Neck Rupture and Scission Neutrons in Nuclear Fission}

\title{Minimal history of scission neutrons}
  

\author{{Ibrahim Abdurrahman}} 
\affiliation{Theoretical Division, Los Alamos National Laboratory, Los Alamos, NM 87545, USA}   
\author{{Matthew Kafker}}
\affiliation{Department of Physics, University of Washington, Seattle, WA 98195--1560, USA}
\author{{Aurel Bulgac}}
\affiliation{Department of Physics, University of Washington, Seattle, WA 98195--1560, USA}
\author{{Ionel Stetcu}}
\affiliation{Theoretical Division, Los Alamos National Laboratory, Los Alamos, NM 87545, USA}   

%\author{\textcolor{red}{Kyle Godbey}}
%\affiliation{FRIB, Michigan State University, East Lansing, MI 48824, USA}

%\author{\textcolor{red}{Guillaume Scamps}}
%\affiliation{Department of Physics, University of Washington, Seattle, WA 98195--1560, USA}
   
\date{\today}


\maketitle   
Nuclear fission was experimentally discovered by \textcite{Hahn:1939} in 1939.  In the same year, it was named and its main mechanism was explained by \textcite{Meitner:1939}. Surprisingly, the idea of scission neutrons (SNs) is almost as old as the discovery of fission itself, also being first considered in 1939.  At the time, Bohr and Wheeler conjectured three sources for neutron emission during fission: first delayed neutrons occurring on the time scale of seconds, second neutrons evaporated from the fission fragments (FFs) as a result of their excitation, now known as prompt neutrons, and finally, and much more speculatively, neutrons formed due to the neck rupture~\cite{Bohr:1939}:\\ \\
\emph{We consider briefly the third possibility that the neutrons in question are produced during the fission process itself.  In this connection attention may be called to observations on the manner in which a fluid mass of unstable form divides into two smaller masses of greater stability; it is found that tiny droplets are generally formed in the space where the original enveloping surface was torn apart.}
\\ \\ 
For a long time after, the idea remained dormant.  In the late 40s, the first experiments investigating the directional properties of neutrons, produced from the fission of $^{235}$U(n,f), were conducted.  At the time there was a consensus that the experimental results were consistent with the assumption that all neutrons are isotropically emitted from fully accelerated FFs, and any brief consideration for SNs were quickly dismissed~\cite{Debenedetti:1948}. As experiments were refined in the 50s conclusions remained the same~\cite{Fraser:1952,Fraser:1954}.  It was not until the 60s, starting with experiments conducted by \textcite{Bowman:1962}, that the idea of SNs would re-emerge with considerable momentum. 

In 1962, his group computed the angular distributions of neutrons emitted from the spontaneous fission of $^{252}$Cf, observing deviations from the "isotropic hypothesis" (which states all neutrons would be emitted isotropically in the intrinsic frame of fully accelerated FFs).  Using a model for SNs proposed by \textcite{Stavinsky:1959} in 1959 and \textcite{Fuller:1962} in 1962, who extended the work of \textcite{Halpern:1959} from scission alphas to scission neutrons, \textcite{Bowman:1962} predicted that SNs would comprise of $\sim$ 10\% of prompt neutrons emitted during fission. Such models assume SNs are treated as being isotropically emitted from the region of the neck rupture. In particular, Fuller proposed the sudden rise of the nuclear potential at the neck during scission would lead to an expulsion of nuclear matter, and estimated scission neutrons would comprise 10-20\% of neutrons emitted, in the case of the spontaneous fission $^{252}$Cf, with each carrying an average of 2-6 MeV of energy~\cite{Fuller:1962}.  Subsequent experiments mostly agreed with the findings of \textcite{Bowman:1962}~\cite{Kapoor:1963,Skarsvaag:1963}, with one proposing an alternative explanation to account for inconsistencies between experimental results and the isotropic emission hypothesis, namely pre-scission neutron emission after the nucleus crosses the saddle~\cite{Kapoor:1963}. 

It is important to re-stress, conclusions pertaining to the neck rupture and related phenomena, such as SNs, cannot be probed directly by experiment.  As a result, all specific conclusions about the emission mechanism of SNs are extremely model dependent, and, beyond the 60s, experimental efforts to confirm the existence of and/or estimate the effects of SNs led to mixed results, with estimates placing the number of SNs as low as 1\% to as high as 15\%~\cite{Gavron:1974, Pringle:1975, Franklyn:1978, Franklyn:1986, Jorgensen:1988, Samant:1995, Hwang:1999, Vorobyev:2010, Guseva:2018, Vorobyev:2018}.  Significantly, the vast majority assumed a simple model for SN emission, namely, all SNs are expelled isotropically in the laboratory frame, close to the neck rupture, and typically with low kinetic energy.  This, despite the greater variance in  the theoretical models available over time.  To illustrate the great historical confusion surrounding the topic consider the following quote from~\textcite{Wagemans:1991} in 1991: \\ \\
\emph{Up to a few years ago, it was generally admitted that 15 to 20\% of neutrons emitted during the fission process were scission neutrons.  This has been contradicted by recent results ...  These measurements indicated that probably only 1.1 $\pm$ 0.3\% of the neutrons are scission neutrons and that their producitons mechanism is similar to that of satellite droplets in the disintegration of liquid jets.}
\\ 

In general, the majority of models proposed and/or used to model SN emission can be classified into three distinct classes. First, neutrons emitted due to the sudden change in the nuclear potential during the neck rupture, historically, the most commonly evoked model~\cite{Boneh:1974, Carjan:2007, Milek:1988}. In this case the number and kinetic energies of the SNs are determined by the amount of matter in the neck region and the time it takes the neck to rupture. The majority of SNs are expected to have higher kinetic energy (relative to prompt neutrons), and either be emitted isotropically from the region of the neck rupture or (in some models) perpendicular to the scission axis.  Second, fragments formed as satellite droplets, which could comprise of both neutrons and protons, in cases of fission where the neck is highly elongated~\cite{Brosa:1992}.  This mechanism would have a similar signal to the first case, except the kinetic energies of the SNs would be significantly lower.  Last, SNs emitted in front the FFs via the so called catapult mechanism: as the neck ruptures the tails of the densities of FFs are "snatched" inside, travel through the nucleus with high energy, and emerge on the other-side as SNs~\cite{Madler:1985}. Here, the majority of the SN signal would be seen as polar emission (in the direction of the FFs) with high kinetic energy, all with respect to the lab frame.  All of the above proposed mechanisms have distinct features, and hence unique signals for determining which is primarily responsible for SNs. In addition, recent hybrid approaches, based on the Time-Dependent Schr$\rm \ddot{o}$dinger Equation in polar coordinates (TDSE2D), have been used to model SNs~\cite{Rizea:2008, Carjan:2010, Carjan:2015, Capote:2016, Carjan:2019}.

%%%%%%%%%%%%%%%%%%%%%%%%%%%%%%%%%%%%%%%%%%%%%

% These are needed to avoid a babel error.
\providecommand{\selectlanguage}[1]{}
\renewcommand{\selectlanguage}[1]{}

\bibliography{local_fission}

\end{document}
