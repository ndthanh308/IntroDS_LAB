%% ****** Start of file apstemplate.tex ****** %
%%
%%
%%   This file is part of the APS files in the REVTeX 4 distribution.
%%   Version 4.1r of REVTeX, August 2010
%%
%%
%%   Copyright (c) 2001, 2009, 2010 The American Physical Society.
%%
%%   See the REVTeX 4 README file for restrictions and more information.
%% 
%
% This is a template for producing manuscripts for use with REVTEX 4.0
% Copy this file to another name and then work on that file.
% That way, you always have this original template file to use.
%
% Group addresses by affiliation; use superscriptaddress for long 
% author lists, or if there are many overlapping affiliations.
% For Phys. Rev. appearance, change preprint to twocolumn.
% Choose pra, prb, prc, prd, pre, prl, prstab, prstper, or rmp for journal
%  Add 'draft' option to mark overfull boxes with black boxes
%  Add 'showpacs' option to make PACS codes appear
%  Add 'showkeys' option to make keywords appear
\documentclass[10pt, aps,prc,twocolumn,superscriptaddress,preprintnumbers,
amsmath, 
floatfix,
longbibliography,
nofootinbib
]{revtex4-1}
\usepackage[T1]{fontenc}
%\usepackage[utf8x]{inputenc} 
\usepackage[utf8]{inputenc}
\usepackage{adjustbox}          % Used to adjust figure frames
\usepackage[caption=false]{subfig}
\usepackage{url}
\usepackage{color}
\usepackage{float}
\usepackage[pdftex,colorlinks=true, linkcolor = blue, citecolor=blue,urlcolor=blue, bookmarksnumbered=true, bookmarksopen=true]{hyperref}
\usepackage{longtable}
\usepackage{amsfonts}
\usepackage{dsfont}
\usepackage{wrapfig,bm} 
\usepackage[normalem]{ulem}
\usepackage{MnSymbol}
\usepackage{float}
\setlength{\belowcaptionskip}{1pt} 

%\documentclass[aps,prl,preprint,superscriptaddress]{revtex4-1} 
%\documentclass[aps,prl,reprint,groupedaddress]{revtex4-1}

\newcommand{\etal}{{\bfet al.,\;}} 
\newcommand{\eqn}[1]{Eq.~(\ref{#1})}
\newcommand{\fig}[1]{Fig.~\ref{#1}}
%\newcommand{\baa}{\begin{align}}
%\newcommand{\eaa}{\end{align}}
\newcommand{\beq}{\begin{equation}}
\newcommand{\eeq}{\end{equation}}
\newcommand{\bea}{\begin{eqnarray}}
\newcommand{\eea}{\end{eqnarray}}
\newcommand{\Tr}{\textrm{Tr}} 
\newcommand{\RE}{\textrm{Re}}
\newcommand{\IM}{\textrm{Im}}


\begin{document}
\title{Neck Rupture and Scission Neutrons in Nuclear Fission}

%\title{The most non-equilibrium stages of nuclear fission: the neck rupture and the emission of scission nucleons}
  

\author{{Ibrahim Abdurrahman}} 
\affiliation{Theoretical Division, Los Alamos National Laboratory, Los Alamos, NM 87545, USA}   
\author{{Matthew Kafker}}
\affiliation{Department of Physics, University of Washington, Seattle, WA 98195--1560, USA}
\author{{Aurel Bulgac}}
\affiliation{Department of Physics, University of Washington, Seattle, WA 98195--1560, USA}
\author{{Ionel Stetcu}}
\affiliation{Theoretical Division, Los Alamos National Laboratory, Los Alamos, NM 87545, USA}   

%\author{\textcolor{red}{Kyle Godbey}}
%\affiliation{FRIB, Michigan State University, East Lansing, MI 48824, USA}

%\author{\textcolor{red}{Guillaume Scamps}}
%\affiliation{Department of Physics, University of Washington, Seattle, WA 98195--1560, USA}
   
\date{\today}

 
\begin{abstract}

Just before a nucleus fissions a neck is formed between the emerging fission fragments. 
It is widely accepted that this neck undergoes a rather violent rupture, despite no direct experimental evidence, and only a few contentious theoretical treatments of this fission stage were ever performed in the more than eight decades since nuclear fission was experimentally observed by Hahn and Strassmann and described by Meitner and Frisch in 1939. 
In the same year, Bohr and Wheeler conjectured that the fission of the nuclear liquid drop 
would likely be accompanied by the rapid formation of tiny droplets, later identified with either scission neutrons 
or other ternary fission fragments, a process which has not yet been discussed in a fully quantum many-body framework. 
The main difficulty in addressing both of these stages of nuclear fission is both are highly non-equilibrium processes. Here we will present 
the first fully microscopic characterization of the scission mechanism, along with the spectrum and the spatial distribution of scission 
neutrons, and some upper limit estimates for the emission of charged particles.

\end{abstract}  

\preprint{NT@UW-23-08,LA-UR-23-25884}

\maketitle   


%\section{Introduction}




Nuclear fission was experimentally discovered by \textcite{Hahn:1939} in 1939.  Later in 1939, it was named and its main mechanism was explained by \textcite{Meitner:1939}.  It is a quantum many-body process of extreme complexity, with various parts of the process occurring at vastly different timescales.
The total time it takes from the moment a neutron initiates the formation of a compound nucleus 
until all final fission products have attained their equilibrium state after $\beta$-decay, 
 can be as long as the life-time of the Universe ${\cal O}(10^{17})$ sec., and spans almost 40 orders of magnitude relative to the time it takes a nucleon to cross a nucleus ${\cal O}(10^{-22})$ sec.

The compound system, formed by a
low-energy neutron~\cite{Hahn:1939} interacting with a target nucleus, evolves through many distinct stages. The first stage is a relatively slow quasi-equilibrium evolution, that lasts until the compound 
system~\cite{Bohr:1936} 
reaches the outer saddle-point at $\approx 10^{-14}$ sec.~\cite{Gonnenwein:2014}. During this stage, the nucleus, with an initial prolate intrinsic shape and axial symmetry, evolves into a nucleus with triaxial shape, and eventually into a reflection asymmetric and axially symmetric elongated shape near the outer fission barrier~\cite{Ryssens:2015},  these being clear examples of quantum phase transitions.
The second stage is a highly non-equilibrium evolution from 
saddle-to-scission~\cite{Bulgac:2016,Bulgac:2019c,Bulgac:2020}, when the primordial fission fragments (FFs) properties are defined within a duration of $\approx 5\times10^{-21}$ sec.~\cite{Gonnenwein:2014}. Even though this second stage is much faster than the first stage, it corresponds to rather slow dynamics, relative to the third stage (scission).
In this stage, the compound nucleus undergoes a relatively rapid ``divorce'' into two FFs, lasting $\approx 10^{-22}$ sec. This stage is also known in literature as the  neck rupture and is another example of a quantum phase transition.  
This is followed by a fourth stage in which the FFs are Coulomb accelerated during an interval of time of ${\cal O}(10^{-18})$ sec., when the FFs achieve 
an equilibration of their shapes.
In spite of the fact that the initial compound nucleus is a relatively cold system with a very small spin, the primordial FFs are very hot and have relatively large spins as well \cite{Bulgac:2019c,Bulgac:2021}. 
These highly excited FFs emit prompt neutrons for an interval of time up to about ${\cal O}(10^{-14})$ sec., followed by the emission of the majority of
prompt $\gamma$-rays  for an interval of time until about ${\cal O}(10^{-3})$ sec., which is further followed by much slower $\beta$-decays,
virtually to times of the order of the life-time of the Universe. 
Other processes are also possible, such as delayed neutron or $\gamma$ emission after $\beta$-decay, or fission after $\beta$-decay. 

With the exception of the saddle-to-scission configuration and the neck rupture, all the other stages of fission are relatively slow quasi-equilibrium processes. 
The fission dynamics after the compound nucleus reaches to the outer saddle point historically has typically been
described in terms of the potential energy surface of the nucleus, determined by its shape~\cite{Gonnenwein:2014,Ring:2004,Pomorski:2012,Schunck:2016},
not compatible with either experiment or recent microscopic studies~\cite{Bulgac:2016,Bulgac:2019c,Bender:2020}.
Approximately at the top of the outer saddle, the nucleus starts forming a barely seen ``wrinkle'', where eventually the neck between the two FFs is formed. 
This ``wrinkle'' tends to appear when the fissioning nucleus is at a young age, and, as our results below will show, their position hardly changes in time, at least until the FFs fully separate. This is similar to the formation of ``skin-wrinkles'' in 
humans.  At the top of the outer saddle the nucleus starts a relatively slow dissipative evolution 
towards scission~\cite{Bulgac:2016}.  In this period, the fissioning nucleus 
becomes more elongated and the neck becomes more and more pronounced. 
The nuclear fluid behaves as nuclear molasses, with a very small collective velocity~\cite{Bulgac:2016,Bulgac:2019c,Bulgac:2020},  while at the same time the
intrinsic temperature of the system gradually increases. The bond between the two partners slowly weakens until the neck, which was still keeping them together, reaches 
a critical small diameter of approximately 3 fm and ruptures, exactly where the initial ``wrinkle'' formed much earlier at the top of the outer saddle. 
This dramatic ``divorce'' of the two emerging FFs is  a rather short-time event. 
For \textcite{Brosa:1990} scission was the defining stage of fission, where the total kinetic energy (TKE) of the FFs is defined along with the average FF properties. 
The Brosa model assumes that the nucleus is a very viscous fluid, with a long neck that ruptures 
at a random position, and is widely invoked today in many models, even though it has no microscopic foundation and its claimed grounding in experimental data do not necessarily support a unique interpretation. Additionally, the Brosa random neck rupture model contradicts the theoretical assumptions of 
other popular approaches, such as the scission-point model of \textcite{Wilkins:1976}, 
where the FF formation is based on statistical equilibrium~\cite{Lemaitre:2015,Lemaitre:2021}, and the recently updated Brownian motion model~\cite{Albertsson:2020}. 
The drama of scission is followed by unavoidable debris characteristic of such dramatic separations, 
the scission neutrons, envisioned as early as 1939 by \textcite{Bohr:1939}.  Potentially other heavier fragments,
usually termed as ternary fission products~\cite{Vandenbosch:1973, Rose:1984, Wagemans:1991}, are created as well.  We relegate a  brief review of the history of scission neutrons as an online supplementary material \cite{supplement}. 


% Time series figure.
% Figure environment removed


% Neck dynamics
% Figure environment removed
%\vspace{0.5cm}


% Number of scission neutrons figure.
% Figure environment removed
%

%% Kinetic energy figure.
% Figure environment removed
%

%\section{Neck Formation and Decay Dynamics } 

As recognized by \textcite{Meitner:1939} nuclear fission must proceed from an initially relatively compact shape with positive Gaussian  curvatures
everywhere, to an intermediate shape, when the neck is formed, with a negative Gaussian  curvature,  
before the systems breaks up into two separate fission fragments, both with positive Gaussian  curvatures everywhere. 
(For any 3D smooth surface, the Gaussian curvature is defined as the product of the inverse principal curvature radii.) 
The nuclear shape changes from positive Gaussian curvature and axially symmetric, to axially asymmetric, back to axially symmetric ~\cite{Ryssens:2015}, 
to negative Gaussian curvature and axially symmetric shape, and eventually to shapes with positive overall Gaussian  curvatures, which are all examples of quantum phase transitions.  This has been demonstrated in a real-time description of nuclear fission dynamics, starting at the top of the outer barrier ~\cite{Bulgac:2016}.

In these simulations, we started by placing the initial compound nucleus 
near the top of the outer barrier in a very large simulation volume, in order to allow to the emitted nucleons enough time to decouple from the FFs 
after the neck rupture. 
We have performed a range of simulations for $^{235}$U(n$_{th}$,f), $^{239}$Pu(n$_{th}$,f), and $^{252}$Cf(sf),
using the nuclear energy density functional (NEDF)  SeaLL1~\cite{Shi:2018} in simulation volumes $48^2\times120$  and $48^2\times100$ fm$^3$, 
with a lattice constant of 1 fm, for further technical details see Ref.~\cite{Shi:2020}. 
The SeaLL1 NEDF is defined by only 8 basic nuclear parameters, each related to specific nuclear properties known for decades, and contains the smallest number of phenomenological parameters of any NEDF to date ~\cite{Shi:2018,Bulgac:2022c}.
We started the simulations at various points near 
the outer fission barrier, by constraining the system deformations $Q_{20}, Q_{30}$, see Refs.~\cite{Bulgac:2016,Bulgac:2019c,Bulgac:2020}, where 
one can find more details about how the final results vary with the choice of initial conditions. 
Our simulation volume of $48^2\times120$ fm$^3$ required the use of the entire supercomputer Summit, and we still could not follow the emission of nucleons for a long time, 
since the emitted nucleons are reflected back at the boundary relatively rapidly, see the lowest row of Fig.~\ref{fig:tseries}. In the transversal direction the reflection from the boundaries occurs 
earlier than along the fission axis, and that has effected some of the properties of the nucleons emitted perpendicular to the fission axis, however the effect on the total number of nucleons emitted is minor.  Additionally, the number of scission
nucleons is not large enough to noticeably alter the FF properties.


From here, we will concentrate on the dynamics of the neck formation and rupture, followed by the emission of nucleons,   all treated   
within the time-dependent density functional extended to superfluid fermionic systems~\cite{Bulgac:2019}. The integrated neck density, shown in Fig.~\ref{fig:neck}, is defined as
\begin{align}
 n_{\mathrm{neck},\tau}(t) = \int \!\!dxdy \, n_\tau(x,y,z_\mathrm{neck},t), \quad \tau = n, \, p,
\end{align}
separately for neutrons and protons, where the $z_{neck}$ is the position along the fission axis $Oz$ where the neck has the smallest radius.  The neck decays relatively slowly at scission, until its diameter reaches about 3 fm, after-which it undergoes a very rapid decay.  Different curves illustrated in the lower panel correspond to trajectories started at various initial conditions for the deformations $Q_{20}, Q_{30}$ close to the outer fission barrier. The time to reach scission can vary significantly, depending on the initial values of the deformations $Q_{20}, Q_{30}$ and on the NEDF used, typically ranging from 1,000 to 3,000 fm/c.


These microscopic results illustrate several points, which were unknown until now, due to the absence 
of any detailed quantum many-body simulation of fission dynamics.  First, the ``wrinkle'' in the nuclear density, where the neck is eventually formed and where the nucleus eventually scissions, is determined a long time before the nucleus reaches scission. Within the TDDFT framework, it is not random, and instead is determined by the quasiparticle energy spectrum and/or the shell structure around the outer fission barrier, where the trajectory was initiated. At the time when the neck reaches a critical diameter of $\approx 3 $ fm, the nuclear surface tension and the shape of the compound around the neck region, can no longer counteract the strong Coulomb repulsion between the preformed FFs causing the system to snap. 
One should keep in mind that as intrinsic temperature of the compound nucleus increases the surface tension decreases.    
The geometry of the nuclear shape changes dramatically at this stage, from exhibiting a neck region where the Gaussian curvature is negative, to two separated FFs with 
surfaces characterized by predominantly positive Gaussian curvatures. 

Second, the proton neck completes its rupture earlier than the neutron neck does, see lower panel in
 Fig.~\ref{fig:neck}, resulting in the neck being mostly sustained by the neutrons just before the full rupture. In this time interval, the number of neutrons per unit area at the neck varies by an order of magnitude.
The protons in the emerging FFs separate about 50-100 fm/c before the neutrons neck ruptures.   
Additionally, the integrated neutron and proton densities at $z_{\mathrm{neck}}$ asymptotically reach almost equilibrium values, after the neck ruptures.

 Third, the rupture is unarguably the fastest stage of the fission dynamics, starting from the capture the incident neutron and formation of the compound nucleus, until all fission products have emitted. The decay times are 15 fm/c and 35 fm/c for proton and neutron necks respectively, which are significantly faster processes than the time it takes the fastest nucleon to communicate any information or facilitate any kind of equilibrium between the two preformed FFs, which is at minimum $ \sim 160$ fm/c. 

 Fourth, the neck decay dynamics displays a clear universality irrespective of the initial conditions, as shown in the lower panel of Fig.~\ref{fig:neck}.  The time to full FF separation, the FF masses and charges, FF excitation energies, and the TKE all vary significantly~\cite{Bulgac:2019c,Bulgac:2020}, however, the neck dynamics is essentially unchanged for both proton and neutron components.

Last, the scission mechanism emerging from a fully microscopic treatment of the fission dynamics is totally at odds with previous models, including the Brosa random rupture model and the scission-point models. TDDFT extended to superfluid systems is the only theoretical framework so far in literature in which the scission is treated without any unchecked assumptions or fitting parameters, fitted to achieve agreement with data.
 



%\section{Emission of Scission Nucleons}

Consider the case of a long neck described in the Brosa model, whose existence was argued for in order to describe both TKE and FF yields distributions. If a long neck was to randomly rupture somewhere in the middle, the surface of the emerging FFs would initially have at least one rather long ``nose,'' where the Gaussian  curvature is negative. These noses would eventually have to be swallowed into the FFs, leading to a particle current lasting a noticeable time interval.  This involves a larger number of particles, thus more scission nucleons,  longer equilibration times, and would also cause the FFs deformations to relax at a later time. Before particle equilibration can be reached however, the presence of a long remnant neck in the FFs and its re-absorption would likely lead to a significant number of nucleons being emitted in the direction of motion of each fragment, an aspect which will discuss in more detail below.

In reality, the neck rupture is a very fast ``healing'' process of the nuclear surface in the neck region.  Unlike a gas in a punctured balloon, which would rapidly escape the enclosure, 
due to the presence of the nuclear ``skin'' and strong surface tension, the nucleus behaves as a fluid. 
The  surface tension quickly ``heals'' the ``wound,'' however a small fraction of matter manages to escape like a gas, with no droplet formation,
see Fig.~\ref{fig:tseries}. The potential condensation of this emitted gas cannot be described within the present framework. In Fig.~\ref{fig:tseries} we show several 
representative frames for neutrons and protons of the neck formation and emission of nucleons. 
The most remarkable features of this process are the following.

 As visible in Fig.~\ref{fig:neck} the proton neck completes its rupture before the neutron neck, however in two stages.  
 Immediately at the scission, which in Fig.~\ref{fig:tseries} is identified with the rupture of the proton neck, 
 a number of nucleons are emitted in the plane perpendicular to the fission axis (see second time frame).  After a sufficient time for nucleons to propagate from the neck to the nose of each FF, scission nucleons also appear propagating in front of each FF.  In 1984, it was suggested by \textcite{Madler:1985}, that the reabsorption of the neck stumps by the FFs, being a relatively rapid process, could act as a ``catapult'', which is more appropriately described as a slingshot, and ``push'' nucleons out of the front of the FFs.  It is also important to note, the three neutron clouds, two in front of the FFs and one transverse ring perpendicular to the fission axis, appears across all considered trajectories and nuclei hence far.  

 Remembering that the TKE is roughly 171-186 MeV, at the end of the full Coulomb acceleration the light and heavy FFs will have an average kinetic energy per 
 nucleon of about 1 MeV and 0.5 MeV respectively, which is significantly lower than the average kinetic energy of the scission neutrons, see Fig.~\ref{fig:ke}, given by 3.51 $\pm$ 0.25 MeV, 3.42 $\pm$ 0.27 MeV, and 2.67 $\pm$ 0.24 MeV for $^{236}$U, $^{240}$Pu, and $^{252}$Cf respectively.  
As a result, the FFs will never have a chance to catch up with them.  Additionally, the scission neutrons are essentially free, since their total interaction energy, estimated via~\cite{Bulgac:2018},
\begin{equation}
  E_n^{\mathrm{int}} =\int dV[ a_n n_n^{5/3} + b_n n_n^2 + c_n n_n^{7/3}] \ll E_n^{\mathrm{kin}},
 \end{equation}
comprises less than 1 percent of their kinetic energy.  

 The total number of emitted neutrons is about $0.30\pm 0.05$, $0.26 \pm 0.05 $, and $0.55\pm0.02$ per fission event for $^{236}$U, $^{240}$Pu, 
 and $^{252}$Cf respectively, which is a considerable portion (roughly 15 \%) of the total emitted prompt neutrons, see
 Refs.~\cite{Vandenbosch:1973,Wagemans:1991,Bulgac:2019c, Bulgac:2020,Stetcu:2021}. At the same time the number of emitted protons is about 
 two orders of magnitude lower, see Fig.~\ref{fig:nsci}.  
The nucleons are emitted in roughly equal numbers both transverse to the scission axis and in front of the FFs.


%\section{Conclusions} 

Here, we presented the first fully microscopic study of the neck in fission dynamics and the emission of scission nucleons, performed without any uncontrollable  
approximations and unchecked assumptions.  We have clarified several aspects of, what is arguably, the most non-equilibrium and fastest stage of nuclear fission dynamics.  Within TDDFT, the neck 
rupture is not a random process, as previously argued in various phenomenological models.  Additionally, it appears that the neck rupture has universal  dynamics, irrespective 
of nucleus considered or the initial conditions, beyond the top of the outer fission barrier.  This universality carries over to the emission of scission neutrons, whose signal always appears as three distinct clouds, one transverse to the fission axis and two in front of each FF, in almost equal proportions.  
% In this initial study we could not test if these aspects 
% depend on the excitation energy of the compound nucleus. 
The aspects of the neck dynamics discussed above, 
% since they will likely never be directly confirmed in experiments, 
can serve as a firm theoretical input for any semi-phenomenological approach to study FF properties~\cite{Becker:2013,Vogt:2009,Litaize:2012}. 
Additionally, we have established that an average of about
15 \% of the total neutrons are emitted at scission per fission event. The idea of scission neutrons, proposed by \textcite{Bohr:1939}, is almost as old as nuclear fission itself.  The existence of scission neutrons has been debated over the years ~\cite{Wilson:1947, Debenedetti:1948, Stavinsky:1959, Fuller:1962, Bowman:1962, Kapoor:1963, Skarsvaag:1963, Gavron:1974, Boneh:1974, Pringle:1975,Franklyn:1978, Franklyn:1986, Jorgensen:1988,  Milek:1988,Brosa:1992,Samant:1995,Hwang:1999,Carjan:2007, Rizea:2008,Carjan:2010,Carjan:2015, Guseva:2018, Vorobyev:2018, Carjan:2019}, see also \textit{Historical Note} in Ref.\cite{Wagemans:1991}, and their experimental confirmation is still an open question. This small fraction of scission neutrons carry noticeable kinetic energy, and what is more surprising is that their energy spectrum ranges up to roughly 18 MeV. Along with scission neutrons, a very small fraction of protons are emitted as well, and their fraction suggest a reasonable theoretical upper bound for the emission of $\alpha$-particles 
and other charged nuclei.
   
   \vspace{0.5cm}         
{\bf Acknowledgements} \\

We thank Kyle Godbey and Guillaume Scamps for many useful discussions.  The work of I.A. and I.S. was supported by the U.S. Department of Energy through the Los
Alamos National Laboratory. The Los Alamos National
Laboratory is operated by Triad National Security, LLC,
for the National Nuclear Security Administration of the U.S.
Department of Energy Contract No. 89233218CNA000001.
I.A. and I.S. gratefully acknowledges partial support and computational
resources provided by the Advanced Simulation and
Computing (ASC) Program.  The work of M.K. and A.B. was supported by the US DOE, Office of Science, Grant No. DE-FG02-97ER41014 and
also partially by NNSA cooperative Agreement
DE-NA0003841, and is greatly appreciated. 
This research used resources of the Oak Ridge
Leadership Computing Facility, which is a U.S. DOE Office of Science
User Facility supported under Contract No. DE-AC05-00OR22725. 

%\appendix* \label{secref:appendix}
%
%\section{Brief, and very likely incomplete, historical overview of scission neutrons} 
%
%
%Nuclear Fission, despite its discovery over 80 years ago, still lacks a complete microscopic description, a remarkable world record in quantum many body theory. As a phenomena, it is highly complicated, with various parts occurring at enormously different time scales; from the formation of the compound nucleus beyond the barrier, at O(10$^7$) fm/c (10$^{-16}$ s) \cite{Bulgac:2019}, to the evolution of the system from a compact object into two (or more) fragments, at O(10$^4$) fm/c (10$^{-19}$ s) \cite{Bulgac:2019}, and finally to fully accele
%rated fragments and their subsequent emission of neutrons and gammas, beginning at O(10$^5$) fm/c (10$^{-18}$ s) \cite{Gonnenwein:2014}.  The middle part of the process, the formation of the fission fragments (FFs) or scission, occurs at a time scale too fast to be probed directly by experiment; however, the characteristics of the fission fragments (FFs) at scission, quantities such as FF masses, charges, excitation energies, total kinetic energy (TKE), and spin distributions, are highly important in determining the properties of neutrons and gammas emitted at later times, which can be measured experimentally.  Additionally, at this stage, the system undergoes highly non-equilibrium evolution, and hence cannot be described via statistical models.  As a result, predictions for scission FF observables are typically computed using microscopic theories.  Currently, there exists only one unified approach for computing all important average FF observables (at scission and shortly after) using realistic initial conditions, namely the Time-Dependent Superfluid Local Density Approximation (TDSLDA) or equivalently Time-Dependent Density Functional Theory extended to superfluid systems \cite{Jin:2021}.
%
%TDSLDA has already previously been applied to great effect in providing new insights into many long-lasting questions in fission.  Such questions include: is the adiabatic approximation valid for describing the evolution of the compound till scission  \cite{Bulgac:2019}? What are the characteristics of the FF intrinsic spin distributions \cite{Bulgac:2021}? What is the nature of their correlations  \cite{Bulgac:2022}?  And more \textcolor{red}{(might want to cite additional references for the "and more" claim)}.  In this study, TDSLDA is used to investigate the properties of SNs, the first time such a topic has been addressed in a fully microscopic manner.  
%
%Surprisingly, the idea of SNs is almost as old as the discovery of fission itself, being first considered in 1939.  At the time, Bohr and Wheeler conjectured three sources for neutron emission during fission: first delayed neutrons occurring on the time scale of seconds, second neutrons evaporated from the FFs as a result of their excitation, now known as prompt neutrons, and finally, and much more speculatively, neutrons formed due to the neck rupture \cite{Bohr:1939}:\\
%\emph{We consider briefly the third possibility that the neutrons in question are produced during the fission process itself.  In this connection attention may be called to observations on the manner in which a fluid mass of unstable form divides into two smaller masses of greater stability; it is found that tiny droplets are generally formed in the space where the original enveloping surface was torn apart.}
%
%
%For a while, the idea remained dormant.  In the late 40s, the first experiments investigating the directional properties of neutrons produced from the fission of $^{235}$U(n,f) were conducted, with a consensus that the experimental results were consistent with the assumption of isotropic evaporation of neutrons from fully accelerated FFs \cite{Wilson:1947,Debenedetti:1948} \textcolor{red}{(we should consider a acronym for such neutrons, due to how often they are referenced in both the literature and in this paper)}.  Any brief consideration for SNs were quickly dismissed \cite{Debenedetti:1948}. As experiments were refined in the 50s conclusions remained the same \cite{Fraser:1952,Fraser:1954}.  It was not until the 60s, starting with experiments conducted by Bowman et al. \cite{Bowman:1962}, that the idea of SNs would re-emerge with considerable momentum. 
%
%\textcolor{green}{I left off editing here before I went on vacation}. 
%
%In 1962, his group computed the angular distributions of neutrons emitted from the spontaneous fission of $^{252}$Cf, observing deviations from the "isotropic" hypothesis (which stated all neutrons would be emitted isotropically in the intrinsic frame of fully accelerated FFs).  Using a model for SNs proposed by Stravinsky in 1959 \cite{Stavinsky:1959} and Fuller in 1962 \cite{Fuller:1962}, who extended the work of Harper in 1959  \cite{Halpern:1959} from scission alphas to scission neutrons, Bowman et al. predicted that SNs would comprise of $\sim$ 10\% of prompt neutrons emitted during fission \cite{Bowman:1962}. Such models assume SNs are treated as being isotropically emitted from \textcolor{blue}{the region of the neck rupture}. In particular, Fuller proposed the sudden rise of the nuclear potential at the neck during scission would lead to an expulsion of nuclear matter, and estimated scission neutrons would comprise 10-20\% of neutrons emitted, in the case of the spontaneous fission $^{252}$Cf, with each carrying an average of 2-6 MeV of energy \cite{Fuller:1962}.  Subsequent experiments confirmed the findings of Bowman et al. \cite{Kapoor:1963,Skarsvaag:1963}, with one proposing alternative explanations to account for inconsistencies between experimental results and the isotropic emission hypothesis, such a pre-scission neutron emission after the nucleus crosses the saddle \cite{Kapoor:1963}.  At this point, it is very important to re-stress experiments can only measure properties of fission after the FFs are fully accelerated.  This means all specific conclusions about the emission mechanism of neutrons are extremely model dependent, partially explaining the inconsistency in results concerning SNs.
%
%From then, a significant number of subsequent experiments were carried out to confirm the existence of and/or estimate the effects of SNs \cite{Gavron:1974, Pringle:1975, Franklyn:1978, Franklyn:1986, Jorgensen:1988, Samant:1995, Hwang:1999, Petrov:2009, Vorobyev:2010, Sokolov:2010, Guseva:2018, Vorobyev:2018}.  Conclusions were mixed, with estimates placing the number of SNs as low as 1 \% and as high as 15 \%. Of importance, the vast majority assumed a simple model, namely all SNs are emitted isotropically from the laboratory frame, typically with low kinetic energy, despite the great variance in  the theoretical models over time.  
%In spite of the significant differences, almost all models can be classified into three distinct classes.
%
% The sudden change in the nuclear potential as the nucleus evolves from the saddle point to scission and past the neck rupture, where the number and kinetic energies of the SNs depend on the time it takes the neck to rupture, after which all neutrons are either emitted isotropically, or (in some models) perpendicular to the scission axis, from the lab frame \cite{Boneh:1974, Serot:2000, Carjan:2007, Milek:1988}.  
% 
% Cases of fission where the neck is high elongated form satellite droplets between the two fragments, which could comprise of neutrons or both neutrons and protons \cite{Brosa:1992}.
% 
% The catapult mechanism; as the neck ruptures the tails of the densities of FFs are "snatched" inside, travel through the nucleus with high energy, and emerge on the other-side as SNs \cite{Madler:1985}. 
%
%For class (i) the majority of SNs would have high kinetic energy, relative to PNs, and be emitted isotropically from the center of the FFs, all with respect to the lab frame.  Class (ii) is similar, except the kinetic energies of the SNs would be significantly lower.  This class of models, specifically for SNs and not ternary fission, is less popular than the other two.  For class (iii) the majority of the SN signal would be seen as polar emission (in the direction of the FFs) with high kinetic energy, all with respect to the lab frame.  All of the above proposed mechanisms have distinct features, and hence unique signals for determining which is primarily responsible for SNs.  Based on our microscopic results (discussed below), we believe reality will be a combination of all three.
%
%It is notable that in the \textcite{Wagemans:1991} 1991 monograph, where surprisingly the names of Meitner and Frisch~\cite{Meitner:1939}, conned the term nuclear fission and correctly identified the fission mechanism as the competition between nuclear surface tension and Coulomb repulsion,  where not mentioned in the introductory historical note, and where only the work of \textcite{Bohr:1939} inspired by Meitner and Frisch is quoted, it is concluded that perhaps only about 1\% of 
%the total emitted neutrons can be attributed to scission.   
%
%It is important to also highlight the work of Carjan and his collaborators, who in recent times modeled SNs by simulating the neck rupture via the Time-Dependent Schr$\rm \ddot{o}$dinger Equation in polar coordinates (TDSE2D) \cite{Rizea:2008, Carjan:2010, Carjan:2015, Capote:2016, Carjan:2019}.  Of greatest significance, they found the SN angular distributions and kinetic energy distributions exhibit similar properties to PFNs for $\rm ^{235}U(n_{th},f)$.  If proven true, all previous predictions of the contribution of SNs to the PFN spectra, which all assume vastly different angular and kinetic energy distributions for the SN component relative to the remainder of the PFN component, would have substantially underestimated the number of SNs.  Additionally, popular statistical models of neutron emission, such as CGMF and FREYA, do not include SNs \cite{Talou:2021,Randrup:2009}.  However, CGMF does include an anisotropy factor (number of neutrons emitted at $0^{\circ}$ over $90^{\circ}$ with respect to the axis of scission), which would partly account for the effect of SNs on the PFN angular distributions.
%
%Tangential related, although not the focus of this work, is the question of charged particle emission at scission or ternary fission.  Unlike SNs, there has been certainty that charged particles are formed during scission as long ago as the late 40s when the first alpha particles were detected \cite{Demers:1946,Farwell:1947}, and the subsequent realization (shortly after) that alphas are primarily emitted at $\sim$ $80^{\circ}$  with respect to the LFF \cite{Wollan:1947,Marshall:1949,Titterton:1951,Allen:1950,Feather:1947,Nobles:1962}.  Additional charged particles, such as Tritium \cite{Albenesius:1959,Watson:1961,Marshall:1966}; and heavier charged particles, such as $^{6}$He \cite{Goward:1949,Whetstone:1965,Dakowski:1967,Cosper:1967,Krogulski:1969,Chwaszczewska:1967}, and isotopes of Li and Be \cite{Vorobiev:1969, Blocki:1969}; were detected later.  Of importance, because prompt alphas are primarily emitted at $90^{\circ}$ with respect to the fission axis, the majority must originate at the neck, and consequently they can be used as good probes for scission configurations  \cite{Fraenkel:1967}. \textcolor{red}{(Need to find sources for even heavier charged particles detected today).}
%


%%%%%%%%%%%%%%%%%%%%%%%%%%%%%%%%%%%%%%%%%%%%%

% These are needed to avoid a babel error.
\providecommand{\selectlanguage}[1]{}
\renewcommand{\selectlanguage}[1]{}

\bibliography{local_fission}%,SciNeuIntroBib}

\end{document}
