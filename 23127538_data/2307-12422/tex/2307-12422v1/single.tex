\section{The Single Pool Protocol}\label{sec:single}
In this section, we provide the definition of a pool in a blockchain system and we describe the rules of a single pool in \textsc{FruitChain}, denoted by $\single$, that includes all the parties. In this pool, all the parties ask the random oracle, but only the pool leader asks the longest chain, the fruit set and the transaction oracle, and determines the instance that will be used by all the parties for the queries to the random oracle.  In the next section, we will prove that joining this ``centralised'' pool is an EVP.  
    
\subsection{Definition of a Pool}\label{subsec:definition_pool}
%
Intuitively, a pool of some protocol $\Pi$ comprises a subset of parties in $\Pi$ that collaborate by interacting internally according to some well-specified communication pattern and guidelines. Formally, we provide the following definition.

\begin{definition}[Pool]\label{def:pool}
Let $\Pi$ be a blockchain protocol with parties $\party{1},\ldots,\party{n}$. A \emph{pool of $\Pi$} is a quadruple $\langle\mbf{V},\mbf{E},\mc{F}_\msf{comm},\tilde{\Pi}\rangle$, where
\begin{itemize}
    \item $\mbf{V}\subseteq\{\party{1},\ldots,\party{n}\}$ is a subset of parties in $\Pi$.
    \item $\mbf{E}\subseteq\{(\party{i},\party{j})|\party{i},\party{j}\in\mbf{V}\}$ is a subset of pairs of parties in $\mbf{V}$ that determines the available simplex communication connections among parties in $\mbf{V}$.
    \item $\mc{F}_\msf{comm}$ is a communication functionality that supports the parties' interaction w.r.t. $\mbf{E}$.
    \item $\tilde{\Pi}$ is a protocol executed by parties in $\mbf{V}$ that captures the execution instructions for the parties in $\mbf{V}$. In addition, $\tilde{\Pi}$ allows parties to have access to the Diffuse functionality (cf. Subsection~\ref{subsec:framework_execution}), hence to the messages exchanged during the execution of $\Pi$.%
\end{itemize}
\end{definition}

\subsection{A Single Pool of $\protocolevp$}\label{subsec:single_pool}
%
Given Definition~\ref{def:pool}, we specify a \emph{single pool of $\protocolevp$ with leader $\party{L}$} as the quadruple $\langle\mbf{V},\mbf{E},\mc{F}_\msf{auth}(\mbf{E}),\single\rangle$ where
\begin{itemize}
    \item $\mbf{V}:=\{\party{1},\ldots,\party{n}\}$, i.e., all the parties collaborate. For some $i^*\in[n]$, we have that $\party{L}=\party{{i^*}}$.
    \item $\mbf{E}:=\{(\party{L},\party{i}),(\party{i},\party{L})\}_{i\in[n]\setminus\{i^*\}}$. Namely, the pool leader $\party{L}$ can communicate with every other party and vice versa. Note that the non leader parties do not communicate with each other.
    \item $\mc{F}_\msf{auth}(\mbf{E})$ is the \emph{message authentication functionality} w.r.t. $\mbf{E}$, defined in the spirit of~\cite{Canetti04} as follows:
\begin{itemize}
    \item[$\blacktriangleright$] Upon receiving $(\textsc{Send},\party{j},M)$ from $\party{i}$, if $(\party{i},\party{j})\in\mbf{E}$, then $\mc{F}_\msf{auth}(\mbf{E})$ sends the message $(\textsc{Sent},\party{i},M)$ to $\party{j}$.
\end{itemize}
Similar to~\cite{Canetti04}, $\mc{F}_\msf{auth}(\mbf{E})$ can be implemented via digital signatures and some setup assumption, such as the presence of a certification authority or the out-of-band exchange of verification keys among the parties in the pool. 
   %
   \item $\single$ is executed by $\party{1},\ldots,\party{n}$ and defines each party's deviation from the protocol $\protocolevp$. During the execution of $\single$, $\party{L}$ takes over the cost for setting up an instance to the random oracle in each round. Then, all parties contribute to the fruit and block mining effort w.r.t. this instance. Upon successful mining of a block, $\party{L}$ shares the rewards that correspond to the fruits included in this block according to the guidelines. The protocol $\single$ is formally introduced in the following subsection.
\end{itemize}



\subsection{Protocol Description}\label{subsec:single_description}
First, we describe an additional oracle that $\single$ utilizes and provide its overview. 
\par 
The \emph{light transaction verification oracle}  $\mc{O}_\msf{ltx}$: receives as input a record of transactions $\record$ and a transaction $\tx$. It outputs $1$ if the transaction $\tx$ is included in the record $\record$ and $\tx$ is valid \footnote{The transaction validity is defined in a protocol-specific manner.}, and $0$ otherwise. The party can make up to $1$ query per round and the cost of a single query is $C_\msf{ltx}$\footnote{This oracle reflects a procedure similar to the ``simplified payment verification'' \url{https://wiki.bitcoinsv.io/index.php/Simplified_Payment_Verification}. $C_\msf{ltx}$ is significantly lower compared to  $C_\msf{tx}$. }.\\[2pt]
%
\indent\emph{Overview of $\single$.}
During each round, the pool leader asks $\mc{O}_\msf{lc},\mc{O}_\msf{fs}$, and $\mc{O}_\msf{tx}$, and creates the instance that will be used for the queries to the random oracle. Then, it sends this instance to the pool members. The pool leader and the other members ask the random oracle $q$ queries when they are activated. When a fruit or a block is produced, they send it to the Diffuse functionality (at most one block per round). 
\par Both the pool leader and the other pool members count the cost that the pool leader should incur for the oracles $\mc{O}_\msf{lc},\mc{O}_\msf{fs},\mc{O}_\msf{tx}$. When a block that uses the instance sent by the pool leader has been diffused, the pool leader creates a payment transaction in the next round. The payments are as follows: if the cost that the pool leader incurred for creating the instances since the last block is higher than the block's rewards (which is equal to the number of fruits multiplied by the fruit reward $\rf$), then the pool leader holds all the rewards. If the block's rewards are higher, then the pool leader subtracts the cost and shares the remaining rewards equally among all the members of the pool including itself. Note that the members will check if the payments have been computed correctly via the light transaction verification oracle \footnote{A similar countermeasure has also been used in P2pool
\url{https://bitcoinmagazine.com/technical/p2pool-bitcoin-mining-decentralization}}.
In addition, to prevent block withholding attacks (cf. \cite{ Rosenfeld2011AnalysisOB,7163020}), both the pool leader and the other pool members check if the diffused fruits and blocks use the instance sent by the pool leader; if not, they abandon the pool.
% 
The $\single$ protocol is presented in  Figures~\ref{fig:single_leader} and~\ref{fig:single_other}.
%
%

% Figure environment removed
\normalsize
% Figure environment removed
\normalsize
%
\begin{remark}\label{rem:payment}
As explained in Figure~\ref{fig:single_leader}, the pool leader carries out the payments of the other parties by including them in a special transaction $\tx_T$. Since the exact payment method does not affect our analysis, we do not provide details on the format of $\tx_T$. In practice, each party $\party{i}\neq\party{L}$ could provide $\party{L}$ with a fresh public key $\msf{pk}_i$, and $\party{L}$ would include a payment linked to $\msf{pk}_i$ in $\tx_T$.
\end{remark}