\section{$\single$ as an EVP}\label{sec:equilibrium_proof}

In this section, we provide our main result. Namely, that the protocol $\single$ is an EVP according to Definition~\ref{def:EVP}. In our theorem statement, we quantify over a class of adversaries whose strategy does not result in the mining of blocks that are ``almost'' empty (i.e., they contain only the special payment transaction $\tx_T$). As we shortly explain, restricting to this class is meaningful and does not harm the generality of our result. In particular, we define the following type of adversary.

For some round $T$, let $\{\mbf{V}_1,\ldots,\mbf{V}_{K_T}\}$ be the partition of the party set $\{\party{1},\ldots,\party{n}\}$ such that for every $i\in[K_T]$, the parties in $\mbf{V}_i$ form a pool according to Definition~\ref{def:pool} (trivially, if $\mbf{V}_i$ is a singleton, then the single party in $\mbf{V}_i$ acts on its own). We say that an adversary $\mathcal{A}$ that controls a coalition $\corrupt\subset\{\party{1},\ldots,\party{n}\}$ is \emph{$\mc{O}_\msf{tx}$-respecting} if for every round $T$ and every $\mbf{V}_i$, $i\in[K_T]$, there is at least one party in $\mbf{V}_i$ that asks the transaction oracle $\mc{O}_\msf{tx}$ during $T$. 

We stress that if we lift the above restriction and quantify over all adversaries in our theorem statement, then using similar proof techniques, we can show that a variant of $\single$ where the leader never queries $\mc{O}_\msf{tx}$, ignores its input transactions, and sets the record as the singleton that includes only the special payment transaction (cf. Remark~\ref{rem:variant} for the variant description details), is an EVP. This strategy profile is related to the verifier's dilemma introduced in~\cite{LuuTKS15}, according to which miners are motivated to skip verification of transactions when the cost is significant.  Observe that this variant of $\single$ forms again a single pool with all the parties (which means that is again completely centralized), but violates liveness\footnote{A blockchain protocol satisfies liveness, if every transaction that has been issued and diffused by an honest party  will be included eventually in the ledger with $1-\negl(\secpar)$ probability~\cite{backbone}}. In our main theorem, we do not follow this direction, as for this variant to be functional, it is necessary that there is no external observer that can check the validity of the chain and affect the profit of the parties. This is not true in practice, as external users can easily detect that the blocks are empty, harm the reputation of the system, and thus affect the price of the currency the parties of the pool earn.% (cf. Remark~\ref{rem:variant} for details). 

Our main theorem statement relies on three reasonable assumptions: (i) the expected rewards per random oracle query are higher than the cost of the query and the cost needed to form the instance for the query (recall that the rewards and the costs are in the same unit), (ii) $\pb=\Omega(\frac{1}{nq})$, and (iii) $\pf<\tfrac{1}{2}$. Moreover, the multiplicative approximation factor is zero. Besides, the three dominant terms in the additive approximation factor are justified as follows:
\begin{enumerate}
    \item[(a)] The term $O\big(\rounds(n-1)\big)\pf\reward$ (a small fraction of the adversary's expected total rewards) appears because the EVP notion compares the exact profit of the adversary in the two executions with $1-\negl(\secpar)$ probability.
    \item[(b)] The term $O\big(\log\secpar\sqrt{\rounds}\big)\cost_\msf{lc}$ (the difference between $\umax(\exec{})$ and $\umin(\honexec)$ in the cost of asking the longest chain oracle) is due to the same reason as above.
   \item[(c)] The term $O\big( \rounds(n-1)\big)\cost_\msf{ltx}$ (the difference between $\umax(\exec{})$ and $\umin(\honexec)$ in the cost of asking the light transaction oracle) derives from the fact that in $\umax(\exec{})$, the parties check that they have got paid correctly by the pool leader. Typically, the cost for this check is relatively small.
\end{enumerate}

\begin{theorem}
\label{th:equilibrium}
%Assuming that the adversary cannot participate in the mining process without asking the transaction oracle $\mc{O}_\msf{tx}$ %\footnote{We can enforce this assumption by modifying the transaction and the random oracle as follows: The transaction $\mc{O}_\msf{tx}$ retains a flag that is initialized as negative. The flag becomes positive when there is a round when the pool leader does not ask this oracle. If this happens the transaction oracle sends a message to the random oracle  $\mc{O}_\msf{tx}$ that the flag became positive and the random oracle does not allow the pool leader to ask any queries until the end of the execution.}, following 
Let (i) $\pf\reward>\tfrac{\cost_\msf{lc}+\cost_\msf{fs}+\cost_\msf{tx}}{(1-\frac{\log\secpar}{\sqrt[4]{n}}) \sqrt{n}q}+\cost_\msf{ro}$, (ii) $\pb=\Omega(\frac{1}{nq})$, and (iii) $\pf<\tfrac{1}{2}$. Then, for any $\delta\in \big[\tfrac{\log\secpar}{\sqrt[4]{\rounds n}},1\big)$, the $\single$ protocol is an $(n-1,0,\epsilon')$-EVP according to the utility profit, where
\begin{equation*}
\begin{split}
\epsilon'=&\Big(\big(\tfrac{\log\secpar}{\sqrt{\rounds n}}+\delta\big)\rounds+\log^2\secpar\big(1+\tfrac{1}{\pf}\big) -\big(\tfrac{\log^3\secpar}{\sqrt{\rounds n}}+1+\delta\big)\Big)(n-1)q\pf\reward+\\
&+\tfrac{n-1}{n}\Big((2\log\secpar)\sqrt{\rounds}+1-\tfrac{\log\secpar}{\sqrt{\rounds}}\Big)(1-(1-\pb)^{nq})\cost_\msf{lc}+\\
&+\big(1+\tfrac{\log\secpar}{\sqrt{\rounds}}\big)\rounds(1-(1-\pb)^{nq})(n-1)\cost_\msf{ltx}+\\
&+\tfrac{\log^2\secpar}{n}(\cost_\msf{fs}+\cost_\msf{tx}),
\end{split}    
\end{equation*}
 w.r.t. every   $\mc{O}_\msf{tx}$-respecting adversary $\mc{A}$ and every $\rounds$-admissible environment $\mc{Z}$ that activates the pool leader first in each round.
\end{theorem}
%
%
\begin{proof}
We will assume that the adversary has corrupted a set $\corrupt$ with $n-1$ parties and can deviate from the $\single$ protocol arbitrarily. We will prove that for every $\rounds$-admissible environment $\mc{Z}$ that activates the leader first in each round and for every PPT adversary $\mc{A}$ that controls $\corrupt$, it holds that 
\begin{equation*}\label{EVP}
\umax(\exec{}) \leq \umin(\honexec) + \epsilon \cdot \mid \umin(\honexec) \mid +\epsilon'   
\end{equation*}
with overwhelming probability in the security parameter $\secpar$.
%
\par
Note that we do not quantify over adversaries that control a set $\corrupt'$ with $t'$ parties, where $t'< n-1$. The reason is that for every adversary $\mc{A}'$ that corrupts $\corrupt'$, we can consider an adversary $\mc{A}''$ that corrupts a set $\corrupt''\supset\corrupt'$ with exactly $n-1$ parties and instructs the parties in $\corrupt'$ to deviate exactly like $\mc{A}'$ and the other $n-1-t'$ parties in $\corrupt''\setminus\corrupt'$ to follow the $\single$ protocol. 
%
\par
At this point, we will describe all the possible deviations of the corrupted parties in the set $\corrupt$ as instructed by $
\mc{A}$. The set $\corrupt$ can include either (i) the pool leader and $n-2$ other members of the pool, or (ii) $n-1$ members of the pool and not the pool leader.

 A round $T$ will be called \textit{payment round} if (i) the array of all blocks that were diffused during the previous round $T-1$ was not empty, and (ii) the array of all the blocks and the set of all the fruits diffused during the previous round $T-1$  do not contain a  block $\hat{\block}:=\langle\langle \hat{h}_{-1},\hat{h}_f,\hat{\nonce},$ $\hat{\msf{dig}},\hat{\record},\hat{h}\rangle,\hat{\mbf{F}}\rangle$
  or a fruit $\hat{\mbf{f}}:=\langle \hat{h}_{-1},\hat{h}_f,\hat{\nonce},\hat{\msf{dig}},\hat{\record},\hat{h}\rangle$, respectively, such that $(\hat{\previous},\hat{h}_f,\hat{\msf{dig}},\hat{\record})\neq(inst_1,inst_2,inst_3,inst_4)$.
%
%
\iffalse
\noindent\textbf{1. $\corrupt$ includes $n-1$ members of the pool and not the pool leader:}
\fi

The possible deviations that $\mc{A}$ can perform are any combination of the following strategies.
\begin{enumerate}
    \item[(D1)] $\mc{A}$ instructs a subset of the non leader parties in $\corrupt$ to deviate from step (4) of the $\single$ protocol for one or more rounds, by ignoring the instance received from the pool leader and creating a different instance. Note that this includes the case where the adversary instructs some adversarial parties to abandon the pool. The meaningful possible deviations that we should examine  are the following: 
    \begin{enumerate}
    \item[(i)] the adversarial party ignores the record $\record$ received from the pool leader and updates $inst_4$ with a new record including a transaction that makes the adversarial party as the recipient of the rewards. 
    \item[(ii)] the adversarial party updates $inst_1$ with a hash value of a block different from $h_{-1}$ received from the pool leader. This reflects the scenario where the adversarial party creates a fork.
    \item[(iii)] the adversarial party updates $inst_2$ with the hash of a block that is different from the hash value, $h_f$, of the block received from the pool leader.
    \item[(iv)] the adversarial party updates $inst_3$ with a digest of a fruit set that is different from $\digest(\mbf{F}_\msf{rec})$ received from the pool leader.
    \item[(v)] the adversarial party does not update one or more $inst_i$, $i\in[4]$.
    
    \end{enumerate}
    \item[(D2)] $\mc{A}$ instructs a subset of the non leader parties in $\mathbf{C}$ to ask $\mc{O}_\msf{ltx}$ no queries during one or more payment rounds.
    \item[(D3)] $\mc{A}$ instructs a subset of the parties in $\mathbf{C}$ to ask the oracle $\mc{O}_\msf{ro}$ fewer than $q$ queries during one or more rounds.
    \item[(D4)] $\mc{A}$ instructs a subset of the parties in $\mathbf{C}$ not to send the fruits or the blocks it produces to the Diffuse Functionality for one or more rounds (this is related to block withholding attacks, cf. \cite{ Rosenfeld2011AnalysisOB,7163020}).

    
    \item[(D5)] $\mc{A}$ instructs a subset of the parties in $\mathbf{C}$ to delay arbitrarily to send the fruits or the blocks it produces to the Diffuse Functionality for one or more rounds.
    \item[(D6)] $\mc{A}$ instructs a subset of the parties in $\mathbf{C}$ to abandon the pool and create a new pool that follows different instructions from the $\single$ protocol.
    \item[(D7)] $\mc{A}$ instructs a subset of the parties in $\mathbf{C}$ to remain in the pool even if it receives a block $\hat{\block}:=\langle\langle \hat{h}_{-1},\hat{h}_f,\hat{\nonce},$ $\hat{\msf{dig}},\hat{\record},\hat{h}\rangle,\hat{\mbf{F}}\rangle$ or a fruit $\langle \hat{h}_{-1},\hat{h}_f,\hat{\nonce},\hat{\msf{dig}},\hat{\record},\hat{h}\rangle$ such that $(\hat{\previous},\hat{h}_f,\hat{\msf{dig}},\hat{\record})\neq(inst_1,inst_2,inst_3,inst_4)$.
    \item[(D8)] $\mc{A}$ instructs a subset of the parties in $\mathbf{C}$ to abandon the pool and follow the $\protocolevp$ protocol.
    \item[(D9)] If $\mathbf{C}$ includes the pool leader, $\mc{A}$ instructs the pool leader to ask no query to $\mc{O}_\msf{fs}$ for one or more rounds.
    \item[(D10)] If $\mathbf{C}$ includes the pool leader, $\mc{A}$ instructs the pool leader to ask no query to $\mc{O}_\msf{tx}$ for one or more rounds.
    \item[(D11)] If $\mathbf{C}$ includes the pool leader, $\mc{A}$ instructs the pool leader to ask no query to $\mc{O}_\msf{lc}$ for one or more payment rounds.
    \item[(D12)] If $\mathbf{C}$ includes the pool leader, $\mc{A}$ instructs the pool leader for one or more payment rounds to  create a special transaction $\tx_T$ that pays the party that does not belong to $\corrupt$ a smaller amount than what described in 3(c) of $\single$ for the pool leader (cf. Figure~\ref{fig:single_leader}). 
   
\end{enumerate}
%
Note that deviations D1, D2 apply only to the corrupted non leader parties, deviations D3-D8 apply to all parties in $\corrupt$ and deviations D9-D12 apply only to the corrupted pool leader. 

First, we provide a lower bound that $\umin(\honexec)$ achieves with overwhelming probability.

\begin{claim}\label{claim:H}
If (i) $\pf\reward\geq\frac{\cost_\msf{lc}+\cost_\msf{fs}+\cost_\msf{tx}}{(1-\frac{\log\secpar}{\sqrt{n}}) nq}$ and (ii) $\pb=\Omega(\frac{1}{nq})$, then it holds that
\begin{equation*} 
\begin{split}
&\Pr\big[\umin(\honexec)\geq\big(1-\tfrac{\log\secpar}{\sqrt{\rounds n}}\big)(\rounds-\log^2\secpar)(n-1)q\pf\reward-\\
&-\tfrac{n-1}{n}\big(1+\tfrac{\log\secpar}{\sqrt{\rounds}}\big)\rounds(1-(1-\pb)^{nq})(\cost_\msf{lc}+n\cost_\msf{ltx})-\\
&-\big(\tfrac{n-1}{n}\rounds+\tfrac{\log^2\secpar}{n}\big)(\cost_\msf{fs}+\cost_\msf{tx})-\rounds (n-1)q\cost_\msf{ro}\big]\geq\\
&\geq 1-\negl(\secpar).
\end{split}
\end{equation*}
\end{claim}
%
\textit{Proof of Claim~\ref{claim:H}. }
We say that a block $\block$ is \emph{profitable} if the rewards that derive from the fruits included in $\block$ are higher than the pool leader cost of asking $\cost_\msf{lc},\cost_\msf{fs},\cost_\msf{tx}$ during the mining of $\block$. By the description of $\single$, if $\block$ is profitable, then $\party{L}$ shares the total profit of $\block$ evenly among all $n$ parties of the single pool. Otherwise, $\party{L}$ uses the rewards to cover (part of) the cost of $\block$ while the other parties receive no profit for $\block$.

We show that the probability that a block $\block$ is profitable under the execution $\honexec$ is overwhelming. 
Let $\rho$ be the number of rounds elapsed for mining $\block$. Since the number of the queries the single pool makes per round is $nq$, the number of fruits mined during the mining of $\block$, $Z_0$, follows $\msf{Bin}(\rho nq,\pf)$. By the Chernoff bounds (cf. Appendix~\ref{app:chernoff}),
%
\begin{equation*}
\Pr\big[Z_0<(1-\tfrac{\log\secpar}{\sqrt{n}})\rho nq\pf\big]\leq e^{-\frac{\log^2\secpar}{2n}\rho nq\pf}=\negl(\secpar).    
\end{equation*}
%
Thus, with $1-\negl(\secpar)$ probability, the rewards w.r.t. $\block$ are at least $(1-\frac{\log\secpar}{\sqrt{n}})\rho nq\pf\reward$. Besides, the leader cost for $\block$ is $\cost_\msf{lc}+\rho\cost_\msf{fs}+\rho\cost_\msf{tx}$. 
Since $\pf\reward\geq\frac{\cost_\msf{lc}+\cost_\msf{fs}+\cost_\msf{tx}}{(1-\frac{\log\secpar}{\sqrt{n}}) nq}$ and $\rho\geq 1$, we have that with $1-\negl(\secpar)$ probability, it holds that
%
\[(1-\tfrac{\log\secpar}{\sqrt{n}})\rho nq\pf\reward\geq\rho(\cost_\msf{lc}+\cost_\msf{fs}+\cost_\msf{tx})\geq\cost_\msf{lc}+\rho\cost_\msf{fs}+\rho\cost_\msf{tx},\]
%
i.e., $\Pr[\block\mbox{ is profitable}]\geq1-\negl(\secpar)$.\\[2pt]
%
%-----------------------------------------------------------
%
Next, we define the following random variables. Let $Z$ be the number of mined fruits during $\honexec$ for the first $\rounds-\log^2\secpar$ rounds and $W$ be the number of rounds that at least one block was mined (i.e., the number of calls to $\mc{O}_\msf{lc}$, $\mc{O}_\msf{tlx}$). Since $nq$ mining queries are made per round, $Z\sim\msf{Bin}((\rounds-\log^2\secpar) nq,\pf)$. Besides, the probability that at least one block is produced in some round is $1-(1-\pb)^{nq}$, so $W\sim\msf{Bin}(\rounds,1-(1-\pb)^{nq})$.

Let $t\_reward$ be the total rewards of the single pool and $l\_cost$ be the leader costs in terms of queries to $\mc{O}_\msf{lc},\mc{O}_\msf{fs},\mc{O}_\msf{tx}$ that are shared among the members of the single pool. Let $c\_reward$ be the rewards of the coalition $\corrupt$ and $c\_cost$ be the additional cost that $\corrupt$ incurs besides its share of $l\_cost$.

As shown above, the probability that some block is not profitable is $\negl(\secpar)$. So, by the union bound, the probability that all the blocks are profitable in $\honexec$ is at least $1-\rounds\negl(\secpar)=1-\negl(\secpar)$. As the coalition consists of $n-1$ parties and when all blocks are profitable each party receives an equal profit share, we have that
%
\begin{equation}\label{eq:c_reward}
\Pr\big[c\_reward=\tfrac{n-1}{n}(t\_reward-l\_cost)\big]=1-\negl(\secpar).
\end{equation}
%
Moreover, the probability that no block is produced during the last $\log^2\secpar$ rounds is $(1-\pb)^{(\log^2\secpar)nq}=\negl(\secpar)$, for $\pb=\Omega(\frac{1}{nq})$. Thus, all fruits mined during the first $\rounds-\log^2\secpar$ rounds will be included in the chain with $1-\negl(\secpar)$ probability, so
\begin{equation}\label{eq:t_reward}
\Pr[t\_reward\geq Z\reward]=1-\negl(\secpar).    
\end{equation}
%
For the leader costs that are shared among the pool members, we have that $l\_cost\leq W\cost_\msf{lc}+r_0(\cost_\msf{fs}+\cost_\msf{tx})$, where $r_0$ is the round that the final block was mined in $\honexec$. By the Chernoff bounds, and since $W\sim\msf{Bin}(\rounds,1-(1-\pb)^{nq})$
%
\begin{equation*}
\begin{split}
&\Pr\big[W\geq\big(1+\tfrac{\log\secpar}{\sqrt{\rounds}}\big)\rounds(1-(1-\pb)^{nq})\big]%\leq\\
%&\leq e^{-\frac{\log^2\secpar}{3\rounds}\rounds(1-(1-\pb)^{nq})}
=\negl(\secpar).
\end{split}
\end{equation*}
%
Thus, we have that %with $1-\negl(\secpar)$ probability,
\begin{equation}\label{eq:l_cost}
\begin{split}
\Pr\big[l\_cost&\leq \big(1+\tfrac{\log\secpar}{\sqrt{\rounds}}\big)\rounds(1-(1-\pb)^{nq})\cost_\msf{lc}+r_0(\cost_\msf{fs}+\cost_\msf{tx})\big] =1-\negl(\secpar).
\end{split}
\end{equation}
%
Each party in the coalition makes also queries to $\cost_\msf{ltx}$ and $\cost_\msf{ro}$. In addition, in case the coalition includes $\party{L}$, we take into account the extra cost $(N-r_0)(\cost_\msf{fs}+\cost_\msf{tx})$ for $\party{L}$ in the last $(N-r_0)$ rounds where no block was produced. In any case, $c\_cost\leq W(n-1)\cost_\msf{ltx}+\rounds(n-1)q\cost_\msf{ro}+(N-r_0)(\cost_\msf{fs}+\cost_\msf{tx})$. By the Chernoff bounds,
\begin{equation}\label{eq:c_cost}
\begin{split}
&\Pr\big[c\_cost\leq \big(1+\tfrac{\log\secpar}{\sqrt{\rounds}}\big)\rounds(1-(1-\pb)^{nq})(n-1)\cost_\msf{ltx}+\\
&+\rounds (n-1)q\cost_\msf{ro}+(N-r_0)(\cost_\msf{fs}+\cost_\msf{tx}) \big]=1-\negl(\secpar).    
\end{split}
\end{equation}
%
By Eq.~\eqref{eq:c_reward},~\eqref{eq:t_reward},~\eqref{eq:l_cost},~\eqref{eq:c_cost} and for some lower bound $B$ (to be defined), we get that
%
\begin{equation}\label{eq:H_final}
\begin{split}
&\Pr[\umin(\honexec)\geq B]=\Pr[c\_reward-c\_cost\geq B]\geq \\
%
\geq&\Pr\big[\tfrac{n-1}{n}(t\_reward-l\_cost)-c\_cost\geq B\big]-\negl(\secpar)\geq\\
%
\geq&\Pr\big[\tfrac{n-1}{n}\Big(Z\reward-\big(1+\tfrac{\log\secpar}{\sqrt{\rounds}}\big)\rounds(1-(1-\pb)^{nq})\cost_\msf{lc}-r_0(\cost_\msf{fs}+\cost_\msf{tx})\Big)-\\
&-\big(1+\tfrac{\log\secpar}{\sqrt{\rounds}}\big)\rounds(1-(1-\pb)^{nq})(n-1)\cost_\msf{ltx}-\\
&-\rounds (n-1)q\cost_\msf{ro}-(N-r_0)(\cost_\msf{fs}+\cost_\msf{tx})\geq B\big]-\\
&-\negl(\secpar)\geq\\
%
\geq&\Pr\big[\tfrac{n-1}{n}\Big(Z\reward-\big(1+\tfrac{\log\secpar}{\sqrt{\rounds}}\big)\rounds(1-(1-\pb)^{nq})\cost_\msf{lc}\Big)-\big(\tfrac{n-1}{n}r_0+(\rounds-r_0)\big)(\cost_\msf{fs}+\cost_\msf{tx})-\\
&-\big(1+\tfrac{\log\secpar}{\sqrt{\rounds}}\big)\rounds(1-(1-\pb)^{nq})(n-1)\cost_\msf{ltx}-\rounds (n-1)q\cost_\msf{ro}\geq B\big]-\\
&-\negl(\secpar)=\\
%
\geq&\Pr\big[Z\reward\geq \tfrac{n}{n-1}B+\big(1+\tfrac{\log\secpar}{\sqrt{\rounds}}\big)\rounds(1-(1-\pb)^{nq})(\cost_\msf{lc}+n\cost_\msf{ltx})+\\
& +\tfrac{n}{n-1}\big(\tfrac{n-1}{n}r_0+(\rounds-r_0)\big)(\cost_\msf{fs}+\cost_\msf{tx})+\rounds nq\cost_\msf{ro}\big]-\negl(\secpar).
\end{split}   
\end{equation}
%
As already shown, with $1-\negl(\secpar)$ probability, it holds that $r_0\geq\rounds-\log^2\secpar$. Therefore,
\begin{equation}\label{eq:CfsCtx}
\tfrac{n-1}{n}r_0+(\rounds-r_0)=\rounds-\tfrac{1}{n}r_0\leq\tfrac{n-1}{n}\rounds+\tfrac{\log^2\secpar}{n}.
\end{equation}
%
So, by setting 
%
\begin{equation*}
\begin{split}
&\tfrac{n}{n-1}B+\big(1+\tfrac{\log\secpar}{\sqrt{\rounds}}\big)\rounds(1-(1-\pb)^{nq})(\cost_\msf{lc}+n\cost_\msf{ltx})+\\
&+\tfrac{n}{n-1}\big(\tfrac{n-1}{n}\rounds+\tfrac{\log^2\secpar}{n}\big)(\cost_\msf{fs}+\cost_\msf{tx})+\rounds nq\cost_\msf{ro}=\\
&=\big(1-\tfrac{\log\secpar}{\sqrt{\rounds n}}\big)(\rounds-\log^2\secpar)nq\pf\reward\Leftrightarrow\\
%
\Leftrightarrow& B:=\big(1-\tfrac{\log\secpar}{\sqrt{\rounds n}}\big)(\rounds-\log^2\secpar)(n-1)q\pf\reward-\\
&-\tfrac{n-1}{n}\big(1+\tfrac{\log\secpar}{\sqrt{\rounds}}\big)\rounds(1-(1-\pb)^{nq})(\cost_\msf{lc}+n\cost_\msf{ltx})-\\
&-\big(\tfrac{n-1}{n}\rounds+\tfrac{\log^2\secpar}{n}\big)(\cost_\msf{fs}+\cost_\msf{tx})-\rounds (n-1)q\cost_\msf{ro},
\end{split}    
\end{equation*}
%
and by Eq.~\eqref{eq:H_final},~\eqref{eq:CfsCtx} and the Chernoff bounds, we conclude that
%
\begin{equation*}
\begin{split}
&\Pr[\umin(\honexec)\geq B]\geq\\
%
\geq&\Pr\big[Z\geq\big(1-\tfrac{\log\secpar}{\sqrt{\rounds n}}\big)(\rounds-\log^2\secpar)nq\pf\big]-\negl(\secpar)\geq\\
%
\geq&\big(1-e^{\frac{\log^2\secpar}{\rounds n}(\rounds-\log^2\secpar)nq\pf})-\negl(\secpar)\geq1-\negl(\secpar).
\end{split}    
\end{equation*}
%
$\quad$\hfill$\dashv$


Next, we provide an upper bound for $\umax(\exec{})$ when $\mc{A}$'s strategy derives by combining deviations D2, D3, D8.


\begin{claim}\label{claim:D2_D3_D8}
Let $\mc{A}$ be an adversary whose strategy comprises a combination of deviations D2, D3, and D8.

If (i) $\pf\reward>\mathrm{max}\Big\{\tfrac{\cost_\msf{lc}+\cost_\msf{fs}+\cost_\msf{tx}}{(1-\frac{\log\secpar}{\sqrt[4]{n}}) \sqrt{n}q},3\big(\tfrac{\cost_\msf{lc}+\cost_\msf{fs}+\cost_\msf{tx}}{(n-1)q}+\cost_\msf{ro}\big)\Big\}$, and (ii) $\pf<\tfrac{1}{2}$, then for any $\delta\in \big[\tfrac{\log\secpar}{\sqrt[4]{\rounds n}},1\big)$, it holds that
%
\begin{equation*}
\begin{split}
&\Pr\big[\umax(\exec{})\leq (1+\delta)(\rounds-1)(n-1)q\pf\reward+\log^2\secpar(n-1)q\reward-\\
&-\tfrac{n-1}{n}\big(1-\tfrac{\log\secpar}{\sqrt{\rounds}}\big)(\rounds-1)(1-(1-\pb)^{nq})\cost_\msf{lc}-\\
&-\tfrac{n-1}{n}\rounds(\cost_\msf{fs}+\cost_\msf{tx})-\rounds (n-1)q\cost_\msf{ro}\big]\geq1-\negl(\secpar).
\end{split}
\end{equation*}
%

\end{claim}

\textit{Proof of Claim~\ref{claim:D2_D3_D8}.} Since it comprises a combination of deviations D2, D3, and D8, $\mc{A}$'s strategy can be generally described as follows: Up to some round $r^*$, $\mc{A}$ may instruct the coalition $\corrupt$ to make fewer queries to $\mc{O}_\msf{ltx}$ and $\mc{O}_\msf{ro}$ while remaining members of the single pool. After $r^*$, $\mc{A}$ instructs the coalition to abandon the pool and follow $\protocolevp$ with the difference that the corrupted parties may again make fewer queries to $\mc{O}_\msf{ltx}$ and $\mc{O}_\msf{ro}$.

Let $Q\leq(n-1)q$ be the total number of queries to $\mc{O}_\msf{ro}$ of the coalition per round. Without loss of generality (since we want to upper bound the profit of $\mc{A}$), we assume that the corrupted parties make no queries to $\mc{O}_\msf{ltx}$. We define the following random variables:

 Let $W^-$ be the number of rounds before $r^*$ that at least one block was mined (i.e., the number of calls to $\mc{O}_\msf{lc}$ up to $r^*$). Since the remaining honest party makes $q$ queries to the random oracle per round, the probability that at least one block is produced in some round is $1-(1-\pb)^{q+Q}$, so $W^-\sim\msf{Bin}(r^*-1,1-(1-\pb)^{q+Q})$.  

Let $Z^-$ be the number of fruits mined before $r^*$. Since $q+Q$ queries are made by all parties per round,  $Z^-\sim\msf{Bin}((r^*-1)(q+Q),\pf)$.

Let $t\_reward$ be the total rewards of the single pool up to $r^*$ and $l\_cost$ be the leader costs up to $r^*$ in terms of queries to $\mc{O}_\msf{lc},\mc{O}_\msf{fs},\mc{O}_\msf{tx}$ that are shared among the members of the single pool. Let $c\_reward^-$ be the rewards of $\corrupt$ up to $r^*$ and $c\_cost^-$ be the additional cost up to $r^*$ that $\corrupt$ incurs besides its share of $l\_cost$.

Let $W^+$ be the number of rounds from $r^*$ to $\rounds-1$ that at least one block was mined (i.e., the number of calls to $\mc{O}_\msf{lc}$ after $r^*$). It holds that $W^+\sim\msf{Bin}(\rounds-r^*,1-(1-\pb)^{q+Q})$.

Let $Z^+$ be the number of fruits mined by $\corrupt$ from $r^*$ to $\rounds-1$. Since the parties in $\corrupt$ ask $Q$ queries per round, it holds that $Z^+\sim\msf{Bin}((\rounds-r^*)Q,\pf)$.

Let $c\_reward^+$ be the rewards of $\corrupt$ after $r^*$ and $c\_cost^+$ be the total cost that $\corrupt$ incurs after $r^*$.

Assume that $Q\geq\sqrt{n}q$ (the case where $Q<\sqrt{n}q$ will be studied later). Similarly to Claim~\ref{claim:H}, we can show that if $\pf\reward\geq\frac{\cost_\msf{lc}+\cost_\msf{fs}+\cost_\msf{tx}}{(1-\frac{\log\secpar}{\sqrt[4]{n}}) \sqrt{n}q}$, then with $1-\negl(\secpar)$ probability, all blocks of the execution are profitable. In particular, let $\rho$ be the number of rounds elapsed for mining a block $\block$. Since the number of the queries that all parties make per round is $q+Q$, the number of fruits mined during the mining of $\block$, $Z_0$, follows $\msf{Bin}(\rho (q+Q),\pf)$. By the Chernoff bounds and given that $Q\geq \sqrt{n}q$,
%
\begin{equation*}
\begin{split}
\Pr\big[Z_0<(1-\tfrac{\log\secpar}{\sqrt[4]{n}})\rho (q+Q)\pf\big]&\leq e^{-\frac{\log^2\secpar}{2\sqrt{n}}\rho (q+Q)\pf}\leq e^{-\frac{\log^2\secpar}{2\sqrt{n}}\rho \sqrt{n}q\pf}=\negl(\secpar). 
\end{split}
\end{equation*}
%
So, with $1-\negl(\secpar)$ probability, the rewards w.r.t. $\block$ are at least $(1-\frac{\log\secpar}{\sqrt[4]{n}})\rho (q+Q)\pf\reward$. Besides, the leader cost for $\block$ is $\cost_\msf{lc}+\rho\cost_\msf{fs}+\rho\cost_\msf{tx}$. 
Since $\pf\reward\geq\frac{\cost_\msf{lc}+\cost_\msf{fs}+\cost_\msf{tx}}{(1-\frac{\log\secpar}{\sqrt[4]{n}}) \sqrt{n}q}$ and $\rho\geq 1$, we have that with $1-\negl(\secpar)$ probability, it holds that
%
\begin{equation*}
\begin{split}
&(1-\tfrac{\log\secpar}{\sqrt[4]{n}})\rho (q+Q)\pf\reward>(1-\tfrac{\log\secpar}{\sqrt[4]{n}})\rho \sqrt{n}q\pf\reward\geq\\
%
\geq&\rho(\cost_\msf{lc}+\cost_\msf{fs}+\cost_\msf{tx})\geq\cost_\msf{lc}+\rho\cost_\msf{fs}+\rho\cost_\msf{tx},
\end{split}
\end{equation*}
%
i.e., $\Pr[\block\mbox{ is profitable}]\geq1-\negl(\secpar)$.



Thus, since by definition, $t\_reward\leq Z^-\reward$ and $l\_cost=W^-\cost_\msf{lc}+r^*(\cost_\msf{fs}+\cost_\msf{tx})$, and given that $\corrupt$ has $n-1$ parties, we have that
with $1-\negl(\secpar)$ probability,
%
\begin{equation}\label{eq:c_reward-}
c\_reward^-\leq\tfrac{n-1}{n}\big(Z^-\reward-W^-\cost_\msf{lc}+r^*(\cost_\msf{fs}+\cost_\msf{tx})\big).
\end{equation}
%
Besides, we directly get that 
\begin{equation}\label{eq:c_cost-}
c\_cost^-\geq 0\cost_\msf{ltx}+r^*Q\cost_\msf{ro}=r^*Q\cost_\msf{ro}.
\end{equation}
Upon abandoning the pool, for the coalition $\corrupt$ it holds that
\begin{equation}\label{eq:c_reward+}
c\_reward^+\leq Z^+\reward
\end{equation}
%
\begin{equation}\label{eq:c_cost+}
c\_cost^+\geq W^+\cost_\msf{lc}+(\rounds-r^*)(\cost_\msf{fs}+\cost_\msf{tx})-(\rounds-r^*)Q\cost_\msf{ro}
\end{equation}
The above lower bound for $c\_cost^+$ holds because $\mc{A}$ follows a combination of D2, D3, and D8, so for every round after $r^*$, there is at least one corrupted party that interacts with $\mc{O}_\msf{lc},\mc{O}_\msf{fs},\mc{O}_\msf{tx}$ according to $\protocolevp$ (on behalf of $\corrupt$).

By Eq.~\eqref{eq:c_reward-},~\eqref{eq:c_cost-},~\eqref{eq:c_reward+},~\eqref{eq:c_cost+} and for lower bound $B$ (to be defined), we have that
%
\begin{equation*}
\begin{split}
&\Pr[\umax(\exec{})\geq B]=\\
%
=&\Pr[(c\_reward^- - c\_cost^-)+(c\_reward^+-c\_cost^+)\geq B]\leq\\
%
\leq&\Pr\big[(\tfrac{n-1}{n}Z^-+Z^+)\reward-(\tfrac{n-1}{n}W^-+W^+)\cost_\msf{lc}-\\
&-\big(\tfrac{n-1}{n}r^*(\cost_\msf{fs}+\cost_\msf{tx})+(\rounds-r^*)(\cost_\msf{fs}+\cost_\msf{tx})\big)-\rounds Q\cost_\msf{ro}\geq B\big]+\negl(\secpar).\\
%
\leq&\Pr\big[(\tfrac{n-1}{n}Z^-+Z^+)\reward-\tfrac{n-1}{n}(W^-+W^+)\cost_\msf{lc}-\tfrac{n-1}{n}\rounds(\cost_\msf{fs}+\cost_\msf{tx})-\rounds Q\cost_\msf{ro}\geq B\big]+\negl(\secpar).
\end{split}
\end{equation*}
%
Now observe that the random variable $W^-+W^+$ follows $\msf{Bin}((r^*-1)+(\rounds-r^*),1-(1-\pb)^{q+Q})$, i.e., $W^-+W^+\sim\msf{Bin}(\rounds-1,1-(1-\pb)^{q+Q})$. So, by the Chernoff bounds,
\begin{equation*}
\Pr\big[W^-+W^+\leq\big(1-\tfrac{\log\secpar}{\sqrt{\rounds}}\big)(\rounds-1)(1-(1-\pb)^{q+Q})\big]=\negl(\secpar).
\end{equation*}
%
Hence, we have that
%
\begin{equation}\label{eq:bound_all_cases}
\begin{split}
&\Pr[\umax(\exec{})\geq B]\leq\Pr\big[(\tfrac{n-1}{n}Z^-+Z^+)\reward\geq B+\\
&+\tfrac{n-1}{n}\big(1-\tfrac{\log\secpar}{\sqrt{\rounds}}\big)(\rounds-1)(1-(1-\pb)^{q+Q})\cost_\msf{lc}+\tfrac{n-1}{n}\rounds(\cost_\msf{fs}+\cost_\msf{tx})+\rounds Q\cost_\msf{ro}\big]+\negl(\secpar).
\end{split}
\end{equation}
%
We study the following cases for the value $r^*$:

\textbf{Case 1:} $r^*<\log^2\secpar$. Since $Z^-\leq(r^*-1)(q+Q)$ and by Eq.~\eqref{eq:bound_all_cases},
%
\begin{equation*}
\begin{split}
&\Pr[\umax(\exec{})\geq B]\leq\Pr\big[Z^+\reward\geq B+\\
&+\tfrac{n-1}{n}\big(1-\tfrac{\log\secpar}{\sqrt{\rounds}}\big)(\rounds-1)(1-(1-\pb)^{q+Q})\cost_\msf{lc}+\\
&+\tfrac{n-1}{n}\rounds(\cost_\msf{fs}+\cost_\msf{tx})+\rounds Q\cost_\msf{ro}-\tfrac{n-1}{n}(r^*-1)(q+Q)\reward\big]+\\
&+\negl(\secpar).
\end{split}
\end{equation*}
%
To apply the Chernoff bounds for $Z^+$, we want to set $B$ such that for every $Q$, it holds that
\begin{equation*}
\begin{split}
& B+\tfrac{n-1}{n}\big(1-\tfrac{\log\secpar}{\sqrt{\rounds}}\big)(\rounds-1)(1-(1-\pb)^{q+Q})\cost_\msf{lc}+\\
&+\tfrac{n-1}{n}\rounds(\cost_\msf{fs}+\cost_\msf{tx})+\rounds Q\cost_\msf{ro}-\\
&-\tfrac{n-1}{n}(r^*-1)(q+Q)\reward\geq\big(1+\tfrac{\log\secpar}{\sqrt[4]{\rounds n}}\big)(\rounds-r^*)Q\pf\reward\Leftrightarrow\\
%
\Leftrightarrow&B\geq\Big(\tfrac{n-1}{n}\big(1-\tfrac{\log\secpar}{\sqrt{\rounds}}\big)(\rounds-1)(1-\pb)^q\cost_\msf{lc}\Big)(1-\pb)^Q+\\
+&\Big(\big(1+\tfrac{\log\secpar}{\sqrt[4]{\rounds n}}\big)(\rounds-r^*)\pf\reward+\tfrac{n-1}{n}(r^*-1)\reward-\rounds \cost_\msf{ro}\Big)Q+\\
+&\tfrac{n-1}{n}(r^*-1)q\reward-\tfrac{n-1}{n}\big(1-\tfrac{\log\secpar}{\sqrt{\rounds}}\big)(\rounds-1)\cost_\msf{lc}-\tfrac{n-1}{n}\rounds(\cost_\msf{fs}+\cost_\msf{tx}).
\end{split}
\end{equation*}
%
We observe that the right term of the above inequality can be expressed as function of $Q$ of the form $f(Q)=a\cdot x^Q+b\cdot Q+c$. 

Next, we show that, if $\pf\reward>2\cost_\msf{ro}$,
then $f(Q)$ has a maximum at $(n-1)q$ in the range  $[0,(n-1)q]$, i.e,. when the coalition asks all available queries. In particular, we want to set $B$ as an upper bound of
\begin{equation*}
\begin{split}
&\Big(\tfrac{n-1}{n}\big(1-\tfrac{\log\secpar}{\sqrt{\rounds}}\big)(\rounds-1)(1-\pb)^q\cost_\msf{lc}\Big)(1-\pb)^Q+\\
+&\Big(\big(1+\tfrac{\log\secpar}{\sqrt[4]{\rounds n}}\big)(\rounds-r^*)\pf\reward+\tfrac{n-1}{n}(r^*-1)\reward-\rounds \cost_\msf{ro}\Big)Q+\\
+&\tfrac{n-1}{n}(r^*-1)q\reward-\tfrac{n-1}{n}\big(1-\tfrac{\log\secpar}{\sqrt{\rounds}}\big)(\rounds-1)\cost_\msf{lc}-\tfrac{n-1}{n}\rounds(\cost_\msf{fs}+\cost_\msf{tx}).
\end{split}
\end{equation*}
%
To do so, we study the function $f(Q)=a\cdot x^Q+b\cdot Q+c$, where
%
\begin{align*}
x&=1-\pb\\
    a&=\tfrac{n-1}{n}\big(1-\tfrac{\log\secpar}{\sqrt{\rounds}}\big)(\rounds-1)(1-\pb)^q\cost_\msf{lc}\\
    b&=\big(1+\tfrac{\log\secpar}{\sqrt[4]{\rounds n}}\big)(\rounds-r^*)\pf\reward+\tfrac{n-1}{n}(r^*-1)\reward-\rounds \cost_\msf{ro}\\
    c&=\tfrac{n-1}{n}(r^*-1)q\reward-\tfrac{n-1}{n}\big(1-\tfrac{\log\secpar}{\sqrt{\rounds}}\big)(\rounds-1)\cost_\msf{lc}-\tfrac{n-1}{n}\rounds(\cost_\msf{fs}+\cost_\msf{tx})
\end{align*}
%
If $\pf\reward>2\cost_\msf{ro}>\tfrac{Nn}{(N-1)(n-1)}\cost_\msf{ro}$, then it is easy to see that 
\begin{equation*}
\begin{split}
b&=\big(1+\tfrac{\log\secpar}{\sqrt[4]{\rounds n}}\big)(\rounds-r^*)\pf\reward+\tfrac{n-1}{n}(r^*-1)\reward-\rounds \cost_\msf{ro}>\\
%
&>\tfrac{n-1}{n}(\rounds-r^*)\pf\reward+\tfrac{n-1}{n}(r^*-1)\pf\reward-\rounds \cost_\msf{ro}=\\
&=\tfrac{n-1}{n}(\rounds-1)\pf\reward-\rounds \cost_\msf{ro}>0.
\end{split}    
\end{equation*}
%
In order to find the maximum of $f(Q)$ for $Q\in[0,(n-1)q]$, we compute
%
\begin{equation*}
\begin{split}
&f'(Q)=0\Rightarrow  a\cdot\ln x\cdot x^Q+b=0\Rightarrow Q=\dfrac{\ln\big(\frac{b}{a\cdot\ln (1/x)}\big)}{\ln x}
\end{split}    
\end{equation*}
%
Since $b>0$ and $\ln x<0$, we have that $f'$ is increasing. Thus, $\frac{\ln\big(\frac{b}{a\cdot\ln (1/x)}\big)}{\ln x}$ is a minimum for $f$. In addition, $p_b$ is typically a small value so $x$ is close to $1$. Consequently, we may assume that $\ln (1/x)$ is sufficiently small so that $\frac{b}{a\cdot\ln (1/x)}>1$. The latter implies that $\frac{\ln\big(\frac{b}{a\cdot\ln (1/x)}\big)}{\ln x}<0$, so given that $f'$ is increasing, we get that $f'(Q)>0$ for $Q\in[0,(n-1)q]$. Therefore, the maximum of $f$ in $[0,(n-1)q]$ is $(n-1)q$. 

By the above, we have that
\begin{equation*}
\begin{split}
f(Q)&\leq\Big(\tfrac{n-1}{n}\big(1-\tfrac{\log\secpar}{\sqrt{\rounds}}\big)(\rounds-1)(1-\pb)^q\cost_\msf{lc}\Big)(1-\pb)^{(n-1)q}+\\
&\quad+\Big(\big(1+\tfrac{\log\secpar}{\sqrt[4]{\rounds n}}\big)(\rounds-r^*)\pf\reward+\tfrac{n-1}{n}(r^*-1)\reward-\rounds \cost_\msf{ro}\Big)(n-1)q+\\
&\quad+\tfrac{n-1}{n}(r^*-1)q\reward-\tfrac{n-1}{n}\big(1-\tfrac{\log\secpar}{\sqrt{\rounds}}\big)(\rounds-1)\cost_\msf{lc}-\tfrac{n-1}{n}\rounds(\cost_\msf{fs}+\cost_\msf{tx})=\\
%
&=\big(1+\tfrac{\log\secpar}{\sqrt[4]{\rounds n}}\big)(\rounds-r^*)(n-1)q\pf\reward+(r^*-1)(n-1)q\reward-\\
&\quad-\tfrac{n-1}{n}\big(1-\tfrac{\log\secpar}{\sqrt{\rounds}}\big)(\rounds-1)(1-(1-\pb)^{nq})\cost_\msf{lc}-\tfrac{n-1}{n}\rounds(\cost_\msf{fs}+\cost_\msf{tx})-\rounds (n-1)q\cost_\msf{ro}.
\end{split}
\end{equation*}
%
Moreover, given that $\pf<\frac{1}{2}<\frac{1}{1+\tfrac{\log\secpar}{\sqrt[4]{\rounds n}}}$ and $r^*<\log^2\secpar$, we have that 
%
\begin{equation*}
\begin{split}
&\big(1+\tfrac{\log\secpar}{\sqrt[4]{\rounds n}}\big)(\rounds-r^*)(n-1)q\pf\reward+(r^*-1)(n-1)q\reward=\\
=&\big(1+\tfrac{\log\secpar}{\sqrt[4]{\rounds n}}\big)\rounds(n-1)q\pf\reward+\big(1-\big(1+\tfrac{\log\secpar}{\sqrt[4]{\rounds n}}\big)\pf\big)(n-1)q\reward r^*-(n-1)q\reward<\\
<&\big(1+\tfrac{\log\secpar}{\sqrt[4]{\rounds n}}\big)\rounds(n-1)q\pf\reward+\big(1-\big(1+\tfrac{\log\secpar}{\sqrt[4]{\rounds n}}\big)\pf\big)(n-1)q\reward \log^2\secpar-(n-1)q\reward=\\
=&\big(1+\tfrac{\log\secpar}{\sqrt[4]{\rounds n}}\big)(\rounds-\log^2\secpar)(n-1)q\pf\reward+(\log^2\secpar-1)(n-1)q\reward.
\end{split}
\end{equation*}
Therefore, we set the upper bound for $f(Q)$ as
%
\begin{equation}\label{eq:bound_case_1}
\begin{split}
B&=\big(1+\tfrac{\log\secpar}{\sqrt[4]{\rounds n}}\big)(\rounds-\log^2\secpar)(n-1)q\pf\reward+(\log^2\secpar-1)(n-1)q\reward-\\
&\quad-\tfrac{n-1}{n}\big(1-\tfrac{\log\secpar}{\sqrt{\rounds}}\big)(\rounds-1)(1-(1-\pb)^{nq})\cost_\msf{lc}-\tfrac{n-1}{n}\rounds(\cost_\msf{fs}+\cost_\msf{tx})-\rounds (n-1)q\cost_\msf{ro}.
\end{split}
\end{equation}
%
For this value of $B$ and by the Chernoff bounds, we have that %
\begin{equation*}
\begin{split}
&\Pr[\umax(\exec{})\geq B]\leq\\
%
\leq&\Pr\big[Z^+\geq\big(1+\tfrac{\log\secpar}{\sqrt[4]{\rounds n}}\big)(\rounds-r^*)Q\pf\big]+\negl(\secpar)\leq\\
%
\leq&e^{-\frac{\log^2\secpar}{3\sqrt{\rounds n}}(\rounds-r^*)Q\pf}+\negl(\secpar)\leq\\
%
\leq&e^{-\frac{\log^2\secpar}{3\sqrt{\rounds n}}(\rounds-\log^2\secpar)\sqrt{n}q\pf}+\negl(\secpar)\leq\negl(\secpar).
\end{split}
\end{equation*}

\textbf{Case 2:} $\rounds-r^*<\log^2\secpar$. By the fact that $Z^+<(\rounds-r^*)Q$, by Eq.~\eqref{eq:bound_all_cases}, we get that  
%
\begin{equation*}
\begin{split}
&\Pr[\umax(\exec{})\geq B]\leq\Pr\big[\tfrac{n-1}{n}Z^-\reward\geq B+\\
&+\tfrac{n-1}{n}\big(1-\tfrac{\log\secpar}{\sqrt{\rounds}}\big)(\rounds-1)(1-(1-\pb)^{q+Q})\cost_\msf{lc}+\\
&+\tfrac{n-1}{n}\rounds(\cost_\msf{fs}+\cost_\msf{tx})+\rounds Q\cost_\msf{ro}-(\rounds-r^*)Q\reward\big]+\\
&+\negl(\secpar).
\end{split}
\end{equation*}
%
To apply the Chernoff bounds for $Z^-$, we want to set $B$ such that for every $Q$, it holds that
\begin{equation*}
\begin{split}
& \tfrac{n}{n-1}B+\big(1-\tfrac{\log\secpar}{\sqrt{\rounds}}\big)(\rounds-1)(1-(1-\pb)^{q+Q})\cost_\msf{lc}+\\
&+\rounds(\cost_\msf{fs}+\cost_\msf{tx})+\tfrac{n}{n-1}\rounds Q\cost_\msf{ro}-\\
&-\tfrac{n}{n-1}(\rounds-r^*)Q\reward\geq\big(1+\tfrac{\log\secpar}{\sqrt[4]{\rounds n}}\big)(r^*-1)(q+Q)\pf\reward\Leftrightarrow\\
%
\Leftrightarrow&B\geq\Big(\tfrac{n-1}{n}\big(1-\tfrac{\log\secpar}{\sqrt{\rounds}}\big)(\rounds-1)(1-\pb)^q\cost_\msf{lc}\Big)(1-\pb)^Q+\\
+&\Big(\tfrac{n-1}{n}\big(1+\tfrac{\log\secpar}{\sqrt[4]{\rounds n}}\big)(r^*-1)\pf\reward+(\rounds-r^*)\reward-\rounds \cost_\msf{ro}\Big)Q+\\
+&\tfrac{n-1}{n}\big(1+\tfrac{\log\secpar}{\sqrt[4]{\rounds n}}\big)(r^*-1)q\pf\reward-\tfrac{n-1}{n}\big(1-\tfrac{\log\secpar}{\sqrt{\rounds}}\big)(\rounds-1)\cost_\msf{lc}-\tfrac{n-1}{n}\rounds(\cost_\msf{fs}+\cost_\msf{tx}).
\end{split}
\end{equation*}
%
Namely, we want to set $B$ as an upper bound of
\begin{equation*}
\begin{split}
&\Big(\tfrac{n-1}{n}\big(1-\tfrac{\log\secpar}{\sqrt{\rounds}}\big)(\rounds-1)(1-\pb)^q\cost_\msf{lc}\Big)(1-\pb)^Q+\\
+&\Big(\tfrac{n-1}{n}\big(1+\tfrac{\log\secpar}{\sqrt[4]{\rounds n}}\big)(r^*-1)\pf\reward+(\rounds-r^*)\reward-\rounds \cost_\msf{ro}\Big)Q+\\
+&\tfrac{n-1}{n}\big(1+\tfrac{\log\secpar}{\sqrt[4]{\rounds n}}\big)(r^*-1)q\pf\reward-\tfrac{n-1}{n}\big(1-\tfrac{\log\secpar}{\sqrt{\rounds}}\big)(\rounds-1)\cost_\msf{lc}-\tfrac{n-1}{n}\rounds(\cost_\msf{fs}+\cost_\msf{tx}).
\end{split}
\end{equation*}
%
%
To do so, we study the function $g(Q)=a'\cdot x^Q+b'\cdot Q+c'$, where
%
\begin{align*}
x&=1-\pb\\
    a'&=\tfrac{n-1}{n}\big(1-\tfrac{\log\secpar}{\sqrt{\rounds}}\big)(\rounds-1)(1-\pb)^q\cost_\msf{lc}\\
    b'&=\tfrac{n-1}{n}\big(1+\tfrac{\log\secpar}{\sqrt[4]{\rounds n}}\big)(r^*-1)\pf\reward+(\rounds-r^*)\reward-\rounds \cost_\msf{ro}\\
    c'&=\tfrac{n-1}{n}\big(1+\tfrac{\log\secpar}{\sqrt[4]{\rounds n}}\big)(r^*-1)q\pf\reward-\tfrac{n-1}{n}\big(1-\tfrac{\log\secpar}{\sqrt{\rounds}}\big)(\rounds-1)\cost_\msf{lc}-\tfrac{n-1}{n}\rounds(\cost_\msf{fs}+\cost_\msf{tx})
\end{align*}
%
If $\pf\reward>2\cost_\msf{ro}>\tfrac{Nn}{(N-1)(n-1)}\cost_\msf{ro}$, then it is easy to see that 
\begin{equation*}
\begin{split}
b'&=\tfrac{n-1}{n}\big(1+\tfrac{\log\secpar}{\sqrt[4]{\rounds n}}\big)(r^*-1)\pf\reward+(\rounds-r^*)\reward-\rounds \cost_\msf{ro}>\\
&>\tfrac{n-1}{n})(r^*-1)\pf\reward+(\rounds-r^*)\pf\reward-\rounds \cost_\msf{ro}>\\
&>\tfrac{n-1}{n})(\rounds-1)\pf\reward-\rounds \cost_\msf{ro}>0.
\end{split}    
\end{equation*}
%
In order to find the maximum of $g(Q)$ for $Q\in[0,(n-1)q]$, we compute
%
\begin{equation*}
\begin{split}
&g'(Q)=0\Rightarrow  a'\cdot\ln x\cdot x^Q+b'=0\Rightarrow Q=\dfrac{\ln\big(\frac{b'}{a'\cdot\ln (1/x)}\big)}{\ln x}
\end{split}    
\end{equation*}
%
Just like function $f$ in Case 1, we can conclude that the maximum of $g$ in $[0,(n-1)q]$ is $(n-1)q$. Thus, we have that
%
\begin{equation*}
\begin{split}
g(Q)&\leq\Big(\tfrac{n-1}{n}\big(1-\tfrac{\log\secpar}{\sqrt{\rounds}}\big)(\rounds-1)(1-\pb)^q\cost_\msf{lc}\Big)(1-\pb)^{(n-1)q}+\\
&\quad+\Big(\tfrac{n-1}{n}\big(1+\tfrac{\log\secpar}{\sqrt[4]{\rounds n}}\big)(r^*-1)\pf\reward+(\rounds-r^*)\reward-\rounds \cost_\msf{ro}\Big)(n-1)q+\\
&\quad+\tfrac{n-1}{n}\big(1+\tfrac{\log\secpar}{\sqrt[4]{\rounds n}}\big)(r^*-1)q\pf\reward-\tfrac{n-1}{n}\big(1-\tfrac{\log\secpar}{\sqrt{\rounds}}\big)(\rounds-1)\cost_\msf{lc}-\tfrac{n-1}{n}\rounds(\cost_\msf{fs}+\cost_\msf{tx})=\\
%
&=\big(1+\tfrac{\log\secpar}{\sqrt[4]{\rounds n}}\big)(r^*-1)(n-1)q\pf\reward+(\rounds-r^*)(n-1)q\reward-\\
&\quad-\tfrac{n-1}{n}\big(1-\tfrac{\log\secpar}{\sqrt{\rounds}}\big)(\rounds-1)(1-(1-\pb)^{nq})\cost_\msf{lc}-\tfrac{n-1}{n}\rounds(\cost_\msf{fs}+\cost_\msf{tx})-\rounds (n-1)q\cost_\msf{ro}.
\end{split}
\end{equation*}
%
%
Given that $\pf<\frac{1}{2}<\frac{1}{1+\tfrac{\log\secpar}{\sqrt[4]{\rounds n}}}$ and $\rounds-r^*<\log^2\secpar$, we have that 
%
\begin{equation*}
\begin{split}
&\big(1+\tfrac{\log\secpar}{\sqrt[4]{\rounds n}}\big)(r^*-1)(n-1)q\pf\reward+(\rounds-r^*)(n-1)q\reward=\\
=&\rounds(n-1)q\reward-\big(1-\big(1+\tfrac{\log\secpar}{\sqrt[4]{\rounds n}}\big)\pf\big)(n-1)q\reward r^*-\big(1+\tfrac{\log\secpar}{\sqrt[4]{\rounds n}}\big)(n-1)q\pf\reward<\\
%
<&\rounds(n-1)q\reward-\big(1-\big(1+\tfrac{\log\secpar}{\sqrt[4]{\rounds n}}\big)\pf\big)(n-1)q\reward (\rounds-\log^2\secpar)-\big(1+\tfrac{\log\secpar}{\sqrt[4]{\rounds n}}\big)(n-1)q\pf\reward=\\
=&\big(1+\tfrac{\log\secpar}{\sqrt[4]{\rounds n}}\big)(\rounds-\log^2\secpar-1)(n-1)q\pf\reward+\log^2\secpar(n-1)q\reward.
\end{split}
\end{equation*}
%
Therefore, we set the upper bound for $g(Q)$ as
%
\begin{equation}\label{eq:bound_case_2}
\begin{split}
B&=\big(1+\tfrac{\log\secpar}{\sqrt[4]{\rounds n}}\big)(\rounds-\log^2\secpar-1)(n-1)q\pf\reward+\log^2\secpar(n-1)q\reward-\\
&\quad-\tfrac{n-1}{n}\big(1-\tfrac{\log\secpar}{\sqrt{\rounds}}\big)(\rounds-1)(1-(1-\pb)^{nq})\cost_\msf{lc}-\tfrac{n-1}{n}\rounds(\cost_\msf{fs}+\cost_\msf{tx})-\rounds (n-1)q\cost_\msf{ro}.
\end{split}
\end{equation}
For this value of $B$ and by the Chernoff bounds, we have that %
\begin{equation*}
\begin{split}
&\Pr[\umax(\exec{})\geq B]\leq\\
%
\leq&\Pr\big[Z^-\geq\big(1+\tfrac{\log\secpar}{\sqrt[4]{\rounds n}}\big)(r^*-1)(q+Q)\pf\big]+\negl(\secpar)\leq\negl(\secpar).
\end{split}
\end{equation*}

%Since $Z^+<(\rounds-r^*)Q$ and by Eq.~\eqref{eq:bound_all_cases},
%
\iffalse
\begin{equation*}
\begin{split}
&\Pr[\umax(\exec{})\geq B]\leq\\
\leq&\Pr\big[\tfrac{n-1}{n}Z^-\reward\geq B+\\
+&\tfrac{n-1}{n}\big(1-\tfrac{\log\secpar}{\sqrt{\rounds}}\big)(\rounds-1)(1-(1-\pb)^{q+Q})\cost_\msf{lc}+\\
+&\tfrac{n-1}{n}\rounds(\cost_\msf{fs}+\cost_\msf{tx})+\rounds Q\cost_\msf{ro}-(\rounds-r^*)Q\reward\big]+\negl(\secpar).
\end{split}
\end{equation*}
%
To apply the Chernoff bounds for $Z^-$, we want to set $B$ such that for every $Q$, it holds that
\begin{equation*}
\begin{split}
& \tfrac{n}{n-1}B+\big(1-\tfrac{\log\secpar}{\sqrt{\rounds}}\big)(\rounds-1)(1-(1-\pb)^{q+Q})\cost_\msf{lc}+\\
&+\rounds(\cost_\msf{fs}+\cost_\msf{tx})+\tfrac{n}{n-1}\rounds Q\cost_\msf{ro}-\\
&-\tfrac{n}{n-1}(\rounds-r^*)Q\reward\geq\big(1+\tfrac{\log\secpar}{\sqrt[4]{\rounds n}}\big)(r^*-1)(q+Q)\pf\reward\Leftrightarrow\\
%
\Leftrightarrow&B\geq\Big(\tfrac{n-1}{n}\big(1-\tfrac{\log\secpar}{\sqrt{\rounds}}\big)(\rounds-1)(1-\pb)^q\cost_\msf{lc}\Big)(1-\pb)^Q+\\
+&\Big(\tfrac{n-1}{n}\big(1+\tfrac{\log\secpar}{\sqrt[4]{\rounds n}}\big)(r^*-1)\pf\reward+(\rounds-r^*)\reward-\rounds \cost_\msf{ro}\Big)Q+\\
+&\tfrac{n-1}{n}\big(1+\tfrac{\log\secpar}{\sqrt[4]{\rounds n}}\big)(r^*-1)q\pf\reward-\\
-&\tfrac{n-1}{n}\big(1-\tfrac{\log\secpar}{\sqrt{\rounds}}\big)(\rounds-1)\cost_\msf{lc}-\tfrac{n-1}{n}\rounds(\cost_\msf{fs}+\cost_\msf{tx}).
\end{split}
\end{equation*}
%
%Similarly to Case 1, we finally get that (cf. Appendix~\ref{app:case_2}) 
%
\fi

\textbf{Case 3:} $\log^2\secpar\leq r^*\leq \rounds-\log^2\secpar$. In this case, by the Chernoff bounds, we have that for $\delta\in(0,1)$
%
\begin{align*}
\Pr\big[Z^-\geq\big(1+\delta\big)(r^*-1)(q+Q)\pf\big]%&\leq e^{-\frac{\delta^2}{3}(r^*-1)(q+Q)\pf}=\\
=\negl(\secpar).\\
%
\Pr\big[Z^+\geq\big(1+\delta\big)(\rounds-r^*)Q\pf\big]%&\leq e^{-\frac{\delta^2}{3}(\rounds-r^*)Q\pf}=\\
=\negl(\secpar).
\end{align*}
%
By the above, with $1-\negl(\secpar)$ probability, it holds that 
\begin{equation*}
\begin{split}&\tfrac{n-1}{n}Z^- + Z^+<\\
<&\Big(\tfrac{n-1}{n}(r^*-1)(q+Q)+(\rounds-r^*)Q\Big)(1+\delta)\pf=\\
=&\Big(\big(\tfrac{n-1}{n}(q+Q)-Q\big)r^*-\tfrac{n-1}{n}(q+Q)+\rounds Q\Big)(1+\delta)\pf.
\end{split}
\end{equation*}
%
Since $Q\leq(n-1)q$, it holds that $\tfrac{n-1}{n}(q+Q)-Q\geq0$. So, given that $r^*\leq\rounds$, we have that with $1-\negl(\secpar)$ probability
\begin{equation*}
\begin{split}&\tfrac{n-1}{n}Z^- + Z^+<
%<&\Big(\big(\tfrac{n-1}{n}(q+Q)-Q\big)\rounds-\tfrac{n-1}{n}(q+Q)+\rounds Q\Big)(1+\delta)\pf=\\
\tfrac{n-1}{n}(\rounds-1)(q+Q)(1+\delta)\pf.
\end{split}
\end{equation*}
%
Thus, by Eq.~\eqref{eq:bound_all_cases}, we set $B$ such that for every $Q$, it holds 
\begin{equation*}
\begin{split}
& B+\tfrac{n-1}{n}\big(1-\tfrac{\log\secpar}{\sqrt{\rounds}}\big)(\rounds-1)(1-(1-\pb)^{q+Q})\cost_\msf{lc}+\\
&+\tfrac{n-1}{n}\rounds(\cost_\msf{fs}+\cost_\msf{tx})+\rounds Q\cost_\msf{ro}\geq\\
&\geq(1+\delta)\tfrac{n-1}{n}(\rounds-1)(q+Q)\pf\reward\Leftrightarrow\\
%
\Leftrightarrow&B\geq\Big(\tfrac{n-1}{n}\big(1-\tfrac{\log\secpar}{\sqrt{\rounds}}\big)(\rounds-1)(1-\pb)^q\cost_\msf{lc}\Big)(1-\pb)^Q+\\
+&\Big((1+\delta)\tfrac{n-1}{n}(\rounds-1)\pf\reward-\rounds \cost_\msf{ro}\Big)Q+\\
+&(1+\delta)\tfrac{n-1}{n}(\rounds-1)q\pf\reward -\tfrac{n-1}{n}\big(1-\tfrac{\log\secpar}{\sqrt{\rounds}}\big)(\rounds-1)\cost_\msf{lc}-\tfrac{n-1}{n}\rounds(\cost_\msf{fs}+\cost_\msf{tx}).
\end{split}
\end{equation*}
%
Namely, we want to set $B$ as an upper bound of
%
\begin{equation*}
\begin{split}
&\Big(\tfrac{n-1}{n}\big(1-\tfrac{\log\secpar}{\sqrt{\rounds}}\big)(\rounds-1)(1-\pb)^q\cost_\msf{lc}\Big)(1-\pb)^Q+\\
+&\Big((1+\delta)\tfrac{n-1}{n}(\rounds-1)\pf\reward-\rounds \cost_\msf{ro}\Big)Q+\\
+&(1+\delta)\tfrac{n-1}{n}(\rounds-1)q\pf\reward -\tfrac{n-1}{n}\big(1-\tfrac{\log\secpar}{\sqrt{\rounds}}\big)(\rounds-1)\cost_\msf{lc}-\tfrac{n-1}{n}\rounds(\cost_\msf{fs}+\cost_\msf{tx}).
\end{split}
\end{equation*}
%
%
To do so, we study the function $h(Q)=a''\cdot x^Q+b''\cdot Q+c''$, where
%
\begin{align*}
x&=1-\pb\\
    a''&=\tfrac{n-1}{n}\big(1-\tfrac{\log\secpar}{\sqrt{\rounds}}\big)(\rounds-1)(1-\pb)^q\cost_\msf{lc}\\
    b''&=(1+\delta)\tfrac{n-1}{n}(\rounds-1)\pf\reward-\rounds \cost_\msf{ro}\\
    c''&=(1+\delta)\tfrac{n-1}{n}(\rounds-1)q\pf\reward-\tfrac{n-1}{n}\big(1-\tfrac{\log\secpar}{\sqrt{\rounds}}\big)(\rounds-1)\cost_\msf{lc}-\tfrac{n-1}{n}\rounds(\cost_\msf{fs}+\cost_\msf{tx})
\end{align*}
%
If $\pf\reward>2\cost_\msf{ro}>\tfrac{Nn}{(N-1)(n-1)}\cost_\msf{ro}$, then it is easy to see that 
\begin{equation*}
\begin{split}
b''&=(1+\delta)\tfrac{n-1}{n}(\rounds-1)\pf\reward-\rounds \cost_\msf{ro}>0.
\end{split}    
\end{equation*}
%
In order to find the maximum of $h(Q)$ for $Q\in[0,(n-1)q]$, we compute
%
\begin{equation*}
\begin{split}
&h'(Q)=0\Rightarrow  a''\cdot\ln x\cdot x^Q+b''=0\Rightarrow Q=\dfrac{\ln\big(\frac{b''}{a''\cdot\ln (1/x)}\big)}{\ln x}
\end{split}    
\end{equation*}
%
Just like function $f$ in Case 1, we can conclude that the maximum of $h$ in $[0,(n-1)q]$ is $(n-1)q$. Thus, we have that
%
\begin{equation*}
\begin{split}
h(Q)&\leq\Big(\tfrac{n-1}{n}\big(1-\tfrac{\log\secpar}{\sqrt{\rounds}}\big)(\rounds-1)(1-\pb)^q\cost_\msf{lc}\Big)(1-\pb)^{(n-1)q}+\\
&\quad+\Big((1+\delta)\tfrac{n-1}{n}(\rounds-1)\pf\reward-\rounds \cost_\msf{ro}\Big)(n-1)q+\\
&\quad+(1+\delta)\tfrac{n-1}{n}(\rounds-1)q\pf\reward -\tfrac{n-1}{n}\big(1-\tfrac{\log\secpar}{\sqrt{\rounds}}\big)(\rounds-1)\cost_\msf{lc}-\tfrac{n-1}{n}\rounds(\cost_\msf{fs}+\cost_\msf{tx})=\\
%
&=(1+\delta)(\rounds-1)(n-1)q\pf\reward-\\
&\quad-\tfrac{n-1}{n}\big(1-\tfrac{\log\secpar}{\sqrt{\rounds}}\big)(\rounds-1)(1-(1-\pb)^{nq})\cost_\msf{lc}-\tfrac{n-1}{n}\rounds(\cost_\msf{fs}+\cost_\msf{tx})-\rounds (n-1)q\cost_\msf{ro}.
\end{split}
\end{equation*}
%
%
%
Therefore, we set the upper bound for $h(Q)$ as
%
%
\begin{equation}\label{eq:bound_case_3}
\begin{split}
B&=(1+\delta)(\rounds-1)(n-1)q\pf\reward-\\
&\quad-\tfrac{n-1}{n}\big(1-\tfrac{\log\secpar}{\sqrt{\rounds}}\big)(\rounds-1)(1-(1-\pb)^{nq})\cost_\msf{lc}-\tfrac{n-1}{n}\rounds(\cost_\msf{fs}+\cost_\msf{tx})-\rounds (n-1)q\cost_\msf{ro}.
\end{split}
\end{equation}
%
For this value of $B$, we get $\Pr[\umax(\exec{})\geq B]=\negl(\secpar).$
%\begin{equation*}
%\begin{split}
%&\Pr[\umax(\exec{})\geq B]=\negl(\secpar).
%\end{split}
%\end{equation*}
\smallskip

Given Cases 1,2, and 3, we provide a final bound that dominates all three upper bounds in Eq.~\eqref{eq:bound_case_1},~\eqref{eq:bound_case_2}, and~\eqref{eq:bound_case_3}, respectively. In particular, for any $\delta\in \big[\tfrac{\log\secpar}{\sqrt[4]{\rounds n}},1\big)$, we set
%
\begin{equation}\label{eq:final_bound}
\begin{split}
B&=(1+\delta)(\rounds-1)(n-1)q\pf\reward+\log^2\secpar(n-1)q\reward-\\
&\quad-\tfrac{n-1}{n}\big(1-\tfrac{\log\secpar}{\sqrt{\rounds}}\big)(\rounds-1)(1-(1-\pb)^{nq})\cost_\msf{lc}-\tfrac{n-1}{n}\rounds(\cost_\msf{fs}+\cost_\msf{tx})-\rounds (n-1)q\cost_\msf{ro}.
\end{split}
\end{equation}
%
Clearly, the above bound dominates the ones in Eq.~\eqref{eq:bound_case_1},~\eqref{eq:bound_case_2}, and and~\eqref{eq:bound_case_3}. Thus, for this value of $B$, we conclude that %
\begin{equation*}
\begin{split}
%&\Pr[\umax(\exec{})\geq B]=\negl(\secpar)\Rightarrow\\
\Pr[\umax(\exec{})\leq B]\geq1-\negl(\secpar).
\end{split}
\end{equation*}
%
Recall that the analysis so far was given that $Q\geq \sqrt{n}q$. To complete the proof of the claim, we will show that for $Q<\sqrt{n}q$, the profit of the coalition $\corrupt$ cannot exceed the bound in Eq.~\eqref{eq:final_bound}, except with $\negl(\secpar)$ probability. 

If $Q<\sqrt{n}q$, then all the parties make less than $(\sqrt{n}+1)q$ random oracle queries in total per round. Let $Z$ be the number of fruits mined during the execution and $\tilde{Z}$ a random variable that follows $\msf{Bin}(\rounds(\sqrt{n}+1)q,\pf)$. By the Chernoff bounds,
%
\begin{equation*}
\begin{split}
\Pr\big[Z\geq\big(1+\tfrac{\log\secpar}{\sqrt[4]{\rounds n}}\big)\rounds(\sqrt{n}+1)q\pf\big]%&\leq e^{-\frac{\log^2\secpar}{3\sqrt{\rounds n}}\rounds(\sqrt{n}+1)q\pf}=\\
=\negl(\secpar).  
\end{split}    
\end{equation*}
%
The latter implies that with $1-\negl(\secpar)$ probability the total rewards, that are clearly greater than the profit of the coalition, are no more than $\big(1+\tfrac{\log\secpar}{\sqrt[4]{\rounds n}}\big)\rounds(\sqrt{n}+1)q\pf\reward$. 

We show that if $\pf\reward>3\big(\tfrac{\cost_\msf{lc}+\cost_\msf{fs}+\cost_\msf{tx}}{(n-1)q}+\cost_\msf{ro}\big)$, then it holds that for any $\delta\in \big[\tfrac{\log\secpar}{\sqrt[4]{\rounds n}},1\big)$,
%
\begin{equation}\label{eq:less_nq}
\begin{split}
&\big(1+\tfrac{\log\secpar}{\sqrt[4]{\rounds n}}\big)\rounds(\sqrt{n}+1)q\pf\reward<\\
<&(1+\delta)(\rounds-1)(n-1)q\pf\reward+\log^2\secpar(n-1)q\reward-\\
&-\tfrac{n-1}{n}\big(1-\tfrac{\log\secpar}{\sqrt{\rounds}}\big)(\rounds-1)(1-(1-\pb)^{nq})\cost_\msf{lc}-\tfrac{n-1}{n}\rounds(\cost_\msf{fs}+\cost_\msf{tx})-\rounds (n-1)q\cost_\msf{ro}.
\end{split}
\end{equation}
%
Namely, since $\big(1+\tfrac{\log\secpar}{\sqrt[4]{\rounds n}}\big)\rounds(\sqrt{n}+1)<\tfrac{1}{2}(\rounds-1)(n-1)$ for typical values of $\rounds,n$, it holds that
\[\big(1+\tfrac{\log\secpar}{\sqrt[4]{\rounds n}}\big)\rounds(\sqrt{n}+1)q\pf\reward<\tfrac{1}{2}(1+\delta)(\rounds-1)(n-1)q\pf\reward.\]
%
Besides, if $\pf\reward>3\big(\tfrac{\cost_\msf{lc}+\cost_\msf{fs}+\cost_\msf{tx}}{(n-1)q}+\cost_\msf{ro}\big)$, then
%
\begin{equation*}
\begin{split}
&(1+\delta)(\rounds-1)(n-1)q\pf\reward+\log^2\secpar(n-1)q\reward-\\
&-\tfrac{n-1}{n}\big(1-\tfrac{\log\secpar}{\sqrt{\rounds}}\big)(\rounds-1)(1-(1-\pb)^{nq})\cost_\msf{lc}-\\
&-\tfrac{n-1}{n}\rounds(\cost_\msf{fs}+\cost_\msf{tx})-\rounds (n-1)q\cost_\msf{ro}>\\
>&(\rounds-1)(n-1)q\pf\reward-\rounds\cost_\msf{lc}-\rounds(\cost_\msf{fs}+\cost_\msf{tx})-\rounds (n-1)q\cost_\msf{ro}=\\
%
=&(\rounds-1)(n-1)q\pf\reward-\rounds(n-1)q\big(\tfrac{\cost_\msf{lc}+\cost_\msf{fs}+\cost_\msf{tx}}{(n-1)q}+\cost_\msf{ro}\big)>\\
%
>&(\rounds-1)(n-1)q\pf\reward-\tfrac{1}{3}\rounds(n-1)q\pf\reward>\\
%
>&(\rounds-1)(n-1)q\pf\reward-\tfrac{1}{2}(\rounds-1)(n-1)q\pf\reward=\\
=&\tfrac{1}{2}(\rounds-1)(n-1)q\pf\reward.
\end{split}
\end{equation*}
By the above, we get Eq.~\eqref{eq:less_nq}, which completes the proof of the claim.
\hfill$\dashv$
%
%Next, we show that $\mc{A}$ does not gain in terms of profit when the its strategy is a combination of D1, D4-D7, D9-D12.  
%
\begin{claim}\label{claim:other_deviations}
For every   $\mc{O}_\msf{tx}$-respecting adversary $\mc{A}$ that performs a combination of deviations D1, D4-D7, D9-D12 and every $\rounds$-admissible environment $\mc{Z}$ that activates the pool leader first in each round, it holds $\umax(\exec{}) \leq \umin(\honexec)$ with $1-\negl(\secpar)$ probability.
\end{claim}
\textit{Proof of Claim~\ref{claim:other_deviations}. }
D1 is captured by D6 due to step (2) (cf. Figure~\ref{fig:single_leader}) and step (2a) (cf. Figure~\ref{fig:single_other}) in the $\single$ protocol for the leader and the non leader, respectively. In more detail, if a party in $\corrupt$ does not update $inst_i$, $i\in[4]$ as instructed by $\single$ and sends its inconsistent fruits and/or blocks, then during the next round the honest parties following $\single$ will dissolve the pool. This happens because the honest parties will detect the deviation. Thus, the outcome of D1  can be captured by D6 where at some round $\mc{A}$ instructs a subset of the parties in $\mathbf{C}$ to abandon the pool. 

D12 is captured by D6 for the case where all the corrupted parties abandon the pool and follow $\protocolevp$ protocol. This happens because if the leader does not pay a non leader party, then this party will detect this via steps (2b) and (3), it will leave the pool and it will follow $\protocolevp$.  

D7 is not effective in our setting, because the parties that are not corrupted by $\mc{A}$  follow the $\single$ protocol and thus they will never produce a block $\hat{\block}:=\langle\langle \hat{h}_{-1},\hat{h}_f,\hat{\nonce},$ $\hat{\msf{dig}},\hat{\record},\hat{h}\rangle,\hat{\mbf{F}}\rangle$ or a fruit $\langle \hat{h}_{-1},\hat{h}_f,\hat{\nonce},\hat{\msf{dig}},\hat{\record},\hat{h}\rangle$ so that $(\hat{\previous},\hat{h}_f,\hat{\msf{dig}},\hat{\record})\neq(inst_1,inst_2,inst_3,inst_4)$. 

D10 is not performed by an $\mc{O}_\tx$-respecting adversary. 

D6 is captured by any combination of D2-D5 and D7-D12:  let us assume that a subset of the corrupted parties abandons the pool and creates a new pool following different instructions from the $\single$ pool. Recall that the utility of the adversary is the sum of the utilities of all the corrupted parties. Thus, the way of sharing the rewards among the corrupted parties does not affect the utility of the adversary.   

D9, D11 have the same effect as D3 for the case where $\mc{A}$ instructs all the corrupted parties to abstain by asking no queries to the random oracle $\mc{O}_\msf{ro}$. The reason is that if the pool leader does not ask the oracles $\mc{O}_\msf{fs}$, $\mc{O}_\msf{lc}$, it cannot create $inst_i$, $i\in[4]$ needed for all the parties to ask the random oracle and produce valid fruits that will give the rewards to the pool when a block is produced.  

D4 will offer to the adversary lower utility than D3 for the case where $\mc{A}$ instructs all the corrupted parties to abstain by asking no queries the random oracle $\mc{O}_\msf{ro}$. The reason that is if the adversary asks the random oracle but does not send its fruits or blocks, it incurs the cost of $C_\msf{ro}$ without getting any more rewards from the fruits it produces. 

Regarding D5, we do not consider deviations that either hinge on the assumption that the blocks can include unlimited number of fruits and/or they demand that the adversary is aware of the round when $\mc{Z}$ will terminate the execution.   
%
%
%
\hfill$\dashv$
\\








Since $\pf\reward>\tfrac{\cost_\msf{lc}+\cost_\msf{fs}+\cost_\msf{tx}}{(1-\frac{\log\secpar}{\sqrt[4]{n}}) \sqrt{n}q}+\cost_\msf{ro}$, $\pb=\Omega(\frac{1}{nq})$, and $\pf<\tfrac{1}{2}$, for an $\mc{O}_\msf{tx}$-respecting adversary $\mc{A}$, the conditions for all Claims~\ref{claim:H},~\ref{claim:D2_D3_D8}, and~\ref{claim:other_deviations} hold. Therefore, for any  $\mc{O}_\msf{tx}$-respecting adversary $\mc{A}$ and $\delta\in \big[\tfrac{\log\secpar}{\sqrt[4]{\rounds n}},1\big)$, with $1-\negl(\secpar)$ probability, it holds that
%
\begin{equation*}
\begin{split}
&\umax(\exec{})-\umin(\honexec)\leq\\
%\leq&\Big( (1+\delta)(\rounds-1)(n-1)q\pf\reward+\\
%&+\log^2\secpar(n-1)q\reward-\\
%&-\tfrac{n-1}{n}\big(1-\tfrac{\log\secpar}{\sqrt{\rounds}}\big)(\rounds-1)(1-(1-\pb)^{nq})\cost_\msf{lc}-\\
%&-\tfrac{n-1}{n}\rounds(\cost_\msf{fs}+\cost_\msf{tx})-\rounds (n-1)q\cost_\msf{ro}\Big)-
%\end{split}    
%\end{equation*}
%
%\begin{equation*}
%\begin{split}
%-&\Big(\big(1-\tfrac{\log\secpar}{\sqrt{\rounds n}}\big)(\rounds-\log^2\secpar)(n-1)q\pf\reward-\\
%&-\tfrac{n-1}{n}\big(1+\tfrac{\log\secpar}{\sqrt{\rounds}}\big)\rounds(1-(1-\pb)^{nq})(\cost_\msf{lc}+n\cost_\msf{ltx})-\\
%&-\big(\tfrac{n-1}{n}\rounds+\tfrac{\log^2\secpar}{n}\big)(\cost_\msf{fs}+\cost_\msf{tx})-\rounds (n-1)q\cost_\msf{ro}\Big)=\\
%
\leq&\Big(\big(\tfrac{\log\secpar}{\sqrt{\rounds n}}+\delta\big)\rounds+\log^2\secpar\big(1+\tfrac{1}{\pf}\big)-\big(\tfrac{\log^3\secpar}{\sqrt{\rounds n}}+1+\delta\big)\Big)(n-1)q\pf\reward+\\
&+\tfrac{n-1}{n}\Big((2\log\secpar)\sqrt{\rounds}+1-\tfrac{\log\secpar}{\sqrt{\rounds}}\Big)(1-(1-\pb)^{nq})\cost_\msf{lc}+\\
&+\big(1+\tfrac{\log\secpar}{\sqrt{\rounds}}\big)\rounds(1-(1-\pb)^{nq})(n-1)\cost_\msf{ltx}+\tfrac{\log^2\secpar}{n}(\cost_\msf{fs}+\cost_\msf{tx}).
\end{split}    
\end{equation*}

Thus, according to Definition~\ref{def:EVP}, the $\single$ protocol is $(n-1,0,\epsilon')$-EVP, for $\epsilon'$ as in theorem statement.

\end{proof}


\begin{remark}\label{rem:variant}
If we remove the assumption of Theorem \ref{th:equilibrium}, then we can prove that instead of the $\single$ protocol, the following strategy profile, denoted by $\mathcal{S}$, is EVP according to the utility profit: all the parties follow all the instructions of the $\single$ protocol except that:
\begin{enumerate} 
\item  the pool leader ignores step (7),(9) for all the rounds. \item in step (8), if the round is a \textit{payment round}, the pool leader sets $inst_4\leftarrow \tx_T$, where $\tx_T$ is the special transaction with the payments, otherwise it does nothing.
\item the members in step (3) do not add $C_\msf{tx}$ in cost.
\end{enumerate}
\end{remark}
