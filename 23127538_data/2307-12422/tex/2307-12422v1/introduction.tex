    \section{Introduction}\label{sec:introduction}
Bitcoin introduced by Nakamoto \cite{nakamoto} is the first established decentralized cryptocurrency. The Bitcoin blockchain protocol formulates a ledger that consists of a chain of blocks that include transactions. This ledger is maintained without the need of a trusted party and is extended by peer to peer nodes called \emph{miners}. Each miner extends the chain when it manages to solve a Proof-of-Work (PoW) puzzle \cite{puzzle,puzzle2,puzzle3,puzzle4} using computational/hashing power. If more than one chains have been formed, the longest of them constitutes the ledger. The economic incentives of the miners to participate in the mining process are (i) the newly-minted rewards they earn when they produce a block that extends the ledger, and (ii) the transaction fees  which constitute the ``tip'' for the miner who includes a transaction in its block.
%

Although Bitcoin has been characterized as the biggest financial innovation of the fourth industrial revolution \cite{LI2021120383}, it has been criticized for various reasons, such as its vulnerability to \emph{selfish mining} attacks \cite{selfish} and its tendency to centralization \cite{DBLP:journals/corr/abs-1811-08572, gap, DBLP:journals/corr/abs-1904-02368,6824541}. In more detail, when a malicious attacker performs a selfish mining attack, it reduces the fraction of the blocks in the ledger that belong to the honest miners (the miners that follow the Bitcoin protocol). As far as  Bitcoin centralization is concerned, miners are organized into mining pools. Currently, only four pools constitute the majority of the computational power \footnote{\url{https://btc.com/stats/pool.}}. The miners who join pools solve computational puzzles of lower difficulty (partial PoW), they get paid regularly according to the pool rules, and their rewards have lower variance compared to solo mining~\cite{Rosenfeld2011AnalysisOB, 7163020, https://doi.org/10.48550/arxiv.1905.05999, 10.1007/978-3-662-54970-4_28}. 
%
\par In order to prevent selfish mining attacks and achieve \emph{fairness}, Pass et al.~\cite{fruitchain} propose a blockchain protocol called \emph{FruitChain} that uses the 2-for-1 PoW technique (introduced in \cite{backbone}). According to \cite{fruitchain}, a protocol satisfies fairness  when
with overwhelming probability, in every long enough segment of the ledger, any honest set of parties is guaranteed to hold a fraction of blocks that is very close to its relative computational power. In addition, \cite{fruitchain} states that the FruitChain protocol could reduce the variance of the rewards similarly to mining pools, but in a ``fully decentralized way''. Namely, in FruitChain, the parties can produce via mining either blocks or \emph{fruits}, where fruits have much lower difficulty than blocks and can play the role of partial PoW in mining pools. To this direction, many follow up papers correlate (FruitChain protocol's)  decentralization with reducing the variance of the rewards. Some notable examples of such works are the following: (i)  \cite{236314} recognizes high variance as the main motivation for joining a pool; (ii) \cite{9496180} states that mining pools are unnecessary, because miners can produce fruits in short time; (iii) according to \cite{Huang_2021}, when the parties' rewards are concentrated with high probability to their initial resources, the parties lose their motivation to form mining pools; (iv) \cite{Sarkar2020ANB} states that a distribution of block rewards that is equitable makes the formation of mining pools redundant; (v) \cite{kellerconsensus} states that there is no mean of pool existence in Fruitchain; (vi) according to \cite{10.1145/3449301.3449335}, as the fruit can be mined in a very short period, pools are not necessary. Other works  that correlate the incentives of the parties to form/join a pool with the variance of their rewards are  \cite{Tanniru2021,article, journals/iacr/FitziGKR18, 10.1145/3318041.3355458,9724503,kellerconsensus,https://doi.org/10.48550/arxiv.2104.01918}.\\[2pt]
%
\indent\emph{Our results.}
In this paper, we revisit the decentralization of the FruitChain protocol \cite{fruitchain} and argue that, contrary to the common perception, \emph{lower variance of the rewards does not eliminate the tendency of PoW blockchain protocols to centralization}. In particular, we focus on another motivation of the miners to form pools, which is to share the cost of creating the \emph{instance} (i.e., the block header that the miners iteratively hash applying different nonce in each iteration) they need to solve the PoW puzzle. By utilizing a notion of equilibrium called \emph{equilibrium with virtual payoffs} (EVP) presented in \cite{EVP}, we prove that in FruitChain, there is a completely centralized EVP, where all the parties form a single pool controlled by a single party (pool leader) responsible for creating the instance used for the PoW task. In more detail, 
\begin{enumerate}[itemsep=2pt]
\item We abstract the procedures of the FruitChain protocol as oracles and we assign a cost to each procedure. 

\item We provide a formal definition of a pool in a blockchain system. We treat the pool as a description of a subset of parties, along with their communication setting and their execution guidelines. Although our definition is generic, we focus on PoW pools where the collective rewards of the pool are shared among its members in ``off-chain'' manner, i.e., not enforced by the underlying blockchain. 

\item We introduce a set of rules of a completely centralized pool that includes all the miners in the FruitChain system. In this single pool, only the pool leader decides which chain constitutes the ledger and the contents of the blocks that will extend this chain, but all the members (including itself) share the cost of this procedure. To the best of our knowledge, this is the first work that examines the construction of mining pools in the FruitChain system.

\item We prove that in the FruitChain system, joining this completely centralized pool and following its rules is an EVP. Note that we are able to prove this statement because the rules of this pool disincentivize the members of the pool from (i) attacking each other and (ii) making the betrayal of the pool more profitable than sticking to the pool and sharing the costs.

\item We discuss possible directions to mitigate the tendency to centralization due to sharing verification costs.
\end{enumerate}
%
%
\indent\emph{Other related work}.
In \cite{EVP}, Kiayias et al. present the EVP notion that we use in our results. As far as decentralization is concerned, the authors prove that in Bitcoin, forming a single dictatorial pool is EVP, when the cost of processing transactions is not negligible. In our work, we show that although the FruitChain protocol offers lower variance in the miners' rewards, the centralization problem is not solved. Apart from deploying the EVP framework to study the (de)centralization of another protocol, our formal treatment extends and improves the one of~\cite{EVP} w.r.t. the following aspects:
\begin{itemize}
    \item We take into account not only the cost of processing transactions and the cost of making random oracle queries (as \cite{EVP} does), but also the cost of finding the longest chain and the cost for finding all the fruits whose digest should be included in the instance needed for mining.
    \item In \cite{EVP}, it is assumed that when a pool member ignores the instance of the pool leader and mines on a different instance, the pool leader can detect this deviant behavior and remove the member from the pool. This assumption excludes strategies where the deviant member mines under a different identity (public key) than the one the pool leader has recorded. In our case, we address this by having the pool leader dissolve the pool whenever it detects the aforementioned deviant behavior.
\end{itemize}

In \cite{9230398}, the authors prove that if a ``fair'' reward function is used in their setting, there is no equilibrium with more than one pool. Note that the setting in~\cite{9230398} (i) refers to proof of stake protocols, which means that there is no cost for mining, and (ii) is deterministic, so it cannot  capture the probabilistic nature of  the FruitChain protocol. 
Some other works that consider cost sharing as a possible reason for centralization in blockchain protocols are \cite{breidenbach2021chainlink,Natoli2019DeconstructingBA, 10.1145/3318041.3355470}. Other works related to decentralization in blockchain protocols are \cite{Gencer2018DecentralizationIB, 10.1145/3318041.3355463,azouvi2021levels,https://doi.org/10.48550/arxiv.2112.09941,9488812}.


  















