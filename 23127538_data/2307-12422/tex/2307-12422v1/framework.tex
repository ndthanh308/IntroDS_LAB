\section{Framework}\label{sec:framework}
%
At this point, we will give an overview of the notion of \textit{coalition-safe equilibria with virtual payoff|} presented in \cite{EVP}. This notion generalizes the equilibrium notion presented in \cite{fruitchain}, and it is based on the execution model of \cite{backbone} and the ``real-world'' protocol execution model of   \cite{Canetti2,Canetti1,UC2,Canetti4}. It examines two executions of a blockchain protocol $\Pi$. In the first execution, all the participants follow the protocol and in the second execution, there exists a strategic coalition that deviates trying to maximize its collective utility.  

\subsection{Notation}\label{subsec:framework_notation}
We use $\secpar$ as the security parameter. We write $X\sim\mc{D}$ to denote that the random variable $X$ follows the distribution $\mc{D}$. The mean of the random variable $X$ is denoted by $E[X]$. By $\msf{Bin}(K,p)$, we denote the binomial distribution with $K$ trials and success probability $p$. We write $\negl(\secpar)$ to denote that a function is negligible in $\secpar$, i.e., asymptotically smaller than the inverse of any polynomial. By $|x|$, we denote the absolute value of $x$. We use `$||$' to denote the concatenation operation.


\subsection{Protocol Execution Model}\label{subsec:framework_execution}
%
The execution model for a blockchain protocol $\Pi$ comprises an environment $\mc{Z}$, an adversary $\mc{A}$ and the participating parties $\party{1}, \ldots, \party{n}$. The protocol execution is progressing in rounds. The environment reflects the external world to the protocol and  decides the number of rounds the execution will run. \emph{$\rounds$-admissible} environment will be the environment that performs the protocol $\rounds$ rounds, where $\rounds$ is a polynomial in the security parameter $\secpar$ that is higher than $\secpar$. Before the beginning of the execution the adversary chooses which parties it will control. Let $\corrupt$ be the set of parties controlled by the adversary and $\honest$ the set of the remaining parties called ``honest''. 
%
\par During each round, the environment gives inputs to the parties and activates them in a round-robin fashion (cf.~\cite{backbone}).   The parties that belong to $\honest$ follow the protocol and the parties that belong to $\corrupt$ follow the instructions of the adversary. Note that the adversary reflects a \emph{strategic coalition} that deviates from the protocol in a way that maximizes its collective utility (the sum of the utilities of all the parties that belong to $\honest$).
%
\par The communication between the parties is controlled by a functionality called \textit{Diffuse functionality} defined in \cite{backbone}. This functionality guarantees that every message sent from an honest party will be delivered to every other honest party by the end of each round. It allows though the adversary to rearrange the order of the messages during the round. This means that the adversary is allowed to deliver its messages first. The adversary that follows the protocol but rearranges the messages so that it delivers its messages first is denoted by $\innocent$. Note that the adversary may send some of its messages to a subset of the honest parties, so the honest parties at the end of a round can have different local view, thus local chain.
%
\par During the execution, the parties interact with a number of oracles $\mc{O}_1,\ldots,\mc{O}_{w_\Pi}$ that are protocol-specific for $\Pi$. For instance, $\mc{O}_j$ can be a random oracle, a signing oracle, a transaction validity oracle, etc. There is a limited number of queries that each party can make to each of $\mc{O}_1,\ldots,\mc{O}_{w_\Pi}$ per round that is denoted by $q_1,\ldots,q_{w_\Pi}$, respectively. In addition, we consider that each single query to $\mc{O}_1,\ldots,\mc{O}_{w_\Pi}$ has a non-zero cost denoted by $\cost_1,\ldots,\cost_{w_\Pi}$, respectively. 
%
\par If we fix the environment $\mc{Z}$ and the adversary $\mc{A}$, then the execution can be seen as a random variable denoted by $\exec{}$.

\subsection{Coalition-safe Equilibria with Virtual Payoff}\label{subsec:framework_equilibria}
%
The notion of \textit{coalition-safe equilibria with virtual payoff} examines the executions $\exec{}$ and $\honexec$ for an arbitrary environment $\mc{Z}$ and an arbitrary adversary $\mc{A}$ that corrupts a set $\corrupt$ including at most $\upper$ parties, where $\upper$ will be a parameter in this definition. In $\honexec$, the adversary follows the protocol but rearranges the messages to deliver its messages first. In $\exec{i}$ the adversary deviates arbitrarily. 
%
\par The notion compares the utility of the adversary in these two executions. Both executions have the same number rounds (the environment is admissible) so that  their utilities can be compared in a meaningful way. The utility of the adversary will be the sum of the utilities of the parties in $\corrupt$. Note that each honest party can have a different view on the utility of each other party and thus, the utility of the adversary. The reason is that the utility is computed based on the rewards which, in turn, are based on the parties', potentially different, local chains. The notion uses the \textit{lowest utility} of the adversary among all the honest parties' local views for the execution $\honexec$, denoted by $\umin(\honexec)$, and the \textit{highest} utility for the execution $\exec{}$, denoted by $\umax(\exec{})$.
Note that $\umin(\honexec)$ and $\umax(\exec{})$ are random variables over the coins of the adversary, the environment, the parties and the oracles.
%
\begin{definition}\label{def:EVP}
Let $\epsilon,\epsilon'$ be non-negative real values. A protocol $\Pi$ is \emph{$(\upper,\epsilon,\epsilon')$-equilibrium with virtual payoffs} (EVP) when for every $\rounds$-admissible environment $\mc{Z}$ and for every PPT adversary $\mc{A}$ that controls an arbitrary set $\corrupt$ of at most $\upper$ parties it holds that 
\begin{equation}\label{eq:EVP}
\umax(\exec{}) \leq \umin(\honexec) + \epsilon \cdot \mid \umin(\honexec) \mid +\epsilon'   
\end{equation}
with overwhelming (i.e., $1-\negl(\secpar)$) probability.
\end{definition}
According to Eq.~\eqref{eq:EVP}, the closer that the values of $\epsilon$ and $\epsilon'$ get to $0$, the ``tighter'' the equilibrium is. Some examples of utility functions of the adversary are: (i) absolute rewards, (ii) absolute rewards minus absolute cost (profit) and (iii) relative rewards. 
%In the case of \textit{absolute rewards minus absolute cost} the utility can take negative values. Thus, the right part of the equation uses absolute value.
Note that if the adversary can deviate from the protocol and increase significantly its utility on the view of just one honest party with a non-negligible probability, then the protocol does not satisfy this notion.

For the rest of the paper, we refer to the framework presented in this section as the \emph{EVP framework}.


