\section{Discussion}\label{sec:discussion}

 We believe that proposals such as Smartpool~\cite{10.5555/3241189.3241299} that give the transaction verification back to the miners, and \cite{236314} where the miners validate the transactions, have similar tendency to centralization; like in the FruitChain system, the miners in~\cite{10.5555/3241189.3241299,236314} have incentives to collude in order to share the transaction verification costs. Note that as  \cite{Azouvi2021SoK} states, any decentralised system can be executed in a centralised manner. Thus, the fact that a system is designed so that miners can process the transactions in a decentralised manner does not imply that they have incentives to do so.
\par As our results indicate, apart from reducing the variance of the rewards, further research is needed to incentivize decentralization in PoW protocols. One possible research direction is to disincentivize parallel mining (e.g., \cite{9136680,10.1145/2810103.2813621}).
\par In more detail, Miller et al. \cite{10.1145/2810103.2813621} propose a non-outsourceable puzzle that does not allow miners in a pool to provide the leader with a proof that they indeed mine for the pool. This could constitute a possible countermeasure for PoW systems where centralization is motivated by sharing transaction verification costs. The reason is that this non-outsourceable puzzle could render abandoning the pool more profitable than sticking to the pool and share the costs. However, such countermeasure seems incompatible with any blockchain protocol that, like FruitChain, uses the 2-for-1 PoW technique \cite{backbone} to reduce the variance of the rewards and mitigate selfish mining attacks. This is because the ``easier'' puzzle used in the 2-for-1 PoW technique  can serve as proof of mining in a pool.
\par  To overcome the above incompatibility, the design challenge is to construct PoW puzzles that disincentivize the formation of pools, while being applicable to blockchain protocols that satisfy fairness~\cite{fruitchain}.
One candidate solution to this direction could be built upon the PoW puzzle in \cite{9136680} that (i) is non-parallelizable, (i.e., it is computed by the miners serially), and (ii) is used in a consensus mechanism that guarantees fairness.

Note that in order to eliminate the problem of centralization, it is necessary that the miners do not use the same set of transactions in their puzzles. Else, they will still have incentives to collude and share the transaction verification cost. 

\paragraph{Acknowledgements.}
Zacharias was supported by Input Output (\url{https://iohk.io}) through their funding of the Edinburgh Blockchain Technology Lab. Part of this work was conducted while Stouka was a research associate at the Edinburgh Blockchain Technology Lab.