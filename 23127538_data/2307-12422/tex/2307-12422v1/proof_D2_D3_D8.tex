
Next, we provide an upper bound for $\umax(\exec{})$ when $\mc{A}$'s strategy derives by combining deviations D2, D3, D8.


\begin{claim}\label{claim:D2_D3_D8}
Let $\mc{A}$ be an adversary whose strategy comprises a combination of deviations D2, D3, and D8.

If (i) $\pf\reward>\mathrm{max}\Big\{\tfrac{\cost_\msf{lc}+\cost_\msf{fs}+\cost_\msf{tx}}{(1-\frac{\log\secpar}{\sqrt[4]{n}}) \sqrt{n}q},3\big(\tfrac{\cost_\msf{lc}+\cost_\msf{fs}+\cost_\msf{tx}}{(n-1)q}+\cost_\msf{ro}\big)\Big\}$, and (ii) $\pf<\tfrac{1}{2}$, then for any $\delta\in \big[\tfrac{\log\secpar}{\sqrt[4]{\rounds n}},1\big)$, it holds that
%
\begin{equation*}
\begin{split}
&\Pr\big[\umax(\exec{})\leq (1+\delta)(\rounds-1)(n-1)q\pf\reward+\log^2\secpar(n-1)q\reward-\\
&-\tfrac{n-1}{n}\big(1-\tfrac{\log\secpar}{\sqrt{\rounds}}\big)(\rounds-1)(1-(1-\pb)^{nq})\cost_\msf{lc}-\\
&-\tfrac{n-1}{n}\rounds(\cost_\msf{fs}+\cost_\msf{tx})-\rounds (n-1)q\cost_\msf{ro}\big]\geq1-\negl(\secpar).
\end{split}
\end{equation*}
%

\end{claim}

\textit{Proof of Claim~\ref{claim:D2_D3_D8}.} Since it comprises a combination of deviations D2, D3, and D8, $\mc{A}$'s strategy can be generally described as follows: Up to some round $r^*$, $\mc{A}$ may instruct the coalition $\corrupt$ to make fewer queries to $\mc{O}_\msf{ltx}$ and $\mc{O}_\msf{ro}$ while remaining members of the single pool. After $r^*$, $\mc{A}$ instructs the coalition to abandon the pool and follow $\protocolevp$ with the difference that the corrupted parties may again make fewer queries to $\mc{O}_\msf{ltx}$ and $\mc{O}_\msf{ro}$.

Let $Q\leq(n-1)q$ be the total number of queries to $\mc{O}_\msf{ro}$ of the coalition per round. Without loss of generality (since we want to upper bound the profit of $\mc{A}$), we assume that the corrupted parties make no queries to $\mc{O}_\msf{ltx}$. We define the following random variables:

 Let $W^-$ be the number of rounds before $r^*$ that at least one block was mined (i.e., the number of calls to $\mc{O}_\msf{lc}$ up to $r^*$). Since the remaining honest party makes $q$ queries to the random oracle per round, the probability that at least one block is produced in some round is $1-(1-\pb)^{q+Q}$, so $W^-\sim\msf{Bin}(r^*-1,1-(1-\pb)^{q+Q})$.  

Let $Z^-$ be the number of fruits mined before $r^*$. Since $q+Q$ queries are made by all parties per round,  $Z^-\sim\msf{Bin}((r^*-1)(q+Q),\pf)$.

Let $t\_reward$ be the total rewards of the single pool up to $r^*$ and $l\_cost$ be the leader costs up to $r^*$ in terms of queries to $\mc{O}_\msf{lc},\mc{O}_\msf{fs},\mc{O}_\msf{tx}$ that are shared among the members of the single pool. Let $c\_reward^-$ be the rewards of $\corrupt$ up to $r^*$ and $c\_cost^-$ be the additional cost up to $r^*$ that $\corrupt$ incurs besides its share of $l\_cost$.

Let $W^+$ be the number of rounds from $r^*$ to $\rounds-1$ that at least one block was mined (i.e., the number of calls to $\mc{O}_\msf{lc}$ after $r^*$). It holds that $W^+\sim\msf{Bin}(\rounds-r^*,1-(1-\pb)^{q+Q})$.

Let $Z^+$ be the number of fruits mined by $\corrupt$ from $r^*$ to $\rounds-1$. Since the parties in $\corrupt$ ask $Q$ queries per round, it holds that $Z^+\sim\msf{Bin}((\rounds-r^*)Q,\pf)$.

Let $c\_reward^+$ be the rewards of $\corrupt$ after $r^*$ and $c\_cost^+$ be the total cost that $\corrupt$ incurs after $r^*$.

Assume that $Q\geq\sqrt{n}q$ (the case where $Q<\sqrt{n}q$ will be studied later). Similarly to Claim~\ref{claim:H}, we can show that if $\pf\reward\geq\frac{\cost_\msf{lc}+\cost_\msf{fs}+\cost_\msf{tx}}{(1-\frac{\log\secpar}{\sqrt[4]{n}}) \sqrt{n}q}$, then with $1-\negl(\secpar)$ probability, all blocks of the execution are profitable. In particular, let $\rho$ be the number of rounds elapsed for mining a block $\block$. Since the number of the queries that all parties make per round is $q+Q$, the number of fruits mined during the mining of $\block$, $Z_0$, follows $\msf{Bin}(\rho (q+Q),\pf)$. By the Chernoff bounds and given that $Q\geq \sqrt{n}q$,
%
\begin{equation*}
\begin{split}
\Pr\big[Z_0<(1-\tfrac{\log\secpar}{\sqrt[4]{n}})\rho (q+Q)\pf\big]&\leq e^{-\frac{\log^2\secpar}{2\sqrt{n}}\rho (q+Q)\pf}\leq e^{-\frac{\log^2\secpar}{2\sqrt{n}}\rho \sqrt{n}q\pf}=\negl(\secpar). 
\end{split}
\end{equation*}
%
So, with $1-\negl(\secpar)$ probability, the rewards w.r.t. $\block$ are at least $(1-\frac{\log\secpar}{\sqrt[4]{n}})\rho (q+Q)\pf\reward$. Besides, the leader cost for $\block$ is $\cost_\msf{lc}+\rho\cost_\msf{fs}+\rho\cost_\msf{tx}$. 
Since $\pf\reward\geq\frac{\cost_\msf{lc}+\cost_\msf{fs}+\cost_\msf{tx}}{(1-\frac{\log\secpar}{\sqrt[4]{n}}) \sqrt{n}q}$ and $\rho\geq 1$, we have that with $1-\negl(\secpar)$ probability, it holds that
%
\begin{equation*}
\begin{split}
&(1-\tfrac{\log\secpar}{\sqrt[4]{n}})\rho (q+Q)\pf\reward>(1-\tfrac{\log\secpar}{\sqrt[4]{n}})\rho \sqrt{n}q\pf\reward\geq\\
%
\geq&\rho(\cost_\msf{lc}+\cost_\msf{fs}+\cost_\msf{tx})\geq\cost_\msf{lc}+\rho\cost_\msf{fs}+\rho\cost_\msf{tx},
\end{split}
\end{equation*}
%
i.e., $\Pr[\block\mbox{ is profitable}]\geq1-\negl(\secpar)$.



Thus, since by definition, $t\_reward\leq Z^-\reward$ and $l\_cost=W^-\cost_\msf{lc}+r^*(\cost_\msf{fs}+\cost_\msf{tx})$, and given that $\corrupt$ has $n-1$ parties, we have that
with $1-\negl(\secpar)$ probability,
%
\begin{equation}\label{eq:c_reward-}
c\_reward^-\leq\tfrac{n-1}{n}\big(Z^-\reward-W^-\cost_\msf{lc}+r^*(\cost_\msf{fs}+\cost_\msf{tx})\big).
\end{equation}
%
Besides, we directly get that 
\begin{equation}\label{eq:c_cost-}
c\_cost^-\geq 0\cost_\msf{ltx}+r^*Q\cost_\msf{ro}=r^*Q\cost_\msf{ro}.
\end{equation}
Upon abandoning the pool, for the coalition $\corrupt$ it holds that
\begin{equation}\label{eq:c_reward+}
c\_reward^+\leq Z^+\reward
\end{equation}
%
\begin{equation}\label{eq:c_cost+}
c\_cost^+\geq W^+\cost_\msf{lc}+(\rounds-r^*)(\cost_\msf{fs}+\cost_\msf{tx})-(\rounds-r^*)Q\cost_\msf{ro}
\end{equation}
The above lower bound for $c\_cost^+$ holds because $\mc{A}$ follows a combination of D2, D3, and D8, so for every round after $r^*$, there is at least one corrupted party that interacts with $\mc{O}_\msf{lc},\mc{O}_\msf{fs},\mc{O}_\msf{tx}$ according to $\protocolevp$ (on behalf of $\corrupt$).

By Eq.~\eqref{eq:c_reward-},~\eqref{eq:c_cost-},~\eqref{eq:c_reward+},~\eqref{eq:c_cost+} and for lower bound $B$ (to be defined), we have that
%
\begin{equation*}
\begin{split}
&\Pr[\umax(\exec{})\geq B]=\\
%
=&\Pr[(c\_reward^- - c\_cost^-)+(c\_reward^+-c\_cost^+)\geq B]\leq\\
%
\leq&\Pr\big[(\tfrac{n-1}{n}Z^-+Z^+)\reward-(\tfrac{n-1}{n}W^-+W^+)\cost_\msf{lc}-\\
&-\big(\tfrac{n-1}{n}r^*(\cost_\msf{fs}+\cost_\msf{tx})+(\rounds-r^*)(\cost_\msf{fs}+\cost_\msf{tx})\big)-\rounds Q\cost_\msf{ro}\geq B\big]+\negl(\secpar).\\
%
\leq&\Pr\big[(\tfrac{n-1}{n}Z^-+Z^+)\reward-\tfrac{n-1}{n}(W^-+W^+)\cost_\msf{lc}-\tfrac{n-1}{n}\rounds(\cost_\msf{fs}+\cost_\msf{tx})-\rounds Q\cost_\msf{ro}\geq B\big]+\negl(\secpar).
\end{split}
\end{equation*}
%
Now observe that the random variable $W^-+W^+$ follows $\msf{Bin}((r^*-1)+(\rounds-r^*),1-(1-\pb)^{q+Q})$, i.e., $W^-+W^+\sim\msf{Bin}(\rounds-1,1-(1-\pb)^{q+Q})$. So, by the Chernoff bounds,
\begin{equation*}
\Pr\big[W^-+W^+\leq\big(1-\tfrac{\log\secpar}{\sqrt{\rounds}}\big)(\rounds-1)(1-(1-\pb)^{q+Q})\big]=\negl(\secpar).
\end{equation*}
%
Hence, we have that
%
\begin{equation}\label{eq:bound_all_cases}
\begin{split}
&\Pr[\umax(\exec{})\geq B]\leq\Pr\big[(\tfrac{n-1}{n}Z^-+Z^+)\reward\geq B+\\
&+\tfrac{n-1}{n}\big(1-\tfrac{\log\secpar}{\sqrt{\rounds}}\big)(\rounds-1)(1-(1-\pb)^{q+Q})\cost_\msf{lc}+\tfrac{n-1}{n}\rounds(\cost_\msf{fs}+\cost_\msf{tx})+\rounds Q\cost_\msf{ro}\big]+\negl(\secpar).
\end{split}
\end{equation}
%
We study the following cases for the value $r^*$:

\textbf{Case 1:} $r^*<\log^2\secpar$. Since $Z^-\leq(r^*-1)(q+Q)$ and by Eq.~\eqref{eq:bound_all_cases},
%
\begin{equation*}
\begin{split}
&\Pr[\umax(\exec{})\geq B]\leq\Pr\big[Z^+\reward\geq B+\\
&+\tfrac{n-1}{n}\big(1-\tfrac{\log\secpar}{\sqrt{\rounds}}\big)(\rounds-1)(1-(1-\pb)^{q+Q})\cost_\msf{lc}+\\
&+\tfrac{n-1}{n}\rounds(\cost_\msf{fs}+\cost_\msf{tx})+\rounds Q\cost_\msf{ro}-\tfrac{n-1}{n}(r^*-1)(q+Q)\reward\big]+\\
&+\negl(\secpar).
\end{split}
\end{equation*}
%
To apply the Chernoff bounds for $Z^+$, we want to set $B$ such that for every $Q$, it holds that
\begin{equation*}
\begin{split}
& B+\tfrac{n-1}{n}\big(1-\tfrac{\log\secpar}{\sqrt{\rounds}}\big)(\rounds-1)(1-(1-\pb)^{q+Q})\cost_\msf{lc}+\\
&+\tfrac{n-1}{n}\rounds(\cost_\msf{fs}+\cost_\msf{tx})+\rounds Q\cost_\msf{ro}-\\
&-\tfrac{n-1}{n}(r^*-1)(q+Q)\reward\geq\big(1+\tfrac{\log\secpar}{\sqrt[4]{\rounds n}}\big)(\rounds-r^*)Q\pf\reward\Leftrightarrow\\
%
\Leftrightarrow&B\geq\Big(\tfrac{n-1}{n}\big(1-\tfrac{\log\secpar}{\sqrt{\rounds}}\big)(\rounds-1)(1-\pb)^q\cost_\msf{lc}\Big)(1-\pb)^Q+\\
+&\Big(\big(1+\tfrac{\log\secpar}{\sqrt[4]{\rounds n}}\big)(\rounds-r^*)\pf\reward+\tfrac{n-1}{n}(r^*-1)\reward-\rounds \cost_\msf{ro}\Big)Q+\\
+&\tfrac{n-1}{n}(r^*-1)q\reward-\tfrac{n-1}{n}\big(1-\tfrac{\log\secpar}{\sqrt{\rounds}}\big)(\rounds-1)\cost_\msf{lc}-\tfrac{n-1}{n}\rounds(\cost_\msf{fs}+\cost_\msf{tx}).
\end{split}
\end{equation*}
%
We observe that the right term of the above inequality can be expressed as function of $Q$ of the form $f(Q)=a\cdot x^Q+b\cdot Q+c$. 

Next, we show that, if $\pf\reward>2\cost_\msf{ro}$,
then $f(Q)$ has a maximum at $(n-1)q$ in the range  $[0,(n-1)q]$, i.e,. when the coalition asks all available queries. In particular, we want to set $B$ as an upper bound of
\begin{equation*}
\begin{split}
&\Big(\tfrac{n-1}{n}\big(1-\tfrac{\log\secpar}{\sqrt{\rounds}}\big)(\rounds-1)(1-\pb)^q\cost_\msf{lc}\Big)(1-\pb)^Q+\\
+&\Big(\big(1+\tfrac{\log\secpar}{\sqrt[4]{\rounds n}}\big)(\rounds-r^*)\pf\reward+\tfrac{n-1}{n}(r^*-1)\reward-\rounds \cost_\msf{ro}\Big)Q+\\
+&\tfrac{n-1}{n}(r^*-1)q\reward-\tfrac{n-1}{n}\big(1-\tfrac{\log\secpar}{\sqrt{\rounds}}\big)(\rounds-1)\cost_\msf{lc}-\tfrac{n-1}{n}\rounds(\cost_\msf{fs}+\cost_\msf{tx}).
\end{split}
\end{equation*}
%
To do so, we study the function $f(Q)=a\cdot x^Q+b\cdot Q+c$, where
%
\begin{align*}
x&=1-\pb\\
    a&=\tfrac{n-1}{n}\big(1-\tfrac{\log\secpar}{\sqrt{\rounds}}\big)(\rounds-1)(1-\pb)^q\cost_\msf{lc}\\
    b&=\big(1+\tfrac{\log\secpar}{\sqrt[4]{\rounds n}}\big)(\rounds-r^*)\pf\reward+\tfrac{n-1}{n}(r^*-1)\reward-\rounds \cost_\msf{ro}\\
    c&=\tfrac{n-1}{n}(r^*-1)q\reward-\tfrac{n-1}{n}\big(1-\tfrac{\log\secpar}{\sqrt{\rounds}}\big)(\rounds-1)\cost_\msf{lc}-\tfrac{n-1}{n}\rounds(\cost_\msf{fs}+\cost_\msf{tx})
\end{align*}
%
If $\pf\reward>2\cost_\msf{ro}>\tfrac{Nn}{(N-1)(n-1)}\cost_\msf{ro}$, then it is easy to see that 
\begin{equation*}
\begin{split}
b&=\big(1+\tfrac{\log\secpar}{\sqrt[4]{\rounds n}}\big)(\rounds-r^*)\pf\reward+\tfrac{n-1}{n}(r^*-1)\reward-\rounds \cost_\msf{ro}>\\
%
&>\tfrac{n-1}{n}(\rounds-r^*)\pf\reward+\tfrac{n-1}{n}(r^*-1)\pf\reward-\rounds \cost_\msf{ro}=\\
&=\tfrac{n-1}{n}(\rounds-1)\pf\reward-\rounds \cost_\msf{ro}>0.
\end{split}    
\end{equation*}
%
In order to find the maximum of $f(Q)$ for $Q\in[0,(n-1)q]$, we compute
%
\begin{equation*}
\begin{split}
&f'(Q)=0\Rightarrow  a\cdot\ln x\cdot x^Q+b=0\Rightarrow Q=\dfrac{\ln\big(\frac{b}{a\cdot\ln (1/x)}\big)}{\ln x}
\end{split}    
\end{equation*}
%
Since $b>0$ and $\ln x<0$, we have that $f'$ is increasing. Thus, $\frac{\ln\big(\frac{b}{a\cdot\ln (1/x)}\big)}{\ln x}$ is a minimum for $f$. In addition, $p_b$ is typically a small value so $x$ is close to $1$. Consequently, we may assume that $\ln (1/x)$ is sufficiently small so that $\frac{b}{a\cdot\ln (1/x)}>1$. The latter implies that $\frac{\ln\big(\frac{b}{a\cdot\ln (1/x)}\big)}{\ln x}<0$, so given that $f'$ is increasing, we get that $f'(Q)>0$ for $Q\in[0,(n-1)q]$. Therefore, the maximum of $f$ in $[0,(n-1)q]$ is $(n-1)q$. 

By the above, we have that
\begin{equation*}
\begin{split}
f(Q)&\leq\Big(\tfrac{n-1}{n}\big(1-\tfrac{\log\secpar}{\sqrt{\rounds}}\big)(\rounds-1)(1-\pb)^q\cost_\msf{lc}\Big)(1-\pb)^{(n-1)q}+\\
&\quad+\Big(\big(1+\tfrac{\log\secpar}{\sqrt[4]{\rounds n}}\big)(\rounds-r^*)\pf\reward+\tfrac{n-1}{n}(r^*-1)\reward-\rounds \cost_\msf{ro}\Big)(n-1)q+\\
&\quad+\tfrac{n-1}{n}(r^*-1)q\reward-\tfrac{n-1}{n}\big(1-\tfrac{\log\secpar}{\sqrt{\rounds}}\big)(\rounds-1)\cost_\msf{lc}-\tfrac{n-1}{n}\rounds(\cost_\msf{fs}+\cost_\msf{tx})=\\
%
&=\big(1+\tfrac{\log\secpar}{\sqrt[4]{\rounds n}}\big)(\rounds-r^*)(n-1)q\pf\reward+(r^*-1)(n-1)q\reward-\\
&\quad-\tfrac{n-1}{n}\big(1-\tfrac{\log\secpar}{\sqrt{\rounds}}\big)(\rounds-1)(1-(1-\pb)^{nq})\cost_\msf{lc}-\tfrac{n-1}{n}\rounds(\cost_\msf{fs}+\cost_\msf{tx})-\rounds (n-1)q\cost_\msf{ro}.
\end{split}
\end{equation*}
%
Moreover, given that $\pf<\frac{1}{2}<\frac{1}{1+\tfrac{\log\secpar}{\sqrt[4]{\rounds n}}}$ and $r^*<\log^2\secpar$, we have that 
%
\begin{equation*}
\begin{split}
&\big(1+\tfrac{\log\secpar}{\sqrt[4]{\rounds n}}\big)(\rounds-r^*)(n-1)q\pf\reward+(r^*-1)(n-1)q\reward=\\
=&\big(1+\tfrac{\log\secpar}{\sqrt[4]{\rounds n}}\big)\rounds(n-1)q\pf\reward+\big(1-\big(1+\tfrac{\log\secpar}{\sqrt[4]{\rounds n}}\big)\pf\big)(n-1)q\reward r^*-(n-1)q\reward<\\
<&\big(1+\tfrac{\log\secpar}{\sqrt[4]{\rounds n}}\big)\rounds(n-1)q\pf\reward+\big(1-\big(1+\tfrac{\log\secpar}{\sqrt[4]{\rounds n}}\big)\pf\big)(n-1)q\reward \log^2\secpar-(n-1)q\reward=\\
=&\big(1+\tfrac{\log\secpar}{\sqrt[4]{\rounds n}}\big)(\rounds-\log^2\secpar)(n-1)q\pf\reward+(\log^2\secpar-1)(n-1)q\reward.
\end{split}
\end{equation*}
Therefore, we set the upper bound for $f(Q)$ as
%
\begin{equation}\label{eq:bound_case_1}
\begin{split}
B&=\big(1+\tfrac{\log\secpar}{\sqrt[4]{\rounds n}}\big)(\rounds-\log^2\secpar)(n-1)q\pf\reward+(\log^2\secpar-1)(n-1)q\reward-\\
&\quad-\tfrac{n-1}{n}\big(1-\tfrac{\log\secpar}{\sqrt{\rounds}}\big)(\rounds-1)(1-(1-\pb)^{nq})\cost_\msf{lc}-\tfrac{n-1}{n}\rounds(\cost_\msf{fs}+\cost_\msf{tx})-\rounds (n-1)q\cost_\msf{ro}.
\end{split}
\end{equation}
%
For this value of $B$ and by the Chernoff bounds, we have that %
\begin{equation*}
\begin{split}
&\Pr[\umax(\exec{})\geq B]\leq\\
%
\leq&\Pr\big[Z^+\geq\big(1+\tfrac{\log\secpar}{\sqrt[4]{\rounds n}}\big)(\rounds-r^*)Q\pf\big]+\negl(\secpar)\leq\\
%
\leq&e^{-\frac{\log^2\secpar}{3\sqrt{\rounds n}}(\rounds-r^*)Q\pf}+\negl(\secpar)\leq\\
%
\leq&e^{-\frac{\log^2\secpar}{3\sqrt{\rounds n}}(\rounds-\log^2\secpar)\sqrt{n}q\pf}+\negl(\secpar)\leq\negl(\secpar).
\end{split}
\end{equation*}

\textbf{Case 2:} $\rounds-r^*<\log^2\secpar$. By the fact that $Z^+<(\rounds-r^*)Q$, by Eq.~\eqref{eq:bound_all_cases}, we get that  
%
\begin{equation*}
\begin{split}
&\Pr[\umax(\exec{})\geq B]\leq\Pr\big[\tfrac{n-1}{n}Z^-\reward\geq B+\\
&+\tfrac{n-1}{n}\big(1-\tfrac{\log\secpar}{\sqrt{\rounds}}\big)(\rounds-1)(1-(1-\pb)^{q+Q})\cost_\msf{lc}+\\
&+\tfrac{n-1}{n}\rounds(\cost_\msf{fs}+\cost_\msf{tx})+\rounds Q\cost_\msf{ro}-(\rounds-r^*)Q\reward\big]+\\
&+\negl(\secpar).
\end{split}
\end{equation*}
%
To apply the Chernoff bounds for $Z^-$, we want to set $B$ such that for every $Q$, it holds that
\begin{equation*}
\begin{split}
& \tfrac{n}{n-1}B+\big(1-\tfrac{\log\secpar}{\sqrt{\rounds}}\big)(\rounds-1)(1-(1-\pb)^{q+Q})\cost_\msf{lc}+\\
&+\rounds(\cost_\msf{fs}+\cost_\msf{tx})+\tfrac{n}{n-1}\rounds Q\cost_\msf{ro}-\\
&-\tfrac{n}{n-1}(\rounds-r^*)Q\reward\geq\big(1+\tfrac{\log\secpar}{\sqrt[4]{\rounds n}}\big)(r^*-1)(q+Q)\pf\reward\Leftrightarrow\\
%
\Leftrightarrow&B\geq\Big(\tfrac{n-1}{n}\big(1-\tfrac{\log\secpar}{\sqrt{\rounds}}\big)(\rounds-1)(1-\pb)^q\cost_\msf{lc}\Big)(1-\pb)^Q+\\
+&\Big(\tfrac{n-1}{n}\big(1+\tfrac{\log\secpar}{\sqrt[4]{\rounds n}}\big)(r^*-1)\pf\reward+(\rounds-r^*)\reward-\rounds \cost_\msf{ro}\Big)Q+\\
+&\tfrac{n-1}{n}\big(1+\tfrac{\log\secpar}{\sqrt[4]{\rounds n}}\big)(r^*-1)q\pf\reward-\tfrac{n-1}{n}\big(1-\tfrac{\log\secpar}{\sqrt{\rounds}}\big)(\rounds-1)\cost_\msf{lc}-\tfrac{n-1}{n}\rounds(\cost_\msf{fs}+\cost_\msf{tx}).
\end{split}
\end{equation*}
%
Namely, we want to set $B$ as an upper bound of
\begin{equation*}
\begin{split}
&\Big(\tfrac{n-1}{n}\big(1-\tfrac{\log\secpar}{\sqrt{\rounds}}\big)(\rounds-1)(1-\pb)^q\cost_\msf{lc}\Big)(1-\pb)^Q+\\
+&\Big(\tfrac{n-1}{n}\big(1+\tfrac{\log\secpar}{\sqrt[4]{\rounds n}}\big)(r^*-1)\pf\reward+(\rounds-r^*)\reward-\rounds \cost_\msf{ro}\Big)Q+\\
+&\tfrac{n-1}{n}\big(1+\tfrac{\log\secpar}{\sqrt[4]{\rounds n}}\big)(r^*-1)q\pf\reward-\tfrac{n-1}{n}\big(1-\tfrac{\log\secpar}{\sqrt{\rounds}}\big)(\rounds-1)\cost_\msf{lc}-\tfrac{n-1}{n}\rounds(\cost_\msf{fs}+\cost_\msf{tx}).
\end{split}
\end{equation*}
%
%
To do so, we study the function $g(Q)=a'\cdot x^Q+b'\cdot Q+c'$, where
%
\begin{align*}
x&=1-\pb\\
    a'&=\tfrac{n-1}{n}\big(1-\tfrac{\log\secpar}{\sqrt{\rounds}}\big)(\rounds-1)(1-\pb)^q\cost_\msf{lc}\\
    b'&=\tfrac{n-1}{n}\big(1+\tfrac{\log\secpar}{\sqrt[4]{\rounds n}}\big)(r^*-1)\pf\reward+(\rounds-r^*)\reward-\rounds \cost_\msf{ro}\\
    c'&=\tfrac{n-1}{n}\big(1+\tfrac{\log\secpar}{\sqrt[4]{\rounds n}}\big)(r^*-1)q\pf\reward-\tfrac{n-1}{n}\big(1-\tfrac{\log\secpar}{\sqrt{\rounds}}\big)(\rounds-1)\cost_\msf{lc}-\tfrac{n-1}{n}\rounds(\cost_\msf{fs}+\cost_\msf{tx})
\end{align*}
%
If $\pf\reward>2\cost_\msf{ro}>\tfrac{Nn}{(N-1)(n-1)}\cost_\msf{ro}$, then it is easy to see that 
\begin{equation*}
\begin{split}
b'&=\tfrac{n-1}{n}\big(1+\tfrac{\log\secpar}{\sqrt[4]{\rounds n}}\big)(r^*-1)\pf\reward+(\rounds-r^*)\reward-\rounds \cost_\msf{ro}>\\
&>\tfrac{n-1}{n})(r^*-1)\pf\reward+(\rounds-r^*)\pf\reward-\rounds \cost_\msf{ro}>\\
&>\tfrac{n-1}{n})(\rounds-1)\pf\reward-\rounds \cost_\msf{ro}>0.
\end{split}    
\end{equation*}
%
In order to find the maximum of $g(Q)$ for $Q\in[0,(n-1)q]$, we compute
%
\begin{equation*}
\begin{split}
&g'(Q)=0\Rightarrow  a'\cdot\ln x\cdot x^Q+b'=0\Rightarrow Q=\dfrac{\ln\big(\frac{b'}{a'\cdot\ln (1/x)}\big)}{\ln x}
\end{split}    
\end{equation*}
%
Just like function $f$ in Case 1, we can conclude that the maximum of $g$ in $[0,(n-1)q]$ is $(n-1)q$. Thus, we have that
%
\begin{equation*}
\begin{split}
g(Q)&\leq\Big(\tfrac{n-1}{n}\big(1-\tfrac{\log\secpar}{\sqrt{\rounds}}\big)(\rounds-1)(1-\pb)^q\cost_\msf{lc}\Big)(1-\pb)^{(n-1)q}+\\
&\quad+\Big(\tfrac{n-1}{n}\big(1+\tfrac{\log\secpar}{\sqrt[4]{\rounds n}}\big)(r^*-1)\pf\reward+(\rounds-r^*)\reward-\rounds \cost_\msf{ro}\Big)(n-1)q+\\
&\quad+\tfrac{n-1}{n}\big(1+\tfrac{\log\secpar}{\sqrt[4]{\rounds n}}\big)(r^*-1)q\pf\reward-\tfrac{n-1}{n}\big(1-\tfrac{\log\secpar}{\sqrt{\rounds}}\big)(\rounds-1)\cost_\msf{lc}-\tfrac{n-1}{n}\rounds(\cost_\msf{fs}+\cost_\msf{tx})=\\
%
&=\big(1+\tfrac{\log\secpar}{\sqrt[4]{\rounds n}}\big)(r^*-1)(n-1)q\pf\reward+(\rounds-r^*)(n-1)q\reward-\\
&\quad-\tfrac{n-1}{n}\big(1-\tfrac{\log\secpar}{\sqrt{\rounds}}\big)(\rounds-1)(1-(1-\pb)^{nq})\cost_\msf{lc}-\tfrac{n-1}{n}\rounds(\cost_\msf{fs}+\cost_\msf{tx})-\rounds (n-1)q\cost_\msf{ro}.
\end{split}
\end{equation*}
%
%
Given that $\pf<\frac{1}{2}<\frac{1}{1+\tfrac{\log\secpar}{\sqrt[4]{\rounds n}}}$ and $\rounds-r^*<\log^2\secpar$, we have that 
%
\begin{equation*}
\begin{split}
&\big(1+\tfrac{\log\secpar}{\sqrt[4]{\rounds n}}\big)(r^*-1)(n-1)q\pf\reward+(\rounds-r^*)(n-1)q\reward=\\
=&\rounds(n-1)q\reward-\big(1-\big(1+\tfrac{\log\secpar}{\sqrt[4]{\rounds n}}\big)\pf\big)(n-1)q\reward r^*-\big(1+\tfrac{\log\secpar}{\sqrt[4]{\rounds n}}\big)(n-1)q\pf\reward<\\
%
<&\rounds(n-1)q\reward-\big(1-\big(1+\tfrac{\log\secpar}{\sqrt[4]{\rounds n}}\big)\pf\big)(n-1)q\reward (\rounds-\log^2\secpar)-\big(1+\tfrac{\log\secpar}{\sqrt[4]{\rounds n}}\big)(n-1)q\pf\reward=\\
=&\big(1+\tfrac{\log\secpar}{\sqrt[4]{\rounds n}}\big)(\rounds-\log^2\secpar-1)(n-1)q\pf\reward+\log^2\secpar(n-1)q\reward.
\end{split}
\end{equation*}
%
Therefore, we set the upper bound for $g(Q)$ as
%
\begin{equation}\label{eq:bound_case_2}
\begin{split}
B&=\big(1+\tfrac{\log\secpar}{\sqrt[4]{\rounds n}}\big)(\rounds-\log^2\secpar-1)(n-1)q\pf\reward+\log^2\secpar(n-1)q\reward-\\
&\quad-\tfrac{n-1}{n}\big(1-\tfrac{\log\secpar}{\sqrt{\rounds}}\big)(\rounds-1)(1-(1-\pb)^{nq})\cost_\msf{lc}-\tfrac{n-1}{n}\rounds(\cost_\msf{fs}+\cost_\msf{tx})-\rounds (n-1)q\cost_\msf{ro}.
\end{split}
\end{equation}
For this value of $B$ and by the Chernoff bounds, we have that %
\begin{equation*}
\begin{split}
&\Pr[\umax(\exec{})\geq B]\leq\\
%
\leq&\Pr\big[Z^-\geq\big(1+\tfrac{\log\secpar}{\sqrt[4]{\rounds n}}\big)(r^*-1)(q+Q)\pf\big]+\negl(\secpar)\leq\negl(\secpar).
\end{split}
\end{equation*}

%Since $Z^+<(\rounds-r^*)Q$ and by Eq.~\eqref{eq:bound_all_cases},
%
\iffalse
\begin{equation*}
\begin{split}
&\Pr[\umax(\exec{})\geq B]\leq\\
\leq&\Pr\big[\tfrac{n-1}{n}Z^-\reward\geq B+\\
+&\tfrac{n-1}{n}\big(1-\tfrac{\log\secpar}{\sqrt{\rounds}}\big)(\rounds-1)(1-(1-\pb)^{q+Q})\cost_\msf{lc}+\\
+&\tfrac{n-1}{n}\rounds(\cost_\msf{fs}+\cost_\msf{tx})+\rounds Q\cost_\msf{ro}-(\rounds-r^*)Q\reward\big]+\negl(\secpar).
\end{split}
\end{equation*}
%
To apply the Chernoff bounds for $Z^-$, we want to set $B$ such that for every $Q$, it holds that
\begin{equation*}
\begin{split}
& \tfrac{n}{n-1}B+\big(1-\tfrac{\log\secpar}{\sqrt{\rounds}}\big)(\rounds-1)(1-(1-\pb)^{q+Q})\cost_\msf{lc}+\\
&+\rounds(\cost_\msf{fs}+\cost_\msf{tx})+\tfrac{n}{n-1}\rounds Q\cost_\msf{ro}-\\
&-\tfrac{n}{n-1}(\rounds-r^*)Q\reward\geq\big(1+\tfrac{\log\secpar}{\sqrt[4]{\rounds n}}\big)(r^*-1)(q+Q)\pf\reward\Leftrightarrow\\
%
\Leftrightarrow&B\geq\Big(\tfrac{n-1}{n}\big(1-\tfrac{\log\secpar}{\sqrt{\rounds}}\big)(\rounds-1)(1-\pb)^q\cost_\msf{lc}\Big)(1-\pb)^Q+\\
+&\Big(\tfrac{n-1}{n}\big(1+\tfrac{\log\secpar}{\sqrt[4]{\rounds n}}\big)(r^*-1)\pf\reward+(\rounds-r^*)\reward-\rounds \cost_\msf{ro}\Big)Q+\\
+&\tfrac{n-1}{n}\big(1+\tfrac{\log\secpar}{\sqrt[4]{\rounds n}}\big)(r^*-1)q\pf\reward-\\
-&\tfrac{n-1}{n}\big(1-\tfrac{\log\secpar}{\sqrt{\rounds}}\big)(\rounds-1)\cost_\msf{lc}-\tfrac{n-1}{n}\rounds(\cost_\msf{fs}+\cost_\msf{tx}).
\end{split}
\end{equation*}
%
%Similarly to Case 1, we finally get that (cf. Appendix~\ref{app:case_2}) 
%
\fi

\textbf{Case 3:} $\log^2\secpar\leq r^*\leq \rounds-\log^2\secpar$. In this case, by the Chernoff bounds, we have that for $\delta\in(0,1)$
%
\begin{align*}
\Pr\big[Z^-\geq\big(1+\delta\big)(r^*-1)(q+Q)\pf\big]%&\leq e^{-\frac{\delta^2}{3}(r^*-1)(q+Q)\pf}=\\
=\negl(\secpar).\\
%
\Pr\big[Z^+\geq\big(1+\delta\big)(\rounds-r^*)Q\pf\big]%&\leq e^{-\frac{\delta^2}{3}(\rounds-r^*)Q\pf}=\\
=\negl(\secpar).
\end{align*}
%
By the above, with $1-\negl(\secpar)$ probability, it holds that 
\begin{equation*}
\begin{split}&\tfrac{n-1}{n}Z^- + Z^+<\\
<&\Big(\tfrac{n-1}{n}(r^*-1)(q+Q)+(\rounds-r^*)Q\Big)(1+\delta)\pf=\\
=&\Big(\big(\tfrac{n-1}{n}(q+Q)-Q\big)r^*-\tfrac{n-1}{n}(q+Q)+\rounds Q\Big)(1+\delta)\pf.
\end{split}
\end{equation*}
%
Since $Q\leq(n-1)q$, it holds that $\tfrac{n-1}{n}(q+Q)-Q\geq0$. So, given that $r^*\leq\rounds$, we have that with $1-\negl(\secpar)$ probability
\begin{equation*}
\begin{split}&\tfrac{n-1}{n}Z^- + Z^+<
%<&\Big(\big(\tfrac{n-1}{n}(q+Q)-Q\big)\rounds-\tfrac{n-1}{n}(q+Q)+\rounds Q\Big)(1+\delta)\pf=\\
\tfrac{n-1}{n}(\rounds-1)(q+Q)(1+\delta)\pf.
\end{split}
\end{equation*}
%
Thus, by Eq.~\eqref{eq:bound_all_cases}, we set $B$ such that for every $Q$, it holds 
\begin{equation*}
\begin{split}
& B+\tfrac{n-1}{n}\big(1-\tfrac{\log\secpar}{\sqrt{\rounds}}\big)(\rounds-1)(1-(1-\pb)^{q+Q})\cost_\msf{lc}+\\
&+\tfrac{n-1}{n}\rounds(\cost_\msf{fs}+\cost_\msf{tx})+\rounds Q\cost_\msf{ro}\geq\\
&\geq(1+\delta)\tfrac{n-1}{n}(\rounds-1)(q+Q)\pf\reward\Leftrightarrow\\
%
\Leftrightarrow&B\geq\Big(\tfrac{n-1}{n}\big(1-\tfrac{\log\secpar}{\sqrt{\rounds}}\big)(\rounds-1)(1-\pb)^q\cost_\msf{lc}\Big)(1-\pb)^Q+\\
+&\Big((1+\delta)\tfrac{n-1}{n}(\rounds-1)\pf\reward-\rounds \cost_\msf{ro}\Big)Q+\\
+&(1+\delta)\tfrac{n-1}{n}(\rounds-1)q\pf\reward -\tfrac{n-1}{n}\big(1-\tfrac{\log\secpar}{\sqrt{\rounds}}\big)(\rounds-1)\cost_\msf{lc}-\tfrac{n-1}{n}\rounds(\cost_\msf{fs}+\cost_\msf{tx}).
\end{split}
\end{equation*}
%
Namely, we want to set $B$ as an upper bound of
%
\begin{equation*}
\begin{split}
&\Big(\tfrac{n-1}{n}\big(1-\tfrac{\log\secpar}{\sqrt{\rounds}}\big)(\rounds-1)(1-\pb)^q\cost_\msf{lc}\Big)(1-\pb)^Q+\\
+&\Big((1+\delta)\tfrac{n-1}{n}(\rounds-1)\pf\reward-\rounds \cost_\msf{ro}\Big)Q+\\
+&(1+\delta)\tfrac{n-1}{n}(\rounds-1)q\pf\reward -\tfrac{n-1}{n}\big(1-\tfrac{\log\secpar}{\sqrt{\rounds}}\big)(\rounds-1)\cost_\msf{lc}-\tfrac{n-1}{n}\rounds(\cost_\msf{fs}+\cost_\msf{tx}).
\end{split}
\end{equation*}
%
%
To do so, we study the function $h(Q)=a''\cdot x^Q+b''\cdot Q+c''$, where
%
\begin{align*}
x&=1-\pb\\
    a''&=\tfrac{n-1}{n}\big(1-\tfrac{\log\secpar}{\sqrt{\rounds}}\big)(\rounds-1)(1-\pb)^q\cost_\msf{lc}\\
    b''&=(1+\delta)\tfrac{n-1}{n}(\rounds-1)\pf\reward-\rounds \cost_\msf{ro}\\
    c''&=(1+\delta)\tfrac{n-1}{n}(\rounds-1)q\pf\reward-\tfrac{n-1}{n}\big(1-\tfrac{\log\secpar}{\sqrt{\rounds}}\big)(\rounds-1)\cost_\msf{lc}-\tfrac{n-1}{n}\rounds(\cost_\msf{fs}+\cost_\msf{tx})
\end{align*}
%
If $\pf\reward>2\cost_\msf{ro}>\tfrac{Nn}{(N-1)(n-1)}\cost_\msf{ro}$, then it is easy to see that 
\begin{equation*}
\begin{split}
b''&=(1+\delta)\tfrac{n-1}{n}(\rounds-1)\pf\reward-\rounds \cost_\msf{ro}>0.
\end{split}    
\end{equation*}
%
In order to find the maximum of $h(Q)$ for $Q\in[0,(n-1)q]$, we compute
%
\begin{equation*}
\begin{split}
&h'(Q)=0\Rightarrow  a''\cdot\ln x\cdot x^Q+b''=0\Rightarrow Q=\dfrac{\ln\big(\frac{b''}{a''\cdot\ln (1/x)}\big)}{\ln x}
\end{split}    
\end{equation*}
%
Just like function $f$ in Case 1, we can conclude that the maximum of $h$ in $[0,(n-1)q]$ is $(n-1)q$. Thus, we have that
%
\begin{equation*}
\begin{split}
h(Q)&\leq\Big(\tfrac{n-1}{n}\big(1-\tfrac{\log\secpar}{\sqrt{\rounds}}\big)(\rounds-1)(1-\pb)^q\cost_\msf{lc}\Big)(1-\pb)^{(n-1)q}+\\
&\quad+\Big((1+\delta)\tfrac{n-1}{n}(\rounds-1)\pf\reward-\rounds \cost_\msf{ro}\Big)(n-1)q+\\
&\quad+(1+\delta)\tfrac{n-1}{n}(\rounds-1)q\pf\reward -\tfrac{n-1}{n}\big(1-\tfrac{\log\secpar}{\sqrt{\rounds}}\big)(\rounds-1)\cost_\msf{lc}-\tfrac{n-1}{n}\rounds(\cost_\msf{fs}+\cost_\msf{tx})=\\
%
&=(1+\delta)(\rounds-1)(n-1)q\pf\reward-\\
&\quad-\tfrac{n-1}{n}\big(1-\tfrac{\log\secpar}{\sqrt{\rounds}}\big)(\rounds-1)(1-(1-\pb)^{nq})\cost_\msf{lc}-\tfrac{n-1}{n}\rounds(\cost_\msf{fs}+\cost_\msf{tx})-\rounds (n-1)q\cost_\msf{ro}.
\end{split}
\end{equation*}
%
%
%
Therefore, we set the upper bound for $h(Q)$ as
%
%
\begin{equation}\label{eq:bound_case_3}
\begin{split}
B&=(1+\delta)(\rounds-1)(n-1)q\pf\reward-\\
&\quad-\tfrac{n-1}{n}\big(1-\tfrac{\log\secpar}{\sqrt{\rounds}}\big)(\rounds-1)(1-(1-\pb)^{nq})\cost_\msf{lc}-\tfrac{n-1}{n}\rounds(\cost_\msf{fs}+\cost_\msf{tx})-\rounds (n-1)q\cost_\msf{ro}.
\end{split}
\end{equation}
%
For this value of $B$, we get $\Pr[\umax(\exec{})\geq B]=\negl(\secpar).$
%\begin{equation*}
%\begin{split}
%&\Pr[\umax(\exec{})\geq B]=\negl(\secpar).
%\end{split}
%\end{equation*}
\smallskip

Given Cases 1,2, and 3, we provide a final bound that dominates all three upper bounds in Eq.~\eqref{eq:bound_case_1},~\eqref{eq:bound_case_2}, and~\eqref{eq:bound_case_3}, respectively. In particular, for any $\delta\in \big[\tfrac{\log\secpar}{\sqrt[4]{\rounds n}},1\big)$, we set
%
\begin{equation}\label{eq:final_bound}
\begin{split}
B&=(1+\delta)(\rounds-1)(n-1)q\pf\reward+\log^2\secpar(n-1)q\reward-\\
&\quad-\tfrac{n-1}{n}\big(1-\tfrac{\log\secpar}{\sqrt{\rounds}}\big)(\rounds-1)(1-(1-\pb)^{nq})\cost_\msf{lc}-\tfrac{n-1}{n}\rounds(\cost_\msf{fs}+\cost_\msf{tx})-\rounds (n-1)q\cost_\msf{ro}.
\end{split}
\end{equation}
%
Clearly, the above bound dominates the ones in Eq.~\eqref{eq:bound_case_1},~\eqref{eq:bound_case_2}, and and~\eqref{eq:bound_case_3}. Thus, for this value of $B$, we conclude that %
\begin{equation*}
\begin{split}
%&\Pr[\umax(\exec{})\geq B]=\negl(\secpar)\Rightarrow\\
\Pr[\umax(\exec{})\leq B]\geq1-\negl(\secpar).
\end{split}
\end{equation*}
%
Recall that the analysis so far was given that $Q\geq \sqrt{n}q$. To complete the proof of the claim, we will show that for $Q<\sqrt{n}q$, the profit of the coalition $\corrupt$ cannot exceed the bound in Eq.~\eqref{eq:final_bound}, except with $\negl(\secpar)$ probability. 

If $Q<\sqrt{n}q$, then all the parties make less than $(\sqrt{n}+1)q$ random oracle queries in total per round. Let $Z$ be the number of fruits mined during the execution and $\tilde{Z}$ a random variable that follows $\msf{Bin}(\rounds(\sqrt{n}+1)q,\pf)$. By the Chernoff bounds,
%
\begin{equation*}
\begin{split}
\Pr\big[Z\geq\big(1+\tfrac{\log\secpar}{\sqrt[4]{\rounds n}}\big)\rounds(\sqrt{n}+1)q\pf\big]%&\leq e^{-\frac{\log^2\secpar}{3\sqrt{\rounds n}}\rounds(\sqrt{n}+1)q\pf}=\\
=\negl(\secpar).  
\end{split}    
\end{equation*}
%
The latter implies that with $1-\negl(\secpar)$ probability the total rewards, that are clearly greater than the profit of the coalition, are no more than $\big(1+\tfrac{\log\secpar}{\sqrt[4]{\rounds n}}\big)\rounds(\sqrt{n}+1)q\pf\reward$. 

We show that if $\pf\reward>3\big(\tfrac{\cost_\msf{lc}+\cost_\msf{fs}+\cost_\msf{tx}}{(n-1)q}+\cost_\msf{ro}\big)$, then it holds that for any $\delta\in \big[\tfrac{\log\secpar}{\sqrt[4]{\rounds n}},1\big)$,
%
\begin{equation}\label{eq:less_nq}
\begin{split}
&\big(1+\tfrac{\log\secpar}{\sqrt[4]{\rounds n}}\big)\rounds(\sqrt{n}+1)q\pf\reward<\\
<&(1+\delta)(\rounds-1)(n-1)q\pf\reward+\log^2\secpar(n-1)q\reward-\\
&-\tfrac{n-1}{n}\big(1-\tfrac{\log\secpar}{\sqrt{\rounds}}\big)(\rounds-1)(1-(1-\pb)^{nq})\cost_\msf{lc}-\tfrac{n-1}{n}\rounds(\cost_\msf{fs}+\cost_\msf{tx})-\rounds (n-1)q\cost_\msf{ro}.
\end{split}
\end{equation}
%
Namely, since $\big(1+\tfrac{\log\secpar}{\sqrt[4]{\rounds n}}\big)\rounds(\sqrt{n}+1)<\tfrac{1}{2}(\rounds-1)(n-1)$ for typical values of $\rounds,n$, it holds that
\[\big(1+\tfrac{\log\secpar}{\sqrt[4]{\rounds n}}\big)\rounds(\sqrt{n}+1)q\pf\reward<\tfrac{1}{2}(1+\delta)(\rounds-1)(n-1)q\pf\reward.\]
%
Besides, if $\pf\reward>3\big(\tfrac{\cost_\msf{lc}+\cost_\msf{fs}+\cost_\msf{tx}}{(n-1)q}+\cost_\msf{ro}\big)$, then
%
\begin{equation*}
\begin{split}
&(1+\delta)(\rounds-1)(n-1)q\pf\reward+\log^2\secpar(n-1)q\reward-\\
&-\tfrac{n-1}{n}\big(1-\tfrac{\log\secpar}{\sqrt{\rounds}}\big)(\rounds-1)(1-(1-\pb)^{nq})\cost_\msf{lc}-\\
&-\tfrac{n-1}{n}\rounds(\cost_\msf{fs}+\cost_\msf{tx})-\rounds (n-1)q\cost_\msf{ro}>\\
>&(\rounds-1)(n-1)q\pf\reward-\rounds\cost_\msf{lc}-\rounds(\cost_\msf{fs}+\cost_\msf{tx})-\rounds (n-1)q\cost_\msf{ro}=\\
%
=&(\rounds-1)(n-1)q\pf\reward-\rounds(n-1)q\big(\tfrac{\cost_\msf{lc}+\cost_\msf{fs}+\cost_\msf{tx}}{(n-1)q}+\cost_\msf{ro}\big)>\\
%
>&(\rounds-1)(n-1)q\pf\reward-\tfrac{1}{3}\rounds(n-1)q\pf\reward>\\
%
>&(\rounds-1)(n-1)q\pf\reward-\tfrac{1}{2}(\rounds-1)(n-1)q\pf\reward=\\
=&\tfrac{1}{2}(\rounds-1)(n-1)q\pf\reward.
\end{split}
\end{equation*}
By the above, we get Eq.~\eqref{eq:less_nq}, which completes the proof of the claim.
\hfill$\dashv$