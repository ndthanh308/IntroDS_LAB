%
%Next, we show that $\mc{A}$ does not gain in terms of profit when the its strategy is a combination of D1, D4-D7, D9-D12.  
%
\begin{claim}\label{claim:other_deviations}
For every   $\mc{O}_\msf{tx}$-respecting adversary $\mc{A}$ that performs a combination of deviations D1, D4-D7, D9-D12 and every $\rounds$-admissible environment $\mc{Z}$ that activates the pool leader first in each round, it holds $\umax(\exec{}) \leq \umin(\honexec)$ with $1-\negl(\secpar)$ probability.
\end{claim}
\textit{Proof of Claim~\ref{claim:other_deviations}. }
D1 is captured by D6 due to step (2) (cf. Figure~\ref{fig:single_leader}) and step (2a) (cf. Figure~\ref{fig:single_other}) in the $\single$ protocol for the leader and the non leader, respectively. In more detail, if a party in $\corrupt$ does not update $inst_i$, $i\in[4]$ as instructed by $\single$ and sends its inconsistent fruits and/or blocks, then during the next round the honest parties following $\single$ will dissolve the pool. This happens because the honest parties will detect the deviation. Thus, the outcome of D1  can be captured by D6 where at some round $\mc{A}$ instructs a subset of the parties in $\mathbf{C}$ to abandon the pool. 

D12 is captured by D6 for the case where all the corrupted parties abandon the pool and follow $\protocolevp$ protocol. This happens because if the leader does not pay a non leader party, then this party will detect this via steps (2b) and (3), it will leave the pool and it will follow $\protocolevp$.  

D7 is not effective in our setting, because the parties that are not corrupted by $\mc{A}$  follow the $\single$ protocol and thus they will never produce a block $\hat{\block}:=\langle\langle \hat{h}_{-1},\hat{h}_f,\hat{\nonce},$ $\hat{\msf{dig}},\hat{\record},\hat{h}\rangle,\hat{\mbf{F}}\rangle$ or a fruit $\langle \hat{h}_{-1},\hat{h}_f,\hat{\nonce},\hat{\msf{dig}},\hat{\record},\hat{h}\rangle$ so that $(\hat{\previous},\hat{h}_f,\hat{\msf{dig}},\hat{\record})\neq(inst_1,inst_2,inst_3,inst_4)$. 

D10 is not performed by an $\mc{O}_\tx$-respecting adversary. 

D6 is captured by any combination of D2-D5 and D7-D12:  let us assume that a subset of the corrupted parties abandons the pool and creates a new pool following different instructions from the $\single$ pool. Recall that the utility of the adversary is the sum of the utilities of all the corrupted parties. Thus, the way of sharing the rewards among the corrupted parties does not affect the utility of the adversary.   

D9, D11 have the same effect as D3 for the case where $\mc{A}$ instructs all the corrupted parties to abstain by asking no queries to the random oracle $\mc{O}_\msf{ro}$. The reason is that if the pool leader does not ask the oracles $\mc{O}_\msf{fs}$, $\mc{O}_\msf{lc}$, it cannot create $inst_i$, $i\in[4]$ needed for all the parties to ask the random oracle and produce valid fruits that will give the rewards to the pool when a block is produced.  

D4 will offer to the adversary lower utility than D3 for the case where $\mc{A}$ instructs all the corrupted parties to abstain by asking no queries the random oracle $\mc{O}_\msf{ro}$. The reason that is if the adversary asks the random oracle but does not send its fruits or blocks, it incurs the cost of $C_\msf{ro}$ without getting any more rewards from the fruits it produces. 

Regarding D5, we do not consider deviations that either hinge on the assumption that the blocks can include unlimited number of fruits and/or they demand that the adversary is aware of the round when $\mc{Z}$ will terminate the execution.   
%
%
%
\hfill$\dashv$
\\






