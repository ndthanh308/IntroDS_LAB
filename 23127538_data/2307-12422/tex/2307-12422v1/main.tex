 \documentclass[10pt,a4paper]{article}
\usepackage[utf8]{inputenc}
        \usepackage{geometry}
        \geometry{left=32mm,right=32mm}
\usepackage{authblk}
\usepackage{cite}
\usepackage{amsmath,amssymb,amsfonts,amsthm}
\usepackage{algorithmic}
\usepackage{graphicx}
\usepackage{textcomp}
\usepackage{xcolor}
\usepackage{mdframed}
\usepackage{boxedminipage}
\usepackage{caption}
\usepackage{subcaption}
\usepackage{float}
\usepackage{enumerate}
\usepackage{hyperref}
\def\BibTeX{{\rm B\kern-.05em{\sc i\kern-.025em b}\kern-.08em
    T\kern-.1667em\lower.7ex\hbox{E}\kern-.125emX}}
\usepackage{enumitem}
\setlist{topsep=2pt, itemsep=1pt}

\newcommand\calF{\mathcal{F}}
\newcommand\calG{\mathcal{G}}
\newcommand\calM{\mathcal{M}}
\newcommand\calV{\mathcal{V}}
\newcommand\calU{\mathcal{U}}
\newcommand\calW{\mathcal{W}}
\newcommand\calP{\mathcal{P}}
\newcommand\calD{\mathbb{D}}
%%%%%%%%%%%%%%%%%
%% macros introduced by Luke 
\newcommand\mydef[1]{{\bf\em #1}}
%%%%%%%%%%%%%%%%%

\newcommand{\numviparams}{{| \lambda |}}
\newcommand{\scoreaccvars}[1]{s_1^{#1}, \ldots, s_{\numviparams}^{#1}}
\newcommand{\scoreaccvar}[2]{s_{#1}^{#2}}
\newcommand{\isdeterm}[1]{\text{Deterministic}({#1})}


\newcommand{\expect}[1]{\mathbb{E}\left[{#1}\right]}
\newcommand{\var}[1]{\mathbb{V}\left[ {#1} \right]}
\newcommand{\expectdist}[2]{\mathbb{E}_{#1}\left[ {#2} \right]}
\newcommand{\vardist}[2]{\mathbb{V}_{#1}\left[ {#2} \right]}
\newcommand{\cov}[2]{\mathbb{C}\text{ov}[{#1}][{#2}]}
\newcommand{\covv}[1]{\mathbb{C}\text{ov}[{#1}]}
\newcommand{\corr}[1]{\mathbb{C}\text{orr}[{#1}]}

\newcommand{\fix}[1]{\mathit{fix}\left({#1}\right)}
\newcommand{\sbr}[1]{\left\llbracket {#1} \right\rrbracket}
\newcommand{\ctxtype}[3]{{#1} \cong_\text{ctx} {#2} : {#3}}
\newcommand{\bigstep}[3]{{#1} \Downarrow_{#2} {#3}}


% PCF types
\newcommand{\bool}{\mathit{bool}}
\newcommand{\nat}{\mathit{nat}}

\newcommand{\ctx}[1]{\mathcal{C}\left[ {#1}\right] }
\newcommand{\pcft}[1]{\text{PCF}_{#1}}

\newcommand{\nfl}{\mathbb{N}_\bot}
\newcommand{\bfl}{\mathbb{B}_\bot}

% PCF constructs
\newcommand{\succc}[1]{\mathbf{succ}({#1})}
\newcommand{\succcn}[2]{\mathbf{succ}^{#1}({#2})}
\newcommand{\zero}{\mathbf{0}}
\newcommand{\zerotest}[1]{\mathbf{zero}\left({#1}\right)}
\newcommand{\pred}[1]{\mathbf{pred}\left( {#1} \right)}
\newcommand{\predn}[2]{\mathbf{pred}^{#1}\left( {#2} \right)}
\def\solvable{\#}

\newcommand{\true}{\mathbf{true}}
\newcommand{\false}{\mathbf{false}}
\newcommand{\pcffix}[1]{\mathbf{fix}\left({#1}\right)}
\newcommand{\pcffn}[3]{\mathbf{fn}~{#1}:{#2}\mathpunct{.}{#3}}
\newcommand{\pairtype}[2]{{#1} * {#2}}
\newcommand{\pairexp}[2]{\mathbf{pair}({#1}, {#2})}
\newcommand{\leftexp}[1]{\mathbf{left}({#1})}
\newcommand{\rightexp}[1]{\mathbf{right}({#1})}

\newcommand{\RationalPos}{\mathbb{Q}^{+}}

\newcommand{\meas}[1]{\mathbb{M}\left( {#1} \right) }
\newcommand{\integ}[1]{\sbr{#1}_I}

\newcommand{\notbigstep}[2]{{#1}~\cancel{\Downarrow}_{#2}}
\newcommand{\subtrace}[3]{{#1}^{{#2} \ldots {#3}}}
\newcommand{\supp}[1]{\textsf{supp}\left({#1}\right)}
\newcommand{\dom}[1]{\textsf{Dom}\left({#1}\right)}
\newcommand{\suppk}[2]{\textsf{Supp}^{#1}\left({#2}\right)}
\newcommand{\tracespace}{\bigcup_{n \in \mathbb{N}}[0, 1]^n}
\newcommand{\generictracespace}{\mathbb{T}}
\newcommand{\nnreals}{\mathbb{R}_{\geq 0}}
\newcommand{\posreals}{\mathbb{R}_{> 0}}
\newcommand{\reals}{\mathbb{R}}

\newcommand{\unrollkM}[2]{\textsf{unroll}_{#1}\left({#2}\right)}
\newcommand{\nphmcint}[5]{\Psi_\textsf{NP}\left({#1}, {#2}, {#3}, {#4}, {#5}\right)}

%SPCF constructs
\newcommand{\spcfvalues}{\Lambda^0_v}

\newcommand{\prevalueM}[1]{\textsf{value}^{-1}_{#1}(\spcfvalues{})}
\newcommand{\num}[1]{\underline{#1}}

% \theoremstyle{definition}
% \newtheorem{thm}{Theorem}
% \newtheorem{lem}{Lemma}
% \newtheorem{defn}{Definition}
% \newtheorem{conj}{Conjecture}
% \newtheorem{prop}{Proposition}

%\theoremstyle{definition}
%\newtheorem{defn}{Definition}[section]
%\newtheorem{example}[defn]{Example}
%
%
%\theoremstyle{plain}
%\newtheorem{thm}{Theorem}[section]
%\newtheorem{lem}[thm]{Lemma}
%\newtheorem{cor}[thm]{Corollary}
%\newtheorem{conj}[thm]{Conjecture}
%\newtheorem{prop}[thm]{Proposition}
%\newtheorem{remark}[thm]{Remark}

%% Proofs
%\let\oldproof\proof
%\renewcommand{\proof}{\color{blue}\oldproof}


\definecolor{codegreen}{rgb}{0,0.6,0}
\definecolor{codegray}{rgb}{0.5,0.5,0.5}
\definecolor{codepurple}{rgb}{0.58,0,0.82}
\definecolor{backcolour}{rgb}{0.95,0.95,0.92}

\lstdefinestyle{myStyle}{
    belowcaptionskip=1\baselineskip,
    breaklines=true,
    frame=none,
    basicstyle=\footnotesize\ttfamily,
    keywordstyle=\bfseries\color{green!40!black},
    commentstyle=\itshape\color{purple!40!black},
    identifierstyle=\color{blue},
    backgroundcolor=\color{gray!10!white},
    %backgroundcolor=\color{backcolour}, 
    numberstyle=\tiny\color{codegray},
    stringstyle=\color{codepurple},
    breakatwhitespace=false,                          
    keepspaces=true,                 
    numbers=left,       
    numbersep=5pt,                  
    showspaces=false,                
    showstringspaces=false,
    showtabs=false,                  
    tabsize=2,
}

% argmin/argmax
\DeclareMathOperator*{\argmax}{arg\,max}
\DeclareMathOperator*{\argmin}{arg\,min}

% Concatenation of lists
\newcommand\doubleplus{+\kern-1.3ex+\kern0.8ex}

% Program configurations
\newcommand{\tuple}[1]{\ensuremath{\langle #1 \rangle}}
% Rule based definitions
\newcommand{\Rule}[4][]{\ensuremath{\inferrule*[lab={\hypertarget{#2}{(\TirName{#2})}},#1]{#3}{#4}}}

% Calligraphic symbols
\newcommand{\calI}{{\mathcal I}} 
\newcommand{\calT}{{\mathcal T}}

%  Macro for new Y operator.
\newcommand{\yBounded}[3]{\mu^{#1}_{#2}\rvert_{#3}}

%%%%%%%%%%%%%%%%%
 
%%%%%%%%%%%%%%%%%

\newcommand{\expv}{\mathbb{E}}

\newcommand{\combTr}[2]{\left[\begin{matrix}
		#1\\
		#2
	\end{matrix} \right]}

\newcommand{\exType}[2]{\left\{\begin{matrix}
		#1\\
		#2
	\end{matrix} \right\}}
\newcommand{\myint}[1]{ [#1]}
\newcommand{\Uniform}{\ensuremath{\mathrm{Uniform}}}
\newcommand{\Normal}{\ensuremath{\mathrm{normal}}}
\DeclareMathOperator{\abs}{abs}
\DeclareMathOperator{\pdf}{pdf}

\newcommand{\intConf}[1]{\lceil#1\rceil}
\newcommand{\tr}{\boldsymbol{t}}

\newcommand{\sample}{\tt{sample}}
%\newcommand{\fix}{\texttt{fix}}
%\newcommand{\num}[1]{\underline{#1}}
\newcommand{\myif}{\texttt{if}}
\newcommand{\mylet}{\texttt{let} \, }
\newcommand{\myin}{\, \texttt{in} \,}
\newcommand{\mythen}{\, \texttt{then} \,}
\newcommand{\myelse}{\, \texttt{else} \,}
\newcommand{\score}{\tt{score}}
\newcommand{\tick}{\tt{tick}}

\newcommand{\term}{\tt{term}}
\newcommand{\pv}{\mathbf{v}}
\newcommand{\rv}{\mathbf{r}}

\newcommand{\interval}{\mathfrak{I}}

\newcommand{\typeReal}{\textbf{\textsf{R}}}

\newcommand{\symbolInt}{\myint{\cdot}}

\newcommand{\LambdaInterval}{\Lambda_{\interval}}
\newcommand{\LambdaSymbolic}{\Lambda_{\text{sym}}}

\newcommand{\toIntervalTerm}[1]{#1^{2\interval}}

%Others
\newcommand{\Sset}{\mathbb{S}}
\newcommand{\Iset}{\mathbb{I}}
\newcommand{\Rset}{\mathbb{R}}
\newcommand{\Nset}{\mathbb{N}}
\newcommand{\Zset}{\mathbb{Z}}

\newcommand{\Term}{\mathbb{T}}
\newcommand{\prob}{\mathbb{P}}
\newcommand{\expt}{\mathbb{E}}


\newcommand{\Leb}{\tt{Leb}}
\newcommand{\Red}{\tt{Red}}
\newcommand{\cost}{\text{cost}}

%\newcommand{\intervalab}[2]{\underline{[#1,#2]}}
\newcommand{\intervalab}{\underline{[a,b]}}
\newcommand{\interI}{\mathcal{I}}
\newcommand{\trans}{\mathcal{T}}

\newcommand{\iv}{\mathbb{I}}

% Programming language constructs
\newcommand{\lit}[1]{\underline{#1}}
\newcommand{\letIn}[1]{\mathsf{let}\,{#1}\,\mathsf{in}\,}
\newcommand{\fixLam}[2]{\mu {#1} {#2}.}
\newcommand{\ifElse}[3]{\mathsf{if} (#1 \le \num{0}) \, {#2} \,\mathsf{else}\, {#3}}

%%Basic notions
\newcommand{\pspace}{(\Omega,\mathcal{F},\probm)}
\newcommand{\probm}{\mathbb{P}}
\newcommand{\condexpv}[2]{{\expt}{\left[{#1} \mid {#2}\right]}}

\newcommand{\stdConf}[1]{(#1)}
%\newcommand{\intConf}[1]{\lceil#1\rceil}
%\newcommand{\intConf}[1]{(#1)}
%\newcommand{\symConf}[1]{\langle\!\langle  #1 \rangle\!\rangle}
%\newcommand\symPath[1]{(#1)}
\newcommand{\symPath}[1]{\langle\!\langle  #1 \rangle\!\rangle}
\newcommand\symConf[1]{(#1)}

\newcommand{\ifSimple}[3]{\mathsf{if}(#1, #2, #3)}
%\newcommand{\ifElse}[3]{\mathsf{if} (#1 \le 0) \, \allowbreak {#2} \, \allowbreak \mathsf{else}\, {#3}}
%\newcommand{\ifElse}[3]{\ifSimple{#1}{#2}{#3}}

%\newcommand{\trace}{\mathsf{s}}
%
%\newcommand\defn[1]{{\bf \em #1}}
\newcommand{\traces}{\mathbb{T}}
%
%\newcommand{\stdConf}[1]{(#1)}
%%\newcommand{\intConf}[1]{\lceil#1\rceil}
%\newcommand{\intConf}[1]{(#1)}
%%\newcommand{\symConf}[1]{\langle\!\langle  #1 \rangle\!\rangle}
%%\newcommand\symPath[1]{(#1)}
%\newcommand{\symPath}[1]{\langle\!\langle  #1 \rangle\!\rangle}
%\newcommand\symConf[1]{(#1)}

\newcommand{\valueSem}[1]{\mathsf{val}_{#1}} % value (semantics)
\newcommand{\weightSem}[1]{\mathsf{wt}_{#1}} % weight (semantics)
\newcommand{\measureSem}[1]{\llbracket #1 \rrbracket}
\newcommand{\posterior}{\mathsf{posterior}}


%%%%%%%%%
% 
%%%%%%%%
\newcommand{\loc}{\ell}
\newcommand{\locs}{\mathit{L}}
\newcommand{\blocs}{\mathit{L}_{\mathrm{b}}}

\newcommand{\iflocs}{\mathit{L}_{\mathrm{if}}}
\newcommand{\looplocs}{\mathit{L}_{\mathrm{while}}}

\newcommand{\alocs}{\mathit{L}_{\mathrm{a}}}
\newcommand{\wlocs}{\mathit{L}_{\mathrm{w}}}
\newcommand{\rlocs}{\mathit{L}_{\mathrm{r}}}
\newcommand{\Alocs}[1]{\mathit{L}_{\mathrm{A}}^{\mathsf{#1}}}
\newcommand{\Dlocs}{\mathit{L}_{\mathrm{nd}}}
\newcommand{\transitions}{{\rightarrow}}

%%% 
\newcommand{\plocs}{\mathit{L}_{\mathrm{p}}}
\newcommand{\tlocs}{\mathit{L}_{\mathrm{t}}}

\newcommand{\lin}{\loc_\mathrm{init}}
\newcommand{\lout}{\loc_\mathrm{out}}
\newcommand{\val}[1]{\mbox{\sl Val}_{#1}}

\newcommand{\pvars}{V_\mathrm{p}}
\newcommand{\rvars}{V_{\mathrm{r}}}
\newcommand{\pre}{\mathrm{pre}}

\newcommand{\sle}{\sqsubseteq}
\newcommand{\sge}{\sqsupseteq}

\newcommand{\lfp}{\mathrm{lfp}}
\newcommand{\gfp}{\mathrm{gfp}}

\newcommand{\rdvarjdis}{\mathcal D}
\newcommand{\sampset}{\textit{supp}}

\newcommand{\upd}{\mbox{\sl upd}}
\newcommand{\wet}{\mbox{\sl wt}}
\newcommand{\transset}{\mathfrak T}
\newcommand{\valin}{\pv_{\mathrm{init}}}
\newcommand{\ret}{\mbox{\sl ret}}

\newcommand{\win}{w_{\mathrm{init}}}

\newcommand{\sampdpd}{\overline{\Upsilon}}

\newcommand{\outmap}{\text{O}}
\newcommand{\sat}[1]{\langle #1 \rangle}
\newcommand{\monoid}{\mbox{\sl Monoid}}
\newcommand{\handelmanformat}{(\dagger)}

\newcommand{\trunc}{\mathcal{B}}

\newcommand{\ewt}{\mbox{\sl ewt}}
\newcommand{\statemap}{\text{St}}

\newcommand{\valrd}{{\mathbf{r}}}
\newcommand{\frmloc}{\ell^{\mathrm{src}}}
\newcommand{\toloc}{\ell^{\mathrm{dst}}}

\newcommand{\monomials}{\mathbf{M}}

%------------------ENVIRONMENTS--------------------------
\setlength{\fboxrule}{1.3pt}
\newenvironment{thickframe}{\begin{center}\begin{boxedminipage}{0.98\textwidth}}{\end{boxedminipage}\end{center}}

\newtheorem{theorem}{Theorem}
\newtheorem{proposition}{Proposition}
\newtheorem{definition}{Definition}
\theoremstyle{definition}
\newtheorem{remark}{Remark}
\theoremstyle{theorem}

\author[1]{Aikaterini-Panagiota Stouka}
\author[2]{Thomas Zacharias}

\affil[1]{Nethermind, UK}
\affil[2]{The University of Edinburgh, UK}
 \date{}
\begin{document}
\newtheorem{claim}{Claim}[theorem]

\title{On the (De)centralization of FruitChains}
%



\maketitle

\begin{abstract}
One of the most important features of blockchain protocols is
decentralization, as their main contribution is that they formulate a distributed ledger that will be maintained and extended without the need of a trusted party. Bitcoin has been criticized for its tendency to centralization, as very few pools control the majority of the hashing power. Pass et al. proposed FruitChain [PODC 17] and claimed that this blockchain protocol mitigates the formation of pools by reducing the variance of the rewards in the same way as mining pools, but in a fully decentralized fashion. Many follow up papers consider that the problem of centralization in Proof-of-Work (PoW) blockchain systems can be solved via lower rewards' variance, and that in FruitChain the formation of pools is unnecessary. 

Contrary to the common perception, in this work, we prove that lower variance of the rewards does not eliminate the tendency of the PoW blockchain protocols to centralization; miners have also other incentives to create large pools, and specifically to share the cost of creating the instance they need to solve the PoW puzzle.

We abstract the procedures of FruitChain as oracles and assign to each of them a cost. Then, we provide a formal definition of a pool in a blockchain
system, and by utilizing the notion of equilibrium with virtual payoffs (EVP) [AFT 21], we prove that there is a completely centralized EVP, where all the parties form a single pool controlled by one party called the pool leader. The pool leader is responsible for creating the instance used for the PoW procedure. To the best of our knowledge, this
is the first work that examines the construction of mining
pools in the FruitChain system. \smallskip

\noindent\textbf{Keywords:} FruitChains, decentralization, incentives, Proof-of-Work, mining pools
\end{abstract}


\section{Introduction}
Current quantum hardware is unable to carry out universal quantum computations due to the buildup of errors that occur during the computation. 
The magnitude of the individual error is currently above the value that the Threshold Theorem requires in order to kick-start quantum error correction and fault-tolerant quantum computation~\cite[Section 10.6]{nielsen_chuang_2010}. 
Although the experimentally achieved fidelity rates are promising and the error bounds are inching closer to the required threshold, we will have to work for the foreseeable future with quantum hardware with errors that build-up during the computation.  This implies that we can only do a limited number of steps before the output of the computation has become completely uncorrelated with the intended one.

For fault-tolerant quantum computing, we repeat four steps: 
1) We apply a number of single and two-qubit quantum gates, in parallel whenever possible; 
2) We perform a syndrome measurement on a subset of the qubits; 
3) We perform fast classical computations to determine which errors have occurred and how to correct them; 
and, 4) We apply correction terms based on the classical computations.
We then repeat these four steps with a next sequence of gates. 
These four steps are essential to fault-tolerant quantum computing. 


The starting point of this work is to use the four steps outlined above, not to carry out error correction and fault-tolerant computation, but to enhance short, constant-depth, {\em uncorrected} quantum circuits that perform single qubit gates and {\em nearest-neighbor} two qubit gates. 
Since in the long run we will have to implement error-correction and fault-tolerant computation anyhow, and this is done by such a four-step process, why not make other use of this architecture? Moreover, on some of the quantum hardware platforms, these operations are already in place.
Embracing this idea we naturally arrive at the question: what is the computational power of \textit{low-depth} quantum-classical circuits organized as in the four steps outlined above? 
We thus investigate circuits that execute a small, ideally constant, number of stages, where at each stage we may apply, in parallel, single qubit gates and {\em nearest-neighbor} two qubit gates, followed by measurements, followed by low-depth classical computations of which the outcome can control quantum gates in later stages. 
It is not clear, at first, whether such circuits, especially with constant depth, can do anything remotely useful. 
But we will see that this is indeed the case: many quantum computations can be done by such circuits in constant depth. 
By parallelizing quantum computations in this way, we improve the overall computational capabilities of these circuits, as we do not incur errors on qubits that are idle, simply because qubits are not idle for a very long time. 
Furthermore, reducing the depth of quantum circuits, at the cost of increasing width, allows the circuit to be run faster even if errors occur.

The first usage of such a four-step layout, not to do error correction, but to perform computations, can be found in the paradigm of measurement-based quantum computing~\cite{gottesman1999demonstrating,raussendorf2001one,jozsa2006introduction,clark2007generalised}: 
A universal form of quantum computing where a quantum state is prepared and operations are performed by measuring qubits in different bases, depending on previous measurements and intermediate measurements.

\citeauthor{PhamSvore2013} were the first to formalize the four-step protocol for performing computations~\cite{PhamSvore2013}. They included specific hardware topologies by considering two-dimensional graphs for imposing constraints on qubit interactions. In their model, they develop circuits for particularly useful multi-qubit gates, including specifying costs in the width, number of qubits, depth, number of concurrent time steps, size, and total number of non-Identity operations.
As a result, they find an algorithm that factors integers in polylogarithmic depth.
\citeauthor{Browne:2011} showed that the main tool in the work by \citeauthor{PhamSvore2013}, the fan-out gate, can also be replaced by additional log-depth classical computations in the measurement-based quantum computing setting~\cite{Browne:2011}.

More recently, \citeauthor{Cirac:2021} introduced a scheme to implement unitary operations involving quantum circuits combined with Local Operations and Classical Communication ($\mathsf{LOCC}$) channels: $\mathsf{LOCC}$-assisted quantum circuits~\cite{Cirac:2021}. Similarly to the four-step scheme we just described, they allow for a short depth circuit to be run on the qubits, followed by one round of $\mathsf{LOCC}$, in which ancilla qubits are measured and local unitaries are applied based on the measurement outcomes. They show that in this model any 1D transitionally invariant matrix-product state (MPS) with fixed bond dimension is in the same phase of matter as the trivial state. Similar ideas can be found in~\cite{TVV_NonAbelianTopologicalOrder_2022, tantivasadakarn2021long}.

In this work, we introduce a new model, called \textit{Local Alternating Quantum-Classical Computations} ($\LAQCC$). In this model we alternate between running quantum circuits (constrained by locality), ending in the measurement of a subset of qubits, and fast classical computations based on the measurement results. The outcome of the classical computations are then used to control future quantum circuits. We allow for flexibility in this model, by giving different constraints to the power of both the quantum circuits and the classical circuits as well as the number of alternations between them. 
Most attention will be given to $\LAQCC$ containing quantum circuits of constant depth, classical circuits of logarithmic depth and at most a constant number of alternations between them. 
Any circuit constructed in this model is considered to be of constant depth. 
We restrict ourselves to logarithmic depth classical computations, as this is the first natural and non-trivial extension beyond constant-depth classical computations. 
Constant-depth classical computations do however also have an equivalent constant-depth quantum implementation.

The definition of $\LAQCC$ sharpens the original definition of \citeauthor{PhamSvore2013} by adding constraints to the intermediate classical computations. This allows us to bound the power of $\LAQCC$ from above. 

The main result of \citeauthor{Cirac:2021}, that 1D translational invariant MPS with fixed bond dimension can be prepared by $\mathsf{LOCC}$-assisted circuits, relies on local symmetries of the MPS. These symmetries allow them to prepare local states (on a constant number of qubits) and glue them together by doing one round of the appropriate entangling measurement and corrections, after which they run a round of local unitaries to get the desired result. This general scheme for preparing states that exhibit an MPS description with the appropriate local symmetries requires only geometrically local unitaries and one round of measurement and corrections an therefore is accessible in $\LAQCC$. Studying different local symmetries, known as Symmetry Protected Topological (SPT) phases of matter, to find measurement-based constant depth circuits for states is a broad ongoing field of research~\cite{TVV_NonAbelianTopologicalOrder_2022, tantivasadakarn2021long, smith2023deterministic}. 
All these schemes have a $\LAQCC$ implementation.

%$\LAQCC$-circuits also exist for general schemes of preparing local states, based on the local tensors, and gluing them together using one round of entangled measurement and corrections, based on the local symmetry. 
%The main result of \citeauthor{Cirac:2021}, that 1D translational invariant MPS with fixed bond dimension can be prepared by $\mathsf{LOCC}$-assisted circuits, relies heavily on local symmetries of the MPS and as a result also has an equivalent $\LAQCC$ implementation. 
%The corrections applied after the measurement round are local unitaries depending on the local symmetries of the MPS. 

 

%This general scheme of preparing local states, based on the local tensors, and gluing it together by doing one round of entangled measurement and corrections, based on the local symmetry, is accessible in $\LAQCC$.
Note however that \citeauthor{Cirac:2021} also suggest a circuit for the $W$-state.
This circuit uses sequentially and dependent measurement-based corrections of the ancilla qubits. 
These dependent measurements translate to sequential alternations between the quantum and classical circuits and therefore increase the total depth to linear depth, exceeding the constant-depth constraints imposed by $\LAQCC$-circuits. 

We study the power of the $\LAQCC$ model with respect to state preparation, showing that even with only constant quantum-depth and logarithmic classical depth it remains possible to prepare states with long-range entanglement.
Another surprising result is that it is unlikely that $\LAQCC$ circuits are classically simulatable. We show that any instantaneous quantum polynomial-time (IQP) circuit~\cite{Bremner2010,Shepherd2009} has an $\LAQCC$ implementation.
Classical simulation of IQP circuits implies the collapse of the polynomial hierarchy to the third level, which is not believed to be true~\cite{Bremner2017}. Therefore, we expect that $\LAQCC$ circuits are unlikely to be classically simulatable. We bound the power of $\LAQCC$ by showing that it is contained in $\QNC^1$, the class of polynomial-size, log-depth circuits.

Next, we also study the power that intermediate classical calculations can add to quantum computations, by considering a new model that alternates between polynomially many polynomial-depth quantum circuits and unbounded classical computations
We study this model by doing a complexity theoretical analysis, where we draw inspiration from the notions of complexity given by \citeauthor{RosenthalYuen:2022}, \citeauthor{MetgerYuen:2023}, and \citeauthor{Aaronson:2004}.
All three complexity notions are based on the notion of state preparation, instead of more traditional definition of complexity such as the decidability of a computational problem. 
The first two consider classes based on sequences of quantum states preparable by a polynomial-sized quantum circuit, where the circuits are uniformly generated by a computational class, for instance, the class $\mathsf{PSPACE}$, which results in the complexity class $\mathsf{StatePSPACE}$~\cite{RosenthalYuen:2022,MetgerYuen:2023}.
The third notion considers a relative complexity, where the complexity is measured between two given states, and is measured by the number of gates, from a given gate-set, required to transform one state in another state~\cite{Aaronson:2004}. 
For our definition of state preparation complexity, we drop the uniformity constraint from~\cite{RosenthalYuen:2022,MetgerYuen:2023} and define a class as $\mathsf{StateX}$, which refers to states preparable by circuits of type $\mathsf{X}$. 
As an example, if $\mathsf{X} = \QNC^0$, this results in the class $\mathsf{StateQNC^0}$, which is the set of states preparable from the $\ket{0}^n$ state by poly-size constant-depth circuits. 
This notion is similar to the relative complexity from~\cite{Aaronson:2004}, where one state is the  $\ket{0}^n$ state and instead of counting the number of gates we consider the set of states preparable by a fixed number of gates. Using this notion of complexity we show that any state preparable by an $\LAQCC^*$ circuit is also preparable by a $\mathsf{PostQPoly}$ circuit, the class of circuits of polynomial depth with an additional post-selection gate. 

All Clifford circuits have a constant-depth $\LAQCC$ implementation, implying that any stabilizer state can be implemented by a constant-depth $\LAQCC$ circuit, see Section~\ref{sec:clifford_circuits} for a proof of this statement. 
Efficient circuits for stabilizer states have been known already through measurement-based quantum computing. Therefore this paper focuses on the preparation of non-stabilizer states, and as a surprising result we find novel constant-depth protocols for four very natural classes of non-stabilizer states.
Despite the extensive research into these four classes of non-stabilizer states and the many applications of them, no efficient constant- or low-depth state preparation protocols are known yet. We specifically consider these four classes as they are all often used as initial states in other algorithms.

The first state is a uniform superposition over an arbitrary number of states. 
This state finds applications in many quantum algorithms, as they often start with a uniform superposition over multiple states. 
This superposition is often achieved by applying Hadamard gates to every qubit due to its simplicity to prepare. 
Yet, the analysis of many algorithms, such as Shor's algorithm~\cite{Shor:1997}, would benefit from a different initial superposition. 
The circuit to prepare the uniform superposition over an arbitrary number of states uses an exact version of Grover search as a subroutine, that turns a probabilistic circuit, with a known constant probability of success, into a deterministic circuit. 
We use the circuit for preparing a uniform superposition over an arbitrary number of states as a subroutine in the next two quantum state preparation protocols. 

The second state is the $W$-state, the uniform superposition over all computational basis states of Hamming-weight~$1$, a natural long-ranged entangled state that displays a fundamentally nonequivalent type of entanglement from the Greenberger–Horne–Zeilinger state~\cite{WState:2000}, for which $\LAQCC$-type constant-depth circuits were previously known~\cite{PhamSvore2013, Cirac:2021}. 
The $W$-state is often used as benchmark for new quantum hardware~\cite{Haffner2005,Neeley2010,GarciaPerez:2021}. 
A novel way to prepare the $W$-state therefore gives a new way to benchmark different quantum devices with each other. 
A circuit for preparing the $W$-state was given in~\cite{Cirac:2021}, but this implementation requires sequentially alternating measurements followed by local unitaries, which in the $\LAQCC$ model is not considered to be of constant depth. 
We improve this protocol by giving an $\LAQCC$ implementation of the $W$-state, based on a compress-uncompress method that links the one-hot and binary encoding of integers.

The third state considered is the Dicke state, a generalization of the $W$-state, a superposition over all computational basis states with Hamming-weight $k$~\cite{Dicke:1954}. 
Dicke states have relevance in various practical settings.
For instance, for quantum game theory~\cite{zdemir2007}, quantum storage~\cite{Bacon_Compress:2006,Plesch:2010}, quantum error correction~\cite{ouyang2014permutation}, quantum metrology~\cite{toth2012multipartite}, and quantum networking~\cite{prevedel2009experimental}. 
Dicke states have been used as a starting state for variational optimization algorithms, most notably Quantum Alternating Operator Ansatz (QAOA)~\cite{Hadfield2019}, to find solutions to problems such as Maximum k-vertex Cover~\cite{Brandhofer2022,cook2020quantum}.
The ground states of physical Hamiltonians describing one-dimensional chains tend to show a resemblance to Dicke states such as states resulting from the Bethe ansatz, making them an ideal starting state when investigating the ground state behavior of these Hamiltonians~\cite{TDL_BetheAnsatzDerivation:2010,B_ExcitedStateQuantumPhaseTransitions:2013,DickeTransitions:2021}. 
For instance, the algorithm by \citeauthor{van2021preparing}, who give an algorithm to prepare the Bethe ansatz eigenstates of the spin-1/2 XXZ spin chain, starts by first preparing a Dicke state~\cite{van2021preparing}. 
A Dicke-state preparation protocol based on the compress-uncompress methodology used in the $W$-state furthermore finds applications in entanglement distillation, where the entanglement of a large state is concentrated on only a few qubits. 
Efficient deterministic circuits for preparing Dicke states have been proposed by \citeauthor{bartschi2019deterministic}~\cite{bartschi2019deterministic, bartschi2022deterministic_short_depth}. 
They provide a quantum circuit of depth $\mathO(k \log(\frac{n}{k}))$, allowing arbitrary connectivity, to prepare a Dicke state, which they conjecture to be optimal when $k$ is constant. 
In this work, we provide a constant-depth $\LAQCC$ circuit below their conjectured bound already for constant $k$. 
However, this does not directly disprove their conjecture, as we allow for intermediate measurements and classical computations. 
More significantly, we even construct constant-depth $\LAQCC$ circuits for $k = \mathO(\sqrt{n})$ greatly improving their bound.
This construction extends the compress-uncompress method for the $W$-state combined with additional subroutines. 

We continue with a log-depth state preparation protocol for the Dicke-state for arbitrary $k$. 
This protocol implements an efficient transformation between the factoradic number representation and the combinatorial number representation of a positive integer. 
The combinatorial number representation relates directly to the Dicke state. 
The provided efficient transformation between number representation systems might be of independent interest. 

We conclude by modifying our protocol for preparing a Dicke-state to a protocol that prepares quantum many-body scar states in constant-depth. 
These states have low entanglement and longer coherence times than states with similar energy density.
These characteristics make many-body scar states interesting to analyze and relevant within physics.
Many-body scar states appear for instance in the AKLT model~\cite{AKLT:1987,MRBAR:2018,MRB:2018} and different spin models~\cite{SI:2019,MOBFR:2020}.
Known methods for preparing these states have polynomial-depth~\cite{Gustafson:2023}, whereas our circuit has constant depth. 

% We conclude by studying the power that intermediate classical calculations can add to quantum computations. 
% In this study, we define a new model that relaxes constant-depth quantum circuits to polynomial depth quantum circuits, log-depth classical calculations to unbounded classical computations and a constant number of alternations to a polynomial number of alternations. 
% We call this model $\LAQCC^*$. 
% We study this model by doing a complexity theoretical analysis, where we draw inspiration from the notions of complexity given by \citeauthor{RosenthalYuen:2022}, \citeauthor{MetgerYuen:2023}, and \citeauthor{Aaronson:2004}.
% All three complexity notions are based on the notion of state preparation, instead of more traditional definition of complexity such as the decidability of a computational problem. 
% The first two consider classes based on sequences of quantum states preparable by a polynomial-sized quantum circuit, where the circuits are uniformly generated by a computational class, for instance, the class $\mathsf{PSPACE}$, which results in the complexity class $\mathsf{StatePSPACE}$~\cite{RosenthalYuen:2022,MetgerYuen:2023}.
% The third notion considers a relative complexity, where the complexity is measured between two given states, and is measured by the number of gates, from a given gate-set, required to transform one state in another state~\cite{Aaronson:2004}. 
% For our definition of state preparation complexity, we drop the uniformity constraint from~\cite{RosenthalYuen:2022,MetgerYuen:2023} and define a class as $\mathsf{StateX}$, which refers to states preparable by circuits of type $\mathsf{X}$. 
% As an example, if $\mathsf{X} = \QNC^0$, this results in the class $\mathsf{StateQNC^0}$, which is the set of states preparable from the $\ket{0}^n$ state by poly-size constant-depth circuits. 
% This notion is similar to the relative complexity from~\cite{Aaronson:2004}, where one state is the  $\ket{0}^n$ state and instead of counting the number of gates we consider the set of states preparable by a fixed number of gates. Using this notion of complexity we show that any state preparable by an $\LAQCC^*$ circuit is also preparable by a $\mathsf{PostQPoly}$ circuit, the class of circuits of polynomial depth with an additional post-selection gate. 

\paragraph{Summary of results}
\begin{itemize}
    \item We give a new definition of a computational model that captures the power of the four step process: applying a constant number of layers of one- and two-qubit gates; performing a syndrome measurement; perform a fast classical computation determining corrections; apply corrections. We call this model \emph{Local Alternating Quantum Classical Computations}, or $\LAQCC$ for short. In this model we bound the allowed quantum operations, intermediate classical calculations, and number of rounds separately. In Section~\ref{sec:LAQCC_model} we define this model and give a list of operations based on results from literature contained in this computational model. In some of these operations we explicitly use that we allow for multiple, but at most constant, rounds  of corrections.
    \item  We show show that there exist $\LAQCC$ circuits that can not be weakly simulated in Section~\ref{sec:IQP_in_LAQCC}. We further show that for every $\LAQCC$ circuit there exists a $\QNC^1$ circuit simulating it perfectly, in Section~\ref{sec:LAQCC_in_QNC1}.
    \item We introduce a new type computational complexity for preparing states and show that the extension of $\LAQCC$ where we allow a polynomial number of rounds and unbounded classical computation, is contained in $\mathsf{PostQPoly}$, the class of polynomial circuits with post-selection, in Section~\ref{sec:Complexity results}.
    \item We show a protocol to prepare the uniform superposition state of size $q$ in $\LAQCC$ using $\mathO(\ceil{\log_2(q)}^2)$ qubits in Section~\ref{sec:superposition_modulo_q}. 
    \item We show a protocol to prepare the $W_n$ state in $\LAQCC$ using $\mathO(n\log(n))$ qubits in Section~\ref{sec:W_state_in_LAQCC}.
    \item We show two ways of preparing the Dicke-$(n,k)$ state. The first method is in $\LAQCC$, works up to $k = \mathO(\sqrt{n})$, uses $\mathO(n^2\log(n))$ qubits, and is found in Section~\ref{sec:dicke:small_k}. The second method is in $\LAQCC\text{-}\mathsf{LOG}$ (an extension of $\LAQCC$ allowing for logarithmic number of alterations instead of constant), works for any $k$, uses $\mathO(\text{poly}(n))$ qubits, and is found in Section~\ref{sec:Dicke_in_LAQCC_LOG}. 
    \item We extend on our $\LAQCC$ method of generating Dicke-$(n,k)$ states for $k = \mathO(\sqrt{n})$ and show a protocol to generate many-body scar states for a particular Hamiltonian in $\LAQCC$ (Section~\ref{sec:many_body_scar}). 
\end{itemize}
Summarized in a table, we provide the following state generation protocols:
\begin{table}[htb]
\centering
\begin{tabular}{l|l|l|l}
\textbf{State description} & \textbf{Width} & \textbf{Depth} & \textbf{Implementation}\\
\hline 
Uniform superposition mod $q$: $\frac{1}{\sqrt{q}} \sum_{i = 0}^{q-1}\ket{i}$ & $\mathO(\ceil{\log^2 q})$ & $\mathO(1)$ & Section~\ref{sec:superposition_modulo_q}\\

$W$-state: $\frac{1}{\sqrt{n}}\sum_{i = 0}^{n-1}\ket{e_i}$ & $\mathO(n \log n)$ & $\mathO(1)$ & Section~\ref{sec:W_state_in_LAQCC}\\

Dicke-$(n,k)$, $k = \mathO(\sqrt{n})$: $\binom{n}{k}^{-1/2}\sum_{x \in \{0,1\}^n: |x| = k} \ket{x}$ &  $\mathO(n^2\log n)$ & $\mathO(1)$ 
&Section~\ref{sec:dicke:small_k}\\

Dicke-$(n,k)$: $\binom{n}{k}^{-1/2}\sum_{x \in \{0,1\}^n: |x| = k} \ket{x}$ & $\mathO(\text{poly}(n))$ & $\mathO(\log n)$ &Section~\ref{sec:Dicke_in_LAQCC_LOG}\\

QMBS: $\ket{S_k} = \frac{1}{k! \sqrt{\mathcal N(n,k)}}(Q^\dagger)^k \ket{\Omega}$ &  $\mathO(n^2\log n)$ & $\mathO(1)$  &  Section~\ref{sec:many_body_scar}
\end{tabular}
\caption{Summary of state preparation protocols given in this paper.}
\label{tab:sate_prep}
\end{table}
In the entry for the quantum many-body scar state $Q$ denotes the raising operator and $\mathcal N(n,k)=\binom{n-k-1}{k}$. 
Section~\ref{sec:many_body_scar} will provide more details on the variables and the implementation. 

\paragraph{Organization of the paper}
\noindent We first introduce relevant preliminaries in Section~\ref{sec:preliminaries}. 
In Section~\ref{sec:LAQCC_model} we formally define the class of Local Alternating Quantum-Classical Computations ($\LAQCC$). We also show that any Clifford circuit can be implemented in constant depth $\LAQCC$ (a result based on a result from measurement-based quantum computing~\cite{jozsa2006introduction}). 
This result allows us to give many useful multi-qubit gates and routines in Section~\ref{sec:gates_created_in_LAQCC}. 
Beyond that we show that constant depth $\LAQCC$ circuits are contained in $\QNC^1$ and that any $\mathsf{IQP}$ circuit has an $\LAQCC$ implementation.
We conclude this section with an analysis of a more powerful instantiation of $\LAQCC$ and show an inclusion with respect to the class $\mathsf{PostQPoly}$, which is the class of circuits of polynomial depth with one additional post-selection gate. 
In Section~\ref{sec:state_prep_in_LAQCC} we give $\LAQCC$ circuit implementations for preparing the uniform superposition over an arbitrary number of states, the $W$-state and the Dicke state up to $k = \mathO(\sqrt{n})$. We furthermore give a log-depth circuit implementation for preparing the Dicke state for any $k$. We conclude by showing a $\LAQCC$ circuit for generating many body scar states of a particular type of Hamiltonian.


% !TeX root = ../MVFD_arxiv.tex


\section{The framework}\label{sec2}
In this section we  present a formal mathematical setup for the local regularity  for bivariate   stochastic processes  (also called scalar random fields, or simply random fields) and the data observed for such processes.

\subsection{Data}\label{sec:data}
Consider $N$ independent realizations, also called sheets, $X^{(1)},\ldots,X^{(j)},\ldots X^{(N)}$   of a stochastic process $X $ defined on a continuous domain $\cT\in\mathbb R^2$. For simplicity, we here focus on domains $\mathcal T$ in the plane, the extension to higher dimensions would not raise different challenges. For the purpose of describing our methodology, we distinguish three observational scenarios of the $N$ realizations. 
First, the ideal, infeasible situation where the sheets $X^{(j)}$ are \emph{completely observed}, that is without error over the entire domain $\cT$.  
The second case is the one where the $X^{(j)}$ are observed (measured) at some \emph{discrete points} in the domain $\cT$, \emph{without noise}. The domain points can be fixed to be the same for all the $X^{(i)}$'s (common design), or can be randomly drawn for each sheets separately (independent design). Finally, the most realistic scenario is the one where in the second case we admit that the realizations of $X$  are observed at discrete domain points \emph{with noise}. 



 To formally describe the second and third scenarios, let  $M_1, \dotsc,M_N$ be an independent sample of an integer-valued random variable $M$ with expectation $\EE[M]=\Mmu$. 
% which increases with $N$. 
In the independent design case, for each $1\leq j \leq N$, and given  $M_j$, let $\Tnm\in\cT$,  $1\leq m \leq  M_j$, be a random sample of a random vector $\TT\in\cT$. The $\Tnm$'s represent the observation points for the realization $\Xp{j}$. We assume that the realizations of $X$, $M$ and $\TT$ are mutually independent. 
In the common design case, $M\equiv \mathfrak m$ and the $\Tnm$'s are the same for all $j$.  Let $\mathcal T_{obs}^{(j)} $ denote the set of observation times $ \Tnm $, $1\leq m \leq M_j$, on the sheet $\Xp{j}$. With common design,  $\mathcal T_{obs}^{(j)} $ does not depend on $j$, while with independent design the expected cardinal of $\mathcal T_{obs}^{(j)} $ can be random with mean $\mathfrak m$.   The following presentation  includes both independent design and common design cases. \color{black}  Finally, the data  consist of  the pairs  $(\Ynm , \Tnm ) \in\mathbb R \times \cT $ where $\Ynm$ is defined as
\begin{equation}\label{model_eq}
	\Ynm = \Xn (\Tnm) + \enm, \quad\text{with}  \quad  \enm = \sigma(\Tnm,\Xn(\Tnm)) \unm, 
	\quad 1\leq i \leq N,  \; 1\leq m \leq M_j.
\end{equation}
Here, the $\unm  \in\mathbb R $ are independent copies of a centered variable $e$ with unit variance, and $\sigma^2(\cdot,\cdot)\geq 0$ is some unknown, bounded conditional variance  function which account for possibly heteroscedastic measurement errors. The case  $\sigma(t,x)\equiv 0$ corresponds to our second scenario, while in the third scenario we have positive conditional variance. 

For each $1\leq j \leq N$,  let $\widetilde X^{(j)}$ denote an observable approximation of $X^{(j)}$. If   the sheets $X^{(j)}$ were  completely observed, as in our infeasible first scenario, $\widetilde X^{(j)} = X^{(j)}$. 
When $X^{(j)}$ are observed  only at  some discrete points $\Tnm$,  arbitrary  $\widetilde X^{(j)}(\Tt)$ can be obtained by simple interpolation or defined equal to the value of $\widetilde X^{(j)}$ at the nearest neighbor of $\Tt$.  Finally, with noisy, discretely observed sheets,   $\widetilde X^{(j)}$  is a pilot nonparametric estimator  of $X^{(j)}$, such as kernel smoothing, splines \emph{etc}.  


Let us next introduce a general class of stochastic processes (random fields)  $X$ with  irregular realizations $X^{(j)}$, for which the regularity can vary over the domain $\mathcal T$. 


\subsection{A class of multivariate processes}
Let $\mathcal{T}$ be an open, bounded bi-dimensional rectangle with the closure included in $(0,\infty)^2$. In the following, $H_1,H_2 : \mathcal T \to (0,1)$ are two continuously differentiable functions such that 
\begin{equation}\label{low_thres}
\min_{i=1,2} \inf_{t\in\mathcal T} H_i(\Tt) >0.
\end{equation} 	
Let $\overline{H} = \max\{H_1 ,H_2\}$.
We also consider the vector-valued function $\mathbf{L}=(L_1^{(1)},L_2^{(1)},L_1^{(2)},L_2^{(2)}),$  where the components  are  non-negative, Lipschitz  continuous functions defined on $\mathcal{T}$ such that 
\begin{equation}\label{id_L}
	L_i^{(1)}(\Tt) +L_i^{(2)}(\Tt) >0,\qquad \forall \Tt\in\mathcal T, \; i=1,2.
\end{equation}
%We denote $\mathbf{L}=(L_1^{(1)},L_2^{(1)},L_1^{(2)},L_2^{(2)}),$ and assume that a constant $C_{\mathbf L}$ exists such that 
%\begin{equation}\label{up_thres}
%C_{\mathbf{L}}=\max_{i,j=1,2} \sup_{t\in\mathcal T} L_j^{(i)}(\Tt) <\infty.
%\end{equation} 	

Let $X$ be a real-valued, second order stochastic process defined on $(0,\infty)^2$. Let $(e_1,e_2)$ be the canonical basis of $\mathbb R ^2,$ and, for sufficiently small scalars $\Delta$, let   
$$
\theta_{\Tt}^{(i)}(\Delta)=\EE\left[\left\{X\left(\Tt-\frac{\Delta}{2}e_i\right)-X\left(\Tt+\frac{\Delta}{2}e_i\right)\right\}^2\right],\quad i=1,2.
$$ 


\begin{definition}\label{def}
Let $H_1$, $H_2$ satisfy \eqref{low_thres}.	The class $\mathcal {H}^{H_1,H_2}(\mathbf{L},\mathcal{T})$ is the set of stochastic processes $X$ satisfying the following condition:  constants $  \Delta_0, C,\beta>0$ exist such that 
	for any $\Tt\in \mathcal T$ and $ 0<\Delta\leq\Delta_0$,  
	\begin{equation}\label{as_repr}
		\left|\theta_{\Tt}^{(i)}(\Delta)-L_1^{(i)}(\Tt)\Delta^{2H_1(\Tt)} -L_2^{(i)}(\Tt)\Delta^{2H_2(\Tt)}\right|\leq C\Delta^{2 \overline{H}(\Tt)+\beta}, \quad i=1,2.
	\end{equation}
	Let  
	$$
	\mathcal{H}^{H_1,H_2} %=\mathcal{H}^{H_1,H_2}(\mathcal{T})
	=\bigcup_{\mathbf{L}}\mathcal {H}^{H_1,H_2}(\mathbf{L},\mathcal{T}) ,
	$$
	where the union is taken over the set of four-dimensional functions $\mathbf{L}$ with non negative positives Lipschitz  continuous components satisfying \eqref{id_L} 
	%and \eqref{up_thres}. 
	The functions $H_1,H_2$ define the local regularity of the process, while $\mathbf{L}$ represent  the local Hölder constants.
\end{definition}

Definition \ref{def}  is general, and extends the local regularity notion considered by \cite{GKP} for processes defined on a compact interval on the real line.  A main example we have in mind is the  multi-fractional Brownian  sheet (MfBs) with a time-deformation. MfBs  is a generalization of the standard fractional Brownian sheet, where the Hurst parameter is allowed to vary along the  domain. The definition of this general class of  processes and some of their properties are provided in Section \ref{BfMs}.




\section{The FruitChain Protocol}\label{sec:description}

In this section, we devise an adaptation of the FruitChain protocol~\cite{fruitchain} to the EVP framework of~\cite{EVP} outlined in Section~\ref{sec:framework}. In our adaptation, we take into account the cost of a random oracle query as well as the costs of deciding on a new local state, validating retrieved messages, and extracting sequences of records of transactions. Before the presentation of our adaptation, we provide an overview of the original  protocol description.

\subsection{Overview of the FruitChain Protocol}\label{subsec:overview}
In the FruitChain protocol \cite{fruitchain}, miners store transactions in \textit{fruits} instead of blocks. In order for a miner to create a fruit, it needs to perform PoW, as it does to produce blocks, yet fruit mining has lower difficulty. In more detail, the miner performs a 2-for-1 PoW procedure introduced in \cite{backbone}. In particular, the miner computes hashes of a specific input, where the prefix and the suffix of the hash determine whether a block or a fruit has been mined, respectively. Fruits are stored in blocks and they need to be \textit{recent} i.e., every fruit points to a block that is not far from the latest block of the ledger. 

At a high level, the FruitChain protocol prevents selfish mining attacks~\cite{selfish} because even if an attacker withholds a block, the fruits of this block that are still recent can be stored in a later block. The restriction of recency exists so that an attacker is not able to precompute an excessive amount of fruits and reveal it later, thus disrupting the \textit{chain quality} of the protocol\footnote{Chain quality was introduced in \cite{backbone} and is related to the fraction of the adversarial blocks in a sufficiently long segment of the ledger.}.  

\subsection{Parameterization and Basic Concepts}\label{subsec:basic}
%
The FruitChain PoW protocol is parameterized by:
\begin{enumerate}
    \item A \emph{random oracle} $\oracle(\cdot)$ that outputs strings of length $\geq 2\secpar$. The oracle responds to (block and fruit) mining queries.
    \item A \emph{collision resistant hash function} (CRHF) $\digest(\cdot)$, utilized to digest sets of fruits.
    \item A \emph{block mining hardness parameter} $\pb$. This is the probability that the $\oracle(\cdot)$ response leads to the successful mining of a block. 
    \item A \emph{fruit mining hardness parameter} $\pf$. This is the probability that the $\oracle(\cdot)$ response leads to the successful mining of a fruit. Probability $\pf$ is significantly greater {than $\pb$}.
    \item A \emph{recency parameter} $\recency$ that determines how far back can a fruit ``hang'', i.e., the fruit needs to point to an earlier block in the chain which is not too far from the block which records the fruit itself. 
\end{enumerate}

The structure of a valid fruit $\fruit$ is denoted by $\fruit=\langle \previous,h_f,\nonce,$ $\msf{dig},\record,h\rangle$, where 
\begin{itemize}
    \item[-] $\previous$ points to the previous block's reference.
    \item[-] $h_f$ is the \emph{pointer} of $\fruit$ to the block that $\fruit$ is hanging from.
    \item[-] $\record$ is the record to be contained in $\fruit$.
    %\item[-] $\nonce$ is a random nonce denoting the solution to a computational puzzle that is derived from the pair $(\previous,\record)$.
    \item[-] $\nonce$ is a random nonce denoting a solution to the computational puzzle that derives from conditions (1),(2) in Definition~\ref{def:valid_fruit}.
    \item[-] $\msf{dig}$ is the digest of some set of fruits $\mbf{F}$.%, necessary for the verification of the $\fruit$'s validity (see below).
    \item[-] $h$ is the reference of $\fruit$, i.e., a hash of the previous fields.
\end{itemize}
%
\begin{definition}[Fruit validity]\label{def:valid_fruit}
A fruit $\fruit:=\langle \previous,h_f,\nonce,\msf{dig},\record,h\rangle$ is \emph{valid}, if the following hold:
\begin{enumerate}
    \item $\oracle(\previous||h_f||\nonce||\msf{dig}||\record)=h$.
    \item $[h]_{-\secpar}<D_{\pf}$, where $[h]_{-\secpar}$ denotes the last $\secpar$ bits of $h$, and $D_{\pf}$ is the difficulty value such that the probability that an input satisfies the relation is $\pf$.
    
\end{enumerate}
We say that a fruit set $\mbf{F}$ is \emph{valid}, if either it contains only valid fruits, or $\mbf{F}=\emptyset$.
\end{definition}

The structure of a valid block $\block$ is denoted by $\block:=\langle\langle \previous,h_f,\nonce,$ $\msf{dig},\record,h\rangle,\mbf{F}\rangle$, where 
\begin{itemize}
    \item[-] $\previous$ points to the previous block's reference.
    \item[-] $h_f$ is some fruit pointer.
    \item[-] $\record$ is the record to be contained in some fruit.
    \item[-] $\nonce$ is a random nonce denoting a solution to the computational puzzle that derives from condition (3),(4) in Definition~\ref{def:valid_block}.
    \item[-] $\msf{dig}$ is the digest to the set of fruits $\mbf{F}$.
    \item[-] $h$ is the reference of $\block$, i.e., a hash of the previous fields.
    \item[-] $\mbf{F}$ is the fruit set to be included in $\block$.
\end{itemize}
%
\begin{definition}[Block validity]\label{def:valid_block}
A block $\block=\langle\langle \previous,h_f,\nonce,\msf{dig},\record,$ $h\rangle,\mbf{F}\rangle$ is \emph{valid}, if the following hold:
\begin{enumerate}
\item $\msf{dig}=\digest(\mbf{F})$.
\item $F$ is a valid fruit set.
    \item $\oracle(\previous||h_f||\nonce||\msf{dig}||\record)=h$.
    \item $[h]_{:\secpar}<D_{\pb}$, where $[h]_{:\secpar}$ denotes the first $\secpar$ bits of $h$, and $D_{\pb}$ is the difficulty value such that the probability that an input satisfies the relation is less than $\pb$. 
\end{enumerate}
\end{definition}

\begin{remark}
By making a query to $\oracle(\cdot)$, the party cannot know in advance whether the response hash value $h$ will lead to the successful mining of a fruit and/or block, or neither of two. Thus, the fields $\previous$ and $\msf{dig}$ are included in a fruit (resp. $h_f$ and $\record$ are included in a block) only for mining purposes. 
\end{remark}


Let $\chain_i$ be the ledger state in the view of party $\party{i}$. By $\chain_i[j]:=\langle\langle \previous^{i,j},$ $h_f^{i,j},\nonce^{i,j},\msf{dig}^{i,j},$ $\record^{i,j},h^{i,j}\rangle,\mbf{F}^{i,j}\rangle$, we denote the $j$-th block of $\chain_i$ and by $|\chain_i|$ the length of $\chain_i$. 

\begin{definition}[Fruit recency]\label{def:fruit_recency}
A fruit  $\fruit:=\langle \previous,h_f,\nonce,\msf{dig},\record,h\rangle$ is \emph{recent} w.r.t. $\chain_i$ if it points to some of the last $\recency\cdot\kappa$ blocks of $\chain_i$, i.e., there exists some $k>|\chain_i|-\recency\cdot\kappa$ such that $h_f=h^{i,k}$. 
\end{definition}

\begin{definition}[Chain validity]\label{def:valid_chain}
A chain $\chain_i$ is \emph{valid}, if the following hold:
\begin{enumerate}
    \item The chain is rooted at the special ``genesis'' block, i.e., 
    \[\chain_i[0]=\langle\langle0,0,0,0,\bot,\oracle(0,0,0,0,\bot)\rangle,\emptyset\rangle\;.\]
    \item Each block is valid according to Definition~\ref{def:valid_block} and refers to the previous block's reference, i.e., 
    \[\forall j\in[|\chain_i|]: \previous^{i,j}=h^{i,j-1}\;. \] 
    \item For every $j\in[|\chain_i|]$ and every $\fruit:=\langle \previous,h_f,\nonce,\msf{dig},\record,$ $h\rangle\in\mbf{F}^{i,j}$, there exists some $k>j-\recency\cdot\kappa$ such that $h_f=h^{i,k}$.
\end{enumerate}
\end{definition}



%\subsection{Protocol Description}\label{subsec:protocol_description}
For completeness, we describe the FruitChain PoW protocol $\protocol$ as presented in~\cite{fruitchain}, parameterized by $\oracle(\cdot),\digest(\cdot),$ $\pb,\pf,\recency$, in Figure~\ref{fig:fruitchain_description}.

\subsection{The FruitChain Protocol in the EVP Framework}\label{subsec:fruitchain_framework}
%
We abstract the FruitChain protocol $\protocol$ as the protocol $\protocolevp$, that specifies the oracles below:
\begin{enumerate}
    \item The \emph{longest chain oracle} $\mc{O}_\msf{lc}$: receives as input a state $\chain$ and an array of blocks $\tilde{\mathbb{B}}$. It stores $\tilde{\mathbb{B}}$ in its memory.
    Given $\chain$ and all the arrays of blocks that are stored in its memory, it constructs a set $\mbf{A}$ that includes all the chains that can be formed. It checks which of these chains are valid according to Definition \ref{def:valid_chain} and constructs a set $\mbf{A}'\subset \mbf{A}$ with these chains. 
    It finds the longest chain(s) of $\mbf{A}'$ denoted by $\chain^1, \ldots, \chain^l$. If $l> 1$, then it finds $i_0 \in \lbrace 1,\ldots,l\rbrace$ such that the last block in $\chain^{i_0}$, denoted by $\chain^{i_0} [|\chain^{i_0}|]$, appears first in $\tilde{\mathbb{B}}$.
    If $|\chain^{i_0}|>|chain|$, it sets $\chain' \leftarrow \chain^{i_0}$, otherwise $\chain' \leftarrow \chain$. It sets as $h_f$ the reference of $\chain'[\mathsf{max}\{1,$ $|\chain'|-\kappa\}]$, and as $\previous$ the reference of $\chain'[|\chain'|-1]$. %It executes the procedure $\mathtt{Extract\_Fruit}(\chain')$ (without checking again if $\chain'$ is valid) and extracts the sequence of records
    %$\big(\record_{1},\ldots,\record_{\ell_2}\big)$. 
    It extracts the sequence of records
    $\big(\record_{1},\ldots,\record_{\ell_2}\big)$ included in the fruits of $\chain'$ (by executing the procedure $\mathtt{Extract\_Fruit}(\chain')$ in Figure~\ref{fig:fruitchain_description} without checking again if $\chain'$ is valid).
    It outputs $\chain',\previous,h_f$ and $\big(\record_{1},\ldots,\record_{\ell_2}\big)$. The party can make up to $1$ query per round and the single query cost is $C_\msf{lc}$.
    
    
    
   \iffalse Input: $\chain$, diffuse fruits+blocks. Output: $\chain',\previous,h_f$, sequence of records.\fi
    
    %
    \item The \emph{fruit set oracle} $\mc{O}_\msf{fs}$: receives as input a state $\chain'$ and two sets of fruits $\tilde{\mbf{F}},\tilde{\mbf{F'}}$. It finds the subset $\mbf{X}$ of $\tilde{\mbf{F'}}$ that includes all the valid fruits of $\tilde{\mbf{F'}}$ according to Definition~\ref{def:valid_fruit}. It returns the set of fruits $\mbf{F}=\tilde{\mbf{F}} \cup \mbf{X} $, the set $\mbf{F}_\msf{rec}\subseteq\mbf{F}$ of valid fruits that are recent w.r.t. $\chain'$ for recency parameter $r$ (Definition~\ref{def:fruit_recency}) and are not already in $\chain'$, and the digest $\digest(\mbf{F}_\msf{rec})$. 
    The party can make up to $1$ query per round and the single query cost is $C_\msf{fs}$.
    %
    \item The \emph{transaction oracle} $\mc{O}_\msf{tx}$: receives as input a set of transactions $\{\tx_1,\ldots,\tx_{\ell_1}\}$ and a sequence of records $(\record_1,\ldots,\record_{\ell_2})$. It computes a record of transactions $\record$ that includes all the transactions of $\{\tx_1,\ldots,\tx_{\ell_1}\}$ that are valid according to $(\record_1,\ldots,\record_{\ell_2})$, where transaction validity is defined in a protocol-specific manner. It outputs $\record$. The party can make up to $1$ query per round and the single query cost is $C_\msf{tx}$.
    %
    \item The \emph{random oracle} $\mc{O}_\msf{ro}$:  on a general query $x\in\{0,1\}^*$ checks if there is a stored pair $(x,\cdot)$. If there is not such a pair, it randomly samples an image denoted by $H(x)$ from $\{0,1\}^{2\secpar}$, stores $(x,H(x))$ and returns $H(x)$. Otherwise, it returns $(x,H(x))$.
    In our setting, the queries will have the form $\previous||h_f||\nonce||\digest(\mbf{F}_\msf{rec})||\record$ and the response will be a reference~$h$. The party can make up to $q$ queries per round and the single query cost is $C_\msf{ro}$.
    
    
 
\end{enumerate}
% Figure environment removed
\normalsize

% Figure environment removed
\normalsize

Given the description of $\mc{O}_\msf{lc},\mc{O}_\msf{fs},\mc{O}_\msf{tx},\mc{O}_\msf{ro}$, and the terminology in Subsection~\ref{subsec:basic}, the blockchain protocol $\protocolevp$ is presented in Figure~\ref{fig:fruitchain_evp}. At this point, we provide an overview of the  $\protocolevp$ protocol.\\[2pt]
%
\indent\emph{Overview of  $\protocolevp$.}
Each party stores all the fruits that are valid. Note that the validity of each fruit (cf. Definition \ref{def:valid_fruit}) does not depend on which chain constitutes the ledger, unlike the recency of the fruit (cf. Definition \ref{def:fruit_recency}). %something that does not happen with the ``recency'' of the fruit according (cf. Definition \ref{def:fruit_recency}). Recall that a fruit is recent only if it points to a block that is buried under at most $\recency\cdot\kappa$ blocks, where $\recency$ is the recency parameter and $\kappa$ the security parameter.
\par 
During each round, when a party is activated, it receives a set of transactions $\{\tx_1,\ldots,\tx_{\ell_1^T}\}$ as input from the environment, and retrieves all the fruits and blocks diffused in the previous round. Then, it gives as input to the longest chain oracle $\mc{O}_\msf{lc}$ its current chain and the blocks it retrieved, and it receives as output (i) the chain $\chain'_i$ that the party will extend and is the longest valid chain, (ii) the hash of the last block of this chain, (iii) the hash of the block to which the fruits that will be produced in this round will point and (iv) all the records of $\chain'_i$.
\par Next, the party gives as input to the fruit set oracle $\mc{O}_\msf{fs}$ the chain $\chain'_i$, the set of the valid fruits it retains and the fruits it retrieved. It receives as output (i) the updated set of the valid fruits that includes also the fruits that it retrieved and were valid (ii) the valid fruits that are recent w.r.t. $\chain'_i$  and are not already in  $\chain'_i$, and (iii) the digest of the set of these fruits which works as a ``fingerprint''.
\par  Afterwards, the party makes a query to the transaction oracle $\mc{O}_\msf{tx}$ with input the transactions $\{\tx_1,\ldots,\tx_{\ell_1^T}\}$ it received from the environment and the records $(\record_1,\ldots,\record_{\ell_2^T})$ that received from $\mc{O}_\msf{lc}$. It outputs a record that includes the transactions of $\{\tx_1,\ldots,\tx_{\ell_1^T}\}$ that were valid w.r.t. $(\record_1,\ldots,\record_{\ell_2^T})$ (transaction validity is defined in a protocol-specific way) and will be included in the instance that will be used for the queries to the random oracle $\mc{O}_\msf{ro}$.
\par Finally, it makes $q$ queries to the random oracle with input an \emph{instance} that includes: (a) the hash of the last block in $\chain'_i$, (b) the hash of the block to which the new fruits will point, (c) a nonce, (d) the digest of the recent fruits, and (e) the output of the transaction oracle. When it receives an output from the random oracle, it checks (i) if the last $\secpar$ bits are lower than $D_{\pf}$, and (ii) if the first $\secpar$ bits are lower than $D_{\pb}$. If (i) holds, a fruit has been produced, so it sends this fruit to the Diffuse functionality. If (ii) holds, then a block has been produced, so it sends this block to the Diffuse functionality and stops checking (ii) in the queries.\\[2pt] 
%
\indent\emph{Assignment of rewards.}
We consider that every fruit included in a block that is part of the ledger mints  $\reward$ rewards and sends them to the party specified in the `` coinbase '' transaction\footnote{\url{https://en.bitcoin.it/wiki/Coinbase}}  in the fruit's record $\record$. Note that the rewards and costs of querying the protocol's oracles are in the same unit.



\noindent Below, we discuss the differences between the descriptions of $\protocolevp$ and $\protocol$.

\begin{enumerate}
    \item The environment provides transactions, not records. In addition, $\protocolevp$ checks if the transactions provided by the environment are valid according to the party's local chain. We do not specify when a transaction is valid according to the party's local chain because this depends on the format of the transactions the protocol accepts.  
    \item (i) The party makes $q$ queries to the random oracle $\mc{O}_\msf{ro}$ per round (instead of $1$) (ii) each party can produce at most one block per round. These modifications are in line with the execution model of \cite{backbone}.
    \item The party returns just `$\msf{complete}$'  and the number of the round to the environment, instead of a sequence of records. Note that as the party diffuses its blocks, the environment can receive them via its interaction with the adversary. %So, there is no need for the party to execute procedure $\mathtt{Extract\_Fruit}(\chain_i)$.  
    \item The party diffuses only the blocks that it received during the previous round and the blocks and fruits that it has produced during the current round; unlike in $\protocol$, the party does not diffuse its whole chain every time it produces a new block. On the other hand, the longest chain oracle $\mc{O}_\msf{lc}$ stores all the blocks it receives from the beginning of the execution. This approach reflects the realistic setting where the parties' local chains are not communicated over the network during the mining process.
    Instead, only the newly mined blocks are normally diffused, and the miners can reconstruct all possible chains given the received blocks they have recorded throughout the execution \footnote{\url{https://wiki.bitcoinsv.io/index.php/Main_Page}}. Note that when the network is synchronous, which means that at the end of each round all the messages diffused by honest parties are delivered to every other honest party, the approach in $\protocolevp$ ``implies'' the one in $\protocol$. Namely, due to synchronicity,
    in the beginning of round $T$, every honest party $
    \party{i}$ can \emph{recursively} reconstruct the local state $\chain_j$ of another honest party $
    \party{j}$  given its view of $
\chain_j$ in the beginning of the previous round $T-1$ and the blocks of $
    \party{j}$ that $
    \party{i}$ received by the end of $T-1$ (note that all honest parties' states are initialized as $\langle\langle0,0,0,0,\bot,\oracle(0,0,0,0,\bot)\rangle,\emptyset\rangle$, so recursive reconstruction is feasible across honest parties as rounds progress).
    Besides, at any moment during round $T$, $\party{i}$ can reconstruct $\chain_j$ via its view of $\chain_j$ in the beginning of $T$ and the blocks and fruits received from $
    \party{j}$ since the beginning of the round.      
    %
    \item The recipient of the fruit's rewards is the party specified in the ``coinbase'' transaction. In \cite{fruitchain} the fruit's rewards are shared evenly among the miners of the fruits that belong to a preceding part of the ledger. Note that we follow the approach of \cite{EVP} which is equivalent to the approach of \cite{fruitchain} when we do not take into account transaction fees, which means that we assume that each fruit gives the same rewards.
    
\end{enumerate}


































\section{The Single Pool Protocol}\label{sec:single}
In this section, we provide the definition of a pool in a blockchain system and we describe the rules of a single pool in \textsc{FruitChain}, denoted by $\single$, that includes all the parties. In this pool, all the parties ask the random oracle, but only the pool leader asks the longest chain, the fruit set and the transaction oracle, and determines the instance that will be used by all the parties for the queries to the random oracle.  In the next section, we will prove that joining this ``centralised'' pool is an EVP.  
    
\subsection{Definition of a Pool}\label{subsec:definition_pool}
%
Intuitively, a pool of some protocol $\Pi$ comprises a subset of parties in $\Pi$ that collaborate by interacting internally according to some well-specified communication pattern and guidelines. Formally, we provide the following definition.

\begin{definition}[Pool]\label{def:pool}
Let $\Pi$ be a blockchain protocol with parties $\party{1},\ldots,\party{n}$. A \emph{pool of $\Pi$} is a quadruple $\langle\mbf{V},\mbf{E},\mc{F}_\msf{comm},\tilde{\Pi}\rangle$, where
\begin{itemize}
    \item $\mbf{V}\subseteq\{\party{1},\ldots,\party{n}\}$ is a subset of parties in $\Pi$.
    \item $\mbf{E}\subseteq\{(\party{i},\party{j})|\party{i},\party{j}\in\mbf{V}\}$ is a subset of pairs of parties in $\mbf{V}$ that determines the available simplex communication connections among parties in $\mbf{V}$.
    \item $\mc{F}_\msf{comm}$ is a communication functionality that supports the parties' interaction w.r.t. $\mbf{E}$.
    \item $\tilde{\Pi}$ is a protocol executed by parties in $\mbf{V}$ that captures the execution instructions for the parties in $\mbf{V}$. In addition, $\tilde{\Pi}$ allows parties to have access to the Diffuse functionality (cf. Subsection~\ref{subsec:framework_execution}), hence to the messages exchanged during the execution of $\Pi$.%
\end{itemize}
\end{definition}

\subsection{A Single Pool of $\protocolevp$}\label{subsec:single_pool}
%
Given Definition~\ref{def:pool}, we specify a \emph{single pool of $\protocolevp$ with leader $\party{L}$} as the quadruple $\langle\mbf{V},\mbf{E},\mc{F}_\msf{auth}(\mbf{E}),\single\rangle$ where
\begin{itemize}
    \item $\mbf{V}:=\{\party{1},\ldots,\party{n}\}$, i.e., all the parties collaborate. For some $i^*\in[n]$, we have that $\party{L}=\party{{i^*}}$.
    \item $\mbf{E}:=\{(\party{L},\party{i}),(\party{i},\party{L})\}_{i\in[n]\setminus\{i^*\}}$. Namely, the pool leader $\party{L}$ can communicate with every other party and vice versa. Note that the non leader parties do not communicate with each other.
    \item $\mc{F}_\msf{auth}(\mbf{E})$ is the \emph{message authentication functionality} w.r.t. $\mbf{E}$, defined in the spirit of~\cite{Canetti04} as follows:
\begin{itemize}
    \item[$\blacktriangleright$] Upon receiving $(\textsc{Send},\party{j},M)$ from $\party{i}$, if $(\party{i},\party{j})\in\mbf{E}$, then $\mc{F}_\msf{auth}(\mbf{E})$ sends the message $(\textsc{Sent},\party{i},M)$ to $\party{j}$.
\end{itemize}
Similar to~\cite{Canetti04}, $\mc{F}_\msf{auth}(\mbf{E})$ can be implemented via digital signatures and some setup assumption, such as the presence of a certification authority or the out-of-band exchange of verification keys among the parties in the pool. 
   %
   \item $\single$ is executed by $\party{1},\ldots,\party{n}$ and defines each party's deviation from the protocol $\protocolevp$. During the execution of $\single$, $\party{L}$ takes over the cost for setting up an instance to the random oracle in each round. Then, all parties contribute to the fruit and block mining effort w.r.t. this instance. Upon successful mining of a block, $\party{L}$ shares the rewards that correspond to the fruits included in this block according to the guidelines. The protocol $\single$ is formally introduced in the following subsection.
\end{itemize}



\subsection{Protocol Description}\label{subsec:single_description}
First, we describe an additional oracle that $\single$ utilizes and provide its overview. 
\par 
The \emph{light transaction verification oracle}  $\mc{O}_\msf{ltx}$: receives as input a record of transactions $\record$ and a transaction $\tx$. It outputs $1$ if the transaction $\tx$ is included in the record $\record$ and $\tx$ is valid \footnote{The transaction validity is defined in a protocol-specific manner.}, and $0$ otherwise. The party can make up to $1$ query per round and the cost of a single query is $C_\msf{ltx}$\footnote{This oracle reflects a procedure similar to the ``simplified payment verification'' \url{https://wiki.bitcoinsv.io/index.php/Simplified_Payment_Verification}. $C_\msf{ltx}$ is significantly lower compared to  $C_\msf{tx}$. }.\\[2pt]
%
\indent\emph{Overview of $\single$.}
During each round, the pool leader asks $\mc{O}_\msf{lc},\mc{O}_\msf{fs}$, and $\mc{O}_\msf{tx}$, and creates the instance that will be used for the queries to the random oracle. Then, it sends this instance to the pool members. The pool leader and the other members ask the random oracle $q$ queries when they are activated. When a fruit or a block is produced, they send it to the Diffuse functionality (at most one block per round). 
\par Both the pool leader and the other pool members count the cost that the pool leader should incur for the oracles $\mc{O}_\msf{lc},\mc{O}_\msf{fs},\mc{O}_\msf{tx}$. When a block that uses the instance sent by the pool leader has been diffused, the pool leader creates a payment transaction in the next round. The payments are as follows: if the cost that the pool leader incurred for creating the instances since the last block is higher than the block's rewards (which is equal to the number of fruits multiplied by the fruit reward $\rf$), then the pool leader holds all the rewards. If the block's rewards are higher, then the pool leader subtracts the cost and shares the remaining rewards equally among all the members of the pool including itself. Note that the members will check if the payments have been computed correctly via the light transaction verification oracle \footnote{A similar countermeasure has also been used in P2pool
\url{https://bitcoinmagazine.com/technical/p2pool-bitcoin-mining-decentralization}}.
In addition, to prevent block withholding attacks (cf. \cite{ Rosenfeld2011AnalysisOB,7163020}), both the pool leader and the other pool members check if the diffused fruits and blocks use the instance sent by the pool leader; if not, they abandon the pool.
% 
The $\single$ protocol is presented in  Figures~\ref{fig:single_leader} and~\ref{fig:single_other}.
%
%

% Figure environment removed
\normalsize
% Figure environment removed
\normalsize
%
\begin{remark}\label{rem:payment}
As explained in Figure~\ref{fig:single_leader}, the pool leader carries out the payments of the other parties by including them in a special transaction $\tx_T$. Since the exact payment method does not affect our analysis, we do not provide details on the format of $\tx_T$. In practice, each party $\party{i}\neq\party{L}$ could provide $\party{L}$ with a fresh public key $\msf{pk}_i$, and $\party{L}$ would include a payment linked to $\msf{pk}_i$ in $\tx_T$.
\end{remark}
\section{$\single$ as an EVP}\label{sec:equilibrium_proof}

In this section, we provide our main result. Namely, that the protocol $\single$ is an EVP according to Definition~\ref{def:EVP}. In our theorem statement, we quantify over a class of adversaries whose strategy does not result in the mining of blocks that are ``almost'' empty (i.e., they contain only the special payment transaction $\tx_T$). As we shortly explain, restricting to this class is meaningful and does not harm the generality of our result. In particular, we define the following type of adversary.

For some round $T$, let $\{\mbf{V}_1,\ldots,\mbf{V}_{K_T}\}$ be the partition of the party set $\{\party{1},\ldots,\party{n}\}$ such that for every $i\in[K_T]$, the parties in $\mbf{V}_i$ form a pool according to Definition~\ref{def:pool} (trivially, if $\mbf{V}_i$ is a singleton, then the single party in $\mbf{V}_i$ acts on its own). We say that an adversary $\mathcal{A}$ that controls a coalition $\corrupt\subset\{\party{1},\ldots,\party{n}\}$ is \emph{$\mc{O}_\msf{tx}$-respecting} if for every round $T$ and every $\mbf{V}_i$, $i\in[K_T]$, there is at least one party in $\mbf{V}_i$ that asks the transaction oracle $\mc{O}_\msf{tx}$ during $T$. 

We stress that if we lift the above restriction and quantify over all adversaries in our theorem statement, then using similar proof techniques, we can show that a variant of $\single$ where the leader never queries $\mc{O}_\msf{tx}$, ignores its input transactions, and sets the record as the singleton that includes only the special payment transaction (cf. Remark~\ref{rem:variant} for the variant description details), is an EVP. This strategy profile is related to the verifier's dilemma introduced in~\cite{LuuTKS15}, according to which miners are motivated to skip verification of transactions when the cost is significant.  Observe that this variant of $\single$ forms again a single pool with all the parties (which means that is again completely centralized), but violates liveness\footnote{A blockchain protocol satisfies liveness, if every transaction that has been issued and diffused by an honest party  will be included eventually in the ledger with $1-\negl(\secpar)$ probability~\cite{backbone}}. In our main theorem, we do not follow this direction, as for this variant to be functional, it is necessary that there is no external observer that can check the validity of the chain and affect the profit of the parties. This is not true in practice, as external users can easily detect that the blocks are empty, harm the reputation of the system, and thus affect the price of the currency the parties of the pool earn.% (cf. Remark~\ref{rem:variant} for details). 

Our main theorem statement relies on three reasonable assumptions: (i) the expected rewards per random oracle query are higher than the cost of the query and the cost needed to form the instance for the query (recall that the rewards and the costs are in the same unit), (ii) $\pb=\Omega(\frac{1}{nq})$, and (iii) $\pf<\tfrac{1}{2}$. Moreover, the multiplicative approximation factor is zero. Besides, the three dominant terms in the additive approximation factor are justified as follows:
\begin{enumerate}
    \item[(a)] The term $O\big(\rounds(n-1)\big)\pf\reward$ (a small fraction of the adversary's expected total rewards) appears because the EVP notion compares the exact profit of the adversary in the two executions with $1-\negl(\secpar)$ probability.
    \item[(b)] The term $O\big(\log\secpar\sqrt{\rounds}\big)\cost_\msf{lc}$ (the difference between $\umax(\exec{})$ and $\umin(\honexec)$ in the cost of asking the longest chain oracle) is due to the same reason as above.
   \item[(c)] The term $O\big( \rounds(n-1)\big)\cost_\msf{ltx}$ (the difference between $\umax(\exec{})$ and $\umin(\honexec)$ in the cost of asking the light transaction oracle) derives from the fact that in $\umax(\exec{})$, the parties check that they have got paid correctly by the pool leader. Typically, the cost for this check is relatively small.
\end{enumerate}

\begin{theorem}
\label{th:equilibrium}
%Assuming that the adversary cannot participate in the mining process without asking the transaction oracle $\mc{O}_\msf{tx}$ %\footnote{We can enforce this assumption by modifying the transaction and the random oracle as follows: The transaction $\mc{O}_\msf{tx}$ retains a flag that is initialized as negative. The flag becomes positive when there is a round when the pool leader does not ask this oracle. If this happens the transaction oracle sends a message to the random oracle  $\mc{O}_\msf{tx}$ that the flag became positive and the random oracle does not allow the pool leader to ask any queries until the end of the execution.}, following 
Let (i) $\pf\reward>\tfrac{\cost_\msf{lc}+\cost_\msf{fs}+\cost_\msf{tx}}{(1-\frac{\log\secpar}{\sqrt[4]{n}}) \sqrt{n}q}+\cost_\msf{ro}$, (ii) $\pb=\Omega(\frac{1}{nq})$, and (iii) $\pf<\tfrac{1}{2}$. Then, for any $\delta\in \big[\tfrac{\log\secpar}{\sqrt[4]{\rounds n}},1\big)$, the $\single$ protocol is an $(n-1,0,\epsilon')$-EVP according to the utility profit, where
\begin{equation*}
\begin{split}
\epsilon'=&\Big(\big(\tfrac{\log\secpar}{\sqrt{\rounds n}}+\delta\big)\rounds+\log^2\secpar\big(1+\tfrac{1}{\pf}\big) -\big(\tfrac{\log^3\secpar}{\sqrt{\rounds n}}+1+\delta\big)\Big)(n-1)q\pf\reward+\\
&+\tfrac{n-1}{n}\Big((2\log\secpar)\sqrt{\rounds}+1-\tfrac{\log\secpar}{\sqrt{\rounds}}\Big)(1-(1-\pb)^{nq})\cost_\msf{lc}+\\
&+\big(1+\tfrac{\log\secpar}{\sqrt{\rounds}}\big)\rounds(1-(1-\pb)^{nq})(n-1)\cost_\msf{ltx}+\\
&+\tfrac{\log^2\secpar}{n}(\cost_\msf{fs}+\cost_\msf{tx}),
\end{split}    
\end{equation*}
 w.r.t. every   $\mc{O}_\msf{tx}$-respecting adversary $\mc{A}$ and every $\rounds$-admissible environment $\mc{Z}$ that activates the pool leader first in each round.
\end{theorem}
%
%
\begin{proof}
We will assume that the adversary has corrupted a set $\corrupt$ with $n-1$ parties and can deviate from the $\single$ protocol arbitrarily. We will prove that for every $\rounds$-admissible environment $\mc{Z}$ that activates the leader first in each round and for every PPT adversary $\mc{A}$ that controls $\corrupt$, it holds that 
\begin{equation*}\label{EVP}
\umax(\exec{}) \leq \umin(\honexec) + \epsilon \cdot \mid \umin(\honexec) \mid +\epsilon'   
\end{equation*}
with overwhelming probability in the security parameter $\secpar$.
%
\par
Note that we do not quantify over adversaries that control a set $\corrupt'$ with $t'$ parties, where $t'< n-1$. The reason is that for every adversary $\mc{A}'$ that corrupts $\corrupt'$, we can consider an adversary $\mc{A}''$ that corrupts a set $\corrupt''\supset\corrupt'$ with exactly $n-1$ parties and instructs the parties in $\corrupt'$ to deviate exactly like $\mc{A}'$ and the other $n-1-t'$ parties in $\corrupt''\setminus\corrupt'$ to follow the $\single$ protocol. 
%
\par
At this point, we will describe all the possible deviations of the corrupted parties in the set $\corrupt$ as instructed by $
\mc{A}$. The set $\corrupt$ can include either (i) the pool leader and $n-2$ other members of the pool, or (ii) $n-1$ members of the pool and not the pool leader.

 A round $T$ will be called \textit{payment round} if (i) the array of all blocks that were diffused during the previous round $T-1$ was not empty, and (ii) the array of all the blocks and the set of all the fruits diffused during the previous round $T-1$  do not contain a  block $\hat{\block}:=\langle\langle \hat{h}_{-1},\hat{h}_f,\hat{\nonce},$ $\hat{\msf{dig}},\hat{\record},\hat{h}\rangle,\hat{\mbf{F}}\rangle$
  or a fruit $\hat{\mbf{f}}:=\langle \hat{h}_{-1},\hat{h}_f,\hat{\nonce},\hat{\msf{dig}},\hat{\record},\hat{h}\rangle$, respectively, such that $(\hat{\previous},\hat{h}_f,\hat{\msf{dig}},\hat{\record})\neq(inst_1,inst_2,inst_3,inst_4)$.
%
%
\iffalse
\noindent\textbf{1. $\corrupt$ includes $n-1$ members of the pool and not the pool leader:}
\fi

The possible deviations that $\mc{A}$ can perform are any combination of the following strategies.
\begin{enumerate}
    \item[(D1)] $\mc{A}$ instructs a subset of the non leader parties in $\corrupt$ to deviate from step (4) of the $\single$ protocol for one or more rounds, by ignoring the instance received from the pool leader and creating a different instance. Note that this includes the case where the adversary instructs some adversarial parties to abandon the pool. The meaningful possible deviations that we should examine  are the following: 
    \begin{enumerate}
    \item[(i)] the adversarial party ignores the record $\record$ received from the pool leader and updates $inst_4$ with a new record including a transaction that makes the adversarial party as the recipient of the rewards. 
    \item[(ii)] the adversarial party updates $inst_1$ with a hash value of a block different from $h_{-1}$ received from the pool leader. This reflects the scenario where the adversarial party creates a fork.
    \item[(iii)] the adversarial party updates $inst_2$ with the hash of a block that is different from the hash value, $h_f$, of the block received from the pool leader.
    \item[(iv)] the adversarial party updates $inst_3$ with a digest of a fruit set that is different from $\digest(\mbf{F}_\msf{rec})$ received from the pool leader.
    \item[(v)] the adversarial party does not update one or more $inst_i$, $i\in[4]$.
    
    \end{enumerate}
    \item[(D2)] $\mc{A}$ instructs a subset of the non leader parties in $\mathbf{C}$ to ask $\mc{O}_\msf{ltx}$ no queries during one or more payment rounds.
    \item[(D3)] $\mc{A}$ instructs a subset of the parties in $\mathbf{C}$ to ask the oracle $\mc{O}_\msf{ro}$ fewer than $q$ queries during one or more rounds.
    \item[(D4)] $\mc{A}$ instructs a subset of the parties in $\mathbf{C}$ not to send the fruits or the blocks it produces to the Diffuse Functionality for one or more rounds (this is related to block withholding attacks, cf. \cite{ Rosenfeld2011AnalysisOB,7163020}).

    
    \item[(D5)] $\mc{A}$ instructs a subset of the parties in $\mathbf{C}$ to delay arbitrarily to send the fruits or the blocks it produces to the Diffuse Functionality for one or more rounds.
    \item[(D6)] $\mc{A}$ instructs a subset of the parties in $\mathbf{C}$ to abandon the pool and create a new pool that follows different instructions from the $\single$ protocol.
    \item[(D7)] $\mc{A}$ instructs a subset of the parties in $\mathbf{C}$ to remain in the pool even if it receives a block $\hat{\block}:=\langle\langle \hat{h}_{-1},\hat{h}_f,\hat{\nonce},$ $\hat{\msf{dig}},\hat{\record},\hat{h}\rangle,\hat{\mbf{F}}\rangle$ or a fruit $\langle \hat{h}_{-1},\hat{h}_f,\hat{\nonce},\hat{\msf{dig}},\hat{\record},\hat{h}\rangle$ such that $(\hat{\previous},\hat{h}_f,\hat{\msf{dig}},\hat{\record})\neq(inst_1,inst_2,inst_3,inst_4)$.
    \item[(D8)] $\mc{A}$ instructs a subset of the parties in $\mathbf{C}$ to abandon the pool and follow the $\protocolevp$ protocol.
    \item[(D9)] If $\mathbf{C}$ includes the pool leader, $\mc{A}$ instructs the pool leader to ask no query to $\mc{O}_\msf{fs}$ for one or more rounds.
    \item[(D10)] If $\mathbf{C}$ includes the pool leader, $\mc{A}$ instructs the pool leader to ask no query to $\mc{O}_\msf{tx}$ for one or more rounds.
    \item[(D11)] If $\mathbf{C}$ includes the pool leader, $\mc{A}$ instructs the pool leader to ask no query to $\mc{O}_\msf{lc}$ for one or more payment rounds.
    \item[(D12)] If $\mathbf{C}$ includes the pool leader, $\mc{A}$ instructs the pool leader for one or more payment rounds to  create a special transaction $\tx_T$ that pays the party that does not belong to $\corrupt$ a smaller amount than what described in 3(c) of $\single$ for the pool leader (cf. Figure~\ref{fig:single_leader}). 
   
\end{enumerate}
%
Note that deviations D1, D2 apply only to the corrupted non leader parties, deviations D3-D8 apply to all parties in $\corrupt$ and deviations D9-D12 apply only to the corrupted pool leader. 

First, we provide a lower bound that $\umin(\honexec)$ achieves with overwhelming probability.

\begin{claim}\label{claim:H}
If (i) $\pf\reward\geq\frac{\cost_\msf{lc}+\cost_\msf{fs}+\cost_\msf{tx}}{(1-\frac{\log\secpar}{\sqrt{n}}) nq}$ and (ii) $\pb=\Omega(\frac{1}{nq})$, then it holds that
\begin{equation*} 
\begin{split}
&\Pr\big[\umin(\honexec)\geq\big(1-\tfrac{\log\secpar}{\sqrt{\rounds n}}\big)(\rounds-\log^2\secpar)(n-1)q\pf\reward-\\
&-\tfrac{n-1}{n}\big(1+\tfrac{\log\secpar}{\sqrt{\rounds}}\big)\rounds(1-(1-\pb)^{nq})(\cost_\msf{lc}+n\cost_\msf{ltx})-\\
&-\big(\tfrac{n-1}{n}\rounds+\tfrac{\log^2\secpar}{n}\big)(\cost_\msf{fs}+\cost_\msf{tx})-\rounds (n-1)q\cost_\msf{ro}\big]\geq\\
&\geq 1-\negl(\secpar).
\end{split}
\end{equation*}
\end{claim}
%
\textit{Proof of Claim~\ref{claim:H}. }
We say that a block $\block$ is \emph{profitable} if the rewards that derive from the fruits included in $\block$ are higher than the pool leader cost of asking $\cost_\msf{lc},\cost_\msf{fs},\cost_\msf{tx}$ during the mining of $\block$. By the description of $\single$, if $\block$ is profitable, then $\party{L}$ shares the total profit of $\block$ evenly among all $n$ parties of the single pool. Otherwise, $\party{L}$ uses the rewards to cover (part of) the cost of $\block$ while the other parties receive no profit for $\block$.

We show that the probability that a block $\block$ is profitable under the execution $\honexec$ is overwhelming. 
Let $\rho$ be the number of rounds elapsed for mining $\block$. Since the number of the queries the single pool makes per round is $nq$, the number of fruits mined during the mining of $\block$, $Z_0$, follows $\msf{Bin}(\rho nq,\pf)$. By the Chernoff bounds (cf. Appendix~\ref{app:chernoff}),
%
\begin{equation*}
\Pr\big[Z_0<(1-\tfrac{\log\secpar}{\sqrt{n}})\rho nq\pf\big]\leq e^{-\frac{\log^2\secpar}{2n}\rho nq\pf}=\negl(\secpar).    
\end{equation*}
%
Thus, with $1-\negl(\secpar)$ probability, the rewards w.r.t. $\block$ are at least $(1-\frac{\log\secpar}{\sqrt{n}})\rho nq\pf\reward$. Besides, the leader cost for $\block$ is $\cost_\msf{lc}+\rho\cost_\msf{fs}+\rho\cost_\msf{tx}$. 
Since $\pf\reward\geq\frac{\cost_\msf{lc}+\cost_\msf{fs}+\cost_\msf{tx}}{(1-\frac{\log\secpar}{\sqrt{n}}) nq}$ and $\rho\geq 1$, we have that with $1-\negl(\secpar)$ probability, it holds that
%
\[(1-\tfrac{\log\secpar}{\sqrt{n}})\rho nq\pf\reward\geq\rho(\cost_\msf{lc}+\cost_\msf{fs}+\cost_\msf{tx})\geq\cost_\msf{lc}+\rho\cost_\msf{fs}+\rho\cost_\msf{tx},\]
%
i.e., $\Pr[\block\mbox{ is profitable}]\geq1-\negl(\secpar)$.\\[2pt]
%
%-----------------------------------------------------------
%
Next, we define the following random variables. Let $Z$ be the number of mined fruits during $\honexec$ for the first $\rounds-\log^2\secpar$ rounds and $W$ be the number of rounds that at least one block was mined (i.e., the number of calls to $\mc{O}_\msf{lc}$, $\mc{O}_\msf{tlx}$). Since $nq$ mining queries are made per round, $Z\sim\msf{Bin}((\rounds-\log^2\secpar) nq,\pf)$. Besides, the probability that at least one block is produced in some round is $1-(1-\pb)^{nq}$, so $W\sim\msf{Bin}(\rounds,1-(1-\pb)^{nq})$.

Let $t\_reward$ be the total rewards of the single pool and $l\_cost$ be the leader costs in terms of queries to $\mc{O}_\msf{lc},\mc{O}_\msf{fs},\mc{O}_\msf{tx}$ that are shared among the members of the single pool. Let $c\_reward$ be the rewards of the coalition $\corrupt$ and $c\_cost$ be the additional cost that $\corrupt$ incurs besides its share of $l\_cost$.

As shown above, the probability that some block is not profitable is $\negl(\secpar)$. So, by the union bound, the probability that all the blocks are profitable in $\honexec$ is at least $1-\rounds\negl(\secpar)=1-\negl(\secpar)$. As the coalition consists of $n-1$ parties and when all blocks are profitable each party receives an equal profit share, we have that
%
\begin{equation}\label{eq:c_reward}
\Pr\big[c\_reward=\tfrac{n-1}{n}(t\_reward-l\_cost)\big]=1-\negl(\secpar).
\end{equation}
%
Moreover, the probability that no block is produced during the last $\log^2\secpar$ rounds is $(1-\pb)^{(\log^2\secpar)nq}=\negl(\secpar)$, for $\pb=\Omega(\frac{1}{nq})$. Thus, all fruits mined during the first $\rounds-\log^2\secpar$ rounds will be included in the chain with $1-\negl(\secpar)$ probability, so
\begin{equation}\label{eq:t_reward}
\Pr[t\_reward\geq Z\reward]=1-\negl(\secpar).    
\end{equation}
%
For the leader costs that are shared among the pool members, we have that $l\_cost\leq W\cost_\msf{lc}+r_0(\cost_\msf{fs}+\cost_\msf{tx})$, where $r_0$ is the round that the final block was mined in $\honexec$. By the Chernoff bounds, and since $W\sim\msf{Bin}(\rounds,1-(1-\pb)^{nq})$
%
\begin{equation*}
\begin{split}
&\Pr\big[W\geq\big(1+\tfrac{\log\secpar}{\sqrt{\rounds}}\big)\rounds(1-(1-\pb)^{nq})\big]%\leq\\
%&\leq e^{-\frac{\log^2\secpar}{3\rounds}\rounds(1-(1-\pb)^{nq})}
=\negl(\secpar).
\end{split}
\end{equation*}
%
Thus, we have that %with $1-\negl(\secpar)$ probability,
\begin{equation}\label{eq:l_cost}
\begin{split}
\Pr\big[l\_cost&\leq \big(1+\tfrac{\log\secpar}{\sqrt{\rounds}}\big)\rounds(1-(1-\pb)^{nq})\cost_\msf{lc}+r_0(\cost_\msf{fs}+\cost_\msf{tx})\big] =1-\negl(\secpar).
\end{split}
\end{equation}
%
Each party in the coalition makes also queries to $\cost_\msf{ltx}$ and $\cost_\msf{ro}$. In addition, in case the coalition includes $\party{L}$, we take into account the extra cost $(N-r_0)(\cost_\msf{fs}+\cost_\msf{tx})$ for $\party{L}$ in the last $(N-r_0)$ rounds where no block was produced. In any case, $c\_cost\leq W(n-1)\cost_\msf{ltx}+\rounds(n-1)q\cost_\msf{ro}+(N-r_0)(\cost_\msf{fs}+\cost_\msf{tx})$. By the Chernoff bounds,
\begin{equation}\label{eq:c_cost}
\begin{split}
&\Pr\big[c\_cost\leq \big(1+\tfrac{\log\secpar}{\sqrt{\rounds}}\big)\rounds(1-(1-\pb)^{nq})(n-1)\cost_\msf{ltx}+\\
&+\rounds (n-1)q\cost_\msf{ro}+(N-r_0)(\cost_\msf{fs}+\cost_\msf{tx}) \big]=1-\negl(\secpar).    
\end{split}
\end{equation}
%
By Eq.~\eqref{eq:c_reward},~\eqref{eq:t_reward},~\eqref{eq:l_cost},~\eqref{eq:c_cost} and for some lower bound $B$ (to be defined), we get that
%
\begin{equation}\label{eq:H_final}
\begin{split}
&\Pr[\umin(\honexec)\geq B]=\Pr[c\_reward-c\_cost\geq B]\geq \\
%
\geq&\Pr\big[\tfrac{n-1}{n}(t\_reward-l\_cost)-c\_cost\geq B\big]-\negl(\secpar)\geq\\
%
\geq&\Pr\big[\tfrac{n-1}{n}\Big(Z\reward-\big(1+\tfrac{\log\secpar}{\sqrt{\rounds}}\big)\rounds(1-(1-\pb)^{nq})\cost_\msf{lc}-r_0(\cost_\msf{fs}+\cost_\msf{tx})\Big)-\\
&-\big(1+\tfrac{\log\secpar}{\sqrt{\rounds}}\big)\rounds(1-(1-\pb)^{nq})(n-1)\cost_\msf{ltx}-\\
&-\rounds (n-1)q\cost_\msf{ro}-(N-r_0)(\cost_\msf{fs}+\cost_\msf{tx})\geq B\big]-\\
&-\negl(\secpar)\geq\\
%
\geq&\Pr\big[\tfrac{n-1}{n}\Big(Z\reward-\big(1+\tfrac{\log\secpar}{\sqrt{\rounds}}\big)\rounds(1-(1-\pb)^{nq})\cost_\msf{lc}\Big)-\big(\tfrac{n-1}{n}r_0+(\rounds-r_0)\big)(\cost_\msf{fs}+\cost_\msf{tx})-\\
&-\big(1+\tfrac{\log\secpar}{\sqrt{\rounds}}\big)\rounds(1-(1-\pb)^{nq})(n-1)\cost_\msf{ltx}-\rounds (n-1)q\cost_\msf{ro}\geq B\big]-\\
&-\negl(\secpar)=\\
%
\geq&\Pr\big[Z\reward\geq \tfrac{n}{n-1}B+\big(1+\tfrac{\log\secpar}{\sqrt{\rounds}}\big)\rounds(1-(1-\pb)^{nq})(\cost_\msf{lc}+n\cost_\msf{ltx})+\\
& +\tfrac{n}{n-1}\big(\tfrac{n-1}{n}r_0+(\rounds-r_0)\big)(\cost_\msf{fs}+\cost_\msf{tx})+\rounds nq\cost_\msf{ro}\big]-\negl(\secpar).
\end{split}   
\end{equation}
%
As already shown, with $1-\negl(\secpar)$ probability, it holds that $r_0\geq\rounds-\log^2\secpar$. Therefore,
\begin{equation}\label{eq:CfsCtx}
\tfrac{n-1}{n}r_0+(\rounds-r_0)=\rounds-\tfrac{1}{n}r_0\leq\tfrac{n-1}{n}\rounds+\tfrac{\log^2\secpar}{n}.
\end{equation}
%
So, by setting 
%
\begin{equation*}
\begin{split}
&\tfrac{n}{n-1}B+\big(1+\tfrac{\log\secpar}{\sqrt{\rounds}}\big)\rounds(1-(1-\pb)^{nq})(\cost_\msf{lc}+n\cost_\msf{ltx})+\\
&+\tfrac{n}{n-1}\big(\tfrac{n-1}{n}\rounds+\tfrac{\log^2\secpar}{n}\big)(\cost_\msf{fs}+\cost_\msf{tx})+\rounds nq\cost_\msf{ro}=\\
&=\big(1-\tfrac{\log\secpar}{\sqrt{\rounds n}}\big)(\rounds-\log^2\secpar)nq\pf\reward\Leftrightarrow\\
%
\Leftrightarrow& B:=\big(1-\tfrac{\log\secpar}{\sqrt{\rounds n}}\big)(\rounds-\log^2\secpar)(n-1)q\pf\reward-\\
&-\tfrac{n-1}{n}\big(1+\tfrac{\log\secpar}{\sqrt{\rounds}}\big)\rounds(1-(1-\pb)^{nq})(\cost_\msf{lc}+n\cost_\msf{ltx})-\\
&-\big(\tfrac{n-1}{n}\rounds+\tfrac{\log^2\secpar}{n}\big)(\cost_\msf{fs}+\cost_\msf{tx})-\rounds (n-1)q\cost_\msf{ro},
\end{split}    
\end{equation*}
%
and by Eq.~\eqref{eq:H_final},~\eqref{eq:CfsCtx} and the Chernoff bounds, we conclude that
%
\begin{equation*}
\begin{split}
&\Pr[\umin(\honexec)\geq B]\geq\\
%
\geq&\Pr\big[Z\geq\big(1-\tfrac{\log\secpar}{\sqrt{\rounds n}}\big)(\rounds-\log^2\secpar)nq\pf\big]-\negl(\secpar)\geq\\
%
\geq&\big(1-e^{\frac{\log^2\secpar}{\rounds n}(\rounds-\log^2\secpar)nq\pf})-\negl(\secpar)\geq1-\negl(\secpar).
\end{split}    
\end{equation*}
%
$\quad$\hfill$\dashv$


Next, we provide an upper bound for $\umax(\exec{})$ when $\mc{A}$'s strategy derives by combining deviations D2, D3, D8.


\begin{claim}\label{claim:D2_D3_D8}
Let $\mc{A}$ be an adversary whose strategy comprises a combination of deviations D2, D3, and D8.

If (i) $\pf\reward>\mathrm{max}\Big\{\tfrac{\cost_\msf{lc}+\cost_\msf{fs}+\cost_\msf{tx}}{(1-\frac{\log\secpar}{\sqrt[4]{n}}) \sqrt{n}q},3\big(\tfrac{\cost_\msf{lc}+\cost_\msf{fs}+\cost_\msf{tx}}{(n-1)q}+\cost_\msf{ro}\big)\Big\}$, and (ii) $\pf<\tfrac{1}{2}$, then for any $\delta\in \big[\tfrac{\log\secpar}{\sqrt[4]{\rounds n}},1\big)$, it holds that
%
\begin{equation*}
\begin{split}
&\Pr\big[\umax(\exec{})\leq (1+\delta)(\rounds-1)(n-1)q\pf\reward+\log^2\secpar(n-1)q\reward-\\
&-\tfrac{n-1}{n}\big(1-\tfrac{\log\secpar}{\sqrt{\rounds}}\big)(\rounds-1)(1-(1-\pb)^{nq})\cost_\msf{lc}-\\
&-\tfrac{n-1}{n}\rounds(\cost_\msf{fs}+\cost_\msf{tx})-\rounds (n-1)q\cost_\msf{ro}\big]\geq1-\negl(\secpar).
\end{split}
\end{equation*}
%

\end{claim}

\textit{Proof of Claim~\ref{claim:D2_D3_D8}.} Since it comprises a combination of deviations D2, D3, and D8, $\mc{A}$'s strategy can be generally described as follows: Up to some round $r^*$, $\mc{A}$ may instruct the coalition $\corrupt$ to make fewer queries to $\mc{O}_\msf{ltx}$ and $\mc{O}_\msf{ro}$ while remaining members of the single pool. After $r^*$, $\mc{A}$ instructs the coalition to abandon the pool and follow $\protocolevp$ with the difference that the corrupted parties may again make fewer queries to $\mc{O}_\msf{ltx}$ and $\mc{O}_\msf{ro}$.

Let $Q\leq(n-1)q$ be the total number of queries to $\mc{O}_\msf{ro}$ of the coalition per round. Without loss of generality (since we want to upper bound the profit of $\mc{A}$), we assume that the corrupted parties make no queries to $\mc{O}_\msf{ltx}$. We define the following random variables:

 Let $W^-$ be the number of rounds before $r^*$ that at least one block was mined (i.e., the number of calls to $\mc{O}_\msf{lc}$ up to $r^*$). Since the remaining honest party makes $q$ queries to the random oracle per round, the probability that at least one block is produced in some round is $1-(1-\pb)^{q+Q}$, so $W^-\sim\msf{Bin}(r^*-1,1-(1-\pb)^{q+Q})$.  

Let $Z^-$ be the number of fruits mined before $r^*$. Since $q+Q$ queries are made by all parties per round,  $Z^-\sim\msf{Bin}((r^*-1)(q+Q),\pf)$.

Let $t\_reward$ be the total rewards of the single pool up to $r^*$ and $l\_cost$ be the leader costs up to $r^*$ in terms of queries to $\mc{O}_\msf{lc},\mc{O}_\msf{fs},\mc{O}_\msf{tx}$ that are shared among the members of the single pool. Let $c\_reward^-$ be the rewards of $\corrupt$ up to $r^*$ and $c\_cost^-$ be the additional cost up to $r^*$ that $\corrupt$ incurs besides its share of $l\_cost$.

Let $W^+$ be the number of rounds from $r^*$ to $\rounds-1$ that at least one block was mined (i.e., the number of calls to $\mc{O}_\msf{lc}$ after $r^*$). It holds that $W^+\sim\msf{Bin}(\rounds-r^*,1-(1-\pb)^{q+Q})$.

Let $Z^+$ be the number of fruits mined by $\corrupt$ from $r^*$ to $\rounds-1$. Since the parties in $\corrupt$ ask $Q$ queries per round, it holds that $Z^+\sim\msf{Bin}((\rounds-r^*)Q,\pf)$.

Let $c\_reward^+$ be the rewards of $\corrupt$ after $r^*$ and $c\_cost^+$ be the total cost that $\corrupt$ incurs after $r^*$.

Assume that $Q\geq\sqrt{n}q$ (the case where $Q<\sqrt{n}q$ will be studied later). Similarly to Claim~\ref{claim:H}, we can show that if $\pf\reward\geq\frac{\cost_\msf{lc}+\cost_\msf{fs}+\cost_\msf{tx}}{(1-\frac{\log\secpar}{\sqrt[4]{n}}) \sqrt{n}q}$, then with $1-\negl(\secpar)$ probability, all blocks of the execution are profitable. In particular, let $\rho$ be the number of rounds elapsed for mining a block $\block$. Since the number of the queries that all parties make per round is $q+Q$, the number of fruits mined during the mining of $\block$, $Z_0$, follows $\msf{Bin}(\rho (q+Q),\pf)$. By the Chernoff bounds and given that $Q\geq \sqrt{n}q$,
%
\begin{equation*}
\begin{split}
\Pr\big[Z_0<(1-\tfrac{\log\secpar}{\sqrt[4]{n}})\rho (q+Q)\pf\big]&\leq e^{-\frac{\log^2\secpar}{2\sqrt{n}}\rho (q+Q)\pf}\leq e^{-\frac{\log^2\secpar}{2\sqrt{n}}\rho \sqrt{n}q\pf}=\negl(\secpar). 
\end{split}
\end{equation*}
%
So, with $1-\negl(\secpar)$ probability, the rewards w.r.t. $\block$ are at least $(1-\frac{\log\secpar}{\sqrt[4]{n}})\rho (q+Q)\pf\reward$. Besides, the leader cost for $\block$ is $\cost_\msf{lc}+\rho\cost_\msf{fs}+\rho\cost_\msf{tx}$. 
Since $\pf\reward\geq\frac{\cost_\msf{lc}+\cost_\msf{fs}+\cost_\msf{tx}}{(1-\frac{\log\secpar}{\sqrt[4]{n}}) \sqrt{n}q}$ and $\rho\geq 1$, we have that with $1-\negl(\secpar)$ probability, it holds that
%
\begin{equation*}
\begin{split}
&(1-\tfrac{\log\secpar}{\sqrt[4]{n}})\rho (q+Q)\pf\reward>(1-\tfrac{\log\secpar}{\sqrt[4]{n}})\rho \sqrt{n}q\pf\reward\geq\\
%
\geq&\rho(\cost_\msf{lc}+\cost_\msf{fs}+\cost_\msf{tx})\geq\cost_\msf{lc}+\rho\cost_\msf{fs}+\rho\cost_\msf{tx},
\end{split}
\end{equation*}
%
i.e., $\Pr[\block\mbox{ is profitable}]\geq1-\negl(\secpar)$.



Thus, since by definition, $t\_reward\leq Z^-\reward$ and $l\_cost=W^-\cost_\msf{lc}+r^*(\cost_\msf{fs}+\cost_\msf{tx})$, and given that $\corrupt$ has $n-1$ parties, we have that
with $1-\negl(\secpar)$ probability,
%
\begin{equation}\label{eq:c_reward-}
c\_reward^-\leq\tfrac{n-1}{n}\big(Z^-\reward-W^-\cost_\msf{lc}+r^*(\cost_\msf{fs}+\cost_\msf{tx})\big).
\end{equation}
%
Besides, we directly get that 
\begin{equation}\label{eq:c_cost-}
c\_cost^-\geq 0\cost_\msf{ltx}+r^*Q\cost_\msf{ro}=r^*Q\cost_\msf{ro}.
\end{equation}
Upon abandoning the pool, for the coalition $\corrupt$ it holds that
\begin{equation}\label{eq:c_reward+}
c\_reward^+\leq Z^+\reward
\end{equation}
%
\begin{equation}\label{eq:c_cost+}
c\_cost^+\geq W^+\cost_\msf{lc}+(\rounds-r^*)(\cost_\msf{fs}+\cost_\msf{tx})-(\rounds-r^*)Q\cost_\msf{ro}
\end{equation}
The above lower bound for $c\_cost^+$ holds because $\mc{A}$ follows a combination of D2, D3, and D8, so for every round after $r^*$, there is at least one corrupted party that interacts with $\mc{O}_\msf{lc},\mc{O}_\msf{fs},\mc{O}_\msf{tx}$ according to $\protocolevp$ (on behalf of $\corrupt$).

By Eq.~\eqref{eq:c_reward-},~\eqref{eq:c_cost-},~\eqref{eq:c_reward+},~\eqref{eq:c_cost+} and for lower bound $B$ (to be defined), we have that
%
\begin{equation*}
\begin{split}
&\Pr[\umax(\exec{})\geq B]=\\
%
=&\Pr[(c\_reward^- - c\_cost^-)+(c\_reward^+-c\_cost^+)\geq B]\leq\\
%
\leq&\Pr\big[(\tfrac{n-1}{n}Z^-+Z^+)\reward-(\tfrac{n-1}{n}W^-+W^+)\cost_\msf{lc}-\\
&-\big(\tfrac{n-1}{n}r^*(\cost_\msf{fs}+\cost_\msf{tx})+(\rounds-r^*)(\cost_\msf{fs}+\cost_\msf{tx})\big)-\rounds Q\cost_\msf{ro}\geq B\big]+\negl(\secpar).\\
%
\leq&\Pr\big[(\tfrac{n-1}{n}Z^-+Z^+)\reward-\tfrac{n-1}{n}(W^-+W^+)\cost_\msf{lc}-\tfrac{n-1}{n}\rounds(\cost_\msf{fs}+\cost_\msf{tx})-\rounds Q\cost_\msf{ro}\geq B\big]+\negl(\secpar).
\end{split}
\end{equation*}
%
Now observe that the random variable $W^-+W^+$ follows $\msf{Bin}((r^*-1)+(\rounds-r^*),1-(1-\pb)^{q+Q})$, i.e., $W^-+W^+\sim\msf{Bin}(\rounds-1,1-(1-\pb)^{q+Q})$. So, by the Chernoff bounds,
\begin{equation*}
\Pr\big[W^-+W^+\leq\big(1-\tfrac{\log\secpar}{\sqrt{\rounds}}\big)(\rounds-1)(1-(1-\pb)^{q+Q})\big]=\negl(\secpar).
\end{equation*}
%
Hence, we have that
%
\begin{equation}\label{eq:bound_all_cases}
\begin{split}
&\Pr[\umax(\exec{})\geq B]\leq\Pr\big[(\tfrac{n-1}{n}Z^-+Z^+)\reward\geq B+\\
&+\tfrac{n-1}{n}\big(1-\tfrac{\log\secpar}{\sqrt{\rounds}}\big)(\rounds-1)(1-(1-\pb)^{q+Q})\cost_\msf{lc}+\tfrac{n-1}{n}\rounds(\cost_\msf{fs}+\cost_\msf{tx})+\rounds Q\cost_\msf{ro}\big]+\negl(\secpar).
\end{split}
\end{equation}
%
We study the following cases for the value $r^*$:

\textbf{Case 1:} $r^*<\log^2\secpar$. Since $Z^-\leq(r^*-1)(q+Q)$ and by Eq.~\eqref{eq:bound_all_cases},
%
\begin{equation*}
\begin{split}
&\Pr[\umax(\exec{})\geq B]\leq\Pr\big[Z^+\reward\geq B+\\
&+\tfrac{n-1}{n}\big(1-\tfrac{\log\secpar}{\sqrt{\rounds}}\big)(\rounds-1)(1-(1-\pb)^{q+Q})\cost_\msf{lc}+\\
&+\tfrac{n-1}{n}\rounds(\cost_\msf{fs}+\cost_\msf{tx})+\rounds Q\cost_\msf{ro}-\tfrac{n-1}{n}(r^*-1)(q+Q)\reward\big]+\\
&+\negl(\secpar).
\end{split}
\end{equation*}
%
To apply the Chernoff bounds for $Z^+$, we want to set $B$ such that for every $Q$, it holds that
\begin{equation*}
\begin{split}
& B+\tfrac{n-1}{n}\big(1-\tfrac{\log\secpar}{\sqrt{\rounds}}\big)(\rounds-1)(1-(1-\pb)^{q+Q})\cost_\msf{lc}+\\
&+\tfrac{n-1}{n}\rounds(\cost_\msf{fs}+\cost_\msf{tx})+\rounds Q\cost_\msf{ro}-\\
&-\tfrac{n-1}{n}(r^*-1)(q+Q)\reward\geq\big(1+\tfrac{\log\secpar}{\sqrt[4]{\rounds n}}\big)(\rounds-r^*)Q\pf\reward\Leftrightarrow\\
%
\Leftrightarrow&B\geq\Big(\tfrac{n-1}{n}\big(1-\tfrac{\log\secpar}{\sqrt{\rounds}}\big)(\rounds-1)(1-\pb)^q\cost_\msf{lc}\Big)(1-\pb)^Q+\\
+&\Big(\big(1+\tfrac{\log\secpar}{\sqrt[4]{\rounds n}}\big)(\rounds-r^*)\pf\reward+\tfrac{n-1}{n}(r^*-1)\reward-\rounds \cost_\msf{ro}\Big)Q+\\
+&\tfrac{n-1}{n}(r^*-1)q\reward-\tfrac{n-1}{n}\big(1-\tfrac{\log\secpar}{\sqrt{\rounds}}\big)(\rounds-1)\cost_\msf{lc}-\tfrac{n-1}{n}\rounds(\cost_\msf{fs}+\cost_\msf{tx}).
\end{split}
\end{equation*}
%
We observe that the right term of the above inequality can be expressed as function of $Q$ of the form $f(Q)=a\cdot x^Q+b\cdot Q+c$. 

Next, we show that, if $\pf\reward>2\cost_\msf{ro}$,
then $f(Q)$ has a maximum at $(n-1)q$ in the range  $[0,(n-1)q]$, i.e,. when the coalition asks all available queries. In particular, we want to set $B$ as an upper bound of
\begin{equation*}
\begin{split}
&\Big(\tfrac{n-1}{n}\big(1-\tfrac{\log\secpar}{\sqrt{\rounds}}\big)(\rounds-1)(1-\pb)^q\cost_\msf{lc}\Big)(1-\pb)^Q+\\
+&\Big(\big(1+\tfrac{\log\secpar}{\sqrt[4]{\rounds n}}\big)(\rounds-r^*)\pf\reward+\tfrac{n-1}{n}(r^*-1)\reward-\rounds \cost_\msf{ro}\Big)Q+\\
+&\tfrac{n-1}{n}(r^*-1)q\reward-\tfrac{n-1}{n}\big(1-\tfrac{\log\secpar}{\sqrt{\rounds}}\big)(\rounds-1)\cost_\msf{lc}-\tfrac{n-1}{n}\rounds(\cost_\msf{fs}+\cost_\msf{tx}).
\end{split}
\end{equation*}
%
To do so, we study the function $f(Q)=a\cdot x^Q+b\cdot Q+c$, where
%
\begin{align*}
x&=1-\pb\\
    a&=\tfrac{n-1}{n}\big(1-\tfrac{\log\secpar}{\sqrt{\rounds}}\big)(\rounds-1)(1-\pb)^q\cost_\msf{lc}\\
    b&=\big(1+\tfrac{\log\secpar}{\sqrt[4]{\rounds n}}\big)(\rounds-r^*)\pf\reward+\tfrac{n-1}{n}(r^*-1)\reward-\rounds \cost_\msf{ro}\\
    c&=\tfrac{n-1}{n}(r^*-1)q\reward-\tfrac{n-1}{n}\big(1-\tfrac{\log\secpar}{\sqrt{\rounds}}\big)(\rounds-1)\cost_\msf{lc}-\tfrac{n-1}{n}\rounds(\cost_\msf{fs}+\cost_\msf{tx})
\end{align*}
%
If $\pf\reward>2\cost_\msf{ro}>\tfrac{Nn}{(N-1)(n-1)}\cost_\msf{ro}$, then it is easy to see that 
\begin{equation*}
\begin{split}
b&=\big(1+\tfrac{\log\secpar}{\sqrt[4]{\rounds n}}\big)(\rounds-r^*)\pf\reward+\tfrac{n-1}{n}(r^*-1)\reward-\rounds \cost_\msf{ro}>\\
%
&>\tfrac{n-1}{n}(\rounds-r^*)\pf\reward+\tfrac{n-1}{n}(r^*-1)\pf\reward-\rounds \cost_\msf{ro}=\\
&=\tfrac{n-1}{n}(\rounds-1)\pf\reward-\rounds \cost_\msf{ro}>0.
\end{split}    
\end{equation*}
%
In order to find the maximum of $f(Q)$ for $Q\in[0,(n-1)q]$, we compute
%
\begin{equation*}
\begin{split}
&f'(Q)=0\Rightarrow  a\cdot\ln x\cdot x^Q+b=0\Rightarrow Q=\dfrac{\ln\big(\frac{b}{a\cdot\ln (1/x)}\big)}{\ln x}
\end{split}    
\end{equation*}
%
Since $b>0$ and $\ln x<0$, we have that $f'$ is increasing. Thus, $\frac{\ln\big(\frac{b}{a\cdot\ln (1/x)}\big)}{\ln x}$ is a minimum for $f$. In addition, $p_b$ is typically a small value so $x$ is close to $1$. Consequently, we may assume that $\ln (1/x)$ is sufficiently small so that $\frac{b}{a\cdot\ln (1/x)}>1$. The latter implies that $\frac{\ln\big(\frac{b}{a\cdot\ln (1/x)}\big)}{\ln x}<0$, so given that $f'$ is increasing, we get that $f'(Q)>0$ for $Q\in[0,(n-1)q]$. Therefore, the maximum of $f$ in $[0,(n-1)q]$ is $(n-1)q$. 

By the above, we have that
\begin{equation*}
\begin{split}
f(Q)&\leq\Big(\tfrac{n-1}{n}\big(1-\tfrac{\log\secpar}{\sqrt{\rounds}}\big)(\rounds-1)(1-\pb)^q\cost_\msf{lc}\Big)(1-\pb)^{(n-1)q}+\\
&\quad+\Big(\big(1+\tfrac{\log\secpar}{\sqrt[4]{\rounds n}}\big)(\rounds-r^*)\pf\reward+\tfrac{n-1}{n}(r^*-1)\reward-\rounds \cost_\msf{ro}\Big)(n-1)q+\\
&\quad+\tfrac{n-1}{n}(r^*-1)q\reward-\tfrac{n-1}{n}\big(1-\tfrac{\log\secpar}{\sqrt{\rounds}}\big)(\rounds-1)\cost_\msf{lc}-\tfrac{n-1}{n}\rounds(\cost_\msf{fs}+\cost_\msf{tx})=\\
%
&=\big(1+\tfrac{\log\secpar}{\sqrt[4]{\rounds n}}\big)(\rounds-r^*)(n-1)q\pf\reward+(r^*-1)(n-1)q\reward-\\
&\quad-\tfrac{n-1}{n}\big(1-\tfrac{\log\secpar}{\sqrt{\rounds}}\big)(\rounds-1)(1-(1-\pb)^{nq})\cost_\msf{lc}-\tfrac{n-1}{n}\rounds(\cost_\msf{fs}+\cost_\msf{tx})-\rounds (n-1)q\cost_\msf{ro}.
\end{split}
\end{equation*}
%
Moreover, given that $\pf<\frac{1}{2}<\frac{1}{1+\tfrac{\log\secpar}{\sqrt[4]{\rounds n}}}$ and $r^*<\log^2\secpar$, we have that 
%
\begin{equation*}
\begin{split}
&\big(1+\tfrac{\log\secpar}{\sqrt[4]{\rounds n}}\big)(\rounds-r^*)(n-1)q\pf\reward+(r^*-1)(n-1)q\reward=\\
=&\big(1+\tfrac{\log\secpar}{\sqrt[4]{\rounds n}}\big)\rounds(n-1)q\pf\reward+\big(1-\big(1+\tfrac{\log\secpar}{\sqrt[4]{\rounds n}}\big)\pf\big)(n-1)q\reward r^*-(n-1)q\reward<\\
<&\big(1+\tfrac{\log\secpar}{\sqrt[4]{\rounds n}}\big)\rounds(n-1)q\pf\reward+\big(1-\big(1+\tfrac{\log\secpar}{\sqrt[4]{\rounds n}}\big)\pf\big)(n-1)q\reward \log^2\secpar-(n-1)q\reward=\\
=&\big(1+\tfrac{\log\secpar}{\sqrt[4]{\rounds n}}\big)(\rounds-\log^2\secpar)(n-1)q\pf\reward+(\log^2\secpar-1)(n-1)q\reward.
\end{split}
\end{equation*}
Therefore, we set the upper bound for $f(Q)$ as
%
\begin{equation}\label{eq:bound_case_1}
\begin{split}
B&=\big(1+\tfrac{\log\secpar}{\sqrt[4]{\rounds n}}\big)(\rounds-\log^2\secpar)(n-1)q\pf\reward+(\log^2\secpar-1)(n-1)q\reward-\\
&\quad-\tfrac{n-1}{n}\big(1-\tfrac{\log\secpar}{\sqrt{\rounds}}\big)(\rounds-1)(1-(1-\pb)^{nq})\cost_\msf{lc}-\tfrac{n-1}{n}\rounds(\cost_\msf{fs}+\cost_\msf{tx})-\rounds (n-1)q\cost_\msf{ro}.
\end{split}
\end{equation}
%
For this value of $B$ and by the Chernoff bounds, we have that %
\begin{equation*}
\begin{split}
&\Pr[\umax(\exec{})\geq B]\leq\\
%
\leq&\Pr\big[Z^+\geq\big(1+\tfrac{\log\secpar}{\sqrt[4]{\rounds n}}\big)(\rounds-r^*)Q\pf\big]+\negl(\secpar)\leq\\
%
\leq&e^{-\frac{\log^2\secpar}{3\sqrt{\rounds n}}(\rounds-r^*)Q\pf}+\negl(\secpar)\leq\\
%
\leq&e^{-\frac{\log^2\secpar}{3\sqrt{\rounds n}}(\rounds-\log^2\secpar)\sqrt{n}q\pf}+\negl(\secpar)\leq\negl(\secpar).
\end{split}
\end{equation*}

\textbf{Case 2:} $\rounds-r^*<\log^2\secpar$. By the fact that $Z^+<(\rounds-r^*)Q$, by Eq.~\eqref{eq:bound_all_cases}, we get that  
%
\begin{equation*}
\begin{split}
&\Pr[\umax(\exec{})\geq B]\leq\Pr\big[\tfrac{n-1}{n}Z^-\reward\geq B+\\
&+\tfrac{n-1}{n}\big(1-\tfrac{\log\secpar}{\sqrt{\rounds}}\big)(\rounds-1)(1-(1-\pb)^{q+Q})\cost_\msf{lc}+\\
&+\tfrac{n-1}{n}\rounds(\cost_\msf{fs}+\cost_\msf{tx})+\rounds Q\cost_\msf{ro}-(\rounds-r^*)Q\reward\big]+\\
&+\negl(\secpar).
\end{split}
\end{equation*}
%
To apply the Chernoff bounds for $Z^-$, we want to set $B$ such that for every $Q$, it holds that
\begin{equation*}
\begin{split}
& \tfrac{n}{n-1}B+\big(1-\tfrac{\log\secpar}{\sqrt{\rounds}}\big)(\rounds-1)(1-(1-\pb)^{q+Q})\cost_\msf{lc}+\\
&+\rounds(\cost_\msf{fs}+\cost_\msf{tx})+\tfrac{n}{n-1}\rounds Q\cost_\msf{ro}-\\
&-\tfrac{n}{n-1}(\rounds-r^*)Q\reward\geq\big(1+\tfrac{\log\secpar}{\sqrt[4]{\rounds n}}\big)(r^*-1)(q+Q)\pf\reward\Leftrightarrow\\
%
\Leftrightarrow&B\geq\Big(\tfrac{n-1}{n}\big(1-\tfrac{\log\secpar}{\sqrt{\rounds}}\big)(\rounds-1)(1-\pb)^q\cost_\msf{lc}\Big)(1-\pb)^Q+\\
+&\Big(\tfrac{n-1}{n}\big(1+\tfrac{\log\secpar}{\sqrt[4]{\rounds n}}\big)(r^*-1)\pf\reward+(\rounds-r^*)\reward-\rounds \cost_\msf{ro}\Big)Q+\\
+&\tfrac{n-1}{n}\big(1+\tfrac{\log\secpar}{\sqrt[4]{\rounds n}}\big)(r^*-1)q\pf\reward-\tfrac{n-1}{n}\big(1-\tfrac{\log\secpar}{\sqrt{\rounds}}\big)(\rounds-1)\cost_\msf{lc}-\tfrac{n-1}{n}\rounds(\cost_\msf{fs}+\cost_\msf{tx}).
\end{split}
\end{equation*}
%
Namely, we want to set $B$ as an upper bound of
\begin{equation*}
\begin{split}
&\Big(\tfrac{n-1}{n}\big(1-\tfrac{\log\secpar}{\sqrt{\rounds}}\big)(\rounds-1)(1-\pb)^q\cost_\msf{lc}\Big)(1-\pb)^Q+\\
+&\Big(\tfrac{n-1}{n}\big(1+\tfrac{\log\secpar}{\sqrt[4]{\rounds n}}\big)(r^*-1)\pf\reward+(\rounds-r^*)\reward-\rounds \cost_\msf{ro}\Big)Q+\\
+&\tfrac{n-1}{n}\big(1+\tfrac{\log\secpar}{\sqrt[4]{\rounds n}}\big)(r^*-1)q\pf\reward-\tfrac{n-1}{n}\big(1-\tfrac{\log\secpar}{\sqrt{\rounds}}\big)(\rounds-1)\cost_\msf{lc}-\tfrac{n-1}{n}\rounds(\cost_\msf{fs}+\cost_\msf{tx}).
\end{split}
\end{equation*}
%
%
To do so, we study the function $g(Q)=a'\cdot x^Q+b'\cdot Q+c'$, where
%
\begin{align*}
x&=1-\pb\\
    a'&=\tfrac{n-1}{n}\big(1-\tfrac{\log\secpar}{\sqrt{\rounds}}\big)(\rounds-1)(1-\pb)^q\cost_\msf{lc}\\
    b'&=\tfrac{n-1}{n}\big(1+\tfrac{\log\secpar}{\sqrt[4]{\rounds n}}\big)(r^*-1)\pf\reward+(\rounds-r^*)\reward-\rounds \cost_\msf{ro}\\
    c'&=\tfrac{n-1}{n}\big(1+\tfrac{\log\secpar}{\sqrt[4]{\rounds n}}\big)(r^*-1)q\pf\reward-\tfrac{n-1}{n}\big(1-\tfrac{\log\secpar}{\sqrt{\rounds}}\big)(\rounds-1)\cost_\msf{lc}-\tfrac{n-1}{n}\rounds(\cost_\msf{fs}+\cost_\msf{tx})
\end{align*}
%
If $\pf\reward>2\cost_\msf{ro}>\tfrac{Nn}{(N-1)(n-1)}\cost_\msf{ro}$, then it is easy to see that 
\begin{equation*}
\begin{split}
b'&=\tfrac{n-1}{n}\big(1+\tfrac{\log\secpar}{\sqrt[4]{\rounds n}}\big)(r^*-1)\pf\reward+(\rounds-r^*)\reward-\rounds \cost_\msf{ro}>\\
&>\tfrac{n-1}{n})(r^*-1)\pf\reward+(\rounds-r^*)\pf\reward-\rounds \cost_\msf{ro}>\\
&>\tfrac{n-1}{n})(\rounds-1)\pf\reward-\rounds \cost_\msf{ro}>0.
\end{split}    
\end{equation*}
%
In order to find the maximum of $g(Q)$ for $Q\in[0,(n-1)q]$, we compute
%
\begin{equation*}
\begin{split}
&g'(Q)=0\Rightarrow  a'\cdot\ln x\cdot x^Q+b'=0\Rightarrow Q=\dfrac{\ln\big(\frac{b'}{a'\cdot\ln (1/x)}\big)}{\ln x}
\end{split}    
\end{equation*}
%
Just like function $f$ in Case 1, we can conclude that the maximum of $g$ in $[0,(n-1)q]$ is $(n-1)q$. Thus, we have that
%
\begin{equation*}
\begin{split}
g(Q)&\leq\Big(\tfrac{n-1}{n}\big(1-\tfrac{\log\secpar}{\sqrt{\rounds}}\big)(\rounds-1)(1-\pb)^q\cost_\msf{lc}\Big)(1-\pb)^{(n-1)q}+\\
&\quad+\Big(\tfrac{n-1}{n}\big(1+\tfrac{\log\secpar}{\sqrt[4]{\rounds n}}\big)(r^*-1)\pf\reward+(\rounds-r^*)\reward-\rounds \cost_\msf{ro}\Big)(n-1)q+\\
&\quad+\tfrac{n-1}{n}\big(1+\tfrac{\log\secpar}{\sqrt[4]{\rounds n}}\big)(r^*-1)q\pf\reward-\tfrac{n-1}{n}\big(1-\tfrac{\log\secpar}{\sqrt{\rounds}}\big)(\rounds-1)\cost_\msf{lc}-\tfrac{n-1}{n}\rounds(\cost_\msf{fs}+\cost_\msf{tx})=\\
%
&=\big(1+\tfrac{\log\secpar}{\sqrt[4]{\rounds n}}\big)(r^*-1)(n-1)q\pf\reward+(\rounds-r^*)(n-1)q\reward-\\
&\quad-\tfrac{n-1}{n}\big(1-\tfrac{\log\secpar}{\sqrt{\rounds}}\big)(\rounds-1)(1-(1-\pb)^{nq})\cost_\msf{lc}-\tfrac{n-1}{n}\rounds(\cost_\msf{fs}+\cost_\msf{tx})-\rounds (n-1)q\cost_\msf{ro}.
\end{split}
\end{equation*}
%
%
Given that $\pf<\frac{1}{2}<\frac{1}{1+\tfrac{\log\secpar}{\sqrt[4]{\rounds n}}}$ and $\rounds-r^*<\log^2\secpar$, we have that 
%
\begin{equation*}
\begin{split}
&\big(1+\tfrac{\log\secpar}{\sqrt[4]{\rounds n}}\big)(r^*-1)(n-1)q\pf\reward+(\rounds-r^*)(n-1)q\reward=\\
=&\rounds(n-1)q\reward-\big(1-\big(1+\tfrac{\log\secpar}{\sqrt[4]{\rounds n}}\big)\pf\big)(n-1)q\reward r^*-\big(1+\tfrac{\log\secpar}{\sqrt[4]{\rounds n}}\big)(n-1)q\pf\reward<\\
%
<&\rounds(n-1)q\reward-\big(1-\big(1+\tfrac{\log\secpar}{\sqrt[4]{\rounds n}}\big)\pf\big)(n-1)q\reward (\rounds-\log^2\secpar)-\big(1+\tfrac{\log\secpar}{\sqrt[4]{\rounds n}}\big)(n-1)q\pf\reward=\\
=&\big(1+\tfrac{\log\secpar}{\sqrt[4]{\rounds n}}\big)(\rounds-\log^2\secpar-1)(n-1)q\pf\reward+\log^2\secpar(n-1)q\reward.
\end{split}
\end{equation*}
%
Therefore, we set the upper bound for $g(Q)$ as
%
\begin{equation}\label{eq:bound_case_2}
\begin{split}
B&=\big(1+\tfrac{\log\secpar}{\sqrt[4]{\rounds n}}\big)(\rounds-\log^2\secpar-1)(n-1)q\pf\reward+\log^2\secpar(n-1)q\reward-\\
&\quad-\tfrac{n-1}{n}\big(1-\tfrac{\log\secpar}{\sqrt{\rounds}}\big)(\rounds-1)(1-(1-\pb)^{nq})\cost_\msf{lc}-\tfrac{n-1}{n}\rounds(\cost_\msf{fs}+\cost_\msf{tx})-\rounds (n-1)q\cost_\msf{ro}.
\end{split}
\end{equation}
For this value of $B$ and by the Chernoff bounds, we have that %
\begin{equation*}
\begin{split}
&\Pr[\umax(\exec{})\geq B]\leq\\
%
\leq&\Pr\big[Z^-\geq\big(1+\tfrac{\log\secpar}{\sqrt[4]{\rounds n}}\big)(r^*-1)(q+Q)\pf\big]+\negl(\secpar)\leq\negl(\secpar).
\end{split}
\end{equation*}

%Since $Z^+<(\rounds-r^*)Q$ and by Eq.~\eqref{eq:bound_all_cases},
%
\iffalse
\begin{equation*}
\begin{split}
&\Pr[\umax(\exec{})\geq B]\leq\\
\leq&\Pr\big[\tfrac{n-1}{n}Z^-\reward\geq B+\\
+&\tfrac{n-1}{n}\big(1-\tfrac{\log\secpar}{\sqrt{\rounds}}\big)(\rounds-1)(1-(1-\pb)^{q+Q})\cost_\msf{lc}+\\
+&\tfrac{n-1}{n}\rounds(\cost_\msf{fs}+\cost_\msf{tx})+\rounds Q\cost_\msf{ro}-(\rounds-r^*)Q\reward\big]+\negl(\secpar).
\end{split}
\end{equation*}
%
To apply the Chernoff bounds for $Z^-$, we want to set $B$ such that for every $Q$, it holds that
\begin{equation*}
\begin{split}
& \tfrac{n}{n-1}B+\big(1-\tfrac{\log\secpar}{\sqrt{\rounds}}\big)(\rounds-1)(1-(1-\pb)^{q+Q})\cost_\msf{lc}+\\
&+\rounds(\cost_\msf{fs}+\cost_\msf{tx})+\tfrac{n}{n-1}\rounds Q\cost_\msf{ro}-\\
&-\tfrac{n}{n-1}(\rounds-r^*)Q\reward\geq\big(1+\tfrac{\log\secpar}{\sqrt[4]{\rounds n}}\big)(r^*-1)(q+Q)\pf\reward\Leftrightarrow\\
%
\Leftrightarrow&B\geq\Big(\tfrac{n-1}{n}\big(1-\tfrac{\log\secpar}{\sqrt{\rounds}}\big)(\rounds-1)(1-\pb)^q\cost_\msf{lc}\Big)(1-\pb)^Q+\\
+&\Big(\tfrac{n-1}{n}\big(1+\tfrac{\log\secpar}{\sqrt[4]{\rounds n}}\big)(r^*-1)\pf\reward+(\rounds-r^*)\reward-\rounds \cost_\msf{ro}\Big)Q+\\
+&\tfrac{n-1}{n}\big(1+\tfrac{\log\secpar}{\sqrt[4]{\rounds n}}\big)(r^*-1)q\pf\reward-\\
-&\tfrac{n-1}{n}\big(1-\tfrac{\log\secpar}{\sqrt{\rounds}}\big)(\rounds-1)\cost_\msf{lc}-\tfrac{n-1}{n}\rounds(\cost_\msf{fs}+\cost_\msf{tx}).
\end{split}
\end{equation*}
%
%Similarly to Case 1, we finally get that (cf. Appendix~\ref{app:case_2}) 
%
\fi

\textbf{Case 3:} $\log^2\secpar\leq r^*\leq \rounds-\log^2\secpar$. In this case, by the Chernoff bounds, we have that for $\delta\in(0,1)$
%
\begin{align*}
\Pr\big[Z^-\geq\big(1+\delta\big)(r^*-1)(q+Q)\pf\big]%&\leq e^{-\frac{\delta^2}{3}(r^*-1)(q+Q)\pf}=\\
=\negl(\secpar).\\
%
\Pr\big[Z^+\geq\big(1+\delta\big)(\rounds-r^*)Q\pf\big]%&\leq e^{-\frac{\delta^2}{3}(\rounds-r^*)Q\pf}=\\
=\negl(\secpar).
\end{align*}
%
By the above, with $1-\negl(\secpar)$ probability, it holds that 
\begin{equation*}
\begin{split}&\tfrac{n-1}{n}Z^- + Z^+<\\
<&\Big(\tfrac{n-1}{n}(r^*-1)(q+Q)+(\rounds-r^*)Q\Big)(1+\delta)\pf=\\
=&\Big(\big(\tfrac{n-1}{n}(q+Q)-Q\big)r^*-\tfrac{n-1}{n}(q+Q)+\rounds Q\Big)(1+\delta)\pf.
\end{split}
\end{equation*}
%
Since $Q\leq(n-1)q$, it holds that $\tfrac{n-1}{n}(q+Q)-Q\geq0$. So, given that $r^*\leq\rounds$, we have that with $1-\negl(\secpar)$ probability
\begin{equation*}
\begin{split}&\tfrac{n-1}{n}Z^- + Z^+<
%<&\Big(\big(\tfrac{n-1}{n}(q+Q)-Q\big)\rounds-\tfrac{n-1}{n}(q+Q)+\rounds Q\Big)(1+\delta)\pf=\\
\tfrac{n-1}{n}(\rounds-1)(q+Q)(1+\delta)\pf.
\end{split}
\end{equation*}
%
Thus, by Eq.~\eqref{eq:bound_all_cases}, we set $B$ such that for every $Q$, it holds 
\begin{equation*}
\begin{split}
& B+\tfrac{n-1}{n}\big(1-\tfrac{\log\secpar}{\sqrt{\rounds}}\big)(\rounds-1)(1-(1-\pb)^{q+Q})\cost_\msf{lc}+\\
&+\tfrac{n-1}{n}\rounds(\cost_\msf{fs}+\cost_\msf{tx})+\rounds Q\cost_\msf{ro}\geq\\
&\geq(1+\delta)\tfrac{n-1}{n}(\rounds-1)(q+Q)\pf\reward\Leftrightarrow\\
%
\Leftrightarrow&B\geq\Big(\tfrac{n-1}{n}\big(1-\tfrac{\log\secpar}{\sqrt{\rounds}}\big)(\rounds-1)(1-\pb)^q\cost_\msf{lc}\Big)(1-\pb)^Q+\\
+&\Big((1+\delta)\tfrac{n-1}{n}(\rounds-1)\pf\reward-\rounds \cost_\msf{ro}\Big)Q+\\
+&(1+\delta)\tfrac{n-1}{n}(\rounds-1)q\pf\reward -\tfrac{n-1}{n}\big(1-\tfrac{\log\secpar}{\sqrt{\rounds}}\big)(\rounds-1)\cost_\msf{lc}-\tfrac{n-1}{n}\rounds(\cost_\msf{fs}+\cost_\msf{tx}).
\end{split}
\end{equation*}
%
Namely, we want to set $B$ as an upper bound of
%
\begin{equation*}
\begin{split}
&\Big(\tfrac{n-1}{n}\big(1-\tfrac{\log\secpar}{\sqrt{\rounds}}\big)(\rounds-1)(1-\pb)^q\cost_\msf{lc}\Big)(1-\pb)^Q+\\
+&\Big((1+\delta)\tfrac{n-1}{n}(\rounds-1)\pf\reward-\rounds \cost_\msf{ro}\Big)Q+\\
+&(1+\delta)\tfrac{n-1}{n}(\rounds-1)q\pf\reward -\tfrac{n-1}{n}\big(1-\tfrac{\log\secpar}{\sqrt{\rounds}}\big)(\rounds-1)\cost_\msf{lc}-\tfrac{n-1}{n}\rounds(\cost_\msf{fs}+\cost_\msf{tx}).
\end{split}
\end{equation*}
%
%
To do so, we study the function $h(Q)=a''\cdot x^Q+b''\cdot Q+c''$, where
%
\begin{align*}
x&=1-\pb\\
    a''&=\tfrac{n-1}{n}\big(1-\tfrac{\log\secpar}{\sqrt{\rounds}}\big)(\rounds-1)(1-\pb)^q\cost_\msf{lc}\\
    b''&=(1+\delta)\tfrac{n-1}{n}(\rounds-1)\pf\reward-\rounds \cost_\msf{ro}\\
    c''&=(1+\delta)\tfrac{n-1}{n}(\rounds-1)q\pf\reward-\tfrac{n-1}{n}\big(1-\tfrac{\log\secpar}{\sqrt{\rounds}}\big)(\rounds-1)\cost_\msf{lc}-\tfrac{n-1}{n}\rounds(\cost_\msf{fs}+\cost_\msf{tx})
\end{align*}
%
If $\pf\reward>2\cost_\msf{ro}>\tfrac{Nn}{(N-1)(n-1)}\cost_\msf{ro}$, then it is easy to see that 
\begin{equation*}
\begin{split}
b''&=(1+\delta)\tfrac{n-1}{n}(\rounds-1)\pf\reward-\rounds \cost_\msf{ro}>0.
\end{split}    
\end{equation*}
%
In order to find the maximum of $h(Q)$ for $Q\in[0,(n-1)q]$, we compute
%
\begin{equation*}
\begin{split}
&h'(Q)=0\Rightarrow  a''\cdot\ln x\cdot x^Q+b''=0\Rightarrow Q=\dfrac{\ln\big(\frac{b''}{a''\cdot\ln (1/x)}\big)}{\ln x}
\end{split}    
\end{equation*}
%
Just like function $f$ in Case 1, we can conclude that the maximum of $h$ in $[0,(n-1)q]$ is $(n-1)q$. Thus, we have that
%
\begin{equation*}
\begin{split}
h(Q)&\leq\Big(\tfrac{n-1}{n}\big(1-\tfrac{\log\secpar}{\sqrt{\rounds}}\big)(\rounds-1)(1-\pb)^q\cost_\msf{lc}\Big)(1-\pb)^{(n-1)q}+\\
&\quad+\Big((1+\delta)\tfrac{n-1}{n}(\rounds-1)\pf\reward-\rounds \cost_\msf{ro}\Big)(n-1)q+\\
&\quad+(1+\delta)\tfrac{n-1}{n}(\rounds-1)q\pf\reward -\tfrac{n-1}{n}\big(1-\tfrac{\log\secpar}{\sqrt{\rounds}}\big)(\rounds-1)\cost_\msf{lc}-\tfrac{n-1}{n}\rounds(\cost_\msf{fs}+\cost_\msf{tx})=\\
%
&=(1+\delta)(\rounds-1)(n-1)q\pf\reward-\\
&\quad-\tfrac{n-1}{n}\big(1-\tfrac{\log\secpar}{\sqrt{\rounds}}\big)(\rounds-1)(1-(1-\pb)^{nq})\cost_\msf{lc}-\tfrac{n-1}{n}\rounds(\cost_\msf{fs}+\cost_\msf{tx})-\rounds (n-1)q\cost_\msf{ro}.
\end{split}
\end{equation*}
%
%
%
Therefore, we set the upper bound for $h(Q)$ as
%
%
\begin{equation}\label{eq:bound_case_3}
\begin{split}
B&=(1+\delta)(\rounds-1)(n-1)q\pf\reward-\\
&\quad-\tfrac{n-1}{n}\big(1-\tfrac{\log\secpar}{\sqrt{\rounds}}\big)(\rounds-1)(1-(1-\pb)^{nq})\cost_\msf{lc}-\tfrac{n-1}{n}\rounds(\cost_\msf{fs}+\cost_\msf{tx})-\rounds (n-1)q\cost_\msf{ro}.
\end{split}
\end{equation}
%
For this value of $B$, we get $\Pr[\umax(\exec{})\geq B]=\negl(\secpar).$
%\begin{equation*}
%\begin{split}
%&\Pr[\umax(\exec{})\geq B]=\negl(\secpar).
%\end{split}
%\end{equation*}
\smallskip

Given Cases 1,2, and 3, we provide a final bound that dominates all three upper bounds in Eq.~\eqref{eq:bound_case_1},~\eqref{eq:bound_case_2}, and~\eqref{eq:bound_case_3}, respectively. In particular, for any $\delta\in \big[\tfrac{\log\secpar}{\sqrt[4]{\rounds n}},1\big)$, we set
%
\begin{equation}\label{eq:final_bound}
\begin{split}
B&=(1+\delta)(\rounds-1)(n-1)q\pf\reward+\log^2\secpar(n-1)q\reward-\\
&\quad-\tfrac{n-1}{n}\big(1-\tfrac{\log\secpar}{\sqrt{\rounds}}\big)(\rounds-1)(1-(1-\pb)^{nq})\cost_\msf{lc}-\tfrac{n-1}{n}\rounds(\cost_\msf{fs}+\cost_\msf{tx})-\rounds (n-1)q\cost_\msf{ro}.
\end{split}
\end{equation}
%
Clearly, the above bound dominates the ones in Eq.~\eqref{eq:bound_case_1},~\eqref{eq:bound_case_2}, and and~\eqref{eq:bound_case_3}. Thus, for this value of $B$, we conclude that %
\begin{equation*}
\begin{split}
%&\Pr[\umax(\exec{})\geq B]=\negl(\secpar)\Rightarrow\\
\Pr[\umax(\exec{})\leq B]\geq1-\negl(\secpar).
\end{split}
\end{equation*}
%
Recall that the analysis so far was given that $Q\geq \sqrt{n}q$. To complete the proof of the claim, we will show that for $Q<\sqrt{n}q$, the profit of the coalition $\corrupt$ cannot exceed the bound in Eq.~\eqref{eq:final_bound}, except with $\negl(\secpar)$ probability. 

If $Q<\sqrt{n}q$, then all the parties make less than $(\sqrt{n}+1)q$ random oracle queries in total per round. Let $Z$ be the number of fruits mined during the execution and $\tilde{Z}$ a random variable that follows $\msf{Bin}(\rounds(\sqrt{n}+1)q,\pf)$. By the Chernoff bounds,
%
\begin{equation*}
\begin{split}
\Pr\big[Z\geq\big(1+\tfrac{\log\secpar}{\sqrt[4]{\rounds n}}\big)\rounds(\sqrt{n}+1)q\pf\big]%&\leq e^{-\frac{\log^2\secpar}{3\sqrt{\rounds n}}\rounds(\sqrt{n}+1)q\pf}=\\
=\negl(\secpar).  
\end{split}    
\end{equation*}
%
The latter implies that with $1-\negl(\secpar)$ probability the total rewards, that are clearly greater than the profit of the coalition, are no more than $\big(1+\tfrac{\log\secpar}{\sqrt[4]{\rounds n}}\big)\rounds(\sqrt{n}+1)q\pf\reward$. 

We show that if $\pf\reward>3\big(\tfrac{\cost_\msf{lc}+\cost_\msf{fs}+\cost_\msf{tx}}{(n-1)q}+\cost_\msf{ro}\big)$, then it holds that for any $\delta\in \big[\tfrac{\log\secpar}{\sqrt[4]{\rounds n}},1\big)$,
%
\begin{equation}\label{eq:less_nq}
\begin{split}
&\big(1+\tfrac{\log\secpar}{\sqrt[4]{\rounds n}}\big)\rounds(\sqrt{n}+1)q\pf\reward<\\
<&(1+\delta)(\rounds-1)(n-1)q\pf\reward+\log^2\secpar(n-1)q\reward-\\
&-\tfrac{n-1}{n}\big(1-\tfrac{\log\secpar}{\sqrt{\rounds}}\big)(\rounds-1)(1-(1-\pb)^{nq})\cost_\msf{lc}-\tfrac{n-1}{n}\rounds(\cost_\msf{fs}+\cost_\msf{tx})-\rounds (n-1)q\cost_\msf{ro}.
\end{split}
\end{equation}
%
Namely, since $\big(1+\tfrac{\log\secpar}{\sqrt[4]{\rounds n}}\big)\rounds(\sqrt{n}+1)<\tfrac{1}{2}(\rounds-1)(n-1)$ for typical values of $\rounds,n$, it holds that
\[\big(1+\tfrac{\log\secpar}{\sqrt[4]{\rounds n}}\big)\rounds(\sqrt{n}+1)q\pf\reward<\tfrac{1}{2}(1+\delta)(\rounds-1)(n-1)q\pf\reward.\]
%
Besides, if $\pf\reward>3\big(\tfrac{\cost_\msf{lc}+\cost_\msf{fs}+\cost_\msf{tx}}{(n-1)q}+\cost_\msf{ro}\big)$, then
%
\begin{equation*}
\begin{split}
&(1+\delta)(\rounds-1)(n-1)q\pf\reward+\log^2\secpar(n-1)q\reward-\\
&-\tfrac{n-1}{n}\big(1-\tfrac{\log\secpar}{\sqrt{\rounds}}\big)(\rounds-1)(1-(1-\pb)^{nq})\cost_\msf{lc}-\\
&-\tfrac{n-1}{n}\rounds(\cost_\msf{fs}+\cost_\msf{tx})-\rounds (n-1)q\cost_\msf{ro}>\\
>&(\rounds-1)(n-1)q\pf\reward-\rounds\cost_\msf{lc}-\rounds(\cost_\msf{fs}+\cost_\msf{tx})-\rounds (n-1)q\cost_\msf{ro}=\\
%
=&(\rounds-1)(n-1)q\pf\reward-\rounds(n-1)q\big(\tfrac{\cost_\msf{lc}+\cost_\msf{fs}+\cost_\msf{tx}}{(n-1)q}+\cost_\msf{ro}\big)>\\
%
>&(\rounds-1)(n-1)q\pf\reward-\tfrac{1}{3}\rounds(n-1)q\pf\reward>\\
%
>&(\rounds-1)(n-1)q\pf\reward-\tfrac{1}{2}(\rounds-1)(n-1)q\pf\reward=\\
=&\tfrac{1}{2}(\rounds-1)(n-1)q\pf\reward.
\end{split}
\end{equation*}
By the above, we get Eq.~\eqref{eq:less_nq}, which completes the proof of the claim.
\hfill$\dashv$
%
%Next, we show that $\mc{A}$ does not gain in terms of profit when the its strategy is a combination of D1, D4-D7, D9-D12.  
%
\begin{claim}\label{claim:other_deviations}
For every   $\mc{O}_\msf{tx}$-respecting adversary $\mc{A}$ that performs a combination of deviations D1, D4-D7, D9-D12 and every $\rounds$-admissible environment $\mc{Z}$ that activates the pool leader first in each round, it holds $\umax(\exec{}) \leq \umin(\honexec)$ with $1-\negl(\secpar)$ probability.
\end{claim}
\textit{Proof of Claim~\ref{claim:other_deviations}. }
D1 is captured by D6 due to step (2) (cf. Figure~\ref{fig:single_leader}) and step (2a) (cf. Figure~\ref{fig:single_other}) in the $\single$ protocol for the leader and the non leader, respectively. In more detail, if a party in $\corrupt$ does not update $inst_i$, $i\in[4]$ as instructed by $\single$ and sends its inconsistent fruits and/or blocks, then during the next round the honest parties following $\single$ will dissolve the pool. This happens because the honest parties will detect the deviation. Thus, the outcome of D1  can be captured by D6 where at some round $\mc{A}$ instructs a subset of the parties in $\mathbf{C}$ to abandon the pool. 

D12 is captured by D6 for the case where all the corrupted parties abandon the pool and follow $\protocolevp$ protocol. This happens because if the leader does not pay a non leader party, then this party will detect this via steps (2b) and (3), it will leave the pool and it will follow $\protocolevp$.  

D7 is not effective in our setting, because the parties that are not corrupted by $\mc{A}$  follow the $\single$ protocol and thus they will never produce a block $\hat{\block}:=\langle\langle \hat{h}_{-1},\hat{h}_f,\hat{\nonce},$ $\hat{\msf{dig}},\hat{\record},\hat{h}\rangle,\hat{\mbf{F}}\rangle$ or a fruit $\langle \hat{h}_{-1},\hat{h}_f,\hat{\nonce},\hat{\msf{dig}},\hat{\record},\hat{h}\rangle$ so that $(\hat{\previous},\hat{h}_f,\hat{\msf{dig}},\hat{\record})\neq(inst_1,inst_2,inst_3,inst_4)$. 

D10 is not performed by an $\mc{O}_\tx$-respecting adversary. 

D6 is captured by any combination of D2-D5 and D7-D12:  let us assume that a subset of the corrupted parties abandons the pool and creates a new pool following different instructions from the $\single$ pool. Recall that the utility of the adversary is the sum of the utilities of all the corrupted parties. Thus, the way of sharing the rewards among the corrupted parties does not affect the utility of the adversary.   

D9, D11 have the same effect as D3 for the case where $\mc{A}$ instructs all the corrupted parties to abstain by asking no queries to the random oracle $\mc{O}_\msf{ro}$. The reason is that if the pool leader does not ask the oracles $\mc{O}_\msf{fs}$, $\mc{O}_\msf{lc}$, it cannot create $inst_i$, $i\in[4]$ needed for all the parties to ask the random oracle and produce valid fruits that will give the rewards to the pool when a block is produced.  

D4 will offer to the adversary lower utility than D3 for the case where $\mc{A}$ instructs all the corrupted parties to abstain by asking no queries the random oracle $\mc{O}_\msf{ro}$. The reason that is if the adversary asks the random oracle but does not send its fruits or blocks, it incurs the cost of $C_\msf{ro}$ without getting any more rewards from the fruits it produces. 

Regarding D5, we do not consider deviations that either hinge on the assumption that the blocks can include unlimited number of fruits and/or they demand that the adversary is aware of the round when $\mc{Z}$ will terminate the execution.   
%
%
%
\hfill$\dashv$
\\








Since $\pf\reward>\tfrac{\cost_\msf{lc}+\cost_\msf{fs}+\cost_\msf{tx}}{(1-\frac{\log\secpar}{\sqrt[4]{n}}) \sqrt{n}q}+\cost_\msf{ro}$, $\pb=\Omega(\frac{1}{nq})$, and $\pf<\tfrac{1}{2}$, for an $\mc{O}_\msf{tx}$-respecting adversary $\mc{A}$, the conditions for all Claims~\ref{claim:H},~\ref{claim:D2_D3_D8}, and~\ref{claim:other_deviations} hold. Therefore, for any  $\mc{O}_\msf{tx}$-respecting adversary $\mc{A}$ and $\delta\in \big[\tfrac{\log\secpar}{\sqrt[4]{\rounds n}},1\big)$, with $1-\negl(\secpar)$ probability, it holds that
%
\begin{equation*}
\begin{split}
&\umax(\exec{})-\umin(\honexec)\leq\\
%\leq&\Big( (1+\delta)(\rounds-1)(n-1)q\pf\reward+\\
%&+\log^2\secpar(n-1)q\reward-\\
%&-\tfrac{n-1}{n}\big(1-\tfrac{\log\secpar}{\sqrt{\rounds}}\big)(\rounds-1)(1-(1-\pb)^{nq})\cost_\msf{lc}-\\
%&-\tfrac{n-1}{n}\rounds(\cost_\msf{fs}+\cost_\msf{tx})-\rounds (n-1)q\cost_\msf{ro}\Big)-
%\end{split}    
%\end{equation*}
%
%\begin{equation*}
%\begin{split}
%-&\Big(\big(1-\tfrac{\log\secpar}{\sqrt{\rounds n}}\big)(\rounds-\log^2\secpar)(n-1)q\pf\reward-\\
%&-\tfrac{n-1}{n}\big(1+\tfrac{\log\secpar}{\sqrt{\rounds}}\big)\rounds(1-(1-\pb)^{nq})(\cost_\msf{lc}+n\cost_\msf{ltx})-\\
%&-\big(\tfrac{n-1}{n}\rounds+\tfrac{\log^2\secpar}{n}\big)(\cost_\msf{fs}+\cost_\msf{tx})-\rounds (n-1)q\cost_\msf{ro}\Big)=\\
%
\leq&\Big(\big(\tfrac{\log\secpar}{\sqrt{\rounds n}}+\delta\big)\rounds+\log^2\secpar\big(1+\tfrac{1}{\pf}\big)-\big(\tfrac{\log^3\secpar}{\sqrt{\rounds n}}+1+\delta\big)\Big)(n-1)q\pf\reward+\\
&+\tfrac{n-1}{n}\Big((2\log\secpar)\sqrt{\rounds}+1-\tfrac{\log\secpar}{\sqrt{\rounds}}\Big)(1-(1-\pb)^{nq})\cost_\msf{lc}+\\
&+\big(1+\tfrac{\log\secpar}{\sqrt{\rounds}}\big)\rounds(1-(1-\pb)^{nq})(n-1)\cost_\msf{ltx}+\tfrac{\log^2\secpar}{n}(\cost_\msf{fs}+\cost_\msf{tx}).
\end{split}    
\end{equation*}

Thus, according to Definition~\ref{def:EVP}, the $\single$ protocol is $(n-1,0,\epsilon')$-EVP, for $\epsilon'$ as in theorem statement.

\end{proof}


\begin{remark}\label{rem:variant}
If we remove the assumption of Theorem \ref{th:equilibrium}, then we can prove that instead of the $\single$ protocol, the following strategy profile, denoted by $\mathcal{S}$, is EVP according to the utility profit: all the parties follow all the instructions of the $\single$ protocol except that:
\begin{enumerate} 
\item  the pool leader ignores step (7),(9) for all the rounds. \item in step (8), if the round is a \textit{payment round}, the pool leader sets $inst_4\leftarrow \tx_T$, where $\tx_T$ is the special transaction with the payments, otherwise it does nothing.
\item the members in step (3) do not add $C_\msf{tx}$ in cost.
\end{enumerate}
\end{remark}

\section{Discussion}
\label{sec: discussion}
\kmsdelete{In this work} We study \kmsreplace{Fairness-Aware PAC learning}{Fair-ERM} in the malicious noise model, and  in some cases allow 
the learner to maintain optimal overall accuracy despite the signal in Group $B$ being almost entirely washed out.
%when we allow learners to use the
%$\PQ$ randomized expansion of the hypothesis class $\mathcal{H}$
In particular we show that different fairness constraints have fundamentally different behavior in the presence of Malicious Noise, in terms of the amount of accuracy loss that a given level of Malicious Noise could cause a fairness-constrained learner to incur. 
The key to achieving our results, which are more optimistic than those in \cite{lampert}, is allowing for improper learners using the (P,Q)-randomized expansions of the given class $\mathcal{H}$.
%We \kmsreplace{present a picture of the}{prove upper and lower bounds on}
%accuracy loss for a range of fairness notions, given \kmsreplace{this simple randomization step.}{learning over $\PQ$.
%In general our results indicate Fair-ERM (given learning over $\PQ$) is more robust than claimed in \cite{lampert}.
The type of smoothness we create by using $\PQ$ seems to be a natural property that is likely shared by many natural hypothesis classes.

Fairness notions are motivated as a response to learned disparities when there is \kmsdelete{data corruption or} systemic error affecting \kmsdelete{the data for}
one group. 
Fairness notions are supposed to mitigate this by ruling out classifiers that have worse performance on a sub-group. 
This can peg both classifiers at a lower level of performance \kmsdelete{(e.g that the lower subgroup)} in order to \emph{motivate} \cite{hardt16} improving the data collection or labelling process to obtain more reliable performance. 
%So in \kmsreplace{some}{a} sense, sensitivity of the fairness notion to poor sub-group performance caused by malicious noise is the \textit{point} of fairness constraints! 
However, it also desirable that fairness constraints perform gracefully when subject to Malicious Noise because fairness constraints will be used in contexts where the data is unreliable and noisy and this might not be known to the learner.
This tension, exposed by our work, motivates 
%a revisiting of fairness notions from first principles approach and trying to axiomatize the 
%desired properties of a fairness intervention a la cryptography and privacy. \footnote{Work in multi-calibration \cite{multicalib} is a viable direction for this problem but it is unclear how 
%that and related notions behave with unreliable data. }
on going work studying the sensitivity level of fairness constraints. 
%If we we are to take a view, if a classifier is deployed 


\bibliographystyle{plain}
\bibliography{references}
%
%
\appendix

\begin{comment}
\section{System Architecture}
\label{appendix:architecture}
\system has a novel modularized system architecture with three key components: 
\emph{StreamManager}, 
\emph{TxnManager} and \emph{TxnScheduler}. 
These components are instantiated in each thread locally.
The execution outline of \system is presented in Algorithm~\ref{alg:algo}.
Transactional stream processing is continuous and potentially never ends (Line 1$\sim$8).
The dependency resolution and execution of state transactions are separated into two non-overlapping phases by punctuations~\cite{Tucker:2003:EPS:776752.776780} (Line 2 and 5), which guarantees that no subsequent input event will have a smaller timestamp. 
Effectively, a batch of state transactions is collected during the first phase, and processed during the second phase.

In the first phase (i.e., stream processing phase), 
the \emph{StreamManager} conducts preprocessing for every input event ($e$). Similar to some prior works~\cite{tstream}, state transactions may be issued but not immediately processed during preprocessing (Line 3).
The \emph{pre\_processing} and \emph{post\_processing} functions are exposed as APIs to users.
The \emph{TxnManager} handles dependency resolution (Line 4) among state transactions and insert decomposed operations to construct a \tpg. We discuss the detailed two-phase \tpg construction process in Section~\ref{subsec:construction}.

In the second phase  (i.e., transaction processing phase), 
the \emph{TxnManager} is first involved again to refine (Line 6) the constructed \tpg with further dependency resolution.
The \emph{TxnScheduler} 
schedules operations for concurrent execution based on the constructed \tpg according to the three dimensions of scheduling decisions (Line 7). 
In particular, a scheduling decision model $M$ is instantiated based on the constructed \tpg (Line 14).
\textbf{\circled{1}} Guided by $M$, execution threads adopt an exploration strategy (Section~\ref{subsec:explore}) to explore the constructed \tpg for operations available to be scheduled constrained by dependencies. 
\textbf{\circled{2}} 
During exploration, one or multiple operations may be treated as the 
% basic 
unit of scheduling (Section~\ref{subsec:granularity}). 
Subsequently, \textbf{\circled{3}} every thread executes operation(s) in the unit of scheduling with various abort handling mechanisms (Section~\ref{subsec:abort_handling}).
Only when state transactions are processed (i.e., committed or aborted) can the associated input events be postprocessed (Line 8) by the \emph{StreamManager} based on transaction processing results.
\end{comment}

\begin{comment}
\begin{algorithm}
\footnotesize
    \KwData{$e$ \tcp{Input event}}
    \KwData{$txn_{ts}$ \tcp{State transaction}}
    \KwData{$G$ \tcp{The currently constructed TPG}}
    \While{!finish processing of input streams}{
        \eIf(\tcp*[h]{Phase 1}){\text{$e$ is not a $punctuation$}}{
                $txn_{ts}$ $\gets$ PRE\_Processing($e$)\;
                \textbf{TPG\_Construction}($G$, $txn_{ts}$)\; 
          }(\tcp*[h]{Phase 2}){
                \textbf{TPG\_Refinement}($G$)\; 
                \textbf{TXN\_Scheduling}($G$)\; 
                POST\_Processing()\;
          }
    }
    
    \SetKwFunction{FMain}{TPG\_Construction}
    \SetKwProg{Fn}{Function}{:}{}
    \Fn{\FMain{$G$, $txn_{ts}$}}{
        $O_{1..k}$ $\gets$ \textbf{Partition} $txn_{ts}$\;
        \ForEach{\text{operation $O_{i}$ $\in$ $O_{1..k}$}}{
            \textbf{Identify} its \ld\;
            $G$ $\gets$ $G$ + $O_{i}$ \;
        }
    }
    \SetKwFunction{FMain}{TPG\_Refinement}
    \SetKwProg{Fn}{Function}{:}{}
    \Fn{\FMain{$G$}}{
        \ForEach{\text{vertex $e_{i}$ $\in$ $G$}}{
            \textbf{Identify} its \td, \pd\;
        }
    }
    
    \SetKwFunction{FMain}{TXN\_Scheduling}
    \SetKwProg{Fn}{Function}{:}{}
    \Fn{\FMain{$G$}}{
        $M$ $\gets$ Instantiated with $G$;\tcp{A decision model}
        \While{!finish scheduling of $G$
        }{
          \textbf{\circled{2}} $Scheduling Unit$ $\gets$ \textbf{\circled{1}} \emph{Explore}($G$, $M$)\; 
            \textbf{\circled{3}} \emph{Execute with Abort Handling} ($Scheduling Unit$)\; 
        }
    }
  \caption{Execution Outline of \system}
  \label{alg:algo}
\end{algorithm}
\end{comment}
\end{document}
