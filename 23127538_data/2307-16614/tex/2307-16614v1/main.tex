\documentclass[lettersize,journal]{IEEEtran}
\usepackage{amsmath,amsfonts}
\usepackage{algorithm}
\usepackage{array}
\usepackage[caption=false,font=normalsize,labelfont=sf,textfont=sf]{subfig}
\usepackage{textcomp}
\usepackage{stfloats}
\usepackage{url}
\usepackage{verbatim}
\usepackage{graphicx}
\usepackage{cite}
\hyphenation{op-tical net-works semi-conduc-tor IEEE-Xplore}
% updated with editorial comments 8/9/2021
% My package
\usepackage{siunitx}
\usepackage{xcolor}
\usepackage{longtable}
\usepackage{dsfont}
\usepackage{subfig} %[position=top]
\usepackage{multirow}
\usepackage{textcomp}
\usepackage{tabularx}
\usepackage{enumerate}
\usepackage{nameref}
\usepackage{graphicx}
% \interfootnotelinepenalty=10000
\usepackage{wrapfig}
\usepackage{makecell}
\usepackage{algpseudocode}
\usepackage{mathtools}
\DeclarePairedDelimiter\abs{\lvert}{\rvert}
\usepackage{comment}
% \newcommand*{\myfont}{\fontfamily{OldStandard}\selectfont}
\newcommand{\MHCT}{{\fontfamily{OldStandard}\selectfont Multi-Head\,Co-Training}}
\newcommand{\MHCTPSC}{{\fontfamily{OldStandard}\selectfont Multi-Head\,Co-Training+PSC}}
\usepackage{booktabs}
\makeatletter
\def\RS#1{\zap@space#1 \@empty}
\usepackage{bbm}
\usepackage[T1]{fontenc}


\begin{document}

\title{LaplaceConfidence: a Graph-based Approach for Learning with Noisy Labels}

\author{Mingcai Chen $^1$, Yuntao Du $^1$, Wei Tang $^2$, Baoming Zhang $^1$, Hao Cheng $^1$, Shuwei Qian $^1$, Chongjun Wang $^{1*}$\thanks{$^*$Corresponding author}\\
$^1$ State Key Laboratory for Novel Software Technology at Nanjing University, Nanjing University\\
$^2$ Department of Neurology, University Medical Center Groningen, University of Groningen

        % <-this % stops a space
% <-this % stops a space
}

% The paper headers
% \markboth{IEEE Transactions on Knowledge and Data Engineering}%
% {Shell \MakeLowercase{\textit{et al.}}: A Sample Article Using IEEEtran.cls for IEEE Journals}

\IEEEpubid{0000--0000/00\$00.00~\copyright~2021 IEEE}
% Remember, if you use this you must call \IEEEpubidadjcol in the second
% column for its text to clear the IEEEpubid mark.

\maketitle

\begin{abstract}
In real-world applications, perfect labels are rarely available, making it challenging to develop robust machine learning algorithms that can handle noisy labels. Recent methods have focused on filtering noise based on the discrepancy between model predictions and given noisy labels, assuming that samples with small classification losses are clean. This work takes a different approach by leveraging the consistency between the learned model and the entire noisy dataset using the rich representational and topological information in the data. We introduce LaplaceConfidence, a method that to obtain label confidence (i.e., clean probabilities) utilizing the Laplacian energy. Specifically, it first constructs graphs based on the feature representations of all noisy samples and minimizes the Laplacian energy to produce a low-energy graph. Clean labels should fit well into the low-energy graph while noisy ones should not, allowing our method to determine data's clean probabilities. Furthermore, LaplaceConfidence is embedded into a holistic method for robust training, where co-training technique generates unbiased label confidence and label refurbishment technique better utilizes it. We also explore the dimensionality reduction technique to accommodate our method on large-scale noisy datasets. Our experiments demonstrate that LaplaceConfidence outperforms state-of-the-art methods on benchmark datasets under both synthetic and real-world noise.
% Code available at \url{https://anonymous.4open.science/r/LaplaceConfidence-07D0}.
\end{abstract}
\begin{IEEEkeywords}
Learning with noisy labels, graph energy, dimensionality reduction.
\end{IEEEkeywords}

\title{Inclusive photon multiplicity at forward pseudorapidities in pp and p--Pb collisions \\ at $\sqrt{s_{\rm NN}}$ = 5.02~TeV with ALICE}
\author[1*]{Abhi Modak (for the ALICE Collaboration)}
\affil[1]{Bose Institute, Kolkata, India}
\affil[*]{Address correspondence to: abhi.modak@cern.ch}

\onehalfspacing
\maketitle

\date{}

%%%%%% Abstract %%%%%
\begin{abstract}

Global observables such as the pseudorapidity distributions of particle multiplicities in the final state are crucial to shed light into the physics processes involved in hadronic collisions. In proton--lead (p--Pb) collisions at Large Hadron Collider (LHC) energies, such measurements provide an important baseline to understand lead--lead (Pb--Pb) results by disentangling hot nuclear matter effects from the ones due to the cold nuclear matter. Multiplicity measurements can also put constraints on theoretical models describing the initial stages of the collision, e.g., to what degree the nucleon and the nuclei interact as dilute (partons) or dense (CGC-like) fields. The study of inclusive photon multiplicity aims to provide complementary measurements to those obtained with charged particles.

In these proceedings, the pseudorapidity distributions of inclusive photons at forward pseudorapidity (2.3~$<~\eta_{\rm \,lab}~<$~3.9) in pp and p--Pb collisions at $\sqrt{s_{\rm NN}}$ = 5.02 TeV are presented. The data samples were collected using the Photon Multiplicity Detector (PMD) of ALICE. The multiplicity dependence of photon production in p--Pb collisions is presented and a comparison with charged-particle distributions measured at mid-pseudorapidity is shown. The results are also compared with predictions from Monte Carlo event generators.

\end{abstract}

\section{Introduction}
One of the primary goals of heavy-ion collision experiments, such as ALICE, is to study and understand the properties of the deconfined state of nuclear matter, commonly known as the quark--gluon plasma (QGP). The first step in characterizing the produced QGP matter in these collisions is the measurement of pseudorapidity distributions of produced final-state particles. Such studies in pp and p--Pb collisions are also important as they provide the baselines for the interpretation of the measurements in heavy-ion collisions. This contribution reports the measurements of inclusive photon multiplicities for minimum bias pp, p--Pb collisions and for various multiplicity classes in p--Pb collisions at $\sqrt{s\rm_{NN}}$~=~5.02~TeV.

\section{Data analysis}

This analysis was performed using the ALICE~\cite{ALICE:Exp} data collected in 2013 during LHC Run~1 for p--Pb collisions and in 2015 during LHC Run~2 for pp collisions. The data from p--Pb collisions were recorded for two beam configurations: in one (denoted as p--Pb), the lead beam travelled towards positive $\eta_{\rm \,lab}$ and in the other configuration (denoted as Pb--p) it moved towards negative $\eta_{\rm \,lab}$. The pp data were analysed for inelastic events, whereas measurements in p--Pb collisions were performed for non-single diffractive interactions. Events with the reconstructed primary vertex position along the beam line, $|v_{z}|<10$ cm, from the nominal interaction point were considered. The multiplicity classes were determined by measuring the charged-particle multiplicity in the outer layer of the Silicon Pixel Detector~\cite{spd} at mid-pseudorapidity (denoted as CL1 estimator) and the energy deposited in the Pb-remnant side of the neutron calorimeter~\cite{zdc} at beam rapidity (denoted as ZNA estimator)~\cite{ALICE:ChPrpPbCent}. The raw distributions of photons were obtained by counting the number of reconstructed clusters (in the preshower plane of the Photon Multiplicity Detector (PMD)~\cite{pmd}) that satisfied the photon--hadron discrimination thresholds~\cite{ALICE:PMDpp,ALICE:PMDpPb}. The distributions were then corrected for various instrumental effects (detector inefficiency, limited acceptance, contaminations from hadron clusters and secondary particles produced in interactions with surrounding materials of the PMD) using a Bayesian unfolding method~\cite{BayesUnfold}. Systematic uncertainties from various sources (effect of upstream material in front of the PMD, hadron and secondary photon contaminations, event generator dependence, unfolding method) were estimated and then added in quadrature. The total systematic uncertainty was found to be around 9--10\%~\cite{ALICE:PMDpPb}.

% Figure environment removed

\section{Results and discussion}

Figure~\ref{dndeta_MB} presents the pseudorapidity distributions (d$N_{\rm \gamma}$/d$\eta_{\rm \,lab}$) of inclusive photons in pp, p--Pb, and Pb--p collisions at $\sqrt{s_{\rm NN}}$~=~5.02~TeV measured within 2.3~$<~\eta_{\rm \,lab}~<$~3.9 together with the measurements of charged-particle multiplicities (d$N_{\rm ch}$/d$\eta_{\rm \,lab}$) at mid-pseudorapidity~\cite{ALICE:ChPrppMB,ALICE:ChPrpPbMB,CMS:ChPrpPbMB}. The data from pp and Pb--p collisions are reflected around $\eta_{\rm \,lab}$~=~0 to extend the measurements in the region, $-3.9<\eta_{\rm \,lab}<-2.3$. The d$N_{\rm \gamma}$/d$\eta_{\rm \,lab}$ at forward pseudorapidity smoothly matches with the d$N_{\rm ch}$/d$\eta_{\rm \,lab}$ at mid-pseudorapidity indicating that the production mechanisms for charged and neutral pions are similar. The predictions from various MC models are also displayed in Fig.~\ref{dndeta_MB} and show similar values for photon (solid lines) and charged-particle (dashed lines) multiplicities at forward and backward pseudorapidities, while at mid-pseudorapidity the d$N_{\rm \gamma}$/d$\eta_{\rm \,lab}$ differs from the d$N_{\rm ch}$/d$\eta_{\rm \,lab}$. This difference is due to a mass effect in the transformation between $\mathrm{d}N/\mathrm{d}y$ and $\mathrm{d}N/\mathrm{d}\eta$ at $\eta \approx 0$. Both HIJING (v1.36)~\cite{hijing} and DPMJET (v3.0-5)~\cite{dpmjet} event generators fairly describe the measured d$N_{\rm ch}$/d$\eta_{\rm \,lab}$ in p--Pb collisions. The DPMJET slightly underpredicts the d$N_{\rm \gamma}$/d$\eta_{\rm \,lab}$ in the p-going side and reproduces the same within uncertainties in the Pb-going side. For pp collisions, both EPOS LHC~\cite{eposlhc} and PYTHIA 8 (v8.243) with the Monash 2013~tune~\cite{pythia8_monash} overestimate the photon and charged-particle multiplicity.

% Figure environment removed

Figure~\ref{dndeta_cent} shows the pseudorapidity distributions of both photons and charged particles measured in p--Pb collisions for three multiplicity classes (0--5\%, 20--40\% and 80--100\%) determined with the CL1 (top panel) and ZNA (bottom panel) estimators. The particle density in the highest multiplicity class (0--5\%) when considering the CL1 (ZNA) estimator reaches values thrice (twice) as large as those in minimum bias p--Pb collisions. A clear asymmetric shape is observed for d$N_{\rm ch}$/d$\eta_{\rm \,lab}$ in the highest multiplicity class (0--5\%) and the shape becomes symmetric, like in pp, in the lowest multiplicity class (80--100\%). HIJING describes the d$N_{\rm \gamma}$/d$\eta_{\rm \,lab}$ at forward pseudorapidity within the measurement uncertainties. For 80--100\% event class, HIJING overestimates (underestimates) the d$N_{\rm ch}$/d$\eta_{\rm \,lab}$ for the CL1 (ZNA) estimator.

\section{Conclusion}

The pseudorapidity distributions of inclusive photons were measured over a kinematic region of 2.3~$<~\eta_{\rm \,lab}~<$~3.9 for minimum bias pp, p--Pb, and Pb--p collisions and for different multiplicity classes in p--Pb collisions at $\sqrt{s_{\rm NN}}$~=~5.02~TeV. The d$N_{\rm \gamma}$/d$\eta_{\rm \,lab}$ at forward pseudorapidity was observed to follow the trend of similar measurements of charged particles at mid-pseudorapidity. The predictions from various MC describe the data within 15--20\%. These results will help to establish baselines for the interpretation of Pb--Pb collision data.

\printbibliography


\section{Acknowledgements}
This paper is supported by the National Natural Science Foundation of China (Grant No. 62192783, U1811462), the Collaborative Innovation Center of Novel Software Technology and Industrialization at Nanjing University.

\bibliographystyle{IEEEtran}
\bibliography{egbib}
% \begin{thebibliography}{1}
% \bibliographystyle{IEEEtran}

% \bibitem{ref1}
% {\it{Mathematics Into Type}}. American Mathematical Society. [Online]. Available: https://www.ams.org/arc/styleguide/mit-2.pdf

% \bibitem{ref2}
% T. W. Chaundy, P. R. Barrett and C. Batey, {\it{The Printing of Mathematics}}. London, U.K., Oxford Univ. Press, 1954.

% \bibitem{ref3}
% F. Mittelbach and M. Goossens, {\it{The \LaTeX Companion}}, 2nd ed. Boston, MA, USA: Pearson, 2004.

% \bibitem{ref4}
% G. Gr\"atzer, {\it{More Math Into LaTeX}}, New York, NY, USA: Springer, 2007.

% \bibitem{ref5}M. Letourneau and J. W. Sharp, {\it{AMS-StyleGuide-online.pdf,}} American Mathematical Society, Providence, RI, USA, [Online]. Available: http://www.ams.org/arc/styleguide/index.html

% \bibitem{ref6}
% H. Sira-Ramirez, ``On the sliding mode control of nonlinear systems,'' \textit{Syst. Control Lett.}, vol. 19, pp. 303--312, 1992.

% \bibitem{ref7}
% A. Levant, ``Exact differentiation of signals with unbounded higher derivatives,''  in \textit{Proc. 45th IEEE Conf. Decis.
% Control}, San Diego, CA, USA, 2006, pp. 5585--5590. DOI: 10.1109/CDC.2006.377165.

% \bibitem{ref8}
% M. Fliess, C. Join, and H. Sira-Ramirez, ``Non-linear estimation is easy,'' \textit{Int. J. Model., Ident. Control}, vol. 4, no. 1, pp. 12--27, 2008.

% \bibitem{ref9}
% R. Ortega, A. Astolfi, G. Bastin, and H. Rodriguez, ``Stabilization of food-chain systems using a port-controlled Hamiltonian description,'' in \textit{Proc. Amer. Control Conf.}, Chicago, IL, USA,
% 2000, pp. 2245--2249.

% \end{thebibliography}


\newpage

% \section{Biography Section}
% If you have an EPS/PDF photo (graphicx package needed), extra braces are
%  needed around the contents of the optional argument to biography to prevent
%  the LaTeX parser from getting confused when it sees the complicated
%  $\backslash${\tt{includegraphics}} command within an optional argument. (You can create
%  your own custom macro containing the $\backslash${\tt{includegraphics}} command to make things
%  simpler here.)
 
% \vspace{11pt}

% \bf{If you include a photo:}\vspace{-33pt}
% \begin{IEEEbiography}[{% Figure removed}]{Michael Shell}
% Use $\backslash${\tt{begin\{IEEEbiography\}}} and then for the 1st argument use $\backslash${\tt{includegraphics}} to declare and link the author photo.
% Use the author name as the 3rd argument followed by the biography text.
% \end{IEEEbiography}

% \vspace{11pt}

% \bf{If you will not include a photo:}\vspace{-33pt}
% \begin{IEEEbiographynophoto}{John Doe}
% Use $\backslash${\tt{begin\{IEEEbiographynophoto\}}} and the author name as the argument followed by the biography text.
% \end{IEEEbiographynophoto}

% \clearpage

\end{document}
    
    
