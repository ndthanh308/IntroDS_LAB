%%
%% This is file `sample-sigconf.tex',
%% generated with the docstrip utility.
%%
%% The original source files were:
%%
%% samples.dtx  (with options: `sigconf')
%% 
%% IMPORTANT NOTICE:
%% 
%% For the copyright see the source file.
%% 
%% Any modified versions of this file must be renamed
%% with new filenames distinct from sample-sigconf.tex.
%% 
%% For distribution of the original source see the terms
%% for copying and modification in the file samples.dtx.
%% 
%% This generated file may be distributed as long as the
%% original source files, as listed above, are part of the
%% same distribution. (The sources need not necessarily be
%% in the same archive or directory.)
%%
%%
%% Commands for TeXCount
%TC:macro \cite [option:text,text]
%TC:macro \citep [option:text,text]
%TC:macro \citet [option:text,text]
%TC:envir table 0 1
%TC:envir table* 0 1
%TC:envir tabular [ignore] word
%TC:envir displaymath 0 word
%TC:envir math 0 word
%TC:envir comment 0 0
%%
%%
%% The first command in your LaTeX source must be the \documentclass
%% command.
%%
%% For submission and review of your manuscript please change the
% command to \documentclass[manuscript, screen, review]{acmart}.
%%
%% When submitting camera ready or to TAPS, please change the command
%% to \documentclass[sigconf]{acmart} or whichever template is required
%% for your publication.
%%
%%
\documentclass[sigconf, screen, review]{acmart}
% \documentclass[sigconf]{acmart}

%%
%% \BibTeX command to typeset BibTeX logo in the docs
\AtBeginDocument{%
  \providecommand\BibTeX{{%
    Bib\TeX}}}

%% Rights management information.  This information is sent to you
%% when you complete the rights form.  These commands have SAMPLE
%% values in them; it is your responsibility as an author to replace
%% the commands and values with those provided to you when you
%% complete the rights form.
\setcopyright{acmcopyright}
\copyrightyear{2023}
\acmYear{2023}
\acmDOI{XXXXXXX.XXXXXXX}

%% These commands are for a PROCEEDINGS abstract or paper.
\acmConference[ACM MM '23]{31st ACM International Conference on Multimedia}{October 29--November 3,
  2023}{Ottawa, Canada}
%%
%%  Uncomment \acmBooktitle if the title of the proceedings is different
%%  from ``Proceedings of ...''!
%%
%%\acmBooktitle{Woodstock '18: ACM Symposium on Neural Gaze Detection,
%%  June 03--05, 2018, Woodstock, NY}
\acmPrice{15.00}
\acmISBN{978-1-4503-XXXX-X/18/06}

\usepackage{algorithm}
\usepackage{siunitx}
\usepackage{xcolor}
\usepackage{longtable}
\usepackage{dsfont}
\usepackage{subfig} %[position=top]
\usepackage{multirow}
\usepackage{textcomp}
\usepackage{tabularx}
\usepackage{enumerate}
\usepackage{nameref}
\usepackage{graphicx}
% \interfootnotelinepenalty=10000
\usepackage{wrapfig}
\usepackage{makecell}
\usepackage{algpseudocode}
\usepackage{mathtools}
\DeclarePairedDelimiter\abs{\lvert}{\rvert}
\usepackage{comment}
\usepackage{booktabs}
\makeatletter
\def\RS#1{\zap@space#1 \@empty}
\usepackage{bbm}
\usepackage[T1]{fontenc}

%%
%% Submission ID.
%% Use this when submitting an article to a sponsored event. You'll
%% receive a unique submission ID from the organizers
%% of the event, and this ID should be used as the parameter to this command.
\acmSubmissionID{1165}

%%
%% For managing citations, it is recommended to use bibliography
%% files in BibTeX format.
%%
%% You can then either use BibTeX with the ACM-Reference-Format style,
%% or BibLaTeX with the acmnumeric or acmauthoryear sytles, that include
%% support for advanced citation of software artefact from the
%% biblatex-software package, also separately available on CTAN.
%%
%% Look at the sample-*-biblatex.tex files for templates showcasing
%% the biblatex styles.
%%

%%
%% The majority of ACM publications use numbered citations and
%% references.  The command \citestyle{authoryear} switches to the
%% "author year" style.
%%
%% If you are preparing content for an event
%% sponsored by ACM SIGGRAPH, you must use the "author year" style of
%% citations and references.
%% Uncommenting
%% the next command will enable that style.
%%\citestyle{acmauthoryear}


%%
%% end of the preamble, start of the body of the document source.
\begin{document}

%%
%% The "title" command has an optional parameter,
%% allowing the author to define a "short title" to be used in page headers.
\title{LaplaceConfidence: A Graph-based Approach for Learning with Noisy Labels}
% \shorttitle{LaplaceConfidence}

%%
%% The "author" command and its associated commands are used to define
%% the authors and their affiliations.
%% Of note is the shared affiliation of the first two authors, and the
%% "authornote" and "authornotemark" commands
%% used to denote shared contribution to the research.
\author{Anonymous Author(s)\\  Submission ID:1165}
% \author{Ben Trovato}
% \authornote{Both authors contributed equally to this research.}
% \email{trovato@corporation.com}
% \orcid{1234-5678-9012}
% \author{G.K.M. Tobin}
% \authornotemark[1]
% \email{webmaster@marysville-ohio.com}
% \affiliation{%
%   \institution{Institute for Clarity in Documentation}
%   \streetaddress{P.O. Box 1212}
%   \city{Dublin}
%   \state{Ohio}
%   \country{USA}
%   \postcode{43017-6221}
% }

% \author{Lars Th{\o}rv{\"a}ld}
% \affiliation{%
%   \institution{The Th{\o}rv{\"a}ld Group}
%   \streetaddress{1 Th{\o}rv{\"a}ld Circle}
%   \city{Hekla}
%   \country{Iceland}}
% \email{larst@affiliation.org}

% \author{Valerie B\'eranger}
% \affiliation{%
%   \institution{Inria Paris-Rocquencourt}
%   \city{Rocquencourt}
%   \country{France}
% }

% \author{Aparna Patel}
% \affiliation{%
%  \institution{Rajiv Gandhi University}
%  \streetaddress{Rono-Hills}
%  \city{Doimukh}
%  \state{Arunachal Pradesh}
%  \country{India}}

% \author{Huifen Chan}
% \affiliation{%
%   \institution{Tsinghua University}
%   \streetaddress{30 Shuangqing Rd}
%   \city{Haidian Qu}
%   \state{Beijing Shi}
%   \country{China}}

% \author{Charles Palmer}
% \affiliation{%
%   \institution{Palmer Research Laboratories}
%   \streetaddress{8600 Datapoint Drive}
%   \city{San Antonio}
%   \state{Texas}
%   \country{USA}
%   \postcode{78229}}
% \email{cpalmer@prl.com}

% \author{John Smith}
% \affiliation{%
%   \institution{The Th{\o}rv{\"a}ld Group}
%   \streetaddress{1 Th{\o}rv{\"a}ld Circle}
%   \city{Hekla}
%   \country{Iceland}}
% \email{jsmith@affiliation.org}

% \author{Julius P. Kumquat}
% \affiliation{%
%   \institution{The Kumquat Consortium}
%   \city{New York}
%   \country{USA}}
% \email{jpkumquat@consortium.net}

%%
%% By default, the full list of authors will be used in the page
%% headers. Often, this list is too long, and will overlap
%% other information printed in the page headers. This command allows
%% the author to define a more concise list
%% of authors' names for this purpose.
% \renewcommand{\shortauthors}{Trovato et al.}
% 准备期刊投稿,修改之前的做噪声标签的工作,然后增加一个类别不平衡的设定。动机就是两个、一个是大规模数据集很有可能同时是noisy和类别不平衡的。另一个是它本身是一个比较有意思的问题,就是这中困难样本挖掘的问题。当然要定义困难样本是比较模糊的,但是找tail-类别的噪声标签定义就比较明确。
%%
%% The abstract is a short summary of the work to be presented in the
%% article.
\begin{abstract}
In real-world applications, perfect labels are rarely available, making it challenging to develop robust machine learning algorithms that can handle noisy labels. Recent methods have focused on filtering noise based on the discrepancy between model predictions and given noisy labels, assuming that samples with small classification losses are clean. This work takes a different approach by leveraging the consistency between the learned model and the entire noisy dataset using the rich representational and topological information in the data. We introduce LaplaceConfidence, a method that to obtain label confidence (i.e., clean probabilities) utilizing the Laplacian energy. Specifically, it first constructs graphs based on the feature representations of all noisy samples and minimizes the Laplacian energy to produce a low-energy graph. Clean labels should fit well into the low-energy graph while noisy ones should not, allowing our method to determine data's clean probabilities. Furthermore, LaplaceConfidence is embedded into a holistic method for robust training, where co-training technique generates unbiased label confidence and label refurbishment technique better utilizes it. We also explore the dimensionality reduction technique to accommodate our method on large-scale noisy datasets. Our experiments demonstrate that LaplaceConfidence outperforms state-of-the-art methods on benchmark datasets under both synthetic and real-world noise.
Code available at \url{https://anonymous.4open.science/r/LaplaceConfidence-07D0}.
\end{abstract}

%%
%% The code below is generated by the tool at http://dl.acm.org/ccs.cfm.
%% Please copy and paste the code instead of the example below.
%%
\begin{CCSXML}
<ccs2012>
   <concept>
       <concept_id>10010147.10010257.10010282</concept_id>
       <concept_desc>Computing methodologies~Learning settings</concept_desc>
       <concept_significance>300</concept_significance>
       </concept>
 </ccs2012>
\end{CCSXML}

\ccsdesc[300]{Computing methodologies~Learning settings}

% \begin{CCSXML}
% <ccs2012>
%  <concept>
%   <concept_id>10010520.10010553.10010562</concept_id>
%   <concept_desc>Computer systems organization~Embedded systems</concept_desc>
%   <concept_significance>500</concept_significance>
%  </concept>
%  <concept>
%   <concept_id>10010520.10010575.10010755</concept_id>
%   <concept_desc>Computer systems organization~Redundancy</concept_desc>
%   <concept_significance>300</concept_significance>
%  </concept>
%  <concept>
%   <concept_id>10010520.10010553.10010554</concept_id>
%   <concept_desc>Computer systems organization~Robotics</concept_desc>
%   <concept_significance>100</concept_significance>
%  </concept>
%  <concept>
%   <concept_id>10003033.10003083.10003095</concept_id>
%   <concept_desc>Networks~Network reliability</concept_desc>
%   <concept_significance>100</concept_significance>
%  </concept>
% </ccs2012>
% \end{CCSXML}

% \ccsdesc[500]{Computer systems organization~Embedded systems}
% \ccsdesc[300]{Computer systems organization~Redundancy}
% \ccsdesc{Computer systems organization~Robotics}
% \ccsdesc[100]{Networks~Network reliability}

%%
%% Keywords. The author(s) should pick words that accurately describe
%% the work being presented. Separate the keywords with commas.
\keywords{learning with noisy labels, graph energy, dimensionality reduction}
%% A "teaser" image appears between the author and affiliation
%% information and the body of the document, and typically spans the
%% page.
% \begin{teaserfigure}
%   % Figure removed
%   \caption{Seattle Mariners at Spring Training, 2010.}
%   \Description{Enjoying the baseball game from the third-base
%   seats. Ichiro Suzuki preparing to bat.}
%   \label{fig:teaser}
% \end{teaserfigure}

% \received{20 February 2007}
% \received[revised]{12 March 2009}
% \received[accepted]{5 June 2009}

%%
%% This command processes the author and affiliation and title
%% information and builds the first part of the formatted document.
\maketitle
\title{Inclusive photon multiplicity at forward pseudorapidities in pp and p--Pb collisions \\ at $\sqrt{s_{\rm NN}}$ = 5.02~TeV with ALICE}
\author[1*]{Abhi Modak (for the ALICE Collaboration)}
\affil[1]{Bose Institute, Kolkata, India}
\affil[*]{Address correspondence to: abhi.modak@cern.ch}

\onehalfspacing
\maketitle

\date{}

%%%%%% Abstract %%%%%
\begin{abstract}

Global observables such as the pseudorapidity distributions of particle multiplicities in the final state are crucial to shed light into the physics processes involved in hadronic collisions. In proton--lead (p--Pb) collisions at Large Hadron Collider (LHC) energies, such measurements provide an important baseline to understand lead--lead (Pb--Pb) results by disentangling hot nuclear matter effects from the ones due to the cold nuclear matter. Multiplicity measurements can also put constraints on theoretical models describing the initial stages of the collision, e.g., to what degree the nucleon and the nuclei interact as dilute (partons) or dense (CGC-like) fields. The study of inclusive photon multiplicity aims to provide complementary measurements to those obtained with charged particles.

In these proceedings, the pseudorapidity distributions of inclusive photons at forward pseudorapidity (2.3~$<~\eta_{\rm \,lab}~<$~3.9) in pp and p--Pb collisions at $\sqrt{s_{\rm NN}}$ = 5.02 TeV are presented. The data samples were collected using the Photon Multiplicity Detector (PMD) of ALICE. The multiplicity dependence of photon production in p--Pb collisions is presented and a comparison with charged-particle distributions measured at mid-pseudorapidity is shown. The results are also compared with predictions from Monte Carlo event generators.

\end{abstract}

\section{Introduction}
One of the primary goals of heavy-ion collision experiments, such as ALICE, is to study and understand the properties of the deconfined state of nuclear matter, commonly known as the quark--gluon plasma (QGP). The first step in characterizing the produced QGP matter in these collisions is the measurement of pseudorapidity distributions of produced final-state particles. Such studies in pp and p--Pb collisions are also important as they provide the baselines for the interpretation of the measurements in heavy-ion collisions. This contribution reports the measurements of inclusive photon multiplicities for minimum bias pp, p--Pb collisions and for various multiplicity classes in p--Pb collisions at $\sqrt{s\rm_{NN}}$~=~5.02~TeV.

\section{Data analysis}

This analysis was performed using the ALICE~\cite{ALICE:Exp} data collected in 2013 during LHC Run~1 for p--Pb collisions and in 2015 during LHC Run~2 for pp collisions. The data from p--Pb collisions were recorded for two beam configurations: in one (denoted as p--Pb), the lead beam travelled towards positive $\eta_{\rm \,lab}$ and in the other configuration (denoted as Pb--p) it moved towards negative $\eta_{\rm \,lab}$. The pp data were analysed for inelastic events, whereas measurements in p--Pb collisions were performed for non-single diffractive interactions. Events with the reconstructed primary vertex position along the beam line, $|v_{z}|<10$ cm, from the nominal interaction point were considered. The multiplicity classes were determined by measuring the charged-particle multiplicity in the outer layer of the Silicon Pixel Detector~\cite{spd} at mid-pseudorapidity (denoted as CL1 estimator) and the energy deposited in the Pb-remnant side of the neutron calorimeter~\cite{zdc} at beam rapidity (denoted as ZNA estimator)~\cite{ALICE:ChPrpPbCent}. The raw distributions of photons were obtained by counting the number of reconstructed clusters (in the preshower plane of the Photon Multiplicity Detector (PMD)~\cite{pmd}) that satisfied the photon--hadron discrimination thresholds~\cite{ALICE:PMDpp,ALICE:PMDpPb}. The distributions were then corrected for various instrumental effects (detector inefficiency, limited acceptance, contaminations from hadron clusters and secondary particles produced in interactions with surrounding materials of the PMD) using a Bayesian unfolding method~\cite{BayesUnfold}. Systematic uncertainties from various sources (effect of upstream material in front of the PMD, hadron and secondary photon contaminations, event generator dependence, unfolding method) were estimated and then added in quadrature. The total systematic uncertainty was found to be around 9--10\%~\cite{ALICE:PMDpPb}.

% Figure environment removed

\section{Results and discussion}

Figure~\ref{dndeta_MB} presents the pseudorapidity distributions (d$N_{\rm \gamma}$/d$\eta_{\rm \,lab}$) of inclusive photons in pp, p--Pb, and Pb--p collisions at $\sqrt{s_{\rm NN}}$~=~5.02~TeV measured within 2.3~$<~\eta_{\rm \,lab}~<$~3.9 together with the measurements of charged-particle multiplicities (d$N_{\rm ch}$/d$\eta_{\rm \,lab}$) at mid-pseudorapidity~\cite{ALICE:ChPrppMB,ALICE:ChPrpPbMB,CMS:ChPrpPbMB}. The data from pp and Pb--p collisions are reflected around $\eta_{\rm \,lab}$~=~0 to extend the measurements in the region, $-3.9<\eta_{\rm \,lab}<-2.3$. The d$N_{\rm \gamma}$/d$\eta_{\rm \,lab}$ at forward pseudorapidity smoothly matches with the d$N_{\rm ch}$/d$\eta_{\rm \,lab}$ at mid-pseudorapidity indicating that the production mechanisms for charged and neutral pions are similar. The predictions from various MC models are also displayed in Fig.~\ref{dndeta_MB} and show similar values for photon (solid lines) and charged-particle (dashed lines) multiplicities at forward and backward pseudorapidities, while at mid-pseudorapidity the d$N_{\rm \gamma}$/d$\eta_{\rm \,lab}$ differs from the d$N_{\rm ch}$/d$\eta_{\rm \,lab}$. This difference is due to a mass effect in the transformation between $\mathrm{d}N/\mathrm{d}y$ and $\mathrm{d}N/\mathrm{d}\eta$ at $\eta \approx 0$. Both HIJING (v1.36)~\cite{hijing} and DPMJET (v3.0-5)~\cite{dpmjet} event generators fairly describe the measured d$N_{\rm ch}$/d$\eta_{\rm \,lab}$ in p--Pb collisions. The DPMJET slightly underpredicts the d$N_{\rm \gamma}$/d$\eta_{\rm \,lab}$ in the p-going side and reproduces the same within uncertainties in the Pb-going side. For pp collisions, both EPOS LHC~\cite{eposlhc} and PYTHIA 8 (v8.243) with the Monash 2013~tune~\cite{pythia8_monash} overestimate the photon and charged-particle multiplicity.

% Figure environment removed

Figure~\ref{dndeta_cent} shows the pseudorapidity distributions of both photons and charged particles measured in p--Pb collisions for three multiplicity classes (0--5\%, 20--40\% and 80--100\%) determined with the CL1 (top panel) and ZNA (bottom panel) estimators. The particle density in the highest multiplicity class (0--5\%) when considering the CL1 (ZNA) estimator reaches values thrice (twice) as large as those in minimum bias p--Pb collisions. A clear asymmetric shape is observed for d$N_{\rm ch}$/d$\eta_{\rm \,lab}$ in the highest multiplicity class (0--5\%) and the shape becomes symmetric, like in pp, in the lowest multiplicity class (80--100\%). HIJING describes the d$N_{\rm \gamma}$/d$\eta_{\rm \,lab}$ at forward pseudorapidity within the measurement uncertainties. For 80--100\% event class, HIJING overestimates (underestimates) the d$N_{\rm ch}$/d$\eta_{\rm \,lab}$ for the CL1 (ZNA) estimator.

\section{Conclusion}

The pseudorapidity distributions of inclusive photons were measured over a kinematic region of 2.3~$<~\eta_{\rm \,lab}~<$~3.9 for minimum bias pp, p--Pb, and Pb--p collisions and for different multiplicity classes in p--Pb collisions at $\sqrt{s_{\rm NN}}$~=~5.02~TeV. The d$N_{\rm \gamma}$/d$\eta_{\rm \,lab}$ at forward pseudorapidity was observed to follow the trend of similar measurements of charged particles at mid-pseudorapidity. The predictions from various MC describe the data within 15--20\%. These results will help to establish baselines for the interpretation of Pb--Pb collision data.

\printbibliography

\newpage
\bibliographystyle{ACM-Reference-Format}
\bibliography{egbib}


%%
%% If your work has an appendix, this is the place to put it.
\appendix


\end{document}
\endinput
%%
%% End of file `sample-sigconf.tex'.
