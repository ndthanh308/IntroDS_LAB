\documentclass[floatfix,reprint,amsmath,amssymb,aps,prl]{revtex4-2}
\bibliographystyle{apsrev4-2}

\usepackage{graphicx}
\usepackage{bm}
\usepackage[hidelinks]{hyperref}
\usepackage[nameinlink,capitalize]{cleveref}
\usepackage{physics}

\newcommand{\gm}{\gamma_{-}}
\newcommand{\gp}{\gamma_{+}}
\newcommand{\avgS}[2]{\expval{S^{#1}}_{#2}} 

\begin{document}

\title{Macroscopic quantum synchronization effects}
\author{Tobias Nadolny}
\affiliation{Department of Physics, University of Basel, Klingelbergstrasse 82, 4056 Basel, Switzerland}
 \author{Christoph Bruder}
\affiliation{Department of Physics, University of Basel, Klingelbergstrasse 82, 4056 Basel, Switzerland}

\date{July 28, 2023}

\begin{abstract}
We theoretically describe macroscopic quantum synchronization effects occurring in a network of all-to-all coupled quantum limit-cycle oscillators.
The coupling causes a transition to synchronization as indicated by the presence of global phase coherence.
We demonstrate that the microscopic quantum properties of the oscillators qualitatively shape the synchronization behavior in a macroscopically large system.
The resulting dynamics features universal behavior, quantum effects, and emergent behavior not visible at the level of two coupled oscillators.
\end{abstract}
\maketitle

What can be inferred about the behavior of a large system, given that its constituent parts are well understood?
Features of its building blocks can remain visible on a macroscopic scale.
Alternatively, the detailed microscopic description of its parts may become irrelevant such that the large system behaves universally.
Yet another possibility is the emergence of a behavior not present in the individual parts.
We discuss this issue
in the context of quantum synchronization and find instances of all three scenarios.

In the presence of a sufficiently strong coupling, limit-cycle oscillators adjust their frequencies and exhibit coherence even in the presence of noise and disorder in their natural frequencies.
This phenomenon is called synchronization; it appears, e.g., in physical, biological, engineered, and social systems~\cite{10.1063/5.0026335,Buck,bridge}
and has been extensively studied in classical nonlinear dynamics~\cite{acebronKuramotoModelSimple2005,pikovsky_rosenblum_kurths_2001,sync}.

Recently, understanding synchronization of quantum oscillators has attracted a great deal of interest~\cite{
Shepelyansky2006,
Chia2020,Lifshits2021,
ludwig_marquardt2013,lee_QuantumSynchronizationQuantum_2013,walterQuantumSynchronizationDriven2014,
rouletSynchronizingSmallestPossible2018,parra-lopezSynchronizationTwolevelQuantum2020,
PhysRevLett.125.013601,PhysRevA.105.062206,zhang2023observing,koppenhoferQuantumSynchronizationIBM2020,
Holland2014,zhuSynchronizationInteractingQuantum2015,
lee_QuantumSynchronizationQuantum_2013,Fazio2013,Davis-Tilley_2018,PhysRevE.107.024204,Witthaut2017,lorenzoQuantumSynchronisationClustering2021,Waechtler2023,SongHeshan2017b,ishibashiOscillationCollapseCoupled2017,
lorchGenuineQuantumSignatures2016, lorchQuantumSynchronizationBlockade2017,Cooper2019,Solano2019a,Chia2023,
PhysRevE.107.024204,
zhuSynchronizationInteractingQuantum2015, Witthaut2017,
Fazio2013,lorenzoQuantumSynchronisationClustering2021,
Zambrini2012,Fazio2013,Ameri2015,Armour2015,Bastidas2015,Armour2016,Weiss_2016,SongHeshan2017a,SongHeshan2017b,Talitha2017,WuJin-Hui2017,amitaiQuantumEffectsAmplitude2018,Kwek_Squeezing2018,Solano2019b,Jaseem2020,Fazio2020,Zambrini2021,Lutz2022,Parvinder2022,Biswabibek2022,BucaJaksch2022,KatoNakao2023,PhysRevA.99.033818,Davidsen2023,rouletQuantumSynchronizationEntanglement2018,koppenhoferOptimalSynchronizationDeep2019,jaseemQuantumSynchronisationNanoscale2020,solanki2022symmetries,
PhysRevLett.131.030401}.
Quantum limit-cycle oscillators~\cite{Chia2020,Lifshits2021} can be implemented using harmonic 
oscillators~\cite{ludwig_marquardt2013,lee_QuantumSynchronizationQuantum_2013,walterQuantumSynchronizationDriven2014} or few-level systems~\cite{rouletSynchronizingSmallestPossible2018,parra-lopezSynchronizationTwolevelQuantum2020} supplemented by gain and loss.
Experimental realizations span systems of cold atoms~\cite{PhysRevLett.125.013601}, nuclear spins~\cite{PhysRevA.105.062206}, trapped ions~\cite{zhang2023observing}, and a simulation on an IBM quantum computer~\cite{koppenhoferQuantumSynchronizationIBM2020}.
Synchronization of large networks of quantum oscillators has been discussed, particularly two-level systems~\cite{Holland2014,zhuSynchronizationInteractingQuantum2015}
and harmonic oscillators~\cite{lee_QuantumSynchronizationQuantum_2013,ludwig_marquardt2013,ishibashiOscillationCollapseCoupled2017,PhysRevE.107.024204,Davis-Tilley_2018,SongHeshan2017b,Waechtler2023,Witthaut2017,lorenzoQuantumSynchronisationClustering2021,Fazio2013}.
In most cases, their synchronization is similar to that of classical noisy oscillators.
Some quantum features in such systems are discussed in Refs.~\cite{zhuSynchronizationInteractingQuantum2015,Witthaut2017,lorenzoQuantumSynchronisationClustering2021,Fazio2013,PhysRevE.107.024204},
e.g., the presence of entanglement and quantum discord.
Quantum effects beyond the influence of quantum noise that lead to a synchronization behavior qualitatively different from classical expectations have been studied mostly on the level of one, two, or three coupled oscillators~\cite{lorchGenuineQuantumSignatures2016, lorchQuantumSynchronizationBlockade2017,Solano2019a,Cooper2019,Chia2023}.
So far, it remained an open question whether these effects survive when increasing the system size.
Will they remain visible, become irrelevant, or will new behavior emerge?

In this work, we show that in a macroscopic ensemble of interacting quantum oscillators, the synchronization behavior is qualitatively shaped by their quantum nature.
The features of the constituent parts indeed remain visible in a large network of quantum oscillators.
Both their wave-like character leading to interference and the discreteness of their energy levels manifest themselves in the macroscopic synchronization behavior.
This corresponds to the first of the three possibilities mentioned before: quantum effects persist in large systems.
We additionally observe the other two scenarios, universal and emergent behavior.
Some aspects of the synchronization of two large groups can be understood by classical synchronization models.
In this case, the microscopic details are irrelevant.
Finally, the possibility of oscillators anti-aligning their phases leads to a kind of frustration in the many-body system and hence the suppression of collective synchronization.

\textit{System.---}
% Figure environment removed
%
A sketch of the system considered in this work is shown in \cref{fig:system}. It comprises two groups, each containing $N$ three-level oscillators with states $\ket{0}$, $\ket{1}$ and $\ket{2}$.
In the frame rotating with the common frequency $\omega_z$ (see \cref{fig:system}), the time evolution is governed by the quantum master equation
$\dot \rho = -i[H_0 + H_\mathrm{int},\rho] +  \mathcal{L} \rho$, with
%
\begin{align}
    H_0 &=  \sum_{i} \frac{\delta}{2} \left( S^z_{A, i} - S^z_{B, i} \right)
    +
    K\left( \dyad{2}{2}_{A,i}+\dyad{2}{2}_{B,i} \right) \nonumber
    \\
     H_\mathrm{int} &= 
     \frac{V}{N} \sum_{\sigma}\sum_{i < j}    S^+_{\sigma, i} S^-_{\sigma, j} +
     \frac{V_{AB}}{N}
     \sum_{i, j}
    S^+_{A, i} S^-_{B, j} 
    + \mathrm{h.c.} \nonumber
    \\
    \mathcal{L} &= \sum_{\sigma, i}
    \left(
    \gp \mathcal{D} \left[\dyad{1}{0}_{\sigma,i} \right]
    +
    \gm \mathcal{D} \left[\dyad{1}{2}_{\sigma,i} \right]
    \right) \,.
    \label{eq:system}
\end{align}
In the sums, $i$ and $j$ take values from $1$ to $N$ and $\sigma \in \{A,B\}$ is the group label.
The spin--1 operators are defined as $S^z=\dyad{2}{2}-\dyad{0}{0}$, $S^+=\sqrt{2}(\dyad{2}{1}+\dyad{1}{0})$, and $S^- = (S^+)^\dag$.
The Lindblad dissipator is $\mathcal{D}[o]\rho = o\rho o^\dag -(o^\dag o \rho + \rho o^\dag o)/2$.

The bare Hamiltonian $H_0$ describes the coherent dynamics in the absence of any coupling. The two groups labeled $A$ and $B$ differ by the detuning $\delta$ between them.
The parameter $K$ sets the asymmetry in energy differences between levels $\ket{1}$ and $\ket{2}$, and the levels $\ket{0}$ and $\ket{1}$.
The interaction among the oscillators is described by $H_\mathrm{int}$.
All oscillators are reactively coupled to all others.
The coupling strength within each group is $V$, while the coupling strength between oscillators of different groups is $V_{AB}$.
Finally, each three-level oscillator is incoherently driven to the level $\ket{1}$, with strength $\gp$ ($\gm$) from level $\ket{0}$ ($\ket{2}$).
Due to these gain and loss processes, each three-level oscillator forms a limit cycle~\cite{rouletSynchronizingSmallestPossible2018}, whose population (measured by $S^z$) is stabilized, while the phase of the amplitude (measured by $S^+$) is free.

Because of the exponential growth of the Hilbert space size, solving Eq.~\eqref{eq:system} becomes intractable for large $N$.
We employ a mean-field treatment, which for the case of an all-to-all coupling discussed here gives an exact solution for the macroscopic dynamics in the limit $N \rightarrow \infty$~\cite{RevModPhys.52.569}.
This approach corresponds to neglecting all correlations between operators, 
or, equivalently, using the product Ansatz,
$\rho = \bigotimes_{\sigma,i} \rho_{\sigma,i}$~\cite{leeEntanglementTongueQuantum2014}.
Since all oscillators within each group are identical, their time evolution can be described in terms of two three-level oscillators with density matrices $\rho_A$ and $\rho_B$ coupled to the mean amplitudes $\avgS{+}{\sigma} = 1/N \sum_i \expval{S_{\sigma,i}^+} = \Tr[\rho_\sigma S^+]$ of each group.
Consequently, \cref{eq:system} reduces to the two coupled nonlinear master equations
\begin{align}
    \begin{split}
        \dot \rho_A &= 
        -i[H_A +
        V_{AB}\left(  S^+ \avgS{-}{B} + \mathrm{h.c.} \right)
        ,
        \rho_A] +  \tilde{\mathcal{L}} \rho_A\, ,
    \\
        \dot \rho_B &= 
        -i[H_B +
        V_{AB}\left(  S^+ \avgS{-}{A} + \mathrm{h.c.} \right)
        ,
        \rho_B] +  \tilde{\mathcal{L}} \rho_B \, ,
    \end{split}
    \label{eq:mf2}
\end{align}
where
$H_\sigma = \pm \frac{\delta}{2} S^z +
    K \dyad{2}{2} + V \left( S^+ \avgS{-}{\sigma} + \mathrm{h.c.} \right)$
and
$\tilde{\mathcal{L}} = 
    \gp \mathcal{D} \left[\dyad{1}{0} \right]
    +
    \gm \mathcal{D} \left[\dyad{1}{2} \right]
$.
The two groups differ by the sign in front of $\delta/2$, which is a plus (minus) for group $A$ ($B$).

To obtain the results discussed in this work, we numerically time-integrate the nonlinear master equations~(\ref{eq:mf2}).
Additionally, we perform a stability analysis of the unsynchronized state $\rho_\sigma = \dyad{1}{1}$, which is a solution of $\dot \rho_\sigma = 0$. 
%
% Figure environment removed

\textit{Synchronization of a single group.---}
We begin to analyze the behavior of a single group by setting the inter-group coupling $V_{AB}$ to zero. For simplicity, the group subscript $\sigma \in \{A,B\}$ is omitted in this section.
To investigate the state of the group, we utilize the mean amplitude $\expval{S^+} = 1/N \sum_i^N \expval{S_i^+}$.
The phase $\phi_i$ of each oscillator is defined through $\langle S^+_i \rangle = \exp(i\phi_i) \abs{\langle S^+_i \rangle}$.
In the absence of any coupling, all oscillators exhibit random phases due to the intrinsic quantum noise.
Therefore, we expect the mean amplitude to vanish in the limit of infinitely many oscillators $N\rightarrow\infty$.
This conclusion is obtained from the mean-field analysis by noting that for small coupling strength, the group of oscillators converges to the steady state $\rho = \dyad{1}{1}$ exhibiting no phase preference since $\expval{S^+} = 0$.
The coupling among the oscillators, however, tends to align their phases.
Similar to the synchronization transition in the Kuramoto model~\cite{Kuramoto1975,acebronKuramotoModelSimple2005}, we find a critical coupling strength $V_c$ beyond which the group synchronizes.
The critical coupling usually depends on both the noise and the frequency disorder inherent in the system.
For the case of a single group, there is no frequency disorder since all oscillators are identical.
There is however intrinsic noise due to quantum fluctuations that increases with the rates $\gm$ and $\gp$ at which each oscillator couples to the environment.

\Cref{fig:1group}(a) displays the time evolution of the mean amplitude in both the unsynchronized and synchronized regimes. 
Below the critical coupling strength, the zero-amplitude state is stable.
For $V>V_c$, in the synchronized regime, the alignment of phases leads to a finite amplitude in the long-time limit with persistent oscillations of $\Re{\expval{S^+}}$.
The frequency of this oscillation will be further addressed when discussing the behavior of two coupled groups.

To analyze the presence or absence of synchronization among the oscillators, we use the time-average of the (in general time-dependent) absolute value of the amplitude $\abs{\expval{S^+}}_t$ in the steady state.
Figure~\ref{fig:1group}(b) depicts this order parameter as a function of the coupling strength, showing a sharp transition between the unsynchronized and synchronized states.
This resembles transitions to a (boundary) time crystal~\cite{iemini_BoundaryTimeCrystals_2018} or to superradiance~\cite{Kirton_2019} that also result from the competition of coherent and incoherent dynamics in open quantum systems.

So far, we have set $\gp/\gm=1/2$.
We now present the order parameter as a function of both the coupling strength and the ratio $\gp/\gm$ in~\cref{fig:1group}(c).
Most notably, for equal gain and loss rates, the critical coupling diverges, i.e., the transition to synchronization disappears.
This is a macroscopic manifestation of the quantum synchronization interference blockade.
A single three-level quantum limit-cycle oscillator subject to an external drive does not synchronize to it when gain and loss rates are equal due to destructive interference~\cite{rouletSynchronizingSmallestPossible2018,koppenhoferOptimalSynchronizationDeep2019, solanki2022symmetries}.
Our result reveals that the wave-like character of the oscillators allowing for interference also shapes the dynamics in the macroscopic ensemble.

In \cite{solanki2022symmetries}, it is shown that for an asymmetric three-level oscillator, the interference blockade is lifted for \textit{any} non-zero value of $K$. 
In contrast, we find that the blockade in the macroscopic ensemble is lifted only for $K<0$, but persists for $K\ge0$, see \cref{fig:1group}(c).
To understand this apparent discrepancy, we examine the microscopic quantum synchronization behavior.
\Cref{fig:1group}(d) shows the phase distribution $s(\phi)$ of a single oscillator coupled to an external drive in the steady state for various values of $K$.
The distribution $s(\phi)$ indicates the phase response of the driven oscillator~\cite{supp}.
In the case $\gamma_+ = \gamma_-$ and $K = 0$, the phase shifts $\phi = 0$ and $\phi = \pi$ between oscillator and drive are equally likely.
For the ensemble of oscillators, this  implies that any mean field present causes each oscillator to align itself both in and out of phase with the mean field.
Hence, the response of each oscillator will not amplify the coherence of the group,
which leads to the absence of synchronization, i.e., the macroscopic interference blockade.
For $K<0$, however, each oscillator preferably aligns its phase with the mean field leading to synchronization of the group. On the other hand, for $K>0$, each oscillator favors a phase shift of $\pi$ with respect to the mean field, hindering synchronization.
Therefore, unlike the interference blockade of a single oscillator, the macroscopic interference blockade is only lifted for negative $K$.
For other ratios of $\gp/\gm$, we similarly find that the phase distribution tends to peak closer to $\phi = 0$ ($\pi$) for negative (positive) values of $K$, which is reflected by the respective critical coupling strengths shown in \cref{fig:1group}(c) being smaller (larger).
To summarize, as a first important result of our work, we find that the interference blockade persists on the macroscopic scale with a significant difference.
While for a single oscillator, the interference blockade vanishes for any kind of asymmetry, in the large ensemble, frustration (in the sense of oscillators aligning out of phase) causes the blockade to persist for $K>0$.

\textit{Synchronization of two groups.---}
We now consider the full model where one half of the oscillators is detuned by $\delta$ from the other half.
We identify three different behaviors in the long-time limit.
The first is the absence of any synchronization, indicated by both amplitudes $\avgS{+}{\sigma}$ vanishing.
Secondly, all oscillators of both groups can fully synchronize.
The third behavior is a state of partial synchronization, i.e., all oscillators within each group are synchronized internally but not with the oscillators of the other group.

To distinguish full and partial synchronization, we compare the oscillation frequencies of both groups.
For this purpose, we compute the discrete Fourier transform of the amplitudes in the long-time limit to obtain the spectra $P_{\sigma}(\omega)=|\mathrm{FT}\{\expval{S^{+}}_{\sigma}(t)\}|$ for each group.
Figures~\ref{fig:2groupsAnh0}(a) and (b) display the spectra as a function of the detuning $\delta$.
We set $V>V_c$ such that the oscillators are synchronized within each group.
%
% Figure environment removed
%
For sufficiently small detuning compared to the inter-group coupling strength, we find a fully synchronized state as indicated by the identical spectra in this regime.
Since each spectrum is dominated by one frequency, we continue the analysis using the two frequencies at which the spectra peak, $\omega_{\sigma} = \mathrm{argmax}_\omega P_\sigma(\omega)$.
The frequency difference $\omega_{A} - \omega_{B}$ between the two groups is displayed in~\cref{fig:2groupsAnh0}(c).
At small $\delta$, the frequencies are equal and the two groups are synchronized, while for large detunings, $\omega_A$ and $\omega_B$ differ by $\delta$.
This corresponds to the dynamics described by the Adler equation~\cite{adler,pikovsky_rosenblum_kurths_2001} for classical phase oscillators.
To further demonstrate this correspondence, we show the frequency difference in \cref{fig:2groupsAnh0}(d) as a function of the inter-group coupling strength.
For small values of $V_{AB} < V$, we observe an Arnold tongue~\cite{pikovsky_rosenblum_kurths_2001}, i.e., the locking range increases linearly with increasing coupling.
Both individually synchronized groups of oscillators can be regarded as two large oscillators which synchronize when their coupling is larger than their detuning.
In this case, the microscopic details are irrelevant and the behavior is universal.
For $V_{AB} > V$, the inter-group coupling dominates, so that the analogy of two large coupled oscillators fails and the spectra show more than one dominant frequency component~\cite{supp}.
%
% Figure environment removed

The previous analysis was done for a symmetric three-level structure, i.e., $K=0$.
We now vary the asymmetry parameter $K$ in addition to the detuning $\delta$ and show the resulting frequency difference, the order parameter $\abs{\avgS{+}{A}}_t$, and the phase difference between the two groups in \cref{fig:2groupsVfix}.
Remarkably, for large $\abs{K}/\gm$, we find synchronization only if $\abs{\delta} \sim \abs{K}$,
while synchronization is absent for $\delta$  around zero.
This is a macroscopic manifestation of the quantum synchronization blockade~\cite{lorchQuantumSynchronizationBlockade2017}:
the two groups synchronize when they are distinct, but not if they are similar.
This is in contrast to the expected behavior that a greater similarity of oscillators increases their tendency to synchronize (see~\cref{fig:2groupsAnh0}(d)).
The absence of synchronization is caused by the discrete energy spectrum of the oscillators.
The effect of the coupling between two oscillators is suppressed when $\abs{K}$ significantly differs from $\abs{\delta}$ because the dominant transitions are off-resonant~\cite{supp}.
Only \textit{close to} the resonances 
$K =\delta$ and $K =-\delta$, there is strong phase alignment of the two oscillators.
This explains the microscopic synchronization blockade, however, it does not yet fully capture the macroscopic quantum synchronization blockade, since we find synchronization only \textit{below} the lines $K = \delta$ and $K = -\delta$.
Below these lines, oscillators of different groups tend to align their phases, while above, they favor opposite phases~\cite{supp}.
This becomes apparent in the phase difference of the two groups, see~\cref{fig:2groupsVfix}(c) which shows the relative phase approaching $\pi$ close to the diagonals.
Each oscillator reacts to the mean fields of both groups, such that their influence cancels if they have opposite phases.
Hence, there is an additional blockade of synchronization for $K>\delta$, which is not present in the case of two oscillators.
To summarize the analysis of two groups, we have shown that parts of their dynamics can be understood as a universal synchronization transition (see \cref{fig:2groupsAnh0}).
More importantly, we have demonstrated that the quantum synchronization blockade persists in the macroscopic system and features an emergent additional regime where synchronization is absent.

\textit{Experimental considerations.---}
Possible experimental realizations include superconducting circuits~\cite{lorchQuantumSynchronizationBlockade2017,Nigg2018}, or trapped ions~\cite{lee_QuantumSynchronizationQuantum_2013,Armour2015, lorchQuantumSynchronizationBlockade2017}.
Since there is no all-to-all coupling of infinitely many oscillators in real systems, we discuss finite-size effects in the supplementary material~\cite{supp}.
There, we show that the lifetime of the coherence in a single group increases linearly with the number of oscillators.
Global synchronization can persist for finite-range interactions in networks of classical oscillators~\cite{acebronKuramotoModelSimple2005} and quantum oscillators~\cite{zhuSynchronizationInteractingQuantum2015}.
The consequences of finite-range interactions in this particular system are left for a future study.


\textit{Conclusion.---}
While quantum effects in synchronization have been studied in few-oscillator systems, it remained an open question whether these effects are visible on a macroscopic scale.
To address this issue, we have investigated the synchronization behavior of two large groups of reactively coupled quantum limit-cycle oscillators.
We have demonstrated that quantum features of synchronization persist in this macroscopic system.
For a single group, we have shown that destructive interference manifests itself as an absence of synchronization for similar gain and loss rates.
For two detuned groups of oscillators with an asymmetric level structure, their quantized nature leads to synchronization of dissimilar oscillators, in contrast to the intuition that a greater similarity increases the tendency to synchronization.
In addition to showing the persistence of microscopic features on the macroscopic scale, we have found that certain aspects of the dynamics are qualitatively explained by universal models of synchronization.
Finally, we uncovered emergent behavior caused by the relative phase differences between oscillators.
In particular, phase frustration in the many-body system results in a blockade of synchronization not visible on the level of two coupled oscillators.

\begin{acknowledgements}
\textbf{Acknowledgments}
We would like to thank M. Koppenh\"ofer and N. L\"orch for stimulating discussions. Furthermore, we acknowledge the use of QuTiP~\cite{Johansson_2013} and QuantumCumulants.jl~\cite{Plankensteiner2022quantumcumulantsjl}, as well as
financial support from the Swiss National Science Foundation individual grant (grant no. 200020 200481).
\end{acknowledgements}

%apsrev4-2.bst 2019-01-14 (MD) hand-edited version of apsrev4-1.bst
%Control: key (0)
%Control: author (72) initials jnrlst
%Control: editor formatted (1) identically to author
%Control: production of article title (-1) disabled
%Control: page (0) single
%Control: year (1) truncated
%Control: production of eprint (0) enabled
\begin{thebibliography}{72}%
\makeatletter
\providecommand \@ifxundefined [1]{%
 \@ifx{#1\undefined}
}%
\providecommand \@ifnum [1]{%
 \ifnum #1\expandafter \@firstoftwo
 \else \expandafter \@secondoftwo
 \fi
}%
\providecommand \@ifx [1]{%
 \ifx #1\expandafter \@firstoftwo
 \else \expandafter \@secondoftwo
 \fi
}%
\providecommand \natexlab [1]{#1}%
\providecommand \enquote  [1]{``#1''}%
\providecommand \bibnamefont  [1]{#1}%
\providecommand \bibfnamefont [1]{#1}%
\providecommand \citenamefont [1]{#1}%
\providecommand \href@noop [0]{\@secondoftwo}%
\providecommand \href [0]{\begingroup \@sanitize@url \@href}%
\providecommand \@href[1]{\@@startlink{#1}\@@href}%
\providecommand \@@href[1]{\endgroup#1\@@endlink}%
\providecommand \@sanitize@url [0]{\catcode `\\12\catcode `\$12\catcode
  `\&12\catcode `\#12\catcode `\^12\catcode `\_12\catcode `\%12\relax}%
\providecommand \@@startlink[1]{}%
\providecommand \@@endlink[0]{}%
\providecommand \url  [0]{\begingroup\@sanitize@url \@url }%
\providecommand \@url [1]{\endgroup\@href {#1}{\urlprefix }}%
\providecommand \urlprefix  [0]{URL }%
\providecommand \Eprint [0]{\href }%
\providecommand \doibase [0]{https://doi.org/}%
\providecommand \selectlanguage [0]{\@gobble}%
\providecommand \bibinfo  [0]{\@secondoftwo}%
\providecommand \bibfield  [0]{\@secondoftwo}%
\providecommand \translation [1]{[#1]}%
\providecommand \BibitemOpen [0]{}%
\providecommand \bibitemStop [0]{}%
\providecommand \bibitemNoStop [0]{.\EOS\space}%
\providecommand \EOS [0]{\spacefactor3000\relax}%
\providecommand \BibitemShut  [1]{\csname bibitem#1\endcsname}%
\let\auto@bib@innerbib\@empty
%</preamble>
\bibitem [{\citenamefont {Goldsztein}\ \emph {et~al.}(2021)\citenamefont
  {Goldsztein}, \citenamefont {Nadeau},\ and\ \citenamefont
  {Strogatz}}]{10.1063/5.0026335}%
  \BibitemOpen
  \bibfield  {author} {\bibinfo {author} {\bibfnamefont {G.~H.}\ \bibnamefont
  {Goldsztein}}, \bibinfo {author} {\bibfnamefont {A.~N.}\ \bibnamefont
  {Nadeau}},\ and\ \bibinfo {author} {\bibfnamefont {S.~H.}\ \bibnamefont
  {Strogatz}},\ }\href {https://doi.org/10.1063/5.0026335} {\bibfield
  {journal} {\bibinfo  {journal} {Chaos: An Interdisciplinary Journal of
  Nonlinear Science}\ }\textbf {\bibinfo {volume} {31}},\ \bibinfo {pages}
  {023109} (\bibinfo {year} {2021})}\BibitemShut {NoStop}%
\bibitem [{\citenamefont {Buck}(1938)}]{Buck}%
  \BibitemOpen
  \bibfield  {author} {\bibinfo {author} {\bibfnamefont {J.~B.}\ \bibnamefont
  {Buck}},\ }\href {http://www.jstor.org/stable/2808377} {\bibfield  {journal}
  {\bibinfo  {journal} {The Quarterly Review of Biology}\ }\textbf {\bibinfo
  {volume} {13}},\ \bibinfo {pages} {301} (\bibinfo {year} {1938})}\BibitemShut
  {NoStop}%
\bibitem [{\citenamefont {Strogatz}\ \emph {et~al.}(2005)\citenamefont
  {Strogatz}, \citenamefont {Abrams}, \citenamefont {McRobie}, \citenamefont
  {Eckhardt},\ and\ \citenamefont {Ott}}]{bridge}%
  \BibitemOpen
  \bibfield  {author} {\bibinfo {author} {\bibfnamefont {S.~H.}\ \bibnamefont
  {Strogatz}}, \bibinfo {author} {\bibfnamefont {D.}~\bibnamefont {Abrams}},
  \bibinfo {author} {\bibfnamefont {F.}~\bibnamefont {McRobie}}, \bibinfo
  {author} {\bibfnamefont {B.}~\bibnamefont {Eckhardt}},\ and\ \bibinfo
  {author} {\bibfnamefont {E.}~\bibnamefont {Ott}},\ }\href
  {https://doi.org/10.1038/43843a} {\bibfield  {journal} {\bibinfo  {journal}
  {Nature}\ }\textbf {\bibinfo {volume} {438}},\ \bibinfo {pages} {43}
  (\bibinfo {year} {2005})}\BibitemShut {NoStop}%
\bibitem [{\citenamefont {Acebr{\'o}n}\ \emph {et~al.}(2005)\citenamefont
  {Acebr{\'o}n}, \citenamefont {Bonilla}, \citenamefont {P{\'e}rez~Vicente},
  \citenamefont {Ritort},\ and\ \citenamefont
  {Spigler}}]{acebronKuramotoModelSimple2005}%
  \BibitemOpen
  \bibfield  {author} {\bibinfo {author} {\bibfnamefont {J.~A.}\ \bibnamefont
  {Acebr{\'o}n}}, \bibinfo {author} {\bibfnamefont {L.~L.}\ \bibnamefont
  {Bonilla}}, \bibinfo {author} {\bibfnamefont {C.~J.}\ \bibnamefont
  {P{\'e}rez~Vicente}}, \bibinfo {author} {\bibfnamefont {F.}~\bibnamefont
  {Ritort}},\ and\ \bibinfo {author} {\bibfnamefont {R.}~\bibnamefont
  {Spigler}},\ }\href {https://doi.org/10.1103/RevModPhys.77.137} {\bibfield
  {journal} {\bibinfo  {journal} {Reviews of Modern Physics}\ }\textbf
  {\bibinfo {volume} {77}},\ \bibinfo {pages} {137} (\bibinfo {year}
  {2005})}\BibitemShut {NoStop}%
\bibitem [{\citenamefont {Pikovsky}\ \emph {et~al.}(2001)\citenamefont
  {Pikovsky}, \citenamefont {Rosenblum},\ and\ \citenamefont
  {Kurths}}]{pikovsky_rosenblum_kurths_2001}%
  \BibitemOpen
  \bibfield  {author} {\bibinfo {author} {\bibfnamefont {A.}~\bibnamefont
  {Pikovsky}}, \bibinfo {author} {\bibfnamefont {M.}~\bibnamefont
  {Rosenblum}},\ and\ \bibinfo {author} {\bibfnamefont {J.}~\bibnamefont
  {Kurths}},\ }\href {https://doi.org/10.1017/CBO9780511755743} {\emph
  {\bibinfo {title} {Synchronization: A Universal Concept in Nonlinear
  Sciences}}},\ Cambridge Nonlinear Science Series\ (\bibinfo  {publisher}
  {Cambridge University Press},\ \bibinfo {year} {2001})\BibitemShut {NoStop}%
\bibitem [{\citenamefont {Strogatz}(2003)}]{sync}%
  \BibitemOpen
  \bibfield  {author} {\bibinfo {author} {\bibfnamefont {S.~H.}\ \bibnamefont
  {Strogatz}},\ }\href@noop {} {\emph {\bibinfo {title} {Sync: The Emerging
  Science of Spontaneous Order}}}\ (\bibinfo  {publisher} {Hyperion},\ \bibinfo
  {year} {2003})\BibitemShut {NoStop}%
\bibitem [{\citenamefont {Zhirov}\ and\ \citenamefont
  {Shepelyansky}(2006)}]{Shepelyansky2006}%
  \BibitemOpen
  \bibfield  {author} {\bibinfo {author} {\bibfnamefont {O.~V.}\ \bibnamefont
  {Zhirov}}\ and\ \bibinfo {author} {\bibfnamefont {D.~L.}\ \bibnamefont
  {Shepelyansky}},\ }\href {https://doi.org/10.1140/epjd/e2006-00011-9}
  {\bibfield  {journal} {\bibinfo  {journal} {E. Phys. J. D}\ }\textbf
  {\bibinfo {volume} {38}},\ \bibinfo {pages} {375} (\bibinfo {year}
  {2006})}\BibitemShut {NoStop}%
\bibitem [{\citenamefont {Chia}\ \emph {et~al.}(2020)\citenamefont {Chia},
  \citenamefont {Kwek},\ and\ \citenamefont {Noh}}]{Chia2020}%
  \BibitemOpen
  \bibfield  {author} {\bibinfo {author} {\bibfnamefont {A.}~\bibnamefont
  {Chia}}, \bibinfo {author} {\bibfnamefont {L.~C.}\ \bibnamefont {Kwek}},\
  and\ \bibinfo {author} {\bibfnamefont {C.}~\bibnamefont {Noh}},\ }\href
  {https://doi.org/10.1103/PhysRevE.102.042213} {\bibfield  {journal} {\bibinfo
   {journal} {Phys. Rev. E}\ }\textbf {\bibinfo {volume} {102}},\ \bibinfo
  {pages} {042213} (\bibinfo {year} {2020})}\BibitemShut {NoStop}%
\bibitem [{\citenamefont {Ben~Arosh}\ \emph {et~al.}(2021)\citenamefont
  {Ben~Arosh}, \citenamefont {Cross},\ and\ \citenamefont
  {Lifshitz}}]{Lifshits2021}%
  \BibitemOpen
  \bibfield  {author} {\bibinfo {author} {\bibfnamefont {L.}~\bibnamefont
  {Ben~Arosh}}, \bibinfo {author} {\bibfnamefont {M.~C.}\ \bibnamefont
  {Cross}},\ and\ \bibinfo {author} {\bibfnamefont {R.}~\bibnamefont
  {Lifshitz}},\ }\href {https://doi.org/10.1103/PhysRevResearch.3.013130}
  {\bibfield  {journal} {\bibinfo  {journal} {Phys. Rev. Res.}\ }\textbf
  {\bibinfo {volume} {3}},\ \bibinfo {pages} {013130} (\bibinfo {year}
  {2021})}\BibitemShut {NoStop}%
\bibitem [{\citenamefont {Ludwig}\ and\ \citenamefont
  {Marquardt}(2013)}]{ludwig_marquardt2013}%
  \BibitemOpen
  \bibfield  {author} {\bibinfo {author} {\bibfnamefont {M.}~\bibnamefont
  {Ludwig}}\ and\ \bibinfo {author} {\bibfnamefont {F.}~\bibnamefont
  {Marquardt}},\ }\href {https://doi.org/10.1103/PhysRevLett.111.073603}
  {\bibfield  {journal} {\bibinfo  {journal} {Phys. Rev. Lett.}\ }\textbf
  {\bibinfo {volume} {111}},\ \bibinfo {pages} {073603} (\bibinfo {year}
  {2013})}\BibitemShut {NoStop}%
\bibitem [{\citenamefont {Lee}\ and\ \citenamefont
  {Sadeghpour}(2013)}]{lee_QuantumSynchronizationQuantum_2013}%
  \BibitemOpen
  \bibfield  {author} {\bibinfo {author} {\bibfnamefont {T.~E.}\ \bibnamefont
  {Lee}}\ and\ \bibinfo {author} {\bibfnamefont {H.~R.}\ \bibnamefont
  {Sadeghpour}},\ }\href {https://doi.org/10.1103/PhysRevLett.111.234101}
  {\bibfield  {journal} {\bibinfo  {journal} {Physical Review Letters}\
  }\textbf {\bibinfo {volume} {111}},\ \bibinfo {pages} {234101} (\bibinfo
  {year} {2013})}\BibitemShut {NoStop}%
\bibitem [{\citenamefont {Walter}\ \emph {et~al.}(2014)\citenamefont {Walter},
  \citenamefont {Nunnenkamp},\ and\ \citenamefont
  {Bruder}}]{walterQuantumSynchronizationDriven2014}%
  \BibitemOpen
  \bibfield  {author} {\bibinfo {author} {\bibfnamefont {S.}~\bibnamefont
  {Walter}}, \bibinfo {author} {\bibfnamefont {A.}~\bibnamefont {Nunnenkamp}},\
  and\ \bibinfo {author} {\bibfnamefont {C.}~\bibnamefont {Bruder}},\ }\href
  {https://doi.org/10.1103/PhysRevLett.112.094102} {\bibfield  {journal}
  {\bibinfo  {journal} {Physical Review Letters}\ }\textbf {\bibinfo {volume}
  {112}},\ \bibinfo {pages} {094102} (\bibinfo {year} {2014})}\BibitemShut
  {NoStop}%
\bibitem [{\citenamefont {Roulet}\ and\ \citenamefont
  {Bruder}(2018{\natexlab{a}})}]{rouletSynchronizingSmallestPossible2018}%
  \BibitemOpen
  \bibfield  {author} {\bibinfo {author} {\bibfnamefont {A.}~\bibnamefont
  {Roulet}}\ and\ \bibinfo {author} {\bibfnamefont {C.}~\bibnamefont
  {Bruder}},\ }\href {https://doi.org/10.1103/PhysRevLett.121.053601}
  {\bibfield  {journal} {\bibinfo  {journal} {Physical Review Letters}\
  }\textbf {\bibinfo {volume} {121}},\ \bibinfo {pages} {053601} (\bibinfo
  {year} {2018}{\natexlab{a}})}\BibitemShut {NoStop}%
\bibitem [{\citenamefont {{Parra-L{\'o}pez}}\ and\ \citenamefont
  {Bergli}(2020)}]{parra-lopezSynchronizationTwolevelQuantum2020}%
  \BibitemOpen
  \bibfield  {author} {\bibinfo {author} {\bibfnamefont {{\'A}.}~\bibnamefont
  {{Parra-L{\'o}pez}}}\ and\ \bibinfo {author} {\bibfnamefont {J.}~\bibnamefont
  {Bergli}},\ }\href {https://doi.org/10.1103/PhysRevA.101.062104} {\bibfield
  {journal} {\bibinfo  {journal} {Physical Review A}\ }\textbf {\bibinfo
  {volume} {101}},\ \bibinfo {pages} {062104} (\bibinfo {year}
  {2020})}\BibitemShut {NoStop}%
\bibitem [{\citenamefont {Laskar}\ \emph {et~al.}(2020)\citenamefont {Laskar},
  \citenamefont {Adhikary}, \citenamefont {Mondal}, \citenamefont {Katiyar},
  \citenamefont {Vinjanampathy},\ and\ \citenamefont
  {Ghosh}}]{PhysRevLett.125.013601}%
  \BibitemOpen
  \bibfield  {author} {\bibinfo {author} {\bibfnamefont {A.~W.}\ \bibnamefont
  {Laskar}}, \bibinfo {author} {\bibfnamefont {P.}~\bibnamefont {Adhikary}},
  \bibinfo {author} {\bibfnamefont {S.}~\bibnamefont {Mondal}}, \bibinfo
  {author} {\bibfnamefont {P.}~\bibnamefont {Katiyar}}, \bibinfo {author}
  {\bibfnamefont {S.}~\bibnamefont {Vinjanampathy}},\ and\ \bibinfo {author}
  {\bibfnamefont {S.}~\bibnamefont {Ghosh}},\ }\href
  {https://doi.org/10.1103/PhysRevLett.125.013601} {\bibfield  {journal}
  {\bibinfo  {journal} {Phys. Rev. Lett.}\ }\textbf {\bibinfo {volume} {125}},\
  \bibinfo {pages} {013601} (\bibinfo {year} {2020})}\BibitemShut {NoStop}%
\bibitem [{\citenamefont {Krithika}\ \emph {et~al.}(2022)\citenamefont
  {Krithika}, \citenamefont {Solanki}, \citenamefont {Vinjanampathy},\ and\
  \citenamefont {Mahesh}}]{PhysRevA.105.062206}%
  \BibitemOpen
  \bibfield  {author} {\bibinfo {author} {\bibfnamefont {V.~R.}\ \bibnamefont
  {Krithika}}, \bibinfo {author} {\bibfnamefont {P.}~\bibnamefont {Solanki}},
  \bibinfo {author} {\bibfnamefont {S.}~\bibnamefont {Vinjanampathy}},\ and\
  \bibinfo {author} {\bibfnamefont {T.~S.}\ \bibnamefont {Mahesh}},\ }\href
  {https://doi.org/10.1103/PhysRevA.105.062206} {\bibfield  {journal} {\bibinfo
   {journal} {Phys. Rev. A}\ }\textbf {\bibinfo {volume} {105}},\ \bibinfo
  {pages} {062206} (\bibinfo {year} {2022})}\BibitemShut {NoStop}%
\bibitem [{\citenamefont {Zhang}\ \emph {et~al.}(2023)\citenamefont {Zhang},
  \citenamefont {Wang}, \citenamefont {Wang}, \citenamefont {Zhang},
  \citenamefont {Wu}, \citenamefont {Jie},\ and\ \citenamefont
  {Lu}}]{zhang2023observing}%
  \BibitemOpen
  \bibfield  {author} {\bibinfo {author} {\bibfnamefont {L.}~\bibnamefont
  {Zhang}}, \bibinfo {author} {\bibfnamefont {Z.}~\bibnamefont {Wang}},
  \bibinfo {author} {\bibfnamefont {Y.}~\bibnamefont {Wang}}, \bibinfo {author}
  {\bibfnamefont {J.}~\bibnamefont {Zhang}}, \bibinfo {author} {\bibfnamefont
  {Z.}~\bibnamefont {Wu}}, \bibinfo {author} {\bibfnamefont {J.}~\bibnamefont
  {Jie}},\ and\ \bibinfo {author} {\bibfnamefont {Y.}~\bibnamefont {Lu}},\
  }\href@noop {} {\bibinfo {title} {Observing quantum synchronization of a
  single trapped-ion qubit}} (\bibinfo {year} {2023}),\ \Eprint
  {https://arxiv.org/abs/2205.05936} {arXiv:2205.05936} \BibitemShut {NoStop}%
\bibitem [{\citenamefont {Koppenh{\"o}fer}\ \emph {et~al.}(2020)\citenamefont
  {Koppenh{\"o}fer}, \citenamefont {Bruder},\ and\ \citenamefont
  {Roulet}}]{koppenhoferQuantumSynchronizationIBM2020}%
  \BibitemOpen
  \bibfield  {author} {\bibinfo {author} {\bibfnamefont {M.}~\bibnamefont
  {Koppenh{\"o}fer}}, \bibinfo {author} {\bibfnamefont {C.}~\bibnamefont
  {Bruder}},\ and\ \bibinfo {author} {\bibfnamefont {A.}~\bibnamefont
  {Roulet}},\ }\href {https://doi.org/10.1103/PhysRevResearch.2.023026}
  {\bibfield  {journal} {\bibinfo  {journal} {Physical Review Research}\
  }\textbf {\bibinfo {volume} {2}},\ \bibinfo {pages} {023026} (\bibinfo {year}
  {2020})}\BibitemShut {NoStop}%
\bibitem [{\citenamefont {Xu}\ \emph {et~al.}(2014)\citenamefont {Xu},
  \citenamefont {Tieri}, \citenamefont {Fine}, \citenamefont {Thompson},\ and\
  \citenamefont {Holland}}]{Holland2014}%
  \BibitemOpen
  \bibfield  {author} {\bibinfo {author} {\bibfnamefont {M.}~\bibnamefont
  {Xu}}, \bibinfo {author} {\bibfnamefont {D.~A.}\ \bibnamefont {Tieri}},
  \bibinfo {author} {\bibfnamefont {E.~C.}\ \bibnamefont {Fine}}, \bibinfo
  {author} {\bibfnamefont {J.~K.}\ \bibnamefont {Thompson}},\ and\ \bibinfo
  {author} {\bibfnamefont {M.~J.}\ \bibnamefont {Holland}},\ }\href
  {https://doi.org/10.1103/PhysRevLett.113.154101} {\bibfield  {journal}
  {\bibinfo  {journal} {Phys. Rev. Lett.}\ }\textbf {\bibinfo {volume} {113}},\
  \bibinfo {pages} {154101} (\bibinfo {year} {2014})}\BibitemShut {NoStop}%
\bibitem [{\citenamefont {Zhu}\ \emph {et~al.}(2015)\citenamefont {Zhu},
  \citenamefont {Schachenmayer}, \citenamefont {Xu}, \citenamefont {Herrera},
  \citenamefont {Restrepo}, \citenamefont {Holland},\ and\ \citenamefont
  {Rey}}]{zhuSynchronizationInteractingQuantum2015}%
  \BibitemOpen
  \bibfield  {author} {\bibinfo {author} {\bibfnamefont {B.}~\bibnamefont
  {Zhu}}, \bibinfo {author} {\bibfnamefont {J.}~\bibnamefont {Schachenmayer}},
  \bibinfo {author} {\bibfnamefont {M.}~\bibnamefont {Xu}}, \bibinfo {author}
  {\bibfnamefont {F.}~\bibnamefont {Herrera}}, \bibinfo {author} {\bibfnamefont
  {J.~G.}\ \bibnamefont {Restrepo}}, \bibinfo {author} {\bibfnamefont {M.~J.}\
  \bibnamefont {Holland}},\ and\ \bibinfo {author} {\bibfnamefont {A.~M.}\
  \bibnamefont {Rey}},\ }\href {https://doi.org/10.1088/1367-2630/17/8/083063}
  {\bibfield  {journal} {\bibinfo  {journal} {New Journal of Physics}\ }\textbf
  {\bibinfo {volume} {17}},\ \bibinfo {pages} {083063} (\bibinfo {year}
  {2015})}\BibitemShut {NoStop}%
\bibitem [{\citenamefont {Mari}\ \emph {et~al.}(2013)\citenamefont {Mari},
  \citenamefont {Farace}, \citenamefont {Didier}, \citenamefont {Giovannetti},\
  and\ \citenamefont {Fazio}}]{Fazio2013}%
  \BibitemOpen
  \bibfield  {author} {\bibinfo {author} {\bibfnamefont {A.}~\bibnamefont
  {Mari}}, \bibinfo {author} {\bibfnamefont {A.}~\bibnamefont {Farace}},
  \bibinfo {author} {\bibfnamefont {N.}~\bibnamefont {Didier}}, \bibinfo
  {author} {\bibfnamefont {V.}~\bibnamefont {Giovannetti}},\ and\ \bibinfo
  {author} {\bibfnamefont {R.}~\bibnamefont {Fazio}},\ }\href
  {https://doi.org/10.1103/PhysRevLett.111.103605} {\bibfield  {journal}
  {\bibinfo  {journal} {Phys. Rev. Lett.}\ }\textbf {\bibinfo {volume} {111}},\
  \bibinfo {pages} {103605} (\bibinfo {year} {2013})}\BibitemShut {NoStop}%
\bibitem [{\citenamefont {Davis-Tilley}\ \emph {et~al.}(2018)\citenamefont
  {Davis-Tilley}, \citenamefont {Teoh},\ and\ \citenamefont
  {Armour}}]{Davis-Tilley_2018}%
  \BibitemOpen
  \bibfield  {author} {\bibinfo {author} {\bibfnamefont {C.}~\bibnamefont
  {Davis-Tilley}}, \bibinfo {author} {\bibfnamefont {C.~K.}\ \bibnamefont
  {Teoh}},\ and\ \bibinfo {author} {\bibfnamefont {A.~D.}\ \bibnamefont
  {Armour}},\ }\href {https://doi.org/10.1088/1367-2630/aae947} {\bibfield
  {journal} {\bibinfo  {journal} {New Journal of Physics}\ }\textbf {\bibinfo
  {volume} {20}},\ \bibinfo {pages} {113002} (\bibinfo {year}
  {2018})}\BibitemShut {NoStop}%
\bibitem [{\citenamefont {Bandyopadhyay}\ and\ \citenamefont
  {Banerjee}(2023)}]{PhysRevE.107.024204}%
  \BibitemOpen
  \bibfield  {author} {\bibinfo {author} {\bibfnamefont {B.}~\bibnamefont
  {Bandyopadhyay}}\ and\ \bibinfo {author} {\bibfnamefont {T.}~\bibnamefont
  {Banerjee}},\ }\href {https://doi.org/10.1103/PhysRevE.107.024204} {\bibfield
   {journal} {\bibinfo  {journal} {Phys. Rev. E}\ }\textbf {\bibinfo {volume}
  {107}},\ \bibinfo {pages} {024204} (\bibinfo {year} {2023})}\BibitemShut
  {NoStop}%
\bibitem [{\citenamefont {Witthaut}\ \emph {et~al.}(2017)\citenamefont
  {Witthaut}, \citenamefont {Wimberger}, \citenamefont {Burioni},\ and\
  \citenamefont {Timme}}]{Witthaut2017}%
  \BibitemOpen
  \bibfield  {author} {\bibinfo {author} {\bibfnamefont {D.}~\bibnamefont
  {Witthaut}}, \bibinfo {author} {\bibfnamefont {S.}~\bibnamefont {Wimberger}},
  \bibinfo {author} {\bibfnamefont {R.}~\bibnamefont {Burioni}},\ and\ \bibinfo
  {author} {\bibfnamefont {M.}~\bibnamefont {Timme}},\ }\href
  {https://doi.org/10.1038/ncomms14829} {\bibfield  {journal} {\bibinfo
  {journal} {Nature Comm.}\ }\textbf {\bibinfo {volume} {8}},\ \bibinfo {pages}
  {14829} (\bibinfo {year} {2017})}\BibitemShut {NoStop}%
\bibitem [{\citenamefont {Lorenzo}\ \emph {et~al.}(2022)\citenamefont
  {Lorenzo}, \citenamefont {Militello}, \citenamefont {Napoli}, \citenamefont
  {Zambrini},\ and\ \citenamefont
  {Palma}}]{lorenzoQuantumSynchronisationClustering2021}%
  \BibitemOpen
  \bibfield  {author} {\bibinfo {author} {\bibfnamefont {S.}~\bibnamefont
  {Lorenzo}}, \bibinfo {author} {\bibfnamefont {B.}~\bibnamefont {Militello}},
  \bibinfo {author} {\bibfnamefont {A.}~\bibnamefont {Napoli}}, \bibinfo
  {author} {\bibfnamefont {R.}~\bibnamefont {Zambrini}},\ and\ \bibinfo
  {author} {\bibfnamefont {G.~M.}\ \bibnamefont {Palma}},\ }\href
  {https://doi.org/10.1088/1367-2630/ac51a9} {\bibfield  {journal} {\bibinfo
  {journal} {New J. of Phys.}\ }\textbf {\bibinfo {volume} {24}},\ \bibinfo
  {pages} {023030} (\bibinfo {year} {2022})}\BibitemShut {NoStop}%
\bibitem [{\citenamefont {W\"achtler}\ and\ \citenamefont
  {Platero}(2023)}]{Waechtler2023}%
  \BibitemOpen
  \bibfield  {author} {\bibinfo {author} {\bibfnamefont {C.~W.}\ \bibnamefont
  {W\"achtler}}\ and\ \bibinfo {author} {\bibfnamefont {G.}~\bibnamefont
  {Platero}},\ }\href {https://doi.org/10.1103/PhysRevResearch.5.023021}
  {\bibfield  {journal} {\bibinfo  {journal} {Phys. Rev. Research}\ }\textbf
  {\bibinfo {volume} {5}},\ \bibinfo {pages} {023021} (\bibinfo {year}
  {2023})}\BibitemShut {NoStop}%
\bibitem [{\citenamefont {Li}\ \emph {et~al.}(2017{\natexlab{a}})\citenamefont
  {Li}, \citenamefont {Li},\ and\ \citenamefont {Song}}]{SongHeshan2017b}%
  \BibitemOpen
  \bibfield  {author} {\bibinfo {author} {\bibfnamefont {W.}~\bibnamefont
  {Li}}, \bibinfo {author} {\bibfnamefont {C.}~\bibnamefont {Li}},\ and\
  \bibinfo {author} {\bibfnamefont {H.}~\bibnamefont {Song}},\ }\href
  {https://doi.org/10.1103/PhysRevE.95.022204} {\bibfield  {journal} {\bibinfo
  {journal} {Phys. Rev. E}\ }\textbf {\bibinfo {volume} {95}},\ \bibinfo
  {pages} {022204} (\bibinfo {year} {2017}{\natexlab{a}})}\BibitemShut
  {NoStop}%
\bibitem [{\citenamefont {Ishibashi}\ and\ \citenamefont
  {Kanamoto}(2017)}]{ishibashiOscillationCollapseCoupled2017}%
  \BibitemOpen
  \bibfield  {author} {\bibinfo {author} {\bibfnamefont {K.}~\bibnamefont
  {Ishibashi}}\ and\ \bibinfo {author} {\bibfnamefont {R.}~\bibnamefont
  {Kanamoto}},\ }\href {https://doi.org/10.1103/PhysRevE.96.052210} {\bibfield
  {journal} {\bibinfo  {journal} {Physical Review E}\ }\textbf {\bibinfo
  {volume} {96}},\ \bibinfo {pages} {052210} (\bibinfo {year}
  {2017})}\BibitemShut {NoStop}%
\bibitem [{\citenamefont {L{\"o}rch}\ \emph {et~al.}(2016)\citenamefont
  {L{\"o}rch}, \citenamefont {Amitai}, \citenamefont {Nunnenkamp},\ and\
  \citenamefont {Bruder}}]{lorchGenuineQuantumSignatures2016}%
  \BibitemOpen
  \bibfield  {author} {\bibinfo {author} {\bibfnamefont {N.}~\bibnamefont
  {L{\"o}rch}}, \bibinfo {author} {\bibfnamefont {E.}~\bibnamefont {Amitai}},
  \bibinfo {author} {\bibfnamefont {A.}~\bibnamefont {Nunnenkamp}},\ and\
  \bibinfo {author} {\bibfnamefont {C.}~\bibnamefont {Bruder}},\ }\href
  {https://doi.org/10.1103/PhysRevLett.117.073601} {\bibfield  {journal}
  {\bibinfo  {journal} {Physical Review Letters}\ }\textbf {\bibinfo {volume}
  {117}},\ \bibinfo {pages} {073601} (\bibinfo {year} {2016})}\BibitemShut
  {NoStop}%
\bibitem [{\citenamefont {L{\"o}rch}\ \emph {et~al.}(2017)\citenamefont
  {L{\"o}rch}, \citenamefont {Nigg}, \citenamefont {Nunnenkamp}, \citenamefont
  {Tiwari},\ and\ \citenamefont
  {Bruder}}]{lorchQuantumSynchronizationBlockade2017}%
  \BibitemOpen
  \bibfield  {author} {\bibinfo {author} {\bibfnamefont {N.}~\bibnamefont
  {L{\"o}rch}}, \bibinfo {author} {\bibfnamefont {S.~E.}\ \bibnamefont {Nigg}},
  \bibinfo {author} {\bibfnamefont {A.}~\bibnamefont {Nunnenkamp}}, \bibinfo
  {author} {\bibfnamefont {R.~P.}\ \bibnamefont {Tiwari}},\ and\ \bibinfo
  {author} {\bibfnamefont {C.}~\bibnamefont {Bruder}},\ }\href
  {https://doi.org/10.1103/PhysRevLett.118.243602} {\bibfield  {journal}
  {\bibinfo  {journal} {Physical Review Letters}\ }\textbf {\bibinfo {volume}
  {118}},\ \bibinfo {pages} {243602} (\bibinfo {year} {2017})}\BibitemShut
  {NoStop}%
\bibitem [{\citenamefont {Dutta}\ and\ \citenamefont
  {Cooper}(2019)}]{Cooper2019}%
  \BibitemOpen
  \bibfield  {author} {\bibinfo {author} {\bibfnamefont {S.}~\bibnamefont
  {Dutta}}\ and\ \bibinfo {author} {\bibfnamefont {N.~R.}\ \bibnamefont
  {Cooper}},\ }\href {https://doi.org/10.1103/PhysRevLett.123.250401}
  {\bibfield  {journal} {\bibinfo  {journal} {Phys. Rev. Lett.}\ }\textbf
  {\bibinfo {volume} {123}},\ \bibinfo {pages} {250401} (\bibinfo {year}
  {2019})}\BibitemShut {NoStop}%
\bibitem [{\citenamefont {Eneriz}\ \emph {et~al.}(2019)\citenamefont {Eneriz},
  \citenamefont {Rossatto}, \citenamefont {Cardenas-Lopez}, \citenamefont
  {Solano},\ and\ \citenamefont {Sanz}}]{Solano2019a}%
  \BibitemOpen
  \bibfield  {author} {\bibinfo {author} {\bibfnamefont {H.}~\bibnamefont
  {Eneriz}}, \bibinfo {author} {\bibfnamefont {D.~Z.}\ \bibnamefont
  {Rossatto}}, \bibinfo {author} {\bibfnamefont {F.~A.}\ \bibnamefont
  {Cardenas-Lopez}}, \bibinfo {author} {\bibfnamefont {E.}~\bibnamefont
  {Solano}},\ and\ \bibinfo {author} {\bibfnamefont {M.}~\bibnamefont {Sanz}},\
  }\href {https://doi.org/10.1038/s41598-019-56468-x} {\bibfield  {journal}
  {\bibinfo  {journal} {Scientific Reports}\ }\textbf {\bibinfo {volume} {9}},\
  \bibinfo {pages} {19933} (\bibinfo {year} {2019})}\BibitemShut {NoStop}%
\bibitem [{\citenamefont {Shen}\ \emph {et~al.}(2023)\citenamefont {Shen},
  \citenamefont {Mok}, \citenamefont {Noh}, \citenamefont {Liu}, \citenamefont
  {Kwek}, \citenamefont {Fan},\ and\ \citenamefont {Chia}}]{Chia2023}%
  \BibitemOpen
  \bibfield  {author} {\bibinfo {author} {\bibfnamefont {Y.}~\bibnamefont
  {Shen}}, \bibinfo {author} {\bibfnamefont {W.-K.}\ \bibnamefont {Mok}},
  \bibinfo {author} {\bibfnamefont {C.}~\bibnamefont {Noh}}, \bibinfo {author}
  {\bibfnamefont {A.~Q.}\ \bibnamefont {Liu}}, \bibinfo {author} {\bibfnamefont
  {L.-C.}\ \bibnamefont {Kwek}}, \bibinfo {author} {\bibfnamefont
  {W.}~\bibnamefont {Fan}},\ and\ \bibinfo {author} {\bibfnamefont
  {A.}~\bibnamefont {Chia}},\ }\href
  {https://doi.org/10.1103/PhysRevA.107.053713} {\bibfield  {journal} {\bibinfo
   {journal} {Phys. Rev. A}\ }\textbf {\bibinfo {volume} {107}},\ \bibinfo
  {pages} {053713} (\bibinfo {year} {2023})}\BibitemShut {NoStop}%
\bibitem [{\citenamefont {Giorgi}\ \emph {et~al.}(2012)\citenamefont {Giorgi},
  \citenamefont {Galve}, \citenamefont {Manzano}, \citenamefont {Colet},\ and\
  \citenamefont {Zambrini}}]{Zambrini2012}%
  \BibitemOpen
  \bibfield  {author} {\bibinfo {author} {\bibfnamefont {G.~L.}\ \bibnamefont
  {Giorgi}}, \bibinfo {author} {\bibfnamefont {F.}~\bibnamefont {Galve}},
  \bibinfo {author} {\bibfnamefont {G.}~\bibnamefont {Manzano}}, \bibinfo
  {author} {\bibfnamefont {P.}~\bibnamefont {Colet}},\ and\ \bibinfo {author}
  {\bibfnamefont {R.}~\bibnamefont {Zambrini}},\ }\href
  {https://doi.org/10.1103/PhysRevA.85.052101} {\bibfield  {journal} {\bibinfo
  {journal} {Phys. Rev. A}\ }\textbf {\bibinfo {volume} {85}},\ \bibinfo
  {pages} {052101} (\bibinfo {year} {2012})}\BibitemShut {NoStop}%
\bibitem [{\citenamefont {Ameri}\ \emph {et~al.}(2015)\citenamefont {Ameri},
  \citenamefont {Eghbali-Arani}, \citenamefont {Mari}, \citenamefont {Farace},
  \citenamefont {Kheirandish}, \citenamefont {Giovannetti},\ and\ \citenamefont
  {Fazio}}]{Ameri2015}%
  \BibitemOpen
  \bibfield  {author} {\bibinfo {author} {\bibfnamefont {V.}~\bibnamefont
  {Ameri}}, \bibinfo {author} {\bibfnamefont {M.}~\bibnamefont
  {Eghbali-Arani}}, \bibinfo {author} {\bibfnamefont {A.}~\bibnamefont {Mari}},
  \bibinfo {author} {\bibfnamefont {A.}~\bibnamefont {Farace}}, \bibinfo
  {author} {\bibfnamefont {F.}~\bibnamefont {Kheirandish}}, \bibinfo {author}
  {\bibfnamefont {V.}~\bibnamefont {Giovannetti}},\ and\ \bibinfo {author}
  {\bibfnamefont {R.}~\bibnamefont {Fazio}},\ }\href
  {https://doi.org/10.1103/PhysRevA.91.012301} {\bibfield  {journal} {\bibinfo
  {journal} {Phys. Rev. A}\ }\textbf {\bibinfo {volume} {91}},\ \bibinfo
  {pages} {012301} (\bibinfo {year} {2015})}\BibitemShut {NoStop}%
\bibitem [{\citenamefont {Hush}\ \emph {et~al.}(2015)\citenamefont {Hush},
  \citenamefont {Li}, \citenamefont {Genway}, \citenamefont {Lesanovsky},\ and\
  \citenamefont {Armour}}]{Armour2015}%
  \BibitemOpen
  \bibfield  {author} {\bibinfo {author} {\bibfnamefont {M.~R.}\ \bibnamefont
  {Hush}}, \bibinfo {author} {\bibfnamefont {W.}~\bibnamefont {Li}}, \bibinfo
  {author} {\bibfnamefont {S.}~\bibnamefont {Genway}}, \bibinfo {author}
  {\bibfnamefont {I.}~\bibnamefont {Lesanovsky}},\ and\ \bibinfo {author}
  {\bibfnamefont {A.~D.}\ \bibnamefont {Armour}},\ }\href
  {https://doi.org/10.1103/PhysRevA.91.061401} {\bibfield  {journal} {\bibinfo
  {journal} {Phys. Rev. A}\ }\textbf {\bibinfo {volume} {91}},\ \bibinfo
  {pages} {061401(R)} (\bibinfo {year} {2015})}\BibitemShut {NoStop}%
\bibitem [{\citenamefont {Bastidas}\ \emph {et~al.}(2015)\citenamefont
  {Bastidas}, \citenamefont {Omelchenko}, \citenamefont {Zakharova},
  \citenamefont {Sch\"oll},\ and\ \citenamefont {Brandes}}]{Bastidas2015}%
  \BibitemOpen
  \bibfield  {author} {\bibinfo {author} {\bibfnamefont {V.~M.}\ \bibnamefont
  {Bastidas}}, \bibinfo {author} {\bibfnamefont {I.}~\bibnamefont
  {Omelchenko}}, \bibinfo {author} {\bibfnamefont {A.}~\bibnamefont
  {Zakharova}}, \bibinfo {author} {\bibfnamefont {E.}~\bibnamefont
  {Sch\"oll}},\ and\ \bibinfo {author} {\bibfnamefont {T.}~\bibnamefont
  {Brandes}},\ }\href {https://doi.org/10.1103/PhysRevE.92.062924} {\bibfield
  {journal} {\bibinfo  {journal} {Phys. Rev. E}\ }\textbf {\bibinfo {volume}
  {92}},\ \bibinfo {pages} {062924} (\bibinfo {year} {2015})}\BibitemShut
  {NoStop}%
\bibitem [{\citenamefont {Davis-Tilley}\ and\ \citenamefont
  {Armour}(2016)}]{Armour2016}%
  \BibitemOpen
  \bibfield  {author} {\bibinfo {author} {\bibfnamefont {C.}~\bibnamefont
  {Davis-Tilley}}\ and\ \bibinfo {author} {\bibfnamefont {A.~D.}\ \bibnamefont
  {Armour}},\ }\href {https://doi.org/10.1103/PhysRevA.94.063819} {\bibfield
  {journal} {\bibinfo  {journal} {Phys. Rev. A}\ }\textbf {\bibinfo {volume}
  {94}},\ \bibinfo {pages} {063819} (\bibinfo {year} {2016})}\BibitemShut
  {NoStop}%
\bibitem [{\citenamefont {Weiss}\ \emph {et~al.}(2016)\citenamefont {Weiss},
  \citenamefont {Kronwald},\ and\ \citenamefont {Marquardt}}]{Weiss_2016}%
  \BibitemOpen
  \bibfield  {author} {\bibinfo {author} {\bibfnamefont {T.}~\bibnamefont
  {Weiss}}, \bibinfo {author} {\bibfnamefont {A.}~\bibnamefont {Kronwald}},\
  and\ \bibinfo {author} {\bibfnamefont {F.}~\bibnamefont {Marquardt}},\ }\href
  {https://doi.org/10.1088/1367-2630/18/1/013043} {\bibfield  {journal}
  {\bibinfo  {journal} {New Journal of Physics}\ }\textbf {\bibinfo {volume}
  {18}},\ \bibinfo {pages} {013043} (\bibinfo {year} {2016})}\BibitemShut
  {NoStop}%
\bibitem [{\citenamefont {Li}\ \emph {et~al.}(2017{\natexlab{b}})\citenamefont
  {Li}, \citenamefont {Zhang}, \citenamefont {Li},\ and\ \citenamefont
  {Song}}]{SongHeshan2017a}%
  \BibitemOpen
  \bibfield  {author} {\bibinfo {author} {\bibfnamefont {W.}~\bibnamefont
  {Li}}, \bibinfo {author} {\bibfnamefont {W.}~\bibnamefont {Zhang}}, \bibinfo
  {author} {\bibfnamefont {C.}~\bibnamefont {Li}},\ and\ \bibinfo {author}
  {\bibfnamefont {H.}~\bibnamefont {Song}},\ }\href
  {https://doi.org/10.1103/PhysRevE.96.012211} {\bibfield  {journal} {\bibinfo
  {journal} {Phys. Rev. E}\ }\textbf {\bibinfo {volume} {96}},\ \bibinfo
  {pages} {012211} (\bibinfo {year} {2017}{\natexlab{b}})}\BibitemShut
  {NoStop}%
\bibitem [{\citenamefont {Weiss}\ \emph {et~al.}(2017)\citenamefont {Weiss},
  \citenamefont {Walter},\ and\ \citenamefont {Marquardt}}]{Talitha2017}%
  \BibitemOpen
  \bibfield  {author} {\bibinfo {author} {\bibfnamefont {T.}~\bibnamefont
  {Weiss}}, \bibinfo {author} {\bibfnamefont {S.}~\bibnamefont {Walter}},\ and\
  \bibinfo {author} {\bibfnamefont {F.}~\bibnamefont {Marquardt}},\ }\href
  {https://doi.org/10.1103/PhysRevA.95.041802} {\bibfield  {journal} {\bibinfo
  {journal} {Phys. Rev. A}\ }\textbf {\bibinfo {volume} {95}},\ \bibinfo
  {pages} {041802(R)} (\bibinfo {year} {2017})}\BibitemShut {NoStop}%
\bibitem [{\citenamefont {Du}\ \emph {et~al.}(2017)\citenamefont {Du},
  \citenamefont {Fan}, \citenamefont {Zhang},\ and\ \citenamefont
  {Wu}}]{WuJin-Hui2017}%
  \BibitemOpen
  \bibfield  {author} {\bibinfo {author} {\bibfnamefont {L.}~\bibnamefont
  {Du}}, \bibinfo {author} {\bibfnamefont {C.-H.}\ \bibnamefont {Fan}},
  \bibinfo {author} {\bibfnamefont {H.-X.}\ \bibnamefont {Zhang}},\ and\
  \bibinfo {author} {\bibfnamefont {J.-H.}\ \bibnamefont {Wu}},\ }\href
  {https://doi.org/10.1038/s41598-017-16115-9} {\bibfield  {journal} {\bibinfo
  {journal} {Scientific Reports}\ }\textbf {\bibinfo {volume} {7}},\ \bibinfo
  {pages} {15834} (\bibinfo {year} {2017})}\BibitemShut {NoStop}%
\bibitem [{\citenamefont {Amitai}\ \emph {et~al.}(2018)\citenamefont {Amitai},
  \citenamefont {Koppenh{\"o}fer}, \citenamefont {L{\"o}rch},\ and\
  \citenamefont {Bruder}}]{amitaiQuantumEffectsAmplitude2018}%
  \BibitemOpen
  \bibfield  {author} {\bibinfo {author} {\bibfnamefont {E.}~\bibnamefont
  {Amitai}}, \bibinfo {author} {\bibfnamefont {M.}~\bibnamefont
  {Koppenh{\"o}fer}}, \bibinfo {author} {\bibfnamefont {N.}~\bibnamefont
  {L{\"o}rch}},\ and\ \bibinfo {author} {\bibfnamefont {C.}~\bibnamefont
  {Bruder}},\ }\href {https://doi.org/10.1103/PhysRevE.97.052203} {\bibfield
  {journal} {\bibinfo  {journal} {Physical Review E}\ }\textbf {\bibinfo
  {volume} {97}},\ \bibinfo {pages} {052203} (\bibinfo {year}
  {2018})}\BibitemShut {NoStop}%
\bibitem [{\citenamefont {Sonar}\ \emph {et~al.}(2018)\citenamefont {Sonar},
  \citenamefont {Hajdu\ifmmode~\check{s}\else \v{s}\fi{}ek}, \citenamefont
  {Mukherjee}, \citenamefont {Fazio}, \citenamefont {Vedral}, \citenamefont
  {Vinjanampathy},\ and\ \citenamefont {Kwek}}]{Kwek_Squeezing2018}%
  \BibitemOpen
  \bibfield  {author} {\bibinfo {author} {\bibfnamefont {S.}~\bibnamefont
  {Sonar}}, \bibinfo {author} {\bibfnamefont {M.}~\bibnamefont
  {Hajdu\ifmmode~\check{s}\else \v{s}\fi{}ek}}, \bibinfo {author}
  {\bibfnamefont {M.}~\bibnamefont {Mukherjee}}, \bibinfo {author}
  {\bibfnamefont {R.}~\bibnamefont {Fazio}}, \bibinfo {author} {\bibfnamefont
  {V.}~\bibnamefont {Vedral}}, \bibinfo {author} {\bibfnamefont
  {S.}~\bibnamefont {Vinjanampathy}},\ and\ \bibinfo {author} {\bibfnamefont
  {L.-C.}\ \bibnamefont {Kwek}},\ }\href
  {https://doi.org/10.1103/PhysRevLett.120.163601} {\bibfield  {journal}
  {\bibinfo  {journal} {Phys. Rev. Lett.}\ }\textbf {\bibinfo {volume} {120}},\
  \bibinfo {pages} {163601} (\bibinfo {year} {2018})}\BibitemShut {NoStop}%
\bibitem [{\citenamefont {Cardenas-Lopez}\ \emph {et~al.}(2019)\citenamefont
  {Cardenas-Lopez}, \citenamefont {Sanz}, \citenamefont {Retamal},\ and\
  \citenamefont {Solano}}]{Solano2019b}%
  \BibitemOpen
  \bibfield  {author} {\bibinfo {author} {\bibfnamefont {F.~A.}\ \bibnamefont
  {Cardenas-Lopez}}, \bibinfo {author} {\bibfnamefont {M.}~\bibnamefont
  {Sanz}}, \bibinfo {author} {\bibfnamefont {J.~C.}\ \bibnamefont {Retamal}},\
  and\ \bibinfo {author} {\bibfnamefont {E.}~\bibnamefont {Solano}},\ }\href
  {https://doi.org/10.1002/qute.201800076} {\bibfield  {journal} {\bibinfo
  {journal} {Adv. Quantum Techn.}\ }\textbf {\bibinfo {volume} {2}},\ \bibinfo
  {pages} {UNSP 1800076} (\bibinfo {year} {2019})}\BibitemShut {NoStop}%
\bibitem [{\citenamefont {Jaseem}\ \emph
  {et~al.}(2020{\natexlab{a}})\citenamefont {Jaseem}, \citenamefont
  {Hajdu\ifmmode~\check{s}\else \v{s}\fi{}ek}, \citenamefont {Solanki},
  \citenamefont {Kwek}, \citenamefont {Fazio},\ and\ \citenamefont
  {Vinjanampathy}}]{Jaseem2020}%
  \BibitemOpen
  \bibfield  {author} {\bibinfo {author} {\bibfnamefont {N.}~\bibnamefont
  {Jaseem}}, \bibinfo {author} {\bibfnamefont {M.}~\bibnamefont
  {Hajdu\ifmmode~\check{s}\else \v{s}\fi{}ek}}, \bibinfo {author}
  {\bibfnamefont {P.}~\bibnamefont {Solanki}}, \bibinfo {author} {\bibfnamefont
  {L.-C.}\ \bibnamefont {Kwek}}, \bibinfo {author} {\bibfnamefont
  {R.}~\bibnamefont {Fazio}},\ and\ \bibinfo {author} {\bibfnamefont
  {S.}~\bibnamefont {Vinjanampathy}},\ }\href
  {https://doi.org/10.1103/PhysRevResearch.2.043287} {\bibfield  {journal}
  {\bibinfo  {journal} {Phys. Rev. Res.}\ }\textbf {\bibinfo {volume} {2}},\
  \bibinfo {pages} {043287} (\bibinfo {year} {2020}{\natexlab{a}})}\BibitemShut
  {NoStop}%
\bibitem [{\citenamefont {Es'haqi-Sani}\ \emph {et~al.}(2020)\citenamefont
  {Es'haqi-Sani}, \citenamefont {Manzano}, \citenamefont {Zambrini},\ and\
  \citenamefont {Fazio}}]{Fazio2020}%
  \BibitemOpen
  \bibfield  {author} {\bibinfo {author} {\bibfnamefont {N.}~\bibnamefont
  {Es'haqi-Sani}}, \bibinfo {author} {\bibfnamefont {G.}~\bibnamefont
  {Manzano}}, \bibinfo {author} {\bibfnamefont {R.}~\bibnamefont {Zambrini}},\
  and\ \bibinfo {author} {\bibfnamefont {R.}~\bibnamefont {Fazio}},\ }\href
  {https://doi.org/10.1103/PhysRevResearch.2.023101} {\bibfield  {journal}
  {\bibinfo  {journal} {Phys. Rev. Res.}\ }\textbf {\bibinfo {volume} {2}},\
  \bibinfo {pages} {023101} (\bibinfo {year} {2020})}\BibitemShut {NoStop}%
\bibitem [{\citenamefont {Cabot}\ \emph {et~al.}(2021)\citenamefont {Cabot},
  \citenamefont {Giorgi},\ and\ \citenamefont {Zambrini}}]{Zambrini2021}%
  \BibitemOpen
  \bibfield  {author} {\bibinfo {author} {\bibfnamefont {A.}~\bibnamefont
  {Cabot}}, \bibinfo {author} {\bibfnamefont {G.~L.}\ \bibnamefont {Giorgi}},\
  and\ \bibinfo {author} {\bibfnamefont {R.}~\bibnamefont {Zambrini}},\ }\href
  {https://doi.org/10.1088/1367-2630/ac29fe} {\bibfield  {journal} {\bibinfo
  {journal} {New J. of Phys.}\ }\textbf {\bibinfo {volume} {23}},\ \bibinfo
  {pages} {103017} (\bibinfo {year} {2021})}\BibitemShut {NoStop}%
\bibitem [{\citenamefont {Schmolke}\ and\ \citenamefont
  {Lutz}(2022)}]{Lutz2022}%
  \BibitemOpen
  \bibfield  {author} {\bibinfo {author} {\bibfnamefont {F.}~\bibnamefont
  {Schmolke}}\ and\ \bibinfo {author} {\bibfnamefont {E.}~\bibnamefont
  {Lutz}},\ }\href {https://doi.org/10.1103/PhysRevLett.129.250601} {\bibfield
  {journal} {\bibinfo  {journal} {Phys. Rev. Lett.}\ }\textbf {\bibinfo
  {volume} {129}},\ \bibinfo {pages} {250601} (\bibinfo {year}
  {2022})}\BibitemShut {NoStop}%
\bibitem [{\citenamefont {Solanki}\ \emph
  {et~al.}(2022{\natexlab{a}})\citenamefont {Solanki}, \citenamefont {Jaseem},
  \citenamefont {Hajdu\ifmmode~\check{s}\else \v{s}\fi{}ek},\ and\
  \citenamefont {Vinjanampathy}}]{Parvinder2022}%
  \BibitemOpen
  \bibfield  {author} {\bibinfo {author} {\bibfnamefont {P.}~\bibnamefont
  {Solanki}}, \bibinfo {author} {\bibfnamefont {N.}~\bibnamefont {Jaseem}},
  \bibinfo {author} {\bibfnamefont {M.}~\bibnamefont
  {Hajdu\ifmmode~\check{s}\else \v{s}\fi{}ek}},\ and\ \bibinfo {author}
  {\bibfnamefont {S.}~\bibnamefont {Vinjanampathy}},\ }\href
  {https://doi.org/10.1103/PhysRevA.105.L020401} {\bibfield  {journal}
  {\bibinfo  {journal} {Phys. Rev. A}\ }\textbf {\bibinfo {volume} {105}},\
  \bibinfo {pages} {L020401} (\bibinfo {year}
  {2022}{\natexlab{a}})}\BibitemShut {NoStop}%
\bibitem [{\citenamefont {Bandyopadhyay}\ and\ \citenamefont
  {Banerjee}(2022)}]{Biswabibek2022}%
  \BibitemOpen
  \bibfield  {author} {\bibinfo {author} {\bibfnamefont {B.}~\bibnamefont
  {Bandyopadhyay}}\ and\ \bibinfo {author} {\bibfnamefont {T.}~\bibnamefont
  {Banerjee}},\ }\href {https://doi.org/10.1103/PhysRevE.106.024216} {\bibfield
   {journal} {\bibinfo  {journal} {Phys. Rev. E}\ }\textbf {\bibinfo {volume}
  {106}},\ \bibinfo {pages} {024216} (\bibinfo {year} {2022})}\BibitemShut
  {NoStop}%
\bibitem [{\citenamefont {Buca}\ \emph {et~al.}(2022)\citenamefont {Buca},
  \citenamefont {Booker},\ and\ \citenamefont {Jaksch}}]{BucaJaksch2022}%
  \BibitemOpen
  \bibfield  {author} {\bibinfo {author} {\bibfnamefont {B.}~\bibnamefont
  {Buca}}, \bibinfo {author} {\bibfnamefont {C.}~\bibnamefont {Booker}},\ and\
  \bibinfo {author} {\bibfnamefont {D.}~\bibnamefont {Jaksch}},\ }\href
  {https://doi.org/10.21468/SciPostPhys.12.3.097} {\bibfield  {journal}
  {\bibinfo  {journal} {Scipost Physics}\ }\textbf {\bibinfo {volume} {12}},\
  \bibinfo {pages} {097} (\bibinfo {year} {2022})}\BibitemShut {NoStop}%
\bibitem [{\citenamefont {Kato}\ and\ \citenamefont
  {Nakao}(2023)}]{KatoNakao2023}%
  \BibitemOpen
  \bibfield  {author} {\bibinfo {author} {\bibfnamefont {Y.}~\bibnamefont
  {Kato}}\ and\ \bibinfo {author} {\bibfnamefont {H.}~\bibnamefont {Nakao}},\
  }\href {https://doi.org/10.1088/1367-2630/acb6e8} {\bibfield  {journal}
  {\bibinfo  {journal} {New J. of Phys.}\ }\textbf {\bibinfo {volume} {25}},\
  \bibinfo {pages} {023012} (\bibinfo {year} {2023})}\BibitemShut {NoStop}%
\bibitem [{\citenamefont {Liao}\ \emph {et~al.}(2019)\citenamefont {Liao},
  \citenamefont {Chen}, \citenamefont {Xie}, \citenamefont {He},\ and\
  \citenamefont {Lin}}]{PhysRevA.99.033818}%
  \BibitemOpen
  \bibfield  {author} {\bibinfo {author} {\bibfnamefont {C.-G.}\ \bibnamefont
  {Liao}}, \bibinfo {author} {\bibfnamefont {R.-X.}\ \bibnamefont {Chen}},
  \bibinfo {author} {\bibfnamefont {H.}~\bibnamefont {Xie}}, \bibinfo {author}
  {\bibfnamefont {M.-Y.}\ \bibnamefont {He}},\ and\ \bibinfo {author}
  {\bibfnamefont {X.-M.}\ \bibnamefont {Lin}},\ }\href
  {https://doi.org/10.1103/PhysRevA.99.033818} {\bibfield  {journal} {\bibinfo
  {journal} {Phys. Rev. A}\ }\textbf {\bibinfo {volume} {99}},\ \bibinfo
  {pages} {033818} (\bibinfo {year} {2019})}\BibitemShut {NoStop}%
\bibitem [{\citenamefont {Lau}\ \emph {et~al.}(2023)\citenamefont {Lau},
  \citenamefont {Davidsen},\ and\ \citenamefont {Simon}}]{Davidsen2023}%
  \BibitemOpen
  \bibfield  {author} {\bibinfo {author} {\bibfnamefont {H.~W.~H.}\
  \bibnamefont {Lau}}, \bibinfo {author} {\bibfnamefont {J.}~\bibnamefont
  {Davidsen}},\ and\ \bibinfo {author} {\bibfnamefont {C.}~\bibnamefont
  {Simon}},\ }\href {https://doi.org/10.1038/s41598-023-35061-3} {\bibfield
  {journal} {\bibinfo  {journal} {Scientific Reports}\ }\textbf {\bibinfo
  {volume} {13}},\ \bibinfo {pages} {8590} (\bibinfo {year}
  {2023})}\BibitemShut {NoStop}%
\bibitem [{\citenamefont {Roulet}\ and\ \citenamefont
  {Bruder}(2018{\natexlab{b}})}]{rouletQuantumSynchronizationEntanglement2018}%
  \BibitemOpen
  \bibfield  {author} {\bibinfo {author} {\bibfnamefont {A.}~\bibnamefont
  {Roulet}}\ and\ \bibinfo {author} {\bibfnamefont {C.}~\bibnamefont
  {Bruder}},\ }\href {https://doi.org/10.1103/PhysRevLett.121.063601}
  {\bibfield  {journal} {\bibinfo  {journal} {Physical Review Letters}\
  }\textbf {\bibinfo {volume} {121}},\ \bibinfo {pages} {063601} (\bibinfo
  {year} {2018}{\natexlab{b}})}\BibitemShut {NoStop}%
\bibitem [{\citenamefont {Koppenh{\"o}fer}\ and\ \citenamefont
  {Roulet}(2019)}]{koppenhoferOptimalSynchronizationDeep2019}%
  \BibitemOpen
  \bibfield  {author} {\bibinfo {author} {\bibfnamefont {M.}~\bibnamefont
  {Koppenh{\"o}fer}}\ and\ \bibinfo {author} {\bibfnamefont {A.}~\bibnamefont
  {Roulet}},\ }\href {https://doi.org/10.1103/PhysRevA.99.043804} {\bibfield
  {journal} {\bibinfo  {journal} {Physical Review A}\ }\textbf {\bibinfo
  {volume} {99}},\ \bibinfo {pages} {043804} (\bibinfo {year}
  {2019})}\BibitemShut {NoStop}%
\bibitem [{\citenamefont {Jaseem}\ \emph
  {et~al.}(2020{\natexlab{b}})\citenamefont {Jaseem}, \citenamefont {Hajdu{\v
  s}ek}, \citenamefont {Vedral}, \citenamefont {Fazio}, \citenamefont {Kwek},\
  and\ \citenamefont
  {Vinjanampathy}}]{jaseemQuantumSynchronisationNanoscale2020}%
  \BibitemOpen
  \bibfield  {author} {\bibinfo {author} {\bibfnamefont {N.}~\bibnamefont
  {Jaseem}}, \bibinfo {author} {\bibfnamefont {M.}~\bibnamefont {Hajdu{\v
  s}ek}}, \bibinfo {author} {\bibfnamefont {V.}~\bibnamefont {Vedral}},
  \bibinfo {author} {\bibfnamefont {R.}~\bibnamefont {Fazio}}, \bibinfo
  {author} {\bibfnamefont {L.-C.}\ \bibnamefont {Kwek}},\ and\ \bibinfo
  {author} {\bibfnamefont {S.}~\bibnamefont {Vinjanampathy}},\ }\href
  {https://doi.org/10.1103/PhysRevE.101.020201} {\bibfield  {journal} {\bibinfo
   {journal} {Physical Review E}\ }\textbf {\bibinfo {volume} {101}},\ \bibinfo
  {pages} {020201(R)} (\bibinfo {year} {2020}{\natexlab{b}})}\BibitemShut
  {NoStop}%
\bibitem [{\citenamefont {Solanki}\ \emph
  {et~al.}(2022{\natexlab{b}})\citenamefont {Solanki}, \citenamefont {Mehdi},
  \citenamefont {Hajdušek},\ and\ \citenamefont
  {Vinjanampathy}}]{solanki2022symmetries}%
  \BibitemOpen
  \bibfield  {author} {\bibinfo {author} {\bibfnamefont {P.}~\bibnamefont
  {Solanki}}, \bibinfo {author} {\bibfnamefont {F.~M.}\ \bibnamefont {Mehdi}},
  \bibinfo {author} {\bibfnamefont {M.}~\bibnamefont {Hajdušek}},\ and\
  \bibinfo {author} {\bibfnamefont {S.}~\bibnamefont {Vinjanampathy}},\
  }\href@noop {} {\bibinfo {title} {Symmetries and synchronization blockade}}
  (\bibinfo {year} {2022}{\natexlab{b}}),\ \Eprint
  {https://arxiv.org/abs/2212.09388} {arXiv:2212.09388} \BibitemShut {NoStop}%
\bibitem [{\citenamefont {Murtadho}\ \emph {et~al.}(2023)\citenamefont
  {Murtadho}, \citenamefont {Vinjanampathy},\ and\ \citenamefont
  {Thingna}}]{PhysRevLett.131.030401}%
  \BibitemOpen
  \bibfield  {author} {\bibinfo {author} {\bibfnamefont {T.}~\bibnamefont
  {Murtadho}}, \bibinfo {author} {\bibfnamefont {S.}~\bibnamefont
  {Vinjanampathy}},\ and\ \bibinfo {author} {\bibfnamefont {J.}~\bibnamefont
  {Thingna}},\ }\href {https://doi.org/10.1103/PhysRevLett.131.030401}
  {\bibfield  {journal} {\bibinfo  {journal} {Phys. Rev. Lett.}\ }\textbf
  {\bibinfo {volume} {131}},\ \bibinfo {pages} {030401} (\bibinfo {year}
  {2023})}\BibitemShut {NoStop}%
\bibitem [{\citenamefont {Spohn}(1980)}]{RevModPhys.52.569}%
  \BibitemOpen
  \bibfield  {author} {\bibinfo {author} {\bibfnamefont {H.}~\bibnamefont
  {Spohn}},\ }\href {https://doi.org/10.1103/RevModPhys.52.569} {\bibfield
  {journal} {\bibinfo  {journal} {Rev. Mod. Phys.}\ }\textbf {\bibinfo {volume}
  {52}},\ \bibinfo {pages} {569} (\bibinfo {year} {1980})}\BibitemShut
  {NoStop}%
\bibitem [{\citenamefont {Lee}\ \emph {et~al.}(2014)\citenamefont {Lee},
  \citenamefont {Chan},\ and\ \citenamefont
  {Wang}}]{leeEntanglementTongueQuantum2014}%
  \BibitemOpen
  \bibfield  {author} {\bibinfo {author} {\bibfnamefont {T.~E.}\ \bibnamefont
  {Lee}}, \bibinfo {author} {\bibfnamefont {C.-K.}\ \bibnamefont {Chan}},\ and\
  \bibinfo {author} {\bibfnamefont {S.}~\bibnamefont {Wang}},\ }\href
  {https://doi.org/10.1103/PhysRevE.89.022913} {\bibfield  {journal} {\bibinfo
  {journal} {Physical Review E}\ }\textbf {\bibinfo {volume} {89}},\ \bibinfo
  {pages} {022913} (\bibinfo {year} {2014})}\BibitemShut {NoStop}%
\bibitem [{\citenamefont {Kuramoto}(1975)}]{Kuramoto1975}%
  \BibitemOpen
  \bibfield  {author} {\bibinfo {author} {\bibfnamefont {Y.}~\bibnamefont
  {Kuramoto}},\ }in\ \href@noop {} {\emph {\bibinfo {booktitle} {International
  Symposium on Mathematical Problems in Theoretical Physics}}},\ \bibinfo
  {editor} {edited by\ \bibinfo {editor} {\bibfnamefont {H.}~\bibnamefont
  {Araki}}}\ (\bibinfo  {publisher} {Springer Berlin Heidelberg},\ \bibinfo
  {address} {Berlin, Heidelberg},\ \bibinfo {year} {1975})\ p.\ \bibinfo
  {pages} {420}\BibitemShut {NoStop}%
\bibitem [{\citenamefont {Iemini}\ \emph {et~al.}(2018)\citenamefont {Iemini},
  \citenamefont {Russomanno}, \citenamefont {Keeling}, \citenamefont
  {Schir{\`o}}, \citenamefont {Dalmonte},\ and\ \citenamefont
  {Fazio}}]{iemini_BoundaryTimeCrystals_2018}%
  \BibitemOpen
  \bibfield  {author} {\bibinfo {author} {\bibfnamefont {F.}~\bibnamefont
  {Iemini}}, \bibinfo {author} {\bibfnamefont {A.}~\bibnamefont {Russomanno}},
  \bibinfo {author} {\bibfnamefont {J.}~\bibnamefont {Keeling}}, \bibinfo
  {author} {\bibfnamefont {M.}~\bibnamefont {Schir{\`o}}}, \bibinfo {author}
  {\bibfnamefont {M.}~\bibnamefont {Dalmonte}},\ and\ \bibinfo {author}
  {\bibfnamefont {R.}~\bibnamefont {Fazio}},\ }\href
  {https://doi.org/10.1103/PhysRevLett.121.035301} {\bibfield  {journal}
  {\bibinfo  {journal} {Physical Review Letters}\ }\textbf {\bibinfo {volume}
  {121}},\ \bibinfo {pages} {035301} (\bibinfo {year} {2018})}\BibitemShut
  {NoStop}%
\bibitem [{\citenamefont {Kirton}\ \emph {et~al.}(2019)\citenamefont {Kirton},
  \citenamefont {Roses}, \citenamefont {Keeling},\ and\ \citenamefont
  {Dalla~Torre}}]{Kirton_2019}%
  \BibitemOpen
  \bibfield  {author} {\bibinfo {author} {\bibfnamefont {P.}~\bibnamefont
  {Kirton}}, \bibinfo {author} {\bibfnamefont {M.~M.}\ \bibnamefont {Roses}},
  \bibinfo {author} {\bibfnamefont {J.}~\bibnamefont {Keeling}},\ and\ \bibinfo
  {author} {\bibfnamefont {E.~G.}\ \bibnamefont {Dalla~Torre}},\ }\href
  {https://doi.org/10.1002/qute.201800043} {\bibfield  {journal} {\bibinfo
  {journal} {Advanced Quantum Technologies}\ }\textbf {\bibinfo {volume} {2}},\
  \bibinfo {pages} {1800043} (\bibinfo {year} {2019})}\BibitemShut {NoStop}%
\bibitem [{sup()}]{supp}%
  \BibitemOpen
  \href@noop {} {}\bibinfo {note} {See Supplemental Material, which includes
  Refs.~\cite{cumulants_kubo,oppenheim1999discrete}, for details on the
  microscopic results, additional figures, a discussion of finite-size effects,
  and more information on the methods.}\BibitemShut {Stop}%
\bibitem [{\citenamefont {Adler}(1946)}]{adler}%
  \BibitemOpen
  \bibfield  {author} {\bibinfo {author} {\bibfnamefont {R.}~\bibnamefont
  {Adler}},\ }\href {https://doi.org/10.1109/JRPROC.1946.229930} {\bibfield
  {journal} {\bibinfo  {journal} {Proceedings of the IRE}\ }\textbf {\bibinfo
  {volume} {34}},\ \bibinfo {pages} {351} (\bibinfo {year} {1946})}\BibitemShut
  {NoStop}%
\bibitem [{\citenamefont {Nigg}(2018)}]{Nigg2018}%
  \BibitemOpen
  \bibfield  {author} {\bibinfo {author} {\bibfnamefont {S.~E.}\ \bibnamefont
  {Nigg}},\ }\href {https://doi.org/10.1103/PhysRevA.97.013811} {\bibfield
  {journal} {\bibinfo  {journal} {Phys. Rev. A}\ }\textbf {\bibinfo {volume}
  {97}},\ \bibinfo {pages} {013811} (\bibinfo {year} {2018})}\BibitemShut
  {NoStop}%
\bibitem [{\citenamefont {Johansson}\ \emph {et~al.}(2013)\citenamefont
  {Johansson}, \citenamefont {Nation},\ and\ \citenamefont
  {Nori}}]{Johansson_2013}%
  \BibitemOpen
  \bibfield  {author} {\bibinfo {author} {\bibfnamefont {J.}~\bibnamefont
  {Johansson}}, \bibinfo {author} {\bibfnamefont {P.}~\bibnamefont {Nation}},\
  and\ \bibinfo {author} {\bibfnamefont {F.}~\bibnamefont {Nori}},\ }\href
  {https://doi.org/10.1016/j.cpc.2012.11.019} {\bibfield  {journal} {\bibinfo
  {journal} {Computer Physics Communications}\ }\textbf {\bibinfo {volume}
  {184}},\ \bibinfo {pages} {1234} (\bibinfo {year} {2013})}\BibitemShut
  {NoStop}%
\bibitem [{\citenamefont {Plankensteiner}\ \emph {et~al.}(2022)\citenamefont
  {Plankensteiner}, \citenamefont {Hotter},\ and\ \citenamefont
  {Ritsch}}]{Plankensteiner2022quantumcumulantsjl}%
  \BibitemOpen
  \bibfield  {author} {\bibinfo {author} {\bibfnamefont {D.}~\bibnamefont
  {Plankensteiner}}, \bibinfo {author} {\bibfnamefont {C.}~\bibnamefont
  {Hotter}},\ and\ \bibinfo {author} {\bibfnamefont {H.}~\bibnamefont
  {Ritsch}},\ }\href {https://doi.org/10.22331/q-2022-01-04-617} {\bibfield
  {journal} {\bibinfo  {journal} {{Quantum}}\ }\textbf {\bibinfo {volume}
  {6}},\ \bibinfo {pages} {617} (\bibinfo {year} {2022})}\BibitemShut {NoStop}%
\bibitem [{\citenamefont {Kubo}(1962)}]{cumulants_kubo}%
  \BibitemOpen
  \bibfield  {author} {\bibinfo {author} {\bibfnamefont {R.}~\bibnamefont
  {Kubo}},\ }\href {https://doi.org/10.1143/JPSJ.17.1100} {\bibfield  {journal}
  {\bibinfo  {journal} {Journal of the Physical Society of Japan}\ }\textbf
  {\bibinfo {volume} {17}},\ \bibinfo {pages} {1100} (\bibinfo {year}
  {1962})}\BibitemShut {NoStop}%
\bibitem [{\citenamefont {Oppenheim}\ \emph {et~al.}(1999)\citenamefont
  {Oppenheim}, \citenamefont {Schafer},\ and\ \citenamefont
  {Buck}}]{oppenheim1999discrete}%
  \BibitemOpen
  \bibfield  {author} {\bibinfo {author} {\bibfnamefont {A.}~\bibnamefont
  {Oppenheim}}, \bibinfo {author} {\bibfnamefont {R.}~\bibnamefont {Schafer}},\
  and\ \bibinfo {author} {\bibfnamefont {J.}~\bibnamefont {Buck}},\ }\href
  {https://books.google.ch/books?id=Bv1SAAAAMAAJ} {\emph {\bibinfo {title}
  {Discrete-time Signal Processing}}},\ Prentice Hall international editions\
  (\bibinfo  {publisher} {Prentice Hall},\ \bibinfo {year} {1999})\BibitemShut
  {NoStop}%
\end{thebibliography}%

\pagebreak
\clearpage
\onecolumngrid
\begin{center}
\textbf{\large Supplemental material: Macroscopic quantum synchronization effects}
\end{center}
\newcounter{sfigure}
\renewcommand{\thefigure}{S\arabic{sfigure}}
\stepcounter{sfigure}

\setcounter{table}{0}
\setcounter{page}{1}
\makeatletter
\renewcommand{\theequation}{S\arabic{equation}}

\appendix
\section{1.~Microscopic description}
We present results regarding the microscopic synchronization behavior of three-level quantum oscillators.
These results are used to explain the synchronization behavior observed on a macroscopic scale,
in particular regarding the macroscopic interference blockade and the influence of an asymmetry in the level structure of the oscillators shown in \cref{fig:1group}(c), as well as the quantum synchronization blockade displayed in \cref{fig:2groupsVfix}.


\subsection{1.1.~One oscillator coupled to external drive}
We discuss the microscopic scenario of one three-level system oscillator coupled to an external harmonic drive.
This explains the macroscopic effects discussed in the main text regarding synchronization of one group (see \cref{fig:1group}),
viz., the existence of an interference blockade for $K\geq 0$ and its absence for $K<0$.

The master equation for the oscillator coupled to an external drive with strength $\Omega$ resonant with the natural frequency of the oscillator is
\begin{align}
\begin{split}
\dot \rho =
    &-i \left[ K \dyad{2} + \Omega (S^+ + S^-),\rho
    \right] +
    \gp \mathcal{D} \left[\dyad{1}{0} \right]\rho
    +
    \gm \mathcal{D} \left[\dyad{1}{2} \right]\rho \, .
    \label{eq:master1}
\end{split}
\end{align}
The external drive has a similar effect as the coupling to the mean field.
In general, it induces a phase preference as indicated by the phase distribution~\cite{lee_QuantumSynchronizationQuantum_2013,rouletSynchronizingSmallestPossible2018}
\begin{equation}
    s(\phi) = \int_0^\pi \mathrm{d} \theta \sin \theta Q(\theta,\phi) - \frac{1}{2\pi} \, .
\end{equation}
Here, we use the Husimi-Q function
\begin{equation}
    Q(\theta,\phi) = \frac{3}{4\pi} \expval{\rho}{\theta,\phi} \, 
\end{equation}
and spin--coherent states
\begin{equation}
    \ket{\theta,\phi} =
    \exp(-i\phi S^z)
    \exp(-i\theta S^y) \ket{2} \, ,
\end{equation}
with the spin--1 operator $S^y = i(S^- - S^+)/2$.
Note that this phase distribution was generalized to $SU(3)$ coherent states in Refs.~\cite{jaseemQuantumSynchronisationNanoscale2020,solanki2022symmetries}, including two free phases. For our discussion, however, it is enough to consider one phase.
We compute the steady state of the master equation \cref{eq:master1} and show the resulting phase distributions $s(\phi)$ in \cref{fig:1group}(d). The drive represents the influence of the mean field on the oscillator. We choose its strength $\Omega/\gm = 1/10$.
As discussed in the main text, we find that for negative $K$, the oscillator tends to align in phase with the drive, whereas for positive $K$ it tends to anti-align, hindering synchronization.
For $K = 0$, and $\gp = \gm$, the coherences resulting from the external drive have opposite phases and their contributions to the phase distribution partially cancel.
This is the destructive interference, which results in two equal peaks at $\phi=0$ and $\phi=\pi$ in the phase distribution, hence the interference blockade.

\subsection{1.2.~Two coupled oscillators}
\label{app:micro2}
\Cref{eq:system} for a group size of $N=1$
\begin{align}
\begin{split}
\dot \rho =
    &-i \Bigl[ \frac{\delta}{2}(S^z_A - S^z_B) + K (\dyad{2}_A+\dyad{2}_B) +
    % \\
    V_{AB} (S^+_A S^-_B + S^+_B S^-_A),\rho
    \Bigr] +
    \\
    &+ \bigl(
    \gp \mathcal{D} \left[\dyad{1}{0}_A \right] + 
    \gm \mathcal{D} \left[\dyad{1}{2}_A \right] +
   \gp \mathcal{D} \left[\dyad{1}{0}_B \right] + 
    \gm \mathcal{D} \left[\dyad{1}{2}_B \right] \bigr) \rho \, ,
    \label{eq:supp_master2}
\end{split}
\end{align}
describes the dynamics of two coupled, detuned oscillators. We omitted the subscript $i=j=1$.
We analyse their steady state using the phase distribution for the relative phase $\phi_{AB}$
\begin{align}
\begin{split}
    s(\phi_{AB}) =
    \int &
    \mathrm{d} \theta_A \mathrm{d} \theta_B  \mathrm{d} \phi_A \mathrm{d} \phi_B \sin \theta_A \sin \theta_B
    \times
    Q(\theta_A,\theta_B,\phi_A,\phi_B) \times \delta(\phi_{AB}-\phi_A+\phi_B) - \frac{1}{2\pi}
    \, ,
    \end{split}
\end{align}
generalizing the Husimi-Q function to two oscillators
\begin{equation}
    Q(\theta_A,\theta_B,\phi_A,\phi_B) = \frac{9}{16\pi^2} \expval{\rho}{\theta_A,\phi_A,\theta_B,\phi_B} \, 
\end{equation}
using the tensor product of spin-coherent states
\begin{equation}
    \ket{\theta_A,\phi_A,\theta_B,\phi_B} =  \ket{\theta_A,\phi_A} \otimes \ket{\theta_B,\phi_B}\, .
\end{equation}
Since the coupling between any two oscillators is small, we use the first-order result for the steady state of the master equation~\eqref{eq:supp_master2} to compute the Husimi-Q function and $s(\phi_{AB})$.
We show the phase distribution for different values of $\delta$ and $K$ in \cref{fig:2group_micro}(a).
%
% Figure environment removed
%
\stepcounter{sfigure}
If $\delta = -K$, the energy difference between states $\ket{1}$ and $\ket{2}$ of species $A$ is equal to that between states $\ket{0}$ and $\ket{1}$ of species $B$.
Thus, the transition $\ket{1}_A\otimes \ket{1}_B \leftrightarrow \ket{2}_A \otimes \ket{0}_B$ is resonant which leads to a strongly varying phase distribution.
As an example, we show the phase distribution for $\delta= -K = -10\gm$ in the top left of \cref{fig:2group_micro}(a).
We also find a strongly varying phase distribution when the energy difference between states $\ket{0}$ and $\ket{1}$ of species $A$ is equal to that between states $\ket{1}$ and $\ket{2}$ of species $B$ (see bottom left plot where $\delta=K=-12\gm$). In this case, the transition $\ket{1}_A\otimes \ket{1}_B \leftrightarrow \ket{0}_A \otimes \ket{2}_B$ is resonant.
Note that these two transitions are the most important ones, since the limit-cycle (product) state $\ket{1}_A \otimes \ket{1}_B$ is the most populated one.
The large phase alignment between the two groups leads to their full synchronization.
When $\abs{\delta}$ differs significantly from $\abs{K}$, the influence of the coupling is suppressed since the dominant transition are off-resonant.
This causes a comparably flat phase distribution (see middle panels on each side of \cref{fig:2group_micro}(a), and (b), showing the maximum of the phase distribution.).
This is the microscopic quantum synchronization blockade.

As discussed in the main text, in the macroscopic ensemble we find an extended synchronization blockade:
In the regions just above $K=\delta$ and $K=-\delta$, synchronization is also absent.
To explain this we need to consider the preferred relative phase $\phi_{AB}^\mathrm{max} = \mathrm{argmax}_{\phi_{AB}} s(\phi_{AB})$, which we plot as a function of detuning and asymmetry parameter in \cref{fig:2group_micro}(c).
The relative phase is zero below the resonances and $\pi$ above them. At the resonances, they cross $\pm \pi/2$.
The phase shift between $A$ and $B$ below the resonances is positive for $K\lesssim -\delta$, and negative for $K \lesssim \delta$.
This is reflected in the phase difference of the macroscopic ensembles, see \cref{fig:2groupsVfix}.

Since each oscillator is subject to the mean fields of both groups, the resulting effect can be smaller if both groups tend toward opposite phases, i.e., their mean fields tend toward opposite signs.
We find that $\phi_{AB} = \pm \pi/2$, which is exactly at the resonances, is the threshold where synchronization can still occur.
In the macroscopic ensemble, this results in a phase shift of $\pm \pi$.
If the relative phase of two oscillators is even closer to $\pi$, there is no synchronization.

To highlight the relation to the macroscopic phase diagram, we show in \cref{fig:2group_micro}(d) a bitmap with white indicating both the amplitude surpassing a certain threshold and the phase being closer to $0$ (or $2\pi$) than to $\pi$.
As explained before, these two are reasonable assumptions for the requirements of synchronization of the two groups.
Indeed we find that the resulting region agrees with the region where the macroscopic system fully synchronizes. 
The value of the amplitude threshold is a free parameter that we chose to be $5\times 10^{-3}$.
Qualitative features, such as the general X-shape whose bottom diagonals are broader than the top ones, or the cut-off at the resonances are independent of this choice.

\section{2.~Supplementary Figures}
We show more spectra complementing \cref{fig:2groupsAnh0,fig:2groupsVfix} in Figs.~\ref{fig:supp_adler_spectra} and \ref{fig:supp_blockade_spectra}.
In particular, they highlight that for the results presented in this work, it is indeed sufficient to use the frequency difference to characterize the state of the system.
The full spectrum shows other less-pronounced frequency components.
This changes when the inter-group coupling dominates the intra-group coupling (see Fig.~S2 middle to right), where there is more than one dominant frequency, or one group synchronizes more strongly than the other group.
%
% Figure environment removed
\stepcounter{sfigure} 
%
% Figure environment removed
\stepcounter{sfigure} 

\section{3.~Finite-size analysis}
To understand the influence of the group size, we go beyond the mean-field treatment and include some correlations between the observables.
A systematic approach is to truncate higher-order cumulants~\cite{cumulants_kubo}.
For this analysis, we truncate at the second-order correlations, i.e., neglect correlations between three and more observables.
To do so, we use the Julia package QuantumCumulants.jl~\cite{Plankensteiner2022quantumcumulantsjl} which provides an automatized way of deriving equations of motions including correlations up to a set order and converting them to Julia functions that can be integrated numerically.
Implementing this for our system of one group, i.e., \cref{eq:system} with $V_{AB}=0$, and time-integrating, we obtain the results shown in \cref{fig:supp_finite_N}.
% Figure environment removed
\stepcounter{sfigure} 
%
We find that the lifetime as measured by the time for the absolute value of the amplitude to decay to $1/e$ increases linearly with the number of oscillators.
The lifetime of the coherence in a group of 500 oscillators reaches $T\gm \approx 30$, four times larger than the lifetime in the absence of coupling.

\section{4.~Supplementary information on methods}
We present additional information on the numerical calculations used to obtain the results of this work ($\mathbb{I}$ denotes the unit matrix). Note that these details, including initial states or integration times, do not influence the results. All spectra are obtained using a discrete Fourier transform after smoothing the amplitudes with a Hann window~\cite{oppenheim1999discrete}.

\Cref{fig:1group}(a): 
The initial state is $\rho_0 = \mathbb{I}/3 + \dyad{1}{2} \times (1+2i)/10 $. The coupling strengths are $V/(\gm+\gp) = 1/5$  and $V/(\gm+\gp) = 3/5$.

\Cref{fig:1group}(b): 
The initial state is $\rho_0 = \mathbb{I}/3 + \dyad{1}{2} \times 1/10 $.
We integrate for a total time of $5000/\gm$, saving the state of the system in $5000$ equally spaced time points.
The time average for the order parameter is performed over the final half of the time points.
The calculation is done for 320 equally spaced values of $V/(\gm+\gp)$.

\Cref{fig:1group}(c):
The initial state is $\rho_0 = \mathbb{I}/3 + \dyad{1}{2} \times 1/10 $.
We integrate for a total time of $10000/\gm$, saving the state of the system in $10000$ equally spaced time points.
The time average for the order parameter is performed over the last $1000$ time points.
The resolution of the grey-scale image is $255\times 255$.

\Cref{fig:2groupsAnh0} and \cref{fig:supp_adler_spectra}:
The initial state for group $A$ is $\rho_0 = \mathbb{I}/3 + \dyad{1}{2} \times 1/10 $, and that of group $B$ is $\rho_0 = \mathbb{I}/3$.
We integrate for a total time of $1000/\gm$, saving the state of the system in $10000$ equally spaced time points.
The last half is used to compute the spectrum.
The resolution of \cref{fig:2groupsAnh0}(c) is $255\times 255$.

\Cref{fig:2groupsVfix}:
The initial state for group $A$ is $\rho_0 = \mathbb{I}/3 + \dyad{1}{2} \times 1/10 $, and that of group $B$ is $\rho_0 = \mathbb{I}/3$.
We integrate for a total time of $500/\gm$, saving the state of the system in $10000$ equally spaced time points.
For the frequency difference (Panel (a)), we use the last half of the samples.
For the order parameter (Panel (b)), we average over the last 1000 samples.
For the relative phase (Panel (c)), we take the values at the final time of integration.
The resolution is $255\times 255$.

\Cref{fig:supp_blockade_spectra}:
Same as for \cref{fig:supp_adler_spectra}, but with total integration time $1000/\gm$.
The second column of \cref{fig:supp_adler_spectra} is the same as the second column of \cref{fig:supp_blockade_spectra}, obtained with total integration time $1000/\gm$.

\end{document}