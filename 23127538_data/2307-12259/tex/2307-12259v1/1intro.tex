\section{Introduction}

Billiards, of course, needs no introduction.
However, it has two exotic cousins which are less
well known, {\it symplectic billiards\/} and
{\it tiling billiards\/}.   In this paper I will
unite these two topics. I
call the new game {\it symplectic tiling billiards\/}.
Perhaps anyone who knows about both
symplectic billiards and tiling billiards could stop reading
now and define symplectic tiling billiards for themselves just
based on the name.

For ease of exposition
I will stick to the polygonal cases of all these
topics.
Symplectic billiards is perhaps best played on
a pair of polygons, $A$ and $B$, as shown in
Figure 1.  Starting with a pair
$(a_1,b_2) \in \partial A \times \partial B$ one
produces a pair $(a_3,b_4) \in \partial A \times \partial B$
using the rule below.

\begin{center}
\resizebox{!}{2in}{% Figure removed}
\newline
{\bf Figure 1:\/} Symplectic Billiards Defined
\end{center}

In words, the line connecting
$a_1$ to $a_3$ is parallel to the side of $B$
containing $b_2$ and the line connecting
$b_2$ to $b_4$ is parallel to the side of $A$
containing $a_3$.   One then iterates and
considers the dynamics.   I first learned about
symplectic billiards from Peter Albers and
Serge Tabachnikov.  We 
later wrote a paper [{\bf ABSST\/}] about the
subject, proving a few foundational results.
There is also upcoming work of Fabian Lander and
Jannik Westermann on symplectic
polygonal billiards.  The two-table
perspective appears in their work.

Tiling billiards is a variant of billiards played
on the edges of a planar tiling.  Figure 2 shows
the rule. 

\begin{center}
\resizebox{!}{2.7in}{% Figure removed}
\newline
{\bf Figure 2:\/} Tiling Billiards
\end{center}

The rule is essentially the same as for
billiards, except that the trajectory
refracts through the edges rather than
bouncing off them.  I first learned about
tiling billiards from Serge Tabachnikov.
Now there is a growing literature on the
subject.  One recent paper is
[{\bf BDFI\/}].
A later version of this paper will have
a more complete bibliography.

In \S 2 I will define {\it symplectic tiling
billiards\/} and make a few general
remarks about it. I will also show
the results of a few easy experiments.

In \S 3 I will consider a special case of
this game, in which the planar tilings
involved consist of a finite number of
infinite sectors bounded by rays emanating
from the origin.  I will prove a result about
periodic orbits in this situation.

The periodic orbits result from \S 3 turns
out to be useful for giving a clean
correspondence between convex planar
mechanical linkages and convex polygons
with fixed angles.  This correspondence
is akin to the one given by
Kapovich and Millson [{\bf KM\/}], but it
is more direct.  The Kapovich-Millson
corrspondence involves the Riemann mapping
theorem while the one here is elementary.

One can use our correspondence (or the
one given by Kapovich-Millson) to give
a hyperbolic structure to the moduli
space of planar  linkages.  The hyperbolic
structure here comes from Thurston's approach
to the modui spaces of polyhedra given in
[{\bf T\/}]. In other words, the corresondence
simply allows one to import the Thurston
hyperbolic structure to the linkage case.  I will
explain all this in \S 4.

I was
inspired to think about this topic while
listening to a great talk given by
Juergen Richter-Gebert [{\bf R-G\/}]
about the special case of equilateral pentagons.
Richter-Gebert has a different way to
give a hyperbolic structure in this case,
which perhaps is more canonical.  In
any case, our method works in general
and is readily computable.

This paper is a meal that I threw
together based on ideas that were dropped
on my plate while I dined in Heidelberg
and Marseille this summer. I would like to
thank Peter Albers, Diana Davis, Aaron Fenyes, Fabian Lander, 
Juergen Richter-Gebert, Sergei Tabachnikov, and
Jannik Westermann
for stimulating conversations about this material.
I probably had the key idea while in free-fall
riding the Hurricane Loop waterslide at
Miramar water park in Weinheim.
I would also like to thank the University of
Heidelberg for their continued support, in the
form of a Mercator Fellowship, and also the
National Science Foundation for their continued
support.

\newpage