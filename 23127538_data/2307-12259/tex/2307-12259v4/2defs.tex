\section{Symplectic Tiling Billiards}

\subsection{Basic Definition}

For us, a {\it tiling\/} is a subdivision of the plane into
convex polygonal regions.  These polygonal regions
are allowed to be unbounded.    The simplest unbounded
case is that of a {\it sector\/}, namely a region bounded
by rays which make an angle
of less than $\pi$ with each other.

Given a tiling $A$, a {\it particle\/} on $A$ is a
point contained on the interior of an edge of $A$,
together with a choice of a direction pointing into
one of the two regions of $A$ adjacent to $e$.
One could encode the direction by a vector
transverse to $e$.

We say that two tilings $A$ and $B$ are
{\it transverse\/} if no edge of $A$ is parallel to
an edge of $B$.
Suppose that $(A,B)$ are transverse and
$(a_1,b_2)$ are a pair of particles, with $a_1$ being
a particle of $A$ and $b_2$ being a particle of $B$.

We define $a_3$ as follows.  The line connecting
$a_1,a_3$ is parallel to the edge of $B$ containing
$b_2$.  The direction at $a_3$ goes in the same
direction as the direction at $a_1$.  That is, one
and the same vector along the line $\overline{a_1a_3}$
would serve as a transverse vector.   We define
$b_4$ in the same way, swapping the roles of
$A$ and $B$.  Figure 2.1 shows the construction.

\begin{center}
\resizebox{!}{2.8in}{% Figure removed}
\newline
{\bf Figure 2.1:\/} Symplectic Billiards Defined
\end{center}


\subsection{Remarks on the Definition}

Here are some comments about the basic definition.
\newline
\newline
\noindent
{\bf Unbounded Tiles:\/}
One subtle point about this definition is that
the point $a_3$ or the point $b_4$ might not be
defined in case $A$ or $B$ has unbounded tiles.
What happens here is that the relevant ray
simply heads off to infinity without intersecting
an edge of the tiling.  We allow this, and indeed
it might present an interesting case to study, but the
squeamish reader could avoid this problem by
only considering tilings with bounded tiles.
\newline
\newline
{\bf Lack of Transversality:\/}
One might also want to consider the case when
$A$ and $B$ are not transverse.  For instance,
one might like to play this game on a single tiling,
setting $A=B$.  As in symplectic billiards, one
requires that the particles
$a_1$ and $b_2$ are not contained in parallel edges.
\newline
\newline
{\bf Affine Symmetry:\/}
Like symplectic billiards, symplectic tiling
billiards is affinely natural.  If $T$ is an
affine transformation of the plane, then
$T$ maps the orbits relative to the pair
$(A,B)$ to the orbits relative to the pair
$(A',B')$ where $A'=T(A)$ and $B'=T(B)$.
Also, if $T$ is a dilation
then the orbits relative to $(A,B)$ are the
same as the orbits relative to $(A,B')$.

It might be interesting to study symplectic
tiling billiards on tilings which have affine
symmetry.  These are called
affine crystallographic groups.
\newline
\newline
{\bf Half Translation Surfaces:\/}
Symplectic tiling billiards can also be
played on a torus.   This is equivalent to
considering the game relative to a pair
of doubly periodic tilings, and then
considering the orbits on the quotient space.

More generally, one can play the game relative
to a half-translation surface.  Recall that a half-translation
surface is a metric on a surface in which all but
a discrete set of points are locally isometric
to the Euclidean plane and the remaining
points are cone points having cone angle
$\pi k$ for various integers $k$.
One additional requirement for these surfaces
is that there is a global parallel line field.
(This is not quite implied by the other
conditions.)

Let's say that a tiling of a translation surface
is a decomposition of the surface into convex
polygons such that every cone point appears
as a vertex.   Other points might be vertices
as well.  Choices of globally parallel line
fields would give a way to line up the two
tilings.

\subsection{Rotated Square Grids}

Here I make some remarks about some
experiments I did with
symplectic tiling billiards in
the case when $A$ and $B$ are both
square grids.  In this case, the only
parameter is the way $A$ and $B$ are
rotated with respect to each other.

Given
$t \in \Q$ define
\begin{equation}
  \label{RAT}
  z_t=\frac{1-t^2}{1+t^2} + i \bigg(\frac{2t}{1+t^2}\bigg).
\end{equation}
This is the usual rational parametrization of the unit circle.
We normalize so that $A$ is the usual
square grid.
Let $A_t$ denote the result of multiplying the usual
square grid by $z_t$.
The program I wrote uses exact rational
arithmetic to explore this case of
rationally rotated square grids.

For the pair $(A,A_0)$, which is the same
as just playing the game on $A$, the
orbits are just rays.
The choice $t=1/3$ yields $z_t=(4/5)+(3/5)i$.
This is the simplest non-trivial rational case.
It is based on the $(3,4,5)$ right triangle.
Let's take a look.

Figure 4 shows a picture of an orbit
with respect to $(A,A_{1/3})$.
The picture on
the right shows a close-up of the most
complicated part of the orbit on the right.
(I have also continued the orbit a bit further
on the right.)
The slightly thicker segments on the right
are actually unions of extremely close
parallel segments.

\begin{center}
\resizebox{!}{2.5in}{% Figure removed}
\newline
    {\bf Figure 2.2:\/} The left half of an orbit on $(A,A_{1/3})$.
\end{center}

The orbit seems to be bounded and aperiodic.  
I did not attempt to prove this, but
my rational calculations reveal that
the numerators and denominators of the
coordinates of the vertices are
tending to $\infty$.  Geometrically, the orbit has
an attracting limit cycle.  All this
would not be hard to prove; one would
look at the first return map to a suitable
interval, get an interval exchange
transformation, and check that it had
an attracting fixed point.  Here is a concrete
conjecture.

\begin{conjecture}
All orbits on $(A,A_{1/3})$ are bounded and
get attracted to a limit cycle.
The limit cycle itself is a periodic orbit.
\end{conjecture}

Theorem \ref{LR} gives some justification for the
nature of Figure 2.2 above and Figure 2.3 below.

\begin{lemma}
  \label{shoot}
  For a pair of rationally rotated grids, it is impossible
  for an orbit on the left (or on the right) to remain
  forever on the $4$ edges incident to a single vertex.
\end{lemma}

\startproof 
We will suppose that this happens on the left.
Let $V_L$ be the vertex on the left that
the orbit moves around.  As the orbit
on the left side moves around $V_L$ is
must always make a $\pi/2$ degree turn,
either clockwise or counterclockwise.
But then the orbit on the right must forever
hit horizontal and vertical edges in alternation.
But then there is a vertex $V_R$ on the right
such that the orbit on the right stays on the
$4$ edges incident to $V_R$.

Suppose the orbit on the left gets closer to
$V_L$ after $4$ steps.  Then, by symmetry,
the orbit spirals in towards $V_L$ in a
geometric series: the distance to $V_L$
drops by a definite factor $\lambda<1$ at
each revolution.   Theorem \ref{LR} 
says that on the right the distance from
the orbit to $V_R$ increases by $1/\lambda$
after each revolution.  (This result would
be easy to work out by hand in our setting here.)
But then the orbit
on the right eventually escapes and we
have a contradiction.

The other possibility is that the orbit on the
left is periodic.   But then the grids are
rotated by $\pi/4$ degrees relative to
each other.  These are not rationally
rotated grids.
\endproof

Lemma \ref{shoot} gives some explanation for
why the orbits in Figure 2.2 and Figure 2.3 seem
to spiral close to a vertex and then suddenly
shoot out.
\newline

Figure 2.3 shows a picture of an unbounded orbit
on $(A,A_{7/11})$.  I picked this parameter somewhat
randomly.  After making two big spirals, the orbit
starts heading southeast in a periodic pattern with
a drift.  The orbit is clearly unbounded, though I
did not attempt a proof.


\begin{center}
\resizebox{!}{4.5in}{% Figure removed}
\newline
    {\bf Figure 2.3:\/} A symplectic billiard orbit on $(A,A_{7/11})$.
\end{center}

Some parameters seem to support both unbounded orbits
and bounded orbits. It would be nice to classify the
rational parameters according to which kinds of
orbits they support.

For irrationally rotated square grids,
one can sometimes get entirely periodic orbits.
For instance, if we play on $(A,B)$, where $B$ is obtained
by rotating $A$ $\pi/4$ radians, then all orbits are
periodic, and they make squares in each factor.
Compare the end of
the proof of Lemma \ref{shoot}.








\newpage
