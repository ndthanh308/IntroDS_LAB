\section{Hyperbolic Structure}

\subsection{The Thurston Construction}
\label{hypXX}

William Thurston's paper [{\bf T\/}] constructs
complex hyperbolic structures on spaces of
flat cone spheres. See my notes [{\bf Sch1\/}]
for an exposition of [{\bf T\/}].
A special case of a flat
cone sphere is the double of a convex equiangular
polygon.   The corresponding subspace sits as a
totally real slice of the convex hyperbolic moduli
space.  This imparts a real hyperbolic structure
on the space of convex equiangular $N$-gons.

In this section I will give an elementary account of
the construction which does not go through complex
hyperbolic geometry. It is possible that I learned this
construction from Thurston when I was a graduate
student at Princeton University and it is also possible
that I worked it out myself sometime later.
There are a number of similar
accounts in the literature.  See e.g.
[{\bf BG\/}] for the general case and
[{\bf Cal\/}] for the pentagonal case.
\newline
\newline
{\bf Linear Coordinates:\/}
We start
with $N$ parallel familes of lines, with each family being
parallel to a different $N$th root of unity.  These families
are cyclically ordered, according to the roots of unity.
We interpret an equiangular $N$-gon as a selection
$\ell_1,...,\ell_n$ of lines, one from
each family.  The vertices of the $N$-gon
are given by $\ell_1 \cap \ell_2$, $\ell_2 \cap \ell_3$, etc.
This interpretration gives a natural identification
of the space of equiangular $N$-gons with $\R^N$.
To get a concrete coordinatization we could pick
some line $L$ in the plane, not parallel to any of
the families, and then use the intersection
$\ell_1 \cap L,...,\ell_N \cap L$ give $N$
linear coordinates.  In other words, we
are identifying $\R^N$ with $L \times ... \times L$.
A different choice of $L$ would give us a linear
change of coordinates.

We now mod out by translations.  This identifies
the space of equiangular $N$-gons mod isometry
with $\R^{N-2}$.   Figure 5.1 shows, in the pentagon
case, how we can introduce concrete coordinates
on $\R^{N-2}$ which are linear functions of
the coordinates discussed above.

\begin{center}
\resizebox{!}{1.2in}{% Figure removed}
\newline
{\bf Figure 5.1:\/}  Coordinates on the space of equiangular pentagons
\end{center}


It is important to emphasize that these coordinates
$A,B,C$ are signed distances.  They look nice in the
convex case but they are defined even in the non-convex
cases.  For instance, in vector notation,
$$A=\big((\ell_2 \cap \ell_5) - (\ell_3 \cap \ell_5) \big)\cdot (1,0).$$
Also, it is important to note that any system of coordinates based
on a similar contruction (with other choices) would result in
a linear change of variables implemented by a matrix
with algebraic entries.

Figure 5.2 shows similar coordinates for the case of hexagons
and $7$-gons.  We had to make some choices to get these
coordinates, but any similar system
would be related by a change of coordinates implemented by
an algebraic matrix.


\begin{center}
\resizebox{!}{1.25in}{% Figure removed}
\newline
{\bf Figure 5.2:\/}  The case of hexagons and $7$-gons.
\end{center}

\noindent
{\bf The Signed Area:\/}
Now we consider the signed area in these coordinates.
In the pentagonal case we have
\begin{equation}
  \label{area}
  {\rm area\/} = -\alpha A^2 - \beta B^2  + \gamma C^2,
\end{equation}
where $\alpha,\beta,\gamma$ are positive constants that
do not depend on the choice of pentagon.
Geometrically, the area of the pentagon is the
area of the big triangle minus the area of the two small
triangles. The area of the big triangle is a quadratic function
of $C$ and the constant only depends on the shape of the
triangle.  Likewise the areas of the smaller two triangles are
quadratic functions in $A$ and $B$ with the same properties.

For hexagons and $7$-gons, Equation \ref{area} would
respectively have $4$ and $5$ quadratic terms with
constant coefficients and all but one being negative.
In the general case there would be $N$ quadratic terms with
constant coefficients with all but one being negative.
Speaking more abstractly,
the area of the $N$-gon given by
$V=(A,B,C,D,...)$ has the form
$Q(V,V)$ where $Q$ is a quadratic form of signature $(1,N-3)$.
\newline
\newline
{\bf The Lorentz Model:\/}
Now, we are interested in the space of {\it equivalence classes\/} of
equiangular $N$-gons.  Up to translation, we can get a unique
representative of each equivalence class by scaling so that the
area is $1$.  But then we can identify our space of equivalence
classes with one sheet of the hyperboloid in $\R^{1,N-3}$ given by
$Q(V,V)=1$.   This is a well-known {\it Lorentz model\/}
of $\H^{N-3}$, hyperbolic
space of dimension $N-3$.
\newline
\newline
{\bf Interaction with Convexity:\/}
Let ${\cal C\/}_N$ be the domain in
$\H^{N-3}$ corresponding to the
space of strictly convex equiangular $N$-gons.
In general, ${\cal C\/}_N$ is the interior of a convex polyhedral
domain.   The points on the boundary of
${\cal C\/}_N$ correspond to
degenerate polygons in which one or more
edge has collapsed to a point.

For instance, in the case of pentagons,
the boundary of ${\cal C\/}_5$ has
$5$ edges and $5$ vertices.
Referring to Figure 5.1, two of the edges
correspond to $B=0$ and $C=0$, and their
vertex intersection corresponds to $B=C=0$.
There is an elegant way to see the geometry
of ${\cal C\/}_5$.
Figure 5.3 shows the
{\it butterfly move\/} $B_2$ for pentagons.

\begin{center}
\resizebox{!}{3in}{% Figure removed}
\newline
{\bf Figure 5.3:\/}  The butterfly move $B_2$.
\end{center}

The $5$ fixed lines in the hyperbolic
plane $f_1,f_3,f_5,f_2,f_4$ are consecutively
perpendicular in the cyclic sense.
The reason they are perpendicular is that
the corresponding butterfly moves commute! 
These fixed lines are the extensions
of the edges of a regular right-angled pentagon.
The interior of this pentagon is the space of
convex unit equiangular pentagons modulo
similarity.

The case of hexagons is also possible to understand.
In this case ${\cal C\/}_6$ has $6$ faces and $5$
vertices.  Two of the vertices lie in $\H^3$.  These
correspond to the two equilateral triangles we get
by collapsing either the even or the odd edges of
our hexagon.   The other three vertices are
ideal vertices.  These correspond to the hexagons one
gets by letting a pair of opposite sides get very long.
Figure 5.4 shows what we mean.

\begin{center}
\resizebox{!}{.9in}{% Figure removed}
\newline
{\bf Figure 5.4:\/}  Hexagons near each of the $5$ vertices of ${\cal
  D\/}_6$.
\end{center}

Whenever two of these faces intersect, the corresponding
butterfly moves commute. Thus, all the faces which meet
do so at right angles.  The domain ${\cal C\/}_6$ in fact
is the interior of a triangular bi-pyramid.  Each half of the
triangular-bi-pyramid is obtained by coning one face of
a regular ideal octahedron to the center of mass.  This
half is a pyramid, one of whose faces corresponds to the
octahedron.  Call this the {\it blue face\/}.   Call the
other faces the {\it red faces\/}.  The red faces meet at
right angles and each red face meets the blue face at
an angle of $\pi/4$.  We get ${\cal C\/}_6$ by gluing
two of these pyramids together across their blue faces
and we are left with the $6$ red faces.


In general, if $P=\ell_1,..,\ell_N$ then
$B_k(P)=\ell_1',...,\ell_N'$ where
$\ell_j'=\ell_j$ for all $j \not = k$ and
the two lines $\ell_k, \ell_k'$ are equidistant
from the intersection $\ell_{k-1} \ell_{k+1}$.
The operation $B_k$ is linear and preserves
signed area.  Hence $B_k$ is a Lorenz transformation
and induces a hyperbolic isometry on the hyperbolic
structure we have explaned.  Moreover $B_k$ is an
involution.  The fixed point set of $B_k$ is the set of
all degenerate polygons in which $\ell_{k-1}, \ell_k,\ell_{k+1}$
have a common point.  This is a codimension one set.
Hence $B_k$ is a hyperbolic reflection.
Note that $B_{a}$ and $B_b$ commute as long as
$a,b$ are not cyclically consecutive.  The corresponding fixed
point sets are perpendicular hyperplanes.




\subsection{Putting it Together}
\label{put}

For each $N \geq 5$, the Thurston construction produces an open
polyhedral convex domain ${\cal C\/}_N \subset
\H^{N-3}$ whose interior
parametrizes the equivalence classes of
strictly convex equiangular $N$-gons.
Now we get to the punchline.
Using our correspondence
from Equation \ref{assoc}
we get the same hyperbolic structure on the space of
equivalence classes of strictly convex equilateral polygons.
\newline
\newline
\noindent
{\bf Proof of Theorem \ref{algebra}:\/}
Suppose that $L$ is an equilateral polygon with algebraic vertices.
Then by Lemma
\ref{algebra2} we can find a representative $P_L$
having algebraic vertices.   When we
scale $P_L$ to have unit area we are
scaling by an algebraic number, so we can
take $P_L$ to have unit area.
But then our special coordinates for $P_L$ are
also algebraic.  Any choice of special coordinates
would have this property, because they all differ
by the action of an algebraic linear matrix.
So, the coordinates of $P_L$ in Lorentz space
$\R^{1,N-3}$ are algebraic.  This is the same
as saying that the coordinates in $\H^{N-3}$ are algebraic.
Conversely, if $P_L$ has algebraic coordinates in
$\H^{N-3}$, then we can take a representative of
$P_L$ such that the special coordinates taken above
are all algebraic and one of the vertices is algebraic.
But then when we reconstruct $P_L$ from a single
vertex and from the coordinates we get an algebraic
polygon.  But then, by Lemma \ref{algebra2} again,
$[L]$ is also algebraic.
\endproof

\noindent
{\bf Remark:\/}
Finding 
the coordinates in $\R^{1,N-3}$ of a given class of
equiangular polygon is straightforward.
We just choose our coordinates as above and
compute. Thus, in view of the discussion in
\S \ref{compute}, it is easy to accurately estimate
the point in $\H^{N-3}$ which parametrizes a
given equilateral $N$-gon.

\subsection{Beyond the Convex Case}
\label{all}

Now we consider general equilateral polygons.
First of all, we widen our equivalence relation so
that two polygons are equivalent if and only if
there is a similarity which maps one to the other.
The similarity here need not be orientation preserving.
If we restrict our attention to the strictly convex case,
this widening of the equivalence relation changes
nothing, because the counterclockwise-oriented convex
polygons we have been considering above
are equivalent in the wider sense if and only if
they are equivalent in the narrow sense.

  The group $S_N$ of permutations
  acts naturally and continuously on the moduli space of equilateral $N$-gons:
We can encode an equilateral $N$-gon by an ordered list
$e_1,...,e_N$ of unit vectors.   Given a permutation
$\pi \in S_N$ we get the new list
$e_{\pi(1)},...,e_{\pi(N)}$ of edges, and we can build a
unique equivalence class of $N$-gon which corresponds
to this list.  The only thing we need from our vectors
is that they sum to zero, and this is unchanged by
permutation.  Figure 5.5 shows this in action for the
regular pentagon.

\begin{center}
\resizebox{!}{1.2in}{% Figure removed}
\newline
{\bf Figure 5.5:\/}  Regular pentagon permutations
\end{center}

We call an $N$-gon  {\it generic\/} if some
permutation makes it strictly convex. The set of
generic $N$-gons is open and dense, and invariant
under the action of $S_N$.
Topologically this subset is homeomorphic to
\begin{equation}
  f(N)=N!/(2N)
\end{equation}
copies of ${\cal C\/}_N$.   The reason why
$f(N)$ has the form it does is that the dihedral group,
which has order $2N$, permutes the convex classes.
We give a hyperbolic
structure to the subset of generic equilateral $N$-gons,
declaring it to be a disconnected union of $f(N)$
copies of ${\cal C\/}_N$.   By construction, $S_N$ acts
isometrically on our big space.  We call the
component corresponding to the strictly convex $N$-gons
the {\it convex component\/}.

We now enlarge
our space by taking the closures of all the components.
At the moment we still have a disjoint union of
hyperbolic polyhedra whose union (redundantly)
parametrizes all the equilateral $N$-gons.
Finally, we form an identification space by identifying
points in our union which represent the same
(equivalence class of) $N$-gon.  This is our hyperbolic structure on
the moduli space of equilateral $N$-gons.
We denote it by ${\cal A\/}_N$.
\newline
\newline
{\bf Remark:\/}
Technically, when $N$ is even, we have to
add to ${\cal A\/}_N$ the ideal points.  These
correspond to $N$-gons which lie in a single line.
For $N=6$ there are $10$ such.


\subsection{Pentagons}

In this section we explore
${\cal A\/}_5$ and prove Theorem \ref{penta}.
Before taking the quotient, we have
$12$ disjoint copies of ${\cal C\/}_5$, which is
a regular right angled hyperbolic pentagon.  These
pentagons are then glued edge-to-edge.
It turns out that $4$ are glued around each
vertex.  Figure 5.6 illustrates this for one
of the vertices of $C$, the copy of
${\cal C\/}_5$ that corresponds to
the convex pentagonal linkages. 
(I use the word {\it linkage\/} in this section to
avoid confusion;
the moduli space is also composed of pentagons.)
The vertices of
$C$ are certain isosceles triangles, in which
two pairs of consecutive edges point in the
same direction.
       
\begin{center}
\resizebox{!}{2.2in}{% Figure removed}
\newline
{\bf Figure 5.6:\/}  The local picture around a vertex.
\end{center}

The left panel of Figure 5.6 shows the vertex linkage.
The middle  panel shows the $4$ kinds of linkages which
contribute to the pentagons which glue together
around the vertex.  The right panel shows a hand-drawn
approximation of how the corresponding $4$ pentagons
would fit together if developed into the hyperbolic plane.
The subgroup generated by
the transpositions $(12)$ and $(34)$ fixes the
vertex and permutes the $4$ pentagons around it.
What makes this work is that $(12)$ and $(34)$
generate the dihedral subgroup of order $4$.

By symmetry, the picture is the same at every
vertex of our space.  Thus, the space we
get has a global hypebolic structure: It is isometric
to a very symmetric hyperbolic surface $\Sigma$.
The surface $\sigma$ has $12$ faces, and
$12 \times (5/2)=30$ edges and
$12 \times (5/4)=15$ vertices.  Hence
the surface has Euler characteristic
$\chi(\Sigma)=-3$.  
Given the classification of surfaces,
we can identify $\Sigma$ topologically as
the connected sum of a genus $2$ surface and
a projective plane.


\subsection{Hexagons}
\label{hexa}

In this section we explore ${\cal A\/}_6$ and
prove Theorem \ref{hex}.

Figure 5.7 shows the $5$ hexagons which
correspond to the $5$ vertices of ${\cal C\/}_6$.
The triangles correspond to vertices n
$\H^3$ and the segments correspond to
ideal vertices.

\begin{center}
\resizebox{!}{1.2in}{% Figure removed}
\newline
{\bf Figure 5.7:\/}  The vertics of ${\cal C\/}_6$.
\end{center}

The leftmost vertex is stabilized by the
order $8$ group $(\Z/2)^3$ generated by
the permutations $(12)$ and $(34)$ and $(56)$.
Thus, $8$ copies of ${\cal C\/}_6$ fit around
this vertex. Given the right-angles involved, the
identification space ${\cal A\/}_6$
is locally isometric to
$\H^3$ even along the vertices and edges.
The second triangular vertex has the same
kind of story, except that now the
permutations involves are $(23)$ and $(45)$ and $(61)$.

Figure 5.8 shows the hexagons corresponding to points
along the edge of
${\cal C\/}_6$ which connects the first two of the
ideal vertices shown above.

\begin{center}
\resizebox{!}{1.7in}{% Figure removed}
\newline
{\bf Figure 5.8:\/}  An edge of
${\cal C\/}_6$ connecting two ideal vertices.
\end{center}

This edge is stabilized by the order $4$ group
generated by $(23)$ and $(56)$.   From this, we
see that $4$ copies of ${\cal C\/}_6$ fit around
this edge.  Once again, given the right-angled
property of the faces of ${\cal C\/}_6$, this
means that our space ${\cal A\/}_6$ is locally
isometric to $\H^3$ around this edge.
The same story goes for the other two
edges of ${\cal C\/}_6$ that connect
ideal vertices.

Our analysis shows that ${\cal A\/}_6$ is locally isometric
to $\H^3$ in a neighborhood of every point of
${\cal C\/}_6$.   By symmetry, the same statement
holds for all points of ${\cal A\/}_6$.  Hence
${\cal A\/}_6$ is a hyperbolic $3$-manifold.
We can cut each copy of ${\cal C\/}_6$ in half
along the ideal triangle that is the convex hull of
the ideal vertices.   Each half is a pyramid obtained
by coning a regular ideal octahedron to the center
of mass.  In ${\cal A\/}_6$ we have $8$ of these
pyramids fitting together around each finite vertex to make
an ideal octahedron.
Thus ${\cal A\/}_6$ is tiled by regular ideal octahedra.
How many?

Well, ${\cal A\/}_6$ is obtained by gluing together
$f(6)=60$ copies of ${\cal C\/}_6$.
Each copy supplies $2$ pyramids, and we need
$8$ pyramids to make an ideal octahedron.
Thus, each copy of ${\cal C\/}_6$ supplies
$1/4$ of an octahedron.   We conclude
that ${\cal A\/}_6$ is tiled by $15$ regular
ideal hyperbolic octahedra.

\begin{center}
\resizebox{!}{1.5in}{% Figure removed}
\newline
{\bf Figure 5.9:\/}  Shapes of nearly degenerate hexagons
\end{center}

The cusps of ${\cal A\/}_6$ are the degenerate hexagons
corresponding to the permutations of the
last 3 shown in Figure 5.7.
Figure 5.9 shows the representative shapes
of all $10$ degenerate hexagons.  There are
$3$ of the first kind, $6$ of the second kind,
and one of the third kind.  To get a
comprehensible picture we have
taken nearly degenerate hexagons rather
than actually degenerate ones.

Finally, Figure 5.10 shows the shapes of
the $15$ hexagons corresponding to the
centers of the ideal octahedra.  There are
respectively $2,6,3,3,1$ of these.

\begin{center}
\resizebox{!}{1in}{% Figure removed}
\newline
{\bf Figure 5.10:\/}  Shapes of the central hexagons
\end{center}

We have picked representative labelings.


\newpage

