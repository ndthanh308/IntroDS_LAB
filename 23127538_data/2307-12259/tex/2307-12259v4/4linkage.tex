\section{Correspondence between Polygons}

\subsection{The Main Construction}

We consider planar equilateral polygons in which all
the sides have the same length.  We also consider
planar equiangular polygons in which all the interior
angles are the same.  We consider these polygons
up to the equivalence of orientation preserving
similarity.  In both cases we only consider strictly
convex polygons which are oriented counterclockwise
around the regions they bound.  In the equilateral
case we normalize so that the sides have
unit length.  In the equiangular case we 
normalize so that the sides are parallel to the
relevant roots of unity.
\newline
\newline
\noindent
{\bf Main Construction:\/}
Let $L$ be a strictly convex $N$-gon with unit vector edges
$\beta_2,\beta_4,...,\beta_{2N}$.  Since $L$ is closed, we have
$\beta_2+ ... + \beta_{2N}=0$.
Let $B_k$ be the ray through the origin that starts at $0$ and
contains $\beta_k$.
The fact that $L$ is strictly convex means that
$B_2,...,B_{2N}$ is an $N$-sunburst.
By construction, $B$ is a balanced sunburst.
A different representative of $[L]$ would give rise
to a balanced sunburst $B'$ which is a rotation of $B$.
Let $A$ be the regular $N$-sunburst.
Applying Theorem \ref{two}, we know that
there is a unique phase modification
$(A,B')$ which is a weave and which has periodic orbits.
We choose any periodic orbit and consider the
convex $N$-gon $P_L$ which is the image that
the orbit traces around $B$.   Our association
is \begin{equation}
  \label{assoc}
  [L] \to [P_L].
\end{equation}
The labeling is such that the $k$th vertex of $P_L$ lies
on the edge $B_k$.  
By construction, this association
is well-defined, independent of choices.
\newline

\noindent
{\bf Remarks:\/}
(1)   
The construction above only works in the
strictly convex case, but we discuss the general
case in \S \ref{all}.
\newline
(2)
Spherical duality interchanges equiangular and equilateral
spherical polygons.   One might wonder if one could get a
different correspondence like Equation \ref{assoc} by taking
a limit of this process as the sphere radius tends to infinity.
(One of the referees posed this question.)
I think that this probably will not work.
Consider what happens when we have
a unit-sized regular polygon on a growing sphere. Then
the diameter of the dual polygon tends to infinity and
it seems impossible to extract a Euclidean limit.
The same problem would happen for other polygons.

\subsection{Computing the Correspondence}
\label{compute}

In the introduction we claimed that the correspondence in
Equation \ref{assoc} is easy to compute.   Figures 1.3 and
3.1 come from our computer program which does the
computations.  Here we discuss the method.
Once we have the pair $(A,B)$ we find the intervals.
$I_1,...,I_N$ where $I_k \subset S^2$ is the interval
of rotations $R_{\theta}$ such that
$(A,R_{\theta}(B))$ is $k$-intervoven. See \S \ref{weaveexist}.

Next, we compute $I=\bigcap I_k$.   We have the
map $h: I \to \R_+$ which
computes the holonomy $h(\theta)$
of the orbits associated
to the pair $(A,R_{\theta}(B))$ for $\theta \in I$.
We want to solve the equation $h(\theta)=1$.
We use the bisection method.  We choose
two values $\theta_1,\theta_2$ respectively
very near the two boundary points of $I$.
We then perform the following bisection algorithm.
\begin{enumerate}
  \item Start with a pair $(\theta_1,\theta_2)$ such that
    $h(\theta_1)<1<h(\theta_2)$.
  \item Let $\theta_3=(\theta_1+\theta_2)/2$.
  \item If $h(\theta_1)>1$ replace
    $(\theta_1,\theta_2)$ with
    $(\theta_1,\theta_3)$.
  \item If $h(\theta_1) < 1$ replace
            $(\theta_1,\theta_2)$ with
    $(\theta_3,\theta_2)$.
  \item Return to Step 1 using the smaller angle  interval.
    \end{enumerate}
      If we iterate this $M$ times we get the
      correct value of $\theta$ up to about
      $2^{-M}$.
      Once we have our good approximation of
      $\theta$ it is a simple matter to actually
      compute the symplectic tiling orbit.
      \newline
      \newline
      {\bf Remark:\/}
The construction in [{\bf KM\/}] starts with
an equiangular $N$-gon, and then take the Riemann map
to the unit disk,  This gives them $N$ unit complex numbers
$u_1,...,u_N$.
They then use the fact (Springborn's Theorem [{\bf Sp\/}]) that there is
a unique point in the hyperbolic plane such that a
Moebius transformaton $M$ mapping this point to the origin
carries $u_1,...,u_n$ to unit complex numbers whose sum is $0$.
(This part of the construction is similar to ours.)
The numbers $u_1',...,u_n'$ are then interpreted as the
direction vectors of a unit equilateral $N$-gon.
Here we have set $u_k'=M(u_k)$.  This construction
produces a unique equilateral $N$-gon up to scale.
This method requires
      something like the computation of the Riemann map.
      There are several methods for doing this for the
      case at hand.  Concretely, one could numerically
      integrate the Christoffel transform.   Alternatively,
      one could use elegant circle-packing methods
      [{\bf St\/}].  All these methods seem very
            computationally involved.
      
\subsection{Bijective Nature of the Correspondence}

We fix some integer $N \geq 3$ and consider
all our constructions relative to $N$.  In this section
we prove that our association in Equation \ref{assoc} is
a bijection.   We explain first how to construct the
inverse map and then we justify the assumptions needed
for the construction to work.

Let $[P]$ be an equivalence class of convex equiangular $N$-gons.
Let $P$ be some representative.  Let
$P_2,P_4,...,P_{2n}$ be the vertices of $P$.
Say that a point $p$ in the region
bounded by $P$ is a {\it balance point\/} if the sum
$\sum \beta_k=0$, where $\beta_k$ is the unit vector parallel
to $P_k-p$.    We will show below that there is a unique
balance point.  This result is similar in spirit to
Springborn's Theorem [{\bf Sp\/}].

Given our balance point $p$, we
let $B$ be the balanced sunburst defined by
the vectors $\beta_2,...,\beta_{2n}$ we have just defined.
Finally, let $L$ be the equilateral $N$-gon whose successive edges are
parallel to these vectors.  The balance condition guarantees
that $L$ is closed and the sunburst condition guarantees
that $L$ is strictly convex.
Our construction shows how to recover $[L]$ from $[P_L]$, and
this shows that our association is injective.
At the same time, our construction shows that our association
is surjective.  Hence, our association in
Equation \ref{assoc} is bijective.

\begin{lemma}
  A strictly convex $N$-gon has at most one balance point.
\end{lemma}

\startproof
For this proof we do not use the equiangular property.
We will suppose this is false and derive a contradiction.
Let $P$ be a strictly convex polygon which supposedly
has at least $2$ balance points.  We rotate the picture so
that both balance points $p_1,p_2$ lie on the $X$-axis
and $p_1$ is on the left.

\begin{center}
\resizebox{!}{1.7in}{% Figure removed}
\newline
{\bf Figure 4.1:\/}  The angles with the $X$-axis.
\end{center}

For each $i=1,2$ let $\{\beta_{ij}\}$ denote the set of
  unit complex numbers that are parallel to the
  vectors $P_j-p_i$.  The basic property is that
  the angle that the vector $\beta_{1j}$ makes with the
  $x$-axis is less than the angle that $\beta_{2j}$ makes
  with the $x$-axis.   Figure 4.1 shows this in action.
  From this we see that the center of mass of
  $\{\beta_{2j}\}$ must lie to the left of the center
    of mass of $\{\beta_{1j}\}$, contradicting the claim
    that both centers of mass are the origin.
    \endproof



    \begin{lemma}
      \label{exist}
  A strictly convex equiangular $N$-gon has a balance point.
\end{lemma}

\startproof
This proof  uses the equiangular property.
When $N=3$ the polygon must be an equilateral
triangle, and then the center of symmetry does the job.
Likewise, when $N=4$ the polygon must be a rectangle,
and again the center of symmetry does the job.  So,
we take $N \geq 5$.

Let $P$ be a strictly convex equiangular $N$-gon.   We consider a
simply connected domain $D$ in the plane as follows.
We start with the closure of the domain bounded by $P$ and then
we chop off small isosceles triangular
neighborhoods of the vertices of $P$.
Figure 4.2 shows the picture.
The boundary of $D$ is a convex $2N$-gon which equals
$P$ at most places but then takes small ``shortcuts'' into
the interior of the region bounded by $P$ near the vertices. 


For each $p \in D$ we consider the vector
$V_p = \sum \beta_{p,j}$, where $\beta_{p.j}$ is
the unit vector parallel to the one pointing from $p$ to $P_j$.
We are looking for a place where $V_p$ vanishes.
If suffices to prove that $V_p$ points inward at
$\partial D$.  (As the proof develops, we will explain
more precisely what this means.)
The boundary $\partial D$ has two kinds of points, those which also
lie in $P$ and those which do not.
We consider these kinds of points in turn.

\begin{center}
\resizebox{!}{1.4in}{% Figure removed}
\newline
{\bf Figure 4.2:\/}  The domain $D$ and two kinds of boundary points
\end{center}



The point $p$ in Figure 4.2 also lies in $P$.
For points like $p$, all but two of the
vectors $\beta_{p.j}$
point into $D$ and the other two point along $\partial D$.
Let us say this a bit more formally.
The $p$ lies in an edge $e_p$ of $P$,
and $e_p$ is contained in a line $L_p$. The line $L_p$ bounds a
halfplane $H_p$ that contains the other vertices of $P$, and
$\beta_{p,j}$ points into $H_p$ for all but $2$ vertices.
For the other two vertices, namely the vertices of $e_p$,
the two unit vectors in question point along $L_p$.
Adding up all these unit vectors, we see that $V_p$
points into $H_p$.  This is to say that $V_p$ points
into $D$.

The point $q$ in Figure 4.2 does not lie in $P$.
We rotate the picture so that the edge containing
$q$ is vertical, as shown in Figure 4.2.
We also relabel so that $q$ is near $P_1$.
The $x$-coordinate of $\beta_{p,1}$ is at most $1$.
The remaining vertices lie on lines which make an
angle of at least $\pi/N$ with the vertical line
through $P_1$.  Here we are using the equiangularity
condition.     But this means that the sum of the
$x$-coordinates of our remaining vectors is
at most
\begin{equation}
  \label{negsum}
  -(N-1) \times (\sin(\pi/N) - \epsilon).
  \end{equation}
Here $\epsilon$ is a number can make as small as
we like by controlling the size of the isosceles
neighborhoods we used in defining $D$.
As long as we take $\epsilon$ sufficiently small,
the number in Equation \ref{negsum} is
always less than (meaning more negative than)
$-2$.    Hence the vector $V_q$ has negative
$x$-coordinate.  This shows that $V_q$ points
into $D$.

Now we know that our vector field $p \to V_p$ is inward-pointing
on $\partial D$.  A well known result about the index of vectorfields
now shows that $V_p$ vanishes somewhere in the interior of $D$.
Hence $P$ has a balance point.
\endproof


\subsection{Algebraic Nature of the Correspondence}
\label{algXX}


In this section we explain the sense in which the
association in Equation \ref{assoc} is algebraic.
The result here feeds into the proof of Theorem
\ref{algebra}.  

\begin{lemma}
  \label{alg}
  If $[L]$ is algebraic then $[P_L]$ is algebraic.
\end{lemma}

\startproof
When $L$ is algebraic, the rays $B_2,B_4,...,B_{2N}$
are also algebraic.  For instance, they have
algebraic slopes.  The holonomy $\lambda$
of an orbit
associated to $(A,B)$ is an algebraic function of
these slopes.    To understand the phase-modification
part of the construction we choose a rational
parametrization of the circle, as in
Equation \ref{RAT}.    For each $t \in \R \cup \infty$
the slopes of the $t$-rotated sunburst $B'$ are
rational functions in $t$, with algebraic coefficients.

We think of the holonomy $\lambda(t)$ as a function of
the rotation parameter $t$.  The function $\lambda(t)$ is
also a rational function with algebraic coefficients.
Setting $\lambda(t)=1$ and solving, we see that
the choice of $t$ which makes $(A,B')$ have periodic
orbits is algebraic.  But then if we scale so that one of the
vertices of $P_L$ is algebraic then all the vertices will
be algebraic.
\endproof


\begin{lemma}
  \label{algebra2}
  $[P_L]$ if and only if $[L]$ is algebraic.
  \end{lemma}

  \startproof
  In view of Lemma \ref{alg}, we just need to prove
  that $[L]$ is algebraic when $[P_L]$ is algebraic.
  Let $P=P_L$ be an algebraic representative.
  We claim that the balance point of $P$ is algebraic.
  (I am grateful to Joe Silverman for supplying the proof.)

We will use complex notation.
Let $P= (p_1,...,p_N) \in \C$. 
For each sequence
$\epsilon=(\epsilon_1,...,\epsilon_N) \in \{\pm 1\}^N$ define
\begin{equation}
  F_{\epsilon}(P,z)=\sum_{i=1}^N \epsilon_i \frac{z-p_i}{\overline z-\overline p_i} = 0.
\end{equation}
The balance point solves the equation
$F_{\epsilon}(P,z)$ when $\epsilon=(1,...,1)$.
The product
\begin{equation}
  G(z,p)=\prod_{\epsilon \in \{\pm 1\}^N} F_{\epsilon}(z,p)
\end{equation}
is unchanged if we change the signs of any subset of the
square roots in the last equation.   Hence $G(P,z) \in \Q(P,z)$,
the ring of rational functions in $z$ and $P$.  Hence
the roots of $G(P,z)$, for algebraic $P$, are also algebraic.
The balance point is one such root.

Since the balance point is algebraic,
the rays describing the $B$ sunburst are
algebraic.  So, if we start with $a_1$ and $a_2$ algebraic,
the whole orbit remains algebraic.
\endproof

\noindent
{\bf Remark:\/}
The algebraic structure of the balance point might be
quite complicated.  Consider the modest example of an
integer pentagon with vertices
$$(0,0), \hskip 15 pt
(1,0), \hskip 15 pt (2,2), \hskip 15 pt
(1,2) \hskip 15 pt (0,1).$$
Peter Doyle played around with this
in Mathematica and found that the minimal polynomial
for the first coordinate of the balance point has
degree $48$ and the coefficients mostly have about
$30$ digits.


\newpage















