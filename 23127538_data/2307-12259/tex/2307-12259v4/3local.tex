\section{Sunbursts}

\subsection{Basic Definitions}
\label{mainres}

The goal in this chapter is to prove
Theorem \ref{local} and related results.
An $N$-{\it sunburst\/} is a union of $N$ rays emanating from
the origin such that the convex hull of the rays is the whole plane.
An $N$-sunburst defines a tiling in the plane in which the
tiles are unbounded sectors based at the origin.
In this section we will consider symplectic tiling billiards
with respect to two $N$-sunbursts $A$ and $B$.
The number $N$ is the same for both $A$ and $B$.

We orient the rays of $A$ outward
and the rays of $B$ inward, as shown in Figure 3.1.
(Compare Figure 1.3.)
For the entire chapter, we restrict our attention to
the situation where
we have an orbit that starts with $a_1 \in A_1$ and $b_2 \in B_2$,
so that the particle at $a_1$ points into the sector bounded
by $A_1, A_3$ and the particle at $b_2$ points into the
sector bounded by $B_2, B_4$.  When we say that
$(A,B)$ has periodic orbits, we implicitly mean this kind.
By dilation symmetry, one orbit on $(A,B)$ is periodic
if and only if they all are.

\begin{center}
\resizebox{!}{2.6in}{% Figure removed}
\newline
{\bf Figure 3.1:\/}  A periodic orbit relative to a pair of
$5$-sunbursts
\end{center}

Let $(A,B)$ be a pair of sunbursts.  Call an
orbit $\cal O$ of $(A,B)$ {\it woven\/} if the restriction
of ${\cal O\/}$ to $A$, which we call
${\cal O\/}_A$, circulates counterclockwise
around $A$ and
if (with the obvious notation) ${\cal O\/}_B$
circulates counterclockwise around $B$.
The orbit shown in Figure 3.1 is both
woven and periodic.  In this situation
${\cal O\/}_A$ and ${\cal O\/}_B$ are
both convex polygons.

\subsection{A Criterion for Periodicity}
\label{leftright}

Say that ${\cal O\/}$ is {\it left-convex\/}
(respectively {\it right-convex\/})
if ${\cal O\/}_A$ (respectively ${\cal O\/}_B$) is a closed convex polygon. 
In the first version of this paper, I considered
left-convex orbits but did not inquire as to
whether left-convex orbits were also
right-convex and hence periodic.
However, Jannik Westermann read the first
version of the paper and asked this question.
He  noticed that a woven orbit seems to be
left-convex if and only if it is right-convex.
Jannik gave a geometric proof of this
fact for pairs of $3$-sunbursts.  Subsequently, I found
a proof of the general case. 

    \begin{theorem}
      \label{LR}
      An woven orbit relative to a pair of $N$-sunbursts is left-convex if
      and only if it is right-convex if and only if it is periodic.
    \end{theorem}

    The rest of this section is devoted to proving Theorem \ref{LR}.
    Our proof goes
    through some elementary complex analysis.
    I also used this idea in [{\bf Sch2\/}].
    I'd prefer a geometric proof, but I don't have one.
        \newline
    
A {\it complex $N$-sunburst\/} is an
ordered list of $N$ complex lines through the origin
in $\C^2$.  We will usually drop the word {\it complex\/}
in our discussion. We will be interested in a pair
$(A,B)$ of $N$-sunbursts.  We write these
as $A=A_1,A_3,...,A_{2N-1}$ and
$B=B_2,B_4,...,B_{2N}$.

We say that $(A,B)$ is a {\it good pair\/} if
$A_i \not = B_{i \pm 1}$ for all indices.
Let $(A,B)$ be a good pair.
Given $z_1 \in A_1-\{0\}$ we let
$z_3= B_2' \cap A_3$ where $B_2'$ is
the complex line through $z_1$ parallel to $B_2$.
Since $(A,B)$ is a good pair, we have
$z_3 \in A_3-\{0\}$.  In the same way
we define $z_5=B_4' \cap A_5 $, etc.
This gives us points
$z_7,...,z_{2N-1},z_{2N+1}$.
The ratio
$$\lambda_A=z_{2N+1}/z_1$$
  makes sense because both points
  lie in $A_1$.  Also, by scaling symmetry,
  $\lambda_A$
  is independent of the choice of $z_1$.
  We would get the same value
  if we started with $z_3 \in A_3$ and set
  $\lambda_A=z_{2N+3}/z_3$. Etc.
    Likewise we define $\lambda_B$.
  
  \begin{theorem}
    \label{main}
    We have $\lambda_A \lambda_B=1$ for all good pairs $(A,B)$.
  \end{theorem}

  Theorem \ref{main} applies
  in the real case to the pairs of
  sunbursts considered in Theorem \ref{LR}.
  The corresponding woven orbits are left-convex if and only
  if $\lambda_A=1$ and right-convex if and only if
  $\lambda_B=1$.  But Theorem \ref{main}
  says in particular that $\lambda_A=1$ if and only
  if $\lambda_B=1$.  Thus Theorem \ref{main}
  implies Theorem \ref{LR}.
  Now we prove Theorem \ref{main}.

  Let $f(A,B)=\lambda_A \lambda_B$.
    There exists a good pairs $(A_0,B)_0$ such that
    $f(A,B)=1$.  Take $A_0$ to be the regular $N$-sunburst and
  $B_0$ to be the suitably rotated copy.
  Call two good pairs $(A,B)$ and $(A',B')$ {\it closely related\/}
  if they differ only in the placement of a single line.
  For instance, we might have $A=A'$ and $B_k=B_k'$ except
  when $k=2$.   Any two good pairs can be connected by
  a finite sequence of closely related good pairs.
  In other words, we can get from one pair to the other by
  moving one line at a time.    
  In particular, we can start
  with $(A_0,B_0)$ and then reach an arbitrary good pair
  through a finite sequence of closely related pairs.
  For this reason, the next result implies Theorem \ref{main}.

  \begin{lemma}
    $f(A,B)=f(A',B')$ if $(A,B)$ and $(A',B')$ are
    closely related.
  \end{lemma}

  \startproof
  Given the invariance properties of $\lambda_A$ and $\lambda_B$
  discussed above, it suffices to prove our result in the special
  case already mentioned: $A'=A$ and $B_k=B_k'$ except when $k=2$.
  We can identify the space of complex lines through the origin
  with the Riemann sphere $\C \cup \infty$ in the usual way.
  We are just talking about the complex projective line here.
  Given $\zeta \in \C \cup \infty$ let
  $B(\zeta)$ be the $B$-sunburst obtained by replacing $B_2$ with
  the complex line $B_2(\zeta)$
  corresponding to $\zeta$.  Then there are
  parameters $\zeta,\zeta'$ such that
  $B=B(\zeta)$ and $B'=B(\zeta')$. We define
  $f(\zeta)=f(A,B(\zeta))$.
  
  There are $2$ bad values of $\zeta$ where $f$ is undefined, namely when
  $B_2(\zeta)=A_1$ or $B_2(\zeta)=A_3$.
  Given the nature of the construction,
  $f$ is a holomorphic function of $\zeta$, defined away from
  $2$ points on the Riemann sphere.

  Let us analyze the behavior of $f$ at the two bad values.
  We use the notation $g \sim h$ to denote the statement
  that the ratio $|g/h|$ is uniformly bounded away from
  both $0$ and $\infty$.   Here $g$ and $h$ are functions
which depend on the varying choice of $\zeta$.
  
  Suppose  that $B_2(\zeta)$ makes an angle of $\epsilon$ with $A_3$ and
  $|z_1|=1$.  Then $|z_3| \sim 1/\epsilon$.  The rest of the points
  are not affected much by the change.  This gives
  $\lambda_A \sim 1/\epsilon$.    A similar analysis shows
  that $\lambda_B \sim \epsilon$.  Hence
  $f$ is bounded in a neighborhood of
  the parameter $\zeta$ where
  $B_2(\zeta)=A_3$.  A similar analysis works for the other
  bad parameter.

  Since $f$ is a meromorphic function on the Riemann sphere
  with no poles, $f$ is constant.
  \endproof



  \subsection{Weaves}
    

  Let $(A,B)$ be a pair
  of $N$-sunbursts.
  For an even index $k$, we say that $(A,B)$ is {\it woven at\/} $k$
  if $B_{k}$ is parallel to a vector that starts at some interior point of
$A_{k-1}$ and ends at some interior point of $A_{k+1}$.
We call $(A,B)$ a {\it weave\/} if it is woven at $k$ for all
$k=2,4,...,2N$.
The pair $(A,B)$ in Figure 3.1 is a weave.
Our definition is more symmetric than it looks. 
Let $-B$ denote the sunburst obtained by
reflecting $B$ through the origin.

\begin{lemma}
  \label{switch}
  $(-B,A)$ is a weave if and only if $(A,B)$ is a
   weave.
\end{lemma}

\startproof
If $(A,B)$ is a weave, then
so is $(-A,-B)$.  We are just turning the picture upside down.
So, it suffices to prove the ``if'' direction.
Note that now the
weave property for $(-B,A)$ involves odd indices.

Let $\alpha=A$ and $\beta=-B$. Thus,
our new pair is $(\beta,\alpha)$.
The rays of $\beta$ are oriented
outward and the rays of $\alpha$
are oriented inward.  In particular
$\beta_k=B_k$ and $\alpha_k=-A_k$
for all relevant indices.

\begin{center}
\resizebox{!}{1.4in}{% Figure removed}
\newline
{\bf Figure 3.2:\/}  The relevant points and lines
\end{center}

We show that $(\beta,\alpha)$ is
woven at $3$.   The same argument works
for the other odd indices.
Say that an {\it avatar\/} of a ray is
a vector based at the origin and parallel to
the ray.
We normalize by an affine
transformation so that $A_1$ is the positive
$X$-axis and $A_3$ is the positive $Y$-axis.
Then, since $(A,B)$ a weave, $B_2$ has an avatar that
points into the $(-,+)$ quadrant.
Since $(A,B)$ is a weave, $B_4$ has an avatar
that points into the left halfplane.
Since $B$ is a sunburst with inwardly oriented
rays that progress counter-clockwise around as
the index increases, any avatar of $B_4$ lies
beneath any avatar of $B_2$.


The avatars of $\beta_2$ and $\beta_4$ equal
the avatars of $B_4$ and $B_4$ and
$\alpha_3$ is the negative $X$-axis.  From
what we have said above,
$\alpha_3$ is parallel to a vector connecting
a point on $\beta_2$ to a point on $\beta_4$.
Hence $(\beta,\alpha)$ is woven at $3$.
\endproof


 We call the system $(A,B')$ a {\it phase modification\/} of $(A,B)$
  if $B'$ is obtained by rotating $B$ about the origin by some angle.

\begin{theorem}
  \label{one}
  Let $(A,B)$ be a weave.  Then there is
  a  unique phase modification $(A,B')$ of $(A,B)$ which is a
  weave with periodic woven orbits.
\end{theorem}

The rest of this section is devoted to proving Theorem \ref{one}. 

\begin{lemma}
  \label{swap}
  Suppose $(A,B)$ is a weave.
  Then the orbit of $(a_1,b_2)$ is  woven.
\end{lemma}

\startproof
Since $(A,B)$ is a weave, the ray emanating
from $a_1$ and pointing in the direction of $B_2$ intersects
$A_3$.  Thus $a_3$ is well-defined.   According to our definition
in terms of particle, the transverse vector at $a_3$ points
into the sector bounded by $A_3$ and $A_5$.
Because $(B,-A)$ is also a weave, the ray
through $b_2$ and parallel to $-A_3$ intersects
$B_4$. Thus $b_4$ is well-defined.    According to our definition
in terms of particle, the transverse vector at $b_4$ points
into the sector bounded by $B_4$ and $B_6$.  Continuing
like this, we see that the forward orbit is woven.
\endproof

For each index, there is an open interval of phase modifications which
keep that part of the oriented weave condition.
The intersection $I$ parametrizes the phase modifications that are
weaves.

Suppose that $(A,B)$ is a weave.
Let $\lambda_A(A,B)$ be the function from
\S \ref{leftright}.   Since the orbits are woven, we have
$\lambda_A(A,B) \in (0,\infty)$. These orbits are
periodic iff
$\lambda_A(A,B)=1$.  We identify $I$ with $(0,1)$
and we orient $I$ so that as $t$ increases in $(0,1)$ the
corresponding sunburst $B'=B_t$ rotates clockwise.
We define $\lambda(t)=\lambda_A(A,B_t)$.
We want to see that there is a unique value $t \in (0,1)$ such
that $\lambda(t)=1$.  Theorem \ref{two} follows immediately
from the combination of the next two results.

\begin{lemma}
  \label{monotone}
  $\lambda$ is strictly increasing on $(0,1)$.
\end{lemma}

\startproof
Consider the orbit
$a_1(t),b_2(t),a_3(t),...$ from
Lemma \ref{swap}.  Here our notation
reflects the fact that this orbit dependn $t \in (0,1)$.
Notice that the ratio $\lambda_k(t)=\|a_{k+2}(t)\|/\|a_k(t)\|$
depends only on the triple of rays
$A_k, B_{k+1}(t),A_{k+1}$.  As $t$
increases, the ray $B_{k+1}(t)$ rotates
clockwise.  But then $\lambda_k(t)$ is
strictly  increasing.  Since
\begin{equation}
  \label{prod1}
  \lambda(t)=\lambda_1(t) \times ... \times \lambda_{2k-1}(t),
  \end{equation}
$\lambda(t)$ is also strictly increasing.
\endproof

\begin{lemma}
  $\lambda(t) \to 0$ as $t \to 0$ and
  $\lambda(t) \to \infty$ as $t \to 1$.
\end{lemma}

\startproof
We continue with the notation from the
proof of Lemma \ref{monotone}.
As $\lambda \to 0$, one of the $B$-rays
converges to one of the $A$-rays.
But then for the corresponding
index $k$, the quantity $\lambda_i(t)$ exits
every compact subset of $(0,\infty)$.
Since this quantity is decreasing, we see that
$\lambda_i(t) \to 0$.  At the same time,
the remaining factors in Equation \ref{prod1} do
not increase. Hence $\lambda(t) \to 0$.
The same kind of argument shows that
$\lambda(t) \to \infty$ as $t \to 1$.
\endproof

\subsection{A Calculus Interlude}

In this section we prove  an inequality that we use
in the next section.

\begin{lemma}
  \label{calc2}
  \label{calc3}
  Let $N \geq 4$ and
  $k=4,6,...,N$ and
  $y^*=\sin(\pi(k-2)/(2n))$.
    Then
  \begin{equation}
    \label{messy}
    ((k/2)-1) - y^*(N-(k/2)+1) <0.
    \end{equation}
\end{lemma}

\startproof
The function
$f(t)=\sin(\pi t)-t/(1-t)$ is positive on
$I=(0,1/2)$ because
$f(0)=f(1/2)=0$ and
$$f''(t)=-\pi^2 \sin(t) -\frac{2}{(1-t)^2} - \frac{2t}{(1-t)^3}<0.$$
Now,
let $t = (k-2)/2N$.  Note that
the conditions on $k$ give
$t \in (0,1/2)$.  The positivity of $f$ gives
$$y^* = \sin(\pi t) > \frac{t}{1-t}=\frac{k/2-1}{N-(k/2)+1}.$$
The last equality requires a bit of algebra.  Equation
\ref{messy} is a rearrangement of this inequality.
\endproof

\begin{center}
\resizebox{!}{1.1in}{% Figure removed}
\newline
{\bf Figure 3.3:\/}  A plot of equation \ref{messy} for $N=100$ and $k=4,...,50$.
\end{center}


\subsection{Existence of Weaves}
\label{weaveexist}

Now we prove our restatement of
Theorem \ref{local}.

\begin{theorem}
  \label{two}
  If $A$ is regular and $B$ is balanced, then
  $(A,B)$ has a unique phase modification which is both a
  weave and has woven periodic orbits.
\end{theorem}

By Theorem \ref{one}, it suffices to prove that
$(A,B)$ has a phase modification which is a weave.
Let $S^1$ be the set of all
phase modifications $(A,B')$ of $(A,B)$.  For each even index $k$
there is an interval $I_k \subset S^1$ which parametrizes the
phase modofications that are woven at $k$.  We just need to prove that
$I=\bigcap I_k$ is nonempty.
Define
\begin{equation}
  \mu_N=\pi - \frac{2 \pi}{N}.
\end{equation}

\begin{lemma}
  Each interval $I_k$ has angular length $\mu_N$.
\end{lemma}

\startproof
Choose any point
$p \in A_{2k-1}$.  Then as $q \in A_{2k+1}$
moves from the origin to $\infty$, the vector
$\overrightarrow{pq}$ sweeps out an angle of $\mu_N$.
\endproof

When $B=A$, the
intervals $I_1,...,I_{2k-1}$ all coincide, by symmetry.
If $B$ is a general balanced $N$-sunburst,
we can find a homotopy $t \to B(t)$, through
balanced $N$-sunbursts, such that
$B(0)=A$ and $B(1)=B$.
For all indices $i,j$ we will show that the
relative displacement of $I_j(t)$ with respect to
$I_i(t)$ is less than $\mu_N$.
This means that all pairs of intervals intersect for all $t$.

\begin{lemma}
  Suppose that all the intervals $I_i(t)$ and $I_j(t)$
  intersect for all $t \in [0,1]$.  Then
  all the intervals $\{I_k(1)\}$ have a common intersection.
\end{lemma}

\startproof 
Recall a case of Helly's Theorem:
If a finite collection of open intervals in $\R$ pairwise
intersect, then their intersection is nonempty.  We think
of $\R$ as the universal cover of $S^1$.
We can lift our intervals to $\R$, so that for
$t=0$ they are all the same interval and the
lifts vary continuously with $t$.  But then the result about
relative displacement still holds, and all pairs of lifted
intervals intersect for all $t \in [0,1]$.  By Helly's Theorem,
all the lifted intervals intersect (in particular) for $t=1$.
Pushing the intersection point
down to $S^1$, we see that all the intervals
$\{I_k(1)\}$ also intersect.
\endproof

To finish our proof of Theorem \ref{two} we need to establish
our claim about the relative displacement of pairs of intervals.
Without loss of generality, it suffices to consider the
intervals $I_0(t)$ and $I_k(t)$ for $k=2,4,...,2N-2$.
The relative displacement of our intervals does not change if
we post-compose our homotopy $B(t)$ with an arbitrary
continuous family of rotations. So, it suffices to
consider the case when $A_1$ is the positive $X$-axis
and $B_0(t)$ is the positive $X$-axis for all $t \in [0,1]$.

Let $\theta_k(t)$ denote the angle between $B_k(t)$ and
the positive $X$-axis.  We have $\theta_0(t)=0$ for all $t$
and $\theta_k(0)=\pi k/N$ for all $k=0,2,4,...,2N-2$.
The relative displacement between $I_k(1)$ and $I_k(0)$ is
$|\theta_k(1)-\theta_k(0)|$.  We just have to prove that
this quantity is less than $\mu_N$ for all even $k$.

  \begin{lemma}
    $\theta_k(1)-\theta_k(0)<\mu_N$ for all $k=2,...,2N-1$.
  \end{lemma}

  \startproof
  We will assume that there is an index $k$ such that
  $\theta_k(1)-\theta_k(0) \geq \mu_N$ and
  derive a contradiction.  By assumption
  $$\theta_k(1) \geq \theta_k(0) + \pi - \frac{2\pi}{N} =
  \pi + \frac{k \pi}{N} -\frac{2 \pi}{N} =
  \pi +\frac{(k-2)\pi}{N}.$$
  

  We first treat two special cases.
  \begin{itemize}
  \item When $k=2$ we have $\theta_2(1) \geq \pi$.
    This contradicts the fact that $\theta_2(1)<\pi$ is the angle
    between $B_0(1)$ and $B_2(1)$.
\item   When $k \geq N+2$ we have
  $\theta_k(1) \geq 2\pi$.  This contradicts the fact that
$\theta_k(1)<2\pi$ for $k=2,4,...,2N-2$.
\end{itemize}

So, we only need consider the cases
$k=4,6,...,N$.  Compare Lemma \ref{calc2}.
  Let $\beta_j$ denote the unit vector along
  $B_j(1)$.  We rotate the picture about
  the origin so that $\beta_0$ and $\beta_k$ have
  the same negative $y$-coordinate, namely
  $-y^*$, the quantity from Lemma \ref{calc2},
  with $\beta_0$ being on the right and
  $\beta_k$ being on the left.
  The $(k/2)-1$ unit vectors $\beta_2,...,\beta_{k-2}$
  all have $y$-coordinates less or equal to $+1$.  The remaining
  $N-(k/2)+1$ unit vectors,
  being between $\beta_k$ and $\beta_0$ in the
  circular order, all have $y$ coordinates less or
  equal to $-y^*$.   The average of the $y$-coordinates
  of all $N$ unit vectors is at therefore at most
  $$(k/2-1) - y^* (N-(k/2)+1)<0.$$
  The inequality comes from Equation \ref{messy}.
  This contradicts the fact that $B(1)$ is balanced.
  \endproof
  

  \begin{lemma}
    $\theta_k(0)-\theta_k(1)<\mu_N$ for all $k=2,...,2N-1$.
  \end{lemma}

  \startproof
  This follows from the previous result and reflection symmetry.
  Let $\overline B$ denote the balanced $N$-sunburst obtained by
  reflecting $B$ in the $X$-axis and dihedrally relabeling the rays
  so that they again go counterclockwise around the origin.  Applying
  the previous result to $\overline B$ and we get
    $\overline \theta_k(1)-\overline \theta_k(0)<\mu$ for all
  $k=2,..,2N-2$.  The homotopy we take is just the original homotopy
  conjugated by reflection in the origin, with the points dihedrally relabeled.
  At the same time, we have
  $\theta_k(t)=2\pi - \overline \theta_{2N-k}(t).$
  Hence
    $$\theta_k(0)-\theta_k(1) =
  -\overline \theta_{2N-k}(0)+\overline \theta_{2N-k}(1)=
  \overline \theta_{2N-k}(1)-\overline \theta_{2N-k}(0)<\mu.$$
This completes the proof.
  \endproof


  \noindent
  {\bf Remark:\/}
  One might wonder about relaxing the hypotheses of
  Theorem \ref{two}.   For instance, is the result true
  for $(A,B)$ if both $A$ and $B$ are regular?    In the
  even case, at least, one needs extra hypotheses.
  When $N$ is even and $A$ is not centrally symmetric,
  there always exists a centrally symmetric (and hence balanced)
  sunburst $B$ such that $(A,B)$ has no phase
  modification which is a weave.  The sunburst $B$
  consists of two clusters of $N/2$ nearly identical lines
  which are diametrically opposed from each other.
  We leave the details of this to the interested reader.
  In the odd case I have not been able to think of an
  easy counter-example like this.
  
  


\newpage
