

\documentclass[12pt]{article}
\usepackage{amsfonts}
\usepackage{amsmath}
\usepackage{epsfig}

\title{Symplectic Tiling Billiards, Planar Linkages, and Hyperbolic
  Geometry}
\author{Richard Evan Schwartz \thanks{Supported by N.S.F. Grant DMS-2102802}}

\newtheorem{theorem}{Theorem}[section]
\newtheorem{proposition}[theorem]{Proposition}
\newtheorem{lemma}[theorem]{Lemma}
\newtheorem{sublemma}[theorem]{Sub-Lemma}
\newtheorem{remarks}[theorem]{Remarks}
\newtheorem{corollary}[theorem]{Corollary}
\newtheorem{conjecture}[theorem]{Conjecture}
\newtheorem{question}[theorem]{Question}

\def\startproof{{\bf {\medskip}{\noindent}Proof: }}
\def\definition{{\bf {\bigskip}{\noindent}Definition: }}
\def\endproof{$\spadesuit$  \newline}


\def\bp{|\underline{\overline{\times}}|}
\def\A{\mbox{\boldmath{$A$}}}% 
\def\B{\mbox{\boldmath{$B$}}}% 
\def\C{\mbox{\boldmath{$C$}}}% 
\def\D{\mbox{\boldmath{$D$}}}% 
\def\E{\mbox{\boldmath{$E$}}}% 
\def\F{\mbox{\boldmath{$F$}}}% 
\def\G{\mbox{\boldmath{$G$}}}% 
\def\H{\mbox{\boldmath{$H$}}}% 
\def\I{\mbox{\boldmath{$I$}}}% 
\def\J{\mbox{\boldmath{$J$}}}% 
\def\K{\mbox{\boldmath{$K$}}}% 
\def\L{\mbox{\boldmath{$L$}}}% 
\def\M{\mbox{\boldmath{$M$}}}% 
\def\N{\mbox{\boldmath{$N$}}}% 
\def\O{\mbox{\boldmath{$O$}}}% 
\def\P{\mbox{\boldmath{$P$}}}% 
\def\Q{\mbox{\boldmath{$Q$}}}% 
\def\R{\mbox{\boldmath{$R$}}}% 
\def\T{\mbox{\boldmath{$T$}}}% 
\def\U{\mbox{\boldmath{$U$}}}% 
\def\V{\mbox{\boldmath{$V$}}}% 
\def\W{\mbox{\boldmath{$W$}}}% 
\def\X{\mbox{\boldmath{$X$}}}% 
\def\Y{\mbox{\boldmath{$Y$}}}% 
\def\Z{\mbox{\boldmath{$Z$}}}% 
\def\heartsuit{\mbox{\boldmath{$Z$}}}% 

\begin{document}
\maketitle

\begin{abstract}
 In this paper I will unite two games,
  symplectic billiards and tiling billiards. The new
  game is called symplectic tiling billiards.
  I will prove a result about periodic orbits of
  symplectic tiling billiards in a
  very special case and then show how this result
  combines with the construction in 
  Thurston's paper {\it Shapes of Polyhedra\/} to
  give hyperbolic structures on moduli spaces
  of planar equilateral polygons.  One corollary
  is that the configuration space of the hexagonal
  planar linkage with unit-length rods (modulo
  isometry)  has an
  algebraically defined hyperbolic structure in which
  it is a $10$-cusped hyperbolic $3$-manifold
  that is tiled by $15$ regular ideal octahedra.
  The $10$ cusps correspond to the $10$ maximally degenerate
  configurations.
      \end{abstract}

\section{Introduction}

Vision-based Bird's Eye View (BEV) representation\cite{lu2021graph,xie2023x, yang2023bevformer, bartoccioni2023lara, lin2022sparse4d} is an emerging perception formulation for autonomous driving. It transforms and maps the information from the image space to a unified 3D BEV space, which can be used for various perception tasks like 3D object detection and BEV map segmentation. Moreover, the unified BEV space can directly fuse other modalities like LiDAR without any cost, which is of great scalability.

As shown in \cref{fig:first}, the essence of BEV representation is the 2D to 3D mapping, which is a one-to-many ill-posed problem because a point in the image space corresponds to infinite collinear 3D points along the camera ray. To resolve this problem, we need to add an extra condition to make the 2D to 3D mapping a one-to-one well-posed problem. For the added extra condition, there are two kinds of methods, which are LSS\cite{philion2020lift} and OFT\cite{roddick2018orthographic}. LSS proposes to predict latent depth as the extra condition, which is implicitly estimated by end-to-end training. OFT directly maps the 2D information to 3D in the one-to-many fashion, while a network in BEV space is needed to implicitly select the dense mapped information in the vertical or height direction, which is also realized by end-to-end training. Both methods use extra depth or height conditions to resolve the mapping problem, but the extra condition is implicitly trained and used. In this way, the correctness of the mapping is not guaranteed, which might affect the performance of BEV representation.

% Figure environment removed

Motivated by the above observations, we propose to explicitly add and model extra conditions to realize better 2D to 3D mapping.
Similarly, some works\cite{park2021pseudo,li2022bevdepth} propose to directly learn depth as the extra condition with depth pre-training or LiDAR information. Different from using depth, we explicitly model the height condition in the mapping for the following reasons. First, we prove that height in the BEV space is equivalent to depth in the image space for the 2D to 3D mapping problem. Both ways can provide equivalent conditions to resolve the problem of mapping. In this way, we can realize well-defined one-to-one mapping between 2D and 3D. Second, the height information in the BEV space can be retrieved from the BEV annotations without any other data modalities like LiDAR, while depth condition needs extra pre-training or LiDAR. In this work, we use the height information from the object's 3D bounding box, which can be directly accessed from the ground truth. Third, the modeling in height can fit arbitrary camera rigs and types. For example, on NuScenes\cite{nuscenes2019}, the focal length of the backward camera is different from other cameras, resulting in different depth estimation patterns. In other words, different depth estimation network is needed for different cameras. While for the height condition, no matter which kind of camera configuration is used, it is processed with the same pattern in the BEV space. In this way, the height condition is more robust and flexible.

In this work, we propose a network that explicitly models height in the BEV space, which fulfills the condition needed for 2D to 3D mapping, termed as HeightFormer. Moreover, based on the height modeling, self-recursive height predictors are proposed to introduce the uncertainty of heights and segmentation maps which are used in the BEV query mask mechanism to produce high-precision detection results. In summary, the main contributions of our work are summarized as follows:
% \begin{itemize}
    % \item
    1) We give theoretical proof of the equivalence between height-based methods in BEV and depth-based methods in images, which is the basis of our work. The proof also demonstrates the feasibility of detection in the BEV space generated with predicted heights.
    % \item 
    2) {{\color{blue}}We propose to explicitly model heights in the BEV space without extra LiDAR supervision. A self-recursive predictor is proposed to model heights and a corresponding segmentation-based query mask is designed to handle positions whose heights cannot be defined.
    % \item 
    3) Experiments on NuScenes\cite{nuscenes2019} show that the proposed HeightFormer achieves the SOTA performance. Extensive quantitative and qualitative results show that it is feasible and effective to model heights in the BEV space and construct the BEV representation with predicted heights. The generalization analysis also shows that the proposed method can be applied to different methods, as a plugin and as compensation for depth modeling. }
% \end{itemize} 

\section{Symplectic Tiling Billiards}

\subsection{Basic Definition}

For us, a {\it tiling\/} is a subdivision of the plane into
convex polygonal regions.  These polygonal regions
are allowed to be unbounded.    The simplest unbounded
case is that of a {\it sector\/}, namely a region bounded
by rays which make an angle
of less than $\pi$ with each other.

Given a tiling $A$, a {\it particle\/} on $A$ is a
point contained on the interior of an edge of $A$,
together with a choice of a direction pointing into
one of the two regions of $A$ adjacent to $e$.
One could encode the direction by a vector
transverse to $e$.

We say that two tilings $A$ and $B$ are
{\it transverse\/} if no edge of $A$ is parallel to
an edge of $B$.
Suppose that $(A,B)$ are transverse and
$(a_1,b_2)$ are a pair of particles, with $a_1$ being
a particle of $A$ and $b_2$ being a particle of $B$.

We define $a_3$ as follows.  The line connecting
$a_1,a_3$ is parallel to the edge of $B$ containing
$b_2$.  The direction at $a_3$ goes in the same
direction as the direction at $a_1$.  That is, one
and the same vector along the line $\overline{a_1a_3}$
would serve as a transverse vector.   We define
$b_4$ in the same way, swapping the roles of
$A$ and $B$.  Figure 2.1 shows the construction.

\begin{center}
\resizebox{!}{2.8in}{% Figure removed}
\newline
{\bf Figure 2.1:\/} Symplectic Billiards Defined
\end{center}


\subsection{Remarks on the Definition}

Here are some comments about the basic definition.
\newline
\newline
\noindent
{\bf Unbounded Tiles:\/}
One subtle point about this definition is that
the point $a_3$ or the point $b_4$ might not be
defined in case $A$ or $B$ has unbounded tiles.
What happens here is that the relevant ray
simply heads off to infinity without intersecting
an edge of the tiling.  We allow this, and indeed
it might present an interesting case to study, but the
squeamish reader could avoid this problem by
only considering tilings with bounded tiles.
\newline
\newline
{\bf Lack of Transversality:\/}
One might also want to consider the case when
$A$ and $B$ are not transverse.  For instance,
one might like to play this game on a single tiling,
setting $A=B$.  As in symplectic billiards, one
requires that the particles
$a_1$ and $b_2$ are not contained in parallel edges.
\newline
\newline
{\bf Affine Symmetry:\/}
Like symplectic billiards, symplectic tiling
billiards is affinely natural.  If $T$ is an
affine transformation of the plane, then
$T$ maps the orbits relative to the pair
$(A,B)$ to the orbits relative to the pair
$(A',B')$ where $A'=T(A)$ and $B'=T(B)$.
Also, if $T$ is a dilation
then the orbits relative to $(A,B)$ are the
same as the orbits relative to $(A,B')$.

It might be interesting to study symplectic
tiling billiards on tilings which have affine
symmetry.  These are called
affine crystallographic groups.
\newline
\newline
{\bf Half Translation Surfaces:\/}
Symplectic tiling billiards can also be
played on a torus.   This is equivalent to
considering the game relative to a pair
of doubly periodic tilings, and then
considering the orbits on the quotient space.

More generally, one can play the game relative
to a half-translation surface.  Recall that a half-translation
surface is a metric on a surface in which all but
a discrete set of points are locally isometric
to the Euclidean plane and the remaining
points are cone points having cone angle
$\pi k$ for various integers $k$.
One additional requirement for these surfaces
is that there is a global parallel line field.
(This is not quite implied by the other
conditions.)

Let's say that a tiling of a translation surface
is a decomposition of the surface into convex
polygons such that every cone point appears
as a vertex.   Other points might be vertices
as well.  Choices of globally parallel line
fields would give a way to line up the two
tilings.

\subsection{Rotated Square Grids}

Here I make some remarks about some
experiments I did with
symplectic tiling billiards in
the case when $A$ and $B$ are both
square grids.  In this case, the only
parameter is the way $A$ and $B$ are
rotated with respect to each other.

Given
$t \in \Q$ define
\begin{equation}
  \label{RAT}
  z_t=\frac{1-t^2}{1+t^2} + i \bigg(\frac{2t}{1+t^2}\bigg).
\end{equation}
This is the usual rational parametrization of the unit circle.
We normalize so that $A$ is the usual
square grid.
Let $A_t$ denote the result of multiplying the usual
square grid by $z_t$.
The program I wrote uses exact rational
arithmetic to explore this case of
rationally rotated square grids.

For the pair $(A,A_0)$, which is the same
as just playing the game on $A$, the
orbits are just rays.
The choice $t=1/3$ yields $z_t=(4/5)+(3/5)i$.
This is the simplest non-trivial rational case.
It is based on the $(3,4,5)$ right triangle.
Let's take a look.

Figure 4 shows a picture of an orbit
with respect to $(A,A_{1/3})$.
The picture on
the right shows a close-up of the most
complicated part of the orbit on the right.
(I have also continued the orbit a bit further
on the right.)
The slightly thicker segments on the right
are actually unions of extremely close
parallel segments.

\begin{center}
\resizebox{!}{2.5in}{% Figure removed}
\newline
    {\bf Figure 2.2:\/} The left half of an orbit on $(A,A_{1/3})$.
\end{center}

The orbit seems to be bounded and aperiodic.  
I did not attempt to prove this, but
my rational calculations reveal that
the numerators and denominators of the
coordinates of the vertices are
tending to $\infty$.  Geometrically, the orbit has
an attracting limit cycle.  All this
would not be hard to prove; one would
look at the first return map to a suitable
interval, get an interval exchange
transformation, and check that it had
an attracting fixed point.  Here is a concrete
conjecture.

\begin{conjecture}
All orbits on $(A,A_{1/3})$ are bounded and
get attracted to a limit cycle.
The limit cycle itself is a periodic orbit.
\end{conjecture}

Theorem \ref{LR} gives some justification for the
nature of Figure 2.2 above and Figure 2.3 below.

\begin{lemma}
  \label{shoot}
  For a pair of rationally rotated grids, it is impossible
  for an orbit on the left (or on the right) to remain
  forever on the $4$ edges incident to a single vertex.
\end{lemma}

\startproof 
We will suppose that this happens on the left.
Let $V_L$ be the vertex on the left that
the orbit moves around.  As the orbit
on the left side moves around $V_L$ is
must always make a $\pi/2$ degree turn,
either clockwise or counterclockwise.
But then the orbit on the right must forever
hit horizontal and vertical edges in alternation.
But then there is a vertex $V_R$ on the right
such that the orbit on the right stays on the
$4$ edges incident to $V_R$.

Suppose the orbit on the left gets closer to
$V_L$ after $4$ steps.  Then, by symmetry,
the orbit spirals in towards $V_L$ in a
geometric series: the distance to $V_L$
drops by a definite factor $\lambda<1$ at
each revolution.   Theorem \ref{LR} 
says that on the right the distance from
the orbit to $V_R$ increases by $1/\lambda$
after each revolution.  (This result would
be easy to work out by hand in our setting here.)
But then the orbit
on the right eventually escapes and we
have a contradiction.

The other possibility is that the orbit on the
left is periodic.   But then the grids are
rotated by $\pi/4$ degrees relative to
each other.  These are not rationally
rotated grids.
\endproof

Lemma \ref{shoot} gives some explanation for
why the orbits in Figure 2.2 and Figure 2.3 seem
to spiral close to a vertex and then suddenly
shoot out.
\newline

Figure 2.3 shows a picture of an unbounded orbit
on $(A,A_{7/11})$.  I picked this parameter somewhat
randomly.  After making two big spirals, the orbit
starts heading southeast in a periodic pattern with
a drift.  The orbit is clearly unbounded, though I
did not attempt a proof.


\begin{center}
\resizebox{!}{4.5in}{% Figure removed}
\newline
    {\bf Figure 2.3:\/} A symplectic billiard orbit on $(A,A_{7/11})$.
\end{center}

Some parameters seem to support both unbounded orbits
and bounded orbits. It would be nice to classify the
rational parameters according to which kinds of
orbits they support.

For irrationally rotated square grids,
one can sometimes get entirely periodic orbits.
For instance, if we play on $(A,B)$, where $B$ is obtained
by rotating $A$ $\pi/4$ radians, then all orbits are
periodic, and they make squares in each factor.
Compare the end of
the proof of Lemma \ref{shoot}.








\newpage

\section{Sunbursts}

\subsection{Basic Definitions}
\label{mainres}

The goal in this chapter is to prove
Theorem \ref{local} and related results.
An $N$-{\it sunburst\/} is a union of $N$ rays emanating from
the origin such that the convex hull of the rays is the whole plane.
An $N$-sunburst defines a tiling in the plane in which the
tiles are unbounded sectors based at the origin.
In this section we will consider symplectic tiling billiards
with respect to two $N$-sunbursts $A$ and $B$.
The number $N$ is the same for both $A$ and $B$.

We orient the rays of $A$ outward
and the rays of $B$ inward, as shown in Figure 3.1.
(Compare Figure 1.3.)
For the entire chapter, we restrict our attention to
the situation where
we have an orbit that starts with $a_1 \in A_1$ and $b_2 \in B_2$,
so that the particle at $a_1$ points into the sector bounded
by $A_1, A_3$ and the particle at $b_2$ points into the
sector bounded by $B_2, B_4$.  When we say that
$(A,B)$ has periodic orbits, we implicitly mean this kind.
By dilation symmetry, one orbit on $(A,B)$ is periodic
if and only if they all are.

\begin{center}
\resizebox{!}{2.6in}{% Figure removed}
\newline
{\bf Figure 3.1:\/}  A periodic orbit relative to a pair of
$5$-sunbursts
\end{center}

Let $(A,B)$ be a pair of sunbursts.  Call an
orbit $\cal O$ of $(A,B)$ {\it woven\/} if the restriction
of ${\cal O\/}$ to $A$, which we call
${\cal O\/}_A$, circulates counterclockwise
around $A$ and
if (with the obvious notation) ${\cal O\/}_B$
circulates counterclockwise around $B$.
The orbit shown in Figure 3.1 is both
woven and periodic.  In this situation
${\cal O\/}_A$ and ${\cal O\/}_B$ are
both convex polygons.

\subsection{A Criterion for Periodicity}
\label{leftright}

Say that ${\cal O\/}$ is {\it left-convex\/}
(respectively {\it right-convex\/})
if ${\cal O\/}_A$ (respectively ${\cal O\/}_B$) is a closed convex polygon. 
In the first version of this paper, I considered
left-convex orbits but did not inquire as to
whether left-convex orbits were also
right-convex and hence periodic.
However, Jannik Westermann read the first
version of the paper and asked this question.
He  noticed that a woven orbit seems to be
left-convex if and only if it is right-convex.
Jannik gave a geometric proof of this
fact for pairs of $3$-sunbursts.  Subsequently, I found
a proof of the general case. 

    \begin{theorem}
      \label{LR}
      An woven orbit relative to a pair of $N$-sunbursts is left-convex if
      and only if it is right-convex if and only if it is periodic.
    \end{theorem}

    The rest of this section is devoted to proving Theorem \ref{LR}.
    Our proof goes
    through some elementary complex analysis.
    I also used this idea in [{\bf Sch2\/}].
    I'd prefer a geometric proof, but I don't have one.
        \newline
    
A {\it complex $N$-sunburst\/} is an
ordered list of $N$ complex lines through the origin
in $\C^2$.  We will usually drop the word {\it complex\/}
in our discussion. We will be interested in a pair
$(A,B)$ of $N$-sunbursts.  We write these
as $A=A_1,A_3,...,A_{2N-1}$ and
$B=B_2,B_4,...,B_{2N}$.

We say that $(A,B)$ is a {\it good pair\/} if
$A_i \not = B_{i \pm 1}$ for all indices.
Let $(A,B)$ be a good pair.
Given $z_1 \in A_1-\{0\}$ we let
$z_3= B_2' \cap A_3$ where $B_2'$ is
the complex line through $z_1$ parallel to $B_2$.
Since $(A,B)$ is a good pair, we have
$z_3 \in A_3-\{0\}$.  In the same way
we define $z_5=B_4' \cap A_5 $, etc.
This gives us points
$z_7,...,z_{2N-1},z_{2N+1}$.
The ratio
$$\lambda_A=z_{2N+1}/z_1$$
  makes sense because both points
  lie in $A_1$.  Also, by scaling symmetry,
  $\lambda_A$
  is independent of the choice of $z_1$.
  We would get the same value
  if we started with $z_3 \in A_3$ and set
  $\lambda_A=z_{2N+3}/z_3$. Etc.
    Likewise we define $\lambda_B$.
  
  \begin{theorem}
    \label{main}
    We have $\lambda_A \lambda_B=1$ for all good pairs $(A,B)$.
  \end{theorem}

  Theorem \ref{main} applies
  in the real case to the pairs of
  sunbursts considered in Theorem \ref{LR}.
  The corresponding woven orbits are left-convex if and only
  if $\lambda_A=1$ and right-convex if and only if
  $\lambda_B=1$.  But Theorem \ref{main}
  says in particular that $\lambda_A=1$ if and only
  if $\lambda_B=1$.  Thus Theorem \ref{main}
  implies Theorem \ref{LR}.
  Now we prove Theorem \ref{main}.

  Let $f(A,B)=\lambda_A \lambda_B$.
    There exists a good pairs $(A_0,B)_0$ such that
    $f(A,B)=1$.  Take $A_0$ to be the regular $N$-sunburst and
  $B_0$ to be the suitably rotated copy.
  Call two good pairs $(A,B)$ and $(A',B')$ {\it closely related\/}
  if they differ only in the placement of a single line.
  For instance, we might have $A=A'$ and $B_k=B_k'$ except
  when $k=2$.   Any two good pairs can be connected by
  a finite sequence of closely related good pairs.
  In other words, we can get from one pair to the other by
  moving one line at a time.    
  In particular, we can start
  with $(A_0,B_0)$ and then reach an arbitrary good pair
  through a finite sequence of closely related pairs.
  For this reason, the next result implies Theorem \ref{main}.

  \begin{lemma}
    $f(A,B)=f(A',B')$ if $(A,B)$ and $(A',B')$ are
    closely related.
  \end{lemma}

  \startproof
  Given the invariance properties of $\lambda_A$ and $\lambda_B$
  discussed above, it suffices to prove our result in the special
  case already mentioned: $A'=A$ and $B_k=B_k'$ except when $k=2$.
  We can identify the space of complex lines through the origin
  with the Riemann sphere $\C \cup \infty$ in the usual way.
  We are just talking about the complex projective line here.
  Given $\zeta \in \C \cup \infty$ let
  $B(\zeta)$ be the $B$-sunburst obtained by replacing $B_2$ with
  the complex line $B_2(\zeta)$
  corresponding to $\zeta$.  Then there are
  parameters $\zeta,\zeta'$ such that
  $B=B(\zeta)$ and $B'=B(\zeta')$. We define
  $f(\zeta)=f(A,B(\zeta))$.
  
  There are $2$ bad values of $\zeta$ where $f$ is undefined, namely when
  $B_2(\zeta)=A_1$ or $B_2(\zeta)=A_3$.
  Given the nature of the construction,
  $f$ is a holomorphic function of $\zeta$, defined away from
  $2$ points on the Riemann sphere.

  Let us analyze the behavior of $f$ at the two bad values.
  We use the notation $g \sim h$ to denote the statement
  that the ratio $|g/h|$ is uniformly bounded away from
  both $0$ and $\infty$.   Here $g$ and $h$ are functions
which depend on the varying choice of $\zeta$.
  
  Suppose  that $B_2(\zeta)$ makes an angle of $\epsilon$ with $A_3$ and
  $|z_1|=1$.  Then $|z_3| \sim 1/\epsilon$.  The rest of the points
  are not affected much by the change.  This gives
  $\lambda_A \sim 1/\epsilon$.    A similar analysis shows
  that $\lambda_B \sim \epsilon$.  Hence
  $f$ is bounded in a neighborhood of
  the parameter $\zeta$ where
  $B_2(\zeta)=A_3$.  A similar analysis works for the other
  bad parameter.

  Since $f$ is a meromorphic function on the Riemann sphere
  with no poles, $f$ is constant.
  \endproof



  \subsection{Weaves}
    

  Let $(A,B)$ be a pair
  of $N$-sunbursts.
  For an even index $k$, we say that $(A,B)$ is {\it woven at\/} $k$
  if $B_{k}$ is parallel to a vector that starts at some interior point of
$A_{k-1}$ and ends at some interior point of $A_{k+1}$.
We call $(A,B)$ a {\it weave\/} if it is woven at $k$ for all
$k=2,4,...,2N$.
The pair $(A,B)$ in Figure 3.1 is a weave.
Our definition is more symmetric than it looks. 
Let $-B$ denote the sunburst obtained by
reflecting $B$ through the origin.

\begin{lemma}
  \label{switch}
  $(-B,A)$ is a weave if and only if $(A,B)$ is a
   weave.
\end{lemma}

\startproof
If $(A,B)$ is a weave, then
so is $(-A,-B)$.  We are just turning the picture upside down.
So, it suffices to prove the ``if'' direction.
Note that now the
weave property for $(-B,A)$ involves odd indices.

Let $\alpha=A$ and $\beta=-B$. Thus,
our new pair is $(\beta,\alpha)$.
The rays of $\beta$ are oriented
outward and the rays of $\alpha$
are oriented inward.  In particular
$\beta_k=B_k$ and $\alpha_k=-A_k$
for all relevant indices.

\begin{center}
\resizebox{!}{1.4in}{% Figure removed}
\newline
{\bf Figure 3.2:\/}  The relevant points and lines
\end{center}

We show that $(\beta,\alpha)$ is
woven at $3$.   The same argument works
for the other odd indices.
Say that an {\it avatar\/} of a ray is
a vector based at the origin and parallel to
the ray.
We normalize by an affine
transformation so that $A_1$ is the positive
$X$-axis and $A_3$ is the positive $Y$-axis.
Then, since $(A,B)$ a weave, $B_2$ has an avatar that
points into the $(-,+)$ quadrant.
Since $(A,B)$ is a weave, $B_4$ has an avatar
that points into the left halfplane.
Since $B$ is a sunburst with inwardly oriented
rays that progress counter-clockwise around as
the index increases, any avatar of $B_4$ lies
beneath any avatar of $B_2$.


The avatars of $\beta_2$ and $\beta_4$ equal
the avatars of $B_4$ and $B_4$ and
$\alpha_3$ is the negative $X$-axis.  From
what we have said above,
$\alpha_3$ is parallel to a vector connecting
a point on $\beta_2$ to a point on $\beta_4$.
Hence $(\beta,\alpha)$ is woven at $3$.
\endproof


 We call the system $(A,B')$ a {\it phase modification\/} of $(A,B)$
  if $B'$ is obtained by rotating $B$ about the origin by some angle.

\begin{theorem}
  \label{one}
  Let $(A,B)$ be a weave.  Then there is
  a  unique phase modification $(A,B')$ of $(A,B)$ which is a
  weave with periodic woven orbits.
\end{theorem}

The rest of this section is devoted to proving Theorem \ref{one}. 

\begin{lemma}
  \label{swap}
  Suppose $(A,B)$ is a weave.
  Then the orbit of $(a_1,b_2)$ is  woven.
\end{lemma}

\startproof
Since $(A,B)$ is a weave, the ray emanating
from $a_1$ and pointing in the direction of $B_2$ intersects
$A_3$.  Thus $a_3$ is well-defined.   According to our definition
in terms of particle, the transverse vector at $a_3$ points
into the sector bounded by $A_3$ and $A_5$.
Because $(B,-A)$ is also a weave, the ray
through $b_2$ and parallel to $-A_3$ intersects
$B_4$. Thus $b_4$ is well-defined.    According to our definition
in terms of particle, the transverse vector at $b_4$ points
into the sector bounded by $B_4$ and $B_6$.  Continuing
like this, we see that the forward orbit is woven.
\endproof

For each index, there is an open interval of phase modifications which
keep that part of the oriented weave condition.
The intersection $I$ parametrizes the phase modifications that are
weaves.

Suppose that $(A,B)$ is a weave.
Let $\lambda_A(A,B)$ be the function from
\S \ref{leftright}.   Since the orbits are woven, we have
$\lambda_A(A,B) \in (0,\infty)$. These orbits are
periodic iff
$\lambda_A(A,B)=1$.  We identify $I$ with $(0,1)$
and we orient $I$ so that as $t$ increases in $(0,1)$ the
corresponding sunburst $B'=B_t$ rotates clockwise.
We define $\lambda(t)=\lambda_A(A,B_t)$.
We want to see that there is a unique value $t \in (0,1)$ such
that $\lambda(t)=1$.  Theorem \ref{two} follows immediately
from the combination of the next two results.

\begin{lemma}
  \label{monotone}
  $\lambda$ is strictly increasing on $(0,1)$.
\end{lemma}

\startproof
Consider the orbit
$a_1(t),b_2(t),a_3(t),...$ from
Lemma \ref{swap}.  Here our notation
reflects the fact that this orbit dependn $t \in (0,1)$.
Notice that the ratio $\lambda_k(t)=\|a_{k+2}(t)\|/\|a_k(t)\|$
depends only on the triple of rays
$A_k, B_{k+1}(t),A_{k+1}$.  As $t$
increases, the ray $B_{k+1}(t)$ rotates
clockwise.  But then $\lambda_k(t)$ is
strictly  increasing.  Since
\begin{equation}
  \label{prod1}
  \lambda(t)=\lambda_1(t) \times ... \times \lambda_{2k-1}(t),
  \end{equation}
$\lambda(t)$ is also strictly increasing.
\endproof

\begin{lemma}
  $\lambda(t) \to 0$ as $t \to 0$ and
  $\lambda(t) \to \infty$ as $t \to 1$.
\end{lemma}

\startproof
We continue with the notation from the
proof of Lemma \ref{monotone}.
As $\lambda \to 0$, one of the $B$-rays
converges to one of the $A$-rays.
But then for the corresponding
index $k$, the quantity $\lambda_i(t)$ exits
every compact subset of $(0,\infty)$.
Since this quantity is decreasing, we see that
$\lambda_i(t) \to 0$.  At the same time,
the remaining factors in Equation \ref{prod1} do
not increase. Hence $\lambda(t) \to 0$.
The same kind of argument shows that
$\lambda(t) \to \infty$ as $t \to 1$.
\endproof

\subsection{A Calculus Interlude}

In this section we prove  an inequality that we use
in the next section.

\begin{lemma}
  \label{calc2}
  \label{calc3}
  Let $N \geq 4$ and
  $k=4,6,...,N$ and
  $y^*=\sin(\pi(k-2)/(2n))$.
    Then
  \begin{equation}
    \label{messy}
    ((k/2)-1) - y^*(N-(k/2)+1) <0.
    \end{equation}
\end{lemma}

\startproof
The function
$f(t)=\sin(\pi t)-t/(1-t)$ is positive on
$I=(0,1/2)$ because
$f(0)=f(1/2)=0$ and
$$f''(t)=-\pi^2 \sin(t) -\frac{2}{(1-t)^2} - \frac{2t}{(1-t)^3}<0.$$
Now,
let $t = (k-2)/2N$.  Note that
the conditions on $k$ give
$t \in (0,1/2)$.  The positivity of $f$ gives
$$y^* = \sin(\pi t) > \frac{t}{1-t}=\frac{k/2-1}{N-(k/2)+1}.$$
The last equality requires a bit of algebra.  Equation
\ref{messy} is a rearrangement of this inequality.
\endproof

\begin{center}
\resizebox{!}{1.1in}{% Figure removed}
\newline
{\bf Figure 3.3:\/}  A plot of equation \ref{messy} for $N=100$ and $k=4,...,50$.
\end{center}


\subsection{Existence of Weaves}
\label{weaveexist}

Now we prove our restatement of
Theorem \ref{local}.

\begin{theorem}
  \label{two}
  If $A$ is regular and $B$ is balanced, then
  $(A,B)$ has a unique phase modification which is both a
  weave and has woven periodic orbits.
\end{theorem}

By Theorem \ref{one}, it suffices to prove that
$(A,B)$ has a phase modification which is a weave.
Let $S^1$ be the set of all
phase modifications $(A,B')$ of $(A,B)$.  For each even index $k$
there is an interval $I_k \subset S^1$ which parametrizes the
phase modofications that are woven at $k$.  We just need to prove that
$I=\bigcap I_k$ is nonempty.
Define
\begin{equation}
  \mu_N=\pi - \frac{2 \pi}{N}.
\end{equation}

\begin{lemma}
  Each interval $I_k$ has angular length $\mu_N$.
\end{lemma}

\startproof
Choose any point
$p \in A_{2k-1}$.  Then as $q \in A_{2k+1}$
moves from the origin to $\infty$, the vector
$\overrightarrow{pq}$ sweeps out an angle of $\mu_N$.
\endproof

When $B=A$, the
intervals $I_1,...,I_{2k-1}$ all coincide, by symmetry.
If $B$ is a general balanced $N$-sunburst,
we can find a homotopy $t \to B(t)$, through
balanced $N$-sunbursts, such that
$B(0)=A$ and $B(1)=B$.
For all indices $i,j$ we will show that the
relative displacement of $I_j(t)$ with respect to
$I_i(t)$ is less than $\mu_N$.
This means that all pairs of intervals intersect for all $t$.

\begin{lemma}
  Suppose that all the intervals $I_i(t)$ and $I_j(t)$
  intersect for all $t \in [0,1]$.  Then
  all the intervals $\{I_k(1)\}$ have a common intersection.
\end{lemma}

\startproof 
Recall a case of Helly's Theorem:
If a finite collection of open intervals in $\R$ pairwise
intersect, then their intersection is nonempty.  We think
of $\R$ as the universal cover of $S^1$.
We can lift our intervals to $\R$, so that for
$t=0$ they are all the same interval and the
lifts vary continuously with $t$.  But then the result about
relative displacement still holds, and all pairs of lifted
intervals intersect for all $t \in [0,1]$.  By Helly's Theorem,
all the lifted intervals intersect (in particular) for $t=1$.
Pushing the intersection point
down to $S^1$, we see that all the intervals
$\{I_k(1)\}$ also intersect.
\endproof

To finish our proof of Theorem \ref{two} we need to establish
our claim about the relative displacement of pairs of intervals.
Without loss of generality, it suffices to consider the
intervals $I_0(t)$ and $I_k(t)$ for $k=2,4,...,2N-2$.
The relative displacement of our intervals does not change if
we post-compose our homotopy $B(t)$ with an arbitrary
continuous family of rotations. So, it suffices to
consider the case when $A_1$ is the positive $X$-axis
and $B_0(t)$ is the positive $X$-axis for all $t \in [0,1]$.

Let $\theta_k(t)$ denote the angle between $B_k(t)$ and
the positive $X$-axis.  We have $\theta_0(t)=0$ for all $t$
and $\theta_k(0)=\pi k/N$ for all $k=0,2,4,...,2N-2$.
The relative displacement between $I_k(1)$ and $I_k(0)$ is
$|\theta_k(1)-\theta_k(0)|$.  We just have to prove that
this quantity is less than $\mu_N$ for all even $k$.

  \begin{lemma}
    $\theta_k(1)-\theta_k(0)<\mu_N$ for all $k=2,...,2N-1$.
  \end{lemma}

  \startproof
  We will assume that there is an index $k$ such that
  $\theta_k(1)-\theta_k(0) \geq \mu_N$ and
  derive a contradiction.  By assumption
  $$\theta_k(1) \geq \theta_k(0) + \pi - \frac{2\pi}{N} =
  \pi + \frac{k \pi}{N} -\frac{2 \pi}{N} =
  \pi +\frac{(k-2)\pi}{N}.$$
  

  We first treat two special cases.
  \begin{itemize}
  \item When $k=2$ we have $\theta_2(1) \geq \pi$.
    This contradicts the fact that $\theta_2(1)<\pi$ is the angle
    between $B_0(1)$ and $B_2(1)$.
\item   When $k \geq N+2$ we have
  $\theta_k(1) \geq 2\pi$.  This contradicts the fact that
$\theta_k(1)<2\pi$ for $k=2,4,...,2N-2$.
\end{itemize}

So, we only need consider the cases
$k=4,6,...,N$.  Compare Lemma \ref{calc2}.
  Let $\beta_j$ denote the unit vector along
  $B_j(1)$.  We rotate the picture about
  the origin so that $\beta_0$ and $\beta_k$ have
  the same negative $y$-coordinate, namely
  $-y^*$, the quantity from Lemma \ref{calc2},
  with $\beta_0$ being on the right and
  $\beta_k$ being on the left.
  The $(k/2)-1$ unit vectors $\beta_2,...,\beta_{k-2}$
  all have $y$-coordinates less or equal to $+1$.  The remaining
  $N-(k/2)+1$ unit vectors,
  being between $\beta_k$ and $\beta_0$ in the
  circular order, all have $y$ coordinates less or
  equal to $-y^*$.   The average of the $y$-coordinates
  of all $N$ unit vectors is at therefore at most
  $$(k/2-1) - y^* (N-(k/2)+1)<0.$$
  The inequality comes from Equation \ref{messy}.
  This contradicts the fact that $B(1)$ is balanced.
  \endproof
  

  \begin{lemma}
    $\theta_k(0)-\theta_k(1)<\mu_N$ for all $k=2,...,2N-1$.
  \end{lemma}

  \startproof
  This follows from the previous result and reflection symmetry.
  Let $\overline B$ denote the balanced $N$-sunburst obtained by
  reflecting $B$ in the $X$-axis and dihedrally relabeling the rays
  so that they again go counterclockwise around the origin.  Applying
  the previous result to $\overline B$ and we get
    $\overline \theta_k(1)-\overline \theta_k(0)<\mu$ for all
  $k=2,..,2N-2$.  The homotopy we take is just the original homotopy
  conjugated by reflection in the origin, with the points dihedrally relabeled.
  At the same time, we have
  $\theta_k(t)=2\pi - \overline \theta_{2N-k}(t).$
  Hence
    $$\theta_k(0)-\theta_k(1) =
  -\overline \theta_{2N-k}(0)+\overline \theta_{2N-k}(1)=
  \overline \theta_{2N-k}(1)-\overline \theta_{2N-k}(0)<\mu.$$
This completes the proof.
  \endproof


  \noindent
  {\bf Remark:\/}
  One might wonder about relaxing the hypotheses of
  Theorem \ref{two}.   For instance, is the result true
  for $(A,B)$ if both $A$ and $B$ are regular?    In the
  even case, at least, one needs extra hypotheses.
  When $N$ is even and $A$ is not centrally symmetric,
  there always exists a centrally symmetric (and hence balanced)
  sunburst $B$ such that $(A,B)$ has no phase
  modification which is a weave.  The sunburst $B$
  consists of two clusters of $N/2$ nearly identical lines
  which are diametrically opposed from each other.
  We leave the details of this to the interested reader.
  In the odd case I have not been able to think of an
  easy counter-example like this.
  
  


\newpage

\section{Correspondence between Polygons}

\subsection{The Main Construction}

We consider planar equilateral polygons in which all
the sides have the same length.  We also consider
planar equiangular polygons in which all the interior
angles are the same.  We consider these polygons
up to the equivalence of orientation preserving
similarity.  In both cases we only consider strictly
convex polygons which are oriented counterclockwise
around the regions they bound.  In the equilateral
case we normalize so that the sides have
unit length.  In the equiangular case we 
normalize so that the sides are parallel to the
relevant roots of unity.
\newline
\newline
\noindent
{\bf Main Construction:\/}
Let $L$ be a strictly convex $N$-gon with unit vector edges
$\beta_2,\beta_4,...,\beta_{2N}$.  Since $L$ is closed, we have
$\beta_2+ ... + \beta_{2N}=0$.
Let $B_k$ be the ray through the origin that starts at $0$ and
contains $\beta_k$.
The fact that $L$ is strictly convex means that
$B_2,...,B_{2N}$ is an $N$-sunburst.
By construction, $B$ is a balanced sunburst.
A different representative of $[L]$ would give rise
to a balanced sunburst $B'$ which is a rotation of $B$.
Let $A$ be the regular $N$-sunburst.
Applying Theorem \ref{two}, we know that
there is a unique phase modification
$(A,B')$ which is a weave and which has periodic orbits.
We choose any periodic orbit and consider the
convex $N$-gon $P_L$ which is the image that
the orbit traces around $B$.   Our association
is \begin{equation}
  \label{assoc}
  [L] \to [P_L].
\end{equation}
The labeling is such that the $k$th vertex of $P_L$ lies
on the edge $B_k$.  
By construction, this association
is well-defined, independent of choices.
\newline

\noindent
{\bf Remarks:\/}
(1)   
The construction above only works in the
strictly convex case, but we discuss the general
case in \S \ref{all}.
\newline
(2)
Spherical duality interchanges equiangular and equilateral
spherical polygons.   One might wonder if one could get a
different correspondence like Equation \ref{assoc} by taking
a limit of this process as the sphere radius tends to infinity.
(One of the referees posed this question.)
I think that this probably will not work.
Consider what happens when we have
a unit-sized regular polygon on a growing sphere. Then
the diameter of the dual polygon tends to infinity and
it seems impossible to extract a Euclidean limit.
The same problem would happen for other polygons.

\subsection{Computing the Correspondence}
\label{compute}

In the introduction we claimed that the correspondence in
Equation \ref{assoc} is easy to compute.   Figures 1.3 and
3.1 come from our computer program which does the
computations.  Here we discuss the method.
Once we have the pair $(A,B)$ we find the intervals.
$I_1,...,I_N$ where $I_k \subset S^2$ is the interval
of rotations $R_{\theta}$ such that
$(A,R_{\theta}(B))$ is $k$-intervoven. See \S \ref{weaveexist}.

Next, we compute $I=\bigcap I_k$.   We have the
map $h: I \to \R_+$ which
computes the holonomy $h(\theta)$
of the orbits associated
to the pair $(A,R_{\theta}(B))$ for $\theta \in I$.
We want to solve the equation $h(\theta)=1$.
We use the bisection method.  We choose
two values $\theta_1,\theta_2$ respectively
very near the two boundary points of $I$.
We then perform the following bisection algorithm.
\begin{enumerate}
  \item Start with a pair $(\theta_1,\theta_2)$ such that
    $h(\theta_1)<1<h(\theta_2)$.
  \item Let $\theta_3=(\theta_1+\theta_2)/2$.
  \item If $h(\theta_1)>1$ replace
    $(\theta_1,\theta_2)$ with
    $(\theta_1,\theta_3)$.
  \item If $h(\theta_1) < 1$ replace
            $(\theta_1,\theta_2)$ with
    $(\theta_3,\theta_2)$.
  \item Return to Step 1 using the smaller angle  interval.
    \end{enumerate}
      If we iterate this $M$ times we get the
      correct value of $\theta$ up to about
      $2^{-M}$.
      Once we have our good approximation of
      $\theta$ it is a simple matter to actually
      compute the symplectic tiling orbit.
      \newline
      \newline
      {\bf Remark:\/}
The construction in [{\bf KM\/}] starts with
an equiangular $N$-gon, and then take the Riemann map
to the unit disk,  This gives them $N$ unit complex numbers
$u_1,...,u_N$.
They then use the fact (Springborn's Theorem [{\bf Sp\/}]) that there is
a unique point in the hyperbolic plane such that a
Moebius transformaton $M$ mapping this point to the origin
carries $u_1,...,u_n$ to unit complex numbers whose sum is $0$.
(This part of the construction is similar to ours.)
The numbers $u_1',...,u_n'$ are then interpreted as the
direction vectors of a unit equilateral $N$-gon.
Here we have set $u_k'=M(u_k)$.  This construction
produces a unique equilateral $N$-gon up to scale.
This method requires
      something like the computation of the Riemann map.
      There are several methods for doing this for the
      case at hand.  Concretely, one could numerically
      integrate the Christoffel transform.   Alternatively,
      one could use elegant circle-packing methods
      [{\bf St\/}].  All these methods seem very
            computationally involved.
      
\subsection{Bijective Nature of the Correspondence}

We fix some integer $N \geq 3$ and consider
all our constructions relative to $N$.  In this section
we prove that our association in Equation \ref{assoc} is
a bijection.   We explain first how to construct the
inverse map and then we justify the assumptions needed
for the construction to work.

Let $[P]$ be an equivalence class of convex equiangular $N$-gons.
Let $P$ be some representative.  Let
$P_2,P_4,...,P_{2n}$ be the vertices of $P$.
Say that a point $p$ in the region
bounded by $P$ is a {\it balance point\/} if the sum
$\sum \beta_k=0$, where $\beta_k$ is the unit vector parallel
to $P_k-p$.    We will show below that there is a unique
balance point.  This result is similar in spirit to
Springborn's Theorem [{\bf Sp\/}].

Given our balance point $p$, we
let $B$ be the balanced sunburst defined by
the vectors $\beta_2,...,\beta_{2n}$ we have just defined.
Finally, let $L$ be the equilateral $N$-gon whose successive edges are
parallel to these vectors.  The balance condition guarantees
that $L$ is closed and the sunburst condition guarantees
that $L$ is strictly convex.
Our construction shows how to recover $[L]$ from $[P_L]$, and
this shows that our association is injective.
At the same time, our construction shows that our association
is surjective.  Hence, our association in
Equation \ref{assoc} is bijective.

\begin{lemma}
  A strictly convex $N$-gon has at most one balance point.
\end{lemma}

\startproof
For this proof we do not use the equiangular property.
We will suppose this is false and derive a contradiction.
Let $P$ be a strictly convex polygon which supposedly
has at least $2$ balance points.  We rotate the picture so
that both balance points $p_1,p_2$ lie on the $X$-axis
and $p_1$ is on the left.

\begin{center}
\resizebox{!}{1.7in}{% Figure removed}
\newline
{\bf Figure 4.1:\/}  The angles with the $X$-axis.
\end{center}

For each $i=1,2$ let $\{\beta_{ij}\}$ denote the set of
  unit complex numbers that are parallel to the
  vectors $P_j-p_i$.  The basic property is that
  the angle that the vector $\beta_{1j}$ makes with the
  $x$-axis is less than the angle that $\beta_{2j}$ makes
  with the $x$-axis.   Figure 4.1 shows this in action.
  From this we see that the center of mass of
  $\{\beta_{2j}\}$ must lie to the left of the center
    of mass of $\{\beta_{1j}\}$, contradicting the claim
    that both centers of mass are the origin.
    \endproof



    \begin{lemma}
      \label{exist}
  A strictly convex equiangular $N$-gon has a balance point.
\end{lemma}

\startproof
This proof  uses the equiangular property.
When $N=3$ the polygon must be an equilateral
triangle, and then the center of symmetry does the job.
Likewise, when $N=4$ the polygon must be a rectangle,
and again the center of symmetry does the job.  So,
we take $N \geq 5$.

Let $P$ be a strictly convex equiangular $N$-gon.   We consider a
simply connected domain $D$ in the plane as follows.
We start with the closure of the domain bounded by $P$ and then
we chop off small isosceles triangular
neighborhoods of the vertices of $P$.
Figure 4.2 shows the picture.
The boundary of $D$ is a convex $2N$-gon which equals
$P$ at most places but then takes small ``shortcuts'' into
the interior of the region bounded by $P$ near the vertices. 


For each $p \in D$ we consider the vector
$V_p = \sum \beta_{p,j}$, where $\beta_{p.j}$ is
the unit vector parallel to the one pointing from $p$ to $P_j$.
We are looking for a place where $V_p$ vanishes.
If suffices to prove that $V_p$ points inward at
$\partial D$.  (As the proof develops, we will explain
more precisely what this means.)
The boundary $\partial D$ has two kinds of points, those which also
lie in $P$ and those which do not.
We consider these kinds of points in turn.

\begin{center}
\resizebox{!}{1.4in}{% Figure removed}
\newline
{\bf Figure 4.2:\/}  The domain $D$ and two kinds of boundary points
\end{center}



The point $p$ in Figure 4.2 also lies in $P$.
For points like $p$, all but two of the
vectors $\beta_{p.j}$
point into $D$ and the other two point along $\partial D$.
Let us say this a bit more formally.
The $p$ lies in an edge $e_p$ of $P$,
and $e_p$ is contained in a line $L_p$. The line $L_p$ bounds a
halfplane $H_p$ that contains the other vertices of $P$, and
$\beta_{p,j}$ points into $H_p$ for all but $2$ vertices.
For the other two vertices, namely the vertices of $e_p$,
the two unit vectors in question point along $L_p$.
Adding up all these unit vectors, we see that $V_p$
points into $H_p$.  This is to say that $V_p$ points
into $D$.

The point $q$ in Figure 4.2 does not lie in $P$.
We rotate the picture so that the edge containing
$q$ is vertical, as shown in Figure 4.2.
We also relabel so that $q$ is near $P_1$.
The $x$-coordinate of $\beta_{p,1}$ is at most $1$.
The remaining vertices lie on lines which make an
angle of at least $\pi/N$ with the vertical line
through $P_1$.  Here we are using the equiangularity
condition.     But this means that the sum of the
$x$-coordinates of our remaining vectors is
at most
\begin{equation}
  \label{negsum}
  -(N-1) \times (\sin(\pi/N) - \epsilon).
  \end{equation}
Here $\epsilon$ is a number can make as small as
we like by controlling the size of the isosceles
neighborhoods we used in defining $D$.
As long as we take $\epsilon$ sufficiently small,
the number in Equation \ref{negsum} is
always less than (meaning more negative than)
$-2$.    Hence the vector $V_q$ has negative
$x$-coordinate.  This shows that $V_q$ points
into $D$.

Now we know that our vector field $p \to V_p$ is inward-pointing
on $\partial D$.  A well known result about the index of vectorfields
now shows that $V_p$ vanishes somewhere in the interior of $D$.
Hence $P$ has a balance point.
\endproof


\subsection{Algebraic Nature of the Correspondence}
\label{algXX}


In this section we explain the sense in which the
association in Equation \ref{assoc} is algebraic.
The result here feeds into the proof of Theorem
\ref{algebra}.  

\begin{lemma}
  \label{alg}
  If $[L]$ is algebraic then $[P_L]$ is algebraic.
\end{lemma}

\startproof
When $L$ is algebraic, the rays $B_2,B_4,...,B_{2N}$
are also algebraic.  For instance, they have
algebraic slopes.  The holonomy $\lambda$
of an orbit
associated to $(A,B)$ is an algebraic function of
these slopes.    To understand the phase-modification
part of the construction we choose a rational
parametrization of the circle, as in
Equation \ref{RAT}.    For each $t \in \R \cup \infty$
the slopes of the $t$-rotated sunburst $B'$ are
rational functions in $t$, with algebraic coefficients.

We think of the holonomy $\lambda(t)$ as a function of
the rotation parameter $t$.  The function $\lambda(t)$ is
also a rational function with algebraic coefficients.
Setting $\lambda(t)=1$ and solving, we see that
the choice of $t$ which makes $(A,B')$ have periodic
orbits is algebraic.  But then if we scale so that one of the
vertices of $P_L$ is algebraic then all the vertices will
be algebraic.
\endproof


\begin{lemma}
  \label{algebra2}
  $[P_L]$ if and only if $[L]$ is algebraic.
  \end{lemma}

  \startproof
  In view of Lemma \ref{alg}, we just need to prove
  that $[L]$ is algebraic when $[P_L]$ is algebraic.
  Let $P=P_L$ be an algebraic representative.
  We claim that the balance point of $P$ is algebraic.
  (I am grateful to Joe Silverman for supplying the proof.)

We will use complex notation.
Let $P= (p_1,...,p_N) \in \C$. 
For each sequence
$\epsilon=(\epsilon_1,...,\epsilon_N) \in \{\pm 1\}^N$ define
\begin{equation}
  F_{\epsilon}(P,z)=\sum_{i=1}^N \epsilon_i \frac{z-p_i}{\overline z-\overline p_i} = 0.
\end{equation}
The balance point solves the equation
$F_{\epsilon}(P,z)$ when $\epsilon=(1,...,1)$.
The product
\begin{equation}
  G(z,p)=\prod_{\epsilon \in \{\pm 1\}^N} F_{\epsilon}(z,p)
\end{equation}
is unchanged if we change the signs of any subset of the
square roots in the last equation.   Hence $G(P,z) \in \Q(P,z)$,
the ring of rational functions in $z$ and $P$.  Hence
the roots of $G(P,z)$, for algebraic $P$, are also algebraic.
The balance point is one such root.

Since the balance point is algebraic,
the rays describing the $B$ sunburst are
algebraic.  So, if we start with $a_1$ and $a_2$ algebraic,
the whole orbit remains algebraic.
\endproof

\noindent
{\bf Remark:\/}
The algebraic structure of the balance point might be
quite complicated.  Consider the modest example of an
integer pentagon with vertices
$$(0,0), \hskip 15 pt
(1,0), \hskip 15 pt (2,2), \hskip 15 pt
(1,2) \hskip 15 pt (0,1).$$
Peter Doyle played around with this
in Mathematica and found that the minimal polynomial
for the first coordinate of the balance point has
degree $48$ and the coefficients mostly have about
$30$ digits.


\newpage
















\section{Hyperbolic Structure}

\subsection{The Thurston Construction}
\label{hypXX}

William Thurston's paper [{\bf T\/}] constructs
complex hyperbolic structures on spaces of
flat cone spheres. See my notes [{\bf Sch1\/}]
for an exposition of [{\bf T\/}].
A special case of a flat
cone sphere is the double of a convex equiangular
polygon.   The corresponding subspace sits as a
totally real slice of the convex hyperbolic moduli
space.  This imparts a real hyperbolic structure
on the space of convex equiangular $N$-gons.

In this section I will give an elementary account of
the construction which does not go through complex
hyperbolic geometry. It is possible that I learned this
construction from Thurston when I was a graduate
student at Princeton University and it is also possible
that I worked it out myself sometime later.
There are a number of similar
accounts in the literature.  See e.g.
[{\bf BG\/}] for the general case and
[{\bf Cal\/}] for the pentagonal case.
\newline
\newline
{\bf Linear Coordinates:\/}
We start
with $N$ parallel familes of lines, with each family being
parallel to a different $N$th root of unity.  These families
are cyclically ordered, according to the roots of unity.
We interpret an equiangular $N$-gon as a selection
$\ell_1,...,\ell_n$ of lines, one from
each family.  The vertices of the $N$-gon
are given by $\ell_1 \cap \ell_2$, $\ell_2 \cap \ell_3$, etc.
This interpretration gives a natural identification
of the space of equiangular $N$-gons with $\R^N$.
To get a concrete coordinatization we could pick
some line $L$ in the plane, not parallel to any of
the families, and then use the intersection
$\ell_1 \cap L,...,\ell_N \cap L$ give $N$
linear coordinates.  In other words, we
are identifying $\R^N$ with $L \times ... \times L$.
A different choice of $L$ would give us a linear
change of coordinates.

We now mod out by translations.  This identifies
the space of equiangular $N$-gons mod isometry
with $\R^{N-2}$.   Figure 5.1 shows, in the pentagon
case, how we can introduce concrete coordinates
on $\R^{N-2}$ which are linear functions of
the coordinates discussed above.

\begin{center}
\resizebox{!}{1.2in}{% Figure removed}
\newline
{\bf Figure 5.1:\/}  Coordinates on the space of equiangular pentagons
\end{center}


It is important to emphasize that these coordinates
$A,B,C$ are signed distances.  They look nice in the
convex case but they are defined even in the non-convex
cases.  For instance, in vector notation,
$$A=\big((\ell_2 \cap \ell_5) - (\ell_3 \cap \ell_5) \big)\cdot (1,0).$$
Also, it is important to note that any system of coordinates based
on a similar contruction (with other choices) would result in
a linear change of variables implemented by a matrix
with algebraic entries.

Figure 5.2 shows similar coordinates for the case of hexagons
and $7$-gons.  We had to make some choices to get these
coordinates, but any similar system
would be related by a change of coordinates implemented by
an algebraic matrix.


\begin{center}
\resizebox{!}{1.25in}{% Figure removed}
\newline
{\bf Figure 5.2:\/}  The case of hexagons and $7$-gons.
\end{center}

\noindent
{\bf The Signed Area:\/}
Now we consider the signed area in these coordinates.
In the pentagonal case we have
\begin{equation}
  \label{area}
  {\rm area\/} = -\alpha A^2 - \beta B^2  + \gamma C^2,
\end{equation}
where $\alpha,\beta,\gamma$ are positive constants that
do not depend on the choice of pentagon.
Geometrically, the area of the pentagon is the
area of the big triangle minus the area of the two small
triangles. The area of the big triangle is a quadratic function
of $C$ and the constant only depends on the shape of the
triangle.  Likewise the areas of the smaller two triangles are
quadratic functions in $A$ and $B$ with the same properties.

For hexagons and $7$-gons, Equation \ref{area} would
respectively have $4$ and $5$ quadratic terms with
constant coefficients and all but one being negative.
In the general case there would be $N$ quadratic terms with
constant coefficients with all but one being negative.
Speaking more abstractly,
the area of the $N$-gon given by
$V=(A,B,C,D,...)$ has the form
$Q(V,V)$ where $Q$ is a quadratic form of signature $(1,N-3)$.
\newline
\newline
{\bf The Lorentz Model:\/}
Now, we are interested in the space of {\it equivalence classes\/} of
equiangular $N$-gons.  Up to translation, we can get a unique
representative of each equivalence class by scaling so that the
area is $1$.  But then we can identify our space of equivalence
classes with one sheet of the hyperboloid in $\R^{1,N-3}$ given by
$Q(V,V)=1$.   This is a well-known {\it Lorentz model\/}
of $\H^{N-3}$, hyperbolic
space of dimension $N-3$.
\newline
\newline
{\bf Interaction with Convexity:\/}
Let ${\cal C\/}_N$ be the domain in
$\H^{N-3}$ corresponding to the
space of strictly convex equiangular $N$-gons.
In general, ${\cal C\/}_N$ is the interior of a convex polyhedral
domain.   The points on the boundary of
${\cal C\/}_N$ correspond to
degenerate polygons in which one or more
edge has collapsed to a point.

For instance, in the case of pentagons,
the boundary of ${\cal C\/}_5$ has
$5$ edges and $5$ vertices.
Referring to Figure 5.1, two of the edges
correspond to $B=0$ and $C=0$, and their
vertex intersection corresponds to $B=C=0$.
There is an elegant way to see the geometry
of ${\cal C\/}_5$.
Figure 5.3 shows the
{\it butterfly move\/} $B_2$ for pentagons.

\begin{center}
\resizebox{!}{3in}{% Figure removed}
\newline
{\bf Figure 5.3:\/}  The butterfly move $B_2$.
\end{center}

The $5$ fixed lines in the hyperbolic
plane $f_1,f_3,f_5,f_2,f_4$ are consecutively
perpendicular in the cyclic sense.
The reason they are perpendicular is that
the corresponding butterfly moves commute! 
These fixed lines are the extensions
of the edges of a regular right-angled pentagon.
The interior of this pentagon is the space of
convex unit equiangular pentagons modulo
similarity.

The case of hexagons is also possible to understand.
In this case ${\cal C\/}_6$ has $6$ faces and $5$
vertices.  Two of the vertices lie in $\H^3$.  These
correspond to the two equilateral triangles we get
by collapsing either the even or the odd edges of
our hexagon.   The other three vertices are
ideal vertices.  These correspond to the hexagons one
gets by letting a pair of opposite sides get very long.
Figure 5.4 shows what we mean.

\begin{center}
\resizebox{!}{.9in}{% Figure removed}
\newline
{\bf Figure 5.4:\/}  Hexagons near each of the $5$ vertices of ${\cal
  D\/}_6$.
\end{center}

Whenever two of these faces intersect, the corresponding
butterfly moves commute. Thus, all the faces which meet
do so at right angles.  The domain ${\cal C\/}_6$ in fact
is the interior of a triangular bi-pyramid.  Each half of the
triangular-bi-pyramid is obtained by coning one face of
a regular ideal octahedron to the center of mass.  This
half is a pyramid, one of whose faces corresponds to the
octahedron.  Call this the {\it blue face\/}.   Call the
other faces the {\it red faces\/}.  The red faces meet at
right angles and each red face meets the blue face at
an angle of $\pi/4$.  We get ${\cal C\/}_6$ by gluing
two of these pyramids together across their blue faces
and we are left with the $6$ red faces.


In general, if $P=\ell_1,..,\ell_N$ then
$B_k(P)=\ell_1',...,\ell_N'$ where
$\ell_j'=\ell_j$ for all $j \not = k$ and
the two lines $\ell_k, \ell_k'$ are equidistant
from the intersection $\ell_{k-1} \ell_{k+1}$.
The operation $B_k$ is linear and preserves
signed area.  Hence $B_k$ is a Lorenz transformation
and induces a hyperbolic isometry on the hyperbolic
structure we have explaned.  Moreover $B_k$ is an
involution.  The fixed point set of $B_k$ is the set of
all degenerate polygons in which $\ell_{k-1}, \ell_k,\ell_{k+1}$
have a common point.  This is a codimension one set.
Hence $B_k$ is a hyperbolic reflection.
Note that $B_{a}$ and $B_b$ commute as long as
$a,b$ are not cyclically consecutive.  The corresponding fixed
point sets are perpendicular hyperplanes.




\subsection{Putting it Together}
\label{put}

For each $N \geq 5$, the Thurston construction produces an open
polyhedral convex domain ${\cal C\/}_N \subset
\H^{N-3}$ whose interior
parametrizes the equivalence classes of
strictly convex equiangular $N$-gons.
Now we get to the punchline.
Using our correspondence
from Equation \ref{assoc}
we get the same hyperbolic structure on the space of
equivalence classes of strictly convex equilateral polygons.
\newline
\newline
\noindent
{\bf Proof of Theorem \ref{algebra}:\/}
Suppose that $L$ is an equilateral polygon with algebraic vertices.
Then by Lemma
\ref{algebra2} we can find a representative $P_L$
having algebraic vertices.   When we
scale $P_L$ to have unit area we are
scaling by an algebraic number, so we can
take $P_L$ to have unit area.
But then our special coordinates for $P_L$ are
also algebraic.  Any choice of special coordinates
would have this property, because they all differ
by the action of an algebraic linear matrix.
So, the coordinates of $P_L$ in Lorentz space
$\R^{1,N-3}$ are algebraic.  This is the same
as saying that the coordinates in $\H^{N-3}$ are algebraic.
Conversely, if $P_L$ has algebraic coordinates in
$\H^{N-3}$, then we can take a representative of
$P_L$ such that the special coordinates taken above
are all algebraic and one of the vertices is algebraic.
But then when we reconstruct $P_L$ from a single
vertex and from the coordinates we get an algebraic
polygon.  But then, by Lemma \ref{algebra2} again,
$[L]$ is also algebraic.
\endproof

\noindent
{\bf Remark:\/}
Finding 
the coordinates in $\R^{1,N-3}$ of a given class of
equiangular polygon is straightforward.
We just choose our coordinates as above and
compute. Thus, in view of the discussion in
\S \ref{compute}, it is easy to accurately estimate
the point in $\H^{N-3}$ which parametrizes a
given equilateral $N$-gon.

\subsection{Beyond the Convex Case}
\label{all}

Now we consider general equilateral polygons.
First of all, we widen our equivalence relation so
that two polygons are equivalent if and only if
there is a similarity which maps one to the other.
The similarity here need not be orientation preserving.
If we restrict our attention to the strictly convex case,
this widening of the equivalence relation changes
nothing, because the counterclockwise-oriented convex
polygons we have been considering above
are equivalent in the wider sense if and only if
they are equivalent in the narrow sense.

  The group $S_N$ of permutations
  acts naturally and continuously on the moduli space of equilateral $N$-gons:
We can encode an equilateral $N$-gon by an ordered list
$e_1,...,e_N$ of unit vectors.   Given a permutation
$\pi \in S_N$ we get the new list
$e_{\pi(1)},...,e_{\pi(N)}$ of edges, and we can build a
unique equivalence class of $N$-gon which corresponds
to this list.  The only thing we need from our vectors
is that they sum to zero, and this is unchanged by
permutation.  Figure 5.5 shows this in action for the
regular pentagon.

\begin{center}
\resizebox{!}{1.2in}{% Figure removed}
\newline
{\bf Figure 5.5:\/}  Regular pentagon permutations
\end{center}

We call an $N$-gon  {\it generic\/} if some
permutation makes it strictly convex. The set of
generic $N$-gons is open and dense, and invariant
under the action of $S_N$.
Topologically this subset is homeomorphic to
\begin{equation}
  f(N)=N!/(2N)
\end{equation}
copies of ${\cal C\/}_N$.   The reason why
$f(N)$ has the form it does is that the dihedral group,
which has order $2N$, permutes the convex classes.
We give a hyperbolic
structure to the subset of generic equilateral $N$-gons,
declaring it to be a disconnected union of $f(N)$
copies of ${\cal C\/}_N$.   By construction, $S_N$ acts
isometrically on our big space.  We call the
component corresponding to the strictly convex $N$-gons
the {\it convex component\/}.

We now enlarge
our space by taking the closures of all the components.
At the moment we still have a disjoint union of
hyperbolic polyhedra whose union (redundantly)
parametrizes all the equilateral $N$-gons.
Finally, we form an identification space by identifying
points in our union which represent the same
(equivalence class of) $N$-gon.  This is our hyperbolic structure on
the moduli space of equilateral $N$-gons.
We denote it by ${\cal A\/}_N$.
\newline
\newline
{\bf Remark:\/}
Technically, when $N$ is even, we have to
add to ${\cal A\/}_N$ the ideal points.  These
correspond to $N$-gons which lie in a single line.
For $N=6$ there are $10$ such.


\subsection{Pentagons}

In this section we explore
${\cal A\/}_5$ and prove Theorem \ref{penta}.
Before taking the quotient, we have
$12$ disjoint copies of ${\cal C\/}_5$, which is
a regular right angled hyperbolic pentagon.  These
pentagons are then glued edge-to-edge.
It turns out that $4$ are glued around each
vertex.  Figure 5.6 illustrates this for one
of the vertices of $C$, the copy of
${\cal C\/}_5$ that corresponds to
the convex pentagonal linkages. 
(I use the word {\it linkage\/} in this section to
avoid confusion;
the moduli space is also composed of pentagons.)
The vertices of
$C$ are certain isosceles triangles, in which
two pairs of consecutive edges point in the
same direction.
       
\begin{center}
\resizebox{!}{2.2in}{% Figure removed}
\newline
{\bf Figure 5.6:\/}  The local picture around a vertex.
\end{center}

The left panel of Figure 5.6 shows the vertex linkage.
The middle  panel shows the $4$ kinds of linkages which
contribute to the pentagons which glue together
around the vertex.  The right panel shows a hand-drawn
approximation of how the corresponding $4$ pentagons
would fit together if developed into the hyperbolic plane.
The subgroup generated by
the transpositions $(12)$ and $(34)$ fixes the
vertex and permutes the $4$ pentagons around it.
What makes this work is that $(12)$ and $(34)$
generate the dihedral subgroup of order $4$.

By symmetry, the picture is the same at every
vertex of our space.  Thus, the space we
get has a global hypebolic structure: It is isometric
to a very symmetric hyperbolic surface $\Sigma$.
The surface $\sigma$ has $12$ faces, and
$12 \times (5/2)=30$ edges and
$12 \times (5/4)=15$ vertices.  Hence
the surface has Euler characteristic
$\chi(\Sigma)=-3$.  
Given the classification of surfaces,
we can identify $\Sigma$ topologically as
the connected sum of a genus $2$ surface and
a projective plane.


\subsection{Hexagons}
\label{hexa}

In this section we explore ${\cal A\/}_6$ and
prove Theorem \ref{hex}.

Figure 5.7 shows the $5$ hexagons which
correspond to the $5$ vertices of ${\cal C\/}_6$.
The triangles correspond to vertices n
$\H^3$ and the segments correspond to
ideal vertices.

\begin{center}
\resizebox{!}{1.2in}{% Figure removed}
\newline
{\bf Figure 5.7:\/}  The vertics of ${\cal C\/}_6$.
\end{center}

The leftmost vertex is stabilized by the
order $8$ group $(\Z/2)^3$ generated by
the permutations $(12)$ and $(34)$ and $(56)$.
Thus, $8$ copies of ${\cal C\/}_6$ fit around
this vertex. Given the right-angles involved, the
identification space ${\cal A\/}_6$
is locally isometric to
$\H^3$ even along the vertices and edges.
The second triangular vertex has the same
kind of story, except that now the
permutations involves are $(23)$ and $(45)$ and $(61)$.

Figure 5.8 shows the hexagons corresponding to points
along the edge of
${\cal C\/}_6$ which connects the first two of the
ideal vertices shown above.

\begin{center}
\resizebox{!}{1.7in}{% Figure removed}
\newline
{\bf Figure 5.8:\/}  An edge of
${\cal C\/}_6$ connecting two ideal vertices.
\end{center}

This edge is stabilized by the order $4$ group
generated by $(23)$ and $(56)$.   From this, we
see that $4$ copies of ${\cal C\/}_6$ fit around
this edge.  Once again, given the right-angled
property of the faces of ${\cal C\/}_6$, this
means that our space ${\cal A\/}_6$ is locally
isometric to $\H^3$ around this edge.
The same story goes for the other two
edges of ${\cal C\/}_6$ that connect
ideal vertices.

Our analysis shows that ${\cal A\/}_6$ is locally isometric
to $\H^3$ in a neighborhood of every point of
${\cal C\/}_6$.   By symmetry, the same statement
holds for all points of ${\cal A\/}_6$.  Hence
${\cal A\/}_6$ is a hyperbolic $3$-manifold.
We can cut each copy of ${\cal C\/}_6$ in half
along the ideal triangle that is the convex hull of
the ideal vertices.   Each half is a pyramid obtained
by coning a regular ideal octahedron to the center
of mass.  In ${\cal A\/}_6$ we have $8$ of these
pyramids fitting together around each finite vertex to make
an ideal octahedron.
Thus ${\cal A\/}_6$ is tiled by regular ideal octahedra.
How many?

Well, ${\cal A\/}_6$ is obtained by gluing together
$f(6)=60$ copies of ${\cal C\/}_6$.
Each copy supplies $2$ pyramids, and we need
$8$ pyramids to make an ideal octahedron.
Thus, each copy of ${\cal C\/}_6$ supplies
$1/4$ of an octahedron.   We conclude
that ${\cal A\/}_6$ is tiled by $15$ regular
ideal hyperbolic octahedra.

\begin{center}
\resizebox{!}{1.5in}{% Figure removed}
\newline
{\bf Figure 5.9:\/}  Shapes of nearly degenerate hexagons
\end{center}

The cusps of ${\cal A\/}_6$ are the degenerate hexagons
corresponding to the permutations of the
last 3 shown in Figure 5.7.
Figure 5.9 shows the representative shapes
of all $10$ degenerate hexagons.  There are
$3$ of the first kind, $6$ of the second kind,
and one of the third kind.  To get a
comprehensible picture we have
taken nearly degenerate hexagons rather
than actually degenerate ones.

Finally, Figure 5.10 shows the shapes of
the $15$ hexagons corresponding to the
centers of the ideal octahedra.  There are
respectively $2,6,3,3,1$ of these.

\begin{center}
\resizebox{!}{1in}{% Figure removed}
\newline
{\bf Figure 5.10:\/}  Shapes of the central hexagons
\end{center}

We have picked representative labelings.


\newpage


\documentclass[11pt]{article}

\parskip 0.1in
\usepackage{geometry}
\geometry{margin=1in}
\setlength\parindent{0pt}

% Some important packages
\usepackage{amsmath, amsfonts, bm, amssymb, enumitem, tabularx}
\usepackage{amsfonts, graphicx, grffile,fullpage, float, latexsym, mathtools}
\usepackage{hyperref}
\hypersetup{
    colorlinks=true, %set true if you want colored links
    citecolor=black,
     linkcolor=blue,
     urlcolor=blue,
    linktoc=all,     %set to all if you want both sections and subsections linked
    linkcolor=black,  %choose some color if you want links to stand out
}
\usepackage{cleveref}
\usepackage{multicol}
\usepackage{tikz}

% Bibliography packages
\usepackage[sort, numbers, compress]{natbib}
\renewcommand{\bibname}{References}
\renewcommand{\bibsection}{\subsubsection*{\bibname}}

% Import Shorthands
\usepackage{Shorthands}

\title{A Data-Driven Approach to Synthesizing Dynamics-Aware Trajectories for Underactuated Robotic Systems}
\author{Anusha Srikanthan$^{1}$, Fengjun Yang$^{1}$, Igor Spasojevic$^{1}$, \\ Dinesh Thakur$^{2}$, Vijay Kumar$^{1}$, Nikolai Matni$^{1}$ % <-this % stops a space
\thanks{This research is in part supported by NSF award CPS-2038873, NSF CAREER award ECCS-2045834, NSF Grant CCR-2112665, and a Google Research Scholar award.
$^{1}$are with GRASP Lab, University of Pennsylvania, Philadelphia, USA
        {\tt\small \{sanusha, fengjun, igorspas, kumar, nmatni\}@seas.upenn.edu}.
$^{2}$was with GRASP Lab at the time of writing this paper.}}

\begin{document}

\maketitle

\begin{abstract}
We consider joint trajectory generation and tracking control for under-actuated robotic systems.
%
A common solution is to use a layered control architecture, where the top layer uses a simplified model of system dynamics for trajectory generation, and the low layer ensures approximate tracking of this trajectory via feedback control.
%
While such layered control architectures are standard and work well in practice, selecting the simplified model used for trajectory generation typically relies on engineering intuition and experience. 
%
In this paper, we propose an alternative data-driven approach to \emph{dynamics-aware trajectory generation}.
%
We show that a suitable augmented Lagrangian reformulation of a global nonlinear optimal control problem results in a layered decomposition of the overall problem into trajectory planning and feedback control layers.
%
Crucially, the resulting trajectory optimization is dynamics-aware, in that, it is modified with a \emph{tracking penalty regularizer} encoding the dynamic feasibility of the generated trajectory.
%
We show that this tracking penalty regularizer can be learned from system rollouts for independently-designed low layer feedback control policies, and instantiate our framework in the context of a unicycle and a quadrotor control problem in simulation.
%
Further, we show that our approach handles the sim-to-real gap through experiments on the quadrotor hardware platform without any additional training. For both the synthetic unicycle example and the quadrotor system, our framework shows significant improvements in both computation time and dynamic feasibility in simulation and hardware experiments.
\end{abstract}

\section{Introduction}

Modularity is a guiding principle behind the design of numerous autonomous platforms. 
For example, the autonomy stack of a typical robot consists of separate modules for perception, planning, and control~\cite{paden2016survey}.
In spite of the requirement of safely executing tasks in real time with limited on board computational resources, such modules usually operate at different frequencies and levels of abstraction. 
Roughly speaking, higher levels of abstraction allow for faster decision making.  
However, if the degree of abstraction varies among the different modules beyond a suitable threshold, the system as a whole can behave in unexpected, unsafe ways. 
By and large, choosing the right level of abstraction in robotics applications has remained somewhat of an art. 
We focus on developing a quantitative method of bridging the potential mismatch between the trajectory planning and control modules in a data-driven manner.


Although trajectory planning and control have been among the most extensively studied areas of robotics, numerous problems remain to be solved. 
In particular, graph-search-based path planning algorithms can find it challenging to account for complex nonlinear system dynamics. Similarly, real-time optimization-based methods for generating trajectories typically use a simplified or a reduced order dynamics model of the agent.  
In contrast, low-level feedback control policies often rely on more accurate, detailed dynamics of the system being controlled in order to track a reference trajectory planned by some of the aforementioned approaches.
While intuitive and conceptually appealing, this layered approach only works well if the outputs of higher layers are compatible with the abilities of lower layers.  

In this paper, we focus on the interplay between trajectory generation and feedback control.  
Rather than imposing such a layered architecture on the control stack, we show that it can be \emph{derived} via a suitable relaxation of a global nonlinear optimal control problem that jointly encodes both the trajectory generation and feedback control problems. 
Crucially, the resulting trajectory generation optimization problem is dynamics-aware, in that it is modified with a \emph{tracking penalty regularizer} that encodes the dynamic feasibility of a generated trajectory.  
While this tracking penalty does not in general admit a closed-form expression, 
we show that it can be interpreted as a cost-to-go. Hence, it can be learned from system roll-outs for any feedback control policy by leveraging tools from the learning literature. Finally we evaluate our framework using unicycle and quadrotor control, and compare our approach in simulation to standard approaches to quadrotor trajectory generation. 
Our extensive experiments demonstrate that our data-driven dynamics-aware framework allows for faster computation of trajectories that can be tracked accurately in both simulation and hardware. 
Our contributions are as follows:
\begin{itemize}
    \item We derive a layered control architecture composed of a dynamics-aware trajectory generator top layer, and a feedback control low layer.  In contrast to existing work, our trajectory generation problem is naturally dynamics-aware, and includes a tracking penalty regularizer that encodes the ability of the low-layer feedback control policy to track a given reference trajectory.
    \item We show how this tracking penalty can viewed as the cost-to-go for a particular system, and hence be learned from system rollouts. %using policy evaluation tools from the reinforcement learning literature.
    \item We apply our data-driven dynamics-aware trajectory generation framework to both a unicycle and a quadrotor control problem. We demonstrate that our approach generates aggressive and easy to track trajectories compared to standard methods for the two systems in consideration. 
\end{itemize}

In what follows, we first formulate the dynamics-aware trajectory planning problem in Section \ref{sec:problem-formulation} and introduce the unicycle and waypoint tracking problem as running examples. In Section \ref{sec:layeredarch}, we introduce a result that shows how a relaxation of the underlying nonlinear controls problem naturally leads to a trajectory optimization problem that includes a regularizer that captures the tracking cost of the given controller. 
%
We also describe our supervised learning approach to learn the cost-to-go function that characterizes the feedback control layer's ability to track a given reference trajectory. In Section \ref{sec:method-uni}, we apply our approach to dynamics-aware trajectory generation to both the unicycle and quadrotor systems. We present two compelling simulation experiments in Section \ref{sec:experiments} to show that our method leverages previous trajectory data to approximate the cost-to-go function and the learned function can be applied to generate easy-to-track trajectories before discussing the results and future work in Section \ref{sec:conclusion}.

\section{Related Work}
Our work builds on the literature of multi-rate/hierarchical control, data-driven nonlinear model-predictive-control (NMPC), and trajectory generation for quadrotors.  Here we attempt to provide an overview of most directly relevant related works from these different branches of the robotics literature.

\subsection{Multi-rate and hierarchical control}
There is a rich literature on combining trajectory generation with low-level control, see for example~\cite{rosolia2020multi, herbert2017fastrack, wabersich2018linear, yin2020optimization, singh2018robust, singh2017robust, gurriet2018towards} and references therein.  While these results offer differing degrees of guarantees and generality, we note that none of them derive the layered control architecture that they propose.  Rather, modification to either the trajectory generation or tracking problems are made to ensure that the chosen interplay between the two layers leads to desirable results.  In contrast, we derive this layered structure, and show how this naturally leads to the inclusion of a tracking penalty regularizer in the trajectory generation problem.  Our work is complementary to the existing literature due to the fact that modifying any of the proposed trajectory generation optimization problems in~\cite{rosolia2020multi, herbert2017fastrack, wabersich2018linear, yin2020optimization, singh2018robust, singh2017robust, gurriet2018towards}  with our proposed regularizer will only lead to more dynamically feasible trajectories.

Another closely-related line of work in this spirit is the ``Layering as Optimization Decomposition'' framework proposed in~\cite{chiang2007layering}, which shows that network utility maximization problems can be suitably relaxed to recover the layered architecture of network control protocol stacks.  This approach was extended to linear optimal control problems in~\cite{matni2016theory}, %by one of the authors, 
but was limited to LQR control for which the tracking penalty admits a closed-form expression.  In contrast, we significantly generalize these results to nonlinear dynamical systems and present a data-driven approach to approximating the tracking penalty.

\subsection{Data-driven NMPC}
Due to the underlying difficulty of solving a NMPC problem that jointly encodes trajectory generation and low-level control, data-driven approaches to improve controller performance on tracking tasks have emerged. Broadly, learning can be applied to: (i) directly obtain the control policy \cite{levine2014learning, levine2016end}, (ii) learn uncertainties in the dynamics and the cost function used for NMPC \cite{kabzan2019learning, williams2017information, ostafew2016robust, rosolia2019learning}, and (iii) learn a low dimensional state representation for NMPC from high dimensional data \cite{kaufmann2019beauty, drews2017aggressive, song2022policy}. Our work broadly fits into this overall line of work in that we propose a data-driven method to learn a tracking penalty regularizer that directly encodes the closed-loop dynamics' ability to track a generated trajectory using offline trajectories.  To the best of our knowledge, ours is the first approach to suggest this compact encoding of the low layer closed-loop dynamics into a cost-to-go function. %However, we differ from prior in that we use this cost to generate `easy to track' reference trajectories as opposed to including it within the controller policy optimization.  

\subsection{Trajectory generation for quadrotors}
%\textbf{Anusha and Igor please update this, this is not my area of expertise!  Just focus on standard approaches like min-jerk, differential flatness, etc.  Highlight the general decoupling from low-level dynamics of prior work to differentiate our paper.}


In general, trajectory generation for quadrotors is a computationally challenging problem.
% , as the high dimensional nature of the state space precludes grid-based methods. % relying on a naive discretization of allowed ranges of individual components of their state.
% Further, the highly nonlinear dynamics makes it difficult to directly apply linear control techniques, and as quadrotors live in a non-Euclidean state space, even numerically integrating their state forward in time for a fixed trajectory of inputs requires care. 
%
The landmark paper \cite{mellinger2011minimum} established \textit{differential flatness} \cite{fliess1995flatness} of quadrotor dynamics. 
It showed that trajectories of position and yaw angle of the quadrotor, i.e., the flat outputs, may be specified \textit{independently} of one another, and that their time derivatives yield the underlying trajectory of states and inputs required to induce them. 
Additionally, \cite{mellinger2011minimum} initiated a line of work~\cite{RichterBryRoy2016,HehnDAndreaRealTimeTRO15,MuellerDAndreaCompEffPrimitiveTRO15,LiuSE3SearchRAL18} using piecewise polynomials to represent trajectories of flat outputs.
Nevertheless, these approaches decouple trajectory generation and tracking control. 
%
As a consequence, there is no model of the specific hardware used for control introduced in the planning layer. 
%A consequence of this is that the resulting trajectories were either dynamically infeasible or more conservative than necessary. 
Our current work, on the other hand, provides a principled way of generating trajectories cognizant of the dynamic capabilities of the closed-loop robotic system.

\iffalse
\fi 

\section{Problem Formulation} \label{sec:problem-formulation}
 
We consider a finite-horizon, discrete-time nonlinear dynamical system
\begin{equation} \label{eq:dynamics}
x_{t+1} = f(x_t, u_t), \ t=0,\dots, N,
\end{equation}
with state $x_t \in \mathcal{X} \subseteq \mathbb{R}^n$ and control input $u_t \in \mathcal{U} \subseteq \mathbb{R}^k$ at time $t$. 
Our task is to solve the following constrained optimal control problem (OCP):
\begin{equation} \label{prob:master-problem}
\begin{array}{rl}
    \underset{x_{0:N},u_{0:N-1}}{\mathrm{minimize}} & \mathcal{C}(x_{0:N}) + \sum_{t=0}^{N-1}\|D_t u_t\|_2^2  \\
     \text{s.t.} & x_{t+1} = f(x_t,u_t), \, t=0,\dots,N,\\
     & x_{0:N} \in \mathcal{R},
\end{array}    
\end{equation}
where $\mathcal{C} : \mathcal{X}^{N+1} \rightarrow \mathbb{R}$ is a trajectory cost function, $D_0, D_1, ...., D_{N-1} \in \mathbb{R}^{l \times k}$
are matrices that penalize control effort, and $\mathcal{R}$ defines the feaible region of $x_{0:N}$. %
%

OCPs of the form \eqref{prob:master-problem} are %often solved as a subroutine in
an essential component of MPC schemes for robotic applications. In such settings, the trajectory cost function $\mathcal{C}$ is typically chosen to e.g., capture high-level task objectives or reward smooth trajectories, whereas the state constraint $\mathcal{R}$ is often used to encode e.g., obstacle avoidance, waypoint constraints, or other mission-specific requirements. %
 %
In the generality stated above, the OCP~\eqref{prob:master-problem} is %typically intractable 
difficult to solve exactly except in the simplest of cases.  Under suitable regularity assumptions, good heuristics exist for finding an approximate solution. However, due to their computational complexity, these heuristics typically lead to the solve time being unacceptably large for applications with fast dynamics such as quadrotor control.

A number of works in the robotics literature approach the computational complexity by using a \emph{layered control architecture} to decompose OCP~\eqref{prob:master-problem} into tractable subproblems.  
For example, a two-layer approach would solve a reference trajectory generation problem at the \emph{top planning layer} using simplified dynamics (typically at a slower frequency). This reference trajectory would then be sent to the \emph{low tracking layer} where a feedback control policy, operating in real time, attempts to follow the reference trajectory.
%

While conceptually appealing, the above approach has several shortcomings. The critical one is the lack of guarantees that the generated reference trajectories can be adequately tracked by the feedback control policy. This could be due to unmodelled dynamics, saturation limits, etc., of the hardware in use for control. In this paper, we address this shortcoming by \emph{deriving} a layered architecture via a relaxation of the original OCP~\eqref{prob:master-problem}, that naturally leads to a \emph{closed-loop dynamics-aware} trajectory generation problem.


\section{Layering as Optimal Control Decomposition} \label{sec:layeredarch}

We show how a suitable relaxation of the OCP~\eqref{prob:master-problem} naturally results in a layered control architecture.  Such an optimization decomposition approach to layered control was first introduced in \cite{matni2016theory} for linear-quadratic control. In this section, we extend it to general nonlinear systems. 
%

\subsection{An augmented Lagrangian relaxation}
We first introduce a redundant reference trajectory variable $r_{0:N}$ constrained to equal the state trajectory, i.e., satisfying $x_{0:N}=r_{0:N}$, to the OCP~\eqref{prob:master-problem}:
\begin{equation}
    \begin{aligned}
        \underset{r_{0:N}, x_{0:N}, u_{0:N-1}}{\mathrm{minimize}}&\quad \mathcal{C}(r_{0:N}) + \sum_{t=0}^{N-1}\lVert D_t u_{t} \rVert_2^2 \\
        \text{s.t.}&\quad x_{t+1} = f(x_t, u_t), \\
        & \quad r_{0:N} \in \mathcal{R}, \\
        & \quad x_{0:N} = r_{0:N}.
    \end{aligned}
\end{equation}
We then relax this redundant equality constraint to a soft-constraint in the objective function, resulting in the augmented Lagrangian reformulation:
\begin{equation} \label{prob:relaxed-problem}
    \begin{array}{rl}
        \underset{r_{0:N}}{\mathrm{min.}}& \mathcal{C}(r_{0:N}) + \underset{x_{0:N}, u_{0:N-1}}{\mathrm{min.}} \sum_{t=0}^{N-1}\left(\lVert D_t u_{t} \rVert_2^2 + \rho \lVert r_t - x_t \rVert_2^2\right) + \rho \lVert r_N - x_N \rVert_2^2\\
        \text{s.t.}&r_{0:N} \in \mathcal{R}, \quad \quad \quad \text{ s.t. } x_{t+1} = f(x_t, u_t)
    \end{array}
\end{equation}
where the weight $\rho>0$ specifies the soft-penalty associated with the constraint $r_{0:N}=x_{0:N}$. Furthermore, we have strategically grouped terms to highlight the nested structure of the resulting optimization problem. Immediately, problem~\eqref{prob:relaxed-problem} admits a layered interpretation: the inner minimization over state and input trajectories $x_{0:N}$ and $u_{0:N-1}$ is a traditional feedback control problem, seeking to optimally track the reference trajectory $r_{0:N}$. The outer optimization over the trajectory $r_{0:N}$ seeks to optimally ``plan'' a reference trajectory for the inner minimization to follow. %a that is not subject to dynamics constraints, but is penalized for dynamic infeasibility; on the lower layer, one jointly optimizes the variables $x_{0:N}$ and $u_{0:N-1}$ to design the a controller that can faithfully track the reference trajectory while minimizing control inputs.
To further highlight the layered nature of the resulting relaxation, we define the tracking penalty
\begin{equation}\label{eq:tracking-cost}
\begin{aligned}
    g_{\rho}^{track}(x_0, r_{0:N}) :=
    \underset{x_{0:N}, u_{0:N-1}}{\mathrm{min}}&\quad \sum_{t=0}^{N-1}\left(\lVert D_t u_{t} \rVert_2^2 + \rho \lVert r_t - x_t \rVert_2^2\right) + \rho \lVert r_N - x_N \rVert_2^2\\
        \text{s.t.}&\quad \text{dynamics \eqref{eq:dynamics}}. 
\end{aligned}
\end{equation}
The tracking penalty $g_{\rho}^{track}(x_0, r_{0:N})$ captures how well a given trajectory $r_{0:N}$ can be tracked by a low layer control sequence $u_{0:N-1}$ given the initial condition $x_0$, and is naturally interpreted as the cost-to-go associated with an augmented system (see \S\ref{sec:learning}).  We observe that the optimal control problem defining the tracking cost~\eqref{eq:tracking-cost} is a standard nonlinear reference tracking problem with quadratic cost, and can be approximately solved using tools from nonlinear feedback control~\cite{roulet2019iterative}.  We therefore let $\pi(x_t,r_{0:N})$ denote the feedback control policy which (approximately) solves problem~\eqref{eq:tracking-cost}, and use $g_{\rho,\pi}^{track}(x_0, r_{0:N})$ to denote the resulting cost-to-go that it induces. While a closed-form expression for the tracking penalty $g_{\rho,\pi}^{track}(x_0, r_{0:N})$ is only available in special cases, e.g., see~\cite{matni2016theory} for the linear quadratic control case, we show in \S\ref{sec:learning} that it can be learned from data.

Assuming that an accurate estimate of the tracking penalty can be obtained, the OCP~\eqref{prob:relaxed-problem} can now be reduced to the \emph{static} optimization problem (i.e., without any constraints enforcing the dynamics~\eqref{eq:dynamics}):
\begin{equation} \label{prob:layered-problem}
    \begin{aligned}
        \underset{r_{0:N}}{\mathrm{minimize}}&\quad \mathcal{C}(r_{0:N}) + g_{\rho, \pi}^{track}(x_0, r_{0:N}) \\
        \text{s.t.} & \quad r_{0:N} \in \mathcal{R}.
    \end{aligned}
\end{equation}
We may view~\eqref{prob:layered-problem} as a family of trajectory optimization problems parametrized by $\rho$. In the limit as $\rho \nearrow \infty$, optimal trajectories prioritize the reference tracking performance. On the other hand, in the limit as $\rho \searrow 0$, the optimal trajectories minimize the $\mathcal{C}$ cost oblivious to the dynamics constraints. 


% Figure environment removed


\subsection{Learning the tracking penalty through policy evaluation}
\label{sec:learning}
Computing the tracking penalty $g_\rho^{track}$ for general nonlinear dynamics and experimental hardware platforms with black-box feedback control policies is intractable. We therefore propose a supervised learning approach to learning the tracking penalty \eqref{eq:tracking-cost} from data as shown in Figure \ref{fig:layered-arch-quad}. 

We define the following augmented dynamical system with states $\mu_t \in \R^{(N+1)n}$ and control inputs $u_t \in \R^{k}$. %On a high level, 
The state $\mu_t$ is constructed by concatenating the nominal state $x_t$ and the reference trajectory $r_{t:t+N}$ of length $N$ starting at time $t$, i.e., $\mu_t = (x_t, r_{t:t+N})\in \R^{(N+1)n}$. Letting $\mu_t^x := x_t$ and $\mu_t^r := r_{t:t+N}$, the augmented system dynamics can be written as $\mu_{t+1} = h(\mu_t, u_t)$, where
\begin{equation}\label{eq:aug_dyn}
    h(\mu_t, u_t) :=  \begin{bmatrix} f(\mu_t^x, u_t) \\ Z \mu_t^r\end{bmatrix}.
\end{equation}
Here $Z \in \{0,1\}^{Nn \times Nn}$ is the block-upshift operator, i.e., a block matrix with $I_n$ along the first block super-diagonal, and zero elsewhere. The state $\mu_t^x=x_t$ evolves in exactly %term 
the same way as in the true dynamics~\eqref{eq:dynamics}, whereas the reference trajectory $\mu_t^r := r_{t:t+N}$ is shifted forward in time via $Z\mu_t^r= r_{t+1:t+1+N}$.  Fixing policy $\pi(\mu_t)$, we define the policy dependent tracking cost
\begin{equation}\label{eq:pi-tracking}
g_{\rho,\pi}^{track}(x_0, r_{0:N}) =
\sum_{t=0}^{N-1} \rho \left\lVert \mu_t^x - [\mu_t^r]_1 \right\rVert_2^2 + \lVert D_t u_t \rVert_2^2 + \rho \left\lVert \mu_N^x - [\mu_N^r]_1 \right\rVert_2^2.
\end{equation}
We note that this corresponds exactly to the objective function defining the tracking penalty~\eqref{eq:tracking-cost} evaluated under the control sequence $u_t = \pi(\mu_t).$  As such, the policy dependent tracking cost $g_{\rho,\pi}^{track}(x_0, r_{0:N})$ is naturally viewed as an upper-bound to the true optimal tracking cost, where the sub-optimality is dependent on the quality of the chosen policy $\pi$.  In particular, we have that $g_{\rho,\pi^\star}^{track} = g_{\rho}^{track}$ for any optimal policy $\pi^\star$ that solves the optimal control problem~\eqref{eq:tracking-cost}.



% We define the stage-cost function of the augmented system to be
% \begin{equation*}
%     q(\mu_t, u_t) = \rho \left\lVert \mu_t^x - [\mu_t^r]_1 \right\rVert_2^2 + \lVert D_t u_t \rVert_2^2.
% \end{equation*} 
% and compute the cost-to-go function for each trajectory using the 

Noting that the policy dependent tracking penalty~\eqref{eq:pi-tracking} is defined in terms of stage-wise costs, we can interpret $g_{\rho,\pi}^{track}(x_0, r_{0:N})$ as a cost-to-go function associated with the Markov Decision Process defined by the cost~\eqref{eq:pi-tracking}, dynamics~\eqref{eq:aug_dyn}, and policy $\pi$.  We therefore use Monte Carlo sampling \cite{sutton2018reinforcement} to generate a set of $\mathcal{T}$ trajectories of horizon length $N$ given by $(x^{(i)}_{0:N}, u^{(i)}_{0:N-1}, r^{(i)}_{0:N})_{i=1}^{\mathcal{T}}$, 
where $x_{0:N}^{(i)}$ and $u^{(i)}_{0:N-1}$ are the $i$-th state and input trajectories collected from applying feedback control policy $\pi$ to track reference trajectories $r^{(i)}$.  We also compute the associated tracking cost labels $y^{(i)} := g_{\rho,\pi}^{track}(x_0^{(i)},r_{0:N}^{(i)})$.  We then used supervised learning to approximate the policy dependent tracking penalty~\eqref{eq:pi-tracking} by solving the following supervised learning problem %empirical risk minimization problem
$$
\begin{array}{rl}
     \mathrm{minimize}_{g\in\mathcal G} \ \sum_{i=1}^\mathcal{T}(g(x_0^{(i)},r_{0:N}^{(i)}) - y^{(i)})^2,
\end{array}
$$
over a suitable function class $\mathcal{G}$, e.g., feedforward neural networks, see \S\ref{sec:experiments} for more details.

\section{Dynamics-Aware Trajectory Generation for Under-Actuated \\Robotic Systems}
\label{sec:experiments}
We showed the flexibility of our framework by applying it to both a unicycle and a quadrotor control problem.  For each platform, we formulated a global planning and control problem, which is then subsequently relaxed according to the methods proposed in \S\ref{sec:layeredarch} to yield a dynamics-aware planning problem and a feedback control layer. We now evaluate our methods experimentally and demonstrate their effectiveness in simulation and in real-world experiments.

\subsection{Unicycle Control}
\label{sec:method-uni}


% \subsection{Reformulation of OCP for unicycle}

We consider the continuous time unicycle dynamics
$$\begin{bmatrix}
    \dot x_1 \\ \dot x_2 \\ \dot \theta
\end{bmatrix} = \left[\begin{array}{cc}
        \cos{\theta} & 0 \\
        \sin{\theta} & 0 \\
        0 & 1
    \end{array}\right] \left[\begin{array}{c}
         v \\
         \omega 
    \end{array}\right]
$$ where $(x_1,x_2)\in\R^2$ are the system's Cartesian coordinates,  $\theta$ is the heading angle, and $v$, $\omega$ are the instantaneous linear and angular velocities, respectively.  Letting $x = (x_1,x_2,\theta)$ and $u=(v, \omega)$, we can compactly write the dynamics as $\dot x = g^{cts}(x)u$, for suitably defined $g(x)\in\R^{3 \times 2}$. Letting $x_{t+1}=f_{uni}(x_t,u_t)$ be the \texttt{rk4} discretization of these continuous dynamics, we can then pose the global problem
\begin{equation}\label{eq:uni-global}
    \begin{array}{rl}
         \underset{{x_{0:N},u_{0:N-1}}}{\mathrm{minimize}}& \displaystyle\sum_{\tau\in\mathcal{T}_w} \|x_\tau - w_\tau\|_2^2 + \sum_{t=0}^{N-1} u_t^T R u_t  \\
         \text{subject to} &  x_{t+1}=f_{uni}(x_t,u_t),
    \end{array}
\end{equation}
where $w_\tau$ such that $\tau\in \mathcal{T}_w\subseteq \{0,\dots,N\}$ are waypoints that the unicycle should traverse at time $\tau$, and $R>0$ is a positive definite control cost matrix.

To instantiate the layering framework proposed in \S\ref{sec:layeredarch}, we fix a low layer continuous time feedback control policy as
$$
\pi_{uni}(x,r)=g(x)^{\dag}(\dot r + K_p(x-r)),
$$
We then define a new control input, $\Delta r=\dot r$, to obtain the continuous time closed-loop dynamics 
\begin{equation}
\begin{array}{rcl}
\dot x &=&  g(x)g(x)^\dag(\dot r + K_p(x-r))\\
\dot r &=& \Delta r
\end{array} . 
\end{equation}
Letting $\bar x := (x, r),$ we compactly rewrite the continuous time closed-loop dynamics as
\begin{equation}\label{eq:uni-cts-dyn}
\dot{\bar{x}}=f^{cts}_{\pi, uni}(\bar x, \Delta r)
\end{equation}
for an appropriately defined $f^{cts}_{\pi,uni}$.
Finally, we obtain the discrete time dynamics $f_{\pi,uni}(\bar x_t, \Delta r_t)$ used in the experiments below via a \texttt{rk4} discretization of the continuous time dynamics \eqref{eq:uni-cts-dyn}.

Given the fixed closed-loop dynamics using the policy $\pi_{uni}$, the dynamics-aware trajectory generation problem~\eqref{prob:layered-problem} is then given by
\begin{equation}\label{eq:uni-ref}
    \begin{array}{rl}
        \underset{r_{0:N}}{\mathrm{minimize}} &\displaystyle\sum_{\tau\in\mathcal{T}_w} \|r_\tau - w_\tau\|_2^2 + g_{\rho, \pi_{uni}}^{track}(x_0, r_{0:N}) 
        %\\
    %\text{s.t.} & r_0 = x_0, \ r_N = x_N
    \end{array}
\end{equation}
where $g_{\rho, \pi_{uni}}^{track}(x_0, r_{0:N})$ is the policy dependent tracking penalty~\eqref{eq:pi-tracking} induced by the closed-loop dynamics $\bar x_{t+1} = f_{\pi,uni}(\bar x_t, \Delta r_t)$.

\paragraph{Data collection}
In order to estimate the policy dependent tracking penalty $g_{\rho, \pi_{uni}}^{track}(x_0, r_{0:N})$, we sample reference trajectories and roll them out on the closed-loop system.  In order to appropriately shape the landscape of the learned penalty, we sample both easy and difficult to track reference trajectories.  Towards that end, we generate \emph{easy to track} reference trajectories by using Iterative LQR (ilqr)~\cite{li2004iterative} to approximately solve the finite horizon constrained optimal control problem 
\begin{equation}\label{eq:uni-ocp}
    \begin{array}{rl}
 \underset{v_{0:N}, x_{0:N}}{\mathrm{minimize}} & \sum_{\tau\in\mathcal{T}_w} \|x_\tau - w_\tau\|_2^2 +\sum_{t=0}^{N-1} \Delta r_t^T R_w \Delta r_t  \\
        %&\text{s. t. } \dot{x}_t = \dot{r}_t + K_p (x_t - r_t), \forall t \in \{0, \cdots, N-1\}
        \text{s.t.} & \bar x_{t+1} = f_{\pi,uni}(\bar x_t, \Delta r_t)
        \\
        &x_0 = r_0,\ x_N = r_N 
    \end{array}
\end{equation}
where $R_w$ is a positive definite matrix penalizing variations in the reference trajectory. Additionally, we also generate state independent polynomial reference trajectories that are oblivious to the low layer closed-loop dynamics of the system and only satisfy the initial and terminal state constraints. This strikes a balance between having low cost but hard to compute ilqr trajectories and high cost but easy to compute polynomial trajectories. At inference, we solve a constrained optimization by applying gradient descent on the dynamics-aware trajectory generation problem~\eqref{eq:uni-ref}. 

We generate $500$ trajectories by sampling initial and goal locations from a uniform distribution over $[0, 2]^{2}$ and $[1, 3]^{2}$, respectively. Between each initial location and goal, we sample one waypoint by choosing a convex combination of the two points. The heading angles for the initial state is sampled at random from a uniform distribution on the interval $[0, \pi]$ and the goal heading angles are set to 0. As described in Section \ref{sec:method-uni}, we run the constrained ilqr algorithm on the closed loop dynamics \eqref{eq:uni-ocp} until convergence enforcing the initial and terminal state constraints. Additionally, we augment the training dataset with $500$ polynomial reference trajectories with randomly sampled initial, waypoint and goal conditions from the fixed intervals mentioned above. In this way, we include both easy to track trajectories (generated by ilqr) and difficult to track trajectories (polynomial) in order to appropriately shape the optimization landscape of the learned tracking penalty.  For testing, we generate $50$ trajectories with one or two waypoints each from the fixed intervals using the polynomial reference generation method described above.

\paragraph{Training} We train a multi-layer perceptron network with $3$ hidden layers of $\{1000, 500, 200\}$ neurons, respectively, with Exponential Linear Unit (ELU) activation functions. ---see~\nameref{sec:appendix} for more details. 
We train a separate network for each value of $\rho$ using a batch size of $64$, learning rate $10^{-4}$ and run for $2500$ epochs. The entire network is setup using the optimized \texttt{JAX} \cite{jax2018github}, \texttt{Optax} and \texttt{Flax} libraries. The loss function is optimized using stochastic gradient descent (SGD) with momentum set to $0.9$. At test time, i.e., when we compute trajectories to be tracked by the low-level controller, we freeze the weights of the network and run projected gradient descent (PGD) using \texttt{jaxopt} to locally solve the dynamics-aware trajectory planning problem~\eqref{eq:uni-ref}. We set the maximum number of iterations for projected gradient descent to $50$. %$\beta > 0$ is the weight placed on the supervised learning loss. In the next section, we detail how this learned tracking cost function can be applied to solve trajectory optimization problems. We start with showing results for a simple unicycle dynamical system to understand the challenges of applying our approach to a simulated system and test on robotic experiments designed for the quadrotor system. %in the context of quadrotors.

\paragraph{Results} We provide two types of evaluations on the learned policy dependent tracking penalty. First, we evaluate the network predictions for different values of relaxation weight $\rho$ on a test dataset consisting of $50$ trajectories generated independently and in an identical way to the training dataset. Next, we plot the relative tracking cost in Figure~\ref{fig:cost-uni}, where we compute the ratio of the tracking errors incurred by the trajectories returned by the dynamics-aware problem~\eqref{eq:uni-ref} to those incurred by polynomial interpolating (i.e., not dynamics-aware) trajectories---we emphasize these tracking costs are computed via rollouts of the actual closed-loop system on the trajectories. The lower the value of the relative cost, the more significant the tracking performance gain obtained from using our approach. Our results indicate that for appropriately chosen tracking weight $\rho>0$ the trajectories generated using our method are on average easier to track than polynomial interpolating trajectories. In Figure \ref{fig:pgd-uni}, we show the performance of our network ($\rho=0.1$) in planning reference trajectories that are of lower tracking cost (from the closed-loop dynamics simulation) with each gradient step. We also evaluated the average run time of our approach on $200$ trajectories and found that our algorithm is almost twice as fast as compared to the run time of the ilqr algorithm. Our network on average takes $6.9 \pm 0.7$ seconds to converge compared to ilqr which takes $11.4 \pm 0.5$ seconds. We conjecture that the results can be further improved by using convex parameterizations for the tracking penalty, such as input-convex-neural-networks (ICNN): we leave exploring this direction to future work. We also note that the approach is sensitive to the number of gradient steps based on the choice of $\rho$. 

% \noindent \textbf{Simulations on the Unicycle}

% Figure environment removed

% Figure environment removed

\subsection{Quadrotor Control}
\label{sec:method-quad}
We consider the waypoint following problem where one is given a sequence of times $(\tau_i)_{i = 0}^W \in [0, T] \cap \mathbb{Z}$ and waypoints $\mathcal{W} = \{(p
_i, \psi_i)\}_{i = 0}^{W} \subseteq \mathbb{R}^3 \times S^1$, each specifying the desired position and yaw angle of the quadrotor. The goal is to generate a trajectory that passes through the waypoints at corresponding times. These conditions can be formulated in the OCP problem \eqref{prob:master-problem} by encoding the state constraints
\begin{equation*}
    p_{\tau_i} = p_i,\ \psi_{\tau_i} = \psi_i, \, \text{for } {i=0,\dots,W} %\subseteq \mathcal{R}.
\end{equation*}
within the constraint set $\mathcal{R}$. We consider a discrete time dynamical system $x_{t+1}=f_{q}(x_t,u_t)$  obtained by numerically integrating (using the Runge-Kutta scheme) a quadrotor with the following equations of motion:
\begin{equation}   \label{eq:quad_dynamics}
\dot x := 
\begin{bmatrix}
\dot{p} \\
\dot{v} \\
\dot{R} \\ 
\end{bmatrix}
=
\begin{bmatrix}
v \\ 
Re_3 c + g \\ 
R [\omega \times] \\ 
\end{bmatrix}.
\end{equation}
Here, the state $x$ of the agent consists of its position ($p \in \mathbb{R}^3$), velocity ($v \in \mathbb{R}^3$), and orientation ($R \in SO(3)$) with respect to the world frame, while the control input $u=(c,\omega)$ consist of total thrust, $c \in \mathbb{R}$, and angular velocity, $\omega \in \mathbb{R}^3$.

We can then pose the global problem that seeks to find a dynamically feasible \emph{minimium jerk} trajectory that passes through all of the waypoints:
\begin{equation}\label{eq:quad-global}
    \begin{array}{rl}
         \underset{{x_{0:N},u_{0:N-1}}}{\mathrm{minimize}}& \sum_{t=0}^{N-1} \|\dddot p_t\|_2^2 + \| \dot \psi_t\|_2^2 + \| u_t\|_2^2  \\
         \text{subject to} &  x_{t+1}=f_{q}(x_t,u_t), \\
         & p_{\tau_i} = p_i,\ \psi_{\tau_i} = \psi_i, \, \text{for } {i=0,\dots,W}
    \end{array}
\end{equation}



To instantiate the layering framework proposed in \S\ref{sec:layeredarch}, we fix a tracking control policy $\pi_{q}(x_t,r_{t:t+N})$ for the given dynamics $f_{q}(x_t,u_t)$, and a way to collect data on this tracking policy. In the experiments that follow, we use an SE(3) geometric controller \cite{SE3Controller}, but any tracking controller, e.g., a PID \cite{PIDController}, or even an RL-based controller~\cite{RLController}, can be equally accommodated by our framework.    Fixing the policy $\pi_q$, we can define the resulting closed-loop dynamics $x_{t+1}=f_{\pi,q}(x_t,r_{t:t+N})$ and corresponding policy dependent tracking penalty $g^{track}_{\rho,\pi_{q}}(x_0,r_{0:N})$ as in~\eqref{eq:pi-tracking}.

We now describe how to use the penalty $g^{track}_{\rho,\pi_{q}}(x_0,r_{0:N})$ to generate dynamics-aware trajectories that interpolate the waypoints $(p_i,\psi_i)$. First, we note that from differential flatness~\cite{fliess1995flatness,greeff2018flatness}, it suffices for us to generate trajectories for $x, y, z$, and $\psi$. We take the widely-adopted approach of parameterizing trajectories as piecewise polynomials of order $k_r$ that smoothly interpolates between waypoints. Specifically, each segment is parametrized by a polynomial
%\begin{equation*}
%    \begin{bmatrix} x_i(\tau_i+t) \\ y_i(\tau_i+t) \\ z_i(\tau_i+t) \\ \psi_i(\tau_i+t) \end{bmatrix} = 
%    \begin{bmatrix}
%    \sum_{k=0}^{k_r-1} c_{i,k}^x t^k \\ \sum_{k=0}^{k_r-1} c_{i,k}^y t^k \\
%    \sum_{k=0}^{k_r-1} c_{i,k}^z t^k \\ \sum_{k=0}^{k_r-1} c_{i,k}^\psi t^k
%    \end{bmatrix},\ 0 \leq t \leq \tau_{i+1}-\tau_i,
%\end{equation*}
where $c_{i,k}^j$ denotes the $k$-th coefficient of polynomial $i$ for the dimension $j \in \{x, y, z, \psi\}$. 

With this parameterization, we can now recast the dynamics-aware trajectory generation problem~\eqref{prob:layered-problem} as one in the coefficients of the polynomial:
\begin{equation} \label{prob:quad-problem}
    \begin{array}{rl}
        \underset{\{c_{i,k}^j\}}{\mathrm{minimize}} &\displaystyle\sum_{\tau\in\mathcal{T}_w} \|r_\tau - w_\tau\|_2^2 + g_{\rho, \pi_q}^{track}(x_0, r_{0:N}) 
        %\text{subject to} &  r_{0:N} = \mathcal{M}(\{c_{i,k}^j\}) \\
        %& \{c_{i,k}^j\} \text{ satisfy constraints \eqref{eq:bc-constraint}}. 
    \end{array}
\end{equation}
where $r_{0:N}$ is a linear map of the polynomial coefficients.

\paragraph{Data Collection} We generate $125$ trajectories by sampling the $x, y, z, \psi$ amplitudes of the Lissajous curves from a uniform distribution on the intervals $[-0.65, 0.65], [-0.55, 0.55], [-0.55, 0.55], [-0.6\pi, 0.6\pi]$, respectively. We select $5$ equally spaced waypoints on the Lissajous curves, parametrized by the following equations:
\begin{equation}\label{eq:lissajous}
    \begin{aligned}
    x_n = A_x\left(1 - \cos{\frac{2\pi n}{T}}\right),\
    y_n = A_y \left(\sin{\frac{2 \pi n}{T}}\right) \\
    z_n = A_z \left(\sin{\frac{2 \pi n}{T}}\right),\
    \psi_n = A_{\psi} \left(\sin{\frac{2 \pi n}{T}} \right)
    \end{aligned}
\end{equation}
where the time period of each trajectory is $3s$ and each second is discretized into 100 time steps. The discretization interval is selected based on the frequency of the feedback layer $SE(3)$ tracking controller~\cite{SE3Controller}, and $N=300$ is our planning horizon. We generate piece-wise polynomial reference trajectories $r_t \in \mathbb{R}^4$ denoting $x, y, z$ positions and $\psi$, for each segment between waypoints by minimizing the sum of squares of jerk and yaw angular velocity. We forward simulate the tracking controller using a state-of-the-art real-time quadrotor physics simulator on the ROS platform \cite{mohta2018fast} to record the system rollouts. 

\paragraph{Training} We train a multi-layer perceptron composed of 3 hidden layers with $\{500, 400, 200\}$ neurons, respectively, and ELU activation functions. ---see~\nameref{sec:appendix} for more details. 
Similar to the unicycle setup, we train a separate neural network for each value of $\rho$ using a batch size of 64, learning rate $10^{-3}$, and run for $2000$ epochs. The network implementation uses \texttt{Optax}, \texttt{Flax}, and \texttt{JAX} libraries for optimization and the loss function used is SGD with momentum set to be $0.9$. At test time, we use the `L-BFGS-B' solver from \texttt{jaxopt} to locally solve the dynamics-aware trajectory planning problem~\eqref{prob:quad-problem} where the objective function represents a trade-off between satisfying waypoints and the tracking cost of the $SE(3)$ geometric controller. We do no additional training for the hardware experiments. 

\paragraph{Results}
We evaluate the policy-dependent tracking penalty on trajectories generated independently and in an identical way to the training dataset, however, we allow for replanning after every $300$ time steps. We compare the tracking performance of our dynamics-aware framework with two trajectory generation methods, the standard minimum-jerk based planner satisfying constraints described in equations~\eqref{eq:quad-global} and polynomial trajectories that satisfy waypoint constraints but no smoothness constraints. Figure \ref{fig:traj-quad} shows the full path of the trajectory in blue, the replanned trajectories for every $300$ time steps in green and the red arrows correspond to the odometry states from the simulator. On the top left are results from using the minimum jerk planner, the bottom left shows the polynomial trajectories without smoothness constraints and on the right we show our dynamics-aware planner that solves the trajectory generation in~\eqref{prob:quad-problem}. Our planner is able to recover trajectories of low tracking cost by replanning with the learned tracking penalty every $300$ time steps. We also evaluate the tracking cost from the $SE(3)$ dynamics for different values of $\rho$, as shown in Figure \ref{fig:cost-quad}. We observe that the learned tracking penalty faithfully approximates the tracking cost function of the low layer $SE(3)$ feedback controller and that the dynamics-aware trajectories synthesized using the learned tracking penalty achieve a significant reduction in the tracking cost for every tracking weight value $\rho>0$ that we tested. Finally, as shown in Figure \ref{fig:traj-quad-hw}, we demonstrate our dynamics-aware trajectories on the Qualcomm-Snapdragon based hardware platform \cite{loianno2016estimation} to show that our method handles the sim-to-real gap without any additional training. 

\paragraph{Hardware} We use the hummingbird quadrotor platform running a VOXL Flight - PX4 Autonomy controller with on-board visual inertial odometry and inertial measurement unit (IMU) sensors for localization. We treat the quadrotor system as a remote work station and establish a communication interface using the ROS platform from a laptop to transmit the position commands for the low layer $SE(3)$ feedback controller. The position commands are $14$-dimensional vectors composed of position, velocity, acceleration, jerk, yaw angles and yaw angular speed computed using $x, y, z, \psi$ references. We generate trajectories online using our dynamics-aware planner and transmit the commands over WiFi to the quadrotor for execution and record the reference trajectory and the odometry states from each run. We plot the $x, y, z$ co-ordinates of the reference trajectories and odometry measurements across time as shown in Figure~\ref{fig:traj-quad-hw}. We note that our dynamics-aware framework is able to generate trajectories that are safe to be deployed and tracked by the $SE(3)$ controller even without enforcing smoothness constraints. In future work, we would like to eliminate latencies arising from communicating the commands over a network and aim towards running the dynamics-aware framework using the limited onboard compute of the quadrotor platform. 

% Figure environment removed



% Figure environment removed

% Figure environment removed

\section{Conclusion} \label{sec:conclusion}
We showed that the familiar two layer architecture composed of a trajectory planning layer and a low-layer tracking controller can be derived via a suitable relaxation of a global optimization problem.  The result of this relaxation is a regularized trajectory planning problem, wherein the original state objective function is augmented with a tracking penalty which captures the low layer closed-loop system's ability to track a given reference trajectory.  We further observed that this penalty can be interpreted as the cost-to-go of an augmented system, and showed how it could be learned from data.  We demonstrated our results on waypoint tracking problems for a unicycle system and a quadrotor system in simulation and hardware. %realistic quadrotor simulation.  
In both cases, our method yielded significantly easier to track trajectories than simple polynomial interpolations between waypoints.  Future work will look to develop more systematic approaches to collecting trajectory data for training the tracking penalty, and to derive statistical guarantees for the learned tracking penalty.

\newpage

\appendix
\section{Network Parameterization and Training}\label{sec:appendix}
%\textbf{Network Parameterization and Training:}
We parameterize the tracking cost function as the exponential of a multi-layer perceptron $\phi(\cdot; \theta): \R^{(N+1)n} \to \R$ with parameter $\theta$ and minimize the loss between the labels and predictions %Bellman error on the collected data, with additional supervision directly on the collected value, 
by solving the empirical risk minimization problem
$\mathrm{minimize}_{g\in\mathcal G} \ \sum_{i=1}^\mathcal{T}(\phi(\mu^{(i)};\theta) - \log(y^{(i)}))^2$
where $\mathcal{T}$ is the number of collected trajectories, and $(\mu^{(i)}, y^{(i)})_{i=1}^\mathcal{T}$ are labeled pairs of the augmented state (containing the initial condition and reference trajectory) and the tracking cost it incurs (as described in \S\ref{sec:learning}).  At test time, we set the tracking penalty to be $\exp(\phi(\mu; \theta)$, thus ensuring that it is non-negative for all $\mu$. We found that this log reparameterization leads to more stable training of the network.  

\clearpage
\bibliographystyle{abbrvnat}

\bibliography{refs}


\end{document}



\end{document}




