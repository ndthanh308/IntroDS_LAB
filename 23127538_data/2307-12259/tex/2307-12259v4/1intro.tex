\section{Introduction}

Billiards, of course, needs no introduction.
However, it has two exotic cousins which are less
well known, {\it symplectic billiards\/} and
{\it tiling billiards\/}.   In this paper I will
unite these two topics. I
call the new game {\it symplectic tiling billiards\/}.
Perhaps anyone who knows about both
symplectic billiards and tiling billiards could stop reading
now and define symplectic tiling billiards for themselves just
based on the name.

For ease of exposition
I will stick to the polygonal cases of all these
topics.
Symplectic billiards is perhaps best played on
a pair of polygons, $A$ and $B$, as shown in
Figure 1.1.  Starting with a pair
$(a_1,b_2) \in \partial A \times \partial B$ one
produces a pair $(a_3,b_4) \in \partial A \times \partial B$
using the rule below.

\begin{center}
\resizebox{!}{1.3in}{% Figure removed}
\newline
{\bf Figure 1.1:\/} Symplectic Billiards Defined
\end{center}

In words, the line connecting
$a_1$ to $a_3$ is parallel to the side of $B$
containing $b_2$ and the line connecting
$b_2$ to $b_4$ is parallel to the side of $A$
containing $a_3$.   One then iterates and
considers the dynamics.   I first learned about
symplectic billiards from Peter Albers and
Serge Tabachnikov.  We 
later wrote a paper [{\bf ABSST\/}] about the
subject, proving a few foundational results.
The two-table perspective is explored extensively in the more recent
work [{\bf ALW\/}].

Tiling billiards is a variant of billiards played
on the edges of a planar tiling.  Figure 1.2 shows
the rule. 

\begin{center}
\resizebox{!}{2.2in}{% Figure removed}
\newline
{\bf Figure 1.2:\/} Tiling Billiards
\end{center}

The rule is essentially the same as for
billiards, except that the trajectory
refracts through the edges rather than
bouncing off them.  I first learned about
tiling billiards from Serge Tabachnikov.
Now there is a growing literature on the
subject.  See
[{\bf BDFI\/}] and the references therein.

In \S 2 I will define {\it symplectic tiling
billiards\/} and make a few general
remarks about it. I will also show
the results of a few easy experiments.
The game is played relative to a pair of
tilings of the plane, though one could
specialize to the case where the two
tilings are the same.

In \S 3 I will consider a special case of
this game, a kind of ``local version'',
in which the planar tilings
involved each consist of $N$
infinite sectors bounded by $N$ rays emanating
from the origin.  We call such a tiling an
$N$-{\it sunburst\/}.   We call the $N$-sunburst
{\it balanced\/} if the sum of the $N$  unit vectors
parallel to the rays is $0$.   As a special
case, we call the $N$-sunburst {\it regular\/}
if the rays are parallel to the $N$th roots of unity.
Figure 1.3 shows a pair $(A,B)$  of $7$-sunbursts
where $A$ is regular and $B$ is balanced. The rays
of $A$ point outwards and the rays of $B$ point inwards,
as indicated by the arrows.

\begin{center}
\resizebox{!}{2.2in}{% Figure removed}
\newline
{\bf Figure 1.3:\/} Adapted polygons supported by a pair of
$7$-sunbursts
\end{center}

Figure 1.3 shows a very special situation.  Each sunburst
has an inscribed counterclockwise-oriented convex polygon such that the edges and
rays having the same label are parallel in the oriented sense
indicated by the arrows.  Let us provisionally
call such polygons {\it adapted polygons\/} and 
say that the pair of sunbursts {\it supports\/} them.
Later on
we will see that these polygons are  periodic orbits for
symplectic tiling billiards.

We say that a pair $(A,B')$ of sunbursts is a
{\it phase modification\/} of the pair $(A,B)$ if
$B'$ is obtained from $B$ by rotating about the origin.
Here is a restatement of one of our main results,
Theorem \ref{two}.

\begin{theorem}
  \label{local}
  A pair $(A,B)$ of $N$-sunbursts, with $A$ regular and $B$ balanced,
  has a unique phase modification which supports adapted $N$-gons.
\end{theorem}

In \S 4 we consider planar equilateral polygons.
All the edges have the same length.
If we fix the number $N$ of sides, then the
moduli space of equilateral $N$-gons modulo
similarity is equivalent to the configuration
space of the mechanical linkage made from
$N$ unit-length rods.
Theorem \ref{local} gives a very clean
bijection between similarity
classes of strictly convex equilateral $N$-gons and
similarity classes of strictly convex equiangular $N$-gons.

Here is the idea.   Let $L$ be an equilateral $N$-gon.
By taking the rays parallel to the
edges of $L$, in order, we get a balanced
$N$-sunburst $B$.  We let $A$ be the regular
$N$-sunburst.  We apply Theorem \ref{local}
to $(A,B)$ to get a phase modification $(A,B')$ which
supports
adapted polygons.  The similarity class
of the adapted $N$-gon inscribed in $B'$ (as in the right side of
Figure 1.3) gives
us our class of equiangular $N$-gon $P_L$.
Our association respects the equivalence class
and this gives us our bijection $[L] \to [P_L]$.

Our correspondence
is akin to the one given by
Misha Kapovich and John Millson [{\bf KM\/}], but it
is more direct, more algebraic, and easier to compute.
See \S \ref{compute} and \S \ref{algXX}.
The Kapovich-Millson
corrspondence involves the Riemann mapping
theorem, a transcendental construction.

In \S \ref{hypXX}  we recall some features of
William Thurston's famous {\it Shapes of Polyhedra\/}
construction [{\bf T\/}] of a complex hyperbolic structure
on the moduli space of polyhedra with prescribed
cone angles.
When we restrict our attention to
polyhedra which are doubles of strictly
convex equiangular polygons, we get a
real hyperbolic structure on the space of
similarity classes of
strictly convex equiangular polygons.
All this is well-known,
and I will give a self-contained account
in \S \ref{hypXX}.
Thurston's construction realizes the moduli space
of strictly convex equiangular $N$-gons as the
interior of convex polytope in hyperbolic space $\H^{N-3}$.
The polytope is canonically defined up to the action
of algebraic matrices. Hence it makes
sense to talk about algebraic points in this polytope.
 When $N$ is odd the polytope
is bounded and when $N$ is even it has some
ideal vertices.

We use our correspondence
$[L]\to [P_L]$ to give
a hyperbolic structure to the moduli
space ${\cal C\/}_N$
of strictly convex equilateral $N$-gons.
We call this the {\it algebraic hyperbolic structure\/}.
Equipped with the algbraic hyperbolic structure,
${\cal C\/}_N$ is just the Thurston polytope in
$\H^{N-3}$.
Our correspondence just imports the
Thurston construction to the equilateral case.

\begin{theorem}
  \label{algebra}
  Relative to the algebraic hyperbolic structure,
  a similarity class in ${\cal C\/}_N$ has a
  representative  with algebraic coordinates
    if and only if the class has algebraic coordinates
  in $\H^{N-3}$.
\end{theorem}

Our construction only
works in the strictly convex case but there is
trick to extend the construction to the general case.
Using the action of the permutation group,
which acts on the space ${\cal A\/}_N$
of all equilateral $M$-gons,
we can extend our hyperbolic structure on
${\cal C\/}_N$ to one on ${\cal A\/}_N$.
See \S \ref{put}.
In the even case, we need to adjoin the
ideal vertices to ${\cal A\/}_N$ in order to
include the degenerate polygons that lie in a
single line.

When $N>6$ the space ${\cal A\/}_N$ has
various conical singularities because the
various copies of ${\cal C\/}_N$ do not
fit nicely together around codimension $2$
faces.
The picture of ${\cal A\/}_N$ is very satisfying
when $N=5,6$.

\begin{theorem}
  \label{penta}
  Relative to the algebraic hyperbolic structure,
    ${\cal A\/}_5$  is a
hyperbolic surface of Euler characteristic $-3$.
The space is tiled by $12$ regular right angled
pentagons which meet $4$ around
each vertex.
\end{theorem}

\begin{theorem}
  \label{hex}
  Relative to the algebraic hyperbolic structure,
  ${\cal A\/}_6$  is a finite volume
  $10$-cusped hyperbolic $3$-manifold
  that is tiled by $15$ regular ideal
  octahedra which meet $4$ around each edge.
\end{theorem}

One thing that inspired me to
consider equilateral polygons is that I had recently
heard a great talk given by 
Juergen Richter-Gebert [{\bf R-G\/}]
about his hyperbolic structure on the space of
equilateral pentagons.
Richter-Gebert has a different way to
give a hyperbolic structure in the pentagonal case.
His construction, like that in [{\bf KM\/}], is
transcendental and seemingly hard to compute.

I would describe the original version of the
paper as a meal that I threw
together based on ideas that were dropped
on my plate while I dined in Heidelberg
and Marseille during a very happy summer
in 2023.  I probably had the key idea for this paper
while in free-fall
riding the Hurricane Loop waterslide at
Miramar water park in Weinheim.
Thanks to the probing comments of
the anonymous referees, and also thanks to a
great insight of Jannik Westermann
which I describe in \S \ref{leftright},
this version of the paper is both deeper
and sharper than the original.

I thank Peter Albers, Diana Davis, Peter Doyle,
Aaron Fenyes, Fabian Lander, 
Juergen Richter-Gebert, Joe Silverman,
Sergei Tabachnikov,
Jannik Westermann, and two anonymous referees
for helpful discussions about this paper.
Finally, I am grateful for the support I've had
during this time period from the University of Heidelberg,
from CIRM (Luminy), from
the National Science Foundation, and from the Simons
Foundation.

\newpage
