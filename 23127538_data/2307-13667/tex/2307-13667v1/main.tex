\documentclass{amsart}
\usepackage[utf8]{inputenc}
\usepackage[english]{babel}
\usepackage{enumitem}
\usepackage{hyperref}



\title{Characterising quasi-isometries of the free group}

\newcommand{\nest}{\sqsubseteq}
\usepackage{mathabx}
\newcommand{\propnest}{\sqsubsetneq}
\newcommand{\orth}{\bot}
\newcommand{\transverse}{\pitchfork}
\newcommand{\gate}{\mathfrak g}
\newcommand{\E}{\mathbb E}
\newcommand{\p}{\mathbb P}
\newcommand{\bb}{\mathbb}
\newcommand{\drift}{\mathsf{drift}}
\newcommand{\mc}{\mathcal}
\newcommand\norm[1]{\lVert#1\rVert}
\newcommand{\ul}{\underline}



%\usepackage{amsthm}
%\usepackage{amsmath}
\usepackage{dsfont}


\newtheorem{theorem}{Theorem}[section]
\newtheorem{thm}{Theorem}[section]
\newtheorem{lemma}[theorem]{Lemma}
\newtheorem{corollary}[theorem]{Corollary}
\newtheorem{conjecture}[theorem]{Conjecture}
\newtheorem{proposition}[theorem]{Proposition}
\newtheorem{prop}[theorem]{Proposition}
\newtheorem{prop-defn}[theorem]{Proposition-Definition}
\newtheorem{claim}{Claim}
\newtheorem*{theorem:main}{Main Theorem} 
\theoremstyle{definition} 
\newtheorem{defn}[theorem]{Definition}
\newtheorem{definition}[theorem]{Definition}
\newtheorem{remark}[theorem]{Remark}
\newtheorem{question}[theorem]{Question}
\newtheorem*{remark*}{Remark}
\newtheorem*{remarks*}{Remarks}
\newtheorem{conv}[theorem]{Convention}
\newtheorem{example}[theorem]{Example}
\newtheorem{assumptions}{Assumption}[section]
\newtheorem{notation}[theorem]{Notation}


\usepackage{graphicx}
\usepackage{commath}
\usepackage{hyperref}
\sloppy

\sloppy
\textwidth=16cm \textheight=23cm
\addtolength{\topmargin}{-40pt} \addtolength{\oddsidemargin}{-2cm}
\addtolength{\evensidemargin}{-2cm}
%\usepackage{showkeys}
\usepackage{dsfont}

\usepackage{tikz}



\usepackage{amssymb}
\usepackage{mathtools} 
\usepackage{float}


\newtheorem{thmintro}{Theorem}
\renewcommand{\thethmintro}{\Alph{thmintro}}



\DeclarePairedDelimiter\floor{\lfloor}{\rfloor}
\DeclarePairedDelimiter\ceil{\lceil}{\rceil}

\author[Antoine Goldsborough]{Antoine Goldsborough}
	\address{Maxwell Institute and Department of Mathematics, Heriot-Watt University, Edinburgh, UK}
	\email{ag2017@hw.ac.uk}

\author{Stefanie Zbinden}
	\address{Maxwell Institute and Department of Mathematics, Heriot-Watt University, Edinburgh, UK}
	\email{sz2020@hw.ac.uk}

\begin{document}

\maketitle



\begin{abstract}
We introduce the notion of mixed subtree quasi-isometries, which are self quasi-isometries of regular trees built in a specific inductive way. We then show that any self quasi-isometry of a regular tree is at bounded distance from a mixed-subtree quasi-isometry. Since the free group is quasi-isometric to a regular tree, this provides a way to describe all self quasi-isometries of the free group. In doing this, we also give a way of constructing quasi-isometries of the free group.  
\end{abstract}

\section{Introduction}

Quasi-isometries are the most fundamental maps in geometric group theory. However, for most metric spaces, very little is know about their quasi-isometry group and there are no known tangible ways to describe all quasi-isometries, except in some cases where quasi-isometric rigidity is known. Notable exceptions to this are Baumslag-Solitar groups which are described in \cite{WhyteBaumslag} and 3-dimensional solvable lie groups which have been studied by Eskin, Fisher and Whyte in \cite{EskinFisherWhyte07,EskinFisherWhyte, EskinFisherWhyte2}. 

With this paper, we add the free group $\mathbb{F}_2$, or more generally regular trees, to the list of spaces where all quasi-isometries up to bounded distance can be described. 
In particular, we introduce the notion of a $D$-mixed subtree quasi-isometry which is a type of quasi-isometry from regular trees to themselves. While a precise definition can be found in Section \ref{sec:qi_of_reg_trees}, the main idea behind them is the following; having defined the quasi-isometry for vertices $v$ at distance $nD$ from the root, one next defines what the quasi-isometry does on the next level, that is, vertices at distance $(n+1)D$ from the root. Moreover, the valid choices of extending the map to the vertices at distance $(n+1)D$ only depend on which of the vertices of distance $nD$ are mapped to the same vertex, but is otherwise independent of the choices made previously. 

Our main theorem below states that a map from a regular tree to itself is a quasi-isometry if and only if it is at bounded distance from a mixed-subtree quasi-isometry.

\begin{thm}\label{thm:main}
     Let $T$ be a regular tree of degree at least 3, rooted at $v_0$. Let $f: T \to T$ be a $(C,C)$-quasi-isometry such that $f(v_0)=v_0$. Then there is a constant $D$ only depending on $C$ and a $D$-deep mixed subtree quasi-isometry $g: T\to T$ such that $f$ and $g$ are at bounded distance from each other. 
\end{thm}

Since regular trees of degree at least 3 and non-elementary free groups are quasi-isometric, the theorem above describes quasi-isometries of the free group $\mathbb F_2.$



Thanks to this independence mentioned above, mixed-subtree quasi-isometries are a useful tool to construct quasi-isometries with certain desired properties. For example, this technique was used in \cite{goldsboroughzbinden}, where the authors build a self quasi-isometry of $\mathbb F_2$ with the property that the push-forward of a simple random walk by this quasi-isometry does not have a well-defined drift.


We suspect that there might be other applications of this construction. For instance, one might want to consider `random quasi-isometries' of $\mathbb F_2$ and properties of a `generic' quasi-isometry. Further, this characterisation might allow to better understand the quasi-isometry group $QI(\mathbb F_2)$.


\smallskip
\textbf{Outline.} In Section \ref{sec:prelim} we introduce the relevant notation and prove some of the technical results about quasi-isometries of trees. In particular, we extend a result of \cite{Nairne} and show that any quasi-isometry is at bounded distance from an order-preserving quasi-isometry. In Section \ref{sec:qi_of_reg_trees} we describe mixed-subtree quasi-isometries and prove Theorem \ref{thm:main}, which states that a map from a rooted tree of degree at least 3 to itself is a quasi-isometry if and only if it is at bounded distance from a mixed-subtree quasi-isometry.\\

\smallskip
\textbf{Acknowledgments.} We would like to thank Oli Jones, Alice Kerr, Patrick Nairne and our supervisor Alessandro Sisto for helpful discussions and feedback.


\section{Preliminaries}\label{sec:prelim}

In this section, we introduce the relevant notation and some preliminary lemmas.  Throughout this paper, we will view $\mathbb F_2$ as a rooted tree. Therefore, our results will cover self-quasi-isometries of rooted trees. 

\begin{definition}
Let $(X, d)$ be a metric space, we say that a map $f: X\to X$ is a \emph{$C$-quasi-isometric embedding} if
\begin{align*}
    \frac{d(x, y)}{C} - C\leq d(f(x), f(y)) \leq C d(x, y) + C.
\end{align*}
for all $x, y\in X$.

Further, we say that a $C$-quasi-isometric embedding $f: X\to X$ is a \emph{$C$-quasi-isometry} if there exists a constant $D$ such that for all $y\in X$ there there exists $x\in X$ such that $d(y, f(x))\leq D$. 
\end{definition}

%\textbf{Notation:} We will say that $f$ is a \emph{C-quasi-isometry} if $f$ is a $(C,C)$-quasi-isometry.


\begin{definition} Let $(X, d)$ be a metric space. Two maps $f, g : X\to X$ are \emph{$C$-bounded} if $d(f(x), g(x))\leq C$ for all $x\in X$. They are \emph{bounded} if they are $C$-bounded for some constant $C$.
\end{definition}




\subsection{Notation on trees}

Let $T$ be a rooted tree and $w\in T$ a vertex.  We denote the subtree rooted at $w$ by $T_w$. 
Vertices $v\in T_w$ are called \emph{descendants} of $w$ and $w$ is called an ancestor of $v$. Further, a vertex $v\in T_w$ is a \emph{$D$-child} of $w$ if $d(v, w) = D$. We denote the parent of a vertex $v\in T$ by $p(v)$ and say that the parent of the root is itself.% With this notation for $v\in T$, $v$ is the $k$-child of $p^k(v)$.\marginpar{This statement is wrong.}

We will view a path between vertices $u$ and $v$ as a sequence of neighbouring vertices $u=u_0,u_1, \ldots, u_n=v$, denoted by $(u_0, \ldots, u_n)$. If a path $(u_0, \ldots, u_n)$ is geodesic (or equivalently non-backtracking) we also denote it by $[u_0, u_n]$. 


\begin{definition}
For a subset $U \subseteq T$ of a rooted tree $T$ at $v_0$, we define the \emph{lowest common ancestor} of $U$ as the (unique) vertex $v\in T$ furthest away from $v_0$ such that every vertex $u\in U$ is a descendant of $v$. We will denote this vertex $v$ as $LCA(U)$.
\end{definition}

Observe that if $v = LCA(U)$, then there exists a pair of vertices $x, y\in U$ such that $v$ lies on $[x, y]$.

\begin{definition}
    Let $S$ be a finite subtree of a rooted tree $T$. We say that the \emph{boundary} of $S$, denoted by $\partial S$, is the set of vertices $v\in T\setminus S$ whose parent $p(v)$ is in $S$.
\end{definition}

\begin{definition}
Let $T$ be a tree rooted at $v_0$. A map $f: T\to T$ is \emph{order-preserving} if for every pair of vertices $u, v\in T$ with $v\in T_u$ we have that $f(v)\in T_{f(u)}$.
\end{definition}

In \cite{Nairne}, Nairne shows that every $(1,C)$-quasi-isometries between spherically homogeneous trees is at bounded distance from an order-preserving quasi-isometry. In Lemma \ref{lemma:close_to_growingQI} we extend this result and show that any $C$-quasi-isometry of a rooted tree to itself is at bounded distance from an order-preserving quasi-isometry. 

\subsection{Properties of quasi-isometries of trees} We state and prove three key technical lemmas about properties of quasi-isometries of trees.

The following lemma states that the image of the geodesic $[u, v]$ under a quasi-isometry $f$ coarsely surjects onto the geodesic $[f(u), f(v)]$.  

\begin{lemma}\label{lemma:close_to_geodesic}
Let $T$ be a tree and let $f: T\to T$ be a $C$-quasi-isometry. For every pair of vertices $u, v\in  T$ and vertex $a\in [f(u), f(v)]$ there exists a vertex $b\in [u, v]$ such that $d(f(b), a)\leq C$.
\end{lemma}

% Figure environment removed

\begin{proof}
Let $[u, v] = (u_0, \ldots , u_n)$. For $1\leq i \leq n$, define $x_i = f(u_i)$ and let $y_i$ be the closest point projection of $x_i$ onto $[f(u), f(v)]$. This is depicted in Figure \ref{fig:path}. 
Let $j$ be the largest index such that $y_j\in [f(u), a]$. Then the path $[x_j, y_j][y_j, y_{j+1}][y_{j+1}, x_{j+1}]$ is non-backtracking and hence a geodesic from $x_j$ to $x_{j+1}$ going through $a$. Since $f$ is a $C$-quasi-isometry, $ d(x_j, a)+d(a, x_{j+1}) = d(x_j ,x_{j+1}) \leq 2C$. So $\max\{d(x_j, a)+d(a, x_{j+1})\}\leq C$. 
\end{proof}


The following lemma states that every quasi-isometry between a rooted tree and itself is at bounded distance from a order-preserving quasi-isometry. This extends the result of \cite{Nairne} where this is shown for $(1, C)$-quasi-isometries between spherically homogeneous trees.

\begin{lemma}\label{lemma:close_to_growingQI}
Let $T$ be a tree rooted at $v_0$ and let $f:T \to T$ be a $C$-quasi-isometry. The map $f$ is at bounded distance from an order-preserving quasi-isometry. Moreover, if $f(v_0) = v_0$, then $f$ is at $K$-bounded distance from an order-preserving $(2K+C)$-quasi-isometry for some $K$ depending only on $C$.
\end{lemma}

% Figure environment removed
\begin{proof}
It suffices to show the moreover part with $K =  3C^3+2C$. Define $g: T\to T$ via $g(v) := LCA(f(T_v))$. Clearly, $g$ is order-preserving. It remains to show that $g$ is at $K$-bounded distance from $f$ since it then follows that $g$ is a $(C, 2K+C)$-quasi-isometry .

Let $u\in T$ be a vertex, we will show that $d(f(u),g(u)) \leq K$. We have $f(u)\in T_{g(u)}$, thus by Lemma \ref{lemma:close_to_geodesic}, there exists $w\in [v_0, u]$ such that $d(f(w), g(u))\leq C$. This is depicted in Figure \ref{fig:close_to_growing}. Since $g(u) = LCA(f(T_u))$, there exist vertices $x, y\in T_u$ such that $g(u)\in [f(x), f(y)]$.  Again by Lemma \ref{lemma:close_to_geodesic}, there exists a vertex $z\in [x, y]\subset T_u$ with $d(g(u), f(z))\leq C$. In particular, $d(f(w), f(z))\leq 2C$.

Observe that $u\in [w, z]$. Hence, $d(u, z)\leq d(w, z)\leq 3C^2$. Therefore, $d(g(u), f(u))\leq d(g(u), f(z))+d(f(z), f(u))\leq 3C^3+2C = K$.
\end{proof}

The following lemma states that if $f$ is an order-preserving quasi-isometry and two vertices $u, v$ have the same distance from the root $f(u)$ cannot be a descendant of $f(v)$, unless they are close. This lemma is a key ingredient in the proof of Lemma \ref{lemma:every_qi_is_mixed}.

\begin{lemma}\label{lemma:same_dist_properties}
Let $T$ be a tree rooted at $v_0$ and let $f: T \to T$ be an order-preserving $C$-quasi-isometry. Let $u, v\in T$ be vertices such that $d(v_0, u) = d(v_0, v)$ and $f(u)\in T_{f(v)}$. Then $d(f(u), f(v))\leq K$ and $d(u, v)\leq K$ for some constant $K$ depending only on $C$.
\end{lemma}

% Figure environment removed

\begin{proof}
Let $ w = LCA(\{u, v\})$. Since $f$ is order-preserving, $f(v)$ lies on $[f(u), f(w)]$. This is depicted in Figure \ref{fig:same_height}. By Lemma \ref{lemma:close_to_geodesic}, there exists a vertex $x\in [u,w]$ such that $d(f(x), f(v))\leq C$. Thus $d(w, v)\leq d(x,v )\leq 2C^2$. Since $d(v_0, u) = d(v_0, v)$, $d(u, v) = 2d(w, v)$ and hence $d(f(u), f(v))\leq 4C^3 + C$. So choosing $K = 4C^3 + C$ works.
\end{proof}

\section{Quasi-isometries of regular trees}\label{sec:qi_of_reg_trees}
\textbf{Notation:} For the rest of this section, $T$ denotes a regular tree of degree $d \geq 3$ rooted at a vertex $v_0$.\\

In this section, we describe a way of building quasi-isometries, which we call \emph{mixed-subtree quasi-isometries}, of regular trees to themselves. We further show that any quasi-isometry is at bounded distance from a mixed-subtree quasi-isometry. The key idea behind mixed-subtree quasi-isometries is that they are quasi-isometries which are defined iteratively for vertices further and further away from the root. Moreover, at each step, the allowed choices are in some sense independent from the choices for earlier vertices.

\textbf{Construction:} Let $D$ be a constant. We will inductively (on the distance to $v_0$) construct a quasi-isometry $f$. Define $f(v_0) = v_0$.

% Figure environment removed

Assume we have defined $f$ for some vertex $x\in T$ but not for any descendants of $x$. Let $v = f(x)$ and $X  = \{x_1,\ldots x_k\}= f^{-1}(v)$. Let $B$ be the set of all $D$-children of vertices $x\in X$. We now define $f(h)$ for all vertices $h\in B$. 
\smallskip

Choose any function $f':B \to T_v$ satisfying the following properties.
\begin{enumerate}
    \item\label{item:image_is_boundary_subtree}  $\mathrm{Im}(f')=\partial S$ for some finite subtree $S$ of $T_v$ containing $v$. 
    \item\label{item:kind_of_injective}  If $f'(w) = f'(w')$, then $w$ and $w'$ are $D$-children of the same vertex $x\in X$.
\end{enumerate}
Define $f|_{B} = f'$ and $f(w) = v$ for all $k$-children $w$ of one of the $x\in X$ with $k < D$.

We call any map $f$ constructed this way a \emph{$D$-deep mixed-subtree} quasi-isometry.

The following lemma shows that mixed-subtree quasi-isometries are indeed quasi-isometries. 

\begin{lemma}\label{lemma:is_qi}
For any choice of functions $f'$, the map $f$ constructed is an order-preserving $C$-quasi-isometry, where $C$ only depends on $D$ and $T$.
\end{lemma}
% \begin{remark}\label{rem:subset_conditions}
%    If $v\not\in T_x$, then $f(v)\not\in T_v$. (for all $x\in T$)
% \end{remark}

\begin{proof}
    Let $K = d^D$, where $d$ is the degree of $T$ and let $C = 2 K^2$. For $n\geq 1$, let $Y_n$ be the set of all $Dn$-children of $v_0$ and let $Y = (\cup_{n\geq 1}Y_n )\cup\{v_0\}$. Before showing that $f$ is an order-preserving quasi-isometry, we show that $f(Y_n) = \partial T_n$ for some finite subtree $T_n$ of $T$ containing $T_{n-1}\cup \partial T_{n-1}$ if $n\geq 2$ and containing $v_0$ otherwise. For $n= 1$, this follows directly from \eqref{item:image_is_boundary_subtree} and for $n\geq 2$ this follows by induction. 

    It is clear from the construction that $f$ is order-preserving. Together with the statement above, we have for all $x, x'\in Y_n$, with $f(x)\neq f(x')$ that $f(T_x)$ is disjoint from $T_{f(x')}$.
    In particular, $f(x)$ and $f(x')$ lie on the geodesic from any vertex of $f(T_x)$ to  $T_{f(x')}$.

    Further, the construction ensures that for any vertex $v\in T$ we have that $\abs{f^{-1}(v)}\cap Y_n \leq K$. So any subtree $S$ in the definition of a function $f'$ satisfies that $\abs{S}\leq K^2$. In other words, if $u\in \partial T_n$ and $v\in \partial T_{n+1}$, then $d(u, v)\leq K^2$. This shows that $f$ is $K^2$-coarsely surjective: let $x\in T$ and let $m$ be the smallest integer such that $x\in T_m$. If $m=1$, then $d(x, f(v_0))\leq K^2$, otherwise, $x$ lies on the geodesic between a pair of vertices $v\in \partial T_{m-1}$ and $u\in \partial  T_m$ (which are both in the image of $f$), implying that $d(x, v)\leq d(u, v)\leq K^2$.

    It remains to show that 
    \begin{align*}
        \frac{d(u, v)}{C} - C\leq d(f(u), f(v))\leq C d(u, v) + C
    \end{align*}
    for all vertices $u, v\in T$. To show the right half of the inequality, it is enough to show that for all neighbours $u, v\in T$, we have $d(f(u), f(v))\leq C$. This follows directly from the definition of $f$ and the observations above. Next we show the left half of the inequality. Let $u, v\in T$ be vertices. Define $u_0 = u$ and for $i\geq 1$ define $u_i$ as the lowest ancestor of $u_{i-1}$ which is contained in $Y$. Observe that $d(u, u_i) \leq Di$ while $d(f(u), f(u_i))\geq i-1$. Define $v_i$ for $i\geq 0$ accordingly. Let $j , k$ be minimal such that $u_j = v_k$. From the observations above, we can see that for all $0\leq i\leq j-2$ and $0\leq i' \leq k-2$, $f(u_i)$ and $f(v_{i'})$ lie on the geodesic between $u$ and $v$. Hence, $d(f(u), f(v)) \geq k+j -2$, while $d(u, v)\leq (k+j)D$. The statement follows.
\end{proof}

We are now ready to prove the following lemma which together with Lemma \ref{lemma:is_qi} states that a map $g : T\to T$ is a quasi-isometry if and only if it is at bounded distance from a mixed-subtree quasi-isometry. The lemma is a slightly more detailed version of Theorem \ref{thm:main}.

\begin{lemma}\label{lemma:every_qi_is_mixed}
Let $g : T\to T$ be a $C$-quasi-isometry. There exists a constant $D>0$ and a $D$-deep mixed subtree quasi-isometry $f$ such that $g$ and $f$ are at bounded distance. Moreover, if $g(v_0) = v_0$, then $D$ only depends on $T$ and $C$.
\end{lemma}
\begin{proof}
By Lemma \ref{lemma:close_to_growingQI}, which states that all quasi-isometries are at bounded distance from order-preserving quasi-isometries, it suffices to show the moreover part for an order-preserving quasi-isometry. So we assume in the following that $g$ is order-preserving. 

Let $K$ be the constant of Lemma \ref{lemma:same_dist_properties} and let $D = C(C+K)+1$.
We show that we can iteratively choose suitable functions $f'$ while defining $f$ to make sure that;
\begin{enumerate}[label = \roman*)]
    \item $d(f(u), g(u))\leq K$,\label{cond:bounded}
    \item $g(u)\subseteq T_{f(u)}$,\label{cond:subset}
\end{enumerate}
for the vertices $u$ whose distance $d(u, v_0)$ is divisible by $D$ (which are exactly the vertices $u$ which lie in the domain of one of the functions $f'$). 

Let $v$ be a vertex, $X = f^{-1}(v)$ and assume that $f$ is not defined for any children of elements of $X$. Observe that by construction, all the distances $d(v_0, x)$ are equal for $x\in X$. Let $B$ be the set of all $D$-children of elements of $X$ and let $A = g(B)$. By \ref{cond:subset},  $A\subset T_v$. For $b\in B$, define $f'(b)$ as the vertex $a\in A$ closest to $v_0$ which satisfies $g(b)\in T_{a}$. Observe that $g(b)\in T_{f'(b)}$.

Note that $f(b) = g(b')$ for some $b'\in B$. It follows from Lemma \ref{lemma:same_dist_properties} that $d(f(b), g(b))\leq K$ for all $b\in B$. Therefore, $g|_B$ and $f'$ are at $K$-bounded distance. By \ref{cond:bounded}, $d(f(x), g(x))\leq K$ for all $x\in X$. Hence for a $k$-child $w$ of some $x\in X$ for $k< D$ we have $f(w) = v$ and hence $d(f(w), g(w))\leq K + d(g(v), g(w))\leq K+2C$.

It only remains to show that $f'$ as defined above is a valid choice, that is, $f'$ satisfies \eqref{item:image_is_boundary_subtree} and \eqref{item:kind_of_injective}. For \eqref{item:image_is_boundary_subtree}, define $S = \{ x \in T_v | \text{$x\not\in T_a $ for all $a\in A$}\}$. If $w\in \partial S$, then $w\in T_{a_w}$ for some $a_w\in A$ while its parent is not in $T_{a_w}$. It follows that $w = a_w$. Further, for any $b\in g^{-1}(a_w)$ we have that $f'(b) \in [a_w, v_0]$ but $f'(b)\not\in S$. Hence $f'(b) = a_w$ implying that $a_w \in \mathrm{Im} (f')$. Thus $\partial S = \mathrm{Im}(f')$ is finite, and $S$ can only be infinite if there exists a vertex $u\in S$ with $T_u\subseteq S$. In the latter case, since $g$ is a quasi-isometry, there exists a vertex $u'\in T$ such that $g(u')\in T_u$ and $d(v_0, g(u'))>2CD$. It follows from the proof of \ref{lemma:is_qi} that $u'\in T_{x}$ for some $x\in X$ and hence $u'\in T_b$ for some $b\in B$. But $g(T_b)\subset T_{f'(b)}$ is disjoint from $S$, a contradiction. Thus $S$ is indeed finite. It remains to show that $v\in S$, or in other words, that $v\not\in A$. Let $b\in B$ be a $D$-child of some vertex $x\in X$. By \ref{cond:bounded}, $d(g(b), v)\geq d(g(b), g(x)) - K > 0$, so indeed $g(b)\neq v$. Since this is true for all $b\in B$, $v\not\in A$.

Next we prove \eqref{item:kind_of_injective}. Let $b$ be a $D$-child of $x$ and $b'$ be a $D$-child of $x'$ with $x\neq x'\in X$. We have $d(b, b')\geq 2D$. Thus $d(g(b), g(b'))\geq 2D/C - C > 2K+C$, which implies that $d(f'(b), f'(b')) > C$. In particular, $f'(b)\neq f'(b')$.
\end{proof}

\bibliography{main}
\bibliographystyle{alpha}




\end{document}
