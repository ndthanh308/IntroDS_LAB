\section{Related Works}

\paragraph{Similarity Embedding} Triplet Network \cite{van01, Schroff_2015_CVPR} uses distance calculation to embed images into a space; images in the same category are placed close and those in different categories are far apart. This algorithm has been widely used for diverse subjects such as face recognition and image retrieval. However, as it learns from a single embedding space, it is unsuitable for embedding multiple subjects with multiple categories. Multiple learning models must be created separately according to the number of categories to increase the sophistication level. 
\vspace*{-10pt}
%
% \paragraph{Fashion Image Retrieval} Most studies on fashion image retrieval ~\cite{kiapour2015where,Simo01,Gajic_2018_CVPR_Workshops, Kucer01, Park_2019_CVPR_Workshops} use fashion datasets classified by categories or instances \cite{liu2016deepfashion, Yuying01} rather than attributes. Hence, they use the triplet loss-based learning method described above.
% \vspace*{-15pt}
% %
% \paragraph{CNN-based Attention} Previous works have explored attention mechanisms according to the channel \cite{Hu01, wang01, Woo01} \& spatial attention \cite{Woo01} concept. Through the technique, the network gives more importance to attended objects. CroW \cite{Kalantidis01}is an image retrieval technique based on early CNNs, a concept related to spatial and channel methods; it is applied to retrieval but is not learning-based. Additionally, GLAM \cite{SCH01}, based on channel and spatial attention learning, was applied to retrieval tasks using channel and spatial attention networks.
% \vspace*{-15pt}
\paragraph{Image Retrieval via CNN-based Embedding} Image Retrieval is a common task in computer vision, which is finding relevant images based on a query image. Recent works have explored the CNN-based embedding and attention mechanisms to improve image retrieval. Some works leverage attention mechanisms according to the channel-wise \cite{Hu01, wang01, Woo01} and spatial-wise \cite{Woo01} concepts to assign more importance to attended object in the image. Understanding the detailed characteristics of objects is crucial in image retrieval. This is particularly significant in the fashion domain, where even the same type of clothing can have various attributes such as color, material, and length. Therefore, to excel in attribute-based retrieval,, it is required to recognize disentangled representation for each attribute. The nature of this task is suitable for demonstrating the effectiveness of multi-space embedding. Thus, we show the efficacy of CCA through a fashion attribute-specific retrieval task.

\vspace*{-10pt}
%
\paragraph{CNN based Attributes-Specific Embedding} 
\autoref{fig:fig2} outlines the concepts of existing attribute-specific embedding, similar to our current study. CSN \cite{veit2017} converts the condition into a mask-like representation for multi-space embedding. The mask can be easily applied to the fully connected layer (FC).
ASEN \cite{ma2020fine} joins the attention mechanism with a condition for multi-space embedding. A variation, ASEN++ \cite{dong2021fine}, extended ASEN to 2 stages. These multi-stage techniques are excluded from this study for a fair comparison. 
M2Fashion \cite{Wan01} adds a classifier to the ASEN base.
Unlike CSN, CAMNet \cite{song2022} was extended to 3D feature maps and applied to the spatial attention mechanism, thus enhancing performance. These studies are CNN-based, not self-attention-based like the present study. The recent ViT \cite{VIT} has been successfully applied to many vision tasks. However, there has been no technique of multi-space embedding for specific attributes, as described in this study.

