\subsection{Benchmarking}
\label{sec:eval}

\autoref{tab:fashionAI}, \autoref{tab:DARN}, and \autoref{tab:DeepFashion} present the evaluations for mAP using the metrics in \autoref{sec:Metrics}. \autoref{tab:zappos50k} shows the triplet prediction metric results. In all tables, our method outperforms the SOTA models CSN \cite{veit2017} and ASEN \cite{ma2020fine}. 
% Table A5가 번역에 빠져있어서 제가 추가했습니다.
\vspace*{-10pt}
%
% \begin{table*} [tb!]
\renewcommand{\arraystretch}{1.25} % 1.25
\centering 
\resizebox{2\columnwidth}{!}
{
\begin{tabular}{l*{10}{c}}
\toprule
\multirow{2}{*}{{\tb{Method}}} & \multirow{2}{*}{{\tb{Backbone}}} & \multirow{2}{*}{{\tb{mAP}}}& \multicolumn{8}{c}{{\tb{mAP for each attribute}}}  \\
\cmidrule(l){4-11}
 & & &  \tb{skirt length} & \tb{sleeve length} & \tb{coat length} & \tb{pant length} & \tb{collar design} & \tb{lapel design} & \tb{neckline design} & \tb{neck design} \\
\toprule
Random baseline \cite{ma2020fine}  &  R50  &  15.79 & 17.20 & 12.50 & 13.35 & 17.45 & 22.36 & 21.63 & 11.09 & 21.19 \\
Triplet network \cite{ma2020fine}  &  R50  & 38.52 & 48.38 & 28.14 & 29.82 & 54.56 & 62.58 & 38.31 & 26.64 & 40.02  \\
CSN  \cite{ma2020fine}    &  R50 & 53.52 & 61.97 & 45.06 & 47.30 & 62.85 & 69.83 & 54.14 & 46.56 & 54.47  \\
ASEN \cite{ma2020fine}  & R50 & {61.02} & {64.44} & {54.63} & {51.27} & {63.53} & {70.79} & {65.36} & {59.50} & {58.67} \\
CAMNet \cite{song2022}  & R50   & \color{black}{{61.97}} & 64.14 & \color{black}{{56.22}} & \color{black}{{53.05}} &  \color{black}{{65.67}} & \color{black}{{72.60}} & \color{black}{67.74} & \color{black}{{63.05}} &  \color{black}{{61.97}} \\
ASEN++ \cite{dong2021fine} & R50 &  64.31 & 66.34 & 57.53 & 55.51 & 68.77 & 72.94 & 66.95 & 66.81 & \textbf{67.01} \\

TF-CSN$^\dagger$  & ViT & \color{black}{{64.86}} & \color{black}{{66.73}} & \color{black}{{59.58}} & \color{black}{{59.94}} & \color{black}{{70.91}} & \color{black}{{71.45}} & \color{black}{{68.17}} & \color{black}{{64.92}} & \color{black}{{62.33}} \\
TF-ASEN$^\dagger$  & ViT & \color{black}{{64.21}} & \color{black}{{65.86}} & \color{black}{{60.11}} & \color{black}{{59.74}} & \color{black}{{70.20}} & \color{black}{{70.80}} & \color{black}{{67.01}} & \color{black}{{64.08}} & \color{black}{{59.48}} \\
\toprule
\multicolumn{10}{c}{{\tb{Ours }}} \\
\hline
\toprule
CCA (Type-1)  & ViT & \color{black}{{66.06}} & \color{black}{{67.20}} & \color{black}{{62.34}} & \color{black}{{60.47}} & \color{black}{{70.29}} & \color{black}{\textbf{75.93}} & \color{black}{{70.32}} & \color{black}{{65.76}} & \color{black}{{61.04}} \\
CCA (Type-2) & ViT & \color{black}{\textbf{69.03}} & \color{black}{\textbf{69.55}} & \color{black}{\textbf{65.92}} & \color{black}{\textbf{64.43}} & \color{black}{\textbf{72.74}} & \color{black}{{75.39}} & \color{black}{\textbf{71.89}} & \color{black}{\textbf{70.42}} & \color{black}{{63.85}} \\

\bottomrule
\end{tabular}
 }
 \vspace{-8pt}
 \caption{mAP comparisons of our methods against other studies on FashionAI. Bold: the best results among all methods. 
 %Blue: our results better than the counterparts.
 Bold black: the best results among the counterparts. TF is Transformer. R50 is ResNet50. $\dagger$ indicates our reproduced results.}
\label{tab:fashionAI}
\end{table*}
\paragraph{FashionAI} 
In \autoref{tab:fashionAI}, our method achieves SOTA performance for all categories except neck design. Overall, we achieve a +4.72\% performance improvement.
% In \autoref{tab:fashionAI}, our method achieves SOTA performance for all categories except neck design. For each evaluation item, our method yields the following performance improvements: +3.21\% for skirt length, +8.39\% for sleeve length, +8.92\% for coat length, +3.97\% for pant length, +2.45\% for collar design, +4.15\% for lapel design, and +3.61\% for neckline design. Overall, this corresponds to a performance improvement of +4.72\%.

\begin{table*} [tb!]
\renewcommand{\arraystretch}{1.2}
\centering 
\resizebox{2\columnwidth}{!}{
\begin{tabular}{l*{11}{c}}
\toprule
\multirow{2}{*}{\textbf{Method}} &  \multirow{2}{*}{{\tb{Backbone}}} & \multirow{2}{*}{\textbf{mAP}} & \multicolumn{9}{c}{\textbf{mAP for each attribute}}  \\
\cmidrule(l){4-12}
 & & &  \tb{clothes category} & \tb{clothes button} & \tb{clothes color} & \tb{clothes length} & \tb{clothes pattern} & \tb{clothes shape} & \tb{collar shape} & \tb{sleeve length} & \tb{sleeve shape} \\
%\cmidrule(l){1-11}
\toprule
Random baseline \cite{ma2020fine} & R50 & 32.26 & 8.49 & 24.45 & 12.54 & 29.90 & 43.26 & 39.76 & 15.22 & 63.03 & 55.54 \\
Triplet network  \cite{ma2020fine}  & R50 & 40.14 & 23.59 & 38.07 & 16.83 & 39.77 & 49.56 & 47.00 & 23.43 & 68.49 & 56.48  \\
CSN \cite{ma2020fine}               & R50 & 50.86  & 34.10 & 44.32 & 47.38 & 53.68 & 54.09 & 56.32 & 31.82 & 78.05 & 58.76\\
ASEN \cite{ma2020fine}               & R50 & {53.31} & {36.69} & {46.96} & {51.35} & {56.47} & {54.49} & {60.02} & {34.18} & {80.11} & {60.04} \\
CAMNet \cite{song2022}$^\dagger$ & R50   & 44.32 & 25.24 & 38.02 & 47.01 & 45.25 & 48.35 & 45.57 & 23.33 & 71.69 & 55.89  \\ 
M2Fashion \cite{Wan01}             & R50 & {54.29} & {36.91} & {48.03} & {51.14} & {57.51} & {56.09} & 60.77 & {35.05} & {81.13} & {62.23} \\
ASEN++ \cite{dong2021fine} & R50 & 55.94  & 40.15 & 50.42 & 53.78 & 60.38 & 57.39 & 59.88 & 37.65 & 83.91 & 60.70 \\
TF-CSN$^\dagger$  & ViT & {62.85} & {48.65} & {60.71} & {53.27} & {66.18} & {63.70} & {72.75} & {45.95} & {88.36}  & 66.35 \\
TF-ASEN$^\dagger$  & ViT & 33.52 & 6.20 & 23.28 & 31.24 & 31.37 & 41.16 & 39.02 & 15.57 & 60.88 & 54.16 \\
\toprule
\multicolumn{10}{c}{\textbf{Ours}} \\
\hline
\toprule
CCA (Type-1)  & ViT & 66.78 & 51.56 & 65.55 & 55.94 & 72.95 & 66.97 & 75.80 & 51.37 & 90.08 & \color{black}{\textbf{71.44}} \\
CCA (Type-2) & ViT & \color{black}{\textbf{68.09}} & \color{black}{\textbf{53.04}} & \color{black}{\textbf{68.21}} & \color{black}{\textbf{56.65}} & \color{black}{\textbf{74.71}} & \color{black}{\textbf{70.12}} & \color{black}{\textbf{77.03}} & \color{black}{\textbf{52.51}} & \color{black}{\textbf{90.23}} & \color{black}{{70.99}} \\

\bottomrule
 \end{tabular}
}
\centering
\vspace{-8pt}
\caption{mAP comparisons of our methods against other studies on DARN. $\dagger$ indicates our reproduced results.} 
\label{tab:DARN}
\end{table*}



\begin{table*} [tb!]
\renewcommand{\arraystretch}{1.2}
\centering 
\resizebox{2\columnwidth}{!}{
\begin{tabular}{l*{8}{c}}
\toprule
\multirow{2}{*}{\textbf{Method}} & \multirow{2}{*}{\Th{\tb{Backbone}}} & \multirow{2}{*}{\textbf{mAP}} & \multicolumn{5}{c}{\textbf{mAP for each attribute}}  \\
\cmidrule(l){4-8}
 & & & \tb{texture-related} & \tb{fabric-related} & \tb{shape-related} & \tb{part-related} & \tb{style-related} \\
%\cmidrule(l){1-11}
\toprule
Random baseline \cite{ma2020fine} & R50 & 3.38 & 6.69 & 2.69 & 3.23 & 2.55 & 1.97 \\
Triplet network  \cite{ma2020fine}  & R50 & 7.36 &  13.26 & 6.28 & 9.49 & 4.43 & 3.33 \\
CSN \cite{ma2020fine}  & R50 & 8.01 & 14.09 & 6.39 & 11.07 & 5.13 & 3.49 \\
ASEN \cite{ma2020fine}  & R50 & 8.74 & 15.13 & 7.11 & 12.39 & 5.51 & 3.56 \\
ASEN++ \cite{dong2021fine} & R50 & 9.64 & 15.60 & 7.67 & 14.31 & 6.60 & 4.07 \\
TF-CSN$^\dagger$  & ViT & 10.04 & 15.27 & 8.11 & 14.91 & 7.40 & 4.51 \\
TF-ASEN$^\dagger$  & ViT & 8.53 & 13.98 & 6.56 & 13.39 & 5.61 & 3.13 \\
\toprule
\multicolumn{8}{c}{\textbf{Ours}} \\
\hline
\toprule
CCA (Type-1) & ViT & 10.64 & 16.18 & 8.38 & 15.98 & 7.99 & 4.78 \\
CCA (Type-2) & ViT & \color{black}{\textbf{11.04}} & \color{black}{\textbf{16.76}} & \color{black}{\textbf{8.42}} & \color{black}{\textbf{16.83}} & \color{black}{\textbf{8.47}} & \color{black}{\textbf{4.92}} \\
\bottomrule
 \end{tabular}
}
\centering
\caption{mAP comparisons of our methods against other studies on DeepFashion.} 
\label{tab:DeepFashion}
\end{table*}



\vspace*{-10pt}
%
\paragraph{DARN} 
In \autoref{tab:DARN}, the proposed model yields SOTA performance for all items. Averaged across the board, it shows a significant performance improvement of +12.15\%.
% In \autoref{tab:DARN}, the proposed model yields SOTA performance for all items. For each evaluation item, our method yields the following performance improvements: +12.89\% for the clothes category, +17.79\% for clothes button, +2.87\% for clothes color, +14.33\% for clothes length, +12.73\% for clothes pattern, +16.26\% for clothes shape, +14.86\% for collar shape, +6.32\% for sleeve length, and +10.29\% for sleeve shape. Overall, this corresponds to a performance improvement of +12.15\%.

\vspace*{-10pt}
%
%\input{exp/sota_deepfashion.tex}
\paragraph{DeepFashion}
% In \autoref{tab:DeepFashion}, the proposed model yields SOTA performance for all items with the following performance improvements: +1.16\% for texture-related, +0.75\% for fabric-related, +2.52\% for shape-related, +1.87\% for part-related, and +0.85\% for style-related. Overall, this corresponds to a performance improvement of +1.4\%. As shown in \autoref{tab:datasets}, although it consists of only five attributes, these contain many more classes than FashionAI and DARN at 1000, resulting in a relatively low mAP value.
In \autoref{tab:DeepFashion}, the proposed model yields SOTA performance for all items. Overall, we achieve a performance improvement of +1.4\%. As shown in \autoref{tab:datasets}, although it consists of only five attributes, these contain many more classes than FashionAI and DARN at 1000, resulting in a relatively low mAP value.



\begin{table} [tb!]
\renewcommand{\arraystretch}{1.2}
\tiny
\centering 
\resizebox{1\columnwidth}{!}{
\begin{tabular}{lr}
\toprule
\textbf{Method} & \textbf{Prediction Accuracy(\%)}\\
\midrule
Random baseline \cite{ma2020fine}          & 50.00 \\
Triplet network \cite{veit2017} & 76.28 \\
CSN \cite{veit2017}                       & 89.27 \\
ASEN \cite{ma2020fine}                     & 90.79 \\
ADDE-C \cite{hou2021learning}                     & 91.37 \\
TF-CSN$^\dagger$                      & 94.78 \\
TF-ASEN$^\dagger$                    & 94.56 \\
\toprule
\multicolumn{2}{c}{\textbf{Ours}} \\
\hline
\toprule
CCA (Type-1)                     & {\textbf{94.98}} \\
CCA (Type-2)                    & 94.85 \\
\bottomrule
\end{tabular}
 }
 \vspace{-8pt}
 \caption{Performance of triplet prediction on Zappos50k.}
\label{tab:zappos50k}
\end{table}

\vspace*{-10pt}

%3.61
\paragraph{Zappos50K} 
\autoref{tab:zappos50k} presents the triplet prediction metric results. Our method achieved SOTA performance, with a +3.61\% improvement compared to the previous method. Unlike the aforementioned datasets, the Zappos50K dataset is relatively simple, as indicated by the category composition in \autoref{tab:datasets} and the example in \autoref{fig:app2}.