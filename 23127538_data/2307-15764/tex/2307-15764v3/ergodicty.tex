\documentclass[11pt]{amsart}
\usepackage[margin=3cm]{geometry}
\title{Geometric Ergodicity and Wasserstein Continuity of 
Non-Linear Filters}

\author{Yunus Emre Demirci}
\address{Queen's University, Department of Mathematics and Statistics, Kingston, Ontario, Canada}
 \email{21yed@queensu.ca}
 
\author{Serdar Yüksel}
\address{Queen's University, Department of Mathematics and Statistics, Kingston, Ontario, Canada} 
 \email{yuksel@queensu.ca}  
 
\usepackage{graphicx}%
\usepackage{multirow}%
\usepackage{amsmath,amssymb,amsfonts}%
\usepackage{amsthm}%
\usepackage{mathrsfs}%
\usepackage[title]{appendix}%
\usepackage{xcolor}%
\usepackage{textcomp}%
\usepackage{manyfoot}%
\usepackage{booktabs}%
\usepackage{algorithm}%
\usepackage{algorithmicx}%
\usepackage{algpseudocode}%
\usepackage{listings}%
\usepackage{amsmath,nicefrac,upgreek,mathtools,enumerate,url}
\usepackage{amsthm,amsmath,amsfonts,amssymb}
\usepackage[numbers]{natbib}
\usepackage{enumitem}
\usepackage{verbatim}
\newcommand{\no}{\noindent}
 \newcommand{\su}{\textnormal{supp }}
\newcommand{\bea}{\begin{eqnarray}}
\newcommand{\ena}{\end{eqnarray}}
\newcommand{\beas}{\begin{eqnarray*}}
\newcommand{\enas}{\end{eqnarray*}}
\newcommand{\beq}{\begin{equation}}
\newcommand{\enq}{\end{equation}}
\def\qed{\hfill \mbox{\rule{0.5em}{0.5em}}}
\newcommand{\bbox}{\hfill $\Box$}
\newcommand{\Lip}{\operatorname{Lip}}

\newcommand\norm[1]{\left\lVert#1\right\rVert}
\newcommand\normx[1]{\Vert#1\Vert}
\newcommand{\T}{\mathcal{T}}
\newcommand{\Q}{\mathcal{Q}}
\newcommand{\Z}{\mathcal{Z}}
\newcommand{\X}{\mathbb{X}}
\newcommand{\Y}{\mathbb{Y}}
\newcommand{\U}{\mathbb{U}}
\newcommand{\B}{\mathbb{B}}
\newcommand{\PP}{\mathbb{P}}
\usepackage{bbm}
\newcommand{\1}{\mathbbm{1}}
\usepackage{color}
\newcommand{\sy}[1]{{\color{magenta} #1}}
%%%%

%%%%%=============================================================================%%%%
%%%%  Remarks: This template is provided to aid authors with the preparation
%%%%  of original research articles intended for submission to journals published 
%%%%  by Springer Nature. The guidance has been prepared in partnership with 
%%%%  production teams to conform to Springer Nature technical requirements. 
%%%%  Editorial and presentation requirements differ among journal portfolios and 
%%%%  research disciplines. You may find sections in this template are irrelevant 
%%%%  to your work and are empowered to omit any such section if allowed by the 
%%%%  journal you intend to submit to. The submission guidelines and policies 
%%%%  of the journal take precedence. A detailed User Manual is available in the 
%%%%  template package for technical guidance.
%%%%%=============================================================================%%%%

%% as per the requirement new theorem styles can be included as shown below
\newtheorem{theorem}{Theorem}%  meant for continuous numbers
%%\newtheorem{theorem}{Theorem}[section]% meant for sectionwise numbers
%% optional argument [theorem] produces theorem numbering sequence instead of independent numbers for Proposition
\newtheorem{proposition}[theorem]{Proposition}% 
%%\newtheorem{proposition}{Proposition}% to get separate numbers for theorem and proposition etc.


\newtheorem{example}{Example}%
\newtheorem{remark}{Remark}%

\newtheorem{definition}{Definition}%

\newtheorem{assumption}{Assumption}%
\newtheorem{corollary}{Corollary}%
\newtheorem{lemma}{Lemma}%


\keywords{Non-linear filtering, unique ergodicity, filter stability}

\begin{document}


%%==================================%%
%% Sample for unstructured abstract %%
%%==================================%%

\begin{abstract}  In this paper, we present conditions for the geometric ergodicity 
and Wasserstein regularity
of non-linear filter processes. While previous studies have mainly focused on unique ergodicity 
and the weak Feller property, our work 
extends these findings in three directions:
(i) We present conditions on the geometric ergodicity 
of non-linear filters, 
(ii) we obtain further conditions on unique ergodicity 
for the case where the state space is compact which complements prior work involving countable spaces, 
and 
(iii) as a by-product of our analysis, we obtain 
Wasserstein continuity of non-linear filters with stronger regularity.
\end{abstract}

%%================================%%
%% Sample for structured abstract %%
%%================================%%

% \abstract{\textbf{Purpose:} The abstract serves both as a general introduction to the topic and as a brief, non-technical summary of the main results and their implications. The abstract must not include subheadings (unless expressly permitted in the journal's Instructions to Authors), equations or citations. As a guide the abstract should not exceed 200 words. Most journals do not set a hard limit however authors are advised to check the author instructions for the journal they are submitting to.
% 
% \textbf{Methods:} The abstract serves both as a general introduction to the topic and as a brief, non-technical summary of the main results and their implications. The abstract must not include subheadings (unless expressly permitted in the journal's Instructions to Authors), equations or citations. As a guide the abstract should not exceed 200 words. Most journals do not set a hard limit however authors are advised to check the author instructions for the journal they are submitting to.
% 
% \textbf{Results:} The abstract serves both as a general introduction to the topic and as a brief, non-technical summary of the main results and their implications. The abstract must not include subheadings (unless expressly permitted in the journal's Instructions to Authors), equations or citations. As a guide the abstract should not exceed 200 words. Most journals do not set a hard limit however authors are advised to check the author instructions for the journal they are submitting to.
% 
% \textbf{Conclusion:} The abstract serves both as a general introduction to the topic and as a brief, non-technical summary of the main results and their implications. The abstract must not include subheadings (unless expressly permitted in the journal's Instructions to Authors), equations or citations. As a guide the abstract should not exceed 200 words. Most journals do not set a hard limit however authors are advised to check the author instructions for the journal they are submitting to.}

%%\pacs[JEL Classification]{D8, H51}

%%\pacs[MSC Classification]{35A01, 65L10, 65L12, 65L20, 65L70}

\maketitle

\section{Introduction}

In this paper, we study the geometric ergodicity and Wasserstein regularity of non-linear filter processes. In our paper, we consider the cases where state space of the hidden Markov model is a compact Polish metric space and the observation space is a Polish metric space. Before we state our main results, we present some preliminaries in the following.
    
%    While previous studies have mainly focused on unique ergodicity 
%    and the weak Feller property, our work 
%    extends these findings in three directions:
%    (i) We present conditions on the geometric ergodicity 
%    of non-linear filters, 
%    (ii) we obtain further conditions on unique ergodicity 
%    for the case where the state space is compact which complements prior work involving countable spaces, 
%    and 
%    (iii) as a by-product of our analysis, we obtain 
%    Wasserstein continuity of non-linear filters with stronger regularity.

%The study of the long-term behavior of hidden Markov models is a classical problem; Blackwell \cite{blackwell1959entropy} studied the entropy rate of the stochastic process defined by $Y_n = h(X_n)$, where $h$ is a non-invertible function and $(X_n)_{n\geq 0}$ is a 
%stationary Markov chain. With the filter process \( (\pi_n)_{n\geq 0}\), defined as \(\pi_n := P(X_n \in \cdot | Y_1,..., Y_n)\). 
%Blackwell's work raised the question of whether the filter process has a unique stationary measure, or in other words, 
%whether it is uniquely ergodic. Blackwell conjectured that this was the case for irreducible underlying Markov chains, 
%but a counterexample by Kaijser \cite{Kaijser} showed that this conjecture was incorrect. The problem of fully characterizing the unique ergodicity of the filtering process has since then received significant attention. 
%
%As we will see in literature review, while the unique ergodicity problem has been studied extensively, 
%there has been no study, to our knowledge, on the geometric 
%ergodicity of non-linear filters. Our paper addresses this gap.


%A crucial step in our analysis toward geometric ergodicity is to establish the 
%Wasserstein continuity and equi-continuity of non-linear filters, which we present as 
%a separate contribution; this generalizes recent studies on weak Feller continuity of non-linear filters.
%Additionally, we establish new conditions that ensure unique ergodicity which complements prior work. 

\subsection{Notation and preliminaries}
Let $(\mathbb{X}, d)$ denote a compact metric space which is the state space of a partially observed Markov process. Let $\mathbb{B}(\mathbb{X})$ be its Borel $\sigma$-field. We denote by $\mathcal{Z}:=\mathbb{P}(\mathbb{X})$ the set of probability measures on $(\mathbb{X}, \mathbb{B}(\mathbb{X}))$ under the weak topology. Let $\mathbb{P}(\mathcal{Z})$ denote the probability measures on $\mathcal{Z}$, equipped with the weak topology. Let $\mathbb{C}(\mathbb{X})$ be the set of all continuous, bounded functions on $\mathbb{X}$. We let $\mathcal{T}$ be the transition kernel of the model which is a stochastic kernel from $\PP(\mathbb{X})$ to $\PP(\mathbb{X})$. Here and throughout the paper $\mathbb{N}$ denotes the set of positive integers. Let $D$ denote the diameter of $\mathbb{X}$, defined as $D = \operatorname{diam}(\mathbb{X}) := \sup\{d(x, y) \mid x, y \in \mathbb{X}\}$.

Let $\mathbb{Y}$ be a Polish space denoting the observation space of the model, and let the state be observed through an observation channel $\Q$. $\Q(\cdot \mid x)$ is a probability measure on $\mathbb{Y}$ for every $x \in \mathbb{X}$, and $\Q(A \mid \cdot): \mathbb{X} \rightarrow[0,1]$ is a Borel measurable function for every $A \in \B(\mathbb{Y})$. 
The update rules of the system are determined by $\mathcal{T}$ and $\Q$:
$$
\operatorname{Pr}\left(\left(X_0, Y_0\right) \in B\right)=\int_B \mu\left(d x_0\right) \Q\left(d y_0 \mid x_0\right), \quad B \in \B(\mathbb{X} \times \mathbb{Y})
$$
where $\mu$ is the (prior) distribution of the initial state $X_0$, and
$$
\operatorname{Pr}\left(\left(X_n, Y_n\right) \in B \mid(X, Y)_{[0, n-1]}=(x, y)_{[0, n-1]}\right)=\int_B \mathcal{T}\left(d x_n \mid x_{n-1}\right) \Q\left(d y_n \mid x_n\right),
$$
$B \in \B(\mathbb{X} \times \mathbb{Y})$, $n \in \mathbb{N}$.

We note that, using stochastic realization results \cite[Lemma~1.2]{gihman2012controlled} \cite[Lemma~3.1]{BorkarRealization}, \cite[Lemma F]{aumann1961mixed}), the above processes also admit a functional representation: In particular, we can realize the above with
\begin{eqnarray}\label{updateEq}
 X_{k+1} &=& F(X_k, W_k) \label{updateEq1} \\
Y_{k} &=& G(X_k, V_k) \label{updateEq2} 
\end{eqnarray}
where $F, G$ are measurable functions and $W_k, V_{k}$ are mutually independent i.i.d. noise processes (taking values without loss in $[0,1]$) and $X_0 \sim \mu$.

\begin{definition}
The filter process is defined as the sequence of conditional probability measures $\pi_n^{\mu}(\cdot)=P^{\mu}\left(X_n \in \cdot \mid Y_{[0, n]}\right), n \in \mathbb{N}$
where $P^{\mu}$ is the probability measure induced by the prior $\mu$.
\end{definition}

Such a partially observed Markov Process (POMP) can be reduced to a completely observable Markov process \cite{Rhe74, Yus76}, whose states are the filter variables, where the state at time $n$ is $\Z$-valued
$$
Z_n:=\operatorname{Pr}\left\{X_n \in \cdot \mid Y_0, \ldots, Y_n\right\}.
$$
The transition kernel $\eta: \mathcal{Z} \rightarrow \mathbb{P}(\mathcal{Z})$ of the filter process can be constructed as follows. Let
\begin{align}\label{F_def}
F(z, y)(\cdot):=\operatorname{Pr}\left\{X_{n+1} \in \cdot \mid Z_n=z, Y_{n+1}=y\right\}
\end{align}
Then,
\begin{align}\label{etatra}
\eta(\cdot \mid z)&=\int_{\mathbb{Y}} \1_{\{F(z, y_1) \in \cdot \}}\operatorname{Pr}(y_1\in dy \mid z_0=z)\\
& =   \int \1_{\{F(z, y_{n+1}) \in \cdot\}}\Q(dy_{n+1}|x_{n+1}){\T}(dx_{n+1}|x_n)z(dx_n). \nonumber
\end{align}

Hence, the filter process is a completely observable Markov process with the state space $\mathcal{Z}$ and transtion kernel $\eta$.

\begin{definition}\label{Dobrushincoefficient}[\cite{dobrushin1956central}, Equation 1.16.] For a kernel operator $K: S_1 \rightarrow \mathcal{P}\left(S_2\right)$ 
we define the Dobrushin coefficient as:
$$
\delta(K)=\inf \sum_{i=1}^n \min \{ K\left(x, A_i\right), K\left(y, A_i\right)\}
$$
where the infimum is over all $x, y \in S_1$ and all partitions $\left\{A_i\right\}_{i=1}^n$ of $S_2$.
\end{definition}

A sequence of probability measures $\left\{\mu_{n}\right\}_{n\in\mathbb{N}}$ from $\mathcal{Z}$ converges weakly to $\mu \in \mathcal{Z}$ if for every bounded continuous function $f: \mathbb{X} \to \mathbb{R}$,
$$
\int_{\mathbb{X}} f(x) \mu_{n}(d x) \rightarrow \int_{\mathbb{X}} f(x) \mu(d x) \quad \text { as } \quad n \rightarrow \infty .
$$
We note that $\mathcal{Z}$ itself is a Polish space with respect to the topology of weak convergence for probability measures when $\mathbb{X}$ is a complete, separable and metric space \cite[ Chapter 2, section 6]{Par67}. 

A sequence of probability measures $\left\{\mu_{n}\right\}_{n\in\mathbb{N}}$ from $\mathcal{Z}$ converges in total variation to $\mu \in \mathcal{Z}$ if
$$
\sup _{A \in \mathbb{B}(\mathbb{X})}\left|\mu_{n}(A)-\mu(A)\right| \rightarrow 0 \text { as } n \rightarrow \infty .$$
One way to metrize $\mathcal{Z}$ under the weak convergence topology is via the bounded Lipschitz metric (\cite{villani2008optimal}, p.109) defined as follows 
\begin{align}\label{boundedlip}
&\rho_{BL}(\mu, \nu) :=\sup \left\{\int_{\mathbb{X}} f(x) \mu(d x)-\int_{\mathbb{X}} f(x) \nu(d x) : f \in \operatorname{BL_1}(\mathbb{X}) \right\},
\end{align}
$\mu, \nu \in \mathcal{Z}$, where
\begin{align*}
\operatorname{BL}_1(\mathbb{X}):=\left\{f: \mathbb{X} \mapsto \mathbb{R},
\norm{f}_{\infty}+\norm{f}_{L}\leq 1 \right\}, \; \norm{f}_{\infty}=\sup_{x\in \mathbb{X}}|f(x)|,
\; \norm{f}_{L}=\sup_{x\neq y} \frac{|f(x)-f(y)|}{d(x,y)}.
\end{align*}
By, e.g., \cite{Bil99}, if $(\mathbb{X}, d)$ is compact, then
the space of probability measures 
$\PP(\mathbb{X})$ equipped with the bounded Lipschitz metric 
is also a compact metric space.

When $\mathbb{X}$ is compact, the Kantorovich-Rubinstein metric (equivalent to the Wasserstein metric of order $1$) (\cite{Bog07}, Theorem 8.3.2) is another metric that metrizes the weak topology on $\mathcal{Z}$. We denote the
$1-$Wasserstein metric by $W_1$, given by
\begin{align}\label{defkappanorm}
&W_1(\mu, \nu):=\sup \left\{\int_{\mathbb{X}} f(x) \mu(d x)-\int_{\mathbb{X}} f(x) \nu(d x) : f \in \operatorname{Lip}(\mathbb{X}) \right\},
\end{align}
$\mu, \nu \in \mathcal{Z}$, where
\begin{align*}
\operatorname{Lip}(\mathbb{X})=\{f:\mathbb{X}\to \mathbb{R},\; \norm{f}_{L}\leq 1\}.
\end{align*}
\section{Main Results and Literature Review}

\begin{definition}
    A Markov chain or transition kernel of a 
    Markov model with state space $\mathbb{X}$ is said to be 
    (uniformly) geometrically
    Wasserstein ergodic if 
    there exists a probability measure 
    $\mu\in P(\mathbb{X})$ and $\rho > 1$ such that 
    \begin{align*}
        \sup_{x\in \mathbb{X}} \rho^n W_1(P^{\delta_x}(X_n\in \cdot), \mu)\to 0
        \quad \text{as } n\to \infty.
    \end{align*} 
\end{definition}

In the following, we present several assumptions; followed by our main results which will utilize their corresponding assumptions.

\begin{assumption} \label{assumption1}
\begin{enumerate}
\item[(i)]  $(\mathbb{X},d)$ is a compact metric space with diameter $D=\sup_{x,y \in \mathbb{X}} (d(x,y))$.
\item[(ii)] There exists $\alpha\in \mathbb{R}_+$ such that 
$\left\|\mathcal{T}(\cdot \mid x)-\mathcal{T}\left(\cdot \mid x^{\prime}\right)\right\|_{T V} \leq d(x,x^{\prime}) \alpha$
for all $x,x' \in \mathbb{X}$.
\item[(iii)] $\alpha D (3-2\delta(\Q)) < 2$.
\end{enumerate}
\end{assumption}

We note that Assumption \ref{assumption1}(ii) implies, by \cite{KSYWeakFellerSysCont}, the weak Feller property of the non-linear filtering kernel $\eta$ defined in (\ref{etatra}).

\begin{definition}[\cite{MeynBook}, Section 6.1.2, pg 137]
    For a Markov chain with transition kernel $P$, 
    a point $x$ is topologically reachable if 
    for every y and every open neighborhood 
    $O$ of $x$, there exists $k>0$ such that $P^k(y, O)>0$.
\end{definition}


\begin{definition}[\cite{MeynBook}, Section 18.4.2, pg 468]
    For a Markov chain with transition kernel $P$, 
    a point $x$ is topologically aperiodic if $P^k(x, O)>0$
    for every open neighborhood 
    $O$ of $x$ and for all $k\in \mathbb{N}$ sufficiently large.
\end{definition}

\begin{assumption}\label{ErgodicityHMM}
\begin{enumerate}
\item[(i)] $(\mathbb{X},d)$ is a compact metric space. 
\item[(ii)] Assumption \ref{assumption1}-ii is fulfilled.
\item[(iii)] The filter process has a topologically reachable state.
\end{enumerate}
\end{assumption}

\begin{comment}
\begin{assumption} \label{ContinuosY} 
    $\mathbb{Y}$ is a Polish space with 
    $\sigma$-finite reference measure $\varphi$. 
    $g(x,y)$ is the density
    function of the transition kernel $Q$ 
    with respect to reference measure 
    $\varphi$, i.e $Q(A|x) = \int_A g(x,y) \varphi(dy)$.
    \begin{enumerate}[label=(\roman*)]
    \item There exists an observation realization $\bar{y} \in \mathbb{Y}$ such that 
    for some positive $\epsilon$, 
    $g(x,y)>\epsilon$ for all $x\in X$. 
    \item The family of functions $\{g(x,\cdot): x \in \mathbb{X}\}$ 
    is equicontinuous at $\bar{y}$. 
    \end{enumerate}
\end{assumption}
\end{comment}

Our main results are as follows:
\begin{theorem}[\bf{Geometric Ergodicity}]\label{Main} Under Assumption \ref{assumption1}, 
    the filter process is geometrically Wasserstein ergodic. 
\end{theorem}

\begin{theorem}[\bf{Unique Ergodicity}]\label{Main2} 
Under Assumption \ref{ErgodicityHMM},
the filter process is uniquely ergodic. 
Moreover, if the topologically reachable state
is aperiodic, then the filter process
converges weakly to the unique stationary 
distribution for any prior distribution.
\end{theorem}

For Markov chains which are not irreducible, 
topologically reachable states play a crucial role, 
not unlike small or petite sets serving as recurrent 
sets for the irreducible setting. Our analysis, 
in Section \ref{existence_reachable}, establishes the existence 
of a reachable state and its implications for unique 
ergodicity and weak convergence.

The weak Feller property of the filter process has attracted 
significant interest with applications on near optimality of 
finite approximations in stochastic control. In the paper, 
we will obtain a generalization of the weak-Feller property. 
Notably, we will show the following, 
which generalizes \cite[Theorem 7(i)]{kara2020near};
the proof is in the appendix.

\begin{lemma}\label{Regularity}[Generalization of \cite[Theorem 7(i)]{kara2020near}]
    Suppose $\mathbb{X}$ and $\mathbb{Y}$ are Polish spaces.
    Assume that Assumption \ref{assumption1}-ii holds,
    we have for all $n \in \mathbb{N}$ the following inequality:
    $$
    \rho_{B L}\left(\eta^n(\cdot \mid z), 
    \eta^n\left(\cdot \mid z^{\prime}\right)\right) \leq 
    3\left(1+\alpha \right) \rho_{B L}\left(z, z^{\prime}\right)
    $$
    for all $z,z' \in \mathcal{Z}$.
\end{lemma}
\begin{proof}
    The proof can be found in the Appendix.
\end{proof}

Moreover, this lemma  demonstrates that the filter process satisfies the properties of an $L$-chain
(Theorem \ref{L-chain}). In other words, for every bounded Lipschitz function $f$,
the sequence $E[f(Z_n)]$ exhibits equicontinuity over $n$. 

\begin{theorem}\label{Talpha}
    Suppose that $Y$ is a Polish space.
    Under the Assumptions \ref{assumption1}-i, \ref{assumption1}-ii,
    we have
    $$
    W_{1}\left(\eta(\cdot \mid z_0), \eta\left(\cdot \mid z_0^{\prime}\right)\right) \leq 
    \left(\frac{\alpha D (3-2\delta(\Q))}{2}\right)W_{1}\left(z_0, z_0^{\prime}\right)
    $$
    for all $z_0,z_0' \in \mathcal{Z}$.
\end{theorem}

We conclude this section with some examples and applications.
For the case with finite $\mathbb{X}$, consider 
the discrete metric $d$ 
defined as follows:
$$d\left(x, x^{\prime}\right)= \begin{cases}1 & \text { if } x \neq x^{\prime} \\ 0 & \text { if } x=x^{\prime}.\end{cases}$$
With this choice of metric, the diameter $D$ is equal to $1$.
By Dobrushin's contraction theorem 
\cite{dobrushin1956central}, we have 
$\norm{\mathcal{T}(\cdot \mid x)-\mathcal{T}\left(\cdot \mid x^{\prime}\right)}_{TV} \leq \norm{ \delta_x- \delta_{x^{\prime}}}_{TV}(1-\delta({\T})) =2 (1-\delta({\T}))$.
Hence, we can choose $\alpha$ to be $2(1-\delta({\T}))$, 
leading to the following result.

\begin{corollary}\label{speccor}
    If $\mathbb{X}$ is finite and $(1-\delta({\T}))(3-2\delta(\Q)) < 1$
    then the filter process is geometrically Wasserstein ergodic.
\end{corollary}

\begin{example}\label{ex1}
    Let $\mathbb{X}=\{0, 1, 2, 3\}$, $\mathbb{Y}=\{0, 1\}$, $\epsilon \in (0,1)$ and let the transition and measurement matrices be given by
    \begin{align}
    \mathcal{T}=\left(\begin{array}{cccc}
    1/3 & 1/3 & 1/6 & 1/6 \\
    0 & 1/2 & 1/6 & 1/3 \\
    1/3 & 1/6 & 1/6 & 1/3 \\
    1/6 & 1/3 & 1/3 & 1/6
    \end{array}\right)
    \quad 
    \Q=\left(\begin{array}{ccc}
    1 & 0 \\
    1-\epsilon & \epsilon \\
    \epsilon & 1-\epsilon\\
    \epsilon & 1-\epsilon
    \end{array}\right).
    \end{align}
\end{example}
Since $\delta({\T})=2/3$ and $\delta(\Q)>0$, 
Example \ref{ex1} satisfies the condition 
$(1-\delta({\T}))(3-2\delta(\Q))<2$ stated in 
Corollary \ref{speccor}. Therefore, we can 
conclude that the filter process associated 
with Example \ref{ex1} is geometrically Wasserstein ergodic.

From Theorem \ref{Main2}, we derive:

\begin{corollary}\label{Main2Cor}
If $\mathbb{X}$ is finite and Assumption 
\ref{ErgodicityHMM}-iii is met, 
the filter process is uniquely ergodic.
\end{corollary}

We finally provide an example where $X$ is continuos.
Let $N(\mu,\sigma^2)$ represent normal distribution with mean 
$\mu$ and variance $\sigma^2$.
Furthermore, let $\bar{N}(\mu,\sigma^2)$ denote truncated version 
of $N(\mu,\sigma^2)$, where the support is restricted to $[0,1] \subset \mathbb{R}$.
Its probability density function $f$ is given by
$$f(x ; \mu, \sigma)=\frac{1}{\sigma} 
\frac{\varphi\left(\frac{x-\mu}{\sigma}\right)}
{\Phi\left(\frac{1-\mu}{\sigma}\right)-\Phi\left(\frac{0-\mu}{\sigma}\right)}$$
Here, $\varphi(\cdot)$ is the probability density function 
of the standard normal distribution and 
$\Phi(\cdot)$ is its cumulative distribution function.
Now let us construct the example:
\begin{example}\label{ContE}
Let $X=[0,1]$ with Euclidean norm, 
$\sigma>0$ and transition kernel 
of the hidden model
be ${\T}(.|x)=\bar{N}(x,\sigma^2)$.
\end{example}

First, we will show that this transition kernel satisfies
Assumption \ref{assumption1}-ii for some $\alpha$.
For any $0\leq x < y \leq 1$:
\begin{align*}
    &\frac{\lVert {\T}(.|y)-{\T}(.|x) \rVert _{TV}}{y-x}
    =\frac{\lVert \bar{N}(y,\sigma^2)-\bar{N}(x,\sigma^2) \rVert _{TV}}{y-x}\\
    &\leq \frac{\lVert N(y,\sigma^2)-N(x,\sigma^2) \rVert _{TV}}{y-x}=
    \frac{2-4\Phi\left(\frac{x-y}{2\sigma}\right)}{y-x}\\
    &=\frac{2}{\sigma}\left(\frac{\Phi(0)-\Phi(\frac{x-y}{2\sigma})}{\frac{y-x}{2\sigma}}\right)
    \leq \frac{2}{\sigma} \Phi'(0)= 
    \frac{\sqrt{2}}{\sigma \sqrt{\pi}}
\end{align*}
The last inequality follows from the fact that 
$\Phi$ is convex on the
negative half-plane.
Therefore Assumption \ref{assumption1}-ii satisfied 
for $\alpha=\sigma \sqrt{2/ \pi}$. 

If we choose $\sigma>3/\sqrt{2\pi}$, then Assumption \ref{assumption1}-iii
satisfied for every observation kernel. 
Under this condition 
we can use Theorem \ref{Main} to conclude that 
the filter is geometrically
Wasserstein ergodic.


\subsection{Comparison with the Literature}
%In the following section, we summarize some previous research on the 
%unique ergodicity of the filter process. 

Kaijser \cite{Kaijser} deals 
with the study of unique ergodicity in 
the context of filtering processes for 
finite state space and finite 
observation space, and this study 
involves the use of the Furstenberg-Kesten 
theory of random matrix products. Kochman and Reeds \cite{kochman2006simple} prove more a general result in a later 
study \cite{kochman2006simple}, also for the finite space case; and assume that the closure of finite multiplications of transition matrices has a rank one element:

%In our proof for a more general case, we 
%utilize a similar technique, which involves 
%relying on a well-known theorem (\cite{MeynBook}, 
%Theorem 18.4.4).

\textbf{Condition KR}. Suppose $\mathbb{X}$ and $\mathbb{Y}$ are 
finite sets, and the transition matrix of the
hidden Markov model is irreducible. 
Define the cone of matrices as
$$
\mathcal{K}=\left\{c M\left(y_1\right) 
\cdots M\left(y_n\right): n \in \mathbb{N}, 
y_1, \ldots, y_n \in \mathbb{Y}, 
c \in \mathbb{R}_{+}\right\} .
$$
Then the closure of $\mathcal{K}$ contains a matrix of rank 1.
Here, $M(y)_{|\mathbb{X}|\times|\mathbb{X}|}$ 
is a  matrix such that
$$M(y)_{ij}=Pr(X_{n+1}=j, Y_{n+1}=y|X_{n}=i).$$

\cite{kochman2006simple} shows 
that any continuous bounded function is 
equicontinuous under the filter process and utilizes the existence of a matrix with 
rank 1 to prove the existence of reachable states and establish that under Condition KR the filter 
process is uniquely ergodic. \cite{kochman2006simple} also shows that if the transition matrix of the 
hidden model is aperiodic then filter process weakly converges to a stationary measure for any initial distribution. A later study by Kaijser \cite{kaijser2011markov} extends the results of Kochman and Reeds to countable $\mathbb{X}$. 

Chigansky and Van Handel \cite{chigansky2010complete} obtain necessary and sufficient conditions for unique ergodicity of the filter 
in the case where $\mathbb{X}$ and $\mathbb{Y}$ both take values in a countable state space.  Additionally, the condition presented in \cite{chigansky2010complete} is also sufficient when the observation space is $\mathbb{R}^d$. \cite{chigansky2010complete} notes also that when $\mathbb{X}$ is finite and $\mathbb{Y}$ is countable, Condition KR is a necessary and sufficient condition for unique ergodicity of the filter process. 

If $X_n$ is an ergodic process, then filter stability and unique ergodicity are closely related properties in both discrete and continuous-time settings 
\cite{Budhiraja, Stettner1989, DiMasiStettner2005ergodicity}. 
Chigansky and Van Handel \cite{chigansky2010complete} 
utilized the previous results on filter stability to propose 
new sufficient conditions for unique ergodicity. %For more detailed results for filter stability, 
%refer to \cite{Handel,HandelUniformObs,chigansky2009intrinsic}. 

\begin{definition}[Nondegeneracy] There exists a probability measure
$\varphi$ on $\mathbb{B}(\mathbb{Y})$ and 
a strictly positive measurable 
function $g: \mathbb{X} \times \mathbb{Y} \rightarrow (0, \infty) $ such that
$$
\Q(A|x)=\int_A g(x, y) \varphi(d y) 
\quad \text { for all } A \in \mathbb{B}(\mathbb{Y}), x \in \mathbb{X}. 
$$
\end{definition}

(\cite{Handel} Theorem $5.2$) shows that if hidden Markov model is strongly ergodic 
(the signal process
converges to the ergodic measure 
in total variation)
and the observation channel is non-degenerate, then the filter is stable. In a setting where the observation state is either countable or finite, the non-degeneracy condition implies that \(\Q(y|x)>0\) for all \(x\in \mathbb{X}\) and \(y\in \mathbb{Y}\)
It's important to note that Example \ref{ex1} and Example \ref{ContE} do not satisfy the non-degeneracy condition.

\cite{mcdonald2020exponential,mcdonald2018stability} provide further conditions for filter stability, and thus, in view of the above, unique ergodicity. Recall that for a kernel $K$, $\delta(K)$ is the Dobrushin coefficient defined in Definition \ref{Dobrushincoefficient}.
If the inequality $(1-\delta(T))(2-\delta(Q))<1$ is satisfied, then the filter is exponentially stable. Earlier studies via the Dobrushin coefficient analysis with complementary conditions include \cite{del2001stability}.

Our analysis also relies on results from \cite{kara2020near}, 
which we will discuss in detail in the following section. The weak Feller continuity properties of filters has been studied in \cite{CrDo02,FeKaZg14,KSYWeakFellerSysCont} under complementary conditions. We refer the reader to \cite[Theorem 7]{kara2020near}, which builds upon \cite{KSYWeakFellerSysCont}, providing further refinements with explicit moduli 
of continuity for the weak Feller property. Further results on the weak Feller property, 
including a converse theorem statement involving 
\cite{KSYWeakFellerSysCont}, can be found in 
\cite{feinberg2022markov, feinberg2023equivalent}, 
which consider both the necessity and sufficiency conditions for the weak Feller property.

{\bf Comparison.} Compared with the aforementioned studies, our paper contributes the following: Theorem \ref{Main} presents the first known result on the geometric ergodicity of the filter process to our knowledge. This result applies to compact state spaces.
Corollary \ref{speccor} establishes a new result on the geometric ergodicity of the filter process for finite state spaces. Theorem \ref{Main2} and 
Corollary \ref{Main2Cor} generalized (and is consistent, for the finite setup, with) the result obtained by 
Kochman and Reeds \cite{kochman2006simple} and utilizes 
the existence of a reachable state of the filter 
process in the context of unique ergodicity. 
A significant aspect of our analysis 
involves the Wasserstein contraction property of non-linear filters, as presented in Theorem \ref{Talpha}.

\section{Proofs}
\subsection{Wasserstein Continuity of the Filter Kernel}
Recall that $\eta$, defined in (\ref{etatra}), is the kernel of the filter update equation. \cite{kara2020near} introduced a 
regularity condition for the controlled filter process. 
\begin{proof}[Proof of Theorem \ref{Talpha}]
We adapt the argument from Kara and Y\"{u}ksel (\cite{kara2020near}, Theorem 7).
We equip \(\mathcal{Z}\) with the metric $W_1$ to define the Lipschitz seminorm $\norm{f}_L$ of any Borel measurable function $f:\mathcal{Z}\to \mathbb{R}$.
\begin{align}
&W_{1}\left(\eta(\cdot \mid z_0), \eta\left(\cdot \mid z_0^{\prime}\right)\right)\nonumber\\
&=\sup _{f\in \operatorname{Lip}(\mathcal{Z})}\left|\int_{\mathcal{Z}} f\left(z_1\right) \eta\left(d z_1 \mid z_0^{\prime}\right)-\int_{\mathcal{Z}} f\left(z_1\right) \eta\left(d z_1 \mid z_0\right)\right| \nonumber\\
& =\sup _{f \in \operatorname{Lip}(\mathcal{Z})}\left|\int_{\mathbb{Y}} f\left(z_1\left(z_0^{\prime}, y_1\right)\right) P\left(d y_1 \mid z_0^{\prime}\right)-\int_{\mathbb{Y}} f\left(z_1\left(z_0,  y_1\right)\right) P\left(d y_1 \mid z_0\right)\right|\label{kappabeta1}\\
& =\sup _{f \in \operatorname{Lip}(\mathcal{Z}),\;\norm{f}_\infty\leq D/2}\left|\int_{\mathbb{Y}} f\left(z_1\left(z_0^{\prime}, y_1\right)\right) P\left(d y_1 \mid z_0^{\prime}\right)-\int_{\mathbb{Y}} f\left(z_1\left(z_0,  y_1\right)\right) P\left(d y_1 \mid z_0\right)\right|, \label{Kappa_beta} 
\end{align}
where $z_1\left(z_0, y\right):=F\left(z_0, y\right)$. Since, for any $f:\mathcal{Z}\to R$ such that $\norm{f}_L\leq 1$, we know that $|f(z)-f(z^{\prime})|\leq d(z,z^{\prime})\leq \operatorname{diam}(\mathcal{Z})\leq D$ 
and subtracting a constant from $f$ does not change the expression (\ref{kappabeta1}), without any loss of generality, we can assume that $\norm{f}_\infty\leq D/2$.

For any $f:\mathcal{Z}\to \mathbb{R}$ such that $\norm{f}_L\leq 1$ and $\norm{f}_\infty\leq D/2$, we have 
\begin{align}
&\left|\int_{\mathbb{Y}} f\left(z_1\left(z_0^{\prime}, y_1\right)\right) P\left(d y_1 \mid z_0^{\prime}\right)-\int_{\mathbb{Y}} f\left(z_1\left(z_0,  y_1\right)\right) P\left(d y_1 \mid z_0\right)\right| \nonumber\\
&\leq \left|\int_{\mathbb{Y}} f\left(z_1\left(z_0^{\prime}, y_1\right)\right) P\left(d y_1 \mid z_0^{\prime}\right)-\int_{\mathbb{Y}} f\left(z_1\left(z_0^{\prime}, y_1\right)\right) P\left(d y_1 \mid z_0\right)\right| \nonumber\\
&+ \int_{\mathbb{Y}}\left|f\left(z_1\left(z_0^{\prime}, y_1\right)\right)-f\left(z_1\left(z_0, y_1\right)\right)\right| P\left(d y_1 \mid z_0\right) \nonumber\\
&\leq\frac{D}{2}\left\|P\left(\cdot \mid z_0^{\prime} \right)-P\left(\cdot \mid z_0\right)\right\|_{T V}\\
&+ \int_{\mathbb{Y}}\left|f\left(z_1\left(z_0^{\prime}, y_1\right)\right)-f\left(z_1\left(z_0, y_1\right)\right)\right| P\left(d y_1 \mid z_0\right).\label{second}
\end{align}
We first analyze the first term:
\begin{align}\label{TV1}
&\left\|P\left(\cdot \mid z_0^{\prime}\right)-P\left(\cdot \mid z_0\right)\right\|_{T V}=\sup _{\|g\|_{\infty} \leq 1}\left|\int g\left(y_1\right) P\left(d y_1 \mid z_0^{\prime}\right)-\int g\left(y_1\right) P\left(d y_1 \mid z_0\right)\right|\nonumber \\
&=\sup _{\|g\|_{\infty} \leq 1}\left|\int g\left(y_1\right) Q\left(d y_1 \mid x_1\right) \mathcal{T}\left(d x_1 \mid z_0^{\prime}\right)-\int g\left(y_1\right) Q\left(d y_1 \mid x_1\right) \mathcal{T}\left(d x_1 \mid z_0\right)\right|\nonumber\\
&\leq (1-\delta(Q))\left\|\mathcal{T}\left(d x_1 \mid z_0^{\prime}\right)-\mathcal{T}\left(d x_1 \mid z_0\right)\right\|_{T V}
\end{align}
by Dobrushin contraction theorem \cite{dobrushin1956central}, where $\T (d x_1 \mid z_0)=\int \T (d x_1 \mid x_0) z_0(dx_0)$

 
\begin{align}\label{TV2}
&\left\|\mathcal{T}\left(d x_1 \mid z_0^{\prime}\right)-\mathcal{T}\left(d x_1 \mid z_0\right)\right\|_{T V}\nonumber\\ 
&= \sup _{\|g\|_{\infty} \leq 1}\left(\int g\left(x_1\right) \T\left(d x_1 \mid z_0^{\prime}\right)-\int g\left(x_1\right) \T\left(d x_1 \mid z_0\right)\right)\nonumber\\
&=\sup _{\|g\|_{\infty} \leq 1}\left(\int\left(\int g\left(x_1\right) \T\left(d x_1 \mid x_0\right)z_0^{\prime}(dx_0)-\int g\left(x_1\right) \T\left(d x_1 \mid x_0\right)z_0(dx_0)\right)\right)\nonumber\\
&= \sup _{\|g\|_{\infty} \leq 1}\left(\int \tilde{g_g}(x_0)z_0^{\prime}(dx_0)- \tilde{g_g}(x_0)z_0(dx_0)\right)
\end{align}
where
\begin{align*}
    \tilde{g_g}(x)= \int g\left(x_1\right) T\left(d x_1 \mid x\right).
\end{align*}
For all $x_0^{\prime}, x_0\in \mathbb{X}$, 
by Assumption \ref{assumption1}-ii, we have, 
\begin{align*}
\|\mathcal{T}(d x_1 \mid x_0^{\prime})-\mathcal{T}(d x_1 \mid x_0)\|_{T V}\leq \alpha d(x_0,x_0^{\prime}).
\end{align*}
As a result, we get that $\tilde{g}/\alpha \in Lip(\mathbb{X})$, since $|\tilde{g}(x_0)-\tilde{g}(x_0^\prime)|\leq\left\|\mathcal{T}\left(d x_1 \mid x_0^{\prime}\right)-\mathcal{T}\left(d x_1 \mid x_0\right)\right\|_{T V}\leq \alpha d(x_0,x_0^{\prime})$.

Then, by inequality (\ref{TV1}) and (\ref{TV2}) we can write 
\begin{align}\label{TV}
\left\|P\left(\cdot \mid z_0^{\prime}\right)-P\left(\cdot \mid z_0\right)\right\|_{T V} \leq \alpha(1-\delta(Q))W_{1}\left(z_0^{\prime}, z_0\right).
\end{align}
Finally, we can analyze the second term in (\ref{second})
\begin{align}\label{eq3}
&\int_{\mathbb{Y}}\left|f\left(z_1\left(z_0^{\prime}, y_1\right)\right)-f\left(z_1\left(z_0, y_1\right)\right)\right| P\left(d y_1 \mid z_0\right)\nonumber\\
&\leq \int_{\mathbb{Y}}W_1(z_1\left(z_0^{\prime}, y_1\right), z_1\left(z_0, y_1\right)) P\left(d y_1 \mid z_0\right)\nonumber\\
&=\int_{\mathbb{Y}} \sup_{g \in \operatorname{Lip}(\mathbb{X})}\left(\int_{\mathbb{X}} g(x_1)z_1\left(z_0^\prime, y_1\right)(dx_1) - \int_{\mathbb{X}} g(x_1)z_1\left(z_0, y_1\right)(dx_1)\right)P\left(d y_1 \mid z_0\right)
\end{align}
If we look at the term inside
\begin{align}\label{inside}
&\sup_{g \in \operatorname{Lip}(\mathbb{X})}\left(\int_{\mathbb{X}} g(x_1)z_1\left(z_0^\prime, y_1\right)(dx_1) - \int_{\mathbb{X}} g(x_1)z_1\left(z_0, y_1\right)(dx_1)\right)\\
&=\sup_{g \in \operatorname{Lip}(\mathbb{X})}\left(\int_{\mathbb{X}} g(x_1)(z_1\left(z_0^\prime, y_1\right)- z_1\left(z_0, y_1\right))(dx_1)\right)\nonumber\\
&=\sup_{g \in \operatorname{Lip}(\mathbb{X})}\left(\int_{\mathbb{X}} g(x_1)w_{y_1}(dx_1)\right),\nonumber
\end{align}
where $w_{y_1}=(z_1\left(z_0^\prime, y_1\right)- z_1\left(z_0, y_1\right))$ 
which is a signed measure on $\mathbb{X}$. 
$Lip(\mathbb{X})$ is closed, uniformly bounded and equicontinuos with respect to the sup-norm topology, 
so by the Arzela-Ascoli theorem $\Lip(\mathbb{X})$ is compact. 
Since a continuous function on a compact set attains its supremum,
 the set
$$
A_y:=\left\{h_y(x)=\arg\sup_{g \in \operatorname{Lip}(\mathbb{X})}\left(\int_{\mathbb{X}} g(x)w_{y}(dx)\right):h_y(x)\in \Lip(\mathbb{X})\right\}
$$
is nonempty for every $y\in \mathbb{Y}$. The integral is continuous under respect to sup-norm, i.e., 
$$ 
\left|\int_{\mathbb{X}} g(x)w_{y}(dx)-\int_{\mathbb{X}} h(x)w_{y}(dx)\right|\leq\norm{g-h}_\infty \quad \forall g,h\in \Lip(\mathbb{X}). 
$$
Then, $A_y$ is a closed set under sup-norm.
$\mathbb{Y}$ and $\operatorname{Lip}(\mathbb{X})$ are Polish spaces, and define $\Gamma=(y,A_y)$. 
$A_y$ is closed for each $y\in \mathbb{Y}$ and $\Gamma$ is Borel measurable. 
So there exists a measurable function $h:\mathbb{Y}\to \operatorname{Lip}(\mathbb{X})$ 
such that $h(y)\in A_y$ for all $y \in \mathbb{Y}$ 
by the Measurable Selection Theorem 
\footnote{[\cite{himmelberg1976optimal}, Theorem 2][Kuratowski Ryll-Nardzewski Measurable Selection Theorem]
Let $\mathbb{X}, \mathbb{Y}$ be Polish spaces and $\Gamma=(x, \psi(x))$ where $\psi(x) \subset \mathbb{Y}$ be such that, $\psi(x)$ is closed for each $x \in \mathbb{X}$ and let $\Gamma$ be a Borel measurable set in $\mathbb{X} \times \mathbb{Y}$. Then, there exists at least one measurable function $f: \mathbb{X} \rightarrow \mathbb{Y}$ such that $\{(x, f(x)), x \in \mathbb{X}\} \subset \Gamma$.
}. 
Let us define $g_y=h(y)$.
After that we can continue with the equation (\ref{eq3}),
\begin{align}
&\int_{\mathbb{Y}} \sup_{g \in \operatorname{Lip}(\mathbb{X})}\left(\int_{\mathbb{X}} g(x_1)z_1\left(z_0^\prime, y_1\right)(dx_1) - \int_{\mathbb{X}} g(x_1)z_1\left(z_0, y_1\right)(dx_1)\right)P\left(d y_1 \mid z_0\right)\nonumber\\
&=\int_{\mathbb{Y}} \left(\int_{\mathbb{X}} g_{y_1}(x_1)z_1\left(z_0^\prime, y_1\right)(dx_1) - \int_{\mathbb{X}} g_{y_1}(x_1)z_1\left(z_0, y_1\right)(dx_1)\right) P\left(d y_1 \mid z_0\right)\label{inside2}\\
&=\int_{\mathbb{Y}} \int_{\mathbb{X}} g_{y_1}(x_1)z_1\left(z_0^\prime, y_1\right)(dx_1) P(d y_1 \mid z_0)-
\int_{\mathbb{Y}}\int_{\mathbb{X}} g_{y_1}(x_1)z_1\left(z_0, y_1\right)(dx_1) P(d y_1 \mid z_0)\nonumber\\
&=\int_{\mathbb{Y}} \int_{\mathbb{X}} g_{y_1}(x_1)z_1\left(z_0^\prime, y_1\right)(dx_1) P(d y_1 \mid z_0)-
\int_{\mathbb{Y}}\int_{\mathbb{X}} g_{y_1}(x_1)z_1\left(z_0^\prime, y_1\right)(dx_1) P(d y_1 \mid z_0^\prime)\label{gy1}\\
&+\int_{\mathbb{Y}} \int_{\mathbb{X}} g_{y_1}(x_1)z_1\left(z_0^\prime, y_1\right)(dx_1) P(d y_1 \mid z_0^\prime)-
\int_{\mathbb{Y}}\int_{\mathbb{X}} g_{y_1}(x_1)z_1\left(z_0, y_1\right)(dx_1) P(d y_1 \mid z_0)\label{gy2}
\end{align}

For the first term, we can write by the same argument as earlier
\begin{align*}
\norm{\int_{\mathbb{X}} g_{y_1}(x_1)z_1\left(z_0^\prime, y_1\right)(dx_1)}_\infty\leq\norm{g_{y_1}}_\infty \leq D/2 .
\end{align*}
So,
\begin{align}\label{sup-1}
&\int_{\mathbb{Y}} \int_{\mathbb{X}} g_{y_1}(x_1)z_1\left(z_0^\prime, y_1\right)(dx_1) P(d y_1 \mid z_0)-
\int_{\mathbb{Y}}\int_{\mathbb{X}} g_{y_1}(x_1)z_1\left(z_0^\prime, y_1\right)(dx_1) P(d y_1 \mid z_0^\prime)\nonumber\\
&\leq \frac{D}{2} \left\|P\left(\cdot \mid z_0^{\prime}\right)-P\left(\cdot \mid z_0\right)\right\|_{T V}\nonumber\\
& \leq \alpha \frac{D}{2} (1-\delta(Q))W_{1}\left(z_0^{\prime}, z_0\right)
\end{align}
by inequality (\ref{TV}).

For the second term,  we can write by smoothing
\begin{align}\label{sup-2}
&\int_{\mathbb{Y}} \int_{\mathbb{X}} g_{y_1}(x_1)z_1\left(z_0^\prime, y_1\right)(dx_1) P(d y_1 \mid z_0^\prime)-
\int_{\mathbb{Y}}\int_{\mathbb{X}} g_{y_1}(x_1)z_1\left(z_0, y_1\right)(dx_1) P(d y_1 \mid z_0)\nonumber\\
&=\int_{\mathbb{Y}} \int_{\mathbb{X}} g_{y_1}(x_1)Q\left(d y_1 \mid x_1\right) \mathcal{T}\left(d x_1 \mid z_0^{\prime}\right)-\int_{\mathbb{Y}} \int_{\mathbb{X}} g_{y_1}(x_1) Q\left(d y_1 \mid x_1\right) \mathcal{T}\left(d x_1 \mid z_0\right)\nonumber\\
&=\int_{\mathbb{X}} \int_{\mathbb{Y}} g_{y_1}(x_1)Q\left(d y_1 \mid x_1\right) \mathcal{T}\left(d x_1 \mid z_0^{\prime}\right)-\int_{\mathbb{X}} \int_{\mathbb{Y}} g_{y_1}(x_1) Q\left(d y_1 \mid x_1\right) \mathcal{T}\left(d x_1 \mid z_0\right)\nonumber\\
&=\int_{\mathbb{X}} \omega(x_1) \mathcal{T}\left(d x_1 \mid z_0^{\prime}\right)-\int_{\mathbb{X}}\omega(x_1) \mathcal{T}\left(d x_1 \mid z_0\right)
\end{align}
where $$\omega(x_1)=\int_{\mathbb{Y}} g_{y_1}(x_1)Q\left(d y_1 \mid x_1\right).$$
The first equality is a consequence of (\ref{etatra}), and the second follows from Fubini's theorem, since both integrals are bounded by $1$.

For any $x^\prime, x^{\prime\prime} \in \mathbb{X}$, 
\begin{align*}
&\int_{\mathbb{X}} \omega(x) \mathcal{T}\left(d x \mid x^{\prime\prime}\right)-\int_{\mathbb{X}}\omega(x) \mathcal{T}\left(d x \mid x^{\prime}\right)\\
&\leq \norm{\omega}_\infty \left\|\mathcal{T}(\cdot \mid x^{\prime\prime})-\mathcal{T}\left(\cdot \mid x^{\prime}\right)\right\|_{T V}\\
&\leq \norm{\omega}_\infty\alpha d(x^{\prime\prime},x^{\prime})\\
&\leq\alpha \frac{D}{2} d(x^{\prime\prime}, x^{\prime})
\end{align*}
So, by definition of the $W_1$ norm (\ref{defkappanorm}), we have
\begin{align}\label{sup-2-1}
&\int_{\mathbb{X}} \omega(x_1) \mathcal{T}\left(d x_1 \mid z_0^{\prime}\right)-\int_{\mathbb{X}}\omega(x_1) \mathcal{T}\left(d x_1 \mid z_0\right)\nonumber\\
&=\int_{\mathbb{X}}\int_{\mathbb{X}} \omega(x_1) \mathcal{T}\left(d x_1 \mid x_0\right)z^\prime_0(dx_0)-\int_{\mathbb{X}} \int_{\mathbb{X}}\omega(x_1) \mathcal{T}\left(d x_1 \mid x_0\right)z_0(dx_0)\nonumber\\
&\leq  \alpha \frac{D}{2} W_{1}\left(z_0, z_0^{\prime}\right).
\end{align}
So, by the inequalities (\ref{gy1}), (\ref{gy2}), (\ref{sup-1}), (\ref{sup-2}), (\ref{sup-2-1}) we get
\begin{align}\label{last2}
& \int_{\mathbb{Y}}\left|f\left(z_1\left(z_0^{\prime}, y_1\right)\right)-f\left(z_1\left(z_0, y_1\right)\right)\right| P\left(d y_1 \mid z_0\right) \nonumber\\
&\leq \alpha \frac{D}{2} (2-\delta(Q))W_{1}\left(z_0^{\prime}, z_0\right).
\end{align}
If we take the supremum of the equation (\ref{Kappa_beta}) over all $f\in \Lip(\mathcal{Z})$, then by using the inequalities (\ref{second}),(\ref{TV}), and (\ref{last2}), we can write:
\begin{align}\label{imp}
&W_{1}\left(\eta(\cdot \mid z_0), \eta\left(\cdot \mid z_0^{\prime}\right)\right)
\leq \left(\frac{\alpha D (3-2\delta(Q))}{2}\right)W_{1}\left(z_0, z_0^{\prime}\right)\end{align}
\end{proof}

\subsection{Geometric Wasserstein Ergodicity}

Now we proceed to prove our theorem on geometric ergodicity. 
We note again that Assumption \ref{assumption1}-ii implies, 
by \cite{KSYWeakFellerSysCont}, 
the weak Feller property of the non-linear 
filtering kernel $\eta$ defined in (\ref{etatra}). 
Therefore, since the state space of filter process 
is compact, 
a Krylov-Bogoliubov argument implies 
that there exists at least one invariant probability measure.

\begin{proof}[Proof of Theorem \ref{Main} ]
Let $\beta=\alpha D (3-2\delta(Q))/2$. 
Under the Assumption \ref{assumption1}(iii), $\beta<1$.

Let $f\in \Lip(\mathcal{Z})$. For any $z^*_0, z_0 \in \mathcal{Z}$,
\begin{align}\label{eq2}
&\left|\int_\mathcal{Z} \eta^n(d z_n\mid z_0) f(z_n)-\int_\mathcal{Z} \eta^n(d z_n\mid z^*_0) f(z_n)\right|\nonumber\\
&=\left|\int_\mathcal{Z}\int_\mathcal{Z} \eta^{n-1}(d z_{n-1}\mid z_0)\eta(d z_n\mid z_{n-1}) f(z_n)-\int_\mathcal{Z}\int_\mathcal{Z} \eta^{n-1}(d z_{n-1}\mid z^*_0)\eta(d z_n\mid z_{n-1}) f(z_n)\right| \nonumber\\
&=\left|\int_\mathcal{Z} \eta^{n-1}(d z_{n-1}\mid z_0)\left(\int_\mathcal{Z}\eta(d z_n\mid z_{n-1}) f(z_n)\right)-\int_\mathcal{Z} \eta^{n-1}(d z_{n-1}\mid z^*_0)\left(\int_\mathcal{Z}\eta(d z_n\mid z_{n-1}) f(z_n)\right)\right| \nonumber\\
&= \left|\int_\mathcal{Z} \eta^{n-1}(d z_{n-1}\mid z_0) g(z_{n-1})-\int_\mathcal{Z} \eta^{n-1}(d z_{n-1}\mid z^*_0) g(z_{n-1})\right|
\end{align}
where
$$g(z)= \int_\mathcal{Z}\eta(d z_1\mid z) f(z_1).$$

For any $z, \bar{z} \in \mathcal{Z}$,
\[\left|g(z)-g(\bar{z})\right|\leq \beta W_{1}\left(z, \bar{z}^{\prime}\right)\]
by Theorem \ref{Talpha}. As a result, $g/\beta \in \Lip(\mathcal{Z})$.

Taking the supremum over $f \in \Lip(\mathcal{Z})$ in inequality \eqref{eq2}, we obtain the following: 
\begin{align}\label{eqrec}
    &\sup_{f\in \Lip(\mathcal{Z})}\left(\int_\mathcal{Z} \eta^n(d z_n\mid z_0) f(z_n)-\int_\mathcal{Z} \eta^n(d z_n\mid z^*_0) f(z_n)\right)\nonumber\\
    &\leq \beta \sup_{f\in \Lip(\mathcal{Z})}\left(\int_\mathcal{Z} \eta^{n-1}(d z_{n-1}\mid z_0) f(z_{n-1})-\int_\mathcal{Z} \eta^{n-1}(d z_{n-1}\mid z^*_0) f(z_{n-1})\right)
\end{align}
Then, by induction
\begin{align}\label{Lipalpha}
    W_{1}\left(\eta^n(\cdot \mid z_0), \eta^n\left(\cdot \mid z_0^*\right)\right)
    \leq \beta^{n} W_{1}\left(z_0, z_0^*\right)
\end{align}
There exists an invariant probability measure 
by Krylov-Bogoliubov 
argument (\cite{Hernandez-Lerma2003}, Theorem 7.2.3) 
and the inequality implies that the 
filter process is uniquely ergodic.
Furthermore, for any 
$\mu,\nu \in \mathcal{P}(\mathcal{Z})$, we have
\begin{align}\label{Lipalpha2}
    W_{1}\left(\eta^n(\cdot \mid \mu), \eta^n\left(\cdot \mid \nu\right)\right)
    \leq \beta^{n} \sup_{z_0, z_0^* \in \mathcal{Z}}W_{1}\left(z_0, z_0^*\right)
    \leq \beta^{n} D,
\end{align}
which shows that the filter process is geometrically Wasserstein ergodic.
\end{proof}

\subsection{Unique Ergodicity}
\begin{lemma}\label{Lip} Under the Assumption \ref{assumption1}-ii, 
    for each function $f\in BL_1(\mathcal{Z})$ with compact 
    support, the sequence of functions 
    $\left\{\int_\mathcal{Z} \eta^n(d z_n\mid z) f(z_n), n \in \mathcal{Z}_{+}\right\}$ 
    is equi-continuous on compact sets under the bounded-Lipschitz norm.
\end{lemma}
\begin{proof}
    The proof is a direct conclusion of Lemma \ref{Regularity}.
\end{proof}

We define the $L$-chain properly as follows:
\begin{definition}
A Markov chain with transition kernel $P$ is called an $L$-chain if for each Lipschitz continuous function with compact support, the sequence of functions $\{\int P^n(d y\mid z) f(y), n \in \mathcal{Z}_{+}\}$ are equi-continuous.
\end{definition}
\begin{theorem}\label{L-chain} Under Assumption \ref{assumption1}-i,ii 
    the filter process with transition kernel $\eta$ is an L-chain. 
\end{theorem}
\begin{proof} Let $f$ be a Lipschitz continuous function with compact support. 
Then, $\left\{\int \eta^n(d y\mid z) f(y), n \in \mathcal{Z}_{+}\right\}$ is equi-continuous, because $f/(\norm{f}_L+\norm{f}_\infty) \in BL_1(Z)$ and by Lemma \ref{Lip}.
\end{proof}

An $L$-chain, by definition, satisfies the weak Feller property. 
Therefore, proving that a chain is an $L$-chain is sufficient 
to show that it also satisfies the weak Feller property. 
Hence, in the theorems that we will present later, 
it is not necessary to explicitly mention that the chain 
satisfies the weak Feller property. Additionally, 
under Assumption \ref{ErgodicityHMM}-ii, 
where the transition kernel of the hidden Markov source is continuous in total variation, 
we know that the weak Feller property holds \cite{KSYWeakFellerSysCont}.

\begin{definition}
Let $\pi$ be a probability measure on $\mathbb{X}$ with metric $d$. The topological support of $\pi$ is defined with
$$
\operatorname{supp} \pi:=\left\{x: \pi\left(B_r(x)\right)>0\right\}, \quad \forall r>0,
$$
where $B_r(x)=\{y \in \mathbb{X}: d(x, y)<r\}$.
\end{definition}

\begin{theorem}[\cite{Hernandez-Lerma2003}, Theorem 7.2.3] \label{compactness}
    Let $\left\{x_t\right\}$ be a weak Feller Markov
    process taking values in a compact subset of a complete,
    separable metric space. Then $\left\{x_t\right\}$
    admits an invariant probability measure.
\end{theorem}

\begin{theorem}\label{Essential}
     Let a weak Feller Markov chain be an L-chain, the state space $\mathcal{Z}$ be compact, 
     and a topologically reachable state $z^*$ exist. Then, (i) there exists a unique invariant probability measure, and (ii)
     if the reachable state $z^*$ is topologically aperiodic, the Markov process converges weakly to the unique invariant
     probability measure for every initial prior.
\end{theorem}

\begin{proof} By Theorem \ref{compactness} there is an invariant probability measure. Suppose that there were two different probability measures $\nu_1, \nu_2$. We may assume $\nu_1$ and $\nu_2$ to be ergodic, via an ergodic decomposition argument of invariant measures [\cite{Tweedie-94}, Theorem 6.1].

Since $z^*$ is reachable state,  we have that $B_r\left(z^*\right)$ is visited under either of the probability measures in finite time for any given $r>0$. So, $z^*$ must be belong to the support of $\nu_1$ and $\nu_2$.
Then, we have two sequences $x_n, y_n$ which converge to $z^*$ where $x_n, y_n$ belong to 
the topological support sets of $\nu_1, \nu_2$, respectively. 

Now, by equi-continuity and the Arzela-Ascoli theorem, for every bounded and Lipschitz continuous function $f$ we have that
\begin{align}\label{subseq}
\eta^{(N)}(f)(z):=\frac{1}{N} \sum_{k=0}^{N-1} \int \eta^k(z, d y) f(y)
\end{align}
has a subsequence which converges (in the sup norm) to a limit $F_f^*: \mathcal{Z} \rightarrow \mathbb{R}$, where is $F_f^*$ continuous.
The above imply that, every bounded and Lipschitz continuous function $f$, the term
$$
\lim _{n \rightarrow \infty}\left|\lim _{N \rightarrow \infty} \eta^{(N)}(f)\left(y_n\right)-\eta^{(N)}(f)\left(x_n\right)\right|=0 .
$$
Suppose not; there would be an $\epsilon>0$ and a subsequence $n_k$ for which the difference
$$
\left|\lim _{N \rightarrow \infty} \eta^{(N)}(f)\left(y_{n_k}\right)-\eta^{(N)}(f)\left(x_{n_k}\right)\right|>\epsilon .
$$
However, for each fixed $n_k$, we have that
$$
\lim _{N \rightarrow \infty} \eta^{(N)}(f)\left(y_{n_k}\right)
$$
converges by the ergodicity of $\nu_1$ to 
$\left\langle\nu_1, f\right\rangle:=\int_{\mathcal{Z}}f(z)\nu_1(dz)$ 
and the limit, by the Arzela-Ascoli theorem, will be equal to $F_f^*\left(y_{n_k}\right)$ (as every converging subsequence would have to converge to the limit; which also implies that the subsequential convergence in (\ref{subseq}) is in fact a sequential convergence). The same argument applies for $\eta^{(N)}(f)\left(x_{n_k}\right) \rightarrow\left\langle\nu_2, f\right\rangle=F_f^*\left(x_{n_k}\right)$.

The above would then imply that $\left|F_f^*\left(y_{n_k}\right)-F_f^*\left(x_{n_k}\right)\right| \geq \epsilon$ for every $\left(y_{n_k}, x_{n_k}\right)$. 
This would be a contradiction due to the continuity of $F_f^*$. 
Therefore, the time averages of $f$ under $\nu_1$ and $\nu_2$ will be arbitrarily close to each other. 
Then, for every bounded and Lipschitz continuous function $\left\langle\nu_1, f\right\rangle-\left\langle\nu_2, f\right\rangle=0$, therefore $\rho_{BL}(\nu_1, \nu_2)=0$. 
Since $\rho_{BL}$ is a metric on $P(Z)$ (or that bounded and Lipschitz continuous functions form a separating class), this implies that the probability measures $\nu_1$ and $\nu_2$ must be equal.

This proves that there exist a unique
 invariant probability measure.
 Let this invariant measure be $\nu$. 

(ii) Now let us prove the second part of the theorem. 
 That is, if $z^*$ is an topological aperiodic state, 
 then for any initial state, 
 the Markov process weakly converges to $\nu$. 
 For this, we will show that the approach given in 
 \cite[Theorem 18.4.4-ii]{MeynBook} for $e-$chains
is also valid for $L-$chains. 

First, let us take any bounded and 
continuous Lipschitz function $f$. 
Without loss of generality, let us accept 
$\int_\mathcal{Z}f(z)\nu(dz)=0$ as we can add a constant to $f$.

Due to the Arzela-Ascoli theorem and equicontinuity,
\begin{align}\label{seq}
\eta^{N}(f)(z):=\int \eta^N(z, d y) f(y)
\end{align}
has a subsequence which converges 
uniformly (in the sup norm) to a 
limit $H_f^*: \mathcal{Z} \rightarrow \mathbb{R}$, 
where $H_f^*$ is continuous. Therefore, 
for a certain series $k_i$, the series 
$\eta^{k_i}(f)$ uniformly converges to $H_f^*$. For every $k\in\mathbb{N}$,
\begin{align*}
&\langle \nu,|\eta^k(f)|\rangle=
\int_\mathcal{Z} |\eta^k(f)|(z)\nu(dz)=
\int_\mathcal{Z} \eta(|\eta^k(f)|)(z)\nu(dz)\\
&=\int_\mathcal{Z}\int_\mathcal{Z} |\eta^k(f)|(y)\eta(z,dy)\nu(dz)=
\int_\mathcal{Z}\int_\mathcal{Z} |\eta^{k}(f)(y)\eta(z,dy)|\nu(dz)\\
&\geq \int_\mathcal{Z}|\eta^{k+1}(f)(z)|\nu(dz)=
\langle \nu,|\eta^{k+1}(f)|\rangle
\end{align*}
the last inequality follows from Jensen inequality.

From the Monotone Convergence Theorem, 
the sequence $\langle \nu,|\eta^k(f)|\rangle$ has a real limit,
denoted as $v$. 
Since the sequence $\eta^{k_i}(f)$ converges to $H_f^*$ for some 
series $k_i$, and the sequence $\eta^{k_i+m}(f)$ converges 
to $\eta^m(H_f^*)$, we have 
\begin{align}\label{H_f_abs}
\langle \nu,|\eta^m(H_f^*)|\rangle=
\langle \nu,|H_f^*|\rangle=v \quad \forall m \in \mathbb{N}.
\end{align}

First, using aperiodicity, we will prove that
$H_f^*$ takes the value $0$ for every element 
in the topological support of $\nu$. 
We will then show that $\eta^N(f)$ converges to $0$ 
for every element 
in the topological support of $\nu$.
Following this, we will prove that $H_f^*$
is $0$ for every element. 
Finally, we will prove that $\eta^N(f)$ 
converges to $0$.

Consider any open neighborhood around $z^*$, 
let us call it $O$. 
The lower semi-continuity of $\eta^k(.,O)$ 
follows from the weak Feller property of 
$\eta$ and the Portmanteau theorem. 
Consequently, for sufficiently large $k$, 
an open neighborhood $\bar{O}$ 
exists around $z^*$ such that 
$\eta^k(z,O)>0$ for every $z\in\bar{O}$.
 Since $z^*$ is reachable, the Markov chain starting from 
 $\nu$ visits $\bar{O}$ within a finite time
 from almost every initial state. 
 This proves that $\nu(O)>0$; note that (as already known) this implies that $z^*$ is in the topological support of $\nu$.

 Now, let us show that $H_f^*(z)=0$ 
 for all $z$ in the topological support of $\nu$. 
 Otherwise, the sets $O^+=\{z:H_f^*(z)>0\}$ and $O^-=\{z:H_f^*(z)<0\}$ 
 would both have a positive measure in $\nu$ 
 since $\langle\nu, H_f^*\rangle
 =\left\langle\nu, f\right\rangle=0$. 
 Since the function $H_f^*$ is continuous, 
 both sets are open. Due to ergodicity, 
 and similar to above arguments, 
 we know that for any bounded continuous 
 function $h$, $F_h^*(z^*)$ is equal to 
 $\left\langle\nu, h\right\rangle$. 
 Since $\nu(O^+)>0$ and $O^+$ is open, 
 we can define the function $h$ to be $0$ outside $O^+$ 
 and $\left\langle\nu, h\right\rangle>0$. 
 In this case, $F_h^*(z^*)>0$ 
 which proves that there exists a 
 $k_1\in \mathbb{N}$ such that 
 $\eta^{k_1}(z^*,O^+)>0$. Similarly, 
 there exists a $k_2\in \mathbb{N}$ 
 such that $\eta^{k_2}(z^*,O^-)>0$. 
 As $\eta^{k}(.,O^+)$ is lower semi-continuous, 
 there exists an open neighborhood 
 $O^\prime$ around $z^*$ such that for 
 every $z\in O^\prime$, $\eta^{k_1}(z,O^+)>0$ 
 and $\eta^{k_2}(z,O^-)>0$. 
 As $z^*$ is topologically aperiodic, 
 for all sufficiently large $k$, $\eta^{k}(z^*,O^\prime)>0$. 
 Therefore, there exists an $l\in \mathbb{N}$ 
 such that $\eta^{l}(z^*,O^+)>0$ and $\eta^{l}(z^*,O^-)>0$. 
 Again, since $\eta^{l}(.,O^+)$ and $\eta^{l}(.,O^-)$ 
 are lower semi-continuous, 
 there exists an open neighborhood $N$ around 
 $z^*$ such that for every $z\in N$, $\eta^{l}(z,O^+)>0$ 
 and $\eta^{l}(z,O^-)>0$. 
 Therefore, for every $z \in N$, 
 $|\eta^l(H_f^*)(z)|<\eta^l(|H_f^*|)(z)$. 
 Since $z^*$ is in the topological support, 
 we know that $\nu(N)>0$. From here, we reach the conclusion that
 \begin{align}\label{ineqH}
 \langle \nu,|\eta^l(H_f^*)|\rangle <
 \langle\nu,\eta^l(|H_f^*|)\rangle =
 \langle\nu, |H_f^*|\rangle
 \end{align}
 where the equality comes from $\nu$ being stationary. 
 This contradicts the equality (\ref{H_f_abs}).

This proves that $H_f^*(z)=0$ for every $z$ 
in the topological support of $\nu$. Therefore, the series 
$\eta^{k_i}(f)$ uniformly converges to $0$
in the topological support of $\nu$. So, for any $\epsilon>0$, 
there exists a $k\in \mathbb{N}$ such that for every 
$z$ in the topological support of $\nu$, $|\eta^{k}(f)(z)|<\epsilon$ holds. 
The set
\begin{align*}
O^k_\epsilon=\{ z\in \mathcal{Z}: |\eta^k(f)(z)|<\epsilon \}
\end{align*}
is an open set and contains all elements in the topological 
support of $\nu$. Therefore, $\nu(O^k_\epsilon)=1$.
For any $m \in \mathbb{N}$ and for any $z$ in the topological support of $\nu$,
\begin{align*}
&|\eta^{k+m}(f)(z)|=
\bigg|\int_\mathcal{Z}\eta^m(z,dy)\eta^k(f)(y) \bigg|=\\
&\bigg|\int_{O^k_\epsilon}\eta^m(z,dy)\eta^k(f)(y) + 
\int_{\mathcal{Z} \setminus O^k_\epsilon}\eta^m(z,dy)\eta^k(f)(y)\bigg|=\\
&\bigg|\int_{O^k_\epsilon}\eta^m(z,dy)\eta^k(f)(y)\bigg| \leq\epsilon.
\end{align*}
This final equality arises because
$z$ is within the topological support of $\nu$ and 
$\nu$ is an invariant probability measure,
leading to $\eta^m(z,\mathcal{Z}\setminus O^k_\epsilon)=0$.
This shows that $\eta^N(f)$ converges to $0$ 
for every element 
in the topological support of $\nu$.

Now, let us prove this for every $z\in \mathcal{Z}$.
The sequence $\bar{\eta}^N(z,.):=\frac{1}{N}\sum_{i=1}^{N} \eta^i(z,.)$
converges to $\nu$, because since $\mathcal{Z}$ is compact, 
the sequence has a limit, and the limit is a stationary probability measure; by uniqueness, it is $\nu$.

For any $\epsilon>0$, the set
\begin{align*}
O_\epsilon=\{ z\in \mathcal{Z}: \limsup_{k \to \infty}
|\eta^k(f)|<\epsilon \}
\end{align*}
is an open set and contains all elements in the topological 
support of $\nu$. 
Therefore, $\nu(O_\epsilon)=1$, 
meaning for any $z\in\mathcal{Z}$, 
we have $\lim_{N \to \infty}\bar{\eta}^N(z,O_\epsilon)=1$. 
From this, for a sufficiently large 
$M\in \mathbb{N}$, $\eta^M(z,O_\epsilon)>1-\epsilon/
\norm{f}_\infty$ holds.

As a result,
\begin{align*}
&|H_f^*(z)|\leq \limsup_{k \to \infty} |\eta^{k+M}(f)(z)|\\
&\leq \norm{f}_\infty \eta^M(z,O_\epsilon^C)+
\limsup_{k \to \infty} \int_{O_\epsilon}
\eta^M(z,dy) |\eta^k(f)(y)|\leq 2\epsilon
\end{align*}

Thus, for every $z\in \mathcal{Z}$, $H_f^*(z)=0$. 
The convergence is uniform,
so the series 
$\eta^{k_i}(f)$ uniformly converges to $0$.
For any $\epsilon>0$, 
there exists a $k\in \mathbb{N}$ such that for every 
$z\in \mathcal{Z}$, $|\eta^{k}(f)(z)|<\epsilon$ holds. 
For any $m \in \mathbb{N}$,
\begin{align*}
|\eta^{k+m}(f)(z)|=
\bigg|\int_\mathcal{Z}\eta^m(z,dy)\eta^k(f)(y) \bigg|\leq\epsilon
\end{align*}

Therefore, $\eta^{N}(f)(z)$ uniformly converges to 
$\langle\nu, f\rangle=0$.
This is valid for all bounded and Lipschitz continuous 
functions and these functions are dense in the set of 
bounded continuous functions.  Consequently, 
for every $z\in \mathcal{Z}$, $\eta^N(z,.)$ weakly converges to $\nu$.
\end{proof}

\begin{proof}[Proof of Theorem \ref{Main2}]    
    Under Assumptions \ref{ErgodicityHMM}-i and ii, 
    we have proved that the filter process is an $L$-chain 
    (Theorem \ref{L-chain}). We also know that 
    $\mathcal{Z}$ is compact. 
    Furthermore, under Assumptions \ref{ErgodicityHMM}-iii,
    there is a topologically 
    reachable state for the filter process. 
    We complete the proof using Theorem \ref{Essential}.
\end{proof}

\section{Existence of a reachable state for the filter process}
\label{existence_reachable}
This section introduces assumptions guaranteeing 
the existence of a reachable state for the filter process.

First, we present a condition via mixing properties. 

\begin{definition}
    Two non-negative measures $\mu, \mu^{\prime}$ on $(\mathbb{X},\B(\mathbb{X}))$ are comparable, if there exist positive constants $0<a \leq b$, such that
    $$
    a \mu^{\prime}(A) \leq \mu(A) \leq b \mu^{\prime}(A)
    $$
    for any Borel subset $A \subset \mathbb{X}$.
\end{definition}

\begin{definition}[Mixing kernel]
    The non-negative kernel $K$ defined on $\mathbb{X}$ is mixing, if there exists a constant $0<\varepsilon \leq 1$, and a non-negative measure $\lambda$ on $\mathbb{X}$, such that
    $$
    \varepsilon \lambda(A) \leq K(x, A) \leq \frac{1}{\varepsilon} \lambda(A)
    $$
    for any $x \in \mathbb{X}$, and any Borel subset $A \subset \mathbb{X}$.
    \end{definition}


\begin{definition}(Hilbert metric).  Let $\mu, \nu$ be two non-negative finite measures. We define the Hilbert metric on such measures as
    \begin{equation}
    h(\mu, \nu)= \begin{cases}\log \left(\frac{\sup _{A \mid \nu(A)>0} \frac{\mu(A)}{\nu(A)}}{\inf _{A \mid \nu(A)>0} \frac{\mu(A)}{\nu(A)}}\right) & \text { if } \mu, \nu \text { are comparable } \\ 0 & \text { if } \mu=\nu=0 \\ \infty & \text { else }\end{cases}
    \end{equation}
\end{definition}

Note that $h(a\mu, b\nu) = h(\mu, \nu)$ for any positive scalars $a, b$. Therefore, the Hilbert metric is a useful metric for nonlinear filters since it is invariant under normalization, and the following lemma demonstrates that it bounds the total-variation distance.

\begin{lemma}[\cite{le2004stability}, Lemma 3.4.]\label{h-TV}
    Let $\mu, \mu^{\prime}$ be two non-negative finite measures,
    \begin{enumerate}
    \item[i.] $
    \left\|\mu-\mu^{\prime}\right\|_{TV} \leq \frac{2}{\log 3} h\left(\mu, \mu^{\prime}\right) .
    $
    \item[ii.] 
    If the nonnegative kernel $K$ is a mixing kernel with constant $\epsilon$, then 
    $
    h\left(K \mu, K \mu^{\prime}\right) \leq \frac{1}{\varepsilon^2}\left\|\mu-\mu^{\prime}\right\|_{TV}.
    $
    \end{enumerate}
\end{lemma}

\begin{assumption} \label{mixing_kernel_con} 
    \begin{enumerate}
    \item[i.] $\mathbb{Y}$ is countable and 
    there exists an observation realization 
    $y' \in \mathbb{Y}$ such that for some 
    positive $\epsilon$, $\Q(y'|x)>\epsilon$ 
    for every $x\in \mathbb{X}$. 
    \item[ii.] The transition kernel ${\T}$ is a mixing kernel.
   \end{enumerate}
\end{assumption}

\begin{lemma}[\cite{le2004stability}, Lemma 3.8]\label{Birkoff} 
    (Birkhoff contraction coefficient). 
    The nonnegative linear operator on $\mathcal{M}^{+}(\mathbb{X})$ (positive measures on $\mathbb{X}$) 
    associated with a nonnegative kernel $K$ defined on $\mathbb{X}$, is a contraction under the Hilbert metric, and
    $$
    \tau(K):=\sup _{0<h\left(\mu, \mu^{\prime}\right)<\infty} \frac{h\left(K \mu, K \mu^{\prime}\right)}{h\left(\mu, \mu^{\prime}\right)}=\tanh \left[\frac{1}{4} H(K)\right]
    $$
    where
    $$
    H(K):=\sup _{\mu, \mu^{\prime}} h\left(K \mu, K \mu^{\prime}\right)
    $$
    is over nonnegative measures. $\tau(K)$ is called the Birkhoff contraction coefficient.
    Notice that $H(K)<\infty$ implies $\tau(K)<1$.
\end{lemma}

Recall from equation (\ref{F_def}) that $
F(z, y)(\cdot)=\operatorname{Pr}\left\{X_{n+1} \in \cdot \mid Z_n=z, Y_{n+1}=y\right\}
$.
        \begin{lemma}\label{clm}
    Under Assumption \ref{mixing_kernel_con}, 
    there exists a constant $c<1$  such that 
    \begin{align}
        h(F(\mu, y'), F(\nu, y'))\leq c h(\mu, \nu)
    \end{align}
    for every comparable $\mu,\nu\in \Z$.
    \end{lemma}
    \begin{proof}
    For the fixed observation $y'$ 
    given in Assumption \ref{h-TV},
    define the nonnegative kernel 
    $K: \mathcal{M}^{+}(\mathbb{X}) \to \mathcal{M}^{+}(\mathbb{X})$ 
    by
    $$
    K \rho\left(d x^{\prime}\right)=\int_\mathbb{X}\rho(dx) Q(y'|x')T(dx'|x),
    $$
    for any nonnegative measure $\rho \in \mathcal{M}^{+}(\mathbb{X})$.
    Define $$N^\mu_{y'}:=\int_X\int_XQ(y'|\Bar{x})T(d\Bar{x}|x)\mu(dx).$$
    $N^\mu_{y'}, N^\nu_{y'}>0$ because of Assumption \ref{mixing_kernel_con}. 
    
    \begin{align*}
    &P^{\mu}\left(X_1\in A \mid Y_{1}=y'\right)=\frac{\int_X\int_A Q(y'|x')T(dx'|x)\mu(dx)}{N^\mu_{y'}}=\frac{(K\mu)(A)}{N^\mu_{y'}}.
    \end{align*}
    Therefore,
    \begin{align}
        h(F(\mu, y'), F(\nu, y'))=h(P^{\mu}\left(X_1\in \cdot \mid Y_{1}=y'\right),P^{\nu}\left(X_1\in \cdot \mid Y_{1}=y'\right))=h(K\mu, K\nu)
    \end{align}
    By Lemma \ref{Birkoff}
    $$
    \tau(K)=\sup _{0<h\left(\mu, \nu\right)<\infty} \frac{h\left(K \mu, K \nu\right)}{h\left(\mu, \nu\right)}<1
    $$
    only if $H(K)=\sup _{\mu, \mu^{\prime}} h\left(K \mu, K\mu^{\prime}\right)<\infty$. 
    Given that $Q(y'|x) > \epsilon$ for every $x \in \mathbb{X}$, it follows that
    \begin{align}\label{K_T}
    \int_{\X} {\T}(A|x)\mu(dx)\geq (K\mu)(A) \geq \epsilon \int_{\X}  {\T}(A|x)\mu(dx)
    \end{align}
    for every $A\in \B(\mathbb{X})$.
    
    Let $\epsilon_{\T}$ be constant of the
    mixing kernel $\T$, i.e., 
    there exists a non-negative measure $\lambda$ on $\mathbb{X}$, such that
    $$
    \epsilon_{\T} \lambda(A) \leq T(A|x) \leq \frac{1}{\epsilon_{\T}} \lambda(A)
    $$
    for any $x \in \mathbb{X}$, and any Borel subset $A \subset \mathbb{X}$.
    Thus, from equation (\ref{K_T}), we obtain:
    \begin{align}
        \frac{\mu(\X)}{\epsilon_{\T}} \lambda(A) \geq (K\mu)(A) \geq \epsilon \mu(\X) \epsilon_{\T}  \lambda(A)
    \end{align}
    for every $A\in \B(\mathbb{X})$.

    Hence, for any two finite positive measures 
$\mu, \nu$, we obtain:
    \begin{align}\label{mu_nu}
        \frac{\mu(\X)}{\nu(\X)\epsilon_{\T}^2 \epsilon} 
        \geq \frac{(K\mu)(A)}{(K\nu)(A)} \geq 
        \frac{\mu(\X)\epsilon_{\T}^2 \epsilon}{\nu(\X)}.
    \end{align}

    Lemma \ref{Birkoff} and equation (\ref{mu_nu})
    conclude the proof:
\begin{align*}
    & \tau(K)=\sup _{0<h\left(\mu, \nu\right)<\infty} \frac{h\left(K \mu, K \nu\right)}{h\left(\mu, \nu\right)}
    =\tanh \left[\frac{1}{4} H(K)\right]\\
    & =\tanh \left[\frac{1}{4} \sup _{\mu, \nu} h\left(K \mu, K \nu\right)\right]
    \\ & \leq \tanh \left[\frac{1}{4} \log \left( \frac{1}{\epsilon_{\T}^4 \epsilon^2} \right)\right]
    \\ & =\frac{1-\epsilon_{\T}^2 \epsilon}{1+\epsilon_{\T}^2 \epsilon}<1.
\end{align*}
\end{proof}
\begin{lemma}\label{Cauchy}
    Under Assumption \ref{mixing_kernel_con},
    there exists an aperiodic reachable state 
    for filter process.
\end{lemma}
\begin{proof}
Define 
$\pi_n^{\mu *}(\cdot)=P^{\mu}\left(X_n \in \cdot \mid Y_{[0, n]}=(y',\dots, y')\right), n \in \mathbb{N}$ 
where $P^{\mu}$ is the probability measure induced by the prior $\mu$.
First we prove that $\pi_n^{\mu *}$ is 
a Cauchy sequence on $\Z$ under total-variation metric.
By Lemma \ref{clm},
$
h\left(\pi_{n+2}^{\mu *}, \pi_{n+1}^{\mu *}\right) \leq c h\left(\pi_{n+1}^{\mu *}, \pi_{n}^{\mu *}\right)
$
for some $c<1$, since $\pi_{n+1}^{\mu *}$ 
and $\pi_{n}^{\mu *}$ are comparable for $n\geq1$
under Assumption \ref{mixing_kernel_con}. 
Thus, the sequence is a Cauchy sequence under the Hilbert metric, and consequently under the total-variation metric.
Since $\Z$ is a complete metric space, 
there exists a probability measure $\pi^{\mu *}\in \mathcal{Z}$ which is the limit of the Cauchy sequence $(\pi_n^{\mu *})_{n\in N}$. 
Now we will show that the limit is
independent of the initial measure $\mu$.
For any two initial measures 
$\mu$ and $\nu$, 
$F(\mu, y')$ and $F(\nu, y')$ are comparable
since $K$ is a mixing kernel. Therefore, 
$
h\left(\pi_{n+1}^{\mu *}, \pi_{n+1}^{\nu *}\right) \leq c h\left(\pi_{n}^{\mu *}, \pi_{n}^{\nu *}\right)
$ for $n\geq 1$. Consequently, the limits of 
both sequences are the same. Thus, 
the limit point is independent of the initial measure. 
Let $\pi^*$ be the limit point.

For any open neighborhood $O$ of $\pi^{*}$, 
there exist $\kappa>0$
such that $O \supset B_\kappa(\pi^{*}):=
\{\mu\in \mathcal{Z}: \rho_{BL}(\mu,\pi^{*})<\kappa\}$.
Under Assumption \ref{mixing_kernel_con},
we know that $Pr^{\mu}(Y_1=\bar{y})>\epsilon$ 
for any $\mu$.
Since convergence in total variation implies
convergence in the bounded Lipschitz metric,
there exists a sufficiently large $M$ such that,
for any $n>M$,
$\pi_n^{\mu *}(\cdot)\in O$ and 
$Pr^{\mu}(Y_{[0, n]}=(\bar{y},\dots, \bar{y}))>0$
This is true for any initial prior $\mu$,
so $\pi^{*}$ is topologically reachable.
It is also true for $\mu=\pi^{*}$, 
meaning $\pi^{*}$ is topologically aperiodic.
Thus, $\pi ^*$ is a topologically 
reachable aperiodic state. 

\end{proof}

For some applications, a reachable state follows from more direct arguments, e.g. if there exists a state which is uniquely recoverable from a measurement with positive measure, or cases where non-informative measurements take place with positive probability. Let us note the latter in more detail:

\begin{assumption} \label{positive_eq} 
    The transition kernel ${\T}$ is uniformly ergodic with
    invariant probability measure $\pi^{*}$. Furthermore, there exists an observation realization 
    $\bar{y} \in \mathbb{Y}$ such that for some 
    positive $\epsilon$, $\Q(\bar{y}|x)=\epsilon$ 
    for every $x\in \mathbb{X}$. 
\end{assumption}

\begin{definition}
    \begin{align}
        \pi_n^{\mu *}(\cdot)=P^{\mu}\left(X_n \in \cdot \mid Y_{[0, n]}=(\bar{y},\dots, \bar{y})\right), n \in \mathbb{N}
    \end{align} where $P^{\mu}$ is the probability measure induced by the prior $\mu$.
\end{definition}

\begin{lemma}\label{reachability}
Under Assumption \ref{positive_eq}, 
$\pi_n^{\mu *}(\cdot)$ converges to the unique 
invariant probability
measure $\pi^{*}$
for any initial measure $\mu \in \Z$ under total variation. 
Furthermore,
the limit measure $\pi^{*}$ is a 
topologically reachable aperiodic state for the filter process.
\end{lemma}
\begin{proof}
\begin{align*}
    & \pi_n^{\mu *}(A)=P^{\mu}\left(X_n \in A \mid Y_{[0, n]}=(\bar{y},\dots, \bar{y})\right)
    \nonumber \\ & = \frac{P^{\mu}\left(Y_n = \bar{y} \mid X_n \in A,  Y_{[0, n-1]}=(\bar{y},\dots, \bar{y})\right) P^{\mu}\left(X_n \in A \mid Y_{[0, n-1]}=(\bar{y},\dots, \bar{y})\right)}
    {P^{\mu}\left(Y_n = \bar{y} \mid Y_{[0, n-1]}=(\bar{y},\dots, \bar{y})\right)}
    \nonumber \\ & =\frac{P^{\mu}\left(Y_n = \bar{y} \mid X_n \in A,  Y_{[0, n-1]}=(\bar{y},\dots, \bar{y})\right) P^{\mu}\left(X_n \in A \mid Y_{[0, n-1]}=(\bar{y},\dots, \bar{y})\right)}
    {\int_{x_n \in \mathbb{X}} P^{\mu}\left(Y_n = \bar{y} \mid X_n=x_n,  Y_{[0, n-1]}=(\bar{y},\dots, \bar{y})\right)P^{\mu}\left(X_n \in d x_n \mid Y_{[0, n-1]}=(\bar{y},\dots, \bar{y})\right) }
    \nonumber \\ & = \frac{\epsilon P^{\mu}\left(X_n \in A \mid Y_{[0, n-1]}=(\bar{y},\dots, \bar{y})\right)}
    {\int_{x_n \in \mathbb{X}} \epsilon P^{\mu}\left(X_n \in d x_n \mid Y_{[0, n-1]}=(\bar{y},\dots, \bar{y})\right)}
    = P^{\mu}\left(X_n \in A \mid Y_{[0, n-1]}=(\bar{y},\dots, \bar{y})\right).
\end{align*}
The second line results from the Bayesian update equation. 
The last line derives from the fact that the current 
observation is independent of past observations given 
the current state, and the fact that 
under Assumption \ref{positive_eq}, 
$\Q(\bar{y}|x) = \epsilon$ for every $x \in \mathbb{X}$.

Then, the recursive application of the update rule leads to
\begin{align*}
    \pi_n^{\mu *}(A)=P^{\mu}\left(X_n \in A \mid Y_{[0, n]}=(\bar{y},\dots, \bar{y})\right)
    =P^{\mu}\left(X_n \in A \right).
\end{align*}
which by the uniform ergodicity of the transition kernel 
${\T}$, implies that $\pi_n^{\mu *}(\cdot)$ converges 
to $\pi^*$ under the total variation.
$\pi^{ *}$ is an aperiodic reachable state, as can 
be seen from the same argument presented 
in the final part of Lemma \ref{Cauchy}.

\end{proof}

\section{Conclusion}
In this paper, we have studied the
geometric ergodicity, unique ergodicity 
and Wasserstein continuity
of non-linear filter processes. 
Our assumptions significantly expand 
the existing findings in three main 
directions: First, we have presented 
novel conditions for the geometric 
ergodicity of non-linear filters, 
making a new contribution to the 
literature. Second, we have extended 
the conditions for unique ergodicity 
to scenarios where the state space is compact. 
Third, as a by-product of our analysis, 
we have derived conditions for Wasserstein 
continuity of non-linear filters.


\begin{appendices}

\section{Proof of Lemma \ref{Regularity}}\label{Proof_Lemma_Regularity}

\begin{proof}[Proof of Lemma \ref{Regularity}]
    We equip $\mathcal{Z}$ 
    with the metric $\rho_{BL}$ to define the bounded-Lipschitz 
    norm $|f|_{BL}$ for any Borel measurable function 
    $f: \mathcal{Z}\rightarrow \mathbb{R}$. 
    In this proof, we use a similar technique as above, 
    but this time we will treat the measurements 
    $y_1^n := (y_1, y_2, \ldots, y_n)$ as a single measurement 
    variable. This allows us to establish a uniform bound 
    over $n$. Let us now continue with the proof. Similarly
    to analyze in inequality (\ref{second}) we can get
\begin{align} \label{secondlemma}
& \rho_{B L}\left(\eta^n(\cdot \mid z), \eta^n\left(\cdot \mid z^{\prime}\right)\right)
\leq\left\|Pr\left(\cdot \mid z_0^{\prime}\right)-Pr\left(\cdot \mid z_0\right)\right\|_{T V} \\
& \quad+\sup _{\|f\|_{B L} \leq 1} \int_{\mathbb{Y}^n}\left|f\left(z_n\left(z_0^{\prime}, y_1^n\right)\right)-f\left(z_n\left(z_0, y_1^n\right)\right)\right| 
Pr\left(d y_1^n \mid z_0\right).
\end{align}
The first term, similar to inequality (\ref{TV}), is bounded as follows:
\begin{align}\label{nsteptv}
\left\|Pr\left(\cdot \mid z_0^{\prime}\right)-Pr\left(\cdot \mid z_0\right)\right\|_{T V} \leq\left(1+\alpha\right) \rho_{B L}\left(z_0^{\prime}, z_0\right).
\end{align}
For the second term in (\ref{secondlemma}), following a similar analysis as in (\ref{gy2}):
\begin{align}\label{secondlemma2}
&\int_{\mathbb{Y}^n}\left|f\left(z_n\left(z_0^{\prime}, y_1^n\right)\right)
-f\left(z_n\left(z_0, y_1^n\right)\right)\right| Pr\left(d y_1^n \mid z_0\right)\nonumber\\
&\leq \int_{\mathbb{Y}^n} \left( \int_{\mathbb{X}} g_{y_1^n}\left(x_n\right) z_n\left(z_0^\prime, y_1^n\right)(d x_n)\right) 
\left( Pr\left(d y_1^n \mid z_0\right)- Pr\left(d y_1^n \mid z_0^\prime \right) \right) \\
&+ \int_{\mathbb{X}} \int_{\mathbb{Y}^n} g_{y_1^n}\left(x_n\right) z_n\left(z_0^\prime, y_1^n\right)(d x_n) Pr\left(d y_1^n \mid z_0^\prime\right) 
- \int_{\mathbb{X}} \int_{\mathbb{Y}^n} g_{y_1^n}\left(x_n\right) z_n\left(z_0, y_1^n\right)(dx_n)Pr\left(d y_1^n \mid z_0\right) 
\end{align}
for some measurable function $g_{y_1^n}:\mathbb{Y}^n \to \operatorname{BL_1}(\mathbb{X})$.
For the first term, following a similar analysis as in (\ref{sup-1}), we have:
\begin{align}\label{secondfirstterm}
    &\int_{\mathbb{Y}^n} \left( \int_{\mathbb{X}} g_{y_1^n}\left(x_n\right) z_n\left(z_0^\prime, y_1^n\right)(d x_n)\right) 
    \left( Pr\left(d y_1^n \mid z_0\right)- Pr\left(d y_1^n \mid z_0^\prime \right) \right) \nonumber \\
    &\leq\left(1+\alpha\right) \rho_{B L}\left(z_0^{\prime}, z_0\right).
\end{align}
For the second term, following a similar analysis as in (\ref{sup-2}),
we have:
\begin{align}
&\nonumber \int_{\mathbb{X}} \int_{\mathbb{Y}^n} g_{y_1^n}\left(x_n\right) z_n\left(z_0^\prime, y_1^n\right)(d x_n) Pr\left(d y_1^n \mid z_0^\prime\right) 
- \int_{\mathbb{X}} \int_{\mathbb{Y}^n} g_{y_1^n}\left(x_n\right) z_n\left(z_0, y_1^n\right)(dx_n)Pr\left(d y_1^n \mid z_0\right) \\
& \leq\left(1+\alpha\right) \rho_{B L}\left(z_0^{\prime}, z_0\right)\label{second2term}.
\end{align}
By using the inequalities (\ref{nsteptv}),(\ref{secondfirstterm}), and (\ref{second2term}), we can write:
$$
\rho_{B L}\left(\eta^n(\cdot \mid z), 
\eta^n\left(\cdot \mid z^{\prime}\right)\right) \leq 
3\left(1+\alpha \right) \rho_{B L}\left(z, z^{\prime}\right).
$$
\end{proof}



%%=============================================%%
%% For submissions to Nature Portfolio Journals %%
%% please use the heading ``Extended Data''.   %%
%%=============================================%%

%%=============================================================%%
%% Sample for another appendix section			       %%
%%=============================================================%%

%% \section{Example of another appendix section}\label{secA2}%
%% Appendices may be used for helpful, supporting or essential material that would otherwise 
%% clutter, break up or be distracting to the text. Appendices can consist of sections, figures, 
%% tables and equations etc.

\end{appendices}

%%===========================================================================================%%
%% If you are submitting to one of the Nature Portfolio journals, using the eJP submission   %%
%% system, please include the references within the manuscript file itself. You may do this  %%
%% by copying the reference list from your .bbl file, paste it into the main manuscript .tex %%
%% file, and delete the associated \verb+\bibliography+ commands.                            %%
%%===========================================================================================%%
\bibliographystyle{plain}
\bibliography{SerdarBibliography.bib}

\end{document}
