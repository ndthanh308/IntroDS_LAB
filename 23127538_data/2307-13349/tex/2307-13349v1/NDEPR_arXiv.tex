\documentclass[reprint,superscriptaddress,amsmath,amssymb,prl,showpacs,floatfix]{revtex4-1}
\usepackage{cases}
\usepackage{amsmath}
\usepackage{amssymb}
\usepackage{amsfonts}
\usepackage{amssymb}
\usepackage{dcolumn}
\usepackage{bm}
\usepackage{hyperref}
\usepackage{graphicx}
\usepackage{upgreek}
\usepackage{subfigure}
\usepackage{color}
\usepackage{textcomp}
%\usepackage[export]{adjustbox}
\UseRawInputEncoding

\begin{document}



\title{In situ electron paramagnetic resonance spectroscopy using single nanodiamond sensors}
%{Zero-field electron paramagnetic resonance spectroscopy using spin sensors in single tumbling nanodiamonds}


%% Notice placement of commas and superscripts and use of &
%% in the author list

\author{Zhuoyang Qin}
\altaffiliation{These authors contributed equally to this work.}
\affiliation{CAS Key Laboratory of Microscale Magnetic Resonance and School of Physical Sciences, University of Science and Technology of China, Hefei 230026, China}
\affiliation{CAS Center for Excellence in Quantum Information and Quantum Physics, University of Science and Technology of China, Hefei 230026, China}

\author{Zhecheng Wang}
\altaffiliation{These authors contributed equally to this work.}
\affiliation{CAS Key Laboratory of Microscale Magnetic Resonance and School of Physical Sciences, University of Science and Technology of China, Hefei 230026, China}
\affiliation{CAS Center for Excellence in Quantum Information and Quantum Physics, University of Science and Technology of China, Hefei 230026, China}

\author{Fei Kong}
\email{kongfei@ustc.edu.cn}
\affiliation{CAS Key Laboratory of Microscale Magnetic Resonance and School of Physical Sciences, University of Science and Technology of China, Hefei 230026, China}
\affiliation{CAS Center for Excellence in Quantum Information and Quantum Physics, University of Science and Technology of China, Hefei 230026, China}

\author{Jia Su}
\affiliation{CAS Key Laboratory of Microscale Magnetic Resonance and School of Physical Sciences, University of Science and Technology of China, Hefei 230026, China}
\affiliation{CAS Center for Excellence in Quantum Information and Quantum Physics, University of Science and Technology of China, Hefei 230026, China}

\author{Zhehua Huang}
\affiliation{CAS Key Laboratory of Microscale Magnetic Resonance and School of Physical Sciences, University of Science and Technology of China, Hefei 230026, China}
\affiliation{CAS Center for Excellence in Quantum Information and Quantum Physics, University of Science and Technology of China, Hefei 230026, China}

\author{Pengju Zhao}
\affiliation{CAS Key Laboratory of Microscale Magnetic Resonance and School of Physical Sciences, University of Science and Technology of China, Hefei 230026, China}
\affiliation{CAS Center for Excellence in Quantum Information and Quantum Physics, University of Science and Technology of China, Hefei 230026, China}

\author{Sanyou Chen}
\affiliation{CAS Key Laboratory of Microscale Magnetic Resonance and School of Physical Sciences, University of Science and Technology of China, Hefei 230026, China}
\affiliation{CAS Center for Excellence in Quantum Information and Quantum Physics, University of Science and Technology of China, Hefei 230026, China}
\affiliation{School of Biomedical Engineering and Suzhou Institute for Advanced Research, University of Science and Technology of China, Suzhou 215123, China}

\author{Qi Zhang}
\affiliation{CAS Key Laboratory of Microscale Magnetic Resonance and School of Physical Sciences, University of Science and Technology of China, Hefei 230026, China}
\affiliation{CAS Center for Excellence in Quantum Information and Quantum Physics, University of Science and Technology of China, Hefei 230026, China}
\affiliation{School of Biomedical Engineering and Suzhou Institute for Advanced Research, University of Science and Technology of China, Suzhou 215123, China}

\author{Fazhan Shi}
\email{fzshi@ustc.edu.cn}
\affiliation{CAS Key Laboratory of Microscale Magnetic Resonance and School of Physical Sciences, University of Science and Technology of China, Hefei 230026, China}
\affiliation{CAS Center for Excellence in Quantum Information and Quantum Physics, University of Science and Technology of China, Hefei 230026, China}
\affiliation{School of Biomedical Engineering and Suzhou Institute for Advanced Research, University of Science and Technology of China, Suzhou 215123, China}
\affiliation{Hefei National Laboratory, University of Science and Technology of China, Hefei 230088, China}


\author{Jiangfeng Du}
\email{djf@ustc.edu.cn}
\affiliation{CAS Key Laboratory of Microscale Magnetic Resonance and School of Physical Sciences, University of Science and Technology of China, Hefei 230026, China}
\affiliation{CAS Center for Excellence in Quantum Information and Quantum Physics, University of Science and Technology of China, Hefei 230026, China}
\affiliation{Hefei National Laboratory, University of Science and Technology of China, Hefei 230088, China}
\affiliation{School of Physics, Zhejiang University, Hangzhou 310027, China}

\begin{abstract}
An ultimate goal of electron paramagnetic resonance (EPR) spectroscopy is to analyze molecular dynamics in place where it occurs, such as in a living cell. The nanodiamond (ND) hosting nitrogen-vacancy (NV) centers will be a promising EPR sensor to achieve this goal. However, ND-based EPR spectroscopy remains elusive, due to the challenge of controlling NV centers without well-defined orientations inside a flexible ND. Here, we show a generalized zero-field EPR technique with spectra robust to the sensor's orientation. The key is applying an amplitude modulation on the control field, which generates a series of equidistant Floquet states with energy splitting being the orientation-independent modulation frequency. We acquire the zero-field EPR spectrum of vanadyl ions in aqueous glycerol solution with embedded single NDs, paving the way towards \emph{in vivo} EPR.
\end{abstract}

\maketitle


Electron paramagnetic resonance (EPR) spectroscopy is a well-established technique for analyzing molecules containing unpaired electrons. Among its widespread applications in diverse scientific fields \cite{Loubser1978,Hagen2006,Roessler2018}, a featured one is the study of dynamic processes, such as monitoring redox reactions \cite{Bruckner2010} and unraveling molecular motions \cite{Borbat2001}. Performing those studies in living cells is an active research topic \cite{Su2019,Bonucci2020,Pierro2023}, where an ultimate goal is promoting the EPR detection to single-cell level. Towards this goal, an essential but unmet precondition is developing suitable EPR sensors with both high spin sensitivity and good biocompatibility. The conventional EPR sensor is a macroscopic resonant microwave cavity with limited spin sensitivity. In the past decades, numerous microscopic EPR sensors have been developed to improve the spin sensitivity, including magnetic resonance force microscopy \cite{Rugar2004}, scanning tunneling microscopy \cite{Baumann2015}, and superconducting microresonator \cite{Ranjan2020,Budoyo2020}, but they require cryogenic temperatures and high vacuum environments. Alternatively, nitrogen-vacancy (NV) centers in diamond can also serve as EPR sensors with single-spin sensitivity \cite{Schlipf2017,Pinto2020}, even at ambient conditions \cite{Grinolds2013,Shi2015,Shi2018}. Furthermore, the diamond hosting NV centers can shrink to nanometer size, making itself more flexible to be \emph{in situ} sensor, such as magnetometry inside polymers \cite{Price2019}, relaxometry in lipid bilayer \cite{Kaufmann2013}, and thermometry in living cells \cite{Kucsko2013}. However, using this flexible nanodiamond (ND) as an EPR sensor remains challenging.

The flexibility of NDs, on the other hand, also brings the uncertainty of their orientations, which prevents the hosted NV centers from measuring EPR spectra. It is because the NV center has an anisotropic response to magnetic fields with a principal axis along the N-V axis \cite{Doherty2013}. In the presence of external static or oscillating magnetic field, random tumbling of the ND will lead to variations in the transition frequency or strength of the hosted NV center, and thus prevents the current EPR detection schemes. For instance, both the double electron-electron resonance (DEER) \cite{Grinolds2013,Shi2015,Shi2018,Grotz2011,Mamin2012} and the cross-polarization schemes \cite{Belthangady2013,Hall2016} require precise quantum controls of the NV spin states. To overcome this challenge, an active approach is either manipulating the ND orientation, such as, by using optical tweezers \cite{Geiselmann2013}, {tracking the ND orientation \cite{Igarashi2020} and then adjusting the control field,} or
optimizing the control pulses \cite{Konzelmann2018}. Besides, another passive but technically simpler way is to develop detection schemes that are naturally insensitive to the orientation. An illuminating example is the zero-field EPR \cite{Bramley1983,Kong2018}, where the resonance frequency does not depend on the spin target's orientation, although still depend on the NV sensor's orientation.

Here, we generalize the robustness of zero-field EPR technique to not only the target's but also the sensor's orientations. By applying an extra amplitude modulation on the control field, the modulation frequency, rather than the field strength, will determine the resonance condition. We experimentally demonstrate it by performing the EPR measurements on P1 centers with different driving field strength, showing that the peak position is indeed field-strength independent. To further show the robustness of our method, we immerse the ND in an aqueous glycerol solution of vanadyl sulfate, and then use the hosted NV centers to detect the vanadyl ions. Although both the ND and the ions are tumbling, we can still acquire a clear EPR spectrum. Our results show the possibility of using flexible NDs as EPR sensors to enable \emph{in situ} and even \emph{in vivo} EPR measurements.

% Figure environment removed

\section*{Results}
\subsection{Scheme of zero-field ND-EPR}$\\$
Considering that a ND is placed inside a sample containing paramagnetic molecules, where all of them are tumbling randomly (Fig.~\ref{scheme}a), the task is using the NV center inside the ND to measure the EPR spectrum of the molecules. In the absence of external static magnetic field, the energy levels of both the sensor and the target are irrelevant to their orientations \cite{Bramley1983}. The Hamiltonian of this sensor-target system can be written as
\begin{equation}
H_{0} = D S_z^2 + d_{ij}S_i T_j + \omega T_z,
\label{H0}
\end{equation}
where $\mathbf{S}$ and $\mathbf{T}$ are the spin operator of the NV sensor and the spin target, respectively, $d_{ij}$ ($i,j=x,y,z$) is the dipole-dipole coupling between them, $D=2.87$ GHz is the zero-field splitting of NV sensor, {and $\omega$ is the energy splitting of the target spin induced by the intrinsic interaction.} 



There usually exists a large energy mismatch between the sensor and the target ($|D-\omega|\gg d$), and thus the dipole-dipole coupling between them has undetectable influence on the sensor. A driving field can eliminate this energy mismatch by bringing the NV center from lab frame to dress frame (Fig.~\ref{scheme}b). A direct way is applying a resonant microwave (MW) field $B_1 \cos D t$ (Fig.~\ref{scheme}c) \cite{Kong2018}, and then we have
\begin{equation}
H_{\text{I}} = \frac{\Omega}{2} S_x + d_{zj}S_z T_j + \omega T_z
\label{H1}
\end{equation}
in the interaction picture, where $\Omega=\gamma_{\text{NV}}B_{1}\sin{\theta}$ is the Rabi frequency, $\gamma_{\text{NV}} = -28.03$ GHz/T is the gyromagnetic ratio of the NV electron spin, and $\theta$ is the angle between the MW magnetic field $\mathbf{B_1}$ and the N-V axis. The energy gap of the sensor becomes $\Omega/2$. By sweeping $\Omega$, a resonant cross-relaxation process will happen when $\Omega/2 = \omega$, resulting in a reduction of the photoluminescence (PL) rate of the NV center. Experimentally, the sweep is performed on $B_1$ rather than $\Omega$, so the actual resonance condition is
\begin{equation}
	\gamma_{\text{NV}}B_{1}\sin{\theta} = \omega.
	\label{resonance1}
\end{equation}
Figure~\ref{scheme}d gives a simulation of the dependence of spectra on $\theta$. One can see the peak position varies dramatically when $\theta$ deviates from $\pi/2$. For a randomly tumbling ND, the spectrum will show line broadening and become asymmetric. Note that the two $\pi/2$ pulses in Fig.~\ref{scheme}c will also deteriorate, resulting in weaker signal.

To address this issue, we perform an periodical amplitude modulation on the continuous driving MW of the form $B_1 \cos ft \cos D t$ (Fig.~\ref{scheme}e), and then the Hamiltonian Eq.~\ref{H1} turns into (Supplementary Note 1)
\begin{equation}
H_{\text{II}} = \sum_{m=\text{odd}}2J_m(\frac{\kappa}{2})d_{zj}\sin mft S_y T_{j}+\omega T_{z},
\label{H2}
\end{equation}
where $J_{m}$ is the $m$-th order Bessel functions of the first kind, and $\kappa=\Omega/f$ is the relative driving index. This periodical modulation creates a series of Floquet side bands with splitting determined by the modulation frequency $f$. In the situation of $\Omega \ll f$, only the first-order term matters. The dipolar coupling will induce an additional longitudinal relaxation on the NV sensor with rate of (Supplementary Note 1)
\begin{equation}
	\Gamma_1^{'} = \frac{3 \kappa^2}{64}\frac{(d_{zx}^2+d_{zy}^2)\Gamma_2}{{\Gamma_2}^2+(f-\omega)^2},
	\label{rate}
\end{equation}
where $\Gamma_2=\Gamma_{2,\text{NV}}+\Gamma_{2,\text{tar}}$ is the total decoherence rate. After a evolution time of $t$, the signal contrast will be 
\begin{equation}
	S = \frac{2}{3}e^{-\Gamma_1 t}(1-e^{-\Gamma_1^{'} t}),
	\label{signal}
\end{equation}
where $\Gamma_1 = 1/T_1$ is the intrinsic relaxation rate of the NV sensor. So the resonance condition becomes
\begin{equation}
	f = \omega.
	\label{resonance2}
\end{equation}
The energy mismatch can also be removed at the price of a reduced coupling by a factor of $\kappa/4$. 
Now the resonance condition does NOT depend on $\theta$, but the signal strength does (Fig.~\ref{scheme}f). The tumbling induced line broadening is completely removed.

\subsection{EPR measurements with fixed NDs}$\\$
To better see the dependence of EPR spectra on the Rabi frequency $\Omega$, we first perform the experimental demonstrations on fixed NDs, where $\Omega$ is well-defined and adjustable. We place the NDs on a coverslip by spin coating, and observe them on a home-built confocal microscope (Fig.~\ref{P1spectrum}a). The 532 nm green laser and the red fluorescence are used to polarized and read out the spin state of NV centers, respectively. Figure~\ref{P1spectrum}b shows the Rabi oscillation of the NV centers in one ND. {As each ND contains an average of 12-14 NV centers, the Fourier transform of the Rabi oscillation shows different peaks, corresponding} to differently oriented NV centers. Here $\Omega$ is defined as the dominated Rabi frequency. 

% Figure environment removed

By applying the pulse sequence given in Fig.~\ref{scheme}e, we can directly get the zero-field EPR spectrum of P1 centers (Fig.~\ref{P1spectrum}c), which are another kind of defects in diamond. Since the signal strength depends on the relative driving index $\kappa$ (Eq.~\ref{rate} {and Eq.~\ref{signal}), which is proportional to $B_1/f$}, we sweep the driving amplitude $B_1$ accordingly during the sweeping of $f$, and keep $\kappa$ as a constant. According to previous measurements on bulk diamonds \cite{Kong2018}, the zero-field EPR spectrum of $^{14}$N P1 centers should have three peaks at 18 MHz, 130 MHz, and 148 MHz. The first peak is difficult to observe for NDs, because it merges with the broad background peak around $f=0$ MHz (Fig.~\ref{P1spectrum}c). Here we focus on the last two. As shown in Fig.~\ref{P1spectrum}c, two clear peaks appear at the expected positions, {even though the ND contains several differently oriented NV centers with different effective driving amplitudes}. Besides, we repeat the measurement with different $\kappa$ to simulate the rotation of NDs. The peak position indeed shows independence on $\kappa$ (Fig.~\ref{P1spectrum}d), while the peak contrasts show positive dependence on $\kappa$ (Fig.~\ref{P1spectrum}e), consisting with the theoretical prediction. Therefore, it is promising that the ND-EPR spectrum will be robust to the tumbling of NDs.



% Figure environment removed

\subsection{EPR measurements with tumbling NDs}$\\$
We then perform the measurement on tumbling NDs to directly show the robustness of our scheme. For a free ND in aqueous solutions, it has both rotational and transnational diffusion. To keep the ND staying at the focus of laser, extra techniques such as wide-field excitation and charge-coupled device (CCD) detection \cite{Steinert2013} or real-time tracking \cite{Hou2020} are required, which are beyond the scope of this work. In order to simplify the measurement, we use a soft `string', which is a PEG molecule, to tether the ND on the surface of a coverslip (see Methods). Its length is roughly 120 nm, much larger than the ND diameter ($\sim$ 40 nm) and smaller than the laser spot ($\sim$ 800 nm). So the rotational motion is nearly unperturbed, while the transnational motion is restricted. We use a short mPEG molecule to control the density of NDs, so that single NDs can be observed (Supplementary Note 2). We put a glycerol aqueous solution (glycerol:water=9:1) of vanadyl sulfate {with a concentration of 25 mM} on the ND-tethered coverslip (see Methods), and then use the embedded ND to sensing the vanadyl ions (Fig.~\ref{VOspectrum}a). {Here we use the glycerol-water mixture rather than the pure water in order to reduce the rotational diffusion rate of vanadyl ions, which we will discuss below.} As shown in Fig.~\ref{VOspectrum}b, the Rabi oscillation decays quickly, corresponding to a wide distribution of Rabi frequency. The oscillation also changes in time slowly, confirming the tumbling of NDs in the aqueous solution. The irregular oscillation shows the difficulty of precisely controlling the spin in a tumbling ND. Nevertheless, we can still clearly acquire the zero-field EPR spectrum of vanadyl ions. 


The ion exists as [VO(H$_2$O)$_5$]$^{2+}$ in aqueous solution at moderately low pH \cite{Grant1999}. It consists of a electron spin $S=1/2$ and a nuclear spin $I=7/2$ with hyperfine interaction between them. At zero magnetic field, the spin Hamiltonian is
\begin{equation}
H_{\text{VO}} = A_{\perp}(S_x I_x + S_y I_y) + A_{\parallel}S_z I_z,
\label{HV}
\end{equation}
where $A_{\perp}=$ 208.5 MHz, $A_{\parallel}=$ 547 MHz \cite{Grant1999}, and we neglect the small nuclear quadrupole coupling term. The eigenstates can be written as $|T,m_T\rangle$ ($T=4,3, m_T = \pm T, \pm (T-1), \ldots, 0$), where $\mathbf{T}=\mathbf{S}+\mathbf{I}$ is the total angular momentum. The eigenenergies are $\left(-A_{\parallel} \pm \sqrt{m_T^2 A_{\parallel}^2 + (16-m_T^2) A_{\perp}^2} \right)/4$, where plus and minus correspond to $T =$ 4 and 3, respectively. Due to the axial symmetry ($A_x = A_y = A_{\perp}$), all the $m_T\neq 0$ states are doubly degenerate, and thus 16 states occupy 9 energy levels (Fig.~\ref{VOspectrum}c). In general, all the transitions that meet the selection rule $\Delta m_T= 0 , \pm1$ are observable. For each transition, the vanadyl ion can be simplified to a two-level system $\omega T_z$ as in Eq.~\ref{H0} with a modified dipolar coupling to the NV sensor (Supplementary Note 3). 

Figure~\ref{VOspectrum}d gives a simulated EPR spectrum of vanadyl ions, where up to 12 peaks are observable. However, Peaks 5-8 will overlap with the strong signal of P1 centers, and thus be hardly to resolve. There also exists a strong background peak at $D/2=1435$ MHz. Because the amplitude-modulated driving field can be divided into two microwaves with frequencies of $D-f$ and $D+f$, each of which alone can also be used for the EPR measurement. {This off-resonant driving field is technically simper, but will induce a second-order shift of the resonance frequency depending on the relative driving index $\kappa$ (Supplementary Note 4). In the case of poor spectral resolution, the off-resonant drive is similar with the amplitude-modulate drive.} When $f\approx D/2$, the driving field itself will be an very strong artificial signal (Supplementary Note 4). Besides, observations of higher-frequency peaks require stronger driving power. Therefore, our measurement focus on the middle range, which is enough to extract the hyperfine constants. As shown in Fig.~\ref{VOspectrum}d, three clear peaks appear at $780$ MHz, $950$ MHz, and $1150$ MHz, possibly corresponding to peak 1, 2, and 10, respectively. Peak 9 should also exist at the middle of peak 1 and 2 with height $\sim$ 67\% of peak 10 (Supplementary Note 3). But if considering that the axial microwave also contributes to the EPR signal, the relative peak height may reduce down to $\sim$ 32\% (Supplementary Note 5). Such a weak signal is hardly to observe with current signal-to-noise ratio.

The theoretical frequencies of peak 1, 2, and 10 are $4A_{\perp}$, $\sqrt{A_{\parallel}^2+15A_{\perp}^2}$, and $(\sqrt{A_{\parallel}^2+15A_{\perp}^2}+\sqrt{4A_{\parallel}^2+12A_{\perp}^2})/2$, respectively. We then use these values as peak positions to fit the spectrum (Fig.~\ref{VOspectrum}d). The fitted result gives $A_{\perp}^{\text{fit}}=195\pm2$ MHz and $A_{\parallel}^{\text{fit}}=579\pm8$ MHz, which is slightly different from previous measurements with conventional EPR \cite{Grant1999}. {Quantitative calculation shows that the signal contrast in Fig.~\ref{VOspectrum}d can be hardly explained by freely diffused ions (Supplementary Note 3). There may exist an absorption layer of vanadyl ions on the ND surface \cite{Staudacher2015}, which contributes to the main signal.} Since the hyperfine constants of vanadyl ions strongly depends on the local ligand environment \cite{Smith2002}, we think the diamond surface might change this environment, and thus lead to different hyperfine interaction. Repeated measurements on different NDs show different results (Supplementary Note 6), suggesting that the hyperfine constant is indeed ND dependent. {Here we cannot perform a blank control because the NV centers in NDs are slowly losing spin contrast. To speed up the measurement, we repeat the experiment on ND ensembles (Supplementary Note 6), which confirms the signal indeed comes from the vanadyl ions. The additional line broadening in the ensemble ND-EPR spectrum consists with the assumption of ND-dependent hyperfine constant. Moreover, ensemble measurements on conventional EPR spectrometers rule out the dependence of hyperfine constant on glycerol ligands (Supplementary Note 7). As the ND contains multiple NV centers, each of them may measure different spectrum because of the different position in ND. But considering the signal strongly depends on the NV depth (defined by the minimum distance of the NV center from the ND surface), it is more likely the shallowest NV dominates the signal.}

According to Eq.~\ref{rate}, the spectral linewidth, {defined by the full width at the half maximum (FWHM)}, is determined by {$2\Gamma_2 = 2\Gamma_{2,\text{NV}}+2\Gamma_{2,\text{VO}}$}. For the ND we used, $\Gamma_{2,\text{NV}} \sim$ {12} MHz is estimated from the resonance spectrum of the NV center itself (Supplementary Note 2). The relaxation of the vanadyl ion $\Gamma_{2,\text{VO}}$ is contributed by the intrinsic relaxation $R_{\text{VO}}^{\text{int}}$, the dipole-dipole interaction between ions $R_{\text{VO}}^{\text{dip}}$, and the rotational diffusion of ions $R_{\text{VO}}^{\text{rot}}$ \cite{Simpson2017}. The intrinsic relaxation $R_{\text{VO}}^{\text{int}}$ is estimated to be $<$ {12} MHz according to the X band EPR spectrum \cite{Angerman1971}. The dipole-dipole mediated relaxation rate $R_{\text{VO}}^{\text{dip}}=c\times272$ MHz$\cdot$M$^{-1}$ \cite{Simpson2017}. For $c=25$ mM, this rate is 7 MHz. The rotational diffusion rate is calculated by \cite{Maclaurin2013}
\begin{equation}
	R_{\text{rot}} = \frac{k_B T}{8\pi r_0^3 \eta} \approx 2 \text{ MHz},
	\label{rotdiff}
\end{equation}
where $k_B$ is the Boltzmann constant, $T=293$ K is the temperature, $r_0\sim0.37$ nm is the radius of the aqueous vanadyl ion [VO(H$_2$O)$_5$]$^{2+}$ \cite{Angerman1971}, and $\eta=0.3$ Pa$\cdot$s is the viscosity of $9:1$ glycerol/water mixtures. {$R_{\text{rot}}$ will increase to $\sim$ 600 MHz in pure water, and then measurements will be impossible.} Note the transitional diffusion of the vanadyl ion also contributes to the line broadening \cite{Simpson2017}, but is negligible here ($\sim$ kHz) because the viscosity of the glycerol/water mixtures is much higher than the pure water. Therefore, the estimated linewidth is $\lesssim$ {66 MHz, roughly} consisting with the measured spectrum. {Here $R_{\text{VO}}^{\text{dip}}$ is underestimated and $R_{\text{rot}}$ is overestimated for ions in the absorption layer. As we perform the measurement at ambient conditions, the Zeeman splitting induced by the geomagnetic field ($\sim$ 50 \textmu{}T) will also contribute to the line broadening, which is negligible ($<$ 2.8 MHz) here.}


\section*{Discussion}
In conclusion, we have presented a robust method for EPR spectroscopy base on a nanometer-size sensor, even the sensor itself is randomly tumbling. By deploying a amplitude-modulated driving field on the NV center, the resonance condition convert from the NV orientation-dependent driving amplitude to the NV orientation-independent modulation frequency, and thus robust to the tumbling of the host ND. As a demonstration, we show that the zero-field EPR spectrum of P1 centers is indeed robust to the variations of driving amplitude. Moreover, we measure a clear EPR spectrum of vanadyl ions with the ND sensor immersed in a solution of vanadyl sulfate. The extracted hyperfine constants may be used to study the different local environment in the future. {This measurement is also robust to the presence of other ions because the peak positions at zero field are determined solely by the characteristic intrinsic interaction \cite{Bramley1983,Kong2018}.} Our method opens the way to nanoscale EPR measurements in complex biological environment, such as \emph{in vivo} EPR inside a single cell. 

The vanadium has been discovered in many biological systems and participates in various biochemical reactions \cite{Willsky1990}, for example, mimic the effect of insulin on glucose oxidation \cite{Shechter1980}, although the mechanism is still unclear. Nanoscale EPR studies of the vanadyl ion may benefit the understanding of its interaction with biological molecules, {if the spectral resolution can be improved. Except for the intrinsic relaxation $R_{\text{VO}}^{\text{int}}$, all other components contributing to the line broadening can be removed by some technical improvements. For example, $\Gamma_{2,\text{NV}}$ can be reduced to submegahertz by using high-purity NDs \cite{Knowles2014}. $R_{\text{VO}}^{\text{dip}}$ can be directly reduced by lowing the ion concentration. $R_{\text{rot}}$ can be removed by measuring solid-state spectrum. Moreover, even $R_{\text{VO}}^{\text{int}}$ itself can be reduced to $\sim 2$ MHz if utilizing the noise-insensitive transitions \cite{Bramley1985}. We note the fundamental limit may be even better than this value if single ions can be detected \cite{Kong2020}. By then, magnetic shielding or compensation will be required.}

{
Although we have solved the problem of random tumbling of NDs, there still exists other challenges for biological applications of ND-EPR, such as cellular uptake of NDs, microwave heating, and measurement efficiency. The ND we used may be too large ($\sim$ 40 nm) to compatible with single-cell studies. Fortunately, some progress has been made to reduce the ND size \cite{Tisler2009,Knowles2014,Alkahtani2019}. It is reported that even 5-nm NDs can contain NV$^{-}$ centers \cite{Terada2019}. Reductions of the ND size will inevitably leads to poorer charge-state stability and coherence time of NV centers inside the ND. However, for near-surface NV centers with the same depth, the ND size will play a marginal role. Since our scheme is highly surface-sensitive (Supplementary Note 3), smaller NDs will not deteriorate the performance, but increase the probability of finding near-surface NV centers. Besides, surface engineering of NDs is another effective way to improve cellular uptake \cite{Terada2018}. It is also helpful for increasing the charge-state stability and the coherence time of near-surface NV centers \cite{Ryan2018,Laube2019}.
}	

{
The heating effect is the main damage of microwave radiation on living cells \cite{Cao2020}. It is indeed a problem for the detection of vanadyl ions and other paramagnetic targets having strong intrinsic interactions, because the resonance frequency and accordingly required microwave power are high. A direct but inefficient way to control the average microwave power is prolonging the idle time. During our measurement of vanadyl ions, the duty cycle is 1:19, corresponding to an average power of $<$ 1 W. A better way is to optimize the microwave antenna to improve the radiation efficiency. In the future, our demonstration can be generalized to the detection of radicals with a relatively lower resonance frequency \cite{Kong2018,Shi2018}, and then the microwave heating issue can be directly avoided. 
} 

{
The current demonstration is still time consuming due to the poor measurement efficiency and signal contrast. For instance, the data in Fig.~\ref{VOspectrum}d costs nearly one week, where 95 \% of the time is wasted to control the average microwave power. The spin-to-charge conversion technique \cite{Shields2015} may be an excellent solution, because it can not only achieve better readout fidelity \cite{Irber2021,Zhang2021}, but also utilize the idle time to perform the slow charge-state readout. As described above, surface engineering is a possible way to improve the property of near-surface NV centers, which can thus increase the signal contrast. For example, the current Rabi contrast of only 10 \% (Fig.~\ref{VOspectrum}b) can be improved to $\sim$ 30 \% via the charge-state improvement, corresponding to a signal enhancement of 3 and time saving of nearly an order of magnitude. Another benefit is shallower NV centers may be usable, which will significantly increase the signal contrast due to its strong dependence on the NV depth.} Considering the spectrum is robust to the orientation of both the sensor and the target, we can utilize ensemble of NDs simultaneously to achieve high parallel efficiency, {although with sacrifice of spatial resolution.}

\section{METHODS}
\subsection{Experimental setup}
The optical part of our setup is a home-built confocal microscopy, where a diode laser (CNI MGL-III-532) generates the excitation light, and the photoluminescence is detected by an avalanche photodiode (Perkin Elmer SPCM-AQRH-14). The microwave part consists of an arbitrary waveform generator (Keysight M8190a), a microwave amplifier (Mini-circuits ZHL-16W-43+), and a coplanar waveguide.

\subsection{Chemical preparations of tumbling NDs}

The surface of the ND we use (Ad\'{a}mas, Carboxylated 40 nm Red Fluorescent ND in DI water, $\leqslant$ 1.5 ppm NV, {12-14 NV centers per particle,} 1 mg/mL) is originally terminated with carboxyl groups. To realize the biotinylation, we cover the ND surface with amine-PEG3-biotin. The detailed procedure is as follows: we freshly prepare 100 \textmu{}L solution containing 1 mM amine-PEG3-biotin (EZ-Link$^{\text{TM}}$), 5 mM EDC (1-[3-(Dimethylamino)propyl]-3-ethylcarbodiimide methiodide, Sigma-Aldrich) in 100 mM MES (4-Morpholineethanesulfonic acid sodium salt, pH 5.0),  and mix it with 10 \textmu{}L ND suspension. The reaction is allowed to proceed at room temperature for 30 min. We then add 20 \textmu{}L of 5 mM EDC to the ND mixture and wait 30 min, repeating 2 times to maximize the amount of amine-PEG3-biotin bound to the ND surface.

A coverslip is used as the substrate for bonding the NDs. Before use, the coverslip is thoroughly cleaned by the following procedure. First, we sonicate the coverslip with MilliQ water for 15 min to remove dirt. After that, we replace the MilliQ water with acetone, and sonicate the coverslip for a further 15 min, rinsing it 3 times with MilliQ water to remove any acetone residue. The coverslip is then sonicated with 1 M KOH for 20 min and rinsed with MilliQ water. Finally, we immerse the coverslip in  Piranha solution (3:1 mixture of concentrated sulphuric acid and 30\% hydrogen peroxide) for 30 min at 90 $^{\circ}$C and rinse it with MilliQ water. After cleaning, we modify the surface of the coverslip with amino group. We prepare the  aminosilylation solution by adding 10 mL of methanol, 0.5 mL of acetic acid, and 0.3 mL of APTES (3-aminopropyltrimethoxysilane, Sigma-Aldrich) to a clean beaker. Then we rinse the cleaned coverslip with methanol and sink it in the aminosilylation solution. The reaction is allowed to proceed at room temperature for 20-30 min, during which time the coverslip is sonicated in the aminosilylation solution once for 1 minute. The coverslip is then rinsed 3 times with methanol. To tether the NDs to the surface of the coverslip and to maintain the rotational movement of the NDs, we bind long-chain biotinylated PEG to the surface of the coverslip, and use short-chain mPEG to control the density of the biotin termination. We prepare a PEG mixture of 0.8 mg biotinylated NHS-ester PEG (20,000 Da, Aladdin) and 8 mg  of NHS-ester mPEG (5,000 Da, Aladdin) in a 100 \textmu{}L tube, add 64 \textmu{}L of 0.1 M NaHCO$_3$ solution, and pipette it to dissolve them completely. We then drop the PEGylation solution onto the amino-silanated coverslip and keep the environment moist to prevent the solution from drying out. We incubate the coverslip in a dark and humid environment for 5 hours, then rinse the coverslip with MilliQ water.

The final step is to attach the biotinylated ND to the biotinylated coverslip using streptavidin. We drop 1 mg/mL streptavidin solution (Sangong Biotech) onto the biotinylated coverslip, wait for 30 min and then rinse the coverslip with MilliQ water. Then we add the biotinylated ND suspension obtained in the previous step to the coverslip, wait for 30 min again and rinse the coverslip with MilliQ water.

\subsection{Chemical preparations of vanadyl ions}

We dissolve the vanadium sulfate pentahydrate powder in the deoxygenated MilliQ water to make a 100 \textmu{}L 250 mM VO$^{\text{2+}}$ solution, then mix it with 900 \textmu{}L deoxygenated glycerol to obtain glycerol aqueous solution (glycerol: water=9:1) of 25 mM VO$^{\text{2+}}$. The solvents are deoxygenated to prevent oxidation of the vanadyl ions. The detailed deoxygenation operations are purging the MilliQ water with N$_{\text{2}}$ under reduced pressure and placing the glycerol container in liquid nitrogen, then purging the glycerol with N$_{\text{2}}$ under reduced pressure. To keep the solution acidic, we add 10 \textmu{}L of 1 M sulfuric acid prepared with deoxygenated MilliQ water to the solution and mix it thoroughly. We then seal a tiny drop of the solution between the ND-bonded coverslip and the coplanar waveguide. All the above operations are done under nitrogen atmosphere in a glove box.

\section*{Data availability} 
All data needed to evaluate the conclusions in the paper are present in the paper and/or the Supplementary Information.

\renewcommand\refname{Reference}

\bibliography{NDEPR_ref}

\bibliographystyle{naturemag}

\section*{Acknowledgments}
{We thank X. C. Su and S. Wang for helpful discussions.} \textbf{Funding:} This work was supported by the National Natural Science Foundation of China (Grant Nos. T2125011, 31971156, 81788101), the National Key R\&D Program of China (Grant No. 2018YFA0306600), the CAS (Grant Nos. XDC07000000, GJJSTD20200001, Y201984, YSBR-068), Innovation Program for Quantum Science and Technology (Grant Nos. 2021ZD0302200, 2021ZD0303204), the Anhui Initiative in Quantum Information Technologies (Grant No. AHY050000), Hefei Comprehensive National Science Center, and the Fundamental Research Funds for the Central Universities. This work was partially carried out at the USTC Center for Micro and Nanoscale Research and Fabrication.

\section*{Author contributions}
J.D. and F.S. supervised the entire project. F.K. and F.S. designed the experiments. Z.Q., Z.W., J.S., and F.K. prepared the sample and performed the experiments. F.K. and Z.H. carried out the calculations. F.K., Z.Q., and F.S. wrote the manuscript. All authors discussed the results and commented on the manuscript. 


\section*{Competing interests} 
The authors declare no competing interests. 


\end{document}
