\begin{table*}[!t]
\label{tab:ablation}
\caption{\small \textbf{Ablation studies on Photograph module in TAP pre-training pipeline.} We choose PointMLP as the backbone model and conduct ablation studies on the ScanObjectNN dataset from two aspects: overall architectural designs and query designs. In Table(a), we first investigate the effectiveness of cross-attention design compared with direct projection (Model A$_1$) and direct project with self-attention (Model A$_2$). Then we analyze the influence of the different number of attention layers and feature channels in Model B and Model C. Finally, we discuss whether pad token in memory builder is beneficial in Model D. In Table(b), we conduct further experiments to compare different approaches for query designing: (1) Using learnable query based on the given viewpoints (Model E) or use mathematical formulations derived in Eq.~\ref{eq:line}. (2) The information we need when we mathematically encode pose information into init query status: (i) \textbf{Origin}: the coordinate of origin point $O$ that the optical line passes through. (ii) \textbf{Direction}: The normalized direction of the optical line. (iii) \textbf{PE}: The position embedding for each grid.}
\vspace{-6pt}
\centering
\begin{minipage}{.54\textwidth}
\begin{subtable}{\textwidth}
    \setlength\tabcolsep{3pt}
    \caption{Overall Architectural Designs.}
    \vspace{-5pt}
    \centering
    \label{tab:abl_arch}
    \newcolumntype{g}{>{\columncolor{Gray}}c}
    \adjustbox{width=\textwidth}{
    \begin{tabular}{@{\hskip 3pt}>{\columncolor{white}[3pt][\tabcolsep]}c|c|ccc| >{\columncolor{Gray}[\tabcolsep][3pt]}g@{\hskip 3pt}}
    \toprule
        Model     & Attention Type  & LayerNum. &  Channels & Mem.Pad  &   Acc.(\%) \\
    \midrule
        A$_1$      & None  &  -- &   256   & \xmark    & 87.6~\cb{(-0.9)} \\
        A$_2$        & SelfAttn  &  6 layers &  256     & \xmark      & 87.8~\cb{(-0.7)} \\
    \midrule
        B       & CrossAttn  & 2 layers &    256  & \cmark      & 87.9~\cb{(-0.6)} \\
        C       & CrossAttn  & 6 layers&   512   & \cmark      & 87.8~\cb{(-0.7)} \\
        D        & CrossAttn &  6 layers&   256      & \xmark     & 88.3~\cb{(-0.2)} \\
    \midrule
        TAP      & CrossAttn & 6 layers &  256    & \cmark       & 88.5 \\ 
    \bottomrule
    \end{tabular}}
\end{subtable}
% \vspace{5pt}

\end{minipage} \hfill
\begin{minipage}{.44\textwidth}
\begin{subtable}{\textwidth}
    \setlength\tabcolsep{3pt}
    \caption{Query Designs.}
    \vspace{-5pt}
    \label{tab:abl_query}
    \newcolumntype{g}{>{\columncolor{Gray}}c}
    \adjustbox{width=\textwidth}{
    \begin{tabular}{@{\hskip 3pt}>{\columncolor{white}[3pt][\tabcolsep]}c|c|ccc |>{\columncolor{Gray}[\tabcolsep][3pt]}g@{\hskip 3pt}}
    \toprule
        Model   & Query Type  & Origin & Direction & PE   & Acc.(\%) \\
    \midrule
        E       & Learnable    & \xmark      & \xmark      &  \xmark & 87.5~\cb{(-1.0)}        \\
    \midrule
        F$_1$   & Formula   & \xmark      & \cmark     & \cmark  &  86.5~\cb{(-2.0)} \\
        F$_2$    & Formula    & \cmark     & \xmark      &  \xmark & 87.8~\cb{(-0.7)}       \\
        F$_3$   & Formula   & \cmark     & \cmark     & \xmark  & 88.0~\cb{(-0.5)}        \\
        F$_4$   & Formula   & \cmark      & \xmark     & \cmark  & 88.1~\cb{(-0.4)}    \\
    \midrule
        TAP     & Formula   & \cmark    & \cmark      & \cmark  & 88.5        \\
    \bottomrule
    \end{tabular}}
\end{subtable}
\end{minipage}
\end{table*}