We present an LOB snapshot of one stock in Table \ref{table:LOB} which consists of five levels of the ask/bid prices and volumes.
\emph{Ask/bid prices and volumes} indicate that there are specific volumes of stock to sell/buy at the specified prices.
The first level of ask/bid price (i.e., the lowest ask price or the highest bid price) is referred to as \emph{the best ask/bid price}.
The \emph{mid price} is the average of the best ask price and the best bid price, and the \emph{spread} is the gap between them.
For example, the mid price is $(\$29.11 + \$29.01) / 2 = \$29.06$ and the spread is $\$29.11-\$29.01=\$0.10$ on the given snapshot.
The traders can trade via two types of orders: market orders (MOs) and limit orders (LOs).
An MO specifies the volume and is executed immediately with the best available price.
For example, an MO that sells $500$ shares of the stock will be executed at the average price $(\$29.01 \times 100 + \$29.00 \times 300 + \$28.99 \times 100) / 500 = \$29.00$. 
We observe that the average execution price is lower than the mid price, and the gap $\$29.06 - \$29.00=\$0.06$ is referred to as the \emph{temporary market impact}.
An LO specifies the volume as well as the price such that the trader will buy/sell the asset with a price no higher/lower than the preset price.
If LO is not executed immediately, it will be left in the LOB and appended to the order queue on the corresponding price level. 

\begin{table}[tbhp]
\centering
\begin{tabular}{ crr  }
 & Price & Volume \\
 \hline
 Ask 5 & \$29.15 & 10,000 \\
 Ask 4 & \$29.14 & 2,000 \\
 Ask 3 & \$29.13 & 1,000 \\
 Ask 2 & \$29.12 & 100 \\
 Ask 1 & \$29.11 & 200 \\
 \hline
 Bid 1 & \$29.01 & 100 \\
 Bid 2 & \$29.00 & 300 \\
 Bid 3 & \$28.99 & 800 \\
 Bid 4 & \$29.95 & 1,100 \\
 Bid 5 & \$29.09 & 1,900 \\
 \hline
\end{tabular}
\vspace{0.5cm}
\caption{A snapshot of the limit order book.}
\label{table:LOB}
\end{table}