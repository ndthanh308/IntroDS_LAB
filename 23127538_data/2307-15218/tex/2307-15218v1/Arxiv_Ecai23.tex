% !TeX document-id = {ee98884b-93e3-41d4-a7ba-3a60185f8427}
% !TeX TXS-program:bibliography = txs:///bibtex
% !TeX TXS-program:compile = txs:///pdflatex

\documentclass{ecai}

%\documentclass{article}
%\pdfpagewidth=8.5in
%\pdfpageheight=11in
%\usepackage{ijcai23}

% Use the postscript times font!
\usepackage[english]{babel}
\usepackage{times}
\usepackage{soul}
\usepackage{url}
\usepackage[hidelinks]{hyperref}
\usepackage[utf8]{inputenc}
\usepackage[small]{caption}
\usepackage{graphicx, latexsym}
\usepackage{amsmath,thm-restate}
\usepackage[thmmarks, amsthm]{ntheorem}
\usepackage{booktabs}
\usepackage{algorithm}
\usepackage[capitalize]{cleveref}
\usepackage{apptools}
\usepackage{chngcntr}
\usepackage{subcaption}
%\usepackage{ntheorem}
%\usepackage{algorithmic}
\usepackage[switch]{lineno}

\interfootnotelinepenalty = 10000


%\AtAppendix{\counterwithin{lemma}{section}}
%\AtAppendix{\counterwithin{definition}{section}}
%\AtAppendix{\counterwithin{example}{section}}

% Comment out this line in the camera-ready submission
%\linenumbers
\urlstyle{same}


% the following package is optional:
\usepackage{latexsym}

% See https://www.overleaf.com/learn/latex/theorems_and_proofs
% for a nice explanation of how to define new theorems, but keep
% in mind that the amsthm package is already included in this
% template and that you must *not* alter the styling.
%\newtheorem{theorem}{Theorem}
%\newtheorem{example}{Example}
%\newtheorem{lemma}{Lemma}
%\newtheorem{definition}{Definition}
%\newtheorem{corollary}{Corollary}
%\newtheorem*{theorem*}{Theorem}
%\newtheorem*{corollary*}{Corollary}
%\newtheorem{conjecture}{Conjecture}


\renewtheorem{theorem}{Theorem}
\newtheorem{example}[theorem]{Example}
\newtheorem{lemma}[theorem]{Lemma}
\newtheorem{definition}[theorem]{Definition}
\newtheorem{corollary}[theorem]{Corollary}
%\newtheorem*{theorem*}{Theorem}
%\newtheorem*{corollary*}{Corollary}
\newtheorem{conjecture}[theorem]{Conjecture}
\newtheorem{remark}[theorem]{Remark}
%\renewtheorem*{proof}{Proof}


\numberwithin{equation}{theorem}


\AtAppendix{\renewtheorem{theorem}{Theorem}[section]}


\usepackage{nico}
\usepackage{xparse,microtype}

%\usepackage{thmtools, thm-restate,apptools}
\usepackage{subcaption}

\usepackage{tikz}
\usepackage{pgfplots}
\pgfplotsset{compat=1.13}
\usepgfplotslibrary{external}
\tikzexternalize[prefix=tikz/]
\usepgfplotslibrary{colorbrewer}
% See http://www.traag.net/wp/wp-content/uploads/2014/06/display_colors1.png
\tikzexternaldisable
\tikzset{state/.style={rond5,minimum size=0.65cm}}
\tikzset{sink/.style={diam,minimum size=0.65cm}}
\tikzset{target/.style={rect,minimum size=0.6cm}}

\newcommand{\PO}{Player~$1$\xspace}
\newcommand{\PT}{Player~$2$\xspace}
\newcommand{\PLi}{Player~$i$\xspace}
\newcommand{\PLj}{Player~$j$\xspace}
\newcommand{\frPG}{frugal-parity~game\xspace}
\newcommand{\frBudg}{frugal-budget\xspace}
\newcommand{\frBG}{frugal-\buchi~game\xspace}
\renewcommand{\ni}{- \! i\xspace}
\newcommand{\PLni}{Player~$\ni$\xspace}

\newcommand{\cstrat}{\sigma_{\textrm{cont}}}
\newcommand{\dstrat}{\sigma_{\textrm{disc}}}

\newcommand{\frPgame}[1]{{\cal{#1}_{\texttt{frP}}}}

\newcommand{\Max}{\text{Max}\xspace}
\newcommand{\Min}{\text{Min}\xspace}

\newcommand{\zug}[1]{\langle #1  \rangle}
\newcommand{\stam}[1]{}
\renewcommand{\set}[1]{\{ #1  \}}


\newcommand{\rch}{\texttt{Reach}\xspace}
\newcommand{\buchi}{B{\"u}chi~}
\newcommand{\pari}{\texttt{Parity}\xspace}
\newcommand{\wpari}{\texttt{WkParity}\xspace}
\newcommand{\frpari}{\texttt{FrParity}\xspace}
\newcommand{\frbuchi}{\texttt{Fr\buchi}\xspace}
\newcommand{\mapindex}{\sigma}

\newcommand{\buch}{\texttt{B{\"u}chi}\xspace}
\newcommand{\thresh}{\texttt{Th}\xspace}
\newcommand{\ThR}[2]{\thresh^{\rch}_{#1}(#2)\xspace}
\newcommand{\ThB}[2]{\thresh^\buch_{#1}(#2)\xspace}

\newcommand{\Fd}{F_d}
\newcommand{\Fsubd}{F_{-d}\xspace}
\newcommand{\AttrFd}{F_d \cup \Attr{F_d}\xspace}


\def\TUG{\operatorname{TOW}}
\def\tug{\operatorname{tow}}


\newcommand\valgvb{\operatorname{val}_G(v,b)}
\newcommand\valg{\operatorname{val}_G}
\newcommand\valbarg{\overline{\operatorname{val}}_G}
\newcommand\supp{\operatorname{supp}}



%\AtAppendix{\renewtheorem{theorem}{Theorem}[section]}

%\def\fl#1{\lfloor #1\rfloor}
%\def\ceil#1{\lceil #1\rceil}



\NewDocumentCommand{\Ngh}{mg}{\texttt{N}_{\IfNoValueTF{#2}{}{#2}}(#1)}

%\newcommand{\ThPo}[2]{\thresh^{\pari}_1(#1, #2)}
%\newcommand{\ThPt}[2]{\thresh^{\pari}_2(#1, #2)}
\newcommand{\ThwP}[3]{\thresh^{\wpari}_{#1}(#2, #3)}
\NewDocumentCommand{\ThPo}{mg}{\thresh^{\pari}_1(\IfNoValueTF{#2}{}{#2,\xspace} #1)}
\NewDocumentCommand{\ThPt}{mg}{\thresh^{\pari}_2(\IfNoValueTF{#2}{}{#2,\xspace} #1)}
\NewDocumentCommand{\ThfrP}{mgg}{\thresh^{\frpari}_{#1}\IfNoValueTF{#2}{}{(\IfNoValueTF{#3}{}{#3,\xspace} #2)}}
\NewDocumentCommand{\ThfrB}{mmg}{\thresh^{\frbuchi}_{#1}(\IfNoValueTF{#3}{}{#3, \xspace} #2)}
%\newcommand{\ThfrP}[2]{\thresh^{\pari}_{#1}(#2)}
\newcommand{\ThP}[3]{\thresh^{\pari}_{#1}(#2, #3)}
\newcommand{\ThPhi}[2]{\thresh^{\Phi(#1)}(#2)}

\NewDocumentCommand{\frP}{g}{\texttt{frP}\IfNoValueTF{#1}{}{(#1)}}
\NewDocumentCommand{\frB}{g}{\texttt{frB}\IfNoValueTF{#1}{}{(#1)}}
\NewDocumentCommand{\Sld}{mg}{\texttt{S}_{#1}\IfNoValueTF{#2}{}{(#2)}\xspace}
\NewDocumentCommand{\Sed}{mg}{\texttt{T}_{#1}\IfNoValueTF{#2}{}{(#2)}\xspace}
\NewDocumentCommand{\FixSld}{g}{\texttt{S}\IfNoValueTF{#1}{}{(#1)}\xspace}
\NewDocumentCommand{\FixSed}{g}{\texttt{T}\IfNoValueTF{#1}{}{(#1)}\xspace}
\NewDocumentCommand{\buildto}{mg}{\texttt{t}_{#1}^0\IfNoValueTF{#2}{}{(#2)}\xspace}
\NewDocumentCommand{\buildti}{mgg}{\texttt{t}_{#1}^{\IfNoValueTF{#3}{i}{#3}}\IfNoValueTF{#2}{}{(#2)}\xspace}

%\NewDocumentCommand{\finreach}{mgg}{f_{\IfNoValueTF{#3}{v}{#3}}^{\IfNoValueTF{#2}{i}{#2}}(#1)}
%#1 = budget, #2 = iteration, #3 = vertex
%\NewDocumentCommand{\thresreach}{mg}{f_{\IfNoValueTF{#2}{v}{#2}}(#1)}

\DeclarePairedDelimiter\ceil{\lceil}{\rceil}
\DeclarePairedDelimiter\floor{\lfloor}{\rfloor}

%For Djordje
\usepackage[T1]{fontenc}


\newcommand{\Ti}[2]{T_i^{#1}(#2)}
\newcommand{\Tj}[2]{T_j^{#1}(#2)}
\newcommand{\Tif}[1]{T_i(#1)}
\newcommand{\Tjf}[1]{T_j(#1)}

\newcommand{\Stb}[1]{#1_{\texttt{Par}}}
\newcommand{\Buc}[1]{#1_{\texttt{Buc}}}

\newcommand{\Attr}[1]{\mathtt{Attr}(#1)}



\newcommand{\A}{{\cal A}}
\newcommand{\B}{{\cal B}}
\newcommand{\C}{{\cal C}}
\newcommand{\D}{{\cal D}}
\newcommand{\E}{{\cal E}}
\newcommand{\G}{{{\cal G}}}
\renewcommand{\H}{{\cal H}}
\newcommand{\M}{{\cal M}}
\renewcommand{\P}{{\cal P}}
\renewcommand{\S}{{\cal S}}
\newcommand{\play}{\text{play}\xspace}


\newcommand{\Real}{\mathbb{R}}
\newcommand{\Nat}{\mathbb{N}}
\newcommand{\Natstr}{\mathbb{N}^*}
\newcommand{\Natstro}{\mathbb{N}^* \setminus \mathbb{N}}
\newcommand{\Rat}{\mathbb{Q}}
\newcommand{\Z}{\mathbb{Z}}
\newcommand{\race}[1]{\text{race}(#1)\xspace}




\NewDocumentCommand{\stepbudget}{mmm}{T_{#3}^{#2}(#1)}
%#1 = budget, #2 = iteration, #3 = vertex
\NewDocumentCommand{\budget}{mm}{T_{#2}(#1)}
\NewDocumentCommand{\gamestepbound}{mm}{\mathsf{Steps}_{#1}(#2)}
\NewDocumentCommand{\stepop}{mmmm}{\mathrm{step}_{#4}(#1, #2, #3)}
\NewDocumentCommand{\step}{mmm}{\mathrm{step}_{#3}(#1, #2)}

% SPACE

% == FLOATS ==
%
\setlength{\textfloatsep}{0.5\textfloatsep} % Space below/above a float if its [t]
\setlength{\intextsep}{0.5\intextsep} % Space below/above a float if its [h]
\setlength{\floatsep}{0.5\floatsep} % Space between floats
%\setlength{\abovecaptionskip}{0.75\abovecaptionskip}
%\setlength{\belowcaptionskip}{0.75\belowcaptionskip}

%\setcounter{topnumber}{2}
%\setcounter{totalnumber}{3}
%\renewcommand{\topfraction}{0.8}

% == EQUATIONS ==
\AtBeginDocument{
\setlength{\abovedisplayskip}{0.5\abovedisplayskip}
\setlength{\abovedisplayshortskip}{0.5\abovedisplayshortskip}
\setlength{\belowdisplayskip}{0.5\belowdisplayskip}
\setlength{\belowdisplayshortskip}{0.5\belowdisplayshortskip}
}

% == FOOTNOTES ==

%\setlength{\footnotesep}{0.25cm}
%\setlength{\skip\footins}{0.5cm}

% == NAUGHTY BITS ==

%\renewcommand{\baselinestretch}{0.98} % >=0.97
%\advance\textwidth5mm % <= 6mm
%\advance\hoffset-2.5mm
%\advance\textheight3mm
%\advance\voffset-1.5mm
%%
%\setlength{\marginparwidth}{3.3cm}
%\setlength{\marginparsep}{0.1cm}

%\let\subparagraph\relax
%\usepackage{titlesec}
%\titlespacing{\section}{0pt}{2ex}{1ex}
%\titlespacing{\subsection}{0pt}{1ex}{0.5ex}
%\titlespacing{\subsubsection}{0pt}{0.5ex}{0.25ex}




\crefname{lemma}{Lem.}{Lemms.}
\crefname{theorem}{Thm.}{Thms.}
\crefname{corollary}{Cor.}{Cors.}
\crefname{equation}{Eq.}{Eqs.}
\crefname{figure}{Fig.}{Figs.}
\crefname{tabular}{Tab.}{Tabs.}

\begin{document}
	
\begin{frontmatter}


\title{Reachability Poorman Discrete-Bidding Games}

\author[A]{\fnms{Guy}~\snm{Avni}}
\author[B,C]{\fnms{Tobias}~\snm{Meggendorfer}}%\orcid{0000-0002-1712-2165}}
\author[A]{\fnms{Suman}~\snm{Sadhukhan}}
\author[D]{\fnms{Josef}~\snm{Tkadlec}}
\author[C]{\fnms{Đorđe}~\snm{Žikelić}} % use of \orcid{} is optional

\address[A]{University of Haifa}
\address[B]{Technical University of Munich}
\address[C]{Institute of Science and Technology Austria}
\address[D]{Harvard University}

%\relatedversiondetails{A full version of the paper is available at}{https://arxiv.org/abs/2210.02773}
%\Copyright{Guy Avni, Tobias Meggendorfer, Suman Sadhukhan, Josef Tkadlec, and Đorđe Žikelić}




\begin{abstract}
We consider {\em bidding games}, a class of two-player zero-sum {\em graph games}. The game proceeds as follows. Both players have bounded budgets. 
A token is placed on a vertex of a graph, in each turn the players simultaneously submit bids, and the higher bidder moves the token, where we break bidding ties in favor of \PO. 
\PO wins the game iff the token visits a designated target vertex. 
We consider, for the first time, {\em poorman discrete-bidding} in which the granularity of the bids is restricted and the higher bid is paid to the bank. 
Previous work either did not impose granularity restrictions or considered {\em Richman} bidding  (bids are paid to the opponent).
%In contrast, in {\em Richman continuous-bidding} bids can be arbitrarily small and are paid to the other player. 
While the latter mechanisms are technically more accessible, the former is more appealing from a practical standpoint. Our study focuses on {\em threshold budgets}, which is the necessary and sufficient initial budget required for \PO to ensure winning against a given \PT budget. 
We first show existence of thresholds. 
In DAGs, we show that threshold budgets can be approximated with error bounds by thresholds under continuous-bidding and that they exhibit a periodic behavior. We identify closed-form solutions in special cases. We implement and experiment with an algorithm to find threshold budgets.
%We address two limitations of previously studied mechanisms. First, most work on bidding games focused on {\em continuous bidding} which allows arbitrarily small bids. While these games exhibit an interesting mathematical structure, their practical applicability is highly questionable. We consider {\em discrete bidding}, which restricts the granularity of the bids. Second, discrete-bidding games have only been studied under {\em Richman} bidding in which the higher bidder pays the lower bidder. While Richman bidding is technically accessible, in many practical applications bids are paid to the bank, called {\em poorman} bidding. We consider, for the first time, poorma discrete-bidding games. 
\end{abstract}
\end{frontmatter}

\section{Introduction}
Two-player zero-sum {\em graph games} are a fundamental model with numerous applications, e.g., in reactive synthesis~\cite{PR89} and multi-agent systems~\cite{AHK02}. %short, verification \cite{EJS93}, 
%and a deep connection with foundations of logic \cite{Rab69}. 
A graph game is played on a finite directed graph as follows. A token is placed on a vertex, and the players move it throughout the graph. We consider {\em reachability} games in which \PO wins iff the token visits a designated target vertex. Traditional graph games are {\em turn-based}: the players alternate turns in moving the token. We consider {\em bidding games}~\cite{LLPU96,LLPSU99} in which an ``auction'' (bidding) determines which player moves the token in each turn. 

Several concrete bidding mechanisms have been defined.
In all mechanisms, both players have bounded budgets.
In each turn, both players simultaneously submit bids that do not exceed their budgets, and the higher bidder moves the token.
%guy: later
%The auction is won by the higher bidder, ties are broken according to a predefined tie-breaking mechanism, discussed later.
%Then, the winner moves the token.
The mechanisms differ in three orthogonal properties. {\em Who pays:} In {\em first-price} bidding only the winner pays the bid, whereas in {\em all-pay} bidding both players pay their bids. {\em Who is the recipient:} In {\em Richman} bidding (named after David Richman) payments are made to the other player, 
 %(i.e.\ the sum of budgets remains constant), 
 in {\em poorman} bidding payments are made to the ``bank'', i.e.\ the bid is lost. 
%guy: This seems irrelevant to me. I suggest to keep this part of the introduction as short as possible so the reader reaches the example. 
%(There are several further, less studied variants~\cite[Sec.~6]{LLPSU99}.) 
{\em Restrictions on bids:} In {\em continuous-bidding} no restrictions are imposed and bids can be arbitrarily small, whereas in {\em discrete-bidding} budgets and bids are restricted to be integers.

In this work, we study, for the first time, first-price poorman discrete-bidding games. 
This combination addresses two limitations of previously-studied models. First, most work on bidding games focused on continuous-bidding games, where a rich mathematical structure was identified in the form of an intriguing equivalence with a class of stochastic games called {\em random-turn games}~\cite{PSSW09}, in particular for infinite-duration games~\cite{AHC19,AHI18,AHZ21,AJZ21}. These results, however, rely on bidding strategies that prescribe arbitrarily small bids. Employing such strategies in practice is questionable -- after all, money is discrete.
%. However, in practice, it is seldom the case that granularity of bids is not restricted, thus employing such strategies is highly questionable. 
%Discrete bidding addresses this limitation.
Second, discrete-bidding games have only been studied under Richman bidding~\cite{DP10,AAH21,AS22}.
The advantage of Richman over poorman bidding is that, as a rule of thumb, the former is technically more accessible. In terms of modeling capabilities, however, while Richman bidding is confined to so called {\em scrip systems} that provide fairness using an internal currency, poorman bidding captures a wide range of settings since it coincides with the popular first-price auction.

The central quantity that we focus on is the {\em threshold budget} in a vertex, which is a necessary and sufficient budget for \PO to ensure winning the game. Formally, a {\em configuration} of a bidding game is a triple $\zug{v, B_1, B_2}$, where $v$ denotes the vertex on which the token is placed and $B_i$ is \PLi's budget, for $i \in \set{1,2}$. For an initial vertex $v$, we call a function $T_v: \Nat \rightarrow \Nat$ the threshold budgets at $v$ if for every configuration $c = \zug{v, B_1, B_2}$, \PO wins from $c$ if $B_1 \geq T_v(B_2)$ and loses from $c$ if $B_1 \leq T_v(B_2)-1$. We stress that we focus only on {\em pure} strategies.

% Figure environment removed


\begin{example}\label{ex:TOW}
\normalfont
Consider the game that is depicted in Fig.~\ref{fig:TOW}, where we break bidding ties in favor of \PO.
In this example, we identify the first few thresholds. In \cref{thm:tug2}, we show that the thresholds in this game are $T_{v_1}(B_2) = \lfloor B_2/\phi \rfloor$ and $T_{v_2}(B_2) = \lfloor B_2 \cdot \phi \rfloor$, %(\cref{thm:tug2}), 
where $\phi \approx 1.618$ is the golden ratio.\footnote{We encourage the reader to read more about these two sequences in \url{https://oeis.org/A000201} and \url{https://oeis.org/A005206}. See also Remark~\ref{rem:Wythoff}.}
%Here we prove this directly for the first few instances.
First, when both budgets are $0$, all biddings result in ties, which \PO wins and forces the game to $t$.
Second, we argue that \PO wins from $\zug{v_1, 0, 1}$. Indeed, \PO bids $0$. In order to avoid losing, \PT must bid $1$, wining the bidding and pays the bid to the bank. Thus, the next configuration is $\zug{v_2, 0,0}$, from which \PO wins.
Third, we show that $T_{v_2}(1) = 1$. Indeed, \PT wins from $\zug{v_2,0,1}$ by bidding $1$. On the other hand, from $\zug{v_2, 1, 1}$ \PO wins since by bidding $1$, he forces the game to $\zug{v_1,0,1}$, from which he wins.
Finally, $T_{v_1}(2) > 0$ since \PT can force two consecutive wins when the budgets are $\zug{0,2}$, and $T_{v_1}(2) = 1$ since by bidding $1$, \PO forces \PT to pay at least $2$ in order not to lose immediately, and he wins from $\zug{v_2,1,0}$. 
\end{example}



\paragraph{Applications.} 
In {\em sequential first-price auctions} $m$ items are sold sequentially in independent first-price auctions (e.g.,~\cite{LST12b,GS01}). The popularity of these auctions stems from their simplicity. Indeed, in each round of the auction, a user is only asked to bid for the current item on sale, whereas in {\em combinatorial auctions}, users need to provide an exponential input: a valuation for each subset of items.
%rather than providing valuations for exponentially many subsets of items as in {\em combinatorial auctions}, the users are only asked to provide a single bid at each round of the auction. 
Two-player sequential auctions are a special case of bidding games played on DAGs. Each vertex $v$ represents an auction for an item. A path from the root to $v$ represents the outcomes of previous rounds, i.e., a subset of items that \PO has purchased so far.  For a target bundle $T$ of items, this modeling allows us to obtain a bidding strategy that is guaranteed to purchase at least the bundle $T$ no matter how the opponent bids. Indeed, we solve the corresponding bidding game with the \PO objective of reaching a vertex in which $T$ is purchased.  We can also capture a quantitative setting in which \PO associates a value with each bundle of items. Given a target value $k$, we set \PO's target to be vertices that represent a purchased bundle of value at least $k$. We can then either find the threshold budget for obtaining value $k$ or fix the initial budgets and optimize over $k$.


%We elaborate on the construction of a bidding game that corresponds to a sequential auction. Each vertex $v$ is associated with a pair $\zug{S, i}$, where $S \subseteq \set{1,\ldots, m}$ represents a collection of items that \PO has already purchased and $i$ is the item on sale. The neighbors of $v$ are associated with $\zug{S \cup \set{i}, j}$ and $\zug{S, j}$ respectively denoting \PO purchasing $i$ or not. For a target bundle $T \subseteq \set{1,\ldots,m}$, a leaf is winning w.r.t. $T$ iff the items that \PO purchased contains $T$, and \PO's goal is to reach a winning leaf. Bidding games allow variations of this basic setting. For example, selling identical items gives rise to a game on a DAG rather than a tree. Additionally, 


Next, we describe two important classes of continuous poorman-bidding games that are technically challenging, and we argue that it is appealing to bypass this challenge by considering their discrete-bidding variants. Our study lays the basis for these extensions. 
First, all-pay poorman bidding games constitute a dynamic version of the well-known Colonel Blotto games \cite{Bor21}: we think of budgets as resources with no inherent value (e.g., time or energy) and a strategy invests the resources in order to achieve a goal. 
%short For example, in a ``best of 7'' tournament (e.g., as in the NBA playoffs), each team needs to decide which players play in each match, where the team that invests more in a match will win it (that is, we associate budgets with players' strengths and matches with biddings), and the goal is to win the tournament. 
In fact, many applications of Colonel Blotto games are dynamic, thus all-pay bidding games are arguably a more accurate model~\cite{AIT20}. 
All-pay poorman bidding games are surprisingly technically complex, e.g., already in extremely simple games, optimal strategies rely on infinite-support distributions, and have never been studied under discrete bidding.
Second, the study of partial-observation bidding games was initiated recently~\cite{AJZ23}. Poorman bidding is both appealing from a theoretical and practical standpoint but is technically complex. Again, it is appealing to consider partial-information in combination with discrete bidding.

Finally, poorman discrete bidding are amenable to extensions such as multi-player games or non-zero-sum games~\cite{MKT18}.

\subsubsection*{Our Contribution}

\paragraph{Existence of thresholds.} 
In discrete-bidding games, one needs to explicitly state how bidding ties are resolved~\cite{AAH21}. Throughout the paper, we always break ties in favor of \PO. 
%short The appeal of this mechanism is in its simplicity. 
We start by showing existence of thresholds in every game, including games that are not DAGs. Our techniques are adapted from~\cite{AAH21} for Richman discrete-bidding games. We note that existence of thresholds coincides with {\em determinacy}: from every configuration, one of the players has a {\em pure} winning strategy. We point out that while determinacy holds in turn-based games for a wide range of objectives~\cite{Mar75}, determinacy of bidding games is not immediate due to the {\em concurrent} choice of bids. For example, {\em matching pennies} is a very simple concurrent game that is not determined: neither player can ensure winning.
%short For example, consider the concurrent game {\em matching pennies} which consists of one round in which both players simultaneously choose either Heads (0) or Tails (1), and \PO wins iff the XOR of the players' choices is 1. Matching pennies is not determined since, clearly, no player can ensure winning by revealing their choice before the other player. 

\stam{OLD
The basic question we study is the existence of thresholds, which is equivalent to {\em determinacy}: from every configuration, one of the players has a {\em pure} winning strategy. Determinacy  in Richman discrete-bidding games was studied in~\cite{AAH21}, which shows that determinacy depends on the mechanism employed to resolve bidding ties. We point out that while determinacy holds in turn-based games for a wide range of objectives~\cite{Mar75}, determinacy of bidding games is not immediate since they constitute a subclass of {\em concurrent} graph games, which are not in general determined:
For example, neither player can ensure winning {\em matching pennies} (with pure strategies).
%short For example, consider the concurrent game {\em matching pennies} which consists of one round in which both players simultaneously choose either Heads (0) or Tails (1), and \PO wins iff the XOR of the players' choices is 1. Matching pennies is not determined since, clearly, no player can ensure winning by revealing their choice before the other player. 

\stam{
We focus on two tie-breaking mechanisms. First, we consider {\em advantage-based} tie breaking, which was defined in~\cite{DP10} under Richman discrete-bidding, where determinacy was shown for reachability objectives. For infinite-duration objectives, determinacy was obtained in~\cite{AAH21}, and improved algorithms were shown in~\cite{AS22}. Given these positive results, we find it surprising that poorman discrete-bidding games under advantage based tie-breaking are not determined! That is, we show a game and an initial configuration from which neither player has a pure winning strategy. 
}

%In particular to discrete bidding games, unlike continuous bidding games, determinacy depends on the tie-breaking considered. 
%For example, Richman discrete bidding games are not determined for \emph{alternating} tie-breaking~\cite{AAH21}. 
The mechanism that we consider, and on which we focus throughout the rest of the paper, breaks ties in favor of \PO. We show that this mechanism admits determinacy, as is the case in Richman bidding~\cite{AAH21}. %Beyond the theoretical appeal, we argue that this mechanism addresses the inherent imbalance between the players: \PO needs to reach the target whereas \PT only needs to avoid it. 
}


\paragraph{Threshold budgets in DAGs.}
In continuous bidding, each vertex $v$ is associated with a {\em threshold ratio} which is a value $t \geq 0$ such that when the ratio between the two players' budgets is $t+\epsilon$, \PO wins, and when the ratio is $t-\epsilon$, \PT wins~\cite{LLPSU99}. 

First, we bound the discrete thresholds based on continuous ratios as follows. Let $t_v$ denote the continuous ratio at a vertex~$v$. Then, for every $B_2 \in \Nat$, we show that $T_v(B_2)$ lies in the {\em pipe}: $(B_2 - n) \cdot t_v \leq T_v(B_2) \leq B_2 \cdot t_v$, where $n$ is the number of vertices in the game.
We point out that the width of the pipe is fixed, so for large budgets $B_2$ the value $T_v(B_2)/B_2$ tends to the threshold ratio $t_v$.
%the larger the budgets, the better the approximation ratio that we can guarantee. 
%We point out that the relative width of the pipe tends to $0$ as the budgets grow, that is, the larger the budgets, the better the approximation that we can guarantee. 

Second, we show that threshold budgets in DAGs exhibit a periodic behavior. While we view this as a positive result, it has a negative angle: The periods are surprisingly complex even for fairly simple games, so even though we identify a compact representation for the thresholds in Example~\ref{ex:TOW}, we do not expect a compact representation in general games. 

Third, in continuous-bidding games, the compact representation of the thresholds (i.e., each vertex being associated with a ratio) is the key to obtaining a linear-time backwards-inductive algorithm to compute thresholds in DAGs. Under discrete bidding, given a \PT budget $B_2$, we present a pseudo-linear algorithm to find $T(B_2)$, namely its running time is linear in the size of the game and in $B_2$.

Fourth, we obtain closed-form solutions for a class of games called {\em race games}: for $a,b \in \Nat$, the race game $\race{a,b}$ ends within $a+b$ turns, \PO wins the game if he wins $a$ biddings before \PT wins $b$ biddings. For example, a ``best of 7'' tournament (as in the NBA playoffs) is $\race{4,4}$.

\stam{
{\bf Periodicity of the threshold function.} We show that every vertex $v$ in a game played on a DAG, there are values $u_x,u_y \in \Nat$ such that for a large enough budget $B_2$, we have $T(B_2 + u_x) = T(B_2) + u_y$. This interesting theoretical finding has a useful practical application: in order to compute $T(B_2)$, one needs to compute the thresholds for small initial budgets and extrapolate to larger initial budgets. (TODO: check and correct this. Do we need to add assumptions like different continuous thresholds?)

{\bf Closed-form solutions.} We identify closed-form solutions to some games. First, beyond the solution we describe in Example~\ref{ex:TOW}, we show a closed-form solution for a {\em tug-of-war} game (a game played on an undirected path) with three internal vertices, and show experimental evidence that closed-form solutions are unlikely for games beyond that. Second, for $a,b \in \Nat$, the {\em race game} $\race{a,b}$ (e.g.,~\cite{}) ends within $a+b$ turns, \PO wins the game if he wins $a$ biddings before \PT wins $b$ biddings. For example, a ``best out of 7'' tournament (as in the NBA playoffs) is $\race{4,4}$. We show that the threshold in $\race{a,b}$ is $....$.  
}
% Figure environment removed

\begin{example}
\label{ex:2race+1}
\normalfont
We illustrate some of our main results. In Fig.~\ref{fig:intro-race}, we depict the threshold budgets in three vertices of a game %, the root and its two children $v_1$ and $v_2$,
as a function of \PT's budget. %First, note that the $r$ is stable: upon winning a bidding at $r$, \PO will always prefer to proceed to $v_2$ since intuitively, he needs less budget to win from there. 
First, the discrete thresholds reside in a ``pipe'' %below the line 
with slope equal to the corresponding continuous ratio (\cref{stm:pipe}).  Second, $v_1$ and $v_2$ are roots of race games, thus their thresholds are simple step functions (\cref{thm:thresholdrace}).
%their thresholds are step-wise as we prove theoretically,
Moreover, they lie exactly on the boundary of the pipe infinitely often, i.e.\ the pipe bound is tight (\cref{stm:pipe_tight}).
Third, the threshold budgets are periodic (\cref{thm:periodicity}), we have $T_r(B_2 + 45) = T_r(B_2) + 32$. We find it surprising that in such a simple game both the periodicity in the root node and the irregularity within this period are comparatively large.
\end{example}
%
\paragraph{Implementation and Experiments.} 
We provide a pseudo-polynomial algorithm to find the threshold budget given the initial budget of \PT in general games together with a specialized, faster variant for DAGs. We implement the algorithm, experiment with it, and develop conjectures based on our findings. Beyond the theoretical interest, the running time we observed is extremely fast, illustrating the practicality of finding exact thresholds. %We experiment with the algorithm, show experimental validation for our theoretical findings, and draw conjectures on poorman discrete-bidding games both for games played on DAGs and on general graphs. 



%\subsection{Related work}
%Infinite-duration bidding games

%Colonel Blotto

%Tournaments ???







\section{Preliminaries}
	A reachability bidding game is \(\G  = \zug{V, E, t, s}\), where \(V\) is the set of vertices, \(E \subseteq V \times V\) is the set of edges, \PO's {\em target} is $t \in V$, a {\em sink} $s \in V$ has no path to $t$ and we think of $s$ as \PT's target, we assume that all other vertices have a path to both $t$ and $s$.
	We write $N(v) = \{u \mid (v, u) \in E\}$ to denote the \emph{neighbours} of $v$.
	
	A {\em configuration} of \(\calG\) is of the form \(c = \zug{v, B_1, B_2}\), where \(v \in V\) is the vertex on which the token is placed and $B_i$ is the budget of \PLi, for $i \in \set{1, 2}$. At $c$, both players simultaneously choose {\em actions}, and the pair of actions determines the next configuration. For $i \in \set{1,2}$, \PLi's action is a pair $\zug{b_i, u_i}$, where $b_i \leq B_i$ is an {\em integer} bid that does not exceed the available budget and $u_i \in N(v)$ is a neighbor of $v$ to move to upon winning the bidding. If $b_1 \geq b_2$, then \PO moves the token and pays ``the bank'', thus the next configuration is $\zug{u_1, B_1 - b_1, B_2}$. Dually, when $b_2 > b_1$, the next configuration is $\zug{u_2, B_1, B_2 - b_2}$. 
%	When $b_1 = b_2$, the winner of the bidding is determined according to $\M$. We focus on the following two tie-breaking mechanisms:
%	\begin{itemize}
%	\item {\bf Advantage-based~\cite{DP10}:} One of the players has the {\em advantage}, which we denote by $*$. Consider a configuration $\zug{v, B_1^*, B_2}$ in which \PO has the advantage (the definition for \PT is dual), let $\zug{u_i, b_i}$ be \PLi's action, for $i \in \set{1, 2}$, and assume $b_1 = b_2$. Then, \PO chooses between
%	(1) winning the bidding and passing the advantage to the other player,
%%	(1) use the advantage by winning the bidding then the advantage changes hands,
%	thus the next configuration is $\zug{u_1, B_1 - b_1, B_2^*}$; or
%	(2) losing the bidding and keeping the advantage,
%%	(2) keep the advantage by losing the bidding, 
%	thus the next configuration is $\zug{u_2, B_1^*, B_2 - b_2}$. 
%	\item {\bf \PO always wins ties:} 
%	Formally, let \(\zug{u_i, b_i}\) be \PLi's action for \(i \in \{1, 2\}\) and \(b_1 = b_2\), then next configuration is \(\zug{u_1, B_1 - b_1, B_2}\).
%	\end{itemize}
%	
	
	A {\em strategy} is a function that maps each configuration to an action.\footnote{In full generality, strategies map {\em histories} of configurations to actions. However, {\em positional} strategies suffice for reachability games.} A pair of strategies $\sigma_1$, $\sigma_2$, and an initial configuration $c_0$ gives rise to a unique {\em play} denoted by $\play(c_0, \sigma_1, \sigma_2)$, which is defined inductively. The inductive step, namely the definition of how a configuration is updated given two actions from the strategies, is described above. Let $\play(c_0, \sigma_1, \sigma_2) = c_0, c_1, \ldots$, where $c_i = \zug{v_i, B^i_1, B^i_2}$. The {\em path} that corresponds to $\play(c_0, \sigma_1, \sigma_2)$ is $v_0,v_1,\ldots$
	
	

\begin{definition}[Winning Strategies]
A \PO strategy \(\sigma_1\) is called a \emph{winning strategy} from configuration \(c_0\) iff  for any \PT strategy \(\sigma_2\),  \(\play(c_0, \sigma_1, \sigma_2)\) visits the target $t$. 
On the other hand, a \PT strategy \(\sigma_2\) is a winning strategy from \(c_0\) iff for any \PO strategy \(\sigma_1\), \(\play(c_0, \sigma_1, \sigma_2)\) \emph{does not visit} the target \(t\).
For $i \in \set{1, 2}$, we say that \PLi \ {\em wins} from $c_0$ if he has a winning strategy from $c_0$. 
\end{definition}

Throughout the paper, we focus on the necessary and sufficient budget that \PO needs for winning, given a \PT budget, defined formally as follows. 

\begin{definition}[Threshold budgets]
Consider a vertex $v \in V$. The {\em threshold budget} at $v$ is a function $T_v: \Nat \rightarrow \Nat$ such that for every $B_2 \in \Nat$:
\begin{itemize}
\item \PO wins from $\zug{v, T_v(B_2), B_2}$, and 
\item \PT wins from $\zug{v, T_v(B_2) -1, B_2}$. 
\end{itemize}
\end{definition}




\stam{ %OLD
	A discrete-bidding games \(\calG\) is \(\zug{V, E, B_1, B_2, \calO}\), where \(V\) is the set of vertices, \(E \subseteq V \times V\) is the set of edges, and \(B_1, B_2 \in \bbN^*\) are the two players' initial budget respectively, and \(\calO\) denotes the \PO's objective in the game.
%	We often denote the total budget \(B_1 + B_2\) by \(B\), where \([B]\) denotes \(\{0, 1, \ldots B\}\).
	
	Intuitively, on each turn, both players simultaneously choose a bid which does not exceed their respective budgets. 
	The higher bidder moves the token.
	We study \emph{Poorman} bidding, where the bidder(s) pays the \emph{bank}.
	In \emph{first-price} Poorman bidding, only the higher bidder pays the bank, while in case of \emph{all-pay} Poorman bidding \emph{both} players pays their respective bids to the bank.
	
	We consider two types of tie-breaking mechanism: 
	\begin{itemize}
		\item \textbf{One player wins ties.} One of the player always wins ties. If the objectives are not symmetric, this may raise two different interesting scenarios to investigate.
		
		\item \textbf{Advantage based.} Exactly one of the players holds the advantage at any turn. 
		When a tie occurs, the player who holds the advantage has two choices: \emph{either} to use it, which makes him the winner of the bid and the advantage passed on to the other player,  \emph{or} to hold on to it, which makes the other player the winner of the current bid. 
	\end{itemize}

	For the subsequent sections, we consider the \emph{\PO wins ties} as the default tie-breaking mechanism, unless otherwise mentioned.
	
	A configuration in \(\calG\) is of the form \(\zug{v, k, l}\) where \(v\) is the current location in the game graph, and \(k, l\) are the budgets for \PO and \PT respectively.
	From a configuration \(c  = \zug{v, k, l}\), if \PO bids \(b_1 \leq k\) and \PT bids \(b_2 \leq l\), then 
	\begin{itemize}
		\item  For first-price bidding, if \(b_1 \geq b_2\) and suppose \PO chooses \(v'\), then the next configuration is \(\zug{v', k - b_1, l}\).
		Otherwise suppose \PT chooses \(v''\) upon winning the bidding, then the next configuration in the game is \(\zug{v'', k, l - b_2}\).
		
		\item For all play bidding, if \(b_1 \geq b_2\) and suppose \PO chooses \(v'\), then the next configuration is \(\zug{v', k - b_1, l - b_2}\).
		Otherwise, suppose \PT choose \(v''\) upon winning the bidding, then the next configuration in the game is \(\zug{v'', k - b_1, l - b_2}\). 
	\end{itemize}
	A (finite) history of discrete bidding games is a sequence of configurations \(\zug{v_1, k_1, l_1}, \zug{v_2, k_2, l_2}, \ldots \zug{v_n, k_n, l_n}\), where the subsequent configuration of \(\zug{v_{i}, k_i, l_i}\) is determined depending on each player's bid as above.
	For the current purpose, we consider reachability objective for \PO, denoted by \(\calO \subseteq V\), and we call a play \(\pi = \zug{v_j, k_j, l_j}_{j \geq 0}\) \emph{satisfies} \(\calO\) iff there is an \(j \geq 0\) such that \(v_j \in \calO\).
	
	A strategy for $\PLi$ for \(i \in \set{1, 2}\) is function from the set of legal histories to a choice of bid and next vertex, of the form \(\sigma: (V \times \bbN \times \bbN)^* \rightarrow \bbN \times V\).
	It is imperative that a player can choose their bid only within their current budget.
	Once we fix strategies \(\sigma_i\) for \(i \in \{1, 2\}\) and an initial configuration \(\zug{v_0, k_0, l_0}\), we get an unique play.
	
	
\begin{definition}[Surely Winning Strategies]
		A \PO strategy \(\sigma_1\) is called a \emph{surely winning strategy} from configuration \(\zug{v, k, l}\) iff  for any \PT strategy \(\sigma_2\), the unique play \(\zug{\zug{v, k, l}, \sigma_1, \sigma_2}\) satisfies \(\calO\).
\end{definition}
\todo{Semantics of bidding games as concurrent games}
}

\section{Existence of Thresholds}\label{sec:determinacy}
In this section we show the existence of threshold budgets in games played on general graphs. 
\begin{definition}{\bf (Determinacy).} 
A game is {\em determined} if from every configuration, one of the players has a pure winning strategy.
\end{definition}

We claim that determinacy is equivalent to existence of thresholds. It is not hard to deduce both implications from the following observation. An additional budget cannot harm a player; namely, if \PO wins from a configuration $\zug{v, B_1, B_2}$, he also wins from $\zug{v, B'_1, B_2}$, for $B'_1 > B_1$, and dually for \PT. 
%	On the other hand, when threshold budget exists for every vertex in the game, \PO wins from any configuration of the form \(\zug{v, B_1, B_2}\) when \(B_1 \geq T_v(B_2)\), and \PT wins when \(B_1 < T_v(B_2)\), thus covering all configurations.  

%\subsection{Advantage-based tie-breaking}
%Given the positive results on Richman discrete-bidding games under advantage-based tie-breaking~\cite{DP10,AAH21,AS22}, we find the following negative result surprising.
%
%\begin{theorem}
%	Poorman discrete-bidding games under advantage based tie-breaking are not determined.
%\end{theorem}
%\begin{proof}
%	% Figure environment removed
%	Consider the game that is depicted in Fig.~\ref{fig:notdetermined} in which \PO needs to win two consecutive biddings. We claim that neither player wins from configuration \(c_0 = \langle v_0, 2^*, 1 \rangle\). We assume that at the time of bidding, the player with the advantage decides whether the advantage will be used; namely, his bid is either $b^*$, i.e.\ use the advantage in case of a tie, or $b$, do not use it. Intuitively, in each turn, a winning strategy reveals a bid and lets the other player respond to it. In the following, we show that no matter which bid is revealed in $c_0$, the other player has a winning response. 
%	
%	\textit{\PO cannot win}.
%	If \PO bids $0$, $0^*$, or $1$, \PT responds by winning the bidding and proceeding to $s$. If \PO bids $1^*$ (and the case of higher bids is similar), \PT responds with \(1\), and we proceed to \(\zug{v_1, 1, 1^*}\) from which \PT wins by bidding $1^*$.
%	
%	\textit{\PT cannot win}.
%	If \PT bids \(1\), \PO bids \(1^*\), and we proceed to \(\zug{v_1, 1, 0^*}\). If \PT bids \(0\), \PO bids \(0^*\), and we proceed to \(\zug{v_1, 2, 1^*}\). In both cases \PO can guarantee winning the next bidding.
%\end{proof}






%\subsection{\PO always wins ties}
%We report on the determinacy of poorman discrete-bidding games when \PO always wins ties. 

%In the rest of this section, we show the following - 


In the rest of this section, we prove determinacy of poorman discrete-bidding games. Our proof is based on a technique that was developed in \cite{AAH21} to show determinacy of \emph{Richman} discrete-bidding. We illustrate the key ideas. 
% under {\em transducer-based tie breaking}, which is more general than ``\PO always wins ties''. 
Consider a reachability bidding game \(\G = \zug{V, E, t, s}\) 
%where \PO always wins ties, 
and a configuration \(c = \zug{v, B_1, B_2}\). 
%We show that if \PO does not win from $c$, then \PT wins from $c$. 
We define a {\em bidding matrix} $M_c$ that corresponds to $c$. For $\zug{b_1, b_2} \in \set{0,\ldots, B_1} \times \set{0,\ldots, B_2}$, the \((b_1, b_2)^{\text{th}}\) entry in $M_c$ is associated with \PLi bidding $b_i$, for $i \in \set{1,2}$. We label entries in $M_c$ by $1$ or $2$ as follows. 
Let $\G_1$ denote a turn-based game that is the same as $\G$ only that in each turn, \PO reveals his bid first and \PT responds. Technically, once both players reveal their bids, the game proceeds to an intermediate vertex $i_{b_1, b_2} = \zug{b_1, b_2, c}$. Since $G_1$ is turn-based, it is determined, thus one of the players has a winning strategy from $i_{b_1, b_2}$. We label the \((b_1, b_2)^{\text{th}}\) entry in $M_c$ by $i \in \set{1,2}$ iff \PLi wins from $i_{b_1, b_2}$. For \(i \in \{1, 2\}\), we call a row or a column of \(M_c\) a \(i\)\emph{-row} or \(i\)\emph{-column}, respectively, if all its entries are labeled \(i\).


%intuitively captures the winner of the game with respect to each pair of bids by the two players at $c$. Formally, $M_c$ has $B_1$ rows and $B_2$ columns, each row corresponds to a \PO bid, and each column to a \PT bid. An entry in \(M_c\) is either \(1\) or \(2\), defined as follows. Let $\zug{b_1, b_2} \in \set{0,\ldots, B_1} \times \set{0,\ldots, B_2}$ and let $\zug{B'_1, B'_2}$ be the updated budgets after the players bid $\zug{b_1, b_2}$ at $c$. We set the \((b_1, b_2)^{\text{th}}\) entry in $M_c$ to be $1$ iff (1) either $b_1 \geq b_2$ and there exists a neighbor $v'$ of $v$ such that \PO wins from configuration $\zug{v', B'_1, B'_2}$, or (2) $b_2 > b_1$ and \PO wins from every configuration $\zug{v', B'_1, B'_2}$, for a neighbor $v'$ of $v$. For \(i \in \{1, 2\}\), we call a row or a column of \(M_c\) a \(i\)\emph{-row} or \(i\)\emph{-column}, respectively, if all its entries are \(i\).

\begin{definition}{\bf (Local Determinacy)}
	A bidding game $\G$ is called \emph{locally determined} if for every configuration $c$, the bidding matrix $M_c$ either has a $1$-row or a $2$-column. 
\end{definition}

Local determinacy is used as follows. 
It can be shown that if \PO wins from $c$, then $M_c$ has a $1$-row. 
%Consider a locally-determined game and a configuration $c$. If \PO wins from $c$, then $M_c$ necessarily has a $1$-row; indeed, a winning \PO strategy guarantees that even if \PO reveals his bid at $c$, the next configuration will be winning for \PO. 
More importantly, suppose that \PO does not win in $c$, then local determinacy implies that there is a $2$-column, say $b_2$. This means that when \PT bids $b_2$ in $\G$, the game proceeds to a configuration $c'$ from which \PO does not win. 
%can reveal her bid and force the game to a configuration that is not winning for \PO.
In reachability games, since \PT's goal is to avoid the target, traversing non-losing configurations for \PT is in fact winning.\footnote{The theorem is stated for reachability objectives and it is extended in~\cite{AAH21} to richer objectives.}

\begin{lemma}{\bf (\cite[Theorem 3.5]{AAH21})}\label{lem:locallydetermined}
	If a reachability bidding game \(\G\) is locally determined, then \(\G\) is determined.
\end{lemma}

Local determinacy of poorman discrete-bidding games follows from the following observations on bidding matrices.  
%We make the following three observations regarding the bidding matrices.
\begin{restatable}{lemma}{observations}
	\label{lem:observations}
	Consider a poorman discrete-bidding game \(\G\) where \PO always wins tie, and consider a configuration \(c = \langle v, B_1, B_2 \rangle \).			(1) Entries in \(M_c\) in a column above the top-left to bottom-right diagonal are equal: for bids \(b_2 > b_1 > b_1'\), we have $M_c[b_1, b_2] = M_c[b_1', b_2]$. (2) Entries on a row, left of the diagonal are equal: for bids \(b_1 > b_2 > b_2'\), we have $M_c[b_1, b_2] = M_c[b_1, b_2']$. (3) The entry immediately under the diagonal equals the entry on the diagonal: For a bid $b$, we have $M_c[b, b]= M_c[b, b-1]$.
\end{restatable}


\begin{proof}
	If \(b_2 > b_1 > b_1'\) then \PT wins the current bidding for both pair of bids \(\zug{b_1, b_2}\) and \(\zug{b_1', b_2}\). 
	Thus \PT controls the corresponding intermediate vertex, and moves the token as per her choice. 
	As a result, only \PT's budget gets decreased by \(b_2\).
	Therefore, all the transitions that are available from \(\zug{c, b_1, b_2}\) are also available from \(\zug{c,b_1', b_2}\), and vice-versa.
	In other words, whoever wins from \(\zug{c, b_1, b_2}\) also wins from \(\zug{c, b_1', b_2}\), hence the entries are same. 
	The argument is similar when \(b_1 > b_2 > b_2'\).
	
	For the third observation, the tie-breaking mechanism, \PO always wins tie, plays the key role.
	For both the cases: when the bids are \(\zug{b, b}\), and when it is \(\zug{b, b -1}\) from a configuration \(c\), \PO wins the current bidding. 
	As a result the token moves according to his choice, his budget gets decreased by \(b\), while \PT's budget remains unchanged. 
	Therefore, all the available transitions from the \PO controlled vertex \(\zug{c, b, b}\) and \(\zug{c, b, b-1}\) are the same, and whoever wins from one, also wins from the other. 
	Thus the entries in \(M_c\) are same.
\end{proof}

%<<<<<<< HEAD
%\begin{proof}[Proof sketch]
%	(See the supplementary material for the full proof)
%	
%	The main argument is that because in poorman bidding, only the winner's budget alters after a bidding, it does not matter what the other player bids as long as his bid remains a losing bid. 
%%	The full technical proof can be found in the supplementary material. 
%\end{proof}
%=======
%%guy: this is too vague. It's better not to include it at all, I think. 
%%\begin{proof}[Proof sketch]
%%	(See the supplementary material for the full proof)
%%	The main argument is that since under poorman bidding, only the winner's budget is altered after a bidding, it does not matter what the other player bids as long as his bid remains a losing bid. 
%%\end{proof}
%>>>>>>> 9cd46c38b89cf01f4a44f1e94e2a1d262765acb9


The proof of \cite[Theorem 4.5]{AAH21} shows that a game whose bidding matrices have the properties of~\cref{lem:observations} is locally determined, irrespective of whether Richman or poorman bidding is employed. Combining with~\cref{lem:locallydetermined}, we obtain the following. %have \cref{thm:determinacy}. 

\begin{restatable}{theorem}{determinacy}\label{thm:determinacy}
	Reachability poorman discrete-bidding games are determined.
\end{restatable}






%In this section, we establish the determinacy of poorman discrete-bidding games when \PO always wins ties. Our proof is based on \cite{AAH21}, which proves determinacy of Richman discrete-bidding under a tie-breaking mechanism called {\em transducer-based tie breaking}, which is more general than \PO wins ties. 
%
%Consider a reachability bidding game \(\G = \zug{V, E, t, s}\) where \PO always wins ties, and consider a configuration \(c = \zug{v, B_1, B_2}\). We show that if \PO does not win from $c$, then \PT wins from $c$. The {\em bidding matrix} $M_c$ intuitively captures the winner of the game with respect to each pair of bids by the two players at $c$. Formally, $M_c$ has $B_1$ rows and $B_2$ columns, each row corresponds to a \PO bid, and each column to a \PT bid. An entry in \(M_c\) is either \(1\) or \(2\), defined as follows. Let $\zug{b_1, b_2} \in \set{0,\ldots, B_1} \times \set{0,\ldots, B_2}$ and let $\zug{B'_1, B'_2}$ be the updated bids after the players bid $\zug{b_1, b_2}$ at $c$. We set the \((b_1, b_2)^{\text{th}}\) entry in $M_c$ to be $1$ iff (1) either $b_1 \geq b_2$ and there exists a neighbor $v'$ of $v$ such that \PO wins from configuration $\zug{v', B'_1, B'_2}$, or (2) $b_2 > b_1$ and \PO wins from every configuration $\zug{v', B'_1, B'_2}$, for a neighbor $v'$ of $v$. For \(i \in \{1, 2\}\), we call a row or a column of \(M_c\) a \(i\)\emph{-row} or \(i\)\emph{-column}, respectively, if all its entries are \(i\).
%
%\begin{definition}{\bf (Local Determinacy)}
%A bidding game $\G$ is called \emph{locally determined} if for every configuration $c$, the bidding matrix $M_c$ either has a $1$-column or a $2$-row. 
%\end{definition}
%
%Consider a locally-determined game and a configuration $c$. If \PO wins from $c$, then $M_c$ necessarily has a $1$-row; indeed, a winning \PO strategy guarantees that even if \PO reveals his bid at $c$, the next configuration will be winning for \PO. More importantly, if \PO does not win in $c$, local determinacy implies that \PT can reveal her bid and force the game to a configuration that is not winning for \PO. In reachability games, since \PT's goal is to avoid the target, traversing non-losing configurations for \PT is in fact winning. 
%
%\begin{lemma}\label{lem:locallydetermined}\cite{AAH21}
%If a reachability bidding game \(\G\) is locally determined, then \(\G\) is determined.
%\end{lemma}
%
%We make the following three observations regarding the bidding matrices (see the supplementary material for the proof). 
%%The main idea, which is same as \cite[lemma 4.3 and 4.4]{AAH21}, is that the subsequent configuration from a configuration only depends on the winner and the winning bid for our current bidding mechanism (first-price, poorman, discrete).  
%
%	\begin{restatable}{lemma}{observations}
%		\label{lem:observations}
%		Consider a poorman discrete-bidding game \(\G\) where \PO always wins tie, and consider a configuration \(c = \langle v, B_1, B_2 \rangle \).			(1) Entries in \(M_c\) in a column above the top-left to bottom-right diagonal are equal: for bids \(b_2 > b_1 > b_1'\), we have $M_c[b_1, b_2] = M_c[b_1', b_2]$. (2) Entries on a row, left of the diagonal are equal: for bids \(b_1 > b_2 > b_2'\), we have $M_c[b_1, b_2] = M_c[b_1, b_2']$. (3) The entry immediately under the diagonal equals the entry on the diagonal: For a bid $b$, we have $M_c[b, b]= M_c[b, b-1]$.
%	\end{restatable}
%
%	
%The proof of \cite[Theorem 4.5]{AAH21} shows that a game whose bidding matrices have the properties of Lem.~\ref{lem:observations} is locally determined, irrespective of whether Richman or poorman bidding is employed. Combining with Lem.~\ref{lem:locallydetermined}, we have the following. 
%
%	\begin{restatable}{theorem}{determinacy}\label{thm:determinacy}
%		Reachability poorman discrete-bidding games are determined when \PO always wins ties.
%	\end{restatable}













%%%%%%%%%%%%%%%%%%%%%%%%%%%
	
	


	





\section{Threshold Budgets for Games on DAGs}\label{sec:pipeandstabilization}
In this section, we focus on games played on directed acyclic graphs (DAGs). We present two main results:
First, the \textit{Pipe theorem} that relates the threshold budgets to the threshold ratio in the continuous-bidding game; and,
second, the \textit{Periodicity theorem} which shows that the threshold budgets eventually exhibit a periodic behavior.
Throughout this section, let \(\calG = \zug{V, E, t, s}\) be a game with $\zug{V,E}$ a DAG.
%Namely, given the threshold values $t_1\le t_2\le\dots\le t_k$ at all descendants of a fixed node, the threshold value $t$ at that node can be computed by solving a single linear equation
%$t= t_k(t_1+1)/(t_k+1)$.

\subsection{Relating Discrete and Continuous Thresholds}
We call the following theorem the \textit{Pipe theorem} since it shows that the threshold budgets $T_v(B_2)$ lie in a ``pipe'' below a line whose slope is the threshold ratio $t_v$ (see Example~\ref{ex:2race+1}). We note that threshold ratios can be computed in DAGs in time polynomial in the size of the game (a fact we also exploit later on in our algorithm on DAGs), thus an immediate corollary of the Pipe theorem is an efficient approximation algorithm to computing the threshold budgets. In Corollary~\ref{stm:pipe_tight}, we show that the lower bound is tight. For a vertex $v$, let {\em $\textnormal{max-path}(v)$} denote the length of the longest path from $v$ to either $t$ or $s$. Note that $\textnormal{max-path}(v) \leq |V| - 1$. 

\begin{restatable}[Pipe theorem]{theorem}{pipe} \label{stm:pipe}
	Given $v\in V$, denote by $t_v$ the threshold ratio in the continuous-bidding game at $v$. Then, for every initial budget $B_2\in\mathbb{N}$ of \PT, we have
	\begin{equation*}
		t_v \cdot (1- \textnormal{max-path}(v) / B_2)  \leq T_v(B_2) / B_2 \leq t_v.
	\end{equation*}
	The right-hand side inequality holds even when $\G$ is not a DAG.
\end{restatable}

%\begin{proof}[Proof sketch] (See supplementary material for the full proof.)
%%	In what follows, we outline the key ideas behind the proof. The full proof can be found in the supplementary material.
%	To prove the right-hand side inequality, we show that if \PO has initial budget of at least $t_v \cdot B_2$ then \PO can win by following a winning strategy in the {\em continuous-bidding} game and {\em rounding down} the bids. More formally, let $\cstrat$ be a winning strategy for \PO under continuous-bidding when the game starts in $v$, \PO's initial budget is at least $t_v \cdot B_2$, and \PT's initial budget is $B_2$. We define a \PO strategy $\dstrat$ as follows. Whenever $\cstrat$ prescribes a pair $\zug{b, u}$, where $b$ is a bid and $u \in V$ is the vertex to move to upon winning, $\dstrat$ prescribes $\zug{\lfloor b\rfloor, u}$. 
%	
%To prove the left-hand side inequality, we show that if \PO has initial budget strictly less than $t_v \cdot (B_2-\textnormal{max-path}(v))$ and \PT has initial budget $B_2$, then \PT can win by following the winning strategy in a {\em continuous-bidding} game and rounding the bids up. More formally, let $\cstrat$ be a winning strategy for \PT under continuous-bidding when the game starts in $v$, \PO's initial budget is at most $t_v \cdot (B_2-\textnormal{max-path}(v))$  and \PT's initial budget is $B_2 - \textnormal{max-path}(v)$. Suppose that $\cstrat$ prescribes $\zug{b, u}$, then $\dstrat$ for \PT prescribes $\zug{\lceil b\rceil, u}$. 
%The fact that \PT always has enough budget to bid $\lceil b\rceil$ follows from the fact that the game necessarily ends within $\textnormal{max-path}(v)$ turns.
%\end{proof}

\begin{proof}
	{\em Right-hand-side inequality.} We first prove the right-hand-side inequality. To prove that $T_v(B_2) / B_2 \leq t_v$, it suffices to prove that, for each $\epsilon > 0$, \PO has a winning strategy if the game starts in $v$, \PO's initial budget is at least $t_v\cdot B_2 + \epsilon$ and \PT's initial budget is $B_2$. If we are able to prove this claim, it will then follow that $T_v(B_2) / B_2 \leq t_v + \epsilon$ holds for every $\epsilon>0$, therefore $T_v(B_2) / B_2 \leq t_v$.
	
	Fix $\epsilon > 0$. We construct the winning strategy of \PO as follows. By the definition of the continuous threshold $t_v$, we know that \PO has a winning strategy in the poorman {\em continuous-bidding} game. Moreover, it was shown in~\cite[Theorem 7]{LLPSU99} that \PO has a {\em memoryless} winning strategy, i.e.~a strategy in which the bids and token moves in each turn depend only on the position of the token and the players' budgets. We take such strategy $\sigma_{\textrm{cont}}$. We then construct a winning strategy $\sigma_{\textrm{disc}}$ of \PO in the poorman discrete-bidding game as follows:
	\begin{itemize}
		\item At each turn, if \PO under $\sigma_{\textrm{cont}}$ would bid $b$, then \PO under $\sigma_{\textrm{disc}}$ bids $\lfloor b \rfloor$.
		\item If \PO wins the bidding, then the token is moved to the vertex perscribed by $\sigma_{\textrm{cont}}$. 
	\end{itemize}
	
	We show that $\sigma_{\text{disc}}$ is indeed winning for \PO. To do this, we prove that $\sigma_{\textrm{disc}}$ preserves the invariant that, whenever the token is in some vertex $v'$, the ratio of players' budgets is positive and strictly greater than $t_{v'}$. This invariant implies that the token does not reach the sink state as the continuous threshold in the sink state is infinite. Thus, as a poorman discrete-bidding game ends in finitely many steps, this then implies that the game must eventually reach \PO's target state and therefore that $\sigma_{\text{disc}}$ is winning for \PO.
	
	We prove the invariant by the induction on the length of the game play. The base case holds by the assumption that, in the initial vertex $v$, \PO's initial budget is at least $t_v\cdot B_2 + \epsilon$ and \PT's initial budget is $B_2$. Now, for the induction hypothesis, suppose that the token is in vertex $v'$ after finitely many steps, with \PT's budget $B_2'$ and \PO's budget at least $t_{v'} \cdot B_2' + \epsilon'$ for some $\epsilon'>0$. We show that the invariant is preserved in the next step. Suppose that \PO under $\sigma_{\textrm{disc}}$ bids $\lfloor b \rfloor$ where $b$ is the bid of \PO under $\sigma_{\textrm{cont}}$. In what follows, we use the fact that $\sigma_{\textrm{cont}}$ preserves the ratio invariant which was established in the proof of~\cite[Theorem 7]{LLPSU99}. We distinguish between two cases:
	\begin{itemize}
		\item If \PO wins the bidding and moves the token to $v''$, then the ratio of budgets at the next step is
		\begin{equation*}
			\begin{split}
				\frac{t_{v'} \cdot B_2' + \epsilon' - \lfloor b\rfloor}{B_2'} &\geq \frac{t_{v'} \cdot B_2' + \epsilon' - b}{B_2'} \\
				&> \frac{t_{v'} \cdot B_2' - b}{B_2'} \\
				&\geq t_{v''}.
			\end{split}
		\end{equation*}
		The last inequality follows from the fact that the fraction in the second line is the subsequent ratio of budgets under the continuous-bidding winning strategy.
		\item If \PT wins the bidding, then \PT had to bid at least $\lfloor b \rfloor + 1$. Suppose that \PT moves the token to $v''$ upon winning. Then the ratio of budgets at the next step is at least
		\begin{equation*}
			\begin{split}
				\frac{t_{v'} \cdot B_2' + \epsilon'}{B_2' - \lfloor b\rfloor - 1 } &\geq \frac{t_{v'} \cdot B_2' + \epsilon'}{B_2' - b} \\
				&> \frac{t_{v'} \cdot B_2'}{B_2' - b} \\
				&\geq t_{v''}.
			\end{split}
		\end{equation*}
		The first inequality follow by observing that $\lfloor b \rfloor + 1 \geq  b$, and the third inequality follows from the fact that the second fraction is an upper bound on the subsequent ratio of budgets under the continuous-bidding winning strategy.
	\end{itemize}
	Hence, the invariant claim for poorman discrete-bidding follows by induction on the length of the game play, thus $\sigma_{\text{disc}}$ is indeed winning for \PO and the right-hand-side inequality in the theorem follows.
	
	\noindent {\em Left-hand-side inequality.} We now prove the left-hand-side inequality. If $B_2 < \textrm{max-path}(v) $, the claim trivially follows. Otherwise, it suffices to prove that \PT has a winning strategy if the game starts in $v$, \PO's initial budget is strictly less than $t_v\cdot (B_2 - \textrm{max-path}(v))$ and \PT's initial budget is $B_2$.
	
	Suppose that $B_2 \geq \textrm{max-path}(v)$ and let $B_1  < t_v\cdot (B_2 - \textrm{max-path}(v) )$ be the initial budget of \PO. We construct the winning strategy of \PT as follows. Let $\sigma_{\textrm{cont}}$ be the memoryless winning strategy of \PT under {\em continuous-bidding} when the game starts in $v$, \PT's initial budget is $B_2 - \textrm{max-path}(v)$ and \PO's initial budget is $B_1$. Since  $B_1  < t_v\cdot (B_2 - \textrm{max-path}(v))$, such a strategy exists by the definition of the continuous threshold $t_v$ and by~\cite[Theorem 7]{LLPSU99} which shows that it is possible to pick a memoryless winning strategy. We then construct a winning strategy $\sigma_{\textrm{disc}}$ of \PT in the poorman discrete-bidding game when \PT has initial budget $B_2$ and \PO has initial budget $B_1$ as follows:
	\begin{itemize}
		\item At each turn, if \PT under $\sigma_{\textrm{cont}}$ would bid $b$, then \PT under $\sigma_{\textrm{disc}}$ bids $\lceil b \rceil$.
		\item If \PT wins the bidding, then the token is moved to the vertex perscribed by $\sigma_{\textrm{cont}}$. 
	\end{itemize}
	Note that, if we show that the bids $\lceil b \rceil$ under $\sigma_{\textrm{disc}}$ are legal (i.e.~do not exceed available budget), then $\sigma_{\textrm{disc}}$ is clearly winning for \PT. Indeed, $\sigma_{\textrm{cont}}$ is winning for \PT, the bids of $\sigma_{\textrm{disc}}$ are always as least as big as those of $\sigma_{\textrm{cont}}$ and the token moves under two strategies coincide. So we only need to prove that the bids $\lceil b \rceil$ under $\sigma_{\textrm{disc}}$ are legal. But this follows from the fact that the underlying graph is a DAG and thus the game takes at most $\textrm{max-path}(v)$ turns before it reaches either the target or the sink vertex. Hence, as the bids are legal under $\sigma_{\textrm{cont}}$ when \PT has initial budget $B_2 - \textrm{max-path}(v)$, the bids are also legal under $\sigma_{\textrm{disc}}$ as \PT can bid $b + 1 \geq \lceil b \rceil$ in each turn. This concludes the proof of the left-hand-side of the inequality.
\end{proof}



%We show later that these bounds are tight.
An immediate corollary of Thm.~\ref{stm:pipe} is that the ratio $T_v(B_2)/B_2$ tends to $t_v$. %short? the continuous-bidding threshold ratio $t_v$.
%First, the discrete-bidding ratio $T_v(B_2)/B_2$ at every vertex $v$ converges to the continuous threshold $t_v$  at $v$ as $B_2$ tends to infinity, since $\textrm{max-path}(v)/B_2 \leq \frac{|V|-1}{B_2}$ converges to $0$.

\begin{corollary}[Convergence to continuous ratios]\label{cor:convergence}
	For every $v \in V$ we have $\lim_{B_2\rightarrow\infty} T_v(B_2) / B_2 = t_v$.
\end{corollary}



\subsection{Periodicity of Threshold Budgets}
The following theorem shows that for any fixed $v\in V$ the function $T_v(\cdot)$ that yields the threshold budgets exhibits an eventually periodic behavior, as seen in Example~\ref{ex:2race+1}. %we have $B = 0$, $u_x = 45$, and $u_y = 32$.

\begin{restatable}[Periodicity theorem]{theorem}{periodicity} \label{thm:periodicity}
For any vertex $v\in V$ there exist values $B, u_x,u_y \in\Nat$ such that for all $B_2\ge B$ we have $T_v(B_2+u_x)=T_v(B_2)+u_y$.
Moreover, the values $B$, $u_x$, $u_y$ can be computed in polynomial time.
\end{restatable}

\stam{
\begin{proof}[Proof sketch] 
(See supplementary material for the full proof.)
The proof is by induction with respect to the topological order of the graph.
If $v$ is a leaf, then the claim is obvious. % (and we can set $B=0$).
Consider $v$ that is not a leaf.
The proof is based on three ingredients.
First, intuitively, when the children of $v$ have different threshold ratios then their pipes diverge.
Let $v^-$ and $v^+$ respectively denote the children whose pipe is lowest and highest.
By Thm.~\ref{stm:pipe}, under discrete-bidding, for large budgets, \PO and \PT will respectively proceed to $v^-$ and $v^+$ upon winning the bidding in $v$.
%Second, we use the fact that a minimum of two periodic functions is itself periodic to take care of the case that the threshold ratios are not unique. 


Second, we show that if $v^-$ satisfies $T_{v^-}(B_2+u^-_x)=T_{v^-}(B_2)+u^-_y$ and
$v^+$ satisfies $T_{v^+}(B_2+u^+_x)=T_{v^+}(B_2)+u^+_y$ (both for large enough $B_2$),
then $v$ satisfies the same equality with $u_x=u^-_x\cdot (u^+_x+u^+_y)$ and $u_y=u^+_y\cdot(u^-_x+u^-_y)$. We illustrate the idea using~\cref{fig:u-climbing}, which depicts a configuration $c=\zug{v,B_1,B_2}$ as a point $[B_2, B_1]$ in the plane. Consider first the left image. Suppose that \PO bids $b$ from $\zug{v, B_1, B_2}$ (see point $P$). The case that \PO wins the bidding corresponds to ``stepping down'' from $[B_2, B_1]$ to $[B_2, B_1-b]$. Note that the token moves to $v^-$. Thus, a necessary condition for $B_1 \geq T_v(B_2)$ is $B_2 - b \geq T_{v^-}(B_2)$. The second case is when \PT bids $b+1$ and wins the bidding, which corresponds to ``stepping left'' to $[B_2 - (b+1), B_1]$, the token moves to $v^+$, and we obtain a second necessary condition $B_1 \geq T_{v^+}(B_2 - (b+1))$. Then, given configurations on the thresholds of $v^-$ and $v^+$ (depicted as $Q$ and $R$), the ``lowest'' point that satisfies both conditions is a point on the threshold of $v$. The right part of~\cref{fig:u-climbing} shows how the period of $T_v$ is determined by the periods of $T_{v^+}$ and $T_{v^-}$.
% Figure environment removed


Third, if multiple children have the same threshold ratio, we reduce to the previous case by using the fact that a minimum of two periodic functions over integers is itself periodic. %to take care of the case that the threshold ratios are not unique. 
%This completes the induction proof.
\end{proof}}

\begin{proof}
	We proceed by induction with respect to the topological order of the graph.
	For the target $v$ we set $(B,u_x,u_y)=(0,1,0)$ and for the sink we set $(B,u_x,u_y)=(0,0,1)$.
	Next, given a non-leaf vertex $v$, suppose that among its children there are $k$ distinct continuous threshold ratios, and denote them by $t_1<t_2<\dots<t_k$.
	Note that whenever \PO wins a bidding at vertex $v$ (while \PT has budget $B$), he moves the token to a child $u\in N(v)$ of $v$ with minimal value $T_u(B)$.
	We claim that for large enough $B$, the only relevant children $u$ are those with $t_{u}=t_1$.
	Indeed, consider two children $u,w\in N(v)$, one with $t_u=t_1$ and the other one with $t_w\ne t_1$.
	Then by \cref{stm:pipe}, for $B>t_2\cdot n / (t_2-t_1)$ we have
	\[T_w(B) \ge t_w\cdot (B-n) \ge t_2\cdot (B-n) > t_1\cdot B \ge T_u(B),
	\]
	thus \PO would prefer to move to $u$ rather than to $w$.
	
	Next, let $V^-=\{u\in N(v) \mid t_u=t_1\}$ be a set of those ``relevant'' children of $v$.
	We say that a function $f\colon \Nat\to\Nat$ is \textit{$(x,y)$-climbing} if it satisfies $f(B+x)=f(B)+y$ for all large enough $B$.
	By induction assumption, for each $u\in V^-$ the function $T_{u}(B)$ is $(u_x,u_y)$-climbing with a slope $u_y/u_x$. By \cref{stm:pipe}, this slope is equal to $t_1$.
	Thus, the function $T_u(B) - t_1\cdot B$ is periodic with period $u_x$.
	The function $\min_{u\in V^-}\{T_u(B) - t_1\cdot B\}$ is then periodic with the period equal to the least common multiple $l=\operatorname{lcm}\{u_x\mid u\in V^-\}$ of the respective periods.
	Therefore, the function $T_{v^-}(B) := \min_{u\in V^-}\{T_u(B)\}$ is $(l,l\cdot t_1)$-climbing.
	To summarize, the moves of \PO (upon winning a bidding) are faithfully represented by him moving the token to a vertex $v^-$ for which the threshold budgets are $(l,l\cdot t_1)$-climbing.
	Completely analogously we show that the moves of \PT are faithfully represented by her moving the token to a vertex $v^+$ with $(l',l'\cdot t_k)$-climbing threshold budgets.
	
	From now on, for ease of notation suppose that $T_{v^-}$ is $(u_x,u_y)$-climbing and that $T_{v^+}$ is $(w_x,w_y)$-climbing (for some integers $u_x,u_y,w_x,w_y$). We will show that $T_v$ is $(u_x\cdot(w_x+w_y), w_y\cdot(u_x+u_y))$-climbing. This will complete the induction proof.
	
	To prove this claim, it is convenient to represent each configuration $c=\zug{v,B_1,B_2}$ as a point in the plane with coordinates $[B_2,B_1]$, see~\cref{fig:u-climbing}.
	\PO can then force a win from a configuration $c=\zug{v,B_1,B_2}$ if and only if 
	the point $P=[B_2,B_1]$ lies on or above the threshold function~$T_v$.
	
	
	% Figure environment removed
	
	
	Take any point $P=(P_x,P_y)$.
	Let $Q=(Q_x=P_x,Q_y)$ be the furthest point below $P$ that still lies on or above $T_{v^-}$, and
	let $R=(R_x,R_y=P_y)$ be the closest point to the left of $P$ that lies on or above $T_{v^+}$.
	Note that $P$ lies on or above $T_v$ if and only if the distances $d_y:=P_y-Q_y$ and $d_x:=P_x-R_x$ satisfy $d_x\le d_y+1$. Indeed, if the inequality holds then \PO can force a win by bidding $d_y$, whereas in the other case \PT can force a win by bidding $d_y+1$. 
	
	As a final step, we show that at some point further along the curves $T_{v^-}$ and $T_{v^+}$, the two distances $d_x$, $d_y$ increase by the same margin.
	Specifically, chain $u_x+u_y$ copies of a vector $(w_x,w_y)$ starting from point $R$ to get to point $R'=(R'_x,R'_y)$, and, similarly,
	chain $w_x+w_y$ copies of $(u_x,u_y)$ from $Q$ to $Q'=(Q'_x,Q'_y)$.
	Finally, let $P'$ be the point above $Q'$ and to the right of $R'$.
	Then a straightforward algebraic manipulation shows that distances from $P'$ to $Q'$ and to $R'$ both increased by $u_xw_y - w_xu_y$.
	Indeed, without loss of generality set $Q=(d_x,0)$ and $R=(0,d_y)$.
	Then we have
	\[Q'=(d_x+ (w_x+w_y)u_x, 0+(w_x+w_y)u_y)\]
	and
	\[R'=( (u_x+u_y)w_x, d_y+(u_x+u_y)w_y),\]
	so 
	\[P'=(Q'_x,R'_y)=(d_x+ (w_x+w_y)u_x, d_y+(u_x+u_y)w_y)\]
	and finally
	\begin{align*}
		P'_y-Q'_y&=d_y+(u_x+u_y)w_y - (w_x+w_y)u_y\\
		&= d_y + (u_xw_y - w_xu_y),\\
		P'_x-R'_x &= d_x+ (w_x+w_y)u_x - (u_x+u_y)w_x \\
		&= d_x + (w_yu_x - u_yw_x),
	\end{align*}
	concluding the proof. 
\end{proof}


This result implies that for each $v\in V$, the function $T_v(\cdot)$ can be finitely represented: let $B$ be \PT's budget when the period ``kicks in'', then for all $B' \leq B$, the value $T_v(B')$ is stored explicitly and these values can be extrapolated to find $T_v(B'')$ for $B'' > B$. %Note that this means that an algorithm to compute $T_v(B_2)$ can terminate once it reaches $B$.

We point out that periodicity may indeed appear only ``eventually'', as illustrated by \cref{fig:eventually_periodic}; namely, only at $B = 7$ state $(2, 2)$ continuously is an optimal choice and the periodic behaviour is observed.
Replacing $\race{5,4}$ with $\race{2x + 1, 2x}$ leads to quickly growing periodicity thresholds $B$.
Finally, we note that on non-DAGs, the behaviour is not necessarily periodic, as illustrated by \cref{thm:tug2} below.


\stam{
Let $v$ that is not a leaf, by using Thm.~\ref{stm:pipe} and the fact that a minimum of two periodic functions is itself periodic, we are able to reduce to the case in which there exists
a single descendant $v^-$ of $v$ where \PO always moves the token (upon winning a bidding), and
a single descendant $v^+$ of $v$ where \PT always moves the token.
%We say that a function $f\colon\Nat\to\Nat$ is $(u_x,u_y)$-\textit{climbing} if there exists an integer $B$ such that $f(B_2+u_x)=f(B_2)+u_y$ holds for all $B_2>B$.
Then we show that if $v^-$ satisfies $T_{v^-}(B_2+u^-_x)=T_{v^-}(B_2)+u^-_y$ and
$v^+$ satisfies $T_{v^+}(B_2+u^+_x)=T_{v^+}(B_2)+u^+_y$ (both for all large enough $B_2$),
then $v$ satisfies the same equality with $u_x=u^-_x\cdot (u^+_x+u^+_y)$ and $u_y=u^+_y\cdot(u^-_x+u^-_y)$. This completes the induction proof.
\end{proof}
As a consequence of Thm.~\ref{thm:periodicity}, we obtain that for each $v\in V$ the whole function $T_v(\cdot)$ can be represented using a finite amount of information -- its values before the period kicks in, and its values over one period.\todo{Sell this more; maybe "finite representation"}
Moreover, the periodic behaviour indeed appears only eventually, as illustrated by Figure~\ref{fig:eventually_periodic}.
There, it takes up to $B = 79$ until state $(2, 2)$ continuously is an optimal choice.
Only then do we observe periodic behaviour.
}



%For instance, in a game where the winner of the first bidding can choose between playing \race(11,10) or \race(2,2), the 
%
%
%
% Figure environment removed
\stam{We evaluate the root node of a game offering a choice between playing \race{11,10} and \race{2,2}. We depict the ``winning moves'' of \PO, i.e.\ for each budget of \PT, which of the two options \PO has to choose when winning with an optimal bid, denoted $v^-$ in the proof of Thm.~\ref{thm:periodicity}.}


%We draw important corollaries from this result.
% Apart from its theoretical importance, the result has practical consequences: it gives rise to an approximation algorithm for computing the thresholds that runs in linear time. In contrast, the best algorithm to compute exact thresholds on DAGs (presented in the following section) is linear in the numerical value of the input budgets, which is exponential when the budgets are given in binary.

%
%We first review the results on thresholds in continuous-bidding games. 
%\begin{theorem}
%{\bf (Continuous thresholds).} \cite{LLPSU99}
%Consider a poorman continuous-bidding  game $\G$. When \PLi's initial budget is $B_i$, for $i \in \set{1,2}$, the {\em ratio} is $B_1/(B_1 + B_2)$. For each vertex $v$, there exists a {\em continuous threshold}, which is $t_v \in [0,1]$ such that for all $\epsilon >0$, \PO wins when the ratio is $t_v+\epsilon$, and \PT wins when it is $t_v-\epsilon$. Moreover, when $\G$ is a DAG, computing the continuous thresholds can be done in time that is linear in the number of vertices in $\G$.
%\end{theorem}
%
%The following theorem is the main result of this section. We call it the \textit{pipe theorem} since it shows that the threshold budgets lie in a ``pipe'' below the continuous threshold (see Example~\ref{ex:2race+1}). For a game $\G = \zug{V, E, t, s}$ and a vertex $v$, let {\em $\textnormal{max-path}(v)$} denote the length of the longest path from $v$ to the target $t$ or the sink $s$. Note that $\textnormal{max-path}(v) \leq |V| - 1$. 
%
%\todo{double check that we are using the ratios in the same way (I don't think so!)}
%
%\begin{restatable}[Pipe theorem]{theorem}{pipe} \label{stm:pipe}
%	Consider a poorman discrete-bidding game \(\calG = \zug{V, E, t, s}\), where $\zug{V,E}$ is a DAG. Consider a vertex $v\in V$ and let $t_v$ denote the continuous threshold at $v$. Then, for every initial budget $B_2\in\mathbb{N}$ of \PT, we have
%	\[ t_v \cdot (1- \frac{\textnormal{max-path}(v)}{B_2})  \leq \frac{T_v(B_2)}{B_2} \leq t_v \]
%Furthermore, the right-hand-side inequality holds even when $\G$ is not a DAG.
%\end{restatable}
%
%\begin{proof}[Proof sketch]
%	In what follows, we outline the key ideas behind the proof. The full proof can be found in the supplementary material. To prove the right-hand-side inequality, we show that if \PO has initial budget of at least $t_v \cdot B_2$ then \PO can win by following a winning strategy in the {\em continuous-bidding} game and {\em rounding down} the bids. More formally, let $\cstrat$ be a winning strategy for \PO under continuous-bidding when the game starts in $v$, \PO's initial budget is at least $t_v \cdot B_2$, and \PT's initial budget is $B_2$. We define a \PO strategy $\dstrat$ as follows. Whenever $\cstrat$ prescribes a pair $\zug{b, u}$, where $b$ is a bid and $u \in V$ is the vertex to move to upon winning, then $\dstrat$ prescribes $\zug{\lfloor b\rfloor, u}$. 
%	
%To prove the left-hand-side inequality, we show that if \PO has initial budget strictly less than $t_v \cdot (B_2-\textnormal{max-path}(v))$ and \PT has initial budget $B_2$, then \PT can win by following the winning strategy in the {\em continuous-bidding} game with the initial budget $B_2-\textnormal{max-path}(v)$ and rounding the bids up. More formally, let $\cstrat$ be a winning strategy for \PT under continuous-bidding when the game starts in $v$, \PO's initial budget is at most $t_v \cdot (B_2-\textnormal{max-path}(v))$  and \PT's initial budget is $B_2-\textnormal{max-path}(v)$. Suppose that $\cstrat$ prescribes $\zug{b, u}$, then $\dstrat$ for \PT prescribes $\zug{\lceil b\rceil, u}$. 
%The fact that \PT always has enough budget to bid $\lceil b\rceil$ follows from the fact that the game necessarily ends within $\textnormal{max-path}(v)$ turns.
%\end{proof}
%
%Next, we draw three useful corollaries from the pipe theorem.
%
%\subsubsection{Large budgets}
%%The pipe theorem has two useful corollaries. The first is that 
%%its bounds are asymptotically tight, in the sense that they imply that 
%First, the discrete-bidding ratio $T_v(B_2)/B_2$ at every vertex $v$ converges to the continuous threshold $t_v$  at $v$ as $B_2$ tends to infinity, since $\textrm{max-path}(v)/B_2 \leq \frac{|V|-1}{B_2}$ converges to $0$.
%
%\begin{corollary}[Convergence to continuous-thresholds]\label{cor:convergence}
%	$\lim_{B_2\rightarrow\infty}\frac{T_v(B_2)}{B_2} = t_v$, for every $v\in V$.
%\end{corollary}
%
%\subsubsection{Stabilization}
%Intuitively, we say that a vertex $v$ is {\em stable} if there is a neighbor $u \in V$ of $v$ that \PO always proceeds to upon winning the bidding at $v$, no matter the budgets. We note that all vertices are stable under continuous-bidding. Under discrete-bidding, the definition needs care: we say that $v$ {\em stabilizes after} $B_2$ if there is a neighbor $u$ of $v$ such that for every $B_2' \geq B_2$, there is a winning \PO strategy from configuration $\zug{v, T_v(B'_2), B'_2}$ that proceeds to $u$ upon winning the bidding in $v$. The second corollary of the pipe theorem gives a sufficient condition for stabilization. 
%
%%concerns \PO's selection of the successor vertex to which the token should be moved upon winning the bidding. %For each $v\in V$, denote by $v^-$ a neighbouring vertex of $v$ at which the continuous threshold is minimized. We denote one such vertex to be $v^-$ if the minimum is attained at multiple neighbouring vertices.
%%In particular, we show that there exists a lower bound $B$ on \PT's initial budget such that, whenever \PT's initial budget is $B_2 \geq B$ and \PO's initial budget exceeds the threshold budget $T_v(B_2)$ at some initial vertex $v$, then \PO has a winning strategy that moves the token to a neighbour of $v$ which minimizes the continuous threshold among all neighbours of $v$. We call the following corollary the stabilization lemma, since it shows that there exists a lower bound $B$ on \PT's initial budget beyond which the token to which \PO moves the token upon winning the bid {\em stabilizes} within the set of neighbouring vertices at which the continuous threshold is minimized.
%%In other words, there exists a lower bound $B$ on \PT's initial budget beyond which the successor state $v^-$ of $v$ under some \PO's winning strategy {\em stabilizes} and does not depend on exact values of initial budgets.
%
%\begin{corollary}[Stabilization lemma]\label{cor:stabilization}
%	%For each $v \in V$, let $v^-$ be some neighbouring vertex of $v$ in $\calG$ at which the continuous threshold is minimized, i.e.~$v^- \in \text{argmin}_{(v,v')\in E} t_{v'}$, and let
%	Let
%	\[ B = 1 + \max_{v',v''\in V, t_{v'} > t_{v''}} \textnormal{max-path}(v') \cdot \frac{t_{v'}}{t_{v'} - t_{v''}}. \]
%	Then, for each initial vertex $v \in V$, initial budget $B_2 \geq B$ of \PT and initial budget $B_1 \geq T_v(B_2)$ of \PO, there exists a winning strategy of \PO that upon winning the bidding moves the token from $v$ to $v^- \in \text{argmin}_{(v,v')\in E} t_{v'}$.
%\end{corollary}
%\begin{proof}
%	Our choice of $B$ ensures that, whenever $t_{v'} > t_{v''}$, we also have $t_{v'} \cdot (B - \textnormal{max-path}(v')) > t_{v''} \cdot B$. Hence, for every initial budget $B_2 \geq B$ of \PT and for every pair of vertices $v',v''\in V$ with $t_{v'} \neq t_{v''}$, we have that the two interval bounds $t_{v'} \cdot (B_2 - \textnormal{max-path}(v')) \leq T_{v'}(B_2) \leq t_{v'} \cdot B_2$ and $t_{v''} \cdot (B_2 - \textnormal{max-path}(v'')) \leq T_{v''}(B_2) \leq t_{v''} \cdot B_2$ on threshold budgets for discrete-bidding games are disjoint. But this also implies that, whenever $t_{v'} > t_{v''}$ and $B_2 \geq B$, we must also have $T_{v'}(B_2) > T_{v''}(B_2)$. Hence, as a winning strategy of \PO moves the token to a vertex that minimizes the discrete-bidding threshold budget needed to win, this monotonicity result also implies that the winning strategy should move the token to a vertex that minimizes the continuous threshold.
%\end{proof}
%
%(((Instability in race games?)))
%
%%\subsubsection{Finite representation of $T_v(\cdot)$}
%\subsubsection{Eventual periodicity}
%As a third corollary, we show that for any $v\in V$ the function $T_v(\cdot)$ is eventually ``periodic'', and thus it can be represented using a finite amount of information -- its values before the period kicks in, and its values over one period (see Example~\ref{ex:2race+1}).
%
%\begin{corollary}\label{cor:periodicity}
%For any DAG and any vertex $v$ there exist values $B, u_x,u_y \in\Nat$ such that for all $B_2>B$ we have $T_v(B_2+u_x)=T_v(B_2)+u_y$.
%\end{corollary}
%
%The intuition behind the proof is that on DAGs, all continuous thresholds are rational numbers and so,
%once a node $v$ has stabilized, the situation will ``repeat'' with a certain integer period length that is at most the lowest common multiple of the period lengths of the descendants. %. In particular, if the threshold budgets at
% In particular, a suitable value of $B$ is the one given in Corollary~\ref{cor:stabilization}.
%Moreover, the values $u_x$, $u_y$ can be computed efficiently similarly as the continuous thresholds.
%See Appendix for a full proof.
%














\stam{%OLD


\section{Threshold Budgets for Games on DAGs}\label{sec:pipeandstabilization}

We now focus on poorman discrete-bidding games played on directed acyclic graphs (DAGs). The central result of this section are upper and lower bounds that relate the threshold budgets in poorman discrete-bidding games to the threshold budgets in poorman continuous-bidding games. Furthermore, we show that our bounds are asymptotically tight. %In particular, we show that as the budget of Player~2 grows, the ratio of the initial budgets that Player~1 needs to win in the poorman discrete-bidding game converges to the ratio in the poorman continuous-bidding game.

The practical importance of our bounds is that they yield a practical algorithm for computing asymptotically tight bounds on threshold budgets. In particular, to compute bounds on threshold budgets in poorman discrete-bidding games, one can use our bounds together with the method for computing threshold budgets in poorman continuous-bidding games~\cite{xxx}. CITE THE 90s PAPER The complexity of the algorithm is in \textsc{PSPACE}.

The following theorem is the main result of this section. We call it the \textit{Pipe theorem}, as it provides a two-sided bound on threshold budgets in poorman discrete-bidding games where both bounds have linear dependence on the initial budget of \PT. Recall, given a continuous-bidding poorman game, the continuous threshold at a vertex $v$ is a value $t_v \in [0,1]$ such that when the ratio between the two players' budgets is $t_v+\epsilon$, \PO wins, and when the ratio is $t_v-\epsilon$, \PT wins.

\begin{theorem}[Pipe theorem]
	Consider a poorman discrete-bidding game \(\calG = \zug{V, E, B_1, B_2, \calO}\) for which the underlying game graph $(V,E)$ is a DAG. Let $v\in V$ be a vertex and let $t_v$ denote the continuous threshold at $v$. Then, for every initial budget $B_2\in\mathbb{N}$ of \PT, we have
	\[ t_v \cdot (1- \frac{\textnormal{max-path}(v)}{B_2})  \leq \frac{T_v(B_2)}{B_2} \leq t_v, \]
	where {\em $\textnormal{max-path}(v)$} is the length of the longest path from $v$ to the target vertex or the sink vertex of the game. Note that $\textnormal{max-path}(v) \leq |V| - 1$. Furthermore, the right-hand-side inequality holds even if the game graph $(V,E)$ is not a DAG.
\end{theorem}

\begin{proof}[Proof sketch]
	In what follows, we outline the key ideas behind the proof. The full proof can be found in the Appendix. To prove the right-hand-side inequality, we show that if \PO has initial budget of at least $t_v \cdot B_2$ then \PO can win by following the winning strategy in the {\em continuous-bidding} game and {\em rounding down} the bids. In other words, if $\sigma_{\textrm{continuous}}$ is a winning strategy for \PO under continous-bidding when the game starts in $v$, \PO's initial budget is at least $t_v \cdot B_2$ and \PT's initial budget is $B_2$, then the strategy $\sigma_{\textrm{discrete}}$ of \PO which
	\begin{itemize}
		\item bids $\lfloor b\rfloor$ whenever $\sigma_{\textrm{continuous}}$ bids $b$, and
		\item upon winning moves the token to the vertex perscribed by $\sigma_{\textrm{continuous}}$
	\end{itemize}
	is winning for \PO under discrete-bidding. To prove the left-hand-side inequality, we show that if \PO has initial budget strictly less than $t_v \cdot (B_2-\textnormal{max-path}(v))$ and \PT has initial budget $B_2$, then \PT can win by following the winning strategy in the {\em continuous-bidding} game with the initial budget $B_2-\textnormal{max-path}(v)$ and {\em rounding up} the bids. In other words, if $\sigma_{\textrm{continuous}}$ is a winning strategy for \PT under continuous-bidding when the game starts in $v$, \PO's initial budget is at least $t_v \cdot (B_2-\textnormal{max-path}(v))$  and \PT's initial budget is $B_2-\textnormal{max-path}(v)$, then the strategy $\sigma_{\textrm{discrete}}$ of \PO which
	\begin{itemize}
		\item bids $\lceil b\rceil$ whenever $\sigma_{\textrm{continuous}}$ bids $b$, and
		\item upon winning moves the token to the vertex perscribed by $\sigma_{\textrm{continuous}}$
	\end{itemize}
	is winning for \PT under-discrete bidding with initial budget $B_2$. The fact that \PT always has enough budget to bid $\lceil b\rceil$ follows from the fact that the game is played on a DAG so needs to end in at most $\textnormal{max-path}(v)$ turns.
\end{proof}

Pipe theorem has two useful corollaries. The first is that its bounds are asymptotically tight, in the sense that they imply that the discrete-bidding ratio $T_v(B_2)/B_2$ at every vertex $v$ converges to the continuous threshold $t_v$  at $v$ as $B_2\rightarrow\infty$, since $\textrm{max-path}(v)/B_2 \leq \frac{|V|-1}{B_2}$ converges to $0$.

\begin{corollary}
	$\lim_{B_2\rightarrow\infty}\frac{T_v(B_2)}{B_2} = t_v$ for every $v\in V$.
\end{corollary}

The second corollary concerns \PO's selection of the successor vertex to which the token should be moved upon winning the bidding. %For each $v\in V$, denote by $v^-$ a neighbouring vertex of $v$ at which the continuous threshold is minimized. We denote one such vertex to be $v^-$ if the minimum is attained at multiple neighbouring vertices.
In particular, we show that there exists a lower bound $B$ on \PT's initial budget such that, whenever \PT's initial budget is $B_2 \geq B$ and \PO's initial budget exceeds the threshold budget $T_v(B_2)$ at some initial vertex $v$, then \PO has a winning strategy that moves the token to a neighbour of $v$ which minimizes the continuous threshold among all neighbours of $v$. We call the following corollary the stabilization lemma, since it shows that there exists a lower bound $B$ on \PT's initial budget beyond which the token to which \PO moves the token upon winning the bid {\em stabilizes} within the set of neighbouring vertices at which the continuous threshold is minimized.
%In other words, there exists a lower bound $B$ on \PT's initial budget beyond which the successor state $v^-$ of $v$ under some \PO's winning strategy {\em stabilizes} and does not depend on exact values of initial budgets.

\begin{corollary}[Stabilization lemma]
	%For each $v \in V$, let $v^-$ be some neighbouring vertex of $v$ in $\calG$ at which the continuous threshold is minimized, i.e.~$v^- \in \text{argmin}_{(v,v')\in E} t_{v'}$, and let
	Let
	\[ B = 1 + \max_{v',v''\in V, t_{v'} > t_{v''}} \textnormal{max-path}(v') \cdot \frac{t_{v'}}{t_{v'} - t_{v''}}. \]
	Then, for each initial vertex $v \in V$, initial budget $B_2 \geq B$ of \PT and initial budget $B_1 \geq T_v(B_2)$ of \PO, there exists a winning strategy of \PO that upon winning the bidding moves the token from $v$ to $v^- \in \text{argmin}_{(v,v')\in E} t_{v'}$.
\end{corollary}

\begin{proof}
	Our choice of $B$ ensures that, whenever $t_{v'} > t_{v''}$, we also have $t_{v'} \cdot (B - \textnormal{max-path}) > t_{v''} \cdot B$. Hence, for every initial budget $B_2 \geq B$ of \PT and for every pair of vertice $v',v''\in V$ with $t_{v'} \neq t_{v''}$, we have that the two interval bounds $t_{v'} \cdot (B_2 - \textnormal{max-path}) \leq T_{v'}(B_2) \leq t_{v'} \cdot B_2$ and $t_{v''} \cdot (B_2 - \textnormal{max-path}) \leq T_{v''}(B_2) \leq t_{v''} \cdot B_2$ on threshold budgets for discrete-bidding games are disjoint. But this also implies that, whenever $t_{v'} > t_{v''}$ and $B_2 \geq B$, we must also have $T_{v'}(B_2) > T_{v''}(B_2)$. Hence, as a winning strategy of \PO moves the token to a vertex that minimizes the discrete-bidding threshold budget needed to win, this monotonicity result also implies that the winning strategy should move the token to a vertex that minimizes the continuous threshold.
\end{proof}
}


\section{Closed-form Solutions}\label{sec:closed-form}

In this section, we show closed-form solutions for threshold budgets in two special classes of games. 


\subsection{Race Games}

Race games are a class of games played on DAGs. For $a, b \in \Nat$, the race game $\race{a,b}$ ends within $a+b$ turns, \PO wins the game if he wins $a$ biddings before \PT wins $b$ biddings. The key property of race games that we employ is that for each vertex $v$ independent of the budgets, there is a neighbor $v_i$ such that \PLi proceeds to $v_i$ upon winning the bidding at $v$, for $i \in \set{1,2}$. 
\cref{fig:race} depicts \race{3,3}.
% Figure environment removed

In the following, we establish closed-form of threshold budgets at any vertex of a race game \race{a, b} by induction. 
% See the supplementary material for details and examples.



\begin{restatable}{theorem}{thresholdrace}
\label{thm:thresholdrace}
Let $v$ be the root of a race game $\race{a,b}$. Then $T_v(B_2) = a\cdot \floor{B_2/b}$.
\end{restatable}

\begin{proof}
	First note that, the threshold budget for \PO is \(0\) at \(t\), and \(\infty\) at \(s\).
	Let us denote any vertex of the race game as \(v_{x,y}\) where \(x\) and \(y\) are the minimum distance from the vertex to \(t\) and \(s\),  respectively.
	In this notation, the root \(v\) of \(\race{a,b}\) is referred to as \(v_{a,b}\).
	
	Note that, a subgame of \(\race{a,b}\) rooted at any vertex \(v_{x, y}\) is \(\race{x,y}\) itself. 
	We, in fact, show in the following: \(T_{v_{x, y}} = x \cdot \floor{\frac{B}{y}}\) which implies what we require.
	
	Let us now consider \(v_{1,1}\).
	At this vertex, \PO has to win the bid, otherwise \PT simply moves the token to \(s\).
	Because \PT has a budget of \(B\), and \PO wins all ties,
	his threshold budget at this vertex is \(B\), and he bids his whole budget.
	
	By induction on \(x\), we can argue that for any vertex of the form \(v_{x, 1}\), the threshold budget is \(xB\), because \PO has to win all \(x\) the bids to prevent the token reaching \(s\).
	Thus he has to bid at least \(B\) at all those \(x\) bids.
	In fact, if he has budget at most \(xB - 1\) at vertex \(v_{x,1}\), then \PT has a winning strategy:
	she bids \(B\) until she wins.
	
	Similarly, by induction on \(y\), we claim that for any vertex of the form \(v_{1, y}\), the threshold budget is \(\floor{\frac{B}{y}}\).
	The base case of this induction is \(v_{1,1}\), for which we showed earlier that the statement is true.
	Let us assume it is true for \(v_{1,y-1}\), and we prove the claim for \(v_{1,y}\).
	
	We suppose \PO's budget at \(v_{1, y}\) is \(\floor{\frac{B}{y}}\) while \PT's budget is \(B\).
	We claim that the winning strategy for \PO at vertex \(v_{1,y}\) is to bid his whole budget itself.
	If he wins the bid, he moves the token to target.
	Otherwise, \PT wins the bid by at least bidding \(\floor{\frac{B}{y}} +1\), and of course, she moves the token to \(v_{1, y-1}\) .
	Thus, her budget at \(v_{1, y-1}\) is at most \(B - (\floor{\frac{B}{y}} +1)\), while \PO's budget remains \(\floor{\frac{B}{y}}\).
	From the induction hypothesis, we know when \PT has a budget \(B - (\floor{\frac{B}{y}} +1)\), \PO's threshold budget for surely winning is \(\floor{\frac{B - (\floor{\frac{B}{y}} +1)}{y-1}}\).
	If we can show that, \(\floor{\frac{B}{y}} \geq \floor{\frac{B - (\floor{\frac{B}{y}} +1)}{y-1}}\), we are done.
	We show this in the following:
	\begin{align*}
		B - y \cdot \floor{\frac{B}{y}} &\leq y-1\\
		\implies B - \floor{\frac{B}{y}} &\leq (y-1) \cdot \floor{\frac{B}{y}} + (y-1)\\
		\implies \frac{B - \floor{\frac{B}{y}}}{y-1} &\leq \floor{\frac{B}{y}} + 1\\
		\implies \frac{B - (\floor{\frac{B}{y}}+1)}{y-1} &\leq \floor{\frac{B}{y}} + \frac{y-2}{y-1}
	\end{align*}
	
	Because \(\floor{\frac{B}{y}}\) itself is an integer, by taking \(\floor{}\) on the both side, we get \(\floor{\frac{B}{y}} \geq \floor{\frac{B - (\floor{\frac{B}{y}} +1)}{y-1}}\).
	Therefore, \(\floor{\frac{B}{y}}\) is a sufficient budget for \PO to surely win from vertex \(v_{1,y}\).
	
	Now, we need to show that this is also necessary budget for him.
	In fact, we show that when \PO has budget at most \(\floor{\frac{B}{y}} - 1\), while \PT's budget is \(B\), she has a surely winning strategy from \(v_{1,y}\).
	Her winning strategy is bidding \(\floor{\frac{B}{y}}\), until she reaches \(s\).
	Because \(B \geq y \cdot \floor{\frac{B}{y}}\), she can actually bids likewise.
	At each vertex, \PO's budget will be strictly less than what she is bidding, therefore he looses all the \(y\) bids, and the token indeed reaches the safety vertex.
	
	For a general vertex \(v_{x, y}\), we argue by induction which goes like above.
	We assume that for \(v_{x-1,y}\) and \(v_{x, y-1}\), which are the only two neighbours of \(v_{x, y}\), the threshold budget for \PO is \((x-1) \cdot \floor{\frac{B}{y}}\) and \(x \cdot \floor{\frac{B}{y-1}}\), respectively.
	
	We suppose \PO's budget at \(v_{x, y}\) is \(x \cdot \floor{\frac{B}{y}}\), and \PT's budget is \(B\).
	We claim that his wining strategy at the first bid is to bid \(\floor{\frac{B}{y}}\).
	We show that irrespective of where the token gets placed at the next vertex, he will have the respective threshold budget at that vertex.
	
	If he wins the bid at \(v_{x, y}\), his new budget becomes \((x-1)\cdot \floor{\frac{B}{y}}\), which is exactly what he needs to surely win from \(v_{x-1, y}\).
	If he looses, and the token gets placed at \(v_{x, y-1}\), \PT's budget becomes at most \(B - \floor{\frac{B}{y}}+1)\).
	It remains to show that \(x \cdot \floor{\frac{B -  \floor{\frac{B}{y}}+1)}{y-1}} \leq x\cdot \floor{\frac{B}{y}}\), which is true as we have earlier established \(\floor{\frac{B}{y}} \geq \floor{\frac{B - (\floor{\frac{B}{y}} +1)}{y-1}}\).
	It proves that \(x\cdot \floor{\frac{B}{y}}\) is the sufficient budget for \PO to surely win from \(v_{x, y}\).
	
	Finally, if \PO's budget is at most \(x \cdot \floor{\frac{B}{y}} - 1\) and \PT's budget is \(B\), then \PT wins the game if she bids \(\floor{\frac{B}{y}}\) at each bidding.
	This can be shown by another inductive argument where we assume the statement being true for vertices \(v_{x-1, y}\) and \(v_{x, y-1}\), and follow the same steps that we did for \PO above.
\end{proof}

With the exact closed-form of threshold budgets for race games, we now show that the bounds in \cref{stm:pipe} are tight.

\begin{corollary}  \label{stm:pipe_tight}
	For every rational number $q = n / m$, there exist infinitely many games $\G$ with vertex $v$ such that $t_v = q$ and for infinitely many $B$ the lower and upper bound of~\cref{stm:pipe} actually is an equality for some $B_2 > B$.
\end{corollary}
\begin{proof}
	Choose $\G = \race{n, m}$ (or any multiple thereof) and insert the closed form of \cref{thm:thresholdrace}.
	Note that in a race game $\textnormal{max-path}(v)$ of the root vertex $v$ clearly is $\max(n,m)$.
\end{proof}



%\begin{proof}[Proof Sketch]
%	At a vertex which is a step away from \PO's target, he bids with his whole budget. 
%	On the other hand, at a vertex which is a step away from the safety vertex, \PO needs to bid at least as much as \PT's budget.
%	These gives the necessary budgets at vertices one-step away from either sink. 
%	For other vertices, we use induction on these steps. 
%	Full proof can be found in the supplementary material.
%\end{proof}



\subsection{Tug-of-War games}
Given an integer \(n \geq 1\), a \textit{tug-of-war} game $\TUG(n)$ is a game played on a chain with $n+2$ nodes, namely $n$ interior nodes and two endpoints $s$ and $t$. We develop closed-form representations of thresholds in  \(\TUG(2)\) and \(\TUG(3)\) (both depicted in \cref{fig:tow}). 
For integers $k\in[1,n]$ and $b\ge 0$, we denote by $\tug(n,k,b)$ the smallest budget that \PO needs to win the tug-of-war game $\TUG(n)$ at the vertex  that is $k$ steps from his target $t$, when the opponent has budget $b$.
%Here we state closed-form solutions for \(\TUG(2)\) (shown in \cref{ex:TOW}) and \(\TUG(3)\).
%The technical proofs can be found in the supplementary material.

% Figure environment removed


\begin{restatable}{theorem}{tugtwo}\label{thm:tug2}
	For \(b \geq 0\), we have 
        \(\tug(2,1,b)=\floor{b/\phi}\) and \(\tug(2,2,b)=\floor{b \cdot \phi}\),
       where \(\phi = (\sqrt{5}+1)/2 \approx 1.618\) is the golden ratio.
\end{restatable}

\stam{
\begin{proof}
	To simplify notation, we use the same vertex names as in Fig.~\ref{fig:TOW} and, for a \PT budget $b$, we denote by \(t_b = \tug(2,1,b)\) and \(u_b = \tug(2,2,b)\), the thresholds in $v_1$ and $v_2$, respectively.
	The core of the proof follows from the following properties of \(t_b\) and \(u_b\):
	\begin{enumerate}
		\item\label{itm:tuga} \(t_0 = u_0 = 0\)
		
		\item\label{itm:tugb} \(u_b = t_b + b\) for any \(b \geq 1\)
		
		\item\label{itm:tugc} \(t_b = \min_x\{\max (x, u_{b - 1-x}) \mid 0 \leq x \leq b\}\) for any \(b \geq 1\)
	\end{enumerate}
	\cref{itm:tuga} is trivial: both players bid $0$, \PO wins ties, thus he wins all biddings (see Example~\ref{ex:TOW}).
	For \cref{itm:tugb}, consider the configuration $\zug{v_2, u_b, b}$. Since $v_2$ neighbors $s$, it is dominant for \PT to bid all her budget $b$. In order to avoid losing, \PO must bid $b$, and the game proceeds to $\zug{v_1, u_b-b, b}$, thus $t_b = u_b-b$. For \cref{itm:tugc}, consider a configuration $\zug{v_1, x, b}$ from which \PO wins, i.e., $x \geq t_b$. Note that it is dominant for \PO to bid his whole budget $x$. In order to avoid losing, \PT must bid $x+1$, and proceed to $\zug{v_2, x, b-(x+1)}$ from which \PO wins, thus $x \geq u_{b-(x+1)}$, and $t_b$ is obtained from the minimal such $x$. 
	
	This gives us the system of three equations with three unknowns (for a fixed \(b\)), thus existence of an unique solution, if any. 
	In the supplementary material, we verify that the expressions \(t_b = \floor{\frac{b}{\phi}}\) and \(u_b = \floor{b \cdot \phi}\) satisfy the equations.
	\end{proof}}
	%, detailed explanation of this can be found in the supplementary material. 	

\begin{proof}
	To simplify the notation, let us assume \(t_b = \tug(2,1,b)\), and \(u_b = \tug(2,2,b)\).
	We first claim that \(t_b\) and \(u_b\) are the unique solution to the following system of recurrence relations. 
	\begin{enumerate}
		\item\label{itm:tuga} \(t_0 = u_0 = 0\)
		
		\item\label{itm:tugb} \(u_b = t_b + b\) for any \(b \geq 1\)
		
		\item\label{itm:tugc} \(t_b = \min_x\{\max (x, u_{b - 1-x}) \mid 0 \leq x \leq b\}\) for any \(b \geq 1\)
	\end{enumerate}
	\cref{itm:tuga} is obvious because \PO bids \(0\) at every step and he wins ties, when \PT has a budget \(0\).
	
	\PO needs to win at the vertex which is \(2\) steps away from his target, otherwise \PT moves the token to the other end-point.
	Therefore, \PO needs to bid \(b\), and his new budget should be, by definition, at least \(t_b\) upon winning.
	This gives us \cref{itm:tugb}.
	
	Finally, at the vertex which is a single step away from \PO's target, he needs to optimize what his bid would be between \(0\) and \(b\) so that even if he loses the current bid, he would have enough budget at the next step to win from there (i.e, \(u_b\)).
	This gives us \cref{itm:tugc}.
	
	Moreover, the system of equations has a unique solutions, as there are as many equations as there are unknowns (\(t_b, u_b\) for a fixed \(b\)).
	Hence, it is enough to show that the expressions \(t_b = \floor{\frac{b}{\phi}}\) and \(u_b = \floor{b \cdot \phi}\) satisfy those equations.
	Clearly, \(\floor{\frac{0}{\phi}} = \floor{0 \cdot \phi} = 0\), so \cref{itm:tuga} holds.
	Next note that the golden ratio satisfy \(\phi = 1 + 1/\phi\). 
	Thus,
	\[ u_b = \floor{b \cdot \phi} = \floor{b \cdot (1 + 1/\phi)} = \floor{b + b/\phi} = b + \floor{b/\phi} = b + t_b\]
	
	implying \cref{itm:tugb} holds too.
	
	Finally, note that the function $f\colon x\to x$ is increasing, hence to verify \cref{itm:tugc} we need to show two inequalities for any $b\geq 1$:
	\begin{enumerate}
		\item For $x=\floor{b/\phi}$ we have $\floor{(b-1-x)\cdot \phi}\leq \floor{b/\phi}$.
		\item For $x=\floor{b/\phi}-1$ we have $\floor{(b-1-x)\cdot \phi}\geq \floor{b/\phi}$.
	\end{enumerate}
	In both cases, we will do this by checking that the insides of the two floor functions being compared satisfy the same inequality. Upon plugging in $x$, it thus suffices to show
	\[(b-1-\floor{b/\phi})\cdot \phi \leq b/\phi
	\quad\text{and}\quad
	(b-\floor{b/\phi})\cdot \phi \geq b/\phi.
	\]
	From $\phi=1+1/\phi$ we have $b\cdot\phi-b/\phi=b$, so the desired inequalities rewrite as
	\[ b-\phi \leq \floor{b/\phi}\cdot\phi
	\quad\text{and}\quad
	\floor{b/\phi}\cdot\phi\leq b.
	\]
	Those two inequalities follow from the obvious inequalities $b/\phi-1\leq \floor{b/\phi} \leq b/\phi$ after multiplying by $\phi$.
\end{proof}

%	Moreover, the system of equations has a unique solutions, as there are as many equations as there are unknowns (\(t_b, u_b\) for a fixed \(b\)).
%	Hence, it is enough to show that the expressions \(t_b = \floor{\frac{b}{\phi}}\) and \(u_b = \floor{b \cdot \phi}\) satisfy those equations.
%	Clearly, \(\floor{\frac{0}{\phi}} = \floor{0 \cdot \phi} = 0\), so \cref{itm:tuga} holds.
%	Next note that the golden ratio satisfy \(\phi = 1 + 1/\phi\). 
%	Thus,
%	\[ u_b = \floor{b \cdot \phi} = \floor{b \cdot (1 + 1/\phi)} = \floor{b + b/\phi} = b + \floor{b/\phi} = b + t_b\]
%	
%	implying \cref{itm:tugb} holds too.
%	
%	Finally, note that the function $f\colon x\to x$ is increasing, hence to verify \cref{itm:tugc} we need to show two inequalities for any $b\geq 1$:
%	\begin{enumerate}
%		\item For $x=\floor{b/\phi}$ we have $\floor{(b-1-x)\cdot \phi}\leq \floor{b/\phi}$.
%		\item For $x=\floor{b/\phi}-1$ we have $\floor{(b-1-x)\cdot \phi}\geq \floor{b/\phi}$.
%	\end{enumerate}
%	In both cases, we will do this by checking that the insides of the two floor functions being compared satisfy the same inequality. Upon plugging in $x$, it thus suffices to show
%	\[(b-1-\floor{b/\phi})\cdot \phi \leq b/\phi
%	\quad\text{and}\quad
%	(b-\floor{b/\phi})\cdot \phi \geq b/\phi.
%	\]
%	From $\phi=1+1/\phi$ we have $b\cdot\phi-b/\phi=b$, so the desired inequalities rewrite as
%	\[ b-\phi \leq \floor{b/\phi}\cdot\phi
%	\quad\text{and}\quad
%	\floor{b/\phi}\cdot\phi\leq b.
%	\]
%	Those two inequalities follow from the obvious inequalities $b/\phi-1\leq \floor{b/\phi} \leq b/\phi$ after multiplying by $\phi$.
%\end{proof}


\stam{OLD	
	We first claim that \(t_b\) and \(u_b\) are the unique solutions to the following system of recurrence relations. 
	\begin{enumerate}
		\item\label{itm:tuga} \(t_0 = u_0 = 0\)
		
		\item\label{itm:tugb} \(u_b = t_b + b\) for any \(b \geq 1\)
		
		\item\label{itm:tugc} \(t_b = \min_x\{\max (x, u_{b - 1-x}) \mid 0 \leq x \leq b\}\) for any \(b \geq 1\)
	\end{enumerate}
	\cref{itm:tuga} is obvious because \PO bids \(0\) at every step and he wins ties, when \PT has a budget \(0\).
	
	\PO needs to win at the vertex which is \(2\) steps away from his target, otherwise \PT moves the token to the other end-point.
	Therefore, \PO needs to bid \(b\), and his new budget should be, by definition, at least \(t_b\) upon winning.
	This gives us \cref{itm:tugb}.
	
	Finally, at the vertex which is a single step away from \PO's target, he needs to optimize what his bid would be between \(0\) and \(b\) so that even if he loses the current bid, he would have enough budget at the next step to win from there (i.e, \(u_b\)).
	This gives us \cref{itm:tugc}.
	
	This gives us the system of three equations with three unknowns (for a fixed \(b\)), thus existence of an unique solution, if any. 
	We can indeed verify that the expressions \(t_b = \floor{\frac{b}{\phi}}\) and \(u_b = \floor{b \cdot \phi}\) satisfy those equations, detailed explanation of this can be found in the supplementary material. 
}

\begin{remark}
\label{rem:Wythoff}
\normalfont
The closed-form solution in \cref{thm:tug2} has a striking similarity to a classic result in Combinatorial Game Theory. {\em Wythoff Nim} is played by two players who alternate turns in removing chips from two stacks. A configuration of the game is $\zug{s_1, s_2}$, for integers $s_1 \geq s_2 \geq 0$, representing the number of chips placed on each stack. A player has two types of actions: (1) choose a stack and remove any $k>0$ chips from that stack, i.e., proceed to $\zug{s_1 -k, s_2}$ or $\zug{s_1, s_2 - k}$, or (2) remove any $k >0$ chips from both stacks, i.e., proceed to $\zug{s_1 - k, s_2 -k}$. The player who cannot move loses. Wythoff~\cite{Wyt07} identified the configurations from which the first player to move loses. Trivially, $\zug{0,0}$ is losing, followed by $\zug{1, 2}, \zug{3,5}, \ldots$. In general, the $n$-th losing configuration is $\zug{\lfloor n \cdot \phi \rfloor, \lfloor n \cdot \phi \rfloor + n}$. Note the similarity to the thresholds in  $v_2$ and $v_1$, which can be written respectively as $\zug{\lfloor b \cdot \phi \rfloor, \lfloor b \cdot \phi \rfloor - b}$, for $b \geq 0$.
\end{remark}



\begin{restatable}{theorem}{tugthree}\label{thm:tug3}
	For $b\ge 1$ we have
	\(\tug(3,1,b)=\floor{\frac{b-1}{2}}\), \(\tug(3,2,b)=b-1\), and \(\tug(3,3,b)=2b-1\).
\end{restatable}


\begin{proof}
	We proceed similarly to the proof of~\cref{thm:tug2}.
	This time, we need to check that the expressions
	\[t_b=\floor{(b-1)/2}, \quad u_b=b-1, \quad\text{and}\quad v_b=2b-1
	\]
	satisfy the relations
	\begin{enumerate}
		\item\label{itm:tug3a} $t_1=u_1=0$, $v_1=1$,
		\item\label{itm:tug3b} $v_b=u_b+b$ for any $b\geq 2$,
		\item\label{itm:tug3c} $u_b= \min_x \{  \max\{t_b+x, v_{b-1-x}\} \mid 0\leq x \leq b\}$ for any $b\geq 2$.
		\item\label{itm:tug3d} $t_b= \min_x \{  \max\{x, u_{b-1-x}\} \mid 0\leq x \leq b\}$ for any $b\geq 2$.
	\end{enumerate}
	This time, both \cref{itm:tug3a} and \cref{itm:tug3b} follow by direct substitution.
	
	Regarding \cref{itm:tug3c}, we need to show that
	\[ b-1=  \min_x \{  \max\{\floor{(b-1)/2}+x, 2b-3-2x \} \mid 0\leq x \leq b\}
	\]
	To that end, we distinguish two cases based on the parity of $b$.
	If $b=2k$ is even then we need to show
	\[ 2k-1 = \min_x\{ \max\{k-1+x, 4k-3-2x \} \mid 0\leq x \leq 2k\},
	\]
	and indeed the minimum on the right-hand side is attained for $x=k-1$ and is equal to $2k-1$ as desired.
	Similarly, if $b=2k+1$ is odd then we need to show
	\[ 2k = \min_x\{ \max\{k+x, 4k-1-2x \} \mid 0\leq x \leq 2k\},
	\]
	and indeed the minimum on the right-hand side is attained for $x=k$ and is equal to $2k$ as desired.
	
	Finally, regarding \cref{itm:tug3d} we have
	$u_{b-1-x}=b-2-x$, hence the two numbers inside the $\max(\cdot)$ function always sum up to $b-2$.
	If $b=2k$ is even, then the minimum is $(b-2)/2=k-1 = \floor{(b-1)/2}=t_b$ as desired.
	If $b=2k+1$ is odd then the minimum is $\ceil{(b-2)/2}=k=\floor{(b-1)/2}=t_b$ as desired again.
\end{proof}


\stam{
\begin{proof}
	We proceed similarly to the proof of~\cref{thm:tug2}.
	This time, we need to check that the expressions
	\[t_b=\floor{(b-1)/2}, \quad u_b=b-1, \quad\text{and}\quad v_b=2b-1
	\]
	satisfy the relations
	\begin{enumerate}
		\item\label{itm:tug3a} $t_1=u_1=0$, $v_1=1$,
		\item\label{itm:tug3b} $v_b=u_b+b$ for any $b\geq 2$,
		\item\label{itm:tug3c} $u_b= \min_x \{  \max\{t_b+x, v_{b-1-x}\} \mid 0\leq x \leq b\}$ for any $b\geq 2$.
		\item\label{itm:tug3d} $t_b= \min_x \{  \max\{x, u_{b-1-x}\} \mid 0\leq x \leq b\}$ for any $b\geq 2$.
	\end{enumerate}
	This time, both \cref{itm:tug3a} and \cref{itm:tug3b} follow by direct substitution.
	
	Regarding \cref{itm:tug3c}, we need to show that
	\[ b-1=  \min_x \{  \max\{\floor{(b-1)/2}+x, 2b-3-2x \} \mid 0\leq x \leq b\}
	\]
	
	To that end, we distinguish two cases based on the parity of $b$.
	This analysis can be found in the supplementary material. 
	
	
%	If $b=2k$ is even then we need to show
%	\[ 2k-1 = \min_x\{ \max\{k-1+x, 4k-3-2x \} \mid 0\leq x \leq 2k\},
%	\]
%	and indeed the minimum on the right-hand side is attained for $x=k-1$ and is equal to $2k-1$ as desired.
%	Similarly, if $b=2k+1$ is odd then we need to show
%	\[ 2k = \min_x\{ \max\{k+x, 4k-1-2x \} \mid 0\leq x \leq 2k\},
%	\]
%	and indeed the minimum on the right-hand side is attained for $x=k$ and is equal to $2k$ as desired.
	
	Finally, regarding \cref{itm:tug3d} we have
	$u_{b-1-x}=b-2-x$, hence the two numbers inside the $\max(\cdot)$ function always sum up to $b-2$.
	Here too, we analyse by distinguishing the parity of \(b\), and the detailed argument can be found in the supplementary material. 
%	If $b=2k$ is even, then the minimum is $(b-2)/2=k-1 = \floor{(b-1)/2}=t_b$ as desired.
%	If $b=2k+1$ is odd then the minimum is $\ceil{(b-2)/2}=k=\floor{(b-1)/2}=t_b$ as desired again.
\end{proof}}


We note that for $n \ge 4$ the situation gets surprisingly more complicated.
For $n=5$ the threshold budgets do eventually converge to a simple pattern, but only from around $b=4\cdot 10^3$ on.
In contrast, for $n\in\{4,6\}$ the threshold budgets exhibit no clear pattern up until $b=10^6$.
Moreover, while the pipe theorem \cref{stm:pipe} seems to hold for $n \leq 5$ (experimentally validated up to $b = 10^7$), it is (quickly) violated for $n \geq 6$.
This suggests that a simple closed form solution for general games is unlikely, given that these structurally similar games behave so differently.



\begin{algorithm}[!t]
    \caption{\method{}}\label{alg: iterative training}
    \begin{algorithmic}
        \State Input: dataset $\gD$, oracle $\oracle$, balanced synthetic dataset size $N$
        \State $i \leftarrow 0$
        \State $\theta_i \leftarrow \argmin_\theta \gL^{\text{DM}}_\theta$ \Comment{train baseline DM, \cref{eq:score-matching}}
        \State ${s_i} \leftarrow s_{\theta_i}(\rvx_t; t)$
        \While{not done}
            \State $\synth^+_i,\synth^-_i \leftarrow$ generate samples from DM with score function ${s_i}$ and label with $\oracle$
            \While{$\min(|\synth^+_i|,|\synth^-_i|)<N$}
             \State $\synth^+,\synth^- \leftarrow$ generate more samples  from DM with score function ${s_i}$ and label with $\oracle$
             \State $\synth^+_i \leftarrow \synth^+_i \cup \synth^+$, $\synth^-_i \leftarrow \synth^-_i \cup \synth^-$
            \EndWhile
            \State $\alpha_i \leftarrow |\synth^+_i| / (|\synth^+_i| + |\synth^-_i|)$ \Comment{Estimate class prior probabilities for Bayes optimal classifier} 
            \State $\synth^+_i \leftarrow \subsample(N,\synth^+_i), \synth^-_i \leftarrow \subsample(N,\synth^-_i)$ \Comment{balance dataset for IS classifier training}
            \State $\phi_i \leftarrow \argmin_\phi \hat{\gL}^{\text{cls}}_\phi(\alpha_i, \synth^+_i, \synth^-_i)$ \Comment{train guidance classifier, \cref{eq: classifier loss}}
            \State $i \leftarrow i+1$

            \If{distill}
                \State $\psi \leftarrow \argmin_\psi \gL^{\text{dtl}}_\psi$\Comment{distill into single DM, \cref{eq: distillation}}
                \State ${s_i} \leftarrow  s_{\psi}(\rvx_t; t)$ 
            \Else  \Comment{``stack'' guidance classifiers}
                \State ${s_i} \leftarrow {s_{i-1}} + \nabla_{\rvx_t}\log C_{\phi_i}(\rvx_t; t)$ \Comment{See \cref{eq:gen-neg-score}}
            \EndIf
        \EndWhile
        \State \Return DM score function $s_i$
    \end{algorithmic}
\end{algorithm}


\section{Conclusion}
We study, for the first time, bidding games that combine poorman with discrete bidding. 
%short We present a mix of positive and negative results. 
On the negative side, threshold budgets in poorman discrete-bidding games exhibit complex behavior already in simple games, in particular in games with cycles. On the positive side, we identify interesting structure: we prove determinacy, in DAGs, we relate the threshold budgets with continuous ratios, and prove that thresholds are periodic. Additionally, our implementation efficiently computes  exact solutions to non-trivial games. We particularly invite the interested reader to explore bidding games using it, the code will be available on demand.

Our work opens several venues for future work:
%Prior work has predominantly focused on continuous bidding or on Richman discrete-bidding. 
%Poorman discrete-bidding is appealing from both a theoretical and practical standpoint. 

Theoretically, 
we left several open problems and conjectures. Beyond that, 
poorman discrete-bidding is more amendable to extensions when compared with poorman continuous-bidding, which quickly becomes technically challenging, or Richman discrete-bidding, which is a rigid mechanism. For example, it is interesting to introduce into the basic model,  multi-players or complex objectives, e.g., that take into account left over budgets~\cite{HDM12}. 
%short In addition, it is interesting to prove or disprove our conjectures towards gaining a better understanding of the model. 

Practically, poorman is more popular than Richman bidding since it coincides with the popular first-price auction and discrete- is more popular than continuous-bidding since most if not all practical applications employ some granularity constraints on bids. It is interesting to develop applications based on these games. For example, to analyze and develop bidding strategies 
%short with guarantees 
in sequential auctions or fair allocation of goods~\cite{BEF22}. 
%short A step beyond solving a game is {\em mechanism design}: develop the arena on which the game is played so as to guarantee a desirable outcome of the game~\cite{MM+22}. 
Further, it is interesting to study {\em mechanism design}: synthesize an arena so that the game has guarantees (e.g.,~\cite{MM+22}). 


\acknowledgements{This research was supported in part by ISF grant no. 1679/21, ERC CoG 863818 (FoRM-SMArt) and the European Union’s Horizon 2020 research and innovation programme under the Marie Skłodowska-Curie Grant Agreement No. 665385.}







%% The file named.bst is a bibliography style file for BibTeX 0.99c
%\bibliographystyle{ecai}
\bibliography{ga}

%\bibliography{ijcai23}

%\cleardoublepage
%\appendix
%\begin{comment}
\section{System Architecture}
\label{appendix:architecture}
\system has a novel modularized system architecture with three key components: 
\emph{StreamManager}, 
\emph{TxnManager} and \emph{TxnScheduler}. 
These components are instantiated in each thread locally.
The execution outline of \system is presented in Algorithm~\ref{alg:algo}.
Transactional stream processing is continuous and potentially never ends (Line 1$\sim$8).
The dependency resolution and execution of state transactions are separated into two non-overlapping phases by punctuations~\cite{Tucker:2003:EPS:776752.776780} (Line 2 and 5), which guarantees that no subsequent input event will have a smaller timestamp. 
Effectively, a batch of state transactions is collected during the first phase, and processed during the second phase.

In the first phase (i.e., stream processing phase), 
the \emph{StreamManager} conducts preprocessing for every input event ($e$). Similar to some prior works~\cite{tstream}, state transactions may be issued but not immediately processed during preprocessing (Line 3).
The \emph{pre\_processing} and \emph{post\_processing} functions are exposed as APIs to users.
The \emph{TxnManager} handles dependency resolution (Line 4) among state transactions and insert decomposed operations to construct a \tpg. We discuss the detailed two-phase \tpg construction process in Section~\ref{subsec:construction}.

In the second phase  (i.e., transaction processing phase), 
the \emph{TxnManager} is first involved again to refine (Line 6) the constructed \tpg with further dependency resolution.
The \emph{TxnScheduler} 
schedules operations for concurrent execution based on the constructed \tpg according to the three dimensions of scheduling decisions (Line 7). 
In particular, a scheduling decision model $M$ is instantiated based on the constructed \tpg (Line 14).
\textbf{\circled{1}} Guided by $M$, execution threads adopt an exploration strategy (Section~\ref{subsec:explore}) to explore the constructed \tpg for operations available to be scheduled constrained by dependencies. 
\textbf{\circled{2}} 
During exploration, one or multiple operations may be treated as the 
% basic 
unit of scheduling (Section~\ref{subsec:granularity}). 
Subsequently, \textbf{\circled{3}} every thread executes operation(s) in the unit of scheduling with various abort handling mechanisms (Section~\ref{subsec:abort_handling}).
Only when state transactions are processed (i.e., committed or aborted) can the associated input events be postprocessed (Line 8) by the \emph{StreamManager} based on transaction processing results.
\end{comment}

\begin{comment}
\begin{algorithm}
\footnotesize
    \KwData{$e$ \tcp{Input event}}
    \KwData{$txn_{ts}$ \tcp{State transaction}}
    \KwData{$G$ \tcp{The currently constructed TPG}}
    \While{!finish processing of input streams}{
        \eIf(\tcp*[h]{Phase 1}){\text{$e$ is not a $punctuation$}}{
                $txn_{ts}$ $\gets$ PRE\_Processing($e$)\;
                \textbf{TPG\_Construction}($G$, $txn_{ts}$)\; 
          }(\tcp*[h]{Phase 2}){
                \textbf{TPG\_Refinement}($G$)\; 
                \textbf{TXN\_Scheduling}($G$)\; 
                POST\_Processing()\;
          }
    }
    
    \SetKwFunction{FMain}{TPG\_Construction}
    \SetKwProg{Fn}{Function}{:}{}
    \Fn{\FMain{$G$, $txn_{ts}$}}{
        $O_{1..k}$ $\gets$ \textbf{Partition} $txn_{ts}$\;
        \ForEach{\text{operation $O_{i}$ $\in$ $O_{1..k}$}}{
            \textbf{Identify} its \ld\;
            $G$ $\gets$ $G$ + $O_{i}$ \;
        }
    }
    \SetKwFunction{FMain}{TPG\_Refinement}
    \SetKwProg{Fn}{Function}{:}{}
    \Fn{\FMain{$G$}}{
        \ForEach{\text{vertex $e_{i}$ $\in$ $G$}}{
            \textbf{Identify} its \td, \pd\;
        }
    }
    
    \SetKwFunction{FMain}{TXN\_Scheduling}
    \SetKwProg{Fn}{Function}{:}{}
    \Fn{\FMain{$G$}}{
        $M$ $\gets$ Instantiated with $G$;\tcp{A decision model}
        \While{!finish scheduling of $G$
        }{
          \textbf{\circled{2}} $Scheduling Unit$ $\gets$ \textbf{\circled{1}} \emph{Explore}($G$, $M$)\; 
            \textbf{\circled{3}} \emph{Execute with Abort Handling} ($Scheduling Unit$)\; 
        }
    }
  \caption{Execution Outline of \system}
  \label{alg:algo}
\end{algorithm}
\end{comment}

\end{document}

