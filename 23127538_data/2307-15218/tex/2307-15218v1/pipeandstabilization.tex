

\section{Threshold Budgets for Games on DAGs}\label{sec:pipeandstabilization}
In this section, we focus on games played on directed acyclic graphs (DAGs). We present two main results:
First, the \textit{Pipe theorem} that relates the threshold budgets to the threshold ratio in the continuous-bidding game; and,
second, the \textit{Periodicity theorem} which shows that the threshold budgets eventually exhibit a periodic behavior.
Throughout this section, let \(\calG = \zug{V, E, t, s}\) be a game with $\zug{V,E}$ a DAG.
%Namely, given the threshold values $t_1\le t_2\le\dots\le t_k$ at all descendants of a fixed node, the threshold value $t$ at that node can be computed by solving a single linear equation
%$t= t_k(t_1+1)/(t_k+1)$.

\subsection{Relating Discrete and Continuous Thresholds}
We call the following theorem the \textit{Pipe theorem} since it shows that the threshold budgets $T_v(B_2)$ lie in a ``pipe'' below a line whose slope is the threshold ratio $t_v$ (see Example~\ref{ex:2race+1}). We note that threshold ratios can be computed in DAGs in time polynomial in the size of the game (a fact we also exploit later on in our algorithm on DAGs), thus an immediate corollary of the Pipe theorem is an efficient approximation algorithm to computing the threshold budgets. In Corollary~\ref{stm:pipe_tight}, we show that the lower bound is tight. For a vertex $v$, let {\em $\textnormal{max-path}(v)$} denote the length of the longest path from $v$ to either $t$ or $s$. Note that $\textnormal{max-path}(v) \leq |V| - 1$. 

\begin{restatable}[Pipe theorem]{theorem}{pipe} \label{stm:pipe}
	Given $v\in V$, denote by $t_v$ the threshold ratio in the continuous-bidding game at $v$. Then, for every initial budget $B_2\in\mathbb{N}$ of \PT, we have
	\begin{equation*}
		t_v \cdot (1- \textnormal{max-path}(v) / B_2)  \leq T_v(B_2) / B_2 \leq t_v.
	\end{equation*}
	The right-hand side inequality holds even when $\G$ is not a DAG.
\end{restatable}

%\begin{proof}[Proof sketch] (See supplementary material for the full proof.)
%%	In what follows, we outline the key ideas behind the proof. The full proof can be found in the supplementary material.
%	To prove the right-hand side inequality, we show that if \PO has initial budget of at least $t_v \cdot B_2$ then \PO can win by following a winning strategy in the {\em continuous-bidding} game and {\em rounding down} the bids. More formally, let $\cstrat$ be a winning strategy for \PO under continuous-bidding when the game starts in $v$, \PO's initial budget is at least $t_v \cdot B_2$, and \PT's initial budget is $B_2$. We define a \PO strategy $\dstrat$ as follows. Whenever $\cstrat$ prescribes a pair $\zug{b, u}$, where $b$ is a bid and $u \in V$ is the vertex to move to upon winning, $\dstrat$ prescribes $\zug{\lfloor b\rfloor, u}$. 
%	
%To prove the left-hand side inequality, we show that if \PO has initial budget strictly less than $t_v \cdot (B_2-\textnormal{max-path}(v))$ and \PT has initial budget $B_2$, then \PT can win by following the winning strategy in a {\em continuous-bidding} game and rounding the bids up. More formally, let $\cstrat$ be a winning strategy for \PT under continuous-bidding when the game starts in $v$, \PO's initial budget is at most $t_v \cdot (B_2-\textnormal{max-path}(v))$  and \PT's initial budget is $B_2 - \textnormal{max-path}(v)$. Suppose that $\cstrat$ prescribes $\zug{b, u}$, then $\dstrat$ for \PT prescribes $\zug{\lceil b\rceil, u}$. 
%The fact that \PT always has enough budget to bid $\lceil b\rceil$ follows from the fact that the game necessarily ends within $\textnormal{max-path}(v)$ turns.
%\end{proof}

\begin{proof}
	{\em Right-hand-side inequality.} We first prove the right-hand-side inequality. To prove that $T_v(B_2) / B_2 \leq t_v$, it suffices to prove that, for each $\epsilon > 0$, \PO has a winning strategy if the game starts in $v$, \PO's initial budget is at least $t_v\cdot B_2 + \epsilon$ and \PT's initial budget is $B_2$. If we are able to prove this claim, it will then follow that $T_v(B_2) / B_2 \leq t_v + \epsilon$ holds for every $\epsilon>0$, therefore $T_v(B_2) / B_2 \leq t_v$.
	
	Fix $\epsilon > 0$. We construct the winning strategy of \PO as follows. By the definition of the continuous threshold $t_v$, we know that \PO has a winning strategy in the poorman {\em continuous-bidding} game. Moreover, it was shown in~\cite[Theorem 7]{LLPSU99} that \PO has a {\em memoryless} winning strategy, i.e.~a strategy in which the bids and token moves in each turn depend only on the position of the token and the players' budgets. We take such strategy $\sigma_{\textrm{cont}}$. We then construct a winning strategy $\sigma_{\textrm{disc}}$ of \PO in the poorman discrete-bidding game as follows:
	\begin{itemize}
		\item At each turn, if \PO under $\sigma_{\textrm{cont}}$ would bid $b$, then \PO under $\sigma_{\textrm{disc}}$ bids $\lfloor b \rfloor$.
		\item If \PO wins the bidding, then the token is moved to the vertex perscribed by $\sigma_{\textrm{cont}}$. 
	\end{itemize}
	
	We show that $\sigma_{\text{disc}}$ is indeed winning for \PO. To do this, we prove that $\sigma_{\textrm{disc}}$ preserves the invariant that, whenever the token is in some vertex $v'$, the ratio of players' budgets is positive and strictly greater than $t_{v'}$. This invariant implies that the token does not reach the sink state as the continuous threshold in the sink state is infinite. Thus, as a poorman discrete-bidding game ends in finitely many steps, this then implies that the game must eventually reach \PO's target state and therefore that $\sigma_{\text{disc}}$ is winning for \PO.
	
	We prove the invariant by the induction on the length of the game play. The base case holds by the assumption that, in the initial vertex $v$, \PO's initial budget is at least $t_v\cdot B_2 + \epsilon$ and \PT's initial budget is $B_2$. Now, for the induction hypothesis, suppose that the token is in vertex $v'$ after finitely many steps, with \PT's budget $B_2'$ and \PO's budget at least $t_{v'} \cdot B_2' + \epsilon'$ for some $\epsilon'>0$. We show that the invariant is preserved in the next step. Suppose that \PO under $\sigma_{\textrm{disc}}$ bids $\lfloor b \rfloor$ where $b$ is the bid of \PO under $\sigma_{\textrm{cont}}$. In what follows, we use the fact that $\sigma_{\textrm{cont}}$ preserves the ratio invariant which was established in the proof of~\cite[Theorem 7]{LLPSU99}. We distinguish between two cases:
	\begin{itemize}
		\item If \PO wins the bidding and moves the token to $v''$, then the ratio of budgets at the next step is
		\begin{equation*}
			\begin{split}
				\frac{t_{v'} \cdot B_2' + \epsilon' - \lfloor b\rfloor}{B_2'} &\geq \frac{t_{v'} \cdot B_2' + \epsilon' - b}{B_2'} \\
				&> \frac{t_{v'} \cdot B_2' - b}{B_2'} \\
				&\geq t_{v''}.
			\end{split}
		\end{equation*}
		The last inequality follows from the fact that the fraction in the second line is the subsequent ratio of budgets under the continuous-bidding winning strategy.
		\item If \PT wins the bidding, then \PT had to bid at least $\lfloor b \rfloor + 1$. Suppose that \PT moves the token to $v''$ upon winning. Then the ratio of budgets at the next step is at least
		\begin{equation*}
			\begin{split}
				\frac{t_{v'} \cdot B_2' + \epsilon'}{B_2' - \lfloor b\rfloor - 1 } &\geq \frac{t_{v'} \cdot B_2' + \epsilon'}{B_2' - b} \\
				&> \frac{t_{v'} \cdot B_2'}{B_2' - b} \\
				&\geq t_{v''}.
			\end{split}
		\end{equation*}
		The first inequality follow by observing that $\lfloor b \rfloor + 1 \geq  b$, and the third inequality follows from the fact that the second fraction is an upper bound on the subsequent ratio of budgets under the continuous-bidding winning strategy.
	\end{itemize}
	Hence, the invariant claim for poorman discrete-bidding follows by induction on the length of the game play, thus $\sigma_{\text{disc}}$ is indeed winning for \PO and the right-hand-side inequality in the theorem follows.
	
	\noindent {\em Left-hand-side inequality.} We now prove the left-hand-side inequality. If $B_2 < \textrm{max-path}(v) $, the claim trivially follows. Otherwise, it suffices to prove that \PT has a winning strategy if the game starts in $v$, \PO's initial budget is strictly less than $t_v\cdot (B_2 - \textrm{max-path}(v))$ and \PT's initial budget is $B_2$.
	
	Suppose that $B_2 \geq \textrm{max-path}(v)$ and let $B_1  < t_v\cdot (B_2 - \textrm{max-path}(v) )$ be the initial budget of \PO. We construct the winning strategy of \PT as follows. Let $\sigma_{\textrm{cont}}$ be the memoryless winning strategy of \PT under {\em continuous-bidding} when the game starts in $v$, \PT's initial budget is $B_2 - \textrm{max-path}(v)$ and \PO's initial budget is $B_1$. Since  $B_1  < t_v\cdot (B_2 - \textrm{max-path}(v))$, such a strategy exists by the definition of the continuous threshold $t_v$ and by~\cite[Theorem 7]{LLPSU99} which shows that it is possible to pick a memoryless winning strategy. We then construct a winning strategy $\sigma_{\textrm{disc}}$ of \PT in the poorman discrete-bidding game when \PT has initial budget $B_2$ and \PO has initial budget $B_1$ as follows:
	\begin{itemize}
		\item At each turn, if \PT under $\sigma_{\textrm{cont}}$ would bid $b$, then \PT under $\sigma_{\textrm{disc}}$ bids $\lceil b \rceil$.
		\item If \PT wins the bidding, then the token is moved to the vertex perscribed by $\sigma_{\textrm{cont}}$. 
	\end{itemize}
	Note that, if we show that the bids $\lceil b \rceil$ under $\sigma_{\textrm{disc}}$ are legal (i.e.~do not exceed available budget), then $\sigma_{\textrm{disc}}$ is clearly winning for \PT. Indeed, $\sigma_{\textrm{cont}}$ is winning for \PT, the bids of $\sigma_{\textrm{disc}}$ are always as least as big as those of $\sigma_{\textrm{cont}}$ and the token moves under two strategies coincide. So we only need to prove that the bids $\lceil b \rceil$ under $\sigma_{\textrm{disc}}$ are legal. But this follows from the fact that the underlying graph is a DAG and thus the game takes at most $\textrm{max-path}(v)$ turns before it reaches either the target or the sink vertex. Hence, as the bids are legal under $\sigma_{\textrm{cont}}$ when \PT has initial budget $B_2 - \textrm{max-path}(v)$, the bids are also legal under $\sigma_{\textrm{disc}}$ as \PT can bid $b + 1 \geq \lceil b \rceil$ in each turn. This concludes the proof of the left-hand-side of the inequality.
\end{proof}



%We show later that these bounds are tight.
An immediate corollary of Thm.~\ref{stm:pipe} is that the ratio $T_v(B_2)/B_2$ tends to $t_v$. %short? the continuous-bidding threshold ratio $t_v$.
%First, the discrete-bidding ratio $T_v(B_2)/B_2$ at every vertex $v$ converges to the continuous threshold $t_v$  at $v$ as $B_2$ tends to infinity, since $\textrm{max-path}(v)/B_2 \leq \frac{|V|-1}{B_2}$ converges to $0$.

\begin{corollary}[Convergence to continuous ratios]\label{cor:convergence}
	For every $v \in V$ we have $\lim_{B_2\rightarrow\infty} T_v(B_2) / B_2 = t_v$.
\end{corollary}



\subsection{Periodicity of Threshold Budgets}
The following theorem shows that for any fixed $v\in V$ the function $T_v(\cdot)$ that yields the threshold budgets exhibits an eventually periodic behavior, as seen in Example~\ref{ex:2race+1}. %we have $B = 0$, $u_x = 45$, and $u_y = 32$.

\begin{restatable}[Periodicity theorem]{theorem}{periodicity} \label{thm:periodicity}
For any vertex $v\in V$ there exist values $B, u_x,u_y \in\Nat$ such that for all $B_2\ge B$ we have $T_v(B_2+u_x)=T_v(B_2)+u_y$.
Moreover, the values $B$, $u_x$, $u_y$ can be computed in polynomial time.
\end{restatable}

\stam{
\begin{proof}[Proof sketch] 
(See supplementary material for the full proof.)
The proof is by induction with respect to the topological order of the graph.
If $v$ is a leaf, then the claim is obvious. % (and we can set $B=0$).
Consider $v$ that is not a leaf.
The proof is based on three ingredients.
First, intuitively, when the children of $v$ have different threshold ratios then their pipes diverge.
Let $v^-$ and $v^+$ respectively denote the children whose pipe is lowest and highest.
By Thm.~\ref{stm:pipe}, under discrete-bidding, for large budgets, \PO and \PT will respectively proceed to $v^-$ and $v^+$ upon winning the bidding in $v$.
%Second, we use the fact that a minimum of two periodic functions is itself periodic to take care of the case that the threshold ratios are not unique. 


Second, we show that if $v^-$ satisfies $T_{v^-}(B_2+u^-_x)=T_{v^-}(B_2)+u^-_y$ and
$v^+$ satisfies $T_{v^+}(B_2+u^+_x)=T_{v^+}(B_2)+u^+_y$ (both for large enough $B_2$),
then $v$ satisfies the same equality with $u_x=u^-_x\cdot (u^+_x+u^+_y)$ and $u_y=u^+_y\cdot(u^-_x+u^-_y)$. We illustrate the idea using~\cref{fig:u-climbing}, which depicts a configuration $c=\zug{v,B_1,B_2}$ as a point $[B_2, B_1]$ in the plane. Consider first the left image. Suppose that \PO bids $b$ from $\zug{v, B_1, B_2}$ (see point $P$). The case that \PO wins the bidding corresponds to ``stepping down'' from $[B_2, B_1]$ to $[B_2, B_1-b]$. Note that the token moves to $v^-$. Thus, a necessary condition for $B_1 \geq T_v(B_2)$ is $B_2 - b \geq T_{v^-}(B_2)$. The second case is when \PT bids $b+1$ and wins the bidding, which corresponds to ``stepping left'' to $[B_2 - (b+1), B_1]$, the token moves to $v^+$, and we obtain a second necessary condition $B_1 \geq T_{v^+}(B_2 - (b+1))$. Then, given configurations on the thresholds of $v^-$ and $v^+$ (depicted as $Q$ and $R$), the ``lowest'' point that satisfies both conditions is a point on the threshold of $v$. The right part of~\cref{fig:u-climbing} shows how the period of $T_v$ is determined by the periods of $T_{v^+}$ and $T_{v^-}$.
% Figure environment removed


Third, if multiple children have the same threshold ratio, we reduce to the previous case by using the fact that a minimum of two periodic functions over integers is itself periodic. %to take care of the case that the threshold ratios are not unique. 
%This completes the induction proof.
\end{proof}}

\begin{proof}
	We proceed by induction with respect to the topological order of the graph.
	For the target $v$ we set $(B,u_x,u_y)=(0,1,0)$ and for the sink we set $(B,u_x,u_y)=(0,0,1)$.
	Next, given a non-leaf vertex $v$, suppose that among its children there are $k$ distinct continuous threshold ratios, and denote them by $t_1<t_2<\dots<t_k$.
	Note that whenever \PO wins a bidding at vertex $v$ (while \PT has budget $B$), he moves the token to a child $u\in N(v)$ of $v$ with minimal value $T_u(B)$.
	We claim that for large enough $B$, the only relevant children $u$ are those with $t_{u}=t_1$.
	Indeed, consider two children $u,w\in N(v)$, one with $t_u=t_1$ and the other one with $t_w\ne t_1$.
	Then by \cref{stm:pipe}, for $B>t_2\cdot n / (t_2-t_1)$ we have
	\[T_w(B) \ge t_w\cdot (B-n) \ge t_2\cdot (B-n) > t_1\cdot B \ge T_u(B),
	\]
	thus \PO would prefer to move to $u$ rather than to $w$.
	
	Next, let $V^-=\{u\in N(v) \mid t_u=t_1\}$ be a set of those ``relevant'' children of $v$.
	We say that a function $f\colon \Nat\to\Nat$ is \textit{$(x,y)$-climbing} if it satisfies $f(B+x)=f(B)+y$ for all large enough $B$.
	By induction assumption, for each $u\in V^-$ the function $T_{u}(B)$ is $(u_x,u_y)$-climbing with a slope $u_y/u_x$. By \cref{stm:pipe}, this slope is equal to $t_1$.
	Thus, the function $T_u(B) - t_1\cdot B$ is periodic with period $u_x$.
	The function $\min_{u\in V^-}\{T_u(B) - t_1\cdot B\}$ is then periodic with the period equal to the least common multiple $l=\operatorname{lcm}\{u_x\mid u\in V^-\}$ of the respective periods.
	Therefore, the function $T_{v^-}(B) := \min_{u\in V^-}\{T_u(B)\}$ is $(l,l\cdot t_1)$-climbing.
	To summarize, the moves of \PO (upon winning a bidding) are faithfully represented by him moving the token to a vertex $v^-$ for which the threshold budgets are $(l,l\cdot t_1)$-climbing.
	Completely analogously we show that the moves of \PT are faithfully represented by her moving the token to a vertex $v^+$ with $(l',l'\cdot t_k)$-climbing threshold budgets.
	
	From now on, for ease of notation suppose that $T_{v^-}$ is $(u_x,u_y)$-climbing and that $T_{v^+}$ is $(w_x,w_y)$-climbing (for some integers $u_x,u_y,w_x,w_y$). We will show that $T_v$ is $(u_x\cdot(w_x+w_y), w_y\cdot(u_x+u_y))$-climbing. This will complete the induction proof.
	
	To prove this claim, it is convenient to represent each configuration $c=\zug{v,B_1,B_2}$ as a point in the plane with coordinates $[B_2,B_1]$, see~\cref{fig:u-climbing}.
	\PO can then force a win from a configuration $c=\zug{v,B_1,B_2}$ if and only if 
	the point $P=[B_2,B_1]$ lies on or above the threshold function~$T_v$.
	
	
	% Figure environment removed
	
	
	Take any point $P=(P_x,P_y)$.
	Let $Q=(Q_x=P_x,Q_y)$ be the furthest point below $P$ that still lies on or above $T_{v^-}$, and
	let $R=(R_x,R_y=P_y)$ be the closest point to the left of $P$ that lies on or above $T_{v^+}$.
	Note that $P$ lies on or above $T_v$ if and only if the distances $d_y:=P_y-Q_y$ and $d_x:=P_x-R_x$ satisfy $d_x\le d_y+1$. Indeed, if the inequality holds then \PO can force a win by bidding $d_y$, whereas in the other case \PT can force a win by bidding $d_y+1$. 
	
	As a final step, we show that at some point further along the curves $T_{v^-}$ and $T_{v^+}$, the two distances $d_x$, $d_y$ increase by the same margin.
	Specifically, chain $u_x+u_y$ copies of a vector $(w_x,w_y)$ starting from point $R$ to get to point $R'=(R'_x,R'_y)$, and, similarly,
	chain $w_x+w_y$ copies of $(u_x,u_y)$ from $Q$ to $Q'=(Q'_x,Q'_y)$.
	Finally, let $P'$ be the point above $Q'$ and to the right of $R'$.
	Then a straightforward algebraic manipulation shows that distances from $P'$ to $Q'$ and to $R'$ both increased by $u_xw_y - w_xu_y$.
	Indeed, without loss of generality set $Q=(d_x,0)$ and $R=(0,d_y)$.
	Then we have
	\[Q'=(d_x+ (w_x+w_y)u_x, 0+(w_x+w_y)u_y)\]
	and
	\[R'=( (u_x+u_y)w_x, d_y+(u_x+u_y)w_y),\]
	so 
	\[P'=(Q'_x,R'_y)=(d_x+ (w_x+w_y)u_x, d_y+(u_x+u_y)w_y)\]
	and finally
	\begin{align*}
		P'_y-Q'_y&=d_y+(u_x+u_y)w_y - (w_x+w_y)u_y\\
		&= d_y + (u_xw_y - w_xu_y),\\
		P'_x-R'_x &= d_x+ (w_x+w_y)u_x - (u_x+u_y)w_x \\
		&= d_x + (w_yu_x - u_yw_x),
	\end{align*}
	concluding the proof. 
\end{proof}


This result implies that for each $v\in V$, the function $T_v(\cdot)$ can be finitely represented: let $B$ be \PT's budget when the period ``kicks in'', then for all $B' \leq B$, the value $T_v(B')$ is stored explicitly and these values can be extrapolated to find $T_v(B'')$ for $B'' > B$. %Note that this means that an algorithm to compute $T_v(B_2)$ can terminate once it reaches $B$.

We point out that periodicity may indeed appear only ``eventually'', as illustrated by \cref{fig:eventually_periodic}; namely, only at $B = 7$ state $(2, 2)$ continuously is an optimal choice and the periodic behaviour is observed.
Replacing $\race{5,4}$ with $\race{2x + 1, 2x}$ leads to quickly growing periodicity thresholds $B$.
Finally, we note that on non-DAGs, the behaviour is not necessarily periodic, as illustrated by \cref{thm:tug2} below.


\stam{
Let $v$ that is not a leaf, by using Thm.~\ref{stm:pipe} and the fact that a minimum of two periodic functions is itself periodic, we are able to reduce to the case in which there exists
a single descendant $v^-$ of $v$ where \PO always moves the token (upon winning a bidding), and
a single descendant $v^+$ of $v$ where \PT always moves the token.
%We say that a function $f\colon\Nat\to\Nat$ is $(u_x,u_y)$-\textit{climbing} if there exists an integer $B$ such that $f(B_2+u_x)=f(B_2)+u_y$ holds for all $B_2>B$.
Then we show that if $v^-$ satisfies $T_{v^-}(B_2+u^-_x)=T_{v^-}(B_2)+u^-_y$ and
$v^+$ satisfies $T_{v^+}(B_2+u^+_x)=T_{v^+}(B_2)+u^+_y$ (both for all large enough $B_2$),
then $v$ satisfies the same equality with $u_x=u^-_x\cdot (u^+_x+u^+_y)$ and $u_y=u^+_y\cdot(u^-_x+u^-_y)$. This completes the induction proof.
\end{proof}
As a consequence of Thm.~\ref{thm:periodicity}, we obtain that for each $v\in V$ the whole function $T_v(\cdot)$ can be represented using a finite amount of information -- its values before the period kicks in, and its values over one period.\todo{Sell this more; maybe "finite representation"}
Moreover, the periodic behaviour indeed appears only eventually, as illustrated by Figure~\ref{fig:eventually_periodic}.
There, it takes up to $B = 79$ until state $(2, 2)$ continuously is an optimal choice.
Only then do we observe periodic behaviour.
}



%For instance, in a game where the winner of the first bidding can choose between playing \race(11,10) or \race(2,2), the 
%
%
%
% Figure environment removed
\stam{We evaluate the root node of a game offering a choice between playing \race{11,10} and \race{2,2}. We depict the ``winning moves'' of \PO, i.e.\ for each budget of \PT, which of the two options \PO has to choose when winning with an optimal bid, denoted $v^-$ in the proof of Thm.~\ref{thm:periodicity}.}


%We draw important corollaries from this result.
% Apart from its theoretical importance, the result has practical consequences: it gives rise to an approximation algorithm for computing the thresholds that runs in linear time. In contrast, the best algorithm to compute exact thresholds on DAGs (presented in the following section) is linear in the numerical value of the input budgets, which is exponential when the budgets are given in binary.

%
%We first review the results on thresholds in continuous-bidding games. 
%\begin{theorem}
%{\bf (Continuous thresholds).} \cite{LLPSU99}
%Consider a poorman continuous-bidding  game $\G$. When \PLi's initial budget is $B_i$, for $i \in \set{1,2}$, the {\em ratio} is $B_1/(B_1 + B_2)$. For each vertex $v$, there exists a {\em continuous threshold}, which is $t_v \in [0,1]$ such that for all $\epsilon >0$, \PO wins when the ratio is $t_v+\epsilon$, and \PT wins when it is $t_v-\epsilon$. Moreover, when $\G$ is a DAG, computing the continuous thresholds can be done in time that is linear in the number of vertices in $\G$.
%\end{theorem}
%
%The following theorem is the main result of this section. We call it the \textit{pipe theorem} since it shows that the threshold budgets lie in a ``pipe'' below the continuous threshold (see Example~\ref{ex:2race+1}). For a game $\G = \zug{V, E, t, s}$ and a vertex $v$, let {\em $\textnormal{max-path}(v)$} denote the length of the longest path from $v$ to the target $t$ or the sink $s$. Note that $\textnormal{max-path}(v) \leq |V| - 1$. 
%
%\todo{double check that we are using the ratios in the same way (I don't think so!)}
%
%\begin{restatable}[Pipe theorem]{theorem}{pipe} \label{stm:pipe}
%	Consider a poorman discrete-bidding game \(\calG = \zug{V, E, t, s}\), where $\zug{V,E}$ is a DAG. Consider a vertex $v\in V$ and let $t_v$ denote the continuous threshold at $v$. Then, for every initial budget $B_2\in\mathbb{N}$ of \PT, we have
%	\[ t_v \cdot (1- \frac{\textnormal{max-path}(v)}{B_2})  \leq \frac{T_v(B_2)}{B_2} \leq t_v \]
%Furthermore, the right-hand-side inequality holds even when $\G$ is not a DAG.
%\end{restatable}
%
%\begin{proof}[Proof sketch]
%	In what follows, we outline the key ideas behind the proof. The full proof can be found in the supplementary material. To prove the right-hand-side inequality, we show that if \PO has initial budget of at least $t_v \cdot B_2$ then \PO can win by following a winning strategy in the {\em continuous-bidding} game and {\em rounding down} the bids. More formally, let $\cstrat$ be a winning strategy for \PO under continuous-bidding when the game starts in $v$, \PO's initial budget is at least $t_v \cdot B_2$, and \PT's initial budget is $B_2$. We define a \PO strategy $\dstrat$ as follows. Whenever $\cstrat$ prescribes a pair $\zug{b, u}$, where $b$ is a bid and $u \in V$ is the vertex to move to upon winning, then $\dstrat$ prescribes $\zug{\lfloor b\rfloor, u}$. 
%	
%To prove the left-hand-side inequality, we show that if \PO has initial budget strictly less than $t_v \cdot (B_2-\textnormal{max-path}(v))$ and \PT has initial budget $B_2$, then \PT can win by following the winning strategy in the {\em continuous-bidding} game with the initial budget $B_2-\textnormal{max-path}(v)$ and rounding the bids up. More formally, let $\cstrat$ be a winning strategy for \PT under continuous-bidding when the game starts in $v$, \PO's initial budget is at most $t_v \cdot (B_2-\textnormal{max-path}(v))$  and \PT's initial budget is $B_2-\textnormal{max-path}(v)$. Suppose that $\cstrat$ prescribes $\zug{b, u}$, then $\dstrat$ for \PT prescribes $\zug{\lceil b\rceil, u}$. 
%The fact that \PT always has enough budget to bid $\lceil b\rceil$ follows from the fact that the game necessarily ends within $\textnormal{max-path}(v)$ turns.
%\end{proof}
%
%Next, we draw three useful corollaries from the pipe theorem.
%
%\subsubsection{Large budgets}
%%The pipe theorem has two useful corollaries. The first is that 
%%its bounds are asymptotically tight, in the sense that they imply that 
%First, the discrete-bidding ratio $T_v(B_2)/B_2$ at every vertex $v$ converges to the continuous threshold $t_v$  at $v$ as $B_2$ tends to infinity, since $\textrm{max-path}(v)/B_2 \leq \frac{|V|-1}{B_2}$ converges to $0$.
%
%\begin{corollary}[Convergence to continuous-thresholds]\label{cor:convergence}
%	$\lim_{B_2\rightarrow\infty}\frac{T_v(B_2)}{B_2} = t_v$, for every $v\in V$.
%\end{corollary}
%
%\subsubsection{Stabilization}
%Intuitively, we say that a vertex $v$ is {\em stable} if there is a neighbor $u \in V$ of $v$ that \PO always proceeds to upon winning the bidding at $v$, no matter the budgets. We note that all vertices are stable under continuous-bidding. Under discrete-bidding, the definition needs care: we say that $v$ {\em stabilizes after} $B_2$ if there is a neighbor $u$ of $v$ such that for every $B_2' \geq B_2$, there is a winning \PO strategy from configuration $\zug{v, T_v(B'_2), B'_2}$ that proceeds to $u$ upon winning the bidding in $v$. The second corollary of the pipe theorem gives a sufficient condition for stabilization. 
%
%%concerns \PO's selection of the successor vertex to which the token should be moved upon winning the bidding. %For each $v\in V$, denote by $v^-$ a neighbouring vertex of $v$ at which the continuous threshold is minimized. We denote one such vertex to be $v^-$ if the minimum is attained at multiple neighbouring vertices.
%%In particular, we show that there exists a lower bound $B$ on \PT's initial budget such that, whenever \PT's initial budget is $B_2 \geq B$ and \PO's initial budget exceeds the threshold budget $T_v(B_2)$ at some initial vertex $v$, then \PO has a winning strategy that moves the token to a neighbour of $v$ which minimizes the continuous threshold among all neighbours of $v$. We call the following corollary the stabilization lemma, since it shows that there exists a lower bound $B$ on \PT's initial budget beyond which the token to which \PO moves the token upon winning the bid {\em stabilizes} within the set of neighbouring vertices at which the continuous threshold is minimized.
%%In other words, there exists a lower bound $B$ on \PT's initial budget beyond which the successor state $v^-$ of $v$ under some \PO's winning strategy {\em stabilizes} and does not depend on exact values of initial budgets.
%
%\begin{corollary}[Stabilization lemma]\label{cor:stabilization}
%	%For each $v \in V$, let $v^-$ be some neighbouring vertex of $v$ in $\calG$ at which the continuous threshold is minimized, i.e.~$v^- \in \text{argmin}_{(v,v')\in E} t_{v'}$, and let
%	Let
%	\[ B = 1 + \max_{v',v''\in V, t_{v'} > t_{v''}} \textnormal{max-path}(v') \cdot \frac{t_{v'}}{t_{v'} - t_{v''}}. \]
%	Then, for each initial vertex $v \in V$, initial budget $B_2 \geq B$ of \PT and initial budget $B_1 \geq T_v(B_2)$ of \PO, there exists a winning strategy of \PO that upon winning the bidding moves the token from $v$ to $v^- \in \text{argmin}_{(v,v')\in E} t_{v'}$.
%\end{corollary}
%\begin{proof}
%	Our choice of $B$ ensures that, whenever $t_{v'} > t_{v''}$, we also have $t_{v'} \cdot (B - \textnormal{max-path}(v')) > t_{v''} \cdot B$. Hence, for every initial budget $B_2 \geq B$ of \PT and for every pair of vertices $v',v''\in V$ with $t_{v'} \neq t_{v''}$, we have that the two interval bounds $t_{v'} \cdot (B_2 - \textnormal{max-path}(v')) \leq T_{v'}(B_2) \leq t_{v'} \cdot B_2$ and $t_{v''} \cdot (B_2 - \textnormal{max-path}(v'')) \leq T_{v''}(B_2) \leq t_{v''} \cdot B_2$ on threshold budgets for discrete-bidding games are disjoint. But this also implies that, whenever $t_{v'} > t_{v''}$ and $B_2 \geq B$, we must also have $T_{v'}(B_2) > T_{v''}(B_2)$. Hence, as a winning strategy of \PO moves the token to a vertex that minimizes the discrete-bidding threshold budget needed to win, this monotonicity result also implies that the winning strategy should move the token to a vertex that minimizes the continuous threshold.
%\end{proof}
%
%(((Instability in race games?)))
%
%%\subsubsection{Finite representation of $T_v(\cdot)$}
%\subsubsection{Eventual periodicity}
%As a third corollary, we show that for any $v\in V$ the function $T_v(\cdot)$ is eventually ``periodic'', and thus it can be represented using a finite amount of information -- its values before the period kicks in, and its values over one period (see Example~\ref{ex:2race+1}).
%
%\begin{corollary}\label{cor:periodicity}
%For any DAG and any vertex $v$ there exist values $B, u_x,u_y \in\Nat$ such that for all $B_2>B$ we have $T_v(B_2+u_x)=T_v(B_2)+u_y$.
%\end{corollary}
%
%The intuition behind the proof is that on DAGs, all continuous thresholds are rational numbers and so,
%once a node $v$ has stabilized, the situation will ``repeat'' with a certain integer period length that is at most the lowest common multiple of the period lengths of the descendants. %. In particular, if the threshold budgets at
% In particular, a suitable value of $B$ is the one given in Corollary~\ref{cor:stabilization}.
%Moreover, the values $u_x$, $u_y$ can be computed efficiently similarly as the continuous thresholds.
%See Appendix for a full proof.
%














\stam{%OLD


\section{Threshold Budgets for Games on DAGs}\label{sec:pipeandstabilization}

We now focus on poorman discrete-bidding games played on directed acyclic graphs (DAGs). The central result of this section are upper and lower bounds that relate the threshold budgets in poorman discrete-bidding games to the threshold budgets in poorman continuous-bidding games. Furthermore, we show that our bounds are asymptotically tight. %In particular, we show that as the budget of Player~2 grows, the ratio of the initial budgets that Player~1 needs to win in the poorman discrete-bidding game converges to the ratio in the poorman continuous-bidding game.

The practical importance of our bounds is that they yield a practical algorithm for computing asymptotically tight bounds on threshold budgets. In particular, to compute bounds on threshold budgets in poorman discrete-bidding games, one can use our bounds together with the method for computing threshold budgets in poorman continuous-bidding games~\cite{xxx}. CITE THE 90s PAPER The complexity of the algorithm is in \textsc{PSPACE}.

The following theorem is the main result of this section. We call it the \textit{Pipe theorem}, as it provides a two-sided bound on threshold budgets in poorman discrete-bidding games where both bounds have linear dependence on the initial budget of \PT. Recall, given a continuous-bidding poorman game, the continuous threshold at a vertex $v$ is a value $t_v \in [0,1]$ such that when the ratio between the two players' budgets is $t_v+\epsilon$, \PO wins, and when the ratio is $t_v-\epsilon$, \PT wins.

\begin{theorem}[Pipe theorem]
	Consider a poorman discrete-bidding game \(\calG = \zug{V, E, B_1, B_2, \calO}\) for which the underlying game graph $(V,E)$ is a DAG. Let $v\in V$ be a vertex and let $t_v$ denote the continuous threshold at $v$. Then, for every initial budget $B_2\in\mathbb{N}$ of \PT, we have
	\[ t_v \cdot (1- \frac{\textnormal{max-path}(v)}{B_2})  \leq \frac{T_v(B_2)}{B_2} \leq t_v, \]
	where {\em $\textnormal{max-path}(v)$} is the length of the longest path from $v$ to the target vertex or the sink vertex of the game. Note that $\textnormal{max-path}(v) \leq |V| - 1$. Furthermore, the right-hand-side inequality holds even if the game graph $(V,E)$ is not a DAG.
\end{theorem}

\begin{proof}[Proof sketch]
	In what follows, we outline the key ideas behind the proof. The full proof can be found in the Appendix. To prove the right-hand-side inequality, we show that if \PO has initial budget of at least $t_v \cdot B_2$ then \PO can win by following the winning strategy in the {\em continuous-bidding} game and {\em rounding down} the bids. In other words, if $\sigma_{\textrm{continuous}}$ is a winning strategy for \PO under continous-bidding when the game starts in $v$, \PO's initial budget is at least $t_v \cdot B_2$ and \PT's initial budget is $B_2$, then the strategy $\sigma_{\textrm{discrete}}$ of \PO which
	\begin{itemize}
		\item bids $\lfloor b\rfloor$ whenever $\sigma_{\textrm{continuous}}$ bids $b$, and
		\item upon winning moves the token to the vertex perscribed by $\sigma_{\textrm{continuous}}$
	\end{itemize}
	is winning for \PO under discrete-bidding. To prove the left-hand-side inequality, we show that if \PO has initial budget strictly less than $t_v \cdot (B_2-\textnormal{max-path}(v))$ and \PT has initial budget $B_2$, then \PT can win by following the winning strategy in the {\em continuous-bidding} game with the initial budget $B_2-\textnormal{max-path}(v)$ and {\em rounding up} the bids. In other words, if $\sigma_{\textrm{continuous}}$ is a winning strategy for \PT under continuous-bidding when the game starts in $v$, \PO's initial budget is at least $t_v \cdot (B_2-\textnormal{max-path}(v))$  and \PT's initial budget is $B_2-\textnormal{max-path}(v)$, then the strategy $\sigma_{\textrm{discrete}}$ of \PO which
	\begin{itemize}
		\item bids $\lceil b\rceil$ whenever $\sigma_{\textrm{continuous}}$ bids $b$, and
		\item upon winning moves the token to the vertex perscribed by $\sigma_{\textrm{continuous}}$
	\end{itemize}
	is winning for \PT under-discrete bidding with initial budget $B_2$. The fact that \PT always has enough budget to bid $\lceil b\rceil$ follows from the fact that the game is played on a DAG so needs to end in at most $\textnormal{max-path}(v)$ turns.
\end{proof}

Pipe theorem has two useful corollaries. The first is that its bounds are asymptotically tight, in the sense that they imply that the discrete-bidding ratio $T_v(B_2)/B_2$ at every vertex $v$ converges to the continuous threshold $t_v$  at $v$ as $B_2\rightarrow\infty$, since $\textrm{max-path}(v)/B_2 \leq \frac{|V|-1}{B_2}$ converges to $0$.

\begin{corollary}
	$\lim_{B_2\rightarrow\infty}\frac{T_v(B_2)}{B_2} = t_v$ for every $v\in V$.
\end{corollary}

The second corollary concerns \PO's selection of the successor vertex to which the token should be moved upon winning the bidding. %For each $v\in V$, denote by $v^-$ a neighbouring vertex of $v$ at which the continuous threshold is minimized. We denote one such vertex to be $v^-$ if the minimum is attained at multiple neighbouring vertices.
In particular, we show that there exists a lower bound $B$ on \PT's initial budget such that, whenever \PT's initial budget is $B_2 \geq B$ and \PO's initial budget exceeds the threshold budget $T_v(B_2)$ at some initial vertex $v$, then \PO has a winning strategy that moves the token to a neighbour of $v$ which minimizes the continuous threshold among all neighbours of $v$. We call the following corollary the stabilization lemma, since it shows that there exists a lower bound $B$ on \PT's initial budget beyond which the token to which \PO moves the token upon winning the bid {\em stabilizes} within the set of neighbouring vertices at which the continuous threshold is minimized.
%In other words, there exists a lower bound $B$ on \PT's initial budget beyond which the successor state $v^-$ of $v$ under some \PO's winning strategy {\em stabilizes} and does not depend on exact values of initial budgets.

\begin{corollary}[Stabilization lemma]
	%For each $v \in V$, let $v^-$ be some neighbouring vertex of $v$ in $\calG$ at which the continuous threshold is minimized, i.e.~$v^- \in \text{argmin}_{(v,v')\in E} t_{v'}$, and let
	Let
	\[ B = 1 + \max_{v',v''\in V, t_{v'} > t_{v''}} \textnormal{max-path}(v') \cdot \frac{t_{v'}}{t_{v'} - t_{v''}}. \]
	Then, for each initial vertex $v \in V$, initial budget $B_2 \geq B$ of \PT and initial budget $B_1 \geq T_v(B_2)$ of \PO, there exists a winning strategy of \PO that upon winning the bidding moves the token from $v$ to $v^- \in \text{argmin}_{(v,v')\in E} t_{v'}$.
\end{corollary}

\begin{proof}
	Our choice of $B$ ensures that, whenever $t_{v'} > t_{v''}$, we also have $t_{v'} \cdot (B - \textnormal{max-path}) > t_{v''} \cdot B$. Hence, for every initial budget $B_2 \geq B$ of \PT and for every pair of vertice $v',v''\in V$ with $t_{v'} \neq t_{v''}$, we have that the two interval bounds $t_{v'} \cdot (B_2 - \textnormal{max-path}) \leq T_{v'}(B_2) \leq t_{v'} \cdot B_2$ and $t_{v''} \cdot (B_2 - \textnormal{max-path}) \leq T_{v''}(B_2) \leq t_{v''} \cdot B_2$ on threshold budgets for discrete-bidding games are disjoint. But this also implies that, whenever $t_{v'} > t_{v''}$ and $B_2 \geq B$, we must also have $T_{v'}(B_2) > T_{v''}(B_2)$. Hence, as a winning strategy of \PO moves the token to a vertex that minimizes the discrete-bidding threshold budget needed to win, this monotonicity result also implies that the winning strategy should move the token to a vertex that minimizes the continuous threshold.
\end{proof}
}
