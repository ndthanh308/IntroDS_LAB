\begin{center}
	\Large\textbf{Supplementary Material}
\end{center}

\section{Proof of Section~\ref{sec:determinacy}}

\stam{
Consider a reachability bidding game \(\G = \zug{V, E, t, s}\) where \PO always wins ties, and consider a configuration \(c = \zug{v, B_1, B_2}\). We show that if \PO does not win from $c$, then \PT wins from $c$. The {\em bidding matrix} $M_c$ intuitively captures the winner of the game with respect to each pair of bids by the two players at $c$. Formally, $M_c$ has $B_1$ rows and $B_2$ columns, each row corresponds to a \PO bid, and each column to a \PT bid. An entry in \(M_c\) is either \(1\) or \(2\), defined as follows. Let $\zug{b_1, b_2} \in \set{0,\ldots, B_1} \times \set{0,\ldots, B_2}$ and let $\zug{B'_1, B'_2}$ be the updated budgets after the players bid $\zug{b_1, b_2}$ at $c$. We set the \((b_1, b_2)^{\text{th}}\) entry in $M_c$ to be $1$ iff (1) either $b_1 \geq b_2$ and there exists a neighbor $v'$ of $v$ such that \PO wins from configuration $\zug{v', B'_1, B'_2}$, or (2) $b_2 > b_1$ and \PO wins from every configuration $\zug{v', B'_1, B'_2}$, for a neighbor $v'$ of $v$. For \(i \in \{1, 2\}\), we call a row or a column of \(M_c\) a \(i\)\emph{-row} or \(i\)\emph{-column}, respectively, if all its entries are \(i\).

\begin{definition}{\bf (Local Determinacy)}
	A bidding game $\G$ is called \emph{locally determined} if for every configuration $c$, the bidding matrix $M_c$ either has a $1$-column or a $2$-row. 
\end{definition}

Consider a locally-determined game and a configuration $c$. If \PO wins from $c$, then $M_c$ necessarily has a $1$-row; indeed, a winning \PO strategy guarantees that even if \PO reveals his bid at $c$, the next configuration will be winning for \PO. More importantly, if \PO does not win in $c$, local determinacy implies that \PT can reveal her bid and force the game to a configuration that is not winning for \PO. In reachability games, since \PT's goal is to avoid the target, traversing non-losing configurations for \PT is in fact winning. 

\begin{lemma}\label{lem:locallydetermined}\cite{AAH21}
	If a reachability bidding game \(\G\) is locally determined, then \(\G\) is determined.
\end{lemma}

We make the following three observations regarding the bidding matrices.}

\stam{
\observations*


\begin{proof}
		If \(b_2 > b_1 > b_1'\) then \PT wins the current bidding for both pair of bids \(\zug{b_1, b_2}\) and \(\zug{b_1', b_2}\). 
		Thus \PT controls the corresponding intermediate vertex, and moves the token as per her choice. 
		As a result, only \PT's budget gets decreased by \(b_2\).
		Therefore, all the transitions that are available from \(\zug{c, b_1, b_2}\) are also available from \(\zug{c,b_1', b_2}\), and vice-versa.
		In other words, whoever wins from \(\zug{c, b_1, b_2}\) also wins from \(\zug{c, b_1', b_2}\), hence the entries are same. 
		The argument is similar when \(b_1 > b_2 > b_2'\).
		
		For the third observation, the tie-breaking mechanism, \PO always wins tie, plays the key role.
		For both the cases: when the bids are \(\zug{b, b}\), and when it is \(\zug{b, b -1}\) from a configuration \(c\), \PO wins the current bidding. 
		As a result the token moves according to his choice, his budget gets decreased by \(b\), while \PT's budget remains unchanged. 
		Therefore, all the available transitions from the \PO controlled vertex \(\zug{c, b, b}\) and \(\zug{c, b, b-1}\) are the same, and whoever wins from one, also wins from the other. 
		Thus the entries in \(M_c\) are same.
\end{proof}}


%The proof of \cite[Theorem 4.5]{AAH21} shows that a game whose bidding matrices have the properties of Lem.~\ref{lem:observations} is locally determined, irrespective of whether Richman or poorman bidding is employed. Combining with Lem.~\ref{lem:locallydetermined}, we have the following. 
%
%\determinacy*

\section{Proofs of Section~\ref{sec:pipeandstabilization}}\label{app:proofspipe}

%\begin{theorem*}
%	Consider a poorman discrete-bidding game \(\calG = \zug{V, E, B_1, B_2, \calO}\) for which the underlying game graph $(V,E)$ is a DAG. Let $v\in V$ be a vertex and let $t_v$ denote the continuous threshold at $v$. Then, for every initial budget $B_2\in\mathbb{N}$ of \PT, we have
%	\[ t_v \cdot (1- \frac{\textnormal{max-path}(v)}{B_2})  \leq \frac{T_v(B_2)}{B_2} \leq t_v, \]
%	where {\em $\textnormal{max-path}(v)$} is the length of the longest path from $v$ to the target vertex or the sink vertex of the game. Note that $\textnormal{max-path}(v) \leq |V| - 1$. Furthermore, the right-hand-side inequality holds even if the game graph $(V,E)$ is not a DAG.
%\end{theorem*}
\stam{
\pipe*
\begin{proof}
	{\em Right-hand-side inequality.} We first prove the right-hand-side inequality. To prove that $T_v(B_2) / B_2 \leq t_v$, it suffices to prove that, for each $\epsilon > 0$, \PO has a winning strategy if the game starts in $v$, \PO's initial budget is at least $t_v\cdot B_2 + \epsilon$ and \PT's initial budget is $B_2$. If we are able to prove this claim, it will then follow that $T_v(B_2) / B_2 \leq t_v + \epsilon$ holds for every $\epsilon>0$, therefore $T_v(B_2) / B_2 \leq t_v$.
	
	Fix $\epsilon > 0$. We construct the winning strategy of \PO as follows. By the definition of the continuous threshold $t_v$, we know that \PO has a winning strategy in the poorman {\em continuous-bidding} game. Moreover, it was shown in~\cite[Theorem 7]{LLPSU99} that \PO has a {\em memoryless} winning strategy, i.e.~a strategy in which the bids and token moves in each turn depend only on the position of the token and the players' budgets. We take such strategy $\sigma_{\textrm{cont}}$. We then construct a winning strategy $\sigma_{\textrm{disc}}$ of \PO in the poorman discrete-bidding game as follows:
	\begin{itemize}
		\item At each turn, if \PO under $\sigma_{\textrm{cont}}$ would bid $b$, then \PO under $\sigma_{\textrm{disc}}$ bids $\lfloor b \rfloor$.
		\item If \PO wins the bidding, then the token is moved to the vertex perscribed by $\sigma_{\textrm{cont}}$. 
	\end{itemize}
	
	We show that $\sigma_{\text{disc}}$ is indeed winning for \PO. To do this, we prove that $\sigma_{\textrm{disc}}$ preserves the invariant that, whenever the token is in some vertex $v'$, the ratio of players' budgets is positive and strictly greater than $t_{v'}$. This invariant implies that the token does not reach the sink state as the continuous threshold in the sink state is infinite. Thus, as a poorman discrete-bidding game ends in finitely many steps, this then implies that the game must eventually reach \PO's target state and therefore that $\sigma_{\text{disc}}$ is winning for \PO.
	
	We prove the invariant by the induction on the length of the game play. The base case holds by the assumption that, in the initial vertex $v$, \PO's initial budget is at least $t_v\cdot B_2 + \epsilon$ and \PT's initial budget is $B_2$. Now, for the induction hypothesis, suppose that the token is in vertex $v'$ after finitely many steps, with \PT's budget $B_2'$ and \PO's budget at least $t_{v'} \cdot B_2' + \epsilon'$ for some $\epsilon'>0$. We show that the invariant is preserved in the next step. Suppose that \PO under $\sigma_{\textrm{disc}}$ bids $\lfloor b \rfloor$ where $b$ is the bid of \PO under $\sigma_{\textrm{cont}}$. In what follows, we use the fact that $\sigma_{\textrm{cont}}$ preserves the ratio invariant which was established in the proof of~\cite[Theorem 7]{LLPSU99}. We distinguish between two cases:
	\begin{itemize}
		\item If \PO wins the bidding and moves the token to $v''$, then the ratio of budgets at the next step is
		\begin{equation*}
		\begin{split}
			\frac{t_{v'} \cdot B_2' + \epsilon' - \lfloor b\rfloor}{B_2'} &\geq \frac{t_{v'} \cdot B_2' + \epsilon' - b}{B_2'} \\
			&> \frac{t_{v'} \cdot B_2' - b}{B_2'} \\
			&\geq t_{v''}.
		\end{split}
		\end{equation*}
		The last inequality follows from the fact that the fraction in the second line is the subsequent ratio of budgets under the continuous-bidding winning strategy.
		\item If \PT wins the bidding, then \PT had to bid at least $\lfloor b \rfloor + 1$. Suppose that \PT moves the token to $v''$ upon winning. Then the ratio of budgets at the next step is at least
		\begin{equation*}
		\begin{split}
			\frac{t_{v'} \cdot B_2' + \epsilon'}{B_2' - \lfloor b\rfloor - 1 } &\geq \frac{t_{v'} \cdot B_2' + \epsilon'}{B_2' - b} \\
			&> \frac{t_{v'} \cdot B_2'}{B_2' - b} \\
			&\geq t_{v''}.
		\end{split}
		\end{equation*}
		The first inequality follow by observing that $\lfloor b \rfloor + 1 \geq  b$, and the third inequality follows from the fact that the second fraction is an upper bound on the subsequent ratio of budgets under the continuous-bidding winning strategy.
	\end{itemize}
	Hence, the invariant claim for poorman discrete-bidding follows by induction on the length of the game play, thus $\sigma_{\text{disc}}$ is indeed winning for \PO and the right-hand-side inequality in the theorem follows.
	
	\noindent {\em Left-hand-side inequality.} We now prove the left-hand-side inequality. If $B_2 < \textrm{max-path}(v) $, the claim trivially follows. Otherwise, it suffices to prove that \PT has a winning strategy if the game starts in $v$, \PO's initial budget is strictly less than $t_v\cdot (B_2 - \textrm{max-path}(v))$ and \PT's initial budget is $B_2$.
	
	Suppose that $B_2 \geq \textrm{max-path}(v)$ and let $B_1  < t_v\cdot (B_2 - \textrm{max-path}(v) )$ be the initial budget of \PO. We construct the winning strategy of \PT as follows. Let $\sigma_{\textrm{cont}}$ be the memoryless winning strategy of \PT under {\em continuous-bidding} when the game starts in $v$, \PT's initial budget is $B_2 - \textrm{max-path}(v)$ and \PO's initial budget is $B_1$. Since  $B_1  < t_v\cdot (B_2 - \textrm{max-path}(v))$, such a strategy exists by the definition of the continuous threshold $t_v$ and by~\cite[Theorem 7]{LLPSU99} which shows that it is possible to pick a memoryless winning strategy. We then construct a winning strategy $\sigma_{\textrm{disc}}$ of \PT in the poorman discrete-bidding game when \PT has initial budget $B_2$ and \PO has initial budget $B_1$ as follows:
	\begin{itemize}
		\item At each turn, if \PT under $\sigma_{\textrm{cont}}$ would bid $b$, then \PT under $\sigma_{\textrm{disc}}$ bids $\lceil b \rceil$.
		\item If \PT wins the bidding, then the token is moved to the vertex perscribed by $\sigma_{\textrm{cont}}$. 
	\end{itemize}
	Note that, if we show that the bids $\lceil b \rceil$ under $\sigma_{\textrm{disc}}$ are legal (i.e.~do not exceed available budget), then $\sigma_{\textrm{disc}}$ is clearly winning for \PT. Indeed, $\sigma_{\textrm{cont}}$ is winning for \PT, the bids of $\sigma_{\textrm{disc}}$ are always as least as big as those of $\sigma_{\textrm{cont}}$ and the token moves under two strategies coincide. So we only need to prove that the bids $\lceil b \rceil$ under $\sigma_{\textrm{disc}}$ are legal. But this follows from the fact that the underlying graph is a DAG and thus the game takes at most $\textrm{max-path}(v)$ turns before it reaches either the target or the sink vertex. Hence, as the bids are legal under $\sigma_{\textrm{cont}}$ when \PT has initial budget $B_2 - \textrm{max-path}(v)$, the bids are also legal under $\sigma_{\textrm{disc}}$ as \PT can bid $b + 1 \geq \lceil b \rceil$ in each turn. This concludes the proof of the left-hand-side of the inequality.
\end{proof}}



\stam{
\periodicity*
\begin{proof}
We proceed by induction with respect to the topological order of the graph.
For the target $v$ we set $(B,u_x,u_y)=(0,1,0)$ and for the sink we set $(B,u_x,u_y)=(0,0,1)$.
Next, given a non-leaf vertex $v$, suppose that among its children there are $k$ distinct continuous threshold ratios, and denote them by $t_1<t_2<\dots<t_k$.
Note that whenever \PO wins a bidding at vertex $v$ (while \PT has budget $B$), he moves the token to a child $u\in N(v)$ of $v$ with minimal value $T_u(B)$.
We claim that for large enough $B$, the only relevant children $u$ are those with $t_{u}=t_1$.
Indeed, consider two children $u,w\in N(v)$, one with $t_u=t_1$ and the other one with $t_w\ne t_1$.
Then by \cref{stm:pipe}, for $B>t_2\cdot n / (t_2-t_1)$ we have
\[T_w(B) \ge t_w\cdot (B-n) \ge t_2\cdot (B-n) > t_1\cdot B \ge T_u(B),
\]
thus \PO would prefer to move to $u$ rather than to $w$.

Next, let $V^-=\{u\in N(v) \mid t_u=t_1\}$ be a set of those ``relevant'' children of $v$.
We say that a function $f\colon \Nat\to\Nat$ is \textit{$(x,y)$-climbing} if it satisfies $f(B+x)=f(B)+y$ for all large enough $B$.
By induction assumption, for each $u\in V^-$ the function $T_{u}(B)$ is $(u_x,u_y)$-climbing with a slope $u_y/u_x$. By \cref{stm:pipe}, this slope is equal to $t_1$.
Thus, the function $T_u(B) - t_1\cdot B$ is periodic with period $u_x$.
The function $\min_{u\in V^-}\{T_u(B) - t_1\cdot B\}$ is then periodic with the period equal to the least common multiple $l=\operatorname{lcm}\{u_x\mid u\in V^-\}$ of the respective periods.
Therefore, the function $T_{v^-}(B) := \min_{u\in V^-}\{T_u(B)\}$ is $(l,l\cdot t_1)$-climbing.
To summarize, the moves of \PO (upon winning a bidding) are faithfully represented by him moving the token to a vertex $v^-$ for which the threshold budgets are $(l,l\cdot t_1)$-climbing.
Completely analogously we show that the moves of \PT are faithfully represented by her moving the token to a vertex $v^+$ with $(l',l'\cdot t_k)$-climbing threshold budgets.

From now on, for ease of notation suppose that $T_{v^-}$ is $(u_x,u_y)$-climbing and that $T_{v^+}$ is $(w_x,w_y)$-climbing (for some integers $u_x,u_y,w_x,w_y$). We will show that $T_v$ is $(u_x\cdot(w_x+w_y), w_y\cdot(u_x+u_y))$-climbing. This will complete the induction proof.

To prove this claim, it is convenient to represent each configuration $c=\zug{v,B_1,B_2}$ as a point in the plane with coordinates $[B_2,B_1]$, see~\cref{fig:u-climbing}.
\PO can then force a win from a configuration $c=\zug{v,B_1,B_2}$ if and only if 
the point $P=[B_2,B_1]$ lies on or above the threshold function~$T_v$.


% Figure environment removed


Take any point $P=(P_x,P_y)$.
Let $Q=(Q_x=P_x,Q_y)$ be the furthest point below $P$ that still lies on or above $T_{v^-}$, and
let $R=(R_x,R_y=P_y)$ be the closest point to the left of $P$ that lies on or above $T_{v^+}$.
Note that $P$ lies on or above $T_v$ if and only if the distances $d_y:=P_y-Q_y$ and $d_x:=P_x-R_x$ satisfy $d_x\le d_y+1$. Indeed, if the inequality holds then \PO can force a win by bidding $d_y$, whereas in the other case \PT can force a win by bidding $d_y+1$. 

As a final step, we show that at some point further along the curves $T_{v^-}$ and $T_{v^+}$, the two distances $d_x$, $d_y$ increase by the same margin.
Specifically, chain $u_x+u_y$ copies of a vector $(w_x,w_y)$ starting from point $R$ to get to point $R'=(R'_x,R'_y)$, and, similarly,
 chain $w_x+w_y$ copies of $(u_x,u_y)$ from $Q$ to $Q'=(Q'_x,Q'_y)$.
Finally, let $P'$ be the point above $Q'$ and to the right of $R'$.
Then a straightforward algebraic manipulation shows that distances from $P'$ to $Q'$ and to $R'$ both increased by $u_xw_y - w_xu_y$.
Indeed, without loss of generality set $Q=(d_x,0)$ and $R=(0,d_y)$.
Then we have
\[Q'=(d_x+ (w_x+w_y)u_x, 0+(w_x+w_y)u_y)\]
and
\[R'=( (u_x+u_y)w_x, d_y+(u_x+u_y)w_y),\]
so 
\[P'=(Q'_x,R'_y)=(d_x+ (w_x+w_y)u_x, d_y+(u_x+u_y)w_y)\]
and finally
\begin{align*}
 P'_y-Q'_y&=d_y+(u_x+u_y)w_y - (w_x+w_y)u_y\\
 &= d_y + (u_xw_y - w_xu_y),\\
 P'_x-R'_x &= d_x+ (w_x+w_y)u_x - (u_x+u_y)w_x \\
 &= d_x + (w_yu_x - u_yw_x). 
\end{align*} 
\end{proof}}







\section{Proofs of Section \ref{sec:closed-form}}
%
%\begin{lemma}
%	At any vertex \(v\) (other than the target and safety vertex) of the race game, if \PT has a budget \(B\), then \PO's threshold budget by which he surely wins the race game is \(x \cdot \floor{\frac{B}{y}}\), where \(x\) and \(y\) is the minimum distance from \(v\) to the target vertex and the safety vertex respectively.
%\end{lemma}
\subsection*{Race games}

\stam{
Recall that in a race games \(\race{a, b}\), \PO wins iff he can win at \(a\) bidding before \PT wins \(b\) biddings.
\cref{fig:race} depicts \(\race{3, 3}\), where a vertex \(v_{x, y}\) depicts \PO needs to win \(x\) many bidding from that vertex before \PT wins \(y\) biddings, in order to win the game. 


% Figure environment removed}

\stam{
\thresholdrace*
\begin{proof}
	First note that, the threshold budget for \PO is \(0\) at \(t\), and \(\infty\) at \(s\).
	Let us denote any vertex of the race game as \(v_{x,y}\) where \(x\) and \(y\) are the minimum distance from the vertex to \(t\) and \(s\),  respectively.
	In this notation, the root \(v\) of \(\race{a,b}\) is referred to as \(v_{a,b}\).
	
	Note that, a subgame of \(\race{a,b}\) rooted at any vertex \(v_{x, y}\) is \(\race{x,y}\) itself. 
	We, in fact, show in the following: \(T_{v_{x, y}} = x \cdot \floor{\frac{B}{y}}\) which implies what we require.
	
	Let us now consider \(v_{1,1}\).
	At this vertex, \PO has to win the bid, otherwise \PT simply moves the token to \(s\).
	Because \PT has a budget of \(B\), and \PO wins all ties,
	his threshold budget at this vertex is \(B\), and he bids his whole budget.
	
	By induction on \(x\), we can argue that for any vertex of the form \(v_{x, 1}\), the threshold budget is \(xB\), because \PO has to win all \(x\) the bids to prevent the token reaching \(s\).
	Thus he has to bid at least \(B\) at all those \(x\) bids.
	In fact, if he has budget at most \(xB - 1\) at vertex \(v_{x,1}\), then \PT has a winning strategy:
	she bids \(B\) until she wins.
	
	Similarly, by induction on \(y\), we claim that for any vertex of the form \(v_{1, y}\), the threshold budget is \(\floor{\frac{B}{y}}\).
	The base case of this induction is \(v_{1,1}\), for which we showed earlier that the statement is true.
	Let us assume it is true for \(v_{1,y-1}\), and we prove the claim for \(v_{1,y}\).
	
	We suppose \PO's budget at \(v_{1, y}\) is \(\floor{\frac{B}{y}}\) while \PT's budget is \(B\).
	We claim that the winning strategy for \PO at vertex \(v_{1,y}\) is to bid his whole budget itself.
	If he wins the bid, he moves the token to target.
	Otherwise, \PT wins the bid by at least bidding \(\floor{\frac{B}{y}} +1\), and of course, she moves the token to \(v_{1, y-1}\) .
	Thus, her budget at \(v_{1, y-1}\) is at most \(B - (\floor{\frac{B}{y}} +1)\), while \PO's budget remains \(\floor{\frac{B}{y}}\).
	From the induction hypothesis, we know when \PT has a budget \(B - (\floor{\frac{B}{y}} +1)\), \PO's threshold budget for surely winning is \(\floor{\frac{B - (\floor{\frac{B}{y}} +1)}{y-1}}\).
	If we can show that, \(\floor{\frac{B}{y}} \geq \floor{\frac{B - (\floor{\frac{B}{y}} +1)}{y-1}}\), we are done.
	We show this in the following:
	\begin{align*}
		B - y \cdot \floor{\frac{B}{y}} &\leq y-1\\
		\implies B - \floor{\frac{B}{y}} &\leq (y-1) \cdot \floor{\frac{B}{y}} + (y-1)\\
		\implies \frac{B - \floor{\frac{B}{y}}}{y-1} &\leq \floor{\frac{B}{y}} + 1\\
		\implies \frac{B - (\floor{\frac{B}{y}}+1)}{y-1} &\leq \floor{\frac{B}{y}} + \frac{y-2}{y-1}
	\end{align*}
	
	Because \(\floor{\frac{B}{y}}\) itself is an integer, by taking \(\floor{}\) on the both side, we get \(\floor{\frac{B}{y}} \geq \floor{\frac{B - (\floor{\frac{B}{y}} +1)}{y-1}}\).
	Therefore, \(\floor{\frac{B}{y}}\) is a sufficient budget for \PO to surely win from vertex \(v_{1,y}\).
	
	Now, we need to show that this is also necessary budget for him.
	In fact, we show that when \PO has budget at most \(\floor{\frac{B}{y}} - 1\), while \PT's budget is \(B\), she has a surely winning strategy from \(v_{1,y}\).
	Her winning strategy is bidding \(\floor{\frac{B}{y}}\), until she reaches \(s\).
	Because \(B \geq y \cdot \floor{\frac{B}{y}}\), she can actually bids likewise.
	At each vertex, \PO's budget will be strictly less than what she is bidding, therefore he looses all the \(y\) bids, and the token indeed reaches the safety vertex.
	
	For a general vertex \(v_{x, y}\), we argue by induction which goes like above.
	We assume that for \(v_{x-1,y}\) and \(v_{x, y-1}\), which are the only two neighbours of \(v_{x, y}\), the threshold budget for \PO is \((x-1) \cdot \floor{\frac{B}{y}}\) and \(x \cdot \floor{\frac{B}{y-1}}\), respectively.
	
	We suppose \PO's budget at \(v_{x, y}\) is \(x \cdot \floor{\frac{B}{y}}\), and \PT's budget is \(B\).
	We claim that his wining strategy at the first bid is to bid \(\floor{\frac{B}{y}}\).
	We show that irrespective of where the token gets placed at the next vertex, he will have the respective threshold budget at that vertex.
	
	If he wins the bid at \(v_{x, y}\), his new budget becomes \((x-1)\cdot \floor{\frac{B}{y}}\), which is exactly what he needs to surely win from \(v_{x-1, y}\).
	If he looses, and the token gets placed at \(v_{x, y-1}\), \PT's budget becomes at most \(B - \floor{\frac{B}{y}}+1)\).
	It remains to show that \(x \cdot \floor{\frac{B -  \floor{\frac{B}{y}}+1)}{y-1}} \leq x\cdot \floor{\frac{B}{y}}\), which is true as we have earlier established \(\floor{\frac{B}{y}} \geq \floor{\frac{B - (\floor{\frac{B}{y}} +1)}{y-1}}\).
	It proves that \(x\cdot \floor{\frac{B}{y}}\) is the sufficient budget for \PO to surely win from \(v_{x, y}\).
	
	Finally, if \PO's budget is at most \(x \cdot \floor{\frac{B}{y}} - 1\) and \PT's budget is \(B\), then \PT wins the game if she bids \(\floor{\frac{B}{y}}\) at each bidding.
	This can be shown by another inductive argument where we assume the statement being true for vertices \(v_{x-1, y}\) and \(v_{x, y-1}\), and follow the same steps that we did for \PO above.
\end{proof}}

\subsection*{Tug-of-War games}

\stam{
Recall that in Tug-of-War games, \(\TUG(n)\) is a tug-of-games with \(n +2 \) nodes (\(n\) internal nodes and \(2\) endpoints).
One of the endpoints is winning for the reachability player. 
We denote by \(\tug(n, k,b)\) the threshold budget for \PO for winning \(\TUG(n)\) from a vertex which is \(k\) steps away from his target and when \PT's budget is \(b\).
\begin{example}
	\cref{fig:tow} depicts tug-of-war games for \(n = 2\) and \(3\).
	In these two games, when opponent has a budget \(b = 100\), the threshold budgets of the reachability player are \(\tug(2, 1, 100) = \floor{100/\phi} = 61\) and \(\tug(3,3,100) = 2 \cdot 100 -1 = 199\), respectively.
\end{example}

%\begin{example}
%	 In the two depicted games, when opponent has budget $b=100$, the threshold budgets of the reachability player are
%		$\tug(2,1,100)=\floor{100/\phi}=61$ and $\tug(3,3,100)=2\cdot 100-1=199$, respectively.
%		% Figure environment removed
%\end{example}

% Figure environment removed}


\tugtwo*

\stam{
\begin{proof}
	To simplify the notation, let us assume \(t_b = \tug(2,1,b)\), and \(u_b = \tug(2,2,b)\).
	We first claim that \(t_b\) and \(u_b\) are the unique solution to the following system of recurrence relations. 
	\begin{enumerate}
		\item\label{itm:tuga} \(t_0 = u_0 = 0\)
		
		\item\label{itm:tugb} \(u_b = t_b + b\) for any \(b \geq 1\)
		
		\item\label{itm:tugc} \(t_b = \min_x\{\max (x, u_{b - 1-x}) \mid 0 \leq x \leq b\}\) for any \(b \geq 1\)
	\end{enumerate}
\cref{itm:tuga} is obvious because \PO bids \(0\) at every step and he wins ties, when \PT has a budget \(0\).

\PO needs to win at the vertex which is \(2\) steps away from his target, otherwise \PT moves the token to the other end-point.
Therefore, \PO needs to bid \(b\), and his new budget should be, by definition, at least \(t_b\) upon winning.
This gives us \cref{itm:tugb}.

Finally, at the vertex which is a single step away from \PO's target, he needs to optimize what his bid would be between \(0\) and \(b\) so that even if he loses the current bid, he would have enough budget at the next step to win from there (i.e, \(u_b\)).
This gives us \cref{itm:tugc}.

Moreover, the system of equations has a unique solutions, as there are as many equations as there are unknowns (\(t_b, u_b\) for a fixed \(b\)).
Hence, it is enough to show that the expressions \(t_b = \floor{\frac{b}{\phi}}\) and \(u_b = \floor{b \cdot \phi}\) satisfy those equations.
Clearly, \(\floor{\frac{0}{\phi}} = \floor{0 \cdot \phi} = 0\), so \cref{itm:tuga} holds.
Next note that the golden ratio satisfy \(\phi = 1 + 1/\phi\). 
Thus,
\[ u_b = \floor{b \cdot \phi} = \floor{b \cdot (1 + 1/\phi)} = \floor{b + b/\phi} = b + \floor{b/\phi} = b + t_b\]

implying \cref{itm:tugb} holds too.

Finally, note that the function $f\colon x\to x$ is increasing, hence to verify \cref{itm:tugc} we need to show two inequalities for any $b\geq 1$:
\begin{enumerate}
	\item For $x=\floor{b/\phi}$ we have $\floor{(b-1-x)\cdot \phi}\leq \floor{b/\phi}$.
	\item For $x=\floor{b/\phi}-1$ we have $\floor{(b-1-x)\cdot \phi}\geq \floor{b/\phi}$.
\end{enumerate}
In both cases, we will do this by checking that the insides of the two floor functions being compared satisfy the same inequality. Upon plugging in $x$, it thus suffices to show
\[(b-1-\floor{b/\phi})\cdot \phi \leq b/\phi
\quad\text{and}\quad
(b-\floor{b/\phi})\cdot \phi \geq b/\phi.
\]
From $\phi=1+1/\phi$ we have $b\cdot\phi-b/\phi=b$, so the desired inequalities rewrite as
\[ b-\phi \leq \floor{b/\phi}\cdot\phi
\quad\text{and}\quad
\floor{b/\phi}\cdot\phi\leq b.
\]
Those two inequalities follow from the obvious inequalities $b/\phi-1\leq \floor{b/\phi} \leq b/\phi$ after multiplying by $\phi$.
\end{proof}}

\stam{
\tugthree*

\begin{proof}
	We proceed similarly to the proof of~\cref{thm:tug2}.
	This time, we need to check that the expressions
	\[t_b=\floor{(b-1)/2}, \quad u_b=b-1, \quad\text{and}\quad v_b=2b-1
	\]
	satisfy the relations
	\begin{enumerate}
		\item\label{itm:tug3a} $t_1=u_1=0$, $v_1=1$,
		\item\label{itm:tug3b} $v_b=u_b+b$ for any $b\geq 2$,
		\item\label{itm:tug3c} $u_b= \min_x \{  \max\{t_b+x, v_{b-1-x}\} \mid 0\leq x \leq b\}$ for any $b\geq 2$.
		\item\label{itm:tug3d} $t_b= \min_x \{  \max\{x, u_{b-1-x}\} \mid 0\leq x \leq b\}$ for any $b\geq 2$.
	\end{enumerate}
	This time, both \cref{itm:tug3a} and \cref{itm:tug3b} follow by direct substitution.
	
	Regarding \cref{itm:tug3c}, we need to show that
	\[ b-1=  \min_x \{  \max\{\floor{(b-1)/2}+x, 2b-3-2x \} \mid 0\leq x \leq b\}
	\]
	To that end, we distinguish two cases based on the parity of $b$.
	If $b=2k$ is even then we need to show
	\[ 2k-1 = \min_x\{ \max\{k-1+x, 4k-3-2x \} \mid 0\leq x \leq 2k\},
	\]
	and indeed the minimum on the right-hand side is attained for $x=k-1$ and is equal to $2k-1$ as desired.
	Similarly, if $b=2k+1$ is odd then we need to show
	\[ 2k = \min_x\{ \max\{k+x, 4k-1-2x \} \mid 0\leq x \leq 2k\},
	\]
	and indeed the minimum on the right-hand side is attained for $x=k$ and is equal to $2k$ as desired.
	
	Finally, regarding \cref{itm:tug3d} we have
	$u_{b-1-x}=b-2-x$, hence the two numbers inside the $\max(\cdot)$ function always sum up to $b-2$.
	If $b=2k$ is even, then the minimum is $(b-2)/2=k-1 = \floor{(b-1)/2}=t_b$ as desired.
	If $b=2k+1$ is odd then the minimum is $\ceil{(b-2)/2}=k=\floor{(b-1)/2}=t_b$ as desired again.
\end{proof}}



%\begin{corollary}\label{cor:periodicity}
%For any DAG and any vertex $v$ there exist values $B, u_x,u_y \in\Nat$ such that for all $B_2>B$ we have $T_v(B_2+u_x)=T_v(B_2)+u_y$.
%\end{corollary}
%\begin{proof}
%We say that a function $f$ is $u=(u_x,u_y)$-\textit{climbing} if $f(B_2+u_x)=f(B_2)+u_y$ holds for all large enough $B$.
%Note that a minimum of several climbing functions is itself a climbing function.
%Indeed, given the corresponding parameters $(u^1_x,u^1_y),\dots,(u^k_x,u^k_y)$, it suffices to take $u_x=\lcm_i(u^i_x)$ and $uy=\lcm_i(u^i_y)$ to be the lowest common multiples (and likewise for maximum).
%
%Thus, it remains to show that 
%
%We proceed by induction with respect to the topological order of the graph.
%
%If $v$ is a leaf than the claim is obvious (and we can set $B=0$).
%Now suppose the claim holds for all descendants of $v$.
%Let $V^-=\argmin_{(v,v')\in E} t_{v'}$ %(resp.\ $V^+=\argmax_{(v,v')\in E} t_{v'}$)
%be the set of descendants of $v$ with the lowest %(resp. largest)
% continuous thresholds.
%Take $B$ from Corollary~\ref{cor:stabilization} and $B_2>B$.
%We know that, upon winning a bidding, Player 1 will move the token to some node $v^-\in V^-$.
% By induction assumption, for each $v_i\in V^-$ there exist integers $x_i,y_i$ such that $T_{v_i}(B_2+x_i)=T_{v_i}(B_2)+y_i$.
% Let $u^-_x=\lcm_i(x_i)$ and $u^-_y=\lcm_i(y_i)$ be the lowest common multiples of the $x_i$'s and $y_i$'s, respectively. 
%\end{proof}


\section{Proof of Section~\ref{sec:thresholdgeneral}}

%\begin{theorem*}
%	For a DAG game and any budget $B_2$ of \PT, the threshold budget $\budget{B_2}{v}$ can be determined in $\mathcal{O}(B_2 \cdot \log(|V|) \cdot |E|)$ steps for all vertices.
%\end{theorem*}
\stam{
\lineardagalgo*
\begin{proof}
	Fix the input as in the assumptions.

	Firstly, we see that each vertex of the DAG is evaluated exactly once and we can, in one step, directly compute $\budget{B}{v}$ for all $0 \leq B \leq B_2$ one vertex at a time:
	Sort the vertices in reverse topological order.
	Observe that, by assumption, sink and target are the only leaves, for which computing $\budget{B}{v}$ is trivial.
	Then, inductively, whenever we compute $\budget{B}{v}$, the values $\budget{\cdot}{v'}$ of all successors $v' \in N(v)$ are already known.
	Thus, we can directly compute $\budget{B}{v} = \step{T_\circ}{v}{B}$.
	(Note that this reasoning also can be applied to the SCC decomposition of general games.)
	%(For a different perspective, recall that DAG games end after at most $\mathrm{height}(G)$ steps, i.e.\ we need at most this many iterations.)

	Secondly, using Thm.~\ref{stm:pipe}, we can derive bounds on the optimal bid:
	We know that the threshold $\budget{B}{v}$ in a particular state $v$ has to lie between the lower and upper bounds given by the theorem -- a linearly sized interval.
	This however does not immediately give us bounds on the bids.
	Using the above approach of processing vertices in reverse topological order, whenever we handle a given vertex $v$, all of its successors are already solved.
	Together, we know (i)~a linearly sized interval of potential \emph{thresholds} for $v$, say $[B^-, B^+]$ and (ii)~the exact thresholds in all successor vertices.
	Note that in order to use Thm.~\ref{stm:pipe} computationally, we first need to determine the continuous ratios $t_v$ for every vertex.
	We explain afterwards how this can be achieved in linear time, too.

	We define $T' = \min_{v' \in N(v)} \budget{B}{v'}$ the smallest threshold over all successors against $B$, i.e.\ the minimum budget \PO needs to win after winning the bid in $v$ (and paying for it).
	As an immediate observation, we see that an optimal bid can never be larger than $B^+ - T'$:
	If \PO would bid more than $B^+ - T'$, \PT bids 0 in response, leaving \PO with a budget of less than $T'$, which is required to win.

	For the lower bound, we prove that at least one optimal bid is at least $B^- - T'$.
	(This does not exclude optimal bids which are smaller than $B^- - T'$.)
	Suppose that the threshold budget is $\budget{B}{v'} = B_1$ and there is a winning strategy for \PO with a bid $b < B^- - T'$.
	We consider the bid $b' = B^- - T' > b$.
	If \PO wins with $b'$, a budget of $B_1 - b' = (B_1 - B^-) + T')$ is left, which is at least $T'$, since $B_1 \geq B^-$ by assumption.
	By definition, \PO can pick a successor from which a winning strategy with budget $T'$ or larger exists.
	For the losing case, recall that the bid $b$ was winning.
	This means that \PO can win if \PT wins by bidding $b + 1$.
	In particular, in every successor of $v$ a budget of $B_1$ is sufficient to win against $B - (b + 1)$ (which is \PT's budget afterwards).
	Thus, if \PO instead bids $b'$ and \PT wins (by bidding $b' + 1$), observe that $B - (b' + 1) < B - (b + 1)$, since $b + 1 > b' + 1$ -- \PT is left with even less budget than before.

	Together, we know that an optimal bid exists in the interval $[B^- - T', B^+ - T']$.
	Thus, we can restrict ourselves to checking all possible bids in this interval.
	Observe that $B^+ - B^-$ is linear in the size of the graph by Thm.~\ref{stm:pipe}, in particular it is bounded by the number of vertices times the largest continuous ratio.
	Moreover, we can apply the binary search idea of Thm.~\ref{stm:complexity}.
	In summary, we obtain a complexity of $\mathcal{O}(\log(B^+ - B^-) \cdot N(v))$ to determine $\budget{B}{v}$ for a vertex $v$ and budget $0 \leq B \leq B_2$.

	It remains to prove complexity and size bounds on $t_v$.
	First, we observe that given the ratios $t_{v'}$ of all successors, we can immediately compute $t_v = t_v^+ / (1 + t_v^+) \cdot (1 + t_v^-)$, where $t_v^+ = \max_{v' \in N(v)} t_{v'}$ and $t_v^- = \min_{v' \in N(v)} t_{v'}$ (using the results of, e.g., \cite[Sec.~3]{LLPSU99}).
	Note that $t_v^+ = \infty$ if $s \in N(v)$.
	In that case, we have $t_v = 1 + t_v^-$.
	As such, we can again obtain all ratios by a linear pass in reverse topological order.
	Moreover, the bit size of $t_v$ is bounded by the sum of bit sizes of $t_v^-$ and $t_v^+$, i.e.\ $|t_v|_\# \in \mathcal{O}(|t_v^+|_\# + |t_v^-|_\#)$, where $|v|_\#$ denotes the size of the representation of $v$.
	Since the ratio of sink and target are trivial (i.e.\ of size $1$), we obtain, as a crude upper bound, $|t_v|_\# \in \mathcal{O}(|V|^2)$ for all $v$.
	This means that evaluating the equation takes at most $\mathcal{O}(|V|^2 \log |V|)$ time (note that $|N(v)| \leq |V|^2$) and we can obtain $t_v$ for all vertices in time $\mathcal{O}(|V|^3 \log |V|)$.

	We also directly obtain a bound on the magnitude of $t_v$:
	Clearly, $t_v \leq 1 + t_v^-$, i.e.\ $t_v \in \mathcal{O}(|V|)$.
	Also, this bound is tight:
	In $\race{1,n}$, \PO needs $|V| - 2$ times the budget of \PT, since its required to win $|V|$ in a row without alternative.
	Consequently, the ``height'' of the pipe, i.e.\ $B^+ - B^-$ is at most of size $|V|^2$.

	Combining all results, we obtain that the overall complexity of this algorithm is bounded by
	\begin{equation*}
		\mathcal{O}(|V|^3 \log(|V|) + B_2 \cdot \log(|V|) \cdot |E|).
	\end{equation*}
	Note that if $B_2 \leq |V|^2$ we can employ our ``classical'' algorithm which simply applies binary search from $0$ to $B_2$ in reverse topological order, yielding a complexity of $\mathcal{O}(B_2 \cdot \log(B_2) \cdot |E|)$ (requiring for each vertex $\mathcal{O}(\log(B_2) \cdot N(v))$).
	Otherwise, i.e.\ $B_2 \geq |V|^2$, $B_2 \cdot \log(|V|) \cdot |E|$ dominates $|V|^3 \log(|V|)$ (recall that $|V| \leq |E|$), proving the claim.
\end{proof}}



