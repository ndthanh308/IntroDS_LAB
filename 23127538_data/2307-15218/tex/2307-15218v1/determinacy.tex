\section{Existence of Thresholds}\label{sec:determinacy}
In this section we show the existence of threshold budgets in games played on general graphs. 
\begin{definition}{\bf (Determinacy).} 
A game is {\em determined} if from every configuration, one of the players has a pure winning strategy.
\end{definition}

We claim that determinacy is equivalent to existence of thresholds. It is not hard to deduce both implications from the following observation. An additional budget cannot harm a player; namely, if \PO wins from a configuration $\zug{v, B_1, B_2}$, he also wins from $\zug{v, B'_1, B_2}$, for $B'_1 > B_1$, and dually for \PT. 
%	On the other hand, when threshold budget exists for every vertex in the game, \PO wins from any configuration of the form \(\zug{v, B_1, B_2}\) when \(B_1 \geq T_v(B_2)\), and \PT wins when \(B_1 < T_v(B_2)\), thus covering all configurations.  

%\subsection{Advantage-based tie-breaking}
%Given the positive results on Richman discrete-bidding games under advantage-based tie-breaking~\cite{DP10,AAH21,AS22}, we find the following negative result surprising.
%
%\begin{theorem}
%	Poorman discrete-bidding games under advantage based tie-breaking are not determined.
%\end{theorem}
%\begin{proof}
%	% Figure environment removed
%	Consider the game that is depicted in Fig.~\ref{fig:notdetermined} in which \PO needs to win two consecutive biddings. We claim that neither player wins from configuration \(c_0 = \langle v_0, 2^*, 1 \rangle\). We assume that at the time of bidding, the player with the advantage decides whether the advantage will be used; namely, his bid is either $b^*$, i.e.\ use the advantage in case of a tie, or $b$, do not use it. Intuitively, in each turn, a winning strategy reveals a bid and lets the other player respond to it. In the following, we show that no matter which bid is revealed in $c_0$, the other player has a winning response. 
%	
%	\textit{\PO cannot win}.
%	If \PO bids $0$, $0^*$, or $1$, \PT responds by winning the bidding and proceeding to $s$. If \PO bids $1^*$ (and the case of higher bids is similar), \PT responds with \(1\), and we proceed to \(\zug{v_1, 1, 1^*}\) from which \PT wins by bidding $1^*$.
%	
%	\textit{\PT cannot win}.
%	If \PT bids \(1\), \PO bids \(1^*\), and we proceed to \(\zug{v_1, 1, 0^*}\). If \PT bids \(0\), \PO bids \(0^*\), and we proceed to \(\zug{v_1, 2, 1^*}\). In both cases \PO can guarantee winning the next bidding.
%\end{proof}






%\subsection{\PO always wins ties}
%We report on the determinacy of poorman discrete-bidding games when \PO always wins ties. 

%In the rest of this section, we show the following - 


In the rest of this section, we prove determinacy of poorman discrete-bidding games. Our proof is based on a technique that was developed in \cite{AAH21} to show determinacy of \emph{Richman} discrete-bidding. We illustrate the key ideas. 
% under {\em transducer-based tie breaking}, which is more general than ``\PO always wins ties''. 
Consider a reachability bidding game \(\G = \zug{V, E, t, s}\) 
%where \PO always wins ties, 
and a configuration \(c = \zug{v, B_1, B_2}\). 
%We show that if \PO does not win from $c$, then \PT wins from $c$. 
We define a {\em bidding matrix} $M_c$ that corresponds to $c$. For $\zug{b_1, b_2} \in \set{0,\ldots, B_1} \times \set{0,\ldots, B_2}$, the \((b_1, b_2)^{\text{th}}\) entry in $M_c$ is associated with \PLi bidding $b_i$, for $i \in \set{1,2}$. We label entries in $M_c$ by $1$ or $2$ as follows. 
Let $\G_1$ denote a turn-based game that is the same as $\G$ only that in each turn, \PO reveals his bid first and \PT responds. Technically, once both players reveal their bids, the game proceeds to an intermediate vertex $i_{b_1, b_2} = \zug{b_1, b_2, c}$. Since $G_1$ is turn-based, it is determined, thus one of the players has a winning strategy from $i_{b_1, b_2}$. We label the \((b_1, b_2)^{\text{th}}\) entry in $M_c$ by $i \in \set{1,2}$ iff \PLi wins from $i_{b_1, b_2}$. For \(i \in \{1, 2\}\), we call a row or a column of \(M_c\) a \(i\)\emph{-row} or \(i\)\emph{-column}, respectively, if all its entries are labeled \(i\).


%intuitively captures the winner of the game with respect to each pair of bids by the two players at $c$. Formally, $M_c$ has $B_1$ rows and $B_2$ columns, each row corresponds to a \PO bid, and each column to a \PT bid. An entry in \(M_c\) is either \(1\) or \(2\), defined as follows. Let $\zug{b_1, b_2} \in \set{0,\ldots, B_1} \times \set{0,\ldots, B_2}$ and let $\zug{B'_1, B'_2}$ be the updated budgets after the players bid $\zug{b_1, b_2}$ at $c$. We set the \((b_1, b_2)^{\text{th}}\) entry in $M_c$ to be $1$ iff (1) either $b_1 \geq b_2$ and there exists a neighbor $v'$ of $v$ such that \PO wins from configuration $\zug{v', B'_1, B'_2}$, or (2) $b_2 > b_1$ and \PO wins from every configuration $\zug{v', B'_1, B'_2}$, for a neighbor $v'$ of $v$. For \(i \in \{1, 2\}\), we call a row or a column of \(M_c\) a \(i\)\emph{-row} or \(i\)\emph{-column}, respectively, if all its entries are \(i\).

\begin{definition}{\bf (Local Determinacy)}
	A bidding game $\G$ is called \emph{locally determined} if for every configuration $c$, the bidding matrix $M_c$ either has a $1$-row or a $2$-column. 
\end{definition}

Local determinacy is used as follows. 
It can be shown that if \PO wins from $c$, then $M_c$ has a $1$-row. 
%Consider a locally-determined game and a configuration $c$. If \PO wins from $c$, then $M_c$ necessarily has a $1$-row; indeed, a winning \PO strategy guarantees that even if \PO reveals his bid at $c$, the next configuration will be winning for \PO. 
More importantly, suppose that \PO does not win in $c$, then local determinacy implies that there is a $2$-column, say $b_2$. This means that when \PT bids $b_2$ in $\G$, the game proceeds to a configuration $c'$ from which \PO does not win. 
%can reveal her bid and force the game to a configuration that is not winning for \PO.
In reachability games, since \PT's goal is to avoid the target, traversing non-losing configurations for \PT is in fact winning.\footnote{The theorem is stated for reachability objectives and it is extended in~\cite{AAH21} to richer objectives.}

\begin{lemma}{\bf (\cite[Theorem 3.5]{AAH21})}\label{lem:locallydetermined}
	If a reachability bidding game \(\G\) is locally determined, then \(\G\) is determined.
\end{lemma}

Local determinacy of poorman discrete-bidding games follows from the following observations on bidding matrices.  
%We make the following three observations regarding the bidding matrices.
\begin{restatable}{lemma}{observations}
	\label{lem:observations}
	Consider a poorman discrete-bidding game \(\G\) where \PO always wins tie, and consider a configuration \(c = \langle v, B_1, B_2 \rangle \).			(1) Entries in \(M_c\) in a column above the top-left to bottom-right diagonal are equal: for bids \(b_2 > b_1 > b_1'\), we have $M_c[b_1, b_2] = M_c[b_1', b_2]$. (2) Entries on a row, left of the diagonal are equal: for bids \(b_1 > b_2 > b_2'\), we have $M_c[b_1, b_2] = M_c[b_1, b_2']$. (3) The entry immediately under the diagonal equals the entry on the diagonal: For a bid $b$, we have $M_c[b, b]= M_c[b, b-1]$.
\end{restatable}


\begin{proof}
	If \(b_2 > b_1 > b_1'\) then \PT wins the current bidding for both pair of bids \(\zug{b_1, b_2}\) and \(\zug{b_1', b_2}\). 
	Thus \PT controls the corresponding intermediate vertex, and moves the token as per her choice. 
	As a result, only \PT's budget gets decreased by \(b_2\).
	Therefore, all the transitions that are available from \(\zug{c, b_1, b_2}\) are also available from \(\zug{c,b_1', b_2}\), and vice-versa.
	In other words, whoever wins from \(\zug{c, b_1, b_2}\) also wins from \(\zug{c, b_1', b_2}\), hence the entries are same. 
	The argument is similar when \(b_1 > b_2 > b_2'\).
	
	For the third observation, the tie-breaking mechanism, \PO always wins tie, plays the key role.
	For both the cases: when the bids are \(\zug{b, b}\), and when it is \(\zug{b, b -1}\) from a configuration \(c\), \PO wins the current bidding. 
	As a result the token moves according to his choice, his budget gets decreased by \(b\), while \PT's budget remains unchanged. 
	Therefore, all the available transitions from the \PO controlled vertex \(\zug{c, b, b}\) and \(\zug{c, b, b-1}\) are the same, and whoever wins from one, also wins from the other. 
	Thus the entries in \(M_c\) are same.
\end{proof}

%<<<<<<< HEAD
%\begin{proof}[Proof sketch]
%	(See the supplementary material for the full proof)
%	
%	The main argument is that because in poorman bidding, only the winner's budget alters after a bidding, it does not matter what the other player bids as long as his bid remains a losing bid. 
%%	The full technical proof can be found in the supplementary material. 
%\end{proof}
%=======
%%guy: this is too vague. It's better not to include it at all, I think. 
%%\begin{proof}[Proof sketch]
%%	(See the supplementary material for the full proof)
%%	The main argument is that since under poorman bidding, only the winner's budget is altered after a bidding, it does not matter what the other player bids as long as his bid remains a losing bid. 
%%\end{proof}
%>>>>>>> 9cd46c38b89cf01f4a44f1e94e2a1d262765acb9


The proof of \cite[Theorem 4.5]{AAH21} shows that a game whose bidding matrices have the properties of~\cref{lem:observations} is locally determined, irrespective of whether Richman or poorman bidding is employed. Combining with~\cref{lem:locallydetermined}, we obtain the following. %have \cref{thm:determinacy}. 

\begin{restatable}{theorem}{determinacy}\label{thm:determinacy}
	Reachability poorman discrete-bidding games are determined.
\end{restatable}






%In this section, we establish the determinacy of poorman discrete-bidding games when \PO always wins ties. Our proof is based on \cite{AAH21}, which proves determinacy of Richman discrete-bidding under a tie-breaking mechanism called {\em transducer-based tie breaking}, which is more general than \PO wins ties. 
%
%Consider a reachability bidding game \(\G = \zug{V, E, t, s}\) where \PO always wins ties, and consider a configuration \(c = \zug{v, B_1, B_2}\). We show that if \PO does not win from $c$, then \PT wins from $c$. The {\em bidding matrix} $M_c$ intuitively captures the winner of the game with respect to each pair of bids by the two players at $c$. Formally, $M_c$ has $B_1$ rows and $B_2$ columns, each row corresponds to a \PO bid, and each column to a \PT bid. An entry in \(M_c\) is either \(1\) or \(2\), defined as follows. Let $\zug{b_1, b_2} \in \set{0,\ldots, B_1} \times \set{0,\ldots, B_2}$ and let $\zug{B'_1, B'_2}$ be the updated bids after the players bid $\zug{b_1, b_2}$ at $c$. We set the \((b_1, b_2)^{\text{th}}\) entry in $M_c$ to be $1$ iff (1) either $b_1 \geq b_2$ and there exists a neighbor $v'$ of $v$ such that \PO wins from configuration $\zug{v', B'_1, B'_2}$, or (2) $b_2 > b_1$ and \PO wins from every configuration $\zug{v', B'_1, B'_2}$, for a neighbor $v'$ of $v$. For \(i \in \{1, 2\}\), we call a row or a column of \(M_c\) a \(i\)\emph{-row} or \(i\)\emph{-column}, respectively, if all its entries are \(i\).
%
%\begin{definition}{\bf (Local Determinacy)}
%A bidding game $\G$ is called \emph{locally determined} if for every configuration $c$, the bidding matrix $M_c$ either has a $1$-column or a $2$-row. 
%\end{definition}
%
%Consider a locally-determined game and a configuration $c$. If \PO wins from $c$, then $M_c$ necessarily has a $1$-row; indeed, a winning \PO strategy guarantees that even if \PO reveals his bid at $c$, the next configuration will be winning for \PO. More importantly, if \PO does not win in $c$, local determinacy implies that \PT can reveal her bid and force the game to a configuration that is not winning for \PO. In reachability games, since \PT's goal is to avoid the target, traversing non-losing configurations for \PT is in fact winning. 
%
%\begin{lemma}\label{lem:locallydetermined}\cite{AAH21}
%If a reachability bidding game \(\G\) is locally determined, then \(\G\) is determined.
%\end{lemma}
%
%We make the following three observations regarding the bidding matrices (see the supplementary material for the proof). 
%%The main idea, which is same as \cite[lemma 4.3 and 4.4]{AAH21}, is that the subsequent configuration from a configuration only depends on the winner and the winning bid for our current bidding mechanism (first-price, poorman, discrete).  
%
%	\begin{restatable}{lemma}{observations}
%		\label{lem:observations}
%		Consider a poorman discrete-bidding game \(\G\) where \PO always wins tie, and consider a configuration \(c = \langle v, B_1, B_2 \rangle \).			(1) Entries in \(M_c\) in a column above the top-left to bottom-right diagonal are equal: for bids \(b_2 > b_1 > b_1'\), we have $M_c[b_1, b_2] = M_c[b_1', b_2]$. (2) Entries on a row, left of the diagonal are equal: for bids \(b_1 > b_2 > b_2'\), we have $M_c[b_1, b_2] = M_c[b_1, b_2']$. (3) The entry immediately under the diagonal equals the entry on the diagonal: For a bid $b$, we have $M_c[b, b]= M_c[b, b-1]$.
%	\end{restatable}
%
%	
%The proof of \cite[Theorem 4.5]{AAH21} shows that a game whose bidding matrices have the properties of Lem.~\ref{lem:observations} is locally determined, irrespective of whether Richman or poorman bidding is employed. Combining with Lem.~\ref{lem:locallydetermined}, we have the following. 
%
%	\begin{restatable}{theorem}{determinacy}\label{thm:determinacy}
%		Reachability poorman discrete-bidding games are determined when \PO always wins ties.
%	\end{restatable}













%%%%%%%%%%%%%%%%%%%%%%%%%%%
	
	

