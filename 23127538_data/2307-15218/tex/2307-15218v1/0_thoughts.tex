\pagebreak


\section{Claims We Believe / Think About}

\newcommand{\goal}{g}
\newcommand{\sink}{\underline{s}}
\newcommand{\dist}{\mathsf{dist}}

Without loss of generality: We have two distinguished states $\goal$ (goal) and $\sink$ (sink), where $\sink$ is the only state that cannot reach the $\goal$.

Notation $T(s, B_2)$ - Budget P1 needs to win against P2 budget $B_2$ in $s$.
$B(s, B_2)$ - Optimal bids of P1 with budget $T(B_2)$ in $s$.
$c(s)$ continuous ratio in vertex $s$ ($B_1 / B_2$ as opposed to $B_1 / (B_1 + B_2)$).


\subsection{DAGs}

\begin{lemma}
	The games never take longer than the height of the DAG.
	In particular, if P1 is winning, the game ends after at most $\dist(s, \goal)$ steps, if P2 is winning it takes $\dist(s, \sink)$.
\end{lemma}
\begin{proof}
	Follows immediately: If there are more moves in the game than the longest path, we must have reached the goal or some region from which the goal is not reachable.
\end{proof}

\begin{lemma}
	Pipe Conjecture: $c(s) \cdot (B_2 - \dist(s, \sink)) \leq T(s, B_2) \leq c(s) \cdot B_2$
\end{lemma}

\begin{proof}
	Upper bound: Replicate the continuous strategy of P1 rounding down (Djordje), we can maintain the bound as invariant.

	Lower bound:
	In this case, P1 has less than $c(s) (B_2 - \dist(s, \sink))$.
	In other words, $B_2 > \frac{1}{c(s)} B_1 + \dist(s, \sink)$.
	Note that P2 has a winning strategy in the continuous game if $B_2 > \frac{1}{c(s)} B_1$.
	Replicate the continuous strategy of P2 rounded up.
	By the above lemma, the game is over after $\dist(s, \sink)$ steps, thus the rounding up costs at most $\dist(s, \sink)$ additional budget, which P2 has available.
\end{proof}

\begin{lemma}
	Both sides of the inequality are tight.
	Formally, there exist infinitely many games $G$ such that for any $B$ there exist budgets $B^+, B^- \geq B$ where $T(s, B^+) = c(s) \cdot B^+$ and $T(s, B^-) = c(s) \cdot (B^- - \dist(s, \sink))$.
\end{lemma}
\begin{proof}
	Should happen in any race game and follow directly from the characterization.
\end{proof}

\begin{lemma}
	$v_-$ never stabilizes.
\end{lemma}
\begin{proof}
	Example: Choose between race(2,2) and race(3,3) (should work with any co-prime n and m in race(n,n) + race(m,m)).
	Needs proof but should be simple, the behaviour is 6-periodic (general: lcm(n,m)).
\end{proof}

\begin{lemma}
	If $c(v_i) \neq c(v_j)$ for all successors $v_i \neq v_j$ of $v$, then $v_-$ eventually stabilizes.
\end{lemma}
\begin{proof}
	Someone knows it.
\end{proof}

\begin{remark}
	Can we say anything about relation of discrete to continuous bids?
\end{remark}

\subsection{General}

Both pipe and bid conjecture do not hold on general games (witnesses by large enough tug-of-war).

%\begin{lemma}
%	For any vertex \(v\) and a \PT budget \(B_2\), if \(T(v, B_2) \neq \infty\) then there exists a bid \(b <= B_2\) such that the following inequality holds:
%	\begin{align*}
%		T(v, B_2) - b &\geq T(v-, B_2)\\
%		T(v, B_2) &\geq T(v+, B_2 - (b+1))
%	\end{align*}
%where \(T(v-, B) = \min_{u \in N(v)} T(u, B)\) and \(T(v+, B) = \max_{u \in N(v)} T(u, B)\) for any budget \(B\).
%\end{lemma}
%
%\begin{proof}
%	\(T(v, B_2) \neq \infty\) means there is a winning bid for \PO, and that must satisfies the above inequalities.
%\end{proof}

%\begin{lemma}
%	For any vertex \(v\) and any \PT budget \(B_2\), if \(T(v, B_2) \neq \infty\) then 
%\end{lemma}

\begin{lemma}
	Suppose that some bid is optimal infinitely often, i.e.\ $c \in B(s, B_2)$ for infinitely many $B_2$.
	Then, $B(s, B_2) = \{0\}$.
\end{lemma}
\begin{proof}
	TODO?
\end{proof}

\begin{lemma}
	Let $a, c \in B(s, B_2)$ and $a \leq b \leq c$.
	Then $b \in B(s, B_2)$.
	In other words, the set of winning bids is an interval.
\end{lemma}
\begin{proof}
	From \(c < b\) and the fact that \(b \in B(s, B_2)\), we have \(T(s, B_2) - c \geq T(s, B_2) - b \geq T(s_-, B_2) \).
	Therefore, if \PT bids \(0\) in response to \PO's bidding of \(c\) from \(s\) while he has budget \(T(s, B_2)\), he wins.
	
	On the other hand, because \(T\) is monotonically non-decreasing function, we have \(T(s_+, B_2 - (a+1)) \geq T(s_+, B_2 - (c+1))\).
	Because \(a\) is a wining bid by hypothesis, we already have \(T(s, B_2) \geq T(s_+, B_2 - (a+1))\).
	Hence, \(T(s, B_2) \geq T(s_+, B_2 - (a+1))\), which implies if \PT wins the bid at \(s\) by bidding \(c+1\) in response to \PO's bid of \(c\), whatever successor vertex she chooses, \PO wins from there.
	
	Therefore, \(c\) is indeed a winning bid from \PO from \(s\) when he has budget \(T(s, B_2)\).
\end{proof}

\begin{lemma}
	Let $l, h$ such that $B(s, B_2) = [l, h]$.
	Then, $h = T(s, B_2) - T(s_-, B_2)$ and either $l = h$ or $T(s, B_2) = T(s_+, B_2 - (l + 1))$.
\end{lemma}
\begin{proof}
 	Because \(h\) is a winning bid, it must holds \(T(s, B_2) - h \geq T(s_-, B_2)\).
 	Suppose towards contradiction that \(T(s, B_2) - h > T(s_-, B_2)\), which implies \(T(s, B_2) - (h+1) \geq T(s_-, B_2)\).
 	This means if \PO bids \(h+1\), and \PT bids \(0\), she loses the game.
 	On the other hand, if she wins the bid at \(s\) by bidding \(h+2\) against \PO's bid \(h+1\), the game ends up in the budget configuration \(\langle T(s, B_2), B_2 - (h+2) \rangle\) at a \PT's chosen successor.
 	Because \(h\) is a winning bid by the hypothesis, the budget configuration \(\langle T(s, B_2), B_2 - (h+1)\) is winning for \PO at any \PT's chosen successor. 
 	Hence, \(\langle T(s, B_2), B_2 - (h+2) \rangle \) must also be winning for him.
 	This shows that \PO wins if he bids \(h+1\) at \(s\), contradicting our assumption that \(h\) is the highest possible winning bid.
 	Therefore, \(h = T(s, B_2) - T(s_-, B_2)\).
 	
 	For the next part, we consider \(l \neq h\).
 	Because \(l\) is a winning bid for \PO, the following must holds: \(T(s, B_2) \geq T(s_+, B_2 - (l+1))\).
 	Suppose towards contradiction that, the inequality is strict, i.e, \(T(s, B_2) > T(s_+, B_2 - (l+1))\)
 	We define \(T'(s, B_2) = T(s, B_2) -1\).
 	The above strict inequality implies \(T'(s, B_2)  \geq T(s_+, B_2 - (l+1))\).
 	Because \(l \neq h\), we have at least \(l+1\) is also a winning bid for \PO when he has budget \(T(s, B_2)\).
 	In particular, if \PT bids \(0\) against \PO's bid of \(l+1\) from \(s\), she loses, i.e, \(T(s, B_2) - (l+1) \geq T(s_-, B_2)\), which is equivalent to say \(T'(s, B_2) - l \geq T(s_-, B_2)\).
 	By combining the two inequalities \(T'(s, B_2)  \geq T(s_+, B_2 - (l+1))\) and \(T'(s, B_2) - l \geq T(s_-, B_2)\), we have a \PO's winning budget \(T'(s, B_2) < T(s, B_2)\), and a corresponding winning bid \(l\) from \(s\).
 	This contradicts the minimality of \(T(s, B_2)\) as a winning budget for \PO.
 	Therefore, we have \(T(s, B_2) = T(s_+, B_2 - (l+1))\) when \(l \neq h\).
 	
 	
 	
 	
\end{proof}

\begin{lemma}
	Weak Pipe Conjecture: $c(s) \cdot (B_2 - \mathcal{O}(\log B_2)) \leq T(s, B_2) \leq c(s) \cdot B_2$
\end{lemma}
\begin{proof}
	Upper bound: Same reasoning as in the DAG case.

	Lower bound: We would need a termination-time bound.
\end{proof}

\begin{lemma}
	Termination time: P1 can win against $B_2$ if and only if P1 can win $G[\mathcal{O}(\log B_2)]$ (where $G[n]$ is the $n$-step unfolding of $G$).
\end{lemma}
\begin{proof}
	No idea.
\end{proof}
This would probably imply the lower bound of the weak pipe conjecture by applying the strong one to the unrolled game.
Also implies a bound of $\mathcal{O}(B_2 \cdot \log(B_2) \cdot |S|)$ for the VI.

\begin{lemma}
	Every $|S|$ steps, there has to be a bid of at least $B2 / |S|$ (or some other factor).
\end{lemma}
\begin{proof}
	This seems sort of obvious but no idea how to prove it.
\end{proof}
This would probably imply the above.

\begin{lemma}
	Weak bid conjecture: Let again $b_c = \mathsf{cont\_bid}(s)$.
	Then, $B(s, B_2) \subseteq $ is $[b_c \cdot B_2 - \mathcal{O}(\log B_2), \lceil b_c \cdot B_2 \rceil]$.
	Implies that $|B(s, B_2)| \in \mathcal{O}(\log B_2)$.
\end{lemma}


\subsection{Open Questions / Ideas}

\begin{itemize}
	\item Are first-price, poorman, \PO wins ties, reachability bidding games determined for any graph game? Are they determined for DAGs?
	
	\item  Like Race-games, do the surely winning threshold budgets for reachability in DAGs have any close form? then, possibly for general graphs?
	
	\item  How fast the discrete ratios converge to the the continuous ratios?
	
	\item Do there exist games where the switching of $v_-$ follows an irregular pattern?
\end{itemize}
