\section{Closed-form Solutions}\label{sec:closed-form}

In this section, we show closed-form solutions for threshold budgets in two special classes of games. 


\subsection{Race Games}

Race games are a class of games played on DAGs. For $a, b \in \Nat$, the race game $\race{a,b}$ ends within $a+b$ turns, \PO wins the game if he wins $a$ biddings before \PT wins $b$ biddings. The key property of race games that we employ is that for each vertex $v$ independent of the budgets, there is a neighbor $v_i$ such that \PLi proceeds to $v_i$ upon winning the bidding at $v$, for $i \in \set{1,2}$. 
\cref{fig:race} depicts \race{3,3}.
% Figure environment removed

In the following, we establish closed-form of threshold budgets at any vertex of a race game \race{a, b} by induction. 
% See the supplementary material for details and examples.



\begin{restatable}{theorem}{thresholdrace}
\label{thm:thresholdrace}
Let $v$ be the root of a race game $\race{a,b}$. Then $T_v(B_2) = a\cdot \floor{B_2/b}$.
\end{restatable}

\begin{proof}
	First note that, the threshold budget for \PO is \(0\) at \(t\), and \(\infty\) at \(s\).
	Let us denote any vertex of the race game as \(v_{x,y}\) where \(x\) and \(y\) are the minimum distance from the vertex to \(t\) and \(s\),  respectively.
	In this notation, the root \(v\) of \(\race{a,b}\) is referred to as \(v_{a,b}\).
	
	Note that, a subgame of \(\race{a,b}\) rooted at any vertex \(v_{x, y}\) is \(\race{x,y}\) itself. 
	We, in fact, show in the following: \(T_{v_{x, y}} = x \cdot \floor{\frac{B}{y}}\) which implies what we require.
	
	Let us now consider \(v_{1,1}\).
	At this vertex, \PO has to win the bid, otherwise \PT simply moves the token to \(s\).
	Because \PT has a budget of \(B\), and \PO wins all ties,
	his threshold budget at this vertex is \(B\), and he bids his whole budget.
	
	By induction on \(x\), we can argue that for any vertex of the form \(v_{x, 1}\), the threshold budget is \(xB\), because \PO has to win all \(x\) the bids to prevent the token reaching \(s\).
	Thus he has to bid at least \(B\) at all those \(x\) bids.
	In fact, if he has budget at most \(xB - 1\) at vertex \(v_{x,1}\), then \PT has a winning strategy:
	she bids \(B\) until she wins.
	
	Similarly, by induction on \(y\), we claim that for any vertex of the form \(v_{1, y}\), the threshold budget is \(\floor{\frac{B}{y}}\).
	The base case of this induction is \(v_{1,1}\), for which we showed earlier that the statement is true.
	Let us assume it is true for \(v_{1,y-1}\), and we prove the claim for \(v_{1,y}\).
	
	We suppose \PO's budget at \(v_{1, y}\) is \(\floor{\frac{B}{y}}\) while \PT's budget is \(B\).
	We claim that the winning strategy for \PO at vertex \(v_{1,y}\) is to bid his whole budget itself.
	If he wins the bid, he moves the token to target.
	Otherwise, \PT wins the bid by at least bidding \(\floor{\frac{B}{y}} +1\), and of course, she moves the token to \(v_{1, y-1}\) .
	Thus, her budget at \(v_{1, y-1}\) is at most \(B - (\floor{\frac{B}{y}} +1)\), while \PO's budget remains \(\floor{\frac{B}{y}}\).
	From the induction hypothesis, we know when \PT has a budget \(B - (\floor{\frac{B}{y}} +1)\), \PO's threshold budget for surely winning is \(\floor{\frac{B - (\floor{\frac{B}{y}} +1)}{y-1}}\).
	If we can show that, \(\floor{\frac{B}{y}} \geq \floor{\frac{B - (\floor{\frac{B}{y}} +1)}{y-1}}\), we are done.
	We show this in the following:
	\begin{align*}
		B - y \cdot \floor{\frac{B}{y}} &\leq y-1\\
		\implies B - \floor{\frac{B}{y}} &\leq (y-1) \cdot \floor{\frac{B}{y}} + (y-1)\\
		\implies \frac{B - \floor{\frac{B}{y}}}{y-1} &\leq \floor{\frac{B}{y}} + 1\\
		\implies \frac{B - (\floor{\frac{B}{y}}+1)}{y-1} &\leq \floor{\frac{B}{y}} + \frac{y-2}{y-1}
	\end{align*}
	
	Because \(\floor{\frac{B}{y}}\) itself is an integer, by taking \(\floor{}\) on the both side, we get \(\floor{\frac{B}{y}} \geq \floor{\frac{B - (\floor{\frac{B}{y}} +1)}{y-1}}\).
	Therefore, \(\floor{\frac{B}{y}}\) is a sufficient budget for \PO to surely win from vertex \(v_{1,y}\).
	
	Now, we need to show that this is also necessary budget for him.
	In fact, we show that when \PO has budget at most \(\floor{\frac{B}{y}} - 1\), while \PT's budget is \(B\), she has a surely winning strategy from \(v_{1,y}\).
	Her winning strategy is bidding \(\floor{\frac{B}{y}}\), until she reaches \(s\).
	Because \(B \geq y \cdot \floor{\frac{B}{y}}\), she can actually bids likewise.
	At each vertex, \PO's budget will be strictly less than what she is bidding, therefore he looses all the \(y\) bids, and the token indeed reaches the safety vertex.
	
	For a general vertex \(v_{x, y}\), we argue by induction which goes like above.
	We assume that for \(v_{x-1,y}\) and \(v_{x, y-1}\), which are the only two neighbours of \(v_{x, y}\), the threshold budget for \PO is \((x-1) \cdot \floor{\frac{B}{y}}\) and \(x \cdot \floor{\frac{B}{y-1}}\), respectively.
	
	We suppose \PO's budget at \(v_{x, y}\) is \(x \cdot \floor{\frac{B}{y}}\), and \PT's budget is \(B\).
	We claim that his wining strategy at the first bid is to bid \(\floor{\frac{B}{y}}\).
	We show that irrespective of where the token gets placed at the next vertex, he will have the respective threshold budget at that vertex.
	
	If he wins the bid at \(v_{x, y}\), his new budget becomes \((x-1)\cdot \floor{\frac{B}{y}}\), which is exactly what he needs to surely win from \(v_{x-1, y}\).
	If he looses, and the token gets placed at \(v_{x, y-1}\), \PT's budget becomes at most \(B - \floor{\frac{B}{y}}+1)\).
	It remains to show that \(x \cdot \floor{\frac{B -  \floor{\frac{B}{y}}+1)}{y-1}} \leq x\cdot \floor{\frac{B}{y}}\), which is true as we have earlier established \(\floor{\frac{B}{y}} \geq \floor{\frac{B - (\floor{\frac{B}{y}} +1)}{y-1}}\).
	It proves that \(x\cdot \floor{\frac{B}{y}}\) is the sufficient budget for \PO to surely win from \(v_{x, y}\).
	
	Finally, if \PO's budget is at most \(x \cdot \floor{\frac{B}{y}} - 1\) and \PT's budget is \(B\), then \PT wins the game if she bids \(\floor{\frac{B}{y}}\) at each bidding.
	This can be shown by another inductive argument where we assume the statement being true for vertices \(v_{x-1, y}\) and \(v_{x, y-1}\), and follow the same steps that we did for \PO above.
\end{proof}

With the exact closed-form of threshold budgets for race games, we now show that the bounds in \cref{stm:pipe} are tight.

\begin{corollary}  \label{stm:pipe_tight}
	For every rational number $q = n / m$, there exist infinitely many games $\G$ with vertex $v$ such that $t_v = q$ and for infinitely many $B$ the lower and upper bound of~\cref{stm:pipe} actually is an equality for some $B_2 > B$.
\end{corollary}
\begin{proof}
	Choose $\G = \race{n, m}$ (or any multiple thereof) and insert the closed form of \cref{thm:thresholdrace}.
	Note that in a race game $\textnormal{max-path}(v)$ of the root vertex $v$ clearly is $\max(n,m)$.
\end{proof}



%\begin{proof}[Proof Sketch]
%	At a vertex which is a step away from \PO's target, he bids with his whole budget. 
%	On the other hand, at a vertex which is a step away from the safety vertex, \PO needs to bid at least as much as \PT's budget.
%	These gives the necessary budgets at vertices one-step away from either sink. 
%	For other vertices, we use induction on these steps. 
%	Full proof can be found in the supplementary material.
%\end{proof}



\subsection{Tug-of-War games}
Given an integer \(n \geq 1\), a \textit{tug-of-war} game $\TUG(n)$ is a game played on a chain with $n+2$ nodes, namely $n$ interior nodes and two endpoints $s$ and $t$. We develop closed-form representations of thresholds in  \(\TUG(2)\) and \(\TUG(3)\) (both depicted in \cref{fig:tow}). 
For integers $k\in[1,n]$ and $b\ge 0$, we denote by $\tug(n,k,b)$ the smallest budget that \PO needs to win the tug-of-war game $\TUG(n)$ at the vertex  that is $k$ steps from his target $t$, when the opponent has budget $b$.
%Here we state closed-form solutions for \(\TUG(2)\) (shown in \cref{ex:TOW}) and \(\TUG(3)\).
%The technical proofs can be found in the supplementary material.

% Figure environment removed


\begin{restatable}{theorem}{tugtwo}\label{thm:tug2}
	For \(b \geq 0\), we have 
        \(\tug(2,1,b)=\floor{b/\phi}\) and \(\tug(2,2,b)=\floor{b \cdot \phi}\),
       where \(\phi = (\sqrt{5}+1)/2 \approx 1.618\) is the golden ratio.
\end{restatable}

\stam{
\begin{proof}
	To simplify notation, we use the same vertex names as in Fig.~\ref{fig:TOW} and, for a \PT budget $b$, we denote by \(t_b = \tug(2,1,b)\) and \(u_b = \tug(2,2,b)\), the thresholds in $v_1$ and $v_2$, respectively.
	The core of the proof follows from the following properties of \(t_b\) and \(u_b\):
	\begin{enumerate}
		\item\label{itm:tuga} \(t_0 = u_0 = 0\)
		
		\item\label{itm:tugb} \(u_b = t_b + b\) for any \(b \geq 1\)
		
		\item\label{itm:tugc} \(t_b = \min_x\{\max (x, u_{b - 1-x}) \mid 0 \leq x \leq b\}\) for any \(b \geq 1\)
	\end{enumerate}
	\cref{itm:tuga} is trivial: both players bid $0$, \PO wins ties, thus he wins all biddings (see Example~\ref{ex:TOW}).
	For \cref{itm:tugb}, consider the configuration $\zug{v_2, u_b, b}$. Since $v_2$ neighbors $s$, it is dominant for \PT to bid all her budget $b$. In order to avoid losing, \PO must bid $b$, and the game proceeds to $\zug{v_1, u_b-b, b}$, thus $t_b = u_b-b$. For \cref{itm:tugc}, consider a configuration $\zug{v_1, x, b}$ from which \PO wins, i.e., $x \geq t_b$. Note that it is dominant for \PO to bid his whole budget $x$. In order to avoid losing, \PT must bid $x+1$, and proceed to $\zug{v_2, x, b-(x+1)}$ from which \PO wins, thus $x \geq u_{b-(x+1)}$, and $t_b$ is obtained from the minimal such $x$. 
	
	This gives us the system of three equations with three unknowns (for a fixed \(b\)), thus existence of an unique solution, if any. 
	In the supplementary material, we verify that the expressions \(t_b = \floor{\frac{b}{\phi}}\) and \(u_b = \floor{b \cdot \phi}\) satisfy the equations.
	\end{proof}}
	%, detailed explanation of this can be found in the supplementary material. 	

\begin{proof}
	To simplify the notation, let us assume \(t_b = \tug(2,1,b)\), and \(u_b = \tug(2,2,b)\).
	We first claim that \(t_b\) and \(u_b\) are the unique solution to the following system of recurrence relations. 
	\begin{enumerate}
		\item\label{itm:tuga} \(t_0 = u_0 = 0\)
		
		\item\label{itm:tugb} \(u_b = t_b + b\) for any \(b \geq 1\)
		
		\item\label{itm:tugc} \(t_b = \min_x\{\max (x, u_{b - 1-x}) \mid 0 \leq x \leq b\}\) for any \(b \geq 1\)
	\end{enumerate}
	\cref{itm:tuga} is obvious because \PO bids \(0\) at every step and he wins ties, when \PT has a budget \(0\).
	
	\PO needs to win at the vertex which is \(2\) steps away from his target, otherwise \PT moves the token to the other end-point.
	Therefore, \PO needs to bid \(b\), and his new budget should be, by definition, at least \(t_b\) upon winning.
	This gives us \cref{itm:tugb}.
	
	Finally, at the vertex which is a single step away from \PO's target, he needs to optimize what his bid would be between \(0\) and \(b\) so that even if he loses the current bid, he would have enough budget at the next step to win from there (i.e, \(u_b\)).
	This gives us \cref{itm:tugc}.
	
	Moreover, the system of equations has a unique solutions, as there are as many equations as there are unknowns (\(t_b, u_b\) for a fixed \(b\)).
	Hence, it is enough to show that the expressions \(t_b = \floor{\frac{b}{\phi}}\) and \(u_b = \floor{b \cdot \phi}\) satisfy those equations.
	Clearly, \(\floor{\frac{0}{\phi}} = \floor{0 \cdot \phi} = 0\), so \cref{itm:tuga} holds.
	Next note that the golden ratio satisfy \(\phi = 1 + 1/\phi\). 
	Thus,
	\[ u_b = \floor{b \cdot \phi} = \floor{b \cdot (1 + 1/\phi)} = \floor{b + b/\phi} = b + \floor{b/\phi} = b + t_b\]
	
	implying \cref{itm:tugb} holds too.
	
	Finally, note that the function $f\colon x\to x$ is increasing, hence to verify \cref{itm:tugc} we need to show two inequalities for any $b\geq 1$:
	\begin{enumerate}
		\item For $x=\floor{b/\phi}$ we have $\floor{(b-1-x)\cdot \phi}\leq \floor{b/\phi}$.
		\item For $x=\floor{b/\phi}-1$ we have $\floor{(b-1-x)\cdot \phi}\geq \floor{b/\phi}$.
	\end{enumerate}
	In both cases, we will do this by checking that the insides of the two floor functions being compared satisfy the same inequality. Upon plugging in $x$, it thus suffices to show
	\[(b-1-\floor{b/\phi})\cdot \phi \leq b/\phi
	\quad\text{and}\quad
	(b-\floor{b/\phi})\cdot \phi \geq b/\phi.
	\]
	From $\phi=1+1/\phi$ we have $b\cdot\phi-b/\phi=b$, so the desired inequalities rewrite as
	\[ b-\phi \leq \floor{b/\phi}\cdot\phi
	\quad\text{and}\quad
	\floor{b/\phi}\cdot\phi\leq b.
	\]
	Those two inequalities follow from the obvious inequalities $b/\phi-1\leq \floor{b/\phi} \leq b/\phi$ after multiplying by $\phi$.
\end{proof}

%	Moreover, the system of equations has a unique solutions, as there are as many equations as there are unknowns (\(t_b, u_b\) for a fixed \(b\)).
%	Hence, it is enough to show that the expressions \(t_b = \floor{\frac{b}{\phi}}\) and \(u_b = \floor{b \cdot \phi}\) satisfy those equations.
%	Clearly, \(\floor{\frac{0}{\phi}} = \floor{0 \cdot \phi} = 0\), so \cref{itm:tuga} holds.
%	Next note that the golden ratio satisfy \(\phi = 1 + 1/\phi\). 
%	Thus,
%	\[ u_b = \floor{b \cdot \phi} = \floor{b \cdot (1 + 1/\phi)} = \floor{b + b/\phi} = b + \floor{b/\phi} = b + t_b\]
%	
%	implying \cref{itm:tugb} holds too.
%	
%	Finally, note that the function $f\colon x\to x$ is increasing, hence to verify \cref{itm:tugc} we need to show two inequalities for any $b\geq 1$:
%	\begin{enumerate}
%		\item For $x=\floor{b/\phi}$ we have $\floor{(b-1-x)\cdot \phi}\leq \floor{b/\phi}$.
%		\item For $x=\floor{b/\phi}-1$ we have $\floor{(b-1-x)\cdot \phi}\geq \floor{b/\phi}$.
%	\end{enumerate}
%	In both cases, we will do this by checking that the insides of the two floor functions being compared satisfy the same inequality. Upon plugging in $x$, it thus suffices to show
%	\[(b-1-\floor{b/\phi})\cdot \phi \leq b/\phi
%	\quad\text{and}\quad
%	(b-\floor{b/\phi})\cdot \phi \geq b/\phi.
%	\]
%	From $\phi=1+1/\phi$ we have $b\cdot\phi-b/\phi=b$, so the desired inequalities rewrite as
%	\[ b-\phi \leq \floor{b/\phi}\cdot\phi
%	\quad\text{and}\quad
%	\floor{b/\phi}\cdot\phi\leq b.
%	\]
%	Those two inequalities follow from the obvious inequalities $b/\phi-1\leq \floor{b/\phi} \leq b/\phi$ after multiplying by $\phi$.
%\end{proof}


\stam{OLD	
	We first claim that \(t_b\) and \(u_b\) are the unique solutions to the following system of recurrence relations. 
	\begin{enumerate}
		\item\label{itm:tuga} \(t_0 = u_0 = 0\)
		
		\item\label{itm:tugb} \(u_b = t_b + b\) for any \(b \geq 1\)
		
		\item\label{itm:tugc} \(t_b = \min_x\{\max (x, u_{b - 1-x}) \mid 0 \leq x \leq b\}\) for any \(b \geq 1\)
	\end{enumerate}
	\cref{itm:tuga} is obvious because \PO bids \(0\) at every step and he wins ties, when \PT has a budget \(0\).
	
	\PO needs to win at the vertex which is \(2\) steps away from his target, otherwise \PT moves the token to the other end-point.
	Therefore, \PO needs to bid \(b\), and his new budget should be, by definition, at least \(t_b\) upon winning.
	This gives us \cref{itm:tugb}.
	
	Finally, at the vertex which is a single step away from \PO's target, he needs to optimize what his bid would be between \(0\) and \(b\) so that even if he loses the current bid, he would have enough budget at the next step to win from there (i.e, \(u_b\)).
	This gives us \cref{itm:tugc}.
	
	This gives us the system of three equations with three unknowns (for a fixed \(b\)), thus existence of an unique solution, if any. 
	We can indeed verify that the expressions \(t_b = \floor{\frac{b}{\phi}}\) and \(u_b = \floor{b \cdot \phi}\) satisfy those equations, detailed explanation of this can be found in the supplementary material. 
}

\begin{remark}
\label{rem:Wythoff}
\normalfont
The closed-form solution in \cref{thm:tug2} has a striking similarity to a classic result in Combinatorial Game Theory. {\em Wythoff Nim} is played by two players who alternate turns in removing chips from two stacks. A configuration of the game is $\zug{s_1, s_2}$, for integers $s_1 \geq s_2 \geq 0$, representing the number of chips placed on each stack. A player has two types of actions: (1) choose a stack and remove any $k>0$ chips from that stack, i.e., proceed to $\zug{s_1 -k, s_2}$ or $\zug{s_1, s_2 - k}$, or (2) remove any $k >0$ chips from both stacks, i.e., proceed to $\zug{s_1 - k, s_2 -k}$. The player who cannot move loses. Wythoff~\cite{Wyt07} identified the configurations from which the first player to move loses. Trivially, $\zug{0,0}$ is losing, followed by $\zug{1, 2}, \zug{3,5}, \ldots$. In general, the $n$-th losing configuration is $\zug{\lfloor n \cdot \phi \rfloor, \lfloor n \cdot \phi \rfloor + n}$. Note the similarity to the thresholds in  $v_2$ and $v_1$, which can be written respectively as $\zug{\lfloor b \cdot \phi \rfloor, \lfloor b \cdot \phi \rfloor - b}$, for $b \geq 0$.
\end{remark}



\begin{restatable}{theorem}{tugthree}\label{thm:tug3}
	For $b\ge 1$ we have
	\(\tug(3,1,b)=\floor{\frac{b-1}{2}}\), \(\tug(3,2,b)=b-1\), and \(\tug(3,3,b)=2b-1\).
\end{restatable}


\begin{proof}
	We proceed similarly to the proof of~\cref{thm:tug2}.
	This time, we need to check that the expressions
	\[t_b=\floor{(b-1)/2}, \quad u_b=b-1, \quad\text{and}\quad v_b=2b-1
	\]
	satisfy the relations
	\begin{enumerate}
		\item\label{itm:tug3a} $t_1=u_1=0$, $v_1=1$,
		\item\label{itm:tug3b} $v_b=u_b+b$ for any $b\geq 2$,
		\item\label{itm:tug3c} $u_b= \min_x \{  \max\{t_b+x, v_{b-1-x}\} \mid 0\leq x \leq b\}$ for any $b\geq 2$.
		\item\label{itm:tug3d} $t_b= \min_x \{  \max\{x, u_{b-1-x}\} \mid 0\leq x \leq b\}$ for any $b\geq 2$.
	\end{enumerate}
	This time, both \cref{itm:tug3a} and \cref{itm:tug3b} follow by direct substitution.
	
	Regarding \cref{itm:tug3c}, we need to show that
	\[ b-1=  \min_x \{  \max\{\floor{(b-1)/2}+x, 2b-3-2x \} \mid 0\leq x \leq b\}
	\]
	To that end, we distinguish two cases based on the parity of $b$.
	If $b=2k$ is even then we need to show
	\[ 2k-1 = \min_x\{ \max\{k-1+x, 4k-3-2x \} \mid 0\leq x \leq 2k\},
	\]
	and indeed the minimum on the right-hand side is attained for $x=k-1$ and is equal to $2k-1$ as desired.
	Similarly, if $b=2k+1$ is odd then we need to show
	\[ 2k = \min_x\{ \max\{k+x, 4k-1-2x \} \mid 0\leq x \leq 2k\},
	\]
	and indeed the minimum on the right-hand side is attained for $x=k$ and is equal to $2k$ as desired.
	
	Finally, regarding \cref{itm:tug3d} we have
	$u_{b-1-x}=b-2-x$, hence the two numbers inside the $\max(\cdot)$ function always sum up to $b-2$.
	If $b=2k$ is even, then the minimum is $(b-2)/2=k-1 = \floor{(b-1)/2}=t_b$ as desired.
	If $b=2k+1$ is odd then the minimum is $\ceil{(b-2)/2}=k=\floor{(b-1)/2}=t_b$ as desired again.
\end{proof}


\stam{
\begin{proof}
	We proceed similarly to the proof of~\cref{thm:tug2}.
	This time, we need to check that the expressions
	\[t_b=\floor{(b-1)/2}, \quad u_b=b-1, \quad\text{and}\quad v_b=2b-1
	\]
	satisfy the relations
	\begin{enumerate}
		\item\label{itm:tug3a} $t_1=u_1=0$, $v_1=1$,
		\item\label{itm:tug3b} $v_b=u_b+b$ for any $b\geq 2$,
		\item\label{itm:tug3c} $u_b= \min_x \{  \max\{t_b+x, v_{b-1-x}\} \mid 0\leq x \leq b\}$ for any $b\geq 2$.
		\item\label{itm:tug3d} $t_b= \min_x \{  \max\{x, u_{b-1-x}\} \mid 0\leq x \leq b\}$ for any $b\geq 2$.
	\end{enumerate}
	This time, both \cref{itm:tug3a} and \cref{itm:tug3b} follow by direct substitution.
	
	Regarding \cref{itm:tug3c}, we need to show that
	\[ b-1=  \min_x \{  \max\{\floor{(b-1)/2}+x, 2b-3-2x \} \mid 0\leq x \leq b\}
	\]
	
	To that end, we distinguish two cases based on the parity of $b$.
	This analysis can be found in the supplementary material. 
	
	
%	If $b=2k$ is even then we need to show
%	\[ 2k-1 = \min_x\{ \max\{k-1+x, 4k-3-2x \} \mid 0\leq x \leq 2k\},
%	\]
%	and indeed the minimum on the right-hand side is attained for $x=k-1$ and is equal to $2k-1$ as desired.
%	Similarly, if $b=2k+1$ is odd then we need to show
%	\[ 2k = \min_x\{ \max\{k+x, 4k-1-2x \} \mid 0\leq x \leq 2k\},
%	\]
%	and indeed the minimum on the right-hand side is attained for $x=k$ and is equal to $2k$ as desired.
	
	Finally, regarding \cref{itm:tug3d} we have
	$u_{b-1-x}=b-2-x$, hence the two numbers inside the $\max(\cdot)$ function always sum up to $b-2$.
	Here too, we analyse by distinguishing the parity of \(b\), and the detailed argument can be found in the supplementary material. 
%	If $b=2k$ is even, then the minimum is $(b-2)/2=k-1 = \floor{(b-1)/2}=t_b$ as desired.
%	If $b=2k+1$ is odd then the minimum is $\ceil{(b-2)/2}=k=\floor{(b-1)/2}=t_b$ as desired again.
\end{proof}}


We note that for $n \ge 4$ the situation gets surprisingly more complicated.
For $n=5$ the threshold budgets do eventually converge to a simple pattern, but only from around $b=4\cdot 10^3$ on.
In contrast, for $n\in\{4,6\}$ the threshold budgets exhibit no clear pattern up until $b=10^6$.
Moreover, while the pipe theorem \cref{stm:pipe} seems to hold for $n \leq 5$ (experimentally validated up to $b = 10^7$), it is (quickly) violated for $n \geq 6$.
This suggests that a simple closed form solution for general games is unlikely, given that these structurally similar games behave so differently.

