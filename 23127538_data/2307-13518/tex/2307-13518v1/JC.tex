\documentclass[pra, twocolumn,showkeys]{revtex4}
\usepackage{bbm}
\usepackage{natbib}
\usepackage{graphics}
\usepackage{amsmath}
\usepackage{mathrsfs}
\usepackage{theorem}
\usepackage{float}
\usepackage{hyperref}
\usepackage{rotating}
\usepackage{amssymb}
\usepackage[english]{babel}
\usepackage{color}
%\usepackage{multirow}
\usepackage{fancybox}
%\usepackage[switch]{lineno}


\newcommand{\ket}[1]{|#1\rangle}
\newcommand{\bra}[1]{\langle #1|}
\newcommand{\inp}[2]{\langle #1 | #2\rangle}
\newcommand{\be}{\begin{eqnarray}}
\newcommand{\ee}{\end{eqnarray}}
\newtheorem{theorem}{Theorem}
%\newtheorem{theorem1}{Definition}


\begin{document}

\title{Quatnum Phase transition in the spin-boson model with rotating-wave approximation}
\author{H. T. Cui $^{1}$}
\email{cuiht01335@aliyun.com}
\author{Y. A. Yan $^{1}$}
\email{yunan@ldu.edu.cn}
\author{M. Qin $^{1}$}
%\email{qinming@ldu.edu.cn}
\author{X. X. Yi $^{2}$}
\email{yixx@nenu.edu.cn}
\affiliation{$^1$ School of Physics and Optoelectronic Engineering \& Institute of Theoretical Physics, Ludong University, Yantai 264025, China}
\affiliation{$^2$ Center for  Quantum Sciences, Northeast Normal University, Changchun 130024, China}
\date{\today}

\begin{abstract}
The study of phase transition in dissipative quantum systems based on the Liouvillian mostly relies on the time-local master equation, which  becomes difficult to attain when the coupling between the system and its environment is strong. To surmount this difficulty, the complex discretization approximation for environment is proposed to study the quantum phase transition in the spin-boson model under rotating-wave approximation. By this approach, a nonhermitian effective Hamiltonian is proposed to simulate the exact dynamics of spin. It is found that the ground state of this Hamiltonian dominates spin dynamics in the single-excitation subspace. Depending on the energy gap and the amplitude of ground state on a special basis state, three distinct phases can be identified, which describe the exponential decaying, localized and intermediate dynamics of spin respectively. Moreover,  these phases are stable  against the increasing of the total energy when extended to the  double-excitation subspace.
\end{abstract}

%\keywords{}

\maketitle

\section{introduction}
The study of open quantum systems has revealed that phase transitions can occur, even though the system has a finite number of degrees of freedom. Unlike the phase transitions in closed systems under thermodynamic conditions, the phase transitions in open quantum systems are accompanied by significant changes in the system's dissipative dynamics. To characterize this transition, the real part of the system's Liouvillian eigenvalues, which correspond to the decay rate of eigenmodes, is studied. It has been found that the vanishing of the real part of the eigenvalue indicates a non-analytical change in the dissipative dynamics of the system, which typically signals the onset of a phase transition \cite{kessler12}.

Recent experimental advances have allowed physicists to investigate regimes  beyond Markovian approximation, such as in solid-state \cite{solid-state} and artificial light-matter systems \cite{light-matter}. In this situation, the Liouvillian is difficult to determine due to the absence of a time-local master equation. This raise the question of whether  phase transitions can still occur in the absence of a Liouvillian and how they can be characterized. To address this,  perturbational expansion or numerical simulation  may be used to incorporate the influence of environment \cite{vega2017}, which typically has an infinite number of degrees of freedom. However, the resulting dynamical equations  are often very complex, making it difficult  to derive the Liouvillian \cite{breuer2002}.


An alternative approach  is to approximate the spectrum of the environment as discrete.  Thus, an effective  Hamiltonian with finite dimensional Hilbert space  can be constructed to simulate the dynamics of system \cite{kazansky97, discretization}. However, this approach requires  a large Hilbert space dimension  to accurately  predicate the long-term behavior of system, which would become unaffordable in computation. To surmount this difficulty,  a complex discretization approximation was proposed  by the authors \cite{cui23}, which allows for a more  efficient computation and more faithful depiction of the dissipative dynamics of the system due to the occurrence of complex eigenvalues, the imaginary parts of which represent the decay rate of corresponding eigenmodes.

The similarity of the Liouvillian and the complex Hamiltonian encourages us to explore the phase transition by solving the complex Hamiltonian. As an exemplification, the spin-boson model with rotating-wave approximation is chosen.  It is known that the spin-boson model undergoes a quantum phase transition in the sub-Ohmic or Ohmic regime from a delocalized phase with a single ground state and no magnetization, to a localized phase with two-fold degenerate ground state and a nonzero magnetization \cite{leggett, weiss}. However, it is difficult  to investigate the phase transition in spin-boson model due to the nonpreservation of total excitations. The rotating-wave approximation is a reasonable choice to obtain tractable solutions \cite{breuer2002}. This approach allows us to determine the phase transition rigorously by finding singularity of the eigenmodes. A similar research has been implemented in Ref. \cite{wang19},  where the effective  Hamiltonian is constructed in real space. In contrast to the previous study,  the current approach provides a better understanding of  the phase transition. The localization-decaying transition can be characterized faithfully by the vanishing  energy gap of  ground state. Additionally,  an intermediate phase between localized and decaying phases can be identified in the sub-Ohmic regime  by checking the eigenfunctions of complex Hamiltonian. Moreover, these phases are stable  against the increasing of the total energy, when extended to the  double-excitation subspace.


The remaining of the paper is organized as follows. The model and method are introduced in Section II, with particular  remarks on the complex discretization approximation. In section III, the open dynamics of spin is discussed explicitly in the single-excitation subspace, using constructed complex Hamiltonian. Resultantly, the phase diagram is founded explicitly. We find that in addition to the localized and decaying phases, an intermediate phase can be identified by the amplitude of ground state function on a special basis state. The spin dynamics is examined explicitly to illustrate the distinct property of the three phases.   In section IV, we extend the discussion into the double-excitation subspace. The critical observation is that the spin dynamics in this case is nearly the same as in single-excitation subspace.  Finally, conclusions  are offered in section V.


% Figure environment removed



% Figure environment removed

% Figure environment removed


\section{Model and method}

The Hamiltonian for the spin boson model under rotating wave approximation can be written as
\be\label{H}
H=\Delta \sigma_+ \sigma_- +\int \text{d}k \omega(k) a^{\dagger}_k a_k + \int \text{d}k  \left[ g(k) a^{\dagger}_k \sigma_- + g^*(k) a_k \sigma_+\right],
\ee
where $\sigma_{\pm}= \left(\sigma_x \pm \mathbbm{i} \sigma_y\right)/2$ with $\sigma_i (i=x, y, z)$ the Pauli matrices and $\mathbbm{i}$ is the imaginary unit. $ a^{\dagger}_k ( a_k)$ denotes the creation (annihilation) operator of the mode $k$ in environment, which is coupled to the spin via coupling strength $g_k$. In this paper, the spectral density for environment is chosen as
\be\label{J}
J(\omega)= \eta \omega \left(\frac{\omega}{\omega_c}\right)^{s-1}e^{-x/\omega_c},
\ee
where $\eta$ depicts the coupling strength, and $\omega_c$ is the cut-off of frequency. Depending the value of $s$, the environment is classified as sub-Ohmic $\left( 0< s < 1 \right)$, Ohmic $\left( s=1 \right)$ and super-Ohmic  $\left( s>1 \right)$.

Although the simplicity of Eq. \eqref{H}, the exact spin dynamics can be obtained only in the single-excitation subspace. In this case, the state of total system can be written in a discrete form as
\be \label{psit}
\ket{\psi(t)}=\alpha(t)\ket{e} \ket{0}^{\otimes N} + \ket{g} \left(\sum_{k} \beta_k(t) \ket{1}_k\right),
\ee
in which $N$ denotes the number of modes in environment,  $\ket{1}_k = a^{\dagger}_k \ket{0}_k$ with the vacuum state $\ket{0}_k$, and $\ket{e}= \sigma_+ \ket{g}$.  Substituting Eq. \eqref{psit} into the Schr\"{o}dinger equation, one gets
\be\label{alphat}
\mathbbm{i}\frac{\partial }{\partial t}\alpha (t)&=& \Delta \alpha(t)- \mathbbm{i} \int_0^t \text{d}\tau \alpha(\tau)\int_0^{\infty} \text{d} \omega J(\omega)e^{-\mathbbm{i} \omega(t-\tau)},
\ee
where the last term stems from the memory effect of environment. The equation above can be solved rigorously by iterative method.  However, duo to the memory effect, the calculation will become exhaustive for a long time evolution. For more than one excitation, the  dynamical equation similar to Eq. \eqref{alphat} is absent. Thus to tackle the spin dynamics, a feasible approach is to discretize the continuum in environment as a finite set of modes \cite{discretization}. Consequently, an effective Hamiltonian with finite dimension can be constructed, which is used to obtain the total dynamics by exact diagonalization. However, this approach suffers from the  recurrence due to the effect of finite dimension, which make the evaluation unaffordable for a long term evolution.

Recently, the authors have proposed to extend the discretization approximation into the complex plane to improve  the simulation of spin dynamics\cite{cui23}, which is known as the complex discretization approximation (CDA). In contrast to its counterpart in real space, this approach allows  for the construction of a non-Hermitian effective Hamiltonian $H_{\text{dis}}$, which shows complex eigenvalues with  the negative imaginary part. Consequently, $H_{\text{dis}}$ can provide accurate description for  the decaying dynamics of spin. An overview of this method is presented in \ref{appendixCDA}. The validity of CDA is demonstrated in Fig. \ref{fig:exact}, which shows that the evaluation is in good agreement with the exact results. Further details can be found in our paper\cite{cui23}. A brief introduction to this method is provided in \ref{appendixCDA}.

However, it is observed that the convergence is very slow for large $\eta$ when $s=1$, resulting in a weak difference from the exact calculation as shown in Fig. \ref{fig:exact}(c3). To improve the evaluation, we use $\widetilde{H}_{\text{dis}}=\sqrt{H_{\text{dis}} \cdot H^{\dagger}_{\text{dis}}}$  to simulate the long-term behavior of spin dynamics. As shown in Fig. \ref{fig:tildeH} for $s=1$,  $\widetilde{H}_{\text{dis}}$ can provide perfect simulation for the long term behavior of spin dynamics.  The  reason  may be  that the contribution of complex eigenmodes for $H_{\text{dis}}$ would tend to vanish for a long term evolution. Thus, the unwanted decaying effect has to be eliminated in order to attain the correct result. On the other hand,  $H^{\dagger}_{\text{dis}}$ can display the complex eigenvalues with positive imaginary part, which implies a gain effect. Thus, $\widetilde{H}_{\text{dis}}$ can balance the two opposite effects, allowing the spin  to reach the steady status quickly. It is worth stressing that  $\widetilde{H}_{\text{dis}}$ could not depict the complicated  spin dynamics in the short term  as shown in Fig.\ref{fig:tildeH} (c), which generally is a consequence of the interference in the eigenmodes of environment. Therefore, in the following discussion,  we will combine $H_{\text{dis}}$ and $\widetilde{H}_{\text{dis}}$ together  to determine the steady dynamics  of spin.

\section{Quantum Phase Transition in the single-excitation subspace}

At zero temperature, the phase transition can occur  when the energy gap of ground state disappears \cite{sachdev}. Similar to the case of  Liouvillian \cite{kessler12}, the ground state of $H_{\text{dis}}$ can be defined as the eigenfunction, of which the eigenvalue displays the  smallest real part. By exact diagonalization of $H_{\text{dis}}$, two different features can be found for the ground state. For small $\eta$, the calculation shows that the real parts for all eigenvalues are nonnegative, which makes up the energy band bounded in open interval $\left(0, 2\omega_c R\right)$. In contrast, with the increase of $\eta$, the ground state appears as a single eigenmode outside the band, showing eigenvalue negative real part. Thus,  the energy gap is defined as the negative real part of eigenvalue for the ground state. In Fig. \ref{fig:Hdis} (a1)-(a3), the real part of eigenvalue for the ground state is plotted versus the coupling strength $\eta$. It is evident that a critical point $\eta_I$ can be decided by the zero real part.


From the point of quantum phase transition, the occurrence of energy gap implies that the dynamics in the total system would demonstrate intrinsic variation. In order to examine this point,  the spin dynamics is explored by checking the survival probability $\left| \inp{\psi(t)} {\psi(0)}\right|^2$ for the initial state $\ket{\psi(0)}=\ket{e} \ket{0}^{\otimes N} $. As shown in Fig. \ref{fig:spindynamics}, the survival probability   can present distinct behaviors depending on the value of $\eta$. For small $\eta$, the survival probability decays exponentially. However, when $\eta$ increase,  the survival probability is  robust against decay, which means that  the excitation may be localized in the spin. Our calculations reveal that the transition from exponential decay to localization is always accompanied by the emergence of the energy gap in the ground state.

Interestingly, an intermediate situation between the decay and localization can be found, as shown by red-dashed line in Fig. \ref{fig:spindynamics} (a) and (b). This situation is characterized  by a slow decay in the survival probability. Our analysis reveals that this phenomenon can only occur when the energy gap  is open, i.e. $\eta>\eta_I$. Furthermore, this intermediate dynamics disappears when $\eta$ is greater than a critical value $\eta_{II}$, beyond which the spin  dynamics begin to localize. In \ref{appendixforIII}, a detailed  examination of spin dynamics near $\eta_{II}$ is provided in  Fig. \ref{fig:eta}. It is clear that the spin dynamics can vary significantly  when $\eta> \eta_{II}$. However, it is difficult to determine $\eta_{II}$  only through analyzing the spin dynamics since the variation is very subtle. Therefore, an analytical approach is necessary to decide $\eta_{II}$.


For this purpose, we investigate the wave function of ground state  explicitly to determine the threshold $\eta_{II}$ below which the intermediate dynamics appears. We observe that the amplitude of the ground state on the basis state $\ket{e} \ket{0}^{\otimes N}$ (labeled as $G_e$) is correlated with the spin dynamics for $\eta>\eta_I$. When the imaginary part of $G_e$ is nonzero, the spin dynamics decays slowly . While if it is zero, the localization can occur for spin dynamics. In Fig. \ref{fig:Hdis}(b1)-(b3), the imaginary part of $G_e$ is plotted to illustrate its relevance to $\eta$. It is direct that a critical value $\eta_II$ can be found by the vanishing imaginary part of $G_e$. We see that  the intermediate dynamics can be found only for $0 \leq s \leq \sim 0.5$. For larger $s$, the numerical calculation illustrates that $\eta_I$ and $\eta_{II}$ would tend to be the same, and the slow decaying corresponds to the critical dynamics. In Fig. \ref{fig:spindynamics}(c), the evolution of survival probability is plotted for $\eta=0.1$ (corresponding to $\eta_I$) and $0.5$ when $s=1$. We can see that the variation of spin dynamics near $\eta_I$ is very intricate.


In summary, a phase diagram can be constructed in Fig. \ref{fig:phase} to identify the three distinct phases (labelled by I, II and III), based on the energy gap and the imaginary part $G_e$. In phase I,  the ground state is  embedded in  the band, resulting in a zero energy gap. Thus, the excitation in spin decays exponentially into the environment. In phase II,  the energy gap is nonzero  and $G_e$ is real completely, leading to a stable spin dynamics and a finite probability of preserving the excitation. It is stressed that this phase is protected by both the energy gap and real $G_e$. The phase III is characterized by a nonzero energy gap and a nonzero imaginary part of $G_e$, resulting in a stretched decay of spin. The three regions are separated  by critical points $\eta_I$ or $\eta_{II}$ respectively. The observation of $\eta_I$ and $\eta_{II}$ implies that the dissipative phase transition can display more complex feature, which cannot be captured uniquely by the energy gap of ground state. Moreover, the approach of CDA demonstrates the special ability to depict the exotic dynamics in open quantum system.

It is worth noting that the relationship between region III and the imaginary part of $G_e$ is evident, yet the physical explanation for it is difficult to provide. This is due to the non-hermiticity of $H_{\text{dis}}$, which results in complex dynamics of spin. As seen in Eq. \ref{U} in \ref{appendixCDA}, the complex $\mathbbm{E}_n$ not only causes decay in the dynamics, but also interference in the different paths of evolution. Therefore, the occurrence of real $G_e$ may significantly alter the interference. Unfortunately, there is currently no analytical discussion on this matter.

\section{spin dynamics in double-excitation subspace}

% Figure environment removed

It is a theoretical challenge to broaden the discussion into the double-excitation subspace, as this would drastically increase the dimension of physical space, making the exact treatment of dynamics very exhaustive. Generally, the wave function in the double-excitation subspace can be written as
\be
\ket{\phi(t)}=\ket{e} \sum_{k=1}^{N}\alpha_k (t) a^{\dagger}_{k} \ket{0}^{\otimes N} + \ket{g}\sum_{k\leq k'}\beta_{k, k'}(t) a^{\dagger}_{k} a^{\dagger}_{k'}\ket{0}^{\otimes N}
\ee
It is evident that the spin state $\ket{e}$ is strongly correlated to the environment's status, thus making the dynamical equation similar to Eq. \eqref{psit} absent. To characterize the spin dynamics, a useful approach is to approximate the environment's spectrum as discrete, which can then be used to solve the dynamics of a closed many-body system \cite{shi16}.

As have done in the previous section, we use $H_{\text{dis}}$ to simulate spin dynamics in the double-excitation subspace. The probability  defined as
\be\label{pe}
P_e(t)= \sum_{k=1}^{N} \left| \alpha_k (t) \right|^2.
\ee
is evaluated for arbitrary time $t$, which describe the probability of spin on excited state $\ket{e}$. The initial state is chosen as $\ket{\phi (0)}=1/\sqrt{N} \ket{e} \sum_{k=1}^{N} a^{\dagger}_{k} \ket{0}^{\otimes N} $ to avoid the dependence on special mode in environment. We chose $N=200$ for simulation, which corresponds to the Hilbert space of dimension $\sim 2.0 \times 10^5$. In this case,  $P_e(t)$ can be evaluated faithfully up to $t\sim 10$. For greater $N$, the calculation becomes too consumptive to implement. The evaluation of $P_e(t)$ is illustrated by empty circles in Fig. \ref{fig:double}. Because of finite $N$, $P_e(t)$ displays significant fluctuation when $t>\sim 10$. For $s=1$, the convergency of computation is very slow, which induces strong fluctuation and large deviation.

An interesting observation is that the evolution of $P_e(t)$  is similar to that of  $\left|\inp{\psi(t)}{\psi (0)}\right|^2$ in single excitation subspace. This is  verified  further using Gauss quadratures method in real space \cite{discretization}, as illustrated by Fig. \ref{fig:realdouble} in \ref{appendixrealGQ}. This implies that the phase diagram in single-excitation subspace is robust against increasing  energy in the total system. This picture can be understood  by noting that the discretized environment corresponds to an interaction-free bosonic system, in which the excitations can freely populate the eigenmodes  without any interference. Thus, the decay  of excitation in spin is independent of the status of environment.

\section{Conclusion}

In conclusion, the quantum phases in the spin-boson model under rotating-wave approximation was investigated in this paper. To simulate the open dynamics of spin, a nonhermitian effective Hamiltonian $H_{\text{dis}}$ was  constructed using complex discretization approximation for the environment. The validity of this approach was confirmed  by examining the spin dynamics in single-excitation subspace, which was determined exactly by solving Eq. \ref{alphat}. By performing exact diagonalization of $H_{\text{dis}}$, the spin dynamics could be further examined by analyzing the ground state, which revealed eigenvalue with the smallest real part. It was found that the spin dynamics in single-excitation subspace is strongly related to two key properties of the ground state,  the energy gap, defined  as the negative real part of eigenvalue, and the amplitude on the basis state $\ket{e} \ket{0}^{\otimes N}$, labelled as $G_e$. When the energy gap is zero, the spin dynamics, measured by the survival probability of initial state $\ket{\psi (0)}=\ket{1} \ket{0}^{\otimes k}$, decays exponentially. On the other hand, if the energy gap is finite, two distinct features can be observed depending on the imaginary part of $G_e$. When $\Im (G_e)=0$,  the survival probability stabilizes eventually at a finite value, which indicates the localization of excitation in the spin. While for nonzero $\Im (G_e)=0$, a stretch decaying was observed. This implies that there would exist an intermediate case between  dissipative and localized spin dynamics. Two critical points $\eta_I$ or $\eta_{II}$ can be identified to  separates the three types of spin dynamics, decided by whether the energy gap is zero or $\Im (G_e)=0$. Additionally, $\eta_{II}$ display obvious relevance  to the spectral function $J(\omega)$. When $0< s<\sim 0.5$, $\eta_{II}$ is always greater than $\eta_I$. While for $s>0.5$, $\eta_{II}$ tends to coincide with $\eta_I$. However, a physical origin of $\eta_{II}$ is still unknown.

The spin dynamics was also investigated  in the double-excitation subspace by evaluating $P_e(t)$ defined in Eq. \eqref{pe}. Interestingly,  the same evolution as the survival probability in single-excitation subspace is observed. This can be attributed to the fact that the discretized environment corresponds to the interaction-free boson model, which enables the excitation  to  travel in the environment without the interference from the other excitation. This finding would allow for an exact analysis of the effects of finite temperature  on the phase transition, which  is usually done using   perturbational method or  high-temperature approximation \cite{tamascelli19}. Further research on this topic will be presented in the future.


\section*{ACKNOWLEDGEMENTS}
H.T.C. acknowledges the support of Natural Science Foundation of Shandong Province under Grant No. ZR2021MA036. Y. A. Yan acknowledges the support of National Natural Science Foundation of China (NSFC) under Grant No. 21973036.  M.Q. acknowledges the support of NSFC under Grant No. 11805092 and  Natural Science Foundation of Shandong Province under Grant No. ZR2018PA012. X.X.Y. acknowledges the support of NSFC under Grant No. 12175033 and National Key R$\&$D Program of China (No. 2021YFE0193500).



\renewcommand\thefigure{A\arabic{figure}}
\renewcommand\theequation{A\arabic{equation}}
\renewcommand\thesection{Appendix \Roman{section}}
\setcounter{equation}{0}
\setcounter{figure}{0}
\setcounter{section}{0}

\section{A brief introduction to complex discretization approximation}\label{appendixCDA}

% Figure environment removed


% Figure environment removed


%In this section, a brief introduction to the complex discretization approximation (CDA)  is presented. The comprehensive discussion can be found in our recent paper \cite{cui23}.

The main idea for CDA is the transformation
\be\label{complextransformation}
d_n&=& \int_{\Gamma}\text{d}z \sqrt{\frac{w(z)}{\mathbbm{i} z}} \eta_n(z)  a_z\nonumber\\
d_n^{\ddag}&=& \int_{\Gamma}\text{d}z \sqrt{\frac{w(z)}{\mathbbm{i} z}} \eta_n(z)  a_z^{\dagger}
\ee
and the inverse
\be\label{bz}
a_z&=&\sqrt{\frac{w(z)}{\mathbbm{i} z}} \sum_{n=0}^{N-1}\eta_n(z) d_n\nonumber\\
a_z^{\dagger}&=&\sqrt{\frac{w(z)}{\mathbbm{i} z}} \sum_{n=0}^{N-1}\eta_n(z) d_n^{\ddag},
\ee
where $a^{\dagger}_z ( a_z)$ is the complex generalization of $ a^{\dagger}_k ( a_k)$  by replacing $k$ with $z=x+\mathbbm{i}y$. By  weight function $w(z)$, the complex polynomial $\eta_n(z)$ of degree $n$ can be constructed through defining the inner product
\be \label{complexinp}
\left\langle f, g \right \rangle_{\Gamma} = \int_{\Gamma} \frac{\text{d}z}{\mathbbm{i} z} w(z) f(z)g(z).
\ee
It is stressed that  the inner product is defined deliberately without complex conjugation to construct the three-term recurrence relation for $\eta_n(z)$. $\Gamma$ denotes the semi-circle with radius $R$  in the lower half complex plane, centered at point $\left(R, 0\right)$. The variable $z$ is given by $z=R(1+e^{\mathbbm{i}\theta})$, where $\theta\in \left[\pi, 2\pi\right]$, and its real part is ensured to be nonnegative.  The parameter $R$ is responsible for limiting the energy mode in the environment, and thus has a direct effect on the accuracy and efficiency of the calculation.

The polynomials $\eta_n(z)$ possesses two properties, which are crucial for the success of evaluation. One is the orthonormality defined as
\be
\left\langle \eta_m, \eta_n \right \rangle_{\Gamma}=\delta_{m, n}.
\ee
The other is the recurrence relation
\be \label{complexrecurrence}
\sqrt{\nu_{n+1}} \eta_{n+1}(z)= \left(z -\mathbbm{i} \mu_n\right) \eta_{n}(z) -\sqrt{\nu_{n}}\eta_{n-1}(z),
\ee
where
\be
\mathbbm{i}\mu_n=\left\langle z \eta_n, \eta_n\right\rangle_{\Gamma},
\nu_n=\frac{A^2_{n-1}}{A_n^2},
\ee
$A_n$ denotes the coefficient of  $z^n$ in $\eta_n(z)$. With help of these two properties and transformation Eq. \eqref{bz} together, $H$ can be transformed into
\be
H&=&\Delta \sigma_+ \sigma_- + \omega_c\sum_{n=1}^{N}\left(\mathbbm{i}\mu_n d_n^{\ddag}d_n + \sqrt{\nu_{n}}d_n^{\ddag}d_{n-1}+ \text{h. c. } \right)+\nonumber\\
&& \sum_{n=1}^{N}\int_{\Gamma}\text{d}z \sqrt{\frac{w(z)}{\mathbbm{i} z}} \eta_n(z)\left[g(z) \sigma_+ d_n + g^*(z) d_n^{\ddagger}\sigma_-\right],
\ee
where $N$ denotes the highest degree of polynomial adopted to simulate the dynamics of the total system.
To obtain the form above, the replacement $\omega_k \rightarrow \omega_c z $ and $g_k \rightarrow g(z)= \sqrt{\eta} \omega_c z^{s/2} e^{- z/2}$ are applied according to the spectral density  $J(\omega)$ defined in Eq. \eqref{J}.  The next step is to define the new mode operator
\be \label{tildeD}
\widetilde{d}_i&=& \sqrt{w_i} \sum_{n=1}^{N} \eta_n(z_i) d_n,\nonumber \\
\widetilde{d}_i^{\ddagger}&=& \sqrt{w_i} \sum_{n=1}^{N} \eta_n(z_i) d_n^{\ddagger},
\ee
where $z_i$ denotes the root of $\eta_N(z)$, and $w_i$ is the corresponding weight. Both $z_i$ and $w_i$ can be obtained directly from the recurrence relation Eq. \eqref{complexrecurrence}. Finally,
$H$ can be simplified as
\be
H_{\text{dis}}&=&\Delta \sigma_+ \sigma_- +\sum_{n=1}^{N} z_n \widetilde{d}_n^{\ddagger}\widetilde{d}_n +\sum_{n=1}^{N} \left(\mathbbm{g}_n \sigma_+ \widetilde{d}_n + \text{h. c. }\right),
\ee
where $\mathbbm{g}_n= \sqrt{\frac{\mathbbm{i} z_n}{w(z_n)}} \sqrt{w_n}g(z_n)$. Since $H_{\text{dis}}$ is nonhermitian, the evolution operator is defined as
\be\label{U}
U(t)= \sum_n e^{-\mathbbm{i E}_n t} \ket{n}_R {_L \bra{n}},
\ee
where $\ket{n}_R$ denotes the right eigenfunction of  $H_{\text{eff}}$ with eigenvalue $\mathbbm{E}_n$, and $\ket{n}_L$ denotes the left eigenfunction with eigenvalue $\mathbbm{E}^*_n$ \cite{brody14}.

The validity of CDA is demonstrated  in  Fig. \ref{fig:exact} and \ref{fig:tildeH}, where the exact result is obtained by solving Eq. \eqref{alphat}. Nonetheless, due to the time-consuming nature of the iterative method for exact approach,  the plots are restricted  to $t=10$ with a step length of  $10^{-4}$. The weight function chosen is $w(z)=1$, which might cause the appearance of complex eigenmodes in the environment. It should be noted that choosing $g(z)$ as the weight function will not generate a non-Hermitian effective Hamiltonian.



\section{Supplementary discussion about the quantum phase transition in the single-excitation subspace} \label{appendixforIII}

% Figure environment removed

% Figure environment removed

In Fig. \ref{fig:RandN}, the relevance of $\eta_I$ and $\eta_{II}$ to the computational parameters $R$ and $N$ is presented. For $N$, it is observed that the values of both $\eta_I$ and $\eta_{II}$ reach a steady state with the rise of $N$.  However,  for $s=1$,  $\eta_I$ and $\eta_{II}$ are observed  to merge into a single value close to $0.1$. On the other hand, with the increase of $R$, $\eta_I$ and $\eta_{II}$ increase very slowly for $s = 0$ and $s = 0.2$. For $s = 1$, $\eta_{II}$ decreases gradually, while $\eta_I$ stays constant. The odd features can be attributed to the computational property of $R$ as stated in Ref. \cite{cui23}. In this work, it is indicated that although the increase of $R$ can enhance the accuracy of evaluation, the computational efficiency may be substantially reduced. Consequently, a suitable balance of $R$ and $N$ needs to be chosen for optimal performance.  In this paper, $R=6$ and $N=1000$ are chosen for $s=0$ and $0.2$, but $N=2000$ for $s=1$. The explicit calculation shows that this choice  offers a good balance between the precision and efficiency.

Fig. \ref{fig:eta} shows the subtle changes in spin dynamics near the critical point $\eta_{II}$, which is represented by the dot-dashed line. Evidently, the spin dynamics can display different features across $\eta_{II}$. For $s=0$ or $0.2$, the curves in figure display  a significant increase  when $\eta> \eta_{II}$ for $t>\sim 3$. A similar observation can be found for $s=1$,  although the changes is not easy to notice  because $\eta_{I}$ and  $\eta_{II}$ tend to be the same in this case. This demonstrates the presence of an intermediate phase through spin dynamics.  However, it is difficult to determine $\eta_{II}$ solely based on spin dynamics, as the variation is very subtle. Fortunately, the construction of $H_{\text{dis}}$ provides an analytical way to identify the phase transition, as discussed in the main text.


\section{Approaching spin dynamics in double-excitation subspace by Gauss quadrature in real space}\label{appendixrealGQ}

% Figure environment removed

In Fig. \ref{fig:realdouble}, $P_e(t)$ is reevaluated using  Gauss quadrature method \cite{discretization}. The weight function $W(\omega)= \left(\omega/\omega_c\right)^{s}e^{-\omega/\omega_c}$ is chosen to construct the chain Hamiltonian and the number of lattice sites is set to $N=200$ in comparison with the evaluation in Fig. \ref{fig:double}. It is evident that $P_e(t)$ has a similar evolution as the survival probability $\left|\inp{\psi(t)}{\psi (0)}\right|^2$ in single-excitation subspace. However, in contrast to CDA approach,  strong fluctuations due to finite $N$ appears after $t \sim 2$, which limits the application of Gauss quadrature in the simulation of long-term dynamics in open quantum systems.


\begin{thebibliography}{99}

\bibitem{kessler12} E. M. Kessler, G. Giedke, A. Imamoglu, S. F. Yelin, M. D. Lukin, and J. I. Cirac, {\it Dissipative phase transtion in a cental spin system}, Phys. Rev. A {\bf 86}, 012116 (2012).

\bibitem{solid-state}I. M. Georgescu, S. Ashhab, Franco Nori, {\it Qauntum Simulation}, Rev. Mod. Phys. {\bf 86}, 154 (2014); M. Gong, {\it et.al.}, {\it Quantum walks on a programmable two-dimensional 62-qubit superconducting processor}, Science, {\bf 372}, 948-952 (2021);  Q. Zhu, {\it et.al.,} {\it Quantum computational advantage via 60-qubit 24-cycle random circuit sampling}, Science Bulletin, {\bf 67}(3),  240-245 (2022).

\bibitem{light-matter}A. F. Kockum,  A. Miranowicz, S. De Liberato, S. Savasta and F. Nori,{\it Ultrastrong coupling between light and matter}, Nat. Rev. Phys. {\bf 1}, 19-40 (2019); P. Forn-D\`{i}az, L. Lamata, E. Rico, J. Kono, and E. Solano, {it Ultrastrong coupling regimes of light-matter interaction}, Rev. Mod. Phys. {\bf 91}, 025005.

\bibitem{vega2017}I. de Vega,  D. Alonso, {\it Dynamics of non-Markovian open quantum systems}, Rev. Mod. Phys. {\bf 89}, 015001 (2017); H. Weimer, A. Kshertrimayum, R. Or\'us, {\it Simulation methods for open quantum many-body systems}, Rev. Mod. Phys. {\bf 93}, 015008 (2021).

\bibitem{breuer2002} H. P. Breuer, F. Petruccione, {\it The Theory of Open Quantum Systems}, Oxford University Press (2002).


\bibitem{kazansky97}R. S. Burkey, C. D. Cantrell, {\it Discretization in the quasi-continuum}, J. Opt. Soc. Am. B {\bf 1}, 169-175 (1984); A. K. Kazansky, {\it Precise anlaysis of resonance decay law in atomic physics}, J. Phys. B: At. Mol. Opt. Phys. {\bf 30}, 1404-1410 (1997); N. Shenvi, J. R. Schmidt, S. T. Edwards, and J. C. Tully, {\it Efficient decretization of the continuum through complex contour deformation}, Phys. Rev. A {\bf 78}, 022502 (2008).

\bibitem{discretization} A. W. Chin, \'{A}. Rivas, S. F. Huelga, and M. B. Plenio, {\it Exact mapping between system-reservior quantum models and semi-infinite discrete chains using orthogonal polynomials}, J. Math. Phys. {\bf 51}, 092109 (2010).

\bibitem{cui23} H. T. Cui, Y. A. Yan, M. Qin, and X. X. Yi, {\it Complex discretization approximation for the full dynamics of system-environment quantum models}, arXiv: 2303.06584 [quant-ph] (2023).

\bibitem{leggett}  A. J. Leggett, S. Chakravarty, A. T. Dorsey, Matthew P. A. Fisher, A. Garg, and W. Zwerger, {\it Dynamics of the dissipative two-state system}, Rev. Mod. Phys. {\bf 59}, 1-85 (1987).

\bibitem{weiss}U. Weiss, {\it Quantum Dissipative Systems} (World Scientific, Singapore, 1999).

\bibitem{wang19} Yan-Zhi Wang, Shu He, Liwei Duan, and Qing-Hu Chen, {\it Quantum phase transition in the spin-boson model without counterrotating terms}, Phys. Rev. B {\bf 100}, 115106 (2016).

\bibitem{sachdev} S. Sachdev, {\it Quantum Phase transition} (Cambidge University Press, Cambridge, 2011).

\bibitem{shi16} Tao Shi, Ying-Hai Wu, A. Gonz\'alez-Tudela, and J. I. Cirac, {\it Bound states in Boson Imputiry Models}, Phys. Rev. X {\bf 6}, 021027 (2016).

\bibitem{tamascelli19} D. Tamascelli, A. Smirne, J. Lim, S. F. Huelga, and M. B. Plenio, {\it Efficient simulation of finite-temperature open quantum systems}, Phys. Rev. Lett. {\bf 123}, 090402 (2019).

\bibitem{brody14}Dorje C Brody, {\it Biorthogonal quantum mechanics}. J. Phys. A: Math. Thoer. {\bf 47}, 035305 (2014).
\end{thebibliography}

\end{document} 