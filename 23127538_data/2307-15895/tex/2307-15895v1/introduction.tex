\section{Introduction}

Auditing frameworks, such as Linux Audit~\cite{audit} or Sysdig~\cite{sysdig}, play a vital role in provenance analysis systems for enterprise security. Many companies build Security Operations Centers (SOC)~\cite{soc, asi,carbonblack,decianno2014indicators,Falco,anafcheh2018intrusion,gomez2019improvements} based on auditing frameworks. Moreover, a large body of research work uses auditing frameworks to develop intrusion detection systems~\cite{gui2019progressive,king_backtracking_2003-1,hossain_sleuth_2017,camflow,LPM,hifi,dap,9152771,berlin2015malicious,du2017deeplog,wang2020you,Falco,han2020sigl,hassan2019nodoze}. We expect this topic to remain relevant for both industry and academia due to the amount of related work.

Unsurprisingly, attackers are engaged in compromising auditing frameworks. Recent studies have also shown that attackers compromise the kernel module of auditing frameworks to prevent attack traces from being recorded. Paccagnella \textit{et al.}~\cite{paccagnella2020logging} proposed a race condition attack in which an attacker with root privileges can compromise the kernel module of auditing frameworks to hide their malicious actions. To protect the kernel module of auditing frameworks, researchers have proposed multiple approaches, such as Hardlog~\cite{ahmad2022hardlog}, KennyLoggings~\cite{paccagnella2020logging}, and QuickLog~\cite{281386}. There are also many user-space attacks that mainly involve log tampering ~\cite{bowers2014pillarbox,logtempering2,logtempering3,logtempering4}. Meanwhile, several cryptographic-based approaches have been proposed to secure the user space transmission and storage of logs~\cite{Forwardsecure,syslog-ng,10.1145/3052973.3053034,Custos}.

In addition to the vulnerabilities mentioned above, this paper concentrates on the \textit{super producer threat}. This threat does not require attackers to have root privileges to launch attacks on the kernel module. In essence, the attacker can either disable provenance data collectors or intensify DoS attacks on the system under observation by creating a super producer, a process that produces a large number of system provenance events in a brief span of time, with user privilege.

Specifically, by generating a large amount of provenance data that auditing frameworks cannot process in time with a reasonable amount of system resources, a super producer puts the current auditing framework into the \textit{data integrity vs. efficiency} dilemma. On one hand,  auditing frameworks may adopt the \textit{pro-performance} strategy~\cite{sysdig,audit,lttng}, which drops system events if there is too much provenance data. However, under this strategy, attackers may launch the \textit{\ac{pdos} attack}, in which the attacker uses the super producer to evict the events of malicious behaviors from the event buffers of auditing frameworks.  On the other hand,  auditing frameworks may adopt the \textit{pro-integrity} strategy~\cite{Camquery}, which elastically allocates sufficient resources to ensure that all provenance data can be processed in time. Unfortunately, this strategy may lead to the \textit{\ac{pados} attack}, in which attackers exploit the pro-integrity strategy to degrade the performance of the whole system, amplifying the DoS attack on the server. Notably, the \ac{pados} attack can even break the protection of \code{cgroup}.

The main reason for the super producer threat is that the design of existing auditing frameworks breaks the resource and logic isolation of processes, which is critical for modern OSes to achieve high performance for concurrent tasks. Existing solutions collect system events by intercepting system calls in the OS kernel and then processing the collected events in a centralized user-space collector~\cite{sysdig,lttng,Camquery,audit}. The centralized collector handles provenance data equally regardless of the priority, importance, and resource quota of the processes that generate the provenance data. Therefore, a super producer can generate a massive amount of provenance data that occupies all the processing power of the centralized handler and causes the  ``data integrity vs. efficiency dilemma''.

In this paper, we present a comprehensive analysis of the super producer threat and propose a novel auditing framework, \toolname, that balances the trade-off between ``data integrity and efficiency''. Specifically, \toolname surpasses existing solutions in two aspects. First, it guarantees the integrity of provenance data. \toolname records all provenance data faithfully regardless of the workload. Therefore, a super producer cannot conceal the traces of attacks by generating too many system events, thus mitigating the \ac{pdos} attack. Second, \toolname prevents the super producer from degrading the performance of the whole system, avoiding the \ac{pados} attack.

The key insight of our design is to provide isolation to the provenance data collector so that each process consumes its own resource quota to handle the provenance data generated by itself. The logic behind this design is as follows. Like user-space log events (e.g., log4j events), provenance events reflect the status of the running processes. Thus, provenance data should be considered as the logs of the corresponding processes, instead of the OS. Therefore, each process should spend its own resource quota to handle the provenance data it generates.

By isolating the provenance data of each process, we can naturally mitigate the super producer threat. This way, a process that generates a huge amount of provenance data in a short time can only affect its own performance, since it has a limited resource quota. It also cannot interfere with the processing power of other processes. Therefore, the super producer cannot stop the system from recording provenance events of attacks by overwhelming the auditing frameworks.

The main challenge of isolating provenance data from different processes is to achieve efficiency. We use a \textit{threadlet-based} approach that inserts the provenance data processing logic into the memory of running applications. This approach leverages the process isolation strategy of the OS directly, eliminating the extra overhead of adding a new isolation strategy to the auditing framework. This approach also reduces process scheduling and its associated cache miss costs.

We thoroughly evaluate \toolname with eight different hardware configurations and five baselines.
Our evaluation shows that \toolname faithfully records all provenance data, preventing the \ac{pdos} attack. On the contrary, current pro-performance auditing frameworks (\ie Sysdig) can drop up to 90\% of provenance data while the super producer is running. \toolname can also prevent the \ac{pados} attack. Specifically, \toolname only slows down three popular applications by 4.0\% on average, regardless of the workload generated by the super producer. For comparison, existing pro-integrity auditing frameworks can slow down the applications by up to 59.1\% on average. More importantly, when the super producer increases its workload, the application performance decreases accordingly with existing auditing frameworks. \toolname is also efficient compared with the SOTA pro-performance collector, Sysdig. \toolname has, on average, 6.30\% less application overhead than Sysdig. In summary, our evaluation proves that \toolname can address the super producer threat while incurring lower system overhead than existing auditing frameworks.   

To sum up, this paper makes the following contributions:
\begin{itemize}
    \item To the best of our knowledge, this is the first thorough systematic study on the super producer threat and related attacks to auditing frameworks.
    \item We identify that the root cause of the super producer threat is the lack of resource isolation in the user space component of existing auditing frameworks.
    \item To address the super producer threat, we propose a novel auditing framework \toolname that efficiently isolates resources for provenance data handling by enforcing processes to consume their own resource quota to handle the provenance data generated by themselves.
    \item Extensive experiments demonstrate that \toolname can address the super producer threat, as well as its efficiency.  
\end{itemize}
\textbf{Availability:}  \toolname is available at: \url{https://github.com/PKU-ASAL/NoDrop}