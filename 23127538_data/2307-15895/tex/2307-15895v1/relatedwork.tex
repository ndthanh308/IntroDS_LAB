\section{Related Work}

Provenance analysis has been widely applied in different security tasks, such as APT attack investigation~\cite{LPM,Spade,king_backtracking_2003-1,MaZX16,camflow,hifi,gao2018saql,gao2018aiql,ji2017rain,ji2018enabling,liu2018towards,pasquier2018ccs} and detecting stealthy security risks~\cite{berlin2015malicious,du2017deeplog,han2020ndss,gu2015leaps,hassan2019nodoze,hossain_sleuth_2017,liu2019log2vec,milajerdi2019holmes,milajerdi2019poirot,oprea2015detection,pei2016hercule,shen2018tiresias,shen2019attack2vec,wang2020you,9152771,han2020sigl}.  There are also methods for precisely and clearly interpreting events to explain applications' behaviors~\cite{bates2017transparent,hassan2020omegalog,kwon2018mci,beep,ma2015accurate,ma2017mpi,Pass,yang2020uiscope}. \toolname benefits these tasks by providing a more reliable data source. Attackers regularly engage in anti-forensic activities to cover their tracks~\cite{asi}. Several cryptographic-based approaches are proposed to secure logs~\cite{paccagnella2020logging,Forwardsecure,syslog-ng,10.1145/3052973.3053034,Custos}, but none discuss the security of the user-space component of auditing frameworks.

Threadlet is a short sequence of instructions with self-contained memory~\cite{kogge2004piglets,4228404,9188165,9622823,9652820}. \toolname borrows this concept but implements it differently as a piece of code instrumented to a host thread. \toolname provides protections such as MPK, address randomization, and heap isolation.