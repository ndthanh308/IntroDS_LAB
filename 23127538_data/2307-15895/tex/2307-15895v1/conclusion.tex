\section{Discussion}

\toolname may allow a malicious process to compromise the consumer residents in its memory. To this end, we adopt a comprehensive solution that combines address randomization, dedicated heap, MPK protection, and ensuring the atomic execution of the consumer, as discussed in \S\ref{sec:consumer}.  Thus, although the consumer shares the same memory space as user applications, they are still protected. We notice that MPK is available in most of the latest Intel server and client-side CPUs. ARM, AMD, RISC-V, PowerPC, and Itanium CPUs ~\cite{mpk1,mpk2,mpk3,mpk4,vahldiek2019erim} also have similar mechanisms.

Although \toolname prevents attackers from slowing down other applications,  the attacker can still slow down a process by injecting the super producer logic into the process directly. We consider the thread model of this attack too strong for \toolname. Indeed, as long as the attacker can compromise a process, it is straightforward to slow down the process. How to protect a running process from hijacking is beyond the scope of this paper.  

Windows provides the ETW~\cite{etw} framework for provenance collection, but it only has a kernel module and leaves the user-space logic for customization. Thus, we cannot find an ``official'' user-space component for ETW. Nevertheless, the super producer vulnerability is about process scheduling and isolation, which is general to both Linux and Windows. \vspace{-2ex}

\section{Conclusion}

This paper is the first to identify the super producer threat to existing auditing frameworks. Through thorough experiments and case studies, we find attackers can either disable existing auditing frameworks or paralyze the whole system with a super producer. Based on our discovery, we propose a novel auditing framework, \toolname, that addresses the super producer threat by providing resource isolation. Our evaluation shows that \toolname prevents the super producer threat while introducing 6.30\% lower application overhead on average across eight different hardware configurations than Sysdig. 

\section{Acknowledgement}
We sincerely thank our Shepherd and all the anonymous reviewers for their valuable comments. This work was partly supported by the National Key Research and Development Program (No. 2022YFB4501802), the National Natural Science Foundation of China (No. 62172009, No. 62141208) and Huawei Research Fund. 