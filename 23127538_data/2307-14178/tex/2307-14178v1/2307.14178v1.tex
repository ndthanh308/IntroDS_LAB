\documentclass[11pt]{article}
\usepackage{pxfonts}
\usepackage{yfonts}
\usepackage{dsfont}
\usepackage{graphicx}
\usepackage{relsize}

\parindent 0pt
\parskip 7pt

\addtolength{\textwidth}{3cm}
\addtolength{\oddsidemargin}{-1.5cm}
\addtolength{\textheight}{4cm}
\addtolength{\topmargin}{-2cm}

\hfuzz=7 pt
\font\bigbf=cmbx10 at 16pt
\font\medbf=cmbx10 at 13pt


\def\bel{\begin{equation}\label}
\def\eeq{\end{equation}}
\def\ds{\displaystyle}
\def\endproof{\hphantom{MM}
\hfill\llap{$\square$}\goodbreak}
\def\argmax{\hbox{arg}\!\max}
\def\argmin{\hbox{arg}\!\min}
\def\mt{\longrightarrow}
\def\v{\vskip 1em}
\def\vsk{\vskip 40em}
\def\ve{\varepsilon}
\def\R{\mathbb R}
\def\Z{\mathbb Z}
\def\C{\mathfrak{C}}
\def\Cx{\mathds C}
\def\N{{\bf N}}
\def\exp{{\bf exp}}
\def\Re{{\bf Re}}
\def\Im {{\bf Im}}
\def\S{{\bf S}}
\def\Sz{\mathfrak{S}}
\def\F{\mathfrak{F}}
\def\G{\mathfrak{G}}
\def\HH{\mathcal{H}}
\def\E{\mathcal{E}}
\def\O{{\bf O}}
\def\Q{{\bf Q}}
\def\D{{\bf D}}
\def\J{{\bf J}}
\def\K{{\bf K}}
\def\A{{\bf A}}
\def\B{{\bf B}}
\def\H{{\bf H}}
\def\L{{\bf L}}
\def\U{\mathcal{U}}
\def\V{\mathcal{V}}
\def\BMO{{\bf BMO}}
\def\RD{{\bf RD}}
\def\p{{\partial}}
\def\a{{\bf a}}
\def\b{{\bf b}}
\def\i{{\bf i}}
\def\Tilde{\widetilde}
\def\Hat{\widehat}
\def\const{{\bf const}}
\def\bar{\overline}
\def\supp{{\bf supp}}
\def\sign{{\bf sign}}
\def\dist{{\bf dist}}
\def\diam{\hbox{diam}}
\def\I{{\bf I}}
\def\II{{\bf II}}
\def\III{{\bf III}}
\def\IV{{\bf IV}}
\def\M{{\bf M}}
\def\G{{\bf G}}
\def\q{\mathfrak{q}}
\def\Vol{{\bf Vol}}
\def\Cup{{\bigcup}}
\def\Cap{{\bigcap}}
\def\la{{\langle}}
\def\ra{{\rangle}}
\def\alpha{\alphaup}
\def\beta{\betaup}
\def\gamma{\gammaup}
\def\delta{\deltaup}
\def\xi{{\xiup}}
\def\eta{{\etaup}}
\def\tau{{\tauup}}
\def\rho{{\rhoup}}
\def\phi{{\phiup}}
\def\psi{{\psiup}}
\def\lambda{{\lambdaup}}
\def\omega{\omegaup}
\def\varphi{{\varphiup}}
\def\gamma{{\gammaup}}
\def\c{{\bf c}}
\def\t{{\bf t}}
\def\s{{\bf s}}
\def\r{{\bf r}}
\def\h{{\bf h}}
\def\m{{\bf m}}
\def\n{{\bf n}}
\def\j{\jmath}

\def\T{\mathbf{T}}
\def\TT{\mathfrak{T}}


\renewcommand{\theequation}{\thesection.
\arabic{equation}}
\renewcommand{\thesection}{\arabic{section}}
\newtheorem{thm}{Theorem}[section]

\newtheorem{cor}{Corollary}[section]
\newtheorem{lemma}{Lemma}[section]
\newtheorem{prop}{Proposition}[section]
\newtheorem{remark}{Remark}[section]
\newtheorem{definition}{Definition}[section]


\begin{document}
 \[\begin{array}{cc}\hbox{\LARGE{\bf On the multi-parameter Fourier integral operator}}
 \end{array}\]

 \[\hbox{Jinhua Cheng~~~~and ~~~~Zipeng Wang}\]

 \begin{abstract}
We study a family of Fourier integral operators whose symbol and phase function satisfy  a multi-parameter  characteristics. We consider symbol function $\sigma(x,y,\xi)$ with compact support in $~(x,y)$ and such that  $~|\p_\xi^\alpha\p_{x,y}^\beta\sigma(x,y,\xi)|\le \C (1+|\xi|)^m\prod_{i=1}^n(1+|\xi^i|)^{-\alpha_i}$.
We require that phase function $~\Phi(x,\xi)=\Phi_1(x^1,\xi^1)+\Phi_2(x^2,\xi^2)+\cdots+\Phi_n(x^n,\xi^n)~$, and each of them satisfies the same conditions as before. As a result,  we get a  desired  better $\L^p$-regularity result.
\end{abstract}
\section{Introduction}
\setcounter{equation}{0}
 Let $f$ be a Schwartz function.
A Fourier integral operator $\F$  defined by
\bel{Ff}
\Big(\F f\Big)(x)~=~\int_{\R^\N} f(y)\Omega(x,y)dy
\eeq
whose kernel is given by an oscillatory integral
\bel{Kernel}
\Omega(x,y)~=~\int_{\R^\N}e^{2\pi\i \left(\Phi(x,\xi)-y\cdot\xi\right)}\sigma(x,y,\xi)d\xi.
\eeq
Symbol function  $\sigma(x,y,\xi)\in\mathcal{C}^\infty(\R^\N\times\R^\N\times\R^\N)$ has a compact support in  $x$ and $y$. 
Phase function $\Phi(x,\xi)$ is real, homogeneous of degree $1$ in $\xi$ and smooth for every $x$ and $\xi\neq0$. Moreover, it satisfies   
 the nondegeneracy condition
\bel{nondegeneracy}
 \det\left({\p^2\Phi\over \p x_i\p \xi_j}\right)\left(x,\xi\right)~\neq~0,\qquad \xi~\neq~0
 \eeq
on the support of $\sigma(x,y,\xi)$.  



$\diamond$  {\small Throughout, we regard $\C$ as a generic constant depending on its subindices.}

We say $\sigma\in \S^m$ if 
\bel{class}
\left|\p_\xi^\alpha\p_{x,y}^\beta \sigma(x,y,\xi)\right|~\leq~\C_{\alpha~\beta}\left(1+|\xi|\right)^m \left({1\over 1+|\xi|}\right)^{|\alpha|}
\eeq
for every multi-indices $\alpha,\beta$.
 
For $\sigma\in \S^0$, 
$\F$ defined in (\ref{Ff})-(\ref{nondegeneracy}) is  bounded on $\L^2(\R^\N)$. See the papers  by  Eskin \cite{Eskin} and H\"{o}rmander \cite{Hormander}. In contrast to this $\L^2$-result,  it is well known that $\F$ of order zero is not bounded on $\L^p(\R^\N)$ if $p\neq2$. 
The optimal $\L^p$-estimate was first investigated by   Duistermaat and H\"{o}rmander  \cite{Duistermaat-Hormander} and  then by  Colin de Verdi\'{e}re and Frisch \cite{Colin-Frisch},  Brenner \cite{Brenner}, Peral \cite{Peral}, Miyachi \cite{Miyachi}, Beals \cite{Beals} and eventually obtained by Seeger, Sogge and Stein \cite{S.S.S}.


 
 


 \v
 
 

{\bf Theorem A: ~Seeger, Sogge and Stein  (1991)}\\
 {\it Let $\F$ defined as (\ref{Ff})-(\ref{class}). Suppose $\sigma\in \S^m$ for $-(\N-1)/2<m\leq0$. We have}
\bel{theoremone}
\begin{array}{cc}\ds
\left\| \F f\right\|_{\L^p(\R^\N)}~\leq~\C_{p~\sigma~\Phi}~\left\| f\right\|_{\L^p(\R^\N)},\qquad 1<p<\infty,
\\\\ \ds
 whenever\qquad \left| {1\over 2}-{1\over p}\right|~\leq~{-m\over \N-1}.
\end{array}
\eeq










{\bf Remark 1.1 Theorem A }~{\it
 is sharp. $\F$ defined in  (\ref{Ff})-(\ref{class}) is not bounded on $\L^p(\R^\N)$ if $\left|1/2-1/p\right|>-m/(\N-1),~(1-\N)/2\leq m\leq0$.  Regarding estimates can be found at {\bf 6.13}, chapter IX in the book of Stein \cite{Stein}.}

Wang \cite{Wang 2} ~extended {\bf Theorem A} to product spaces, that is, they proved the same results by considering symbol class $\S^m$ such that   $~\sigma \in \S^m$ if 
\bel{Class}
\left|\p_\xi^\alpha\p_{x,y}^\beta \sigma(x,y,\xi)\right|~\leq~\C_{\alpha~\beta}~\left(1+|\xi|\right)^m\prod_{i=1}^n \left({1\over 1+|\xi^i|}\right)^{\alpha_i}
\eeq
for every multi-indices $\alpha,\beta~$,  where $~\xi^i\in \R^{\N_i}~,i=1,2,\dots,n$ and $~\N=\N_1+\N_2+\cdots+\N_n$.






In this paper, we extend {\bf Theorem A} to product spaces by considering $\F$ defined within a multi-parameter setting.  

We further require
\bel{phasesum}
\Phi(x,\xi)=\Phi_1(x^1,\xi^1)+\Phi_2(x^2,\xi^2)+\cdots+\Phi_n(x^n,\xi^n)
\eeq
 where $~(x^i,\xi^i)\in \R^{\mathbf{N}_i}\times\R^{\mathbf{N}_i}~$ for $~i=1,2,\dots,n.~$
Each $~\Phi_i(x^i,\xi^i),i=1,2,\dots,n~$ is real, and homogeneous of degree $1$ in $~\xi^i~$ and smooth for every $x^i$. Moreover, it satisfies the non-degeneracy condition
\bel{nondeneracies}
    \det\left[\frac{\partial^2\Phi _i}{\partial x^i\partial \xi^i}\right](x^i,\xi^i)\neq 0
\eeq


at $~\xi^i\neq 0~$ on the support of $~\sigma(x,y,\xi)~$.





The study of such operators  that  commute with a multi-parameter family of dilations  dates back to the time of  Jessen, Marcinkiewicz and Zygmund.  Over the several past  decades,
 a number of pioneering  results  have been accomplished, for example   
by Robert Fefferman \cite{R.Fefferman}, Fefferman and Stein \cite{R-F.S}, Chang and  Fefferman \cite{Chang-Fefferman}, Cordoba and Fefferman \cite{Cordoba-Fefferman} 
and M\"{u}ller, Ricci and Stein \cite{M.R.S}. 
\v

Our main result is stated below.

{\bf Theorem One}~~
 {\it Let $\F$ defined in (\ref{Ff})-(\ref{Kernel}) and  (\ref{Class})-(\ref{nondeneracies}). Suppose $\sigma\in \S^m$ for $-(\N-n)/2<m\leq0$. We have}
 
\bel{theoremtwo}
\begin{array}{cc}\ds
\left\| \F f\right\|_{\L^p(\R^\N)}~\leq~\C_{p~\sigma~\Phi}~\left\| f\right\|_{\L^p(\R^\N)},\qquad 1<p<\infty,
\\\\ \ds
 whenever\qquad \left| {1\over 2}-{1\over p}\right|~\leq~{-m\over \N-n}.
\end{array}
\eeq























We aim to show 
\bel{aim}
     ||\F f ||_{\mathbf{L}^1(\R^{\mathbf{N}})}\le \C||f||_{\mathbf{H}^1(\R^{\mathbf{N}})},  \quad \sigma\in \S^{-(\mathbf{N}-n)/2}.
\eeq

Together with the $\mathbf{L}^2-$ boundedness of $\F$ for $\sigma\in \S^0$ and duality argument, we can prove the desired result by carrying out on interpolation argument set out at $4.9$, chapter IX of Stein \cite{Stein}.

Let $a$ be an $\mathbf{H}^1-atom$ associated to the ball $B_r(x_o)$ centered on some $x_o\in\R^\mathbf{N}$ with radius $r>0$. 







In order to prove $(\ref{aim})$, it suffices to have
\bel{aim2}
     \int_{\R^\mathbf{N}}|\F a(x)|dx\le \C, \qquad \sigma\in \S^{-(\mathbf{N}-n)/2}.
\eeq
due to the characterization of $\H^1-$Hardy space established by Fefferman and Stein \cite{FC.S}.

 In fact, we can assume that $0<r<1$, since if $r>1$, the estimate is easy, because of our assumption that the symbol of $\F$ has compact $x-$support. Indeed,
 \[
     \int_{\R^\mathbf{N}}|\F(a)|dx\le C||\F a||_{\mathbf{L}^2}\le \C||a||_{\mathbf{L}^2}.
\]

The first inequality holds because $\F(a)$ has fixed compact support; the second follows from the $\mathbf{L}^2$ boundedness property already proved. Now since $a$ is an atom, $|a(x)|\le |B|^{-1}$, We have $||a||_{\mathbf{L}^2}\le |B|^{-\frac{1}{2}}\le \C$. 
And (\ref{aim}) holds in the case.


In section $2$ we decompose operator $\F$ into infinitely many $\F_{\ell j}$ via "cone decomposition" on frequency space. In section $3$ we make a further dyadic decomposition on frequency space and define the so-called \emph{region of influence}. Then in  sections $4$ and $5$ we prove that when $\N_i\ge 2,~i=1,2,\cdots ,n$, then 
inside the region of influence~ (\ref{aim2})~ is true by applying {\bf lemma one} and Schwarz's inequality. In section $6$ we prove  when $\N_i\ge 2,~i=1,2,\cdots ,n$, then 
on the complement of the region of influence~ (\ref{aim2})~ is true by applying {\bf lemma two}, which will be proved in section $7$.
In the last section, we prove the case when there exists $1\le i\le n$ such that $\N_i=1$, ~(\ref{aim2}) is true by adjusting the definition of the region of influence and introducing another parameter.




















\section{Cone decomposition}
\setcounter{equation}{0}
Cone decomposition on frequency  arises naturally if we want to make full use of the condition of the differential inequalities   ~(\ref{Class})~.
Let $~\xi=(\tau,\lambda^1,\dots,\lambda^{n-1})\in \R^{\mathbf{N}_n}\times\R^{\mathbf{N}_1}\times\cdots\times\R^{\mathbf{N}_{n-1}}~$.  By symmetry, it suffices to assume $~|\tau|=\max_{i=1,2,\dots,n}|\xi^i|~$. Throughout, $~\ell~$ denotes an $~(n-1)-$tuple $~(\ell_1,\dots, \ell_{n-1})~$ where $~\ell_i, i=1,\dots,n-1~$ are nonnegative integers. Also, we let $~\ell_n=0~$ for convenience. Let $\I$ and $\J$ be  subsets of $\{1,2,\dots,n-1\}$ such that $\I\cup\J=\{1,2,\dots,n-1\}$, and $~\K=\I\cup\{n\}~$. Their cardinalities are denoted by $~|\I|~$ ,$~|\J|~$  and $~|\K|~$, respectively. 








Let $~\varphi~$ be  a smooth bump-function on $\R$ such that $~\varphi(t)=1~$ for $~|t|\le 1~$ and $~\varphi(t)=0~$ for $~|t|>2~$. Define
\bel{delta_t}
\begin{array}{cc}\ds
 \delta_{\ell }(\xi)~=~\prod_{i=1}^{n-1}\delta_{\ell_i}(\xi),
 \\\\ \ds
 \delta_{\ell_i}(\xi)~=~\varphi\left(2^{\ell_i}{|\lambda^i|\over |\tau|}\right)-\varphi\left(2^{\ell_i+1}{|\lambda^i|\over|\tau|}\right),\qquad i~=~1,2,\ldots,n-1.
 \end{array}
\eeq

Observe that $~\delta_{\ell j}(\xi)~$ is supported in



\bel{Cone}
\Lambda_{\ell }~=~\Bigg\{ ~ (\tau,\lambda)\in\R\times\R^{n-1}~\colon~      2^{-\ell_i-1}~<~ {|\lambda_i|\over|\tau|}~<~2^{-\ell_i+1},~i=1,2,\dots,n-1\Bigg\}.
\eeq










 Let $j>0$ be an integer, define 
 
\bel{}
\begin{array}{cc}
\ell_\I=\{(\ell_1,\ell_2,\dots,\ell_{n-1})\in \Z^{n-1}:\quad
  0~\leq~\ell_i~<~ j,\quad i\in\I,\quad\ell_i~\ge~j,\quad i\in\J\end{array}\}.
\eeq
and 
\bel{delta l_I}
\delta_{\ell_\I}=\sum_{\ell_\I}\prod_{i=1}^{n-1}\delta_{\ell_i}.
\eeq


Therefore, we have 
\bel{}
\sum_{\ell}\delta_\ell=\sum_{\I}\delta_{\ell_\I}
\eeq

where the second sum take over all the subsets of $\{1,2,\dots,n-1\}$.

Let $\varphi$ be the smooth {\it bump}-function  as before.  Consider
\bel{phi}
\phi_j(\xi)=\varphi(2^{-j}|\xi|)-\varphi(2^{-j+1}|\xi|).
\eeq

Recall
$\delta_\ell(\xi)$ defined in (\ref{delta_t})-(\ref{Cone}).
Note that $\sum_\ell \delta_\ell(\xi)=\sum_{j}\phi_j(\xi)\equiv1$.
Define
\bel{delta_lj}
\begin{array}{lr}\ds
\delta_{\ell j}(\xi)~=~\prod_{i\in\I}\delta_{\ell_i}(\xi)  \prod_{i\in\J}~ \sum_{\ell_i} \delta_{\ell_i}(\xi)
\\\\ \ds~~~~~~~~~
~=~\prod_{i\in\I}\delta_{\ell_i}(\xi)\prod_{i\in\J}~\left\{ \sum_{\ell_i} \varphi\left(2^{\ell_i} {|\lambda^i|\over|\tau|}\right)-\varphi\left(2^{\ell_i+1}{|\lambda^i|\over |\tau|}\right)\right\}
\\\\ \ds~~~~~~~~~
~=~\prod_{i\in\I}\delta_{\ell_i}(\xi)\prod_{i\in\J}~\varphi\left(2^j{|\lambda^i|\over|\tau|}\right)
\end{array}
\eeq


 Observe that $\delta_{\ell j}(\xi)$ is supported in
\bel{Lambda_lj}
\begin{array}{lr}\ds
\Lambda_{\ell j}~=~\Bigg\{ ~ (\tau,\lambda)\in\R^{\N_n}\times\R^{\N-\N_n}~\colon~      2^{-\ell_i-1}~<~ {|\lambda^i|\over|\tau|}~<~2^{-\ell_i+1},~i\in\I 
\\\\ \ds~~~~~~~~~~~~~~~~~~~~~~~~~~~~~~~~~~~~~~~~~
\hbox{and}\qquad 2^{-j-1}~<~ {|\lambda^i|\over|\tau|}~<~2^{-j+1},~i\in\J~\Bigg\}.
\end{array}
\eeq


From $(\ref{delta_lj})-(\ref{Lambda_lj})$ and direct computation, we find
\bel{delta Diff Ineq}
\left| \p_\tau^\alpha \p_\lambda^\beta  \delta_{\ell j}(\tau,\lambda)\right|~\leq~\C_{\alpha~\beta}~ \left({1\over |\tau|}\right)^{\alpha}\prod_{i\in\I} \left({1\over |\lambda^i|}\right)^{\beta_i}
\eeq
 for every multi-indices $\alpha,\beta$.










 
Moreover, by definition of $\delta_\ell(\xi)$ in (\ref{delta_t}) and $\delta_{\ell j}(\xi)$ in (\ref{delta_lj}), we have
\bel{delta Sum}
\sum_{\I}\prod_{i\in\I}\sum_{\ell_i\le j} \delta_{\ell j}(\xi)~\equiv~\sum_{\I}\delta_{\ell_\I}(\xi)=\sum_{\ell}\delta_\ell(\xi).
\eeq


Now we can make a further decomposition

\bel{delta Sum2}
\sum_{j\in\Z}\sum_{\I}\prod_{i\in\I}\sum_{\ell_i\le j} \phi_{j}(\xi) \delta_{\ell j}(\xi)~\equiv~\sum_{\ell}\delta_\ell(\xi).
\eeq

Consider
\bel{F_lj}
\F_{\ell j} f(x)=\int_{\R^\N}f(y)\Omega_{\ell j}(x,y)dy,
\eeq
where 
\bel{Omega_t,j}
\Omega_{\ell j}(x,y)~=~\int_{\R^\N} e^{2\pi\i\left(\Phi(x,\xi)-y\cdot\xi\right)}\delta_{\ell j}(\xi)\phi_j(\xi)\sigma(x,y,\xi)d\xi
\eeq


For convenience, we define 
\bel{Partial}
\begin{array}{cc}\ds
\F^\sharp_{\ell } f(x)~=~\int_{\R^\N} f(y)\Omega^\sharp_{\ell  }(x,y)dy,
\\\\ \ds
\Omega^\sharp_{\ell }(x,y)~=~\sum_{j>0}\int_{\R^\N}e^{2\pi\i \left(\Phi(x,\xi)-y\cdot\xi\right)}\delta_{\ell j }(\xi) \phi_j(\xi)\sigma(x,y,\xi)d\xi.
\end{array}
\eeq
The "negative" part is fairly easy to handle, for if we let

\bel{negative}
\begin{array}{cc}\ds
\F^{\flat}_{\ell } f(x)~=~\int_{\R^\N} f(y)\Omega^{\flat}_{\ell  }(x,y)dy,
\\\\ \ds
\Omega^{\flat}_{\ell }(x,y)~=~\sum_{j\le 0}\int_{\R^\N}e^{2\pi\i \left(\Phi(x,\xi)-y\cdot\xi\right)}\delta_{\ell j }(\xi) \phi_j(\xi) \sigma(x,y,\xi) d\xi.
\end{array}
\eeq
Therefore we need to consider 

\bel{decom}
\F a(x)~\equiv ~\sum_{j\in\Z}\sum_{\I}\prod_{i\in\I}\sum_{\ell_i\le j}\F_{\ell j} a(x)
\equiv \sum_{\I} \F_{\I}a(x)
\eeq
Then it is easy to see

\bel{decom2}
|\F a(x)|~\le ~\sum_{\I}\prod_{i\in\I}\sum_{\ell_i}\left|\sum_{j\in\Z }\F_{\ell j} a(x)\right|
\equiv \sum_{\I} \F^\flat_{\I}a(x)+\sum_{\I}\F^{\sharp}_{\I}a(x)
\eeq

where
\bel{two component}
\F^\flat_{\I}a(x)= \prod_{i\in\I}\sum_{\ell_i}\F^\flat_{\ell}a(x),\qquad \F^\sharp_{\I}a(x)= \prod_{i\in\I}\sum_{\ell_i}\F^\sharp_{\ell}a(x)
\eeq

Since there are finitely many $\I$'s, if suffice to prove (\ref{aim2}) for $\F^\flat_{\I}$ and $\F^\sharp_{\I}$.


 Now consider the partial operator $\F^\flat_{\I}$
 compute directly and note that $\sigma$ has compact support in $(x,y)\in\R^\N\times\R^\N$,


\bel{part0}
\begin{array}{lr}\ds
\int_{\R^\N}\left|\F^\flat_{\I} a(x)\right| dx
~\doteq~\int_{\R^\N}\left|\int_{\R^\N}a(y)~\prod_{i\in\I}\sum_{\ell_i}\Omega^\flat_{\ell j}(x,y)dy\right| dx
\\ \ds~~~~~~~~~~~~~~~~~~~~~~~~~~~~~~~~~~~~~~~~~~~~~~~~~~~~~~~~~~~~~~~~~~~~~~~~~~
~~~~~~~~~~~~~~~~~~~~~~~~~~~~~
 \hbox{\small{by (\ref{delta_lj}) and (\ref{delta Sum})}}
\\\\ \ds~~~~~~~~~~~~~~~~~~~~~~~~~~~~~~~
~\leq~\prod_{i\in\I}\sum_{\ell_i}\int_{\supp\sigma}\left\{\int_{\R^\N} |a(y)\Omega^{\flat}_{\ell j}(x,y)|dy\right\}dx

\\\\ \ds~~~~~~~~~~~~~~~~~~~~~~~~~~~~~~~
~\leq~\sum_{j\leq0}~\prod_{i\in\I}\left\{\sum_{\ell_i}
2^{(j-\ell_i)\N_i}2^{j\N_n}\right\}
\\\\ \ds~~~~~~~~~~~~~~~~~~~~~~~~~~~~~~~
\le \C.
\end{array}
\eeq 















%For each partial operator $\F_{\ell j}$,
%Consider  a subset $\Q^{\I}_{r\ell}(x_o)\subset \R^{\N_\I}\times\R^{\N_n}$, so-called the {\it region of influence},  satisfying
%\bel{Q norm}
%|\Q^{\I}_{r\ell}(x_o)|~\leq~\C~r^{|\I|+1}2^{\epsilon (\ell)}.
%\eeq
%where $\epsilon(\ell)$ is a linear function of $(\ell_1,\dots,\ell_{n-1})$ and will be determined momentarily.


%In the  following  sections, we will first  prove that for $\sigma\in\S^{-{\N-n\over 2}}$,
%\bel{inside}
%\int_{\Q^\I_{r\ell}}\int_{\R^{\N_\J}}|\F a(x)|dx\le \C,\qquad\int_{\Q^\I_{r\ell}}\int_{\R^{\N_\J}}|\F^* a(x)|dx\le \C,
% \eeq

%then we prove

%\bel{outside}
%\int_{\prod_{i\in\I\cup\{n\}}\R^{\N_i}\setminus \Q^\I_{r\ell}}\int_{\prod_{i\in\J}\R^{\N_i}}|\F_{\I}a(x)|dx\le \C,\qquad
%\int_{\prod_{i\in\I\cup\{n\}}\R^{\N_i}\setminus \Q^\I_{r\ell}}\int_{\prod_{i\in\J}\R^{\N_i}}|\F^*_{\I}a(x)|dx\le \C.
% \eeq























































\section{  Region of influence}
\setcounter{equation}{0}










Let $~j>0~$, for every subspace $~\R^{\mathbf{N}_i}, ~i\in \K.$,  we consider a collection of points $~\{^v\xi^i_j\}_v~$ uniformly distributed on on $~\mathbb{S}^{\mathbf{N}_i-1}~$ with grid length equal to $~2^{-(j-\ell_i)}~$  multiplied by some suitable constant if $~i\in \K$. Note that $\ell_n=0$ in this section.

Denote $~\mathbf{V}_i,~ i\in \K~$ to be the set consisting every index $~v~$ of $~\{^v\xi^i_j\}_v$.

\begin{enumerate}
    \item [(1)] For every given $\xi^i\in \R^{\mathbf{N}_i}, i\in\K$, there exists a $~^v\xi^i_j~$ such that $~\left|\frac{\xi^i}{|\xi^i|}-\ ^v\xi^i_j\right|\le 2^{-(j-\ell_i)/2}$.

    
     \item [(2)] There are at most a constant multiple of $~2^{(j-\ell_i)\left[\frac{\N_i-1}{2}\right]}~$ elements inside $~\mathbf{V}_i, ~i\in\K$.

% \item [(3)] For every given $\xi^n\in \R^{\mathbf{N}_n}, $, there exists a $^v\xi^n_j$ such that $\left|\frac{\xi^n}{|\xi^n|}-\ ^v\xi^n_j\right|\le 2^{-j/2}$.

    
%\item [(4)] There are at most a constant multiple of $2^{j\left[\frac{\mathbf{N}_n-1}{2}\right]}$ elements inside $\mathbf{V}_n$.
\end{enumerate}

Let $\varphi\in C_0^{\infty}(\R)$ be the smooth bump-function as before. Consider
\bel{}
    ^v\varphi^i_j(\xi^i)=\varphi\left( 2^{ (j-\ell_i)/2} \left|\frac{\xi^i}{|\xi^i|}-\ ^v\xi^i_j\right| \right),\qquad i\in\K
\eeq
whose support lies inside 
\bel{}
    ^v\Gamma_j^i=\left\{\xi^i\in\R^{\mathbf{N}_i}:  \left|\frac{\xi^i}{|\xi^i|}-\ ^v\xi^i_j\right|\le 3\cdot 2^{-(j-\ell_i)/2}  \right\}, \qquad i\in\K
\eeq
respectively.
Furthermore,  we find 
\bel{theta}
    ^v\vartheta_j(\xi)=\prod_{i\in\K }\ ^v\vartheta_j^i(\xi^i),\qquad^v\vartheta_j^i(\xi^i)=\ ^v\varphi_j^i(\xi^i){\Bigg/}\sum_{v\in \mathbf{V}_i}\ ^v\varphi_j^i(\xi^i).
\eeq

Let $\mathbf{L}_v$ be a linear transformation such that $\xi^i=\mathbf{L}_v^i\eta^i$ for every $i\in\K$ of which $\mathbf{L}_v^i$ is an $\mathbf{N}_i\times\mathbf{N}_i$- matrix with $\det \mathbf{L}_v^i=1$. In particular, $\eta_1^i$ is in the same direction of $^v\xi^i_j$. Denote $^v\eta_j^i=\eta_1^i/|\eta_1^i|$.  We have
\bel{lineartrans}
^v\xi_j^i=\mathbf{L}_v^i\ ^v\eta_j^i.
\eeq
A direct computation shows 
\bel{}
    \left|\partial_{\eta^i}^{\alpha}\ ^v\vartheta_j^i\left(\mathbf{L}_v^i\eta^i\right)\right|\le \C_{\alpha}2^{|\alpha|(j-\ell_i)/2}|\eta^i|^{-|\alpha|}, \qquad i\in\K
\eeq

for every multi-index $\alpha$.

Denote $r=|\xi^i|=|\eta^i|$. For every $\mathbf{L}_v\eta^i=\xi^i\in \ ^v\Gamma_j^i$, the angle between $\eta^i$ and $\eta_1^i$ is bounded by $\arcsin(2\cdot 2^{-(j-\ell_i)/2}), i\in \K$.   Write $\eta^i=(\eta_1^i,\eta_{\dag}^i)\in\R\times\R^{\mathbf{N}_i-1}$. By using the Polar coordinates, we have 
\bel{}
    \frac{\partial}{\partial\eta_1^i}=\left[\frac{\partial r}{\partial \eta_1^i}\right]\frac{\partial}{\partial r}+\mathbf{O}(2^{-(j-\ell_i)/2})\cdot\nabla_{\eta_{\dag}^i}, \qquad  \qquad i\in\K.
\eeq
Note that $\partial_r \ ^v\vartheta_j^i\equiv 0$ because $^v\vartheta_j^i(\xi^i)=\ ^v\vartheta_j^i(\mathbf{L}_v^i\eta^i)$ defined in $(4.3)$ is homogeneous of degree zero in $\eta^i$. From $(4.4)$ and $(4.5)$, for every multi-index $\alpha$, we have

\bel{}
    \left|\partial_{\eta_1^i}^{\alpha}\ ^v\vartheta_j^i\left(\mathbf{L}_v^i\eta^i\right)\right|\le \C_{\alpha} |\eta^i|^{-|\alpha|}, \qquad i\in\K
\eeq
\bel{}
    \left|\partial_{\eta_{\dag}^i}^{\alpha}\ ^v\vartheta_j^i\left(\mathbf{L}_v^i\eta^i\right)\right|\le \C_{\alpha}2^{|\alpha|(j-\ell_i)/2}|\eta^i|^{-|\alpha|}, \qquad i\in\K
\eeq



Let $~\gamma\in \R~$ be a constant,  and $~\ell_M=\max\{\ell_i~:~i\in\I\}$.  


Consider the rectangles 


\bel{rectangle1}
\begin{array}{cc}\ds
       ^v\mathbf{R}^i_j(x_o^i)=\Big\{x^i\in\R^{\mathbf{N}_i}:\left| \left(^T\mathbf{L}^i_vx_o^i-\nabla_{\eta^i}\Phi(x^i,\mathbf{L}^i_v \ ^v\eta^i_j) \right)_1\right|\le 4\cdot 2^{-(j-\ell_i)}2^{\gamma \ell_M},\ds
    \\\\
  \left\{ \sum_{k=2}^{\mathbf{N}_i} \left(^T\mathbf{L}^i_vx_o^i-\nabla_{\eta^i}\Phi(x^i,\mathbf{L}^i_v \ ^v\eta^i_j) \right)^2_k\right\}^{\frac{1}{2}} \le 4\cdot 2^{-(j-\ell_i)/2}2^{\gamma \ell_M}\Big \},~~i\in\K.
\end{array}
\eeq







Let $~j_0\ge 0~$ is an integer such that 
\bel{j_0}
2^{-j_0}\le r<2^{-j_0+1}.
\eeq
The \emph{region of influence} $~\mathbf{Q}^\K_{r\ell}(x_o)~$ is defined by
\bel{region of influence}
    \mathbf{Q}^\K_{r\ell}(x_o)=\bigcup_{j\ge j_0}\left\{\bigotimes_{i\in\K}\bigcup_{v\in \mathbf{V}_i}\  ^v\mathbf{R}_j^i(x_o^i)\right\}.
\eeq
Recall that there are at most  a constant multiple of $~2^{(j-\ell_i)(\mathbf{N}_i-1)/2}~$ elements in $~\mathbf{V}_i$ for $i\in\K~$ .  We have

\bel{Q est}
\begin{array}{lc}\ds
 |\mathbf{Q}^\K_{rl}(x_o)| \le \sum_{j\ge j_0
       }\prod_{i\in\K}\left\{\sum_{v\in \mathbf{V}_i}|^v\mathbf{R}_j^i(x_o^i)|\right\} 
       \\\\ \ds
  \le      \C ~ \sum_{j\ge j_0}\prod_{i\in \K} 2^{{(j-\ell_i)}(\mathbf{N}_i-1)/2}2^{-(j-\ell_i)}2^{-(j-\ell_i)(\mathbf{N}_i-1)/2}2^{\gamma \mathbf{N}_i\ell_M}
\\\\ \ds
\le \C ~\sum_{j\ge j_0} 2^{-|\K|j} 2^{\gamma \N_\K\ell_M} \prod_{i\in \I}2^{\ell_i}
\\\\ \ds
\le \C~r^{|\K|} 2^{\gamma \N_\K\ell_M}\prod_{i\in \I}2^{\ell_i}
\end{array}
\eeq



















\section{Inside the region of influence}
\setcounter{equation}{0}
In order to estimate $\F^\sharp_{\ell}a$, we write
\bel{}
\int_{\R^\N}|\F^\sharp_{\ell}a(x)|dx=\int_{\R^{\N_\J}}\int_{\Q^\K_{r \ell}}|\F^\sharp_{\ell}a(x)|dx+\int_{\R^{\N_\J}}\int_{\R^{\N_\K}\setminus\Q^\K_{r \ell}}|\F^\sharp_{\ell}a(x)|dx
\eeq

where
\bel{N}
~\N_\I=\sum_{i\in\I}\N_i,\qquad ~\N_\J=\sum_{i\in\J}\N_i,\qquad \N_\K=\N_\I+\N_n.
\eeq

Suppose that $\I=\{i_1,i_2,\dots,i_{|\I|}\}~$ and $~\J=\{j_1,j_2,\dots,j_{|\J|}\}~$.
we denote
\[
 ~x^\I=(x^{i_1},x^{i_2},\dots,x^{i_{|\I|}}),\qquad ~x^\J=(x^{j_1},x^{j_2},\dots,x^{j_{|\J|}}), \qquad ~x^\K=(x^\I,x^n)
\]

and 
\[
dx^\I=\prod_{i\in\I}dx^i,\qquad dx^\J=\prod_{i\in\J}dx^i, \qquad dx^\K=dx^\I\cdot dx^n.
\]


Now suppose $~\sigma\in \S^{-\frac{\N-n}{2}}$.





Applying Schwarz's inequality and note that $\sigma$ has compact support in $x$,  we have

\bel{}
\begin{array}{lc}\ds
  \ds
    \int_{\Q^\K_{r\ell}}\int_{\R^{\N_\J}}|\F^\sharp_{\ell }a(x)|dx
 \le \int_{\R^\J} \left\{\int_{\R^{\N_\K}}|\F^\sharp_{\ell}a|^2dx^\K\right\}^{\frac{1}{2}}dx^\J\cdot |\Q^{\K}_{r\ell}|^{\frac{1}{2}}.
 \ds
\end{array}
\eeq









Now applying the following {\bf Lemma one},  we have

\bel{Schwarz}
\begin{array}{lc}\ds
  \ds
    \int_{\Q^\K_{r\ell}}\int_{\R^{\N_\J}}|\F^\sharp_{\ell }a(x)|dx
 \le \C \prod_{i\in\I}2^{-\frac{\N-n}{2}\left[{\N_i\over \N_\K}\right] \ell_i}\int_{\R^\J}||a(\cdot,x^\J)||_{\L^p(\R^{\N_\K})}d x^\J
 \cdot |\Q^{\K}_{r\ell}|^{\frac{1}{2}}
 \\\\   \ds
  \le \C \prod_{i\in\I}2^{-\frac{\N-n}{2}\left[{\N_i\over \N_\K}\right] \ell_i}r^{(-1+\frac{1}{p})\N_\K}
 \cdot r^{{|\K|\over2}}\prod_{i\in \I}2^{{\ell_i\over2}}2^{\gamma \N_\K{\ell_M\over2}}, ~~~~~~\textit{for } \quad 
{\N-n\over 2\N_\K}~=~{1\over p}-{1\over 2}
 \\\\ \ds
   \le \C \prod_{i\in\I}2^{-\frac{\N-n}{2}\left[{\N_i\over \N_\K}\right] \ell_i} 2^{{\ell_i\over2}} 2^{\gamma \N_\K{\ell_M\over2}}.  ~~~~~~~~~~~~~~~~~~~~~~~~~~~~~~~~~~~~~\textit{for } \quad 
\N_\J\ge |\J|
 \\\\   \ds
\end{array}
\eeq









{\bf Lemma One}~~{\it Suppose $~\sigma\in \S^m~$  for $-\N/2<m\leq0~$. For every $~\ell~$, and $~a\in \H^1(\R^\N)~$, a is supported in a ball $~B(r)~$ with radius $~r$.
we have


\bel{F_t 2,p result}
\begin{array}{cc}\ds
 \int_{\R^{\N_\J}}\left\| \F^\sharp_\ell a(\cdot,x^\J)\right\|_{\L^2(\R^{\N_\K})}dx^\J
 ~\leq~\C_{p}~ \prod_{i\in\I}2^{m\left[{\N_i\over \N_\K}\right] \ell_i}  \int_{\R^{\N_\J}}\left\| a(\cdot,x^\J)\right\|_{\L^p(\R^{\N_\K})}dx^\J~
 \\\\ \ds
 ~\leq~\C_{p}~ \prod_{i\in\I}2^{m\left[{\N_i\over \N_\K}\right] \ell_i} r^{(-1+\frac{1}{p})\N_\K}\qquad \textit{for } \qquad 
{-m\over \N_\K}~=~{1\over p}-{1\over 2},
\end{array}
\eeq

where
\[
\left\| \F^\sharp_\ell a(\cdot,x^\J)\right\|_{\L^2(\R^{\N_\K})}=\left\{\int_{\R^{\N_\K}} |\F^\sharp_\ell a (x)|^2 dx^{\K}\right\}^\frac{1}{2}
\]
and 
\[
\left\| a(\cdot,x^\J)\right\|_{\L^p(\R^{\N_\K})}=\left\{\int_{\R^{\N_\K}} |a (x)|^p dx^{\K}\right\}^\frac{1}{p}.
\]
}

Now consider the case that  for each $1\le i\le n$, $~\N_i\ge 2$, we take $\gamma=-\frac{1}{2\N_K}$, then from (\ref{Schwarz}) we have


\bel{Schwarz2}
\begin{array}{lc}\ds
  \ds
    \int_{\Q^\K_{r\ell}}\int_{\R^{\N_\J}}|\F^\sharp_{\ell }a(x)|dx
   \le \C \prod_{i\in\I}2^{-\frac{\N-n}{2}\left[{\N_i\over \N_\K}\right] \ell_i} 2^{{\ell_i\over2}} 2^{\gamma \N_\K{\ell_M\over2}}
   \\\\ \ds 
   \le ~\C  \prod_{i\in\I}2^{-\left[{\N_\J+\N-2n\over 2 \N_\K}\right] \ell_i} 2^{\gamma \N_\K{\ell_M\over2}}   \le ~\C  2^{-{\ell_M\over4}},
   \\\\ \ds 
\end{array}
\eeq

which implies

\bel{part1}
\begin{array}{lc}\ds
  \ds
    \int_{\Q^\K_{r\ell}}\int_{\R^{\N_\J}}|\F^\sharp_{\I }a(x)|dx
   \le \C.
\end{array}
\eeq


































\section{Proof of Lemma One}
\setcounter{equation}{0}
we begin to prove (\ref{F_t 2,p result}) in {\bf Lemma One}.

Suppose $\sigma\in\S^m$ for $-\N/2<m<0$. Let $\delta_{\ell j}(\xi)$ defined in (\ref{delta_lj}).
We write
\[
\Phi_\J(x,\xi)=\sum_{i\in\J}\Phi_i(x^i,\xi^i), \qquad \Phi_\K(x,\xi)=\sum_{i\in\K}\Phi_i(x^i,\xi^i),
\]
\[
\Hat{f}(\xi^\K,y^\J)=\int_{\R^{\N_\K}}f(y^\K,y^\J)e^{-2\pi\i y^\K\cdot\xi^\K}dy^\K.
\]
Hence we can write 
\bel{F_t decom}
\begin{array}{lc}
\F_{\ell j} f(x)~=~  \int_{\R^\N} e^{2\pi\i\Phi(x,\xi)} \delta_{\ell j}(\xi)\phi_j(\xi)\Hat{f}(\xi) \sigma(x,y,\xi)d\xi
\\\\ \ds
~=~\int_{\R^{\N_\J}}  \int_{\R^{\N_\J}} e^{2\pi\i\Phi_\J(x,\xi)} \int_{\R^{\N_\K}} e^{2\pi\i\Phi_{\K}(x,\xi)} 
 \delta_{\ell j}(\xi)\phi_j(\xi)\int_{\R^{\N_\J}}\Hat{f}(\xi^\K,y^\J)e^{-2\pi\i y^\J\cdot\xi^\J}dy^\J \sigma(x,y,\xi)d\xi^\K   d\xi^\J 
\\\\ \ds
~=~  \int_{\R^{\N_\J}}dy^\J  \int_{\R^{\N_\J}}e^{2\pi\i(\Phi_\J(x,\xi)-y^\J\cdot\xi^\J)}  d\xi^\J \int_{\R^{\N_\K}} e^{2\pi\i\Phi_{\K}} 
 \delta_{\ell j}(\xi)\phi_j(\xi)\Hat{f}(\xi^\K,y^\J) \sigma(x,y,\xi) d\xi^\K .
\\\\ \ds
~=~  \int_{\R^{\N_\J}}\int_{\R^{\N_\J}}e^{2\pi\i(\Phi_\J(x,\xi)-y^\J\cdot\xi^\J)} \G_{\ell j}f(x^\K,y^\J,\xi^\J) d\xi^\J dy^\J  .
\\\\ \ds
\end{array}
\eeq

Now we may write 
\bel{F_t decom2}
\begin{array}{lr}\ds
\G_{\ell j}f(x^\K,y^\J,\xi^\J)~=~ \int_{\R^{\N_\K}} e^{2\pi\i\Phi_{\K}(x,\xi)} 
\sigma(x,\xi) \delta_{\ell j}(\xi)\phi_j(\xi)\Hat{f}(\xi^\K,y^\J) \sigma(x,y,\xi) d\xi^\K
\\\\ \ds
~=~ \prod_{i\in\I}2^{m\left[{\N_i\over \N_\K}\right] \ell_i}\int_{\R^{\N_\K}} e^{2\pi\i\Phi_{\K}(x,\xi)}\sigma(x,y,\xi)\left(1+|\xi^\K|^2\right)^{-{m\over2}} 
\\\\ \ds ~~~~~~~~~~~~~~
\left\{\delta_{\ell j}(\xi)\phi_j(\xi)\prod_{i\in\I}2^{-\left[{\N_i\over \N_\K}\right] \ell_i}\left(1+|\xi^\K|^2\right)^{m\over2}\right\}
\Hat{f}(\xi^\K,y^\J)
d\xi^\K
\\\\ \ds
~\doteq~\prod_{i\in\I}2^{m\left[{\N_i\over \N_\K}\right] \ell_i}\int_{\R^{\N_\K}} e^{2\pi\i\Phi_{\K}(x,\xi)}\sigma(x,y,\xi)\left(1+|\xi^\K|^2\right)^{-{m\over2}} \Hat{\T_{\ell j} f}(\xi^\K,y^\J,\xi^\J)d\xi^\K.
\end{array}
\eeq
Observe that in the support $~\delta_{\ell j}(\xi)\phi_j(\xi)~$, $~|\xi^\J|\le \C$ for every $j>0$. Hence we know
$~\sigma(x,y,\xi)(1+|\xi^\K|^2)^{-{m\over 2}}\in\S^0(\R^{\N_\K})$ with respect to $\xi^\K$. 
Moreover $\Phi_{\K}(x,\xi)$ satisfies the nondegeneracy condition.
Now We have
\bel{convolution T_tf}
\begin{array}{cc}\ds
\Big(\T_{\ell j} f\Big)(x^\K,y^\J,\xi^\J)~=~\int_{\R^{\N_\K}}f(y^\K,y^\J)\mathcal{K}_{\ell j}(x^\K-y^\K,\xi^\J)dy^\K,
\\\\ \ds
\mathcal{K}_{\ell j}(x^\K,\xi^\J)~=\prod_{i\in\I}2^{-m\left[{\N_i\over \N_\K}\right]\ell_i}\int_{\R^{\N_\K}} e^{2\pi\i x^\K\cdot\xi^\K} \delta_{\ell j}(\xi) \phi_{j}(\xi)\left(1+|\xi^\K|^2\right)^{m\over 2} d\xi^\K. 
\end{array}     
\eeq

We denote
\bel{}
\G_{\ell}f=\sum_{j>0}\G_{\ell j},\qquad \T_{\ell}f=\sum_{j>0}\T_{\ell j},\qquad
\mathcal{K}_{\ell}f=\sum_{j>0}\mathcal{K}_{\ell j}.
\eeq

By Plancherel's theorem, and the $\L^2$ boundedness of $\F$ of order zero, we have
\bel{Plancherel}
\left\|\G_\ell f(\cdot,\xi^\J,y^\J)\right\|_{\L^2(\R^{\N_\K})}~\le~\left\|\T_\ell f(\cdot,\xi^\J,y^\J)\right\|_{\L^2(\R^{\N_\K})}
\eeq
Write $x=(z,w^\I,w^\J)\in \R^{\N_n}\times\R^{\N_\I}\times\R^{\N_\J}$,  $y=(u,v^\I,v^\J)\in\R^{\N_n}\times\R^{\N_\I}\times\R^{\N_\J}$, whose dual variable is $\xi=(\tau,\lambda^\I,\lambda^\J)\in\R^{\N_n}\times\R^{\N_\I}\times\R^{\N_\J}$. From (\ref{convolution T_tf})-(\ref{phi}), we have
\bel{K_t sum}
\begin{array}{lr}\ds
\mathcal{K}_{\ell j}((x^\I,x^n),\lambda^\J)~=\prod_{i\in\I}2^{-m\left[{\N_i\over \N_\K}\right]\ell_i}\int_{\R^{\N_\K}} e^{2\pi\i (w^\I\cdot\lambda^\I+z\cdot\tau)} \delta_{\ell j}(\xi) \phi_{j}(\xi)\left(1+|\lambda^\I|^2+|\tau|^2\right)^{m\over 2} d\lambda^\I d\tau. 
\\\\ \ds~~~~~~~~~
~=~ \prod_{i\in\I}2^{-m\left[{\N_i\over \N_\K}\right] \ell_i}\int_{\R^{\N_\K}} e^{2\pi\i (z\cdot\tau+w^\I\cdot\lambda^\I)} 
%\\\\ \ds~~~~~~~~~
\delta_{\ell j}(\tau,\lambda^\I,\lambda^\J) \phi_j(\tau,\lambda^\I,\lambda^\J)\left(1+\tau^2+|\lambda^\I|^2\right)^{m\over 2} d\tau d\lambda^\I .
\end{array}
\eeq
Note that  $\phi_j(\xi)$ is supported in the dyadic annuli $2^{j-1}\leq|\xi|\leq2^{j+1}$. 
On the other hand, $\delta_{\ell j}(\xi)$ defined in (\ref{delta_lj}) is supported in the dyadic cone $\Lambda_{\ell j}$  given in (\ref{Lambda_lj}).

We have $|\tau|\leq\C 2^j~$ , $~|\lambda^i|\leq\C 2^{j-\ell_i}, i\in\I$ and $~|\lambda^i|\le \C, i\in\J$ so that
\bel{support}
\left| \supp\delta_{\ell j}(\tau,\lambda^\I,\lambda^\J) \phi_j(\tau,\lambda^\I,\lambda^\J)\left(1+\tau^2+|\lambda^\I|^2\right)^{m\over 2}\right|~\leq~\C~ 2^{j\N_n}\prod_{i\in\I}2^{(j-\ell_i)\N_i}.
\eeq
Recall $\delta_{\ell j}(\xi)$ satisfying  the differential inequality in (\ref{delta Diff Ineq}). We have
\bel{Computation Est1}
\begin{array}{lr}\ds
\left|\partial_\tau^M  \delta_{\ell j}(\tau,\lambda^\I,\lambda^\J) \phi_j(\tau,\lambda^\I,\lambda^\J)\left(1+\tau^2+|\lambda^\I|^2\right)^{m\over 2}\right|
~\leq~\C_M~\left(1+\tau^2+|\lambda^\I|^2\right)^{m\over 2}\left({1\over |\tau|}\right)^M
\\\\ \ds~~~~~~~~~~~~~~~~~~~~~~~~~~~~~~~~~~~~~~~~~~~~~~~~~~~~~~~~~~~
~\leq~\C_M ~2^{jm}2^{-jM},
\\\\ \ds
\left|\partial_{\lambda_i}^{N_i}  \delta_{\ell j}(\tau,\lambda^\I,\lambda^\J) \phi_j(\tau,\lambda^\I,\lambda^\J)\left(1+\tau^2+|\lambda^\I|^2\right)^{m\over 2}\right|
~\leq~\C_{N_i}~\left(1+\tau^2+|\lambda^\I|^2\right)^{m\over 2}\left({1\over |\lambda_i|}\right)^{N_i}
\\\\ \ds~~~~~~~~~~~~~~~~~~~~~~~~~~~~~~~~~~~~~~~~~~~~~~~~~~~~~~~~~~~~
~\leq~\C_{N_i}~2^{jm}2^{-(j-\ell_i)N_i},
\\\\ \ds
\end{array}
\eeq
for every $M\ge1$ and $N_i\ge1, i\in \I$.

Let $N=N_1+N_2+\cdots+N_{|\I|}$. An $M+N$-fold integration by parts $w.r.t~(\tau,\lambda^\I)$ gives 
\bel{K_t by parts} 
\begin{array}{lr}\ds
\prod_{i\in\I}2^{-m\left[{\N_i\over \N_\K}\right] \ell_i}\int_{\R^{\R_\I+\N_n}} e^{2\pi\i (z\cdot\tau+w^\I\cdot\lambda^\I)} 
%\\\\ \ds~~~~~~~~~
\delta_{\ell j}(\tau,\lambda^\I,\lambda^\J) \phi_j(\tau,\lambda^\I,\lambda^\J)\left(1+\tau^2+|\lambda^\I|^2\right)^{m\over 2} d\tau d\lambda^\I.
\\\\ \ds
~\leq~\C_{M~N}~\prod_{i\in\I}2^{-m \left[{\N_i\over \N_\K}\right]\ell_i} ~|z|^{-M}\prod_{i\in\I} |w^i|^{-N_i} \Bigg|\iiint_{\R^{\N_n}\times\R^{\N_\I}\times\R^{\N_J}} e^{2\pi\i (z\cdot \tau+w^\I\cdot\lambda^\I)} 
\\\\ \ds~~~~~~~
\partial_\tau^M\prod_{i\in\I}\p_{\lambda^i}^{N_i}\delta_{\ell j}(\tau,\lambda^\I,\lambda^\J) \phi_j(\tau,\lambda^\I,\lambda^\J)\left(1+\tau^2+|\lambda^\I|^2\right)^{m\over 2} d\tau d\lambda^\I\Bigg|
\\\\ \ds
~\leq~\C_{M~N}~ \prod_{i\in\I} 2^{-m \left[{\N_i\over \N_\K}\right] \ell_i}\left\{2^{jm} 2^{j\N_n}\prod_{i\in\I}2^{(j-\ell_i)\N_i}\right\}\left(2^{j}|z|\right)^{-M}\prod_{i\in\I}\left(2^{j-\ell_i}|w_i|\right)^{-N_i}
\\ \ds ~~~~~~~~~~~~~~~~~~~~~~~~~~~~~~~~~~~~~~~~~~~~~~~~~~~~~~~~~~~~~~~~~~~~~~~~~~~~~~~~~~~~~~~~~
\quad \hbox{\small{by (\ref{support})-(\ref{Computation Est1})}}
\\\\ \ds
~=~\C_{M~N}~2^{j\N_n\left(1+{m\over \N_\K}\right)}  \left(2^{j}|z|\right)^{-M}\prod_{i\in\I} 2^{(j-\ell_i)\N_i\left(1+ {m \over \N_\K}\right)}\left(2^{j-\ell_i}|w_i|  \right)^{-N_i}.
\end{array}
\eeq
We choose 
\bel{N,M tau}
\begin{array}{cc}\ds
M=0~~~\hbox{if}~~~|z|\leq2^{-j}\qquad \hbox{or}\qquad M=\N_n ~~~\hbox{if}~~~ |z|>2^{-j}; 
\\\\ \ds
N_i=0~~~ \hbox{if}~~~ |w_i|\leq2^{-j+\ell_i}\qquad \hbox{or}\qquad N_i=\N_i~~~\hbox{if}~~~ |w_i|>2^{-j+\ell_i},\qquad i\in\I.
\end{array}
\eeq
From (\ref{K_t sum}) and (\ref{K_t by parts}), we have
\bel{Sum K_t est} 
\begin{array}{lr}\ds
\left|\mathcal{K}_{\ell }((w^\I,z),\lambda^\J)~=\  \right|~\leq~\C_{M~N~L} \sum_{j>0}2^{j\N_n\left(1+{m\over \N_\K}\right)}  \left(2^{j}|z|\right)^{-M}\prod_{i\in\I} 2^{(j-\ell_i)\N_i\left(1+{m\over \N_\K}\right)}\left(2^{j-\ell_i}|w_i|\right)^{-N_i}
\\\\ \ds~~~~~~~~~~~~~~~~ 
~\leq~\C_{M~N}  \left\{\sum_{j>0}2^{j\N_n\left(1+{m\over \N_\K}\right)}  \left(2^j|z|\right)^{-M}\right\}
\prod_{i\in\I} \left\{\sum_{j>0} 2^{(j-\ell_i)\N_i\left(1+{m\over \N_\K}\right)}\left(2^{j-\ell_i}|w_i|\right)^{-N_i}\right\}
\\\\ \ds~~~~~~~~~~~~~~~~
~=~\C~  \left\{\sum_{|z|\leq2^{-j}} 2^{j\N_n\left(1+{m\over \N_\K}\right)}~+~\sum_{|z|>2^{-j}}2^{j\N_n\left(1+{m\over \N_\K}\right)} \left(2^{j}|z|\right)^{-1}\right\}
\\\\ \ds~~~~~~~~~~~~~~~~~~~~~
\times \prod_{i\in\I} \left\{\sum_{|w_i|\leq2^{-j+\ell_i}} 2^{(j-\ell_i)\N_i\left(1+{m\over \N_\K}\right)}~+~\sum_{|w_i|>2^{-j+\ell_i}}2^{(j-\ell_i)\N_i\left(1+{m\over \N_\K}\right)} \left(2^{j-\ell_i}|w_i|\right)^{-1}\right\}
\\ \ds~~~~~~~~~~~~~~~~~~~~~~~~~~~~~~~~~~~~~~~~~~~~~~~~~~~~~~~~~~~~~~~~~~~~~~~~~~~~~~~~~~~~~~~~~~~~~~~~~~~~~~~~~~~~~~~~~~~~~~~~
\hbox{\small{by (\ref{N,M tau})}}
\\\\ \ds~~~~~~~~~~~~~~~~
~\leq~\C~ \left({1\over |z|}\right)^{\N_n\left(1+{m\over \N_\K}\right)}\prod_{i\in\I}\left({1\over|w_i|}\right)^{\N_i\left(1+{m\over \N_\K}\right)}.
\end{array}
\eeq


Now recall the famous Hardy-Littlewood-Sobolev inequality.


{\bf Hardy-Littlewood-Sobolev inequality} \quad {\it  For  $0<\alpha<\mathbf{N}$, we have}

\bel{Hardy-Littlewood-Sobolev}
    \left\{\int_{\R^\mathbf{N}}\left|\int_{\R^\mathbf{N}}f(y)\left(\frac{1}{|x-y|}\right)^{\mathbf{N}-\alpha}dy  \right|^q dx\right\}^{\frac{1}{q}}\le C_{p,q}||f||_{\mathbf{L}^p(\R^\mathbf{N})}
\eeq

{\it for $1<p<q<\infty$ if and only if }
\bel{balanced}
    \frac{\alpha}{\mathbf{N}}=\frac{1}{p}-\frac{1}{q}.
\eeq

Let $-\frac{m}{\mathbf{N_\K}}=\frac{1}{p}-\frac{1}{2}$ and $L$ large enough. By applying \textbf{Hardy-Littlewood-Sobolev inequality} on every coordinate subspace and using Minkowski integral inequality, we have




\bel{T_t regularity}
\begin{array}{lc}\ds
\left\|\T_\ell f(\cdot,y^\J,\lambda^\J)\right\|_{\L^2(\R^{\N_\K})}~=~\left\{\int_{\R^{\N_\K}} \Big(f\ast\mathcal{K}_\ell \Big)^2(w^\I,z,y^\J,\lambda^\J)dw^\I dz\right\}^{1\over 2}
\\\\ \ds~~~~~~~~~~~~~~~~~~~
~\leq~\C~\Bigg\{\int_{\R^{\N_\K}} \Bigg\{\int_{\R^{\N_\K}} \left|f(v^\I,u,y^\J)\right|
\left({1\over |z-u|}\right)^{\N_n\left(1+{m\over \N_\K}\right)} 
\\\\ \ds
~~~~~~~~~~~~~~~~~~~~~~~~
\prod_{i\in\I}\left({1\over|w^i-v^i|}\right)^{\N_i\left(1+{m\over \N_\K}\right)}dv^\I du\Bigg\}^2dw^\I dz \Bigg\}^{1\over 2}
\\ ~~~~~~~~~~~~~~~~~~~~~~~~~~~~~~~~~~~~~~~~~~~~~~~~~~~~~~~~~~~~~~~~~~~~~~~~~~~~~~~~~~~~~~~~~~~~~~~~~~
 \hbox{\small{by (\ref{Sum K_t est})}}
\\\\ \ds
~~~~~~~~~~~~~~~~~~~~~~

~\leq~\C_p~\left\| f(\cdot,y^\J)\right\|_{\L^p(\R^{\N_\K})}.
\end{array}
\eeq
Therefore, using (\ref{Plancherel}) and (\ref{T_t regularity}), note that $\sum_{j>0}\delta_{\ell j}(\xi)\phi_j(\xi)$ has compact support in $\xi^\J$ and  $~\sigma$ has compact support in $x$, it follows

\bel{lemma1}
\begin{array}{cc}
||\F_{\ell}a(\cdot,x^\J)||_{\L^2(\R^{\N_\K})}\le\left\{
\int_{\R^{\N_\K}}
\left|\int_{\R^{\N_\J}} \int_{\R^{\N_\J}}
\G_{\ell}a(x^\K,\xi^\J,y^\J)
 d\xi^\J dy^\J
\right|^2 dx^\K
\right\}^{\frac{1}{2}}
\\\\ \ds
\le \int_{\R^{\N_\J}} \int_{\R^{\N_\J}}
\left\{\int_{\R^{\N_\K}}
|\G_{\ell}a(x^\K,\xi^\J,y^\J)|^2
 dx^\K\right\}^{\frac{1}{2}}
  d\xi^\J dy^\J
  \\\\ \ds
\le \int_{\R^{\N_\J}} \int_{\R^{\N_\J}}
\left\{\int_{\R^{\N_\K}}
|\T_{\ell}a(x^\K,\xi^\J,y^\J)|^2
 dx^\K\right\}^{\frac{1}{2}}
  d\xi^\J dy^\J
  \\\\ \ds
  \le\C_{p}~ \prod_{i\in\I}2^{m\left[{\N_i\over \N_\K}\right] \ell_i}  \int_{\R^{\N_\J}} \left\{\int_{\R^{\N_\K}}|a(x^\K,x^\J)|^pdx^\K\right\}^{\frac{1}{p}} dx^\J
\end{array}
\eeq

Hence we have proved (\ref{F_t 2,p result}).
















































































\section{An $\H^1\mt\L^1$-estimate}
\setcounter{equation}{0}
 The following estimates of the kernel are key to our final results.


{\bf Lemma Two}~~{\it  
Suppose $\sigma\in\S^{-{\N-n\over 2}}$.  we have
\bel{Est1}
\int_{\R^\N} \left|\Omega_{\ell j}(x,y)\right| dx~\leq~\C_{}~
\prod_{i\in\I}2^{-\ell_i{\N_i-1\over 2}}\prod_{i\in\J}2^{-j\frac{\mathbf{N}_i-1}{2}},
\eeq
\bel{Est2}
\int_{\R^\N}\left|\Omega_{\ell j}(x,y)-\Omega_{\ell j}(x,x_o)\right|dx~\leq~\C_{}~2^j|y-x_o|~
\prod_{i\in\I}2^{-\ell_i{\N_i-1\over2}}\prod_{i\in\J}2^{-j\frac{\mathbf{N}_i-1}{2}}
\eeq
and
\bel{Est3}
\int_{\R^{\N_\J}}\int_{\R^{\N_\K}\setminus\Q^\K_{r\ell}(x_o)}\left|\Omega_{\ell j}(x,y)\right|dx~\leq~\C_{}~{2^{-j}\over r}~\prod_{i\in\I} 2^{-\ell_i{\N_i-1\over 2}}\prod_{i\in\J}2^{-j\frac{\mathbf{N}_i-1}{2}}2^{-\gamma \ell_M},\qquad y\in B_r(x_o)
\eeq
whenever $j>j_0$.}
\v

Now if we  consider the case, for each $1\le i\le n~$, $\N_i\ge2$ and take $\gamma=-\frac{1}{2\N_\K}$. There {\bf Lemma two} imlpies


 
\bel{Est1 2}
\int_{\R^\N} \left|\Omega_{\ell j}(x,y)\right| dx~\leq~\C_{}~
\prod_{i\in\I}2^{-\frac{\ell_i}{2}}
\eeq
\bel{Est2 2}
\int_{\R^\N}\left|\Omega_{\ell j}(x,y)-\Omega_{\ell j}(x,x_o)\right|dx~\leq~\C_{}~2^j|y-x_o|~
\prod_{i\in\I}2^{-\frac{\ell_i}{2}}
\eeq
and
\bel{Est3 2}
\int_{\R^{\N_\J}}\int_{\R^{\N_\K}\setminus\Q^\K_{r\ell}(x_o)}\left|\Omega_{\ell j}(x,y)\right|dx~\le \C~{2^{-j}\over r}~\prod_{i\in\I} 2^{-\frac{\ell_i}{2}} 2^{\frac{\ell_M}{2\N_\K}},\qquad y\in B_r(x_o)
\eeq
whenever $j>j_0$.


Consider $j\le j_0$.  We  write
\bel{F_t cancella}
\int_{\R^n} a(y)\Omega_{\ell j}(x,y)dy~=~\int_{B_r(x_o)} a(y) \left(\Omega_{\ell j}(x,y)-\Omega_{\ell j}(x,x_o)\right)dy
\eeq 
because 
$\int_{B_r(x_o)} a(y)dy=0$ and $a$ is supported in $B_r(x_o)$. 

By using (\ref{Est2 2}) and (\ref{F_t cancella}), we find
\bel{Norm Est1}
\begin{array}{lr}\ds
\int_{\R^n}\left|\int_{\R^n} a(y)\Omega_{\ell j}(x,y)dy\right|dx
~\leq~\int_{B_r(x_o)} |a(y)|\left\{\int_{\R^n} \left|\Omega_{\ell j}(x,y)-\Omega_{\ell j}(x,x_o)\right| dx\right\} dy
\\\\ \ds~~~~~~~~~~~~~~~~~~~~~~~~~~~~~~~~~~~~~~~~~~~~~~
~\leq~ \C~2^j|y-x_o|~\prod_{i\in\I}2^{-\frac{\ell_i}{2}}
\\\\ \ds~~~~~~~~~~~~~~~~~~~~~~~~~~~~~~~~~~~~~~~~~~~~~~
~\leq~\C 2^j r~\prod_{i\in\I}2^{-\frac{\ell_i}{2}},\qquad y\in B_r(x_o).
\end{array}
\eeq

By summing over all regarding $\ell$ and $j$ s, we have
\bel{Sum1}
\begin{array}{lr}\ds
 \sum_{2^j\leq r^{-1}} ~\prod_{i\in\I}\sum_{\ell_i}~\int_{\R^n}\left|\int_{\R^n} a(y)\Omega_{\ell j}(x,y)dy\right|dx
\\\\ \ds
~\leq~\C~r\sum_{2^j\leq r^{-1}} 2^j 

 \prod_{i\in\I}\sum_{\ell_i}2^{-\frac{\ell_i}{2}}
\qquad
\hbox{\small{by (\ref{F_t cancella})-(\ref{Norm Est1})}}
\\\\ \ds
~\leq~\C~r\sum_{2^j\leq r^{-1}} 2^j
\\\\ \ds
~\leq~\C.
\end{array}
\eeq
For $j>j_0$,  (\ref{Est3}) implies
\bel{Norm Est2}
\begin{array}{lr}\ds
\int_{\R^{\N)\J}}\int_{\R^{\N_\K}\setminus\Q^\K_{r \ell}(x_o)}\left|\int_{\R^\N} a(y)\Omega_{\ell j}(x,y)dy\right|dx
~\leq~\int_{B_r(x_o)} |a(y)|\left\{\int_{\R^{\N_\J}}\int_{\R^{\N_\K}\setminus\Q^\K_{r \ell}(x_o)} \left|\Omega_{\ell j}(x,y)\right| dx\right\} dy
\\\\ \ds~~~~~~~~~~~~~~~~~~~~~~~~~~~~~~~~~~~~~~~~~~~~~~~~~~~~~~~
~\leq~\C~{2^{-j}\over r}~\prod_{i\in\I}2^{-\frac{\ell_i}{2}} 2^{\frac{\ell_M}{2\N_\K}}.
\end{array}
\eeq
By summing over all regarding $\ell$ and $j$ s, we have
\bel{Sum2}
\begin{array}{lr}\ds
\sum_{j>j_0}~\prod_{i\in\I}\sum_{\ell_i}~\int_{\R^{\N_\J}}\int_{\R^{\N_\K}\setminus\Q^\K_{r \ell }(x_o)}\left|\int_{\R^\N} a(y)\Omega_{\ell j}(x,y)dy\right|dx
\\\\ \ds
~\leq~\C~r^{-1}\sum_{j>j_0} 2^{-j}~
 \prod_{i\in\I}\sum_{\ell_i}2^{-\frac{\ell_i}{2}}2^{\frac{\ell_M}{2\N_\K}}
 \qquad
\hbox{\small{by (\ref{Norm Est2})}}
\\\\ \ds
~\leq~\C~~r^{-1}\sum_{j>j_0} 2^{-j}
\\\\ \ds
~\leq~\C .
\end{array}
\eeq

Now (\ref{Sum1}) and $(\ref{Sum2})$ implies

\bel{part2}
\begin{array}{lc}\ds
    \int_{\R^{\N_\K}\setminus \Q^\K_{r\ell}}\int_{\R^{\N_\J}}|\F^\sharp_{\I }a(x)|dx
   \le \C.
\end{array}
\eeq
Therefore, Combine (\ref{part0}), (\ref{part1}) and  (\ref{part2}), we get the desired result.



In conclusion,  We have proved {\bf Theorem One} under the conditions, for each $1\le i\le n$, $\N_i\ge 2$.

{\bf Lemma two} shows that if there is $\N_i=1$ for $1\le i\le n$,~ then the "decaying" property {\it w.r.t} $~\ell_i$ would not exist. Hence it is more difficult for this case. However, we can still handle this situation by change the definition of region of influence accordingly.



























































































\section{Proof of Lemma Two}
\setcounter{equation}{0}


Recall $\Omega_{\ell j}(x,y)$ from $(\ref{Omega_t,j})$. We write

\bel{Omega_lj}
\begin{array}{cc}\ds
        \Omega_{\ell j}(x,y)=\sum_{v\in \cup_{i\in\K}\mathbf{V}_i}\Omega_{\ell j}^v(x,y),
\\\\ \ds
        \Omega_{\ell j}^v(x,y)=\int_{\R^\mathbf{N}} e^{2\pi \mathbf{i}(\Phi(x,\xi)-y\cdot\xi) }\ ^v\vartheta_j(\xi)\delta_{\ell j}(\xi)\phi_j(\xi)\sigma(x,\xi)d\xi
\end{array}
\eeq










where $^v\vartheta_j, \delta_{\ell j}$ and $\phi_j$ are defined in $(\ref{theta}), (\ref{delta_lj})$ and $(\ref{phi})$ respectively.

From $(\ref{phasesum})$, we rewrite

\bel{Omega v}
\begin{array}{cc}\ds
\Omega_{\ell j}^v(x,y)=\int_{\R^\mathbf{N}} e^{2\pi \mathbf{i}(\Phi(x,\xi)-y\cdot\xi) }\ ^v\vartheta_j(\xi)\delta_{\ell j}(\xi)\phi_j(\xi)\sigma(x,\xi)d\xi
\\\\ \ds
=\int_{\R^\mathbf{N}} \left\{ \prod_{i=1}^n e^{2\pi \mathbf{i}(\Phi_i(x^i,\xi^i)-y^i\cdot\xi^i) }\ ^v\vartheta_j^i(\xi)\right\}\delta_{\ell j}(\xi)\phi_j(\xi)\sigma(x,\xi)d\xi\\\\ \ds
=\int_{\R^\mathbf{N}} \left\{ \prod_{i=1}^n e^{2\pi \mathbf{i}(\Phi_i(x^i,\mathbf{L}_v^i\eta^i)-y^i\cdot\mathbf{L}_v^i\eta^i) }\ ^v\vartheta_j^i(\mathbf{L}_v^i\eta^i)\right\}\delta_{\ell j}(\mathbf{L}_v\eta)\phi_j(\mathbf{L}_v\eta)\sigma(x,\mathbf{L}_v\eta)d\eta.
\end{array}
\eeq













Consider 


\bel{Phi v}
\begin{array}{cc}\ds
\Phi_i(x^i,\mathbf{L}_v^i\eta^i)-y^i\cdot\mathbf{L}_v^i\eta^i=\left(\nabla_{\eta^i}\Phi_i(x^i,\mathbf{L}_v^i\ ^v\eta_j^i)-\ ^T\mathbf{L}_v^iy^i \right)\cdot\eta^i+\Psi_i(x^i,\eta^i),
\\\\ \ds
    \Psi_i(x^i,\eta^i)=\Phi_i(x^i,\mathbf{L}_v^i\eta^i)-\nabla_{\eta^i}\Phi_i(x^i,\mathbf{L}_v^i\ ^v\eta_j^i)\cdot\eta^i, \quad i\in\K.
\end{array}
\eeq









Note that $\Phi_i, i=1,2,\dots,n$ satisfies the non-degeneracy condition as $(\ref{nondeneracies})$. Recall \textbf{4.5}, chapter $IX$ of Stein $[5]$. We have

\bel{}
    \left| \partial_{\eta_1^i}^{\alpha}\Psi_i(x^i,\eta^i) \right|\le \C_{\alpha}2^{-|\alpha|(j-\ell_i)},\qquad \left| \partial_{\eta_{\dag}^i}^{\beta}\Psi_i(x^i,\eta^i) \right|\le \C_{\beta}2^{-|\alpha|(j-\ell_i)},\qquad i\in\K.
\eeq
for every multi-indices $\alpha,\beta$ whenever $2^{j-\ell_i-2}\le |\eta^i|\le2^{j-\ell_i+2},i\in\K.$



From $(\ref{Omega v})$ and $(\ref{Phi v})$, we have

\bel{Theta v}
\begin{array}{cc}\ds
        \Omega_{\ell j}^v(x,y)&=\int_{\R^\mathbf{N}}  
 e^{2\pi \mathbf{i}(\Phi_\J(x,\xi)-y^\J\cdot\xi^\J )}
        
        \prod_{i\in \K} e^{2\pi \mathbf{i}\left( \nabla_{\eta^i} \Phi_i(x^i,\mathbf{L}_v^i\ ^v\eta_j^i)-^T\mathbf{L}_v^iy^i\right)\cdot\eta^i }\Theta_{\ell j}^v(x,\eta)d\eta.
        \\\\ \ds
     \Theta_{\ell j}^v(x,\eta)&=\left\{\prod_{i\in\K} e^{2\pi \mathbf{i}\Psi_i(x^i,\eta^i) } \ ^v\vartheta_j^i(\mathbf{L}_v^i\eta^i)\right\}   \delta_{\ell j}(\mathbf{L}_v\eta)\phi_j(\mathbf{L}_v\eta)\sigma(x,\mathbf{L}_v\eta).
\end{array}
\eeq






































Define the differential operator


\bel{}
    \mathcal{D}=I-\sum_{i\in\K}\left[2^{2(j-\ell_i)}(\partial_{\eta_1^i})^2+2^{j-\ell_i}\Delta_{\eta_{\dag}^i}\right]
\eeq
Let $\Theta_{\ell j}^v(x,\eta)$ defined in $(\ref{Theta v})$. We have
\bel{}
    \left|\mathcal{D}^N\Theta_{\ell j}^v(x,\eta)\right|\le \C_N 2^{-j\left(\frac{\mathbf{N}-n}{2} \right)},\qquad N\ge 0.
\eeq
On the other hand, we have
\bel{}
    \left|\textbf{supp}\Theta_{\ell j}^v(x,\eta)\right|\le \C\prod_{i\in\K}2^{j-\ell_i}2^{(j-\ell_i)\left(\frac{\mathbf{N}_i-1}{2}\right) }.
\eeq


Thus we obtain that


\bel{Omega size}
\begin{array}{cc}\ds
        |\Omega_{\ell j}^v(x,y)|\le \C 2^{-j \frac{\mathbf{N}-n}{2}}\prod_{i\in\I}2^{j-\ell_i}2^{(j-\ell_i)\left(\frac{\mathbf{N}_i-1}{2}\right) } 2^{j}2^{j\left({\N_n-1\over 2}\right)} 
        \\\\ \ds
   \Bigg\{1+\sum_{i\in\K} \left(2^{j-\ell_i}\left|\left(\nabla_{\eta^i}\Phi_i(x^i,\mathbf{L}_v^i\ ^v\eta_j^i)-\ ^T\mathbf{L}_v^iy^i \right)_1\right| +2^{(j-\ell_i)/2}\left|\left(\nabla_{\eta^i}\Phi_i(x^i,\mathbf{L}_v^i\ ^v\eta_j^i)-\ ^T\mathbf{L}_v^iy^i \right)'\right|\right )\Bigg\}^{-2N}.
   \\\\ \ds
\end{array}
\eeq











Here $(\cdot)_1$ indicates the component in the direction $^v\eta_j^i$, and $(\cdot)'$ denotes the orthogonal component. 

In carrying out the estimate for $\int_{\R^{\mathbf{N}}}|\Omega_{\ell j}^v(x,y|dx)$, we use the majorization $(\ref{Omega size})$, as well as the change of variables

\[
    x^i \mapsto \mathbf{L}_v^i \nabla_{\eta^i}\Phi_i(x^i,\mathbf{L}_v^i\ ^v\eta_j^i),\qquad i\in\K.
\]

whose Jacobian is bounded from below, as we indicated above. Note that $\sigma$ has compact support in $x$, the result is


\bel{}
\begin{array}{cc}\ds
   \int_{\R^{\mathbf{N}}}|\Omega_{\ell j}^v(x,y)|dx\le \C_N 2^{-j \frac{\mathbf{N}-n}{2}}\prod_{i\in\K}2^{j-\ell_i}2^{(j-\ell_i)\left(\frac{\mathbf{N}_i-1}{2}\right) } 
   \\\\ \ds
    \int_{\R^{\N_\K}}\Bigg\{1+\sum_{i\in
\K} \left(2^{j-\ell_i}\left|\left(x^i-y^i \right)_1\right| +2^{(j-\ell_i)/2}\left|\left(x^i-y^i \right)'\right|\right )
   \Bigg\}^{-2N}
  dx^\K
\end{array}
\eeq








If  we choose $N$ so that $2N>\mathbf{N}$. Therefore
\bel{}
    \int_{\R^{\mathbf{N}}}|\Omega_{\ell j}^v(x,y)|dx\le \C 2^{-j \frac{\mathbf{N}-n}{2}}.
\eeq

However, there are essentially

\[
\prod_{i\in \K}2^{(j-\ell_i)\frac{\mathbf{N}_i-1}{2}}
\]

components in the decomposition (\ref{Omega_lj}). Thus we have 
\bel{}
    \int_{\R^{\mathbf{N}}}|\Omega_{\ell j}(x,y)|dx\le \C \prod_{i\in\I}2^{-\ell_i\frac{\mathbf{N}_i-1}{2}} \prod_{i\in\J}2^{-j\frac{\mathbf{N}_i-1}{2}}.
\eeq

A similar estimate holds for $\nabla_y\Omega_{lj}^v(x,y)$, once we observe that the differentiation in $y$ introduces factors bounded by $2^j$. As a result,

\[
    \int_{\R^{\mathbf{N}}}|\nabla_y\Omega_{\ell j}^v(x,y)|dx\le \C \prod_{i\in\I}2^{-\ell_i\frac{\mathbf{N}_i-1}{2}} \prod_{i\in\J}2^{-j\frac{\mathbf{N}_i-1}{2}}\cdot 2^j.
\]

and so 
    \bel{}
        \int_{\R^N}|\Omega_{\ell j}(x,y)-\Omega_{\ell j}(x,x_o)|dx\le \C2^j|y-x_o|\prod_{i\in\I}2^{-\ell_i\frac{\N_i-1}{2}}\prod_{i\in\J}2^{-j\frac{\mathbf{N}_i-1}{2}},
    \eeq

Let us now estimate 
\[
\int_{\R^{\N_\J}}\int_{\R^{\N_\K}\setminus\Q^\K_{r\ell}(x_o)} |\Omega_{\ell j}^v(x,y)|dx
\]

Recall that $j_0$ is an integer such that 
\[
2^{-j_0}\le r<2^{-j_0+1}.
\]

Then there is a unit vector $^\mu\xi_{j_0}=\mathbf{L}_{\mu}\ ^{\mu}\eta_{j_0}$ such that for each $i\in\K$, we have

\bel{}
\begin{array}{lc}\ds
      \left|^v\xi_j^i-\ ^{\mu}\xi_{j_0}^i\right|\le 2^{-(j_0-\ell_i)/2} \qquad i\in \K,
\end{array}
\eeq













We have the inclusion 
\[
    ^c\mathbf{Q}^\K_{r \ell}(x_o)\subset \  ^c\left(\bigotimes_{i\in\K} \bigcup_{v\in \mathbf{V}_i}\  ^v\mathbf{R}_{j_0}^{i}(x_o^i)\right).
\]
 hence for $x\in  \ ^c\left(\bigotimes_{i\in\K}\bigcup_{v\in \mathbf{V}_i}\  ^v\mathbf{R}_{j_0}^{i}(x_o^i)\right) $, there exists $i\in\K$, such that 
 
\bel{}
\begin{array}{lc}\ds
   2^{j_0-\ell_i}\left| \left(^T\mathbf{L}^i_{\mu}x_o^i-\nabla_{\eta^i}\Phi(x^i,\mathbf{L}^i_{\mu} \ ^{\mu}\eta^i_{j_0}) \right)_1\right|+ 2^{(j_0-\ell_i)/2}\left| \left(^T\mathbf{L}^i_{\mu}x_o^i-\nabla_{\eta^i}\Phi(x^i,\mathbf{L}^i_{\mu} \ ^{\mu}\eta^i_{j_0}) \right)'\right|\ge 4\cdot 2^{\gamma \ell_M}, \ i\in \K
   \\ \ds 
\end{array}
\eeq











If $y\in B_r(x_o)$, then $|y-x_o|\le 2^{-j_0+1}$, we get as a consequence


\bel{}
\begin{array}{lc}\ds
   2^{j-\ell_i}\left| \left(^T\mathbf{L}^i_{\mu}x_o^i-\nabla_{\eta^i}\Phi(x^i,\mathbf{L}^i_{\mu} \ ^{\mu}\eta^i_{j_0}) \right)_1\right|+ 2^{(j-\ell_i)/2}\left| \left(^T\mathbf{L}^i_{\mu}x_o^i-\nabla_{\eta^i}\Phi(x^i,\mathbf{L}^i_{\mu} \ ^{\mu}\eta^i_{j_0}) \right)'\right|\ge 4\cdot 2^{j-j_0}2^{\gamma \ell_M}, \ i\in \K
   \\ \ds 
\end{array}
\eeq





















 
when $j\ge j_0$. Inserting this  in the bound $(\ref{Omega size})$ and arguing as before, we obtain




\bel{}
\begin{array}{cc}\ds
   \int_{\R^{\N_\J}}\int_{\R^{\N_\K}\setminus\Q^\K_{r \ell}(x_o)}  |\Omega_{\ell j}^v(x,y)|dx\le \C_N 2^{-j \frac{\mathbf{N}-n}{2}}\prod_{i\in\K}2^{j-\ell_i}2^{(j-\ell_i)\left(\frac{\mathbf{N}_i-1}{2}\right) } 
   2^{-j+j_0}2^{-\gamma \ell_M}
   \\\\ \ds
\int_{\R^{\N_\K }}\Bigg\{1+\sum_{i\in\K} \left(2^{j-\ell_i}\left|\left(x^i-y^i \right)_1\right| +2^{(j-\ell_i)/2}\left|\left(x^i-y^i \right)'\right|\right )
   \Bigg\}^{1-2N} dx^{\K}
\end{array}
\eeq





















If we choose $N$ so that $2N-1>\mathbf{N}$,  the result is 
\bel{}
   \int_{\R^{\N_\J}}\int_{\R^{\N_\K}\setminus\Q^\K_{r \ell}(x_o)}  |\Omega_{\ell j}^v(x,y)|dx\le \C 2^{-j \frac{\mathbf{N}-n}{2}}\frac{2^{-j}}{r}2^{-\gamma \ell_M}.
\eeq

Taking into account that there are essentially 
$\prod_{i\in \K}2^{(j-\ell_i)\frac{\mathbf{N}_i-1}{2}}$ terms involved, then we have 
\bel{}
   \int_{\R^{\N_\J}}\int_{\R^{\N_\K}\setminus\Q^\K_{r \ell}(x_o)}  
\left|\Omega_{\ell j}(x,y)\right|dx~\leq~\C~{2^{-j}\over r}~\prod_{i\in\I} 2^{-\ell_i{\N_i-1\over 2}}\prod_{i\in\J}2^{-j\frac{\mathbf{N}_i-1}{2}}2^{-\gamma \ell_M}.
\eeq










\section{General cases}
\setcounter{equation}{0}
Let $\I'$ be a nonempty subset of $\{1,2,\dots,n\}$, and 
\[
\N_i=1,\qquad i\in\I',\qquad \N_i\ge 2,\qquad i\in \{1,2,\dots,n\}\setminus\I'.
\]
The case $\I'=\{1,2,\dots,n\}$ is trivial, since then $\Phi_i(x^i,\xi^i)\sim x^i\cdot\xi^i$, and Fourier integral operators are essentially trivial modifications of pseudo-differential operators. Hnece {\bf Theorem One} is true, see \cite{Wang*}. It suffices to consider the case $\I\subset\I'  $. Recall the defintion of $j_0$ in (\ref{j_0}), let $~\epsilon>0~$ be a constant, which will be determined momentarily. Also, we need to change slightly the definition of region of influence.


Let $\epsilon>0$ and suppose $j_0\ge \epsilon\ell_M$.
The set $\mathbf{Q}^\K_{r\ell \epsilon}(x_o)$ is defined by
\bel{}
    \mathbf{Q}^\K_{r\ell\epsilon}(x_o)=\bigcup_{j\ge j_0-\epsilon\ell_M}\left\{\bigotimes_{i\in\K}\bigcup_{v\in \mathbf{V}_i}\  ^v\mathbf{R}_j^i(x_o^i)\right\}.
\eeq
Recall that there are at most  a constant multiple of $2^{(j-\ell_i)(\mathbf{N}_i-1)/2}$ elements in $\mathbf{V}_i$ for $i\in\K$ .  We have

\bel{Q est 2}
\begin{array}{lc}\ds
 |\mathbf{Q}^\K_{rl\epsilon}(x_o)| \le \sum_{j\ge j_0-\epsilon\ell_M
       }\prod_{i\in\K}\left\{\sum_{v\in \mathbf{V}_i}|^v\mathbf{R}_j^i(x_o^i)|\right\} 
       \\\\ \ds
  \le      \C ~ \sum_{j\ge j_0-\epsilon\ell_M}\prod_{i\in \K} 2^{{(j-\ell_i)}(\mathbf{N}_i-1)/2}2^{-(j-\ell_i)}2^{-(j-\ell_i)(\mathbf{N}_i-1)/2}2^{\gamma \mathbf{N}_i\ell_M}
\\\\ \ds
\le \C ~\sum_{j\ge j_0-\epsilon\ell_M} 2^{-|\K|j}\prod_{i\in \I}2^{\ell_i}2^{\gamma \N_\K\ell_M}
\\\\ \ds
\le \C~r^{|\K|}\prod_{i\in \I}2^{\ell_i}2^{(\epsilon|\K|+\gamma \N_\K)\ell_M}
\end{array}
\eeq








Now (\ref{Schwarz2}) becomes

\bel{Schwarz3}
\begin{array}{lc}\ds
  \ds
    \int_{\Q^\K_{r\ell\epsilon}}\int_{\R^{\N_\J}}|\F^\sharp_{\ell }a(x)|dx
   \le \C \prod_{i\in\I}2^{-\left[{\N-n\over 2 \N_\K}\right] \ell_i} 2^{{\ell_i\over2}} 2^{(\epsilon|\K|+\gamma \N_\K){\ell_M\over2}},
 \\\\   \ds
\end{array}
\eeq
Since we need to sum all $\ell$'s, In order the sum converge, we require

\bel{condition1}
\frac{\N-n}{\N_\K}-1-\epsilon|\K|-\gamma\N_\K>0 \qquad {\it for~~ all}\qquad \K.
\eeq

Let us now consider the "outside" region of influence.

{\bf Case one:~ $\N_\K\neq |\K|$.}

In this case, there must exist a $i\in\J$ such that  $\N_i\ge 2.~$
In worst case, {\bf Lemma two} would become 

\bel{Est13}
\int_{\R^\N} \left|\Omega_{\ell j}(x,y)\right| dx~\leq~\C
\eeq
\bel{Est23}
\int_{\R^\N}\left|\Omega_{\ell j}(x,y)-\Omega_{\ell j}(x,x_o)\right|dx~\leq~\C ~2^j|y-x_o|
\eeq
and
\bel{Est33}
\int_{\R^{\N_\J}}\int_{\R^{\N_\K}\setminus\Q^\K_{r\ell\epsilon}(x_o)}\left|\Omega_{\ell j}(x,y)\right|dx~\leq~\C_{\sigma~\Phi}~{2^{-j}\over r}~
2^{-(\epsilon+\gamma )\ell_M},\qquad y\in B_r(x_o)
\eeq
whenever $j>j_0-\epsilon \ell_M$.



Let
$0<h<1$ be a constant and sufficiently close to $0$.
Consider $j\le j_0-h\epsilon \ell_M$. 




By using (\ref{Est2 2}) and (\ref{F_t cancella}), we find
\bel{Norm Est12}
\begin{array}{lr}\ds
\int_{\R^n}\left|\int_{\R^n} a(y)\Omega_{\ell j}(x,y)dy\right|dx
~\leq~\int_{B_r(x_o)} |a(y)|\left\{\int_{\R^n} \left|\Omega_{\ell j}(x,y)-\Omega_{\ell j}(x,x_o)\right| dx\right\} dy
\\\\ \ds~~~~~~~~~~~~~~~~~~~~~~~~~~~~~~~~~~~~~~~~~~~~~~
~\leq~ \C~2^j|y-x_o|~
\\\\ \ds~~~~~~~~~~~~~~~~~~~~~~~~~~~~~~~~~~~~~~~~~~~~~~
~\leq~\C~2^j r~,\qquad y\in B_r(x_o).
\end{array}
\eeq

By summing over all regarding $\ell$ and $j$ s, we have
\bel{Sum12}
\begin{array}{lr}\ds
 \sum_{j\leq j_0-h\epsilon\ell_M} ~\prod_{i\in\I}\sum_{\ell_i}~
 \int_{\R^n}\left|\int_{\R^n} a(y)\Omega_{\ell j}(x,y)dy\right|dx
\\\\ \ds
~\leq~\C~ \prod_{i\in\I}\sum_{\ell_i}
\sum_{j\leq j_0-h\epsilon\ell_M} 2^jr 

\qquad
\hbox{\small{by (\ref{F_t cancella})-(\ref{Norm Est12})}}
\\\\ \ds
~\leq~\C ~\prod_{i\in\I}\sum_{\ell_i}2^{-h\epsilon\ell_M}
\\\\ \ds
~\leq~\C.
\end{array}
\eeq
For $j>j_0-h\epsilon \ell_M$,  (\ref{Est3}) implies
\bel{Norm Est22}
\begin{array}{lr}\ds
\int_{\R^{\N)\J}}\int_{\R^{\N_\K}\setminus\Q^\K_{r \ell\epsilon}(x_o)}\left|\int_{\R^\N} a(y)\Omega_{\ell j}(x,y)dy\right|dx
~\leq~\int_{B_r(x_o)} |a(y)|\left\{\int_{\R^{\N_\J}}\int_{\R^{\N_\K}\setminus\Q^\K_{r \ell\epsilon}(x_o)} \left|\Omega_{\ell j}(x,y)\right| dx\right\} dy
\\\\ \ds~~~~~~~~~~~~~~~~~~~~~~~~~~~~~~~~~~~~~~~~~~~~~~~~~~~~~~~
~\leq~\C ~{2^{-j}\over r}~
2^{-(\epsilon+\gamma )\ell_M} .
\end{array}
\eeq
By summing over all regarding $\ell$ and $j$ s, we have
\bel{Sum22}
\begin{array}{lr}\ds
\sum_{j>j_0-h\epsilon\ell_M }~\prod_{i\in\I}\sum_{\ell_i}~\int_{\R^{\N_\J}}\int_{\R^{\N_\K}\setminus\Q^\K_{r \ell \epsilon}(x_o)}\left|\int_{\R^\N} a(y)\Omega_{\ell j}(x,y)dy\right|dx
\\\\ \ds
~\leq~\C ~ \prod_{i\in\I}\sum_{\ell_i} \sum_{j>j_0-h\epsilon\ell_M} r^{-1}2^{-j}2^{-(\epsilon+\gamma)\ell_M}~

 \qquad
\hbox{\small{by (\ref{Norm Est22})}}
\\\\ \ds
~\leq~\C~ \prod_{i\in\I}\sum_{\ell_i}  2^{-((1-h)\epsilon+\gamma)\ell_M}~
\\\\ \ds
~\leq~\C,  \qquad
\hbox{\small{Provided that}}\qquad \epsilon+\gamma>0.
\\\\ \ds
\end{array}
\eeq

Solve the inequalities
\bel{}
\begin{array}{lc}
     0< \epsilon<1,
      \\\\ \ds
    \epsilon +\gamma>0,
    \\\\ \ds
    \epsilon|\K|+\gamma\N_\K<  \frac{\N-n}{\N_\K}-1 \qquad {\it for~~ all}\qquad \K.
\end{array}
\eeq

It suffices to verify that 
\[
\frac{1}{\N_\K-|\K|}\left(1-\frac{\N-n}{\N_\K}\right)<1,
\]

and this is true once we note that $\N_\K-|\K|\ge 1$.
We can choose $\epsilon=\epsilon_0$   and $\gamma=\gamma_0$  such that  the above inequalities are satisfied. 



Now consider the case $j_0\le \epsilon_0\ell_M$. We use the definition of region of influence in (\ref{region of influence}), with $\gamma$ to be determined, then we have 


\bel{Schwarz4}
\begin{array}{lc}\ds
  \ds
    \int_{\Q^\K_{r\ell}}\int_{\R^{\N_\J}}|\F^\sharp_{\ell }a(x)|dx
   \le \C \prod_{i\in\I}2^{-\left[{\N-n\over 2 \N_\K}\right] \ell_i} 2^{{\ell_i\over2}} 2^{\gamma \N_\K{\ell_M\over2}},
 \\\\   \ds
\end{array}
\eeq
Since we need to sum all $\ell$'s, In order the sum converge, we require

\bel{condition12}
\frac{\N-n}{\N_\K}-1-\gamma\N_\K>0 \qquad {\it for~~ all}\qquad \K.
\eeq


Let us now consider the "outside" region of influence. Similarly, applying {\bf Lemma two} 

For $j>j_0$,  (\ref{Est3}) implies
\bel{Norm Est24}
\begin{array}{lr}\ds
\int_{\R^{\N)\J}}\int_{\R^{\N_\K}\setminus\Q^\K_{r \ell}(x_o)}\left|\int_{\R^\N} a(y)\Omega_{\ell j}(x,y)dy\right|dx
~\leq~\int_{B_r(x_o)} |a(y)|\left\{\int_{\R^{\N_\J}}\int_{\R^{\N_\K}\setminus\Q^\K_{r \ell}(x_o)} \left|\Omega_{\ell j}(x,y)\right| dx\right\} dy
\\\\ \ds~~~~~~~~~~~~~~~~~~~~~~~~~~~~~~~~~~~~~~~~~~~~~~~~~~~~~~~
~\leq~\C~{2^{-j}\over r}~
2^{-\gamma \ell_M} .
\end{array}
\eeq
Note that if $~j>\ell_M~$, then $~j>\ell_M\ge j_0/\epsilon_0>j_0$. 


By summing over all regarding $\ell$ and $j$ s, we have
\bel{Sum24}
\begin{array}{lr}\ds
\sum_{j>0 }~\prod_{i\in\I}\sum_{\ell_i<j}~\int_{\R^{\N_\J}}\int_{\R^{\N_\K}\setminus\Q^\K_{r \ell }(x_o)}\left|\int_{\R^\N} a(y)\Omega_{\ell j}(x,y)dy\right|dx
\\\\ \ds
~\le ~\prod_{i\in\I}\sum_{\ell_i}\sum_{j>\ell_M}~\int_{\R^{\N_\J}}\int_{\R^{\N_\K}\setminus\Q^\K_{r \ell }(x_o)}\left|\int_{\R^\N} a(y)\Omega_{\ell j}(x,y)dy\right|dx
\\\\ \ds
~\leq~\C~ \prod_{i\in\I}\sum_{\ell_i} \sum_{j>\ell_M} r^{-1}2^{-j}2^{-\gamma\ell_M}~

 \qquad
\hbox{\small{by (\ref{Norm Est24})}}
\\\\ \ds
~\leq~\C~ \prod_{i\in\I}\sum_{\ell_i} 2^{\epsilon_0\ell_M}2^{-\ell_M} 2^{-\gamma\ell_M}~
\\\\ \ds
~\leq~\C.
\end{array}
\eeq
Provided that
\bel{condition22}
\epsilon_0-1-\gamma<0.
\eeq


Then (\ref{condition12}) and $\ref{condition22}$ is equivalent to 
\bel{condition23}
\epsilon_0<1+\frac{1}{\N_\K}\left(\frac{\N-n}{\N_\K}-1\right).
\eeq
Hence it suffices to 
Solve the inequalities
\bel{condition 24}
\begin{array}{lc}
     0< \epsilon_0<1+\frac{1}{\N_\K}\left(\frac{\N-n}{\N_\K}-1\right),
      \\\\ \ds
    \epsilon_0 +\gamma_0>0,
    \\\\ \ds
    \epsilon_0|\K|+\gamma_0\N_\K<  \frac{\N-n}{\N_\K}-1 \qquad {\it for~~ all}\qquad \K.
\end{array}
\eeq
Note that $~1+\frac{1}{\N_\K}\left(\frac{\N-n}{\N_\K}-1\right)>0~$, since $\N_K\ge 2$. Therefore the only constraint condition is
\bel{final condition}
\frac{1}{\N_\K-|\K|}\left(1-\frac{\N-n}{\N_\K}\right)<1+\frac{1}{\N_\K}\left(\frac{\N-n}{\N_\K}-1\right),
\eeq
 and this is also true .

{\bf Case two:~ $\N_\K=  |\K|$.}

In worst case, {\bf Lemma two} would become 

\bel{Est14}
\int_{\R^\N} \left|\Omega_{\ell j}(x,y)\right| dx~\leq~\C2^{-\frac{j}{2}}
\eeq
\bel{Est24}
\int_{\R^\N}\left|\Omega_{\ell j}(x,y)-\Omega_{\ell j}(x,x_o)\right|dx~\leq~\C ~2^{\frac{j}{2}}|y-x_o| 
\eeq
and
\bel{Est34}
\int_{\R^{\N_\J}}\int_{\R^{\N_\K}\setminus\Q^\K_{r\ell\epsilon}(x_o)}\left|\Omega_{\ell j}(x,y)\right|dx~\leq~\C_{\sigma~\Phi}~ {2^{-\frac{3}{2}j}\over r}~
2^{-(\epsilon+\gamma )\ell_M},\qquad y\in B_r(x_o)
\eeq
whenever $j>j_0-\epsilon \ell_M$.





Similar to case one, we have
\bel{}
\begin{array}{lr}\ds
\int_{\R^n}\left|\int_{\R^n} a(y)\Omega_{\ell j}(x,y)dy\right|dx
~\leq~\int_{B_r(x_o)} |a(y)|\left\{\int_{\R^n} \left|\Omega_{\ell j}(x,y)-\Omega_{\ell j}(x,x_o)\right| dx\right\} dy
\\\\ \ds~~~~~~~~~~~~~~~~~~~~~~~~~~~~~~~~~~~~~~~~~~~~~~
~\leq~\C~2^{\frac{j}{2}} r~,\qquad y\in B_r(x_o).
\end{array}
\eeq
and 
\bel{}
\begin{array}{lr}\ds
 \sum_{j\leq j_0-\epsilon\ell_M} ~\prod_{i\in\I}\sum_{\ell_i}~
 \int_{\R^n}\left|\int_{\R^n} a(y)\Omega_{\ell j}(x,y)dy\right|dx
\\\\ \ds
~\leq~\C~ \prod_{i\in\I}\sum_{\ell_i}
\sum_{j\leq j_0-\epsilon\ell_M} 2^{\frac{j}{2}}r 

~\leq~\C ~\prod_{i\in\I}\sum_{\ell_i}2^{-\epsilon\ell_M}
~\leq~\C.
\end{array}
\eeq
For $j>j_0-\epsilon \ell_M$,  we have 
\bel{}
\begin{array}{lr}\ds
\int_{\R^{\N)\J}}\int_{\R^{\N_\K}\setminus\Q^\K_{r \ell\epsilon}(x_o)}\left|\int_{\R^\N} a(y)\Omega_{\ell j}(x,y)dy\right|dx
~\leq~\int_{B_r(x_o)} |a(y)|\left\{\int_{\R^{\N_\J}}\int_{\R^{\N_\K}\setminus\Q^\K_{r \ell\epsilon}(x_o)} \left|\Omega_{\ell j}(x,y)\right| dx\right\} dy
\\\\ \ds~~~~~~~~~~~~~~~~~~~~~~~~~~~~~~~~~~~~~~~~~~~~~~~~~~~~~~~
~\leq~\C ~{2^{-\frac{3}{2}j}\over r}~
2^{-(\epsilon+\gamma )\ell_M} .
\end{array}
\eeq
By summing over all regarding $\ell$ and $j$ s, we have
\bel{}
\begin{array}{lr}\ds
\sum_{j>j_0-\epsilon\ell_M }~\prod_{i\in\I}\sum_{\ell_i}~\int_{\R^{\N_\J}}\int_{\R^{\N_\K}\setminus\Q^\K_{r \ell \epsilon}(x_o)}\left|\int_{\R^\N} a(y)\Omega_{\ell j}(x,y)dy\right|dx
\\\\ \ds
~\leq~\C ~ \prod_{i\in\I}\sum_{\ell_i} \sum_{j>j_0-\epsilon\ell_M} r^{-1}2^{-\frac{3}{2}j}2^{-(\epsilon+\gamma)\ell_M}~
~\leq~\C~ \prod_{i\in\I}\sum_{\ell_i}  2^{-(\frac{3}{2}\epsilon+\gamma)\ell_M}~
\\\\ \ds
~\leq~\C,  \qquad
\hbox{\small{Provided that}}\qquad  \frac{3}{2}\epsilon+\gamma>0.
\\\\ \ds
\end{array}
\eeq

Solve the inequalities
\bel{}
\begin{array}{lc}
      0<\epsilon<1
      \\\\ \ds
   3 \epsilon +2\gamma>0,
    \\\\ \ds
    \epsilon+\gamma <\frac{1}{|\K|} \left( \frac{\N-n}{|\K|}-1\right) \qquad {\it for~~ all}\qquad \K.
\end{array}
\eeq

It suffices to verify that 
\[
\frac{2}{|\K|}\left(1-\frac{\N-n}{|\K|}\right)<1,
\]

and this is true once we note that $|\K|\ge 2$.
We can choose $\epsilon=\epsilon_0$   and $\gamma=\gamma_0$  such that  the above inequalities are satisfied. 



Now consider the case $j_0\le \epsilon_0\ell_M$. We use the definition of region of influence in (\ref{region of influence}), with $\gamma$ to be determined, then we have 


\bel{}
\begin{array}{lc}\ds
  \ds
    \int_{\Q^\K_{r\ell}}\int_{\R^{\N_\J}}|\F^\sharp_{\ell }a(x)|dx
   \le \C \prod_{i\in\I}2^{-\left[{\N-n\over 2 \N_\K}\right] \ell_i} 2^{{\ell_i\over2}} 2^{\gamma \N_\K{\ell_M\over2}},
 \\\\   \ds
\end{array}
\eeq
Since we need to sum all $\ell$'s, In order the sum converge, we require

\bel{constraint f1}
\frac{\N-n}{\N_\K}-1-\gamma\N_\K>0 \qquad {\it for~~ all}\qquad \K.
\eeq


Let us now consider the "outside" region of influence. Similarly, applying {\bf Lemma two} 

For $j>j_0$,  
\bel{}
\begin{array}{lr}\ds
\int_{\R^{\N)\J}}\int_{\R^{\N_\K}\setminus\Q^\K_{r \ell}(x_o)}\left|\int_{\R^\N} a(y)\Omega_{\ell j}(x,y)dy\right|dx
~\leq~\int_{B_r(x_o)} |a(y)|\left\{\int_{\R^{\N_\J}}\int_{\R^{\N_\K}\setminus\Q^\K_{r \ell}(x_o)} \left|\Omega_{\ell j}(x,y)\right| dx\right\} dy
\\\\ \ds~~~~~~~~~~~~~~~~~~~~~~~~~~~~~~~~~~~~~~~~~~~~~~~~~~~~~~~
~\leq~\C~{2^{-\frac{3}{2}j}\over r}~
2^{-\gamma \ell_M} .
\end{array}
\eeq
By summing over all regarding $\ell$ and $j$ s, we have
\bel{constraint f2}
\begin{array}{lr}\ds
\sum_{j>0 }~\prod_{i\in\I}\sum_{\ell_i<j}~\int_{\R^{\N_\J}}\int_{\R^{\N_\K}\setminus\Q^\K_{r \ell }(x_o)}\left|\int_{\R^\N} a(y)\Omega_{\ell j}(x,y)dy\right|dx
\\\\ \ds
~\le ~\prod_{i\in\I}\sum_{\ell_i}\sum_{j>\ell_M}~\int_{\R^{\N_\J}}\int_{\R^{\N_\K}\setminus\Q^\K_{r \ell }(x_o)}\left|\int_{\R^\N} a(y)\Omega_{\ell j}(x,y)dy\right|dx
\\\\ \ds
~\leq~\C~ \prod_{i\in\I}\sum_{\ell_i} \sum_{j>\ell_M} r^{-1}2^{-\frac{3}{2}j}2^{-\gamma\ell_M}~
\\\\ \ds
~\leq~\C~ \prod_{i\in\I}\sum_{\ell_i} 2^{\epsilon_0\ell_M}2^{-\frac{3}{2}\ell_M} 2^{-\gamma\ell_M}~
\\\\ \ds
~\leq~\C.\qquad~~~~~~~~~~~~~~~~\textit{provided}~~~~~~ \epsilon_0-\frac{3}{2}-\gamma<0.
\end{array}
\eeq

Now (\ref{constraint f1}) and (\ref{constraint f2}) is equivalent to 
\[
\epsilon_0<\frac{3}{2}+\frac{1}{|\K|}\left(\frac{\N-n}{|\K|}-1\right).
\]
Hence it suffices to 
Solve the inequalities
\bel{condition 25}
\begin{array}{lc}
     0< \epsilon_0<\min\left\{ \frac{3}{2}+\frac{1}{|\K|}\left(\frac{\N-n}{|\K|}-1\right),1\right\},
      \\\\ \ds
    3\epsilon_0 +2\gamma_0>0,
    \\\\ \ds
    \epsilon_0+\gamma_0< \frac{1}{|\K|}\left( \frac{\N-n}{|\K|}-1\right) \qquad {\it for~~ all}\qquad \K.
\end{array}
\eeq
Note that $~\frac{3}{2}+\frac{1}{|\K|}\left(\frac{\N-n}{|\K|}-1\right)>0~$, since $|\K|\ge 2$. Therefore the only constraint condition is
\bel{final condition2}
\frac{2}{|\K|}\left(1-\frac{\N-n}{|\K|}\right)<\min\left\{ \frac{3}{2}+\frac{1}{|\K|}\left(\frac{\N-n}{|\K|}-1\right),1\right\}
\eeq
which is obvious since 
 \[
 \frac{2}{|\K|}\left(1-\frac{\N-n}{|\K|}\right)< 1 .
 \]











\begin{thebibliography}{100}
{\small \bibitem{S.S.S}A.~Seeger,~C.~D.~Sogge and E.~M.~Stein, {\it Regularity Properties of Fourier Integral Operators}, Annals of Mathematics, {\bf 134}: 231-251, 1991.}













{\small \bibitem{FC1} C.~Fefferman, {\it A note on Spherical Summation Multipliers}, Israel Journal of Mathematics, {\bf 15}: 44-52, 1973.}





{\small \bibitem{Fefferman} C.~Fefferman, {\it Characterizations of Bounded Mean Oscillation}, Bulletin of American Mathematical Society, {\bf 77}: 587-588, 1971.}


{\small \bibitem{FC.S}C.~Fefferman and E.~M.~Stein, {\it $\H^p$ Spaces of Several Variables}, Acta Mathematica, {\bf 129}: 137-193, 1972. }




{\small \bibitem{Stein} E.~M.~Stein, {\it
Harmonic Analysis: Real-Variable Methods, Orthogonality and Oscillatory Integrals},
 Princeton University Press, 1993.}






{\small \bibitem{Sogge}C.~D.~Sogge, {\it Fourier Integrals in Classical Analysis}, Cambridge Tracts in Mathematics, $\sharp$105, Cambridge University Press, 1993. }









{\small \bibitem{Hormander}L. H\"{o}rmander, {\it Fourier integral operators I}, Acta Mathematica {\bf 127}: 79-183, 1971.}




{\small \bibitem{Duistermaat-Hormander}J. J. Duistermaat and L. H\"{o}rmander, {\it Fourier integral operators II}, Acta Mathematica {\bf 122}: 183-269, 1972.}


{\small \bibitem{Eskin} G. I. Eskin, {\it Degenerate elliptic pseudo-differential operators of principal type} (Russian),
Mat. Sbornik {\bf 82}: (124) 585-628, 1970; English translation, Math. USSR Sbornik {\bf 11}: 
539-582, 1970.}


{\small \bibitem{Colin-Frisch}  Y. Colin de Verdi\'{e}re and M. Frisch, {\it R\'{e}gularit\'{e} Lipschitzienne et solutions de l'\'{e}quation des ondes sur une vari\'{e}t\'{e} Riemannienne compacte}, Ann. Scient. Ecole Norm. Sup. {\bf 9}: 539-565, 1976.}











{\small \bibitem{Brenner}P. Brenner, {\it $\L^p\mt\L^{p'}$-estimates for Fourier integral operators related to hyperbolic equations}, Math. Z. {\bf 152}: 273-286, 1977.}


{\small \bibitem{Peral}J. Peral, {\it $\L^p$-estimates for the wave equation}, Journal of Functional Analysis {\bf 36}: 114-145, 1980.}

{\small \bibitem{Miyachi}  A. Miyachi, {\it On some estimates for the wave equation in $\L^p$ and $\H^p$}, J. Fac. Sci. Tokyo, Sci. IA {\bf 27}: 331-354, 1980.}


{\small \bibitem{Beals}M. Beals, {\it $\L^p$-Boundedness of Fourier Integral Operators}, Mem. Amer. Math. Soc. {\bf 264}: 1982.}





{\small \bibitem{Hardy-Littlewood}
G.~H.~Hardy and J.~E.~Littlewood, {\it Some Properties of Fractional Integrals}, 
Mathematische Zeitschrift {\bf 27}: 565-606, 1928.}






{\small \bibitem{Sobolev}
S.~L.~Sobolev, {\it On a Theorem of Functional Analysis}, Matematicheskii Sbornik {\bf 46}: 471-497, 1938.}







{\small \bibitem{R.Fefferman}R.~Fefferman, {\it Harmonic Analysis on Product Spaces}, Annals of Mathematics {\bf 126}: 109-130, 1987.}



{\small \bibitem{R-F.S}R.~Fefferman and E.~M.~Stein, {\it Singular Integrals on Product Spaces},
Advances in Mathematics {\bf 45}:117-143, 1982.}




{\small \bibitem{Cordoba-Fefferman}A.~C\'{o}rdoba and R.~Fefferman, {\it A geometric Proof of the Strong Maximal Theorem}, Annals of Mathematics {\bf 102}: 95-100, 1975.}










{\small \bibitem{Chang-Fefferman}S.~Y.~A.~Chang and R.~Fefferman, {\it The Colder\'{o}n-Zygmund Decomposition on Product Domains}, American Journal of Mathematics {\bf 104}: 455-468, 1982.}

{\small \bibitem{M.R.S}D. M\"{u}ller,~~F.~Ricci,~~E.~M.~Stein,  
{\it Marcinkiewicz Multipliers and Multi-parameter structures on Heisenberg (-type) group, I}, 
Inventiones Mathematicae {\bf 119}: 199-233, 1995.}





{\small \bibitem{Wang} Z.~Wang,~ {\it Stein-Weiss inequality on product spaces}, Revista Matem\'{a}tica Iberoamericana {\bf 37}: no.5, 1641-1667, 2021.}

\bibitem{Wang*}Z.~Wang,~ {\it Singular integrals of nonconvolution type on product spaces}, Ph.D thesis, 2015.

\bibitem{Wang 2}Z.~Wang,  ~{\it Regularity of multi-parameter Fourier integral operator}, arXiv 2007.02262, 2022.





\end{thebibliography}
{\small Department of Mathematics, Westlake University}


 {\small email: chengjinhua@westlake.edu.cn}
 


{\small Department of Mathematics, Westlake University}


 {\small email: wangzipeng@westlake.edu.cn}













\end{document}







