\documentclass[12pt]{amsart}

\usepackage{graphicx,amsthm}
\usepackage{enumerate}
\usepackage{empheq}
\usepackage{amssymb}
\usepackage[margin=1in]{geometry}
\usepackage{pst-all}
\usepackage{array}
\usepackage{tikz}
\usepackage{tikz-cd}
\usepackage{amsmath,amscd,amsfonts,bbm,mathabx,graphicx}
\allowdisplaybreaks
\usepackage[onehalfspacing]{setspace}

\usepackage[all]{xy} 

\usepackage{caption}
\usepackage{subcaption}


\usepackage[mathscr]{euscript}
\usepackage{booktabs} 
\usepackage{ amssymb }
\usepackage{hyperref}

%%%%%%%%%%%%%%%%%%%%%%%%%%%%%%%%%%%%%%%%%%%%%%%%%%%%%%%%%%%%
%  Environments
%%%%%%%%%%%%%%%%%%%%%%%%%%%%%%%%%%%%%%%%%%%%%%%%%%%%%%%%%%%%

\newtheorem{theorem}{Theorem}[section]
\newtheorem{lemma}[theorem]{Lemma}
\newtheorem{proposition}[theorem]{Proposition}
\newtheorem{corollary}[theorem]{Corollary}
\newtheorem{conjecture}[theorem]{Conjecture}

\theoremstyle{definition}
\newtheorem{definition}[theorem]{Definition}
\newtheorem{example}[theorem]{Example}

%\theoremstyle{remark}
\newtheorem{remark}[theorem]{Remark}

\numberwithin{equation}{section}

\newtheorem{thmy}{Theorem}
\renewcommand{\thethmy}{\Alph{thmy}} % "letter-numbered" theorems
\newenvironment{thmx}{\stepcounter{theorem}\begin{thmy}}{\end{thmy}}

%%%%%%%%%%%%%%%%%%%%%%%%%%%%%%%%%%%%%%%%%%%%%%%%%%%%%%%%%%%%

\newcommand{\Q}{\mathbb{Q}}
\newcommand{\C}{\mathbb{C}}
\newcommand{\Z}{\mathbb{Z}}
\newcommand{\R}{\mathbb{R}}
\DeclareMathOperator{\GL}{GL}
\newcommand{\Cstar}{\mathbb{C}^{\ast}}
\newcommand{\flag}[1]{\mathcal{F}\ell(\mathbb{C}^{#1})} %%
\newcommand{\Xwo}[1]{X_{#1}^{\circ}}
\newcommand{\Xw}[1]{X_{#1}}
\newcommand{\Owo}[1]{\Omega_{#1}^{\circ}}
\newcommand{\Ow}[1]{\Omega_{#1}}
\newcommand{\Owho}[1]{\Omega_{#1,h}^{\circ}}
\newcommand{\Owh}[1]{\Omega_{#1,h}}
\newcommand{\Awh}[1]{A_{#1,h}}
\newcommand{\swh}[1]{\sigma_{#1,h}}
\DeclareMathOperator{\Hess}{Hess}
\DeclareMathOperator{\Hom}{Hom}
\newcommand{\Gh}{\Gamma_h}
\DeclareMathOperator{\codim}{\codim}
\newcommand{\Sym}{\mathcal{S}}
\DeclareMathOperator{\supp}{supp}
\newcommand{\Sn}[1]{\mathfrak{S}_{#1}} %%
\newcommand{\ve}{\mathbf{e}}
\newcommand{\hpat}[1]{\overline{#1}} %% Patterns

\newcommand{\uni}{\underline{i}}
\newcommand{\Gwh}{G_{w,h}}
\newcommand{\Ghw}{G_{w,h}}	
\newcommand{\vPQi}{w_{P,Q}^{(i)}}
\newcommand{\tvPQi}{\widetilde{w}_{P,Q}^{(i)}}
\newcommand{\tsvPQi}{\widetilde{\sigma}_{P,Q}^{(i)}}
\allowdisplaybreaks

\newcommand{\La}{\Lambda_k}
\newcommand{\hats}{\widehat{\sigma}}

%%%%%%%%%%%%%%%%%%%%%%%%%%%%%%%%%%%%
% For comments
%%%%%%%%%%%%%%%%%%%%%%%%%%%%%%%%%%%%
\definecolor{cadmiumgreen}{rgb}{0.0, 0.42, 0.24}
\newcommand{\sj}[1]{\textcolor{cadmiumgreen}{#1}}
	
%%%%%%%%%%%%%%%%%%%%%%%%%%%%%%%%%%%%
%
%  Little LaTeX tricks for editing: ``To Do'' and ``Fix'' as well as \comment
%     --- from Peter Garfield, 16 Jan 2004, and Yael Karshon for \comment
%
%%%%%%%%%%%%%%%%%%%%%%%%%%%%%%%%%%%%


%\textwidth=125mm
%\textheight=185mm
%\parindent=8mm
%\evensidemargin=0pt
%\oddsidemargin=0pt
%\frenchspacing


%%%%%%%%%%%%%%%%%%%%%%%%%%%%%%%%%%%%%%%%%%%%%%%%%%%%%%%%%%%%
%  MACROS for this particular document
%%%%%%%%%%%%%%%%%%%%%%%%%%%%%%%%%%%%%%%%%%%%%%%%%%%%%%%%%%%%


\newcommand{\la}{\lambda}
\newcommand{\x}{\mathbf{x}}
\renewcommand{\a}{\mathbf{a}}
\renewcommand{\b}[1]{\mathbf{#1}}





\begin{document}

\title[Pattern avoidance and smoothness of Hessenberg Schubert varieties]{Pattern avoidance and smoothness of Hessenberg Schubert varieties}
%\thanks{}




\author{Soojin Cho}
\address{Department of Mathematics, Ajou University, Suwon  16499, Republic of Korea}
\email{chosj@ajou.ac.kr}

\author{JiSun Huh}
\address{Department of Mathematics, Ajou University, Suwon  16499, Republic of Korea}
\email{hyunyjia@ajou.ac.kr}

\author{Seonjeong Park}
\address{Department of Mathematics Education, Jeonju University, Jeonju 55069, Republic of Korea}
\email{seonjeongpark@jj.ac.kr}

\thanks{This work was supported by the National Research Foundation of Korea [NRF-2020R1A2C1A01011045]}

\begin{abstract}  
A \emph{Hessenberg Schubert variety} is the closure of a Schubert cell inside a given Hessenberg variety.  
We consider the smoothness of Hessenberg Schubert varieties of regular semisimple Hessenberg varieties of type $A$ in this paper. 
We use known combinatorial characterizations of torus fixed points of a Hessenberg Schubert variety $\Omega_{w, h}$ to find a necessary condition for $\Omega_{w, h}$  to be smooth, in terms of the pattern avoidance of the permutation $w$. First, we show useful theorems regarding the structure of the subposet of the Bruhat order induced by the torus fixed points of $\Omega_{w, h}$. Then we apply them to prove that 
the regularity of the associated graph, which is known to be a necessary condition for the smoothness of $\Omega_{w, h}$, is completely characterized by the avoidance of the patterns we found.  
\end{abstract}

\keywords{Hessenberg variety, Hessenberg Schubert variety, smoothness, pattern avoidance, GKM graph, Bruhat order, $h$-Bruhat order}

\subjclass[2010]{Primary 14M15, 05E14; Secondary 14L30, 57S12}
\maketitle

\setcounter{tocdepth}{1}
%\tableofcontents


%%%%%%%%%%%%%%%%%%%%%%%%%%%%%%%%%%%
\section{Introduction} \label{sec:intro}

Schubert varieties of the full flag varieties form an important class of complex projective varieties that appear in many areas, including algebraic geometry, representation theory, and combinatorics. Schubert varieties of the full flag variety of type $A_{n-1}$ are subvarieties of $\GL_n(\mathbb C)/B$, where $B$ is the Borel subgroup of upper triangular matrices in $\GL_n(\mathbb C)$. They are indexed by the permutations in the symmetric group $\mathfrak{S}_n$, the Weyl group of $\GL_n(\mathbb C)$, and defined as the closure of the Schubert cell; $X_w\coloneqq \overline{BwB/B}$ for $w\in \mathfrak S_n$. The closure of the opposite Schubert cell $\Omega_w^\circ\coloneqq B^-wB/B$ is the opposite Schubert variety $\Omega_w$ of $\GL_n(\mathbb C)/B$, where $B^-$ is the subgroup of lower triangular matrices in $\GL_n(\mathbb C)$.
The characterization of the smoothness of $X_w$ (hence of $\Omega_w$) was done in different ways; in terms of pattern avoidance, the regularity of the Bruhat subgraph, the Poincar\'e polynomial of the cohomology, and Kazhdan--Lusztig polynomial; see Section 13.2 in \cite{BL}. We write some of them, which are relevant to our work, in the following theorem. Note that a maximal torus $T$ contained in $B$ acts on the flag variety $\GL_n(\mathbb{C})/B$ as left multiplication and the opposite Schubert varieties are $T$-invariant subvarieties of $\GL_n(\mathbb{C})/B$.

\begin{theorem}\label{thm:known_smoothness} The following statements are equivalent:
\begin{enumerate}
\item The opposite Schubert variety $\Omega_w$ is smooth.
\item The subgraph of the Bruhat graph of $\mathfrak S_n$ (equivalently, the GKM graph of $\GL_n(\mathbb C)/B$) induced by the set $\Omega_w^T$ of torus fixed points is a regular graph.
\item The permutation $w$ avoids patterns $2143$ and $1324$.
\end{enumerate}
\end{theorem}



Hessenberg varieties are subvarieties of the full flag variety, which were introduced by De Mari, Shayman, and Proceci \cite{DPS} in the 1990s. 
A diagonal matrix $S$ with distinct eigenvalues and a nondecreasing function $h\colon \{1, \dots, n\} \to \{1, \dots, n\}$ with $h(i)\geq i$ for $i=1, \dots, n$, defines a regular semisimple Hessenberg variety $\Hess(S, h)$ of type $A$. Owing to their interesting characteristics, Hessenberg varieties have become one of the central objects of research in the related areas, including combinatorics and algebraic geometry. A surprising instance is that the well-known Stanley--Stembridge conjecture in algebraic combinatorics is shown to be equivalent to a conjecture on the $\mathfrak{S}_n$-module structure of the cohomology of the corresponding  Hessenberg variety $\Hess(S, h)$ \cite{BC,G-P}. In this context, Cho, Hong, and Lee \cite{CHL}, considered the minus cell decomposition (Bia{\l}ynicki--Birula decomposition) $\bigsqcup_{w \in \mathfrak{S}_n}\Omega_{w, h}^\circ$ of $\Hess(S, h)$ and the cohomology class $\sigma_{w, h}$ of $\Omega_{w, h}\coloneqq \overline{\Omega_{w, h}^\circ}$ to investigate the $\mathfrak{S}_n$-module structure of the cohomology space, where the closure $\Omega_{w, h}$ is called an \emph{opposite Hessenberg Schubert variety}. In the same paper, they showed that the support of the class $\sigma_{w, h}$ is the set  $\Omega_{w, h}^T$  of torus fixed points of the Hessenberg Schubert variety and provided an explicit combinatorial description of the torus fixed points. In the Hessenberg variety $\Hess(S, h)$, the permutations are partitioned according to the Hessenberg function $h$, and there is a set of natural representatives of the partition, called the \emph{generators} for the Hessenberg function $h$; see  Section~3 of \cite{CHL2}.
Notably, Hessenberg Schubert varieties corresponding to the permutations in the same part resemble each other. For example, all of them have the same number of $T$-fixed points, where the action of the torus $T$ on a Hessenberg Schubert variety is induced from that of $T$ on the flag variety~\cite{HP}.

In this paper, we aim to extend the results (in Theorem~\ref{thm:known_smoothness}) regarding the characterization of the smoothness of Schubert varieties $\Omega_w$ to Hessenberg Schubert varieties $\Omega_{w, h}$. The Hessenberg variety $\Hess(S, h)$ is known to be a GKM variety and its GKM graph (denoted by $\Gamma_h$) is a subgraph of the GKM graph of the full flag variety, with the same set $\mathfrak S_n$ of vertices~\cite{DPS}. This defines a partial order on $\mathfrak S_n$, which we call the \emph{$h$-Bruhat order}. We first 
interpret the set $\Omega_{w, h}^T$, which has been shown in \cite{HP} as the interval $[w, w_0]$ in the Bruhat order for a generator~$w$, in terms of the $h$-Bruhat order. We use $[n]$ for the set $\{1, 2, \dots, n\}$.


\begin{thmx}[Theorem~\ref{thm:interval}]\label{thmx:1}
Let $w\in \mathfrak{S}_n$ be a generator for a given Hessenberg function $h\colon [n] \to  [n]$. Then, $\Omega_{w, h}^T=[w, w_0]_h$, the interval between $w$ and $w_0$ in the $h$-Bruhat order. 
\end{thmx}
 
 Let $\Gamma_{w,h}$ be the subgraph of $\Gamma_h$ induced by the set $\Omega_{w, h}^T$ of torus fixed points.  In the following theorem, we show that for a generator $w$, the degree of the vertices in $\Gamma_{w,h}$ is nondecreasing with respect to the $h$-Bruhat order. This theorem is essential for characterizing of the smoothness of $\Omega_{w, h}$.  
For two permutations $u$ and $v$, we use $u\preceq_h v$ to denote the $h$-Bruhat order relation.

 \begin{thmx}[Theorem~\ref{thm:increasing}]\label{thmx:2}
For a Hessenberg function $h$ and a generator $w\in \mathfrak S_n$, if $u$ and $v$ are vertices in $\Gamma_{w,h}$ with $u\preceq_h v$, then $\deg(u)\leq \deg(v)$.
\end{thmx}


We then consider the regularity of $\Gamma_{w,h}$ in the relation with patterns in the permutation $w$, where the patterns depend upon the given Hessenberg functions $h$. We found all the patterns that $w$ must avoid for $\Gamma_{w,h}$ to be regular as stated in the following theorem. The patterns are defined in Definitions~\ref{def:pattern4} and~\ref{def:pattern5} for the definition of the patterns.  

\begin{thmx}[Theorems \ref{thm:irregular}, \ref{thm:regular}  and \ref{thm:main}]\label{thmx:3}
Let $h\colon [n] \to [n]$ be a Hessenberg function.

\begin{enumerate}
    \item If $w\in \mathfrak S_n$ is a generator for $h$, then the graph $\Gamma_{w,h}$ is regular if and only if $w$ avoids all the associated patterns $\hpat{2143}$, $\hpat{1324}$, $\hpat{1243}$, $\hpat{2134}$, $\hpat{1423}$, $\hpat{2314}$, and $\hpat{2413}$. 

    \item Let $w\in \mathfrak S_n$ be a permutation. Then the graph $\Gamma_{w,h}$ is regular if and only if $w$ avoids all the associated patterns $\hpat{2143}$, $\hpat{1324}$, $\hpat{1243}$, $\hpat{2134}$, $\hpat{1423}$, $\hpat{2314}$, $\hpat{25314}$, $\hpat{24315}$, $\hpat{14325}$, and $\hpat{15324}$. 
\end{enumerate}
\end{thmx}

Because the smoothness of $\Omega_{w, h}$ implies the regularity of the subgraph $\Gamma_{w, h}$, as stated in  Proposition~\ref {prop:Hessenberg Schubert variety}, our result provides a necessary condition for the smoothness of $\Omega_{w, h}$ in terms of pattern avoidance. 

\begin{thmx}[Theorems~\ref{thm:not-smooth_generator} and \ref{thm:not-smooth}]\label{thmx:4}
Let $h\colon [n] \to [n]$ be a Hessenberg function.
\begin{enumerate}
    \item If $w\in \mathfrak S_n$, a generator for $h$, contains one of the seven patterns in $\{ \hpat{2143}$, $\hpat{1324}$, $\hpat{1243}$, $\hpat{2134}$, $\hpat{1423}$, $\hpat{2314}$, $\hpat{2413}\}$, then  $\Omega_{w, h}$ is not smooth.
    \item If a permutation $w\in \mathfrak S_n$ contains one of the ten patterns in  $\{ \hpat{2143}$, $\hpat{1324}$, $\hpat{1243}$, $\hpat{2134}$, $\hpat{1423}$, $\hpat{2314}$, $\hpat{25314}$, $\hpat{24315}$, $\hpat{14325}$,  $\hpat{15324}\}$, then  $\Omega_{w, h}$ is not smooth.
\end{enumerate}

\end{thmx}

We believe that the pattern avoidance condition we found is also an equivalent condition for the smoothness of 
$\Ow{w,h}$ (see Conjecture~\ref{conj:equivalent}).

 
 This paper is organized as follows. In Section~\ref{sec:2}, first, we set up notation and terminology, and then review some basic theories on Schubert varieties and Hessenberg Schubert varieties. In Section~\ref{sec:GKM}, we focus on the GKM graph $\Gamma_h$ of a Hessenberg variety and its subgraph $\Gamma_{w, h}$. The set of vertices of $\Gamma_{w, h}$ is characterized in terms of the $h$-Bruhat order, and a nice injective map from the set of edges of a vertex to the set of edges of an incident vertex is introduced. Further, seven patterns that generator permutations must avoid are presented such that the corresponding Hessenberg Schubert variety is smooth. In Section~\ref{sec:regular}, we attempt to prove that avoiding the seven patterns of length~$4$ for generators and ten patterns of length $4$ or~$5$ for an arbitrary permutation is enough to guarantee that the corresponding graph $\Gamma_{w, h}$  is regular.


%%%%%%%%%%%%%%%%%%%%%%%%%%%%%%%%%%%%%%%%%%%%%%%%%%%%%%%%%%%%%%
\section{Preliminaries}\label{sec:2}

%\input{2.tex}

\subsection{Basic terminologies and properties.}
Let $\mathfrak{S}_n$ be a symmetric group on $[n]$. For a permutation $w\in\mathfrak{S}_n$, we use the one-line notation 
$w=w(1)w(2)\cdots w(n).$ For $1\leq i<j\leq n$, the permutation that acts on $[n]$ by swapping $i$ and $j$ is called a \emph{transposition}, and it is denoted by~$(i,j)$. The \emph{Bruhat order} on $\mathfrak{S}_n$ is the transitive closure of the relation
$$u\prec v \quad \text{ if and only if } \quad v=u(i,j) \text{ for some $(i,  j)$ and } \ell(u)<\ell(v)\,.$$
The \emph{length} of $w$ is defined by the number of inversions and denoted by $\ell(w)$, i.e.,
$\ell(w)=\left|\{ (i,j)\mid w(i,j) \prec w \}  \right|$.
Then the poset $(\Sn{n},\prec)$ is a graded poset whose rank function is given by the length of a permutation.
Notably, $(\Sn{n},\prec)$ has the unique minimal element $e = 1\, 2\,\cdots\,n$ and the unique maximal element $w_0=n\,(n-1)\,\cdots\,1.$  The \emph{Bruhat interval} $[v,w]$ is the subposet of $(\Sn{n},\prec)$ defined by $[v,w]=\{u\in \Sn{n}\mid v\preceq u\preceq w\}$.
The \emph{Bruhat graph} for $\Sn{n}$ is the graph with the vertex set $\Sn{n}$ and edges $\{v,w\}$ if $w=u(i,j)$ for some $(i,j)$. 

For $w \in \Sn{n}$ and $p \in \Sn{k}$ with $k \leq n$, we say that the permutation $w$ \emph{contains} the pattern $p$ if a sequence $1 \leq i_1 < \cdots < i_k \leq n$ exists such that $w(i_1)\cdots w(i_k)$ has the same relative order with $p(1)\cdots p(k)$. If $w$ does not contain $p$, then we say that $w$ \emph{avoids} $p$ or is \emph{$p$-avoiding}. 

For positive integers $k_1\leq k_2$, we denote the set $\{ k_1, k_1+1, \dots, k_2 \}$ by $[k_1, k_2]$.
For a permutation $w\in \mathfrak{S}_n$,
we let $w[k_1,k_2]$ be the set $\{w(i)\mid k_1\leq i \leq k_2\}$.
We use $A\!\!\uparrow$ to represent the increasing arrangement $\{a_1 < a_2 < \cdots < a_k \}$ of the elements in $A$, where $A$ is a set of $k$ integers. 
For two finite sets $A$ and $B$ of integers of the same cardinality, $A\!\!\uparrow\leq B\!\!\uparrow$ means $a_i\leq b_i$ for all $i$, where $a_i$ and $b_i$ are the $i$th elements of $A\!\!\uparrow$ and $B\!\!\uparrow$, respectively.

There are many equivalent conditions to $u\prec v$ in the Bruhat order; we give some of them that we use in this paper.
We refer {to} the book~\cite{BB} for more general theories on (Bruhat orders of) Coxeter groups.  

\begin{proposition}\label{prop:Bruhat order}
    For $u, v\in \mathfrak{S}_n$, the following statements are equivalent:
    \begin{enumerate}
        \item $u\preceq v$ in the Bruhat order.
        \item $u[k]\!\!\uparrow\leq v[k]\!\!\uparrow$ for all $k\in [n]$.\item $u[k]\!\!\uparrow\leq v[k]\!\!\uparrow$ for all $k\in [n-1]-D(v)$, where $D(v)\coloneqq \{ i\in [n-1]\mid v(i)>v(i+1)\}$ is the set of \emph{descents} of $v$.
    \end{enumerate}
\end{proposition}

\begin{lemma}\cite[Lemma 2.1]{Bil}\label{lemma:Bruhat order}
    Suppose two permutations $u$ and $v$ agree everywhere except on positions $i_1<\cdots<i_k$. Then $u\preceq v$ in the Bruhat order if and only if $\{ u(i_1), \dots, u(i_j)\}\!\!\uparrow \leq \{ v(i_1), \dots, v(i_j)\}\!\!\uparrow $ for all $j=1, \dots, k$. In particular, 
    $u\prec u(i,j)$ if and only if $u(i)<u(j)$.
\end{lemma}

The Bruhat order on $\Sn{n}$ satisfies the chain property as follows.

\begin{proposition}[Chain Property]\label{prop:chain}
    If $u\prec v$ in $\mathfrak S_n$, then there exist elements $v_i\in \mathfrak S_n$, satisfying $\ell(v_i)=\ell(u)+i$ for $0\leq i\leq k$, and $u=v_0\prec v_1\prec \cdots \prec v_k=v$.
\end{proposition}

%---------------------------------------
\subsection{Flag varieties, Schubert varieties, and Opposite Schubert varieties.} 

The flag variety $\flag{n}$ is the homogeneous space $\GL_n(\mathbb{C})/B$, where $B$ is the Borel subgroup of upper triangular matrices in $\GL_n(\mathbb{C})$. Then $\flag{n}$ is a smooth projective variety of (complex) dimension $\binom{n}{2}$, and it can be identified with the set
\[\flag{n}=\{(\{0\}\subset V_1 \subset V_2\subset\cdots \subset V_n=\C^n)\mid \dim_{\C}V_i=i\text{ for }i=1,\dots,n\}\]
of chains of subspaces of $\C^n$. Each element of $\flag{n}$ is called a flag.
For instance, an element $w \in \Sn{n}$ defines a flag given by
$$(\{0\}\subset\langle \ve_{w(1)}\rangle \subset \langle \ve_{w(1)},\ve_{w(2)}\rangle\subset\cdots\subset V_n =\C^n),$$
where $\ve_1,\dots,\ve_n$ are the standard basis vectors in $\C^n$. We denote by $wB$ this standard coordinate flag.


Let $G=\GL_n(\C)$ and $B^-$ be the Borel subgroup of lower triangular matrices in $G$.
Then, the left action of $B$ (respectively, $B^-$) on $G/B$ has finitely many orbits $BwB/B$ (respectively, $B^- wB/B$), where $w$ is a permutation in $\mathfrak{S}_n$, and we get a cell decomposition called the \emph{Bruhat decomposition}:
\begin{equation}\label{eq:flag-decomposition}
    G/B= \bigsqcup_{w \in \mathfrak{S}_n} BwB/B\,= \bigsqcup_{w \in \mathfrak{S}_n} B^- wB/B\,.
\end{equation}
For each $w\in\mathfrak{S}_n$, the cells $BwB/B$ and $B^-wB/B$ are isomorphic to $\C^{\ell(w)}$ and $\C^{\binom{n}{2}-\ell(w)}$, respectively. We call $\Xwo{w}\coloneqq BwB/B$ the \emph{Schubert cell} and $\Owo{w}\coloneqq B^-wB/B$ the \emph{opposite Schubert cell} indexed by~$w$. The (Zariski) closures  $\Xw{w}\coloneqq \overline{\Xwo{w}}$ and $\Ow{w}\coloneqq \overline{\Owo{w}}$ are called the \emph{Schubert variety} and the \emph{opposite Schubert variety} indexed by~$w$, respectively. Notably,  $\Ow{w}=w_0\Xw{w_0w}$ because $B^{-}=w_0Bw_0$, and we only consider opposite Schubert varieties throughout the paper.

For $v,w\in\Sn{n}$, we have
\[v\prec w\quad \text{ if and only if }\quad \Ow{w} \subset \Ow{v},\]
and we obtain the cell decomposition of $\Ow{w}$ as follows:
\begin{equation}\label{eq:Schubert-decomposition}
    \Ow{w}= \bigsqcup_{v\succeq w} \Owo{v}\,.
\end{equation}
An opposite Schubert variety is not necessarily smooth. It is known from \cite{LSa} that $\Ow{w}$ is smooth if and only if $w$ avoids patterns $2143$ and $1324$.

We refer a reader to \cite{Fulton} and references therein for the geometry and combinatorics related to flag varieties and their subvarieties.

\subsection{Torus actions and GKM varieties.}
Let $X$ be a complex projective variety with an action of algebraic torus $T\cong (\C^\ast)^n$. Then, the variety $X$ is called a \emph{GKM (Goresky--Kottwitz--MacPherson) variety} if it satisfies the following three conditions:
\begin{itemize}
    \item The fixed point set $X^T$ consists of isolated points.
    \item There are finitely many one-dimensional orbits of $T$ on $X$.
    \item The space $X$ is equivariantly formal with respect to the action of $T$.%, that is, $H^\ast_T(X;\Q)$ is a free $H^\ast_T(BT;\Q)$-module.
\end{itemize}
Note that the equivariantly formal condition is quite technical, but \eqref{eq:flag-decomposition} and \eqref{eq:Schubert-decomposition} imply that $\flag{n}$ and $\Ow{w}$ are equivariantly formal with respect to every torus action. We refer a reader to \cite{GKM, GZ, GT} for the GKM theory.

For a GKM variety $X$, the boundary of each one-dimensional $T$-orbit contains two $T$-fixed points and the closure of each one-dimensional orbit is isomorphic to $\C P^1$ with fixed points at the north and south poles. Based on the zero- and one-dimensional orbits of a GKM variety, we construct the GKM graph $\Gamma$ as follows:
\begin{enumerate}
    \item the vertex set of $\Gamma$ is identified with the set $X^T$ of $T$-fixed points; and
    \item two vertices $p$ and $q$ of $X^T$ are connected by an edge if there exists a one-dimensional orbit whose closure has $p$ and $q$ as the $T$-fixed points.
\end{enumerate}
Let $E_\Gamma$ be the set of directed edges of $\Gamma$.\footnote{Note that $E_\Gamma$ is not the set of edges of $\Gamma$; 
the cardinality of $E_\Gamma$ is twice that of the edge set.} The \emph{axial function} of a GKM graph $\Gamma$ is a map $\alpha\colon E_\Gamma\to \mathfrak{t}^\ast$ given by $\alpha(e)$ is the $T$-weight on the corresponding one-dimensional orbit, where $\mathfrak{t}^\ast$ is the dual of the Lie algebra $\mathfrak{t}$ of $T$.
When $X$ is a smooth manifold of (complex) dimension $d$, the GKM graph of $X$ is a $d$-valent graph and the tangent space $T_pX$ for $p\in X^T$ is isomorphic to $\C(\alpha(e_1))\oplus\cdots\oplus\C(\alpha(e_d))$ as a $T$-representation, where $e_1,\dots,e_d$ are the edges emanating from $p$ in $E_\Gamma$.

Now, we let $T$ be the set of diagonal matrices in $G$. Then $T$ is isomorphic to $(\C^\ast)^n$ and acts on $G/B$ by left multiplication, and the set of $T$-fixed points of $G/B$ consists of the standard coordinate flags. Thus, we identify $(G/B)^T$ with $\Sn{n}$. Furthermore, the flag variety $\flag{n}$ with the action of $T$ becomes a GKM variety, and the GKM graph $\Gamma$ of $\flag{n}$ is the Bruhat graph for $\Sn{n}$.

For every $w\in \mathfrak{S}_n$, the opposite Schubert variety $\Ow{w}$ is a GKM subvariety of $\flag{n}$. The set of $T$-fixed points of $\Ow{w}$ is identified with the Bruhat interval $[w,w_0]$ and the GKM graph of $\Ow{w}$ is the subgraph of $\Gamma$ induced by $[w,w_0]$.

%----------------
\subsection{Hessenberg-Schubert varieties.}
  
A \emph{Hessenberg function} is a nondecreasing function $h\colon [n] \to [n]$ satisfying $h(i)\geq i$ for each $i\in [n]$. 
Let $S$ be a regular semisimple linear operator on $\mathbb C^n$. That is, a Jordan form of $S$ is a diagonal matrix with $n$ distinct eigenvalues. 
 A regular semisimple \emph{Hessenberg variety} determined by $h$ and $S$ is defined as
$$\Hess(S, h)\coloneqq \{(\{0\}\subset V_1 \subset \cdots \subset V_n) \in \flag{n} \,\, |\,\, S(V_i)\subseteq V_{h(i)} \mbox{ for } i=1, \dots, n \} \,.$$
Then, $\Hess(S,h)$ is a smooth projective variety of (complex) dimension $d_h\coloneqq \sum_i (h(i)-i)$ \cite{DPS}.

We often use the list of values $(h(1), h(2), \dots, h(n))$ of $h$ to represent a Hessenberg function $h\colon [n] \to [n]$. Moreover, we can depict a Hessenberg function using a graph. For each Hessenberg function $h$, the graph with the vertex set $[n]$ and the edge set $\{\{i,j\} \mid 1\leq i<j\leq h(i) \}$ is called the \emph{incomparability graph} of $h$. 
For instance, when $h=(n,n,\dots,n)$, the complete graph $K_n$ is the incomparability graph of $h$.


Using~\eqref{eq:flag-decomposition}, we obtain an affine paving of $\Hess(S,h)$
\begin{equation*}\label{eq:Hess_paving}
\Hess(S, h)=  \bigsqcup_{w \in \mathfrak{S}_n} \left(\Omega^{\circ}_w \cap \Hess(S, h)\right),
\end{equation*}
so $\Hess(S,h)$ is also equivariantly formal with respect to every torus action \cite{DLP,Kostant}.
For each $w\in\Sn{n}$, the intersection $\Omega_{w, h}^\circ \coloneqq  \Omega^{\circ}_w \cap \Hess(S, h)$ is isomorphic to $\C^{d_h-\ell_h(w)}$, and it is called the \emph{(opposite) Hessenberg Schubert cell} indexed by $w$, where $$\ell_h(w)= \left|  \{ i<j \mid w(i)> w(j), \,\, j\leq h(i)   \}  \right|\,.$$

\begin{proposition}[\cite{T2}] A regular semisimple Hessenberg variety $ \Hess(S, h)$  is a $T$-invariant subvariety of $\flag{n}$, and it is a GKM variety with the associated GKM graph $\Gamma_h=(V, E)$, where $V= \mathfrak{S}_n$ and $E=\{ \{u, v\} \mid v=u(i,j) \mbox{ for }1\leq  i<j\leq h(i) \}$.
\end{proposition}


\begin{remark}
\begin{enumerate}
    \item For a Hessenberg function $h$, if $S$ and $S'$ are regular semisimple linear operators on $\mathbb C^n$, then their associated Hessenberg varieties are diffeomorphic:
$$\Hess(S, h)\approx \Hess(S', h)\, .$$ 
    \item When $h=(n, n, \dots, n)$,  $\Hess(S, h)$ is the full flag variety $\flag{n}$.
\end{enumerate}
\end{remark}

The closure $\Omega_{w, h}\coloneqq \overline{\Omega_{w, h}^\circ}$ is called the \emph{(opposite) Hessenberg Schubert variety} indexed by~$w$, and it is shown in \cite{HP} that $\Ow{w,h}$ is an irreducible component of 
$$\Ow{w}\cap \Hess(S,h)= \bigsqcup_{v \succeq w}\Owo{v,h}\,.$$ 
Since $\Ow{w}$ and $\Hess(S,h)$ are $T$-invariant subvarieties of $\flag{n}$ and $\Ow{w,h}$ is irreducible, $\Ow{w,h}$ is also $T$-invariant.
However, we cannot conclude that $\Ow{w,h}$ is a GKM variety because the equivariant formality of $\Ow{w,h}$ is unknown for the action of~$T$. Fortunately, we can acquire some geometric information of $\Ow{w,h}$ from the fact that $\Ow{w,h}$ is an irreducible $T$-invariant subvariety of $\flag{n}$. Let $\Gamma_{w,h}$ be the subgraph of $\Gamma_h$ induced by $\Ow{w,h}^T$.

\begin{proposition}[\cite{CK}, {Lemma 2.4}]\label{prop:Hessenberg Schubert variety} For $w\in\Sn{n}$ and $v\in \Ow{w,h}^T$, 
$$\dim T_{v}(\Ow{w,h})\geq \deg(v)\geq \dim \Ow{w,h}=d_h-\ell_h(w),$$ where $T_{v}(\Ow{w,h})$ is the Zariski tangent space of $\Ow{w,h}$ at $v\in \Ow{w,h}^T$. When $\Ow{w,h}$ is smooth, all the inequalities in the above are equal.
\end{proposition}

   


For a given Hessenberg function $h\colon [n] \to [n]$, the set of \emph{generators} is defined by
$$\{w\in \mathfrak S_n \mid w^{-1}(w(i)+1)\leq h(i) \text{ for all } w(i)\leq n-1 \}\,.$$
When $h=(3, 3, 4, 4)$, $2134$ is a generator but 
$w=1324$ is not because $h(2)=3<w^{-1}(4)=4$.
In \cite{CHL2}, it is shown that the Białynicki--Birula basis elements corresponding to the set of generators satisfying $\ell_h(w)=k$ 
form a module generator set of the $\mathfrak S_n$-module $H^{2k}(\Hess(S,h);\C)$.

For a permutation $w$, there is a unique generator $\widetilde{w}$ satisfying that \begin{equation}\label{eq:generator}
\widetilde{w}(i)<\widetilde{w}(j) \quad \text{ if and only if } \quad w(i)<w(j)\, \quad \text{ for all } i<j \text{ with } j\leq h(i)\,.
\end{equation}
See Proposition 3.8 in \cite{CHL2}. 
We call $\widetilde{w}$ the \emph{corresponding generator of $w$}. Then $\Ow{w,h}^T$ is characterized as follows.
\begin{proposition}[\cite{CHL}, \cite{HP}]\label{prop:h-fixed points} Let $h$ be a Hessenberg function on $[n]$. For a permutation $w\in \mathfrak S_n$, the following statements hold.
\begin{enumerate}
    \item If w is a generator, then $\Omega_{w, h}^T$ is identified with the Bruhat interval $[w, w_0]$. 
    \item If $w$ is not a generator, then $\Omega_{w, h}^T$ is identified with the set $w\widetilde{w}^{-1}[\widetilde{w}, w_0]$.
\end{enumerate}
In particular, $\Ow{w,h}^T$ and $\Ow{\widetilde{w},h}^T$ have the same cardinality.
\end{proposition}


It should be noted that if $w$ is not a generator, then $\Ow{w,h}^T$ may not form an interval. For example, when $h=(3,3,4,4)$, $w=1324$ is not a generator and the corresponding generator of $w$ is $\widetilde{w}=1423$. In this case, $w_0\in \Ow{w,h}^T$, but the permutation $1432\not\in\Ow{w,h}^T$ even though 
$w\prec 1432 \prec w_0$. 
Nevertheless, we can show that $\Ow{w,h}^T$ and $\Ow{\widetilde{w},h}^T$ induce the isomorphic subgraphs of $\Gamma_h$.

\begin{proposition}\label{prop:iso}
    For $w\in\Sn{n}$, $\Gamma_{w,h}$ is isomorphic to $\Gamma_{\widetilde{w},h}$.
\end{proposition}
\begin{proof}
    For each $w\in \Sn{n}$, by definition of $\Gamma_{w,h}$, there is an edge $\{u,v\}\in E(\Gamma_{w,h})$ if and only if there exists a transposition $(i,j)$ such that $u(i,j)=v$ and $1\leq i<j\leq h(i)$. If $w$ is not a generator, then $\Ow{w,h}^T=\{w\widetilde{w}^{-1}u\mid u\in [\widetilde{w},w_0]\}$ by Proposition~\ref{prop:h-fixed points}. Therefore, for $u'=w\widetilde{w}^{-1}u$ and $v'=w\widetilde{w}^{-1}v$ in $\Ow{w,h}^T$, $\{u',v'\}$ is an edge of $\Gamma_{w,h}$ if and only if $\{u,v\}$ is an edge of $\Gamma_{\widetilde{w},h}$.
\end{proof}



%%%%%%%%%%%%%%%%%%%%%%%%%%%%%%%%%%%%%%%%%%%%%%%%%%%%%%%%%%%%%%
\section{Properties of  \texorpdfstring{$h$}{h}-Bruhat order and induced subgraphs of GKM graphs}\label{sec:GKM}




%------------------------------------------------------------------

In this section, we first introduce the notion of the $h$-Bruhat order and investigate the properties of the order in relation to the Bruhat order. Subsequently, in the second subsection, we proceed to look at the subgraphs $\Gamma_{w,h}$ inside $\Gamma_h$; we define a nice injective map from the set of incident edges of $u$ to that of $v$ when $v$ is greater than $u$ in the $h$-Bruhat order. Finally, we use the injection to introduce seven special patterns that a generator $w$ for $h$ has to avoid for $\Gamma_{w,h}$ to be regular.
%-------------------------------------------------------------------------------------------------------------------------------------
\subsection{The $h$-Bruhat order}\label{sec:interval}

Proposition~\ref{prop:h-fixed points} characterizes the $T$-fixed points of Hessenberg Schubert varieties for a generator $w\in\mathfrak{S}_n$ in relation to the Bruhat order; $\Ow{w,h}^T=[w,w_0]$. It is natural to ask if $\Ow{w,h}^T$ can be characterized in terms of an order on $\mathfrak S_n$, which respects the Hessenberg function $h$ in some way. Indeed, we can think of a natural order that respects $h$: 

\begin{definition}
Let $h\colon [n] \to [n]$ be a Hessenberg function.
 \begin{enumerate}
 \item The \emph{$h$-Bruhat order} on $\mathfrak S_n$ is the transitive closure of the following relation
 $$u\prec_h v \quad \text{ if and only if }\quad v=u(i,j) \text{ for some $(i,  j)$ such that $j\leq h(i)$ and } \ell(v)>\ell(u)\,.$$
 \item For permutations $u\prec_h v$, let $[u, v]_h$ denote the \emph{$h$-Bruhat interval} 
 $\{w \mid u\preceq_h w \preceq_h v\}$. 
 \end{enumerate}
\end{definition}

It is clear that $v\prec_h w$ implies $v\prec w$; however, the converse does not hold true in general. For instance, when $h=(4,4,4,5,5)$, $54132\prec 54231$, but $54132\not\prec_h 54231$. The unique maximum permutation $w_0$ in the Bruhat order is not necessarily the maximum in the $h$-Bruhat order: if $h=(2,3,3,5,5)$, then $54231\not\prec_h 54321=w_0$. However, we obtain the following lemma when $w$ is a generator.

\begin{lemma}\label{lem:h-maximum} For a Hessenberg function $h$, if $w$ is a generator and $w\preceq_h u$, then $u \preceq_h w_0$.
\end{lemma}

\begin{proof}
Let $h$ be a Hessenberg function on $[n]$.
If $h(i)>i$ for all $i<n$ then we can successively apply the adjacent transpositions to $u$ on the right to make a chain $u\prec_h u_1 \prec_h u_2\prec_h\cdots\prec_h u_k\prec_h w_0$; first move $1$ to the $n$th position, then move $2$ to the $(n-1)$st position, and so on until we obtain $w_0$. 

If $h(i)=i$ for some $i<n$, let $i_1<i_2<\cdots <i_{k}=n$ be the such numbers; $h(i_j)=i_j$ for $j=1, \dots, k$, and $i_0\coloneqq 0$. 
Then, since $w$ is a generator and $w\preceq_h u$,
for $j=1,2,\dots,k$
\[
u[i_{j-1}+1,i_j]=w[i_{j-1}+1,i_j]=[n-i_j+1,n-i_{j-1}]\,.
\]
Each $u[i_{j-1}+1, i_{j}]$ 
can be arranged in increasing order by applying a sequence of adjacent transpositions $(a, a+1)$, 
$a=i_{j-1}+1, \dots, i_{j}-1$, 
to $w$.  
This shows that there is a chain $u\prec_h u_1 \prec_h u_2\prec_h\cdots\prec_h u_k\prec_h w_0$. 
\end{proof}

There is a nice result on the $T$-fixed set of $\Omega_{w,h}$ in relation to the $h$-Bruhat order.
 
\begin{proposition}\cite[Proposition 2.11]{CHL}\label{prop:h-interval}
Let $h$ be a Hessenberg function and $w$ be a permutation. If $v\in \Ow{w,h}^T$, then $w\preceq_h v$. In particular, when $w$ is a generator if $w\preceq v$, then $w\preceq_h v$.
\end{proposition}

Now, we can characterize the $T$-fixed points of the Hessenberg Schubert variety in terms of the $h$-Bruhat order. 
Proposition~\ref{prop:h-interval} together with Lemma~\ref{lem:h-maximum} prove the following theorem.


\begin{theorem}\label{thm:interval}
Let $w$ be a generator for a Hessenberg function $h$. Then, as sets
 $$[w, w_0]_h=[w, w_0]\,.$$ 
\end{theorem}


Finally, we show that the chain property of the Bruhat order extends to the $h$-Bruhat order.

\begin{theorem}[Chain Property of the $h$-Bruhat order]\label{thm:h-chain}
    Let $h$ be a Hessenberg function on $[n]$. If $u\prec_h v$ in $\mathfrak S_n$, then there exist $v_i\in \mathfrak S_n$ satisfying $\ell(v_i)=\ell(u)+i$ for $0\leq i\leq k$, and $u=v_0\prec_h v_1\prec_h \cdots \prec_h v_k=v$. 
\end{theorem}

\begin{proof} We show the theorem when $v=u(a,b)$. That is, we show that `if  $u\prec_h u(a,b)$ then there is a chain $u=v_0\prec_h v_1 \prec_h\cdots\prec_h v_k\coloneqq u(a, b)$ with $\ell(v_i)=\ell(u)+i$ for $i=1, \dots, k$.' This will prove the theorem by the definition of the $h$-Bruhat order. 

Suppose that $u\prec_h u(a,b)$. Then $b\leq h(a)$, and by Proposition~\ref{prop:chain} there is a chain 
  $u=v_0\prec v_1\prec \cdots \prec v_k=u(a,b)$ with $\ell(v_i)=\ell(u)+i$ for $0\leq i\leq k$.
The main observation is that, if we only look at the subword $u(a) u(a+1) \cdots u(b)$ of $u$ and $u(b) u(a+1) \cdots u(a)$ of $u(a,b)$, then they are permutations of $\{u(a), u(a+1) \cdots, u(b)\}$. Thus, we can locally apply the chain property to obtain a chain from $u(a) u(a+1) \cdots u(b)$ to $u(b) u(a+1) \cdots u(a)$, which means that we can assume that $v_{i+1}=v_i(a_i, b_i)$ for $a\leq a_i< b_i\leq b$. This implies that  $b_i\leq h(a_i)$ and $v_i\prec_h v_{i+1}$, which completes the proof.
\end{proof}







%-------------------------------------------------------------------------------------------------------------------------------------
\subsection{An injection for the induced subgraphs of GKM graphs}\label{sec:injection}

Recall that for a generator $w$
for a Hessenberg function $h$, $\Gamma_{w,h}$ is the induced subgraph of $\Gamma_h$ whose vertex set is $[w, w_0]=[w, w_0]_h$. Note that $\{u,v\}$ is an edge of $\Gamma_{w,h}$ if and only if $w\preceq u,v$ and $u=v(i,j)$ for some $(i,j)$ with $j \leq h(i)$. For each vertex $u$ in $\Gamma_{w,h}$, we denote by
$$
E_{w,h}(u) \coloneqq \{ (i,j) \mid u(i,j) \succeq w \text{~for~} 1\leq i<j\leq h(i) \}
$$
the set of transpositions $(i,j)$ satisfying that $\{u,u(i,j)\}$ is an edge of $\Gamma_{w,h}$. Thus, it is evident that 
$$
E_{w,h}(w)=\{ (i,j) \mid w(i)<w(j) \text{~for~} 1\leq i<j\leq h(i) \}\,.
$$

For vertices $u$ and $v$ in $\Gamma_{w,h}$ satisfying that $v=u(a,b)$ for some $(a,b) \in E_{w,h}(u)$ and $u\prec_h v$, we define a map $\phi_{uv} \colon E_{w,h}(u) \to E_{w,h}(v)$ by $(i,j) \mapsto (\overline{i},\overline{j})$, where
$$
(\overline{i},\overline{j}) \coloneqq \begin{cases}
(b,j) & \mbox{if $i=a$, $j>b$, and $(b,j)\not \in E_{w,h}(u) $},\\
(i,a) & \mbox{if $i<a$, $j=b$, and $(i,a)\not \in E_{w,h}(u) $},\\
(i,j) & \mbox{otherwise}.
\end{cases}
$$

 
\begin{lemma}\label{lem:injection}
Let $w$ be a generator for a Hessenberg function $h$. If $u$ and $v$ are vertices in $\Gamma_{w,h}$ satisfying that $v=u(a,b)$ for some $(a,b) \in E_{w,h}(u)$ and $u\prec_h v$, then the map $\phi_{uv} \colon E_{w,h}(u) \to E_{w,h}(v)$ is well-defined and injective.
\end{lemma}

\begin{proof}
From the definition of $\phi_{uv}$, it is enough to show that the map $\phi_{uv}$ is well-defined. Note that from the assumption, we have  $a<b\leq h(a)$, $u(a)<u(b)$, and $v(a,b)=u \succeq w$.
We show that $\phi_{uv}$ is well-defined by showing $\overline{i}<\overline{j}\leq h(\overline{i})$ and $ v(\overline{i},\overline{j})\succeq w$. Let $(i,j)\in E_{w,h}(u)$. If $(i,j)=(a,b)$, then $\phi_{uv}(a,b)=(a,b)$. If $\{i,j\} \cap \{a,b\}=\emptyset$, then $(\overline{i},\overline{j})=(i,j)$ so that we have $\overline{i}<\overline{j}\leq h(\overline{i})$ and
$$
v(\overline{i},\overline{j})=v(i,j)=u(a,b)(i,j)=u(i,j)(a,b)\succ u(i,j) \succeq w\,.
$$
Hence, if $|\{i,j\}\cap\{a,b\}| \neq 1$, then $\phi_{uv}$ is well-defined. For the remaining $(i,j)$, we split them into four cases as follows.

\textbf{Case i)} Let $i=a$ and $j < b$. Since $\phi_{uv}(a,j)=(a,j)$, we only need to show that $v(a,j)\succeq w$. In this case, since $(u(a,j))(j)=u(a)<u(b)=(u(a,j))(b)$, by Lemma~\ref{lemma:Bruhat order}
$$
v(a,j)=u(a,j)(j,b)\succ u(a,j) \succeq w\,.
$$

\textbf{Case ii)} Let $i>a$ and $j=b$. Similarly to the above case, since $\phi_{uv}(i,b)=(i,b)$, let us show $v(i,b)\succeq w$. Since $(u(i,b))(a)=u(a)<u(b)=(u(i,b))(i)$, we have
$$
v(i,b)=u(i,b)(a,i)\succ u(i,b) \succeq w\,.
$$

\textbf{Case iii)} Let $i<a$ and $j\in\{a,b\}$. If $(i,a)\in E_{w,h}(u)$, then $\phi_{uv}(i,a)=(i,a)$. Since $(u(i,a))(i)=u(a)<u(b)=(u(i,a))(b)$, we have
$$
v(i,a)=u(a,b)(i,a)=u(i,a)(i,b)\succ u(i,a) \succeq w\,.
$$
In addition, if $(i,b)\in E_{w,h}(u)$, then $\phi_{uv}(i,b)=(i,b)$. Since $u(i,a),u(i,b)\succeq w$, and 
$$
(v(i,b))[k]=\begin{cases}
(u(i,a))[k] & \mbox{if $k<a$},\\
(u(i,b))[k] & \mbox{otherwise},
\end{cases}
$$
for all $k$, we have $v(i,b)\succeq w$ by Proposition~\ref{prop:Bruhat order} (2).

On the other hand, if $(i,a)\not\in E_{w,h}(u)$ and $(i,b)\in E_{w,h}(u)$, then $\phi_{uv}(i,b)=(i,a)$. Here, $i<a<b\leq h(i)$. In this case, we need to show that $v(i,a)\succeq w$. If $u(i)<u(b)$, then $v(i,a)\succ v \succeq w$. Otherwise, we have $u(a)<u(b)<u(i)$, so it follows that 
$$
v(i,a)=u(i,b)(a,b)\succ u(i,b)\succeq w
$$
since $(u(i,b))(a)=u(a)<u(i)=(u(i,b))(b)$.

\textbf{Case iv)} Let $i\in\{a,b\}$ and $j>b$. Similarly to Case iii), if $(b,j)\in E_{w,h}(u)$, then $\phi_{uv}(b,j)=(b,j)$. Since $(u(b,j))(a)=u(a)<u(b)=(u(b,j))(j)$, we have
$$
v(b,j)=u(a,b)(b,j)=u(b,j)(a,j)\succ u(b,j) \succeq w\,.
$$
In addition, if $(a,j)\in E_{w,h}(u)$, then $\phi_{uv}(a,j)=(a,j)$ and $v(a,j)\succeq w$ as $u(a,j),u(b,j)\succeq w$ and for all $k$
\[
(v(a,j))[k]=\begin{cases}
(u(a,j))[k] & \mbox{if 
$k<b$,
}\\
(u(b,j))[k] & \mbox{otherwise}.
\end{cases}
\]

On the other hand, if $(b,j)\not\in E_{w,h}(u)$ and $(a,j)\in E_{w,h}(u)$, then $\phi_{uv}(a,j)=(b,j)$. Here, $a<b<j\leq h(a)\leq h(b)$. In this case, we need to show that $v(b,j)\succeq w$. If $u(a)<u(j)$, then $v(b,j)\succ v \succeq w$. Otherwise, we have $u(j)<u(a)<u(b)$, so it follows that 
$$
v(b,j)=u(a,j)(a,b)\succ u(a,j)\succeq w
$$
since $(u(a,j))(a)=u(j)<u(b)=(u(a,j))(b)$.

Thus, we conclude that $\phi_{uv}$ is well-defined.
\end{proof}

\begin{example} 
Consider a generator $w=2134$ for the Hessenberg function $h=(3,3,4,4)$.
The graph $\Gamma_{w,h}$ is depicted in Figure~\ref{fig:2134}.

\begin{enumerate}
    \item If $u=2314$ and $v=u(3,4)=2341$, then we have $E_{w,h}(u)=\{(1,2),(2,3),(3,4)\}$ and $E_{w,h}(v)=\{(1,2), (1,3), (2,3),(3,4)\}$. The map $\phi_{uv}\colon E_{w,h}(u) \to E_{w,h}(v)$ is an injection defined by $\phi_{uv}(i,j)=(i,j)$ for all $(i,j)\in E_{w,h}(u)$.
    \item If $u=3124$ and $v=u(2,3)=3214$, then we have $E_{w,h}(u)=\{(1,3),(2,3),(3,4)\}$ and $E_{w,h}(v)=\{(1,2),(2,3),(3,4)\}$. The map $\phi_{uv}$ sends $(1,3)$ to $(1,2)$ and fixes the others.
\end{enumerate}

\end{example}

The following theorem is immediate from the previous lemma due to Theorem~\ref{thm:h-chain} and Lemma~\ref{lem:h-maximum}.

\begin{theorem}\label{thm:increasing}
For a Hessenberg function $h$ and its generator $w$, if $u$ and $v$ are vertices in $\Gamma_{w,h}$ with $u\preceq_h v$, then $|E_{w,h}(u)|\leq |E_{w,h}(v)|$. Furthermore, $|E_{w,h}(w)|\leq |E_{w,h}(w_0)|.$
\end{theorem}

As an example, Figure~\ref{fig:2134} shows that the strict inequality holds in the previous theorem.
Here, $|E_{w,h}(u)|=3$ and $|E_{w,h}(v)|=4$ for red vertices $u$ and black vertices $v$, respectively.

 
% Figure environment removed

%-------------------------------------------------------------------------------------------------------------------------------------
\subsection{Associated patterns for irregularity}\label{sec:irregularity}

Recall that for a Hessenberg function $h=(n,n,\dots,n)$, the induced subgraph $\Gamma_{w,h}$ is irregular if and only if $w\in \mathfrak{S}_n$ contains pattern $2143$ or pattern $1324$. We provide seven patterns associated with a Hessenberg function $h$ for irregularity of $\Gamma_{w,h}$ when $w$ is a generator. Using the injection $\phi_{uv}$ that we defined in the previous section, we show that $\Gamma_{w,h}$ is not regular when a generator $w$ contains one of the seven patterns.


\begin{definition}\label{def:pattern4}
Let $h$ be a Hessenberg function on $[n]$. For a permutation $w\in \mathfrak{S}_n$, we say that $w$ contains the associated pattern
\begin{enumerate}
\item $\hpat{2143}$ if $w(j)<w(i)<w(\ell)<w(k)$ for some $i<j<k<\ell\leq h(i)$;
\item $\hpat{1324}$ if $w(i)<w(k)<w(j)<w(\ell)$ for some $i<j<k<\ell\leq h(j)$ with $k\leq h(i)$;
\item $\hpat{1243}$ if $w(i)<w(j)<w(\ell)<w(k)$ for some $i<j<k<\ell\leq h(j)$ with $j\leq h(i)<\ell$;
\item $\hpat{2134}$ if $w(j)<w(i)<w(k)<w(\ell)$ for some $i<j<k<\ell\leq h(k)$ with $k\leq h(i)<\ell$; 
\item $\hpat{1423}$ if 
$w(i)<w(k)<w(\ell)<w(j)$ 
for some $i<j<k<\ell\leq h(j)$ with $k\leq h(i)<\ell$;
\item $\hpat{2314}$ if 
$w(k)<w(i)<w(j)<w(\ell)$ 
for some $i<j<k<\ell\leq h(j)$ with $k\leq h(i)<\ell$;
\item $\hpat{2413}$ if 
$w(k)<w(i)<w(\ell)<w(j)$ 
for some $i<j\leq h(i)<k\leq h(j)<\ell\leq h(k)$.\end{enumerate}
\end{definition}

Figure~\ref{fig:patterns} presents the appropriate induced subgraphs of the incomparability graphs of Hessenberg functions for the associated patterns. 

% Figure environment removed

\begin{remark}
    When $h=(2, 3, \dots, n, n)$, the corresponding Hessenberg variety is the permutohedral variety, which is a smooth projective toric variety. From the definition, every permutation avoids the patterns (1) to (6). Furthermore, every generator of the permutohedral variety avoids the associated pattern $\hpat{2413}$, see \cite[Lemma 5.21]{CHL}. It should be noted that $\Gamma_{w,h}$ is regular, and $\Omega_{w, h}$ is smooth for every $w\in \mathfrak S_n$ because $\Gamma_{w,h}$ forms a face containing $w$ and $w_0$ in the permutohedron if $w$ is a generator.
\end{remark}

\begin{theorem}\label{thm:irregular}
For a Hessenberg function $h$, let $w$ be a generator. If $w$ contains one of associated patterns in the set $\{ \hpat{2143},  \hpat{1324}, \hpat{1243},\hpat{2134}, \hpat{1423}, \hpat{2314}, \hpat{2413} \}$, then the graph $\Gamma_{w,h}$ is irregular. 
\end{theorem}

\begin{proof}
By Lemma~\ref{lem:injection}, the map $\phi_{uv}\colon E_{w,h}(u)\to E_{w,h}(v)$ is injective for every edge $\{u,v\}\in \Gamma_{w,h}$ with $u\prec v$. For a permutation $w$ containing an associated pattern in Definition~\ref{def:pattern4}, to show that $\Gamma_{w,h}$ is irregular, we find an appropriate edge $\{u,v\}$ of $\Gamma_{w,h}$ with $u\prec v$ such that $\phi_{uv}$ is not surjective.

\textbf{Case $\hpat{2143}$.} Suppose that $w(j)<w(i)<w(\ell)<w(k)$ for some 
$i<j<k<\ell\leq h(i)$. 
Set $u\coloneqq w(i,k)(j,k)$ and $v\coloneqq u(k,\ell)$. Note that $(i,k)\in E_{w,h}(v)$ since $k\leq h(i)$ and
$$
v(i,k)=w(j,\ell)(i,j)\succ w(j,\ell)\succ w\,.
$$
In this case, if $\phi_{uv}(\tau)=(i,k)$, then $\tau\in \{ (i,k), (i,\ell) \}$. However, $u(i,k)=w(i,j)\prec w$ and $u(i,\ell)\not \succeq w$ since
$$
(u(i,\ell))[k]\!\!\uparrow=\{w(1),w(2),\dots,w(k-1),w(\ell)\}\!\!\uparrow < w[k]\!\!\uparrow.
$$
Therefore $(i,k), (i,\ell)\not \in E_{w,h}(u)$ and $\phi_{uv}$ is not surjective. 

\textbf{Case $\hpat{1324}$.} Suppose that $w(i)<w(k)<w(j)<w(\ell)$ for some $i<j<k<\ell\leq h(j)$ with $k\leq h(i)$.
Set $u\coloneqq w(k,\ell)$ and $v\coloneqq u(i,k)$. Note that $(j,\ell)\in E_{w,h}(v)$ since 
$\ell \leq h(j)$ and 
$$
v(j,\ell)=w(i,k)(j,\ell)(i,j)\succ w(i,k)(j,\ell)\succ w(i,k) \succ w\,.
$$ 
By the definition of $\phi_{uv}$, if $\phi_{uv}(\tau)=(j,\ell)$, then $\tau$ must be $(j,\ell)$. However,
$$
(u(j,\ell))[j]\!\!\uparrow=\{w(1),w(2),\dots,w(j-1),w(k)\}\!\!\uparrow < w[j]\!\!\uparrow,
$$
and this yields that $u(j,\ell)\not\succeq w$. Hence $(j,\ell)\not \in E_{w,h}(u)$ and $\phi_{uv}$ is not surjective. 

\textbf{Case $\hpat{1243}$.} Suppose that $w(i)<w(j)<w(\ell)<w(k)$ for some $i<j<k<\ell\leq h(j)$ with $j\leq h(i)<\ell$. Set $u\coloneqq w(j,k)$ and $v\coloneqq u(i,j)$. Note that $(j,\ell)\in E_{w,h}(v)$ since 
$\ell\leq h(j)$ and $v(j,\ell) \succ v \succeq w$. In this case, if $\phi_{uv}(\tau)=(j,\ell)$, then $\tau$ must be $(j,\ell)$ since $h(i)<\ell$. However, $u(j,\ell)\not \succeq w$ since
$$
(u(j,\ell))[k]\!\!\uparrow=\{w(1),w(2),\dots,w(k-1),w(\ell)\}\!\!\uparrow < w[k]\!\!\uparrow.
$$
So $(j,\ell)\not \in E_{w,h}(u)$ and $\phi_{uv}$ is not surjective. 

\textbf{Case $\hpat{2134}$.} Suppose that $w(j)<w(i)<w(k)<w(\ell)$ for some $i<j<k<\ell\leq h(k)$ with $k\leq h(i)<\ell$. Set $u\coloneqq w(j,k)$ and $v\coloneqq u(k,\ell)$. Note that $(i,k)\in E_{w,h}(v)$ since 
$k\leq h(i)$ and $v(i,k) \succ v \succeq w$. Similarly to the above case, if $\phi_{uv}(\tau)=(i,k)$, then $\tau$ must be $(i,k)$ since $h(i)<\ell$. However, $u(i,k)\not \succeq w$ since
$$
(u(i,k))[i]\!\!\uparrow=\{w(1),w(2),\dots,w(i-1),w(j)\}\!\!\uparrow < w[i]\!\!\uparrow.
$$
Thus $(i,k)\not \in E_{w,h}(u)$ and $\phi_{uv}$ is not surjective. 

\textbf{Case $\hpat{1423}$.} Suppose that $w(i)<w(k)<w(\ell)<w(j)$ 
for some $i<j<k<\ell\leq h(j)$ with $k\leq h(i)<\ell$. Set $u\coloneqq w(i,k)$ and $v\coloneqq u(i,j)$. Note that $(j,\ell)\in E_{w,h}(v)$ since 
$\ell \leq h(j)$ and $v(j,\ell) \succ v \succeq w$. Similarly to the above case, if $\phi_{uv}(\tau)=(j,\ell)$, then $\tau$ must be $(j,\ell)$ since $h(i)<\ell$. However, $u(j,\ell)\not \succeq w$ since
$$
(u(j,\ell))[k]\!\!\uparrow=((w[k]-\{w(j)\})\cup\{w(\ell)\})\!\!\uparrow < w[k]\!\!\uparrow.
$$
Thus $(j,\ell)\not \in E_{w,h}(u)$ and $\phi_{uv}$ is not surjective. 

\textbf{Case $\hpat{2314}$.} Suppose that $w(k)<w(i)<w(j)<w(\ell)$ for some $i<j<k<\ell\leq h(j)$ with $k\leq h(i)<\ell$. Set $u\coloneqq w(j,\ell)$ and $v\coloneqq u(k,\ell)$. Note that $(i,k)\in E_{w,h}(v)$ since 
$k \leq h(i)$ and $v(i,k) \succ v \succeq w$. Again, if $\phi_{uv}(\tau)=(i,k)$, then $\tau$ must be $(i,k)$ since $h(i)<\ell$. However, $u(i,k)\not \succeq w$ since
$$
(u(i,k))[i]\!\!\uparrow=\{w(1),w(2),\dots,w(i-1),w(k)\}\!\!\uparrow < w[i]\!\!\uparrow.
$$
Thus $(i,k)\not \in E_{w,h}(u)$ and $\phi_{uv}$ is not surjective. 

\textbf{Case $\hpat{2413}$.} Suppose that $w(k)<w(i)<w(\ell)<w(j)$ for some $i<j\leq h(i)<k\leq h(j)<\ell\leq h(k)$. Set $u\coloneqq w(i,j)$ and $v\coloneqq u(k,\ell)$. Note that $(j,k)\in E_{w,h}(v)$ since 
$k \leq h(j)$ and $v(j,k) \succ v \succeq w$. Again, if $\phi_{uv}(\tau)=(j,k)$, then $\tau$ must be $(j,k)$ since $h(j)<\ell$. However, $u(j,k)\not \succeq w$ since
$$
(u(j,k))[j]\!\!\uparrow=((w[j]-\{w(i)\})\cup\{w(k)\})\!\!\uparrow < w[j]\!\!\uparrow.
$$
Thus $(j,k)\not \in E_{w,h}(u)$ and $\phi_{uv}$ is not surjective. This completes the proof.

\end{proof}

As a consequence of Theorem~\ref{thm:irregular}, we can provide a necessary condition for $\Omega_{w, h}$ to be smooth when $w$ is a generator.

\begin{theorem}\label{thm:not-smooth_generator}
    For a Hessenberg function $h$, if $w$ is a generator containing one of the associated patterns in the set $\{ \hpat{2143}, \hpat{1324}, \hpat{1243},\hpat{2134}, \hpat{1423}, \hpat{2314}, \hpat{2413} \}$, then the Hessenberg Schubert variety $\Omega_{w, h}$ is not smooth.
\end{theorem}
\begin{proof}
    If $w$  is a generator containing one of the associated patterns in the set above, then $\Gamma_{w,h}$ is an irregular graph, 
    so $\deg(w_0)=|E_{w,h}(w_0)|>|E_{w,h}(w)|=\dim\Ow{w,h}$ due to Theorem~\ref{thm:increasing}.
    %so there is a vertex $v$ such that $\deg(v)>\dim\Ow{w,h}$. 
    Therefore, $\Ow{w,h}$ is not smooth by Proposition~\ref{prop:Hessenberg Schubert variety}.
\end{proof}


%%%%%%%%%%%%%%%%%%%%%%%%%%%%%%%%%%%%%%%%%%%%%%%%%%%%%%%%%%%%%%

\section{Regular subgraphs of GKM graphs}\label{sec:regular}

In this section, we aim to characterize the regularity of $\Gamma_{w,h}$ in terms of pattern avoidance. First, we focus on the generators for the given Hessenberg function~$h$ and extend our work described in Theorem~\ref{thm:irregular} by considering the converse of the theorem. The generators for~$h$ avoiding the seven patterns are shown to have nice properties that ensure the regularity of~$\Gamma_{w,h}$. Subsequently, we assess more patterns to deal with an arbitrary permutation. The fact that a unique generator $\widetilde{w}$ exists for each $w$, where  $\Gamma_{w,h}$ and $\Gamma_{\widetilde{w},h}$ are isomorphic graphs, plays an essential role.


\subsection{Subgraphs of GKM graphs for well-organized generators}\label{sec:organized}

Here, we define the notion of well-organized permutation and study the properties of well-organized generators.

For a permutation $w\in\mathfrak{S}_n$, we denote 
$$
Y(w)\coloneqq \{w(i) \mid i\geq w^{-1}(1) \mbox{~and~} w(i)\leq w(n)\}\,.
$$
We always denote $Y(w)=\{y_0,y_1,\dots,y_{r}\}$ with $1=y_0<y_1<\cdots<y_{r}=w(n)$. Furthermore, we say that $w$ is a \emph{well-organized} if it satisfies that $w^{-1}(y_0)<w^{-1}(y_1)<\cdots<w^{-1}(y_{r})=n$. 
The following lemma shows a property of a generator.

\begin{lemma}\label{lem:y}
For a Hessenberg function $h$, if $w$ is a generator with $Y(w)=\{y_0,y_1,\dots,y_{r}\}$, then $w^{-1}(y_i)\leq h(w^{-1}(y_{i-1}))$ for $1\leq i \leq r$.
\end{lemma}

\begin{proof}
Recall that $w^{-1}(a+1)\leq h(w^{-1}(a))$ for $1\leq a\leq n-1$ since $w$ is a generator.
If $y_{i}=y_{i-1}+1$, then $w^{-1}(y_i)=w^{-1}(y_{i-1}+1)\leq h(w^{-1}(y_{i-1}))$. On the other hand, if $y_i > y_{i-1}+1$, then $y_{i}-1\not \in Y(w)$ and $w^{-1}(y_{i}-1)<w^{-1}(1)\leq w^{-1}(y_{i-1})$ so that 
$$
w^{-1}(y_i)\leq h(w^{-1}(y_i-1))\leq h(w^{-1}(1))\leq h(w^{-1}(y_{i-1}))\,.
$$ 
This proves the lemma.
\end{proof}

For a well-organized permutation $w$, we let $\overline{w}_m \coloneqq (1,y_m)\overline{w}_{m-1}$ for $m=1,2,\dots,r$, where $\overline{w}_0 \coloneqq w$. Note that $\overline{w}_m=\overline{w}_{m-1}(w^{-1}(y_{m-1}),w^{-1}(y_m))$ and $\overline{w}_0 \prec \overline{w}_1 \prec \cdots \prec \overline{w}_{r}$. For example, $w=213654$ is well-organized permutation with $Y(w)=\{1,3,4\}$. Here, 
$$
\overline{w}_0=213654 \prec \overline{w}_1=231654 \prec \overline{w}_2=234651\,.
$$ 

The following proposition shows several properties of $\overline{w}_{r}$ for a well-organized generator $w$ with $Y(w)=\{y_0,y_1,\dots, y_{r}\}$. For the sake of simplicity, let  $\overline{w}\coloneqq \overline{w}_{r}$.

\begin{proposition}\label{prop:wbar}
For a Hessenberg function $h$ on $[n]$, if $w$ is a well-organized generator with $Y(w)=\{y_0,y_1,\dots,y_{r}\}$, then the following hold.

\begin{enumerate}
\item $\overline{w}$ is a generator.
\item $[\overline{w},w_0]=\{u\in [w,w_0] \mid u(n)=1\}$.
\item $\{(i,n)\in E_{w,h}(\overline{w}) \}= \{(w^{-1}(y_s),n) \mid h(w^{-1}(y_s))=n \text{~for~} 0\leq s < r \}$.
\end{enumerate}

\end{proposition}

\begin{proof}

Note that $\overline{w}^{-1}(a)=w^{-1}(a)$ for all $a\not\in Y(w)$.

\begin{enumerate}
\item 
Since $w$ is a generator, 
for $1\leq a\leq n-1$, 
if $a,a+1\not\in Y(w)$, then 
$$
\overline{w}^{-1}(a+1)=w^{-1}(a+1)\leq h(w^{-1}(a))=h(\overline{w}^{-1}(a))\,.
$$
Hence, it suffices to show that $\overline{w}^{-1}(a+1)\leq h(\overline{w}^{-1}(a))$ for $1\leq a\leq n-1$ such that $a\in Y(w)$ or $a+1\in Y(w)$. First, consider the case $a\not \in Y(w)$ and $a+1 \in Y(w)$. Let $y_s-1\not \in Y(w)$ for some $1\leq s \leq r$. 
Since $w$ is a well-organized generator, we have 
$$
\overline{w}^{-1}(y_s)=w^{-1}(y_{s-1}) <w^{-1}(y_s) \leq h(w^{-1}(y_s-1))=h(\overline{w}^{-1}(y_s-1))\,.
$$
Now consider the case $a=y_s$ for $0\leq s\leq r$. If $r=0$, then $\overline{w}^{-1}(2)\leq h(\overline{w}^{-1}(1))=n$. If $s=r$, then by Lemma~\ref{lem:y} we have 
$$
\overline{w}^{-1}(y_{r}+1)< n =w^{-1}(y_{r})\leq h(w^{-1}(y_{r-1}))=h(\overline{w}^{-1}(y_{r}))\,.
$$
For $0<s<r$, if $y_s+1\in Y(w)$, then $y_s+1=y_{s+1}$ so that $\overline{w}^{-1}(y_s+1)=w^{-1}(y_s)$. From Lemma~\ref{lem:y} it follows that
$$
\overline{w}^{-1}(y_s+1)=w^{-1}(y_s) \leq h(w^{-1}(y_{s-1}))=h(\overline{w}^{-1}(y_s))\,.
$$
On the other hand, if $y_s+1\not \in Y(w)$, then we have
$$
\overline{w}^{-1}(y_s+1)=w^{-1}(y_s+1)<w^{-1}(1)<w^{-1}(y_s)\leq h(w^{-1}(y_{s-1}))=h(\overline{w}^{-1}(y_s))\,.
$$
\item Since $[\overline{w},w_0]\subseteq\{u\in [w,w_0]\mid u(n)=1\}$, it suffices to show that $\overline{w}\preceq u$ for each $u\in\Sn{n}$ with $w\preceq u$ and $u(n)=1$.
Suppose that $u[k]\!\!\uparrow< \overline{w}[k]\!\!\uparrow$ for some $k$. Since $w[m]=\overline{w}[m]$ for $m<w^{-1}(1)$ or $m=n$, it follows that $w^{-1}(1)\leq k<n$ and $\overline{w}[k]=(w[k]-\{1\})\cup\{y_s\}$ for some $s>0$. More precisely, since $w$ is well-organized, if we denote 
$w[k]\!\!\uparrow$, $u[k]\!\!\uparrow$, and $\overline{w}[k]\!\!\uparrow$ by ordered sets $\{a_1,a_2,\dots,a_k\}$,
$\{b_1,b_2,\dots,b_k\}$, and $\{c_1,c_2,\dots,c_k\}$, respectively, then $a_i=i$ and $c_i=i+1$ for $1\leq i<y_s$, and $a_i=c_i$ for $y_s\leq i\leq k$. Note that $w\preceq u$ implies that $a_i\leq b_i$ for $1\leq i \leq k$. Hence, if $a_i\leq b_i<c_i$ for some $i$, then $i<y_s$ so that $i\leq c_i<i+1$, or equivalently, $\{c_1,c_2,\dots,c_i\}=[i]$. This contradicts that $u(n)=1$ and $k<n$. Therefore, there is no such $k$.
\item We first check the inclusion ($\subseteq$). Let $(i,n)\in E_{w,h}(\overline{w})$. Then $h(i)=n$ and $\overline{w}(i)<y_{r}$ since $\overline{w}(i,n)\succeq w$ and 
$(\overline{w}(i,n))[n-1]=(w[n-1]-\{\overline{w}(i)\})\cup\{y_{r}\}$.
Note that $i<w^{-1}(1)$ yields that $(\overline{w}(i,n))[i]=w[i-1]\cup\{1\}$ and $\overline{w}(i,n)\not\succeq w$. Hence, $i\geq w^{-1}(1)$ so that $\overline{w}(i)=y_{s+1}$, or equivalently $w(i)=y_s$,  for some $0\leq s< r$. Therefore $(i,n)=(w^{-1}(y_s),n)$ and $h(w^{-1}(y_s))=n$.

Now we check the inclusion ($\supseteq$). Assume that $w(i)=y_s$ and $h(i)=n$ for some $i<n$.  
By Lemma~\ref{lemma:Bruhat order} $\overline{w}(i,n)\succeq w$ because
$\overline{w}(i,n)$ and $w$ agree everywhere except on positions $w^{-1}(y_0), w^{-1}(y_1), \dots, w^{-1}(y_{r})$, and on these positions $\overline{w}(i,n)$ and $w$ have the subsequences $y_1\dots y_s y_0 y_{s+2} \dots y_{r} y_{s+1}$ and $y_0 \dots y_{s-1} y_s y_{s+1} \dots y_{r-1}y_{r}$, respectively.
Therefore $(w^{-1}(y_s),n)\in E_{w,h}(\overline{w})$.
\end{enumerate}
\end{proof}

For a well-organized permutation $w$, if $w$ satisfies that $y_i=i+1$ (respectively, $w(n-i)=y_{r-i}$) for $0\leq i\leq r$, then it is called a \emph{well-organized permutation of the first} (respectively, \emph{second}) \emph{kind}.
For example, permutations $461523$ and $426135$ are of the first and second types, respectively, and $213654$ is a permutation neither of the first type nor of the second type. 
In the following, we show that the map $\phi_{\overline{w}_{m-1}\overline{w}_{m}}$ is a bijection if $w$ is a well-organized generator that satisfies a certain condition.

\begin{proposition}\label{prop:organized}
For a Hessenberg function $h$ on $[n]$, let $w$ be a well-organized generator with $Y(w)=\{y_0,y_1,\dots,y_{r}\}$ for some $r\geq 1$. For $1\leq m \leq r$, let $\overline{w}_m=\overline{w}_{m-1}(a,b)$. If $h(p)\geq b$ for all 
$(p,a)\in E_{w,h}(\overline{w}_m)$ such that $p<w^{-1}(1)$, then
$$
\phi_{\overline{w}_{m-1}\overline{w}_{m}}\colon E_{w,h}(\overline{w}_{m-1}) \to E_{w,h}(\overline{w}_{m})
$$
is a bijection and $\phi_{\overline{w}_{m-1}\overline{w}_{m}}(i,j)=(i,j)$ except 
$\phi_{\overline{w}_{m-1}\overline{w}_{m}}(p,b)=(p,a)$ for $p<w^{-1}(1)$.

In particular, if $w$ is a well-organized generator of the first kind,
then $E_{w,h}(\overline{w}_{m-1})=E_{w,h}(\overline{w}_{m})$ and $\phi_{\overline{w}_{m-1}\overline{w}_{m}}$ is an identity map.
\end{proposition}

\begin{proof}
By Lemma~\ref{lem:injection}, it suffices to check the surjectivity of $\phi_{\overline{w}_{m-1}\overline{w}_{m}}$. 
Let 
$$
A \coloneqq E_{w,h}(\overline{w}_{m})- \{(p,a)\mid p<w^{-1}(1)\}\,.
$$
We show that 
$\phi_{\overline{w}_{m-1}\overline{w}_{m}}(i,j)=(i,j)$
for $(i,j)\in A$
and then find the preimage of $(p,a)$ under $\phi_{\overline{w}_{m-1}\overline{w}_{m}}$
for each
$(p,a)\in E_{w,h}(\overline{w}_{m})$
with $p<w^{-1}(1)$.
Note that since $\overline{w}_m=(1,y_m)\overline{w}_{m-1}=\overline{w}_{m-1}(a,b)$, we get $\overline{w}_{m-1}(a)=1$ and $\overline{w}_{m-1}(b)=y_m$.
 
We first show that $A\subseteq E_{w,h}(\overline{w}_{m-1})$.
Let $(i,j)\in A$. It suffices to show that $\overline{w}_{m-1}(i,j)\succeq w$. Note that if $\overline{w}_{m-1}(i)<\overline{w}_{m-1}(j)$, then $\overline{w}_{m-1}(i,j)\succ \overline{w}_{m-1} \succeq w$. Now we assume that $\overline{w}_{m-1}(i)>\overline{w}_{m-1}(j)$. 
Then $i \neq a$ and $i \neq b$ since $\overline{w}_{m-1}(a)=1$ and 
$\overline{w}_{m-1}(b)=y_m=\min\overline{w}_{m-1}[a+1,n]$, respectively.
Note that if $i<b$ and $j>a$, then
$$
(\overline{w}_{m}(i,j))[k]=(w[k]-\{1,\overline{w}_{m}(i)\})\cup\{y_m,w(j)\}
$$
for $\max\{a,i\}\leq k <\min\{b,j\}$.
It follows from $\overline{w}_{m}(i,j)\succeq w$ that $\overline{w}_{m}(i)<w(j)$. However, this contradicts $w(i)>w(j)$ since $w(i) \leq \overline{w}_{m}(i)$.  
Hence 
it remains the case when $i>b$ or $j\leq a$.
For the remaining $(i,j)$, we split them into two cases as follows.

\begin{itemize}
\item 
Let $i>b$ or $j<a$. Then $\overline{w}_{m-1}(i,j)\succeq w$ since
$$
(\overline{w}_{m-1}(i,j))[k]=\begin{cases}
w[k] & \mbox{if $a\leq k<b$},\\
(\overline{w}_{m}(i,j))[k] & \mbox{otherwise}.
\end{cases}
$$
\item
Let $j=a$ and $i\geq w^{-1}(1)$. If $w(i)>y_m$, then
\[
(\overline{w}_{m}(i,a))[i]=(w[i-1]-\{1\})\cup\{y,y_m\} 
\]
for some $y<y_m$, which contradicts  $\overline{w}_{m}(i,a)\succeq w$. Therefore, $w(i)<y_m$ so that $w(i)=y_s$ for some $s<m-1$. 
By Lemma~\ref{lemma:Bruhat order} $\overline{w}_{m-1}(i,a)\succeq w$ since
$\overline{w}_{m-1}(i,a)$ and $w$ agree everywhere except on positions $w^{-1}(y_0), w^{-1}(y_1), \dots, w^{-1}(y_{m-1})$, and on these positions $\overline{w}_{m-1}(i,a)$ and $w$ have the subsequences $y_1\dots y_s y_0 y_{s+2} \dots y_{m-1} y_{s+1}$ and $y_0 \dots y_{s-1} y_s y_{s+1} \dots y_{m-2}y_{m-1}$, respectively.
\end{itemize}
Therefore, $A\subseteq E_{w,h}(\overline{w}_{m-1})$.

Now consider the sets $P\coloneqq \{(p,j)\mid p<a,~j\in\{a,b\}\}$ and $Q\coloneqq \{(i,q)\mid i\in\{a,b\},~q>b\}$.

\begin{itemize}
\item Suppose that $(i,j)\in A-(P\cup Q)$. It follows from the definition of the map that $\phi_{\overline{w}_{m-1}\overline{w}_{m}}$ fixes all $(i,j)$.
\item Suppose that $(i,q)\in Q$. Note that $\overline{w}_{m}(i,q)\succeq w$
since $\overline{w}_{m}(a)=y_m<\overline{w}_{m}(q)$ and $\overline{w}_{m}(b)=1$.
This yields that if $h(a)\geq q$, then $(a,q),(b,q)\in A$; if $h(b)\leq q <h(a)$, then $(a,q)\not \in A$, $(b,q) \in A$, and $(a,q)\not \in E_{w,h}(\overline{w}_{m-1})$. In any case, we have $\phi_{\overline{w}_{m-1}\overline{w}_{m}}(i,q)=(i,q)$ for $(i,q)\in A\cap Q$.
\item Suppose that $(p,j)\in P$. Note that if $p<w^{-1}(1)$, then $\overline{w}_{m}(p,b)\not \succeq w$ since
$(\overline{w}_m(p,b))(p)=1$. Thus if $(p,b)\in A$, then $p\geq w^{-1}(1)$ and $\overline{w}_m(p)=y_s$ for some $s<m$. This yields that $(p,a), (p,b)\in A$, and $\phi_{\overline{w}_{m-1}\overline{w}_{m}}(p,j)=(p,j)$ for $(p,j)\in A\cap P$. 


\end{itemize}
This shows that
$\phi_{\overline{w}_{m-1}\overline{w}_{m}}(i,j)=(i,j)$
for all $(i,j)\in A$.

Now we assume that $(p,a) \in E_{w,h}(\overline{w}_m)$ and $p<w^{-1}(1)$ and find the preimage of $(p,a)$ under $\phi_{\overline{w}_{m-1}\overline{w}_{m}}$.
By the definition of $\phi_{\overline{w}_{m-1}\overline{w}_{m}}$,
if $\phi_{\overline{w}_{m-1}\overline{w}_{m}}(\tau)=(p,a)$ for some $\tau$, then $\tau \in \{(p,a), (p,b)\}$. 
Note that $(p,a)\in E_{w,h}(\overline{w}_m)$ implies that $w(p)<y_m$
and $h(p)\geq b$ from the assumption. Since $\overline{w}_{m-1}(a)=1$ and $\overline{w}_{m-1}(b)=y_m$, we have $\overline{w}_{m-1}(p,a)\not\succeq w$ and $\overline{w}_{m-1}(p,b)\succeq w$, respectively. Therefore, $(p,a)\not\in E_{w,h}(\overline{w}_{m-1})$ and $(p,b)\in E_{w,h}(\overline{w}_{m-1})$, so 
$\phi_{\overline{w}_{m-1}\overline{w}_{m}}(p,b)=(p,a)$
when $p<w^{-1}(1)$. Hence the map $\phi_{\overline{w}_{m-1}\overline{w}_{m}}$ is bijective.

Note that if $w$ is a well-organized generator of the first kind, then $w(p)>y_m$ for all $p<w^{-1}(1)$. Hence, $E_{w,h}(\overline{w}_{m})=A$ and the map $\phi_{\overline{w}_{m-1}\overline{w}_{m}}$ is an identity map.
This proves the proposition.
\end{proof}


%-------------------------------------------------------------------------------------------------------------------------------------
\subsection{Subgraphs of GKM graphs for pattern avoiding generators}\label{sec:regularproof}

In this subsection, 
we first investigate the relationship between the properties of well-organized generators in Section~\ref{sec:organized} and the generators avoiding the associated patterns in Definition~\ref{def:pattern4}. Subsequently, we  
show the regularity of $\Gamma_{w,h}$, for a generator $w$,  which can be characterized in accordance with the pattern avoidance of $w$.

For simplicity, we assume that $Y(w)=\{y_0,y_1,\dots,y_r\}$ in this subsection unless otherwise stated.
\begin{lemma}\label{lem:1324}
    Let $w$ be a generator for a Hessenberg function~$h$. If $w$ avoids the associated pattern $\hpat{1324}$, then $w$ is well-organized. 
\end{lemma}

\begin{proof}
    Suppose that $w$ is not well-organized so that we have 
$w^{-1}(y_{i})<w^{-1}(y_j)<w^{-1}(y_{i+1})$ for some $i$ and $j$ with $0<i+1<j<r$.  
Let $k$ be the smallest integer such that $k>j$ and $w^{-1}(y_{i+1})<w^{-1}(y_k)$.
Note that if $w^{-1}(y_{k-1})<w^{-1}(y_i)$, then the subsequence $y_i y_j y_{i+1} y_k$ gives the associated pattern $\hpat{1324}$ since $w^{-1}(y_{k})\leq h(w^{-1}(y_{k-1}))\leq h(w^{-1}(y_i))$ by Lemma~\ref{lem:y}. On the other hand, if $w^{-1}(y_i)<w^{-1}(y_{k-1})$, then the subsequence $y_i y_{k-1} y_{i+1} y_k$ gives the associated pattern $\hpat{1324}$ since $w^{-1}(y_{i+1})\leq h(w^{-1}(y_{i}))$ and $w^{-1}(y_{k})\leq h(w^{-1}(y_{k-1}))$ by Lemma~\ref{lem:y}. Hence, $w$ always contains the associated pattern $\hpat{1324}$.
\end{proof}

Note that for a generator $w$, avoiding the associated pattern $\hpat{1324}$ is not a necessary condition to be well-organized. For instance, when $h=(7,7,7,7,7,7,7)$, the permutation $4651273$ is well-organized but contains the pattern $1324$.

If a well-organized generator avoids certain patterns in Definition~\ref{def:pattern4}, then it is of the first or second type.  

\begin{lemma}\label{lem:first-second-kind}
    Assume that $w$ is a well-organized generator for a Hessenberg function~$h$.
    \begin{enumerate}
        \item If $h(w^{-1}(1))<n$ and $w$ avoids the associated pattern $\hpat{2134}$, then $w$ is a well-organized generator of the first kind. 
        In addition, if $w$ also avoids $\hpat{1243}$ and $\hpat{1423}$, then $w(i)\in Y(w)$ for all $i$ such that $h(i)=n$ .
        \item If $h(w^{-1}(1))=n$ and $w$ avoids the associated patterns $\hpat{2143}$ and $\hpat{2134}$, then $w$ is a well-organized generator of the first or the second kind.
    \end{enumerate}
\end{lemma}

\begin{proof}
\begin{enumerate}
\item 
First note that $y_{r}>2$ and $y_{r}-1\in Y(w)$, otherwise we have $h(w^{-1}(1))=n$.
Now suppose that $\{x+1,x+2,\dots,y_{r}\}\subset Y(w)$ and $x \not \in Y(w)$ for some $1<x<y_{r}$.
Since $w^{-1}(x+1)\leq h(w^{-1}(x))$ and $h(w^{-1}(1))<n$, we have $w^{-1}(x+j)\leq h(w^{-1}(x))<w^{-1}(x+j+1)$ for some $1\leq j<y_{r}-x$. Then the subsequence $x1(x+j)(x+j+1)$ gives the associated pattern $\hpat{2134}$ since $w^{-1}(x+j+1)\leq h(w^{-1}(x+j))$. Hence, there is no such $x$, and $w$ is a well-organized generator of the first kind. 

For the latter part, suppose that there is  $i$ satisfying that $h(i)=n$ and $w(i)>r+1=w(n)$. Let $x$ be the smallest integer satisfying that $1\leq x <r$ and $h(w^{-1}(x+1))=n$. Note that if $w^{-1}(x+1)<i$, then the subsequence $x (x+1) w(i) (r+1)$ gives the associated pattern $\hpat{1243}$. On the other hand, if $w^{-1}(x+1)>i$, then the subsequence $x\, w(i)\, (x+1)\,  (r+1)$ gives the associated pattern $\hpat{1423}$. Therefore there is no such $i$.
\item 
Suppose that $w$ is not of the first kind so that $y_{s}-1\not \in Y(w)$ for some $1\leq s \leq r$. Now assume that there is $x$ satisfying that $x>w(n)$ and $w^{-1}(1)<w^{-1}(x)$. If $w^{-1}(x)<w^{-1}(y_s)$, then the subsequence $(y_s-1) 1 x y_s$ gives the associated pattern $\hpat{2143}$ since $w^{-1}(y_{s})\leq h(w^{-1}(y_{s}-1))$. Hence, $w^{-1}(y_s)<w^{-1}(x)$. In this case, if $w^{-1}(x)\leq h(w^{-1}(y_s-1))$, then the subsequence $(y_s-1) 1 x y_{r}$ gives the associated pattern $\hpat{2143}$ since $h(w^{-1}(1))=n$. On the other hand, if $w^{-1}(x)> h(w^{-1}(y_s-1))$, then the subsequence $(y_s-1) 1 y_s x $ gives the associated pattern $\hpat{2134}$ since  $w^{-1}(y_{s})\leq h(w^{-1}(y_{s}-1))$ and $h(w^{-1}(1))=n$, but this contradicts the fact that $w$ avoids $\hpat{2143}$ and $\hpat{2134}$. Therefore there is no such $x$, and $w$ is well-organized of the second kind. 
\end{enumerate}
This completes the proof.
\end{proof}

\begin{proposition}\label{prop:size of Ewh}
    Let $w$ be a generator for a Hessenberg function~$h$. Assume that $w$ avoids the three associated patterns $\hpat{2143}$, $\hpat{1324}$, and $\hpat{2134}$. Then the following hold.
    \begin{enumerate}
        \item If $w$ also avoids the associated pattern $\hpat{2314}$, then $|E_{w,h}(w)|=|E_{w,h}(\overline{w})|$.
\item If $w$ also avoids the two associated patterns $\hpat{1243}$  and $\hpat{1423}$, then 
$$
\{(i,n)\in E_{w,h}(w_0) \}= \{(i,n) \mid n-k+1 \leq i<n\},
$$
where $k=|\{y_i \mid h(w^{-1}(y_i))=n\}|$.
    \end{enumerate}
\end{proposition}
\begin{proof}
    From the assumption, $w$ is a well-organized generator of the first or the second kind by Lemmas~\ref{lem:1324} and~\ref{lem:first-second-kind}.
    \begin{enumerate}
        \item Note that from Proposition~\ref{prop:organized} if $w$ is well-organized of the first kind, then we have 
$$
E_{w,h}(w)=E_{w,h}(\overline{w}_0)=E_{w,h}(\overline{w}_1)=\cdots= E_{w,h}(\overline{w}_{r})=E_{w,h}(\overline{w})\,.
$$ 
Hence we only need to consider when $w$ is well-organized of the second kind. Let $Y(w)\neq [r+1]$. It follows from Lemma~\ref{lem:first-second-kind} that $h(w^{-1}(1))=n$. 
Suppose that for some $1\leq m\leq r$, the injection $\phi_{\overline{w}_{m-1}\overline{w}_{m}}\colon E_{w,h}(\overline{w}_{m-1}) \to E_{w,h}(\overline{w}_{m})$ is not surjective. Let $\overline{w}_m=\overline{w}_{m-1}(a,b)$. Then there exists $p$ such that $p<w^{-1}(1)$, $w(p)<y_m$, $(p,a)\in E_{w,h}(\overline{w}_m)$, and $a\leq h(p)<b$ by Proposition~\ref{prop:organized}. Note that $w(a)=y_{m-1}$ and $w(b)=y_{m}$.
If $w(p)<y_{m-1}$, then the subsequence $w(p)1y_{m-1}y_{m}$ gives the associated pattern $\hpat{2134}$. Hence $y_{m-1}<w(p)<y_{m}$ and $y_{m-1}\neq y_m-1$. Note that $p<w^{-1}(y_m-1)<w^{-1}(1)$ since $h(p)<w^{-1}(y_m)\leq h(w^{-1}(y_m-1))$. Then the subsequence $w(p)(y_m-1)y_{m-1}y_{m}$ gives the associated pattern $\hpat{2314}$. Thus there is no such $p$ for all $m=1,2,\dots,r$ so that 
$$
|E_{w,h}(w)|=|E_{w,h}(\overline{w}_0)|=|E_{w,h}(\overline{w}_1)|=\cdots= |E_{w,h}(\overline{w}_{r})|=|E_{w,h}(\overline{w})|\,.
$$  
\item Let $|\{y_i \mid h(w^{-1}(y_i))=n\}|=k$ and $|\{i \mid h(i)=n\}|=m$. Then $1\leq k \leq m$.
For $n-m+1\leq i<n$, note that $(w_0(i,n))[i]\!\!\uparrow \geq w[i]\!\!\uparrow$ if and only if $w^{-1}(1)\leq i$ since 
\begin{align*}
w[i]&=[n]-\{w(i+1),w(i+2),\dots,w(n)\},\\
(w_0(i,n))[i]&=[n]-\{2,3,\dots,n-i+1\}\,.
\end{align*}
In addition, $(w_0(i,n))[n-1]\!\!\uparrow \geq w[n-1]\!\!\uparrow$ if and only if $n-i+1\leq w(n)$.
From Proposition~\ref{prop:Bruhat order} (3), it follows that $w_0(i,n)\succeq w$ if and only if 
$(w_0(i,n))[i]\!\!\uparrow \geq w[i]\!\!\uparrow$ and $(w_0(i,n))[n-1]\!\!\uparrow \geq w[n-1]\!\!\uparrow$ since
$\{i,n-1\}=[n-1]-D(w_0(i,n))$. Hence 
$$
\{(i,n)\in E_{w,h}(w_0) \}= \{(i,n) \mid \max\{w^{-1}(1), n+1-w(n), n-m+1\} \leq i<n\}\,.
$$ 
If $h(w^{-1}(1))<n$, then $w^{-1}(1)<n-m+1$, $w(n)=r+1$, and $m=k\leq r$ by Lemma~\ref{lem:first-second-kind} (1). On the other hand, if $h(w^{-1}(1))=n$, then from Lemma~\ref{lem:first-second-kind} (2) it follows that $k=r+1$, $w^{-1}(1)=n-r=n-k+1$, and $w(n)>r$. 
In any case, we have 
$\max\{w^{-1}(1), n+1-w(n), n-m+1\}=n-k+1$, 
as we desired.
    \end{enumerate}
    This completes the proof.
\end{proof}

\begin{proposition}\label{prop:chain_w}
Let $w$ be a generator for a Hessenberg function~$h$. If $w$ avoids all of the associated patterns $\hpat{2143}$, $\hpat{1324}$, $\hpat{1243}$, $\hpat{2134}$, $\hpat{1423}$, $\hpat{2314}$, and $\hpat{2413}$, then so does $\overline{w}$.
\end{proposition}

\begin{proof}
Since $w$ avoids the associated patterns $\hpat{2143}$, $\hpat{1324}$, and  $\hpat{2134}$, $w$ is well-organized of the first or the second kind by Lemma~\ref{lem:first-second-kind}. Note that if $w$ is well-organized of the first kind, then $w(i)<w(j)$ if and only if $\overline{w}(i)<\overline{w}(j)$ for all $1\leq i<j <n$. Moreover, since $\overline{w}(n)=1$ and none of the seven associated patterns ends with $1$, $\overline{w}$ avoids all of the associated patterns if $w$ is well-organized of the first kind.

Now we consider the case when $w$ is well-organized of the second kind. Then $h(w^{-1}(1))=n$ by Lemma~\ref{lem:first-second-kind}. In the following, we show that if $\overline{w}$ contains one of the associated patterns in Definition~\ref{def:pattern4}, then so does $w$. Note that $\overline{w}$ and $w$ may contain different associated patterns. For a given associated pattern, 
we suppose that $\overline{w}$ contains the subsequence $\overline{v}\coloneqq \overline{w}(i)\overline{w}(j)\overline{w}(k)\overline{w}(\ell)$ with $1\leq i<j<k<\ell\leq n$ that gives  the pattern. 
Since $w$ is well-organized of the second kind that avoids all of the patterns, we may assume that the subsequence $\overline{v}$ can be written as one of the forms $w(i)y_{q+1} y_{s+1} y_{m+1}$, $w(i)w(j)y_{s+1} y_{m+1}$, and $w(i)w(j)w(k)y_{m+1}$ for some $0\leq q<s<m<r$. 


\textbf{Case $\hpat{2143}$.} 
Suppose that $\overline{w}(j)<\overline{w}(i)<\overline{w}(\ell)<\overline{w}(k)$ for some 
$\ell\leq h(i)$. 
It suffices to consider the subsequence $\overline{v}=w(i)w(j)w(k)y_{m+1}$. Then $w^{-1}(y_m)=\ell\leq h(i)$, and we may assume that $y_m<w(i)$. 
Note that if $w^{-1}(y_{m+1})\leq h(i)$, then $w$ contains the subsequence $w(i)w(j)w(k) y_{m+1}$ that gives the associated pattern $\hpat{2143}$. On the other hand, if $w^{-1}(y_{m+1})> h(i)$, then $w^{-1}(y_{m+1}-1)>i$ and $w(i)(y_{m+1}-1) y_m y_{m+1}$ gives the associated pattern $\hpat{2314}$.


\textbf{Case $\hpat{1324}$.} 
Suppose that $\overline{w}(i)<\overline{w}(k)<\overline{w}(j)<\overline{w}(\ell)$, $k\leq h(i)$, and $\ell\leq h(j)$.
It suffices to consider the subsequence $\overline{v}=w(i)w(j)y_{s+1} y_{m+1}$ or $\overline{v}=w(i)w(j)w(k)y_{m+1}$.
First, let $\overline{v}=w(i)w(j)y_{s+1} y_{m+1}$. Then $w^{-1}(y_s)\leq h(i)$ and $w^{-1}(y_m) \leq h(j)$, and we may assume that $y_s<w(i)$ or $y_m<w(j)$. 
Note that if $w^{-1}(y_{m+1})\leq h(i)$, then $w$ contains $w(i)w(j) y_{s+1} y_{m+1}$ that gives the associated pattern $\hpat{1324}$. Now we assume that $w^{-1}(y_{m+1})> h(i)$, and consider three possible cases according to $y_s$ and $y_m$. 
\begin{itemize}
\item Let $y_s>w(i)$ and $y_m<w(j)$. If $w^{-1}(y_{m+1})\leq h(j)$, then $w$ contains $w(i)w(j) y_s y_{m+1}$ that gives the associated pattern $\hpat{1324}$. On the other hand, if $w^{-1}(y_{m+1})> h(j)$, then $w(j)(y_{m+1}-1) y_m y_{m+1}$ gives the associated pattern $\hpat{2314}$.
\item Let $y_s<w(i)$ and $y_m>w(j)$. If $w^{-1}(y_{m+1})\leq h(j)$, then $w(i)w(j) y_s y_{m+1}$ gives the associated pattern $\hpat{2314}$. On the other hand, if $w^{-1}(y_{m+1})> h(j)$, then $w(j)y_s y_m y_{m+1}$ gives the associated pattern $\hpat{2134}$.
\item Let $y_s<w(i)$ and $y_m<w(j)$. If $w^{-1}(y_{m+1})\leq h(j)$, then $w$ contains $w(i)w(j) y_s y_{m+1}$ that gives the associated pattern $\hpat{2314}$. On the other hand, if $w^{-1}(y_{m+1})> h(j)$, then $w(j)(y_{m+1}-1) y_s y_{m+1}$ gives the associated pattern $\hpat{2314}$.
\end{itemize}
Now let $\overline{v}=w(i)w(j)w(k)y_{m+1}$. It follows that $w^{-1}(y_m) \leq h(j)$, and we may assume that $y_m<w(j)$. In this case if $w^{-1}(y_{m+1})\leq h(j)$, then $w$ contains the subsequence $w(i)w(j)w(k) y_{m+1}$ that gives the associated pattern $\hpat{1324}$. On the other hand, if $w^{-1}(y_{m+1})> h(j)$, then $w(j)(y_{m+1}-1) y_m y_{m+1}$ gives the associated pattern $\hpat{2314}$.

\textbf{Case $\hpat{1243}$.} 
Suppose that $\overline{w}(i)<\overline{w}(j)<\overline{w}(\ell)<\overline{w}(k)$ and $j\leq h(i)<\ell\leq h(j)$. 
It suffices to consider the subsequence $\overline{v}=w(i)w(j)w(k)y_{m+1}$. 
In this case we have $h(i)<w^{-1}(y_m) \leq h(j)$, and we may assume that $y_m<w(j)$. Note that if $w^{-1}(y_{m+1})\leq h(j)$, then the subsequence $w(i)w(j)w(k) y_{m+1}$ of $w$ gives the associated pattern $\hpat{1243}$. On the other hand, if $w^{-1}(y_{m+1})> h(j)$, then $w(j)(y_{m+1}-1) y_m y_{m+1}$ gives the associated pattern $\hpat{2314}$.

\textbf{Case $\hpat{2134}$.} 
Suppose that $\overline{w}(j)<\overline{w}(i)<\overline{w}(k)<\overline{w}(\ell)$ and $k\leq h(i)<\ell\leq h(k)$.
In this case, we have to consider all of the forms of $\overline{v}$. If $\overline{v}$ is one of the forms $w(i)y_{q+1} y_{s+1} y_{m+1}$ and $\overline{v}=w(i)w(j)y_{s+1} y_{m+1}$, then we have $w^{-1}(y_s)\leq h(i)<w^{-1}(y_m) \leq h(w^{-1}(y_s))$, and we may assume that $y_s<w(i)$. If $w^{-1}(y_{s+1})\leq h(i)$, then $w$ contains the subsequence $w(i) y_s y_{s+1} y_{m+1}$ that gives the associated pattern $\hpat{2134}$. On the other hand, if $w^{-1}(y_{s+1})> h(i)$, then the subsequence $w(i) (y_{s+1}-1) y_s y_{s+1}$ gives the associated pattern $\hpat{2314}$.

Now let $\overline{v}=w(i)w(j)w(k)y_{m+1}$. Then $h(i)<w^{-1}(y_m) \leq h(k)$, and we may assume that $y_m<w(k)$. If $w^{-1}(y_{m+1})\leq h(k)$, then $w$ contains the subsequence $w(i)w(j)w(k) y_{m+1}$ that gives the associated pattern $\hpat{2134}$. On the other hand, if $w^{-1}(y_{m+1})> h(k)$, then $w(k) (y_{m+1}-1) y_m y_{m+1}$ gives the associated pattern $\hpat{2314}$ in $w$.

\textbf{Case $\hpat{1423}$.} 
Suppose that $\overline{w}(i)<\overline{w}(k)<\overline{w}(\ell)<\overline{w}(j)$ and $k\leq h(i)<\ell\leq h(j)$. 
It suffices to consider the subsequence $\overline{v}=w(i)w(j)y_{s+1} y_{m+1}$ or $\overline{v}=w(i)w(j)w(k)y_{m+1}$.
First, let $\overline{v}=w(i)w(j)y_{s+1} y_{m+1}$. Then it follows that $w^{-1}(y_s)\leq h(i)<w^{-1}(y_m) \leq h(j)$, and we may assume that $y_s<w(i)$. If $w^{-1}(y_{s+1})\leq h(i)$, then $w$ contains the subsequence $w(i) y_s y_{s+1} y_{m+1}$ that gives the associated pattern $\hpat{2134}$. On the other hand, if $w^{-1}(y_{s+1})> h(i)$, then $w(i)(y_{s+1}-1) y_s y_{s+1}$ gives the associated pattern $\hpat{2314}$ in $w$.

Now let $\overline{v}=w(i)w(j)w(k)y_{m+1}$. Then it follows that $h(i)<w^{-1}(y_m) \leq h(j)$, and we may assume that $y_m<w(k)$. If $w^{-1}(y_{m+1})\leq h(j)$, then $w$ contains the subsequence $w(i)w(j)w(k) y_{m+1}$ that gives the associated pattern $\hpat{1423}$. On the other hand, if $w^{-1}(y_{m+1})> h(j)$, then $w$ contains the associated pattern $\hpat{2314}$ or $\hpat{2413}$. Indeed, if $w^{-1}(1)\leq h(i)$, then $w(i) (y_{m+1}-1) 1 y_{m+1}$ gives $\hpat{2314}$; if $w^{-1}(1)> h(i)$, then $w(i) w(j) 1 y_{m+1}$ gives $\hpat{2413}$.

\textbf{Case $\hpat{2314}$.} 
Suppose that $\overline{w}(k)<\overline{w}(i)<\overline{w}(j)<\overline{w}(\ell)$ and $k\leq h(i)<\ell\leq h(j)$. 
Again, it suffices to consider the subsequence $\overline{v}=w(i)w(j)y_{s+1} y_{m+1}$ or $\overline{v}=w(i)w(j)w(k)y_{m+1}$.
First, let $\overline{v}=w(i)w(j)y_{s+1} y_{m+1}$. Then $w^{-1}(y_s)\leq h(i)<w^{-1}(y_m) \leq h(j)$, and we may assume that $y_m<w(j)$. Accordingly, the subsequence $w(i) (y_{m+1}-1) y_s y_{m+1}$ gives the associated pattern $\hpat{2314}$.

Now we let $\overline{v}=w(i)w(j)w(k)y_{m+1}$, so we have $h(i)<w^{-1}(y_m) \leq h(j)$. We may assume that $y_m<w(j)$. If $w^{-1}(y_{m+1})\leq h(j)$, then  $w$ contains $w(i)w(j)w(k) y_{m+1}$ that gives the associated pattern $\hpat{2314}$. On the other hand, if $w^{-1}(y_{m+1})> h(j)$, then the subsequence $w(j) (y_{m+1}-1) y_m y_{m+1}$ gives the associated pattern $\hpat{2314}$.

\textbf{Case $\hpat{2413}$.} 
Suppose that $\overline{w}(k)<\overline{w}(i)<\overline{w}(\ell)<\overline{w}(j)$ for $j\leq h(i)<k\leq h(j)<\ell\leq h(k)$. 
Again, it suffices to consider the subsequence $\overline{v}=w(i)w(j)y_{s+1} y_{m+1}$ or $\overline{v}=w(i)w(j)w(k)y_{m+1}$. First, let $\overline{v}=w(i)w(j)y_{s+1} y_{m+1}$. Then $h(i)<w^{-1}(y_s)\leq h(j)<w^{-1}(y_{m})\leq h(w^{-1}(y_s))$, and we may assume that $y_m<w(i)$. Accordingly, $w$ contains the subsequence $w(i) w(j) y_s y_{m+1}$ that gives the associated pattern $\hpat{2413}$.

Now let $\overline{v}=w(i)w(j)w(k)y_{m+1}$. Then we have $h(j)<w^{-1}(y_{m})\leq h(k)$, and we may assume that $y_m<w(i)$. If $w^{-1}(y_{m+1})\leq h(k)$, then $w$ contains $w(i)w(j)w(k) y_{m+1}$ that gives the associated pattern $\hpat{2413}$. On the other hand, if $w^{-1}(y_{m+1})> h(k)$, then $w$ contains $w(k) (y_{m+1}-1) 1 y_{m+1}$ that gives the associated pattern $\hpat{2314}$.

Thus we conclude that if $\overline{w}$ contains one of the seven associated patterns, then so does $w$. This completes the proof.
\end{proof}

In the following, we show that the converse of Theorem~\ref{thm:irregular} is also true. Therefore, we characterize the regularity of $\Gamma_{w,h}$ in terms of pattern avoidance when $w$ is a generator.

\begin{theorem}\label{thm:regular}
Let $w$ be a generator for a Hessenberg function~$h$. The graph $\Gamma_{w,h}$ is regular if $w$ avoids all of the associated patterns $\hpat{2143}$, $\hpat{1324}$, $\hpat{1243}$, $\hpat{2134}$, $\hpat{1423}$, $\hpat{2314}$, and $\hpat{2413}$. 
\end{theorem}

\begin{proof}
Suppose that $w$ avoids all of the associated patterns. We prove that $\Gamma_{w,h}$ is regular by induction on $n$. For $n=1$, $\Gamma_{w,h}$ is a single vertex graph, and this is certainly regular. Suppose that $n \geq2$ and $\Gamma_{w',h'}$ is regular for any Hessenberg function $h'$ on $[n-1]$ and its generator $w'$ that avoids all of the associated patterns. Now we consider a Hessenberg function $h$ on $[n]$ and its generator $w$ that avoids all of the associated patterns. From Theorem~\ref{thm:increasing} it suffices to show that $|E_{w,h}(w)|=|E_{w,h}(w_0)|$. From Lemma~\ref{lem:first-second-kind} and Proposition~\ref{prop:size of Ewh} (1), it follows that $w$ is a well-organized generator of the first or the second kind satisfying that $|E_{w,h}(w)|=|E_{w,h}(\overline{w})|$. Hence we only need to show that  $|E_{w,h}(\overline{w})|=|E_{w,h}(w_0)|$.
By Proposition~\ref{prop:wbar}, we see that $\overline{w}$ is a generator for $h$, and $\Gamma_{\overline{w},h}$ is the induced subgraph of $\Gamma_{w,h}$ whose vertex set is $\{u\in[w,w_0] \mid u(n)=1\}$, and 
$$
|E_{w,h}(\overline{w})|=|E_{\overline{w},h}(\overline{w})|+|\{ y_i \in Y(w)\mid h(w^{-1}(y_i))=n \text{~for~} 0\leq i < r \}|\,.
$$
On the other hand, by Proposition~\ref{prop:size of Ewh} (2), we have
$$
|E_{w,h}(w_0)|=|E_{\overline{w},h}(w_0)|+|\{y_i \in Y(w)\mid h(w^{-1}(y_i))=n \text{~for~} 0\leq i \leq r\}|-1\,.
$$
Thus, it suffices to show that $\Gamma_{\overline{w},h}$ is regular. Let $h'$ be the Hessenberg function on $[n-1]$ whose incomparability graph is obtained from the incomparability graph of $h$ by deleting the vertex $n$ and the edges incident to it. Let $w'\in \mathfrak{S}_{n-1}$ be the permutation defined by $w'(i)=\overline{w}(i)-1$ for $i=1,2,\dots,n-1$. As $\overline{w}$ is a generator for $h$, it follows that $w'$ is a generator for $h'$. Moreover, it follows from Proposition~\ref{prop:chain_w} that $w'$ avoids all the patterns. Accordingly, we can apply the induction hypothesis to $w'$ and $h'$ and conclude that $\Gamma_{w',h'}$ is regular. This completes the proof.
\end{proof}

%-------------------------------------------------------------------------------------------------------------------------------------
\subsection{Associated patterns for an arbitrary permutation}\label{sec:all}

Now, we consider not only generators but also all the permutations and provide a set of patterns that characterize the regularity of $\Gamma_{w,h}$. We define a few more patterns that are needed.

\begin{definition}\label{def:pattern5}
Let $h$ be a Hessenberg function on $[n]$. For a permutation $w\in \mathfrak{S}_n$, we say that $w$ contains the associated pattern
\begin{enumerate}
\item $\hpat{25314}$ if $w(\ell)<w(i)<w(k)<w(m)<w(j)$;
\item $\hpat{24315}$ if $w(\ell)<w(i)<w(k)<w(j)<w(m)$;
\item $\hpat{14325}$ if $w(i)<w(\ell)<w(k)<w(j)<w(m)$;
\item $\hpat{15324}$ if $w(i)<w(\ell)<w(k)<w(m)<w(j)$; 
\end{enumerate}
for some $i<j<k\leq h(i) <\ell \leq h(j) <m\leq h(k)$.
\end{definition}

Figure~\ref{fig:addedpatterns} shows the induced subgraph of the incomparability graph of $h$ for the new associated patterns. 

% Figure environment removed

\begin{remark}\label{remark:pattern5}
 Four patterns in Definition~\ref{def:pattern5} are all the possible ones that appear when we consider the permutations satisfying the following two conditions:
 $$ w(i)<w(k)<w(j) \quad \text{ and }  \quad   w(\ell)<w(k)<w(m)\,. $$
\end{remark}

These new patterns are related to the associated pattern $\hpat{2413}$.

\begin{lemma}\label{lem:2413}
Let $h$ be a Hessenberg function.
If $w$ is a generator 
that avoids the associated patterns $\hpat{1243}$, $\hpat{2134}$, and $\hpat{1423}$,
then $w$ contains the associated pattern $\hpat{2413}$ if and only if $w$ contains associated pattern $\hpat{25314}$.
\end{lemma}

\begin{proof}
Let $w$ contain the associated pattern $\hpat{25314}$, i.e., $w(\ell)<w(i)<w(k)<w(m)<w(j)$ for some $i<j<k\leq h(i) <\ell \leq h(j) <m\leq h(k)$. Then $w(\ell)<w(i)<w(m)<w(j)$ and $i<j\leq h(i) <\ell \leq h(j) <m\leq h(\ell)$. Accordingly, $w$ contains the associated pattern $\hpat{2413}$. 

Conversely, 
let $w$ contain $w(i)w(j)w(\ell)w(m)$ that gives the associated pattern $\hpat{2413}$. Note that $w(i)+1<w(m)$ since $w$ is a generator and $h(i)<m$. Consider the following set
$$
X\coloneqq \{w(s) \mid w(i)<w(s)<w(m) \text{ for } j<s<\ell\}\,. 
$$
Suppose that $X$ is empty. Let $y$ be the smallest integer satisfying that $\ell<w^{-1}(y)$ and $w(i)<y$. Then 
$w(i)+1<y$ and $i<w^{-1}(y-1)<j$ since $h(i)<\ell$ and $X$ is empty.
Moreover $y<w(m)$ since $h(j)<m$. Accordingly, $w$ contains the subsequence $w(i)(y-1)w(j)y$ giving the pattern $\hpat{1243}$, which is a contradiction. Hence $X$ is not empty.

Now we show that $w(i)w(j)w(k)w(\ell)w(m)$ gives the associated pattern $\hpat{25314}$ by taking $w(k)=\max X$. It suffices to show that $k\leq h(i)$ and $h(k)\geq m$. 
Note that if $h(i)< w^{-1}(x)$  for all $x\in X$, then $w$ contains the subsequence $w(i)zw(j)x$ for some $x\in X$ and $z$ satisfying that $i<w^{-1}(z)<j$, $w(i)<z<x$, and $h(w^{-1}(z))\leq w^{-1}(x)$. Again, it contradicts the fact that $w$ avoids the associated pattern $\hpat{1243}$. Therefore $h(i)\geq w^{-1}(x)$ for some $x\in X$. Note that if $w^{-1}(x_1)\leq h(i)< w^{-1}(x_2)$ for some $x_1,x_2\in X$, then 
$x_1> x_2$
since $w$ avoids the associated pattern $\hpat{1423}$. 
It follows that
$k\leq h(i)$. In this case, if $h(k)<m$, then $w$ contains $w(k)w(\ell)w(s)w(m)$ for some $s$ satisfying that $\ell<s<m$ and $w(k)<w(s)<w(m)$, which gives the associated pattern $\hpat{2134}$. Thus $h(k)\geq m$, and 
this completes the proof.
\end{proof}

Recall that by Theorems~\ref{thm:irregular} and~\ref{thm:regular}, we can characterize the regularity of 
the graph $\Gamma_{w,h}$ for a generator $w$ as follows: 
 $\Gamma_{w,h}$ is regular if and only if $w$ avoids associated patterns $\hpat{2143}$, $\hpat{1324}$, $\hpat{1243}$, $\hpat{2134}$, $\hpat{1423}$, $\hpat{2314}$, and $\hpat{2413}$.  Owing to Lemma~\ref{lem:2413}, we can replace the pattern $\hpat{2413}$ with a new pattern $\hpat{25314}$ in the characterization of the regularity of $\Gamma_{w,h}$.

 
\begin{remark}\label{rmk:generators} Let $w$ be a generator for a Hessenberg function $h$. Then the following are equivalent:
\begin{enumerate}
    \item $w$ avoids associated patterns $\hpat{2143}$, $\hpat{1324}$, $\hpat{1243}$, $\hpat{2134}$, $\hpat{1423}$, $\hpat{2314}$, and \,\,$\hpat{2413}$.
    \item $w$ avoids associated patterns $\hpat{2143}$, $\hpat{1324}$, $\hpat{1243}$, $\hpat{2134}$, $\hpat{1423}$, $\hpat{2314}$, and\,\, $\hpat{25314}$. 
\end{enumerate}
\end{remark}


Now, we consider all permutations and not solely the generators for $h$. Recall from Proposition~\ref{prop:h-fixed points} that if $w$ is not a generator for a Hessenberg function $h$, then $\Omega_{w, h}^T$ is identified with the set $w\widetilde{w}^{-1}[\widetilde{w}, w_0]$, where $\widetilde{w}$ is the corresponding generator for $h$ satisfying \eqref{eq:generator}.
Moreover, Proposition~\ref{prop:iso} tells us that two graphs $\Gamma_{w,h}$ and $\Gamma_{\widetilde{w},h}$ are isomorphic, which allows us to consider the graph $\Gamma_{\widetilde{w},h}$ of the  corresponding generator $\widetilde{w}$  of $w$ instead of $\Gamma_{w,h}$.



\begin{lemma}\label{lem:allpatterns}
For a Hessenberg function $h$ and a permutation $w$, let $\widetilde{w}$ be the corresponding generator of $w$. 
\begin{enumerate}
\item For each associated pattern $p\in \{ \hpat{2143}, \hpat{1324}, \hpat{1243},\hpat{2134}, \hpat{1423}, \hpat{2314}\}$, $w$ contains $p$ if and only if $\widetilde{w}$ contains $p$.
\item 
Let $w$ avoid the associated pattern $\hpat{1324}$. 
Then $w$ contains one of the associated patterns $\hpat{25314}$, $\hpat{24315}$, $\hpat{14325}$, and $\hpat{15324}$ if and only if $\widetilde{w}$ contains the associated pattern $\hpat{25314}$. 
\end{enumerate}
\end{lemma}

\begin{proof}
\begin{enumerate}
\item Suppose that $w$ contains the subsequence $w(i)w(j)w(k)w(\ell)$ that gives an associated pattern $p\in \{ \hpat{2143},  \hpat{1324}, \hpat{1243},\hpat{2134}, \hpat{1423}, \hpat{2314}\}$ for some $i<j<k<\ell$. Due to (\ref{eq:generator}), for any $p$, the set $A\coloneqq\{\widetilde{w}(i),\widetilde{w}(j),\widetilde{w}(k),\widetilde{w}(\ell)\}$ is totally ordered because
\begin{itemize}
    \item for $p=\hpat{2143}$, $\ell\leq h(i)$;
    \item for $p\in\{\hpat{1324},\hpat{1243},\hpat{1423}\}$, $\widetilde{w}(i)=\min A$, the position of the second minimum of $p$ is not greater than $h(i)$, and $\ell\leq h(j)$;
    \item for $p\in\{\hpat{2134},\hpat{2314}\}$, $\widetilde{w}(\ell)=\max A$ and $k\leq h(i)$.
\end{itemize}
 Therefore, $\widetilde{w}$ contains $p$ if $w$ contains $p$. Since the roles of $w$ and $\widetilde{w}$ are symmetric, the proof is completed. 
\item Let $\widetilde{w}$ contain the associated pattern $\hpat{25314}$, i.e., $\widetilde{w}(\ell)<\widetilde{w}(i)<\widetilde{w}(k)<\widetilde{w}(m)<\widetilde{w}(j)$ for some $i<j<k\leq h(i) <\ell \leq h(j) <m\leq h(k)$. Then from~\eqref{eq:generator}, $w$ satisfies $w(i)<w(k)<w(j)$ and $w(\ell)<w(k)<w(m)$. Therefore, by Remark~\ref{remark:pattern5}, $w$ contains one of the associated patterns $\hpat{25314}$, $\hpat{24315}$, $\hpat{14325}$, and $\hpat{15324}$.

Conversely, let $w$ contain the subsequence $w(i)w(j)w(k)w(\ell)w(m)$ that gives an associated pattern in $P\coloneqq\{\hpat{25314},\hpat{24315}, \hpat{14325}, \hpat{15324}\}$. By (\ref{eq:generator}) and Remark~\ref{remark:pattern5}, the subsequence $\widetilde{w}(i)\widetilde{w}(j)\widetilde{w}(k)\widetilde{w}(\ell)\widetilde{w}(m)$ gives an associated pattern $p \in P$. We show that $p\not\in\{\hpat{24315}, \hpat{14325},\hpat{15324}\}$ by using the fact that $\widetilde{w}$ is a generator. 
\begin{itemize}
\item[$\bullet$] For $p\in \{\hpat{24315}, \hpat{14325}\}$, let $x$ be the smallest integer satisfying that $\widetilde{w}(j)<x$ and $k<\widetilde{w}^{-1}(x)$. Then $\widetilde{w}^{-1}(x-1)<k$. If $i<\widetilde{w}^{-1}(x-1)$ (respectively, $i>\widetilde{w}^{-1}(x-1)$), then $\widetilde{w}$ contains $\widetilde{w}(i)(x-1)\widetilde{w}(k)x$ (respectively, $\widetilde{w}(i)\widetilde{w}(j)\widetilde{w}(k)x$) that gives the associated pattern $\hpat{1324}$. This contradicts the assumption by (1).
\item[$\bullet$] For $p=\hpat{15324}$, let $x$ be the smallest integer satisfying that $\widetilde{w}(i)<x$ and $k<\widetilde{w}^{-1}(x)$. Then $\widetilde{w}^{-1}(x-1)<k$ and $x\leq w(\ell)$. If $m<\widetilde{w}^{-1}(x)$ (respectively, $m>\widetilde{w}^{-1}(x)$), then $\widetilde{w}$ contains $(x-1)\widetilde{w}(k)\widetilde{w}(\ell)\widetilde{w}(m)$ (respectively, $(x-1)\widetilde{w}(k)x\widetilde{w}(m)$) that gives the associated pattern $\hpat{1324}$. This contradicts the assumption by (1).
\end{itemize}
Thus $p=\hpat{25314}$, as we desired.
\end{enumerate}
\end{proof}

The following is our main theorem.

\begin{theorem}\label{thm:main}
For a Hessenberg function $h\colon [n] \to [n]$, let $w$ be a permutation on $[n]$. The graph $\Gamma_{w,h}$ is regular if and only if $w$ avoids all of the associated patterns $\hpat{2143}$, $\hpat{1324}$, $\hpat{1243}$, $\hpat{2134}$, $\hpat{1423}$, $\hpat{2314}$, $\hpat{25314}$, $\hpat{24315}$, $\hpat{14325}$, and $\hpat{15324}$. 
\end{theorem}

\begin{proof}
Let $\widetilde{w}$ be the corresponding generator of $w$. Let us denote the associated pattern sets,
\begin{align*}
&A\coloneqq\{\hpat{2143}, \hpat{1324}, \hpat{1243}, \hpat{2134}, \hpat{1423}, \hpat{2314} \},\\
&B\coloneqq\{\hpat{2143}, \hpat{1324}, \hpat{1243}, \hpat{2134}, \hpat{1423}, \hpat{2314}, \hpat{2413}\},\\
&C\coloneqq\{\hpat{2143}, \hpat{1324}, \hpat{1243}, \hpat{2134}, \hpat{1423}, \hpat{2314}, \hpat{25314}, \hpat{24315}, \hpat{14325}, \hpat{15324}\}\,.
\end{align*}

If $w$ contains an associated pattern in $A$, then so does $\widetilde{w}$ by Lemma~\ref{lem:allpatterns} (1). For a permutation $w$ that avoids all of the associated patterns in $A$, if $w$ contains an associated pattern in $\{\hpat{25314}, \hpat{24315}, \hpat{14325}, \hpat{15324}\}$, then $\widetilde{w}$ contains the associated pattern $\hpat{2413}$ by Lemmas~\ref{lem:2413} and \ref{lem:allpatterns} (2).
Hence $\widetilde{w}$ contains an associated pattern in $B$ if $w$ contains an associated pattern in $C$, and $\Gamma_{w,h}$ is irregular by Proposition~\ref{prop:iso} and Theorem~\ref{thm:irregular}.

Similarly, if $w$ avoids all of the associated patterns in $C$, then $\widetilde{w}$ avoids all of the associated patterns in $B$, and $\Gamma_{w,h}$ is regular by Proposition~\ref{prop:iso} and Theorem~\ref{thm:regular}. This completes the proof.
\end{proof}

Once again, by Proposition~\ref{prop:Hessenberg Schubert variety}, we can provide a necessary condition for $\Omega_{w, h}$ to be smooth, which is an extension of Theorem~\ref{thm:not-smooth_generator}. 


\begin{theorem}\label{thm:not-smooth}
    For a Hessenberg function $h$, if a permutation $w$ contains any one of the associated patterns  
    $\hpat{2143}$, $\hpat{1324}$, $\hpat{1243}$, $\hpat{2134}$, $\hpat{1423}$, $\hpat{2314}$, $\hpat{25314}$, $\hpat{24315}$, $\hpat{14325}$, and $\hpat{15324}$,  
    then the Hessenberg Schubert variety $\Omega_{w, h}$ is not smooth.
\end{theorem}


We believe that the pattern avoidance condition that we found is also an equivalent condition for the smoothness of $\Ow{w,h}$, and we make the following conjecture. In this paper, we proved that the second and third conditions are equivalent and are necessary conditions for the smoothness of the Hessenberg Schubert variety.
\begin{conjecture}\label{conj:equivalent} For a Hessenberg function $h\colon [n] \to [n]$, the following statements are equivalent:
\begin{enumerate}
\item The Hessenberg Schubert variety $\Omega_{w, h}$ is smooth.
\item The subgraph $\Gamma_{w, h}$ of the GKM graph $\Gamma_h$ induced by the torus fixed points in $\Omega_{w, h}^T$ is a regular graph.
\item The permutation $w$ avoids all the patterns $\hpat{2143}$, $\hpat{1324}$, $\hpat{1243}$, $\hpat{2134}$, $\hpat{1423}$, $\hpat{2314}$, $\hpat{25314}$, $\hpat{24315}$, $\hpat{14325}$ and  $\hpat{15324}$.
\end{enumerate}
\end{conjecture}
%%%%%%%%%%%%%%%%%%%%%%%%%%%%%%%%%%%%%%%%%%%%%%%%%%%%%%%%%%%%%%



%%%%%%%%%%%%%%%%%%%%%%%%%%%%%%%%%%%%%%%%%%%%%%%%%%%%%%%%%%%%%%%%%%%%%%%%%%%%
\bibliographystyle{amsplain}
%\bibliography{GKM}
\begin{thebibliography}{10}

\bibitem{BL}
Sara Billey and V.~Lakshmibai, \emph{Singular loci of {S}chubert varieties},
  Progress in Mathematics, vol. 182, Birkh\"{a}user Boston, Inc., Boston, MA,
  2000.

\bibitem{Bil}
Sara~C. Billey, \emph{Pattern avoidance and rational smoothness of {S}chubert
  varieties}, Adv. Math. \textbf{139} (1998), no.~1, 141--156.

\bibitem{BB}
Anders Bj\"{o}rner and Francesco Brenti, \emph{Combinatorics of {C}oxeter
  groups}, Graduate Texts in Mathematics, vol. 231, Springer, New York, 2005.

\bibitem{BC}
Patrick Brosnan and Timothy~Y. Chow, \emph{Unit interval orders and the dot
  action on the cohomology of regular semisimple {H}essenberg varieties}, Adv.
  Math. \textbf{329} (2018), 955--1001. \MR{3783432}

\bibitem{CK}
James~B. Carrell and Jochen Kuttler, \emph{Smooth points of {$T$}-stable
  varieties in {$G/B$} and the {P}eterson map}, Invent. Math. \textbf{151}
  (2003), no.~2, 353--379. \MR{1953262}

\bibitem{CHL2}
Soojin Cho, Jaehyun Hong, and Eunjeong Lee, \emph{{Permutation Module
  Decomposition of the Second Cohomology of a Regular Semisimple Hessenberg
  Variety}}, International Mathematics Research Notices (2022), \, rnac328.

\bibitem{CHL}
\bysame, \emph{Bases of the equivariant cohomologies of regular semisimple
  {H}essenberg varieties}, Adv. Math. \textbf{423} (2023), Paper No. 109018.

\bibitem{DLP}
Corrado De~Concini, George Lusztig, and Claudio Procesi, \emph{Homology of the
  zero-set of a nilpotent vector field on a flag manifold}, J. Amer. Math. Soc.
  \textbf{1} (1988), no.~1, 15--34. \MR{924700}

\bibitem{DPS}
Filippo De~Mari, Claudio Procesi, and Mark~A. Shayman, \emph{Hessenberg
  varieties}, Trans. Amer. Math. Soc. \textbf{332} (1992), no.~2, 529--534.
  \MR{1043857}

\bibitem{Fulton}
William Fulton, \emph{Young tableaux with applications to representation theory
  and geometry}, London Mathematical Society Student Texts, vol.~35, Cambridge
  University Press, 1997.

\bibitem{GT}
Rebecca Goldin and Julliana Tymoczko, \emph{Which {H}essenberg varieties are
  {GKM}?}, {arXiv:2301.09741}, 2023.

\bibitem{GKM}
Mark Goresky, Robert Kottwitz, and Robert MacPherson, \emph{Equivariant
  cohomology, {K}oszul duality, and the localization theorem}, Invent. Math.
  \textbf{131} (1998), no.~1, 25--83. \MR{1489894}

\bibitem{G-P}
Mathieu Guay-Paquet, \emph{A second proof of the {S}hareshian--{W}achs
  conjecture, by way of a new {H}opf algebra}, 2016.

\bibitem{GZ}
Victor Guillemin and Catalin Zara, \emph{1-skeleta, {B}etti numbers, and
  equivariant cohomology}, Duke Math. J. \textbf{107} (2001), no.~2, 283--349.
  \MR{1823050}

\bibitem{HP}
Megumi Harada and Martha Precup, \emph{{Torus fixed point sets of Hessenberg
  Schubert varieties in regular semisimple Hessenberg varieties}}, Osaka
  Journal of Mathematics \textbf{60} (2023), no.~3, 637 -- 652.

\bibitem{Kostant}
Bertram Kostant, \emph{Flag manifold quantum cohomology, the {T}oda lattice,
  and the representation with highest weight {$\rho$}}, Selecta Math. (N.S.)
  \textbf{2} (1996), no.~1, 43--91. \MR{1403352}

\bibitem{LSa}
V.~Lakshmibai and B.~Sandhya, \emph{Criterion for smoothness of {S}chubert
  varieties in {${\rm Sl}(n)/B$}}, Proc. Indian Acad. Sci. Math. Sci.
  \textbf{100} (1990), no.~1, 45--52.

\bibitem{T2}
Julianna~S. Tymoczko, \emph{Permutation actions on equivariant cohomology of
  flag varieties}, Toric topology, Contemp. Math., vol. 460, Amer. Math. Soc.,
  Providence, RI, 2008, pp.~365--384.

\end{thebibliography}


\end{document}