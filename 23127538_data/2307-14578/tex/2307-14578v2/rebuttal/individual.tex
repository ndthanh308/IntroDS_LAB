\noindent \textbf{\rone (Q1) Explanation on Cross Modality Distillation:} The motivation to use distillation is to leverage useful information for human identification from color space. As demonstrated in [20], ensembling the RGB and silhouette information enhances performance, suggesting that complementary, discriminative information exists in both modalities.

\textbf{(Q2) Color space is removed during testing:} To ensure a fair comparison with mainstream gait recognition models, we only employ silhouette in the test phase. In real-life applications, we train the model with all available data, including RGB images as a distillation branch; during testing, our gait recognition relies solely on the silhouette for privacy protection. GADER benefits from the feature space learned by an RGB-based recognition system during training, thereby enhancing the ability to distinguish identities. In practice, the RGB feature can also be ensembled if available as in [20].

\textbf{(Q3) Additional Convolution Layer:} As demonstrated in \textit{Line 507-512}, since each modality possesses its discriminative advantages, directly constraining them to be similar is suboptimal. So we refer [55] to handle the multi-modality distillation.

\textbf{(Q4) Discussion on Backbones:} The main focus of this paper is to introduce an effective training strategy to enhance existing backbones. As shown in Fig.1 and Table 1,2,3, experiments show that our strategies consistently improve performance with various backbones.

\textbf{(Q5) Verification on BRIAR:} The verification results are on all conditions, we will make it clear.

\textbf{(Q6) Analysis of Failure Cases:} As stated in Section A.4, we found that curated datasets, e.g. GREW, also contain unqualified gait fragments. Additionally, the window length, a hyperparameter, needs to be fine-tuned for different datasets, which also contributes to failure cases. However, the gait detector is generally able to accurately identify sequence qualification across various datasets in most cases.
We will include a brief section in the main paper discussing the failure cases.



\noindent \textbf{\rthree (Q1) Improvement in Table 5(a):} The performance on CASIA-B is relatively saturated, as can be observed in various related work~[9,15,16,38,56]. Also, the enhancement in CASIA-B is achieved without additional parameters, demonstrating the effectiveness of ratio attention and distillation.

\textbf{(Q2) Evaluation on Gait Detector:} We assess our gait detector's performance using the confusion matrix, as shown in Table \ref{tab:confusion}. The false positive, true positive, false negative and true negative are 4.5\%, 53.2\%, 3.5\% and 38.8\%, respectively. The results demonstrate that the gait detector distinguishes gait and non-gait effectively.

\textbf{(Q3) Evaluation Setting:} For evaluation, we only reduce the number of test probes.  The gait detector's objective is to filter out sequences unsuitable for gait recognition, such as those depicting standing or incomplete bodies, and leave them to Person ReID. As shown in Table B, gait recognition relies more on temporal information while ReID does not. Therefore, it is optimal to allocate each model to focus on data aligned with its strengths, and the role of the gait detector is to classify the data accordingly. The last row of Table 6 (28.3\% $\rightarrow$ 42.5\%) further supports our assumption.


\begin{table}
    \centering
    \begin{tabular}{c|c|c}
    \hline
    \backslashbox{Prediction}{Label}
         & Unqualified & Qualified\\
         \hline
        Unqualified & 38.8 & 4.5\\ \hline
        Qualified & 3.5 & 53.2\\ \hline
    \end{tabular}
    \caption{}
    \label{tab:confusion}
\end{table}



\textbf{(Q4) Symbols not Well-defined:} We will carefully define all symbols mentioned in the formulas. 
\begin{itemize}
    \item E2: Ratio is calculated based on the height, similar to the process employed in silhouette normalization.
    \item E3: $\mV$ is defined in \textit{Line 308}, representing video.
    \item E4: $F^{f}_i$ and $F^{s}_i$ are $i$th layer of RGB and silhouette feature extraction, respectively.
    \item E5: $\gD$ represents distance, i.e. Euclidean distance.
\end{itemize}

\textbf{(Q5) Ratio Attention Explanation:} The concept behind ratio attention is from the observation that the changes in ratio correlate with the viewpoint, supported by Figure E. Although the distance may affect the absolute value of the ratio, its impact on ratio change is trivial. This notion is further validated by GADER's performance on the BRIAR dataset, demonstrating effectiveness in various distance scenarios.


\textbf{\rfive} We thank the reviewer for the appreciation of our work and accurate interpretations of our results and for recognizing the writing quality, novelty, and thoroughness of experiments. 
