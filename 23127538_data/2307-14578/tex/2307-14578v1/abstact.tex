

\begin{abstract}
   % Gait recognition identifies subjects based on their body shapes and walking patterns. While previous silhouette-based approaches have shown to perform well for indoor scenes, they fail to account for complex situations in outdoor/unconstrained scenes. To better address these scenarios, we propose an end-to-end approach that combines gait detection that leverages the Double Helical Pattern, a classic gait representation, to assess when gait information is present and gait recognition (GAR) which uses an RGB branch that produces good representations for silhouette features to learn from. This design allows us to only use the silhouette modality at the inference time while maintaining high performance. Extensive experiments on indoor and outdoor datasets demonstrate that the proposed method outperforms state-of-the-art methods for gait recognition and verification tasks, particularly a 20.6\% performance improvement for an unconstrained dataset.




   Gait recognition holds the promise of robustly identifying subjects based on their walking patterns instead of color information. While previous approaches have performed well for curated indoor scenes, they have significantly impeded applicability in unconstrained situations, e.g. outdoor, long distance scenes. We propose an end-to-end GAit DEtection and Recognition (GADER) algorithm for human authentication in challenging outdoor scenarios. Specifically, GADER leverages a Double Helical Signature to detect the fragment of human movement and incorporates a novel gait recognition method, which learns representations by distilling from an auxiliary RGB recognition model. At inference time, GADER only uses the silhouette modality but benefits from a more robust representation. Extensive experiments on indoor and outdoor datasets demonstrate that the proposed method outperforms the State-of-The-Arts for gait recognition and verification, with a significant 20.6\% improvement on unconstrained, long distance scenes.



    %; particularly, it is robust in long distance recognition and preserves privacy better
    
    % Gait recognition identifies subjects through body shapes and walking patterns. While silhouette-based methods developed previously have been shown to perform well for indoor scenes, they often fail in outdoor situations. To better address these scenarios, we propose an end-to-end approach for gait detection and recognition (GADER) from unconstrained video sequences acquired at various altitudes and ranges. Gait detection is accomplished by leveraging the Double Helical Pattern, a classic gait representation, to assess when gait information is present. Furthermore, we introduce a novel gait recognition model (GAR), which introduces an RGB branch that produces good representations for silhouette features to learn from. At inference time, this allows us to use the silhouette modality while maintaining high performance. Extensive experiments on CASIA-B and BRIAR datasets demonstrate that the proposed method outperforms state-of-the-art methods for gait recognition and verification tasks. %performing better regardless of indoor or outdoor cases. 
\end{abstract}