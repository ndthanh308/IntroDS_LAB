%%%%%%%%%%%%%%%%%%%%%%%%%%%%%%%%%%%%%%%%%%%%%%%%%%%%%%%%%%%%
%%% ELIFE ARTICLE TEMPLATE
%%%%%%%%%%%%%%%%%%%%%%%%%%%%%%%%%%%%%%%%%%%%%%%%%%%%%%%%%%%%
%%% PREAMBLE 
\documentclass[9pt%,lineno
]{elife}
% Use the onehalfspacing option for 1.5 line spacing
% Use the doublespacing option for 2.0 line spacing
% Please note that these options may affect formatting.
% Additionally, the use of the \newcommand function should be limited.

\usepackage{lipsum} % Required to insert dummy text
\usepackage[version=4]{mhchem}
\newcommand\hmmax{0}
\newcommand\bmmax{0}
\usepackage{siunitx}
\DeclareSIUnit\Molar{M}
%\usepackage[colorlinks=true,urlcolor=blue]{hyperref}
\usepackage[normalem]{ulem}
\usepackage{mathtools}
%\usepackage{amsfonts}
\usepackage{graphicx}
\usepackage{stmaryrd}
\usepackage{amsmath}
%\usepackage{amssymb}
%\usepackage{lineno}
\usepackage{xcolor}
\usepackage{color}
\usepackage{bbold}
\usepackage{bm}

\definecolor{red}{rgb}{0.75,0,0}
\definecolor{blue}{rgb}{0,0,0.75}
\definecolor{green}{rgb}{0,0.5,0}
\definecolor{airforceblue}{rgb}{0.36, 0.54, 0.66}
\newcommand{\Livio}[1]{{\color{airforceblue} {\bf LNC:} #1}}
\newcommand{\Dimitris}[1]{{\color{green} [{\bf DK:} #1]}}
\newcommand{\luca}[1]{{\color{blue} [{\bf LG:} #1]}}

\newcommand{\red}[1]{{\color{red}#1}}
\newcommand{\blue}[1]{{\color{blue}#1}}
\newcommand{\green}[1]{{\color{green}#1}}

\newcommand{\revision}[1]{{\color{magenta}#1}}

\newcommand{\fa}{\bm{f}^{({\rm a})}}

\newcommand{\St}{\bm{\sigma}}
\newcommand{\Sp}{\bm{\sigma}^{({\rm p})}}
\newcommand{\Sa}{\bm{\sigma}^{({\rm a})}}
\newcommand{\Se}{\bm{\sigma}^{({\rm e})}}
\newcommand{\Sv}{\bm{\sigma}^{({\rm v})}}
\newcommand{\Sr}{\bm{\sigma}^{({\rm r})}}

\newcommand{\traceless}[1]{\left\llbracket #1 \right\rrbracket}

\renewcommand{\figurename}{Fig.}

\DeclareMathOperator{\Arg}{Arg}




%%%%%%%%%%%%%%%%%%%%%%%%%%%%%%%%%%%%%%%%%%%%%%%%%%%%%%%%%%%%
%%% ARTICLE SETUP
%%%%%%%%%%%%%%%%%%%%%%%%%%%%%%%%%%%%%%%%%%%%%%%%%%%%%%%%%%%%

\title{Collective epithelial migration mediated by the unbinding of hexatic defects}

\author[1]{Dimitrios Krommydas}
\author[1,2]{Livio Nicola Carenza}
\author[1*]{Luca Giomi}
\affil[1]{Instituut-Lorentz, Universiteit Leiden, P.O. Box 9506, 2300 RA Leiden, The Netherlands}
\affil[2]{Physics Department, College of Sciences, Ko{\c c} University, Rumelifeneri Yolu 34450 Sar\i{}yer, Istanbul,  T{\" u}rkiye}

\corr{giomi@lorentz.leidenuniv.nl}{LG}

%%%%%%%%%%%%%%%%%%%%%%%%%%%%%%%%%%%%%%%%%%%%%%%%%%%%%%%%%%%%
%%% ARTICLE START
%%%%%%%%%%%%%%%%%%%%%%%%%%%%%%%%%%%%%%%%%%%%%%%%%%%%%%%%%%%%

\begin{document}

\maketitle

\begin{abstract}
Collective cell migration in epithelia relies on {\em cell intercalation}: a local remodelling of the cellular network that allows neighbouring cells to swap their positions. Unlike foams and passive cellular fluid, in epithelial intercalation these rearrangements crucially depend on activity. During these processes, the local geometry of the network and the contractile forces generated therein conspire to produce a burst of remodelling events, which collectively give rise to a vortical flow at the mesoscopic length scale. In this article we formulate a continuum theory of the mechanism driving this process, built upon recent advances towards understanding the hexatic (i.e. $6-$fold ordered) structure of epithelial layers. Using a combination of active hydrodynamics and cell-resolved numerical simulations, we demonstrate that cell intercalation takes place via the unbinding of topological defects, naturally initiated by fluctuations and whose late-times dynamics is governed by the interplay between passive attractive forces and active self-propulsion. Our approach sheds light on the structure of the cellular forces driving collective migration in epithelia and provides an explanation of the observed extensile activity of {\em in vitro} epithelial layers.
\end{abstract}

\section{Introduction}

From humble soap froths~\citep{Weaire1999,Graner2008,Marmottant2008} down to epithelial layers, confluent cellular fluids use intercalation to flow, even in the absence of gaps and interstitial structures. At the heart of the process is a mechanism known as {\em topological rearrangement process of the first kind}, or T1 for brevity, through which the vertices of a honeycomb network merge and then split, thereby leading to a remodelling of the network's topology. In epithelia, this strategy allows cells to migrate, while preserving the structural integrity of the tissue, hence its major biological functionalities. These include the barrier function, which allows epithelial layers to maintain homeostasis, ensure nutrient transport and filter out harmful pathogens, as well as extrusion and replacement of apoptotic cell~\citep{Irvine1994,Skoglund2000,WalckShannon2014,Tetley2016,Tetley2018,Pare2020,Rauzi2020}.

While relying exclusively of local rearrangements of the cellular network, intercalation gives rise to surprisingly organized patterns, where cells are able to migrate collectively over distances orders of magnitude larger than the average cell size. In the morphogenesis of {\em Drosophila}, for instance, cell intercalation drives a major cellular rearrangement known as germ-band extension, in which a layer of cells initially localized in the ventral region of the developing embryo, folds over the dorsal region upon extending by approximatively two and half time its initial length. In this process, individual cells persistently move across the anterior-posterior axis more than ten times their body length. In {\em Drosophila}, intercalation is also the main process driving the salivary gland tube formation, in which cells radially converge towards a central pit and eventually escape from the tangent plane of the embryo, thereby giving rise to the tubular structure~\citep{Roper2018}.

In the context of cancer progression, the role of cell intercalation has been recently debated in relation with various stages of the so-called {\em metastatic cascade}: i.e. biomechanical pathway leading to the formation of a secondary tumor~\citep{cheung2016}. The latter is schematically divided into three main phases: {\em 1)} detachment of cell clusters from a primary tumor and invasion of the extracellular matrix (ECM); {\em 2)} penetration (i.e. {\em intravasation}), circulation and expulsion (i.e. {\em extravasation}) of the clusters in and from the blood stream; {\em 3)} colonization of a healthy tissue and the proliferation of a secondary tumor. Along the cascade, metastatic cells undergo multiple phenotypic switches, aimed at maximizing their chances of success and survival within the surrounding microenvironment. These, in turn, determine the cells' motility mode, which can vary from individual (e.g. ameboid or mesenchymal) to collective (e.g. intercalation-based or flocking guided by a small number leader cells localized at the front of the cluster)~\citep{friedl2009,Friedl2019,Trepat2012,Payne2024}. 

Yet, while being highly regulated at the biochemical level~\citep{Lecuit2008,YAMADA2005,Mai2008,Dunn2014,Viasnoff2004,ZALLEN2004,Lecuit2004}, cell intercalation cannot be separated from the mechanical forces originating it and whose nature, spatial organization and dynamics are still largely unknown. In this article, we provide a topological insight into the mechanics of cell intercalation, by leveraging on recent advances toward deciphering orientational order in epithelial layers~\citep{Li2018,Pasupalak2020,Durand2019,Armengol2022a,Armengol2022b,Eckert2022,Cislo2023}. The latter originates from the cells' anisotropic shape and results in the emergence of liquid crystal phases collectively known as $p-$atics, with $p$ an integer reflecting the symmetry of the system under rotation by $2\pi/p$. The honeycomb structure of epithelial layers, in particular, has been shown to give rise to {\em hexatic} order (i.e. $p=6$) at length scales ranging from one to dozens of cells, depending on the cells' density and molecular repertoire, as well as the mechanical properties of the substrate~\citep{,Armengol2022a,Armengol2022b,Eckert2022}. In the language of liquid crystals, the cell-wide morphological transformations underlying T1 processes can be described in terms of topological defects known as {\em disclinations}: i.e. point-like singularities in the otherwise regular configuration of a continuous $p-$atic order parameter -- i.e. $\Psi_{p}=\langle e^{ip\vartheta} \rangle$, with $\vartheta$ the orientation of the individual building blocks and $\langle\cdots\rangle$ the ensemble average~\citep{Giomi2022a,Giomi2022b} -- around which the average cellular orientation rotates by $2\pi s$, with $s=\pm 1/p,\,\pm 2/p\ldots$ the winding number or ``strength'' of the defect. In passive two-dimensional matter, disclinations mediate the transition from solid to liquid via a process known as Kosterlitz-Thouless-Halperin-Nelson-Young (KTHNY) melting scenario~\citep{Kosterlitz1972,Kosterlitz1973,Nelson1979,Young1979,Kosterlitz2016}. According to this, the hierarchical unbinding of neutral defect complexes -- i.e. for which $\sum_{i}s_{i}=0$ -- renders the system progressively more disordered. Our working hypothesis is that the competition between active and passive forces drives a similar unbinding mechanism in epithelial layers. In the following, we clarify the various steps and possible outcomes of this process and test this hypothesis against both hydrodynamic and cell-resolved numerical simulations.

\section{Results}

\subsection{Cell intercalation and T1 cycle}

A typical cell intercalation is illustrated in Fig.~\ref{Fig:fig1}a for a cluster of four cells, hereafter referred to as {\em primary} cell cluster. A T1 occurs when the internal $3-$fold coordinated vertices shrink until merging into a $4-$fold vertex, and then split along the orthogonal direction. Such an {\em internal} T1, however, leaves the number and, importantly, the positions of {\em external} vertices of each cell unchanged. Thus, despite this concept having received little attention in the literature (see, e.g., Refs.~\citep{Julicher2010,Fletcher2014,Rauzi2020,Julicher2022,Sknepnek2023,Jain2023}), it is impossible to achieve collective migration by means of {\em isolated} T1 processes. 

%To make progress, here we introduce the notion of T1 {\em avalanche}: i.e. a series of T1 processes involving the external vertices of the primary cluster. 
A full cell intercalation then consists of a burst of internal and a peripheral T1 processes, as is schematically summarized in Fig.~\ref{Fig:fig1}a. These steps do not necessarily occur in a sequential order, but are most often simultaneous. Furthermore, since collective migration is a phenomenon that occurs roughly homogeneously across the entire cell layer, there is no single initial T1, but a uniform distribution of seeds. %, from which avalanches emanate. 
Crucially, internal T1 processes do {\em not} always trigger %a T1 avalanche and eventually 
a full intercalation. After the first T1, where the internal $3-$fold vertices shrink to a $4-$fold vertex, the cluster may reverse its dynamics and return to its original configuration (see Fig.~\ref{Fig:fig1}b). In the following, we refer to this scenario as T1 {\em cycle}: i.e. a direct T1 followed by an inverse T1, which does not permanently alter the configuration of the honeycomb network. Intuitively, and as we show next, whether the initial T1 triggers %an avalanche
a full intercalation, hence collective migration, or a cycle, is determined by the configuration of contractile stresses exerted by the cells, which, in turn, are modulated by the local geometry of the primary cluster.

% Figure environment removed

As a starting point, we focus on the intermediate configuration of primary cell cluster comprising two orthogonal pairs of pentagonal and heptagonal cells, as shown in the central column of Fig.~\ref{Fig:fig1}. This configuration, which corresponds to the most elementary short-ranged excitation of a honeycomb network, can be described in the language of topological defects as a quadrupole of $\pm 1/6$ disclinations. As in two-dimensional melting, such a defect-structure can arise spontaneously, as consequence of spatiotemporal fluctuations of physical or biological nature. The two complementary remodelling events following the intermediate configuration, i.e. cell intercalation (Fig.~\ref{Fig:fig1}a) and  the T1 cycle (Fig.~\ref{Fig:fig1}b), correspond instead to the two possible ``fates'' of this initial excitation. Upon unbinding [Figs.~\ref{Fig:fig1}a-(i) and \ref{Fig:fig1}b-(i)], the four defects comprising the quadrupole can further unbind into two $\pm 1/6$ disclination dipoles [Fig.~\ref{Fig:fig1}a-(ii)], or annihilate [Fig.~\ref{Fig:fig1}b-(ii)], thereby restoring the initial defect-free configuration. As we demonstrate next, the former scenario corresponds to cell intercalation and the latter to the T1 cycle. To this end, we identify three {\em geometric} requirements that a model of cell intercalation must fulfill, regardless of the desired level of biophysical accuracy. {\em 1)} The average orientation of the cells must rotate by $\pi/6$ with respect to initial configuration. {\em 2)} Both the primary cluster and its surrounding cells, must perform a local {\em convergent extension}: i.e. move inward in one direction, and outward along the orthogonal one~\citep{Skoglund2008,Blanchard2017,Wang2020,IoratimUba2023}. We stress that the adjective {\em local}, is used here to distinguish this process from convergent extension as intended in developmental biology, where the same rearrangement occurs at the scale of the entire organism. {\em 3)} In order to remodel the external vertices and initiate cell intercalation, the primary cluster must undergo a spontaneous shear deformation. 

In the following, we show that our construction not only fulfills these requirements, but, harnessing the predictive power of active hydrodynamics, provides readily testable experimental predictions. To this end, we numerically integrate the hydrodynamic equations of active hexatic liquid crystals, introduced by Armengol-Collado {\em et al}. in Ref.~\citep{Armengol2022b} (see Methods). To follow the fate of the primary cell cluster after an internal T1, we assume the cells to be initially horizontally oriented, so that the phase $\theta$ of the hexatic order parameter, $\Psi_{6}=|\Psi_{6}|e^{6i\theta}$, is $\theta=0$, and construct a configuration featuring a quadrupole of $\pm 1/6$ disclinations [see Figs.~\ref{Fig:fig2}a-(i) and \ref{Fig:fig2}b-(i)]. Along one {\em full} loop encircling each of these defects, $\theta$ changes by $\pm\pi/3$, with the sign reflecting that of the defect's winding number. Since the external boundary of the primary cluster consists of four {\em half} loops, this implies that $\theta$ varies in the range $-\pi/6 \le \theta \le \pi/6$ around the quadruple, as indicated by the alternating blue and red tones in Fig.~\ref{Fig:fig2}a-(ii) and \ref{Fig:fig2}b-(ii). Once the defect quadrupole breaks into two dipoles, this new orientation propagates from the boundary of the primary cluster into the space between the dipoles [see Fig.~\ref{Fig:fig2}a-(iii)], while leaving the orientation of the cells in the exterior essentially undistorted. Thus, the unbinding of a $\pm 1/6$ defect quadrupole from a defect-free configuration and its break up into two dipoles drives a $\pi/6$ rotation of the cells between the dipoles [see Fig.~\ref{Fig:fig2}a-(iv)], consistently with our first requirement. Conversely, if defects annihilate [see Fig.~\ref{Fig:fig2}b-(iii)], the cells' initial orientation is restored after a transient orientational fluctuation [see Fig.~\ref{Fig:fig2}b-(iv)]. 

% Figure environment removed

In order to address the second and third requirements, we look at the configuration of the velocity field $\bm{v}$, corresponding to the average velocity of the cells in the surroundings of the primary cluster. As well documented in the theoretical~\citep{Giomi2014,Giomi2015,Hoffmann2020} and experimental~\citep{Saw2017,Kawaguchi2017,BlanchMercader2018,Balasubramaniam2021,Yashunsky2022} literature of active nematic liquid crystals, the distortion induced by topological defects drives a flow, whose structure and direction is determined by the defect's strength and the magnitude of the active stresses collectively exerted by the cells. Immediately after unbinding, an approximated expression for the velocity of the flow caused by the defect quadrupole can be analytically calculated (see Methods). Calling $\bm{r}=|\bm{r}|(\cos\phi\,\bm{e}_{x}+\sin\phi\,\bm{e}_{y})$ the distance from the center of the primary cluster and $\ell$ the cluster's size, this is given by 
\begin{equation}\label{eq:velocity}
\bm{v} \approx \frac{120\alpha_{6}\ell^{4}}{\eta} \left[\left(4-6\log{\frac{\bm{|r|}}{\ell}}-3\,\frac{\bm{|r|}^2}{\ell^{2}}\right) \cos6\phi \right] \frac{\bm{r}}{|\bm{r}|^{8}}\;.
\end{equation}
where $\alpha_{6}$ a constant, with dimensions of force over volume, embodying the active stresses exerted by the cells and modulated by the local hexatic order, and $\eta$ the shear viscosity (see Methods for details). Thus, in close proximity the primary cluster, where $|\bm{r}|/\ell \gtrapprox 1$, Eq.~\eqref{eq:velocity} gives $v_{x}(x,0) \sim \alpha_{6}/(\eta \ell^{4})\,x$ and $v_{y}(y,0) \sim - \alpha_{6}/(\eta\ell^{4})\,y$. In agreement with our second requirement, this is typical structure of a {\em stagnation flow}, whose realization at the cellular scale is a local convergent extension. We stress that, consistently with the cooperative and mesoscale nature of cell intercalation, Eq.~\eqref{eq:velocity} cannot be obtained from the mere superposition of the flows individually sourced by the defects, as a consequence of the non-linear dependence of the order parameter on the average cellular orientation. Finally, we notice that the specific direction of motion -- i.e. the sign of $\bm{v}$ -- depends solely on $\alpha_{6}$, which, in turn, is either positive or negative depending on whether the active stresses exerted within the cell layer are respectively \textit{contractile} or \textit{extensile}.

As the cellular layer starts remodelling and the defects comprising the quadrupole migrate away from their original position, a solution of the hydrodynamic equations becomes analytically inaccessible, but can be obtained from a numerical integration of the hydrodynamic equations and is displayed in Figs.~\ref{Fig:fig2}a and \ref{Fig:fig2}b, for two different $\alpha_{6}$ values. When $\alpha_{6}$ is large and {\em negative}, the quadrupole splits into two $\pm 1/6$ dipoles moving away from each other at an angle of approximately $5\pi/6$ (see Methods for details). Such a shear deformation is further enhanced by the coupling between hexatic order and flow, which, in a way not dissimilar to flow alignment effects in nematic liquid crystals, drives a rotation of the local orientation~\citep{DeGennes1993}. This biases the unbinding dynamics of the defect dipoles, thereby setting, in concert with the passive Coulomb-like forces at play, the direction along which the dipoles move away from each other~\citep{Krommydas2023}. Finally, switching off active stresses -- i.e. $\alpha_{6}=0$, shown in Fig.~\ref{Fig:fig2}b -- suppresses both defect unbinding and shear. Consistently, inverting the direction of active forces from extensile to contractile -- i.e. $\alpha_{6}>0$, see Methods -- results in a speed up of defects annihilation, hence of the T1 cycle, but never lead to the unbinding of $\pm 1/6$ pairs, thus to the onset of cell intercalation.

We conclude this section by stressing that the scenario emerging from our hydrodynamic analysis, of which Eq.~\eqref{eq:velocity} represents a central outcome, is further corroborated by recent numerical work by Erdemci-Tandogan {\em et al.}~\citep{Manning} and Das {\em et al}.~\citep{MaxBi}, who, using a different cell-resolved model of epithelia -- i.e. the Vertex model~\citep{honda1980} -- showed that the rate of T1 processes enhances the fluidity of tissues. Consistently, the speed of the stagnation flow sourced by cell intercalation increases like $\eta^{-1}$, indicating that, the faster cells intercalate, the smaller $\eta$, the more fluid is the epithelial layer.

\subsection{Collective Cell Migration}

Having demonstrated the viability of our hydrodynamic approach, we next outline a number of general predictions, the most striking of which is that collective epithelial migration, as it results from cell intercalation, is a process of activity-guided defect unbinding. To this end, we perform numerical simulations of the multiphase field model (MPF)~\citep{Loewe2020}, which have been proved to capture various aspects of epithelial organization, including the recently found hexanematic multiscale order~\citep{Armengol2022a,Armengol2022b,Eckert2022}. In this approach, cells are modeled as droplets of immiscible fluid phases, undergoing a persistent random walk. Each cell is characterized by a nominal propulsion speed $v_0$, equal for all cells, and a fluctuating direction of motion, with rotational diffusion coefficient $D_r$. A detailed description of the MPF model and numerical details is provided in Methods. In order to test the correlation between topological defects, T1 processes and collective migration, we fix the speed $v_{0}$ and the total number of cells and vary the persistence length of the cells trajectories by tuning the rotational diffusion coefficient $D_{r}$. Fig.~\ref{Fig:fig3}a,b show a typical configuration of MPF simulations, colored according to the normalized hexatic and nematic longitudinal stress, respectively, experienced by cells in the tissue. Note that the negative value of the longitudinal stress corresponds to extensile stresses (see the Methods section for details).

To track topological defects, we first polygonize the cells by detecting their contour and mark the vertices, as shown in Fig.~\ref{Fig:fig3}c. For each cell, we then compute the {\em shape function}
\begin{equation}
\gamma_{6} = \frac{\sum_{v=1}^{V} |\bm{r}_{v}|^{6} e^{6i\phi_{v}}}{\sum_{v=1}^{V} |\bm{r}_{v}|^{6}}\;,
\label{eqn:DefinitionOrderParameter}
\end{equation}
where $\bm{r}_{v}$, with $v=1,\,2\ldots\,V$ is the positions of the $v-$th vertex with respect to the centroid of the polygon and $\phi_{v}=\arctan(y_{v}/x_{v})$ its orientation with respect to the $x-$axis. As demonstrated in Ref.~\citep{Armengol2022b}, the phase $\Arg(\gamma_{6})/6$ of the shape function identifies the $6-$fold orientation of a cell and is represented as white six-legged stars in Fig.~\ref{Fig:fig3}d. The shape function is then coarse-grained over the length scale $R=1.5R_{\rm cell}$ to reconstruct the {\em shape parameter} $\Gamma_6=\langle \gamma_{6} \rangle_{R}$. Finally, topological defects are identified as singularities in the otherwise smoothly varying field $\Gamma_{6}=\Gamma_{6}(\bm{r})$. T1 processes, conversely, are readily detected by tracking those events leading to a recombination of the change of neighbors of a given cell (see Figs.~\ref{Fig:fig3}c). 

% Figure environment removed

For each $D_{r}$ value, we reconstruct the probability of finding a T1 at a given distance from a defect (see Figs.~\ref{Fig:fig3}c and \ref{Fig:fig3}d) and compare it with that of an arbitrary cell. Both distributions are approximately Gaussian and have vanishing mean if expressed in terms of the signed distance $\Delta x$ (see Fig.~\ref{Fig:fig3}e). Furthermore, for larger $D_{r}$ values, where the short correlation time renders the system more disordered and dense of defects, the two probability distributions are hardly distinguishable, with the probability of finding a defect in proximity of a T1 only slightly larger than that associated with an arbitrary cell. As $D_{r}$ is decreased also the density of defects decreases and the latter probability distribution becomes flatter and flatter, while the former remains unchanged, thereby confirming that T1 processes are {\em de facto} a realization of hexatic defect unbinding. To correlate structure and dynamics we show in Fig.~\ref{Fig:fig3}f, the mean squared displacement of T1 cells versus the local defect density. Our data confirm the existence of two different classes: i.e. ``slow'' and ``fast'' cells, respectively denoted by purple and orange tones. While for both classes of cells the mean squared displacement is roughly uniform for all values of the local defect density, fast cells only appear where the density is higher, thus providing an alternative signature of cell intercalation and T1 cycles. Cells undergoing a T1 cycle oscillate about their initial position, but do not participate to collective migration and, therefore, exhibit small mean square displacement. By contrast intercalating cells drive collective migration and, based on the correspondence identified here, can only be found in regions of high defect densities. Note that the difference in the mean square displacement of cells undergoing intercalation is at least and order of magnitude larger than that of cells affected by an isolated T1 cycle (Fig.~\ref{Fig:fig3}f). Hence, although isolated internal T1's (and by extension T1 cycles) can have small long-ranged effects, those effects are negligible in comparison the collective long-ranged motion induced by a full cell intercalation.

Finally, following Refs.~\citep{MaxBi,Manning} we study the temporal correlations of T1 cycles and cell intercalation. Fig.~\ref{Fig:fig3}g) shows a plot of the average time between two intercalation events and the average period of T1 cycles, for varying rotational diffusion coefficients $D_{r}$. The former is a monotonically decreasing function of $D_{r}$, indicating that rotational noise improves the performance of cell intercalation thus favoring a faster collective migration. Conversely, the dynamics of T1 cycles is essentially unaffected by rotational diffusion. Both behaviors can be readily rationalized on the basis of our hydrodynamic description. As previously observed in relation to Fig.~\ref{Fig:fig3}e, rotational noise favors the proliferation of hexatic defects, hence the intercalation rate. Conversely, being defect annihilation driven by the passive attractive forces resulting from the entropic elasticity of the hexatic phase, once a T1 cycle is initiated, its duration is essentially independent on the noise strength.

Before concluding, we discuss three especially striking aspects of collective cell migration highlighted by our approach and amenable to experimental scrutiny. First, from the active flow given by Eq.~\eqref{eq:velocity} and neglecting irrelevant inertial effects, one can estimate the magnitude of the active forces at play: i.e. $F_{\rm active} \sim \alpha_{6}/\ell^{3}$, where $\ell$ denotes the range of the distortion caused by the unbinding dislocations. Such a scaling form results primarily from the $6-$fold structure of cellular forces under the assumption -- still to be verified in hexatic epithelial layers, but consistent with previous observations on nematic cell cultures~\citep{Duclos2016,Saw2017,Kawaguchi2017,BlanchMercader2018,Yashunsky2022} --  that $\alpha_{6}$ is, at least approximately, spatially uniform. Second, in order for these forces to prompt defect unbinding, thereby triggering cell intercalation, they must overcome the passive Coulomb-like forces driving their annihilation. At the length scale of the primary cluster, the magnitude of the latter is roughly given by $F_{\rm passive} \sim L_{6}/\ell$, with $L_{6}$ the orientational stiffness of the hexatic phase (see Methods). Equating $F_{\rm active}$ and $F_{\rm passive}$ provides an estimate of the typical size of the primary cluster at which the cellular layer becomes unstable to collective migration: i.e. $\ell_{6}=\sqrt{|\alpha_{6}|/L_{6}}$. This active hexatic length scale was identified in Ref.~\citep{Armengol2022b} and is believed to play a role analog to that of the active nematic length scale $\ell_{2}=\sqrt{L_{2}/|\alpha_{2}|}$ \citep{Giomi2015}. The latter is the fundamental parameter controlling the collective behavior of active nematic liquid crystals and, depending on how it compares with other extrinsic and intrinsic length scales, determines the hydrodynamic stability~\citep{Duclos2018} of active nematics, the distribution of the vortex area in chaotic cytoskeletal flows~\citep{Guillamat2017,Lemma2019}, the size of nematic domains in bacterial colonies~\citep{You2018} and {\em in vitro} cultures of spindle-like cells~\citep{BlanchMercader2018} etc. Similarly, we expect that confining epithelia at length scales smaller than $\ell_{6}$ has the effect of suppressing cell intercalation and possibly render necessary different locomotion strategies and possibly favoring a switch to mesenchymal phenotypes. Lastly, in agreement with recent experimental evidence~\citep{Saw2017,BlanchMercader2018,Balasubramaniam2021}, our analysis shows that only an {\em extensile} activity can fuel collective migration in confluent epithelia and, leveraging on the familiar language of topological defects, it further provides a simple key to rationalizing the mechanical advantage of such a biological strategy: i.e. only extensile forces can provide the type of repulsive interactions that are necessary for defects to unbind, thus to trigger cell intercalation.

\section{Discussion}

In conclusion, we have investigated the physical mechanisms behind cell intercalation in confluent epithelial layers using analytics, and a combination of continuum and discrete modelling. After having established that cell intercalation, hence collective migration, requires a burst of correlated T1 processes. we demonstrated how the latter originates from the unbinding of neutral quadrupoles of hexatic disclinations. As in the KTHNY melting scenario~\citep{Nelson1979}, the latter are spontaneously generated by fluctuations and can either annihilate, thereby restoring the original configuration (T1 cycles), or further unbind in two pairs of $\pm 1/6$ disclinations, decreasing the translational and orientation order and increasing its fluidity. As these pairs move away from each other by activity, they stir and shear the cellular network, thereby initiating further T1 processes, which, cooperatively, give rise to cell migration. Our theory sheds light on the structure of the cellular forces driving collective migration in epithelia, suggests the possibility of a confinement-induced switch to mesenchymal phenotypes and provides an explanation of the observed extensile activity of {\em in vitro} epithelial layers~\citep{Saw2017,BlanchMercader2018,Balasubramaniam2021}. 

\section{Methods}

\subsection{Hydrodynamic equations of epithelial layers}

Our hydrodynamic equations of epithelial layers have been given in Ref.~\citep{Armengol2022b} and account for both hexatic and nematic order, with the former being dominant at short and the latter at long length scales. Since cell intercalation occurs at small length scales, here we can ignore nematic order and focus solely on the hydrodynamics of the hexatic phase. In addition to the standard density $\rho$ and velocity $\bm{v}$, this can be described in terms of the $6-$fold order parameter tensor $\bm{Q}_{6} = 4 \left\llbracket \langle \bm{\nu}^{\otimes 6}\rangle \right\rrbracket=4|\Psi_{6}|\left\llbracket \bm{n}^{\otimes 6}\right\rrbracket$, where $\bm{\nu}=\cos\vartheta\,\bm{e}_{x}+\sin\vartheta\,\bm{e}_{y}$ and $\bm{n}=\cos\theta\,\bm{e}_{x}+\sin\theta\,\bm{e}_{y}$ are respectively the fluctuating and average orientation of the hexatic building blocks and the operator $\left\llbracket \cdots \right\rrbracket$ renders its argument traceless and symmetric (see Refs.~\citep{Giomi2022a,Giomi2022b} for a general introduction to $p-$atic hydrodynamics). The short scale hydrodynamic equations of the epithelial layer are then given, in the most generic form, by
\begin{subequations}\label{eq:hydrodynamics}
\begin{gather}
\frac{D\rho}{Dt}+\rho\nabla\cdot\bm{v} = (k_{\rm d}-k_{\rm a})\rho\;,\\
\rho\,\frac{D\bm{v}}{Dt} = \nabla\cdot\bm{\sigma}+\bm{f}\;,\\
\frac{D\bm{Q}_{6}}{Dt} = \Gamma_{6}\bm{H}_{6} + 6 \big \llbracket \bm{Q}_{6}\cdot\bm{\omega} \big \rrbracket + \lambda_{6} \big\llbracket \nabla^{\otimes 4}\bm{u} \big\rrbracket + \Bar{\lambda}_6 \mbox{tr}(\bm{u}) \bm{Q}_6 + \nu_{6} \big \llbracket \bm{u}^{\otimes 3} \big \rrbracket\;.
\end{gather}	
\end{subequations}
with $D/Dt=\partial_{t}+\bm{v}\cdot\nabla$ is the material derivative. In Eq.~(\ref{eq:hydrodynamics}a), $k_{\rm d}$ and $k_{\rm a}$ the cell division and apoptosis rates, here assumed equal. For simplicity, we also assume uniform density throughout the system to that Eq.~(\ref{eq:hydrodynamics}a) reduces to the standard incompressibility condition $\nabla\cdot\bm{v}=0$. In Eq.~(\ref{eq:hydrodynamics}b) $\bm{\sigma}$ is the total stress tensor and $\bm{f}$ an external body force. In Eq.~(\ref{eq:hydrodynamics}c), $\bm{u}=[\nabla\bm{v}+(\nabla\bm{v})^{\intercal}]/2$ and $\bm{\omega}=[\nabla\bm{v}-(\nabla\bm{v})^{\intercal}]/2$, with $\intercal$ indicating transposition, are respectively the strain rate and vorticity tensors and entail the coupling between hexatic order and flow, with $\lambda_{6}$ and $\nu_{6}$ material constants and $\left( \nabla^{\otimes n} \right)_{i_{1} i_{2} ... i_{n}} = \partial_{i_{1}} \partial_{i_{2}}\ldots\,\partial_{i_{n}}$.  Because of incompressibility, the term $\Bar{\lambda}_6 \mbox{tr}(\bm{u}) \bm{Q}_6 $ in Eq.~(\ref{eq:hydrodynamics}c) vanishes. The tensor $\bm{H}_{6}=-\delta F/\delta\bm{Q}_{6}$ is the hexatic analog of the molecular tensor, dictating the relaxation dynamics of the order parameter tensor toward the minimum of the orientational free energy 
\begin{equation}
F = \int {\rm d}A\,\left(\frac{L_{6}}{2}\,|\nabla \bm{Q}_{6}|^{2}+\frac{A_{6}}{2}\,|\bm{Q}_{6}|^{2}+\frac{B_{6}}{4}\,|\bm{Q}_{6}|^{4}\right)\;,
\end{equation}
where $|\cdots|^{2}$ is the Euclidean norm and is such that $|\bm{Q}_{6}|^{2} = |\Psi_{6}|^{2}/2$. The constant $L_{6}$ is the order parameter stiffness, while $A_{6}$ and $B_{6}$ are phenomenological constants setting the magnitude of the coarse-grained complex order parameter at equilibrium: $|\Psi_{6}| = |\Psi_{6}^{(0)}| = \sqrt{-2 A_{6}/B_{6}}$, when $H_{i_{1}i_{2}\cdots\,i_{6}}=0$.

The stress tensor figuring in Eq.~(\ref{eq:hydrodynamics}b) can be customarily decomposed in a passive and an active contributions: i.e. $\St=\Sp+\Sa$.The passive stress, in turn, can be expressed as $\Sp=-P\mathbb{1}+\Sv+\Se+\Sr$, where $P$ is the pressure and $\Sv=2\eta\traceless{\bm{u}}$ the viscous stress, with $\eta$ the shear viscosity. The tensor $\sigma_{ij}^{({\rm e})} = -L_{6}\partial_{i}\bm{Q}_{6}\odot\partial_{j}\bm{Q}_{6}$ is the elastic stress, arising in response of a static deformation of a fluid patch and the symbol $\odot$ indicates a contraction of all matching indices of the two operands yielding a tensor whose rank equates the number of unmatched indices (two in this case). Finally $\Sr=-\lambda_{6}\nabla^{\otimes 4}\odot\bm{H}_{6}+3\left(\bm{Q}_{6}\cdot\bm{H}_{6}-\bm{H}_{6}\cdot\bm{Q}_{6}\right)$ is the reactive stress tensor, which embodies the conservative forces arising in response to flow-induced distortions of the hexatic orientation. The active stress tensor $\Sa$ was introduced in Ref.~\citep{Armengol2022b} on the basis of phenomenological and microscopic arguments and is given by $\Sa=\alpha_{6}\nabla^{\otimes 4}\odot\bm{Q}_{6}$, with $\alpha_{6}$ a constant. 
 
\subsection{Multiphase field Model for Epithelial Tissues}

\subsubsection{\label{app:B1}The model}

The multiphase field model is a cell-resolved model where each cell is described in a two-dimensional space by a concentration field $\varphi_{c}=\varphi_{c}(\bm{r})$,  with $c=1,\,2\ldots\,N_{\rm cell}$, and $N_{\rm cell}$ the total number of cells in the systems~\citep{Loewe2020,Monfared2023}. The equilibrium state is defined by the free energy $\mathcal{F}=\int {\rm d}A f$, where the free energy density $f$ is 
\begin{equation}\label{eqn:freeenergy_multiphase}
f 
= \frac{\alpha}{4} \sum_{c} \varphi_{c}^{2}(\varphi_{c}-\varphi_{0})^{2} 
+ \frac{k_{\varphi}}{2}\sum_{c}|\nabla\varphi_{c}|^{2} 
+ \epsilon\sum_{c<c'}\varphi_{c}^{2}\varphi_{c'}^{2} 
+ k_{\rm ad} \sum_{c<c'} \nabla \varphi_{c}\cdot \nabla \varphi_{c'}
+ \sum_{c} \lambda\left(1-\frac{1}{\pi \varphi_{0}^{2}R_{\rm cell}^{2}} \int {\rm d}A\,\varphi_{c}^{2}\right)^{2}\;.
\end{equation}
Here $\alpha>0$ and $k_\varphi>0$ are material parameters which can be used to tune the surface tension $\sigma = \sqrt{8k_\varphi \alpha}$ and the interfacial thickness $\xi=\sqrt{2 k_\varphi /\alpha}$ of isolated cells and thermodynamically favor spherical cell shapes, so that the concentration field is large (i.e. $\varphi_{c} \approx \varphi_{0}$) inside the cells and zero outside.  The repulsive bulk term proportional to $\epsilon>0$ captures the fact that cells cannot overlap, while the term proportional to $k_{\rm ad}>0$ models the interfacial adhesion between cells, ultimately favoring tissue confluency in a crowded environment~\citep{Monfared2023}. The term proportional to $\lambda>0$ forces the cells' area around its nominal value $\pi R^2_{\rm cell}$, with $R_{\rm cell}$ the preferential cell radius. The phase field $\varphi_c$ evolves according to the Allen-Cahn equation
\begin{equation}\label{eqn:allen_cahn}
\partial_{t} \varphi_{c}+\bm{v}_{c} \cdot \nabla \varphi_{c} = - M\,\frac{\delta \mathcal{F}}{\delta \varphi_{c}}\;,
\end{equation}
where $\bm{v}_{c}=v_{0}(\cos\theta_{c}\,\bm{e}_{x}+\sin\theta_{c}\,\bm{e}_{y})$ is the velocity at which the $c-$th cell self-propels, with $v_{0}$ a constant speed and $\theta_{c}$ the angle defining the nominal direction of cell migration.  The latter evolves according to the stochastic equation
\begin{equation}
\frac{{\rm d}\theta_{c}}{{\rm d}t} = \eta_{c}\;,
\end{equation}
where $\eta_{c}$ is a noise term with correlation function $\langle \eta_{c}(t)\eta_{c'}(t') \rangle = 2D_{r}\delta_{cc'}\delta(t-t')$ and $D_{r}$ a constant controlling noise diffusivity. The constant $M$ in Eq.~\eqref{eqn:allen_cahn} is the mobility measuring the relevance of thermodynamic relaxation with respect to non-equilibrium cell migration. Eq.~\eqref{eqn:allen_cahn} is solved with a finite-difference approach through a predictor-corrector finite difference Euler scheme implementing second order stencil for space derivatives~\citep{Carenza2019}. 

We have integrated the dynamical equations with a number of cells $N_{\rm cells}=440$ in a system of size $384 \times 403$ with periodic boundary conditions. The model parameters values used in the simulations are as follows: $\alpha = 0.2$, $k_{\varphi} = 0.2$, $\epsilon = 0.1$, $k_{\rm ad}=0.005$, $\lambda=600$, $R_{\rm cell}=11.5$, $v_0=0.006$. The noise variance $D_{r}$ was varied in the range  $10^{-6} \le D_{r} < 8 \times 10^{-5}$. 

\subsubsection{Cell segmentation}
\label{app:B2}
In order to find the $6-$fold orientation of each cell, we proceeded to the cell segmentation of the simulated configuration. This procedure consists of the following steps. First, we define the thresholded density of the whole cell layer as
\begin{equation}\label{eq:thresholded_density}
\Phi = \sum_{c=1}^{N_{\rm cell}} \vartheta_{H}(\varphi_{c} - \varphi_{\rm th})\;,
\end{equation}
where $\vartheta_{H}$ is the Heaviside theta function such that $\vartheta_{H}(x)=1$ if $x \geq 0$ and $\vartheta_{H}(x)=0$ otherwise. Here $\varphi_{\rm th}$ is the threshold marking the boundary of each cell. In particular we choose $\varphi_{\rm th} = \varphi_0/2$. The resulting field is $\Phi=1$ inside each cell; $\Phi=2$ at the interface of two cells; $\Phi=3$ or $\Phi=4$ on the vertices. As at the interface the field $\varphi_c$ of each cell smoothly changes from $\varphi_0$ to $0$, therefore dropping below the threshold $\varphi_{\rm th}$, it is possible to find pixels where $\Phi=0$. These spurious features are adjusted by replacing Eq.~\eqref{eq:thresholded_density} with an average over the pixels neighboring that where $\Phi$ vanishes. That is
\begin{equation}
\Phi  = \sum_{\langle x,y \rangle} \sum_{c=1}^{N_{\rm cell}}\frac{\vartheta_{H}(\varphi_{c})}{\max \left\{ \sum_{\langle x,y \rangle} \vartheta_{H}(\varphi_{c}), 1 \right\}}\;,
\end{equation}
where $\sum_{\langle x,y \rangle}$ stands for a sum over the nearest neighbors of the pixel where $\Phi=0$. The procedure is reiterated until no change occurs in two consecutive iterations. Finally, upon segmenting the thresholded density, we identify the cell's vertices $\{ \bm{r}_{v} \}_{c}$, as those points where $\varphi_{\rm tot} > 2$. Tissue rearrangement events are identified by tracking changes in the list of neighbors of each cell.

\subsubsection{Cell orientation, coarse-graining and topological defects}
\label{app:B3}

The $6-$fold orientation of a cell is computed via the shape function $\gamma_{6}$ defined in Eq.~\eqref{eqn:DefinitionOrderParameter}, introduced in Ref.~\citep{Armengol2022a}. This construction is schematically illustrated in Fig.~\ref{Fig:ExtDatFig1}. The single-cell orientation can then be coarse-grained to construct a continuous description of the cellular tissue~\citep{Armengol2022a}. To do so, we use the shape order parameter $\Gamma_{6}=\Gamma_{6}(\bm{r})$, constructed upon averaging the shape function $\gamma_{6}$ of the segmented cells whose center of mass, $\bm{r}_{c}$, lies within a disk of radius $R$ centered in $\bm{r}$. That is
\begin{equation}
\Gamma_{6}(\bm{r}) = \dfrac{1}{N_{\rm disk}} \sum_{c=1}^{N_{\rm cell}} \gamma_{6}(\bm{r}_c) \vartheta_H (R - |\bm{r}-\bm{r}_c|)\;,
\label{eqn:explicit_cg}
\end{equation}
where $N_{\rm disk} = \sum_{c}  \vartheta_H (R - |\bm{r}-\bm{r}_c|)$ is the number of cells whose centers lie within the disk, and the coarse-graining radius is fixed to be $R=1.5 R_{\rm cell}$. 
% Figure environment removed 
We choose to sample the shape function on a square grid with grid-spacing equal to the nominal cell radius $R_{\rm cell}$. Topological defects are then identified computing the winding number along each unit cell:
\begin{equation}
s = \frac{1}{2\pi}\oint_{\square} {\rm d}\theta = \frac{1}{2\pi}\sum_{n=1}^{4}\left[\theta(\bm{r}_{n+1})-\theta(\bm{r}_{n})\right]\,{\rm mod}\,\frac{2\pi}{6}\;,
\end{equation}
where the symbol $\square$ denotes a square unit cell in the interpolation grid and $\theta=\Arg(\Gamma_{6})/6$ the phase of the shape order parameter.

\subsubsection{Nematic and Hexatic stress in MPF simulations}
To characterize the properties of intercellular stresses in MPF simulations, we show in Fig.~\ref{Fig:fig3}a,b the longitudinal hexatic,nematic stress in the tissue. This is computed by contracting the nematic and hexatic concentration gradient tensor for each cell $c$ ($\bm{\sigma}^{\rm (nem)}_c= - k_{\varphi} (\nabla \phi_c)^{\otimes 2}$, $\bm{\sigma}^{\rm (hex)}_c = - k_{\varphi} (\nabla \varphi_c)^{\otimes 6}$) with the cellular shape tensor of the corresponding order ($\bm{Q}_{c,p} = \bm{n}_{c,p}^{\otimes p}$, with $p=2,6$). Here, $\bm{n}_{c,p}=(\cos\theta_{c,p}, \sin\theta_{c,p})$ with $\theta_{c,p} = {\rm Arg}(\gamma_{c,p})/p$ the angular orientation of the complex shape function $\gamma_p$ relative to the $c-$th cell (see Fig.~\ref{Fig:ExtDatFig1}).
Therefore, the explicit expression of the longitudinal nematic and hexatic stress is given by
\begin{eqnarray}
\sigma^{\rm (nem)}_{\parallel} = \sum_{c=1}^{N_{\rm cell}} \bm{\sigma}^{\rm (nem)}_c : \bm{Q}_{c,2} = - k_{\varphi} \sum_{c=1}^{N_{\rm cell}} \sum_{i,j} (\partial_i \varphi_c)(\partial_j \varphi_c) (n_{2,c})_i (n_{2,c})_j \; , \\
\sigma^{\rm (hex)}_{\parallel} = \sum_{c=1}^{N_{\rm cell}} \bm{\sigma}_c^{\rm (hex)} : \bm{Q}_{c,6} = - k_{\varphi} \sum_{c=1}^{N_{\rm cell}} \sum_{i_1,\dots i_6} (\partial_{i_1} \varphi_c) \dots (\partial_{i_6} \varphi_c) (n_{6,c})_{i_1} \dots (n_{6,c})_{i_6} \; .
\end{eqnarray}
Notice that positive (negative) values of the longitudinal stress correspond to contractile (extensile) stresses for both nematic and hexatic contributions.

\subsubsection{Correlating cellular mean square displacement and defect density}

To build the scatter plot shown in Fig.~\ref{Fig:fig3}f correlating cellular mean square displacement and defect density, we proceed as follows. First, we divide the system in squared regions of size $\Delta=35$, containing roughly $3.5$ cells. Then we observe each of this regions for a time window of $25 \times 10^{3}$ iterations, tracking both the mean square displacement of the cells in each subregion and the mean defect density therein. Finally, we record these observation for each subregion and for each time window analyzed and we use these measurements as entries to build the scatter plot in Fig.~\ref{Fig:fig3}f.

\subsubsection{Analysis of characteristic time of cell intercalation and T1 cycles}

To analyze the temporal statistics of remodeling events we start differentiating these between T1 cycles and cell intercalation. T1 cycles are identified as events where cells first change their neighbors at time $t$ and then they return to their initial configuration at a later time $t+T$. Analogously, for cell intercalation events--leading to a permanent tissue remodeling--we measure the time interval $T$ between two consecutive intercalation events. The results of the analysis at varying the rotational diffusion $D_{r}$ are shown in Fig.~\ref{Fig:fig3}g. For each value of $D_{r}$ considered, we first identify cell intercalations and T1 cycles for each cell in the system, then we average the measured time intervals, obtaining the distribution of mean values, explicitly shown in the box plot in the inset of Fig.~\ref{Fig:fig3}g. The statistical analysis shows that the two populations (T1 cycles and intercalation events) are significantly different, with T1 cycles occurring faster than cell intercalation. We repeat this analysis at varying $D_{r}$, and we plot in Fig.~\ref{Fig:fig3}g the characteristic time obtained as the mean value of the cell intercalations and T1 cycles distributions. Importantly we find that, while the time interval between cell intercalation events sensibly depends on the rotational diffusion, the typical timescale of T1 cycles does not.

\subsection{Active flow of a hexatic defect quadrupole}

\subsubsection{Scalar order parameter}

In this section we provide a derivation of Eq.~\eqref{eq:velocity}. To this end, let $\Psi_{6}=|\Psi_{6}|e^{6i\theta}$ be the hexatic complex order parameter and consider a quadrupole of $\pm 1/6$ disclinations equidistantly placed from the center of the primary cluster. The phase $\theta=\theta(\bm{r})$ is then given by the convolution of the average orientation in the surrounding of each defect, that is
\begin{equation}
\label{eq:theta_convolution}
 \theta 
 = -\frac{1}{6} \arctan\left(\frac{y}{x-\ell}\right) 
 - \frac{1}{6} \arctan\left(\frac{y}{x+\ell}\right)
 + \frac{1}{6} \arctan\left(\frac{y+\ell}{x}\right) 
 + \frac{1}{6} \arctan\left(\frac{y-\ell}{x}\right)\;,
\end{equation}
where $\ell$ is distance from the center. The quadrupolar distance $\ell$ is by definition taken to be small compared to the size of the system. Thus, expanding Eq.~\eqref{eq:theta_convolution} for $|\bm{r}|/\ell \gg 1$, we obtain the simpler expression
\begin{equation}
\label{theta quad exp}
\theta = -\frac{2\ell^2\sin 2\phi}{3|\bm{r}|^2} + \mathcal{O}\left(|\ell/\bm{r}|^{6}\right)\;.
\end{equation}
Notice that the Taylor expansion features {\em only} the quadrupolar term of order $\ell^{2}$ and is exact up to $6-$th order in $\ell/|\bm{r}|$; i.e. the dipolar term, of order $\mathcal{O}(\ell/|\bm{r}|)$, and all other terms up to the $6-$th order vanish identically. 

This result is extremely robust, and can be derived in a number of ways. For instance, Eq.~\eqref{eq:theta_convolution} can be obtained from the solution of the Poisson equation
\begin{equation}\label{eq:poisson}
\nabla^{2}\varphi = \rho_{\rm d}
\end{equation}
where $\varphi$ is a dual field such that $\partial_{i}\theta=-\epsilon_{ij}\partial_{j}\varphi$ and the right-hand side of the Eq.~\eqref{eq:poisson} is analogous to the electrostatic charge density \citep{Chaikin1995}. At large distance from the defects, Eq.~\eqref{eq:poisson} can be solved by multipole expansion \citep{Jackson1999}, that is:
\begin{equation}
\varphi = a_{0}\log \frac{r_{0}}{|\bm{r}|}+\sum_{n=1}^{\infty}\frac{a_{n}\cos n\theta+b_{n}\sin n\theta}{|\bm{r}|^{n}}\;,
\end{equation}
where $r_{0}$ is an irrelevant length scale and $a_{n}$ and $b_{n}$ are coefficients given by
\begin{subequations}\label{multipole coefficients}
\begin{gather}
a_n= \frac{1}{n} \int {\rm d}A\,|\bm{r}|^n \cos{(n\phi)} \rho_{\rm d}\;, \\[5pt]
b_n= \frac{1}{n} \int {\rm d}A\,|\bm{r}|^n \sin{(n\phi)} \rho_{\rm d}\;.
\end{gather}
\end{subequations}
Thus, up to the quadrupole term, the expansion of $\varphi$ is given by
\begin{equation}\label{diretor expansion}
\varphi = a_0 \log{\frac{r_{0}}{|\bm{r}|}} + \frac{a_1 \cos{\phi} + b_1 \sin{\phi}}{|\bm{r}|}+
\frac{a_2 \cos{2\phi} + b_2 \sin{2\phi}}{|\bm{r}|^2} + \cdots
\end{equation}
As in electrostatics, the density $\rho_{\rm d}$ is given by
\begin{equation}
\label{defect density T1}
\rho_{\rm d} = \frac{1}{6} \Big[- \delta(\bm{r} - \ell\bm{e}_{x}) - \delta(\bm{r} + \ell\bm{e}_{x}) + \delta(\bm{r} - \ell\bm{e}_{y}) + \delta(\bm{r} + \ell\bm{e}_{y}) \Big]\;,
\end{equation}
where $\ell$ is again the distance from the center of the primary cluster. Now, because the defect quadrupole has, by construction, vanishing total strength and dipole moment, $a_{0}=0$ and $a_{1}=b_{1}=0$. Of the quadrupolar terms, on the other hand, $a_{2}=-  \ell^{2}/3$ and $b_{2}=0$, thus
\begin{equation}
\label{theta quad}
\varphi = - \frac{ \ell^{2}\cos{2\phi} }{3|\bm{r}|^2}\;.
\end{equation}
Finally, going from $\varphi$ to the original field $\theta$ one finds
\begin{equation}
\label{real theta quad?}
\theta = -\frac{2 \ell^2 \sin{2\phi} }{3|\bm{r}|^2}\;,
\end{equation}
thus confirming the expression given in Eq.~\eqref{theta quad exp}.

\subsubsection{Active Force}

To shed light on the structure of the cellular flow triggered by a T1 process, we solve the Stokes equation in the presence of an active force of the form $\fa=\nabla\cdot\Sa$, where 
\begin{equation}\label{active hexatic stress}
\Sa = \alpha_6 \nabla^{\otimes 4} \odot \boldsymbol{Q}_6,
\end{equation}
is the active hexatic stress tensor introduced in Ref.~\citep{Armengol2022b}. Calculating the divergence gives
\begin{equation}
\label{eq:active_force}
    \fa = 960\,\frac{\alpha_{6}\ell^{2}}{|\bm{r}|^{7}}
\Bigg\{ \Bigg[
-3\cos 7\phi+\frac{\ell^2}{|\bm{r}|^{2}}\left(3\cos 5\phi-14\cos 9\phi\right)
\Bigg]\bm{e}_{x}
+\Bigg[
3\sin 7\phi-\frac{\ell^2}{|\bm{r}|^{2}}\left(3\sin 5\phi-14\sin 9\phi\right)
\Bigg]\bm{e}_{y}
\Bigg\}\;,
\end{equation}
up to correction of order $\mathcal{O}(|\ell/r|^{6})$. A plot of the force field is shown Fig.~\ref{Fig:fig1_SI}a.

\subsubsection{Flow field}

To reconstruct the cellular motion generated by a T1 process, we solve the incompressible Stokes equation for the flow sourced the active force $\fa$: i.e.
\begin{subequations}
\begin{gather}
\eta\nabla^{2}\bm{v}-\nabla P + \fa = \bm{0}\;,\\[5pt]
\nabla\cdot\bm{v} = 0\;,
\end{gather}	
\end{subequations}
where $\eta$ is the shear viscosity and $P$ the pressure. To this end, we turn to the Oseen formal solution
\begin{equation}
\label{Oseenv}
\bm{v}(\bm{r})= \int_{0}^{2\pi}{\rm d}\phi'\int_{\ell}^{R} {\rm d}r'r'\,\bm{G}(\bm{r}-\bm{r}') \cdot \fa(\bm{r}')\;,
\end{equation}
where
\begin{equation}
\label{OseenTensor}
\bm{G}(\bm{r}) =
\frac{1}{4\pi \eta} \Bigg[\left(\log{\frac{\mathcal{L}}{|\bm{r}|}}-1\right)\mathbb{1} +
\frac{\bm{r}\otimes\bm{r}}{|\bm{r}|^2} \Bigg]\;,
\end{equation}
is the two-dimensional Oseen tensor (see e.g. Ref.~\citep{Giomi2015}), with $\mathcal{L}$ a constant, and $R$ is a large distance cut-off. Without loss of generality, one can set $\mathcal{L}= R\sqrt{e}$ in Eq.~\eqref{OseenTensor}. To calculate the integrals in Eq.~\eqref{Oseenv}, we make use of the logarithmic expansion
\begin{equation}
\label{logexp}
     \log{\frac{|\bm{r}-\bm{r}'|}{\mathcal{L}}} = \log{\frac{r_>}{\mathcal{L}}} - \sum^\infty_1 \frac{1}{m}\left(\frac{r_>}{r_<}\right)^m \cos{[m(\phi-\phi')]}\;,
\end{equation}
with $r_{\gtrless}$ the maximum (minimum) between $|\bm{r}|$ and $|\bm{r}'|$, and of the orthogonality of trigonometric functions 
\begin{equation}
\int^{2\pi}_0 \text{d}\phi'  \cos{[m(\phi-\phi')]}\cos{n\phi'} = \pi \cos{n\phi}~ \delta_{mn}\;.
\end{equation}
The resulting flow field surrounding the defect quadrupole is then given by 
\begin{align}
\frac{\bm{v}}{2\alpha_6 \ell^2/\eta}  
&=-\Bigg[\frac{30\left(\ell^2-\bm{|r|}^2\right)^2}{\bm{|r|}^{12}} \left(3 \bm{|r|}^2 \cos {8\phi} + 14 \ell^2 \cos{10\phi}\right)
+\frac{60 }{\bm{|r|}^{8}}\cos {6\phi} \left(3\bm{|r|}^2+6 \ell^2 \log \frac{\bm{|r|}}{\ell}-4 \ell^2\right)\Bigg]\bm{r}\notag\\[5pt]
&+6\left(\frac{6 }{\bm{|r|}^5}-\frac{5 \ell^2}{\bm{|r|}^{7}}\right) (\cos{5\phi}~\bm{e}_x-\sin{5\phi}~\bm{e}_y) 
+\frac{30}{7 } \left(\frac{6 \ell^2}{\bm{|r|}^7}-\frac{7}{\bm{|r|}^5}\right) (\cos{7\phi}~\bm{e}_x-\sin{7\phi}~\bm{e}_y)\notag\\[5pt]
&+\frac{35}{3} \ell^2 \left(\frac{8 \ell^2}{\bm{|r|}^9}-\frac{9}{\bm{|r|}^7}\right)  (\cos{9\phi}~\bm{e}_x-\sin{9\phi}~\bm{e}_y)\;.
\end{align}
Fig.~\ref{Fig:fig1_SI}b shows a plot of this flow, while Fig.~\ref{Fig:fig1_SI}c shows a plot of the short distance approximation given in Eq.~\eqref{eq:velocity} of the main text.
% Figure environment removed

\subsection{Numerical simulations of defect annihilation and unbinding}

\subsubsection{Numerical model and validation} 

The time-dependent flows shown in Fig.~\ref{Fig:fig2}a(i) and Fig.~\ref{Fig:fig2}b(i) are obtained by numerically integrating Eqs.~\eqref{eq:hydrodynamics}  using a vorticity-stream function finite difference scheme. All equations are discretized on a two-dimensional square grid of sizes $256 \times 256$ and $1024 \times 1024$ with periodic boundary conditions. For both grid sizes, the grid spacing is $\Delta x = \Delta y = 1$ and the time stepping $\Delta t = 0.1$. The validity of this numerical approach is benchmarked by many numerical studies on liquid crystals and active matter (see for example Refs.~\citep{Krommydas2023,Giomi2022a,Giomi2022b,Giomi2014}). In all simulations, we set: $\rho=1$, $\eta=1$, $L_6=0.5$, $A_6=-0.2$, $B_6=0.4$, $\Gamma_6 =1$ and $\lambda_6 = 1.11$. All parameters are expressed in the arbitrary units used in the numerical simulations. 

\subsubsection{Defect annihilation and unbinding}

To construct the initial configuration of $\Psi_{6}$, we set $\ell=7$, $|\Psi_{6}|=1$ and take $\theta$ as given in Eq.~\eqref{eq:theta_convolution} inside a disk of radius $R_{D}= 28$ and random outside.  We then thermalize this configuration by keeping the orientation of the order parameter in the disk fixed, and relaxing $\Psi_{6}$ everywhere else. This allows us to obtain a defect-free configuration where $|\Psi_6| \approx 1$ everywhere, except that close to the defect cores where $|\Psi_6| \approx 0$. Notice that, on a doubly periodical domain, $\sum_{i}s_{i}=0$. Therefore, no other topological defect is found at the end of such relaxation procedure. Simulations are carried out until the total free energy relative variation drops under $0.1 \%$ with respect to two consecutive iterations. This corresponds to a state where defects have annihilated, and the hexatic liquid crystal has achieved a smooth configuration everywhere in the simulation box. For both annihilation and unbinding numerical experiments, we scan $\alpha_{6}$ for a wide range of positive and negative values. For any negative values of activity, we obtain increasingly sheared versions of the flow pattern in Fig.~\ref{Fig:fig2}a(i). Similarly, for positive values of $\alpha_{6}$ we obtain increasingly sheared versions of the same flow pattern, but with the direction of the flow inverted. 

% Figure environment removed

\subsubsection{Active defect dipole annihilation: the origin of the unbinding}

In this section we provide a brief account of the annihilation dynamics of a pair of $\pm 1/6$ active hexatic defects, in which it is possible to recognize the fundamental mechanism driving defect unbinding. To this end, we place the defects on the $x-$axis at a distance of $\Delta x = 64$ and construct the initial configuration of the hexatic order parameter $\Psi_{6}=e^{6i\theta}$ by setting $\theta = \pm \arctan[y/(x\pm\Delta x/2)]$ inside a disk of radius $R_{D}= 5$ centred at the defect cores, and random outside. We thermalize this configuration by keeping the phase of the order parameter in the two disks fixed, while relaxing $\Psi_{6}$ everywhere else. As before, this procedure allows us to obtain a state where $|\Psi_{6}| \approx 1$ everywhere, except that close to the two defect cores where $|\Psi_{6}| \approx 0$. We use this as the initial state for our annihilation experiment. Simulations are carried out until defects have annihilated and the total free energy relative variation drops under $0.1 \%$ with respect to two consecutive iterations. The model parameters, expressed in lattice units, are again: $\Delta t=1$, $\rho=1$, $\eta=1$, $L_6=0.5$, $A_6=-0.2$, $B_6=0.4$, $\Gamma_6 =1$ and $\lambda_6 = 1.11$. 

Fig.~\ref{Fig:fig2_SI} shows the trajectories of the positive (red) and negative (blue) defects during annihilation, for three realizations of the activity parameter $\alpha_{6}$, that is $\alpha_{6}=0.1$ (contractile), $\alpha_{6}=0$ (passive) and $\alpha_{6}=-0.1$ (extensile). For contractile activity, the backflow sourced by the active stress, Eq.~\eqref{active hexatic stress}, annihilation is sped up with respect to the passive case. By contrast, for extensile activity, annihilation is delayed. The same effects leads to the break up of the quadrupole into two defect pairs, provided the repulsive forces introduced by the active flow overcome the attractive Coulomb-like forces between defects.

\section{Acknowledgments}

This work is supported by the  European Union via the ERC-CoGgrant HexaTissue and by Netherlands Organization for Scientific Research (NWO/OCW). L.~N.~C. acknowledges the support of the Postdoctoral EMBO Fellowship ALTF 353-2023. Part of this work was carried out on the Dutch national e-infrastructure with the support of SURF through the Grant 2021.028 for computational time. The computer codes used to produced the numerical data presented in the article are available upon request. 

\bibliography{Biblio.bib}

\end{document}
