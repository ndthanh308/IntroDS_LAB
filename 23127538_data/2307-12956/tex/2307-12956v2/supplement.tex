\documentclass[pre,superscriptaddress,onecolumn,notitlepage]{revtex4-1}
\pdfoutput=1
\usepackage[colorlinks=true,urlcolor=blue]{hyperref}
\usepackage[normalem]{ulem}
\usepackage[utf8]{inputenc}
\usepackage[english]{babel}
\usepackage[T1]{fontenc}
\usepackage{mathtools}
\usepackage{stmaryrd}
\usepackage{graphicx}
\usepackage{placeins}
\usepackage{amsfonts}
\usepackage{amssymb}
\usepackage{makecell}
\usepackage{rotating}
\usepackage{multirow}
\usepackage{booktabs}
\usepackage{arydshln}
\usepackage{amsmath}
\usepackage{xcolor}
\usepackage{amsthm}
\usepackage{xcolor}
\usepackage{bbold}
\usepackage{bm}

\definecolor{red}{rgb}{0.75,0,0}
\definecolor{blue}{rgb}{0,0,0.75}
\definecolor{green}{rgb}{0,0.5,0}

\newcommand{\red}[1]{{\color{red}#1}}
\newcommand{\blue}[1]{{\color{blue}#1}}
\newcommand{\Livio}[1]{{\color{blue}#1}}
\newcommand{\green}[1]{{\color{green}#1}}
\newcommand{\luca}[1]{{\color{blue} [{\bf LG:} #1]}}

\newcommand{\traceless}[1]{\left\llbracket #1 \right\rrbracket}

\renewcommand{\tablename}{Tab.}
\renewcommand{\figurename}{Fig.}
\renewcommand{\thetable}{S\arabic{table}}
\renewcommand{\thefigure}{S\arabic{figure}}
\renewcommand{\thesection}{S\arabic{section}}
\renewcommand{\theequation}{S\arabic{equation}}

\newcommand{\fa}{\bm{f}^{({\rm a})}}

\newcommand{\St}{\bm{\sigma}}
\newcommand{\Sp}{\bm{\sigma}^{({\rm p})}}
\newcommand{\Sa}{\bm{\sigma}^{({\rm a})}}
\newcommand{\Se}{\bm{\sigma}^{({\rm e})}}
\newcommand{\Sv}{\bm{\sigma}^{({\rm v})}}
\newcommand{\Sr}{\bm{\sigma}^{({\rm r})}}

\DeclareMathOperator{\tr}{tr}
\DeclareMathOperator{\Arg}{Arg}

\begin{document}

\title{Supplemental Information for: 
Collective epithelial migration is mediated by the unbinding of hexatic defects}
\author{Dimitrios Krommydas}
\affiliation{Instituut-Lorentz, Universiteit Leiden, P.O. Box 9506, 2300 RA Leiden, The Netherlands}
\author{Livio Nicola Carenza}
\affiliation{Instituut-Lorentz, Universiteit Leiden, P.O. Box 9506, 2300 RA Leiden, The Netherlands}
\author{Luca Giomi}
\affiliation{Instituut-Lorentz, Universiteit Leiden, P.O. Box 9506, 2300 RA Leiden, The Netherlands}
\date{\today}

\maketitle

\tableofcontents

\section{Active flow in the surrounding of a hexatic defect quadrupole}

\subsection{Scalar order parameter}

In this supplementary section we provide a derivation of Eq.~(1) in the main text. To this end, let $\Psi_{6}=|\Psi_{6}|e^{6i\theta}$ be the hexatic complex order parameter and consider a quadrupole of $\pm 1/6$ disclinations equidistantly placed from the center of the primary cluster. The phase $\theta=\theta(\bm{r})$ is then given by the convolution of the average orientation in the surrounding of each defect, that is
\begin{equation}
\label{eq:theta_convolution}
\theta 
 = -\frac{1}{6} \arctan\left(\frac{y}{x-\ell}\right) 
 - \frac{1}{6} \arctan\left(\frac{y}{x+\ell}\right)
 + \frac{1}{6} \arctan\left(\frac{y+\ell}{x}\right) 
 + \frac{1}{6} \arctan\left(\frac{y-\ell}{x}\right)\;,
\end{equation}
where $\ell$ is distance from the center. The quadrupolar distance $\ell$ is by definition taken to be small compared to the size of the system. Thus, expanding Eq.~\eqref{eq:theta_convolution} for $|\bm{r}|/\ell \gg 1$, we obtain the simpler expression
\begin{equation}
\label{theta quad exp}
\theta = -\frac{2\ell^2\sin 2\phi}{3|\bm{r}|^2} + \mathcal{O}\left(|\ell/\bm{r}|^{6}\right)\;.
\end{equation}
Notice that the Taylor expansion features {\em only} the quadrupolar term of order $\ell^{2}$ and is exact up to $6-$th order in $\ell/|\bm{r}|$; i.e. the dipolar term, of order $\mathcal{O}(\ell/|\bm{r}|)$, and all other terms up to the $6-$th order vanish identically. 

This result is extremely robust, and can be derived in a number of ways. For instance, Eq.~\eqref{eq:theta_convolution} can be obtained from the solution of the Poisson equation
\begin{equation}\label{eq:poisson}
\nabla^{2}\varphi = \rho_{\rm d}
\end{equation}
where $\varphi$ is a dual field such that $\partial_{i}\theta=-\epsilon_{ij}\partial_{j}\varphi$ and the right-hand side of the Eq.~\eqref{eq:poisson} is analogous to the electrostatic charge density~[52]. At large distance from the defects, Eq.~\eqref{eq:poisson} can be solved by multipole expansion~[53], that is:
\begin{equation}
\varphi = a_{0}\log \frac{r_{0}}{|\bm{r}|}+\sum_{n=1}^{\infty}\frac{a_{n}\cos n\theta+b_{n}\sin n\theta}{|\bm{r}|^{n}}\;,
\end{equation}
where $r_{0}$ is an irrelevant length scale and $a_{n}$ and $b_{n}$ are coefficients given by
\begin{subequations}\label{multipole coefficients}
\begin{gather}
a_n= \frac{1}{n} \int {\rm d}A\,|\bm{r}|^n \cos{(n\phi)} \rho_{\rm d}\;, \\[5pt]
b_n= \frac{1}{n} \int {\rm d}A\,|\bm{r}|^n \sin{(n\phi)} \rho_{\rm d}\;.
\end{gather}
\end{subequations}
Thus, up to the quadrupole term, the expansion of $\varphi$ is given by
\begin{equation}\label{diretor expansion}
\varphi = a_0 \log{\frac{r_{0}}{|\bm{r}|}} + \frac{a_1 \cos{\phi} + b_1 \sin{\phi}}{|\bm{r}|} + \frac{a_2 \cos{2\phi} + b_2 \sin{2\phi}}{|\bm{r}|^2} + \cdots
\end{equation}
As in electrostatics, the density $\rho_{\rm d}$ is given by
\begin{equation}
\label{defect density T1}
\rho_{\rm d} = \frac{1}{6} \left[- \delta(\bm{r} - \ell\bm{e}_{x}) - \delta(\bm{r} + \ell\bm{e}_{x}) + \delta(\bm{r} - \ell\bm{e}_{y}) + \delta(\bm{r} + \ell\bm{e}_{y}) \right]\;,
\end{equation}
where $\ell$ is again the distance from the center of the primary cluster. Now, because the defect quadrupole has, by construction, vanishing total strength and dipole moment, $a_{0}=0$ and $a_{1}=b_{1}=0$. Of the quadrupolar terms, on the other hand, $a_{2}=-  \ell^{2}/3$ and $b_{2}=0$, thus
\begin{equation}
\label{theta quad}
\varphi = - \frac{ \ell^{2}\cos{2\phi} }{3|\bm{r}|^2}\;.
\end{equation}
Finally, going from $\varphi$ to the original field $\theta$ one finds
\begin{equation}
\label{real theta quad?}
\theta = -\frac{2 \ell^2 \sin{2\phi} }{3|\bm{r}|^2}\;,
\end{equation}
thus confirming the expression given in Eq.~\eqref{theta quad exp}.

\subsection{Active Force}

To shed light on the structure of the cellular flow triggered by a T1 process, we solve the Stokes equation in the presence of an active force of the form $\fa=\nabla\cdot\Sa$, where 
\begin{equation}\label{active hexatic stress}
\Sa = \alpha_6 \nabla^{\otimes 4} \odot \boldsymbol{Q}_6,
\end{equation}
is the active hexatic stress tensor introduced in Ref.~[20]. Calculating the divergence gives
\begin{equation}\label{active force}
\fa = 960\,\frac{\alpha_{6}\ell^{2}}{|\bm{r}|^{7}}
\left\{
\left[
-3\cos 7\phi+\frac{\ell^2}{|\bm{r}|^{2}}\left(3\cos 5\phi-14\cos 9\phi\right)
\right]\bm{e}_{x}+
\left[
3\sin 7\phi-\frac{\ell^2}{|\bm{r}|^{2}}\left(3\sin 5\phi-14\sin 9\phi\right)
\right]\bm{e}_{y}
\right\}\;,
\end{equation}
up to correction of order $\mathcal{O}(|\ell/r|^{6})$. A plot of the force field is shown Fig.~\ref{Fig:fig1_SI}a.

\subsection{Flow field}

To reconstruct the cellular motion generated by a T1 process, we solve the incompressible Stokes equation for the flow sourced the active force $\fa$: i.e.
\begin{subequations}
\begin{gather}
\eta\nabla^{2}\bm{v}-\nabla P + \fa = \bm{0}\;,\\[5pt]
\nabla\cdot\bm{v} = 0\;,
\end{gather}	
\end{subequations}
where $\eta$ is the shear viscosity and $P$ the pressure. To this end, we turn to the Oseen formal solution
\begin{equation}
\label{Oseenv}
\bm{v}(\bm{r})= \int_{0}^{2\pi}{\rm d}\phi'\int_{\ell}^{R} {\rm d}r'r'\,\bm{G}(\bm{r}-\bm{r}') \cdot \fa(\bm{r}')\;,
\end{equation}
where
\begin{equation}
\label{OseenTensor}
\bm{G}(\bm{r}-\bm{r}') = \frac{1}{4\pi \eta} \left[\left(\log{\frac{\mathcal{L}}{|\bm{r}-\bm{r}'|}}-1\right)\mathbb{1}+\frac{(\bm{r}-\bm{r}')\otimes(\bm{r}-\bm{r}')}{|\bm{r}-\bm{r}'|^2} \right]\;,
\end{equation}
is the two-dimensional Oseen tensor (see e.g. Ref.~[45] main text), with $\mathcal{L}$ a constant, and $R$ is a large distance cut-off. Without loss of generality, one can set $\mathcal{L}= R\sqrt{e}$ in Eq.~\eqref{OseenTensor}. To calculate the integrals in Eq.~\eqref{Oseenv}, we make use of the logarithmic expansion
\begin{equation}
\label{logexp}
     \log{\frac{|\bm{r}-\bm{r}'|}{\mathcal{L}}} = \log{\frac{r_>}{\mathcal{L}}} - \sum^\infty_1 \frac{1}{m}\left(\frac{r_>}{r_<}\right)^m \cos{[m(\phi-\phi')]}\;,
\end{equation}
with $r_{\gtrless}$ the maximum (minimum) between $|\bm{r}|$ and $|\bm{r}'|$, and of the orthogonality of trigonometric functions 
\begin{equation}
\int^{2\pi}_0 \text{d}\phi'  \cos{[m(\phi-\phi')]}\cos{n\phi'} = \pi \cos{n\phi}~ \delta_{mn}\;.
\end{equation}
The resulting flow field surrounding the defect quadrupole is then given by 
\begin{equation}
\begin{split}
\label{fullv}
 \frac{\bm{v}}{2\alpha_6 \ell^2/\eta}  = &    -30\left[\left(\ell^2-\bm{|r|}^2\right)^2 \left(14 \ell^2 \cos{10\phi}+3 \bm{|r|}^2 \cos {8\phi} \right)+2 \bm{|r|}^4 \cos {6\phi} \left(-4 \ell^2+6 \ell^2 \log \frac{\bm{|r|}}{\ell}+3\bm{|r|}^2\right)\right]\frac{\bm{r}}{\bm{|r|}^{12}}
     \\ &
     +6 \left(\frac{6 }{\bm{|r|}^5}-\frac{5 \ell^2}{\bm{|r|}^{7}}\right) (\cos{5\phi}~\bm{e}_x-\sin{5\phi}~\bm{e}_y) + \frac{30}{7 } \left(\frac{6 \ell^2}{\bm{|r|}^7}-\frac{7}{\bm{|r|}^5}\right) (\cos{7\phi}~\bm{e}_x-\sin{7\phi}~\bm{e}_y)
     \\ &
     +\frac{35}{3} \ell^2 \left(\frac{8 \ell^2}{\bm{|r|}^9}-\frac{9}{\bm{|r|}^7}\right)  (\cos{9\phi}~\bm{e}_x-\sin{9\phi}~\bm{e}_y)\;.
\end{split}
\end{equation}
Fig.~\ref{Fig:fig1_SI}b shows a plot of this flow, while Fig.~\ref{Fig:fig1_SI}c shows a plot of the short distance approximation given in Eq.~(1) of the main text.

% Figure environment removed


% Figure environment removed

\section{Numerical simulations of defect annihilation and unbinding}

\subsection{Numerical model and validation} 

The time-dependent flows shown in Fig.~2 of the main text are obtained by numerically integrating Eqs.~(2a) and (2c) of the main text using a vorticity-stream function finite difference scheme. All equations are discretized on a two-dimensional square grid of sizes $256 \times 256$ and $1024 \times 1024$ with periodic boundary conditions. For both grid sizes, the grid spacing is $\Delta x = \Delta y = 1$ and the time stepping $\Delta t = 0.1$. The validity of this numerical approach is benchmarked by many numerical studies on liquid crystals and active matter (see for example Refs.~[23,24,38] of the main text). In all simulations, we set: $\rho=1$, $\eta=1$, $L_6=0.5$, $A_6=-0.2$, $B_6=0.4$, $\Gamma_6 =1$ and $\lambda_6 = 1.11$. All parameters are expressed in the arbitrary units used in the numerical simulations. 

\subsection{Defect annihilation and unbinding}

To construct the initial configuration of $\Psi_{6}$, we set $\ell=7$, $|\Psi_{6}|=1$ and take $\theta$ as given in Eq.~\eqref{eq:theta_convolution} inside a disk of radius $R_{D}= 28$ and random outside.  We then thermalize this configuration by keeping the orientation of the order parameter in the disk fixed, and relaxing $\Psi_{6}$ everywhere else. This allows us to obtain a defect-free configuration where $|\Psi_6| \approx 1$ everywhere, except that close to the defect cores where $|\Psi_6| \approx 0$. Notice that, on a doubly periodical domain, $\sum_{i}s_{i}=0$. Therefore, no other topological defect is found at the end of such relaxation procedure. Simulations are carried out until the total free energy relative variation drops under $0.1 \%$ with respect to two consecutive iterations. This corresponds to a state where defects have annihilated, and the hexatic liquid crystal has achieved a smooth configuration everywhere in the simulation box. For both annihilation and unbinding numerical experiments, we scan $\alpha_{6}$ for a wide range of positive and negative values. For any negative values of activity, we obtain increasingly sheared versions of the flow pattern in Fig.~2a-(i). Similarly, for positive values of $\alpha_{6}$ we obtain increasingly sheared versions of the same flow pattern, but with the direction of the flow inverted. 

\subsection{Active defect dipole annihilation: the origin of the unbinding}

In this section we provide a brief account of the annhilation dynamics of a pair of $\pm 1/6$ active hexatic defects, in which it is possible to recognize the fundamental mechanism driving defect unbinding. To this end, we place the defects on the $x-$axis at a distance of $\Delta x = 64$ and construct the initial configuration of the hexatic order parameter $\Psi_{6}=e^{6i\theta}$ by setting $\theta = \pm \arctan[y/(x\pm\Delta x/2)]$ inside a disk of radius $R_{D}= 5$ centred at the defect cores, and random outside. We thermalize this configuration by keeping the phase of the order parameter in the two disks fixed, while relaxing $\Psi_{6}$ everywhere else. As before, this procedure allows us to obtain a state where $|\Psi_{6}| \approx 1$ everywhere, except that close to the two defect cores where $|\Psi_{6}| \approx 0$. We use this as the initial state for our annihilation experiment. Simulations are carried out until defects have annihilated and the total free energy relative variation drops under $0.1 \%$ with respect to two consecutive iterations. The model parameters, expressed in lattice units, are again: $\Delta t=1$, $\rho=1$, $\eta=1$, $L_6=0.5$, $A_6=-0.2$, $B_6=0.4$, $\Gamma_6 =1$ and $\lambda_6 = 1.11$. 

Fig.~\ref{Fig:fig2_SI} shows the trajectories of the positive (red) and negative (blue) defects during annihilation, for three realizations of the activity parameter $\alpha_{6}$, that is $\alpha_{6}=0.1$ (contractile), $\alpha_{6}=0$ (passive) and $\alpha_{6}=0.1$ (extensile). For contractile activity, the backflow sourced by the active stress, Eq.~\eqref{active hexatic stress}, annihilation is sped up with respect to the passive case. By contrast, for extensile activity, annihilation is delayed. The same effects leads to the break up of the quadrupole into two defect pairs, provided the repulsive forces introduced by the active flow overcome the attractive Coulomb-like forces between defects. 
\end{document}