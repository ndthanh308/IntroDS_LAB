\documentclass[11pt,a4paper]{amsart}

%\usepackage[fullpage]{geometry}
\usepackage{fullpage}
\usepackage{amsmath}
\usepackage{amsfonts}
\usepackage{amssymb}
\usepackage{amsthm}
\usepackage[utf8]{inputenc}
\usepackage{breqn}
\usepackage{afterpage}
\usepackage{longtable}
\usepackage{indentfirst}
\usepackage{caption}
\usepackage{subcaption}
\usepackage{graphicx}
\graphicspath{ {./images/} }
\newtheorem{theorem}{Theorem}[section]
\newtheorem{lemma}[theorem]{Lemma}
\newtheorem{corollary}[theorem]{Corollary}
\newtheorem{proposition}[theorem]{Proposition}
\newtheorem{conjecture}[theorem]{Conjecture}
\theoremstyle{definition}
\newtheorem{definition}[theorem]{Definition}
\newtheorem{example}[theorem]{Example}
\newtheorem{xca}[theorem]{Exercise}



\newtheorem{theorem-definition}[theorem]{Theorem-Definition}


\theoremstyle{remark}
\newtheorem{remark}[theorem]{Remark}

\numberwithin{equation}{section}




\usepackage{color}
\usepackage{xcolor}
\def\blue{\color{blue}}
\def\red{\color{red}}
\def\green{\color{violet}}


\usepackage{bbold}


\newcommand{\abs}[1]{\lvert#1\rvert}


\newcommand{\lam}{\lambda}
\newcommand{\Pk}{\mathbb{P}^k}
\newcommand{\C}{\mathbb{C}}
\newcommand{\R}{\mathbb{R}}
\newcommand{\N}{\mathbb{N}}
\newcommand{\Z}{\mathbb{Z}}
\renewcommand{\P}{\mathbb{P}}
\renewcommand{\H}{\mathcal{H}}
\renewcommand{\L}{\mathcal{L}}
\newcommand{\J}{\mathcal{J}}
\newcommand{\F}{\mathcal{F}}
\newcommand{\M}{\mathcal{M}}
\newcommand{\G}{\mathcal{G}}

\DeclareMathOperator{\Card}{Card}
\DeclareMathOperator{\jac}{Jac}
\DeclareMathOperator{\Supp}{Supp}
\DeclareMathOperator{\diam}{diam}

\DeclareMathOperator{\Lip}{Lip}
\DeclareMathOperator{\dist}{dist}


\renewcommand{\epsilon}{\varepsilon}



\newcommand{\blankbox}[2]{%
  \parbox{\columnwidth}{\centering


    \setlength{\fboxsep}{0pt}%
    \fbox{\raisebox{0pt}[#2]{\hspace{#1}}}%
  }%
}

\begin{document}

\title{Holomorphic motions of weighted periodic points}



\author{Fabrizio Bianchi}
\author{Maxence Br\'evard}

\address{CNRS, Univ. Lille, UMR 8524 - Laboratoire Paul Painleve, F-59000 Lille, France}
\email{fabrizio.bianchi@univ-lille.fr}
\address{Université de Toulouse - IMT, UMR CNRS 5219, 31062 Toulouse Cedex, France}
\email{maxence.brevard@univ-toulouse.fr, mbrevard@hotmail.fr}




%\curraddr{}

%\email{AAA}



\subjclass[2010]{}
\date{}


\keywords{}

\begin{abstract}
We study the holomorphic motions of repelling
periodic points
in stable families of endomorphisms of $\mathbb P^k(\mathbb C)$.
In particular, we establish an asymptotic equidistribution
of the graphs associated to such periodic points with respect to natural measures in the space of all holomorphic motions of points in the Julia sets.
\end{abstract}


\maketitle


%{\red [maybe split intro in two parts: one only with text, and the second with precise statements, definitions, etc; maybe some details etc go in later sections. We can see this at the end]}


\section{Introduction}



A \emph{holomorphic family of endomorphims of $\mathbb P^k$} is a pair $(M,f)$, where $M$ is a complex manifold and $f\colon M\times \mathbb P^k\to M\times \mathbb P^k$ is a holomorphic map of the form $f(\lambda,z)=(\lambda, f_\lambda (z))$, where each $f_\lambda$ is an endomorphism of $\mathbb P^k$ of the same algebraic degree $d$. 
%Unless explicitly noticed {\red check if needed}, w
We always assume 
%in the following
that $M$ is connected and simply connected, and that $d\geq 2$.
%{\green
%Set $d \ge 2$, $k\ge 1$ and denote by $\mathcal H_d(\P^k)$ the space of holomorphic endomorphisms of $\P^k$ of algebraic degree $d$.
%Let $(M,f)$ be a holomorphic family of endomorphisms, parametrized by some connected, simply connected complex manifold $M$.
%In other words, the holomorphic map $f:M\times\P^k \to M \times \P^k$ is of the form $f(\lam,z) = (\lam, f_\lam(z))$, where $f_\lam \in \mathcal H_d(\P^k)$ for any $\lam \in M$.
The following fundamental result due 
by Lyubich \cite{Ly83a}, Mañ\'e-Sad-Sullivan \cite{MSS83}, and DeMarco \cite{dM03} defines and 
characterizes \emph{stability} within such families when $k=1$, see also
\cite{Lev82,Prz85,Si81} 
for further characterizations and previous results in the polynomial case. Recall that
Freire-Lopes-Mañ\'e \cite{FLM} and Lyubich \cite{Ly83b} proved that each rational map $f_\lam$ admits a unique invariant measure of maximal entropy $\mu_\lam$,
whose support, denoted as $J_\lam$, is the Julia set of $f_\lam$.

\begin{theorem-definition}
Let $(M,f)$ be a holomorphic family of rational maps as above.
%{\green Assume the $2d-2$ critical points $c_j : M \to \P^1$ are marked.} 
The following conditions are equivalent:
\begin{enumerate}
\item the Julia sets $J_\lam$ move holomorphically with $\lam$;
\item $dd^c L(\lam)\equiv 0$, where $L(\lam):= \int \log |f'_\lambda|\mu_\lam$
is the Lyapunov exponent of the measure of maximal entropy $\mu_\lambda$ of $f_\lambda$;
\item the repelling periodic points of $f_\lam$ move holomorphically with $\lam$.
%\item the families $(f^n\circ c_j)_{n\in \N}$ are normal.
\end{enumerate}
We say that the family is \emph{stable} if any (hence, all) of the above conditions hold. 
\end{theorem-definition}


Recall that a family of Borel sets $E_\lam \subset \P^1$ \emph{move holomorphically with $\lam$} if
there exists a set $\L$ of holomorphic functions $\gamma : M \to \P^1$ such that
the graphs $\Gamma_{\gamma_1}, \Gamma_{\gamma_2}\subset M\times \mathbb P^1$
of two distinct functions $\gamma_1, \gamma_2 \in \L$ do not intersect,
and for any parameter $\lam \in M$, we have $\L_\lam := \{\gamma(\lam), \gamma \in \L \} = E_\lam$.
A crucial point here is that, if the sets $E_\lam$ move holomorphically with $\lam$, the same is true for $\overline {E_\lam}$. This fact, usually referred to as the $\lambda$--lemma, is a consequence of the Hurwitz theorem for (one-dimensional) holomorphic maps.
%{\red references to other equivalences}


\medskip



A generalization of the above result for families in any dimension $k\geq 1$ was
proved by Berteloot, Dupont, and the first author in \cite{BBD18}.
As the Hurwitz theorem fails in higher dimensions,
the approach replaces the holomorphic motion of the Julia sets by a \emph{measurable holomorphic motion},
namely a family $\L$ of non-intersecting graphs $\gamma \colon M \to \P^k$ such that for any parameter $\lam \in M$, we have $\mu_\lam(\L_\lam) = 1$,
where $\mu_\lam$ is the unique  measure of
maximal entropy of the system $(\P^k, f_\lam)$ (see 
%{\red ref: BriendDuval, FS} 
\cite{BD01,DS10,FS94}).
We still denote its support as $J_\lambda$. The result is as follows, see also
\cite{BBD18,BB22,B19} for details and further characterizations and
\cite{BB18}
for an explanation of the strategy of the proof.
%{\green in terms of post-critical normality}.
%{\red we will see, if later we need lamination we need to define at some point. not here}

\begin{theorem-definition}\label{t:bbd}
Let $(M,f)$ be a holomorphic family of endomorphisms of $\mathbb P^k$. 
The following conditions are equivalent:
\begin{enumerate}
\item there exists a measurable holomorphic motion for the Julia sets $J_\lam$;
%{\red maybe comment/explain somewhere }
\item $dd^c L(\lam)\equiv 0$, 
where $L(\lam):=\int \log |Df_\lam|\mu_\lambda$
is the sum of the  Lyapunov exponents of the measure of maximal entropy $\mu_\lambda$ of $f_\lambda$.
\end{enumerate}
We say that the family is \emph{stable} if any (hence, all) of the above conditions hold. 

Moreover, the following condition
\begin{itemize}
\item[(3)] the repelling periodic points of $f_\lam$ move holomorphically with $\lam$
\end{itemize}
implies the above two, and is equivalent to them if $k=2$, or if $M$ is an open connected and simply connected
subset of 
 the space 
 $\mathcal H_d(\P^k)$ of all holomorphic endomorphisms of $\P^k$ of algebraic degree $d$.
 \end{theorem-definition}

It is still an open question whether the stability of 
a general family of endomorphisms in any dimension is equivalent to the motion of all the repelling periodic cycles. %We give below (see \ref{?}) {\red [todo, or maybe not]}
%an example of a possible scenario where such motion could fail.
On the other hand, it was proved in \cite{B16,B19} that the stability of a family implies at least
a weaker version of the motion of the repelling cycles. Notice that, in turn,
this weaker notion is still sufficient
to imply stability, see also \cite{B18}.
The definition is as follows.



 \begin{definition}\label{def:motion-rep-asympt}
  We say that \emph{asymptotically all} repelling cycles move holomorphically  on $M$
  if
  there exists a countable set
  $\mathcal P= \cup_{n\in \N^*} \mathcal P_n$ of holomorphic functions $\gamma : M \to \P^k$
  satisfying the following properties:
  %{\red [SEE mail about this and main thm; in any case, maybe M'?]}
 \begin{enumerate}
 \item  $\mathrm{Card }\: \mathcal P_n = d^{kn} + o(d^{kn})$;
 \item the point $\gamma(\lambda)$
 belongs to $J_\lambda$ and is a fixed point of $f_\lambda^n$, for every $\lambda \in M$ and $\gamma \in \mathcal P_n$;
 \item  for all open subsets $M'\Subset M$, we have
\[ \frac{\mathrm{Card } \{
 \gamma \in \mathcal P_n \: \colon \: \gamma(\lambda) \mbox{ is repelling for all } \lambda \in M'\} }{d^{kn}}\underset{n\to \infty}{\longrightarrow} 1.\]
\end{enumerate}
\end{definition}


%{\red [need to cite \cite{B18} in some way around here; quickly, or maybe later if we put the statement with limsup in a final section]}

The cycles in Definition \ref{def:motion-rep-asympt}
can be seen, in some sense, as generic with respect to the measure of maximal entropy.
More precisely, consider the space $\mathcal J$, defined as
\begin{equation}\label{e:J}
\mathcal J := \{
\gamma : M\to \mathbb P^k \: \colon \: \gamma(\lam) \in J_\lam \quad \forall \lam \in M
\}, \end{equation}
where the maps $\gamma$ are holomorphic. 
$\mathcal J$ is a metric space, see Section \ref{s:prelim-inverses} for details.
%{\red [maybe the details not here in the introduction but in section 2?]}
%{\red [Other thing: since we always work on a relatively cpt subset Omega of M, maybe we can take just the distance on the closure of Omega?]}
%{\red maybe if we say dist we should define it}
We can turn $\mathcal J$ into a topological dynamical system by defining a natural map $\mathcal F$ as
\begin{equation}\label{e:F}
\mathcal F (\gamma) (\lambda) := f_\lambda (\gamma(\lambda)).
\end{equation}
A
\emph{web} is a probability measure
$\mathcal M$
compactly supported on $\mathcal J$ which is invariant under $\mathcal F$. An \emph{equilibrium web} 
\cite{BBD18}
is a web such that
\[
(p_\lam)_* \mathcal M = \mu_\lam \quad  \forall \lambda \in M,
\]
where we denote by $p_\lam\colon \mathcal J \to \mathbb P^k$ the natural map $\gamma \mapsto \gamma(\lambda)$. The properties of equilibrium webs are crucial in the approach in \cite{BBD18}.

\medskip


With this terminology, and owing to the equidistribution
of the repelling periodic cycles with respect to the measure of maximal entropy \cite{BD99},
we see in particular that
the fact that the existence of $\mathcal P$ as in Definition \ref{def:motion-rep-asympt} is equivalent to stability as in Theorem-Definition \ref{t:bbd}
leads to the following equidistribution result for periodic graphs in any stable family. This can be seen as a weak version of the implication (1)$\Rightarrow$(3).




\begin{theorem}\label{t:graph_equidistribution}
Let $(M,f)$ be a stable family of
endomorphisms of $\mathbb P^k$.
Then, for every 
$M'\Subset M$ and every
$n\in \mathbb N^*$, there exists
a non-empty subset $\mathcal P_n \subset \J$ of motions $\gamma$
of $n$-periodic points
such that 
$\gamma(\lam)$ is repelling for all $\lam \in M'$ and
\[
\lim_{n\to \infty} d^{-kn} \sum_{\gamma \in \mathcal P_n}
\delta_{\gamma} = \mathcal M,
\]
where $\mathcal M$ is an equilibrium web.
 \end{theorem}


 


% \begin{equation}\label{e:equid:0}
% d^{-kn} \sum_{\gamma\in \mathcal P_n} \delta_\gamma \to \mathcal M,
% \end{equation}



%\medskip

In this paper, we address the question of the motion of cycles which similarly equidistribute invariant measures in  a much larger class, that we now introduce.

\medskip

Let $f$ be an endomorphism of $\mathbb P^k$ 
of algebraic degree $d\geq 2$
and let $\phi$ be a real continuous function on $\mathbb P^k$ (usually called a \emph{weight}).
The \emph{pressure} of $\phi$ is defined as 
\[
P(\phi) := \sup_{\nu} 
\big(h_\nu + \int \phi \nu\big)
\]
where 
the supremum is over all $f$-invariant measures $\nu$
and
$h_\nu$ denotes the measure-theoretic entropy of $\nu$. An 
\emph{equilibrium state} 
for the weight $\phi$ is defined  as a maximizer of the pressure function, i.e., as an invariant measure $\mu_\phi$ satisfying
\[
h_{\mu_\phi} + \int \phi \mu =  P(\phi).
\]
Assume that $f$ satisfies the following condition:


\medskip\noindent
{\bf (A)} \hspace{1cm} the local degree of the iterate $f^n$ satisfies
$$\lim_{n\to\infty} \frac{1}{n} \log\max_{a\in\P^k}\deg(f^n,a) =0.$$

\medskip\noindent
Here, $\deg(f^n,a)$ is the multiplicity of $a$ as a solution of the
equation $f^n(z)=f^n(a)$. Note that generic endomorphisms of $\P^k$ satisfy
 this condition, 
 see \cite{DS10b}.
 Let $\phi$ satisfy
 %{\red [for now here, maybe in text later and refer]}
%\begin{equation}\label{eq:phi}

\medskip\noindent
{\bf (B)}
 \hspace{2cm}
$\|\phi\|_{\log^q} 
<\infty    
\, \mbox{ for some } q>2 \quad \quad \mbox{ and } \quad \quad 
\Omega (\phi)
%:= \max (\phi)-\min (\phi)
<\log d$,
%\end{equation}

\medskip\noindent
%for some $q>2$, 
where we define $\Omega(\phi):=\max (\phi)-\min (\phi) $,
\[\|\phi\|_{\log^q} 
:= 
\sup_{a,b \in \P^k} |\phi(a)-\phi(b)|\cdot (\log^\star \dist (a,b) )^q,\] 
and
$\log^\star (\cdot) = 1+|\log (\cdot)|$.
The existence and uniqueness of the equilibrium state $\mu_\phi$  for $f$
under the
assumptions
{\bf (A)} and {\bf (B)}
have been proved 
in \cite{UZ13,BD23}
see also \cite{D12,SUZ,BD22}
for further properties of these measures, and
\cite{PU} and references therein
%{\red [cite something]}
for previous results in dimension 1. 
The case $\phi=0$ corresponds to the case of the measure of maximal entropy, see for instance
\cite{DS10} and references therein
for an account of this case.


\medskip


% \cite{dTh08, D12}
By the definition of pressure and the assumption on $\Omega(\phi)$, all these equilibrium states satisfy $h_{\mu_\phi}> \log d^{k-1}$. It is proved in \cite{BR22}
that, given a stable family 
$(f_\lam)_{\lam\in M}$
and any $\lam_0\in M$,
for any $f_{\lambda_0}$-invariant  measure $\nu$ 
satisfying $h_\nu >\log d^{k-1}$, it is possible to construct an associated web $\mathcal M_{\lam_0, \nu}$
with the property that
%{\red [maybe here cite Dujardin-Lyubich and Berger-D in some way, especially if we mention a lamination here/use it later]}
\[
(p_{\lambda_0})_* \mathcal M_{\lam_0, \nu} = \nu,
\]
as well as the associated lamination.
This in particular applies to the equilibrium states as above. 
%We will denote by $\mathcal M_{\lam_0, \mu_\phi}$ the web associated to the equilibrium state $\mu_{\phi}$ for $\phi$ at $\lam_0$.
The following is our main result.


%{\red [CHECK: it seems that we need condition (A) only at $\lambda_0$, right?]}


\begin{theorem}\label{t:main}
Let $(M,f)$ be a stable family of
endomorphisms of $\mathbb P^k$.
Take $\lam_0 \in M$ and assume that $f_{\lam_0}$ satisfies condition 
{\bf (A)}. 
Let $\phi : \mathbb P^k\to \mathbb R$
 satisfy 
 %$\|\phi\|_{\log^q}<\infty$ for some $q>2$ and $\Omega(\phi)<\log d$
 {\bf (B)}
% Take $\lam_0\in M$ 
and let $\mu_\phi$ be the equilibrium state for $f_{\lam_0}$
 associated to $\phi$. 
Then, for every 
$M'\Subset M$ and every
$n\in \mathbb N$, there exists
a non-empty subset $\mathcal P_{\phi,n} \subset \J$ of motions 
$\gamma$ of $n$-periodic points
such that 
%over $M'$
$\gamma(\lam)$ is repelling for all $\lam \in M'$ and
\[
\lim_{n\to \infty} e^{- n P(\phi)} \sum_{\gamma \in \mathcal P_{\phi,n}}
e^{\phi (\gamma(\lambda_0)) + \dots + \phi ( f^{n-1}_{\lambda_0} (\gamma(\lambda_0)))} \delta_{\gamma} = \mathcal M_{\lam_0,\mu_\phi}.
\]
% In particular, the $\phi$-cycles at $\lambda_0$ move holomorphically over $M'$. 
 \end{theorem}
%{\red [a priori on the left the graphs are on M', on the right on M; need a comment that we can restrict parameter for a measure]}


Observe that, in particular, Theorem \ref{t:main} generalizes
to general weights $\phi$
Theorem \ref{t:graph_equidistribution} ,
the latter corresponding to the case $\phi = 0$.


% gives the asymptotic holomorphic motion of repelling periodic cycles with respect to
% any equilibrium state at any parameter, generalizing to any weight $\phi$ 
% (which corresponds to the case $\phi=0$).
%{\red [likely change/remove this comment if we make the statements 1.3 and 1.4 more similar]}

\medskip

At the parameter $\lam_0$, the equidistribution of repelling periodic points with respect to the equilibrium state $\mu_\phi$ has been established in \cite[Theorem 4.10]{BD22}. The proof follows the now classical strategy by Briend-Duval \cite{BD99},
%(in a simplified version due to Buff \cite{Buf05})
who showed this result for the measure of maximal entropy, which corresponds to the case $\phi=0$. On the other hand, when $\phi\neq 0$, as the Jacobian of $\mu_\phi$ is not constant, the proof requires more precise estimates on the contraction along generic inverse branches for $\mu_\phi$, which in turn follow from delicate distortion estimates along inverse branches due to Berteloot-Dupont-Molino \cite{BD19,BDM}.
%see also \cite{BD19}. 
In the current paper, we adapt this strategy
in the setting of the dynamical system
$(\mathcal J, \mathcal F, \mathcal M_{\lam_0, \phi})$. This requires to precisely control the contraction of $f$ on tubes 
(i.e., tubular neighbourhoods, of uniform radius in $\lambda$, 
of the graphs
in $M'\times \mathbb \P^k$ of elements of $\mathcal J$) centered at $\mathcal M_{\lam_0, \phi}$-generic elements of $\mathcal J$. As a result, we get the motions of repelling periodic points as repelling periodic elements for $\mathcal F$.


\subsection*{Acknowledgments}
%The first author would like to thank SIMONS ETC.
The first author would like to thank the Simons Foundation, Laura De Marco, and Mattias Jonsson, for supporting and organizing the Simons Symposium 
on Algebraic, Complex, and Arithmetic Dynamics
in August 2022. This work was motivated by questions and discussions which arised during such event.

This project has received funding from
 the French government through the Programme
 Investissement d'Avenir
 (I-SITE ULNE /ANR-16-IDEX-0004,
 LabEx CEMPI /ANR-11-LABX-0007-01,
ANR QuaSiDy /ANR-21-CE40-0016,
ANR PADAWAN /ANR-21-CE40-0012-01)
managed by the Agence Nationale de la Recherche.
%{\red [check if you need to add something]}


\section{Nice inverse branches for expanding webs}\label{s:prelim-inverses}


\subsection{Transfer operators on $(\mathcal J, \mathcal F)$.}

%\subsection{Stable families of endomorphisms of $\mathbb P^k$}

%{\red [general comment: at some point we need to restrict the parameter space to a relatively compact subset; I dont do that for now; maybe something needs to be adapted, for instance in the defs of the tubes: $T_\Omega$ etc]}

%- I am trying to write this part more in general, not only for web coming from equilibrium states. But in practice, we must have assumptions to get the limit formula for disintegration I think. I try to use $\Lambda$ at the beginning, but maybe we have to reduce and directly use a sum of weigthed preimages there already.


\medskip


%- no phi in this section, i put M3 condition and then we say in next section that those for phi are ok. So this section is "general". }{\blue NEED TO CHECK THAT WE can really do this without already using only equilibrium states - I am more and more sceptical about this, but I'll think a bit more}

%{\red maybe remove stable}
We consider 
%{\red actually we do not need stable in the first part no?} {\green I agree}
in this section a holomorphic family $(M,f)$
of endomorphisms of $\mathbb P^k$.
%BBD if not before
We let $\mathcal J$ be as in \eqref{e:J}
(observe that stability is not required to define this set).
%{\red [maybe here the details about the distance. Maybe also define $\mathcal J_\Omega$ and the distance only on Omega - easier if we can take $\bar \Omega$?]} 
We can turn $\mathcal J$ into a Polish (i.e.,  separable complete metric) space $(\J, \dist_\J)$ as a closed subset of the space $\mathcal O(M,\P^k)$ endowed with the metric of local uniform convergence.
More precisely,
the distance between two elements $\gamma_1, \gamma_2 \in \J$ is given by
\[
\dist_\J(\gamma_1,\gamma_2) := \sum_{n=0}^{+\infty} 2^{-n} \max\left(1, \sup_{\lam \in K_n}\dist_{\P^k}(\gamma_1(\lam),\gamma_2(\lam))\right),
\]
where the family $(K_n)_{n\in \N}$ is an exhaustion of $M$, 
namely a nested sequence of compact sets whose union is $M$ and such that $K_n$ is a subset of the interior of $K_{n+1}$ for every $n\in \N$.
The resulting topology is independent of the choice of the exhaustion.


Let  
$\mathcal F\colon \mathcal J\to \mathcal J$ be as in \eqref{e:F}. 
For $\gamma \in \mathcal J$, we denote by $\Gamma_\gamma$ the graph of $\gamma$ in $M\times \mathbb P^k$.
We denote by $\mathcal J_s$ the subset of $\mathcal J$ given by
$$\mathcal J_s := \{\gamma\in\mathcal{J}:\Gamma_\gamma\cap GO(C_f)
  \ne \emptyset\},$$
and set $\mathcal X := \mathcal J \setminus \mathcal J_s$.
Observe that every element $\gamma \in\mathcal X$ admits $d^{k}$ well defined inverse elements by
$\mathcal F$ in $\mathcal X$, i.e.,
elements $\gamma'\in \mathcal X$ such that $\mathcal F (\gamma')=\gamma$.

We let $\psi\colon \mathcal J\to \R$ be a continuous function 
and define the (transfer) operator $\Lambda_\psi$ acting on measurable real functions on $\mathcal J$ as
%
%\colon \mathcal C^0(\mathcal X) \to \mathcal C^0(\mathcal X)$ by
\begin{equation}\label{e:def-Lambda-psi}
\Lambda_\psi (g)(\gamma)
=\sum_{\mathcal F(\gamma')=\gamma} e^{\psi (\gamma')} g (\gamma').
\end{equation}
Observe that the operator $\Lambda_\psi$ preserves positivity. However, even when $g$ is continuous, 
$\Lambda_\psi(g)$ needs not be continuous
(as, for example, the system $\mathcal F\colon \mathcal J\to \mathcal J$ may not have a well defined degree). On the other hand, as every $\gamma\in\mathcal X$ has precisely
$d^{k}$ preimages under $\mathcal F$, which are also in $\mathcal X$, the operator $\Lambda_\psi$ defines a continuous operator from 
$\mathcal C^0(\mathcal X)$
to
$\mathcal C^0(\mathcal X)$.
%{\red [say better]}. 

%As $\Lambda_\psi$ does not preserve continuous function, we cannot 
%automatically define a dual operator on measures on $\mathcal J$. On the other,
Given a positive measure $\mathcal N$ satisfying $\mathcal N (\mathcal J_s)=0$,
the measure $\mathcal N$  integrates any continuous functions on $\mathcal X$, and we can define $\Lambda_\psi^*\mathcal N$ by the relation
\[
\langle \Lambda_\psi^* \mathcal N, g\rangle=
\langle \mathcal N , \Lambda_\psi (g)
\rangle,
\]
where $g$ is any continuous function on $\mathcal J$.
%{\red does it make sense like this?}
We will use
the operator 
$\Lambda_\psi^*$
only on measures vanishing on $\mathcal J_s$. %satisfying the above condition.

\begin{lemma}\label{l:rho-N-equivalent}
The following assertions are equivalent:
%{\red statement only for positive}
\begin{enumerate}
\item there exists a continuous and strictly positive
function $\theta\colon \mathcal X \to \mathbb R$ such that $\Lambda_\psi^n (g)\to c_g \theta$ for every continuous function $g\colon \mathcal J \to \mathbb R$, where the constant $c_g$ depends linearly and continuously on $g$;
\item
there exists a positive 
measure $\mathcal N$
on $\mathcal J$, satisfying $\mathcal N(\mathcal J_s)=0$, such that $(\Lambda^*_\psi)^n \delta_\gamma\to c_\gamma \mathcal N$
for every $\gamma \in \mathcal X$, where $c_\gamma$ is a
%non-negative 
%(resp. positive)
strictly positive
 constant depending continuously on $\gamma \in \mathcal X$.
\end{enumerate}
\end{lemma}


\begin{proof}
(1)$\Rightarrow$(2)
Define a measure $\mathcal N$ on $\mathcal X$ 
by setting $\langle\mathcal N, g\rangle := c_g$
for every continuous function $g\colon \mathcal X\to \mathbb R$, where $c_g$ is as in (1).
We can extend $\mathcal N$ to a measure on $\mathcal J$ by setting $\mathcal N (\mathcal J \setminus \mathcal X)=0$. Such measure is positive since $c_g \geq 0$ for every non-negative $g$.


\medskip




For every $\gamma \in \mathcal X$ and continuous function $g\colon \mathcal X\to \mathbb R$,
we have
\[
\langle
(\Lambda_\psi^*)^n\delta_\gamma, g\rangle=
\langle \delta_\gamma, \Lambda^n_\psi (g)\rangle
\to
\langle \delta_\gamma, c_g \theta\rangle= \theta(\gamma)c_g.
\]
By the definition of $\mathcal N$, and setting $c_\gamma:= \theta (\gamma)$, 
this shows that $(\Lambda_\psi^*)^n \delta_\gamma \to c_\gamma \mathcal N$. 


\medskip

(2)$\Rightarrow$(1)
Define a function $\theta\colon \mathcal X\to \mathbb R$ by $\theta(\gamma) := c_\gamma$, for every $\gamma \in \mathcal X$, where $c_\gamma$
 is as in (2).
For every $\gamma \in \mathcal X$ and continuous function $g\colon \mathcal J \to \mathbb R$ we have
\[
\Lambda_\psi^n (g) (\gamma)
=
\langle \delta_\gamma, \Lambda_\psi^n (g)\rangle
=
\langle 
(\Lambda_\psi^*)^n \delta_\gamma, g\rangle
\to \langle c_\gamma  \mathcal  N, g\rangle=
c_\gamma\langle \mathcal N, g\rangle.
\]
 By the definition of $\theta$, and setting $c_g:= \langle\mathcal N,g\rangle$, this shows that
 $\Lambda_\psi^n (g)\to c_g \theta$.
\end{proof}


\begin{remark} The two assertions in Lemma \ref{l:rho-N-equivalent} stay equivalent if $\theta$
is assumed to just be non-negative in (1) and $c_\gamma$ is assumed to just be non-negative in (2).
\end{remark}
%{\red lemma below says that the family is stable, but a priori we do not need to assume it, right?}
%{\green Agreed}

For every graph $\gamma \in \mathcal X$ and every integer $n\in \N$, define the measure

\begin{equation}\label{e:web_definition}
\mathcal M_{\gamma,n}
:=
\theta \cdot \theta(\gamma)^{-1}
\cdot (\Lambda_\psi^*)^n \delta_\gamma =
\theta(\gamma)^{-1} \sum_{\mathcal F^n (\gamma')}
e^{\psi(\gamma') + \ldots + \psi(\mathcal F^{n-1} (\gamma'))} \theta(\gamma')\delta_{\gamma'}.
\end{equation}

\begin{lemma}\label{l:criterion-conditions}
If the  conditions in Lemma \ref{l:rho-N-equivalent} are satisfied
%{\red positive rho},
and $\theta$ and $\mathcal N$ are as in that lemma, 
the measure $\mathcal M := \theta \mathcal N$ is well defined and the following properties hold:
\begin{enumerate}
\item $\Lambda_\psi\theta = \theta$;
\item $\Lambda_\psi^* \mathcal N= \mathcal N$;
\item $\mathcal M$ is a probability measure and $\mathcal M (\mathcal J_s)=0$;
%{\red this means already acritical}
\item $\mathcal M$ is $\mathcal F$-invariant;

\item 
%, we have
%the following convergence holds:
%{\green $\theta(\gamma) > 0$ and}
%\[
$\mathcal M_{\gamma,n} \to \mathcal M$
for every
$\gamma \in \mathcal X$;
%\]
\item $\F^{-1}(\Supp \M) \subseteq \Supp \M$.
\end{enumerate}
\end{lemma}



\begin{proof}
(1)
This is a consequence of the first condition in Lemma \ref{l:rho-N-equivalent}. Indeed, as $\Lambda_\psi$ is continuous on $\mathcal C^0 (\mathcal X)$, we have
%{\red check}
\[
c_\theta \theta =
\lim_{n\to \infty} \Lambda_\psi^{n+1} \theta
=
\Lambda_\psi 
(\lim_{n\to \infty} \Lambda_\psi^n \theta) =
%\lim_{n\to \infty} \Lambda^{n+1}_\psi \theta 
\Lambda_\psi (c_\theta \theta)
= c_\theta \Lambda_\psi  \theta.
\]
As $c_\theta>0$, we deduce 
that $\theta = \Lambda_\psi (\theta)$, as desired.
Observe in particular that this implies that $c_\theta=1$, 
since  $\Lambda_\psi^n \theta=\theta$ for every $n\in \mathbb N$  and $\Lambda_\psi^n \theta \to c_\theta \theta$. %valid for every $n\in \mathbb N$, that $c_\rho=1$.

\medskip

(2)
Observe that $\Lambda_\psi^* \mathcal N$ is well defined since $\mathcal N(\mathcal J_s)=0$, and still
satisfies  $(\Lambda_\psi^* \mathcal N) (\mathcal J_s)=0$. It is enough to check that 
$\langle \Lambda_\psi^* \mathcal N, g\rangle = \langle\mathcal N, g\rangle$ for every $g \in \mathcal C^0 (\mathcal J)$.
With the notation of Lemma \ref{l:rho-N-equivalent}, we also have $\langle\mathcal N, g\rangle=c_g$, where $c_g$ is characterized by the convergence $\Lambda_\psi^n g \to c_g \rho$.
Observe in particular that $c_g = c_{\Lambda_\psi g}$. It follows that,
for every $g$ as above, we have
\[
\langle
 \Lambda_\psi^* \mathcal N, g
\rangle =
\langle
\mathcal N, \Lambda_\psi g\rangle
=
c_{\Lambda_\psi g}= c_g = \langle\mathcal N, g\rangle.
\]

(3)
As $\theta$ is positive on $\mathcal X$, $\mathcal N(\mathcal J_s)=0$, and   $ \langle\mathcal N, \theta\rangle=c_\theta =1$, we have  $\theta \in L^1 (\mathcal N)$.
As $\mathcal N(\mathcal J_s)=0$, we have $\mathcal M (\mathcal J_s)=0$. As $\theta$ is non-negative and $\mathcal N$ is a positive measure, $\mathcal M$ is a positive measure. It is a probability measure since
\[\mathcal M (\mathcal X) = \langle\theta\mathcal N, \mathbb 1_{\mathcal X}\rangle
=
\langle\mathcal N, \theta\rangle = c_\theta =1.\]

(4)
We need to check that $\langle\mathcal M, g\rangle
=
\langle \M,g \circ \mathcal F \rangle$
for every continuous function $g$ on $\mathcal J$.
As $\mathcal M (\mathcal J_s)=0$, the pairing can be computed on $\mathcal X$. A direct computation gives that
\[
\langle \mathcal M, g \circ \mathcal F
\rangle
=
\langle \theta \mathcal N, g \circ \mathcal F\rangle
=
\langle \mathcal N, \theta\cdot g \circ \mathcal F\rangle
=
\langle \mathcal N, \Lambda_\psi( \theta\cdot g \circ \mathcal F)\rangle,\]
where in the last step we used the equality
$\Lambda_\psi^*\mathcal N = \mathcal N$. It follows from the definition of $\Lambda_\psi$ and the $\Lambda_\psi$-invariance of $\theta$
that
\[
\Lambda_\psi (\theta \cdot g \circ F) = g \cdot \Lambda_\psi (\theta) = g \theta.
\]
Together with the previous equalities, this gives
\[
\langle \mathcal M, g \circ \mathcal F
\rangle
=
\langle
\mathcal N, \theta  g\rangle =
\langle\mathcal M, g\rangle,
\]
as desired.

\medskip


(5)
Take $g\in \mathcal C^0 (\mathcal J)$. By the second assertion of Lemma \ref{l:rho-N-equivalent}, and recalling that $c_\gamma=\theta(\gamma)$ for every
$\gamma\in \mathcal X$,
we have
\[
\langle \mathcal M_{\gamma, n}, g\rangle
=
\theta(\gamma)^{-1}
\langle
\theta 
\cdot (\Lambda_\psi^*)^n \delta_\gamma, g
\rangle
\to
\theta(\gamma)^{-1}
\langle c_\gamma \mathcal N, \theta g\rangle
=
\langle\mathcal M, g\rangle.
\]
The assertion follows.

\medskip


(6)
Since $\theta$ is positive, we have $\Supp \M = \Supp \mathcal N$. Take $\gamma \in \Supp \mathcal N$ and $\gamma' \in \J$ such that $\F(\gamma') = \gamma$.
Fix a small open set $U$ containing $\gamma$ and let $U'$ be the connected component of $\mathcal F^{-1} (U)$ containing $\gamma'$ (which is open since $\mathcal F$ is continuous).
%For $\eta>0$, the open set $U := \F(B(\gamma', \eta))$ contains $\gamma$.
Since $\gamma \in \Supp \mathcal N$, we have $\mathcal N(U)>0$.
By the definition of $\Lambda_\psi$, we have that
$\Lambda_\psi^*\mathcal N (U')>0$.
This gives that $\gamma' \in \Supp \Lambda^*_\psi \mathcal N$.
%it is then sufficient to prove that $\mathcal N(U) > 0$.
%Observe that, by definition, $\mathcal N(U) = 0$ would imply $\Lambda_\psi^*{\mathcal N}(B(\gamma, \eta)) = 0$.
By (2), it follows that $\gamma'\in \Supp \mathcal N$, as desired.

\end{proof}




%{\red [stop here]}

%\medskip

%{\red we can split here is family needs not stable above, and observe that the condition below imply stable}

\subsection{Estimates of contraction along inverse branches.}
We assume in the following that the family $(M,f)$
admits
%ergodic
%{\red [needed? Or just the slices?]}
a web $\mathcal M$, i.e., an
%n ergodic
$\mathcal F$-invariant compactly supported
probability measure on $\mathcal J$, satisfying the following properties:

\begin{itemize}
\item[{\bf (M1)}] $\mathcal M$ is \emph{acritical}, i.e., we have $\mathcal M (\mathcal J_s)=0$;
%$$\mathcal J_s := \{\gamma\in\mathcal{J}:\Gamma_\gamma\cap GO(C_f)
%  \ne \emptyset\}.$$
\item[{\bf (M2)}] there exists a constant
$A_1>0$ such that, 
for every $\lam \in M$, 
the probability measure $(p_\lam)_* (\mathcal M)$ is ergodic and  
the Lyapunov exponents
of $(p_\lam)_* (\mathcal M)$ are strictly larger than $A_1$;
\item[{\bf (M3)}]
there
%{\red [maybe this definition goes above the other lemma too]}
exists a continuous
%measurable {\red check}
function $\psi\colon \mathcal X \to \R$, a strictly positive continuous function 
%{\red check maybe positive, etc}
$\theta\colon  \mathcal X\to \mathbb R$, and a positive measure $\mathcal N$ on $\mathcal J$ 
with $\mathcal N (\mathcal J_s)=0$
such that
\begin{enumerate}
\item 
$\mathcal M = \theta \mathcal N$,
\item $\Lambda_\psi \theta= \theta$,
\item $\Lambda_\psi^* \mathcal N= \mathcal N$;

%the dual $\Lambda_\psi^*$
%of the operator $\Lambda_\psi$
%%%the continuous operator on $\mathcal C^0 (\mathcal X)$ 
%defined as in \eqref{e:def-Lambda-psi}
%%%%%\[\Lambda_\psi (g):= \sum_{\mathcal F (\gamma')= \gamma} e^{\psi (\gamma')}g (\gamma')\]
 %admits a unique probability measure $\mathcal N$ on $\mathcal X$ such that $\Lambda_\psi^* \mathcal N= \mathcal N$, and $\mathcal M = \rho \mathcal N$, where $\rho$ is given by Lemma \ref{l:general-Lambda}.
%\item[{\bf (M3)}] there exist positive functions
%{\red maybe continuous}
%$b, a_n \colon \mathcal J \to \mathbb R$ such that, 
\item for $\mathcal M$-almost every $\gamma \in \mathcal J$, we have
%following convergence hold:
%\[
$\mathcal M_{\gamma,n} \to \mathcal M$,
%:= b(\gamma)^{-1} \sum_{\mathcal F^n (\gamma')=\gamma}
%a_n (\gamma')\delta_{\gamma'} \to \mathcal M.
where $\M_{\gamma, n}$ is defined as in \eqref{e:web_definition}.
\end{enumerate}
%\item {\red check if lamination needed later to define things, and likely better way to write all; check if maybe something is redundant}
\end{itemize}

%\medskip


%{\blue I arrived here; likely first we give a list of condition, and then the lemma of the equivalences, and then the lemma saying that the properties are satisfied in those cases. Just check if now it seems correct}



\medskip

%{\blue [comments to fix]

Observe that condition {\bf (M1)} is sufficient to imply  the stability of the family $(M,f)$ in the sense of Theorem-Definition \ref{t:bbd}, see \cite[Theorem 4.1]{BBD18}.
%In particular, $\mathcal M$-almost every $\gamma$ admits $d^{n}$ well defined inverse $\gamma'$, i.e., elements $\gamma'\in \mathcal J$ such that $\mathcal F (\gamma')=\gamma$. The sum in {\bf (M3)} is over such $d^{k}$ elements.
%{\red [some comment on the fact that it makes sense to restrict to $\mathcal X$ etc]}
%Although a priori $\mathcal J$ is not compact, since we will only work on the (compact) support of $\mathcal M$, we will assume that this is the case (a possible way to do this is also to lift the family to a family of homogeneous endomorphisms of $\mathbb C^{k+1}$).
%{\red [check later if this comment is really needed, or maybe somewhere else]}
%}
Conversely, a stable family always admits at least a web $\mathcal M$ satisfying the above assumptions.
To this purpose, it is enough to consider an acritical equilibrium web $\mathcal M_0$, 
as constructed in \cite{BBD18}. This web corresponds to the case of $(p_\lam)_* \mathcal M= \mu_\lam$
(the measure of maximal entropy) for all $\lam \in M$, and $\psi \equiv - k \log d$.
An asymptotic contraction property along generic inverse branches for 
$\mathcal M_0$ is proved in \cite[Proposition 4.2 and 4.3]{BBD18}. This property is the key to getting the measurable holomorphic motion in Theorem-Definition \ref{t:bbd}. It was generalized in \cite{BR22} for the
larger class of webs satisfying {\bf (M1)} and {\bf (M2)}, see Proposition \ref{p:bbd-gen} below.
%Building on such contraction property, 
We aim here at establishing a more quantitative version of such results, under the extra condition {\bf (M3)}. In particular, all this will also apply 
to the equilibrium web $\mathcal M_0$ as above.


\medskip


%{\red maybe split subsection here}

%We will denote by $\mathcal X:= \Supp \mathcal M \setminus \mathcal J_s$. 
Given 
%{\red [why the $\Supset$ in this part, is it necessary? Then answer could be yes if  we use the simplified distance, but i dont think we do it here]}
$\Omega\subset M$, 
$\gamma\in \mathcal X$ and $\eta>0$,
%, and a subset $\Omega\subset M$,
we denote by
$T_\Omega(\gamma, \eta)$ the $\eta$-neighbourhood of the graph $\Gamma_\gamma$ of $\gamma$ in $\Omega \times \P^k$, i.e.,
\[
T_\Omega (\gamma, \eta) := \{
(\lambda, z) \in \Omega \times \P^k\colon \dist_{\P^k}(z,\gamma(\lambda))< \eta
\}.
\]
We call such neighbourhood a \emph{tube} at $\gamma$ over $\Omega$. Observe that a tube $T_\Omega(\gamma, \eta)$ corresponds to the ball $\mathcal B_\Omega (\gamma, \eta)$ in the metric space $(\mathcal J, \dist_{\Omega})$,
%{\red [somewhere define distance for this, maybe before when we say that $\mathcal J$ is a metric space]}
%{\blue not completely true, if the tube is on Omega and the distance for maps on M, no?}
where the distance $\dist_\Omega$ is given by
\begin{equation}\label{e:distance-Omega}
\dist_\Omega (\gamma_1, \gamma_2) := \sup_{\lam \in \Omega}
\dist_{\P^k} (\gamma_1(\lambda),\gamma_2 (\lambda)).
\end{equation}
%{\blue Abusing notation, we will identify graphs on $\Omega$ in the support of any web $\mathcal M$ with the graphs in }
%{\red check the max 1?}

%{\red [check notation for ball later]}

%{\red [check if (likely) more generally we need to give a def over some $\Omega\Subset M$]}

%{\red define slice $A_{|\lambda}$ of an open set $A$ or similar, if needed/useful below}

% {\red SEE mail; need to use something like $\mathcal J_\Omega$ or similar?}
% {\green I see the problem. We cannot consider inverse branches of $\gamma \in \mathcal X$ over the whole parameter space $M$, due to the lack of controll outside $\Omega$/$M'$. However, if $\gamma \in \mathcal X_\M := \Supp \M \setminus \J_s$, we should be able to extend the graph. If $\gamma(\lam_0)$ were repelling periodic, we could conlude with \cite[Lemma 2.5]{BBD18}. I feel we should have already answered to this question in Lemma \ref{l:criterion-conditions} (6).}
%{\red [I need to check again this. I think it is the last thing left?]}
Given a tube $T= T_\Omega (\gamma,\eta)$,
the \emph{slice}
$T_{|\lambda}$ is the ball
$B(\gamma(\lam), \eta) = T \cap (\{\lambda\}\times \P^k)$.
More generally, given the image of a tube $T$ by a holomorphic map $g\colon T\to \Omega\times \P^k$ fibered over $M$, we define the slice $g(T)_{|\lam}$
of $g(T)$ at $\lambda$
as $g(T)\cap \{\lam\}\times \P^k$.

%{\red check if needed to define relatively cpt tubes etc}

\medskip

We fix in what follows a constant $0<A_0<A_1$, where $A_1$ is given in {\bf (M2)}.


\begin{definition}
Given $\Omega\subset M$,
$\gamma \in \mathcal X$, a tube $T$ at $\gamma$ over $\Omega$,
and $n \in \mathbb N$, we say that a map
$g \colon T\to g(T)$ is a 
\emph{$m$-good inverse branch of $f$ of order $n$ on $T$} if
\begin{enumerate}
\item $g \circ f^n = id_{g(T)}$;
\item for all $\lam\in \Omega$, $\diam f^l_\lam (g(T)_{|\lam})\leq e^{-m- (n-l)A_0}$ for all $0 \le l \leq n$.
\end{enumerate}
%where $A_0$ is as in {\bf (M2)}.
\end{definition}

Observe that, by definition, we have diam $T_{|\lam}\leq e^{-m}$ for every tube $T$ admitting a $m$-good inverse branch as above. 

\medskip

%Moreover,
Given an inverse branch $g$
of $f^m$ defined on a tube $T$,
given any $\gamma\in \mathcal J$ with $\Gamma_\gamma \subset  T$ 
we can in particular associate to such 
inverse branch a map
$\gamma_g$
%of the element $\gamma$ with 
such that $\Gamma_{\gamma_g}\subset g(T)$ and $\mathcal F (\gamma_g)=\gamma$. In particular, the association $\gamma \mapsto \gamma_g$ defines a map $\mathcal G$ on the ball $\mathcal B_\Omega (\gamma,\eta)$, that we can see as an inverse branch for $\mathcal F$ over such ball.
%The following is the transposition of Definition
%
%
%{\red maybe $\mathcal G$ more general for the inverses of graphs, or $\mathcal F^{-1}_g$}
%We will denote by %$\mathcal G \colon\gamma\mapsto \gamma'$  this correspondence.
%$\gamma_{g}$ this inverse.
%{\red not very clear maybe, check if useful, we only need the inverse of the "center" below}

%{\red from a $g$, define the $G$ on graphs}




Given $\Omega\subset M$ and a tube $T$ at $\gamma\in \mathcal X$ over $\Omega$,
we denote by $\mathcal M_{T,n}^{(m)}$
the measure
%{\red notation if then needed for balls}
%{\red now change with defs above, maybe use $\Lambda_\phi$ and $\Lambda_\psi^*$}
\begin{equation}\label{e:M-good-branches}
\mathcal M^{(m)}_{T,n}:=
%e^{- P(\phi) n} \rho(\gamma(\lambda_0))
\theta (\gamma)^{-1}
\sum_{\gamma_g}
%e^{\phi(\gamma' (\lambda_0)) + \dots \phi (\mathcal F^{n-1} (\gamma' (\lambda_0)))} \rho(\delta (\lambda_0)) 
e^{\psi(\gamma_g)+ \ldots +\psi(\mathcal F^{n-1}(\gamma_g))}
\theta (\gamma_g) \delta_{\gamma_g},
\end{equation}
%{\red need $\mathcal G$}
where the sum is over the preimages $\gamma_g$  of $\gamma$
associated to $m$-good inverse branches $g$ of $f$
of order $n$ on $T$.

\begin{remark}
We have $\mathcal M^{(m)}_{T,n}\leq
%{\green \rho(\gamma)} 
\mathcal M_{\gamma,n}$ for all $n\geq 0$. Hence, the condition \textbf{(M3)} ensures that any limit value $\mathcal M'_T$ of the sequence 
$\{\mathcal M^{(m)}_{T,n}\}_{n}$
satisfies $\|\mathcal M'_T\|\leq 1$.
%{\red [do we need M3 to say this?]}
%{\green Oh yes, since $||\M_{\gamma,n}||$ is not $1$ a priori !}
\end{remark}


\begin{definition}
Given $\Omega \subset M$ and $m>1$, we say that a tube $T$ at $\gamma\in \mathcal X$ over $\Omega$
is 
\emph{$m$-nice} 
if
%\begin{enumerate}
%\item $\inf_{T_{|\lambda_0}} \rho \geq (1-1/m) \sup_{T_{|\lambda_0}}\rho$;
%{\red [adapt, now rho is function on graphs for now; check the letter if changed above] }
%{\blue maybe we remove this condition here and we say later in the construction that we can assume this up to taking balls small?}
%{\green [I think it clarifies a lot to put the condition here (instead of shrinking all the $D_i$'s after the construction). We only need this condition just before section 3.3, right ? Maybe just write a comment]}
%{\red need to check this, maybe we can just remove and say it is automatic later? it should be automatic with $\rho$ continuous, we can just say later that we can also assume this; or maybe we fix an abstract continuous function and we ask this, and then we apply for rho; TODO FOR b now}
%\item
$\|\mathcal M^{(m)}_{T,n}\|\geq 1-1/m$ for all $n$ sufficiently large.
%\end{enumerate}
We say that a ball $\mathcal B_\Omega(\gamma, \eta)$ is $m$-nice if the tube $T_\Omega(\gamma, \eta)$ is $m$-nice.
%{\green Note that, up to shrinking $\eta$, the condition $(1)$ is automatic since $\rho$ is positive continuous and $T_{|\lam_0}$ is relatively compact.}\\ \\
%{\green !!! For this, we need the function $\rho$ to be defined on all $\J$, not only $\mathcal X$.}
%{\red I agree; this is why I proposed to remove (1) and just say it later in section 3, when rho is continuous on J}
\end{definition}


%{\red  REMOVED (1): $\inf_{T_{|\lambda_0}} \rho \geq (1-1/m) \sup_{T_{|\lambda_0}}\rho$; ADD LATER}

%{\red [say ball nice if tube nice}

\begin{remark}
Every $m$-nice tube $T(\gamma,\eta)$ satisfies $\eta\leq e^{-m}$, and $\|\mathcal M'_{T}\|\geq 1-1/m$ for every limit value $\mathcal M'_T$ of the sequence $\mathcal M^{(m)}_{T,n}$.
\end{remark}

The following is the main result of this section, giving a quantitative control on the contraction along $\mathcal M$-generic
inverse branches of $\mathcal F$.

\begin{proposition}\label{p:l411}
Let $\mathcal M$ be a compactly supported and 
ergodic 
%{\red [attention: we need to have ergodic later then]}
web
satisfing {\bf (M1)}, {\bf (M2)}, and {\bf (M3)}.
For every $\Omega \subset M$ and $\gamma\in \mathcal X$, the tube $T_\Omega(\gamma,\eta)$
is $m$-nice if $\eta$ is sufficiently small.
%Nice balls; 
%{\red SEE LEMMA 4.11 BD}
%{\red need to fix things if needed, maybe rho}
\end{proposition}



%Lemma 4.12 is Proposition \ref{p:bbd-gen} of contraction, already in previous section}
In order to prove Proposition \ref{p:l411}, we will make use of the natural extension $(\hat \J, \hat \F, \hat \M)$ of the system $(\mathcal J, \mathcal F, \mathcal M)$.
We refer to \cite[Section 10.4]{CFS} and \cite[Theorem 2.7.1]{PU} for the general setting of Lebesgue spaces.
The main difficulty lies in proving that $\hat \M$ is $\sigma$-additive.
 We recall here how to construct the extension of the Polish space $(\J,\dist_\J)$, introducing some notations that will be used later.

\medskip

%By {\bf (M1)}, the subset 
%{\red [need to decide notation for $\mathcal X$, above is $\mathcal J\setminus \mathcal J_s$. We can keep that]}
%$\mathcal X := \Supp \mathcal M \setminus \mathcal J_s$ satisfies $\mathcal M (\mathcal X)=1$
%and the map
%$\mathcal F\colon \mathcal X \to \mathcal X$ is then defined and surjective. 
%{\red [say preimages well def, number, etc]}
%{\green The natural extension $\widehat{\mathcal S}$ of $\Supp \mathcal M$ is still well defined. I think we should consider $\mathcal X = \Supp \mathcal M$, and say $\widehat{\mathcal S}$ is compact (by Tychonoff ; we don't need axiom of choice for the countable product of compact metric spaces) and separable because of $M3$.}

%Recall that $\J$ is a closed subset of the Polish space $\mathcal O(M,\P^k)$.
We denote by $\hat {\mathcal J}$ the subspace
\[
\hat {\mathcal J} := \{\hat \gamma = (\gamma_n)_{n\in \mathbb Z} \in \mathcal J^\Z \colon \mathcal F (\gamma_n) = \gamma_{n+1}, n\in \Z\},\]
by $\pi_n\colon \hat{\mathcal J} \to \mathcal J$ the projection defined as $\pi_n (\hat \gamma)=\gamma_n$ and by $\hat {\mathcal F}\colon\hat {\mathcal J}\to \hat{\mathcal J}$ the shift map
\[
\hat{\mathcal F}
(\hat \gamma)
%=v\hat {\mathcal F} (\dots, \gamma_{-1}, \gamma_0, \gamma_1, \dots) 
%=
%(\dots \gamma_0, \gamma_1, \gamma_2, \dots)
=
(\mathcal F (\gamma_n))_{n\in \mathbb Z} = (\gamma_{n+1})_{n\in \mathbb Z}.
\]
The maps satisfy $\pi_n \circ\hat {\mathcal F}= \mathcal F \circ \pi_n$ for all $n\in \mathbb Z$.
%The space $\hat \J$ is a closed subspace of the Polish space $\J^\Z$, endowed with the product topology.
We endow $\J$ with its Borel $\sigma$-algebra.
If $\mathcal B \subset \J$ is a Borel set, define $\mathcal A_{n,\mathcal B} := \pi_n^{-1} (\mathcal B)=\{
\hat \gamma \in \hat \J \colon \gamma_n \in \mathcal B
\} \subset \hat {\J}$.
We may consider the set $\mathcal A_{n,\mathcal B}$ as a subset of $\hat \J^{\{-n,\dots,n\}} := \{(\gamma_k)_{-n\le k\le n} \in \J^{\{-n,\dots,n\}}\: : \: \F(\gamma_k) = \gamma_{k+1}, -n\le k < n \}$ as well.

A \emph{cylinder} is a finite intersection of subsets of  $\hat {\mathcal J}$ of the form $\mathcal A_{n,\mathcal B}$,
for some $n\in \Z$ and open set $\mathcal B \subseteq \mathcal J$.
The family of all the cylinders forms a basis for the product topology.
It is also a $\pi$-system
(see \cite[Chapter 1]{Kal}) whose generated $\sigma$-algebra is the Borel $\sigma$-algebra
of $\hat{\mathcal J}$.
By the monotone classes theorem  (see for instance \cite[Theorem 1.1]{Kal}), any measure on $\hat \J^{\{-n,\dots, n\}}$ is thus defined by its values on the cylinders.
Given $\mathcal B_{-n}, \dots, \mathcal B_n \subset \J$ open sets, define
\begin{equation}\label{e:def-hat-M-cylinder}
\hat {\mathcal M}_n (\mathcal A_{-n,\mathcal B_{-n}} \cap \dots \cap \mathcal A_{n, \mathcal B_n})
:=
\mathcal M (\mathcal B_{-n} \cap \F^{-1}(\mathcal B_{-(n-1)}) \cap \dots \cap \F^{-2n}(\mathcal B_{n})).
\end{equation}

The probability measure $\hat {\mathcal M}_n$ is well defined on $\hat \J^{\{-n,\dots,n\}}$.
The invariance of ${\mathcal M}$ and the fact that $\gamma_k \in \mathcal B$ if and only if $\gamma_{k-1} \in \F^{-1}(\mathcal B)$
yield the following consistency condition
\[
\hat \M_n (\mathcal A_{k,\mathcal B}) = \hat \M_{n+1} (\mathcal A_{k,\mathcal B}),
\]
for every Borel set $\mathcal B \subset \J$ and integers $n\in \N$, $k \in \{-n,\dots, n \}$.
% For every $m>0$ and Borel sets
% $\mathcal B_0, \dots, \mathcal B_m \subset \mathcal J$, we have
% \[
% \begin{aligned}
% \hat {\mathcal M}
% (
% \{
% \hat \gamma \colon \gamma_0 \in \mathcal B_0,
% \dots, \gamma_m \in \mathcal B_m
% \}
% )
% & =
% \hat {\mathcal M}
% (\{
% \hat \gamma \colon
% x_{-m} \in \mathcal F^{-m} (\mathcal B_0)\cap \dots \cap \mathcal B_{-m}
% \})\\
% &=
% {\mathcal M}
% (\{
% \mathcal F^{-m} (\mathcal B_0) \cap \dots \cap \mathcal B_{-m}
% \}).\end{aligned}
% \]
By Kolmogorov extension theorem,
%{see for example \green (\cite[Corollary 8.22]{Kal})}, 
%% [this should be ok]
the measure $\hat \M$ then admits a unique extension on $\hat \J$.

%We can extend $\hat{\mathcal M}$ to a probability measure on $\hat \Sigma$. We still denote by $\hat {\mathcal M}$ this extension.
Observe that $\hat {\mathcal M}$ is then
$\hat{\mathcal F}$-invariant and satisfies $(\pi_n)_* \hat{\mathcal M} = \mathcal M$ for every $n\in \Z$.

%{\red see email. I think we only need to use one direction (the past). My previous explanation was not good?}


%{\red NATURAL EXTENSION, done in appendix BR now; redefine all here, a bit differently}

%{\red give details about construction of $\hat {\mathcal M}$ as needed below}

%{\red define here disintegration $\hat {\mathcal M}^\gamma$}


\medskip


In the following, we will use the disintegration of $\hat {\mathcal M}$ with respect to $\mathcal M$ and the projection $\pi_0$, and more precisely 
 the \emph{conditional measures} $\hat{\mathcal M}^\gamma$ of $\hat {\mathcal M}$ on $\{\gamma_0=\gamma\}$. These measures
are uniquely defined for $\mathcal M$-almost every $\gamma \in \mathcal J$ and are characterized by the property that
\[
\langle
\hat {\mathcal M} , g\rangle
= \int_{\mathcal X}
\langle \hat{\mathcal M}^\gamma ,g\rangle
\mathcal M(\gamma) 
\]
for all non-negative bounded measurable functions $g \colon \mathcal J \to \mathbb R$
%{\green We may cite \cite[Theorem 8.1]{Kal}, but we need to replace $g$ by $L^1(\hat \M)$ functions. Is it a problem ?} 
(see \cite{B23} for more details about the construction).
%{\green
%[In practice, the conditional measure is always well defined since the lamination allows to reduce the construction to one fiber (see \cite{B23} for more details).]}
We now describe a useful approximation of the conditional measures $\hat{\mathcal M}^\gamma$, that we will need later.

\medskip


For every $n>0$, consider the projection $\pi^n\colon \hat {\mathcal J} \to \mathcal J^{n+1}$ defined as
\[
\pi^n := (\pi_{-n},\dots, \pi_{0} ).
\]
By {\bf (M1)}, 
$\mathcal X$ satisfies $\mathcal M (\mathcal X)=1$ and the map
$\mathcal F\colon \mathcal X \to \mathcal X$ is well defined and surjective.
For every $(\gamma_{-n}, \dots, \gamma_0)\in \mathcal X^{n+1}$ with $\mathcal F (\gamma_j) = \gamma_{j+1}$ for every $-n \leq j\leq -1$, we choose a representative $\hat \beta \in \hat {\mathcal J}$ such that $\beta_j = \gamma_j$ for all $-n\leq j\leq 0$. Observe that, given any $\gamma_0 \in$ $\mathcal X$, there are $d^{kn}$ elements as above in $\mathcal X^{n+1}$, hence $d^{kn}$ representatives. We denote by
$\hat {\mathcal Z}_n$ the collection of such representatives.



%{\red add part defs}

\medskip


For every $\gamma \in \mathcal X$, define 
%{\red double check this part if I missed some rho }
%{\green \checkmark}
%{\red [not sure it it is safe to divide by $\rho(\gamma)$ in the formulas, check later; in case we need to NOT divide by rho, and moltiply by rho the RHS in Lemma below]}
\[
\hat{\mathcal M}^\gamma_n
:=
\theta(\gamma)^{-1}
\sum_{ \hat \beta \in \hat{\mathcal Z}_n \colon \beta_0 = \gamma }
%a_n (\beta_{-n})
%\delta_{\hat \beta}.
e^{\psi (\beta_{-n}) + \ldots + \psi(\beta_{-1})}
\theta(\beta_{-n})
\delta_{\hat z}.
\]
Observe in particular that, by {\bf (M1)} and the fact that $\theta$ is positive on $\mathcal X$,
$\hat { \mathcal M}^\gamma_n$ is defined for $\mathcal M$-almost every $\gamma \in \mathcal J$.

\begin{lemma}\label{l:disintegration}
%{\red THE CLAIM ABOUT THE DISINTEGRATION; WE CAN PUT IT HERE; CHECK IF OK WITHOUT SPECIFIC PHI}
For every $\gamma \in \mathcal X$, we have
%{\red [maybe multiply by $\rho (\gamma)$ to avoid the division above]}
\[
\lim_{n\to \infty} \hat{\mathcal M}^\gamma_n  =\hat {\mathcal M}^{\gamma}.
\]
\end{lemma}

\begin{proof}
%{\red TODO}
It is enough to show that, for $\gamma \in \mathcal X$, we have
\[
\lim_{n\to \infty} \hat{\mathcal M}^\gamma_{n} (\mathcal A_{-i, \mathcal B}) = \hat{\mathcal M}^\gamma (\mathcal A_{-i,\mathcal B})
\mbox{ for all }  i \geq 0 \mbox{ and Borel set } \mathcal B.
\]
%{\red double check if the notations are ok}
%{\green \checkmark}
Hence, the assertion is
a consequence of the following two identities:

\begin{enumerate}
\item $\hat {\mathcal M}^\gamma_n (\mathcal A_{-i,\mathcal B})=
\hat {\mathcal M}^\gamma_i (\mathcal A_{-i,\mathcal B})$
for every $\gamma \in \mathcal X$ and all $n> i \geq 0$;
\item $\int \hat{\mathcal M}^\gamma_i (\mathcal A_{-i, \mathcal B}) \mathcal M (\gamma)= \mathcal M (\mathcal B) $ for all $i\geq 0$.
\end{enumerate}

\medskip

We start proving the first identity above. This is a consequence of the identity $\Lambda_\psi \theta = \theta$. Indeed, for every $n>i\geq 0$ and Borel set $\mathcal B$, we have
\[
\begin{aligned}
\hat{\mathcal M}^\gamma_n (\mathcal A_{-i, \mathcal B})
%&= \hat{\mathcal M}^\gamma_n (\mathcal A_{-i,\mathcal B}\cap \pi_0^{-1} (\gamma) ) \text{ \green why this line ?}\\
&= \theta(\gamma)^{-1}
\sum_{\hat \beta \in \hat {\mathcal Z}_n \colon \beta_0=\gamma}
e^{\psi (\beta_{-n}) + \ldots + \psi (\beta_{-1})}
\theta (\beta_{-n})
 \delta_{\hat \beta} (\mathcal A_{-i, \mathcal B})\\
 &= \theta(\gamma)^{-1}
 \sum_{\hat \beta \in \hat {\mathcal Z}_i \colon \beta_0=\gamma}
 (\Lambda_{\psi}^{n-i}\theta) (\beta_{-i})
e^{\psi (\beta_{-i}) + \ldots + \psi (\beta_{-1})}
 \delta_{\hat \beta} (\mathcal A_{-i, \mathcal B})\\
 &=\theta(\gamma)^{-1}
 \sum_{\hat \beta \in \hat {\mathcal Z}_i \colon \beta_0=\gamma}
 \theta (\beta_{-i})
e^{\psi (\beta_{-i}) + \ldots + \psi (\beta_{-1})}
 \delta_{\hat \beta} (\mathcal A_{-i, \mathcal B})\\
 &=\hat{\mathcal M}^\gamma_i (\mathcal A_{-i, \mathcal B}),
\end{aligned}
\]
which proves the first identity.


\medskip


Let us now prove the second identity. This time, we will use
the fact that $\mathcal N$ is a fixed point for $\Lambda^*_\psi$.
%{\red [check if ok, otherwise need to do define and use the Jacobian for $\mathcal M$}.
For all $i\geq 0$,
we have
%{\red [need to define indicatrix if not done already before]}
\[
\begin{aligned}
\int \hat{\mathcal M}^\gamma_i (\mathcal A_{-i, \mathcal B}) \mathcal M (\gamma)
&=
\int 
\Big(
\theta(\gamma)^{-1}
\sum_{\hat \beta \in \hat {\mathcal Z}_i \colon \beta_0=\gamma}
 \theta (\beta_{-i})
e^{\psi (\beta_{-i}) + \ldots + \psi (\beta_{-1})}
 \delta_{\hat \beta} (\mathcal A_{-i, \mathcal B})
\Big) \mathcal M(\gamma)\\
&=
\int 
\Big(
\theta(\gamma)^{-1}
\sum_{\mathcal F^i (\gamma')=\gamma}
 \theta ({\gamma'})
e^{\psi (\gamma') + \ldots + \psi (\mathcal F^{i-1} (\gamma'))}
\mathbb 1_{\mathcal B} (\gamma)
\Big) \mathcal M(\gamma)
\\
&=
\big\langle \mathcal M, \theta^{-1}
%\mathcal F^i_* 
%\big( e^{\psi  + {\green \psi \circ \mathcal F + } \dots + \psi {\green \circ \mathcal F^{i-1}}} \theta \mathbb 1_{\mathbb B}\big) 
\Lambda^i_\psi (\theta \mathbb 1_{\mathcal B})
\big\rangle
%{\green \text{ maybe write } \theta^{-1}\Lambda_\psi^i(\theta \mathbb 1_{\mathcal B})}
= 
%\Big\langle 
%\mathcal N,\mathcal F^i_* 
%\big(e^{\psi (\gamma') + \dots + \psi (\gamma')} \theta \mathbb 1_{\mathbb B}\big) 
%\Big\rangle
\big\langle
\mathcal N,\Lambda_\psi^i(\theta \mathbb 1_{\mathcal B}) 
\big\rangle
\\
& = \langle  (\Lambda_\psi^i)^* \mathcal N, \theta \mathbb 1_{\mathcal B}\rangle
= \langle \mathcal N, \theta \mathbb 1_{\mathcal B}\rangle = \mathcal M (\mathcal B),
\end{aligned}
\]
where we denoted by $\mathbb 1_{\mathcal B}$ the indicatrix function of $\mathcal B$ and in the last step we used the identity $\mathcal M = \theta \mathcal N$. The proof is complete.
%
%
%
%
%REMARK: WE CANNOT DO THE FIRST STEP, ie the estimate with $n>i$ to  reduce the problem to i. We must do directly the limit in n for all i}
%
%{\red if we put proof, maybe say that the proof in BD is specific of that case}
%
%{\blue ATTENTION: I THOUGHT ABOUT THIS; maybe we really need some control on the Jacovian of $\mathcal M$, see $\mu_\phi$ in BD; if so, we need  to put $\phi$ etc before, and this part becomes less general.}
%
%{\red now it should be ok with the definitions above, I hope}
\end{proof}



For every $n\geq 0$, we denote by $f_{\hat \gamma}^{-n}$ the inverse branch of $f^n$, defined in a neighbourhood of $\Gamma_\gamma$, and such that $f_{\hat \gamma}^{-n} (\gamma(\lam)) = \gamma_{-n} (\lam)$ for all $\lam \in M$.
Such branch exists for every $\gamma \in \mathcal X$, but a priori it is only defined in some (non-controlled) 
neighbourhood of $\Gamma_{\gamma}$.
The following proposition (see
 \cite[Propositions 4.2 and 4.3]{BBD18} for the case of the equilibrium web and %\cite[Proposition {\red A.?}]{BR} 
 \cite[Proposition A.1]{BR22}
 for the general case) 
gives a uniform control in $\lambda$
on the size of the neighbourhood of $\gamma (\lambda)$
where such inverses are defined.
%{\red [see comments]}
%{\red [need a sentence about the fact the we do not put the p, see BBD]}

%{\red [I write the statement without the p for simplicity; need to a comment to say it is ok]}


%{\red [If not only p=1, just iterate the system otherwise -  should be ok since $p$ is fixed/finite; cite BriendDuval or somewhere in my thesis where i use p=1]}



%{\red  [check about the Omega and all the definitions for tubes etc]}

%The following is from BR / BBD for equilibrium web

%{\red need to localize in M too, tube is local}

\begin{proposition}\label{p:bbd-gen}
Let $(M,f)$ be a stable
holomorphic family of endomorphisms of $\mathbb P^k$.
%Fix $\lambda_0\in M$ and 
%an open neighbourhood $\Omega\Subset M$ of $\lambda_0$.
Let $\mathcal M$ be 
%an ergodic compactly supported
a web satisfying {\bf (M1)} and  {\bf (M2)}.
%
%
%given by REF.
 Then, for every open set $\Omega \Subset M$ and $0<A<A_1$,
 there exists
 %$p\geq 1$,
 a Borel subset $\hat {\mathcal Y}\subseteq \hat {\mathcal J}$ with
 $\hat {\mathcal M} (\hat { \mathcal Y})=1$, and two
 measurable functions $\hat \eta_{A} \colon \hat {\mathcal Y} \to ]0,1]$
 and $\hat l_A \colon \hat {\mathcal Y}\to [1,+\infty[$
 %, and a constant $0<A<A_0$ 
 which satisfy the following properties.

 For every $\hat \gamma \in \hat{\mathcal Y}$ and every $n\in  \mathbb N^*$ the iterated inverse branch $f_{\hat \gamma}^{-n}$ is defined and Lipschitz on the tubular neighbourhood 
 $T_{\Omega} (\gamma_0, \hat \eta_A (\hat \gamma))$
 of the graph $\Gamma_{\gamma_0}\cap (\Omega \times \P^k)$ of $\gamma_0$, 
 and we have
 %{\red say better the second, maybe define something like $\tilde \Lip$}
 \[
f_{\hat \gamma}^{-n} (T_{\Omega} (\gamma_0, \hat \eta_A (\hat \gamma) ))\subset T_{\Omega} (\gamma_{-n}, e^{-nA})
 \quad
 \mbox{ and } \quad
 \widetilde \Lip ( f^{-n}_{\hat \gamma} )\leq \hat l_A  (\hat \gamma) e^{-nA},
 \]
 where $\widetilde \Lip
 ( f^{-n}_{\hat \gamma} ):= \sup_{\lam \in \Omega} \Lip \big( (f_{\hat \gamma}^{-n})_{|B(\gamma_0(\lam), \hat \eta_A)} \big)$.
\end{proposition}




We can now prove Proposition \ref{p:l411}.


\begin{proof}[Proof of Proposition \ref{p:l411}]
%{\red the proof of lemma 4.11 bd to be done for graphs, using Proposition \ref{p:bbd-gen} instead of Lemma 4.12 in BD.}
%
%{\red disintegration has been put above in separate lemma}
%
Fix $m>0$, a constant $A_0< A <A_1$ 
and a positive integer $r$.
For every $N\in \mathbb N$, define
\[
\hat {\mathcal Y}_N := \{
\hat \gamma \in \hat{\mathcal Y} \:\colon\:
\eta_A (\hat \gamma) \geq N^{-1} \mbox{ and } l_A (\hat \gamma)\leq N 
\}.
\]
By Proposition \ref{p:bbd-gen},
we 
have $\lim_{N\to \infty} \hat{\mathcal M} (\hat{\mathcal Y}_N)=1$. 
Fix $N_0 = N_0 (m, r)$ such that
$\hat{\mathcal M}(\hat{\mathcal Y}_N)> 1-1/(2m^{r+1})$
for all $N\geq N_0$. Then, by Markov inequality and the definition of the measures $\hat {\mathcal M}^\gamma$,
we see that there exists a subset $\mathcal X_r \subset\mathcal X$
such that $\mathcal M (\mathcal X_r) > 1-1/m^r$ and
\begin{equation}\label{e:disintegration-error-m}
\hat {\mathcal M}^\gamma
(\hat {\mathcal Y}_N\cap \{\gamma_0 = \gamma\})
> 1-1/(2m) 
\quad
\mbox{ for all } \gamma \in \mathcal X_r
\quad \mbox{ and } \quad
N\geq N_0.\end{equation}
In order to prove the statement, it is enough to prove the assertion for all $\gamma \in\mathcal X_r$.
Fix one such $\gamma$. It follows from
Proposition \ref{p:bbd-gen}
that all inverse branches defined 
on $T_\Omega (\gamma, e^{-m}/(2N))$ and
corresponding to $\hat \gamma \in \hat {\mathcal Y_N}\cap \{\gamma_0=\gamma\}$ are $m$-good for all $n$.
It follows from Lemma \ref{l:disintegration} and \eqref{e:disintegration-error-m} that
\[
\hat {\mathcal M}^\gamma_n
(\hat {\mathcal Y}_N\cap \{\gamma_0 = \gamma\})
> 1-1/m 
\quad
\mbox{ for all } 
n  \mbox{ large enough}.
\]
This implies that, for all $n$ sufficiently large, we have
$\|\mathcal M^{(m)}_{T,n} \|> 1-1/m$, 
for $T= T_\Omega(\gamma, e^{-m}/(2N))$. By definition, this means that $T$ is $m$-nice.
%{\red continue, it should be ok if disintegration is ok, it should be just formal}
\end{proof}




\section{Proof of Theorem \ref{t:main}}

%{\red comments. in this section

%- ss1: recall/define the web that we use, moving the eq states, and satisfying M1,2,3

%- ss2: main proposition of construction (Lemma 4.14 BD)

%- ss3 proof of thm for that web, using Proposition \ref{p:l411}

%}



%{\red [maybe in this section we drop $\mathcal M_{\lam_0, \mu_\phi}$ and just use $\mathcal  M_\phi$, for simplicity. I write just $\mathcal M$ for now, to be fixed later]}



%{\red [maybe we drop the omega and we say that we just work locally for simplicity. Need to fix this]}



%{\red [here or below: sentence saying that we follows BD, and mention mention Xavier trick]}

\subsection{Equilibrium states and associated webs}
\label{ss:equilibrium}
We work here in the assumptions of Theorem \ref{t:bbd}.
In particular, we assume that $f_{\lam_0}$
satisfies condition {\bf (A)} and $\phi$ satisfies the
conditions in {\bf (B)} in the Introduction.
%the assumptions in that theorem.
Up to replacing $\phi$ by $\phi- P(\phi)$, we can assume that $P(\phi)=0$.
%{\green maybe precise that $\implies \lam = 1$ in \cite{BD23} ?}{\red [ok, later when we cite BD23]}.
We denote by $\Lambda_\phi$ the (Ruelle-Perron-Frobenius) transfer operator
\[
\Lambda_{\phi} (g) (y) := \sum_{f_{\lam_0}(x)=y}e^{\phi (x)} g(x)
\]
acting on $\mathcal C^0 (\P^k)$, where the elements in the sum are counted with multiplicity.
By \cite[Theorem 1.1]{BD23}, there exists 
a unique 
probability measure $m_\phi$
%{\red [maybe other notation, anyway this measure is not too important/used here]}
and
a unique (up to multiplicative constant) strictly positive
continuous function $\rho_\phi \colon \P^k \to \R$ such that
%{\red [here we need to change, depending on the HP in section 2]}
\[
\Lambda_\phi^n (g) \to \langle m_\phi, g \rangle \rho_\phi
\quad \mbox{ and }
\quad (\Lambda_\phi^n)^* \delta_x \to \rho_\phi (x) m_\phi
\]
for all continuous function $g\colon \P^k \to \R$ and all $x \in \P^k$.
%{\red need scaling ratio}
We normalize $\rho_\phi$ so that $\langle m_\phi, \rho_\phi\rangle=1$. The probability measure $\mu_\phi := \rho_\phi m_\phi$ is then $f_{\lam_0}$-invariant, and is the 
unique 
equilibrium state  for $f_{\lam_0}$
associated to $\phi$.
%(which is in particular an $f_{\lam_0}$-invariant measure), and a continuous function
%$\rho\colon \P^k \to \R$ with the property that, for every $x\in \P^k$, 
%\[
%\sum_{f(y)=x} e^{\phi(y)}\rho(y) = e^{P(\phi)}\rho(x).
%\]
%Such function is unique up to multiplicative constant. We fix a choice of $\rho$ in the following.
In particular, for every
$x \in \P^k$, the measures  
\[
\mu_{x,n} := \rho_\phi \cdot 
\rho_\phi(x)^{-1}  
(\Lambda_\phi^n)^*  \delta_x=
%e^{-P(\phi)n}
\rho_\phi^{-1}(x)\sum_{f^n(y)=x} e^{\phi(y)+ \ldots +\phi(f^{n-1} (y))} \rho_\phi(y)\delta_y
\]
satisfy
$\mu_{x,n}\to \mu_\phi \mbox{ as } n \to \infty$.

\medskip

By \cite[Proposition 4.9]{BD23}, $\mu_\phi$ satisfies
$h_{\mu_\phi}> \log d^{k-1}$. Hence, by
\cite{BR22}, there exists an ergodic 
%{\red [check: unique, as measure]}{\green [(M1) + $\M(\L)= 1$ guarantees uniqueness as a measure. I suggest we put it in the paper with Karim.]}
web $\mathcal M_{\lam_0,\mu_\phi}$ on $\mathcal J$ with the property that
$(p_{\lam_0})_{*} \mathcal M_{\lam_0, \mu_\phi}=\mu_\phi$
and
which satisfies {\bf (M1)} and {\bf (M2)}.
%{\red [maybe need the laminarity to say better uniqueness; check/add if needed]}
The following lemma, proved in \cite{BR}, gives in particular the uniqueness of such web.
%{\red [is this lemma enough to recover the ergodicity needed in 2.8? Or this should come from the paper with Karim already?]}

\begin{lemma}\label{l:strong-uniqueness}
Let $(M,f)$ 
be a stable family of endomorphisms of $\mathbb P^k$. Let $\mathcal M_1$ and $\mathcal M_2$ be two
probability measures compactly supported on $\mathcal J$. Assume that $\mathcal M_1$ is an acritical web and that there exists $\lam_0 \in M$ such that
$(p_{\lam_0})_* \mathcal M_1= (p_{\lam_0})_*\mathcal M_2$.
Then $\mathcal M_1 = \mathcal M_2$.
%
%an acritical web and $\mathcal M'$ be any p 
\end{lemma}
%\medskip
%{\red double check that this is really correct before we send this paper}


Define the functions
$\theta, \psi \colon \mathcal J \to \R$ as
%{\green We should say $\rho$, $\psi$ are continuous for LU convergence ?}
\[
\theta (\gamma) := \rho_\phi (\gamma (\lam_0))
\quad
\mbox{ and }
\quad 
\psi (\gamma) :=
%:=e^{-P(\phi) n} e^{\phi (\gamma(\lam_0)) + \dots + \phi(f^{n}_{\lam_0} (\gamma(\lam_0)))} \rho (\gamma(\lam_0))
\phi(\gamma (\lam_0)).
\]
Observe that these functions are continuous on $\mathcal J$.
In particular, for every $m\in \N^*$, any graph $\gamma \in \J$ admits an open neighbourhood $\mathcal A \subset \J$ such that
\begin{equation}\label{e:small_enough}
\inf_{\mathcal A}\theta \geq (1-1/m) \sup_{\mathcal A}\theta.
\end{equation}
To see this, take an open neighbourhood $\mathcal A' \subset \J$ of $\gamma$.
By the continuity of $\theta$ at $\gamma$, and the fact that $\rho_\phi$ is positive on the compact set $\P^k$, there exists an open neighbourhood $\mathcal A \subset \mathcal A'$ such that
\[
\theta(\gamma_1) - \theta(\gamma_2) \le \frac{\inf_{\mathcal A'} \theta}{m}
\quad 
\mbox{ for any } \gamma_1, \gamma_2 \in \mathcal A.
\]
Observe that $\inf_{\mathcal A'} \theta \le \inf_{\mathcal A}\theta \le \sup_{\mathcal A} \theta$.
Taking respectively the supremum over $\gamma_1$ and the infimum over $\gamma_2$
proves the claim.

\medskip


Consider the operator $\Lambda_\psi$
%\colon \mathcal C^0 (\mathcal X) \to \mathcal C^0 (\mathcal X)$ 
defined as in \eqref{e:def-Lambda-psi}.
For $\gamma \in \mathcal X$, set $\M_{\gamma,n}$ as defined in \eqref{e:web_definition}.
%{\green Is it still clear that $\Lambda_\psi \in \mathcal L_c(\mathcal C^0(\mathcal J))$ ?}
It follows from the above and Lemmas \ref{l:rho-N-equivalent},
\ref{l:criterion-conditions}, and
\ref{l:strong-uniqueness},
that $\theta$ satisfies $\Lambda_\psi (\theta)= \theta$
and that there exists a unique probability measure $\mathcal N$ on $\mathcal X$
which is a fixed point
for $\Lambda_\psi^*$.
Furthermore, the web $\mathcal M_{\lam_0, \mu_\phi}$ satisfies
$\mathcal M_{\lam_0, \mu_\phi}= \theta \mathcal N$, and for any $\gamma \in \mathcal X$ we have
%and, 
%{\red [maybe now remove this, or anyway reduce/reverse order]}
%for every $\gamma \in \mathcal X$, we have
\begin{equation}\label{e:conv-42-web}
\begin{aligned}
\M_{\gamma,n}
&=
\theta(\gamma)^{-1} 
\sum_{\mathcal F^n (\gamma')=\gamma}
e^{\psi(\gamma') + \ldots + \psi(\mathcal F^{n-1}(\gamma'))}
\theta (\gamma')
\delta_{\gamma'}
\\
&=
\theta (\gamma(\lambda_0))^{-1}
\sum_{\mathcal F^n (\gamma')=\gamma} e^{\phi(\gamma' (\lambda_0)) + \ldots +\phi (\mathcal F^{n-1} (\gamma') (\lambda_0))} \theta (\gamma' (\lambda_0)) \delta_{\gamma'}\\
& \to \mathcal M_{\lam_0, \mu_\phi}
\mbox{ as } n\to \infty.
\end{aligned}
\end{equation}
In particular, $\mathcal M_{\lam_0, \mu_\phi}$ satisfies condition {\bf (M3)}. 






%{\red [CHECK IF above is ok; should follow from \cite{BR22}, or your paper with Karim, check; maybe lemma with more details in case if needed]}







%BD eq states; (maybe equidistribution repelling)

%{\red if needed... maybe just in intro; if we can find a way to be more "light" in the intro, we put statements here}




\medskip



%By [BD or what recalled above], for any continuous function $g \colon \mathcal J \to \R$, we have
%{\red notation scalar for integral}
%\[
%\langle\mathcal M_{\gamma,n}, g\rangle
%\to
%\langle \mathcal M_{\lambda_0,\mu_\phi} , g\rangle
%\mbox{ as } n\to \infty
%\]
%In particular, this implies that $\mathcal M_{\gamma,n}\to \mathcal M_{\lam_0, \phi}$ as $n\to \infty$
%for all $\gamma \in \mathcal J$.

%\medskip

We conclude this section with the following lemma,
that we will need in the next section. 
Recall that $q>2$
%is as in Theorem \ref{t:main}
and $\|\phi\|_{\log^q}<\infty$.

\begin{lemma}\label{l:l413}
%{\red lemma 4.13 bd; series}
There exists a positive constant $C=C(A_0,q)$ such that, for all $n\in \mathbb N$, $m\geq 0$, and every $m$-good inverse branch $g\colon T \to g(T)$ of $f$ of order $n$ on a tube $T$, and for all sequences of graphs
$\{\gamma^1_l\}_{0 \leq l\leq  n-1}$ and
$\{\gamma^2_l\}_{0 \leq l \leq n-1}$
%with $0\leq l \leq n-1$ and
with $\Gamma_{\gamma_l^1}, \Gamma_{\gamma_l^2}\subset 
f^l (g(T))$ for all $0\leq l \leq n-1$,
we have
%{\red maybe write with psi if needed later}
\[
\sum_{l=0}^{n-1}
|\psi (\gamma_l^1) -\psi( \gamma^2_l)|\leq C m^{-(q-1)}.
\]
\end{lemma}

\begin{proof}
It follows 
from the definition of $\psi$ that
\[
\sum_{l=0}^{n-1}
|\psi (\gamma_l^1) -\psi( \gamma^2_l)|
=
\sum_{l=0}^{n-1}
|\phi (\gamma_l^1 (\lam_0)) -\phi( \gamma^2_l (\lambda_0))|.
\]
The assertion is a consequence of \cite[Lemma 4.13]{BD23}.
%{\red Since it is an estimate at $\lambda_0$, it should be enough to apply BD 4.13 (or proof is not long in case).}
\end{proof}





%We fix in what follows a positive constant $A_0 < A$, where $A$ is the positive constant given by Proposition \ref{p:bbd-gen}.






\subsection{Construction of repelling graphs}\label{ss:construction}

%{\red [some text; need to fix notations for tubes and balls; I mix them for now]}

For simplicity, we denote in this section by $\mathcal M$ the web $\mathcal M_{\lam_0,\mu_\phi}$ defined in the previous section.
We also recall that we are assuming,
with no loss of generality, that $P(\phi)=0$.
We will 
fix a relatively compact open subset $M'\Subset M$, and only consider the graphs of elements of $\mathcal J$ as graphs over $M'$. In particular, all distances and balls are with respect to the distance $\dist_{M'}$, see \eqref{e:distance-Omega}.


%{\red [question. As we discussed, if we use this distance we lose the completeness. but if we work only on Supp M, do we recover completeness thanks to the lemma in BBD actually?]}
%{\green [The support is not closed for the topology of uniform convergence on M'.]}

%{\blue [Recall notations maybe: we can use $\psi(\gamma)$ instead of $\phi(\gamma(\lam_0))$ below, it's shorter; we can remove pressure as we put 0 (recall, leave in the statement but we say in the proof that we can assume it's zero)]}



The following is the main result of this section and the key estimate to prove Theorem \ref{t:main}. Thanks to the results proved in Section \ref{s:prelim-inverses}, we can follow
the strategy of \cite[Lemma 4.14]{BD22}. In particular we also employ a trick due to Buff \cite{B05} which simplifies the original proof of the equidistribution of repelling points with respect to the measure of maximal entropy in %Briend-Duval's proof.
\cite{BD99}.

\begin{proposition}\label{p:l414}
%{\red [lemma 4.14, estimate THIS IS THE MAIN LEMMA; CHECK if sets in $\mathcal J$ is enough or need more general; I guess we actually want to use tubes as before]}
Let $\mathfrak U$ be a finite collection of disjoint 
open subsets of $\mathcal J$. For every $m>0$ there exists $n(m, \mathcal \mathfrak U)>m$
and, for every $n> N(m,\mathfrak U)$, a set $\mathcal Q_{m,n}$ of motions of repelling periodic points of period $n$ such that, for all $ \mathcal U\in \mathfrak U$ we have
%{\red maybe notation psi, to recall etc}
\[
(1-1/m) \mathcal M (\mathcal U)
\leq
%e^{-nP(\phi)}
\sum_{\gamma \in \mathcal Q_{m,n}\cap U}
e^{\psi (\gamma) + \ldots + \psi (\F^{n-1} (\gamma))}
\leq 
(1+1/m) \mathcal M (\mathcal U)
\]
\end{proposition}

%{\red [maybe notation $T$ tube gives $\mathcal T$ a ball in $\mathcal J$? Does it make sense, yes? I use the same letter for now, we'll have to check/change later]}

%{\red [maybe we call a nice ball in $\mathcal J$ a ball associated to a nice tube... and everything becomes even more the same. In case, we say that we give sketch, for example skip something]}



Before proving Proposition \ref{p:l414}, we make some preliminary simplifications. First of all, it is enough to prove the statement for a single open set $\mathcal U$. The general case follows
setting $n(m,\mathfrak U):= \max_{\mathcal U \in \mathfrak U} n(n,U)$.

Assume that $\mathcal M (\mathcal U)=0$. In that case, it is enough to take
$n(m,\mathcal U)=m+1$ and $\mathcal Q_{m,n}= \emptyset$. Hence, we can assume that $\mathcal M (\mathcal U)>0$.

Fix integers $m_2 \gg m_1 \gg m$. By Proposition \ref{p:l411}, for $\mathcal M$-almost every $\gamma$, the tube $T_{M'}(\gamma, \eta)$ 
(and hence the ball $\mathcal B_{M'}(\gamma, \eta)$ of $\mathcal J$)
%{\red [need attention on tube and balls]}
is $m_2$-nice if $\eta$ is sufficiently small (depending on $\gamma$).
It follows that, for $\mathcal U$ as above, we can find a finite union of $m_2$-nice balls 
$\mathcal B_i\subset \mathcal J$ with $\mathcal B_i\Subset U$,
satisfying \eqref{e:small_enough},
whose centers belong to $\Supp \M$,
%{\red [precise what this means, fibered relatively compact; do at the beginning all the defs for tubes and balls together]} 
and with the property that
$\mathcal M (\mathcal U \setminus \cup_i \mathcal B_i) < \mathcal M (\mathcal U)/m_2$.
%{\red [here for example we need to replace the tubes with balls to make the difference]}. 
Hence, it is enough to prove the statement for each of the balls  $\mathcal B_i$.
Namely,  the above gives that, in order to prove Proposition \ref{p:l414},
it is enough to 
 to prove the following statement.
% {\red [maybe again statement/proposition]}

%{\red need that also T has 3.1?}

%\medskip

\begin{proposition}\label{p:l414-reduced}
%{\red [Maybe we need to say: for $m_2$ suff large...; again maybe psi]}
For every $m_1>0$ there exists $\bar m_2 (m_1)>m_1$ such that, for all  $m_2> \bar m_2$
%sufficiently large (depending on $m_1$) such that the following assertion holds.
%
%Let 
and 
every
$m_2$-nice ball
$\mathcal T= \mathcal B(\gamma, \eta)$
with $\gamma \in \Supp \M$ and satisfying \eqref{e:small_enough} (with $\mathcal A = \mathcal T$ and $m = m_2$),
there exists $n(m_2)> m_2$ and, for all $n\geq n(m_2)$ a set $\mathcal Q_n\subset \mathcal T\cap \Supp \mathcal M$
of motions of repelling $n$-periodic points 
%in $T \cap \Supp \mathcal M$
with the property that
\[
(1-1/m_1) \mathcal M (\mathcal T)
\leq
%e^{-nP(\phi)}
\sum_{\gamma \in \mathcal Q_n}
e^{\psi (\gamma) + \ldots +\psi (\F^{n-1} (\gamma))}
\leq 
(1+1/m_1) \mathcal M (\mathcal T).
\]
%{\red [(possibility) maybe here and everywhere we can write $a_n$ or similar for the coefficients to simplify all the sums, when we don't really use what they are]}
\end{proposition}

%{\red [we need to say somewhere what is $\mathcal M (T)$ for a tube, since a priori a tube is in $M\times P^k$; maybe change notation and use $\mathcal B$ for a ball in the graphs, which corresponds to a tube in practice, for example $\mathcal B_T$; or $\mathcal T$; I don't do this for now]}


Since $\M(\mathcal T) > 0$, we fix an integer $m_3 \ge m_2/ \mathcal M(\mathcal T)$
and a ball $\mathcal T^\star := \mathcal B_{M'}(\gamma, \eta^\star)$, where $\eta^\star < \eta$ is such that 
$\mathcal M( \mathcal T^\star)> (1-1/m_2) \mathcal M (\mathcal T)$.
%{\red [check if this is ok/possible} {\green yes, because in a Polish space, every measure is regular]}. 
We also denote by $T$ and $T^\star$
the tubes corresponding to $\mathcal T$ and $\mathcal T^\star$. 
%{\red now defined before, check/say recall/uniformize Omega or M'}
Recalling that the support of $\mathcal M$ is compact,
we also 
choose a finite family of disjoint $m_3$-nice balls %{\red [balls etc; maybe the Di are not really tubes with constant radius, check]}
$\mathcal D_i := \mathcal B_{M'}(\gamma_i, \eta_i)$ such that
$\mathcal M (\cup_i \mathcal D_i) > 1-1/m_3$ and satisfying \eqref{e:small_enough} (with $\mathcal A = \mathcal D_i$ and $m=m_3$).
Set $\mathcal D:= \cup_i \mathcal D_i$.
We also fix balls 
%{\red [idem as above; check; maybe uses laminarity? {\green still no}]} 
$\mathcal D^\star_i := \mathcal B_{M'}(\gamma_i,\eta^\star_i) \subset \mathcal D_i$ with $\eta_i^* < \eta_i$, $\mathcal M (\cup_i \mathcal D^\star_i)> 1-1/m_3$
and set $\mathcal D^\star := \cup_i \mathcal D^\star_i$.
%{\red [likely rewrite better; defs are used later]}
%
%{\red the next two could become a single lemma, items 1,2, in case; not very important}
We will denote by $D_i$
the tube corresponding to $\mathcal D_i$ and set $D:=\cup D_i$.


\begin{lemma}\label{l:claim1}
There exists an integer $M_1 = M_1 (m_2, \mathcal T,\mathcal T^\star, \mathcal D_i)$
such that
\[
(1-4/m_2)
\mathcal M (\mathcal T)
\leq
\mathcal M^{(m_3)}_{D_i, N} (\mathcal T^\star)
\leq
(1+4/m_2)
\mathcal M(\mathcal T)
\mbox{ for all } i \mbox{ and } N\geq M_1.
\]
\end{lemma}

Recall that $\mathcal M^{(m_3)}_{D_i,N}$ 
is defined in \eqref{e:M-good-branches}.

\begin{proof}
Since the balls $\mathcal D_i$ are $m_3$-nice
and $m_3 \ge m_2 / \mathcal M (\mathcal T)$,
we have
\[
\mathcal M^{(m_3)}_{D_i,N}
\geq 
(1-\mathcal M(\mathcal T)/m_2)
\quad
\mbox{ for all } i \mbox{ and all } N
\mbox{ large enough}.
\]
%where
%{\blue $D_i$ is the tube corresponding to $\mathcal D_i$ and}
%we recall that $\mathcal M^{(m_3)}_{D_i,N}$ 
%is defined in \eqref{e:M-good-branches}
%{\red [check if maybe repeat def in this section/define for balls/etc]}.
As $\mathcal M^{(m_3)}_{D_i,N}\leq \mathcal M_{\gamma_i, N}$ and $\|\mathcal M_{\gamma_i, N}\|= 1 + o(1)$, it follows that
\[
\|\mathcal M_{\gamma_i, N} 
-\mathcal M^{(m_3)}_{D_i, N}\|\leq 
\mathcal M (\mathcal T) / m_2 + o(1).
\]
Hence, 
it is enough to prove that
\[
(1-2/m_2)\mathcal M (\mathcal T)
\leq 
\mathcal M_{\gamma_i, N} (\mathcal T^\star)
\leq 
(1+2/m_2)\mathcal M(\mathcal T)
\mbox{ for all } i \mbox{ and all } N
\mbox{ large enough}.
\]
As $\mathcal M(\mathcal T^\star)\geq (1-1/m_2) \mathcal M(\mathcal T)$, this is a consequence of \eqref{e:conv-42-web}.
%{\red COMPLETE}
\end{proof}

\begin{lemma}\label{l:claim2}
There exists an integer $M_2 = M_2 (m_2, \mathcal T, \mathcal D^\star)$
such that
\[
(1-4/m_2)
\leq
\mathcal M^{(m_2)}_{T, N} (\mathcal D^\star)
\leq
(1+4/m_2)
\mbox{ for all } i \mbox{ and } N\geq M_2.
\]
\end{lemma}

\begin{proof}
Since $\mathcal T$ is $m_2$-nice,
%{\red notation tube},
we have
\[
\|\mathcal M^{(m_2)}_{T,N}
\|
\geq (1-1/m_2)
\mbox{ for all } N \mbox{ large enough}.
\]
Recall also that $\mathcal M^{(m_2)}_{T,N} \leq \mathcal M_{\gamma, N}$
and that 
$\|\mathcal M_{\gamma,N}\|\leq 1+o(1)$. So, in order to prove the statement, it is enough to prove that
\[
1-2/m_2 \leq
\mathcal M_{\gamma,N} (\mathcal D^\star)
\leq 1+1/m_2.
\]
Since $\mathcal M (\mathcal D^\star)\geq 1-1/m_3$ and $m_3 \ge m_2$, this follows from \eqref{e:conv-42-web}.
\end{proof}


%{\red blabla defs, paragraph 1}



%{\red blabla comments after the rep points/cycles, fix constants}

We now construct the set $\mathcal Q$ of motions of repelling periodic points as in Proposition \ref{p:l414-reduced}.

\medskip

%{\red [recall $D$ is not tube but union of tubes! Or maybe not even tubes...]}

For every $N_1$ sufficiently large,
every element
in the support
%{\red [again notation balls/tubes]}
of $\mathbb 1_{\mathcal T^\star} \mathcal M^{(m_3)}_{D_i, N_1}$
corresponds to an $m_3$-good inverse branch of $f:M\times \P^k \to M\times \P^k$ %{\red [fibered map, def/ref/recall]} 
of order $N_1$ on the tube $D_i$, which maps $D_i$
to
%{\green [Why not $T^*$ here ?]} 
%{\red [because only the center is in $T^\star$. With N1 big all the image is in T]}
a 
%{\red [fibered, maybe define $\Subset$ for tubes or similar at the beginning]}
%relatively compact {\red [give name, not a tube, need more general for preimages/images]} 
subset of the tube $T$
whose slice at any $\lambda$ is relatively compact in $T_{|\lambda}$.
In the same way, for every $N_2$ sufficiently large, every element in the support of 
$\mathbb 1_{\mathcal D^\star}\mathcal M^{(m_2)}_{T, N_2}$
corresponds to an $m_2$-good inverse branch of order $N_2$
sending the tube $T$ to a 
%relatively compact {\red [define above]} 
subset of $D= \cup_i D_i$.

\medskip

We define the collection $\{g_j\}$ to be the compositions
of such inverse branches, giving inverse branches for $f^{N_1+N_2}$ sending the tube 
$T$ to
%{\green [$T^*$ again ?]}
%{\red [as before]}
a
%{\red def}
subset of $T$ whose slice at any $\lambda$ is relatively compact in $T_{|\lambda}$.
Given any such $g_j$, we write $g_j = g_j^{(1)} \circ g_j^{(2)}$, where
$g_j^{(2)}$ is an
inverse branch of $f^{N_2}$ on $T$ with image in $D$ and $g_j^{(1)}$ is an inverse branch of $f^{N_1}$ on some $D_i$ with image in $T$. For every 
$j$, we also set $i=i(j)$  where $i$ is defined
by $g^{(2)}_j (T) \subset D_i$.
%{\red define the $\mathcal G_j$}\\

By definition, the inverse branch $g_j$ is of the form $g_j(\lam, z) = (\lam, g_j^\lam(z))$, where $\lam \mapsto g_j^\lam(z)$ is holomorphic.
For $\gamma' \in \mathcal T$, set $\mathcal G_j(\gamma')(\lam) := g_j^\lam(\gamma'(\lam))$, so that $\mathcal G_j (\mathcal T) \subseteq \mathcal T^*$.
One can define in a similar way $\mathcal G_j^{(1)} : \mathcal D_i \to \mathcal T^*$ and $\mathcal G_j^{(2)} : \mathcal T \to \mathcal D_i$ and remark that $\G_j = \G_j^{(1)} \circ \G_j^{(2)}$.


\begin{lemma}\label{l:completeness}
The space $(\Supp \M, \dist_{M'})$ is complete.
\end{lemma}
\begin{proof}
Let $(\gamma_n)_{n\in \N}$ be a Cauchy sequence.
As $\Supp \M$ is compact with respect to the local uniform convergence, there exists a subsequence $\gamma_{n_j}$ such that
$\gamma_{n_j}$ converges locally uniformly to a graph $\gamma \in \Supp \M$.
In particular, $\gamma_{n_j}$ converges to $\gamma$ on $M'$ and the Cauchy sequence $(\gamma_n)_{n\in \N}$ admits a cluster value, hence converges.
\end{proof}



\begin{lemma}\label{l:motions}
For $N=N_1+N_2$ large enough, the operator $\G_j$ admits a fixed point $\gamma_j^{\infty} \in \mathcal T \cap \Supp \M$.
Furthermore, $\gamma_j^\infty(\lam)$ is a periodic point of period $N$ of $f_\lam$, for any parameter $\lam \in M$,
and it is repelling for any $\lam \in M'$.
\end{lemma}


\begin{proof}
Since $\gamma \in \Supp \M$, by Lemma \ref{l:criterion-conditions} (6) and Lemma \ref{l:completeness},
it is sufficient to prove that $\G_j$ is a contraction on a closed subset of $\mathcal T$ containing $\gamma$.
%{\red [CHECK: now we are using distance on M'/Omega: not complete. Adapt and say there is just limit as repelling periodic graph on M'/Omega. Later we say we can extend]}
Take $\gamma_1, \gamma_2 \in \mathcal T$. 
%{\red need to say who is $\hat \gamma$, it should follow $G_j$ etc. Otherwise just put an implicit constant}
By Proposition \ref{p:bbd-gen}, we have
\begin{equation}\label{e:contraction}
\dist_{M'}(\G_j(\gamma_1), \G_j(\gamma_2)) \lesssim %\hat l_p(\hat \gamma) 
e^{-NA} \dist_{M'}(\gamma_1,\gamma_2),
\end{equation}
where the implicit constant is independent of $N$.
Hence, for all $N$ sufficiently large,
%$N > N_0:= \frac{\ln \hat l_p(\hat \gamma)}{A}$, 
$\G_j$ is 
a contraction on the closed set
$\overline{\mathcal T^*} = \{\gamma' \in \J \colon \dist_\J(\gamma,\gamma') \le \eta^*\}$.
By the Banach fixed point theorem, applied to the complete metric space $\overline{\mathcal T^*}\cap \Supp \M$, the map $\G_j$ admits a  fixed point
$\gamma_j^\infty \in \mathcal T \cap \Supp \M$.
In particular, observe that $\gamma_j^\infty(\lam)$ lies in $J_\lam$ for any $\lam \in M$.
To conclude, it remains to prove that $\gamma_j^\infty(\lam)$ is an $N$-periodic point for every $\lam \in M$ and 
is repelling for every $\lam \in M'$.
%{\red maybe not Omega, especially if distance is with J; we'll see}.

Recall that $T=T(\gamma, \eta)$ is the tube associated to the ball $\mathcal T$.
Remark that $f^N \colon g_j(T) \to T$ is onto,
hence $f^N\circ g_j = id_T$.
For all $\lambda \in M'$,
the equality $\gamma_j^\infty(\lam) = \G_j(\gamma_j^\infty)(\lam) = g_j(\gamma_j^\infty(\lam))$
then leads to $f_\lam^N(\gamma_j^\infty(\lam)) = \gamma_j^\infty(\lam)$. This shows that $\gamma_j^\infty$ is indeed the motion of an $N$-periodic point over $M'$.
In order to extend it to a motion over all $M$, it is sufficient to show that $\gamma_j^\infty(\lam)$ is repelling for any $\lam \in M'$.
Indeed, since $\gamma_j^\infty \in \Supp \M$, \cite[Lemma 2.5]{BBD18} then allows us to conclude.

% Combining \eqref{e:contraction} with the fact that the point $(\lam,z_\lam)$ lies in $T$, where $z_\lam:= f_\lam^N(\gamma_j^\infty(\lam))$,
% yields
% \[
% \lvert \jac_{z_\lam}(g_j^\lam) \rvert \le \hat l_p(\hat \gamma)e^{-NA},
% \]
% which leads to $\lvert \jac_{\gamma_j^\infty}f_\lam^N \rvert \ge \frac{e^{NA}}{\hat l_p(\hat\gamma))} > 1$ as soon as $N > N_0$.
% {\red large Jacobian not enough to be repelling, could have one very expanding direction and one attracting for example}\\
% {\green [Ok. Then simply apply \eqref{e:contraction} with $\F^N(\gamma_i)$ instead of $\gamma_i$, for $\gamma_i \in \F^{-N}(\mathcal T)$ ]}
Observe that
\eqref{e:contraction} also 
gives
%implies that 
a positive constant $c$ such that
\[
\dist_{M'}(\F^{qN}(\gamma_1),\gamma_j^\infty) 
\gtrsim 
%\left(
%\frac{
e^{qNA}
%}{\hat l_p(\hat \gamma)}\right)^q 
\dist_{M'}(\gamma_1, \gamma_j^\infty),
\]
for every positive integer $q \in \N^*$ and any graph $\gamma_1 \in \mathcal T \left( \G_j^q(\gamma_1), 
%\left(\hat l_p(\hat \gamma)
c\cdot e^{-qNA}
%\right)^q
\right)
$, where the implicit constant is independent of $q$.
%Since $N > N_0$, t
This proves that $\gamma_j^\infty(\lam)$ is repelling for any $\lam \in M'$, and concludes the proof.
\end{proof}




% \begin{lemma}\label{l:motions}
% For every $j$, the sequence $g_j^{n}$ converges uniformly to a holomorphic map $g_j^\infty\colon T\to T$, whose image is the graph of the motion $\sigma_j$
% of a repelling 
% $(N_1+N_2)$-periodic point.
% %
% {\red [see how to say better; in any case, this should say that at the limit you have only a graph, and it is repelling]}
% %
% %
% %
% %
% %the limit graphs give motions of repelling points
% \end{lemma}

% \begin{proof}
% For any fixed $\lambda \in M$, the map
% $g_j$ has a unique fixed point in
% the ball $T_{|\lambda}$. Such point is attracting for
% $g_j$, hence repelling for $f_\lam^{N_1+N_2}$.
% {\red [more details here]}
% \end{proof}





We can now conclude the proof of Proposition \ref{p:l414-reduced}. As explained above, this also completes the proof of Proposition \ref{p:l414}.


\begin{proof}[End of the proof of Proposition \ref{p:l414-reduced}]
%{\red TODO, MAIN POINT, proof using the lemmas before, should be like BD}
%
%{\red In general, I am trying to split more in steps; we can split in step also this if it makes sense}
%
%{\red maybe the first paragraph or two after the two claim in BD, where we define things, can be outside the proof}
%
%
%
We continue to use the notations introduced above. 
Up to possibly increasing $M_1$ and $M_2$ in Lemmas \ref{l:claim1} and \ref{l:claim2}, we can
assume that we can take $N_1=M_1$ and $N_2=M_2$ in the construction before Lemma \ref{l:motions}.
%{\red [check, maybe better before the lemma]}
We 
set $n(m_2):= M_1 (m_2)+M_2 (m_2)$, for a fixed choice of sufficiently large $m_2$ and $m_3$ as above. For every $n> n(n_2)$, we
denote by $\mathcal Q_n$ the set of the motions of $n$-repelling point $\sigma_j$ given by Lemma \ref{l:motions}
applied with $N_1 = M_1(m_2)$ and $N_2 = n-M_2 (m_2)$.
Every element $\sigma\in \mathcal Q_n$ is then the motion of
a repelling $n$-periodic point.

%{\red [fix the other $m_1$ etc here]}

%{\red [Define $\mathcal Q$ here as the set of repelling graphs in the lemma]}




%{\red [Define $\mathcal G_j$ as the corresponding map to $g_j$ on graphs in the tube, and similarly $\mathcal G^{(1)}_j$ and $\mathcal G^{(2)}_j$; here or before when we do the $g_j$]}


%{\red TODO, some before proof, some above}








%We will prove the statement with the set $\mathcal Q$ {\red [change sentence when defined above]}  of motions of repelling periodic points constructed above.


Set
\[
\mathcal M_n :=
\sum_{\sigma \in \mathcal Q_n}
e^{\psi (\gamma) + \ldots + \psi (\mathcal F^{n-1} (\gamma))}
\delta_\gamma
=
\sum_j e^{\psi (\sigma_j) + \ldots + \psi (\mathcal F^{n-1} (\sigma_j))}
\delta_{\sigma_j},
\]
where $\mathcal Q_n = \{\sigma_j\}$, and
%{\red [the motions given by the lemma in $\mathcal Q$]}, 
\[\begin{aligned}
\tilde{\mathcal M}_n :=
\sum_j
\Big( e^{\psi (\mathcal G_j^{(1)} (\gamma_i(j))) 
+ \ldots +
\psi (\mathcal F^{N_1-1} \circ \mathcal G_j^{(1)} (\gamma_{i(j)}))}
& \cdot \frac{\theta (\mathcal G_j^{(1)} (\gamma_{i(j)})) }{\theta (\gamma_{i(j)})}\\
& \cdot e^{\psi (\mathcal G_j^{2)} (\gamma) 
+ \ldots +
\psi (\mathcal F^{N_2-1} \circ \mathcal G_j^{(2)} (\gamma))}
\frac{\theta (\mathcal G_j^{(2)} (\gamma))}{\theta (\gamma)}
\Big)
\delta_{\mathcal G_j (\gamma)},
\end{aligned}\]
where we recall that $\mathcal T= B(\gamma, \eta)$ 
and $\mathcal D_i =B(\gamma_i, \eta_i)$. Observe in particular that
\begin{equation}\label{e:MntildeT}
\tilde {\mathcal M}_n (\mathcal T)
=
\sum_i \mathcal M_{T,N_2}^{(m_2)} (\mathcal D^\star_i)
\cdot 
\mathcal M_{D_i, N_1}^{(m_3)} (\mathcal T^\star).
\end{equation}
By Lemma \ref{l:claim1} and the fact that 
$\sum_i \mathcal M^{(m_2)}_{T, N_2} (\mathcal D_i^\star)=
\mathcal M^{(m_2)}_{T,N_2} (\mathcal D^\star)$, 
we have
\[
(1-4/m_2)
\mathcal M(\mathcal T)
\mathcal M^{(m_2)}_{T,N_2} (\mathcal D^\star)
\leq 
\tilde{\mathcal M}_n (\mathcal T)
\leq
(1+4/m_2)
\mathcal M (\mathcal T)
\mathcal M^{(m_2)}_{T,N_2} (\mathcal D^\star).
\]
By Lemma \ref{l:claim2}, this gives
\begin{equation}\label{e:estimate-tilde}
(1-10/m_2)
\mathcal M(\mathcal T)
\leq 
\tilde{\mathcal M}_n (\mathcal T)
\leq
(1+10/m_2)
\mathcal M (\mathcal T).
\end{equation}
Hence, 
up to taking $m_2$ sufficiently large,
%{\red [here we likely need to take $m_2$ large, to get the estimate with $m_1$ in the Fact]}, 
the desired estimate holds if we replace $\mathcal M_n (\mathcal T)$ with
$\tilde {\mathcal M}_n (\mathcal T)$.
It is then enough to compare
$\mathcal M_n (\mathcal T)$ with
$\tilde {\mathcal M}_n (\mathcal T)$.

%{\red [maybe, if this is more clear, the estimate for tilde can become another lemma, which is a consequence of the two lemmas/claims, and then we just conclude of the proof of the Proposition. We can see later]}

\medskip





Observe that, for all $i$ and $j$,
the graphs of $\gamma$ and $\mathcal G_j^{(1)}(\gamma_{i(j)})$
belong to $T$ and 
those of $\gamma_{i(j)}$
and $\mathcal G_j^{(2)} (\gamma)$ belong to $D_i$. Hence, 
since the tubes $T$ and $D_i$ are $m_2$-nice
and satisfy \eqref{e:small_enough}
(with $\mathcal A$ replaced by $\mathcal T$ and $\mathcal D_i$, respectively), we have
%{\red [TODO: here is where we  need to precise "small enough" etc, if we remove (1) in the def of nice. Here or at the beginning of the proof, putting a condition with number and refer to that here]}
\[
\Big|
\frac{\theta ( \mathcal G_j^{(1)} (\gamma_{i(j)}) )}{\theta (\gamma)}
-1\Big|\le \frac{2}{m_2}
\mbox{ and }
\Big|
\frac{\theta (\mathcal G_j^{(2)} (\gamma))}{\theta(\gamma_{i(j)})}
-1\Big|\le \frac{2}{m_2}
\mbox{ for all }
j.\]
%{\green [It should be $\le (m_2 -1)^{-1}$. In the paper with Dinh, you put $\lesssim m_2^{-1}$. I think it is better.]}{\red yes, ok}
It follows from these inequalities and Lemma \ref{l:l413}
that 
$\lvert\mathcal M_n (\mathcal T)- \tilde{\mathcal M}_n (\mathcal T)\rvert  \lesssim
\tilde{\mathcal M}_n (\mathcal T)/m_2$. 
This, together with \eqref{e:estimate-tilde}, concludes the proof.
%
%
%To conclude, it is then enough to show that
%\[ (1-1/(2m_1)) \mathcal M (T) \leq \tilde {\mathcal M}_n (T)  \leq (1+ 1/(2m_1)) \mathcal M (T) \]
%{\red with $m_2$ large enough, need to adapt slightly for the statement and choice m1 etc}
%
%{\red continue}
%
%By definition 
%
\end{proof}




\subsection{Proof of Theorem \ref{t:main}}\label{ss:proof-main}

%{\red some sentence; if only this proof, the environment proof not needed below}


%\begin{proof}[Proof of Theorem \ref{t:main}]



%{\red TODO; If everything is done correctly, it should follow from the Proposition \ref{p:l414},  need to be sure to what is U etc now}

We can now conclude the proof of Theorem \ref{t:main}. Recall that we are assuming that $P(\phi)=0$ and that 
we denote by $\mathcal M$ the web $\mathcal M_{\lam_0, \mu_\phi}$, whose construction is detailed in Section \ref{ss:equilibrium}.

%{\blue [sentence: this is now as in BD23/ a sentence like this; also recall that we can assume that pressure is 0; recall what else is needed]}


%{\red $\mathcal M$ is always $\mathcal M_\phi$ below}


\medskip

We first build a  family
%{\red [not sure of the name, is it standard?]}
of measurable partitions of $(\J, \M)$ whose diameter converges to $0$.
%{\red [maybe partition of the support, but ok]}.

\begin{lemma}\label{l:partition}
There exists a sequence $(\mathfrak U_i)_{i\in \N}$ of finite families of disjoint open sets $\mathfrak U_i = \{\mathcal U_{i,j}\}_{1\leq j \leq J_i}$ satisfying the following properties:
\begin{enumerate}
\item[{\bf (U1)}] for all $i \in \N$, we have $\mathcal M (\cup_{1\leq j \leq J_i} \mathcal U_{i,j})=1$;
\item[{\bf (U2)}] $\sup_{i\in\N} (i \cdot \max_{1\leq j \leq J_i}\diam_{\J} \mathcal U_{i,j})\leq 1$;
\item[{\bf (U3)}] for all $i\geq 2$ and $1\leq j \leq J_i$
there exists $1\leq j'\leq J_{i-1}$ such that
$\mathcal U_{i,j} \subset \mathcal U_{i-1,j'}$.
\end{enumerate}
\end{lemma}


\begin{proof}
We prove the lemma by induction on $i\in \mathbb N$.
We set $J_0 = 1$ and $\mathcal U_{0,1} = \J$.
The family $\mathfrak U_0 :=\{U_{0,1}\}$
satisfies conditions {\bf (U1)} and {\bf (U2)}, and condition 
{\bf (U3)} is empty in this case.
%{\red a priori need to check U2 no? maybe put a constant in U2? Or at the end say we only take $i\leq i_0$ and replace $i$ by $i-i_0$?}

\medskip

We now do the induction step. Assume that we have built, for some $i \in \N$, a finite family $\mathfrak U_i = \{\mathcal U_{i,j}\}_{1\le j \le J_i}$ satisfying \textbf{(U1)}, \textbf{(U2)}, and \textbf{(U3)}.
Consider the interval $I:= (\frac{1}{2(i+2)}, \frac{1}{2(i+1)})$.
%and take $\epsilon \in I$.
As $\Supp \M \subseteq  \cup_{\gamma \in \Supp \mathcal M} \mathcal B_{\J}(\gamma, \frac{1}{2(i+2)})$ 
%{\red [I do not understand the =, is there an intersection over i somewhere? Or inclusion? also next equality]}
and $\Supp \mathcal M$ is compact, there exists $m_i \in \N$ and a collection $\{\gamma_{i,m}\}_{1\leq m\leq m_i} \subset \Supp \M$
%, for $1\le m \le m_i$,
such that, for any $\epsilon \in I$, $\Supp \M \subseteq \cup_{m=1}^{m_i} \mathcal B_{\J}(\gamma_{i,m}, \epsilon)$.
Set
%{\red [I guess the index is m]}
$B_{i,m}^\epsilon := \mathcal B_{\J}(\gamma_{i,m},\epsilon)$ and $D_{i,m}^\epsilon := \partial \mathcal B_{i,m}^\epsilon$.

\medskip

\noindent \textbf{Claim.} The set $A_i:=\{\epsilon \in I \::\:\M(\cup_{1\le m\le m_i} D_{i,m}^\epsilon) > 0\}$ is countable.

\begin{proof}
As the family $\{D_{i,m}^\epsilon\}_{1\leq m \leq m_i}$ is finite, 
it suffices to prove that 
$A_{i,m} := \{\epsilon \in I \::\: \M(D_{i,m}^\epsilon) > 0\}$
is countable
for every 
$1\le m \le m_i$.
Assume this is not true for some $A_{i,\bar m}$.

%{\red [is there a reason for $\alpha+1$ instead of just $\alpha$?]}
%{\green [We could also take $\alpha \in \N^*$]}
For $\alpha \in \mathbb N$, set $A_{i,\bar m}^\alpha = \{\epsilon \in I \::\: \M(D_{i,\bar m}^\epsilon) > \frac{1}{\alpha +1}\}$. Observe that
$A_{i,\bar m} = \cup_{\alpha \in \N} A_{i,\bar m}^\alpha$.
Hence, there exists $\alpha \in \N$
such that $\Card(A_{i,\bar m}^\alpha) = +\infty$. As $D_{i,\bar m}^{\epsilon}\cap D_{i,\bar m}^{\epsilon'}=\emptyset$
for all $\epsilon \neq \epsilon' \in I$, we have
%the family $(D_i^\epsilon)_{\epsilon \in I}$ is pairwise disjoint
%{\red [what does this mean]}. 
%Hence
\begin{equation*}
1 \ge \M(\cup_{\epsilon \in A_{i,\bar m}^\alpha}D_{i,\bar m}^\epsilon) = \sum_{\epsilon \in A_{i,\bar m}^\alpha} \M(D_{i,\bar m}^\epsilon) \ge \frac{\Card(A_{i,\bar m}^\alpha)}{\alpha+1} = +\infty.
\end{equation*}
This gives a contradiction, and the proof of the claim is complete.\end{proof}


Fix now $\epsilon \in I$ such that
$\M(D_{i,m}^\epsilon) = 0$
for all $1\le m \le m_i$.
Denote the connected components of $\cup_{j=1}^{J_i} \mathcal U_{i,j} \cap B_{i,1}^\epsilon$ by $\mathcal U_{i+1,1}, \dots, \mathcal U_{i+1, l_1}$. By induction, 
for $1<m\le m_i$, define also $\mathcal U_{i+1, l_{m-1}+1}, \dots,\mathcal U_{i+1,l_m}$ as the connected components of
$$\bigcup_{j=1}^{J_i} \mathcal U_{i,j} \cap \left( B_{i,m}^\epsilon \setminus \bigcup_{l=1}^{l_{m-1}} \overline{\mathcal{U}_{i+1,l}} \right).$$
Set $J_{i+1} := l_{m_i}$. 
By definition, the sets $\mathcal U_{i+1,l}$ are open and pairwise disjoint.
We claim that
\[
\bigcup_{l=1}^{J_{i+1}} \mathcal U_{i+1,l} \supset \left(
\bigcup_{m=1}^{m_i} B_{i,m}^\epsilon \setminus \bigcup_{l=1}^{J_{i+1}} \partial \mathcal U_{i+1,l}\right)
\cap \bigcup_{j=1}^{J_i} \mathcal U_{i,j}.
\]






Fix $m\leq m_i$ 
and take $\gamma \in B_{i,m}^\epsilon\setminus
\cup_{1\le l\le J_{i+1}}\partial \mathcal U_{i+1,l}$,
for some $m \le m_i$. We can assume that $\gamma$ does not satisfy this property for all $m'<m$.
By the definition of the sets $\mathcal U_{i+1,l}$,
this implies that $\gamma \notin \mathcal U_{i+1,l}$, for every $l\le l_{m-1}$.
%{\red I do not understand}.
Hence $\gamma \in B_{i,m}^\epsilon \setminus \bigcup_{l=1}^{l_m-1} \overline{\mathcal U_{i+1,l}}$.
This proves the claim.

Remark that $\bigcup_{l=1}^{J_{i+1}} \partial \mathcal U_{i+1,l} \subset \bigcup_{m=1}^{m_i} D_{i,m}^\epsilon$.
As $\mathfrak U_i$ satisfies 
\textbf{(U1)},
the choice of $\epsilon$
yields
%{\red check, used here?}
\begin{equation*}
%\begin{aligned}
\M\left(\bigcup_{l=1}^{J_{i+1}} \mathcal U_{i+1,l} \right) 
\ge \M \left( \bigcup_{m=1}^{m_i} B_{i,m}^\epsilon \setminus \bigcup_{l=1}^{J_{i+1}} \partial \mathcal U_{i+1,l} \right)
\ge \M\left(\bigcup_{m=1}^{m_i} B_{i,m}^\epsilon\right) - \M\left(\bigcup_{m=1}^{m_i} D_{i,m}^\epsilon \right)
 = 1.
%\end{aligned}
\end{equation*}



This proves that $\mathfrak U_{i+1}$ satisfies \textbf{(U1)}. The property \textbf{(U2)} comes from 
the definition of the $B^\epsilon_{i,m}$'s
and the fact that, by construction, 
for every $l$ there exists $m(l)$ such that  
$\mathcal U_{i+1,l} \subset B_{i,m(l)}^\epsilon$.
Finally, the fact that $\mathcal U_{i+1,l}$ is a connected subset of $\cup_{j=1}^{J_i} \mathcal U_{i,j}$ gives \textbf{(U3)}. The proof is complete.
\end{proof}




% For every $i\in \mathbb N$, construct a finite family of disjoint open sets $\mathfrak U = \{\mathcal U_{i,j}\}_{1\leq j \leq J_i} $ satisfying the following properties:
% \begin{enumerate}
% \item[{\bf (U1)}] $\mathcal M (\cup_{1\leq j \leq J_i} \mathcal U_{i,j})=1$;
% \item[{\bf (U2)}] for all $1\leq j \leq J_i$, we have $\diam \mathcal U_{i,j}\leq 1/i$ {\red [fibered diameter, or do on $\P^k$ and then move them with lamination; check the contruction in general]};
% \item[{\bf (U3)}] for all $i\geq 2$ and $1\leq j \leq J_i$
% there exists $1\leq j'\leq J_{i-1}$ such that
% $\mathcal U_{i,j} \subset \mathcal U_{i-1,j'}$.
% \end{enumerate}

% {\red [say how we can do them; likely we do open sets on $\P^k$ at $\lam_0$, and then consider the subsets of the lamination whose slide at $\lam_0$ are these sets; check. In case, we can first build on $\P^k$ for $\mu_\phi$ and then move them]}
% {\blue [I leave the details of the construction, as we discussed]}

\medskip


For every $n\in \mathbb N$, define 
$i_n := \max \{ m\leq n \colon n \geq n(m, \mathfrak U_m)\}$, where $n(m, \mathfrak U_m)$ is given by Proposition \ref{p:l414}.
For every  $\alpha >0$, set $m_\alpha := \lfloor \alpha+1 \rfloor$. Then, for any $n \ge \max(m,n(m,\mathcal U_m))$, we have $i_n \ge m_\alpha > \alpha$.
In particular,
we have $i_n\to \infty$ as $n\to \infty$.
%{\red [explain better maybe]}. 
 
For every $n\in \mathbb N$, we apply Proposition \ref{p:l414} with ${\mathfrak U}_{i_n}$
instead of $\mathfrak U$.
This gives a collection $\mathcal P_{\phi,n}\subset \cup_{1\leq j \leq J_{i_n}} \mathcal U_{i_n, j}$ of motions of repelling $n$-periodic points, satisfying the properties in that statement.
We define
\begin{equation}\label{eq:M'n}
\mathcal M'_n :=
\sum_{\sigma \in \mathcal P_{\phi,n}}
e^{\psi (\sigma) + \ldots + \psi (\mathcal F^{n-1} \sigma)}\delta_\sigma.
\end{equation}

\begin{lemma}\label{l:limit-1}
Any limit $\mathcal M'$ of the sequence $\{\mathcal M'_n\}_{n \in \mathbb N}$
has mass 1.
\end{lemma}

\begin{proof}
%This follows from Proposition \ref{p:l414} and the property {\bf (U1)} of the family $\mathfrak U$.
%{\red [give more details if needed; check that 
%{\bf (U2)} not needed]}~\\
 By definition, for every $n\in \mathbb N$ we have
\[
\lVert \mathcal M'_n \rVert =
\sum_{\sigma \in \mathcal P_{\phi,n}}
e^{\psi (\sigma) + \ldots +\psi (\mathcal F^{n-1} \sigma)}
= \sum_{j=1}^{J_{i_n}} \sum_{\sigma \in \mathcal P_{\phi,n}\cap \mathcal U_{i_n,j}}
e^{\psi (\sigma) + \ldots +\psi (\mathcal F^{n-1} \sigma)}.
\]
Since the family $\mathfrak U_n$ is pairwise disjoint, Proposition \ref{p:l414}  yields
\[
\left(1-\frac{1}{i_n}\right) \M_\phi\left(\bigcup_{j=1}^{J_{i_n}}\mathcal U_{i_n,j}\right) \le \lVert \M_n' \rVert \le \left(1+\frac{1}{i_n}\right) \M_\phi\left(\bigcup_{j=1}^{J_{i_n}} \mathcal U_{i_n,j}\right).
\]
We conclude by Property \textbf{(U1)}
%{\red [check, I think it's U1, right?]}
and letting $n\to \infty$.
\end{proof}

%By Lemma \ref{l:limit-1} and the properties 1 and 2 of the open sets $\mathcal U_{i,j}$, in order to prove Theorem \ref{t:main} it is enough to prove that 

\begin{lemma}\label{l:liminf}
For all $i^\star \in \mathbb N$ and $1\leq j^\star \leq J_{i^\star}$ we have
\[
\liminf_{n\to \infty}
\mathcal M'_n (\mathcal U_{i^\star, j^\star})
\geq \mathcal M (\mathcal U_{i^\star, j^\star}).
%\mbox{ for all }
%i^\star \in \mathbb N \mbox{ and }
%1 \leq j^\star \leq J_{i^\star}.
\]
\end{lemma}


\begin{proof}
Fix $i^\star, j^\star$ as in the statement and 
$\epsilon >0$. It is enough to prove that, for all $n$ sufficiently large, we have
\[
\mathcal M'_n (\mathcal U_{i^\star, j^\star})
\geq \mathcal M  (\mathcal U_{i^\star, j^\star})
-\epsilon.
\]
It is enough to consider only those $n$ for which $i_n > i^\star$. For all such $n$, by 
{\bf (U1)} 
and pairwise disjointness,
%{\red [first property of the sets; ref]}
we have
\[
\mathcal M  (\mathcal U_{i^\star, j^\star})
=
\sum_{j'} \mathcal M (\mathcal U_{i_n, j'})
\quad \mbox{ and}
\quad
\mathcal M'_n  (\mathcal U_{i^\star, j^\star})
\ge 
\sum_{j'} \mathcal M'_n (\mathcal U_{i_n, j'}),
\]
where the two sums are
over the $j'$ such that $U_{i_n,j'}\subset U_{i^\star, j^\star}$.
Using the same convention for the sums,
by the definition of $\mathcal M'_n$ and Proposition \ref{p:l414}, we have 
\[
\mathcal M  (\mathcal U_{i^\star, j^\star}) -
\mathcal M'_n  (\mathcal U_{i^\star, j^\star})
\leq
\sum_{j'}
\mathcal M  (\mathcal U_{i_n, j'}) -
\mathcal M'_n  (\mathcal U_{i_n, j'})
\leq
i_n^{-1}
\sum_{j'}
\M(\mathcal U_{i_n, j'})
=
i_n^{-1} \M(\mathcal U_{i^*,j^*}).
\]
The assertion follows.
\end{proof}

\begin{proof}[End of the proof of Theorem \ref{t:main}]
We show that the sequence $\{\M'_n\}_{n\in \N}$ as in \eqref{eq:M'n}
converges to $\M$.
Let $\mathcal A \subset \J$ be a closed set. Consider $\mathcal A' := \bigcap_{n\in \N} \bigcup_{\mathcal U \in \mathfrak U_{i_n}^{\mathcal A}}\mathcal U$, where $\mathfrak U_i^{\mathcal A} := \{\mathcal U\in \mathfrak U_i \::\:
\mathcal U \cap \mathcal A \neq \emptyset \}$. By \textbf{(U1)}, for every $n\in \mathbb N$ we have
\[
\M\left(
\bigcup_{\mathcal U 
%\cap \mathcal A \neq \emptyset
\in \mathfrak U_{i_n}^{\mathcal A}} \mathcal U\right) = 1 - \M\left(\bigcup_{\mathcal U 
%\cap \mathcal A= \emptyset
\in \mathfrak U_{i_n} \setminus \mathfrak U_{i_n}^{\mathcal A}
}\mathcal U\right) \ge 1 - \M(\mathcal A^c)=
\mathcal M (\mathcal A).
\]
%where the unions above are taken over the open sets $\mathcal U \in \mathfrak U_i$.
Remark that property \textbf{(U3)} ensures that $\mathcal A'$ is a countable non-increasing intersection.
%{\red [not sure what this means]}.
Letting $n\to \infty$, we deduce that $\M(\mathcal A') \ge \M(\mathcal A)$.

\medskip

\noindent
\textbf{Claim.} We have $\mathcal A'\subseteq \mathcal A$. In particular, we have $\mathcal M(\mathcal A') = \mathcal M( \mathcal A)$.

\begin{proof}
Since $\mathcal M(\mathcal A')\leq\mathcal M(A)$, it is enough to show that $\mathcal A'\subseteq \mathcal A$.
Take $\gamma \in \mathcal A'$.
%Let $\gamma \in \mathcal J$ such that 
For any $n \in \mathbb N$, there exist some open set $\mathcal U \in \mathfrak U_{i_n}$ and $\gamma_n \in \mathcal J$ with
$\gamma \in \mathcal U$ and $\gamma_n \in \mathcal U \cap \mathcal A$.
%which contains $\gamma$, and some $\gamma_i \in \mathcal U_i \cap \mathcal A$. 
By property \textbf{(U2)}, we have $\dist_{\J}(\gamma,\gamma_n) < 1/{i_n}$. We deduce that $\gamma = \displaystyle \lim_{n\to \infty}\gamma_n \in \overline{\mathcal A} = \mathcal A$.
\end{proof}



Let now $\M'$ be any limit of the sequence $\{\M_n'\}_{n\in \N}$. Lemma \ref{l:liminf} and the Claim above give
%{\red I do not understand, the limit should be in n no?}
that
%{\red check}
\[
\M(\mathcal A) = \M(\mathcal A') = \lim_{n\to \infty} \sum_{\mathcal U \in \mathfrak U_{i_n}^{\mathcal A}}\M(\mathcal U)
\le \lim_{n\to \infty} \sum_{\mathcal U \in \mathfrak U_{i_n}^{\mathcal A}}\M'(\mathcal U)
= \M'(\mathcal A') \le \M'(\mathcal A).
\]
As the closed set $\mathcal A$ was chosen arbitrarily,
%for every closed set $\mathcal A \subset \mathcal J$.
Lemma \ref{l:limit-1} and the fact that $\|\mathcal M\|=1$
imply that
%The fact that $\lVert \M \rVert = \lVert \M' \rVert$ imply 
that $\M = \M'$.
This shows that $\mathcal M'_n\to \mathcal M$ as $n\to \infty$, and completes the proof.
\end{proof}






\bibliographystyle{alpha}
\begin{thebibliography}{9}



\bibitem{B18} François Berteloot,
Les cycles r\'epulsifs bifurquent en cha\^{i}ne,
{\it Annali di Matematica Pura ed Applicata}
{\bf 197} (2018), no. 2, 517-520.

\bibitem{BB18} François Berteloot and Fabrizio Bianchi, Stability and bifurcations in projective holomorphic dynamics, in
{\it Dynamical systems}, Banach Center Publications {\bf 115} (2018), Polish Academy of Sciences, Institute of Mathematics, Warsaw, 37–71.


\bibitem{BBD18} François Berteloot, Fabrizio Bianchi, and Christophe Dupont,
Dynamical stability and Lyapunov exponents for holomorphic endomorphisms of $\mathbb{P}^k$,
{\it Annales Scientifiques de l'{\'E}cole Normale Sup\'erieure (4)}
{\bf 51} (2018), no. 1, 215-262.


\bibitem{BB22} 
François Berteloot and Maxence Br\'evard, Ramification current, post-critical normality and stability of
holomorphic endomorphisms of $\mathbb P^k$, {\it Fundamenta Mathematicae}, to appear (2022),
{\tt arXiv:2211.13636}.


\bibitem{BD19}
François
Berteloot 
and Christophe Dupont,
A Distortion Theorem for iterated inverse branches of holomorphic endomorphisms of $\mathbb P^k$,
{\it Journal of the London Mathematical Society} {\bf 99} (2019), no. 1
153-172.


\bibitem{BDM}
François Berteloot, Christophe Dupont, and Laura Molino,
Normalization of bundle holomorphic contractions and applications to dynamics,
{\it Annales de l'Institut Fourier} {\bf 58} (2008), no. 6, 2137-2168.

\bibitem{B16} Fabrizio Bianchi, \textit{Motions of Julia sets and dynamical stability in several complex variables,} PhD
Thesis, Université Toulouse III Paul Sabatier and Università di Pisa (2016).

\bibitem{B19} Fabrizio Bianchi, 
Misiurewicz parameters and dynamical stability of polynomial-like maps of large
topological degree,
{\it Mathematische Annalen}
{\bf 373} (2019), no. 3-4 901–928.



\bibitem{BD23} Fabrizio Bianchi and Tien-Cuong Dinh, Equilibrium states of endomorphisms of $\mathbb P^k$I: existence and properties,
{\it Journal de mathématiques pures et appliquées}
{\bf 172}
 (2023), 164-201.

\bibitem{BD22} Fabrizio Bianchi and Tien-Cuong Dinh, Equilibrium states of endomorphisms of $\mathbb P^k$ II: spectral gap and limit theorems, 
{\it preprint} (2022), {\tt  arXiv:2204.02856}.



\bibitem{BR22} Fabrizio Bianchi and Karim Rakhimov,
Strong probabilistic stability
 in holomorphic families
of endomorphisms of $\mathbb P^k  (\mathbb C)$ and polynomial-like maps,
{\it preprint} 
(2022), {\tt arxiv:2206.04165}.




\bibitem{B23} Maxence Br{\'e}vard, \emph{\'Etude des composantes de stabilité pour les familles d'endomorphismes holomorphes des espaces projectifs complexes}, PhD thesis, Université Toulouse III Paul Sabatier (2023).


\bibitem{BR} Maxence Br{\'e}vard and Karim Rakhimov, Propagation of equilibrium states in stable families of endomorphisms of $\mathbb P^k$, {\it preprint} (2023).



\bibitem{BD99} Jean-Yves Briend and Julien Duval, Exposants de Liapounoff et distribution des points périodiques
d'un endomorphisme de $\C\P^k$, 
{\it Acta Mathematica} {\bf 182} (1999), no. 2, 143–157.


\bibitem{BD01} Jean-Yves Briend and Julien Duval,
Deux caract\'{e}risations de la mesure d'\'{e}quilibre d'un endomorphisme de {${\rm P}^k(\bold C)$},
{\it Publ. Math. Inst. Hautes \'{E}tudes Sci.},
{\bf 93}, (2001), 145-159.
%, and erratum in {\bf 109} (2009), 295-296.




 \bibitem{B05} 
Xavier Buff,
La mesure d'\'equilibre d'un endomorphisme de $\mathbb P^k(\mathbb C)$, {\it S\'eminaire Bourbaki}
{\bf 47} (2004-2005),  33-70.


\bibitem{CFS}
Isaac P. Cornfeld,
 Sergej V. Fomin, and  Yakov G. Sinai,
 {\it Ergodic theory}, Vol. 245,  Springer Science \& Business Media (2012)


 
\bibitem{DS10} Tien-Cuong Dinh and Nessim Sibony, \emph{Dynamics in several complex variables: endomorphisms of projective spaces and polynomial-like mappings},
in  \emph{Holomorphic dynamical systems}, Eds. G. Gentili, J. Guenot, G. Patrizio, Lect. Notes in Math. 1998 (2010), Springer, Berlin, 165–294.


\bibitem{DS10b}
Tien-Cuong Dinh and Nessim Sibony.
Equidistribution speed for endomorphisms of projective
spaces. 
{\it Mathematische Annalen} {\bf 347}
(2010), no. 3, 613-626.

\bibitem{dM03} Laura DeMarco, 
Dynamics of rational maps: Lyapunov exponents, bifurcations, and capacity, 
{\it Mathematische Annalen}
{\bf 326} (2003), no. 1, 43–73.

\bibitem{D12} Christophe Dupont,
Large entropy measures for endomorphisms of $\mathbb{CP}^k$,
{\it Israel Journal of Mathematics} 
{\bf 192} (2012), 505–533.

\bibitem{FS94} John Erik Fornaess and Nessim Sibony,
Complex dynamics in higher dimensions,
{\it Complex potential theory (Montreal, PQ, 1993)}, NATO Adv. Sci. Inst. Ser. C Math. Phys.
Sci.
{\bf 439}, Kluwer Acad. Publ., Dordrecht, (1994), 131–186.

\bibitem{FLM} Alexandre Freire, Artur Lopes, Ricardo Mañ\'e,
An invariant mesure for rational maps,
\emph{Bulletin of the Brazilian Mathematical Society}
{\bf 14} (1983), no. 1, 45-62.

\bibitem{Kal} Olav Kallenberg,
\emph{Foundations of Modern Probability},
Springer Cham (2022).

\bibitem{Lev82}
Genadi M. Levin, 
On the irregular values of the parameter of a family of polynomial mappings, 
{\it Uspekhi Matematicheskikh Nauk}
{\bf 37} (1982), no. 3, 189-190.



\bibitem{Ly83a} Mikhail Lyubich, 
Some typical properties of the dynamics of rational mappings, 
{\it Russian Mathematical Surveys}
{\bf 38} (1983), no. 5, 154–155.

\bibitem{Ly83b} Mikhail Lyubich,
Entropy properties of rational endomorphisms of the Riemann sphere,
\emph{Ergodic Theory Dynam. Systems}
{\bf 3}, (1983), 351-385.

\bibitem{MSS83} Ricardo Mañé, Paulo Sad, and Dennis Sullivan, 
On the dynamics of rational maps, 
{\it Annales Scientifiques de l'{\'E}cole Normale Superi{\'e}ure (4)}
{\bf 16} (1983), no. 2, 193–217.


\bibitem{Prz85} Feliks Przytycki, 
Hausdorff dimension of harmonic measure on the boundary of an
attractive basin for a holomorphic map,
{\it Inventiones mathematicae}
{\bf 80} (1985), no. 1,
161–179.



\bibitem{PU}
Feliks Przytycki and Mariusz Urbański
{\it Conformal fractals: ergodic theory methods} (Vol. 371). Cambridge University Press (2010).

\bibitem{Si81} Nessim Sibony, Seminar talk at Orsay, October (1981).



\bibitem{SUZ} 
Micha{\l} Szostakiewicz, Mariusz Urbański, and Anna Zdunik,
Stochastics and thermodynamics for equilibrium measures of holomorphic endomorphisms on complex projective spaces,
{\it Monatshefte für Mathematik} {\bf 174} (2014), no. 1, 141-162.


\bibitem{UZ13} Mariusz Urbański and Anna Zdunik, Equilibrium measures for holomorphic endomorphisms of complex projective spaces,
{\it Fundamenta Mathematicae}
{\bf 220} (2013) 23-69.


%{\red other refs eq state dim 1}



%{\red maybe Berger Dujardin Lyubich for Henon...}



%{\red check order biblio}


\end{thebibliography}

\end{document}

%%%remove later:

\section{Removed}




is a continuous operator, and send positive functions to positive functions.
We denote by $\Lambda_\psi^*$ the dual operator, acting on measures on $\mathcal X$. The following lemma is standard,  we give the proof for completeness.
%{\red I hope}.
The explicit form of the operator is only needed for the invariance in the third item.

{\red [in the same way I use psi instead of phi here, we may want a different letter for $\rho$, since $\rho$ is used later on $\P^k$]}

\begin{lemma}\label{l:general-Lambda} Let $\psi,\Lambda_\psi$ be as above.
Assume that there exists a unique probability measure $\mathcal N$ on $\mathcal X$ which is a fixed point for $\Lambda^*_\psi$, i.e., such that $\Lambda^*_\psi \mathcal N = \mathcal N$. Then
\begin{enumerate}
\item $\Lambda_\psi$ extends to a continuous 
{\red [check] }
operator from $L^1 (\mathcal N)$ to $L^1 (\mathcal N)$;
\item there exist a unique fixed points $\rho\in L^1 (\mathcal N)$ for the operator $\Lambda_\psi$;
\item the measure $\mathcal M := \rho \mathcal N$ is $\mathcal F$-invariant probability measure;
\item for every 
$\gamma \in \mathcal X$ and every continuous function $g\colon \mathcal J \to \R$, we have
\[
\langle 
(\Lambda_\psi^n)^* \delta_\gamma, g\rho
\rangle \to \rho(\gamma) \langle \mathcal M, g\rangle.
\]
\end{enumerate}
\end{lemma}

 \begin{proof}
 {\red [check all this and possibly write better]}

 
Take $g\in L^1(\mathcal N)$. We have
\[
\langle \mathcal N, |\Lambda_\psi (g)|\rangle
\leq
\langle
\mathcal N, \Lambda_\psi (|g|)
\rangle
=
\langle \Lambda^*_\psi \mathcal N, |g|\rangle = \langle 
\mathcal N, |g|\rangle.
\]
Hence, we have $\|\Lambda_\psi (g)\|_{L^1(\mathcal N)}\leq \|g\|_{L^1 (\mathcal N)}$ This proves the first assertion.


\medskip

Since $\mathcal N$
is the unique fixed point for $\Lambda$,
 for every $\gamma \in \mathcal X$ we have
\begin{equation}\label{e:conv-rho-gamma}
(\Lambda_\psi^n)^* \delta_\gamma \to \rho_\gamma \mathcal N,
\end{equation}
where $\rho_\gamma$ 
is a non-negative constant depending on $\gamma$. We define the function $\rho(\gamma):=\rho_\gamma$. 
%Observe that $\rho(\gamma)\geq 0$ for every $\gamma \in \mathcal X$.

\medskip




\medskip




Take any $g \in \mathcal C^0 (\mathcal X)$. For any $\gamma \in \mathcal X$,
we have
\begin{equation}\label{e:conv-g}
\Lambda_\psi^n (g) (\gamma)
=
\langle\delta_\gamma, \Lambda_\psi^n (g) \rangle
=
\langle
(\Lambda_\psi^n)^* \delta_\gamma, g
\rangle\to
\rho(\gamma) \langle g, \mathcal N\rangle.
\end{equation}

\medskip

We deduce from \eqref{e:conv-g} 
and the fact that $(\Lambda_\psi)^* \mathcal N= \mathcal N$ that
\[
\langle \mathcal N, g\rangle =
\lim_{n \to \infty}
\langle (\Lambda^n_\psi)^* \mathcal N, g\rangle
=
\lim_{n\to \infty}
\langle
\mathcal N, \Lambda^n_\psi (g)\rangle
=
\langle
\mathcal N, 
\rho
\rangle
\langle \mathcal N, g\rangle
\]
for every $g \in \mathcal C^0(\mathcal X)$.
This shows that $\langle \mathcal N, \rho\rangle=1$. Hence,
as $\rho$ is non-negative,
it belongs to $L^1 (\mathcal N)$.
Moreover,
$\mathcal M = \rho \mathcal N$ is a probability measure.

\medskip



Since $\Lambda_\psi$ extends to $L^1 (\mathcal N)$, the last item follows from \eqref{e:conv-g},
up to replacing $g$ with $g\rho$.
%{\red need rho in L1 }

\medskip

%{\red maybe this paragraph not needed}
%Define $\mathbb 1 $ to be the function identically equal to $1$ on $\mathcal X$, i.e., such that $\mathbb 1 (\gamma)=1$ for all $\gamma \in \mathcal X$. For every $\gamma \in \mathcal X$, we have
%\[ \Lambda_\psi^n (\mathbb 1)  (\gamma) = \langle \delta_\gamma, \Lambda^n_\psi ( \mathbb 1) \rangle = \langle (\Lambda_\psi^n)^* \delta_\gamma, \mathbb 1\rangle \to \rho(\gamma) \langle\mathcal N, \mathbb 1\rangle = \rho (\gamma), \]
%where the convergence is given by \eqref{e:conv-rho-gamma}, and in the last step we used the fact that $\mathcal N$ is a probability measure. This shows that $\Lambda_\psi^n (\mathbb 1) \to \rho$. By continuity
%{\red check/explain, need rho in L1 already maybe?}, this gives $\Lambda \rho = \rho$.



Finally, for every $g \in \mathcal C^0 (\mathcal X)$
we have
\[
\langle
\mathcal M, g \circ \mathcal F
\rangle
=
\langle
\mathcal N, \rho (g \circ \mathcal F)
\rangle
=
\langle
\mathcal N, \Lambda_\psi ( \rho (g \circ f))\rangle
=
\langle
\mathcal N, \rho g
\rangle
=
\langle
\mathcal M, g\rangle,
\]
where in the second equality we used that $\mathcal N= \Lambda^*_\psi \mathcal N$ and in the third 
we used the explicit expression of $\Lambda_\psi$.
%{\red attention, maybe needed to already define $\Lambda$ as sums with $\psi$ to get the last equality. this also helps to say positive etc if it is explicit. maybe we do like that}
%
\end{proof}




\end{document}
