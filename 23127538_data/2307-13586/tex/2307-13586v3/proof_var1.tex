





Before proceeding, we recall that 
%
\begin{align}
&T_4 = \sum_{k=1}^K \left( \sum_{h=1}^H \widehat{r}_h^k(s_h^k,a_h^k) - V_1^{\pi^k}(s_1^k) \right),\nonumber
\\&T_5 = \sum_{k=1}^K \sum_{h=1}^H \mathbb{V}\big(\widehat{P}_{s_h^k,a_h^k,h},V_{h+1}^k\big),\nonumber
\\ &T_6 = \sum_{k=1}^K \sum_{h=1}^H \mathbb{V}\big(P_{s_h^k,a_h^k,h},V_{h+1}^k\big),\nonumber
 \end{align}
 %
and that 
%
$$
	B =4000(\log_2K)^3\log(3SAH)\log\frac{1}{\delta'}
	\qquad\text{and}\qquad
	\delta' = \frac{\delta}{200SAH^2K^2}.
$$


\subsubsection{Bounding $T_2$}
%
Recall that when proving \eqref{eq:boundt2}, we have demonstrated that (see \eqref{eq:boundt2o-temp}) 
%
\begin{align}
	T_2 &\leq \frac{460}{9} \sqrt{2SAH (\log_2K) \Big( \log \frac{1}{\delta'} \Big) T_5 } \nonumber
\\ & \qquad  +4\sqrt{SAH(\log_2K)\log \frac{1}{\delta'}}\sqrt{\sum_{k,h}\left(\widehat{\sigma}_h^k(s_h^k,a_h^k)- \big(\widehat{r}_h^k(s_h^k,a_h^k)\big)^2\right)} 
	+ \frac{1088}{9} SAH^2(\log_2K)\log \frac{1}{\delta'} .\label{eq:local3}
 \end{align}
%
This motivates us to bound the sum $\sum_{k,h}\big(\widehat{\sigma}_h^k(s_h^k,a_h^k)- \big(\widehat{r}_h^k(s_h^k,a_h^k)\big)^2\big)$, 
which we accomplish via the following lemma.
%
\begin{lemma}\label{lemma:bdrv}
With probability at least $1-4SAHK\delta'$, one has
%
\begin{align}
\sum_{k,h}\left(\widehat{\sigma}_h^k(s_h^k,a_h^k)- \big(\widehat{r}_h^k(s_h^k,a_h^k) \big)^2\right)\leq  6K\mathrm{var}_1 + 242SAH^3(\log_2K)\log \frac{1}{\delta'}.
\end{align}
%
\end{lemma}
%
Combining Lemma~\ref{lemma:bdrv} with \eqref{eq:local3}, we can readily derive
%
\begin{align}
	T_2&\leq \frac{460}{9} \sqrt{2SAH (\log_2K) \Big( \log \frac{1}{\delta'} \Big) T_5 }  +12\sqrt{SAH(\log_2K)\log \frac{1}{\delta'}}\sqrt{2K\mathrm{var}_1} \notag\\
	& \qquad\qquad + 157SAH^2(\log_2K)\log \frac{1}{\delta'} \label{eq:nbt2}
\end{align}
%
with probability at least $1-4SAHK\delta'$. 
%



\begin{proof}[Proof of Lemma~\ref{lemma:bdrv}]
%
For notational convenience, let us define the variance of $R_h(s,a)$ as $v_h(s,a)$. 


Firstly, we control each $\widehat{\sigma}_h^k(s_h^k,a_h^k)- (\widehat{r}_h^k(s_h^k,a_h^k))^2$ with $v_h(s,a)$. Fix $(s,a,h,k)$. 
Applying Lemma~\ref{lemma:con} shows that, with probability at least $1-2\delta'$,
%
\begin{align}
N_h^k(s,a)\left(\widehat{\sigma}_h^k(s_h^k,a_h^k)- \big(\widehat{r}_h^k(s_h^k,a_h^k)\big)^2\right)\leq 3N_h^k v_h(s,a)+ H^2\log \frac{1}{\delta'}.
\end{align}
%
This allows us to deduce that, with probability exceeding $1-2SAHK\delta'$,
%
\begin{align}
\sum_{k,h}\left(\widehat{\sigma}_h^k(s_h^k,a_h^k)- \big(\widehat{r}_h^k(s_h^k,a_h^k)\big)^2\right)
 &\leq 3\sum_{k,h}v_h(s_h^k,a_h^k) + \sum_{k,h}\frac{H^2\log \frac{1}{\delta'} }{N_h^k(s_h^k,a_h^k)} \notag\\
	& \leq 3\sum_{k,h}v_h(s_h^k,a_h^k) 
	+2SAH^3(\log_2K)\log \frac{1}{\delta'}.\label{eq:bdvr1}
\end{align}


It then suffices to bound the sum $\sum_{k,h}v_h(s_h^k,a_h^k)$. 
Towards this end, let 
%
$$
	\widetilde{V}_h^k(s) \coloneqq \mathbb{E}_{\pi^k}\left[\sum_{h'=h}^H v_{h'}(s_{h'},a_{h'}) \,\Big|\, s_h = s \right]
$$ 
%
be the value function with rewards taken to be $\{v_h(s,a)\}$ and the policy selected as $\pi^k$. 
It is clearly seen that $$\widetilde{V}^k_h(s,a)\leq H^2.$$
%
In view of Lemma~\ref{lemma:self-norm}, we can obtain
%
\begin{align}
\sum_{k=1}^K\sum_{h=1}^Hv_h(s_h^k,a_h^k) - \sum_{k=1}^K \widetilde{V}_1^k(s_1^k) 
&= \sum_{k=1}^K \left( \sum_{h=1}^H \big\langle e_{s_{h+1}^k}-P_{s_h^k,a_h^k,h},\,\widetilde{V}^k_{h+1} \big\rangle \right) \nonumber
\\ & \leq 2\sqrt{2\sum_{k=1}^K \sum_{h=1}^H \mathbb{V}\big(P_{s_h^k,a_h^k,h},\widetilde{V}_{h+1}^k \big)\log\frac{1}{\delta'}} + 3H^2 \log\frac{1}{\delta'} \label{eq:local0}
\end{align}
%
with probability at least $1-2SAHK\delta'$. 
Moreover, invoking Lemma~\ref{lemma:self-norm} once again reveals that 
%
\begin{align}
&\sum_{k=1}^K \sum_{h=1}^H \mathbb{V}\big(P_{s_h^k,a_h^k,h},\widetilde{V}_{h+1}^k\big)\nonumber
 \\ & = \sum_{k=1}^K \sum_{h=1}^H \big\langle P_{s_h^k,a_h^k,h} - e_{s_{h+1}^k}  ,\, (\widetilde{V}_{h+1}^k)^2 \big\rangle  \nonumber
 \\ & \qquad  + \sum_{k=1}^H \sum_{h=1}^H \left(\big(\widetilde{V}_{h+1}^k(s_{h+1}^k)\big)^2- \big(\widetilde{V}_h^k(s_h^k)\big)^2 \right)
	+\sum_{k=1}^K \sum_{h=1}^H \left( \big(\widetilde{V}_h^k(s_h^k)\big)^2 - \big(\big\langle P_{s_h^k,a_h^k,h}, \widetilde{V}_{h+1}^k \big\rangle\big)^2\right) \nonumber
 \\ & \leq 2\sqrt{8H^4\sum_{k=1}^K \sum_{h=1}^H \mathbb{V}\big(P_{s_h^k,a_h^k,h},\widetilde{V}_{h+1}^k \big)\log \frac{1}{\delta'}   } +2H^2\sum_{k=1}^K\sum_{h=1}^H v_h(s_h^k,a_h^k) 
	+ 3H^4\log \frac{1}{\delta'} \nonumber
 \\ & \leq 4H^2\sum_{k=1}^K\sum_{h=1}^H v_h(s_h^k,a_h^k)+42H^4\log \frac{1}{\delta'} \label{eq:local1}
\end{align}
%
with probability at least $1-2SAHK\delta'$. 
Combine \eqref{eq:local0} and \eqref{eq:local1} to yield 
%
\begin{align}
\sum_{k=1}^K\sum_{h=1}^K v_h(s_h^k,a_h^k) & \leq \sum_{k=1}^K \widetilde{V}_1^k(s_1^k) + 2\sqrt{8H^2\sum_{k=1}^K\sum_{h=1}^Hv_h(s_h^k,a_h^k) \log\frac{1}{\delta'}+ 84H^4\log^2 \frac{1}{\delta'}}+3H^2\log \frac{1}{\delta'} \nonumber
\\ & \leq   2\sum_{k=1}^K \widetilde{V}_1^k(s_1^k)  +80H^2\log \frac{1}{\delta'} \nonumber
\\ & \leq 2K\mathrm{var}_1 +80H^2\log \frac{1}{\delta'} \label{eq:bdddv}
\end{align}
% 
with probability exceeding $1-4SAHK\delta'$. 
\end{proof}


\subsubsection{Bounding $T_4$}

We now move on to the term $T_4$, 
which can be written as $T_4 = \widecheck{T}_1+\widecheck{T}_2$ with
%
\begin{align*}
	\widecheck{T}_1 = \sum_{k=1}^K \sum_{h=1}^H  \left( \widehat{r}_h^k(s_h^k,a_h^k)-r_h(s_h^k,a_h^k) \right) \\
	\widecheck{T}_2 = \sum_{k=1}^K \left( \sum_{h=1}^H r_h(s_h^k,a_h^k) - V_1^{\pi^k}(s_1^k) \right).
\end{align*}
%
This leaves us with two quantities to control. 


To begin with, let us look at $\widecheck{T}_1$. In view of Lemma~\ref{bennet} and the union bound over $(s,a,h,k)$, 
we see that, with probability at least $1-2SAHK\delta'$,
%
\begin{align}
\widehat{r}_h^k(s,a)-r_h(s,a)\leq \sqrt{\frac{2v_h(s,a)\log \frac{1}{\delta'}}{N_h^k(s,a)}}+ \frac{H\log \frac{1}{\delta'}}{N_h^k(s,a)}.
\end{align}
%
As a result, we obtain
%
\begin{align}
|\widecheck{T}_1| &  \leq \sum_{k=1}^K \sum_{h=1}^H\left(  \sqrt{\frac{2v_h(s_h^k,a_h^k)\log \frac{1}{\delta'}}{N_h^k(s_h^k,a_h^k)}}+ \frac{H\log \frac{1}{\delta'}}{N_h^k(s_h^k,a_h^k)}  \right)\nonumber
	\\ & \leq \sqrt{4SAH (\log_2 K)\log \frac{1}{\delta'}}\cdot \sqrt{\sum_{k=1}^K\sum_{h=1}^H v_h(s_h^k,a_h^k)} + 2SAH^2 (\log_2 K)\log \frac{1}{\delta'}.\label{eq:wc3}
\end{align}
%
In view of \eqref{eq:bdddv}, with probability exceeding $1-4SAHK\delta'$ we have
%
\begin{align}
\sum_{k=1}^K\sum_{h=1}^H v_h(s_h^k,a_h^k)\leq 2K\mathrm{var}_1 + 80H^2\log \frac{1}{\delta'}.\label{eq:wc1}
\end{align}
%
Consequently, we arrive at
%
\begin{align}
	|\widecheck{T}_1| \leq \sqrt{8SAHK\mathrm{var}_1 (\log_2K)\log \frac{1}{\delta'} }+ 20SAH^2 (\log_2 K)\log \frac{1}{\delta'}.\label{eq:cht1}
\end{align}


Next, we proceed to bound $\widecheck{T}_2$. Towards this, we make the observation that
%
\begin{align}
\widecheck{T}_2 = \sum_{k=1}^K \sum_{h=1}^H \big\langle e_{s_{h+1}^k}-P_{s_h^k,a_h^k,h} ,\, V_{h+1}^{\pi^k} \big\rangle.
\end{align}
%
Applying Lemma~\ref{lemma:self-norm} shows that, with probability at least $1-2SAHK\delta'$, 
%
\begin{align}
|\widecheck{T}_2|  & \leq 2\sqrt{2\sum_{k=1}^K\sum_{h=1}^H \mathbb{V}(P_{s_h^k,a_h^k,h},V_{h+1}^{\pi^k})\log \frac{1}{\delta'} }+ 3H\log \frac{1}{\delta'}\nonumber
\\ & \leq 2\sqrt{4\sum_{k=1}^K \sum_{h=1}^H \big(\mathbb{V}(P_{s_h^k,a_h^k,h},V_{h+1}^{\star}) + \mathbb{V}(P_{s_h^k,a_h^k,h}, V^{\star}_{h+1}-V_{h+1}^{\pi^k})  \big)\log \frac{1}{\delta'}}+3H\log \frac{1}{\delta'}.\label{eq:cls0}
\end{align}
%
Continue the calculation to derive
%
\begin{align}
 & \sum_{k=1}^{K}\sum_{h=1}^{H}\mathbb{V}(P_{s_{h}^{k},a_{h}^{k},h},V_{h+1}^{\star}-V_{h+1}^{\pi^{k}})\nonumber\\
 & =\sum_{k=1}^{K}\sum_{h=1}^{H}\left(\big\langle P_{s_{h}^{k},a_{h}^{k},h},(V_{h+1}^{\star}-V_{h+1}^{\pi^{k}})^{2}\big\rangle-\big(\big\langle P_{s_{h}^{k},a_{h}^{k},h},\,V_{h+1}^{\star}-V_{h+1}^{\pi^{k}}\big\rangle\big)^{2}\right)\nonumber\\
 & \leq\sum_{k=1}^{K}\sum_{h=1}^{H}\Big\langle P_{s_{h}^{k},a_{h}^{k},h}-e_{s_{h+1}^{k}},\,(V_{h+1}^{\star}-V_{h+1}^{\pi^{k}})^{2}\Big\rangle\nonumber\\
 & \quad+2H\sum_{k=1}^{K}\sum_{h=1}^{H}\max\left\{ \left(V_{h}^{\star}(s_{h}^{k})-r_{h}(s_{h}^{k},a_{h}^{k})-\big\langle P_{s_{h}^{k},a_{h}^{k},h},V_{h+1}^{\star}\big\rangle\right)-\left(V_{h}^{\pi^{k}}(s_{h}^{k})-r_{h}(s_{h}^{k},a_{h}^{k})-\big\langle P_{s_{h}^{k},a_{h}^{k},h},V_{h+1}^{\pi^{k}}\big\rangle\right),0\right\} \nonumber\\
 & \leq2\sqrt{8H^{2}\sum_{k=1}^{K}\sum_{h=1}^{H}\mathbb{V}(P_{s_{h}^{k},a_{h}^{k},h},V_{h+1}^{\star}-V_{h+1}^{\pi^{k}})\log\frac{1}{\delta'}}\nonumber\\
 & \qquad\qquad+2H\sum_{k=1}^{K}\sum_{h=1}^{H}\left(V_{h}^{\star}(s_{h}^{k})-r_{h}(s_{h}^{k},a_{h}^{k})-\big\langle P_{s_{h}^{k},a_{h}^{k},h},V_{h+1}^{\star}\big\rangle\right)+3H^{2}\log\frac{1}{\delta'}.
	\label{eq:csl1}
\end{align}
%
Here, \eqref{eq:csl1} holds with probability at least $1-2SAHK\delta'$, a consequence of Lemma~\ref{lemma:self-norm}
 and Lemma~\ref{lemma:sqv}.



To further bound the right-hand side of \eqref{eq:csl1}, we develop the following upper bound: 
%
\begin{align}
 & \sum_{k=1}^{K}\sum_{h=1}^{H}\left(V_{h}^{\star}(s_{h}^{k})-r_{h}(s_{h}^{k},a_{h}^{k})-\big\langle P_{s_{h}^{k},a_{h}^{k},h},V_{h+1}^{\star}\big\rangle\right)\nonumber\\
 & =\sum_{k=1}^{K}\left(V_{1}^{\star}(s_{1}^{k})-V_{1}^{\pi^{k}}(s_{1}^{k})\right)+\sum_{k=1}^{K}\big(V_{1}^{\pi^{k}}(s_{1}^{k})-\sum_{h=1}^{H}r_{h}(s_{h}^{k},a_{h}^{k})\big)+\sum_{k=1}^{K}\sum_{h=1}^{H}\big\langle e_{s_{h+1}^{k}}-P_{s_{h}^{k},a_{h}^{k},h},\,V_{h+1}^{\star}\big\rangle.
	\label{eq:d1}
\end{align}
%
Note that the first term on the right-hand side \eqref{eq:d1} is exactly $\mathsf{Regret}(K)=T_1+T_2+T_3+T_4$, the second term on the right-hand side \eqref{eq:d1} corresponds to $-T_4$, 
whereas the third term on the right-hand side \eqref{eq:d1} can be bounded by 
%
\begin{align}
\sum_{k=1}^K\sum_{h=1}^H \big\langle e_{s_{h+1}^k}-P_{s_h^k,a_h^k,h}, V_{h+1}^{\star} \big\rangle
	\leq 2\sqrt{2\sum_{k=1}^K\sum_{h=1}^H \mathbb{V}(P_{s_h^k,a_h^k,h},V_{h+1}^{\star})\log \frac{1}{\delta'}}+3H\log \frac{1}{\delta'}
\end{align}
%
with probability at least $1-2SAHK\delta'$.
%
It then implies the validity of the following bound with probability exceeding $1-8SAHK\delta'$: 
%
\begin{align}
& \sum_{k=1}^K \sum_{h=1}^H \left(V_{h}^{\star}(s_h^k)-r_h(s_h^k,a_h^k) - \big\langle P_{s_h^k,a_h^k,h}, V_{h+1}^{\star} \big\rangle \right) \nonumber
\\ & \leq T_1+T_2+T_3+2|T_4| + 2\sqrt{2\sum_{k=1}^K\sum_{h=1}^H\mathbb{V}(P_{s_h^k,a_h^k,h},V_{h+1}^{\star})\log \frac{1}{\delta'} } + 55H\log \frac{1}{\delta'} .
\end{align}
%
Combining these bounds with \eqref{eq:csl1}, we can use a little algebra to further obtain 
%
\begin{align}
&\sum_{k=1}^K\sum_{h=1}^H \mathbb{V}(P_{s_h^k,a_h^k,h},V_{h+1}^{\star}-V_{h+1}^{\pi^k}) \nonumber
	\\ & \leq   4H\left(T_1+T_2+T_3+2|T_4|+2\sqrt{2\sum_{k=1}^K\sum_{h=1}^H\mathbb{V}(P_{s_h^k,a_h^k,h},V_{h+1}^{\star})\log \frac{1}{\delta'}} \right)    +262H^2\log \frac{1}{\delta'} 
	\label{eq:wc2}
\end{align}
%
with probability at least $1-8SAHK\delta'$. 
If we define $T_{10} = \sum_{k=1}^K\sum_{h=1}^H\mathbb{V}(P_{s_h^k,a_h^k,h},V_{h+1}^{\star})$,  
then substituting \eqref{eq:wc2} into \eqref{eq:cls0} yields: with probability exceeding $1-10SAHK\delta'$, 
%
\begin{align}
|\widecheck{T}_2|&\leq 2\sqrt{8K\mathrm{var}_1 \log \frac{1}{\delta'} } +  8\sqrt{H\left(T_1+T_2+T_3+2|T_4|+2\sqrt{2T_{10}\log \frac{1}{\delta'}}\right)\log \frac{1}{\delta'}} +107H\log\frac{1}{\delta'}  \nonumber
\\ &  \leq 11\sqrt{T_{10}\log \frac{1}{\delta'}} +16\sqrt{H(T_1+T_2+T_3+2|T_4|)\log \frac{1}{\delta'}} + 115H\log \frac{1}{\delta'}.
\end{align}
%


Combining the above bound on $|\widecheck{T}_2|$ with \eqref{eq:cht1},  with probability exceeding $1-10SAHK\delta'$
\begin{align}
|T_4|&\leq |\widecheck{T}_1| + |\widecheck{T}_2| \notag\\
	&\leq
	18\sqrt{SAHT_{10}(\log_2K)\log \frac{1}{\delta'}} +16\sqrt{H(T_1+T_2+T_3+2|T_4|)\log \frac{1}{\delta'}} + 135SAH^2(\log_2K)\log \frac{1}{\delta'},\nonumber
\end{align}
%
which together with a little algebra yields
%
\begin{align}
|T_4| & \leq   36\sqrt{SAHT_{10}(\log_2K)\log \frac{1}{\delta'}} + 32\sqrt{H(T_1+T_2+T_3)\log \frac{1}{\delta'}}+           306SAH^2(\log_2K)\log \frac{1}{\delta'}.\label{eq:nbt4}
\end{align}
%


\subsubsection{Bounding $T_5$ and $T_6$}

We now turn attention to the terms $T_5$ and $T_6$. 
Towards this, 
we start with the following lemma. 
%
\begin{lemma}\label{lemma:empv}
With probability at least $1-2SAHK\delta'$, one has
%
\begin{align}
 T_5 \leq  5T_6 +8BSAH^3.\label{eq:var5x}
\end{align}
%
\end{lemma}
%
\begin{proof}[Proof of Lemma~\ref{lemma:empv}]
Direct computation gives 
%
\begin{align}
 & \sum_{k,h}\mathbb{V}(\widehat{P}_{s_{h}^{k},a_{h}^{k},h}^{k},V_{h+1}^{k})\nonumber\\
 & =\sum_{k,h}\left(\big\langle\widehat{P}_{s_{h}^{k},a_{h}^{k},h}^{k},(V_{h+1}^{k})^{2}\big\rangle-\big(\big\langle\widehat{P}_{s_{h}^{k},a_{h}^{k},h},V_{h+1}^{k}\big\rangle\big)^{2}\right)\nonumber\\
 & \leq\sum_{k,h}\left(\big\langle P_{s_{h}^{k},a_{h}^{k},h}^{k},(V_{h+1}^{k})^{2}\big\rangle-\big(\big\langle P_{s_{h}^{k},a_{h}^{k},h},V_{h+1}^{k}\big\rangle\big)^{2}\right)+\sum_{k,h}\big\langle\widehat{P}_{s_{h}^{k},a_{h}^{k},h}-P_{s_{h}^{k},a_{h}^{k},h},(V_{h+1}^{k})^{2}\big\rangle\\
 & \qquad\qquad+2H\sum_{k,h}\big\langle\widehat{P}_{s_{h}^{k},a_{h}^{k},h}-P_{s_{h}^{k},a_{h}^{k},h},V_{h+1}^{k}\big\rangle\nonumber\\
 & \leq\sum_{k,h}\mathbb{V}(P_{s_{h}^{k},a_{h}^{k},h}^{k},V_{h+1}^{k})+\sum_{k,h}\big\langle\widehat{P}_{s_{h}^{k},a_{h}^{k},h}-P_{s_{h}^{k},a_{h}^{k},h},\,(V_{h+1}^{k})^{2}\big\rangle+2H\sum_{k,h}\big\langle\widehat{P}_{s_{h}^{k},a_{h}^{k},h}-P_{s_{h}^{k},a_{h}^{k},h},V_{h+1}^{k}\big\rangle\nonumber\\
 & =T_{5}+T_{7}+2HT_{1}.
%
\end{align}
%
Invoking Lemma~\ref{lemma:decouple} to bound $T_7$ and $T_1$, we obtain 
%
\begin{align}
 \sum_{k,h}\mathbb{V}(\widehat{P}_{s_h^k,a_h^k,h}^k, V_{h+1}^k) &\leq \sum_{k,h}\mathbb{V}(P_{s_h^k,a_h^k,h}^k  , V_{h+1}^k) +  6\sqrt{ \sum_{k,h} \mathbb{V}(P_{s_h^k,a_h^k,h},V_{h+1}^k)BSAH^3 }+3BSAH^3\nonumber
 \\ & \leq 5\sum_{k,h}\mathbb{V}(P_{s_h^k,a_h^k,h}^k  , V_{h+1}^k) +8BSAH^3 \label{eq:var5}
\end{align}
%
with probability exceeding $1-2SAHK\delta'$. 
\end{proof}




In view of Lemma~\ref{lemma:empv}, it suffices to bound $T_6 = \sum_{k,h}\mathbb{V}(P_{s_h^k,a_h^k,h},V_{h+1}^k)$. 
%
Given that $\mathsf{Var}(X+Y)\leq 2(\mathsf{Var}(X)+\mathsf{Var}(Y))$ holds for any two random variables $X,Y$, we have 
%
\begin{align}
	T_6&=\sum_{k,h} \mathbb{V}\big(P_{s_h^k,a_h^k,h},V_{h+1}^k\big)   \leq  2\sum_{k,h} \mathbb{V}\big(P_{s_h^k,a_h^k,h},V_{h+1}^{\star} \big) + 2\sum_{k,h}\mathbb{V}\big(P_{s_h^k,a_h^k,h},V_{h+1}^k -V_{h+1}^{\star} \big)\nonumber
\\ 
	& \leq 3   K\mathrm{var}_1 +\sum_{k=1}^K \left( \sum_{h=1}^H \mathbb{V}\big(P_{s_h^k,a_h^k,h},V_{h+1}^{\star} \big) - 3\mathrm{var}_1 \right) + 2\sum_{k,h}\mathbb{V}\big(P_{s_h^k,a_h^k,h},V_{h+1}^k -V_{h+1}^{\star} \big).\label{eq:var0}
\end{align}
%
To further upper bound the right-hand side of \eqref{eq:var0}, we make note of the following lemmas. 
%
\begin{lemma}\label{lemma:k1}
With probability at least $1-4SAHK\delta'$, it holds that
\begin{align}
T_{10} -2K\mathrm{var}_1=\sum_{k=1}^K \left( \sum_{h=1}^H \mathbb{V}\big(P_{s_h^k,a_h^k,h},V_{h+1}^{\star} \big) -2\mathrm{var}_1 \right)  \leq 80H^2\log \frac{1}{\delta'}.
\end{align}
\end{lemma}
%
% For the left term $\sum_{k,h}\mathbb{V}(P_{s_h^k,a_h^k,h},V_{h+1}^k -V_{h+1}^{\star} )$,   we have the following lemma.
%
\begin{lemma}\label{lemma:bdv1}
%
With probability at least $1-2\delta'$, it holds that
\begin{align}
 & \sum_{k,h}\mathbb{V}\big(P_{s_h^k,a_h^k,h}, V_{h+1}^k -V_{h+1}^{\star}\big)   \leq 4\sqrt{BH^2\sum_{k,h}\mathbb{V}(P_{s_h^k,a_h^k,h},V_{h+1}^k)}+ 4H\sum_{k,h}b_h^k(s_h^k,a_h^k)+ 3BSAH^3.\nonumber
\end{align}
\end{lemma}
%
Combining Lemma~\ref{lemma:k1} and Lemma~\ref{lemma:bdv1} with \eqref{eq:var0}, we see that with probability at least $1-6SAHK\delta'$,
%
\begin{align}
T_6 
	&= \sum_{k,h}\mathbb{V}(P_{s_h^k,a_h^k,h},V_{h+1}^k)\nonumber
%\\ 
%	& \leq  2\sum_{k,h}\mathbb{V}(P_{s_h^k,a_h^k,h},V_{h+1}^{\star})+2\sum_{k,h}\mathbb{V}(P_{s_h^k,a_h^k,h},V_{h+1}^k - V_{h+1}^{\star})\nonumber
\\ & \leq 4K\mathrm{var}_1 +  8\sqrt{BSAH^3 T_6}+8HT_2 + 7BSAH^3, 
	\nonumber
%\\ & \leq 8K\mathrm{var}_1 + 16HT_2 + 78BSAH^3.\label{eq:nbt6}
\end{align}
%
and as a result, 
%
\begin{align}
T_6 
	%&= \sum_{k,h}\mathbb{V}(P_{s_h^k,a_h^k,h},V_{h+1}^k)\nonumber
%\\ 
%	& \leq  2\sum_{k,h}\mathbb{V}(P_{s_h^k,a_h^k,h},V_{h+1}^{\star})+2\sum_{k,h}\mathbb{V}(P_{s_h^k,a_h^k,h},V_{h+1}^k - V_{h+1}^{\star})\nonumber
%\\ & \leq 4K\mathrm{var}_1 +  8\sqrt{BSAH^3 T_6}+8HT_2 + 7BSAH^3\nonumber
%\\ 
	& \leq 8K\mathrm{var}_1 + 16HT_2 + 78BSAH^3.\label{eq:nbt6}
\end{align}
%
This taken collectively with Lemma~\ref{lemma:empv} yields, with probability at least $1-8SAHK\delta'$, 
%
\begin{align}
	T_5 =\sum_{k,h}\mathbb{V}(\widehat{P}_{s_h^k,a_h^k,h},V_{h+1}^k)\leq 40K\mathrm{var}_1 + 80HT_2+398BSAH^3.\label{eq:nbt5}
\end{align}
%
To finish our bounds on $T_5$ and $T_6$, it remains to establish Lemma~\ref{lemma:k1} and Lemma~\ref{lemma:bdv1}. 



\begin{proof}[Proof of Lemma~\ref{lemma:k1}]
%
Let $\overline{R}_{h}^{\star}(s,a) = \mathbb{V}(P_{s,a,h},V_{h+1}^{\star})$, and define
\begin{align}
\overline{V}^k_h (s) = \mathbb{E}\left[ \sum_{h'=h}^H \overline{R}_{h'}(s_{h'},a_{h'}) \,\Big|\, s_h = s\right].\nonumber
\end{align}
%
Then $\overline{V}_h^k(s)\leq \mathrm{var}_1\leq H^2$. 
It then follows that
%
\begin{align}
  \sum_{h=1}^H  \mathbb{V}(P_{s_h^k,a_h^k,h},V_{h+1}^{\star})-\mathrm{var}_1 \nonumber & =\sum_{h=1}^H  \overline{R}_h^{\star}(s_h^k,a_h^k)-\mathrm{var}_1  \nonumber
 \\ & \leq \sum_{h=1}^H \overline{R}_h^{\star}(s_h^k,a_h^k) - \overline{V}^k_1(s_1^k)\nonumber
 \\ & = \sum_{h=1}^H \left(e_{s_{h+1}^k} - P_{s_{h}^k,a_h^k,h}  \right)\overline{V}^k_{h+1}.
\end{align}
%
Note that $\overline{V}^k$ depends only on $\pi^k$, which is determined before the beginning of the $k$-th episode.  
Consequently, applying Lemma~\ref{lemma:self-norm} reveals that, with probability at least $1-2SAHK\delta'$,
\begin{align}
 & \sum_{k=1}^K \left( \sum_{h=1}^H \mathbb{V}(P_{s_h^k,a_h^k,h},V_{h+1}^{\star}) - \overline{V}_1^k(s_1^k)  \right) \nonumber
 \\ & \leq 2\sqrt{2\sum_{k=1}^K \sum_{h=1}^H \mathbb{V}\big(P_{s_h^k,a_h^k,h},\overline{V}_{h+1}^k \big)\log \frac{1}{\delta'} } + 3H^2\log \frac{1}{\delta'} .\label{eq:xlll1}
\end{align}
%
Regarding the sum of variance terms on the right-hand side of \eqref{eq:xlll1},  
one can further bound
%
\begin{align}
 & \sum_{k=1}^{K}\sum_{h=1}^{H}\mathbb{V}\big(P_{s_{h}^{k},a_{h}^{k},h},\overline{V}_{h+1}^{k}\big)\nonumber\\
 & =\sum_{k=1}^{K}\sum_{h=1}^{H}\left(\big\langle P_{s_{h}^{k},a_{h}^{k},h},\,(\overline{V}_{h+1}^{k})^{2}\big\rangle-\big(\big\langle P_{s_{h}^{k},a_{h}^{k},h},\overline{V}_{h+1}^{k}\big\rangle\big)^{2}\right)\nonumber\\
 & =\sum_{k=1}^{K}\sum_{h=1}^{H}\big\langle P_{s_{h}^{k},a_{h}^{k},h}-e_{s_{h+1}^{k}},\,(\overline{V}_{h+1}^{k})^{2}\big\rangle\nonumber\\
 & \qquad+\sum_{k=1}^{H}\sum_{h=1}^{H}\left(\big(\overline{V}_{h+1}^{k}(s_{h+1}^{k})\big)^{2}-\big(\overline{V}_{h}^{k}(s_{h}^{k})\big)^{2}\right)+\sum_{k=1}^{K}\sum_{h=1}^{H}\left(\big(\overline{V}_{h}^{k}(s_{h}^{k})\big)^{2}-\big(\big\langle P_{s_{h}^{k},a_{h}^{k},h},\overline{V}_{h+1}^{k}\big\rangle\big)^{2}\right)\nonumber\\
 & \leq2\sqrt{8H^{4}\sum_{k=1}^{K}\sum_{h=1}^{H}\mathbb{V}\big(P_{s_{h}^{k},a_{h}^{k},h},\overline{V}_{h+1}^{k}\big)\log\frac{1}{\delta'}}+2H^{2}\sum_{k=1}^{K}\sum_{h=1}^{H}\overline{R}_{h}(s_{h}^{k},a_{h}^{k})+3H^{4}\log\frac{1}{\delta'}
	\label{eq:xllll2}
\end{align}
%
with probability at least $1-2SAHK\delta'$.  
Here, the last inequality arises from Lemma~\ref{lemma:self-norm} and Lemma~\ref{lemma:sqv} as well as
the fact that $\overline{V}_h^k(s_h^k) = \overline{R}_h(s_h^k,a_h^k)+P_{s_h^k,a_h^k,h}\overline{V}_{h+1}^k$.
%
It then follows from elementary algebra that
%
\begin{align}
\sum_{k=1}^K \sum_{h=1}^H \mathbb{V}(P_{s_h^k,a_h^k,h},\overline{V}_{h+1}^k)  \leq 4H^2\sum_{k=1}^K \sum_{h=1}^H \overline{R}_h(s_h^k,a_h^k) + 42 H^4\log \frac{1}{\delta'}.\label{eq:xlll3}
\end{align}
%


Substituting \eqref{eq:xlll3} into \eqref{eq:xlll1} gives
%
\begin{align}
\sum_{k=1}^K \sum_{h=1}^H \mathbb{V}(P_{s_h^k,a_h^k,h},V_{h+1}^{\star}) \leq \sum_{k=1}^H \overline{V}_1^k(s_1^k) +2\sqrt{8 H^2 \sum_{k=1}^K \sum_{h=1}^H \mathbb{V}(P_{s_h^k,a_h^k,h},V_{h+1}^{\star}) \log \frac{1}{\delta'}     } + 21H^2\log \frac{1}{\delta'},
	\nonumber
\end{align}
%
thus indicating that
%
\begin{align}
\sum_{k=1}^K \sum_{h=1}^H \mathbb{V}(P_{s_h^k,a_h^k,h},V_{h+1}^{\star}) \leq 2\sum_{k=1}^K \overline{V}_1^k(s_1^k)+ 84H^2\log \frac{1}{\delta'}
	\leq 2K\mathrm{var}_1 + 84H^2\log \frac{1}{\delta'} .\nonumber
\end{align}
%
The proof of Lemma~\ref{lemma:k1} is thus completed. 
%
\end{proof}



\begin{proof}[Proof of Lemma~\ref{lemma:bdv1}]
%
We make the observation that 
%
\begin{align}
 & \sum_{k,h}\mathbb{V}\big(P_{s_{h}^{k},a_{h}^{k},h},V_{h+1}^{k}-V_{h+1}^{\star}\big)\nonumber\\
 & =\sum_{k,h}\left(\big\langle P_{s_{h}^{k},a_{h}^{k},h},\,(V_{h+1}^{k}-V_{h+1}^{\star})^{2}\big\rangle-\big(\big\langle P_{s_{h}^{k},a_{h}^{k},h},\,V_{h+1}^{k}-V_{h+1}^{\star}\big\rangle\big)^{2}\right)\notag\\
 & =\sum_{k,h}\left(\big\langle P_{s_{h}^{k},a_{h}^{k},h}-e_{s_{h+1}^{k}},\,(V_{h+1}^{k}-V_{h+1}^{\star})^{2}\big\rangle\right)\nonumber\\
 & \qquad\qquad+\sum_{k,h}\left((V_{h+1}^{k}(s_{h+1}^{k})-V_{h+1}^{\star}(s_{h+1}^{k}))^{2}-\big(\big\langle P_{s_{h}^{k},a_{h}^{k},h},\,V_{h+1}^{k}-V_{h+1}^{\star}\big\rangle\big)^{2}\right)\nonumber\\
 & =\sum_{k,h}\left(\big\langle P_{s_{h}^{k},a_{h}^{k},h}-e_{s_{h+1}^{k}},\,(V_{h+1}^{k}-V_{h+1}^{\star})^{2}\big\rangle\right)+\sum_{k,h}\left((V_{h}^{k}(s_{h}^{k})-V_{h}^{\star}(s_{h}^{k}))^{2}-\big(\big\langle P_{s_{h}^{k},a_{h}^{k},h},\,V_{h+1}^{k}-V_{h+1}^{\star}\big\rangle\big)^{2}\right).
	\label{eq:var1}
 \end{align}
%
%Let $T_{101} = \sum_{k,h} \left(    (P_{s_h^k,a_h^k,h}-\textbf{1}_{s_{h+1}^k}) (V_{h+1}^k - V_{h+1}^{\star})^2   \right)$ and $T_{102} =\sum_{k,h} \left( (V_{h}^k(s_{h}^k)- V_{h}^{\star}(s_{h}^k))^2 - ((P_{s_h^k,a_h^k,h} (V_{h+1}^k - V_{h+1}^{\star}) )^2   ) \right) $. 
%
According to Lemma~\ref{lemma:self-norm} and Lemma~\ref{lemma:sqv}, we see that with probability exceeding $1-\delta'$, 
%
\begin{align}
	&\sum_{k,h}     \big\langle P_{s_h^k,a_h^k,h}-e_{s_{h+1}^k},\, (V_{h+1}^k - V_{h+1}^{\star})^2   \big\rangle \notag\\
	&\quad \leq 2\sqrt{2}\sqrt{4H^2\sum_{k,h}\mathbb{V}\big(P_{s_h^k,a_h^k,h}, V_{h+1}^k - V_{h+1}^{\star} \big)\log\frac{1}{\delta'}} + 3H^2\log \frac{1}{\delta'}.
	\label{eq:var3}
\end{align}
%
In addition, with probability at least $1-\delta'$ one has
%
 \begin{align}
 & \sum_{k,h}\left(\big(V_{h}^{k}(s_{h}^{k})-V_{h}^{\star}(s_{h}^{k})\big)^{2}-\big(\big\langle P_{s_{h}^{k},a_{h}^{k},h},\,V_{h+1}^{k}-V_{h+1}^{\star}\big\rangle\big)^{2}\right)\nonumber\\
 & \leq2H\sum_{k,h}\max\big\{ V_{h}^{k}(s_{h}^{k})-\big\langle P_{s_{h}^{k},a_{h}^{k},h},V_{h+1}^{k}\big\rangle-\big(V_{h}^{\star}(s_{h}^{k})-\big\langle P_{h}^{k},V_{h+1}^{\star}\big\rangle\big),0\big\}\nonumber\\
 & \leq2H\sum_{k,h}\max\big\{ V_{h}^{k}(s_{h}^{k})-\big\langle P_{s_{h}^{k},a_{h}^{k},h},V_{h+1}^{k}\big\rangle-r_{h}(s_{h}^{k},a_{h}^{k}),\,0\big\}\nonumber\\
 & \leq2H\sum_{k,h}\max\big\{\big\langle\widehat{P}_{s_{h}^{k},a_{h}^{k},h}-P_{s_{h}^{k},a_{h}^{k},h},V_{h+1}^{k}\big\rangle,0\big\}+2H\sum_{k,h}b_{h}^{k}\nonumber\\
 & \leq2\sqrt{BSAH^{3}\sum_{k,h}\mathbb{V}\big(P_{s_{h}^{k},a_{h}^{k},h},V_{h+1}^{k}\big)}+2H\sum_{k,h}b_{h}^{k}(s_{h}^{k},a_{h}^{k})+BSAH^{3}.
	 \label{eq:var4}
 \end{align}
%
It then follows that, with probability at least $1-2\delta'$,
\begin{align}
 & \sum_{k,h}\mathbb{V}\big(P_{s_h^k,a_h^k,h}, V_{h+1}^k -V_{h+1}^{\star}\big)  \leq 4\sqrt{BSAH^3\sum_{k,h}\mathbb{V}\big(P_{s_h^k,a_h^k,h},V_{h+1}^k\big)}+ 4H\sum_{k,h}b_h^k(s_h^k,a_h^k)+ 3BSAH^3, 
\end{align}
%
thereby concluding the proof.
%
\end{proof}




\subsubsection{Putting all this together}
%
To finish up, let us rewrite the inequalities $\eqref{eq:obt1}-\eqref{eq:obt8}$ as follows, 
with \eqref{eq:obt2}, \eqref{eq:obt4}, \eqref{eq:obt5} and \eqref{eq:obt6}  replaced by  \eqref{eq:nbt2}, \eqref{eq:nbt4}
\eqref{eq:nbt5} and \eqref{eq:nbt6}, respectively:
%
\begin{align}
& T_1 \leq \sqrt{128BSAHT_6}+24BSAH^2, \nonumber
\\ & T_7 \leq H\sqrt{512BSAHT_6}+24BSAH^3, \nonumber
\\ & T_9 \leq \sqrt{128BSAHT_6}+24BSAH^2, \nonumber
\\ & T_2\leq 100 \sqrt{B SAHT_5}+140BSAH^2, \nonumber
 \\ & T_3 \leq \sqrt{8BT_6}+3H\log\frac{1}{\delta'}  ,\nonumber
 \\ & T_4 \leq \sqrt{BSAHT_{10}}+32\sqrt{BH(T_1+T_2+T_3)}+BSAH^2,\nonumber
 \\ & T_5 \leq 40K\mathrm{var}_1 + 80HT_2+398BSAH^3  ,\nonumber
 \\ &  T_6 \leq  8K\mathrm{var}_1 + 16HT_2 + 78BSAH^3 ,\nonumber
 \\ & T_8 \leq \sqrt{32BH^2T_6 } + 3BH^2 ,\nonumber
\end{align}
%
where we recall that $B =4000 (\log_2K)^3\log(3SA)\log \frac{1}{\delta'} $. 
In addition, it follows from Lemma~\ref{lemma:k1} that 
$$ T_{10} \leq 2K\mathrm{var}_1 + 80BH^2.$$ 
%
Solving the inequalities above reveals that, with probability exceeding $1-200SAH^2K^2\delta'$, 
%
\begin{align}
\mathsf{Regret}(K)= T_1+T_2+T_3+T_4 \leq O\left( \sqrt{BSAHK\mathrm{var_1} }+ BSAH^2 \right).\label{eq:rbvar1}
\end{align}
%
One can thus conclude the proof by recalling that $\delta'=\frac{\delta}{200SAH^2K^2}$.
