
\subsection{Proof of Lemma~\ref{lemma:key2}}\label{app:pfkey2}
%
It suffices to develop an upper bound on the cardinality of $\mathcal{C}$ (cf.~\eqref{eq:defn-C-choice}). 
%
Setting   
%
\begin{equation}
	M = \log_2 K \qquad\qquad \text{and} \qquad\qquad N = SAH,
\end{equation}
%
we find it helpful to introduce the following useful sets: 
%
\begin{subequations}
\begin{align}
	\mathcal{C}^{\mathsf{distinct}}(l) 
	&\coloneqq \Big\{ \mathcal{I} = \{\mathcal{I}^1,\ldots,\mathcal{I}^l\} \mid \mathcal{I}^{1}\leq \cdots \leq\mathcal{I}^{l}  ,  \mathcal{I}^{\tau}\in \{0,1,\cdots,M\}^N 
	\text{ and } \mathcal{I}^{\tau}\neq \mathcal{I}^{\tau+1}  ~(\forall\tau) 
 \Big\} ;
	\\
	\mathcal{C}^{\mathsf{distinct}} &\coloneqq \bigcup_{l\geq 1}\mathcal{C}^{\mathsf{distinct}}(l). 
\end{align}
\end{subequations}
% 
In words, $\mathcal{C}^{\mathsf{distinct}}(l)$ can be viewed as the set of non-decreasing length-$l$ paths in $\{0,1,\cdots,M\}^N$, with all points on a path being distinct;   
 $\mathcal{C}^{\mathsf{distinct}}$ thus consists of all such paths  regardless of the length.
%



We first establish a connection between $|\mathcal{C}|$ and $\big|\mathcal{C}^{\mathsf{distinct}}\big|$. 
Define the operator $\mathsf{Proj}:\mathcal{C}\to \mathcal{C}^{\mathsf{distinct}}$ 
that maps each $\mathcal{I}\in \mathcal{C}$ to $\mathcal{I}^{\mathsf{distinct}}\in \mathcal{C}^{\mathsf{distinct}}$, 
where $\mathcal{I}^{\mathsf{distinct}}$ is composed of all distinct elements in $\mathcal{I}$ 
(in other words, this operator simply removes redundancy in $\mathcal{I}$). Let us looking at the following set 
%
\[
	\mathcal{B}(\mathcal{I}^{\mathsf{distinct}}) \coloneqq \big\{ \mathcal{I}\in \mathcal{C} \mid \mathsf{Proj}(\mathcal{I})=\mathcal{I}^{\mathsf{distinct}} \big\}
\]
%
for each $\mathcal{I}^{\mathsf{distinct}}\in \mathcal{C}^{\mathsf{distinct}}$. 
%
Since $\mathcal{I}^{\mathsf{distinct}}$ is a non-decreasing path with all its points being distinct, 
there are at most $MN+1$ elements in each $\mathcal{I}^{\mathsf{distinct}}$. Hence, the size of $\mathcal{B}(\mathcal{I}^{\mathsf{distinct}}) $ is at most the number of solutions to the following equations
%
\begin{align}
	\sum_{i=1}^{MN+1} x_i = K \qquad \text{and} \qquad x_i\in \mathbb{N} ~~\text{for all } 1\leq i \leq MN+1\nonumber.
\end{align}
%
Elementary combinatorial arguments then reveal that 
%
\begin{align*}
	\big| \mathcal{B}(\mathcal{I}^{\mathsf{distinct}})  \big | \leq 
\left( \begin{array}{c} K+MN \\ MN
\end{array}\right) 
%= \frac{(K+MN)!}{(MN)! K!}
\leq (K+MN)^{MN}\leq (2K)^{MN}
\end{align*}
%
for each $\mathcal{I}^{\mathsf{distinct}}$, 
provided that $K\geq MN=SAH\log_2K$. 
We then arrive at 
%
\begin{equation}
	|\mathcal{C}|\leq \big|\mathcal{C}^{\mathsf{distinct}}\big|\cdot (2K)^{MN}. 
	\label{eq:C-Cdistinct}
\end{equation}
%



Everything then boils down to bounding $|\mathcal{C}^{\mathsf{distinct}}|$. 
To do so, let us first look at the set $\mathcal{C}^{\mathsf{distinct}}(MN+1)$, 
as each path in $\mathcal{C}^{\mathsf{distinct}}$ cannot have length more than $MN+1$.  
For each $\mathcal{I}^{\mathsf{distinct}} = \{\widetilde{\mathcal{I}}^1,\widetilde{\mathcal{I}}^2,\ldots, \widetilde{\mathcal{I}}^{MN+1}\}\in \mathcal{C}^{\mathsf{distinct}}(MN+1)$, 
it is easily seen that
%
\begin{itemize}
%
	\item	$\widetilde{\mathcal{I}}^1 = [0,0,\ldots,0]^{\top}$ and $\widetilde{\mathcal{I}}^{MN+1}=[M,M,\ldots, M]^{\top}$.
%
	\item For each $1\leq \tau \leq MN$, $\widetilde{\mathcal{I}}^{\tau}$ and $\widetilde{\mathcal{I}}^{\tau+1}$ differ only in one element (i.e., their Hamming distance is 1). 
%
\end{itemize}
%
In other words, we can view $\mathcal{I}^{\mathsf{distinct}}$ as an $MN$-step path from  $ [0,0,\ldots,0]^{\top}$ to $[M,M,\ldots, M]^{\top}$, 
with each step moving in one dimension. 
Clearly, each step has at most $N$ directions to choose from, meaning that there are at most $N^{MN}$ such paths. This implies that 
%
\begin{align*}
	\big|\mathcal{C}^{\mathsf{distinct}}(MN+1)\big|\leq N^{MN}.
\end{align*}
%
To finish up, we further observe that for each $\mathcal{I}^{\mathsf{distinct}}\in \mathcal{C}^{\mathsf{distinct}}$, there exists some $\widetilde{\mathcal{I}}^{\mathsf{distinct}}\in \mathcal{C}^{\mathsf{distinct}}(MN+1)$ such that $\mathcal{I}^{\mathsf{distinct}}\subseteq \widetilde{\mathcal{I}}^{\mathsf{distinct}}$.  
This observation together with basic combinatorial arguments indicates that
%
\begin{align*}
	\big|\mathcal{C}^{\mathsf{distinct}}\big| \leq  2^{MN+1} \big|\mathcal{C}^{\mathsf{distinct}}(MN+1)\big|\leq (2N)^{MN+1}, 
\end{align*}
%
which taken collectively with \eqref{eq:C-Cdistinct} leads to the advertised bound  
%
\begin{align*}
	|\mathcal{C}|\leq (2K)^{MN} \big|\mathcal{C}^{\mathsf{distinct}}\big|
	%\leq (2K)^{MN} 2^{MN+1} \big|\mathcal{C}^{\mathsf{distinct}}(MN+1)\big|
	\leq (4KN)^{MN+1} \leq (4KN)^{MN+1}.
\end{align*}
%





\subsection{Proof of Lemma~\ref{lemma:uniform}}\label{app:pfuniform}
%
Let us begin by considering any fixed total profile $\mathcal{I}\in \mathcal{C}$, 
any fixed integer $l$ obeying $2\leq l\leq\log_{2}K+1$, and any given
feasible sequence $\{X_{h,s,a}\}_{(s,a,h)\in\mathcal{S}\times\mathcal{A}\times[H]}$.
Recall that (i) $\widehat{P}_{s,a,h}^{(l)}$ is computed based on
the $l$-th batch of data comprising $2^{l-2}$ independent samples
from $\mathcal{D}^{\mathsf{expand}}$ (see Definition~\ref{def:filt2});
and (ii) each $X_{h+1,s,a}$ is given by a deterministic function of $\mathcal{I}$
and the empirical models for steps $h'\in[h+1,H]$. Consequently,
Lemma~\ref{lemma:self-norm} together with Definition~\ref{def:filt2} tells us that:
with probability at least $1-\delta'$, one has
%
\begin{align}
	%\Bigg|
	&\sum_{s,a,h}\big\langle \widehat{P}_{s,a,h}^{(l)}-P_{s,a,h}, X_{h+1,s,a} \big\rangle \notag\\
	%\Bigg|
	&\qquad 
	\leq \sqrt{\frac{8}{2^{l-2}}\sum_{s,a,h}\mathbb{V}\big(P_{s,a,h},X_{h+1,s,a}\big)\log\frac{3\log_{2}(SAHK)}{\delta'}}+\frac{4H}{2^{l-2}}\log\frac{3\log_{2}(SAHK)}{\delta'},
	\label{eq:xx1-aux}
\end{align}
%
where we view the left-hand side of \eqref{eq:xx1-aux} as a martingale sequence from $h=H$ back to $h=1$. 


Moreover,  given that each $X_{h,s,a}$ has at most $K+1$ different
choices (since we assume $|\mathcal{X}_{h,\mathcal{I}}|\leq K+1$), 
there are no more than $(K+1)^{SAH}\leq (2K)^{SAH}$ possible choices of the feasible sequence
$\{X_{h,s,a}\}_{(s,a,h)\in\mathcal{S}\times\mathcal{A}\times[H]}$. 
In addition, it has been shown in Lemma~\ref{lemma:key2} that there are no more than $(4SAHK)^{2SAH\log_{2}K}$ possibilities of the total profile $\mathcal{I}$.
%
Taking the union bound over all these choices and replacing $\delta'$
in (\ref{eq:xx1-aux}) with $\delta'/\big((4SAHK)^{2SAH\log_{2}K} (2K)^{SAH}\log_2K\big)$, we can demonstrate that with probability at least $1-\delta'$, 
%
\begin{align}
 & %\Bigg|
	\sum_{s,a,h}\big\langle \widehat{P}_{s,a,h}^{(l)}-P_{s,a,h}, X_{h+1,s,a} \big\rangle
	%\Bigg|
	\notag\\
	& \leq\sqrt{\frac{8}{2^{l-2}}\sum_{s,a,h}\mathbb{V}\big(P_{s,a,h},X_{h+1,s,a}\big)\left(2SAH\log_{2}K\log(4SAHK)+SAH\log (2K)+\log\frac{3\log_{2}^{2}(SAHK)}{\delta'}\right)}\nonumber\\
	& \qquad+\frac{4H}{2^{l-2}}\left(2SAH\log_{2}K\log(4SAHK)+SAH\log (2K)+\log\frac{3\log_{2}^{2}(SAHK)}{\delta'}\right) \notag\\
 & \leq\sqrt{\frac{8}{2^{l-2}}\sum_{s,a,h}\mathbb{V}\big(P_{s,a,h},X_{h+1,s,a}\big)\left(6SAH\log_{2}^{2}K+\log\frac{1}{\delta'}\right)}
	+\frac{4H}{2^{l-2}}\left(6SAH\log_{2}^{2}K+\log\frac{1}{\delta'}\right)
	\label{eq:xx1-aux-123}
\end{align}
%
holds simultaneously for all $\mathcal{I}\in \mathcal{C}$, all $2\leq l\leq\log_{2}K+1$, and all feasible sequences $\{X_{h,s,a}\}_{(s,a,h)\in\mathcal{S}\times\mathcal{A}\times[H]}$. 


Finally, recalling our assumption $0\in \mathcal{X}_{h+1,\mathcal{I}}$, we see that 
for every total profile $\mathcal{I}$ and its associated feasible sequence  $\{X_{h,s,a}\}$, 
%
\[
	\sum_{s,a,h}\max\Big\{\big\langle\widehat{P}_{s,a,h}^{(l)}-P_{s,a,h},X_{h+1,s,a}\big\rangle,0\Big\}\in\bigg\{ \sum_{s,a,h}\big\langle\widehat{P}_{s,a,h}^{(l)}-P_{s,a,h},\widetilde{X}_{h+1,s,a}\big\rangle \,\Big|\, \widetilde{X}_{h+1,s,a}\in\mathcal{X}_{h+1,\mathcal{I}}, \forall (s,a,h)\bigg\} 
\]
%
holds true.  Consequently, the uniform upper bound on the right-hand side of \eqref{eq:xx1-aux-123} continues to be a valid upper bound on 
$\sum_{s,a,h}\max\big\{\big\langle\widehat{P}_{s,a,h}^{(l)}-P_{s,a,h},X_{h+1,s,a}\big\rangle,0\big\}$. This concludes the proof. 




\subsection{Proof of Lemma~\ref{lemma:decouple}}\label{app:pfdecouple}


We begin by making the following claim, which we shall establish towards the end of this subsection.   
%
\begin{claim}
	\label{claim:PV-l-UB}
	With probability exceeding $1-\delta'$, 
%
\begin{align}
 & \sum_{s,a,h}\Big\langle\widehat{P}_{s,a,h}^{(l)}-P_{s,a,h},V_{h+1}^{k_{l,j,s,a,h}}\Big\rangle \notag\\
 & \qquad\leq\sqrt{\frac{8}{2^{l-2}}\sum_{s,a,h}\mathbb{V}\big(P_{s,a,h},V_{h+1}^{k_{l,j,s,a,h}}\big)\left(6SAH\log_{2}^{2}K+\log\frac{1}{\delta'}\right)}+\frac{4H}{2^{l-2}}\left(6SAH\log_{2}^{2}K+\log\frac{1}{\delta'}\right)
	\label{eq:claim-PV-l-UB}
\end{align}
%
holds simultaneously for all $l=1,\ldots,\log_{2}K$ and all $j=1,\ldots,2^{l-1}$, 
where $k_{l,j,s,a,h}$ stands for the episode index of the sample that visits $(s,a,h)$ for the $(2^{l-1}+j)$-th time in the online learning process. 
\end{claim}
%


Assuming the validity of Claim~\ref{claim:PV-l-UB} for the moment, 
we can combine this claim with the decomposition~\eqref{eq:PV-sum-decompose} and applying the Cauchy-Schwarz inequality to reach
%
\begin{align*}
 & \sum_{k=1}^{K}\sum_{h=1}^{H}\Big\langle\widehat{P}_{s_{h}^{k},a_{h}^{k},h}^{k}-P_{s_{h}^{k},a_{h}^{k},h},V_{h+1}^{k}\Big\rangle\leq\sum_{l=1}^{\log_{2}K}\sum_{j=1}^{2^{l-1}}\sum_{s,a,h}\Big\langle\widehat{P}_{s,a,h}^{(l)}-P_{s,a,h},V_{h+1}^{k_{l,j,s,a,h}}\Big\rangle+SAH^{2}\\
 & \leq\sum_{l=1}^{\log_{2}K}\sum_{j=1}^{2^{l-1}}\sqrt{\frac{8}{2^{l-2}}\sum_{s,a,h}\mathbb{V}\big(P_{s,a,h},V_{h+1}^{k_{l,j,s,a,h}}\big)\left(6SAH\log_{2}^{2}K+\log\frac{1}{\delta'}\right)}\\
 & \qquad\qquad+\sum_{l=1}^{\log_{2}K}\sum_{j=1}^{2^{l-1}}\frac{4H}{2^{l-2}}\left(6SAH\log_{2}^{2}K+\log\frac{1}{\delta'}\right)+SAH^{2}\\
 & \leq\sum_{l=1}^{\log_{2}K}\sqrt{16\sum_{j=1}^{2^{l-1}}\sum_{s,a,h}\mathbb{V}\big(P_{s,a,h},V_{h+1}^{k_{l,j,s,a,h}}\big)\left(6SAH\log_{2}^{2}K+\log\frac{1}{\delta'}\right)}\\
 & \qquad\qquad+\sum_{l=1}^{\log_{2}K}8H\left(6SAH\log_{2}^{2}K+\log\frac{1}{\delta'}\right)+SAH^{2}\\
 & \leq\sqrt{16(\log_{2}K)\sum_{l=1}^{\log_{2}K}\sum_{j=1}^{2^{l-1}}\sum_{s,a,h}\mathbb{V}\big(P_{s,a,h},V_{h+1}^{k_{l,j,s,a,h}}\big)\left(6SAH\log_{2}^{2}K+\log\frac{1}{\delta'}\right)}\\
 & \qquad\qquad+\left(48SAH^{2}\log_{2}^{3}K+8H(\log_{2}K)\log\frac{1}{\delta'}\right)+SAH^{2}\\
 & \leq\sqrt{16(\log_{2}K)\sum_{k=1}^{K}\sum_{h=1}^{H}\mathbb{V}\big(P_{s_{h}^{k},a_{h}^{k},h},V_{h+1}^{k}\big)\left(6SAH\log_{2}^{2}K+\log\frac{1}{\delta'}\right)}+49SAH^{2}\log_{2}^{3}K+8H(\log_{2}K)\log\frac{1}{\delta'}.
\end{align*}
%
Here, the last inequality is valid due to our assumption $V_{h+1}^k=0$ ($\forall k > K$) and the identity
%
\begin{align}
 & \sum_{k=1}^{K}\sum_{h=1}^{H}\mathbb{V}\big(P_{s_{h}^{k},a_{h}^{k},h},V_{h+1}^{k}\big)\nonumber\\
 & =\sum_{l=1}^{\log_{2}K}\sum_{s,a,h}\sum_{j=1}^{2^{l-1}}\mathbb{V}\big(P_{s,a,h},V_{h+1}^{k_{l,j,s,a,h}}\big)+\sum_{k=1}^{K}\sum_{h=1}^{H}\mathds{1}\Big\{ N_{h}^{k,\mathsf{all}}(s_{h}^{k},a_{h}^{k})=0\Big\}\mathbb{V}\big(P_{s_{h}^{k},a_{h}^{k},h},V_{h+1}^{k}\big).\nonumber
\end{align}
%
This establishes our advertised bound on $\sum_{k,h}\big\langle\widehat{P}_{s_{h}^{k},a_{h}^{k},h}^{k}-P_{s_{h}^{k},a_{h}^{k},h},V_{h+1}^{k}\big\rangle$, 
provided that Claim~\ref{claim:PV-l-UB} is valid.  




Before proceeding to the proof of Claim~\ref{claim:PV-l-UB}, 
we note that the other two quantities $\sum_{k,h}\max\big\{\big\langle\widehat{P}_{s_{h}^{k},a_{h}^{k},h}^{k}-P_{s_{h}^{k},a_{h}^{k},h},V_{h+1}^{k}\big\rangle,0\big\}$ 
and $\sum_{k,h}\big\langle\widehat{P}_{s_{h}^{k},a_{h}^{k},h}^{k}-P_{s_{h}^{k},a_{h}^{k},h},\big(V_{h+1}^{k}\big)^{2}\big\rangle$ 
can be upper bounded using exactly the same arguments, which we omit for the sake of brevity. 
In particular, the latter quantity further satisfies 
%
\begin{align*} 
	& \sum_{k=1}^{K}\sum_{h=1}^{H}\Big\langle\widehat{P}_{s_{h}^{k},a_{h}^{k},h}^{k}-P_{s_{h}^{k},a_{h}^{k},h},\big(V_{h+1}^{k}\big)^{2}\Big\rangle\\
 & \leq\sqrt{16(\log_{2}K)\sum_{k=1}^{K}\sum_{h=1}^{H}\mathbb{V}\big(P_{s_{h}^{k},a_{h}^{k},h},\big(V_{h+1}^{k}\big)^{2}\big)\left(6SAH\log_{2}^{2}K+\log\frac{1}{\delta'}\right)}+49SAH^{3}\log_{2}^{3}K+8H^2(\log_{2}K)\log\frac{1}{\delta'}\\
 & \leq8H\sqrt{(\log_{2}K)\sum_{k=1}^{K}\sum_{h=1}^{H}\mathbb{V}\big(P_{s_{h}^{k},a_{h}^{k},h},V_{h+1}^{k}\big)\left(6SAH\log_{2}^{2}K+\log\frac{1}{\delta'}\right)}+49SAH^{3}\log_{2}^{3}K+8H^2(\log_{2}K)\log\frac{1}{\delta'},
\end{align*}
%
where the last inequality follows from Lemma~\ref{lemma:sqv} and
the fact that $0\leq V_{h+1}^{k}(s)\le H$ for all $s\in\mathcal{S}$. 





\begin{proof}[Proof of Claim~\ref{claim:PV-l-UB}]


To invoke Lemma~\ref{lemma:decouple} to prove this claim,  we need to choose the set $\{\mathcal{X}_{h,\mathcal{I}}\}$ properly to include the true value function estimates $\{V_h^k\}$. 
To do so, we find it helpful to first introduce an auxiliary algorithm tailored to each total profile. 
Specifically, for each $\mathcal{I} \in \mathcal{C}$ (cf.~\eqref{eq:defn-C-choice}), consider the following updates operating upon the expanded sample set $\mathcal{D}^{\mathsf{expand}}$.  
%
% where $\widehat{r}^{(j)}_h(s,a)$ (resp.~$\widehat{\sigma}^{(j)}_h(s,a)$) denotes the empirical reward (resp.~squared reward) w.r.t.~$(s,a,h)$ based on the $j$-th batch of data. 
%

\begin{algorithm}[ht]
	\DontPrintSemicolon
	\caption{Monotoinic Value Propagation for a given total profile $\mathcal{I}\in \mathcal{C}$ ($\mathtt{MVP}(\mathcal{I})$) \label{alg:main-fixed-profile}}
	
	\textbf{initialization: } set $V_{H+1}^{k,\mathcal{I}}(s)\leftarrow H$ for all $s\in \mathcal{S}$ and $1\leq k\leq K$. \\
	\For{$k=1,2,\ldots,K$} {
			\For{$h=H,H-1,...,1$} {
				%
				\For{$(s,a)\in \mathcal{S}\times \mathcal{A}$} { \vspace{-1ex}
					\begin{align*} 
						\vspace{-3ex}
						j & \leftarrow I_{s,a,h}^k,~~ n \leftarrow 2^{j-2}, \\
						b_h(s,a) &\leftarrow c_1 \sqrt{\frac{   \mathbb{ V}\big(\widehat{P}^{(j)}_{s,a,h} ,V_{h+1}^{k,\mathcal{I}}\big) \log \frac{1}{\delta'}  }{ \max\{n,1 \} }}+c_2 \sqrt{\frac{\big(\widehat{\sigma}^{(j)}_h(s,a)- (\widehat{r}^{(j)}_h(s,a))^2 \big)\log \frac{1}{\delta'}}{\max\{n,1\}}} 
						 +c_3\frac{H\log \frac{1}{\delta'}}{ \max\{n,1\}  },  
						\\
						Q_h^{k,\mathcal{I}}(s,a) &\leftarrow \min\Big\{    \widehat{r}^{(j)}_h(s,a)+\langle \widehat{P}^{(j)}_{s,a,h}, V_{h+1}^{k,\mathcal{I}} \rangle +b_h(s,a)    ,H\Big\}, \\
						V^{k,\mathcal{I}}_{h}(s) &\leftarrow \max_{a}Q^{k,\mathcal{I}}_{h}(s,a).
					\end{align*}
					\vspace{-3ex}
				}
			}
			%
	}
\end{algorithm}
%
If we construct
%
\begin{align}
	\mathcal{X}_{h,\mathcal{I}} \coloneqq \Big\{ V_h^{k,\mathcal{I}} \mid 1\leq k\leq K\Big\} \cup \{0\}, \qquad \forall h\in [H] \text{ and }\mathcal{I}\in \mathcal{C}, 
	\label{eq:X-I-ours}
\end{align}
%
then it can be easily seen that $\{\mathcal{X}_{h,\mathcal{I}}\}$ satisfies the properties stated right before  Lemma~\ref{lemma:uniform}.  
As a consequence, applying Lemma~\ref{lemma:uniform} yields
%
\begin{align}
 & \sum_{s,a,h}\Big\langle\widehat{P}_{s,a,h}^{(l)}-P_{s,a,h},X_{h+1,s,a}\Big\rangle \notag\\
 & \qquad\leq\sqrt{\frac{8}{2^{l-2}}\sum_{s,a,h}\mathbb{V}\big(P_{s,a,h},X_{h+1,s,a}\big)\left(6SAH\log_{2}^{2}K+\log\frac{1}{\delta'}\right)}+\frac{4H}{2^{l-2}}\left(6SAH\log_{2}^{2}K+\log\frac{1}{\delta'}\right)
	\label{eq:claim-PV-l-UB-temp}
\end{align}
%
simultaneously for all $l=1,\ldots,\log_{2}K$, all $\mathcal{I}\in \mathcal{C}$, and all sequences $\{X_{h,s,a}\}$ obeying $X_{h,s,a}\in \mathcal{X}_{h,\mathcal{I}}$, $\forall (s,a,h)$.  


To finish up, denote by $\mathcal{I}^{\mathsf{true}}$ the true total profile resulting from the online learning process. 
Given the way we couple $\mathcal{D}^{\mathsf{expand}}$ and $\mathcal{D}^{\mathsf{original}}$ (see the beginning of Section~\ref{sec:decoupling-all}), 
	we can easily see that the true value function estimate $\{V_{h}^{k}\}$ obeys
%
\begin{equation}
	V_{h}^{k} = V_{h}^{k,\mathcal{I}^{\mathsf{true}}} \in \mathcal{X}_{h,\mathcal{I}^{\mathsf{true}}},\qquad 1\leq k \leq K. 
	\label{eq:Vk-Itrue}
\end{equation}
%
The claimed result then follows immediately from \eqref{eq:Vk-Itrue} and the uniform bound \eqref{eq:claim-PV-l-UB-temp}. 
\end{proof}




%
%A few properties about Algorithm~\ref{alg:main-fixed-profile} are in order. 
%%
%%To ease presentation,  our discussions in this subsection are conditioned on the randomness of the rewards (i.e., we focus only on the randomness underlying the empirical transition kernels).
%%
%\begin{itemize}
%	\item[(a)] Given any fixed total profile $\mathcal{I} \in \mathcal{C} $, each estimate $V^{k}_{h}$ computed by Algorithm~\ref{alg:main-fixed-profile} is a {\em deterministic} mapping of $\big\{ \widehat{P}_{h'}^{(j)} \mid h'>h, j\leq \log_2K \big\}$  (when conditioned on the randomness of the rewards); as a result,  $V^{k}_{h}$ is statistically independent of  $\big\{ \widehat{P}_{h_0}^{(j)} \mid h_0\leq h, j\leq \log_2K \big\}$. 
%
%	\item[(b)] Once the total profile $\mathcal{I}^{\mathsf{true}}$ of the visitation counts of Algorithm~\ref{alg:main} is revealed, 
%		then the value estimates of Algorithm~\ref{alg:main} are equivalent to those of Algorithm~\ref{alg:main-fixed-profile} with $\mathcal{I}=\mathcal{I}^{\mathsf{true}}$. 
%\end{itemize}
%%
%












\begin{comment}


\subsection{Proof of Lemma~\ref{lemma:key1}}\label{app:pfseckey1}


 
 Since $\mathcal{X}_{h+1}$ has at most $W$ elements, we can write $\mathcal{X}_{h+1} = \{x_{h+1}(w)  \}_{w=1}^W$, where each $x_{h+1}(w)$ could be regarded as a function of $\{\widehat{P}^{(J^{k}_{s,a,h})}_{s,a,h'}, \widehat{r}^{(J^k_{s,a,h'})_{h'}(s,a)}, \widehat{\sigma}^{(J^k_{s,a,h'})}_{h'}(s,a)\}_{ h+1\leq h'\leq H, 1\leq k\leq K,(s,a)}$  and $\{J^k\}_{k=1}^K$. Now we fix a group of indices $\{w_{s,a,h}\}_{(s,a,h)}$ where $1\leq w_{s,a,h}\leq W$ for each $(s,a,h)$.

Fix $2\leq l \leq \log_2(K)+1$. We consider to bound the term
%
\begin{align}
T(l,\{w_{s,a,h}\}_{(s,a,h)}) \coloneqq 2^{l-2}\sum_{s,a,h}(\widehat{P}^{(l)}_{s,a,h} - P_{s,a,h}) x_{h+1}(w_{s,a,h}).\nonumber 
\end{align}
%
Recall that $\widehat{P}^{(l)}_{s,a,h}$ is the empirical transition of 
the $l$-th batch of $(s,a,h)$, i.e., the empirical transition of the $2^{l-2}+1$-th to $2^{l-1}$-th samples of $(s,a,h)$. Also recall the definition of $\mathcal{F}_{\mathrm{gen}} = \{\widetilde{F}(z)\}_{z=1}^{SAHK}$ in Definition~\ref{def:filt2}. 
 Conditioned on $\widetilde{\mathcal{F}}((H-h)\cdot SAK)$, $\{x_{h+1}(w_{s,a,h})\}_{(s,a)}$ is fixed, and $\{2^{l-2}\widehat{P}^{(l)}_{s,a,h}\}_{(s,a)}$ are mutually independent multinomial random variables.   Let $v_{s,a,h}(t,l)$ be the next state of the $t$-th sample of  the $ l$-th batch of $(s,a,h)$.   
By writing 
%
\begin{align}
T(l,\{w_{s,a,h}\}_{(s,a,h)}) = \sum_{s,a,h}\sum_{\tau=1}^{2^{l-2}}  ( \textbf{1}_{v_{s,a,h}(\tau,l)} -P_{s,a,h}) \cdot x_{h+1}(w_{s,a,h}),
\end{align}
%
using Lemma~\ref{lemma:self-norm}, with probability at least $1-10SAH^2K^2\delta'$,
%
\begin{align}
T(l,\{w_{s,a,h}\}_{(s,a,h)})  \leq 2\sqrt{2}\cdot \sqrt{ 2^{l-2}\sum_{s,a,h} \mathbb{V}(P_{s,a,h},x_{h+1}(w_{s,a,h}))   \log \frac{1}{\delta'}  } + 3H \log \frac{1}{\delta'}.\label{eq:xx1}
\end{align}


Note that $\{w_{s,a,h}\}_{(s,a,h)}$ has at most $(W)^{SAH}$ choices.
 Applying the union bound and rescaling $\delta'$ to $\frac{\delta'}{|W|^{SAH}}$, we see that: with probability at least $1-10SAH^2K^2\delta'$, 
 %
\begin{align}
 & T(l,\{w_{s,a,h}\}_{(s,a,h)})\nonumber\\ &= 2^{l-2}\sum_{s,a,h} (\widehat{P}_{s,a,h}^{(l)}-P_{s,a,h} )x_{h+1}(w_{s,a,h}) \nonumber
 \\ & \leq 2\sqrt{2}\cdot \sqrt{ 2^{l-2}\sum_{s,a,h} \mathbb{V}(P_{s,a,h},x_{h+1}(w_{s,a,h}))   \bigg(2SAH\log W+\log\frac{1}{\delta'}\bigg)  } + 6H\bigg(SAH\log K+\log\frac{1}{\delta'}\bigg) \label{eq:xx2}
\end{align}
%
holds for any $\{w_{s,a,h}\}_{(s,a,h)}$ such that $1\leq w_{s,a,h}\leq W,\forall (s,a,h)$. 



For $l=1$, we have that $\sum_{s,a,h} (\widehat{P}_{s,a,h}^{(l)}-P_{s,a,h} )x_{h+1}(w_{s,a,h}) \leq SAH^2$ trivially.



Now we rewrite 
%
\begin{align}
 & \sum_{k=1}^K \sum_{h=1}^H \left( \widehat{P}^{(J^k_{s_h^k,a_h^k,h})}_{s_h^k,a_h^k,h}-P_{s_h^k,a_h^k,h} \right) X_{h+1}^k  \nonumber
 \\ & = \sum_{l=0}^{\log_2(K)} \sum_{s,a,h} (\widehat{P}_{s,a,h}^{(l)}-P_{s,a,h}) \sum_{k=1}^K \mathbb{I}[(s_h^k,a_h^k)=(s,a), I^k_{s,a,h}=l] X_{h+1}^k\nonumber
	\\ & \leq   \sum_{l=1}^{\log_2(K)} \sum_{o = 1}^{2^{l-1}}  \sum_{s,a,h} (\widehat{P}_{s,a,h}^{(l)}-P_{s,a,h}) \sum_{k=1}^K \mathbb{I}[(s_h^k,a_h^k)=(s,a), I^k_{s,a,h}=l,N_{s_h^k,a_h^k,h}^{k,\mathsf{all}} = 2^{l-1}+o] X_{h+1}^k + SAH^2\nonumber
 \\ & = \sum_{l=1}^{\log_2(K)} \sum_{o = 1}^{2^{l-1}}  \sum_{s,a,h} (\widehat{P}_{s,a,h}^{(l)}-P_{s,a,h}) X_{h+1}^{k_{l,o,s,a,h}} + SAH^2,
	\label{eq:f2}
\end{align}
%
where $k_{l,o,s,a,h}$ denotes the index of the $(2^{l-1}+o)$-th sample of $(s,a,h)$ in the online learning process. Recall that $N^{K+1,\mathsf{all}}_h(s,a)$ is the total visit count of $(s,a,h)$ in  $K$ episodes. 
If $\overline{N}^{K+1}_{h}(s,a)< 2^{l-1}+o$, we set $k_{l,o,s,a,h}=\infty$ and $X_{h+1}^{\infty}=0$.
% Note that there exists $\mathcal{J}\in \mathcal{C}$ such that $\mathcal{I}\subset \mathcal{J}$. 
Fix $2\leq l \leq \log_2(K)+1$ and $1\leq o \leq 2^{l-1}$, we can find $\{w^{(l,o)}_{s,a,h}\}_{(s,a,h)}$ be such that $x_{h+1}(w^{(l,o)}_{s,a,h})=X_{h+1}^{k_{l,o,s,a,h}}$ for any proper $(s,a,h)$. 
Using \eqref{eq:xx2}  for $(l,o)$ such that  $2\leq l \leq \log_2(K)+1$ and  $1\leq o \leq 2^{l-1}$ , with probability exceeding $1-2K\cdot 10SAH^2K^2\delta'$,
\begin{align}
&  \sum_{l=1}^{\log_2(K)} \sum_{o = 1}^{2^{l-1}}  \sum_{s,a,h} (\widehat{P}_{s,a,h}^{(l)}-P_{s,a,h}) X_{h+1}^{k_{l,o,s,a,h}}  \nonumber
\\ & \leq \sum_{l=1}^{\log_2(K)} \sum_{o = 1}^{2^{l-1}}\frac{1}{2^{l-2}}\Bigg( \sqrt{16 \cdot 2^{l-2} \sum_{s,a,h}\mathbb{V}(P_{s,a,h},X_{h+1}^{k_{l,o,s,a,h}})\cdot \bigg(SAH\log W+\log\frac{1}{\delta'}\bigg) } \nonumber
\\ & \qquad \qquad \qquad \qquad \qquad \qquad \qquad \qquad \qquad \qquad \qquad \qquad \qquad + 6H\bigg(SAH\log W+\log\frac{1}{\delta'}\bigg) \Bigg) \nonumber
\\ & \leq \sum_{l=1}^{\log_2(K)}\sqrt{32 \cdot \sum_{s,a,h}\sum_{o=1}^{2^{l-1}} \mathbb{V}(P_{s,a,h},X^{k_{l,o,s,a,h}}_{h+1}) \cdot \bigg(SAH\log W+\log\frac{1}{\delta'}\bigg) } \nonumber
\\ & \qquad\qquad \qquad \qquad \qquad \qquad \qquad \qquad \qquad \qquad \qquad \qquad  + \sum_{l=1}^{\log_2(K)} 12H\bigg(SAH\log W+\log\frac{1}{\delta'}\bigg)\nonumber
\\ & \leq \sqrt{64\log_2(K) \sum_{k=1}^K \sum_{h=1}^H \mathbb{V}(P_{s_h^k,a_h^k,h},X_{h+1}^k) \cdot (SAH\log(W)+\log(\frac{1}{\delta'})) } \nonumber
	\\ &\qquad\qquad \qquad \qquad \qquad \qquad \qquad \qquad \qquad \qquad \qquad \qquad + 12(\log_2K)H \bigg(SAH\log W+\log\frac{1}{\delta'}\bigg).\label{eq:f} 
\end{align}
In the last inequality, we have applied Cauchy's inequality and the fact that 
\begin{align}
& \sum_{k=1}^K\sum_{h=1}^H \mathbb{V}(P_{s_h^k,a_h^k,h},X_{h+1}^k) \nonumber
\\ & =
\sum_{l=1}^{\log_2(K)}\sum_{s,a,h}\sum_{o=1}^{2^{l-1}}\mathbb{V}(P_{s,a,h},X^{k_{l,o,s,a,h}}_{h+1})
 + \sum_{s,a,h}\sum_{k=1}^K \mathbb{I}\Big[(s_h^k,a_h^k)=(s,a), \overline{N}^k_{h}(s_h^k,a_h^k) = 0\Big]\mathbb{V}(P_{s,a,h},X_{h+1}^k)  .\nonumber
\end{align}


Using \eqref{eq:f2} and \eqref{eq:f}, and replacing $\delta$ with $\delta/(20SAH^2K^3)$, we finish the proof.




\subsection{Proof of Lemma~\ref{lemma:key3}}\label{app:pfkey3}


Without loss of generality, we write $\widehat{P}^{(I^k)}=\widehat{P}^k = \{\widehat{P}^k_{s,a,h}\}_{(s,a,h)} = \{\widehat{P}_{s,a,h}^{(I^k_{s,a,h})}\}_{(s,a,h)}$. Then we can regard $\mathcal{X}_{h+1}$ as a function of $\{\widehat{P}^{(I^k)}\}_{k=1}^K$, i.e., 
$\mathcal{X}_{h+1}=  \mathcal{X}_{h+1}^k(\{\widehat{P}^{(I^k)}\}_{k=1}^K, \{I^k\}_{k=1}^K  )$.
Let $\widetilde{\mathcal{E}}$ be the event where there exists $X^k_{h+1} \in \mathcal{X}_{h+1}(\{\widehat{P}^{(I^k)}\}_{k=1}^K, \{I^k\}_{k=1}^K),\forall (h,k)\in [H]\times [K]$ such that \eqref{eq:star} 
does not hold. So it suffices to prove that $\mathrm{Pr}(\widetilde{\mathcal{E}})\leq \delta$.
Fix $\mathcal{J} =\{J^k\}_{k=1}^K\in \mathcal{C}$.
We consider the event $\mathcal{E}(\mathcal{J})$ where there exists a sequence $X^k_{h+1} \in \mathcal{X}_{h+1}(\{\widehat{P}^{(J^k)}\}_{k=1}^K, \{J^k\}_{k=1}^K),\forall (h,k)\in [H]\times [K]$ such that
\begin{align}
& \sum_{k=1}^K \sum_{h=1}^H (\widehat{P}^{(J^k_{s,a,h})}_{s_h^k,a_h^k,h}-P_{s_h^k,a_h^k,h}) X^k_{h+1} \nonumber
\\ & \leq  \sqrt{L\sum_{k=1}^K \sum_{h=1}^H \mathbb{V}(P_{s_h^k,a_h^k,h},X^k_{h+1})\left(SAH\log(W) + \log(|\mathcal{C}|)+\log(1/\delta)\right)  }\nonumber
\\ & \qquad\qquad \qquad \qquad \qquad \qquad \qquad \qquad \qquad + LH(SAH\log(W) + \log(|\mathcal{C}|)+\log(1/\delta))\label{eq:star2},
\end{align}
does not hold, where $L= 200 (\log_2(K)+1)^2$. 
 Let $\widetilde{\mathcal{E}}(\mathcal{J})$ be the event where there exists a sequence $X^k_{h+1} \in \mathcal{X}_{h+1}(\{\widehat{P}^{(J^k)}\}_{k=1}^K),\forall (h,k)\in [H]\times [K]$ such that \eqref{eq:star} 
does not holds and $\mathcal{I}=\mathcal{J}$. Because $|\mathcal{C}|\leq (4SAHK))^{SAH\log_2(K)+1}$, we have $\widetilde{\mathcal{E}}(\mathcal{J})\subset \mathcal{E}(\mathcal{J})$.
With Lemma~\ref{lemma:key1}, we learn that $\mathrm{Pr}(\widetilde{\mathcal{E}}(\mathcal{J}))\leq \mathrm{Pr}(\mathcal{E}(\mathcal{J})) \leq \frac{\delta}{|\mathcal{C}|}$. As a result, we have  $\mathrm{Pr}(\cup_{\mathcal{J}\in \mathcal{C}}\widetilde{\mathcal{E}}(\mathcal{J}))\leq \delta$. By noting that $\widetilde{\mathcal{E}}\subset \cup_{\mathcal{J}\in \mathcal{C}}\widetilde{\mathcal{E}}(\mathcal{J})$, we obtain that $\mathrm{Pr}(\widetilde{\mathcal{E}})\leq \delta$. The proof is completed.



\end{comment}
