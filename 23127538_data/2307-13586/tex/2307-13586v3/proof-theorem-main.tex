

This section is devoted to proving Theorem~\ref{thm1}. 
For notational convenience, let $B$ be a logarithmic term
%
\begin{equation}
	B=4000 (\log_2 K)^3 \log(3SAH)\log\frac{1}{\delta'} 
	\label{eq:assumption-K-proof-B}, 
\end{equation}
%
where we recall that $\delta$ is the confidence parameter in Algorithm~\ref{alg:main} and $\delta' = \frac{\delta}{200SAH^2K^2}$. 
When $K\leq BSAH$, the claimed result in Theorem~\ref{thm1} holds trivially since
%
\[
\mathsf{Regret}(K)=\sum_{k=1}^{K}\left(V_{1}^{\star}(s_{1}^{k})-V_{1}^{\pi^{k}}(s_{1}^{k})\right)
\leq HK 
= \min\left\{ \sqrt{BSAH^{3}K},HK\right\}.
%=\widetilde{O}\left(\min\left\{ \sqrt{SAH^{3}K},HK\right\} \right).
\]
%
As a result, it suffices to focus on the scenario with 
%
\begin{equation}
	K\geq BSAH \qquad \text{with } B=4000 (\log_2 K)^3 \log(3SAH)\log\frac{1}{\delta'} .
	\label{eq:assumption-K-proof}
\end{equation}
% 
 
  
%Let $\pi^k$ be the policy in the $k$-th episode.  Let $\overline{N}_{h}^k(s,a)$ be the count  of $(s,a,h)$ before the $k$-th episode and $N_h^k(s,a)$ be the count of the doubling batch used to compute the value function in the $k$-th episode. In particular, when $\overline{N}_h^k(s,a)=0$, we define $N_h^k(s,a)=1$.   Let $V_h^k $ and $ Q_h^k$ be respectively the value of $ V_h$ and $Q_h$ before the $k$-th episode for all proper $(s,a,k,h)$. Recall that $\widehat{P}^k_{s,a,h}$ is the value of $\widehat{P}_{s,a,h}$ before the $k$-th episode. Let $\widehat{r}_h^k(s,a)$ be the empirical reward function before the $k$-th episode of $(s,a)$. Let $\widehat{\sigma}_h^k(s,a)$ be the empirical variance before the $k$-th episode for the state-action pair $(s,a)$, i.e., the value of $\widehat{\sigma}_h(s,a)$ before the $k$-th episode.

%For the sake of notational simplicity, we shall adopt a slightly different bonus term throughout the proof:
%
%Recall that 
%\begin{align} 
%	b_h(s,a) = c_1 \sqrt{\frac{   \mathbb{ V}(\widehat{P}_{s,a,h} ,V_{h+1}) \log \frac{1}{\delta'}  }{ \max\{N_h(s,a),1 \} }}+c_2 \sqrt{\frac{\big(\widehat{\sigma}_h(s,a)- (\widehat{r}_h(s,a))^2 \big)\log \frac{1}{\delta'}}{\max\{N_h(s,a),1\}}} 
%+c_3\frac{H\log \frac{1}{\delta'}}{ \max\{N_h(s,a) ,1\}  },  \label{eq:update1-proof}  
%\end{align}
%
%where the log term is taken to be $\log \frac{1}{\delta'}$ as opposed to $\log \frac{SAHK}{\delta}$;  the aim then becomes proving that the advertised result in Theorem~\ref{thm1} holds with probability $1-SAKH\delta$. 
%\yxc{check} 

%\subsection{Regret analysis}
%\label{sec:regret-decomp}


Our regret analysis for Algorithm~\ref{alg:main} consists of several steps described below. 

\paragraph{Step 1: the optimism principle.} 
%
To begin with, we justify that the running estimates of Q-function and value function in Algorithm~\ref{alg:main} are always upper bounds on the optimal Q-function and the optimal value function,  respectively,  
thereby guaranteeing optimism in the face of uncertainty. 
%
\begin{lemma}[Optimism]\label{lemma:opt}
With probability exceeding $1-4SAHK\delta'$, one has
%
\begin{equation}
	Q_h^k(s,a)\geq Q_h^{\star}(s,a) \qquad  \text{and}  \qquad V^k_h(s)\geq V^{\star}_h(s)
\end{equation}
%
for all $(s,a,h,k)$. 
 \end{lemma}
 %
 \begin{proof} See Appendix~\ref{sec:proof-lemma:opt}. \end{proof}




 
%\simon{let's add full details in the final version because I think this paper gonna be the standard reference for tabular MDP.}


\paragraph{Step 2: regret decomposition.}
%
In view of the optimism shown in Lemma~\ref{lemma:opt}, 
the regret can be upper bounded by 
%
\begin{align}
\mathsf{Regret}(K) & =\sum_{k=1}^{K}\big(V_{1}^{\star}(s_{1}^{k})-V_{1}^{\pi^{k}}(s_{1}^{k})\big)\leq\sum_{k=1}^{K}\big(V_{1}^{k}(s_{1}^{k})-V_{1}^{\pi^{k}}(s_{1}^{k})\big)
	\label{eq:regret-UB1}
\end{align}
%
with probability at least $1-4SAHK\delta'$. 
In order to control the right-hand side of \eqref{eq:regret-UB1}, 
we first make note of the following upper bound on $V_{1}^{k}(s_{1}^{k})$. 
%
\begin{lemma}\label{lemma:decomdetail}
For every $1\leq k\leq K$, one has
%
$$
	V_1^k(s_1^k) \leq \sum_{h=1}^{H} \left( \big\langle \widehat{P}^k_{s_h^k,a_h^k,h} - P_{s_h^k,a_h^k,h}, V_{h+1}^k \big\rangle + b_h^k(s_h^k,a_h^k) + \widehat{r}_h^k(s_h^k,a_h^k) + \big\langle P_{s_h^k,a_h^k,h}-e_{s_{h+1}^k}, V_{h+1}^k \big\rangle \right) .
$$
\end{lemma}
%
\begin{proof}[Proof of Lemma~\ref{lemma:decomdetail}]
%
From the construction of $V_h^k$ and $Q_h^k$, it is seen that, for each $1\leq h\leq H$,  
%
\begin{align}
	V_h^k(s_h^k) & = Q_h^k(s_h^k, a_h^k) \leq \widehat{r}_h^k(s_h^k,a_h^k) + \widehat{P}^k_{s_h^k,a_h^k,h}V_{h+1}^k + b_h^k(s_h^k,a_h^k) \nonumber
\\ &  =  \big\langle \widehat{P}^k_{s_h^k,a_h^k,h} - P_{s_h^k,a_h^k,h}, V_{h+1}^k \big\rangle + b_h^k(s_h^k,a_h^k) + \widehat{r}_h^k(s_h^k,a_h^k) + \big\langle P_{s_h^k,a_h^k,h}-e_{s_{h+1}^k}, V_{h+1}^k \big\rangle + V_{h+1}^k(s_{h+1}^k).\nonumber
\end{align}
%
Applying this relation recursively over $1\leq h\leq H$ gives 
%
\begin{align*}
 & V_1^k(s_1^k) \nonumber
 \\ & \leq  \sum_{h=1}^{H} \left( \big\langle \widehat{P}^k_{s_h^k,a_h^k,h} - P_{s_h^k,a_h^k,h}, V_{h+1}^k \big\rangle + b_h^k(s_h^k,a_h^k) 
	+ \widehat{r}_h^k(s_h^k,a_h^k) + \big\langle P_{s_h^k,a_h^k,h}-e_{s_{h+1}^k}, V_{h+1}^k \big\rangle \right) + V_{H+1}^k(s_{H+1}^k),
\end{align*}
%
which combined with $V_{H+1}^k=0$ concludes the proof. 
\end{proof}


Combine Lemma~\ref{lemma:decomdetail} with \eqref{eq:regret-UB1} to show that, with probability at least $1-4SAHK\delta'$, 
%
\begin{align}
\mathsf{Regret}(K) & \leq\underset{\eqqcolon\,T_{1}}{\underbrace{\sum_{k=1}^{K}\sum_{h=1}^{H}\big\langle\widehat{P}_{s_{h}^{k},a_{h}^{k},h}^{k}-P_{s_{h}^{k},a_{h}^{k},h},V_{h+1}^{k}\big\rangle}}+\underset{\eqqcolon\,T_{2}}{\underbrace{\sum_{k=1}^{K}\sum_{h=1}^{H}b_{h}^{k}(s_{h}^{k},a_{h}^{k})}}\nonumber\\
 & \quad+\underset{\eqqcolon\,T_{3}}{\underbrace{\sum_{k=1}^{K}\sum_{h=1}^{H}\big\langle P_{s_{h}^{k},a_{h}^{k},h}-e_{s_{h+1}^{k}},V_{h+1}^{k}\big\rangle}}+\underset{\eqqcolon\,T_{4}}{\underbrace{\sum_{k=1}^{K}\left(\sum_{h=1}^{H}\widehat{r}_{h}^{k}(s_{h}^{k},a_{h}^{k})-V_{1}^{\pi^{k}}(s_{1}^{k})\right)}}, 
	\label{eq:decomposition}
\end{align}
%
leaving us with four terms to control. 
In particular, $T_1$ has already been upper bounded in Section~\ref{sec:tec1}, and hence we shall describe how to bound $T_2,\ldots,T_4$ in the sequel.    


%where $b_h^k(s_h^k,a_h^k): = c_1\sqrt{\frac{\mathbb{V}(  \widehat{P}^k_{s_h^k,a_h^k,h},V_{h+1}^k)\log(\frac{1}{\delta'})}{N^k_{h}(s_h^k,a_h^k) }} +c_2 \sqrt{\frac{\left( \widehat{\sigma}_h^k(s,a)- (\widehat{r}_h^k(s,a))^2 \right)\log(\frac{1}{\delta'})}{N_h^k(s_h^k,a_h^k)}}+ c_3\frac{H\log(\frac{1}{\delta'})}{N^k_{h}(s_h^k,a_h^k)}$.



% Recall the definition that $\pi^k_h(s) = \arg\max_a Q_h^k(s,a)$. With probability $1-2SAHK\delta$, 




% Define $T_1 =\sum_{k=1}^K \sum_{h=1}^H \left(    (\widehat{P}^k_{s_h^k,a_h^k,h}   - P_{s_h^k,a_h^k,h})V_{h+1}^k \right) $, $T_2 = \sum_{k=1}^K \sum_{h=1}^H  b_h^k(s_h^k,a_h^k)$, $T_3 =\sum_{k=1}^K\sum_{h=1}^H ( P_{s_h^k,a_h^k,h}-\mathbf{1}_{s_{h+1}^k})V_{h+1}^k $ and $T_4 = \sum_{k=1}^K \left(\sum_{h=1}^H \widehat{r}^k_{h}(s^k_h,a^k_h)- V_1^{\pi^k}(s_1^k) \right) $.




\paragraph{Step 3.1: bounding the terms $T_2,T_3$ and $T_4$.}
%
In this section, we seek to bound the terms  $T_2,T_3$ and $T_4$ defined in the regret decomposition \eqref{eq:decomposition}.  
To do so, we find it helpful to first introduce the following quantities that capture some sort of aggregate variances: 
%
\begin{subequations}
\label{eq:defn-T56-proof}
\begin{align}
	T_5 &\coloneqq \sum_{k=1}^K\sum_{h=1}^H\mathbb{V}\big(\widehat{P}^k_{s_h^k,a_h^k,h},V_{h+1}^k \big),
	\label{eq:defn-T5-proof} \\
	T_6 &\coloneqq \sum_{k=1}^K \sum_{h=1}^H\mathbb{V} \big(P_{s_h^k,a_h^k,h},V_{h+1}^k \big) ,
	\label{eq:defn-T6-proof}
\end{align}
\end{subequations}
%
with $T_5$ denoting certain empirical variance and $T_6$ the true variance. 
With these quantities in place, we claim that the following bounds hold true. 
%
\begin{lemma}\label{lem:bound-T234}
%
With probability exceeding $1-15SAH^2K^2\delta'$, one has 
%\begin{itemize}
%	\item Regarding the non-negative term $T_2$, we have the following connection between $T_2$ and $T_5$: 
%
\begin{subequations}
\label{eq:boundt234}
\begin{align}
	T_2 &\leq 61\sqrt{2SAH(\log_2 K)\Big(\log\frac{1}{\delta'}\Big)T_5} +  8\sqrt{SAH^3K(\log_2 K)\log\frac{1}{\delta'}}+
	151 SAH^2(\log_2K)\log\frac{1}{\delta'},
	\label{eq:boundt2} \\
	|T_3|  &\leq \sqrt{ 8 T_6 \log \frac{1}{\delta'}  } + 3H\log \frac{1}{\delta'}, \label{eq:boundt3} \\
	|T_4| &\leq 6\sqrt{2SAH^3K(\log_2K)\log \frac{1}{\delta'} } +  55SAH^2(\log_2 K)\log \frac{1}{\delta'}.\label{eq:bdt_4f}
\end{align}
\end{subequations}
%
%with probability exceeding $1-2SAHK\delta'$. 
%
%	\item With regards to the term $T_3$, with probability at least $1-10SAH^2K^2\delta'$ one has
%
%\begin{align}
% |T_3|   \leq 2\sqrt{2}\cdot \sqrt{  T_6 \log \frac{1}{\delta'}  } + 3H\log \frac{1}{\delta'}. \label{eq:boundt3}
%\end{align}
%
%
%	\item When it comes to the term $T_4$, with probability at least $1-3SAHK\delta'$ we have
%
%\begin{equation}
%	|T_4| \leq 6\sqrt{2SAH^3K(\log_2K)\log \frac{1}{\delta'} } +  55SAH^2(\log_2 K)\log \frac{1}{\delta'}.\label{eq:bdt_4f}
%\end{equation}
%
%\end{itemize}
%
\end{lemma}
%
\begin{proof} See Appendix~\ref{app:pflem:bound-T234}. \end{proof}



\paragraph{Step 3.2: bounding the aggregate variances $T_5$ and $T_6$.} 
%
The previous bounds on $T_2$ and $T_3$ stated in Lemma~\ref{lem:bound-T234} depend respectively on the aggregate variance $T_5$ and $T_6$  (cf.~\eqref{eq:defn-T5-proof} and \eqref{eq:defn-T6-proof}), 
which we would like to control now. 
By introducing the following quantities: 
%
\begin{subequations}
\label{eq:defn-T789-proof}
\begin{align}
T_{7} & \coloneqq\sum_{k=1}^{K}\sum_{h=1}^{H}\Big\langle\widehat{P}_{s_{h}^{k},a_{h}^{k},h}^{k}-P_{s_{h}^{k},a_{h}^{k},h},\big(V_{h+1}^{k}\big)^{2}\Big\rangle,\label{eq:defn-T7-proof}\\
T_{8} & \coloneqq\sum_{k=1}^{K}\sum_{h=1}^{H}\Big\langle P_{s_{h}^{k},a_{h}^{k},h}-e_{s_{h+1}^{k}},\big(V_{h+1}^{k}\big)^{2}\Big\rangle,\label{eq:defn-T8-proof}\\
T_9 & \coloneqq \sum_{k=1}^{K}\sum_{h=1}^{H}\max\Big\{\Big\langle\widehat{P}_{s_{h}^{k},a_{h}^{k},h}^{k}-P_{s_{h}^{k},a_{h}^{k},h},V_{h+1}^{k}\Big\rangle,0\Big\}, 
\label{eq:defn-T9-proof}
\end{align}
\end{subequations}
%
we can upper bound $T_5$ and $T_6$ through the following lemma.  
%
\begin{lemma}
\label{lem:bound-T56}
With probability at least $1-4SAHK\delta'$,
%
\begin{subequations}
\label{eq:boundt56}
\begin{align}
T_{5} & \leq T_{7}+T_8+ 2HT_{2}+6KH^{2},
	%+\sqrt{32H^{2}T_{6}\log\frac{1}{\delta'}}+3H^{2}\log\frac{1}{\delta'} ,
\label{eq:boundt5} \\
T_{6} & \leq 2HT_{2}+6KH^{2}
	+\sqrt{32H^{2}T_{6}\log\frac{1}{\delta'}}+3H^{2}\log\frac{1}{\delta'}
	+2HT_{9}, 
\label{eq:boundt6} \\
|T_{8}|&
	%\leq2\sqrt{2}\sqrt{\sum_{k,h}\mathbb{V}\Big(\widehat{P}_{s_{h}^{k},a_{h}^{k},h}^{k},\big(V_{h+1}^{k}\big)^{2}\Big)\log\frac{1}{\delta'}}+3H^{2}\log\frac{1}{\delta'}
	\leq \sqrt{32H^{2}T_{6}\log\frac{1}{\delta'}}+3H^{2}\log\frac{1}{\delta'}	 . 
	\label{eq:boundt8}
\end{align}
\end{subequations}
%
\end{lemma}
%
\begin{proof} See Appendix~\ref{sec:pflem:bound-T56}. \end{proof}  






\iffalse

\subsection{Bounding the terms $T_1,T_7$ and $T_9$}
In this section, we will prove a key lemma to deal with the error terms $T_1,T_7$ and $T_9$. This lemma controls the error term $\sum_{k,h}(\widehat{P}_{s_h^k,a_h^k,h}-P_{s_h^k,a_h^k,h}) X_{h}^k$ under some mild conditions, where we do not require $\{X_{h}^k\}_{h,k}$ is conditionally independent of $\{\widehat{P}^k_h\}_{h,k}$. By this lemma, we can control the error terms above by letting $\mathcal{X}_h^k = \{ V_{h+1}^k, (V_{h+1}^k)^2/H, \textbf{0}\}$. % This consititutes our key novelty

%\simon{add intuitions what $\mathcal{X}_h^k$ will be like later, and we why need this lemma}


\begin{lemma}\label{lemma:key}
For each $k,h$,  let a set of $S$-dimensional vectors  $\mathcal{X}_h^k$  be a function of $\{\widehat{P}^k_{s,a,h'}\}_{h'\geq h+1,s,a}$ such that $\|X\|_{\infty}\leq H, \forall X\in \mathcal{X}_h^k$. Let $L = \max_{h,k}|\mathcal{X}_h^k|$. Then with probability $1- 10\delta$, for any sequence $\{X_h^k\}$ such that  $X_h^k\in \mathcal{X}_h^k$, it holds that
\begin{align}
 &  \sum_{k=1}^K \sum_{h=1}^H \left( \widehat{P}_{s_h^k,a_h^k,h}-P_{s_h^k,a_h^k,h} \right) X_h^k  \nonumber
 \\ & \leq  \sqrt{32\log_2(K) \sum_{k=1}^K \sum_{h=1}^H \mathbb{V}(P_{s_h^k,a_h^k,h},X_h^k) \cdot (4SAH\log_2(K)\log(SALH)+\log(\frac{1}{\delta'})) } + 12\log_2(K)H (4SAH\log_2(K)\log(SALH)+\log(\frac{1}{\delta'})).\label{eq:key}
\end{align}
\end{lemma}
\begin{proof}




 Recall the definition of $i^k_{s,a,h}$. Define a \emph{profile} as $\{i_{s,a,h}\}_{s,a,h}$ with $i_{s,a,h}\in [\log_2(K)]$. We say two profiles $i\leq j$ iff $i_{s,a,h}\leq j_{s,a,h}$ for any $(s,a,h)$.
 
 Let $\mathcal{I} =\left\{  \{i^k_{s,a,h}\}_{s,a,h}     | k = 1,2,\ldots,K  \right\}$.  Since there are at most $SAH(\log_2(K)+1)$ updates, the size of  $\mathcal{I}$ is at most $SAH(\log_2(K)+1)$.  

\simon{this paragraph might be the most important technical paragraph. We can add more details and a plot.}
Let $M=SAH(\log_2(K)+1)$ and $\mathcal{C}:= \{  \{j^1,j^2,\ldots,j^{M}\} |j_l\in [\log_2(K)]^{SAH}, j^l\neq j^{l'} \forall l\neq l', j^l\leq j^{l+1}, \forall 1\leq l\leq M-1\}$. In words, $\mathcal{C}$ is the set of an increasing path in the set $[\log_2(K)]^{SAH}$ from $[0,0,\ldots,0]^{\top}$ to $[\log_2(K),\log_2(K),\ldots,\log_2(K)]^{\top}$. Since there are at most $M$ steps and in each step we have at most $SAH$ choices, the size of $\mathcal{C}$ is at most $(SAH)^{M}$. Let $\overline{\mathcal{C}} = \{  \{j^1,j^2,j^3,\ldots,j^q\}| j^l<j^{l+1},1\leq l \leq q-1, q\leq M \}$. Then the size of $\overline{\mathcal{C}}$ is at most $2^{M}\cdot |\mathcal{C}|=(2SAH)^M$. 
\simon{without using the increasing path property, we will have size $(\log_2K)^{SAHM}$.}

Recalling the definition of $\mathcal{I}$, we always have that $\mathcal{I}\in \overline{\mathcal{C}}$.
%Because $i^1\leq i^2\leq \ldots,\leq i^K$, $\mathcal{I}$ is determined by an increasing path in $[\log_2(K)]^{SAH}$ (it is possible that $\mathcal{I}$ does not end with $[\log_2(K),\log_2(K),\ldots, \log_2(K)]^{\top}$). Then we can always find some $\mathcal{J}\in \mathcal{C}$ such that $\mathcal{I}\subset \mathcal{J}$.

% Now we fix $\mathcal{J}\in \overline{\mathcal{C}}$. Let $\delta'\in (0,1)$ be a confidence parameter and $\log(\frac{1}{\delta'})' = \log(2/\delta')$. Assume $\mathcal{J} =\{j^1,j^2,\ldots, j^M\}$ such that $j^1\leq j^2\leq \cdots \leq j^M$.  For $j\in [\log_2(K)]^{SAH}$, we use $\widehat{P}(j)$ to denote the tuple $\{ \widehat{P}_{s,a,h}(j_{s,a,h})  \}_{s,a,h}$ where $\widehat{P}_{s,a,h}^{(l)}$ is the empirical transition model computed using the $l$-th batch of $(s,a,h)$ for $1\leq l \leq \log_2(K)$, and $\widehat{P}_{s,a,h}^{(l)} = \frac{1}{S}\cdot \textbf{1}$ if $l=0$.  We also define $\widehat{P}_h(j)=\{ \widehat{P}_{s,a,h'}(j_{s,a,h})  \}_{h'\geq h+1,s,a}$.%\simon{what is $j^k$?}


\end{proof}

As a result, with probability $1-\delta$ it holds that 
\begin{align}
 &  \sum_{k=1}^K \sum_{h=1}^H \left( \widehat{P}_{s_h^k,a_h^k,h}-P_{s_h^k,a_h^k,h} \right) X_h^k  \nonumber
 \\ & \leq  \sqrt{32\log_2(K) \sum_{k=1}^K \sum_{h=1}^H \mathbb{V}(P_{s_h^k,a_h^k,h},X_h^k) \cdot (4SAH\log_2(K)\log(SALH)+\log(\frac{1}{\delta'})) } + 12\log_2(K)H (4SAH\log_2(K)\log(SALH)+\log(\frac{1}{\delta'})).\nonumber
\end{align}
The proof is completed.

\fi 





\paragraph{Step 3.3: bounding the terms $T_1$, $T_7$ and $T_9$.} 
%
Taking a look at the above bounds on $T_2,\ldots,T_6$, 
we see that one still needs to deal with the terms $T_1$, $T_7$ and $T_9$ (see \eqref{eq:decomposition}, \eqref{eq:defn-T7-proof} and \eqref{eq:defn-T9-proof}, respectively). 
As it turns out, these quantities have already been bounded in Section~\ref{sec:tec}. 
Specifically, Lemma~\ref{lemma:decouple} tells us that: with probability at least $1-\delta'$, 
%
%\begin{align*} 
%	T_1 &\leq T_9 
%	%\sum_{k=1}^{K}\sum_{h=1}^{H}\Big\langle\widehat{P}_{s_{h}^{k},a_{h}^{k},h}^{k}-P_{s_{h}^{k},a_{h}^{k},h},V_{h+1}^{k}\Big\rangle
%	%\leq \sum_{k=1}^{K}\sum_{h=1}^{H}\max\Big\{\Big\langle\widehat{P}_{s_{h}^{k},a_{h}^{k},h}^{k}-P_{s_{h}^{k},a_{h}^{k},h},V_{h+1}^{k}\Big\rangle,0\Big\} \\
%  \leq \sqrt{16(\log_{2}K)\sum_{k=1}^{K}\sum_{h=1}^{H}\mathbb{V}\big(P_{s_{h}^{k},a_{h}^{k},h},V_{h+1}^{k}\big)\left(6SAH\log_{2}^{2}K+\log\frac{1}{\delta'}\right)}+49SAH^{2}\log_{2}^{3}K+8H(\log_{2}K)\log\frac{1}{\delta'} ;\\
%%\end{align*}
%%
%%and
%%
%%\begin{align*} 
%	%& \sum_{k=1}^{K}\sum_{h=1}^{H}\Big\langle\widehat{P}_{s_{h}^{k},a_{h}^{k},h}^{k}-P_{s_{h}^{k},a_{h}^{k},h},\big(V_{h+1}^{k}\big)^{2}\Big\rangle\\
%	T_7 
% & \leq8H\sqrt{(\log_{2}K)\sum_{k=1}^{K}\sum_{h=1}^{H}\mathbb{V}\big(P_{s_{h}^{k},a_{h}^{k},h},V_{h+1}^{k}\big)\left(6SAH\log_{2}^{2}K+\log\frac{1}{\delta'}\right)}+49SAH^{3}\log_{2}^{3}K+8H^2(\log_{2}K)\log\frac{1}{\delta'}.
%\end{align*}
% 
\begin{subequations}
\label{eq:boundt179}
\begin{align}
	T_1\leq T_9&\leq \sqrt{ B SAH \sum_{k=1}^K \sum_{h=1}^H \mathbb{V}(P_{s_h^k,a_h^k,h},V_{h+1}^k)}+BSAH^2 = \sqrt{BSAH T_6}+BSAH^2,\label{eq:boundt1}
 \\ 
	T_7 &\leq H \sqrt{ BSAH  \sum_{k=1}^K \sum_{h=1}^H \mathbb{V}(P_{s_h^k,a_h^k,h}, V_{h+1}^k) }    + BSAH^3 = H\sqrt{BSAH T_6}+BSAH^3 ,\label{eq:boundt7}
 %\\ & T_9 \leq \sqrt{B SAH \sum_{k=1}^K \sum_{h=1}^H \mathbb{V}(P_{s_h^k,a_h^k,h},V_{h+1}^k)}+BSAH^2= \sqrt{BSAH T_6}+BSAH^2.\label{eq:boundt9}
\end{align}
\end{subequations}
%
where we recall that $B=4000(\log_2K)^3\log(3SAH)\log\frac{1}{\delta'} $. 


%for which we remind the reader that
%
%\begin{align}
%	T_1 &= \sum_{k=1}^K \sum_{h=1}^H  \Big\langle \widehat{P}^k_{s_h^k,a_h^k,h}  -P_{s_h^k,a_h^k,h} , V_{h+1}^k \Big\rangle
%;\nonumber
%	\\  T_7  &= \sum_{k=1}^K \sum_{h=1}^H  \Big\langle \widehat{P}^k_{s_h^k,a_h^k,h}  -P_{s_h^k,a_h^k,h} , (V_{h+1}^k)^2\Big\rangle;\nonumber
%	\\  T_9 &=  \sum_{k=1}^{K}\sum_{h=1}^{H}\max\Big\{\Big\langle\widehat{P}_{s_{h}^{k},a_{h}^{k},h}^{k}-P_{s_{h}^{k},a_{h}^{k},h},V_{h+1}^{k}\Big\rangle,0\Big\}.\nonumber
%\end{align}
%
%Notably, all these quantities involve weighted sums of $\widehat{P}^k_{s_h^k,a_h^k,h}$. 



%Recall $B=4000\log^2_3(K)\log(3SAH)\log(\frac{1}{\delta'}) $. By the update rule \eqref{eq:updateq}, $V_{h+1}^k$ is determined by the $\{ \widehat{P}^k_{s,a,h'} \}_{(h+1\leq h'\leq H ,s,a)}$ and $\{I^k_{s,a,h'}\}_{h+1\leq h'\leq H,s,a}$. 
%Using Lemma~\ref{lemma:key3} with $\mathcal{X}_{h+1} =\{V_{h+1}^k\}_{k=1}^K \cup\{0\}, \{(V_{h+1}^k)^2 /H\}_{k=1}^K\cup\{0\}$ and $\{ V_{h+1}^k\}_{k=1}^K \cup\{0\}$, and noting that $\mathsf{Var}(X^2)\leq 4\|X\|^2_{\infty}\mathsf{Var}(X)$ (see Lemma~\ref{lemma:sqv}), we have 
%
%\begin{align}
% & T_1\leq \sqrt{ B SAH \sum_{k=1}^K \sum_{h=1}^H \mathbb{V}(P_{s_h^k,a_h^k,h},V_{h+1}^k)}+BSAH^2 = \sqrt{BSAH T_6}+BSAH^2;\label{eq:boundt1}
% \\ & T_7 \leq H \sqrt{ 4BSAH  \sum_{k=1}^K \sum_{h=1}^H \mathbb{V}(P_{s_h^k,a_h^k,h}, V_{h+1}^k) }    + 4BSAH^3 = H\sqrt{4BSAH T_6}+4BSAH^3 ;\label{eq:boundt7}
% \\ & T_9 \leq \sqrt{B SAH \sum_{k=1}^K \sum_{h=1}^H \mathbb{V}(P_{s_h^k,a_h^k,h},V_{h+1}^k)}+BSAH^2= \sqrt{BSAH T_6}+BSAH^2.\label{eq:boundt9}
%\end{align}
%
 

\paragraph{Step 4: putting all pieces together.}
%
The previous bounds \eqref{eq:boundt234}, \eqref{eq:boundt56} and \eqref{eq:boundt179} indicate that: 
%and \eqref{eq:boundt9} 
with probability at least $1-100SAH^2K^2\delta'$, one has
%
\begin{subequations}
\label{eq:all-bounds-summary}
\begin{align}
	T_2 &\leq  \sqrt{B SAHT_5} +  \sqrt{BSAH^3K}+ BSAH^2,\label{eq:obt2}
	\\  T_3 &\leq \sqrt{BT_6}+HB  ,\label{eq:obt3}
	\\  T_4 &\leq \sqrt{ BSAH^3K}+BSAH^2,\label{eq:obt4}
	\\  T_5 &\leq T_7 + T_8 + 2H T_2 + 6KH^2 ,\label{eq:obt5}
	\\  T_6 &\leq \sqrt{B H^2T_6} + 2HT_2 + 2HT_9 + BH^2 + 6KH^2,\label{eq:obt6}
	\\  T_8 &\leq \sqrt{BH^2T_6 } + BH^2 ,\label{eq:obt8}
	\\  T_1 &\leq \sqrt{BSAHT_6} + BSAH^2,\label{eq:obt1}
	\\  T_7 &\leq H\sqrt{BSAHT_6} + BSAH^3,\label{eq:obt7}
	\\  T_9 &\leq \sqrt{BSAHT_6}+BSAH^2,\label{eq:obt9}
\end{align}
\end{subequations}
%
where we again use $B=4000(\log_2K)^3\log(3SAH)\log\frac{1}{\delta'} $.   


 To solve the inequalities \eqref{eq:all-bounds-summary}, we resort to the elementary AM-GM inequality: if $a\leq \sqrt{bc}+d$ for some $b,c\geq 0$, then it follows that $a \leq \epsilon b + \frac{1}{2\epsilon}c +d$ for any $\epsilon>0$. This basic inequality combined with  \eqref{eq:all-bounds-summary} gives
 %
 \begin{align}
	 HT_2 &\leq \epsilon T_5 + \left(\frac{1}{2\epsilon}+1\right) BSAH^3+ \frac{3}{2}BSAH^3 + \frac{1}{2} KH^2 ,\nonumber
	 \\  T_6 &\leq \epsilon T_6 + 2HT_2 + 2HT_9 + \left(1+\frac{1}{2\epsilon}\right) BH^2 + 6KH^2,\nonumber
	 \\  HT_9 &\leq \epsilon T_6 +\left( \frac{1}{2\epsilon}+1\right)BSAH^3,\nonumber
	 \\  T_8 &\leq \epsilon T_6 + \left( \frac{1}{2\epsilon}+1\right)BH^2,\nonumber
	 \\  T_7 &\leq  \epsilon T_6 + \left(\frac{1}{2\epsilon} +1\right)BSAH^3,\nonumber
 \end{align}
 %
which in turn result in
%
\begin{align}
	 T_5 &\leq T_7+T_8 + 2HT_2 + 6KH^2\leq 2\epsilon T_5+2\epsilon T_6 +  \left( \frac{1}{\epsilon}+2\right) BSAH^3+6KH^2;\nonumber
\\  T_{6} & \leq\epsilon T_{6}+2HT_{2}+2HT_{9}+\left(1+\frac{1}{2\epsilon}\right)BH^{2}+6KH^{2} 
	\leq3\epsilon T_{6}+2\epsilon T_{5}+\left(\frac{3}{\epsilon}+8\right)BSAH^{3}+7KH^{2}. \notag
\end{align}
%
By taking $\epsilon= 1/20$, we arrive at 
%
\begin{align}
	T_5+T_6\lesssim BSAH^3+KH^2 \asymp KH^2,
\end{align}
%
where the last relation holds due to our assumption $K\geq SAHB$ (cf.~\eqref{eq:assumption-K-proof}).  
%We then have that $T_5,T_6 = O(KH^2 + BSAH^3) = O(KH^2)$. 
Substituting this into \eqref{eq:all-bounds-summary}  yields
%
\begin{align}
	T_1 \lesssim \sqrt{BSAH^3K},
	%\quad  T_7+ T_8 \lesssim \sqrt{BSAH^5K}, 
	\quad T_2 \lesssim \sqrt{BSAH^3K}, \quad  T_3 \lesssim \sqrt{BKH^2}
	\quad \text{and} \quad
	T_4 &\lesssim \sqrt{ BSAH^3K},
\end{align}
%
provided that $K\geq SAHB$. 
These bounds taken collectively with \eqref{eq:decomposition} readily give
%
\[
	\mathsf{Regret}(K) \lesssim \sqrt{ BSAH^3K} . 
	%\leq \widetilde{O}\left(\sqrt{SAH^3K\log \frac{1}{\delta'}} \right).
\]
%
 



Combining the two scenarios (i.e., $K\geq BSAH$ and $K\leq BSAH$) reveals that with probability at least $1-100SAH^2K^2\delta'$, 
%
\[
	\mathsf{Regret}(K) \lesssim \min\big\{ \sqrt{ BSAH^3K} , HK \big\}
	\lesssim  
	\min\bigg\{ 
	\sqrt{ BSAH^3K \log^5 \frac{SAHK}{\delta'}} , HK \bigg\}.
	%\leq \widetilde{O}\left(\sqrt{SAH^3K\log \frac{1}{\delta'}} \right).
\]
%
The proof of Theorem~\ref{thm1} is thus completed by recalling that $\delta' = \frac{\delta}{200SAH^2K^2}$.

