\documentclass[english,10pt]{article}




\usepackage[latin9]{inputenc} 
\usepackage[T1]{fontenc}    % use 8-bit T1 fonts
\usepackage{lmodern}
\usepackage{geometry}
\geometry{margin=1in}
%\usepackage{times}

\usepackage[unicode=true,
 bookmarks=false,
 breaklinks=false,pdfborder={0 0 1},colorlinks=false]
 {hyperref}
\hypersetup{
 colorlinks,citecolor=blue,filecolor=blue,linkcolor=blue,urlcolor=blue}%       % simple URL typesetting
\usepackage{booktabs}       % professional-quality tables
\usepackage{xcolor}
\usepackage{cite} 
\usepackage{bm}
\usepackage{amsmath,amsthm,amssymb,mathtools}
\usepackage{amssymb}
\usepackage{nicefrac}       % compact symbols for 1/2, etc.
\usepackage{microtype}      % microtypography
\usepackage{appendix}
\usepackage{enumerate}
\usepackage[linesnumbered,ruled,vlined]{algorithm2e}
\usepackage{algorithmic} 
\usepackage{xspace}
\usepackage{graphicx}
\usepackage{tabularx, makecell, booktabs}
\usepackage{colortbl}
\usepackage{tablefootnote}
\usepackage{natbib}
\usepackage{verbatim}
\usepackage{adjustbox}
\usepackage{nicematrix}
\usepackage{multirow}
\usepackage{dsfont}



\DeclareFontFamily{U}{mathx}{}
\DeclareFontShape{U}{mathx}{m}{n}{<-> mathx10}{}
\DeclareSymbolFont{mathx}{U}{mathx}{m}{n}
\DeclareMathAccent{\widecheck}{0}{mathx}{"71}



\allowdisplaybreaks

\setlength{\parskip}{0.2\baselineskip}
\newcommand{\mymid}{\,|\,}

\newtheorem{theorem}{\textbf{Theorem}}
\newtheorem{fact}[theorem]{\textbf{Fact}}
\newtheorem{claim}[theorem]{\textbf{Claim}}
\newtheorem{lemma}[theorem]{\textbf{Lemma}}
\newtheorem{proposition}[theorem]{\textbf{Proposition}}
%\newtheorem{remark}{Remark}
\newtheorem{corollary}{\textbf{Corollary}}

\theoremstyle{theorem}\newtheorem{remark}{\textbf{Remark}}
\theoremstyle{theorem}\newtheorem{definition}{\textbf{Definition}}
\theoremstyle{theorem}\newtheorem{assumption}{\textbf{Assumption}}



\newcommand{\simon}[1]{\textcolor{cyan}{[Simon: #1]}}


\newcommand{\zihan}[1]{\textcolor{red}{[Zihan: #1]}}

\definecolor{yxc}{RGB}{255,0,0}
\newcommand{\yxc}[1]{\textcolor{yxc}{[YXC: #1]}}



\title{Settling the Sample Complexity of Online Reinforcement Learning\footnotetext{Accepted for presentation at the
Conference on Learning Theory (COLT) 2024.}}
% Optimal  Reinforcement Learning without the Burn-in Cost: 
% by Decoupling  Transition and Value Estimations

\author{%
	Zihan Zhang\thanks{Department of Electrical and Computer Engineering, Princeton University; email: \texttt{\{zz5478,jasonlee\}@princeton.edu}.}\\
  Princeton  \and
 Yuxin Chen\thanks{Department of Statistics and Data Science, University of Pennsylvania; email: \texttt{yuxinc@wharton.upenn.edu}.}\\
 UPenn
 \and
Jason D.~Lee\footnotemark[1] \\
 Princeton 
 \and
 Simon S.~Du\thanks{Paul G. Allen School of Computer Science and Engineering, University of Washington; email: \texttt{ssdu@cs.washington.edu}.}\\
 U.~Washington
}
\date{July 2023; ~~Revised: May 2024}

%\def\usebigfont{1}
\ifdefined\usebigfont
\renewcommand{\mddefault}{bx}
\renewcommand\seriesdefault{b}
\renewcommand\bfdefault{bx} % just to make sure
\usepackage{times}
\usepackage[fontsize=13pt]{scrextend}
\makeatletter
\@ifpackageloaded{geometry}{\AtBeginDocument{\newgeometry{letterpaper,left=1.56in,right=1.56in,top=1.71in,bottom=1.77in}}}{\usepackage[letterpaper,left=1.56in,right=1.56in,top=1.71in,bottom=1.77in]{geometry}}
\AtBeginDocument{\newgeometry{letterpaper,left=1.56in,right=1.56in,top=1.71in,bottom=1.77in}}
\makeatother
\else
\fi
\begin{document}
\maketitle

\begin{abstract}
    \begin{abstract}

The Fast Reciprocal Square Root Algorithm is a well-established approximation technique consisting of two stages: first, a coarse approximation is obtained by manipulating the bit pattern of the floating point argument using integer instructions, and second, the coarse result is refined through one or more steps, traditionally using Newtonian iteration but alternatively using improved expressions with carefully chosen numerical constants found by other authors. The algorithm was widely used before microprocessors carried built-in hardware support for computing reciprocal square roots. At the time of writing, however, there is in general no hardware acceleration for computing other fixed fractional powers. This paper generalises the algorithm to cater to all rational powers, and to support any polynomial degree(s) in the refinement step(s), and under the assumption of unlimited floating point precision provides a procedure which automatically constructs provably optimal constants in all of these cases. It is also shown that, under certain assumptions, the use of monic refinement polynomials yields results which are much better placed with respect to the cost/accuracy tradeoff than those obtained using general polynomials. Further extensions are also analysed, and several new best approximations are given.

\end{abstract}

\end{abstract}

\setcounter{tocdepth}{2}
\tableofcontents

\section{Introduction}\label{sec:intro}
% Figure environment removed

\section{Introduction}
Automatic 3D reconstruction of clothed humans using image inputs has gained increasing significance due to its potential applications in a wide array of AR/VR scenarios. High-fidelity reconstructions typically depend on sophisticated capture systems, which are developed with dense camera arrays~\cite{collet2015high,joo2015panoptic,joo2018total}, programmable light-stages~\cite{Vlasic2009, guo2019relightables}, and depth sensors~\cite{newcombe2011kinectfusion,DoubleFusion,BodyFusion,dou2016fusion4d,newcombe2015dynamicfusion}. However, stringent capture environments equipped with complex hardware pose significant challenges for consumer-level applications.


In this context, considerable research effort has been dedicated to developing methods that allow for more flexible capture configurations, such as utilizing a few RGB inputs. Among these works, learning implicit functions \cite{iccv2020PIFu, saito2020pifuhd, hong2021stereopifu} has proven effective in achieving highly detailed reconstructions by integrating the advancements of deep neural networks. These methods employ large multi-layer perceptrons (MLPs) to predict the occupancy probability or truncated signed distance function (TSDF) value of every queried 3D point based on its associated local feature, which is extracted from images. They can recover a continuous surface at arbitrary resolutions without topology restrictions.


However, in typical MLP-based implicit networks, the occupancy or TSDF value at each location is solved independently with planar image features, rendering them less capable of addressing challenging cases such as occlusions. Consequently, these methods suffer from generalization and robustness issues, particularly when tackling strong occlusions caused by large motion or multiple interacting humans. 
Some follow-up studies  \cite{zheng2021deepmulticap,zheng2021pamir,huang2020arch} utilize an extra geometric model, SMPL~\cite{Loper2015}, to improve robustness by introducing strong shape priors. 
Their success typically relies on the assumption of geometrical similarity \cite{huang2020arch} between the shape prior and target reconstruction, making them intractable for handling complex cases with loose clothes and sensitive to errors in SMPL model fitting.



%\ping{this paragraph sounds like `TSDF is better than MLP/SMPL, and we use TSDF to solve the problem'. But in Sec 3, we are telling a different story, saying `MLP needs a 3D convolutional encoder'. We need to make these two sections consistent.}\sicong{I think in this paragraph we claim that the TSDF}


%We opt for Trucated Signed Distance Funtion (TSDF) volumetric representations as they are naturally suitable for convolution operations, which have shown remarkable performance for learning hierarchical features on 2D visual perception tasks \cite{SunXLW19}. 
%Meanwhile, TSDF also describes the gradual geometry change around shape surface, which is not reflected by occupancy volume. 

We instead revisit the 3D volumetric representation and resort to 3D convolutional neural networks (CNNs) for feature learning, due to their impressive performance in feature learning and the ability to incorporate spatial context. However, volumetric methods and 3D convolution involve discretization, which might raise concerns regarding whether a discretized volume can preserve subtle geometric details as continuous representations learned in implicit functions. We investigate the relationship between volume resolution and quantization error on synthetic data by converting target mesh objects to TSDF volumes, as shown in Figure~\ref{fig:quantization_error}. We observe that the quantization errors are significantly reduced by increasing volume resolution and become nearly negligible when reaching a relatively high resolution (e.g., 512 or higher). In other words, achieving fine-detailed reconstruction is not supposed to be restricted by the use of volume representations as long as a proper volume resolution is utilized. Therefore, we present a method with high-resolution feature volumes, e.g., 256 and 512, while traditional volumetric methods \cite{varol18_bodynet,gilbert2018volumetric} are often limited to much lower resolutions, such as 32 or 128.



On the other hand, an increase in volume resolution may lead to a cubic growth of memory overhead \cite{8100085}. Reducing memory costs while guaranteeing the granularity of volumetric representations is necessary for pursuing high-quality reconstruction. Thus, we adopt a coarse-to-fine approach and cull away irrelevant voxels to build a sparse high-resolution feature volume. At the coarse level, the network computes an initial TSDF by applying a U-Net with sparse 3D CNN \cite{3DSemanticSegmentationWithSubmanifoldSparseConvNet} on the sparse feature volume, which is carved by a visual hull. Through our experiments, it turns out that more than 95\% of the volume grids are discarded by the visual hull culling, making the sparse 3D CNN efficient. At the fine level, the network focuses on a narrow band near the zero-level set of the initial TSDF and discretizes the narrow band with smaller voxels. By employing this narrow-band culling, we further shrink the sampling space, resulting in a relatively small range of grid numbers (usually 300K--500K in our experiments) even with a high volume resolution of 512. The remaining voxels in the narrow band are associated with features that fuse high-frequency information from the computed normal maps upon the low-frequency shape from the coarse level to compute the TSDF at high resolution. The final mesh is then extracted from the TSDF using the Marching-Cube algorithm ~\cite{Lorensen87marchingcubes}.
% Different from the u-net sturcture to preserve global topology context, we then apply a shallow 3dcnn to compute the final TSDF $D_{final}$ which contain more local geometry detail.




% \ping{this paragraph can be expanded. It is an important contribution and often ignored by other works. stress on the novel idea of regressing blending weights instead of colors}

In addition to geometry, high-quality mesh texture is also a crucial factor contributing to visual appearance. Directly computing a color field in 3D space, as in \cite{iccv2020PIFu}, struggles to capture high-frequency texture details, while the neural radiance field (NeRF) \cite{yu2020pixelnerf} or the DoubleField~\cite{shao2022doublefield} require expensive per-instance optimization and are often unstable for sparse input images. In contrast, we adopt an image-based rendering approach to compute a texture atlas map, which is efficient and widely supported in existing computer graphics tools. 
Specifically, we compute a blending weight at each 3D point on the mesh surface to determine its color as a weighted average of the colors at its image projections. The blending weights can be computed at a relatively coarse resolution, e.g., 512 volume resolution in our case, and leave texture details to the high-resolution images, such as 1K or 2K. Unlike previous methods that generate blurry texturing results under sparse input, our method generalizes well on both synthetic and real data with just a few input views. 
Figure~\ref{fig:teaser} shows two examples reconstructed by our method. Despite the challenging garment, pose, and occlusion, our method recovers faithful shape, normal, and texture on the right.

%with a wide variety of poses and clothing styles, and it is also adaptive to handle input image with arbitrary resolutions.
%\sicong{For this concern we claim that when the resolution of dicretized volume meets certain threshold (which is 256 in our experiment), the quantization error can be neglected.} 



In summary, the main contributions of this paper are as follows:
\begin{itemize}
\vspace{-0.1in}
  \item 
  We revisit the 3D volumetric representation and demonstrate that it can support clothed human reconstruction with equal or even better performance compared to implicit representation. 
  \item 
  We develop a memory and computation-efficient method for high-resolution volumetric reconstruction using sophisticated sparse 3D CNN, coarse-to-fine estimation, and voxel culling by visual hull and narrow bands. 
  \item 
  We introduce a novel method to compute a texture atlas map, which captures rich appearance details from high-resolution input images.
  \item 
  We achieve impressive results on standard benchmark datasets Twindom and MultiHuman, significantly reducing the point-2-surface (P2S) precision to approximately 0.2cm from just six input views, with more than $50\%$ error reduction compared to the state-of-the-art methods, including DoubleField~\cite{shao2022doublefield} and PIFuHD~\cite{saito2020pifuhd}.
\end{itemize}




\subsection{Related works}




\label{sec:rel}
Let us take a moment to discuss several related theoretical works on tabular RL; 
there has also been an active line of research that exploits low-dimensional function approximation to further reduce sample complexity, 
which is beyond the scope of this paper. 


Our discussion below focuses on two mainstream approaches that have received widespread adoption: the model-based approach and the model-free approach. 
In a nutshell, 
model-based algorithms decouple model estimation and policy learning, and often use the learned transition kernel to
 compute the value function and find a desired policy. 
In stark contrast, the model-free approach attempts to estimate the optimal value function and optimal policy directly without explicit estimation of the model. 
%While both approaches are capable of achieving asymptotic sample optimality, 
%thus far only the model-based approach has been shown to fully eliminate the burn-in cost, regardless of the sampling mechanism in use \citet{li2020breaking, li2022settling}.  
%On the other hand, model-free algorithms are often more advantageous in terms of memory complexity. 
In general, model-free algorithms only require $O(SAH)$ memory for the purpose of storing Q-functions and value functions, while the model-based counterpart might require $O(S^2AH)$ space in order to store the estimated transition kernel.  



\paragraph{Sample complexity for RL with a simulator and offline RL.} 
%
As an idealistic setting that separates the problem of exploration from that of estimation, 
RL with a simulator (or generative model) has been studied by numerous works, 
allowing the learner to query any state-action pairs and draw independent samples~\citep{kearns1998finite,kakade2003sample,azar2013minimax,agarwal2020model,wainwright2019variance,wainwright2019stochastic,sidford2018near,sidford2018variance,chen2020finite,li2020breaking,pananjady2020instance,li2023q,li2022minimax,even2003learning,shi2023curious,beck2012error,cui2021minimax}.
While both model-based and model-free approaches are capable of achieving asymptotic sample optimality \citep{sidford2018variance,wainwright2019variance,azar2013minimax,agarwal2020model}, 
all model-free algorithms that enjoy asymptotically optimal sample complexity suffer from dramatic burn-in cost. 
Thus far,  only the model-based approach has been shown to fully eliminate the burn-in cost  
for both discounted  infinite-horizon MDPs and inhomogeneous finite-horizon MDPs \citep{li2020breaking}.  
The optimal sample complexity for time-homogeneous finite-horizon MDPs remains open.  





\paragraph{Sample complexity for offline RL.} 
%
The emergent subfield of offline RL is concerned with learning based purely on a pre-collected dataset 
\citep{levine2020offline}. 
A frequently used mathematical model assumes that historical data are collected (often independently) using some behavior policy, 
and the key challenges (compared with RL with a simulator) come from distribution shift and incomplete data coverage. 
%
The sample complexity of offline RL has been the focus of 
a large strand of recent works, with asymptotically optimal sample complexity achieved by multiple algorithms~\citep{jin2021pessimism,xie2021policy,yin2022near,ren2021nearly,shi2022pessimistic,qu2020finite,yan2022efficacy,rashidinejad2021bridging,li2022settling,li2021sample,wang2022gap}.
Akin to the simulator setting, the fully optimal sample complexity (without burn-in cost) has only been achieved via the model-based approach when it comes to discounted infinite-horizon and inhomogeneous finite-horizon settings~\citep{li2022settling}. 
All model-free methods incur substantial burn-in cost. Time-homogeneous finite-horizon MDPs also remain unsettled. 


\paragraph{Sample complexity for online RL.} 
Obtaining optimal sample complexity (or regret bound) in online RL without incurring any burn-in cost 
has been one of the most fundamental open problems in RL theory. 
In fact, the past decades have witnessed a flurry of activity towards improving the sample efficiency of online RL,  
partial examples including \citet{kearns1998near,brafman2002r,kakade2003sample,strehl2006pac,strehl2008analysis,kolter2009near,bartlett2009regal,jaksch2010near,szita2010model,lattimore2012pac,osband2013more,dann2015sample,agralwal2017optimistic,dann2017unifying,jin2018q,efroni2019tight,fruit2018efficient,zanette2019tighter,cai2019provably,dong2019q,russo2019worst,pacchiano2020optimism,neu2020unifying,zhang2020almost,zhang2020reinforcement,tarbouriech2021stochastic,xiong2021randomized,menard2021ucb,wang2020long,li2021settling,li2021breaking,domingues2021episodic,zhang2022horizon,li2023minimax,ji2023regret}. 
% 
Unfortunately, no work has settled this problem completely: 
the state-of-the-art result for model-based algorithms still incurs a burn-in that scales at least quadratically in $S$ 
\citep{zhang2020reinforcement}, while the burn-in cost of the best model-free algorithms (particularly with the aid of variance reduction introduced in \citet{zhang2020almost}) still suffers from sub-optimal horizon dependency \citep{li2021breaking}.  








\subsection{Notation} 
%
Before proceeding, let us introduce a set of notation to be used throughout. 
%
For any set $\mathcal{X}$,  $\Delta(\mathcal{X})$ represents the set of probability distributions over the set $\mathcal{X}$. 
For any positive integer $N$, we denote $[N]=\{1,\ldots,N\}$.
For any two vectors $x,y$ with the same dimension, 
we use $xy$ to abbreviate $x^{\top}y$. 
For any integer $S>0$, any probability vector $p\in \Delta([S])$ and another vector $v=[v_i]_{1\leq i\leq S}$, 
we denote by $\mathbb{V}(p,v) \coloneqq p v^2 - (pv)^2$ the associated variance,  
%Let $\{V^*_h\}_{h=1}^H$ and $\{Q_h^*\}_{h=1}^H$ denote respectively the optimal value function and $Q$-function. 
where $v^2=[v_i^2]_{1\leq i\leq S}$ represents element-wise square of $v$. 
%Let $[N]$ denote the set $\{1,\ldots,N\}$. 
Let $\textbf{1}$ and $\textbf{0}$ indicate respectively the all-one vector and the all-zero vector. Let $\textbf{1}_{s}$ denote the vector with $1$ at the $s$-th coordinate and $0$ at other coordinates.
 We shall often use $\{\cdot\}_{(\lambda)}$ as shorthand for $\{\cdot\}_{\lambda \in \Lambda}$, where $\Lambda$ is the set of all proper choices of the index $\lambda$; 
for example, $\{\cdot \}_{(s,a,h,k)}$ denotes $\{\cdot \}_{(s,a,h,k)\in \mathcal{S}\times \mathcal{A}\times [H]\times [K]}$.  Without loss of generality, we assume throughout that  $K$ is a power of $2$ to streamline presentation.  



%




\section{Problem formulation}
\label{sec:pre}
\vspace{-2mm}
\section{Preliminaries} \label{sec:pre}
%In this section, we first present a formal definition of PageRank. Then we formulate the problem of single-node PageRank computation. Moreover, we provide a brief introduction to Personalized PageRank, which is a variant of PageRank and always plays a vital role in PageRank computation. 
This section introduces several basic concepts that are frequently adopted in the single-node PageRank computation. 
Table~\ref{tbl:def-notation} shows the notations that are frequently used in this paper. 

\begin{table} [t]
%\vspace{-4mm}
\centering
\renewcommand{\arraystretch}{1.3}
\begin{small}
\tblcapup
\caption{\rev Table of notations.}\label{tbl:def-notation}
\vspace{-4mm}
%\resizebox{0.9\linewidth}{!}{%
\resizebox{1\linewidth}{!}{%
%\tblcapdown
%p{2.3in}
\begin{tabular} 
%{|l|p{2.3in}|} \hline
{|c@{\hspace{1mm}}|P{2.25in}|} \hline
{\bf Notation} &  {\bf Description}  \\ \hline
$G=(V,E)$ & undirected graph with vertex set $V$ and edge set $E$ \\ \hline
$n, m$ & the numbers of nodes and edges in $G$ \\ \hline
$N(u)$ & the adjacency list of node $u$\\ \hline
%$n_u=|N_{out}(u)|$ & the number of $u$'s out neighbors\\ \hline
$\A$ & the adjacency matrix of $G$\\ \hline
$d_u$ & the degree of node $u$ \\ \hline
$\davg$ & the average node degree of the graph\\ \hline
$\dmax$ & the maximum node degree of the graph\\ \hline
$\D$ & the diagonal degree matrix that $D_{uu}=d_u$\\ \hline
$\P=\A\D^{-1}$ & the transitional probability matrix\\ \hline
%$\A_{uv}$ & the edge weight of edge $(u,v)\in E$ \\ \hline
%$\|\A\|_1=\sum_{(u,v)\in E}\A_{uv}$ &  the total weights of all edges in $G$\\ \hline
%$N_{in}(u), N_{out}(u)$	& 	the in/out neightbor set of node $u$	\\ \hline
%$d_{in}(u), d_{out}(u)$ & the in/out degree of node $u$ \\ \hline
%$\mathbf{D}$ & the diagonal degree matrix of $G$\\ \hline
%$\bar{d}=\frac{1}{n}\cdot \sum_{u\in V}d_u$ & the average out-degree in graph $G$\\ \hline
%$\A_u^*=\max_{v\in N_{out}(u)}\A_{uv}$ & the maximum edge weight of all the edges connecting node $u$ and its out-neighbors\\ \hline
%$\vp_u$ & the probability distribution for the weighted sampling with replacement at node $u$ that $\vp_u(v)=\frac{\A_{uv}}{d(u)}$ for $\forall v\in N_{out}(u)$\\ \hline
%For the random walk based on weighted sampling, the walk currently at node $u$ needs tsample a out-neighbor of $u$ from the probability distribution $\vp_u$. Specifically, $\vp_u(v)=\frac{\A_{uv}}{d(u)}$ for $\forall v\in N_{out}(u)$, and $\|\vp_u\|_1=1$. \\ \hline
%$\C$, $\rho_v$, $\X$, $\mu(\C)$ & Given a set $\C$, in which every node $v\in \C$ has an associated probability $p_v\in [0,1]$, the coin flip sampling on set $\C$ aims to sample a subset $\X$ that $\forall v\in \C$ is independently sampled into $\X$ with probability $p_v$. Additionally, $\mu(\C)$ denotes the expected output size that $\mu(\C)=\sum_{v\in \C}p_v$. \\ \hline
%Given a set $N_{out}(u)$, the weighted sampling with coin flip samples a subset from $N_{out}(u)$ that each $v\in N_{out}(u)$ is independently selected with probability $\rho_v\in [0,1]$. \\ \hline
%$N_{i}(u), N_{o}(u)$	& 	the in-/out-neighbor set of node $u$	\\ \hline
%$d_u$ & the degree of node $u$ on undirected graph\\ \hline
%$d_{i}(u), d_{o}(u)$ & the in-/out-degree of node $u$ \\ \hline
%$a,b$ & the Laplacian parameters\\ \hline
%$\vec{x}$ & the graph signal vector in $\mathcal{R}^n$, $\left\| \vec{x} \right\|_{2}=1$ \\ \hline
%$w_i, Y_i$ & the $i$-th weight  and partial sum $Y_i=\sum_{k=i}^\infty w_k$\\ \hline
%$L_E = \sum_{i=0}^\infty i w_i $  & the average propagation length\\ \hline
%$\vec{w}$ & the weighted vector with $w_i$ as the $i_{th}$ entry\\ \hline
$\alpha$	& the teleport probability that an $\alpha$-discounted random walk terminates at each step \\ \hline
$\vpi(t),\epi(t)$ & the true and estimated PageRank of node $t$. \\ \hline
$\vpi_t,\epi_t$ & the true and estimated Personalized PageRank vectors with regard to node $t$. \\ \hline
%$\vp_u$ & the sampling probability vector of $\forall v\in N_{out}(u)$ in the random walk process, i.e., $\vp_u(v)=\frac{\A_{uv}}{d_u}$ \\ \hline
%$\vec{r}^{(i)},\hat{\vec{r}}^{(i)}$ & the true and estimated $i$-hop residue vectors in $\mathcal{R}^n$\\ \hline
%$\vec{q}^{(i)},\hat{\vec{q}}^{(i)}$ & the true and estimated $i$-hop reserve vectors in $\mathcal{R}^n$\\ \hline
%$\e_r,\delta$ & the relative error and threshold \\ \hline
%$\e_r, \delta, p_f$ & parameters of relative error, relative error threshold and failure probability \\ \hline
$c$ & constant relative error\\ \hline
$\tilde{O}$ & the Big-Oh natation ignoring the log factors \\ \hline
%$\pi_{\ell}(s,t), \epi_{\ell}(s,t)$	& 	the exact and estimated $\ell$-hop PPR value of node $t$ with respect to $s$. \\ \hline
%$\pi_{\ell}(s,t), r_{\ell}(s,t)$	& 	the reserve and residue of $t$ during $\ell$-hop PPR push from $s$.\\ \hline
%$X_{\ell}(u,v)$	&	the backward push's increments from node $u$ (in level $\ell$) to node v (in level $\ell+1$)\\ \hline
%$C_{\ell}(u,v)$	&	cost in the backward push from node $u$ (in level $\ell$) to node $v$	\\ \hline
%$r_s^f(u), r_t^b(u)$	&	the Node Income	of $u$ in forward / backward Push start from node $s / t$\\ \hline
%$r_s^f(u,v), r_t^b(u,v)$	&	the Edge Saving	of edge ($u,v$) in forward / backward Push start from node $s / t$\\ \hline
%$p_s^f(u,v), p_t^b(u,v)$	&	the Edge Expense of edge ($u,v$) in forward / backward Push start from node $s / t$\\ \hline			
%$c$          &   the decay factor in the definition of SimRank                   \\ \hline
%$\e, \e_{min}$         &   additive error parameter and error required for exactness ($\e_{min} = 10^{-7}$)          \\ \hline
% $\e_r$         &   the maximum relative error allowed in top-$k$ SimRank queries
%$P$, $D$   & the transition matrix and the diagonal correction matrix\\ \hline
%$\vec{\pi}_i, \vec{\pi}_i^\ell,$   & the Personalized PageRank and $\ell$-hop Personalized PageRank vectors of node $v_i$\\ \hline
%$ \vec{h}_i^\ell$   &  the $\ell$-hop Hitting Probability vector of $v_i$\\	\hline
%$\rf(s,t)$, $\pif(s,t)$ & The reserve and residue of $t$ from $s$ in the forward search \\
%\hline
%$\frsum$ & The sum of all nodes' residues during in the forward
%search from $s$\\
%\hline
%$h^{\l}(v_i, v_j)$ & the hitting probability (HP) from node $v_i$ to node $v_j$ at step $\l$ (see Section~\ref{sec:our-overview}) \\ \hline
\end{tabular}
}
\vspace{-4mm}
\end{small}
\end{table}

\subsection{PageRank}
%\header{\bf Single-Node PageRank. } 
Given an undirected and unweighted graph $G=(V,E)$ with $n$ nodes and $m$ edges, the PageRank vector $\vpi$ is an $n$-dimensional vector, which can be mathematically formulated as: 
\begin{align}~\label{eqn:def_pagerank}
\vpi=(1-\alpha)\A\D^{-1}\cdot \vpi +\frac{\alpha}{n}\cdot \bm{1}. 
\end{align}
Here $\A$ denotes the adjacency matrix of the graph, $\D$ is the diagonal degree matrix that $\D_{uu}=d_u$, $\bm{1}\in \mathbb{R}^n$ denotes an all-one vector, and $\alpha$ is a {\em constant} damping factor, which is strictly less than $1$ (i.e., $\alpha \in (0,1)$). For each node $t\in V$, we use $\vpi(t)$ to denote the PageRank value of node $t$. %We may also use $\vpi(t)$ to denote node $t$'s {\em single-node PageRank} of $t$ for simplicity. 
According to the definition formula given in Equation~\eqref{eqn:def_pagerank}, the PageRank value of node $t$ satisfies the following recurrence relation: 
%Specifically, for each node $t\in V$, 
%we can further derive: 
\begin{align}\label{eqn:ite_pagerank}
\vpi(t)=(1-\alpha)\sum_{u\in N(t)}\frac{\vpi(u)}{d_u}+\frac{\alpha}{n}, 
\end{align}
where $u$ is one of the neighbor of node $t$, and $d_u$ denotes the degree of node $u$. In particular, Equation~\eqref{eqn:ite_pagerank} also indicates a lower bound of any node's PageRank that $\vpi(t)\ge \frac{\alpha}{n}$ for each $t\in V$. 

\header{\bf $\alpha$-random walk. } 
By the definition formula of PageRank vector $\vpi$ given in Equation~\eqref{eqn:def_pagerank}, 
%Reviewing the recurrence relation for $\vpi(t)$ given in Equation~\eqref{eqn:ite_pagerank}, the PageRank vector $\vpi$ actually indicates the stationary distribution of the Markov chain that at each step: 
%the PageRank vector $\vpi$ actually corresponds to the stationary distribution of a random walk based Markov chain. Specifically 
we can further derive: 
%In the seminal paper of PageRank~\cite{page1999pagerank}, Page et al. also propose an alternative definition of the PageRank vector $\vpi$ using the stationary distribution of a Markov chain: 
\begin{align}\label{eqn:power_series}
\vpi=\left(\mathbf{I}-(1-\alpha)\A \D^{-1}\right)^{-1}\cdot \left(\frac{\alpha}{n}\cdot \bm{1}\right). 
\end{align} 
As pointed out in~\cite{lofgren2015PHDthesis}, Equation~\eqref{eqn:power_series} can be solved using a power series expansion~\cite{avrachenkov2007monte}: 
\begin{align}\label{eqn:ite_power_method}
\vpi=\sum_{i=0}^{\infty} \alpha(1-\alpha)^i\cdot (\A\D^{-1})^{i}\cdot \frac{1}{n}\cdot \bm{1}, 
\end{align}
where $\vpi$ corresponds to a random walk probability distribution. Specifically, %a random walk
%which actually indicates the probability distribution of a random walk 
%of length $L \sim G(\alpha)$~\cite{fogaras2005MC, jeh2003scaling, fogaras2003start}. Specifically, 
a random walk on the graph is a sequence of nodes $W=\{w_0, w_1, w_2, \ldots\}$ that the $i$-th step (i.e., the node $w_i$) in the walk is selected uniformly at random from the neighbor of node $w_{i-1}$. The PageRank value of node $t$ equals to the probability that a so called {\em $\alpha$-random walk} (or $\alpha$-discounted random walks in some literature)~\cite{wang2017fora, wang2020RBS} simulated from a uniformly selected source node $s$ terminates at node $t$. Note that in each step (e.g., currently at node $u$), an $\alpha$-random walk: 
\begin{itemize}
    \item with probability $(1-\alpha)$, select a neighbor $v$ uniformly at random from the adjacency list $N(u)$ of node $u$, and moves from $u$ to $v$; 
    \item with probability $\alpha$, terminates at the current node $u$. 
\end{itemize}
Therefore, the length $L$ of an $\alpha$-random walk is a geometrical random number following the geometric distribution $L\sim G(\alpha)$. The expectation of $L$ is therefore a constant that $\E\left[L\right]=\frac{1}{\alpha}$. 
%a random walk of length $L$ terminates at node $t$, where the source node $w_0$ is selected uniformly at random from the vertex set $V$ and the length of the walk $L$ is a geometrical random number generated according to the geometric distribution $G(\alpha)$. Such random walk is called as {\em $\alpha$-random walk} (or $\alpha$-discounted random walks in some literature)~\cite{wang2017fora, wang2020RBS}. %The damping factor $\alpha$ is a constant strictly less than $1$ (i.e., $\alpha \in (0,1)$). 

\header{\bf Problem Definition. } In this paper, we concern the problem of single-node PageRank computation. Specifically, given a target node $t$, a relative error parameter $\rela$, and a failure probability parameter $p_f$, we aim to derive a $(c,p_f)$ approximation of $\vpi(t)$, which is formally defined as follows. 
%Given an undirected graph $G=(V,E)$ with a target node $t\in V$, a teleport probability $\alpha \in (0,1)$, and two parameters $\rela \in [0,1]$ and $\pf \in [0,1]$, we aim to compute an $(\e_r, \pf)$ approximation of $\vpi(t)$, which is defined as below.
\begin{definition} [$(\rela,\pf)$-Approximation of Single-Node PageRank]\label{def:problem}
%\begin{definition} [Estimating Single-Node PageRank with Constant Relative Error]
%\label{def:def_relaerr}
%Given a target node $t$ on graph $G=(V,E)$, the task of estimating single-node PageRank with constant relative error $c\in (0,1)$ aims to compute an estimator $\epi(t)$ of the single-node PageRank $\vpi(t)$, such that 
Given a target node $t$ in the graph $G=(V,E)$, %an estimator $\epi(t)$ of the single-node PageRank $\vpi(t)$, two parameters $\rela\in (0,1)$ and $\pf\in (0,1)$, 
$\epi(t)$ is an $(\rela, \pf)$-approximation of the single-node PageRank $\vpi(t)$ if 
\vspace{-1mm}
\begin{equation}\nonumber %\label{eqn:def_maxerr}
\left|\epi(t) - \vpi(t) \right| \le \rela \cdot \vpi(t)
\end{equation}
%simultaneously hold with probability no less than $1-p_f$. 
holds with probability at least $1-\pf$. %, where $c$ and $p_f$ are both constants in $(0,1)$.  
\end{definition}

Note that in a line of research~\cite{bressan2018sublinear, lofgren2014FastPPR, wang2020RBS}, $\rela$ is set as a {\em constant} and thus is omitted in the Big-Oh notation. In this paper, we assume $\rela$ is a constant following this convention. Additionally, we assume $\pf$ is also a {\em constant} without loss of generality. 
%we allow a small failure probability $\pf$ for approximating the single-node PageRank $\vpi(t)$ within the desired accuracy. , we also  since 
It's worth mentioning that a constant failure probability $\pf$ can be easily reduced to arbitrarily small with only adding a log factor to the running time by utilizing the Median-of-Mean trick~\cite{charikar2002mediantrick}.


\subsection{Personalized PageRank}
%\subsection{Personalized PageRank}
%\header{\bf Personalized PageRank (PPR). }
Apart from PageRank, the seminal paper~\cite{page1999pagerank} also propose a variant of PageRank, called Personalized PageRank (PPR), to evaluate the {\em personalized} centrality of graph vertices with respect to a given source node. The definition formula of PPR is analogous to that of PageRank except for the initial distribution: % the $\frac{1}{n}\cdot \bm{1}$ term in Equation~\eqref{eqn:def_pagerank} is replaced with an one-hot vector $\bm{e}_s$: 
\begin{align}\label{eqn:def_ppr}
\vpi_s=(1-\alpha)\A \D^{-1}\cdot \vpi_s +\alpha \bm{e}_s. 
\end{align}
Specifically, $\vpi_s \in \mathcal{R}^{n}$ is called the single-source PPR vector, where $\vpi_s(t)$ denotes the PPR value of node $t$ with respect to node $s$. $\bm{e}_s$ is an one-hot vector that $\bm{e}_s(s)=1$ and $\bm{e}_s(u)=0$ if $u\neq s$. Analogously, by applying the power series expansion~\cite{avrachenkov2007monte}, we can derive: 
\begin{align}\label{eqn:ite_power_method_ppr}
\vpi_s=\sum_{\ell=0}^{\infty}\alpha (1-\alpha)^\ell \left(\A \D^{-1}\right)^{\ell}\cdot \bm{e}_s. 
\end{align}
Equation~\eqref{eqn:ite_power_method_ppr} provides a probabilistic interpretation on the PPR score. Specifically, the PPR value $\vpi_s(u)$ corresponds to the probability that an $\alpha$-random walk generated from node $s$ terminates at node $u$. Additionally, by comparing Equation~\eqref{eqn:ite_power_method_ppr} with Equation ~\eqref{eqn:ite_power_method}, we note that the PageRank score $\vpi(t)$ is actually an average over all $\vpi_u(t)$ for $\forall u\in V$: 
%the relation between PageRank and Personalized PageRank can be directly derived from Equation~\eqref{eqn:ite_power_method} and ~\eqref{eqn:ite_power_method_ppr}: 
%\begin{fact}\label{fact:PageRank_PPR}
\begin{align}\label{eqn:PageRank_PPR}
\vpi(t)=\frac{1}{n}\cdot\sum_{s\in V}\vpi_s(t). 
\end{align}
%\end{fact}
%Recall that the PageRank value of node $t$ can be interpreted as a probability that an $\alpha$-random walk simulated from a {\em random} node terminates at node $t$. In comparison, the PPR value of node $t$ w.r.t node $s$ corresponds to the probability that an $\alpha$-random walk starting from {\em node $s$} terminates at node $t$. And Equation~\eqref{eqn:PageRank_PPR} indicates that $\vpi(t)$ is an average over all $\vpi_s(t)$ for $\forall s\in V$. 
In particular, on undirected graphs, PPR vectors exhibit an underlying {\em reversibility property} that for any node-pair $(u,v)\in V^2$~\cite{lofgren2015BiPPRundirected}:   
%Furthermore, a crucial property held only by the PPR values on undirected graphs is: 
\begin{align}\label{eqn:birectional_ppr}
\vpi_u(v)\cdot d_u=\vpi_v(u) \cdot d_v. 
\end{align}

\header{\bf $\boldsymbol{\ell}$-hop PPR. } Given a source node $s$, a target node $t$ and an integer $\ell\ge 0$, the $\ell$-hop PPR $\vpi_s^{(\ell)}(t)$ corresponds to the probability that an $\alpha$-random walk generated from node $s$ terminates at node $t$ exactly in its $\ell$-th step. The $\ell$-hop PPR vector $\vpi_s^{(\ell)}$ is defined as below. 
%we further present the definition of $\ell$-hop PPR vector $\vpi_s^{(\ell)}$: 
\begin{align}\label{eqn:def_lhopppr}
\vpi^{(\ell)}_s=\alpha (1-\alpha)^\ell \cdot \left(\A \D^{-1}\right)^{\ell} \bm{e}_s. 
\end{align}
By Equation~\eqref{eqn:def_lhopppr} and Equation~\eqref{eqn:ite_power_method}, we can thus derive $\vpi_s=\sum_{\ell=0}^{\infty} \vpi_s^{(\ell)}$. %From the probabilistic aspect, the $\ell$-hop PPR of node $t$ w.r.t node $s$ equals the probability that an $\alpha$-random walk starting from node $s$ terminates at node $t$ exactly at the $\ell$-th step. 
Moreover, the $\ell$-hop PPR value $\vpi^{(\ell)}_s(u)$ admits the following recursive equation that for each node $v\in V$ and each integer $\ell\ge 1$:  
%\begin{fact}\label{fact:PPR_recur}
\begin{align}\label{eqn:PPR_recur}
\vpi_t^{(\ell+1)}(v)=\sum_{u\in N(v)}\frac{(1-\alpha)}{d_u}\cdot \vpi_t^{(\ell)}(u). 
\end{align}
%\end{fact}
Moreover, the $\ell$-hop PPR vector also exhibits the reversibility property on undirected graphs. More specifically, for every two nodes $u,v$ in an undirected $G$ and every $\ell\in \{0,1, \ldots \}$, we have: 
%\begin{fact}\label{fact:undirectedPPR}
\begin{align}\label{eqn:undirectedPPR}
\vspace{-2mm}
\vpi^{(\ell)}_s(t)\cdot d_s=\vpi^{(\ell)}_t(s)\cdot d_t. 
\end{align}
%\end{fact} 

%\header{\bf Evolving Set. } 



%\header{\bf Problem Definition. } 






%%% Local Variables:
%%% mode: latex
%%% TeX-master: "paper"
%%% End:




\section{A model-based algorithm: Monotonic Value Propagation}\label{sec:alg}


In this section, 
we formally describe our algorithm: a simple variation of the model-based algorithm called {\em Monotonic Value Propagation} proposed by \citet{zhang2020reinforcement}. 
%
We present the full procedure in Algorithm~\ref{alg:main}, and point out several key ingredients. 
%
\begin{itemize}
	\item {\em Optimistic updates using upper confidence bounds (UCB).} The algorithm implements the optimism principle in the face of uncertainty 
by adopting the frequently used UCB-based framework \citep{azar2017minimax,jin2018q}. 
More specifically, the learner maintains upper estimates for both the value and Q-function, 
by calculating the following optimistic Bellman equation backward (from $h=H,\ldots,1$): 
%
\begin{subequations}
\begin{align}
Q_{h}(s,a)\, & \leftarrow\,\min\big\{\widehat{r}_{h}(s,a)+\langle\widehat{P}_{s,a,h},V_{h+1}\rangle+b_{h}(s,a),H\big\}, 
	\label{eq:Qh-UCB-informal}\\
V_{h}(s)\, & \leftarrow\, \max\nolimits_{a}Q_{h}(s,a). 
\end{align}
\end{subequations}
%
Here, $Q_{h}$ (resp.~$V_h$) is the running estimate for the Q-function (resp.~value function), 
		$\widehat{r}_{h}(s,a) \in \mathbb{R}$ is an estimate of the mean reward at $(s,a,h)$, 
$\widehat{P}_{s,a,h}\in \mathbb{R}^S$ indicates an estimate of the transition probability vector from $(s,a,h)$, 
whereas $b_{h}(s,a)\geq 0$ is some suitably chosen bonus term that compensates for the uncertainty.   



	\item {\em Monotonic bonus functions.} Another crucial step in order to ensure near-optimal regret lies in careful designs of the data-driven bonus terms $\{b_h(s,a)\}$ in \eqref{eq:Qh-UCB-informal}. 
		Here, we adopt the monotonic bonus function for MVP originally proposed in \citet{zhang2020reinforcement}, 
		to be made precise in \eqref{eq:update1}. 
		Compared to the bonus function in $\mathtt{Euler}$~\citep{zanette2019tighter} and $\mathtt{UCBVI}$~\citep{azar2017minimax},  the monotonic bonus form has a cleaner structure that effectively avoid large lower order terms. In order to enable variance-aware regret, we also need to keep track of the empirical variance of the (stochastic) immediate rewards.


	\item {\em An epoch-based procedure and a doubling trick.} 
%
		A key step of our algorithm is to update the empirical transition kernel and empirical rewards in an epoch-based fashion. 
		More concretely, the whole learning process is divided into several consecutive epochs via a simple doubling rule.
		That is, once the number of visits to a $(s,a,h)$-tuple reaches a power of 2, we end the current epoch,  reconstruct the empirical transition kernel and rewards using data from this epoch (cf.~lines~\ref{line:r-hsa-update} and \ref{line:P-hsa-update} of Algorithm~\ref{alg:main}), compute the Q-function and value function using the newly updated transition kernel and rewards (cf.~\eqref{eq:updateq}), and then start a new epoch. In each epoch, the learned policy is induced by the optimistic Q-function estimate computed based on the empirical transition kernel of the {\em current} epoch. 




\end{itemize}




\begin{remark}[Doubling batch]
We note that a doubling update rule has also been used in the original MVP \citep{zhang2020reinforcement}. 
A major difference between our modified version and the original one is that: when the visitation count for some $(s,a,h)$ reaches $2^i$ for some integer $i$, we only use the second half of the samples (i.e., the $\{2^{i-1}+j\}_{j=1}^{2^{i-1}}$-th samples) to compute the empirical model, whereas the original MVP makes use of all the $2^i$ samples. This step is crucial for decoupling statistical dependence.

\end{remark}







\begin{algorithm}[h]
	\DontPrintSemicolon
\caption{Monotoinic Value Propagation (MVP)~\citep{zhang2020reinforcement}\label{alg:main}}
%\begin{algorithmic}[ht]
	\textbf{input:} state space $\mathcal{S}$, action space $\mathcal{A}$, horizon $H$, total number of episodes $K$, confidence parameter $\delta$, 
	%$\mathcal{L}=\{1,2,\ldots, 2^{\log_2K}\}$, 
	$c_1=\frac{460}{9}$, $c_2 = 2\sqrt{2}$ and $c_3=\frac{544}{9}$.  \\
%\textbf{Initialize:} . \\
	\textbf{initialization: } for all $(s,a,s',h)\in \mathcal{S}\times \mathcal{A}\times\mathcal{S}\times [H]$, set $\theta_h(s,a)\leftarrow 0$, $\kappa_h(s,a)\leftarrow 0$, $\overline{N}_h(s,a,s')\leftarrow 0$, $N_h(s,a,s')\leftarrow 0$, $n_h(s,a)\leftarrow 0$, $Q_h(s,a)\leftarrow H-h+1$, $V_h(s)\leftarrow H-h+1$. \\
	\For{$k=1,2,...$} {
		Set $\pi^k$ such that $\pi_h^k(s) = \arg\max_{a}Q_h(s,a)$ for all $s\in \mathcal{S}$ and $h\in [H]$. {\color{blue}\tcc{policy iterate.}}
		\For {$h=1,2,...,H$} {
			Observe $s_{h}^k$, 
			take action $ a_h^k= \arg\max_{a}Q_h(s_h^k,a)$, 
%	%	\STATE{ Receive reward $r_h^k$ (receive $c_h^k$ and set $r_h^k = -c_h^k$ for the cost case) and  observe $s_{h+1}^k$.}
			receive  $r_h^k$,  observe $s_{h+1}^k$. \label{line:choose_action} 
			{\color{blue}\tcc{sampling.}}
			$(s,a,s')\leftarrow (s_h^k,a_h^k,s_{h+1}^k)$. \\
			Update $\overline{N}_h(s,a) \leftarrow  \overline{N}_h( s,a )+1$, $N_h(s,a,s') \leftarrow   N_h(s,a,s')+1$, $\theta_h(s,a)\leftarrow \theta_h(s,a)+r_h^k$, $\kappa_h(s,a)\leftarrow \kappa_h(s,a)+(r_h^k)^2$. \\
		{\color{blue}\tcc{perform updates using data of this epoch.}}
		\If{$\overline{N}_h(s,a)\in \{1,2,\ldots, 2^{\log_2K}\}$ \label{line:rp_update_start} }   {
			$n_h(s,a)\leftarrow \sum_{\widetilde{s}}N_h(s,a,\widetilde{s})$. 
			{\color{blue}\tcp{number of visits to $(s,a,h)$ in this epoch.}}
			$\widehat{r}_h(s,a)\leftarrow \frac{\theta_h(s,a)}{n_h(s,a)}$. \label{line:r-hsa-update}
			{\color{blue}\tcp{empirical rewards of this epoch.}} 
			$\widehat{\sigma}_h(s,a)\leftarrow \frac{\kappa_h(s,a)}{n_h(s,a) }  $. 
			{\color{blue}\tcp{empirical squared rewards of this epoch.}}
			$\widehat{P}_{s,a,h}(\widetilde{s}) \leftarrow  \frac{N_h(s,a,\widetilde{s})}{n_h(s,a)}$ for all $\widetilde{s} \in \mathcal{S}$.  \label{line:P-hsa-update}
			{\color{blue}\tcp{empirical transition for this epoch.}}
			%
			Set TRIGGERED = TRUE, 
			and $\theta_h(s,a)\leftarrow 0$, $\kappa_h(s,a)\leftarrow 0$,  $N_h(s,a,\widetilde{s})\leftarrow 0$  for all $\widetilde{s}\in \mathcal{S}$. 
%		\ENDIF \label{line:rp_update_end}
		}
		}
		{\color{blue}\tcc{optimistic Q-estimation using empirical model of this epoch.}}
		\If {\textnormal{TRIGGERED= TRUE}} {
			Set TRIGGERED = FALSE, and $V_{H+1}(s)\leftarrow 0$ for all $s\in \mathcal{S}$. \\
			\For{$h=H,H-1,...,1$} {
				%
				\For{$(s,a)\in \mathcal{S}\times \mathcal{A}$} {

					%\vspace{-3ex}
					\begin{align} 
						\vspace{-3ex}
						b_h(s,a) &\leftarrow c_1 \sqrt{\frac{   \mathbb{ V}(\widehat{P}_{s,a,h} ,V_{h+1}) \log \frac{1}{\delta}  }{ \max\{n_h(s,a),1 \} }}+c_2 \sqrt{\frac{\big(\widehat{\sigma}_h(s,a)- (\widehat{r}_h(s,a))^2 \big)\log \frac{1}{\delta}}{\max\{n_h(s,a),1\}}} \qquad\qquad\qquad\qquad\qquad\qquad\nonumber\\
						&\qquad\qquad\qquad +c_3\frac{H\log \frac{1}{\delta}}{ \max\{n_h(s,a) ,1\}  },  \label{eq:update1}  \\
						Q_h(s,a) &\leftarrow \min\big\{    \widehat{r}_h(s,a)+\langle \widehat{P}_{s,a,h}, V_{h+1} \rangle +b_h(s,a)    ,H\big\},\,
						V_{h}(s) \leftarrow \max_{a}Q_h(s,a).
						\label{eq:updateq}
					\end{align}
					\vspace{-3ex}
				}
			}
			%
		}
	}
%\end{algorithmic}
\end{algorithm}






















\begin{comment}

\begin{algorithm}
	\DontPrintSemicolon
\caption{Monotoinic Value Propagation (MVP)~\citep{zhang2020reinforcement}\label{alg:main}}
\begin{algorithmic}[ht]
\STATE{\textbf{Input:} number of states $S$, number of actions $A$, horizon $H$, total number of episodes $K$, confidence parameter $\delta$;}
\STATE{\textbf{Initialize:} $\mathcal{L}=\{1,2,\ldots, 2^{\log_2(K)}\}$; $c_1=\frac{460}{9}, c_2 = 2\sqrt{2},  c_3=\frac{544}{9}$;}
\STATE{\textbf{Initialization: } $\theta_h(s,a)\leftarrow 0, \kappa_h(s,a)\leftarrow 0, \overline{N}_h(s,a,s')\leftarrow 0, N_h(s,a,s')\leftarrow 0, n_h(s,a)\leftarrow 0, Q_h(s,a)\leftarrow H-h+1, V_h(s)\leftarrow H-h+1, \forall (s,a,s',h)\in \mathcal{S}\times \mathcal{A}\times\mathcal{S}\times [H]$;}
\FOR {$k=1,2,...$}
		\FOR {$h=1,2,...,H$}
		\STATE{ Observe $s_{h}^k$;}
		\STATE{ Take action $ a_h^k= \arg\max_{a}Q_h(s_h^k,a)$;} \label{line:choose_action}
	%	\STATE{ Receive reward $r_h^k$ (receive $c_h^k$ and set $r_h^k = -c_h^k$ for the cost case) and  observe $s_{h+1}^k$.}
 \STATE{ Receive reward $r_h^k$  and  observe $s_{h+1}^k$.}
		\STATE{ Set $(s,a,s')\leftarrow (s_h^k,a_h^k,s_{h+1}^k)$;.}
		\STATE{ Set $\overline{N}_h(s,a) \leftarrow  \overline{N}_h( s,a )+1$, \ \,$N_h(s,a,s') \leftarrow   N_h(s,a,s')+1$, $\theta_h(s,a)\leftarrow \theta_h(s,a)+r_h^k$, $\kappa_h(s,a)\leftarrow \kappa_h(s,a)+(r_h^k)^2$.}
		\STATE{ \verb|\\| \emph{Update empirical rewards and transition probability}}
		\IF {$\overline{N}_h(s,a)\in \mathcal{L}$}  \label{line:rp_update_start}
  \STATE{Set $\widehat{r}_h(s,a)\leftarrow \frac{\theta_h(s,a)}{\sum_{\widetilde{s}'}N_h(s,a,\widetilde{s}')}$;}
  \STATE{Set $\widehat{\sigma}_h(s,a)\leftarrow \frac{\kappa_h(s,a)}{\sum_{\widetilde{s}'}N_h(s,a,\widetilde{s'}) } - (\widehat{r}_h(s,a))^2 $}
		\STATE Set $\widehat{P}_{s,a,h,\widetilde{s}} \leftarrow  \frac{N_h(s,a,\widetilde{s})}{\sum_{\widetilde{s}'}N_h(s,a,\widetilde{s}')}$ for all $\widetilde{s} \in \mathcal{S}$.
		%			\ENDFOR
  \STATE{}
		\STATE{ Set $n_h(s,a)\leftarrow \sum_{\widetilde{s}'}N_h(s,a,\widetilde{s}')$;}
		\STATE{ Set TRIGGERED = TRUE.}
  \STATE{$\theta_h(s,a)\leftarrow 0, \kappa_h(s,a)\leftarrow 0,  N_h(s,a,\widetilde{s})\leftarrow 0, \forall \widetilde{s}\in \mathcal{S}$;}
		\ENDIF \label{line:rp_update_end}
		\ENDFOR
		\STATE{ \verb|\\| \emph{Update $Q$-function}}
		\IF {TRIGGERED}
  \STATE{$V_{H+1}(s)\leftarrow 0,\forall s\in \mathcal{S}$;}
		\FOR{$h=H,H-1,...,1$}
		\FOR{$(s,a)\in \mathcal{S}\times \mathcal{A}$}
		%		  \vspace{-3ex}
		\STATE 	 {		%\vspace{-0.5cm} 	
			%					\vspace{-0.5cm}
			%			\simon{Maybe move to the line $N(s,a) \in \mathcal{L}$ so it's clear for the reader that we update the policy only in the set.}
			Set
			\begin{align} 
			~~~~~~~~~~~~~&b_h(s,a)\leftarrow c_1 \sqrt{\frac{   \mathbb{ V}(\widehat{P}_{s,a,h} ,V_{h+1}) \log(\frac{1}{\delta})  }{ \max\{n_h(s,a),1 \} }}+c_2 \sqrt{\frac{(\widehat{\sigma}_h(s,a)- (\widehat{r}_h(s,a))^2 )\log(\frac{1}{\delta})}{\max\{n_h(s,a),1\}}}+c_3\frac{H\log(\frac{1}{\delta})}{ \max\{n_h(s,a) ,1\}  },  \label{eq:update1}  \\
			%\\ 	\hspace{-20ex} 	&  \left\{\begin{array}{l}   Q_h(s,a)\leftarrow \min\{    \widehat{r}_h(s,a)+\widehat{P}_{s,a,h} V_{h+1} +b_h(s,a)    ,H\} \quad  \text{for reward case} \\ Q_h(s,a)\leftarrow \max\{\min \left\{ \widehat{r}_h(s,a) + \widehat{P}_{s,a,h}V_{h+1}+b_h(s,a), 0 \right\} ,-H\} \quad \text{for cost case}\label{eq:updateq}\end{array}\right.
   \hspace{-20ex} 	&    Q_h(s,a)\leftarrow \min\{    \widehat{r}_h(s,a)+\widehat{P}_{s,a,h} V_{h+1} +b_h(s,a)    ,H\}\label{eq:updateq}
			\\ & V_{h}(s) \leftarrow \max_{a}Q_h(s,a).\nonumber
			\end{align}
			\vspace{-3ex}
		}
		\ENDFOR
		\ENDFOR
		\STATE{ Set TRIGGERED = FALSE}
		\ENDIF
		\ENDFOR
\end{algorithmic}
\end{algorithm}

\end{comment}











\section{Key technical innovations}\label{sec:tec}


In this section, we point out the technical hurdles the previous approach encounters when mitigating the burn-in cost, 
and put forward a new strategy to overcome such hurdles. 




\subsection{Technical barriers in prior theory} 
%


\paragraph{A high-level diagnosis of the technical obstacles.} 
%
Let us first single out a technical challenge on a high level. 
In the regret analysis, one central step is to control the error term $(\widehat{P}-P)V$,  where $\widehat{P}$ represents a certain empirical transition kernel (constructed based on collected data), 
$P$ stands for the true transition kernel, and $V$ is a certain value function estimate. 
The analytical difficult arises in that $V$ is often statistically dependent on $\widehat{P}$. 
A couple of strategies have been adopted in prior works to address this issue. 
%
\begin{itemize}
	\item The first strategy, which has been commonly used for model-based algorithms, 
decomposes the error term as \citep{azar2017minimax,dann2017unifying,zhang2020reinforcement} $$(\widehat{P}-P)V =(\widehat{P}-P)V^* +(\widehat{P}-P)(V-V^*).$$ 
Given $V^*$ is independent of $\widehat{P}$, one can apply Bernstein-style concentration inequalities to control the first term $(\widehat{P}-P)V^*$. As for the second term $(\widehat{P}-P)(V-V^*)$, note that $V-V^*$ might become exceedingly small when $K$ is large enough;  
if this were the case, then one could simply bound this term via $\|\widehat{P}-P\|_1  \|V-V^*\|_{\infty}$, which would become a negligible lower-order term. This approach, however, becomes problematic when $K$ is not large enough, 
as this crude bound 
leads to an extra $\widetilde{O}(\sqrt{S})$ factor in the lower-order term  due to the use of $\|\widehat{P}-P\|_1$. 

	\item We now turn to the analysis of model-free algorithms \citep{jin2018q,zhang2020almost,li2021breaking,menard2021ucb}. 
		One way that has been used in earlier analyses (e.g.,  \citet{jin2018q}) can be described as follows: 
		the learner first computes a value estimate $V$, and then employs news samples to construct $\widehat{P}$, 
		which facilitates the analysis of $(\widehat{P}-P)V$ owing to certain independence between $(\widehat{P}-P)$ and $V$. 
		Nevertheless, this strategy falls short of sample efficiency (even in an asymptotic large-sample sense), given that only the samples collected after computation of $V$ are utilized. 
		To enable asymptotic sample optimality, \citet{zhang2020reinforcement} proposed a solution called reference-advantage decomposition (or variance reduction). This strategy maintains a reference value estimate $V^{\mathrm{ref}}$ (computed using a previous batch of data in a way that obeys $V \approx V^{\mathrm{ref}}$) and decomposes $$(\widehat{P}-P)V = (\widehat{P}-P)V^{\mathrm{ref}}+(\widehat{P}-P)(V-V^{\mathrm{ref}}),$$
where the first term can be easily controlled if $\widehat{P}$ is based on data collected after $V^{\mathrm{ref}}$ is determined, and the second term vanishes if $V\approx V^{\mathrm{ref}}$.  
%		
Unfortunately, this strategy also fails to enable all-regime optimality, 
since even computing the first version of $V^{\mathrm{ref}}$ at the initial stage already requires a large sample size. 

\end{itemize}





\paragraph{A closer inspection on prior analysis for UCB-based algorithms.} 
%
Next, let us take a closer inspection on the regret analysis for UCB-based model-based algorithms, 
in order to better illuminate the part that calls for novel analysis. 

 

In each episode $k = 1,\ldots, K$, we update our estimates for the Q-function and the value function as follows:
%
%\begin{align}
% & V^k_{H+1}(s)=0, \qquad Q^k_{H+1}(s,a)=0, \qquad \forall (s,a)\in \mathcal{S}\times \mathcal{A} .\nonumber
% \end{align}
 working backward (i.e., $h=H,H-1,\ldots, 1$), we set 
 \begin{subequations}
 \begin{align}
	 Q^k_{h}(s,a) &= \min\Big\{ \widehat{r}^k_h(s,a) +  \big\langle \widehat{P}_{s,a,h}^k, V_{h+1}^{k} \big\rangle + b_h^k(s,a),\, H \Big\} ,
  &&\forall (s,a)\in \mathcal{S}\times \mathcal{A}\label{eq:updateqq}
 \\  V^k_h(s)&=\max_{a}Q_h^k(s,a), &&\forall s\in \mathcal{S}.\label{eq:updatevv}\end{align}
 \end{subequations}
%
Here, $\widehat{r}^k_h(s,a)$  
%$\widehat{\sigma}_h^k(s,a)$ 
and $\widehat{P}_{s,a,h}^k$ represent respectively the empirical reward 
%reward square 
and the empirical transition model for the $k$-th episode, and $b_h^k(s,a)$ stands for a bonus function 
%computed based on $(\widehat{P}_{s,a,h}^k,V^k_{h+1},\widehat{r}_h^k(s,a),\widehat{\sigma}_h^k(s,a))$
properly chosen to ensure that $Q_h^k(s,a)\geq Q^*_{h}(s,a)$ with high probability. 
These are computed from the collected data.
%\yxc{TODO} 
 We will specify $\widehat{P}_{s,a,h}^k$ and $\widehat{r}_h^k(s,a)$ 
%and $\widehat{\sigma}_h^k(s,a)$ 
later in Section~\ref{sec:tec1}, and $b_h^k(s,a)\geq 0$ has been described in Section~\ref{sec:alg}. 
It has been shown using standard decomposition arguments that~\citep{jaksch2010near,azar2017minimax,zhang2020reinforcement} 
%
\begin{align}
\mathsf{Regret}(K) \lesssim  \sum_{k,h} b_h^k\big(s_h^k,a_h^k\big) + 
	\underset{\eqqcolon\, T_{\mathrm{err}}}{\underbrace{ \bigg| \sum_{k,h} \Big(\widehat{P}^k_{s_h^k,a_h^k,h}-P_{s_h^k,a_h^k,h} \Big)V_{h+1}^k  \bigg| }}
	+ \bigg| \sum_{k,h} \Big(\widehat{r}_h^k\big(s_h^k,a_h^k\big)-r_h\big(s_h^k,a_h^k \big)\Big) \bigg|.
	\label{eq:key}
\end{align}
%
In order to achieve full-range optimal regret, 
one needs to bound the three terms on the right-hand side of \eqref{eq:key} carefully.  
The first term can be bounded in a rate-optimal manner 
(i.e., $\widetilde{O} \big(\sqrt{ \mathbb{V}(P_{s_h^k,a_h^k,h},V_{h+1}^k) /N} +H/N \big)$)  
if we adopt the bonus construction in \citet{zhang2020reinforcement} for the original MVP  (here, we omit the bonus tailored to stochastic rewards for simplicity). In the meantime, the third term on the right-hand side of \eqref{eq:key} 
can be be easily coped with via standard Bernstein-style concentration inequalities. 


%For the first term, i.e., the sum of the bonuses, we hope there are no explicit terms with high-order dependence on $(S,A,H)$ when designing the bonus function. 
% If we can adopt the bonus construction in MVP, which has the simple form of $\widetilde{O} \big(\sqrt{ \mathbb{V}(\widehat{P}_{s_h^k,a_h^k,h},V_{h+1}^k) /N} +H/N \big)$ (we omit the bonus for reward for simplicity), which will have no lower order term. 
% On the other hand, the third term could be dealt with via standard Bernstein-style concentration inequalities.

%\paragraph{The analysis used in prior works.} 

The term that is the most challenging to control is the second error term $T_{\mathrm{err}}$ on the right-hand side of \eqref{eq:key}. 
Given the statistical dependency between $\widehat{P}^k_{s_h^k,a_h^k,h}$ and $V_{h+1}^k$, 
it is often difficult to directly apply concentration inequalities.\footnote{This is different from the simulator and offline RL setting for inhomogeneous MDPs \citep{li2020breaking,li2022settling}, as $\widehat{P}^k_{s_h^k,a_h^k,h}$ and $V_{h+1}^k$ are (or can be made) independent therein.}
To see this, note that the estimation of $\widehat{P}^{k-1}_{s_h^k,a_h^k,h}$ determines the policy $\pi^{k}$ for the $k$-th round, which in turns affects $\{\widehat{P}^{k}_{s,a,h}\}_{(s,a,h)}$ and $V_{h+1}^k$. On the other hand, $\widehat{P}_{s_h^k,a_h^k,h}^k$ is highly correlated with  $\widehat{P}^{k-1}_{s_h^k,a_h^k,h}$ for most $(s,a,h,k)$-tuples, thus implying that $V_{h+1}^k$ is not independent of $\widehat{P}^k_{s_h^k,a_h^k,h}$.
%
In most prior analysis for model-based algorithms \citep{azar2017minimax,dann2017unifying,zanette2019tighter,zhang2020reinforcement}, this term $T_{\mathrm{err}}$ is decomposed as 
%
\begin{align*}
	&\sum_{k,h} \Big(\widehat{P}^k_{s_h^k,a_h^k,h}-P_{s_h^k,a_h^k,h}\Big)V_{h+1}^k 
	%\nonumber\\
	%&\qquad \qquad 
	=\sum_{k,h} \Big(\widehat{P}^k_{s_h^k,a_h^k,h}-P_{s_h^k,a_h^k,h}\Big)V_{h+1}^*+ \sum_{k,h} \Big(\widehat{P}_{s_h^k,a_h^k,h}-P_{s_h^k,a_h^k,h}\Big)\big(V_{h+1}^k-V_{h+1}^*\big).
\end{align*}
%
The first term above can be bounded easily since $V_{h+1}^*$ is fixed and independent of $\widehat{P}^k_{s_{h}^k,a_{h}^k,h}$. 
In comparison, the second term on the right-hand side of the above equation is a lower-order term, which would vanish as $V_{h+1}^*$ converges to $V_{h+1}^*$ (which would happen as $K$ becomes large enough). Such arguments, however, are loose when analyzing the initial stage of the learning process --- given that $V_{h+1}^k-V_{h+1}^*$ is not sufficiently small --- resulting in a potentially large lower-order term and hence large burn-in cost.




\subsection{A novel approach to decouple $V$ from $\widehat{P}$}\label{sec:tec1}
%
To address the above-mentioned issue, the key lies in decoupling the statistical dependence between $\widehat{P}^k_{s_h^k,a_h^k,h}$ and $V_{h+1}^k$. 
Let us first look at the relationship between $\widehat{P}^k_{s_h^k,a_h^k,h}$ 
 and $V_{h+1}^k$.  Let $\{N^k_h(s,a)\}_{(s,a,h)}$ be the number of visits to a state-action-step tuple $(s,a,h)$ before the $k$-th episode. 
Note that $V_{h+1}^k$ is determined by the samples after the $h$-th step up to the $k$-th episode.
 % Then we wonder how these samples is related to $\widehat{P}_{s_h^k,a_h^k,h}$.  
We can find that, $\widehat{P}^k_{s_h^k,a_h^k,h}$ \emph{at most decides the count} after the $h$-th step.
Therefore, if we pretend that the visitation counts $\{N^k_{h}(s,a)\}_{s,a,h,k}$ are fixed independent of $\widehat{P}^k_{s_h^k,a_h^k,h}$, then we can obtain a  desired high-probability upper bound $\widetilde{O}\big( \sqrt{\mathbb{V}(P_{s_h^k,a_h^k,h}, V_{h+1}^k)/N_{h}^k(s_h^k,a_h^k)} \big)$ on the quantity of interest $(\widehat{P}^k_{s_h^k,a_h^k,h}-P_{s_h^k,a_h^k,h})V_{h+1}^k$. 
One natural strategy is then to first develop such bounds for $\{N_{h}^k(s,a)\}_{s,a,h,k}$, and 
then invoke a covering argument that applies a union bound over  all possible choices of $\{N_{h}^k(s,a)\}_{s,a,h,k}$. 

Unfortunately, there are exponentially many choices for $\{N_{h}^k(s,a)\}_{s,a,h,k}$, 
thus preventing one from invoking the uniform convergence argument. 
In order to perform proper compression of the set of all possible choices of $\{N_{h}^k(s,a)\}_{(s,a,h,k)}$, 
we introduce doubling batches (as described in Section~\ref{sec:alg})  
during estimation of the value functions and Q-functions. 
 To facilitate analysis, we have the following definitions.
 
\begin{definition}[Doubling batch and estimations of transitions, rewards, and squared rewards]\label{def:batch} For any $(s,a,h)$, the $i$-th batch for $(s,a,h)$ is the collection of $2^{i-2}+j$-th sample for $j=1,2,\ldots,2^{i-2}$ for $i\geq 2$, and the first sample for $i=1$.\footnote{ It is possible that the total count of $(s,a,h)$ is less than $K$ after $K$ episodes. In this case, we add some virtual samples to fill the $\log_2(K)+1$ batches.} We define $\widehat{P}^{(j)}_{s,a,h}$,  $\widehat{r}^{(j)}_h(s,a)$ and $\widehat{\sigma}^{(j)}_h(s,a)$ to be the empirical transition probability, the empirical reward, and the empirical squared reward of the $j$-th batch for $(s,a,h)$,  respectively. For completeness, we define the $0$-th batch for each $(s,a,h)$ as an empty set, and set $\widehat{P}^{(0)}_{s,a,h} = \frac{1}{S}\mathbf{1}$, $\widehat{r}^{(0)}_h(s,a)= 0$ and $\widehat{\sigma}^{(0)}_h(s,a)=0$ for the $0$-th batch.  
\end{definition}

 
  \begin{definition}[Profile]
Fix $k\in [K]$. Let $I^k_{s,a,h}$ be the largest integer obeying $2^{I^k_{s,a,h}-1}\leq N_{h}^k(s,a)$ for each $(s,a,h)$. In particular, when $N_h^k(s,a)=0$, we set $I^k_{s,a,h}=0$.  
	The profile for the $k$-th episode is defined as 
%
\begin{equation}
	\mathcal{I}^k \coloneqq \big\{I^k_{s,a,h} \big\}_{(s,a,h)\in \mathcal{S}\times \mathcal{A}\times [H]}.
	\label{eq:defn-profile-Ik}
\end{equation}
%
We further let $\mathcal{I} \coloneqq \{\mathcal{I}^k\}_{k=1}^K$ be the total profile.
  \end{definition}


With regards to the online RL, we can define a natural filtration induced by the sequential learning process. The formal definition is as follows. 
%
  \begin{definition}[Online filtration]\label{filt1}
	  For any $(h,k)\in [H+1]\times [K]$, let $\mathcal{F}_h^k$ be the $\sigma$-algebra induced by events happening before the $h$-th step in the $k$-th episode. Then $\{\mathcal{F}_h^k\}_{(h,k)\in [H]\times [K]}$ --- with proper ordering in accordance with the sequential learning process --- constructs a filtration $\mathcal{F}_{\mathrm{online}}$, which we shall refer to as ``online filtration'' throughout. 
  \end{definition}


In order to facilitate analysis, we find it helpful to introduce another filtration below tailored to a generative model (which can be defined in a more flexible way than the online counterpart).  
%
  \begin{definition}[Generative filtration]\label{def:filt2}
    Consider an order over all state-action pairs in $\mathcal{S}\times \mathcal{A}$ such that  
 $\mathcal{S}\times \mathcal{A} = \{ (s^{(i)},a^{(i)})\}_{i=1}^{SA}$. 
 %The samples  are draw in an increasing order from the $H$-th step to the first step, where each $(s,a,h)$ is sampled for $K$ times.  
  Let us employ the following sampling order of the state-action-step tuples: 
 %
 \begin{align}
	 \begin{array}{cccc}
(s^{(1)},a^{(1)},H) & (s^{(2)},a^{(2)},H) & \cdots & (s^{(SA)},a^{(SA)},H)\\
(s^{(1)},a^{(1)},H-1) & (s^{(2)},a^{(2)},H-1) & \cdots & (s^{(SA)},a^{(SA)},H-1)\\
 & \cdots\\
(s^{(1)},a^{(1)},1) & (s^{(2)},a^{(2)},1) & \cdots & (s^{(SA)},a^{(SA)},1)
\end{array}
 \end{align}
 %
where for each $(s,a,h)$ we draw $K$ independent sample transitions from the generative model.  
	  For any $1\leq t\leq K$, define $\overline{\mathcal{F}}_{s,a,h}(t)$ to be the $\sigma$-algebra induced by events happening after the $t$-th sample of $(s,a,h)$ is collected. For any $1\leq z\leq SAHK$,  define $\widetilde{\mathcal{F}}(z) \coloneqq \overline{\mathcal{F}}_{s^{(i)},a^{(i)},h}(t)$, where $(i,h,t)$ is chosen be such that $z = (H-h)\cdot SA\cdot K+ (i-1)\cdot K  + t $. Then (a proper ordering of) $\{\widetilde{F}(z)\}_{z=1}^{SAKH}$ constructs a
 filtration $\mathcal{F}_{\mathrm{gen}}$, which we shall refer to as ``generative filtration'' throughout.
  \end{definition}

In words, there are a total number of  $SAHK$ samples to be collected ($K$ i.i.d.~samples for each $(s,a,h)$), 
and we introduce a sequential ordering of them, with $\widetilde{F}(z)$ denoting the $\sigma$-algebra after the $z$-th sample. 
For convenience, we assume that all initial states have been generated from $\mu$ before the online learning process starts. Then  $\widetilde{F}(SAHK)$ could be viewed as an expansion of $\mathcal{F}_{H+1}^k$, since one could simulate the whole online learning process using the $SAHK$ independent samples in the generative filtration. In other words, for each event in the online filtration $\mathcal{F}_{\mathrm{online}}$, it is measurable w.r.t.~the generative filtration $\mathcal{F}_{\mathrm{gen}}$. 
This allows one to conduct analysis based on the generative filtration (as we shall detail momentarily). 
% In particular, in Lemma~\ref{lemma:key1}, we will prove a concentration inequality under the generative filtration. 
We also remark that we will only use the generative filtration $\mathcal{F}_{\mathrm{gen}}$ when necessary, 
given that the analysis under the online filtration $\mathcal{F}_{\mathrm{online}}$ is more natural for an online learning problem. 



\iffalse
\paragraph{Decoupling $V$ from $\widehat{P}$ with generative filtration} Below we present the high-level idea to use generative filtration to decouple $V$ from $\widehat{P}$.  Fix $\{n_h(s,a)\}_{(s,a,h)}\in [K]^{SAH}$. Define $\widehat{P}_{s,a,h}$ be the empirical transition model of the first $n_{h}(s,a)$ samples of $(s,a,h)$ under the generative filtration. Now we consider the value function $V$ defined as:
\begin{align}
& V_{H+1}(s) = 0,  \forall s\in \mathcal{S};\nonumber
\\ & V_{h}(s) = \min\{\max_{a} r_h(s,a) + \widehat{P}_{s,a,h}V_{h+1}+b(\widehat{P}_{s,a,h},V_{h+1},n_h(s,a)) ,H\},\forall s\in \mathcal{S}, h = H,H-1,\ldots,1,\label{eq:ls1}
\end{align}
where $b(\cdot,\cdot,\cdot)$ is some proper bonus function. 
Conditioned on  $\widetilde{\mathcal{F}}(SAK(H-h))$, $V_{h+1}$ is fixed, and $\{n_h(s,a)\widehat{P}_{s,a,h}\}_{s,a,h}$ are  mutually independent  multinomial random variables. As a result,  the term
\begin{align}
\sum_{h=1}^H \sum_{s,a}n_h(s,a) (\widehat{P}_{s,a,h} - P_{s,a,h})V_{h+1}
\end{align}
could be viewed as a martingale difference. With concentration inequality (Lemma~\ref{lemma:self-norm}), we can bound this term as 
\begin{align}
\sum_{h=1}^H \sum_{s,a}n_h(s,a) (\widehat{P}_{s,a,h} - P_{s,a,h})V_{h+1} \leq O\left(  \sqrt{\sum_{h=1}^H \sum_{s,a} n_h(s,a)\mathrm{Var}(P_{s,a,h},V_{h+1})\log(\frac{SAH}{\delta})  } + H \log(\frac{SAH}{\delta})\right)\label{eq:cpx1}
\end{align}
with probability $1-\delta$. In comparison, if we bound each $(\widehat{P}_{s,a,h}-P_{s,a,h})V_{h+1}$ by $O\left( \sqrt{\frac{\mathbb{V}(P_{s,a,h},V_{h+1})\log(\frac{SAH}{\delta})}{n_h(s,a)}} + \frac{H\log(\frac{SAH}{\delta})}{n_h(s,a)} \right)$, 
 the total bound would be: with probability $1-\delta$
 \begin{align}
\sum_{h=1}^H \sum_{s,a}n_h(s,a)(\widehat{P}_{s,a,h}-P_{s,a,h})V_{h+1}\leq O\left(  \sqrt{SAH\sum_{h=1}^H \sum_{s,a} n_h(s,a)\mathrm{Var}(P_{s,a,h},V_{h+1})\log(\frac{SAH}{\delta})  } + SAH^2 \log(\frac{SAH}{\delta})\right)\label{eq:cpx21}.
 \end{align}

However,   \eqref{eq:cpx1} only holds for fixed $\{n_h(s,a)\}_{(s,a,h)}$, while \eqref{eq:cpx21} holds for any $\{n_h(s,a)\}_{(s,a,h)}$
\fi 



\paragraph{Doubling batch updates.}
%
We  update the value function and Q-function with the doubling batches.
Namely, in the $k$-th episode, 
 we choose $\widehat{P}^k_{s,a,h}= \widehat{P}^{(I^k_{s,a,h})}_{s,a,h}$, $\widehat{r}^k_h(s,a) = \widehat{r}^{(I^k_{s,a,h})}_h(s,a)$ and $\widehat{\sigma}^k_h(s,a) = \widehat{\sigma}^{(I^k_{s,a,h})}_h(s,a)$ for any $(s,a,h,k)$. Our update rule is a slightly different from previous doubling update rules \citep{jaksch2010near}, where the algorithm keeps running a policy until the visitation count of some $(s,a,h)$ doubles and then uses the whole dataset to compute the empirical transition model. In contrast, we divide the whole dataset into disjoint batches following Definition~\ref{def:batch}, and only use the latest batch to compute the empirical transition model. We design this update rule because the samples in different batches are not correlated, which could help us decouple the value function and empirical transition model.
Crucially, our update rule preserves sample efficiency, since the latest batch always contains at least half of the samples. 

 
  


\subsection{Key lemmas}
\label{sec:key-lemmas-tec}
 
As discussed above, under the generative filtration  $\mathcal{F}_{\mathrm{gen}}$, 
the random vector $\widehat{P}^k_{s_h^k,a_h^k,h}$ is conditionally independent of $V_{h+1}^k$ for any $(k,h)\in [K]\times [H]$ if we fix $\mathcal{I}$.
Then we can view the error term $$T_{\mathrm{err}}=\sum_{k=1}^K \sum_{h=1}^H \Big(\widehat{P}^k_{s_h^k,a_h^k,h}-P_{s_h^k,a_h^k,h} \Big) V_{h+1}^k$$  as a martingale difference and obtain a desired  bound. 
Following this intuition, we introduce our key lemma to bound the error term $T_{\mathrm{err}}$ with the doubling batch updates mentioned above.  





  Let $\mathcal{C}$ be a set which contains all possible values of the total profiles $\mathcal{I}$.
One key novelty is to obtain a tight bound on $|\mathcal{C}|$, which we will discuss later.


 Now,  fix any $\mathcal{J}\in \mathcal{C}$, and consider the event $\mathcal{E}(\mathcal{J},\delta)$ defined as follows:
 %
  \begin{align}
	  \mathcal{E}(\mathcal{J},\delta) \coloneqq \left\{\mathcal{I}=\mathcal{J}, \, T_{\mathrm{err}} \leq \sqrt{L\sum_{k=1}^K \sum_{h=1}^H\mathbb{V} \big(P_{s_h^k,a_h^k,h},V_{h+1}^k \big)\bigg(SAH+\log \frac{1}{\delta} \bigg) } + L H \bigg(SAH+\log \frac{1}{\delta} \bigg) \right\}, 
  \end{align}
  %
  where $L$ is a logarithmic term in $(S,A,H,K)$ to be defined shortly. We claim that 
  %
  \begin{equation}
	  \mathrm{Pr}\big(\mathcal{E}(\mathcal{J},\delta)\big)\geq 1- \delta 
	  \label{eq:Pr-E-delta-bound}
  \end{equation}
  %
  for any $\mathcal{J}\in \mathcal{C}$ and $\delta \in (0,1)$.  
Then by applying the union bound over $\mathcal{J} \in \mathcal{C}$ and rescaling $\delta$ as $\delta/|\mathcal{C}|$, we obtain that with probability at least $1-\delta$, 
%
\begin{align}
	T_{\mathrm{err}}\leq  L\sqrt{\sum_{k=1}^K \sum_{h=1}^H\mathbb{V} \big( P_{s_h^k,a_h^k,h},V^k_{h+1} \big)\bigg(SAH+\log\frac{|C|}{\delta}\bigg) } + LH\bigg(SAH+\log\frac{|C|}{\delta}\bigg).\nonumber
\end{align}
%


Consequently, we are left with accomplishing the following two steps:
%
\begin{enumerate}
    \item  Prove that $\mathrm{Pr}(\mathcal{E}(\mathcal{J},\delta))\geq 1-\delta$ (i.e., \eqref{eq:Pr-E-delta-bound}).
    \item Determine $\mathcal{C}$ and  bound $|\mathcal{C}|$ properly.
\end{enumerate}
% (\romannumeral1); (\romannumeral2) 

\paragraph{Proof of inequality~\eqref{eq:Pr-E-delta-bound}.}
%
Towards this end, we need the lemma below. 
%
\begin{lemma}\label{lemma:key1}
Fix any $\mathcal{J} = \{J^k\}_{k=1}^K$. For each $h \times [H]$, let $\mathcal{X}_{h+1}$ be a set of vectors obeying 
\begin{itemize}

\item $\|X\|_{\infty}\leq H$, $\forall X\in \mathcal{X}_{h+1}$; 
\item  $\mathcal{X}_{h+1}$ is determined by $\{\widehat{P}^{(J^k_{s,a,h'})}_{s,a,h'},\widehat{r}^{(J^k_{s,a,h'})}_{h'}(s,a),\widehat{\sigma}^{(J^k_{s,a,h'})}_{h'}(s,a)\}_{ h'\in [h+1, H], k\in [K], (s,a)}$ and $\{J^k\}_{k=1}^K$; 
\item $|\mathcal{X}_{h+1}|\leq W$ for any $1\leq h \leq H$ and some $W\in \mathbb{N}$; 
\item the all-zero vector $\textbf{0}\in \mathcal{X}_{h+1}$ for each $h\in [H]$.

\end{itemize}
%
Then with probability at least $1-\delta$, it holds that
%
\begin{align}
& \Bigg| \sum_{k=1}^K \sum_{h=1}^H \Big(\widehat{P}^{(J^k_{s_h^k,a_h^k,h})}_{s_h^k,a_h^k,h}-P_{s_h^k,a_h^k,h} \Big) X^k_{h+1} \Bigg| \nonumber
	\\ & \leq  \sqrt{L\sum_{k=1}^K \sum_{h=1}^H \mathbb{V}\big(P_{s_h^k,a_h^k,h},X^k_{h+1}\big)\bigg(SAH\log W + \log\frac{1}{\delta}\bigg)  } 
	+ LH\bigg(SAH\log W + \log\frac{1}{\delta}\bigg)
\end{align}
%
for any sequence $\{X^k_{h+1}\}_{(h,k)}$ such that $X^k_{h+1} \in \mathcal{X}_{h+1},\forall (h,k)\in [H]\times [K]$, 
where $L = 200(\log_2(K)+1)^2$.
\end{lemma}

The proof of Lemma~\ref{lemma:key1} is based on a martingale concentration inequality in the view of $\mathcal{F}_{\mathrm{gen}}$. We first fix the choice of $X_{h+1}^k\in \mathcal{X}_{h+1}$ for each $(h,k)$, and then verify that $\sum_{s,a}\sum_{h=1}^H 2^{l-2}(\widehat{P}^{(l)}_{s,a,h}-P_{s,a,h}) Y_{s,a,h}$ is a martingale difference for any $l\geq 2$ and $\{Y_{s,a,h}\}_{(s,a,h)}$ as long as  $Y_{s,a,h}$ is selected from $\{X_{h+1}^k\}_{k=1}^K$ according to some specific rule for any $(s,a,h)$. 
See Appendix~\ref{app:pfseckey1} for the proof of Lemma~\ref{lemma:key1}. 



\paragraph{Bounding the size of possible profiles $\left|\mathcal{C}\right|$.}
%
Next, we turn to the second problem concerning $\left|\mathcal{C}\right|$. 
%
Let us choose 
%
\begin{equation}\mathcal{C} \coloneqq 
	\Big\{ \mathcal{J}=\{J^1,J^2,\ldots,J^K\} \,\Big|\, J^{\tau}\leq J^{\tau+1}, \forall 1\leq \tau \leq K-1,  J^{\tau}\in \big\{0\cup[\log_2K]\big\}^{SAH},\forall \tau \Big\}.
	\label{eq:defn-C-choice}
\end{equation}
%
Given that $i^k\leq i^{k+1}$ for $1\leq k\leq K-1$, it is easily seen that $\mathcal{I}\in \mathcal{C}$. The lemma below serves to upper bound the size of $\mathcal{C}$.
%
\begin{lemma}\label{lemma:key2} 
	The choice \eqref{eq:defn-C-choice} obeys 
$|\mathcal{C}|\leq (4SAHK)^{SAH(\log_2K +1)}$.
\end{lemma}

In proving Lemma~\ref{lemma:key2}, we use the increasing property that  $J^{\tau}\leq J^{\tau+1},\forall 1\leq\tau \leq K-1$  for $ \mathcal{J} = \{J^1,J^2,\ldots,J^{K}\}\in \mathcal{C}$. The naive bound for the size of  $\mathcal{C}$ is $(\log_2(K)+1)^{SAHK}$, which is too large for our purpose. By virtue of the increasing property, we are actually counting the number of increasing paths in the $SAH$-dimensional grid $ \left\{[\log_2(K)]\cup {0}\right\}^{SAH}$. For each  increasing path, there are at most $SAH(\log_2(K)+1)$ steps and at most $SAH$ directions for each step. Then the proof can be completed with some primitive combinatorial  computations. 
The detailed proof can be found in Appendix~\ref{app:pfkey2}. 



With Lemma~\ref{lemma:key1} and Lemma~\ref{lemma:key2} in mind, we can invoke a uniform convergence argument to reach the lemma below; 
the proof of this lemma is postponed to Appendix~\ref{app:pfkey3}. 
%
\begin{lemma}\label{lemma:key3} Recall that $\mathcal{I}=\{I^k\}_{k=1}^K$ is the total profile and the fact that $\widehat{P}^{k}_{s,a,h}=\widehat{P}^{(I^k_{s,a,h})}_{s,a,h}, \widehat{r}^k_{h}(s,a) = \widehat{r}^{(I^k_{s,a,h})}_h(s,a), \widehat{\sigma}_h^k(s,a)=\widehat{\sigma}_h^{(I_{s,a,h}^k)}(s,a)$ for any proper $(s,a,k,h)$. 
 For each $h \times [H]$, let $\mathcal{X}_{h+1}$ be a set of vectors be such that: (1) $\|X\|_{\infty}\leq H$, $\forall X\in \mathcal{X}_{h+1}$ ; (2)  $\mathcal{X}_{h+1}$ is determined by $\{\widehat{P}^{k}_{s,a,h'}, \widehat{r}^k_{h'}(s,a),\widehat{\sigma}^k_{h'}(s,a)\}_{ h+1\leq h'\leq H, 1\leq k\leq K, (s,a)}$ and $\{I^k\}_{k=1}^K$; (3) $|\mathcal{X}_{h+1}|\leq W$ for any $1\leq h \leq H$ and some $W\in \mathbb{N}$; (4) the zero vector $\textbf{0}\in \mathcal{X}_{h+1}$ for each $h\in [H]$. Then with probability at least $1-\delta$, it holds that
\begin{align}
& \Bigg| \sum_{k=1}^K \sum_{h=1}^H \Big(\widehat{P}^{k}_{s_h^k,a_h^k,h}-P_{s_h^k,a_h^k,h}\Big) X^k_{h+1}  \Bigg| \nonumber
\\ & \leq  \sqrt{L_1\sum_{k=1}^K \sum_{h=1}^H \mathbb{V}(P_{s_h^k,a_h^k,h},X^k_{h+1})\bigg( SAH\log W + \log \frac{1}{\delta} \bigg)  } 
	+ L_1H \bigg( SAH\log W + \log \frac{1}{\delta} \bigg) \label{eq:star} 
\end{align}
for any sequence $\{X^k_{h+1}\}_{h,k}$ such that $X^k_{h+1} \in \mathcal{X}_{h+1},\forall (h,k)\in [H]\times [K]$, 
where $L_1 = 4000\log^2_2(K)\log(SAHK)$.
\end{lemma}
%
 



%The full proofs of Lemma~\ref{lemma:key1}, Lemma~\ref{lemma:key2} and Lemma~\ref{lemma:key3} can be found in Appendix~\ref{app:mfsectec}.


In Algorithm~\ref{alg:main}, we compute $V_{h+1}^k$  by the following rule.
\begin{align}
& V^k_{H+1}(s) = 0 ;\forall s\in \mathcal{S}; \nonumber
\\ & V_{h'}^k(s) = \min\left\{ \max_{a} \left( \widehat{r}^k_{h'}(s,a)+ \widehat{P}^k_{s,a,h'}V_{h'+1}^k + b(\widehat{P}^k_{s,a,h'},\widehat{r}^k_{h'}(s,a),\widehat{\sigma}_{h'}^k(s,a), n_h^k(s,a) \right)  ,H\right\}
	%\nonumber\\
%&~~~~~~~~~~~~~~~~~~~~~~~~~~~~~~~~~~~~~~~~~~~~~~~~~~~~~~~~~~~~~~~~~~~~~~~~~~~~~~~~~~~~~~~~~~~~~~~~~~~~~~\forall s\in \mathcal{S}, h'=H,H-1,\ldots, h+1.\nonumber
\end{align}
%
for all $s\in \mathcal{S}$ and $h'=H,H-1,\ldots, h+1$. 
Here $b(\cdot,\cdot,\cdot,\cdot)$ is some proper bonus function and $n_h^k(s,a)$ is the size of the $I^k_{s,a,h}$-th batch of $(s,a,h)$. 
Then $V_{h+1}^k$ is determined by $\{\widehat{P}^{k}_{s,a,h'}, \widehat{r}^{k}_{h'}(s,a),\widehat{\sigma}^{k}_{h'}(s,a)\}_{ h+1\leq h'\leq H, 1\leq k'\leq k, (s,a)}$ and $\{I^k\}_{k=1}^K$, 
thus allowing us to apply Lemma~\ref{lemma:key3} to bound $T_{\mathrm{err}}$ by choosing $\mathcal{X}_{h+1} = \{V_{h+1}^k\}_{k=1}^K$.
In addition, 
it is worth noting that Lemma~\ref{lemma:key3} is more general compared to our original target to bound $T_{\mathrm{err}}$, since  $\mathcal{X}_{h+1}$ can be chosen as arbitrary functions. 





\section{Proof of Theorem~\ref{thm1}}\label{app:thmmain}


This section is devoted to proving Theorem~\ref{thm1}. 
For notational convenience, let $B$ be a logarithmic term
%
\begin{equation}
	B=4000 (\log_2 K)^3 \log(3SAH)\log\frac{1}{\delta'} 
	\label{eq:assumption-K-proof-B}, 
\end{equation}
%
where we recall that $\delta$ is the confidence parameter in Algorithm~\ref{alg:main} and $\delta' = \frac{\delta}{200SAH^2K^2}$. 
When $K\leq BSAH$, the claimed result in Theorem~\ref{thm1} holds trivially since
%
\[
\mathsf{Regret}(K)=\sum_{k=1}^{K}\left(V_{1}^{\star}(s_{1}^{k})-V_{1}^{\pi^{k}}(s_{1}^{k})\right)
\leq HK 
= \min\left\{ \sqrt{BSAH^{3}K},HK\right\}.
%=\widetilde{O}\left(\min\left\{ \sqrt{SAH^{3}K},HK\right\} \right).
\]
%
As a result, it suffices to focus on the scenario with 
%
\begin{equation}
	K\geq BSAH \qquad \text{with } B=4000 (\log_2 K)^3 \log(3SAH)\log\frac{1}{\delta'} .
	\label{eq:assumption-K-proof}
\end{equation}
% 
 
  
%Let $\pi^k$ be the policy in the $k$-th episode.  Let $\overline{N}_{h}^k(s,a)$ be the count  of $(s,a,h)$ before the $k$-th episode and $N_h^k(s,a)$ be the count of the doubling batch used to compute the value function in the $k$-th episode. In particular, when $\overline{N}_h^k(s,a)=0$, we define $N_h^k(s,a)=1$.   Let $V_h^k $ and $ Q_h^k$ be respectively the value of $ V_h$ and $Q_h$ before the $k$-th episode for all proper $(s,a,k,h)$. Recall that $\widehat{P}^k_{s,a,h}$ is the value of $\widehat{P}_{s,a,h}$ before the $k$-th episode. Let $\widehat{r}_h^k(s,a)$ be the empirical reward function before the $k$-th episode of $(s,a)$. Let $\widehat{\sigma}_h^k(s,a)$ be the empirical variance before the $k$-th episode for the state-action pair $(s,a)$, i.e., the value of $\widehat{\sigma}_h(s,a)$ before the $k$-th episode.

%For the sake of notational simplicity, we shall adopt a slightly different bonus term throughout the proof:
%
%Recall that 
%\begin{align} 
%	b_h(s,a) = c_1 \sqrt{\frac{   \mathbb{ V}(\widehat{P}_{s,a,h} ,V_{h+1}) \log \frac{1}{\delta'}  }{ \max\{N_h(s,a),1 \} }}+c_2 \sqrt{\frac{\big(\widehat{\sigma}_h(s,a)- (\widehat{r}_h(s,a))^2 \big)\log \frac{1}{\delta'}}{\max\{N_h(s,a),1\}}} 
%+c_3\frac{H\log \frac{1}{\delta'}}{ \max\{N_h(s,a) ,1\}  },  \label{eq:update1-proof}  
%\end{align}
%
%where the log term is taken to be $\log \frac{1}{\delta'}$ as opposed to $\log \frac{SAHK}{\delta}$;  the aim then becomes proving that the advertised result in Theorem~\ref{thm1} holds with probability $1-SAKH\delta$. 
%\yxc{check} 

%\subsection{Regret analysis}
%\label{sec:regret-decomp}


Our regret analysis for Algorithm~\ref{alg:main} consists of several steps described below. 

\paragraph{Step 1: the optimism principle.} 
%
To begin with, we justify that the running estimates of Q-function and value function in Algorithm~\ref{alg:main} are always upper bounds on the optimal Q-function and the optimal value function,  respectively,  
thereby guaranteeing optimism in the face of uncertainty. 
%
\begin{lemma}[Optimism]\label{lemma:opt}
With probability exceeding $1-4SAHK\delta'$, one has
%
\begin{equation}
	Q_h^k(s,a)\geq Q_h^{\star}(s,a) \qquad  \text{and}  \qquad V^k_h(s)\geq V^{\star}_h(s)
\end{equation}
%
for all $(s,a,h,k)$. 
 \end{lemma}
 %
 \begin{proof} See Appendix~\ref{sec:proof-lemma:opt}. \end{proof}




 
%\simon{let's add full details in the final version because I think this paper gonna be the standard reference for tabular MDP.}


\paragraph{Step 2: regret decomposition.}
%
In view of the optimism shown in Lemma~\ref{lemma:opt}, 
the regret can be upper bounded by 
%
\begin{align}
\mathsf{Regret}(K) & =\sum_{k=1}^{K}\big(V_{1}^{\star}(s_{1}^{k})-V_{1}^{\pi^{k}}(s_{1}^{k})\big)\leq\sum_{k=1}^{K}\big(V_{1}^{k}(s_{1}^{k})-V_{1}^{\pi^{k}}(s_{1}^{k})\big)
	\label{eq:regret-UB1}
\end{align}
%
with probability at least $1-4SAHK\delta'$. 
In order to control the right-hand side of \eqref{eq:regret-UB1}, 
we first make note of the following upper bound on $V_{1}^{k}(s_{1}^{k})$. 
%
\begin{lemma}\label{lemma:decomdetail}
For every $1\leq k\leq K$, one has
%
$$
	V_1^k(s_1^k) \leq \sum_{h=1}^{H} \left( \big\langle \widehat{P}^k_{s_h^k,a_h^k,h} - P_{s_h^k,a_h^k,h}, V_{h+1}^k \big\rangle + b_h^k(s_h^k,a_h^k) + \widehat{r}_h^k(s_h^k,a_h^k) + \big\langle P_{s_h^k,a_h^k,h}-e_{s_{h+1}^k}, V_{h+1}^k \big\rangle \right) .
$$
\end{lemma}
%
\begin{proof}[Proof of Lemma~\ref{lemma:decomdetail}]
%
From the construction of $V_h^k$ and $Q_h^k$, it is seen that, for each $1\leq h\leq H$,  
%
\begin{align}
	V_h^k(s_h^k) & = Q_h^k(s_h^k, a_h^k) \leq \widehat{r}_h^k(s_h^k,a_h^k) + \widehat{P}^k_{s_h^k,a_h^k,h}V_{h+1}^k + b_h^k(s_h^k,a_h^k) \nonumber
\\ &  =  \big\langle \widehat{P}^k_{s_h^k,a_h^k,h} - P_{s_h^k,a_h^k,h}, V_{h+1}^k \big\rangle + b_h^k(s_h^k,a_h^k) + \widehat{r}_h^k(s_h^k,a_h^k) + \big\langle P_{s_h^k,a_h^k,h}-e_{s_{h+1}^k}, V_{h+1}^k \big\rangle + V_{h+1}^k(s_{h+1}^k).\nonumber
\end{align}
%
Applying this relation recursively over $1\leq h\leq H$ gives 
%
\begin{align*}
 & V_1^k(s_1^k) \nonumber
 \\ & \leq  \sum_{h=1}^{H} \left( \big\langle \widehat{P}^k_{s_h^k,a_h^k,h} - P_{s_h^k,a_h^k,h}, V_{h+1}^k \big\rangle + b_h^k(s_h^k,a_h^k) 
	+ \widehat{r}_h^k(s_h^k,a_h^k) + \big\langle P_{s_h^k,a_h^k,h}-e_{s_{h+1}^k}, V_{h+1}^k \big\rangle \right) + V_{H+1}^k(s_{H+1}^k),
\end{align*}
%
which combined with $V_{H+1}^k=0$ concludes the proof. 
\end{proof}


Combine Lemma~\ref{lemma:decomdetail} with \eqref{eq:regret-UB1} to show that, with probability at least $1-4SAHK\delta'$, 
%
\begin{align}
\mathsf{Regret}(K) & \leq\underset{\eqqcolon\,T_{1}}{\underbrace{\sum_{k=1}^{K}\sum_{h=1}^{H}\big\langle\widehat{P}_{s_{h}^{k},a_{h}^{k},h}^{k}-P_{s_{h}^{k},a_{h}^{k},h},V_{h+1}^{k}\big\rangle}}+\underset{\eqqcolon\,T_{2}}{\underbrace{\sum_{k=1}^{K}\sum_{h=1}^{H}b_{h}^{k}(s_{h}^{k},a_{h}^{k})}}\nonumber\\
 & \quad+\underset{\eqqcolon\,T_{3}}{\underbrace{\sum_{k=1}^{K}\sum_{h=1}^{H}\big\langle P_{s_{h}^{k},a_{h}^{k},h}-e_{s_{h+1}^{k}},V_{h+1}^{k}\big\rangle}}+\underset{\eqqcolon\,T_{4}}{\underbrace{\sum_{k=1}^{K}\left(\sum_{h=1}^{H}\widehat{r}_{h}^{k}(s_{h}^{k},a_{h}^{k})-V_{1}^{\pi^{k}}(s_{1}^{k})\right)}}, 
	\label{eq:decomposition}
\end{align}
%
leaving us with four terms to control. 
In particular, $T_1$ has already been upper bounded in Section~\ref{sec:tec1}, and hence we shall describe how to bound $T_2,\ldots,T_4$ in the sequel.    


%where $b_h^k(s_h^k,a_h^k): = c_1\sqrt{\frac{\mathbb{V}(  \widehat{P}^k_{s_h^k,a_h^k,h},V_{h+1}^k)\log(\frac{1}{\delta'})}{N^k_{h}(s_h^k,a_h^k) }} +c_2 \sqrt{\frac{\left( \widehat{\sigma}_h^k(s,a)- (\widehat{r}_h^k(s,a))^2 \right)\log(\frac{1}{\delta'})}{N_h^k(s_h^k,a_h^k)}}+ c_3\frac{H\log(\frac{1}{\delta'})}{N^k_{h}(s_h^k,a_h^k)}$.



% Recall the definition that $\pi^k_h(s) = \arg\max_a Q_h^k(s,a)$. With probability $1-2SAHK\delta$, 




% Define $T_1 =\sum_{k=1}^K \sum_{h=1}^H \left(    (\widehat{P}^k_{s_h^k,a_h^k,h}   - P_{s_h^k,a_h^k,h})V_{h+1}^k \right) $, $T_2 = \sum_{k=1}^K \sum_{h=1}^H  b_h^k(s_h^k,a_h^k)$, $T_3 =\sum_{k=1}^K\sum_{h=1}^H ( P_{s_h^k,a_h^k,h}-\mathbf{1}_{s_{h+1}^k})V_{h+1}^k $ and $T_4 = \sum_{k=1}^K \left(\sum_{h=1}^H \widehat{r}^k_{h}(s^k_h,a^k_h)- V_1^{\pi^k}(s_1^k) \right) $.




\paragraph{Step 3.1: bounding the terms $T_2,T_3$ and $T_4$.}
%
In this section, we seek to bound the terms  $T_2,T_3$ and $T_4$ defined in the regret decomposition \eqref{eq:decomposition}.  
To do so, we find it helpful to first introduce the following quantities that capture some sort of aggregate variances: 
%
\begin{subequations}
\label{eq:defn-T56-proof}
\begin{align}
	T_5 &\coloneqq \sum_{k=1}^K\sum_{h=1}^H\mathbb{V}\big(\widehat{P}^k_{s_h^k,a_h^k,h},V_{h+1}^k \big),
	\label{eq:defn-T5-proof} \\
	T_6 &\coloneqq \sum_{k=1}^K \sum_{h=1}^H\mathbb{V} \big(P_{s_h^k,a_h^k,h},V_{h+1}^k \big) ,
	\label{eq:defn-T6-proof}
\end{align}
\end{subequations}
%
with $T_5$ denoting certain empirical variance and $T_6$ the true variance. 
With these quantities in place, we claim that the following bounds hold true. 
%
\begin{lemma}\label{lem:bound-T234}
%
With probability exceeding $1-15SAH^2K^2\delta'$, one has 
%\begin{itemize}
%	\item Regarding the non-negative term $T_2$, we have the following connection between $T_2$ and $T_5$: 
%
\begin{subequations}
\label{eq:boundt234}
\begin{align}
	T_2 &\leq 61\sqrt{2SAH(\log_2 K)\Big(\log\frac{1}{\delta'}\Big)T_5} +  8\sqrt{SAH^3K(\log_2 K)\log\frac{1}{\delta'}}+
	151 SAH^2(\log_2K)\log\frac{1}{\delta'},
	\label{eq:boundt2} \\
	|T_3|  &\leq \sqrt{ 8 T_6 \log \frac{1}{\delta'}  } + 3H\log \frac{1}{\delta'}, \label{eq:boundt3} \\
	|T_4| &\leq 6\sqrt{2SAH^3K(\log_2K)\log \frac{1}{\delta'} } +  55SAH^2(\log_2 K)\log \frac{1}{\delta'}.\label{eq:bdt_4f}
\end{align}
\end{subequations}
%
%with probability exceeding $1-2SAHK\delta'$. 
%
%	\item With regards to the term $T_3$, with probability at least $1-10SAH^2K^2\delta'$ one has
%
%\begin{align}
% |T_3|   \leq 2\sqrt{2}\cdot \sqrt{  T_6 \log \frac{1}{\delta'}  } + 3H\log \frac{1}{\delta'}. \label{eq:boundt3}
%\end{align}
%
%
%	\item When it comes to the term $T_4$, with probability at least $1-3SAHK\delta'$ we have
%
%\begin{equation}
%	|T_4| \leq 6\sqrt{2SAH^3K(\log_2K)\log \frac{1}{\delta'} } +  55SAH^2(\log_2 K)\log \frac{1}{\delta'}.\label{eq:bdt_4f}
%\end{equation}
%
%\end{itemize}
%
\end{lemma}
%
\begin{proof} See Appendix~\ref{app:pflem:bound-T234}. \end{proof}



\paragraph{Step 3.2: bounding the aggregate variances $T_5$ and $T_6$.} 
%
The previous bounds on $T_2$ and $T_3$ stated in Lemma~\ref{lem:bound-T234} depend respectively on the aggregate variance $T_5$ and $T_6$  (cf.~\eqref{eq:defn-T5-proof} and \eqref{eq:defn-T6-proof}), 
which we would like to control now. 
By introducing the following quantities: 
%
\begin{subequations}
\label{eq:defn-T789-proof}
\begin{align}
T_{7} & \coloneqq\sum_{k=1}^{K}\sum_{h=1}^{H}\Big\langle\widehat{P}_{s_{h}^{k},a_{h}^{k},h}^{k}-P_{s_{h}^{k},a_{h}^{k},h},\big(V_{h+1}^{k}\big)^{2}\Big\rangle,\label{eq:defn-T7-proof}\\
T_{8} & \coloneqq\sum_{k=1}^{K}\sum_{h=1}^{H}\Big\langle P_{s_{h}^{k},a_{h}^{k},h}-e_{s_{h+1}^{k}},\big(V_{h+1}^{k}\big)^{2}\Big\rangle,\label{eq:defn-T8-proof}\\
T_9 & \coloneqq \sum_{k=1}^{K}\sum_{h=1}^{H}\max\Big\{\Big\langle\widehat{P}_{s_{h}^{k},a_{h}^{k},h}^{k}-P_{s_{h}^{k},a_{h}^{k},h},V_{h+1}^{k}\Big\rangle,0\Big\}, 
\label{eq:defn-T9-proof}
\end{align}
\end{subequations}
%
we can upper bound $T_5$ and $T_6$ through the following lemma.  
%
\begin{lemma}
\label{lem:bound-T56}
With probability at least $1-4SAHK\delta'$,
%
\begin{subequations}
\label{eq:boundt56}
\begin{align}
T_{5} & \leq T_{7}+T_8+ 2HT_{2}+6KH^{2},
	%+\sqrt{32H^{2}T_{6}\log\frac{1}{\delta'}}+3H^{2}\log\frac{1}{\delta'} ,
\label{eq:boundt5} \\
T_{6} & \leq 2HT_{2}+6KH^{2}
	+\sqrt{32H^{2}T_{6}\log\frac{1}{\delta'}}+3H^{2}\log\frac{1}{\delta'}
	+2HT_{9}, 
\label{eq:boundt6} \\
|T_{8}|&
	%\leq2\sqrt{2}\sqrt{\sum_{k,h}\mathbb{V}\Big(\widehat{P}_{s_{h}^{k},a_{h}^{k},h}^{k},\big(V_{h+1}^{k}\big)^{2}\Big)\log\frac{1}{\delta'}}+3H^{2}\log\frac{1}{\delta'}
	\leq \sqrt{32H^{2}T_{6}\log\frac{1}{\delta'}}+3H^{2}\log\frac{1}{\delta'}	 . 
	\label{eq:boundt8}
\end{align}
\end{subequations}
%
\end{lemma}
%
\begin{proof} See Appendix~\ref{sec:pflem:bound-T56}. \end{proof}  






\iffalse

\subsection{Bounding the terms $T_1,T_7$ and $T_9$}
In this section, we will prove a key lemma to deal with the error terms $T_1,T_7$ and $T_9$. This lemma controls the error term $\sum_{k,h}(\widehat{P}_{s_h^k,a_h^k,h}-P_{s_h^k,a_h^k,h}) X_{h}^k$ under some mild conditions, where we do not require $\{X_{h}^k\}_{h,k}$ is conditionally independent of $\{\widehat{P}^k_h\}_{h,k}$. By this lemma, we can control the error terms above by letting $\mathcal{X}_h^k = \{ V_{h+1}^k, (V_{h+1}^k)^2/H, \textbf{0}\}$. % This consititutes our key novelty

%\simon{add intuitions what $\mathcal{X}_h^k$ will be like later, and we why need this lemma}


\begin{lemma}\label{lemma:key}
For each $k,h$,  let a set of $S$-dimensional vectors  $\mathcal{X}_h^k$  be a function of $\{\widehat{P}^k_{s,a,h'}\}_{h'\geq h+1,s,a}$ such that $\|X\|_{\infty}\leq H, \forall X\in \mathcal{X}_h^k$. Let $L = \max_{h,k}|\mathcal{X}_h^k|$. Then with probability $1- 10\delta$, for any sequence $\{X_h^k\}$ such that  $X_h^k\in \mathcal{X}_h^k$, it holds that
\begin{align}
 &  \sum_{k=1}^K \sum_{h=1}^H \left( \widehat{P}_{s_h^k,a_h^k,h}-P_{s_h^k,a_h^k,h} \right) X_h^k  \nonumber
 \\ & \leq  \sqrt{32\log_2(K) \sum_{k=1}^K \sum_{h=1}^H \mathbb{V}(P_{s_h^k,a_h^k,h},X_h^k) \cdot (4SAH\log_2(K)\log(SALH)+\log(\frac{1}{\delta'})) } + 12\log_2(K)H (4SAH\log_2(K)\log(SALH)+\log(\frac{1}{\delta'})).\label{eq:key}
\end{align}
\end{lemma}
\begin{proof}




 Recall the definition of $i^k_{s,a,h}$. Define a \emph{profile} as $\{i_{s,a,h}\}_{s,a,h}$ with $i_{s,a,h}\in [\log_2(K)]$. We say two profiles $i\leq j$ iff $i_{s,a,h}\leq j_{s,a,h}$ for any $(s,a,h)$.
 
 Let $\mathcal{I} =\left\{  \{i^k_{s,a,h}\}_{s,a,h}     | k = 1,2,\ldots,K  \right\}$.  Since there are at most $SAH(\log_2(K)+1)$ updates, the size of  $\mathcal{I}$ is at most $SAH(\log_2(K)+1)$.  

\simon{this paragraph might be the most important technical paragraph. We can add more details and a plot.}
Let $M=SAH(\log_2(K)+1)$ and $\mathcal{C}:= \{  \{j^1,j^2,\ldots,j^{M}\} |j_l\in [\log_2(K)]^{SAH}, j^l\neq j^{l'} \forall l\neq l', j^l\leq j^{l+1}, \forall 1\leq l\leq M-1\}$. In words, $\mathcal{C}$ is the set of an increasing path in the set $[\log_2(K)]^{SAH}$ from $[0,0,\ldots,0]^{\top}$ to $[\log_2(K),\log_2(K),\ldots,\log_2(K)]^{\top}$. Since there are at most $M$ steps and in each step we have at most $SAH$ choices, the size of $\mathcal{C}$ is at most $(SAH)^{M}$. Let $\overline{\mathcal{C}} = \{  \{j^1,j^2,j^3,\ldots,j^q\}| j^l<j^{l+1},1\leq l \leq q-1, q\leq M \}$. Then the size of $\overline{\mathcal{C}}$ is at most $2^{M}\cdot |\mathcal{C}|=(2SAH)^M$. 
\simon{without using the increasing path property, we will have size $(\log_2K)^{SAHM}$.}

Recalling the definition of $\mathcal{I}$, we always have that $\mathcal{I}\in \overline{\mathcal{C}}$.
%Because $i^1\leq i^2\leq \ldots,\leq i^K$, $\mathcal{I}$ is determined by an increasing path in $[\log_2(K)]^{SAH}$ (it is possible that $\mathcal{I}$ does not end with $[\log_2(K),\log_2(K),\ldots, \log_2(K)]^{\top}$). Then we can always find some $\mathcal{J}\in \mathcal{C}$ such that $\mathcal{I}\subset \mathcal{J}$.

% Now we fix $\mathcal{J}\in \overline{\mathcal{C}}$. Let $\delta'\in (0,1)$ be a confidence parameter and $\log(\frac{1}{\delta'})' = \log(2/\delta')$. Assume $\mathcal{J} =\{j^1,j^2,\ldots, j^M\}$ such that $j^1\leq j^2\leq \cdots \leq j^M$.  For $j\in [\log_2(K)]^{SAH}$, we use $\widehat{P}(j)$ to denote the tuple $\{ \widehat{P}_{s,a,h}(j_{s,a,h})  \}_{s,a,h}$ where $\widehat{P}_{s,a,h}^{(l)}$ is the empirical transition model computed using the $l$-th batch of $(s,a,h)$ for $1\leq l \leq \log_2(K)$, and $\widehat{P}_{s,a,h}^{(l)} = \frac{1}{S}\cdot \textbf{1}$ if $l=0$.  We also define $\widehat{P}_h(j)=\{ \widehat{P}_{s,a,h'}(j_{s,a,h})  \}_{h'\geq h+1,s,a}$.%\simon{what is $j^k$?}


\end{proof}

As a result, with probability $1-\delta$ it holds that 
\begin{align}
 &  \sum_{k=1}^K \sum_{h=1}^H \left( \widehat{P}_{s_h^k,a_h^k,h}-P_{s_h^k,a_h^k,h} \right) X_h^k  \nonumber
 \\ & \leq  \sqrt{32\log_2(K) \sum_{k=1}^K \sum_{h=1}^H \mathbb{V}(P_{s_h^k,a_h^k,h},X_h^k) \cdot (4SAH\log_2(K)\log(SALH)+\log(\frac{1}{\delta'})) } + 12\log_2(K)H (4SAH\log_2(K)\log(SALH)+\log(\frac{1}{\delta'})).\nonumber
\end{align}
The proof is completed.

\fi 





\paragraph{Step 3.3: bounding the terms $T_1$, $T_7$ and $T_9$.} 
%
Taking a look at the above bounds on $T_2,\ldots,T_6$, 
we see that one still needs to deal with the terms $T_1$, $T_7$ and $T_9$ (see \eqref{eq:decomposition}, \eqref{eq:defn-T7-proof} and \eqref{eq:defn-T9-proof}, respectively). 
As it turns out, these quantities have already been bounded in Section~\ref{sec:tec}. 
Specifically, Lemma~\ref{lemma:decouple} tells us that: with probability at least $1-\delta'$, 
%
%\begin{align*} 
%	T_1 &\leq T_9 
%	%\sum_{k=1}^{K}\sum_{h=1}^{H}\Big\langle\widehat{P}_{s_{h}^{k},a_{h}^{k},h}^{k}-P_{s_{h}^{k},a_{h}^{k},h},V_{h+1}^{k}\Big\rangle
%	%\leq \sum_{k=1}^{K}\sum_{h=1}^{H}\max\Big\{\Big\langle\widehat{P}_{s_{h}^{k},a_{h}^{k},h}^{k}-P_{s_{h}^{k},a_{h}^{k},h},V_{h+1}^{k}\Big\rangle,0\Big\} \\
%  \leq \sqrt{16(\log_{2}K)\sum_{k=1}^{K}\sum_{h=1}^{H}\mathbb{V}\big(P_{s_{h}^{k},a_{h}^{k},h},V_{h+1}^{k}\big)\left(6SAH\log_{2}^{2}K+\log\frac{1}{\delta'}\right)}+49SAH^{2}\log_{2}^{3}K+8H(\log_{2}K)\log\frac{1}{\delta'} ;\\
%%\end{align*}
%%
%%and
%%
%%\begin{align*} 
%	%& \sum_{k=1}^{K}\sum_{h=1}^{H}\Big\langle\widehat{P}_{s_{h}^{k},a_{h}^{k},h}^{k}-P_{s_{h}^{k},a_{h}^{k},h},\big(V_{h+1}^{k}\big)^{2}\Big\rangle\\
%	T_7 
% & \leq8H\sqrt{(\log_{2}K)\sum_{k=1}^{K}\sum_{h=1}^{H}\mathbb{V}\big(P_{s_{h}^{k},a_{h}^{k},h},V_{h+1}^{k}\big)\left(6SAH\log_{2}^{2}K+\log\frac{1}{\delta'}\right)}+49SAH^{3}\log_{2}^{3}K+8H^2(\log_{2}K)\log\frac{1}{\delta'}.
%\end{align*}
% 
\begin{subequations}
\label{eq:boundt179}
\begin{align}
	T_1\leq T_9&\leq \sqrt{ B SAH \sum_{k=1}^K \sum_{h=1}^H \mathbb{V}(P_{s_h^k,a_h^k,h},V_{h+1}^k)}+BSAH^2 = \sqrt{BSAH T_6}+BSAH^2,\label{eq:boundt1}
 \\ 
	T_7 &\leq H \sqrt{ BSAH  \sum_{k=1}^K \sum_{h=1}^H \mathbb{V}(P_{s_h^k,a_h^k,h}, V_{h+1}^k) }    + BSAH^3 = H\sqrt{BSAH T_6}+BSAH^3 ,\label{eq:boundt7}
 %\\ & T_9 \leq \sqrt{B SAH \sum_{k=1}^K \sum_{h=1}^H \mathbb{V}(P_{s_h^k,a_h^k,h},V_{h+1}^k)}+BSAH^2= \sqrt{BSAH T_6}+BSAH^2.\label{eq:boundt9}
\end{align}
\end{subequations}
%
where we recall that $B=4000(\log_2K)^3\log(3SAH)\log\frac{1}{\delta'} $. 


%for which we remind the reader that
%
%\begin{align}
%	T_1 &= \sum_{k=1}^K \sum_{h=1}^H  \Big\langle \widehat{P}^k_{s_h^k,a_h^k,h}  -P_{s_h^k,a_h^k,h} , V_{h+1}^k \Big\rangle
%;\nonumber
%	\\  T_7  &= \sum_{k=1}^K \sum_{h=1}^H  \Big\langle \widehat{P}^k_{s_h^k,a_h^k,h}  -P_{s_h^k,a_h^k,h} , (V_{h+1}^k)^2\Big\rangle;\nonumber
%	\\  T_9 &=  \sum_{k=1}^{K}\sum_{h=1}^{H}\max\Big\{\Big\langle\widehat{P}_{s_{h}^{k},a_{h}^{k},h}^{k}-P_{s_{h}^{k},a_{h}^{k},h},V_{h+1}^{k}\Big\rangle,0\Big\}.\nonumber
%\end{align}
%
%Notably, all these quantities involve weighted sums of $\widehat{P}^k_{s_h^k,a_h^k,h}$. 



%Recall $B=4000\log^2_3(K)\log(3SAH)\log(\frac{1}{\delta'}) $. By the update rule \eqref{eq:updateq}, $V_{h+1}^k$ is determined by the $\{ \widehat{P}^k_{s,a,h'} \}_{(h+1\leq h'\leq H ,s,a)}$ and $\{I^k_{s,a,h'}\}_{h+1\leq h'\leq H,s,a}$. 
%Using Lemma~\ref{lemma:key3} with $\mathcal{X}_{h+1} =\{V_{h+1}^k\}_{k=1}^K \cup\{0\}, \{(V_{h+1}^k)^2 /H\}_{k=1}^K\cup\{0\}$ and $\{ V_{h+1}^k\}_{k=1}^K \cup\{0\}$, and noting that $\mathsf{Var}(X^2)\leq 4\|X\|^2_{\infty}\mathsf{Var}(X)$ (see Lemma~\ref{lemma:sqv}), we have 
%
%\begin{align}
% & T_1\leq \sqrt{ B SAH \sum_{k=1}^K \sum_{h=1}^H \mathbb{V}(P_{s_h^k,a_h^k,h},V_{h+1}^k)}+BSAH^2 = \sqrt{BSAH T_6}+BSAH^2;\label{eq:boundt1}
% \\ & T_7 \leq H \sqrt{ 4BSAH  \sum_{k=1}^K \sum_{h=1}^H \mathbb{V}(P_{s_h^k,a_h^k,h}, V_{h+1}^k) }    + 4BSAH^3 = H\sqrt{4BSAH T_6}+4BSAH^3 ;\label{eq:boundt7}
% \\ & T_9 \leq \sqrt{B SAH \sum_{k=1}^K \sum_{h=1}^H \mathbb{V}(P_{s_h^k,a_h^k,h},V_{h+1}^k)}+BSAH^2= \sqrt{BSAH T_6}+BSAH^2.\label{eq:boundt9}
%\end{align}
%
 

\paragraph{Step 4: putting all pieces together.}
%
The previous bounds \eqref{eq:boundt234}, \eqref{eq:boundt56} and \eqref{eq:boundt179} indicate that: 
%and \eqref{eq:boundt9} 
with probability at least $1-100SAH^2K^2\delta'$, one has
%
\begin{subequations}
\label{eq:all-bounds-summary}
\begin{align}
	T_2 &\leq  \sqrt{B SAHT_5} +  \sqrt{BSAH^3K}+ BSAH^2,\label{eq:obt2}
	\\  T_3 &\leq \sqrt{BT_6}+HB  ,\label{eq:obt3}
	\\  T_4 &\leq \sqrt{ BSAH^3K}+BSAH^2,\label{eq:obt4}
	\\  T_5 &\leq T_7 + T_8 + 2H T_2 + 6KH^2 ,\label{eq:obt5}
	\\  T_6 &\leq \sqrt{B H^2T_6} + 2HT_2 + 2HT_9 + BH^2 + 6KH^2,\label{eq:obt6}
	\\  T_8 &\leq \sqrt{BH^2T_6 } + BH^2 ,\label{eq:obt8}
	\\  T_1 &\leq \sqrt{BSAHT_6} + BSAH^2,\label{eq:obt1}
	\\  T_7 &\leq H\sqrt{BSAHT_6} + BSAH^3,\label{eq:obt7}
	\\  T_9 &\leq \sqrt{BSAHT_6}+BSAH^2,\label{eq:obt9}
\end{align}
\end{subequations}
%
where we again use $B=4000(\log_2K)^3\log(3SAH)\log\frac{1}{\delta'} $.   


 To solve the inequalities \eqref{eq:all-bounds-summary}, we resort to the elementary AM-GM inequality: if $a\leq \sqrt{bc}+d$ for some $b,c\geq 0$, then it follows that $a \leq \epsilon b + \frac{1}{2\epsilon}c +d$ for any $\epsilon>0$. This basic inequality combined with  \eqref{eq:all-bounds-summary} gives
 %
 \begin{align}
	 HT_2 &\leq \epsilon T_5 + \left(\frac{1}{2\epsilon}+1\right) BSAH^3+ \frac{3}{2}BSAH^3 + \frac{1}{2} KH^2 ,\nonumber
	 \\  T_6 &\leq \epsilon T_6 + 2HT_2 + 2HT_9 + \left(1+\frac{1}{2\epsilon}\right) BH^2 + 6KH^2,\nonumber
	 \\  HT_9 &\leq \epsilon T_6 +\left( \frac{1}{2\epsilon}+1\right)BSAH^3,\nonumber
	 \\  T_8 &\leq \epsilon T_6 + \left( \frac{1}{2\epsilon}+1\right)BH^2,\nonumber
	 \\  T_7 &\leq  \epsilon T_6 + \left(\frac{1}{2\epsilon} +1\right)BSAH^3,\nonumber
 \end{align}
 %
which in turn result in
%
\begin{align}
	 T_5 &\leq T_7+T_8 + 2HT_2 + 6KH^2\leq 2\epsilon T_5+2\epsilon T_6 +  \left( \frac{1}{\epsilon}+2\right) BSAH^3+6KH^2;\nonumber
\\  T_{6} & \leq\epsilon T_{6}+2HT_{2}+2HT_{9}+\left(1+\frac{1}{2\epsilon}\right)BH^{2}+6KH^{2} 
	\leq3\epsilon T_{6}+2\epsilon T_{5}+\left(\frac{3}{\epsilon}+8\right)BSAH^{3}+7KH^{2}. \notag
\end{align}
%
By taking $\epsilon= 1/20$, we arrive at 
%
\begin{align}
	T_5+T_6\lesssim BSAH^3+KH^2 \asymp KH^2,
\end{align}
%
where the last relation holds due to our assumption $K\geq SAHB$ (cf.~\eqref{eq:assumption-K-proof}).  
%We then have that $T_5,T_6 = O(KH^2 + BSAH^3) = O(KH^2)$. 
Substituting this into \eqref{eq:all-bounds-summary}  yields
%
\begin{align}
	T_1 \lesssim \sqrt{BSAH^3K},
	%\quad  T_7+ T_8 \lesssim \sqrt{BSAH^5K}, 
	\quad T_2 \lesssim \sqrt{BSAH^3K}, \quad  T_3 \lesssim \sqrt{BKH^2}
	\quad \text{and} \quad
	T_4 &\lesssim \sqrt{ BSAH^3K},
\end{align}
%
provided that $K\geq SAHB$. 
These bounds taken collectively with \eqref{eq:decomposition} readily give
%
\[
	\mathsf{Regret}(K) \lesssim \sqrt{ BSAH^3K} . 
	%\leq \widetilde{O}\left(\sqrt{SAH^3K\log \frac{1}{\delta'}} \right).
\]
%
 



Combining the two scenarios (i.e., $K\geq BSAH$ and $K\leq BSAH$) reveals that with probability at least $1-100SAH^2K^2\delta'$, 
%
\[
	\mathsf{Regret}(K) \lesssim \min\big\{ \sqrt{ BSAH^3K} , HK \big\}
	\lesssim  
	\min\bigg\{ 
	\sqrt{ BSAH^3K \log^5 \frac{SAHK}{\delta'}} , HK \bigg\}.
	%\leq \widetilde{O}\left(\sqrt{SAH^3K\log \frac{1}{\delta'}} \right).
\]
%
The proof of Theorem~\ref{thm1} is thus completed by recalling that $\delta' = \frac{\delta}{200SAH^2K^2}$.







\section{Extensions}
\label{sec:extensions}



With the refined error bound derived in Section~\ref{sec:key-lemmas-tec}, 
we can readily obtain more refined regret bounds for Algorithm~\ref{alg:main} to reflect the role of several problem-dependent quantities.  Most of the arguments in the analysis are similar to those in the previous work \citet{zhou2023sharp}. 
Detailed proofs are postponed to Appendix~\ref{sec:appfirst} and Appendix~\ref{app:var}.
% Our results show that 




%\subsection{Value-based and cost-based regret bounds}\label{sec:value-cost}



%We restate Theorem~\ref{thm:first} as below.
%
%
%\noindent\textbf{Theorem 2. (Restated)}\emph{
%For any $K \ge 1$, with probability $1-\delta$, the regret of Algorithm~\ref{alg:main} is 
%$$\widetilde{O}\left(\min\{\sqrt{SAH^2K v^*}+SAH^2,Kv^* \}\right)$$ where $v^*$ is the value of the optimal policy, i.e., $v^* =\max_{\pi} \mathbb{E}_{\pi,s_1\sim \mu}\left[\sum_{h=1}^H r_h\right]$.}

% The proof of Theorem~\ref{thm:first} is similar to that of Theorem~\ref{thm1}, except for that we provide refined analysis for some of the regret terms. See  Appendix~\ref{sec:appfirst} for more details.


%
\paragraph{Value-based regret bounds.} 
%
Thus far, we have not yet introduced the crucial quantity $v^{\star}$ in Theorem~\ref{thm:first}, 
which we define now. 
%
When the initial states are drawn from $\mu$, $v^{star}$ stands for the weighted optimal value: 
%
\begin{equation}
	v^{*} \coloneqq \mathbb{E}_{s\sim \mu}\big[ V_1^*(s) \big]. 
	\label{eq:defn-vstar-formal}
\end{equation}
%
Encouragingly, 
the value-dependent regret bound in Theorem~\ref{thm:first} is still minimax-optimal, 
as asserted by the following lower bound. 
%
\begin{theorem}\label{thm:lb1} 
Consider any $p\in [0,1]$ and $K\geq 1$. 
For any learning algorithm, there exists an MDP with $S$ states, $A$ actions and horizon $H$
	obeying $v^*\leq  Hp$ and 
	%
	\begin{equation}
		\mathbb{E}\big[\mathsf{Regret}(K)\big]  \gtrsim \min\big\{ \sqrt{SAH^3Kp},\, KHp \big\} .
	\end{equation}
	%
\end{theorem}
%
In fact, the construction of the hard instance (as required in Theorem~\ref{thm:lb1}) is quite simple. 
Design a new branch with $0$ reward and set the probability of reaching this branch to be $1-p$. 
Also, with probability $p$, we direct the learner to a hard instance with regret $\Omega(\min\{\sqrt{SAH^3Kp},KpH\})$ and optimal value $H$. This guarantees that the optimal value $v^* \leq Hp$ and that the expected regret is at least $\Omega(\min\{ \sqrt{SAH^3Kp},KHp   \}) \gtrsim \min\{ \sqrt{SAH^2Kv^*},Kv^*   \}$.  
See Appendix~\ref{app:lb} for more details.


%Recall that in Section~\ref{sec:pre}, we assume the reward satisfies Assumption~\ref{assum1}, which means $R_{h,s,a}\in \Delta ([0,H])$ and $\sum_{h=1}^H r_h \leq H$. 
%


%
\paragraph{Cost-based regret bounds.} 
%
Next, we turn to the cost-aware regret bound as in Corollary~\ref{thm:cost}. 
Note that all other results except for Corollary~\ref{thm:cost} are about rewards as opposed to cost. 
In order to facilitate discussion, let us first formally introduce the cost-based scenarios. 



Suppose that the reward distributions $\{R_{h,s,a}\}_{(s,a,h)}$ are replaced with the cost distributions $\{C_{h,s,a}\}_{(s,a,h)}$, 
where each distribution $C_{h,s,a}\in \Delta([0,H])$ has mean $c_h(s,a)$. 
In the $h$-th step of an episode, the learner pays an immediate cost $c_h\sim C_{h,s_h,a_h}$ instead of receiving an immediate reward $r_h$,  
and the objective of the learner is instead to minimize the total cost $\sum_{h=1}^H c_h$ (in an expected sense). 
The optimal cost quantity $c^*$ is then defined as 
%
\begin{equation}
	c^* \coloneqq \min_{\pi}\mathbb{E}_{\pi,s_1\sim \mu}\bigg[\sum_{h=1}^H c_h \bigg]. 
\label{eq:defn-cstar-formal}
\end{equation}
%
Similarly, we can re-define the $Q$-function and value function as follows:
%
\begin{align}
	 \mathtt{Q}_{h}^{\pi}(s,a) &\coloneqq \mathbb{E}_{\pi}\left[\sum_{h'=h}^H c_{h'} \,\Big|\, (s_h,a_h)=(s,a)\right] ,
	 && \forall (s,a,h)\in \mathcal{S}\times \mathcal{A}\times [H], \nonumber
	\\ 
	\mathtt{V}_h^{\pi}(s) &\coloneqq \mathbb{E}_{\pi}\left[\sum_{h'=h}^H c_{h'} \,\Big|\, s_h = s\right],
	&& \forall (s,h)\in \mathcal{S}\times \times [H], 
	\nonumber
\end{align}
%
where we use different fonts to differentiate them from the original Q-function and value function. 
The optimal cost function is then given by $\mathtt{Q}_h^*(s,a) = \min_{\pi}\mathtt{Q}_h^{\pi}(s,a)$ and $\mathtt{V}_h^*(s)=  \min_{\pi}\mathtt{V}_h^{\pi}(s)$. 
Given the definitions above, 
we overload the notation $\mathsf{Regret}(K)$ to denote the regret for the cost-based scenario as 
$$
	\mathsf{Regret}(K) \coloneqq \sum_{k=1}^K \Big( \mathtt{V}^{\pi^k}_{1}(s_1^k) -  \mathtt{V}_1^*(s_1^k) \Big).
$$
%
One can also simply regard the cost minimization problem as  reward maximization with negative rewards by choosing $r_h = -c_h$. 
This way allows us to apply Algorithm~\ref{alg:main} directly, except that \eqref{eq:updateq} is replaced by 
%
\begin{align}
Q_h(s,a) \,\leftarrow\, \max\left\{\min \left\{ \widehat{r}_h(s,a) + \widehat{P}_{s,a,h}V_{h+1}+b_h(s,a), \, 0 \right\} ,\,-H \right\}.
	\label{eq:updatecost}
\end{align}
%
%
%We restate Theorem~\ref{thm:cost} as below.
%
%\noindent \textbf{Theorem 3. (Restated)}
%\emph{
%For any $K \ge 1$, with probability $1-\delta$, the regret of Algorithm~\ref{alg:main} is} 
%$$\widetilde{O}\left(\min\{\sqrt{SAH^2K c^*}+SAH^2,K(H-c^*) \}\right).$$
%
%
Note that the proof of Corollary~\ref{thm:cost} closely resembles that of Theorem~\ref{thm:first}, 
which can be found in Appendix~\ref{app:cost}.
%
%, except that $\mathbb{E}[\sum_{k=1}^K\sum_{h=1}^H r_h^k] \leq Kv^*$, while $\mathbb{E}[\sum_{k=1}^K\sum_{h=1}^H c_h^k] \geq Kc^*$. Refer to Appendix~\ref{app:cost} for more details.



To confirm the tightness of  Corollary~\ref{thm:cost}, we develop the following matching lower bound, 
which basically employs the same hard instance as in the proof of Theorem~\ref{thm:lb1}. 
%
\begin{corollary}\label{corollary:costlb}
Consider any $p\in [0,\frac{1}{4}]$ and any $K\geq 1$.
For any algorithm, one can construct an MDP with $S$ states, $A$ actions and horizon $H$ 
 obeying $c^*= \Theta(Hp)$ and 
	%
	\[
		\mathbb{E}\big[\mathsf{Regret}(K)\big] \gtrsim \min\big\{ \sqrt{SAH^3Kp}+SAH^2, \,KH(1-p) \big\} .
	\]
	%
\end{corollary}





\paragraph{Variance-dependent regret bound.}
%\input{variance}

The final regret bound presented in Theorem~\ref{thm:var} depends on a sort of variance metrics. 
Towards this end, let us first make precise the variance metrics of interest:
%
\begin{itemize}
	\item[(i)] The first variance metric is defined as 
		%
		\begin{equation}
			\mathrm{var}_1 \coloneqq \max_{\pi}\mathbb{E}_{\pi}\Bigg[\sum_{h=1}^H \mathbb{V}\big(P_{s_h,a_h,h},V_{h+1}^*\big)+\sum_{h=1}^H \mathrm{Var}\big(R_h(s_h,a_h)\big) \Bigg],
			\label{eq:defn-var1}
		\end{equation}
		%
		where $\{(s_h,a_h)\}_{1\leq h\leq H}$ represents a sample trajectory under policy $\pi$. 
		This captures the maximal possible expected sum of variance with respect to the optimal value function $\{V_{h}^*\}_{h=1}^H$. 
		
	\item[(ii)] Another useful variance metric is defined as
		%
		\begin{equation}
			\mathrm{var}_2 \coloneqq \max_{\pi,s}\mathrm{Var}_{\pi}\bigg[\sum_{h=1}^H r_h \,\Big|\, s_1=s\bigg],
			\label{eq:defn-var2}
		\end{equation}
		%
		where $\{r_h\}_{1\leq h\leq H}$ denotes a sample sequence of immediate rewards under policy $\pi$. 
		This indicates the maximal possible variance of the accumulative reward. 
%
\end{itemize}
%
The interested reader is referred  to \citet{zhou2023sharp} for further discussion about these two metrics. 
Our final variance metric is then defined as
%
\begin{align}
	\mathrm{var} \coloneqq \min\big\{\mathrm{var}_1,\mathrm{var}_2 \big\} .
	\label{eq:defn-var-formal}
\end{align}
%


 %and restate Theorem~\ref{thm:var} as below.

%\noindent\textbf{Theorem 4. (Restated)}\emph{
% With probability $1-\delta$, the regret of Algorithm~\ref{alg:main} is at most $\tilde{O}(\min\{\sqrt{SAHK\mathrm{var}}+SAH^2,HK\})$.}


With the above metric $\mathrm{var}$ in mind, we can then revisit Theorem~\ref{thm:var}. 
When the transition model is fully deterministic, the regret bound in Theorem~\ref{thm:var} simplifies to $$\mathsf{Regret}(K)\leq \widetilde{O}\big(\min\big\{SAH^2,\,HK\big\}\big)$$ for any $K\geq 1$, which is roughly the cost of visiting each state-action pair. 
%
The full proof of Theorem~\ref{thm:var} is postponed to Appenndix~\ref{app:var}. 
%depend on tighter bounds for $T_2$, $T_4$, $T_5$ and $T_6$. Refer to Appenndix~\ref{app:var} for more details. 



To finish up, let us develop a matching lower bound to corroborate the tightness and optimality of Theorem~\ref{thm:var}. 
%
\begin{theorem}\label{thm:lb3}
Consider any $p\in [0,1]$ and any $K\geq 1$. For any algorithm, one can find an MDP with $S$ states, $A$ actions, and horizon $H$ satisfying $\max\{\frac{\mathrm{var}_1}{H^2},\frac{\mathrm{var}_2}{H^2}\}\leq p$ and 
	$$	
		\mathbb{E}\big[\mathsf{Regret}(K)\big]  \gtrsim \min\big\{\sqrt{SAH^3Kp}+SAH^2,\,KH\big\}.
	$$
\end{theorem}

The proof of Theorem~\ref{thm:lb3} resembles that of Theorem~\ref{thm:lb1}, except that we need to construct a hard instance when $K\leq SAH/p$. For this purpose, we construct a fully deterministic MDP  (i.e., all of its transitions are deterministic and all rewards are fixed), and show that the learner has to visit about half of the state-action-layer tuples in order to learn a near-optimal policy. 
The proof details are deferred to Appendix~\ref{app:lb}.











\section{Discussion}
\section{Conclusion and Future Work}
In this work, I design corruption-robust algorithms for the Lipschitz contextual search problem. I present the \emph{agnostic checking} technique and demonstrate its effectiveness in designing corruption-robust algorithms. There are several open problems for future research. First, in the algorithm I propose for pricing loss, the schedule for agnostic checks is fixed upfront. Can the learner design an adaptive checking schedule for the pricing loss? Second, this work assumes the learner has knowledge of the Lipschitz constant $L$. Can the learner design efficient no-regret algorithms without knowledge of $L$? 


% \simon{add discounted, horizon-free, model-free, Thompson sampling, multi-agent, function approximation}
% However, this technique is limited to the finite-horizon inhomogeneous MDP because the correlation is much more complicated beyond this case. 
%For example, for the discounted infinite-horizon MDP, \cite{li2020breaking} decoupled the empirical model $\hat{P}$ with the value function $V$ using the \emph{leave-one-out} technique given a simulator. 
%When it comes to the online case, a simple toy example below show the corelation between $\hat{P}$ and $V$. Saying that we aim to bound $()$



\section*{Acknowledgement}
We thank for Qiwen Cui for helpful discussions. 
Y.~Chen is supported in part by the Alfred P.~Sloan Research Fellowship, the Google Research Scholar Award, the AFOSR grants FA9550-19-1-0030 and FA9550-22-1-0198, 
the ONR grant N00014-22-1-2354,  and the NSF grants CCF-2221009 and CCF-1907661.  JDL acknowledges support of the ARO under MURI Award W911NF-11-1-0304,  the Sloan Research Fellowship, NSF CCF 2002272, NSF IIS 2107304,  NSF CIF 2212262, ONR Young Investigator Award, and NSF CAREER Award 2144994. 
SSD acknowledges the support of NSF IIS 2110170, NSF
DMS 2134106, NSF CCF 2212261, NSF IIS 2143493,
NSF CCF 2019844, and NSF IIS 2229881.


%\newpage
\appendix
\begin{comment}
\section{System Architecture}
\label{appendix:architecture}
\system has a novel modularized system architecture with three key components: 
\emph{StreamManager}, 
\emph{TxnManager} and \emph{TxnScheduler}. 
These components are instantiated in each thread locally.
The execution outline of \system is presented in Algorithm~\ref{alg:algo}.
Transactional stream processing is continuous and potentially never ends (Line 1$\sim$8).
The dependency resolution and execution of state transactions are separated into two non-overlapping phases by punctuations~\cite{Tucker:2003:EPS:776752.776780} (Line 2 and 5), which guarantees that no subsequent input event will have a smaller timestamp. 
Effectively, a batch of state transactions is collected during the first phase, and processed during the second phase.

In the first phase (i.e., stream processing phase), 
the \emph{StreamManager} conducts preprocessing for every input event ($e$). Similar to some prior works~\cite{tstream}, state transactions may be issued but not immediately processed during preprocessing (Line 3).
The \emph{pre\_processing} and \emph{post\_processing} functions are exposed as APIs to users.
The \emph{TxnManager} handles dependency resolution (Line 4) among state transactions and insert decomposed operations to construct a \tpg. We discuss the detailed two-phase \tpg construction process in Section~\ref{subsec:construction}.

In the second phase  (i.e., transaction processing phase), 
the \emph{TxnManager} is first involved again to refine (Line 6) the constructed \tpg with further dependency resolution.
The \emph{TxnScheduler} 
schedules operations for concurrent execution based on the constructed \tpg according to the three dimensions of scheduling decisions (Line 7). 
In particular, a scheduling decision model $M$ is instantiated based on the constructed \tpg (Line 14).
\textbf{\circled{1}} Guided by $M$, execution threads adopt an exploration strategy (Section~\ref{subsec:explore}) to explore the constructed \tpg for operations available to be scheduled constrained by dependencies. 
\textbf{\circled{2}} 
During exploration, one or multiple operations may be treated as the 
% basic 
unit of scheduling (Section~\ref{subsec:granularity}). 
Subsequently, \textbf{\circled{3}} every thread executes operation(s) in the unit of scheduling with various abort handling mechanisms (Section~\ref{subsec:abort_handling}).
Only when state transactions are processed (i.e., committed or aborted) can the associated input events be postprocessed (Line 8) by the \emph{StreamManager} based on transaction processing results.
\end{comment}

\begin{comment}
\begin{algorithm}
\footnotesize
    \KwData{$e$ \tcp{Input event}}
    \KwData{$txn_{ts}$ \tcp{State transaction}}
    \KwData{$G$ \tcp{The currently constructed TPG}}
    \While{!finish processing of input streams}{
        \eIf(\tcp*[h]{Phase 1}){\text{$e$ is not a $punctuation$}}{
                $txn_{ts}$ $\gets$ PRE\_Processing($e$)\;
                \textbf{TPG\_Construction}($G$, $txn_{ts}$)\; 
          }(\tcp*[h]{Phase 2}){
                \textbf{TPG\_Refinement}($G$)\; 
                \textbf{TXN\_Scheduling}($G$)\; 
                POST\_Processing()\;
          }
    }
    
    \SetKwFunction{FMain}{TPG\_Construction}
    \SetKwProg{Fn}{Function}{:}{}
    \Fn{\FMain{$G$, $txn_{ts}$}}{
        $O_{1..k}$ $\gets$ \textbf{Partition} $txn_{ts}$\;
        \ForEach{\text{operation $O_{i}$ $\in$ $O_{1..k}$}}{
            \textbf{Identify} its \ld\;
            $G$ $\gets$ $G$ + $O_{i}$ \;
        }
    }
    \SetKwFunction{FMain}{TPG\_Refinement}
    \SetKwProg{Fn}{Function}{:}{}
    \Fn{\FMain{$G$}}{
        \ForEach{\text{vertex $e_{i}$ $\in$ $G$}}{
            \textbf{Identify} its \td, \pd\;
        }
    }
    
    \SetKwFunction{FMain}{TXN\_Scheduling}
    \SetKwProg{Fn}{Function}{:}{}
    \Fn{\FMain{$G$}}{
        $M$ $\gets$ Instantiated with $G$;\tcp{A decision model}
        \While{!finish scheduling of $G$
        }{
          \textbf{\circled{2}} $Scheduling Unit$ $\gets$ \textbf{\circled{1}} \emph{Explore}($G$, $M$)\; 
            \textbf{\circled{3}} \emph{Execute with Abort Handling} ($Scheduling Unit$)\; 
        }
    }
  \caption{Execution Outline of \system}
  \label{alg:algo}
\end{algorithm}
\end{comment}


\bibliography{ref}
\bibliographystyle{apalike}



\end{document}

