

In this section, we gather several useful results that prove useful in our analysis. We use $\mathds{1}\{\mathcal{E}\}$ to denote the indicator of the event $\mathcal{E}$.
%
The first result below is a user-friendly version of the celebrated Freedman inequality \citep{freedman1975tail}, 
a martingale counterpart to the Bernstein inequality. See \citet[Lemma~11]{zhang2020model} for the proof. 
%
\begin{lemma}[Freedman's inequality]\label{lemma:self-norm}
	Let $(M_n)_{n\geq 0}$ be a martingale such that $M_0=0$ and $|M_n-M_{n-1}|\leq c$  $(\forall n\geq 1)$ 
	hold for some quantity $c>0$. 
	Define $\mathsf{Var}_{n} \coloneqq \sum_{k=1}^n \mathbb{E}\left[  (M_{k}-M_{k-1})^2 \mymid \mathcal{F}_{k-1}\right]$ for every $n\geq 0$, where $\mathcal{F}_k$ is the $\sigma$-algebra generated by $(M_1,...,M_{k})$. Then for any integer $n\geq 1$ and any $\epsilon,\delta>0$, one has 
%
\begin{align}
	\mathbb{P} \left[       |M_n|\geq 2\sqrt{2}\sqrt{\mathsf{Var}_n \log\frac{1}{\delta} } +2\sqrt{\epsilon \log\frac{1}{\delta} } +2c\log\frac{1}{\delta} \right]\leq 2\left(\log_2\left(\frac{nc^2}{\epsilon}\right) +1 \right)\delta.\nonumber
\end{align}
%
\end{lemma}
%
Next, letting $\mathsf{Var}(X)$ represent the variance of $X$, 
we record a basic inequality connecting $\mathsf{Var}(X^2)$ with $\mathsf{Var}(X)$ for any bounded random variable $X$. 
%
\begin{lemma}[Lemma 30 in \citep{chen2021implicit}]\label{lemma:sqv}
	Let $X$ be a random variable, and denote by $C_{\max}$ the largest possible value of $X$. 
	%Let $\mathsf{Var}(X)$ denote the variance of $X$. 
	Then we have $\mathsf{Var}(X^2)\leq 4 C_{\max}^2 \mathsf{Var}(X)$.
\end{lemma}
%
Now, we turn to an intimate connection between the sum of a sequence of bounded non-negative random variables and the sum of their associated conditional random variables (with each random variable conditioned on the past), which is a consequence of basic properties about supermartingales.   
%
\begin{lemma}[Lemma 10 in \citep{zhang2022horizon}]\label{lemma:con}
Let $X_1,X_2,\ldots$ be a sequence of random variables taking value in $[0,l]$. 
For any $k\geq 1$, let $\mathcal{F}_k$ be the $\sigma$-algebra generated by $(X_1,X_2,\ldots,X_k)$, and define 
	$Y_k \coloneqq \mathbb{E}[X_k \mymid \mathcal{F}_{k-1}]$. Then for any $\delta>0$, we have 
%
\begin{align}
& \mathbb{P}\left[ \exists n, \sum_{k=1}^n X_k \geq  3\sum_{k=1}^n Y_k+ l\log\frac{1}{\delta}\right]\leq \delta\nonumber
\\  & \mathbb{P}\left[  \exists n,  \sum_{k=1}^n Y_k \geq 3\sum_{k=1}^n X_k + l\log\frac{1}{\delta}  \right]    \leq \delta .\nonumber 
\end{align}
%
\end{lemma}


%
The next two lemmas are concerned with concentration inequalities for the sum of i.i.d.~bounded random variables: 
the first one is a version of the Bennet inequality, and the second one is an empirical Bernstein inequality (which replaces the variance in the standard Bernstein inequality with the empirical variance). 
%
\begin{lemma}[Bennet's inequality]\label{bennet}
Let $Z,Z_1,...,Z_n$  be i.i.d.~random variables with values in $[0,1]$ and let $\delta>0$. Define $\mathbb{V}Z = \mathbb{E}\left[(Z-\mathbb{E}Z)^2 \right]$. Then one has
%
\begin{align}
\mathbb{P}\left[ \left|\mathbb{E}\left[Z\right]-\frac{1}{n}\sum_{i=1}^n Z_i  \right| > \sqrt{\frac{  2\mathbb{V}Z \log(2/\delta)}{n}} +\frac{\log(2/\delta)}{n} \right]\leq \delta.\nonumber
\end{align}
%
\end{lemma}
%
%The following lemma presents an empirical Bernstein inequality that allows one to replace the variance parameter in the standard Bernstein inequality with the empirical variance. 
%
\begin{lemma}[Theorem 4 in  \citet{maurer2009empirical}]\label{empirical bernstein}
Consider any $\delta>0$ and any integer $n\geq 2$. 
Let $Z,Z_1,...,Z_n$  be a collection of i.i.d.~random variables falling within $[0,1]$. 
Define the empirical mean $\overline{Z} \coloneqq \frac{1}{n}\sum_{i=1}^n Z_{i}$ and empirical variance $\widehat{V}_n  \coloneqq \frac{1}{n}\sum_{i=1}^n (Z_i- \overline{Z})^2$. Then we have
%
\begin{align}
\mathbb{P}\left[ \left|\mathbb{E}\left[Z\right]-\frac{1}{n}\sum_{i=1}^n Z_i  \right| > \sqrt{\frac{  2\widehat{V}_n \log(2/\delta)}{n-1}} +\frac{7\log(2/\delta)}{3(n-1)} \right] \leq \delta.\nonumber
\end{align}
%
\end{lemma}


Moreover, we record a simple fact concerning the visitation counts $\{N_h^k(s_h^k,a_h^k)\}$. 
%
\begin{lemma}\label{lemma:doubling}
%
Recall the definition of $N_h^k(s_h^k,a_h^k)$ in Algorithm~\ref{alg:main}. It holds that
%
\begin{align}
\sum_{k=1}^K \sum_{h=1}^H \frac{1}{\max\{ N_h^k(s_h^k,a_h^k),1\}}\leq 2SAH\log_2 K
\end{align}
%
\end{lemma}
\begin{proof}
%
In view of the doubling batch update rule, it is easily seen that: for any given $(s,a,h)$, 
%
\begin{align}
	\sum_{k=1}^K \frac{1}{\max\{ N_h^k(s_h^k,a_h^k),1\}}  \mathds{1}\Big\{(s,a)=\big(s_h^k,a_h^k \big) \Big\} \leq 2\log_2 K,
\end{align}
%
since each $(s,a,h)$ is associated with at most $\log_2 K $ epochs. 
Summing over $(s,a,h)$ completes the proof.
\end{proof}


As it turns out, Lemma~\ref{lemma:doubling} together with the Freedman inequality allows one to control the difference between the empirical rewards and the true mean rewards, as stated below. 
%
\begin{lemma}\label{lemma:bdempr}
With probability exceeding $1-2SAHK\delta'$, it holds that
%
\begin{align}
  \sum_{k=1}^{K}\sum_{h=1}^{H}\left|\widehat{r}_{h}^{k}(s_{h}^{k},a_{h}^{k})-r_{h}(s_{h}^{k},a_{h}^{k})\right|
% & \leq4SAH^{2}+4\sqrt{\sum_{k=1}^{K}\sum_{h=1}^{H}\frac{H\log\frac{1}{\delta'}}{N_{h}^{k}(s_{h}^{k},a_{h}^{k})}}\cdot\sqrt{\sum_{k=1}^{K}\sum_{h=1}^{H}r_{h}(s_{h}^{k},a_{h}^{k})}+24\sum_{k=1}^{K}\sum_{h=1}^{H}\frac{H\log\frac{1}{\delta'}}{N_{h}^{k}(s_{h}^{k},a_{h}^{k})}.\nonumber
	& \leq 4\sqrt{2SAH^{2}(\log_{2}K)\log\frac{1}{\delta'}}\sqrt{\sum_{k=1}^K\sum_{h=1}^Hr_{h}(s_{h}^{k},a_{h}^{k})}+52SAH^{2}(\log_{2}K)\log\frac{1}{\delta'} ;
	\nonumber \\
	\sum_{k=1}^{K}\sum_{h=1}^{H}\widehat{r}_{h}^{k}(s_{h}^{k},a_{h}^{k})&\leq2\sum_{k=1}^{K}\sum_{h=1}^{H}r_{h}(s_{h}^{k},a_{h}^{k})+60SAH^{2}(\log_{2}K)\log\frac{1}{\delta'}. \nonumber
\end{align}
%
\end{lemma}
%
As an immediate consequence of Lemma~\ref{lemma:bdempr} and the basic fact $\sum_{k,h}r_{h}(s_{h}^{k},a_{h}^{k})\leq KH$, we have
%
\begin{align}
\sum_{k=1}^{K}\sum_{h=1}^{H}\widehat{r}_{h}^{k}(s_{h}^{k},a_{h}^{k}) & \leq2\sum_{k=1}^{K}\sum_{h=1}^{H}r_{h}(s_{h}^{k},a_{h}^{k})+60SAH^{2}(\log_{2}K)\log\frac{1}{\delta'}\nonumber \\
 & \leq2KH+60SAH^{2}(\log_{2}K)\log\frac{1}{\delta'}\leq3KH\label{eq:sum-empirical-r-UB}
\end{align}
%
with probability exceeding $1-2SAHK\delta'$, 
where the last inequality holds true under the assumption~\ref{eq:assumption-K-proof}. 
%
\begin{proof}[Proof of Lemma~\ref{lemma:bdempr}]
%	
In view of Lemma~\ref{empirical bernstein} and the union bound, with probability $1-2SAHK\delta'$ we have 
%
\begin{align}
\left| \widehat{r}_{h}^{k}(s,a)-r_{h}(s,a) \right| & \leq2\sqrt{2}\sqrt{\frac{\left(\widehat{\sigma}_{h}^{k}(s_{h}^{k},a_{h}^{k})-\big(\widehat{r}_{h}^{k}(s_{h}^{k},a_{h}^{k})\big)^{2}\right)\log\frac{1}{\delta'}}{N_{h}^{k}(s,a)}}+\frac{28H\log\frac{1}{\delta'}}{3N_{h}^{k}(s,a)} \notag\\
 & \leq2\sqrt{2}\sqrt{\frac{H\widehat{r}_{h}^{k}(s,a)\log\frac{1}{\delta'}}{N_{h}^{k}(s,a)}}+\frac{28H\log\frac{1}{\delta'}}{3N_{h}^{k}(s,a)} 
	\notag
\end{align}
%
simultaneously for all $(s,a,h,k)$ obeying $N_h^k(s,a)>2$, 
	where we take advantage of the basic fact $\widehat{\sigma}_h^k(s_h^k,a_h^k) \leq H\widehat{r}_h^k(s,a)$ (since each immediate reward is upper bounded by $H$). 
Solve the inequality above to obtain 
%
\begin{align}
\left|\widehat{r}_h^k(s,a) - r_h(s,a)\right|\leq  4\sqrt{\frac{Hr_h(s,a)\log \frac{1}{\delta'}}{N_h^k(s,a)}} + 24\frac{H\log \frac{1}{\delta'}}{N_h^k(s,a)}.\label{eq:bdt4_0}
\end{align}
%
It is then seen that
%
\begin{align}
  \sum_{k,h}\left|\widehat{r}_{h}^{k}(s_{h}^{k},a_{h}^{k})-r_{h}(s_{h}^{k},a_{h}^{k})\right|
 & \leq4SAH^{2}+\sum_{k,h}\left(4\sqrt{\frac{Hr_{h}(s_{h}^{k},a_{h}^{k})\log\frac{1}{\delta'}}{N_{h}^{k}(s_{h}^{k},a_{h}^{k})}}+24\frac{H\log\frac{1}{\delta'}}{N_{h}^{k}(s_{h}^{k},a_{h}^{k})}\right)\nonumber\\
 & \leq4SAH^{2}+4\sqrt{\sum_{k,h}\frac{H\log\frac{1}{\delta'}}{N_{h}^{k}(s_{h}^{k},a_{h}^{k})}}\cdot\sqrt{\sum_{k,h}r_{h}(s_{h}^{k},a_{h}^{k})}+24\sum_{k,h}\frac{H\log\frac{1}{\delta'}}{N_{h}^{k}(s_{h}^{k},a_{h}^{k})}. \nonumber
\end{align}
%
Here, the second inequality arises from Cauchy-Schwarz,  
%the third inequality is a consequence of  Lemma~\ref{lemma:doubling}, 
whereas the term $4SAH^2$ accounts for those state-action pairs with $N_h^k(s,a)\leq 2$ (since there are at most $2SAH$ such occurances and it holds that $\left|\widehat{r}_{h}^{k}(s_{h}^{k},a_{h}^{k})-r_{h}(s_{h}^{k},a_{h}^{k})\right|\leq 2H$). 
This together with Lemma~\ref{lemma:doubling} then leads to 
%
\begin{align*}
  \sum_{k,h}\left|\widehat{r}_{h}^{k}(s_{h}^{k},a_{h}^{k})-r_{h}(s_{h}^{k},a_{h}^{k})\right|
 & \leq4SAH^{2}+4\sqrt{2SAH^{2}(\log_{2}K)\log\frac{1}{\delta'}}\sqrt{\sum_{k,h}r_{h}(s_{h}^{k},a_{h}^{k})}+48SAH^{2}(\log_{2}K)\log\frac{1}{\delta'}
	\nonumber \\
 & \leq 4\sqrt{2SAH^{2}(\log_{2}K)\log\frac{1}{\delta'}}\sqrt{\sum_{k,h}r_{h}(s_{h}^{k},a_{h}^{k})}+52SAH^{2}(\log_{2}K)\log\frac{1}{\delta'}.
	\nonumber
\end{align*}
%
Moreover, the AM-GM inequality implies that
%
\begin{align*}
\sum_{k,h}\widehat{r}_{h}^{k}(s_{h}^{k},a_{h}^{k})-\sum_{k,h}r_{h}(s_{h}^{k},a_{h}^{k}) & \leq \sum_{k=1}^{K}\sum_{h=1}^{H}r_{h}(s_{h}^{k},a_{h}^{k})+8SAH^{2}(\log_{2}K)\log\frac{1}{\delta'}+52SAH^{2}(\log_{2}K)\log\frac{1}{\delta'}
\end{align*}
\[
\Longrightarrow\qquad\sum_{k,h}\widehat{r}_{h}^{k}(s_{h}^{k},a_{h}^{k})\leq 2\sum_{k,h}r_{h}(s_{h}^{k},a_{h}^{k})+60SAH^{2}(\log_{2}K)\log\frac{1}{\delta'},
\]
thus concluding the proof.    
\end{proof}


