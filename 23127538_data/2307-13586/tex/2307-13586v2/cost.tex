

We now turn to the proof of Theorem~\ref{thm:cost}. 
For notational convenience, we shall use $r$ to denote the negative cost (namely, $r_h= -c_h$, $\widehat{r}_h=-\widehat{c}_h$, and so on) throughout this section. 
We shall also use the following notation (and similar quantities like $Q_h^k$, $V_h^k$, $\ldots$)
%
\begin{align}
	Q_h(s,a)&\leftarrow \max\left\{\min \Big\{ \widehat{r}_h(s,a) + \widehat{P}_{s,a,h}V_{h+1}+b_h(s,a), 0 \Big\} ,-H \right\},\nonumber \\
	V_h(s) & \leftarrow \max_a Q_h(s,a),
	\nonumber
\end{align}
%
in order to be consistent with the reward-based setting. 


Akin to the proof of Theorem~\ref{thm:first},  
we need to bound the quantities $T_1,\ldots,T_9$ introduced previously (see \eqref{eq:decomposition}, \eqref{eq:defn-T56-proof} and \eqref{eq:defn-T789-proof}). 
We note that the analysis for $T_1$, $T_3$, $T_7$, $T_8$ and $T_9$ in Appendix~\ref{sec:appfirst} readily applies to the negative reward case herein. 
Thus, it suffices to develop bounds on $T_2$, $T_4$, $T_5$ and $T_6$ to capture their dependency on $c^{\star}$, 
which forms the main content of the remainder of  this section.



\paragraph{Bounding $T_2$.}
%
Recall from \eqref{eq:T2-decomposition-main1} that 
%
\begin{align}
T_{2} 
	%& =\sum_{k=1}^{K}\sum_{h=1}^{H}b_{h}^{k}(s_{h}^{k},a_{h}^{k})\nonumber\\
 & =\frac{460}{9}\sum_{k,h}\sqrt{\frac{\mathbb{V}\big(\widehat{P}_{s_{h}^{k},a_{h}^{k},h}^{k},V_{h+1}^{k}\big)\log\frac{1}{\delta'}}{N_{h}^{k}(s_{h}^{k},a_{h}^{k})}}+\nonumber\\
 & \qquad2\sqrt{2}\sum_{k,h}\sqrt{\frac{\left(\widehat{\sigma}_{h}^{k}(s_{h}^{k},a_{h}^{k})-\big(\widehat{r}_{h}^{k}(s_{h}^{k},a_{h}^{k})\big)^{2}\right)\log\frac{1}{\delta'}}{N_{h}^{k}(s_{h}^{k},a_{h}^{k})}} +\frac{544}{9}\sum_{k,h}\frac{H\log\frac{1}{\delta'}}{N_{h}^{k}(s_{h}^{k},a_{h}^{k})}.
\label{eq:cc1}
\end{align}
%
In what follows, let us bound the three terms on the right-hand side of \eqref{eq:cc1} separately. 
\begin{itemize}
	\item 
For the first and the third terms on the right-hand side of \eqref{eq:cc1}, invoking the Cauchy-Schwarz inequality and Lemma~\ref{lemma:doubling} gives
%
\begin{align}
\sum_{k,h}\sqrt{\frac{\mathbb{V}\big(\widehat{P}_{s_{h}^{k},a_{h}^{k},h}^{k},V_{h+1}^{k}\big)\log\frac{1}{\delta'}}{N_{h}^{k}(s_{h}^{k},a_{h}^{k})}} & \leq\sqrt{2SAH(\log_{2}K)\Big(\log\frac{1}{\delta'}\Big)\sum_{k,h}\mathbb{V}\big(\widehat{P}_{s_{h}^{k},a_{h}^{k},h}^{k},V_{h+1}^{k}\big)}\nonumber\\
 & =\sqrt{2SAH(\log_{2}K)\Big(\log\frac{1}{\delta'}\Big)T_{5}}\label{eq:cterm1} 
 \end{align}
 %
with $T_5$ defined in \eqref{eq:defn-T5-proof},  and in addition, 
 %
 \begin{align}
	 \sum_{k,h} \frac{H\log \frac{1}{\delta'}}{N_h^k(s_h^k,a_h^k)} \leq 2SAH^2(\log_2K)\log \frac{1}{\delta'}.\label{eq:cterm2}
\end{align}
%

\item 
Let us turn to the second term on the right-hand side of \eqref{eq:cc1}. Observing the basic fact that
%
\begin{align}
 \widehat{\sigma}_h^k(s_h^k,a_h^k) - \big(\widehat{r}_h^k(s_h^k,a_h^k)\big)^2 \leq  -H \widehat{r}_h^k(s_h^k,a_h^k),\nonumber
\end{align}
%
we can combine it with Lemma~\ref{lemma:doubling} to derive 
%
\begin{align}
 & \sqrt{\frac{\left(\widehat{\sigma}_{h}^{k}(s_{h}^{k},a_{h}^{k})-\big(\widehat{r}_{h}^{k}(s_{h}^{k},a_{h}^{k})\big)^{2}\right)\log\frac{1}{\delta'}}{N_{h}^{k}(s_{h}^{k},a_{h}^{k})}}\leq\sqrt{2SAH(\log_{2}K)\log\frac{1}{\delta'}}\sqrt{H\sum_{k,h}-\widehat{r}_{h}^{k}(s_{h}^{k},a_{h}^{k})}\nonumber\\
 & \leq\sqrt{2SAH^{2}(\log_{2}K)\log\frac{1}{\delta'}}\sqrt{-T_{4}+3Kc^{\star}+\sum_{k=1}^{K}\Big(-V_{1}^{\pi^{k}}(s_{1}^{k})+V_{1}^{\star}(s_{1}^{k})\Big)+\sum_{k=1}^{K}\Big(-V_{1}^{\star}(s_{1}^{k})-3c^{\star}\Big)},\label{eq:cterm3}
\end{align}
%
where the last inequality invokes the definition of $T_4$ (see \eqref{eq:decomposition}). 
%
By virtue of Lemma~\ref{lemma:con} and the definition \eqref{eq:defn-cstar-formal} of $c^{\star}$, one can show that
%
\begin{align}
\sum_{k=1}^K -V_1^{\star}(s_1^k)\leq 3Kc^{\star} + H\log \frac{1}{\delta'}\nonumber
\end{align}
%
with probability exceeding $1-\delta'$. 
In addition, we note that
%
\begin{align}
\sum_{k=1}^K \Big(-V_1^{\pi^k}(s_1^k) +V_1^{\star}(s_1^k)  \Big) = \mathsf{Regret}(K) = T_1+T_2+T_3+T_4.
\end{align}
%
Taking these properties together with \eqref{eq:cterm3} yields
%
\begin{align*}
 & \sqrt{\frac{\left(\widehat{\sigma}_{h}^{k}(s_{h}^{k},a_{h}^{k})-\big(\widehat{r}_{h}^{k}(s_{h}^{k},a_{h}^{k})\big)^{2}\right)\log\frac{1}{\delta'}}{N_{h}^{k}(s_{h}^{k},a_{h}^{k})}}\nonumber\\
 & \qquad\leq\sqrt{2SAH^{2}(\log_{2}K)\log\frac{1}{\delta'}}\sqrt{T_{1}+T_{2}+T_{3}+2|T_{4}|+3Kc^{\star}+H\log\frac{1}{\delta'}}
\end{align*}
%
\end{itemize}
%
Putting the above results together, we can deduce that, with probability exceeding $1-\delta'$,
%
\begin{align}
	T_2 &  \leq 90\sqrt{SAH(\log_2K)\Big(\log\frac{1}{\delta'}\Big)T_5}  \nonumber\\ &  + 4\sqrt{SAH^2(\log_2K)\log\frac{1}{\delta'}}\sqrt{T_1+T_2+T_3+2|T_4|+3Kc^{\star}+H\log\frac{1}{\delta'}} + 130 SAH^2(\log_2 K)\log \frac{1}{\delta'}.\label{eq:ccbt2}
\end{align}


\paragraph{Bounding $T_4$.}
%
When it comes to the quantity $T_4$, we recall that
%
\begin{align}
 T_4 
	%& = \sum_{k=1}^K \left( \sum_{h=1}^H \widehat{r}_h^k(s_h^k,a_h^k)-V_{1}^{\pi^k}(s_1^k)\right) \nonumber
 %\\ 
	& = \underset{\eqqcolon\,\widecheck{T}_1}{\underbrace{ \sum_{k=1}^K \left( \sum_{h=1}^H \widehat{r}_h^k(s_h^k,a_h^k)-r_{h}(s_h^k,a_h^k)\right) }} 
	+ \underset{\eqqcolon\,\widecheck{T}_2}{\underbrace{ \sum_{k=1}^K \left( \sum_{h=1}^H r_h(s_h^k,a_h^k) - V_{1}^{\pi^k}(s_1^k) \right) }}.
	\label{eq:T4-1234567}
 \end{align}
%
%Also recall that $\widecheck{T}_1 =\sum_{k=1}^K \left( \sum_{h=1}^H \widehat{r}_h^k(s_h^k,a_h^k)-r_{h}(s_h^k,a_h^k)\right)  $ and $\widecheck{T}_2 =\sum_{k=1}^K \left( \sum_{h=1}^H r_h(s_h^k,a_h^k) - V_{1}^{\pi^k}(s_1^k) \right) $. 
%
To control $T_4$, we first make note of the following result that bounds the empirical reward (for the case with negative rewards), 
which assists in bounding the term $\widecheck{T}_1$.
%
\begin{lemma}\label{lemma:bdempc}
With probability at least $1-2SAHK\delta'$, it holds that
\begin{align}
%& \sum_{k=1}^K \sum_{h=1}^H \left| \widehat{r}_h^k(s_h^k,a_h^k) - r_h(s_h^k,a_h^k)\right|\nonumber
%\\ & \leq  4SAH^2 +4\sqrt{\sum_{k=1}^K\sum_{h=1}^H \frac{H\log\frac{1}{\delta'}}{N_h^k(s_h^k,a_h^k)}}\cdot \sqrt{\sum_{k=1}^K \sum_{h=1}^H -r_h(s_h^k,a_h^k)}+24\sum_{k=1}^K\sum_{h=1}^H\frac{H\log\frac{1}{\delta'}}{N_h^k(s_h^k,a_h^k)}.\nonumber
 & \sum_{k=1}^{K}\sum_{h=1}^{H}\left|\widehat{r}_{h}^{k}(s_{h}^{k},a_{h}^{k})-r_{h}(s_{h}^{k},a_{h}^{k})\right|\nonumber\\
	& \leq 4\sqrt{2SAH^{2}(\log_{2}K)\log\frac{1}{\delta'}}\cdot\sqrt{\sum_{k=1}^{K}\sum_{h=1}^{H} \big(-r_{h}(s_{h}^{k},a_{h}^{k}) \big)}+52SAH^{2}(\log_{2}K)\log\frac{1}{\delta'}.\nonumber
\end{align}
\end{lemma}
%
\begin{proof}
The proof basically follows the same arguments as in the proof of Lemma~\ref{lemma:bdempr}, except that $r$ is now replaced with $-r$.
\end{proof}
%
\noindent 
Lemma~\ref{lemma:bdempc} tells us that with probability at least $1-3SAHK\delta'$,  
%
\begin{align}
	|\widecheck{T}_1 |& \leq 4\sqrt{2SAH^{2}(\log_{2}K)\log\frac{1}{\delta'}}\cdot\sqrt{\sum_{k=1}^{K}\sum_{h=1}^{H}\big(-r_{h}(s_{h}^{k},a_{h}^{k}) \big)}+52SAH^{2}(\log_{2}K)\log\frac{1}{\delta'} \nonumber
	\\ & \leq 4\sqrt{2SAH^2(\log_2 K)}\cdot \sqrt{-\widecheck{T}_2 + 3Kc^{\star}+\sum_{k=1}^K \big(-V_1^{\star}(s_1^k)- 3c^{\star}\big) } + 52SAH^2(\log_2 K)\log \frac{1}{\delta'} \nonumber
  \\ & \leq 4\sqrt{2SAH^2(\log_2 K)}\cdot \sqrt{\widecheck{T}_2 + 3Kc^{\star} } + 60SAH^2(\log_2 K)\log \frac{1}{\delta'} .\label{eq:cct1}
\end{align}
%
Here, the last line uses the fact (see Lemma~\ref{lemma:con}) that, with probability exceeding $1-\delta'$, 
%
\begin{align}
	\sum_{k=1}^K \big(-V_1^{\star}(s_1^k) \big) \leq 3Kc^{\star} + H\log \frac{1}{\delta'}. \label{eq:addc}
\end{align}
%

In addition, the Freedman inequality in Lemma~\ref{lemma:self-norm} combined with \eqref{eq:addc} implies that, with probability at least $1-3SAHK\delta$,
%
\begin{align}
|\widecheck{T}_2| & \leq 2\sqrt{2\sum_{k=1}^K \mathbb{E}_{\pi^k}\left[\left(\sum_{h=1}^H r_h(s_h,a_h) \right)^2 \,\Big|\, s_1=s_1^k\right]\log \frac{1}{\delta} }+3H\log \frac{1}{\delta} \nonumber
\\ & \leq 2\sqrt{2H\sum_{k=1}^K \mathbb{E}_{\pi^k}\left[\sum_{h=1}^H  -r_h(s_h,a_h) \,\Big|\, s_1=s_1^k \right]\log \frac{1}{\delta} }+3H\log \frac{1}{\delta} \nonumber
\\ & = 2\sqrt{2H\left( \sum_{k=1}^K \left(- V_1^{\pi^k}(s_1^k) +V_1^{\star}(s_1^k) \right) + \sum_{k=1}^K\left(-V_1^{\star}(s_1^k)-3c^{\star}\right)+3Kc^{\star}\right)\log \frac{1}{\delta}}
	+3H\log \frac{1}{\delta} \label{eq:cct1.5}
\\ & \leq 3Kc^{\star}  +T_1+T_2+T_3+T_4 + 9H\log \frac{1}{\delta}.
	\label{eq:cct2}
\end{align}
%
Combining \eqref{eq:cct1}, \eqref{eq:cct1.5} with \eqref{eq:cct2} reveals that, with probability at least $1-4SAHK\delta$,
%
\begin{align}
	& |\widecheck{T}_1|\leq     16\sqrt{SAH^2(Kc^{\star}+T_1+T_2+T_3+T_4)(\log_2 K)\log \frac{1}{\delta}}  + 200SAH^2(\log_2K)\log\frac{1}{\delta}\nonumber
\\ & |\widecheck{T}_2| \leq  2\sqrt{2H(3Kc^{\star}+T_1+T_2+T_3+T_4)\log \frac{1}{\delta}}+9H\log \frac{1}{\delta} .\nonumber
\end{align}
%
As a result, substitution into \eqref{eq:T4-1234567} leads to
%
\begin{align}
	|T_4| \leq 22\sqrt{SAH^2(Kc^{\star} +T_1+T_2+T_3+T_4) (\log_2 K)\log \frac{1}{\delta}}  + 209SAH^2 (\log_2 K)\log\frac{1}{\delta}.\label{eq:ccbt4}
\end{align}
%



\paragraph{Bounding $T_5$.}

Invoking the arguments in  \eqref{eq:boundt5} and recalling the update rule \eqref{eq:updatecost}, we obtain
%
\begin{align}
	T_{5} & \leq\sum_{k=1}^{K}\sum_{h=1}^{H}\Big\langle\widehat{P}_{s_{h}^{k},a_{h}^{k},h}^{k}-P_{s_{h}^{k},a_{h}^{k},h},\,\big(V_{h+1}^{k}\big)^{2}\Big\rangle+\sum_{k=1}^{K}\sum_{h=1}^{H}\Big\langle P_{s_{h}^{k},a_{h}^{k},h}-e_{s_{h+1}^{k}},\,\big(V_{h+1}^{k}\big)^{2}\Big\rangle\\
 & \qquad+2H\sum_{k=1}^{K}\sum_{h=1}^{H}\big[-r_{h}(s_{h}^{k},a_{h}^{k})\big].\nonumber
\end{align}
%
Moreover, we recall that
%
\begin{align}
	\sum_{k=1}^K \sum_{h=1}^H  \big[-r_h(s_h^k,a_h^k) \big] &  = - \widecheck{T}_2 -\sum_{k=1}^K V_1^{\pi^k}(s_1) \leq -\widecheck{T}_2 + \sum_{k=1}^K V_1^{\star}(s_1^k).\label{eq:cx1}
\end{align}
%
By virtue of \eqref{eq:addc}, one sees that with probability at least $1-5SAHK\delta$,
%
\begin{align}
	\sum_{k=1}^K \sum_{h=1}^H \big[ -r_h(s_h^k,a_h^k) \big] \leq  2\sqrt{2H(3Kc^{\star} + T_1+T_2+T_3+T_4)\log \frac{1}{\delta} } + 3Kc^{\star} + 10H\log \frac{1}{\delta}.
\end{align}
%
Consequently, we arrive at
%
\begin{align}
T_5 & \leq T_7 + T_8+2HT_2 + 4\sqrt{2H^3(3Kc^{\star} +T_1+T_2+T_3+T_4)\log \frac{1}{\delta}}+ 6HKc^{\star} + 20H^2\log \frac{1}{\delta}\label{eq:ccbt5}
\end{align}
%
with probability exceeding $1-5SAHK\delta$.



\paragraph{Bounding $T_6$.}
%
Invoking the arguments in  \eqref{eq:boundt5}, \eqref{eq:addc} and \eqref{eq:cx1}, and recalling the update rule \eqref{eq:updatecost}, 
we can demonstrate that 
%
\begin{align}
T_{6} & \leq2\sqrt{8T_{6}\log\frac{1}{\delta}}+3H^{2}\log\frac{1}{\delta}+2H\sum_{k=1}^{K}\sum_{h=1}^{H}\max\big\{\big\langle P_{s_{h}^{k},a_{h}^{k},h},V_{h+1}^{k}\big\rangle-V_{h}^{k}(s_{h}^{k}),0\big\}\nonumber\\
 & \leq2\sqrt{8T_{6}\log\frac{1}{\delta}}+3H^{2}\log\frac{1}{\delta}+2HT_{9}+2H\sum_{k=1}^{K}\sum_{h=1}^{H}\left[-r_{h}(s_{h}^{k},a_{h}^{k})\right]\nonumber\\
 & \leq2\sqrt{8T_{6}\log\frac{1}{\delta}}+3H^{2}\log\frac{1}{\delta}+2HT_{9}\nonumber\\
 & \qquad\qquad\qquad+2H\left(2\sqrt{2H(3Kc^{\star}+T_{1}+T_{2}+T_{3}+T_{4})\log\frac{1}{\delta}}+3Kc^{\star}+10H\log\frac{1}{\delta}\right)
	\label{eq:ccbt6}
\end{align}
%
with probability at least $1-3SAHK\delta$. 


\paragraph{Putting all this together.}


Armed with the preceding bounds, we are ready to establish the claimed regret bound. 
% 
By solving \eqref{eq:ccbt2},\eqref{eq:boundt3},\eqref{eq:ccbt4},\eqref{eq:ccbt5},\eqref{eq:ccbt6},\eqref{eq:boundt8},\eqref{eq:boundt1} and \eqref{eq:boundt7}, 
we can show that, with probability exceeding $1-100SAH^2K\delta$, 
%
\begin{align*}
	T_6 &\lesssim HKc^{\star}+BSAH^3, \\
	T_1 &\lesssim  \sqrt{BSAH^2Kc^{\star}}+BSAH^2, \\
	T_7+T_8 &\lesssim  \sqrt{BSAH^4Kc^{\star}}+BSAH^3, \\
	T_5 &\lesssim  HKc^{\star}+BSAH^2, \\
	T_2 &\lesssim  \sqrt{BSAH^2Kc^{\star}}+BSAH^2, \\
	T_3 &\lesssim  \sqrt{BHKc^{\star}}+BSAH^2. 
\end{align*}
%
%$T_6 = O(HKc^{\star}+BSAH^3)$, $T_1 = O(\sqrt{BSAH^2Kc^{\star}}+BSAH^2)$, $T_7,T_8 = O(\sqrt{BSAH^4Kc^{\star}}+BSAH^3)$, $T_5 = O(HKc^{\star}+BSAH^2)$, $T_2 = O(\sqrt{BSAH^2Kc^{\star}}+BSAH^2)$ and $T_3 = O(\sqrt{BHKc^{\star}}+BSAH^2)$. 
%
We then readily conclude that the total regret is bounded by $$O\big(\sqrt{BSAH^2Kc^{\star}}+BSAH^2 \big).$$ 
In addition, the regret bound is trivially upper bounded by $O\big(K(H-c^{\star})\big)$. 
The proof is thus completed by combining these two regret bounds and replacing $\delta'$ with $\frac{\delta}{100SAH^2K}$. 

