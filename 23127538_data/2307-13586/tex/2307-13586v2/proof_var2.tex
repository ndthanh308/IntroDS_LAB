



Following similar arguments as in the proof of Lemma~\ref{lemma:var1}, 
we focus on bounding $T_2,T_4,T_5$ and $T_6$ in terms of $\mathrm{var}_2$.



\subsubsection{Bounding $T_2$}

Recall that $\delta'$ is defined as $\delta' = \frac{\delta}{200SAH^2K^2}$, 
and that we have demonstrated in \eqref{eq:boundt2o-temp} that
%
\begin{align}
T_2 &\leq \frac{460}{9}\sqrt{2SAH T_5 (\log_2K)\log \frac{1}{\delta'} }\nonumber
\\ & \qquad +4\sqrt{SAH(\log_2K)\log \frac{1}{\delta'}}\sqrt{\sum_{k,h}\left(\widehat{\sigma}_h^k(s_h^k,a_h^k)- \big(\widehat{r}_h^k(s_h^k,a_h^k) \big)^2\right)} 
	+ \frac{1088}{9}SAH^2(\log_2K)\log \frac{1}{\delta'}.\label{eq:local31}
 \end{align}
%
To bound the right-hand side of \eqref{eq:local31}, 
we resort to the following lemma. 
%
 \begin{lemma}\label{lemma:ks2}
With probability at least $1-4SAHK\delta'$, one has
%
\begin{align}
\sum_{k,h}\left(\widehat{\sigma}_h^k(s_h^k,a_h^k)- (\widehat{r}_h^k(s_h^k,a_h^k))^2\right) \leq 6K\mathrm{var}_2 + 242H^2(\log_2K)\log \frac{1}{\delta'}.
\end{align}
%
 \end{lemma}
 %
\begin{proof}
Recall that in Lemma~\ref{lemma:bdrv}, we have shown that with probability at least $1-4SAHK\delta'$,
%
\begin{align}
    \sum_{k=1}^K\sum_{h=1}^H\left(\widehat{\sigma}_h^k(s_h^k,a_h^k)- \big(\widehat{r}_h^k(s_h^k,a_h^k) \big)^2\right)\leq 3\sum_{k=1}^K  \widetilde{V}_1^k(s_1^k)+2SAH^3(\log_2K)\log \frac{1}{\delta'}.
\end{align}
%
We then complete the proof by observing that
%
\begin{align}
	\widetilde{V}_1^k(s_1^k) &\leq  \widetilde{V}_1^k(s_1^k)+\mathbb{E}_{\pi^k}\left[\sum_{h=1}^H \mathbb{V}\big(P_{s_h,a_h,h},V_{h+1}^{\pi^k} \big) \,\Big|\, s_1=s_1^k\right] \notag\\
	&= \mathsf{Var}_{\pi^k}\left[\sum_{h=1}^H r_h(s_h,a_h) \,\Big|\, s_1=s_1^k\right]\leq \mathrm{var}_2.
\end{align}
\end{proof}
 %
Combining Lemma~\ref{lemma:ks2} with \eqref{eq:local31} gives: with probability at least $1-4SAHK\delta'$,
%
\begin{align}
	T_2 &\leq \frac{460}{9}\sqrt{2SAH T_5 (\log_2K)\log \frac{1}{\delta'} } +12\sqrt{SAH(\log_2K)\log \frac{1}{\delta'}}\sqrt{2K\mathrm{var}_2}  \notag\\
	&\qquad\qquad + 157SAH^2(\log_2K)\log \frac{1}{\delta'}.\label{eq:xbt2}
\end{align}




\subsubsection{Bounding $T_4$}

Recall that $T_4 = \widecheck{T}_1 + \widecheck{T}_2$, 
where 
%
\begin{align*}
	\widecheck{T}_1 &= \sum_{k=1}^K\sum_{h=1}^H \left(\widehat{r}_h^k(s_h^k,a_h^k) -r_h(s_h^k,a_h^k) \right),\\
	\widecheck{T}_2 &= \sum_{k=1}^K\left(\sum_{h=1}^H r_h(s_h^k,a_h^k) -V_{1}^{\pi^k}(s_1^k) \right).
\end{align*}
%
Repeating similar arguments employed in the proof of Lemma~\ref{lemma:bdrv} and using \eqref{eq:wc3}, we see that with probability exceeding $1-6SAHK\delta'$, 
%
\begin{align}
	|\widecheck{T}_1| & \leq \sqrt{4SAH (\log_2 K)\log \frac{1}{\delta'}}\cdot \sqrt{\sum_{k=1}^K\sum_{h=1}^H v_h(s_h^k,a_h^k)} + 2SAH^2(\log_2K)\log \frac{1}{\delta'} \nonumber
\\ & \leq \sqrt{8SAHK\mathrm{var}_2(\log_2K)\log \frac{1}{\delta'}}+ 20SAH^2(\log_2K)\log \frac{1}{\delta'}.\nonumber
\end{align}
%
In addition, from Lemma~\ref{lemma:self-norm} and the definition of $\mathrm{var}_2$, we see that 
%
\begin{align}
|\widecheck{T}_2|\leq 2\sqrt{2K\mathrm{var}_2 \log \frac{1}{\delta'}}+3H\log \frac{1}{\delta'} 
\end{align}
%
with probability at least $1-2SAHK\delta'$. 
% 
Therefore, with probability at least $1-8SAHK\delta'$, it holds that
%
\begin{align}
T_4 \leq 4\sqrt{2SAHK\mathrm{var}_2(\log_2K)\log \frac{1}{\delta'}}+ 23SAH^2(\log_2K)\log \frac{1}{\delta'}.\label{eq:xbt4}
\end{align}





\subsubsection{Bounding $T_5$ and $T_6$}




Recall that Lemma~\ref{lemma:empv} asserts that with probability exceeding $1-2\delta'$, 
$$
	T_5\leq 5T_6+8BSAH^3.
$$
%
Hence, it suffices to bound $T_6$.


From the elementary inequality $\mathsf{Var}(X+Y)\leq 2\mathsf{Var}(X)+ 2\mathsf{Var}(Y)$, we obtain
%
\begin{align}
	T_6 &=\sum_{k,h} \mathbb{V}\big(P_{s_h^k,a_h^k,h},V_{h+1}^k\big)   \leq  2\sum_{k,h} \mathbb{V}\big(P_{s_h^k,a_h^k,h},V_{h+1}^{\pi^k} \big) 
	+ 2\sum_{k,h}\mathbb{V}\big(P_{s_h^k,a_h^k,h},V_{h+1}^k -V_{h+1}^{\pi^k}\big )\nonumber
\\ 
	& \leq 3   K\mathrm{var}_2 +\sum_{k=1}^K \left( \sum_{h=1}^H \mathbb{V}\big(P_{s_h^k,a_h^k,h},V_{h+1}^{\pi^k} \big) - 3\mathrm{var}_2 \right) + 2\sum_{k,h}\mathbb{V}\big(P_{s_h^k,a_h^k,h},V_{h+1}^k -V_{h+1}^{\pi^k} \big).\label{eq:var01}
\end{align}
%
To bound the right-hand side of \eqref{eq:var01}, we resort to the following two lemmas.
%
\begin{lemma}\label{lemma:kx1}
With probability  at least $1-4SAHK\delta'$, it holds that
\begin{align}
\sum_{k=1}^K \left( \sum_{h=1}^H \mathbb{V}(P_{s_h^k,a_h^k,h},V_{h+1}^{\pi^k} ) -2\mathrm{var}_2 \right)  \leq 80H^2\log \frac{1}{\delta'}.
\end{align}
\end{lemma}
%
\begin{lemma}\label{lemma:bdv11}
With probability exceeding $1-4SAKH\delta'$, it holds that
%
\begin{align}
 & \sum_{k,h}\mathbb{V}\big(P_{s_h^k,a_h^k,h}, V_{h+1}^k -V_{h+1}^{\pi^k}\big)   \leq 4\sqrt{BH^2\sum_{k,h}\mathbb{V}\big(P_{s_h^k,a_h^k,h},V_{h+1}^k\big)}+ 4H\sum_{k,h}b_h^k(s_h^k,a_h^k)+ 3BSAH^3.\nonumber
\end{align}
%
\end{lemma}
%
With Lemma~\ref{lemma:kx1} and Lemma~\ref{lemma:bdv11} in place, we can demonstrate that with probability at least $1-6SAHK\delta'$,
%
\begin{align}
T_6&
	%=\sum_{k,h}\mathbb{V}\big(P_{s_h^k,a_h^k,h},V_{h+1}^k\big)\nonumber
%\\ & 
	\leq  2\sum_{k,h}\mathbb{V}\big(P_{s_h^k,a_h^k,h},V_{h+1}^{\pi^k}\big)+2\sum_{k,h}\mathbb{V}\big(P_{s_h^k,a_h^k,h},V_{h+1}^k - V_{h+1}^{\pi^k}\big)\nonumber
\\ & \leq 4K\mathrm{var}_2 +  8\sqrt{BSAH^3 T_6}+8HT_2 + 7BSAH^3,\nonumber
%\\ & \leq 8K\mathrm{var}_2 + 16HT_2 + 78BSAH^3.\label{eq:xbt6}
\end{align}
%
\begin{align}
\Longrightarrow \qquad 
	T_6 \leq 8K\mathrm{var}_2 + 16HT_2 + 78BSAH^3.\label{eq:xbt6}
\end{align}
%
Taking this result together with  Lemma~\ref{lemma:empv} gives, with probability exceeding $1-8SAHK\delta'$, 
%
\begin{align}
T_5 =\sum_{k,h}\mathbb{V}\big(\widehat{P}_{s_h^k,a_h^k,h},V_{h+1}^k\big)\leq 40K\mathrm{var}_2 + 80HT_2+398BSAH^3.\label{eq:xbt5}
\end{align}
%
To finish establishing the above bounds on $T_5$ and $T_6$, it suffices to prove Lemma~\ref{lemma:kx1} and Lemma~\ref{lemma:bdv11}, 
which we accomplish in the sequel. 


%\yxc{???} We can then derive
%%
%\begin{align}
%\sum_{k,h}\mathbb{V}(P_{s_h^k,a_h^k,h}, V_{h+1}^k) &  \leq 2 \sum_{k,h}\mathbb{V}(P_{s_h^k,a_h^k,h},V_{h+1}^{\pi^k}) + 2\sum_{k,h}\mathbb{V}(P_{s_h^k,a_h^k,h}, V_{h+1}^k -V_{h+1}^{\pi^k} )\nonumber
%\\ & \leq 6\sum_{k=1}^K\widecheck{\mathrm{var}}^k + \sum_{k=1}^K\left( \sum_{h=1}^H \mathbb{V}(P_{s_h^k,a_h^k,h},V_{h+1}^{\pi^k}) - 3\widecheck{\mathrm{var}}^k \right) + 2\sum_{k,h}\mathbb{V}(P_{s_h^k,a_h^k,h}, V_{h+1}^k -V_{h+1}^{\pi^k} )\nonumber
%\\ & \leq 6K\mathrm{var}_2 + \sum_{k=1}^K\left( \sum_{h=1}^H \mathbb{V}(P_{s_h^k,a_h^k,h},V_{h+1}^{\pi^k}) - 3\widecheck{\mathrm{var}}^k \right) + 2\sum_{k,h}\mathbb{V}(P_{s_h^k,a_h^k,h}, V_{h+1}^k -V_{h+1}^{\pi^k} ).\label{eq:var6}
%\end{align}
%
%
%
%By Lemma~\ref{lemma:kx1}, Lemma~\ref{lemma:empv} and Lemma~\ref{lemma:bdv11}, we know that with probability at least $1-18SAHK\delta'$, 
%\begin{align}
% & T_5 \leq O\left( K\mathrm{var}_2 + H\sqrt{T_6\left(SAH+\log\frac{1}{\delta'}\right)} +T_2+H^2\left(SAH+\log \frac{1}{\delta'}\right)    \right) ;\label{eq:nbbt5}
%\\ & T_6 \leq O\left( K\mathrm{var}_2 + H\sqrt{T_6\left(SAH+\log\frac{1}{\delta'}\right)} +T_2+H^2\left(SAH+\log \frac{1}{\delta'}\right)    \right) .\label{eq:nbbt6}
%\end{align}



\begin{proof}[Proof of Lemma~\ref{lemma:kx1}]
%
For notational convenience, define 
	%$\widecheck{R}^k_{h}(s,a) = \mathbb{V}(P_{s,a,h},V_{h+1}^{\pi^k})$. Define
\begin{align}
	\widecheck{R}^k_{h}(s,a) = \mathbb{V}(P_{s,a,h},V_{h+1}^{\pi^k}) \qquad \text{and} \qquad
\widecheck{V}^k_h (s) = \mathbb{E}\left[ \sum_{h'=h}^H \widecheck{R}^k_{h'}(s_{h'},a_{h'}) \,\Big|\, s_h = s\right].\nonumber
\end{align}
%
It is easily seen that $\widecheck{V}_h^k(s)\leq\mathrm{var}_2 \leq  H^2$. 


We also make the observation that 
%
\begin{align}
  \sum_{h=1}^H  \mathbb{V}(P_{s_h^k,a_h^k,h},V_{h+1}^{\pi^k})-\mathrm{var}_2 \nonumber & =\sum_{h=1}^H  \widecheck{R}_h^k(s_h^k,a_h^k)-\mathrm{var}_2  \nonumber
 \\ & \leq \sum_{h=1}^H \widecheck{R}_h^k(s_h^k,a_h^k) - \widecheck{V}^k_1(s_1^k)\nonumber
 \\ & = \sum_{h=1}^H \big\langle e_{s_{h+1}^k} - P_{s_{h}^k,a_h^k,h},\,\widecheck{V}^k_{h+1}   \big\rangle.
\end{align}
%
Note that $\widecheck{V}^k$ only depends on $\pi^k$, which is determined before the $k$-th episode starts.  
 Lemma~\ref{lemma:self-norm} then tells us that, with probability at least $1-2SAHK\delta'$,
%
\begin{align}
 & \sum_{k=1}^K \left( \sum_{h=1}^H \mathbb{V}\big(P_{s_h^k,a_h^k,h},V_{h+1}^{\pi^k}\big) - \widecheck{V}_1^k(s_1^k)  \right) \nonumber
 \\ & \leq 2\sqrt{2\sum_{k=1}^K \sum_{h=1}^H \mathbb{V}\big(P_{s_h^k,a_h^k,h},\widecheck{V}_{h+1}^k\big)\log \frac{1}{\delta'} } + 3H^2\log \frac{1}{\delta'}.\label{eq:xlll11}
\end{align}
%
Further, it is observed that with probability at least $1-2SAHK\delta'$, 
%
\begin{align}
 & \sum_{k=1}^{K}\sum_{h=1}^{H}\mathbb{V}\big(P_{s_{h}^{k},a_{h}^{k},h},\widecheck{V}_{h+1}^{k}\big)\nonumber\\
 & =\sum_{k=1}^{K}\sum_{h=1}^{H}\left(\big\langle P_{s_{h}^{k},a_{h}^{k},h},(\widecheck{V}_{h+1}^{k})^{2}\big\rangle-\big(\big\langle P_{s_{h}^{k},a_{h}^{k},h},\widecheck{V}_{h+1}^{k}\big\rangle\big)^{2}\right)\nonumber\\
 & =\sum_{k=1}^{K}\sum_{h=1}^{H}\big\langle P_{s_{h}^{k},a_{h}^{k},h}-e_{s_{h+1}^{k}},\,(\widecheck{V}_{h+1}^{k})^{2}\big\rangle\nonumber\\
 & \qquad+\sum_{k=1}^{H}\sum_{h=1}^{H}\left( \big(\widecheck{V}_{h+1}^{k}(s_{h+1}^{k}) \big)^{2}- \big(\widecheck{V}_{h}^{k}(s_{h}^{k})\big)^{2}\right)
	+\sum_{k=1}^{K}\sum_{h=1}^{H}\left(\big(\widecheck{V}_{h}^{k}(s_{h}^{k})\big)^{2}-\left(\big\langle P_{s_{h}^{k},a_{h}^{k},h},\widecheck{V}_{h+1}^{k}\big\rangle\right)^{2}\right)\nonumber\\
 & \leq2\sqrt{8H^{4}\sum_{k=1}^{K}\sum_{h=1}^{H}\mathbb{V}\big(P_{s_{h}^{k},a_{h}^{k},h},\widecheck{V}_{h+1}^{k}\big)\log\frac{1}{\delta'}}+2H^{2}\sum_{k=1}^{K}\sum_{h=1}^{H}\widecheck{R}_{h}(s_{h}^{k},a_{h}^{k})+3H^{4}\log\frac{1}{\delta'}.
	\label{eq:xllll21}
\end{align}
%
Here, the last inequality results from Lemma~\ref{lemma:self-norm}, Lemma~\ref{lemma:sqv}  and the fact that 
	$\widecheck{V}_h^k(s_h^k) = \widecheck{R}_h(s_h^k,a_h^k)+ \langle P_{s_h^k,a_h^k,h}, \widecheck{V}_{h+1}^k \rangle$. 
% 
It then follows that
\begin{align}
\sum_{k=1}^K \sum_{h=1}^H \mathbb{V}\big(P_{s_h^k,a_h^k,h},\widecheck{V}_{h+1}^k \big)  \leq 4H^2\sum_{k=1}^K \sum_{h=1}^H \widecheck{R}_h(s_h^k,a_h^k) + 42 H^4\log \frac{1}{\delta'}.\label{eq:xlll31}
\end{align}
%
Taking \eqref{eq:xlll11} and \eqref{eq:xlll31} together leads to
%
\begin{align}
\sum_{k=1}^K \sum_{h=1}^H \mathbb{V}\big(P_{s_h^k,a_h^k,h},V_{h+1}^{\pi^k}\big) 
	\leq \sum_{k=1}^H \widecheck{V}_1^k(s_1^k) +2\sqrt{8 H^2 \sum_{k=1}^K \sum_{h=1}^H \mathbb{V}\big(P_{s_h^k,a_h^k,h},V_{h+1}^{\pi^k}\big) \log \frac{1}{\delta'}    } + 21H^2\log \frac{1}{\delta'},\nonumber
\end{align}
%
which further implies that
%
\begin{align}
	\sum_{k=1}^K \sum_{h=1}^H \mathbb{V}\big(P_{s_h^k,a_h^k,h},V_{h+1}^{\pi^k}\big) 
	\leq 2\sum_{k=1}^K \widecheck{V}_1^k(s_1^k)+ 84H^2\log \frac{1}{\delta'} 
	\leq 2K\mathrm{var}_2 + 84H^2\log \frac{1}{\delta'}.\nonumber
\end{align}
%
This concludes the proof. 
\end{proof}


%For the left term $\sum_{k,h}\mathbb{V}(P_{s_h^k,a_h^k,h},V_{h+1}^k -V_{h+1}^{\pi^k})$,   we have the lemma below.
%
%
\begin{proof}[Proof of Lemma~\ref{lemma:bdv1}]
A little algebra gives
%
\begin{align}
 & \sum_{k,h}\mathbb{V}\big(P_{s_{h}^{k},a_{h}^{k},h},V_{h+1}^{k}-V_{h+1}^{\pi^{k}}\big)\nonumber\\
 & =\sum_{k,h}\left(\big\langle P_{s_{h}^{k},a_{h}^{k},h},\,(V_{h+1}^{k}-V_{h+1}^{\pi^{k}})^{2}\big\rangle-\big(\big\langle P_{s_{h}^{k},a_{h}^{k},h},V_{h+1}^{k}-V_{h+1}^{\pi^{k}}\big\rangle\big)^{2}\right)\notag\\
 & =\sum_{k,h}\left(\big\langle P_{s_{h}^{k},a_{h}^{k},h}-e_{s_{h+1}^{k}},(V_{h+1}^{k}-V_{h+1}^{\pi^{k}})^{2}\big\rangle\right)\nonumber\\
 & \qquad\qquad+\sum_{k,h}\left(\big(V_{h+1}^{k}(s_{h+1}^{k})-V_{h+1}^{\pi^{k}}(s_{h+1}^{k})\big)^{2}-\big(\big\langle P_{s_{h}^{k},a_{h}^{k},h},V_{h+1}^{k}-V_{h+1}^{\pi^{k}}\big\rangle\big)^{2}\right)\nonumber\\
 & =\sum_{k,h}\left(\big\langle P_{s_{h}^{k},a_{h}^{k},h}-e_{s_{h+1}^{k}},(V_{h+1}^{k}-V_{h+1}^{\pi^{k}})^{2}\big\rangle\right)+\sum_{k,h}\left(\big(V_{h}^{k}(s_{h}^{k})-V_{h}^{\pi^{k}}(s_{h}^{k})\big)^{2}-\big(\big\langle P_{s_{h}^{k},a_{h}^{k},h},V_{h+1}^{k}-V_{h+1}^{\pi^{k}}\big\rangle\big)^{2}\right).
	\label{eq:var11}
 \end{align}
%
From Lemma~\ref{lemma:self-norm} and Lemma~\ref{lemma:sqv}, we can show that with probability $1-2SAKH\delta'$, 
%
\begin{align}
 & \sum_{k,h}\big\langle P_{s_{h}^{k},a_{h}^{k},h}-e_{s_{h+1}^{k}},(V_{h+1}^{k}-V_{h+1}^{\pi^{k}})^{2}\big\rangle\\
 & \qquad\leq2\sqrt{2}\sqrt{4H^{2}\sum_{k,h}\mathbb{V}\big(P_{s_{h}^{k},a_{h}^{k},h},V_{h+1}^{k}-V_{h+1}^{\pi^{k}}\big)\log\frac{1}{\delta'}}+3H^{2}\log\frac{1}{\delta'}. 
	\label{eq:var31}
\end{align}
%
Additionally, with probability at least $1-2SAKH\delta'$, 
%
 \begin{align}
  & \sum_{k,h}\left\{ \big(V_{h}^{k}(s_{h}^{k})-V_{h}^{\pi^{k}}(s_{h}^{k})\big)^{2}-\big(\big\langle P_{s_{h}^{k},a_{h}^{k},h},V_{h+1}^{k}-V_{h+1}^{\pi^{k}}\big\rangle)^{2}\right\} \nonumber\\
 & \leq2H\sum_{k,h}\max\Big\{ V_{h}^{k}(s_{h}^{k})-\big\langle P_{s_{h}^{k},a_{h}^{k},h},V_{h+1}^{k}\big\rangle-\big(V_{h}^{\pi^{k}}(s_{h}^{k})-\big\langle P_{h}^{k},V_{h+1}^{\pi^{k}}\big\rangle\big),0\Big\}\nonumber\\
 & =2H\sum_{k,h}\max\Big\{ V_{h}^{k}(s_{h}^{k})-\big\langle P_{s_{h}^{k},a_{h}^{k},h},V_{h+1}^{k}\big\rangle-r_{h}(s_{h}^{k},a_{h}^{k}),0\Big\}\nonumber\\
 & \leq2H\sum_{k,h}\max\Big\{\big\langle\widehat{P}_{s_{h}^{k},a_{h}^{k},h}-P_{s_{h}^{k},a_{h}^{k},h},V_{h+1}^{k}\big\rangle,0\Big\}+2H\sum_{k,h}b_{h}^{k}(s_{h}^{k},a_{h}^{k})\nonumber\\
 & \leq2\sqrt{BSAH^{3}\sum_{k,h}\mathbb{V}\big(P_{s_{h}^{k},a_{h}^{k},h},V_{h+1}^{k}\big)}+2H\sum_{k,h}b_{h}^{k}(s_{h}^{k},a_{h}^{k})+BSAH^{3}.
	 \label{eq:var41}
 \end{align}
%
It then follows that
%
\begin{align}
 & \sum_{k,h}\mathbb{V}(P_{s_h^k,a_h^k,h}, V_{h+1}^k -V_{h+1}^{\pi^k})  \leq 4\sqrt{BSAH^3\sum_{k,h}\mathbb{V}(P_{s_h^k,a_h^k,h},V_{h+1}^k)}+ 4H\sum_{k,h}b_h^k(s_h^k,a_h^k)+ 3BSAH^3
\end{align}
%
with probability at least $1-4SAKH\delta'$. 
The proof is thus complete.
%
\end{proof}



\iffalse
\begin{lemma}\label{lemma:l2}
With probability at least $1-4SAKH\delta$,
\begin{align}
\sum_{k,h} \mathbb{V}(P_{s_h^k,a_h^k,h}, V_{h+1}^k-V_{h+1}^{\pi^k}) \leq O\left(   H\sqrt{\sum_{k,h}\mathbb{V}(P_{s_h^k,a_h^k,h},V_{h+1}^k)\left(SAH+\log(\frac{1}{\delta'}\right))} +\sum_{k,h}b_h^k(s_h^k,a_h^k)+ H^2\left(SAH+\log \frac{1}{\delta'} \right)\right).
\end{align}
\end{lemma}
%
\begin{proof}[Proof of Lemma~\ref{lemma:l2}] With Lemma~\ref{lemma:self-norm}, with probability exceeding $1-2SAHK\delta$
\begin{align}
 & \sum_{k,h}\mathbb{V}(P_{s_h^k,a_h^k,h}, V^k_{h+1}- V_{h+1}^{\pi^k}) \nonumber
\\ &= \sum_{k,h} \left(   P_{s_h^k,a_h^k,h} (V_{h+1}^k - V_{h+1}^{\pi^k})^2 - (P_{s_h^k,a_h^k,h} (V_{h+1}^k-V_{h+1}^{\pi^k}))^2      \right)\nonumber
\\ &  = \sum_{k,h}\left( (P_{s_h^k,a_h^k,h}-e_{s_{h+1}^k})(V_{h+1}^k -V_{h+1}^{\pi^k})^2  \right) +\sum_{k,h} \left(  (V_{h+1}^k(s_{h+1}^k) - V_{h+1}^{\pi^k}(s_{h+1}^k))^2  - (P_{s_h^k,a_h^k,h}(V_{h+1}^k -V_{h+1}^{\pi^k}))^2  \right)\nonumber
\\ & \leq  \sum_{k,h}\left( (P_{s_h^k,a_h^k,h}-e_{s_{h+1}^k})(V_{h+1}^k -V_{h+1}^{\pi^k})^2  \right) +\sum_{k,h} \left(  (V_{h}^k(s_{h}^k) - V_{h}^{\pi^k}(s_{h}^k))^2  - (P_{s_h^k,a_h^k,h}(V_{h+1}^k -V_{h+1}^{\pi^k}))^2  \right)\nonumber
\\ & \leq O\left( H\sqrt{\sum_{k,h}\mathbb{V}(P_{s_h^k,a_h^k,h},V_{h+1}^k -V_{h+1}^{\pi^k})  \log(\frac{1}{\delta'})}  +H^2\log(\frac{1}{\delta'}) \right) \nonumber
\\ & \quad \quad \qquad \qquad +2H\sum_{k,h} \max\{     (V_{h}^k(s_h^k)-P_{s_h^k,a_h^k,h} V_{h+1}^k) - (V_{h}^{\pi^k}(s_h^k)-P_{s_h^k,a_h^k,h}V_{h+1}^{\pi^k})   ,0\}.\label{eq:v81}
\end{align}

By the definition of $V_{h}^k$ and $V_{h}^{\pi^k}$, we have
\begin{align}
& V_h^k(s_h^k)\leq r_h(s_h^k,a_h^k) + \widehat{P}^k_{s_h^k,a_h^k,h} + b_h^k(s_h^k,a_h^k)\nonumber
\\ & V_{h}^{\pi^k}(s_h^k) = r_h(s_h^k,a_h^k) +P_{s_h^k,a_h^k,h}V_{h+1}^{\pi^k}.\nonumber
\end{align}
It then holds that
\begin{align}
&\max\{     (V_{h}^k(s_h^k)-P_{s_h^k,a_h^k,h} V_{h+1}^k) - (V_{h}^{\pi^k}(s_h^k)-P_{s_h^k,a_h^k,h}V_{h+1}^{\pi^k})   ,0\} \nonumber
\\ & \leq \max\{(\widehat{P}_{s_h^k,a_h^k,h}-P_{s_h^k,a_h^k,h})V_{h+1}^{k},0 \} +\sum_{k,h}b_h^k(s_h^k,a_h^k).\label{eq:v7}
\end{align}

By Lemma~\ref{lemma:key3}, we further have that, with probability $1-2SAKH\delta$,
\begin{align}
\sum_{k,h}\max\{(\widehat{P}_{s_h^k,a_h^k,h}-P_{s_h^k,a_h^k,h})V_{h+1}^{k},0 \} \leq O\left( \sqrt{\sum_{k,h}\mathbb{V}(P_{s_h^k,a_h^k,h}, V_{h+1}^{k})\left(SAH+\log(\frac{1}{\delta'}) \right) }+H\left(SAH+\log(\frac{1}{\delta'})\right)    \right).\label{eq:v8}
\end{align}

By \eqref{eq:v81}, \eqref{eq:v7} and \eqref{eq:v8}, we obtain 
\begin{align}
&\sum_{k,h}\mathbb{V}(P_{s_h^k,a_h^k,h},V_{h+1}^k - V_{h+1}^{\pi^k})\nonumber
\\ & \leq O\left(H \sqrt{\sum_{k,h}\mathbb{V}(P_{s_h^k,a_h^k,h},V_{h+1}^k - V_{h+1}^{\pi^k})\log(\frac{1}{\delta'})}+ H\sqrt{\sum_{k,h}\mathbb{V}(P_{s_h^k,a_h^k,h},V_{h+1}^k)\left(SAH+\log(\frac{1}{\delta'}\right)} \right)\nonumber
\\ & \qquad \qquad \qquad \qquad \qquad \qquad \qquad \qquad \qquad \qquad\qquad \qquad+ O\left( \sum_{k,h}b_h^k(s_h^k,a_h^k)+ H^2\left(SAH+\log(\frac{1}{\delta'})\right) \right)\nonumber
\end{align}
holds with probability at least $1-4SAKH\delta$, which further implies that
\begin{align}
\sum_{k,h}\mathbb{V}(P_{s_h^k,a_h^k,h},V_{h+1}^k - V_{h+1}^{\pi^k}) \leq O\left(   H\sqrt{\sum_{k,h}\mathbb{V}(P_{s_h^k,a_h^k,h},V_{h+1}^k)\left(SAH+\log(\frac{1}{\delta'}\right))} +\sum_{k,h}b_h^k(s_h^k,a_h^k)+ H^2\left(SAH+\log(\frac{1}{\delta'})\right)\right) .
\end{align}
This completes the proof.
\end{proof}

\fi 









\subsubsection{Putting all pieces together}

Recall that $B =4000 (\log_2 K)^3 \log(3SA)\log \frac{1}{\delta'}$. 
The last step is to rewrite the inequalities $\eqref{eq:obt1}-\eqref{eq:obt8}$ as follows with \eqref{eq:obt2}, \eqref{eq:obt4}, \eqref{eq:obt5} and \eqref{eq:obt6}  replaced by  \eqref{eq:xbt2},\eqref{eq:xbt4}
\eqref{eq:xbt5} and \eqref{eq:xbt6} respectively:
%
\begin{align}
& T_1 \leq \sqrt{128BSAHT_6}+24BSAH^2,\nonumber
\\ & T_7 \leq H\sqrt{512BSAHT_6}+24BSAH^3,\nonumber
\\ & T_9 \leq \sqrt{128BSAHT_6}+24BSAH^2,\nonumber
\\ & T_2\leq 100 \sqrt{BSAHT_5}+140BSAH^2,\nonumber
 \\ & T_3 \leq \sqrt{8BT_6}+3H\log \frac{1}{\delta'} ,\nonumber
 \\ & T_4 \leq \sqrt{BSAHK\mathrm{var}_2}+BSAH^2;\nonumber
 \\ & T_5 \leq 40K\mathrm{var}_2 + 80HT_2+398BSAH^3  ,\nonumber
 \\ &  T_6 \leq  8K\mathrm{var}_2 + 16HT_2 + 78BSAH^3 ,\nonumber
 \\ & T_8 \leq \sqrt{32BH^2T_6 } + 3BH^2 ,\nonumber
\end{align}
%
which are valid with probability at least $1-200SAH^2K^2\delta'$. 
Solving the inequalities listed above, we can readily conclude that
%
\begin{align}
\mathsf{Regret}(K)= T_1+T_2+T_3+T_4 \leq O\left( \sqrt{BSAHK\mathrm{var_2} }+ BSAH^2 \right).\label{eq:rbvar2}
\end{align}
%
This finishes the proof by recalling that $\delta' = \frac{\delta}{200SAH^2K^2}$.
