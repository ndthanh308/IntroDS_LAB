
Recall that 
%
\begin{equation}
	B=4000 (\log_2 K)^3 \log(3SAH)\log\frac{1}{\delta'} 
	\qquad \text{with }\delta' = \frac{\delta}{200SAH^2K^2}. 
	\label{eq:defn-B-first}
\end{equation}
%
Consider first the scenario where $K\leq \frac{BSAH^2}{v^{\star}}$: 
the regret bound can be upper bounded by 
%
\begin{align}
\mathbb{E}\big[\mathsf{Regret}(K)\big] & =\mathbb{E}\left[\sum_{k=1}^{K}\Big(V_{1}^{\star}(s_{1}^{k})-V_{1}^{\pi^{k}}(s_{1}^{k})\Big)\right]\leq\mathbb{E}\left[\sum_{k=1}^{K}V_{1}^{\star}(s_{1}^{k})\right]=K\mathbb{E}_{s_{1}\sim\mu}\big[V_{1}^{\star}(s_{1})\big]\notag\\
 & =Kv^{\star}=\min\Big\{\sqrt{BSAH^{2}Kv^{\star}}
	%+BSAH^{2}
	,Kv^{\star}\Big\}.
	\label{eq:E-regret-UB-easy-first}
\end{align}
%
As a result, 
the remainder of the proof is dedicated to the the case with
%
\begin{equation}
	K\geq \frac{BSAH^2}{v^{\star}} .
	\label{eq:K-focus-first}
\end{equation}
%
%throughout the rest of this section. 



%\begin{align}
%	\mathsf{Regret}(K) &= \sum_{k=1}^K \Big( V_1^{\star}(s_1^k) -  V_1^{\pi^k} (s_1^k) \Big) 
%	\leq \sum_{k=1}^K  V_1^{\star}(s_1^k) 
%	\leq 3K \mathbb{E}_{s_1\sim \mu}\big[V_1^{\star}(s_1)\big]  + H\log \frac{1}{\delta'} \notag\\ 
%	&= 3Kv^{\star} + H\log \frac{1}{\delta'} 
%	\leq  3\min\Big\{\sqrt{BSAH^{2}Kv^{\star}}+BSAH^2,Kv^{\star}\Big\}
%\end{align}
%%
%with probability at least $1-\delta'$, 
%where we have invoked Lemma~\ref{lemma:con}. \yxc{check}





To begin with, recall that the proof of Theorem~\ref{thm1} in Section~\ref{app:thmmain} consists of bounding the quantities $T_1,\ldots,T_9$ (see \eqref{eq:decomposition}, \eqref{eq:defn-T56-proof} and \eqref{eq:defn-T789-proof}) and recall that $\delta' = \frac{\delta}{200SAH^2K^2}$. 
%
%defined as follows:  
%%
%\begin{align}
%& T_1= \sum_{k=1}^K \sum_{h=1}^H \left( \widehat{P}^k_{s_h^k,a_h^k,h}-P_{s_h^k,a_h^k,h} \right)V_{h+1}^k ;\nonumber
%\\ &  T_2 = \sum_{k=1}^K \sum_{h=1}^H b_h^k(s_h^k,a_h^k);\nonumber
%\\ & T_3 =\sum_{k=1}^K \sum_{h=1}^H (P_{s_h^k,a_h^k,h}-\textbf{1}_{s_{h+1}^k})V_{h+1}^k;\nonumber
%\\ & T_4 = \sum_{k=1}^K \left( \sum_{h=1}^H \widehat{r}_h^k(s_h^k,a_h^k)-V_{1}^{\pi^k}(s_1^k)\right);\nonumber
%\\ & T_5  = \sum_{k=1}^K \sum_{h=1}^H \mathbb{V}(\widehat{P}^k_{s_h^k,a_h^k,h},V_{h+1}^k);\nonumber
%\\ & T_6 = \sum_{k=1}^K \sum_{h=1}^H \mathbb{V}(P_{s_h^k,a_h^k,h},V_{h+1}^k);\nonumber
%\\ & T_7  = \sum_{k=1}^K \sum_{h=1}^H \left( \widehat{P}^k_{s_h^k,a_h^k,h} - P_{s_h^k,a_h^k,h} \right) (V_{h+1}^k)^2;\nonumber
%\\ & T_8 =\sum_{k=1}^K \sum_{h=1}^H (P_{s_h^k,a_h^k,h}-\textbf{1}_{s_{h+1}^k})(V_{h+1}^k)^2;\nonumber
%\\ & T_9  = \sum_{k=1}^K\sum_{h=1}^H \max\left\{ (\widehat{P}^k_{s_h^k,a_h^k,h}-P_{s_h^k,a_h^k,h})V_{h+1}^k, 0  \right\}.\nonumber
%\end{align}
%%
%where we recall that $\delta' = \frac{\delta}{200SAH^2K^2}$.  
%
In order to establish Theorem~\ref{thm:first}, we need to develop tighter bounds on some of these quantities (i.e., $T_2$, $T_4$, $T_5$ and $T_6$)  to reflect their dependency on $v^{\star}$ (cf.~\eqref{eq:defn-vstar-formal}). 

%continue by proving tighter bounds for some of these terms with respect to $v^{\star}$.


%Also recall \eqref{eq:obt2}-\eqref{eq:obt9}. We will provide refined analysis for the bound of $T_2$, $T_4$, $T_5$ and $T_6$, and leave other bounds invariant.


\paragraph{Bounding $T_2$.}
%
Recall that we have shown in \eqref{eq:boundt2o} that
%
\begin{align}
 & T_{2}\leq\frac{460}{9}\sqrt{2SAH(\log_{2}K)\Big(\log\frac{1}{\delta'}\Big)T_{5}}\nonumber\\
 & \qquad+4\sqrt{SAH^{2}(\log_{2}K)\log\frac{1}{\delta'}}\sqrt{\sum_{k,h}\widehat{r}_{h}^{k}(s_{h}^{k},a_{h}^{k})}+\frac{1088}{9}SAH^{2}(\log_{2}K)\log\frac{1}{\delta'}.
\nonumber
\end{align}
%
In view of the definition of $T_4$ (cf.~\eqref{eq:decomposition}) as well as the fact that $\sum_{k=1}^K V_1^{\star}(s_1^k)\leq 3Kv^{\star} + H\log \frac{1}{\delta'}$ holds with probability at least $1-\delta'$ (see Lemma~\ref{lemma:con}),
we arrive at
%
\begin{align}
\sum_{k,h}\widehat{r}_{h}^{k}(s_{h}^{k},a_{h}^{k})\leq T_{4}+\sum_{k}V_{1}^{\pi_{k}}(s_{1}^{k})\leq T_{4}+\sum_{k}V_1^{\star}(s_{1}^{k})\leq T_{4}+3Kv^{\star}+H\log\frac{1}{\delta'},
	\label{eq:sum-rhat-vstar}
\end{align}
%
which in turn gives
%
 \begin{align}
T_{2} & \leq\frac{460}{9}\sqrt{2SAH(\log_{2}K)\Big(\log\frac{1}{\delta'}\Big)T_{5}}\nonumber\\
 & \qquad\qquad+4\sqrt{SAH^{2}(\log_{2}K)\log\frac{1}{\delta'}}\sqrt{T_{4}+3Kv^{\star}}+130SAH^{2}(\log_{2}K)\log\frac{1}{\delta'}.
\label{eq:nbft2}
 \end{align}


\paragraph{Bounding $T_4$.}
%
When it comes to the quantity $T_4$ (cf.~\eqref{eq:decomposition}), we make the observation that 
%
\begin{align}
 T_4 
	%& = \sum_{k=1}^K \left( \sum_{h=1}^H \widehat{r}_h^k(s_h^k,a_h^k)-V_{1}^{\pi^k}(s_1^k)\right) \nonumber
 %\\
	& 
	=\underset{\eqqcolon\,\widecheck{T}_{1}}{\underbrace{\sum_{k=1}^{K}\left(\sum_{h=1}^{H}\widehat{r}_{h}^{k}(s_{h}^{k},a_{h}^{k})-r_{h}(s_{h}^{k},a_{h}^{k})\right)}}+\underset{\eqqcolon\,\widecheck{T}_{2}}{\underbrace{\sum_{k=1}^{K}\left(\sum_{h=1}^{H}r_{h}(s_{h}^{k},a_{h}^{k})-V_{1}^{\pi^{k}}(s_{1}^{k})\right)}}.
	\label{eq:T4-decompose-T12-check}
 \end{align}
%
% Let $\widecheck{T}_1 =\sum_{k=1}^K \left( \sum_{h=1}^H \widehat{r}_h^k(s_h^k,a_h^k)-r_{h}(s_h^k,a_h^k)\right)  $ and $\widecheck{T}_2 =\sum_{k=1}^K \left( \sum_{h=1}^H r_h(s_h^k,a_h^k) - V_{1}^{\pi^k}(s_1^k) \right) $
%
Repeating the arguments for \eqref{eq:sum-rhat-vstar} yields
%
%
\begin{align}
	\sum_{k,h} r_{h}(s_{h}^{k},a_{h}^{k})\leq \widecheck{T}_{2}+\sum_{k}V_{1}^{\pi_{k}}(s_{1}^{k})\leq \widecheck{T}_{2}+\sum_{k}V_1^{\star}(s_{1}^{k})\leq \widecheck{T}_{2}+3Kv^{\star}+H\log\frac{1}{\delta'}
	\label{eq:sum-rnohat-vstar}
\end{align}
%
 with probability at least $1-\delta'$. 
Combining this with Lemma~\ref{lemma:bdempr},
%and Lemma~\ref{lemma:doubling}, 
we see that   
%
\begin{align}
	\widecheck{T}_1 & \leq 4\sqrt{2SAH^2\log_2 K \log \frac{1}{\delta'}} \sqrt{\sum_{k=1}^K\sum_{h=1}^H r_h(s_h^k,a_h^k)} + 52SAH^2(\log_2 K)\log \frac{1}{\delta'} \nonumber
	\\ & \leq 4\sqrt{2SAH^2\log_2 K \log \frac{1}{\delta'}} \sqrt{\widecheck{T}_2 + 3Kv^{\star}} + 60SAH^2(\log_2 K)\log \frac{1}{\delta'}\label{eq:ct1}
\end{align}
%
with probability exceeding $1-3SAHK\delta'$. 
In addition,  Lemma~\ref{lemma:self-norm} tells us that
%
\begin{align}
\widecheck{T}_2 & \leq 2\sqrt{2\sum_{k=1}^K \mathbb{E}_{\pi^k,s_1\sim \mu}\left[\left(\sum_{h=1}^H r_h(s_h,a_h) \right)^2 \right]\log \frac{1}{\delta'}}+3H^2\log \frac{1}{\delta'} \nonumber
\\ & \leq 2\sqrt{2H\sum_{k=1}^K \mathbb{E}_{\pi^k,s_1\sim \mu}\left[\sum_{h=1}^H r_h(s_h,a_h)  \right]\log \frac{1}{\delta'}}+3H\log \frac{1}{\delta'} \nonumber
\\ & \leq 2\sqrt{2KHv^{\star}\log \frac{1}{\delta'}}+3H\log \frac{1}{\delta'} \label{eq:ct1.5}
\\ & \leq 2Kv^{\star} + 5H\log \frac{1}{\delta'}\label{eq:ct2}
\end{align}
%
with probability at least $1-2SAHK\delta'$, 
where the expectation operator $\mathbb{E}_{\pi^k,s_1\sim \mu}[\cdot]$ is taken over the randomness of a trajectory $\{(s_h,a_h)\}$ 
generated under policy $\pi^k$ and initial state $s_1\sim \mu$, 
the last line arises from the AM-GM inequality, 
and the penultimate line makes use of Assumption~\ref{assum1} and the fact that 
%
\[
\mathbb{E}_{\pi^{k},s_{1}\sim\mu}\left[\sum_{h=1}^{H}r_{h}(s_{h},a_{h})\right]=\mathbb{E}_{s_{1}\sim\mu}\left[V_{1}^{\pi^{k}}(s_{1})\right]\leq\mathbb{E}_{s_{1}\sim\mu}\left[V_{1}^{\star}(s_{1})\right]=v^{\star}.
\]
%
Taking \eqref{eq:ct1}, \eqref{eq:ct1.5} and \eqref{eq:ct2} together, we can demonstrate that with probability exceeding $1-5SAHK\delta'$,
%
\begin{subequations}
\begin{align}
	& \widecheck{T}_1\leq     13\sqrt{SAH^2Kv^{\star}(\log_2 K)\log \frac{1}{\delta'}}  + 80SAH^2(\log_2 K)\log\frac{1}{\delta'},\label{eq:check-T-bound-first}
\\ & \widecheck{T}_2 \leq  2\sqrt{2KHv^{\star}\log \frac{1}{\delta'}}+3H\log\frac{1}{\delta'} .
\end{align}
\end{subequations}
%
Substitution into \eqref{eq:T4-decompose-T12-check} reveals that: with probability exceeding $1-5SAHK\delta'$, 
%
\begin{align}
	T_4 \leq 15\sqrt{SAH^2Kv^{\star}(\log_2K)\log \frac{1}{\delta'}}  + 83SAH^2(\log_2K)\log\frac{1}{\delta'}.\label{eq:fnbt4}
\end{align}


\paragraph{Bounding $T_5$.}
%
Recall that we have proven in \eqref{eq:boundt5-intermediate-13} that
%
%
\begin{align}
	T_{5} & \leq T_{7}+T_{8}+2HT_2+2H\sum_{k=1}^{K}\sum_{h=1}^{H}\widehat{r}^k_{h}(s_{h}^{k},a_{h}^{k}).\label{eq:T5-UB-first-123}
\end{align}
%
%\begin{align}
%T_5  & \leq \sum_{k=1}^K \sum_{h=1}^H   (\widehat{P}^k_{s_h^k,a_h^k,h} -P_{s_h^k,a_h^k,h})(V^k_{h+1})^2   + \sum_{k=1}^K \sum_{h=1}^H (P_{s_h^k,a_h^k,h} - \textbf{1}_{s_{h+1}^k} )  (V_{h+1}^k)^2  \nonumber
%\\ & \qquad \qquad\qquad \qquad \qquad \qquad   +2H \sum_{k=1}^K \sum_{h=1}^H b_h^k(s_h^k,a_h^k)+2H\sum_{k=1}^K \sum_{h=1}^H r_h(s_h^k,a_h^k).\nonumber
%\end{align}
%
With \eqref{eq:sum-rnohat-vstar} and \eqref{eq:ct2} in place, we can deduce that, with probability at least $1-3SAHK\delta'$, 
%
\begin{align}
\sum_{k,h}r_{h}(s_{h}^{k},a_{h}^{k}) & \leq\widecheck{T}_{2}+3Kv^{\star}+H\log\frac{1}{\delta'}\leq5Kv^{\star}+6H\log\frac{1}{\delta'}. 
	\label{eq:sum-r-UB-first}
\end{align}
%
Moreover, under the assumption~\eqref{eq:K-focus-first}, we can further bound \eqref{eq:check-T-bound-first} as
%
\[
	\widecheck{T}_{1}\leq\sqrt{BSAH^{2}Kv^{\star}}+BSAH^{2}\leq2Kv^{\star}
\]
%
with probability exceeding $1-3SAHK\delta'$, which combined with \eqref{eq:sum-r-UB-first} and the assumption~\eqref{eq:K-focus-first} results in
%
\begin{align}
\sum_{k,h}\widehat{r}_{h}^{k}(s_{h}^{k},a_{h}^{k})=\sum_{k,h}r_{h}(s_{h}^{k},a_{h}^{k})+\widecheck{T}_{1} & \leq7Kv^{\star}+6H\log\frac{1}{\delta'}\leq8Kv^{\star}.
	\label{eq:sum-hat-r-UB-first}
\end{align}
%
Substitution into \eqref{eq:T5-UB-first-123} indicates that: with probability exceeding $1-6SAHK\delta'$, 
%
\begin{align}
T_5\leq  T_7 + T_8+2HT_2 + 16HKv^{\star} .\label{eq:fnbt5}
%+ 10H^2\log \frac{1}{\delta'}
\end{align}
%





\paragraph{Bounding $T_6$.}
%
Making use of our bounds \eqref{eq:boundt6-intermediate-135}, \eqref{eq:boundt8} and \eqref{eq:sum-hat-r-UB-first}, we can readily derive
%
\begin{align}
T_{6} & \leq T_{8}+2HT_{2}+2HT_{9}+2H\sum_{k=1}^{K}\sum_{h=1}^{H}\widehat{r}_{h}(s_{h}^{k},a_{h}^{k})\nonumber\\
 %& \leq\sqrt{32T_{6}\log\frac{1}{\delta'}}+3H^{2}\log\frac{1}{\delta'}+2HT_{2}+2HT_{9}+2H\sum_{k=1}^{K}\sum_{h=1}^{H}\widehat{r}_{h}(s_{h}^{k},a_{h}^{k})\nonumber\\
 & \leq \sqrt{32T_{6}\log\frac{1}{\delta'}}+2HT_{9}+16HKv^{\star}+3H^{2}\log\frac{1}{\delta'}+2HT_{2}
	\label{eq:fnbt6-first}
\end{align}
%
with probability at least $1-16SAH^2K^2\delta'$. 





\paragraph{Putting all pieces together.}
%
% Assume that $K\geq \frac{BSAH^2}{v^{\star}}$. Then $\sqrt{BSAH^2Kv^{\star}}\geq BSAH^2$. 
%
Recalling our choice of $B$ (cf.~\eqref{eq:defn-B-first}), 
we can see from \eqref{eq:nbft2}, \eqref{eq:boundt3}, \eqref{eq:fnbt4}, \eqref{eq:fnbt5}, \eqref{eq:fnbt6-first}, \eqref{eq:boundt8}, \eqref{eq:boundt1} and \eqref{eq:boundt7} that
%
\begin{subequations}
	\label{eq:all-T-bounds-first}
\begin{align}
T_{2} & \leq\sqrt{BSAHT_{5}}+\sqrt{BSAH^{2}(T_{4}+3Kv^{\star})}+BSAH^{2},\\
T_{3} & \leq\sqrt{BT_{6}}+BH,\\
T_{4} & \leq\sqrt{BSAH^{2}Kv^{\star}}+BSAH^{2},\\
T_{5} & \leq T_{7}+T_{8}+2HT_{2}+16HKv^{\star},\\
T_{6} & \leq\sqrt{BT_{6}}+2HT_{9}+16HKv^{\star}+BH^{2}+2HT_{2},\\
T_{8} & \leq\sqrt{BH^{2}T_{6}}+BH^{2},\\
T_{1}\leq T_{9} & \leq\sqrt{BSAHT_{6}}+BSAH^{2},\\
T_{7} & \leq H\sqrt{BSAHT_{6}}+BSAH^{3}.
\end{align}
\end{subequations}
%
Solving \eqref{eq:all-T-bounds-first} under the assumption $K\geq \frac{BSAH^2}{v^{\star}}$ allows us to demonstrate that
%
\begin{subequations}
\begin{align}
	T_6 &\lesssim BHKv^{\star} \\
	T_1\leq T_9 &\lesssim \sqrt{B^2SAH^2Kv^{\star}}\\
	T_7+T_8 &\lesssim \sqrt{B^2SAH^4Kv^{\star}} \\
	T_5 &\lesssim BHKv^{\star} \\
	T_2 &\lesssim \sqrt{B^2SAH^2Kv^{\star}} \\
	T_3 &\lesssim \sqrt{B^2HKv^{\star}} \\
	T_4 & \lesssim \sqrt{BSAH^{2}Kv^{\star}}
\end{align}
\end{subequations}
%
with probability exceeding $1-200SAH^2K^2\delta'$. 
Putting these bounds together with \eqref{eq:decomposition}, we arrive at
%
\[
	\mathsf{Regret}(K)\leq T_{1}+T_{2}+T_{3}+T_{4} 
	\lesssim B\sqrt{SAH^2Kv^{\star}} 
\]
%
with probability exceeding $1-200SAH^2K^2\delta'$. 
Replacing $\delta'$ with $\frac{\delta}{200SAH^2K^2}$ and taking $\delta=\frac{1}{2KH}$ give
%
\begin{align*}
	\mathbb{E}\big[\mathsf{Regret}(K)\big]
	&\lesssim(1-\delta)B\sqrt{SAH^{2}Kv^{\star}}+\delta Kv^{\star}\lesssim B\sqrt{SAH^{2}Kv^{\star}}+1 \asymp B\sqrt{SAH^{2}Kv^{\star}} \\
	&  
	\asymp \min\big\{ B\sqrt{SAH^{2}Kv^{\star}}, B Kv^{\star} \big\} 
	\asymp  \min\big\{ \sqrt{SAH^{2}Kv^{\star}}, Kv^{\star}\big\} \log^{5}(SAHK) 
	,
\end{align*}
%
provided that $K\geq \frac{BSAH^2}{v^{\star}}$. 
Taking this collectively with \eqref{eq:E-regret-UB-easy-first} concludes the proof. 

