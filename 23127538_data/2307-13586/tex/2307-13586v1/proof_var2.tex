



Following the arguments in the proof of Lemma\ref{lemma:var1}, 
we now bound $T_2,T_4,T_5$ and $T_6$ with respect to $\mathrm{var}_2$.



\subsubsection{Bound of $T_2$}

Recall in \eqref{eq:boundt2}, we show that
\begin{align}
T_2 &\leq \frac{460}{9}\sqrt{2SAH\log_2(K)\log(\frac{1}{\delta})T_5}\nonumber
\\ & \qquad\qquad +4\sqrt{SAH\log_2(K)\log(\frac{1}{\delta})}\sqrt{\sum_{k,h}\left(\hat{\sigma}_h^k(s_h^k,a_h^k)- (\hat{r}_h^k(s_h^k,a_h^k))^2\right)} + \frac{1088}{9}SAH^2\log_2(K)\log(\frac{1}{\delta}).\label{eq:local31}
 \end{align}

 \begin{lemma}\label{lemma:ks2}
With probability $1-4SAHK\delta$,
\begin{align}
\sum_{k,h}\left(\hat{\sigma}_h^k(s_h^k,a_h^k)- (\hat{r}_h^k(s_h^k,a_h^k))^2\right) \leq 6K\mathrm{var}_2 + 242H^2\log_2(K)\log(\frac{1}{\delta}).
\end{align}
 \end{lemma}
\begin{proof}
Recall in Lemma~\ref{lemma:bdrv}, we show that with probability $1-4SAHK\delta$,
\begin{align}
    \sum_{k=1}^K\sum_{h=1}^H\left(\hat{\sigma}_h^k(s_h^k,a_h^k)- (\hat{r}_h^k(s_h^k,a_h^k))^2\right)\leq 3\sum_{k=1}^K  \tilde{V}_1^k(s_1^k)+2SAH^3\log_2(K)\log(\frac{1}{\delta}).
\end{align}
We then complete the proof by noting that
\begin{align}
\tilde{V}_1^k(s_1^k)\leq  \tilde{V}_1^k(s_1^k)+\mathbb{E}_{\pi^k}\left[\sum_{h=1}^H \mathbb{V}(P_{s_h,a_h,h},V_{h+1}^{\pi^k})|s_1=s_1^k\right]  = \mathrm{Var}_{\pi^k}\left[\sum_{h=1}^H r_h(s_h,a_h)|s_1=s_1^k\right]\leq \mathrm{var}_2.
\end{align}
\end{proof}
 

By Lemma~\ref{lemma:ks2}, with probability $1-4SAHK\delta$,
\begin{align}
T_2\leq \frac{460}{9}\sqrt{2SAH\log_2(K)\log(\frac{1}{\delta})T_5} +12\sqrt{SAH\log_2(K)\log(\frac{1}{\delta})}\sqrt{2K\mathrm{var}_1} + 157SAH^2\log_2(K)\log(\frac{1}{\delta}).\label{eq:xbt2}
\end{align}


\subsubsection{Bound of $T_4$}

Recall that $T_4 = \check{T}_1 + \check{T}_2$ where $\check{T}_1 = \sum_{k=1}^K\sum_{h=1}^H \left(\hat{r}_h^k(s_h^k,a_h^k) -r_h(s_h^k,a_h^k) \right)$ and $\check{T}_2 = \sum_{k=1}^K\left(\sum_{h=1}^H r_h(s_h^k,a_h^k) -V_{1}^{\pi^k}(s_1^k) \right)$.

Following the arguments in Lemma~\ref{lemma:bdrv} and \eqref{eq:wc3}, with probability $1-6SAHK\delta$,
\begin{align}
|\check{T}_1| & \leq \sqrt{4SAH\log_2(K)\log(\frac{1}{\delta})}\cdot \sqrt{\sum_{k=1}^K\sum_{h=1}^H v_h(s_h^k,a_h^k)} + 2SAH^2\log_2(K)\log(\frac{1}{\delta}) \nonumber
\\ & \leq \sqrt{8SAHK\mathrm{var}_1\log_2(K)\log(\frac{1}{\delta})}+ 20SAH^2\log_2(K)\log(\frac{1}{\delta}).\nonumber
\end{align}

On the other hand, by Lemma~\ref{lemma:self-norm} and the definition of $\mathrm{var}_2$, with probability $1-2SAHK\delta$
\begin{align}
|\check{T}_2|\leq 2\sqrt{2K\mathrm{var}_2 \log(\frac{1}{\delta})}+3H\log(\frac{1}{\delta}).
\end{align}

Therefore, with probability $1-8SAHK\delta$,
\begin{align}
T_4 \leq 4\sqrt{2SAHK\mathrm{var}_2\log_2(K)\log(\frac{1}{\delta})}+ 23SAH^2\log_2(K)\log(\frac{1}{\delta}).\label{eq:xbt4}
\end{align}

\subsubsection{Bounds of $T_5$ and $T_6$}




Recall Lemma~\ref{lemma:empv} states that with probability $1-2\delta$, $T_5\leq 5T_6+8BSAH^3$. So it suffices to bound $T_5$.


Because $\mathrm{Var}(X+Y)\leq 2(\mathrm{Var}(X)+\mathrm{Var}(Y))$ for any two random variable $X,Y$ with finite variance, we have that
\begin{align}
\sum_{k,h} \mathbb{V}(P_{s_h^k,a_h^k,h},V_{h+1}^k)  & \leq  2\sum_{k,h} \mathbb{V}(P_{s_h^k,a_h^k,h},V_{h+1}^{\pi^k} ) + 2\sum_{k,h}\mathbb{V}(P_{s_h^k,a_h^k,h},V_{h+1}^k -V_{h+1}^{\pi^k} )\nonumber
\\ & \leq 3   K\mathrm{var}_1 +\sum_{k=1}^K \left( \sum_{h=1}^H \mathbb{V}(P_{s_h^k,a_h^k,h},V_{h+1}^{\pi^k} ) - 3\mathrm{var}_1 \right) + 2\sum_{k,h}\mathbb{V}(P_{s_h^k,a_h^k,h},V_{h+1}^k -V_{h+1}^{\pi^k} ).\label{eq:var01}
\end{align}




\begin{lemma}\label{lemma:kx1}
With probability $1-4SAHK\delta$, it holds that
\begin{align}
\sum_{k=1}^K \left( \sum_{h=1}^H \mathbb{V}(P_{s_h^k,a_h^k,h},V_{h+1}^{\pi^k} ) -2\mathrm{var}_2 \right)  \leq 80H^2\log(\frac{1}{\delta}).
\end{align}
\end{lemma}


\begin{proof}
Let $\check{R}^k_{h}(s,a) = \mathbb{V}(P_{s,a,h},V_{h+1}^{\pi^k})$. Define
\begin{align}
\check{V}^k_h (s) = \mathbb{E}\left[ \sum_{h'=h}^H \check{R}^k_{h'}(s_{h'},a_{h'}) |s_h = s\right].\nonumber
\end{align}
Then $\check{V}_h^k(s)\leq\mathrm{var}_2 \leq  H^2$. 
We have that
\begin{align}
  \sum_{h=1}^H  \mathbb{V}(P_{s_h^k,a_h^k,h},V_{h+1}^{\pi^k})-\mathrm{var}_2 \nonumber & =\sum_{h=1}^H  \check{R}_h^k(s_h^k,a_h^k)-\mathrm{var}_2  \nonumber
 \\ & \leq \sum_{h=1}^H \check{R}_h^k(s_h^k,a_h^k) - \check{V}^k_1(s_1^k)\nonumber
 \\ & = \sum_{h=1}^H \left(\textbf{1}_{s_{h+1}^k} - P_{s_{h}^k,a_h^k,h}  \right)\check{V}^k_{h+1}.
\end{align}

Note that $\check{V}^k$ only depends on $\pi^k$, which is determined before the $k$-th episode start.  
With Lemma~\ref{lemma:self-norm}, with probability $1-2SAHK\delta$,
\begin{align}
 & \sum_{k=1}^K \left( \sum_{h=1}^H \mathbb{V}(P_{s_h^k,a_h^k,h},V_{h+1}^{\pi^k}) - \check{V}_1^k(s_1^k)  \right) \nonumber
 \\ & \leq 2\sqrt{2\sum_{k=1}^K \sum_{h=1}^H \mathbb{V}(P_{s_h^k,a_h^k,h},\check{V}_{h+1}^k)\log(\frac{1}{\delta}) } + 3H^2\log(\frac{1}{\delta}).\label{eq:xlll11}
\end{align}

We further bound
\begin{align}
 & \sum_{k=1}^K \sum_{h=1}^H \mathbb{V}(P_{s_h^k,a_h^k,h},\check{V}_{h+1}^k) \nonumber
 \\ &  =\sum_{k=1}^K \sum_{h=1}^H \left( P_{s_h^k,a_h^k,h} (\check{V}_{h+1}^k)^2 - (P_{s_h^k,a_h^k,h}\check{V}_{h+1}^k)^2 \right)\nonumber
 \\ & = \sum_{k=1}^K \sum_{h=1}^H \left( P_{s_h^k,a_h^k,h} -\textbf{1}_{s_{h+1}^k}  \right)(\check{V}_{h+1}^k)^2  \nonumber
 \\ & \qquad \qquad + \sum_{k=1}^H \sum_{h=1}^H \left((\check{V}_{h+1}^k(s_{h+1}^k))^2- (\check{V}_h^k(s_h^k))^2 \right)+\sum_{k=1}^K \sum_{h=1}^H \left( (\check{V}_h^k(s_h^k))^2 - (P_{s_h^k,a_h^k,h}\check{V}_{h+1}^k)^2\right) \nonumber
 \\ & \leq 2\sqrt{8H^4\sum_{k=1}^K \sum_{h=1}^H \mathbb{V}(P_{s_h^k,a_h^k,h},\check{V}_{h+1}^k)\log(\frac{1}{\delta})   } +2H^2\sum_{k=1}^K\sum_{h=1}^H \check{R}_h(s_h^k,a_h^k) + 3H^4\log(\frac{1}{\delta}).\label{eq:xllll21}
\end{align}
Here the last inequality is by Lemma~\ref{lemma:self-norm} and Lemma~\ref{lemma:sqv} (with probability $1-2SAHK\delta$) and the fact that $\check{V}_h^k(s_h^k) = \check{R}_h(s_h^k,a_h^k)+P_{s_h^k,a_h^k,h}\check{V}_{h+1}^k$.

It then follows that
\begin{align}
\sum_{k=1}^K \sum_{h=1}^H \mathbb{V}(P_{s_h^k,a_h^k,h},\check{V}_{h+1}^k)  \leq 4H^2\sum_{k=1}^K \sum_{h=1}^H \check{R}_h(s_h^k,a_h^k) + 42 H^4\log(\frac{1}{\delta}).\label{eq:xlll31}
\end{align}

By \eqref{eq:xlll11} and \eqref{eq:xlll31}, we learn that
\begin{align}
\sum_{k=1}^K \sum_{h=1}^H \mathbb{V}(P_{s_h^k,a_h^k,h},V_{h+1}^{\pi^k}) \leq \sum_{k=1}^H \check{V}_1^k(s_1^k) +2\sqrt{8 H^2 \sum_{k=1}^K \sum_{h=1}^H \mathbb{V}(P_{s_h^k,a_h^k,h},V_{h+1}^{\pi^k}) \log(\frac{1}{\delta})     } + 21H^2\log(\frac{1}{\delta}),\nonumber
\end{align}
which further implies that
\begin{align}
\sum_{k=1}^K \sum_{h=1}^H \mathbb{V}(P_{s_h^k,a_h^k,h},V_{h+1}^{\pi^k}) \leq 2\sum_{k=1}^K \check{V}_1^k(s_1^k)+ 84H^2\log(\frac{1}{\delta}) \leq 2K\mathrm{var}_2 + 84H^2\log(\frac{1}{\delta}).\nonumber
\end{align}

The proof is finished.







\end{proof}


For the left term $\sum_{k,h}\mathbb{V}(P_{s_h^k,a_h^k,h},V_{h+1}^k -V_{h+1}^{\pi^k})$,   we have the lemma below.
\begin{lemma}\label{lemma:bdv11}
With probability $1-4SAKH\delta$, it holds that
\begin{align}
 & \sum_{k,h}\mathbb{V}(P_{s_h^k,a_h^k,h}, V_{h+1}^k -V_{h+1}^{\pi^k})   \leq 4\sqrt{BH^2\sum_{k,h}\mathbb{V}(P_{s_h^k,a_h^k,h},V_{h+1}^k)}+ 4H\sum_{k,h}b_h^k(s_h^k,a_h^k)+ 3BSAH^3.\nonumber
\end{align}
\end{lemma}
\begin{proof}[Proof of Lemma~\ref{lemma:bdv1}]
Direct computation gives that
\begin{align}
 & \sum_{k,h}\mathbb{V}(P_{s_h^k,a_h^k,h},V_{h+1}^k - V_{h+1}^{\pi^k})  \nonumber
 \\ & =\sum_{k,h} \left(  P_{s_h^k,a_h^k,h} (V_{h+1}^k - V_{h+1}^{\pi^k})^2 - (P_{s_h^k,a_h^k,h} (V_{h+1}^k - V_{h+1}^{\pi^k}) )^2       \right)
 \\ &  = \sum_{k,h} \left(    (P_{s_h^k,a_h^k,h}-\textbf{1}_{s_{h+1}^k}) (V_{h+1}^k - V_{h+1}^{\pi^k})^2   \right) \nonumber
 \\ & \qquad \qquad \qquad + \sum_{k,h} \left( (V_{h+1}^k(s_{h+1}^k)- V_{h+1}^{\pi^k}(s_{h+1}^k))^2 - ((P_{s_h^k,a_h^k,h} (V_{h+1}^k - V_{h+1}^{\pi^k}) )^2   ) \right)\nonumber
 \\ &  =\sum_{k,h} \left(    (P_{s_h^k,a_h^k,h}-\textbf{1}_{s_{h+1}^k}) (V_{h+1}^k - V_{h+1}^{\pi^k})^2   \right) + \sum_{k,h} \left( (V_{h}^k(s_{h}^k)- V_{h}^{\pi^k}(s_{h}^k))^2 - ((P_{s_h^k,a_h^k,h} (V_{h+1}^k - V_{h+1}^{\pi^k}) )^2   ) \right).\label{eq:var11}
 \end{align}


By Lemma~\ref{lemma:self-norm} and Lemma~\ref{lemma:sqv}, with probability $1-2SAKH\delta$, it holds that
\begin{align}
 \sum_{k,h} \left(    (P_{s_h^k,a_h^k,h}-\textbf{1}_{s_{h+1}^k}) (V_{h+1}^k - V_{h+1}^{\pi^k})^2   \right)\leq 2\sqrt{2}\sqrt{4H^2\sum_{k,h}\mathbb{V}(P_{s_h^k,a_h^k,h}, V_{h+1}^k - V_{h+1}^{\pi^k} )\log(\frac{1}{\delta})} + 3H^2\log(\frac{1}{\delta}).\label{eq:var31}
\end{align}

On the other hand, with probability $1-2SAKH\delta$,
 \begin{align}
 & \sum_{k,h} \left( (V_{h}^k(s_{h}^k)- V_{h}^{\pi^k}(s_{h}^k))^2 - ((P_{s_h^k,a_h^k,h} (V_{h+1}^k - V_{h+1}^{\pi^k}) )^2   ) \right)\nonumber
 \\ & \leq  2H\sum_{k,h} \max\{ V_h^k(s_h^k)-P_{s_h^k,a_h^k,h} V_{h+1}^k - (V^{\pi^k}_{h}(s_h^k)-P_{h}^k V_{h+1}^{\pi^k}) ,0  \}  \nonumber
\\ & = 2H \sum_{k,h} \max\{V_h^k(s_h^k) - P_{s_h^k,a_h^k,h}V_{h+1}^k-r_h(s_h^k,a_h^k),0\} \nonumber
\\ & \leq 2H\sum_{k,h}\max\{(\hat{P}_{s_h^k,a_h^k,h}-P_{s_h^k,a_h^k,h})V_{h+1}^k ,0\} +2H\sum_{k,h} b_h^k(s_h^k,a_h^k)\nonumber
\\ & \leq  2\sqrt{BSAH^3\sum_{k,h}\mathbb{V}(P_{s_h^k,a_h^k,h}, V_{h+1}^k)}+2H\sum_{k,h}b_h^k(s_h^k,a_h^k) + BSAH^3.\label{eq:var41}
 \end{align}



It then follows that, with probability $1-4SAKH\delta$,
\begin{align}
 & \sum_{k,h}\mathbb{V}(P_{s_h^k,a_h^k,h}, V_{h+1}^k -V_{h+1}^{\pi^k})  \leq 4\sqrt{BSAH^3\sum_{k,h}\mathbb{V}(P_{s_h^k,a_h^k,h},V_{h+1}^k)}+ 4H\sum_{k,h}b_h^k(s_h^k,a_h^k)+ 3BSAH^3.
\end{align}


The proof is completed.





\end{proof}


By Lemma~\ref{lemma:kx1} and Lemma~\ref{lemma:bdv11}, we have that with probability $1-6SAHK\delta$,
\begin{align}
T_6:=&\sum_{k,h}\mathbb{V}(P_{s_h^k,a_h^k,h},V_{h+1}^k)\nonumber
\\ & \leq  2\sum_{k,h}\mathbb{V}(P_{s_h^k,a_h^k,h},V_{h+1}^{\pi^k})+2\sum_{k,h}\mathbb{V}(P_{s_h^k,a_h^k,h},V_{h+1}^k - V_{h+1}^{\pi^k})\nonumber
\\ & \leq 4K\mathrm{var}_2 +  8\sqrt{BSAH^3 T_6}+8HT_2 + 7BSAH^3\nonumber
\\ & \leq 8K\mathrm{var}_2 + 16HT_2 + 78BSAH^3.\label{eq:xbt6}
\end{align}

By Lemma~\ref{lemma:empv} and \eqref{eq:xbt6}, with probability $1-8SAHK\delta$, it holds that
\begin{align}
T_5:=\sum_{k,h}\mathbb{V}(\hat{P}_{s_h^k,a_h^k,h},V_{h+1}^k)\leq 40K\mathrm{var}_2 + 80HT_2+398BSAH^3.\label{eq:xbt5}
\end{align}



Then we have 
\begin{align}
\sum_{k,h}\mathbb{V}(P_{s_h^k,a_h^k,h}, V_{h+1}^k) &  \leq 2 \sum_{k,h}\mathbb{V}(P_{s_h^k,a_h^k,h},V_{h+1}^{\pi^k}) + 2\sum_{k,h}\mathbb{V}(P_{s_h^k,a_h^k,h}, V_{h+1}^k -V_{h+1}^{\pi^k} )\nonumber
\\ & \leq 6\sum_{k=1}^K\check{\mathrm{var}}^k + \sum_{k=1}^K\left( \sum_{h=1}^H \mathbb{V}(P_{s_h^k,a_h^k,h},V_{h+1}^{\pi^k}) - 3\check{\mathrm{var}}^k \right) + 2\sum_{k,h}\mathbb{V}(P_{s_h^k,a_h^k,h}, V_{h+1}^k -V_{h+1}^{\pi^k} )\nonumber
\\ & \leq 6K\mathrm{var}_2 + \sum_{k=1}^K\left( \sum_{h=1}^H \mathbb{V}(P_{s_h^k,a_h^k,h},V_{h+1}^{\pi^k}) - 3\check{\mathrm{var}}^k \right) + 2\sum_{k,h}\mathbb{V}(P_{s_h^k,a_h^k,h}, V_{h+1}^k -V_{h+1}^{\pi^k} ).\label{eq:var6}
\end{align}



\iffalse
\begin{lemma}\label{lemma:l2}
With probability $1-4SAKH\delta$,
\begin{align}
\sum_{k,h} \mathbb{V}(P_{s_h^k,a_h^k,h}, V_{h+1}^k-V_{h+1}^{\pi^k}) \leq O\left(   H\sqrt{\sum_{k,h}\mathbb{V}(P_{s_h^k,a_h^k,h},V_{h+1}^k)\left(SAH+\log(\frac{1}{\delta}\right))} +\sum_{k,h}b_h^k(s_h^k,a_h^k)+ H^2\left(SAH+\log(\frac{1}{\delta})\right)\right).
\end{align}
\end{lemma}
\begin{proof}[Proof of Lemma~\ref{lemma:l2}] With Lemma~\ref{lemma:self-norm}, with probability $1-2SAHK\delta$
\begin{align}
 & \sum_{k,h}\mathbb{V}(P_{s_h^k,a_h^k,h}, V^k_{h+1}- V_{h+1}^{\pi^k}) \nonumber
\\ &= \sum_{k,h} \left(   P_{s_h^k,a_h^k,h} (V_{h+1}^k - V_{h+1}^{\pi^k})^2 - (P_{s_h^k,a_h^k,h} (V_{h+1}^k-V_{h+1}^{\pi^k}))^2      \right)\nonumber
\\ &  = \sum_{k,h}\left( (P_{s_h^k,a_h^k,h}-\textbf{1}_{s_{h+1}^k})(V_{h+1}^k -V_{h+1}^{\pi^k})^2  \right) +\sum_{k,h} \left(  (V_{h+1}^k(s_{h+1}^k) - V_{h+1}^{\pi^k}(s_{h+1}^k))^2  - (P_{s_h^k,a_h^k,h}(V_{h+1}^k -V_{h+1}^{\pi^k}))^2  \right)\nonumber
\\ & \leq  \sum_{k,h}\left( (P_{s_h^k,a_h^k,h}-\textbf{1}_{s_{h+1}^k})(V_{h+1}^k -V_{h+1}^{\pi^k})^2  \right) +\sum_{k,h} \left(  (V_{h}^k(s_{h}^k) - V_{h}^{\pi^k}(s_{h}^k))^2  - (P_{s_h^k,a_h^k,h}(V_{h+1}^k -V_{h+1}^{\pi^k}))^2  \right)\nonumber
\\ & \leq O\left( H\sqrt{\sum_{k,h}\mathbb{V}(P_{s_h^k,a_h^k,h},V_{h+1}^k -V_{h+1}^{\pi^k})  \log(\frac{1}{\delta})}  +H^2\log(\frac{1}{\delta}) \right) \nonumber
\\ & \quad \quad \qquad \qquad +2H\sum_{k,h} \max\{     (V_{h}^k(s_h^k)-P_{s_h^k,a_h^k,h} V_{h+1}^k) - (V_{h}^{\pi^k}(s_h^k)-P_{s_h^k,a_h^k,h}V_{h+1}^{\pi^k})   ,0\}.\label{eq:v81}
\end{align}

By the definition of $V_{h}^k$ and $V_{h}^{\pi^k}$, we have that
\begin{align}
& V_h^k(s_h^k)\leq r_h(s_h^k,a_h^k) + \hat{P}^k_{s_h^k,a_h^k,h} + b_h^k(s_h^k,a_h^k)\nonumber
\\ & V_{h}^{\pi^k}(s_h^k) = r_h(s_h^k,a_h^k) +P_{s_h^k,a_h^k,h}V_{h+1}^{\pi^k}.\nonumber
\end{align}
It then holds that
\begin{align}
&\max\{     (V_{h}^k(s_h^k)-P_{s_h^k,a_h^k,h} V_{h+1}^k) - (V_{h}^{\pi^k}(s_h^k)-P_{s_h^k,a_h^k,h}V_{h+1}^{\pi^k})   ,0\} \nonumber
\\ & \leq \max\{(\hat{P}_{s_h^k,a_h^k,h}-P_{s_h^k,a_h^k,h})V_{h+1}^{k},0 \} +\sum_{k,h}b_h^k(s_h^k,a_h^k).\label{eq:v7}
\end{align}

By Lemma~\ref{lemma:key3}, we further have that, with probability $1-2SAKH\delta$,
\begin{align}
\sum_{k,h}\max\{(\hat{P}_{s_h^k,a_h^k,h}-P_{s_h^k,a_h^k,h})V_{h+1}^{k},0 \} \leq O\left( \sqrt{\sum_{k,h}\mathbb{V}(P_{s_h^k,a_h^k,h}, V_{h+1}^{k})\left(SAH+\log(\frac{1}{\delta}) \right) }+H\left(SAH+\log(\frac{1}{\delta})\right)    \right).\label{eq:v8}
\end{align}

By \eqref{eq:v81}, \eqref{eq:v7} and \eqref{eq:v8}, we obtain that 
\begin{align}
&\sum_{k,h}\mathbb{V}(P_{s_h^k,a_h^k,h},V_{h+1}^k - V_{h+1}^{\pi^k})\nonumber
\\ & \leq O\left(H \sqrt{\sum_{k,h}\mathbb{V}(P_{s_h^k,a_h^k,h},V_{h+1}^k - V_{h+1}^{\pi^k})\log(\frac{1}{\delta})}+ H\sqrt{\sum_{k,h}\mathbb{V}(P_{s_h^k,a_h^k,h},V_{h+1}^k)\left(SAH+\log(\frac{1}{\delta}\right)} \right)\nonumber
\\ & \qquad \qquad \qquad \qquad \qquad \qquad \qquad \qquad \qquad \qquad\qquad \qquad+ O\left( \sum_{k,h}b_h^k(s_h^k,a_h^k)+ H^2\left(SAH+\log(\frac{1}{\delta})\right) \right)\nonumber
\end{align}
holds with probability $1-4SAKH\delta$, which further implies that
\begin{align}
\sum_{k,h}\mathbb{V}(P_{s_h^k,a_h^k,h},V_{h+1}^k - V_{h+1}^{\pi^k}) \leq O\left(   H\sqrt{\sum_{k,h}\mathbb{V}(P_{s_h^k,a_h^k,h},V_{h+1}^k)\left(SAH+\log(\frac{1}{\delta}\right))} +\sum_{k,h}b_h^k(s_h^k,a_h^k)+ H^2\left(SAH+\log(\frac{1}{\delta})\right)\right) .
\end{align}
The proof is completed.
\end{proof}

\fi 


By Lemma~\ref{lemma:kx1}, \ref{lemma:empv} and Lemma~\ref{lemma:bdv11},  with probability $1-18SAHK\delta$, it holds that
\begin{align}
 & T_5 \leq O\left( K\mathrm{var}_2 + H\sqrt{T_6\left(SAH+\log(\frac{1}{\delta})\right)} +T_2+H^2(SAH+\log(\frac{1}{\delta}))    \right) ;\label{eq:nbbt5}
\\ & T_6 \leq O\left( K\mathrm{var}_2 + H\sqrt{T_6\left(SAH+\log(\frac{1}{\delta})\right)} +T_2+H^2(SAH+\log(\frac{1}{\delta}))    \right) .\label{eq:nbbt6}
\end{align}








\subsubsection{Putting All Together}
Recall $B =4000 \log_2^3(K)\log(3SA)\log(\frac{1}{\delta})$. 
We rewrite the inequalities $\eqref{eq:obt1}-\eqref{eq:obt8}$ as follows with \eqref{eq:obt2}, \eqref{eq:obt4}, \eqref{eq:obt5} and \eqref{eq:obt6}  replaced by  \eqref{eq:xbt2},\eqref{eq:xbt4}
\eqref{eq:xbt5} and \eqref{eq:xbt6} respectively. With probability $1-200SAH^2K^2\delta$, it holds that
\begin{align}
& T_1 \leq \sqrt{128BSAHT_6}+24BSAH^2;\nonumber
\\ & T_7 \leq H\sqrt{512BSAHT_6}+24BSAH^3;\nonumber
\\ & T_9 \leq \sqrt{128BSAHT_6}+24BSAH^2;\nonumber
\\ & T_2\leq 100 \sqrt{BSAHT_5}+140BSAH^2;\nonumber
 \\ & T_3 \leq \sqrt{8BT_6}+3H\log(\frac{1}{\delta})  ;\nonumber
 \\ & T_4 \leq \sqrt{BSAHK\mathrm{var}_2}+BSAH^2;\nonumber
 \\ & T_5 \leq 40K\mathrm{var}_2 + 80HT_2+398BSAH^3  ;\nonumber
 \\ &  T_6 \leq  8K\mathrm{var}_2 + 16HT_2 + 78BSAH^3 ;\nonumber
 \\ & T_8 \leq \sqrt{32BH^2T_6 } + 3BH^2 .\nonumber
\end{align}

Solving the inequalities above, we obtain that
\begin{align}
\mathrm{Regret}(K)= T_1+T_2+T_3+T_4 \leq O\left( \sqrt{BSAHK\mathrm{var_2} }+ BSAH^2 \right).\label{eq:rbvar2}
\end{align}
The proof is completed by replacing $\delta$ with $\frac{\delta}{200SAH^2K^2}$.  
