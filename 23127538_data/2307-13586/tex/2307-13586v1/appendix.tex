
\section{Technical lemmas}

\begin{lemma}\label{lemma:self-norm}
Let $(M_n)_{n\geq 0}$ be a martingale such that $M_0=0$ and $|M_n-M_{n-1}|\leq c$ for some $c>0$ and any $n\geq 1$. Let $\mathrm{Var}_{n} = \sum_{k=1}^n \mathbb{E}\left[  (M_{k}-M_{k-1})^2 |\mathcal{F}_{k-1}\right]$ for $n\geq 0$, where $\mathcal{F}_k = \sigma(M_1,...,M_{k})$. Then for any positive integer $n$, and any $\epsilon,\delta>0$, one has 
%
\begin{align}
	\mathbb{P} \left[       |M_n|\geq 2\sqrt{2}\sqrt{\mathrm{Var}_n \ln\frac{1}{\delta} } +2\sqrt{\epsilon \ln\frac{1}{\delta} } +2c\ln\frac{1}{\delta} \right]\leq 2\left(\log_2\left(\frac{nc^2}{\epsilon}\right) +1 \right)\delta.\nonumber
\end{align}
%
\end{lemma}
%
\begin{lemma}[Lemma 30 in \cite{chen2021implicit}]\label{lemma:sqv}
Let $X$ be a random variable and $\|X\|_{\infty}$ denotes the largest possible value of $X$. Let $\mathrm{Var}(X)$ denote the variance of $X$. Then $\mathrm{Var}(X^2)\leq 4 \|X\|_{\infty}^2 \mathrm{Var}(X)$.
\end{lemma}

\begin{lemma}[Lemma 10 in \cite{zhang2022horizon}]\label{lemma:con}
Let $X_1,X_2,\ldots$ be a sequence of random variables taking value in $[0,l]$. Define $\mathcal{F}_k =\sigma(X_1,X_2,\ldots,X_{k-1})$ and $Y_k = \mathbb{E}[X_k|\mathcal{F}_k]$ for $k\geq 1$. For any $\delta>0$, we have 
%
\begin{align}
& \mathbb{P}\left[ \exists n, \sum_{k=1}^n X_k \geq  3\sum_{k=1}^n Y_k+ l\ln\frac{1}{\delta}\right]\leq \delta\nonumber
\\  & \mathbb{P}\left[  \exists n,  \sum_{k=1}^n Y_k \geq 3\sum_{k=1}^n X_k + l\ln\frac{1}{\delta}  \right]    \leq \delta .\nonumber 
\end{align}
%
\end{lemma}

\begin{lemma}[Bennet's inequality]\label{bennet}
Let $Z,Z_1,...,Z_n$  be i.i.d.~random variables with values in $[0,1]$ and let $\delta>0$. Define $\mathbb{V}Z = \mathbb{E}\left[(Z-\mathbb{E}Z)^2 \right]$. Then one has
%
\begin{align}
\mathbb{P}\left[ \left|\mathbb{E}\left[Z\right]-\frac{1}{n}\sum_{i=1}^n Z_i  \right| > \sqrt{\frac{  2\mathbb{V}Z \ln(2/\delta)}{n}} +\frac{\ln(2/\delta)}{n} \right]\leq \delta.\nonumber
\end{align}
%
\end{lemma}

\begin{lemma}[Theorem 4 in  \cite{maurer2009empirical}  ]\label{empirical bernstein}
Let $Z,Z_1,...,Z_n$ ($n\geq 2$) be i.i.d.~random variables with values in $[0,1]$ and let $\delta>0$. Define $\overline{Z} = \frac{1}{n}\sum_{i=1}^n Z_{i}$ and $\widehat{V}_n  = \frac{1}{n}\sum_{i=1}^n (Z_i- \overline{Z})^2$. Then we have
%
\begin{align}
\mathbb{P}\left[ \left|\mathbb{E}\left[Z\right]-\frac{1}{n}\sum_{i=1}^n Z_i  \right| > \sqrt{\frac{  2\widehat{V}_n \ln(2/\delta)}{n-1}} +\frac{7\ln(2/\delta)}{3(n-1)} \right] \leq \delta.\nonumber
\end{align}
%
\end{lemma}


\begin{lemma}\label{lemma:doubling}
%
Recall the definition of $N_h^k(s_h^k,a_h^k)$ in Algorithm~\ref{alg:main}. It holds that
%
\begin{align}
\sum_{k=1}^K \sum_{h=1}^H \frac{1}{\max\{ N_h^k(s_h^k,a_h^k),1\}}\leq 2SAH\log_2 K
\end{align}
%
\end{lemma}
\begin{proof}
%
By definition, for any fixed $(s,a,h)$ we have
%
\begin{align}
\sum_{k=1}^K \frac{1}{ N_h^k(s_h^k,a_h^k)}  \mathbb{I}\Big[(s,a)=\big(s_h^k,a_h^k \big) \Big] \leq \log_2 K+1.
\end{align}
%
Summing over all $(s,a,h)$ completes the proof.
\end{proof}


%\section{Discussion for the Discounted MDP}
%In the discounted case, the random variables are more complicated. Using leave-one-out method, we can ensure $(\widehat{P}_{s,a}-P_{s,a})$ is independent of $V^{(s,a)}$, but the sum of error $\sum_{s,a}(\widehat{P}_{s,a}-P_{s,a}) V^{(s,a)}$ is not a martingale (actually it is hard to find a group of $\sigma$-field to make it a martingale), since $V^{(s,a)}$ also depends on $\{ \widehat{P}_{s',a'} \}_{(s',a')\neq (s,a)}$. Deeper analysis shows that to control the error term, we need to find concentration inequalities for a group of coupling random variables, e.g., to prove that $X+XY+Y\leq O(\sqrt{(\mathrm{Var}(X)+\mathrm{Var}(XY)+\mathrm{Var}(Y))\iota})$ with probability $1-\delta$. The example above seems easy, but we are not aware of efficient concentration when the number of random variables is large.


\section{Missing proofs in Section~\ref{sec:tec}}\label{app:mfsectec}


\subsection{Proof of Lemma~\ref{lemma:key1}}\label{app:pfseckey1}


 
 Since $\mathcal{X}_{h+1}$ has at most $W$ elements, we can write $\mathcal{X}_{h+1} = \{x_{h+1}(w)  \}_{w=1}^W$, where each $x_{h+1}(w)$ could be regarded as a function of $\{\widehat{P}^{(J^{k}_{s,a,h})}_{s,a,h'}, \widehat{r}^{(J^k_{s,a,h'})_{h'}(s,a)}, \widehat{\sigma}^{(J^k_{s,a,h'})}_{h'}(s,a)\}_{ h+1\leq h'\leq H, 1\leq k\leq K,(s,a)}$  and $\{J^k\}_{k=1}^K$. Now we fix a group of indices $\{w_{s,a,h}\}_{(s,a,h)}$ where $1\leq w_{s,a,h}\leq W$ for each $(s,a,h)$.

Fix $2\leq l \leq \log_2(K)+1$. We consider to bound the term
%
\begin{align}
T(l,\{w_{s,a,h}\}_{(s,a,h)}) \coloneqq 2^{l-2}\sum_{s,a,h}(\widehat{P}^{(l)}_{s,a,h} - P_{s,a,h}) x_{h+1}(w_{s,a,h}).\nonumber 
\end{align}
%
Recall that $\widehat{P}^{(l)}_{s,a,h}$ is the empirical transition of 
the $l$-th batch of $(s,a,h)$, i.e., the empirical transition of the $2^{l-2}+1$-th to $2^{l-1}$-th samples of $(s,a,h)$. Also recall the definition of $\mathcal{F}_{\mathrm{gen}} = \{\widetilde{F}(z)\}_{z=1}^{SAHK}$ in Definition~\ref{def:filt2}. 
 Conditioned on $\widetilde{\mathcal{F}}((H-h)\cdot SAK)$, $\{x_{h+1}(w_{s,a,h})\}_{(s,a)}$ is fixed, and $\{2^{l-2}\widehat{P}^{(l)}_{s,a,h}\}_{(s,a)}$ are mutually independent multinomial random variables.   Let $v_{s,a,h}(t,l)$ be the next state of the $t$-th sample of  the $ l$-th batch of $(s,a,h)$.   
By writing 
%
\begin{align}
T(l,\{w_{s,a,h}\}_{(s,a,h)}) = \sum_{s,a,h}\sum_{\tau=1}^{2^{l-2}}  ( \textbf{1}_{v_{s,a,h}(\tau,l)} -P_{s,a,h}) \cdot x_{h+1}(w_{s,a,h}),
\end{align}
%
using Lemma~\ref{lemma:self-norm}, with probability at least $1-10SAH^2K^2\delta'$,
%
\begin{align}
T(l,\{w_{s,a,h}\}_{(s,a,h)})  \leq 2\sqrt{2}\cdot \sqrt{ 2^{l-2}\sum_{s,a,h} \mathbb{V}(P_{s,a,h},x_{h+1}(w_{s,a,h}))   \log \frac{1}{\delta'}  } + 3H \log \frac{1}{\delta'}.\label{eq:xx1}
\end{align}


%With $(z,s,a,h,t)$ satisfying $z = (H-h)\cdot SA\cdot 2^{l-2}+ (i-1)\cdot 2^{l-2} + t $, we define $M^{l}(z):= 2^{l-1}\sum_{\alpha=1}^{SA}\sum_{h'\geq h+1}(\widehat{P}^{(l)}_{s^{(\alpha)},a^{(\alpha)},h'}-P_{s^{(\alpha)},a^{(\alpha)},h'} )x^k_{h'+1}(w_{s,a,h'})  + \sum_{\alpha=1}^{i-1}(\widehat{P}^{(l)}_{s^{(\alpha)},a^{(\alpha)},h}-P_{s^{(\alpha)},a^{(\alpha)},h} )x^k_{h'+1}(w_{s,a,h'}) +\sum_{\tau=1}^t ( \textbf{1}_{s'_{s,a,h}(\tau)} x^{k}_{h+1}(w_{s,a,h})-P_{s,a,h} x^{k}_{h+1}(w_{s,a,h})$. 


%Note that $Y_{s^{(i)},a^{(i)},h}$ is measurable with respect to $\mathcal{F}^{l}(z-1)$ for $z = (H-h)\cdot SA\cdot 2^l + (i-1)\cdot 2^l + t $. We then obtain that $\{M^l(z)\}_{z=1}^{SAH\cdot 2^l}$ is a martingale with respect to $\{\mathcal{F}^{l}(z)\}_{z=1}^{SAH\cdot 2^l}$.






Note that $\{w_{s,a,h}\}_{(s,a,h)}$ has at most $(W)^{SAH}$ choices.
 Applying the union bound and rescaling $\delta'$ to $\frac{\delta'}{|W|^{SAH}}$, we see that: with probability at least $1-10SAH^2K^2\delta'$, 
 %
\begin{align}
 & T(l,\{w_{s,a,h}\}_{(s,a,h)})\nonumber\\ &= 2^{l-2}\sum_{s,a,h} (\widehat{P}_{s,a,h}^{(l)}-P_{s,a,h} )x_{h+1}(w_{s,a,h}) \nonumber
 \\ & \leq 2\sqrt{2}\cdot \sqrt{ 2^{l-2}\sum_{s,a,h} \mathbb{V}(P_{s,a,h},x_{h+1}(w_{s,a,h}))   \bigg(2SAH\log W+\log\frac{1}{\delta'}\bigg)  } + 6H\bigg(SAH\log K+\log\frac{1}{\delta'}\bigg) \label{eq:xx2}
\end{align}
%
holds for any $\{w_{s,a,h}\}_{(s,a,h)}$ such that $1\leq w_{s,a,h}\leq W,\forall (s,a,h)$. 



For $l=1$, we have that $\sum_{s,a,h} (\widehat{P}_{s,a,h}^{(l)}-P_{s,a,h} )x_{h+1}(w_{s,a,h}) \leq SAH^2$ trivially.



Now we rewrite 
%
\begin{align}
 & \sum_{k=1}^K \sum_{h=1}^H \left( \widehat{P}^{(J^k_{s_h^k,a_h^k,h})}_{s_h^k,a_h^k,h}-P_{s_h^k,a_h^k,h} \right) X_{h+1}^k  \nonumber
 \\ & = \sum_{l=0}^{\log_2(K)} \sum_{s,a,h} (\widehat{P}_{s,a,h}^{(l)}-P_{s,a,h}) \sum_{k=1}^K \mathbb{I}[(s_h^k,a_h^k)=(s,a), I^k_{s,a,h}=l] X_{h+1}^k\nonumber
 \\ & \leq   \sum_{l=1}^{\log_2(K)} \sum_{o = 1}^{2^{l-1}}  \sum_{s,a,h} (\widehat{P}_{s,a,h}^{(l)}-P_{s,a,h}) \sum_{k=1}^K \mathbb{I}[(s_h^k,a_h^k)=(s,a), I^k_{s,a,h}=l,\overline{N}_{s_h^k,a_h^k,h}^k = 2^{l-1}+o] X_{h+1}^k + SAH^2\nonumber
 \\ & = \sum_{l=1}^{\log_2(K)} \sum_{o = 1}^{2^{l-1}}  \sum_{s,a,h} (\widehat{P}_{s,a,h}^{(l)}-P_{s,a,h}) X_{h+1}^{k_{l,o,s,a,h}} + SAH^2,
	\label{eq:f2}
\end{align}
%
where $k_{l,o,s,a,h}$ denotes the index of the $(2^{l-1}+o)$-th sample of $(s,a,h)$ in the online learning process. Recall that $\overline{N}^{K+1}_h(s,a)$ is the total visit count of $(s,a,h)$ in  $K$ episodes. 
If $\overline{N}^{K+1}_{h}(s,a)< 2^{l-1}+o$, we set $k_{l,o,s,a,h}=\infty$ and $X_{h+1}^{\infty}=0$.
% Note that there exists $\mathcal{J}\in \mathcal{C}$ such that $\mathcal{I}\subset \mathcal{J}$. 
Fix $2\leq l \leq \log_2(K)+1$ and $1\leq o \leq 2^{l-1}$, we can find $\{w^{(l,o)}_{s,a,h}\}_{(s,a,h)}$ be such that $x_{h+1}(w^{(l,o)}_{s,a,h})=X_{h+1}^{k_{l,o,s,a,h}}$ for any proper $(s,a,h)$. 
Using \eqref{eq:xx2}  for $(l,o)$ such that  $2\leq l \leq \log_2(K)+1$ and  $1\leq o \leq 2^{l-1}$ , with probability exceeding $1-2K\cdot 10SAH^2K^2\delta'$,
\begin{align}
&  \sum_{l=1}^{\log_2(K)} \sum_{o = 1}^{2^{l-1}}  \sum_{s,a,h} (\widehat{P}_{s,a,h}^{(l)}-P_{s,a,h}) X_{h+1}^{k_{l,o,s,a,h}}  \nonumber
\\ & \leq \sum_{l=1}^{\log_2(K)} \sum_{o = 1}^{2^{l-1}}\frac{1}{2^{l-2}}\Bigg( \sqrt{16 \cdot 2^{l-2} \sum_{s,a,h}\mathbb{V}(P_{s,a,h},X_{h+1}^{k_{l,o,s,a,h}})\cdot \bigg(SAH\log W+\log\frac{1}{\delta'}\bigg) } \nonumber
\\ & \qquad \qquad \qquad \qquad \qquad \qquad \qquad \qquad \qquad \qquad \qquad \qquad \qquad + 6H\bigg(SAH\log W+\log\frac{1}{\delta'}\bigg) \Bigg) \nonumber
\\ & \leq \sum_{l=1}^{\log_2(K)}\sqrt{32 \cdot \sum_{s,a,h}\sum_{o=1}^{2^{l-1}} \mathbb{V}(P_{s,a,h},X^{k_{l,o,s,a,h}}_{h+1}) \cdot \bigg(SAH\log W+\log\frac{1}{\delta'}\bigg) } \nonumber
\\ & \qquad\qquad \qquad \qquad \qquad \qquad \qquad \qquad \qquad \qquad \qquad \qquad  + \sum_{l=1}^{\log_2(K)} 12H\bigg(SAH\log W+\log\frac{1}{\delta'}\bigg)\nonumber
\\ & \leq \sqrt{64\log_2(K) \sum_{k=1}^K \sum_{h=1}^H \mathbb{V}(P_{s_h^k,a_h^k,h},X_{h+1}^k) \cdot (SAH\log(W)+\log(\frac{1}{\delta'})) } \nonumber
	\\ &\qquad\qquad \qquad \qquad \qquad \qquad \qquad \qquad \qquad \qquad \qquad \qquad + 12(\log_2K)H \bigg(SAH\log W+\log\frac{1}{\delta'}\bigg).\label{eq:f} 
\end{align}
In the last inequality, we have applied Cauchy's inequality and the fact that 
\begin{align}
& \sum_{k=1}^K\sum_{h=1}^H \mathbb{V}(P_{s_h^k,a_h^k,h},X_{h+1}^k) \nonumber
\\ & =
\sum_{l=1}^{\log_2(K)}\sum_{s,a,h}\sum_{o=1}^{2^{l-1}}\mathbb{V}(P_{s,a,h},X^{k_{l,o,s,a,h}}_{h+1})
 + \sum_{s,a,h}\sum_{k=1}^K \mathbb{I}\Big[(s_h^k,a_h^k)=(s,a), \overline{N}^k_{h}(s_h^k,a_h^k) = 0\Big]\mathbb{V}(P_{s,a,h},X_{h+1}^k)  .\nonumber
\end{align}


Using \eqref{eq:f2} and \eqref{eq:f}, and replacing $\delta$ with $\delta/(20SAH^2K^3)$, we finish the proof.


\subsection{Proof of Lemma~\ref{lemma:key2}}\label{app:pfkey2}
Let $M = \log_2(K)$ and $N = SAH$. 
Let $\check{\mathcal{C}}(l):= \{ \mathcal{J} = \{J^1,J^2,\ldots,J^l\}| J^{\tau}<J^{\tau+1},\forall 1\leq \tau \leq l-1  , J^{\tau}\in \{0\cup [M]\}^N, \forall \tau
 \}$ and $\check{\mathcal{C}} = \cup_{l\geq 1}\check{\mathcal{C}}(l)$. In words, $\check{\mathcal{C}}(l)$ is the set of \emph{strict} increasing path in $\{0\cup [M]\}^N$ with length $l$ and $\check{\mathcal{C}}$ is the set of all \emph{strict} increasing path.

We define $\mathrm{Proj}:\mathcal{C}\to \check{\mathcal{C}}$ by mapping $\mathcal{J}\in \mathcal{C}$ to $\check{\mathcal{J}}\in \check{\mathcal{C}}$, where $\check{\mathcal{J}}$ is the set of all different elements in $\mathcal{J}$. Let $\mathcal{F}(\check{J}):=\{ \mathcal{J}\in \mathcal{C}| \mathrm{Proj}(\mathcal{J})=\check{\mathcal{J}}\}$  for each $\check{\mathcal{J}}\in \check{\mathcal{C}}$. Because $\check{\mathcal{J}}$ is a \emph{strict} increasing path, there are at most $MN+1$ elements in $\check{\mathcal{J}}$. As a result, the size of $\mathcal{F}(\mathcal{J})$ is at most the solution of the equation below
\begin{align}
\sum_{i=1}^{MN+1} x_i = K, x_i\in \mathbb{N}, \forall 1\leq i \leq MN+1\nonumber
\end{align}
which is $\left( \begin{array}{c} K+MN \\ MN
\end{array}\right) = \frac{(K+MN)!}{(MN)! K!}\leq (K+MN)^{MN}\leq (2K)^{MN}$. It then holds that $|\mathcal{C}|\leq |\check{\mathcal{C}}|\cdot (2K)^{MN}$.

We further consider the set $\check{\mathcal{C}}(MN+1)$. For $\check{\mathcal{J}} = \{J^1,J^2,\ldots, J^{MN+1}\}\in \check{\mathcal{C}}(MN+1)$, with pigeonhole principle, we have that $J^1 = [0,0,\ldots,0]^{\top}$ and $J^{MN+1}=[M,M,\ldots, M]^{\top}$. Moreover, for each $1\leq \tau \leq MN$, $J^{\tau}$ and $J^{\tau+1}$ differ only at one dimension with distance $1$. In this way, we can view $\check{\mathcal{J}}$ as an $MN$-step increasing path from  $ [0,0,\ldots,0]^{\top}$ to $[M,M,\ldots, M]^{\top}$. In each step, we have at most $N$ directions. As a result, there are at most $N^{MN}$ such paths, which implies that $|\check{\mathcal{C}}(MN+1)|\leq N^{MN}$. Finally, noting that for any $\check{\mathcal{J}}\in \check{\mathcal{C}}$, there exists some $\check{\mathcal{J}}'\in \check{\mathcal{C}}(MN+1)$ such that $\check{\mathcal{J}}\subset \check{\mathcal{J}}'$, we conclude that $|\mathcal{C}|\leq (2K)^{MN}|\check{\mathcal{C}}|\leq (2K)^{MN} 2^{MN+1}|\check{\mathcal{C}}(MN+1)|\leq (4KN)^{MN+1}$. 


\subsection{Proof of Lemma~\ref{lemma:key3}}\label{app:pfkey3}


Without loss of generality, we write $\widehat{P}^{(I^k)}=\widehat{P}^k = \{\widehat{P}^k_{s,a,h}\}_{(s,a,h)} = \{\widehat{P}_{s,a,h}^{(I^k_{s,a,h})}\}_{(s,a,h)}$. Then we can regard $\mathcal{X}_{h+1}$ as a function of $\{\widehat{P}^{(I^k)}\}_{k=1}^K$, i.e., 
$\mathcal{X}_{h+1}=  \mathcal{X}_{h+1}^k(\{\widehat{P}^{(I^k)}\}_{k=1}^K, \{I^k\}_{k=1}^K  )$.
Let $\widetilde{\mathcal{E}}$ be the event where there exists $X^k_{h+1} \in \mathcal{X}_{h+1}(\{\widehat{P}^{(I^k)}\}_{k=1}^K, \{I^k\}_{k=1}^K),\forall (h,k)\in [H]\times [K]$ such that \eqref{eq:star} 
does not hold. So it suffices to prove that $\mathrm{Pr}(\widetilde{\mathcal{E}})\leq \delta$.
Fix $\mathcal{J} =\{J^k\}_{k=1}^K\in \mathcal{C}$.
We consider the event $\mathcal{E}(\mathcal{J})$ where there exists a sequence $X^k_{h+1} \in \mathcal{X}_{h+1}(\{\widehat{P}^{(J^k)}\}_{k=1}^K, \{J^k\}_{k=1}^K),\forall (h,k)\in [H]\times [K]$ such that
\begin{align}
& \sum_{k=1}^K \sum_{h=1}^H (\widehat{P}^{(J^k_{s,a,h})}_{s_h^k,a_h^k,h}-P_{s_h^k,a_h^k,h}) X^k_{h+1} \nonumber
\\ & \leq  \sqrt{L\sum_{k=1}^K \sum_{h=1}^H \mathbb{V}(P_{s_h^k,a_h^k,h},X^k_{h+1})\left(SAH\log(W) + \log(|\mathcal{C}|)+\log(1/\delta)\right)  }\nonumber
\\ & \qquad\qquad \qquad \qquad \qquad \qquad \qquad \qquad \qquad + LH(SAH\log(W) + \log(|\mathcal{C}|)+\log(1/\delta))\label{eq:star2},
\end{align}
does not hold, where $L= 200 (\log_2(K)+1)^2$. 
 Let $\widetilde{\mathcal{E}}(\mathcal{J})$ be the event where there exists a sequence $X^k_{h+1} \in \mathcal{X}_{h+1}(\{\widehat{P}^{(J^k)}\}_{k=1}^K),\forall (h,k)\in [H]\times [K]$ such that \eqref{eq:star} 
does not holds and $\mathcal{I}=\mathcal{J}$. Because $|\mathcal{C}|\leq (4SAHK))^{SAH\log_2(K)+1}$, we have $\widetilde{\mathcal{E}}(\mathcal{J})\subset \mathcal{E}(\mathcal{J})$.
With Lemma~\ref{lemma:key1}, we learn that $\mathrm{Pr}(\widetilde{\mathcal{E}}(\mathcal{J}))\leq \mathrm{Pr}(\mathcal{E}(\mathcal{J})) \leq \frac{\delta}{|\mathcal{C}|}$. As a result, we have  $\mathrm{Pr}(\cup_{\mathcal{J}\in \mathcal{C}}\widetilde{\mathcal{E}}(\mathcal{J}))\leq \delta$. By noting that $\widetilde{\mathcal{E}}\subset \cup_{\mathcal{J}\in \mathcal{C}}\widetilde{\mathcal{E}}(\mathcal{J})$, we obtain that $\mathrm{Pr}(\widetilde{\mathcal{E}})\leq \delta$. The proof is completed.


\section{Regret analysis (proof of Theorem~\ref{thm1})}\label{app:thmmain}


This section is devoted to the proof of Theorem~\ref{thm1}. Below we assume that $K\geq BSAH$, where 
 $B=4000\log^3_2(K)\log(3SAH)\log(\frac{1}{\delta})$.   
Let $\pi^k$ be the policy in the $k$-th episode.  Let $\bar{N}_{h}^k(s,a)$ be the count 
 of $(s,a,h)$ before the $k$-th episode and $N_h^k(s,a)$ be the count of the doubling batch used to compute the value function in the $k$-th episode. In particular, when $\bar{N}_h^k(s,a)=0$, we define $N_h^k(s,a)=1$.  
Let $V_h^k $ and $ Q_h^k$ be respectively the value of $ V_h$ and $Q_h$ before the $k$-th episode for all proper $(s,a,k,h)$. Recall that $\hat{P}^k_{s,a,h}$ is the value of $\hat{P}_{s,a,h}$ before the $k$-th episode. Let $\hat{r}_h^k(s,a)$ be the empirical reward function before the $k$-th episode of $(s,a)$. Let $\hat{\sigma}_h^k(s,a)$ be the empirical variance before the $k$-th episode for the state-action pair $(s,a)$, i.e., the value of $\hat{\sigma}_h(s,a)$ before the $k$-th episode.

\subsection{Optimism}
\begin{lemma}\label{lemma:opt}
With probability $1-4SAHK\delta$, $Q_h^k(s,a)\geq Q_h^*(s,a)$  and $V^k_h(s)\geq V^*_h(s)$ for any proper $(s,a,h,k)$. 
 \end{lemma}
 \begin{proof}
 For $p\in \Delta^{S}$, $v\in \mathbb{R}^S, \|v\|_{\infty}\leq H $ and $n\in N^+$,
 let 
 $$f(p,v,n)=pv + \max\left\{\frac{20}{3}\sqrt{\frac{\mathbb{V}(p,v)\log(\frac{1}{\delta})} {n}} ,\frac{400}{9}\cdot \frac{H\log(\frac{1}{\delta})}{n} \right\}.$$ Direct computation shows that $f(p,v,n)$ is non-decreasing in each dimension of $v$ as below.

Despite two possible points such that $\frac{20}{3}\sqrt{\frac{\mathbb{V}(p,v)\log(\frac{1}{\delta})} {n}}  =\frac{400}{9}\cdot \frac{H\log(\frac{1}{\delta})}{n} $,
 \begin{align}
	\frac{\partial f}{\partial v(s)} &=p(s)+\frac{20}{3}\mathbb{I}\left[  \frac{20}{3} \sqrt{  \frac{ \mathbb{V}(p,v)  \log(\frac{1}{\delta})} {n}  }\geq  \frac{400}{9}\frac{H\log(\frac{1}{\delta})}{n} \right] \frac{ p(s)(v(s)- pv )\log(\frac{1}{\delta})}{\sqrt{n\mathbb{V}(p,v) \log(\frac{1}{\delta})}}\nonumber
	\\ & \geq \min\{ p(s)+ p(s)\frac{(v(s)-pv)}{H},p(s) \}\nonumber
	\\ & = 0.
\end{align}

Fix $h,k$. In the case $N_{h}^k(s,a) \leq 2$, $Q_h^k(s,a)=H-h+1 \geq Q_h^*(s,a)$ and $_h^k(s)=H-h+1\geq V_h^*(s)$ for any proper $(s,a,h)$. 
Assume $N_{h}^k(s,a)>2$ and $Q_{h+1}^k\geq Q_{h+1}^*$. It then follows that $V_{h+1}^k \geq V_{h+1}^*$. According to the update rule in \eqref{eq:update1}, we either have that $Q_h^k(s,a)=H-h+1$ or
\begin{align}
&Q_h^k(s,a)  \nonumber\\ & = \hat{r}_h^k(s,a)+\hat{P}^k_{s,a,h}V_{h+1}^k + c_1\sqrt{\frac{\mathbb{V}(\hat{P}^k_{s,a,h},V_{h+1}^k)\log(\frac{1}{\delta})}{N_{h}^k(s,a)}} +c_2\sqrt{\frac{\left( \hat{\sigma}_h^k(s,a) -  (\hat{r}_h^k(s,a))^2 \right)\log(\frac{1}{\delta})}{N_{h}^k(s,a)}}+ c_3 \frac{H\log(\frac{1}{\delta})}{N_{h}^k(s,a)} \nonumber
\\ & \geq \hat{r}^k_{h}(s,a) +2\sqrt{2}\sqrt{\frac{\left( \hat{\sigma}_h^k(s,a) -  (\hat{r}_h^k(s,a))^2 \right)\log(\frac{1}{\delta})}{N_{h}^k(s,a)}} +\frac{28H\log(\frac{1}{\delta})}{3N_{h}^k(s,a)} + f(\hat{P}_{s,a,h}^k, V_{h+1}^k,N_{h}^k(s,a)) \nonumber
\\ & \geq \hat{r}^k_{h}(s,a) +2\sqrt{2}\sqrt{\frac{\left( \hat{\sigma}_h^k(s,a) -  (\hat{r}_h^k(s,a))^2 \right)\log(\frac{1}{\delta})}{N_{h}^k(s,a)}} +\frac{28H\log(\frac{1}{\delta})}{3N_{h}^k(s,a)}  + f(\hat{P}_{s,a,h}^k , V_{h+1}^*, N_{h}^k(s,a)) .
 \end{align}
for any $(s,a)$.



By Lemma \ref{empirical bernstein}, and recalling the definition of $\sigma_h^k(s,a)$, we have that,
\begin{align}
	& \mathbb{P}\left[|(\hat{P}^k_{s,a,h}-P_{s,a,h})V_{h+1}^*| >   2\sqrt{\frac{ \mathbb{ V}(\hat{P}^k_{s,a,h}, V_{h+1}^*  )\iota }{ N^k_h(s,a)}} +\frac{14\iota}{3n^k(s,a)} \right] \nonumber
	\\ & \leq  \mathbb{P}\left[|(\hat{P}^k_{s,a,h}-P_{s,a,h})V_{h+1}^*| >   \sqrt{\frac{2 \mathbb{ V}(\hat{P}^k_{s,a,h}, V_{h+1}^*  )\iota }{ N^k_h(s,a)-1}} +\frac{7\iota}{3N^k_h(s,a)-1} \right] \nonumber
	\\ & \leq 2\delta \label{eq_lemma1_ref.5}
\end{align}
and
\begin{align}
	& \mathbb{P}\left[ |\hat{r}^k_h(s,a)-r(s,a)| > 2\sqrt{ \frac{\left( \hat{\sigma}_h^k(s,a) -  (\hat{r}_h^k(s,a))^2 \right)\iota }{N_h^k(s,a)}} +\frac{28H\iota}{3N_h^k(s,a)}  \right]\nonumber
%	\\ & \leq \mathbb{P}\left[ |\hat{r}^k_h(s,a)-r_h(s,a)|  >2\sqrt{\frac{  \hat{\mathrm{Var}}_h^k(s,a) \iota}{N_h^k(s,a)-1}} +\frac{14H\iota}{3(N_h^k(s,a)-1)}     \right]\nonumber
	\\ & \leq 2\delta,\label{eq_lemma1_ref1}
\end{align}
%where $\hat{\mathrm{ Var}}_h^k(s,a) \leq \hat{r}_h^k(s,a)$\footnote{ $\mathbb{E}\left[(Z-\mathbb{E}[Z])^2 \right]\leq  H\mathbb{E}\left[Z\right]$ for $Z\in [0,H]$.} is the empirical variance of $R_h(s,a)$ computed by the $N_h^k(s,a)$ samples.

Note that \eqref{eq_lemma1_ref.5} implies that $f(\hat{P}^k_{s,a,h},V_{h+1}^*,N_h^k(s,a))\geq P_{s,a,h}V_{h+1}^*$.

Therefore, with probability $1-4\delta$, $Q_h^k(s,a)\geq r_h(s,a)+P_{s,a,h}V_{h+1}^*= Q_h^*(s,a)$. By induction, we learn that with probability $1-4SAHK\delta$, $Q_h^k(s,a)\geq Q_h^*(s,a)$ for any proper $(s,a,h,k)$. It then follows $V_h^k = \max_a Q_h^k(s,a)\geq \max_a Q_h^*(s,a)\geq V_h^*(s)$. The proof is completed.
%Via a union bound over all $(s,a)$ and $i$, we obtain that $\mathbb{P}\left[\mathcal{E}_1\cap \mathcal{E}_2\right]\geq 1 -2SA(\log_{2}KH+1)\delta$. The proof is completed.

%On the other hand, using empirical Bernstein's inequality,  we can verify that $f(\hat{P}_{s,a,h}^k, V_{h+1}^*, N_{s,a,h}^k)\geq Q_{h}^*(s,a)$ with probability $1-2\delta$. With a union bound on the failure probability, we obtain that with probability $1-2SAHK\delta$, $Q_h^k(s,a)\geq Q_h^*(s,a)$  and $V^k_h(s)\geq V^*_h(s)$ for any proper $(s,a,h,k)$. 
 \end{proof}

\subsection{Regret decomposition}
%\simon{let's add full details in the final version because I think this paper gonna be the standard reference for tabular MDP.}
Recall the definition that $\pi^k_h(s) = \arg\max_a Q_h^k(s,a)$. With probability $1-2SAHK\delta$, 
\begin{align}
\mathrm{Regret}(K):  & = \sum_{k=1}^K (V^*_1(s_1^k) - V^{\pi^k}_1(s_1^k)) \nonumber
\\ &  \leq  \sum_{k=1}^K \left( V^{k}_1(s_1^k) - V^{\pi^k}_1(s_1^k) \right) \nonumber
\\ &  \leq  \sum_{k=1}^K \sum_{h=1}^H \left(    (\hat{P}^k_{s_h^k,a_h^k,h}   - P_{s_h^k,a_h^k,h})V_{h+1}^k + b_h^k(s_h^k,a_h^k) \right) + \sum_{k=1}^K \sum_{h=1}^H ( P_{s_h^k,a_h^k,h}-\mathbf{1}_{s_{h+1}^k})V_{h+1}^k\nonumber
\\ & \qquad \qquad \qquad \qquad \qquad \qquad \qquad \qquad  + \sum_{k=1}^K \left(\sum_{h=1}^H \hat{r}^k_h(s^k_h,a^k_h)- V_1^{\pi^k}(s_1^k) \right),\label{eq:decomposition}
\end{align}
where $b_h^k(s_h^k,a_h^k): = c_1\sqrt{\frac{\mathbb{V}(  \hat{P}^k_{s_h^k,a_h^k,h},V_{h+1}^k)\log(\frac{1}{\delta})}{N^k_{h}(s_h^k,a_h^k) }} +c_2 \sqrt{\frac{\left( \hat{\sigma}_h^k(s,a)- (\hat{r}_h^k(s,a))^2 \right)\log(\frac{1}{\delta})}{N_h^k(s_h^k,a_h^k)}}+ c_3\frac{H\log(\frac{1}{\delta})}{N^k_{h}(s_h^k,a_h^k)}$. Here the first inequality is by Lemma~\ref{lemma:opt}, and the second inequality is by the Lemma below:

\begin{lemma}\label{lemma:decomdetail}
For each 
$k\in [K]$, 
$$V_1^k(s_1^k) \leq \sum_{h=1}^{H} \left( (\hat{P}^k_{s_h^k,a_h^k,h} - P_{s_h^k,a_h^k,h})V_{h+1}^k + b_h^k(s_h^k,a_h^k) + r_h(s_h^k,a_h^k) + (P_{s_h^k,a_h^k,h}-\textbf{1}_{s_{h+1}^k})V_{h+1}^k \right) .$$
\end{lemma}

\begin{proof}
By definition, for each $h\in [H]$
\begin{align}
V_h^k(s_h^k) &  \leq r_h(s_h^k,a_h^k) + \hat{P}^k_{s_h^k,a_h^k,h}V_{h+1}^k + b_h^k(s_h^k,a_h^k) \nonumber
\\ &  =  (\hat{P}^k_{s_h^k,a_h^k,h} - P_{s_h^k,a_h^k,h})V_{h+1}^k + b_h^k(s_h^k,a_h^k) + r_h(s_h^k,a_h^k) + (P_{s_h^k,a_h^k,h}-\textbf{1}_{s_{h+1}^k})V_{h+1}^k + V_{h+1}^k(s_{h+1}^k)\nonumber
\end{align}

Taking sum over $h\in [H]$, we have that 

\begin{align}
 & V_1^k(s_1^k) \nonumber
 \\ & \leq  \sum_{h=1}^{H} \left( (\hat{P}^k_{s_h^k,a_h^k,h} - P_{s_h^k,a_h^k,h})V_{h+1}^k + b_h^k(s_h^k,a_h^k) + r_h(s_h^k,a_h^k) + (P_{s_h^k,a_h^k,h}-\textbf{1}_{s_{h+1}^k})V_{h+1}^k \right) + V_{H+1}^k(s_{H+1}^k).
\end{align}
The proof is completed because $V_{H+1}^k=\textbf{0}$.
\end{proof}

Define $T_1 =\sum_{k=1}^K \sum_{h=1}^H \left(    (\hat{P}^k_{s_h^k,a_h^k,h}   - P_{s_h^k,a_h^k,h})V_{h+1}^k \right) $, $T_2 = \sum_{k=1}^K \sum_{h=1}^H  b_h^k(s_h^k,a_h^k)$, $T_3 =\sum_{k=1}^K\sum_{h=1}^H ( P_{s_h^k,a_h^k,h}-\mathbf{1}_{s_{h+1}^k})V_{h+1}^k $ and $T_4 = \sum_{k=1}^K \left(\sum_{h=1}^H \hat{r}^k_{h}(s^k_h,a^k_h)- V_1^{\pi^k}(s_1^k) \right) $.


We can easily bound $T_2,T_3$ and $T_4$ as in the following.

\paragraph{Bound of $T_2$} By definition, we can write
\begin{align}
T_2  & =\sum_{k=1}^K \sum_{h=1}^Hb_h^k(s_h^k,a_h^k) \nonumber
\\ & = \frac{460}{9}\sum_{k=1}^K\sum_{h=1}^H  \sqrt{\frac{\mathbb{V}(\hat{P}^k_{s_h^k,a_h^k,h},V_{h+1}^k)\log(\frac{1}{\delta})}{N_h^k(s_h^k,a_h^k)}}  +  2\sqrt{2} \sum_{k=1}^K \sum_{h=1}^H \sqrt{\frac{\left(\hat{\sigma}_h^k(s_h^k,a_h^k)- (\hat{r}_h^k(s_h^k,a_h^k))^2\right) \log(\frac{1}{\delta})}{N_h^k(s_h^k,a_h^k)}} \nonumber
\\ & \qquad\qquad \qquad \qquad \qquad \qquad \qquad \qquad \qquad \qquad \qquad \qquad+ \frac{544}{9}\sum_{k=1}^K \sum_{h=1}^H \frac{H\log(\frac{1}{\delta})}{N_h^k(s_h^k,a_h^k)}.
\end{align}


Using Cauchy's inequality and Lemma~\ref{lemma:doubling}, we obtain
\begin{align}
T_2  &  \leq \frac{460}{9}  \sqrt{2SAH \log_2(K)\log(\frac{1}{\delta})\sum_{k,h}\mathbb{V}(\hat{P}^k_{s_h^k,a_h^k,h},V_{h+1}^k )} \nonumber\\ & \qquad \qquad   + 4\sqrt{SAH\log_2(K)\log(\frac{1}{\delta})}\sqrt{\sum_{k,h}\left(\hat{\sigma}_h^k(s_h^k,a_h^k)- (\hat{r}_h^k(s_h^k,a_h^k))^2\right)}  + \frac{1088}{9}SAH^2\log_2(K)  \log(\frac{1}{\delta}) \nonumber
\\ & = \frac{460}{9}  \sqrt{2SAH \log_2(K)\log(\frac{1}{\delta}) T_5}\nonumber
\\& \qquad  + 4\sqrt{SAH\log_2(K)\log(\frac{1}{\delta})}\sqrt{\sum_{k,h}\left(\hat{\sigma}_h^k(s_h^k,a_h^k)- (\hat{r}_h^k(s_h^k,a_h^k))^2\right)} +\frac{1088}{9}SAH^2\log_2(K)  \log(\frac{1}{\delta})\nonumber
\\ & \leq \frac{460}{9}  \sqrt{2SAH \log_2(K)\log(\frac{1}{\delta}) T_5} \nonumber
\\ & \qquad \qquad \qquad \qquad + 4\sqrt{SAH^2\log_2(K)\log(\frac{1}{\delta})}\sqrt{\sum_{k,h}\hat{r}_h^k(s_h^k,a_h^k)} +\frac{1088}{9}SAH^2\log_2(K)  \log(\frac{1}{\delta}),\label{eq:boundt2o}
\end{align}
where we define $T_5 = \sum_{k=1}^K\sum_{h=1}^H\mathbb{V}(\hat{P}^k_{s_h^k,a_h^k,h},V_{h+1}^k )$, and the last inequality is by the fact that $\left(\hat{\sigma}_h^k(s_h^k,a_h^k)- (\hat{r}_h^k(s_h^k,a_h^k))^2\right)\leq H\hat{r}_h^k(s,a)$ for any proper $(s,a,h,k)$. 

We bound $\sum_{k,h}\hat{r}_h^k(s_h^k,a_h^k)$ by means of the lemma below.

\begin{lemma}\label{lemma:bdempr}
With probability $1-2SAHK\delta$, it holds that
\begin{align}
& \sum_{k=1}^K \sum_{h=1}^H \left| \hat{r}_h^k(s_h^k,a_h^k) - r_h(s_h^k,a_h^k)\right|\nonumber
\\ & \leq  4SAH^2 +4\sqrt{\sum_{k=1}^K\sum_{h=1}^H \frac{H\log(\frac{1}{\delta})}{N_h^k(s_h^k,a_h^k)}}\cdot \sqrt{\sum_{k=1}^K \sum_{h=1}^H r_h(s_h^k,a_h^k)}+24\sum_{k=1}^K\sum_{h=1}^H\frac{H\log(\frac{1}{\delta})}{N_h^k(s_h^k,a_h^k)}.\nonumber
\end{align}
\end{lemma}
\begin{proof}[Proof of Lemma~\ref{lemma:bdempr}]
In view of Lemma~\ref{empirical bernstein}, with probability $1-2SAHK\delta$ we have 
\begin{align}
\hat{r}_h^k(s,a) -r_h(s,a) & \leq 2\sqrt{2}\sqrt{\frac{\left(\hat{\sigma}_h^k(s_h^k,a_h^k)- (\hat{r}_h^k(s_h^k,a_h^k))^2\right)\log(\frac{1}{\delta})}{N_h^k(s,a)}} + \frac{28H\log(\frac{1}{\delta})}{3N_h^k(s,a)}
\\ & \leq  2\sqrt{2}\sqrt{\frac{H\hat{r}_h^k(s,a)\log(\frac{1}{\delta})}{N_h^k(s,a)}} + \frac{28H\log(\frac{1}{\delta})}{3N_h^k(s,a)}
\end{align}
for any proper $(s,a,h,k)$ be such that $N_h^k(s,a)>2$. Solve the inequality above, we can obtain 
\begin{align}
|\hat{r}_h^k(s,a) - r_h(s,a)|\leq  4\sqrt{\frac{Hr_h(s,a)\log(\frac{1}{\delta})}{N_h^k(s,a)}} + 24\frac{H\log(\frac{1}{\delta})}{N_h^k(s,a)}.\label{eq:bdt4_0}
\end{align}

It is then seen that
\begin{align}
   & \sum_{k=1}^K \sum_{h=1}^H \left| \hat{r}_h^k(s_h^k,a_h^k) - r_h(s_h^k,a_h^k)\right| \nonumber
   \\& \leq  4SAH^2 + \sum_{k=1}^K \sum_{h=1}^H\left( 4\sqrt{\frac{Hr_h(s_h^k,a_h^k)\log(\frac{1}{\delta})}{N_h^k(s_h^k,a_h^k)}}+24\frac{H\log(\frac{1}{\delta})}{N_h^k(s_h^k,a_h^k)} \right)
\nonumber
\\ & \leq 4SAH^2 +4\sqrt{\sum_{k=1}^K\sum_{h=1}^H \frac{H\log(\frac{1}{\delta})}{N_h^k(s_h^k,a_h^k)}}\cdot \sqrt{\sum_{k=1}^K \sum_{h=1}^H r_h(s_h^k,a_h^k)}+24\sum_{k=1}^K\sum_{h=1}^H\frac{H\log(\frac{1}{\delta})}{N_h^k(s_h^k,a_h^k)} .\nonumber
\end{align}
Here, the second inequality is due to Cauchy's inequality and the third inequality is a consequence of  Lemma~\ref{lemma:doubling}. We have thus completed the proof. 
\end{proof}

With Lemma~\ref{lemma:bdempr} and Lemma~\ref{lemma:doubling} in place, and noting that $\sum_{k=1}^K \sum_{h=1}^H r_h(s_h^k,a_h^k)\leq KH$, with probability $1-2SAHK\delta$,
\begin{align}
T_2\leq 61\sqrt{2SAH\log_2(K)\log(\frac{1}{\delta})T_5} +  
 145SAH^2\log_2(K)\log(\frac{1}{\delta})\label{eq:boundt2}
\end{align}

\paragraph{Bound of $T_3$} By virtue of Lemma~\ref{lemma:self-norm}, we see that with probability $1-10SAH^2K^2\delta$,
\begin{align}
T_3  \leq 2\sqrt{2}\cdot \sqrt{  T_6 \log(\frac{1}{\delta}) } + \log(\frac{1}{\delta}) + 2H\log(\frac{1}{\delta})\leq 2\sqrt{2}\cdot \sqrt{  T_6 \log(\frac{1}{\delta}) } + 3H\log(\frac{1}{\delta}),\label{eq:boundt3}
\end{align}
where $T_6 =\sum_{k=1}^K \sum_{h=1}^H\mathbb{V}(P_{s_h^k,a_h^k,h},V_{h+1}^k ) $. 


\paragraph{Bound of $T_4$} 
Note that
\begin{align}
T_4 = \sum_{k=1}^K\sum_{h=1}^H \left(\hat{r}_h^k(s_h^k,a_h^k) - r_h(s_h^k,a_h^k)  \right) + \sum_{k=1}^K \left( \sum_{h=1}^H r_h(s_h^k,a_h^k) - \mathbb{V}_1^{\pi^k}(s_1^k) \right).
\end{align}

For the first term, using Lemma~\ref{lemma:bdempr} and noting that $\sum_{k=1}^K\sum_{h=1}^H r_h(s_h^k,a_h^k)\leq KH$, with probability $1-2SAHK\delta$,
\begin{align}
 & \sum_{k=1}^K\sum_{h=1}^H (\hat{r}_h^k(s_h^k,a_h^k)-r_{h}(s_h^k,a_h^k))
\\ & \leq 4SAH^2 +4\sqrt{2\log_2(K)SAH^2}\cdot \sqrt{KH} + 24\log_2(K)SAH^2\log(\frac{1}{\delta})\nonumber
\\ & \leq  4\sqrt{2SAH^3K \log_2(K)\log(\frac{1}{\delta})}+28 SAH^2 \log_2(K)\log(\frac{1}{\delta}).\label{eq:bdt4_2}
\end{align}

Noting that $E_k:=\sum_{h=1}^H r_{h}(s_h^k,a_h^k)- V_1^{\pi^k}(s_1^k)$ is a zero-mean random variable bounded by $H$, by Lemma~\ref{lemma:self-norm}, with probability $1-2\delta$, it holds that
\begin{align}
\sum_{k=1}^K E_k \leq 2\sqrt{2}\cdot \sqrt{\sum_{k=1}^K \mathrm{Var}(E_k)\log(\frac{1}{\delta}) } + 3H^2\log(\frac{SAHK}{\delta}) \leq 2\sqrt{2KH^2\log(\frac{1}{\delta})} + 3H^2 \log(\frac{SAHK}{\delta}),\label{eq:bdt4_1}
\end{align}
where $\mathrm{Var}(E_k)$ denoting teh variance of $E_k$  conditioned on the $\mathcal{F}_{1}^{k}$.

Putting \eqref{eq:bdt4_2} and \eqref{eq:bdt4_1} together, we obtain that with probability $1-4SAHK\delta$, 
\begin{align}
T_4 \leq 6\sqrt{2SAH^3K\log_2(K)\log(\frac{1}{\delta})} +  31SAH^2\log_2(K)\log(\frac{SAHK}{\delta}).\label{eq:bdt_4f}
\end{align}

\paragraph{Bound of $T_5$ and $T_6$} Direct computation gives that
\begin{align}
T_5 : & = \sum_{k=1}^K \sum_{h=1}^H \mathbb{V}(\hat{P}^k_{s_h^k,a_h^k,h},V_{h+1}^k) \nonumber
\\ & = \sum_{k=1}^K \sum_{h=1}^H  \left( (\hat{P}^k_{s_h^k,a_h^k,h} -P_{s_h^k,a_h^k,h})(V^k_{h+1})^2  + P_{s_h^k,a_h^k,h}(V^k_{h+1})^2   - (\hat{P}^k_{s_h^k,a_h^k,h} V_{h+1}^k)^2  \right) \nonumber
\\ & \leq \sum_{k=1}^K \sum_{h=1}^H   (\hat{P}^k_{s_h^k,a_h^k,h} -P_{s_h^k,a_h^k,h})(V^k_{h+1})^2   + \sum_{k=1}^K \sum_{h=1}^H (P_{s_h^k,a_h^k,h} - \textbf{1}_{s_{h+1}^k} )  (V_{h+1}^k)^2  \nonumber
\\ & \qquad \qquad\qquad \qquad \qquad \qquad   +2H \sum_{k=1}^K \sum_{h=1}^H \max\{   V_{h}^k(s_h^k) - \hat{P}^k_{s_h^k,a_h^k,h} V_{h+1}^k  , 0\} \nonumber
\\ & \leq \sum_{k=1}^K \sum_{h=1}^H   (\hat{P}^k_{s_h^k,a_h^k,h} -P_{s_h^k,a_h^k,h})(V^k_{h+1})^2   + \sum_{k=1}^K \sum_{h=1}^H (P_{s_h^k,a_h^k,h} - \textbf{1}_{s_{h+1}^k} )  (V_{h+1}^k)^2  \nonumber
\\ & \qquad \qquad\qquad \qquad \qquad \qquad   +2H \sum_{k=1}^K \sum_{h=1}^H b_h^k(s_h^k,a_h^k)+2H\sum_{k=1}^K \sum_{h=1}^H r_h(s_h^k,a_h^k).\label{eq:boundt5}
\end{align}

Let $T_7  =  \sum_{k=1}^K \sum_{h=1}^H   (\hat{P}^k_{s_h^k,a_h^k,h} -P_{s_h^k,a_h^k,h})(V^k_{h+1})^2  $ and $T_8 = \sum_{k=1}^K \sum_{h=1}^H (P_{s_h^k,a_h^k,h} - \textbf{1}_{s_{h+1}^k} )  (V_{h+1}^k)^2$.

Using Freedman's inequality and the fact that $\mathrm{Var}(X^2)\leq 4 H^2\mathrm{Var}(X)$ for any random variable $X$ with support on $[-H,H]$, we have that
\begin{align}
T_8 \leq 2\sqrt{2}\sqrt{ 4 H^2T_6\log(\frac{1}{\delta})  } + 3H^2\log(\frac{1}{\delta}) .\label{eq:boundt8}
\end{align}


In a similar way, we can bound $T_6$ as 
\begin{align}
T_6 : & =\sum_{k=1}^K \sum_{h=1}^H \mathbb{V}(P_{s_h^k,a_h^k,h} ,V_{h+1}^k ) \nonumber
\\ &  = \sum_{k=1}^K \sum_{h=1}^H \left( P_{s_h^k,a_h^k,h} - \textbf{1}_{s_{h+1}^k} \right) (V_{h+1}^k)^2  + \sum_{k=1}^K \sum_{h=1}^H (V_{h+1}^k)^2(s_{h+1}^k) - \sum_{k=1}^K \sum_{h=1}^H (P_{s_h^k,a_h^k,h}V_{h+1}^k)^2 \nonumber
\\ & \leq 2\sqrt{2}\sqrt{4H^2T_6 \log(\frac{1}{\delta})}+3H^2 \log(\frac{1}{\delta})  + 2H \sum_{k=1}^K \sum_{h=1}^H \max\{  V_{h}^k(s_h^k) - P_{s_h^k,a_h^k,h}V_{h+1}^k,0  \} \nonumber
\\ & \leq 2\sqrt{2}\sqrt{4H^2T_6 \log(\frac{1}{\delta})}+3H^2 \log(\frac{1}{\delta}) + 2H T_2 \nonumber
\\ & \qquad \qquad \qquad \qquad \qquad \qquad + 2H \sum_{k=1}^K \sum_{h=1}^H \max\{ (\hat{P}^k_{s_h^k,a_h^k,h} - P_{s_h^k,a_h^k,h})V_{h+1}^k,0 \}+2KH^2.\label{eq:boundt6}
\end{align}

Let $T_9 = \sum_{k=1}^K \sum_{h=1}^H \max\{ (\hat{P}^k_{s_h^k,a_h^k,h} - P_{s_h^k,a_h^k,h})V_{h+1}^k,0 \}$.




\iffalse

\subsection{Bound of The Error Term}
In this section, we will prove a key lemma to deal with the error terms $T_1,T_7$ and $T_9$. This lemma controls the error term $\sum_{k,h}(\hat{P}_{s_h^k,a_h^k,h}-P_{s_h^k,a_h^k,h}) X_{h}^k$ under some mild conditions, where we do not require $\{X_{h}^k\}_{h,k}$ is conditionally independent of $\{\hat{P}^k_h\}_{h,k}$. By this lemma, we can control the error terms above by letting $\mathcal{X}_h^k = \{ V_{h+1}^k, (V_{h+1}^k)^2/H, \textbf{0}\}$. % This consititutes our key novelty

%\simon{add intuitions what $\mathcal{X}_h^k$ will be like later, and we why need this lemma}


\begin{lemma}\label{lemma:key}
For each $k,h$,  let a set of $S$-dimensional vectors  $\mathcal{X}_h^k$  be a function of $\{\hat{P}^k_{s,a,h'}\}_{h'\geq h+1,s,a}$ such that $\|X\|_{\infty}\leq H, \forall X\in \mathcal{X}_h^k$. Let $L = \max_{h,k}|\mathcal{X}_h^k|$. Then with probability $1- 10\delta$, for any sequence $\{X_h^k\}$ such that  $X_h^k\in \mathcal{X}_h^k$, it holds that
\begin{align}
 &  \sum_{k=1}^K \sum_{h=1}^H \left( \hat{P}_{s_h^k,a_h^k,h}-P_{s_h^k,a_h^k,h} \right) X_h^k  \nonumber
 \\ & \leq  \sqrt{32\log_2(K) \sum_{k=1}^K \sum_{h=1}^H \mathbb{V}(P_{s_h^k,a_h^k,h},X_h^k) \cdot (4SAH\log_2(K)\log(SALH)+\log(\frac{1}{\delta})) } + 12\log_2(K)H (4SAH\log_2(K)\log(SALH)+\log(\frac{1}{\delta})).\label{eq:key}
\end{align}
\end{lemma}
\begin{proof}




 Recall the definition of $i^k_{s,a,h}$. Define a \emph{profile} as $\{i_{s,a,h}\}_{s,a,h}$ with $i_{s,a,h}\in [\log_2(K)]$. We say two profiles $i\leq j$ iff $i_{s,a,h}\leq j_{s,a,h}$ for any $(s,a,h)$.
 
 Let $\mathcal{I} =\left\{  \{i^k_{s,a,h}\}_{s,a,h}     | k = 1,2,\ldots,K  \right\}$.  Since there are at most $SAH(\log_2(K)+1)$ updates, the size of  $\mathcal{I}$ is at most $SAH(\log_2(K)+1)$.  

\simon{this paragraph might be the most important technical paragraph. We can add more details and a plot.}
Let $M=SAH(\log_2(K)+1)$ and $\mathcal{C}:= \{  \{j^1,j^2,\ldots,j^{M}\} |j_l\in [\log_2(K)]^{SAH}, j^l\neq j^{l'} \forall l\neq l', j^l\leq j^{l+1}, \forall 1\leq l\leq M-1\}$. In words, $\mathcal{C}$ is the set of an increasing path in the set $[\log_2(K)]^{SAH}$ from $[0,0,\ldots,0]^{\top}$ to $[\log_2(K),\log_2(K),\ldots,\log_2(K)]^{\top}$. Since there are at most $M$ steps and in each step we have at most $SAH$ choices, the size of $\mathcal{C}$ is at most $(SAH)^{M}$. Let $\bar{\mathcal{C}} = \{  \{j^1,j^2,j^3,\ldots,j^q\}| j^l<j^{l+1},1\leq l \leq q-1, q\leq M \}$. Then the size of $\bar{\mathcal{C}}$ is at most $2^{M}\cdot |\mathcal{C}|=(2SAH)^M$. 
\simon{without using the increasing path property, we will have size $(\log_2K)^{SAHM}$.}

Recalling the definition of $\mathcal{I}$, we always have that $\mathcal{I}\in \bar{\mathcal{C}}$.
%Because $i^1\leq i^2\leq \ldots,\leq i^K$, $\mathcal{I}$ is determined by an increasing path in $[\log_2(K)]^{SAH}$ (it is possible that $\mathcal{I}$ does not end with $[\log_2(K),\log_2(K),\ldots, \log_2(K)]^{\top}$). Then we can always find some $\mathcal{J}\in \mathcal{C}$ such that $\mathcal{I}\subset \mathcal{J}$.

% Now we fix $\mathcal{J}\in \bar{\mathcal{C}}$. Let $\delta'\in (0,1)$ be a confidence parameter and $\log(\frac{1}{\delta})' = \log(2/\delta')$. Assume $\mathcal{J} =\{j^1,j^2,\ldots, j^M\}$ such that $j^1\leq j^2\leq \cdots \leq j^M$.  For $j\in [\log_2(K)]^{SAH}$, we use $\hat{P}(j)$ to denote the tuple $\{ \hat{P}_{s,a,h}(j_{s,a,h})  \}_{s,a,h}$ where $\hat{P}_{s,a,h}^{(l)}$ is the empirical transition model computed using the $l$-th batch of $(s,a,h)$ for $1\leq l \leq \log_2(K)$, and $\hat{P}_{s,a,h}^{(l)} = \frac{1}{S}\cdot \textbf{1}$ if $l=0$.  We also define $\hat{P}_h(j)=\{ \hat{P}_{s,a,h'}(j_{s,a,h})  \}_{h'\geq h+1,s,a}$.%\simon{what is $j^k$?}


\end{proof}

As a result, with probability $1-\delta$ it holds that 
\begin{align}
 &  \sum_{k=1}^K \sum_{h=1}^H \left( \hat{P}_{s_h^k,a_h^k,h}-P_{s_h^k,a_h^k,h} \right) X_h^k  \nonumber
 \\ & \leq  \sqrt{32\log_2(K) \sum_{k=1}^K \sum_{h=1}^H \mathbb{V}(P_{s_h^k,a_h^k,h},X_h^k) \cdot (4SAH\log_2(K)\log(SALH)+\log(\frac{1}{\delta})) } + 12\log_2(K)H (4SAH\log_2(K)\log(SALH)+\log(\frac{1}{\delta})).\nonumber
\end{align}
The proof is completed.

\fi 

\subsection{Bounds of the error terms}

Now we  deal with $T_1,T_7$ and $T_9$. We first recall the definition.
\begin{align}
 & T_1 = \sum_{k=1}^K \sum_{h=1}^H  \left( \hat{P}^k_{s_h^k,a_h^k,h}  -P_{s_h^k,a_h^k,h} \right) V_{h+1}^k
;\nonumber
\\ & T_7  = \sum_{k=1}^K \sum_{h=1}^H  \left( \hat{P}^k_{s_h^k,a_h^k,h}  -P_{s_h^k,a_h^k,h} \right) (V_{h+1}^k)^2;\nonumber
\\ & T_9 =  \sum_{k=1}^K \sum_{h=1}^H \max\{ (\hat{P}^k_{s_h^k,a_h^k,h} - P_{s_h^k,a_h^k,h})V_{h+1}^k,0 \}.\nonumber
\end{align}

Recall $B=4000\log^2_3(K)\log(3SAH)\log(\frac{1}{\delta}) $. By the update rule \eqref{eq:updateq}, $V_{h+1}^k$ is determined by the $\{ \hat{P}^k_{s,a,h'} \}_{(h+1\leq h'\leq H ,s,a)}$ and $\{I^k_{s,a,h'}\}_{h+1\leq h'\leq H,s,a}$. 
Using Lemma~\ref{lemma:key3} with $\mathcal{X}_{h+1} =\{V_{h+1}^k\}_{k=1}^K \cup\{0\}, \{(V_{h+1}^k)^2 /H\}_{k=1}^K\cup\{0\}$ and $\{ V_{h+1}^k\}_{k=1}^K \cup\{0\}$, and noting that $\mathrm{Var}(X^2)\leq 4\|X\|^2_{\infty}\mathrm{Var}(X)$, we have that with probability $1-30 SAH^2K^2 \delta$,
\begin{align}
 & T_1\leq \sqrt{ B SAH \sum_{k=1}^K \sum_{h=1}^H \mathbb{V}(P_{s_h^k,a_h^k,h},V_{h+1}^k)}+BSAH^2 = \sqrt{BSAH T_6}+BSAH^2;\label{eq:boundt1}
 \\ & T_7 \leq H \sqrt{ 4BSAH  \sum_{k=1}^K \sum_{h=1}^H \mathbb{V}(P_{s_h^k,a_h^k,h}, V_{h+1}^k) }    + 4BSAH^3 = H\sqrt{4BSAH T_6}+4BSAH^3 ;\label{eq:boundt7}
 \\ & T_9 \leq \sqrt{B SAH \sum_{k=1}^K \sum_{h=1}^H \mathbb{V}(P_{s_h^k,a_h^k,h},V_{h+1}^k)}+BSAH^2= \sqrt{BSAH T_6}+BSAH^2.\label{eq:boundt9}
\end{align}

\subsection{Putting all pieces together}

Rewrite  \eqref{eq:boundt2},\eqref{eq:boundt3},\eqref{eq:bdt_4f},\eqref{eq:boundt5},\eqref{eq:boundt6},\eqref{eq:boundt8},\eqref{eq:boundt1},\eqref{eq:boundt7} and \eqref{eq:boundt9} as below. With probability $1-100SAH^2K\delta$,

\begin{align}
 & T_2\leq 61 \sqrt{B SAHT_5}+145BSAH^2;\label{eq:obt2}
 \\ & T_3 \leq \sqrt{8BT_6}+3HB  ;\label{eq:obt3}
 \\ & T_4 \leq 9\sqrt{ BSAH^3K}+31BSAH^2;\label{eq:obt4}
 \\ & T_5 \leq T_7 + T_8 + 2H T_2 + 2KH^2 ;\label{eq:obt5}
 \\ & T_6 \leq \sqrt{32B H^2T_6} + 2HT_2 + 2HT_9 + 3H^2B + 2KH^2;\label{eq:obt6}
 \\ & T_8 \leq \sqrt{32BH^2T_6 } + 3BH^2 ;\label{eq:obt8}
 \\ & T_1\leq \sqrt{BSAHT_6} + BSAH^2;\label{eq:obt1}
 \\ & T_7 \leq H\sqrt{4BSAHT_6} + 4BSAH^3;\label{eq:obt7}
 \\ & T_9 \leq \sqrt{BSAHT_6}+BSAH^2.\label{eq:obt9}
\end{align}

 To solve the inequalities \eqref{eq:obt2} to \eqref{eq:obt9}, we use the fact that $a\leq \sqrt{bc}+d$ implies that $a \leq \epsilon b + \frac{1}{2\epsilon}c +d$ for any $b,c\geq 0, a,d\in \mathbb{R}$ and any $\epsilon>0$. Using this arguments, we have
 \begin{align}
 & HT_2 \leq \epsilon T_5 + (\frac{1}{2\epsilon}+1)61BSAH^3+145BSAH^3;\nonumber
 \\ & T_6 \leq \epsilon T_6 + 2HT_2 + 2HT_9 + (3+\frac{32}{\epsilon})BH^2 + 2KH^2;\nonumber
 \\ & HT_9 \leq \epsilon T_6 +\left( \frac{1}{2\epsilon}+1\right)BSAH^3;\nonumber
 \\ & T_8 \leq \epsilon T_6 + \left( \frac{16}{\epsilon}+1\right)BH^2;\nonumber
 \\ & T_7 \leq  \epsilon T_6 + \left(\frac{2}{\epsilon} +4\right)BSAH^3.\nonumber
 \end{align}
Then we have that
\begin{align}
& T_5 \leq T_7+T_8 + 2HT_2 + 2KH^2\leq 2\epsilon T_5+2\epsilon T_6 +  \left( \frac{100}{\epsilon}+300\right) BSAH^3+2KH^2;\nonumber
\\ & T_6 \leq 3 \epsilon T_6 + 2\epsilon T_5 + \left( \frac{200}{\epsilon}+300\right) BSAH^3+2KH^2.
\end{align}

 Choosing $\epsilon= \frac{1}{20}$, we learn that $T_5+T_6\leq O(BSAH^3+KH^2)$. 
Recall that  $K\geq SAHB$. We then have that $T_5,T_6 = O(KH^2 + BSAH^3) = O(KH^2)$. As a result, we have that $T_1 = O(\sqrt{BSAH^3K})$,  $T_7, T_8 = O(\sqrt{BSAH^5K})$, $T_2 = O(\sqrt{BSAH^3K})$ and $T_3 = O(\sqrt{BKH^2})$. We then conclude that the total regret is bounded by $\tilde{O}(\sqrt{SAH^3K\log(\frac{1}{\delta})})$. The proof of Theorem~\ref{thm1} is completed by replacing $\delta$ with $\frac{\delta}{100SAH^2K}$.





\section{Proof of the value-based regret bound (proof of Theorem~\ref{thm:first})}\label{sec:appfirst}

Recall that 
%
\begin{equation}
	B=4000 (\log_2 K)^3 \log(3SAH)\log\frac{1}{\delta'} 
	\qquad \text{with }\delta' = \frac{\delta}{200SAH^2K^2}. 
	\label{eq:defn-B-first}
\end{equation}
%
Consider first the scenario where $K\leq \frac{BSAH^2}{v^{\star}}$: 
the regret bound can be upper bounded by 
%
\begin{align}
\mathbb{E}\big[\mathsf{Regret}(K)\big] & =\mathbb{E}\left[\sum_{k=1}^{K}\Big(V_{1}^{\star}(s_{1}^{k})-V_{1}^{\pi^{k}}(s_{1}^{k})\Big)\right]\leq\mathbb{E}\left[\sum_{k=1}^{K}V_{1}^{\star}(s_{1}^{k})\right]=K\mathbb{E}_{s_{1}\sim\mu}\big[V_{1}^{\star}(s_{1})\big]\notag\\
 & =Kv^{\star}=\min\Big\{\sqrt{BSAH^{2}Kv^{\star}}
	%+BSAH^{2}
	,Kv^{\star}\Big\}.
	\label{eq:E-regret-UB-easy-first}
\end{align}
%
As a result, 
the remainder of the proof is dedicated to the the case with
%
\begin{equation}
	K\geq \frac{BSAH^2}{v^{\star}} .
	\label{eq:K-focus-first}
\end{equation}
%
%throughout the rest of this section. 



%\begin{align}
%	\mathsf{Regret}(K) &= \sum_{k=1}^K \Big( V_1^{\star}(s_1^k) -  V_1^{\pi^k} (s_1^k) \Big) 
%	\leq \sum_{k=1}^K  V_1^{\star}(s_1^k) 
%	\leq 3K \mathbb{E}_{s_1\sim \mu}\big[V_1^{\star}(s_1)\big]  + H\log \frac{1}{\delta'} \notag\\ 
%	&= 3Kv^{\star} + H\log \frac{1}{\delta'} 
%	\leq  3\min\Big\{\sqrt{BSAH^{2}Kv^{\star}}+BSAH^2,Kv^{\star}\Big\}
%\end{align}
%%
%with probability at least $1-\delta'$, 
%where we have invoked Lemma~\ref{lemma:con}. \yxc{check}





To begin with, recall that the proof of Theorem~\ref{thm1} in Section~\ref{app:thmmain} consists of bounding the quantities $T_1,\ldots,T_9$ (see \eqref{eq:decomposition}, \eqref{eq:defn-T56-proof} and \eqref{eq:defn-T789-proof}) and recall that $\delta' = \frac{\delta}{200SAH^2K^2}$. 
%
%defined as follows:  
%%
%\begin{align}
%& T_1= \sum_{k=1}^K \sum_{h=1}^H \left( \widehat{P}^k_{s_h^k,a_h^k,h}-P_{s_h^k,a_h^k,h} \right)V_{h+1}^k ;\nonumber
%\\ &  T_2 = \sum_{k=1}^K \sum_{h=1}^H b_h^k(s_h^k,a_h^k);\nonumber
%\\ & T_3 =\sum_{k=1}^K \sum_{h=1}^H (P_{s_h^k,a_h^k,h}-\textbf{1}_{s_{h+1}^k})V_{h+1}^k;\nonumber
%\\ & T_4 = \sum_{k=1}^K \left( \sum_{h=1}^H \widehat{r}_h^k(s_h^k,a_h^k)-V_{1}^{\pi^k}(s_1^k)\right);\nonumber
%\\ & T_5  = \sum_{k=1}^K \sum_{h=1}^H \mathbb{V}(\widehat{P}^k_{s_h^k,a_h^k,h},V_{h+1}^k);\nonumber
%\\ & T_6 = \sum_{k=1}^K \sum_{h=1}^H \mathbb{V}(P_{s_h^k,a_h^k,h},V_{h+1}^k);\nonumber
%\\ & T_7  = \sum_{k=1}^K \sum_{h=1}^H \left( \widehat{P}^k_{s_h^k,a_h^k,h} - P_{s_h^k,a_h^k,h} \right) (V_{h+1}^k)^2;\nonumber
%\\ & T_8 =\sum_{k=1}^K \sum_{h=1}^H (P_{s_h^k,a_h^k,h}-\textbf{1}_{s_{h+1}^k})(V_{h+1}^k)^2;\nonumber
%\\ & T_9  = \sum_{k=1}^K\sum_{h=1}^H \max\left\{ (\widehat{P}^k_{s_h^k,a_h^k,h}-P_{s_h^k,a_h^k,h})V_{h+1}^k, 0  \right\}.\nonumber
%\end{align}
%%
%where we recall that $\delta' = \frac{\delta}{200SAH^2K^2}$.  
%
In order to establish Theorem~\ref{thm:first}, we need to develop tighter bounds on some of these quantities (i.e., $T_2$, $T_4$, $T_5$ and $T_6$)  to reflect their dependency on $v^{\star}$ (cf.~\eqref{eq:defn-vstar-formal}). 

%continue by proving tighter bounds for some of these terms with respect to $v^{\star}$.


%Also recall \eqref{eq:obt2}-\eqref{eq:obt9}. We will provide refined analysis for the bound of $T_2$, $T_4$, $T_5$ and $T_6$, and leave other bounds invariant.


\paragraph{Bounding $T_2$.}
%
Recall that we have shown in \eqref{eq:boundt2o} that
%
\begin{align}
 & T_{2}\leq\frac{460}{9}\sqrt{2SAH(\log_{2}K)\Big(\log\frac{1}{\delta'}\Big)T_{5}}\nonumber\\
 & \qquad+4\sqrt{SAH^{2}(\log_{2}K)\log\frac{1}{\delta'}}\sqrt{\sum_{k,h}\widehat{r}_{h}^{k}(s_{h}^{k},a_{h}^{k})}+\frac{1088}{9}SAH^{2}(\log_{2}K)\log\frac{1}{\delta'}.
\nonumber
\end{align}
%
In view of the definition of $T_4$ (cf.~\eqref{eq:decomposition}) as well as the fact that $\sum_{k=1}^K V_1^{\star}(s_1^k)\leq 3Kv^{\star} + H\log \frac{1}{\delta'}$ holds with probability at least $1-\delta'$ (see Lemma~\ref{lemma:con}),
we arrive at
%
\begin{align}
\sum_{k,h}\widehat{r}_{h}^{k}(s_{h}^{k},a_{h}^{k})\leq T_{4}+\sum_{k}V_{1}^{\pi_{k}}(s_{1}^{k})\leq T_{4}+\sum_{k}V_1^{\star}(s_{1}^{k})\leq T_{4}+3Kv^{\star}+H\log\frac{1}{\delta'},
	\label{eq:sum-rhat-vstar}
\end{align}
%
which in turn gives
%
 \begin{align}
T_{2} & \leq\frac{460}{9}\sqrt{2SAH(\log_{2}K)\Big(\log\frac{1}{\delta'}\Big)T_{5}}\nonumber\\
 & \qquad\qquad+4\sqrt{SAH^{2}(\log_{2}K)\log\frac{1}{\delta'}}\sqrt{T_{4}+3Kv^{\star}}+130SAH^{2}(\log_{2}K)\log\frac{1}{\delta'}.
\label{eq:nbft2}
 \end{align}


\paragraph{Bounding $T_4$.}
%
When it comes to the quantity $T_4$ (cf.~\eqref{eq:decomposition}), we make the observation that 
%
\begin{align}
 T_4 
	%& = \sum_{k=1}^K \left( \sum_{h=1}^H \widehat{r}_h^k(s_h^k,a_h^k)-V_{1}^{\pi^k}(s_1^k)\right) \nonumber
 %\\
	& 
	=\underset{\eqqcolon\,\widecheck{T}_{1}}{\underbrace{\sum_{k=1}^{K}\left(\sum_{h=1}^{H}\widehat{r}_{h}^{k}(s_{h}^{k},a_{h}^{k})-r_{h}(s_{h}^{k},a_{h}^{k})\right)}}+\underset{\eqqcolon\,\widecheck{T}_{2}}{\underbrace{\sum_{k=1}^{K}\left(\sum_{h=1}^{H}r_{h}(s_{h}^{k},a_{h}^{k})-V_{1}^{\pi^{k}}(s_{1}^{k})\right)}}.
	\label{eq:T4-decompose-T12-check}
 \end{align}
%
% Let $\widecheck{T}_1 =\sum_{k=1}^K \left( \sum_{h=1}^H \widehat{r}_h^k(s_h^k,a_h^k)-r_{h}(s_h^k,a_h^k)\right)  $ and $\widecheck{T}_2 =\sum_{k=1}^K \left( \sum_{h=1}^H r_h(s_h^k,a_h^k) - V_{1}^{\pi^k}(s_1^k) \right) $
%
Repeating the arguments for \eqref{eq:sum-rhat-vstar} yields
%
%
\begin{align}
	\sum_{k,h} r_{h}(s_{h}^{k},a_{h}^{k})\leq \widecheck{T}_{2}+\sum_{k}V_{1}^{\pi_{k}}(s_{1}^{k})\leq \widecheck{T}_{2}+\sum_{k}V_1^{\star}(s_{1}^{k})\leq \widecheck{T}_{2}+3Kv^{\star}+H\log\frac{1}{\delta'}
	\label{eq:sum-rnohat-vstar}
\end{align}
%
 with probability at least $1-\delta'$. 
Combining this with Lemma~\ref{lemma:bdempr},
%and Lemma~\ref{lemma:doubling}, 
we see that   
%
\begin{align}
	\widecheck{T}_1 & \leq 4\sqrt{2SAH^2\log_2 K \log \frac{1}{\delta'}} \sqrt{\sum_{k=1}^K\sum_{h=1}^H r_h(s_h^k,a_h^k)} + 52SAH^2(\log_2 K)\log \frac{1}{\delta'} \nonumber
	\\ & \leq 4\sqrt{2SAH^2\log_2 K \log \frac{1}{\delta'}} \sqrt{\widecheck{T}_2 + 3Kv^{\star}} + 60SAH^2(\log_2 K)\log \frac{1}{\delta'}\label{eq:ct1}
\end{align}
%
with probability exceeding $1-3SAHK\delta'$. 
In addition,  Lemma~\ref{lemma:self-norm} tells us that
%
\begin{align}
\widecheck{T}_2 & \leq 2\sqrt{2\sum_{k=1}^K \mathbb{E}_{\pi^k,s_1\sim \mu}\left[\left(\sum_{h=1}^H r_h(s_h,a_h) \right)^2 \right]\log \frac{1}{\delta'}}+3H^2\log \frac{1}{\delta'} \nonumber
\\ & \leq 2\sqrt{2H\sum_{k=1}^K \mathbb{E}_{\pi^k,s_1\sim \mu}\left[\sum_{h=1}^H r_h(s_h,a_h)  \right]\log \frac{1}{\delta'}}+3H\log \frac{1}{\delta'} \nonumber
\\ & \leq 2\sqrt{2KHv^{\star}\log \frac{1}{\delta'}}+3H\log \frac{1}{\delta'} \label{eq:ct1.5}
\\ & \leq 2Kv^{\star} + 5H\log \frac{1}{\delta'}\label{eq:ct2}
\end{align}
%
with probability at least $1-2SAHK\delta'$, 
where the expectation operator $\mathbb{E}_{\pi^k,s_1\sim \mu}[\cdot]$ is taken over the randomness of a trajectory $\{(s_h,a_h)\}$ 
generated under policy $\pi^k$ and initial state $s_1\sim \mu$, 
the last line arises from the AM-GM inequality, 
and the penultimate line makes use of Assumption~\ref{assum1} and the fact that 
%
\[
\mathbb{E}_{\pi^{k},s_{1}\sim\mu}\left[\sum_{h=1}^{H}r_{h}(s_{h},a_{h})\right]=\mathbb{E}_{s_{1}\sim\mu}\left[V_{1}^{\pi^{k}}(s_{1})\right]\leq\mathbb{E}_{s_{1}\sim\mu}\left[V_{1}^{\star}(s_{1})\right]=v^{\star}.
\]
%
Taking \eqref{eq:ct1}, \eqref{eq:ct1.5} and \eqref{eq:ct2} together, we can demonstrate that with probability exceeding $1-5SAHK\delta'$,
%
\begin{subequations}
\begin{align}
	& \widecheck{T}_1\leq     13\sqrt{SAH^2Kv^{\star}(\log_2 K)\log \frac{1}{\delta'}}  + 80SAH^2(\log_2 K)\log\frac{1}{\delta'},\label{eq:check-T-bound-first}
\\ & \widecheck{T}_2 \leq  2\sqrt{2KHv^{\star}\log \frac{1}{\delta'}}+3H\log\frac{1}{\delta'} .
\end{align}
\end{subequations}
%
Substitution into \eqref{eq:T4-decompose-T12-check} reveals that: with probability exceeding $1-5SAHK\delta'$, 
%
\begin{align}
	T_4 \leq 15\sqrt{SAH^2Kv^{\star}(\log_2K)\log \frac{1}{\delta'}}  + 83SAH^2(\log_2K)\log\frac{1}{\delta'}.\label{eq:fnbt4}
\end{align}


\paragraph{Bounding $T_5$.}
%
Recall that we have proven in \eqref{eq:boundt5-intermediate-13} that
%
%
\begin{align}
	T_{5} & \leq T_{7}+T_{8}+2HT_2+2H\sum_{k=1}^{K}\sum_{h=1}^{H}\widehat{r}^k_{h}(s_{h}^{k},a_{h}^{k}).\label{eq:T5-UB-first-123}
\end{align}
%
%\begin{align}
%T_5  & \leq \sum_{k=1}^K \sum_{h=1}^H   (\widehat{P}^k_{s_h^k,a_h^k,h} -P_{s_h^k,a_h^k,h})(V^k_{h+1})^2   + \sum_{k=1}^K \sum_{h=1}^H (P_{s_h^k,a_h^k,h} - \textbf{1}_{s_{h+1}^k} )  (V_{h+1}^k)^2  \nonumber
%\\ & \qquad \qquad\qquad \qquad \qquad \qquad   +2H \sum_{k=1}^K \sum_{h=1}^H b_h^k(s_h^k,a_h^k)+2H\sum_{k=1}^K \sum_{h=1}^H r_h(s_h^k,a_h^k).\nonumber
%\end{align}
%
With \eqref{eq:sum-rnohat-vstar} and \eqref{eq:ct2} in place, we can deduce that, with probability at least $1-3SAHK\delta'$, 
%
\begin{align}
\sum_{k,h}r_{h}(s_{h}^{k},a_{h}^{k}) & \leq\widecheck{T}_{2}+3Kv^{\star}+H\log\frac{1}{\delta'}\leq5Kv^{\star}+6H\log\frac{1}{\delta'}. 
	\label{eq:sum-r-UB-first}
\end{align}
%
Moreover, under the assumption~\eqref{eq:K-focus-first}, we can further bound \eqref{eq:check-T-bound-first} as
%
\[
	\widecheck{T}_{1}\leq\sqrt{BSAH^{2}Kv^{\star}}+BSAH^{2}\leq2Kv^{\star}
\]
%
with probability exceeding $1-3SAHK\delta'$, which combined with \eqref{eq:sum-r-UB-first} and the assumption~\eqref{eq:K-focus-first} results in
%
\begin{align}
\sum_{k,h}\widehat{r}_{h}^{k}(s_{h}^{k},a_{h}^{k})=\sum_{k,h}r_{h}(s_{h}^{k},a_{h}^{k})+\widecheck{T}_{1} & \leq7Kv^{\star}+6H\log\frac{1}{\delta'}\leq8Kv^{\star}.
	\label{eq:sum-hat-r-UB-first}
\end{align}
%
Substitution into \eqref{eq:T5-UB-first-123} indicates that: with probability exceeding $1-6SAHK\delta'$, 
%
\begin{align}
T_5\leq  T_7 + T_8+2HT_2 + 16HKv^{\star} .\label{eq:fnbt5}
%+ 10H^2\log \frac{1}{\delta'}
\end{align}
%





\paragraph{Bounding $T_6$.}
%
Making use of our bounds \eqref{eq:boundt6-intermediate-135}, \eqref{eq:boundt8} and \eqref{eq:sum-hat-r-UB-first}, we can readily derive
%
\begin{align}
T_{6} & \leq T_{8}+2HT_{2}+2HT_{9}+2H\sum_{k=1}^{K}\sum_{h=1}^{H}\widehat{r}_{h}(s_{h}^{k},a_{h}^{k})\nonumber\\
 %& \leq\sqrt{32T_{6}\log\frac{1}{\delta'}}+3H^{2}\log\frac{1}{\delta'}+2HT_{2}+2HT_{9}+2H\sum_{k=1}^{K}\sum_{h=1}^{H}\widehat{r}_{h}(s_{h}^{k},a_{h}^{k})\nonumber\\
 & \leq \sqrt{32T_{6}\log\frac{1}{\delta'}}+2HT_{9}+16HKv^{\star}+3H^{2}\log\frac{1}{\delta'}+2HT_{2}
	\label{eq:fnbt6-first}
\end{align}
%
with probability at least $1-16SAH^2K^2\delta'$. 





\paragraph{Putting all pieces together.}
%
% Assume that $K\geq \frac{BSAH^2}{v^{\star}}$. Then $\sqrt{BSAH^2Kv^{\star}}\geq BSAH^2$. 
%
Recalling our choice of $B$ (cf.~\eqref{eq:defn-B-first}), 
we can see from \eqref{eq:nbft2}, \eqref{eq:boundt3}, \eqref{eq:fnbt4}, \eqref{eq:fnbt5}, \eqref{eq:fnbt6-first}, \eqref{eq:boundt8}, \eqref{eq:boundt1} and \eqref{eq:boundt7} that
%
\begin{subequations}
	\label{eq:all-T-bounds-first}
\begin{align}
T_{2} & \leq\sqrt{BSAHT_{5}}+\sqrt{BSAH^{2}(T_{4}+3Kv^{\star})}+BSAH^{2},\\
T_{3} & \leq\sqrt{BT_{6}}+BH,\\
T_{4} & \leq\sqrt{BSAH^{2}Kv^{\star}}+BSAH^{2},\\
T_{5} & \leq T_{7}+T_{8}+2HT_{2}+16HKv^{\star},\\
T_{6} & \leq\sqrt{BT_{6}}+2HT_{9}+16HKv^{\star}+BH^{2}+2HT_{2},\\
T_{8} & \leq\sqrt{BH^{2}T_{6}}+BH^{2},\\
T_{1}\leq T_{9} & \leq\sqrt{BSAHT_{6}}+BSAH^{2},\\
T_{7} & \leq H\sqrt{BSAHT_{6}}+BSAH^{3}.
\end{align}
\end{subequations}
%
Solving \eqref{eq:all-T-bounds-first} under the assumption $K\geq \frac{BSAH^2}{v^{\star}}$ allows us to demonstrate that
%
\begin{subequations}
\begin{align}
	T_6 &\lesssim BHKv^{\star} \\
	T_1\leq T_9 &\lesssim \sqrt{B^2SAH^2Kv^{\star}}\\
	T_7+T_8 &\lesssim \sqrt{B^2SAH^4Kv^{\star}} \\
	T_5 &\lesssim BHKv^{\star} \\
	T_2 &\lesssim \sqrt{B^2SAH^2Kv^{\star}} \\
	T_3 &\lesssim \sqrt{B^2HKv^{\star}} \\
	T_4 & \lesssim \sqrt{BSAH^{2}Kv^{\star}}
\end{align}
\end{subequations}
%
with probability exceeding $1-200SAH^2K^2\delta'$. 
Putting these bounds together with \eqref{eq:decomposition}, we arrive at
%
\[
	\mathsf{Regret}(K)\leq T_{1}+T_{2}+T_{3}+T_{4} 
	\lesssim B\sqrt{SAH^2Kv^{\star}} 
\]
%
with probability exceeding $1-200SAH^2K^2\delta'$. 
Replacing $\delta'$ with $\frac{\delta}{200SAH^2K^2}$ and taking $\delta=\frac{1}{2KH}$ give
%
\begin{align*}
	\mathbb{E}\big[\mathsf{Regret}(K)\big]
	&\lesssim(1-\delta)B\sqrt{SAH^{2}Kv^{\star}}+\delta Kv^{\star}\lesssim B\sqrt{SAH^{2}Kv^{\star}}+1 \asymp B\sqrt{SAH^{2}Kv^{\star}} \\
	&  
	\asymp \min\big\{ B\sqrt{SAH^{2}Kv^{\star}}, B Kv^{\star} \big\} 
	\asymp  \min\big\{ \sqrt{SAH^{2}Kv^{\star}}, Kv^{\star}\big\} \log^{5}(SAHK) 
	,
\end{align*}
%
provided that $K\geq \frac{BSAH^2}{v^{\star}}$. 
Taking this collectively with \eqref{eq:E-regret-UB-easy-first} concludes the proof. 



\section{Proof of Corollary~\ref{thm:cost}}\label{app:cost}

In this section, we will use $r$ to denote the negative reward, that is, $r= -c$. Recall \eqref{eq:updatecost}:
\begin{align}
Q_h(s,a)\leftarrow \max\{\min \left\{ \hat{r}_h(s,a) + \hat{P}_{s,a,h}V_{h+1}+b_h(s,a), 0 \right\} ,-H\}.\nonumber
\end{align}



Recall the definition of $T_1$-$T_9$. We note that the analysis of $T_1,T_3,T_7,T_8$ and $T_9$ in Appendix~\ref{sec:appfirst} applies for the case the reward function is negative. So it suffices to provide bounds for $T_2,T_4,T_5$ and $T_6$ with respect to $c^*$.



\paragraph{Bound of $T_2$}
Recall that 
\begin{align}
T_2  & = \sum_{k=1}^K \sum_{h=1}^H b_h^k(s_h^k,a_h^k) \nonumber
\\ & =\sum_{k=1}^K \sum_{h=1}^H \Bigg( \frac{460}{9}  \sqrt{\frac{\mathbb{V}(\hat{P}^k_{s_h^k,a_h^k,h},V_{h+1}^k)\log(\frac{1}{\delta})}{N_h^k(s_h^k,a_h^k)}} \nonumber\\ & \qquad \qquad \qquad \qquad   +  2\sqrt{2} \sqrt{\frac{\left(\hat{\sigma}_h^k(s_h^k,a_h^k)- (\hat{r}_h^k(s_h^k,a_h^k))^2\right) \log(\frac{1}{\delta})}{N_h^k(s_h^k,a_h^k)}}+ \frac{544}{9}\frac{H\log(\frac{1}{\delta})}{N_h^k(s_h^k,a_h^k)}\Bigg).\label{eq:cc1}
\end{align}
For the first and third term in right hand side of \eqref{eq:cc1}, we can use Cauchy's inequality to obtain that 
\begin{align}
 & \sum_{k=1}^K \sum_{h=1}^H \sqrt{\frac{\mathbb{V}(\hat{P}^k_{s_h^k,a_h^k,h},V_{h+1}^k)\log(\frac{1}{\delta})}{N_h^k(s_h^k,a_h^k)}} \nonumber
 \\ & \leq   \sqrt{2SAH\log_2(K)\log(\frac{1}{\delta})\sum_{k,h}\mathbb{V}(\hat{P}^k_{s_h^k,a_h^k,h},V_{h+1}^k)}\nonumber
 \\ & = \sqrt{2SAH\log_2(K)\log(\frac{1}{\delta})T_5} \label{eq:cterm1} 
 \end{align}
 and
 \begin{align}
\sum_{k=1}^K \sum_{h=1}^H \frac{H\log(\frac{1}{\delta})}{N_h^k(s_h^k,a_h^k)} \leq 2SAH^2\log_2(K)\log(\frac{1}{\delta}).\label{eq:cterm2}
\end{align}

For the second term, noting that
\begin{align}
\left( \hat{\sigma}_h^k(s_h^k,a_h^k) - (\hat{r}_h^k(s_h^k,a_h^k)^2)\right) \leq  -H \hat{r}_h^k(s_h^k,a_h^k),\nonumber
\end{align}
we have 
\begin{align}
&\sqrt{\frac{\left(\hat{\sigma}_h^k(s_h^k,a_h^k)- (\hat{r}_h^k(s_h^k,a_h^k))^2\right) \log(\frac{1}{\delta})}{N_h^k(s_h^k,a_h^k)}}\nonumber
\\ &  \leq \sqrt{2SAH\log_2(K)\log(\frac{1}{\delta})}\sqrt{H\sum_{k,h}-\hat{r}_h^k(s_h^k,a_h^k)} \nonumber
\\ & \leq \sqrt{2SAH^2\log_2(K)\log(\frac{1}{\delta})}\sqrt{T_4 + 3Kc^* + \sum_{k=1}^K (-V_1^{\pi^k}(s_1^k) +V_1^*(s_1^k)  )+\sum_{k=1}^K (-V_1^*(s_1^k)-3c^*) } .\label{eq:cterm3}
\end{align}

By Lemma~\ref{lemma:con}, with probability $1-\delta$,
\begin{align}
\sum_{k=1}^K -V_1^*(s_1^k)\leq 3Kc^* + H\log(\frac{1}{\delta}).\nonumber
\end{align}
On the other hand, we note that
\begin{align}
\sum_{k=1}^K (-V_1^{\pi^k}(s_1^k) +V_1^*(s_1^k)  ) = \mathrm{Regret}(K) = T_1+T_2+T_3+T_4.
\end{align}

Putting all together, we obtain that, with probability $1-\delta$,
\begin{align}
T_2 &  \leq 90\sqrt{SAH\log_2(K)\log(\frac{1}{\delta})T_5}  \nonumber\\ & \qquad + 4\sqrt{SAH^2\log_2(K)\log(\frac{1}{\delta})}\sqrt{T_1+T_2+T_3+2T_4+3Kv^*} + 130 SAH^2\log_2(K)\log(\frac{1}{\delta}).\label{eq:ccbt2}
\end{align}


\paragraph{Bound of $T_4$}


Recall that
\begin{align}
 T_4 & = \sum_{k=1}^K \left( \sum_{h=1}^H \hat{r}_h^k(s_h^k,a_h^k)-V_{1}^{\pi^k}(s_1^k)\right) \nonumber
 \\ & = \sum_{k=1}^K \left( \sum_{h=1}^H \hat{r}_h^k(s_h^k,a_h^k)-r_{h}(s_h^k,a_h^k)\right) + \sum_{k=1}^K \left( \sum_{h=1}^H r_h(s_h^k,a_h^k) - V_{1}^{\pi^k}(s_1^k) \right).
 \end{align}

Also recall that $\check{T}_1 =\sum_{k=1}^K \left( \sum_{h=1}^H \hat{r}_h^k(s_h^k,a_h^k)-r_{h}(s_h^k,a_h^k)\right)  $ and $\check{T}_2 =\sum_{k=1}^K \left( \sum_{h=1}^H r_h(s_h^k,a_h^k) - V_{1}^{\pi^k}(s_1^k) \right) $. We continue with a lemma to bound the empirical reward for negative reward function.

\begin{lemma}\label{lemma:bdempc}
With probability $1-2SAHK\delta$, it holds that
\begin{align}
& \sum_{k=1}^K \sum_{h=1}^H \left| \hat{r}_h^k(s_h^k,a_h^k) - r_h(s_h^k,a_h^k)\right|\nonumber
\\ & \leq  4SAH^2 +4\sqrt{\sum_{k=1}^K\sum_{h=1}^H \frac{H\log(\frac{1}{\delta})}{N_h^k(s_h^k,a_h^k)}}\cdot \sqrt{\sum_{k=1}^K \sum_{h=1}^H -r_h(s_h^k,a_h^k)}+24\sum_{k=1}^K\sum_{h=1}^H\frac{H\log(\frac{1}{\delta})}{N_h^k(s_h^k,a_h^k)}.\nonumber
\end{align}
\end{lemma}
The proof of Lemma~\ref{lemma:bdempc} is basically the same as that of Lemma~\ref{lemma:bdempr}, except for that $r$ is replaced with $-r$.


By Lemma~\ref{lemma:bdempc} and Lemma~\ref{lemma:doubling}, with probability $1-3SAHK\delta$,  
\begin{align}
 |\check{T}_1 |& \leq 4\sqrt{2SAH^2\log_2(K)}\cdot \sqrt{\sum_{k=1}^K\sum_{h=1}^H -r_h(s_h^k,a_h^k)} + 52SAH^2\log_2(K)\log(\frac{1}{\delta}) \nonumber
  \\ & \leq 4\sqrt{2SAH^2\log_2(K)}\cdot \sqrt{\check{T}_2 + 3Kc^*+\sum_{k=1}^K (-V_1^{*}(s_1^k)- 3c^*) } + 52SAH^2\log_2(K)\log(\frac{1}{\delta}) \nonumber
  \\ & \leq 4\sqrt{2SAH^2\log_2(K)}\cdot \sqrt{\check{T}_2 + 3Kc^* } + 60SAH^2\log_2(K)\log(\frac{1}{\delta}) ,\label{eq:cct1}
\end{align}
where in the last line we use the fact 
\begin{align}
\sum_{k=1}^K -V_1^*(s_1^k)\leq 3Kc^* + H\log(\frac{1}{\delta})\label{eq:addc}
\end{align}
with probability $1-\delta$ (Lemma~\ref{lemma:con}).

On the other hand, by Lemma~\ref{lemma:self-norm} and \eqref{eq:addc}, with probability $1-3SAHK\delta$,
\begin{align}
|\check{T}_2| & \leq 2\sqrt{2\sum_{k=1}^K \mathbb{E}_{\pi^k,}\left[\left(\sum_{h=1}^H r_h(s_h,a_h) \right)^2 |s_1=s_1^k\right]\log(\frac{1}{\delta})}+3H^2\log(\frac{1}{\delta}) \nonumber
\\ & 2\sqrt{2H\sum_{k=1}^K \mathbb{E}_{\pi^k}\left[\sum_{h=1}^H  -r_h(s_h,a_h) |s_1=s_1^k \right]\log(\frac{1}{\delta})}+3H\log(\frac{1}{\delta}) \nonumber
\\ & \leq 2\sqrt{2H\left( \sum_{k=1}^K \left(- V_1^{\pi^k}(s_1^k) +V_1^*(s_1^k) \right) + \sum_{k=1}^K\left(-V_1^*(s_1^k)-3c^*\right)+3Kc^*\right)\log(\frac{1}{\delta})}+3H\log(\frac{1}{\delta}) \label{eq:cct1.5}
\\ & \leq 3Kc^*  +T_1+T_2+T_3+T_4 + 9H\log(\frac{1}{\delta}).\label{eq:cct2}
\end{align}

Combining \eqref{eq:cct1}, \eqref{eq:cct1.5} with \eqref{eq:cct2}, with probability $1-4SAHK\delta$,
\begin{align}
& |\check{T}_1|\leq     16\sqrt{SAH^2(Kc^*+T_1+T_2+T_3+T_4)\log_2(K)\log(\frac{1}{\delta})}  + 200SAH^2\log_2(K)\log(\frac{1}{\delta})\nonumber
\\ & |\check{T}_2| \leq  2\sqrt{2H(3Kc^*+T_1+T_2+T_3+T_4)\log(\frac{1}{\delta})}+9H\log(\frac{1}{\delta}) .\nonumber
\end{align}

As a result, we have that
\begin{align}
|T_4| \leq 22\sqrt{SAH^2(Kc^* +T_1+T_2+T_3+T_4)\log_2(K)\log(\frac{1}{\delta})}  + 209SAH^2\log_2(K)\log(\frac{1}{\delta}).\label{eq:ccbt4}
\end{align}

\paragraph{Bound of $T_5$}

Using the arguments in  \eqref{eq:boundt5}, and noting the update rule \eqref{eq:updatecost}, we have
\begin{align}
T_5  & \leq \sum_{k=1}^K \sum_{h=1}^H   (\hat{P}^k_{s_h^k,a_h^k,h} -P_{s_h^k,a_h^k,h})(V^k_{h+1})^2   + \sum_{k=1}^K \sum_{h=1}^H (P_{s_h^k,a_h^k,h} - \textbf{1}_{s_{h+1}^k} )  (V_{h+1}^k)^2  +2H\sum_{k=1}^K \sum_{h=1}^H -r_h(s_h^k,a_h^k).\nonumber
\end{align}

Recall that
\begin{align}
\sum_{k=1}^K \sum_{h=1}^H -r_h(s_h^k,a_h^k) &  = - \check{T}_2 -\sum_{k=1}^K V_1^{\pi^k}(s_1) \leq -\check{T}_2 + \sum_{k=1}^K V_1^{*}(s_1^k).\label{eq:cx1}
\end{align}
By \eqref{eq:addc}, with probability $1-5SAHK\delta$,
\begin{align}
\sum_{k=1}^K \sum_{h=1}^H -r_h(s_h^k,a_h^k) \leq  2\sqrt{2H(3Kc^* + T_1+T_2+T_3+T_4)\log(\frac{1}{\delta})} + 3Kc^* + 10H\log(\frac{1}{\delta}).
\end{align}
As a result, we have that
\begin{align}
T_5 & \leq T_7 + T_8+2HT_2 + 4\sqrt{2H^3(3Kc^* +T_1+T_2+T_3+T_4)\log(\frac{1}{\delta})}+ 6HKc^* + 20H^2\log(\frac{1}{\delta}).\label{eq:ccbt5}
\end{align}
with probability $1-5SAHK\delta$.



\paragraph{Bound of $T_6$}

Using the arguments in  \eqref{eq:boundt5}, \eqref{eq:addc} and \eqref{eq:cx1}, and noting the update rule \eqref{eq:updatecost}, with probability $1-3SAHK\delta$
\begin{align}
T_6 & \leq 2\sqrt{8T_6\log(\frac{1}{\delta})} + 3H^2\log(\frac{1}{\delta})+2H \sum_{k=1}^K\sum_{h=1}^H \max\{P_{s_h^k,a_h^k,h}V_{h+1}^k-V_{h}^k(s_h^k),0 \} \nonumber
\\ & \leq  2\sqrt{8T_6\log(\frac{1}{\delta})} + 3H^2\log(\frac{1}{\delta})+ 2HT_9 + 2H\sum_{k=1}^K\sum_{h=1}^H -r_h(s_h^k,a_h^k)\nonumber
\\ & \leq 2\sqrt{8T_6\log(\frac{1}{\delta})} + 3H^2\log(\frac{1}{\delta})+ 2HT_9 \nonumber
\\ & \qquad \qquad \qquad + 2H \left(2\sqrt{2H (3Kc^*+T_1+T_2+T_3+T_4)\log(\frac{1}{\delta})  }+3Kc^*+10H\log(\frac{1}{\delta})\right).\label{eq:ccbt6}
\end{align}

\paragraph{Putting all together}


Solving \eqref{eq:ccbt2},\eqref{eq:boundt3},\eqref{eq:ccbt4},\eqref{eq:ccbt5},\eqref{eq:ccbt6},\eqref{eq:boundt8},\eqref{eq:boundt1},\eqref{eq:boundt7} and \eqref{eq:boundt9}, we have that, with probability $1-100SAH^2K\delta$, $T_6 = O(HKc^*+BSAH^3)$, $T_1 = O(\sqrt{BSAH^2Kc^*}+BSAH^2)$, $T_7,T_8 = O(\sqrt{BSAH^4Kc^*}+BSAH^3)$, $T_5 = O(HKc^*+BSAH^2)$, $T_2 = O(\sqrt{BSAH^2Kc^*}+BSAH^2)$ and $T_3 = O(\sqrt{BHKc^*}+BSAH^2)$. We then conclude that the total regret is bounded by $O(\sqrt{BSAH^2Kc^*}+BSAH^2)$. On the other hand, the regret bound is trivially bounded by $O(K(H-c^*))$. The proof is completed by replacing $\delta$ with $\frac{\delta}{100SAH^2K}$.



\section{Proof of the variance-dependent regret bounds}\label{app:var}

\subsection{Proof of Theorem~\ref{thm:var}}
In this section, we will present the proof of Theorem~\ref{thm:var}. The proof contains two parts, where we respectively prove regret bounds of $\widetilde{O}\left(\min\{\sqrt{SAHK\mathrm{var}_1} +SAH^2,KH\}  \right)$ and $\widetilde{O}\left(\min\{\sqrt{SAHK\mathrm{var}_2} +SAH^2,KH\}  \right)$ . 
Formally we have the following lemmas.

\begin{lemma}\label{lemma:var1}
 With probability exceeding $1-\delta$, the regret of Algorithm~\ref{alg:main} is at most $\widetilde{O}(\min\{\sqrt{SAHK\mathrm{var}_1}+SAH^2,KH\})$
\end{lemma}
\begin{lemma}\label{lemma:var2}
 With probability at least $1-\delta$, the regret of Algorithm~\ref{alg:main} is at most $\widetilde{O}(\min\{\sqrt{SAHK\mathrm{var}_2}+SAH^2,KH\})$
\end{lemma}

Putting the two regret bounds together and rescaling $\delta$ to $\delta/2$, we conclude the proof.

\subsection{Proof of Lemma~\ref{lemma:var1}}







Recall that 
\begin{align}
&T_4 = \sum_{k=1}^K \left( \sum_{h=1}^H \hat{r}_h^k(s_h^k,a_h^k) - V_1^{\pi^k}(s_1^k) \right);\nonumber
\\&T_5 = \sum_{k=1}^K \sum_{h=1}^H \mathbb{V}(\hat{P}_{s_h^k,a_h^k,h},V_{h+1}^k);\nonumber
\\ &T_6 = \sum_{k=1}^K \sum_{h=1}^H \mathbb{V}(P_{s_h^k,a_h^k,h},V_{h+1}^k).\nonumber
 \end{align}
Recall that $B =4000\log^2_3(K)\log(3SAH)\log(\frac{1}{\delta})$.


\subsubsection{Bound of $T_2$}
Recall in \eqref{eq:boundt2}, we show that
\begin{align}
T_2 &\leq \frac{460}{9}\sqrt{2SAH\log_2(K)\log(\frac{1}{\delta})T_5} \nonumber
\\ & \qquad \qquad +4\sqrt{SAH\log_2(K)\log(\frac{1}{\delta})}\sqrt{\sum_{k,h}\left(\hat{\sigma}_h^k(s_h^k,a_h^k)- (\hat{r}_h^k(s_h^k,a_h^k))^2\right)} + \frac{1088}{9}SAH^2\log_2(K)\log(\frac{1}{\delta}).\label{eq:local3}
 \end{align}


Define the variance of $R_h(s,a)$ as $v_h(s,a)$. We then have the following lemma.
\begin{lemma}\label{lemma:bdrv}
With probability $1-4SAHK\delta$,
\begin{align}
\sum_{k,h}\left(\hat{\sigma}_h^k(s_h^k,a_h^k)- (\hat{r}_h^k(s_h^k,a_h^k))^2\right)\leq  6K\mathrm{var}_1 + 242SAH^3\log_2(K)\log(\frac{1}{\delta}).
\end{align}
\end{lemma}
\begin{proof}
We first control each $\hat{\sigma}_h^k(s_h^k,a_h^k)- (\hat{r}_h^k(s_h^k,a_h^k))^2$ with $v_h(s,a)$. Fix $(s,a,h,k)$. Using Lemma~\ref{lemma:con}, with probability $1-2\delta$,
\begin{align}
N_h^k(s,a)\left(\hat{\sigma}_h^k(s_h^k,a_h^k)- (\hat{r}_h^k(s_h^k,a_h^k))^2\right)\leq 3N_h^k v_h(s,a)+ H^2\log(\frac{1}{\delta}).
\end{align}

Then we have that, with probability $1-2SAHK\delta$,
\begin{align}
&\sum_{k,h}\left(\hat{\sigma}_h^k(s_h^k,a_h^k)- (\hat{r}_h^k(s_h^k,a_h^k))^2\right)\nonumber
\\ &\leq 3\sum_{k,h}v_h(s_h^k,a_h^k) + \sum_{k,h}\frac{H^2\log(\frac{1}{\delta})}{N_h^k(s_h^k,a_h^k)} \leq 3\sum_{k,h}v_h(s_h^k,a_h^k) +2SAH^3\log_2(K)\log(\frac{1}{\delta}).\label{eq:bdvr1}
\end{align}


Now it suffices to control $\sum_{k,h}v_h(s_h^k,a_h^k)$. Let $\tilde{V}_h^k(s):=\mathbb{E}_{\pi^k}[\sum_{h'=h}^H v_{h'}(s_{h'},a_{h'})|s_h = s]$ be the value function with reward as $\{v_h(s,a)\}$ and policy $\pi^k$. Then $\tilde{V}^k_h(s,a)\leq H^2$.

Then, by Lemma~\ref{lemma:self-norm}, with probability $1-2SAHK\delta$,
\begin{align}
\sum_{k=1}^K\sum_{h=1}^Hv_h(s_h^k,a_h^k) - \sum_{k=1}^K \tilde{V}_1^k(s_1^k) &= \sum_{k=1}^K \left( \sum_{h=1}^H \left(\textbf{1}_{s_{h+1}^k}-P_{s_h^k,a_h^k,h}\right)\tilde{V}^k_{h+1} \right) \nonumber
\\ & \leq 2\sqrt{2\sum_{k=1}^K \sum_{h=1}^H \mathbb{V}(P_{s_h^k,a_h^k,h},\tilde{V}_{h+1}^k)\log(\frac{1}{\delta})} + 3H^2 \log(\frac{1}{\delta}).\label{eq:local0}
\end{align}

On the other hand, using Lemma~\ref{lemma:self-norm} again, we obtain that with probability $1-2SAHK\delta$
\begin{align}
&\sum_{k=1}^K \sum_{h=1}^H \mathbb{V}(P_{s_h^k,a_h^k,h},\tilde{V}_{h+1}^k)\nonumber
 \\ & = \sum_{k=1}^K \sum_{h=1}^H \left( P_{s_h^k,a_h^k,h} -\textbf{1}_{s_{h+1}^k}  \right)(\tilde{V}_{h+1}^k)^2  \nonumber
 \\ & \qquad \qquad \qquad + \sum_{k=1}^H \sum_{h=1}^H \left((\tilde{V}_{h+1}^k(s_{h+1}^k))^2- (\tilde{V}_h^k(s_h^k))^2 \right)+\sum_{k=1}^K \sum_{h=1}^H \left( (\tilde{V}_h^k(s_h^k))^2 - (P_{s_h^k,a_h^k,h}\tilde{V}_{h+1}^k)^2\right) \nonumber
 \\ & \leq 2\sqrt{8H^4\sum_{k=1}^K \sum_{h=1}^H \mathbb{V}(P_{s_h^k,a_h^k,h},\tilde{V}_{h+1}^k)\log(\frac{1}{\delta})   } +2H^2\sum_{k=1}^K\sum_{h=1}^H v_h(s_h^k,a_h^k) + 3H^4\log(\frac{1}{\delta})\nonumber
 \\ & \leq 4H^2\sum_{k=1}^K\sum_{h=1}^H v_h(s_h^k,a_h^k)+42H^4\log(\frac{1}{\delta}).\label{eq:local1}
\end{align}

By \eqref{eq:local0} and \eqref{eq:local1}, we learn that, with probability $1-4SAHK\delta$,
\begin{align}
\sum_{k=1}^K\sum_{h=1}^K v_h(s_h^k,a_h^k) & \leq \sum_{k=1}^K \tilde{V}_1^k(s_1^k) + 2\sqrt{8H^2\sum_{k=1}^K\sum_{h=1}^Hv_h(s_h^k,a_h^k) \log\frac{1}{\delta}+ 84H^4\log^2(\frac{1}{\delta})}+3H^2\log(\frac{1}{\delta}) \nonumber
\\ & \leq   2\sum_{k=1}^K \tilde{V}_1^k(s_1^k)  +80H^2\log(\frac{1}{\delta})\nonumber
\\ & \leq 2K\mathrm{var}_1 +80H^2\log(\frac{1}{\delta}).\label{eq:bdddv}
\end{align}

\end{proof}

With Lemma~\ref{lemma:bdrv} and \eqref{eq:local3}, with probability $1-4SAHK\delta$,
\begin{align}
T_2\leq \frac{460}{9}\sqrt{2SAH\log_2(K)\log(\frac{1}{\delta})T_5} +12\sqrt{SAH\log_2(K)\log(\frac{1}{\delta})}\sqrt{2K\mathrm{var}_1} + 157SAH^2\log_2(K)\log(\frac{1}{\delta}).\label{eq:nbt2}
\end{align}


\subsubsection{Bound of $T_4$}

Recall that $T_4 = \check{T}_1+\check{T}_2$ where $\check{T}_1 = \sum_{k=1}^K \sum_{h=1}^H  \left( \hat{r}_h^k(s_h^k,a_h^k)-r_h(s_h^k,a_h^k) \right)$ and $\check{T}_2 = \sum_{k=1}^K \left( \sum_{h=1}^H r_h(s_h^k,a_h^k) - V_1^{\pi^k}(s_1^k) \right)$.

We first bound $\check{T}_1$. By Lemma~\ref{bennet} and a union bound over all proper $(s,a,h,k)$, with probability $1-2SAHK\delta$,
\begin{align}
\hat{r}_h^k(s,a)-r_h(s,a)\leq \sqrt{\frac{2v_h(s,a)\log(\frac{1}{\delta})}{N_h^k(s,a)}}+ \frac{H\log(\frac{1}{\delta})}{N_h^k(s,a)}.
\end{align}

As a result, we have that
\begin{align}
|\check{T}_1| &  \leq \sum_{k=1}^K \sum_{h=1}^H\left(  \sqrt{\frac{2v_h(s_h^k,a_h^k)\log(\frac{1}{\delta})}{N_h^k(s_h^k,a_h^k)}}+ \frac{H\log(\frac{1}{\delta})}{N_h^k(s_h^k,a_h^k)}  \right)\nonumber
\\ & \leq \sqrt{4SAH\log_2(K)\log(\frac{1}{\delta})}\cdot \sqrt{\sum_{k=1}^K\sum_{h=1}^H v_h(s_h^k,a_h^k)} + 2SAH^2\log_2(K)\log(\frac{1}{\delta}).\label{eq:wc3}
\end{align}

By \eqref{eq:bdddv}, with probability $1-4SAHK\delta$,
\begin{align}
\sum_{k=1}^K\sum_{h=1}^H v_h(s_h^k,a_h^k)\leq 2K\mathrm{var}_1 + 80H^2\log(\frac{1}{\delta}).\label{eq:wc1}
\end{align}

Then
\begin{align}
|\check{T}_1| \leq \sqrt{8SAHK\mathrm{var}_1\log_2(K)\log(\frac{1}{\delta})}+ 20SAH^2\log_2(K)\log(\frac{1}{\delta}).\label{eq:cht1}
\end{align}

On the other hand, to bound $\check{T}_2$, we have that
\begin{align}
\check{T}_2 = \sum_{k=1}^K \sum_{h=1}^H \left(\textbf{1}_{s_{h+1}^k}-P_{s_h^k,a_h^k,h} \right)V_{h+1}^{\pi^k}.
\end{align}

Using Lemma~\ref{lemma:self-norm}, with probability $1-2SAHK\delta$,
\begin{align}
|\check{T}_2|  & \leq 2\sqrt{2\sum_{k=1}^K\sum_{h=1}^H \mathbb{V}(P_{s_h^k,a_h^k,h},V_{h+1}^{\pi^k})\log(\frac{1}{\delta}) }+ 3H\log(\frac{1}{\delta})\nonumber
\\ & \leq 2\sqrt{4\sum_{k=1}^K \sum_{h=1}^H (\mathbb{V}(P_{s_h^k,a_h^k,h},V_{h+1}^*) + \mathbb{V}(P_{s_h^k,a_h^k,h}, V^*_{h+1}-V_{h+1}^{\pi^k})  )\log(\frac{1}{\delta})}+3H\log(\frac{1}{\delta}).\label{eq:cls0}
\end{align}


Continue the computation,
\begin{align}
&\sum_{k=1}^K\sum_{h=1}^H \mathbb{V}(P_{s_h^k,a_h^k,h},V_{h+1}^* -V_{h+1}^{\pi^k}) \nonumber
\\ & = \sum_{k=1}^K\sum_{h=1}^H \left(  P_{s_h^k,a_h^k,h} (V_{h+1}^* - V_{h+1}^{\pi^k})^2 - (P_{s_h^k,a_h^k,h} (V_{h+1}^* - V_{h+1}^{\pi^k}) )^2       \right)\nonumber
\\ & \leq \sum_{k=1}^K \sum_{h=1}^H \left( P_{s_h^k,a_h^k,h}-\textbf{1}_{s_{h+1}^k} \right) (V_{h+1}^* -V_{h+1}^{\pi^k})^2 \nonumber
\\  &\quad + 2H\sum_{k=1}^K \sum_{h=1}^H \max\left\{   \left( V_{h}^*(s_h^k)-r_{h}(s_h^k,a_h^k) - P_{s_h^k,a_h^k,h}V_{h+1}^*\right)- \left( V_{h}^{\pi^k}(s_h^k) -r_h(s_h^k,a_h^k) - P_{s_h^k,a_h^k,h}V_{h+1}^{\pi^k} \right)    ,0 \right\} \nonumber
\\ & \leq 2\sqrt{8H^2\sum_{k=1}^K\sum_{h=1}^H \mathbb{V}(P_{s_h^k,a_h^k,h},V_{h+1}^* - V_{h+1}^{\pi^k})\log(\frac{1}{\delta})}   
\nonumber
\\ &\qquad \qquad \qquad \qquad +    2H\sum_{k=1}^K \sum_{h=1}^H \left(V_{h}^*(s_h^k)-r_h(s_h^k,a_h^k) -P_{s_h^k,a_h^k,h}V_{h+1}^*\right)   + 3H^2\log(\frac{1}{\delta})\label{eq:csl1}
\end{align}
Here \eqref{eq:csl1} holds with probability $1-2SAHK\delta$ because of Lemma~\ref{lemma:self-norm}
 and Lemma~\ref{lemma:sqv}.

Then we consider to bound
\begin{align}
& \sum_{k=1}^K \sum_{h=1}^H \left(V_{h}^*(s_h^k)-r_h(s_h^k,a_h^k) -P_{s_h^k,a_h^k,h}V_{h+1}^*\right)   \nonumber
\\ & = \sum_{k=1}^K \left( V_1^*(s_1^k) - V_1^{\pi^k}(s_1^k) \right) + \sum_{k=1}^K(V_1^{\pi^k}(s_1^k) - \sum_{h=1}^H r_h(s_h^k,a_h^k)) + \sum_{k=1}^K\sum_{h=1}^H (\textbf{1}_{s_{h+1}^k}-P_{s_h^k,a_h^k,h})V_{h+1}^*.\label{eq:d1}
\end{align}

The first term in the right hand side \eqref{eq:d1} is exactly $\mathrm{Regret}(K)=T_1+T_2+T_3+T_4$, the second term is $-T_4$, and the third term is bounded by 
\begin{align}
\sum_{k=1}^K\sum_{h=1}^H (\textbf{1}_{s_{h+1}^k}-P_{s_h^k,a_h^k,h})V_{h+1}^*\leq 2\sqrt{2\sum_{k=1}^K\sum_{h=1}^H \mathbb{V}(P_{s_h^k,a_h^k,h},V_{h+1}^*)\log(\frac{1}{\delta})}+3H\log(\frac{1}{\delta})
\end{align}
with probability $1-2SAHk\delta$.


It then follows that with probability $1-8SAHK\delta$,
\begin{align}
& \sum_{k=1}^K \sum_{h=1}^H \left(V_{h}^*(s_h^k)-r_h(s_h^k,a_h^k) -P_{s_h^k,a_h^k,h}V_{h+1}^*\right) \nonumber
\\ & \leq T_1+T_2+T_3+2|T_4| + 2\sqrt{2\sum_{k=1}^K\sum_{h=1}^H\mathbb{V}(P_{s_h^k,a_h^k,h},V_{h+1}^*)\log(\frac{1}{\delta})} + 55H\log(\frac{1}{\delta}).
\end{align}

With \eqref{eq:csl1}, we further obtain that, with probability $1-8SAHK\delta$
\begin{align}
&\sum_{k=1}^K\sum_{h=1}^H \mathbb{V}(P_{s_h^k,a_h^k,h},V_{h+1}^*-V_{h+1}^{\pi^k}) \nonumber
\\ & \leq   4H(T_1+T_2+T_3+2|T_4|+2\sqrt{2\sum_{k=1}^K\sum_{h=1}^H\mathbb{V}(P_{s_h^k,a_h^k,h},V_{h+1}^*)\log(\frac{1}{\delta})})    +262H^2\log(\frac{1}{\delta}).\label{eq:wc2}
\end{align}

Define $T_{10} = \sum_{k=1}^K\sum_{h=1}^H\mathbb{V}(P_{s_h^k,a_h^k,h},V_{h+1}^*)$.  
Plugging \eqref{eq:wc2} into \eqref{eq:cls0}, with probability $1-10SAHK\delta$,
\begin{align}
|\check{T}_2|&\leq 2\sqrt{8K\mathrm{var}_1 \log(\frac{1}{\delta})} +  8\sqrt{H(T_1+T_2+T_3+2|T_4|+2\sqrt{2T_{10}\log(\frac{1}{\delta})})\log(\frac{1}{\delta})} +107H\log(\frac{1}{\delta})  \nonumber
\\ &  \leq 11\sqrt{T_{10}\log(\frac{1}{\delta})} +16\sqrt{H(T_1+T_2+T_3+2|T_4|)\log(\frac{1}{\delta})} + 115H\log(\frac{1}{\delta}).
\end{align}

Recalling \eqref{eq:cht1},  with probability $1-10SAHK\delta$
\begin{align}
|T_4|&\leq  18\sqrt{SAHT_{10}\log_2(K)\log(\frac{1}{\delta})} +16\sqrt{H(T_1+T_2+T_3+2|T_4|)\log(\frac{1}{\delta})} + 135SAH^2\log_2(K)\log(\frac{1}{\delta}) \nonumber
\\ & \leq   36\sqrt{SAHT_{10}\log_2(K)\log(\frac{1}{\delta})} + 32\sqrt{H(T_1+T_2+T_3)\log(\frac{1}{\delta})}+           306SAH^2\log_2(K)\log(\frac{1}{\delta}).\label{eq:nbt4}
\end{align}



\subsubsection{Bound of $T_5$ and $T_6$}

We start with the following lemma
\begin{lemma}\label{lemma:empv}
With probability $1-2SAHK\delta$,
\begin{align}
 T_5 \leq  5T_6 +8BSAH^3.\label{eq:var5x}
\end{align}

\end{lemma}
\begin{proof}[Proof of Lemma~\ref{lemma:empv}]
Direct computation gives that
\begin{align}
 & \sum_{k,h}\mathbb{V}(\hat{P}_{s_h^k,a_h^k,h}^k, V_{h+1}^k)\nonumber\\ &  = \sum_{k,h} \left(\hat{P}_{s_h^k,a_h^k,h}^k  (V_{h+1}^k)^2 - (\hat{P}_{s_h^k,a_h^k,h}V_{h+1}^k)^2 \right) \nonumber
\\ & \leq  \sum_{k,h}\left(P_{s_h^k,a_h^k,h}^k  (V_{h+1}^k)^2   - (P_{s_h^k,a_h^k,h}V_{h+1}^k)^2 \right) + \sum_{k,h}\left(\hat{P}_{s_h^k,a_h^k,h}-P_{s_h^k,a_h^k,h}\right)(V_{h+1}^k)^2 + 2H\sum_{k,h}\left(\hat{P}_{s_h^k,a_h^k,h}-P_{s_h^k,a_h^k,h}\right)V_{h+1}^k\nonumber
\\ & \leq  \sum_{k,h}\mathbb{V}(P_{s_h^k,a_h^k,h}^k  ,V_{h+1}^k) + \sum_{k,h}\left(\hat{P}_{s_h^k,a_h^k,h}-P_{s_h^k,a_h^k,h}\right)(V_{h+1}^k)^2 + 2H\sum_{k,h}\left(\hat{P}_{s_h^k,a_h^k,h}-P_{s_h^k,a_h^k,h}\right)V_{h+1}^k\nonumber
\\ & = T_5 + T_7+ 2HT_1.
\end{align}

Using Lemma~\ref{lemma:key3} to bound $T_7$ and $T_1$, with probability $1-2SAHK\delta$, it holds that
\begin{align}
 \sum_{k,h}\mathbb{V}(\hat{P}_{s_h^k,a_h^k,h}^k, V_{h+1}^k) &\leq \sum_{k,h}\mathbb{V}(P_{s_h^k,a_h^k,h}^k  , V_{h+1}^k) +  6\sqrt{ \sum_{k,h} \mathbb{V}(P_{s_h^k,a_h^k,h},V_{h+1}^k)BSAH^3 }+3BSAH^3\nonumber
 \\ & \leq 5\sum_{k,h}\mathbb{V}(P_{s_h^k,a_h^k,h}^k  , V_{h+1}^k) +8BSAH^3.\label{eq:var5}
\end{align}

\end{proof}

By Lemma~\ref{lemma:empv}, it suffices to bound $T_5 = \sum_{k,h}\mathbb{V}(P_{s_h^k,a_h^k,h},V_{h+1}^k)$.


Because $\mathrm{Var}(X+Y)\leq 2(\mathrm{Var}(X)+\mathrm{Var}(Y))$ for any two random variable $X,Y$ with finite variance, we have that
\begin{align}
\sum_{k,h} \mathbb{V}(P_{s_h^k,a_h^k,h},V_{h+1}^k)  & \leq  2\sum_{k,h} \mathbb{V}(P_{s_h^k,a_h^k,h},V_{h+1}^* ) + 2\sum_{k,h}\mathbb{V}(P_{s_h^k,a_h^k,h},V_{h+1}^k -V_{h+1}^* )\nonumber
\\ & \leq 3   K\mathrm{var}_1 +\sum_{k=1}^K \left( \sum_{h=1}^H \mathbb{V}(P_{s_h^k,a_h^k,h},V_{h+1}^* ) - 3\mathrm{var}_1 \right) + 2\sum_{k,h}\mathbb{V}(P_{s_h^k,a_h^k,h},V_{h+1}^k -V_{h+1}^* ).\label{eq:var0}
\end{align}




\begin{lemma}\label{lemma:k1}
With probability $1-4SAHK\delta$, it holds that
\begin{align}
T_{10} -2K\mathrm{var}_1=\sum_{k=1}^K \left( \sum_{h=1}^H \mathbb{V}(P_{s_h^k,a_h^k,h},V_{h+1}^* ) -2\mathrm{var}_1 \right)  \leq 80H^2\log(\frac{1}{\delta}).
\end{align}
\end{lemma}
\begin{proof}




Let $\bar{R}_{h}^*(s,a) = \mathbb{V}(P_{s,a,h},V_{h+1}^*)$. Define
\begin{align}
\bar{V}^k_h (s) = \mathbb{E}\left[ \sum_{h'=h}^H \bar{R}_{h'}(s_{h'},a_{h'}) |s_h = s\right].\nonumber
\end{align}
Then $\bar{V}_h^k(s)\leq \mathrm{var}_1\leq H^2$.

We then have that
\begin{align}
  \sum_{h=1}^H  \mathbb{V}(P_{s_h^k,a_h^k,h},V_{h+1}^*)-\mathrm{var}_1 \nonumber & =\sum_{h=1}^H  \bar{R}_h^*(s_h^k,a_h^k)-\mathrm{var}_1  \nonumber
 \\ & \leq \sum_{h=1}^H \bar{R}_h^*(s_h^k,a_h^k) - \bar{V}^k_1(s_1^k)\nonumber
 \\ & = \sum_{h=1}^H \left(\textbf{1}_{s_{h+1}^k} - P_{s_{h}^k,a_h^k,h}  \right)\bar{V}^k_{h+1}.
\end{align}

Note that $\bar{V}^k$ only depends on $\pi^k$, which is determined before the $k$-th episode start.  
With Lemma~\ref{lemma:self-norm}, with probability $1-2SAHK\delta$,
\begin{align}
 & \sum_{k=1}^K \left( \sum_{h=1}^H \mathbb{V}(P_{s_h^k,a_h^k,h},V_{h+1}^*) - \bar{V}_1^k(s_1^k)  \right) \nonumber
 \\ & \leq 2\sqrt{2\sum_{k=1}^K \sum_{h=1}^H \mathbb{V}(P_{s_h^k,a_h^k,h},\bar{V}_{h+1}^k)\log(\frac{1}{\delta}) } + 3H^2\log(\frac{1}{\delta}).\label{eq:xlll1}
\end{align}

We further bound
\begin{align}
 & \sum_{k=1}^K \sum_{h=1}^H \mathbb{V}(P_{s_h^k,a_h^k,h},\bar{V}_{h+1}^k) \nonumber
 \\ &  =\sum_{k=1}^K \sum_{h=1}^H \left( P_{s_h^k,a_h^k,h} (\bar{V}_{h+1}^k)^2 - (P_{s_h^k,a_h^k,h}\bar{V}_{h+1}^k)^2 \right)\nonumber
 \\ & = \sum_{k=1}^K \sum_{h=1}^H \left( P_{s_h^k,a_h^k,h} -\textbf{1}_{s_{h+1}^k}  \right)(\bar{V}_{h+1}^k)^2  \nonumber
 \\ & \qquad \qquad + \sum_{k=1}^H \sum_{h=1}^H \left((\bar{V}_{h+1}^k(s_{h+1}^k))^2- (\bar{V}_h^k(s_h^k))^2 \right)+\sum_{k=1}^K \sum_{h=1}^H \left( (\bar{V}_h^k(s_h^k))^2 - (P_{s_h^k,a_h^k,h}\bar{V}_{h+1}^k)^2\right) \nonumber
 \\ & \leq 2\sqrt{8H^4\sum_{k=1}^K \sum_{h=1}^H \mathbb{V}(P_{s_h^k,a_h^k,h},\bar{V}_{h+1}^k)\log(\frac{1}{\delta})   } +2H^2\sum_{k=1}^K\sum_{h=1}^H \bar{R}_h(s_h^k,a_h^k) + 3H^4\log(\frac{1}{\delta}).\label{eq:xllll2}
\end{align}
Here the last inequality is by Lemma~\ref{lemma:self-norm} and Lemma~\ref{lemma:sqv} (with probability $1-2SAHK\delta$) and the fact that $\bar{V}_h^k(s_h^k) = \bar{R}_h(s_h^k,a_h^k)+P_{s_h^k,a_h^k,h}\bar{V}_{h+1}^k$.

It then follows that
\begin{align}
\sum_{k=1}^K \sum_{h=1}^H \mathbb{V}(P_{s_h^k,a_h^k,h},\bar{V}_{h+1}^k)  \leq 4H^2\sum_{k=1}^K \sum_{h=1}^H \bar{R}_h(s_h^k,a_h^k) + 42 H^4\log(\frac{1}{\delta}).\label{eq:xlll3}
\end{align}

By \eqref{eq:xlll1} and \eqref{eq:xlll3}, we learn that
\begin{align}
\sum_{k=1}^K \sum_{h=1}^H \mathbb{V}(P_{s_h^k,a_h^k,h},V_{h+1}^*) \leq \sum_{k=1}^H \bar{V}_1^k(s_1^k) +2\sqrt{8 H^2 \sum_{k=1}^K \sum_{h=1}^H \mathbb{V}(P_{s_h^k,a_h^k,h},V_{h+1}^*) \log(\frac{1}{\delta})     } + 21H^2\log(\frac{1}{\delta}),\nonumber
\end{align}
which further implies that
\begin{align}
\sum_{k=1}^K \sum_{h=1}^H \mathbb{V}(P_{s_h^k,a_h^k,h},V_{h+1}^*) \leq 2\sum_{k=1}^K \bar{V}_1^k(s_1^k)+ 84H^2\log(\frac{1}{\delta}) \leq 2K\mathrm{var}_1 + 84H^2\log(\frac{1}{\delta}).\nonumber
\end{align}

The proof is completed.







\end{proof}


For the left term $\sum_{k,h}\mathbb{V}(P_{s_h^k,a_h^k,h},V_{h+1}^k -V_{h+1}^* )$,   we have the lemma below.
\begin{lemma}\label{lemma:bdv1}
With probability $1-2\delta$, it holds that
\begin{align}
 & \sum_{k,h}\mathbb{V}(P_{s_h^k,a_h^k,h}, V_{h+1}^k -V_{h+1}^*)   \leq 4\sqrt{BH^2\sum_{k,h}\mathbb{V}(P_{s_h^k,a_h^k,h},V_{h+1}^k)}+ 4H\sum_{k,h}b_h^k(s_h^k,a_h^k)+ 3BSAH^3.\nonumber
\end{align}
\end{lemma}
\begin{proof}[Proof of Lemma~\ref{lemma:bdv1}]
Direct computation gives that
\begin{align}
 & \sum_{k,h}\mathbb{V}(P_{s_h^k,a_h^k,h},V_{h+1}^k - V_{h+1}^*)  \nonumber
 \\ & =\sum_{k,h} \left(  P_{s_h^k,a_h^k,h} (V_{h+1}^k - V_{h+1}^*)^2 - (P_{s_h^k,a_h^k,h} (V_{h+1}^k - V_{h+1}^*) )^2       \right)
 \\ &  = \sum_{k,h} \left(    (P_{s_h^k,a_h^k,h}-\textbf{1}_{s_{h+1}^k}) (V_{h+1}^k - V_{h+1}^*)^2   \right) \nonumber
 \\ & \qquad \qquad \qquad \qquad + \sum_{k,h} \left( (V_{h+1}^k(s_{h+1}^k)- V_{h+1}^*(s_{h+1}^k))^2 - ((P_{s_h^k,a_h^k,h} (V_{h+1}^k - V_{h+1}^*) )^2   ) \right)\nonumber
 \\ &  =\sum_{k,h} \left(    (P_{s_h^k,a_h^k,h}-\textbf{1}_{s_{h+1}^k}) (V_{h+1}^k - V_{h+1}^*)^2   \right) + \sum_{k,h} \left( (V_{h}^k(s_{h}^k)- V_{h}^*(s_{h}^k))^2 - ((P_{s_h^k,a_h^k,h} (V_{h+1}^k - V_{h+1}^*) )^2   ) \right).\label{eq:var1}
 \end{align}

%Let $T_{101} = \sum_{k,h} \left(    (P_{s_h^k,a_h^k,h}-\textbf{1}_{s_{h+1}^k}) (V_{h+1}^k - V_{h+1}^*)^2   \right)$ and $T_{102} =\sum_{k,h} \left( (V_{h}^k(s_{h}^k)- V_{h}^*(s_{h}^k))^2 - ((P_{s_h^k,a_h^k,h} (V_{h+1}^k - V_{h+1}^*) )^2   ) \right) $. 
By Lemma~\ref{lemma:self-norm} and Lemma~\ref{lemma:sqv}, with probability $1-\delta$, it holds that
\begin{align}
 \sum_{k,h} \left(    (P_{s_h^k,a_h^k,h}-\textbf{1}_{s_{h+1}^k}) (V_{h+1}^k - V_{h+1}^*)^2   \right)\leq 2\sqrt{2}\sqrt{4H^2\sum_{k,h}\mathbb{V}(P_{s_h^k,a_h^k,h}, V_{h+1}^k - V_{h+1}^* )\log(\frac{1}{\delta})} + 3H^2\log(\frac{1}{\delta}).\label{eq:var3}
\end{align}

On the other hand, with probability $1-\delta$,
 \begin{align}
 & \sum_{k,h} \left( (V_{h}^k(s_{h}^k)- V_{h}^*(s_{h}^k))^2 - ((P_{s_h^k,a_h^k,h} (V_{h+1}^k - V_{h+1}^*) )^2   ) \right)\nonumber
 \\ & \leq  2H\sum_{k,h} \max\{ V_h^k(s_h^k)-P_{s_h^k,a_h^k,h} V_{h+1}^k - (V^*_{h}(s_h^k)-P_{h}^k V_{h+1}^*) ,0  \}  \nonumber
\\ & \leq 2H \sum_{k,h} \max\{V_h^k(s_h^k) - P_{s_h^k,a_h^k,h}V_{h+1}^k-r_h(s_h^k,a_h^k),0\} \nonumber
\\ & \leq 2H\sum_{k,h}\max\{(\hat{P}_{s_h^k,a_h^k,h}-P_{s_h^k,a_h^k,h})V_{h+1}^k ,0\} +2H\sum_{k,h} b_h^k \nonumber
\\ & \leq  2\sqrt{BSAH^3\sum_{k,h}\mathbb{V}(P_{s_h^k,a_h^k,h}, V_{h+1}^k)}+2H\sum_{k,h}b_h^k(s_h^k,a_h^k) + BSAH^3.\label{eq:var4}
 \end{align}



It then follows that, with probability $1-2\delta$,
\begin{align}
 & \sum_{k,h}\mathbb{V}(P_{s_h^k,a_h^k,h}, V_{h+1}^k -V_{h+1}^*)  \leq 4\sqrt{BSAH^3\sum_{k,h}\mathbb{V}(P_{s_h^k,a_h^k,h},V_{h+1}^k)}+ 4H\sum_{k,h}b_h^k(s_h^k,a_h^k)+ 3BSAH^3.
\end{align}


The proof is completed.





\end{proof}


By Lemma~\ref{lemma:k1} and Lemma~\ref{lemma:bdv1}, we have that with probability $1-6SAHK\delta$,
\begin{align}
T_6:=&\sum_{k,h}\mathbb{V}(P_{s_h^k,a_h^k,h},V_{h+1}^k)\nonumber
\\ & \leq  2\sum_{k,h}\mathbb{V}(P_{s_h^k,a_h^k,h},V_{h+1}^*)+2\sum_{k,h}\mathbb{V}(P_{s_h^k,a_h^k,h},V_{h+1}^k - V_{h+1}^*)\nonumber
\\ & \leq 4K\mathrm{var}_1 +  8\sqrt{BSAH^3 T_6}+8HT_2 + 7BSAH^3\nonumber
\\ & \leq 8K\mathrm{var}_1 + 16HT_2 + 78BSAH^3.\label{eq:nbt6}
\end{align}

By Lemma~\ref{lemma:empv} and \eqref{eq:nbt6}, with probability $1-8SAHK\delta$, it holds that
\begin{align}
T_5:=\sum_{k,h}\mathbb{V}(\hat{P}_{s_h^k,a_h^k,h},V_{h+1}^k)\leq 40K\mathrm{var}_1 + 80HT_2+398BSAH^3.\label{eq:nbt5}
\end{align}


\subsubsection{Putting All Together}
We rewrite the inequalities $\eqref{eq:obt1}-\eqref{eq:obt8}$ as follows with \eqref{eq:obt2}, \eqref{eq:obt4}, \eqref{eq:obt5} and \eqref{eq:obt6}  replaced by  \eqref{eq:nbt2}, \eqref{eq:nbt4}
\eqref{eq:nbt5} and \eqref{eq:nbt6}. 
Recall $B =4000 \log_2^3(K)\log(3SA)\log(\frac{1}{\delta})$.

\begin{align}
& T_1 \leq \sqrt{128BSAHT_6}+24BSAH^2;\nonumber
\\ & T_7 \leq H\sqrt{512BSAHT_6}+24BSAH^3;\nonumber
\\ & T_9 \leq \sqrt{128BSAHT_6}+24BSAH^2;\nonumber
\\ & T_2\leq 100 \sqrt{B SAHT_5}+140BSAH^2;\nonumber
 \\ & T_3 \leq \sqrt{8BT_6}+3H\log(\frac{1}{\delta})  ;\nonumber
 \\ & T_4 \leq \sqrt{BSAHT_{10}}+32\sqrt{BH(T_1+T_2+T_3)}+BSAH^2;\nonumber
 \\ & T_5 \leq 40K\mathrm{var}_1 + 80HT_2+398BSAH^3  ;\nonumber
 \\ &  T_6 \leq  8K\mathrm{var}_1 + 16HT_2 + 78BSAH^3 ;\nonumber
 \\ & T_8 \leq \sqrt{32BH^2T_6 } + 3BH^2 .\nonumber
\end{align}
On the other hand, by Lemma~\ref{lemma:k1}, we have 
$$ T_{10} \leq 2K\mathrm{var}_1 + 80BH^2.$$ 
Solving the inequalities above, we obtain that, with probability $1-200SAH^2K^2\delta$, 
\begin{align}
\mathrm{Regret}(K)= T_1+T_2+T_3+T_4 \leq O\left( \sqrt{BSAHK\mathrm{var_1} }+ BSAH^2 \right).\label{eq:rbvar1}
\end{align}

The proof is completed by replacing $\delta$ with $\frac{\delta}{200SAH^2K^2}$.  

\subsection{Proof of Lemma~\ref{lemma:var2}}




Following similar arguments as in the proof of Lemma~\ref{lemma:var1}, 
we focus on bounding $T_2,T_4,T_5$ and $T_6$ in terms of $\mathrm{var}_2$.



\subsubsection{Bounding $T_2$}

Recall that $\delta'$ is defined as $\delta' = \frac{\delta}{200SAH^2K^2}$, 
and that we have demonstrated in \eqref{eq:boundt2o-temp} that
%
\begin{align}
T_2 &\leq \frac{460}{9}\sqrt{2SAH T_5 (\log_2K)\log \frac{1}{\delta'} }\nonumber
\\ & \qquad +4\sqrt{SAH(\log_2K)\log \frac{1}{\delta'}}\sqrt{\sum_{k,h}\left(\widehat{\sigma}_h^k(s_h^k,a_h^k)- \big(\widehat{r}_h^k(s_h^k,a_h^k) \big)^2\right)} 
	+ \frac{1088}{9}SAH^2(\log_2K)\log \frac{1}{\delta'}.\label{eq:local31}
 \end{align}
%
To bound the right-hand side of \eqref{eq:local31}, 
we resort to the following lemma. 
%
 \begin{lemma}\label{lemma:ks2}
With probability at least $1-4SAHK\delta'$, one has
%
\begin{align}
\sum_{k,h}\left(\widehat{\sigma}_h^k(s_h^k,a_h^k)- (\widehat{r}_h^k(s_h^k,a_h^k))^2\right) \leq 6K\mathrm{var}_2 + 242H^2(\log_2K)\log \frac{1}{\delta'}.
\end{align}
%
 \end{lemma}
 %
\begin{proof}
Recall that in Lemma~\ref{lemma:bdrv}, we have shown that with probability at least $1-4SAHK\delta'$,
%
\begin{align}
    \sum_{k=1}^K\sum_{h=1}^H\left(\widehat{\sigma}_h^k(s_h^k,a_h^k)- \big(\widehat{r}_h^k(s_h^k,a_h^k) \big)^2\right)\leq 3\sum_{k=1}^K  \widetilde{V}_1^k(s_1^k)+2SAH^3(\log_2K)\log \frac{1}{\delta'}.
\end{align}
%
We then complete the proof by observing that
%
\begin{align}
	\widetilde{V}_1^k(s_1^k) &\leq  \widetilde{V}_1^k(s_1^k)+\mathbb{E}_{\pi^k}\left[\sum_{h=1}^H \mathbb{V}\big(P_{s_h,a_h,h},V_{h+1}^{\pi^k} \big) \,\Big|\, s_1=s_1^k\right] \notag\\
	&= \mathsf{Var}_{\pi^k}\left[\sum_{h=1}^H r_h(s_h,a_h) \,\Big|\, s_1=s_1^k\right]\leq \mathrm{var}_2.
\end{align}
\end{proof}
 %
Combining Lemma~\ref{lemma:ks2} with \eqref{eq:local31} gives: with probability at least $1-4SAHK\delta'$,
%
\begin{align}
	T_2 &\leq \frac{460}{9}\sqrt{2SAH T_5 (\log_2K)\log \frac{1}{\delta'} } +12\sqrt{SAH(\log_2K)\log \frac{1}{\delta'}}\sqrt{2K\mathrm{var}_2}  \notag\\
	&\qquad\qquad + 157SAH^2(\log_2K)\log \frac{1}{\delta'}.\label{eq:xbt2}
\end{align}




\subsubsection{Bounding $T_4$}

Recall that $T_4 = \widecheck{T}_1 + \widecheck{T}_2$, 
where 
%
\begin{align*}
	\widecheck{T}_1 &= \sum_{k=1}^K\sum_{h=1}^H \left(\widehat{r}_h^k(s_h^k,a_h^k) -r_h(s_h^k,a_h^k) \right),\\
	\widecheck{T}_2 &= \sum_{k=1}^K\left(\sum_{h=1}^H r_h(s_h^k,a_h^k) -V_{1}^{\pi^k}(s_1^k) \right).
\end{align*}
%
Repeating similar arguments employed in the proof of Lemma~\ref{lemma:bdrv} and using \eqref{eq:wc3}, we see that with probability exceeding $1-6SAHK\delta'$, 
%
\begin{align}
	|\widecheck{T}_1| & \leq \sqrt{4SAH (\log_2 K)\log \frac{1}{\delta'}}\cdot \sqrt{\sum_{k=1}^K\sum_{h=1}^H v_h(s_h^k,a_h^k)} + 2SAH^2(\log_2K)\log \frac{1}{\delta'} \nonumber
\\ & \leq \sqrt{8SAHK\mathrm{var}_2(\log_2K)\log \frac{1}{\delta'}}+ 20SAH^2(\log_2K)\log \frac{1}{\delta'}.\nonumber
\end{align}
%
In addition, from Lemma~\ref{lemma:self-norm} and the definition of $\mathrm{var}_2$, we see that 
%
\begin{align}
|\widecheck{T}_2|\leq 2\sqrt{2K\mathrm{var}_2 \log \frac{1}{\delta'}}+3H\log \frac{1}{\delta'} 
\end{align}
%
with probability at least $1-2SAHK\delta'$. 
% 
Therefore, with probability at least $1-8SAHK\delta'$, it holds that
%
\begin{align}
T_4 \leq 4\sqrt{2SAHK\mathrm{var}_2(\log_2K)\log \frac{1}{\delta'}}+ 23SAH^2(\log_2K)\log \frac{1}{\delta'}.\label{eq:xbt4}
\end{align}





\subsubsection{Bounding $T_5$ and $T_6$}




Recall that Lemma~\ref{lemma:empv} asserts that with probability exceeding $1-2\delta'$, 
$$
	T_5\leq 5T_6+8BSAH^3.
$$
%
Hence, it suffices to bound $T_6$.


From the elementary inequality $\mathsf{Var}(X+Y)\leq 2\mathsf{Var}(X)+ 2\mathsf{Var}(Y)$, we obtain
%
\begin{align}
	T_6 &=\sum_{k,h} \mathbb{V}\big(P_{s_h^k,a_h^k,h},V_{h+1}^k\big)   \leq  2\sum_{k,h} \mathbb{V}\big(P_{s_h^k,a_h^k,h},V_{h+1}^{\pi^k} \big) 
	+ 2\sum_{k,h}\mathbb{V}\big(P_{s_h^k,a_h^k,h},V_{h+1}^k -V_{h+1}^{\pi^k}\big )\nonumber
\\ 
	& \leq 3   K\mathrm{var}_2 +\sum_{k=1}^K \left( \sum_{h=1}^H \mathbb{V}\big(P_{s_h^k,a_h^k,h},V_{h+1}^{\pi^k} \big) - 3\mathrm{var}_2 \right) + 2\sum_{k,h}\mathbb{V}\big(P_{s_h^k,a_h^k,h},V_{h+1}^k -V_{h+1}^{\pi^k} \big).\label{eq:var01}
\end{align}
%
To bound the right-hand side of \eqref{eq:var01}, we resort to the following two lemmas.
%
\begin{lemma}\label{lemma:kx1}
With probability  at least $1-4SAHK\delta'$, it holds that
\begin{align}
\sum_{k=1}^K \left( \sum_{h=1}^H \mathbb{V}(P_{s_h^k,a_h^k,h},V_{h+1}^{\pi^k} ) -2\mathrm{var}_2 \right)  \leq 80H^2\log \frac{1}{\delta'}.
\end{align}
\end{lemma}
%
\begin{lemma}\label{lemma:bdv11}
With probability exceeding $1-4SAKH\delta'$, it holds that
%
\begin{align}
 & \sum_{k,h}\mathbb{V}\big(P_{s_h^k,a_h^k,h}, V_{h+1}^k -V_{h+1}^{\pi^k}\big)   \leq 4\sqrt{BH^2\sum_{k,h}\mathbb{V}\big(P_{s_h^k,a_h^k,h},V_{h+1}^k\big)}+ 4H\sum_{k,h}b_h^k(s_h^k,a_h^k)+ 3BSAH^3.\nonumber
\end{align}
%
\end{lemma}
%
With Lemma~\ref{lemma:kx1} and Lemma~\ref{lemma:bdv11} in place, we can demonstrate that with probability at least $1-6SAHK\delta'$,
%
\begin{align}
T_6&
	%=\sum_{k,h}\mathbb{V}\big(P_{s_h^k,a_h^k,h},V_{h+1}^k\big)\nonumber
%\\ & 
	\leq  2\sum_{k,h}\mathbb{V}\big(P_{s_h^k,a_h^k,h},V_{h+1}^{\pi^k}\big)+2\sum_{k,h}\mathbb{V}\big(P_{s_h^k,a_h^k,h},V_{h+1}^k - V_{h+1}^{\pi^k}\big)\nonumber
\\ & \leq 4K\mathrm{var}_2 +  8\sqrt{BSAH^3 T_6}+8HT_2 + 7BSAH^3,\nonumber
%\\ & \leq 8K\mathrm{var}_2 + 16HT_2 + 78BSAH^3.\label{eq:xbt6}
\end{align}
%
\begin{align}
\Longrightarrow \qquad 
	T_6 \leq 8K\mathrm{var}_2 + 16HT_2 + 78BSAH^3.\label{eq:xbt6}
\end{align}
%
Taking this result together with  Lemma~\ref{lemma:empv} gives, with probability exceeding $1-8SAHK\delta'$, 
%
\begin{align}
T_5 =\sum_{k,h}\mathbb{V}\big(\widehat{P}_{s_h^k,a_h^k,h},V_{h+1}^k\big)\leq 40K\mathrm{var}_2 + 80HT_2+398BSAH^3.\label{eq:xbt5}
\end{align}
%
To finish establishing the above bounds on $T_5$ and $T_6$, it suffices to prove Lemma~\ref{lemma:kx1} and Lemma~\ref{lemma:bdv11}, 
which we accomplish in the sequel. 


%\yxc{???} We can then derive
%%
%\begin{align}
%\sum_{k,h}\mathbb{V}(P_{s_h^k,a_h^k,h}, V_{h+1}^k) &  \leq 2 \sum_{k,h}\mathbb{V}(P_{s_h^k,a_h^k,h},V_{h+1}^{\pi^k}) + 2\sum_{k,h}\mathbb{V}(P_{s_h^k,a_h^k,h}, V_{h+1}^k -V_{h+1}^{\pi^k} )\nonumber
%\\ & \leq 6\sum_{k=1}^K\widecheck{\mathrm{var}}^k + \sum_{k=1}^K\left( \sum_{h=1}^H \mathbb{V}(P_{s_h^k,a_h^k,h},V_{h+1}^{\pi^k}) - 3\widecheck{\mathrm{var}}^k \right) + 2\sum_{k,h}\mathbb{V}(P_{s_h^k,a_h^k,h}, V_{h+1}^k -V_{h+1}^{\pi^k} )\nonumber
%\\ & \leq 6K\mathrm{var}_2 + \sum_{k=1}^K\left( \sum_{h=1}^H \mathbb{V}(P_{s_h^k,a_h^k,h},V_{h+1}^{\pi^k}) - 3\widecheck{\mathrm{var}}^k \right) + 2\sum_{k,h}\mathbb{V}(P_{s_h^k,a_h^k,h}, V_{h+1}^k -V_{h+1}^{\pi^k} ).\label{eq:var6}
%\end{align}
%
%
%
%By Lemma~\ref{lemma:kx1}, Lemma~\ref{lemma:empv} and Lemma~\ref{lemma:bdv11}, we know that with probability at least $1-18SAHK\delta'$, 
%\begin{align}
% & T_5 \leq O\left( K\mathrm{var}_2 + H\sqrt{T_6\left(SAH+\log\frac{1}{\delta'}\right)} +T_2+H^2\left(SAH+\log \frac{1}{\delta'}\right)    \right) ;\label{eq:nbbt5}
%\\ & T_6 \leq O\left( K\mathrm{var}_2 + H\sqrt{T_6\left(SAH+\log\frac{1}{\delta'}\right)} +T_2+H^2\left(SAH+\log \frac{1}{\delta'}\right)    \right) .\label{eq:nbbt6}
%\end{align}



\begin{proof}[Proof of Lemma~\ref{lemma:kx1}]
%
For notational convenience, define 
	%$\widecheck{R}^k_{h}(s,a) = \mathbb{V}(P_{s,a,h},V_{h+1}^{\pi^k})$. Define
\begin{align}
	\widecheck{R}^k_{h}(s,a) = \mathbb{V}(P_{s,a,h},V_{h+1}^{\pi^k}) \qquad \text{and} \qquad
\widecheck{V}^k_h (s) = \mathbb{E}\left[ \sum_{h'=h}^H \widecheck{R}^k_{h'}(s_{h'},a_{h'}) \,\Big|\, s_h = s\right].\nonumber
\end{align}
%
It is easily seen that $\widecheck{V}_h^k(s)\leq\mathrm{var}_2 \leq  H^2$. 


We also make the observation that 
%
\begin{align}
  \sum_{h=1}^H  \mathbb{V}(P_{s_h^k,a_h^k,h},V_{h+1}^{\pi^k})-\mathrm{var}_2 \nonumber & =\sum_{h=1}^H  \widecheck{R}_h^k(s_h^k,a_h^k)-\mathrm{var}_2  \nonumber
 \\ & \leq \sum_{h=1}^H \widecheck{R}_h^k(s_h^k,a_h^k) - \widecheck{V}^k_1(s_1^k)\nonumber
 \\ & = \sum_{h=1}^H \big\langle e_{s_{h+1}^k} - P_{s_{h}^k,a_h^k,h},\,\widecheck{V}^k_{h+1}   \big\rangle.
\end{align}
%
Note that $\widecheck{V}^k$ only depends on $\pi^k$, which is determined before the $k$-th episode starts.  
 Lemma~\ref{lemma:self-norm} then tells us that, with probability at least $1-2SAHK\delta'$,
%
\begin{align}
 & \sum_{k=1}^K \left( \sum_{h=1}^H \mathbb{V}\big(P_{s_h^k,a_h^k,h},V_{h+1}^{\pi^k}\big) - \widecheck{V}_1^k(s_1^k)  \right) \nonumber
 \\ & \leq 2\sqrt{2\sum_{k=1}^K \sum_{h=1}^H \mathbb{V}\big(P_{s_h^k,a_h^k,h},\widecheck{V}_{h+1}^k\big)\log \frac{1}{\delta'} } + 3H^2\log \frac{1}{\delta'}.\label{eq:xlll11}
\end{align}
%
Further, it is observed that with probability at least $1-2SAHK\delta'$, 
%
\begin{align}
 & \sum_{k=1}^{K}\sum_{h=1}^{H}\mathbb{V}\big(P_{s_{h}^{k},a_{h}^{k},h},\widecheck{V}_{h+1}^{k}\big)\nonumber\\
 & =\sum_{k=1}^{K}\sum_{h=1}^{H}\left(\big\langle P_{s_{h}^{k},a_{h}^{k},h},(\widecheck{V}_{h+1}^{k})^{2}\big\rangle-\big(\big\langle P_{s_{h}^{k},a_{h}^{k},h},\widecheck{V}_{h+1}^{k}\big\rangle\big)^{2}\right)\nonumber\\
 & =\sum_{k=1}^{K}\sum_{h=1}^{H}\big\langle P_{s_{h}^{k},a_{h}^{k},h}-e_{s_{h+1}^{k}},\,(\widecheck{V}_{h+1}^{k})^{2}\big\rangle\nonumber\\
 & \qquad+\sum_{k=1}^{H}\sum_{h=1}^{H}\left( \big(\widecheck{V}_{h+1}^{k}(s_{h+1}^{k}) \big)^{2}- \big(\widecheck{V}_{h}^{k}(s_{h}^{k})\big)^{2}\right)
	+\sum_{k=1}^{K}\sum_{h=1}^{H}\left(\big(\widecheck{V}_{h}^{k}(s_{h}^{k})\big)^{2}-\left(\big\langle P_{s_{h}^{k},a_{h}^{k},h},\widecheck{V}_{h+1}^{k}\big\rangle\right)^{2}\right)\nonumber\\
 & \leq2\sqrt{8H^{4}\sum_{k=1}^{K}\sum_{h=1}^{H}\mathbb{V}\big(P_{s_{h}^{k},a_{h}^{k},h},\widecheck{V}_{h+1}^{k}\big)\log\frac{1}{\delta'}}+2H^{2}\sum_{k=1}^{K}\sum_{h=1}^{H}\widecheck{R}_{h}(s_{h}^{k},a_{h}^{k})+3H^{4}\log\frac{1}{\delta'}.
	\label{eq:xllll21}
\end{align}
%
Here, the last inequality results from Lemma~\ref{lemma:self-norm}, Lemma~\ref{lemma:sqv}  and the fact that 
	$\widecheck{V}_h^k(s_h^k) = \widecheck{R}_h(s_h^k,a_h^k)+ \langle P_{s_h^k,a_h^k,h}, \widecheck{V}_{h+1}^k \rangle$. 
% 
It then follows that
\begin{align}
\sum_{k=1}^K \sum_{h=1}^H \mathbb{V}\big(P_{s_h^k,a_h^k,h},\widecheck{V}_{h+1}^k \big)  \leq 4H^2\sum_{k=1}^K \sum_{h=1}^H \widecheck{R}_h(s_h^k,a_h^k) + 42 H^4\log \frac{1}{\delta'}.\label{eq:xlll31}
\end{align}
%
Taking \eqref{eq:xlll11} and \eqref{eq:xlll31} together leads to
%
\begin{align}
\sum_{k=1}^K \sum_{h=1}^H \mathbb{V}\big(P_{s_h^k,a_h^k,h},V_{h+1}^{\pi^k}\big) 
	\leq \sum_{k=1}^H \widecheck{V}_1^k(s_1^k) +2\sqrt{8 H^2 \sum_{k=1}^K \sum_{h=1}^H \mathbb{V}\big(P_{s_h^k,a_h^k,h},V_{h+1}^{\pi^k}\big) \log \frac{1}{\delta'}    } + 21H^2\log \frac{1}{\delta'},\nonumber
\end{align}
%
which further implies that
%
\begin{align}
	\sum_{k=1}^K \sum_{h=1}^H \mathbb{V}\big(P_{s_h^k,a_h^k,h},V_{h+1}^{\pi^k}\big) 
	\leq 2\sum_{k=1}^K \widecheck{V}_1^k(s_1^k)+ 84H^2\log \frac{1}{\delta'} 
	\leq 2K\mathrm{var}_2 + 84H^2\log \frac{1}{\delta'}.\nonumber
\end{align}
%
This concludes the proof. 
\end{proof}


%For the left term $\sum_{k,h}\mathbb{V}(P_{s_h^k,a_h^k,h},V_{h+1}^k -V_{h+1}^{\pi^k})$,   we have the lemma below.
%
%
\begin{proof}[Proof of Lemma~\ref{lemma:bdv1}]
A little algebra gives
%
\begin{align}
 & \sum_{k,h}\mathbb{V}\big(P_{s_{h}^{k},a_{h}^{k},h},V_{h+1}^{k}-V_{h+1}^{\pi^{k}}\big)\nonumber\\
 & =\sum_{k,h}\left(\big\langle P_{s_{h}^{k},a_{h}^{k},h},\,(V_{h+1}^{k}-V_{h+1}^{\pi^{k}})^{2}\big\rangle-\big(\big\langle P_{s_{h}^{k},a_{h}^{k},h},V_{h+1}^{k}-V_{h+1}^{\pi^{k}}\big\rangle\big)^{2}\right)\notag\\
 & =\sum_{k,h}\left(\big\langle P_{s_{h}^{k},a_{h}^{k},h}-e_{s_{h+1}^{k}},(V_{h+1}^{k}-V_{h+1}^{\pi^{k}})^{2}\big\rangle\right)\nonumber\\
 & \qquad\qquad+\sum_{k,h}\left(\big(V_{h+1}^{k}(s_{h+1}^{k})-V_{h+1}^{\pi^{k}}(s_{h+1}^{k})\big)^{2}-\big(\big\langle P_{s_{h}^{k},a_{h}^{k},h},V_{h+1}^{k}-V_{h+1}^{\pi^{k}}\big\rangle\big)^{2}\right)\nonumber\\
 & =\sum_{k,h}\left(\big\langle P_{s_{h}^{k},a_{h}^{k},h}-e_{s_{h+1}^{k}},(V_{h+1}^{k}-V_{h+1}^{\pi^{k}})^{2}\big\rangle\right)+\sum_{k,h}\left(\big(V_{h}^{k}(s_{h}^{k})-V_{h}^{\pi^{k}}(s_{h}^{k})\big)^{2}-\big(\big\langle P_{s_{h}^{k},a_{h}^{k},h},V_{h+1}^{k}-V_{h+1}^{\pi^{k}}\big\rangle\big)^{2}\right).
	\label{eq:var11}
 \end{align}
%
From Lemma~\ref{lemma:self-norm} and Lemma~\ref{lemma:sqv}, we can show that with probability $1-2SAKH\delta'$, 
%
\begin{align}
 & \sum_{k,h}\big\langle P_{s_{h}^{k},a_{h}^{k},h}-e_{s_{h+1}^{k}},(V_{h+1}^{k}-V_{h+1}^{\pi^{k}})^{2}\big\rangle\\
 & \qquad\leq2\sqrt{2}\sqrt{4H^{2}\sum_{k,h}\mathbb{V}\big(P_{s_{h}^{k},a_{h}^{k},h},V_{h+1}^{k}-V_{h+1}^{\pi^{k}}\big)\log\frac{1}{\delta'}}+3H^{2}\log\frac{1}{\delta'}. 
	\label{eq:var31}
\end{align}
%
Additionally, with probability at least $1-2SAKH\delta'$, 
%
 \begin{align}
  & \sum_{k,h}\left\{ \big(V_{h}^{k}(s_{h}^{k})-V_{h}^{\pi^{k}}(s_{h}^{k})\big)^{2}-\big(\big\langle P_{s_{h}^{k},a_{h}^{k},h},V_{h+1}^{k}-V_{h+1}^{\pi^{k}}\big\rangle)^{2}\right\} \nonumber\\
 & \leq2H\sum_{k,h}\max\Big\{ V_{h}^{k}(s_{h}^{k})-\big\langle P_{s_{h}^{k},a_{h}^{k},h},V_{h+1}^{k}\big\rangle-\big(V_{h}^{\pi^{k}}(s_{h}^{k})-\big\langle P_{h}^{k},V_{h+1}^{\pi^{k}}\big\rangle\big),0\Big\}\nonumber\\
 & =2H\sum_{k,h}\max\Big\{ V_{h}^{k}(s_{h}^{k})-\big\langle P_{s_{h}^{k},a_{h}^{k},h},V_{h+1}^{k}\big\rangle-r_{h}(s_{h}^{k},a_{h}^{k}),0\Big\}\nonumber\\
 & \leq2H\sum_{k,h}\max\Big\{\big\langle\widehat{P}_{s_{h}^{k},a_{h}^{k},h}-P_{s_{h}^{k},a_{h}^{k},h},V_{h+1}^{k}\big\rangle,0\Big\}+2H\sum_{k,h}b_{h}^{k}(s_{h}^{k},a_{h}^{k})\nonumber\\
 & \leq2\sqrt{BSAH^{3}\sum_{k,h}\mathbb{V}\big(P_{s_{h}^{k},a_{h}^{k},h},V_{h+1}^{k}\big)}+2H\sum_{k,h}b_{h}^{k}(s_{h}^{k},a_{h}^{k})+BSAH^{3}.
	 \label{eq:var41}
 \end{align}
%
It then follows that
%
\begin{align}
 & \sum_{k,h}\mathbb{V}(P_{s_h^k,a_h^k,h}, V_{h+1}^k -V_{h+1}^{\pi^k})  \leq 4\sqrt{BSAH^3\sum_{k,h}\mathbb{V}(P_{s_h^k,a_h^k,h},V_{h+1}^k)}+ 4H\sum_{k,h}b_h^k(s_h^k,a_h^k)+ 3BSAH^3
\end{align}
%
with probability at least $1-4SAKH\delta'$. 
The proof is thus complete.
%
\end{proof}



\iffalse
\begin{lemma}\label{lemma:l2}
With probability at least $1-4SAKH\delta$,
\begin{align}
\sum_{k,h} \mathbb{V}(P_{s_h^k,a_h^k,h}, V_{h+1}^k-V_{h+1}^{\pi^k}) \leq O\left(   H\sqrt{\sum_{k,h}\mathbb{V}(P_{s_h^k,a_h^k,h},V_{h+1}^k)\left(SAH+\log(\frac{1}{\delta'}\right))} +\sum_{k,h}b_h^k(s_h^k,a_h^k)+ H^2\left(SAH+\log \frac{1}{\delta'} \right)\right).
\end{align}
\end{lemma}
%
\begin{proof}[Proof of Lemma~\ref{lemma:l2}] With Lemma~\ref{lemma:self-norm}, with probability exceeding $1-2SAHK\delta$
\begin{align}
 & \sum_{k,h}\mathbb{V}(P_{s_h^k,a_h^k,h}, V^k_{h+1}- V_{h+1}^{\pi^k}) \nonumber
\\ &= \sum_{k,h} \left(   P_{s_h^k,a_h^k,h} (V_{h+1}^k - V_{h+1}^{\pi^k})^2 - (P_{s_h^k,a_h^k,h} (V_{h+1}^k-V_{h+1}^{\pi^k}))^2      \right)\nonumber
\\ &  = \sum_{k,h}\left( (P_{s_h^k,a_h^k,h}-e_{s_{h+1}^k})(V_{h+1}^k -V_{h+1}^{\pi^k})^2  \right) +\sum_{k,h} \left(  (V_{h+1}^k(s_{h+1}^k) - V_{h+1}^{\pi^k}(s_{h+1}^k))^2  - (P_{s_h^k,a_h^k,h}(V_{h+1}^k -V_{h+1}^{\pi^k}))^2  \right)\nonumber
\\ & \leq  \sum_{k,h}\left( (P_{s_h^k,a_h^k,h}-e_{s_{h+1}^k})(V_{h+1}^k -V_{h+1}^{\pi^k})^2  \right) +\sum_{k,h} \left(  (V_{h}^k(s_{h}^k) - V_{h}^{\pi^k}(s_{h}^k))^2  - (P_{s_h^k,a_h^k,h}(V_{h+1}^k -V_{h+1}^{\pi^k}))^2  \right)\nonumber
\\ & \leq O\left( H\sqrt{\sum_{k,h}\mathbb{V}(P_{s_h^k,a_h^k,h},V_{h+1}^k -V_{h+1}^{\pi^k})  \log(\frac{1}{\delta'})}  +H^2\log(\frac{1}{\delta'}) \right) \nonumber
\\ & \quad \quad \qquad \qquad +2H\sum_{k,h} \max\{     (V_{h}^k(s_h^k)-P_{s_h^k,a_h^k,h} V_{h+1}^k) - (V_{h}^{\pi^k}(s_h^k)-P_{s_h^k,a_h^k,h}V_{h+1}^{\pi^k})   ,0\}.\label{eq:v81}
\end{align}

By the definition of $V_{h}^k$ and $V_{h}^{\pi^k}$, we have
\begin{align}
& V_h^k(s_h^k)\leq r_h(s_h^k,a_h^k) + \widehat{P}^k_{s_h^k,a_h^k,h} + b_h^k(s_h^k,a_h^k)\nonumber
\\ & V_{h}^{\pi^k}(s_h^k) = r_h(s_h^k,a_h^k) +P_{s_h^k,a_h^k,h}V_{h+1}^{\pi^k}.\nonumber
\end{align}
It then holds that
\begin{align}
&\max\{     (V_{h}^k(s_h^k)-P_{s_h^k,a_h^k,h} V_{h+1}^k) - (V_{h}^{\pi^k}(s_h^k)-P_{s_h^k,a_h^k,h}V_{h+1}^{\pi^k})   ,0\} \nonumber
\\ & \leq \max\{(\widehat{P}_{s_h^k,a_h^k,h}-P_{s_h^k,a_h^k,h})V_{h+1}^{k},0 \} +\sum_{k,h}b_h^k(s_h^k,a_h^k).\label{eq:v7}
\end{align}

By Lemma~\ref{lemma:key3}, we further have that, with probability $1-2SAKH\delta$,
\begin{align}
\sum_{k,h}\max\{(\widehat{P}_{s_h^k,a_h^k,h}-P_{s_h^k,a_h^k,h})V_{h+1}^{k},0 \} \leq O\left( \sqrt{\sum_{k,h}\mathbb{V}(P_{s_h^k,a_h^k,h}, V_{h+1}^{k})\left(SAH+\log(\frac{1}{\delta'}) \right) }+H\left(SAH+\log(\frac{1}{\delta'})\right)    \right).\label{eq:v8}
\end{align}

By \eqref{eq:v81}, \eqref{eq:v7} and \eqref{eq:v8}, we obtain 
\begin{align}
&\sum_{k,h}\mathbb{V}(P_{s_h^k,a_h^k,h},V_{h+1}^k - V_{h+1}^{\pi^k})\nonumber
\\ & \leq O\left(H \sqrt{\sum_{k,h}\mathbb{V}(P_{s_h^k,a_h^k,h},V_{h+1}^k - V_{h+1}^{\pi^k})\log(\frac{1}{\delta'})}+ H\sqrt{\sum_{k,h}\mathbb{V}(P_{s_h^k,a_h^k,h},V_{h+1}^k)\left(SAH+\log(\frac{1}{\delta'}\right)} \right)\nonumber
\\ & \qquad \qquad \qquad \qquad \qquad \qquad \qquad \qquad \qquad \qquad\qquad \qquad+ O\left( \sum_{k,h}b_h^k(s_h^k,a_h^k)+ H^2\left(SAH+\log(\frac{1}{\delta'})\right) \right)\nonumber
\end{align}
holds with probability at least $1-4SAKH\delta$, which further implies that
\begin{align}
\sum_{k,h}\mathbb{V}(P_{s_h^k,a_h^k,h},V_{h+1}^k - V_{h+1}^{\pi^k}) \leq O\left(   H\sqrt{\sum_{k,h}\mathbb{V}(P_{s_h^k,a_h^k,h},V_{h+1}^k)\left(SAH+\log(\frac{1}{\delta'}\right))} +\sum_{k,h}b_h^k(s_h^k,a_h^k)+ H^2\left(SAH+\log(\frac{1}{\delta'})\right)\right) .
\end{align}
This completes the proof.
\end{proof}

\fi 









\subsubsection{Putting all pieces together}

Recall that $B =4000 (\log_2 K)^3 \log(3SA)\log \frac{1}{\delta'}$. 
The last step is to rewrite the inequalities $\eqref{eq:obt1}-\eqref{eq:obt8}$ as follows with \eqref{eq:obt2}, \eqref{eq:obt4}, \eqref{eq:obt5} and \eqref{eq:obt6}  replaced by  \eqref{eq:xbt2},\eqref{eq:xbt4}
\eqref{eq:xbt5} and \eqref{eq:xbt6} respectively:
%
\begin{align}
& T_1 \leq \sqrt{128BSAHT_6}+24BSAH^2,\nonumber
\\ & T_7 \leq H\sqrt{512BSAHT_6}+24BSAH^3,\nonumber
\\ & T_9 \leq \sqrt{128BSAHT_6}+24BSAH^2,\nonumber
\\ & T_2\leq 100 \sqrt{BSAHT_5}+140BSAH^2,\nonumber
 \\ & T_3 \leq \sqrt{8BT_6}+3H\log \frac{1}{\delta'} ,\nonumber
 \\ & T_4 \leq \sqrt{BSAHK\mathrm{var}_2}+BSAH^2;\nonumber
 \\ & T_5 \leq 40K\mathrm{var}_2 + 80HT_2+398BSAH^3  ,\nonumber
 \\ &  T_6 \leq  8K\mathrm{var}_2 + 16HT_2 + 78BSAH^3 ,\nonumber
 \\ & T_8 \leq \sqrt{32BH^2T_6 } + 3BH^2 ,\nonumber
\end{align}
%
which are valid with probability at least $1-200SAH^2K^2\delta'$. 
Solving the inequalities listed above, we can readily conclude that
%
\begin{align}
\mathsf{Regret}(K)= T_1+T_2+T_3+T_4 \leq O\left( \sqrt{BSAHK\mathrm{var_2} }+ BSAH^2 \right).\label{eq:rbvar2}
\end{align}
%
This finishes the proof by recalling that $\delta' = \frac{\delta}{200SAH^2K^2}$.



\section{Minimax lower bounds}\label{app:lb}



In this section, we establish the lower bounds advertised in this paper. 

\subsection{Proof of Theorem~\ref{thm:lb1}}\label{app:lbf}


Consider any given $(S,A,H)$.   We start by establishing the following lemma. 
%
\begin{lemma}\label{lemma:lb2}
Consider any $K'\geq 1$. 
For any algorithm, there exists an MDP instance with $S$ states, $A$ actions, and horizon $H$, such that the regret in $K'$ episodes is at least 
%
\begin{align}
\mathsf{Regret}(K')=\Omega\big( f(K')\big) = \Omega\left(\min\big\{\sqrt{SAH^3K'},K'H\big\}\right).\nonumber
\end{align}
%
\end{lemma}
%
\begin{proof}[Proof of Lemma~\ref{lemma:lb2}]
	Our construction of the hard instance is based on the hard instance JAO-MDP constructed in  \citet{jaksch2010near,jin2018q}. In  \citet[Appendix.D]{jin2018q}, the authors already showed that when $K'\geq C_0SAH$ for some constant $C_0>0$, the minimax regret lower bound is $\Omega(\sqrt{SAH^3K'})$. 
	Hence, it suffices for us to focus on the regime where $K'\leq C_0SAH$. Without loss of generality, we assume $S=A=2$, and the argument to generalize it to arbitrary $(S,A)$ is standard and henc omitted for brevity.  
	

	Recall the construction of JAO-MDP in \citet{jaksch2010near}. Let the two states be $x$ and $y$, and the two actions be $a$ and $b$. 
	The reward  is always equal to $x$ in state $1$ and $1/2$ in state $y$. The probability transition kernel is given by 
	$$
		P_{x,a} =P_{x,b}= [1-\delta,\delta], 
		~~~P_{y,a} = [1-\delta,\delta], ~~~
		P_{y,b}= [1-\delta -\epsilon,\delta+\epsilon],
	$$
	%
	where we choose $\delta = C_1 / H$ and $\epsilon =1/H$. Then the mixing time of the MDP is roughly $O(H)$. By choosing $C_1$ large enough, 
	we can ensure that the MDP is $C_3$-mixing after the first half of the horizons for some proper constant $C_3\in (0,1/2)$.

It is then easy to show that action $b$ is the optimal action for state $y$. 
	Moreover, whenever action $a$ is chosen in state $y$, the learner needs to pay regret $\Omega(\epsilon H)=\Omega(1)$. 
	In addition, to differentiate action $a$ from action $b$ in state $y$ with probability at least $1-\frac{1}{10}$, the learner needs at least $\Omega(\frac{\epsilon}{\delta^2}) = \Omega(H)$ rounds --- let us call it $C_4H$ rounds for some proper constant $C_4>0$. As a result, in the case where $K'\leq C_4H$, the minimax regret is at least $\Omega(K'H^2\epsilon)=\Omega(K'H)$. When $C_4H \leq K' \leq C_0SAH = 4C_0H$, the minimax regret is at least $\Omega(C_4H^2)=\Omega(K'H)$. This concludes the proof.
%
\end{proof}



%Fix the algorithm $\mathcal{G}$. 


With Lemma~\ref{lemma:lb2}, we are ready to prove Theorem~\ref{thm:lb1}. 
Let $\mathcal{M}$ be the hard instance  for $K' = \max\left\{\frac{1}{10}Kp ,1\right\}$ constructed in the proof of Lemma~\ref{lemma:lb2}. 
We construct an MDP $\mathcal{M}'$ as below. 
%
\begin{itemize}
	\item	In the first step, for any state $s$, with probability $p$, the leaner transitions to a copy of $\mathcal{M}$, and with probability $1-p$, the learner transitions to a dumb state with $0$ reward. 
\end{itemize}
%
It can be easily verified that $v^{\star}\leq pH$.
Let $X=X_1+X_2+\dots+X_k$, where $\{X_i\}_{i=1}^K$ are i.i.d.~Bernoulli random variables with mean $p$. Let $g(X,K')$ denote the minimax regret on the hard instance $\mathcal{M}$ in $X$ episodes. Given that $g(X,K')$ is non-decreasing in $X$,  
one sees that $$\mathsf{Regret}(K) \geq  \mathbb{E}\big[g(X,K') \big].$$ 
%
\begin{itemize}
	\item
In the case where $Kp\geq 10$,  Lemma~\ref{lemma:con} tells us that with probability at least $1/2$, $X\geq \frac{1}{10}Kp = K'$, and then it holds that 
		$$
			\mathbb{E}\big[g(X,K')\big] \geq \frac{1}{2} g(K',K')=\frac{1}{2}f(K')= \frac{1}{2} \Omega\left(\min\left\{\sqrt{SAH^3K'},K'H\right\}\right) = \Omega(\sqrt{SAH^3Kp},KHp).
		$$
	\item 
		In the case where $Kp<10$, with probability exceeding $1-(1-p)^K \geq (1-e^{-Kp})\geq  \frac{Kp}{30}$, one has $X\geq 1$. Then one has 
		$$
			\mathbb{E}\big[g(X,K')\big]\geq \frac{Kp}{30}\cdot g(1,K')=\frac{Kp}{30}\cdot g(1,1) =\Omega(KHp). 
		$$
%
\end{itemize}
%
The preceding bounds taken together complete the proof. 




\subsection{Proof of Theorem~\ref{corollary:costlb}}\label{app:lbc}

Without loss of generality, assume that $S=A=2$ (as in the proof of Theorem~\ref{thm:lb1}). Note that $p\leq 1/4$. 
We would like to construct a hard instance for which the learner needs to identify the correct action for each step. 
Let $\mathcal{S}=\{s_1,s_2\}$, and take the initial state to be $s_1$. The transition kernel and cost are chosen as follows. 
%
\begin{itemize}
	\item For any action $a$ and $h$,  set $P_{s_2,a,h} = e_{s_2}$ and $c_h(s_2,a)=0$. 
	\item For any action $a\neq a^{\star}$ and $h$,  set $P_{s_1,a,h} = e_{s_2}$ and $c_h(s_2,a)=1$. 
	\item Set $P_{s_1,a^{\star},h} = e_{s_1}$ and $c_h(s_1,a^{\star}) = p$. 
%
\end{itemize}
%
% Let the initial state be $s_1$.
It can be easily checked that $c^{\star} =Hp$ by choosing $a^{\star}$ for each step. To identify the correct action $a^{\star}$ for at least half of the $H$ steps, we need $\Omega(H)$ episodes, which implies that, there exists a constant $C_5>0$ such that in the first $K\leq C_5H$ episodes, the cost of the learner is at least $\frac{H(1-p)}{2}$. Then the minimax regret is at least 
$$
	\Omega\big(K(H-c^{\star}) \big)=\Omega\big(KH^2(1-p) \big)
$$ 
%
when $K\leq C_5H$. 
In the case where $C_5H \leq K \leq  \frac{100H}{p}$, the minimax regret is at least $$ \Omega\big(H(H-c^{\star})\big)= \Omega\big(H^2(1-p)\big).$$ 

For $K\geq \frac{100H}{p}$, we let $\mathcal{M}$ be the  hard instance with the same transition as that in the proof of Lemma~\ref{lemma:lb2}, and set the cost as $1'2$ for state $x$ and $1$ for state $y$ with respect to $K' = Kp/10 \geq 10H$.  
Let $\mathcal{M}'$ be the MDP such that: in the first step,  with probability $p$, the learner transitions to a copy of $\mathcal{M}$, and with probability $1-p$, the learner transitions to a dumb state with $0$ cost. Then we have $c^{\star} =\Theta(Hp)$. 
It follows from Lemma~\ref{lemma:con} that, with probability exceeding $1/2$, one has $ X\geq \frac{1}{3}Kp - \log 2 \geq \frac{1}{6}Kp$. Then one has 
%
$$\mathsf{Regret}(K)  \geq 
\frac{1}{2} \Omega\left(\min\big\{\sqrt{H^3K'}, K'H \big\}\right)= \Omega\big(\sqrt{H^3Kp}\big).$$ 
%
The proof is thus completed by combining the above minimax regret lower bounds for the three regimes $K\in [1,C_5H]$, $K\in(C_5H,\frac{100H}{p}]$ and $K\in(\frac{100H}{p},\infty]$.





%Mention the lower bound of $\Omega(KH)$ when $K\leq BSAH$ ($K\leq \frac{SAH^2}{v^{\star}}, \frac{SAH^3}{\mathrm{var}_1}, \frac{SAH^3}{\mathrm{var}_2}$)





\subsection{Proof of Theorem~\ref{thm:lb3}}


When $K\geq SAH/p$, the lower bound in Theorem~\ref{thm:lb1} readily applies because the regret is at least $\Omega(\sqrt{SAH^3Kp})$ and the variance $\mathrm{var}$ is at most $pH^2$. 
When $SAH\leq K \leq SAH/p$, the regret is at least $\Omega(SAH^2)=\Omega(\min\{\sqrt{SAH^3Kp}+SAH^2,KH \})$. 
As a result, it suffices to focus on the case where  $1\leq K\leq SAH$, 
Towards this end, we only need the following lemma, which suffices for us to complete the proof.  


% by Lemma~\ref{lemma:lb34}, the minimax regret is at least $\Omega(KH)$.


% For $SAH\leq K \leq SAH/p$, the regret is at least $\Omega(SAH^2)=\Omega(\min\{\sqrt{SAH^3Kp}+SAH^2,KH \})$. The proof is completed.


\begin{lemma}\label{lemma:lb34}
Consider any $1\leq K\leq SAH$. There exists  an MDP instance with $S$ states, $A$ actions, horizon $H$, and $\mathrm{var}_1 = \mathrm{var}_2 = 0$, such that the regret is at least $\Omega(KH)$.
\end{lemma}

\begin{proof}
Let us construct an MDP with deterministic transition; more precisely, for each $(s,a,h)$, there is some $s'$ such that $P_{s,a,h,s'}=1$ and $P_{s,a,h,s''}=0$ for any $s''\neq s'$. 
	The reward function is also chosen to be deterministic. In this case, it is easy to verify that $\mathrm{var}_1 = \mathrm{var}_2 = 0$. 


	We first assume $S=2$. For any action $a$ and horizon $h$, we set $P_{s_2,a,h} = e_{s_2}$ and $r_h(s_2,a)=0$. For any action $a\neq a^{\star}$ and $h$, we also set $P_{s_1,a,h} = e_{s_2}$ and $r_h(s_2,a)=0$. At last, we set $P_{s_1,a^{\star},h} = e_{s_1}$ and $r_h(s_1,a^{\star}) = 1$. In other words, there are a dumb state and a normal state in each step. The learner would naturally hope to find the correct action to avoid the dumb state. Obviously, $V_1^{\star}(s_1)=H$. To find an $\frac{H}{2}$-optimal policy, the learner needs to identify $a^{\star}$ for the first $\frac{H}{2}$ steps, requiring at least $\Omega(HA)$ rounds in expectation. As a result, the minimax regret is at least $\Omega(KH)$ when $K\leq cHA$ for some proper constant $c>0$.  


Let us refer to the hard instance above as a \emph{hard chain}.
	For general $S$, we can construct $d \coloneqq \frac{S}{2}$ hard chains. Let the two states in the $i$-th hard chain be $(s_1(i),s_2(i) )$. We set the initial distribution to be the uniform distribution over $\{s_1(i)\}_{i=1}^d$. Then $V_1^{\star}(s_1(i))=H$ holds for any $1\leq i\leq d$.  Let $\mathsf{Regret}_i(K)$ be the expected regret resulting from the $i$-th hard chain. 
When $K\geq 100S$,  Lemma~\ref{lemma:con} tells us that with probability at least $\frac{1}{2}$, $s_1(i)$ is visited for at least $\frac{K}{10S}\geq 10$ times. As a result, we have $$\mathsf{Regret}_i(K)\geq \frac{1}{2}\cdot \Omega\left(\frac{KH}{S}\right).$$  Summing over $i$, we see that the total regret is at least $\sum_{i=1}^d \mathsf{Regret}_i(K) = \Omega(KH)$. When $K<100S$, with probability at least $1-(1-\frac{1}{S})^K\geq 0.0001\frac{K}{S}$, we know that $s_1(i)$ is visited for at least one time. Therefore, it holds that $\mathsf{Regret}_i(K)\geq \Omega(\frac{KH}{S})$. 
	Summing over $i$, we obtain 
	%
	$$\mathsf{Regret}(K) = \sum_{i=1}^K \mathsf{Regret}_i(K) =\Omega(KH)$$
	as claimed. 
\end{proof}

