

This section is devoted to the proof of Theorem~\ref{thm1}. Below we assume that $K\geq BSAH$, where 
 $B=4000\log^3_2(K)\log(3SAH)\log(\frac{1}{\delta})$.   
Let $\pi^k$ be the policy in the $k$-th episode.  Let $\bar{N}_{h}^k(s,a)$ be the count 
 of $(s,a,h)$ before the $k$-th episode and $N_h^k(s,a)$ be the count of the doubling batch used to compute the value function in the $k$-th episode. In particular, when $\bar{N}_h^k(s,a)=0$, we define $N_h^k(s,a)=1$.  
Let $V_h^k $ and $ Q_h^k$ be respectively the value of $ V_h$ and $Q_h$ before the $k$-th episode for all proper $(s,a,k,h)$. Recall that $\hat{P}^k_{s,a,h}$ is the value of $\hat{P}_{s,a,h}$ before the $k$-th episode. Let $\hat{r}_h^k(s,a)$ be the empirical reward function before the $k$-th episode of $(s,a)$. Let $\hat{\sigma}_h^k(s,a)$ be the empirical variance before the $k$-th episode for the state-action pair $(s,a)$, i.e., the value of $\hat{\sigma}_h(s,a)$ before the $k$-th episode.

\subsection{Optimism}
\begin{lemma}\label{lemma:opt}
With probability $1-4SAHK\delta$, $Q_h^k(s,a)\geq Q_h^*(s,a)$  and $V^k_h(s)\geq V^*_h(s)$ for any proper $(s,a,h,k)$. 
 \end{lemma}
 \begin{proof}
 For $p\in \Delta^{S}$, $v\in \mathbb{R}^S, \|v\|_{\infty}\leq H $ and $n\in N^+$,
 let 
 $$f(p,v,n)=pv + \max\left\{\frac{20}{3}\sqrt{\frac{\mathbb{V}(p,v)\log(\frac{1}{\delta})} {n}} ,\frac{400}{9}\cdot \frac{H\log(\frac{1}{\delta})}{n} \right\}.$$ Direct computation shows that $f(p,v,n)$ is non-decreasing in each dimension of $v$ as below.

Despite two possible points such that $\frac{20}{3}\sqrt{\frac{\mathbb{V}(p,v)\log(\frac{1}{\delta})} {n}}  =\frac{400}{9}\cdot \frac{H\log(\frac{1}{\delta})}{n} $,
 \begin{align}
	\frac{\partial f}{\partial v(s)} &=p(s)+\frac{20}{3}\mathbb{I}\left[  \frac{20}{3} \sqrt{  \frac{ \mathbb{V}(p,v)  \log(\frac{1}{\delta})} {n}  }\geq  \frac{400}{9}\frac{H\log(\frac{1}{\delta})}{n} \right] \frac{ p(s)(v(s)- pv )\log(\frac{1}{\delta})}{\sqrt{n\mathbb{V}(p,v) \log(\frac{1}{\delta})}}\nonumber
	\\ & \geq \min\{ p(s)+ p(s)\frac{(v(s)-pv)}{H},p(s) \}\nonumber
	\\ & = 0.
\end{align}

Fix $h,k$. In the case $N_{h}^k(s,a) \leq 2$, $Q_h^k(s,a)=H-h+1 \geq Q_h^*(s,a)$ and $_h^k(s)=H-h+1\geq V_h^*(s)$ for any proper $(s,a,h)$. 
Assume $N_{h}^k(s,a)>2$ and $Q_{h+1}^k\geq Q_{h+1}^*$. It then follows that $V_{h+1}^k \geq V_{h+1}^*$. According to the update rule in \eqref{eq:update1}, we either have that $Q_h^k(s,a)=H-h+1$ or
\begin{align}
&Q_h^k(s,a)  \nonumber\\ & = \hat{r}_h^k(s,a)+\hat{P}^k_{s,a,h}V_{h+1}^k + c_1\sqrt{\frac{\mathbb{V}(\hat{P}^k_{s,a,h},V_{h+1}^k)\log(\frac{1}{\delta})}{N_{h}^k(s,a)}} +c_2\sqrt{\frac{\left( \hat{\sigma}_h^k(s,a) -  (\hat{r}_h^k(s,a))^2 \right)\log(\frac{1}{\delta})}{N_{h}^k(s,a)}}+ c_3 \frac{H\log(\frac{1}{\delta})}{N_{h}^k(s,a)} \nonumber
\\ & \geq \hat{r}^k_{h}(s,a) +2\sqrt{2}\sqrt{\frac{\left( \hat{\sigma}_h^k(s,a) -  (\hat{r}_h^k(s,a))^2 \right)\log(\frac{1}{\delta})}{N_{h}^k(s,a)}} +\frac{28H\log(\frac{1}{\delta})}{3N_{h}^k(s,a)} + f(\hat{P}_{s,a,h}^k, V_{h+1}^k,N_{h}^k(s,a)) \nonumber
\\ & \geq \hat{r}^k_{h}(s,a) +2\sqrt{2}\sqrt{\frac{\left( \hat{\sigma}_h^k(s,a) -  (\hat{r}_h^k(s,a))^2 \right)\log(\frac{1}{\delta})}{N_{h}^k(s,a)}} +\frac{28H\log(\frac{1}{\delta})}{3N_{h}^k(s,a)}  + f(\hat{P}_{s,a,h}^k , V_{h+1}^*, N_{h}^k(s,a)) .
 \end{align}
for any $(s,a)$.



By Lemma \ref{empirical bernstein}, and recalling the definition of $\sigma_h^k(s,a)$, we have that,
\begin{align}
	& \mathbb{P}\left[|(\hat{P}^k_{s,a,h}-P_{s,a,h})V_{h+1}^*| >   2\sqrt{\frac{ \mathbb{ V}(\hat{P}^k_{s,a,h}, V_{h+1}^*  )\iota }{ N^k_h(s,a)}} +\frac{14\iota}{3n^k(s,a)} \right] \nonumber
	\\ & \leq  \mathbb{P}\left[|(\hat{P}^k_{s,a,h}-P_{s,a,h})V_{h+1}^*| >   \sqrt{\frac{2 \mathbb{ V}(\hat{P}^k_{s,a,h}, V_{h+1}^*  )\iota }{ N^k_h(s,a)-1}} +\frac{7\iota}{3N^k_h(s,a)-1} \right] \nonumber
	\\ & \leq 2\delta \label{eq_lemma1_ref.5}
\end{align}
and
\begin{align}
	& \mathbb{P}\left[ |\hat{r}^k_h(s,a)-r(s,a)| > 2\sqrt{ \frac{\left( \hat{\sigma}_h^k(s,a) -  (\hat{r}_h^k(s,a))^2 \right)\iota }{N_h^k(s,a)}} +\frac{28H\iota}{3N_h^k(s,a)}  \right]\nonumber
%	\\ & \leq \mathbb{P}\left[ |\hat{r}^k_h(s,a)-r_h(s,a)|  >2\sqrt{\frac{  \hat{\mathrm{Var}}_h^k(s,a) \iota}{N_h^k(s,a)-1}} +\frac{14H\iota}{3(N_h^k(s,a)-1)}     \right]\nonumber
	\\ & \leq 2\delta,\label{eq_lemma1_ref1}
\end{align}
%where $\hat{\mathrm{ Var}}_h^k(s,a) \leq \hat{r}_h^k(s,a)$\footnote{ $\mathbb{E}\left[(Z-\mathbb{E}[Z])^2 \right]\leq  H\mathbb{E}\left[Z\right]$ for $Z\in [0,H]$.} is the empirical variance of $R_h(s,a)$ computed by the $N_h^k(s,a)$ samples.

Note that \eqref{eq_lemma1_ref.5} implies that $f(\hat{P}^k_{s,a,h},V_{h+1}^*,N_h^k(s,a))\geq P_{s,a,h}V_{h+1}^*$.

Therefore, with probability $1-4\delta$, $Q_h^k(s,a)\geq r_h(s,a)+P_{s,a,h}V_{h+1}^*= Q_h^*(s,a)$. By induction, we learn that with probability $1-4SAHK\delta$, $Q_h^k(s,a)\geq Q_h^*(s,a)$ for any proper $(s,a,h,k)$. It then follows $V_h^k = \max_a Q_h^k(s,a)\geq \max_a Q_h^*(s,a)\geq V_h^*(s)$. The proof is completed.
%Via a union bound over all $(s,a)$ and $i$, we obtain that $\mathbb{P}\left[\mathcal{E}_1\cap \mathcal{E}_2\right]\geq 1 -2SA(\log_{2}KH+1)\delta$. The proof is completed.

%On the other hand, using empirical Bernstein's inequality,  we can verify that $f(\hat{P}_{s,a,h}^k, V_{h+1}^*, N_{s,a,h}^k)\geq Q_{h}^*(s,a)$ with probability $1-2\delta$. With a union bound on the failure probability, we obtain that with probability $1-2SAHK\delta$, $Q_h^k(s,a)\geq Q_h^*(s,a)$  and $V^k_h(s)\geq V^*_h(s)$ for any proper $(s,a,h,k)$. 
 \end{proof}

\subsection{Regret decomposition}
%\simon{let's add full details in the final version because I think this paper gonna be the standard reference for tabular MDP.}
Recall the definition that $\pi^k_h(s) = \arg\max_a Q_h^k(s,a)$. With probability $1-2SAHK\delta$, 
\begin{align}
\mathrm{Regret}(K):  & = \sum_{k=1}^K (V^*_1(s_1^k) - V^{\pi^k}_1(s_1^k)) \nonumber
\\ &  \leq  \sum_{k=1}^K \left( V^{k}_1(s_1^k) - V^{\pi^k}_1(s_1^k) \right) \nonumber
\\ &  \leq  \sum_{k=1}^K \sum_{h=1}^H \left(    (\hat{P}^k_{s_h^k,a_h^k,h}   - P_{s_h^k,a_h^k,h})V_{h+1}^k + b_h^k(s_h^k,a_h^k) \right) + \sum_{k=1}^K \sum_{h=1}^H ( P_{s_h^k,a_h^k,h}-\mathbf{1}_{s_{h+1}^k})V_{h+1}^k\nonumber
\\ & \qquad \qquad \qquad \qquad \qquad \qquad \qquad \qquad  + \sum_{k=1}^K \left(\sum_{h=1}^H \hat{r}^k_h(s^k_h,a^k_h)- V_1^{\pi^k}(s_1^k) \right),\label{eq:decomposition}
\end{align}
where $b_h^k(s_h^k,a_h^k): = c_1\sqrt{\frac{\mathbb{V}(  \hat{P}^k_{s_h^k,a_h^k,h},V_{h+1}^k)\log(\frac{1}{\delta})}{N^k_{h}(s_h^k,a_h^k) }} +c_2 \sqrt{\frac{\left( \hat{\sigma}_h^k(s,a)- (\hat{r}_h^k(s,a))^2 \right)\log(\frac{1}{\delta})}{N_h^k(s_h^k,a_h^k)}}+ c_3\frac{H\log(\frac{1}{\delta})}{N^k_{h}(s_h^k,a_h^k)}$. Here the first inequality is by Lemma~\ref{lemma:opt}, and the second inequality is by the Lemma below:

\begin{lemma}\label{lemma:decomdetail}
For each 
$k\in [K]$, 
$$V_1^k(s_1^k) \leq \sum_{h=1}^{H} \left( (\hat{P}^k_{s_h^k,a_h^k,h} - P_{s_h^k,a_h^k,h})V_{h+1}^k + b_h^k(s_h^k,a_h^k) + r_h(s_h^k,a_h^k) + (P_{s_h^k,a_h^k,h}-\textbf{1}_{s_{h+1}^k})V_{h+1}^k \right) .$$
\end{lemma}

\begin{proof}
By definition, for each $h\in [H]$
\begin{align}
V_h^k(s_h^k) &  \leq r_h(s_h^k,a_h^k) + \hat{P}^k_{s_h^k,a_h^k,h}V_{h+1}^k + b_h^k(s_h^k,a_h^k) \nonumber
\\ &  =  (\hat{P}^k_{s_h^k,a_h^k,h} - P_{s_h^k,a_h^k,h})V_{h+1}^k + b_h^k(s_h^k,a_h^k) + r_h(s_h^k,a_h^k) + (P_{s_h^k,a_h^k,h}-\textbf{1}_{s_{h+1}^k})V_{h+1}^k + V_{h+1}^k(s_{h+1}^k)\nonumber
\end{align}

Taking sum over $h\in [H]$, we have that 

\begin{align}
 & V_1^k(s_1^k) \nonumber
 \\ & \leq  \sum_{h=1}^{H} \left( (\hat{P}^k_{s_h^k,a_h^k,h} - P_{s_h^k,a_h^k,h})V_{h+1}^k + b_h^k(s_h^k,a_h^k) + r_h(s_h^k,a_h^k) + (P_{s_h^k,a_h^k,h}-\textbf{1}_{s_{h+1}^k})V_{h+1}^k \right) + V_{H+1}^k(s_{H+1}^k).
\end{align}
The proof is completed because $V_{H+1}^k=\textbf{0}$.
\end{proof}

Define $T_1 =\sum_{k=1}^K \sum_{h=1}^H \left(    (\hat{P}^k_{s_h^k,a_h^k,h}   - P_{s_h^k,a_h^k,h})V_{h+1}^k \right) $, $T_2 = \sum_{k=1}^K \sum_{h=1}^H  b_h^k(s_h^k,a_h^k)$, $T_3 =\sum_{k=1}^K\sum_{h=1}^H ( P_{s_h^k,a_h^k,h}-\mathbf{1}_{s_{h+1}^k})V_{h+1}^k $ and $T_4 = \sum_{k=1}^K \left(\sum_{h=1}^H \hat{r}^k_{h}(s^k_h,a^k_h)- V_1^{\pi^k}(s_1^k) \right) $.


We can easily bound $T_2,T_3$ and $T_4$ as in the following.

\paragraph{Bound of $T_2$} By definition, we can write
\begin{align}
T_2  & =\sum_{k=1}^K \sum_{h=1}^Hb_h^k(s_h^k,a_h^k) \nonumber
\\ & = \frac{460}{9}\sum_{k=1}^K\sum_{h=1}^H  \sqrt{\frac{\mathbb{V}(\hat{P}^k_{s_h^k,a_h^k,h},V_{h+1}^k)\log(\frac{1}{\delta})}{N_h^k(s_h^k,a_h^k)}}  +  2\sqrt{2} \sum_{k=1}^K \sum_{h=1}^H \sqrt{\frac{\left(\hat{\sigma}_h^k(s_h^k,a_h^k)- (\hat{r}_h^k(s_h^k,a_h^k))^2\right) \log(\frac{1}{\delta})}{N_h^k(s_h^k,a_h^k)}} \nonumber
\\ & \qquad\qquad \qquad \qquad \qquad \qquad \qquad \qquad \qquad \qquad \qquad \qquad+ \frac{544}{9}\sum_{k=1}^K \sum_{h=1}^H \frac{H\log(\frac{1}{\delta})}{N_h^k(s_h^k,a_h^k)}.
\end{align}


Using Cauchy's inequality and Lemma~\ref{lemma:doubling}, we obtain
\begin{align}
T_2  &  \leq \frac{460}{9}  \sqrt{2SAH \log_2(K)\log(\frac{1}{\delta})\sum_{k,h}\mathbb{V}(\hat{P}^k_{s_h^k,a_h^k,h},V_{h+1}^k )} \nonumber\\ & \qquad \qquad   + 4\sqrt{SAH\log_2(K)\log(\frac{1}{\delta})}\sqrt{\sum_{k,h}\left(\hat{\sigma}_h^k(s_h^k,a_h^k)- (\hat{r}_h^k(s_h^k,a_h^k))^2\right)}  + \frac{1088}{9}SAH^2\log_2(K)  \log(\frac{1}{\delta}) \nonumber
\\ & = \frac{460}{9}  \sqrt{2SAH \log_2(K)\log(\frac{1}{\delta}) T_5}\nonumber
\\& \qquad  + 4\sqrt{SAH\log_2(K)\log(\frac{1}{\delta})}\sqrt{\sum_{k,h}\left(\hat{\sigma}_h^k(s_h^k,a_h^k)- (\hat{r}_h^k(s_h^k,a_h^k))^2\right)} +\frac{1088}{9}SAH^2\log_2(K)  \log(\frac{1}{\delta})\nonumber
\\ & \leq \frac{460}{9}  \sqrt{2SAH \log_2(K)\log(\frac{1}{\delta}) T_5} \nonumber
\\ & \qquad \qquad \qquad \qquad + 4\sqrt{SAH^2\log_2(K)\log(\frac{1}{\delta})}\sqrt{\sum_{k,h}\hat{r}_h^k(s_h^k,a_h^k)} +\frac{1088}{9}SAH^2\log_2(K)  \log(\frac{1}{\delta}),\label{eq:boundt2o}
\end{align}
where we define $T_5 = \sum_{k=1}^K\sum_{h=1}^H\mathbb{V}(\hat{P}^k_{s_h^k,a_h^k,h},V_{h+1}^k )$, and the last inequality is by the fact that $\left(\hat{\sigma}_h^k(s_h^k,a_h^k)- (\hat{r}_h^k(s_h^k,a_h^k))^2\right)\leq H\hat{r}_h^k(s,a)$ for any proper $(s,a,h,k)$. 

We bound $\sum_{k,h}\hat{r}_h^k(s_h^k,a_h^k)$ by means of the lemma below.

\begin{lemma}\label{lemma:bdempr}
With probability $1-2SAHK\delta$, it holds that
\begin{align}
& \sum_{k=1}^K \sum_{h=1}^H \left| \hat{r}_h^k(s_h^k,a_h^k) - r_h(s_h^k,a_h^k)\right|\nonumber
\\ & \leq  4SAH^2 +4\sqrt{\sum_{k=1}^K\sum_{h=1}^H \frac{H\log(\frac{1}{\delta})}{N_h^k(s_h^k,a_h^k)}}\cdot \sqrt{\sum_{k=1}^K \sum_{h=1}^H r_h(s_h^k,a_h^k)}+24\sum_{k=1}^K\sum_{h=1}^H\frac{H\log(\frac{1}{\delta})}{N_h^k(s_h^k,a_h^k)}.\nonumber
\end{align}
\end{lemma}
\begin{proof}[Proof of Lemma~\ref{lemma:bdempr}]
In view of Lemma~\ref{empirical bernstein}, with probability $1-2SAHK\delta$ we have 
\begin{align}
\hat{r}_h^k(s,a) -r_h(s,a) & \leq 2\sqrt{2}\sqrt{\frac{\left(\hat{\sigma}_h^k(s_h^k,a_h^k)- (\hat{r}_h^k(s_h^k,a_h^k))^2\right)\log(\frac{1}{\delta})}{N_h^k(s,a)}} + \frac{28H\log(\frac{1}{\delta})}{3N_h^k(s,a)}
\\ & \leq  2\sqrt{2}\sqrt{\frac{H\hat{r}_h^k(s,a)\log(\frac{1}{\delta})}{N_h^k(s,a)}} + \frac{28H\log(\frac{1}{\delta})}{3N_h^k(s,a)}
\end{align}
for any proper $(s,a,h,k)$ be such that $N_h^k(s,a)>2$. Solve the inequality above, we can obtain 
\begin{align}
|\hat{r}_h^k(s,a) - r_h(s,a)|\leq  4\sqrt{\frac{Hr_h(s,a)\log(\frac{1}{\delta})}{N_h^k(s,a)}} + 24\frac{H\log(\frac{1}{\delta})}{N_h^k(s,a)}.\label{eq:bdt4_0}
\end{align}

It is then seen that
\begin{align}
   & \sum_{k=1}^K \sum_{h=1}^H \left| \hat{r}_h^k(s_h^k,a_h^k) - r_h(s_h^k,a_h^k)\right| \nonumber
   \\& \leq  4SAH^2 + \sum_{k=1}^K \sum_{h=1}^H\left( 4\sqrt{\frac{Hr_h(s_h^k,a_h^k)\log(\frac{1}{\delta})}{N_h^k(s_h^k,a_h^k)}}+24\frac{H\log(\frac{1}{\delta})}{N_h^k(s_h^k,a_h^k)} \right)
\nonumber
\\ & \leq 4SAH^2 +4\sqrt{\sum_{k=1}^K\sum_{h=1}^H \frac{H\log(\frac{1}{\delta})}{N_h^k(s_h^k,a_h^k)}}\cdot \sqrt{\sum_{k=1}^K \sum_{h=1}^H r_h(s_h^k,a_h^k)}+24\sum_{k=1}^K\sum_{h=1}^H\frac{H\log(\frac{1}{\delta})}{N_h^k(s_h^k,a_h^k)} .\nonumber
\end{align}
Here, the second inequality is due to Cauchy's inequality and the third inequality is a consequence of  Lemma~\ref{lemma:doubling}. We have thus completed the proof. 
\end{proof}

With Lemma~\ref{lemma:bdempr} and Lemma~\ref{lemma:doubling} in place, and noting that $\sum_{k=1}^K \sum_{h=1}^H r_h(s_h^k,a_h^k)\leq KH$, with probability $1-2SAHK\delta$,
\begin{align}
T_2\leq 61\sqrt{2SAH\log_2(K)\log(\frac{1}{\delta})T_5} +  
 145SAH^2\log_2(K)\log(\frac{1}{\delta})\label{eq:boundt2}
\end{align}

\paragraph{Bound of $T_3$} By virtue of Lemma~\ref{lemma:self-norm}, we see that with probability $1-10SAH^2K^2\delta$,
\begin{align}
T_3  \leq 2\sqrt{2}\cdot \sqrt{  T_6 \log(\frac{1}{\delta}) } + \log(\frac{1}{\delta}) + 2H\log(\frac{1}{\delta})\leq 2\sqrt{2}\cdot \sqrt{  T_6 \log(\frac{1}{\delta}) } + 3H\log(\frac{1}{\delta}),\label{eq:boundt3}
\end{align}
where $T_6 =\sum_{k=1}^K \sum_{h=1}^H\mathbb{V}(P_{s_h^k,a_h^k,h},V_{h+1}^k ) $. 


\paragraph{Bound of $T_4$} 
Note that
\begin{align}
T_4 = \sum_{k=1}^K\sum_{h=1}^H \left(\hat{r}_h^k(s_h^k,a_h^k) - r_h(s_h^k,a_h^k)  \right) + \sum_{k=1}^K \left( \sum_{h=1}^H r_h(s_h^k,a_h^k) - \mathbb{V}_1^{\pi^k}(s_1^k) \right).
\end{align}

For the first term, using Lemma~\ref{lemma:bdempr} and noting that $\sum_{k=1}^K\sum_{h=1}^H r_h(s_h^k,a_h^k)\leq KH$, with probability $1-2SAHK\delta$,
\begin{align}
 & \sum_{k=1}^K\sum_{h=1}^H (\hat{r}_h^k(s_h^k,a_h^k)-r_{h}(s_h^k,a_h^k))
\\ & \leq 4SAH^2 +4\sqrt{2\log_2(K)SAH^2}\cdot \sqrt{KH} + 24\log_2(K)SAH^2\log(\frac{1}{\delta})\nonumber
\\ & \leq  4\sqrt{2SAH^3K \log_2(K)\log(\frac{1}{\delta})}+28 SAH^2 \log_2(K)\log(\frac{1}{\delta}).\label{eq:bdt4_2}
\end{align}

Noting that $E_k:=\sum_{h=1}^H r_{h}(s_h^k,a_h^k)- V_1^{\pi^k}(s_1^k)$ is a zero-mean random variable bounded by $H$, by Lemma~\ref{lemma:self-norm}, with probability $1-2\delta$, it holds that
\begin{align}
\sum_{k=1}^K E_k \leq 2\sqrt{2}\cdot \sqrt{\sum_{k=1}^K \mathrm{Var}(E_k)\log(\frac{1}{\delta}) } + 3H^2\log(\frac{SAHK}{\delta}) \leq 2\sqrt{2KH^2\log(\frac{1}{\delta})} + 3H^2 \log(\frac{SAHK}{\delta}),\label{eq:bdt4_1}
\end{align}
where $\mathrm{Var}(E_k)$ denoting teh variance of $E_k$  conditioned on the $\mathcal{F}_{1}^{k}$.

Putting \eqref{eq:bdt4_2} and \eqref{eq:bdt4_1} together, we obtain that with probability $1-4SAHK\delta$, 
\begin{align}
T_4 \leq 6\sqrt{2SAH^3K\log_2(K)\log(\frac{1}{\delta})} +  31SAH^2\log_2(K)\log(\frac{SAHK}{\delta}).\label{eq:bdt_4f}
\end{align}

\paragraph{Bound of $T_5$ and $T_6$} Direct computation gives that
\begin{align}
T_5 : & = \sum_{k=1}^K \sum_{h=1}^H \mathbb{V}(\hat{P}^k_{s_h^k,a_h^k,h},V_{h+1}^k) \nonumber
\\ & = \sum_{k=1}^K \sum_{h=1}^H  \left( (\hat{P}^k_{s_h^k,a_h^k,h} -P_{s_h^k,a_h^k,h})(V^k_{h+1})^2  + P_{s_h^k,a_h^k,h}(V^k_{h+1})^2   - (\hat{P}^k_{s_h^k,a_h^k,h} V_{h+1}^k)^2  \right) \nonumber
\\ & \leq \sum_{k=1}^K \sum_{h=1}^H   (\hat{P}^k_{s_h^k,a_h^k,h} -P_{s_h^k,a_h^k,h})(V^k_{h+1})^2   + \sum_{k=1}^K \sum_{h=1}^H (P_{s_h^k,a_h^k,h} - \textbf{1}_{s_{h+1}^k} )  (V_{h+1}^k)^2  \nonumber
\\ & \qquad \qquad\qquad \qquad \qquad \qquad   +2H \sum_{k=1}^K \sum_{h=1}^H \max\{   V_{h}^k(s_h^k) - \hat{P}^k_{s_h^k,a_h^k,h} V_{h+1}^k  , 0\} \nonumber
\\ & \leq \sum_{k=1}^K \sum_{h=1}^H   (\hat{P}^k_{s_h^k,a_h^k,h} -P_{s_h^k,a_h^k,h})(V^k_{h+1})^2   + \sum_{k=1}^K \sum_{h=1}^H (P_{s_h^k,a_h^k,h} - \textbf{1}_{s_{h+1}^k} )  (V_{h+1}^k)^2  \nonumber
\\ & \qquad \qquad\qquad \qquad \qquad \qquad   +2H \sum_{k=1}^K \sum_{h=1}^H b_h^k(s_h^k,a_h^k)+2H\sum_{k=1}^K \sum_{h=1}^H r_h(s_h^k,a_h^k).\label{eq:boundt5}
\end{align}

Let $T_7  =  \sum_{k=1}^K \sum_{h=1}^H   (\hat{P}^k_{s_h^k,a_h^k,h} -P_{s_h^k,a_h^k,h})(V^k_{h+1})^2  $ and $T_8 = \sum_{k=1}^K \sum_{h=1}^H (P_{s_h^k,a_h^k,h} - \textbf{1}_{s_{h+1}^k} )  (V_{h+1}^k)^2$.

Using Freedman's inequality and the fact that $\mathrm{Var}(X^2)\leq 4 H^2\mathrm{Var}(X)$ for any random variable $X$ with support on $[-H,H]$, we have that
\begin{align}
T_8 \leq 2\sqrt{2}\sqrt{ 4 H^2T_6\log(\frac{1}{\delta})  } + 3H^2\log(\frac{1}{\delta}) .\label{eq:boundt8}
\end{align}


In a similar way, we can bound $T_6$ as 
\begin{align}
T_6 : & =\sum_{k=1}^K \sum_{h=1}^H \mathbb{V}(P_{s_h^k,a_h^k,h} ,V_{h+1}^k ) \nonumber
\\ &  = \sum_{k=1}^K \sum_{h=1}^H \left( P_{s_h^k,a_h^k,h} - \textbf{1}_{s_{h+1}^k} \right) (V_{h+1}^k)^2  + \sum_{k=1}^K \sum_{h=1}^H (V_{h+1}^k)^2(s_{h+1}^k) - \sum_{k=1}^K \sum_{h=1}^H (P_{s_h^k,a_h^k,h}V_{h+1}^k)^2 \nonumber
\\ & \leq 2\sqrt{2}\sqrt{4H^2T_6 \log(\frac{1}{\delta})}+3H^2 \log(\frac{1}{\delta})  + 2H \sum_{k=1}^K \sum_{h=1}^H \max\{  V_{h}^k(s_h^k) - P_{s_h^k,a_h^k,h}V_{h+1}^k,0  \} \nonumber
\\ & \leq 2\sqrt{2}\sqrt{4H^2T_6 \log(\frac{1}{\delta})}+3H^2 \log(\frac{1}{\delta}) + 2H T_2 \nonumber
\\ & \qquad \qquad \qquad \qquad \qquad \qquad + 2H \sum_{k=1}^K \sum_{h=1}^H \max\{ (\hat{P}^k_{s_h^k,a_h^k,h} - P_{s_h^k,a_h^k,h})V_{h+1}^k,0 \}+2KH^2.\label{eq:boundt6}
\end{align}

Let $T_9 = \sum_{k=1}^K \sum_{h=1}^H \max\{ (\hat{P}^k_{s_h^k,a_h^k,h} - P_{s_h^k,a_h^k,h})V_{h+1}^k,0 \}$.




\iffalse

\subsection{Bound of The Error Term}
In this section, we will prove a key lemma to deal with the error terms $T_1,T_7$ and $T_9$. This lemma controls the error term $\sum_{k,h}(\hat{P}_{s_h^k,a_h^k,h}-P_{s_h^k,a_h^k,h}) X_{h}^k$ under some mild conditions, where we do not require $\{X_{h}^k\}_{h,k}$ is conditionally independent of $\{\hat{P}^k_h\}_{h,k}$. By this lemma, we can control the error terms above by letting $\mathcal{X}_h^k = \{ V_{h+1}^k, (V_{h+1}^k)^2/H, \textbf{0}\}$. % This consititutes our key novelty

%\simon{add intuitions what $\mathcal{X}_h^k$ will be like later, and we why need this lemma}


\begin{lemma}\label{lemma:key}
For each $k,h$,  let a set of $S$-dimensional vectors  $\mathcal{X}_h^k$  be a function of $\{\hat{P}^k_{s,a,h'}\}_{h'\geq h+1,s,a}$ such that $\|X\|_{\infty}\leq H, \forall X\in \mathcal{X}_h^k$. Let $L = \max_{h,k}|\mathcal{X}_h^k|$. Then with probability $1- 10\delta$, for any sequence $\{X_h^k\}$ such that  $X_h^k\in \mathcal{X}_h^k$, it holds that
\begin{align}
 &  \sum_{k=1}^K \sum_{h=1}^H \left( \hat{P}_{s_h^k,a_h^k,h}-P_{s_h^k,a_h^k,h} \right) X_h^k  \nonumber
 \\ & \leq  \sqrt{32\log_2(K) \sum_{k=1}^K \sum_{h=1}^H \mathbb{V}(P_{s_h^k,a_h^k,h},X_h^k) \cdot (4SAH\log_2(K)\log(SALH)+\log(\frac{1}{\delta})) } + 12\log_2(K)H (4SAH\log_2(K)\log(SALH)+\log(\frac{1}{\delta})).\label{eq:key}
\end{align}
\end{lemma}
\begin{proof}




 Recall the definition of $i^k_{s,a,h}$. Define a \emph{profile} as $\{i_{s,a,h}\}_{s,a,h}$ with $i_{s,a,h}\in [\log_2(K)]$. We say two profiles $i\leq j$ iff $i_{s,a,h}\leq j_{s,a,h}$ for any $(s,a,h)$.
 
 Let $\mathcal{I} =\left\{  \{i^k_{s,a,h}\}_{s,a,h}     | k = 1,2,\ldots,K  \right\}$.  Since there are at most $SAH(\log_2(K)+1)$ updates, the size of  $\mathcal{I}$ is at most $SAH(\log_2(K)+1)$.  

\simon{this paragraph might be the most important technical paragraph. We can add more details and a plot.}
Let $M=SAH(\log_2(K)+1)$ and $\mathcal{C}:= \{  \{j^1,j^2,\ldots,j^{M}\} |j_l\in [\log_2(K)]^{SAH}, j^l\neq j^{l'} \forall l\neq l', j^l\leq j^{l+1}, \forall 1\leq l\leq M-1\}$. In words, $\mathcal{C}$ is the set of an increasing path in the set $[\log_2(K)]^{SAH}$ from $[0,0,\ldots,0]^{\top}$ to $[\log_2(K),\log_2(K),\ldots,\log_2(K)]^{\top}$. Since there are at most $M$ steps and in each step we have at most $SAH$ choices, the size of $\mathcal{C}$ is at most $(SAH)^{M}$. Let $\bar{\mathcal{C}} = \{  \{j^1,j^2,j^3,\ldots,j^q\}| j^l<j^{l+1},1\leq l \leq q-1, q\leq M \}$. Then the size of $\bar{\mathcal{C}}$ is at most $2^{M}\cdot |\mathcal{C}|=(2SAH)^M$. 
\simon{without using the increasing path property, we will have size $(\log_2K)^{SAHM}$.}

Recalling the definition of $\mathcal{I}$, we always have that $\mathcal{I}\in \bar{\mathcal{C}}$.
%Because $i^1\leq i^2\leq \ldots,\leq i^K$, $\mathcal{I}$ is determined by an increasing path in $[\log_2(K)]^{SAH}$ (it is possible that $\mathcal{I}$ does not end with $[\log_2(K),\log_2(K),\ldots, \log_2(K)]^{\top}$). Then we can always find some $\mathcal{J}\in \mathcal{C}$ such that $\mathcal{I}\subset \mathcal{J}$.

% Now we fix $\mathcal{J}\in \bar{\mathcal{C}}$. Let $\delta'\in (0,1)$ be a confidence parameter and $\log(\frac{1}{\delta})' = \log(2/\delta')$. Assume $\mathcal{J} =\{j^1,j^2,\ldots, j^M\}$ such that $j^1\leq j^2\leq \cdots \leq j^M$.  For $j\in [\log_2(K)]^{SAH}$, we use $\hat{P}(j)$ to denote the tuple $\{ \hat{P}_{s,a,h}(j_{s,a,h})  \}_{s,a,h}$ where $\hat{P}_{s,a,h}^{(l)}$ is the empirical transition model computed using the $l$-th batch of $(s,a,h)$ for $1\leq l \leq \log_2(K)$, and $\hat{P}_{s,a,h}^{(l)} = \frac{1}{S}\cdot \textbf{1}$ if $l=0$.  We also define $\hat{P}_h(j)=\{ \hat{P}_{s,a,h'}(j_{s,a,h})  \}_{h'\geq h+1,s,a}$.%\simon{what is $j^k$?}


\end{proof}

As a result, with probability $1-\delta$ it holds that 
\begin{align}
 &  \sum_{k=1}^K \sum_{h=1}^H \left( \hat{P}_{s_h^k,a_h^k,h}-P_{s_h^k,a_h^k,h} \right) X_h^k  \nonumber
 \\ & \leq  \sqrt{32\log_2(K) \sum_{k=1}^K \sum_{h=1}^H \mathbb{V}(P_{s_h^k,a_h^k,h},X_h^k) \cdot (4SAH\log_2(K)\log(SALH)+\log(\frac{1}{\delta})) } + 12\log_2(K)H (4SAH\log_2(K)\log(SALH)+\log(\frac{1}{\delta})).\nonumber
\end{align}
The proof is completed.

\fi 

\subsection{Bounds of the error terms}

Now we  deal with $T_1,T_7$ and $T_9$. We first recall the definition.
\begin{align}
 & T_1 = \sum_{k=1}^K \sum_{h=1}^H  \left( \hat{P}^k_{s_h^k,a_h^k,h}  -P_{s_h^k,a_h^k,h} \right) V_{h+1}^k
;\nonumber
\\ & T_7  = \sum_{k=1}^K \sum_{h=1}^H  \left( \hat{P}^k_{s_h^k,a_h^k,h}  -P_{s_h^k,a_h^k,h} \right) (V_{h+1}^k)^2;\nonumber
\\ & T_9 =  \sum_{k=1}^K \sum_{h=1}^H \max\{ (\hat{P}^k_{s_h^k,a_h^k,h} - P_{s_h^k,a_h^k,h})V_{h+1}^k,0 \}.\nonumber
\end{align}

Recall $B=4000\log^2_3(K)\log(3SAH)\log(\frac{1}{\delta}) $. By the update rule \eqref{eq:updateq}, $V_{h+1}^k$ is determined by the $\{ \hat{P}^k_{s,a,h'} \}_{(h+1\leq h'\leq H ,s,a)}$ and $\{I^k_{s,a,h'}\}_{h+1\leq h'\leq H,s,a}$. 
Using Lemma~\ref{lemma:key3} with $\mathcal{X}_{h+1} =\{V_{h+1}^k\}_{k=1}^K \cup\{0\}, \{(V_{h+1}^k)^2 /H\}_{k=1}^K\cup\{0\}$ and $\{ V_{h+1}^k\}_{k=1}^K \cup\{0\}$, and noting that $\mathrm{Var}(X^2)\leq 4\|X\|^2_{\infty}\mathrm{Var}(X)$, we have that with probability $1-30 SAH^2K^2 \delta$,
\begin{align}
 & T_1\leq \sqrt{ B SAH \sum_{k=1}^K \sum_{h=1}^H \mathbb{V}(P_{s_h^k,a_h^k,h},V_{h+1}^k)}+BSAH^2 = \sqrt{BSAH T_6}+BSAH^2;\label{eq:boundt1}
 \\ & T_7 \leq H \sqrt{ 4BSAH  \sum_{k=1}^K \sum_{h=1}^H \mathbb{V}(P_{s_h^k,a_h^k,h}, V_{h+1}^k) }    + 4BSAH^3 = H\sqrt{4BSAH T_6}+4BSAH^3 ;\label{eq:boundt7}
 \\ & T_9 \leq \sqrt{B SAH \sum_{k=1}^K \sum_{h=1}^H \mathbb{V}(P_{s_h^k,a_h^k,h},V_{h+1}^k)}+BSAH^2= \sqrt{BSAH T_6}+BSAH^2.\label{eq:boundt9}
\end{align}

\subsection{Putting all pieces together}

Rewrite  \eqref{eq:boundt2},\eqref{eq:boundt3},\eqref{eq:bdt_4f},\eqref{eq:boundt5},\eqref{eq:boundt6},\eqref{eq:boundt8},\eqref{eq:boundt1},\eqref{eq:boundt7} and \eqref{eq:boundt9} as below. With probability $1-100SAH^2K\delta$,

\begin{align}
 & T_2\leq 61 \sqrt{B SAHT_5}+145BSAH^2;\label{eq:obt2}
 \\ & T_3 \leq \sqrt{8BT_6}+3HB  ;\label{eq:obt3}
 \\ & T_4 \leq 9\sqrt{ BSAH^3K}+31BSAH^2;\label{eq:obt4}
 \\ & T_5 \leq T_7 + T_8 + 2H T_2 + 2KH^2 ;\label{eq:obt5}
 \\ & T_6 \leq \sqrt{32B H^2T_6} + 2HT_2 + 2HT_9 + 3H^2B + 2KH^2;\label{eq:obt6}
 \\ & T_8 \leq \sqrt{32BH^2T_6 } + 3BH^2 ;\label{eq:obt8}
 \\ & T_1\leq \sqrt{BSAHT_6} + BSAH^2;\label{eq:obt1}
 \\ & T_7 \leq H\sqrt{4BSAHT_6} + 4BSAH^3;\label{eq:obt7}
 \\ & T_9 \leq \sqrt{BSAHT_6}+BSAH^2.\label{eq:obt9}
\end{align}

 To solve the inequalities \eqref{eq:obt2} to \eqref{eq:obt9}, we use the fact that $a\leq \sqrt{bc}+d$ implies that $a \leq \epsilon b + \frac{1}{2\epsilon}c +d$ for any $b,c\geq 0, a,d\in \mathbb{R}$ and any $\epsilon>0$. Using this arguments, we have
 \begin{align}
 & HT_2 \leq \epsilon T_5 + (\frac{1}{2\epsilon}+1)61BSAH^3+145BSAH^3;\nonumber
 \\ & T_6 \leq \epsilon T_6 + 2HT_2 + 2HT_9 + (3+\frac{32}{\epsilon})BH^2 + 2KH^2;\nonumber
 \\ & HT_9 \leq \epsilon T_6 +\left( \frac{1}{2\epsilon}+1\right)BSAH^3;\nonumber
 \\ & T_8 \leq \epsilon T_6 + \left( \frac{16}{\epsilon}+1\right)BH^2;\nonumber
 \\ & T_7 \leq  \epsilon T_6 + \left(\frac{2}{\epsilon} +4\right)BSAH^3.\nonumber
 \end{align}
Then we have that
\begin{align}
& T_5 \leq T_7+T_8 + 2HT_2 + 2KH^2\leq 2\epsilon T_5+2\epsilon T_6 +  \left( \frac{100}{\epsilon}+300\right) BSAH^3+2KH^2;\nonumber
\\ & T_6 \leq 3 \epsilon T_6 + 2\epsilon T_5 + \left( \frac{200}{\epsilon}+300\right) BSAH^3+2KH^2.
\end{align}

 Choosing $\epsilon= \frac{1}{20}$, we learn that $T_5+T_6\leq O(BSAH^3+KH^2)$. 
Recall that  $K\geq SAHB$. We then have that $T_5,T_6 = O(KH^2 + BSAH^3) = O(KH^2)$. As a result, we have that $T_1 = O(\sqrt{BSAH^3K})$,  $T_7, T_8 = O(\sqrt{BSAH^5K})$, $T_2 = O(\sqrt{BSAH^3K})$ and $T_3 = O(\sqrt{BKH^2})$. We then conclude that the total regret is bounded by $\tilde{O}(\sqrt{SAH^3K\log(\frac{1}{\delta})})$. The proof of Theorem~\ref{thm1} is completed by replacing $\delta$ with $\frac{\delta}{100SAH^2K}$.

