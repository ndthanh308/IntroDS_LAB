
In this section, we will use $r$ to denote the negative reward, that is, $r= -c$. Recall \eqref{eq:updatecost}:
\begin{align}
Q_h(s,a)\leftarrow \max\{\min \left\{ \hat{r}_h(s,a) + \hat{P}_{s,a,h}V_{h+1}+b_h(s,a), 0 \right\} ,-H\}.\nonumber
\end{align}



Recall the definition of $T_1$-$T_9$. We note that the analysis of $T_1,T_3,T_7,T_8$ and $T_9$ in Appendix~\ref{sec:appfirst} applies for the case the reward function is negative. So it suffices to provide bounds for $T_2,T_4,T_5$ and $T_6$ with respect to $c^*$.



\paragraph{Bound of $T_2$}
Recall that 
\begin{align}
T_2  & = \sum_{k=1}^K \sum_{h=1}^H b_h^k(s_h^k,a_h^k) \nonumber
\\ & =\sum_{k=1}^K \sum_{h=1}^H \Bigg( \frac{460}{9}  \sqrt{\frac{\mathbb{V}(\hat{P}^k_{s_h^k,a_h^k,h},V_{h+1}^k)\log(\frac{1}{\delta})}{N_h^k(s_h^k,a_h^k)}} \nonumber\\ & \qquad \qquad \qquad \qquad   +  2\sqrt{2} \sqrt{\frac{\left(\hat{\sigma}_h^k(s_h^k,a_h^k)- (\hat{r}_h^k(s_h^k,a_h^k))^2\right) \log(\frac{1}{\delta})}{N_h^k(s_h^k,a_h^k)}}+ \frac{544}{9}\frac{H\log(\frac{1}{\delta})}{N_h^k(s_h^k,a_h^k)}\Bigg).\label{eq:cc1}
\end{align}
For the first and third term in right hand side of \eqref{eq:cc1}, we can use Cauchy's inequality to obtain that 
\begin{align}
 & \sum_{k=1}^K \sum_{h=1}^H \sqrt{\frac{\mathbb{V}(\hat{P}^k_{s_h^k,a_h^k,h},V_{h+1}^k)\log(\frac{1}{\delta})}{N_h^k(s_h^k,a_h^k)}} \nonumber
 \\ & \leq   \sqrt{2SAH\log_2(K)\log(\frac{1}{\delta})\sum_{k,h}\mathbb{V}(\hat{P}^k_{s_h^k,a_h^k,h},V_{h+1}^k)}\nonumber
 \\ & = \sqrt{2SAH\log_2(K)\log(\frac{1}{\delta})T_5} \label{eq:cterm1} 
 \end{align}
 and
 \begin{align}
\sum_{k=1}^K \sum_{h=1}^H \frac{H\log(\frac{1}{\delta})}{N_h^k(s_h^k,a_h^k)} \leq 2SAH^2\log_2(K)\log(\frac{1}{\delta}).\label{eq:cterm2}
\end{align}

For the second term, noting that
\begin{align}
\left( \hat{\sigma}_h^k(s_h^k,a_h^k) - (\hat{r}_h^k(s_h^k,a_h^k)^2)\right) \leq  -H \hat{r}_h^k(s_h^k,a_h^k),\nonumber
\end{align}
we have 
\begin{align}
&\sqrt{\frac{\left(\hat{\sigma}_h^k(s_h^k,a_h^k)- (\hat{r}_h^k(s_h^k,a_h^k))^2\right) \log(\frac{1}{\delta})}{N_h^k(s_h^k,a_h^k)}}\nonumber
\\ &  \leq \sqrt{2SAH\log_2(K)\log(\frac{1}{\delta})}\sqrt{H\sum_{k,h}-\hat{r}_h^k(s_h^k,a_h^k)} \nonumber
\\ & \leq \sqrt{2SAH^2\log_2(K)\log(\frac{1}{\delta})}\sqrt{T_4 + 3Kc^* + \sum_{k=1}^K (-V_1^{\pi^k}(s_1^k) +V_1^*(s_1^k)  )+\sum_{k=1}^K (-V_1^*(s_1^k)-3c^*) } .\label{eq:cterm3}
\end{align}

By Lemma~\ref{lemma:con}, with probability $1-\delta$,
\begin{align}
\sum_{k=1}^K -V_1^*(s_1^k)\leq 3Kc^* + H\log(\frac{1}{\delta}).\nonumber
\end{align}
On the other hand, we note that
\begin{align}
\sum_{k=1}^K (-V_1^{\pi^k}(s_1^k) +V_1^*(s_1^k)  ) = \mathrm{Regret}(K) = T_1+T_2+T_3+T_4.
\end{align}

Putting all together, we obtain that, with probability $1-\delta$,
\begin{align}
T_2 &  \leq 90\sqrt{SAH\log_2(K)\log(\frac{1}{\delta})T_5}  \nonumber\\ & \qquad + 4\sqrt{SAH^2\log_2(K)\log(\frac{1}{\delta})}\sqrt{T_1+T_2+T_3+2T_4+3Kv^*} + 130 SAH^2\log_2(K)\log(\frac{1}{\delta}).\label{eq:ccbt2}
\end{align}


\paragraph{Bound of $T_4$}


Recall that
\begin{align}
 T_4 & = \sum_{k=1}^K \left( \sum_{h=1}^H \hat{r}_h^k(s_h^k,a_h^k)-V_{1}^{\pi^k}(s_1^k)\right) \nonumber
 \\ & = \sum_{k=1}^K \left( \sum_{h=1}^H \hat{r}_h^k(s_h^k,a_h^k)-r_{h}(s_h^k,a_h^k)\right) + \sum_{k=1}^K \left( \sum_{h=1}^H r_h(s_h^k,a_h^k) - V_{1}^{\pi^k}(s_1^k) \right).
 \end{align}

Also recall that $\check{T}_1 =\sum_{k=1}^K \left( \sum_{h=1}^H \hat{r}_h^k(s_h^k,a_h^k)-r_{h}(s_h^k,a_h^k)\right)  $ and $\check{T}_2 =\sum_{k=1}^K \left( \sum_{h=1}^H r_h(s_h^k,a_h^k) - V_{1}^{\pi^k}(s_1^k) \right) $. We continue with a lemma to bound the empirical reward for negative reward function.

\begin{lemma}\label{lemma:bdempc}
With probability $1-2SAHK\delta$, it holds that
\begin{align}
& \sum_{k=1}^K \sum_{h=1}^H \left| \hat{r}_h^k(s_h^k,a_h^k) - r_h(s_h^k,a_h^k)\right|\nonumber
\\ & \leq  4SAH^2 +4\sqrt{\sum_{k=1}^K\sum_{h=1}^H \frac{H\log(\frac{1}{\delta})}{N_h^k(s_h^k,a_h^k)}}\cdot \sqrt{\sum_{k=1}^K \sum_{h=1}^H -r_h(s_h^k,a_h^k)}+24\sum_{k=1}^K\sum_{h=1}^H\frac{H\log(\frac{1}{\delta})}{N_h^k(s_h^k,a_h^k)}.\nonumber
\end{align}
\end{lemma}
The proof of Lemma~\ref{lemma:bdempc} is basically the same as that of Lemma~\ref{lemma:bdempr}, except for that $r$ is replaced with $-r$.


By Lemma~\ref{lemma:bdempc} and Lemma~\ref{lemma:doubling}, with probability $1-3SAHK\delta$,  
\begin{align}
 |\check{T}_1 |& \leq 4\sqrt{2SAH^2\log_2(K)}\cdot \sqrt{\sum_{k=1}^K\sum_{h=1}^H -r_h(s_h^k,a_h^k)} + 52SAH^2\log_2(K)\log(\frac{1}{\delta}) \nonumber
  \\ & \leq 4\sqrt{2SAH^2\log_2(K)}\cdot \sqrt{\check{T}_2 + 3Kc^*+\sum_{k=1}^K (-V_1^{*}(s_1^k)- 3c^*) } + 52SAH^2\log_2(K)\log(\frac{1}{\delta}) \nonumber
  \\ & \leq 4\sqrt{2SAH^2\log_2(K)}\cdot \sqrt{\check{T}_2 + 3Kc^* } + 60SAH^2\log_2(K)\log(\frac{1}{\delta}) ,\label{eq:cct1}
\end{align}
where in the last line we use the fact 
\begin{align}
\sum_{k=1}^K -V_1^*(s_1^k)\leq 3Kc^* + H\log(\frac{1}{\delta})\label{eq:addc}
\end{align}
with probability $1-\delta$ (Lemma~\ref{lemma:con}).

On the other hand, by Lemma~\ref{lemma:self-norm} and \eqref{eq:addc}, with probability $1-3SAHK\delta$,
\begin{align}
|\check{T}_2| & \leq 2\sqrt{2\sum_{k=1}^K \mathbb{E}_{\pi^k,}\left[\left(\sum_{h=1}^H r_h(s_h,a_h) \right)^2 |s_1=s_1^k\right]\log(\frac{1}{\delta})}+3H^2\log(\frac{1}{\delta}) \nonumber
\\ & 2\sqrt{2H\sum_{k=1}^K \mathbb{E}_{\pi^k}\left[\sum_{h=1}^H  -r_h(s_h,a_h) |s_1=s_1^k \right]\log(\frac{1}{\delta})}+3H\log(\frac{1}{\delta}) \nonumber
\\ & \leq 2\sqrt{2H\left( \sum_{k=1}^K \left(- V_1^{\pi^k}(s_1^k) +V_1^*(s_1^k) \right) + \sum_{k=1}^K\left(-V_1^*(s_1^k)-3c^*\right)+3Kc^*\right)\log(\frac{1}{\delta})}+3H\log(\frac{1}{\delta}) \label{eq:cct1.5}
\\ & \leq 3Kc^*  +T_1+T_2+T_3+T_4 + 9H\log(\frac{1}{\delta}).\label{eq:cct2}
\end{align}

Combining \eqref{eq:cct1}, \eqref{eq:cct1.5} with \eqref{eq:cct2}, with probability $1-4SAHK\delta$,
\begin{align}
& |\check{T}_1|\leq     16\sqrt{SAH^2(Kc^*+T_1+T_2+T_3+T_4)\log_2(K)\log(\frac{1}{\delta})}  + 200SAH^2\log_2(K)\log(\frac{1}{\delta})\nonumber
\\ & |\check{T}_2| \leq  2\sqrt{2H(3Kc^*+T_1+T_2+T_3+T_4)\log(\frac{1}{\delta})}+9H\log(\frac{1}{\delta}) .\nonumber
\end{align}

As a result, we have that
\begin{align}
|T_4| \leq 22\sqrt{SAH^2(Kc^* +T_1+T_2+T_3+T_4)\log_2(K)\log(\frac{1}{\delta})}  + 209SAH^2\log_2(K)\log(\frac{1}{\delta}).\label{eq:ccbt4}
\end{align}

\paragraph{Bound of $T_5$}

Using the arguments in  \eqref{eq:boundt5}, and noting the update rule \eqref{eq:updatecost}, we have
\begin{align}
T_5  & \leq \sum_{k=1}^K \sum_{h=1}^H   (\hat{P}^k_{s_h^k,a_h^k,h} -P_{s_h^k,a_h^k,h})(V^k_{h+1})^2   + \sum_{k=1}^K \sum_{h=1}^H (P_{s_h^k,a_h^k,h} - \textbf{1}_{s_{h+1}^k} )  (V_{h+1}^k)^2  +2H\sum_{k=1}^K \sum_{h=1}^H -r_h(s_h^k,a_h^k).\nonumber
\end{align}

Recall that
\begin{align}
\sum_{k=1}^K \sum_{h=1}^H -r_h(s_h^k,a_h^k) &  = - \check{T}_2 -\sum_{k=1}^K V_1^{\pi^k}(s_1) \leq -\check{T}_2 + \sum_{k=1}^K V_1^{*}(s_1^k).\label{eq:cx1}
\end{align}
By \eqref{eq:addc}, with probability $1-5SAHK\delta$,
\begin{align}
\sum_{k=1}^K \sum_{h=1}^H -r_h(s_h^k,a_h^k) \leq  2\sqrt{2H(3Kc^* + T_1+T_2+T_3+T_4)\log(\frac{1}{\delta})} + 3Kc^* + 10H\log(\frac{1}{\delta}).
\end{align}
As a result, we have that
\begin{align}
T_5 & \leq T_7 + T_8+2HT_2 + 4\sqrt{2H^3(3Kc^* +T_1+T_2+T_3+T_4)\log(\frac{1}{\delta})}+ 6HKc^* + 20H^2\log(\frac{1}{\delta}).\label{eq:ccbt5}
\end{align}
with probability $1-5SAHK\delta$.



\paragraph{Bound of $T_6$}

Using the arguments in  \eqref{eq:boundt5}, \eqref{eq:addc} and \eqref{eq:cx1}, and noting the update rule \eqref{eq:updatecost}, with probability $1-3SAHK\delta$
\begin{align}
T_6 & \leq 2\sqrt{8T_6\log(\frac{1}{\delta})} + 3H^2\log(\frac{1}{\delta})+2H \sum_{k=1}^K\sum_{h=1}^H \max\{P_{s_h^k,a_h^k,h}V_{h+1}^k-V_{h}^k(s_h^k),0 \} \nonumber
\\ & \leq  2\sqrt{8T_6\log(\frac{1}{\delta})} + 3H^2\log(\frac{1}{\delta})+ 2HT_9 + 2H\sum_{k=1}^K\sum_{h=1}^H -r_h(s_h^k,a_h^k)\nonumber
\\ & \leq 2\sqrt{8T_6\log(\frac{1}{\delta})} + 3H^2\log(\frac{1}{\delta})+ 2HT_9 \nonumber
\\ & \qquad \qquad \qquad + 2H \left(2\sqrt{2H (3Kc^*+T_1+T_2+T_3+T_4)\log(\frac{1}{\delta})  }+3Kc^*+10H\log(\frac{1}{\delta})\right).\label{eq:ccbt6}
\end{align}

\paragraph{Putting all together}


Solving \eqref{eq:ccbt2},\eqref{eq:boundt3},\eqref{eq:ccbt4},\eqref{eq:ccbt5},\eqref{eq:ccbt6},\eqref{eq:boundt8},\eqref{eq:boundt1},\eqref{eq:boundt7} and \eqref{eq:boundt9}, we have that, with probability $1-100SAH^2K\delta$, $T_6 = O(HKc^*+BSAH^3)$, $T_1 = O(\sqrt{BSAH^2Kc^*}+BSAH^2)$, $T_7,T_8 = O(\sqrt{BSAH^4Kc^*}+BSAH^3)$, $T_5 = O(HKc^*+BSAH^2)$, $T_2 = O(\sqrt{BSAH^2Kc^*}+BSAH^2)$ and $T_3 = O(\sqrt{BHKc^*}+BSAH^2)$. We then conclude that the total regret is bounded by $O(\sqrt{BSAH^2Kc^*}+BSAH^2)$. On the other hand, the regret bound is trivially bounded by $O(K(H-c^*))$. The proof is completed by replacing $\delta$ with $\frac{\delta}{100SAH^2K}$.

