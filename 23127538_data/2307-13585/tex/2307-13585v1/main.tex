\documentclass[11pt]{amsart}

\newcommand{\Addresses}{{
		\bigskip
		\footnotesize
		
		\textsc{Mathematical Institute, University of Oxford, Oxford, OX2 6GG, UK}\par\nopagebreak
		\textit{E-mail addresses:}
		\textit{gorodetsky@maths.ox.ac.uk} 
		\textit{lasse.grimmelt@maths.ox.ac.uk}
}}

\usepackage[margin=1in]{geometry}
\usepackage[all]{xy}
\usepackage{amssymb}
\usepackage{amsfonts}
\usepackage{amsthm}
\usepackage{amsmath}
\usepackage{hyperref}
\usepackage{mathtools}
\usepackage{url}



\newtheorem{thm}{Theorem}
\newtheorem{cor}{Corollary}  
\newtheorem{lem}{Lemma}  
\newtheorem{proposition}{Proposition}
\newtheorem{conj}{Conjecture}

\theoremstyle{plain}

\newtheorem{remark}{Remark}


\theoremstyle{definition}
\newcommand{\PP}{\mathbb{P}}
\newcommand{\ZZ}{\mathbb{Z}}
\newcommand{\FF}{\mathbb{F}}
\newcommand{\EE}{\mathbb{E}}
\newcommand{\NN}{\mathbb{N}}
\newcommand{\RR}{\mathbb{R}}
\newcommand{\Lower}{D}

\numberwithin{equation}{section}
\mathtoolsset{showonlyrefs}


\begin{document}

\author{Ofir Gorodetsky, Lasse Grimmelt}
\title[On a conjecture of Elliott]{On a conjecture of Elliott concerning a probabilistic variant of Titchmarsh's divisor problem}

\date{}



\begin{abstract}
Elliott and Halberstam proved that $\sum_{p<n} 2^{\omega(n-p)}$ is asymptotic to $\phi(n)$. In analogy to the Erd\H{o}s--Kac Theorem, Elliott conjectured that if one restricts the summation to primes $p$ such that $\omega(n-p)\le 2 \log \log n+\lambda(2\log \log n)^{1/2}$ then the sum will be asymptotically proportional to $\phi(n)\int_{-\infty}^{\lambda} e^{-t^2/2}dt$. We confirm this conjecture.
\end{abstract}

\maketitle
\section{Introduction}

Let $\omega(m) = \sum_{p \mid m}1$ be the prime divisor function. Elliott and Halberstam proved, using the Bombieri--Vinogradov Theorem, that \cite{ElliottH}
\begin{equation}\label{eq:elliott}
\sum_{p<n} 2^{\omega(n-p)} = \phi(n) + O\left(\frac{n}{\log n}(\log \log n)^2\right).
\end{equation}
Here and later $p$ denotes a prime and $\phi$ is Euler's totient function. Elliott gave a talk at the meeting held in Urbana--Champaign, June 5–7, 2014, in memory of Paul and Felice Bateman, and Heini Halberstam, where he revisited his works with Halberstam. A telegraphic representation of the talk was published, which contains the following conjecture \cite[Conj.\ B]{Elliott}: for each real $\lambda$,	
\begin{equation}\label{eq:conjb} \sum_{\substack{p <n\\ \omega(n-p)\le 2\log \log n+ \lambda (2\log \log n)^{1/2}}}2^{\omega(n-p)} \to \frac{\phi(n)}{\sqrt{2\pi}}\int_{-\infty}^{\lambda}e^{-u^2/2}du, \qquad n \to \infty.
\end{equation}
In this note we establish the conjecture.
\begin{thm}\label{thm:e}
The asymptotic relation \eqref{eq:conjb} holds for each real $\lambda$. \end{thm}
Without the $2^{\omega(n-p)}$ weight, variants of \eqref{eq:conjb} were known for a long time \cite{Halberstam,Tanaka}.

The strategy of proof of Theorem \ref{thm:e} is as follows. By work of Granvile and Soundararajan \cite{GS}, Theorem \ref{thm:e} can be reduced to understanding the distribution of $2^{\omega(n-p)}$ in artihmetic progressions. More precisely, for $a$ in a certain range it is required to consider
\begin{align*}
    \sum_{p<n:\, a \mid n-p}2^{\omega(n-p)}.
\end{align*}
Similar as in \cite{ElliottH}, we obtain the asymptotics for this by using the distribution of primes in arithmetic progressions. However, as the required range of $a$ is large, larger than any bounded power of $\log n$, the Bombieri-Vinogradov Theorem is no longer enough for this. The requirements for the replacement of the Bombieri-Vinogradov Theorem in the proof of Theorem \ref{thm:e} are strong:
\begin{itemize}
    \item We need to handle arithmetic progressions with modulus larger than $n^{1/2}$.
    \item The weight on the moduli contains a component of non-neglible range to be handled with absolute value, arising from $a$.
    \item The progression to be considered is of the form $p\equiv n (d)$, where $n$ is of the same order of magnitude as the range of summation of $p$.
\end{itemize}
We succeed by showing that these conditions can be handled by an application of a recent result due to Blomer, the second-named author, Li and Myerson \cite[Proposition 6.1]{Blomer}.


\subsection*{Acknowledgements}
This project has received funding from the European Research Council (ERC) under the European Union's Horizon 2020 research and innovation programme (grant agreement No 851318).


\section{Initial reduction}
Theorem \ref{thm:e} can be restated as follows. Let $X_n$ be a random variable taking the value $m \in \{1,2,\ldots,n\}$ with probability proportional to $2^{\omega(m)} \mathbf{1}_{n-m \textup{ a prime}}$. Then
\begin{equation}\label{eq:prob}
\frac{\omega(X_n)-2\log \log n}{(2\log \log n)^{1/2}} \overset{d}{\longrightarrow} N(0,1), \qquad n \to \infty,
\end{equation}
where $N(0,1)$ is the standard gaussian distribution and the arrow indicates convergence in distribution. Let 

\begin{equation}\label{eq:Ck}
    \begin{split}C_k &:= \EE X^k,\\
    C'_k &:= \EE |X|^k, \end{split}
\end{equation}
 where $X\sim N(0,1)$. It is known that for \eqref{eq:prob} to hold it suffices to show that
\begin{equation}\label{eq:mom}
\EE\left( \frac{\omega(X_n)-2 \log \log n}{(2\log \log n)^{1/2}}\right)^k \to C_k, \qquad n \to \infty,
\end{equation}
holds for every integer $k \ge 1$ \cite[Ch.\ 3.3.5]{Durrett}. To approach \eqref{eq:mom} it is convenient to use a general result of Granville and Soundararajan \cite{GS}. 
\begin{proposition}[Granville--Soundararajan]\label{prop:gs}
Let $f \colon\NN \to \RR_{\ge 0}$ be an arithmetic function and  $\mathcal{P}$ be a squarefree integer. Let $h \colon\NN \to \RR_{\ge 0}$ be a multiplicative function with $h(p) \le p$ for $p | \mathcal{P}$. For $a | \mathcal{P}  $ define $r_a$ via
\[ \sum_{m \le n: \, a \mid m}f(m) =\frac{h(a)}{a}\sum_{m \le n}f(m) + r_a.\]
Let
\[ \mu_{\mathcal{P}} = \sum_{p \mid \mathcal{P}} \frac{h(p)}{p}, \qquad \sigma^2_{\mathcal{P}} =  \sum_{p \mid \mathcal{P}} \frac{h(p)}{p}\left(1-\frac{h(p)}{p}\right).\]
Uniformly for $1 \le k \le \sigma_{\mathcal{P}}^{2/3}$ we have
\begin{align}
\sum_{m \le n} &f(m) \big(\sum_{p \mid \mathcal{P}} \mathbf{1}_{p \mid m} -\mu_{\mathcal{P}}\big)^k =\sigma_{\mathcal{P}}^k \sum_{m \le n} f(m) \big( C_k  + O( (k^{3/2}\sigma_{\mathcal{P}}^{-1})^{1+\mathbf{1}_{2 \mid k}}C'_k)\big)+  O\bigg(\mu_{\mathcal{P}}^k  \sum_{\substack{\omega(a) \le k\\ a \mid \mathcal{P} }} |r_a|\bigg),
\end{align}
where $C_k$, $C'_k$ are given by \eqref{eq:Ck}. 
\end{proposition}
For $f \equiv 1$ this is \cite[Prop.\ 3]{GS} and the proof works the same way for general $f \ge 0$. Proposition \ref{prop:gs} allows us to reduce the proof of Theorem \ref{thm:e} to a level-of-distribution result. We now state this reduction in somewhat more general terms, for the proof of Theorem \ref{thm:e} only the case $\theta=2$ will be used.
\begin{cor}\label{cor:gs}
Fix $\theta>0$ and $k \in \NN $. For every $n \gg 1$ let $Y_{n}$ be a random variable taking values in $\{1,2,\ldots,n\}$ and set
\[ I_n:=(\exp(\exp(\log^{1/3}\log  n)), n^{\exp(-\log^{1/3}\log n)}), \qquad T:=n^{1/\log^{1/3} \log n}, \qquad  \mathcal{P} = \prod_{p \in I_n}p.\]
Suppose that, as $n \to \infty$
\begin{align}
\label{eq:cond2}
\sum_{\substack{\omega(a) \le k\\p\mid a \implies p \in [2,T) \setminus I_n\\ \mu^2(a)=1}} \PP( a \textup{ divides } Y_n) &=o\big( (\log \log n)^{k/2}\big),\\
\label{eq:cond1} \sum_{\substack{\omega(a) \le k\\a \mid \mathcal{P}}}\bigg| \PP(a \textup{ divides }Y_{n})-\frac{\theta^{\omega(a)}}{a}\bigg| &=o\big( (\log \log n)^{-k/2}\big).
\end{align}
Then we have
\[ \EE \bigg( \frac{\omega(Y_{n})-\theta \log \log n}{(\theta \log \log n)^{1/2}}\bigg)^k \to C_k, \qquad n \to \infty.\]
\end{cor}
\begin{proof}
We consider a given $n$ and apply Proposition \ref{prop:gs} with $f(m) =\PP(Y_n=m)$, $h(m)=\theta^{\omega(m)}$. Define
\begin{equation}
\omega_{\mathcal{P}}(n) = \sum_{p \mid \mathcal{P}} \mathbf{1}_{p \mid n},\qquad \mu = \sum_{p \mid \mathcal{P}} \frac{\theta}{p} ,\qquad \sigma^2 = \sum_{p \mid \mathcal{P}} \frac{\theta}{p}\left(1-\frac{\theta}{p}\right) \qquad r_a=\PP(a \textup{ divides }Y_{n})-\frac{\theta^{\omega(a)}}{a}.
\end{equation}
We find that 
\begin{equation}\label{eq:app}
	\sum_{m=1}^{n} \PP(Y_n=m)\left(\frac{\omega_{\mathcal{P}}(m)-\mu}{\sigma}\right)^k =  \sum_{m=1}^{n} \PP(Y_n=m)\big( C_k + O( \sigma^{-1-\mathbf{1}_{2 \mid k}})\big)+  O\bigg(  \frac{\mu^k}{\sigma^k} \sum_{\substack{\omega(a) \le k\\ a \mid \mathcal{P}}} |r_a|\bigg)
\end{equation}
holds for every fixed $k \ge 1$. Since $\sum_{p \le x} 1/p = \log \log x + O(1)$ we have
\begin{equation}\label{eq:omegamusigma}
\mu, \sigma^2 = \theta \log \log n+O(\log^{1/3}\log n).
\end{equation}
We express \eqref{eq:app} as
\begin{align}\label{eq:diff} \EE\left(\frac{\omega_{\mathcal{P}}(Y_n)-\mu}{\sigma}\right)^k -C_k \ll  (\log \log n)^{-1/2} + (\log \log n)^{k/2}  \sum_{\substack{\omega(a) \le k\\ a \mid \mathcal{P}}} |r_a|.
\end{align}
By \eqref{eq:cond1}, the right-hand side of \eqref{eq:diff} goes to $0$ as $n \to \infty$. Let $T=n^{1/\log^{1/3} \log n}$. We have the identity
\[ \frac{\omega(Y_n)-\theta\log \log n}{(\theta \log \log n)^{1/2}} = A_1 A_2 + A_3 + A_4 + A_5\]
for
\begin{align}
A_1 &= \frac{\sigma}{(\theta \log \log n)^{1/2}}, \qquad  A_2 = \frac{\omega_{\mathcal{P}}(Y_n)-\mu}{\sigma},\qquad A_3=\frac{\mu-\theta\log \log n}{(\theta\log \log n)^{1/2}},\\
A_4 &= \frac{\sum_{p \in [2,T)\setminus I_n} \mathbf{1}_{p \mid Y_n}}{(\theta\log \log n)^{1/2}},\qquad A_5 = \frac{\sum_{p\in [T,n]}\mathbf{1}_{p \mid Y_n}}{(\theta\log \log n)^{1/2}}.
\end{align}
By \eqref{eq:omegamusigma}, the constants $A_1$ and $A_3$ tend to $1$ and $0$, respectively. We have observed  $\EE A_2^k \to C_k$. An integer $m$ can have at most $\log m/\log T$ prime factors of size at least $T$, implying $A_5 \ll (\log \log n)^{-1/6}$ and so $\EE A_5^k \to 0$ as $n \to \infty$. To handle $A_4$ we use \eqref{eq:cond2} which yields
\begin{multline}
\EE A_4^k = \frac{ \sum_{p_1,\ldots,p_k \in [2,T) \setminus I_n} \PP([p_1,\ldots,p_k]\textup{ divides } Y_n)}{(\theta\log \log n)^{k/2}} \\
\ll (\log \log n)^{-k/2}\sum_{\substack{\omega(a) \le k\\ p \mid a \implies p \in [2,T)\setminus I_n \\ \mu^2(a)=1}} \PP(a \textup{ divides }Y_n) =o(1).
\end{multline}
We have shown $\EE |A_3+A_4+A_5|^k \to 0$ and $\EE (A_1A_2)^k \to C_k$. By Hölder, \eqref{eq:mom} follows.
\end{proof}
Corollary \ref{cor:gs} may also be proved by generalizing an argument of Billingsley \cite{Billingsley}.
To deduce Theorem \ref{thm:e} from Corollary \ref{cor:gs} we apply it with $Y_n=X_n$ and $\theta=2$. This reduces matters to the following lemma, whose proof we postpone to the next section. \begin{lem}\label{lem:titch}
Let $\eta>0$ be sufficiently small. Fix $k \ge 1$. Then
\begin{equation}\label{eq:primes} 
\sum_{\substack{1 \le a \le n^{\eta}\\ \omega(a)\le k \\ (a,n)=1} }\mu^2(a)\left|\PP( a \textup{ divides } X_n)-\frac{2^{\omega(a)}}{a}+\frac{2\Lower_a}{\phi(a)\log n}\right| \ll_k \frac{(\log \log n)^{O_k(1)}}{\log n}
 \end{equation}
 where
 \begin{align}
\Lower_a&:= \sum_{d\mid a^{\infty}} g(d)\log d, \qquad g(d)=\prod_{p^k\mid \mid d} (-1)^{k-1}(p-2)(p-1)^{-k}.
\end{align}
\end{lem}
 We remark that the occurrence of the somewhat unexpected lower order term involving $D_a$ may merit further study.

\begin{proof}[Proof of Theorem \ref{thm:e}, assuming Lemma \ref{lem:titch}]
By Corollary \ref{cor:gs} it suffices to establish \eqref{eq:cond2} and \eqref{eq:cond1} (with $\theta=2$ and $Y_n=X_n$). To deduce this from Lemma \ref{lem:titch}, we must explain how to deal with two issues: the condition $(a,n)=1$ being present in the lemma, and the existence of the lower order term involving $\Lower_a$.
The contribution of $a$-s with $(a,n)>1$ to \eqref{eq:cond2} is negligible since \begin{equation}\label{eq:gcd}
\PP(a \textup{ divides }X_n)\ll n^{-1+o(1)}
\end{equation} 
uniformly for $a$ with $(a,n)>1$. The contribution of $a$-s with $(a,n)>1$ to \eqref{eq:cond1}  is negligible by \eqref{eq:gcd} and by 
\[\sum_{\substack{\omega(a)\le k\\ a\mid \mathcal{P}\\(a,n)>1}}\frac{2^{\omega(a)}}{a} \ll_k \bigg(1+\sum_{p \mid \mathcal{P}} \frac{1}{p} \bigg)^{k-1} \sum_{\substack{p \mid \mathcal{P}\\ p \mid n}} \frac{1}{p} \ll_k \exp(-\exp(\log^{\frac{1}{4}}\log n))\]
where we used $\sum_{p \le x} 1/p = \log \log x+ O(1)$ and the fact that $n$ has $\ll \log n$ prime factors. Next, we use $\sum_{d \mid a^{\infty}} |g(d)| =2^{\omega(a)-\mathbf{1}_{2 \mid a}}$ to bound $\Lower_a$ as
\[ \Lower_a \ll \sum_{p\mid a} \sum_{i\ge 1}\log (p^i) \sum_{p^i \mid \mid d \mid a^{\infty}}|g(d)| \ll 2^{\omega(a)}\sum_{p\mid a} \sum_{i\ge 1}\log (p^i) (p-2)(p-1)^{-i}\ll 2^{\omega(a)}\log a\]
and it follows that for any squarefree $\mathcal{Q}$,
\[ \sum_{\substack{\omega(a)\le k \\ a \mid \mathcal{Q}}} \frac{|\Lower_a|}{\phi(a) \log n} \ll_k \frac{1}{\log n}\bigg(1+\sum_{p \mid \mathcal{Q} }\frac{1}{p}\bigg)^{k-1}\sum_{p \mid \mathcal{Q} }\frac{\log p}{p}.\]
In the notation of Corollary \ref{cor:gs}, we specialize to $\mathcal{Q}=\mathcal{P}$ and $\mathcal{Q}=\prod_{p \in [2,T)\setminus I_n}p$ and find $\Lower_a$ contributes negligibly to \eqref{eq:cond2} and \eqref{eq:cond1} by invoking $\sum_{p \le x} \tfrac{\log p}{p} = \log x + O(1)$ and $\sum_{p \le x} = \log \log x + O(1)$. 
\end{proof}
\section{Proof of Lemma \ref{lem:titch}}
We shall work with arithmetic sums instead of probabilities. Given $a \ge 1$ let \[A_a(n):=\sum_{p<n:\, a \mid n-p}2^{\omega(n-p)}\]
so that $\PP( a \textup{ divides } X_n) = A_a(n)/A_1(n)$. Lemma \ref{lem:titch} is equivalent to
\begin{equation}\label{eq:primes2} 
\sum_{\substack{1 \le a \le n^{\eta}\\ \omega(a)\le k \\ (a,n)=1} }\mu^2(a)\bigg|A_a(n) - \phi(n)\bigg(\frac{2^{\omega(a)}}{a}-\frac{2\Lower_a}{\phi(a)\log n}\bigg)\bigg| \ll_k \frac{n}{\log n}(\log \log n)^{O_k(1)}
 \end{equation}
 where we utilized \eqref{eq:elliott}, $\phi(n) \gg n/\log \log n$ and $\sum_{p \le x}1/p = \log \log x + O(1)$ to replace $A_1(n)$ by $\phi(n)$. We write $2^{\omega(n-p)}$ as a divisor sum using $2^{\omega(m)}= \sum_{d\mid m}\mu^2(d)$ and $\mu^2(d) = \sum_{e^2 \mid d} \mu(e)$, separate the contribution of large square divisors, and apply divisor switching. We have
\[2^{\omega(n)} = \sum_{n=ds}\mu^2(d) =\sum_{n=e^2ds}\mu(e)=\sum_{\substack{n=e^2ds\\ e>V}}\mu(e)+\sum_{\substack{n=e^2ds\\ e\leq V\\ d\leq D}}\mu(e)+\sum_{\substack{n=e^2ds\\ e\leq V\\ s<  n/(e^2 D)}}\mu(e)\]
with the choices
\[    V=n^{2\eta}, \qquad D=n^{1/2+2\eta}.\]
Then, given $a\geq 1$, we split $A_a(n)$ accordingly:
\[    A_a(n)=\sum_{\substack{p<n\\ a|n-p}}\bigg(\sum_{\substack{n-p=e^2ds\\ e>V}}\mu(e)+\sum_{\substack{n-p=e^2ds\\ e\leq V\\ d\leq D}}\mu(e)+\sum_{\substack{n-p=e^2ds\\ e\leq V\\ s<  (n-p)/(e^2 D)}}\mu(e) \bigg)=:S_0(a)+S_1(a)+S_2(a).\]
We write $\pi(x;q,b)$ for the number of primes in $[1,x]$ that are congruent to $b$ modulo $q$. With this notation we have
\begin{align}
\label{eq:divswitch0}  S_0(a)&=\sum_{\substack{V<e\leq \sqrt{n}\\ d\leq n}}\mu(e)\pi(n-1;[de^2,a],n),\\
\label{eq:divswitch1}   S_1(a)&=\sum_{\substack{ e\leq V\\ d\leq D}}\mu(e) \pi(n-1;[de^2,a],n), \\
\label{eq:divswitch2} S_2(a)&=\sum_{\substack{e\leq V\\ s<  (n-2)/(e^2 D)}}\mu(e)\pi(n-se^2D-1;[se^2,a],n).
\end{align}
Write $E(m;q,a)=\pi(m;q,a)-\frac{\textup{Li}(m) \mathbf{1}_{(q,a)=1}}{\varphi(a)}$ and accordingly
\begin{align*}
    S_i(a)&=M_i(a)+E_i(a) \qquad (i=1,2)
\end{align*}
for
\begin{align}
M_1(a) &= \textup{Li}(n-1) \sum_{\substack{e \le V \\ d \le D\\(ed,n)=1}}\frac{\mu(e)}{\phi([de^2,a])} , \qquad E_1(a) = \sum_{\substack{e \le V \\ d \le D\\(ed,n)=1}}\mu(e) E(n-1;[de^2,a],n),\\ 
M_2(a) &= \sum_{\substack{e \le V \\ d<(n-2)/(e^2 D)\\(de,n)=1}} \frac{\mu(e)\textup{Li}(n-de^2D-1)}{\phi([de^2,a])},\qquad
E_2(a) = \sum_{\substack{e \le V \\ d<(n-2)/(e^2 D)\\(de,n)=1}} \mu(e)E(n-de^2D-1;[de^2,a],n).
\end{align}

The rest of the proof is separated into four parts which together establish \eqref{eq:primes2}.
\begin{lem}\label{lem:S0}
We have
$\sum_{a \le n^{\eta}}|S_0(a)|\ll n^{1-\frac{\eta}{2}}$.
\end{lem}
\begin{lem}\label{lem:E1}
For every $B>0$, $\sum_{a \le n^{\eta}}|E_1(a)|\ll_B  n(\log n)^{-B}$.
\end{lem}
\begin{lem}\label{lem:E2}
For every $B>0$,
$\sum_{a \le n^{\eta}}|E_2(a)|\ll_B  n(\log n)^{-B}$.
\end{lem}
\begin{lem}\label{lem:m}
Fix $k \ge 1$. Suppose $a \le n^{\eta}$, $\omega(a) \le k$ and $\mu^2(a)=1$. We have
\begin{align}\label{eq:M1M2} M_{1}(a)+M_{2}(a) =\phi(n) \bigg( \frac{2^{\omega(a)}}{a} - \frac{2D_a}{\phi(a)\log n}\bigg) +O_k\bigg( \frac{n (\log \log n)^2 }{a \log n} \bigg).
\end{align}
\end{lem}
\subsection{Proof of Lemma \ref{lem:S0}}
The upper bound $\pi(n-1;q,n)\leq n/q$ gives
\[    \left|S_0(a)\right|\leq \sum_{\substack{V<e\leq \sqrt{n}\\ d\leq n}}\sum_{\substack{p<n\\ [de^2,a]|n-p}}1 \leq \sum_{\substack{e>V\\ d\leq n}} \frac{n}{[de^2,a]} \le \sum_{\substack{e>V\\ d\leq n}} \frac{n}{de^2} \ll \frac{n \log n}{V}\]
which implies the result.
\subsection{Proof of Lemma \ref{lem:E2}}
Isolating the contribution of $(se ^2a,n)\neq 1$, we have
\begin{align*}
    \sum_{a \le n^{\eta}}|E_2(a)|&=\sum_{a \le n^{\eta}} \bigg|\sum_{\substack{e\leq V\\ s< (n-2)/(e^2D)} }\mu(e) E(n-se^2D-1;[se^2,a],n)\bigg|\\
    &\le \sum_{\substack{d\leq n^{1/2 - \eta} \\ (d,n)=1}}\tau^2(d)\max_{y\leq n} \left|E(y;d,n) \right| + n^{o(1)}
\end{align*}
by the divisor bound, where $d$ stands for $[se^2,a]$. The conclusion follows by applying Cauchy--Schwarz and then the Bombieri--Vinogradov Theorem \cite[Lemma]{ElliottH}.
\subsection{Proof of Lemma \ref{lem:E1}}
As $E_1(a)$ requires us to break the $1/2$-barrier in the level of distribution, it is the more delicate error term.
We have
\begin{align*}
\sum_{a \le n^{\eta}}|E_1(a)|&=\sum_{a \le n^{\eta}}\bigg|\sum_{e \le V} \mu(e)\sum_{d \le D}E(n-1;[de^2,a],n)\bigg|\\ 
&= \sum_{a \le n^{\eta}}\bigg|\sum_{\substack{e \le V\\ b\mid (ae)^\infty}} \mu(e)\sum_{\substack{d \le D\\  b \mid d \\ (d/b,ae)=1}}E(n-1;[de^2,a],n)\bigg|\\
&=\sum_{a \le n^{\eta}}\bigg|\sum_{\substack{e \le V\\ b\mid (ae)^\infty}} \mu(e)\sum_{\substack{d' \le D/b \\ (d',ae)=1}}E(n-1;d'[be^2,a],n)\bigg|,
\end{align*}
where we substituted $d'=db$. Let $E_{1,1}$ denote the contribution to the above with $b\le 
J=n^{2\eta}$ and $E_{1,2}$ the remainder. 
For $E_{1,1}$ we apply the triangle inequality to arrive at
\begin{align*}
    E_{1,1}\leq \sum_{a \le n^{\eta}} \sum_{e \le V}\sum_{\substack{ b\leq J\\ b \mid (ae)^{\infty}} }\mu^2(e)\bigg|\sum_{\substack{d' \le D/b\\ (d',ae)=1}}E(n-1;d'[be^2,a],n)\bigg|.
\end{align*}
To make the summation range of $d'$ independent of the other variables, we apply a standard finer-than-dyadic decomposition on the summation over $d'$ (see e.g.~the proof of \cite[Thm.\ 1.1]{Assing}). It thus suffices to estimate
\begin{align*}
    E_{1,1,D'}=\sum_{a \le n^{\eta}} \sum_{e \le V}\sum_{\substack{ b\leq J\\ b \mid (ae)^{\infty}} }\mu^2(e)\bigg|\sum_{\substack{d\leq D' \\ (d',ae)=1}}E(n-1;d'[be^2,a],n)\bigg|
\end{align*}
for any $D'\leq D$. If $[be^2,a]=g$ then $a$, $b$ and $e$ divide $g$ and the condition $(d',ae)=1$ is equivalent to $(d',g)=1$. Thus 
\[ E_{1,1,D'}\le  \sum_{g \le n^{8\eta}} \tau(g)^3\bigg| \sum_{\substack{d' \le D'\\(d',g)=1}} E(n-1;d'g,n)\bigg|.\]
We quote \cite[Prop.\ 6.1]{Blomer} specialised to $a_2=c=c_0=1$.

\begin{proposition}\label{PrimeAp}
	There exists some absolute constant $\varpi>0$ such that the following holds. 
	Let $x\geq 2$, $d, C\in \mathbb N,  a\in \mathbb{Z}\setminus \{0\}$ such that  
	\begin{align}
	Q\leq x^{1/2+\varpi},\quad |a|\leq x^{1+\varpi}, \quad |\lambda_d|\ll \tau(d)^C
	\end{align} 
hold. Then for any $B>0$ we have 
	\begin{align}
	\sum_{d\leq x^{\varpi}}\lambda_d\sup_{\substack{w \bmod d\\(w,d)=1}}\Big|\sum_{\substack{q\leq Q\\ (q,ad)=1}}\Big(\sum_{\substack{n\leq x\\n\equiv a\bmod q\\ n\equiv w\bmod d}}\Lambda(n)-\frac{1}{\phi(qd)}\sum_{\substack{n\leq x\\ (n, qd)=1}}\Lambda(n)\Big)\Big|\ll_{C,B} x(\log x)^{-B}.
	\end{align}
\end{proposition}
As commented after \cite[Prop.\ 6.1]{Blomer}, this holds for the characteristic function of the primes replacing the von Mangoldt function and so implies $E_{1,1,D'}\ll_B n (\log n)^{-B}$. 

To prove Lemma \ref{lem:E1}, it remains to estimate $E_{1,2}$ and we claim
\begin{align}\label{eq:E12bound}
    E_{1,2}\ll n^{1-\eta/2}.
\end{align}
By the  trivial bound $E(n-1;q,a)\ll 1+n/q$ we have
\[  E_{1,2}\ll n\sum_{\substack{a\leq n^\eta \\ e\leq V \\ d'< D/J}} \frac{1}{d'} \sum_{\substack{J<b\leq D/d'\\ b\mid (ae)^\infty }}\frac{1}{[be^2,a]}.\]
Since $[be^2,a] = be^2a/(be^2,a)$, by dyadic decomposition to get \eqref{eq:E12bound} it suffices to show
\begin{equation}\label{eq:dyadicaeb}
\sum_{\substack{a \sim A\\ e \sim V}} \sum_{\substack{b \sim J \\ b \mid (ae)^{\infty}}} (be^2,a) \ll(2+\log AV)^{C_{\varepsilon}} J^{\varepsilon} A^2 V
\end{equation}
for all $\varepsilon>0$. Here $C_{\varepsilon}>0$ depends only on $\varepsilon>0$ (later occurrences of $C_{\varepsilon}$ might have a different value), $a \sim A$ means $a \in [A,2A)$ and $e \sim V$, $b \sim J$ should be interpreted in the same way. Estimate~\eqref{eq:dyadicaeb} implies $E_{1,2} \ll n^{1+\eta}(\log n)^{C_{\varepsilon}}/J^{1-\varepsilon} \ll n^{1-\frac{\eta}{2}}$ and so ~\eqref{eq:E12bound} if $\varepsilon$ is sufficiently small.

To demonstrate \eqref{eq:dyadicaeb}, let $h=(be^2,a)$. If $h \mid be^2$ then $h=h_1h_2$ for some (not necessarily unique) $h_1 \mid b$ and $h_2 \mid e^2$. This implies $h'_2 \mid e$ for $h'_2 = \prod_{p^k \mid \mid h_2} p^{\lceil k/2\rceil}$. We can bound \eqref{eq:dyadicaeb} by
\[ \le \sum_{h_1 h_2 \mid a \sim A} h_1 h_2  \sum_{\substack{h'_2 \mid e \sim V \\ b \sim J \\ h_1 \mid b \mid (ae)^{\infty}}} 1 \le\sum_{h_1 h_2  \mid a \sim A } h_1 h_2  \sum_{\substack{e' \sim V/h'_2 \\ b' \sim J/h_1 \\ b'  \mid (ae')^{\infty}}} 1 .\]
By Rankin's trick,
\[ \sum_{b \le B, \, b \mid m^{\infty}} 1 \le B^{\varepsilon}\prod_{p \mid m}(1-p^{-\varepsilon})^{-1} \le B^{\varepsilon} \exp(C_{\varepsilon} \sum_{p \mid m} p^{-\varepsilon}) \le B^{\varepsilon} \exp(C_{\varepsilon}\omega(m))\]
for every $\varepsilon>0$. Hence
\[\sum_{\substack{e' \sim V/h'_2 \\ b' \sim J/h_1 \\ b'  \mid (ae')^{\infty}}} 1 \ll \sum_{e' \sim V/h'_2} (J/h_1)^{\varepsilon} \exp(C_{\varepsilon} (\omega(a)+\omega(e'))) \ll (J/h_1)^{\varepsilon} V(2+\log V)^{C_{\varepsilon}}\exp(C_{\varepsilon} \omega(a))\]
and so, since $h_1 h_2 \ll A$,
\begin{align} \sum_{\substack{a \sim A\\ e \sim V}} \sum_{\substack{ b \sim J \\ b \mid (ae)^{\infty}}} (be^2,a) &\ll (2+\log V)^{C_{\varepsilon}} J^{\varepsilon}V \sum_{h_1 h_2 \mid a \sim A } h_1^{1-\varepsilon} h_2 \exp(C_{\varepsilon} \omega(a))\\
&\ll (2+\log V)^{C_{\varepsilon}} J^{\varepsilon} AV\sum_{a \sim A} \exp(C_{\varepsilon}\omega(a)) \tau(a)^2 \ll( 2+\log AV)^{C_{\varepsilon}} J^{\varepsilon} A^2V .
\end{align}
\subsection{Proof of Lemma \ref{lem:m}}
Let $1 \le a \le n^{\eta}$ be squarefree. By completing the $e$ range we have
\begin{align}
M_1(a) &=  \textup{Li}(n-1) \sum_{\substack{e \ge 1 \\ d \le D \\(de,n)=1}}\frac{\mu(e)}{\phi([de^2,a])} +  O\big(\frac{n}{V}\big),\quad 
M_2(a) =  \sum_{\substack{e \ge 1 \\ d<(n-2)/(e^2 D)\\(de,n)=1}} \frac{\mu(e)\textup{Li}(n-de^2D-1)}{\phi([de^2,a])} + O\big( \frac{n}{V}\big).
\end{align}
The $n/V$ error is more than sufficient for Lemma \ref{lem:m}. We now decompose, up to admissible error terms both $M_1(a)$ and $M_2(a)$ as follows. Let
\begin{align*}
    M_1(a)&= M_{1,1}(a)+M_{1,2}(a) + O(n/V),\\
    M_2(a)&= M_{2,1}(a)+M_{2,2}(a) + O(n/V),
\end{align*}
where we split according to $de^2\leq D$ or not
\begin{align*}
    M_{1,1}(a)&=\textup{Li}(n-1) \sum_{\substack{de^2\leq D \\(de,n)=1}}\frac{\mu(e)}{\phi([de^2,a])},\\
     M_{1,2}(a)&=\textup{Li}(n-1) \sum_{\substack{de^2> D\\ d\leq D \\(de,n)=1}}\frac{\mu(e)}{\phi([de^2,a])},
\end{align*}
and where for $M_2$ we use $\textup{Li}(n-de^2D-1) =  \textup{Li}(n-1) + O(de^2 D/\log n)$ so that
\begin{align}
M_{2,1}(a) &= \textup{Li}(n-1) \sum_{\substack{de^2<(n-2)/D \\ (de,n)=1}} \frac{\mu(e)}{\phi([de^2,a])},\\ 
\label{eq:m22}M_{2,2}(a)&\ll \frac{D}{\log n} \sum_{de^2<(n-2)/D } \frac{de^2\mu^2(e)}{\phi([de^2,a])}. 
\end{align} 
We claim that for square-free $1 \le a \le n^{\eta}$ with $\omega(a)\le k$,
\begin{align}\label{eq:M1222bound}
M_{1,2}(a),M_{2,2}(a)&\ll_k \frac{n \log \log n}{a\log n},\\
\label{eq:M1112asymp} M_{1,1}(a)+M_{1,2}(a)\\
   =\textup{Li}(n-1)\frac{2^{\omega(a)}}{a}&\left( \frac{\phi(n)}{n} \log n  + O_k((\log \log n)^2 )\right)-\frac{2\textup{Li}(n-1)\phi(n)}{n}\frac{\Lower_a}{\phi(a)},
\end{align}
which implies Lemma \ref{lem:m}, as we can replace $\textup{Li}(n-1)$ by $n/\log n + O(n/\log^2 n)$, incurring an acceptable error.

We now bound $M_{1,2}(a)$. We have
\[ M_{1,2}(a) \ll \frac{n}{\log n}  \sum_{D<de^2 \le De^2}\frac{\mu^2(e)}{\phi([de^2,a])} \ll \frac{n\log \log n}{\log n} \sum_{D< de^2 \le De^2} \frac{1}{[de^2,a]}.\]
Since $a$ is squarefree, letting $(de^2,a)=g$ there are $g_1,g_2$ (not necessarily unique) with $g=g_1g_2$, $g_1\mid e$ and $g_2 \mid d$. Thus,
\begin{align}
\sum_{D < de^2 \le De^2} \frac{1}{[de^2,a]}&= a^{-1} \sum_{e \ge 1} e^{-2} \sum_{D/e^2 < d \le D} (de^2,a)d^{-1}\\
&\le a^{-1} \sum_{g_1 g_2 \mid a}g_1 g_2 \sum_{g_1 \mid e^2} e^{-2} \sum_{\substack{D/e^2 < d \le D\\ g_2 \mid d }}  d^{-1}\\
& \ll a^{-1}\sum_{g_1 g_2 \mid a}g_1^{-1}(1+ \log g_1)\ll a^{-1}3^{\omega(a)}.
\end{align}
Since $\omega(a)\le k$ in Lemma \ref{lem:m}, this is sufficient for \eqref{eq:M1222bound}.

We now bound $M_{2,2}(a)$. Writing $m=de^2$, we have
\begin{align}
    \label{eq:m222}M_{2,2}(a) &\ll \frac{D}{\log n} \sum_{m< n/D} \frac{m 2^{\#\{p^2 \mid m\}}}{\phi([m,a])}.
\end{align} 
We let $g=(a,m)$ in the right-hand side of \eqref{eq:m222} and observe $1/\phi([m,a]) \le 1/\phi(a)\phi(m/g)$ to find
\[M_{2,2}(a) \ll \frac{D}{\log n \phi(a)} \sum_{g \mid a}g\sum_{g \mid m < n/D }\frac{(m/g) 2^{\#\{p^2 \mid m\}}}{\phi(m/g)}.\]
Let $S(x):=\sum_{m \le X} (m/\phi(m))2^{\#\{p^2 \mid m\}}$. This sum is $ \ll X$ \cite[Thm.\ 2.14]{MV}. Hence
\[ M_{2,2}(a) \ll \frac{D}{\log n \phi(a)} \sum_{g \mid a} g2^{\omega(g)}S\left( \frac{n}{Dg}\right) \ll \frac{n}{\log n\phi(a)} 3^{\omega(a)},\] 
which suffices for \eqref{eq:M1222bound}.

We now prove \eqref{eq:M1112asymp}. We have
\begin{align*}
    M_{1,1}(a)  = \textup{Li}(n-1) \sum_{\substack{m \le D\\(m,n)=1}} \frac{\mu^2(m)}{\phi([m,a])}
\end{align*}
and
\begin{align*}
    M_{2,1}(a) =\textup{Li}(n-1) \sum_{\substack{m<(n-2)/D \\ (m,n)=1}} \frac{\mu^2(m)}{\phi([m,a])}.
\end{align*}
We claim that
\begin{align}\label{eq:final}
    \sum_{\substack{ m<X\\ (m,n)=1}} \frac{\mu^2(m)}{\phi([m,a])} = \frac{2^{\omega(a)}}{a}\left( \frac{\phi(n)}{n} \log X  + O_k((\log \log n)^2 )\right)-\frac{\phi(n)}{n}\frac{\Lower_a}{\phi(a)}
\end{align}
holds uniformly for $a\le n^{\eta}$ and $X\ge n^{1/3}$ with $(a,n)=1$, $\mu^2(a)=1$ and $\omega(a) \le k$. With this we can evaluate $M_{1,1}(a)+M_{2,1}(a)$ by applying \eqref{eq:final} with $X=D$ and $X=n/D$ to obtain \eqref{eq:M1112asymp}.

 
To show \eqref{eq:final}, we introduce the multiplicative function \[ f_{n,a}(m) := \mathbf{1}_{(m,n)=1} \mu^2(m)\phi(a)/\phi([m,a]).\]
Elliott and Halberstam established \cite[p.\ 202]{ElliottH}
\begin{equation}\label{eq:EH} \sum_{m<X} f_{n,1}(m)= \frac{\phi(n)}{n} \log X  + O((\log \log n)^2)
\end{equation}
for $n \le X^5$. We have
$f_{n,a} = f_{n,1} * g_{n,a}$ where $g_{n,a}$ is multiplicative, supported on odd divisors of $a^{\infty}$ and defined on $p^k$ via
$g_{n,a}(p^k)=(-1)^{k-1}(p-2)/(p-1)^{k}$. Hence, by \eqref{eq:EH},
\begin{align} \label{eq:mt}
    \sum_{\substack{ m<X\\ (m,n)=1}} \frac{\mu^2(m)}{\phi([m,a])} = \sum_{m<X} \frac{f_{n,a}(m)}{\phi(a)}= \frac{\phi(n)}{n\phi(a)}\sum_{d \le X/n^{1/5}} g_{n,a}(d) \log (X/d)+  O\left(\frac{E}{\phi(a)}\right)
    \end{align}
    for 
    \begin{equation}\label{eq:Edef} E := \sum_{d <X} |g_{n,a}(d)| (\log \log n)^2+\sum_{d>X/n^{1/5}}|g_{n,a}(d)|\log d.
\end{equation}
To bound the first sum in $E$ we use $\sum_{d\ge 1} |g_{n,a}(d)|=2^{\omega(a)-\mathbf{1}_{2 \mid a}}\ll_k 1$. We turn to the second sum in $E$. If $d$ is in the support of $g_{n,a}$ then
\[ |g_{n,a}(d)| \le \prod_{p \mid d} p \prod_{p^i \mid \mid d} p^{-i} \prod_{p^i \mid \mid d}\left(1+\frac{1}{p-1}\right)^{i-1} \le \frac{a}{d}\prod_{p^i \mid \mid d}\left(1+\frac{1}{p-1}\right)^{i-1}\le \frac{a}{d}  (3/2)^{\log_3 d}.\]
If $\omega(a)\le k$ then there are $\ll_k (\log T)^k$ integers $d \in [T,2T)$ with $d \mid a^{\infty}$, so
\begin{equation}\label{eq:dyadic}
\sum_{d\in [T,2T)} |g_{n,a}(d)|  \ll_k \frac{a(\log T)^k}{T} T^{\log(3/2)/\log 3} .
\end{equation}
This implies that under our assumptions, $E/\phi(a) \ll_k (\log \log n)^2/a$.
With the same $E$ as in \eqref{eq:Edef}, we express the main term in \eqref{eq:mt}  as
\[ \sum_{ d \le X/n^{1/5}} g_{n,a}(d) \log (X/d) = \log X\prod_{p \mid a} \big( \sum_{i=0}^{\infty} g_{n,a}(p^i)\big) - \sum_{d\ge 1} g_{n,a}(d) \log d + O(E)\]
times $\phi(n)/(n\phi(a))$.
The product is $2^{\omega(a)}\phi(a)/a$ and the $d$-sum is $\Lower_a$.



\bibliographystyle{alpha}
\bibliography{references}

\Addresses

\end{document} 


