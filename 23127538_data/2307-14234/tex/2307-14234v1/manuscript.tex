%===============================================================================
% $Id: ifacconf.tex 19 2011-10-27 09:32:13Z jpuente $  
% Template for IFAC meeting papers
% Copyright (c) 2007-2008 International Federation of Automatic Control
%===============================================================================
\documentclass{ifacconf}

\usepackage{graphicx,subcaption}      % include this line if your document contains figures
\graphicspath{{Figures/}}
\let\printglossary\relax
\let\theglossary\relax
\let\endtheglossary\relax
\usepackage{amsmath}
\usepackage{amssymb}
\usepackage{helvet}
\usepackage[acronym, toc]{glossaries}
\usepackage{siunitx}
\usepackage{tikz,bm,color}
\usepackage{enumerate}
\usepackage{mathtools, cuted}
\usepackage{pgfplots}
\usepackage{pgfplotstable}
\usetikzlibrary{external,calc,patterns,decorations.pathmorphing,decorations.markings,decorations.text,arrows,shapes,positioning, backgrounds, intersections}
\usetikzlibrary{spy}
\usepgfplotslibrary{fillbetween}
\usepackage{caption}
\captionsetup{width=.75\textwidth}
\usepackage{hhline}
\usepackage{ctable}
\usepackage{pbox}
%\usepackage{subfig}

\definecolor{cadmiumgreen}{rgb}{0.0, 0.42, 0.24}
\definecolor{PIcolor}{rgb}{0, 0, 1.0}
\definecolor{STSMCcolor}{rgb}{0.929, 0.427, 0.067}
\definecolor{NACcolor}{rgb}{0.0, 0.42, 0.24}
\definecolor{InIcolor}{rgb}{0.247, 0.475, 0.749}
\definecolor{limitColor}{rgb}{0.247, 0, 0}
\pgfplotsset{compat=newest}
\tikzexternalize[prefix=myExtTikz/]
\usepackage{balance}

\newtheorem{problem}{Problem}
\newtheorem{assumption}{Assumption}
\newtheorem{theorem}{Theorem}
\newtheorem{definition}{Definition}
\newtheorem{remark}{Remark}
\newtheorem{proposition}{Proposition}
\newtheorem{property}{Property}
%\renewcommand\qed{$\blacksquare$}
\usepackage{algorithm}
\usepackage{algpseudocode}
\usepackage[normalem]{ulem}
\usepackage{slashbox}
\newcommand\T{\rule{0pt}{2.6ex}}       % Top strut
\newcommand\B{\rule[-1.2ex]{0pt}{0pt}} % Bottom strut

\newcommand\TT{\rule{0pt}{3.2ex}}       % Top strut
\newcommand\BB{\rule[-1.8ex]{0pt}{0pt}} % Bottom strut

\newacronym{AMB}{AMB}{Active Magnetic Bearings}
\newacronym{PMB}{PMB}{Passive Magnetic Bearings}
\newacronym{PMSM}{PMSM}{Permanent Magnet Synchronous Motor}
\newacronym{PMSMs}{PMSMs}{Permanent Magnet Synchronous Motors}
\newacronym{SMC}{SMC}{Sliding Mode Control}
\newacronym{ISS}{ISS}{Input-to-State Stable}
\newacronym{SISO}{SISO}{Single-Input Single-Output}
\newacronym{PID}{PID}{Proportional-Integral-Differential}
\newacronym{PI}{PI}{Proportional-Integral}
\newacronym{P}{P}{Proportional}
\newacronym{STSMC}{STSMC}{Super-twisting Sliding Mode controller}
\newacronym{PE}{PE}{Persistence of Excitation}
\newacronym{ULES}{ULES}{Uniformly Locally Exponentially Stable}
\newacronym{GES}{GES}{Globally Exponentially Stable}
\newacronym{UB}{UB}{Uniformly Bounded}
\newacronym{UGB}{UGB}{Uniformly Globally Bounded}
\newacronym{UGAS}{UGAS}{Uniformly Globally Asymptotically Stable}
\newacronym{UAS}{UAS}{Uniformly Asymptotically Stable}
\newacronym{MAE}{MAE}{Maximum Absolute Error}
\newacronym{RMSE}{RMSE}{Root Mean Square Error}
\newacronym{VSC}{VSC}{Variable-Structure Control}

\makenoidxglossaries
\usepackage{natbib}        % required for bibliography
%===============================================================================
\begin{document}
\begin{frontmatter}

\title{Sliding Mode Control of Active Magnetic Bearings - A Cascaded Architecture} 
% Title, preferably not more than 10 words.

%\thanks[footnoteinfo]{Sponsor and financial support acknowledgment
%goes here. Paper titles should be written in uppercase and lowercase
%letters, not all uppercase.}

\author[First]{Dimitrios Papageorgiou} 
\author[Second]{Ilmar Santos}

\address[First]{Technical University of Denmark, Department of Electrical and Photonics Engineering, Elektrovej 326, 2800 Kgs Lyngby, Denmark (e-mail: dimpa@dtu.dk).}
\address[Second]{Technical University of Denmark, Department of Civil and Mechanical Engineering, Koppels All{\'e} 404, 2800 Kgs Lyngby, Denmark (e-mail: ilsa@dtu.dk).}

\begin{abstract}                % Abstract of not more than 250 words.
	Accurate and robust positioning of rotor axle is essential for efficient and safe operation of high-speed rotational machines with active magnetic bearings. This study presents a cascaded nonlinear control strategy for vertical axial positioning of an active magnetic bearing system. The proposed scheme employs two sliding mode controllers for regulating rotor vertical position and current and an adaptive estimator to invert the nonlinear input mapping. Uniform asymptotic stability is proven for the closed-loop system and the efficacy and performance of the proposed design is evaluated in simulation.
\end{abstract}

\begin{keyword}
	Active magnetic bearings, sliding mode control, cascaded control, nonlinear input mapping inversion
\end{keyword}

\end{frontmatter}
%===============================================================================

\section{Introduction}
	% Figure environment removed

\section{Introduction}
Automatic 3D reconstruction of clothed humans using image inputs has gained increasing significance due to its potential applications in a wide array of AR/VR scenarios. High-fidelity reconstructions typically depend on sophisticated capture systems, which are developed with dense camera arrays~\cite{collet2015high,joo2015panoptic,joo2018total}, programmable light-stages~\cite{Vlasic2009, guo2019relightables}, and depth sensors~\cite{newcombe2011kinectfusion,DoubleFusion,BodyFusion,dou2016fusion4d,newcombe2015dynamicfusion}. However, stringent capture environments equipped with complex hardware pose significant challenges for consumer-level applications.


In this context, considerable research effort has been dedicated to developing methods that allow for more flexible capture configurations, such as utilizing a few RGB inputs. Among these works, learning implicit functions \cite{iccv2020PIFu, saito2020pifuhd, hong2021stereopifu} has proven effective in achieving highly detailed reconstructions by integrating the advancements of deep neural networks. These methods employ large multi-layer perceptrons (MLPs) to predict the occupancy probability or truncated signed distance function (TSDF) value of every queried 3D point based on its associated local feature, which is extracted from images. They can recover a continuous surface at arbitrary resolutions without topology restrictions.


However, in typical MLP-based implicit networks, the occupancy or TSDF value at each location is solved independently with planar image features, rendering them less capable of addressing challenging cases such as occlusions. Consequently, these methods suffer from generalization and robustness issues, particularly when tackling strong occlusions caused by large motion or multiple interacting humans. 
Some follow-up studies  \cite{zheng2021deepmulticap,zheng2021pamir,huang2020arch} utilize an extra geometric model, SMPL~\cite{Loper2015}, to improve robustness by introducing strong shape priors. 
Their success typically relies on the assumption of geometrical similarity \cite{huang2020arch} between the shape prior and target reconstruction, making them intractable for handling complex cases with loose clothes and sensitive to errors in SMPL model fitting.



%\ping{this paragraph sounds like `TSDF is better than MLP/SMPL, and we use TSDF to solve the problem'. But in Sec 3, we are telling a different story, saying `MLP needs a 3D convolutional encoder'. We need to make these two sections consistent.}\sicong{I think in this paragraph we claim that the TSDF}


%We opt for Trucated Signed Distance Funtion (TSDF) volumetric representations as they are naturally suitable for convolution operations, which have shown remarkable performance for learning hierarchical features on 2D visual perception tasks \cite{SunXLW19}. 
%Meanwhile, TSDF also describes the gradual geometry change around shape surface, which is not reflected by occupancy volume. 

We instead revisit the 3D volumetric representation and resort to 3D convolutional neural networks (CNNs) for feature learning, due to their impressive performance in feature learning and the ability to incorporate spatial context. However, volumetric methods and 3D convolution involve discretization, which might raise concerns regarding whether a discretized volume can preserve subtle geometric details as continuous representations learned in implicit functions. We investigate the relationship between volume resolution and quantization error on synthetic data by converting target mesh objects to TSDF volumes, as shown in Figure~\ref{fig:quantization_error}. We observe that the quantization errors are significantly reduced by increasing volume resolution and become nearly negligible when reaching a relatively high resolution (e.g., 512 or higher). In other words, achieving fine-detailed reconstruction is not supposed to be restricted by the use of volume representations as long as a proper volume resolution is utilized. Therefore, we present a method with high-resolution feature volumes, e.g., 256 and 512, while traditional volumetric methods \cite{varol18_bodynet,gilbert2018volumetric} are often limited to much lower resolutions, such as 32 or 128.



On the other hand, an increase in volume resolution may lead to a cubic growth of memory overhead \cite{8100085}. Reducing memory costs while guaranteeing the granularity of volumetric representations is necessary for pursuing high-quality reconstruction. Thus, we adopt a coarse-to-fine approach and cull away irrelevant voxels to build a sparse high-resolution feature volume. At the coarse level, the network computes an initial TSDF by applying a U-Net with sparse 3D CNN \cite{3DSemanticSegmentationWithSubmanifoldSparseConvNet} on the sparse feature volume, which is carved by a visual hull. Through our experiments, it turns out that more than 95\% of the volume grids are discarded by the visual hull culling, making the sparse 3D CNN efficient. At the fine level, the network focuses on a narrow band near the zero-level set of the initial TSDF and discretizes the narrow band with smaller voxels. By employing this narrow-band culling, we further shrink the sampling space, resulting in a relatively small range of grid numbers (usually 300K--500K in our experiments) even with a high volume resolution of 512. The remaining voxels in the narrow band are associated with features that fuse high-frequency information from the computed normal maps upon the low-frequency shape from the coarse level to compute the TSDF at high resolution. The final mesh is then extracted from the TSDF using the Marching-Cube algorithm ~\cite{Lorensen87marchingcubes}.
% Different from the u-net sturcture to preserve global topology context, we then apply a shallow 3dcnn to compute the final TSDF $D_{final}$ which contain more local geometry detail.




% \ping{this paragraph can be expanded. It is an important contribution and often ignored by other works. stress on the novel idea of regressing blending weights instead of colors}

In addition to geometry, high-quality mesh texture is also a crucial factor contributing to visual appearance. Directly computing a color field in 3D space, as in \cite{iccv2020PIFu}, struggles to capture high-frequency texture details, while the neural radiance field (NeRF) \cite{yu2020pixelnerf} or the DoubleField~\cite{shao2022doublefield} require expensive per-instance optimization and are often unstable for sparse input images. In contrast, we adopt an image-based rendering approach to compute a texture atlas map, which is efficient and widely supported in existing computer graphics tools. 
Specifically, we compute a blending weight at each 3D point on the mesh surface to determine its color as a weighted average of the colors at its image projections. The blending weights can be computed at a relatively coarse resolution, e.g., 512 volume resolution in our case, and leave texture details to the high-resolution images, such as 1K or 2K. Unlike previous methods that generate blurry texturing results under sparse input, our method generalizes well on both synthetic and real data with just a few input views. 
Figure~\ref{fig:teaser} shows two examples reconstructed by our method. Despite the challenging garment, pose, and occlusion, our method recovers faithful shape, normal, and texture on the right.

%with a wide variety of poses and clothing styles, and it is also adaptive to handle input image with arbitrary resolutions.
%\sicong{For this concern we claim that when the resolution of dicretized volume meets certain threshold (which is 256 in our experiment), the quantization error can be neglected.} 



In summary, the main contributions of this paper are as follows:
\begin{itemize}
\vspace{-0.1in}
  \item 
  We revisit the 3D volumetric representation and demonstrate that it can support clothed human reconstruction with equal or even better performance compared to implicit representation. 
  \item 
  We develop a memory and computation-efficient method for high-resolution volumetric reconstruction using sophisticated sparse 3D CNN, coarse-to-fine estimation, and voxel culling by visual hull and narrow bands. 
  \item 
  We introduce a novel method to compute a texture atlas map, which captures rich appearance details from high-resolution input images.
  \item 
  We achieve impressive results on standard benchmark datasets Twindom and MultiHuman, significantly reducing the point-2-surface (P2S) precision to approximately 0.2cm from just six input views, with more than $50\%$ error reduction compared to the state-of-the-art methods, including DoubleField~\cite{shao2022doublefield} and PIFuHD~\cite{saito2020pifuhd}.
\end{itemize}

\section{System description} \label{sec:modelling}
	The system considered in this study comprises a vertical-axis rotor enclosed into two \gls{PMB} that stabilise the tilt angle to zero degrees, i.e. they passively guarantee the verticality of the axis \citep{andersen2013dynamics}. Two electromagnets, one on the top and one at the bottom of the rotor can be independently activated by the currents induced into their coils and accelerate the rotor mass along the vertical direction $z$ as shown in Fig. \ref{fig:AMB_system}. A common practice is to relate the currents of the two electromagnets $i_{up}$, $i_{down}$ through a common deviation $i$ from a known constant current value $i_0$, such that $i_{up} = i_0 + i$ and $i_{down} = i_0 - i$ \citep{maslen2009magnetic,chiba2005magnetic}. This achieves the reduction of the control inputs from two to one, i.e. the current deviation $i$. This approach is also adopted in this study. The equations of motion along the z-axis and the dynamics of the current read \citep{dagnaes2018magnetic}:
% Figure environment removed
\begin{align}
	\ddot{z} &= \frac{2 k_z}{m}z + \frac{\mu_0 n^2 A}{4m}\left[ \left( \frac{i_0 + i}{s_0 - z} \right)^2 - \left( \frac{i_0 - i}{s_0 + z} \right)^2 \right] - g + q_z \label{eq:z_dynamics}\\
	\frac{di}{dt} &= \frac{2 \dot{z}}{s_0 + z}i + \frac{2(s_0 + z)}{\mu_0 n^2 A}(u - Ri) + q_i \label{eq:z_dot_dynamics}
\end{align}
where $m$ is the rotor mass, $z$ is the vertical displacement of the rotor, $u$ is the voltage input, $s_0$ is the air gap between rotor and electromagnets at the equilibrium point, $\mu_0$ is vacuum permeability, $n$ is the number of windings in each coil, $R$ is its resistance, $A$ is its cross-sectional area, $k_z$ is the axial stiffness coefficient and $g$ is the gravitational acceleration. The unknown perturbations $q_z(t)$ and $q_i(t)$ are bounded with $\vert q_z(t) \vert \leq Q_z, \; \vert q_i(t) \vert \leq Q_i$, $\forall t \geq 0$ and account for the lumped model uncertainties and disturbances in the two subsystems. Table \ref{tab:par_tab} lists the values of the system parameters.
\begin{table}[bp]
	\begin{center}
		\caption{System parameter values}
		\label{tab:par_tab}
		\begin{tabular}{clc}
	\toprule
	\textbf{Symbol}	& \textbf{Description} & \textbf{Value}\T\B \\
	\specialrule{.2em}{.1em}{-1em}\\
	$m$ 	& Mass of rotor and axle	& $0.588 \; \si{\kilogram}$\T\B \\
	$k_z$ 	& Axial stiffness 			& $-754 \; \si{\newton\meter}$\T\B \\
	$\mu_0$ & Vacuum permeability		& $1.25\cdot 10^{-6} \; \si{\newton\per\ampere^2}$\T\B \\
	$n$ 	& Number of coil windings	& $1480$\T\B \\
	$A$ 	& Cross-sectional area 		& $0.121 \; \si{\meter^2}$\T\B \\
	$s_0$ 	& Air gap size 				& $5\cdot 10^{-3} \; \si{\meter}$\T\B \\
	$i_0$ 	& Bias current				& $0.25 \; \si{\ampere}$\T\B \\
	$R$ 	& Coil resistance			& $41.44 \; \si{\ohm}$\T\B \\
	$g$ 	& gravitational acceleration& $9.81 \; \si{\meter\per\sec^2}$\T\B \\\hline
\end{tabular}
	\end{center}
\end{table}

\section{Control design}\label{sec:control}
	\subsection{Architecture}
	Our \hpc\ system consists of three parts: (1) a user interface which
allows the user to write a proof file, (2) a parser that performs some
basic well-formedness checks on a given proof file and (3) a proof
checking backend that validates a syntactically well formed proof for
correctness.

We provide two primary interfaces for \hpc: a Web-based editor and an
Eclipse plugin. The Web-based editor requires no setup on the user's
part but has fewer features and is slower than the Eclipse plugin,
since much of its functionality comes from making requests to a remote
server. The Eclipse plugin is faster and has more features, but
requires more effort to set up locally.
Figure~\ref{fig:eclipse-plugin-screenshot} shows a screenshot of our
Eclipse plugin. We discuss features of the user interface in more detail.

% Figure environment removed

In both the interfaces, the results of checking a proof are
communicated to the user in two ways: (1) by producing a simplified
list of the checks that is performed for each proof and their status,
with some additional details if the check failed, as is shown in the
``Checker Output'' panel on the top-right-hand side of
Figure~\ref{fig:eclipse-plugin-screenshot}, and (2) by marking
particular areas of the user's conjectures with
\textit{diagnostics}. An error diagnostic can be seen in
Figure~\ref{fig:eclipse-plugin-screenshot}, visualized by the red
underlining of part of the conjecture shown on the left-hand side of
the screenshot. Certain checks may give rise to lower-severity
diagnostics, like warnings or informational notes.

We spent significant time and effort in developing user-facing
messages that are easy to understand, useful and actionable. In fact,
this is the raison d’etre for having Phase I. But we also owe much to
the years of usage that \hpc\ has accumulated and the experience we
have gained perfecting it.

The Eclipse plugin is equipped with several features to improve the
proof-editing experience. Parts of a proof, or even entire proofs, can
be folded (hidden) when desired to maximize the screen space available
for content of interest (using $\oplus$ and $\ominus$ buttons to the left of text
seen in the screenshot) which makes it easy to browse a large proof
file. An outline view is available that allows one to easily identify
and navigate to specific areas of a document. Certain keywords like
Conjecture, Proof, Def etc... are highlighted. Highlighting of matched
parentheses and basic code completion are provided. The Eclipse plugin
also provides links to online documentation describing the conjecture
format we use, and answers to frequently asked questions.

We use the Xtext system~\cite{xtext} to implement both the parser and
a portion of both the user interfaces. Xtext is a Java-based framework
for developing tooling for domain-specific languages. By providing a
grammar, Xtext generates not only a parser but also IDE features like
code completion, syntax highlighting and code folding. The default
behavior that Xtext provides can be modified by providing Java
code. We configure Xtext to generate both an Eclipse plugin and
supporting functionality for the CodeMirror Web-based editor. Phase 0
checks described in the previous section are implemented by the parser
that Xtext generates automatically from the grammar we provide, though
we augment its error-handling abilities to produce more helpful error
messages in certain conditions. Xtext provides the ability to define
arbitrary checks (called ``validators'') for a document written in a
DSL. These validators take in an abstract syntax tree (AST), process
it and mark different parts of the AST with various diagnostics as
needed. Xtext handles mapping these errors back to the text of the
document and rendering them to the user. To integrate our proof
checker backend with Xtext, we define a validator that encodes the
conjecture document in a form readable by our backend, runs the
backend in a subprocess, captures and parses the output from the
backend, and translates that output into Xtext diagnostics.

The backend is responsible for checking the correctness all of the
proofs contained in a document. It performs Phase I and Phase II
checks as described in the previous section and uses the ACL2s Systems
Programming methodology that we have
developed~\cite{walter-acl2-systems-programming}. That is, most of the
backend's code is written in Common Lisp and runs in a Common Lisp
environment that has ACL2 loaded. It interacts with ACL2 through a set
of interface functions, making it easier to identify which code
interacts with the theorem prover and reducing the risk of
accidentally modifying ACL2 internals, thereby making ACL2
unsound. This potential unsoundness is mitigated for our purposes as
\hpc\ produces ACL2s proofs that can be checked by running them
through an unmodified ACL2s instance.

The backend takes in a syntactically correct
proof document that consists of ACL2 expressions interleaved with
conjectures. We define data structures in Common Lisp that track
metadata attached to expressions or parts of a conjecture, which
informs us where certain diagnostics should appear. The proof document
is represented as a sequence of these data structures. In brief, when
given a proof document, the backend performs some setup, sequentially
evaluates each element of the proof document while collecting
diagnostics, and then reports the diagnostics in an XML document. This
XML data is captured and consumed by our Xtext validator and is
translated into Xtext diagnostics, which are visible to the user
through whichever frontend they are using.

	
	\subsection{Position control law}
	Given a smooth position reference signal $r(t)$ and the associated tracking error $e_z \triangleq z - r$, define the sliding variable $\sigma \triangleq \dot{e}_z + c e_z$, where $c > 0$. On the sliding manifold $\mathcal{S}: \sigma = 0$ the dynamics of the vertical displacement $z$ are governed by $\dot{e}_z = -c e_z$, i.e. $e_z^{*} = 0$ is a \gls{GES} equilibrium point, with rate of decay $c$. The dynamics of the sliding variable reads:
\begin{align}
	\dot{\sigma} &= \frac{2 k_z}{m}z + \frac{\mu_0 n^2 A}{4m}\left[ \left( \frac{i_0 + i}{s_0 - z} \right)^2 - \left( \frac{i_0 - i}{s_0 + z} \right)^2 \right] \nonumber\\
				 &- g + q_z + c\dot{e}_z - \ddot{r} \label{eq:sigma_dynamics}
\end{align}
Assigning the value
\begin{equation} \label{eq:z_control_law}
	v^{*} = \frac{4m}{\mu_0 n^2 A} \left[ -\frac{2k_z}{m}z - c(\dot{z} - \dot{r}) + \ddot{r} + g - k\text{sgn}(\sigma) \right]
\end{equation}
to the virtual control input defined in \eqref{eq:v_definition}, where $k > Q_z$ and the signum function is defined by
$$
\text{sgn}(x) \triangleq \begin{cases}
	\displaystyle \frac{x}{\vert x \vert} &\text{, for } x \neq 0\\
	\chi\in[ -1, 1 ] &\text{, for } x = 0
\end{cases}
$$
will drive the rotor axial dynamics onto the sliding manifold $\mathcal{S}$ in finite time \citep{slotine1991applied}. Calculating the appropriate value of the current directly from \eqref{eq:v_definition} can be problematic due to the possible existence of two values for $i$ (provided that the equation has real roots). Numerical ill-posedness is another potential challenge especially under the presence of sensor noise. To overcome these obstacles, an online estimator is designed that dynamically inverts the nonlinear mapping (see \citep{nicosia1994nonlinear} for more details) $v^{*} = v(z,i)$ with respect to $i$, providing thus a smooth estimate $i_{ref}$ of the appropriate current value. The design of this estimator is detailed in the next subsection.
	
	\subsection{Input mapping inversion}
	Let $i^{*}$ be an appropriate value of the current $i$ such that $v(z,i^{*}) = v^{*}$, with $v^{*}$ given in \eqref{eq:z_control_law}. Note that since $v^{*}$ is a state feedback control law, the value for $i$ is generally time-varying. Define the input error $\tilde{v} \triangleq v^{*} - v(z,i_{ref}) = v(z,i^{*}) - v(z,i_{ref})$ and $d(z,\dot{z},\dot{r},\ddot{r}) \triangleq \dot{v}^{*} - \frac{\partial v(z,i_{ref})}{\partial z}\dot{z}$.
\begin{assumption}\label{assum:delta_boundeness}
	The signal $d(z,\dot{z},r,\dot{r},\ddot{r})$ is bounded, i.e. $\exists\Delta_1 > 0$, such that $\vert d(z,\dot{z},r,\dot{r},\ddot{r}) \vert \leq \Delta_1, \; \forall t \geq 0$.
\end{assumption}
\begin{prop} \label{prop:adaptive_law}
	The adaptive law
	\begin{equation} \label{eq:adaptive_law}
		\frac{d i_{ref}}{dt} = \gamma \left( \frac{\partial v(z,i_{ref})}{\partial i_{ref}} \right)^{-1}\text{sgn}(\tilde{v})
	\end{equation}
	with $\gamma > 0$ such that
	\begin{equation} \label{eq:gamma_selection}
		\gamma > \Delta_1 \geq \left \vert \dot{v}^{*} - \frac{\partial v(z,i_{ref})}{\partial z}\dot{z} \right \vert
	\end{equation}
	stabilises the input error $\tilde{v}$ dynamics at the origin in finite time.
\end{prop}
\begin{pf}
	The dynamics of the input error reads:
	\begin{align*}
		\dot{\tilde{v}} &= \dot{v}^{*} - \dot{v}(z,i_{ref}) = \underbrace{\dot{v}^{*} - \frac{\partial v(z,i_{ref})}{\partial z}\dot{z}}_{d(z,\dot{z},\dot{r},\ddot{r})} - \frac{\partial v(z,i_{ref})}{\partial i_{ref}}\frac{d i_{ref}}{dt}
	\end{align*}
	Inserting the adaptive law \eqref{eq:adaptive_law} in the equation above gives
	\begin{equation}\label{eq:inversion_error_dynamics}
		\dot{\tilde{v}} = -\gamma\text{sgn}(\tilde{v}) + d(z,\dot{z},r,\dot{r},\ddot{r})
	\end{equation}
	Under assumption \ref{assum:delta_boundeness}, selecting $\gamma > \Delta_1$ ensures convergence of $\tilde{v}$ to the origin in finite time \citep{shtessel2014sliding}. $\blacksquare$
\end{pf}
\begin{remark}
	 The adaptative law in \eqref{eq:adaptive_law} is implementable only if the gradient $\frac{\partial v(z,i_{ref})}{\partial i_{ref}}$ is not vanishing, i.e. only if changes in $i_{ref}$ affect the input estimation error. This translates to the requirement that $\forall t \geq 0$ the points $(z(t),i_{ref}(t))$ do not belong to the zero set of $\frac{\partial v(z,i_{ref})}{\partial i_{ref}}$, which is the curve $z^2 + \frac{2s_0}{i_0}i_{ref}z + s_0^2 = 0$, i.e. $\forall t \geq 0$
	 \begin{equation} \label{cond:gradient_non_zero}
	 	(z,i_{ref})\notin \mathcal{C} = \left\lbrace z,i_{ref}\in \mathbb{R} \big | z^2 + \frac{2s_0}{i_0}i_{ref}z + s_0^2 = 0 \right\rbrace \; .
	 \end{equation}
 	This can be seen as a controllability condition for the input error dynamics with $\frac{di_{ref}}{dt}$ being the ``input" to the system.
\end{remark}
\vspace{5pt}
\begin{remark}
	The result of Proposition \ref{prop:adaptive_law} does not depend on the selection of the control law $v^{*}$ for the $z$ dynamics. In fact the signals $\dot{v}^{*}$, $\frac{\partial v(z,i_{ref})}{\partial z}\dot{z}$ can be computed at each time instant so long $v^{*}$ is differentiable almost everywhere.
\end{remark}
% To obtain an estimation of $\Delta_1$, the sgn$(x)$ function can be approximated by a continuous function, such $\frac{2}{\pi}\arctan(px), \; p \gg 1$.
	
	\subsection{Current control law}
	The task of the current controller is to ensure that the appropriate current deviation $i_{ref}$ is generated by the \gls{AMB} coils such that the demanded total magnetic force is applied to the rotor mass for its axial positioning. Let the current tracking error be denoted by $e = i - i_{ref}$. Its dynamics reads:
\begin{equation} \label{eq:current_dynamics}
	\dot{e} = \frac{di}{dt} - \frac{di_{ref}}{dt} = \frac{2 \dot{z}}{s_0 + z}i + \frac{2(s_0 + z)}{\mu_0 n^2 A}(u - Ri) + q_i - \frac{di_{ref}}{dt} \; .
\end{equation}
The control law
\begin{equation} \label{eq:i_control_law}
	u = \frac{\mu_0 n^2 A}{2(s_0 + z)}\left[ -\frac{2 \dot{z}}{s_0 + z}i + \frac{di_{ref}}{dt} - k_i\text{sgn}(e) \right] + Ri, \; k_i > Q_i
\end{equation}
gives the closed-loop dynamics $\dot{e} = -k_i\text{sgn}(e) + q_i$, which has a finite-time stable equilibrium point at the origin. Note that $\frac{di_{ref}}{dt}$ is known from \eqref{eq:adaptive_law} by design.
	
	\subsection{Stability analysis}
	The rotor sliding variable closed-loop dynamics can be written as
\begin{align}
	&\dot{\sigma} = \frac{2 k_z}{m}z - g + q_z - \ddot{r} + c\dot{e}_z + \frac{\mu_0 n^2 A}{4m}v(z,i) = \frac{2 k_z}{m}z - g \nonumber\\
				 &+ q_z - \ddot{r} + c\dot{e}_z + \frac{\mu_0 n^2 A}{4m} \left[ v^{*} - \tilde{v} - v(z,i_{ref}) + v(z,i) \right] \nonumber\\
				 &= -k\text{sgn}(\sigma) + q_z + \underbrace{\frac{\mu_0 n^2 A}{4m} \left[ v(z,i_{ref} + e) - \tilde{v} - v(z,i_{ref}) \right]}_{h(z,e,\tilde{v})} \nonumber\\
				 &= -k\text{sgn}(\sigma) + q_z + h(z,e,\tilde{v}) \; .
\end{align}
Together with the dynamics of the input estimation error and the current tracking error, they comprise a feedback interconnection. It is easy to see that when $e = \tilde{v} = 0$ the unperturbed $\sigma$-dynamics have a finite-time stable equilibrium at the origin. Inspired by the approach proposed in \citep{lor2008a}, this feedback can be also viewed as a \emph{cascaded interconnection} of the systems
\begin{align}
	(\Sigma_1) &: \dot{\sigma}  = -k\text{sgn}(\sigma) + q_z + h(z,e,\tilde{v})\\
	(\Sigma_2) &: \bm{\dot{\xi}} \triangleq \begin{bmatrix}
		\dot{\tilde{v}}\\
		\dot{e}
	\end{bmatrix} = \begin{bmatrix}
		-\gamma\text{sgn}(\tilde{v}) + d(z,\sigma - cz)\\
		-k_i\text{sgn}(e) + q_i
	\end{bmatrix}
\end{align}
where the solutions of $(\Sigma_2)$ depend on the \emph{parameter} $\sigma(t;t_0,\sigma_0)$.
\setcounter{thm}{0}
\begin{thm}
	Under the assumption that condition \eqref{cond:gradient_non_zero} holds with $k > Q_z, \; k_i > Q_i$ and $\gamma > \Delta_1$, the closed loop system $(\Sigma_1)-(\Sigma_2)$ has a \gls{UAS} equilibrium point at the origin.
\end{thm}
\begin{pf}
	The unperturbed system $(\Sigma_1)$ with $e = \tilde{v} = 0 \Rightarrow h(z,e,\tilde{v}) = 0$ has a finite-time stable equilibrium point at the origin. This implies the existence of a $\mathcal{C}^1$ positive definite and radially unbounded Lyapunov function $V \triangleq \sigma^2$, for which it holds $\dot{V} \leq -2\bar{k}\vert \sigma \vert$, where $\bar{k} = k - Q_z$. In order to prove that the origin is \gls{UAS}, it is enough to show that Assumptions 1,4,5,7 and the conditions of Theorem 2 from \citep{lor2008a} are satisfied. In the subsequent analysis, the notation introduced in \citep{lor2008a} is adopted to facilitate direct referencing to the original formulation of the assumptions and Theorem 2.
	
	The finite-time stability of the origin of the unperturbed $(\Sigma_1)$ satisfies Assumptions 1 and 5 that require \gls{UGAS} origin instead. Define $V_1(\sigma) \triangleq V(\sigma)$ and the class $\mathcal{K}_{\infty}$ functions $\alpha_1(\sigma) \triangleq \vert \sigma \vert^2 = V_1(\sigma)$, $\alpha_4(\sigma) \triangleq 2\vert \sigma \vert$ and $\alpha_4^{\prime}(\Vert \bm{\xi} \Vert) \triangleq \frac{k\mu_0 n^2 A}{4m}\sqrt{L_v^2 + 1}\Vert \bm{\xi} \Vert$, where $L_v$ is the Lipschitz constant of $v(z,i_{ref} + e)$ with respect to $e$. Moreover, evaluating the time derivative of $V_1$ along the trajectories of the perturbed system gives:
	\begin{align*}
		\dot{V}_1 &\leq -2\bar{k}\vert \sigma \vert + 2k\vert \sigma \vert \frac{\mu_0 n^2 A}{4m}\vert v(z,i_{ref} + e) - \tilde{v} - v(z,i_{ref}) \vert\\
				  &\leq 2k\vert \sigma \vert \frac{\mu_0 n^2 A}{4m}\left( L_v\vert e \vert + \vert \tilde{v} \vert \right) = 2k\vert \sigma \vert \frac{\mu_0 n^2 A}{4m} \begin{bmatrix}
				  		L_v & 1
				  \end{bmatrix} \begin{bmatrix}
				  		\vert e \vert\\
				  		\vert \tilde{v} \vert
			  	  \end{bmatrix}\\
		  	  	  &\leq 2k\vert \sigma \vert \frac{\mu_0 n^2 A}{4m}\sqrt{L_v^2 + 1}\Vert \bm{\xi} \Vert = \alpha_4(\sigma) \alpha_4^{\prime}(\Vert \bm{\xi} \Vert)
	\end{align*}
	Furthermore, if $V_{1,0}$ is a lower bound for $V_1$, then
	$$
		\int_{V_{1,0}}^{\infty} \frac{dw}{\alpha_4(\alpha_1^{-1}(w))} = \int_{V_{1,0}}^{\infty} \frac{dw}{2\sqrt{w}} = \left[ \sqrt{w} \right]_{V_{1,0}}^{\infty} = \infty
	$$
	which shows that Assumption 4 from \citep{lor2008a} is also satisfied. The finite-time stability of the origin of $(\Sigma_2)$ - note that this holds irrespectively of the dependence on the trajectories of $(\Sigma_1)$ - satisfies Assumption 7, which requires \gls{UGAS} for ($\Sigma_2$).
	
	The second and final condition of Theorem 2 in \citep{lor2008a} requires that there exist class $\mathcal{K}$ functions $\alpha_5(\sigma)$, $\alpha_5^{\prime}(\Vert \bm{\xi} \Vert)$ such that 
	$$
		\left \vert \frac{\partial V}{\partial \sigma}h(z,e,\tilde{v}) \right \vert \leq \alpha_5(\sigma)\alpha_5^{\prime}(\Vert \bm{\xi} \Vert)
	$$
	where $V$ was defined earlier. Since $V = V_1$, the foregoing inequality was shown to hold for $\alpha_5(\sigma) \triangleq \alpha_4(\sigma) = 2\vert \sigma \vert$ and $\alpha_5^{\prime}(\Vert \bm{\xi} \Vert) \triangleq \alpha_4^{\prime}(\Vert \bm{\xi} \Vert) = \frac{\mu_0 n^2 A}{4m}\sqrt{L_v^2 + 1}\Vert \bm{\xi} \Vert$. Moreover, it is required that for every positive upper bound $\rho$ of the solutions of $(\Sigma_2)$ $\exists \lambda_{\rho}, \eta_{\rho} > 0$ such that
	$$
		\forall t \geq 0 \; , \vert \sigma \vert \geq \eta_{\rho} \Rightarrow \alpha_5(\sigma) \leq \lambda_{\rho} W(\sigma)
	$$
	where $W$ is a positive semi-definite function. Indeed, selecting $W(\sigma) \triangleq 2\sigma^2$ one obtains:
	$$
		W(\sigma) = 2\sigma^2 \geq \eta_{\rho}2\vert \sigma \vert = \eta_{\rho}\alpha_5(\sigma) \Leftrightarrow \alpha_5(\sigma) \leq \lambda_{\rho}W(\sigma)
	$$
	with $\lambda_{\rho} \triangleq \frac{1}{\eta_{\rho}}$. With this, all conditions of Theorem 2 are satisfied and by Proposition 2 in  \citep{lor2008a}, the origin of $(\Sigma_1) - (\Sigma_2)$ is a \gls{UAS} equilibrium point. $\blacksquare$
\end{pf}
\begin{remark}
	Selecting $k > Q_z, \; k_i > Q_i$ suffices for ensuring finite-time stability of the origins of $\sigma$ and $e$ in the presence of unmodelled dynamics and perturbations in the system. Following the same line of argumentation, $k$ can be chosen larger than $\sup\limits_{t} \left\vert q_z(t) + h(z(t),e(t),\tilde{v}(t)) \right\vert$ to even dominate over the effects of the transients due to the feedback interconnection with $(\Sigma_2)$. Moreover, sgn$(x)$ can be approximated by a continuous function, such $\frac{2}{\pi}\arctan(px), \; p \gg 1$ to alleviate the effect of chattering.
\end{remark}
\begin{remark}
	The control laws \eqref{eq:z_control_law} and \eqref{eq:i_control_law} are only two of several possible options. In fact, provided that condition \eqref{cond:gradient_non_zero} holds, \gls{UAS} of the closed-loop system ($\Sigma_1$)-($\Sigma_2$) is ensured for any selection of $v^{*}$ and $u$ that render the origins of \eqref{eq:sigma_dynamics}, \eqref{eq:current_dynamics} \gls{UGAS}. It should also be noted that \gls{UGAS} of the origin of the cascaded system cannot be claimed since trajectories starting in $\mathcal{C}$ will automatically result in violation of condition \eqref{cond:gradient_non_zero}, which is essential for the finite time stability of the origin of \eqref{eq:inversion_error_dynamics}.
\end{remark}






\section{Simulation results} \label{sec:simulations}
	\subsection{Computer Simulated Experiments}

\subsubsection{MNIST Simulated Data Preparation}

% \subsubsection{MNIST Data Pre-processing}

\Xpolish{The MNIST dataset consists of handwritten digits from 0 to 9, with a default size of 28 by 28 pixels \cite{lecun1998MNIST}. In order to align the data with the Hadamard matrices, we padded the images with black pixels at the edges to resize them to 32 by 32 pixels \cite{lecun1998MNIST}. We then transformed the range of all pixel values from $[0,1]$ to $[0.3, 1]$. This operation enhanced the persuasiveness of the dataset for various noise models, as the added black pixels do not generate Poisson noise, whose variance is proportional to the expected photon counts. \bnote{did we ever vary the dark level (0.3)/ have results showing its effect?}}
{The MNIST dataset, which comprises handwritten digits ranging from 0 to 9, has a default size of 28 by 28 pixels \cite{lecun1998MNIST}. To align the data with Hadamard matrices, we added black pixels to the edges of the images and resized them to 32 by 32 pixels \cite{lecun1998MNIST}. Subsequently, we rescaled the pixel values from the original range of $[0,1]$ to $[0.3, 1]$. This rescaling was aimed at improving the dataset's suitability for different noise models, as the initial black backgrounds do not introduce Poisson noise, which is typically proportional to the expected photon counts.}

\subsubsection{Spectral Datasets and Performance Evaluation}


% \rnote{How well this model can work w/o noise}

% \subsubsection{Performance Evaluation}


\Xpolish{The performances of the models were evaluated through their average classification rates. To ensure a thorough assessment, the model was tested five times independently, each time with a randomly split dataset for 70\% training and 30\% validation. During each assessment, the model generated noisy photon counts, which were then used to train the classifier until the validation rate reached a plateau or declined. The best validation rate achieved during the training process was chosen as the representative performance upper bound for each assessment. The overall performance of the model was determined by averaging the results of all five assessments.}
{The model performances were assessed based on their average classification rates, which were calculated after five independent tests. In each test, the dataset was randomly split into 70\% for training and 30\% for validation. The model generated noisy photon counts during each assessment, and the training continued until the validation rate reached a plateau or declined. The highest validation rate obtained during the training process was considered as the representative upper bound for each assessment. The overall model performance was determined by averaging the results from all five assessments. This comprehensive evaluation ensured the reliability and accuracy of the model performance.}



%%
%% \begin{thm} ... \end{thm}            % Theorem
%% \begin{lem} ... \end{lem}            % Lemma
%% \begin{claim} ... \end{claim}        % Claim
%% \begin{conj} ... \end{conj}          % Conjecture
%% \begin{cor} ... \end{cor}            % Corollary
%% \begin{fact} ... \end{fact}          % Fact
%% \begin{hypo} ... \end{hypo}          % Hypothesis
%% \begin{prop} ... \end{prop}          % Proposition
%% \begin{crit} ... \end{crit}          % Criterion

\section{Conclusions and future work} \label{sec:conclusions}
	%% -*- mode: LaTeX; fill-column: 78; -*-

\section{Concluding Remarks}
\label{sec:conclusions}

In this paper, we presented a novel SMC algorithm, \EventDPOR, tailored to the
characteristics of event-driven multi-threaded programs running under the SC
semantics. The algorithm was proven correct and optimal for event-driven
programs in which the variable accesses of events do not depend on how their
execution is interleaved with other threads.

We have implemented \EventDPOR in the \Nidhugg tool, and we will open-source
our implementation.
%
With a wide range of event-driven programs, we have shown that \EventDPOR
incurs only a moderate constant overhead over its baseline implementation
(\OptimalDPOR), it is exponentially faster than existing state-of-the-art SMC
algorithms in time and number of traces examined on programs where events'
actions do not conflict, and does not suffer from performance degradation
caused by having to examine
% a significant number of
non-serializable executions.
%
%% \bjcom{Should we include:
%% Moreover, in our benchmarks, also those that are not non-branching,
%% \EventDPOR explores only the optimal number of executions, and never
%% had to resort to a potentially expensive decision procedure.}

\EventDPOR assumes that handlers can process their events in arbitrary order.
Directions for future work include to retarget \EventDPOR for event-driven
programs with other policies (e.g., FIFO), and for specific event-driven
execution models.


\balance

\bibliography{mybibl}             % bib file to produce the bibliography
                                                     % with bibtex (preferred)

\end{document}
