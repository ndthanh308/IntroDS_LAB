The rotor sliding variable closed-loop dynamics can be written as
\begin{align}
	&\dot{\sigma} = \frac{2 k_z}{m}z - g + q_z - \ddot{r} + c\dot{e}_z + \frac{\mu_0 n^2 A}{4m}v(z,i) = \frac{2 k_z}{m}z - g \nonumber\\
				 &+ q_z - \ddot{r} + c\dot{e}_z + \frac{\mu_0 n^2 A}{4m} \left[ v^{*} - \tilde{v} - v(z,i_{ref}) + v(z,i) \right] \nonumber\\
				 &= -k\text{sgn}(\sigma) + q_z + \underbrace{\frac{\mu_0 n^2 A}{4m} \left[ v(z,i_{ref} + e) - \tilde{v} - v(z,i_{ref}) \right]}_{h(z,e,\tilde{v})} \nonumber\\
				 &= -k\text{sgn}(\sigma) + q_z + h(z,e,\tilde{v}) \; .
\end{align}
Together with the dynamics of the input estimation error and the current tracking error, they comprise a feedback interconnection. It is easy to see that when $e = \tilde{v} = 0$ the unperturbed $\sigma$-dynamics have a finite-time stable equilibrium at the origin. Inspired by the approach proposed in \citep{lor2008a}, this feedback can be also viewed as a \emph{cascaded interconnection} of the systems
\begin{align}
	(\Sigma_1) &: \dot{\sigma}  = -k\text{sgn}(\sigma) + q_z + h(z,e,\tilde{v})\\
	(\Sigma_2) &: \bm{\dot{\xi}} \triangleq \begin{bmatrix}
		\dot{\tilde{v}}\\
		\dot{e}
	\end{bmatrix} = \begin{bmatrix}
		-\gamma\text{sgn}(\tilde{v}) + d(z,\sigma - cz)\\
		-k_i\text{sgn}(e) + q_i
	\end{bmatrix}
\end{align}
where the solutions of $(\Sigma_2)$ depend on the \emph{parameter} $\sigma(t;t_0,\sigma_0)$.
\setcounter{thm}{0}
\begin{thm}
	Under the assumption that condition \eqref{cond:gradient_non_zero} holds with $k > Q_z, \; k_i > Q_i$ and $\gamma > \Delta_1$, the closed loop system $(\Sigma_1)-(\Sigma_2)$ has a \gls{UAS} equilibrium point at the origin.
\end{thm}
\begin{pf}
	The unperturbed system $(\Sigma_1)$ with $e = \tilde{v} = 0 \Rightarrow h(z,e,\tilde{v}) = 0$ has a finite-time stable equilibrium point at the origin. This implies the existence of a $\mathcal{C}^1$ positive definite and radially unbounded Lyapunov function $V \triangleq \sigma^2$, for which it holds $\dot{V} \leq -2\bar{k}\vert \sigma \vert$, where $\bar{k} = k - Q_z$. In order to prove that the origin is \gls{UAS}, it is enough to show that Assumptions 1,4,5,7 and the conditions of Theorem 2 from \citep{lor2008a} are satisfied. In the subsequent analysis, the notation introduced in \citep{lor2008a} is adopted to facilitate direct referencing to the original formulation of the assumptions and Theorem 2.
	
	The finite-time stability of the origin of the unperturbed $(\Sigma_1)$ satisfies Assumptions 1 and 5 that require \gls{UGAS} origin instead. Define $V_1(\sigma) \triangleq V(\sigma)$ and the class $\mathcal{K}_{\infty}$ functions $\alpha_1(\sigma) \triangleq \vert \sigma \vert^2 = V_1(\sigma)$, $\alpha_4(\sigma) \triangleq 2\vert \sigma \vert$ and $\alpha_4^{\prime}(\Vert \bm{\xi} \Vert) \triangleq \frac{k\mu_0 n^2 A}{4m}\sqrt{L_v^2 + 1}\Vert \bm{\xi} \Vert$, where $L_v$ is the Lipschitz constant of $v(z,i_{ref} + e)$ with respect to $e$. Moreover, evaluating the time derivative of $V_1$ along the trajectories of the perturbed system gives:
	\begin{align*}
		\dot{V}_1 &\leq -2\bar{k}\vert \sigma \vert + 2k\vert \sigma \vert \frac{\mu_0 n^2 A}{4m}\vert v(z,i_{ref} + e) - \tilde{v} - v(z,i_{ref}) \vert\\
				  &\leq 2k\vert \sigma \vert \frac{\mu_0 n^2 A}{4m}\left( L_v\vert e \vert + \vert \tilde{v} \vert \right) = 2k\vert \sigma \vert \frac{\mu_0 n^2 A}{4m} \begin{bmatrix}
				  		L_v & 1
				  \end{bmatrix} \begin{bmatrix}
				  		\vert e \vert\\
				  		\vert \tilde{v} \vert
			  	  \end{bmatrix}\\
		  	  	  &\leq 2k\vert \sigma \vert \frac{\mu_0 n^2 A}{4m}\sqrt{L_v^2 + 1}\Vert \bm{\xi} \Vert = \alpha_4(\sigma) \alpha_4^{\prime}(\Vert \bm{\xi} \Vert)
	\end{align*}
	Furthermore, if $V_{1,0}$ is a lower bound for $V_1$, then
	$$
		\int_{V_{1,0}}^{\infty} \frac{dw}{\alpha_4(\alpha_1^{-1}(w))} = \int_{V_{1,0}}^{\infty} \frac{dw}{2\sqrt{w}} = \left[ \sqrt{w} \right]_{V_{1,0}}^{\infty} = \infty
	$$
	which shows that Assumption 4 from \citep{lor2008a} is also satisfied. The finite-time stability of the origin of $(\Sigma_2)$ - note that this holds irrespectively of the dependence on the trajectories of $(\Sigma_1)$ - satisfies Assumption 7, which requires \gls{UGAS} for ($\Sigma_2$).
	
	The second and final condition of Theorem 2 in \citep{lor2008a} requires that there exist class $\mathcal{K}$ functions $\alpha_5(\sigma)$, $\alpha_5^{\prime}(\Vert \bm{\xi} \Vert)$ such that 
	$$
		\left \vert \frac{\partial V}{\partial \sigma}h(z,e,\tilde{v}) \right \vert \leq \alpha_5(\sigma)\alpha_5^{\prime}(\Vert \bm{\xi} \Vert)
	$$
	where $V$ was defined earlier. Since $V = V_1$, the foregoing inequality was shown to hold for $\alpha_5(\sigma) \triangleq \alpha_4(\sigma) = 2\vert \sigma \vert$ and $\alpha_5^{\prime}(\Vert \bm{\xi} \Vert) \triangleq \alpha_4^{\prime}(\Vert \bm{\xi} \Vert) = \frac{\mu_0 n^2 A}{4m}\sqrt{L_v^2 + 1}\Vert \bm{\xi} \Vert$. Moreover, it is required that for every positive upper bound $\rho$ of the solutions of $(\Sigma_2)$ $\exists \lambda_{\rho}, \eta_{\rho} > 0$ such that
	$$
		\forall t \geq 0 \; , \vert \sigma \vert \geq \eta_{\rho} \Rightarrow \alpha_5(\sigma) \leq \lambda_{\rho} W(\sigma)
	$$
	where $W$ is a positive semi-definite function. Indeed, selecting $W(\sigma) \triangleq 2\sigma^2$ one obtains:
	$$
		W(\sigma) = 2\sigma^2 \geq \eta_{\rho}2\vert \sigma \vert = \eta_{\rho}\alpha_5(\sigma) \Leftrightarrow \alpha_5(\sigma) \leq \lambda_{\rho}W(\sigma)
	$$
	with $\lambda_{\rho} \triangleq \frac{1}{\eta_{\rho}}$. With this, all conditions of Theorem 2 are satisfied and by Proposition 2 in  \citep{lor2008a}, the origin of $(\Sigma_1) - (\Sigma_2)$ is a \gls{UAS} equilibrium point. $\blacksquare$
\end{pf}
\begin{remark}
	Selecting $k > Q_z, \; k_i > Q_i$ suffices for ensuring finite-time stability of the origins of $\sigma$ and $e$ in the presence of unmodelled dynamics and perturbations in the system. Following the same line of argumentation, $k$ can be chosen larger than $\sup\limits_{t} \left\vert q_z(t) + h(z(t),e(t),\tilde{v}(t)) \right\vert$ to even dominate over the effects of the transients due to the feedback interconnection with $(\Sigma_2)$. Moreover, sgn$(x)$ can be approximated by a continuous function, such $\frac{2}{\pi}\arctan(px), \; p \gg 1$ to alleviate the effect of chattering.
\end{remark}
\begin{remark}
	The control laws \eqref{eq:z_control_law} and \eqref{eq:i_control_law} are only two of several possible options. In fact, provided that condition \eqref{cond:gradient_non_zero} holds, \gls{UAS} of the closed-loop system ($\Sigma_1$)-($\Sigma_2$) is ensured for any selection of $v^{*}$ and $u$ that render the origins of \eqref{eq:sigma_dynamics}, \eqref{eq:current_dynamics} \gls{UGAS}. It should also be noted that \gls{UGAS} of the origin of the cascaded system cannot be claimed since trajectories starting in $\mathcal{C}$ will automatically result in violation of condition \eqref{cond:gradient_non_zero}, which is essential for the finite time stability of the origin of \eqref{eq:inversion_error_dynamics}.
\end{remark}




