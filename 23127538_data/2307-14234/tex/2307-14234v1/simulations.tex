The efficacy and performance of the proposed control scheme were tested in two simulation scenarios, namely:
\begin{itemize}
	\item Regulation of $z$ at the origin, i.e. $r(t) \equiv 0$.
	\item Tracking of sinusoidal signal $r(t) = A_r\sin(2\pi f_r t)$.
\end{itemize}
\begin{table}[bp]
\begin{center}
	\caption{Simulation parameter values}
	\label{tab:spar_tab}
	\begin{tabular}{clc}
	\toprule
	\textbf{Symbol}	& \textbf{Description} & \textbf{Value}\T\B \\
	\specialrule{.2em}{.1em}{-1em}\\
	$A_r$ 	& Reference amplitude				& $0.0025 \; \si{\meter}$\T\B \\
	$f_r$ 	& Reference frequency 				& $2 \; \si{\hertz}$\T\B \\
	$c$ 	& Exponential decay rate			& $17$\T\B \\
	$k$ 	& \gls{SMC} gain for $z$			& $25$\T\B \\
	$\gamma$& Adaptation gain 					& $1000$\T\B \\
	$k_i$ 	& \gls{SMC} gain for $i$ 			& $152$\T\B \\
	$p$ 	& Signum approximation steepness factor	& $25$\T\B \\
	$Q_z$ 	& Disturbance amplitude				& $1$\T\B \\\hline
\end{tabular}
\end{center}
\end{table}
A pulse scaled magnetic force disturbance $q_z(t)$ of amplitude $Q_z$ is introduced in both cases to assess the robustness of the closed-loop system. It is assumed that the position sensor is inflicted with zero mean Gaussian noise with variance $10^{-7}$ m, which corresponds to approximately $20\%$ of the allowable axial displacement $s_0$ of the rotor. Table \ref{tab:spar_tab} lists all the simulation parameters.

% Figure environment removed

% Figure environment removed

% Figure environment removed

% Figure environment removed
Figure \ref{fig:signals_z} illustrates the response of the closed-loop system in the case of regulation at $z = 0$. Both the position displacement and the current bias converge to the reference signals, while the sliding surface is reaching a neighbourhood of the origin (due to noise) with radius $6\cdot 10^{-6} \; \si{\meter}$ in finite time. The closed-loop system is insensitive to the effect of the disturbance. The performance of the input mapping inversion scheme is shown in Fig. \ref{fig:signals_u} with the input estimation error reaching zero in less than 0.01 sec. Similar conclusions can be drawn by looking at Fig. \ref{fig:signals_z_track} and \ref{fig:signals_u_track}, which show the closed-loop response during the tracking case. The position displacement $z$ of the rotor is able to follow the sinusoidal reference signal with maximum absolute error $2.3\cdot 10^{-5} \; \si{\meter}$, while at the same time the disturbance is successfully rejected.