Let $i^{*}$ be an appropriate value of the current $i$ such that $v(z,i^{*}) = v^{*}$, with $v^{*}$ given in \eqref{eq:z_control_law}. Note that since $v^{*}$ is a state feedback control law, the value for $i$ is generally time-varying. Define the input error $\tilde{v} \triangleq v^{*} - v(z,i_{ref}) = v(z,i^{*}) - v(z,i_{ref})$ and $d(z,\dot{z},\dot{r},\ddot{r}) \triangleq \dot{v}^{*} - \frac{\partial v(z,i_{ref})}{\partial z}\dot{z}$.
\begin{assumption}\label{assum:delta_boundeness}
	The signal $d(z,\dot{z},r,\dot{r},\ddot{r})$ is bounded, i.e. $\exists\Delta_1 > 0$, such that $\vert d(z,\dot{z},r,\dot{r},\ddot{r}) \vert \leq \Delta_1, \; \forall t \geq 0$.
\end{assumption}
\begin{prop} \label{prop:adaptive_law}
	The adaptive law
	\begin{equation} \label{eq:adaptive_law}
		\frac{d i_{ref}}{dt} = \gamma \left( \frac{\partial v(z,i_{ref})}{\partial i_{ref}} \right)^{-1}\text{sgn}(\tilde{v})
	\end{equation}
	with $\gamma > 0$ such that
	\begin{equation} \label{eq:gamma_selection}
		\gamma > \Delta_1 \geq \left \vert \dot{v}^{*} - \frac{\partial v(z,i_{ref})}{\partial z}\dot{z} \right \vert
	\end{equation}
	stabilises the input error $\tilde{v}$ dynamics at the origin in finite time.
\end{prop}
\begin{pf}
	The dynamics of the input error reads:
	\begin{align*}
		\dot{\tilde{v}} &= \dot{v}^{*} - \dot{v}(z,i_{ref}) = \underbrace{\dot{v}^{*} - \frac{\partial v(z,i_{ref})}{\partial z}\dot{z}}_{d(z,\dot{z},\dot{r},\ddot{r})} - \frac{\partial v(z,i_{ref})}{\partial i_{ref}}\frac{d i_{ref}}{dt}
	\end{align*}
	Inserting the adaptive law \eqref{eq:adaptive_law} in the equation above gives
	\begin{equation}\label{eq:inversion_error_dynamics}
		\dot{\tilde{v}} = -\gamma\text{sgn}(\tilde{v}) + d(z,\dot{z},r,\dot{r},\ddot{r})
	\end{equation}
	Under assumption \ref{assum:delta_boundeness}, selecting $\gamma > \Delta_1$ ensures convergence of $\tilde{v}$ to the origin in finite time \citep{shtessel2014sliding}. $\blacksquare$
\end{pf}
\begin{remark}
	 The adaptative law in \eqref{eq:adaptive_law} is implementable only if the gradient $\frac{\partial v(z,i_{ref})}{\partial i_{ref}}$ is not vanishing, i.e. only if changes in $i_{ref}$ affect the input estimation error. This translates to the requirement that $\forall t \geq 0$ the points $(z(t),i_{ref}(t))$ do not belong to the zero set of $\frac{\partial v(z,i_{ref})}{\partial i_{ref}}$, which is the curve $z^2 + \frac{2s_0}{i_0}i_{ref}z + s_0^2 = 0$, i.e. $\forall t \geq 0$
	 \begin{equation} \label{cond:gradient_non_zero}
	 	(z,i_{ref})\notin \mathcal{C} = \left\lbrace z,i_{ref}\in \mathbb{R} \big | z^2 + \frac{2s_0}{i_0}i_{ref}z + s_0^2 = 0 \right\rbrace \; .
	 \end{equation}
 	This can be seen as a controllability condition for the input error dynamics with $\frac{di_{ref}}{dt}$ being the ``input" to the system.
\end{remark}
\vspace{5pt}
\begin{remark}
	The result of Proposition \ref{prop:adaptive_law} does not depend on the selection of the control law $v^{*}$ for the $z$ dynamics. In fact the signals $\dot{v}^{*}$, $\frac{\partial v(z,i_{ref})}{\partial z}\dot{z}$ can be computed at each time instant so long $v^{*}$ is differentiable almost everywhere.
\end{remark}
% To obtain an estimation of $\Delta_1$, the sgn$(x)$ function can be approximated by a continuous function, such $\frac{2}{\pi}\arctan(px), \; p \gg 1$.