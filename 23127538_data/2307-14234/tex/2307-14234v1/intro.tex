\gls{AMB} are becoming more and more popular in rotating machine systems. This is due to their feature of facilitating contactless rotation, hence rotary machines with \gls{AMB} do not require lubricants, exhibit less wear and support high rotational speed limits. A wide integration of \gls{AMB} into rotary equipment can revolutionise the operating conditions in energy production systems such as flywheels.

Active control of \gls{AMB} systems is instrumental for accurate positioning of the rotor axle, which is necessary for safe and efficient operation of high-speed rotational machines. Feedback control strategies in particular, can ensure robust operation of high-speed rotational machines so that they become available in industrial motion systems, allowing energy-efficient technology to enter manufacturing and production.

Active control of \gls{AMB} systems has been extensively studied in the literature. Robust linear control approaches such as $H_{\infty}$ loop shaping were investigated for disturbance rejection \citep{schroder1997a} and in combination with closed-loop identification \citep{junfeng2011a}. $H_{\infty}$ control of uncertain \gls{AMB} systems was studied in \citep{lee2013a} using a linear parameter-varying systems framework, while the authors in \citep{fittro2002a} proposed a $\mu$-synthesis approach. Adaptive control methods were investigated in several studies such as \citep{gibson2002a} and \citep{long1996a}, where the authors proposed an adaptive backstepping control scheme for compensating for uncertain load change and unbalance disturbance. Adaptive backstepping control was employed in \citep{dong2013a} for a linearised \gls{AMB} system with small parametric uncertainties. Nonlinear backstepping control was proposed in \citep{motee2002a} for regulation of the radial motion of an \gls{AMB} system. An adaptive extension of this scheme was studied in \citep{xu2022a}, where only the mechanical dynamics were considered. \gls{SMC} approaches for linearised \gls{AMB} plants were proposed in \citep{huynh2016a}, \citep{kang2010a} and \citep{jang2005a}. \gls{SMC} for radial motion regulation was developed in \citep{wang2016a}. The nonlinear dynamics of \gls{AMB} systems was considered in several control approaches, however with linear approximations of the magnetic forces such as in \citep{saha2020a} and \citep{kandil2018a}.

The majority of the reported work on control of \gls{AMB} employ linearisation of the magnetic forces expressions, which limits the operating range of the model. Static input inversion was used in \citep{mouille1992a} to compute the appropriate voltage that would generate the demanded control forces. A similar approach was adopted in \citep{song1996a}, where nonlinear adaptive control was designed for an \gls{AMB} system. The inversion of the nonlinear input mapping was achieved by solving a set of second-order algebraic equations at every time instant. Such approaches can be problematic when there is ambiguity regarding multiple solutions, eventually leading to non-smooth current or voltage reference signals.

This paper proposes a cascaded nonlinear control strategy for the vertical axial positioning of the rotor shaft in an \gls{AMB} system. The control scheme comprises nonlinear control laws based on sliding mode principles for the mechanical and current dynamics that are designed separately. An adaptive estimator is designed for inverting the nonlinear input mapping in the \gls{AMB} system. The inherent modularity of this architecture allows not only for easier analysis but also for integration of different methods in the control cascade. More specifically, the contributions of this study pertain to the following:
\begin{itemize}
	\item Development of a modular cascaded control architecture, into which alternative control and estimation algorithms can be integrated.
	\item Development of a \emph{dynamic} estimation strategy for inverting the nonlinear input mapping between desired force and current.
	\item Stability analysis of the cascaded closed-loop system.
\end{itemize}

The remainder of the paper is structured as follows: Section \ref{sec:modelling} gives a description of the \gls{AMB} system and its mathematical dynamical model. The proposed control design along with the stability analysis of the closed-loop system is detailed in Section \ref{sec:control}. Section \ref{sec:simulations} presents the simulation results and discusses the performance of the proposed method. Finally, concluding remarks and reflections on future work are presented in Section \ref{sec:conclusions}.