The system considered in this study comprises a vertical-axis rotor enclosed into two \gls{PMB} that stabilise the tilt angle to zero degrees, i.e. they passively guarantee the verticality of the axis \citep{andersen2013dynamics}. Two electromagnets, one on the top and one at the bottom of the rotor can be independently activated by the currents induced into their coils and accelerate the rotor mass along the vertical direction $z$ as shown in Fig. \ref{fig:AMB_system}. A common practice is to relate the currents of the two electromagnets $i_{up}$, $i_{down}$ through a common deviation $i$ from a known constant current value $i_0$, such that $i_{up} = i_0 + i$ and $i_{down} = i_0 - i$ \citep{maslen2009magnetic,chiba2005magnetic}. This achieves the reduction of the control inputs from two to one, i.e. the current deviation $i$. This approach is also adopted in this study. The equations of motion along the z-axis and the dynamics of the current read \citep{dagnaes2018magnetic}:
% Figure environment removed
\begin{align}
	\ddot{z} &= \frac{2 k_z}{m}z + \frac{\mu_0 n^2 A}{4m}\left[ \left( \frac{i_0 + i}{s_0 - z} \right)^2 - \left( \frac{i_0 - i}{s_0 + z} \right)^2 \right] - g + q_z \label{eq:z_dynamics}\\
	\frac{di}{dt} &= \frac{2 \dot{z}}{s_0 + z}i + \frac{2(s_0 + z)}{\mu_0 n^2 A}(u - Ri) + q_i \label{eq:z_dot_dynamics}
\end{align}
where $m$ is the rotor mass, $z$ is the vertical displacement of the rotor, $u$ is the voltage input, $s_0$ is the air gap between rotor and electromagnets at the equilibrium point, $\mu_0$ is vacuum permeability, $n$ is the number of windings in each coil, $R$ is its resistance, $A$ is its cross-sectional area, $k_z$ is the axial stiffness coefficient and $g$ is the gravitational acceleration. The unknown perturbations $q_z(t)$ and $q_i(t)$ are bounded with $\vert q_z(t) \vert \leq Q_z, \; \vert q_i(t) \vert \leq Q_i$, $\forall t \geq 0$ and account for the lumped model uncertainties and disturbances in the two subsystems. Table \ref{tab:par_tab} lists the values of the system parameters.
\begin{table}[bp]
	\begin{center}
		\caption{System parameter values}
		\label{tab:par_tab}
		\begin{tabular}{clc}
	\toprule
	\textbf{Symbol}	& \textbf{Description} & \textbf{Value}\T\B \\
	\specialrule{.2em}{.1em}{-1em}\\
	$m$ 	& Mass of rotor and axle	& $0.588 \; \si{\kilogram}$\T\B \\
	$k_z$ 	& Axial stiffness 			& $-754 \; \si{\newton\meter}$\T\B \\
	$\mu_0$ & Vacuum permeability		& $1.25\cdot 10^{-6} \; \si{\newton\per\ampere^2}$\T\B \\
	$n$ 	& Number of coil windings	& $1480$\T\B \\
	$A$ 	& Cross-sectional area 		& $0.121 \; \si{\meter^2}$\T\B \\
	$s_0$ 	& Air gap size 				& $5\cdot 10^{-3} \; \si{\meter}$\T\B \\
	$i_0$ 	& Bias current				& $0.25 \; \si{\ampere}$\T\B \\
	$R$ 	& Coil resistance			& $41.44 \; \si{\ohm}$\T\B \\
	$g$ 	& gravitational acceleration& $9.81 \; \si{\meter\per\sec^2}$\T\B \\\hline
\end{tabular}
	\end{center}
\end{table}