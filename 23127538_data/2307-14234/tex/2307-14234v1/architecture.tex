Consider the system in \eqref{eq:z_dynamics}, \eqref{eq:z_dot_dynamics} where all three states are available from measurements. The objective is to control the rotor axle vertical position such that the displacement $z$ is regulated at zero. The position and velocity of the rotor can be controlled by means of the current deviation $i$, which in turn can be controlled through the voltage $u$. Instead of employing backstepping strategies that often lead to complex designs, this study will pursue a modular architecture starting from defining the scaled magnetic force as virtual control input:
\begin{equation} \label{eq:v_definition}
	v(z,i) \triangleq \displaystyle \left( \frac{i_0 + i}{s_0 - z} \right)^2 - \left( \frac{i_0 - i}{s_0 + z} \right)^2, \; z\in(-s_0,s_0) \; .
\end{equation}
This essentially renders the mechanical subsystem an undamped mass-spring system. Next, an adaptive estimator will be deployed to invert the nonlinear mapping from the current deviation $i$ to the scaled magnetic force $v(z,i)$ such that the appropriate current reference signal is generated. Once the reference is obtained, the current dynamics will be regulated to track it such that the appropriate magnetic force, demanded by the controller for the mechanical system is generated and applied on the axle mass to position it at $z = 0$. \gls{SMC} will be used in both the mechanical and the electrical systems to ensure robust finite-time convergence of the controlled variable to the desired values. The overall control architecture is illustrated in Fig. \ref{fig:architecture}.
\tikzsetnextfilename{cascade}
% Figure environment removed