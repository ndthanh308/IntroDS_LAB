\RequirePackage{fix-cm}
\documentclass{svjour3}                     %

\smartqed  %

\newcommand\hmmax{0}
\newcommand\bmmax{0}

\usepackage{graphicx}

%\usepackage[subtle]{savetrees}
%\usepackage[margin=2cm]{geometry}
\usepackage{tikz,amsmath, amssymb,bm,color, amsthm,amsfonts}
\usetikzlibrary{positioning, calc,chains,fit,shapes}
%\usetikzlibrary{circuits.logic.US,circuits.logic.IEC,fit}
\usepackage{enumerate}
\usepackage{comment}
\usepackage{tikz}
\usepackage{graphics}
%\usepackage[cm]{fullpage}
\usepackage{longtable}
\usepackage{mdframed}
\usepackage{caption}
\usepackage{subcaption}
\usepackage{slashbox}
\usepackage{url}
\usepackage{framed}
\usepackage{array}
\usepackage{tabu}
\usepackage{lscape}
\usepackage{multirow}
\usepackage{ulem}
\usepackage{multicol}
\usepackage{placeins}
\usepackage{cite}
\usepackage{enumitem}
\usepackage{mathtools}
%\usepackage[numbers]{natbib}
%\usepackage{mathtools}
%\usepackage{authblk}

\mdfsetup{skipabove=2pt,skipbelow=2pt}
%\setlenght {\marginparwidth }{2cm}
%\usepackage{todonotes}

%\usepackage{floatrow}
%\usepackage{adjustbox}
%\setlength{\extrarowheight}{.05ex}
%\renewcommand\thesubfigure{\roman{subfigure}}


%\newtheorem{theorem}{Theorem}[section]
%\newtheorem{lemma}[theorem]{Lemma}
%\newtheorem{observation}[theorem]{Observation}
%\newtheorem{corollary}[theorem]{Corollary}
%\newtheorem{proposition}[theorem]{Proposition}
%\newtheorem{definition}[theorem]{Definition}
\newtheorem{construction}{Construction}
%\newtheorem{conjecture}{Conjecture}
%\newtheorem{remark}[theorem]{Remark}

\newcommand{\pname}[1]{\textnormal{\textsc{#1}}}
\newcommand{\cclass}[1]{\textnormal{\textsf{#1}}}
\newcommand{\nog}{nine} % no of members in the gang!
\newcommand{\nogd}{nineteen} % no of members in the gang - for deletion/completion
\newcommand{\nogl}{eighteen} % no of members in the larger gang - for editing
\newcommand{\nogld}{thirty eight} % no of members in the larger gang - for deletion/completion
\newcommand{\diffnog}{ten} %
%\newcommand{\dominatedby}{dominated by} %
%\newcommand{\dominatingset}{dominating set} %
%\newcommand{\dominates}{dominates} %
\newcommand{\simulates}{simulates} %
\newcommand{\baseset}{base} %
\newcommand{\issimulatedby}{is simulated by} %

\newcommand{\StarSAT}{\pname{8-SAT$_{\geq 6}$}}
\newcommand{\FSAT}{\pname{4-SAT$_{\geq 2}$}}
\newcommand{\FISAT}{\pname{5-SAT$_{\geq 3}$}}
\newcommand{\SIXSAT}{\pname{6-SAT$_{\geq 4}$}}
\newcommand{\ESAT}{\pname{8-SAT$_{\geq 6}$}}
\newcommand{\KSAT}{\pname{$k$-SAT$_{\geq {k-2}}$}}
\newcommand{\KSATO}{\pname{$k$-SAT}}
\newcommand{\ESATO}{\pname{8-SAT}}
\newcommand{\FSATO}{\pname{4-SAT}}
\newcommand{\FISATO}{\pname{5-SAT}}
\newcommand{\TSAT}{\pname{3-SAT}}
\newcommand{\HED}{\pname{${H}$-free Edge Deletion}}
\newcommand{\AEE}{\pname{${A}$-free Edge Editing}}
\newcommand{\AED}{\pname{${A}$-free Edge Deletion}}
\newcommand{\TSED}{\pname{$t$-star-free Edge Deletion}}
\newcommand{\ATSED}{\pname{Annotated $t$-star-free Edge Deletion}}
\newcommand{\AFSED}{\pname{Annotated $4$-star-free Edge Deletion}}
\newcommand{\FSED}{\pname{$4$-star-free Edge Deletion}}
\newcommand{\FVSED}{\pname{$5$-star-free Edge Deletion}}
\newcommand{\HEE}{\pname{${H}$-free Edge Editing}}
\newcommand{\HEC}{\pname{${H}$-free Edge Completion}}
\newcommand{\HDEE}{\pname{${H'}$-free Edge Editing}}
\newcommand{\HDDEE}{\pname{${H''}$-free Edge Editing}}
\newcommand{\HDED}{\pname{${H'}$-free Edge Deletion}}
\newcommand{\HDEC}{\pname{${H'}$-free Edge Completion}}
\newcommand{\HBEE}{\pname{${\overline{H}}$-free Edge Editing}}
\newcommand{\HBED}{\pname{${\overline{H}}$-free Edge Deletion}}
\newcommand{\HBEC}{\pname{${\overline{H}}$-free Edge Completion}}
\newcommand{\HOEDCE}{\pname{${H_1}$-free Edge Deletion(Completion/Editing)}}
\newcommand{\HEDCE}{\pname{${H}$-free Edge Deletion(Completion/Editing)}}
\newcommand{\HEEDC}{\pname{${H}$-free Edge Editing(Deletion/Completion)}}
\newcommand{\HDEEDC}{\pname{${H'}$-free Edge Editing(Deletion/Completion)}}
\newcommand{\BFED}{\pname{Bow-free Edge Deletion}}
\newcommand{\ABFED}{\pname{Annotated Bow-free Edge Deletion}}
\newcommand{\DTIS}{\pname{Distance-3 Independent Set}}
\newcommand{\SVC}{\pname{Strong Vertex Cover}}
\newcommand{\CLIQUE}{\pname{Clique}}
\newcommand{\IS}{\pname{Independent Set}}
\newcommand{\PFS}{\pname{Propagational-$f$ Satisfiability}}
\newcommand{\RHED}{\pname{Restricted ${H}$-free Edge Deletion}}
\newcommand{\RHEC}{\pname{Restricted ${H}$-free Edge Completion}}
\newcommand{\RHDED}{\pname{Restricted ${H'}$-free Edge Deletion}}
\newcommand{\RHDEC}{\pname{Restricted ${H'}$-free Edge Completion}}
\newcommand{\RHEE}{\pname{Restricted ${H}$-free Edge Editing}}
\newcommand{\PH}{$\cclass{NP} \subseteq \cclass{coNP/poly}$}
\newcommand{\NOPH}{$\cclass{NP} \not\subseteq \cclass{coNP/poly}$}
\newcommand{\LG}{\mathcal{W}}
\newcommand{\LGD}{\mathcal{W}'}
\newcommand{\LGDD}{\mathcal{W}''}


%\let\oldvee\vee
\renewcommand\vee{\boxtimes}

\newcommand\addvmargin[1]{
  \node[fit=(current bounding box),inner ysep=#1,inner xsep=0]{};
}
\setlength{\fboxrule}{0pt}

\newcommand{\defstage}[2]{% PGD Version
  \hfill\\\smallskip\noindent%
  \begin{tabularx}{\textwidth}{|l X|}%
    \hline%
    \multicolumn{2}{|l|}{\textbf{#1}}\\%
    &#2\\\hline%
  \end{tabularx}%
%  \smallskip%
}
\setlength\extrarowheight{15pt}

\newcounter{rowcntr}[table]
\renewcommand{\therowcntr}{\thetable.\arabic{rowcntr}}

% A new columntype to apply automatic stepping
\newcolumntype{N}{>{\refstepcounter{rowcntr}\therowcntr}c}

% Reset the rowcntr counter at each new tabular
\AtBeginEnvironment{longtabu}{\setcounter{rowcntr}{0}}

\newcounter{rowcntra}[table]
\renewcommand{\therowcntra}{\arabic{rowcntra}}

% A new columntype to apply automatic stepping
\newcolumntype{M}{>{\refstepcounter{rowcntra}\therowcntra}c}

% Reset the rowcntr counter at each new tabular
\AtBeginEnvironment{tabular}{\setcounter{rowcntra}{0}}

\newcommand{\NPC}{NP-Complete}


\newcommand{\highlight}[1]{\textcolor{blue}{#1}}
\newcommand{\dhanya}[1]{\textcolor{blue}{dhanya: #1}}


%\newcommand{\XCD1}[1]{\pname{$\chi_{cd}$\ensuremath{(#1)}}}
\newcommand{\XCD}{\pname{$\chi_{cd}$}}
\newcommand{\SC}{\pname{$\omega_{s}$}}

\newcommand{\CDC}{\textsc{CD-coloring}}
\newcommand{\SCP}{\textsc{Separated-Cluster}}
\newcommand{\TD}{\textsc{Total Domination}}
\newcommand{\ISP}{\textsc{Independent Set}}
\newcommand{\CC}{\textsc{Clique Cover}}
\newcommand{\TETHS}{Further, the problem cannot be solved in time \ensuremath{2^{o(|V(G)|)}}, unless the ETH fails}
%\usetikzlibrary{positioning,chains,shapes,calc}
\usetikzlibrary{fit}
\thispagestyle{empty}
\usetikzlibrary{
  graphs,
  graphs.standard
}
 \journalname{Celestial Mechanics and Dynamical Astronomy}
\begin{document}
\title{Stable sets mapping with Taylor differential algebra with application to ballistic capture orbits}




\author{T. Caleb \and
G. Merisio \and
P. di Lizia \and
F. Topputo
}

\institute{Thomas Caleb (corresponding author) \at
              ISAE-SUPAERO, 10 Avenue Édouard Belin, 54032 - 31055, Toulouse, France\\
              OrcID: 0000-0002-4027-3340\\
              \email{thomas.caleb@student.isae-supaero.fr}
           \and
           Gianmario Merisio \at
              Department of Aerospace Science and Technology, Politecnico di Milano, Via La Masa 34, 20156, Milano, Italy\\
              OrcID: 0000-0001-8806-7952\\
              \email{gianmario.merisio@polimi.it}
           \and
           Pierluigi di Lizia  \at
           	  Department of Aerospace Science and Technology, Politecnico di Milano, Via La Masa 34, 20156, Milano, Italy\\
           	    OrcID: 0000-0003-1692-3929\\
              \email{pierluigi.dilizia@polimi.it}
           \and
           Francesco Topputo  \at
              Department of Aerospace Science and Technology, Politecnico di Milano, Via La Masa 34, 20156, Milano, Italy\\
              OrcID: 0000-0002-5369-6887\\
              \email{francesco.topputo@polimi.it}
}


\date{Received: date / Accepted: date}

\maketitle              %
\begin{abstract}
Ballistic capture orbits offer safer Mars injection at longer transfer time. However, the search for such an extremely rare event is a computationally-intensive process. Indeed, it requires the propagation of a grid sampling the whole search space. This work proposes a novel ballistic capture search algorithm based on Taylor differential algebra propagation. This algorithm provides a continuous description of the search space compared to classical grid sampling research and focuses on areas where the nonlinearities are the largest. Two criteria, named consistency and quality, are defined to assess this new algorithm  and to compare its performances with classical grid sampling of the search space around Mars. Results show that differential algebra mapping works on large search spaces, and automatic domain splitting captured the dynamical variations on the whole domain successfully. The consistency criterion shows that more than 87\% of the search space is guaranteed as accurate, and the quality criterion demonstrates that the quality criterion is kept over 80\%.



\keywords{Ballistic capture \and Stable sets \and Weak stability boundary \and Taylor differential algebra \and Mars}
\end{abstract}

\section{Introduction}
Current quantum hardware is unable to carry out universal quantum computations due to the buildup of errors that occur during the computation. 
The magnitude of the individual error is currently above the value that the Threshold Theorem requires in order to kick-start quantum error correction and fault-tolerant quantum computation~\cite[Section 10.6]{nielsen_chuang_2010}. 
Although the experimentally achieved fidelity rates are promising and the error bounds are inching closer to the required threshold, we will have to work for the foreseeable future with quantum hardware with errors that build-up during the computation.  This implies that we can only do a limited number of steps before the output of the computation has become completely uncorrelated with the intended one.

For fault-tolerant quantum computing, we repeat four steps: 
1) We apply a number of single and two-qubit quantum gates, in parallel whenever possible; 
2) We perform a syndrome measurement on a subset of the qubits; 
3) We perform fast classical computations to determine which errors have occurred and how to correct them; 
and, 4) We apply correction terms based on the classical computations.
We then repeat these four steps with a next sequence of gates. 
These four steps are essential to fault-tolerant quantum computing. 


The starting point of this work is to use the four steps outlined above, not to carry out error correction and fault-tolerant computation, but to enhance short, constant-depth, {\em uncorrected} quantum circuits that perform single qubit gates and {\em nearest-neighbor} two qubit gates. 
Since in the long run we will have to implement error-correction and fault-tolerant computation anyhow, and this is done by such a four-step process, why not make other use of this architecture? Moreover, on some of the quantum hardware platforms, these operations are already in place.
Embracing this idea we naturally arrive at the question: what is the computational power of \textit{low-depth} quantum-classical circuits organized as in the four steps outlined above? 
We thus investigate circuits that execute a small, ideally constant, number of stages, where at each stage we may apply, in parallel, single qubit gates and {\em nearest-neighbor} two qubit gates, followed by measurements, followed by low-depth classical computations of which the outcome can control quantum gates in later stages. 
It is not clear, at first, whether such circuits, especially with constant depth, can do anything remotely useful. 
But we will see that this is indeed the case: many quantum computations can be done by such circuits in constant depth. 
By parallelizing quantum computations in this way, we improve the overall computational capabilities of these circuits, as we do not incur errors on qubits that are idle, simply because qubits are not idle for a very long time. 
Furthermore, reducing the depth of quantum circuits, at the cost of increasing width, allows the circuit to be run faster even if errors occur.

The first usage of such a four-step layout, not to do error correction, but to perform computations, can be found in the paradigm of measurement-based quantum computing~\cite{gottesman1999demonstrating,raussendorf2001one,jozsa2006introduction,clark2007generalised}: 
A universal form of quantum computing where a quantum state is prepared and operations are performed by measuring qubits in different bases, depending on previous measurements and intermediate measurements.

\citeauthor{PhamSvore2013} were the first to formalize the four-step protocol for performing computations~\cite{PhamSvore2013}. They included specific hardware topologies by considering two-dimensional graphs for imposing constraints on qubit interactions. In their model, they develop circuits for particularly useful multi-qubit gates, including specifying costs in the width, number of qubits, depth, number of concurrent time steps, size, and total number of non-Identity operations.
As a result, they find an algorithm that factors integers in polylogarithmic depth.
\citeauthor{Browne:2011} showed that the main tool in the work by \citeauthor{PhamSvore2013}, the fan-out gate, can also be replaced by additional log-depth classical computations in the measurement-based quantum computing setting~\cite{Browne:2011}.

More recently, \citeauthor{Cirac:2021} introduced a scheme to implement unitary operations involving quantum circuits combined with Local Operations and Classical Communication ($\mathsf{LOCC}$) channels: $\mathsf{LOCC}$-assisted quantum circuits~\cite{Cirac:2021}. Similarly to the four-step scheme we just described, they allow for a short depth circuit to be run on the qubits, followed by one round of $\mathsf{LOCC}$, in which ancilla qubits are measured and local unitaries are applied based on the measurement outcomes. They show that in this model any 1D transitionally invariant matrix-product state (MPS) with fixed bond dimension is in the same phase of matter as the trivial state. Similar ideas can be found in~\cite{TVV_NonAbelianTopologicalOrder_2022, tantivasadakarn2021long}.

In this work, we introduce a new model, called \textit{Local Alternating Quantum-Classical Computations} ($\LAQCC$). In this model we alternate between running quantum circuits (constrained by locality), ending in the measurement of a subset of qubits, and fast classical computations based on the measurement results. The outcome of the classical computations are then used to control future quantum circuits. We allow for flexibility in this model, by giving different constraints to the power of both the quantum circuits and the classical circuits as well as the number of alternations between them. 
Most attention will be given to $\LAQCC$ containing quantum circuits of constant depth, classical circuits of logarithmic depth and at most a constant number of alternations between them. 
Any circuit constructed in this model is considered to be of constant depth. 
We restrict ourselves to logarithmic depth classical computations, as this is the first natural and non-trivial extension beyond constant-depth classical computations. 
Constant-depth classical computations do however also have an equivalent constant-depth quantum implementation.

The definition of $\LAQCC$ sharpens the original definition of \citeauthor{PhamSvore2013} by adding constraints to the intermediate classical computations. This allows us to bound the power of $\LAQCC$ from above. 

The main result of \citeauthor{Cirac:2021}, that 1D translational invariant MPS with fixed bond dimension can be prepared by $\mathsf{LOCC}$-assisted circuits, relies on local symmetries of the MPS. These symmetries allow them to prepare local states (on a constant number of qubits) and glue them together by doing one round of the appropriate entangling measurement and corrections, after which they run a round of local unitaries to get the desired result. This general scheme for preparing states that exhibit an MPS description with the appropriate local symmetries requires only geometrically local unitaries and one round of measurement and corrections an therefore is accessible in $\LAQCC$. Studying different local symmetries, known as Symmetry Protected Topological (SPT) phases of matter, to find measurement-based constant depth circuits for states is a broad ongoing field of research~\cite{TVV_NonAbelianTopologicalOrder_2022, tantivasadakarn2021long, smith2023deterministic}. 
All these schemes have a $\LAQCC$ implementation.

%$\LAQCC$-circuits also exist for general schemes of preparing local states, based on the local tensors, and gluing them together using one round of entangled measurement and corrections, based on the local symmetry. 
%The main result of \citeauthor{Cirac:2021}, that 1D translational invariant MPS with fixed bond dimension can be prepared by $\mathsf{LOCC}$-assisted circuits, relies heavily on local symmetries of the MPS and as a result also has an equivalent $\LAQCC$ implementation. 
%The corrections applied after the measurement round are local unitaries depending on the local symmetries of the MPS. 

 

%This general scheme of preparing local states, based on the local tensors, and gluing it together by doing one round of entangled measurement and corrections, based on the local symmetry, is accessible in $\LAQCC$.
Note however that \citeauthor{Cirac:2021} also suggest a circuit for the $W$-state.
This circuit uses sequentially and dependent measurement-based corrections of the ancilla qubits. 
These dependent measurements translate to sequential alternations between the quantum and classical circuits and therefore increase the total depth to linear depth, exceeding the constant-depth constraints imposed by $\LAQCC$-circuits. 

We study the power of the $\LAQCC$ model with respect to state preparation, showing that even with only constant quantum-depth and logarithmic classical depth it remains possible to prepare states with long-range entanglement.
Another surprising result is that it is unlikely that $\LAQCC$ circuits are classically simulatable. We show that any instantaneous quantum polynomial-time (IQP) circuit~\cite{Bremner2010,Shepherd2009} has an $\LAQCC$ implementation.
Classical simulation of IQP circuits implies the collapse of the polynomial hierarchy to the third level, which is not believed to be true~\cite{Bremner2017}. Therefore, we expect that $\LAQCC$ circuits are unlikely to be classically simulatable. We bound the power of $\LAQCC$ by showing that it is contained in $\QNC^1$, the class of polynomial-size, log-depth circuits.

Next, we also study the power that intermediate classical calculations can add to quantum computations, by considering a new model that alternates between polynomially many polynomial-depth quantum circuits and unbounded classical computations
We study this model by doing a complexity theoretical analysis, where we draw inspiration from the notions of complexity given by \citeauthor{RosenthalYuen:2022}, \citeauthor{MetgerYuen:2023}, and \citeauthor{Aaronson:2004}.
All three complexity notions are based on the notion of state preparation, instead of more traditional definition of complexity such as the decidability of a computational problem. 
The first two consider classes based on sequences of quantum states preparable by a polynomial-sized quantum circuit, where the circuits are uniformly generated by a computational class, for instance, the class $\mathsf{PSPACE}$, which results in the complexity class $\mathsf{StatePSPACE}$~\cite{RosenthalYuen:2022,MetgerYuen:2023}.
The third notion considers a relative complexity, where the complexity is measured between two given states, and is measured by the number of gates, from a given gate-set, required to transform one state in another state~\cite{Aaronson:2004}. 
For our definition of state preparation complexity, we drop the uniformity constraint from~\cite{RosenthalYuen:2022,MetgerYuen:2023} and define a class as $\mathsf{StateX}$, which refers to states preparable by circuits of type $\mathsf{X}$. 
As an example, if $\mathsf{X} = \QNC^0$, this results in the class $\mathsf{StateQNC^0}$, which is the set of states preparable from the $\ket{0}^n$ state by poly-size constant-depth circuits. 
This notion is similar to the relative complexity from~\cite{Aaronson:2004}, where one state is the  $\ket{0}^n$ state and instead of counting the number of gates we consider the set of states preparable by a fixed number of gates. Using this notion of complexity we show that any state preparable by an $\LAQCC^*$ circuit is also preparable by a $\mathsf{PostQPoly}$ circuit, the class of circuits of polynomial depth with an additional post-selection gate. 

All Clifford circuits have a constant-depth $\LAQCC$ implementation, implying that any stabilizer state can be implemented by a constant-depth $\LAQCC$ circuit, see Section~\ref{sec:clifford_circuits} for a proof of this statement. 
Efficient circuits for stabilizer states have been known already through measurement-based quantum computing. Therefore this paper focuses on the preparation of non-stabilizer states, and as a surprising result we find novel constant-depth protocols for four very natural classes of non-stabilizer states.
Despite the extensive research into these four classes of non-stabilizer states and the many applications of them, no efficient constant- or low-depth state preparation protocols are known yet. We specifically consider these four classes as they are all often used as initial states in other algorithms.

The first state is a uniform superposition over an arbitrary number of states. 
This state finds applications in many quantum algorithms, as they often start with a uniform superposition over multiple states. 
This superposition is often achieved by applying Hadamard gates to every qubit due to its simplicity to prepare. 
Yet, the analysis of many algorithms, such as Shor's algorithm~\cite{Shor:1997}, would benefit from a different initial superposition. 
The circuit to prepare the uniform superposition over an arbitrary number of states uses an exact version of Grover search as a subroutine, that turns a probabilistic circuit, with a known constant probability of success, into a deterministic circuit. 
We use the circuit for preparing a uniform superposition over an arbitrary number of states as a subroutine in the next two quantum state preparation protocols. 

The second state is the $W$-state, the uniform superposition over all computational basis states of Hamming-weight~$1$, a natural long-ranged entangled state that displays a fundamentally nonequivalent type of entanglement from the Greenberger–Horne–Zeilinger state~\cite{WState:2000}, for which $\LAQCC$-type constant-depth circuits were previously known~\cite{PhamSvore2013, Cirac:2021}. 
The $W$-state is often used as benchmark for new quantum hardware~\cite{Haffner2005,Neeley2010,GarciaPerez:2021}. 
A novel way to prepare the $W$-state therefore gives a new way to benchmark different quantum devices with each other. 
A circuit for preparing the $W$-state was given in~\cite{Cirac:2021}, but this implementation requires sequentially alternating measurements followed by local unitaries, which in the $\LAQCC$ model is not considered to be of constant depth. 
We improve this protocol by giving an $\LAQCC$ implementation of the $W$-state, based on a compress-uncompress method that links the one-hot and binary encoding of integers.

The third state considered is the Dicke state, a generalization of the $W$-state, a superposition over all computational basis states with Hamming-weight $k$~\cite{Dicke:1954}. 
Dicke states have relevance in various practical settings.
For instance, for quantum game theory~\cite{zdemir2007}, quantum storage~\cite{Bacon_Compress:2006,Plesch:2010}, quantum error correction~\cite{ouyang2014permutation}, quantum metrology~\cite{toth2012multipartite}, and quantum networking~\cite{prevedel2009experimental}. 
Dicke states have been used as a starting state for variational optimization algorithms, most notably Quantum Alternating Operator Ansatz (QAOA)~\cite{Hadfield2019}, to find solutions to problems such as Maximum k-vertex Cover~\cite{Brandhofer2022,cook2020quantum}.
The ground states of physical Hamiltonians describing one-dimensional chains tend to show a resemblance to Dicke states such as states resulting from the Bethe ansatz, making them an ideal starting state when investigating the ground state behavior of these Hamiltonians~\cite{TDL_BetheAnsatzDerivation:2010,B_ExcitedStateQuantumPhaseTransitions:2013,DickeTransitions:2021}. 
For instance, the algorithm by \citeauthor{van2021preparing}, who give an algorithm to prepare the Bethe ansatz eigenstates of the spin-1/2 XXZ spin chain, starts by first preparing a Dicke state~\cite{van2021preparing}. 
A Dicke-state preparation protocol based on the compress-uncompress methodology used in the $W$-state furthermore finds applications in entanglement distillation, where the entanglement of a large state is concentrated on only a few qubits. 
Efficient deterministic circuits for preparing Dicke states have been proposed by \citeauthor{bartschi2019deterministic}~\cite{bartschi2019deterministic, bartschi2022deterministic_short_depth}. 
They provide a quantum circuit of depth $\mathO(k \log(\frac{n}{k}))$, allowing arbitrary connectivity, to prepare a Dicke state, which they conjecture to be optimal when $k$ is constant. 
In this work, we provide a constant-depth $\LAQCC$ circuit below their conjectured bound already for constant $k$. 
However, this does not directly disprove their conjecture, as we allow for intermediate measurements and classical computations. 
More significantly, we even construct constant-depth $\LAQCC$ circuits for $k = \mathO(\sqrt{n})$ greatly improving their bound.
This construction extends the compress-uncompress method for the $W$-state combined with additional subroutines. 

We continue with a log-depth state preparation protocol for the Dicke-state for arbitrary $k$. 
This protocol implements an efficient transformation between the factoradic number representation and the combinatorial number representation of a positive integer. 
The combinatorial number representation relates directly to the Dicke state. 
The provided efficient transformation between number representation systems might be of independent interest. 

We conclude by modifying our protocol for preparing a Dicke-state to a protocol that prepares quantum many-body scar states in constant-depth. 
These states have low entanglement and longer coherence times than states with similar energy density.
These characteristics make many-body scar states interesting to analyze and relevant within physics.
Many-body scar states appear for instance in the AKLT model~\cite{AKLT:1987,MRBAR:2018,MRB:2018} and different spin models~\cite{SI:2019,MOBFR:2020}.
Known methods for preparing these states have polynomial-depth~\cite{Gustafson:2023}, whereas our circuit has constant depth. 

% We conclude by studying the power that intermediate classical calculations can add to quantum computations. 
% In this study, we define a new model that relaxes constant-depth quantum circuits to polynomial depth quantum circuits, log-depth classical calculations to unbounded classical computations and a constant number of alternations to a polynomial number of alternations. 
% We call this model $\LAQCC^*$. 
% We study this model by doing a complexity theoretical analysis, where we draw inspiration from the notions of complexity given by \citeauthor{RosenthalYuen:2022}, \citeauthor{MetgerYuen:2023}, and \citeauthor{Aaronson:2004}.
% All three complexity notions are based on the notion of state preparation, instead of more traditional definition of complexity such as the decidability of a computational problem. 
% The first two consider classes based on sequences of quantum states preparable by a polynomial-sized quantum circuit, where the circuits are uniformly generated by a computational class, for instance, the class $\mathsf{PSPACE}$, which results in the complexity class $\mathsf{StatePSPACE}$~\cite{RosenthalYuen:2022,MetgerYuen:2023}.
% The third notion considers a relative complexity, where the complexity is measured between two given states, and is measured by the number of gates, from a given gate-set, required to transform one state in another state~\cite{Aaronson:2004}. 
% For our definition of state preparation complexity, we drop the uniformity constraint from~\cite{RosenthalYuen:2022,MetgerYuen:2023} and define a class as $\mathsf{StateX}$, which refers to states preparable by circuits of type $\mathsf{X}$. 
% As an example, if $\mathsf{X} = \QNC^0$, this results in the class $\mathsf{StateQNC^0}$, which is the set of states preparable from the $\ket{0}^n$ state by poly-size constant-depth circuits. 
% This notion is similar to the relative complexity from~\cite{Aaronson:2004}, where one state is the  $\ket{0}^n$ state and instead of counting the number of gates we consider the set of states preparable by a fixed number of gates. Using this notion of complexity we show that any state preparable by an $\LAQCC^*$ circuit is also preparable by a $\mathsf{PostQPoly}$ circuit, the class of circuits of polynomial depth with an additional post-selection gate. 

\paragraph{Summary of results}
\begin{itemize}
    \item We give a new definition of a computational model that captures the power of the four step process: applying a constant number of layers of one- and two-qubit gates; performing a syndrome measurement; perform a fast classical computation determining corrections; apply corrections. We call this model \emph{Local Alternating Quantum Classical Computations}, or $\LAQCC$ for short. In this model we bound the allowed quantum operations, intermediate classical calculations, and number of rounds separately. In Section~\ref{sec:LAQCC_model} we define this model and give a list of operations based on results from literature contained in this computational model. In some of these operations we explicitly use that we allow for multiple, but at most constant, rounds  of corrections.
    \item  We show show that there exist $\LAQCC$ circuits that can not be weakly simulated in Section~\ref{sec:IQP_in_LAQCC}. We further show that for every $\LAQCC$ circuit there exists a $\QNC^1$ circuit simulating it perfectly, in Section~\ref{sec:LAQCC_in_QNC1}.
    \item We introduce a new type computational complexity for preparing states and show that the extension of $\LAQCC$ where we allow a polynomial number of rounds and unbounded classical computation, is contained in $\mathsf{PostQPoly}$, the class of polynomial circuits with post-selection, in Section~\ref{sec:Complexity results}.
    \item We show a protocol to prepare the uniform superposition state of size $q$ in $\LAQCC$ using $\mathO(\ceil{\log_2(q)}^2)$ qubits in Section~\ref{sec:superposition_modulo_q}. 
    \item We show a protocol to prepare the $W_n$ state in $\LAQCC$ using $\mathO(n\log(n))$ qubits in Section~\ref{sec:W_state_in_LAQCC}.
    \item We show two ways of preparing the Dicke-$(n,k)$ state. The first method is in $\LAQCC$, works up to $k = \mathO(\sqrt{n})$, uses $\mathO(n^2\log(n))$ qubits, and is found in Section~\ref{sec:dicke:small_k}. The second method is in $\LAQCC\text{-}\mathsf{LOG}$ (an extension of $\LAQCC$ allowing for logarithmic number of alterations instead of constant), works for any $k$, uses $\mathO(\text{poly}(n))$ qubits, and is found in Section~\ref{sec:Dicke_in_LAQCC_LOG}. 
    \item We extend on our $\LAQCC$ method of generating Dicke-$(n,k)$ states for $k = \mathO(\sqrt{n})$ and show a protocol to generate many-body scar states for a particular Hamiltonian in $\LAQCC$ (Section~\ref{sec:many_body_scar}). 
\end{itemize}
Summarized in a table, we provide the following state generation protocols:
\begin{table}[htb]
\centering
\begin{tabular}{l|l|l|l}
\textbf{State description} & \textbf{Width} & \textbf{Depth} & \textbf{Implementation}\\
\hline 
Uniform superposition mod $q$: $\frac{1}{\sqrt{q}} \sum_{i = 0}^{q-1}\ket{i}$ & $\mathO(\ceil{\log^2 q})$ & $\mathO(1)$ & Section~\ref{sec:superposition_modulo_q}\\

$W$-state: $\frac{1}{\sqrt{n}}\sum_{i = 0}^{n-1}\ket{e_i}$ & $\mathO(n \log n)$ & $\mathO(1)$ & Section~\ref{sec:W_state_in_LAQCC}\\

Dicke-$(n,k)$, $k = \mathO(\sqrt{n})$: $\binom{n}{k}^{-1/2}\sum_{x \in \{0,1\}^n: |x| = k} \ket{x}$ &  $\mathO(n^2\log n)$ & $\mathO(1)$ 
&Section~\ref{sec:dicke:small_k}\\

Dicke-$(n,k)$: $\binom{n}{k}^{-1/2}\sum_{x \in \{0,1\}^n: |x| = k} \ket{x}$ & $\mathO(\text{poly}(n))$ & $\mathO(\log n)$ &Section~\ref{sec:Dicke_in_LAQCC_LOG}\\

QMBS: $\ket{S_k} = \frac{1}{k! \sqrt{\mathcal N(n,k)}}(Q^\dagger)^k \ket{\Omega}$ &  $\mathO(n^2\log n)$ & $\mathO(1)$  &  Section~\ref{sec:many_body_scar}
\end{tabular}
\caption{Summary of state preparation protocols given in this paper.}
\label{tab:sate_prep}
\end{table}
In the entry for the quantum many-body scar state $Q$ denotes the raising operator and $\mathcal N(n,k)=\binom{n-k-1}{k}$. 
Section~\ref{sec:many_body_scar} will provide more details on the variables and the implementation. 

\paragraph{Organization of the paper}
\noindent We first introduce relevant preliminaries in Section~\ref{sec:preliminaries}. 
In Section~\ref{sec:LAQCC_model} we formally define the class of Local Alternating Quantum-Classical Computations ($\LAQCC$). We also show that any Clifford circuit can be implemented in constant depth $\LAQCC$ (a result based on a result from measurement-based quantum computing~\cite{jozsa2006introduction}). 
This result allows us to give many useful multi-qubit gates and routines in Section~\ref{sec:gates_created_in_LAQCC}. 
Beyond that we show that constant depth $\LAQCC$ circuits are contained in $\QNC^1$ and that any $\mathsf{IQP}$ circuit has an $\LAQCC$ implementation.
We conclude this section with an analysis of a more powerful instantiation of $\LAQCC$ and show an inclusion with respect to the class $\mathsf{PostQPoly}$, which is the class of circuits of polynomial depth with one additional post-selection gate. 
In Section~\ref{sec:state_prep_in_LAQCC} we give $\LAQCC$ circuit implementations for preparing the uniform superposition over an arbitrary number of states, the $W$-state and the Dicke state up to $k = \mathO(\sqrt{n})$. We furthermore give a log-depth circuit implementation for preparing the Dicke state for any $k$. We conclude by showing a $\LAQCC$ circuit for generating many body scar states of a particular type of Hamiltonian.


\vspacebeforesection
\section{Background}
\label{sec:background}

In this section, we provide the necessary background information to ensure a comprehensive understanding of the attack described in this paper. We start with a description of the Distributed Hash Table (DHT) used by IPFS, followed by its content resolution mechanisms. We also detail techniques for network size estimation, necessary for our attack detection and mitigation mechanisms.

\vspacebeforesection
\subsection{IPFS DHT}
\label{sec:kad_dht}

We review the features of the Kademlia DHT~\cite{maymounkov2002kademlia} and its \texttt{libp2p} implementation~\cite{libp2p_github} that are the most relevant to our attack.
To participate in the DHT, each peer generates a public/private key pair and derives an identity $\peerid \in \{0,1\}^{256}$ as the hash of its public key.
Ideally, each peer generates a random key pair and, therefore, peer IDs are distributed uniformly and independently over the space $\{0,1\}^{256}$.
While honest nodes follow this rule, malicious nodes may generate and choose from an arbitrary number of key pairs.
Each peer maintains a routing table consisting of $m=256$ buckets.
The $i$-th bucket contains the addresses of up to $k=20$ peers whose peer IDs share a common prefix of exactly $i$ bits with the peer's own peer ID. 

%
A new participant node joins the IPFS network by contacting one of the hardcoded bootstrap nodes. This bootstrap node provides the new node with some initial peers allowing it to join the DHT. The new node uses this information to perform a walk through the DHT towards its own peer ID.
The walk allows to: \textit{(i)}~make sure that there is no other node in the network with the same ID; \textit{(ii)}~discover new peers and fill the newcomer's DHT routing table. At the same time, the newcomer establishes \bitswap~\cite{de2021accelerating} connections to a subset of encountered peers (usually around 300 of them). The core role of the \bitswap protocol is to enable bilateral content transfer and to play the role of a cache for recently-accessed content.

The main DHT operation $\Call{GetClosestPeers}{\key}$ returns the $k=20$ closest peers to $\key$. 
%
In Kademlia, the distance between two keys $x$ and $y$ in the key space is given by $x \oplus y \in \{0,...,2^{256}-1\}$, where $\oplus$ denotes the bitwise XOR operation on the keys; the resulting binary string is interpreted as an integer.
%
When a client wants to find the peers with IDs closest to $\key$, it sends a request to the $\alpha=3$ peers in its routing table whose peer IDs are closest to $\key$. Each of these peers returns the $k$ closest peers to $\key$ in its own routing table and the addresses of these peers. 
%
The client again sends a request to the $\alpha$ peers closest to $\key$, among peers in its routing table and those whose addresses it just received. This process repeats until the client does not find any more peers closer to $\key$.
Due to network churn and imperfect routing tables, we observed in our experiments that successive calls to $\Call{GetClosestPeers}{\key}$ do not always return the same set of $k=20$ peers (we provide more details in \Cref{sec:evaluation}, \Cref{fig:20closest}). This is an important limitation affecting our attack.

\vspacebeforesection
\subsection{Content Resolution in IPFS}
\label{sec:ipfs}

IPFS is a content-centric network.
It allows its participant to request files without specifying their location. 
%
Content is indexed by content IDs $\cid \in \{0,1\}^{256}$ that are derived from a hash of that content.
Both peer IDs and CIDs are used as keys in the DHT.
Each node can play the role of a \provider, \downloader, or \resolver. 
The process of content advertisement and resolution is illustrated in \Cref{fig:add_get_provider}.

%
When a \provider wishes to publish content with a given $\cid$ on IPFS, it creates a \emph{provider record} that contains $cid$ and the \provider's address.
During a $\Call{Provide}{\cid}$ operation, the \provider first uses $\Call{GetClosestPeers}{\cid}$ to locate the $k=20$ peers with their peer IDs closest to $\cid$, 
%
and then sends them a $\mathsf{PutProvider}$ message including the provider record (\Cref{fig:add_get_provider}(a)).
We call the peers that hold provider records for $\cid$ the \emph{resolvers} for $\cid$.

Each CID can have several \providers. In fact, by default, each IPFS client becomes a provider for each piece of content it downloads for a fixed amount of time (12h, 24h, or 48h depending on the client version or custom configuration). As a result, the system provides an auto-scaling feature with supply automatically rising with demand.

%
When a \downloader wishes to fetch a piece of content, it first sends a request to all its \bitswap peers. If none of them has the content, the \downloader uses the DHT-based resolution system. We stress that the \bitswap protocol plays the supporting role of a cache in the dissemination of popular files. However, the mechanism does not provide reliable content resolution, in particular for new or less popular content. %

When \bitswap unstructured search fails, the \downloader resolves $\cid$ using $\Call{FindProviders}{\cid}$. This operation uses a DHT walk identical to that of $\Call{GetClosestPeers}{\cid}$ to find $k$ \resolvers but also queries encountered nodes for a provider record for $\cid$ (\Cref{fig:add_get_provider}(b)). The process terminates when either 20 \providers have been found, or all \resolvers have been asked. Querying all encountered nodes (\ie, not only the designated \resolvers) is useful because some of the encountered nodes may have a provider record in their cache.
%

Upon receiving a provider record, the client connects to the address specified in the provider record to retrieve the actual content (\Cref{fig:add_get_provider}(c)).
Provider records are not authenticated, and therefore malicious \providers may respond with incorrect provider records (or may not respond at all). However, the integrity of the content is preserved because the hash of the retrieved content can be verified against its $\cid$.
%


%

\input{img/add_get_provider.tex}

\vspacebeforesection
\subsection{Network Size Estimator}
\label{sec:netsize}

The number of nodes in a decentralized system is generally unknown due to the avoidance of centralized membership management.
This number is nonetheless useful for optimizations, deciding on individual node configurations, or security mechanisms.
Various methods were proposed for the decentralized estimation of unstructured and structured networks~\cite{eli-sohl-dht-size-estimation,kostoulas2005decentralized, manku2003symphony}.
We use in this work a mechanism developed initially by Protocol Labs as part of a mechanism for decreasing the latency of publishing content in IPFS~\cite{network-size-estimation-notion,network-size-estimation-github-pr}.

%
%
%
%
%
%
%
%
%
%

Each node in the DHT refreshes its routing table periodically (every $10$ minutes in \texttt{libp2p}). 
For this, the node samples $m$ random keys (one for each bucket of its routing table)
%
and queries the DHT to obtain the $k=20$ closest peer IDs to each key.
Using these, the node then computes the average distance between each one of these keys $\key_j$ for $j=1,\dots,m$ and their $i$-th closest peer ID for $i=1,...,k$ (with $m=256$ and $k=20$).
\begin{equation}
    \label{equ:avg-dist}
    \overline{D}_i = \frac{1}{m} \sum_{j=1}^m \operatorname{dist}(\key_j, \peerid_{j}^{(i)})
\end{equation}
where $\peerid_{j}^{(i)}$ is the $i$-th closest peer ID to $\key_j$.
With $N$ peers in the DHT and peer IDs uniformly distributed in the hash space, the expected distance between a $\key$ and its $i$-th closest peer ID is $\frac{2^{256}i}{N+1}$. The node then runs a least square regression to compute the value of $N$ for which the expected distances best fit the empirical average distances, \ie,
\begin{equation}
    \label{equ:netsize-least-squares}
    \hat{N} = \arg\min_{N} \sum_{i=1}^k \left(\overline{D}_i - \frac{2^{256}i}{N+1}\right)^2.
\end{equation}
The resulting estimate $\hat{N}$ can be computed in closed form.
%

When a node starts running, it must perform DHT queries for a few random keys to initialize its network size estimate. 
Since a larger number of queries will result in higher accuracy, making more queries than what is needed to initialize one's routing table is recommended.
Thereafter, keeping the estimate up-to-date does not require any excess DHT queries beyond what is already used for refreshing the routing table as this is done frequently (every 10 minutes).

While the network size estimate has a stochastic variance resulting from the probability distribution of the honest peer IDs, it is hard for an attacker to bias the estimate significantly. Since the estimator uses the density of peer IDs around keys chosen uniformly at random, the adversary would require numerous Sybil nodes (on the order of the whole network size) to significantly affect the peer ID density around those keys.

\section{Cross-Lingual Diffusion Language Model}
\label{sec:XDLMusion}

% In this section, we present our proposed language modeling objectives designed specifically for diffusion and the diffusion model applied for cross-lingual translation. These objectives cater to both monolingual and multilingual data, and they are situated within the diffusion model framework for facilitating cross-lingual translation.

In this section, we present the Cross-lingual Diffusion Language Model (XDLM), which incorporates a pre-training phase on cross-lingual data, utilizing diffusion techniques for the purpose of non-autoregressive machine translation, and a fine-tuning phase generating corresponding text from one language to another language based on the pre-trained model.

% \subsection{Preliminary}
% \subsubsection{Cross-lingual translation}
% (\irene{combine 3.1.1 and 3.1.2 as NAR machine translation, and, there is no such term called \textit{Cross-lingual translation}, all translation is cross-lingual, it should be either \textit{machine translation} or \textit{cross-lingual language model}})

% Cross-lingual translation typically involves generating an output sequence $Y=\{y_1, y_2,…, y_{|Y|}\}$ from a given input sequence $X=\{x_1,x_2,…,x_{|X|}\}$, with each sequence being in a different language. Three common generative paradigms exist for cross-lingual translation: AutoRegressive (AR) generation, Non-AutoRegressive (NAR) generation, and semi-NAR generation. Ordinarily, diffusion models employ the NAR approach for generation tasks.

% \subsubsection{Non-AutoRegressive(NAR) generation}
% The NAR generation follows the conditional probality: 
% $$
% p_{\theta}(Y|X)=\prod_{i=1}^{|Y|} p_{\theta}(y_i|X)
% $$

% Unlike AutoRegressive (AR) generation, all tokens $y_i$$(0\leq i \leq |Y|)$ in the generated sequence Y are predicted concurrently. The generation solely depends on the input sequence X, without any dependency on preceding tokens. This attribute presents a challenge in determining the length of the generated sequence. To address this issue, the prediction of the output sequence is introduced as an auxiliary task \cite{gu2017non}.

\textbf{Non-AutoRegressive (NAR) Machine Translation}
In machine translation, given the input sequence from a source language $X=\{x_1,x_2,…,x_{|X|}\}$, the task is to generate the output sequence of the translation in the target language $Y=\{y_1, y_2,…, y_{|Y|}\}$. In this work, we focus on the Non-AutoRegressive (NAR) translation setting with the diffusion model. Typically, it has the following conditional probability:  
$$
p_{\theta}(Y|X)=\prod_{i=1}^{|Y|} p_{\theta}(y_i|X).
$$

Unlike AutoRegressive (AR) text generation, all tokens $y_i$$(0\leq i \leq |Y|)$ in the generated sequence $Y$ are predicted concurrently. The generation solely depends on the input sequence $X$, without any dependency on preceding tokens. This attribute presents a challenge in determining the length of the generated sequence. To address this issue, the length prediction of the output sequence is introduced as an auxiliary task \cite{gu2017non}. And the training loss is defined as a weighted sum between the translation loss and the length prediction loss.

\textbf{Diffusion Models}
The Denoising Diffusion Probabilistic Model (DDPM) \cite{ho2020denoising} is a parametrized Markov chain, and it is trained using variational inference to generate samples that match the original input data. 
% a diffusion process for generative tasks was introduced by \cite{ho2020denoising}, yielding impressive results.
The diffusion process comprises a noise-adding forward process and a noise-removing backward process, both of which can be viewed as discrete-time Markov processes. During the forward process, the model gradually introduces random noise with different scheduled variance $\beta_1,...,\beta_t$, with the aim of generating a standard Gaussian noise $x_t$ after $t$ turns. This can be formalized as follows:
$$
q(x_{t+1}|x_t)=\mathcal{N}(x_{t+1};\sqrt{1-\beta_{t+1}}x_t,\beta_{t}\mathbf{I}).
$$

The backward process, the reverse of the forward process, attempts to reconstruct the target sequence from the standard noise. Like the forward process, this procedure is also applied incrementally and can be formalized as follows:

$$
    p(x_{t-1}|x_t)=\mathcal{N}(x_{t-1};\mu_{\theta}^{t-1},\sigma_{\theta}^{t-1}),
$$
$$
    \mu_{\theta}^{t-1}=\frac{1}{\sqrt{\alpha_{t}}}(x_t-\frac{\beta_{t}}{\sqrt{1-\overline(\alpha_{t})}}z_{\theta}(x_{t},t)), 
$$
$$
    \sigma_{\theta}^{{t-1}^2}=\frac{1-\overline{\alpha_{t-1}}}{1-\overline{\alpha_{t}}}\dot \beta_{t},
$$

where $\alpha_t=1-\beta_t, \overline{\alpha_{t}}=\prod_{i=1}^t \alpha_{i}$ and $z_\theta$ comes from the prediction of model parameterized by $\theta$. 
In this work, we apply discrete diffusion for text generating and cross-lingual translation. Based on \citet{zheng2023reparameterized}, we follow the proposed discrete diffusion model with the following routing mechanism.

$
    x_{t-1}, v_{t-1} \sim q(x_{t-1},v_{t-1}|x_t,x_0) \\
    q(v_{t-1}|x_t,x_0)=q(v_{t-1})=Bernoulli(\lambda) \\
    q(x_{t-1}|v_{t-1},x_t,x_0)= \\
    v_{t-1}x_t+(1-v^{(1)}_{t-1})q_{noise}, \quad if \quad x_t = x_0 \\
    v_{t-1}x_0+(1-v_{t-1}^{(2)})q_{noise} (x_t), \quad if \quad x_t \neq x_0 \\
$


Which models the joint distribution over both $x$ and $v$. The sampling process here also takes the reparameterized method, which improves flexibility and expressiveness compared to the original process.

% Figure environment removed
\textbf{Translation Diffusion Language Modeling (TDLM)}
% Contrary to previous language modeling objectives for diffusion models, which primarily focus on monolingual data and neglect the potential to harness cross-lingual modeling capabilities from parallel datasets, we propose a pretraining process for parallel language pairs along with a corresponding modeling objective.
Unlike previous diffusion model objectives for language modeling that primarily concentrate on monolingual data, we target to exploit cross-lingual modeling capabilities from parallel datasets. Consequently, we propose a pretraining process named Translation Diffusion Language Modeling (TDLM), aiming at enhancing cross-lingual pretraining with diffusion models. As illustrated in Figure 1, we first concatenate both source and target sentences and generate the corresponding language and position embedding sequences, and then stack them as the input to a diffusion model. 
% we select both source and target sentences, generate their corresponding language and position embedding series, and concatenate them to form the input text stream. 
In a similar vein to \citet{lin2023text}, we random mask 15\% of the tokens to the input as \cite{lample2019cross} designed, tasking the model with predicting the noise and its surrounding text based on the cross-lingual context. This denoising setting assists the model in grasping the cross-lingual context.


\section{Experimental Results}\label{sec:results}
    \subsection{General Results}
        The basic ResSAN model is used to determine reference results which our expanded model can be compared to as it is structurally similar to ResLAN but does not possess the Lidar adaptive components of it. Further, we compare with the full-size PackNet-SAN and the unmodified NLSPN architecture. 
        As it can be seen from Tab.\,\ref{tab:sota-results}, our LiDAR-adaptive ResLAN achieves competitive performance compared to state-of-the-art standard depth completion methods, which are specialized to the unfiltered 64-beam-LiDAR. The performance differences are in the range of a few centimetres in terms of MAE, which is acceptable given the practical advantage that ResLAN can generalize to different beam patterns as will be shown below.

        Furthermore, we compared the architectures for a set of three different input types that contained 64, 32 or 16 LiDAR channels using both filter types on the metrics from the KITTI benchmark. The NLSPN model was trained for the standard depth completion task and then evaluated with different input data. As for the ResSAN models, we trained one model for each input type and tested it for the corresponding one which serve serve as the \emph{Baseline} in Tab.\,\ref{tab:overall-results}. Our ResLAN model was jointly trained for all three settings. As listed in Tab.\,\ref{tab:overall-results}, the ResLAN models outperform the challenging baseline in all metrics for FOV filtering and all but one for sparse filtering. This implies that our LiDAR adaptive model is able to outperform dedicated models in case of very sparse input depth. Fig.\,\ref{fig:comp-plot} shows this is indeed the case for 32 and even more for 16 channels. For FOV-filtered inputs with 16 channels, the ResLAN exhibits approx. $10\%$ smaller MAE than the baseline. As for the NLSPN, it becomes apparent that it is not capable of generalizing to other input types since it shows clearly worse results. The difference is especially pronounced for the FOV filtering where on average more than every fourth predicted pixel is more than $25 \%$ deviating from the ground truth\,($\delta_{1.25}$). Therefore, using a weight-adapting network in combination with differently filtered input depths allows us to train models that outperform their non-adaptive counterparts.

        \begin{table}[]
            \centering
    	    \small
            \vspace{0.4cm}
            \caption{\textbf{Depth estimation result for standard depth completion} when the ResSAN model was only trained for 64 channels and the ResLAN model for multiple tasks. The PackNet-SAN and NLSPN models were trained with the setup that was also used for our model architecture.}
            \footnotesize
            \setlength{\tabcolsep}{5pt}
            \begin{tabular}{@{}lrrrrl@{}}
            \toprule
            \multicolumn{6}{c}{\textbf{Standard LiDAR Depth Completion}}                                                                                                                         \\ \midrule
            \multicolumn{1}{l|}{Method}          & RMSE $\downarrow$            & MAE  $\downarrow$            & iRMSE $\downarrow$             & iMAE $\downarrow$ & $\delta_{1.25}$ $\uparrow$ \\
            \multicolumn{1}{l|}{}                & \multicolumn{1}{l}{{[}mm{]}} & \multicolumn{1}{l}{{[}mm{]}} & \multicolumn{1}{l}{{[}1/km{]}} & {[}1/km{]}        &                            \\ \midrule
            \multicolumn{1}{l|}{PackNet-SAN}     &  914                            &  298                            &  2.78                              &  1.4                 &  99.65 \%                          \\
            \multicolumn{1}{l|}{NLSPN}           &  \textbf{889}                            &   \textbf{263}                           &  \textbf{2.62}                              &   \textbf{1.3}                &   \textbf{99.61} \%                         \\ \midrule
            \multicolumn{1}{l|}{ResSAN (Ours)}   & 948                             &  275                            &  2.75                              &    1.4               &   99.58 \%                         \\
            \multicolumn{1}{l|}{ResLAN (Ours)} &   969                           &  283                            &   2.83                             &   1.4                &  99.56 \%                          \\ \bottomrule
            \end{tabular}
            \vspace{0.2cm}
            \label{tab:sota-results}
        \end{table}

        \begin{table}[]
    	    \centering
    	    \small
    	    \caption{\textbf{Depth estimation results of the two baseline setups and the explicit and implicit ResSAN} when evaluated on a combination of 16, 32 and 64 channel depth inputs. Please note that Specialist Methods need to train three specialized networks, one for each of the three types of inputs while our method only uses one network.}
            \footnotesize
            \setlength{\tabcolsep}{4.8pt}
            \begin{tabular}{@{}lrrrrl@{}}
                \toprule
                \multicolumn{6}{c}{\textbf{Sparse Channel Filter}}                                                                                                                                  \\ \midrule
                \multicolumn{1}{l|}{Method}        & RMSE $\downarrow$            & MAE  $\downarrow$            & iRMSE $\downarrow$             & iMAE $\downarrow$ & $\delta_{1.25}$ $\uparrow$  \\
                \multicolumn{1}{l|}{}              & \multicolumn{1}{l}{{[}mm{]}} & \multicolumn{1}{l}{{[}mm{]}} & \multicolumn{1}{l}{{[}1/km{]}} & {[}1/km{]}        &                             \\ \midrule
                \multicolumn{1}{l|}{NLSPN}         &  1396                            &  437                            & 5.54                               &  2.2                 &  98.82 \%                           \\
                \multicolumn{1}{l|}{Baseline}      & \textbf{1207}                             &  381                            & 4.41                               &  1.8                 &  \textbf{99.37} \%                           \\
                \multicolumn{1}{l|}{ResLAN (Ours)} &  1215                            &  \textbf{378}                            &  \textbf{4.27}                              &  \textbf{1.7}                 &  99.31 \%                           \\ \toprule
                \multicolumn{6}{c}{\textbf{Field-of-View Filter}}                                                                                                                                   \\ \midrule
                \multicolumn{1}{l|}{Method}        & RMSE $\downarrow$            & MAE  $\downarrow$            & iRMSE $\downarrow$             & iMAE $\downarrow$ & $\delta_{1.25}$ $\uparrow$ \\
                \multicolumn{1}{l|}{}              & \multicolumn{1}{l}{{[}mm{]}} & \multicolumn{1}{l}{{[}mm{]}} & \multicolumn{1}{l}{{[}1/km{]}} & {[}1/km{]}        &                             \\ \midrule
                \multicolumn{1}{l|}{NLSPN}         &  2738                            &  1702                            & 12.3                              &  4.3                 &  74.69 \%                           \\
                \multicolumn{1}{l|}{Baseline}      &  1556                            &  525                            &  6.8                              &  3.0                 & 98.14 \%                            \\
                \multicolumn{1}{l|}{ResLAN (Ours)} &  \textbf{1548}                            &  \textbf{519}                            &  \textbf{6.44}                              &  \textbf{2.8}                 & \textbf{98.52 \%}                            \\ \bottomrule
            \end{tabular}
            \label{tab:overall-results}
        \end{table}

        
        
        % Figure environment removed
        
        % Figure environment removed

    \subsection{Filter Effects}
        Comparing the effect of the two different types of depth input filters on the model performance, it becomes apparent that FOV filtering is the more challenging task. In that setting, reducing LiDAR channels is more detrimental to the performance than sparse filtering as it creates regions where no depth information is available. Effectively, the model is forced to perform depth prediction in these regions. These effects are highlighted in the depth images in Fig.\,\ref{fig:dense-maps} where the effect of a 16-channel sparse depth filter and a 16-channel FOV can be compared.

    \subsection{Generalization Capabilities}
        We trained three models for both filter types eaach, so the combinations and number of filtered depth inputs they receive are different. This serves the purpose of testing the generalization capabilities of the ResLAN architecture as well as the robustness to different filter settings. After training, the models were evaluated for the depth input settings they were trained for, as well as for ones they weren't exposed to. Overall, ResLAN shows good generalization capabilities. As one can gather from Fig.\,\ref{fig:explicit-comp} and Fig.\,\ref{fig:implicit-comp}, the consequences of slightly varying sets of input depth settings are limited. The most considerable deviations can be seen when the model is tasked to extrapolate. For instance, the model $\{64, 32, 16\}$ shows a noticeably higher MAE for eight-channel depth inputs than the model that was trained for it. Similar behaviour can be seen for the FOV filtering case as well for the model $\{64, 48, 32\}$ when tasked to generalize for a 16-channel input. There is no such pronounced effect for generalization tasks that lie between two filter settings the model was trained for. At most, it can be observed that models that were trained for a smaller range of filter values perform slightly better than ones that have to cover a wider range. The number of filter settings used in a fixed range does not relevantly influence the model performance, as can be seen, when comparing the two models in Fig.\,\ref{fig:implicit-comp}, which are both trained for a range of 64 to 32 channels but one with three filter settings and the other one with five.
    
    % Figure environment removed
    
    
    % Figure environment removed
\section{Conclusion and Future Work}
In this work, I design corruption-robust algorithms for the Lipschitz contextual search problem. I present the \emph{agnostic checking} technique and demonstrate its effectiveness in designing corruption-robust algorithms. There are several open problems for future research. First, in the algorithm I propose for pricing loss, the schedule for agnostic checks is fixed upfront. Can the learner design an adaptive checking schedule for the pricing loss? Second, this work assumes the learner has knowledge of the Lipschitz constant $L$. Can the learner design efficient no-regret algorithms without knowledge of $L$? 


\begin{dataavailability}

The data sets supporting the findings of this study are available on Zenodo\footnote{\url{https://zenodo.org/} [last accessed \lastdate]} with the identifiers \url{https://doi.org/10.5281/zenodo.6103720} and \url{https://doi.org/10.5281/zenodo.5940101}.

\end{dataavailability}


This work was partially supported by CONICET (Argentina) (PUE 0015-2016), by the Santa Fe province (Argentina) Government under Grant PEICID-2021-170, by the Spanish Government under Grants PGC2018-096367-B-I00 and
PID2021-127685NB-I00 and by the Aragon Government under Grant DGA T45 17R/FSE.


\bibliography{references}    %

\end{document}
