
\section{Methodology} \label{sec:method}

To map \gls*{BC} sets on the search space using \gls*{DA} propagation, a novel classification algorithm is devised and herewith presented. To be successful the algorithm requires an accurate revolution period estimation. Consistency and quality criteria are defined to evaluate performances of the resulting mapping. Finally, fresh representation methods to visualize properly the large resulting amount of data are exposed.

The search space is chosen to maximize the capture ratio $ \mathcal{R}_{c} $ based on the analysis reported in \citet{luo2015analysis}. It is defined in the Mars-centered RTN reference frame at capture epoch $ t_{0} $ on December 9, 2023, at 00:45:18.363 (UTC). At that epoch, Mars's true anomaly with respect to the Sun is equal to \SI{270}{\deg}, maximizing $ \mathcal{R}_{c} $ \citep{luo2015analysis}. The selected plane is defined by inclination $ i = \SI{0.6283}{\radian} $, and right ascension of the ascending node $ \mathrm{RAAN} = \SI{0.6283}{\radian}$. That because, according to Fig. 10 in \citet{luo2015analysis}, such values maximize the capture ratio for Mars. Sought trajectories have osculating eccentricity $ e = 0.99 $ \citep{topputo2015earth}, and mean anomaly $ M = \SI{0}{\radian} $ at the initial epoch $ t_{0} $. If $ R_{\mars} $ is the radius of Mars in \si{\kilo\meter}, then the search space on the plane defined above is a circular crown centered at Mars, from radius $ R_{\mars} + \SI{100}{\kilo\meter} $ up to radius $ 5 R_{\mars} $. Hence, 
\begin{equation}
    \label{eq:search_space}
    \left(r_p, \omega\right) \in \left[R_{\mars} + \text{ \SI{100}{\kilo\meter}}, 5 \cdot R_{\mars}\right] \times \left(-\pi,\pi\right]
\end{equation}
with $r_p$ the radius of the periapsis, and $\omega$ the argument of the periapsis.

The main difference between the point-wise mapping of sets performed by \gls*{GRATIS} \citep{topputo2018trajectory} and the \gls*{DA} mapping presented in this work is that the \gls*{DA} propagator does not allow to count revolutions. It means the sub-domains cannot be classified in the sub-sets defined in \Subsec{sec:set_defs}. New sets are defined to solve this problem. Furthermore, instead of tracking revolutions geometrically, as in \citet{luo2014constructing}, the proposed novel classification algorithm counts revolution periods. Finally, a bridge between the two mapping methods is established.

\subsection{Redefinition of sub-sets and classification algorithm} \label{sec:hatted_sets}
There is no direct method to use the definitions of \Subsec{sec:set_defs} with \gls*{DA}. This is because accurately counting the revolutions of a continuous set of particles around the target cannot be performed as in \textit{Remark 1} and Eq.~(5) in \citet{luo2014constructing}. Instead of using this condition to count completed revolutions, the number of period elapsed since epoch is used. Consequently, \Dset{i} cannot be defined when using \gls*{DA} propagation. Moreover, if a subdomain reached the minimum size allowed by the \gls*{ADS} algorithm, and it tries to split again, the accuracy of the mapping of this sub-domain cannot be guaranteed. Thus, these inconsistent sub-domains need to be ruled out from the sub-sets, so that only the consistent ones are retained.

Therefore, the definitions of \Subsec{sec:set_defs} are adapted to \gls*{DA} propagation as follows:
\begin{enumerate*}[label=\roman*)]
    \item \textit{inconsistent} (sub-set \DAIset{i}) if the sub-domain performed the maximum number of splits allowed and tries to split again before completing the $i$-th period;
    \item \textit{weakly stable} (sub-set \DAWset{i}) if the sub-domain is consistent, and performs $i$ complete periods without escaping or impacting the target or its moons;
    \item \textit{unstable} (sub-set \DAXset{i}) if the sub-domain is consistent, and escapes from the target before completing the $i$-th;
    \item \textit{target--crash} (sub-set \DAKset{i}) if the sub-domain is consistent, and impacts with the target before completing the $i$-th period;    
    \item \textit{moon--crash} (sub-set \DAMset{i}) if the sub-domain is consistent, and impacts with one of the target's moons before completing the $i$-th period.
\end{enumerate*}
Conditions i), and ii)-v) apply after the particle performs $(i-1)$ periods around the target. As in \Subsec{sec:set_defs}, the sub-sets are defined for $ i \in \mathbb{Z} \textbackslash \{0\}  $ and the same considerations about propagation direction still apply. Moreover, the domain that is consistent after $i$ periods is defined as
\begin{equation}
    \hat{\Omega}_i = \Omega \textbackslash \bigcup_{j=1}^{\vert i \vert} \hat{\mathcal{I}}_{\sgn{i}j}.
	\label{eq:def_omega_set}
\end{equation}
In \Fig{fig:graphDA}, a graph reporting the relations between the sub-sets adapted to \gls*{DA} propagation is shown. As usual, a capture set is defined in \gls*{DA} propagation as $ \hat{\mathcal{C}}_{-1}^{n} := \hat{\mathcal{W}}_{n} \cap\hat{ \mathcal{X}}_{-1} $.

% Figure environment removed

\subsection{Classification algorithm} \label{sec:classification}
The classification algorithm shown in \Alg{algo:classification_algorithm} is a while loop divided in two parts:
\begin{enumerate}
    \item the \gls*{DA} propagation of the stable set \DAWset{i - 1} from the current period to the next one, thus from $ T_{i-1} $ to $ T_{i} $;
    \item the extraction of all the sub-domains to classify them in the right set.
\end{enumerate}
A bridge is built between the definitions of \Subsec{sec:set_defs} and the ones of \Subsec{sec:hatted_sets}. The similarities are
\begin{equation}
	\hat{\Omega}_i \rightarrow \Omega, \quad
	\mathcal{\hat{I}}_{i} \rightarrow \emptyset, \quad
	\mathcal{\hat{K}}_{i} \rightarrow \mathcal{K}_{i}, \quad
	\mathcal{\hat{M}}_{i} \rightarrow \mathcal{M}_{i}, \quad
	\mathcal{\hat{X}}_{i} \rightarrow \mathcal{X}_{i}, \quad
	\mathcal{\hat{W}}_{i} \rightarrow \mathcal{W}_{i}, \quad
	\mathcal{\hat{C}}_{-1}^{i} \rightarrow \mathcal{C}_{-1}^{i}, \quad
	\emptyset \rightarrow \mathcal{D}_{i}.
	\label{eq:point-wise_vs_DA}    
\end{equation}
Note that the inconsistent set \DAIset{i} is mapped to the empty set due to the inability to compute these sub-domains with accuracy. Moreover, the acrobatic set \Dset{i} is not mapped by the \gls*{DA} classification algorithm. Nonetheless, this region of the search space represents a small fraction of the total for long enough propagation times \citep{luo2014constructing}.

\begin{algorithm}[tbp]
	\SetAlgoLined
	Set either $ i = 1 $ (forward propagation) or $ i = -1 $ (backward propagation)\;
	Set the search space \DAWset{0} $ = \Omega $\;
	Set capture epoch $ T_{0} = t_{0} $\;
	Set maximum number of periods $ n_{\mathrm{max}} $\;
	\While{$ |i| \leq n_{\mathrm{max}} $}{
		Propagate \DAWset{i - \sgn{i}} from $T_{i-\sgn{i}}$ to $T_{i}$\; 
		Extract inconsistent sub-domains in \DAIset{i}\;
		Extract crash sub-domains in \DAKset{i}\;
		Extract moon-crash sub-domains in \DAMset{i}\;
		Extract escaped sub-domains in \DAXset{i}\;
		Retain remaining sub-domains as \DAWset{i}\;
		Set $ i = i + \sgn{i} $\;
	}
	\caption{Classification algorithm.}
	\label{algo:classification_algorithm}
\end{algorithm}

\subsection{Propagation time span} \label{sec:timespan}
Since it is not possible to track revolutions geometrically, they are tracked by counting the revolution periods. Thus, the need to estimate them with fidelity. These periods are determined by a least-square regression on data from point-wise computations issued by \gls*{GRATIS}\footnote{The time regressions have been derived from \glspl*{IC} propagated taking into account also the gravitational attractions of Uranus (B), and Neptune (B), later discarded due to their negligible influence.}, with respect to the radius of the periapsis $ r_{p} $. Two regression shapes are chosen:
\begin{enumerate}
    \item a square root shape
    \begin{equation}
        \label{eq:sqrt_period}
        T_{\sqrt{\cdot}} = A + B \sqrt{r_{p}}, \ \text{with } A = -0.19329, \ B = 2.10555;
    \end{equation}
    \item a logarithmic shape
    \begin{equation}
        \label{eq:ln_period}
        T_{\ln{(\cdot)}} = A + B \ln{(r_{p})}, \ \text{with } A = -7.35410, \ B = 1.44254.
    \end{equation}
\end{enumerate}
\Fig{fig:regression} shows these regressions for 2 and 6 periods, compared to the dataset generated by \gls*{GRATIS}.

% Figure environment removed

For the rest of this work, the logarithmic shape is chosen over the square root one, due to a better fit. Nevertheless, the regression does not fit well for 2 revolutions at large $r_p$, due to the high concentration of points at small radii, as shown by \Fig{fig:histograms}. Moreover, \Fig{fig:violin_plots} highlights how the distribution of the revolution times is widely spread, with respect to $r_p$, as the whiskers represent the minimum and maximum values, with the median in the middle. Therefore, the regressions displayed in \Eqs{eq:sqrt_period} and~\eqref{eq:ln_period} make up for a strong hypothesis on the behavior on the revolution periods.

% Figure environment removed

% Figure environment removed

\subsection{Consistency and quality criteria} \label{sec:criteria}
Once the classification of the search space is performed, tools need to be developed to evaluate it. Two criteria are defined to do so. The first is the \textit{consistency} criterion, which assesses the parts of the domain where the \gls*{DA} mapping cannot be trusted. The second is the \textit{quality} criterion, which assesses the performances of the \gls*{DA} mapping with respect to point-wise reference mapping carried out with \gls*{GRATIS}.

\paragraph{Consistency criterion.} \label{sec:consistency_def}
The consistency is a number assigned to each sub-domain. After the classification algorithm is performed, each sub-domain is given a consistency of either 1 if it does not belong to an inconsistent set \DAIset{i}, or 0 if it does. The main interest is to compute the global consistency on all of the search space. It can be computed by performing the mean of the consistencies, pondered by the size of each sub-domain. The global consistency will then represent the ratio of the mapping that is not inconsistent. Therefore, it is the proportion of the mapping where \gls*{ADS} guarantees the accuracy of the mapping.

\paragraph{Quality criterion.} \label{sec:quality_def}
The quality criterion is a value assigned to each set issued from \gls*{GRATIS}'s classification. It represents the proportion of the set from \gls*{GRATIS} that is well-mapped by the \gls*{DA} classification algorithm. To be computed, the quality criterion requires a sample propagated point-wise with \gls*{GRATIS}. The quality $ q_\mathcal{A} $ of a \gls*{GRATIS} set $ \mathcal{A} $ is computed as follows:
\begin{enumerate}
    \item for each point $ x \in \mathcal{A} $, if $ x $  belongs to the set $ \mathcal{\hat{A}} $, according to the bridge between \gls*{DA} and point-wise classification exposed in \Eq{eq:point-wise_vs_DA}, then the quality of that point $ q_{x} = 1 $, otherwise $ q_{x} = 0 $;
    \item the quality $q_\mathcal{A}$ of a set $\mathcal{A}$ is the mean of all the values of $\left(q_x\right)_{x\in \mathcal{A}}$. In other words
    \begin{equation}
        q_\mathcal{A} = \frac{\#\left(\mathcal{A}\cap\mathcal{\hat{A}}\right)}{\#\left(\mathcal{A}\right)}.
    \end{equation}
    In the case where $\#\left(\mathcal{A}\right)=0$, then $\#\left(\mathcal{A}\cap\mathcal{\hat{A}}\right)=0$. Thus, by convention $q_\mathcal{A} = 1$;
    \item the confidence intervals on $ q_\mathcal{A} $ are evaluated according to \citet{robert2004monte}, and \citet{hanley1983if}.
\end{enumerate}
The quality represents the probability for a point from the point-wise mapping to be mapped correctly with the \gls*{DA} mapping. Note that since \DAIset{i} is mapped to $\emptyset$, the quality criterion of $\Omega_i$ is less than or equal to the consistency criterion after $i$ revolutions. A schematic representation clarifying the meaning of the quality criterion is shown in \Fig{fig:quality_criterion_scheme}.

% Figure environment removed

\subsection{Representation methods} \label{sec:representation}
The representation of the results produced by this methodology raises two issues. Firstly, due to \gls*{ADS}, fully-split sub-domains have a maximum size of $ \approx \SI{50}{\kilo\meter} $, while the typical length of the search space is $\approx \SI{3000}{\kilo\meter} $. Thus, it is impossible to see them on a global representation. To solve this problem, the resolution of the display is downgraded to a smaller one. Furthermore, instead of plotting each sub-domain independently, the density of sub-domains per pixel is preferred. It allows detecting areas with a high number of sub-domains even on a global visualization.

Secondly, the Cartesian representation of the search space tends to shrink details for low values of the radius of the periapsis $ r_{p} $, while the size of structures located at larges values of $ r_{p} $ is amplified. However, areas with a larger amount of data to visualize are located at small $ r_{p} $. Thus, results are visualized on the $ r_{p} \times \omega $ plane, which is the search space in Keplerian coordinates.
