
\section{Introduction} \label{sec:intro}

\Gls*{BC} allows a spacecraft to approach a planet and enter a temporary orbit about it without requiring maneuvers in between. As part of the low-energy transfers, it is a valuable alternative to Keplerian approaches. Exploiting \gls*{BC} grants several benefits in terms of both cost reduction \citep{belbruno1993sun} and mission versatility \citep{belbruno2000calculation,topputo2015earth}, in general at the cost of longer transfer times \citep{circi2001dynamics,ivashkin2002trajectories}. In the past, the \gls*{BC} mechanism was used to rescue Hiten \citep{belbruno1990ballistic}, and to design insertion trajectories in lunar missions like SMART-1 \citep{racca2002smart} and GRAIL \citep{chung2010trans}. In the near future, BepiColombo will exploit \gls*{BC} orbits to be weakly captured by Mercury \citep{benkhoff2021bepicolombo,schuster2014influence}. \gls*{BC} is an event occurring in extremely rare occasions and requires acquiring a proper state (position and velocity) far away from the target planet \citep{topputo2015earth}. In fact, massive numerical simulations are required to find the specific conditions that support capture \citep{topputo2009computation} and only approximately 1 out of \num{10000} states lead to capture \citep{luo2015analysis}. In a first effort to reduce the computational burden, the variational theory for Lagrangian coherent structures \citep{haller2011variational} was recently applied to find \gls*{BC} opportunities more efficiently \citep{manzi2021flow}.

\Gls*{DA} propagation is a worthy candidate to reduce the computational burden for the search of \gls*{BC} trajectories. It consists of propagating \glspl*{IC}, not as a single point but as an interval around an \gls*{IC}. Thanks to Taylor expansions of the flow, the state of any point in the represented interval can be determined through convenient polynomial evaluations \citep{berz1992high,berz1999modern}. This gain in efficiency comes at the cost of a loss of accuracy due to the finite Taylor expansion. The \gls*{ADS} algorithm offers higher accuracy for less computational burden compared to the increase of the order of the \gls*{DA} mappings \citep{wittig2015propagation}. It splits the initial domain into smaller sub-domains to reduce the approximation error when it grows above a given tolerance.

\gls*{DA} propagation is increasingly used in astrodynamics. Indeed, it provides high performances in uncertainty propagation in the two-body dynamics \citep{valli2013nonlinear}, even for highly nonlinear dynamics with large uncertainties when exploiting \gls*{ADS}, as in the case of Apophis \citep{wittig2015propagation}. Other applications occur in orbital mechanics, such as propagation of probability density functions \citep{wittig2017longterm}, orbit determination \citep{pirovano2021differential}, and generation and study of orbit families in the \gls*{CR3BP} \citep{dilizia2008application,baresi2021highorder}.

\gls*{DA} propagation allows avoiding intensive grid sampling implied by point-wise research of conventional algorithms for designing \gls*{BC} trajectories \citep{hyeraci2010method,luo2014constructing}, as it offers a continuous description of the whole search-space. In addition, the computation of a Taylor expansion provides information such as the partial derivatives of the flow up to an arbitrary order \citep{wittig2017longterm}. Furthermore, the polynomial maps can be manipulated to impose constraints on the flown trajectories \citep{berz1999modern, dilizia2008application}.

This work proposes a methodology to find \gls*{BC} trajectories using \gls*{DA} mapping. Hence, the adaptation of the definition of \gls*{BC} for \gls*{DA}, followed by the definition of criteria to assess the performances of \gls*{DA}-based mapping of \gls*{BC} compared to point-wise mapping. The goal of the paper is to show how to build cartography of \gls*{BC} sets using \gls*{DA} mapping. It proposes an alternative classification algorithm to sort the sub-domains produced by the \gls*{ADS} algorithm in newly defined capture sets. The remainder of the paper is organized as follows. In \Sec{sec:background}, the dynamical model employed is introduced, as well as the \gls*{WSB} concept, the \gls*{BC} mechanism, and the \gls*{DA} propagation. Then, the description of the characterization process, and the mapping-assessment methodology follow in \Sec{sec:method}. Results are presented and discussed in \Sec{sec:results}. Eventually, conclusions are drawn in \Sec{sec:conclusion} together with the presentation of future work.

