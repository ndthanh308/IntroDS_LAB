
\section{Results} \label{sec:results}

In this work, the results from two simulations with initial grids having different resolution are presented. The first simulation relies on a coarse grid providing a low-resolution mapping where the search space is initially divided into $ 32 \times 32 $ domains and the maximum number of splits allowed to the \gls*{ADS} algorithm is 9. On the contrary, the second is computed on a finer grid returning a high-resolution mapping. In the latter, the search space is divided into $ 128 \times 128 $ domains and 10 maximum splits are allowed. Results are visualized in the low-resolution mapping in Cartesian coordinates. Sub-domains densities and last step epochs are computed on the whole search space. Then, the \gls*{DA} classification outcome is rendered for qualitative analysis. Finally, a quantitative analysis of those mappings exploiting consistency and quality criteria follows.

\subsection{Mapping} \label{sec:mapping}
\Fig{fig:cartesian_mapping} represents the low-resolution search space in Cartesian coordinates, before and after the propagation with \gls*{ADS} (left and right, respectively). Before propagation, the search space is divided regularly into small domains for parallelization (see \Fig{fig:car_map_left}). Conversely, an irregular sub-divisions highlighting dynamical changes is shown in \Fig{fig:car_map_right}. This visualization shows the necessity to use different visualization methods. Indeed, it is hard to analyze close to zones where \gls*{ADS} creates a large number of domains, although these are the most interesting regions.

% Figure environment removed

\subsubsection{Density of sub-domains and last step epochs} \label{sec:subdomain}
\Fig{fig:density_mapping} shows the density of sub-domains for the low-resolution and high-resolution mappings (left and right, respectively). They mostly differ because the high-resolution mapping provides more variations. Indeed, the density is either maximal (in grey) or minimal (in white) in the low-resolution one. On the contrary, the high-resolution mapping provides more shades, highlighting several degrees of non-linearity accurately mapped. 

Moreover, epochs of last steps carried out by the propagation scheme are shown in \Fig{fig:last_step_mapping}, in percent of the overall propagation time span. The integration of a sub-domain stops either after a collision occurs or when the sub-domain itself is declared inconsistent. Note that at low $ r_{p} $, the lightest zones on the low-resolution mapping (\Fig{fig:last_step_mapping_left}) become completely white when visualized in high resolution (\Fig{fig:last_step_mapping_right}). Therefore, propagation of these areas is now completed and sub-domains becomes consistent on a finer grid. In these regions, the consistency criterion is expected to rise when the mapping resolution increases.

% Figure environment removed

% Figure environment removed

\subsubsection{Classification} \label{sec:classification_results}
\Fig{fig:classification_2_revolutions} shows the results of the classification algorithm after two revolutions for the low-resolution mapping (left), the high-resolution mapping (middle), and from point-wise propagation using \gls*{GRATIS} (right). The latter used as a reference and derived through point-wise propagation from a sample of $ 10^{5} $ points. Each color represents a different set, following the bridge between sets from \gls*{DA} to point-wise mapping established in \Sec{sec:classification}. Thus, a set \WSBset{A}{}{} from \gls*{GRATIS} classification is colored the same way as \WSBset{\hat{A}}{}{} from \gls*{DA} classification. Since inconsistent sets \DAIset{i} and acrobatic sets \Dset{i} have not corresponding sets in point-wise and DA mappings, respectively, they share the same colors.

% Figure environment removed

While these three mappings have a similar macroscopic look, they differ in many ways. First of all, the sizes of the inconsistent sets decrease dramatically from the low-resolution mapping to the high-resolution one. Moreover, these gains in consistency mainly benefit the representation of the stable set \Wset{2}. Furthermore, the escape sets \DAXset{1} and \DAXset{2} are poorly mapped compared to the results of \gls*{GRATIS} on \Xset{1} and \Xset{2}. These issues in the mapping are mostly due to errors in the approximation of the revolution period.

\Fig{fig:classification_6_revolutions} presents the results of the classification algorithm after six revolutions for the low-resolution mapping (left), the high-resolution mapping (middle), and from point-wise propagation using \gls*{GRATIS} (right). Also in this case the latter mapping is used as reference and it is derived through point-wise propagation from a sample of $ 10^{5} $ points as before. These three charts show simplified mappings. For the sake of clarity, all sets $ \left( \mathcal{A}_{j} \right)_{j \in \llbracket 1,6\rrbracket}$ are united under the name $ \mathcal{A} $, defined as 
\begin{equation}
	\mathcal{A} = \bigcup_{j=1}^{6} {\mathcal{A}_{j}}.
	\label{eq:siplified_mapping_definition}
\end{equation}
This representation method is adopted for all sets but \Wset{6}, and \DAWset{6}.

% Figure environment removed

The same observations can be done for Figs.~\ref{fig:classification_2_revolutions} and~\ref{fig:classification_6_revolutions}. However, \Fig{fig:classification_6_revolutions} shows a strong predominance of the inconsistent set, especially at low resolution. The high-resolution mapping improves these results but still struggles to handle long accurate propagations on such a large search space. Nevertheless, these results show that \gls*{DA} mapping with \gls*{ADS} can highlight changes of behavior all over the search space, at an equivalent or lower computational cost compared to point-wise mapping.

\subsection{Quantitative performance analysis} \label{sec:results_criteria}
After performing a qualitative analysis on the resulting \gls*{DA} mappings, the two criteria developed deliver quantitative data on the computed mappings.

\subsubsection{Consistency criterion} \label{sec:consistency_results}
In \Fig{fig:consistency}, the global consistency rate for both the low- and high-resolution mappings after several revolution periods are shown in bright and dark colors, respectively. The consistency at zero revolution is 100\% since mappings only represent \glspl*{IC}. It appears that the consistency remains high, even after six revolutions, since it stays above 87\%. It means that the size of the inconsistent sets is 13\% or less on the total search space. Moreover, the consistency of the high-resolution mapping is around 5 points above the one of the low-resolution one. It demonstrates the gain of accuracy delivered by these additional computations.

% Figure environment removed


\subsubsection{Quality criterion} \label{sec:quality_results}
\Fig{fig:quality} represents the quality criterion for both low- and high-resolution mappings on the left and right, respectively. Each group of bars corresponds to a different collection of sets. For instance, \Xset{i} for $ i \in \llbracket1,6\rrbracket $. Differently, each color represents a different number of periods, from 1 (light) to 6 (dark). A bar's height represents the quality of that set in percent, while whiskers inform on the confidence interval at 95\% on such value. These estimations of the quality criterion are computed against the sample of size $ 10^{5}$ obtained with \gls*{GRATIS}.

The quality of acrobatic sets \Dset{i} is not shown since they are not mapped by the \gls*{DA} algorithm. In fact, their qualities are automatically set to 0\%. In addition, the quality of moon-crash sets \Mset{i} is always 100\%. Indeed, among the $ 10^{5} $ propagated trajectories, none of them crashed on Mars' moons. The quality of these empty sets is then $100\%$ by convention.

% Figure environment removed

The quality criterion is always higher for the high-resolution mapping than for the low-resolution one. Especially for stable sets \Wset{i} and capture sets \Cset{i} since a larger part of these sets becomes inconsistent during the low-resolution propagation, as shown by \Fig{fig:classification_6_revolutions}. Furthermore, capture sets \Cset{i} are mapped in a better or equal way compared to stable sets \Wset{i}. This is due to the positioning of \Cset{i} farther from Mars \citep{merisio2021characterization}, than most of \Wset{i}. Thus, as highlighted in Figs.~\ref{fig:classification_2_revolutions} and~\ref{fig:classification_6_revolutions}, the consistency of this region is higher, which enables better quality.

Except for those two groups of sets, only slight improvements are remarkable when switching from low resolution to high resolution. Indeed, apart from \Xset{1}, escape sets \Xset{i} and crash sets \Kset{i} fail to be accurately mapped, even though these regions of the search space are consistent. Indeed, the proposed classification algorithm does not track revolutions geometrically but uses period approximations. Hence, major approximations in the classification.

Nevertheless, the poor mapping of these sets hardly impacts the quality of sets \Wset{i} due to their small sizes. Therefore, the global quality never drops below 80\% in low resolution. The overall quality can be improved at a high computational cost, compared to the point-wise mapping, thanks to high-resolution mapping. Such a mapping improves the global classification results by at least 4. These poor performances are mainly due to the inability of the \gls*{DA} mapping engine to track revolutions geometrically instead of temporally. Thus, \gls*{DA} mapping as presented in this work cannot outperform point-wise mapping in terms of classification precision, but it exceeds point-wise mapping concerning computational cost.

