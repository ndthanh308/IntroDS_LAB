
\section{Conclusion} \label{sec:conclusion}

This article presented a methodology to build cartography of \acrlong*{BC} sets using \acrlong*{DA} mapping. It proposes an alternative classification algorithm to sort the sub-domains produced by the \acrlong*{ADS} algorithm in newly defined capture sets. Moreover, instead of tracking revolutions with geometrical methods, as in \citet{luo2014constructing}, they are counted temporally, using regression on point-wise data. This work establishes a bridge between the point-wise mapping and \gls*{DA} mapping, allowing us to assess the performances of this new method. Hence, the introduction of two criteria. The first one is the consistency criterion, which represents the proportion of the search space where the mapping accuracy is guaranteed by the \gls*{ADS} algorithm. The second one is the quality criterion, which represents the success rate of the \gls*{DA} classification algorithm compared to the point-wise one.

Results show that \gls*{DA} mapping of \gls*{BC} sets could be performed on large search spaces, as \gls*{ADS} captured the dynamical variations on the whole domain. Furthermore, the consistency criterion shows that more than 87\% of the search space is guaranteed by the \gls*{ADS} algorithm as accurate, even for low-resolution mappings. Moreover, the quality criterion demonstrates that the global error rate of \gls*{DA} mapping is below 20\%. However, some small sets are either not or poorly mapped using revolution period regression, mainly due to the incapacity to count revolutions via geometrical information. This phenomenon does not disappear when the resolution of the mapping is increased.

Therefore, while \gls*{DA} mapping allows performing good macroscopic cartography of the search space, it does not allow to replace point-wise mapping with \gls*{DA} mapping. However, \gls*{DA} mapping has the advantage of being a continuous mapping instead of a point-wise one. It means that the behavior of any point in the search space is defined, provided that it is consistent. On the contrary, a point-wise mapping delivers precise knowledge on the discrete set of propagated \glspl*{IC} only. As a result, the behavior of the continuous space between two points is unknown and can be challenging to interpolate due to the nonlinear dynamics. On that account, \gls*{DA} mapping of \gls*{BC} sets can prove useful in various situations. For instance, it can be used for fast mapping of a plane, to then target a restricted domain of the search space. Either with a dense point-wise mapping or with a high-resolution \gls*{DA} mapping of the restricted area of interest.
