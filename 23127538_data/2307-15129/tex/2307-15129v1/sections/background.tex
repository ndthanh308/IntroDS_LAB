
\section{Background} \label{sec:background}

Details about dynamical model, \gls*{WSB}, \gls*{BC} phenomenon, \gls*{DA} propagation, and \gls*{ADS} algorithms are herewith presented.

\subsection{Dynamical model}

According to the nomenclature introduced in \citet{luo2014constructing}, a \textit{target} (also referred to as \textit{central body}) and a \textit{primary} are defined. The target being the body around which the motion of the spacecraft is studied (Mars in this work), and the primary being the body around which the target revolves (the Sun). Target and primary masses are $m_{t}$ and $m_{p}$, respectively.

\subsubsection{Reference frames} \label{sec:reference_frame}
In this work, the following reference frames are used: J2000, and \acrshort*{RTN}.

\paragraph{J2000.}
Defined on the Earth's mean equator and equinox, the J2000 is an inertial frame determined from observations of planetary motions which was realized to coincide almost exactly with the \gls*{ICRF} \citep{archinal2011report}. \Gls*{EOM} are integrated in this reference frame.

\paragraph{RTN@$t_{i}$.}
The \gls*{RTN} is an inertial frame frozen at a prescribed epoch $t_{i}$. The frame is centered at the target. The $x$-axis is aligned with the primary--secondary direction, the $z$-axis is normal to the primary--secondary plane in the direction of their angular momentum, and the $y$-axis completes the dextral orthonormal triad. \glspl*{IC} are defined in this frame \citep{luo2015analysis}.

\subsubsection{Ephemerides}
Precise states of the Sun and the major planets are retrieved from the \gls*{JPL}'s planetary ephemerides \textfnc{de440s.bsp}\footnote{Data publicly available at: \url{https://naif.jpl.nasa.gov/pub/naif/generic_kernels/spk/planets/de440s.bsp} [retrieved \lastdate].} (or DE440s) \citep{park2021jpl}. Additionally, the ephemerides \textfnc{mars097.bsp} of Mars (the target) and its moons are employed\footnote{\url{~/spk/satellites/mars097.bsp} [retrieved \lastdate].}. The following generic \gls*{LSK} and \gls*{PCK} are used: \textfnc{naif0012.tls}, \textfnc{pck00010.tpc}, and \textfnc{gm\_de431.tpc}\footnote{Data publicly available at: \url{https://naif.jpl.nasa.gov/pub/naif/generic_kernels/lsk/naif0012.tls}, \url{~/generic_kernels/pck/pck00010.tpc}, and \url{~/generic_kernels/pck/gm_de431.tpc} [retrieved \lastdate]. The \textfnc{gm\_de431.tpc} \gls*{PCK} kernel is used because the new version consistent with the ephemerides DE440s has not been released yet.}.

\subsubsection{Equations of motion}
The \gls*{EOM} used are those of the restricted $N$-body problem. The gravitational attractions of the Sun, Mercury, Venus, Earth (B\footnote{Here B stands for barycenter.}), Mars (central body), Jupiter (B), Saturn (B), Phobos, and Deimos are considered. Additionally, \gls*{SRP} is also included and implemented as a \emph{cannonball} or \emph{spherical} model \citep{scheeres2011dynamics}. The assumed spacecraft specifications needed to evaluate the \gls*{SRP} perturbation are collected in \Tab{tab:sc-specs}. They are compatible with the specifications of a 12U deep-space CubeSat \citep{topputo2021envelop}.

\begin{table}[tbp]
	\centering
	\caption{Assumed spacecraft specifications.}
	\begin{tabular}{lll}
		\toprule
		\textbf{Specification} & \textbf{Symbol} & \textbf{Value} \\
		\midrule
		Mass & $ m $ & \SI{24}{\kilo\gram} \\
		\gls*{SRP} area & $ A $ & \SI{0.32}{\meter\squared} \\
		Coefficient of reflectivity & $ C_{r} $ & \SI{1.3}{} \\
		\bottomrule
	\end{tabular}
	\label{tab:sc-specs}
\end{table}  

The \gls*{EOM}, written in a non-rotating Mars-centered reference frame are \citep{luo2014constructing,merisio2021characterization}
\begin{equation}
	\ddot{\mathbf{r}} = - \frac{\mu_{t}}{r^{3}} \mathbf{r} 
	- \sum\limits_{i \in \mathbb{P}} \mu_{i} \left( \frac{\mathbf{r}_{i}}{r_{i}^{3}} + \frac{\mathbf{r} - \mathbf{r}_{i}}{\left\lVert \mathbf{r} - \mathbf{r}_{i} \right\rVert^{3}} \right)
	+ \frac{Q A}{m} \frac{\mathbf{r} - \mathbf{r}_{\Sun}}{\left\lVert \mathbf{r} - \mathbf{r}_{\Sun} \right\rVert^{3}}
	\label{eq:eom}
\end{equation}
where $ \mu_{t} $ is the gravitational parameter of the target body; $ \mathbf{r} $ is the position vector of the spacecraft with respect to the target and $ r $ is its magnitude; $ \mathbb{P} $ is a set of $ N-2 $ indexes each referring to a perturbing body; $ \mu_{i} $ and $ \mathbf{r}_{i} $ are the gravitational parameter and position vector with respect to the target of the $i$-th body, respectively; $ A $ is the Sun-projected area on the spacecraft for \gls*{SRP} evaluation; $ m $ is the spacecraft mass; $ \mathbf{r}_{\Sun} $ is the position vector of the Sun with respect to the target. Lastly, $ Q $ is equal to
\begin{equation}
	Q = \frac{L C_{r}}{4 \pi c} 
	\label{eq:solarq}
\end{equation}
where $ C_{r} $ is the spacecraft coefficient of reflectivity, $ c = \SI{299792458}{\meter\per\second} $ taken from SPICE \citep{acton1996ancillary,acton2018look} is the speed of light in vacuum, and $ L = S_{\Sun} 4 \pi d_{\mathrm{AU}}^{2} $ is the luminosity of the Sun. The latter is computed from the solar constant\footnote{\url{https://extapps.ksc.nasa.gov/Reliability/Documents/Preferred_Practices/2301.pdf} [last accessed \lastdate].} $ S_{\Sun} = \SI{1367.5}{\watt\per\meter\squared} $  evaluated at $ d_{\mathrm{AU}} = \SI{149597870613.6889}{\meter} $ corresponding to \SI{1}{\AU} according to SPICE \citep{acton1996ancillary,acton2018look}.

\subsubsection{Numerical integration of \gls*{EOM}}
The \gls*{EOM} in \Eq{eq:eom} are integrated in their nondimensional form to avoid ill-conditioning \citep{luo2014constructing}. Nondimensionalization units are reported in \Tab{tab:units}. For point-wise simulations, the numerical integration is carried with the \gls*{DOPRI8} propagation scheme \citep{montenbruck2000satellite}. It is an adaptive step, 8th-order \gls*{RK} integrator with 7th-order error control, the coefficients were derived by Prince and Dormand \citep{prince1981high}. As for the \gls*{DA} simulations, the propagation scheme is \gls*{DOP853} \citep{hairer1993solving}, which is of the same 8th-order \gls*{RK} integrator family. However, the error control is performed with a 3rd-order and a 5th-order estimation. The dynamics are propagated setting the relative tolerance to $ 10^{-12} $ \citep{luo2014constructing}. 

\begin{table}[tbp]
	\centering
	\caption{Nondimensionalization units.}
	\begin{tabular}{llll}
		\toprule
		\textbf{Unit} & \textbf{Symbol} & \textbf{Value} & \textbf{Comment} \\
		\midrule
		Gravity parameter & $ \mathrm{MU} $ & \SI{42828.376}{\kilo\meter\cubed\per\second\squared} & Mars' gravity parameter $ \mu_{t} $ \\
		Length & $ \mathrm{LU} $ & \SI{3396.0000}{\kilo\meter} & Mars' radius $ R_{\mars} $ \\ 
		Time & $ \mathrm{TU} $ & \SI{956.28142}{\second} & $ (\mathrm{LU}^3 / \mathrm{MU})^{0.5} $ \\
		Velocity & $ \mathrm{VU} $ & \SI{3.5512558}{\kilo\meter\per\second} & $ \mathrm{LU} / \mathrm{TU} $ \\
		\bottomrule
	\end{tabular}
	\label{tab:units}
\end{table}


\subsection{Weak stability boundary and ballistic capture mechanism} \label{sec:wsb}

Over the years, the \gls*{WSB} was defined in many different ways. It was initially identified as a fuzzy boundary region placed at approximately \SI{1.5e6}{\kilo\meter} from the Earth in the Sun--Earth direction \citep{belbruno1987lunar,belbruno1990ballistic}. An algorithmic definition followed in \citet{belbruno2004capture}, later extended in \citet{garcia2007note}, \citet{topputo2009computation}, and \citet{silva2012applicability}. Then, the \gls*{WSB} was interpreted as the intersection of three sub-sets of the phase space \citep{topputo2008resonant,belbruno2008resonance}. The \gls*{WSB} concept being closely connected to \gls*{BC} \citep{belbruno2004capture}, a formal definition and a methodology for its derivation from weakly stable and unstable sets were finally proposed in \citet{hyeraci2010method}. To date, despite the effort put in numerous works \citep{garcia2007note,topputo2008resonant,belbruno2008resonance,belbruno2010weak,belbruno2013geometry}, both \gls*{WSB} and \gls*{BC} are still not completely understood. Nonetheless, a connection between celestial and quantum mechanics was recently found exploiting the \gls*{WSB} \citep{belbruno2020relation}, providing a fresh perspective to tackle the problem.

\Gls*{BC} orbits are characterized by \glspl*{IC} escaping the target when integrated backward and performing $n$ revolutions about it when propagated forward, neither impacting or escaping the target. In forward time, particles flying on \gls*{BC} orbits approach the target coming from outside its sphere of influence and remain temporarily captured about it. After a certain time, the particle escapes if an energy dissipation mechanism does not take place to make the capture permanent. To dissipate energy either a breaking maneuver or the target's atmosphere (if available) could be used \citep{luo2021mars}. In this work \gls*{BC} sets for comparison are derived propagating the \gls*{EOM} in \Eq{eq:eom} and following the procedure in \citet{luo2014constructing}.

When searching for \gls*{BC} opportunities, most of the trajectories found are spurious solutions which are typically not useful for mission design purposes \citep{luo2014constructing}. Useful solutions are detected exploiting the regularity index\footnote{In previous works this was referred to as stability index \citep{luo2014constructing,luo2015analysis,luo2017capability}. However, in \citet{deitos2018survey}, the adjustment from \emph{stability} to \emph{regularity} index was proposed to avoid misunderstandings with the periodic orbit stability index. The same nomenclature introduced in \citet{deitos2018survey} is used in this work.} $ S $ and regularity coefficient $ \Delta S_{\%} $ \citep{deitos2018survey}. The aim is seeking for ideal orbits that presents regular post-capture legs resulting in $n$ revolutions about the target which are similar in orientation and shape. Numerical experiments showed that high-quality post-capture orbits are identified by small regularity index and coefficient \citep{deitos2018survey,luo2017capability,luo2015analysis,luo2014constructing}. If the regularity index and coefficient are indicators used to qualitatively judge post-capture legs, capture occurrence is quantitatively measured through the capture ratio $ \mathcal{R}_{\mathcal{C}} $ \citep{luo2015analysis}. Typically, search spaces characterized by larger capture ratio are desirable when looking for \gls*{BC} orbits.

\subsubsection{Definitions of particle stability and sub-sets} \label{sec:set_defs}
A particle stability is inferred using a plane in the three-dimensional physical space \citep{belbruno1993sun}, according to the spatial stability definition provided in \citet{luo2014constructing}. The following indications are used to classify stability, see \citet{luo2014constructing} for more details:
\begin{enumerate*}[label=\arabic*)]
	\item a particle completes a revolution around the target according to \textit{Remark 1} and Eq.~(5) in \citet{luo2014constructing};
	\item a particle escapes from the target according to \textit{Remark 2} and Eq.~(6) in \citet{luo2014constructing};
	\item a particle impacts with the target according to \textit{Remark 3} and Eq.~(7) in \citet{luo2014constructing}.
\end{enumerate*}
Consistent variants of Eq.~(7) in \citet{luo2014constructing} can be derived to locate impacts with target's moons, if present.

Based on its dynamical behavior, a propagated trajectory is said to be:
\begin{enumerate*}[label=\roman*)]
	\item \textit{weakly stable} (sub-set \Wset{i}) if the particle performs $i$ complete revolutions around the target without escaping or impacting with it or its moons;
	\item \textit{unstable} (sub-set \Xset{i}) if the particle escapes from the target before completing the $i$-th revolution;
	\item \textit{target--crash} (sub-set \Kset{i}) if the particle impacts with the target before completing the $i$-th revolution;
	\item \textit{moon--crash} (sub-set \Mset{i}) if the particle impacts with one of the target's moons before completing the $i$-th revolution;
	\item \textit{acrobatic} (sub-set \Dset{i}) if none of the previous conditions occurs within the integration time span.
\end{enumerate*}
Conditions ii)-v) apply after the particle performs $(i-1)$ revolutions around the target. The sub-sets are defined for $ i \in \mathbb{Z} \textbackslash \{0\} $, where the sign of $i$ informs on the propagation direction. When $ i > 0 $ ($ i < 0 $) the \gls*{IC} is propagated forward (backward) in time. The overall domain, union of all sub-sets, is defined as $\Omega$. A graph clarifying the relations between sub-sets is shown in \Fig{fig:graph}. A capture set is defined as $ \mathcal{C}_{-1}^{n} := \mathcal{W}_{n} \cap \mathcal{X}_{-1} $. Therefore, it is the intersection between the stable set in forward time \Wset{n} and the unstable set in backward time \Xset{-1} \citep{luo2014constructing}.

% Figure environment removed


\subsection{Differential algebra} \label{sec:da}

\gls*{DA} propagation consists of assimilating a function $f$ of $v$ variables, contained in $C^{k+1}$, with $T_f^{(k)}$ the Taylor expansion of $f$ at order $k$ \citep{berz1999modern}. The computation of such polynomials can be performed efficiently, and provides a representation of the function $f$ on all of its domain. Moreover, the \gls*{DA} structure ensures that algebraic and functional operations are well-defined, particularly for the numerical solving of ordinary differential equations \citep{berz1992high}.

The main advantage of this method is that the polynomial map only needs to be computed once, and then it is evaluated in an arbitrarily large number of points. In other words, to perform a Monte-Carlo estimation with a sample of size $S$, only one computation of the map is needed, followed by $S$ polynomial evaluations, while classic Monte-Carlo requires $S$ propagations, see \citet{armellin2010asteroid}. The \gls*{DA} engine used in this work is the \gls*{DACE}\footnote{Library available at: \url{https://github.com/dacelib/dace} [last accessed \lastdate].}, implemented by Politecnico di Milano \citep{rasotto2016differential,massari2018differential}.

\subsubsection{Automatic domain splitting} \label{sec:ads}

DA propagation allows reducing the approximation error by increasing the order $k$ of the polynomial mapping. Nevertheless, increasing the order leads to an important growth of the computational time. Thus, the introduction of \gls*{ADS}, see \citet{wittig2015propagation}, and \citet{pirovano2021differential} for more details.

The farther a point is from the constant part of the Taylor expansion, the higher the loss of accuracy. The \gls*{ADS} algorithm controls this error by dividing the domain into halves until the accuracy is below a predetermined tolerance. Therefore, each new sub-domain is represented by its polynomial map causing a much smaller approximation error, at a controlled computational cost \citep{wittig2015propagation}. The \gls*{ADS} creates a division of the initial domain separating areas with different behaviors from one another. The \gls*{ADS} algorithm requires the passing of two parameters, in addition to the order of the \gls*{DA}: the tolerance, and the maximum number of splits allowed. The former being the approximation error threshold before the splitting of a domain occurs, while the latter being the number of times the \gls*{ADS} routine can be applied to a sub-domain.

