
\section{Introduction} \label{sec:intro}

\Gls*{BC} allows a spacecraft to approach a planet and enter a temporary orbit about it without requiring maneuvers in between. As part of the low-energy transfers, it is a valuable alternative to Keplerian approaches. Exploiting \gls*{BC} grants several benefits in terms of both cost reduction \citep{belbruno1993sun} and mission versatility \citep{belbruno2000calculation,topputo2015earth}, in general at the cost of longer transfer times \citep{circi2001dynamics,ivashkin2002trajectories}. In the past, the \gls*{BC} mechanism was used to rescue Hiten \citep{belbruno1990ballistic}, and to design insertion trajectories in lunar missions like SMART-1 \citep{racca2002smart} and GRAIL \citep{chung2010trans}. In the near future, BepiColombo will exploit \gls*{BC} orbits to be weakly captured by Mercury \citep{benkhoff2021bepicolombo,schuster2014influence}. \gls*{BC} is an event occurring in extremely rare occasions and requires acquiring a proper state (position and velocity) far away from the target planet \citep{topputo2015earth}. In fact, massive numerical simulations are required to find the specific conditions that support capture \citep{topputo2009computation} and only approximately 1 out of \num{10000} states lead to capture \citep{luo2015analysis}. In a first effort to reduce the computational burden, the variational theory for Lagrangian coherent structures \citep{haller2011variational} was recently applied to find \gls*{BC} opportunities more efficiently \citep{manzi2021flow}.

\Gls*{DA} propagation is a worthy candidate to reduce the computational burden for the search of \gls*{BC} trajectories. It consists of propagating \glspl*{IC}, not as a single point but as an interval around an \gls*{IC}. Thanks to the Taylor expansions of the flow, the state of any point in the represented interval can be determined through convenient polynomial evaluations \citep{berz1992high,berz1999modern}. This gain in efficiency comes at the cost of a loss of accuracy due to the finite Taylor expansion. The \gls*{ADS} algorithm allows to represent large domains accurately when increasing the order of the polynomial expansions fails to do so \citep{wittig2015propagation}.  Indeed, on poorly-defined domains, raising the order will increase the approximation error on the edges while the error will decrease on already well-defined areas. \Gls*{ADS} splits the initial domain into smaller sub-domains to reduce the approximation error when it grows above a given tolerance.

\gls*{DA} propagation is increasingly used in astrodynamics. Indeed, it provides high performances in uncertainty propagation in the two-body dynamics \citep{valli2013nonlinear}, even for highly nonlinear dynamics with large uncertainties when exploiting \gls*{ADS}, as in the case of Apophis \citep{wittig2015propagation} or Apollo LM-10 also known as Snoopy \citep{caleb2021can}. Other applications occur in orbital mechanics, such as propagation of probability density functions \citep{wittig2017longterm}, maximum a posteriori estimation \citep{servadio2022maximum}, orbit determination \citep{pirovano2021differential,servadio2021differential}, and generation and study of orbit families in the \gls*{CR3BP} \citep{dilizia2008application,baresi2021highorder}.

\gls*{DA} propagation allows avoiding intensive grid sampling implied by point-wise research of conventional algorithms for designing \gls*{BC} trajectories \citep{hyeraci2010method,luo2014constructing}, as it offers a continuous description of the whole search-space. In addition, the computation of a Taylor expansion provides information such as the partial derivatives of the flow up to an arbitrary order \citep{wittig2017longterm}. Furthermore, the polynomial maps can be manipulated to impose constraints on the flown trajectories \citep{berz1999modern, dilizia2008application}.

The goal of this work is to use \gls*{DA} mapping to carry out macroscopic analyses of the phase space about Mars to find \gls*{BC} trajectories. Hence, the adaptation of the definition of \gls*{BC} for \gls*{DA}, followed by the definition of two criteria, named consistency and quality, to assess the performances of \gls*{DA}-based mapping of \gls*{BC} compared to point-wise mapping. Cartography of large \gls*{BC} sets about Mars are computed using \gls*{DA} mapping. Mars is chosen without loss of generality due to its relevance in the long-term exploration. The work proposes an alternative classification algorithm to sort the sub-domains produced by the \gls*{ADS} algorithm in newly defined capture sets. While \gls*{DA} mapping allows performing good macroscopic cartography of the search space, it does not allow to fully replace point-wise mapping. However, \gls*{DA} mapping has the advantage of being continuous, so meaning that the behavior of any point in the search space is defined. On the contrary, a point-wise mapping delivers precise knowledge on the discrete set of propagated \glspl*{IC} only. As a consequence, the behavior of the continuous space between two points is unknown and can be challenging to interpolate due to the nonlinear dynamics. 

The remainder of the paper is organized as follows. In \Sec{sec:background}, the dynamical model employed is introduced, as well as the \gls*{WSB} concept, the \gls*{BC} mechanism, and the \gls*{DA} propagation. Then, the description of the characterization process, and the mapping-assessment methodology follow in \Sec{sec:method}. Results are presented and discussed in \Sec{sec:results}. Eventually, conclusions are drawn in \Sec{sec:conclusion} together with the presentation of future work.







