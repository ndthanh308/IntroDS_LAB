\section{Derivation of a one vertex H-triangulation for $(S^3,7_3)$ and an ideal tetrahedral decomposition of the complementary space $S^3\backslash 7_3$}\label{triangulation}
In this section, we obtain a one vertex H-triangulation for a pair of $S^3$ and its contained knot $7_3$ and an ideal tetrahedral decomposition for the complementary space $S^3\backslash 7_3$ according to the methods described in \cite{MR718149},\cite{MR3486430}.
\begin{prop}\label{one_veretex_H-triantulation}
An example of one vertex H-triangulation $Y=(T_1,\cdots,T_6,\sim)$ for $(S^3,7_3)$ is given in Figure \ref{fig:one_vertex_Htriangulation}.
To explain the way to view this figure, the top left figure about the tetrahedron $T_1^{+}$ is a top view of the tetrahedron with the bottom face $\partial_0(T_1^{+})$ opposite to the vertex $0$. The variable $x$ assigned to the bottom face $\partial_0(T_1^{+})$ is marked outside, and the variables assigned to the other faces are marked inside the corresponding triangles. The same is true for the other tetrahedra. 
% Figure environment removed
% Figure environment removed

Let $\Delta_1(Y)=\{e_0, e_1, \cdots, e_6\}$, and the inverse image of each edge before gluing is marked with an arrow symbol. Figure \ref{fig:one_vertex_Htriangulation_edge} shows the correspondence. The edge corresponding to the knot is $e_0$. 
\end{prop}
\begin{proof}
Let $S^3$ be a standard one-point compactification of $\mathbb{R}^3$, and fix the embedding $S^2\subset S^3$ as the closure of the standard embedding $\mathbb{R}^2\subset \mathbb{R}^3$, $(x,y)\mapsto (x,y,0)$. Let $K$ be a knot $7_3$ in $S^3$, take a small positive number $\epsilon$, and let $K \subset \mathbb{R}^2\times[-\epsilon,\epsilon]\subset\mathbb{R}^3$.
Let the projective diagram $D$ be the image of $K$ by the orthogonal projection $\mathbb{R}^3\rightarrow \mathbb{R}^2$. Let $K$ be cellularly decomposed, and small circles in $D$ represent the images of its vertices by the orthogonal projection.
 Figure \ref{fig:knot-diagram} shows $D$.
% Figure environment removed
The cellular decomposition of $K$ can be extended to a cellular decomposition of $S^3$ by adding new edges with the same vertices.
Here the edges are added so that each intersection point of $D$ is bounded by the orthogonal projective image of the newly added four edges, as in Figure \ref{fig:shaded-knot-diagram}, and each higher dimensional cell is given by a tetrahedral cell with the natural cellular structure contained in $\mathbb{R}^2\times[-\epsilon,\epsilon]$ such that it is projected onto a shaded rectangle containing the intersection point of $D$.
% Figure environment removed
The other higher dimensional cells are given by two 3-dimensional cells $\Tilde{c}_\pm$ which are respectively the common part of the 3-dimensional balls $B_+$, $B_-$ defined by the closure of upper and lower half spaces in $S^3$, i.e., $B_\pm=\overline{\left\{(x,y,z)\mid \pm z\ge 0 \right\}}$ and the complementary space of tetrahedral cells in $S^3$.
From the cellular decomposition of $S^3$ constructed in this way, we create a new cellular complex by an isotopy which starts from the identity map and ends up in the projection to the quotient space by an equivalence relation that identifies all points on $K$ except one edge and collapses each tetrahedron to one edge, as in Figure \ref{fig:isotopy}.

% Figure environment removed
The resulting cellular complex consists of two $3$-dimensional cells $c_\pm$, which are the projective images of $\Tilde{c}_\pm$, and the complementary region of the shaded rectangles containing the intersection points of the diagram $D$ gives the 2-skeleton.

In Figure \ref{fig:oriented-shaded-diagram}, the edge of $K$ that is not collapsed to one point is the uppermost horizontal line segment, and the unshaded region, i.e., three triangular cells, one pentagonal cell, and one hexagonal cell corresponding to the outer region, gives the $2$-dimensional cells. The same kinds of arrows attached to the edges represent the edges that are identified, including their orientations if tetrahedra are collapsed into line segments. 
% Figure environment removed

By gluing two 3-dimensional cells $c_\pm$ along the hexagonal 2-dimensional cell corresponding to the outer region in the projective diagram of the knot, we obtain a cellular complex given by a single 3-dimensional cell whose boundaries are composed of remaining 2-dimensional cells as in Figure \ref{glued-complex}.
Here each 2-dimensional cell appears twice with different orientations corresponding to $c_+$ and $c_-$, and in Figure \ref{glued-complex}, 2-dimensional cells are glued together with the same letters as those marked with $\prime$ on the right shoulder. Figure \ref{glued-complex-2-decomposition} shows the cellular complex given by the cellular decomposition of the 2-dimensional cells represented by $2$ and $2^\prime$ in Figure \ref{glued-complex}. Figure \ref{glued-complex-cut-T0} shows the tetrahedron $T_0$ and the remaining cellular complex obtained by splitting the cellular complex with a new two-dimensional cell bounded by the dashed line in Figure \ref{glued-complex-2-decomposition}.

% Figure environment removed

Figure \ref{fig:T0_remainder} shows the cellular decomposition of $2$-dimensional cells represented by the symbol $6$, $6^\prime$ in the non-tetrahedral cellular complex obtained in Figure \ref{glued-complex-cut-T0} and Figure \ref{fig:glued-complex-cut-T1} represents the tetrahedron $T_1$ and the cellular complex obtained by splitting the tetrahedron $T_1$ by a new $2$-dimensional cell whose boundary is the dashed line in Figure \ref{fig:T0_remainder}.

% Figure environment removed

Similarly, Figure \ref{fig:T1-remainder} shows the cellular decomposition of $2$-dimensional cells represented by the symbol $4$,$4^\prime$ in the non-tetrahedral cellular complex obtained in Figure \ref{fig:glued-complex-cut-T1}, and Figure \ref{fig:T2-cut} represents the tetrahedron $T_2$ and the cellular complex obtained by splitting the tetrahedron $T_2$ by a new $2$-dimensional cell whose boundary is the dashed line in Figure \ref{fig:T1-remainder}.

% Figure environment removed

Figure \ref{fig:T2-remainder} shows the cellular decomposition of the $2$-dimensional cells represented by the symbol $3$, $3^\prime$ in the non-tetrahedral cellular complex obtained in Figure\ref{fig:T2-cut}, and Figure \ref{fig:cut-T3-T6} shows the tetrahedra $T_3,\cdots,T_6$ obtained by splitting them by the $2$-dimensional cells whose boundaries are the dashed lines in Figure \ref{fig:T2-remainder}. 

% Figure environment removed

Furthermore, for the cellular complex consisting of tetrahedra $T_4$, $T_5$, and $T_6$ glued together with their corresponding $2$-dimensional cells, another cellular decomposition consisting of two tetrahedra $T_7,T_8$ is obtained by applying the Pachner $3-2$ move as shown in Figure \ref{fig:7_3-Pachner}.

% Figure environment removed

The tetrahedra $T_0,T_1,T_2,T_3,T_7$, and $T_8$ obtained in the above process give the one vertex H-triangulation.
\end{proof}
\begin{prop}\label{thm:ideal_triangulation}
An example of ideal tetrahedral decompositions of $S^3\backslash 7_3$, $X=(T_1,\cdots,T_5,\sim)$, is given in Figure \ref{fig:ideal_triangulation}.
Here $\Delta_1(X)=\{ e_1, e_2, \cdots, e_5 \}$ and the edges of the inverse image of each edge by $\sim$ are marked with the same arrow symbol. The correspondence between the arrow symbols and the edges $e_1,e_2,\cdots,e_5$ is shown in Figure \ref{fig:ideal_triangulation_edge}.
% Figure environment removed
% Figure environment removed
\end{prop}
\begin{proof}
 For the one vertex H-triangulation obtained by Theorem \ref{one_veretex_H-triantulation}, this decomposition is obtained by collapsing the edge $e_0$ corresponding to the knot to one point and the tetrahedron $T_1^{+}$ by gluing the faces $\partial_2(T_1^{+})$ and $\partial_3(T_1 ^{+})$. In this case, we glue the faces which are assigned the variables $y$ and $z$ in Figure \ref{fig:one_vertex_Htriangulation}. 
\end{proof}