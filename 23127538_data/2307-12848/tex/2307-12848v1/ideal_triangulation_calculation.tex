\section{Calculation of the partition function for the ideal tetrahedral decomposition of the complementary space of $7_3$ knot.}
\label{ideal-triangulation}
In this section, we calculate the partition function of the ideal tetrahedral decomposition $X$ for the knot complement $S^3\backslash 7_3$ obtained by Proposition \ref{thm:ideal_triangulation} in Section \ref{triangulation}.

In general, the second-order homology group with an integer coefficient of the complementary space of the knot embedded in $S^3$ is shown to be trivial, either by using the Mayer-Vietoris exact sequence \cite{MR1472978} or the Alexander duality theorem. Then $H_2(X-\Delta_0(X))=0$. Therefore, $X$ is admissible. 

Let $\alpha_X=(2\pi a_1,2\pi b_1,2\pi c_1,\cdots,2\pi a_5, 2\pi b_5,2\pi c_5)\in\mathscr{S}_X$.

From Figure \ref{fig:ideal_triangulation}, the condition that $\alpha_X\in\mathscr{A}_X$, i.e., the weight of each edge is $2\pi$, is
\begin{align}
e_1: &\quad & b_1+b_2+a_3+b_4+c_4+a_5+b_5&=1\label{it1}\\
e_2:&\quad & b_1+c_1+c_2+b_3+c_3+a_4+c_5&=1\label{it2}\\
e_3:&\quad & a_2+b_3+a_4+b_4+a_5+c_5&=1\label{it3}\\
e_4:&\quad & a_1+c_1+a_2+c_2+a_3&=1\label{it4}\\
e_5:&\quad & a_1+b_2+c_3+c_4+b_5&=1\label{it5}.
\end{align}

Using $a_i+b_i+c_i=\frac{1}{2}$ for $i=1,2,\ldots,5$, 
\begin{empheq}[left={\eqref{it1},\eqref{it2},\eqref{it3},\eqref{it4},\eqref{it5}\iff\empheqlbrace}]{align}
b_1+b_2+a_3&=a_4+c_5 \label{it1'}\\
a_2&=2b_1+c_1 \label{it2''}\\
a_2+b_3&=c_4+b_5 \label{it3'}\\
a_3&=b_1+b_2\label{it4'}.
\end{empheq}
\begin{lemma}\label{non-empty}
$\mathscr{A}_X$ is non-empty.
\end{lemma}
\begin{proof}
$(a_1,b_1,c_1) =(\frac{1}{4},\frac{1}{8},\frac{1}{8})$, 
$(a_2,b_2,c_2)=(\frac{3}{8},\frac{1}{16},\frac{1}{16})$,  
$(a_3,b_3,c_3)=(\frac{3}{16},\frac{1}{8},\frac{3}{16})$, 
$(a_4,b_4,c_4)=(\frac{1}{8},\frac{1}{16},\frac{5}{16})$, 
$(a_5,b_5,c_5)=(\frac{1}{16},\frac{3}{16},\frac{1}{4})$
gives a non-empty element of $\mathscr{A}_X$.
\end{proof}
In the following, we calculate the partition function $Z_{\hbar}(X,\alpha_X)$ of the fully balanced shaped ideal triangulation $(X,\alpha_X)$. 
\begin{theorem}\label{volume-conjecture-1}
The partition function of the fully balanced shaped ideal triangulation $(X,\alpha_X)$ is expressed as follows.
\begin{equation*}
Z_\hbar(X,\alpha_X)=e^{-\frac{\pi}{3}i}e^{i\frac{\phi(\alpha_X)}{\hbar}}
\int_{\mathbb{R}+\frac{i\mu(\alpha_X)}{\sqrt{\hbar}}}J_X(\hbar,x)e^{-\frac{x\lambda(\alpha_X)}{\sqrt{\hbar}}}dx,
\end{equation*}
where $\lambda(\alpha_X)$ and $\mu(\alpha_X)$ are gauge invariant real linear combinations of dihedral angles, $\phi(\alpha_X)$ is a real quadratic polynomial of dihedral angles, and 
\begin{equation*}
\begin{split}
J_X:(\hbar,x)\mapsto
\int_{\mathscr{Y}^\prime} d\mathbf{y}^\prime e^{2i\pi({\mathbf{y}^\prime}^{\rm T}Q\mathbf{y}^\prime-\frac{3}{2}x^2)}e^{\frac{1}{\sqrt{\hbar}}{\mathbf{y}^\prime}^{\rm T}\mathscr{W}}\frac{\Phi_{\mathsf{b}}(Z^\prime)}{\Phi_{\mathsf{b}}(Y^\prime)\Phi_{\mathsf{b}}(W^\prime)\Phi_{\mathsf{b}}(W^\prime-x)\Phi_{\mathsf{b}}(W^\prime+x)}
\end{split}
\end{equation*}
where
\begin{eqnarray*}
\mathscr{Y}^\prime&\coloneqq&(\mathbb{R}-2c_{\mathsf{b}}(c_1+b_1))\times(\mathbb{R}+2c_{\mathsf{b}}(b_2+c_2))\times(\mathbb{R}-2c_{\mathsf{b}}(c_3+b_3))\\
&=&\left(\mathbb{R}-\frac{i}{2\sqrt{\hbar}}(1-2a_1)\right)\times\left(\mathbb{R}+\frac{i}{2\sqrt{\hbar}}(1-2a_2)\right)\times\left(\mathbb{R}-\frac{i}{2\sqrt{\hbar}}(1-2a_3)\right)
\end{eqnarray*}
and let
\begin{equation*}
\mathbf{y}^\prime=\begin{bmatrix}
Y^\prime\\
Z^\prime\\
W^\prime
\end{bmatrix},\quad
\mathscr{W}=\begin{bmatrix}
-\pi\\
0\\
\pi
\end{bmatrix},\quad
Q=\begin{bmatrix}
1 & -\frac{1}{2} & 0\\
-\frac{1}{2} & 0 & \frac{1}{2}\\
0 & \frac{1}{2} &\frac{1}{2}
\end{bmatrix}.
\end{equation*}
\end{theorem}
\begin{proof}
\begin{equation*}
\begin{split}
Z_\hbar(X,\alpha_X)&=\int_{\mathbb{R}^{10}}dxdydzdwdudvdpdqdrds\ \langle y,w\lvert \mathsf{T}(a_1,c_1)\rvert z,x\rangle \langle v,x\lvert \overline{\mathsf{T}}(a_2,c_2)\rvert u,y\rangle \\ 
&\quad\quad\times \langle q,u\lvert \mathsf{T}(a_3,c_3)\rvert p,w\rangle
\langle r,p\lvert \mathsf{T}(a_4,c_4)\rvert s,q\rangle \langle s,z\lvert \mathsf{T}(a_5,c_5)\rvert v,r\rangle \\
&=\int_{\mathbb{R}^{10}}dxdydzdwdudvdpdqdrds\ \delta(y+w-z)e^{-\frac{\pi}{12}i}\psi_{c_1,b_1}(x-w)e^{2\pi iy(x-w)}\\
& \quad\quad\times  \delta(u+y-v)
\psi_{b_2,c_2}(y-x)e^{-\frac{\pi}{12}i}e^{\pi i(y-x)^2}e^{-2\pi iu(x-y)}\delta(q+u-p)e^{-\frac{\pi}{12}i}\\ 
&\quad\quad\times \psi_{c_3,b_3}(w-u)
e^{2\pi iq(w-u)}
\delta(r+p-s)e^{-\frac{\pi}{12}i}\psi_{c_4,b_4}(q-p)e^{2\pi i r(q-p)}\\
&\quad\quad\times\delta(s+z-v)e^{-\frac{\pi}{12}i}\psi_{c_5,b_5}(r-z)e^{2\pi is(r-z)}.
\end{split}
\end{equation*}
Iterated integration with respect to the variables $q$, $p$, $s$, $v$, and $z$, in turn, gives
\begin{equation*}
\begin{split}
Z_\hbar(X,\alpha_X)&=\int_{\mathbb{R}^5}dxdydwdudr\ e^{-\frac{5\pi}{12}i}\psi_{c_1,b_1}(x-w)e^{2\pi iy(x-w)}\psi_{b_2,c_2}(y-x)e^{\pi i(y-x)^2}\\
&\quad\quad\times e^{-2\pi iu(x-y)}\psi_{c_3,b_3}(w-u)
 e^{-2\pi i(r+w)(w-u)}\psi_{c_4,b_4}(-u)e^{-2\pi iru}\\
& \quad\quad\times \psi_{c_5,b_5}(r-y-w)e^{2\pi i(u-w)(r-y-w)}.
\end{split}
\end{equation*}
Transformation of variables,
$\left\{
\begin{aligned}
&A\coloneqq x-w\\
&B\coloneqq y-x\\
&C\coloneqq w-u\\
&D\coloneqq -u\\
&E\coloneqq r-y-w
\end{aligned},
\right.$
gives 
$\left\{
\begin{aligned}
&x=A+C-D\\
&y=A+B+C-D\\
&w=C-D\\
&u=-D\\
&r=A+B+2C-2D+E
\end{aligned}
\right.$
and its Jacobian is 
$
\begin{vmatrix}
1 & 0 & 1& -1&0\\
1 & 1& 1 & -1&0 \\
0 & 0 & 1 & -1&0 \\
0 & 0 & 0 & -1&0\\
1 & 1 & 2 & -2&1
\end{vmatrix}
=-1$. Then 
\begin{equation*}
\begin{split}
Z_{\hbar}(X,\alpha_X)
&=\int_{\mathbb{R}^5}dAdBdCdDdE\ e^{-\frac{5\pi}{12}i}\psi_{c_1,b_1}(A)\psi_{b_2,c_2}(B)\psi_{c_3,b_3}(C)\psi_{c_4,b_4}(D)\psi_{c_5,b_5}(E)\\
&\quad\quad\times e^{\pi iB^2}e^{2\pi i(A^2+AB-3C^2+5CD-2CE-2D^2+DE-BC)}.
\end{split}
\end{equation*}
Since
\begin{equation*}
\int_{\mathbb{R}}dE\ \psi_{c_5,b_5}(E)e^{2\pi iE(-2C+D)}=\widetilde{\psi}_{c_5,b_5}(2C-D)=e^{\pi i(2C-D)^2}e^{-\frac{\pi}{12}i}\psi_{b_5,a_5}(2C-D) , 
\end{equation*}
\begin{equation*}
\begin{split}
Z_{\hbar}(X,\alpha_X)
&=\int_{\mathbb{R}^4}dAdBdCdDe^{-\frac{\pi}{2}i}\psi_{c_1,b_1}(A)\psi_{b_2,c_2}(B)\psi_{c_3,b_3}(C)
\psi_{c_4,b_4}(D)\psi_{b_5,a_5}(2C-D)\\
&\quad\quad\times e^{\pi iB^2}e^{2\pi i(A^2+AB-C^2+3CD-BC)}e^{-3\pi iD^2}\\
&=\int_{\mathbb{R}^4}dAdBdCdD  e^{-\frac{\pi}{2}i}\psi(A-2c_{\mathsf{b}}(b_1+c_1))e^{-4\pi ic_{\mathsf{b}}c_1(A-c_{\mathsf{b}}(c_1+b_1))}\\
&\quad\quad\times e^{-\pi ic_{\mathsf{b}}^2\frac{4(c_1-b_1)+1}{6}}
\psi(B-2c_{\mathsf{b}}(b_2+c_2))e^{-4\pi ic_{\mathsf{b}}b_2(B-c_{\mathsf{b}}(b_2+c_2))}\\
&\quad\quad\times e^{-\pi ic_{\mathsf{b}}^2\frac{4(b_2-c_2)+1}{6}}\psi(C-2c_{\mathsf{b}}(c_3+b_3))e^{-4\pi ic_{\mathsf{b}}c_3(C-c_{\mathsf{b}}(c_3+b_3))}\\
&\quad\quad\times e^{-\pi ic_{\mathsf{b}}^2\frac{4(c_3-b_3)+1}{6}}\psi(D-2c_{\mathsf{b}}(c_4+b_4))e^{-4\pi ic_{\mathsf{b}}c_4(D-c_{\mathsf{b}}(c_4+b_4))}\\
&\quad\quad\times e^{-\pi ic_{\mathsf{b}}^2\frac{4(c_4-b_4)+1}{6}}\psi(2C-D-2c_{\mathsf{b}}(b_5+a_5))e^{-4\pi ic_{\mathsf{b}}b_5(2C-D-c_{\mathsf{b}}(b_5+a_5))}\\
&\quad\quad\times e^{-\pi ic_{\mathsf{b}}^2\frac{4(b_5-a_5)+1}{6}}e^{\pi iB^2}e^{2\pi i(A^2+AB-C^2+3CD-BC)}e^{-3\pi iD^2}.
\end{split}
\end{equation*}
Transformation of variables 
\begin{equation*}
\begin{aligned}
&\Tilde{A}\coloneqq A-2c_{\mathsf{b}}(c_1+b_1)\\
&\Tilde{B}\coloneqq B-2c_{\mathsf{b}}(b_2+c_2)\\
&\Tilde{C}\coloneqq C-2c_{\mathsf{b}}(c_3+b_3)\\
&\Tilde{D}\coloneqq D-2c_{\mathsf{b}}(c_4+b_4)
\end{aligned}
\end{equation*}
gives, using the equations \eqref{it1'}, \eqref{it4'}, and $a_i+b_i+c_i=\frac{1}{2}\quad( i=1, \ldots, 5 )$,
\begin{equation*} 
2C-D-2c_{\mathsf{b}}(a_5+b_5)=2(\Tilde{C}+2c_{\mathsf{b}}(b_3+c_3))-\Tilde{D}-2c_{\mathsf{b}}(b_4+c_4)-2c_{\mathsf{b}}(a_5+b_5)=2\Tilde{C}-\Tilde{D}.
\end{equation*}
Concerning
\begin{equation*}
\begin{split}
&\quad e^{-4\pi ic_{\mathsf{b}}c_1 A}e^{-4\pi ic_{\mathsf{b}}b_2B}e^{-4\pi ic_{\mathsf{b}}c_3C}e^{-4\pi ic_{\mathsf{b}}c_4D}
e^{-4\pi ic_{\mathsf{b}}b_5(2C-D)}e^{\pi iB^2}e^{2\pi i(A^2+AB-C^2+3CD-BC)}
\\
&\quad \times e^{-3\pi iD^2}\\
&= e^{-4\pi ic_{\mathsf{b}}c_1(\Tilde{A}+2c_{\mathsf{b}}(c_1+b_1))}e^{-4\pi ic_{\mathsf{b}}b_2(\Tilde{B}+2c_{\mathsf{b}}(b_2+c_2))}e^{-4\pi ic_{\mathsf{b}}c_3(\Tilde{C}+2c_{\mathsf{b}}(c_3+b_3))}e^{-4\pi ic_{\mathsf{b}}c_4(\Tilde{D}+2c_{\mathsf{b}}(c_4+b_4))}\\
&\quad\times e^{-4\pi ic_{\mathsf{b}}b_5(2\Tilde{C}-\Tilde{D}+2c_{\mathsf{b}}(a_5+b_5))}e^{\pi i\left\{\Tilde{B}+2c_{\mathsf{b}}(b_2+c_2)\right\}^2}e^{2\pi i\left\{\Tilde{A}+2c_{\mathsf{b}}(c_1+b_1)\right\}^2}\\
&\quad\times e^{2\pi i\left\{\Tilde{A}+2c_{\mathsf{b}}(c_1+b_1)\right\}\left\{\Tilde{B}+2c_{\mathsf{b}}(b_2+c_2)\right\}}e^{-2\pi i\left\{\Tilde{C}+2c_{\mathsf{b}}(c_3+b_3)\right\}^2}e^{6\pi i \left\{\Tilde{C}+2c_{\mathsf{b}}(c_3+b_3)\right\}\left\{\Tilde{D}+2c_{\mathsf{b}}(c_4+b_4)\right\}}\\
&\quad\times e^{-2\pi i\left\{\Tilde{B}+2c_{\mathsf{b}}(b_2+c_2)\right\}\left\{\Tilde{C}+2c_{\mathsf{b}}(c_3+b_3)\right\}}e^{-3\pi i\left\{\Tilde{D}+2c_{\mathsf{b}}(c_4+b_4)\right\}^2},
\end{split}
\end{equation*}
the first-order terms concerning $\Tilde{A}$ with constant coefficients in the exponent of $e$ are 
\begin{equation*}
\begin{split}
&\quad 4\pi ic_{\mathsf{b}}\Tilde{A}\left(-c_1+2(b_1+c_1)+b_2+c_2\right)=4\pi ic_{\mathsf{b}}\Tilde{A}\left(-c_1+2\left(\frac{1}{2}-a_1\right)+\frac{1}{2}-a_2\right)\\
&
=4\pi ic_{\mathsf{b}}\left(\frac{3}{2}-2a_1-c_1-2b_1-c_1\right)\quad (\because\eqref{it2''})\\
&=4\pi ic_{\mathsf{b}}\Tilde{A}\left(\frac{3}{2}-2(a_1+b_1+c_1)\right)=2\pi ic_{\mathsf{b}}\Tilde{A},
\end{split}
\end{equation*}
the first-order terms concerning $\Tilde{B}$ with constant coefficients in the exponent of $e$ are
\begin{equation*}
\begin{split}
&\quad 4\pi ic_{\mathsf{b}}\Tilde{B}(-b_2+b_2+c_2+b_1+c_1-(b_3+c_3))=4\pi ic_{\mathsf{b}}\left(c_2+\frac{1}{2}-a_1-\left(\frac{1}{2}-a_3\right)\right)\\
&=4\pi ic_{\mathsf{b}}\Tilde{B}(c_2-a_1+a_3)=4\pi ic_{\mathsf{b}}\Tilde{B}(c_2-a_1+b_1+b_2)\quad(\because\eqref{it4'})\\
&=4\pi ic_{\mathsf{b}}\Tilde{B}\left(\frac{1}{2}-a_2-a_1+b_1\right)=4\pi ic_{\mathsf{b}}\Tilde{B}\left(\frac{1}{2}-2b_1-c_1-a_1+b_1\right)\quad(\because\eqref{it2''})\\
&=4\pi ic_{\mathsf{b}}\Tilde{B}\left(\frac{1}{2}-a_1-b_1-c_1\right)=0,
\end{split}
\end{equation*}
the first-order terms concerning $\Tilde{C}$ with constant coefficients in the exponent of $e$ are
\begin{equation*}
\begin{split}
&\quad 4\pi ic_{\mathsf{b}}\Tilde{C}(-c_3-2b_5-2(b_3+c_3)+3(c_4+b_4)-(b_2+c_2))\\
&=4\pi ic_{\mathsf{b}}\Tilde{C}\left(-c_3-2b_5-2\left(\frac{1}{2}-a_3\right)+3\left(\frac{1}{2}-a_4\right)-\left(\frac{1}{2}-a_2\right)\right)\\
&=4\pi ic_{\mathsf{b}}\Tilde{C}(-c_3-2b_5+2a_3-3a_4+a_2),
\end{split}
\end{equation*}
the first-order terms concerning $\Tilde{D}$ with constant coefficients in the exponent of $e$ are
\begin{equation*}
\begin{split}
&\quad 4\pi ic_{\mathsf{b}}\Tilde{D}(-c_4+b_5+3(b_3+c_3)-3(c_4+b_4))\\
&=4\pi ic_{\mathsf{b}}\Tilde{D}\left(-c_4+b_5+3\left(\frac{1}{2}-a_3\right)-3\left(\frac{1}{2}-a_4\right)\right)\\
&=4\pi ic_{\mathsf{b}}\Tilde{D}(-c_4+b_5-3a_3+3a_4).
\end{split}
\end{equation*}
Since
\begin{equation*}
\begin{split}
&\quad -c_3-2b_5+2a_3-3a_4+a_2-c_4+b_5-3a_3+3a_4\\
&=-c_3-c_4-b_5-a_3+a_2=-c_3-(a_2+b_3)-a_3+a_2\quad(\because\eqref{it3'})\\
&=-a_3-b_3-c_3=-\frac{1}{2},
\end{split}
\end{equation*}
let
\begin{equation*}
\lambda\coloneqq -c_3-2b_5+2a_3-3a_4+a_2 ,    
\end{equation*} 
then
\begin{equation*}
-c_4+b_5-3a_3+3a_4=-\lambda-\frac{1}{2} .     
\end{equation*} 
Therefore, 
\begin{equation*}
\begin{split}
Z_{\hbar}(X,\alpha_X)&=\int_{\mathscr{Y}}d\Tilde{A}d\Tilde{B}d\Tilde{C}d\Tilde{D}\ e^{-\frac{\pi}{2}i}\frac{1}{\Phi_{\mathsf{b}}(\Tilde{A})\Phi_{\mathsf{b}}(\Tilde{B})\Phi_{\mathsf{b}}(\Tilde{C})\Phi_{\mathsf{b}}(\Tilde{D})\Phi_{\mathsf{b}}(2\Tilde{C}-\Tilde{D})}e^{2\pi ic_{\mathsf{b}}\Tilde{A}}\\
&\quad\times e^{4\pi ic_{\mathsf{b}}\lambda\Tilde{C}}e^{4\pi ic_{\mathsf{b}}(-\lambda-\frac{1}{2})\Tilde{D}} e^{\pi i\Tilde{B}^2}e^{2\pi i(\Tilde{A}^2+\Tilde{A}\Tilde{B}-\Tilde{C}^2+3\Tilde{C}\Tilde{D}-\Tilde{B}\Tilde{C})}e^{-3\pi i\Tilde{D}^2}\times\mbox{\textcircled{\scriptsize 1}},
\end{split}
\end{equation*}
where
\begin{equation*}
\begin{split}
\mbox{\textcircled{\scriptsize 1}}&\coloneqq e^{4\pi ic_{\mathsf{b}}^2c_1(c_1+b_1)}e^{-\pi ic_{\mathsf{b}}^2\frac{4(c_1-b_1)+1}{6}}e^{4\pi ic_{\mathsf{b}}^2b_2(b_2+c_2)}e^{-\pi ic_{\mathsf{b}}^2\frac{4(b_2-c_2)+1}{6}}
 e^{4\pi ic_{\mathsf{b}}^2c_3(c_3+b_3)}\\
 &\quad \times e^{-\pi ic_{\mathsf{b}}^2\frac{4(c_3-b_3)+1}{6}}e^{4\pi ic_{\mathsf{b}}^2c_4(c_4+b_4)}
  e^{-\pi ic_{\mathsf{b}}^2\frac{4(c_4-b_4)+1}{6}}e^{4\pi ic_{\mathsf{b}}^2b_5(b_5+a_5)}e^{-\pi ic_{\mathsf{b}}^2\frac{4(b_5-a_5)+1}{6}}\\
&\quad \times e^{-8\pi ic_{\mathsf{b}}^2c_1(c_1+b_1)}e^{-8\pi ic_{\mathsf{b}}^2b_2(b_2+c_2)}e^{-8\pi ic_{b}^2c_3(c_3+b_3)}e^{-8\pi ic_{\mathsf{b}}^2c_4(c_4+b_4)}e^{-8\pi c_{\mathsf{b}}^2b_5(a_5+b_5)}\\
&\quad \times e^{4\pi ic_{\mathsf{b}}^2(b_2+c_2)^2}
e^{8\pi ic_{\mathsf{b}}^2(c_1+b_1)^2}e^{8\pi ic_{\mathsf{b}}^2(b_1+c_1)(b_2+c_2)}
e^{-8\pi ic_{\mathsf{b}}^2(b_3+c_3)^2}e^{24\pi ic_{\mathsf{b}}^2(b_3+c_3)(b_4+c_4)}\\
&\quad \times e^{-8\pi ic_{\mathsf{b}}^2(b_2+c_2)(b_3+c_3)}
e^{-12\pi ic_{\mathsf{b}}^2(c_4+b_4)^2}
\end{split}
\end{equation*}
and let 
\begin{equation*}
\mathscr{Y}\coloneqq (\mathbb{R}-2c_{\mathsf{b}}(c_1+b_1))\times(\mathbb{R}-2c_{\mathsf{b}}(b_2+c_2))\times(\mathbb{R}-2c_{\mathsf{b}}(c_3+b_3))\times(\mathbb{R}-2c_{\mathsf{b}}(c_4+b_4)).
\end{equation*}

By Proposition \ref{property-of-Phi_b}(1), since 
\begin{equation*}
\frac{e^{i\pi\Tilde{B}^2}}{\Phi_{\mathsf{b}}(\Tilde{B})}=e^{i\pi\frac{1+2c_{\mathsf{b}}^2}{6}}\Phi_{\mathsf{b}}(-\Tilde{B}),
\end{equation*} 
\begin{equation}
\begin{split}
Z_{\hbar}(X,\alpha_X)&=e^{-\frac{\pi}{3}i}\int_{\mathscr{Y}}d\Tilde{A}d\Tilde{B}d\Tilde{C}d\Tilde{D}\frac{1}{\Phi_{\mathsf{b}}(\Tilde{A})}e^{\frac{i}{3}\pi c_{\mathsf{b}}^2}\Phi_{\mathsf{b}}(-\Tilde{B})\frac{1}{\Phi_{\mathsf{b}}(\Tilde{C})\Phi_{\mathsf{b}}(\Tilde{D})\Phi_{\mathsf{b}}(2\Tilde{C}-\Tilde{D})}\\
&\quad\times e^{2\pi ic_{\mathsf{b}}\Tilde{A}}e^{4\pi ic_{\mathsf{b}}\lambda\Tilde{C}} e^{4\pi ic_{\mathsf{b}}(-\lambda-\frac{1}{2})\Tilde{D}}e^{2\pi i(\Tilde{A}^2+\Tilde{A}\Tilde{B}-\Tilde{C}^2+3\Tilde{C}\Tilde{D}-\Tilde{B}\Tilde{C})}e^{-3\pi i\Tilde{D}^2}\times\mbox{\textcircled{\scriptsize 1}}. 
\end{split}\label{partition-fun-ideal-for-volume}
\end{equation}
Let
\begin{equation*}
\Tilde{B}^\prime\coloneqq -\Tilde{B},
\end{equation*} 
then
\begin{equation*}
\begin{split}
Z_{\hbar}(X,\alpha_X)&=e^{-\frac{\pi}{3}i}\int_{\Tilde{\mathscr{Y}}} d\Tilde{A}d\Tilde{B}^\prime d\Tilde{C}d\Tilde{D}\frac{1}{\Phi_{\mathsf{b}}(\Tilde{A})}e^{\frac{i}{3}\pi c_{\mathsf{b}}^2}\Phi_{\mathsf{b}}(\Tilde{B}^\prime)\frac{1}{\Phi_{\mathsf{b}}(\Tilde{C})\Phi_{\mathsf{b}}(\Tilde{D})\Phi_{\mathsf{b}}(2\Tilde{C}-\Tilde{D})}\\
&\quad\times e^{2\pi ic_{\mathsf{b}}\Tilde{A}}e^{4\pi ic_{\mathsf{b}}\lambda\Tilde{C}}
 e^{4\pi ic_{\mathsf{b}}(-\lambda-\frac{1}{2})\Tilde{D}}e^{2\pi i(\Tilde{A}^2-\Tilde{A}\Tilde{B}^\prime-\Tilde{C}^2+3\Tilde{C}\Tilde{D}+\Tilde{B}^\prime\Tilde{C})}e^{-3\pi i\Tilde{D}^2}\times\mbox{\textcircled{\scriptsize 1}},
\end{split}
\end{equation*}
where
\begin{equation*}
\Tilde{\mathscr{Y}}\coloneqq (\mathbb{R}-2c_{\mathsf{b}}(c_1+b_1))\times(\mathbb{R}+2c_{\mathsf{b}}(b_2+c_2))\times(\mathbb{R}-2c_{\mathsf{b}}(c_3+b_3))\times(\mathbb{R}-2c_{\mathsf{b}}(c_4+b_4)).
\end{equation*}

Transformation of variables, 
$\left\{
\begin{aligned}
&x\coloneqq \Tilde{C}-\Tilde{D}\\
&Y\coloneqq \Tilde{A}\\
&Z\coloneqq \Tilde{B}^\prime\\
&W\coloneqq \Tilde{C}\\
\end{aligned}
\right.$, 
gives
$\left\{
\begin{aligned}
&\Tilde{A}=Y\\
&\Tilde{B}^\prime=Z\\
&\Tilde{C}=W\\
&\Tilde{D}=W-x\\
\end{aligned}
\right.$
and its Jacobian is 
$
\begin{vmatrix}
0 & 1 & 0 &0\\
0 & 0& 1 & 0\\
0 & 0 & 0 & 1\\
-1 & 0 & 0 & 1 
\end{vmatrix}
=1$. Therefore,  
\begin{equation*}
\begin{split}
Z_{\hbar}(X,\alpha_X)&=e^{-\frac{\pi}{3}i}\int_{\Tilde{\mathscr{Y}}^\prime} dx dY dZ dW e^{\frac{i}{3}\pi c_{\mathsf{b}}^2}\frac{\Phi_{\mathsf{b}}(Z)}{\Phi_{\mathsf{b}}(Y)\Phi_{\rm b}(W)\Phi_{\mathsf{b}}(W-x)\Phi_{\mathsf{b}}(W+x)}\\
&\quad\times  e^{4\pi ic_{\mathsf{b}}(\lambda+\frac{1}{2}) x}
e^{2\pi ic_{\mathsf{b}}(Y-W)}e^{\pi i(-3x^2+2Y^2+W^2-2YZ+2ZW)}\times\mbox{\textcircled{\scriptsize 1}}, 
\end{split}
\end{equation*}
where
\begin{equation*}
\Tilde{\mathscr{Y}}^\prime\coloneqq(\mathbb{R}+2c_{\mathsf{b}}(c_4+b_4-c_3-b_3))\times(\mathbb{R}-2c_{\mathsf{b}}(c_1+b_1))\times(\mathbb{R}+2c_{\mathsf{b}}(b_2+c_2))\times(\mathbb{R}-2c_{\mathsf{b}}(c_3+b_3)),
\end{equation*}
and $(x,Y,Z,W)\in\Tilde{\mathscr{Y}}^\prime$.

Since the integrand of $J_{X}(\hbar,X)$ rapidly decreases if $\alpha_X\in\mathscr{A}_X$ by the property of Proposition \ref{property-of-Phi_b} (3), the value of $J_{X}(\hbar,x)$ does not depend on the choice of $\alpha_X\in\mathscr{A}_X$ by the Bochner-Martinelli formula \cite{MR635928}, which generalizes Cauchy's integral theorem.

Then 
\begin{equation*}
\begin{split}
Z_{\hbar}(X,\alpha_X)&=e^{-\frac{\pi}{3}i}e^{\frac{i}{3}\pi c_{\mathsf{b}}^2}\times\mbox{\textcircled{\scriptsize 1}}\times\int_{\mathbb{R}+\frac{i\mu(\alpha_X)}{\sqrt{\hbar}}} dx J_X(\hbar,x) e^{-\frac{x}{\sqrt{\hbar}}\lambda(\alpha_X)}, \\
\end{split}
\end{equation*}
where
\begin{eqnarray*}
\mu(\alpha_X)&=&a_3-a_4\\
\lambda(\alpha_X)&=&-2\pi(-c_4+b_5-3a_3+3a_4).\\
\end{eqnarray*}

Let
\begin{equation*}
\lambda^\prime\coloneqq\lambda+\frac{1}{2}.
\end{equation*}
We check the gauge invariance of $\lambda^\prime$.  
Let $(\Tilde{X},\alpha_{\Tilde{X}})$ be a shaped pseudo $3$-manifold which is gauge equivalent to $(X,\alpha_X)$ and
\begin{equation*}
g: \Delta_{1}(X)\longrightarrow \mathbb{R}
\end{equation*}
be the map in Definition \ref{gauge-equivalent} associated with the gauge equivalence.
Let 
$\alpha_{\Tilde{X}}=(2\pi\Tilde{a}_1,2\pi\Tilde{b}_1,2\pi\Tilde{c}_1,\ldots,2\pi\Tilde{a}_5,2\pi\Tilde{b}_5,2\pi\Tilde{c}_5)$. Since
\begin{equation*}
\begin{split}
&-\Tilde{c}_3-2\Tilde{b}_5+2\Tilde{a}_3-3\Tilde{a}_4+\Tilde{a}_2\\
=&-\left(c_3+\frac{1}{2}(g(e_4)-g(e_3)+g(e_1)-g(e_2))\right)-2\left(b_5+\frac{1}{2}(g(e_2)-g(e_1)+g(e_3)-g(e_3))\right)\\
&+2\left(a_3+\frac{1}{2}(g(e_2)-g(e_2)+g(e_3)-g(e_5))\right)-3\left(a_4+\frac{1}{2}(g(e_1)-g(e_1)+g(e_3)-g(e_5))\right)\\
&+\left(a_2+\frac{1}{2}(g(e_2)-g(e_5)+g(e_4)-g(e_1))\right)\\
=&-c_3-2b_5+2a_3-3a_4+a_2, 
\end{split}
\end{equation*}
$\lambda$ is gauge invariant. Therefore $\lambda^\prime=\lambda+\frac{1}{2}$ and  $\lambda(\alpha_X)$ are also gauge invariant.\\
Since
\begin{equation*}
\begin{split}
&\quad \Tilde{a}_3-\Tilde{a}_4\\
&=\left(a_3+\frac{1}{2}\left(g(e_2)-g(e_2)+g(e_3)-g(e_5)\right)\right)-\left(a_4+\frac{1}{2}\left(g(e_1)-g(e_1)+g(e_3)-g(e_5)\right)\right)\\
&=a_3-a_4,
\end{split}
\end{equation*} 
$\mu(\alpha_X)$ is gauge invariant.
\end{proof}