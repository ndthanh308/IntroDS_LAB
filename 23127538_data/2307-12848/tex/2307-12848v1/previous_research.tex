\section{Andersen-Kashaev's Teichm\"{u}ller TQFT}\label{A-K-TQFT}
This section summarizes the results of the Teichm\"{u}ller TQFT \cite{MR3227503} of Andersen and Kashaev.
\subsection{Oriented triangulated pseudo 3-manifolds}
Consider the disjoint union of a finite number of copies of a standard 3-simplex in $\mathbb{R}^3$. Each simplex is assumed to have totally ordered vertices. The order of the vertices induces the orientation of the edges. Some pairs of faces of codimension 1 of this disjoint union are glued and identified by an affine homeomorphism, called gluing homeomorphism, which preserves the order of the vertices and reverses the faces' orientations. The quotient space $X$, by this identification, is a CW-complex with oriented edges called an oriented triangulated pseudo 3-manifold. 
For $i\in\left\{0,1,2,3\right\}$, $\Delta_i(X)$ denotes the i-dimensional cell of $X$. For any $i>j$, we define 
\begin{equation*}
\Delta_i^j(X)=\left\{(a,b)\mid a\in\Delta_i(X),b\in\Delta_j(a)\right\}.
\end{equation*} 

We also define the natural map as 
\begin{equation*}
\phi_{i,j}:\Delta_i^j(X)\rightarrow\Delta_i(X),\ \ \phi^{i,j}:\Delta_i^j(X)\rightarrow\Delta_j(X).
\end{equation*} 

Also, the standard boundary map 
\begin{equation*}
\partial_i:\Delta_j(X)\rightarrow \Delta_{j-1}(X),\ \ 0\le i\le j
\end{equation*}
is defined such that for a $j$ dimensional simplex $S=[v_0,v_1,\ldots,v_j]$ in $\mathbb{R}^3$ with ordered vertices $v_0,v_1,\ldots,v_j$,   
\begin{equation*}
\partial_iS=[v_0,\ldots,v_{i-1},v_{i+1},\ldots,v_j],\ \ i\in\{0,\ldots,j\}.
\end{equation*} 
\subsection{Shaped 3-manifolds}
Let $X$ be an oriented triangulated pseudo 3-manifold. 
\begin{definition}
The shape structure on $X$ is an assignment of a positive real number 
called the dihedral angle to each edge of each tetrahedron  
\begin{equation*}
\alpha_X:\Delta_3^1(X)\rightarrow\mathbb{R}_+
\end{equation*}
such that the sum of the dihedral angles at the three edges extending from each vertex of each tetrahedron is $\pi$. 
An oriented triangulated pseudo 3-manifold with a shape structure is called a shaped pseudo 3-manifold. Let $S(X)$ be the set of all shape structures on $X$. \label{shape}
\end{definition}
By Definition \ref{shape}, for any tetrahedron, the dihedral angles at opposite edges are equal. Thus, for each tetrahedron, three dihedral angles are associated with three pairs of opposite edges whose sum is $\pi$. 
\begin{definition}
For each shape structure on X, the weight function
\begin{equation*}
\omega_X:\Delta_1(X)\rightarrow\mathbb{R}_+
\end{equation*}
is defined by mapping each edge $e$ of $X$ to the sum of the dihedral angles around it.  
That is, for any $e\in\Delta_1(X)$, we define 
\begin{equation*}
\omega_X(e)=\sum_{a\in(\phi^{3,1})^{-1}(e)} \alpha_X(a).
\end{equation*}
\end{definition}
\begin{definition}
An edge $e$ of shaped pseudo 3-manifold $X$ is called balanced if it is inside $X$ and $\omega_X(e)=2\pi$. 
An edge that is not balanced is called unbalanced. 
We say that a shaped pseudo 3-manifold $X$ is fully balanced if every edge of $X$ is balanced. 
\end{definition}
\subsection{$\mathbb{Z}/3\mathbb{Z}$ action on the set of opposite edge pairs of a tetrahedron}
Let $X$ be an oriented triangulated pseudo 3-manifold. The orientation on $X$ gives a cyclic ordering to the three edges extending from each tetrahedron vertex. Moreover, this cyclic order induces a cyclic order on the set of opposite pairs of tetrahedron edges.
Let $\Delta_3^{1/p}(X)$ be the set consisting of the opposite edge pairs of all tetrahedra. 
$\Delta_3^{1/p}(X)$ is the quotient set for $\Delta_3^1(X)$ by the equivalence relation generated by the opposite edge pairs of all tetrahedra, and let the corresponding quotient map be
\begin{equation*}
    p:\Delta_3^1(X)\rightarrow\Delta_3^{1/p}(X).
\end{equation*}

Skew-symmetric function
\begin{equation*}
\epsilon_{a,b}\in\{0,\pm 1\},\quad \epsilon_{a,b}=-\epsilon_{b,a},\quad a,b\in\Delta_3^{1/p}(X)
\end{equation*}
is set $\epsilon_{a,b}=0$ if the tetrahedra in which $a$ and $b$ exist are different, and $\epsilon_{a,b}=1$ if the tetrahedra in which $a$ and $b$ exist coincide and $a$ is ahead of $b$ under the cyclic order.
\subsection{Leveled shaped 3-manifolds}
\begin{definition}
A leveled shaped 3-manifold is a pair $(X,l_X)$ consisting of a shaped pseudo 3-manifold $X$ and a real number $l_X\in\mathbb{R}$ called the level. The leveled shaped structure on an oriented triangulated pseudo 3-manifold $X$ is a pair $(\alpha_X,l_X)$ of a shape structure on $X$ and a level. Let $LS(X)$ denote the set of all leveled shaped structures on $X$. 
\end{definition}
\begin{definition}\label{gauge-equivalent}
Two leveled shaped pseudo 3-manifolds $(X,l_X)$ and $(Y,l_Y)$ are said to be gauge equivalent if there exists an isomorphism of cellular structures of $X$ and $Y$ 
\begin{equation*}
h:X\rightarrow Y    
\end{equation*}
and a function
\begin{equation*}
g:\Delta_1(X)\rightarrow \mathbb{R}    
\end{equation*}
which satisfies the following conditions. 
\begin{equation*}
\Delta_1(\partial X)\subset g^{-1}(0)
\end{equation*}
\begin{equation*}
\alpha_Y(h(a))=\alpha_X(a)+\pi \sum_{b\in\Delta_3^1(X)}\epsilon_{p(a),p(b)}g(\phi^{3,1}(b)),\quad\forall a\in\Delta_3^1(X)    
\end{equation*}
\begin{equation*}
l_Y=l_X+\sum_{e\in\Delta_1(X)}g(e)\sum_{a\in (\phi^{3,1})^{-1}(e)}\left(\frac{1}{3}-\frac{\alpha_X(a)}{\pi}\right)
\end{equation*}
\end{definition}
If $X$ and $Y$ are gauge equivalent, weights on edges are gauge invariant in the sense that 
\begin{equation*}
\omega_X=\omega_Y\circ h
\end{equation*}
holds.

\begin{definition}
Two leveled shape structures $(\alpha_X, l_X)$ and $(\alpha_X^\prime, l_X^\prime)$ on a pseudo 3-manifold $X$ are called based gauge equivalent if they are gauge equivalent with $h:X\rightarrow X$ as the identity map in Definition \ref{gauge-equivalent}. 
\end{definition}
The (based) gauge equivalence relation on leveled shaped pseudo 3-manifolds induces a (based) gauge equivalence relation on shaped pseudo 3-manifolds under the map which forgets the level. Let $LS_r(X)$ denote the set of gauge equivalence classes of based leveled shape structures on $X$ and $S_r(X)$ denote the set of gauge equivalence classes of based shape structures on $X$. 
\begin{definition}
A generalized shape structure on $X$ is an assignment of real numbers to each edge of each tetrahedron such that the sum of the real numbers assigned to the three edges extending from each vertex of each tetrahedron is $\pi$. Let $\widetilde{S}(X)$ denote the set of all generalized shape structures on $X$ and $\widetilde{LS}(X)$ denote the set of all generalized leveled shape structures on $X$. The leveled generalized shape structure and its gauge equivalence are defined in the same way as the leveled shaped structure and its gauge equivalence. We denote the space of based gauge equivalence classes of generalized shape structures by $\widetilde{S}_r(X)$ and the space of based gauge equivalence classes of leveled generalized shape structures by $\widetilde{LS}_r(X)$.
\end{definition}

$S_r(X)$ is an open convex subset of $\widetilde{S}_r(X)$.

Let the following map denote the map which assigns the weight function $\omega_X:\Delta_1(X)\rightarrow \mathbb{R}$ to the generalized shape structure $\alpha_X\in \widetilde{S}(X)$,  
\begin{equation*}
\widetilde{\Omega}_X:\widetilde{S}(X)\rightarrow\mathbb{R}^{\Delta_1(X)}.
\end{equation*}

Since this map is gauge equivalent, it induces the following unique map.
\begin{equation*}
\widetilde{\Omega}_{X,r}:\widetilde{S}_r(X)\rightarrow\mathbb{R}^{\Delta_1(X)}.
\end{equation*}

Let $N_0(X)$ be a sufficiently small tubular neighborhood of $\Delta_0(X)$. $\partial N_0(X)$ is a two-dimensional surface. It can be disconnected and with a boundary if $\partial X\neq\emptyset$.
\subsection{3-2 Pachner move}
Let $X$ be a shaped pseudo 3-manifold.
Let $e$ be a balanced edge of $X$ shared by three different tetrahedra $t_1,t_2$, and $t_3$.
Let $S$ be a shaped pseudo 3-submanifold of $X$ consisting of tetrahedra $t_1,t_2$, and $t_3$. There exists another triangulation $S_e$ of $S$ consisting of two tetrahedra $t_4,t_5$ so that the triangulations of $\partial S$ and $\partial S_e$ can be made to coincide. 
This is obtained by removing the edge $e$, then $\Delta_1(S_e)=\Delta_1(S)\backslash{e}$. 
Furthermore, there is a unique shape structure on $S_e$ that induces the same weight as that induced by the shape structure on $S$. 
Let $(\alpha_i,\beta_i,\gamma_i)$ be the dihedral angles of $t_i$ ( for $i=1,2,3$, let $\alpha_i$ be the dihedral angle on $e$). The following relation holds. 
\begin{equation*}
\alpha_4=\beta_2+\gamma_1,\quad\alpha_5=\beta_1+\gamma_2
\end{equation*}
\begin{equation*}
\beta_4=\beta_1+\gamma_3,\quad\beta_5=\beta_3+\gamma_1\\
\end{equation*}
\begin{equation*}
\gamma_4=\beta_3+\gamma_2,\quad\gamma_5=\beta_2+\gamma_3    
\end{equation*}

Since the edge $e$ is balanced, i.e.,  $\alpha_1+\alpha_2+\alpha_3=2\pi$, we can see that the sum of the dihedral angles of $t_4$ and $t_5$ is $\pi$, respectively, taking into consideration the above relation. Furthermore, since the dihedral angles of $t_1,t_2$, and $t_3$ are positive, the dihedral angles of $t_4$ and $t_5$ are also positive by the above equations. Conversely, given the dihedral angles of $t_4$ and $t_5$, it does not necessarily follow that the solutions for the positive dihedral angles of $t_1,t_2$ and $t_3$ are obtained. However, under common dihedral angles of $t_4,t_5$, if multiple solutions of positive dihedral angles of $t_1,t_2,t_3$ are obtained, they are gauge equivalent and  $\alpha_1+\alpha_2+\alpha_3=2\pi$. (Figure \ref{fig:3-2_Pachner}) . 
% Figure environment removed
\begin{definition}
If some shaped pseudo 3-manifold $Y$ is obtained from $X$ by replacing $S$ by $S_e$, we say that it is obtained from $X$ by a shaped 3-2 Pachner move along $e$. 
\end{definition}
A leveled shaped pseudo 3-manifold $(Y,l_Y)$ is said to be obtained by a leveled shaped 3-2 Pachner move if it is obtained from a leveled shaped pseudo 3-manifold $(X,l_X)$ such that the following condition is satisfied.
The condition is that there exists $e\in\Delta_1(X)$ such that $Y=X_e$, and 
\begin{equation*}
l_Y=l_X+\frac{1}{12}\sum_{a\in(\phi^{3,1})^{-1}(e)}\sum_{b\in\Delta_3^1(X)}\epsilon_{p(a),p(b)}\alpha_X(b).
\end{equation*}
This action is defined such that the level-dependent invariant is invariant to leveled shaped 3-2 Pachner moves.
\begin{definition}
A (leveled) shaped pseudo 3-manifold $X$ is called a Pachner refinement of a (leveled) shaped pseudo 3-manifold $Y$ if it satisfies the following condition.
The condition is that there exists a finite sequence of (leveled) shaped pseudo 3-manifolds 
\begin{equation*}
X=X_1,X_2,\ldots,X_n=Y
\end{equation*}
such that for any $i\in\{1,\ldots,n-1\}$, $X_{i+1}$ is obtained from $X_i$ by a (leveled) shaped 3-2 Pachner move.

Two (leveled) shaped pseudo 3-manifolds, $X$ and $Y$, are said to be equivalent if there exist Pachner refinements $X^\prime$, $Y^\prime$ of $X$ and $Y$ such that $X^\prime$ and $Y^\prime$ are gauge equivalent. 
\end{definition}

\begin{definition}
An oriented triangulated pseudo 3-manifold $X$ is said to 
be admissible if  
\begin{equation*}
    S_{r}(X)\neq\emptyset
\end{equation*}
and
\begin{equation*}
H_2(X-\Delta_0(X),\mathbb{Z})=0
\end{equation*} 
\end{definition}

The Andersen-Kashaev TQFT is not defined on all leveled shaped pseudo 3-manifolds, and the invariant of the TQFT is guaranteed to be well-defined only on admissible (leveled) shaped psudo 3-manifolds.
 This is because the positivity of the dihedral angles is necessary to regularize the distribution corresponding to the tetrahedra described below, and $H_2(X-\Delta_0(X),\mathbb{Z})=0$ guarantees that the product of the distributions for all tetrahedra constituting $X$ is determined, and we can push forward the product necessary to define the TQFT invariant.
\begin{definition}
Two admissible (leveled) shaped pseudo 3-manifolds $X$ and $Y$ are said 
to be admissibly equivalent if there exists a gauge equivalence $h^\prime: X^\prime\rightarrow Y^\prime$ for some Pachner refinements $X^\prime$, $Y^\prime$ of $X$ and $Y$ such that the following conditions are satisfied.
\begin{equation*}
\Delta_1(X^\prime)=\Delta_1(X)\cup D_X, \quad \Delta_1(Y^\prime)=\Delta_1(Y)\cup D_Y
\end{equation*}
and
\begin{equation*}
h^\prime(S_r(X^\prime)\cap\widetilde{\Omega}_{X^\prime,r}^{-1}(P_{D_X}))\cap\widetilde{\Omega}_{Y^\prime,r}^{-1}(P_{D_Y})\neq\emptyset,    
\end{equation*}
where
\begin{equation*}
P_{D_X}\coloneqq \left\{\omega:\Delta_1(X^\prime)\rightarrow\mathbb{R}\mid\forall e\in D_X, \omega(e)=2\pi \right\},
\end{equation*}
and $P_{D_Y}$ is defined similarly.
\end{definition}
\subsection{Categroid}
\begin{definition}
A categroid $\mathcal{C}$ consists of a family of objects $\mathrm{Obj} (\mathcal{C})$ and a set of morphisms $\mathrm{Hom}_{\mathcal{C}}(A,B)$ for any two objects  $A$, $B$ in $\mathrm{Obj} (\mathcal{C})$ such that the following two properties are satisfied. 

1. For any three objects $A$, $B$, $C\in\mathrm{Obj}(\mathcal{C})$, 
there exists a subset called composable morphisms 
\begin{equation*}
K_{A,B,C}^{\mathcal{C}}\subset \mathrm{Hom}_{\mathcal{C}}(A,B)\times \mathrm{Hom}_{\mathcal{C}}(B,C) 
\end{equation*}
and a composite map such that the composition of morphisms is associative,
\begin{equation*}
\circ: K_{A,B,C}^{\mathcal{C}}\rightarrow \mathrm{Hom}_{\mathcal{C}}(A,C).
\end{equation*}

2. For any object $A\in\mathrm{Obj}(\mathcal{C})$, there exists an identity morphism $\mathrm{id}_A\in\mathrm{Hom}_{\mathcal{C}}(A,A)$ which can be composite with any morphism $f\in\mathrm{Hom}_{\mathcal{C}}(A,B)$, $g\in\mathrm{Hom}_{\mathcal{C}}(B,A)$ such that the following equation satisfies.
\begin{equation*}
\mathrm{id}_A\circ f = f \quad \mathrm{and}\quad g\circ\mathrm{id}_A =g. 
\end{equation*}
\end{definition}
\subsection{A categroid of admissible leveled shaped pseudo 3-manifolds}
The equivalence class of leveled shaped pseudo 3-manifolds is a morphism of the cobordism category $\mathcal{B}$, where the object is a triangulated surface, and the composition of morphisms is gluing the corresponding parts of the boundary by the CW-homeomorphism that preserves the orientation of the edges and reverses the orientation of the faces, which involves an obvious composition of dihedral angles and a sum of levels. Depending on how the boundary is partitioned, even the same leveled shaped pseudo 3-manifolds can be interpreted as different morphisms of $\mathcal{B}$. However, there is a standard way of partitioning the boundary as described below.

For a tetrahedron $T=[v_0,v_1,v_2,v_3]$ in $\mathbb{R}^3$ with ordered vertices $v_0,v_1,v_2,v_3$, its sign is defined as
\begin{equation*}
{\rm sign}(T)={\rm sign}(\det(v_1-v_0,v_2-v_0,v_3-v_0)),
\end{equation*}
and the sign of the face is defined as 
\begin{equation*}
{\rm sign}(\partial_i T)=(-1)^i{\rm sign}(T),\quad i\in{0,\ldots,3}.
\end{equation*}

For a pseudo 3-manifold $X$, the sign of faces of tetrahedra constituting $X$ induces the following sign function on the faces of the boundary of $X$. 
\begin{equation*}
{\rm sign}_X:\Delta_2(\partial X)\rightarrow \left\{\pm 1\right\}
\end{equation*}

This sign function decomposes the boundary of $X$ into two components $\partial_{-}X$, $\partial_{+}X$ consisting of an equal number of triangles.

In the following (the equivalence class of) a leveled shaped pseudo 3-manifold $X$ is considered as a $\mathcal{B}$-morphism between the objects $\partial_{-}X$ and $\partial_{+}X$. In other words
\begin{equation*}
    X\in{\rm Hom }_{\mathcal{B}}(\partial_{-}X,\partial_{+}X). 
\end{equation*}

The TQFT of Andersen and Kashaev is not defined on the entire $\mathcal{B}$-category, but on the sub-categroid formed by admissible equivalence classes of admissible morphisms.
\begin{definition}
The categroid $\mathcal{B}_{a}$ of admissible leveled shaped pseudo 3-manifolds is a sub-categroid of the category formed by leveled shaped pseudo 3-manifolds, whose morphisms consist of admissible equivalence classes of admissible leveled shaped pseudo 3-manifolds.
The composition map in this sub-categroid is induced from the category $\mathcal{B}$, and
the composable morphisms are
\begin{equation*}
\begin{split}
 K_{A,B,C}^{\mathcal{B}_a}&=
 \{(X_1,X_2)\in\mathrm{Hom}_{\mathcal{B}_a}(A,B)\times \mathrm{Hom}_{\mathcal{B}_a}(B,C)\mid S_r(X_1\circ X_2)\neq\emptyset,\\ &\quad H_2(X_1\circ X_2 -\Delta_0(X_1\circ X_2),\mathbb{Z})=0 \}.     
\end{split}
\end{equation*}
\end{definition}
\subsection{The TQFT functor}
Andersen, Kashaev's TQFT constructs a family of functors $\{F_{\hbar}\}_{\hbar\in\mathbb{R}}$  from a cobordism categroid $\mathcal{B}_a$ consisting of admissible leveled shaped pseudo 3-manifolds to a categroid $\mathcal{D}$ of tempered distributions as described below.

The space $\mathcal{S}^\prime (\mathbb{R}^n)$ of (complex) tempered distributions is the space of continuous linear functionals on the (complex) Schwartz space $\mathcal{S}(\mathbb{R}^n)$.  
By the Schwartz representation theorem, any tempered distribution can be represented by differentiating a continuous function with polynomial growth a finite number of times. Therefore, we can consider a tempered distribution as a function of $\mathbb{R}^n$. For any $\phi\in\mathcal{S}^\prime (\mathbb{R}^n),x\in\mathbb{R}^n$, we denote $\phi(x)\equiv\langle x|\phi\rangle$. This notation should be considered in the usual sense of distributions. For example,
\begin{equation*}
\phi(f)=\int_{\mathbb{R}^n}\phi(x)f(x)dx
\end{equation*}

This formula shows that $\mathcal{S}(\mathbb{R}^n)\subset\mathcal{S}^\prime(\mathbb{R}^n)$.
\begin{definition}
The categroid $\mathcal{D}$ has a finite set as its object, and for two finite sets $n$ and $m$, the set of morphisms from $n$ to $m$ is
 \begin{equation*}
 {\rm Hom}_{\mathcal{D}}(n,m)=\mathcal{S}^\prime(\mathbb{R}^{n\sqcup m}).   
 \end{equation*}
\end{definition}
Let $\mathcal{L}(S(\mathbb{R}^n),\mathcal{S}^\prime(\mathbb{R}^m))$ denote the space of continuous linear maps from $\mathcal{S}(\mathbb{R}^n)$ to $\mathcal{S}^\prime(\mathbb{R}^m)$.

For any $\phi\in L(\mathcal{S}(\mathbb{R}^n)$, $\mathcal{S}^\prime(\mathbb{R}^m)), f\in\mathcal{S}(\mathbb{R}^n)$, $g\in\mathcal{S}(\mathbb{R}^m)$, there exists an isomorphism 
\begin{equation}\label{Schwartz-isom}
\widetilde{\cdot}:\mathcal{L}(\mathcal{S}(\mathbb{R}^n), \mathcal{S}^\prime(\mathbb{R}^m))\rightarrow\mathcal{S}^\prime(\mathbb{R}^{n\sqcup m})
\end{equation}
defined by 
\begin{equation*}
\phi(f)(g)=\widetilde{\phi}(f\otimes g).
\end{equation*}

The composition partially defined in this categroid is defined as follows. Let $n,m,l$ be three finite sets and $A\in{\rm Hom}_{\mathcal{D}}(n,m)$, $B\in{\rm Hom}_{\mathcal{D}}(m,l)$.
By theorem 6.1.2 in \cite{MR1996773}, there exist pullback maps
\begin{equation*}
\pi^{\ast}_{n,m}:\mathcal{S}^\prime(\mathbb{R}^{n\sqcup m})\rightarrow\mathcal{S}^{\prime}(\mathbb{R}^{n\sqcup m\sqcup l}) ,\quad
\pi^{\ast}_{m,l}:\mathcal{S}^\prime(\mathbb{R}^{m\sqcup l})\rightarrow\mathcal{S}^{\prime}(\mathbb{R}^{n\sqcup m\sqcup l}) .
\end{equation*}
By theorem IX.45 in \cite{MR0493420}, 
\begin{equation*}
\pi^\ast_{n,m}(A)\pi^\ast_{m,l}(B)\in\mathcal{S}^\prime(\mathbb{R}^{n\sqcup m \sqcup l})
\end{equation*}
is well-defined if $\pi_{n,m}^\ast(A)$ and $\pi_{m,l}^\ast(B)$ satisfy the following transversality condition.
\begin{equation}\label{wave_front_set}
(WF(\pi_{n,m}^\ast(A))\oplus WF(\pi_{m,l}^\ast(B)))\cap Z_{n\sqcup m\sqcup l}=\emptyset,
\end{equation}
where $Z_{n\sqcup m\sqcup l}$ is the zero section of $T^\ast(\mathbb{R}^{n\sqcup m\sqcup l})$. Furthermore, if $\pi^\ast_{n,m}(A)\pi^\ast_{m,l}(B)$ extends continuously to $\mathcal{S}^\prime(\mathbb{R}^{n\sqcup m\sqcup l })_{m}$, we obtain the following well-defined element.
\begin{equation*}
(\pi_{n,l})_\ast(\pi^\ast_{n,m}(A)\pi^\ast_{m,l}(B))\in\mathcal{S}^\prime(\mathbb{R}^{n \sqcup l}),
\end{equation*}
where $\mathcal{S}^\prime(\mathbb{R}^n)_m$ is defined as follows.
\begin{definition}
$\mathcal{S}(\mathbb{R}^n)_m$ is the set of all $\phi\in C^\infty(\mathbb{R}^n)$ such that  
\begin{equation*}
\sup_{x\in\mathbb{R}^n}|x^\beta\partial^\alpha(\phi)(x)|<\infty
\end{equation*} 
for all multi indices $\alpha,\beta$ such that for $n-m<i\le n$ if $\alpha_i=0$, then $\beta_i=0$. 
Also, $\mathcal{S}^\prime(\mathbb{R}^n)_m$ is a continuous dual of $\mathcal{S}(\mathbb{R}^n)_m$ regarding these seminorms.  
\end{definition}
\begin{definition}
If $A\in{\rm Hom}_D(n,m)$ and $B\in{\rm Hom}_D(m,l)$ satisfy the 
equation (\ref{wave_front_set}) and $\pi^\ast_{n,m}(A)\pi^\ast_{m,l}(B)$ continuously extends to $\mathcal{S}^\prime (\mathbb{R}^{n\sqcup m\sqcup l})_m$, we define 
\begin{equation*}
AB=(\pi_{n,l})_\ast(\pi_{n,m}^\ast(A)\pi_{m,l}^\ast(B))\in{\rm Hom}_{\mathcal{D}}(n,l).
\end{equation*}
\end{definition}
Therefore, the composable morphisms in the categroid $\mathcal{D}$ are 
\begin{equation*}
\begin{split}
K_{n,m,l}^{\mathcal{D}}=&\{(A,B)\in\mathcal{S}^\prime(\mathbb{R}^{n\sqcup m})\times\mathcal{S}^\prime(\mathbb{R}^{m\sqcup l})\mid (\mathrm{WF}(\pi_{n,m}^\ast(A))\oplus WF(\pi_{m,l}^\ast(B)))\cap Z_{n\sqcup m\sqcup l} \\
& =\emptyset, \pi^\ast_{n,m}(A)\pi^\ast_{m,l}(B)\in\mathcal{S}^\prime(\mathbb{R}^{n\sqcup m\sqcup l })_{m}\}.
\end{split}
\end{equation*}

For any $A\in\mathcal{L}(\mathcal{S}(\mathbb{R}^n),\mathcal{S}^\prime(\mathbb{R}^m))$, there 
exists an unique adjoint $A^\ast\in\mathcal{L}(\mathcal{S}(\mathbb{R}^m),\mathcal{S}^\prime(\mathbb{R}^n))$ defined as follows.
For any $f\in\mathcal{S}(\mathbb{R}^m), g\in\mathcal{S}(\mathbb{R}^n)$, 
\begin{equation*}
A^\ast(f)(g)=\overline{A(\overline{g})(\overline{f})}.
\end{equation*}
\begin{definition}
The functor $F: \mathcal{B}_a\rightarrow D$ is called 
a $\ast-$functor if 
\begin{equation*}
F(X^\ast)=F(X)^\ast, 
\end{equation*}
where $X^\ast$ is $X$ with an opposite orientation to $X$, and  $F(X)^\ast$ is the adjoint of $F(X)$.
\end{definition}
Faddeev's quantum dilogarithm plays an important role in constructing 
the functor of Andersen-Kashaev's TQFT.
\begin{definition}
Faddeev's quantum dilogarithm \cite{MR1345554} is a function whose arguments are two complex variables $z$ and $\mathsf{b}$ defined for $| \Im z |<\frac{1}{2}| \mathsf{b}+\mathsf{b}^{-1}|$, 
\begin{equation*}
\Phi_{\mathsf{b}}(z)\coloneqq\exp{\left(\int_{C}\frac{e^{-2izw}dw}{4\sinh{w{\mathsf{b}}}\sinh{(\frac{w}{\mathsf{b}})}w}\right)},
\end{equation*}
where $C$ runs along the real axis and deviates into the upper half-plane in the neighborhood of the origin, and it extends to the meromorphic function on $z\in\mathbb{C}$ by the functional equation
\begin{equation*}
\Phi_{\mathsf{b}}(z-i{\mathsf{b}}^{\pm 1}/2)=(1+e^{2\pi {\mathsf{b}}^{\pm 1}z})\Phi_{\mathsf{b}}(z+i{\mathsf{b}}^{\pm 1}/2).
\end{equation*}
\end{definition}
$\Phi_{\mathsf{b}}(z)$ depends on ${\mathsf{b}}$ only through $\hbar$
defined by
\begin{equation*}
\hbar\coloneqq({\mathsf{b}+\mathsf{b}^{-1}})^{-2}.
\end{equation*}
$\Phi_{\mathsf{b}}$ has the following properties.
\begin{prop}[ \cite{MR3227503}, Appendix A ]\label{property-of-Phi_b}
(1) 
For any ${\mathsf{b}}\in\mathbb{R}_{>0}$, for any $z\in\mathbb{R}+i\left(\frac{-1}{2\sqrt{\hbar}},\frac{1}{2\sqrt{\hbar}}\right)$, 
\begin{equation*}
\Phi_{\mathsf{b}}(z)\Phi_{\mathsf{b}}(-z)=e^{i\frac{\pi}{12}(\mathsf{b}^2+\mathsf{b}^{-2})}e^{i\pi z^2}.  \end{equation*}
(2) 
For any $\mathsf{b}\in\mathbb{R}_{>0}$, for any $z\in\mathbb{R}+i\left(\frac{-1}{2\sqrt{\hbar}},\frac{1}{2\sqrt{\hbar}}\right)$,
\begin{equation*}
\overline{\Phi_{\mathsf{b}}(z)}=\frac{1}{\Phi_{\mathsf{b}}(\overline{z})}.
\end{equation*}
(3) (Behavior at infinity) For any $\mathsf{b}\in\mathbb{R}_{>0}$,
\begin{equation*}
\begin{split}
\Phi_{\mathsf{b}}(z)&\underset{\Re(z)\rightarrow-\infty}{\sim}1,\\
\Phi_{\mathsf{b}}(z)&\underset{\Re(z)\rightarrow\infty}{\sim}e^{i\frac{\pi}{12}(\mathsf{b}^2+\mathsf{b}^{-2})}e^{i\pi z^2}.
\end{split}
\end{equation*}

In particular, for any $\mathsf{b}\in\mathbb{R}_{>0}$, for any 
$d\in\left(\frac{-1}{2\sqrt{\hbar}},\frac{1}{2\sqrt{\hbar}}\right)$,
\begin{equation*}
\begin{split}
\left|\Phi_{\mathsf{b}}(x+id)\right|&\underset{\mathbb{R}\ni x\rightarrow-\infty}{\sim}1,\\
\left|\Phi_{\mathsf{b}}(x+id)\right|&\underset{\mathbb
{R}\ni x\rightarrow +\infty}{\sim}e^{-2\pi xd}.
\end{split}
\end{equation*}
\end{prop}
\begin{theorem}[Andersen, Kashaev]
For any $\hbar\in\mathbb{R}_{+}$, there exists a unique $\ast-$ functor satisfying the following conditions. 
For any $A\in{\rm Ob}(\mathcal{B}_a)$, $F_\hbar(A)=\Delta_2(A)$, and
for any admissible leveled shaped pseudo 3-manifold $(X, l_X)$, the corresponding morphism in $\mathcal{D}$ is
\begin{equation*}
F_\hbar(X,l_X)=Z_\hbar(X)e^{i\pi \frac{l_X}{4\hbar}}\in\mathcal{S}^\prime(\mathbb{R}^{\Delta_2(\partial X)}),
\end{equation*}
where $Z_\hbar(X)$ is defined such that for a tetrahedron $T$ with ${\rm sign}(T)=1$
\begin{equation*}
Z_\hbar(T)(x)=\delta(x_0+x_2-x_1)\frac{\exp\left(2\pi i(x_3-x_2)(x_0+\frac{\alpha_3}{2i\sqrt{\hbar}}
)+\pi i\frac{\varphi_T}{4\hbar}\right)}{\Phi_{\mathsf{b}}\left(x_3-x_2+\frac{1-\alpha_1}{2i\sqrt{\hbar}}\right)}.
\end{equation*}
Here $\delta(t)$ is Dirac's delta function and
\begin{equation*}
\varphi_T\coloneqq\alpha_1\alpha_3+\frac{\alpha_1-\alpha_3}{3}-\frac{2\hbar+1}{6},\quad \alpha_i\coloneqq\frac{1}{\pi }\alpha_T(\partial_0\partial_i T),\quad i\in\{1,2,3\},
\end{equation*}
\begin{equation*}
x_i\coloneqq x(\partial_i(T)),\quad x:\Delta_2(\partial T)\to\mathbb{R}.
\end{equation*}
\end{theorem}
$Z_\hbar(X)$ and $Z_\hbar(T)$ are called partition functions and described in detail in Subsection \ref{partition-func-of-tetrahedron}.
For an admissible pseudo 3-manifold $X$, Andersen-Kashaev's TQFT functor gives the following well-defined function, 
\begin{equation*}
F_\hbar : LS_r(X)\rightarrow \mathcal{S}^\prime (\mathbb{R}^{\partial X}).
\end{equation*}

If $\partial X=\emptyset$, then $\mathcal{S}^\prime(\mathbb{R}^{\partial X})=\mathbb{C}$, and we obtain a complex-valued function on $LS_r(X)$.

In particular, the value of the functor $F_\hbar$ on any fully balanced admissible leveled shaped 3-manifold is a complex number, and it is a topological invariant in the sense that if two fully balanced admissible leveled shaped 3-manifolds are admissibly equivalent, $F_\hbar$ assigns the same complex number to them. 
\subsection{Tetrahedral operators of quantum Teichm\"{u}ller theory}
The operators $\mathsf{q}_i$ and $\mathsf{p}_i$ which act on $\mathcal{S}(\mathbb{R}^n)$ are defined as follows. For any $f\in\mathcal{S}(\mathbb{R}^n)$,
\begin{equation*}
\mathsf{q}_i(f)(t)=t_if(t),\quad \mathsf{p}_i(f)(t)=\frac{1}{2\pi i}\frac{\partial}{\partial t_i}(f)(t),\quad \forall t\in\mathbb{R}^n.
\end{equation*}
These operators are known to extend continuously to $\mathcal{S}^\prime(\mathbb{R}^n)$ and satisfy the following Heisenberg's commutation relations,
\begin{equation*}
[\mathsf{p}_i,\mathsf{p}_j]=[\mathsf{q}_i,\mathsf{q}_j]=0,\quad [\mathsf{p}_i,\mathsf{q}_j]=\frac{1}{2\pi i}\delta_{i,j}.
\end{equation*}

Fix ${\rm{b}}\in\mathbb{C}$ such that ${\Re}({\mathsf{b}})\neq 0$. 
By the spectral theorem, the following operators
\begin{equation*}
\mathsf{u}_i=e^{2\pi\mathsf{bq}_i},\quad\mathsf{v}_i=e^{2\pi\mathsf{bp}_i}
\end{equation*}
can be defined.

The commutation relations between $\mathsf{u}_i$ and $\mathsf{v}_j$ are
\begin{equation*}
[\mathsf{u}_i,\mathsf{u}_j]=[\mathsf{v}_i,\mathsf{v}_j]=0,\quad \mathsf{u}_i\mathsf{v}_j=e^{i2\pi\mathsf{b}^2\delta_{i,j}}\mathsf{v}_j\mathsf{u}_i.
\end{equation*}

According to \cite{MR1607296}, consider the following operations on $\vec{\mathsf{w}}_i=(\mathsf{u}_i,\mathsf{v}_i), i=1,2$.
\begin{equation*}
\vec{\mathsf{w}}_1\cdot\vec{\mathsf{w}}_2\coloneqq(\mathsf{u}_1\mathsf{u}_2, \mathsf{u}_1\mathsf{v}_2+\mathsf{v}_1)
\end{equation*}
\begin{equation*}
\vec{\mathsf{w}}_1\ast\vec{\mathsf{w}}_2\coloneqq(\mathsf{v}_1\mathsf{u}_2(\mathsf{u}_1\mathsf{v}_2+\mathsf{v}_1)^{-1},\mathsf{v}_2(\mathsf{u}_1\mathsf{v}_2+\mathsf{v}_1)^{-1})
\end{equation*}
\begin{theorem}[\cite{MR1607296}]
Let $\psi(z)$ be a solution of the following functional equation
\begin{equation}\label{functional-equation}
\psi\left(z+\frac{i{\mathsf{b}}}{2}\right)=\psi\left(z-\frac{i\mathsf{b}}{2}\right)(1+e^{2\pi\mathsf{b}z}),\quad z\in\mathbb{C}.
\end{equation}
Then the operator 
\begin{equation}\label{T-operator}
\mathsf{T}=\mathsf{T}_{12}\coloneqq e^{2\pi i\mathsf{p}_1\mathsf{q}_2}\psi(\mathsf{q}_1+\mathsf{p}_2-\mathsf{q}_2)=\psi(\mathsf{q}_1-\mathsf{p}_1+\mathsf{p}_2)e^{2\pi i\mathsf{p}_1\mathsf{q}_2}
\end{equation}
is an element of $\mathcal{L}(\mathcal{S}(\mathbb{R}^4),\mathcal{S}(\mathbb{R}^4))$, and satisfies the following equations,
\begin{equation*}
\vec{\mathsf{w}}_1\cdot\vec{\mathsf{w}}_2\mathsf{T}=\mathsf{T}\vec{\mathsf{w}}_1,\quad
\vec{\mathsf{w}}_1\ast\vec{\mathsf{w}}_2\mathsf{T}=\mathsf{T}\vec{\mathsf{w}}_2.
\end{equation*}
\end{theorem}
One solution of the equation (\ref{functional-equation}) is given by Faddeev's quantum dilogarithm as follows.
\begin{equation*}
\psi(z)=\overline{\Phi}_\mathsf{b}(z)\coloneqq\frac{1}{\Phi_{\mathsf{b}}(z)}.
\end{equation*}
In the following sections, $\mathsf{b}$ is taken so that 
\begin{equation*}
\hbar\coloneqq\left(\mathsf{b}+\mathsf{b}^{-1}\right)^{-2}\in\mathbb{R}_+ .
\end{equation*}

\subsection{Charged tetrahedral operators}
For any positive real numbers $a$ and $c$ such that $b\coloneqq\frac{1}{2}-a-c$ is also positive, the charged $\mathsf{T}-$ operators are defined as 
\begin{equation*}
\mathsf{T}(a,c)\coloneqq e^{-\pi ic_{\mathsf{b}}^2\frac{4(a-c)+1}{6}}e^{4\pi ic_{\mathsf{b}}(c\mathsf{q}_2-a\mathsf{q}_1)}\mathsf{T}e^{-4\pi ic_{\mathsf{b}}(a\mathsf{p}_2+c\mathsf{q}_2)}
\end{equation*}
\begin{equation*}
\overline{\mathsf{T}}(a,c)\coloneqq e^{\pi ic_{\mathsf{b}}^2\frac{4(a-c)+1}{6}}e^{-4\pi ic_{\mathsf{b}}(a\mathsf{p}_2+c\mathsf{q}_2)}\overline{\mathsf{T}}e^{4\pi ic_{\mathsf{b}}(c\mathsf{q}_2-a\mathsf{q}_1)},
\end{equation*}
where $\overline{\mathsf{T}}\coloneqq\mathsf{T}^{-1}$ and  
\begin{equation*}
c_{\mathsf{b}}\coloneqq i\frac{(\mathsf{b}+\mathsf{b}^{-1})}{2}.
\end{equation*}
Then $\mathsf{T}(a,c):\mathcal{S}(\mathbb{R}^2)\rightarrow\mathcal{S}(\mathbb{R}^2)$ and $\overline{\mathsf{T}}(a,c):\mathcal{S}({\mathbb{R}^2})\rightarrow\mathcal{S}(\mathbb{R}^2)$.
Substituting the equation (\ref{T-operator}), we obtain
\begin{equation*}
\mathsf{T}(a,c)=e^{2\pi i\mathsf{p}_1\mathsf{q}_2}\psi_{a,c}(\mathsf{q}_1-\mathsf{q}_2+\mathsf{p}_2),
\end{equation*}
where 
\begin{equation*}
\psi_{a,c}(x)\coloneqq\psi(x-2c_{\mathsf{b}}(a+c))e^{-4\pi ic_{\mathsf{b}}a(x-c_{\mathsf{b}}(a+c))}e^{-\pi ic_{\mathsf{b}}^2\frac{4(a-c)+1}{6}}.
\end{equation*}
The following formula holds for $\mathsf{T}(a,c)\in\mathcal{S}^\prime(\mathbb{R}^4)$.
\begin{equation*}
\langle x_0,x_2|\mathsf{T}(a,c)|x_1,x_3\rangle =\delta(x_0+x_2-x_1)\widetilde{\psi}_{a,c}^\prime(x_3-x_2)e^{2\pi ix_0(x_3-x_2)},
\end{equation*}
where
\begin{equation*}
\widetilde{\psi}_{a,c}^\prime(x)\coloneqq e^{-\pi ix^2}\widetilde{\psi}_{a,c}(x),\quad \widetilde{\psi}_{a,c}(x)\coloneqq \int_{\mathbb{R}}\psi_{a,c}(y)e^{-2\pi ixy}dy.
\end{equation*}

The conditions imposed on $a$ and $c$ guarantee that the Fourier integral is absolutely convergent.

The Fourier transformation formula for Faddeev's quantum dilogarithm leads to the following identity.
\begin{equation*}
\widetilde{\psi}_{a,c}^\prime(x)=e^{-\frac{\pi i}{12}}\psi_{c,b}(x).
\end{equation*}
Concerning complex conjugation, 
\begin{equation*}
\overline{\psi_{a,c}(x)}=e^{-\frac{\pi i}{6}}e^{\pi ix^2}\psi_{c,a}(-x)=e^{-\frac{\pi i}{12}}\widetilde{\psi}_{b,c}(-x).
\end{equation*}
In combination with these, 
\begin{equation*}
\overline{\widetilde{\psi}^\prime_{a,c}(x)}=e^{\frac{\pi i}{12}}\overline{\psi_{c,b}(x)}=e^{-\frac{\pi i}{12}}e^{\pi ix^2}\psi_{b,c}(-x).
\end{equation*}

From the above, the following formula for $\overline{\mathsf{T}}(a,c)$ is obtained, 
\begin{eqnarray*}
\langle x,y|\overline{\mathsf{T}}(a,c)|u,v\rangle&=&\overline{\langle u,v|\mathsf{T}(a,c)|x,y\rangle}\\
&=&\delta(u+v-x)\overline{\widetilde{\psi}_{a,c}^\prime(y-v)}e^{-2\pi iu(y-v)}\\
&=&\delta(u+v-x)\psi_{b,c}(v-y)e^{-\frac{\pi i}{12}}e^{\pi i(v-y)^2}e^{-2\pi iu(y-v)}.
\end{eqnarray*}
\subsection{The partition function of a shaped tetrahedron}
\label{partition-func-of-tetrahedron}
Let $T$ be a shaped tetrahedron with ordered vertices $v_i\ ( i=0,1,2,3 )$ in $\mathbb{R}^3$.

We define the partition function 
$Z_\hbar(T)\in\mathcal{S}^\prime(\mathbb{R}^{\Delta_2(\partial T)})$
by the following formula.
\begin{equation}
\langle x|Z_\hbar(T)\rangle=\left\{
  \begin{aligned}
  &\langle x_0,x_2\lvert \mathsf{T}\left(c\left(v_0v_1\right),c\left(v_0v_3\right)\right)\rvert x_1,x_3\rangle\quad\quad {\rm{if}}\quad {\rm sign}(T)=1\\
  &\langle x_1,x_3\lvert \overline{\mathsf{T}}\left(c\left(v_0v_1\right),c\left(v_0v_3\right)\right)\rvert x_0,x_2\rangle\quad\quad {\rm{if}}\quad {\rm sign}(T)=-1
  \end{aligned}
\right., 
\end{equation}
where
\[
x_i=x(\partial _i T),\quad i\in\{0,1,2,3\}
\]
\[
c\coloneqq\frac{1}{2\pi}\alpha_T:\Delta_1(T)\rightarrow\mathbb{R}_+.
\]
\subsection{One vertex H-triangulation}
Let $(M,K)$ be a pair of oriented closed 3-manifold $M$ and a knot $K$ in $M$. A one vertex H-triangulation of $(M,K)$ is a tetrahedral decomposition of $M$ with one vertex and which has one edge representing the knot $K$. If $M=S^3$, we can obtain an example of one vertex H-triangulation from the projective diagram of $K$.  

\subsection{Notations and conditions}
We denote an oriented triangulated pseudo 3-manifold $X$ which consists of $N$ tetrahedra with ordered vertices $0,1,2,3$ by $X=(T_1, \ldots, T_N, \sim)$, where $\sim$ means the equivalence relation generated by gluing between faces which preserve the orientation of edges.

Let $\mathscr{S}_X$ denote the set corresponding to the shape structure $S(X)$ on $X$ defined by 
\begin{equation*}
\begin{split}
\mathscr{S}_X\coloneqq & \left\{ \alpha_X=(2\pi a_1,2\pi b_1,2\pi c_1,\ldots,2\pi a_N,2\pi b_N,2\pi c_N)\in(0,\pi)^{3N}\mid\right.\\
& \quad \left.\forall k\in\{1,\ldots,N\},a_k+b_k+c_k=\frac{1}{2}\right\},
\end{split}
\end{equation*}
 where $2\pi a_k$ (respectively, $2\pi b_k,2\pi c_k$) denotes the value of the dihedral angle on $\overrightarrow{01}$ (respectively, $\overrightarrow{02},\ \overrightarrow{03}$) and its opposite edge of a tetrahedron $T_k$. $\mathscr{S}_X$ is also called a shape structure.

Let $\overline{\mathscr{S}_X}$ denote the closure of $\mathscr{S}_X$ such that $a_k,b_k$, and $c_k$ have values in $[0,\frac{1}{2}]$, and the weight function is defined in an extended way. 

A fully balanced shape structure on $X$ is called an angle structure and is defined by $\mathscr{A}_X\coloneqq\{\alpha_X\in \mathscr{S}_X\mid
\forall e\in \Delta_1(X), \omega_{X}(e)=2\pi\}$ and the extended angle structure on $X$ is defined by
$\overline{\mathscr{A}_X}\coloneqq\{\alpha_X\in \overline{\mathscr{S}_X}\mid
\forall e\in \Delta_1(X), \omega_{X}(e)=2\pi\}$.

A shaped pseudo 3-manifold can be represented as a pair $(X,\alpha_X)$ of an underlying oriented triangulated pseudo 3-manifold $X$ and a shape structure $\alpha_X\in\mathscr{S}_X$. $\omega_{X,\alpha_X}$ denotes the weight function on $(X,\alpha_X)$. 

This paper discusses under the condition $\mathsf{b}\in\mathbb{R}_{>0}$.

\subsection{The conjecture analogous to the volume conjecture}
The following conjecture has been proposed.
\begin{conjecture}[Andersen, Kashaev]\label{Andersen-Kashaev}
Let $M$ be an oriented compact closed 3-manifold. For any hyperbolic knot $K$ in $M$, there exists a smooth function $J_{M,K}(\hbar,x)$ on $\mathbb{R}_{> 0}\times\mathbb{R}$ satisfying the following properties.

(1) For any fully balanced shaped ideal triangulation $X$ of the complementary space of $K$ in $M$, there exists a real linear combination $\lambda$ of gauge-invariant dihedral angles and a real second-order polynomial $\phi$ of (not necessarily gauge-invariant) dihedral angles such that 
\[
Z_{\hbar}(X)=e^{i\frac{\phi}{\hbar}}\int_{\mathbb{R}}J_{M,K}(\hbar,x)e^{-\frac{x\lambda}{\sqrt{\hbar}}}dx.
\]

(2) For any one vertex shaped H-triangulation $Y$ of the pair $(M,K)$, there exists a real second-order polynomial $\varphi$ of dihedral angles such that
\[
\lim_{\omega_{Y}\rightarrow\tau}\Phi_{\rm{b}}\left(\frac{\pi-\omega_{Y}(K)}{2\pi i\sqrt{\hbar}}\right)Z_{\hbar}(Y)=e^{i\frac{\varphi}{\hbar}-i\frac{\pi}{12}}J_{M,K}(\hbar,0),
\]
where $\tau:\Delta_1(Y)\rightarrow\mathbb{R}$ takes the value $0$ on the edge representing the knot $K$ and $2\pi$ on all other edges. 

(3) The hyperbolic volume of the complementary space of $K$ in $M$ is obtained by the following limit,
\[
\lim_{\hbar\rightarrow 0}2\pi\hbar\log|J_{M,K}(\hbar,0)|=-{\rm Vol}(M\backslash K).
\]
\end{conjecture}

Concerning this conjecture, the $(S^3,4_1)$ and $(S^3,5_2)$ cases are proven for a certain triangulation \cite{MR3227503}. 
A similar reformulated theorem is also proven for all twist knots under the condition $\mathsf{b}>0$ and with the specific triangulation \cite{MR3945172}.
In this paper, we aim to prove this conjecture in the case of the pair of $S^3$ and $7_3$ knot, which has yet to proven in previous studies, using a similar method to the proof of \cite{MR3945172}. 
Results similar to (1) and (2) regarding the form of the partition function are confirmed for a specific tetrahedral decomposition and its equivalence class by the calculation of partition functions in Section \ref{ideal-triangulation} and \ref{one-vertex_Htriangulation}. 
As for (3), we prove it by rigorously evaluating the integral using a similar method to the proof of \cite{MR3945172} for twist knots, and we describe the proof in Section \ref{volume-conjecture-strict-proof}.
In other words, the main theorem of this paper is as follows.
\begin{theorem}\label{uemura}
For the hyperbolic knot $7_3$ in $S^3$, there exists a function $J_{S^3,7_3}(\hbar,x)$ on $\mathbb{R}_{> 0}\times\mathbb{C}$ satisfying the following properties. 

(1) For an ideal tetrahedral decomposition $X$ of the complementary space of $7_3$ knot in $S^3$ and for any angle structure $\alpha_X\in\mathscr{A}_X$ on $X$, there exist gauge invariant real linear combinations of dihedral angles $\lambda(\alpha_X)$, $\mu(\alpha_X)$, and a (not necessarily gauge invariant) quadratic polynomial of dihedral angles $\phi(\alpha_X)$ such that  
\[
Z_{\hbar}(X,\alpha_X)=e^{i\frac{\phi(\alpha_X)}{\hbar}}\int_{\mathbb{R}+\frac{i\mu(\alpha_X)}{\sqrt{\hbar}}}J_{S^3,7_3}(\hbar,x)e^{-\frac{x\lambda(\alpha_X)}{\sqrt{\hbar}}}dx.
\]

(2) For a one vertex H-triangulation $Y$ of $(S^3,7_3)$, let $Z$ be a tetrahedron containing an edge $\Vec{K}$ representing $7_3$ knot. For all $\mathsf{b}>0$, all $\tau\in\overline{\mathscr{S}_Z}\times\mathscr{S}_{Y\backslash Z}$ such that $\omega_{Y,\tau}$ takes the value $0$ on the edge $\Vec{K}$ and $2\pi$ on the other edges, there exists a real quadratic polynomial $\varphi(\tau)$ of dihedral angles such that 

\[
\lim_{\alpha_Y\rightarrow\tau,\alpha_Y\in\mathscr{S}_Y}\Phi_{\mathsf{b}}\left(\frac{\pi-\omega_{Y,\alpha_Y}(\Vec{K})}{2\pi i\sqrt{\hbar}}\right)Z_{\hbar}(Y,\alpha_Y)=e^{i\frac{\varphi(\tau)}{\hbar}-i\frac{\pi}{12}}J_{S^3,7_3}(\hbar,0).
\]

(3) The following limit obtains the the hyperbolic volume of the complementary space of $7_3$ knot in $S^3$,
\[
\lim_{\hbar\rightarrow 0^+}2\pi\hbar\log|J_{S^3,7_3}(\hbar,0)|=-{\rm Vol}(S^3\backslash 7_3).
\]
\end{theorem}
We note that we choose the contour of the integral in Theorem\ref{uemura} (1) different from the Conjecture \ref{Andersen-Kashaev} (1) in order to evaluate the integral rigorously. 
Regarding Theorem\ref{uemura} (3), we can also numerically confirm its establishment by evaluating the integral using the saddle-point method shown in Appendix \ref{volume-conjecture}.