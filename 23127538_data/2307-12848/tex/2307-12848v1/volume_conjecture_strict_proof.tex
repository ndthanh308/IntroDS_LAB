\section{Proof of Theorem \ref{uemura}}
\label{volume-conjecture-strict-proof}
In this section, we describe the proof of Theorem \ref{uemura}, referring to the policy of the proof for twist knots \cite{MR3945172}.

By Theorem \ref{volume-conjecture-1}, and Theorem \ref{volume-conjecture-2},
let
\begin{equation*}
J_{S^3,7_3}\coloneqq e^{-\frac{i\pi}{3}}J_{X}, 
\end{equation*}
then it is shown that Theorem \ref{uemura} (1) and (2) hold.

Furthermore, transformation of variables ${\sf{x}}=2\pi\sqrt{\hbar}x$, $Y=2\pi\sqrt{\hbar}Y^\prime$, $Z=2\pi\sqrt{\hbar}Z^\prime$,
$W=2\pi\sqrt{\hbar}W^\prime$ gives
\begin{equation*}
\begin{split}
Z_\hbar(X,\alpha_X)&=e^{-\frac{\pi}{3}i}e^{\frac{i}{3}\pi c_{\rm{b}}^2}\times\mbox{\textcircled{\scriptsize 1}}\times
\int_{\mathbb{R}+2\pi i\mu(\alpha_X)}\mathfrak{J}_X(\hbar,\mathsf{x})e^{-\frac{\mathsf{x}}{2\pi\hbar}\lambda(\alpha_X)}d\mathsf{x},
\end{split}
\end{equation*}
where we define
\begin{equation*}
\mathbf{y}\coloneqq\begin{bmatrix}
Y\\
Z\\
W
\end{bmatrix}
\end{equation*}
and 
\begin{equation*}
\begin{split}
\mathfrak{J}_X:(\hbar,\mathsf{x})\mapsto\left(\frac{1}{2\pi\sqrt{\hbar}}\right)^4\int_{\mathscr{Y}_{\alpha_X}}d\mathbf{y}e^{\frac{i\mathbf{y}^{\rm T}Q\mathbf{y}-\frac{3}{2}i\mathsf{x}^2+\mathbf{y}^{\rm T}\mathscr{W}}{2\pi\hbar}}\frac{\Phi_{\mathsf{b}}(\frac{Z}{2\pi\sqrt{\hbar}})}{\Phi_{\mathsf{b}}(\frac{Y}{2\pi\sqrt{\hbar}})\Phi_{\mathsf{b}}(\frac{W}{2\pi\sqrt{\hbar}})\Phi_{\mathsf{b}}(\frac{W-\mathsf{x}}{2\pi\sqrt{\hbar}})\Phi_{\mathsf{b}}(\frac{W+\mathsf{x}}{2\pi\sqrt{\hbar}})}. 
\end{split}
\end{equation*}
Here the integral domain is 
\begin{equation*}
\mathscr{Y}_{\alpha_X}\coloneqq (\mathbb{R}-i\pi(1-2a_1))\times (\mathbb{R}+i\pi(1-2a_2))\times(\mathbb{R}-i\pi(1-2a_3)).
\end{equation*}
Then 
\begin{equation}\label{J-relation2}
J_X(\hbar,x)=2\pi\sqrt{\hbar}\mathfrak{J}_X(\hbar,(2\pi\sqrt{\hbar})x)
\end{equation}
holds.
By the equation \eqref{J-relation2}, Theorem \ref{uemura} (3) holds by proving the following equation.
\begin{equation*}
\lim_{\hbar\rightarrow 0^{+}}2\pi\hbar \log |J_X(\hbar,0)|=\lim_{\hbar\rightarrow 0^{+}}2\pi\hbar\log|\mathfrak{J}_X(\hbar,0)|=-\mathrm{Vol}(S^3\backslash 7_3)
\end{equation*}

We prove this below.

First, we discuss the geometricity of the ideal tetrahedral decomposition, i.e., the property that positive dihedral angles corresponding to the complete hyperbolic structure of the underlying $3$-dimensional hyperbolic manifold can be assigned to these tetrahedra.

In \cite{Thurston}, Thurston gives a method to check whether a given tetrahedral decomposition has geometricity. A system of gluing equations with respect to the complex parameters associated with the tetrahedra is defined there, and it is shown that if this system has a solution, the solution is unique and corresponds to the complete hyperbolic metric of the tetrahedrally decomposed manifold.

However, this system of equations is difficult to solve. In the 1990s, Casson and Rivin devised another method to prove geometricity (for a review paper, see
\cite{MR2796632}). The idea is to focus on the argument part of the complex gluing equation and to use the volume functional property.

For the ideal tetrahedral decomposition $X$ of $S^3\backslash 7^3$ obtained in Section \ref{ideal-triangulation}, we prove the following theorem.
\begin{theorem}\label{geometric}
$X$ is geometric.
\end{theorem}

To prove Theorem \ref{geometric}, we use the method by Futer and Gu\'{e}ritaud (\cite{MR2796632, MR2255497}).
First, we show that the volume functional is not maximal on the boundary of the space of the extended angle structures, which is non-empty by Lemma \ref{non-empty}.
Then Theorem \ref{geometric} can be proven using the result by Casson and Rivin (Theorem \ref{Casson-Rivin}).
\subsection{The volume functional}
Lobachevsky function $\Lambda :\mathbb{R}\rightarrow \mathbb{R}$ is given by
\begin{equation*}
\Lambda(x)=-\int_0^x \log |2\sin t|dt.
\end{equation*}
It is well-defined and continuous on $\mathbb{R}$ with period $\pi$.
If $T$ is a hyperbolic ideal tetrahedron with dihedral angles $a$, $b$, $c$, then its volume is
\begin{equation*}
{\rm Vol}(T)=\Lambda(a)+\Lambda(b)+\Lambda(c).
\end{equation*}

Let $X= ( T_1, \ldots, T_N, \sim )$ be an ideal tetrahedral decomposition and $\mathscr{A}_X$ be the space of its angle structures, then it is a (possibly empty) convex polyhedron in $\mathbb{R}^{3N}$.

The volume functional $\mathscr{V}:\overline{\mathscr{A}_X}\rightarrow \mathbb{R}$ is defined by assigning to (the extended) angle structure $\alpha_X=(2\pi a_1,2\pi b_1,2\pi c_1,\ldots,2\pi a_N,2\pi b_N,2\pi c_N)$ a real number 
\begin{equation*}
\mathscr{V}(\alpha_X)=\Lambda(2\pi a_1)+\Lambda(2\pi b_1)+\Lambda(2\pi c_1)+\cdots+\Lambda(2\pi a_N)+\Lambda(2\pi b_N)+\Lambda(2\pi c_N).    
\end{equation*}

From \cite[Theorem 6.1, 6.6]{MR2255497} and \cite[Lemma 5.3]{MR2796632},
it is known that $\mathscr{V}$ is strictly concave on $\mathscr{A}_X$ and concave on $\overline{\mathscr{A}_X}$.
Furthermore, the following theorem holds as stated in \cite[Theorem 1.2]{MR2796632}.
\begin{theorem}[Casson-Rivin \cite{MR1283870}]\label{Casson-Rivin}
Let $M$ be an orientable 3-manifold whose boundaries consist of tori, and let $X$ be an ideal tetrahedral decomposition of $M$. Then, an angle structure $\alpha$ corresponds to the unique complete hyperbolic metric on the interior of $M$ if and only if $\alpha$ is a critical point of the volume functional.
\end{theorem}
When the last situation of this theorem holds, we say that the ideal tetrahedral decomposition $X$ of the $3$-manifold $M$ is geometric.
\subsection{Thurston's complex gluing equation}\label{complex-gluing-equation}
There exists a bijective map from shape structures $\{(2\pi a,2\pi b,2\pi c)\in(0,\pi)^3\mid a+b+c=\frac{1}{2}\}$ on a vertex ordered tetrahedron $T$ to $\mathbb{R}+i\mathbb{R}_{>0}$.
By this map, if $(2\pi a,2\pi b,2\pi c)\in\mathscr{S}_T$ maps to $z\in\mathbb{R}+i\mathbb{R}_{>0}$, and we call $z$ a complex shape structure.

In addition, we define
\begin{equation*}
z^\prime\coloneqq \frac{1}{1-z},\quad z^{\prime\prime}\coloneqq \frac{z-1}{z}.
\end{equation*}

The complex shape structure $z$ corresponds to the edge with a dihedral angle of $2\pi a$. Furthermore, let $\epsilon(T)$ be the sign of a tetrahedron $T$, then if $\epsilon(T)=1$, $z^\prime$ corresponds to $2\pi c$ and $z^{\prime\prime}$ corresponds to $2\pi b$. If $\epsilon(T)=-1$, $z^\prime$ corresponds to $2\pi b$ and $z^{\prime\prime}$ corresponds to $2\pi c$.

In this section, we define the complex logarithmic function as for $z\in\mathbb{C}^\ast$,
\begin{equation*}
\mathrm{Log}(z)\coloneqq\log |z|+i\arg(z),\quad (\arg(z)\in(-\pi, \pi] ).
\end{equation*}

In addition, let
$y:=\epsilon(T)(\mathrm{Log}(z)-i\pi)\in \mathbb{R}+i\epsilon(T)(-\pi,0)$.

The relations between $(2\pi a,2\pi b,2\pi c)$, $(z,z^\prime,z^{\prime\prime})$, and $y$ for each sign of $\epsilon(T)$ are as follows.

If $\epsilon(T)=1$,
\begin{eqnarray*}
y+i\pi=\mathrm{Log}(z)=\log \left(\frac{\sin{2\pi c}}{\sin{2\pi b}}\right)+2\pi ia,\\ 
-\mathrm{Log}(1+e^y)=\mathrm{Log}(z^\prime)=\log\left(\frac{\sin{2\pi b}}{\sin{2\pi a}}\right)+2\pi ic,\\
\mathrm{Log}(1+e^{-y})=\mathrm{Log}(z^{\prime\prime})=\log\left(\frac{\sin{2\pi a}}{\sin{2\pi c}}\right)+2\pi ib,\\
y=\log\left(\frac{\sin{2\pi c}}{\sin{2\pi b}}\right)-i\pi(1-2a)\in\mathbb{R}-i\pi(1-2 a),\\
z=-e^y\in\mathbb{R}+i\mathbb{R}_{>0}.\\
\end{eqnarray*}

If $\epsilon(T)=-1$, 
\begin{eqnarray*}
-y+i\pi=\mathrm{Log}(z)=\log \left(\frac{\sin{2\pi b}}{\sin{2\pi c}}\right)+2\pi ia,\\ 
-\mathrm{Log}(1+e^{-y})=\mathrm{Log}(z^\prime)=\log\left(\frac{\sin{2\pi c}}{\sin{2\pi a}}\right)+2\pi ib,\\
\mathrm{Log}(1+e^{y})=\mathrm{Log}(z^{\prime\prime})=\log\left(\frac{\sin{2\pi a}}{\sin{2\pi b}}\right)+2\pi ic,\\
y=\log\left(\frac{\sin{2\pi c}}{\sin{2\pi b}}\right)+i\pi(1-2 a)\in\mathbb{R}+i\pi(1-2a),\\
z=-e^{-y}\in\mathbb{R}+i\mathbb{R}_{>0}.\\
\end{eqnarray*}

For a tetrahedral decomposition $X$ and an angle structure $\alpha_X\in\mathscr{A}_X$, the complex weight function $\omega_{X,\alpha_X}^\mathbb{C}:X^1\rightarrow \mathbb{C}$ is defined by assigning to the edge $e\in X^1$ the logarithmic sum of the complex shapes corresponding to the inverse images of $e$ by $\sim$.

Given an ideal tetrahedral decomposition $X$ of a 3-manifold $M$ with torus $S$ as a boundary component, suppose that $S$ is triangulated by triangles generated by truncating the corners of tetrahedra.
Furthermore, let $\sigma$ be an oriented normal closed curve on $S$.
Then $\sigma$ cuts off the corners of triangles on $S$, and let $z_1, z_2, \cdots, z_l$ be the corresponding complex shapes.
Assume $\epsilon_k=1$ if the corner of the triangle is to the left of $\sigma$ and $\epsilon_k=-1$ if it is to the right. Then 
the complex holonomy is defined as 
\begin{equation}
H^{\mathbb{C}}(\sigma)\coloneqq \sum_{k=1}^l \epsilon_k \mathrm{Log}(z_k).
\end{equation}

The angular holonomy is defined by replacing $\mathrm{Log}(z_k)$ with $\arg(z_k)=\Im(\mathrm{Log}(z_k))$.

The complex gluing edge equation is defined as follows,
\begin{equation*}
\forall e\in X^1,\quad \omega_{X,\alpha_X}^\mathbb{C}(e)=2i\pi.
\end{equation*}

On the other hand, the complex completeness equation is defined by the vanishing of the complex holonomy for any closed curve generating the first order homology $H_1(\partial M)$.

Let $M$ be an orientable $3$-manifold whose boundary consists of tori and given an ideal tetrahedral decomposition $X$. The angle structure $\alpha_X\in\mathscr{A}_X$
corresponds to the unique complete hyperbolic metric on the interior of $M$ if and only if  it satisfies the complex gluing edge equation and the complex completeness equation.
\subsection{The classical dilogarithm}
The classical dilogarithm is defined as follows. For $z\in\mathbb{C}\backslash[1,\infty)$, 
\begin{equation*}
\mathrm{Li}_2(z)\coloneqq -\int_0^z\mathrm{Log}(1-u)\frac{du}{u}.
\end{equation*}
The classical dilogarithm satisfies the following properties.
\begin{theorem}\label{classical-dilog}
(1) ( inversion relation )
\begin{equation*}
\forall z\in\mathbb{C}\backslash[1,\infty),\quad \mathrm{Li}_2\left(\frac{1}{z}\right)=-\mathrm{Li}_2(z)-\frac{\pi^2}{6}-\frac{1}{2}\mathrm{Log}(-z)^2.
\end{equation*}
(2) ( integral form )

For any $y\in\mathbb{R}+i(-\pi,\pi)$,
\begin{equation*}
\frac{-i}{2\pi}\mathrm{Li_2}(-e^y)=\int_{v\in\mathbb{R}+i0^+}\frac{\exp(-i\frac{yv}{\pi})}{4v^2\sinh{v}}dv,
\end{equation*}
where $\mathbb{R}+i0^+$ means the deformed path of the real axis in the complex plane which does not pass through the origin, but the upper half plane in the neighborhood of the origin. In this section, we use this notation. For example, the path may be a semicircle on the upper half-plane centered at the origin in the neighborhood of the origin.
\end{theorem}
\subsection{Bloch-Wigner function}
Bloch-Wigner function $D:\mathbb{C}\rightarrow\mathbb{R}$ is defined as follows.
\begin{equation*}
D(z)\coloneqq
\left\{
\begin{array}{lr}
\Im(\mathrm{Li}_2(z))+\arg(1-z)\log|z|& (\text{if}\  z\in\mathbb{C}\backslash\mathbb{R} ) \\
0 & ( \text{otherwise} ).
\end{array}
\right.    
\end{equation*}
Bloch-Wigner function is continuous on $\mathbb{C}$ and real analytic on $\mathbb{C}\backslash\{0,1\}$.
\begin{theorem}[\cite{MR0662760}]\label{bloch-hyperbolic-volume}
Let $T$ be an ideal tetrahedron in $3$-dimensional hyperbolic space $\mathbb{H}^3$, and the complex shape structure is given by $z$. 
Then the hyperbolic volume $\mathrm{Vol}(T)$ is expressed by the following equation.
\begin{equation*}
\mathrm{Vol}(T)=D(z)=D\left(\frac{z-1}{z}\right)=D\left(\frac{1}{1-z}\right).
\end{equation*}
\end{theorem}
We now give the definition of the asymptotic expansion and state the theorem on the asymptotic expansion necessary for the proof of Theorem \ref{uemura} (3).
\begin{definition}
Let $f:\Omega\rightarrow \mathbb{C}$ be a function on a non-bounded domain $\Omega\subset \mathbb{C}$. (Either convergent or divergent) complex power series $\sum_{n=0}^\infty a_nz^{-n}$ is called an asymptotic expansion of $f$, if for any fixed integer $N\ge 0$, 
\begin{equation*}
f(z)=\sum_{n=0}^N a_n z^{-n}+O (z^{-(N+1)})
\end{equation*}
if $z\rightarrow \infty$.
In this case, we denote
\begin{equation*}
f(z)\underset{z\rightarrow\infty}{\cong} \sum_{n=0}^\infty a_nz^{-n}.
\end{equation*}
\end{definition}
By Section 2.4.5 in \cite{Fedoryuk1989}, the following holds.
\begin{theorem}[Fedoryuk]\label{Fedoryuk}
Let $m\ge 1$ be an integer, and $\gamma^m$ be an $m$-dimensional smooth compact real submanifold of $\mathbb{C}^m$ with a connected boundary.
Denote $z=(z_1,\ldots,z_m)\in\mathbb{C}^m$, $dz=dz_1\cdots dz_m$.
Let $z\mapsto f(z)$ and $z\mapsto S(z)$ be analytic complex-valued functions on the domain $D$ such that $\gamma ^m\subset D\subset \mathbb{C}^m$.
With a parameter $\lambda\in\mathbb{R}$, we define
\begin{equation*}
F(\lambda)=\int_{\gamma^m} f(z)\exp (\lambda S(z))dz.
\end{equation*}

Suppose $\max_{z\in\gamma ^m} \Re S(z)$ is attained only at the point $z^0$, where $z^0$ is an interior point of $\gamma^m$ and a simple saddle point of $S$, i.e., $\nabla S(z^0)=0$ and $\det\mathrm{Hess}(S)(z^0)\neq 0$. 
Then if $\lambda\rightarrow \infty$, the following asymptotic expansion exists.
\begin{equation*}
F(\lambda)\underset{\lambda\rightarrow\infty}{\cong}\left(\frac{2\pi}{\lambda}\right)^{\frac{m}{2}}\frac{\exp(\lambda S(z^0))}{\sqrt{\det{\mathrm{Hess}(S)(z^0)}}}\left[f(z^0)+\sum_{k=1}^\infty c_k\lambda^{-k}\right],
\end{equation*}
where $c_k$ is a complex number, and the choice of the branch of the square root $\sqrt{\det \mathrm{Hess}(S)(z^0)}$ depends on the orientation of $\gamma ^m$.

In particular,
\begin{equation*}
\lim_{\lambda\rightarrow\infty}\frac{1}{\lambda}\log |F(\lambda)|=\Re S(z^0)   
\end{equation*}
holds.
\end{theorem}

By the equations $\eqref{it1'},\eqref{it2''},\eqref{it3'},\eqref{it4'}$, the angle structure $\mathscr{A}_X$ on the ideal tetrahedral decomposition $X$ of $S^3\backslash 7^3$ obtained in Section \ref{ideal-triangulation} is given by 
\begin{equation*}
 \mathscr{A}_X=\{\alpha_X\in\mathscr{S}_X\mid b_1+b_2+a_3=a_4+c_5, a_2=2b_1+c_1, a_2+b_3=c_4+b_5,a_3=b_1+b_2\} .  
\end{equation*}

In general, for a single tetrahedron $T$ constituting a tetrahedral decomposition $X$ with an extended shape structure $\alpha_X\in\overline{\mathscr{S}_X}$, if at least one of the dihedral angles of $T$ is $0$, it is called flat for $\alpha_X$, and if two dihedral angles are $0$ and the third dihedral angle is $\pi$, it is 
called taut for $\alpha_X$. In both cases, the volume of $T$ is $0$.

The following holds in the same way as in Lemma 4.3 in \cite{MR3945172}.
\begin{lemma}\label{flat-taut}
Let $\alpha_X\in\overline{\mathscr{A}_X}\backslash \mathscr{A}_X$ be such that the volume functional on $\overline{\mathscr{A}_X}$ is maximal at $\alpha_X$.
If one dihedral angle of $\alpha_X$ is zero, the other two dihedral angles of the tetrahedron containing this dihedral angle are $0$ and $\pi$. That is, if a tetrahedron is flat for $\alpha_X$, then it is taut.
\end{lemma}
In the four constraints in the definition of $\mathscr{A}_X$, even if we replace the variables like $(a_4,b_4,c_4)\leftrightarrow (c_5,a_5,b_5)$, the constraints are invariant. Therefore, by the concavity of the volume functional and the convexity of $\overline{\mathscr{A}_X}$, there exists an extended shape structure with $(a_4,b_4,c_4)=(c_5,a_5,b_5)$ which has the maximal volume. In this case, the constraints are given by

\begin{empheq}[left={\left\{
\begin{aligned}
b_1+b_2+a_3&=2a_4\\
a_2&=2b_1+c_1\\
a_2+b_3&=2c_4\\
a_3&=b_1+b_2\\
\end{aligned}
\right.\iff\empheqlbrace}]{align}
a_3&=a_4 \label{restriction-1}\\
a_2&=2b_1+c_1\label{restriction-2}\\
a_2+b_3&=2c_4\label{restriction-3}\\
a_3&=b_1+b_2\label{restriction-4}
\end{empheq}
\begin{lemma}\label{volume-maximizer}
Suppose the volume functional on $\overline{\mathscr{A}_X}$ is maximal at $\alpha_X$. Then $\alpha_X$ is not on the boundary $\overline{\mathscr{A}_X}\backslash\mathscr{A}_X$, but necessarily on the interior $\mathscr{A}_X$.
\end{lemma}
\begin{proof}
Since the volume functional takes its maximum value on $(a_4,b_4,c_4)=(c_5,a_5,b_5)$, if it can be shown that the volume functional under the constraints of $(a_4,b_4,c_4)=(c_5,a_5,b_5)$ and the equations \eqref{restriction-1}, \eqref{restriction-2}, \eqref{restriction-3}, \eqref{restriction-4} is maximal, not on the boundary $\overline{\mathscr{A}_X}\backslash\mathscr{A}_X$ but on the interior $\mathscr{A}_X$, the lemma is proven because the volume functional is strictly concave on $\mathscr{A}_X$.

Therefore, by reductio ad absurdum, we derive the contradiction by assuming that the volume functional under the constraints of $(a_4,b_4,c_4)=(c_5,a_5,b_5)$ and the equations \eqref{restriction-1}, \eqref{restriction-2},
 \eqref{restriction-3}, \eqref{restriction-4} takes the maximum value on the boundary $\overline{\mathscr{A}_X}\backslash\mathscr{A}_X$. 

In this case, at least one of $T_1,\ldots,T_4(T_5)$ is a taut tetrahedron by Lemma \ref{flat-taut}.

(i) In the case $T_1$ is taut.

\quad (a) If $(a_1,b_1,c_1)=(\frac{1}{2},0,0)$, by the equation \eqref{restriction-2}, $a_2=0$. Therefore $T_2$ is taut, so $b_2=0$ or $b_2=\frac{1}{2}$.

\quad\quad ($\alpha$) If $b_2=0$, by the equations \eqref{restriction-1} and \eqref{restriction-4}, $a_3=a_4=0$.
Since all tetrahedra are taut, the volume is $0$, which violates the maximality.

\quad\quad ($\beta$) If $b_2=\frac{1}{2}$, $a_3=a_4=\frac{1}{2}$ by the equations \eqref{restriction-1} and \eqref{restriction-4}.
Since all tetrahedra are taut also, in this case, the volume is $0$, which violates the maximality.

\quad (b) If $(a_1,b_1,c_1)=(0,0,\frac{1}{2})$, $a_2=\frac{1}{2}$ by the equation \eqref{restriction-2}.
Therefore, $b_2=c_2=0$. By the equations \eqref{restriction-1} and \eqref{restriction-4}, $a_3=a_4=0$. Therefore, $T_1,\ldots,T_4(T_5)$
are all taut, and the volume is $0$, which violates the maximality.

\quad (c) If $(a_1,b_1,c_1)=(0,\frac{1}{2},0)$, this situation does not exist because $a_2=1>\frac{1}{2}$ by the equation \eqref{restriction-2}.

(ii) In the case $T_2$ is taut.

\quad (a) If $(a_2,b_2,c_2)=(0,0,\frac{1}{2})$, by the equation \eqref{restriction-2}, $2b_1+c_1=0$. Therefore $b_1=c_1=0$, and $T_1$ is taut, so this case is attributed to (i).

\quad (b) If $(a_2,b_2,c_2)=(0,\frac{1}{2},0)$, as in (a), by the equation \eqref{restriction-2}, $b_1=c_1=0$
and $T_1$ is taut. Then this case is attributed to (i).

(iii) In the case $T_3$ is taut.

\quad (a) If $(a_3,b_3,c_3)=(0,0,\frac{1}{2})$ or $(0,\frac{1}{2},0)$, by the equation \eqref{restriction-4}, $b_1=b_2=0$ and this case is attributed to (i).

\quad (b) If $(a_3,b_3,c_3)=(\frac{1}{2},0,0)$, $a_4=\frac{1}{2}$ by the equation \eqref{restriction-1}. Therefore $b_4=c_4=0$. 
By the equation \eqref{restriction-3}, $a_2=0$. By the equation \eqref{restriction-2}, $b_1=c_1=0$, so this case is attributed to (i).

(iv) In the case $T_4(T_5)$ is taut.

\quad (a) If $(a_4,b_4,c_4)=(0,0,\frac{1}{2})$ or $(0,\frac{1}{2},0)$, by the equation \eqref{restriction-1}, $a_3=0$, and by the equation \eqref{restriction-4}, $b_1=b_2=0$. Therefore, this case is attributed to (i).

\quad (b) If $(a_4,b_4,c_4)=(\frac{1}{2},0,0)$, by the equation \eqref{restriction-3}, $a_2=b_3=0$. By the equation \eqref{restriction-2}, $b_1=c_1=0$. Therefore, this case is attributed to (i).

Therefore, the volume functional can be maximal on the boundary if $(a_2,b_2,c_2)=(\frac{1}{2},0,0)$ and $T_1$, $T_3$ and $T_4(T_5)$ are non-flat tetrahedra.

Then 
\begin{empheq}[left={\empheqlbrace}]{align}
a_3&=a_4 \label{restriction-1'}\\
\frac{1}{2}&=2b_1+c_1\label{restriction-2'}\\
\frac{1}{2}+b_3&=2c_4\label{restriction-3'}\\
a_3&=b_1 \label{restriction-4'}.
\end{empheq}

By the equation \eqref{restriction-2'}, $\frac{1}{2}=b_1+(\frac{1}{2}-a_1)$, then $a_1=b_1$.

Let $a_3=b_1=a_4=a_1=\frac{s}{2\pi}$. Then $c_1=\frac{1}{2}-\frac{s}{\pi}$. Let $b_4=\frac{v}{2\pi}$, then $c_4=\frac{1}{2}-\frac{s}{2\pi}-\frac{v}{2\pi}$.
By the equation \eqref{restriction-3'}, $b_3=\frac{1}{2}-\frac{s}{\pi}-\frac{v}{\pi}$. $c_3=\frac{1}{2}-a_3-b_3=\frac{s}{2\pi}+\frac{v}{\pi}$.

Since $0<a_i<\frac{1}{2}$, $0<b_i<\frac{1}{2}$, and $0<c_i<\frac{1}{2}$ for $i=1$, $3$, and $4$, it follows that $0<s$, $0<v$ and $s+v<\frac{\pi}{2}$.

Let $t>0$ be a sufficiently small real number and
\begin{eqnarray*}
a_1^t=a_1+\frac{t}{2\pi},\quad b_1^t=b_1,\quad c_1^t=c_1-\frac{t}{2\pi}\\
a_2^t=a_2-\frac{t}{2\pi},\quad b_2^t=b_2,\quad c_2^t=c_2+\frac{t}{2\pi}\\
a_3^t=a_3,\quad b_3^t=b_3+\frac{t}{2\pi},\quad c_3^t=c_3-\frac{t}{2\pi}\\
a_4^t=a_4,\quad b_4^t=b_4,\quad c_4^t=c_4,
\end{eqnarray*}
then $(a_i^t,b_i^t,c_i^t)\quad(i=1,\cdots,4)$ satisfy the constraints from \eqref{restriction-1} to \eqref{restriction-4}.

Calculation of the derivative of the volume functional $\mathscr{V}$ on the extended angle structure given by $(a_i^t,b_i^t,c_i^t)\quad(i=1,\cdots,4)$ gives, noting that since $b_2^t=b_2=0$, the volume of $T_2$ is zero independently of $t$, 
\begin{eqnarray*}
\left.\frac{\partial \mathscr{V}}{\partial t}\right|_{t=0}&=&\Lambda^\prime(2\pi a_1)-\Lambda^\prime(2\pi c_1)
+\Lambda^\prime(2\pi b_3)-\Lambda^\prime(2\pi c_3)\\
&=&-\log{\sin{2\pi a_1}}+\log{\sin{2\pi c_1}}-\log{\sin{2\pi b_3}}+\log{\sin{2\pi c_3}}\\
&=&-\log{\sin{s}}+\log{\sin{2s}}-\log{\sin{2(s+v)}}+\log{\sin{(s+2v)}}.
\end{eqnarray*}

Then 
\begin{equation*}
\exp{\left(\left.-\frac{\partial \mathscr{V}}{\partial t}\right |_{t=0}\right)}=\frac{\sin{2(s+v)}}{2\cos{s}\sin{(s+2v)}}.
\end{equation*}

Fix $s$. Let $f(v)=\frac{\sin{2(s+v)}}{\sin{(s+2v)}}$, then
\begin{eqnarray*}
f^\prime(v)&=&\frac{2\cos{2(s+v)}\sin{(s+2v)}-2\sin{2(s+v)}\cos{(s+2v)}}{\sin^2{(s+2v)}}\\
&=&\frac{2\{\sin{(s+2v-2(s+v))}\}}{\sin^2{(s+2v)}}=\frac{2\sin{(-s)}}{\sin^2{(s+2v)}}<0.
\end{eqnarray*}

Thus, $f(v)$ strictly monotonically decreases and 
\begin{equation*}
\lim_{v\rightarrow 0^+}\frac{\sin{2(s+v)}}{2(\cos{s})\sin{(s+2v)}}=\lim_{v\rightarrow 0^+}\frac{\sin{2s}}{2\cos{s}\sin{s}}=1.
\end{equation*}

Therefore, $\exp{\left(\left.-\frac{\partial\mathscr{V}}{\partial t}\right |_{t=0}\right)}<1$, that is $\left.\frac{\partial\mathscr{V}}{\partial t}\right |_{t=0}>0$, then the volume functional $\mathscr{V}$ is not maximal at $t=0$, which is a contradiction.
\end{proof}
\begin{proof}[{\bf Proof of Theorem \ref{geometric}}]
By Lemma \ref{non-empty}, $\mathscr{A}_X$ is non-empty. Therefore, since a continuous function on a non-empty compact set has a maximum value, the volume functional $\mathscr{V}:\overline{\mathscr{A}_X}\rightarrow \mathbb{R}$ is maximal at some $\alpha_X\in\overline{\mathscr{A}_X}$.  
Since Lemma \ref{volume-maximizer}, $\alpha_X \notin\overline{\mathscr{A}_X}\backslash\mathscr{A}_X$, that is, $\alpha_X\in\mathscr{A}_X$, it is shown that $X$ is geometric by Theorem \ref{Casson-Rivin}. 
\end{proof}
\subsection{Cusp triangulation and the complex gluing equation}
By removing the neighborhood of each vertex of tetrahedra in Figure \ref{fig:ideal_triangulation}, we cut off the corners of tetrahedra. Then let $\nu(7_3)$ be a tubular neighborhood of the knot $7_3$ in $S^3$, and we obtain a triangulation of the boundary torus $\partial \nu(7_3)$. This triangulation of the torus is shown in Figure \ref{fig:torus_triangulation}.
% Figure environment removed
Here, $i_j$ ($i=1,\ldots,5$, $j=0,\ldots,3$) inside the triangle in Figure \ref{fig:torus_triangulation} denotes the triangle created by truncating the neighborhood of the vertex $j$ of the tetrahedron $T_i$ in Figure \ref{fig:ideal_triangulation}.
The letter marked near each edge of each triangle indicates the variable of Figure \ref{fig:ideal_triangulation} that represents the face of the tetrahedron that includes the edge. The edges with the same letter on the top, bottom, left, and right of the rectangle in the figure are considered to be identical to each other.
The symbols $a,b$, and $c$ on the corners of the triangle indicate the dihedral angles, and for a triangle with $i_j$ marked inside, it means $2\pi a_i, 2\pi b_i, 2\pi c_i$, respectively. Closed curves representing the torus meridian $m_X$ and longitude $l_X$ are drawn by dashed lines in the figure.
Let $z_i$ be a complex shape structure of the tetrahedron $T_i$, then the complex gluing equations are given as follows. Here, $e_1,\ldots,e_5\in X^1$ are shown in Figure \ref{fig:ideal_triangulation}, \ref{fig:ideal_triangulation_edge}.
\begin{eqnarray}
\omega_{X,\alpha}^\mathbb{C}(e_1)=2i\pi &\iff&\mathrm{Log} z_1^{\prime\prime}+\mathrm{Log} z_2^\prime +\mathrm{Log} z_3 +\mathrm{Log} z_4^{\prime\prime} +\mathrm{Log} z_4^\prime \nonumber\\
& & +\mathrm{Log} z_5 +\mathrm{Log} z_5^{\prime\prime} =2i\pi, \label{e1-complex-gluing}\\
\omega_{X,\alpha}^\mathbb{C}(e_2)=2i\pi&\iff&\mathrm{Log} z_1^{\prime\prime} +\mathrm{Log} z_1^\prime +\mathrm{Log} z_2^{\prime\prime} +\mathrm{Log} z_3^{\prime\prime} +\mathrm{Log} z_3^\prime \nonumber\\
& &+\mathrm{Log} z_4+\mathrm{Log} z_5^\prime =2i\pi, \label{e2-complex-gluing}\\
\omega_{X,\alpha}^\mathbb{C}(e_3)=2i\pi&\iff&\mathrm{Log} z_2+\mathrm{Log}z_3^{\prime\prime}+\mathrm{Log} z_4+\mathrm{Log} z_4^{\prime\prime}\nonumber\\
& &+\mathrm{Log} z_5+\mathrm{Log} z_5^\prime=2i\pi,\label{e3-complex-gluing}\\
\omega_{X,\alpha}^\mathbb{C}(e_4)=2i\pi&\iff&\mathrm{Log} z_1+\mathrm{Log} z_1^\prime +\mathrm{Log} z_2+\mathrm{Log} z_2^{\prime\prime} +\mathrm{Log} z_3 =2i\pi, \label{e4-complex-gluing}\\
\omega_{X,\alpha}^\mathbb{C}(e_5)=2i\pi&\iff&\mathrm{Log} z_1 +\mathrm{Log} z_2^\prime +\mathrm{Log} z_3^\prime +\mathrm{Log} z_4^\prime +\mathrm{Log} z_5^{\prime\prime} =2i\pi.\label{e5-complex-gluing}
\end{eqnarray}

The complex completeness equations regarding the meridian $m_X$ and longitude $l_X$ in Figure \ref{fig:torus_triangulation} are
\begin{equation}
H^{\mathbb{C}}(m_X)=0\iff-\mathrm{Log} z_3+\mathrm{Log} z_4=0\iff z_3=z_4\label{meridian-completeness}
\end{equation}
\begin{eqnarray}
& & H^{\mathbb{C}}(l_X)=0\nonumber\\
&\iff&-\mathrm{Log} z_1^{\prime\prime}-\mathrm{Log} z_2^{\prime\prime}+\mathrm{Log} z_5-\mathrm{Log}z_4^\prime+\mathrm{Log} z_3+\mathrm{Log} z_4^\prime
-\mathrm{Log} z_5^\prime\nonumber\\
&&\quad-\mathrm{Log} z_4^{\prime\prime}+\mathrm{Log} z_3^\prime+\mathrm{Log} z_1^{\prime\prime}+\mathrm{Log} z_2^{\prime\prime}
-\mathrm{Log} z_5^{\prime\prime}+\mathrm{Log} z_4^\prime-\mathrm{Log} z_3^\prime=0\nonumber\\
&\iff&2\mathrm{Log} z_5+\mathrm{Log} z_3+\mathrm{Log} z_4^\prime-\mathrm{Log}z_4^{\prime\prime}=i\pi,\label{completeness-longitude}
\end{eqnarray}
where, in the derivation of the equation \eqref{completeness-longitude}, we used $\mathrm{Log} z_i+\mathrm{Log} z_i^\prime+\mathrm{Log} z_i^{\prime\prime}=i\pi\quad (i=1,\ldots,5)$.

Under the assumption that the equation \eqref{meridian-completeness} holds, that is, $z_3=z_4$, the equation \eqref{e1-complex-gluing} is transformed as follows, 
\begin{align}
&\mathrm{Log} z_1^{\prime\prime}+\mathrm{Log} z_2^\prime+\mathrm{Log} z_3+\mathrm{Log} z_3^{\prime\prime}+\mathrm{Log} z_3^\prime 
+\mathrm{Log} z_5+\mathrm{Log} z_5^{\prime\prime}=2i\pi\nonumber\\
\iff&\mathrm{Log} z_1^{\prime\prime}+\mathrm{Log} z_2^{\prime}+i\pi+i\pi-\mathrm{Log} z_5^\prime=2i\pi\nonumber\\
\iff&\mathrm{Log} z_1^{\prime\prime}+\mathrm{Log} z_2^\prime-\mathrm{Log} z_5^\prime=0\tag{$\ref{e1-complex-gluing}^\prime$}.\label{e1-complex-gluing'}
\end{align}

The equation \eqref{e2-complex-gluing} is transformed as 
\begin{align}
&i\pi-\mathrm{Log}z_1+\mathrm{Log} z_2^{\prime\prime}+i\pi+\mathrm{Log}z_5^\prime =2i\pi\nonumber\\
\iff&\mathrm{Log} z_1-\mathrm{Log} z_2^{\prime\prime}-\mathrm{Log}z_5^\prime=0.\tag{$\ref{e2-complex-gluing}^\prime$}\label{e2-complex-gluing'}
\end{align}

The equation \eqref{e3-complex-gluing} is transformed as
\begin{align}
&\mathrm{Log} z_2+i\pi-\mathrm{Log} z_3^\prime+\mathrm{Log} z_3^{\prime\prime}+i\pi-\mathrm{Log} z_5^{\prime\prime}=2i\pi\nonumber\\
\iff&\mathrm{Log} z_2-\mathrm{Log} z_3^\prime+\mathrm{Log} z_3^{\prime\prime}-\mathrm{Log} z_5^{\prime\prime}=0.\tag{$\ref{e3-complex-gluing}^\prime$}\label{e3-complex-gluing'}
\end{align}

The equation \eqref{e4-complex-gluing} is transformed as
\begin{align}
&i\pi-\mathrm{Log} z_1^{\prime\prime}+i\pi-\mathrm{Log} z_2^\prime+\mathrm{Log} z_3=2i\pi\nonumber\\
\iff&\mathrm{Log} z_3=\mathrm{Log} z_1^{\prime\prime}+\mathrm{Log} z_2^\prime.\tag{$\ref{e4-complex-gluing}^\prime$}\label{e4-complex-gluing'}
\end{align}


The equation \eqref{e5-complex-gluing} is transformed as 
\begin{align}
\mathrm{Log} z_1 +\mathrm{Log} z_2^\prime +2\mathrm{Log} z_3^\prime +\mathrm{Log} z_5^{\prime\prime} =2i\pi.\tag{$\ref{e5-complex-gluing}^\prime$}\label{e5-complex-gluing'}
\end{align}

The equation \eqref{completeness-longitude} is transformed as 
\begin{align}
&2\mathrm{Log} z_5+i\pi-\mathrm{Log} z_3^{\prime\prime} -\mathrm{Log} z_3^{\prime\prime}=i\pi\nonumber\\
\iff&\mathrm{Log} z_5=\mathrm{Log} z_3^\prime
\iff z_5=z_3^{\prime\prime}\iff z_3=z_5^\prime.\tag{$\ref{completeness-longitude}^\prime$}\label{completeness-longitude'}
\end{align}

If the equation \eqref{completeness-longitude'} holds, $z_3^\prime=z_5^{\prime\prime}$, thus the equation \eqref{e5-complex-gluing'} is transformed as
\begin{align}
\mathrm{Log} z_1+\mathrm{Log} z_2^{\prime} +3\mathrm{Log} z_3^\prime=2i\pi\tag{$\ref{e5-complex-gluing}^{\prime\prime}$},\label{e5-complex-gluing''}
\end{align}
and the equation \eqref{e1-complex-gluing'} is transformed as
\begin{align*}
\mathrm{Log}z_3=\mathrm{Log}z_1^{\prime\prime}+\mathrm{Log}z_2^\prime,
\end{align*}
and the equation \eqref{e4-complex-gluing'} is derived. In addition, the equation \eqref{e3-complex-gluing'} is transformed as 
\begin{align}
2\mathrm{Log} z_3^\prime=\mathrm{Log} z_2+\mathrm{Log} z_3^{\prime\prime},\tag{$\ref{e3-complex-gluing}^{\prime\prime}$}\label{e3-complex-gluing''}
\end{align}
and the equation \eqref{e2-complex-gluing'} is transformed as 
\begin{align}
\mathrm{Log} z_3=\mathrm{Log} z_1-\mathrm{Log} z_2^{\prime\prime}.\tag{$\ref{e2-complex-gluing}^{\prime\prime}$}\label{e2-complex-gluing''}
\end{align}

On the other hand, transforming the equation \eqref{e5-complex-gluing''} using the equation \eqref{e2-complex-gluing''},
\begin{align*}
&\mathrm{Log} z_3+\mathrm{Log} z_2^{\prime\prime}+\mathrm{Log} z_2^\prime+3\mathrm{Log} z_3^\prime=2i\pi\\
\iff&i\pi-\mathrm{Log} z_2+i\pi-\mathrm{Log} z_3^{\prime\prime}+2\mathrm{Log} z_3^\prime=2i\pi\\
\iff&2\mathrm{Log} z_3^\prime=\mathrm{Log} z_2+\mathrm{Log} z_3^{\prime\prime},
\end{align*}
and the equation \eqref{e3-complex-gluing''} is derived.

Therefore, the equations from \eqref{e1-complex-gluing} to \eqref{completeness-longitude} are satisfied if and only if the equations \eqref{meridian-completeness}, \eqref{completeness-longitude'}, \eqref{e4-complex-gluing'}, \eqref{e2-complex-gluing''}, \eqref{e5-complex-gluing''} are satisfied.

By Theorem \ref{geometric}, $X$ has a unique complex shape structure corresponding to the complete hyperbolic metric, and this complex shape structure is the unique $z=(z_1,z_2,z_3,z_4,z_5)\in(\mathbb{R}+i\mathbb{R}_{>0})^5$ which satisfies the equations \eqref{meridian-completeness}, \eqref{completeness-longitude'}, \eqref{e4-complex-gluing'}, \eqref{e2-complex-gluing''}, and \eqref{e5-complex-gluing''}.  

Let
\begin{equation*}
\mathscr{U}\coloneqq (\mathbb{R}+i(-\pi,0))\times(\mathbb{R}+i(0,\pi))\times(\mathbb{R}+i(-\pi,0)).  
\end{equation*} 

Let $\alpha_X^0\in\mathscr{A}_X$ be the angle structure corresponding to the unique complete hyperbolic metric, which exists by Theorem \ref{geometric}, and let
\begin{equation*}
\mathscr{Y}^0\coloneqq\mathscr{Y}_{\alpha_X^0}.  
\end{equation*}
The potential function $S:\mathscr{U}\rightarrow \mathbb{C}$ is defined as follows, 
\begin{equation}\label{definition-of-S}
S(\mathbf{y})\coloneqq i\mathbf{y}^{\mathrm{T}}Q\mathbf{y}+\mathbf{y}^{\mathrm{T}}\mathscr{W}+i\mathrm{Li}_2(-e^{Y})+3i\mathrm{Li}_2(-e^{W})-i\mathrm{Li}_2(-e^{Z}).
\end{equation}
\subsection{The property of the potential function $S$ on $\mathscr{U}$.}
\begin{lemma}[ \cite{MR3945172}, Lemma 7.2 ]\label{invertible}
Let $m\ge 1$ be an integer, and let $S_1$ and $S_2\in M_m(\mathbb{R})$ be real square matrices of order $m$ such that $S_1$ is a symmetric positive definite matrix and $S_2$ is a symmetric matrix.
Then the complex symmetric matrix $S_1+iS_2$ is invertible.
\end{lemma}
Hereafter, let $y_1\coloneqq Y$, $y_2\coloneqq Z$, and $y_3\coloneqq W$.
\begin{lemma}\label{hessian}
For any $\mathbf{y}\in\mathscr{U}$, the holomorphic Hessian of $S$ is given by
\begin{equation*}
\mathrm{Hess}(S)(\mathbf{y})=\left(\frac{\partial^2S}{\partial y_j\partial y_k}\right)_{j,k\in \{1,2,3\}}(\mathbf{y})=2iQ+i
\begin{pmatrix}
\frac{-1}{1+e^{-y_1}} &  &  \\
 & \frac{1}{1+e^{-y_2}} & \\
  &  &  \frac{-3}{1+e^{-y_3}}
\end{pmatrix}.
\end{equation*}

Furthermore, for any $\mathbf{y}\in\mathscr{U}$, the determinant of $\mathrm{Hess}(S)(\mathbf{y})$ is not $0$.
\end{lemma}
\begin{proof}
The first equation is derived by the second derivative of $S$ and by the fact that for any $y\in\mathbb{R}\pm i(0,\pi)$
\begin{equation*}
\frac{\partial \mathrm{Li}_2(-e^y)}{\partial y}=-\mathrm{Log}(1+e^y)
\end{equation*}
holds.

Let $\mathbf{y}\in\mathscr{U}$, then $Q$ is a real symmetric matrix, so $\Im (\mathrm{Hess}(S)(\mathbf{y}))$ is a symmetric matrix.
Furthermore,
\begin{equation*}
\Re(\mathrm{Hess}(S)(\mathbf{y}))=\begin{pmatrix}
-\Im \left(\frac{-1}{1+e^{-y_1}}\right) & & \\
 & -\Im \left(\frac{1}{1+e^{-y_2}}\right) & \\
 &   &  -\Im \left(\frac{-3}{1+e^{-y_3}}\right) 
\end{pmatrix},
\end{equation*}
and all the diagonal components are negative.
In fact, let $y_1=a+bi\quad(a,b\in\mathbb{R})$, then $b\in(-\pi,0)$ and 
\begin{eqnarray*}
-\Im \left(\frac{-1}{1+e^{-y_1}}\right)&=&\Im \frac{1}{1+e^{-a}(\cos{b}-i\sin{b})}\\
&=&\frac{e^{-a}\sin{b}}{(1+e^{-a}\cos{b})^2+(e^{-a}\sin{b})^2}<0.
\end{eqnarray*} The same applies to the other diagonal components.
Therefore, since $-\mathrm{Hess}(S)(\mathbf{y})$ is invertible by Lemma \ref{invertible}, $\mathrm{Hess}(S)(\mathbf{y})$ is also invertible and the lemma is proven.
\end{proof}
On the other hand, for a tetrahedron $T$ with a shape structure, there exists the following diffeomorphism $\psi_T$.
\begin{equation*}
\psi_T:\mathbb{R}+i\mathbb{R}_{>0}\rightarrow\mathbb{R}-i\epsilon(T)(0,\pi),\quad z\mapsto\epsilon(T)(\mathrm{Log}(z)-i\pi)
\end{equation*}
The inverse map is given by
\begin{equation*}
\psi_T^{-1}:\mathbb{R}-i\epsilon(T)(0,\pi)\rightarrow \mathbb{R}+i\mathbb{R}_{>0},\quad y\mapsto -\exp(\epsilon(T)y).
\end{equation*}
\begin{lemma}\label{completeness-bijection}
Let us consider the following diffeomorphism
\begin{equation*}
\psi\coloneqq\left(\prod_{T\in\{T_1,T_2,T_3\}}\psi_T \right):(\mathbb{R}+i\mathbb{R}_{>0})^{3}\rightarrow \mathscr{U}.    
\end{equation*}
Here, we note that $\epsilon(T_1)=\epsilon(T_3)=1$, and $\epsilon(T_2)=-1$ according to Figure \ref{fig:ideal_triangulation}.

The map $\psi$ induces the bijective map between $\{\mathbf{z}=(z_1,z_2,z_3)\in(\mathbb{R}+i\mathbb{R}_{>0})^3\mid\eqref{e4-complex-gluing'}\land\eqref{e2-complex-gluing''}\land\eqref{e5-complex-gluing''}\}$ and $\{\mathbf{y}\in\mathscr{U}\mid\nabla S(\mathbf{y})=0\}$.
In particular, $S$ has the unique critical point on $\mathscr{U}$ corresponding to the unique complete hyperbolic structure $\mathbf{z}^0$ on the geometric ideal tetrahedral decomposition $X$. Here, we define $z_4^0=z_3^0$ and $z_5^0={z_3^{\prime\prime}}^0$. 
\end{lemma}
\begin{proof}
For any $\mathbf{y}\in\mathscr{U}$,
\begin{equation*}
\nabla S(\mathbf{y})=\begin{pmatrix}
\partial_{y_1} S(\mathbf{y})\\
\partial_{y_2} S(\mathbf{y})\\
\partial_{y_3} S(\mathbf{y})
\end{pmatrix}
=2iQ\mathbf{y}+\mathscr{W}+i\begin{pmatrix}
-\mathrm{Log}(1+e^{y_1})\\
\mathrm{Log}(1+e^{y_2})\\
-3\mathrm{Log}(1+e^{y_3})
\end{pmatrix}.
\end{equation*}
Let $y_1=\psi_{T_1}(z_1)$, $y_3=\psi_{T_3}(z_3)$, then 
\begin{eqnarray*}
\mathrm{Log}(z_1)=y_1+i\pi,&\quad\mathrm{Log}(z_1^\prime)=-\mathrm{Log}(1+e^{y_1}),&\quad\mathrm{Log}(z_1^{\prime\prime})=\mathrm{Log}(1+e^{-y_1})\\
\mathrm{Log}(z_3)=y_3+i\pi,&\quad\mathrm{Log}(z_3^\prime)=-\mathrm{Log}(1+e^{y_3}),&\quad \mathrm{Log}(z_3^{\prime\prime})=\mathrm{Log}(1+e^{-y_3})
\end{eqnarray*}
and let $y_2=\psi_{T_2}(z_2)$, then 
\begin{equation*}
\mathrm{Log}(z_2)=-y_2+i\pi,\quad \mathrm{Log}(z_2^\prime)=-\mathrm{Log}(1+e^{-y_2}),\quad \mathrm{Log}(z_2^{\prime\prime})=\mathrm{Log}(1+e^{y_2}).
\end{equation*}
Since
\begin{eqnarray*}
\nabla S(\mathbf{y})&=i\begin{pmatrix}
2 & -1 & 0\\
-1 & 0 &1 \\
0 & 1 & 1
\end{pmatrix}\begin{pmatrix}
y_1\\
y_2\\
y_3
\end{pmatrix}
+\begin{pmatrix}
-\pi\\
0\\
\pi
\end{pmatrix}
+i\begin{pmatrix}
-\mathrm{Log}(1+e^{y_1})\\
\mathrm{Log}(1+e^{y_2})\\
-3\mathrm{Log}(1+e^{y_3})
\end{pmatrix},
\end{eqnarray*}

\begin{eqnarray*}
\nabla S(\psi(\mathbf{z}))&=i\begin{pmatrix}
2\mathrm{Log}z_1+\mathrm{Log}z_2+\mathrm{Log}z_1^\prime-2i\pi\\
-\mathrm{Log}z_1+\mathrm{Log}z_3+\mathrm{Log}z_2^{\prime\prime}\\
-\mathrm{Log}z_2+\mathrm{Log}z_3+3\mathrm{Log}z_3^\prime-i\pi
\end{pmatrix}.
\end{eqnarray*}
Let 
\begin{equation*}
A=\begin{pmatrix}
1 & 1 & \\
 & 1 & \\
 & -1 & 1
\end{pmatrix}.    
\end{equation*}
Since $|A|=1$, $A$ is a regular matrix and
\begin{eqnarray*}
A\cdot(\nabla S(\psi(\mathbf{z}))&=&i\begin{pmatrix}
\mathrm{Log}z_1+\mathrm{Log}z_1^\prime+\mathrm{Log}z_2+\mathrm{Log}z_2^{\prime\prime}+\mathrm{Log}z_3-2i\pi\\
-\mathrm{Log}z_1+\mathrm{Log}z_3+\mathrm{Log}z_2^{\prime\prime}\\
\mathrm{Log}z_1-\mathrm{Log}z_2-\mathrm{Log}z_2^{\prime\prime}+3\mathrm{Log}z_3^\prime-i\pi
\end{pmatrix}\\
&=&i\begin{pmatrix}
\mathrm{Log}z_3-\mathrm{Log}z_1^{\prime\prime}-\mathrm{Log}z_2^\prime\\
-\mathrm{Log}z_1+\mathrm{Log}z_3+\mathrm{Log}z_2^{\prime\prime}\\
\mathrm{Log}z_1+\mathrm{Log}z_2^\prime+3\mathrm{Log}z_3^\prime-2i\pi
\end{pmatrix}.
\end{eqnarray*}
In this deformation, we used the fact that $\mathrm{Log}z_j+\mathrm{Log}z_j^\prime+\mathrm{Log}z_j^{\prime\prime}=i\pi\quad(j=1,2,3)$.

Therefore, since satisfying $\mathbf{z}\in(\mathbb{R}+i\mathbb{R}_{>0})^3$ and $\eqref{e4-complex-gluing'}\land\eqref{e2-complex-gluing''}\land\eqref{e5-complex-gluing''}$ is equivalent to satisfying $\psi(\mathbf{z})\in\mathscr{U}$ and $\nabla S(\psi(\mathbf{z}))=\mathbf{0}$, the lemma is proven.
\end{proof}
Let
\begin{equation*}
\mathscr{Y}^0=\mathscr{Y}_{\alpha_X^0}=(\mathbb{R}-i\pi(1-2a_1^{0}))\times(\mathbb{R}+i\pi(1-2a_2^{0}))\times(\mathbb{R}-i\pi(1-2a_3^{0})),
\end{equation*}
where $\alpha_X^0\in\mathscr{A}_X$ is the complete hyperbolic angle structure corresponding to the complete hyperbolic complex shape structure $\mathbf{z}^0$.

Let $\mathbf{y}\in\mathscr{Y}^0$ be parametric represented by 
\begin{equation*}
\mathbf{y}=\begin{pmatrix}
y_1\\
y_2\\
y_3
\end{pmatrix}
=\begin{pmatrix}
x_1+id_1^0\\
x_2+id_2^0\\
x_3+id_3^0
\end{pmatrix}
=\mathbf{x}+i\mathbf{d}^0,
\end{equation*}
where $d_1^0=-\pi(1-2a_1^{0})<0$, $d_2^0=\pi(1-2a_2^{0})>0$, $d_3^0=-\pi(1-2a_3^{0})<0$. In addition, let $\mathbf{y}^0\coloneqq\psi(z_1^0,z_2^0,z_3^0)\in\mathscr{Y}^0$. $\mathscr{Y}^0=\mathbb{R}^3+i\mathbf{d}^0\subset \mathbb{C}^3$ is a $\mathbb{R}$-affine subspace of $\mathbb{C}^3$.
\subsection{The concavity of $\Re S$ on each contour $\mathscr{Y}_\alpha$}
\begin{lemma}\label{RS-concave}
For any $\alpha\in\mathscr{A}_X$, the function $\Re S:\mathscr{Y}_\alpha\rightarrow \mathbb{R}$ is strictly concave on $\mathscr{Y}_\alpha$.
\end{lemma}
\begin{proof}
Let $\alpha\in\mathscr{A}_X$. 
Since $\Re S:\mathscr{Y}_\alpha\rightarrow\mathbb{R}$ is twice continuously differentiable as a function of three real variables, 
it is sufficient to show that the Hessian $(\Re S|_{\mathscr{Y}_\alpha})^{\prime\prime}$ is negative definite for all $\mathbf{x}+i\mathbf{d}\in\mathscr{Y}_\alpha$.
Since this Hessian is equal to the real part of the holomorphic Hessian of $S$, by the calculation of Lemma \ref{hessian}, for all $\mathbf{x}\in\mathbb{R}^3$, the following holds.
\begin{eqnarray*}
(\Re S|_{\mathscr{Y}_\alpha})^{\prime\prime}(\mathbf{x}+i\mathbf{d})&=&\Re(\mathrm{Hess}(S)(\mathbf{x}+i\mathbf{d}))\\
&=&\begin{pmatrix}
-\Im\left(\frac{-1}{1+e^{-x_1-id_1}}\right) & & \\
 & -\Im\left(\frac{1}{1+e^{-x_2-id_2}}\right) & \\
 & & -\Im (\frac{-3}{1+e^{-x_3-id_3}})
\end{pmatrix}.
\end{eqnarray*}
Since $d_1,d_3\in(-\pi,0)\quad d_2\in(0,\pi)$, all diagonal components of this matrix are negative.
In fact, 
\begin{equation*}
-\Im\left(\frac{-1}{1+e^{-x_1-id_1}}\right)=\frac{e^{-x_1}\sin{d_1}}{(1+e^{-x_1}\cos{d_1})^2+(e^{-x_1}\sin{d_1})^2}<0,
\end{equation*}
\begin{equation*}
-\Im\left(\frac{1}{1+e^{-x_2-id_2}}\right)=\frac{-e^{-x_2}\sin{d_2}}{(1+e^{-x_2}\cos{d_2})^2+(e^{-x_2}\sin{d_2})^2}<0
\end{equation*}
and the same holds for the other diagonal component. 

Therefore, $(\Re S|_{\mathscr{Y}_\alpha})^{\prime\prime}$ is negative definite at any point on $\mathscr{Y}_\alpha$, and $\Re S|_{\mathscr{Y}_\alpha}$ is strictly concave.
\end{proof}
\subsection{Properties of $\Re S$ on the contour $\mathscr{Y}^0$ corresponding to the complete hyperbolic structure}
The following lemma holds as well as Lemma 7.6 in \cite{MR3945172}.
\begin{lemma}\label{strict-max-of-ReS}
The function $\Re S:\mathscr{Y}^0\rightarrow\mathbb{R}$ has a strictly global maximum at $\mathbf{y}^0\in\mathscr{Y}^0$.
\end{lemma}
\begin{proof}
Since the holomorphic gradient of $S:\mathscr{U}\rightarrow \mathbb{C}$ vanishes at $\mathbf{y}^0$ by Lemma \ref{completeness-bijection}, the real gradient of $\Re S|_{\mathscr{Y}^0}$, which is the real part of the holomorphic gradient of $S$, also vanishes at $\mathbf{y}^0$. Therefore $\mathbf{y}^0$ is the critical point of $\Re S|_{\mathscr{Y}^0}$.
Furthermore, since $\Re S|_{\mathscr{Y}^0}$ is strictly concave by Lemma \ref{RS-concave}, $\Re S|_{\mathscr{Y}^0}$ takes the global maximum at $\mathbf{y}^0$.
\end{proof}
\begin{lemma}\label{rewrite-S}
The function $S:\mathscr{U}\rightarrow \mathbb{C}$ can be rewritten as
\begin{equation*}
S(\mathbf{y})=i\mathrm{Li}_2(-e^{y_1})+i\mathrm{Li}_2(-e^{-y_2})+3i\mathrm{Li}_2(-e^{y_3})+i\mathbf{y}^{\rm T}Q\mathbf{y}+i\frac{y_2^2}{2}+\mathbf{y}^{\rm T}\mathscr{W}+i\frac{\pi^2}{6}.
\end{equation*}
\end{lemma}
\begin{proof}
It is proven by transforming $S(\mathbf{y})$ using the inversion relation formula of Theorem \ref{classical-dilog} (1) where $z=-e^{y_2}$.
\end{proof}
\begin{lemma}\label{relation-between-S-and-Vol}
\begin{equation*}
\Re(S)(\mathbf{y}^0)=-\mathrm{Vol}(S^3\backslash 7_3)    
\end{equation*}
holds.
\end{lemma}
\begin{proof}
By Lemma \ref{rewrite-S}, for all $\mathbf{y}\in\mathscr{U}$,
\begin{equation*}
\Re(S)(\mathbf{y})=-\Im(\mathrm{Li}_2(-e^{y_1}))-\Im(\mathrm{Li}_2(-e^{y_2}))-3\Im(\mathrm{Li}_2(-e^{y_3}))-\Im\left(\mathbf{y}^{\rm T}Q\mathbf{y}+\frac{y_2^2}{2}\right)+
\Re(\mathbf{y}^{\rm T}\mathscr{W}) .
\end{equation*}

For $z\in\mathbb{R}+i\mathbb{R}_{>0}$, the hyperbolic volume of the ideal hyperbolic tetrahedron with the complex shape $z$ can be expressed using the Bloch-Wigner function $D$ as 
$D(z)=\Im(\mathrm{Li}_2(z))+\arg(1-z)\log|z|$.

For $z_1=-e^{y_1}=-e^{x_1+id_1}$, by the relations in Section \ref{complex-gluing-equation}, $\arg(1-z_1)\log|z_1|=-2\pi c_1 x_1$.

Similarly, for $z_2=-e^{-y_2}=-e^{-x_2-id_2}$, $\arg(1-z_2)\log|z_2|=2\pi b_2 x_2$ and for $z_3=-e^{y_3}=-e^{x_3+id_3}$, $\arg(1-z_3)\log|z_3|=-2\pi c_3 x_3$.

Therefore, for $\mathbf{y}\in\mathscr{U}$,
\begin{equation*}
\Re(S)(\mathbf{y})=-D(z_1)-2\pi c_1 x_1-D(z_2)+2\pi b_2 x_2-3D(z_3)-6\pi c_3 x_3-2\mathbf{x}^{\rm T}Q\mathbf{d}-x_2d_2+\mathbf{x}^{\rm T}\mathscr{W} . 
\end{equation*}

Since $\mathbf{z}^0$ is the complex shape structure corresponding to the complete hyperbolic structure on the ideal tetrahedral decomposition $X$, and  $z_3^0=z_4^0$, $z_5^0={z_3^{\prime\prime}}^0$,
using the properties of the Bloch-Wigner function described in Theorem \ref{bloch-hyperbolic-volume}, we can obtain
\begin{eqnarray*}
-\mathrm{Vol}(S^3\backslash 7_3)&=& -D(z_1^0)-D(z_2^0)-D(z_3^0)-D(z_4^0)-D(z_5^0)\\
&=&-D(z_1^0)-D(z_2^0)-D(z_3^0)-D(z_3^0)-D({z_3^{\prime\prime}}^0)\\
&=&-D(z_1^0)-D(z_2^0)-3D(z_3^0).
\end{eqnarray*}

Therefore, let
\begin{equation*}
\mathcal{J}:=\begin{pmatrix}
-2\pi c_1^{0}\\
2\pi b_2^{0}\\
-6\pi c_3^{0}\\
\end{pmatrix}
+\mathscr{W}
-2Q\mathbf{d}^0+\begin{pmatrix}
0\\
-d_2^0\\
0
\end{pmatrix},
\end{equation*}
then it is sufficient to prove $(\mathbf{x}^0)^{\rm T}\cdot\mathcal{J}=0$.

Since $d_2^0=\pi(1-2a_2^{0})=2\pi(b_2^{0}+c_2^{0})$, $2\pi b_2^{0}-d_2^0=-2\pi c_2^{0}$. Thus,
\begin{eqnarray*}
\mathcal{J}=-\begin{pmatrix}
2\pi c_1^{0}\\
2\pi c_2^{0}\\
6\pi c_3^{0}
\end{pmatrix}
+\begin{pmatrix}
-\pi\\
0\\
\pi
\end{pmatrix}+\begin{pmatrix}
-2 & 1 & \\
1 & 0 & -1 \\
 & -1 & -1 
\end{pmatrix}
\begin{pmatrix}
d_1^0\\
d_2^0\\
d_3^0
\end{pmatrix}.
\end{eqnarray*}

Since $d_1^0=-\pi(1-2a_1^{0})$, $d_2^0=\pi(1-2a_2^{0})$, $d_3^0=-\pi(1-2a_3^0)$,
\begin{eqnarray*}
\mathcal{J}&=&\begin{pmatrix}
-2\pi c_1^{0}-\pi +2\pi(1-2a_1^{0})+\pi-2\pi a_2^{0}\\
-2\pi c_2^{0}-\pi+2\pi a_1^{0}+\pi-2\pi a_3^{0}\\
-6\pi c_3^{0}+\pi-\pi+2\pi a_2^{0}+\pi-2\pi a_3^{0}
\end{pmatrix}
=\begin{pmatrix}
2\pi-2\pi c_1^{0}-4\pi a_1^{0}-2\pi a_2^{0}\\
-2\pi c_2^{0}+2\pi a_1^{0}-2\pi a_3^{0}\\
\pi-6\pi c_3^{0}+2\pi a_2^{0}-2\pi a_3^{0}
\end{pmatrix}\\
&=&\begin{pmatrix}
4\pi(b_1^0+c_1^{0})-2\pi c_1^{0}-2\pi a_2^{0}\\
-2\pi c_2^{0}+2\pi a_1^{0}-2\pi a_3^{0}\\
\pi-6\pi c_3^{0}+2\pi a_2^{0}-2\pi a_3^{0}
\end{pmatrix}
=\begin{pmatrix}
2\pi (2b_1^{0}+c_1^{0}-a_2^{0})\\
2\pi (-c_2^{0}+a_1^{0}-a_3^{0})\\
2\pi(\frac{1}{2}-3c_3^{0}+a_2^{0}-a_3^{0})


\end{pmatrix}.
\end{eqnarray*}

$z_3^0=z_4^0$ leads to $a_3^{0}=a_4^{0}$, $b_3^{ 0}=b_4^{ 0}$, $c_3^{ 0}=c_4^{ 0}$ and $z_5^0={z_3^{\prime\prime}}^0$ leads to $a_3^0=c_5^0$, $c_3^0=b_5^0$, $b_3^0=a_5^0$.

Therefore, by the constraints in the definition of $\mathscr{A}_X$,
\begin{equation*}
\left\{
\begin{aligned}
b_1^{0}+b_2^{0}&=a_3^{0}\\
a_2^{0}&=2b_1^{0}+c_1^{0}\\
a_2^{0}+b_3^{0}&=2c_3^{0}\\
\end{aligned}
\right.
\end{equation*}
holds.

Then \begin{eqnarray*}
-c_2^{0}+a_1^{0}-a_3^{0}&=&-c_2^{0}+a_1^{0}-(b_1^{0}+b_2^{0})\quad(\because b_1^{0}+b_2^{0}=a_3^{0})\\
&=&-\left(\frac{1}{2}-a_2^{0}\right)+a_1^{0}-b_1^{0}\\
&=& -\frac{1}{2}+2b_1^{0}+c_1^{0}+a_1^{0}-b_1^{0} \quad (\because a_2^{0}=2b_1^{0}+c_1^{0})\\
&=&0,
\end{eqnarray*}
\begin{eqnarray*}
\frac{1}{2}-3c_3^{0}+a_2^{0}-a_3^{0}&=&\frac{1}{2}-c_3^{0}-(a_2^{0}+b_3^{0})+a_2^{0}-a_3^{0}\quad (\because a_2^{0}+b_3^{0}=2c_3^{0})\\
&=&\frac{1}{2}-c_3^{0}-b_3^{0}-a_3^{0}=0,
\end{eqnarray*}
thus, $\mathcal{J}=0$. The above proves the lemma.
\end{proof}
Thereafter, let $r_0>0$ and $\gamma =\left\{\mathbf{y}\in\mathscr{Y}^0\mid\|\mathbf{y}-\mathbf{y}^0\|\le r_0\right\}$ be the $3$-dimensional ball containing $\mathbf{y}^0$ in $\mathscr{Y}^0$.

\subsection{Asymptotic expansion of the integral on $\mathscr{Y}^0$}
The following is proven in a similar way to the proof of Proposition 7.9. in \cite{MR3945172}.
\begin{prop}\label{rho}

There exists a constant $\rho\in\mathbb{C}^\ast$ such that the following holds.

If $\lambda\rightarrow \infty$,
\begin{equation*}
\int_{\gamma} d\mathbf{y} e^{\lambda S(\mathbf{y})}=\rho \lambda^{-\frac{3}{2}}\exp(\lambda S(\mathbf{y^0}))(1+o_{\lambda\rightarrow\infty}(1)).
\end{equation*}

In particular,
\begin{equation*}
\frac{1}{\lambda}\log\left|\int_{\gamma}d\mathbf{y}e^{\lambda S(\mathbf{y})}\right|\underset{\lambda\rightarrow\infty}{\longrightarrow} \Re S(\mathbf{y}^0)=-\mathrm{Vol}(S^3\backslash 7_3).
\end{equation*}
\end{prop}
\begin{proof}
We apply the saddle point method of Theorem \ref{Fedoryuk}. In Theorem \ref{Fedoryuk}, let $m=3$, $\gamma^3=\gamma$, $z=\mathbf{y}$, $z^0=\mathbf{y}^0$, $D=\mathscr{U}$, $f=1$ and $S$ is defined by the equation \eqref{definition-of-S}. We check the conditions for the application of Theorem \ref{Fedoryuk}.

1) $\mathbf{y}^0$ is the interior point of $\gamma$ by the configuration.

2) $\max_{\gamma} \Re S$ is attained only at $\mathbf{y}^0$ by Lemma \ref{strict-max-of-ReS}.

3) By Lemma \ref{completeness-bijection}, $\nabla S(\mathbf{y}^0)=0$.

4) By Lemma \ref{hessian}, $\det\mathrm{Hess}(\mathbf{y}^0)\neq 0$.

Therefore, by setting $\rho\coloneqq \frac{(2\pi)^\frac{3}{2}}{\sqrt{\det\mathrm{Hess}(S)(\mathbf{y}^0)}}\in\mathbb{C}^\ast$, the first statement is proven by Theorem \ref{Fedoryuk}.

The second statement follows by Lemma \ref{relation-between-S-and-Vol}.
\end{proof}
Next, we calculate the remainder term, that is, the upper bound of the integral on $\mathscr{Y}^0\backslash\gamma$, which excludes the compact ball from the unbounded integral domain.

The following is proven in the same way as the proof of Lemma 7.10 in \cite{MR3945172}.
\begin{lemma}\label{upper-bound}
There exist constants $A,B >0$ such that the following holds. For all $\lambda > A$,
\begin{equation*}
\left|\int_{\mathscr{Y}^0\backslash \gamma}d\mathbf{y}e^{\lambda S(\mathbf{y})}\right|\le Be^{\lambda M},
\end{equation*}
where $M\coloneqq \max_{\partial\gamma}\Re S$.
\end{lemma}
\begin{proof}
First, we perform a variable transformation to $3$-dimensional spherical coordinates.

\begin{equation*}
\mathbf{y}\in\mathscr{Y}^0\backslash\gamma\iff r\Vec{\eta}\in(r_0,\infty)\times\mathbb{S}^{2}
\end{equation*}

Then for all $\lambda>0$, 
\begin{equation*}
\int_{\mathscr{Y}^0\backslash\gamma}d\mathbf{y}e^{\lambda S(\mathbf{y})}=\int_{\mathbb{S}^2}d \mathrm{vol}_{\mathbb{S}^2}\int_{r_0}^\infty r^2e^{\lambda S(r\Vec{\eta})}dr.
\end{equation*}


Therefore, for all $\lambda >0$,
\begin{equation*}
\left|\int_{\mathscr{Y}^0\backslash\gamma}d\mathbf{y}e^{\lambda S(\mathbf{y})}\right|\le 4\pi \sup_{\Vec{\eta}\in\mathbb{S}^2}\int_{r_0}^\infty 
r^2 e^{\lambda \Re(S)(r\Vec{\eta})}dr. 
\end{equation*}
Let $\Vec{\eta}\in\mathbb{S}^2$ and $f=f_{\Vec{\eta}}\coloneqq (r\mapsto \Re(S)(r\Vec{\eta}))$ be the function which restricts $\Re (S)$ on $(r_0,\infty)\Vec{\eta}$.
We find an upper bound for $\int_{r_0}^\infty r^2 e^{\lambda f(r)}$.
By Lemma \ref{RS-concave}, since $\Re(S)$ is strictly concave and $f$ is its restriction on the convex set, $f$ is also strictly concave on $(r_0,\infty)$ (even on $[0,\infty)$).

The function which represents the slope $N:[r_0,\infty)\rightarrow \mathbb{R}$ is defined as $N(r)\coloneqq \frac{f(r)-f(r_0)}{r-r_0}$ for $r>r_0$, $N(r_0):=f^\prime(r_0)$.

The function $N$ is of class $C^1$ and satisfies $N^\prime(r):=\frac{f^\prime(r)-N(r)}{r-r_0}$ for $r>r_0$.
Since $f$ is strictly concave, for any $r\in(r_0,\infty)$, 
$f^\prime(r)<N(r)$, and $N$ decreases on this interval.
Therefore,
\begin{equation*}
\int_{r_0}^\infty r^2 e^{\lambda f(r)}dr=e^{\lambda f(r_0)}\int_{r_0}^\infty r^2e^{\lambda N(r)(r-r_0)}dr\le e^{\lambda f(r_0)}\int_{r_0}^\infty r^2 e^{\lambda N(r_0)(r-r_0)}dr.
\end{equation*}
Note that $N(r_0)=f^\prime(r_0)<0$ by Lemma \ref{RS-concave}, \ref{strict-max-of-ReS}. 

By repeating the integration by parts, we obtain 
\begin{equation*}
\int_{r_0}^\infty r^2 e^{\lambda N(r_0)(r-r_0)}dr=\frac{1}{(\lambda N(r_0))^3}\sum_{k=0}^{2}(-1)^{1-k}\frac{2!}{k!}(\lambda N(r_0))^kr_0^k.
\end{equation*}

Furthermore, since $N(r_0)=f^\prime(r_0)=\langle(\nabla \Re(S)(r_0\Vec{\eta}));\Vec{\eta}\rangle$ and $S$ is holomorphic, $(\Vec{\eta}\mapsto N(r_0)=f^\prime_{\Vec{\eta}}(r_0))$ is a continuous map from $\mathbb{S}^2$ to $\mathbb{R}_{<0}$. Thus, there exist some constants $m_1, m_2>0$ such that  
for all $\Vec{\eta}\in\mathbb{S}^2$, $0<m_1\le |N(r_0)|\le m_2$.

Therefore, the following inequality holds for all $\lambda > \frac{1}{m_1r_0}$.
\begin{eqnarray*}
\int_{r_0}^\infty r^{2}e^{\lambda f(r)}dr &\le& e^{\lambda f(r_0)}\frac{1}{(\lambda N(r_0))^3}\sum_{k=0}^{2}(-1)^{1-k}\frac{2!}{k!}(\lambda N(r_0))^kr_0^k\\
&\le & e^{\lambda f(r_0)}\left|\frac{1}{(\lambda N(r_0))^3}\sum_{k=0}^{2}(-1)^{1-k}\frac{2!}{k!}(\lambda N(r_0))^kr_0^k\right|\\
&\le & e^{\lambda f(r_0)}\frac{1}{|\lambda N(r_0)|^3}\sum_{k=0}^{2}2!|\lambda N(r_0)r_0|^k\\
&\le & e^{\lambda f(r_0)}\frac{3!|\lambda N(r_0)r_0|^3}{|\lambda N(r_0)|^3}=3!r_0^3e^{\lambda f(r_0)}.
\end{eqnarray*}
For all $\Vec{\eta}\in\mathbb{S}^2$, there is a constant $C>0$ independent of $\lambda$ and $\Vec{\eta}$ such that for all $\lambda>\frac{1}{m_1r_0}$, $\int_{r_0}^\infty r^2 e^{\lambda f_{\Vec{\eta}}(r)}dr\le Ce^{\lambda f_{\Vec{\eta}}(r_0)}$. Thus, let $M\coloneqq \max_{\partial\gamma}\Re S$, then for all $\lambda >\frac{1}{m_1r_0}$,
\begin{equation*}
\left|\int_{\mathscr{Y}^0\backslash \gamma}d\mathbf{y}e^{\lambda S(\mathbf(y)}\right|\le 4\pi \sup_{\Vec{\eta}\in\mathbb{S}^2}\int_{r_ 0}^\infty r^2e^{\lambda \Re(S)(r\Vec{\eta})}dr\le 4\pi Ce^{\lambda M}
\end{equation*}
holds.
Thus, by setting $A:=\frac{1}{m_1r_0}$ and $B:=4\pi C$, the lemma is proven. 
\end{proof}
The following is proven in a similar way to the proof of Proposition 7.11 in \cite{MR3945172}.
\begin{prop}
For $\rho\in\mathbb{C}^\ast$, which is identical to the one in Proposition \ref{rho},
if $\lambda\rightarrow\infty$, 
\begin{equation*}
\int_{\mathscr{Y}^0}d\mathbf{y} e^{\lambda S(\mathbf{y})}=\rho\lambda^{-\frac{3}{2}}\exp{(\lambda S(\mathbf{y}^0))}(1+o_{\lambda\rightarrow\infty}(1)).
\end{equation*}
In particular, 
\begin{equation*}
\frac{1}{\lambda}\log{\left|\int_{\mathscr{Y}^0}d\mathbf{y} e^{\lambda S(\mathbf{y})}\right|}\underset{\lambda\rightarrow\infty}{\longrightarrow}\Re S(\mathbf{y}^0)=-\mathrm{Vol}(S^3\backslash 7_3).
\end{equation*}
\end{prop}
\begin{proof}
As in the proof of Proposition \ref{rho}, the second statement follows by the first.

We prove the first statement.

By Lemma \ref{upper-bound}, for all $\lambda > A$,
$\left|\int_{\mathscr{Y}^0\backslash\gamma}d\mathbf{y}e^{\lambda S(\mathbf{y})}\right|\le Be^{\lambda M}$
holds. 

By Lemma \ref{RS-concave}, \ref{strict-max-of-ReS}, since $M<\Re (S)(\mathbf{y}^0)$,
\begin{equation*}
\int_{\mathscr{Y}^0\backslash \gamma}d\mathbf{y} e^{\lambda S(\mathbf{y})}=o_{\lambda\rightarrow\infty}(\lambda^{-\frac{3}{2}}\exp{(\lambda S(\mathbf{y}^0))}).   
\end{equation*}

Therefore, the first statement follows by Proposition \ref{rho} and the following equation.
\begin{equation*}
\int_{\mathscr{Y}^0} d\mathbf{y}e^{\lambda S(\mathbf{y})}=\int_{\gamma} d\mathbf{y}e^{\lambda S(\mathbf{y})}+\int_{\mathscr{Y}^0\backslash \gamma}d\mathbf{y}e^{\lambda S(\mathbf{y})}.
\end{equation*}
\end{proof}
\subsection{Extension of the asymptotic expansion to the quantum dilogarithm}
For $\mathsf{b}>0$, a new potential function $S_{\mathsf{b}}:\mathscr{U}\rightarrow\mathbb{C}$, which is a holomorphic function of three complex variables, is defined by
\begin{equation*}
S_{\sf b}(\mathbf{y}):=i\mathbf{y}^{\rm T}Q\mathbf{y}+\mathbf{y}^{\rm T}\mathscr{W}+2\pi\mathsf{b}^2\mathrm{Log}\left(\frac{\Phi_{\mathsf{b}}(\frac{y_2}{2\pi\mathsf{b}})}{\Phi_{\sf b}(\frac{y_1}{2\pi\mathsf{b}})\Phi_{\mathsf{b}}(\frac{y_3}{2\pi\mathsf{b}})^3}\right).
\end{equation*}

\begin{lemma}[ \cite{MR3945172}, Lemma 7.12 ]
For all $\mathsf{b}\in(0,1)$, for all $y\in\mathbb{R}+i(0,\pi)$,
\begin{equation*}
\Re\left(\mathrm{Log}\left(\Phi_{\sf b}\left(\frac{-\overline{y}}{2\pi\mathsf{b}}\right)\right)-\left(\frac{-i}{2\pi\mathsf{b}^2}\mathrm{Li}_2(-e^{-\overline{y}})\right)\right)
=\Re\left(\mathrm{Log}\left(\Phi_{\sf b}\left(\frac{y}{2\pi\mathsf{b}}\right)\right)-\left(\frac{-i}{2\pi\mathsf{b}^2}\mathrm{Li}_2(-e^y)\right)\right).
\end{equation*}
\end{lemma}
\begin{lemma}[ \cite{MR3945172}, Lemma 7.13 ]\label{upperbound-of-dif}
For all $\delta>0$, there exists a constant $B_{\delta}>0$ such that the following holds.
For all $\mathsf{b}\in(0,1)$, for all $y\in\mathbb{R}+i[\delta,\pi-\delta]$,
\begin{equation*}
\left|\Re\left(\mathrm{Log}\left(\Phi_{\mathsf{b}}\left(\frac{y}{2\pi\mathsf{b}}\right)\right)-\left(\frac{-i}{2\pi\mathsf{b}^2}\mathrm{Li}_2(-e^y)\right)\right)\right|\le B_{\delta}\mathsf{b}^2.
\end{equation*}

Furthermore, $B_{\delta}$ is expressed in the form $B_\delta=\frac{C}{\delta}+C^\prime$ for $C$, $C^\prime>0$.
\end{lemma}
\begin{lemma}[ \cite{MR3945172}, Lemma 7.14 ]\label{upperbound-of-dif-negative}
For all $\delta > 0$, there exists a constant $B_\delta>0$ (which is identical to the one in Lemma \ref{upperbound-of-dif}) such that the following hods.

For all $\mathsf{b}\in(0,1)$, for all $y\in\mathbb{R}-i[\delta,\pi-\delta]$, 
\begin{equation*}
\left|\Re\left(\mathrm{Log}\left(\Phi_{\mathsf{b}}\left(\frac{y}{2\pi\mathsf{b}}\right)\right)-\left(\frac{-i}{2\pi\mathsf{b}^2}\mathrm{Li}_2(-e^y)\right)\right)\right|\le B_\delta \mathsf{b}^2.
\end{equation*}
\end{lemma}
The following holds in a similar way to the proof of Proposition 7.15 in \cite{MR3945172}.
\begin{prop}\label{limit-Sb}
There exists a constant $\rho^\prime\in\mathbb{C}^\ast$ such that the following holds.
If $\mathsf{b}\rightarrow 0^+$,
\begin{eqnarray*}
\int_{\mathscr{Y}^0}d\mathbf{y}e^{\frac{1}{2\pi\mathsf{b}^2}S_{\mathsf{b}}(\mathbf{y})}&=&\int_{\mathscr{Y}^0}d\mathbf{y}e^{\frac{i\mathbf{y}^{\rm T}Q\mathbf{y}+\mathbf{y}^{\rm T}\mathscr{W}}{2\pi\mathsf{b}^2}}\frac{\Phi_{\mathsf{b}}\left(\frac{y_2}{2\pi\mathsf{b}}\right)}{\Phi_{\mathsf{b}}\left(\frac{y_1}{2\pi\mathsf{b}}\right)\Phi_{\mathsf{b}}\left(\frac{y_3}{2\pi\mathsf{b}}\right)^3}\\
&=& e^{\frac{1}{2\pi\mathsf{b}^2}S(\mathbf{y}^0)}\left(\rho^\prime\mathsf{b}^3\left(1+o_{\mathsf{b}\rightarrow 0^+}(1)\right)+O_{\mathsf{b}\rightarrow 0^{+}}(1)\right).
\end{eqnarray*}
In particular, 
\begin{equation*}
2\pi\mathsf{b}^2\log\left|\int_{\mathscr{Y}^0}d\mathbf{y}e^{\frac{1}{2\pi\mathsf{b}^2}S_{\mathsf{b}}(\mathbf{y})}\right|\underset{\mathsf{b}\rightarrow 0^+}{\longrightarrow}\Re S(\mathbf{y}^0)=-\mathrm{Vol}(S^3\backslash 7_3).
\end{equation*}
\end{prop}
\begin{proof}
The second statement follows by the first statement and the fact that 
\begin{equation*}
\rho^\prime \mathsf{b}^3(1+o_{\mathsf{b}\rightarrow 0^+}(1))+O_{\mathsf{b}\rightarrow 0^+}(1)
\end{equation*}
is a polynomial in $\mathsf{b}$ if $\mathsf{b}\rightarrow 0^+$.

We prove the first statement.
The integral over $\mathscr{Y}^0$ is divided into the integral over the compact domain $\gamma$ and the integral over the non-bounded domain $\mathscr{Y}^0\backslash\gamma$.

Note that there exists a constant $\delta$ such that for all $\mathbf{y}=(y_1,y_2,y_3)\in\mathscr{Y}^0$, $\Im(y_1),\Im(y_3)\in[-(\pi-\delta),-\delta]$, $\Im(y_2)\in[\delta,\pi-\delta]$.
Let $(\zeta_1,\zeta_2,\zeta_3)\coloneqq (-1,1,-3)$. By Lemma \ref{upperbound-of-dif}, \ref{upperbound-of-dif-negative},
\begin{eqnarray}
& &\left|\Re\left(\frac{1}{2\pi\mathsf{b}^2}S_{\mathsf{b}}(\mathbf{y})-\frac{1}{2\pi\mathsf{b}^2}S(\mathbf{y})\right)\right|\nonumber\\
&=&\left|\Re\left(
\sum_{j=1}^3\zeta_j\left(\mathrm{Log}\left(\Phi_{\mathsf{ b}}\left(\frac{y_j}{2\pi\mathsf{b}}\right)\right)-\left(\frac{-i}{2\pi\mathsf{b}^2}\mathrm{Li}_2(-e^{y_j})\right)\right)\right)\right|\nonumber\\
&\le&\sum_{j=1}^3|\zeta_j|\left|\Re\left(\mathrm{Log}\left(\Phi_{\mathsf{b}}\left(\frac{y_j}{2\pi\mathsf{b}}\right)\right)-\left(\frac{-i}{2\pi\mathsf{b}^2}\mathrm{Li}_2(-e^{y_j})\right)\right)\right|\nonumber\\
&\le&5B_\delta \mathsf{b}^2.\label{upperbound-5Bb^2}
\end{eqnarray}
Focusing on the compact domain $\gamma$, we prove 
\begin{equation*}
\int_{\gamma} d\mathbf{y}e^{\frac{1}{2\pi\mathsf{b}^2}S_{\mathsf{b}}(\mathbf{y})}=e^{\frac{1}{2\pi\mathsf{b}^2}S(\mathbf{y}^0)}\left(\rho^\prime\mathsf{b}^3(1+o_{\mathsf{b}\rightarrow 0^+}(1))+O_{\mathsf{b}\rightarrow 0^+}(1)\right).
\end{equation*}
By Proposition \ref{rho}, setting $\lambda=\frac{1}{2\pi\mathsf{b}^2}$, $\rho^\prime\coloneqq\rho(2\pi)^{\frac{3}{2}}$, it is sufficient to prove 
\begin{equation*}
\int_\gamma d\mathbf{y}e^{\frac{1}{2\pi\mathsf{b}^2}S(\mathbf{y})}\left(e^{\frac{1}{2\pi\mathsf{b}^2}(S_{\mathsf{b}}(\mathbf{y})-S(\mathbf{y}))}-1\right)=e^{\frac{1}{2\pi\mathsf{b}^2}S(\mathbf{y}^0)}O_{\mathsf{b}\rightarrow 0^+}(1).
\end{equation*}
This equality follows by the upper bound $5B_\delta\mathsf{b}^2$ of the inequality \eqref{upperbound-5Bb^2}, the compactness of $\gamma$, and Lemma \ref{strict-max-of-ReS}.

Finally, for the integral on the unbounded domain $\mathscr{Y}^0\backslash\gamma$, we prove
\begin{equation*}
\int_{\mathscr{Y}^0\backslash\gamma}d\mathbf{y} e^{\frac{1}{2\pi\mathsf{b}^2}S_{\mathsf{b}}(\mathbf{y})}=e^{\frac{1}{2\pi\mathsf{b}^2}S(\mathbf{y}^0)}O_{\mathsf{b}\rightarrow 0^+}(1).
\end{equation*}

By the proof of Lemma \ref{upper-bound}, for all $\mathsf{b}<(2\pi A)^{-\frac{1}{2}}$,
\begin{equation*}
\int_{\mathscr{Y}^0\backslash\gamma}d\mathbf{y}e^{\frac{1}{2\pi\mathsf{b}^2}\Re(S)(\mathbf{y})}\le Be^{\frac{1}{2\pi\mathsf{b}^2}M}.
\end{equation*}

Furthermore, for all $\mathsf{b}\in(0,1)$ and $\mathbf{y}\in\mathscr{Y}^0\backslash\gamma$
\begin{equation*}
e^{\frac{1}{2\pi\mathsf{b}^2}\Re(S_{\mathsf{b}}(\mathbf{y})-S(\mathbf{y}))}\le e^{5B_\delta\mathsf{b}^2}.
\end{equation*}

Let $v\coloneqq\frac{\Re(S)(\mathbf{y}^0)-M}{2}$. Then for all $\mathsf{b}>0$ smaller than both $(2\pi A)^{-\frac{1}{2}}$ and $\left(\frac{v}{10\pi B_\delta}\right)^{\frac{1}{4}}$,
\begin{eqnarray*}
\left|\int_{\mathscr{Y}^0\backslash\gamma}d\mathbf{y}e^{\frac{1}{2\pi\mathsf{b}^2}S_{\mathsf{b}}(\mathbf{y})}\right|&=&\left|\int_{\mathscr{Y}^0\backslash\gamma}
d\mathbf{y}e^{\frac{1}{2\pi\mathsf{b}^2}S(\mathbf{y})}e^{\frac{1}{2\pi\mathsf{b}^2}(S_{\mathsf{b}}(\mathbf{y})-S(\mathbf{y}))}\right|  \\
&\le & \int_{\mathscr{Y}^0\backslash\gamma}d\mathbf{y}e^{\frac{1}{2\pi\mathsf{b}^2}\Re(S)(\mathbf{y})}e^{\frac{1}{2\pi\mathsf{b}^2}\Re(S_{\mathsf{b}}(\mathbf{y})-S(\mathbf{y}))}\\
&\le & Be^{\frac{M}{2\pi\mathsf{b}^2}}e^{5B_\delta\mathsf{b}^2}\le Be^{\frac{1}{2\pi\mathsf{b}^2}(M+v)}\\
&=&e^{\frac{1}{2\pi\mathsf{b}^2}S(\mathbf{y}^0)}O_{\mathsf{b}\rightarrow0^+}(1).
\end{eqnarray*}
The above proves the first statement.
\end{proof}
\subsection{From $\mathsf{b}$ to $\hbar$}
For all $\mathsf{b}>0$, $\hbar:=\mathsf{b}^2(1+\mathsf{b}^2)^{-2}>0$ 
was defined.

For $\mathsf{b}>0$, we define a new potential function $S_{\mathsf{b}}^\prime:\mathscr{U}\rightarrow\mathbb{C}$, which is a holomorphic function of three complex variables as follows.
\begin{equation*}
S_{\mathsf{b}}^\prime (\mathbf{y})\coloneqq i\mathbf{y}^{\rm T}Q\mathbf{y}+\mathbf{y}^{\rm T}\mathscr{W}+2\pi\hbar\mathrm{Log}\left(\frac{\Phi_{\mathsf{b}}(\frac{y_2}{2\pi\sqrt{\hbar}})}{\Phi_{\mathsf{b}}(\frac{y_1}{2\pi\sqrt{\hbar}})\Phi_{\mathsf{b}}(\frac{y_3}{2\pi\sqrt{\hbar}})^3}\right).
\end{equation*}
Note that
\begin{equation*}
\left|\mathfrak{J}_X(\hbar,0)\right|=\left|\left(\frac{1}{2\pi\sqrt{\hbar}}\right)^4 \int_{\mathscr{Y}^0}d\mathbf{y} e^{\frac{1}{2\pi\hbar}S_{\mathsf{b}}^\prime(\mathbf{y})}\right|.
\end{equation*}

Let $c_\delta\coloneqq \sqrt{\frac{\delta}{2(\pi-\delta)}}$, then the following holds by Lemma 7.17 of \cite{MR3945172} and its proof.

\begin{lemma}[ \cite{MR3945172}, Lemma 7.17 ]\label{upperbound-of-dif-b}
For all $\delta\in(0,\frac{\pi}{2})$, there exists a constant $C_\delta$ such that the following holds. For all $\mathsf{b}\in(0,c_\delta)$, and for all $y\in\mathbb{R}+i([-(\pi-\delta),-\delta]\cup[\delta,\pi-\delta])$,
\begin{equation*}
\left|\Re\left(\left(\frac{-i}{2\pi\mathsf{b}^2}\mathrm{Li}_2\left(-e^{y(1+\mathsf{b}^2)}\right)\right)-\left(\frac{-i}{2\pi\mathsf{b}^2}(1+\mathsf{b}^2)^2\mathrm{Li}_2(-e^y)\right)\right)\right|\le C_\delta.
\end{equation*}
\end{lemma}

The following holds in a similar way to the proof of Proposition 7.18 in \cite{MR3945172}.
\begin{prop}\label{limit-Sb-prime}
For $\rho^\prime\in\mathbb{C}^\ast$ defined by Proposition \ref{limit-Sb}, the following equation holds if $\hbar\rightarrow 0^+$.
\begin{eqnarray*}
\int_{\mathscr{Y}^0}d\mathbf{y} e^{\frac{1}{2\pi\hbar}S_{\mathsf{b}}^\prime(\mathbf{y})}&=&\int_{\mathscr{Y}^0}d\mathbf{y} e^{\frac{i\mathbf{y}^{\rm T}Q\mathbf{y}+\mathbf{y}^{\rm T}\mathscr{W}}{2\pi\hbar}}\frac{\Phi_{\mathsf{b}}(\frac{y_2}{2\pi\sqrt{\hbar}})}{\Phi_{\mathsf{b}}(\frac{y_1}{2\pi\sqrt{\hbar}})
\Phi_{\mathsf{b}}(\frac{y_3}{2\pi\sqrt{\hbar}})^3}\\
&=& e^{\frac{1}{2\pi\hbar}S(\mathbf{y}^0)}\left(\rho^\prime\hbar^{\frac{3}{2}}(1+o_{\hbar\rightarrow 0^+}(1))+O_{\hbar\rightarrow 0^+}(1)\right).
\end{eqnarray*}
In particular,
\begin{equation*}
(2\pi\hbar)\log\left|\int_{\mathscr{Y}^0}d\mathbf{y} e^{\frac{1}{2\pi\hbar}S_{\mathsf{b}}^\prime(\mathbf{y})}\right|\underset{\hbar\rightarrow 0^+}{\longrightarrow}
\Re S(\mathbf{y}^0)=-\mathrm{Vol}(S^3\backslash 7_3).
\end{equation*}
\end{prop}
\begin{proof}
The second statement follows from the first statement and Lemma \ref{relation-between-S-and-Vol}.

We prove the first statement.

Take $\delta>0$ so that the absolute values of the imaginary part of the coordinates of any $\mathbf{y}\in\mathscr{Y}^0$ are in $[\delta,\pi-\delta]$.
Let $(\zeta_1,\zeta_2,\zeta_3)\coloneqq (-1,1,-3)$.

Then for all $\mathbf{y}\in\mathscr{Y}^0$ and all $\mathsf{b}\in(0,c_\delta)$, by Lemma \ref{upperbound-of-dif},
\ref{upperbound-of-dif-negative}, and \ref{upperbound-of-dif-b},
\begin{eqnarray*}
& &\left|\Re\left(\frac{1}{2\pi\hbar}S_{\mathsf{b}}^\prime(\mathbf{y})-\frac{1}{2\pi\hbar}S(\mathbf{y})\right)\right|\\
&=&\left|\Re\left(
\sum_{j=1}^3\zeta_j\left(\mathrm{Log}\left(\Phi_{\mathsf{b}}\left(\frac{y_j}{2\pi\sqrt{\hbar}}\right)\right)-\left(\frac{-i}{2\pi\hbar}\mathrm{Li}_2(-e^{y_j})\right)\right)\right)\right|\\
&\le& \sum_{j=1}^3|\zeta_j|\left|\Re\left(\mathrm{Log}\left(\Phi_{\mathsf{b}}\left(\frac{y_j(\mathsf{b}^2+1)}{2\pi\mathsf{b}}\right)\right)-\left(\frac{-i}{2\pi\mathsf{b}^2}\mathrm{Li}_2\left(-e^{y_j(1+\mathsf{b}^2)}\right)\right)\right)\right|\\
& &+ \sum_{j=1}^3|\zeta_j|\left|\Re\left(\left(\frac{-i}{2\pi\mathsf{b}^2}\mathrm{Li}_2\left(-e^{y_j(1+\mathsf{b}^2)}\right)\right)-\left(\frac{-i}{2\pi\mathsf{b}^2}
(1+\mathsf{b}^2)^2\mathrm{Li}_2(-e^{y_j})\right)\right)\right|\\
&\le& 5B_{\frac{\delta}{2}}\mathsf{b}^2+5C_\delta\le 5(B_{\frac{\delta}{2}}+C_\delta)
\end{eqnarray*}
holds.

Let $\lambda=\frac{1}{2\pi\hbar}$, and $\hbar$ is taken to be small enough such that 
\begin{equation*}
0<\mathsf{b}<\min \left\{c_\delta, (2\pi A)^{-\frac{1}{2}}, \left(\frac{v}{10\pi(B_{\frac{\delta}{2}}+C_\delta)}\right)^{\frac{1}{2}}\right\},
\end{equation*}
then the remainder of the proof proceeds in a similar way to the proof of Proposition \ref{limit-Sb}.
\end{proof}
By Proposition \ref{limit-Sb-prime}, since $\lim_{\hbar\rightarrow 0^+}2\pi\hbar\log|\mathfrak{J}_X(\hbar, 0)|=-\mathrm{Vol}(S^3\backslash 7_3)$, Theorem \ref{uemura} (3) is proven.



