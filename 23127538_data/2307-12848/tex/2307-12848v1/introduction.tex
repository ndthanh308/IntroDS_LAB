\section{Introduction}
A Topological Quantum Field Theory (TQFT) is a theory that satisfies the axiom of Atiyah, Segal, and Witten \cite{MR1001453,MR0981378,MR0953828}.
It is defined as a functor from the cobordism category to the module category. The Jones-Witten theory is an example of constructing topological invariants of 3-manifolds based on the TQFT. This invariant is given for the principal $SU(2)$ bundle on an oriented compact 3-manifold $M$ by the path integral using the Chern-Simons functional described in the following equation \eqref{CS}.  

Let $P$ be a principal $G$ bundle on $M$ for a simple, connected, simply-connected compact Lie group $G$, and denote by $\mathcal{A}$ the space of connections on $P$, then $\mathcal{A}$ is identified with the space $\Omega^1(M,\mathfrak{g})$ of first-order differential forms on $M$ with values in the Lie ring $\mathfrak{g}$ of $G$.
In this case, the Chern-Simons functional $CS:\mathcal{A}\rightarrow\mathbb{R}$ is defined by the following equation, \begin{equation}
CS(A)=\int_M {\rm{Tr}}\left(A\wedge dA+\frac{2}{3}A\wedge A\wedge A\right),\quad A\in\mathcal{A}.\label{CS}
\end{equation}

The TQFT based on the path integral or the partition function using the Chern-Simons functional is called the Chern-Simons theory. 
The Jones-Witten theory is for $G=SU(2)$, and the path integral is formally 
expressed as 
\begin{equation*}
Z_k(M)=\int_{\mathcal{A}/\mathcal{G}}\exp\left({\frac{\sqrt{-1}k}{4\pi}CS(A)}\right)\mathcal{D}A, 
\end{equation*}
where $k$ is an integer called the level of the theory, and $\mathcal{G}$ denotes the gauge transformation group \cite{MR0990772}. 

Such topological invariants of 3-manifolds based on the path integral are called quantum invariants. In particular, the first mathematically rigorously defined quantum invariant called the WRT invariant is obtained from the TQFT constructed by Reshetikhin and Turaev \cite{MR1036112,MR1091619,MR1292673}. 

Also related to these quantum invariants, a polynomial knot invariant called the colored Jones polynomial is defined as the path integral on the Wilson loop. In particular, for hyperbolic knots in $S^3$, the volume conjecture, that the hyperbolic volume of the knot complement space appears in the asymptotic expansion of the colored Jones polynomial at some radical root of unity has been proposed by Kashaev, Murakami and Murakami \cite{MR1434238, MR1828373} and proved rigorously for several hyperbolic knots, but the conjecture is not completely solved. The volume conjecture suggests a close relationship between topology and hyperbolic geometry. Chen and Yang have also proposed similar volume conjectures for quantum invariants such as the WRT invariant and the Turaev-Viro invariant obtained by TQFT as well \cite{MR3827806}.

 In contrast, the Chern-Simons theory corresponding to the case where the structure group $G$ is noncompact, compared to the case where $G$ is compact, is challenging to analyze and poorly understood. Recently, Andersen and Kashaev have constructed a generalized TQFT that is expected to have a correspondence with the Chern-Simons theory of $G=SL(2,\mathbb{C})$ by using the quantum Teichm\"{u}ller theory, which has the feature of generating unitary representations of the centrally extended mapping class group of a punctured surface into the infinite-dimensional Hilbert space \cite{MR3227503}. 
 A series of conjectures (Conjecture \ref{Andersen-Kashaev}) analogous to the volume conjecture have been proposed for the invariant defined by this TQFT. This conjecture suggests a new relationship between hyperbolic geometry and topology. Theorems similar to this conjecture have been proved for $4_1$ and $5_2$ knots in $S^3$ \cite{MR3227503} and all twist knots in $S^3$ \cite{MR3945172} for a certain tetrahedrad decomposition.
 
In this paper, we compute and explicitly show the invariant based on this TQFT (hereafter referred to as the Teichm\"{u}ller TQFT) for the hyperbolic knot $7_3$ in $S^3$, which has not been investigated in previous studies. By rigorously evaluating the integral using the saddle point method, we prove a conjecture analogous to the volume conjecture for $7_3$ knot in $S^3$ in a reformulated form for a certain tetrahedral decomposition, and obtain the main result, Theorem \ref{uemura}.

This paper is organized as follows. Section \ref{A-K-TQFT} introduces the Teichm\"{u}ller TQFT proposed by Andersen and Kashaev. Section \ref{triangulation} obtains an ideal tetrahedral decomposition of the complementary space $S^3 \backslash 7_3$ and a tetrahedral decomposition of $S^3$ called one vertex H-triangulation such that a knot $7_3$ in $S^3$ is an edge of a single tetrahedron. 
Section \ref{ideal-triangulation} obtains the partition function defined in the Teichm\"{u}ller TQFT for the ideal tetrahedral decomposition of $S^3\backslash 7_3$. Section \ref{one-vertex_Htriangulation} obtains the partition function for one vertex H-triangulation. 
In Section \ref{volume-conjecture-strict-proof}, we prove Theorem \ref{uemura} analogous to the volume conjecture by evaluating the integral rigorously. In Appendix \ref{volume-conjecture}, an approximate numerical analysis also verifies Theorem \ref{uemura} (3).