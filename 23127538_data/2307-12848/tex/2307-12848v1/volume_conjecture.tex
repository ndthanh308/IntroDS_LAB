\section{The approximate evaluation regarding Theorem\ref{uemura} (3)}
\label{volume-conjecture}
In this appendix, we verify that Theorem \ref{uemura} (3) holds by numerical analysis based on the saddle point method. Therefore, this is not a rigorous proof.
\begin{equation*}
\begin{split}
J_X(\hbar,0)&=
\int_{\mathscr{Y}^\prime} d\mathbf{y}^\prime e^{2i\pi({\mathbf{y}^\prime}^{\rm T}Q\mathbf{y}^\prime)}e^{\frac{1}{\sqrt{\hbar}}{\mathbf{y}^\prime}^{\rm T}\mathscr{W}}\frac{\Phi_{\mathsf{b}}(Z^\prime)}{\Phi_{\mathsf{b}}(Y^\prime)\Phi_{\mathsf{b}}^3(W^\prime)}\\
&=\int_{\mathscr{Y}^\prime} d\mathbf{y}^\prime e^{2i\pi({Y^\prime}^2-Y^\prime Z^\prime +Z^\prime W^\prime +\frac{1}{2}{W^\prime}^2)}e^{\frac{\pi}{\sqrt{\hbar}}(-Y^\prime +W^\prime)}\frac{\Phi_{\mathsf{b}}(Z^\prime)}{\Phi_{\mathsf{b}}(Y^\prime)\Phi_{\mathsf{b}}^3(W^\prime)},
\end{split}
\end{equation*}
where
\begin{eqnarray*}
\mathscr{Y}^\prime
&=&\left(\mathbb{R}-\frac{i}{2\sqrt{\hbar}}(1-2a_1)\right)\times\left(\mathbb{R}+\frac{i}{2\sqrt{\hbar}}(1-2a_2)\right)\times\left(\mathbb{R}-\frac{i}{2\sqrt{\hbar}}(1-2a_3)\right)
\end{eqnarray*}
and 
\begin{equation*}
\mathbf{y}^\prime=\begin{bmatrix}
Y^\prime\\
Z^\prime\\
W^\prime
\end{bmatrix},\quad
\mathscr{W}=\begin{bmatrix}
-\pi\\
0\\
\pi
\end{bmatrix},\quad
Q=\begin{bmatrix}
1 & -\frac{1}{2} & 0\\
-\frac{1}{2} & 0 & \frac{1}{2}\\
0 & \frac{1}{2} &\frac{1}{2}
\end{bmatrix}.
\end{equation*}

By Sections 13.1, 13.2 in \cite{MR3227503},
\begin{eqnarray*}
\int_{\mathbb{R}+\frac{i}{2\sqrt{\hbar}}(1-2a_2)} dZ^\prime 
e^{2\pi iZ^\prime (-Y^\prime +W^\prime)}\Phi_{\mathsf{b}}(Z^\prime)
&=&\frac{e^{-\frac{i}{12}\pi(1-4c_{\mathsf{b}}^2)}e^{2\pi ic_{\mathsf{b}}(-Y^\prime+W^\prime)}}{\Phi_{\mathsf{b}}(Y^\prime -W^\prime -c_{\mathsf{b}})}\\
&=& \frac{e^{-\frac{i}{12}\pi(1+\frac{1}{\hbar})}e^{-\frac{\pi}{\sqrt{\hbar}}(-Y^\prime+W^\prime)}}{\Phi_{\mathsf{b}}(Y^\prime -W^\prime -c_{\mathsf{b}})}.
\end{eqnarray*}

Therefore, in order to show that Theorem \ref{uemura} (3) holds, it is sufficient to prove
\begin{equation}
\lim_{\hbar\rightarrow 0^+}2\pi\hbar\log\left|\int_{\mathscr{Y}}dxdy\frac{e^{i\pi(2x^2+y^2)}}{\Phi_{\mathsf{b}}(x)\Phi_{\mathsf{b}}(x-y-c_{\mathsf{b}})\Phi_{\mathsf{b}}^3(y)}\right|=-\rm{Vol}(S^3\backslash 7_3),
\label{volume-limit}
\end{equation}

where
\begin{eqnarray*}
\mathscr{Y}
&\coloneqq &\left(\mathbb{R}-\frac{i}{2\sqrt{\hbar}}(1-2a_1)\right)
\times\left(\mathbb{R}-\frac{i}{2\sqrt{\hbar}}(1-2a_3)\right).
\end{eqnarray*}

We show the above equation by the approximate evaluation based on the saddle point method. 

Transformation of variables such that $x=\frac{x^\prime}{2\pi\mathsf{b}},y=\frac{y^\prime}{2\pi\mathsf{b}}$ leads to 
\begin{equation}
\begin{split}
&\int_{\mathscr{Y}}dxdy\frac{e^{i\pi(2x^2+y^2)}}{\Phi_{\mathsf{b}}(x)\Phi_{\mathsf{b}}(x-y-c_{\mathsf{b}})\Phi_{\mathsf{b}}^3(y)}\\
&=\frac{1}{(2\pi\mathsf{b})^2}\int_{\mathscr{Y}^\prime}dx^\prime dy^\prime\frac{e^{i\pi \left(2\frac{{x^\prime}^2}{(2\pi\mathsf{b})^2}+\frac{{y^\prime}^2}{(2\pi\mathsf{b})^2}\right)}}{\Phi_{\mathsf{b}}(\frac{x^\prime}{2\pi\mathsf{b}})\Phi_{\mathsf{b}}(\frac{x^\prime-y^\prime}{2\pi\mathsf{b}}-c_{\mathsf{b}})\Phi_{\mathsf{b}}^3(\frac{y^\prime}{2\pi\mathsf{b}})},
\end{split}
\label{2-variable-volume-integral}
\end{equation}
where 
\begin{eqnarray*}
\mathscr{Y}^\prime
&\coloneqq &\left(\mathbb{R}-i\pi (1+\mathsf{b}^2)(1-2a_1)\right)
\times\left(\mathbb{R}-i\pi (1+\mathsf{b}^2)(1-2a_3)\right).
\end{eqnarray*}

$c_{\mathsf{b}}=\frac{i(\mathsf{b}+\mathsf{b}^{-1})}{2}=\frac{i\pi(\mathsf{b}^2+1)}{2\pi\mathsf{b}}$ and 
\begin{equation}
\begin{split}
\int_{\mathscr{Y}}dxdy\frac{e^{i\pi(2x^2+y^2)}}{\Phi_{\mathsf{b}}(x)\Phi_{\mathsf{b}}(x-y-c_{\mathsf{b}})\Phi_{\mathsf{b}}^3(y)}&=\frac{1}{(2\pi\mathsf{b})^2}\int_{\mathscr{Y}^\prime}dxdy\frac{e^{i\pi \left(2\frac{x^2}{(2\pi\mathsf{b})^2}+\frac{y^2}{(2\pi\mathsf{b})^2}\right)}}{\Phi_{\mathsf{b}}(\frac{x}{2\pi\mathsf{b}})\Phi_{\mathsf{b}}(\frac{x-y-i\pi(\mathsf{b}^2+1)}{2\pi\mathsf{b}})\Phi_{\mathsf{b}}^3(\frac{y}{2\pi\mathsf{b}})}
\end{split}.
\label{volume-function1}
\end{equation}
By the equation (A.21) in \cite{MR3700060}, since 
\begin{equation*}
\Phi_{\mathsf{b}}\left(\frac{x}{2\pi{\mathsf{b}}}\right)=\exp\left(\frac{1}{2\pi i\mathsf{b}^2}{\rm{Li}}_2\left(-e^x\right)\right)\left(1+O({\mathsf{b}}^2)\right),
\end{equation*} 
\begin{align}
&\eqref{volume-function1}\notag\\
&=\frac{1}{(2\pi\mathsf{b})^2}\int_{\mathscr{Y}^\prime}dxdy\frac{e^{i\pi \left(2\frac{x^2}{(2\pi{\mathsf{b}})^2}+\frac{y^2}{(2\pi\mathsf{b})^2}\right)}}{\exp{\left(\frac{1}{2\pi i\mathsf{b}^2}{\rm{Li}}_2(-e^x)\right)}\exp{\left(\frac{1}{2\pi i\mathsf{b}^2}({\rm{Li}}_2(e^{x-y})+O(\mathsf{b}^2))\right)}\exp\left(\frac{3}{2\pi i{\mathsf{b}}^2}{\rm Li}_2(-e^y)\right)(1+O({\mathsf{b}}^2))}\notag\\
&=\frac{1}{(2\pi{\mathsf{b}})^2}\int_{\mathscr{Y}^\prime}dxdy e^{\frac{1}{2\pi i{\mathsf{b}}^2}\left(-{\rm{Li}}_2(-e^x)-{\rm{Li}}_2(e^{x-y})-3{\rm Li}_2(-e^y)-x^2-\frac{y^2}{2}+O({\mathsf{b}}^2)\right)}(1+O({\mathsf{b}}^2)).\label{volume-function2}
\end{align}
Let
\begin{equation*}
V(x,y)=-{\rm{Li}}_2(-e^x)-{\rm Li}_2(e^{x-y})-3{\rm Li}_2(-e^y)-x^2-\frac{y^2}{2}.
\end{equation*}
Since 
\begin{equation*}
\frac{\partial}{\partial x}{\rm Li}_2(-e^x)=-\log(1+e^x),
\end{equation*}
\begin{equation*}
\begin{split}
\frac{\partial}{\partial x}V(x,y)&=\log(1+e^x)+\log(1-e^{x-y})-2x\\
&=\log{\frac{(1+e^2)(1-e^{x-y})}{e^{2x}}},
\end{split}
\end{equation*}
and 
\begin{equation*}
\begin{split}
\frac{\partial}{\partial y}V(x,y)&=-\log{(1-e^{x-y})}+3\log{(1+e^y)}-y\\
&=\log{\frac{(1+e^y)^3}{(1-e^{x-y})e^y}}.
\end{split}
\end{equation*}

The stationary points such that
\begin{equation*}
\frac{\partial}{\partial x}V(x,y)=\frac{\partial}{\partial y}V(x,y)=0
\end{equation*}
satisfy
\begin{align}
e^{2x}=(1+e^x)(1-e^{x-y}),\label{saddle-point-cond1}\\
(1-e^{x-y})e^{y}=(1+e^y)^3.\label{saddle-point-cond2}
\end{align}

By the equation $\eqref{saddle-point-cond1}$, 
\begin{equation*}
e^y=-\frac{1+e^x}{e^x-e^{-x}-1}.
\end{equation*}

Substituting this into the equation $\eqref{saddle-point-cond2}$,
\begin{equation*}
(e^{2x}-e^x-1)^2=\frac{(1+2e^x)^3}{e^{3x}}.    
\end{equation*}

Let $e^x=t$, then  
\begin{equation*}
t^3(t^2-t-1)^2=(1+2t)^3.
\end{equation*}

Using Mathematica, we solve the equation numerically with respect to $t$,
\begin{equation*}
\begin{split}
t&=-1,\quad -0.566231,\quad 2.712568,\quad -0.446038-0.121232i,\\
&\quad-0.446083+0.121232i,\quad 0.872869-1.511780i,\quad0.872869+1.511780i.
\end{split}
\end{equation*}

However, if $t=-1$, $e^y=0$, and there is no such complex solution $y$. 

Denoting the value of $V(x,y)$ at the stationary point using $t$, we obtain
\begin{equation*}
\begin{split}
f(t):=&-\left\{{\rm Log}(t)\right\}^2-\frac{1}{2}\left\{{\rm Log}\left(\frac{-1-t}{-1-\frac{1}{t}+t}\right)\right\}^2-{\rm Li}_2(-t)\\
&\quad\quad -{\rm Li}_2\left(\frac{1+t-t^2}{1+t}\right)-3{\rm Li}_2\left(\frac{1+t}{-1-\frac{1}{t}+t}\right).
\end{split}
\end{equation*}
\begin{equation*}
\frac{\partial^2}{\partial x^2}V(x,y)=\frac{e^x}{1+e^x}-\frac{e^{x-y}}{1-e^{x-y}}-2,
\end{equation*}
\begin{equation*}
\frac{\partial^2}{\partial y^2}V(x,y)=-\frac{e^{x-y}}{1-e^{x-y}}+3\frac{e^y}{1+e^y}-1,
\end{equation*}
\begin{equation*}
\frac{\partial^2}{\partial y\partial x}V(x,y)=\frac{\partial^2}{\partial x\partial y}V(x,y)=\frac{e^{x-y}}{1-e^{x-y}}.
\end{equation*}

Let $h(t),i(t),j(t)$ be the value of $\frac{\partial^2}{\partial x^2}V(x,y),\frac{\partial^2}{\partial y^2}V(x,y),\frac{\partial^2}{\partial y\partial x}V(x,y)$ at the stationary point, respectively.
In the neighborhood of the stationary point $(x_0,y_0)$, Taylor expansion gives 

\begin{equation*}
V(x,y)=V(x_0,y_0)+\frac{1}{2}\left\{h(t)(x-x_0)^2+2j(t)(x-x_0)(y-y_0)+i(t)(y-y_0)^2\right\}\cdots.
\end{equation*}

In general, suppose that $A$ is a complex symmetric quadratic matrix and $B$ is a positive definite matrix if $A=B+iC$\quad($B,C$ are real matrices). Then 
\begin{equation*}
\int_{\mathbb{R}^2}\exp(-\frac{1}{2}x^tAx) dx=\frac{2\pi}{\sqrt{\det A}}
\end{equation*}
and the integral on the left-hand side converges.

Therefore, if the integral contour is changed to the contour $x=x_0+\mathbb{R},y=y_0+\mathbb{R}$ which passes through the stationary point $(x_0,y_0)$ and is parallel to the real axes and 
both eigenvalues of the matrix
\begin{equation*}
S(t)=
\begin{pmatrix}
\Im h(t) & \Im j(t) \\
\Im j(t) & \Im i(t) \\
\end{pmatrix}
\end{equation*}
are negative,
let
\begin{equation*}
A(t)=\begin{pmatrix}
 h(t) &  j(t) \\
 j(t) &  i(t) \\
\end{pmatrix},
\end{equation*}
then
\begin{equation*}
\int_{(x_0+\mathbb{R})\times(y_0+\mathbb{R})}dxdye^{\frac{1}{2\pi i{\rm{b}}^2}V(x,y)}=e^{\frac{1}{2\pi i{\mathsf{b}}^2}f(t)}\mathsf{b}^2\left(\frac{-4\pi^2 i}{\sqrt{\det A(t)}}+O({\mathsf{b}})\right).
\end{equation*}

In this case, the left-hand side of the equation $\eqref{volume-limit}$ is equal to 
\begin{equation*}
\Im f(t).
\end{equation*}

By the property of Proposition \ref{property-of-Phi_b} (3), if $0<a_1,a_3,a_1-a_3< \frac{1}{2}$, the integrand of the equation \eqref{2-variable-volume-integral} decreases rapidly at infinity, so by the holomorphicity of the integrand and the Bochner-Martinelli formula, the value of the equation \eqref{2-variable-volume-integral} is invariant if we change the contour with $a_1,a_3$ satisfy the above condition. 
In particular, if $\alpha_X\in\mathscr{A}_X$, by the equations \eqref{it4'} and \eqref{it2''}, $a_1-a_3=a_1-(b_1+b_2)=a_1-b_1-(\frac{1}{2}-a_2-c_2)=a_1-b_1-\frac{1}{2}+2b_1+c_1+c_2=-\frac{1}{2}+(a_1+b_1+c_1)+c_2=c_2$, so, certainly, this condition is satisfied.

In particular, if the imaginary parts of $e^{x_0},e^{y_0}$ are negative and the imaginary part of $e^{x_0-y_0}$ is positive, this condition is satisfied. So, we check the existence of such stationary points $(x_0,y_0)$.

Let $t_3=-0.446038-0.121232i$, then 
\begin{align*}
h(t_3)=-3.67157+1.42489i,\\
i(t_3)=-3.0171-0.961697i,\\
j(t_3)=0.948895-1.80189i
\end{align*}
and the eigenvalues of $S(t_3)$ are 
\begin{equation*}
2.39278,-1.9296.
\end{equation*}

Let $t_5=0.872869-1.511780i$. 
Since $e^y=-\frac{t^2+t}{t^2-t-1}$, if 
$t=t_5$, $e^y=-0.537981 - 1.04357 i$.
Since $e^{x-y}=-\frac{t^2-t-1}{t+1}$, 
if $t=t_5$, $e^{x-y}=0.803839 + 1.25082i$.
\begin{align*}
h(t_5)=-0.445662-1.04125i,\\
i(t_5)=1.81348-3.1839i,\\
j(t_5)=-0.87763+0.780285i,
\end{align*}
and the eigenvalues of  $S(t_5)$ are 
\begin{equation*}
-0.787211,-3.43793.
\end{equation*}

Therefore, the stationary point $(x_0,y_0)$ such that the eigenvalues of $S(t)$ are both negative, the imaginary parts of $e^{x_0},e^{y_0}$ are negative, and the imaginary part of $e^{x_0-y_0}$ is positive exists if $t=t_5$. In this case, 
\begin{equation*}
f(t_5)=2.884158080-4.592125697i
\end{equation*}
and  
\begin{equation*}
\Im f(t_5)=-4.592125697. 
\end{equation*} 

On the other hand, the hyperbolic volume of the complementary space of the knot $7_3$ in $S^3$ is 4.592125697$\cdots$, which numerically confirms the equation $\eqref{volume-limit}$.