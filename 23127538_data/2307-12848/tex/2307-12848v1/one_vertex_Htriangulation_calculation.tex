\section{Calculation of the partition function for one vertex H-triangulation of $(S^3, 7_3)$}
\label{one-vertex_Htriangulation}

In this section, we calculate the partition function for one vertex H-triangulation $Y$ of ($S^3$, $7_3$) obtained from Proposition \ref{one_veretex_H-triantulation} in Section \ref{triangulation}.
Since $Y-\Delta_0(Y)$ is homeomorphic to the space $S^3$ minus one point, 
\begin{equation*}
H_2(Y-\Delta_0(Y),\mathbb{Z})\cong H_2(S^{3}\backslash\text{point},\mathbb{Z})\cong H_2(\mathbb{R}^3,\mathbb{Z})\cong H_2(\text{point},\mathbb{Z}) =0.
\end{equation*}

Therefore, $Y$ is admissible.

Let $\alpha_Y=(2\pi a_1,2\pi b_1,2\pi c_1,\ldots,2\pi a_6,2\pi b_6,2\pi c_6)\in\mathscr{S}_Y$.

From Figure \ref{fig:one_vertex_Htriangulation}, the weight of each edge divided by $2\pi$ is obtained as follows, 
\begin{align*}
e_0:&\quad & a_1\\
e_1:&\quad & a_1+a_2+c_2+a_3+c_3+a_4\\
e_2:&\quad & b_1+c_1+b_2+c_2+c_3+c_6\\
e_3:&\quad & b_1+c_1+b_4+c_4+a_5\\
e_4:&\quad & a_3+b_4+a_5+b_5+a_6+c_6\\
e_5:&\quad & b_2+b_3+a_4+b_5+c_5+a_6+b_6\\
e_6:&\quad & a_2+b_3+c_4+c_5+b_6 .
\end{align*}

Let $\tau=(2\pi a_1^\tau,2\pi b_1^\tau,2\pi c_1^\tau,\cdots,2\pi a_6^\tau,2\pi b_6^\tau, 2\pi c_6^\tau)\in\overline{\mathscr{S}_{T_1}}\times\mathscr{S}_{Y\backslash T_1}$ be the extended shape structure which gives the weight function $\omega_{Y,\tau}:\Delta_1(Y)\longrightarrow\mathbb{R}$ such that each weight on $e_1,e_2,\cdots,e_6$ is $2\pi$, and the weight on $e_0$ is $0$.

Then 
\begin{align}
a_1^\tau+a_2^\tau+c_2^\tau+a_3^\tau+c_3^\tau+a_4^\tau=1\label{da1}\\
b_1^\tau+c_1^\tau+b_2^\tau+c_2^\tau+c_3^\tau+c_6^\tau=1\label{da2}\\
b_1^\tau+c_1^\tau+b_4^\tau+c_4^\tau+a_5^\tau=1\label{da3}\\
a_3^\tau+b_4^\tau+a_5^\tau+b_5^\tau+a_6^\tau+c_6^\tau=1\label{da4}\\
b_2^\tau+b_3^\tau+a_4^\tau+b_5^\tau+c_5^\tau+a_6^\tau+b_6^\tau=1\label{da5}\\
a_2^\tau+b_3^\tau+c_4^\tau+c_5^\tau+b_6^\tau=1\label{da6}\\
a_1^\tau=0.\label{da7}
\end{align}

Using $a_i^\tau+b_i^\tau+c_i^\tau=\frac{1}{2}$ for $i=1,2,\ldots,6$,
\begin{empheq}[left={\eqref{da1},\eqref{da2},\eqref{da3},\eqref{da4},\eqref{da5},\eqref{da6},\eqref{da7}\iff\empheqlbrace}]{align}
&a_3^\tau+b_4^\tau=c_5^\tau+b_6^\tau \label{da1''}\\
&a_4^\tau=b_2^\tau+b_3^\tau=a_5^\tau \label{da2''}\\
&a_4^\tau=c_6^\tau \label{da3''}\\
&a_2^\tau=c_3^\tau+a_4^\tau \label{da4''}\\
&a_1^\tau=0.\label{da5''}
\end{empheq}

The following holds in the same way as in \cite{MR3945172} Lemma 6.2.
\begin{lemma}\label{upper-bound-of-Phi_b}
Let $\hbar >0$, $\delta\in(0,\frac{1}{4})$. Then 
$M_{\delta,\hbar}\coloneqq\max_{z\in\mathbb{R}+\frac{i}{\sqrt{\hbar}}[\delta,\frac{1}{2}-\delta]}|\Phi_{\mathsf{b}}(z)|$ is finite.
\end{lemma}
\begin{proof}
Let $\hbar > 0$, $\delta\in(0,\frac{1}{4})$. Suppose $M_{\delta,\hbar}=\infty$. Then there exists a sequence of points $(z_n)_{n\in\mathbb{N}}\in\left(\mathbb{R}+\frac{i}{\sqrt{\hbar}}[\delta,\frac{1}{2}-\delta]\right)^{\mathbb{N}}$ such that $\left|\Phi_{\mathsf{b}}(z_n)\right|\underset{n\rightarrow\infty}{\rightarrow}\infty$.
If $(\Re(z_n))_{n\in\mathbb{N}}$ is bounded, $(z_n)_{n\in\mathbb{N}}$ is on the compact set and by the continuity of $\left|\Phi_{\mathsf{b}}\right|$, $(|\Phi_{\mathsf{b}}(z_n)|)_{n\in\mathbb{N}}$ is bounded, which is a contradiction. 
Therefore there exists a subsequence tending to $-\infty$ or $\infty$ in $(\Re(z_n))_{n\in\mathbb{N}}$, and the images of this sequence by $\left|\Phi_{\mathsf{b}}\right|$ must tend to $\infty$, which contradicts Proposition \ref{property-of-Phi_b}(3).
\end{proof}
Below we calculate the partition function $Z_\hbar(Y,\alpha_Y)$ of $(Y,\alpha_Y)$. 
\begin{theorem}\label{volume-conjecture-2}
The following holds for the partition function $Z_\hbar(Y,\alpha_Y)$ of $(Y,\alpha_Y)$.
\begin{equation*}
\lim_{\alpha_Y\rightarrow \tau}\Phi_{\mathsf{b}}\left(\frac{\pi- \omega_{Y,\alpha_Y}(e_0)}{2\pi i\sqrt{\hbar}}\right)Z_\hbar(Y,\alpha_Y)=e^{-\frac{5}{12}\pi i}e^{-\frac{i\varphi(\tau)}{\hbar}}J_{X}(\hbar,0),
\end{equation*}
where $\varphi(\tau)$ is a real quadratic polynomial of dihedral angles and $J_{X}(\hbar,x)$ is the function defined in Section \ref{ideal-triangulation}.
\end{theorem}
\begin{proof}
\begin{equation*}
\begin{split}
Z_\hbar(Y,\alpha_Y)&=\int_{\mathbb{R}^{12}}dxdydzdwdpdqdrdsdtdudvda\ \langle x,z\lvert \mathsf{T}(a_1,c_1)\rvert x,y\rangle \langle p,y\lvert \mathsf{T}(a_2,c_2)\rvert q,w\rangle \\ 
& \quad\quad\times \langle s,w\lvert \overline{\mathsf{T}}(a_3,c_3)\rvert r,p\rangle\langle u,r\lvert \mathsf{T}(a_4,c_4)\rvert t,z\rangle \langle v,t\lvert \mathsf{T}(a_5,c_5)\rvert a,u\rangle \\
& \quad\quad\times \langle a,q\lvert \mathsf{T}(a_6,c_6)\rvert s,v\rangle \\
&=\int_{\mathbb{R}^{12}}dxdydzdwdpdqdrdsdtdudvda\ \delta(z)\psi_{c_1,b_1}(y-z)e^{2\pi ix(y-z)}e^{-\frac{\pi}{12}i}\\
&\quad\quad\times \delta(p+y-q)\psi_{c_2,b_2}(w-y) e^{2\pi ip(w-y)}e^{-\frac{\pi}{12}i}\delta(r+p-s)\psi_{b_3,c_3}(p-w)\\
&\quad\quad\times e^{\pi i (p-w)^2}e^{-2\pi ir(w-p)}e^{-\frac{\pi}{12}i}\delta(u+r-t)
 \psi_{c_4,b_4}(z-r)e^{2\pi i u(z-r)}e^{-\frac{\pi}{12}i}\\
 & \quad\quad\times \delta(v+t-a)\psi_{c_5,b_5}(u-t)e^{2\pi i v(u-t)}e^{-\frac{\pi}{12}i}\delta(a+q-s)\psi_{c_6,b_6}(v-q)\\
&\quad\quad\times e^{2\pi i a(v-q)}e^{-\frac{\pi}{12}i}.
\end{split}
\end{equation*}

Iterated integration with respect to the variables $z, u, t, a, s$, and $q$, in turn, gives
\begin{equation*}
\begin{split}
Z_\hbar(Y,\alpha_Y)&=\int_{\mathbb{R}^{6}}dxdydwdpdrdv\ e^{-\frac{\pi}{2}}\psi_{c_1,b_1}(y)\psi_{c_2,b_2}(w-y)\psi_{b_3,c_3}(p-w)\psi_{c_4,b_4}(-r)\\
&\quad\quad\times \psi_{c_5,b_5}(-r)\psi_{c_6,b_6}(v-p-y)
 e^{\pi i(p^2+w^2+2y^2+2xy-2rw+2rv-2yv)}.
\end{split} 
\end{equation*}
Since
\begin{equation*}
\delta(y)=\int_{\mathbb{R}}\ e^{2\pi ixy}dx,   
\end{equation*}
\begin{equation*}
\begin{split}
Z_\hbar(Y,\alpha_Y)&=\int_{\mathbb{R}^5}dydwdpdrdv\ e^{-\frac{\pi}{2}i}\delta(y)
\psi_{c_1,b_1}(y)\psi_{c_2,b_2}(w-y)\psi_{b_3,c_3}(p-w)\psi_{c_4,b_4}(-r)\\
&\quad\quad\times\psi_{c_5,b_5}(-r)\psi_{c_6,b_6}(v-p-y)e^{\pi i(p^2+w^2+2y^2-2rw+2rv-2yv)}\\
&=\int_{\mathbb{R}^4}dwdpdrdv\ e^{-\frac{\pi}{2}i}\psi_{c_1,b_1}(0)\psi_{c_2,b_2}(w)\psi_{b_3,c_3}(p-w)\psi_{c_4,b_4}(-r)\psi_{c_5,b_5}(-r)\\
&\quad\quad\times \psi_{c_6,b_6}(v-p)e^{\pi i(p^2+w^2-2rw+2rv)}.
\end{split}    
\end{equation*}
Transformation of variables, 
$\left\{
\begin{aligned}
&A\coloneqq w\\
&B\coloneqq -r\\
&C\coloneqq v-p\\
&D\coloneqq p-w\\
\end{aligned}
\right.$,
gives
$\left\{
\begin{aligned}
&w=A\\
&p=A+D\\
&r=-B\\
&v=A+C+D\\
\end{aligned}
\right.$
and its Jacobian is 
$
\begin{vmatrix}
1 & 0 & 0 & 0\\
1 & 0& 0 & 1 \\
0 & -1 & 0 & 0 \\
1 & 0 & 1 & 1
\end{vmatrix}
=-1
$. Therefore, 
\begin{equation*}
\begin{split}
Z_{\hbar}(Y,\alpha_Y)&=\int_{\mathbb{R}^4}dAdBdCdD\ e^{-\frac{\pi}{2}i}
\psi_{c_1,b_1}(0)\psi_{c_2,b_2}(A)\psi_{b_3,c_3}(D)\psi_{c_4,b_4}(B)
\psi_{c_5,b_5}(B)\\
&\quad\quad\times \psi_{c_6,b_6}(C)e^{\pi i(2A^2-2BC-2BD+2AD+D^2)}.
\end{split}
\end{equation*}
Since
\begin{equation*}
\int_{\mathbb{R}}dC\ \psi_{c_6,b_6}(C)e^{-2\pi iBC}=\widetilde{\psi}_{c_6,b_6}(B)=e^{\pi iB^2}e^{-\frac{\pi}{12}i}\psi_{b_6,a_6}(B),
\end{equation*}
\begin{equation*}
\begin{split}
Z_{\hbar}(Y,\alpha_Y)&=\int_{\mathbb{R}^3}dAdBdD\ e^{-\frac{7}{12}\pi i}\psi_{c_1,b_1}(0)\psi_{c_2,b_2}(A)\psi_{b_3,c_3}(D)\psi_{c_4,b_4}(B)\psi_{c_5,b_5}(B)\\ &\quad\quad\times \psi_{b_6,a_6}(B)e^{\pi i(2A^2+B^2-2BD+D^2+2AD)}\\
&=e^{-\frac{7}{12}\pi i}\int_{\mathbb{R}^3}dAdBdD \psi(-2c_{\mathsf{b}}(c_1+b_1))e^{-4\pi i c_{\mathsf{b}}c_1(-c_{\mathsf{b}}(c_1+b_1))}e^{-\pi ic_{\mathsf{b}}^2\frac{(4(c_1-b_1)+1)}{6}}\\
&\quad\times \psi(A-2c_{\mathsf{b}}(c_2+b_2))e^{-4\pi ic_{\mathsf{b}}c_2(A-c_{\mathsf{b}}(c_2+b_2))}e^{-\pi i c_{\mathsf{b}}^2\frac{(4(c_2-b_2)+1)}{6}}\psi(D-2c_{\mathsf{b}}(b_3+c_3))\\
&\quad\times e^{-4\pi ic_{\mathsf{b}}b_3(D-c_{\mathsf{b}}(b_3+c_3))}e^{-\pi ic_{\mathsf{b}}^2\frac{(4(b_3-c_3)+1)}{6}}\psi(B-2c_{\mathsf{b}}(c_4+b_4))e^{-4\pi ic_{\mathsf{b}}c_4(B-c_{\mathsf{b}}(c_4+b_4))}\\
&\quad\times e^{-\pi i c_{\mathsf{b}}^2\frac{(4(c_4-b_4)+1)}{6}}\psi(B-2c_{\mathsf{b}}(c_5+b_5))e^{-4\pi ic_{\mathsf{b}}c_5(B-c_{\mathsf{b}}(c_5+b_5))}e^{-\pi ic_{\mathsf{b}}^2\frac{4(c_5-b_5)+1}{6}}\\
&\quad\times \psi(B-2c_{\mathsf{b}}(b_6+a_6))e^{-4\pi ic_{\mathsf{b}}b_6(B-c_{\mathsf{b}}(b_6+a_6))}e^{-\pi ic_{\mathsf{b}}^2\frac{4(b_6-a_6)+1}{6}}\\
&\quad\times e^{\pi i(2A^2+B^2-2BD+D^2+2AD)}\\
&=e^{-\frac{7}{12}\pi i}\int_{\mathbb{R}^3}dAdBdD\frac{1}{\Phi_{\mathsf{b}}(-2c_{\mathsf{b}}(c_1+b_1))}\frac{1}{\Phi_{\mathsf{b}}(A-2c_{\mathsf{b}}(c_2+b_2))}
e^{-4\pi ic_{\mathsf{b}}c_2A}\\
&\quad\times \frac{1}{\Phi_{\mathsf{b}}(B-2c_{\mathsf{b}}(b_4+c_4))}e^{-4\pi ic_{\mathsf{b}}c_4B}\frac{1}{\Phi_{\mathsf{b}}(B-2c_{\mathsf{b}}(b_5+c_5))}
e^{-4\pi ic_{\mathsf{b}}c_5B}\\
&\quad \times \frac{1}{\Phi_{\mathsf{b}}(B-2c_{\mathsf{b}}(a_6+b_6))}
e^{-4\pi ic_{\mathsf{b}}b_6B}\frac{1}{\Phi_{\mathsf{b}}(D-2c_{\mathsf{b}}(b_3+c_3))} e^{-4\pi ic_{\mathsf{b}}b_3D} \\
& \quad \times e^{\pi i(2A^2+B^2-2BD+D^2+2AD)}\times\mbox{\textcircled{\scriptsize 2}},
\end{split}
\end{equation*}
where
\begin{equation*}
\begin{split}
\mbox{\textcircled{\scriptsize 2}}&\coloneqq e^{4\pi i c_{\mathsf{b}}^2c_1(c_1+b_1)}
e^{-\pi ic_{\mathsf{b}}^2\frac{(4(c_1-b_1)+1)}{6}}
e^{4\pi ic_{\mathsf{b}}^2c_2(c_2+b_2)}e^{-\pi i c_{\mathsf{b}}^2\frac{(4(c_2-b_2)+1)}{6}}e^{4\pi ic_{\mathsf{b}}^2b_3(b_3+c_3)}\\
&\quad\times e^{-\pi ic_{\mathsf{b}}^2\frac{(4(b_3-c_3)+1)}{6}}e^{4\pi ic_{\mathsf{b}}^2c_4(c_4+b_4)}e^{-\pi i c_{\mathsf{b}}^2\frac{(4(c_4-b_4)+1)}{6}}  e^{4\pi ic_{\mathsf{b}}^2c_5(c_5+b_5)}\\
&\quad \times e^{-\pi i c_{\mathsf{b}}^2\frac{(4(c_5-b_5)+1)}{6}}e^{4\pi ic_{\mathsf{b}}^2b_6(b_6+a_6)}e^{-\pi i c_{\mathsf{b}}^2\frac{(4(b_6-a_6)+1)}{6}} .
\end{split}
\end{equation*}
Since
\begin{equation*}
\Phi_{\mathsf{b}}\left(\frac{\pi-2\pi a_1}{2\pi i\sqrt{\hbar}}\right)=\Phi_{\mathsf{b}}(2c_{\mathsf{b}}a_1-c_{\mathsf{b}})=\Phi_{\mathsf{b}}(-2c_{\mathsf{b}}(b_1+c_1)),
\end{equation*} 
\begin{equation*}
\begin{split}
&\quad \Phi_{\mathsf{b}}\left(\frac{\pi-2\pi a_1}{2\pi i\sqrt{\hbar}}\right)Z_\hbar(Y,\alpha_Y)\\
&=e^{-\frac{7}{12}\pi i}\int_{\mathbb{R}^3}dAdBdD \frac{1}{\Phi_{\mathsf{b}}(A-2c_{\mathsf{b}}(c_2+b_2))}
e^{-4\pi ic_{\mathsf{b}}c_2A}\frac{1}{\Phi_{\mathsf{b}}(B-2c_{\mathsf{b}}(b_4+c_4))}e^{-4\pi ic_{\mathsf{b}}c_4B}\\
&\quad\times\frac{1}{\Phi_{\mathsf{b}}(B-2c_{\mathsf{b}}(b_5+c_5))}
e^{-4\pi ic_{\mathsf{b}}c_5B}\frac{1}{\Phi_{\mathsf{b}}(B-2c_{\mathsf{b}}(a_6+b_6))}e^{-4\pi ic_{\mathsf{b}}b_6B}\frac{1}{\Phi_{\mathsf{b}}(D-2c_{\mathsf{b}}(b_3+c_3))}\\
&\quad\times e^{-4\pi ic_{\mathsf{b}}b_3D} e^{\pi i(2A^2+B^2-2BD+D^2+2AD)}\times\mbox{\textcircled{\scriptsize 2}}.
\end{split}
\end{equation*}

Take $\delta>0$ such that there exists a neighborhood $\mathfrak{U}$ of $\tau$ in $\overline{\mathscr{S}_{T_1}}\times\mathscr{S}_{Y\backslash T_1}$ such that for each $\alpha_Y\in\mathfrak{U}\cap \mathscr{S}_Y$, the values of $15$ variables $a_2,b_2,c_2,\ldots,a_6,b_6,c_6$ of $\alpha_Y$ are in $(\delta,\frac{1}{2}-\delta)$.
Then for every $\alpha_Y\in\mathfrak{U}\cap\mathscr{S}_Y$, for any $j\in\{2,\ldots,6\}$, any $t\in\mathbb{R}$,
\begin{equation*}
\left|e^{-4\pi ic_{\mathsf{b}}c_jt}\Phi_{\mathsf{b}}\left(t\pm 2c_{\mathsf{b}}(b_j+c_j)\right)^{\pm 1}\right|=\left|e^{\frac{2\pi}{\sqrt{\hbar}}c_jt}\Phi_{\mathsf{b}}\left(t\pm\frac{i}{\sqrt{\hbar}}(b_j+c_j)\right)^{\pm 1}\right| \le M_{\delta,\hbar}e^{-\frac{2\pi}{\sqrt{\hbar}}\delta|t|}
\end{equation*}
holds. In addition, the above inequality holds even if we replace the letters $(b_j, c_j)\rightarrow (c_j, b_j), (b_j, c_j) \rightarrow(a_j, b_j)$ and so on.
In fact, for $t\le 0$ it follows by Lemma \ref{upper-bound-of-Phi_b}, Proposition \ref{property-of-Phi_b} (2), $c_j >\delta$, and $\delta<b_j+c_j=\frac{1}{2}-a_j<\frac{1}{2}-\delta$.
For $t\ge 0$, it follows by $b_j>\delta$ and by the fact
\begin{equation*}
\begin{split}
&\left|\Phi_{\mathsf{b}}\left(t+\frac{i}{\sqrt{\hbar}}(b_j+c_j)\right)\right|
=\left|\Phi_{\mathsf{b}}\left(-t+\frac{i}{\sqrt{\hbar}}(b_j+c_j)\right)\right|\left|e^{i\pi(t+\frac{i}{\sqrt{\hbar}}(b_j+c_j))^2}\right| \\
&\le M_{\delta,\hbar}e^{-\frac{2\pi}{\sqrt{\hbar}}(b_j+c_j)t}
\end{split}
\end{equation*}
because of Proposition \ref{property-of-Phi_b} (1), (2).

Therefore, for every $\alpha_Y\in\mathfrak{U}\cap\mathscr{S}_Y$, every $(A,B,D)\in\mathbb{R}^3$,
\begin{equation*}
\begin{split}
&\left|\frac{1}{\Phi_{\mathsf{b}}(A-2c_{\mathsf{b}}(c_2+b_2))}
e^{-4\pi ic_{\mathsf{b}}c_2A}\frac{1}{\Phi_{\mathsf{b}}(B-2c_{\mathsf{b}}(b_4+c_4))}e^{-4\pi ic_{\mathsf{b}}c_4B}\frac{1}{\Phi_{\mathsf{b}}(B-2c_{\mathsf{b}}(b_5+c_5))}\right.
\\
&\times e^{-4\pi ic_{\mathsf{b}}c_5B}\frac{1}{\Phi_{\mathsf{b}}(B-2c_{\mathsf{b}}(a_6+b_6))}
e^{-4\pi ic_{\mathsf{b}}b_6B}\frac{1}{\Phi_{\mathsf{b}}(D-2c_{\mathsf{b}}(b_3+c_3))} e^{-4\pi ic_{\mathsf{b}}b_3D}\\
&\times\left.e^{\pi i(2A^2+B^2-2BD+D^2+2AD)}\times \mbox{\textcircled{\scriptsize 2}}\right|\le (M_{\delta,\hbar})^5 e^{-\frac{2\pi}{\sqrt{\hbar}}\delta(|A|+3|B|+|D|)}
\end{split}
\end{equation*}
holds.

Since the right-hand side of this inequality is integrable in $\mathbb{R}^3$, because of the dominated convergence theorem, it follows that 
\begin{equation*}
\begin{split}
&\lim_{\alpha_Y\rightarrow \tau}\Phi_{\mathsf{b}}\left(\frac{\pi- \omega_{Y,\alpha_Y}(e_0)}{2\pi i\sqrt{\hbar}}\right)Z_\hbar(Y,\alpha_Y)=e^{-\frac{7}{12}\pi i}\int_{\mathbb{R}^3}dAdBdD \frac{1}{\Phi_{\mathsf{b}}(A-2c_{\mathsf{b}}(c_2^\tau+b_2^\tau))}\\
&\quad\times e^{-4\pi ic_{\mathsf{b}}c_2^\tau A}\frac{1}{\Phi_{\mathsf{b}}(B-2c_{\mathsf{b}}(b_4^\tau+c_4^\tau))} e^{-4\pi ic_{\mathsf{b}}c_4^\tau B}\frac{1}{\Phi_{\mathsf{b}}(B-2c_{\mathsf{b}}(b_5^\tau+c_5^\tau))}
e^{-4\pi ic_{\mathsf{b}}c_5^\tau B}\\
&\quad \times \frac{1}{\Phi_{\mathsf{b}}(B-2c_{\mathsf{b}}(a_6^\tau+b_6^\tau))}
e^{-4\pi ic_{\mathsf{b}}b_6^\tau B}\frac{1}{\Phi_{\mathsf{b}}(D-2c_{\mathsf{b}}(b_3^\tau +c_3^\tau))}e^{-4\pi ic_{\mathsf{b}}b_3^\tau D} \\
&\quad\times e^{\pi i(2A^2+B^2-2BD+D^2+2AD)}\times\left.\mbox{\textcircled{\scriptsize 2}}\right|_{\alpha_Y=\tau}.
\end{split}
\end{equation*}

Transformation of variables,
\begin{equation*}
\begin{aligned}
&\Tilde{A}\coloneqq A-2c_{\mathsf{b}}(c_2^\tau+b_2^\tau)=A-2c_{\mathsf{b}}\left(\frac{1}{2}-a_2^\tau\right)\\
&\Tilde{D}\coloneqq D-2c_{\mathsf{b}}(b_3^\tau+c_3^\tau)=D-2c_{\mathsf{b}}\left(\frac{1}{2}-a_3^\tau\right)\\
&\Tilde{B}\coloneqq B-2c_{\mathsf{b}}(c_4^\tau+b_4^\tau)=B-2c_{\mathsf{b}}\left(\frac{1}{2}-a_4^\tau\right)=B-2c_{\mathsf{b}}\left(\frac{1}{2}-a_5^\tau\right)=B-2c_{\mathsf{b}}(b_5^\tau+c_5^\tau)\\
&=B-2c_{\mathsf{b}}\left(\frac{1}{2}-c_6^\tau\right)=B-2c_{\mathsf{b}}(a_6^\tau+b_6^\tau)
\end{aligned}
\end{equation*}
(We used \eqref{da2''}, \eqref{da3''} in the deformation of $\Tilde{B}$.), gives 
\begin{equation*}
\begin{split}
&\quad e^{-4\pi ic_{\mathsf{b}}c_2^\tau A}e^{-4\pi ic_{\mathsf{b}}b_3^\tau D}e^{-4\pi ic_{\mathsf{b}}c_4^\tau B}e^{-4\pi ic_{\mathsf{b}}c_5^\tau B}e^{-4\pi ic_{\mathsf{b}}b_6^\tau B}e^{\pi i\left(2A^2+(B-D)^2+2AD\right)}\\
&=e^{-4\pi ic_{\mathsf{b}}c_2^\tau \left\{\Tilde{A}+2c_{\mathsf{b}}\left(\frac{1}{2}-a_2^\tau \right)\right\}}e^{-4\pi ic_{\mathsf{b}}b_3\left\{\Tilde{D}+2c_{\mathsf{b}}\left(\frac{1}{2}-a_3^\tau\right)\right\}}e^{-4\pi ic_{\mathsf{b}}(c_4^\tau+c_5^\tau+b_6^\tau)\left\{\Tilde{B}+2c_{\mathsf{b}}\left(\frac{1}{2}-a_4^\tau\right)\right\}}\\
&\quad \times e^{\pi i\left[2\left(\Tilde{A}+2c_{\mathsf{b}}\left(\frac{1}{2}-a_2^\tau\right)\right)^2+\left\{\Tilde{B}+2c_{\mathsf{b}}\left(\frac{1}{2}-a_4^\tau\right)-\Tilde{D}-2c_{\mathsf{b}}\left(\frac{1}{2}-a_3^\tau\right)\right\}^2+2\left(\Tilde{A}+2c_{\mathsf{b}}\left(\frac{1}{2}-a_2^\tau\right)\right)\left(\Tilde{D}+2c_{\mathsf{b}}\left(\frac{1}{2}-a_3^\tau\right)\right)\right]}\\
&=e^{2\pi i\Tilde{A}^2}e^{\pi i\Tilde{B}^2}e^{\pi i\Tilde{D}^2}e^{-2\pi i\Tilde{B}\Tilde{D}}e^{2\pi i\Tilde{A}\Tilde{D}}e^{2\pi ic_{\mathsf{b}}\left(-2c_2^\tau+2\left(1-a_2^\tau\right)+1-2a_3^\tau\right)\Tilde{A}}e^{2\pi ic_{\mathsf{b}}\left(-2\left(c_4^\tau+c_5^\tau+b_6^\tau\right)-2(a_4^\tau-a_3^\tau)\right)\Tilde{B}}\\
&\quad \times e^{2\pi ic_{\mathsf{b}}\left(-2b_3^\tau+2\left(a_4^\tau-a_3^\tau\right)+1-2a_2^\tau\right)\Tilde{D}}e^{-4\pi ic_{\mathsf{b}}^2c_2^\tau(1-2a_2^\tau)}e^{-4\pi ic_{\mathsf{b}}^2b_3^\tau(1-2a_3^\tau)}e^{-4\pi ic_{\mathsf{b}}^2(c_4^\tau+c_5^\tau+b_6^\tau)(1-2a_4^\tau)}\\
&\quad \times e^{2\pi ic_{\mathsf{b}}^2(1-2a_2^\tau)^2}e^{4\pi ic_{\mathsf{b}}^2(a_4^\tau-a_3^\tau)^2}e^{2\pi ic_{\mathsf{b}}^2(1-2a_2^\tau)(1-2a_3^\tau)}.
\end{split}
\end{equation*}

Deformation of the coefficients of the first-order terms on $\tilde{A},\tilde{B},\tilde{D}$ in the exponent of $e$ using the equations from \eqref{da1''} to \eqref{da5''} gives
\begin{equation*}
-2c_2^\tau+2(1-a_2^\tau)+1-2a_3^\tau=1   
\end{equation*}
\begin{equation*}
-2(c_4^\tau+c_5^\tau+b_6^\tau+a_4^\tau-a_3^\tau)=-1
\end{equation*}
\begin{equation*}
-2b_3^\tau+2(a_4^\tau-a_3^\tau)+1-2a_2^\tau=0.
\end{equation*}

Therefore,
\begin{equation*}
\begin{split}
&\quad \lim_{\alpha_Y\rightarrow \tau}\Phi_{\mathsf{b}}\left(\frac{\pi-\omega_{Y,\alpha_Y}(e_0)}{2\pi i\sqrt{\hbar}}\right)Z_{\hbar}(Y,\alpha_Y)\\
&=e^{-\frac{7}{12}\pi i}\int_{\mathscr{Y}_\tau}d\Tilde{A}d\Tilde{B}d\Tilde{D}\frac{1}{\Phi_{\mathsf{b}}(\tilde{A})}\frac{1}{\Phi_{\mathsf{b}}^3(\tilde{B})} \frac{1}{\Phi_{\mathsf{b}}(\tilde{D})}e^{2\pi i\Tilde{A}^2}e^{\pi i\Tilde{B}^2}e^{\pi i\Tilde{D}^2}e^{-2\pi i\Tilde{B}\Tilde{D}}e^{2\pi i\Tilde{A}\Tilde{D}}e^{2\pi ic_{\mathsf{b}}\Tilde{A}}\\
&\quad \times e^{-2\pi ic_{\mathsf{b}}\Tilde{B}}\times \mbox{\textcircled{\scriptsize 3}},
\end{split}
\end{equation*}
where
\begin{equation*}
\begin{split}
\mbox{\textcircled{\scriptsize 3}}&\coloneqq e^{4\pi ic_{\mathsf{b}}^2c_1^\tau(c_1^\tau+b_1^\tau)}e^{-\pi ic_{\mathsf{b}}^2\frac{4(c_1^\tau-b_1^\tau)+1}{6}}e^{4\pi ic_{\mathsf{b}}^2c_2^\tau(c_2^\tau+b_2^\tau)}e^{-\pi ic_{\mathsf{b}}^2\frac{4(c_2^\tau-b_2^\tau)+1}{6}}e^{4\pi ic_{\mathsf{b}}^2b_3^\tau(b_3^\tau+c_3^\tau)}\\
&\quad\times e^{-\pi ic_{\mathsf{b}}^2\frac{4(b_3^\tau-c_3^\tau)+1}{6}}e^{4\pi ic_{\mathsf{b}}^2c_4^\tau(c_4^\tau+b_4^\tau)}e^{-\pi ic_{\mathsf{b}}^2\frac{4(c_4^\tau-b_4^\tau)+1}{6}}e^{4\pi ic_{\mathsf{b}}^2c_5^\tau(c_5^\tau+b_5^\tau)}e^{-\pi ic_{\mathsf{b}}^2\frac{4(c_5^\tau-b_5^\tau)+1}{6}}\\
&\quad\times e^{4\pi ic_{\mathsf{b}}^2b_6^\tau(b_6^\tau+a_6^\tau)}e^{-\pi ic_{\mathsf{b}}^2\frac{4(b_6^\tau-a_6^\tau)+1}{6}}e^{-4\pi ic_{\mathsf{b}}^2c_2^\tau(1-a_2^\tau)}e^{-4\pi ic_{\mathsf{b}}^2b_3^\tau(1-2a_3^\tau)}e^{-4\pi ic_{\mathsf{b}}^2(c_4^\tau+c_5^\tau+b_6^\tau)(1-2a_4^\tau)}\\
&\quad\times e^{2\pi ic_{\mathsf{b}}^2(1-2a_2^\tau)^2} e^{4\pi ic_{\mathsf{b}}^2(a_4^\tau-a_3^\tau)^2}e^{2\pi ic_{\mathsf{b}}^2(1-2a_2^\tau)(1-2a_3^\tau)},
\end{split}
\end{equation*}
and
\begin{equation*}
\mathscr{Y}_\tau\coloneqq(\mathbb{R}-2c_{\mathsf{b}}(b_2^\tau+c_2^\tau))\times
(\mathbb{R}-2c_{b}(b_4^\tau+c_4^\tau))\times(\mathbb{R}-2c_{\mathsf{b}}(b_3^\tau+c_3^\tau)).
\end{equation*}

By Proposition \ref{property-of-Phi_b} (1), since 
\begin{equation*}
\frac{e^{i\pi\Tilde{D}^2}}{\Phi_{\mathsf{b}}(\Tilde{D})}=e^{i\pi\frac{1+2c_{\mathsf{b}}^2}{6}}\Phi_{\mathsf{b}}(-\Tilde{D}),
\end{equation*} 
\begin{equation*}
\begin{split}
&\quad\lim_{\alpha_Y\rightarrow \tau}\Phi_{\mathsf{b}}\left(\frac{\pi-\omega_{Y,\alpha_Y}(e_0)}{2\pi i\sqrt{\hbar}}\right)Z_\hbar(Y,\alpha_Y)\\
&=e^{-\frac{5}{12}\pi i}\int_{\mathscr{Y}_\tau}d\Tilde{A}d\Tilde{B}d\Tilde{D}\frac{1}{\Phi_{\mathsf{b}}(\Tilde{A})}\Phi_{\mathsf{b}}(-\Tilde{D})\frac{1}{\Phi_{\mathsf{b}}^3(\Tilde{B})}e^{2\pi i\Tilde{A}^2}e^{\pi i\Tilde{B}^2}e^{2\pi i\Tilde{D}(\Tilde{A}-\Tilde{B})}e^{2\pi ic_{\mathsf{b}}\Tilde{A}}\\
&\quad\times e^{-2\pi ic_{\mathsf{b}}\Tilde{B}}e^{\frac{i\pi}{3}c_{\mathsf{b}}^2}\times\mbox{\textcircled{\scriptsize 3}}.
\end{split}
\end{equation*}

Let
\begin{equation*}
\Tilde{D}^\prime\coloneqq -\Tilde{D},
\end{equation*}
then 
\begin{equation*}
\begin{split}
&\quad \lim_{\alpha_Y\rightarrow \tau}\Phi_{\mathsf{b}}\left(\frac{\pi-\omega_{Y,\alpha_Y}(e_0)}{2\pi i\sqrt{\hbar}}\right)Z_\hbar(Y,\alpha_Y)\\
&=e^{-\frac{5}{12}\pi i}\int_{\mathscr{Y}_\tau^\prime}d\Tilde{A}d\Tilde{B}d\Tilde{D}^\prime\frac{\Phi_{\mathsf{b}}(\Tilde{D}^\prime)}{\Phi_{\mathsf{b}}(\Tilde{A})\Phi_{\mathsf{b}}^3(\Tilde{B})}e^{2\pi i(\Tilde{A}^2+\frac{\Tilde{B}^2}{2}-\Tilde{D}^\prime\Tilde{A}+\Tilde{B}\Tilde{D}^\prime)}e^{-\frac{\pi}{\sqrt{\hbar}}(\Tilde{A}-\Tilde{B})}e^{-\frac{i\pi}{12\hbar}}\times\mbox{\textcircled{\scriptsize 3}},
\end{split}
\end{equation*}
where
\begin{equation*}
\begin{split}
\mathscr{Y}_\tau^\prime&\coloneqq(\mathbb{R}-2c_{\rm b}(b_2^\tau+c_2^\tau))\times
(\mathbb{R}-2c_{b}(b_4^\tau+c_4^\tau))\times(\mathbb{R}+2c_{\rm b}(b_3^\tau+c_3^\tau))\\
&=\left(\mathbb{R}-\frac{i}{2\sqrt{\hbar}}(1-2a_2^{\tau})\right)\times
\left(\mathbb{R}-\frac{i}{2\sqrt{\hbar}}(1-2a_4^{\tau})\right)\times
\left(\mathbb{R}+\frac{i}{2\sqrt{\hbar}}(1-2a_3^{\tau})\right).
\end{split}
\end{equation*}

Since $\tau$ satisfies the equations from \eqref{da1} to \eqref{da7}, if we define $a_i=a_{i+1}^\tau,b_i=b_{i+1}^\tau,c_i=c_{i+1}^\tau\quad ( i=1, \cdots, 5 )$, then we can see that $( a_1, b_1, c_1, \cdots, a_5, b_5, c_5 )$ satisfy the equations from \eqref{it1} to \eqref{it5} in Section \ref{ideal-triangulation}. Therefore, 
\begin{equation*}
\lim_{\alpha_Y\rightarrow \tau}\Phi_{\mathsf{b}}\left(\frac{\pi-\omega_{Y,\alpha_Y}(e_0)}{2\pi i\sqrt{\hbar}}\right)Z_\hbar(Y,\alpha_Y)=e^{-\frac{5}{12}\pi i}e^{-\frac{i\pi}{12\hbar}}\times\mbox{\textcircled{\scriptsize 3}}\times J_{X}(\hbar,0)
\end{equation*}
is derived.
\end{proof}