\documentclass[aps,prl,twocolumn,superscriptaddress,10pt]{revtex4-2}
\usepackage[dvips]{graphicx}  
%\documentclass[aps,prl,reprint,groupedaddress]{revtex4-1}
% You should use BibTeX and apsrev.bst for references
\usepackage{natbib}
\usepackage{datetime}
\usepackage{color}
\usepackage{ulem}
\usepackage{epstopdf}
\usepackage{amssymb}
\usepackage{amsmath}
\def\d{{\rm d}}
\newcommand{\overbar}[1]{\mkern 1.5mu\overline{\mkern-1.5mu#1\mkern-1.5mu}\mkern 1.5mu}
\newcommand{\olsi}[1]{\,\overline{\!{#1}}} % overline short italic
\def\be{\begin{eqnarray}}
\def\ee{\end{eqnarray}}
\def\der{{\rm d}}
\def\e{{\rm e}}
\def\({\left(}
\def\){\right)}
\def\veps{\varepsilon}
\begin{document}
\title{Size distributions in irreversible particle aggregation}
			% Explanatory text should go in the []'s, actual e-mail
			% address or url should go in the {}'s for \email and \homepage.
\author{Klavs Hansen}
\email{hansen@lzu.edu.cn,KlavsHansen@tju.edu.cn}
\affiliation{Lanzhou Center for Theoretical Physics, Key Laboratory 
of Theoretical Physics of Gansu Province, 
and Key Laboratory of Quantum Theory and Applications of MoE, 
Lanzhou University, Lanzhou, Gansu 730000, China}
\affiliation{Center for Joint Quantum Studies and Department of Physics, 
School of Science, Tianjin University,\\
92 Weijin Road, Tianjin 300072, China}
\date{\today,~\currenttime}

\begin{abstract}
The aggregation of particles in the free molecular regime pertaining to 
cluster growth is determined approximately for kernels describing frequently 
occurring physical situations. 
The mean particle sizes develop close to linearly in time.
Scaling relations are used to derive a linear partial differential equation 
which is solved to show that the size distributions are close to log-normal size 
asymptotically in time.
\end{abstract}
\maketitle

\section{Motivation}

The growth of particles by irreversible molecular accumulation and cluster-cluster 
aggregation in a closed system in quasi-equilibrium is governed by the aggregation 
Schmoluchowski equation \cite{SmoluchowskiZPC1917}.
The equations gained renewed interest with the appearance of 
nanoparticles as technologically interesting species.
With their strongly size-dependent properties \cite{Haberland1994}, control and 
knowledge of size distributions and the factors that determine them therefore 
become of prime interest.
A good understanding of the growth of particles is likewise highly relevant for a 
quantitative description of the kinetics of atmospheric nucleation, in parallel to the
thermodynamic quasi-equilibrium description 
\cite{HoltenJCP2009,FordJMES2004,ElmJAS2020}.
Similarly it is essential for technological purposes where time dependent 
kernels have been suggested to engineer 3d printing for medicine 
\cite{MamontovAMM2017}. 

The present work aims at providing solutions to the equations under the conditions
characterized as the free molecular regime \cite{AldousBernoulli1999}, commonly 
prevailing under nanoparticle formation in supersaturation conditions. 
The methods derived here should be applicable to a wider range of conditions, 
although the general situation is not analyzed in detail. 
The approach is that of mean field theory where, at a given time, a single 
concentration for each particle size describes both the state of the system and the 
growth of the particles. 
 
A few different kernels (the $a$'s in Eq.\ref{smo-gain}) are known to lend 
themselves to exact closed form solutions.
Of special interest here is the solution for the size independent kernel used 
by Smoluchowski, which yields a single exponential decay.
It will be a potential solution to the equations, together with the log-normal 
distribution.
The realized solution will be determined by the exponent of the homogeneous 
kernels. 

\section{Fundamental equation and kernels}

The time development of an irreversibly aggregating particle distribution 
is described by the coupled ordinary differential equations
\begin{equation}
\label{smo-gain}
\frac{\der c_N}{\der t}
= \sum_{i =1}^{N-1} \frac{a_{i,N-i}}{2} c_i c_{N-i} 
- \sum_{i=1}^{\infty} a_{i,N} c_N c_i,
\end{equation}
where $c_i$ denotes the concentrations of particle size $i$. 
The kernels, $a_{i,j}$, are defined by the physical situation, but we can 
assume without any further justification that they are symmetric 
in their indices, $a_{i,j} = a_{j,i}$.


The equations obey total particle number conservation.
As this fact is not always recognized we show it here.
It follows from a rewrite of the first term in the equation
\be
&&\frac{\der }{\der t} \sum_{N=1}^{\infty} Nc_N \\\nonumber
&=&\sum_{N=1}^{\infty} N\(\sum_{i =1}^{N-1} \frac{a_{i,N-i}}{2} c_i c_{N-i} 
- \sum_{i=1}^{\infty} a_{i,N} c_N c_i\),
\ee
as
\be
&&\sum_{N=1}^{\infty} N \sum_{i =1}^{N-1} \frac{a_{i,N-i}}{2} 
c_i c_{N-i}\\\nonumber
&=&\frac{1}{2}\(\sum_{N=1}^{\infty}i\sum_{i=1}^{\infty}a_{i,N}c_ic_N  
+\sum_{N=1}^{\infty}N\sum_{i=1}^{\infty} a_{i,N} c_N c_i\),
\ee
which cancels the last sum in the expression.

Eqs. \ref{smo-gain} represent the time development of the concentrations 
when position dependent concentration fluctuations are ignored 
\cite{AldousBernoulli1999}.
This disregards the fluctuations that must be inherent in the stochastic 
processes described by the Smoluchowski equations 
\cite{MarcusTm1968,LushnikovIAN1979}.
On general grounds we must expect fluctuations to influence the solutions at 
most to second order in their relative values, although this is not guaranteed 
(see ref. \cite{AldousBernoulli1999} and references therein).
Including fluctuations will have the strongest consequences for the low 
intensity clusters. 

The choice of the physical situation gives the following kernels:
\be
a_{i,j} = \sigma_{i,j} v_{i,j} =\pi r_1 ^2
\left(i^{1/3} + j^{1/3} \right)^2 \left(\frac{i+j}{ij} 
\frac{8T}{\pi m_1} \right)^{1/2}.
% \equiv 
%q \left(i^{1/3} + j^{1/3} \right)^2 
%\left(\frac{i+j}{ij}\right)^{1/2}.
\ee
The geometric capture cross section assumed by the squared bracket is 
calculated with the sum of the radii of the two colliding particles.
The radii are proportional to the cube root of the particle number, reflecting a 
constant density for all sizes.
The relative speed in the square root is the average relative thermal speed of the 
particles in thermal equilibrium at temperature $T$, calculated as the 
value for a single particle with the reduced mass \cite{L&Lstatphys}.
The parameter $q$, defined as
\be
\label{qdef}
q \equiv \pi r_1 ^2 
\left( \frac{8T}{\pi m_1} \right)^{1/2},
\ee
has dimension volume per time, and clearly depends only on the aggregating 
material and the temperature.
Together with the total monomer concentration it is used to rewrite the 
equations in a dimensionless form.
For this purpose it is convenient to used the quantity $c_0$, which is defined as 
the reciprocal of the volume that contains a single monomer, bound or not.
The scaled time is then defined as  
\be
\tau \equiv t/q c_0.
\ee
This all gives the kernels the form
\be
\label{kernels}
a_{i,j} = \(i^{1/3}+j^{1/3}\)^2\(\frac{i+j}{ij}\)^{1/2}.
\ee
The scaling of the concentration means that the total particle number is 
normalized to unity:
\be
\sum_{i=1}^{\infty} Nc_N = 1.
\ee

\section{Time development of mean size}

The average particle size is
\be
\label{average-size2}
\olsi{N} = \frac{\sum_{N=1}^{\infty} Nc_N }{\sum_{N=1}^{\infty} c_N} = 
\frac{1}{\sum_{N=1}^{\infty} c_N},
\ee
with the time derivative
\be
\label{smo-meansize}
\frac{\der \olsi{N}}{\der \tau} = 
-\frac{1}{\left( \sum_{N=1}^{\infty} c_N \right)^2} 
\sum_{N=1}^{\infty} \frac{\der c_N}{\der \tau} = - \olsi{N}^2 
\sum_{N=1}^{\infty} \frac{\der c_N}{\der \tau}.
\ee
For the right hand side derivatives, the Smoluchowski equation is used.
Inserting it gives 
\begin{alignat}{2}
&~~~~~~~~\frac{\der \olsi{N}}{\der \tau}= \nonumber\\
-&\olsi{N}^2 \sum_{N=1}^{\infty} \left(
\sum_{i=1}^{N-1} \frac{1}{2} a_{i,N-i} c_i c_{N-i} 
- \sum_{i=1}^{\infty} a_{i,N} c_N c_i  \right).
\end{alignat}
To perform the sums, we first note that they are essentially identical.
This is seen from
\be
\label{doublesum}
\sum_{N=1}^{\infty} \sum_{i=1}^{N-1} a_{i,N-i} c_i c_{N-i} =
\sum_{N=1}^{\infty} \sum_{i=1}^{\infty} a_{i,N} c_i c_N.
\ee
The derivative therefore simplifies to 
\begin{alignat}{2}
\label{Nbarderiv}
\frac{\der \olsi{N}}{\der \tau} = 
\olsi{N}^2 \sum_{N=1}^{\infty}
\sum_{i=1}^{\infty} \frac{1}{2} a_{i,N} c_i c_N.
\end{alignat}
Up to this point the results hold for any set of kernels. 

The expression now needs to be approximated.
For this purpose we note that the $a$'s are slowly varying functions when
neither of the two indices is very small. 
We therefore use the approximation.
\be
\label{a-approx}
a_{i,N} \approx a_{\olsi{N},\olsi{N}} = 4\sqrt{2}\,\olsi{N}^{1/6}.
\ee
Hence 
\be
\frac{\der \olsi{N}}{\der \tau} = \frac{1}{2}4\sqrt{2}\,\olsi{N}^{1/6}
\olsi{N}^2 \sum_{N=1}^{\infty} \sum_{i=1}^{\infty} c_i c_N.
\ee
The two sums in this expression decouple and are both equal to $1/\olsi{N}$
by Eq. (\ref{average-size2}). 
This gives 
\be
\label{meanvalue}
\frac{\der \olsi{N}}{\der \tau} = 2\sqrt{2}\,\olsi{N}^{1/6}.
\ee
It is quite remarkable that this result is obtained without any knowledge 
about the distribution.
Only the relation Eq. (\ref{a-approx}) is required. 
Clearly, other kernels with similar properties can be analyzed similarly.

The time development is then simply
\be
\label{smo-mean-approx}
\olsi{N} = \left(\frac{5\sqrt{2}}{3} \tau +\tau_0 \right)^{6/5}.
\ee

The approximate calculation of this result suggests that a check with a 
numerical simulation is in place.
% Figure environment removed
The numerical integration of the coupled differential equations shown in
Fig. \ref{simulation1} started with monomers at $\tau=0$, which fixes 
$\tau_0$ to be 1.
The expected growth of the mean size with a power slightly above unity is
confirmed by the simulations, and the power of 6/5 on the scaled time is 
reproduced fairly well.
The difference from the predicted value in Eq. (\ref{smo-mean-approx}) 
is a deviation from the predicted multiplicative factor of 16 \%.
Similar relations for the asymptotic forms have been established previously 
by other means (see e.g. \cite{DongenPRL1985}).
The agreement of the result here with those and the numerical calculations 
lends confidence in the applicability of Eq.\ref{a-approx}.


\section{Scaling properties}

The scaling properties of the solutions for different kernels and initial 
conditions have been discussed extensively in refs. \cite{LeyvrazPR2003},
\cite{FournierCMP2005}.
The kernels used in the equations here accommodate scaled solutions on
the form
\be
\label{n-scaled}
c_N = \frac{1}{\olsi{N}^2}\tilde{c}\left(\frac{N}{\olsi{N}} \right),
\ee
where the reciprocal square of the mean size accounts for mass conservation
and the scaling size can be taken as the mean size, $\olsi{N}$, without loss 
of generality. 

The scaling does not determine the scaled abundances $\tilde{c}$ \textit{per se}
but provides a useful tool for their determination.
%\begin{alignat}{2}
 %\sum_{N=1}^{\infty} N c_N &\approx \int_0^{\infty} \frac{N}{\olsi{N}^2} 
%\tilde{c} \left( \frac{N}{\olsi{N}} \right) \der N \\\nonumber
%&= \int_0^{\infty} x \tilde{c}(x) \der x = 1.
%\end{alignat}
The rate of change of each concentration is 
\begin{alignat}{2}
\label{c-deriv0}
\frac{\der c_N}{\der \tau}& = 
-2\frac{\dot{\olsi{N}}}{\olsi{N}\,^3}\tilde{c}\(\frac{N}{\olsi{N}} \)
- \frac{N \dot{\olsi{N}}}{\olsi{N}\,^4} \tilde{c}'\(\frac{N}{\olsi{N}} \)\nonumber \\
&= \frac{\dot{\olsi{N}}}{\olsi{N}\,^3}\left(-2 \tilde{c}\(\frac{N}{\olsi{N}} \)
- \frac{N}{\olsi{N}} \tilde{c}'\(\frac{N}{\olsi{N}} \) \right)\nonumber \\
&\equiv \frac{\dot{\olsi{N}}}{\olsi{N}\,^3}f\(\frac{N}{\olsi{N}} \),
\end{alignat}
where $\tilde{c}'$ indicates the derivative with respect to the argument. 
As indicated, $f$ is a function of $N/\olsi{N}$ only,
and its prefactor only of time (and the initial conditions).

The time derivative is also equal to 
\begin{eqnarray}
\label{c-deriv}
\frac{\der c_N}{\der \tau} &=& 
\sum_{i=1}^{N-1} \frac{1}{2} a_{i,N-i} c_i c_{N-i} - 
\sum_{i=1}^{\infty} a_{i,N} c_N c_i
\\
&\approx& 
\frac{1}{2} \int_0^N a_{i,N-i} c_i c_{N-i} \der i 
-c_N \int_0^{\infty} a_{i,N} c_i \der i.
\end{eqnarray}
The kernels are homogeneous functions with exponent 1/6:
\begin{equation}
\label{a-coeff}
a_{\alpha i, \alpha j} = \alpha^{1/6} a_{i,j}.
\end{equation}
Use of this together with the scaling in Eq. (\ref{n-scaled}) for the 
concentrations allows Eq. (\ref{c-deriv})  to be written, with 
$y \equiv N/\olsi{N}$, as:
\begin{alignat}{2}
\label{c-deriv2}
&~~~~~\frac{\der c_N}{\der \tau} \approx \olsi{N}\,^{-17/6}\times\\
&\nonumber
\left[ \frac{1}{2} 
\int_0^{N/\olsi{N}} a_{\rm x,y-x} \tilde{c}_x \tilde{c}_{\rm y-x} \der x 
-\tilde{c}_y \int_0^{\infty} a_{\rm x,y} \tilde{c}_x \der x\right].
\end{alignat}
The right hand side is a product of $\olsi{N}\,^{-17/6}$ and a function of 
$N/\olsi{N}$.
For notational simplicity it will be denoted by $g$:
\begin{alignat}{2}
\label{c-deriv3}
\frac{\der c_N}{\der \tau} \approx \olsi{N}\,^{-17/6} g\(\frac{N}{\olsi{N}} \).
\end{alignat}
Equating Eqs. (\ref{c-deriv0},\,\ref{c-deriv3}) gives
\begin{equation}
\frac{\dot{\olsi{N}}}{\olsi{N}^{1/6}}
= \frac{g\(\frac{N}{\olsi{N}} \)}
{f\(\frac{N}{\olsi{N}} \)},
\end{equation}
As the left hand side of this equation does not depend on $N$ and the right hand 
side not on time, a separation of variables has been achieved.
The separation constant has already been calculated in Eq. (\ref{meanvalue}) 
to be $2\sqrt{2}$. 
Notably, the specific kernel is manifested only in the power 1/6.
Other kernels will give analogous results with different powers, provided they 
are 1) slowly varying with size, and 2) homogeneous functions with the power
less than unity to avoid gelation \cite{HendriksJSP1983}.


\section{Solution with the scaled abundances}

The scaling properties of the solutions will first be used to establish 
a partial differential equation for $\tilde{c}$. 
The partial derivative of the function with respect to time is, with
Eq.(\ref{n-scaled}), equal to
\begin{eqnarray}
\frac{\partial c_N}{\partial \tau} = 
-2\frac{\dot{\olsi{N}}}{\olsi{N}^3}
\tilde{c}\left(\frac{N}{\olsi{N}}\right)
-\frac{N \dot{\olsi{N}}}{\olsi{N}^4}
\tilde{c}'\left(\frac{N}{\olsi{N}}\right)
\end{eqnarray}
The derivative with respect to size is
\be
\frac{\partial c_N}{\partial N} = 
\frac{1}{\olsi{N}^3} \tilde{c}'\left(\frac{N}{\olsi{N}}\right).
\ee
Substituting this equation into the previous and using 
\be
\frac{\der \olsi{N}}{\der \tau} = \frac{6}{5} \frac{\olsi{N}}{\tau} 
\ee
gives us
\begin{alignat}{2}
\label{partdiffsmolu}
&\frac{\partial c_N}{\partial \tau} = 
-\frac{6}{5\tau} \left( 2c_N + N\frac{\partial c_N}{\partial N}\right)\nonumber\\
&~~~~~~~~~~~~~\Downarrow \nonumber\\
&\frac{\partial \ln(c_N)}{\partial \ln \tau} = 
-\frac{6}{5}\left(2 + \frac{\partial \ln(c_N)}{\partial \ln N}\right).
\end{alignat}
The coefficient 6/5 arises as $1/(1-p)$, where $p$ is the order of the 
homogeneous kernels.

Inspection shows that two types of functions solve the equation.
One is the pure exponential,
\be
\label{singleexp}
c_N = \olsi{N}^{-2}\e^{-N/\olsi{N}}.
\ee
The other is a log-normal function: 
\be
\label{scaled-conc}
c_N = a \olsi{N}^{\,-2} \exp\left(-\frac{1}{2s^2}\left(\ln(N) 
-\ln(N_0) \right)^2 \right).
\ee
We note that the Eq. (\ref{partdiffsmolu}) also holds for constant kernels, 
provided the factor 6/5 is replaced by unity, consistent with it being a 
coefficient derived from the time development.

The analysis so far does not provide the criterion for choosing either of 
these two forms of solutions.
The choice is made by considering the time development of the monomer
for which, from Eq. (\ref{c-deriv}), we have 
\be
\label{monomerderiv}
\frac{\d c_1}{\d \tau} = - c_1\sum_{i=1}^{\infty} a_{i,1}c_i.
\ee
When the kernels are constant, $a_{i,1} = a$, the sum is equal to 
$1/\olsi{N}$.
For this case, Eq.(\ref{Nbarderiv}) shows that 
\be
\olsi{N} = \frac{1}{2}\tau + \olsi{N}(0),
\ee
and thus
\be
\frac{\d c_1}{\d \tau} = - c_1 \frac{2}{\tau}.
\ee
For long times we then have
\be
c_1 \propto \tau^{-2}.
\ee
A comparison with Eq. (\ref{singleexp}) identifies the solution for these kernels 
with the exponential form because for this size the exponential is close to unity. 

Application of Eq. (\ref{monomerderiv}) with the kernels given in 
Eq. (\ref{kernels}) gives
\be
\label{monomerderiv2}
\frac{\d c_1}{\d \tau} &=
 - c_1\sum_{i=1}^{\infty} \(i^{1/3}+1 \)^2\(\frac{i+1}{i}\)^{1/2} c_i\\\nonumber
&\approx - c_1\sum_{i=1}^{\infty} i^{2/3} c_i
\ee
where the approximation is for long times where $\langle i \rangle \gg 1$.
We approximate the sum as
\be
\sum_{i=1}^{\infty} i^{-1/3} i c_i \approx  \olsi{N}^{-1/3} 
\sum_{i=1}^{\infty} i c_i=\olsi{N}^{-1/3}.
\ee
With the known time dependence of the mean size the monomer concentration
is 
\be
c_1 \propto \exp\(-\alpha \tau^{3/5}\),
\ee
with 
\be
\alpha \equiv \(\frac{5}{3}\)^{3/5}2^{-1/5} = 1.18... 
\ee
Clearly this is not consistent with the exponential solution and we
can therefore assign the log-normal distributions to the physical kernels of 
interest here.

The constants of integration $a, s$ and $N_0$ in Eq.(\ref{scaled-conc}) 
are related due to mass conservation and the known time dependence of 
the mean size.
Replacing summation with integration, mass conservation yields
\be
a = \frac{\olsi{N}^2}{N_0^2}
\frac{1}{s\sqrt{2\pi}}{\rm e}^{-2s^2}.
\ee
The reciprocal of the mean size is calculated similarly.
With the value of $a$ known it is calculated to
\begin{alignat}{2}
\olsi{N}^{-1} = \sum_{N=1}^{\infty} c_N = N_0^{-1} {\rm e}^{-3s^2/2}.
\end{alignat}
Hence the average size is larger than the peak value of the size distribution 
by the factor $\exp(3s^2/2)$:
\be
\frac{\olsi{N}}{N_0} =  {\rm e}^{3s^2/2}.
\ee
Inserting this into the result for $a$ gives
\be
a = \frac{1}{s\sqrt{2\pi}}{\rm e}^{s^2}.
\ee
The peak value size, $N_0$, varies with time as $\olsi{N}$, i.e. 
as $\tau^{6/5}$, confirming the scaling properties of the solution.
In particular we have that
\be
\label{log-deriv}
\frac{\dot{\olsi{N}}}{\olsi{N}} = \frac{\dot{N}_0}{N_0}.
\ee

To find the width of the distribution, represented by $s$, we calculate the time 
derivative of the peak size, $N_0$, with both the scaled solution containing the 
unknown $s$ and with the Smoluchowski equation.
From the scaled expression in Eq. (\ref{scaled-conc}) we have
\begin{eqnarray}
\label{scaledsolution}
\dot{c}_{N_0} = -2 a \frac{\dot{\olsi{N}}}{\olsi{N}} \olsi{N}^{\,-2}
= -4 \sqrt{2} a \olsi{N}^{\,-17/6},
\end{eqnarray}
where use was made of Eq. (\ref{log-deriv}) and the known time dependence 
of $\olsi{N}$.
The derivative calculated with the Smoluchowski equation is:
\be
\label{N0}
\dot{c}_{N_0} = \frac{1}{2}\, \sum_{i=1}^{N_0-1} a_{N_0-i,i} c_{N_0-i}c_i 
-c_{N_0} \sum_{i=1}^{\infty} a_{N_0,i} c_i.
\ee
In the second term we approximate the size dependence of the kernels with 
the replacement $i \rightarrow \olsi{N}$.
The remaining sum is then known and given by $1/\olsi{N}$, making this 
term approximately equal to $-c_{N_0} a_{N_0,\olsi{N}}/\olsi{N}$. 
After inserting the expression for $c_{N_0}$ from the 
scaled solution, Eq. (\ref{scaled-conc}), the term becomes:
\be
-c_{N_0} \sum_{i=1}^{\infty} a_{N_0,i} c_i \approx
-a\olsi{N}^{\,-3} a_{N_0,\olsi{N}}.
\ee
The known relation between mean and peak values allows us to 
express the loss term as 
\be
\label{lossterm}
&&-c_{N_0} \sum_{i=1}^{\infty} a_{N_0,i} c_i \approx\nonumber\\
&&-a\, \olsi{N}^{-17/6} \(1 + {\rm e}^{-s^2/2}\)^2 
\(1+{\rm e}^{3s^2/2}\)^{\frac{1}{2}}.
\ee

The gain term in Eq. (\ref{N0}), which is a self-convolution of the abundances,
is calculated with a saddle point expansion of the scaled solutions:
\begin{eqnarray}
\label{gainterm}
&&\frac{1}{2} \sum_{i=1}^{N_0-1} a_{N_0-i,i} c_{N_0-i}c_i 
\nonumber\\
&\approx& a\, \olsi{N}^{-17/6} 2^{-\frac{1}{6}} 
{\rm e}^{-(\ln2)^2/2s^2-3s^2/4},
\end{eqnarray}
where the value of $a$ was also used. 
Equating the two calculations of the time derivative, Eq. (\ref{scaledsolution}) 
and the sum of Eqs. (\ref{lossterm},\ref{gainterm}), 
gives two solutions for $s$.
One is $s=0$, which is physically uninteresting. 
A numerical solution for the relevant value gives $s= 0.98$.

The number gives the Full-Width-Half-Maximum of the distributions 
of 
\be
N_{\frac{1}{2}^+} - N_{\frac{1}{2}^-}
= 4.24 N_0 = 0.34\olsi{N}.
\ee
The standard deviation of the size distribution is
\be
\sigma = 1.5 \olsi{N}.
\ee
The difference between these two values reflect the large difference 
between mean and peak values:
\be
\olsi{N}\approx 4.2N_0.
\ee

\section{Comparison with numerical solutions}

A test of the scaling of the solutions of the equations can be made by 
comparing numerically calculated particle size distributions for different times. 
For spectra sampled at $\tau_1$ and $\tau_2$ we have that
\be
\label{spectrumscaling}
c_{N'}(\tau_2) \frac{\olsi{N}(\tau_2)^2}{\olsi{N}(\tau_1)^2}
= c_N(\tau_1),
\ee
where 
\be
N' \equiv N \frac{\olsi{N}(\tau_2)}{\olsi{N}(\tau_1)}
\ee

The spectra were calculated numerically with a brute force solution of 
Eq.(\ref{smo-gain}) after discretization of the time.
The scaled time steps used were decided in each iteration as 0.0005 divided 
by the sum over all sizes of the absolute rate of change, divided by the mean value.
This conservative value eliminated discretization errors.
Rounding errors were eliminated by frequent normalization to unit total intensity.
The upper limit for particle sizes included in the sums was set to those that 
exceeded $10^{-5}$ times the highest abundance, but changes of all sizes up 
to $N=2\times 10^6$ were updated in each time step.
The distributions for some scaled times are shown in Fig.\ref{simulation2}
The numerical simulations indicate that scaling holds very well. 
%\vspace{-0.5cm}
% Figure environment removed
Fig.\ref{simulation3} shows the comparison of the numerically determined 
distributions with a fitted log-normal distribution.
% Figure environment removed
The fit in Fig.\ref{simulation3} yields the values $(s,\ln(N_0)) = (1.27,5.4)$,
which should be compared with the calculated values of $(0.98,5.9)$.
As is clear from the analysis and comparison of the simulated data with those
at  longer times, the difference between the simulated and calculated values
have reached their asymptotic values.
In particular, they will not increase a longer times.

\section{Summary and discussion}

The log-normal solutions found for the kernels here agree well with the numerical
results.
We note that the solutions in terms of log-normal functions vs. simple 
exponential size dependences depend on the form of the kernels.
Furthermore, the parameters in the log-normal solutions depend in a tractable 
manner on the kernels if they are homogeneous, specifically on the exponent.
The approximation of the Smoluchowski equations leading to a partial 
differential equation that accommodates two fundamentally different solutions
is of interest for the solutions of the special kernels found in refs. 
\cite{DongenJSP1988}, \cite{GoodismanJCP2006} and for the 'free coagulation'
solution in \cite{LushnikovPRE2002}.
The solutions given in the literature are, excluding a few of the exactly solved 
and gellling cases, of the form $N^{-\lambda}\exp\(-aN\)$ for large $N$, where 
$a$ depends on time \cite{DongenPRL1985,VillaricaJCP1993}.
In ref. \cite{VillaricaJCP1993} this is calculated by insertion of the Ansatz 
into the Smoluchowski equations.
But as shown here, the equations in general permit two solutions and the single 
exponential is not the relevant one, except for the constant kernels case.
The solutions found here are approximate but are not limited to small or large
sizes.
In particular it represents fairly accurate the peak of the distributions where
the bulk of the material is found.

The log-normal functional form is commonly applied to describe aggregation 
in cluster and nanoparticle sources.
Empirically it seems also to apply to size distributions generated under conditions 
where re-evaporation is relevant, beyond the condition for irreversible aggregation 
required for the present derivation.
The addition of reversibility requires the introduction of additional parameters 
and relations in the description.
With some simplifying assumptions about these, it should be possible to extend 
the approximate calculations which showed their usefulness in this work.
It remains to be seen how physically realistic these can be made.

\section{Acknowlegdement}

This work has been supported by the NSFC Grant No. 12247101, and 
by the 111 Project under Grant No. B20063.

%\bibliographystyle{apsrev}
%\bibliography{Smoluchowski}
\begin{thebibliography}{18}
\expandafter\ifx\csname natexlab\endcsname\relax\def\natexlab#1{#1}\fi
\expandafter\ifx\csname bibnamefont\endcsname\relax
  \def\bibnamefont#1{#1}\fi
\expandafter\ifx\csname bibfnamefont\endcsname\relax
  \def\bibfnamefont#1{#1}\fi
\expandafter\ifx\csname citenamefont\endcsname\relax
  \def\citenamefont#1{#1}\fi
\expandafter\ifx\csname url\endcsname\relax
  \def\url#1{\texttt{#1}}\fi
\expandafter\ifx\csname urlprefix\endcsname\relax\def\urlprefix{URL }\fi
\providecommand{\bibinfo}[2]{#2}
\providecommand{\eprint}[2][]{\url{#2}}

\bibitem[{\citenamefont{Smoluchowski}(1917)}]{SmoluchowskiZPC1917}
\bibinfo{author}{\bibfnamefont{M.}~\bibnamefont{Smoluchowski}},
  \bibinfo{journal}{Z. Phys. Chem.} \textbf{\bibinfo{volume}{XCII}},
  \bibinfo{pages}{129} (\bibinfo{year}{1917}).

\bibitem[{\citenamefont{Haberland}(1994)}]{Haberland1994}
\bibinfo{editor}{\bibfnamefont{H.}~\bibnamefont{Haberland}}, ed.,
  \emph{\bibinfo{title}{Clusters of Atoms and Molecules}}, Title Springer
  Series in Chemical Physics (\bibinfo{publisher}{Springer Berlin, Heidelberg},
  \bibinfo{year}{1994}), ISBN \bibinfo{isbn}{978-3-642-84331-0}.

\bibitem[{\citenamefont{Holten and van Dongen}(2009)}]{HoltenJCP2009}
\bibinfo{author}{\bibfnamefont{V.}~\bibnamefont{Holten}} \bibnamefont{and}
  \bibinfo{author}{\bibfnamefont{M.~E.~H.} \bibnamefont{van Dongen}},
  \bibinfo{journal}{J. Chem. Phys.} \textbf{\bibinfo{volume}{130}},
  \bibinfo{pages}{014102} (\bibinfo{year}{2009}).

\bibitem[{\citenamefont{Ford}(2004)}]{FordJMES2004}
\bibinfo{author}{\bibfnamefont{I.~J.} \bibnamefont{Ford}}, \bibinfo{journal}{J.
  Mechanical Engineering Science} \textbf{\bibinfo{volume}{218, Part C}},
  \bibinfo{pages}{883} (\bibinfo{year}{2004}).

\bibitem[{\citenamefont{Elm et~al.}(2020)\citenamefont{Elm, Kube\v{c}ka, Besel,
  J\"{a}\"{a}skel\"{a}inen, Halonen, Kurt\'{e}n, and
  Vehkam\"{a}ki}}]{ElmJAS2020}
\bibinfo{author}{\bibfnamefont{J.}~\bibnamefont{Elm}},
  \bibinfo{author}{\bibfnamefont{J.}~\bibnamefont{Kube\v{c}ka}},
  \bibinfo{author}{\bibfnamefont{V.}~\bibnamefont{Besel}},
  \bibinfo{author}{\bibfnamefont{M.~J.}
  \bibnamefont{J\"{a}\"{a}skel\"{a}inen}},
  \bibinfo{author}{\bibfnamefont{R.}~\bibnamefont{Halonen}},
  \bibinfo{author}{\bibfnamefont{T.}~\bibnamefont{Kurt\'{e}n}},
  \bibnamefont{and}
  \bibinfo{author}{\bibfnamefont{H.}~\bibnamefont{Vehkam\"{a}ki}},
  \bibinfo{journal}{J. Aerosol Sci.} \textbf{\bibinfo{volume}{149}},
  \bibinfo{pages}{105621} (\bibinfo{year}{2020}).

\bibitem[{\citenamefont{Mamontov and Hansen}(2017)}]{MamontovAMM2017}
\bibinfo{author}{\bibfnamefont{E.}~\bibnamefont{Mamontov}} \bibnamefont{and}
  \bibinfo{author}{\bibfnamefont{K.}~\bibnamefont{Hansen}},
  \bibinfo{journal}{Applied Mathematical Modelling}
  \textbf{\bibinfo{volume}{51}}, \bibinfo{pages}{109} (\bibinfo{year}{2017}).

\bibitem[{\citenamefont{Aldous}(1999)}]{AldousBernoulli1999}
\bibinfo{author}{\bibfnamefont{D.~J.} \bibnamefont{Aldous}},
  \bibinfo{journal}{Bernoulli} \textbf{\bibinfo{volume}{5}}, \bibinfo{pages}{3}
  (\bibinfo{year}{1999}).

\bibitem[{\citenamefont{Marcus}(1968)}]{MarcusTm1968}
\bibinfo{author}{\bibfnamefont{A.}~\bibnamefont{Marcus}},
  \bibinfo{journal}{Technometrics} \textbf{\bibinfo{volume}{10}},
  \bibinfo{pages}{78} (\bibinfo{year}{1968}).

\bibitem[{\citenamefont{Lushnikov}(1978)}]{LushnikovIAN1979}
\bibinfo{author}{\bibfnamefont{A.}~\bibnamefont{Lushnikov}},
  \bibinfo{journal}{Izv. Akad. Nauk SSSR Fiz. Atmos. Okeana}
  \textbf{\bibinfo{volume}{14}}, \bibinfo{pages}{738} (\bibinfo{year}{1978}).

\bibitem[{\citenamefont{Landau and Lifshitz}(1980)}]{L&Lstatphys}
\bibinfo{author}{\bibfnamefont{L.~D.} \bibnamefont{Landau}} \bibnamefont{and}
  \bibinfo{author}{\bibfnamefont{E.~M.} \bibnamefont{Lifshitz}},
  \emph{\bibinfo{title}{Statistical Physics,\\vol.1}}
  (\bibinfo{publisher}{Pergamon Press}, \bibinfo{year}{1980}).

\bibitem[{\citenamefont{van Dongen and Ernst}(1985)}]{DongenPRL1985}
\bibinfo{author}{\bibfnamefont{P.~G.~J.} \bibnamefont{van Dongen}}
  \bibnamefont{and} \bibinfo{author}{\bibfnamefont{M.~H.} \bibnamefont{Ernst}},
  \bibinfo{journal}{Phys. Rev. Lett.} \textbf{\bibinfo{volume}{54}},
  \bibinfo{pages}{1396} (\bibinfo{year}{1985}).

\bibitem[{\citenamefont{Leyvraz}(2003)}]{LeyvrazPR2003}
\bibinfo{author}{\bibfnamefont{F.}~\bibnamefont{Leyvraz}},
  \bibinfo{journal}{Phys. Rep.} \textbf{\bibinfo{volume}{383}},
  \bibinfo{pages}{95} (\bibinfo{year}{2003}).

\bibitem[{\citenamefont{Fournier and Lauren\c{c}ot}(2005)}]{FournierCMP2005}
\bibinfo{author}{\bibfnamefont{N.}~\bibnamefont{Fournier}} \bibnamefont{and}
  \bibinfo{author}{\bibfnamefont{P.}~\bibnamefont{Lauren\c{c}ot}},
  \bibinfo{journal}{Comm. Math. Phys.} \textbf{\bibinfo{volume}{256}},
  \bibinfo{pages}{589} (\bibinfo{year}{2005}).

\bibitem[{\citenamefont{Hendriks et~al.}(1983)\citenamefont{Hendriks, Ernst,
  and Ziff}}]{HendriksJSP1983}
\bibinfo{author}{\bibfnamefont{E.~M.} \bibnamefont{Hendriks}},
  \bibinfo{author}{\bibfnamefont{M.~H.} \bibnamefont{Ernst}}, \bibnamefont{and}
  \bibinfo{author}{\bibfnamefont{R.~M.} \bibnamefont{Ziff}},
  \bibinfo{journal}{J. Stat. Phys.} \textbf{\bibinfo{volume}{31}}
  (\bibinfo{year}{1983}).

\bibitem[{\citenamefont{van Dongen and Ernst}(1988)}]{DongenJSP1988}
\bibinfo{author}{\bibfnamefont{P.~G.~J.} \bibnamefont{van Dongen}}
  \bibnamefont{and} \bibinfo{author}{\bibfnamefont{M.~H.} \bibnamefont{Ernst}},
  \bibinfo{journal}{J. Stat. Phys.} \textbf{\bibinfo{volume}{50}},
  \bibinfo{pages}{295} (\bibinfo{year}{1988}).

\bibitem[{\citenamefont{Goodisman and Chaiken}(2006)}]{GoodismanJCP2006}
\bibinfo{author}{\bibfnamefont{J.}~\bibnamefont{Goodisman}} \bibnamefont{and}
  \bibinfo{author}{\bibfnamefont{J.}~\bibnamefont{Chaiken}},
  \bibinfo{journal}{J. Chem. Phys.} \textbf{\bibinfo{volume}{125}},
  \bibinfo{pages}{074304} (\bibinfo{year}{2006}).

\bibitem[{\citenamefont{Lushnikov and Kulmala}(2002)}]{LushnikovPRE2002}
\bibinfo{author}{\bibfnamefont{A.~A.} \bibnamefont{Lushnikov}}
  \bibnamefont{and} \bibinfo{author}{\bibfnamefont{M.}~\bibnamefont{Kulmala}},
  \bibinfo{journal}{Phys. Rev. E} \textbf{\bibinfo{volume}{65}},
  \bibinfo{pages}{041604} (\bibinfo{year}{2002}).

\bibitem[{\citenamefont{Villarica et~al.}(1993)\citenamefont{Villarica, Casey,
  Goodisman, and Chaiken}}]{VillaricaJCP1993}
\bibinfo{author}{\bibfnamefont{M.}~\bibnamefont{Villarica}},
  \bibinfo{author}{\bibfnamefont{M.~J.} \bibnamefont{Casey}},
  \bibinfo{author}{\bibfnamefont{J.}~\bibnamefont{Goodisman}},
  \bibnamefont{and} \bibinfo{author}{\bibfnamefont{J.}~\bibnamefont{Chaiken}},
  \bibinfo{journal}{J. Chem. Phys.} \textbf{\bibinfo{volume}{98}},
  \bibinfo{pages}{4610} (\bibinfo{year}{1993}).

\end{thebibliography}

\end{document}