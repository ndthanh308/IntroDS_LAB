\documentclass[amsmath,amssymb,12pt,superscriptaddress,reprint, preprintnumbers, notitlepage,aps,onecolumn,nofootinbib,showpacs,floatfix]{revtex4-1}
\pdfoutput=1
\usepackage{tabularx}
\usepackage[T1]{fontenc}
%\usepackage[utf8]{inputenc}
\usepackage[english]{babel}
\usepackage{amsmath}
\usepackage{graphicx}
\usepackage{dcolumn}
\usepackage{pbox}
\usepackage{amssymb}
\usepackage{epsfig}
\usepackage{slashed}
\usepackage{amssymb}
\usepackage{ mathrsfs }
\usepackage{float}
\usepackage[usenames,dvipsnames]{xcolor}
\usepackage[font=small]{caption}
\usepackage[font=small]{subcaption}
\usepackage{url}
\def\simlt{\stackrel{<}{{}_\sim}}
\def\simgt{\stackrel{>}{{}_\sim}}
\usepackage{multirow}
\usepackage{lineno}


\definecolor{MyLightBlue}{rgb}{0.22,0.51,0.9}
\definecolor{BrickRed}{rgb}{0.8, 0.25, 0.33}
\RequirePackage{hyperref}
\hypersetup{colorlinks, citecolor=blue,linkcolor=BrickRed, urlcolor=MyLightBlue}

\makeatletter
\renewcommand\@makecaption[2]{%
  \par
  \vskip\abovecaptionskip
  \begingroup
  
   \small\rmfamily
    \begingroup
     \samepage
     \flushing
     \let\footnote\@footnotemark@gobble
     \@make@capt@title{#1}{#2}\par
    \endgroup
  \endgroup
  \vskip\belowcaptionskip
}
\makeatother

\DeclareUnicodeCharacter{2212}{-}
\setcounter{secnumdepth}{1}
%%%%%%%%%%%%%%%%%%%%%%%%%%%%%%%%%%%%%%%%%%%%%%%%
\begin{document}
\title{Doubly-charged scalars of the Minimal Left-Right Symmetric Model at Muon Colliders}
\author{Mohamed Belfkir}
\email[E-mail: ]{mohamed.belfkir@cern.ch}
\affiliation{Department of Physics, UAE University, P.O. Box 17551, Al-Ain, United Arab Emirates}
\author{Talal Ahmed Chowdhury}
\email[E-mail: ]{talal@du.ac.bd}
\affiliation{Department of Physics, University of Dhaka, Dhaka 1000, Bangladesh}
\affiliation{Department of Physics and Astronomy, University of Kansas, Lawrence, Kansas 66045, USA}
\affiliation{The Abdus Salam International Centre for Theoretical Physics, Strada Costiera 11, I-34014, Trieste, Italy}
\author{Salah Nasri}
\email[E-mail: ]{snasri@uaeu.ac.ae}
\affiliation{Department of Physics, UAE University, P.O. Box 17551, Al-Ain, United Arab Emirates}
\affiliation{The Abdus Salam International Centre for Theoretical Physics, Strada Costiera 11, I-34014, Trieste, Italy}

\begin{abstract}
    We investigate the prospects of probing the doubly-charged scalars of the minimal Left-Right Symmetric model (MLRSM) at a muon collider. We assess its capability by studying the production of doubly-charged scalars and their subsequent decay into four charged lepton final states containing the same-charge lepton pairs. We find that the channels with same-charge electron and muon pairs, i.e., ($e^{\pm}e^{\pm}\mu^{\mp}\mu^{\mp}$ and its charge conjugated pairs), have the largest sensitivity due to the lowest Standard Model background. Besides, we show that the possibility of using fully polarized initial muon beams in the muon collider can enhance the detection sensitivity of doubly-charged scalars of the MLRSM. Furthermore, we show that one can put exclusion limits on the magnitudes of triplet Yukawa couplings that are directly related to the neutrino sector of the MLRSM for the mass range $1.1-5$ TeV of the doubly-charged scalars.
\end{abstract}


\maketitle

\section{Introduction}\label{intro}
The minimal left-right symmetric model (MLRSM) \cite{Mohapatra:1979ia, Mohapatra:1980yp} is a compelling extension of the Standard Model (SM) that addresses the fundamental question of neutrino mass. Neutrino mass is a significant departure from the SM, and uncovering its underlying mechanisms is crucial to unveil new physics beyond the SM. The MLRSM incorporates the seesaw mechanism \cite{Minkowski:1977sc, Mohapatra:1979ia, Gell-Mann:1979vob, Glashow:1979nm, Yanagida:1979as}, which has emerged as the prevailing explanation for the smallness of neutrino mass. The MLRSM, however, goes beyond the seesaw mechanism's ad hoc nature and offers a more comprehensive framework. It introduces left-right symmetry \cite{Pati:1974yy, Mohapatra:1974gc, Mohapatra:1974hk, Senjanovic:1975rk, Senjanovic:1978ev}, which attributes the left-handed nature of weak interactions to the spontaneous breakdown of parity. The model postulates the existence of right-handed (RH) neutrinos ($\nu_R$), thereby providing a natural explanation for non-vanishing neutrino masses. In the minimal version of the MLRSM, additional Higgs scalars in the form of left-handed and right-handed triplets play a vital role in spontaneous symmetry breaking. Significantly, the mass of the RH charged gauge boson ($W_R$) is directly related to the mass of the RH neutrinos, $M_N$, reinforcing the connection between the smallness of the neutrino mass and the near-maximality of parity violation in weak interactions.

Besides, the MLRSM is emerging as a self-contained and predictive theory of neutrino mass like the Higgs origin of the charged fermion masses in the SM \cite{Nemevsek:2012iq, Senjanovic:2016vxw, Senjanovic:2018xtu, Senjanovic:2019moe, Senjanovic:2023czt}. Moreover, the MLRSM not only predicts the lepton number violation both at low energies through the neutrinoless double beta decay \cite{Racah:1937qq, Furry:1939qr} and at high energies through the Keung-Senjanovic process, i.e., the production of same-sign charged lepton pairs in hadron colliders \cite{Keung:1983uu} but also connects these low-energy and high-energy processes \cite{Tello:2010am, Nemevsek:2011aa, Chakrabortty:2012mh}. Moreover, despite having near maximal parity violation in the low energy weak interactions, the left and right quark mixing matrices are shown to be close to each other \cite{Zhang:2007fn, Senjanovic:2015yea}. The MLRSM has been studied extensively in the Large Hadron Collider (LHC), and in \cite{Nemevsek:2018bbt, CMS:2021dzb, ATLAS:2023cjo}, it was shown that for the mass of the RH neutrinos, $M_{N}\simlt 1$ TeV, the observed lower limit on the mass of the RH gauge boson, $W_{R}$ is $M_{W_{R}}>6.4$ TeV. Also, in \cite{ATLAS:2022pbd}, it was shown that the observed lower limit on the mass of the doubly-charged scalars in the MLRSM is 1.08 TeV.

Recently, the muon colliders are considered to be future lepton colliders as these circular colliders have the potential to achieve center-of-energy in the multi-TeV range with high luminosity \cite{Gunion:1998bc, Palmer:2014nza, Delahaye:2019omf}. In addition, the muon collider has been shown to be a promising lepton collider for carrying out precision studies on the Higgs, gauge boson and Yukawa sectors of the SM as well as probing different BSM scenarios (see for example, \cite{AlAli:2021let}). Therefore, it will be interesting to study the sensitivity of such a future muon collider in probing the BSM scalars and gauge bosons predicted by the MLRSM. In this work, we have studied the sensitivity of a muon collider to probe the LH and RH doubly-charged scalars, $\Delta_{L}^{++}$ and $\Delta^{++}_{R}$, respectively, which are uniquely predicted by the MLRSM. Moreover, we have shown that the possibility of using the polarized incoming muon beams, which can be a distinctive feature of the future muon collider, will distinguish between LH and RH doubly charged scalars of the MLRSM.

The article is organized as follows. In section~\ref{mlrsmsec} we present the model. In section~\ref{tripletmuoncollider} we discuss the relevant parameter space of the model, identify the signals of interest and describe the event reconstruction and selection at the muon collider. Section~\ref{resultsec} includes the results of our analysis. Finally, we conclude in section~\ref{conclusion}. We present the relevant Higgs potential of the MLRSM in appendix~\ref{mlrsm-higgs}. 

\section{Doubly-charged scalars of the MLRSM}\label{mlrsmsec}
The gauge group of the MLRSM is $SU(3)_{c}\times SU(2)_{L}\times SU(2)_{R}\times U(1)_{B-L}$ with a symmetry between the left and right sectors given by the generalized parity ($\mathcal{P}$) or charge conjugation ($\mathcal{C}$) symmetry. Quarks and leptons are represented under the MLRSM gauge group as,
\begin{equation}
    Q_{L,R}=\begin{pmatrix}
        u\\
        d
    \end{pmatrix}_{L,R},\,\,\,
    \ell_{L,R}=\begin{pmatrix}
        \nu\\
        e
    \end{pmatrix}_{L,R}.
\label{lrsm-1}
\end{equation}
Moreover, the electromagnetic charge operator is defined as, $\hat{Q}_{\mathrm{EM}}=T^{3}_{L}+T^{3}_{R}+\frac{\hat{Q}_{BL}}{2}$, where $T^{3}_{L,R}$ are the diagonal generators of $SU(2)_{L,R}$, respectively, and $\hat{Q}_{BL}$ gives the charge under $U(1)_{B-L}$. For quark and lepton doublets, it is simply $\hat{Q}_{BL}=1/3$ and $1$, respectively. 
\begin{widetext}
% Figure environment removed
\end{widetext}
The Higgs sector of the MLRSM \cite{Mohapatra:1979ia, Mohapatra:1980yp, Deshpande:1990ip, Duka:1999uc, Barenboim:2001vu, Kiers:2005gh, Tello:2012qda, Dev:2016dja, Maiezza:2016bzp, Maiezza:2016ybz, BhupalDev:2018xya} contains a bidoublet $\Phi$ and $SU(2)_{L,R}$ scalar triplets, $\Delta_{L,R}$ under the gauge group, as follows,
\begin{equation}
\Phi \in (1,2,2,0),\,\,\,\mathbf{\Delta}_{L}\in (1,3,1,2)\,\,\, \mathrm{and}\,\,\, \mathbf{\Delta}_{R}\in (1,1,3,2),
\label{lrsm-2}
\end{equation}
where,
\begin{equation}
    \Phi=\begin{pmatrix}
        \phi^{0}_{1} & \phi^{+}_{2}\\
        \phi^{-}_{1} & \phi^{0}_{2}
    \end{pmatrix},\,\,\,
    \mathbf{\Delta}_{L,R}=\begin{pmatrix}
        \Delta^{+}/\sqrt{2} & \Delta^{++}\\
        \Delta^{0} & -\Delta^{+}/\sqrt{2}
    \end{pmatrix}_{L,R}.
    \label{lrsm-3}
\end{equation}
The Yukawa sector is given by,
\begin{align}
\mathcal{L}_{Y}&= \overline{Q_{L}}\left(Y_{q}\,\Phi+\tilde{Y}_{q}\,\tilde{\Phi}\right)Q_{R}+\overline{\ell_{L}}\left(Y_{\ell}\,\Phi+\tilde{Y}_{\ell}\,\tilde{\Phi}\right)\ell_{R}\nonumber\\
&+\frac{1}{2}\left(\overline{(\ell_{L})^{c}}\,\epsilon\,Y_{L}\mathbf{\Delta}_{L}\ell_{L}+\overline{(\ell_{R})^{c}}\,\epsilon\,Y_{R}\mathbf{\Delta}_{R}\ell_{R}\right)+\mathrm{h.c},
\label{lrsm-4}
\end{align}
where, $\epsilon=i\sigma_{2}$, $\tilde{\Phi}=\epsilon\Phi\epsilon$ and $(f_{L(R)})^{c}=C\gamma^{0}f^{*}_{L(R)}$ is the usual charge-conjugate spinor.
Now, under the discrete left-right symmetry, the fields transform as follows:
\begin{align}
    \mathcal{P}:&\,\,\,f_{L}\longleftrightarrow f_{R},\,\,\,\Phi\longleftrightarrow \Phi^{\dagger},\,\,\,\mathbf{\Delta}_{L}\longleftrightarrow \mathbf{\Delta}_{R},\nonumber\\
    \mathcal{C}:&\,\,\,f_{L}\longleftrightarrow (f_{R})^{c},\,\,\,\Phi\longleftrightarrow \Phi^{T},\,\,\,\mathbf{\Delta}_{L}\longleftrightarrow \mathbf{\Delta}^{*}_{R},\label{lrsm-5}
\end{align}
and the invariance of the Lagrangian under the left-right symmetry imposes the following conditions on the Yukawa couplings as follows:
\begin{align}
\mathcal{P}:&\,\,\,Y_{q}=Y_{q}^{\dagger},\,\,Y_{l}=Y_{l}^{\dagger},\,\,Y_{L}=Y_{R},\\
\mathcal{C}:&\,\,\,Y_{q}=Y_{q}^{T},\,\,Y_{l}=Y_{l}^{T},\,\,Y_{L}=Y^{*}_{R}.
\end{align}
In this work, we consider the generalized parity $\mathcal{P}$ as the discrete left-right symmetry, so the Yukawa couplings relevant for the scalar triplets become $Y_{L}=Y_{R}=Y$. Besides, the gauge couplings associated with $SU(2)_{L,R}$ are $g_{L}=g_{R}=g$. We also listed the parity symmetric Higgs potential in the appendix.

The symmetry breaking in the MLRSM takes place in two steps. First, at the high scale with the breaking of $SU(2)_{R}\times U(1)_{B-L}\rightarrow U(1)_{Y}$ through the vacuum expectation values of scalar triplets,
\begin{equation}
\langle \Delta^{0}_{R}\rangle =v_{R}/\sqrt{2},\,\,\,\langle \Delta^{0}_{L}\rangle = v_{L}/\sqrt{2},
\label{lrsm-6}
\end{equation}
where we consider $v_{L}\sim 0$, which is set by the seesaw picture of the MLRSM, as it directly contributes to the Majorana mass of the $\nu_{L}$. Afterward, at the lower scale, the SM symmetry is broken by the vev of $\Phi$, which is expressed as, 
\begin{equation}
    \langle\Phi\rangle=v/\sqrt{2}\,\,\mathrm{diag}(\cos\beta, -e^{i a}\sin\beta) 
\end{equation}
where, $\tan\beta=v_{2}/v_{1}$, the ratio between vevs of $\langle\phi_{1,2}^{0}\rangle=v_{1,2}$, $v=\sqrt{v_{1}^2+v_{2}^2}$ being the SM vev of 246 GeV, and the presence of the phase $a$ leads to the spontaneous breaking of CP symmetry by this vev.

Now, as $v_{R}\gg v$, neglecting the $O(v/v_{R})$ terms, the masses of the charged gauge boson, $W_{R}$ and neutral gauge boson, $Z_{R}$ are given by,
\begin{equation}
m_{W_{R}}^{2}\simeq \frac{1}{2} g^2 v_{R}^2,\,\,\, m_{Z_{R}}^{2}\simeq \left(g^2+\frac{1}{2}g^{2}_{B-L}\right)v_{R}^2,
\label{lrsm-7}
\end{equation}
where, the gauge coupling $g_{B-L}$ can be determined using the relation,
$1/e^2=2/g^2+1/g^{2}_{B-L}$. Moreover, the RH neutrino mass matrix is $M_{\nu_{R}}=Y\,v_{R}/\sqrt{2}$, which is proportional to the $v_{R}$ and connects the neutrino mass to the symmetry breaking scale of the MLRSM.

We will  focus on the scalar triplets, specifically the doubly-charged scalars $\Delta^{++}_{L,R}$ of the MLRSM. As $v_{L}\sim 0$, the decay $\Delta^{\pm\pm}_{L}\rightarrow W^{\pm}\,W^{\pm}$ to the same-sign SM charged gauge bosons $W$ is negligible. Now, the presence of flavor changing neutral currents mediated by the scalars, $H^{+},\,H^{0}\,A^{0}$ coming from the bidoublet set their masses to be $\simgt 20$ TeV, which in turn puts the $W_{R}$ mass (or the vev $v_{R}$) also in the multi-TeV range~\cite{Beall:1981ze, Mohapatra:1983ae, Ecker:1983uh, Zhang:2007da, Maiezza:2010ic, Guadagnoli:2010sd, Bertolini:2014sua, Bertolini:2019out}. But the masses of the $\Delta^{++}_{L}$, $\Delta^{+}_{L}$, $\Delta^{0}_{L}$, $\Delta^{++}_{R}$ and $\mathrm{Re}\Delta^{0}_{R}$ (as seen in appendix), can be set to a smaller scale than the $m_{W_{R}}$ in a region of the parameter space of the MLRSM. Therefore, we focus on this parameter space where, apart from the above-mentioned scalars associated with scalar triplets, the BSM particles predicted by the MLRSM are in the $O(\simgt 20)$ TeV range. Hence, $\Delta^{\pm\pm}_{R}\rightarrow W^{\pm}_{R}\,W^{\pm}_{R}$, to the same-sign $W_{R}$ becomes kinematically forbidden in our case. Consequently, the dominant decay modes of the doubly-charged scalars turn out to be the same-sign charged lepton pairs.

\section{Doubly-charged scalars at the muon collider}\label{tripletmuoncollider}
\subsection{The parameter space}
We will concentrate  on the parameter space where $m_{\Delta^{++}_{L}}\simeq m_{\Delta^{+}_{L}}\simeq m_{\Delta^{0}_{L}}$. Besides, $m_{\Delta_{L}^{++}}\simeq m_{\Delta^{++}_{R}}$ when $\rho_{2}\simeq \frac{1}{4}(\rho_{3}-2\rho_{1})$, as can be seen from Eq. \ref{scalarmass1} and \ref{scalarmass4}. Moreover, the cascade decay, $\Delta^{++}_{L}\rightarrow \Delta^{+}_{L}W^{+}$ is suppressed due to small mass difference between $\Delta^{++}_{L}$ and $\Delta^{+}_{L}$. Finally, the dominant decay modes of doubly charged scalars are into the same-charge lepton pairs,
\begin{equation}
    \Delta^{\pm\pm}_{L(R)}\rightarrow \ell^{\pm}_{L(R)}\ell'^{\pm}_{L(R)}\,\,\,\mathrm{with}\,\,\,\Gamma^{\Delta^{++}}_{\ell\ell'}=\kappa \frac{|Y_{\ell\ell'}|^{2}}{16\pi}m_{\Delta^{++}},
    \label{lrsm-8}
\end{equation}
where $\kappa=2$ for $\ell=\ell'$ and $\kappa=1$ for $\ell\neq\ell'$, respectively. Unlike the case of hadron colliders, in the muon collider, the production of the doubly-charged scalars proceeds through the Drell-Yan-like processes (s-channel) with $\gamma^*$ and $Z$, $Z_{R}$ and heavy Higgses, and additional t-channel processes with the exchange of charged leptons. But, as $Z_{R}$ and heavy Higgses have masses in $O(20\,\mathrm{TeV})$, their contributions will be sub-dominant compared to $\gamma^*$ and $Z$ for our considered center-of-mass energy $\sqrt{s}\leq 10$ TeV. Therefore, our analysis focuses on s-channel processes with $\gamma^*$ and $Z$, as shown in Fig.~\ref{feynmandig} (left), and t-channel processes with the exchange of charged leptons, as shown in Fig.~\ref{feynmandig} (right). In addition, the amplitude associated with Fig.~\ref{feynmandig} (left) scales with the couplings $\sim O(g^2 Y^{2})$ whereas the t-channel process given in Fig.~\ref{feynmandig} (right) has $O(Y^{4})$ dependence, where $Y$ denotes the Yukawa couplings between the doubly-charged scalars and charged leptons. For this reason, if $Y_{\ell\ell}\simgt g$, the t-channel dominates the pair productions of doubly-charged scalars and their subsequent decays into same-charge lepton pairs. But the scale of these Yukawa couplings, particularly the off-diagonal ones, are severely constrained by the limits on the charged lepton flavor violating (LFV) processes: $\mathrm{Br}(\mu\rightarrow 3e)< 10^{-12}$ \cite{SINDRUM:1987nra}, $\mathrm{Br}(\tau\rightarrow 3 e)<2.7\times 10^{-8}$ and $\mathrm{Br}(\tau\rightarrow 3 \mu)< 2.1\times 10^{-8}$ \cite{Hayasaka:2010np}, as the doubly-charged scalars mediate them at the tree-level (see for example~\cite{Cirigliano:2004mv}). For this reason, we consider the Yukawa coupling matrix $Y$ to be diagonal for simplicity, i.e., $Y_{ee}=Y_{\mu\mu}=Y_{\tau\tau}$ and $Y_{\ell\ell'}=0$. Moreover, as the muon collider is expected to have center-of-mass energy $\sqrt{s}=10$ TeV with integrated luminosity of $\mathcal{L}=10\,\mathrm{ab}^{-1}$, we focus on the mass range of the doubly-charged scalars in $m_{\Delta^{++}}=1100 - 5000\,\mathrm{GeV}$.

Fig.~\ref{fig:xsec_vs_mass} shows the total cross-section of the doubly-charged scalar pair productions decaying to four leptons $\sigma(\mu\mu\rightarrow\Delta^{++}_{L,R}\Delta^{--}_{L,R}\rightarrow \ell^{+}\ell^{+}\ell^{-}\ell^{-})$ ($\ell= e$ or $\mu$) as a function of the scalar mass for $\sqrt{s}$ = 10 TeV. The cross-section drops significantly, as expected, at around $\sqrt{s}/2$. The $\Delta^{++}_{L}$ shows a slightly higher production rate compared to the $\Delta^{++}_{R}$, which can be enhanced with the polarized initial muon beam. Actually, $\Delta^{++}_{L}$ and $\Delta^{++}_{R}$ couple to the  $Z$ bosons
% Figure environment removed
with $\frac{g}{\cos\theta_{w}}(1-2\sin^{2}\theta_{w})$ and $-2 g \sin\theta_{w}\tan\theta_{w}$, respectively. On the other hand, if the initial muon beams are polarized, i.e., LH and RH muons, the $Z$ boson couplings are $\frac{g}{\cos\theta_{w}}(-\frac{1}{2}+\sin^{2}\theta_{w})$ and $g\sin\theta_{w}\tan\theta_{w}$, respectively. Here $\theta_{w}$ is the Weinberg angle. So the usage of the polarized initial muon beam would lead to distinguishing the Drell-Yan process involving $Z$ boson as the t-channels are highly suppressed for the polarized cases. Fig.~\ref{fig:cross_vs_pol} (a) shows the production cross-section in the four-lepton final state as a function of the initial muon beam polarization. The beam polarization refers to the degree of alignment of the muon spins within the beam line. We can see from Fig.~\ref{fig:cross_vs_pol} (b) that for $m_{\Delta^{++}}$ = 2 TeV, a fully polarized beam leads to about a factor 4 enhancement in the $\Delta^{++}_{L}$ production cross-section compared to $\Delta^{++}_{R}$, thus making $\Delta^{++}_{L}$ and $\Delta^{++}_{R}$ more distinguishable in the muon collider.

% Figure environment removed

\subsection{Analysis}

\subsubsection{Monte Carlo event generation}
\label{mc}

In this work, the signal of interest consists of final states with four leptons. The doubly-charged scalars decay into the same-charge lepton pairs, leading to three different signal categories\footnote{$\tau$ lepton is not considered in the final state leptons of our analysis because $\tau$, having semileptonic decay mode, has different reconstruction routine compared to $e$ and $\mu$ in the colliders.}:
\begin{itemize}
    \item Same flavor - same flavor ($SF-SF$) where doubly-charged scalars decay into the same flavor lepton pairs, either $e^{\pm}e^{\pm}$ or $\mu^{\pm}\mu^{\pm}$. This category includes final states with four muons ($\mu^{+}\mu^{+}\mu^{-}\mu^{-}$), four electrons ($e^{+}e^{+}e^{-}e^{-}$), and two electrons and two muons ($e^{\pm}e^{\pm}\mu^{\mp}\mu^{\mp}$ and its charge conjugated states). We denote these four lepton combinations as $4\mu$, $4e$, and $2e 2\mu$, respectively, in the subsequent text.
    \item Same flavor - different flavor ($SF-DF$) where one of the doubly-charged scalars decays into the same flavor lepton pair, while the other decays into different flavor pair. This category includes final states with three electrons and one muon ($eee\mu$) and three muons and one electron ($e\mu\mu\mu$).
    \item Different flavor - Different flavor ($DF-DF$) where doubly-charged scalars decay into opposite flavor pairs. This category includes final states with two electrons and two muons of the form ($e\mu e\mu$).\\
\end{itemize}
The $SF-DF$ and $DF-DF$ scenarios are not considered for the benchmarks adopted in this work, as the mixing between electrons and muons is set to zero. Only the $SF-SF$ final state is analyzed. For each $SF-SF$ leptonic final state, 50,000 events were generated at a center-of-mass energy of 10 TeV, corresponding to an integrated luminosity of 10 $ab^{-1}$, as defined by the Snowmass muon collider forum \cite{Aime:2022flm}. The analysis also takes into account the muon beam polarization. Two different beam polarization configurations are adopted:
\begin{itemize}
    \item No polarization: The muon beams are considered unpolarized.
    \item Fully polarized beam: One positively charged muon beam is right-handed, and the negatively charged muon beam is left-handed.\\
\end{itemize}
The MLRSM is implemented using the \texttt{Feynrules} package \cite{Alloul:2013bka} to generate the UFO files. Afterward, the event generation is performed at the leading-order (LO) matrix elements with \texttt{MadGraph5\_aMC\@NLO 3.3.0} \cite{Alwall:2014hca}, which is interfaced to \texttt{Pythia 8.186} \cite{Bierlich:2022pfr} for decay chain modeling, parton showering, hadronization, and the description of the underlying event. The dominant Standard Model (SM) background for most channels arises from misidentifying lepton flavors in the final states. The misidentification rate for electrons and muons in muon colliders is less than 0.5\%, making the final states nearly background-free \cite{Yu:2017mpx}. However, the SM background that includes $ZZ^{*}$, $Z\gamma$, and $\gamma\gamma$, was generated with a total of 100,000 events for each channel ($4\mu$, $4e$, and $2e2\mu$), following the same signal generation scheme.

The \texttt{Delphes} \cite{deFavereau:2013fsa} fast simulation is used to emulate the detector reconstruction and performance. Moreover, the muon collider detector card is included in its latest release \cite{delphs_muon_card}, which is a hybrid of CLIC \cite{leogrande2019delphes} and FCC-$hh$ cards \cite{Selvaggi:2717698}. The card assumes a muon (electron) reconstruction efficiency of nearly 100\% (90\%) for $|\eta| < 1.5$, and 98\% (75\%) for $1.5 < |\eta| < 2.5$. The muon energy $p_T$ resolution is approximately 1\%. Additionally, the Delphes card includes an additional hypothetical forward muon detector in the region $|\eta| > 2.5$ with an efficiency of 90\% for $0.5 < p_T < 1$ GeV and 95\% for $p_T > 1$ GeV.

\subsubsection{Events reconstruction and selection}
Leptons, specifically muons or electrons, are reconstructed using the particle-flow tracks collection. Muons must have a transverse momentum ($p_T$) greater than 18 GeV and the pseudorapidity within the range of $|\eta| < 2.5$ in order to be considered in the analysis. Conversely, electrons are required to have $p_T$ > 22 GeV and $|\eta| < 2.5$. An overlap removal procedure is applied to avoid overlap between reconstructed leptons. This involves discarding leptons too close to each other with a distance in the $\Delta R$ less than 0.01, with the lepton having the highest $p_T$ retained.

For the $4\mu$ channel, events must have at least four muons that satisfy the previously mentioned requirements. An electron veto is applied to ensure orthogonality with the $4e$ and $2e2\mu$ channels. Additionally, events are required to have at least two pairs of muons with the same charge. The two pairs with the highest $p_T$ and closest in terms of $\Delta R$ are considered candidates for the doubly-charged scalar particles.

Similarly, for the $4e$ channel, events must have a minimum of four electrons that meet the selection criteria. A muon veto is applied to ensure orthogonality with the $4\mu$ and $2e2\mu$ channels. Events must have at least two pairs of electrons with the same charge. The two pairs with the highest $p_T$ and closest in $\Delta R$ are identified as doubly-charged scalar particles.

In the $2e2\mu$ channel, events must contain at least two electrons and two muons. No veto is applied in this channel. Events must have at least one pair of electrons and one pair of muons with the same charge. Doubly-charged scalars are reconstructed using the pair with the highest $p_T$.

For each selected event, the invariant masses $m_{\ell\ell}$ of each of the doubly-charged scalar candidates are computed, and events are required to satisfy that $m_{\ell\ell}$ within $ m_{\Delta^{++}} \pm 10\%$. In addition, this mass requirement is removed to allow for more background events in $4e$ and $2e2\mu$ channels. Fig.~\ref{fig:mll} shows the distribution of the invariant masses of the leading di-lepton system for each channel for $m_{\Delta^{++}} = $ 2 TeV. The low tails observed in the reconstructed $m_{ee}$ for both $2e2\mu$ and $4e$ channels are due to the FCC-$hh$ tracking resolution, which is part of Delphes's muon collider detector card. For the $2e2\mu$ channel, only a few events (at low $m_{\ell\ell}$) passed our selection. The background in this channel is due to the misidentification rate estimated to be 0.1\% at the muon collider \cite{Forslund:2022xjq}. Thus, the $2e2\mu$ channel is considered with no background. 

% Figure environment removed

\section{Result and discussion}\label{resultsec}
In Fig.~\ref{fig:accxeff}, we present the acceptance times efficiency, which is defined as the number of selected events divided by the expected number of events. The $4\mu$ channel exhibits a higher efficiency of approximately $77\%$ compared to the $4e$ and $2e2\mu$ channels. Consequently, the search for doubly-charged scalar sensitivity is predominantly driven by the $4\mu$ channel. Notably, the acceptance times efficiency remains nearly constant across the scalar mass for both LH and RH scalars. Besides, the initial beam polarization does not influence the final state kinematics, ensuring the efficiency remains unaffected. However, as demonstrated in Figure \ref{fig:cross_vs_pol}, the rate is enhanced by a factor of 3 in the case of $\Delta^{++}_{L}$ when a fully polarised beam is considered. \\
% Figure environment removed

For $m_{\Delta^{++}_{L,R}} $ = 2 TeV, tables \ref{tab:dl2pPol0}, \ref{tab:dl2pPol100}, \ref{tab:dr2pPol0} and \ref{tab:dr2pPol100} are provided to summarize the number of selected signal $N_{signal}$ and SM background events $N_{SM}$ for each channel. The $2e2\mu$ channel is reported to have zero background events, as described earlier. Additionally, the expected Asimov significance~\cite{Cowan:2010js}, as given in Eq.~\ref{eq:Z} where $s$ and $b$ are the signal and background expected yields, respectively, is evaluated to assess the significance of our findings.  
\begin{equation}
    \label{eq:Z}
    Z = \sqrt{ 2[(s+b)log(1+s/b) - s] },
\end{equation}

\input{figures/yield_2TeV_dl2p_S10_Pol0.txt}
\input{figures/yield_2TeV_dl2p_S10_Pol100.txt}
\input{figures/yield_2TeV_dr2p_S10_Pol0.txt}
\input{figures/yield_2TeV_dr2p_S10_Pol100.txt}

When considering an unpolarized initial beam, the Asimov significance, for $\Delta^{++}_{L}\,(\Delta^{++}_{R})$, are  98.7 (93.3), 62.2 (58.6), and 80.23 (75.9) for the $2e2\mu$, $4e$, and $4\mu$ channels, respectively, as seen in Table \ref{tab:dl2pPol0} and \ref{tab:dr2pPol0}. The $\Delta^{++}_{L}$ exhibits a 6\% higher significance compared to the $\Delta^{++}_{R}$, primarily due to a slightly higher cross-section, as shown in Figure \ref{fig:xsec_vs_mass}. The $2e2\mu$ shows the highest significance as it has the lowest background. However, this $6\%$ difference is significantly improved when utilizing a fully polarized beam, resulting in a remarkable factor 2 enhancement in the significance, as seen in Table \ref{tab:dl2pPol100} and \ref{tab:dr2pPol100}. Thus using a polarized beam will significantly enhance the muon collider's sensitivity to probe the doubly-charged scalars of MLRSM.

Figure \ref{fig:sigma_summary} illustrates the Asimov significance for the $4\mu$, $4e$ and $2e2\mu$ channels as a function of $m_{\Delta^{++}_{L,R}}$ for $\sqrt{s}=10$ TeV. As we approach the $\sqrt{s}/2$, the sensitivity decreases significantly, as expected. However, the significance remains prominent even at this point, showcasing the robustness of this analysis's very clean final states. Besides, the combined results of $4\mu$ + $4e$ + $2e2\mu$ are presented where each channel is statistically combined by ensuring the full orthogonality of different signal regions. This combination significantly enhances our sensitivity, achieving a factor of 1.3 improvements, as seen from green continuous (dashed) lines in Fig. \ref{fig:sigma_summary} for unpolarized (polarized) muon beams. However, since the $2e2\mu$ channel has the lowest background, we present the combined results of $4\mu$ and $4e$ channels, represented by magenta continuous (dashed) lines for unpolarized (polarized) muon beams. Remarkably, we see that the combined result of $4e+4\mu$ channels is equivalent to the results of the $2e2\mu$ channel.

% Figure environment removed

Finally, as the cross-section, $\mu^{+}\mu^{-}\rightarrow \Delta^{++}_{L(R)}\Delta^{--}_{L(R)}\rightarrow \ell^{+}\ell^{+}\ell^{-}\ell^{-}$ depends on the Yukawa couplings, $Y_{\ell\ell}$, we explore the limits on their magnitudes for fixed $m_{\Delta^{++}}$ values. In Fig.~\ref{fig:limit_summary}, we present the exclusion limits at a $95\%$ confidence level for the Yukawa couplings as a function of $m_{\Delta^++}$, focusing on $4\mu$ and $4e$ channels. The $4\mu$ and $4e$ channels exhibit similar sensitivity in the low mass region. However, at higher masses around 4 TeV, the $4\mu$ channel significantly outperforms the $4e$ channel, providing a stronger constraint on the Yukawa coupling. Combining the results from the $4\mu$ and $4e$ channels, we achieve a $20\%$ improvement in our exclusion limit in the low mass region. Additionally, when examining the case of $\Delta^{++}_{L}$, considering a fully polarized beam further strengthens the constraint, unlike $\Delta^{++}_{R}$, where the sensitivity is diminished.

% Figure environment removed

\section{Conclusion}
\label{conclusion}
A multi-TeV muon collider can not only probe both the SM and BSM~\cite{ Han:2020pif, Han:2020uak, Yin:2020afe, Han:2021udl, Han:2021lnp, Yang:2020rjt, Bandyopadhyay:2020mnp, Liu:2021jyc, Li:2021lnz, Asadi:2021gah, Franceschini:2021aqd, Anchordoqui:2021vrg, Haghighat:2021djz, BuarqueFranzosi:2021wrv, Bottaro:2021srh, Sen:2021fha, Yang:2021zak, Bandyopadhyay:2021pld, Dermisek:2021mhi, Chiesa:2021qpr, Bause:2021prv, Liu:2021akf, Ruiz:2021tdt, Casarsa:2021rud, Capdevilla:2021kcf, Medina:2021ram, Chen:2021pqi, Chen:2022msz, Hosseini:2022urq, Cesarotti:2022ttv, deBlas:2022aow, MuonCollider:2022xlm, MuonCollider:2022nsa, MuonCollider:2022glg, MuonCollider:2022ded, Aime:2022flm, Abbott:2022jqq, Homiller:2022iax, Forslund:2022xjq, Yang:2022fhw, Bottaro:2022one, Li:2022kkc, Black:2022nzv, Azatov:2022itm, Liu:2022byu, Spor:2022hhn, Black:2022cth, Kwok:2023dck, Li:2023ksw, Jueid:2023zxx, Xu:2023ene, Accettura:2023ked, Garosi:2023bvq, Vignaroli:2023rxr, Maharathy:2023dtp, Chowdhury:2023imd, Barducci:2023gdc, Altmannshofer:2023uci} but also complement the new physics searches at the future hadron colliders. Therefore, in this work, we focused on the possibility of probing the doubly-charged scalars $\Delta^{++}_{L}$ and $\Delta^{++}_{R}$ of the MLRSM in the future multi-TeV muon collider using the process, $\mu^{+}\mu^{-}\rightarrow \Delta^{++}_{L(R)}\Delta^{--}_{L(R)}\rightarrow \ell^{+}\ell^{+}\ell^{-}\ell^{-}$ with $\ell=e,\,\mu$. We choose the parameter space where masses of $W_{R}$ and flavor violating Higgses $H^{+}$, $H^{0}$ and $A^{0}$ are in $O(20)$ TeV. We concentrate on the doubly-charged mass in the range, $m_{\Delta^{++}}=1.1-5$ TeV in the muon collider with center of mass energy $\sqrt{s}=10$ TeV and integrated luminosity, $\mathcal{L}=10\,\mathrm{ab}^{-1}$. Besides, to avoid the stringent constraints coming from the lepton flavour violating processes $\mu\rightarrow 3 e$ and $\tau\rightarrow 3e,\,3\mu$, which are mediated by the exchange of doubly-charged scalars, we consider a diagonal triplet Yukawa matrix $Y$ with $Y_{ee}=Y_{\mu\mu}=Y_{\tau\tau}$. From our sensitivity analysis of probing $\Delta^{++}_{L,R}$ with four lepton final states that contain the same charge pairs in the muon collider, we point out as follows:
\begin{itemize}
 \item The $2e2\mu$ channel has the largest sensitivity of probing $\Delta^{++}_{L,R}$ due to the lowest SM background. The next sensitive channels are the $4\mu$ and $4e$ channels.
 \item Although the production cross-sections of $\Delta^{++}_{L}$ and $\Delta^{++}_{R}$ are comparable in the case of unpolarized initial muon beams, the difference between them becomes significant when the polarized muon beams are used, and furthermore, will allow us to distinguish between $\Delta^{++}_{L}$ and $\Delta^{++}_{R}$ in the muon collider.
 \item The $4\mu$ channel has a stronger exclusion limit on the Yukawa couplings at higher $m_{\Delta^{++}}$ masses. The constraints are improved with a combined analysis of $4\mu$ and $4e$ channels.
\end{itemize}
In conclusion, the doubly-charged scalars $\Delta^{++}_{L}$ and $\Delta^{++}_{R}$ are the unique predictions of the MLRSM which are connected to its explanation of neutrino mass through the interplay of left-right symmetry. The multi-TeV muon collider has the potential to discover them and provide the experimental verification of the MLRSM, thus making it a compelling avenue for future research to unravel the mysteries of neutrino mass and physics beyond the Standard Model.

\subsection*{Acknowledgement}
TAC is grateful to Kyoungchul Kong for illuminating discussions and would like to thank the High Energy Theory Group in the Department of Physics and Astronomy at the University of Kansas for their hospitality and support.  The work of S.N is supported by the United Arab Emirates University (UAEU) under UPAR Grant No. 12S093.

\appendix
\section{Higgs potential of MLRSM}\label{mlrsm-higgs}
The Higgs potential of the parity symmetric MLRSM in the seesaw picture is given by,
\begin{widetext}
\begin{align}
    V=& -\mu_{1}^{2}\mathrm{Tr}\left(\Phi^{\dagger}\Phi \right)-\mu^{2}_{2}\left[\mathrm{Tr}\left(\tilde{\Phi}^{\dagger}\Phi \right)+\mathrm{h.c.}\right]-\mu^{2}_{3}\left[\mathrm{Tr}\left(\mathbf{\Delta}_{L}\mathbf{\Delta}^{\dagger}_{L} \right)+\mathrm{Tr}\left(\mathbf{\Delta}_{R}\mathbf{\Delta}^{\dagger}_{R} \right)\right]+\lambda_{1}\left(\mathrm{Tr}\left(\Phi^{\dagger}\Phi \right)\right)^{2}\nonumber\\
    &+\lambda_{2}\left[\left(\mathrm{Tr}\left(\tilde{\Phi}\Phi^{\dagger} \right)\right)^{2}+\mathrm{h.c}\right]+\lambda_{3}\mathrm{Tr}\left(\tilde{\Phi}\Phi^{\dagger} \right)\,\mathrm{Tr}\left(\tilde{\Phi}^{\dagger}\Phi \right)+\lambda_{4}\mathrm{Tr}\left(\Phi^{\dagger}\Phi \right)\left[ \mathrm{Tr}\left(\tilde{\Phi}\Phi^{\dagger} \right)+\mathrm{h.c}\right]\nonumber\\
    &+ \rho_{1}\left[\left(\mathrm{Tr}\left(\mathbf{\Delta}_{L}\mathbf{\Delta}_{L}^{\dagger} \right)\right)^{2}+\left(\mathrm{Tr}\left(\mathbf{\Delta}_{R}\mathbf{\Delta}_{R}^{\dagger} \right)\right)^{2}\right]+\rho_{2}\left[\mathrm{Tr}\left(\mathbf{\Delta}_{L}\mathbf{\Delta}_{L} \right)\mathrm{Tr}\left(\mathbf{\Delta}^{\dagger}_{L}\mathbf{\Delta}^{\dagger}_{L} \right)+L\rightarrow R\right]+\rho_{3}\mathrm{Tr}\left(\mathbf{\Delta}_{L}\mathbf{\Delta}_{L}^{\dagger} \right)\mathrm{Tr}\left(\mathbf{\Delta}_{R}\mathbf{\Delta}_{R}^{\dagger} \right)\nonumber\\
&+\rho_{4}\left[\mathrm{Tr}\left(\mathbf{\Delta}^{\dagger}_{L}\mathbf{\Delta}_{L}^{\dagger} \right)\mathrm{Tr}\left(\mathbf{\Delta}_{R}\mathbf{\Delta}_{R} \right)+\mathrm{h.c}\right]+\alpha_{1}\mathrm{Tr}\Phi^{\dagger}\Phi\left[\mathrm{Tr}\left(\mathbf{\Delta}_{L}\mathbf{\Delta}_{L}^{\dagger} \right)+\mathrm{Tr}\left(\mathbf{\Delta}_{R}\mathbf{\Delta}_{R}^{\dagger} \right) \right]\nonumber\\
    &+\alpha_{2}\left[ e^{i c}\left\{\mathrm{Tr}\left(\tilde{\Phi}\Phi^{\dagger} \right)\mathrm{Tr}\left(\mathbf{\Delta}_{L}\mathbf{\Delta}_{L}^{\dagger} \right)+\mathrm{Tr}\left(\Phi\tilde{\Phi}^{\dagger} \right)\mathrm{Tr}\left(\mathbf{\Delta}_{R}\mathbf{\Delta}_{R}^{\dagger} \right) \right\}+\mathrm{h.c}\right]+\alpha_{3}\left[\mathrm{Tr}\left(\Phi\Phi^{\dagger}\mathbf{\Delta}_{L}\mathbf{\Delta}_{L}^{\dagger} \right)+\mathrm{Tr}\left(\Phi^{\dagger}\Phi\mathbf{\Delta}_{R}\mathbf{\Delta}_{R}^{\dagger} \right) \right].
    \label{higgs}
\end{align}
\end{widetext}
The masses of the component fields of $\mathrm{\Delta}_{L}$, after mass diagonalization in the limit $a,\,c\rightarrow 0$, are given as,
\begin{align}
    m^{2}_{\Delta^{++}_{L}}&= \frac{1}{2}(\rho_{3}-2\rho_{1})v_{R}^{2}+\frac{1}{2}\alpha_{3}\cos 2\beta\,v^{2},\label{scalarmass1}\\
    m^{2}_{\Delta^{+}_{L}}&\simeq\frac{1}{2}(\rho_{3}-2\rho_{1})v_{R}^{2}+\frac{1}{4}\alpha_{3}\cos 2\beta\,v^{2},\label{scalarmass2}\\
    m^{2}_{\Delta^{0}_{L}}&\simeq\frac{1}{2}(\rho_{3}-2\rho_{1})v_{R}^{2}.\label{scalarmass3}\\
\end{align}
On the other hand, the masses of the component fields of $\mathrm{\Delta}_{R}$ are
\begin{align}
    m^{2}_{\Delta^{++}_{R}}&=2 \rho_{2} v_{R}^2+\frac{1}{2}\alpha_{3}\cos 2\beta\,v^{2},\label{scalarmass4}\\
    m^{2}_{\mathrm{Re}\Delta^{0}_{R}}&\simeq 2 \rho_{1} v_{R}^2\label{scalarmass5}.
\end{align}

\bibliographystyle{style}
\bibliography{references}
\end{document}
