% IEEE overleaf file sharing link: https://www.overleaf.com/7583389676qbddccnrpvqv#9c51e8
% Google Drive link with AIVR Reviews: https://docs.google.com/document/d/1_vPieyEBV61UYwmn7Y6o3qX3Obj4UOOjUJXHzkDyKfI/edit?usp=sharing

% previous overleaf link: https://www.overleaf.com/2382424889svwbtcbbyjsr
% Reviews: The authors should provide more (and more detailed) pictures of their AR system (and more explanations about these systems) so that readers could better understand the details of the different AR-based systems presented and discussed in the paper.





\documentclass[conference]{IEEEtran}
\IEEEoverridecommandlockouts
% The preceding line is only needed to identify funding in the first footnote. If that is unneeded, please comment it out.
\usepackage{cite}
\usepackage{amsmath,amssymb,amsfonts}
\usepackage{algorithmic}
\usepackage{graphicx}
\usepackage{textcomp}
\usepackage{xcolor}
\usepackage{balance} 
\usepackage{enumitem}
%\usepackage{subfig}
\usepackage{booktabs}
%\usepackage{subcaption}
 \usepackage{subcaption}
\usepackage{comment}
\def\BibTeX{{\rm B\kern-.05em{\sc i\kern-.025em b}\kern-.08em
    T\kern-.1667em\lower.7ex\hbox{E}\kern-.125emX}}
\newcommand{\unnumberedparagraph}[1]{\paragraph*{#1}\mbox{}} % for getting rid of paragraph enumerations

\begin{document}
%\title{Technology and Gender in Anatomy Education: \\A Comprehensive Feasibility and Ethnography Study of Screen-based Augmented Reality\\ and 3D Visualization Tools}
\title{A Large-Scale Feasibility and Ethnography Study of Screen-based AR and 3D Visualization Tools for Anatomy Education: Exploring Gender Perspectives in Learning Experience}


\author{\IEEEauthorblockN{Roghayeh Leila Barmaki$^{1}$$^{\S}$\thanks{$^{\S}$ The study was conducted during the postdoctoral fellowship of Roghayeh Leila Barmaki at Johns Hopkins University. Correspondence to rlb@udel.edu.}, %0000-0001-5137-1301
Kangsoo Kim$^{2}$,  
Zhang Guo$^{1}$, 
Qile Wang$^{1}$, \\
Kevin Yu$^{3}$,
Rebecca Pearlman$^{4}$, and
Nassir Navab$^{3, 4}$ 
}


\IEEEauthorblockA{$^{1}$ University of Delaware, Newark, DE, USA
}
\IEEEauthorblockA{$^{2}$ University of Calgary, Calgary, AB, Canada}

\IEEEauthorblockA{$^{3}$ Technical University of Munich, Munich, Germany	
}
\IEEEauthorblockA{$^{4}$Johns Hopkins University, Baltimore, MD, USA
}
}


\maketitle

%\begin{abstract}While anatomy learning is an essential part of medical education, there remain significant challenges in traditional learning methods, e.g., related to the limited resources for performing dissection/pro-section in large classes and the learning difficulty in understanding the complexities of the human body and performing proper mental mapping based on two-dimensional (2D) anatomical structures in a textbook.
%Recently, advanced technologies, such as 3D visualization and augmented reality (AR), have reshaped anatomy learning.In this paper, we introduce two in-house anatomy training solutions that can visualize and superimpose 3D virtual anatomy models with informative labels using a hand-held tablet or a wide-screen AR. To investigate the feasibility and effectiveness of the proposed tablet-based 3D visualization and AR tools, we conducted a large-scale study with \textbf{236} students enrolled in undergraduate premedical programs (95 males, 141 females in 118 dyadic teams).In this study, participant students were split into three groups to use one of the following learning tools in a team-based anatomy painting activity: (1) conventional textbook, (2) hand-held tablet-based 3D visualization, and (3) screen-based AR.The results showed that students who used the tablet-based visualization tool or the AR learning tool reported significantly higher (more positive) scores than those who used a textbook.Though we did not observe a significant difference in knowledge retention among the three learning tools, our further analysis of gender effects revealed that male participants generally reported more positive scores than female participants. Also, the overall experience of mixed-gender dyads was reported to be significantly lower than others in most of the learning experience and performance measures. 
% While discussing the implications of our results in the context of anatomy and medical education, we suggest that tablet-based 3D visualizations and AR learning tools have the potential to be considered as a complement or even replacement for textbooks in anatomy learning interventions.
%While discussing the implications of our results in the context of anatomy and medical education, we highlight the potential of our learning tools with additional considerations related to gender and team dynamics in body painting anatomy learning interventions.
%\end{abstract}

\begin{abstract}
    
Anatomy education is an indispensable part of medical training, but traditional methods face challenges like limited resources for dissection in large classes and difficulties understanding 2D anatomy in textbooks. Advanced technologies, such as 3D visualization and augmented reality (AR), are transforming anatomy learning. This paper presents two in-house solutions that use handheld tablets or screen-based AR to visualize 3D anatomy models with informative labels and in-situ visualizations of the muscle anatomy. To assess these tools, a user study of muscle anatomy education involved 236 premedical students in dyadic teams, with results showing that the tablet-based 3D visualization and screen-based AR tools led to significantly higher learning experience scores than traditional textbook. While knowledge retention didn't differ significantly, ethnographic and gender analysis showed that male students generally reported more positive learning experiences than female students. This study discusses the implications for anatomy and medical education, highlighting the potential of these innovative learning tools considering gender and team dynamics in body painting anatomy learning interventions.
\end{abstract}
\begin{IEEEkeywords}
Screen-based Augmented Reality, Collaborative Learning, Evaluation Methodologies, Human-Computer Interface, Gender and Ethnography.
\end{IEEEkeywords}
\section{Introduction} \label{Sec:Intro}
\noindent
Human anatomy and physiology are vital parts of medical education that involve complex functional structures and movements of the human body. 
Comprehensive learning of anatomy and physiology provides a thorough understanding of human body function, enabling more effective treatment of abnormal or disease states~\cite{BLANCHARD200573}.
However, the complexity of the course poses challenges for students in achieving their desired learning outcomes.
Several factors associated with the learning experience can influence these outcomes, including the learning tools, the quality of the material, the student's prior experience, and their emotional concerns~\cite{o2008development, green2018relationship, chan2019approaches}.

%\vspace{2em}

\noindent The most common practice for students in anatomy education is to use textbooks with static images, but this cannot provide the students with a realistic first-hand and interactive experience, which may not be effective for their learning experience and performance~\cite{leung2020modernising}.
Despite the effectiveness of traditional methods for training anatomy, such as dissection or prosection, these methods have become less feasible nowadays due to limited resources, large class sizes, and mainly the absence of face-to-face learning experiences, to name a few constraints.
Due to such limitations, most medical, dental, and other allied health schools have recently declined the practical laboratory hours for anatomy\cite{leung_anatomy_2006,winkelmann_anatomical_2007}.
%To address the change and issues, various complementary methods have been tried and changed over the view of anatomical education. The explosion of advanced technologies during the last few decades has brought anatomical education to another level, such as immersive learning tools.
In anatomy and physiology education, spatial visualization is likely essential for students to learn the dynamics of anatomical structures and spatial relationships to surrounding structures.
%Two-dimensional (2D) views in the textbook do not fully represent such dynamics and relationships with the complexities of human anatomy, and students need to mentally manipulate three-dimensional (3D) relationships by themselves, which is a big challenge.
The traditional visualization of human anatomy in 2D textbook views insufficiently captures the complexity of human anatomy as students are often required to mentally reconstruct 3D spatial relationships, which presents a considerable challenge.


Virtual/augmented reality (VR/AR) can provide information on dynamics and spatial relationships interactively and intuitively by employing 3D virtual skeletons and organs and adding a virtual information layer on top of the physical body.
While various medical training scenarios have used these technologies~\cite{lovis_mixed_2020,marmulla_augmented_2005,Romand2020}, the use of computer-generated 3D models, in particular, allows students to rotate and locate structures from various views and perspectives in anatomy learning. 
Such dynamic visualization techniques improve student visual-spatial abilities\cite{huk_who_2006,lipponen_challenges_1999,stieff_mental_2007}.
%For those students without any dissection experiences, compared with cadavers, 3D models on the screen or virtual/augmented environments are so much easier to interact with and explore, too.
Moreover, virtual 3D visualizations offer more accessible opportunities to engage and explore anatomical structures than traditional cadaver-based learning for their repeatability and monitoring capabilities.


% In this paper, we investigate the use of modern technologies such as augmented reality as a replacement of textbook in an anatomy learning intervention using learning analytics.
% In a team-based experimental study with \note{301} students, we found that students who used our in-house augmented reality platform, and interactive app were as successful as those who used textbook; they are familiar learning instrument; in knowledge tests.
% Overall satisfaction of students on usability and engagement was also highlighted the potential of the proposed learning tools either as supplement or replacement of textbooks in the future of anatomy education.

%In this paper, we introduce and investigate the use of screen-based VR and AR technologies as a replacement for textbooks in an anatomy learning  intervention (muscle painting) using \expMeasure and analytic methods collected from premedical students. 
Body painting has been shown as an effective tool for learning anatomy and associated clinical skills \cite{cookson2018exploration,Diaz2021learning}. It is a motivating and creative experience for students that provides memorable visual images and encourages multisensory and active participation. 
While body painting suits all students, cultural sensitivity, gendered considerations, and careful negotiation may be necessary to ensure all students are comfortable carrying out the activities.

Inspired by renown methods of anatomy education, in this work, we propose a user study in an actual educational laboratory setting that uses 3D visualizations in an anatomy body painting learning task.
% We have compared our in-house tablet-based 3D visualizations, and our screen-based AR \cite{barmaki_enhancement_2019} with textbook to find any interplay between participants' performance outcomes with learning tools, gender, and team gender composition across these three learning tools.
%using statistical analysis.
%\note{either scope down the RQs to the level of tablet learning tool vs screen-AR, or add separate sentence(s) to clarify these are the general RQs while the scope of our experiment is much narrower.}
In a controlled, team-based large-scale study with $236$ participants, we compare our in-house tablet-based 3D visualizations (\emph{Tablet-3D}), and our screen-based AR (\emph{Screen-AR})~\cite{barmaki_enhancement_2019} with the conventional paper-based \emph{Textbook}. We aimed to find any interplay between the participants' performance outcomes with learning tools, gender, and group gender compositions. 
% with three \textit{learning tools} of the conventional paper-based \emph{Textbook}, tablet-based 3D visualization (\emph{Tablet-3D}), and screen-based AR (\emph{Screen-AR}),
Our research aims to address the following research questions:
\begin{itemize}[leftmargin=*]
    \item \textbf{RQ1:} Do tablet-based visualizations and AR technology improve students' learning experience compared to the traditional textbook in anatomy education?
    \item \textbf{RQ2:} Do tablet-based visualizations and AR technology increase the learning outcomes, e.g., test scores or knowledge retention, in anatomy education?
    \item \textbf{RQ3:} Are there any particular benefits of AR technology in anatomy education experience over tablet-based visualizations?
    \item \textbf{RQ4:} Is the student's gender a factor to interplay with the effects of digital technology, or in general for the learning experience?
\end{itemize}
% \begin{itemize}
%     \item \textbf{RQ1:} Is there a significant difference among student score gain based on the different learning tools? 
%     \item \textbf{RQ2:} Is there a difference in student knowledge retention (score gain) among males and females who took part in the activity? 
%     \item \textbf{RQ3:} Is there any meaningful pattern in the gender composition of teams and their score gain?
%     \item \textbf{RQ4:} Is there a difference among learning tools based on user experience questionnaire?
% \end{itemize}
To answer these research questions, we first introduce our technological tools, tablet-based interactive visualization application and large screen-based AR tool, which can show dynamic anatomical information with interactive life-size 3D virtual models on top of a physical body.
We then report the findings from our large-scale study with $236$ students in teams of two who were participated in the body painting activity using three learning tools.
%We examined various measures and performances, comparing these three learning tools.
%: (1) textbook, (2) tablet-based visualization, and (3) screen-based AR.

We found that students who used our Tablet-3D and Screen-AR conditions had more positive (anatomy) learning experiences than those who used a textbook, according to their self-reported outcomes.
%For example, overall scores for satisfaction, learning perception, and perceived easiness were higher with the proposed digital learning tools than with the textbook.Interestingly, the learning outcomes, e.g., test scores of anatomical knowledge, did not show any differences among the three learning tools, in some ways showing that all three conditions can be interchangeably used without compromising the knowledge representation.
In addition, we analyzed the potential effects of the participant's gender on the performance, which will be elaborated more in the following.
%We found that male participants had higher scores in the learning experience measures than female participants.
%We will describe the details of the study and discuss the implications of the results and findings.

\section{Related Work}\label{Sec:RelatedWork}

\subsection{3D Technologies for Anatomy Education}

%\note{add more references about non-immersive VR (e.g., smartphone/tablet-based 3D visualization tools)@Kyle -added 5 citations}

% \subsubsection{[traditional]}
\noindent Anatomy is a complex subject that cannot be learned only from textbooks\cite{nainggolan2020user}.
Traditional anatomy training is based on the dissection and pro-section of the human body, which provides tangible haptic interactions and realistic environment settings\cite{snelling_attitudes_2003,gunderman_exploring_2005}.
% {(Snelling et al., 2003)}.
%Gunderman and Wilson~\citep{gunderman_exploring_2005}
% Gunderman and Wilson (2005)} 
%has illustrated that dissecting helps students respect life and understand their patients. 
%However, despite the advantages, the problems of dissection have been concerned.
Despite the advantages of dissection, it equally raises concerns.
Studies have shown that the learning outcomes and the quality of dissections may be affected by the quality of the material, students' prior experience, and emotional concerns\cite{trelease_going_2000,winkelmann_anatomical_2007}.
% (Trelease et al., 2000)}.
In particular, both inexperienced and experienced medical and healthcare students are frequently appalled by the fear of death and the unnatural smell of cadavers during the dissection\cite{mclachlan_teaching_2004,winkelmann_anatomical_2007}. 
%The unpleasant learning experience, from the students without any background knowledge, also have been shown in the studies~\citep{mclachlan_teaching_2004,winkelmann_anatomical_2007}, based on the fear of death and the smell of the cadavers during the dissection.
% {(McLachlan et al., 2004; Winkelmann, 2007)}.
Instead of using deceased bodies, clay models provide an alternative solution for educators.
DeHoff et al.\cite{dehoff_learning_2011} found that compared with animal dissections, students had a better learning experience with clay models.
% {(DeHoff et al., 2011)}.
However, clay models cannot present complex anatomical regions or the functional movements of the structures in the anatomical domain.
Additionally, they impose challenges in transportation and storage. 
%are usually difficult to share, store, and carry with. 
Moreover, anatomy course lab hours have gradually decreased in the past decades \cite{leung_anatomy_2006,winkelmann_anatomical_2007}, bringing more challenges to anatomy education.
% \subsubsection{[Non-traditional]}

% (Leung et al., 2006; Winkelmann, 2007)}.
\noindent As a response to the barriers and changes over time, anatomical learning platforms increasingly adapt from traditional methods to digital technology. 
Like any other reshaping processes, some anatomy scholars argue that dynamic visualization compensates for students' low spatial abilities by providing an explicit external representation of the system\cite{hays_spatial_1996,huk_who_2006,stieff_mental_2007,chickness2022novel}.
% (Hays, 1996; Huk, 2006; Stieff, 2007)}.
Increasingly powerful and accessible computer hardware allow 3D visualizations to replace or supplement traditional teaching in healthcare regarding lectures, cadavers, and textbooks\cite{yammine_meta-analysis_2015,golenhofen_use_2019,lemos_design_2019,maresky_virtual_2019}.
%Virtual Human Dissector© (VHD) software~\citep{donnelly2009virtual} is one of the interactive teaching tools for cross-sectional anatomy, capable of reconstructing 3D views from 2D images. 
Donnelly et al.\cite{donnelly2009virtual} investigated the use of Virtual Human Dissector© (VHD) software, interactive teaching tools for cross-sectional anatomy, capable of reconstructing 3D views from 2D images, in anatomy education with self-directed learning and found no significant difference when compared with a students group that learns from using prosection, models, and textbooks. 
Kennan \& Awadh \cite{keenan2019integrating} discussed the effective utilization of visual 3D learning technologies as self-learning resources in the context of cross-sectional anatomy.
They proposed integrating the use of the 3D VHD system with Sectra\cite{barrack2015step}, a medical imaging device, to enhance the understanding of cross-sectional anatomy.

Lim et al.\cite{lim_use_2016} used 3D-printed models instead of traditional cadaveric specimens during the learning of external cardiac anatomy. 
Although 3D printing technology can provide teaching materials, the 3D printed model is limited by the complexity of anatomical regions. 
Equally, researchers found that 3D visualization methods improved student performance by providing multiple anatomical views and different perspectives of 3D rotating models, even on 2D screens \cite{yammine_meta-analysis_2015}.
% (Yammine \& Violato, 2015)}.
Mobile-based applications and web-based 3D games have also been used as efficient learning tools for the study of human skeletal, muscular, and cardiovascular systems to explore more spatial information about the 3D anatomical models \cite{golenhofen_use_2019,lemos_design_2019}.
% {(Golenhofen et al., 2019; Lemos et al., 2019)}. 

VR and AR techniques have been adopted into anatomy education in recent years\cite{maresky_virtual_2019,bork_empirical_2017,silva_emerging_2018,bacca2014augmented}.
% {(Maresky et al., 2019; Nicholson et al., 2006; Silva et al., 2018)}.
As dynamic visualizing tools, they engage students in an immersive environment with audio and visual interactions and stereoscopic 3D models to have a better functional understanding of the anatomical structure and its movement within the 3D body space\cite{hackett_effect_2018,jacob2012lindsay,luursema_role_2008,preim2018survey}.
% {(Hackett \& Proctor, 2018; Luursema et al., 2006, 2008)}.
Duncan‐Vaidya and Stevenson\cite{duncan-vaidya_effectiveness_2020}
% Duncan‐Vaidya \& Stevenson (2020)} 
have found that experience from AR positively influences the learning process of skull anatomy on a similar level as traditional tools such as textbooks or plastic skull models.
Increasing the frequency of learning instances and interactions with models and specimens are advantages of teaching anatomy in AR\cite{Romand2020}.
% \note{(Romand et al., 2020)}.
Kolla et al.\cite{kolla2020medical} examined the effectiveness of VR technology in anatomy education and compared it to traditional teaching methods like lectures and cadaveric dissection.
28 first-year medical students used a VR headset to identify anatomical structures, drew them on a virtual skeleton, and then provided feedback through surveys. 
Their results indicated that VR was highly supported by the students, demonstrating its potential as a valuable tool for learning human anatomy and as a useful complement.
However, all study conditions were conducted within VR settings.
In a different study, the AR controlled group that was randomly selected from a biochemistry course suggests that AR educational apps motivated them to understand the visualized processes\cite{barrow_augmented_2019}.
% {(Barrow et al., 2019)}.
Despite the popularity of using VR/AR methods in anatomy education, a review paper \cite{chytas2022extended} of 152 articles did not identify conclusive evidence of their efficiency over traditional anatomy education methods.



% In this paper, we would like to evaluate the user experience from both individual and team in an anatomy study, which leverages modern anatomical content visualization in 3D with handheld mobile/tablet devices and large-scale AR displays.


\subsection{Measures of Anatomy Learning Experience}
%\note{elaborate a bit more on the references. R2 complained about the part of Nainggolan et al.}
\noindent To analyze the learning experience and evaluate the effectiveness of VR/AR applications, various methods for data collection and different measures were used in previous research.
Kurniawan and Witjaksono~\cite{kurniawan2018human} evaluated the usefulness of the mobile-based AR application by employing the attitude questionnaire to analyze user perception. 
%using the Likert Scale for user perceiving analysis. 
% Nainggolan et al.~\citep{nainggolan2020user} conducted a survey, scored, and evaluated the user agreement level and user satisfaction level of each anatomical function of a VIVE Controller.
Nainggolan et al.~\cite{nainggolan2020user} evaluated the interactivity level of VR controller based on the user's agreement and satisfaction levels, and found that the use of the VR controller in the anatomy learning system was very interactive and satisfactory.
Tanjung et al.~\cite{fahmi2020comparison} conducted a comparison learning experience study to evaluate the level of acceptability and satisfaction towards three anatomical learning systems.
In our previous AR anatomy learning research ~\cite{barmaki_enhancement_2019,bork_empirical_2017,barmaki2020deep, bork2017exploring} and another study by Duncan‐Vaidya and Stevenson~\cite{duncan2021effectiveness} pre- and post-knowledge quizzes, and a usability questionnaires were used for data collection and analysis of the effectiveness of the AR tool.


%learning experience (or user experience) in (anatomy) education.



%learning performance (knowledge retention)
% \subsubsection{Knowledge Retention}

% Compared to individual and competition efforts, collaborative learning strategies provide positive effects on achievement and productivity~\citep{johnson_effects_1981}.
% % {(Johnson et al., 1981)}.
% Previous research provided evidence on reduced time to complete learning tasks in well-format collaboration settings~\citep{williams_support_2001},
% % {(Williams \& Upchurch, 2001)}, 
% improved student's understanding of learning process~\citep{declue_pair_2003},
% % (DeClue, 2003)}, 
% and improved their performance on exams~\citep{declue_pair_2003}, p.\! 1).
% % {(DeClue, 2003, p. 1)}.
% Another study has shown that interactions during the learning process help students' long-term retention of key ideas~\citep{perez-sabater_active_2011}.
% % {(Pérez-Sabater et al., 2011)}.
% With the rapid development of medical science, the explosive knowledge in the anatomy educations raises serious problems and knowledge retention, as the essential learning goal, has been used for learning evaluation.
% In evaluating the performance during the learning process, researchers often use formal assessment~\citep{van_boxtel_collaborative_2000},
% % {(Van Boxtel et al., 2000)}, 
% quizzes, exams~\citep{sangin_facilitating_2011},
% % {(Sangin et al., 2011)}, 
% and self-reported survey~\citep{terenzini_collaborative_2001}
% % {(Terenzini et al., 2001)} 
% has always been adapted to measure students' learning outcome and knowledge retention.


\subsection{Gender Effects in Education}

\noindent In different education domains, gender differences have been discussed in recent studies.
Understanding gender effects in education is not conclusive, and it varies based on different disciplines and tasks \cite{fernandez-sanz_analysis_2012,wegge_age_2008}.
% {(Fernandez-Sanz \& Misra, 2012; Wegge et al., 2008)}.
In the science, technology, engineering, and mathematics (STEM) domain, some non-positive learning experiences for females have been reported due to gender biases at the technical level~\cite{meadows_interactive_2015,nagappan_improving_2003}.
% {(Meadows et al., 2015; Nagappan et al., 2003, p. 1)}.
In the business domain, females' higher management ability in group tasks was highlighted \cite{bear_role_2011,de_paola_teamwork_2018,eagly_female_2003}.
% {(Bear \& Woolley, 2011; De Paola et al., 2018; Eagly \& Carli, 2003)}.
Previous research has also shown that during cognitive tests, females have better information-processing skills than males \cite{rabbitt_unique_1995,schaie_age_1993}.
% {(Rabbitt et al., 1995; Schaie \& Willis, 1993)}.
%Conversely, no significant gender effect has been reported in the study. 
In Andersson's study \cite{andersson_net_2001}
% Andersson's study (2001)}
on explicit spatial and verbal collaborative memory performance, the author reported better retention performance for females, but he argued that there was no main gender effect on team performance. 
Prinsen et al.\cite{prinsen_gender-related_2007}
% {Prinsen et al. (2007)}
noted that, in computer-mediated communication \cite{herring_computer-mediated_1996}
% {(Herring, 1996)}
and computer-supported collaborative learning settings \cite{lehtinen_computer_1999,lipponen_challenges_1999,barmaki2020deep}, 
% {(Lehtinen et al., 1999; Lipponen, 1999)}, 
learning performance for different genders might change by various role distributions. 
In this paper, we will discuss the ethnographic observations related to gender differences (from individual and group composition standpoints. We need to acknowledge that we recruited students from a large laboratory classroom with pre-assigned teams thus, it was not feasible to form ethnographically balanced teams for all study conditions.


\section{Methods and Materials}\label{Sec:Method}

\noindent In this section, we will describe the details of the conducted study with our proposed tablet-based 3D visualization and screen-based AR learning tools.
Relevant hypotheses were established and evaluated to address our general research questions introduced in Section~\ref{Sec:Intro}.
% through a large-scale user study.
This study was approved by the Institutional Review Board (Protocol \#HIRB00005021). %\note{TODO}


\subsection{Digital Anatomy Learning Tools}
\label{Sec:ToolsAndIntervention}
% \subsection{System Overview}

%\note{more details about the learning tools, what information were provided, how the systems were implemented, etc.}
%\note{gender effect: how the participant pairs were created, did they know each other beforehand? mixed gender? etc.} Leila: This was part of the initial description but for some reason was removed along the way.. I added it to participants

% \note{Figure 1} shows our three study conditions of the learning activity.
% Since there were also three concurrent lab sessions, we had one workstation per lab. 
% As shown in \note{Figure 1.c}, the mobile workstation had one Alienware 17 Laptop with GTX 1070m GPU, a 55-inch Samsung TV, and a Microsoft Kinect One tracking sensor; all mounted on a portable TV cart.
% The TV was either streaming the camera feed from Kinect for tablet and textbook settings, or the 3D anatomical overlays atop participants, with a 3D virtual model in the AR setting. 
% The AR setting was designed to use a hand-held clicker to navigate through the learning platform. 
% Students could navigate among the muscle groups and see the highlighted muscles and labels overlay atop their bodies, besides rotating and zooming in/out the virtual 3D model using the clicker. 
% For the tablet setting, we used Samsung Galaxy Tab 2 with 10.1 in screen size---comparable to textbook anatomy diagram size---and installed our interactive app on it.
% In the tablet condition, for sake of consistency, the interactive app had the exact content as the AR system, with two main differences: (a) the interaction with tablet was based on touch, similar to a smartphone, instead of a clicker and (b) self-augmentation was not present.
% Other key properties for 3D anatomy model visualization, including regional and full-body anatomy visualization, 3D model rotation, zoom in and out, toggling the labels for enhancing readability, in addition to the highlighting key muscles for the painting activity, and color-coded labels were similar for AR and tablet platforms (see \note{Figures 1.e} and \note{1.f}). 

% Figure 1. (a-c) Three study settings for students to complete anatomy learning activity, and (d-f): a closer look of learning tools of textbook, tablet, and screen-based AR system, all for a specific muscle group. 

\noindent To investigate the effects of tablet- and AR-based learning tools in anatomy education, we prepared in-house tablet-based 3D visualization and screen-based AR for our user study. 
The learning tools are presented in Figure~\ref{fig:tools}.
% \note{Textbook}
% \note{Tablet-based VR App}
For Tablet-3D, we developed an interactive Android application for hand-held devices, which visualizes 3D virtual models of body muscles and labels on a tablet (see Figure~\ref{fig:tools}(b)).
The visualization of the 3D anatomy included regional and full-body anatomy models.
Participants could rotate, adjust zoom levels, toggle between color-coded virtual labels for enhancing readability, and highlight relevant muscle groups for our user study (see Figure~\ref{fig:tools}(c)).
The developed application was installed on Samsung Galaxy Tab 2 with a 10.1-inch screen.
% ---comparable to textbook anatomy diagram size---and installed our interactive app on it.
% \note{Life-size AR Mirror}

For Screen-AR, we deployed our screen-based AR tool onto a large TV display with a mounted Kinect v2 camera. 
Screen-AR used a split-screen to show an AR view on the left side and a focused view of the anatomical model on the right side. 
%Compared to Tablet-3D, the pose of the virtual anatomy was animated based on the body tracking data of the Kinect v2.
Furthermore, body tracking allowed Screen-AR to superimpose 3D life-size virtual models of human anatomical systems, such as the muscular system, and labels atop the participant's mirrored body.
% in a mirror-like AR visualization.
Participants could navigate between the relevant muscle groups using a hand-held clicker.
Depending on the muscle group, the virtual camera responsible for rendering the scene on the right screen automatically follows and zooms into the selected muscles.
Screen-AR consists of an Alienware 17 Laptop with GTX 1070m GPU, a 55-inch Samsung TV, and a Microsoft Kinect v2 tracking sensor, and all devices were mounted on a mobile TV cart.
% for the mobility of the system.
%The TV displayed the 3D anatomical overlays on top of the participant's view, while the Kinect sensor detected and tracked the participant's body in front of the TV screen.
%For the sake of consistency, both the AR tool had the exact contents as the VR tool, but with two main differences: 
We predefined the same muscle groups for Tablet-3D and Screen-AR for consistency.
% , which were the same for both conditions. 
%The main differences between Tablet-3D and Screen-AR are as follows:the interaction with Tablet-3D was based on touch-screen while Screen-AR was controlled by the clicker and the users' body pose, and Screen-AR, unlike Tablet-3D, provided a self-augmentation view of anatomical models.

\begin{comment}
% Figure environment removed
\end{comment}

% Figure environment removed
%Utilizing these VR/AR learning tools, w

\subsection{Team-based Learning Intervention}

\noindent After consultation with our laboratory instructor, Dr. Pearlman about the possibilities of testing our anatomy learning tools with her large classroom, we were introduced to a mandatory team-based lab activity about human muscle painting\cite{marieb_essentials_2006}.
% {(Marieb \& Jackson, 2006)}
Muscle painting, as a form of body painting, has been shown to be one of the common exercises in a medical curriculum\cite{mcmenamin_body_2008, barmaki_enhancement_2019}, and students execute the muscle painting activity typically with printed 2D visualization of human musculature from the lab manual. So, we prepared our learning tools based on these muscle painting activities, aiming to use them as a replacement for the 2D anatomy textbook and evaluate their learning efficacy and student performance.
%in-house VR and AR learning tools for interactive 3D visualization.
During the intervention, participants worked in a team of two people to collaboratively learn and teach among peers about human muscle groups through a body painting activity.
They were asked to identify and paint major muscle parts on their body with washable painting supplies while using one of a randomly assigned (see Section \ref{Sec:StudyDesign}) learning tool.
%, while using one of the learning tools: (1) textbook or lab manual, (2) tablet-based VR app, and (3) mirror-like AR visualization tool. 
Once one of the participants completed the painter's role, they switched their role to be a paintee with the learning partners.
We showed 40 muscles and labels to the participants to provide extensive anatomical landmarks. 
However, we only asked them to identify 12 major muscles (same for all conditions) and use appropriate body paint to colorize the muscles' location on their arms and legs.

% Our study had three different settings based on instrumental tools. 
% Students in the control group used textbook or the lab manual as their learning tools. 

% In experimental group I, instead of a textbook, students used our in-house interactive app on the tablet as a 3D musculoskeletal visualizing system.

% Experimental group II used a screen-based AR system—also developed internally—where students could see themselves with augmented anatomy visualizations on a large display.

% The knowledge-base information, presented in all instrumental tools, were identical across the board to mitigate potential confounding factors related to student workload and learning (exactly 40 muscle names and their locations were presented to students in all learning tools). 



\subsection{Study Design}
\label{Sec:StudyDesign}

\noindent We conducted a user study using a between-subjects design with three \textit{learning tool} conditions.
A brief description of each \textit{learning tool }is explained in the following.
%, but the details of our in-house VR and AR tools were introduced in Section~\ref{Sec:ToolsAndIntervention}.
\begin{itemize}[leftmargin=*]
    \item \textbf{Textbook}: As a traditional learning method, participants used a textbook during the team-based anatomy learning intervention described in Section~\ref{Sec:ToolsAndIntervention}.
    They identified the corresponding muscle parts on their partner's body while checking the location of target muscles from the textbook (Figure~\ref{fig:tools}(a, d)).
    The textbook could be carried while performing the muscle painting activity.
    \item \textbf{Tablet-3D}: Participants used our tablet-based 3D anatomy application during the learning intervention.
    The participants could carry the tablet that visualizes 3D virtual anatomy models and labels, while performing the painting activity (Figure~\ref{fig:tools}(b, e)).
    \item \textbf{Screen-AR}: Participants used our large screen-based AR anatomy tool during the activity.
    They could see the virtual 3D models and labels directly overlaid on their own bodies through the large screen (Figure~\ref{fig:tools}(c, f)).
\end{itemize}
% \note{we can skip # of students and just talk about lab assignments since participants section is just below. What about this: We had a total of 17 anatomy lab sessions with a maximum capacity of 20 in each lab. Considering the large scale of the study, participants could be randomly assigned to one of the three learning tool conditions by randomly scheduling different conditions to each lab session.}
The study was performed alongside 17 anatomy lab sessions with a maximum capacity of 20 students in each lab.
Considering the large scale of the study, participants could be randomly assigned to one of the three learning tool conditions by scheduling a single condition for each lab session.
This assignment method is termed hierarchical or clustered randomization, which is a common practice in educational studies and clinical trials\cite{davis_application_2002}.
% {(Davis et al., 2002)}.

% \subsection{Collaborative Learning Intervention}

% We conducted a between-subjects study of team-based anatomy learning intervention in a laboratory course of General Biology as part of undergraduate premedical curricula. 
% \note{321} students in \note{138} teams participated in the user study.
% The intervention was inspired by a manual laboratory activity on muscle painting~\citep{marieb_essentials_2006}
% % {(Marieb \& Jackson, 2006)}
% using 2D visualization in the lab manual, but we adapted it for using other learning tools for visualization.
% During this intervention, students worked in teams to learn more about human muscle groups in a body painting activity.
% They were expected to identify and paint major muscles of their body using one of the learning instruments (textbook or lab manual, tablet, and AR) and washable painting supplies. 
% While 40 muscles and labels were shown to students, they were expected to identify 12 muscles out of these and paint them on each other's extremities (six muscles per student for teams of two). 

% Our study had three different settings based on instrumental tools. 
% Students in the control group used textbook or the lab manual as their learning tools. 
% In experimental group I, instead of a textbook, students used our in-house interactive app on the tablet as a 3D musculoskeletal visualizing system.
% Experimental group II used a screen-based AR system—also developed internally—where students could see themselves with augmented anatomy visualizations on a large display.
% The knowledge-base information, presented in all instrumental tools, were identical across the board to mitigate potential confounding factors related to student workload and learning (exactly 40 muscle names and their locations were presented to students in all learning tools). 


\subsection{Participants}
% Each student in this course attended laboratory one afternoon a week for three hours, in a room with up to 25 total students and one teaching assistant. There were three such rooms operating concurrently each day that the laboratory was in session. Considering the size of the laboratory, instructors collaborated closely with all of the 15 teaching assistants and laboratory manager to guide students. Students were assigned by their instructor into teams of two to four for performing their assignments throughout the semester

% In total, we had \note{119??} teams of size two, \note{13?? teams of size three, and 6?? teams of size four}.
\noindent We initially recruited 319 student participants (male: 128, female: 191), although we needed to exclude a subset of the data (described in the following). The recruitment took place via an online flyer from a laboratory course of General Biology as part of undergraduate premedical curricula. The recruitment was part of a general biology lab course, and the students were randomly assigned by their instructor into teams of two to four to perform their assignments throughout the semester.
The initial anatomy learning intervention  had teams of sizes two to four, however, teams of size two were only considered for this work. Larger teams had different task distribution in team, thus excluded. So, our final subset data of interest included \textit{\textbf{236}} participants (95 males and 141 females; age $M\!=\!19.77$, $SD\!=\!1.81$) in \textbf{118} teams of size two 
%\note{[ZG: STATA 32]}
within three age-balanced groups with the following  gender compositions. Male pairs: 62 with age $M\!=\!19.45$, $SD\!=\!1.035$, Female pairs: 108 with age $M\!=\!19.71$, $SD\!=\!1.583$, and Mixed pairs: 66 with age $M\!=\!20.17$, $SD\!=\!2.527$). %Participant's gender and team gender composition interplay with Learning Experience and Learning Performance is investigated. 
Table~\ref{Tab:Participants} shows the number of participants per condition.

% \vspace{-2ex}
\begin{table}[tb]
\caption{The number of participants in each study condition.}
\label{Tab:Participants}
% \vspace{-2ex}
\centering
\begin{tabular}{c|c|c|c}
\toprule
\textbf{Learning Tool} & \textbf{Male} & \textbf{Female} & \textbf{Total} \\
\midrule
Textbook  & 27   & 45     & 72    \\
Tablet-3D & 43   & 45     & 88    \\
Screen-AR & 25   & 51     & 76    \\
\midrule
\textbf{Total}     & 95   & 141    & 236  \\
\bottomrule
\end{tabular}
\vspace{-2ex}
\end{table}


% A total of \note{301 students (179 females) in 138 teams} were selected in this study.
% Although data were collected from \note{321} participants, data from \note{20} individuals were excluded due to incomplete team information. 
% In total, we had \note{119} teams of size two, \note{13 teams of size three, and 6 teams of size four}.
% \note{Table 1} shows the number of individuals and teams assigned to each study condition. 
% % We first assured that participant assignment to study conditions was properly performed, and the study is balanced based on participant demographics (gender) and anatomy prior knowledge.
% % The Chi-square test did not reveal any difference in the distribution of participants based on genders on three study conditions ($\chi^2=3.06$, df=2, ns).
% % Similarly, one-way ANOVA did not identify any interaction of pre-test and study conditions (F(2,298)=0.14, ns).
% Although data were collected from \note{321} participants, data from \note{20} individuals were excluded due to incomplete team information. 
% \note{Table 1} shows the number of individuals and teams assigned to each study condition. 

% We first assured that participant assignment to study conditions was properly performed, and the study is balanced based on participant demographics (gender) and anatomy prior knowledge.
% The Chi-square test did not reveal any difference in the distribution of participants based on genders on three study conditions ($\chi^2=3.06$, df=2, ns).
% Similarly, one-way ANOVA did not identify any interaction of pre-test and study conditions (F(2,298)=0.14, ns).


% \subsection{Study Design and Procedure}
\subsection{Procedure}

% Our user study had a 3$\times$1 between-subjects research design.
% Considering the scale of the study with \note{321} participants, each lab session (out of 17 lab sessions) was randomly assigned to one of the study conditions.
% This method of participant recruitment and assignment is termed hierarchical or clustered randomization, and it is a common practice in educational studies and clinical trials~\citep{davis_application_2002}.
% % {(Davis et al., 2002)}.

% The study procedure was as follows. 
% An online flyer was sent to all undergraduate students enrolled in the General Biology lab, inviting them to participate in the study.
% The study was approved by Institutional Review Board in the institution study carried out (Protocol \# XXXXXXX), and oral informed consent was obtained from each participant before the study commenced. 
% After consent, students completed the online pre-questionnaire individually and then entered the painting activity room with their preassigned teammates. 
% Each team either used (1) anatomy interactive app on the tablet, (2) screen-based AR system as two experimental groups, or (3) textbook as control groups to complete the team-based learning activity.

\noindent The study procedure was as follows. 
An online flyer was sent to all undergraduate students in the General Biology lab, inviting them to participate. Students needed to complete the body painting activity regardless of our study for their lab credits, but they could opt in to participate in our study to perform it slightly differently.
After the oral consent process, participants individually completed an online pre-questionnaire, which asked for their demographic information and evaluated their prior anatomy knowledge. 
Then they entered the learning intervention room with their pre-assigned teammates.
Each team only used one of the learning tools described in Section~\ref{Sec:StudyDesign} to complete the learning task, because of between-subjects study arrangements. 
%: (1) Textbook, (2) Tablet-VR, and (3) Screen-AR, .
% Figure 1 presents study settings, visualization, as well as the painting activity setup. 
% For the AR system, the model (or ``paintee'') was asked to stand closer to the Kinect body tracking sensor to have the appropriate digital anatomical illustrations superimposed on their body. 
% Then, students found the appropriate muscle to be painted using the presenter clicker to navigate the software.
% Next, the painter used the muscle overlay information shown on the TV screen to paint the muscle on the model's skin.
% After finishing painting upper limb muscles, students switched roles in order to paint the lower limb muscles.
% Teams in the tablet and textbook groups also had a workstation in their activity room, but it was served as the data collection purpose from teams. 
% Students of tablet or textbook groups used either the interactive app's 3D visualizations or the laboratory manual anatomy figures to complete the activity. 
% After completing the painting, all of the students completed the online post-questionnaire individually.
% Students then presented their painted limbs to their teaching assistants. 
During the painting activity, the participant in the painter's role tried to find appropriate muscles on the teammate's body (who played the role of a paintee) to be painted while navigating different muscle parts.
Participants in the Tablet-3D or textbook conditions used either the interactive application's 3D visualizations or the laboratory manual anatomy figures to complete the activity.
Particularly for the AR system, the participant in the paintee's role was asked to stand closer to the Kinect body tracking sensor to have appropriate digital anatomical illustrations superimposed on their body. 
The painter used the muscle overlay information on the TV screen to paint.
% After finishing painting upper limb muscles, students switched roles in order to paint the lower limb muscles.
% Teams in the tablet and textbook groups also had a workstation in their activity room, but it was served as the data collection purpose from teams. 
After completing the painting activity, all participants completed the online post-questionnaire individually, which asked about their interaction with the learning tools and evaluated their anatomy knowledge retention.
Before concluding the study session, they presented their painted limbs to their lab assistants.
The activity, including learning intervention and questionnaire completion, took approximately 20--30 minutes.
% , so that their learning performance could be objectively scored. 







% \subsection{Evaluation Measures and Research Questions}
\subsection{Measures and Hypotheses}
\label{Sec:MeasuresHypotheses}

% In this section, we introduce research questions to inform this work based on user study evaluation measures. 
% It is worth noting that our study used both individual and team-based measures to better investigate the impact of various advanced learning tools on anatomy learning.

\noindent This section describes the measures we used for the study, which were collected through questionnaires.
We also introduce several hypotheses that we established based on the measures and our research questions in Section~\ref{Sec:Intro}.
% It is worth noting that our study used both individual and team-based measures to better investigate the impact of various advanced learning tools on anatomy learning.
We used online questionnaires on the Qualtrics platform (Provoto, UT) for designing and collecting pre- and post-questionnaires.


% \subsection{Questionnaires}

% \subsubsection{User Engagement and Satisfaction}
% \subsubsection{Usability}
% \subsubsection{User Experience}
\subsubsection{Learning Experience}

We prepared nine subjective questions to examine the participants in the anatomy learning intervention using different learning tools.
The questions were presented on a five-point Likert scale (1: Strongly Disagree to 5: Strongly Agree) except for the ``willingness to recommend'' measure, which had a scale of 0 to 10 (10 means a strong willingness).
Our questions for each measure is described below. \textit{Textbook, Tablet-3D, or Screen-AR} was replaced by \textit{learning tool} statement, depending on the assigned condition.
% All students who participated in the study took pre- and post-questionnaires before and after the muscle painting activity. 
% The exit survey or post-questionnaire collected student's opinions on their favorite sub-activity, in addition to their user experience, while interacting with one of the learning tools.
% From the usability standpoint, we were interested in addressing the following research questions:
% \textbf{RQ4.} Is there a difference among learning tools based on user experience questionnaire?
\begin{itemize}[leftmargin=*]% \paragraph{Role Preference}
% In this activity:
% o	Both my arm and leg were painted  (1) 
% o	My arm was painted ‎  (4) 
% o	My leg was painted ‎  (5) 
% o	I painted my partner’s skin ‎  (6) 
% o	None of the above - I had a different role (e.g. locating the muscles)  (7) 
% Which part of the activity did you like the most?
% o	Painting  (1) 
% o	Being the model to be painted  (2) 
% o	Both painting and being the model  (3) 
    \item [-] \textbf{Easy to Paint}: ``Using \textit{learning tool} was easy and straightforward for completing the muscle painting activity.''
    \item [-] \textbf{Easy to Find Muscles}: ``I recognized and found the location of muscles easily using \textit{learning tool}.''
    \item [-] \textbf{Satisfaction}: ``Using \textit{learning tool} for the muscle painting activity was satisfying.''
    \item [-] \textbf{Learning Perception}: ``I learned a lot about muscle anatomy using/interacting with \textit{learning tool}.''
    \item [-] \textbf{Enjoyment}: ``I found using \textit{learning tool} to be enjoyable.''
    \item [-] \textbf{Effort to Focus}: ``I had to make an effort to keep my mind on the activity.''
    \item [-] \textbf{Lost Track of Time}: ``Time seemed to pass very quickly during the painting activity.''
    \item [-] \textbf{Willingness to Recommend}: ``I would recommend that my friends use \textit{learning tool} to study human muscles.''
    \item [-] \textbf{Learning Motivation}: ``Using \textit{learning tool} increased my enthusiasm for learning more about human anatomy.''
\end{itemize}

%\note{check references as theoretical foundation for H1, H2, H5, and H6}
\noindent Considering the potential benefits of interactive visualizations in our 3D visualization tool, and more intuitive and direct information display on the real body in the AR tool \cite{blum2012mirracle}, we established the following hypotheses for the  measures:

\begin{itemize} [leftmargin=*]
    \item \textbf{H1}: The participants in the Tablet-3D condition or the Screen-AR condition will have more positive ratings than those in the Textbook condition for all the measures. \\\textit{\textbf{(Textbook $<$ Tablet-3D, Screen-AR)}}
    \item \textbf{H2}: The participants in the Screen-AR condition will further have more positive experience than those in the Tablet-3D condition for all the measures. \textit{\textbf{(Tablet-3D $<$ Screen-AR)}}
\end{itemize}

%Also, given the prior research showed that females tend to be more conservative about being touched than males~\citep{finn2010qualitative} and that males tend to be more interested in computers and technology than females~\citep{Jones2000}, we also established the following hypothesis for the gender difference in the \expMeasure during the intervention:

%\begin{itemize}
%\item \textbf{H3}: The male participants in the study will report more positive \expMeasure than the female participants, particularly in the measures of satisfaction, enjoyment, and willingness to recommend. (Female $<$ Male)
%\end{itemize}


\noindent As Figure~\ref{fig:tools} shows, the anatomy models in all learning tools are male-based. 
Compared to female students, male students can easily recognize and access the muscles according to the same anatomical structures.
Also, given the prior research has shown that females tend to be more conservative about being touched~\cite{finn2010qualitative},
%and that males tend to be more interested in computers and technology than females\cite{Jones2000}, 
we established the following hypotheses to understand the role of the gender, gender composition:

\begin{itemize}[leftmargin=*]
\item \textbf{H3}: The male participants (male pairs) in the study will report more positive ratings than the female participants (females or mixed pairs), particularly in the measures like easy to paint and easy to find muscles.
%, learning perception, and enjoyment. 
\\\textit{\textbf{(Female $<$ Male)}} and \textit{\textbf{(Mixed $<=$ Females $<$ Males)}}
\end{itemize}

\subsubsection{Short-term Knowledge Retention}
% \paragraph{Pre-questionnaire}
% \paragraph{Post-questionnaire}
%\note{R3: is it really a performance measure? or just a short-term memory.}
To examine the learning performance in anatomy education, we evaluated the participant's anatomy knowledge retention.
We collected the participants' anatomy knowledge scores from pre- and post-tests, and calculated the score gain by subtracting the pre-test score from the post-test score.

\begin{itemize}[leftmargin=*]    
    \item [-] \textbf{Pre-Score}: The pre-test was a matching test with five questions in a provided diagram of the human anatomy muscle system. 
    In the anatomy diagram, 15 regions of the body were highlighted, and participants were asked to match five muscle labels provided in the pre-test with one of these 15 body regions.
    The number of correct matches was reported as the pre-test score in the range of [0, 5].
    \item [-] \textbf{Post-Score}: The post-test had a different (lateral) view of human anatomy, with a matching test similar to the pre-test.
    The number of correct matches was reported as the post-test score in the range of [0, 5].
    Both pre- and post-tests were designed and evaluated by anatomy instructors to be at the same level of difficulty.
    \item [-] \textbf{Score Gain}: Score gain, as the difference between pre- and post-test scores, was calculated in [-5, 5] range.
\end{itemize}
% The post-questionnaire had questions related to the intervention (favorite sub-activity), user experience evaluation (see \note{Table 4} for the complete list of the topics), and a post-test. 
% The post-test had a different (lateral) view of human anatomy, with a matching test similar to the pre-test.
% The number of correct matches was reported as the post-test score.
% Both pre- and post-test was designed and evaluated by anatomy instructors to be at the same level of difficulty and with an equal distribution of questions among upper and lower limbs. 
% The average of test scores of all team members was reported as team pre-test and team post-test scores. 	

% \paragraph{Knowledge Retention}
% As mentioned in section 3.3 for questionnaires, we solicitepd both pre- and post-test from students. 
% We reported these values as test scores, and it was in the range of [0, 5]. 
% Score gain, as the difference between pre- and post-test scores, was reported in the range of [-5, 5]. 
% For team scores, we simply identified test scores from each team member, using their Team-ID and averaging the scores. 
% For example, the team post-test score was the average of post-test scores of individual members of the team. 

% \textbf{RQ1.} Is there a significant difference among student score gain based on the different learning tools? 


% \subsubsection{Gender Composition}

% The distribution of males and females in the study was proportional to the overall ratio in medical fields (slight dominance of females).
% For this specific study, we are interested in addressing two research questions related to gender effects on individual and team levels. 

% \textbf{RQ2.} Is there a difference in student knowledge retention (score gain) among males and females who took part in the activity? 

% \textbf{RQ3.} Is there any meaningful pattern in the gender composition of teams and their score gain?


\noindent Based on the positive outcome that we anticipate in the Tablet-3D and Screen-AR conditions, we established the following hypotheses similar to H1 and H2 regarding the learning performance (or the improvement of anatomy knowledge retention):
\begin{itemize}[leftmargin=*]
    \item \textbf{H4}: The participants in the Tablet-3D condition or the Screen-AR condition will have a higher score gain than those in the Textbook condition. 
    \\\textit{\textbf{(Textbook $<$ Tablet-3D, Screen-AR)}}
    \item \textbf{H5}: The participants in the Screen-AR condition will further have a higher score gain than those in the Tablet-3D condition.   \textit{\textbf{  (Tablet-3D $<$ Screen-AR)}}
\end{itemize}


\section{Results}\label{Sec:Results}

\label{Sec:Results}

\noindent This section reports our analysis results considering the hypotheses we established in Section~\ref{Sec:MeasuresHypotheses}.
Since we are also interested in the possible effects of participant's gender on our measures, we have two factors to consider in our analysis: (1) learning tool and (2) participant's gender.
We first conducted two-way ANOVAs with these two factors to see if there is any interaction effect between the factors on both and performance measures.
We did not find any significant interactions between the learning tool and the participant gender; thus, we focused on the main effects of each factor using one-way ANOVAs ($\alpha\!=\!0.05$).
Multiple comparisons with Bonferroni correction were conducted for post-hoc tests.


%\vspace{-2em}
% \begin{table*}[]
% \caption{}
% \label{tab:my-table4}
% \resizebox{\textwidth}{!}{%
% \begin{tabular}{lcccclccccl}
% \hline
% \multicolumn{1}{c}{\textbf{Variable}} & \textbf{F} & \textbf{p} & \textbf{Sig} & \textbf{$\eta_{p}^{2}$} & \multicolumn{1}{c}{\textbf{Variable}} & \textbf{F} & \textbf{p} & \textbf{Sig} & \textbf{$\eta_{p}^{2}$} &  \\ \hline
% Easy to Paint &  &  &  &  & Effort to Focus &  &  &  &  &  \\
% \textit{\hspace*{0.2cm} Learning tool} & 7.786 & 0.001 & * & 0.063 & \textit{\hspace*{0.2cm} Learning tool} & 1.679 & 0.189 &  &  &  \\
% \textit{\hspace*{0.2cm} Gender} & 1.854 & 0.175 &  &  & \textit{\hspace*{0.2cm} Gender} & 0.598 & 0.44 &  &  &  \\
% Easy to Find Muscles &  &  &  &  & \textit{\hspace*{0.2cm} Gender composition} & 1.84 & 0.161 &  &  &  \\
% \textit{\hspace*{0.2cm} Gender} & 7.065 & 0.008 & * & 0.029 & Lost Track of Time &  &  &  &  &  \\
% \textit{\hspace*{0.2cm} Gender composition} & 5.62 & 0.004 & * &  & \textit{\hspace*{0.2cm} Learning tool} & 0.332 & 0.718 &  &  &  \\
% Satisfaction &  &  &  &  & \textit{\hspace*{0.2cm} Gender} & 0.413 & 0.521 &  &  &  \\
% \textit{\hspace*{0.2cm} Learning tool} & 0.207 & 0.813 &  &  & \textit{\hspace*{0.2cm} Gender composition} & 0.36 & 0.696 &  &  &  \\
% \textit{\hspace*{0.2cm} Gender} & 1.564 & 0.212 &  &  & Willingness to Recommend &  &  &  &  &  \\
% \textit{\hspace*{0.2cm} Gender composition} & 3.91 & 0.021 & * &  & \textit{\hspace*{0.2cm} Learning tool} & 4.596 & 0.011 & * & 0.038 &  \\
% Learning Perception &  &  &  &  & \textit{\hspace*{0.2cm} Gender} & 4.323 & 0.039 & * & 0.018 &  \\
% \textit{\hspace*{0.2cm} Learning tool} & 0.009 & 0.991 &  &  & \textit{\hspace*{0.2cm} Gender composition} & 1.19 & 0.306 &  &  &  \\
% \textit{\hspace*{0.2cm} Gender} & 6.245 & 0.013 & * & 0.026 & Learning Motivation &  &  &  &  &  \\
% \textit{\hspace*{0.2cm} Gender composition} & 3.63 & 0.028 & * &  & \textit{\hspace*{0.2cm} Learning tool} & 7.441 & 0.001 & * & 0.06 &  \\
% Enjoyment &  &  &  &  & \textit{\hspace*{0.2cm} Gender} & 4.907 & 0.028 & * & 0.021 &  \\
% \textit{\hspace*{0.2cm} Learning tool} & 9.87 & \textless 0.001 & * & 0.078 & \textit{\hspace*{0.2cm} Gender composition} & 2.95 & 0.055 & * &  &  \\
% \textit{\hspace*{0.2cm} Gender} & 10.008 & 0.002 & * & 0.041 &  & \multicolumn{1}{l}{} & \multicolumn{1}{l}{} & \multicolumn{1}{l}{} & \multicolumn{1}{l}{} &  \\
% \textit{\hspace*{0.2cm} Gender composition} & 4.03 & 0.019 & * &  &  & \multicolumn{1}{l}{} & \multicolumn{1}{l}{} & \multicolumn{1}{l}{} & \multicolumn{1}{l}{} &  \\ \hline
% \end{tabular}%
% }
% \end{table*}


% \begin{table*}[]
% \caption{}
% \label{tab:my-table4}
% \resizebox{\textwidth}{!}{%
% \begin{tabular}{lcccclccccl}
% \hline
% \multicolumn{1}{c}{\textbf{Variable}} & \textbf{F} & \textbf{p} & \textbf{Sig} & \textbf{$\eta_{p}^{2}$} & \multicolumn{1}{c}{\textbf{Variable}} & \textbf{F} & \textbf{p} & \textbf{Sig} & \textbf{$\eta_{p}^{2}$} &  \\ \hline
% \textbf{Learning Experience} & \multicolumn{1}{l}{} & \multicolumn{1}{l}{} & \multicolumn{1}{l}{} & \multicolumn{1}{l}{} & \hspace*{0.2cm} Effort to Focus &  &  &  &  &  \\
% \hspace*{0.2cm} Easy to Paint &  &  &  &  & \textit{\hspace*{0.4cm} Learning tool} & 1.679 & 0.189 &  &  &  \\
% \textit{\hspace*{0.4cm} Learning tool} & 7.786 & 0.001 & * & 0.063 & \textit{\hspace*{0.4cm} Gender} & 0.598 & 0.44 &  &  &  \\
% \textit{\hspace*{0.4cm} Gender} & 1.854 & 0.175 &  &  & \textit{\hspace*{0.4cm} Gender composition} & 1.84 & 0.161 &  &  &  \\
% \hspace*{0.2cm} Easy to Find Muscles &  &  &  &  & \textit{\hspace*{0.2cm} Lost Track of Time} &  &  &  &  &  \\
% \textit{\hspace*{0.4cm} Gender} & 7.065 & 0.008 & * & 0.029 & \textit{\hspace*{0.4cm} Learning tool} & 0.332 & 0.718 &  &  &  \\
% \textit{\hspace*{0.4cm} Gender composition} & 5.62 & 0.004 & * &  & \textit{\hspace*{0.4cm} Gender} & 0.413 & 0.521 &  &  &  \\
% \hspace*{0.2cm} Satisfaction &  &  &  &  & \textit{\hspace*{0.4cm} Gender composition} & 0.36 & 0.696 &  &  &  \\
% \textit{\hspace*{0.4cm} Learning tool} & 0.207 & 0.813 &  &  & \textit{\hspace*{0.2cm} Willingness to Recommend} &  &  &  &  &  \\
% \textit{\hspace*{0.2cm} Gender} & 1.564 & 0.212 &  &  & \textit{\hspace*{0.2cm} Learning tool} & 4.596 & 0.011 & * & 0.038 &  \\
% \textit{\hspace*{0.4cm} Gender composition} & 3.91 & 0.021 & * &  & \textit{\hspace*{0.4cm} Gender} & 4.323 & 0.039 & * & 0.018 &  \\
% \hspace*{0.2cm} Learning Perception &  &  &  &  & \textit{\hspace*{0.4cm} Gender composition} & 1.19 & 0.306 &  &  &  \\
% \textit{\hspace*{0.4cm} Learning tool} & 0.009 & 0.991 &  &  & \textit{\hspace*{0.2cm} Learning Motivation} &  &  &  &  &  \\
% \textit{\hspace*{0.4cm} Gender} & 6.245 & 0.013 & * & 0.026 & \textit{\hspace*{0.4cm} Learning tool} & 7.441 & 0.001 & * & 0.06 &  \\
% \textit{\hspace*{0.4cm} Gender composition} & 3.63 & 0.028 & * &  & \textit{\hspace*{0.4cm} Gender} & 4.907 & 0.028 & * & 0.021 &  \\
% \hspace*{0.2cm} Enjoyment &  &  &  &  & \textit{\hspace*{0.4cm} Gender composition} & 2.95 & 0.055 & * &  &  \\
% \textit{\hspace*{0.4cm} Learning tool} & 9.87 & \textless 0.001 & * & 0.078 & \textbf{Learning Performance} &  &  &  &  &  \\
% \textit{\hspace*{0.4cm} Gender} & 10.008 & 0.002 & * & 0.041 & \textit{\hspace*{0.4cm} Learning tool} & ??? & ??? & \multicolumn{1}{l}{} & \multicolumn{1}{l}{} &  \\
% \textit{\hspace*{0.4cm} Gender composition} & 4.03 & 0.019 & * &  & \textit{\hspace*{0.4cm} Gender} & ??? & ??? & \multicolumn{1}{l}{} & \multicolumn{1}{l}{} &  \\
%  & \multicolumn{1}{l}{} & \multicolumn{1}{l}{} & \multicolumn{1}{l}{} & \multicolumn{1}{l}{} & \textit{\hspace*{0.4cm} Gender composition} & 4.74 & 0.0096 & * & \multicolumn{1}{l}{} &  \\ \hline
% \end{tabular}%
% }
% \end{table*}

% \subsection{User Engagement and Satisfaction}
% \subsection{User Experience}
\subsection{Learning Experience}
\label{Sec:UX_Results}

\noindent Here, we report the results of our analysis on the measures for both learning tool, participant gender, and group gender composition factors.
The detailed results for each measure are described below, and the overviews of the learning tool effects and the gender effects are summarized in Table~\ref{tab:my-table2}.%Figure~\ref{fig:learning_tool_effects} and Figure~\ref{fig:gender_effects}.

%\begin{itemize}
\unnumberedparagraph{\textbf{Easy to Paint}} The one-way ANOVA for the measure of ``easy to paint'' showed a significant effect of the learning tool ($F(2, 233)\!=\!7.786$, \textbf{\itshape{p}\,$=$\,0.0005}; $\eta_{\text{p}}^{2}\!=\!0.063$ - medium to large effect size). % _{\text{p}}^{2}\note{effect size}).
The post-hoc tests revealed that Tablet-3D ($M\!=\!4.67$, $SD\!=\!0.656$) had a higher score than Textbook ($M\!=\!4.26$, $SD\!=\!0.872$; \textbf{\itshape{p}\,$=$\,0.009}) or Screen-AR ($M\!=\!4.18$, $SD\!=\!1.016$; \textbf{\itshape{p}\,$=$\,0.001}).
This suggests that the participants felt the Tablet-3D condition was easier to perform the muscle painting activity than the other two conditions.
The analysis for the participant gender did not show any significant effect. %($F(1, 234)\!=\!1.854$, $p\!=\!0.175$).


    
\unnumberedparagraph{\textbf{Easy to Find Muscles}} For the ``easy to find muscles'' we did not find any significant effect of the learning tool.  
%nor the group gender composition ($F(2, 233)\!=\!2.81$, $p\!=\!0.062$).
% the p-values were close to the significance level. We took a further step and looked into mixed pairs vs. same-gender pairs. 
We found that this measure was significantly different among three group gender compositions ($F(2, 233)\!=\!5.62$, \textbf{\itshape{p}\,$=$\,0.004}; male pairs: $M\!=\!4.645$, $SD\!=\!0.630$; female pairs: $M\!=\!4.398$, $SD\!=\!0.875$; mixed pairs: $M\!=\!4.136$, $SD\!=\!1.006$).
Based on the post-hoc comparison with the Bonferroni test, the group with same gender compositions (males pairs and females pairs) were easier to perform the muscle painting activity than the mixed pairs (\textbf{\itshape{p}\,$=$\,0.005}).
Moreover, we found a significant effect of the participant's gender on this measure, too ($F(1, 234)\!=\!7.065$, \textbf{\itshape{p}\,$=$\,0.008}; $\eta_{\text{p}}^{2}\!=\!0.029$ - small to medium effect size).
The result showed that the male participants ($M\!=\!4.55$, $SD\!=\!0.632$) reported a significantly higher score for the easiness of finding muscles than the female participants ($M\!=\!4.26$, $SD\!=\!0.937$).


\begin{table}[tb]
\caption{The summary of Learning Experience inferential results. (* p\,$<$\,0.05, ** p\,$<$\,0.01, *** p\,$<$\,0.001.)}
\label{tab:my-table2}
\resizebox{\columnwidth}{!}{%
\begin{tabular}{lcccc}
\toprule
\multicolumn{1}{c}{\textbf{Variable}} & \textbf{F} & \textbf{p} & \textbf{Sig} & \textbf{$\eta_{p}^{2}$} \\ \midrule
% Learning Experience &  &  &  &  \\
Easy to Paint &  &  &  &  \\
\hspace*{0.2cm} \textit{Learning tool} & 7.786 & 0.0005 & *** & 0.063 \\
\hspace*{0.2cm} \textit{Gender} & 1.854 & 0.175 &  &  \\
Easy to Find Muscles &  &  &  &  \\
\hspace*{0.2cm} \textit{Gender} & 7.065 & 0.008 & ** & 0.029 \\
\hspace*{0.2cm} \textit{Gender composition} & 5.62 & 0.004 & ** &  \\
Satisfaction &  &  &  &  \\
\hspace*{0.2cm} \textit{Learning tool} & 0.207 & 0.813 &  &  \\
\hspace*{0.2cm} \textit{Gender} & 1.564 & 0.212 &  &  \\
\hspace*{0.2cm} \textit{Gender composition} & 3.91 & 0.021 & * &  \\
Learning Perception &  &  &  &  \\
\hspace*{0.2cm} \textit{Learning tool} & 0.009 & 0.991 &  &  \\
\hspace*{0.2cm} \textit{Gender} & 6.245 & 0.013 & * & 0.026 \\
\hspace*{0.2cm} \textit{Gender composition} & 3.63 & 0.028 & * &  \\
Enjoyment &  &  &  &  \\
\hspace*{0.2cm} \textit{Learning tool} & 9.87 & 0.0001 & *** & 0.078 \\
\hspace*{0.2cm} \textit{Gender} & 10.008 & 0.002 & ** & 0.041 \\
\hspace*{0.2cm} \textit{Gender composition} & 4.03 & 0.019 & * &  \\
Effort to Focus &  &  &  &  \\
\hspace*{0.2cm} \textit{Learning tool} & 1.679 & 0.189 &  &  \\
\hspace*{0.2cm} \textit{Gender} & 0.598 & 0.44 &  &  \\
\hspace*{0.2cm} \textit{Gender composition} & 1.84 & 0.161 &  &  \\
Lost Track of Time &  &  &  &  \\
\hspace*{0.2cm} \textit{Learning tool} & 0.332 & 0.718 &  &  \\
\hspace*{0.2cm} \textit{Gender} & 0.413 & 0.521 &  &  \\
\hspace*{0.2cm} \textit{Gender composition} & 0.36 & 0.696 &  &  \\
Willingness to Recommend &  &  &  &  \\
\hspace*{0.2cm} \textit{Learning tool} & 4.596 & 0.011 & * & 0.038 \\
\hspace*{0.2cm} \textit{Gender} & 4.323 & 0.039 & * & 0.018 \\
\hspace*{0.2cm} \textit{Gender composition} & 1.19 & 0.306 &  &  \\
Learning Motivation &  &  &  &  \\
\hspace*{0.2cm} \textit{Learning tool} & 7.441 & 0.0007 & *** & 0.06 \\
\hspace*{0.2cm} \textit{Gender} & 4.907 & 0.028 & * & 0.021 \\
\hspace*{0.2cm} \textit{Gender composition} & 2.95 & 0.055 &  &  \\ \bottomrule
\end{tabular}%
}
% \newline
% \raggedright
% %\hspace*{0.5cm}
% \textit{Asterisks denote
% significance levels:* p\,$<$\,0.05, ** p\,$<$\,0.01, *** p\,$<$\,0.001.}
\vspace{-2em}
\end{table}
    
\unnumberedparagraph{\textbf{Satisfaction}} We did not find any significant effects of both the learning tool and the participant gender. %for the measure of ``satisfaction''. %(learning tool: $F(2, 233)\!=\!0.207$, $p\!=\!0.813$; gender: ($F(1, 234)\!=\!1.564$, $p\!=\!0.212$).
%\note{[ZG: STATA 15 17]}
However, this ``satisfaction'' measure was significantly different in gender compositions ($F(2, 233)\!=\!3.91$, \textbf{\itshape{p}\,$=$\,0.021}). 
Among three different gender compositions (male pairs: $M\!=\!4.484$, $SD\!=\!0.671$; female pairs: $M\!=\!4.259$, $SD\!=\!0.921$; mixed pairs: $M\!=\!4.045$, $SD\!=\!0.999$), Bonferroni test showed that male pairs had significantly higher satisfaction than mixed pairs (\textbf{\itshape{p}\,$=$\,0.017}).%, and no significant difference was observed based on the gender composition of the groups between male pairs and female pairs, %($p\!=\!0.339$)
%nor between female pairs and mixed pairs. %($p\!=\!0.372$).
    
\unnumberedparagraph{\textbf{Learning Perception}} For this %``learning perception'' 
measure, we did not find a significant effect of the learning tool, %($F(2, 233)\!=\!0.009$, $p\!=\!0.991$), 
but there was a main effect of the participant gender with a statistical significance ($F(1, 234)\!=\!6.245$, \textbf{\itshape{p}\,$=$\,0.013}; $\eta_{\text{p}}^{2}\!=\!0.026$ - small to medium effect size).
    This showed again that the male participants ($M\!=\!4.08$, $SD\!=\!0.834$) had a higher score than the female participants ($M\!=\!3.76$, $SD\!=\!1.062$).
%\note{[ZG: STATA 23]}
Also, a significant effect for the ``learning perception'' measure was observed ($F(2, 233)\!=\!3.63$, \textbf{\itshape{p}\,$=$\,0.028}) among gender compositions of male pairs ($M\!=\!4.177$, $SD\!=\!0.758$), female pairs ($M\!=\!3.778$, $SD\!=\!1.026$), and mixed pairs ($M\!=\!3.803$, $SD\!=\!1.084$).
Bonferroni test indicated that male pairs had a significantly higher score than female pairs (\textbf{\itshape{p}\,$=$\,0.034}).

    
\unnumberedparagraph{\textbf{Enjoyment}} For the ``enjoyment'' we found main effects of the learning tool, the participant gender, and group gender composition factors.
For the learning tool ($F(2, 233)\!=\!9.870$, \textbf{\itshape{p}\,$=$\,0.0001}; $\eta_{\text{p}}^{2}\!=\!0.078$ - medium to large effect size), we further compared the conditions and found that Textbook ($M\!=\!3.78$, $SD\!=\!1.051$) had a lower score than Tablet-3D ($M\!=\!4.32$, $SD\!=\!0.865$; \textbf{\itshape{p}\,$=$\,0.001}) and Screen-AR ($M\!=\!4.41$, $SD\!=\!0.897$; \textbf{\itshape{p}\,$=$\,0.001}).
For the gender effect ($F(1, 234)\!=\!10.008$, \textbf{\itshape{p}\,$=$\,0.002}; $\eta_{\text{p}}^{2}\!=\!0.041$ - small to medium effect size), the result showed that the male participants ($M\!=\!4.42$, $SD\!=\!0.807$) had a higher score than the female participants ($M\!=\!4.02$, $SD\!=\!1.038$).
%\note{[ZG: STATA 25]}
For the gender composition (male pairs: $M\!=\!4.468$, $SD\!=\!0.804$; female pairs: $M\!=\!4.037$, $SD\!=\!1.049$; mixed pairs: $M\!=\!4.152$, $SD\!=\!0.932$), a significant difference was observed ($F(2, 233)\!=\!4.03$, \textbf{\itshape{p}\,$=$\,0.019}).
The Bonferroni post-hoc test showed that the male pairs had a significantly higher score than the female pairs (\textbf{\itshape{p}\,$=$\,0.016}).  

    
\unnumberedparagraph{\textbf{Effort to Focus}} There were no significant effects found in the ``effort to focus'' measure.
% with interaction to learning tool and gender factors.
%(learning tool: $F(2, 233)\!=\!1.679$, $p\!=\!0.189$; gender: $F(1, 234)\!=\!0.598$, $p\!=\!0.440$; and 
%\note{[ZG: STATA 26]}
%gender composition: $F(2, 233)\!=\!1.84$, $p\!=\!0.161$).

    
\unnumberedparagraph{\textbf{Lost Track of Time}} For learning tool and gender-related factors, there were also no significant effects for this measure.% found in the ``lost track of time'' measure (learning tool: $F(2, 233)\!=\!0.332$, $p\!=\!0.718$; gender: $F(1, 234)\!=\!0.413$, $p\!=\!0.521$; %\note{[ZG: STATA 27]}
%gender composition: $F(2, 233)\!=\!0.36$, $p\!=\!0.696$).

    
\unnumberedparagraph{\textbf{Willingness to Recommend}} We found a main effect of the learning tool for the ``willingness to recommend'' ($F(2, 233)\!=\!4.596$, \textbf{\itshape{p}\,$=$\,0.011}; $\eta_{\text{p}}^{2}\!=\!0.038$ - small to medium effect size).
The post-hoc tests revealed that Textbook ($M\!=\!6.96$, $SD\!=\!2.185$) had a lower score than Tablet-3D ($M\!=\!7.82$, $SD\!=\!2.003$; \textbf{\itshape{p}\,$=$\,0.024}) and Screen-AR ($M\!=\!7.83$, $SD\!=\!1.865$; \textbf{\itshape{p}\,$=$\,0.028}).
The gender effect was also found for this ($F(1, 234)\!=\!4.323$, \textbf{\itshape{p}\,$=$\,0.039}; $\eta_{\text{p}}^{2}\!=\!0.018$ - small to medium effect size), which showed that the male participants ($M\!=\!7.89$, $SD\!=\!1.825$) had a higher score than the female participants ($M\!=\!7.33$, $SD\!=\!2.164$).
%\note{[ZG: STATA 28]}
%On the contrary, no gender composition effect was found ($F(2, 233)\!=\!0.12$, $p\!=\!0.889$
%).
%\note{[ZG: STATA 29]}
On the contrary, no group gender composition effect was identified.% ($F(2, 233)\!=\!1.19$, $p\!=\!0.306$).

    
\unnumberedparagraph{\textbf{Learning Motivation}} %For the ``learning motivation'' measure, 
We found the main effects of the learning tool and participant gender.
The effect of the learning tool was significant ($F(2, 233)\!=\!7.441$, \textbf{\itshape{p}\,$=$\,0.0007}; $\eta_{\text{p}}^{2}\!=\!0.060$ - medium effect size), and the post-hoc tests showed that Textbook ($M\!=\!3.57$, $SD\!=\!1.032$) had a lower score than Tablet-3D ($M\!=\!3.94$, $SD\!=\!0.987$; \textbf{\itshape{p}\,$=$\,0.044}) and Screen-AR ($M\!=\!4.17$, $SD\!=\!0.839$; \textbf{\itshape{p}\,$=$\,0.001}).
The gender effect was also significant ($F(1, 234)\!=\!4.907$, \textbf{\itshape{p}\,$=$\,0.028}; $\eta_{\text{p}}^{2}\!=\!0.021$ - small to medium effect size), showing that the male participants ($M\!=\!4.07$, $SD\!=\!0.902$) had a higher score than the female participants ($M\!=\!3.79$, $SD\!=\!1.020$).
%\note{[ZG: STATA 30]}
%For the gender composition (male pairs: $M\!=\!4.145$, $SD\!=\!0.884$; female pairs: $M\!=\!3.769$, $SD\!=\!1.029$; mixed pairs: $M\!=\!3.894$, $SD\!=\!0.963$), a strong tendency towards statistical significance was observed ($F(2, 233)\!=\!2.95$, $p\!=\!0.055$).
Based on Bonferroni test, we further compared the pairs and found that the male pairs had a significantly higher score than the female pairs for ``learning motivation'' (\textbf{\itshape{p}\,$=$\,0.048}). 
    
%\end{itemize}



% % Figure environment removed
% \vspace{-2ex}
% % Figure environment removed

%\vspace{1ex}

%\subsection{Learning Performance}
\subsection{Learning Performance}

\label{Sec:Performance_Results}
% Anatomy Knowledge Retention}

%To guarantee the valid evaluation of knowledge retention, e.g., the post-score or the score gained, we first analyzed if there were any significant differences in the pre-score measure among the learning tool conditions or among different gender groups%---in other words, we evaluated if both learning tool groups, gender groups, and group gender compositions were well-balanced with respect to the prior knowledge of anatomy.
To ensure valid knowledge retention evaluation, we analyzed pre-score differences among learning tool conditions and gender groups. This involved assessing the balance in prior anatomy knowledge within learning tool groups and gender compositions. We did not find any differences in the pre-score measure among the learning tool conditions. However, we found a significant difference between the male and female participants (\textbf{\itshape{p}\,$=$\,0.019}), which showed the male participants ($M\!=\!2.05$, $SD\!=\!1.283$) had a higher pre-score than the female participants ($M\!=\!1.69$, $SD\!=\!1.070$).
%\note{[ZG: STATA 38]}
%A significant difference was found among various group gender compositions, too ($F(2, 233)\!=\!4.74$, \textbf{\itshape{p}\,$=$\,0.0096}).
While controlling the pre-score as a covariate, we conducted one-way ANCOVA to investigate the effects of the learning tool, the participant's gender, or the group gender composition; however, no significant results were found.% there were no significant effects on both measures of the post-score and the score gain. 
% the post-score ($M\!=\!1.92$, $SD\!=\!1.401$) and the score gain ($M\!=\!0.08$, $SD\!=\!1.459$). 
Among different learning tool conditions, compared with the score gain ($M\!=\!0.069$, $SD\!=\!1.437$) and the post-score ($M\!=\!1.92$, $SD\!=\!1.441$) in the Textbook condition, participants in the Tablet-3D condition achieved the highest score gain ($M\!=\!0.10$, $SD\!=\!1.447$) with post-score ($M\!=\!1.82$, $SD\!=\!1.282$); Screen-AR groups achieved the second highest score gain ($M\!=\!0.079$, $SD\!=\!1.512$) with post-score ($M\!=\!2.04$, $SD\!=\!1.501$). 
With respect to the participant's gender, females achieved a higher score gain ($M\!=\!0.20$, $SD\!=\!1.455$) with post-score ($M\!=\!1.89$, $SD\!=\!1.342$) than males' score gain ($M\!=\!-0.084$, $SD\!=\!1.456$) with post-score ($M\!=\!1.97$, $SD\!=\!1.491$).
%\note{[ZG: STATA 39 40 seem not good to report]}
Among three group gender compositions, compared with the score gain ($M\!=\!-0.145$, $SD\!=\!1.401$) and the post-score ($M\!=\!1.887$, $SD\!=\!1.307$) of the male pairs, female groups achieved the highest score gain ($M\!=\!0.269$, $SD\!=\!1.464$) with post-score ($M\!=\!1.852$, $SD\!=\!1.288$); mixed pairs achieved the second highest score gain ($M\!=\!0$, $SD\!=\!1.488$) with post-score ($M\!=\!2.06$, $SD\!=\!1.654$). 
% Two-way ANOVAs with the learning tool, the gender, and the group gender composition factors also did not show any main or interaction effects related to the post-score and the score gain.
There are no main or interaction effects on the post-score and the score gain among the three factors: learning tool, gender, and group gender composition.

\section{Discussion}

\label{Sec:Discussion}

% \note{note after submission: we might need/want to discuss more about the meanings and reasons behind the results for individual measures.}

\noindent In this section, we summarize our findings based on the results while connecting them to our established hypotheses. We also discuss the justifications and implications of our findings and associate them with previous research.
\unnumberedparagraph{\textbf{Digital Tools Improved Learning Experience}} We found several significant effects of the learning tool for some of our measures, which we reported in Section~\ref{Sec:UX_Results}.
In general, the results show that the Tablet-3D and the Screen-AR could provide more positive experience scores than the Textbook, e.g., for the measures of ``enjoyment,'' ``willingness to recommend,'' and ``learning motivation,'' which partly supports our \textbf{H1}.
These results align with previous literature findings that showed 3D visualization technologies increased students' engagement in anatomy learning\cite{hackett_effect_2018,luursema_role_2008}.
% % {(Hackett \& Proctor, 2018; Luursema et al., 2008; Yammine \& Violato, 2015)}.
The positive scores in some of our  measures could be related to the novelty of the VR/AR technology as an anatomy learning tool, and the intuitive and interactive 3D models could also be an important factor in influencing the perception of the learning experience.
Some qualitative comments from the participants in the Tablet-3D condition or the Screen-AR condition post-session also support our reasoning for the positive scores. For example, some of the participants in Tablet-3D or Screen-AR said:
% were also aligned with this positive trends in the learning experience measures.
% \vspace{.1em}
\begin{quote}
\leftskip=-10pt
\rightskip=-10pt
% \\
% \noindent
\textit{``Interface could be refined a little; the model was great.''}\\
\textit{``It was an enjoyable and educational experience.''}\\
\textit{``Super cool to be the model and see my body's muscles.''}\\
\textit{``It was fun to paint a muscle and helped put it into the context of reality instead of simply seeing it on a page.''}
\end{quote}
% enjoyed the muscle painting activity 
% enjoyment, willingness to recommend, and learning motivation

% easy to paint: tablet > textbook, screen-AR
\noindent Interestingly, however, we found no significant benefits of the Screen-AR compared to the Tablet-3D, which we expected in our \textbf{H2}.
Instead, we found a higher score in the Tablet-3D condition than the Screen-AR condition for the ``easy to paint'' measure.
Based on the participants' feedback below, we realized that some participants complained about the intermittent misalignment of virtual contents in the Screen-AR condition.

\begin{quote}
\leftskip=-10pt
\rightskip=-10pt
% \noindent
% ``The calibration for the magic mirror was not great and so it was hard to actually see the muscle projected onto myself and my partner. The diagram on the right of the mirror was much more helpful and it's what allowed us to actually complete the activity.''
% ``The program kept not focusing on the right subject, and sometimes it was hard to see which muscle was being pointed at.''
% ``The model doesn't line up very well with our actual bodies.''
% ``It was a little hard to distinguish where in the muscle was sometimes because the overlay wasn't actually exact/not proportional to my body.''
\textit{``The calibration was not great and it was hard to actually see the muscle projected onto myself and my partner.''}\\
\textit{``Sometimes it was hard to see which muscle was being pointed.''}\\
\textit{``It was a little hard to distinguish where in the muscle was sometimes because the overlay wasn't actually exact/not proportional to my body.''}
\end{quote}

\noindent This implies that the participants were quite sensitive to the accuracy and reliability of body tracking for AR content registration, which could be crucial for the learning experience.
Some major contributors to this AR accuracy issue include the 3D virtual models, which followed a male anatomy limb proportions rather than female, and the physical proximity of the painter and paintee in front of the body tracking system.  

%%%%%%%%%%%%%%%%%%%%%%%%%%%%%%%%%%%%%%%%%%%%%%%%%%%%%%%%%%%%%%%%%%%%%% 
% \note{need for further long-term research to reveal the effects on the knowledge retention}

\unnumberedparagraph{\textbf{No Influence on Learning Performance}} Positive effects of our digital tools on learning performance based on prior research was expected~\cite{maresky_virtual_2019,nicholson_can_2006, barmaki_enhancement_2019}.
As we reported in Section\ref{Sec:Performance_Results}, there were no significant effects of the learning tool or the participant's gender for the learning performance, e.g., the anatomy knowledge retention, which means no evidence to support our \textbf{H4} and \textbf{H5}.
However, prior research showed that positive experience could increase learning performance and retention~\cite{maresky_virtual_2019,nicholson_can_2006}, and our proposed 3D visualization and AR learning tools did promote more positive compared to the traditional textbook-based learning.
In that sense, we are inspired to conduct further research on different learning analytic metrics.
Additionally, the intervention that the participants had in our study was only a one-time session for about 10--15 minutes, which could not be enough to reveal the positive effects of AR on learning performance.
As most learning science studies require longitudinal studies for their efficacy, we will propose a longitudinal study in future.


%%%%%%%%%%%%%%%%%%%%%%%%%%%%%%%%%%%%%%%%%%%%%%%%%%%%%%%%%%%%%%%%%%%%%% 


\unnumberedparagraph{\textbf{Male Users Noted More Positive Experience}} Beyond investigating the learning tool effects, we also established an interesting question about gender effects in anatomy education, as introduced in \textbf{H3} in Section~\ref{Sec:MeasuresHypotheses}.
We found significant effects of the participant gender on various measures: ``easy to find muscles,'' ``learning perception,'' ``enjoyment,'' ``willingness to recommend,'' and ``learning motivation.''
For all those measures, male participants reported higher (more positive) scores than female participants, which partly supports our \textbf{H3}.
 Considering gender composition, similar patterns were observed on the ``easy to find muscles,'' ``learning perception,'' and ``enjoyment'' measures. 
 When comparing gender compositions, we found statistically significant differences between male pairs and mixed pairs, as well as between male pairs and female pairs. Male pairs reported the highest learning experience scores. These finding also provides partial support for \textbf{H3}. We should note that this effect was not associated with a specific \textit{learning tool}.
% , but was found as general results including all types of learning tools.
There could be some aspects of our learning intervention that female participants did not like compared to male participants, which seemed to be the \textit{painting part}, not the \textit{learning tool}. Female participants in our study might be more conservative about the body painting activity than males. 
According to prior research~\cite{cookson2018exploration,finn2010qualitative}, unequal engagement of students in body painting activities may be due to cultural, social, and religious barriers concerning body image, nudity, gender, vulnerability, and embarrassment.
Not surprisingly, such tendencies were reflected in the after-session feedback from some female participants.

\begin{quote}
\leftskip=-10pt
\rightskip=-10pt
% ``(you should) take into account clothes and girls''
% ``I am not a fan of body painting. However, I love the program and think that it is really cool.''
 % \noindent 
\textit{``Painting was unnecessary, but it was cool to see the muscles on my body.''}
% ``I really don't like the painting i dont like having paint on my skin that has touched like 100 other kids before me i would do it if i had clean paint brushes and my own paint but this is just an issue for me sorry i love the computer program though it is so cool, just not the paint.''
% \noindent
\textit{``[The activity may] take into account clothes and girls.''}\\
\textit{``I am not a fan of body painting. However, I love the program and think that it is really cool.'''}\\
\textit{``I don't like having paint on my skin.''}
\end{quote}

\noindent These observations emphasize the importance of gender perspectives in the design. It also recommends exploring a new learning module, one that doesn't involve body painting.
%Novel learning tools could/should also be designed to mitigate the reluctance to the intervention with offering a variety of virtual content or interaction mechanisms.


% \begin{itemize}
% \item \textbf{H3}: The male participants in the study will report more positive user experience than the female participants, particularly in the measures of satisfaction, enjoyment, willingness to recommend. (Female $<$ Male)
% \end{itemize}

\unnumberedparagraph{\textbf{Future Work and Limitations}}
The overarching goal of our study was to explore and understand the possible effects of using advanced learning technologies on anatomy learning and student performance. Identifying anatomical parts in a real body, as in our current study, is primarily acknowledged in clinical anatomy. While not immediately applicable to sessions involving cadavers or anatomy models, exploring this avenue, such as using a Kinect sensor positioned above a patient bed, presents an interesting future path. Our discussion unveils that we require multi-session, long-term, and modified studies to draw more conclusive findings for knowledge retention and task completion time. As noted, some students liked the anatomy learning tools but disliked the body painting activity, so the selection of body painting as baseline activity may have been a confounding factor. In addition, this paper covered dyadic teams and we plan to analyze our data with participants in larger teams while controlling for demographics of age, gender, and prior knowledge level to form more uniform teams. We also plan to expand student learning analysis by collecting and using multi-modal measurements for more effective and objective \textit{learning tool} evaluations. 


\section{Conclusion}
\label{Sec:Conclusions}

\noindent This paper investigated the effects of digital anatomy learning tools, using our in-house tablet-based 3D visualization system and the screen-based AR application overlaying 3D anatomy structures onto the digital mirror image of students. 
We conducted a large-scale study with \textbf{236} premedical students, and compared our digital learning tools with conventional textbook.
Our results indicated that both digital learning tools could improve the learning experience, particularly in ``enjoyment'', ``willingness to recommend'', and ``learning motivation''. 
We also found that male participants generally reported more positive experiences than females.
Our findings include further considerations of learners' gender and performance in body painting anatomy education.
% Since the study with our digital learning tools shows enhanced, progressive short-term learning performance and enjoyment, we are optimistic that the long-term retention of future novel 3D technology will follow a similar trend.  
Based on the enhanced, progressive short-term learning performance and enjoyment in our study, we expect a similar trend in the long-term retention of future 3D technology.


%Although we did not find any significant benefits of our digital learning tools to increase anatomical knowledge retention as a learning performance measure, our findings offer implications in anatomy education, concerning learning technologies and learners' profiles, specifically gender, by providing novel instructional tools for teaching and learning anatomy.

% \note{future research analyzing the group size effects, etc.}

\section*{Acknowledgment}
\noindent We wish to express our gratitude to the study participants and laboratory assistants. We wish to acknowledge the support from the support of the Johns Hopkins University Science of
Learning Institute, and National Science Foundation for award \#2321274. 
%and National Institute of General Medical Sciences of the National Institutes of Health ($P20 GM103446-E$). 
Any opinions, findings, and conclusions expressed in this material are those of the authors.
% and do not necessarily reflect the view of sponsors.
%The preferred spelling of the word ``acknowledgment'' in America is without an ``e'' after the ``g''. Avoid the stilted expression ``one of us (R. B. G.) thanks $\ldots$''. Instead, try ``R. B. G. thanks$\ldots$''. Put sponsor acknowledgments in the unnumbered footnote on the first page.

%\newpage
\balance
\bibliographystyle{IEEEtran}
\bibliography{IEEEabrv,bibfile}

\end{document}
