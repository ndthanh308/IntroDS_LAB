%%
%% This is file `sample-manuscript.tex',
%% generated with the docstrip utility.
%%
%% The original source files were:
%%
%% samples.dtx  (with options: `manuscript')
%% 
%% IMPORTANT NOTICE:
%% 
%% For the copyright see the source file.
%% 
%% Any modified versions of this file must be renamed
%% with new filenames distinct from sample-manuscript.tex.
%% 
%% For distribution of the original source see the terms
%% for copying and modification in the file samples.dtx.
%% 
%% This generated file may be distributed as long as the
%% original source files, as listed above, are part of the
%% same distribution. (The sources need not necessarily be
%% in the same archive or directory.)
%%
%%
%% Commands for TeXCount
%TC:macro \cite [option:text,text]
%TC:macro \citep [option:text,text]
%TC:macro \citet [option:text,text]
%TC:envir table 0 1
%TC:envir table* 0 1
%TC:envir tabular [ignore] word
%TC:envir displaymath 0 word
%TC:envir math 0 word
%TC:envir comment 0 0
%%
%%
%% The first command in your LaTeX source must be the \documentclass command.


\documentclass[sigconf]{acmart}

%%
%% \BibTeX command to typeset BibTeX logo in the docs

\AtBeginDocument{%
  \providecommand\BibTeX{{%
    Bib\TeX}}}
    
% \AtBeginDocument{%
%   \providecommand\BibTeX{{%
%     \normalfont B\kern-0.5em{\scshape i\kern-0.25em b}\kern-0.8em\TeX}}}

%% Rights management information.  This information is sent to you
%% when you complete the rights form.  These commands have SAMPLE
%% values in them; it is your responsibility as an author to replace
%% the commands and values with those provided to you when you
%% complete the rights form.
\setcopyright{acmcopyright}
\copyrightyear{2023}
\acmYear{2023}
\acmDOI{XXXXXXX.XXXXXXX}

%% These commands are for a PROCEEDINGS abstract or paper.
\acmConference[SUI'23]{SUI '23: ACM Symposium on Spatial User Interaction}{October 13--15, 2023}{Sydney, Australia}
\acmPrice{15.00}
\acmISBN{978-1-4503-XXXX-X/18/06}


\usepackage{subfigure}
\usepackage{booktabs}

\newcommand{\note}[1]{{\color{red} NOTE: #1}}
\newcommand{\expMeasure}{learning experience }
%\newcommand{\expMeasure }{learning and usability experience } %user experience?

%%
%% Submission ID.
%% Use this when submitting an article to a sponsored event. You'll
%% receive a unique submission ID from the organizers
%% of the event, and this ID should be used as the parameter to this command.
\acmSubmissionID{2047}

%%
%% The majority of ACM publications use numbered citations and
%% references.  The command \citestyle{authoryear} switches to the
%% "author year" style.
%%
%% If you are preparing content for an event
%% sponsored by ACM SIGGRAPH, you must use the "author year" style of
%% citations and references.
%% Uncommenting
%% the next command will enable that style.
%%\citestyle{acmauthoryear}

%%
%% end of the preamble, start of the body of the document source.
\begin{document}

%%
%% The "title" command has an optional parameter,
%% allowing the author to define a "short title" to be used in page headers.
% \title[Effects of VR/AR Learning Tools and Gender in Anatomy Education]{Effects of VR/AR Anatomy Education Tools and Learner's Gender on the Learning Experience and Performance: A Large-Scale Case Study}

% \title[Effects of VR/AR Learning Tools and Gender in Anatomy Education]%{Effects of VR/AR-based Learning Tools and Gender on the User Experience and Knowledge Retention in Anatomy Education: A Large-Scale  Study}

%\title[Effects of VR/AR Learning Tools in Anatomy Education]{A Large-Scale Anatomy Education Study for Investigating the Effects of VR/AR-based Learning Tools on User Experience and Knowledge Retention}

% \title[VR/AR Learning Tools for Anatomy Education]{A Large-Scale Feasibility Study of Screen-based Virtual and Augmented Reality Solutions for Human Anatomy Education}
\title[A Large-Scale Feasibility Study of Screen-based 3D Visualization and Augmented Reality Tools for Human Anatomy Education]{A Large-Scale Feasibility Study of Screen-based 3D Visualization and Augmented Reality Tools for Human Anatomy Education: Exploring Gender Perspectives in Learning Experience}

% VR/AR Learning Tools for Anatomy Education: Investigating the User Experience, Knowledge Retention and Gender Effects
% VR/AR Learning Tools for Anatomy Education: Investigating the User Experience, Knowledge Retention and Gender Effects in a Large-Scale User Study

%%
%% The "author" command and its associated commands are used to define
%% the authors and their affiliations.
%% Of note is the shared affiliation of the first two authors, and the
%% "authornote" and "authornotemark" commands
%% used to denote shared contribution to the research.
% \author{Ben Trovato}
% \authornote{Both authors contributed equally to this research.}
% \email{trovato@corporation.com}
% \orcid{1234-5678-9012}

% \author{G.K.M. Tobin}
% \authornotemark[1]
% \email{webmaster@marysville-ohio.com}
% \affiliation{%
%   \institution{Institute for Clarity in Documentation}
%   \streetaddress{P.O. Box 1212}
%   \city{Dublin}
%   \state{Ohio}
%   \country{USA}
%   \postcode{43017-6221}
% }

% \author{Lars Th{\o}rv{\"a}ld}
% \affiliation{%
%   \institution{The Th{\o}rv{\"a}ld Group}
%   \streetaddress{1 Th{\o}rv{\"a}ld Circle}
%   \city{Hekla}
%   \country{Iceland}}
% \email{larst@affiliation.org}

% \author{Valerie B\'eranger}
% \affiliation{%
%   \institution{Inria Paris-Rocquencourt}
%   \city{Rocquencourt}
%   \country{France}
% }

\author{Roghayeh Leila Barmaki}
\affiliation{%
  \institution{University of Delaware}
  \city{Newark}
  \state{DE}
  \country{U.S.A}}
\email{rlb@udel.edu}
\orcid{0000-0002-7570-5270}

\author{Kangsoo Kim}
\affiliation{
 \institution{University of Calgary}
  \city{Calgary}
  \state{AB}
  \country{Canada}
  }
\email{kangsoo.kim@ucalgary.ca}
\orcid{0000-0002-0925-378X}

\author{Zhang Guo}
\affiliation{%
  \institution{University of Delaware}
  \city{Newark}
  \state{DE}
  \country{U.S.A}}
\email{guozhang@udel.edu}


\author{Qile Wang}
\affiliation{%
  \institution{University of Delaware}
  \city{Newark}
  \state{DE}
  \country{U.S.A}}
\email{kylewang@udel.edu}
\orcid{0000-0003-0308-6033}

\author {Kevin Yu}
\affiliation{%
    \institution {Technical University of Munich}
    \city {Munich}
    \country {Germany}
}
\email{kevin.yu@tum.de}
    
\author {Rebecca Pearlman}
\affiliation{%
    \institution {Johns Hopkins University}
    \city {Baltimore}
    \state {MD},
    \country {U.S.A}
}
\email{pearlman@jhu.edu}

\author {Nassir Navab}
\affiliation{%
    \institution{Johns Hopkins University}
    \city {Baltimore}
    \state {MD}
    \country {U.S.A}
    }
\email{nassir.navab@jhu.edu}



%%
%% By default, the full list of authors will be used in the page
%% headers. Often, this list is too long, and will overlap
%% other information printed in the page headers. This command allows
%% the author to define a more concise list
%% of authors' names for this purpose.
\renewcommand{\shortauthors}{Barmaki et~al.}

%%
%% The abstract is a short summary of the work to be presented in the
%% article.
\begin{abstract}
% Page limit: 8000 words single column (final version: max. 9 pages double column + references)
% As an essential part of medical education, anatomy learning has been reshaped in recent decades. There are several challenges related to 
% traditional teaching methods, including the limited resources for performing dissection or pro-section in   large classes, or on the flip-side, the difficulty of students to understand the complexities of human body, and to perform proper mental mapping by just viewing two-dimensional anatomical structures in a textbook.
% In this paper, we investigate the use of modern technologies such as augmented and virtual reality as a complement or replacement of textbooks in an anatomy learning intervention.
% In a team-based study with 236 students in undergraduate premedical programs, students who used both of our in-house augmented reality(AR), and virtual reality (VR) interactive applications were as successful as those who used textbook; their familiar learning instrument; and no difference was observed between study conditions based on knowledge retention. 
% For user experience, students reflected that they enjoyed completing the learning activity using the AR and VR learning tools more \note{(F()=, p<0.05)}.
% We also investigated gender effect in our data and identified \note{significantly higher knowledge retention for females rather than males (F(1,299)=4.44, $p\!<\!0.05$).}
% The findings from this work may offer implications on modern learning technologies for anatomy and medical education.
%While anatomy learning is an essential part of medical education, there are significant challenges in traditional learning methods, e.g., related to the limited resources for performing dissection/pro-section in large classes, and the learning difficulty to understand the complexities of human body and to perform proper mental mapping based on two-dimensional anatomical structures in a textbook.
%With advanced technology-based learning tools, such as using virtual/augmented reality (VR/AR), anatomy learning has been reshaped in recent decades.
%In this paper, we developed in-house tablet-based VR application and screen-based AR system that could superimpose and visualize 3D virtual anatomy models with informative labels. 
%To investigate the effects of interactive VR/AR learning tools on \expMeasure and knowledge retention, a large-scale team-based study with 236 students (males: 95, females: 141) from undergraduate premedical programs were recruited.The participant students experienced an anatomy learning intervention, which involves muscle painting activities, using one of the learning tools: (1) conventional textbook, (2) tablet-based VR, and (3) screen-based AR.
%The results showed that students with our VR or AR tool reported higher (more positive) \expMeasure scores than those who used a textbook.We did not observe any difference on knowledge retention among the three learning tools, but our further analysis about gender effects revealed that male participants generally reported more positive \expMeasure scores than female participants.
%While discussing implications of our results in the context of the anatomy and medical education, we suggest that VR/AR learning tools could be a complement or replacement of textbooks in anatomy learning interventions.

% Anatomy is an essential part of medical education, but using new technology for teaching it still remains a huge challenge. In this paper, we introduce and evaluate two in-house screen-based virtual and augmented reality (VR/AR) anatomy training solutions that can visualize and superimpose 3D virtual anatomy models with informative labels on a display. We report the usability and learning experience findings from a large-scale study with 236 (19.77 years/pm 1.07, 141F) undergraduate premedical students. The participant students were split into three groups to use one of the following learning tools in a team-based anatomy painting activity: (1) conventional textbook, (2) tablet-based VR, and (3) screen-based AR. Students who used our VR or AR learning tools reported significantly higher (more positive) learning experience scores than textbook groups. We suggest that VR/AR learning tools have the potential to be considered as a complement or even replacement for textbooks in anatomy learning interventions.

While anatomy learning is an essential part of medical education, there remain significant challenges in traditional learning methods, e.g., related to the limited resources for performing dissection/pro-section in large classes, and the learning difficulty in understanding the complexities of the human body and performing proper mental mapping based on two-dimensional (2D) anatomical structures in a textbook.
Recently, advanced technologies, such as 3D visualization and augmented reality (AR), have reshaped anatomy learning.

In this paper, we introduce two in-house anatomy training solutions that can visualize and superimpose 3D virtual anatomy models with informative labels using a hand-held tablet or a wide-screen AR. 
To investigate the feasibility and effectiveness of the proposed tablet-based 3D visualization and AR tools, we conducted a large-scale study with \textbf{236} students enrolled in undergraduate premedical programs (95 males, 141 females in 118 dyadic teams).
In this study, participant students were split into three groups to use one of the following learning tools in a team-based anatomy painting activity: (1) conventional textbook, (2) hand-held tablet-based 3D visualization, and (3) screen-based AR.
The results showed that students who used the tablet-based visualization tool or the AR learning tool reported significantly higher (more positive) \expMeasure scores than those who used a textbook.
Though we did not observe a significant difference in knowledge retention among the three learning tools, our further analysis of gender effects revealed that male participants generally reported more positive \expMeasure scores than female participants. Also, the overall experience of mixed-gender dyads was reported to be significantly lower than others in most of the learning experience and performance measures. 
% While discussing the implications of our results in the context of anatomy and medical education, we suggest that tablet-based 3D visualizations and AR learning tools have the potential to be considered as a complement or even replacement for textbooks in anatomy learning interventions.
While discussing the implications of our results in the context of anatomy and medical education, we highlight the potential of our learning tools with additional considerations related to gender and team dynamics in body painting anatomy learning interventions.
\end{abstract}

%%
%% The code below is generated by the tool at http://dl.acm.org/ccs.cfm.
%% Please copy and paste the code instead of the example below.
%%
\begin{CCSXML}
<ccs2012>
<concept>
<concept_id>10003120.10003121.10003122</concept_id>
<concept_desc>Human-centered computing~HCI design and evaluation methods</concept_desc>
<concept_significance>500</concept_significance>
</concept>
<concept>
<concept_id>10010405.10010489.10010491</concept_id>
<concept_desc>Applied computing~Interactive learning environments</concept_desc>
<concept_significance>500</concept_significance>
</concept>
<concept>
<concept_id>10003120.10003121.10003124.10010866</concept_id>
<concept_desc>Human-centered computing~Virtual reality</concept_desc>
<concept_significance>500</concept_significance>
</concept>
<concept>
<concept_id>10003120.10003121.10003124.10010392</concept_id>
<concept_desc>Human-centered computing~Mixed / augmented reality</concept_desc>
<concept_significance>500</concept_significance>
</concept>
</ccs2012>
\end{CCSXML}

\ccsdesc[500]{Human-centered computing~HCI design and evaluation methods}
\ccsdesc[500]{Applied computing~Interactive learning environments}
% \ccsdesc[500]{Human-centered computing~Virtual reality}
\ccsdesc[500]{Human-centered computing~Mixed / augmented reality}


%%
%% Keywords. The author(s) should pick words that accurately describe
%% the work being presented. Separate the keywords with commas.
% \keywords{anatomy education, virtual reality, augmented reality, gender, learning experience, knowledge retention}

\begin{keywords} {Augmented and virtual reality, Cooperative/collaborative learning, Evaluation methodologies, Human-computer interface, Post-secondary education}

\end{keywords}


%%
%% This command processes the author and affiliation and title
%% information and builds the first part of the formatted document.
\maketitle

%!TEX root = ../main.tex

\section{Introduction}
\label{sec:intro}

Climate change and the decline of species richness are severe challenges that influence the living conditions of humans around the world.
Especially the dramatic loss of insects~\cite{hallmann2017more,wagner2021insect} plays a crucial role in many ecological processes that affect agriculture and others.
Hence, monitoring insect species populations becomes more important nowadays to better understand insect decline and long-term trends in species distributions.
Furthermore, there are about one million named species on our planet~\cite{stork2018many}, making manual counting of individuals unrealistic.
Consequently, automated monitoring of insects is inevitably required to infer abundance estimations across larger regions.
One possible way is to use camera traps to collect images of insects that computer vision algorithms can then process to recognize the depicted species automatically.

In this paper, we focus on nocturnal insects, mainly nocturnal moths (Lepidoptera).
Even for this subset, there exist hundred thousands of different species worldwide and depending on the habitat, species lists can be narrowed down based on the study region.
For example, image datasets containing hundreds of moth species from Ecuador and Costa Rica are publicly available and can directly be used for evaluating fine-grained recognition algorithms~\cite{Rodner15:FRD}.
Here, we are interested in monitoring moth species in Central Europe.
We present datasets of moth images we have collected so far and our analysis of algorithms for insect localization and species classification.

% Figure environment removed

Our work is part of a larger project called AMMOD\footnote{\scriptsize{AMMOD = \textbf{A}utomated \textbf{M}ultisensor Station for \textbf{M}onitoring \textbf{o}f Bio\textbf{d}iversity (\url{https://ammod.de/})}}, which aims at developing self-sustaining multi-sensor stations for monitoring species diversity~\cite{Waegele22:TAM}.
One component of these stations is a light-based camera trap for nocturnal insects, called the \emph{moth~scanner}~\cite{Radig2021:AVL,Korsch21_DLP}.
It is a non-invasive monitoring system for automatically gathering images at nighttime.
A UV-LED lamp illuminates a white planar surface to attract the insects that land on this surface.
A high-resolution camera takes an image of the whole surface every two minutes.
Our prototype is shown in Figure~\ref{fig:prototype}.

With this setup, we can collect large-scale datasets of nocturnal insects over a long period that can then be used to develop and evaluate appropriate fine-grained species recognition algorithms.
The moth scanner takes several hundred images during one night, and within five months, we collected more than \num{27000} images with our prototype.
In this paper, we refer to the resulting dataset as the \emph{nocturnal insects dataset~(NID)}, and more details are given in Section~\ref{sec:dataset}.
Note that this dataset is supposed to be extended over time as our system will be in operation within the following years.
We plan to maintain multiple sensor stations in parallel at different locations.
Hence, it has the potential to become a valuable source for large-scale learning and continuous learning within a fine-grained domain.

Besides its impact on research in fine-grained recognition, our developments for automated visual monitoring of nocturnal insects are beneficial for ecologists.
Until now, insect monitoring is mainly done by hand and supported by citizen scientists who manually take images of individual insects in their gardens. 
Previously, we published an image dataset of nocturnal moths captured manually by citizen scientists, called \emph{\mbox{EU-Moths}} dataset at a local workshop~\cite{Korsch21_DLP}.
This paper also includes a dataset description and our baseline results for insect localization and species classification.
There are two reasons for this.
First, we want to announce this dataset to a broader audience interested in fine-grained recognition because it can directly be used for algorithm development and evaluation.
Second, we want to highlight the challenges for recognition algorithms that arise when processing automatically captured camera trap images compared to manually taken images with hand-held cameras.

In general, our paper aims to promote the application of moth species identification as a fine-grained visual recognition problem.
We underpin this with existing datasets, results of baseline algorithms, and a light-based camera trap setup that will be used during the following years to automatically collect further large-scale image data.
We believe that research on automated visual identification of hundreds to thousands of different nocturnal moth species can have a major impact on developing fine-grained recognition algorithms in general, and we, therefore, want to share our insights and datasets with the community.

%The rest of the paper is structured as follows. 
%After a short review of related work in Section~\ref{sec:related_work}, we describe the two abovementioned datasets containing images of nocturnal moths in Section~\ref{sec:dataset}.
%The algorithms we applied to both datasets are described in Section~\ref{sec:methods} and we present the achieved results in Section~\ref{sec:results}. 
%We discuss challenges of processing automatically captured images with light-based camera traps in Section~\ref{sec:challenges} that are also important to consider for similar projects, followed by conclusions in Section~\ref{sec:conclusions}.

%\todo{REWRITE from here}

%Before the classification can be performed, we need to perform a detection of the insects.
%At this stage, the application of the state-of-the-art detection models like SSD~\cite{liu2016ssd} or YOLO~\cite{redmon2016you} is an obvious step.
%On the other hand, these models are computationally expensive and other light-weight methods like the MCC blob detector~\cite{bjerge2021automated} are more suitable for the application in the field.
%Unfortunately, to evaluate and compare different detection methods suitable benchmark datasets are missing.

%In this paper, we present a new dataset collected with the help of our prototype.
%In the period of five months, we captured over \num{27000} images in suburban area in Middle Germany.
%For bootstrapping and first evaluations we annotated a subset of these images with bounding boxes for the captured insects.
%The image data and the annotations will be soon publicly available.

%As a first baseline for insect detection task, we evaluated two different methods on the data and present these results further in our paper.
%First, we used a well-established Deep Learning detection model capable of identifying multiple objects in an image, namely the single-shot MultiBox detector (SSD)~\cite{liu2016ssd}.
%As a light-weight alternative that can be easily deployed directly at the computationally limited hardware of the moth scanner, we developed and evaluated a multi-step blob detection algorithm.
% \todo{Edge Computing as buzzword? EdgeAI may be wrong here?}
%First, it reduces the power consumption due to the reduction of computations.
%Further, applying the detection directly at the moth scanner, we can drastically reduce the amount of data that needs to be transmitted  when the system will gather data autonomously in the field.
%The algorithm is closely related to the blob detection method proposed by Bjerge~\etal\cite{bjerge2021automated} but mitigates some of the method's limits.
%We present the idea and the improvement in more detail in Sect.~\ref{sec:methods}.

% !TEX root = ../AttackGraphBasedRiskAnalysis.tex
% !TEX spellcheck = en_US
% !TEX encoding = UTF-8 Unicode

\section{Related Work}\label{sec: related work}
%\todo{Discuss all frameworks regarding consequences.}

Kordy et al.~\cite{DAGpaper} categorize thirty-three frameworks for graphical analysis of attack and defense scenarios into (1) \emph{attack and/or defense modeling}, which focus on the formal aspects of attacks or defenses, and (2) \emph{static or sequential modeling}, which focus on the temporal aspects or dependencies between actions. 
Using the same categorization, this section provides an overview of all the frameworks, and it describes these frameworks that fulfill the majority of properties incorporated in the framework of this article.

By reviewing frameworks from current literature, we identify seven properties for graphically modeling and managing an entire risk landscape.
The first property is \emph{attack vectors}, which enables the relations (shown as edges) between attack steps (shown as nodes) and, therefore, the formation of attack paths (i.e., attack vectors). 
The second property is the \emph{directed acyclic graph (DAG) structure}, thereby enabling linear (i.e., directed) and finite (i.e., acyclic) series of attack steps towards multiple potential attack goals (i.e., graph). 
The third property is \emph{node attributes}, which enables the quantification and, therefore, the evaluation of attack steps. 
The fourth property is \emph{dynamic connectors}, thereby enabling extensive attack refinements (besides the basic AND-OR refinements). 
The fifth property is \emph{edge attributes}, which enables the quantification and, therefore, the evaluation of relations between attack steps. 
The sixth property is \emph{countermeasure nodes}, thereby enabling actions to reduce the negative consequences of attacks.
The final property is \emph{consequence nodes}, enabling the presentation of consequences of successful attacks, which is also necessary to constitute the impact.

\begin{table*}[h]
\rowcolors{2}{gray!10}{gray!40}
\renewcommand{\arraystretch}{1.2}
\caption{Static attack modeling frameworks compared to the seven defined properties.}
\label{tab: static attack modeling}
\noindent\makebox[\textwidth]{%
\begin{tabular}[t]{>{\raggedright}p{0.15\textwidth}>{\raggedright}p{0.06\textwidth}>{\raggedright}p{0.08\textwidth}>{\raggedright}p{0.1\textwidth}>{\raggedright}p{0.09\textwidth}>{\raggedright\arraybackslash}p{0.08\textwidth}>{\raggedright\arraybackslash}p{0.15\textwidth}>{\raggedright\arraybackslash}p{0.1\textwidth}}
\toprule
 & Attack Vectors & DAG Structure & Node Attributes & Dynamic Connectors & Edge Attributes & Countermeasure Nodes & Consequence Nodes
\tabularnewline
\midrule
% \textbf{Static Attack Modeling} & & & & &
% \tabularnewline
Attack Trees & \checkmark & - & (\checkmark) & - & - & - & -
\tabularnewline
Augmented Vulnerability Trees & \checkmark & - & (\checkmark) & - & - & -  & -
\tabularnewline
Augmented Attack Trees & \checkmark & - & (\checkmark) & - & - & -  & -
\tabularnewline
OWA Trees & \checkmark & - & - & (\checkmark)  & (\checkmark) & -  & -
\tabularnewline
Parallel Model for Multi-Parameter Attack Trees & \checkmark & - & (\checkmark) & - & - & -  & -
\tabularnewline
Extended Fault Trees & \checkmark & - & (\checkmark) & - & - & -  & -
\tabularnewline
\bottomrule
\end{tabular}}
\end{table*}

Each one of the thirty-three frameworks presented in this section considers only subsets of the seven identified properties. 
None of these frameworks are suitable, as all seven properties are necessary to perform a full risk assessment.
To overcome this limitation, this article incorporates all seven identified properties into a framework for a graphical solution for performing risk analysis and examines its applicability to different risk analysis standards.

\subsection{Static Attack Modeling}\label{sec: static attack modeling}




Six frameworks for \emph{static attack modeling}, namely \emph{Attack Trees}~\cite{weiss1991}, \emph{Augmented Vulnerability Trees}~\cite{AugmentedVulnerabilityTrees}, \emph{Augmented Attack Trees}~\cite{AugmentedAttackTrees}, \emph{OWA Trees}~\cite{Yager2006OWATA}, \emph{Parallel Model for Multi-Parameter Attack Trees}~\cite{ParallelModelForMultiParameterAttackTrees}, and \emph{Extended Fault Trees}~\cite{ExtendedFaultTrees}, are summarised in Table~\ref{tab: static attack modeling}.
All frameworks fulfill the attack vectors property, but none of them supports the DAG structure, countermeasure nodes, and consequence nodes properties. 
Of the six frameworks, OWA trees stand out as they at least partially fulfill the dynamic connectors and edge attributes properties, despite being the only framework that does not fulfill the node attributes property. 
This section describes attack trees, which was the first graphical security modeling framework, and OWA trees, which is the framework that at least partially fulfills most of the seven identified properties.

\subsubsection{Attack Trees}\label{sec: attack trees}

The first \emph{tree-based approach}, shown as an AND-OR tree structure for graphical security modeling, was the \emph{threat logic trees}, which was introduced by Weiss in 1991~\cite{weiss1991}.
Today, all AND-OR tree structures are referred to as \emph{attack trees}, a term first introduced by Salter et al. in 1998~\cite{Salter1998}.

In attack trees, the root node (i.e., the tree's root) indicates the attack's main goal. 
The main goal is then conjunctively (AND) or disjunctively (OR) refined into sub-goals until they represent basic actions corresponding to atomic components that can be easily understood and quantified. 
Conjunctive refinements indicate that \emph{all} sub-goals need to be fulfilled in order to achieve the main goal, whereas disjunctive refinements indicate that \emph{at least one} sub-goal needs to be fulfilled for achieving the main goal~\cite{weiss1991}.

\subsubsection{OWA Trees}\label{sec: owa trees}

\emph{Ordered weighted averaging (OWA) trees} were proposed by Yager in 2005 to include the concept of \emph{uncertainty} into attack trees~\cite{Yager2006OWATA}. 
This was made possible by replacing the AND-OR nodes with OWA nodes (i.e., quantifiers, such as \emph{most}, \emph{some}, \emph{half of}, etc.) and therefore taking into consideration situations where the number of sub-goals that need to be fulfilled in order to achieve the main goal remains unknown. 
Finally, OWA trees allow for the evaluation of success probability and cost attributes, which can be jointly used to calculate the cheapest and most probable attack.

\subsection{Sequential Attack Modeling}\label{sec: sequential attack modeling}

\begin{table*}[h]
\rowcolors{2}{gray!10}{gray!40}
\renewcommand{\arraystretch}{1.2}
\caption{Sequential attack modeling frameworks compared to the seven defined properties.}
\label{tab: sequential attack modeling}
\noindent\makebox[\textwidth]{%
\begin{tabular}[t]{>{\raggedright}p{0.15\textwidth}>{\raggedright}p{0.06\textwidth}>{\raggedright}p{0.08\textwidth}>{\raggedright}p{0.1\textwidth}>{\raggedright}p{0.09\textwidth}>{\raggedright\arraybackslash}p{0.08\textwidth}>{\raggedright\arraybackslash}p{0.15\textwidth}>{\raggedright\arraybackslash}p{0.1\textwidth}}
\toprule
 & Attack Vectors & DAG Structure & Node Attributes & Dynamic Connectors & Edge Attributes & Countermeasure Nodes & Consequence Nodes
\tabularnewline
\midrule
% \textbf{Sequential Attack Modeling} & & & & &
% \tabularnewline
Cryptographic DAGs & \checkmark & \checkmark & - & - & - & - & - 
\tabularnewline
Fault Trees for Security & \checkmark & - & \checkmark & (\checkmark) & - & -  & -
\tabularnewline
Bayesian Networks for Security & \checkmark & \checkmark & \checkmark & - & \checkmark & - & -
\tabularnewline
Bayesian Attack Graphs & \checkmark & \checkmark & \checkmark & - & \checkmark & - & -
\tabularnewline
Compromise Graphs & \checkmark & \checkmark & - & - & (\checkmark) & - & -
\tabularnewline
Enhanced Attack Trees & \checkmark & - & \checkmark & - & (\checkmark) & - & -
\tabularnewline
Vulnerability Cause Graphs & (\checkmark) & \checkmark & - & - & - & - & -
\tabularnewline
Dynamic Fault Trees for Security & \checkmark & - & (\checkmark) & - & - & - & -
\tabularnewline
Serial Model for Multi-Parameter Attack Trees & \checkmark & - & (\checkmark) & - & - & - & -
\tabularnewline
Improved Attack Trees & \checkmark & - & (\checkmark) & - & - & - & -
\tabularnewline
Time-dependent Attack Trees & \checkmark & \checkmark & (\checkmark) & - & - & - & -
\tabularnewline
\bottomrule
\end{tabular}}
\end{table*}

Eleven frameworks for \emph{sequential attack modeling}, namely \emph{Cryptographic DAGs}~\cite{Meadows1996ARO}, \emph{Fault Trees for Security}~\cite{FaultTreesForSecurity}, \emph{Bayesian Networks for Security}~\cite{BayesianNetworksForSecurity}, \emph{Bayesian Attack Graphs}~\cite{BayesianAttackGraphs}, \emph{Compromise Graphs}~\cite{CompromiseGraphs}, \emph{Enhanced Attack Trees}~\cite{EnhancedAttackTrees}, \emph{Vulnerability Cause Graphs}~\cite{VulnerabilityCauseGraphs}, \emph{Dynamic Fault Trees for Security}~\cite{DynamicFaultTreesForSecurity}, \emph{Serial Model for Multi-Parameter Attack Trees}~\cite{SerilModelForMultiParameterAttackTrees}, \emph{Improved Attack Trees}~\cite{ImprovedAttackTrees}, and \emph{Time-dependent Attack Trees}~\cite{TimeDependentAttackTrees}, are summarised in Table~\ref{tab: sequential attack modeling}. 
Again, none of the frameworks fulfills the countermeasure and consequence nodes property. In addition, only Fault Trees for Security offer a wide range of dynamic connectors, and only Bayesian-based models fulfill the edge attributes property. 
Finally, Compromise Graphs and Enhanced Attack Trees are two frameworks that at least partially fulfill the edge attributes property, and Vulnerability Cause Graphs are the only framework that only partially fulfills the attack vectors property. 
This section describes Cryptographic DAGs, which was the first graph-based approach for security modeling, and Bayesian Attack Graphs, which combine attack trees and Bayesian networks and also fulfill four of the seven identified properties.

\subsubsection{Cryptographic DAGs}\label{sec: cryptographic dags}

\emph{Cryptographic directed acyclic graphs} were proposed by Meadows~\cite{Meadows1996ARO} in 1996 to provide a \emph{novel} simple representation of sequences and dependencies of attack steps towards the main goal of the attack. 
Instead of a tree-based approach, Cryptographic DAGs introduced a \emph{graph-based approach} for security modeling. 
However, they eventually do not offer the possibility to perform risk assessment as other properties are still not fulfilled.

\subsubsection{Bayesian Networks and Bayesian Attack Graphs}\label{sec: bayesian Networks and Attack Graphs}

For the last couple of decades, researchers have been focusing on \emph{Bayesian networks} for the purposes of security modeling.
The origin of Bayesian networks, which are also known as \emph{belief} or \emph{causal networks}, lies in artificial intelligence.
In Bayesian networks, nodes represent events or objects and are associated with probabilistic variables. 
Hence, analyzing the uncertainty of events is also possible. 
Bayesian networks follow a DAG structure, where the directed edges represent the causal dependencies between the nodes~\cite{BayesianAttackGraphs}.

\emph{Bayesian attack graphs} are a fusion of (general) attack trees and (computational procedures) of Bayesian networks, and they were first introduced by Liu and Man in 2005 to analyze network vulnerability scenarios~\cite{BayesianAttackGraphs}. 
Subsequently, calculating general security metrics regarding information system networks~\cite{Frigault2008, Noel2010} and capturing dynamic behavior~\cite{Frigault2008Dyn} was also made possible.

Finally, although Bayesian attack graphs allow for assigning values to nodes and for performing computations using the graphs, they do not allow for a dynamic selection of connectors and for including countermeasures. 
As a result, Bayesian attack graphs cannot be used to perform risk assessment.

\subsection{Static Attack and Defense Modeling}\label{sec: static attack and defense modeling}

\begin{table*}[h]
\rowcolors{2}{gray!10}{gray!40}
\renewcommand{\arraystretch}{1.2}
\caption{Static attack and defense modeling frameworks compared to the seven defined properties.}
\label{tab: static attack and defense modeling}
\noindent\makebox[\textwidth]{%
\begin{tabular}[t]{>{\raggedright}p{0.15\textwidth}>{\raggedright}p{0.06\textwidth}>{\raggedright}p{0.08\textwidth}>{\raggedright}p{0.1\textwidth}>{\raggedright}p{0.09\textwidth}>{\raggedright\arraybackslash}p{0.08\textwidth}>{\raggedright\arraybackslash}p{0.15\textwidth}>{\raggedright\arraybackslash}p{0.1\textwidth}}
\toprule
 & Attack Vectors & DAG Structure & Node Attributes & Dynamic Connectors & Edge Attributes & Countermeasure Nodes & Consequence Nodes
\tabularnewline
\midrule
% \textbf{Static Attack and Defense Modeling}
% \tabularnewline
Anti-Models & \checkmark & - & - & - & - & \checkmark & -
\tabularnewline
Defense Trees & \checkmark & - & (\checkmark) & - & - & \checkmark & -
\tabularnewline
Protection Trees & - & - & \checkmark & - & - & \checkmark & -
\tabularnewline
Security Activity Graphs & \checkmark & \checkmark & (\checkmark) & - & - & \checkmark & -
\tabularnewline
Attack Countermeasure Trees & \checkmark & - & \checkmark & - & - & \checkmark & -
\tabularnewline
Attack-Defense Trees & \checkmark & - & \checkmark & - & - & \checkmark & -
\tabularnewline
Countermeasure Graphs & \checkmark & \checkmark & \checkmark & - & - & \checkmark & -
\tabularnewline
\bottomrule
\end{tabular}}
\end{table*}

Seven frameworks for \emph{static attack and defense modeling}, namely \emph{Anti-Models}~\cite{AntiModels}, \emph{Defense Trees}~\cite{DefenseTrees}, \emph{Protection Trees}~\cite{ProtectionTrees}, \emph{Security Activity Graphs}~\cite{SecurityActivityGraphs}, \emph{Attack Countermeasure Trees}~\cite{AttackCountermeasureTrees}, \emph{Attack-Defense Trees}~\cite{AttackDefenseTrees}, and \emph{Countermeasure Graphs}~\cite{CountermeasureGraphs}, are summarised in Table~\ref{tab: static attack and defense modeling}. 
All frameworks fulfill the countermeasure nodes property, and only Protection Trees do not fulfill the attack vectors property. 
In addition, Anti-Models is the only framework that does not at least partially fulfill the node attributes property.
However, none of these frameworks considers consequence nodes in their design.
This section describes Security Activity Graphs and Countermeasure Graphs, which are the two frameworks that fulfill four of the seven identified properties.

\subsubsection{Security Activity Graphs}\label{sec: security activity graphs}

\emph{Security activity graphs (SAGs)} were developed by Ardi et al.~\cite{SecurityActivityGraphs} in 2006 to improve security throughout the software development process. 
SAGs are loosely based on fault trees, and the root of a SAG is associated with a vulnerability. 
Vulnerability mitigations are modeled using activities (i.e., leaf nodes), which are assigned boolean variables to indicate whether an activity \enquote{is implemented perfectly during software development} (true) or not (false). 
Finally, besides AND-OR gates, which follow a strictly Boolean logic, SAGs also include \emph{split gates}, which allow one activity to be used in several parent activities, thus creating a DAG structure.

However, SAGs lack the ability to represent the consequences of threats and edge attributes, and both are necessary to calculate a risk value.
Furthermore, there is only a limited option for connectors and node attributes.
Therefore, rendering SAGs impractical for risk assessment.

\subsubsection{Countermeasure Graphs}\label{sec: countermeasure graphs}

\emph{Countermeasure graphs} were introduced by Baca and Petersen~\cite{CountermeasureGraphs} in 2010 to simplify countermeasure selection through cumulative voting. 
Countermeasure graphs are created by identifying actors, goals, attacks, and countermeasures. Related events are connected with edges. 
That is, actors are connected to pursued goals and likely executable attacks, and countermeasures are connected to preventable attacks. 
Finally, actors, goals, attacks, and countermeasures are assigned priorities according to the rules of hierarchical cumulative voting. 
Higher assigned priorities imply higher threat levels of the corresponding events and vice versa. 
Using hierarchical cumulative voting, the most effective countermeasures can be identified.

Countermeasure Graphs provide a useful system overview, but the computational rules focus on finding the most effective countermeasure instead of the most likely and severe attack. 
This limitation could at least partially be addressed with the threat level. 
However, the threat level value is determined by the subjective assessment of the graph creator rather than by calculations over meaningful attributes, thereby raising issues of validity.

\subsection{Sequential Attack and Defense Modeling}\label{sec: sequential attack and defense modeling}

\begin{table*}[h]
\rowcolors{2}{gray!10}{gray!40}
\renewcommand{\arraystretch}{1.2}
\caption{Sequential attack and defense modeling frameworks compared to the seven defined properties.}
\label{tab: sequential attack and defense modeling}
\noindent\makebox[\textwidth]{%
\begin{tabular}[t]{>{\raggedright}p{0.15\textwidth}>{\raggedright}p{0.06\textwidth}>{\raggedright}p{0.08\textwidth}>{\raggedright}p{0.1\textwidth}>{\raggedright}p{0.09\textwidth}>{\raggedright\arraybackslash}p{0.08\textwidth}>{\raggedright\arraybackslash}p{0.15\textwidth}>{\raggedright\arraybackslash}p{0.1\textwidth}}
\toprule
 & Attack Vectors & DAG Structure & Node Attributes & Dynamic Connectors & Edge Attributes & Countermeasure Nodes & Consequence Nodes
\tabularnewline
\midrule
% \textbf{Sequential Attack and Defense Modeling} & & & & &
% \tabularnewline
Insecurity Flows & \checkmark & \checkmark & \checkmark & - & - & \checkmark & -
\tabularnewline
Intrusion DAGs & \checkmark & \checkmark & - & - & - & \checkmark & -
\tabularnewline
Bayesian Defense Graphs & \checkmark & \checkmark & (\checkmark) & - & - & \checkmark & -
\tabularnewline
Security Goal Indicator Trees & - & - & - & - & - & \checkmark & -
\tabularnewline
Attack Response Trees & \checkmark & -  & \checkmark & - & - & \checkmark & -
\tabularnewline
Boolean Logic Driven Markov Processes & \checkmark & \checkmark & (\checkmark) & (\checkmark) & - & \checkmark & -
\tabularnewline
Cyber Security Modeling Language & \checkmark & \checkmark & (\checkmark) & - & (\checkmark) & \checkmark & -
\tabularnewline
Security Goal Models & \checkmark & \checkmark & - & - & - & \checkmark & -
\tabularnewline
Unified Parameterizable Attack Trees & \checkmark & - & \checkmark & - & (\checkmark) & \checkmark & -
\tabularnewline
\bottomrule
\end{tabular}}
\end{table*}

Finally, nine frameworks for \emph{sequential attack and defense modeling}, namely \emph{Insecurity Flows}~\cite{InsecurityFlows}, \emph{Intrusion DAGs}~\cite{IntrusionDAGs}, \emph{Bayesian Defense Graphs}~\cite{BayesianDefenseGraphs}, \emph{Security Goal Indicator Trees}~\cite{SecurityGoalIndicatorTrees}, \emph{Attack Response Trees}~\cite{AttackResponseTrees}, \emph{Boolean Logic Driven Markov Process}~\cite{BooleanLogicDrivenMarkovProcess}, \emph{Cyber Security Modeling Language}~\cite{CyberSecurityModelingLanguage2010}, \emph{Security Goal Models}~\cite{SecurityGoalModels}, and \emph{Unified Parameterizable Attack Trees}~\cite{UnifiedParameterizableAttackTrees}, are summarized in Table~\ref{tab: sequential attack and defense modeling}. 
All frameworks fulfill the countermeasure nodes property, and only Security Goal Indicator Trees do not fulfill the attack vectors property. 
In addition, Boolean Logic Driven Markov Processes (BDMPs) is the only framework that offers a wide range of connectors and, therefore, at least partially fulfills the dynamic connectors property. 
This section describes BDMPs and Cyber Security Modeling Language (CySeMoL), which are the two frameworks that at least partially fulfill five of the seven identified properties.

\subsubsection{Boolean Logic Driven Markov Processes}\label{sec: boolean logic driven markov processes}

\emph{Boolean logic driven Markov processes (BDMPs)} are a security modeling framework, which can also be used to perform risk assessment~\cite{BooleanLogicDrivenMarkovProcess}. 
It was invented by Bouissou and Bon~\cite{BooleanLogicDrivenMarkovProcess} in 2003 for the safety and reliability domains, and it was later adapted to security modeling by Piètre-Cambacédès and Bouissou in 2010. 
BDMPs combine the readability of attack trees with the modeling power of Markov chains. 
The root (top event) of a BDMP represents the main goal of the attack, and the leaves represent the attack steps or security events.
BDMPs offer a wide range of node attributes, including time-domain metrics, such as mean-time to success, attack tree-related metrics, such as costs of attacks, boolean indicators, such as specific requirements, and risk assessment tools, such as sensibility graphs.

However, the lack of edge attributes, in addition to issues of usability with respect to leaf nodes and connectors~\cite{BDMPCritic}, render BDMPs impractical for risk assessment.

\subsubsection{Cyber Security Modeling Language}\label{sec: cyber security modeling language}

\emph{Cyber security modeling language (CySeMoL)} was developed by Sommestad et al. in 2010 to assess the cyber security of \emph{supervisory control and data acquisition (SCADA)} system architectures~\cite{CyberSecurityModelingLanguage2010, CyberSecurityModelingLanguage2013}.
Simply modeling the system architecture and the characteristics of the involved assets is sufficient, as CySeMoL already includes information about how attacks and defenses are quantitatively related. 
The attacker is assumed to be a professional penetration tester with a fixed time of one week to perform an attack.
CySeMoL was extended by Holm in 2014 and renamed to \emph{predictive, probabilistic cyber security modeling language ((P$^2$)CySeMoL)}, introducing more flexible and useful computations, the possibility to model assets, attacks, and defenses that are not necessarily SCADA-related, and the option to specify the time needed to perform an attack~\cite{PredictiveProbabilisticCyberSecurityModelingLanguage}.
Computations can be conducted automatically (i.e., without personalized inputs) as (P$^2$)CySeMoL already includes qualitative information gathered from literature reviews, empirical studies, as well as surveys involving domain experts~\cite{CyberSecurityModelingLanguage2010, CyberSecurityModelingLanguage2013, PredictiveProbabilisticCyberSecurityModelingLanguage}.

The results of the computations show the likelihood of an attack. 
However, the severity of an attack is not considered, and therefore the risk of an attack cannot be properly assessed. 
Furthermore, (P$^2$)CySeMoL does not include connectors, and therefore it seems an inconvenient tool for graphical risk assessment.

\subsection{Summary of Remarks}\label{sec2: summary of remarks}

This section provides an overview of thirty-three frameworks for analysis of attack and defense scenarios, and it describes eight of these frameworks in more detail. 
Thirty frameworks fulfill the attack vectors property, sixteen frameworks fulfill the countermeasure nodes property, and only thirteen frameworks fulfill the DAG structure property.
In addition, node/edge attributes and connectors are, in most cases, fixed and limited, thereby reducing the usability and usefulness of the frameworks with respect to the purposes of risk assessment. 
The complex nature and rapid development of (information) systems, attacks, and defenses motivate the need for proper risk management.
Existing methods are mainly consisting of tables with graphical solutions mostly utilized for support, if at all.
As shown in this section, current graphical solutions support threat or vulnerability management and sometimes even calculations to determine which attack vector might be the easiest to execute or, in other terms, which is most probable to occur.
The risk value cannot be equated with probability, though, and is usually determined using the probability of an event and its impact.
However, none of the methods described in this section can represent an event's consequences and impact, rendering them incapable of performing risk assessment.


\section{Methods}
\label{sec:method}

CTVIS builds upon Mask2Former \cite{mask2former}, which is an effective image instance segmentation model (briefly reviewed in Section~\ref{sec:mask2former})\footnote{
%We would like to
Note that CTVIS can be easily combined with other query-based instance segmentation models \cite{idol, detr, deformabledetr} with minor modifications.}. Our CTVIS is motivated by the inference of typical online VIS methods introduced in Section~\ref{sec:inference}. 
Then we detail our consistent training method in Section~\ref{sec:ct}. Finally, Section~\ref{sec:pseudo} presents our goal-oriented pseudo-video generation technique for training VIS models with sparse image-level annotations.

% We closely follow the Notations in MinVIS
\subsection{Brief Overview of Mask2Former} 
\label{sec:mask2former}
Mask2Former \cite{mask2former} composed of three main components: an \emph{image encoder} $\mathcal{E}$ (consist of a backbone and a pixel decoder), a \emph{transformer decoder} $\mathcal{T}$ and a \emph{prediction head} $\mathcal{P}$. Given an input image $I\in \mathbb{R}^{H \times W \times 3}$, $\mathcal{E}$ extracts a set of feature maps $\bm{F}=\mathcal{E}(I)$, where $\bm{F} = \{ F_0 \cdots F_{-1}\}$ is a sequence of multi-scale feature maps, and $F_{-1}$ is the final output of the $\mathcal{E}$ with $1/4$ resolution of $I$. The $N$ raw query embeddings $\hat{Q} \in \mathbb{R}^{N \times C}$ are learnable parameters, where $N$ is a large enough number of outputs and $C$ is the number of channels. Then, $\mathcal{T}$ takes both $\bm{F}$ and $\hat{Q}$ to iteratively refine query embeddings, and consequently outputs $Q \in \mathbb{R}^{N \times C}$. Finally, the prediction head outputs the segmentation masks $M$ and the classification scores $O$. For classification, $O=\mathcal{C}(Q) \in \mathbb{R}^{N \times K}$, where $K$ is the number of object categories. For  segmentation, the masks $M \in \mathbb{R}^{N \times H/4 \times W/4}$ are generated with $M = \sigma(Q \ast F_{-1})$, where $\ast$ denotes the convolution operation and $\sigma(\cdot)$ is the sigmoid function.

\noindent\textbf{Our Modification.} Because CTVIS employs instance embeddings to associate instances during inference, we add a  head (a few MLP layers) to compute the instance embeddings $E \in \mathbb{R}^{N \times C}$ based $Q$. 
% The entire process can be summarized as
% \begin{equation}
% \label{eq:mask2former}
%     O, M, E = Mask2Former(I).
% \end{equation}

\subsection{Inference of CTVIS}
\label{sec:inference}
CTVIS leverages Mask2Former\cite{mask2former} to process each frame 
%(\ie Equation~\eqref{eq:mask2former}) 
and introduces an external memory bank\cite{idol, masktrackrcnn} to store the states of previously detected instances, including classification scores, segmentation masks and instance embeddings. 
% gets the corresponding classification scores, segmentation masks and instance embeddings for each frame. 
% Specially, CTVIS makes instance association frame by frame and introduces an external memory bank to store the states of previously detected instances, including classification scores, segmentation masks and instance embeddings. 
To ease presentation, we assume that CTVIS has already processed $T$ frames out of an input video of $L$ frames, and there are $N$ predicted instances with $N$ instance embeddings $\bold{d}_i \in \mathbb{R}^C$ in the current frame. The memory bank stores for the previous $T$ frames $M$ detected instances, each of which has multiple temporal instance embeddings $\{ \bold{e}^t_j \in \mathbb{R}^C  \}^T_{t=1}$ and a momentum-averaged instance embedding $\hat{\bold{e}}_j^T$, which is computed according to the similarity-guided fusion \cite{sgf}: 
\begin{gather}
    \label{eq:sgf}
    \hat{\bold{e}}^T_j=(1-\beta^T) \hat{\bold{e}}^{T-1}_j+\beta^T \bold{e}^T_j \text {, } \\
    \beta^T=\max \left\{0, \frac{1}{T-1} \sum_{k=1}^{T-1} \Psi_d\left(e^T_j, e^{T-k}_j\right)\right\} , 
\end{gather}

% Figure environment removed 

\noindent where $\Psi_d$ denotes the cosine similarity. % This momentum type brings more flexblity for instance embedding fusion. 
We refer the reader to \cite{sgf} for more details. Next, for each instance $i$ detected in the current frame, we compute its bi-softmax similarity \cite{qdtrack} with respect to the previously detected instance $j$ using


\begin{equation}
\label{equ:bio_softmax}
    f_{i,j}=
    0.5 \cdot \left[\frac{\exp \left(\hat{\mathbf{e}}_j^T \cdot \mathbf{d}_i\right)}{\sum_k \exp \left(\hat{\mathbf{e}}_k^T \cdot \mathbf{d}_i\right)}+\frac{\exp \left(\hat{\mathbf{e}}_j^T \cdot \mathbf{d}_i\right)}{\sum_{l} \exp \left(\hat{\mathbf{e}}_j^T \cdot \mathbf{d}_l\right)}\right] %\cdot 
 %   / 2. 
\end{equation}

Finally, we find the ``best''  
instance ID for $i$ with
\begin{equation}
\hat{j}=\arg \max f_{i,j}, \forall j \in\{1,2, \ldots, M\}.
\end{equation}
If $f_{i,\hat{j}} > 0.5$, we believe that newly detected instance $i$ and instance $\hat{j}$ in the memory bank correspond to the identical target. Otherwise, we initiate a new instance ID in the memory bank. When all frames are processed, the memory bank contains a certain number of instances, each of which takes a classification score list $\{c_i^t\}_{t=1}^{L}$ and a mask list $\{m_i^t\}_{t=1}^{L}$ (recall that $L$ denotes the number of frames). For each instance $i$, we calculate its video-level classification score by averaging the frame-level scores of the object. 
% $\boldsymbol{c_i}$ as follows: $\boldsymbol{c_i} = \Sigma_{t=1}^L c_i^t$. We get the $\{(\boldsymbol{c_i}, \{m_i^t\}_{t=1}^{L})\}_{i=1}^S$ as final outputs. 

% \subsection{Constructing CIs via Consistent Training}
\subsection{Consistent Learning}
\label{sec:ct}

A reliable matching of instances (\ie using Equation~\eqref{equ:bio_softmax}) across time is required to track instances successfully. Hence the extraction of highly discriminative embeddings of objects is of great importance. 
We argue that the discrimination of instance embeddings extracted with recent models \cite{idol, stc} is still inadequate, especially for videos involving object-occlusion, shape-transformation and fast-motion. One main reason is that mainstream contrastive learning methods build CIs (\ie $\{\mathbf{v},\mathbf{k}^+,\mathbf{k}^-\}$) from the reference frame only, which results in the comparison of the anchor embedding against instantaneous instance embeddings in $\mathbf{k}^+$ and $\mathbf{k}^-$. Such embeddings are typically less discriminative and contain noise, which prevents training from learning robust representations. To address this, our CTVIS leverages a memory bank to store MA embeddings, thus supporting contrastive learning from more stable representations. Here our insight is to align the embedding comparison of training with that of inference (such that the two comparisons are consistent). Figure~\ref{fig:main} sketches our CTVIS, which processes the training video frame-by-frame. For an arbitrary frame $t$, CTVIS involves three steps: a) it first takes the Mask2Former and Hungarian matching to compute the instance embeddings, and to match them with GT (highlighted by red, green and purple); b) Then, it builds CIs using MA embeddings within the memory bank, and performs contrastive learning with CIs; and c) It updates the memory bank with noise (\eg the embedding of the \emph{cat} is deliberately added to the memory of the \emph{dog}), which serves the learning from the next frame.

\noindent\textbf{Forward passing and GT assignment.} As shown in Figure~\ref{fig:main}~(a), we first feed the current frame $t$ into Mask2Former to compute the embeddings for queries. Then we employ Hungarian matching to find an optimal match between the decoded instances and the ground truth (GT), such that each GT instance is assigned one instance embedding. Note that Hungarian matching relies on the costs calculated for all (\emph{Decoded-Instance}, \emph{GT-Instance}) pairs. Essentially, each cost measures the similarity between a pair of instances based on their labels and masks.

\noindent\textbf{Construct CIs.} 
After GT assignment, we build the contrastive items for each GT instance using a memory bank. The memory bank stores all detected instances of previous $t-1$ frames, each associated with 1) a series of instance embeddings extracted at different times, and 2) its MA embedding computed by Equation~\eqref{eq:sgf}. 
% To clarify, we only  show the contrastive item of the person in Figure~\ref{fig:main}(b), we select the instance embeddings of the person at current frames as query embedding $v$. For the positive embedding, we select the momentum-averaged embedding of person from the memroy bank
In order to prepare the CIs $\{\mathbf{v}, \mathbf{k}^+, \mathbf{k}^-\}$ for instance $i$ (termed as the \emph{anchor}, \eg the person in Figure~\ref{fig:main}~(a)) at the $t$-th frame, the instance embedding extracted from this frame is used as the anchor embedding $v$.
%
For the positive embedding, we pick from the memory bank the MA embedding of instance $i$.
%
The negative embeddings $\mathbf{k}^-$ include the major negative embeddings and the supplementary negative embeddings. We use the MA embeddings of other instances in the memory bank as the major negative embeddings. We also sample the background query embeddings of previous $t - 1$ frames to form the supplement negative embeddings. Taking as inputs the created CIs, we compute the contrastive loss with
\begin{equation}
\label{eq:loss_embed}
\begin{aligned}
    \mathcal{L}_{\text {emb}} & =-\log \frac{\exp \left(\mathbf{v} \cdot \mathbf{k}^{+}\right)}{\exp \left(\mathbf{v} \cdot \mathbf{k}^{+}\right)+\sum\nolimits_{\mathbf{k}^{-}} \exp \left(\mathbf{v} \cdot \mathbf{k}^{-}\right)} \\
    & =\log \left[1+\sum\nolimits_{\mathbf{k}^{-}} \exp \left(\mathbf{v} \cdot \mathbf{k}^{-}-\mathbf{v} \cdot \mathbf{k}^{+}\right)\right].
\end{aligned}
\end{equation}
As shown in Figure~\ref{fig:main} (c), training with $\mathcal{L}_{\text {emb}}$ pulls the embeddings of positive instances close to the anchor embedding, while pushing the negative embeddings away from it.

\noindent\textbf{Update memory bank.} After computing the $\mathcal{L}_{\text{emb}}$ for each instance in frame $t$, we need to update the memory bank, such that the updated version can be taken to build CIs for frame $t+1$.
%
Unlike the inference stage, for training we can get the ground truth ID of each instance so as to update the memory bank with their embeddings extracted from frame $t$.
%
In comparison, inference can fail to track instances across time (\ie the ID switch issue), especially for complicated scenarios. To alleviate this, we introduce noise to the update of the memory bank, which compels the contrastive learning to tackle the switch of instance IDs.
%
Specifically, each disappeared instance (\eg the dog) in frame $t$ will have a little chance to receive an embedding of other instances (\eg the cat, which is randomly picked from all available instances) in the same frame, which is called the \emph{noise}. 
%
% As illustrated in Figure~\ref{fig:main}~(c), the dog disappeared in frame $t$, and a new instance of cat presents.
%
If the generated random value exceeds a threshold (\eg 0.05), as illustrated in Figure~\ref{fig:main}~(c), we use the noise as the embedding of the disappeared instance at frame $t$. Finally, the MA embeddings are updated for all instances using Equation~\eqref{eq:sgf}. Due to the low similarity between the disappeared instance and the noise, such an update has quite a limited impact on the MA embedding of the instance, which is reidentified later. Indeed, training with noise is able to reduce the chance of ID switch, as demonstrated by the fish example in Figure~\ref{fig:video}. 

\noindent\textbf{Loss.} After processing all frames, The $\mathcal{L}_{\text {emb}}$ values of all CIs are averaged to obtain $L_{\text {emb}}$.
The total training loss is
\begin{equation}
L_{\text{total}} = \lambda_{\text{emb}}L_{\text{emb}} + \lambda_{\text{cls}} L_{\text{cls}} + \lambda_{\text{ce}} L_{\text {ce}} + \lambda_{\text{dice}} L_{\text{dice}},
\end{equation}
where $\lambda$ denotes loss weight. $L_{\text {cls}}$, $L_{\text {ce}}$ and $L_{\text {dice}}$ supervise the per-frame segmentation as suggested in \cite{mask2former}.

% when the ID of an instance changes to another instance in a complicated scene, most current methods always accumulate errors; to ease this issue, we introduce noise training, which directly simulates this situation during the construction of CI. As illustrated in Figure~\ref{fig:main}, the dog disappeared in the third frame, but a new instance of the cat appeared, and we added the cat's embedding to the external memory bank of the dog. Due to the low similarity between the instance embeddings of cats and dogs, it will have little impact on the MA embedding of further dogs that appear in the following frames. As shown in the video scene on the right of Figure~\ref{fig:video}, the wrong instances are corrected to original trajectories through noise training. 

% Here we describe how to build CIs via consistent training. Following the inference pipeline, we build the CIs frame by frame with updating the memory bank. For each frame (expect the first frame),  we will build CIs for each instances. 

% The consistent training aims at constructing CIs frame by frame following the inference stage introduced in Section~\ref{sec:inference}. As shown in Figure~\ref{fig:main}, we sample $T$ temporally adjcent frames as train video $\{ I_t\}_{t=1}^T$. 
% The first step is feeding each frame into a weight-shared Mask2Former and then matching the output with the corresponding ground truth via Hungarian matching. The output of each input frame $I_t$ comprises classification scores, segmentation masks and instance embeddings, formulated as $\{ O_t, M_t, E_t\}$. Then we calculate the pair-wise matching cost, considering both class prediction and the similarity of predicted and the ground truth masks. Next, we use Hungarian matching to assign one predicted instance to each ground truth instance. Specially, for each ground truth instance $j$ at frame $I_t$, we have a matched predicted instance embedding $\mathbf{e}^t_j$. 

% Then we construct contrastive items, each of which consists of anchor/positive/negative embeddings, for each ground truth instances at each frames. 
% % Contrastive items are composed of three parts: anchor embedding, positive embedding and negative embedding. Assume we contruct the contrastive item for the instance $i$ at the $I_t$, we build 
% Here we detail the construction of the CI of the GT instance $i$ at the frame $I_t$ (e.g. the dog at the $I_5$ shown in Figure~\ref{fig:main}). As shown in Figure~\ref{fig:main} (b), the memory bank stores the status of instances of previous frames. For each instance ID in the memory bank, we can get corresponding momentum-averaged embedding from the instance embeddings via Equation~\ref{eq:sgf}. We sample the instance embedding $e^t_j$ at the current frame as the anchor embedding $v$. For the positive embedding, we choose the momentum-averaged embedding of the same instance ID from the memory bank. The negative embeddings $k^-$ of contrastive item compose of two parts: major negative embeddings and supplement negative embeddings. And we select the momentum-averaged embeddings of other instance IDs as major negative embeddings. Furthermore, we sample the background embedding of previous $t - 1$ frames as supplement negative embeddings. Finally, we compute the contrastive loss upon each contrastive item as follows:
% \begin{equation}
% \label{eq:loss_embed}
% \begin{aligned}
%     \mathcal{L}_{\text {embed}} & =-\log \frac{\exp \left(\mathbf{v} \cdot \mathbf{k}^{+}\right)}{\exp \left(\mathbf{v} \cdot \mathbf{k}^{+}\right)+\sum_{\mathbf{k}^{-}} \exp \left(\mathbf{v} \cdot \mathbf{k}^{-}\right)} \\
%     & =\log \left[1+\sum_{\mathbf{k}^{-}} \exp \left(\mathbf{v} \cdot \mathbf{k}^{-}-\mathbf{v} \cdot \mathbf{k}^{+}\right)\right].
% \end{aligned}
% \end{equation}
% As shown in Figure~\ref{fig:main} (c), the $\mathcal{L}_{\text {embed}}$ enforces the embeddings of same instances while draws embeddings of different embeddings far away. 

% In addition, when the ID of an instance changes to another instance in a complicated scene, most current methods always accumulate errors; to ease this issue, we introduce noise training, which directly simulates this situation during the construction of CI. As illustrated in Figure~\ref{fig:main}, the dog disappeared in the third frame, but a new instance of the cat appeared, and we added the cat's embedding to the external memory bank of the dog. Due to the low similarity between the instance embeddings of cats and dogs, it will have little impact on the MA embedding of further dogs that appear in the following frames. As shown in the video scene on the right of Figure~\ref{fig:video}, the wrong instances are corrected to original trajectories through noise training. 

% Finally, we get the final contrastive loss by average on all contrastive items. Besides, we get the loss final loss as follows: $$

% For instance, a contrastive item is generated for contrastive learning that matches the ground truth (GT) on each frame. This contrastive item is mainly composed of three parts:
% \begin{itemize}
% \item The anchor, which is the instance embedding of the current frame
% \item The positive and negative samples, which are the instances matched with GT by calculating the momentum-averaged instance  embedding of the current frame through the similarity-guided fusion mentioned in Section~\ref{sec:inference}
% \item Other embeddings that do not match GT, which are directly treated as negative samples
% \end{itemize}

% When the ID of an instance changes to another instance in a complicated scene, most current methods will always be wrong; thus, we directly simulate this situation in the training phase. As illustrated in Figure~\ref{fig:main}, the dog disappeared in the third frame, but a new instance of the cat appeared, and we added the cat's embedding to the external memory bank $B_{dog}$ of the dog. Due to the low similarity between the instance embeddings of cats and dogs, it will have little impact on the MA embedding of further dogs that appear in the following frames. As shown in the video scene on the right of Figure~\ref{fig:video}, the wrong instances are corrected to original trajectories through noise training.

\vspace{-3mm}
\subsection{Learning from Sparse Annotation}
\label{sec:pseudo}

% Figure environment removed 

We now elaborate on our pseudo-video and mask generation technique, which enables the training of VIS models when only sparse annotations (\eg image data) are available. We take a few widely applied image-augmentation methods, including \emph{random rotation}, \emph{random crop} and \emph{copy\&paste} on source image to create pseudo-videos and the associated instance masks. Note that the pseudo-videos are created by no means to approximate real ones. Instead, they are taken to mimic the movement of targets in reality. 

\noindent\textbf{Rotation.} 
As shown in the first row of Figure~\ref{fig:augs}, the rotation augmentation rotates the source images with several random angles (e.g., $ [-15, 15]$ ) to introduce subtle changes between frames of the pseudo-videos. 

\noindent\textbf{Crop.} 
The rotation augmentation cannot alter the shapes and magnitudes of instances. However, instances deform or/and enter/exit the visible field due to the movement introduced either by the target or the camera. To address this, we apply random crop augmentation to the image, which allows the generated videos to mimic the zooming in/out effect of the camera lens and the shifting of targets. The second and the third rows of Figure~\ref{fig:augs} present two examples of \emph{crop-zoom} and \emph{crop-shift}, respectively. The pseudo-videos generated by such augmentations cover a large proportion of targets' movements.

\noindent\textbf{Copy and Paste.} 
As mentioned earlier, the trajectories of instances in pseudo-videos created by the augmentations share the identical motion direction. To incorporate the relative motion between instances, we also employ the \emph{copy\&paste} augmentation\cite{copypaste}, which copies the instances from another image in the dataset and pastes them into  random locations within the source image. Note that the pasting positions of an instance are typically different across time, which brings the relative motion between different instances (as shown in the fourth row of Figure~\ref{fig:augs}).
% As shown in Figure~\ref{fig:augs}, this operation brings relative between several instances.

% \noindent\textbf{Merge All.} Suppose we want to generate a pseudo-video of T frames. Given an input image $I_{des}$, we randomly select another image from datasets as $I_{src}$. Then we parallelly make the copy\&paste $T$ times each of which copy\&pastes the instances of $I_{src}$ into the $I_{des}$. We get the output as N

% which selects two images, ${img_k}$ and ${img_r}$, from the dataset by random, and applies the aforementioned augmentations to them simultaneously. Subsequently, a subset of the pixel values of the instance from ${img_r}$, exceeding a certain threshold, is selected and pasted onto ${img_k}$. Finally, the necessary modifications to the ground-truth annotation are made, the fully occluded object is removed, and the mask of the partially occluded mask and bounding box is updated. As shown in the fourth row of Figure~\ref{fig:augs}, the copy\&paste augmentation\cite{copypaste}

\section{Results}
\label{sec:results}

\subsection{Pareto-optimal YOLO models}

By computing the proxy metric for model accuracy (mAP$_{50-95}$ in VOC training from scratch) and latency values for the whole \textit{YOLOBench} architecture space on several hardware platforms, we have determined the Pareto sets containing the most promising models (in terms of latency-accuracy trade-off) for each HW platform. Merging the first and second Pareto sets for each device into a single list of best architectures, we arrived at a set of 52 backbone/neck combinations for COCO pre-training (same architectures with different input resolutions are considered as the same data points here, since COCO training is regardless done on a fixed resolution of 640x640). After performing the COCO training for these selected models, we fine-tune these models at 11 different resolutions (from 160x160 to 480x480 with a step of 32) on all downstream datasets (except for COCO), resulting in 572 models total for each dataset.

% Figure environment removed


Finally, given fine-tuned model accuracy on several datasets and latency measurements on several devices, we compute the actual Pareto sets for each particular dataset/HW platform combination. Figure \ref{fig:voc_pareto} shows the Pareto frontiers of \textit{YOLOBench} models fine-tuned on the VOC dataset on 4 different devices. One could observe significant differences in the Pareto frontiers between devices. In particular, the Pareto-optimal set for VIM3 NPU is mostly comprised of YOLOv6 models, with some YOLOv5, YOLOv7, and YOLOv8 models present in the higher accuracy region. This is not the case for Pareto sets of Intel and ARM CPUs that, despite containing a few YOLOv6 models in the lower latency region, also contain many YOLOv5 and YOLOv7 variations in the higher accuracy region (with some other models families like YOLOv3 and YOLOv4 also present in a limited capacity). While the Pareto sets for Intel and ARM CPUs are found to be similar to each other, the Pareto set on Jetson Nano GPU is different from the rest of the devices and is very non-uniform in terms of model family distribution, with YOLOv5, YOLOv6, YOLOv7 and YOLOv8 models all represented across the whole accuracy/latency range. Table \ref{tab:pareto_table} shows representative Pareto-optimal models for 3 different datasets (VOC, SKU-110k, WIDERFACE) and 3 hardware platforms under certain latency thresholds. Note that although there are similarities of model family distributions in Pareto sets computed for different datasets (see Appendix B), the exact optimal model for a given latency threshold depends on the specific dataset of interest.

Next, we looked at statistics of Pareto-optimal models depending on the dataset and hardware platform. Figure \ref{fig:pareto_scaling} shows the distribution of depth factor, width factor, and input resolution values in Pareto frontier models for VOC and SKU-110k datasets on Jetson Nano GPU (data for other datasets and devices available in Appendix B). The general trend observed is that models at lower input resolutions mostly have lower depth and width factors. This means that to achieve optimal results in terms of latency-accuracy trade-off, one has to scale down the width and depth of the architecture before lowering the input resolution. This effect is more pronounced in some datasets (SKU-110k and WIDERFACE), where almost all optimal models are either at the maximal resolution we considered (480x480) with variation in width and depth, or at lower resolutions with minimal width and depth factors. This effect is dataset-dependent, as it is observed to be more relaxed for VOC and COCO datasets, where many optimal models with a variation in width and depth factor are found at resolutions lower than 480x480.

To summarize, we have demonstrated that with a state-of-the-art training pipeline and detection head structure, YOLO-based models with various backbone/neck combinations could achieve good latency-accuracy trade-off in various deployment scenarios, including older backbone/neck structures from YOLOv4 and YOLOv3 models. Furthermore, we have shown that depth/width reduction precedes input resolution down-scaling in optimal YOLO-based detectors.

\subsection{Ranking training-free accuracy predictors}
\label{sec:ranking_zc}


With an increasing number of architecture blocks and hyperparameter combinations, the size of the candidate model space in \textit{YOLOBench} can further grow exponentially. Hence, it is important to develop efficient methods of filtering out bad architecture proposals before running them through the full training pipeline, including pre-training on the COCO dataset. In the field of neural architecture search, recent works have proposed a handful of training-free, {\em zero-cost} (ZC) estimators that have been shown to perform well on various (relatively simple) benchmarks \cite{mellor2021neural, abdelfattah2021zero, li2023zico}.

Zero-cost estimators were originally proposed by Mellor et al. \cite{mellor2021neural}, and later expanded by Abdelfattah et al. \cite{abdelfattah2021zero} as a means to quickly evaluate the performance of an architecture using only a mini-batch of data. These estimators work by extracting statistics obtained from a forward (and/or backward) pass of a few mini-batches of data through the network, hence eliminating the need for full training of the model. Despite the fact that over 20 different zero-cost accuracy estimators have been introduced in recent years, simple baselines like the number of parameters and MAC count are still found to be hard to outperform \cite{li2023zico}.

The vast search space of YOLO-like architectures necessitates the development of effective training-free estimators to filter out bad candidates and reduce the search space. We have examined the performance of a representative subset of zero-cost estimators on \textit{YOLOBench}, namely: 
Fisher~\cite{turner2019blockswap}, GradNorm~\cite{abdelfattah2021zero}, GraSP~\cite{wang2020picking}, JacobCov~\cite{abdelfattah2021zero}, Plain~\cite{abdelfattah2021zero}, SNIP~\cite{lee2018snip}, SynFlow~\cite{tanaka2020pruning}, ZiCo~\cite{li2023zico}, Zen-score~\cite{lin2021zen} and NWOT~\cite{mellor2021neural}. The NWOT metric is computed by measuring the Hamming distance between binary codes produced by each layer's activations \cite{mellor2021neural}. Although originally proposed for ReLU-based networks, we found that it works well in practice for YOLO variations, most of which contain SiLU activations. The NWOT metric can be also computed by looking at signs of each layer's output features before the activation layer to form the binary code. We refer to that version of the NWOT metric as  \textit{NWOT (pre-act)} ("pre-activation"), and find that its performance might differ significantly from the original NWOT metric, primarily because the binary codes are computed before the normalization layers followed by the activations. We also compare the performance of the above metrics with simple baselines such as the number of trainable parameters and MAC count, as well as with a training-based proxy that we have used for pre-select models for \textit{YOLOBench} (mAP$_{50-95}$ of training from scratch on the VOC dataset). 

All zero-cost metrics are computed on randomly initialized models using the same loss function as used for training of all \textit{YOLOBench} models, using a single mini-batch of data with a corresponding image resolution (except for ZiCo, which requires two different mini-batches of data). We empirically evaluate the considered set of zero-cost proxies on \textit{YOLOBench} using the following metrics:

\begin{itemize}
    \item Kendall $\tau$ (global): Kendall rank correlation coefficient evaluated on all \textit{YOLOBench} models
    \item Kendall $\tau$ (top-15\%): Kendall rank correlation coefficient evaluated on the top-15\% performing \textit{YOLOBench} models (in terms of mAP$_{50-95}$ value)
    \item Percentage of all actual Pareto-optimal models in the Pareto set determined with the zero-cost estimator in the zero-cost proxy-latency space (recall for Pareto-optimal model prediction using the ZC-based Pareto set)
\end{itemize}

The last metric effectively measures how accurate the computed Pareto set would be if the proxy values are used instead of actual mAP to rank models. It is calculated by determining Pareto fronts for model rankings based on zero-cost proxies (and real latency measurements) and looking at how many models present in the actual Pareto set are also present in the ZC-based Pareto set. In other words, a recall value of 0.7 would mean that by taking the models from the ZC-based Pareto set as candidates, we find 70\% of all actual Pareto-optimal models in that candidate set. We report values for Pareto fronts computed with latency measurements on the Jetson Nano GPU in Table \ref{tab:zc_table}.

We generally found that all of the zero-cost predictors we considered (except for NWOT) were outperformed by the simple baseline of MAC count both in terms of Kendall-Tau scores as well as in the percentage of predicted Pareto-optimal models (see Table \ref{tab:zc_table}). Furthermore, when compared with using mAP$_{50-95}$ on VOC training from scratch as a predictor, we observed that only NWOT came close to it in terms of ranking scores. We have also found that a pre-activation version of NWOT tends to work better than standard NWOT on \textit{YOLOBench}. For the task of predicting mAP$_{50-95}$ of models fine-tuned on SKU-110k, we notably found that pre-activation NWOT outperforms VOC training from scratch metric in terms of Kendall-Tau scores (possibly due to domain difference between VOC and SKU-110k datasets), but the VOC-based proxy metric still performs better for Pareto-optimal model prediction on SKU-110k.

\begin{table}
  \small
    \caption{COCO mAP and inference latency on Raspberry Pi 4 CPU (TFLite, FP32) for YOLOv8s vs. a model identified from the NWOT-latency Pareto frontier. Mean and standard deviation of inference time over 5 runs (each run done for 100 iterations) shown, with 640x640 input resolution.} %
    \label{tab:timm_coco}
    \vspace*{2mm}
    \begin{tabularx}{\columnwidth}{X|X|X} %
      \hline
      {Model} & {COCO mAP$^{val}_{50-95}$} & {Latency, ms}\\
      \hline
      {YOLOv8s} & {0.4364} & {1476.09 $\pm$ 1.49} \\
      \hline
      {YOLOv8s (HSwish)} & {0.4355} & {1381.62 $\pm$ 7.34} \\
      \hline
      {YOLO-FBNetV3-D-PAN-C3} & {0.4463} & {1355.21 $\pm$ 9.93} \\
      \hline
    \end{tabularx}
\end{table}


In trying to capture all the real Pareto-optimal models using ZC scores, one could increase the size of the ZC-based candidate pool by computing the second (third, fourth) ZC-based Pareto set and adding it to the pool of ZC-based candidates. Such a procedure naturally increases the percentage of actual Pareto-optimal models in that pool, and the full set of actual Pareto-optimal models could be found this way by looking only at a portion of the full dataset (e.g. at first $N$ ZC-based Pareto fronts). In this context, we compute candidate pools consisting of $N$ ZC Pareto fronts for each ZC metric and look at the percentage of actual Pareto-optimal models found in the pool versus the pool size (as \% of the full dataset size). Looking at the pool size is motivated by the observation that the number of models in ZC-based Pareto fronts can significantly vary depending on the ZC metric.

Figure \ref{fig:zc_pareto_voc} shows the percentage of predicted real Pareto-optimal models on the VOC dataset contained in pools of $N$ first Pareto fronts for 4 different predictors (VOC training from scratch, NWOT, pre-activation NWOT and MAC count). For ARM and Intel CPUs, we observe a general trend of VOC training from scratch being the best predictor and MAC count being the worst at all points. Interestingly, for Jetson Nano GPU NWOT performs close to VOC training from scratch for $N = 1,2$ but starts to perform worse with more models in the pool. Surprisingly, the training-free predictors MAC count and pre-activation NWOT outperform VOC training from scratch in predicting Pareto-optimal models on VIM3 NPU.

\subsection{Pareto-optimal detector identification using NWOT score}

To demonstrate the potential of using ZC-based Pareto sets in identifying promising detector architectures with good accuracy-latency trade-off, we have additionally generated multiple candidate architectures based on CNN backbones provided by the \texttt{timm} library \cite{rw2019timm}. The architectures were generated by using one of the 347 CNN-based backbones available in \texttt{timm} as a feature extractor followed by a modified Path Aggregation Network (PAN) (same structure with C3 blocks as in YOLOv5 was used, with the number of channels corresponding to YOLOv5s, without the SPPF block) and a YOLOv8 detection head, as in all other \textit{YOLOBench} models. 

We computed the pre-activation NWOT scores as well as measured inference latency on Raspberry Pi 4 ARM CPU with TFLite for all candidate models. We then used the NWOT score and latency values for each model to compute the Pareto frontier in the NWOT-latency space (see Appendix D). We then trained one of the models identified to belong to the NWOT-based Pareto frontier (YOLO with FBNetV3-D backbone) on the COCO dataset with the same setup used to pre-train \textit{YOLOBench} models (640x640 input resolution, 300 epochs, batch size 64, other hyperparameters set to default of Ultralytics YOLOv8 \cite{ultralytics})\footnote{Note that YOLOv8s results provided by Ultralytics \cite{ultralytics} are higher than the ones we have reported, as the models were trained for 500+ epochs on COCO. However, no script to reproduce those results has been released to date.}. The resulting model was found to be more accurate and faster than YOLOv8s (a model in a similar latency range) when tested on Raspberry Pi 4 CPU with TFLite (FP32, XNNPACK backend) (see Table \ref{tab:timm_coco}). Furthermore, we have looked at the accuracy and latency of a YOLOv8s modification with SiLU activations replaced with Hardswish activations (Table \ref{tab:timm_coco}), as we have observed the choice of activation function to be a significant factor affecting TFLite inference latency. We found that the identified NWOT-Pareto model (also containing Hardswish activations in the backbone, neck and head) still outperformed YOLOv8s-HSwish in terms of latency and accuracy.

\section{Discussion}
\label{Sec:Discussion}

% \note{note after submission: we might need/want to discuss more about the meanings and reasons behind the results for individual measures.}

In this section, we summarize our findings based on the results reported in the previous section while connecting to our established hypotheses.
We also discuss the justifications and implications of our findings and associate them with other prior findings in the literature.

\subsection{Digital Tools Improved Learning Experience}
% \note{emphasize the effects on the learning experience}

We found several significant effects of the learning tool for some of our \expMeasure measures, which we reported in Section~\ref{Sec:UX_Results}.
In general, the results show that the Tablet-3D and the Screen-AR could provide more positive experience scores than the Textbook, e.g., for the measures of ``enjoyment,'' ``willingness to recommend,'' and ``learning motivation,'' which partly supports our \textbf{H1}.
These results align with previous literature findings that showed 3D visualization technologies increased students' engagement in human anatomy learning~\citep{hackett_effect_2018,luursema_role_2008,yammine_meta-analysis_2015}.
% % {(Hackett \& Proctor, 2018; Luursema et al., 2008; Yammine \& Violato, 2015)}.
The positive scores in some of our \expMeasure measures could be related to the novelty of the VR/AR technology as an anatomy learning tool, and the intuitive and interactive 3D models could also be an important factor in influencing the perception of the learning experience.
Some qualitative comments from the participants in the Tablet-3D condition or the Screen-AR condition after the experiment session also support our reasoning for the positive scores. For example, some of the participants in Tablet-3D or Screen-AR said:
% were also aligned with this positive trends in the learning experience measures.
% \vspace{-1em}
\begin{quote}
\leftskip=-10pt
\rightskip=-10pt
% textbook: ``Using technology was far easier than using the textbook.''\\
% tablet: ``Interface could be refined a little but the model was great.''}\\
% tablet: ``It's a great tool!''\\
% AR: ``I really loved the activity!''\\
% AR: ``it was an enjoyable and educational experience.''\\
% AR: ``super cool to be the model and see my body's muscles.''
% textbook: ``It was fun to paint a muscle and helped put it into the context of reality instead of simply seeing it on a page.''
\textit{``Interface could be refined a little but the model was great.''}\\
\textit{``It was an enjoyable and educational experience.''}\\
\textit{``Super cool to be the model and see my body's muscles.''}\\
\textit{``It was fun to paint a muscle and helped put it into the context of reality instead of simply seeing it on a page.''}
\end{quote}
% enjoyed the muscle painting activity 
% enjoyment, willingness to recommend, and learning motivation

% easy to paint: tablet > textbook, screen-AR
Interestingly, however, we found no significant benefits of the Screen-AR compared to the Tablet-3D, which we expected in our \textbf{H2}.
Instead, we found a higher score in the Tablet-3D condition than the Screen-AR condition for the ``easy to paint'' measure.
Based on the participants' feedback below, we realized that some participants complained about the intermittent misalignment of virtual contents in the Screen-AR condition.
\begin{quote}
\leftskip=-10pt
\rightskip=-10pt
% ``The calibration for the magic mirror was not great and so it was hard to actually see the muscle projected onto myself and my partner. The diagram on the right of the mirror was much more helpful and it's what allowed us to actually complete the activity.''
% ``The program kept not focusing on the right subject, and sometimes it was hard to see which muscle was being pointed at.''
% ``The model doesn't line up very well with our actual bodies.''
% ``It was a little hard to distinguish where in the muscle was sometimes because the overlay wasn't actually exact/not proportional to my body.''
\textit{``The calibration (of the AR tool) was not great and so it was hard to actually see the muscle projected onto myself and my partner.}\\
\textit{``Sometimes it was hard to see which muscle was being pointed at.''}\\
\textit{``It was a little hard to distinguish where in the muscle was sometimes because the overlay wasn't actually exact/not proportional to my body.''}
\end{quote}
\noindent This implies that the participants were quite sensitive to the accuracy and reliability of body tracking for AR content registration, which could significantly influence the quality of the learning experience.
Some major contributors to this accuracy issue in the AR tool include the 3D virtual model itself (which followed a male anatomy limb proportions rather than female), and the physical proximity of the painter and paintee in front of the Kinect body tracking system.  

% \begin{itemize}
%     \item \textbf{H1}: The participants in the Tablet-VR condition or the Screen-AR condition will have more positive user experience than those in the Textbook condition for all the user experience measures. (Textbook $<$ Tablet-VR, Screen-AR)
%     \item \textbf{H2}: The participants in the Screen-AR condition will further have more positive user experience than those in the Tablet-VR condition for all the user experience measures. (Tablet-VR $<$ Screen-AR)
% \end{itemize}

%%%%%%%%%%%%%%%%%%%%%%%%%%%%%%%%%%%%%%%%%%%%%%%%%%%%%%%%%%%%%%%%%%%%%% 

\subsection{No Influence on Learning Performance}
% \note{need for further long-term research to reveal the effects on the knowledge retention}

We expected positive effects of our digital tools on learning performance based on prior research~\citep{maresky_virtual_2019,nicholson_can_2006}.
As we reported in Section~\ref{Sec:Performance_Results}, there were no significant effects of the learning tool or the participant's gender for the learning performance, e.g., the anatomy knowledge retention, which means no evidence to support our \textbf{H4} and \textbf{H5}.
However, prior research showed that positive experience could increase learning performance and retention~\citet{maresky_virtual_2019,nicholson_can_2006}, and our proposed 3D visualization and AR learning tools did promote more positive \expMeasure compared to the traditional textbook-based learning.
In that sense, we are encouraged to conduct further research on different learning analytic metrics.
Additionally, the intervention that the participants had in our study was only a one-time session for about 10--15 minutes, which could not be enough to reveal the effects of AR experience on learning performance.
As most learning science studies require long-term data collection and analysis, a multi-session long-term intervention with our proposed digital tools should be considered for further research in the future.

% \begin{itemize}
%     \item \textbf{H4}: The participants in the Tablet-VR condition or the Screen-AR condition will have a higher score gain than those in the Textbook condition. (Textbook $<$ Tablet-VR, Screen-AR)
%     \item \textbf{H5}: The participants in the Screen-AR condition will further have a higher score gain than those in the Tablet-VR condition. (Tablet-VR $<$ Screen-AR)
% \end{itemize}

% As shown in \note{Table 1} and \note{Figure 2}, the mean value and median of score gain for individuals and teams across the conditions are very similar. 
% Textbook has a slightly higher mean of score gain for individuals. 
% However, among teams, students who used AR have better scores.
% No significant difference in score gain was observed among learning tools (F(2,298)=0.15, ns for individuals and F(2,135)=0.05, ns for teams), and all learning tools appear to have similar and comparable outcomes according to the knowledge retention---so the answer to RQ1 was no, there was no significant difference between conditions and score gains are in-distinguishable from each other.
% These findings are in agreement with previous studies in anatomy education which highlighted the potential of using evolving technologies such as MR and AR for enhancement of student learning and student outcomes in anatomical science education~\citep{maresky_virtual_2019,nicholson_can_2006}.
% % {(Maresky et al., 2019; Nicholson et al., 2006)}. 
% Our study also provides further evidence that 3D visualization technologies increase students' engagement and improve their knowledge retention in human anatomy learning~\citep{hackett_effect_2018,luursema_role_2008,yammine_meta-analysis_2015}.
% % {(Hackett \& Proctor, 2018; Luursema et al., 2008; Yammine \& Violato, 2015)}.


%%%%%%%%%%%%%%%%%%%%%%%%%%%%%%%%%%%%%%%%%%%%%%%%%%%%%%%%%%%%%%%%%%%%%% 


\subsection{Male Users Noted More Positive Experience}
% \note{different learning intervention and tools considering the gender difference?}

Beyond investigating the learning tool effects, we also established an interesting question about gender effects in anatomy education, as introduced in \textbf{H3} in Section~\ref{Sec:MeasuresHypotheses}.
We found significant effects of the participant gender on various \expMeasure measures: ``easy to find muscles,'' ``learning perception,'' ``enjoyment,'' ``willingness to recommend,'' and ``learning motivation.''
For all those measures, male participants reported higher (more positive) scores than female participants, which partly supports our \textbf{H3}.
 Considering gender composition, similar patterns were observed on the ``easy to find muscles,'' ``learning perception,'' and ``enjoyment'' measures. Usually, the difference between male pairs and mixed pairs, or male pairs and female pairs, was statistically significant, with male pairs reporting the highest scores, which partly supports our \textbf{H3}. 
 We should note that this effect was not associated with a specific learning tool.
% , but was found as general results including all types of learning tools.
There could be some aspects of our learning intervention that female participants did not like compared to male participants.
As we justified our hypothesis, the 3D model used in the learning tools was based on a male model, which was found to be an issue when calibrating and superimposing anatomical landmarks on the female participant models. Also, the female participants in our study might be more conservative/reluctant to the body painting activity than the males, or the males, on average, have a higher aptitude towards digital technologies than the females.
According to prior research~\citet{cookson2018exploration,finn2010qualitative}, unequal engagement of students in body painting activities may be due to cultural, social, and religious barriers concerning body image, nudity, gender, vulnerability, and embarrassment.
Not surprisingly, such tendencies were revealed in the after-session feedback from some of the female participants.
\begin{quote}
\leftskip=-10pt
\rightskip=-10pt
% ``(you should) take into account clothes and girls''
% ``I am not a fan of body painting. However, I love the program and think that it is really cool.''
 \textit{``Painting was unnecessary, but it was cool to see the muscles on my body.''}
 
% ``I really don't like the painting i dont like having paint on my skin that has touched like 100 other kids before me i would do it if i had clean paint brushes and my own paint but this is just an issue for me sorry i love the computer program though it is so cool, just not the paint.''
\textit{``[The activity should] take into account clothes and girls.''}\\
\textit{``I am not a fan of body painting. However, I love the program and think that it is really cool.'''}\\
\textit{``I really don't like the painting. I don't like having paint on my skin.''}
\end{quote}
\noindent These observations motivate and highlight the importance of learning interventions that should consider gender differences.
Novel learning tools could/should also be designed to mitigate the reluctance to the intervention with more pleasant virtual content or interaction mechanisms while considering the learners' profile.


% \begin{itemize}
% \item \textbf{H3}: The male participants in the study will report more positive user experience than the female participants, particularly in the measures of satisfaction, enjoyment, willingness to recommend. (Female $<$ Male)
% \end{itemize}

\subsection{Future Work}
The overarching goal of our study was to explore and understand the possible effects of using advanced learning technologies on anatomy learning and student performance.
Our discussion reveals that we require multi-session long-term studies to draw conclusive findings, including knowledge retention and task completion time.
In addition, the data set we analyzed in this paper covered the participants in teams of size two.
Hence, there could be various effects of team size in team-based learning interventions, and we plan to analyze our data with participants in larger teams while controlling for gender, age, and other demographic information, to form more holistic teams.
We also plan to expand student learning analysis by collecting and using multi-modal measurements for more effective and objective evaluations to complement self-reported information from questionnaires.
\section{Conclusions}
\label{Sec:Conclusions}

% In this paper, we introduced new learning technologies using 3D modeling visualizations for anatomy education and evaluated these tools in an experimental study with \note{301} students in \note{138} teams.
% We evaluated student knowledge retention, gender effects, and usability on individual and team levels.
% The results indicated that experimental teams who used new learning tools performed better for the knowledge tests than those in the control group. 
% The user satisfaction results showed that students who interacted with 3D models were more motivated and had better user experience than those who used textbook. 
% The relationship between user-study learning tools, knowledge retention, and gender effects on individuals and team performance were also investigated. 
% The findings from this work offer implications in anatomy education, learning technologies and learning analytics by providing novel instructional tools for teaching and learning anatomy.

% The focus of this study was to understand the effects of using advanced learning technologies on anatomy learning and student performance. 
% Similar to any studies some improvements are desired for future.
% The study time was bounded to 3-hour lab sessions, and some students needed to complete the task with time considerations, so in future, we will also plan to investigate impact of task completion time on knowledge retention and mitigate any potential confounding effects.
% Moreover, the data we collected was based on the student self-reported questionnaires and knowledge tests from one relatively large-scale anatomy intervention. 
% In future work, we aim to evaluate the effectiveness of adopting advanced technologies to other anatomy learning scenarios. 
% We also plan to expand student learning analytics by collecting and using multi-modal measurements for more effective, and objective evaluations to complement self-reported information from questionnaires.
%We developed a tablet-based 3D visualization application and a wide screen-based AR system that visualizes 3D anatomy models as our in-house learning tools, and

This paper investigated the effects of digital anatomy learning tools, including a detailed tablet-based visualizations of anatomical systems and our screen-based AR application capable of overlaying 3D anatomical structures on to the digital mirror image of students. 
We conducted a large-scale study with \textbf{236} students from healthcare-related courses and compared our digital learning tools with conventional textbook education.
Our results indicated that both tablet-based 3D visualizations and screen-based AR improved the students' \expMeasure, particularly 
about ``enjoyment'', ``willingness to recommend'', and ``learning motivation''. 
In addition, we found that male participants generally had higher (more positive) \expMeasure scores than female participants in our anatomy learning intervention. Our findings include further considerations on learners' gender on the \expMeasure and performance in anatomy education.

As short-term learning performance indicates, and the enjoyment of learning tasks with our digital learning tools has shown to be perceived positively, we are optimistic that the long-term retention of future novel 3D technology will follow a similar trend.  





%Although we did not find any significant benefits of our digital learning tools to increase anatomical knowledge retention as a learning performance measure, our findings offer implications in anatomy education, concerning learning technologies and learners' profiles, specifically gender, by providing novel instructional tools for teaching and learning anatomy.

% \note{future research analyzing the group size effects, etc.}



% \section{Acknowledgments}
% \note{TODO}
% Identification of funding sources and other support, and thanks to
% individuals and groups that assisted in the research and the
% preparation of the work should be included in an acknowledgment
% section, which is placed just before the reference section in your
% document.




%%
%% The acknowledgments section is defined using the "acks" environment
%% (and NOT an unnumbered section). This ensures the proper
%% identification of the section in the article metadata, and the
%% consistent spelling of the heading.

%%
%% The next two lines define the bibliography style to be used, and
%% the bibliography file.
\bibliographystyle{ACM-Reference-Format}
\balance
\bibliography{ref}

\end{document}
\endinput
%%
%% End of file `sample-manuscript.tex'.
