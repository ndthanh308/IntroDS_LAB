\section{Discussion}
\label{Sec:Discussion}

% \note{note after submission: we might need/want to discuss more about the meanings and reasons behind the results for individual measures.}

In this section, we summarize our findings based on the results reported in the previous section while connecting to our established hypotheses.
We also discuss the justifications and implications of our findings and associate them with other prior findings in the literature.

\subsection{Digital Tools Improved Learning Experience}
% \note{emphasize the effects on the learning experience}

We found several significant effects of the learning tool for some of our \expMeasure measures, which we reported in Section~\ref{Sec:UX_Results}.
In general, the results show that the Tablet-3D and the Screen-AR could provide more positive experience scores than the Textbook, e.g., for the measures of ``enjoyment,'' ``willingness to recommend,'' and ``learning motivation,'' which partly supports our \textbf{H1}.
These results align with previous literature findings that showed 3D visualization technologies increased students' engagement in human anatomy learning~\citep{hackett_effect_2018,luursema_role_2008,yammine_meta-analysis_2015}.
% % {(Hackett \& Proctor, 2018; Luursema et al., 2008; Yammine \& Violato, 2015)}.
The positive scores in some of our \expMeasure measures could be related to the novelty of the VR/AR technology as an anatomy learning tool, and the intuitive and interactive 3D models could also be an important factor in influencing the perception of the learning experience.
Some qualitative comments from the participants in the Tablet-3D condition or the Screen-AR condition after the experiment session also support our reasoning for the positive scores. For example, some of the participants in Tablet-3D or Screen-AR said:
% were also aligned with this positive trends in the learning experience measures.
% \vspace{-1em}
\begin{quote}
\leftskip=-10pt
\rightskip=-10pt
% textbook: ``Using technology was far easier than using the textbook.''\\
% tablet: ``Interface could be refined a little but the model was great.''}\\
% tablet: ``It's a great tool!''\\
% AR: ``I really loved the activity!''\\
% AR: ``it was an enjoyable and educational experience.''\\
% AR: ``super cool to be the model and see my body's muscles.''
% textbook: ``It was fun to paint a muscle and helped put it into the context of reality instead of simply seeing it on a page.''
\textit{``Interface could be refined a little but the model was great.''}\\
\textit{``It was an enjoyable and educational experience.''}\\
\textit{``Super cool to be the model and see my body's muscles.''}\\
\textit{``It was fun to paint a muscle and helped put it into the context of reality instead of simply seeing it on a page.''}
\end{quote}
% enjoyed the muscle painting activity 
% enjoyment, willingness to recommend, and learning motivation

% easy to paint: tablet > textbook, screen-AR
Interestingly, however, we found no significant benefits of the Screen-AR compared to the Tablet-3D, which we expected in our \textbf{H2}.
Instead, we found a higher score in the Tablet-3D condition than the Screen-AR condition for the ``easy to paint'' measure.
Based on the participants' feedback below, we realized that some participants complained about the intermittent misalignment of virtual contents in the Screen-AR condition.
\begin{quote}
\leftskip=-10pt
\rightskip=-10pt
% ``The calibration for the magic mirror was not great and so it was hard to actually see the muscle projected onto myself and my partner. The diagram on the right of the mirror was much more helpful and it's what allowed us to actually complete the activity.''
% ``The program kept not focusing on the right subject, and sometimes it was hard to see which muscle was being pointed at.''
% ``The model doesn't line up very well with our actual bodies.''
% ``It was a little hard to distinguish where in the muscle was sometimes because the overlay wasn't actually exact/not proportional to my body.''
\textit{``The calibration (of the AR tool) was not great and so it was hard to actually see the muscle projected onto myself and my partner.}\\
\textit{``Sometimes it was hard to see which muscle was being pointed at.''}\\
\textit{``It was a little hard to distinguish where in the muscle was sometimes because the overlay wasn't actually exact/not proportional to my body.''}
\end{quote}
\noindent This implies that the participants were quite sensitive to the accuracy and reliability of body tracking for AR content registration, which could significantly influence the quality of the learning experience.
Some major contributors to this accuracy issue in the AR tool include the 3D virtual model itself (which followed a male anatomy limb proportions rather than female), and the physical proximity of the painter and paintee in front of the Kinect body tracking system.  

% \begin{itemize}
%     \item \textbf{H1}: The participants in the Tablet-VR condition or the Screen-AR condition will have more positive user experience than those in the Textbook condition for all the user experience measures. (Textbook $<$ Tablet-VR, Screen-AR)
%     \item \textbf{H2}: The participants in the Screen-AR condition will further have more positive user experience than those in the Tablet-VR condition for all the user experience measures. (Tablet-VR $<$ Screen-AR)
% \end{itemize}

%%%%%%%%%%%%%%%%%%%%%%%%%%%%%%%%%%%%%%%%%%%%%%%%%%%%%%%%%%%%%%%%%%%%%% 

\subsection{No Influence on Learning Performance}
% \note{need for further long-term research to reveal the effects on the knowledge retention}

We expected positive effects of our digital tools on learning performance based on prior research~\citep{maresky_virtual_2019,nicholson_can_2006}.
As we reported in Section~\ref{Sec:Performance_Results}, there were no significant effects of the learning tool or the participant's gender for the learning performance, e.g., the anatomy knowledge retention, which means no evidence to support our \textbf{H4} and \textbf{H5}.
However, prior research showed that positive experience could increase learning performance and retention~\citet{maresky_virtual_2019,nicholson_can_2006}, and our proposed 3D visualization and AR learning tools did promote more positive \expMeasure compared to the traditional textbook-based learning.
In that sense, we are encouraged to conduct further research on different learning analytic metrics.
Additionally, the intervention that the participants had in our study was only a one-time session for about 10--15 minutes, which could not be enough to reveal the effects of AR experience on learning performance.
As most learning science studies require long-term data collection and analysis, a multi-session long-term intervention with our proposed digital tools should be considered for further research in the future.

% \begin{itemize}
%     \item \textbf{H4}: The participants in the Tablet-VR condition or the Screen-AR condition will have a higher score gain than those in the Textbook condition. (Textbook $<$ Tablet-VR, Screen-AR)
%     \item \textbf{H5}: The participants in the Screen-AR condition will further have a higher score gain than those in the Tablet-VR condition. (Tablet-VR $<$ Screen-AR)
% \end{itemize}

% As shown in \note{Table 1} and \note{Figure 2}, the mean value and median of score gain for individuals and teams across the conditions are very similar. 
% Textbook has a slightly higher mean of score gain for individuals. 
% However, among teams, students who used AR have better scores.
% No significant difference in score gain was observed among learning tools (F(2,298)=0.15, ns for individuals and F(2,135)=0.05, ns for teams), and all learning tools appear to have similar and comparable outcomes according to the knowledge retention---so the answer to RQ1 was no, there was no significant difference between conditions and score gains are in-distinguishable from each other.
% These findings are in agreement with previous studies in anatomy education which highlighted the potential of using evolving technologies such as MR and AR for enhancement of student learning and student outcomes in anatomical science education~\citep{maresky_virtual_2019,nicholson_can_2006}.
% % {(Maresky et al., 2019; Nicholson et al., 2006)}. 
% Our study also provides further evidence that 3D visualization technologies increase students' engagement and improve their knowledge retention in human anatomy learning~\citep{hackett_effect_2018,luursema_role_2008,yammine_meta-analysis_2015}.
% % {(Hackett \& Proctor, 2018; Luursema et al., 2008; Yammine \& Violato, 2015)}.


%%%%%%%%%%%%%%%%%%%%%%%%%%%%%%%%%%%%%%%%%%%%%%%%%%%%%%%%%%%%%%%%%%%%%% 


\subsection{Male Users Noted More Positive Experience}
% \note{different learning intervention and tools considering the gender difference?}

Beyond investigating the learning tool effects, we also established an interesting question about gender effects in anatomy education, as introduced in \textbf{H3} in Section~\ref{Sec:MeasuresHypotheses}.
We found significant effects of the participant gender on various \expMeasure measures: ``easy to find muscles,'' ``learning perception,'' ``enjoyment,'' ``willingness to recommend,'' and ``learning motivation.''
For all those measures, male participants reported higher (more positive) scores than female participants, which partly supports our \textbf{H3}.
 Considering gender composition, similar patterns were observed on the ``easy to find muscles,'' ``learning perception,'' and ``enjoyment'' measures. Usually, the difference between male pairs and mixed pairs, or male pairs and female pairs, was statistically significant, with male pairs reporting the highest scores, which partly supports our \textbf{H3}. 
 We should note that this effect was not associated with a specific learning tool.
% , but was found as general results including all types of learning tools.
There could be some aspects of our learning intervention that female participants did not like compared to male participants.
As we justified our hypothesis, the 3D model used in the learning tools was based on a male model, which was found to be an issue when calibrating and superimposing anatomical landmarks on the female participant models. Also, the female participants in our study might be more conservative/reluctant to the body painting activity than the males, or the males, on average, have a higher aptitude towards digital technologies than the females.
According to prior research~\citet{cookson2018exploration,finn2010qualitative}, unequal engagement of students in body painting activities may be due to cultural, social, and religious barriers concerning body image, nudity, gender, vulnerability, and embarrassment.
Not surprisingly, such tendencies were revealed in the after-session feedback from some of the female participants.
\begin{quote}
\leftskip=-10pt
\rightskip=-10pt
% ``(you should) take into account clothes and girls''
% ``I am not a fan of body painting. However, I love the program and think that it is really cool.''
 \textit{``Painting was unnecessary, but it was cool to see the muscles on my body.''}
 
% ``I really don't like the painting i dont like having paint on my skin that has touched like 100 other kids before me i would do it if i had clean paint brushes and my own paint but this is just an issue for me sorry i love the computer program though it is so cool, just not the paint.''
\textit{``[The activity should] take into account clothes and girls.''}\\
\textit{``I am not a fan of body painting. However, I love the program and think that it is really cool.'''}\\
\textit{``I really don't like the painting. I don't like having paint on my skin.''}
\end{quote}
\noindent These observations motivate and highlight the importance of learning interventions that should consider gender differences.
Novel learning tools could/should also be designed to mitigate the reluctance to the intervention with more pleasant virtual content or interaction mechanisms while considering the learners' profile.


% \begin{itemize}
% \item \textbf{H3}: The male participants in the study will report more positive user experience than the female participants, particularly in the measures of satisfaction, enjoyment, willingness to recommend. (Female $<$ Male)
% \end{itemize}

\subsection{Future Work}
The overarching goal of our study was to explore and understand the possible effects of using advanced learning technologies on anatomy learning and student performance.
Our discussion reveals that we require multi-session long-term studies to draw conclusive findings, including knowledge retention and task completion time.
In addition, the data set we analyzed in this paper covered the participants in teams of size two.
Hence, there could be various effects of team size in team-based learning interventions, and we plan to analyze our data with participants in larger teams while controlling for gender, age, and other demographic information, to form more holistic teams.
We also plan to expand student learning analysis by collecting and using multi-modal measurements for more effective and objective evaluations to complement self-reported information from questionnaires.