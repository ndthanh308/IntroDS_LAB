\section{Conclusions}
\label{Sec:Conclusions}

% In this paper, we introduced new learning technologies using 3D modeling visualizations for anatomy education and evaluated these tools in an experimental study with \note{301} students in \note{138} teams.
% We evaluated student knowledge retention, gender effects, and usability on individual and team levels.
% The results indicated that experimental teams who used new learning tools performed better for the knowledge tests than those in the control group. 
% The user satisfaction results showed that students who interacted with 3D models were more motivated and had better user experience than those who used textbook. 
% The relationship between user-study learning tools, knowledge retention, and gender effects on individuals and team performance were also investigated. 
% The findings from this work offer implications in anatomy education, learning technologies and learning analytics by providing novel instructional tools for teaching and learning anatomy.

% The focus of this study was to understand the effects of using advanced learning technologies on anatomy learning and student performance. 
% Similar to any studies some improvements are desired for future.
% The study time was bounded to 3-hour lab sessions, and some students needed to complete the task with time considerations, so in future, we will also plan to investigate impact of task completion time on knowledge retention and mitigate any potential confounding effects.
% Moreover, the data we collected was based on the student self-reported questionnaires and knowledge tests from one relatively large-scale anatomy intervention. 
% In future work, we aim to evaluate the effectiveness of adopting advanced technologies to other anatomy learning scenarios. 
% We also plan to expand student learning analytics by collecting and using multi-modal measurements for more effective, and objective evaluations to complement self-reported information from questionnaires.
%We developed a tablet-based 3D visualization application and a wide screen-based AR system that visualizes 3D anatomy models as our in-house learning tools, and

This paper investigated the effects of digital anatomy learning tools, including a detailed tablet-based visualizations of anatomical systems and our screen-based AR application capable of overlaying 3D anatomical structures on to the digital mirror image of students. 
We conducted a large-scale study with \textbf{236} students from healthcare-related courses and compared our digital learning tools with conventional textbook education.
Our results indicated that both tablet-based 3D visualizations and screen-based AR improved the students' \expMeasure, particularly 
about ``enjoyment'', ``willingness to recommend'', and ``learning motivation''. 
In addition, we found that male participants generally had higher (more positive) \expMeasure scores than female participants in our anatomy learning intervention. Our findings include further considerations on learners' gender on the \expMeasure and performance in anatomy education.

As short-term learning performance indicates, and the enjoyment of learning tasks with our digital learning tools has shown to be perceived positively, we are optimistic that the long-term retention of future novel 3D technology will follow a similar trend.  





%Although we did not find any significant benefits of our digital learning tools to increase anatomical knowledge retention as a learning performance measure, our findings offer implications in anatomy education, concerning learning technologies and learners' profiles, specifically gender, by providing novel instructional tools for teaching and learning anatomy.

% \note{future research analyzing the group size effects, etc.}

