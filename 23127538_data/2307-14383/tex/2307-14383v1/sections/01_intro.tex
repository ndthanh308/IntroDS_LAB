% \firstsection{Introduction}
% \maketitle

\section{Introduction}
\label{Sec:Intro}

%\note{either introduce the terms (AR, VR) as a broader meaning of virtual model visualization or change the terms to mobile 3D model or something.}
%\note{clarify the potential contributions, which we are aware of the weakness already though.}

% \note{need to check the use of term, user experience or learning experience, later throughout the paper}

% Human anatomy and physiology is a vital part of medical education that involves complex functional structures and movements of the human body. 
% It provides a thorough understanding of human body function, enabling more effective treatment of abnormal or disease states.
% Anatomy and physiology are an extremely complex course, and traditional methods of training anatomy such as dissection or pro-section are not feasible nowadays due to limited resources, large class sizes, and mainly the absence of face-to-face learning experiences to name a few constraints. 
% The learning outcomes of these methods may affect by a number of factors such as quality of material, students' prior experience, and their emotional concerns.
% Moreover, learning dynamics of anatomical structures and spatial relationships to surrounding structures shows the importance of spatial visualization in anatomy education.
% However, two-dimensional (2D) views on the textbook do not fully represent the complexities of human anatomy, and students need to mentally manipulate three-dimensional (3D) relationships by themselves, which is a big challenge.
% Equally, most medical, dental, and other allied health schools have declined the practical laboratory hours~\citep{leung_anatomy_2006,winkelmann_anatomical_2007}.
% % {(Leung et al., 2006; Winkelmann, 2007)}.
% To adapt to the change, various complementary methods have been tried and changed over the view of anatomical education. 
% The explosion of advanced technologies during the last few decades has brought anatomical education to another level.

% Among advanced learning technologies, augmented reality (AR) has been used for enhancing learning experiences of student in medical training process by blending digital elements with physical learning environment~\citep{lovis_mixed_2020,marmulla_augmented_2005}
% % (Lovis, 2020; Marmulla et al., 2005; 
% \note{Romand et al., 2020)}.
% Especially in anatomy learning, computer-generated 3D models allow students to rotate and locate structures from various views and perspectives, and dynamic visualization techniques improve student visual-spatial ability without prerequisite~\citep{huk_who_2006,lipponen_challenges_1999,stieff_mental_2007}.
% % (Huk, 2006; Lipponen, 1999; Stieff, 2007).
% Compared with cadavers, those nonrealistic models on the 2D screen or virtual environment, are also easy to interact with and explore for the students without any dissection experience.
% In medical education domain, some interactive applications have been developed for increasing student self-directed learning skills and improve their learning performance~\citep{ang_gamifying_2018},
% % (Ang et al., 2018), 
% which have 3D virtual models built inside the mobile applications or AR/VR-based system.
% Despite the benefits that 3D model offers, AR technology provides the student a sense of connection between anatomic concepts and his/her own body~\note{[citep]}.

% \note{note after submission: we may need to check the term, ``learning experience,'' again}

% VRST Reviews
% • Definition of VR app using tablet is not suitable and misleading (R1, R2, R3): References to sensomotoric only availbable on HMD/CAVE, not on tablet. The reviewers have suggested many seminal papers on AR/VR tech.
% • Confounding factors in the different conditions (R1, R3): technology, interface, usability, the way information is presented differently in each condition.
% • Lacking details on the applications, measures, and procedures (R1, R2, R3), not relevant measures for some condition (R3). There was no strong rationale for the body painting condition (R3); self-designed measures used (R3).
% • Lack of contributions of study results (R1): obvious positive effects compared to conventional method; well researched advantage of technology based learning environment (R2); possible effects of novelty and short term memory rather than learning (R2, R3); unclear statements on gender effects (R3).

% \note{need to talk about ``Anatomical body painting is an effective and popular art-based approach to anatomy education.''C}

Human anatomy and physiology is a vital part of medical education that involves complex functional structures and movements of the human body. 
Comprehensive learning of anatomy and physiology provides a thorough understanding of human body function, enabling more effective treatment of abnormal or disease states~\cite{BLANCHARD200573}.
However, the complexity of the course poses challenges for students in achieving their desired learning outcomes.
Several factors associated with the learning experience can influence these outcomes, including the learning tools, the quality of the material, the students' prior experience, and their emotional concerns~\cite{o2008development, green2018relationship, chan2019approaches}.

The most common practice for students in anatomy education is to use textbooks with static images, but this cannot provide the students with a realistic first-hand and interactive experience, which may not be effective for their learning experience and performance~\cite{leung2020modernising}.
Despite the effectiveness of traditional methods for training anatomy, such as dissection or prosection, these methods have become less feasible nowadays due to limited resources, large class sizes, and mainly the absence of face-to-face learning experiences, to name a few constraints.
Due to such limitations, most medical, dental, and other allied health schools have recently declined the practical laboratory hours for anatomy~\citep{leung_anatomy_2006,winkelmann_anatomical_2007}.
%To address the change and issues, various complementary methods have been tried and changed over the view of anatomical education. The explosion of advanced technologies during the last few decades has brought anatomical education to another level, such as immersive learning tools.
In anatomy and physiology education, spatial visualization is likely essential for students to learn the dynamics of anatomical structures and spatial relationships to surrounding structures.
%Two-dimensional (2D) views in the textbook do not fully represent such dynamics and relationships with the complexities of human anatomy, and students need to mentally manipulate three-dimensional (3D) relationships by themselves, which is a big challenge.
The traditional visualization of human anatomy in two-dimensional (2D) textbook views insufficiently captures the complexity of human anatomy as students are often required to mentally reconstruct three-dimensional (3D) spatial relationships, which presents a considerable challenge.


Virtual/augmented reality (VR/AR) can provide information on dynamics and spatial relationships interactively and intuitively by employing 3D virtual skeletons and organs and adding a virtual information layer on top of the physical body.
While various medical training scenarios have used these technologies~\cite{lovis_mixed_2020,marmulla_augmented_2005,Romand2020}, the use of computer-generated 3D models, in particular, allows students to rotate and locate structures from various views and perspectives in anatomy learning. 
Such dynamic visualization techniques improve student visual-spatial abilities~\citep{huk_who_2006,lipponen_challenges_1999,stieff_mental_2007}.
%For those students without any dissection experiences, compared with cadavers, 3D models on the screen or virtual/augmented environments are so much easier to interact with and explore, too.
Moreover, virtual 3D visualizations offer significantly more accessible opportunities to engage and explore anatomical structures than traditional cadaver-based learning due to digital systems' repeatability and monitoring capabilities.


% In this paper, we investigate the use of modern technologies such as augmented reality as a replacement of textbook in an anatomy learning intervention using learning analytics.
% In a team-based experimental study with \note{301} students, we found that students who used our in-house augmented reality platform, and interactive app were as successful as those who used textbook; they are familiar learning instrument; in knowledge tests.
% Overall satisfaction of students on usability and engagement was also highlighted the potential of the proposed learning tools either as supplement or replacement of textbooks in the future of anatomy education.

%In this paper, we introduce and investigate the use of screen-based VR and AR technologies as a replacement for textbooks in an anatomy learning  intervention (muscle painting) using \expMeasure and analytic methods collected from premedical students. 
Body painting has been identified as an effective tool for learning anatomy and associated clinical skills ~\citep{cookson2018exploration,Diaz2021learning}. It is a motivating and creative experience for students that provides memorable visual images and encourages multisensory and active participation. 
While body painting suits all students, cultural sensitivity, gendered considerations, and careful negotiation may be necessary to ensure all students are comfortable carrying out the activities.

In this work, inspired by the best practices in anatomy education, we developed an intervention that uses 3D visualizations in a body painting learning activity.
We have investigated our in-house tablet-based 3D visualizations and screen-based AR with textbooks in a body painting anatomy learning intervention. We have reported our results on the interplay of participants' \expMeasure, performance outcomes with learning tools, gender, and team gender composition using statistical analysis.
%\note{either scope down the RQs to the level of tablet learning tool vs screen-AR, or add separate sentence(s) to clarify these are the general RQs while the scope of our experiment is much narrower.}
In a controlled study with three conditions of the conventional paper-based \emph{Textbook}, tablet-based 3D visualization (\emph{Tablet-3D}), and screen-based AR (\emph{Screen-AR}), our research aims to address the following research questions:
\begin{itemize}
    \item \textbf{RQ1:} Do tablet-based visualizations and AR technology improve students' learning experience compared to the traditional textbook in anatomy education?
    \item \textbf{RQ2:} Do tablet-based visualizations and AR technology increase the learning outcomes, e.g., test scores or knowledge retention, in anatomy education?
    \item \textbf{RQ3:} Are there any particular benefits of AR technology in anatomy education experience over tablet-based visualizations?
    \item \textbf{RQ4:} Is the student's gender a factor to interplay with the effects of digital technology, or in general for the learning experience?
\end{itemize}
% \begin{itemize}
%     \item \textbf{RQ1:} Is there a significant difference among student score gain based on the different learning tools? 
%     \item \textbf{RQ2:} Is there a difference in student knowledge retention (score gain) among males and females who took part in the activity? 
%     \item \textbf{RQ3:} Is there any meaningful pattern in the gender composition of teams and their score gain?
%     \item \textbf{RQ4:} Is there a difference among learning tools based on user experience questionnaire?
% \end{itemize}
To answer these research questions, we first introduce our tablet-based interactive visualization application and large screen-based AR tool, which can show dynamic anatomical information with interactive life-size 3D virtual models on top of a physical body.
We then report the findings from our large-scale team-based user study with 236 students (male: 95, female: 141) who experienced an anatomy learning intervention with a body painting activity.
We examined various \expMeasure measures and performances comparing the aforementioned three learning tools.
%: (1) textbook, (2) tablet-based visualization, and (3) screen-based AR.

We found that students who used our Tablet-3D and Screen-AR conditions had more positive (anatomy) learning experiences than those who used a textbook, according to their self-reported surveys.
For example, overall scores for satisfaction, learning perception, and perceived easiness were higher with the proposed digital learning tools than with the textbook.
Interestingly, the learning outcomes, e.g., test scores of anatomical knowledge, did not show any differences among the three learning tools, in some ways showing that all three conditions can be interchangeably used without compromising the knowledge representation.
In addition, we analyzed the potential effects of participant's gender on the \expMeasure and performance.
We found that male participants had higher scores in the learning experience measures than female participants.
We will describe the details of the study and discuss the implications of the results and findings.

% This paper is organized as follows. 
% First, we review the literature on the related work of anatomy education and learning analytics.
% We then introduce our proposed anatomy learning intervention.
% Afterwards, we explain the assessment tools used for measuring student achievement.
% Finally, we present findings from our user study and discuss further implications. 

%The remaining sections are organized as follows. 
%Section~\ref{Sec:RelatedWork} reviews the literature on the related work of anatomy education and learning analytics.
%Section~\ref{Sec:Method} introduces our proposed anatomy learning intervention and tools, and describe the conducted study with participant students.
%Section~\ref{Sec:Results} reports the results.
%Finally, we discuss the findings in Section~\ref{Sec:Discussion} before concluding the paper in Section~\ref{Sec:Conclusions}.

% Figure 2.  Gaze following: a task of following people’s gaze in a scene and inferring what they are looking at.

