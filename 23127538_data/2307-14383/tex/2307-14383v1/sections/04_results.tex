\section{Results}
\label{Sec:Results}

This section reports our analysis results considering the hypotheses we established in Section~\ref{Sec:MeasuresHypotheses}.
Since we are also interested in the possible effects of participant's gender on our measures, we have two factors to consider in our analysis: (1) learning tool and (2) participant's gender.
We first conducted two-way ANOVAs with these two factors to see if there is any interaction effect between the factors on both \expMeasure and performance measures.
We did not find any significant interactions between the learning tool and the participant gender; thus, we focused on the main effects of each factor using one-way ANOVAs ($\alpha\!=\!0.05$).
Multiple comparisons with Bonferroni correction were conducted for post-hoc tests.


% \subsection{User Engagement and Satisfaction}
% \subsection{User Experience}
\subsection{Learning Experience}
\label{Sec:UX_Results}

Here we report the results of our analysis on the \expMeasure measures for both learning tool, participant gender, and group gender composition factors.
The detailed results for each \expMeasure measure are described below, and the overviews of the learning tool effects and the gender effects are visualized in Figure~\ref{fig:learning_tool_effects} and Figure~\ref{fig:gender_effects}, respectively.

%\begin{itemize}
\paragraph{\textbf{Easy to Paint}} The one-way ANOVA for the measure of ``easy to paint'' showed a significant effect of the learning tool ($F(2, 233)\!=\!7.786$, \textbf{\itshape{p}\,$=$\,0.001}; $\eta_{\text{p}}^{2}\!=\!0.063$ - medium to large effect size). % _{\text{p}}^{2}\note{effect size}).
The post-hoc tests revealed that Tablet-3D ($M\!=\!4.67$, $SD\!=\!0.656$) had a higher score than Textbook ($M\!=\!4.26$, $SD\!=\!0.872$; \textbf{\itshape{p}\,$=$\,0.009}) or Screen-AR ($M\!=\!4.18$, $SD\!=\!1.016$; \textbf{\itshape{p}\,$=$\,0.001}).
This suggests that the participants felt the Tablet-3D condition was easier to perform the muscle painting activity than the other two conditions.
The analysis for the participant gender factor did not show any significant effect ($F(1, 234)\!=\!1.854$, $p\!=\!0.175$).


    
\paragraph{\textbf{Easy to Find Muscles}} For the ``easy to find muscles'' we did not find any significant effect of the learning tool.  
%nor the group gender composition ($F(2, 233)\!=\!2.81$, $p\!=\!0.062$).
% the p-values were close to the significance level. We took a further step and looked into mixed pairs vs. same-gender pairs. 
We found that the measure ``Easy to Find Muscles'' was signifcantly different among three group gender compositions ($F(2, 233)\!=\!5.62$, \textbf{\itshape{p}\,$=$\,0.004}; male pairs: $M\!=\!4.645$, $SD\!=\!0.630$; female pairs: $M\!=\!4.398$, $SD\!=\!0.875$; mixed pairs: $M\!=\!4.136$, $SD\!=\!1.006$).
Based on the post-hoc comparison with the Bonferroni test, the group with same gender compositions (males pairs and females pairs) was easier to perform the muscle painting activity than than the mixed pairs (\textbf{\itshape{p}\,$=$\,0.005}).
However, we found a significant effect of the participant's gender on this measure ($F(1, 234)\!=\!7.065$, \textbf{\itshape{p}\,$=$\,0.008}; $\eta_{\text{p}}^{2}\!=\!0.029$ - small to medium effect size).
The result showed that the male participants ($M\!=\!4.55$, $SD\!=\!0.632$) reported a significantly higher score for the easiness of finding muscles than the female participants ($M\!=\!4.26$, $SD\!=\!0.937$).

    
\paragraph{\textbf{Satisfaction}} We did not find any significant effects of both the learning tool and the participant gender for the measure of ``satisfaction'' (learning tool: $F(2, 233)\!=\!0.207$, $p\!=\!0.813$; gender: ($F(1, 234)\!=\!1.564$, $p\!=\!0.212$).
%\note{[ZG: STATA 15 17]}
However, the measure of ``satisfaction'' was significantly different in gender compositions ($F(2, 233)\!=\!3.91$, \textbf{\itshape{p}\,$=$\,0.021}). 
Among three different gender compositions (male pairs: $M\!=\!4.484$, $SD\!=\!0.671$; female pairs: $M\!=\!4.259$, $SD\!=\!0.921$; mixed pairs: $M\!=\!4.045$, $SD\!=\!0.999$), Bonferroni test showed that male pairs had higher satisfaction than mixed pairs (\textbf{\itshape{p}\,$=$\,0.017}), and no significant difference was observed based on the gender composition of the groups between male pairs and female pairs ($p\!=\!0.339$), nor between female pairs and mixed pairs ($p\!=\!0.372$).

    
\paragraph{\textbf{Learning Perception}} For the ``learning perception'' measure, we did not find a significant effect of the learning tool ($F(2, 233)\!=\!0.009$, $p\!=\!0.991$), but there was a main effect of the participant gender with a statistical significance ($F(1, 234)\!=\!6.245$, \textbf{\itshape{p}\,$=$\,0.013}; $\eta_{\text{p}}^{2}\!=\!0.026$ - small to medium effect size).
    This showed again that the male participants ($M\!=\!4.08$, $SD\!=\!0.834$) had a higher score than the female participants ($M\!=\!3.76$, $SD\!=\!1.062$).
%\note{[ZG: STATA 23]}
A significant difference based on the ``learning perception'' measure ($F(2, 233)\!=\!3.63$, \textbf{\itshape{p}\,$=$\,0.028}) was observed among gender compositions of male pairs ($M\!=\!4.177$, $SD\!=\!0.758$), female pairs ($M\!=\!3.778$, $SD\!=\!1.026$), and mixed pairs ($M\!=\!3.803$, $SD\!=\!1.084$).
Bonferroni test indicated that male pairs had a significantly higher score than female pairs (\textbf{\itshape{p}\,$=$\,0.034}).

    
\paragraph{\textbf{Enjoyment}} For the ``enjoyment'' we found main effects of the learning tool, the participant gender, and group gender composition factors.
For the learning tool ($F(2, 233)\!=\!9.870$, \textbf{\itshape{p}\,$<$\,0.001}; $\eta_{\text{p}}^{2}\!=\!0.078$ - medium to large effect size), we further compared the conditions and found that Textbook ($M\!=\!3.78$, $SD\!=\!1.051$) had a lower score than Tablet-3D ($M\!=\!4.32$, $SD\!=\!0.865$; \textbf{\itshape{p}\,$=$\,0.001}) and Screen-AR ($M\!=\!4.41$, $SD\!=\!0.897$; \textbf{\itshape{p}\,$<$\,0.001}).
For the gender effect ($F(1, 234)\!=\!10.008$, \textbf{\itshape{p}\,$=$\,0.002}; $\eta_{\text{p}}^{2}\!=\!0.041$ - small to medium effect size), the result showed that the male participants ($M\!=\!4.42$, $SD\!=\!0.807$) had a higher score than the female participants ($M\!=\!4.02$, $SD\!=\!1.038$).
%\note{[ZG: STATA 25]}
For the gender composition (male pairs: $M\!=\!4.468$, $SD\!=\!0.804$; female pairs: $M\!=\!4.037$, $SD\!=\!1.049$; mixed pairs: $M\!=\!4.152$, $SD\!=\!0.932$), a significant difference was observed ($F(2, 233)\!=\!4.03$, \textbf{\itshape{p}\,$=$\,0.019}).
We further compared the results with post-hoc comparisons, and the Bonferroni test showed that the male pairs had a significantly higher score than the female pairs (\textbf{\itshape{p}\,$=$\,0.016}).  

    
\paragraph{\textbf{Effort to Focus}} There were no significant effects found in the ``effort to focus'' measure (learning tool: $F(2, 233)\!=\!1.679$, $p\!=\!0.189$; gender: $F(1, 234)\!=\!0.598$, $p\!=\!0.440$; and 
%\note{[ZG: STATA 26]}
gender composition: $F(2, 233)\!=\!1.84$, $p\!=\!0.161$
).

    
\paragraph{\textbf{Lost Track of Time}} There were also no significant effects found in the ``lost track of time'' measure (learning tool: $F(2, 233)\!=\!0.332$, $p\!=\!0.718$; gender: $F(1, 234)\!=\!0.413$, $p\!=\!0.521$;
and %\note{[ZG: STATA 27]}
gender composition: $F(2, 233)\!=\!0.36$, $p\!=\!0.696$
).

    
\paragraph{\textbf{Willingness to Recommend}} We found a main effect of the learning tool for the ``willingness to recommend'' ($F(2, 233)\!=\!4.596$, \textbf{\itshape{p}\,$=$\,0.011}; $\eta_{\text{p}}^{2}\!=\!0.038$ - small to medium effect size).
The post-hoc tests revealed that Textbook ($M\!=\!6.96$, $SD\!=\!2.185$) had a lower score than Tablet-3D ($M\!=\!7.82$, $SD\!=\!2.003$; \textbf{\itshape{p}\,$=$\,0.024}) and Screen-AR ($M\!=\!7.83$, $SD\!=\!1.865$; \textbf{\itshape{p}\,$=$\,0.028}).
The gender effect was also revealed ($F(1, 234)\!=\!4.323$, \textbf{\itshape{p}\,$=$\,0.039}; $\eta_{\text{p}}^{2}\!=\!0.018$ - small to medium effect size), which showed that the male participants ($M\!=\!7.89$, $SD\!=\!1.825$) had a higher score than the female participants ($M\!=\!7.33$, $SD\!=\!2.164$).
%\note{[ZG: STATA 28]}
%On the contrary, no gender composition effect was found ($F(2, 233)\!=\!0.12$, $p\!=\!0.889$
%).
%\note{[ZG: STATA 29]}
On the contrary, no gender composition effect was found ($F(2, 233)\!=\!1.19$, $p\!=\!0.306$).

    
\paragraph{\textbf{Learning Motivation}} For the ``learning motivation'' measure, we found main effects of the learning tool and the participant gender.
The effect of the learning tool was significant ($F(2, 233)\!=\!7.441$, \textbf{\itshape{p}\,$=$\,0.001}; $\eta_{\text{p}}^{2}\!=\!0.060$ - medium effect size), and the post-hoc tests showed that Textbook ($M\!=\!3.57$, $SD\!=\!1.032$) had a lower score than Tablet-3D ($M\!=\!3.94$, $SD\!=\!0.987$; \textbf{\itshape{p}\,$=$\,0.044}) and Screen-AR ($M\!=\!4.17$, $SD\!=\!0.839$; \textbf{\itshape{p}\,$=$\,0.001}).
The gender effect was also significant ($F(1, 234)\!=\!4.907$, \textbf{\itshape{p}\,$=$\,0.028}; $\eta_{\text{p}}^{2}\!=\!0.021$ - small to medium effect size), showing that the male participants ($M\!=\!4.07$, $SD\!=\!0.902$) had a higher score than the female participants ($M\!=\!3.79$, $SD\!=\!1.020$).
%\note{[ZG: STATA 30]}
For the gender composition (male pairs: $M\!=\!4.145$, $SD\!=\!0.884$; female pairs: $M\!=\!3.769$, $SD\!=\!1.029$; mixed pairs: $M\!=\!3.894$, $SD\!=\!0.963$), a strong tendency towards statistical significance was observed ($F(2, 233)\!=\!2.95$, $p\!=\!0.055$).
Based on Bonferroni test, we further compared the pairs and found that the male pairs had a significantly higher score than the female pairs (\textbf{\itshape{p}\,$=$\,0.048}). 
    
%\end{itemize}

% % Figure environment removed

% % Figure environment removed

% % Figure environment removed


% Figure environment removed

% Figure environment removed

% \note{Table 4 and Figure 6} show the summary of post-questionnaire findings collected from participants after completing the learning activity. 
% Except for favorite sub-activity (with three scales of painting, being painted and both being painted and painting), and recommendation of learning tool (with three scales extracted from Net Promoter Score (NPS) for detractor, passive and promoter), rest of questions followed the well-known 5-point Likert scale (1: strongly disagree, and 5: strongly agree).
% Hence, Figure 6 has five levels, and each dimensions indicate one of the usability topics covered in the questionnaire.


% Table 4. Summary of post-questionnaires reflecting usability, engagement, and motivation of study participants.
% User Evaluation Topics	Textbook
% n = 82
% mean (±SD)	Tablet
% n = 118
% mean (±SD)	AR
% n = 101
% mean (±SD)	F-Value	Post-Hoc Comparison
% Favorite sub-activity	1.84 (±0.88)	1.54 (±0.81)	1.84 (±0.85)	4.52 **	Tablet: AR
% Tablet: Textbook
					
% Anatomy Visualization	4.28 (±0.84)	4.56 (±0.83)	4.25 (±0.96)	4.12 *	Tablet: AR
% User Enjoyment and Engagement	3.78 (±1.04)	4.28 (±0.88)	4.46 (±0.88)	12.73 ***	Textbook: AR
% Textbook: Tablet
% Flow and Loosing Track of Time	4.16 (±0.92)	4.10 (±1.09)	4.30 (±0.85)	1.15, ns	 
% Recommendation of Learning Tool	1.83 (±0.77)	2.12 (±0.76)	2.25 (±0.79)	6.8 ***	Textbook: AR
% Textbook: Tablet
% Increased Learning Motivation	3.56 (±1.01)	3.92 (±0.98)	4.23 (±0.84)	11.42 ***	Textbook: AR
%  Textbook: Tablet
% Note. * shows significance with p < 0.05, ** significance with p < 0.01, and *** significance with p < 0.001; ns=non-significant. 

 
% Figure 6. Bar charts for user evaluation on three conditions. Overall, the experimental conditions of AR and tablet had better outcomes rather than textbook according to usability questionnaire, and the red (textbook) polygon has smallest area in comparison to blue (AR) and green (tablet).

% We performed one-way ANOVA and Bonferroni post-hoc comparisons to identify differences among conditions.
% Post-hoc comparison showed that for most of the usability measures, including user enjoyment and engagement, the recommendation of learning tool, and increased learning motivation, there was a significant difference between Textbook: AR and Textbook: Tablet conditions.
% Moreover, for favorite sub-activity, tablet groups reported lower scores than the other two conditions. 
% Also, students on Tablet groups reported higher scores for anatomy visualization than those on the AR group, and this difference was significant (AR and textbook, however, had very similar scores. Thus, AR was comparable to textbook on visualization).
% As shown on the radar plot in \note{Figure 6}, users have reported more positive responses to two experimental groups of Tablet and AR. 
% RQ4 on user experience was confirmed by observation of most usability measures listed on \note{Table 4}.


% \subsection{Gender Composition}
% \paragraph{Gender Effects}

% For addressing our research questions, we performed one-way ANOVA on score gain and gender.
% The gender effect was present in the data, and females obtained significantly higher score gains in comparison to male counterparts (F(1,299)=4.44, $p\!<\!0.05$, Cohen's d=-0.247)---RQ2 was confirmed by rejecting the null hypothesis. 
% However, team score gains were not statistically different for all-females, all-males or mixed teams (F(2,135)=1.67, ns)---RQ3 was disproved by non-significant findings.
% Summary of mean score gain based on individuals and team gender characteristics is shown in \note{Tables 2 and 3}.
% \note{Figure 4} illustrates the distribution of score gain with boxplots and jitter plots based on individual and team levels.
% Similarly, \note{Figure 5.a} shows the scatterplot of team pre-test and post-test scores among all-males, all-females, and mixed teams in the study. 

% Table 2. Summary of score gain with different gender composition based on individual level and team level
% Gender Composition	Individuals	Teams
% Gender Composition	Female	Male	Total	Females	Males	Mixed	Total
% Observation	n	179	122	301	60	32	46	138
% Score Gain	M±SD	0.229±1.50	-0.131±1.39	0.083±1.46	0.283±1.27	-0.141±0.93	-0.016±1.14	0.085±1.16
% Note. n = Number of observations; M = Mean; SD = Standard deviation; Score gain is in the [-5, 5] range.

% Figure 4. The boxplot with observed data points for score gain across different genders based on (a) individual level and (b) team level. A significant gender difference was observed in the study for score gain based on individuals but not for teams.

% Figure 5. (a) is a scatter plot of team pre-test and team post-test scores among all-male, all-female and mixed gender groups of the study. Regression line and 95\% confidence interval (shaped area) for each gender composition are also included. All groups have positive slopes increased the score. Note that all-female group have a smaller coefficient compare to all-male and mixed gender groups. (b) are distributions of pre-test and post-test scores for different gender groups.

% We conducted the KS test on every pair of gender composition groups.
% Samples from team pre-test score of all-males and all-females (blue and red curves on the top of \note{Figure 5.b}) is statistically significant (D-statistic=0.256, $p\!<\!0.05$).
% Surprisingly, team post-test score from these two gender composition groups (blue and red curves on the bottom of \note{Figure 5.b}) has the most identical distribution (D-statistic=0.081, $p\!=\!0.91$) among others. 
% Overall, all groups increased the score (slops are all positive) but all-females groups have smaller beta1 (coefficient).

% About half of the participants were in females-only teams, which was also a common gender enrollment rate in life sciences and premedical programs~\citep{barzansky_educational_1997}.
% % {(Barzansky et al., 1997)}.
% Although we observed individual gender effect on findings, no significant difference was observed based on the gender composition of teams based on team score gain during the activity. 
% These findings are consistent with previous studies that noted no gender effect in life sciences team studies~\citep{andersson_net_2001,prinsen_gender-related_2007}.
% % {(Andersson, 2001; Prinsen et al., 2007)}.


\subsection{Learning Performance}
\label{Sec:Performance_Results}
% Anatomy Knowledge Retention}

To guarantee the valid evaluation of knowledge retention, e.g., the post-score or the score gained in the anatomy knowledge tests, we first analyzed if there were any significant differences in the pre-score measure among the learning tool conditions or among different gender groups---in other words, we evaluated if both learning tool groups, gender groups, and group gender compositions were well-balanced with respect to the prior knowledge of anatomy.
We did not find any differences in the pre-score measure among the learning tool conditions. However, we found a significant difference between the male and female participants (\textbf{\itshape{p}\,$=$\,0.019}), which showed the male participants ($M\!=\!2.05$, $SD\!=\!1.283$) had a higher pre-score than the female participants ($M\!=\!1.69$, $SD\!=\!1.070$).
%\note{[ZG: STATA 38]}
A significant difference was found among various group gender compositions ($F(2, 233)\!=\!4.74$, \textbf{\itshape{p}\,$=$\,0.0096}).
While controlling the pre-score as a covariate, we conducted one-way ANCOVA to investigate the effects of the learning tool, the participant's gender, or the group gender composition; however, there were no significant effects on both measures of the post-score and the score gain. 
% the post-score ($M\!=\!1.92$, $SD\!=\!1.401$) and the score gain ($M\!=\!0.08$, $SD\!=\!1.459$). 
Among different learning tool conditions, compared with the score gain ($M\!=\!0.069$, $SD\!=\!1.437$) and the post-score ($M\!=\!1.92$, $SD\!=\!1.441$) in the Textbook condition, participants in the Tablet-3D condition achieved the highest score gain ($M\!=\!0.10$, $SD\!=\!1.447$) with post-score ($M\!=\!1.82$, $SD\!=\!1.282$); Screen-AR groups achieved the second highest score gain ($M\!=\!0.079$, $SD\!=\!1.512$) with post-score ($M\!=\!2.04$, $SD\!=\!1.501$). 
With respect to the participant's gender, females achieved a higher score gain ($M\!=\!0.20$, $SD\!=\!1.455$) with post-score ($M\!=\!1.89$, $SD\!=\!1.342$) than males' score gain ($M\!=\!-0.084$, $SD\!=\!1.456$) with post-score ($M\!=\!1.97$, $SD\!=\!1.491$).
%\note{[ZG: STATA 39 40 seem not good to report]}
Among three group gender compositions, compared with the score gain ($M\!=\!-0.145$, $SD\!=\!1.401$) and the post-score ($M\!=\!1.887$, $SD\!=\!1.307$) of the male pairs, female groups achieved the highest score gain ($M\!=\!0.269$, $SD\!=\!1.464$) with post-score ($M\!=\!1.852$, $SD\!=\!1.288$); mixed pairs achieved the second highest score gain ($M\!=\!0$, $SD\!=\!1.488$) with post-score ($M\!=\!2.06$, $SD\!=\!1.654$). 
Two-way ANOVAs with the learning tool, the gender, and the group gender composition factors also did not show any main or interaction effects related to the post-score and the score gain.
% This suggests that the learning tool did not influence the test score, and there is no difference in the scores among the gender groups.



% \begin{itemize}
%     \item [-] \textbf{Pre-Score}: gender effect --> emphasize the importance of score gain, and a connection to ANCOVA.
%     \item [-] \textbf{Post-Score}: none
%     \item [-] \textbf{Score Gain}: none
% \end{itemize}


% Knowledge retention, as the main objective of anatomy learning, was represented by score gain from pre- and post-test scores.
% \note{Table 1} summarizes descriptive statistics for students and teams in each study setting, including the number of observations, mean values, and standard deviations for score gain.

% Table 1. Summary of score gain with different instrumental tools based on individuals and teams.
% 		Individual Measures	Team Measures
% Group	Condition
% (Learning Tool) 	Observation 	Score Gain	Observation 	Score Gain
% 		n	M±SD	n	M±SD
% Control	Textbook 	82 (51 F)
% 0.122±1.49	39	0.105±1.15
% 	Tablet 	118 (63 F)	0.025±1.40	52	0.046±1.10
% 	AR	101(65 F)	0.119±1.52	47	0.112±1.26
% Experiment	Combined (Tablet and AR)	219(128 F)	0.068±1.46	99	0.077±1.17
% Total	Textbook, Tablet, AR	301 (179 F)	0.083±1.46	138	0.085±1.16
% Note. n = Number of observations for individuals and teams; F = number of females; M = Mean; SD = Standard deviation; Score gain is in the [-5, 5] range. Maximum score gain was obtained on textbook condition for individuals, and for AR condition based on teams. Overall, all three conditions have comparable score gains and no significant difference is observed among learning tools. 

% As shown in \note{Table 1} and \note{Figure 2}, the mean value and median of score gain for individuals and teams across the conditions are very similar. 
% Textbook has a slightly higher mean of score gain for individuals. 
% However, among teams, students who used AR have better scores.
% No significant difference in score gain was observed among learning tools (F(2,298)=0.15, ns for individuals and F(2,135)=0.05, ns for teams), and all learning tools appear to have similar and comparable outcomes according to the knowledge retention---so the answer to RQ1 was no, there was no significant difference between conditions and score gains are in-distinguishable from each other.
% These findings are in agreement with previous studies in anatomy education which highlighted the potential of using evolving technologies such as MR and AR for enhancement of student learning and student outcomes in anatomical science education~\citep{maresky_virtual_2019,nicholson_can_2006}.
% % {(Maresky et al., 2019; Nicholson et al., 2006)}. 
% Our study also provides further evidence that 3D visualization technologies increase students' engagement and improve their knowledge retention in human anatomy learning~\citep{hackett_effect_2018,luursema_role_2008,yammine_meta-analysis_2015}.
% % {(Hackett \& Proctor, 2018; Luursema et al., 2008; Yammine \& Violato, 2015)}.

% Figure 2: The boxplot with observed data points for score gain across different learning tools of Textbook, Tablet, and AR, based on (a) individual level and (b) team level. All learning tools lead to relatively similar distributions on score gain, although AR condition had slightly higher score grain mean.

% \note{Figure 3} highlights the scatterplot of test scores (pre-test vs. post-test) for \note{138} teams of the study. 
% Different colors are used to visualize each learning tool.
% According to study findings, especially \note{Figure 3.a} trend lines or slops, all learning tools had similar and comparable outcomes according to knowledge retention.
% This means student performance on knowledge tests were not compromised in experimental groups of AR and tablet; those learning tools which were completely new and unfamiliar platforms to students.
% We performed Kolmogorov-Smirnov (KS) tests on every pair of learning tools in team per-test score and team post-test score to test if distributions are identical. 
% The hypothesis $H_0$ is two samples come from the same distribution. 
% Samples from team pre-test score for tablet and textbook (green and red curved on the top of \note{Figure 3.b} are statistically significant (D-statistic=0.199, $p\!<\!0.05$) which means two distributions are different. 
% D-statistic that is close to 0 means two samples are identical. 
% The most identical distribution is the samples (blue and red curves on the top of \note{Figure 3.b} from the team pre-test score for AR and textbook (D-statistic=0.0996, $p\!=\!0.71$).
% Interestingly, D-statistic increased for the same pair of conditions samples (blue and red curves on the bottom of \note{Figure 3.b}) from team post-test score (D-statistic=0.120, $p\!=\!0.48$).


% Figure 3. (a) is a scatter plot of team pre-test and team post-test scores for three learning tools used in the study. Regression line and 95\% confidence interval (shaped area) for each tool are also included. Slopes are very similar for all learning tools (b) are distributions of pre-test and post-test scores for all three learning tools. Tablet has more spread overall in both pre-test and post-test scores.
