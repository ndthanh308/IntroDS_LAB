\section{Related Work}
\label{Sec:RelatedWork}

\subsection{3D Technologies for Anatomy Education}

%\note{add more references about non-immersive VR (e.g., smartphone/tablet-based 3D visualization tools)@Kyle -added 5 citations}

% \subsubsection{[traditional]}
Anatomy is a complex subject that cannot be learned only from textbooks~\citep{nainggolan2020user}.
Traditional anatomy education is based on the dissection and pro-section of the human body, which provides tangible haptic interactions and realistic environment settings~\citep{snelling_attitudes_2003,gunderman_exploring_2005}.
% {(Snelling et al., 2003)}.
%Gunderman and Wilson~\citep{gunderman_exploring_2005}
% Gunderman and Wilson (2005)} 
%has illustrated that dissecting helps students respect life and understand their patients. 
%However, despite the advantages, the problems of dissection have been concerned.
Despite the advantages dissection offers, it equally raises concerns.
Studies have shown that the learning outcomes and the quality of dissections may be affected by the quality of the material, students' prior experience, and emotional concerns~\citep{trelease_going_2000,winkelmann_anatomical_2007}.
% (Trelease et al., 2000)}.
In particular, both inexperienced and experienced medical and healthcare students are frequently appalled by the fear of death (thanatophobia) and the unnatural smell of cadavers during the dissection~\citep{mclachlan_teaching_2004,winkelmann_anatomical_2007}. 
%The unpleasant learning experience, from the students without any background knowledge, also have been shown in the studies~\citep{mclachlan_teaching_2004,winkelmann_anatomical_2007}, based on the fear of death and the smell of the cadavers during the dissection.
% {(McLachlan et al., 2004; Winkelmann, 2007)}.
Instead of using deceased bodies, clay models provide an alternative solution for educators.
DeHoff et al.~\citep{dehoff_learning_2011} found that compared with animal dissections, students had a better learning experience with clay models.
% {(DeHoff et al., 2011)}.
However, clay models cannot present complex anatomical regions or the functional movements of the structures in the anatomical domain.
Additionally, they impose challenges in transportation and storage. 
%are usually difficult to share, store, and carry with. 
Moreover, anatomy course lab hours have gradually decreased in the past decades~\citep{leung_anatomy_2006,winkelmann_anatomical_2007}, bringing more challenges to anatomy education.
% \subsubsection{[Non-traditional]}

% (Leung et al., 2006; Winkelmann, 2007)}.
As a response to the challenges and changes over time, anatomical learning platforms increasingly adapt from traditional methods to digital technology. 
Like any other reshaping processes, some anatomy scholars argue that dynamic visualization compensates for students' low spatial abilities by providing an explicit external representation of the system~\citep{hays_spatial_1996,huk_who_2006,stieff_mental_2007,chickness2022novel}.
% (Hays, 1996; Huk, 2006; Stieff, 2007)}.
Increasingly powerful and accessible computer hardware allow 3D visualizations to replace or supplement traditional teaching in healthcare regarding lectures, cadavers, and textbooks~\citep{yammine_meta-analysis_2015,golenhofen_use_2019,lemos_design_2019,maresky_virtual_2019}.
%Virtual Human Dissector© (VHD) software~\citep{donnelly2009virtual} is one of the interactive teaching tools for cross-sectional anatomy, capable of reconstructing 3D views from 2D images. 
Donnelly et al.~\cite{donnelly2009virtual} investigated the use of Virtual Human Dissector© (VHD) software, interactive teaching tools for cross-sectional anatomy, capable of reconstructing 3D views from 2D images, in anatomy education with self-directed learning and found no significant difference when compared with a students group that learns from using prosection, models, and textbooks. 
Kennan \& Awadh ~\cite{keenan2019integrating} discussed the effective utilization of visual 3D learning technologies as self-learning resources in the context of cross-sectional anatomy.
They proposed integrating the use of the 3D VHD system with Sectra~\citep{barrack2015step}, a medical imaging device, to enhance the understanding of cross-sectional anatomy.

Lim et al.~\citep{lim_use_2016} used 3D-printed models instead of traditional cadaveric specimens during the learning of external cardiac anatomy. 
Although 3D printing technology is capable of providing teaching materials, the 3D printed model is limited by the complexity of anatomical regions. 
Equally, researchers found that 3D visualization methods improved student performance by providing multiple anatomical views and different perspectives of 3D rotating models, even on 2D screens~\citep{yammine_meta-analysis_2015}.
% (Yammine \& Violato, 2015)}.
Mobile-based applications and web-based 3D games have also been used as efficient learning tools for the study of human skeletal, muscular, and cardiovascular systems to explore more spatial information about the 3D anatomical models~\citep{golenhofen_use_2019,lemos_design_2019}.
% {(Golenhofen et al., 2019; Lemos et al., 2019)}. 

VR and AR techniques have been adopted into anatomy education in recent years~\citep{maresky_virtual_2019,bork_empirical_2017,silva_emerging_2018,bacca2014augmented}.
% {(Maresky et al., 2019; Nicholson et al., 2006; Silva et al., 2018)}.
As dynamic visualizing tools, they engage students in an immersive environment with audio and visual interactions and stereoscopic 3D models to have a better functional understanding of the anatomical structure and its movement within the 3D body space~\citep{hackett_effect_2018,jacob2012lindsay,luursema_role_2008,preim2018survey}.
% {(Hackett \& Proctor, 2018; Luursema et al., 2006, 2008)}.
Duncan‐Vaidya and Stevenson~\citep{duncan-vaidya_effectiveness_2020}
% Duncan‐Vaidya \& Stevenson (2020)} 
have found that experience from AR positively influenced the learning process of skull anatomy, on a similar level as traditional tools such as textbooks or plastic skull models.
Increasing the frequency of learning instances, and interactions with models and specimens are advantages of teaching anatomy in AR~\citep{Romand2020}.
% \note{(Romand et al., 2020)}.
Kolla et al.~\cite{kolla2020medical} examined the effectiveness of VR technology in anatomy education and compared it to traditional teaching methods like lectures and cadaveric dissection.
28 first-year medical students used HTC VIVE, a VR platform to identify anatomical structures, drew them on a virtual skeleton, and then provided feedback through surveys. 
Their results indicated that VR was highly supported by the students, demonstrating its potential as a valuable tool for learning human anatomy and as a useful complement.
However, all study conditions were conducted in VR settings.
In a different study, the AR controlled group that was randomly selected from a biochemistry course suggests that AR educational apps motivated them to understand the visualized processes~\citep{barrow_augmented_2019}.
% {(Barrow et al., 2019)}.
Despite the popularity of using VR/AR methods in anatomy education, a review paper~\citep{chytas2022extended} of 152 articles did not identify conclusive evidence of their enhanced effectiveness over traditional anatomy education methods.



% In this paper, we would like to evaluate the user experience from both individual and team in an anatomy study, which leverages modern anatomical content visualization in 3D with handheld mobile/tablet devices and large-scale AR displays.


\subsection{Measures of Anatomy Learning Experience}
%\note{elaborate a bit more on the references. R2 complained about the part of Nainggolan et al.}
To analyze the learning experience and evaluate the effectiveness of VR/AR applications in general, various methods for data collection and different measures were used in previous research.
Kurniawan and Witjaksono~\citep{kurniawan2018human} evaluated the usefulness of the mobile-based AR application by employing the attitude questionnaire to analyze the user perception. 
%using the Likert Scale for user perceiving analysis. 
% Nainggolan et al.~\citep{nainggolan2020user} conducted a survey, scored, and evaluated the user agreement level and user satisfaction level of each anatomical function of a VIVE Controller.
Nainggolan et al.~\citep{nainggolan2020user} evaluated the interactivity level of VIVE Controller based on the user agreement level and user satisfaction level, and found that the use of the VR controller in anatomy learning system was very interactive and satisfactory.
Tanjung et al.~\citep{fahmi2020comparison} conducted a comparison learning experience study to evaluate the level of acceptability and satisfaction towards three anatomical learning systems.
Duncan‐Vaidya and Stevenson~\citep{duncan2021effectiveness}, and Barmaki et al.~\citep{barmaki_enhancement_2019,bork_empirical_2017,barmaki2020deep} used the pre- and post-knowledge quizzes, and a usability questionnaire for data collection, and analyzed the effectiveness of the AR tool in anatomy learning studies.


%learning experience (or user experience) in (anatomy) education.



%learning performance (knowledge retention)
% \subsubsection{Knowledge Retention}

% Compared to individual and competition efforts, collaborative learning strategies provide positive effects on achievement and productivity~\citep{johnson_effects_1981}.
% % {(Johnson et al., 1981)}.
% Previous research provided evidence on reduced time to complete learning tasks in well-format collaboration settings~\citep{williams_support_2001},
% % {(Williams \& Upchurch, 2001)}, 
% improved student's understanding of learning process~\citep{declue_pair_2003},
% % (DeClue, 2003)}, 
% and improved their performance on exams~\citep{declue_pair_2003}, p.\! 1).
% % {(DeClue, 2003, p. 1)}.
% Another study has shown that interactions during the learning process help students' long-term retention of key ideas~\citep{perez-sabater_active_2011}.
% % {(Pérez-Sabater et al., 2011)}.
% With the rapid development of medical science, the explosive knowledge in the anatomy educations raises serious problems and knowledge retention, as the essential learning goal, has been used for learning evaluation.
% In evaluating the performance during the learning process, researchers often use formal assessment~\citep{van_boxtel_collaborative_2000},
% % {(Van Boxtel et al., 2000)}, 
% quizzes, exams~\citep{sangin_facilitating_2011},
% % {(Sangin et al., 2011)}, 
% and self-reported survey~\citep{terenzini_collaborative_2001}
% % {(Terenzini et al., 2001)} 
% has always been adapted to measure students' learning outcome and knowledge retention.


\subsection{Gender Effects in Education}

In different education domains, gender differences have been discussed in recent studies.
Understanding gender effects in education is not conclusive, and it varies based on different disciplines and tasks~\citep{fernandez-sanz_analysis_2012,wegge_age_2008}.
% {(Fernandez-Sanz \& Misra, 2012; Wegge et al., 2008)}.
In the science, technology, engineering, and mathematics (STEM) domain, negative learning experiences for females due to gender biases at the technical level have been reported~\citep{meadows_interactive_2015,nagappan_improving_2003}.
% {(Meadows et al., 2015; Nagappan et al., 2003, p. 1)}.
In the business domain, females' higher management ability in group tasks was highlighted~\citep{bear_role_2011,de_paola_teamwork_2018,eagly_female_2003}.
% {(Bear \& Woolley, 2011; De Paola et al., 2018; Eagly \& Carli, 2003)}.
Previous research has also shown that during cognitive tests, females have better information-processing skills than males~\citep{rabbitt_unique_1995,schaie_age_1993}.
% {(Rabbitt et al., 1995; Schaie \& Willis, 1993)}.
%Conversely, no significant gender effect has been reported in the study. 
In Andersson's study~\citep{andersson_net_2001}
% Andersson's study (2001)}
on explicit spatial and verbal collaborative memory performance, the author reported better retention performance for females, but he argued that there was no main gender effect on team performance. 
Prinsen et al.~\citep{prinsen_gender-related_2007}
% {Prinsen et al. (2007)}
noted that, in computer-mediated communication~\citep{herring_computer-mediated_1996}
% {(Herring, 1996)}
and computer-supported collaborative learning settings~\citep{lehtinen_computer_1999,lipponen_challenges_1999,barmaki2020deep}, % {(Lehtinen et al., 1999; Lipponen, 1999)}, 
learning performance for different genders might change by various role distributions. 
We will discuss the observed gender differences (from individual and group composition standpoints) in our anatomy learning study, though we need to keep in mind since we worked with pre-assigned teams, it was not feasible to form gender-balanced teams for the work reported here.


%%%%%%%%%%%%%%%%%%%%%%%%%%%%%%%%%%%%%%%%%%%%%%%%%%%%%%%%%%%%%%%%%%%%%%%%%%%%%% 
%%%%%%%%%%%%%%%%%%%%%%%%%%%%%%%%%%%%%%%%%%%%%%%%%%%%%%%%%%%%%%%%%%%%%%%%%%%%%%

% \note{S. E. Kirkley and J. R. Kirkley, “Creating next generation blended learning environments using mixed reality, Video Games and Simulations,” TechTrends, vol. 49, no. 3, pp. 42–53, 2005.}
% \note{motivational function of reinforcement, learning by direct experience or by observing the behavior of others: A. Bandura, Social learning theory. New York: General Learning Press, 1971.}
% \note{reinforcement: \url{https://www.learnalberta.ca/content/inspb1/html/6_positivereinforcement.html}}

% \note{\citep{bork_empirical_2017}F. Bork, R. Barmaki, U. Eck, K. Yu, C. Sandor, and N. Navab, “Empirical Study of Non-Reversing Magic Mirrors for Augmented Reality Anatomy Learning,” in IEEE International Symposium on Mixed and Augmented Reality, 2017, pp. 169–176.}

% \note{additional references}
% \note{B. Preim and P. Saalfeld, “A survey of virtual human anatomy education systems,” Comput. Graph., vol. 71, pp. 132–153, 2018.}

% \note{Z. Guo and R. Barmaki, “Deep neural networks for collaborative learning analytics: Evaluating team collaborations using student gaze point prediction,” Australas. J. Educ. Technol., vol. 36, no. 6, pp. 53–71, Dec. 2020.}

% \note{C. Jacob et al., “LINDSAY virtual human: Multi-scale, agent-based, and interactive,” Adv. Intell. Model. Simul. Artif. Intell. Model. Tech. Scalable Comput., vol. 422, pp. 327–349, 2012.}

% \note{mixed reality as an education tool: C. Jacob et al., “LINDSAY virtual human: Multi-scale, agent-based, and interactive,” Adv. Intell. Model. Simul. Artif. Intell. Model. Tech. Scalable Comput., vol. 422, pp. 327–349, 2012.}


% \note{survey: J. Bacca, R. Fabregat, S. Baldiris, S. Graf, and Kinshuk, “Augmented reality trends in education: A systematic review of research and applications,” Educ. Technol. Soc., vol. 17, no. January 2016, pp. 133–149, 2014.}

% \note{anatomy training: J. R. Keebler, B. S. Patzer, T. J. Wiltshire, and S. M. Fiore, “Augmented Reality Systems in Training,” in The Cambridge Handbook of Workplace Training and Employee Development, K. G. E. Brown, Ed. Cambridge University Press, 2017, pp. 278–292.}