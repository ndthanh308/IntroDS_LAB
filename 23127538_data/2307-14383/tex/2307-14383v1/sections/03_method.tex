\section{Methods and Materials}
% \section{Experiment}
\label{Sec:Method}
%% Figure environment removed
In this section, we will describe the details of the conducted study with our proposed tablet-based 3D visualization and screen-based AR learning tools.
Relevant hypotheses were established and evaluated to address our general research questions introduced in Section~\ref{Sec:Intro}.
% through a large-scale user study.
The study was approved by the Institutional Review Board of the affiliated University (Protocol \# XXXXXXX). %\note{TODO}


\subsection{Digital Anatomy Learning Tools}
\label{Sec:ToolsAndIntervention}
% \subsection{System Overview}

%\note{more details about the learning tools, what information were provided, how the systems were implemented, etc.}
%\note{gender effect: how the participant pairs were created, did they know each other beforehand? mixed gender? etc.} Leila: This was part of the initial description but for some reason was removed along the way.. I added it to participants

% \note{Figure 1} shows our three study conditions of the learning activity.
% Since there were also three concurrent lab sessions, we had one workstation per lab. 
% As shown in \note{Figure 1.c}, the mobile workstation had one Alienware 17 Laptop with GTX 1070m GPU, a 55-inch Samsung TV, and a Microsoft Kinect One tracking sensor; all mounted on a portable TV cart.
% The TV was either streaming the camera feed from Kinect for tablet and textbook settings, or the 3D anatomical overlays atop participants, with a 3D virtual model in the AR setting. 
% The AR setting was designed to use a hand-held clicker to navigate through the learning platform. 
% Students could navigate among the muscle groups and see the highlighted muscles and labels overlay atop their bodies, besides rotating and zooming in/out the virtual 3D model using the clicker. 
% For the tablet setting, we used Samsung Galaxy Tab 2 with 10.1 in screen size---comparable to textbook anatomy diagram size---and installed our interactive app on it.
% In the tablet condition, for sake of consistency, the interactive app had the exact content as the AR system, with two main differences: (a) the interaction with tablet was based on touch, similar to a smartphone, instead of a clicker and (b) self-augmentation was not present.
% Other key properties for 3D anatomy model visualization, including regional and full-body anatomy visualization, 3D model rotation, zoom in and out, toggling the labels for enhancing readability, in addition to the highlighting key muscles for the painting activity, and color-coded labels were similar for AR and tablet platforms (see \note{Figures 1.e} and \note{1.f}). 

% Figure 1. (a-c) Three study settings for students to complete anatomy learning activity, and (d-f): a closer look of learning tools of textbook, tablet, and screen-based AR system, all for a specific muscle group. 

To investigate the effects of tablet- and AR-based learning tools in anatomy education, we prepared in-house tablet-based 3D visualization and screen-based AR for our user study. 
The learning tools are presented in Figure~\ref{fig:tools}.
% \note{Textbook}
% \note{Tablet-based VR App}
For Tablet-3D, we developed an interactive Android application for hand-held devices, which visualizes 3D virtual models of body muscles and labels on a tablet (see Figure~\ref{fig:tools}(b)).
The visualization of the 3D anatomy included regional and full-body anatomy models.
Participants could rotate, adjust zoom levels, toggle between color-coded virtual labels for enhancing readability, and highlight relevant muscle groups for our user study (see Figure~\ref{fig:tools}(c)).
The developed application was installed on Samsung Galaxy Tab 2 with a 10.1-inch screen.
% ---comparable to textbook anatomy diagram size---and installed our interactive app on it.
% \note{Life-size AR Mirror}

For Screen-AR, we deployed our screen-based AR tool onto a large monitor with a mounted Kinect v2 camera. 
Screen-AR used a split-screen to show an AR view on the left side and a focused view of the anatomical model on the right side. 
Compared to Tablet-3D, the pose of the virtual anatomy was animated based on the body tracking data of the Kinect v2.
Furthermore, body tracking allowed Screen-AR to superimpose 3D life-size virtual models of human anatomical systems, such as the musculary system and labels atop the participant's body in a mirror-like AR visualization.
Participants could navigate between the relevant muscle groups using a hand-held clicker.
Depending on the muscle group, the virtual camera responsible for rendering the scene on the right screen automatically follows and zooms into the selected muscles.
Screen-AR consists of an Alienware 17 Laptop with GTX 1070m GPU, a 55-inch Samsung TV, and a Microsoft Kinect v2 tracking sensor, and all devices were mounted on a mobile TV cart.
% for the mobility of the system.
%The TV displayed the 3D anatomical overlays on top of the participant's view, while the Kinect sensor detected and tracked the participant's body in front of the TV screen.
%For the sake of consistency, both the AR tool had the exact contents as the VR tool, but with two main differences: 

We pre-defined the selectable muscle groups for Tablet-3D and Screen-AR, which were the same for both conditions.
The main differences between Tablet-3D and Screen-AR are as follows:
(i) the interaction with Tablet-3D was based on touch-screen while Screen-AR was controlled by the clicker and the users' body pose, and (ii) Screen-AR, unlike Tablet-3D, provided a self-augmentation view of anatomical models.


% Figure environment removed


%Utilizing these VR/AR learning tools, w
\subsection{Team-based Learning Intervention}

We prepared a team-based learning intervention inspired by a manual laboratory activity on muscle painting~\citep{marieb_essentials_2006}.
% {(Marieb \& Jackson, 2006)}
Muscle painting is one of the common exercises in a medical curriculum~\cite{mcmenamin_body_2008}, and students execute the muscle painting activity typically with printed 2D visualization of human musculature from the lab manual. 
We extend the same principle to our 3D applications. 
%in-house VR and AR learning tools for interactive 3D visualization.
During the intervention, participants worked in a team of two people to collaboratively learn and teach among peers about human muscle groups through a body painting activity.
They were asked to identify and paint major muscle parts on their body with washable painting supplies while using one of a randomly assigned (see Section~\ref{Sec:StudyDesign}) learning tool.
%, while using one of the learning tools: (1) textbook or lab manual, (2) tablet-based VR app, and (3) mirror-like AR visualization tool. 
Once one of the participants completed the painter's role, they switched their role to be a paintee with the learning partners.
We showed 40 muscles and labels to the participants to provide extensive anatomical landmarks. 
However, we only asked them to identify 12 major muscles (same for all conditions) and use appropriate body paint to colorize the muscles' location on their arms and legs.

% Our study had three different settings based on instrumental tools. 
% Students in the control group used textbook or the lab manual as their learning tools. 

% In experimental group I, instead of a textbook, students used our in-house interactive app on the tablet as a 3D musculoskeletal visualizing system.

% Experimental group II used a screen-based AR system—also developed internally—where students could see themselves with augmented anatomy visualizations on a large display.

% The knowledge-base information, presented in all instrumental tools, were identical across the board to mitigate potential confounding factors related to student workload and learning (exactly 40 muscle names and their locations were presented to students in all learning tools). 



\subsection{Study Design}
\label{Sec:StudyDesign}

We conducted a user study using a between-subjects design with three learning tool conditions.
A brief description of each learning tool is described in the following.
%, but the details of our in-house VR and AR tools were introduced in Section~\ref{Sec:ToolsAndIntervention}.
\begin{itemize}
    \item \textbf{Textbook}: As a traditional learning method, participants used a textbook during the team-based anatomy learning intervention described in Section~\ref{Sec:ToolsAndIntervention}.
    They identified the corresponding muscle parts on their partner's body while checking the location of target muscles from the textbook (see Figure~\ref{fig:tools}(a, d)).
    The textbook could be carried while performing the muscle painting activity.
    \item \textbf{Tablet-3D}: Participants used our tablet-based 3D anatomy application during the learning intervention.
    The participants could carry the tablet that visualizes 3D virtual anatomy models and labels, while performing the painting activity (see Figure~\ref{fig:tools}(b, e)).
    \item \textbf{Screen-AR}: Participants used our large screen-based AR anatomy tool during the learning intervention.
    The participants could see the virtual 3D models and labels directly overlaid on their own body through the large screen (see Figure~\ref{fig:tools}(c, f)).
\end{itemize}
% \note{we can skip # of students and just talk about lab assignments since participants section is just below. What about this: We had a total of 17 anatomy lab sessions with a maximum capacity of 20 in each lab. Considering the large scale of the study, participants could be randomly assigned to one of the three learning tool conditions by randomly scheduling different conditions to each lab session.}
The study was performed alongside 17 anatomy lab sessions with a maximum capacity of 20 students in each lab.
Considering the large scale of the study, participants could be randomly assigned to one of the three learning tool conditions by scheduling a different condition for each lab session.
This assignment method is termed hierarchical or clustered randomization, which is a common practice in educational studies and clinical trials~\citep{davis_application_2002}.
% {(Davis et al., 2002)}.

% \subsection{Collaborative Learning Intervention}

% We conducted a between-subjects study of team-based anatomy learning intervention in a laboratory course of General Biology as part of undergraduate premedical curricula. 
% \note{321} students in \note{138} teams participated in the user study.
% The intervention was inspired by a manual laboratory activity on muscle painting~\citep{marieb_essentials_2006}
% % {(Marieb \& Jackson, 2006)}
% using 2D visualization in the lab manual, but we adapted it for using other learning tools for visualization.
% During this intervention, students worked in teams to learn more about human muscle groups in a body painting activity.
% They were expected to identify and paint major muscles of their body using one of the learning instruments (textbook or lab manual, tablet, and AR) and washable painting supplies. 
% While 40 muscles and labels were shown to students, they were expected to identify 12 muscles out of these and paint them on each other's extremities (six muscles per student for teams of two). 

% Our study had three different settings based on instrumental tools. 
% Students in the control group used textbook or the lab manual as their learning tools. 
% In experimental group I, instead of a textbook, students used our in-house interactive app on the tablet as a 3D musculoskeletal visualizing system.
% Experimental group II used a screen-based AR system—also developed internally—where students could see themselves with augmented anatomy visualizations on a large display.
% The knowledge-base information, presented in all instrumental tools, were identical across the board to mitigate potential confounding factors related to student workload and learning (exactly 40 muscle names and their locations were presented to students in all learning tools). 


\subsection{Participants}
We initially recruited 319 participants (male: 128, female: 191), although we needed to exclude a subset of the data (described in the following). The recruitment took place via an online flyer from a laboratory course of General Biology as part of undergraduate premedical curricula. The recruitment was part of a general biology lab course, and the students were randomly assigned by their instructor into teams of two to four to perform their assignments throughout the semester. There were a total of 15 lab sessions, accommodating up to 25 students per lab session. Considering the size of the recruitment, it was not feasible to do team-level condition assignments or rearrange the teams for more gender-balanced teams. 
% Each student in this course attended laboratory one afternoon a week for three hours, in a room with up to 25 total students and one teaching assistant. There were three such rooms operating concurrently each day that the laboratory was in session. Considering the size of the laboratory, instructors collaborated closely with all of the 15 teaching assistants and laboratory manager to guide students. Students were assigned by their instructor into teams of two to four for performing their assignments throughout the semester

% In total, we had \note{119??} teams of size two, \note{13?? teams of size three, and 6?? teams of size four}.
The anatomy learning intervention in our study involved teams of sizes two, three, and four.
However, teams of size two were only considered for our analysis in this paper because of the consistent experience of the participants.
For example, participants in the team of size two were guaranteed to have both painting and being painted experience during the intervention, but larger teams had different task distribution, thus excluded.

After excluding larger teams, our subset data of interest for this paper includes \textbf{n=236} participants (95 males and 141 females; age $M\!=\!19.77$, $SD\!=\!1.81$) in \textbf{118} teams of size two 
%\note{[ZG: STATA 32]}
within three age-balanced groups with the following  gender compositions. Male pairs: 62 with age $M\!=\!19.45$, $SD\!=\!1.035$, Female pairs: 108 with age $M\!=\!19.71$, $SD\!=\!1.583$, and Mixed pairs: 66 with age $M\!=\!20.17$, $SD\!=\!2.527$). Participant's gender and team gender composition interplay with Learning Experience and Learning Performance is investigated.Table~\ref{Tab:Participants} shows the number of participants in each study condition.
% \note{age?, other demo info?}
% \note{compensation? purely voluntary?}

\begin{table}[htb]
\caption{The number of participants in each study condition.}
\label{Tab:Participants}
% \vspace{-1ex}
\begin{tabular}{c|c|c|c}
\toprule
\textbf{Study Condition (Learning Tool)} & \textbf{Male} & \textbf{Female} & \textbf{Total} \\
\midrule
\textbf{Textbook}  & 27   & 45     & 72    \\
\textbf{Tablet-3D} & 43   & 45     & 88    \\
\textbf{Screen-AR} & 25   & 51     & 76    \\
\midrule
\textbf{Total}     & 95   & 141    & 236  \\
\bottomrule
\end{tabular}
\end{table}


% A total of \note{301 students (179 females) in 138 teams} were selected in this study.
% Although data were collected from \note{321} participants, data from \note{20} individuals were excluded due to incomplete team information. 
% In total, we had \note{119} teams of size two, \note{13 teams of size three, and 6 teams of size four}.
% \note{Table 1} shows the number of individuals and teams assigned to each study condition. 
% % We first assured that participant assignment to study conditions was properly performed, and the study is balanced based on participant demographics (gender) and anatomy prior knowledge.
% % The Chi-square test did not reveal any difference in the distribution of participants based on genders on three study conditions ($\chi^2=3.06$, df=2, ns).
% % Similarly, one-way ANOVA did not identify any interaction of pre-test and study conditions (F(2,298)=0.14, ns).
% Although data were collected from \note{321} participants, data from \note{20} individuals were excluded due to incomplete team information. 
% \note{Table 1} shows the number of individuals and teams assigned to each study condition. 

% We first assured that participant assignment to study conditions was properly performed, and the study is balanced based on participant demographics (gender) and anatomy prior knowledge.
% The Chi-square test did not reveal any difference in the distribution of participants based on genders on three study conditions ($\chi^2=3.06$, df=2, ns).
% Similarly, one-way ANOVA did not identify any interaction of pre-test and study conditions (F(2,298)=0.14, ns).


% \subsection{Study Design and Procedure}
\subsection{Procedure}

% Our user study had a 3$\times$1 between-subjects research design.
% Considering the scale of the study with \note{321} participants, each lab session (out of 17 lab sessions) was randomly assigned to one of the study conditions.
% This method of participant recruitment and assignment is termed hierarchical or clustered randomization, and it is a common practice in educational studies and clinical trials~\citep{davis_application_2002}.
% % {(Davis et al., 2002)}.

% The study procedure was as follows. 
% An online flyer was sent to all undergraduate students enrolled in the General Biology lab, inviting them to participate in the study.
% The study was approved by Institutional Review Board in the institution study carried out (Protocol \# XXXXXXX), and oral informed consent was obtained from each participant before the study commenced. 
% After consent, students completed the online pre-questionnaire individually and then entered the painting activity room with their preassigned teammates. 
% Each team either used (1) anatomy interactive app on the tablet, (2) screen-based AR system as two experimental groups, or (3) textbook as control groups to complete the team-based learning activity.

The study procedure was as follows. 
An online flyer was sent to all undergraduate students in the General Biology lab, inviting them to participate.
When participants arrived for the study, oral informed consent was obtained from each participant before the study commenced.
After the consent, participants individually completed the online pre-questionnaire, which asked for their demographic information and evaluated their prior anatomy knowledge. 
Then they entered the learning intervention room with their pre-assigned teammates.
Each team only used one of the learning tools described in Section~\ref{Sec:StudyDesign} to complete the learning task. %: (1) Textbook, (2) Tablet-VR, and (3) Screen-AR, .
% Figure 1 presents study settings, visualization, as well as the painting activity setup. 
% For the AR system, the model (or ``paintee'') was asked to stand closer to the Kinect body tracking sensor to have the appropriate digital anatomical illustrations superimposed on their body. 
% Then, students found the appropriate muscle to be painted using the presenter clicker to navigate the software.
% Next, the painter used the muscle overlay information shown on the TV screen to paint the muscle on the model's skin.
% After finishing painting upper limb muscles, students switched roles in order to paint the lower limb muscles.
% Teams in the tablet and textbook groups also had a workstation in their activity room, but it was served as the data collection purpose from teams. 
% Students of tablet or textbook groups used either the interactive app's 3D visualizations or the laboratory manual anatomy figures to complete the activity. 
% After completing the painting, all of the students completed the online post-questionnaire individually.
% Students then presented their painted limbs to their teaching assistants. 
During the painting activity, the participant in the painter's role tried to find appropriate muscles on the teammate's body (who played the role of a paintee) to be painted while navigating different muscle parts.
Participants in the Tablet-3D or textbook conditions used either the interactive application's 3D visualizations or the laboratory manual anatomy figures to complete the activity.
Particularly for the AR system, the participant in the paintee's role was asked to stand closer to the Kinect body tracking sensor to have appropriate digital anatomical illustrations superimposed on their body. 
The painter used the muscle overlay information on the TV screen to paint the muscle on the paintee's skin.
% After finishing painting upper limb muscles, students switched roles in order to paint the lower limb muscles.
% Teams in the tablet and textbook groups also had a workstation in their activity room, but it was served as the data collection purpose from teams. 
After completing the painting, all participants completed the online post-questionnaire individually, which asked their \expMeasure with the learning tool used during the intervention and evaluated their anatomy knowledge retention.
Before terminating the study session, they presented their painted limbs to their teaching assistants.
The activity, including learning intervention and questionnaire completion, took approximately 20--30 minutes.
% , so that their learning performance could be objectively scored. 







% \subsection{Evaluation Measures and Research Questions}
\subsection{Measures and Hypotheses}
\label{Sec:MeasuresHypotheses}

% In this section, we introduce research questions to inform this work based on user study evaluation measures. 
% It is worth noting that our study used both individual and team-based measures to better investigate the impact of various advanced learning tools on anatomy learning.

This section describes the measures we used for the study, which were collected through questionnaires.
We also introduce several hypotheses that we established based on the measures and our research questions in Section~\ref{Sec:Intro}.
% It is worth noting that our study used both individual and team-based measures to better investigate the impact of various advanced learning tools on anatomy learning.
We used online questionnaires on the Qualtrics platform (Provoto, UT) for designing and collecting pre- and post-questionnaires.


% \subsection{Questionnaires}

% \subsubsection{User Engagement and Satisfaction}
% \subsubsection{Usability}
% \subsubsection{User Experience}
\subsubsection{Learning Experience}

We prepared nine subjective questions to examine the participants' \expMeasure in the anatomy learning intervention using different learning tools.
The questions were presented on a five-point Likert scale (1: Strongly Disagree to 5: Strongly Agree) except for the ``willingness to recommend'' measure, which had a scale of 0 to 10 (10 means a strong willingness).
Our exact question for each \expMeasure measure is described below.
% All students who participated in the study took pre- and post-questionnaires before and after the muscle painting activity. 
% The exit survey or post-questionnaire collected student's opinions on their favorite sub-activity, in addition to their user experience, while interacting with one of the learning tools.
% From the usability standpoint, we were interested in addressing the following research questions:
% \textbf{RQ4.} Is there a difference among learning tools based on user experience questionnaire?
\begin{itemize}
% \paragraph{Role Preference}
% In this activity:
% o	Both my arm and leg were painted  (1) 
% o	My arm was painted ‎  (4) 
% o	My leg was painted ‎  (5) 
% o	I painted my partner’s skin ‎  (6) 
% o	None of the above - I had a different role (e.g. locating the muscles)  (7) 
% Which part of the activity did you like the most?
% o	Painting  (1) 
% o	Being the model to be painted  (2) 
% o	Both painting and being the model  (3) 
    \item [-] \textbf{Easy to Paint}: ``Using (textbook/Tablet-3D/Screen-AR) was easy and straightforward for completing the muscle painting activity.''
    \item [-] \textbf{Easy to Find Muscles}: ``I recognized and found the location of muscles easily using (textbook/Tablet-3D/Screen-AR).''
    \item [-] \textbf{Satisfaction}: ``Using (textbook/Tablet-3D/Screen-AR) for the muscle painting activity was satisfying.''
    \item [-] \textbf{Learning Perception}: ``I learned a lot about muscle anatomy using/interacting with (textbook/Tablet-3D/Screen-AR).''
    \item [-] \textbf{Enjoyment}: ``I found using (textbook/Tablet-3D/Screen-AR) to be enjoyable.''
    \item [-] \textbf{Effort to Focus}: ``I had to make an effort to keep my mind on the activity.''
    \item [-] \textbf{Lost Track of Time}: ``Time seemed to pass very quickly during the painting activity.''
    \item [-] \textbf{Willingness to Recommend}: ``I would recommend my friends to use (textbook/Tablet-3D/Screen-AR) for learning about human muscles.''
    \item [-] \textbf{Learning Motivation}: ``Using (textbook/Tablet-3/Screen-AR) increased my enthusiasm for learning more about human anatomy.''
\end{itemize}

%\note{check references as theoretical foundation for H1, H2, H5, and H6}
Considering the potential benefits of interactive visualizations in our 3D visualization tool, and more intuitive and direct information display on the real body in the AR tool \citep{blum2012mirracle}, we established the following hypotheses for the \expMeasure measures:
\begin{itemize}
    \item \textbf{H1}: The participants in the Tablet-3D condition or the Screen-AR condition will have more positive \expMeasure than those in the Textbook condition for all the \expMeasure measures. \\(Textbook $<$ Tablet-3D, Screen-AR)
    \item \textbf{H2}: The participants in the Screen-AR condition will further have more positive \expMeasure than those in the Tablet-3D condition for all the \expMeasure measures. \\(Tablet-3D $<$ Screen-AR)
\end{itemize}

%Also, given the prior research showed that females tend to be more conservative about being touched than males~\citep{finn2010qualitative} and that males tend to be more interested in computers and technology than females~\citep{Jones2000}, we also established the following hypothesis for the gender difference in the \expMeasure during the intervention:

%\begin{itemize}
%\item \textbf{H3}: The male participants in the study will report more positive \expMeasure than the female participants, particularly in the measures of satisfaction, enjoyment, and willingness to recommend. (Female $<$ Male)
%\end{itemize}


As Figure~\ref{fig:tools} shows, the anatomy models in the textbooks, on the tablets, and on the screen-AR are male-based. 
% Compared to female students, male students can easily recognize and access the muscles according to the same anatomy structures.
Also, given the prior research showed that females tend to be more conservative about being touched than males~\citep{finn2010qualitative} and that males tend to be more interested in computers and technology than females~\citep{Jones2000}, we also established the following hypothesis to understand the interplay role of the gender/gender composition in the \expMeasure during the intervention:
\begin{itemize}
\item \textbf{H3}: The male participants (male pairs) in the study will report more positive \expMeasure than the female participants (female pairs or mixed pairs), particularly in the measures of easy to paint, easy to find muscles, satisfaction, learning perception, enjoyment, willingness to recommend, and learning motivation. 

(Female $<$ Male) \&

(Mixed $<=$ Females $<$ Males)
\end{itemize}

\subsubsection{Short-term Knowledge Retention}
% \paragraph{Pre-questionnaire}
% \paragraph{Post-questionnaire}
%\note{R3: is it really a performance measure? or just a short-term memory.}
To examine the learning performance in anatomy education, we evaluated the participant's anatomy knowledge retention.
We collected the participants' anatomy knowledge scores from pre- and post-tests, and calculated the score gain by subtracting the pre-test score from the post-test score.
\begin{itemize}
    \item [-] \textbf{Pre-Score}: The pre-test was a matching test with five questions in a provided diagram of the human anatomy muscle system. 
    In the anatomy diagram, 15 regions of the body were highlighted, and participants were asked to match five muscle labels provided in the pre-test with one of these 15 body regions.
    The number of correct matches was reported as the pre-test score in the range of [0, 5].
    \item [-] \textbf{Post-Score}: The post-test had a different (lateral) view of human anatomy, with a matching test similar to the pre-test.
    The number of correct matches was reported as the post-test score in the range of [0, 5].
    Both pre- and post-test were designed and evaluated by anatomy instructors to be at the same level of difficulty.
    \item [-] \textbf{Score Gain}: Score gain, as the difference between pre- and post-test scores, was calculated in the range of [-5, 5].
\end{itemize}
% The post-questionnaire had questions related to the intervention (favorite sub-activity), user experience evaluation (see \note{Table 4} for the complete list of the topics), and a post-test. 
% The post-test had a different (lateral) view of human anatomy, with a matching test similar to the pre-test.
% The number of correct matches was reported as the post-test score.
% Both pre- and post-test was designed and evaluated by anatomy instructors to be at the same level of difficulty and with an equal distribution of questions among upper and lower limbs. 
% The average of test scores of all team members was reported as team pre-test and team post-test scores. 	

% \paragraph{Knowledge Retention}
% As mentioned in section 3.3 for questionnaires, we solicitepd both pre- and post-test from students. 
% We reported these values as test scores, and it was in the range of [0, 5]. 
% Score gain, as the difference between pre- and post-test scores, was reported in the range of [-5, 5]. 
% For team scores, we simply identified test scores from each team member, using their Team-ID and averaging the scores. 
% For example, the team post-test score was the average of post-test scores of individual members of the team. 

% \textbf{RQ1.} Is there a significant difference among student score gain based on the different learning tools? 


% \subsubsection{Gender Composition}

% The distribution of males and females in the study was proportional to the overall ratio in medical fields (slight dominance of females).
% For this specific study, we are interested in addressing two research questions related to gender effects on individual and team levels. 

% \textbf{RQ2.} Is there a difference in student knowledge retention (score gain) among males and females who took part in the activity? 

% \textbf{RQ3.} Is there any meaningful pattern in the gender composition of teams and their score gain?


Based on the positive \expMeasure that we anticipate in the Tablet-3D and Screen-AR conditions, we established the following hypotheses similar to H1 and H2 regarding the learning performance (or the improvement of anatomy knowledge retention):
\begin{itemize}
    \item \textbf{H4}: The participants in the Tablet-3D condition or the Screen-AR condition will have a higher score gain than those in the Textbook condition. 
 
    (Textbook $<$ Tablet-3D, Screen-AR)
    \item \textbf{H5}: The participants in the Screen-AR condition will further have a higher score gain than those in the Tablet-3D condition. 
  
    (Tablet-3D $<$ Screen-AR)
\end{itemize}


