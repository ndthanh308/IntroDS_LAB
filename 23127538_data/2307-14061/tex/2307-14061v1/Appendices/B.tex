\section{Experiments \& Analysis}
\label{sec:supp_b}

\subsection{Experimental Settings} 
\label{sec:supp_b_exp_setting}
Since fused VLP models contain both multimodal encoder and unimodal encoder, two types of embedding can be perturbed, \textit{i.e.}, multimodal embedding, and unimodal embedding. 
The embeddings can be further divided into the full embedding (denoted as {\texttt{Multi$\rm_{full}$}} or {\texttt{Uni$\rm_{full}$}}) and [CLS] of embedding (denoted as {\texttt{Multi$\rm_{CLS}$}} or {\texttt{Uni$\rm_{CLS}$}}).
For aligned VLP models (\textit{e.g.}, CLIP \cite{Radford2021CLIP}), since the image encoder can be ViT or CNN, only [CLS] of embedding for CLIP$_{\rm ViT}$  is discussed and consider the embedding of CLIP$_{\rm CNN}$ as [CLS] of embedding \cite{Zhang2022Co-attack}.

\begin{table*}[t]
\begin{center}
\small
\renewcommand\arraystretch{1}
  \setlength{\tabcolsep}{3mm}{
		\begin{tabular}{l|l|l|ccc|ccc}
        \toprule
        \multicolumn{9}{c}{\textbf{{\textbf{\fontsize{10.5pt}{\baselineskip}\selectfont{Sep-Attack}}}}} \\
        \midrule
        \multirow{2}{*}{\textbf{\fontsize{11pt}{\baselineskip}\selectfont{Source}}} &
        \multirow{2}{*}{\textbf{\fontsize{11pt}{\baselineskip}\selectfont{Attack}}} & \multirow{2}{*}{\textbf{\fontsize{11pt}{\baselineskip}\selectfont{Target}}} & \multicolumn{3}{c|}{\textbf{\fontsize{10.5pt}{\baselineskip}\selectfont{Image-to-Text}}} & \multicolumn{3}{c}{\textbf{\fontsize{10.5pt}{\baselineskip}\selectfont{Text-to-Image}}}   \\
                    &  &   & R@1  & R@5  & R@10    & R@1  & R@5  & R@10 \\
      \hline
      \multirow{48}{*}{\textbf{ALBEF}}                      & \multirow{4}{*}{\texttt{\textbf{Text}@Uni$\rm_{full}$}}       & ALBEF  & \textbf{8.34$^\ast$} & \textbf{1.40$^\ast$} & \textbf{0.60$^\ast$} & \textbf{21.19$^\ast$} & \textbf{11.36$^\ast$} & \textbf{9.18$^\ast$}   \\
&  & TCL  & 7.90 & 1.21 & 0.30 & 19.45 & 8.87 & 6.26   \\
&  & CLIP$_{\rm ViT}$  & 23.31 & 9.55 & 4.88 & 36.05 & 20.77 & 15.98   \\
&  & CLIP$_{\rm CNN}$  & 26.05 & 9.73 & 5.97 & 38.04 & 21.83 & 16.85   \\
      \cline{2-9}
      & \multirow{4}{*}{\texttt{\textbf{Image}@Uni$\rm_{full}$}}      & ALBEF  & \textbf{62.46$^\ast$} & \textbf{50.70$^\ast$} & \textbf{45.00$^\ast$} & \textbf{68.73$^\ast$} & \textbf{57.38$^\ast$} & \textbf{52.12$^\ast$}   \\
&  & TCL  & 5.48 & 1.21 & 0.80 & 10.43 & 3.33 & 1.89   \\
&  & CLIP$_{\rm ViT}$  & 7.36 & 1.66 & 0.61 & 13.18 & 5.21 & 3.10   \\
&  & CLIP$_{\rm CNN}$  & 10.09 & 2.85 & 1.24 & 15.54 & 6.28 & 3.61   \\
      \cline{2-9}
      & \multirow{4}{*}{\texttt{\textbf{Bi}@Uni$\rm_{full}$}}     & ALBEF  & \textbf{68.93$^\ast$} & \textbf{55.21$^\ast$} & \textbf{49.40$^\ast$} & \textbf{76.33$^\ast$} & \textbf{65.46$^\ast$} & \textbf{59.66$^\ast$}   \\
&  & TCL  & 16.86 & 3.32 & 1.70 & 27.07 & 13.27 & 8.79   \\
&  & CLIP$_{\rm ViT}$  & 25.40 & 9.55 & 4.88 & 36.15 & 20.93 & 15.57   \\
&  & CLIP$_{\rm CNN}$  & 26.82 & 9.73 & 6.49 & 38.80 & 22.34 & 16.90   \\
        \cline{2-9}
       & \multirow{4}{*}{\texttt{\textbf{Text}@Multi$\rm_{full}$}}       & ALBEF  & \textbf{15.43$^\ast$} & \textbf{2.91$^\ast$} & \textbf{1.40$^\ast$} & \textbf{30.54$^\ast$} & \textbf{16.41$^\ast$} & \textbf{12.66$^\ast$}   \\
&  & TCL  & 12.64 & 2.21 & 0.60 & 28.64 & 14.62 & 10.40   \\
&  & CLIP$_{\rm ViT}$  & 26.75 & 10.49 & 5.28 & 41.33 & 24.62 & 19.15   \\
&  & CLIP$_{\rm CNN}$  & 30.27 & 12.16 & 7.11 & 43.43 & 26.64 & 20.96   \\
      \cline{2-9}
      & \multirow{4}{*}{\texttt{\textbf{Image}@Multi$\rm_{full}$}}       & ALBEF  & \textbf{35.97$^\ast$} & \textbf{25.35$^\ast$} & \textbf{21.40$^\ast$} & \textbf{50.54$^\ast$} & \textbf{40.57$^\ast$} & \textbf{37.24$^\ast$}   \\
&  & TCL  & 1.79 & 0.50 & 0.20 & 6.50 & 1.88 & 1.10   \\
&  & CLIP$_{\rm ViT}$  & 7.12 & 1.56 & 0.30 & 13.02 & 5.05 & 3.03   \\
&  & CLIP$_{\rm CNN}$  & 9.83 & 2.85 & 1.34 & 14.75 & 5.56 & 3.23   \\
      \cline{2-9}
      & \multirow{4}{*}{\texttt{\textbf{Bi}@Multi$\rm_{full}$}}       & ALBEF  & \textbf{51.09$^\ast$} & \textbf{36.97$^\ast$} & \textbf{31.90$^\ast$} & \textbf{64.17$^\ast$} & \textbf{52.87$^\ast$} & \textbf{48.73$^\ast$}   \\
&  & TCL  & 16.86 & 4.02 & 1.30 & 32.57 & 16.77 & 12.08   \\
&  & CLIP$_{\rm ViT}$  & 27.48 & 10.70 & 5.69 & 41.78 & 25.02 & 18.88   \\
&  & CLIP$_{\rm CNN}$  & 31.16 & 12.16 & 6.90 & 43.77 & 26.56 & 21.12   \\
        \cline{2-9}
      & \multirow{4}{*}{\texttt{\textbf{Text}@Uni$\rm_{CLS}$}}     & ALBEF  & \textbf{11.57$^\ast$} & \textbf{1.80$^\ast$} & \textbf{1.10$^\ast$} & \textbf{27.46$^\ast$} & \textbf{14.48$^\ast$} & \textbf{10.98$^\ast$}   \\
&  & TCL  & 12.64 & 2.51 & 0.90 & 28.07 & 14.39 & 10.26   \\
&  & CLIP$_{\rm ViT}$  & 29.33 & 11.63 & 6.30 & 43.17 & 26.37 & 19.91   \\
&  & CLIP$_{\rm CNN}$  & 32.69 & 15.43 & 8.65 & 46.11 & 28.43 & 22.14   \\
      \cline{2-9}
      & \multirow{4}{*}{\texttt{\textbf{Image}@Uni$\rm_{CLS}$}}      & ALBEF  & \textbf{52.45$^\ast$} & \textbf{36.57$^\ast$} & \textbf{30.00$^\ast$} & \textbf{58.65$^\ast$} & \textbf{44.85$^\ast$} & \textbf{38.98$^\ast$}   \\
&  & TCL  & 3.06 & 0.40 & 0.10 & 6.79 & 2.21 & 1.20   \\
&  & CLIP$_{\rm ViT}$  & 8.96 & 1.66 & 0.41 & 13.21 & 5.19 & 3.05   \\
&  & CLIP$_{\rm CNN}$  & 10.34 & 2.96 & 1.85 & 14.65 & 5.60 & 3.39   \\
      \cline{2-9}
      & \multirow{4}{*}{\texttt{\textbf{Bi}@Uni$\rm_{CLS}$}}      & ALBEF  & \textbf{65.69$^\ast$} & \textbf{47.60$^\ast$} & \textbf{42.10$^\ast$} & \textbf{73.95$^\ast$} & \textbf{59.50$^\ast$} & \textbf{53.70$^\ast$}   \\
&  & TCL  & 17.60 & 3.72 & 1.90 & 32.95 & 17.10 & 11.90   \\
&  & CLIP$_{\rm ViT}$  & 31.17 & 12.05 & 7.01 & 45.23 & 25.93 & 19.95   \\
&  & CLIP$_{\rm CNN}$  & 32.82 & 15.86 & 9.06 & 45.49 & 28.43 & 22.32   \\
    \cline{2-9}
        & \multirow{4}{*}{\texttt{\textbf{Text}@Multi$\rm_{CLS}$}}       & ALBEF  & \textbf{15.43$^\ast$} & \textbf{2.81$^\ast$} & \textbf{1.30$^\ast$} & \textbf{30.47$^\ast$} & \textbf{15.85$^\ast$} & \textbf{11.85$^\ast$}   \\
&  & TCL  & 13.59 & 3.02 & 1.20 & 30.26 & 15.42 & 11.09   \\
&  & CLIP$_{\rm ViT}$  & 27.12 & 11.94 & 6.50 & 42.53 & 25.20 & 19.36   \\
&  & CLIP$_{\rm CNN}$  & 30.78 & 13.21 & 7.52 & 44.39 & 28.07 & 21.89   \\
      \cline{2-9}
      & \multirow{4}{*}{\texttt{\textbf{Image}@Multi$\rm_{CLS}$}}       & ALBEF  & \textbf{30.76$^\ast$} & \textbf{21.24$^\ast$} & \textbf{17.10$^\ast$} & \textbf{43.85$^\ast$} & \textbf{34.84$^\ast$} & \textbf{31.44$^\ast$}   \\
&  & TCL  & 2.53 & 0.20 & 0.00 & 6.74 & 1.98 & 1.20   \\
&  & CLIP$_{\rm ViT}$  & 7.98 & 1.35 & 0.30 & 12.85 & 5.00 & 3.16   \\
&  & CLIP$_{\rm CNN}$  & 9.96 & 2.64 & 1.75 & 14.92 & 5.65 & 3.37   \\
      \cline{2-9}
      & \multirow{4}{*}{\texttt{\textbf{Bi}@Multi$\rm_{CLS}$}}      & ALBEF  & \textbf{42.13$^\ast$} & \textbf{26.95$^\ast$} & \textbf{22.20$^\ast$} & \textbf{57.76$^\ast$} & \textbf{44.91$^\ast$} & \textbf{39.95$^\ast$}   \\
&  & TCL  & 16.65 & 3.92 & 1.90 & 34.02 & 17.16 & 12.10   \\
&  & CLIP$_{\rm ViT}$  & 28.71 & 11.42 & 6.30 & 42.01 & 24.90 & 18.99   \\
&  & CLIP$_{\rm CNN}$  & 31.03 & 14.16 & 8.96 & 43.98 & 27.17 & 21.30   \\
        \bottomrule[0.3mm]
    \end{tabular}}
\end{center}
\caption{
\textbf{Attack success rates ($\%$)} with different adversarial input modalities under Sep-Attack on image-text retrieval. The adversaries are crafted on ALBEF using Flickr30K. $^\ast$ indicates white-box attacks. A higher ASR indicates better adversarial transferability.}
\label{tab:supp_t8_exp_sep_attack_albef}
\end{table*}





\begin{table*}[t]
\begin{center}
\small
\renewcommand\arraystretch{1}
  \setlength{\tabcolsep}{3mm}{
\begin{tabular}{l|l|l|ccc|ccc}
        \toprule
        \multicolumn{9}{c}{\textbf{{\textbf{\fontsize{10.5pt}{\baselineskip}\selectfont{Sep-Attack}}}}} \\
        \midrule
        \multirow{2}{*}{\textbf{\fontsize{11pt}{\baselineskip}\selectfont{Source}}} &
        \multirow{2}{*}{\textbf{\fontsize{11pt}{\baselineskip}\selectfont{Attack}}} & \multirow{2}{*}{\textbf{\fontsize{11pt}{\baselineskip}\selectfont{Target}}} & \multicolumn{3}{c|}{\textbf{\fontsize{10.5pt}{\baselineskip}\selectfont{Image-to-Text}}} & \multicolumn{3}{c}{\textbf{\fontsize{10.5pt}{\baselineskip}\selectfont{Text-to-Image}}}   \\
                    &  &   & R@1  & R@5  & R@10    & R@1  & R@5  & R@10 \\
      \hline
      \multirow{48}{*}{\textbf{TCL}}                      & \multirow{4}{*}{\texttt{\textbf{Text}@Uni$\rm_{full}$}}       & TCL  & \textbf{9.48$^\ast$} & \textbf{1.51$^\ast$} & \textbf{0.60$^\ast$} & \textbf{23.50$^\ast$} & \textbf{11.83$^\ast$} & \textbf{8.53$^\ast$}   \\
&  & ALBEF  & 9.91 & 1.80 & 0.70 & 23.64 & 12.80 & 10.07   \\
&  & CLIP$_{\rm ViT}$  & 25.89 & 8.41 & 4.57 & 39.79 & 24.22 & 18.62   \\
&  & CLIP$_{\rm CNN}$  & 28.35 & 11.21 & 7.11 & 41.96 & 26.30 & 20.42   \\
      \cline{2-9}
      & \multirow{4}{*}{\texttt{\textbf{Image}@Uni$\rm_{full}$}}      & TCL  & \textbf{45.10$^\ast$} & \textbf{34.07$^\ast$} & \textbf{28.76$^\ast$} & \textbf{53.21$^\ast$} & \textbf{38.27$^\ast$} & \textbf{32.49$^\ast$}   \\
&  & ALBEF  & 3.86 & 0.90 & 0.20 & 7.62 & 2.40 & 1.29   \\
&  & CLIP$_{\rm ViT}$  & 7.12 & 1.66 & 0.71 & 12.82 & 5.35 & 3.14   \\
&  & CLIP$_{\rm CNN}$  & 9.07 & 2.54 & 1.75 & 15.68 & 5.70 & 3.39   \\
      \cline{2-9}
      & \multirow{4}{*}{\texttt{\textbf{Bi}@Uni$\rm_{full}$}}      & TCL  & \textbf{55.95$^\ast$} & \textbf{39.80$^\ast$} & \textbf{33.77$^\ast$} & \textbf{65.38$^\ast$} & \textbf{49.58$^\ast$} & \textbf{42.28$^\ast$}   \\
&  & ALBEF  & 15.02 & 4.01 & 2.60 & 30.47 & 17.04 & 13.28   \\
&  & CLIP$_{\rm ViT}$  & 27.48 & 8.10 & 4.37 & 39.85 & 24.50 & 18.23   \\
&  & CLIP$_{\rm CNN}$  & 29.50 & 11.21 & 7.52 & 42.13 & 26.47 & 20.53   \\
        \cline{2-9}
       & \multirow{4}{*}{\texttt{\textbf{Text}@Multi$\rm_{full}$}}       & TCL  & \textbf{12.86$^\ast$} & \textbf{2.81$^\ast$} & \textbf{1.00$^\ast$} & \textbf{30.33$^\ast$} & \textbf{15.32$^\ast$} & \textbf{10.89$^\ast$}   \\
&  & ALBEF  & 13.24 & 2.61 & 1.20 & 27.13 & 15.16 & 11.28   \\
&  & CLIP$_{\rm ViT}$  & 26.75 & 9.24 & 4.57 & 40.85 & 24.57 & 18.77   \\
&  & CLIP$_{\rm CNN}$  & 28.35 & 11.31 & 6.49 & 42.95 & 26.13 & 20.71   \\
      \cline{2-9}
      & \multirow{4}{*}{\texttt{\textbf{Image}@Multi$\rm_{full}$}}       & TCL  & \textbf{52.05$^\ast$} & \textbf{41.81$^\ast$} & \textbf{35.47$^\ast$} & \textbf{63.05$^\ast$} & \textbf{51.46$^\ast$} & \textbf{46.67$^\ast$}   \\
&  & ALBEF  & 4.38 & 1.50 & 0.90 & 9.87 & 3.28 & 2.10   \\
&  & CLIP$_{\rm ViT}$  & 7.73 & 1.97 & 0.41 & 13.56 & 5.68 & 3.34   \\
&  & CLIP$_{\rm CNN}$  & 9.32 & 2.64 & 1.44 & 14.92 & 5.70 & 3.32   \\
      \cline{2-9}
      & \multirow{4}{*}{\texttt{\textbf{Bi}@Multi$\rm_{full}$}}       & TCL  & \textbf{61.96$^\ast$} & \textbf{48.64$^\ast$} & \textbf{42.08$^\ast$} & \textbf{71.74$^\ast$} & \textbf{61.07$^\ast$} & \textbf{55.83$^\ast$}   \\
&  & ALBEF  & 19.29 & 6.21 & 3.00 & 35.17 & 19.71 & 14.96   \\
&  & CLIP$_{\rm ViT}$  & 26.75 & 9.55 & 5.08 & 41.37 & 24.78 & 18.71   \\
&  & CLIP$_{\rm CNN}$  & 30.78 & 11.31 & 7.42 & 43.53 & 26.23 & 20.42   \\
        \cline{2-9}
      & \multirow{4}{*}{\texttt{\textbf{Text}@Uni$\rm_{CLS}$}}     
      & TCL  & \textbf{14.54$^\ast$} & \textbf{2.31$^\ast$} & \textbf{0.60$^\ast$} & \textbf{29.17$^\ast$} & \textbf{15.03$^\ast$} & \textbf{10.91$^\ast$}   \\
&  & ALBEF  & 11.89 & 2.20 & 0.70 & 26.82 & 14.09 & 10.80   \\
&  & CLIP$_{\rm ViT}$  & 29.69 & 12.77 & 7.62 & 44.49 & 27.47 & 21.00   \\
&  & CLIP$_{\rm CNN}$  & 33.46 & 14.38 & 9.37 & 46.07 & 29.28 & 22.59   \\
      \cline{2-9}
      & \multirow{4}{*}{\texttt{\textbf{Image}@Uni$\rm_{CLS}$}}      & TCL  & \textbf{77.87$^\ast$} & \textbf{65.13$^\ast$} & \textbf{58.72$^\ast$} & \textbf{79.48$^\ast$} & \textbf{66.26$^\ast$} & \textbf{60.36$^\ast$}   \\
&  & ALBEF  & 6.15 & 1.30 & 0.70 & 10.78 & 3.36 & 1.70   \\
&  & CLIP$_{\rm ViT}$  & 7.48 & 1.45 & 0.81 & 13.72 & 5.37 & 3.01   \\
&  & CLIP$_{\rm CNN}$  & 10.34 & 2.75 & 1.54 & 15.33 & 5.77 & 3.28   \\
      \cline{2-9}
      & \multirow{4}{*}{\texttt{\textbf{Bi}@Uni$\rm_{CLS}$}}      & TCL  & \textbf{84.72$^\ast$} & \textbf{73.07$^\ast$} & \textbf{65.43$^\ast$} & \textbf{86.07$^\ast$} & \textbf{74.67$^\ast$} & \textbf{68.83$^\ast$}   \\
&  & ALBEF  & 20.13 & 4.91 & 2.70 & 36.48 & 19.48 & 14.82   \\
&  & CLIP$_{\rm ViT}$  & 31.29 & 12.98 & 7.72 & 44.65 & 26.82 & 20.37   \\
&  & CLIP$_{\rm CNN}$  & 33.33 & 14.27 & 9.89 & 45.80 & 29.18 & 23.02   \\
    \cline{2-9}
        & \multirow{4}{*}{\texttt{\textbf{Text}@Multi$\rm_{CLS}$}}       & TCL  & \textbf{18.34$^\ast$} & \textbf{4.02$^\ast$} & \textbf{1.90$^\ast$} & \textbf{33.90$^\ast$} & \textbf{16.68$^\ast$} & \textbf{11.78$^\ast$}   \\
&  & ALBEF  & 13.66 & 2.30 & 0.90 & 27.90 & 14.11 & 10.31   \\
&  & CLIP$_{\rm ViT}$  & 27.85 & 11.32 & 6.71 & 42.01 & 24.95 & 18.88   \\
&  & CLIP$_{\rm CNN}$  & 30.27 & 13.95 & 8.34 & 44.32 & 27.58 & 21.23   \\
      \cline{2-9}
      & \multirow{4}{*}{\texttt{\textbf{Image}@Multi$\rm_{CLS}$}}       & TCL  & \textbf{37.41$^\ast$} & \textbf{28.04$^\ast$} & \textbf{24.15$^\ast$} & \textbf{48.93$^\ast$} & \textbf{39.01$^\ast$} & \textbf{35.72$^\ast$}   \\
&  & ALBEF  & 2.92 & 0.90 & 0.50 & 8.07 & 2.65 & 1.62   \\
&  & CLIP$_{\rm ViT}$  & 7.48 & 1.77 & 0.41 & 13.34 & 4.91 & 3.10   \\
&  & CLIP$_{\rm CNN}$  & 9.58 & 3.07 & 1.54 & 15.40 & 5.26 & 3.25   \\
      \cline{2-9}
      & \multirow{4}{*}{\texttt{\textbf{Bi}@Multi$\rm_{CLS}$}}      & TCL  & \textbf{47.31$^\ast$} & \textbf{34.77$^\ast$} & \textbf{29.66$^\ast$} & \textbf{60.31$^\ast$} & \textbf{48.07$^\ast$} & \textbf{43.36$^\ast$}   \\
&  & ALBEF  & 18.87 & 5.21 & 2.70 & 34.03 & 18.17 & 13.12   \\
&  & CLIP$_{\rm ViT}$  & 28.47 & 11.63 & 6.30 & 42.53 & 25.53 & 19.06   \\
&  & CLIP$_{\rm CNN}$  & 31.03 & 14.06 & 8.55 & 44.46 & 27.00 & 20.74   \\
        \bottomrule[0.3mm]
    \end{tabular}}
\end{center}
\caption{
\textbf{Attack success rates ($\%$)} with different adversarial input modalities under Sep-Attack on image-text retrieval. The adversaries are crafted on TCL using Flickr30K. $^\ast$ indicates white-box attacks. A higher ASR indicates better adversarial transferability.}
\label{tab:supp_t9_exp_sep_attack_tcl}
\end{table*}





\begin{table*}[t]
\begin{center}
\small
\renewcommand\arraystretch{1}
  \setlength{\tabcolsep}{3mm}{
\begin{tabular}{l|l|l|ccc|ccc}
        \toprule
        \multicolumn{9}{c}{\textbf{{\textbf{\fontsize{10.5pt}{\baselineskip}\selectfont{Sep-Attack}}}}} \\
        \midrule
        \multirow{2}{*}{\textbf{\fontsize{10pt}{\baselineskip}\selectfont{Source}}} &
        \multirow{2}{*}{\textbf{\fontsize{10pt}{\baselineskip}\selectfont{Attack}}} & \multirow{2}{*}{\textbf{\fontsize{10pt}{\baselineskip}\selectfont{Target}}} & \multicolumn{3}{c|}{\textbf{\fontsize{10pt}{\baselineskip}\selectfont{Image-to-Text}}} & \multicolumn{3}{c}{\textbf{\fontsize{10pt}{\baselineskip}\selectfont{Text-to-Image}}}   \\
                    &  &   & R@1  & R@5  & R@10    & R@1  & R@5  & R@10 \\
      \hline
      \multirow{12}{*}{\textbf{CLIP$_{\rm ViT}$}}                      & \multirow{4}{*}{\texttt{\textbf{Text}@Uni}}       & CLIP$_{\rm ViT}$   & \textbf{28.34$^\ast$} & \textbf{11.73$^\ast$} & \textbf{6.81$^\ast$} & \textbf{39.08$^\ast$} & \textbf{24.08$^\ast$} & \textbf{17.44$^\ast$} \\
      &     & CLIP$_{\rm CNN}$  & 30.40 & 11.63 & 5.97 & 37.43 & 24.96 & 18.66  \\
      &     & ALBEF  & 9.59 & 1.30 & 0.40 & 22.64 & 10.95 & 8.17 \\
      &     & TCL    & 11.80 & 1.91 & 0.70 & 25.07 & 12.92 & 8.90  \\
      \cline{2-9}
      & \multirow{4}{*}{\texttt{\textbf{Image}@Uni}}       & CLIP$_{\rm ViT}$   & \textbf{70.92$^\ast$} & \textbf{50.05$^\ast$} & \textbf{42.28$^\ast$} & \textbf{78.61$^\ast$} & \textbf{60.78$^\ast$} & \textbf{51.50$^\ast$}  \\
      &     & CLIP$_{\rm CNN}$  & 5.36 & 1.16 & 0.72 & 8.44 & 2.35 & 1.54 \\
      &     & ALBEF  & 2.50 & 0.40 & 0.10 & 4.93 & 1.44 & 1.01  \\
      &     & TCL    & 4.85 & 0.20 & 0.20 & 8.17 & 2.27 & 1.46  \\
      \cline{2-9}
      & \multirow{4}{*}{\texttt{\textbf{Bi}@Uni}}       & CLIP$_{\rm ViT}$   & \textbf{79.75$^\ast$} & \textbf{63.03$^\ast$} & \textbf{53.76$^\ast$} & \textbf{86.79$^\ast$} & \textbf{75.24$^\ast$} & \textbf{67.84$^\ast$} \\
      &     & CLIP$_{\rm CNN}$  & 30.78 & 12.16 & 6.39 & 39.76 & 25.62 & 19.34 \\
      &     & ALBEF  & 9.59 & 1.30 & 0.50 & 23.25 & 11.22 & 8.01  \\
      &     & TCL    & 11.38 & 2.11 & 0.90 & 25.60 & 12.92 & 9.14 \\
        \hline
      \multirow{12}{*}{\textbf{CLIP$_{\rm CNN}$}}                      & \multirow{4}{*}{\texttt{\textbf{Text}@Uni}}       & CLIP$_{\rm CNN}$   & \textbf{30.40$^\ast$} & \textbf{13.00$^\ast$} & \textbf{7.31$^\ast$} & \textbf{40.10$^\ast$} & \textbf{26.71$^\ast$} & \textbf{20.85$^\ast$} \\
      &     & CLIP$_{\rm ViT}$  & 27.12 & 11.21 & 6.81 & 37.44 & 23.48 & 17.66 \\
      &     & ALBEF  & 8.86 & 1.50 & 0.60 & 23.27 & 11.34 & 8.41 \\
      &     & TCL    & 12.33 & 2.01 & 0.90 & 25.48 & 13.25 & 8.81  \\
      \cline{2-9}
      & \multirow{4}{*}{\texttt{\textbf{Image}@Uni}}       & CLIP$_{\rm CNN}$   & \textbf{86.46$^\ast$} & \textbf{69.13$^\ast$} & \textbf{61.17$^\ast$} & \textbf{92.25$^\ast$} & \textbf{81.00$^\ast$} & \textbf{75.04$^\ast$} \\
      &     & CLIP$_{\rm ViT}$  & 1.10 & 0.52 & 0.41 & 6.60 & 2.73 & 1.48 \\
      &     & ALBEF  & 2.09 & 0.30 & 0.10 & 4.82 & 1.29 & 0.87  \\
      &     & TCL    & 4.00 & 0.40 & 0.20 & 7.81 & 2.09 & 1.34 \\
      \cline{2-9}
      & \multirow{4}{*}{\texttt{\textbf{Bi}@Uni}}       & CLIP$_{\rm CNN}$   & \textbf{91.44$^\ast$} & \textbf{78.54$^\ast$} & \textbf{71.58$^\ast$} & \textbf{95.44$^\ast$} & \textbf{88.48$^\ast$} & \textbf{82.88$^\ast$} \\
      &     & CLIP$_{\rm ViT}$  & 28.34 & 10.8 & 6.30 & 39.43 & 24.34 & 18.36 \\
      &     & ALBEF  & 8.55 & 1.50 & 0.60 & 23.41 & 11.38 & 8.23  \\
      &     & TCL    & 12.64 & 1.91 & 0.70 & 26.12 & 13.44 & 8.96 \\
        \bottomrule[0.3mm]
    \end{tabular}}
\end{center}
\caption{
\textbf{Attack success rates ($\%$)} with different adversarial input modalities under Sep-Attack on image-text retrieval. The adversaries are crafted on CLIP using Flickr30K. $^\ast$ indicates white-box attacks. A higher ASR indicates better adversarial transferability.}
\label{tab:supp_t10_exp_sep_attack_clip}
\end{table*}





\begin{table*}[t]
\begin{center}
\small
\renewcommand\arraystretch{1}
  \setlength{\tabcolsep}{3mm}{
\begin{tabular}{l|l|l|ccc|ccc}
        \toprule
        \multicolumn{9}{c}{\textbf{{\textbf{\fontsize{10.5pt}{\baselineskip}\selectfont{Co-Attack}}}}} \\
        \midrule
        \multirow{2}{*}{\textbf{\fontsize{11pt}{\baselineskip}\selectfont{Source}}} &
        \multirow{2}{*}{\textbf{\fontsize{11pt}{\baselineskip}\selectfont{Attack}}} & \multirow{2}{*}{\textbf{\fontsize{11pt}{\baselineskip}\selectfont{Target}}} & \multicolumn{3}{c|}{\textbf{\fontsize{10.5pt}{\baselineskip}\selectfont{Image-to-Text}}} & \multicolumn{3}{c}{\textbf{\fontsize{10.5pt}{\baselineskip}\selectfont{Text-to-Image}}}   \\
                    &  &   & R@1  & R@5  & R@10    & R@1  & R@5  & R@10 \\
      \hline
      \multirow{12}{*}{\textbf{ALBEF}}                      & \multirow{4}{*}{\texttt{\textbf{Text}@Multi}}       & ALBEF   & \textbf{9.18$^\ast$}    & \textbf{1.50$^\ast$}   & \textbf{1.00$^\ast$}  & \textbf{21.70$^\ast$}   & \textbf{11.96$^\ast$}   & \textbf{9.22$^\ast$}  \\
      &     & TCL  & 9.38   & 1.31   & 0.30   & 20.40  & 9.74   & 6.80  \\
      &     & CLIP$_{\rm ViT}$    & 20.98   & 7.79   & 4.57   & 31.73   & 19.13   & 14.63  \\
      &     & CLIP$_{\rm CNN}$    & 22.48   & 7.40  & 4.12  & 31.94    & 21.36  & 15.59  \\
      \cline{2-9}
      & \multirow{4}{*}{\texttt{\textbf{Image}@Multi}}       & ALBEF   & \textbf{75.50$^\ast$}    & \textbf{59.22$^\ast$}   & \textbf{53.30$^\ast$}   & \textbf{83.63$^\ast$}   & \textbf{75.14$^\ast$}   & \textbf{70.32$^\ast$}  \\
      &     & TCL  & 4.64   & 1.21   & 0.50   & 11.33  & 3.72   & 2.25  \\
      &     & CLIP$_{\rm ViT}$    & 7.24   & 1.97   & 0.51   & 13.53   & 5.23   & 3.01  \\
      &     & CLIP$_{\rm CNN}$    & 10.09   & 2.85  & 1.65  & 15.27    & 6.11  & 3.52  \\
      \cline{2-9}
      & \multirow{4}{*}{\texttt{\textbf{Bi}@Multi}}       & ALBEF   & \textbf{77.16$^\ast$}    & \textbf{64.60$^\ast$}   & \textbf{58.37$^\ast$}   & \textbf{83.86$^\ast$}   & \textbf{74.63$^\ast$}   & \textbf{70.13$^\ast$}  \\
      &     & TCL  & 15.21   & 4.19   & 1.47   & 29.49  & 14.97   & 10.55  \\
      &     & CLIP$_{\rm ViT}$    & 23.60   & 7.82   & 3.93   & 36.48   & 21.09   & 15.76  \\
      &     & CLIP$_{\rm CNN}$    & 25.12   & 8.42  & 5.39  & 38.89    & 22.38  & 17.49  \\
        \hline
    \multirow{12}{*}{\textbf{TCL}}                      & \multirow{4}{*}{\texttt{\textbf{Text}@Multi}}       
    & TCL   & \textbf{12.86}$^\ast$ &\textbf{2.81}$^\ast$ &\textbf{1.0}$^\ast$ &\textbf{30.33}$^\ast$ &\textbf{15.32}$^\ast$ &\textbf{10.89}$^\ast$  \\
      &     & ALBEF  & 13.24 & 2.61 & 1.2 & 27.13 & 15.16 & 11.28  \\
      &     & CLIP$_{\rm ViT}$    & 25.28 & 9.87 & 5.79 & 37.11 & 22.85 & 17.25  \\
      &     & CLIP$_{\rm CNN}$    & 26.18 & 11.21 & 5.25 & 37.84 & 24.65 & 18.71 \\
      \cline{2-9}
      & \multirow{4}{*}{\texttt{\textbf{Image}@Multi}}       
      & TCL   & \textbf{72.5}$^\ast$ &\textbf{55.98}$^\ast$ &\textbf{46.49}$^\ast$ &\textbf{79.26}$^\ast$ &\textbf{64.65}$^\ast$ &\textbf{56.99}$^\ast$  \\
      &     & ALBEF  & 5.94 & 1.6 & 0.8 & 12.16 & 3.79 & 2.26  \\
      &     & CLIP$_{\rm ViT}$    & 7.85 & 1.97 & 0.61 & 13.43 & 5.37 & 3.31  \\
      &     & CLIP$_{\rm CNN}$    & 9.71 & 2.85 & 1.65 & 15.44 & 5.7 & 3.37  \\
      \cline{2-9}
      & \multirow{4}{*}{\texttt{\textbf{Bi}@Multi}}       
      & TCL   & \textbf{78.08}$^\ast$ &\textbf{65.53}$^\ast$ &\textbf{56.81}$^\ast$ &\textbf{87.43}$^\ast$ &\textbf{75.23}$^\ast$ &\textbf{68.87}$^\ast$  \\
      &     & ALBEF  & 22.94 & 6.61 & 3.6 & 40.13 & 22.72 & 17.51  \\
      &     & CLIP$_{\rm ViT}$    & 27.98 & 9.66 & 5.08 & 41.46 & 25.11 & 18.99  \\
      &     & CLIP$_{\rm CNN}$    & 30.78 & 12.47 & 7.52 & 44.19 & 26.93 & 20.63  \\
        \hline
      \multirow{12}{*}{\textbf{CLIP$_{\rm ViT}$}}                      & \multirow{4}{*}{\texttt{\textbf{Text}@Uni}}       & CLIP$_{\rm ViT}$   & \textbf{28.34$^\ast$}    & \textbf{11.73$^\ast$}   & \textbf{6.81$^\ast$}   & \textbf{38.89$^\ast$}   & \textbf{24.08$^\ast$}   & \textbf{17.42$^\ast$}  \\
      &     & CLIP$_{\rm CNN}$  & 29.89   & 11.52   & 5.87   & 37.36  & 24.97  & 18.62  \\
      &     & ALBEF    & 7.61   & 1.00   & 0.30   & 19.97   & 9.58   & 6.59  \\
      &     &  TCL   & 8.43   & 0.90  & 0.30  & 20.90    & 9.96  & 7.03  \\
      \cline{2-9}
      & \multirow{4}{*}{\texttt{\textbf{Image}@Uni}}       & CLIP$_{\rm ViT}$   & \textbf{87.73$^\ast$}    & \textbf{78.09$^\ast$}   & \textbf{72.05$^\ast$}   & \textbf{91.72$^\ast$}   & \textbf{83.32$^\ast$}   & \textbf{78.67$^\ast$}  \\
      &     & CLIP$_{\rm CNN}$  & 7.66  & 1.90   & 1.44   & 9.37  & 3.90   & 2.53  \\
      &     & ALBEF    & 2.50   & 0.60   & 0.20   & 5.80   & 1.78   & 1.11  \\
      &     &  TCL   & 5.27  & 0.40  & 0.20  & 9.12   & 2.75  & 1.48  \\
      \cline{2-9}
      & \multirow{4}{*}{\texttt{\textbf{Bi}@Uni}}       & CLIP$_{\rm ViT}$   & \textbf{93.25$^\ast$}    & \textbf{84.88$^\ast$}   & \textbf{78.96$^\ast$}   & \textbf{95.86$^\ast$}   & \textbf{90.83$^\ast$}   & \textbf{87.36$^\ast$}  \\
      &     & CLIP$_{\rm CNN}$  & 32.52   & 13.78   & 7.52   & 41.82  & 26.77   & 21.10  \\
      &     & ALBEF    & 10.57   & 1.87   & 0.63   & 24.33   & 11.74   & 8.41  \\
      &     &  TCL   & 11.94   & 2.38  & 1.07  & 26.69    & 13.80  & 9.46  \\
        \hline
    \multirow{12}{*}{\textbf{CLIP$_{\rm CNN}$}}                      & \multirow{4}{*}{\texttt{\textbf{Text}@Uni}}       & CLIP$_{\rm CNN}$   & \textbf{30.40$^\ast$}    & \textbf{13.11$^\ast$}   & \textbf{7.21$^\ast$}   & \textbf{40.03$^\ast$}   & \textbf{26.79$^\ast$}   & \textbf{20.74$^\ast$}  \\
      &     & CLIP$_{\rm ViT}$  & 26.99   & 11.11   & 6.81   & 37.37  & 23.48  & 17.64  \\
      &     & ALBEF    & 7.72   & 0.90   & 0.50   & 20.79   & 9.84   & 6.98  \\
      &     &  TCL   & 9.69   & 1.31  & 0.30  & 21.67    & 10.73  & 7.49  \\
      \cline{2-9}
      & \multirow{4}{*}{\texttt{\textbf{Image}@Uni}}       & CLIP$_{\rm CNN}$   & \textbf{88.12$^\ast$}    & \textbf{79.70$^\ast$}   & \textbf{74.87$^\ast$}   & \textbf{93.69$^\ast$}   & \textbf{87.66$^\ast$}   & \textbf{83.03$^\ast$}  \\
      &     & CLIP$_{\rm ViT}$  & 1.84  & 0.10   & 0.30   & 5.51  & 2.50   & 1.02  \\
      &     & ALBEF    & 1.98   & 0.30   & 0.20   & 5.12  & 1.42   & 0.91  \\
      &     &  TCL   & 4.74   & 0.50  & 0.10  & 7.95    & 2.32  & 1.42  \\
      \cline{2-9}
      & \multirow{4}{*}{\texttt{\textbf{Bi}@Uni}}       & CLIP$_{\rm CNN}$   & \textbf{94.76$^\ast$}    & \textbf{87.03$^\ast$}   & \textbf{82.08$^\ast$}   & \textbf{96.89$^\ast$}   & \textbf{92.87$^\ast$}   & \textbf{89.25$^\ast$}  \\
      &     & CLIP$_{\rm ViT}$  & 28.79   & 11.63   & 6.40   & 40.03  & 24.60  & 18.83  \\
      &     & ALBEF    & 8.79  & 1.53   & 0.60   & 23.74   & 11.75   & 8.42  \\
      &     &  TCL   & 13.10   & 2.31  & 0.93  & 26.07    & 13.53  & 9.23  \\
        \bottomrule[0.3mm]
    \end{tabular}}
\end{center}
\caption{
\textbf{Attack success rates ($\%$)} with different adversarial input modalities under Co-Attack on image-text retrieval. The adversaries are crafted using Flickr30K. $^\ast$ indicates white-box attacks. A higher ASR indicates better adversarial transferability.}
\vspace{-5pt}
\label{tab:supp_t11_exp_co_attack}
\end{table*}





\begin{sidewaystable*}[t]
	\centering
	\footnotesize 
	\renewcommand\arraystretch{0.9}
\setlength{\tabcolsep}{2pt}
		\scalebox{0.94}[0.94]{
		\begin{tabular}{ @{\extracolsep{\fill}} l|l|ccc|ccc|ccc|ccc} 
        \toprule[0.3mm]
        \multicolumn{14}{c}{\textbf{Flickr30K  (Image-Text Retrieval)}} \\ \midrule[0.3mm]
			& &  \multicolumn{3}{c}{\textbf{ALBEF}} & \multicolumn{3}{c}{\textbf{TCL}} & \multicolumn{3}{c}{\textbf{CLIP$_{\rm ViT}$}} & \multicolumn{3}{c}{\textbf{CLIP$_{\rm CNN}$}}  \\
			\cmidrule{3-14}
			\multirow{-2}{*}{\textbf{Source}} &\multirow{-2}{*}{\textbf{Attack}} & {R@1} & {R@5} & {R@10} & {R@1} & {R@5} & {R@10} & {R@1} & {R@5} & {R@10} & {R@1} & {R@5} & {R@10} \\
			\midrule
   %%%% ALBEF 
			\multirow{5}{*}{\rotatebox[origin=c]{0}{\textbf{ALBEF}}} 
            & PGD 
                & 52.45$^\ast$ & 36.57$^\ast$ & 30.00$^\ast$   
                & 3.06 & 0.40 & 0.10   
                & 8.96 & 1.66 & 0.41   
                & 10.34 & 2.96 & 1.85  \\
            & BERT-Attack 
                & 11.57$^\ast$ & 1.80$^\ast$ & 1.10$^\ast$  
                & 12.64 & 2.51 & 0.90 
                & 29.33 & 11.63 & 6.30 
                & 32.69 & 15.43 & 8.65  \\
            & Sep-Attack 
                & 65.69$^\ast$ & 47.60$^\ast$ & 42.10$^\ast$
                & 17.60 & 3.72 & 1.90 
                & 31.17 & 12.05 & 7.01
                & 32.82 & 15.86 & 9.06  \\
            & Co-Attack 
                & 77.16$^\ast$  & 64.60$^\ast$  & 58.37$^\ast$  
                & 15.21 & 4.19 & 1.47 
                & 23.60 & 7.82 & 3.93 
                & 25.12 & 8.42 & 5.39 \\
			& \cellcolor{gray! 20} SGA  
                & \cellcolor{gray! 20}\textbf{97.24$\pm$0.22$^\ast$} 
                & \cellcolor{gray! 20}\textbf{94.09$\pm$0.42$^\ast$} 
                & \cellcolor{gray! 20}\textbf{92.30$\pm$0.28$^\ast$}  
                & \cellcolor{gray! 20}\textbf{45.42$\pm$0.60} 
                & \cellcolor{gray! 20}\textbf{24.93$\pm$0.15} 
                & \cellcolor{gray! 20}\textbf{16.48$\pm$0.49} 
                & \cellcolor{gray! 20}\textbf{33.38$\pm$0.35} 
                & \cellcolor{gray! 20}\textbf{13.50$\pm$0.30} 
                & \cellcolor{gray! 20}\textbf{9.04$\pm$0.15} 
                & \cellcolor{gray! 20}\textbf{34.93$\pm$0.99} 
                & \cellcolor{gray! 20}\textbf{17.07$\pm$0.23} 
                & \cellcolor{gray! 20}\textbf{10.45$\pm$0.95} \\
			\midrule
   %%%% TCL 
			\multirow{5}{*}{\rotatebox[origin=c]{0}{\textbf{TCL}}} 
            & PGD 
                & 6.15 & 1.30 & 0.70
                & 77.87$^\ast$ & 65.13$^\ast$ & 58.72$^\ast$
                & 7.48 & 1.45 & 0.81
                & 10.34 & 2.75 & 1.54 \\
            & BERT-Attack 
                & 11.89 & 2.20 & 0.70
                & 14.54$^\ast$ & 2.31$^\ast$ & 0.60$^\ast$
                & 29.69 & 12.77 & 7.62
                & 33.46 & 14.38 & 9.37 \\
            & Sep-Attack    
                & 20.13 & 4.91 & 2.70 
                & 84.72$^\ast$ & 73.07$^\ast$ & 65.43$^\ast$
                & 31.29 & 12.98 & 7.72
                & 33.33 & 14.27 & 9.89 \\
            & Co-Attack 
                & 23.15 & 6.98  & 3.63
                & 77.94$^\ast$  & 64.26$^\ast$  & 56.18$^\ast$  
                & 27.85 & 9.80  & 5.22 
                & 30.74 & 12.09 & 7.28 \\
			& \cellcolor{gray! 20} SGA  
                & \cellcolor{gray! 20}\textbf{48.91$\pm$0.74} 
                & \cellcolor{gray! 20}\textbf{30.86$\pm$0.28} 
                & \cellcolor{gray! 20}\textbf{23.10$\pm$0.42} 
                & \cellcolor{gray! 20}\textbf{98.37$\pm$0.08$^\ast$}  
                & \cellcolor{gray! 20}\textbf{96.53$\pm$0.07$^\ast$} 
                & \cellcolor{gray! 20}\textbf{94.99$\pm$0.28$^\ast$} 
                & \cellcolor{gray! 20}\textbf{33.87$\pm$0.18} 
                & \cellcolor{gray! 20}\textbf{15.21$\pm$0.07} 
                & \cellcolor{gray! 20}\textbf{9.46$\pm$0.43} 
                & \cellcolor{gray! 20}\textbf{37.74$\pm$0.27} 
                & \cellcolor{gray! 20}\textbf{17.86$\pm$0.30} 
                & \cellcolor{gray! 20}\textbf{11.74$\pm$0.00}  \\
			\midrule
   %%% CLIP_ViT
			\multirow{5}{*}{\rotatebox[origin=c]{0}{\textbf{CLIP$_{\rm ViT}$}}} 

            & PGD 
                & 2.50 & 0.40 & 0.10
                & 4.85 & 0.20 & 0.20
                & 70.92$^\ast$ & 50.05$^\ast$ & 42.28$^\ast$
                & 5.36 & 1.16 & 0.72 \\
            & BERT-Attack 
                & 9.59 & 1.30 & 0.40
                & 11.80 & 1.91 & 0.70
                & 28.34$^\ast$ & 11.73$^\ast$ & 6.81$^\ast$
                & 30.40 & 11.63 & 5.97 \\
            & Sep-Attack        
                & 9.59 & 1.30 & 0.50 & 11.38 & 2.11 & 0.90 & 79.75$^\ast$ & 63.03$^\ast$ & 53.76$^\ast$ & 30.78 & 12.16 & 6.39 \\ 
            & Co-Attack 
                & 10.57 & 1.87 & 0.63
                & 11.94 & 2.38 & 1.07
                & 93.25$^\ast$ & 84.88$^\ast$ & 78.96$^\ast$ 
                & 32.52 & 13.78 & 7.52  \\
			& \cellcolor{gray! 20}SGA  
                & \cellcolor{gray! 20}\textbf{13.40$\pm$0.07} 
                & \cellcolor{gray! 20}\textbf{2.46$\pm$0.08} 
                & \cellcolor{gray! 20}\textbf{1.35$\pm$0.07} 
                & \cellcolor{gray! 20}\textbf{16.23$\pm$0.45} 
                & \cellcolor{gray! 20}\textbf{3.77$\pm$0.21} 
                & \cellcolor{gray! 20}\textbf{1.10$\pm$0.14} 
                & \cellcolor{gray! 20}\textbf{99.08$\pm$0.08$^\ast$} 
                & \cellcolor{gray! 20}\textbf{97.25$\pm$0.07$^\ast$} 
                & \cellcolor{gray! 20}\textbf{95.22$\pm$0.15$^\ast$} 
                & \cellcolor{gray! 20}\textbf{38.76$\pm$0.27}
                & \cellcolor{gray! 20}\textbf{19.45$\pm$0.00} 
                & \cellcolor{gray! 20}\textbf{11.95$\pm$0.44}  \\
			\midrule
      %%% CLIP_CNN
			\multirow{5}{*}{\rotatebox[origin=c]{0}{\textbf{CLIP$_{\rm CNN}$}}} 
            & PGD 
                & 2.09 & 0.30 & 0.10 & 4.00 & 0.40 & 0.20 & 1.10 & 0.52 & 0.41 & 86.46$^\ast$ & 69.13$^\ast$ & 61.17$^\ast$ \\
            & BERT-Attack 
                & 8.86 & 1.50 & 0.60 & 12.33 & 2.01 & 0.90 & 27.12 & 11.21 & 6.81 & 30.40$^\ast$ & 13.00$^\ast$ & 7.31$^\ast$ \\
            & Sep-Attack        
                & 8.55 & 1.50 & 0.60 & 12.64 & 1.91 & 0.70 & 28.34 & 10.8 & 6.30 & 91.44$^\ast$ & 78.54$^\ast$ & 71.58$^\ast$ \\
            & Co-Attack 
                & 8.79 & 1.53 & 0.60 
                & 13.10 & 2.31 & 0.93 
                & 28.79 & 11.63 & 6.40 
                & 94.76$^\ast$ & 87.03$^\ast$ & 82.08$^\ast$ \\
			& \cellcolor{gray! 20}SGA  
                & \cellcolor{gray! 20}\textbf{11.42$\pm$0.07} 
                & \cellcolor{gray! 20}\textbf{2.56$\pm$0.07} 
                & \cellcolor{gray! 20}\textbf{1.05$\pm$0.21} 
                & \cellcolor{gray! 20}\textbf{14.91$\pm$0.08} 
                & \cellcolor{gray! 20}\textbf{3.62$\pm$0.14} 
                & \cellcolor{gray! 20}\textbf{1.70$\pm$0.14} 
                & \cellcolor{gray! 20}\textbf{31.24$\pm$0.42} 
                & \cellcolor{gray! 20}\textbf{13.45$\pm$0.07} 
                & \cellcolor{gray! 20}\textbf{8.74$\pm$0.14} 
                & \cellcolor{gray! 20}\textbf{99.24$\pm$0.18$^\ast$} 
                & \cellcolor{gray! 20}\textbf{98.20$\pm$0.30$^\ast$} 
                & \cellcolor{gray! 20}\textbf{95.16$\pm$0.44$^\ast$}   \\
   \midrule[0.3mm]
      %%%% Flickr30K 
			\multicolumn{14}{c}{\textbf{Flickr30K (Text-Image Retrieval)}} \\ \midrule[0.3mm]
			\multirow{5}{*}{\rotatebox[origin=c]{0}{\textbf{ALBEF}}} 
   %%%% ALBEF 
            & PGD 
                & 58.65$^\ast$ & 44.85$^\ast$ & 38.98$^\ast$   
                & 6.79 & 2.21 & 1.20   
                & 13.21 & 5.19 & 3.05   
                & 14.65 & 5.60 & 3.39 \\
            & BERT-Attack 
                & 27.46$^\ast$ & 14.48$^\ast$ & 10.98$^\ast$
                & 28.07 & 14.39 & 10.26
                & 43.17 & 26.37 & 19.91
                & 46.11 & 28.43 & 22.14 \\
            & Sep-Attack 
                & 73.95$^\ast$ & 59.50$^\ast$ & 53.70$^\ast$
                & 32.95 & 17.10 & 11.90
                & \textbf{45.23} & 25.93 & 19.95
                & 45.49 & 28.43 & 22.32 \\
            & Co-Attack 
                & 83.86$^\ast$ & 74.63$^\ast$ & 70.13$^\ast$ 
                & 29.49 & 14.97 & 10.55 
                & 36.48 & 21.09 & 15.76 
                & 38.89 & 22.38 & 17.49  \\
			& \cellcolor{gray! 20}SGA  
                & \cellcolor{gray! 20}\textbf{97.28$\pm$0.15$^\ast$} 
                & \cellcolor{gray! 20}\textbf{94.27$\pm$0.04$^\ast$} 
                & \cellcolor{gray! 20}\textbf{92.58$\pm$0.03$^\ast$} 
                & \cellcolor{gray! 20}\textbf{55.25$\pm$0.06} 
                & \cellcolor{gray! 20}\textbf{36.01$\pm$0.03} 
                & \cellcolor{gray! 20}\textbf{27.25$\pm$0.13} 
                & \cellcolor{gray! 20} 44.16$\pm$0.25 
                & \cellcolor{gray! 20}\textbf{27.35$\pm$0.30} 
                & \cellcolor{gray! 20}\textbf{20.84$\pm$0.04} 
                & \cellcolor{gray! 20}\textbf{46.57$\pm$0.13} 
                & \cellcolor{gray! 20}\textbf{29.16$\pm$0.17} 
                & \cellcolor{gray! 20}\textbf{22.68$\pm$0.00} \\
			\midrule
     %%%% TCL 
			\multirow{5}{*}{\rotatebox[origin=c]{0}{\textbf{TCL}}} 
            & PGD 
                & 10.78 & 3.36 & 1.70
                & 79.48$^\ast$ & 66.26$^\ast$ & 60.36$^\ast$
                & 13.72 & 5.37 & 3.01
                & 15.33 & 5.77 & 3.28 \\
            & BERT-Attack 
                & 26.82 & 14.09 & 10.80
                & 29.17$^\ast$ & 15.03$^\ast$ & 10.91$^\ast$
                & 44.49 & 27.47 & 21.00
                & 46.07 & 29.28 & 22.59 \\
            & Sep-Attack  
                & 36.48 & 19.48 & 14.82
                & 86.07$^\ast$ & 74.67$^\ast$ & 68.83$^\ast$
                & 44.65 & 26.82 & 20.37
                & 45.80 & 29.18 & 23.02 \\
            & Co-Attack 
                & 40.04 & 22.66 & 17.23 & 85.59$^\ast$ & 74.19$^\ast$ & 68.25$^\ast$ & 41.19 & 25.22 & 19.01 & 44.11 & 26.67 & 20.66 \\
			& \cellcolor{gray! 20}SGA 
                & \cellcolor{gray! 20}\textbf{60.34$\pm$0.10} 
                & \cellcolor{gray! 20}\textbf{42.47$\pm$0.22} 
                & \cellcolor{gray! 20}\textbf{34.59$\pm$0.29} 
                & \cellcolor{gray! 20}\textbf{98.81$\pm$0.07$^\ast$} 
                & \cellcolor{gray! 20}\textbf{97.19$\pm$0.03$^\ast$} 
                & \cellcolor{gray! 20}\textbf{95.86$\pm$0.11$^\ast$} 
                & \cellcolor{gray! 20}\textbf{44.88$\pm$0.54} 
                & \cellcolor{gray! 20}\textbf{28.79$\pm$0.28} 
                & \cellcolor{gray! 20}\textbf{21.95$\pm$0.11} 
                & \cellcolor{gray! 20}\textbf{48.30$\pm$0.34} 
                & \cellcolor{gray! 20}\textbf{29.70$\pm$0.02} 
                & \cellcolor{gray! 20}\textbf{23.68$\pm$0.06}   \\
			\midrule
   %%% CLIP_ViT
			\multirow{5}{*}{\rotatebox[origin=c]{0}{\textbf{CLIP$_{\rm ViT}$}}} 
            & PGD 
                & 4.93 & 1.44 & 1.01
                & 8.17 & 2.27 & 1.46
                & 78.61$^\ast$ & 60.78$^\ast$ & 51.50$^\ast$
                & 8.44 & 2.35 & 1.54 \\
            & BERT-Attack 
                & 22.64 & 10.95 & 8.17
                & 25.07 & 12.92 & 8.90
                & 39.08$^\ast$ & 24.08$^\ast$ & 17.44$^\ast$
                & 37.43 & 24.96 & 18.66  \\
            & Sep-Attack      
                & 23.25 & 11.22 & 8.01 & 25.60 & 12.92 & 9.14 & 86.79$^\ast$ & 75.24$^\ast$ & 67.84$^\ast$ & 39.76 & 25.62 & 19.34 \\
            & Co-Attack 
                & 24.33 & 11.74 & 8.41 
                & 26.69 & 13.80 & 9.46
                & 95.86$^\ast$ & 90.83$^\ast$ & 87.36$^\ast$ 
                & 41.82 & 26.77 & 21.10 \\
			& \cellcolor{gray! 20}SGA  
                & \cellcolor{gray! 20}\textbf{27.22$\pm$0.06} 
                & \cellcolor{gray! 20}\textbf{13.21$\pm$0.00} 
                & \cellcolor{gray! 20}\textbf{9.76$\pm$0.11} 
                & \cellcolor{gray! 20}\textbf{30.76$\pm$0.07} 
                & \cellcolor{gray! 20}\textbf{16.36$\pm$0.26} 
                & \cellcolor{gray! 20}\textbf{12.08$\pm$0.06} 
                & \cellcolor{gray! 20}\textbf{98.94$\pm$0.00$^\ast$} 
                & \cellcolor{gray! 20}\textbf{97.53$\pm$0.16$^\ast$} 
                & \cellcolor{gray! 20}\textbf{96.03$\pm$0.08$^\ast$} 
                & \cellcolor{gray! 20}\textbf{47.79$\pm$0.58} 
                & \cellcolor{gray! 20}\textbf{30.36$\pm$0.36} 
                & \cellcolor{gray! 20}\textbf{24.50$\pm$0.37}  \\
			\midrule
   %%% CLIP_CNN
			\multirow{5}{*}{\rotatebox[origin=c]{0}{\textbf{CLIP$_{\rm CNN}$}}} 
             & PGD 
                & 4.82 & 1.29 & 0.87 & 7.81 & 2.09 & 1.34 & 6.60 & 2.73 & 1.48 & 92.25$^\ast$ & 81.00$^\ast$ & 75.04$^\ast$ \\
            & BERT-Attack 
                & 23.27 & 11.34 & 8.41 & 25.48 & 13.25 & 8.81 & 37.44 & 23.48 & 17.66 & 40.10$^\ast$ & 26.71$^\ast$ & 20.85$^\ast$ \\
            & Sep-Attack        
                & 23.41 & 11.38 & 8.23 & 26.12 & 13.44 & 8.96 & 39.43 & 24.34 & 18.36 & 95.44$^\ast$ & 88.48$^\ast$ & 82.88$^\ast$ \\
            & Co-Attack & 23.74 & 11.75 & 8.42 & 26.07 & 13.53 & 9.23 & 40.03 & 24.60 & 18.83 & 96.89$^\ast$ & 92.87$^\ast$ & 89.25$^\ast$ \\
			& \cellcolor{gray! 20}SGA  
                & \cellcolor{gray! 20}\textbf{24.80$\pm$0.28} 
                & \cellcolor{gray! 20}\textbf{12.32$\pm$0.15} 
                & \cellcolor{gray! 20}\textbf{8.98$\pm$0.06}
                & \cellcolor{gray! 20}\textbf{28.82$\pm$0.11} 
                & \cellcolor{gray! 20}\textbf{15.12$\pm$0.11} 
                & \cellcolor{gray! 20}\textbf{10.56$\pm$0.17} 
                & \cellcolor{gray! 20}\textbf{42.12$\pm$0.11} 
                & \cellcolor{gray! 20}\textbf{26.80$\pm$0.05} 
                & \cellcolor{gray! 20}\textbf{20.23$\pm$0.13} 
                & \cellcolor{gray! 20}\textbf{99.49$\pm$0.05$^\ast$} 
                & \cellcolor{gray! 20}\textbf{98.41$\pm$0.06$^\ast$} 
                & \cellcolor{gray! 20}\textbf{97.14$\pm$0.11$^\ast$}  \\
			\bottomrule[0.3mm]		
	\end{tabular}}
	\caption{\textbf{Attack success rate ($\%$)} of four VLP models under existing adversarial attacks and SGA.
	The source column indicates the source models used to generate the adversarial data on Flickr30K.
	 $^\ast$ indicates white-box attacks. A higher ASR indicates better adversarial transferability.}
	\vspace{-8pt}
	\label{tab:supp_t12_SGA_flickr}
\end{sidewaystable*}
\begin{sidewaystable*}[t]
	\centering
	\footnotesize
	\renewcommand\arraystretch{0.9}
\setlength{\tabcolsep}{2pt}
		\scalebox{0.94}[0.94]{
		\begin{tabular}{ @{\extracolsep{\fill}} l|l|ccc|ccc|ccc|ccc}
        \toprule[0.3mm]
        \multicolumn{14}{c}{\textbf{MSCOCO  (Image-Text Retrieval)}} \\ \midrule[0.3mm]
			& &  \multicolumn{3}{c}{\textbf{ALBEF}} & \multicolumn{3}{c}{\textbf{TCL}} & \multicolumn{3}{c}{\textbf{CLIP$_{\rm ViT}$}} & \multicolumn{3}{c}{\textbf{CLIP$_{\rm CNN}$}}  \\
			\cmidrule{3-14}
			\multirow{-2}{*}{\textbf{Source}} &\multirow{-2}{*}{\textbf{Attack}} & {R@1} & {R@5} & {R@10} & {R@1} & {R@5} & {R@10} & {R@1} & {R@5} & {R@10} & {R@1} & {R@5} & {R@10} \\
			\midrule
   %%%% ALBEF 
			\multirow{5}{*}{\rotatebox[origin=c]{0}{\textbf{ALBEF}}} 
            & PGD & 76.70$^\ast$ & 67.49$^\ast$ & 62.47$^\ast$ & 12.46 & 5.00 & 3.14 & 13.96 & 7.33 & 5.21 & 17.45 & 9.08 & 6.45 \\
& BERT-Attack & 24.39$^\ast$ & 10.67$^\ast$ & 6.75$^\ast$ & 24.34 & 9.92 & 6.25 & 44.94 & 27.97 & 22.55 & 47.73 & 29.56 & 23.10 \\
& Sep-Attack & 82.60$^\ast$ & 73.20$^\ast$ & 67.58$^\ast$ & 32.83 & 15.52 & 10.10 & 44.03 & 27.60 & 21.84 & 46.96 & 29.83 & 23.15 \\
            & Co-Attack  & 79.87$^\ast$ & 68.62$^\ast$ & 62.88$^\ast$ & 32.62 & 15.36 & 9.67 & 44.89 & 28.33 & 21.89 & 47.30 & 29.89 & 23.29  \\
			& \cellcolor{gray! 20}SGA  & \cellcolor{gray! 20}\textbf{96.75$\pm$0.11$^\ast$} & \cellcolor{gray! 20}\textbf{92.83$\pm$0.13$^\ast$} & \cellcolor{gray! 20}\textbf{90.37$\pm$0.03$^\ast$} & \cellcolor{gray! 20}\textbf{58.56$\pm$0.06} & \cellcolor{gray! 20}\textbf{39.00$\pm$0.40} & \cellcolor{gray! 20}\textbf{30.68$\pm$0.22} & \cellcolor{gray! 20}\textbf{57.06$\pm$0.51} & \cellcolor{gray! 20}\textbf{39.38$\pm$0.22} & \cellcolor{gray! 20}\textbf{31.55$\pm$0.06} & \cellcolor{gray! 20}\textbf{58.95$\pm$0.19} & \cellcolor{gray! 20}\textbf{42.49$\pm$0.13} & \cellcolor{gray! 20}\textbf{34.84$\pm$0.28}  \\
			\midrule
   %%%% TCL 
			\multirow{5}{*}{\rotatebox[origin=c]{0}{\textbf{TCL}}} 
            & PGD & 10.83 & 5.28 & 3.21 & 59.58$^\ast$ & 51.25$^\ast$ & 47.89$^\ast$ & 14.23 & 7.40 & 4.93 & 17.25 & 8.51 & 6.45 \\
            & BERT-Attack & 35.32 & 15.89 & 10.25 & 38.54$^\ast$ & 19.08$^\ast$ & 12.10$^\ast$ & 51.09 & 31.71 & 25.40 & 52.23 & 33.75 & 27.06 \\
            & Sep-Attack & 41.71 & 21.37 & 14.99 & 70.32$^\ast$ & 59.64$^\ast$ & 55.09$^\ast$ & 50.74 & 31.34 & 24.43 & 51.90 & 34.02 & 26.79 \\
            &  Co-Attack  & 46.08 & 24.87 & 17.11 & 85.38$^\ast$ & 74.73$^\ast$ & 68.23$^\ast$ & 51.62 & 31.92 & 24.87 & 52.13 & 33.80 & 27.09 \\
			& \cellcolor{gray! 20}SGA  & \cellcolor{gray! 20}\textbf{65.93$\pm$0.06} & \cellcolor{gray! 20}\textbf{49.33$\pm$0.35} & \cellcolor{gray! 20}\textbf{40.34$\pm$0.01} & \cellcolor{gray! 20}\textbf{98.97$\pm$0.04$^\ast$} & \cellcolor{gray! 20}\textbf{97.89$\pm$0.12$^\ast$} & \cellcolor{gray! 20}\textbf{96.63$\pm$0.03$^\ast$} & \cellcolor{gray! 20}\textbf{56.34$\pm$0.08} & \cellcolor{gray! 20}\textbf{39.58$\pm$0.21} & \cellcolor{gray! 20}\textbf{32.00$\pm$0.12} & \cellcolor{gray! 20}\textbf{59.44$\pm$0.20} & \cellcolor{gray! 20}\textbf{42.17$\pm$0.21} & \cellcolor{gray! 20}\textbf{34.94$\pm$0.05}  \\
			\midrule
   %%% CLIP_ViT
			\multirow{5}{*}{\rotatebox[origin=c]{0}{\textbf{CLIP$_{\rm ViT}$}}} 
& PGD & 7.24 & 3.10 & 1.65 & 10.19 & 4.23 & 2.50 & 54.79$^\ast$ & 36.21$^\ast$ & 28.57$^\ast$ & 7.32 & 3.64 & 2.79 \\
& BERT-Attack & 20.34 & 8.53 & 4.73 & 21.08 & 7.96 & 4.65 & 45.06$^\ast$ & 28.62$^\ast$ & 22.67$^\ast$ & 44.54 & 29.37 & 23.97 \\
& Sep-Attack & 23.41 & 10.33 & 6.15 & 25.77 & 11.60 & 7.45 & 68.52$^\ast$ & 52.30$^\ast$ & 43.88$^\ast$ & 43.11 & 27.22 & 21.77 \\
             &  Co-Attack  & 30.28 & 13.64 & 8.83 & 32.84 & 15.27 & 10.27 & 97.98$^\ast$ & 94.94$^\ast$ & 93.00$^\ast$ & 55.08 & 38.64 & 31.42  \\
			& \cellcolor{gray! 20}SGA  & \cellcolor{gray! 20}\textbf{33.41$\pm$0.22} & \cellcolor{gray! 20}\textbf{16.73$\pm$0.04} & \cellcolor{gray! 20}\textbf{10.98$\pm$0.25} & \cellcolor{gray! 20}\textbf{37.54$\pm$0.30} & \cellcolor{gray! 20}\textbf{19.09$\pm$0.04} & \cellcolor{gray! 20}\textbf{12.92$\pm$0.31} & \cellcolor{gray! 20}\textbf{99.79$\pm$0.03$^\ast$} & \cellcolor{gray! 20}\textbf{99.37$\pm$0.07$^\ast$} & \cellcolor{gray! 20}\textbf{98.89$\pm$0.04$^\ast$} & \cellcolor{gray! 20}\textbf{58.93$\pm$0.11} & \cellcolor{gray! 20}\textbf{44.60$\pm$0.11} & \cellcolor{gray! 20}\textbf{37.53$\pm$0.74}  \\
			\midrule
      %%% CLIP_CNN
			\multirow{5}{*}{\rotatebox[origin=c]{0}{\textbf{CLIP$_{\rm CNN}$}}} 
            & PGD & 7.01 & 3.03 & 1.77 & 10.08 & 4.20 & 2.38 & 4.88 & 2.96 & 1.71 & 76.99$^\ast$ & 63.80$^\ast$ & 56.76$^\ast$ \\
& BERT-Attack & 23.38 & 10.16 & 5.70 & 24.58 & 9.70 & 5.96 & 51.28 & 33.23 & 26.63 & 54.43$^\ast$ & 38.26$^\ast$ & 30.74$^\ast$ \\
& Sep-Attack & 26.53 & 11.78 & 6.88 & 30.26 & 13.00 & 8.61 & 50.44 & 32.71 & 25.92 & 88.72$^\ast$ & 78.71$^\ast$ & 72.77$^\ast$ \\
            &  Co-Attack  & 29.83 & 13.13 & 8.35 & 32.97 & 15.11 & 9.76 & 53.10 & 35.91 & 28.53 & 96.72$^\ast$ & 94.02$^\ast$ & 91.57$^\ast$ \\
			& \cellcolor{gray! 20}SGA  & \cellcolor{gray! 20}\textbf{31.61$\pm$0.40} & \cellcolor{gray! 20}\textbf{14.27$\pm$0.28} & \cellcolor{gray! 20}\textbf{9.36$\pm$0.01} & \cellcolor{gray! 20}\textbf{34.81$\pm$0.15} & \cellcolor{gray! 20}\textbf{17.16$\pm$0.03} & \cellcolor{gray! 20}\textbf{11.26$\pm$0.04} & \cellcolor{gray! 20}\textbf{56.62$\pm$0.06} & \cellcolor{gray! 20}\textbf{41.31$\pm$0.15} & \cellcolor{gray! 20}\textbf{32.88$\pm$0.10} & \cellcolor{gray! 20}\textbf{99.61$\pm$0.08$^\ast$} & \cellcolor{gray! 20}\textbf{99.02$\pm$0.11$^\ast$} & \cellcolor{gray! 20}\textbf{98.42$\pm$0.17$^\ast$}  \\
			\hline \midrule[0.3mm]
      %%%% MSCOCO 
			\multicolumn{14}{c}{\textbf{MSCOCO (Text-Image Retrieval)}} \\ \midrule[0.3mm]
			\multirow{5}{*}{\rotatebox[origin=c]{0}{\textbf{ALBEF}}} 
   %%%% ALBEF 
            & PGD & 86.30$^\ast$ & 78.49$^\ast$ & 73.94$^\ast$ & 17.77 & 8.36 & 5.32 & 23.10 & 12.74 & 9.43 & 23.54 & 13.26 & 9.61 \\
& BERT-Attack & 36.13$^\ast$ & 23.71$^\ast$ & 18.94$^\ast$ & 33.39 & 20.21 & 15.56 & 52.28 & 38.06 & 32.04 & 54.75 & 41.39 & 35.11 \\
& Sep-Attack & 89.88$^\ast$ & 82.60$^\ast$ & 78.82$^\ast$ & 42.92 & 27.04 & 20.65 & 54.46 & 40.12 & 33.46 & 55.88 & 41.30 & 35.18 \\
            &  Co-Attack  & 87.83$^\ast$ & 80.16$^\ast$ & 75.98$^\ast$ & 43.09 & 27.32 & 21.35 & 54.75 & 40.00 & 33.81 & 55.64 & 41.48 & 35.28 \\
			& \cellcolor{gray! 20}SGA  & \cellcolor{gray! 20}\textbf{96.95$\pm$0.08$^\ast$} & \cellcolor{gray! 20}\textbf{93.44$\pm$0.04$^\ast$} & \cellcolor{gray! 20}\textbf{91.00$\pm$0.06$^\ast$} & \cellcolor{gray! 20}\textbf{65.38$\pm$0.08} & \cellcolor{gray! 20}\textbf{47.61$\pm$0.07} & \cellcolor{gray! 20}\textbf{38.96$\pm$0.07} & \cellcolor{gray! 20}\textbf{65.25$\pm$0.09} & \cellcolor{gray! 20}\textbf{50.42$\pm$0.08} & \cellcolor{gray! 20}\textbf{43.47$\pm$0.12} & \cellcolor{gray! 20}\textbf{66.52$\pm$0.18} & \cellcolor{gray! 20}\textbf{52.44$\pm$0.28} & \cellcolor{gray! 20}\textbf{45.05$\pm$0.07} \\
			\midrule
     %%%% TCL 
			\multirow{5}{*}{\rotatebox[origin=c]{0}{\textbf{TCL}}} 
            & PGD & 16.52 & 8.40 & 5.61 & 69.53$^\ast$ & 60.88$^\ast$ & 57.56$^\ast$ & 22.28 & 12.20 & 9.10 & 23.12 & 12.77 & 9.49 \\
            & BERT-Attack & 45.92 & 30.40 & 23.89 & 48.48$^\ast$ & 31.48$^\ast$ & 24.47$^\ast$ & 58.80 & 43.10 & 36.68 & 61.26 & 46.14 & 39.54 \\
            & Sep-Attack & 52.97 & 36.33 & 28.97 & 78.97$^\ast$ & 69.79$^\ast$ & 65.71$^\ast$ & 60.13 & 44.13 & 37.32 & 61.26 & 45.99 & 38.97 \\
            &  Co-Attack  & 57.09 & 39.85 & 32.00 & 91.39$^\ast$ & 83.16$^\ast$ & 78.05$^\ast$ & 60.46 & 45.16 & 37.73 & 62.49 & 46.61 & 39.74 \\
			& \cellcolor{gray! 20}SGA  & \cellcolor{gray! 20}\textbf{73.30$\pm$0.04} & \cellcolor{gray! 20}\textbf{58.40$\pm$0.09} & \cellcolor{gray! 20}\textbf{50.96$\pm$0.17} & \cellcolor{gray! 20}\textbf{99.15$\pm$0.03$^\ast$} & \cellcolor{gray! 20}\textbf{98.17$\pm$0.02$^\ast$} & \cellcolor{gray! 20}\textbf{97.34$\pm$0.01$^\ast$} & \cellcolor{gray! 20}\textbf{63.99$\pm$0.16} & \cellcolor{gray! 20}\textbf{49.87$\pm$0.09} & \cellcolor{gray! 20}\textbf{42.46$\pm$0.10} & \cellcolor{gray! 20}\textbf{65.70$\pm$0.19} & \cellcolor{gray! 20}\textbf{51.45$\pm$0.06} & \cellcolor{gray! 20}\textbf{44.64$\pm$0.06}  \\
			\midrule
   %%% CLIP_ViT
			\multirow{5}{*}{\rotatebox[origin=c]{0}{\textbf{CLIP$_{\rm ViT}$}}} 
            & PGD & 10.75 & 4.64 & 2.91 & 13.74 & 6.77 & 4.32 & 66.85$^\ast$ & 51.80$^\ast$ & 46.02$^\ast$ & 11.34 & 6.50 & 4.66 \\
& BERT-Attack & 29.74 & 18.13 & 13.73 & 29.61 & 16.91 & 12.66 & 51.68$^\ast$ & 37.12$^\ast$ & 31.02$^\ast$ & 53.72 & 40.13 & 34.32 \\
& Sep-Attack & 34.61 & 21.00 & 16.15 & 36.84 & 22.63 & 17.03 & 77.94$^\ast$ & 66.77$^\ast$ & 60.69$^\ast$ & 49.76 & 37.51 & 31.74 \\
           &  Co-Attack  & 42.67 & 27.20 & 21.46 & 44.69 & 29.42 & 22.85 & 98.80$^\ast$ & 96.83$^\ast$ & 95.33$^\ast$ & 62.51 & 49.48 & 42.63   \\
			& \cellcolor{gray! 20}SGA  & \cellcolor{gray! 20}\textbf{44.64$\pm$0.00} & \cellcolor{gray! 20}\textbf{28.66$\pm$0.13} & \cellcolor{gray! 20}\textbf{22.64$\pm$0.09} & \cellcolor{gray! 20}\textbf{47.76$\pm$0.25} & \cellcolor{gray! 20}\textbf{32.30$\pm$0.04} & \cellcolor{gray! 20}\textbf{25.70$\pm$0.04} & \cellcolor{gray! 20}\textbf{99.79$\pm$0.00$^\ast$} & \cellcolor{gray! 20}\textbf{99.37$\pm$0.01$^\ast$} & \cellcolor{gray! 20}\textbf{98.94$\pm$0.07$^\ast$} & \cellcolor{gray! 20}\textbf{65.83$\pm$0.35} & \cellcolor{gray! 20}\textbf{53.58$\pm$0.25} & \cellcolor{gray! 20}\textbf{46.84$\pm$0.16}  \\
			\midrule
   %%% CLIP_CNN
			\multirow{5}{*}{\rotatebox[origin=c]{0}{\textbf{CLIP$_{\rm CNN}$}}} 
            & PGD & 10.62 & 4.51 & 2.76 & 13.65 & 6.39 & 4.32 & 10.70 & 6.20 & 4.52 & 84.20$^\ast$ & 73.64$^\ast$ & 67.86$^\ast$ \\
& BERT-Attack & 34.64 & 21.13 & 16.25 & 29.61 & 16.91 & 12.66 & 57.49 & 42.73 & 36.23 & 62.17$^\ast$ & 47.80$^\ast$ & 40.79$^\ast$ \\
& Sep-Attack & 39.29 & 24.04 & 18.83 & 41.51 & 26.13 & 20.17 & 57.11 & 41.89 & 35.55 & 92.49$^\ast$ & 85.84$^\ast$ & 81.66$^\ast$ \\
            &  Co-Attack  & 41.97 & 26.62 & 20.91 & 43.72 & 28.62 & 22.35 & 58.90 & 45.22 & 38.72 & 98.56$^\ast$ & 96.86$^\ast$ & 95.55$^\ast$ \\
			& \cellcolor{gray! 20}SGA  & \cellcolor{gray! 20}\textbf{43.00$\pm$0.01} & \cellcolor{gray! 20}\textbf{27.64$\pm$0.04} & \cellcolor{gray! 20}\textbf{21.74$\pm$0.00} & \cellcolor{gray! 20}\textbf{45.95$\pm$0.23} & \cellcolor{gray! 20}\textbf{30.57$\pm$0.00} & \cellcolor{gray! 20}\textbf{24.27$\pm$0.22} & \cellcolor{gray! 20}\textbf{60.77$\pm$0.02} & \cellcolor{gray! 20}\textbf{46.99$\pm$0.11} & \cellcolor{gray! 20}\textbf{40.49$\pm$0.16} & \cellcolor{gray! 20}\textbf{99.80$\pm$0.03$^\ast$} & \cellcolor{gray! 20}\textbf{99.29$\pm$0.06$^\ast$} & \cellcolor{gray! 20}\textbf{98.77$\pm$0.06$^\ast$} \\
			\bottomrule[0.3mm]
	\end{tabular}}
	\caption{\textbf{Attack success rate ($\%$)} of four VLP models under existing adversarial attacks and SGA.
	The source column indicates the source models used to generate the adversarial data on MSCOCO.
	 $^\ast$ indicates white-box attacks. A higher ASR indicates better adversarial transferability.}
	\vspace{-8pt}
	\label{tab:supp_t13_SGA_coco}
\end{sidewaystable*}




\subsection{Transferability Analysis}
\label{sec:supp_b_ana}
Table \ref{tab:supp_t8_exp_sep_attack_albef}, Table \ref{tab:supp_t9_exp_sep_attack_tcl}, Table \ref{tab:supp_t10_exp_sep_attack_clip}, and Table \ref{tab:supp_t11_exp_co_attack} show adversarial transferability among different VLP models and configurations under Sep-Attack and Co-Attack.
We report the attack success rates of the adversarial examples generated by the source model to attack the target models. 

Some observations on adversarial transferability are summarized below:
\begin{itemize}
  \item For all VLP models, attacking two modalities simultaneously shows better adversarial transferability than only attacking a single modality. This is consistent with the observation in \cite{Zhang2022Co-attack} for the white-box setting.
  \item Even though models with exact same architectures but with different pretrain objectives (\textit{e.g.}, ALBEF and TCL), the adversarial examples cannot directly pass through another model with a similar success attack rate. 
  \item The adversarial transferability from fused VLP models to aligned VLP models is higher than that from backward (\textit{e.g.}, from ALBEF or TCL to CLIP-ViT and CLIP-CNN). 
  \item Although ALBEF, TCL, and CLIP-ViT are using ViT as image-encoders, the adversarial transferability from ALBEF or TCL to CLIP-CNN will be higher than that of CLIP-ViT; similarly, the adversarial transferability of CLIP-ViT to CLIP-CNN is higher than that of CLIP-CNN to CLIP-ViT. 
\end{itemize}

\begin{table*}[t]
\begin{center}
\small
\renewcommand\arraystretch{0.81}
  \setlength{\tabcolsep}{2.5mm}{
\begin{tabular}{l|l|l|l|ccc|ccc}
        \toprule[0.3mm]
        \multirow{2}{*}{\textbf{\fontsize{10pt}{\baselineskip}\selectfont{Source}}} &
        \multirow{2}{*}{\textbf{\fontsize{10pt}{\baselineskip}\selectfont{Attack}}} &
        \multirow{2}{*}{\textbf{\fontsize{10pt}{\baselineskip}\selectfont{Target}}} & \multirow{2}{*}{\textbf{\fontsize{10pt}{\baselineskip}\selectfont{Method}}} & \multicolumn{3}{c|}{\textbf{\fontsize{10pt}{\baselineskip}\selectfont{Image-to-Text}}} & \multicolumn{3}{c}{\textbf{\fontsize{10pt}{\baselineskip}\selectfont{Text-to-Image}}}   \\
        &  &  &   & R@1  & R@5  & R@10    & R@1  & R@5  & R@10 \\
      \hline
      \multirow{16}{*}{ALBEF} & \multirow{8}{*}{\texttt{\textbf{Text}@Multi}}  & \multirow{2}{*}{ALBEF} 
            & Co-Attack   & 9.18$^\ast$ & 1.50$^\ast$ & 1.00$^\ast$ & 21.70$^\ast$ & 11.96$^\ast$ & 9.22$^\ast$  \\
      &  &  & SGA  & \textbf{13.03$^\ast$} & \textbf{2.71$^\ast$} & \textbf{1.40$^\ast$} & \textbf{26.17$^\ast$} & \textbf{14.17$^\ast$} & \textbf{10.76$^\ast$}  \\
      \cline{3-10}
      &  & \multirow{2}{*}{TCL}  & Co-Attack  & 9.38 & 1.31 & 0.30 & 20.40 & 9.74 & 6.80 \\
      &  &                       & SGA & \textbf{12.64} & \textbf{2.01} & \textbf{0.80} & \textbf{26.43} & \textbf{13.69} & \textbf{9.32}\\
      \cline{3-10}
      &  & \multirow{2}{*}{CLIP$_{\rm ViT}$} & Co-Attack  & 20.98 & 7.79 & 4.57 & 31.73 & 19.13 & 14.63  \\
      &  &                                   & SGA & \textbf{27.24} & \textbf{11.32} & \textbf{7.52} & \textbf{36.82} & \textbf{22.10} & \textbf{16.99}   \\ 
      \cline{3-10}
      &  & \multirow{2}{*}{CLIP$_{\rm CNN}$} & Co-Attack  & 22.48 & 7.40 & 4.12 & 31.94 & 21.36 & 15.59  \\
      &  &                                   & SGA & \textbf{27.97} & \textbf{13.85} & \textbf{7.62} & \textbf{37.77} & \textbf{24.82} & \textbf{18.46}   \\        
      \cline{2-10}
      
      &  \multirow{8}{*}{\texttt{\textbf{Image}@Multi}}       & \multirow{2}{*}{ALBEF} 
                & Co-Attack  & 75.50$^\ast$ & 59.22$^\ast$ & 53.30$^\ast$ & 83.63$^\ast$ & 75.14$^\ast$ & 70.32$^\ast$  \\
      &  &      & SGA & \textbf{90.82$^\ast$} & \textbf{83.27$^\ast$} & \textbf{79.00$^\ast$} & \textbf{90.08$^\ast$} & \textbf{83.35$^\ast$} & \textbf{79.32$^\ast$}    \\
      \cline{3-10}
      &  & \multirow{2}{*}{TCL}  & Co-Attack  & 4.64 & 1.21 & 0.50 & 11.33 & 3.72 & 2.25 \\
      &  &                             & SGA & \textbf{21.18} & \textbf{9.15} & \textbf{5.91} & \textbf{28.00} & \textbf{13.50} & \textbf{9.16}  \\
      \cline{3-10}
      &  & \multirow{2}{*}{CLIP$_{\rm ViT}$} & Co-Attack  & 7.24 & 1.97 & 0.51 & 13.53 & 5.23 & 3.01   \\
      &  &                                         & SGA & \textbf{10.92} & \textbf{3.53} & \textbf{1.52} & \textbf{16.72} & \textbf{6.70} & \textbf{4.34}   \\   
      \cline{3-10}
      &  & \multirow{2}{*}{CLIP$_{\rm CNN}$} & Co-Attack  & 10.09 & 2.85 & 1.65 & 15.27 & 6.11 & 3.52   \\
      &  &                                         & SGA & \textbf{12.52} & \textbf{3.91} & \textbf{2.47} & \textbf{17.77} & \textbf{7.44} & \textbf{4.65}   \\    
        \hline
        
      \multirow{16}{*}{TCL} & \multirow{8}{*}{\texttt{\textbf{Text}@Multi}}  & \multirow{2}{*}{ALBEF} 
            & Co-Attack & 13.24 & 2.61 & 1.20 & 27.13 & 15.16 & 11.28 \\
      &  &  & SGA & \textbf{10.84} & \textbf{2.71} & \textbf{0.90} & \textbf{24.77} & \textbf{12.22} & \textbf{9.30} \\
      \cline{3-10}
      &  & \multirow{2}{*}{TCL}  & Co-Attack  & 12.86 & 2.81 & 1.00 & 30.33 & 15.32 & 10.89  \\
      &  &                             & SGA & \textbf{13.38$^\ast$} & \textbf{3.72$^\ast$} & \textbf{1.00$^\ast$} & \textbf{27.17$^\ast$} & \textbf{14.06$^\ast$} & \textbf{10.07$^\ast$}  \\
      \cline{3-10}
      &  & \multirow{2}{*}{CLIP$_{\rm ViT}$} & Co-Attack  & 25.28 & 9.87 & 5.79 & 37.11 & 22.85 & 17.25 \\
      &  &                                         & SGA & \textbf{27.98} & \textbf{12.05} & \textbf{7.32} & \textbf{37.69} & \textbf{22.73} & \textbf{17.31}  \\ 
      \cline{3-10}
      &  & \multirow{2}{*}{CLIP$_{\rm CNN}$} & Co-Attack  & 26.18 & 11.21 & 5.25 & 37.84 & 24.65 & 18.71   \\
      &  &                                         & SGA  & \textbf{30.40} & \textbf{13.85} & \textbf{8.14} & \textbf{37.77} & \textbf{25.14} & \textbf{19.41}    \\        
      \cline{2-10}
      
      &  \multirow{8}{*}{\texttt{\textbf{Image}@Multi}}       & \multirow{2}{*}{ALBEF} 
               & Co-Attack  & 5.94 & 1.60 & 0.80 & 12.16 & 3.79 & 2.26  \\
      &  &     & SGA & \textbf{27.11} & \textbf{13.93} & \textbf{9.80} & \textbf{34.49} & \textbf{19.24} & \textbf{13.67}   \\
      \cline{3-10}
      &  & \multirow{2}{*}{TCL}  & Co-Attack  & 72.50 & 55.98 & 46.49 & 79.26 & 64.65 & 56.99   \\
      &  &                       & SGA & \textbf{96.00$^\ast$} & \textbf{92.16$^\ast$} & \textbf{89.28$^\ast$} & \textbf{96.86$^\ast$} & \textbf{92.70$^\ast$} & \textbf{90.19$^\ast$}  \\
      \cline{3-10}
      &  & \multirow{2}{*}{CLIP$_{\rm ViT}$} & Co-Attack  & 7.85 & 1.97 & 0.61 & 13.43 & 5.37 & 3.31   \\
      &  &                                         & SGA & \textbf{10.92} & \textbf{3.53} & \textbf{1.52} & \textbf{16.88} & \textbf{7.15} & \textbf{4.62}    \\   
      \cline{3-10}
      &  & \multirow{2}{*}{CLIP$_{\rm CNN}$} & Co-Attack  & 9.71 & 2.85 & 1.65 & 15.44 & 5.70 & 3.37  \\
      &  &                                         & SGA & \textbf{13.15} & \textbf{4.97} & \textbf{2.37} & \textbf{18.56} & \textbf{7.56} & \textbf{5.13}     \\    
        \hline

      \multirow{16}{*}{CLIP$_{\rm ViT}$} & \multirow{8}{*}{\texttt{\textbf{Text}@Multi}}  & \multirow{2}{*}{ALBEF} 
            & Co-Attack  & 7.61 & 1.00 & 0.30 & 19.97 & 9.58 & 6.59 \\
      &  &  & SGA & \textbf{8.13} & \textbf{1.20} & \textbf{0.40} & \textbf{19.50} & \textbf{8.76} & \textbf{6.59}  \\
      \cline{3-10}
      &  & \multirow{2}{*}{TCL}  & Co-Attack  & 8.43 & 0.90 & 0.30 & 20.90 & 9.96 & 7.03   \\
      &  &                             & SGA & \textbf{8.96} & \textbf{1.01} & \textbf{0.30} & \textbf{21.64} & \textbf{10.59} & \textbf{7.88}    \\
      \cline{3-10}
      &  & \multirow{2}{*}{CLIP$_{\rm ViT}$} & Co-Attack   & 28.34 & 11.73 & 6.81 & 38.89 & 24.08 & 17.42  \\
      &  &                                         & SGA & \textbf{31.78$^\ast$} & \textbf{15.16$^\ast$} & \textbf{8.43$^\ast$} & \textbf{39.43$^\ast$} & \textbf{25.58$^\ast$} & \textbf{19.30$^\ast$}  \\ 
      \cline{3-10}
      &  & \multirow{2}{*}{CLIP$_{\rm CNN}$} & Co-Attack  & 29.89 & 11.52 & 5.87 & 37.36 & 24.97 & 18.62 \\
      &  &                                         & SGA & \textbf{29.89} & \textbf{12.37} & \textbf{6.90} & \textbf{36.40} & \textbf{23.13} & \textbf{18.48}    \\        
      \cline{2-10}
      
      &  \multirow{8}{*}{\texttt{\textbf{Image}@Multi}}       & \multirow{2}{*}{ALBEF} 
              & Co-Attack  & 2.50 & 0.60 & 0.20 & 5.80 & 1.78 & 1.11 \\
      &  &     & SGA & \textbf{3.86} & \textbf{0.70} & \textbf{0.30} & \textbf{7.69} & \textbf{2.73} & \textbf{1.52}  \\
      \cline{3-10}
      &  & \multirow{2}{*}{TCL}  & Co-Attack  & 5.27 & 0.40 & 0.20 & 9.12 & 2.75 & 1.48  \\
      &  &                             & SGA & \textbf{6.43} & \textbf{0.60} & \textbf{0.20} & \textbf{10.93} & \textbf{3.47} & \textbf{2.05}   \\
      \cline{3-10}
      &  & \multirow{2}{*}{CLIP$_{\rm ViT}$} & Co-Attack  & 87.73 & 78.09 & 72.05 & 91.72 & 83.32 & 78.67 \\
      &  &                                         & SGA & \textbf{94.11$^\ast$} & \textbf{88.89$^\ast$} & \textbf{83.64$^\ast$} & \textbf{95.91$^\ast$} & \textbf{90.10$^\ast$} & \textbf{85.98$^\ast$}      \\   
      \cline{3-10}
      &  & \multirow{2}{*}{CLIP$_{\rm CNN}$} & Co-Attack  & 7.66 & 1.90 & 1.44 & 9.37 & 3.90 & 2.53   \\
      &  &                                         & SGA & \textbf{11.24} & \textbf{5.39} & \textbf{2.68} & \textbf{15.68} & \textbf{6.88} & \textbf{5.08}   \\    
        \hline

      \multirow{16}{*}{CLIP$_{\rm CNN}$} & \multirow{8}{*}{\texttt{\textbf{Text}@Multi}}  & \multirow{2}{*}{ALBEF} 
           & Co-Attack  & 7.72 & 0.90 & 0.50 & 20.79 & 9.84 & 6.98 \\
      &  &  & SGA & \textbf{7.82} & \textbf{1.30} & \textbf{0.60} & \textbf{19.93} & \textbf{9.74} & \textbf{7.16} \\
      \cline{3-10}
      &  & \multirow{2}{*}{TCL}  & Co-Attack  & 9.69 & 1.31 & 0.30 & 21.67 & 10.73 & 7.49   \\
      &  &                             & SGA & \textbf{9.59} & \textbf{1.91} & \textbf{0.60} & \textbf{21.88} & \textbf{10.96} & \textbf{7.74}     \\
      \cline{3-10}
      &  & \multirow{2}{*}{CLIP$_{\rm ViT}$} & Co-Attack  & 26.99 & 11.11 & 6.81 & 37.37 & 23.48 & 17.64   \\
      &  &                                         & SGA & \textbf{26.50} & \textbf{11.63} & \textbf{6.40} & \textbf{37.66} & \textbf{22.89} & \textbf{17.01}     \\ 
      \cline{3-10}
      &  & \multirow{2}{*}{CLIP$_{\rm CNN}$} & Co-Attack  & 30.40 & 13.11 & 7.21 & 40.03 & 26.79 & 20.74   \\
      &  &                                         & SGA & \textbf{36.27$^\ast$} & \textbf{17.34$^\ast$} & \textbf{11.02$^\ast$} & \textbf{44.29$^\ast$} & \textbf{29.16$^\ast$} & \textbf{22.82$^\ast$}      \\        
      \cline{2-10}
      
      &  \multirow{8}{*}{\texttt{\textbf{Image}@Multi}}       & \multirow{2}{*}{ALBEF} 
               & Co-Attack  & 1.98 & 0.30 & 0.20 & 5.12 & 1.42 & 0.91  \\
      &  &     & SGA  & \textbf{2.09} & \textbf{0.60} & \textbf{0.20} & \textbf{6.20} & \textbf{1.70} & \textbf{1.19}    \\
      \cline{3-10}
      &  & \multirow{2}{*}{TCL}  & Co-Attack  & 4.74 & 0.50 & 0.10 & 7.95 & 2.32 & 1.42 \\
      &  &                             & SGA & \textbf{4.85} & \textbf{0.70} & \textbf{0.30} & \textbf{9.19} & \textbf{2.63} & \textbf{1.73}      \\
      \cline{3-10}
      &  & \multirow{2}{*}{CLIP$_{\rm ViT}$} & Co-Attack  & 1.84 & 0.10 & 0.30 & 5.51 & 2.50 & 1.02    \\
      &  &                                         & SGA & \textbf{3.19} & \textbf{1.77} & \textbf{0.81} & \textbf{9.34} & \textbf{4.56} & \textbf{2.35}     \\   
      \cline{3-10}
      &  & \multirow{2}{*}{CLIP$_{\rm CNN}$} & Co-Attack & 88.12 & 79.70 & 74.87 & 93.69 & 87.66 & 83.03 \\
      &  &                                         & SGA & \textbf{92.46$^\ast$} & \textbf{86.68$^\ast$} & \textbf{81.98$^\ast$} & \textbf{96.64$^\ast$} & \textbf{91.78$^\ast$} & \textbf{87.87$^\ast$}  \\    
        \bottomrule[0.3mm]
    \end{tabular}}
\end{center}
\vspace{-10pt}
\caption{
\textbf{Attack success rates ($\%$)} on four VLP models under Co-Attack \cite{Zhang2022Co-attack} and SGA with different single adversarial input modalities. The adversaries are crafted on Flickr30K. $^\ast$ indicates white-box attacks.}
\label{tab:supp_t14_exp_unimodal}
\end{table*}







\subsection{Main Results}
\label{sec:supp_b_main_res}
We present a thorough analysis of the performance of our proposed high transferable multimodal attack method, SGA, on the popular benchmark datasets Flickr30K and MSCOCO. 
The experimental results are summarized in Table \ref{tab:supp_t12_SGA_flickr} and Table \ref{tab:supp_t13_SGA_coco}, providing a clear comparison between the performance of our SGA and existing multimodal attack methods across different attack scenarios. 
As we can see, our proposed SGA outperforms the state-of-the-art in all white-box and black-box settings.
Moreover, as illustrated in Table \ref{tab:supp_t14_exp_unimodal}, we conduct extensive experiments on Flickr30K under a unimodal scenario, with perturbed input in either the image or text modality. 
Empirical evidence suggests that even in scenarios where only query data are accessible, the performance of SGA consistently surpasses that of existing methods.

Our results suggest that the proposed SGA can serve as a promising method for evaluating the robustness of multimodal models and improving their security in real-world applications.


\subsection{Ablation Study}
\label{sec:supp_b_ablation}

This section presents the ablation experiments on the augmented multimodal data and the iterative strategy sequence of SGA. To provide a thorough analysis, detailed experimental results are presented and discussed.

\paragraph{Iterative Strategy.}
In this study, we generate adversarial examples through cross-modal guidance. 
This allows for the disruption of multimodal interactions through the collaborative generation of perturbations. 
Notably, our process follows a ``text-image-text" (t-i-t) pipeline.

We have conducted additional experiments to evaluate the effectiveness of our attack strategy. 
As shown in Table \ref{tab:supp_t15_ablation_reverse_multi_alter}, an interesting observation is that reversing the ``t-i-t" pipeline does not significantly impact the results. 
Furthermore, although adding one iteration (t-i-t-i-t) slightly enhances performance, it also doubles the computational cost. 
This suggests that our SGA is not sensitive to the exact order of the pipeline, but rather benefits from cross-modal guidance.


\paragraph{Multi-scale Image Set.}
In SGA, an augmented image set is used to generate adversarial data based on the scale-invariant property of deep learning models.
To verify the effectiveness of the augmented image set, we choose different scale ranges to build the image sets and evaluate the adversarial transferability.
As presented in Table \ref{tab:supp_t16_ablation_img}, there exists a positive correlation between transferability and the scale range, with the highest transferability observed at a scale range of $[0.50,1.50]$ with a step size of 0.25.
The experimental results show that the augmented image set plays a crucial role in increasing the transferability of the generated adversarial data.

\begin{table*}[t]
\begin{center}
\renewcommand\arraystretch{1.1}
\setlength{\tabcolsep}{3mm}
		\scalebox{0.97}[0.97]{
		    \begin{tabular}{c|ccc|ccc}
        \toprule
        \multirow{2}{*}{\textbf{\fontsize{10pt}{\baselineskip}\selectfont{Iterative Strategy}}} & \multicolumn{3}{c|}{\textbf{\fontsize{10pt}{\baselineskip}\selectfont{Image-to-Text}}} & \multicolumn{3}{c}{\textbf{\fontsize{10pt}{\baselineskip}\selectfont{Text-to-Image}}}   \\
            & R@1  & R@5  & R@10    & R@1  & R@5  & R@10 \\
      \midrule
    t-i-t       & 45.42     & 24.93    & 16.48     & 55.25     & 36.01     & 27.25     \\
    i-t-i       & 45.84     & 26.43    & 18.24     & 56.45     & 36.39     & 27.60     \\
    t-i-t-i-t   & 48.37     & 26.63    & 19.44     & 57.19     & 38.08     & 28.72  \\
        \bottomrule
    \end{tabular}}
\end{center}
\caption{ \textbf{Ablation experiment on different iterative strategies.} The dataset is Flickr30K. The source model is ALBEF and the target model is TCL. Attack success rates ($\%$) are utilized to measure the adversarial transferability. }
\vspace{-5pt}
\label{tab:supp_t15_ablation_reverse_multi_alter}
\end{table*}
\begin{table*}[t]
\begin{center}
\renewcommand\arraystretch{1.1}
\setlength{\tabcolsep}{3mm}
		\scalebox{0.97}[0.97]{
		    \begin{tabular}{c|ccc|ccc}
        \toprule
        \multirow{2}{*}{\textbf{\fontsize{10pt}{\baselineskip}\selectfont{Scales}}} & \multicolumn{3}{c|}{\textbf{\fontsize{10pt}{\baselineskip}\selectfont{Image-to-Text}}} & \multicolumn{3}{c}{\textbf{\fontsize{10pt}{\baselineskip}\selectfont{Text-to-Image}}}   \\
            & R@1  & R@5  & R@10    & R@1  & R@5  & R@10 \\
      \midrule
      $[1.00]$                & 34.04    & 13.17      & 8.62      & 44.12     & 25.95     & 19.25     \\
      $[0.75, 1.00, 1.25]$                 & 44.57    & 22.70      & 14.63     & 54.55     & 34.36     & 26.22  \\
      $[0.50, 0.75, 1.00, 1.25,1.50]$      & 45.94    & 24.82      & 16.13     & 55.21     & 35.99     & 27.15   \\
      $[0.25, 0.50, 0.75, 1.00, 1.25,1.50, 1.75]$   & 44.15   & 24.22  & 16.13  & 55.10    & 35.35  & 26.81  \\
        \bottomrule
    \end{tabular}}
\end{center}
\caption{ \textbf{Ablation experiment on the image set.} The dataset is Flickr30K. The source model is ALBEF and the target model is TCL. Attack success rates ($\%$) are utilized to measure the adversarial transferability. }
\vspace{-5pt}
\label{tab:supp_t16_ablation_img}
\end{table*}
\begin{table*}[t]
\begin{center}
\renewcommand\arraystretch{1.1}
\setlength{\tabcolsep}{3mm}
		\scalebox{0.97}[0.97]{
		    \begin{tabular}{c|ccc|ccc}
        \toprule
        \multirow{2}{*}{\textbf{\fontsize{10pt}{\baselineskip}\selectfont{Number of Captions}}} & \multicolumn{3}{c|}{\textbf{\fontsize{10pt}{\baselineskip}\selectfont{Image-to-Text}}} & \multicolumn{3}{c}{\textbf{\fontsize{10pt}{\baselineskip}\selectfont{Text-to-Image}}}   \\
            & R@1  & R@5  & R@10    & R@1  & R@5  & R@10 \\
      \midrule
      1                 & 40.04     & 18.99         & 12.53     & 51.14     & 30.93     & 23.17     \\
      2                 & 45.52     & 22.51         & 15.13     & 54.69     & 33.45     & 25.28  \\
      3                 & 45.84     & 23.82         & 15.43     & 54.67     & 34.69     & 26.58   \\
      4                 & 46.05     & 25.03         & 16.23     & 55.16     & 35.66     & 27.13         \\
      5                 & 45.94     & 24.82         & 16.13     & 55.21     & 35.99     & 27.15         \\
        \bottomrule
    \end{tabular}}
\end{center}
\vspace{-8pt}
\caption{ \textbf{Ablation experiment on the caption set. }The dataset is Flickr30K. The source model is ALBEF and the target model is TCL. Attack success rates ($\%$) are utilized to measure the adversarial transferability. }
\vspace{-5pt}
\label{tab:supp_t17_ablation_txt}
\end{table*}

\paragraph{Multi-pair Caption Set.}
The proposed SGA involves augmenting the original caption into a caption set for the purpose of generating adversarial data. 
To determine the effectiveness of the augmented caption set, various numbers of captions are utilized to construct the caption sets, and the transferability of the resulting adversarial data is evaluated. 
As illustrated in Table \ref{tab:supp_t17_ablation_txt}, the use of multiple captions in the process of crafting adversarial data is observed to have a significant positive impact on adversarial transferability. 
Experimental results demonstrate that the augmented caption set also helps enhance the transferability of the generated adversarial data.


\subsection{Visualization}
\label{sec:supp_b_vis}
Figure \ref{fig:supp_f7_visualization} depicts randomly selected original clean images and the corresponding adversarial examples, and such small perturbations are hard to be perceived.
We magnified the imperceptible perturbation by a factor of 50 for visualization.

