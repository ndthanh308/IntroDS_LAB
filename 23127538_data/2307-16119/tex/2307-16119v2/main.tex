\documentclass[12pt]{amsart}
% \documentclass{scrartcl}
\usepackage[utf8]{inputenc}
\usepackage{amssymb}
\usepackage{amsmath}
\usepackage{mathtools}  
\usepackage{tabulary}
\usepackage{booktabs}
\usepackage{setspace}
\usepackage{environ}
\usepackage{mathrsfs}
\usepackage{amsfonts}
% \usepackage{mathabx}
% \usepackage{MnSymbol}
\usepackage{pict2e}
\usepackage{enumitem}
\usepackage{tikz}
\usepackage[english]{babel}
\usepackage{amsthm}
\usepackage[english]{babel}
\usepackage[numbers]{natbib}

\DeclareMathOperator{\arcsinh}{arcsinh}
\DeclareMathOperator{\arccosh}{arccosh}
\DeclareMathOperator{\crit}{Crit}
\DeclareMathOperator{\inj}{inj}
\DeclareMathOperator{\ind}{ind}
\DeclareMathOperator{\sys}{sys}
\DeclareMathOperator{\ex}{exp}
\DeclareMathOperator{\proj}{proj}
\DeclareMathOperator{\ess}{ess}
\DeclareMathOperator{\argmax}{argmax}
\DeclareMathOperator{\spn}{Span}
\DeclareMathOperator{\Homeo}{Homeo}
\DeclareMathOperator{\Iso}{Iso}
\DeclareMathOperator{\intr}{int}
\DeclareMathOperator{\rank}{rank}
\DeclareMathOperator{\syst}{sys_T}
\DeclareMathOperator{\secsys}{sec.sys}

\makeatletter
\DeclareRobustCommand{\sector}{\mathord{\mathpalette\make@sector\relax}}
\newcommand{\make@sector}[2]{%
  \settoheight{\unitlength}{$#1x$}%
  \begin{picture}(1,1.06)
  \linethickness{.08\unitlength}
  \moveto(0.5,0)
  \lineto(0.842,1)
  \curveto(.6,1.08)(.4,1.08)(0.158,1)
  \closepath
  \strokepath
  \end{picture}%
}
\makeatother

\newcommand*\circled[1]{\tikz[baseline=(C.base)]\node[draw,circle,inner sep=1.2pt,line width=0.2mm,](C) {\small #1};\!}



% \NewEnviron{iproof}[1][Proof]{\begin{proof}[\indent\bfseries #1]\BODY\end{proof}}{}

\newtheorem{theorem}{Theorem}[section]
\newtheorem*{main}{Main Theorem}
\newtheorem{lemma}[theorem]{Lemma}
\newtheorem{proposition}[theorem]{Propostion}
\newtheorem{corollary}[theorem]{Corollary}

\theoremstyle{definition}
\newtheorem{definition}[theorem]{Definition}
\newtheorem*{definition*}{Definition}
\newtheorem{example}[theorem]{Example}
\newtheorem{xca}[theorem]{Exercise}

\theoremstyle{remark}
\newtheorem{remark}[theorem]{Remark}

\numberwithin{equation}{section}

\title{$C^2$-Morse functions on $\overline{\mathcal M}_{g,n}$}
\author{Changjie Chen}

% [An approximation of the systole function]

\usepackage{natbib}
\usepackage{graphicx}
\setstretch{1.25}

\begin{document}

\begin{abstract}
    We present a series of $C^2$-Morse functions on the Deligne-Mumford compactification $\overline{\mathcal M}_{g,n}$ of the moduli space of genus $g$ Riemann/hyperbolic surfaces with $n$ punctures. This series of functions converges to the systole function, which is topologically Morse. We will show that the critical points of our functions approach those of systole sublinearly, stratum-wise, and with the same indices.
\end{abstract}

\maketitle


\section{Introduction}
\noindent
% $\sector$
Morse functions are important in the sense of studying the Morse homology of the base space through the handle decomposition, see \cite{milnor2016morse} for more about Morse theory. The moduli space $\mathcal M_{g,n}$ of $(g,n)$-hyperbolic or Riemann surfaces, as well as its Deligne-Mumford compactification $\overline{\mathcal M}_{g,n}$, is of utmost interest while the topology is still mysterious for general $(g,n)$. Morse theory could be a powerful tool if we understand a Morse function well on either space. We will construct and study a series of Morse functions using Teichm\"uller theory.\\
\\
Note that any geodesic-length function $l_\gamma$ is a Morse function on the Teichm\"uller space. However, it does not descend to the moduli space. It is also not possible to take the arithmetic average over the mapping class group orbit of $\gamma$ because of the size of the mapping class group.\\
\\
While not smooth, topological Morse property has been shown for the 
% Another way to potentially get Morse functions on $\mathcal M_{g,n}$ is through the 
$\textit{systole}$ function, i.e., the minimum of all geodesic-length functions. The systole function depends only on the hyperbolic structure of the base surface and therefore is mapping class group invariant and descends to the moduli space. However, as a minimum function, it is only continuous. Explicit examples with Fenchel-Nielsen coordinates can show it is not even $C^1$.\\
\\
P. Schmutz Schaller showed in \cite{schaller1998geometry} that when $n>0$, the systole function is topologically Morse on $\mathcal M_{g,n}$, which is an analogy of (smooth) Morse functions in topological settings, see \cite{morse1959topologically}. The case of $(g,n)=(1,1)$ has been well studied, for instance, see \cite{ahlfors1953complex}. Schmutz Schaller also calculated all the critical points and the index at each of them for $(g,n)=(0,5), (0,6), (1,2), (2,0)$ in addition to $(0,4)$ and $(1,1)$.\\
\\
The general case was proved by H. Akrout in \cite{akrout2003singularites} a few years later that the systole function is topologically Morse on the moduli space of a hyperbolic surface with or without punctures. Akrout defined a $\textit{minimal\ gradient}$ at a surface $X\in\mathcal T$ to be the gradient vector of the geodesic-length function for a shortest geodesic on $X$ and he developed a theory on so-called \textit{(semi-)eutacticity} differentiating surfaces by the relative position of all minimal gradients for his proof.\\
\\
Aside from the stories originating from geodesic-length functions, as another candidate, $-\log\det^*\Delta_m$ on the space of hyperbolic structures $\{m\}$ on a given surface was conjectured by Sarnak to be a Morse function, where $\det^*\Delta_m$ is the zeta regularized product of all non-zero eigenvalues of $\Delta_m$, see \cite{quine1997extremal}. It seems that very little study followed and this conjecture still remains wide open.\\
\\
In the most popular version, a Morse function is required to be smooth, but to define critical points and nondegeneracy, and to construct the handle decomposition and perform Morse theory, it is allowable to weaken the smoothness condition to $C^2$. We may call them $C^2$-Morse functions. In this paper, we introduce a series of functions named $\syst$, that converges to the systole function and will have (at least) $C^2$-Morse property and more:
\begin{definition*}
$$\syst(X)=-T\log\sum_{\gamma \text{ s.c.g. on } X} e^{-\frac1Tl_\gamma(X)},$$
\end{definition*}
\noindent
where s.c.g. stands for simple closed geodesic. We will show the main theorem as follows across a few sections:
\begin{main}
The series of functions $\syst$ is decreasing in $T$ and converges to $\sys$ as $T\to 0^+$. Moreover, there exists $T_0>0$ such that when $T<T_0$, $\syst$ has the following properties:\\
(1) Every $\syst$ is a $C^2$-Morse function on the Deligne-Mumford compactification $\overline{\mathcal M}_{g,n}$ (with altered differential structure).\\
(2) There is a natural stratum-wise correspondence: $$\crit(\syst)\leftrightarrow\crit(\sys).$$ More precisely, let $\mathcal S\subset\mathcal M_{g,n}$ be a stratum that is isomorphic to $\mathcal M_{g',n'}$, then there is a bijection
\begin{align*}
    \crit(\syst|_{\mathcal S})&\leftrightarrow\crit(\sys|_{\mathcal M_{g',n'}})\\
    p_T&\leftrightarrow p
\end{align*} with the properties $$d_{\text{WP}}(p,p_T)<CT,$$ which implies $p_T\to p$ and consequently $\crit(\syst|_{\mathcal S})\to\crit(\sys|_{\mathcal M_{g',n'}})$, and $$\ind_{\syst}(p_T)=\ind_{\sys}(p).$$
% (2) For each critical point $p$ of the systole function, there exists a unique critical point for $\syst$, denoted by $p_T$, near $p$.\\
% (3) $d(p,p_T)$ is sublinear, i.e., $d(p,p_T)<CT$.\\
% (4) All the critical points of $\syst$ are near the critical points of $\sys$, as described above, i.e., the critical points of $\syst$ will escape any small neighborhood of an ordinary point for $\sys$, and therefore, there is a one-to-one correspondence $p\leftrightarrow p_T$, and as a corollary, $\#Crit(\syst)=\#Crit(\sys)$.\\
% (5) The index of $\syst$ at $p_T$ is the same as the index of $\sys$ at $p$.
\end{main}
\noindent
The author believes that the $\syst$ functions are smooth. To the best of the author's knowledge, they are the first examples in literature, of $C^2$-Morse functions on the moduli space as well as the Deligne-Mumford compactification. With the Morse property, they can be used to study the topology of $\mathcal M_{g,n}$ or $\overline{\mathcal M}_{g,n}$. They can also be used to study the systole function itself with the index-preserving critical point attracting property.\\
\\
As we have established the Morse property of $\syst$ functions, there are many mysteries to be discovered and questions to be answered. In \textbf{[paper to be uploaded]}, the author proves a theorem of nonexistence of low index critical points for $\syst$ (and $\sys$) on large $\mathcal M_{g,n}$, namely for large $\mathcal M_{g,n}$, all low index critical points for $\syst$ live in the Deligne-Mumford boundary. Very ambiguously speaking, this result can be interpreted as low order homology elements of $\mathcal M_{g,n}$ come from the boundary $\partial\mathcal M_{g,n}$.\\
\\
Here is the organization of this paper. In Section \ref{prelim}, we review some basics in hyperbolic geometry and Teichm\"uller theory. In section \ref{Akrout}, we talk about Akrout's theorem about the systole function and his theory of (semi-)eutacticity. We list some powerful theorems in section \ref{thms} that will be used in our later proofs. In section \ref{critptsthry} we study $\syst$ at the first step
and develop general knowledge including the \textit{fan decomposition} to study the critical points. Section \ref{behaviorinmajor} is focused on the behavior of $\syst$ on the \textit{major subspace} while section \ref{behavior} reveals its full local behavior at eutactic points. Section \ref{nondegeneracy} establishes nondegeneracy of $\syst$ near eutactic points. The final pieces of the puzzle to show that $\syst$ is a Morse function on $\mathcal M_{g,n}$ come in section \ref{ordinary} with a study of its behavior near semi-eutactic and non-semi-eutactic points, and section \ref{boundary} for near the Deligne-Mumford boundary. The extension of $\syst$ to the boundary is defined and studied in section \ref{extension} to complete the proof of the main theorem. We show the Weil-Petersson gradient flow of $\syst$ on $\mathcal M_{1,1}$ in section \ref{case}.
\\
% Eutactic:\\
% -Existence in $\rho T$-ball\\
% -Uniqueness in $\rho T$-ball\\
% -Nonexsitence outside $\rho T$-ball\\
% Semi-eutactic:\\
% -Nonexistence\\
% Non semi-eutactic:\\
% -Nonexistence with continuity\\
\\
A table of notations that will be used is attached at the end of this paper.


\section{Preliminaries}
\label{prelim}

\subsection{Teichm\"uller space and moduli space}
Let $X$ be a smooth complete real surface with negative Euler characteristic and $\mathcal M_{-1}(X)$ the set of all hyperbolic metrics on $X$. Let $\textit{Diff}_+(X)$ be the set of all orientation-preserving diffeomorphisms and $\textit{Diff}_0(X)\subset\textit{Diff}_+(X)$ the subset consisting of those homotopic to the identity map, and they act on $\mathcal M_{-1}(X)$ by pullback. The quotient $\textit{Diff}_+(X)/\textit{Diff}_0(X)$ is known as the \textit{mapping class group} $\textit{Mod}(X)$. The \textit{Teichm\"uller space} is $$\mathcal T(X):=\mathcal M_{-1}(X)/\textit{Diff}_0(X)$$ and the \textit{moduli space} is $$\mathcal M(X):=\mathcal M_{-1}(X)/\textit{Diff}_+(X)=\mathcal T(X)/\textit{Mod}(X).$$ We will frequently use a hyperbolic surface $X$ instead of a hyperbolic metric $m$ on the surface $X$, by abuse of notation, to refer to a point in the Teichm\"uller space or moduli space. When $X$ is of genus $g$ and with $n$ punctures, we also use $\mathcal T_{g,n}$ and $\mathcal M_{g,n}$ for $\mathcal T(X)$ and $\mathcal M(X)$ respectively.
% Let $X$ be an oriented hyperbolic surface of genus $g$ with $n$ punctures, i.e., a $(g,n)$-surface. The Teichm\"uller space $\mathcal T(X)$ is the space of equivalence classes of pairs $(Y,f)$, where $f:X\to Y$ is an orientation-preserving diffeomorphism from $X$ to a hyperbolic surface $Y$, and $(Y_1,f_1)\sim(Y_2,f_2)$ if $f_2\circ f_1^{-1}$ is homotopic to a conformal mapping $h:Y_1\to Y_2$. The moduli space $\mathcal M(X)$ is the quotient space of the Teichm\"uller space $\mathcal T(X)$ by modding out the mapping class group $MCG(X)$. Teichm\"uller spaces modeled on different $(g,n)$-surfaces are isomorphic, and so are moduli spaces. Therefore, Teichm\"uller spaces or moduli spaces essentially only depend on the surface type $(g,n)$. In the main part of this paper, we fix a surface type and only use $\mathcal T$ or $\mathcal M$ without indicating the surface type.
\subsection{Weil-Petersson metric}
Let $(X,m)\in\mathcal T$. A \textit{Beltrami differential} on $(X,m)$ is $f\frac{d\overline{z}}{dz}$ on a local chart and a \textit{harmonic Beltrami differential} if it can be expressed as $ds_m^{-2}\overline{\varphi}$ for a holomorphic quadratic differential $\varphi$, whose total space is denoted by $\textit{HB}(X)$. By Bers' embedding, points in Teichm\"uller space can be represented by Beltrami differentials on $X$ and we in fact have an isomorphism $\textit{HB}(X)\cong T_m\mathcal T(X)$.\\
\\
The \textit{Weil-Petersson metric} on $\mathcal T(X)$ is
$$\langle\mu_1,\mu_2\rangle_{\textit{WP}}=\int_X\mu_1\overline{\mu}_2ds_m^2,$$
for $\mu_1,\mu_2\in \textit{HB}(X)\cong T_m\mathcal T(X)$.\\
\\
The Weil-Petersson metric descends to the moduli space since it is mapping class group invariant. See \cite{imayoshi2012introduction} for more details on Teichm\"uller theory.

\subsection{The systole function and topological Morse functions} For any homotopy class $\gamma$ on $X$, there exists a unique geodesic representative with respect to a hyperbolic metric $m$. We define $l_\gamma$ at $(X,m)$ to be the length of the geodesic representative.\\
\\
The systole function \textit{sys} assigns a surface the length of a shortest (nontrivial) closed geodesic, i.e., $$\sys(X)=\min_{\gamma\text{ closed geodesic on }X}l_\gamma(X).$$ As the function respects isometries, it descends to \textit{sys}$:\mathcal T\to\mathbb R_+$, as well as \textit{sys}$:\mathcal M\to\mathbb R_+$.
\begin{definition}
With the same assumptions on $X$ above, let $S(X)$ be the set of all shortest closed geodesics on it.
\end{definition}
\noindent
For any $\gamma\in S(X)$, note that $\gamma$ has to be simple (otherwise it is possible to take a nontrivial shortcut). Therefore, it is true that $$\sys(X)=\min_{\gamma\text{ simple closed geodesic on }X}l_\gamma(X).$$
% \begin{lemma}
% Let $\gamma$ be a shortest geodesic on a hyperbolic surface $X$, then the following are true:\\
% (1) $\gamma$ is simple.\\
% (2) If $X$ has at most 1 punctures, and $\gamma'$ is another shortest geodesic, then $\#\gamma\cap\gamma'\le1$.
% \end{lemma}
% \noindent
It is not hard to see that $\sys$ is continuous but not $C^1$. Examples can be constructed with Fenchel-Nielsen coordinates for instance.\\
\\
To show Akrout's result on $\sys$, we review some definitions.
\begin{definition}[Topological Morse function]
Let $f:M^n\to\mathbb R$ be a continuous function. $x\in M$ is called a ($C^0$-)\textit{ordinary point} if locally $f$ can be expanded to a $C^0$-chart near $x$, otherwise it is called ($C^0$-) critical. A \textit{critical point} $x$ is nondegenerate if there is a local $C^0$-chart $(x^i)$ such that $f-f(x)=(x^1)^2+\cdots+(x^r)^2-(x^{r+1})^2-\cdots(x^n)^2$. In this case the \textit{index} of $f$ at $x$ is defined to be $n-r$. A continuous function is called \textit{topologically Morse} if all the critical points are nondegenerate. For more, see \cite{morse1959topologically}.
\end{definition}


\section{Akrout's Theorem and Eutacticity}
\label{Akrout}
\noindent
We introduce Akrout's (semi-)eutactic condition and his results on the systole function.
\begin{definition}
Let $\{v_i\}\subset\mathbb R^n$ be a finite set of nonzero vectors. It is called \textit{eutactic} if the origin is contained in the interior of the convex hull of $\{v_i\}$, or \textit{semi-eutactic} if the origin is contained in the boundary of the convex hull of $\{v_i\}$. It is called \textit{non-semi-eutactic} if it is neither eutactic or semi-eutactic.
\end{definition}
\noindent
With Euclidean metric, we have the following equivalent definitions:
\begin{lemma}
Let $\{v_i\}\subset\mathbb R^n$ be a finite subset, then\\
(1) $\{v_i\}$ is eutactic if $\max_{\tau\in S^{n-1}}\min_i\{\langle v_i,\tau\rangle\}<0$.\\
(2) $\{v_i\}$ is semi-eutactic if $\max_{\tau\in S^{n-1}}\min_i\{\langle v_i,\tau\rangle\}=0$.\\
(3) $\{v_i\}$ is non-semi-eutactic if $\max_{\tau\in S^{n-1}}\min_i\{\langle v_i,\tau\rangle\}>0$.
\end{lemma}
\begin{definition}
Let $S(X)=\{\gamma_1,\cdots,\gamma_r\}$ be the set of all shortest geodesics on $X$, then the gradient vectors $\nabla l_1,\cdots,\nabla l_r\in T_X\mathcal T$ are called the \textit{minimal gradients} for $X$.
\end{definition}
\begin{definition}
\label{eutactic}
A hyperbolic surface $X\in\mathcal T$ is called \textit{eutactic (semi-eutactic, non-semi-eutactic)} if the set of minimal gradients $\{\nabla l_i\}_{\gamma_i\in S(X)}$ is eutactic (semi-eutactic, non-semi-eutactic) in the tangent space $T_X\mathcal T$ (equivalently $\{dl_i\}_{\gamma_i\in S(X)}$ is eutactic (semi-eutactic, non-semi-eutactic) in the cotangent space $T^*_X\mathcal T$ by duality).
\end{definition}
\noindent
In \cite{akrout2003singularites}, Akrout proved topological Morse property for $\sys$ through eutacticity:
\begin{theorem}
The systole function is topologically Morse on $\mathcal M_{g,n}$. $X$ is a critical point if and only if $X$ is eutactic, and the index is equal to $\rank\{\nabla l_i:\gamma_i\in S(X)\}$.
\end{theorem}
\noindent
This paper is related to but will not use Akrout's theorem on topological Morse property to prove any result.

\section{A few theorems in Teichm\"uller theory}
\label{thms}
\noindent
Below we cite a few results from literature that we will use.
% Wolpert calculated in \cite{wolpert1987geodesic} the second derivatives of geodesic-length functions and proved convexity:
\begin{theorem}[Wolpert \cite{wolpert1987geodesic}]
\label{convex}
Let $\gamma$ be a closed geodesic and $l_\gamma$ the associated geodesic-length function, then $l_\gamma$ is real analytic on $\mathcal T$ and strictly convex along Weil-Petersson geodesics.
\end{theorem}
% \begin{theorem}[Delsarte-Huber-Selberg]
% \label{counting}
% The number $c_X(L)$ of closed geodesics of length $\le L$ on any hyperbolic surface $X$ is asymptotic to $\frac{e^L}{L}$ as $L\to\infty$.
% \end{theorem}
\noindent
We use Birman and Series' version of control of the growth rate of simple closed geodesics by length on hyperbolic surfaces. Let $s_X(L)$ be the number of simple closed geodesics with length at most $L$ on a hyperbolic surface $X$.
% We will need a bound on the growth rate of geodesics by length for convergence reasons. There are a few different versions but here we use a universal upper bound given by Birman and Series in \cite{birman1985geodesics} on the number $s_X(L)$ of simple closed geodesics with length at most $L$ on a hyperbolic surface $X$:
\begin{theorem}[Birman-Series \cite{birman1985geodesics}]
There exists a universal polynomial $P$ such that $s_X(L)\le P(L)$.
\end{theorem}
\noindent
To make it convenient to use, we suppose $P(x)\le ce^x$ for some fixed constant $c$.\\
\\
% \begin{lemma}
% \label{exceptional}
% Let $X$ be a surface of type $(g,n)\neq(1,1),(0,4)$, and $\gamma_1,\gamma_2$ be two different simple closed geodesics on $X$, then $\nabla l_1$ and $\nabla l_2$ are linearly independent in $T_X\mathcal T$.
% \end{lemma}
The following results are also due to Wolpert. See the explicit Weil-Petersson geodesic length-length formula that was given by Riera in \cite{riera2005formula} that may help understand.
% In \cite{wolpert2008behavior}, Wolpert studied the Weil-Petersson gradient vector and Hessian of geodesic-length functions through Riera's explicit length-length formula (see \cite{riera2005formula}). His results greatly help us understand the behavior of geodesic-length functions under Weil-Petersson metric and will be used to study derivatives of $\syst$ later:
\begin{theorem}[Wolpert \cite{wolpert2008behavior} 3.11, 3.16]
\label{norm}
There exists a universal constant $c>0$, such that $$\langle\nabla l_\gamma,\nabla l_\gamma\rangle_{\textit{WP}}\le c(l_\gamma+l_\gamma^2e^{\frac{l_\gamma}{2}})$$ and $$\nabla^2l_\gamma(\cdot,\cdot)<c(1+l_\gamma e^{\frac{l_\gamma}{2}})\langle\cdot,\cdot\rangle_{\textit{WP}},$$ for any simple closed geodesic $\gamma$.
\end{theorem}
% \noindent
% Specially for disjoint or identical geodesics, which for instance can be given by taking shortest geodesics on surfaces close to the boundary of the moduli space, an estimate on Weil-Petersson pairing can be established also due to Wolpert:
\begin{theorem}[Wolpert \cite{wolpert2008behavior} 3.12]
\label{short}
Let $\gamma_i,\gamma_j$ be disjoint or identical geodesics, then $$0<\langle\nabla l_i,\nabla l_j\rangle_{\textit{WP}}=\frac{2}{\pi}l_i\delta_{ij}-O(l_i^2l_j^2),$$
where the right hand side is uniform for $l_i,l_j\le c_0$.
\end{theorem}
\noindent
Given mutually disjoint short geodesics $\beta_i$, let $\gamma_i$'s be a few geodesics disjoint from $\cup \beta_i$, such that $\{\lambda_i:=\nabla l_{\beta_i}^{\frac12},J\lambda_i,\nabla l_i\}$ is a local frame, and with the notations:
% The following estimates on the Weil-Petersson connection is also due to Wolpert, see \cite{wolpert2009extension}:
\begin{theorem}[Wolpert \cite{wolpert2009extension}]
\label{connection}
Let $D$ be the Weil-Petersson connection, then as $l_{\beta_i}\to0$,
    \begin{align*}
        &D_{J\lambda_i}\lambda_i=\frac{3}{2\pi\l_{\beta_i}^{\frac12}}J\lambda_i+O(l_{\beta_i}^{\frac32}),\\
        &D_{\lambda_j}\lambda_i=O(l_{\beta_i}^{\frac32}),\\
        &D_{J\lambda_j}\lambda_i=O(l_{\beta_i}(l_{\beta_j}^{\frac32}+l_{\beta_i}^{\frac12})), \text{when } i\neq j,\\
        &D_{\nabla l_j}\lambda_i=O(l_{\beta_i}),\\
        &D_{\lambda_j}\nabla l_i=O(l_{\beta_j}^{\frac12}),\\
        &D_{J\lambda_j}\nabla l_i=O(l_{\beta_j}^{\frac12}),\\
        &D_{\nabla l_j}\nabla l_i \text{ is continuous}.
    \end{align*}
\end{theorem}
% \noindent
% Kerckhoff proved the length-twist formula in \cite{kerckhoff1983nielsen}:
% \begin{theorem}[Kerckhoff \cite{kerckhoff1983nielsen}]
% \label{lengthtwist}
%     Let $\gamma,\beta$ be two simple closed geodesics, then $$\langle\nabla l_\gamma,\frac{\partial}{\partial\tau_\beta}\rangle=\sum\cos\theta_i,$$ where $\theta_i$ is the angle measured counterclockwise from $\gamma$ to $\beta$ at the $i$-th intersection. If they are disjoint, the inner product is 0.
% \end{theorem}
\noindent
A classic exercise in topology will also be used:
\begin{lemma}
\label{degree}
Let $f,g:S^n\to S^n$ be two self maps where $f(x)+g(x)\neq0$, then $\deg f=\deg g$. Specially, if $g=\textit{id}$, then $\deg f=1$.
\end{lemma}
\begin{proof}
$H(x,t)=\frac{(1-t)f(x)+tg(x)}{||(1-t)f(x)+tg(x)||}$ is a homotopy that sends $f$ to $g$.
\end{proof}
\section{$\syst$ Functions and Fan Decomposition}
\label{critptsthry}
\noindent
In this section, we study the convergence of the $\syst$ functions and develop a method that the author calls \textit{fan decomposition} that will be used to study the behavior of $\syst$ in following sections.
\begin{definition}
We take a family of averages of the length functions associated to all simple closed geodesics on the base surface $X$, indexed by $T>0$:
$$\syst(X)=-T\log\sum_{\gamma \text{ simple closed geodesic on } X} e^{-\frac1Tl_\gamma(X)}.$$
\end{definition}
\begin{remark}
As it respects the mapping class group action, $\syst$ descends to $\mathcal T$ as well as $\mathcal M$. We make no distinguishment between them by abuse of notation.
\end{remark}
\begin{definition}
Let $S(X)$ be the set of all shortest geodesics on $X$.
\end{definition}
% \noindent
% We introduce the second systole function:
\begin{definition}
Let $\textit{sec.sys}(X)$ be the \textit{second systole function} on a hyperbolic surface $X$, taking the value of the length of second shortest geodesics (counted without multiplicity).
\end{definition}
\noindent
Note that the second systole function is upper semi-continuous and thus bounded from above on the thick part of the moduli space.
\begin{theorem}
\label{C0}
    The function $\syst$ is well defined for $T$ sufficiently small, and it uniformly converges to $\sys$ as $T\to0^+$. More specifically, $$0<\syst(X)-\sys(X)<c'T,$$ where $c'$ is a constant depending only on $(g,n)$. Moreover, $\syst:\mathcal M\to\mathbb R$ is at least $C^2$.
\end{theorem}
\begin{proof}
    Write $$\sum_{\gamma}e^{-\frac1Tl_\gamma(X)} = \sum_{n=0}\sum_{n\le l_\gamma<n+1}e^{-\frac1Tl_\gamma(X)},$$
    and by bounding the growth rate
    \begin{align*}
        \sum_{n\le l_\gamma<n+1}e^{-\frac1Tl_\gamma(X)} &\le s_X(n+1)e^{-\frac1Tn}\\
        &\le ce^{n+1}e^{-\frac1Tn}=ce^{(1-\frac1T)n+1}.
    \end{align*}
    Therefore, $\syst$ is well defined for $T<1$, and
    \begin{align*}
        \sum_{\gamma}e^{-\frac1Tl_\gamma(X)} &= \#S(X)e^{-\frac1T\sys(X)}+\sum_{n=\secsys(X)}\sum_{n\le l_\gamma<n+1}e^{-\frac1Tl_\gamma(X)}\\
        &\le \#S(X)e^{-\frac1T\sys(X)}+\sum_{n=\secsys(X)}ce^{(1-\frac1T)n+1}\\
        &= \#S(X)e^{-\frac1T\sys(X)}+c\frac{e^{1+\secsys(X)}}{1-e^{1-\frac1T}}e^{-\frac1T\secsys(X)}.
    \end{align*}
    Note that $\#S(X)\ge1$ is bounded by Birman-Series' theorem as $\sys(g,n):=\max_{X\in\mathcal M_{g,n}}\{\sys(X)\}<\infty$ and $\secsys$ is also bounded from above, then
    \begin{align*}
        |\syst(X)-\sys(X)|<c'T.
    \end{align*}
    The first and second derivatives will be studied in Lemma \ref{C1tail} and Lemma \ref{C2tail}, completing the proof that $\syst$ is at least $C^2$.
\end{proof}
\begin{lemma}
    $\syst$ is decreasing in $T$.
\end{lemma}
\begin{proof}
Note that
    \begin{align*}
        \frac{d}{dT}\syst&=-\log\left(\sum e^{-\frac1Tl}\right)-\frac1T\frac{\sum le^{-\frac1Tl}}{\sum e^{-\frac1Tl}}\\
        &=-\frac{1}{\sum e^{-\frac1Tl}}\left(\log\left(\sum e^{-\frac1Tl}\right)-\sum\left(-\frac1Tl\right)e^{-\frac1Tl}\right)<0.
    \end{align*}
\end{proof}
% \begin{theorem}
% \label{C0}
% For $T$ sufficiently small, $\syst$ is well defined, and it uniformly converges to $\sys$ as $T\to0^+$. More specifically, $$|\syst(X)-(-T\log\sum_ie^{-\frac1Tl_i})|<c'T,$$ where $c'$ is a constant depending only on $(g,n)$. Moreover, $\syst:\mathcal M\to\mathbb R$ is at least $C^2$.
% \end{theorem}
% \begin{proof}
%     Write $$\sum_{\gamma}e^{-\frac1Tl_\gamma(X)} = \sum_{n=0}\sum_{n\le l_\gamma<n+1}e^{-\frac1Tl_\gamma(X)},$$
%     and the 
% \end{proof}
% \begin{proof}
% Choose $C>2\epsilon>0$ such that $\secsys(X)\le C$ for $X\in\mathcal M^{\le\epsilon}$.\\
% When $X\in\mathcal M^{\ge\epsilon}$,
% % \begin{align*}
% %     \sum_{\gamma}e^{-\frac1Tl_\gamma(X)} &= \#S(X)e^{-\frac1T\sys(X)}+\sum_{\sys<l_\gamma\le L_0}e^{-\frac1Tl_\gamma(X)} +\sum_{n\ge L_0}\sum_{n< l_\gamma\le n+1}e^{-\frac1Tl_\gamma(X)}\\
% %     &\le \#S(X)e^{-\frac1T\sys(X)}+\sum_{\sys<l_\gamma\le L_0}e^{-\frac1Tl_\gamma(X)}+\sum_{n\ge2}s_X(n+1)e^{-\frac1Tn}\\
% %     &\le \#S(X)e^{-\frac1T\sys(X)}\left(1+\frac{1}{\#S(X)}\sum_{\sys<l_\gamma\le L_0}e^{-\frac1T(l_\gamma(X)-\sys(X))}\right.\\
% %     &\ + \left.\frac{2}{\#S(X)}\sum_{n\ge L_0}e^{(1-\frac1T)n+1}\right)\\
% %     &=: \#S(X)e^{-\frac1T\sys(X)}\left(1+\delta^0_T(X)\right),
% % \end{align*}
% \begin{align*}
%     \sum_{\gamma}e^{-\frac1Tl_\gamma(X)} &= \#S(X)e^{-\frac1T\sys(X)}+\sum_{n=\secsys(X)}\sum_{n\le l_\gamma<n+1}e^{-\frac1Tl_\gamma(X)}\\
%     &\le \#S(X)e^{-\frac1T\sys(X)}+\sum_{n=\secsys(X)}s_X(n+1)e^{-\frac1Tn}\\
%     &\le \#S(X)e^{-\frac1T\sys(X)}+\sum_{n=\secsys(X)}ce^{(1-\frac1T)n+1}\\
%     &= \#S(X)e^{-\frac1T\sys(X)}+c\frac{e^{1+\secsys(X)}}{1-e^{1-\frac1T}}e^{-\frac1T\secsys(X)}\\
%     &= \#S(X)e^{-\frac1T\sys(X)}+c'e^{-\frac1T\secsys(X)},
% \end{align*}
% where $s_X(L)$ is the number of simple closed geodesics on $X$ of length $\le L$. It is clear that $\frac{c'e^{-\frac1T\secsys(X)}}{e^{-\frac1T\sys(X)}}\to0$ as $T\to0^+$ pointwise. \\
% \\
% When $\secsys(X)>C$,
% \begin{align*}
%     \sum_{\gamma}e^{-\frac1Tl_\gamma(X)} &\le \#S(X)e^{-\frac1T\sys(X)}+c\frac{e^{1+\secsys(X)}}{1-e^{1-\frac1T}}e^{-\frac1T\secsys(X)}\\
%     &=: \#S(X)e^{-\frac1T\sys(X)}+c'e^{-\frac2T\sys(X)}.
% \end{align*}
% \\
% Therefore,
% \begin{align*}
%     |\syst(X)-(-T\log\sum_{\gamma\in S(X)}e^{-\frac1Tl_\gamma})|&=T\log\left(1+\frac{\sum_{\gamma\not\in S(X)}e^{-\frac1Tl_\gamma}}{\sum_ie^{-\frac1Tl_i}}\right)\\
%     &\le T\frac{c'\max\{e^{-\frac1T\secsys(X)},e^{-\frac2T\sys(X)}\}}{e^{-\frac1T\sys(X)}}\\
%     &\le c'T\to0
% \end{align*}
% Note that $$-T\log\sum_ie^{-\frac1Tl_i}=-T\log\#S(X)+\sys(X)\to\sys(X),$$
% thus $\syst\to\sys$ uniformly in $T$.\\
% \\
% It is $C^1$ as a consequence of Lemma \ref{C1tail} below, and $C^2$ due to 
% \end{proof}
\noindent
Given $T$, the first derivatives can be calculated directly: $$\nabla\syst(X)=\frac{\sum_\gamma e^{-\frac1Tl_\gamma(X)}\nabla l_\gamma(X)}{\sum_\gamma e^{-\frac1Tl_\gamma(X)}}.$$
Therefore, to study the critical points of $\syst$, it suffices to study $$\sum_\gamma e^{-\frac1T(l_\gamma(X)-\sys(X_0))}\nabla\syst(X)=\sum_\gamma e^{-\frac1T(l_\gamma(X)-\sys(X_0))}\nabla l_\gamma(X)$$ instead for a fixed $X_0$.
\begin{definition}
\label{setup}
Let $p$ be a critical point of the systole function, and let $$\tilde\Omega_T(X)=\sum_\gamma e^{-\frac1T(l_\gamma(X)-\sys(p))}\nabla l_\gamma(X),$$ which is a rescaling of $\nabla\syst(X)$. We leave out $p$ in the definition of $\tilde\Omega$ for simplicity as there are only finitely many such critical points and the context will be clear when $\tilde\Omega$ is used.\\
Let $u:(-a,a)\to\mathcal T$ be a unit speed geodesic with $u(0)=p$ and tangent vector $\tau:=u'(0)$. We let $$\tilde\Omega_T(v)=\tilde\Omega_T(t,\tau)=\tilde\Omega_T(u(t))$$ by abuse of notation, where $X=u(t)$ and $v=t\tau$, where we omit the exponential map $\ex_p$ at $p$.
Let $\Omega_T$ be the `main part' of $\tilde\Omega_T$ where the sum is taken over $S(p)$. In the rest of the paper, we treat other maps like $\Phi_T$ and $\Psi_T$ in the same manner. We may also use a lowercase superscript $i$ for the $i$-th component of a sum, for example, $$\Omega^i(X)=e^{-\frac1T(l_i(X)-\sys(p))}\nabla l_i(X),$$ and an uppercase superscript $J$ for the partial sum over $J$, for example, $$\Omega^J(X)=\sum_{j\in J} e^{-\frac1T(l_j(X)-\sys(p))}\nabla l_j(X).$$
Let $\pi:\mathbb R^n\setminus\{O\}\to S^{n-1}$ be the standard projection onto the unit sphere.
% \begin{remark}
% It is obvious that $\tilde\Omega_T$ depends on $p$, while since there are only finitely many critical points for the systole function in $\mathcal M$, and $\tilde\Omega_T$ is a renormalized Weil-Petersson gradient vector of $\syst$, whose zeros are what we are most interested in, we don't have to pay much attention to the change of the base point $p$.
% \end{remark}
\begin{remark}
Note that if we choose an orthonormal basis for $T_p\mathcal T$, then by composing the exponential map $\ex_p$, we get a normal coordinate system locally near $p$. Therefore, we can see $\tilde\Omega_T$ as a vector field on $\mathcal T$ near $p$ or $T_p\mathcal T$ near 0. We will recall this later. We will focus on the renormalized gradient vector $\tilde\Omega_T$.
\end{remark}
% \noindent
% We will focus on the renormalized gradient vector $\tilde\Omega_T$ and we have the following estimate:
\begin{lemma}
\label{C1tail}
When $T<1$, $||\tilde\Omega_T(X)-\Omega_T(X)||<c''e^{-\frac1T\secsys(X)}$, where $c''$ is a constant depending on $(g,n)$.
% $$||\nabla\syst(X)-\frac{\sum_ie^{-\frac1Tl_i}\nabla l_i}{\syst(X)}||\le c''\max\{e^{-\frac1T\secsys(X)},e^{-\frac2T\sys(X)}\}$$
\end{lemma}
\begin{proof}
We have the growth rate of the first derivative
\begin{align*}
    \frac{1}{e^{-\frac1T\syst}}||\sum_{n\le l_\gamma<n+1}e^{-\frac1Tl_\gamma(X)}\nabla l_\gamma||
    &\le \frac{1}{e^{-\frac1T(\sys+c'T)}}s_X(n+1)e^{-\frac1Tn}||\nabla l_\gamma||\\
    &\le Cce^{n+1}e^{-\frac1Tn(c(n+1+(n+1)^2e^{\frac{n+1}{2}}))^{\frac12}}\\
    &\le 2Cc^{\frac32}e^{(\frac54-\frac1T)n+\frac54}.
\end{align*}
Therefore, $\syst$ is at least $C^1$. For the approximation with $\Omega_T$, with the help of Theorem \ref{norm}, we have
\begin{align*}
    ||\tilde\Omega_T(X)-\Omega_T(X)|| &\le \sum_{\gamma\not\in S(X)} e^{-\frac1Tl_\gamma(X)}||\nabla l_\gamma(X)||\\
    &= \sum_{n=\secsys(X)}\sum_{n\le l_\gamma<n+1}e^{-\frac1Tl_\gamma(X)}||\nabla l_\gamma(X)||\\
    % &\le \sum_{n=\secsys(X)}s_X(n+1)e^{-\frac1Tn}(c_1(n+1+(n+1)^2e^{\frac{n+1}{2}}))^{\frac12}\\
    % &\le \sum_{n=\secsys(X)}ce^{n+1}e^{-\frac1Tl_\gamma(X)}2\sqrt{c_1}(n+1)e^{\frac{n+1}{4}}\\
    &\le \sum_{n=\secsys(X)}2c\sqrt{c_1}e^{n+1}e^{-\frac1Tl_\gamma(X)}e^{1+\frac{n+1}{4}}\\
    &\le 2c\sqrt{c_1} \frac{e^{\frac54+\frac94\secsys(X)}}{1-e^{\frac94-\frac1T}}e^{-\frac1T\secsys(X)}\\
    &\le c''e^{-\frac1T\secsys(X)}
\end{align*}
% \begin{align*}
%     ||\tilde\Omega_T(X)-\Omega_T(X)|| &< \sum_{\gamma\not\in S(X)} e^{-\frac1Tl_\gamma(X)}||\nabla l_\gamma(X)||\\
%     &= \sum_{\sys(X)<l_\gamma\le L_0}e^{-\frac1Tl_\gamma(X)}||\nabla l_\gamma(X)||\\
%     &+\sum_{n\ge L_0}\sum_{n< l_\gamma\le n+1}e^{-\frac1Tl_\gamma(X)}||\nabla l_\gamma(X)||\\
%     &\le \sum_{\secsys(X)\le l_\gamma\le L_0}e^{-\frac1Tl_\gamma(X)}||\nabla l_\gamma(X)||\\
%     & +\sum_{n\ge L_0}s_X(n+1)e^{-\frac1Tn}(c_1(n+1+(n+1)^2e^{\frac{n+1}{2}}))^{\frac12}\\
%     &\le \sum_{\sys(X)<l_\gamma\le L_0}e^{-\frac1Tl_\gamma(X)}||\nabla l_\gamma(X)|| +\sum_{n\ge L_0}2\sqrt{2c_1}e^{(\frac54(n+1)-\frac1Tn)}
% \end{align*}
The lemma follows.
\end{proof}
\end{definition}
\begin{definition}
\label{space}
Let $X$ be a point in the Teichm\"uller space, and $S(X)=\{\gamma_1,\cdots,\gamma_r\}$ the set of all shortest geodesics on it. Define subspaces of the tangent space $T_X\mathcal T$:
$$\text{the\ } \textit{major subspace}: T_X^{\sys}\mathcal T=\spn\{\nabla l_1,\cdots,\nabla l_r\},$$
$$\text{and\ the\ } \textit{minor subspace}: T_X^{\sys\perp}\mathcal T=(T_X^{\sys}\mathcal T)^\perp.$$ The notions are used mainly when $X$ is eutactic.
\end{definition}
\begin{definition}[Fan decomposition of the major subspace $T_p^{\sys}\mathcal T$]
\noindent\\
Let $p\in\mathcal M$ be a critical point for the systole function, and $S(p)=\{\gamma_i\}_{i\in I}$ where $I=\{1,\cdots,r\}$ is the index set, then the minimal gradient set $\{\nabla l_i\}_{i\in I}$ satisfies the eutactic condition at $p$, say $\sum_{i=1}^r a_i\nabla l_i(p)=0$ for positive $a_i$'s. Let $J$ be a subset of $I$, and we define $F_J\subset T_p^{\sys}\mathcal T$ to be the region consisting of vectors $v$ such that $\argmax_j\{\langle \nabla l_j,v\rangle\}=J$. On the other hand, for any $\tau$ it assigns a multiindex $J=J(\tau)$ by inclusion $\tau\in F_J$. We extend the definition of $J(\cdot)$ to $(T_X^{\sys\perp}\mathcal T)^c$ by $J=J\circ\proj_{T_p^{\sys}\mathcal T}$ by abuse of notation. Let $v_J=\pi(\sum_{j\in J}\nabla l_j)$ be the unitized average.
\end{definition}
\noindent
The fan decomposition concerns the variation of the length of the shortest geodesics at $p$ when the base point is perturbed. Fix the direction $\tau$ of a small perturbation, the geodesics indexed by $J(\tau)$ will be longer than others. If we perturb the base point in the opposite direction $-\tau$, then the geodesics indexed by $J$ will be shorter than others and are the only candidates for the shortest geodesics on $u(-t\tau)$, i.e. $S(u(-t\tau))\subset\{\gamma_j\}_J$.
\begin{lemma}
\label{fan}
The following are true:\\
(1) $\bigcup_J F_J = T_p^{\sys}\mathcal T$.\\
(2) Any $F_J$ is a polygonal cone properly contained in some half space of the major subspace and $\nabla l_j\in\mathbb H(\tau)$ for any $\tau$ and all $j\in J(\tau)$, and therefore $v_J$ is well defined.\\
(3) $F_J\subset\mathbb H(v_J)$, equivalently, $\langle v_J,\tau\rangle>0$ if $\tau\in F_J$.\\
(4) There exists $D>0$ such that $|\langle\nabla l_j,\tau\rangle| > D$ for any $\tau$ unit and all $j\in J(\tau)$, i.e., $$\max_{\tau;j\in J(\tau)}\langle\nabla l_j,\tau\rangle > D \text{ and } \min_{\tau;j\in J(\tau)}\langle\nabla l_j,\tau\rangle < -D.$$
(5) The inner products $\langle\nabla l_j,\tau\rangle$ cannot all be equal and $\#\{i:\langle\nabla l_i,\tau\rangle\ge 0\}>0$, $\#\{i:\langle\nabla l_i,\tau\rangle< 0\}>0$.
\end{lemma}
\begin{proof}
(1) is tautological. For any $0\neq v\in T_p^{\sys}\mathcal T$, by the eutactic condition \ref{eutactic}, there exists at least one $i$ such that $\langle\nabla l_i,v\rangle>0$, otherwise, $v$ would live in the minor subspace $T_p^{\sys\perp}\mathcal T$, from which (2) and (5) follows. (3) follows directly from (2). (4) follows from the finiteness of $\{F_J\}$ and the first two propositions.
\end{proof}
\begin{remark}
If $q\in\mathcal T$ is semi-eutactic, we may do a fan decomposition on the subspace spanned by the maximal eutactic subset of all minimal gradients in a similar way, or a degenerated fan decomposition on the subspace spanned by all minimal gradients. In the latter case, we can similarly establish an inequality resembling (4) above:\\
There exists $D>0$ such that $$\max\langle\nabla l_j,\tau\rangle > D \text{ or } \min\langle\nabla l_j,\tau\rangle < -D$$ for any $\tau\in T^{\sys}\mathcal T$ unit.
\end{remark}
\begin{remark}
Property (4) in Lemma \ref{fan} and the remark above still hold true (with possibly smaller $D$), if we allow $\tau$ to be any vector in $T_*\mathcal T$ with $\angle(\tau,T_*^{\sys\perp}\mathcal T)>\theta$ for fixed $\theta>0$.
\end{remark}
\section{Behavior of $\syst$ in the Major Subspace}
\label{behaviorinmajor}
\noindent
This section provides an insight of the behavior of $\syst$ (though not complete), in a simplified setting, to help one better understand, while very little from this section will be directly used later.\\
\\
We take the following approximation of $\Omega_T$: $$\Phi_T(t,\tau)=\sum_i e^{-\frac tT\langle\nabla l_i(p),\tau\rangle}\nabla l_i(p).$$
Note that rigorously we are considering the vector field $\ex_p^*\Omega_T$ on the tangent bundle $T(T_p\mathcal T)\cong T_p\mathcal T$, as $\ex_p$ is locally an isomorphism but we will continue to use $\Omega_T$ for simplicity by abuse of notation (we will do the same thing a few times in this paper), and the approximation $\Phi_T$ is done in the same space with the help of normal coordinates.
\begin{lemma}
\label{existence1}
Under the notations in Definition \ref{setup} and Definition \ref{space}, $\Phi_T(\cdot,\cdot)$ restricted as a map $T_p^{\sys}\mathcal T\to T_p^{\sys}\mathcal T$ has a zero.
\end{lemma}
\begin{proof}
Consider $\Phi_T(t,-\tau)$ restricted onto the $\rho T$-sphere $S^{d-1}_{\rho T}:=\{t\tau:t=\rho T, \tau \text{\ unit}\}\subset T_p^{\sys}\mathcal T$, where $d=\dim T_p^{\sys}\mathcal T$ is the dimension of the major subspace and $\rho>0$ is a constant depending only on $p$ and it will be determined later.\\
\\
For any $\tau\in T_p^{\sys}\mathcal T$, fix a $j\in J(\tau)$, we then have
\begin{align*}
    \Phi_T(t,-\tau) &= \sum_i e^{-\frac tT\langle\nabla l_i(p),\tau\rangle}\nabla l_i(p)\\
    &= e^{\frac tT\langle\nabla l_j(p),\tau\rangle}\sum_i e^{\frac tT(\langle\nabla l_i(p),\tau\rangle-\langle\nabla l_j(p),\tau\rangle)}\nabla l_i(p).
\end{align*}
Therefore, by (2) in Lemma \ref{fan}, $$\Phi_T(t,-\tau)\sim e^{\rho\langle\nabla l_j(p),\tau\rangle}\sum_{i\in J}\nabla l_i(p)\neq0,$$ when $\rho=\frac tT$ is sufficiently large and hence it induces a self map on $S^{d-1}\subset T_p^{\sys}\mathcal T$ by post-composing the projection $\pi$. It can be formulated as follows:
\begin{align*}
    \pi\circ\Phi_T(\rho,-\tau)|_{\tau\in S^{d-1}} &= \pi\circ\Phi_T(t,-\tau)|_{t\tau\in S^{d-1}_{\rho T}}\\
    &= \pi\left(e^{\rho\langle\nabla l_j(p),\tau\rangle}\sum_i e^{\rho(\langle\nabla l_i(p),\tau\rangle-\langle\nabla l_j(p),\tau\rangle)}\nabla l_i(p)\right)\\
    &\to v_J \text{\ as\ } \rho\to\infty.
\end{align*}
Therefore, there exists $\rho_J>0$ such that $$\angle(\Phi_T(t,-\tau), \tau)<\frac{\pi}{2}$$ when $t>\rho_J T$, by (3) in Lemma \ref{fan}. By finiteness of $\{J\}$, take $\rho>\max_J\{\rho_J\}$ and the angle condition is then satisfied for all $\tau\in S^{d-1}\subset T_p^{\sys}\mathcal T$. By Lemma \ref{degree}, $\Phi_T(\cdot,-\cdot)|_{S^{d-1}_{\rho T}}$ has degree 1, and equivalently, if we don't negate $\tau$, $$\deg(\pi\circ\Phi_T|_{S^{d-1}_{\rho T}})=(-1)^d.$$ Therefore, $\Phi_T$ has a zero in the interior of the $\rho T$-sphere.
\end{proof}
% \begin{remark}
% Any $\rho(T)$-sphere would work for the same argument with possibly smaller $T$, if $\rho(T)\succ\rho T$.
% \end{remark}
\noindent
The following lemma establishes uniqueness, which still holds true for $\tilde\Omega_T$ as we will see later, while the linearity below is only for $\Phi_T$.
\begin{lemma}
\label{linearity}
The zero in the lemma above is unique and it linearly approaches $p$ in $T$.
\end{lemma}
\begin{proof}
Uniqueness: Suppose $v_1\neq v_2$ are two zeros for $\Phi_T(v)$. Consider the function $f(s):=\langle\Phi_T((1-s)v_1+sv_2),v_1-v_2\rangle$ which vanishes at $0$ and $1$. Note that $f$ is strictly increasing since each summand of $$f(s)=\sum_ie^{-\frac1T\langle\nabla l_i,v_1\rangle}e^{\frac sT\langle\nabla l_i,v_1-v_2\rangle}\langle\nabla l_i,v_1-v_2\rangle$$ is non-decreasing and at least one of them is strictly increasing, leading to a contradiction.
\\
Linearity: Suppose $(t_0,\tau_0)$ is the zero for $\Phi_{T_0}$, then $\Phi_T(\frac{t_0}{T_0}T,\tau_0)=\Phi_{T_0}(t_0,\tau_0)=0$, i.e., $(\frac{t_0}{T_0}T,\tau_0)$ is the zero for $\Phi_{T_0}$.
\end{proof}
\noindent
Putting Lemma \ref{existence1} and Lemma \ref{linearity} together, we have:
\begin{theorem}
$\Phi_T$ restricted on $T_p^{\sys}\mathcal T$ has a unique zero, and it linearly approaches $p$ in $T$.
\end{theorem}
\section{Existence of Critical Points}
\label{behavior}
\noindent
Recall that we suppose $p$ is a critical point for $\sys$, i.e., $p$ is eutactic. While the linear approximation above solves the existence of a zero in the major subspace $T_p^{\sys}\mathcal T$, it does not work for detecting that for $\nabla\syst$ when the orthogonal complement $T_p^{\sys\perp}\mathcal T$ is involved.\\
\\
Note that the $t$ component in the coordinates $(t,\tau)$ that we are using for a small neighborhood of $p$ is the radial coordinate. As a reminder, we omit the exponential map `$\ex_p^*$' in $\ex_p^*l_i$, $\ex_p^*\nabla l_i$, and $\ex_p^*\Omega_T$, etc., by abuse of notation, while we use the $(t,\tau)$ coordinates. Since $l_i$ is real analytic, we can write $$l_i(t,\tau)=l_i(p) + t\langle\nabla l_i(p),\tau\rangle+\frac12t^2\nabla^2l_i(p)(\tau,\tau) + O_i(t^3)$$ and $$\nabla l_i(t,\tau)=\nabla l_i(p)+t\nabla^2l_i(p)(\tau,\cdot)+O_i(t^2).$$
We consider the following when $t=\rho T$, i.e., when $t\tau\in S^{n-1}_{\rho T}$,
\begin{align*}
& e^{-\frac1T(l_i(t,\tau)-\sys(p))}\nabla l_i(t,\tau)\\ = & e^{-\frac1T(l_i(t,\tau)-\sys(p))}(\nabla l_i(p)+\rho T\nabla^2l_i(p)(\tau,\cdot)+O(T^2))\\
= & e^{\rho\langle\nabla l_i(p),\tau\rangle+\frac12\rho^2T\nabla^2l_i(p)(\tau,\tau) + O(T^2)}\nabla l_i(p)+\rho Te^{\rho\langle\nabla l_i(p),\tau\rangle + O(T)}\nabla^2l_i(p)(\tau,\cdot)\\ &+ e^{-\frac1T(l_i(t,\tau)-\sys(p))}O(T^2)\\
= & e^{\rho\langle\nabla l_i(p),\tau\rangle+\frac12\rho^2T\nabla^2l_i(p)(\tau,\tau)}\nabla l_i(p)+\rho Te^{\rho\langle\nabla l_i(p),\tau\rangle}\nabla^2l_i(p)(\tau,\cdot) + O(T^2)
\end{align*}
Therefore, if we write
\begin{align*}
    \Omega_T(t,\tau) &= \sum_ie^{-\frac 1T(t\langle\nabla l_i(p),\tau\rangle + \frac12 t^2 \nabla^2l_i(\tau,\tau))}\nabla l_i(p)\\
    &+ t\sum_ie^{-\frac 1T(t\langle\nabla l_i(p),\tau\rangle}\nabla^2l_i(p)(\tau,\cdot)+\epsilon_2(T,t,\tau),
\end{align*}
then $\epsilon_2(T,t,\tau)=O(T^2)$ when $t=\rho T$.
\begin{definition}
\label{ep2}
With the notations above, we denote the `main part' by  $$\Psi_T:=\Omega_T-\epsilon_2,$$ where $\epsilon_2(T,t,\tau)|_{t=\rho T}=O(T^2)$.
\end{definition}
\noindent
We fix a constant $\theta_0$ below that will stay fixed and be used almost everywhere throughout the rest of the paper:
\begin{definition}
Let $0<\theta_0<\frac{\pi}{6}$ be a constant (that depends on the base critical point $p$) that satisfies $$\angle(\nabla^2l_i(\tau,\cdot),\tau)<\frac{\pi}{2}-2\theta_0$$ for all $i$. Existence of $\theta_0$ is due to $\nabla^2 l_i(\tau,\tau)>0$ by convexity in Theorem \ref{convex}, compactness of the unit sphere and finiteness of the number of shortest geodesics on $p$.
\end{definition}
\begin{theorem}
\label{existence2}
There exists $\rho_0>0$ and $\epsilon_0=\epsilon_0(\rho)>0$ such that $\pi\circ\Psi_T(\cdot,\cdot)$ restricted on $S^{n-1}_{\rho T}\subset T_p\mathcal T$ has degree $(-1)^d$ for $\rho>\rho_0$ and $T<\epsilon_0$, where $d:=\dim T_p^{\sys}\mathcal T$ as a reminder.
\end{theorem}
\begin{proof}
Let $i$ be the automorphism given by $$i(v_1+v_2)=v_1-v_2,$$ where $v_1\in T_X^{\sys}\mathcal T$ and $v_2\in T_X^{\sys\perp}\mathcal T$. For example, $i|_{T_X^{\sys}\mathcal T}=\textit{id}|_{T_X^{\sys}\mathcal T}$. To show the conclusion on $\pi\circ\Psi_T(\cdot,\cdot)$, we consider $i\circ\pi\circ\Psi_T(\cdot,-\cdot)$.\\
\\
When $t=\rho T$, $$\Psi_T(t,-\tau)|_{S^{n-1}_{\rho T}}=\sum_ie^{\rho \langle\nabla l_i(p),\tau\rangle}\left(e^{-\frac12 \rho^2T \nabla^2l_i(p)(\tau,\tau)}\nabla l_i(p) -\rho T\nabla^2l_i(p)(\tau,\cdot)\right).$$
Let $$\Psi_1(t,-\tau)=\sum_ie^{\rho\langle\nabla l_i(p),\tau\rangle}e^{-\frac12\rho^2 T \nabla^2l_i(p)(\tau,\tau)}\nabla l_i(p)$$ and $$\Psi_2(t,-\tau)=-\sum_ie^{\rho\langle\nabla l_i(p),\tau\rangle}\rho T\nabla^2l_i(p)(\tau,\cdot).$$
\\
We consider two cases based on the position of $\tau$:\\
\\
Case 1: When $\angle(\tau,T_p^{\sys\perp}\mathcal T)\le\theta_0$ (as a remark that this situation does not happen if $T_p^{\sys}\mathcal T=T_p\mathcal T$), let $\tau'$ be the projection of $\tau$ onto $T_p^{\sys\perp}\mathcal T$, then $\angle(\tau',\tau)\le\theta_0$. Recall that we have $\angle(\nabla^2l_i(p)(\tau,\cdot),\tau)<\frac{\pi}{2}-2\theta_0$ by the choice of $\theta_0$, then it follows that $\angle(-\Psi_2(t,-\tau),\tau)<\frac{\pi}{2}-2\theta_0$. Using triangle inequality, we have $$\angle(-\Psi_2(t,-\tau),\tau')<\angle(-\Psi_2(t,-\tau),\tau)+\angle(\tau,\tau')<\frac{\pi}{2}-2\theta_0+\theta_0=\frac{\pi}{2}-\theta_0,$$ which in other words implies $i(\Psi_2(t,-\tau))\in \mathbb H(\tau')$ by definition of $i$. Note that $\Psi_1\in T_p^{\sys}\mathcal T\subset\partial\mathbb H(\tau')$, thus $$i(\Psi_T(t,-\tau))=\Psi_1(t,-\tau)+i(\Psi_2(t,-\tau))\in \mathbb H(\tau').$$ Therefore, $$\angle(i(\Psi_T(t,-\tau)),\tau)<\angle(\Psi_1+i(\Psi_2),\tau')+\angle(\tau',\tau)<\frac{\pi}{2}+\theta_0.$$
% \\
% When $\theta^\perp(\tau)>\theta_0$, let $\pi^\tau$ be the projection of vectors onto $\tau$ and let $M, m>0$ be the maximum and minimum of $\{||\nabla l_i(p)||,||\nabla^2l_i(p)(\tau,\cdot)||\}_{i,\tau}$ respectively. Suppose $\tau\in F_J\times T_X^{\sys\perp}\mathcal T$, and let $\langle\nabla l_{i_j},\tau\rangle=:C_1>C_2\ge\langle\nabla l_j,\tau\rangle$, for $j\neq i_j$, then
% $$||\pi^\tau(e^{-\frac12 \rho^2T \nabla^2l_{i_j}(\tau,\tau)}\nabla l_{i_j}(p))|| >e^{-\frac12 \rho^2T M}C_1,$$ $$||\pi^\tau(e^{-\frac12 \rho^2T \nabla^2l_{i_j}(\tau,\tau)}\nabla l_j(p))||<e^{-\frac12\rho^2Tm}C_2,$$ for $j\neq i_j$ and $$||\rho T\nabla^2l_{i_j}(p)(\tau,\cdot)||<\rho TM.$$
% Therefore, $$||\Psi_1||=\sum_ie^{\rho\langle\nabla l_i(p),\tau\rangle}e^{-\frac12\rho^2 T \nabla^2l_i(\tau,\tau)}\nabla l_i(p)>ke^{\rho C_1}e^{-\frac12 \rho^2T M}C_1-\#S(p)e^{\rho C_2}e^{-\frac12\rho^2Tm}C_2$$ and $$||\Psi_2||=\sum_ie^{\rho\langle\nabla l_i(p),\tau\rangle}\rho T\nabla^2l_i(p)(\tau,\cdot)<\#S(p)e^{\rho M}\rho TM$$
% when $T$ is small enough, $||\Psi_1||>>||-i(\Psi_2)||=||\Psi_2||$.
% \\
% \\
% When $\theta^\perp(\tau)>\theta_0$, there exists $C>0$ such that $\langle\nabla l_{i_1},\tau\rangle = \cdots = \langle\nabla l_{i_k},\tau\rangle > C$ for any $\tau\in F_J$.\\
% Fix $\rho$, when $T<\rho^{-3}$, $$e^{\rho\langle\nabla l_i(p),\tau\rangle}e^{-\frac12\rho^{-1}\nabla^2l_i(\tau,\tau)}||\nabla l_i(p)|| <e^{\rho\langle\nabla l_i(p),\tau\rangle}e^{-\frac12\rho^2 T \nabla^2l_i(\tau,\tau)}||\nabla l_i(p)||<e^{\rho\langle\nabla l_i(p),\tau\rangle}||\nabla l_i(p)||.$$
% Let $$\hat\Psi_1(-\tau)=\sum e^{\rho\langle\nabla l_i(p),\tau\rangle}e^{-\frac12\rho^{-1}\nabla^2l_i(\tau,\tau)}\nabla l_i(p),$$ then $||\Phi^i||<||\Psi_1^i||<||\hat\Psi_1^i||$ when $T<\rho^{-3}$.\\
% When $\tau\in F_J$, $\pi\circ\Psi_1(t,-\tau)\to\pi\circ\Phi_T(t,-\tau)\to\pi(\sum_j\nabla l_{i_j})$ and $||\Psi_1||>e^{\rho C}m$. On the other hand, $$||\Psi_2(t,-\tau)||<\#S(p)e^{\rho M}\rho TM.$$ Therefore, when $T$ is small enough, $||\Psi_1||>>||-i(\Psi_2)||=||\Psi_2||$.\\
% Consider that $$\pi(\sum e^{\rho\langle\nabla l_i(p),\tau\rangle}e^{-\frac12\rho^{-1}\nabla^2l_i(\tau,\tau)}\nabla l_i(p))\to\pi(\sum_j\nabla l_{i_j})$$ as $\rho\to\infty$, and choose $\rho$ large enough such that $\angle(\Psi_1,\tau)<\frac{\pi}{2}$.
% \\
% \\
% Since $e^{\rho\langle\nabla l_i(p),\tau\rangle}e^{-\frac12\rho^{-1}\nabla^2l_i(\tau,\tau)}$ is asymptotic to $e^{\rho\langle\nabla l_i(p),\tau\rangle}$ as $\rho\to\infty$, and $\pi\circ\Phi_T(t,-\tau)|_{t=\rho T}\to\pi(\sum_j\nabla l_{i_j})$ as $\rho\to\infty$, it is true that $\pi\circ\hat\Psi_1(-\tau)|_{t=\rho T}\to\pi(\sum_j\nabla l_{i_j})$ as well. Note that the convergence of both is independent of $T$. Choose $\rho$ sufficiently large such that $\angle(\Phi_T(t,-\tau)|_{t=\rho T},\tau)<\frac{\pi}{2}$ and $\angle(\hat\Psi_1(-\tau),\tau)<\frac{\pi}{2}$.\\
% \\
% When $T<\rho^{-3}$, $$e^{\rho\langle\nabla l_i(p),\tau\rangle}e^{-\frac12\rho^{-1}\nabla^2l_i(\tau,\tau)}||\nabla l_i(p)|| <e^{\rho\langle\nabla l_i(p),\tau\rangle}e^{-\frac12\rho^2 T \nabla^2l_i(\tau,\tau)}||\nabla l_i(p)||<e^{\rho\langle\nabla l_i(p),\tau\rangle}||\nabla l_i(p)||.$$
% Therefore, $\pi\circ\Psi_1(t,-\tau)\to\pi(\sum_j\nabla l_{i_j})$, and $\angle(\Psi_1(-\tau),\tau)<\frac{\pi}{2}$.
% \\
% \\
\\
Case 2: When $\angle(\tau,T_p^{\sys\perp}\mathcal T)\ge\theta_0$, similar to what we have in (4) in Lemma \ref{fan}, by compactness of the region of such $\tau$'s in this case, there exists $D>0$ such that $\langle\nabla l_j(p),\tau\rangle > D$ for any $\tau$ and all $j\in J(\tau)$. Let $M, m>0$ be the maximum and minimum of $\{||\nabla l_i(p)||,||\nabla^2l_i(p)(\tau,\cdot)||\}_{i,\tau}$ respectively, where positivity is guaranteed by Theorem \ref{convex} again.\\
\\
Let $K=\{k:\langle\nabla l_k,\tau\rangle\le0\}$ that is disjoint from $J(\tau)$. For any $j\in J(\tau)$ and $k\in K$, we have the following estimates:
\begin{align*}
    ||\Psi_1^j(t,-\tau)||&>\langle\Psi_1^j(t,-\tau),\tau\rangle\\
    & = e^{\rho\langle\nabla l_j(p),\tau\rangle}e^{-\frac12\rho^2 T \nabla^2l_j(p)(\tau,\tau)}\langle\nabla l_j(p),\tau\rangle\\
    & > e^{\rho D}e^{-\frac12\rho^2 TM}D,
\end{align*}
\begin{align*}
    ||\Psi_1^k(t,-\tau)||&=e^{\rho\langle\nabla l_k(p),\tau\rangle}e^{-\frac12\rho^2 T \nabla^2l_k(p)(\tau,\tau)}||\nabla l_k(p)||\\
    &\le 2e^{\frac12\rho^2 TM}M,
\end{align*}
and
\begin{align*}
    ||\Psi_2(t,-\tau)||&\le \sum_ie^{\rho\langle\nabla l_i(p),\tau\rangle}\rho T||\nabla^2l_i(p)(\tau,\cdot)||\\
    &\le re^{\rho M}\rho TM,
\end{align*}
where $r=\#S(p)$.\\
\\
% since $\langle\nabla l_k(p),\tau\rangle-\langle\nabla l_j(p),\tau\rangle<-D$,
% \begin{align*}
%     \frac{||\Psi_1^k(t,-\tau)||}{||\Psi_1^j(t,-\tau)||}|_{t=\rho T} & = e^{\rho(\langle\nabla l_k(p),\tau\rangle-\langle\nabla l_j(p),\tau\rangle)}e^{-\frac12\rho^2T(\nabla^2l_k(p)(\tau,\tau)-\nabla^2l_j(p)(\tau,\tau))}\frac{||\nabla l_k(p)||}{||\nabla l_j(p)||}\\
%     & < e^{-\rho D}e^{\frac12\rho^{-1}(M-m)}\frac Mm\\
%     & \to 0 \text{\ as\ $\rho\to\infty$}
% \end{align*}
Note that all $\langle\Psi_1^i(t,-\tau),\tau\rangle\ge0$ for $i\in(J(\tau)\cup K)^C$, then putting all terms together we have
\begin{align*}
    \langle\Psi_T(t,-\tau)|_{t=\rho T},\tau\rangle & = \langle\Psi_1^J+\Psi_1^K+\Psi_1^{(J\cup K)^C}+\Psi_2,\tau\rangle\\
    & \ge \langle\Psi_1^J,\tau\rangle-||\Psi_1^K||-||\Psi_2||\\
    & \ge e^{\rho D}e^{-\frac12\rho^2TM}D-2e^{\frac12\rho^2TM}M-re^{\rho M}\rho TM,
\end{align*}
where $J=J(\tau)$ for short.
If we set $T<e^{-\rho M}$, the expression above will go to $+\infty$ as $\rho\to\infty$. Therefore, we fix $\rho$ such that $$\langle\Psi_T(t,-\tau)|_{t=\rho T},\tau\rangle>0$$
for any $T<e^{-\rho M}$.\\
% , then $||\Psi_1^K(t,-\tau)||=o(||\Psi_1^J(t,-\tau)||)$ restricted on $S^{n-1}_{\rho T}$ for $T<\rho^{-3}$ as $\rho\to\infty$. We can similarly study the direction of the limit of $\Psi_1$ as $\rho\to\infty$: $$\pi\circ\Psi_1^J(t,-\tau)|_{t=\rho T}=\pi(e^{\rho\langle\nabla l_j,\tau\rangle}\sum_{j\in J}e^{-\frac12\rho^2T\nabla^2l_j(\tau,\tau)}\nabla l_j(p))\to w_J(\tau),$$ where $w_J(\tau)$ is a vector in $\mathbb H(\tau)$ depending on $\tau$.\\
% \\
% Fix $0<\epsilon<\frac12$. Since convergence above for both is uniform in $T$ for $T<\rho^{-3}$, there exists $\rho$ sufficiently large, such that $||\Psi_1^K||<\frac12\epsilon||\Psi_1^J||$ and $\angle(\Psi_1^J,w_J)<\frac{\pi}{6}$. We can make $\rho$ independent of the fans $F_J$ by taking the maximum over all $F_J$'s.\\
% \\
% We can establish an estimate of $\Psi_1^i$:
% \begin{align*}
%     ||\Psi_1(t,-\tau)||<\#S(p)\max||\Psi_1^i||<\#S(p)e^{\rho M}M
% \end{align*}
% Since all $\nabla l_j$'s have positive projections onto $\tau$, a lower bound of $||\Psi_1^J||$ can be found:
% \begin{align*}
%     ||\Psi_1^J(t,-\tau)||>\langle\Psi_1^j(t,-\tau),\tau\rangle & = e^{\rho\langle\nabla l_j(p),\tau\rangle}e^{-\frac12\rho^2 T \nabla^2l_j(p)(\tau,\tau)}\langle\nabla l_j(p),\tau\rangle\\
%     & > e^{\rho D}e^{-\frac12\rho^2TM}D \to e^{\rho D}D>0,
% \end{align*}
% as $T\to 0^+$. On the other hand, $$||\Psi_2(t,-\tau)||<re^{\rho M}\rho TM \to 0,$$ as $T\to 0^+$, where $r=\#S(p)$.\\
\\
% Therefore, when $T$ is sufficiently small, $||\Psi_2||<\frac12\epsilon||\Psi_1^J||$, and by the following elementary lemma, $\angle(\Psi_T,\Psi_1^J)<\frac{\pi}{6}$. Putting all together, we have $$\angle(\Psi_T(t,-\tau),\tau)<\angle(\Psi_T,\Psi_1^J)+\angle(\Psi_1^J,w_J)+\angle(w_J,\tau)<\frac{\pi}{6}+\frac{\pi}{6}+\frac{\pi}{2}=\frac{5\pi}{6}.$$
% \begin{align*}
% \langle\Psi_T(t,-\tau)|_{t=\rho T},\tau\rangle & = \langle(\Psi_1^J+\Psi_1^K+\Psi_1^{(J\cup K)^C}+\Psi_2),\tau\rangle\\
% & \ge \langle\Psi_1^J,\tau\rangle-||\Psi_1^K||-||\Psi_2||\\
% & \ge (1-\epsilon)
% \end{align*}
With the choice of $\rho$ and the two cases considered above, $$\deg i\circ\pi\circ\Psi_T(\cdot,-\cdot)=1,$$ following the theorem.
\end{proof}
% \begin{lemma}
% \label{angle}
% Suppose $\epsilon<1$. Let $u,v$ be two vectors such that $||v||<\epsilon||u||$, then the angle $\angle(u,u+v)<\arccosh(\sqrt{1-\epsilon^2})$.
% \end{lemma}
\begin{lemma}
\label{norm2}
With the same notations above, we have the following estimates on the norms when $t=\rho T$:\\
(1) When $\angle(\tau,T_p^{\sys\perp}\mathcal T)\le\theta_0$, $$||\Psi_T(t,-\tau)||>\rho Te^{-\rho M\sin\theta_0}m\sin\theta_0\cos\theta_0,$$
(2) When $\angle(\tau,T_p^{\sys\perp}\mathcal T)\ge\theta_0$, $$||\Psi_T(t,-\tau)||>\frac12e^{\rho D}e^{-\frac12\rho^2TM}D.$$
Together we may claim that $||\Psi_T|_{t=\rho T}||>C(p)T.$
\end{lemma}
\begin{proof}
In the first case, let $\tau'$ be the projection of $\tau$ onto the minor subspace and consider the projection of $\Psi_T$ onto $\tau'$.\\
Note that $$\langle\nabla l_i(p),\tau\rangle=||\nabla l_i(p)||\cos\angle(\nabla l_i,\tau)>-M\sin\theta_0$$ and as a reminder, $\theta_0$ is chosen such that $\angle(\nabla^2l_i(\tau,\cdot),\tau)<\frac{\pi}{2}-2\theta_0$, so $$\angle(\nabla^2l_i(p)(\tau,\cdot),\tau')<\angle(\nabla^2l_i(p)(\tau,\cdot),\tau)+\angle(\tau,\tau')<\frac{\pi}{2}-\theta_0$$ and
\begin{align*}
    \langle\nabla^2l_i(p)(\tau,\cdot),\tau'\rangle&=||\nabla^2l_i(p)(\tau,\cdot)||\cdot||\tau'||\cos\angle(\nabla^2l_i(p)(\tau,\cdot),\tau')\\
    &>m\sin\theta_0\cos\theta_0.
\end{align*}
Therefore,
\begin{align*}
    \langle\Psi_T,-\tau'\rangle&=\sum_ie^{\rho\langle\nabla l_i(p),\tau\rangle}\rho T\langle\nabla^2l_i(p)(\tau,\cdot),\tau'\rangle\\
    &>\rho Te^{-\rho M\sin\theta_0}m\sin\theta_0\cos\theta_0
\end{align*}
The first claim follows. The second follows directly from the definition of $\rho$.
\end{proof}
\noindent
What Theorem \ref{existence2} tells us is that $\Psi_T$ has a zero in the interior of $S^{n-1}_{\rho T}$ by Lemma \ref{degree}. Now we are ready to prove the same conclusion for $\tilde\Omega_T$:
\begin{theorem}[Existence]
\label{main}
With the same $\rho$ above, $\syst$ has a critical point in the interior of the $\rho T$-sphere $S^{n-1}_{\rho T}\subset T_p\mathcal T$ when $T$ is sufficiently small.
\end{theorem}
\begin{proof}
\noindent
To recall the idea, we take a two-step approximation $$\tilde\Omega_T\approx \Omega_T\approx \Psi_T.$$
% we approximate $\tilde\Omega_T$ with $\Omega_T$, and then $\Omega_T$ with $\Psi_T$.
Now let us consider the errors.\\
\\
When $T$ is sufficiently small, any point $X$ in the geodesic ball $\ex_p(B_{\rho T})$ centered at $p$ satisfies that $\secsys(X)>\secsys(p)$, and thus by Lemma \ref{C1tail}, $$||\tilde\Omega_T(X)-\Omega_T(X)||<c''e^{-\frac1T\secsys(X)}<c''e^{-\frac1T\secsys(p)}.$$
For the second approximation, since $\rho$ is already fixed as in Theorem \ref{existence2}, by Definition \ref{ep2}, $$||\Omega_T(X)-\Psi_T(X)||=O(T^2).$$
Therefore, the total error can be given by $$||\tilde\Omega_T(X)-\Psi_T(X)||=O(T^2).$$
Since the angle between $\tau$ and $i\circ\Psi_T(t,-\tau)|_{S^{n-1}_{\rho T}}$ can be bounded by $\frac{\pi}{2}+\theta_0$ (which is not optimal), and $||i\circ\Psi_T(t,-\tau)|_{S^{n-1}_{\rho T}}||>C(p)T$, by the lemma above, with the error taken into effect, Lemma \ref{degree} is satisfied, and $$\deg(\tilde\Omega_T)=\deg(\Psi_T)=(-1)^d.$$There existence follows.
\end{proof}
\begin{corollary}
Suppose uniqueness (which will be proved in Theorem \ref{unique}) is satisfied for the existence above. Let $p_T$ denote the unique critical point for $\syst$ near $p$, then $d(p,p_T)$ is sublinear in $T$, meaning $d(p,p_T)<CT$ for some $C$.
\end{corollary}
\noindent
We also prove the nonexistence of critical points outside of the $\rho T$-ball centered at $p$:
\begin{theorem}[Nonexistence outside the $\rho T$-ball]
There exist $\rho_1>0$, $\epsilon_1=\epsilon_1(\rho)>0$ and $r_p>0$ for each $p$, such that there are no critical points of $\syst$ with distance from $p$ in $[\rho T,r_p]$ for $\rho>\rho_1$ and $T<\min\{\epsilon_1,\frac{r_p}{\rho}\}$.
\end{theorem}
\begin{proof}
We recall that $0<\theta_0<\frac{\pi}{6}$ is chosen such that for all $i$, $$\angle(\nabla^2l_i(\tau,\cdot),\tau)<\frac{\pi}{2}-2\theta_0.$$ For $\tau\not\in T_p^{\sys\perp}\mathcal T$, let $C^{\sys}(\tau,\theta)$ be the cone in the major subspace centered at $\proj_{T_p^{\sys}\mathcal T}\tau$ with angle $\theta$. Note that $v\in C^{\sys}(\tau,\theta)\times T_p^{\sys\perp}\mathcal T$ if and only if $$\angle(\proj_{T_p^{\sys}\mathcal T}v,\proj_{T_p^{\sys}\mathcal T}\tau)<\theta.$$ For example, if $\tau\in T_p^{\sys}\mathcal T$, then $C^{\sys}(\tau,\frac12\pi)\times T_p^{\sys\perp}\mathcal T=\mathbb H(\tau)$.\\
\\
Let $U_p=B(p,r_p)$ be a geodesic ball centered at $p$ and $C>0$ a constant with $Cr_p<\frac12$, such that for any $q\in\overline{U}_p$, if we let $t=d(p,q)$ and $\tau=\frac{1}{t}\ex_p^{-1}(q)$, the following conditions are satisfied:\\
\\
% (1) Following fan decomposition, there exists $0<\theta_1<\frac12\pi$ such that for any $\tau\not\in T_p^{\sys\perp}\mathcal T$, $$C^{\sys}(\tau,\frac12\pi-\theta_1)\cap\{\nabla l_i\}\neq\emptyset,$$ and $l_i(q)>l_j(q)$ for $\nabla l_j(p)\in C^{\sys}(-\tau,\frac12\pi-\frac12\theta_1)$ and $\nabla l_i(p)\in\mathbb H(\tau)$.\\
(1) By continuity, $$|l_i(q)-l_i(p)-t\langle\nabla l_i(p),\tau\rangle|<Ct^2,$$ for all $i$ and $$l_j(q)-l_j(p)<0,$$  for $j\in J(-\tau)$,\\
(2) By continuity, $$||\nabla l_i(q)-\nabla l_i(p)||<Ct,$$ for all $i$ and consequently $$\angle(\nabla l_i(q),\nabla l_i(p))<\frac12\min\{\theta_0,\frac12\pi-\theta_1\},$$ where $\theta_1$ is chosen in Case 2 below, and $$\frac12<\frac{||\nabla l_i(q)||}{||\nabla l_i(p)||}<\frac32,$$
(3) By continuity, $$||\nabla l_i(q)-\nabla l_i(p)-t\nabla^2l_i(\tau,\cdot)||<Ct^2,$$ and by taking projection onto the minor subspace, $$t||\proj_{T_p^{\sys\perp}\mathcal T}\nabla^2l_i(\tau,\cdot)||-Ct^2>0,$$
(4) By continuity, $$\max\{l_i(q)\}<\min_{\gamma\not\in S(p)}{l_\gamma(q)}.$$
$$\sys(p)+(r_p||\nabla l_i||\sin\theta_0+2Cr_p^2)<\min_{\gamma\not\in S(p)}{l_\gamma(q)}$$ for all $i$.\\
\\
Condition (4) guarantees that we only need to consider the sum over $\gamma_i\in S(p)$ since the `tail' will be dominantly smaller. Given these constants and conditions, we again consider the following two cases:\\
\\
Case 1: if $\angle(\tau,T_p^{\sys\perp}\mathcal T)\le\theta_0$, then
$$\angle(\tau,\nabla l_i(p))\ge\frac{\pi}{2}-\theta_0,$$ for all $i\in I$, and therefore,
% \begin{align*}
%     ||\sum_ie^{-\frac1Tl_i(q)}\nabla l_i(q)||&\ge||\proj_{T_p^{\sys\perp}\mathcal T}(\sum_ie^{-\frac1Tl_i(q)}\nabla l_i(q))||\\
%     &\ge\langle \sum_ie^{-\frac1Tl_i(q)}\nabla l_i(q),\tau\rangle
% \end{align*}
\begin{align*}
    e^{-\frac1Tl_i(q)}&>e^{-\frac1T\sys(p)}e^{-\frac1T(t\langle\nabla l_i(p),\tau\rangle+Ct^2)}\\
    &>e^{-\frac1T\sys(p)}e^{-\frac1T(t||\nabla l_i(p)||\sin\theta_0+Ct^2)}\\
    &>e^{-\frac1T\sys(p)}e^{-\frac1T(r_p||\nabla l_i(p)||\sin\theta_0+Cr_p^2)}\\
    &>e^{-\frac1T(\min_{\gamma\not\in S(p)}{l_\gamma(q)}-Cr_p^2)}
\end{align*}
Take the projection of $\sum_ie^{-\frac1Tl_i(q)}\nabla l_i(q)$ onto the minor subspace, then we have, for a single $\nabla l_i(q)$,
\begin{align*}
    ||\proj_{T_p^{\sys\perp}\mathcal T}\nabla l_i(q)||&>t||\proj_{T_p^{\sys\perp}\mathcal T}\nabla^2l_i(\tau,\cdot)||-Ct^2\\
    &\ge \rho T||\proj_{T_p^{\sys\perp}\mathcal T}\nabla^2l_i(\tau,\cdot)||-Cr_p\rho T=:C_1(p)T,
\end{align*}
and for the main part,
\begin{align*}
    &||\sum_ie^{-\frac1Tl_i(q)}\nabla l_i(q)||\\
    &\ge||\proj_{T_p^{\sys\perp}\mathcal T}\sum_ie^{-\frac1Tl_i(q)}\nabla l_i(q)||\\
    &\ge\sum_ie^{-\frac1Tl_i(q)}||\proj_{T_p^{\sys\perp}\mathcal T}(\nabla l_i(q))||\\
    &>\sum_ie^{-\frac1T\sys(p)}e^{-\frac1T(r_p||\nabla l_i(p)||\sin\theta_0+Cr_p^2)}\left(t||\proj_{T_p^{\sys\perp}\mathcal T}\nabla^2l_i(\tau,\cdot)||-Ct^2\right)\\
    &>e^{-\frac1T(\min_{\gamma\not\in S(p)}{l_\gamma(q)}-Cr_p^2)}C_1(p)T.
\end{align*}
We thus have the positivity of the norm of the gradient vector, as  the tail is dominantly smaller:
\begin{align*}
    ||\sum_\gamma e^{-\frac1Tl_i(q)}\nabla l_i(q)||&\ge||\proj_{T_p^{\sys\perp}\mathcal T}\sum_ie^{-\frac1Tl_i(q)}\nabla l_i(q)||-||\sum_{\gamma\not\in S(p)} e^{-\frac1Tl_i(q)}\nabla l_i(q)||\\
    &>C_1(p)Te^{-\frac1T(\min_{\gamma\not\in S(p)}{l_\gamma(q)}-Cr_p^2)}-C_2(p)e^{-\frac1T \min_{\gamma\not\in S(p)}{l_\gamma(q)}}>0.
\end{align*}
Case 2: if $\angle(\tau,T_p^{\sys\perp}\mathcal T)\ge\theta_0$. By eutacticity, there exists $0<\theta_1<\frac12\pi$ such that for any $\tau\not\in T_p^{\sys\perp}\mathcal T$, $$C^{\sys}(\tau,\frac12\pi-\theta_1)\cap\{\nabla l_i\}\neq\emptyset.$$
Let $$J:=\{j:\nabla l_j(p)\in C^{\sys}(-\tau,\frac12\pi-\theta_1)\}$$
(note that $J$ is not short for $J(-\tau)$ in this case) and $$K:=\{k:\nabla l_k(p)\in C^{\sys}(\tau,\frac{\pi}{2}+\frac12\theta_1)\}.$$ By continuity, there exist $D_1<0$ and $D_2$ constant, such that when we can make $r_p$ small enough, for any $q\in U_p=B(p,r_p)$:\\
(1) For $j\in J$, $$\langle\nabla l_j(q),\tau\rangle<D_1.$$
(2) For $k\in K$, $$\langle\nabla l_k(q),\tau\rangle>D_2.$$
(3) For $i\in (J\cup K)^C$, $$\langle\nabla l_i(q),\tau\rangle<0.$$
(4) $D_1<D_2$.\\
\\
For any $j\in J$ and $k\in K$, we have
\begin{align*}
    \frac{||e^{-\frac1Tl_k(q)}\nabla l_k(q)||}{||e^{-\frac1Tl_j(q)}\nabla l_j(q)||}&<\frac{\frac32e^{-\frac1Tl_k(q)}||\nabla l_k(p)||}{\frac12e^{-\frac1Tl_j(q)}||\nabla l_j(p)||}\\
    &=3e^{-\frac1T(l_k(q)-l_j(q))}\frac{||\nabla l_k(p)||}{||\nabla l_j(p)||}\\
    &<3e^{\frac1T(-t(\langle\nabla l_k(p),\tau\rangle-\langle\nabla l_j(p),\tau\rangle)+2Ct^2)}\frac{||\nabla l_k(p)||}{||\nabla l_j(p)||}\\
    &<3e^{\frac1T(t(D_1-D_2)+2Ct^2)}\frac{||\nabla l_k(p)||}{||\nabla l_j(p)||}\\
    &<3e^{\rho(D_1-D_2)+2Cr_p}\frac{||\nabla l_k(p)||}{||\nabla l_j(p)||}<\epsilon,
\end{align*}
where $\epsilon<\frac1r$, if $\rho$ is sufficiently large.\\
% Similar to part (4) in Lemma \ref{fan}, there exist $D_1>0$ and $D_2$, such that $$\max_{\tau\in F_J;j\in J}\{\langle\nabla l_j(p),\tau\rangle\}=:D_1>D_2:=\max_{\tau;k\in K}\{\langle\nabla l_k(p),\tau\rangle\}.$$
We consider the projection of $\sum_ie^{-\frac1Tl_i(q)}\nabla l_i(q)$ onto the major subspace. For any $\nabla l_j(p)\in C^{\sys}(-\tau,\frac12\pi-\theta_1)$,
\begin{align*}
    -\langle e^{-\frac1Tl_j(q)}\nabla l_j(q),\tau\rangle&>-e^{-\frac1T\sys(p)}e^{-\frac1T(l_j(q)-l_j(p))}\langle\nabla l_j(q),\tau\rangle\\
    &>-D_1e^{-\frac1T\sys(p)}e^{\frac1T(-t\langle\nabla l_j,\tau\rangle-Ct^2)}\\
    &>-D_1e^{-\frac1T\sys(p)}e^{\frac1T(-D_1\rho T-Cr_p\rho T)}\\
    &=-D_1e^{-\frac1T\sys(p)}e^{\rho(-D_1-Cr_p)}
\end{align*}
% \begin{align*}
%     ||e^{-\frac1Tl_j(q)}\nabla l_j(q)||&>\frac12e^{-\frac1T\sys(p)}e^{-\frac1T(l_j(q)-l_j(p))}||\nabla l_j(p)||\\
%     &>\frac12e^{-\frac1T\sys(p)}e^{\frac1T(-t\langle\nabla l_j,\tau\rangle-Ct^2)}||\nabla l_j(p)||\\
%     &>\frac12e^{-\frac1T\sys(p)}e^{\frac1T(-D_1\rho T-Cr_p\rho T)}||\nabla l_j(p)||\\
%     &=\frac12e^{-\frac1T\sys(p)}e^{\rho(-D_1-Cr_p)}||\nabla l_j(p)||
% \end{align*}
% \begin{align*}
%     ||e^{-\frac1Tl_i(q)}\nabla l_i(q)||&<\frac32e^{-\frac1T(l_i(q)-l_i(p))}||\nabla l_i(p)||\\
%     &<\frac32e^{-\frac1T(l_j(q)-l_i(p))}||\nabla l_i(p)||
% \end{align*}
Note that, for any $i\in(J\cup K)^C$, $\nabla l_i(q)$ has a negative projection onto $\tau$. Therefore,
\begin{align*}
    ||\sum_Ie^{-\frac1Tl_i(q)}\nabla l_i(q)||&>-\langle\sum_{K^C}e^{-\frac1Tl_i(q)}\nabla l_i(q),\tau\rangle-||\sum_Ke^{-\frac1Tl_k(q)}\nabla l_k(q)||\\
    &\ge-\langle\sum_Je^{-\frac1Tl_j(q)}\nabla l_j(q),\tau\rangle-||\sum_Ke^{-\frac1Tl_k(q)}\nabla l_k(q)||\\
    &\ge-(1-r\epsilon)\langle e^{-\frac1Tl_j(q)}\nabla l_j(q),\tau\rangle
\end{align*}
% If we have any other $\nabla l_i(p)\in C(\tau,\frac12\pi+\frac12\theta_1)$, choose $0<\epsilon<\frac 1r$, where $r=\#S(p)$ as a reminder,
% \begin{align*}
%     \frac{||e^{-\frac1Tl_i(q)}\nabla l_i(q)||}{||e^{-\frac1Tl_j(q)}\nabla l_j(q)||}&<\frac{\frac32e^{-\frac1Tl_i(q)}||\nabla l_i(p)||}{\frac12e^{-\frac1Tl_j(q)}||\nabla l_j(p)||}\\
%     &=3e^{-\frac1T(l_i(q)-l_j(q))}\frac{||\nabla l_i(p)||}{||\nabla l_j(p)||}\\
%     &<3e^{\frac1T(-t(\langle\nabla l_i(p),\tau\rangle-\langle\nabla l_j(p),\tau\rangle)+2Ct^2)}\frac{||\nabla l_i(p)||}{||\nabla l_j(p)||}\\
%     &<3e^{\frac1T(-D't+2Ct^2)}\frac{||\nabla l_i(p)||}{||\nabla l_j(p)||}\\
%     &<3e^{\frac1T(-D'\rho T+2C(\rho T)^2)}\frac{||\nabla l_i(p)||}{||\nabla l_j(p)||}<\epsilon,
% \end{align*}
% when $T$ is small enough, where we modify $r_p$ to be the smaller one of $r_p$ and $\frac {D'}{C}$ in the last step to make the inequality hold.\\
% Note that, if $\nabla l_i(p)\in C(-\tau,\frac12\pi-\frac12\theta_1)$, then $\nabla l_i(q)\in \mathbb H(\tau)$. Therefore,
% \begin{align*}
%     ||\sum_ie^{-\frac1Tl_i(q)}\nabla l_i(q)||&>||\sum_{\nabla l_i(p)\in\overline{C(-\tau,\frac12\pi-\frac12\theta_1)}}e^{-\frac1Tl_i(q)}\nabla l_i(q)||\\
%     &-||\sum_{\nabla l_i(p)\in C(\tau,\frac12\pi+\frac12\theta_1)}e^{-\frac1Tl_i(q)}\nabla l_i(q)||\\
%     &\ge||\langle\sum_{\nabla l_i(p)\in C(-\tau,\frac12\pi-\theta_1)}e^{-\frac1Tl_i(q)}\nabla l_i(q),\tau\rangle||\\
%     &-||\sum_{\nabla l_i(p)\in C(\tau,\frac12\pi+\frac12\theta_1)}e^{-\frac1Tl_i(q)}\nabla l_i(q)||\\
%     &\ge (1-r\epsilon)||e^{-\frac1T(l_j(q)-sys(p))}\nabla l_j(q)||>0
% \end{align*}
Put the tail in,
\begin{align*}
    ||\sum_\gamma e^{-\frac1Tl_i(q)}\nabla l_i(q)||&>-(1-r\epsilon)\langle e^{-\frac1Tl_j(q)}\nabla l_j(q),\tau\rangle-||\sum_{\gamma\not\in S(p)} e^{-\frac1Tl_i(q)}\nabla l_i(q)||\\
    &>-(1-r\epsilon)D_1e^{-\frac1T\sys(p)}e^{\rho(-D_1-Cr_p)}-C_2(p)e^{-\frac1T \min_{\gamma\not\in S(p)}{l_\gamma(q)}}\\
    &>C_3(p)e^{-\frac1T\sys(p)}-C_2(p)e^{-\frac1T \min_{\gamma\not\in S(p)}{l_\gamma(q)}}>0
\end{align*}
The theorem follows.
\end{proof}
\begin{corollary}
Let $p_1,p_2$ be two critical points, then $$d(p_1,p_2)>\max\{r_{p_1},r_{p_2}\}.$$
\end{corollary}

\section{Nondegeneracy and Local Uniqueness of Critical Points}
\label{nondegeneracy}
\noindent
As the actual systole function is topologically Morse, we naturally hope the $C^2$ functions $\syst$ in the approximation family are $C^2$-Morse or even smooth Morse, when $T$ is sufficiently close to 0. To proceed, with the same notations above, we directly calculate the Hessian $\tilde H_{\syst}$ of $\syst$ on the tangent vector field $\tau=u'$ for any geodesic $u$:
\begin{align*}
\tilde H_T(\tau)&:=T\left(\sum_\gamma e^{-{\frac1T}l_\gamma}\right)^2\tilde H_{\syst}(\tau,\tau)\\ &= \left(\sum_\gamma e^{-{\frac1T}l_\gamma}\langle\nabla l_\gamma,\tau\rangle\right)^2-\left(\sum_\gamma e^{-{\frac1T}l_\gamma}\langle\nabla l_\gamma,\tau\rangle^2\right)\left(\sum_\gamma e^{-{\frac1T}l_\gamma}\right)\\
& +T\left(\sum_\gamma e^{-{\frac1T}l_\gamma}\right)\left(\sum_\gamma e^{-{\frac1T}l_\gamma}\nabla^2l_\gamma(\tau,\tau)\right)
\end{align*}
Let $H_T$ be the `main part' of $\tilde H_T$, i.e., where every sum above is over $S(p)$ instead of all simple closed geodesics. By Theorem \ref{norm} and a similar calculation as in Lemma \ref{C1tail} and Theorem \ref{main},
\begin{lemma}
\label{C2tail}
We have
\begin{align*}
    |(\tilde H_T)_p-(H_T)_p|< c'''e^{-\frac1T(\sys(p)+\secsys(p))}.
\end{align*}
\end{lemma}
\begin{proof}
    Note that
    \begin{align*}
        \sum_\gamma e^{-{\frac1T}l_\gamma}\nabla^2l_\gamma(\tau,\tau)\le c\sum_\gamma e^{-{\frac1T}l_\gamma}(1+l_\gamma e^{\frac{l_\gamma}{2}})\le c(p)e^{-\frac1T\secsys(p)},
    \end{align*}
    then the lemma follows from Theorem \ref{C0} and Lemma \ref{C1tail}.
\end{proof}
\noindent
Our hope is that $\tilde H_T$ is nondegenerate at critical points of $\syst$, but first as an observation we have the following:
\begin{lemma}
When $T$ is sufficiently small, $(\tilde H_T)_p:T_p\mathcal T\to\mathbb R$ is nondegenerate.
\end{lemma}
\begin{proof}
Since all $\gamma_i$'s attain the same value at $p$, the above can be simplified as $$(H_T)_p(\tau)=-e^{-\frac2T\sys(p)}\left( r\sum_i\langle\nabla l_i,\tau\rangle^2-\left(\sum_i\langle\nabla l_i,\tau\rangle\right)^2-Tr\sum_i\nabla l_i^2(p)(\tau,\tau)\right),$$ where $r:=\#S(p)$ as a reminder.\\
\\
If $\tau\in T_p^{\sys\perp}\mathcal T$, then $$(\tilde H_T)_p(\tau)>Tre^{-\frac2T\sys(p)}\sum_i\nabla^2l_i(p)(\tau,\tau)-C_pe^{-\frac1T(\sys(p)+\secsys(p))}>0.$$
If $\tau\in T_p^{\sys}\mathcal T$, by part (5) of Lemma \ref{fan} and Cauchy's inequality,
$$r\sum_i\langle\nabla l_i,\tau\rangle^2-\left(\sum_i\langle\nabla l_i,\tau\rangle\right)^2>0.$$
Similar to part (4) in Lemma \ref{fan}, there exists $D>0$ (we may assume the same $D$ for both Lemma \ref{fan} and here for simplicity), such that $$\min_i\{\langle\nabla l_i(p),\tau\rangle\}<-D.$$ If we let $$r_1:=\#\{i:\langle\nabla l_i,\tau\rangle\ge 0\}$$ and $$r_2:=\{i:\langle\nabla l_i,\tau\rangle< 0\},$$ then we have \begin{align*}
& r\sum_i\langle\nabla l_i,\tau\rangle^2-\left(\sum_i\langle\nabla l_i,\tau\rangle\right)^2\\
& = r\left(\sum_{\langle\nabla l_i,\tau\rangle\ge 0}\langle\nabla l_i,\tau\rangle^2 +\sum_{\langle\nabla l_i,\tau\rangle<0}\langle\nabla l_i,\tau\rangle^2\right)\\
& - \left(\sum_{\langle\nabla l_i,\tau\rangle\ge0}\langle\nabla l_i,\tau\rangle+\sum_{\langle\nabla l_i,\tau\rangle\ge0}\langle\nabla l_i,\tau\rangle\right)^2\\
& = \left(r_1\sum_{\langle\nabla l_i,\tau\rangle\ge 0}\langle\nabla l_i,\tau\rangle^2-\left(\sum_{\langle\nabla l_i,\tau\rangle\ge0}\langle\nabla l_i,\tau\rangle\right)^2\right)\\
& + \left(r_2\sum_{\langle\nabla l_i,\tau\rangle< 0}\langle\nabla l_i,\tau\rangle^2-\left(\sum_{\langle\nabla l_i,\tau\rangle<0}\langle\nabla l_i,\tau\rangle\right)^2\right)\\
& +r_2\sum_{\langle\nabla l_i,\tau\rangle\ge 0}\langle\nabla l_i,\tau\rangle^2 + r_1\sum_{\langle\nabla l_i,\tau\rangle< 0}\langle\nabla l_i,\tau\rangle^2\\
& -2\left(\sum_{\langle\nabla l_i,\tau\rangle\ge0}\langle\nabla l_i,\tau\rangle\right)\left(\sum_{\langle\nabla l_i,\tau\rangle<0}\langle\nabla l_i,\tau\rangle\right)\\
& \ge r_2\sum_{\langle\nabla l_i,\tau\rangle\ge 0}\langle\nabla l_i,\tau\rangle^2 + r_1\sum_{\langle\nabla l_i,\tau\rangle< 0}\langle\nabla l_i,\tau\rangle^2\\
& -2\left(\sum_{\langle\nabla l_i,\tau\rangle\ge0}\langle\nabla l_i,\tau\rangle\right)\left(\sum_{\langle\nabla l_i,\tau\rangle<0}\langle\nabla l_i,\tau\rangle\right)\\
& \ge r_2D^2+r_1D^2+2D^2\ge4D^2
\end{align*}
Going back to $(H_T)_p$, we have
$$(H_T)_p(\tau)\le -e^{-\frac2T\sys(p)}\left( 4D^2-Tr^2M\right),$$
and
\begin{align*}
    (\tilde H_T)_p(\tau) & < (H_T)_p(\tau)+C_pe^{-\frac1T(\sys(p)+\secsys(p))}\\
    &\le -e^{-\frac2T\sys(p)}\left( 4D^2-Tr^2M\right)+C_pe^{-\frac1T(\sys(p)+\secsys(p))}
\end{align*}
which is negative when $T$ is sufficiently small as $\secsys>\sys$.\\
\\
Therefore, when $T$ is small enough, $H_T(\tau)<0$.
\end{proof}
\noindent
Given a critical point $p_T$ of $\syst$, the Hessian can be simplified as
$$(\tilde H_T)_{p_T}(\tau) = \left(\sum_\gamma e^{-{\frac1T}l_\gamma}\right)\left(\sum_\gamma e^{-{\frac1T}l_\gamma}\left(-\langle\nabla l_\gamma,\tau\rangle^2 +T\nabla^2l_\gamma(\tau,\tau)\right)\right).$$
\begin{theorem}
Let $p_T$ be a critical point (no uniqueness established yet) for $\syst$ as in Theorem \ref{main} when $T$ is sufficiently small, then $(\tilde H_T)_{p_T}:T_{p_T}\mathcal T\to\mathbb R$ is nondegenerate.
\end{theorem}
\begin{proof}
Let $v_T=\ex_p^{-1}(p_T)\in T_p\mathcal T$, then $d(\ex_p)_{v_T}:T_{v_T}T_p\mathcal T\cong T_p\mathcal T\to T_{p_T}\mathcal T$ is an isomorphism when $T$ is small. By smoothness, let $C=C(p)>0$ satisfy $$|d(\ex_p)_{v_T}(\nabla l_i(p))-\nabla l_i(p_T)|<Cd(p,p_T)<C\rho T.$$
We also let $$\text{tail}=\sum_{\gamma\not\in S(p)} e^{-{\frac1T}l_\gamma}\left(-\langle\nabla l_\gamma,\tau\rangle^2 +T\nabla^2l_\gamma(\tau,\tau)\right),$$
and then $|\text{tail}|<C_pe^{-\frac1T\secsys(p)}$, similar as above.\\
\\
Instead of the tangent space $T_{p_T}\mathcal T$ at $p_T$, we consider the push forward of the tangent subspaces $T_p^{\sys}\mathcal T$ and $T_p^{\sys\perp}\mathcal T$ at $p$. By continuity, results similar to part (4) of Lemma \ref{fan} and the bounds of $\{||\nabla l_i(p_T)||,||\nabla^2l_i(p_T)(\tau,\cdot)||\}_{i,\tau; T\le\epsilon}$ can be established. For simplicity, by abuse of notation, we still denote the bounds by $D,M,m$. We also assume $l_i(p_T)<\frac12\left(\sys(p)+\secsys(p)\right)<\secsys(p)$ for all $\gamma_i\in S(p)$ by making $T$ small enough.\\
\\
When $\tau\in d(\ex_p)_{v_T}(T_p^{\sys}\mathcal T)$,
\begin{align*}
    &\left(\sum_\gamma e^{-{\frac1T}l_\gamma}\right)^{-1}\cdot (\tilde H_T)_{p_T}(\tau)\\
    &\le -e^{-\frac1T\frac12(\sys(p)+\secsys(p))}(D^2-rTM) + C_pe^{-\frac1T\secsys(p)}<0,
\end{align*}
when $T$ is sufficiently small.\\
\\
When $\tau\in d(\ex_p)_{v_T}(T_p^{\sys\perp}\mathcal T)$,
\begin{align*}
    &\left(\sum_\gamma e^{-{\frac1T}l_\gamma}\right)^{-1}\cdot (\tilde H_T)_{p_T}(\tau)\\
    &\ge re^{-\frac1T\frac12(\sys(p)+\secsys(p))}(-(2C\rho T)^2+Tm) - C_pe^{-\frac1T\secsys(p)}>0,
\end{align*}
when $T$ is sufficiently small.\\
\\
The theorem follows since $d(\ex_p)_{v_T}$ is an isomorphism.
\end{proof}
\noindent
If we say the \textit{index} of a (nondegenerate) quadratic form is the number of negative eigenvalues, then we have shown in the proof above:
\begin{corollary}
\label{Hindex}
Let $p$ be a critical point for $\sys$, then $$\ind \tilde H_{\syst}(p)=\ind \tilde H_{\syst}(p_T)=(-1)^d=\ind_{\sys}(p).$$
\end{corollary}
\begin{theorem}[Uniqueness within $\rho T$-ball]
\label{unique}
When $T$ is sufficiently small, there exists a unique critical point $p_T$ within the $\rho T$-ball around $p$.
\end{theorem}
\begin{proof}
Let $p_i,i\in I$ be all the critical points in the interior of the $\rho T$-sphere described as above. Consider the gradient vector field $\nabla\syst$ on the interior of $S_{\rho T}$ that is nonzero on the boundary. Note that $$\ind_{p_i}(\nabla\syst)=\ind \tilde H_{\syst}(p_i).$$ Therefore, when $T$ is sufficiently small, by Theorem \ref{existence2}, we have
\begin{align*}
(-1)^d&=\deg(\nabla\syst|_{S_{\rho T}})=\sum_I\ind_{p_i}(\nabla\syst)\\
&=\sum_I\ind \tilde H_{\syst}(p_i)=\#I\cdot(-1)^d,
\end{align*}
which implies $\#I=1$, i.e., the critical point near $p$ is unique.
\end{proof}
\noindent
The theorem eventually validates the definition we have used above:
\begin{definition}
Suppose $T$ is sufficiently small. Let $p_T$ be the unique critical point near $p$.
\end{definition}
\section{Ordinary Points in the Interior}
\label{ordinary}
\noindent
By Akrout's theorem, ordinary points for the systole function are exactly non-eutactic points, namely, non-semi-eutactic and semi-eutactic points. Similar results are true for $\syst$.
\begin{theorem}[Non-semi-eutactic points]
For any non-semi-eutactic point $q\in\mathcal M$, there exist a neighborhood $V=V(q)$ of $q$ and $T_0=T_0(q)>0$ such that any $q'\in V$ is ordinary for $\syst$ for $T<T_0$.
\end{theorem}
\begin{proof}
Let $I$ be the index set for $S(q)$. Since $q$ does not satisfy the semi-eutactic condition, there exists $\tau\in T_q^{\sys}\mathcal T$ such that $\langle\nabla l_i(q),\tau\rangle>0$ for all $i\in I$, equivalently, $$\nabla l_i(q)\in C(\tau,\frac12\pi-\theta_2)$$ for some $0<\theta_2(q)<\frac12\pi$, where $C(\tau,\theta)$ is the cone in the tangent space $T_q\mathcal T$ centered at $\tau$ with angle $\theta$. Let $V$ be a neighborhood of $q$ such that for any $q'\in V$:\\
(1) $\nabla l_i(q')\in C(\tau,\frac12\pi-\frac12\theta_2)$,\\
(2) $\max\{l_i(q')\}<\min_{\gamma\not\in S(q)}\{l_\gamma(q')\}$.\\
Therefore,
\begin{align*}
    &\sum_\gamma e^{-\frac1Tl_\gamma(q')}||\nabla\syst(q')||=||\sum_\gamma e^{-\frac1Tl_\gamma(q')}\nabla l_\gamma(q')||\\
    &\ge ||\sum_i e^{-\frac1Tl_i(q')}\nabla l_i(q')||-||\sum_{\gamma\not\in S(q)} e^{-\frac1Tl_\gamma(q')}\nabla l_\gamma(q')||\\
    &\ge ||\langle\sum_i e^{-\frac1Tl_i(q')}\nabla l_i(q'),\tau\rangle||-||\sum_{\gamma\not\in S(q)} e^{-\frac1Tl_\gamma(q')}\nabla l_\gamma(q')||\\
    &>\sin(\frac12\theta_2)\sum_i||\nabla l_i(q')||e^{-\frac1T\max\{l_i(q')\}}-C(q)e^{-\frac1T\min_{\gamma\not\in S(q)}\{l_\gamma(q')\}}>0,
\end{align*}
for $T<T_0$ for appropriate $T_0>0$.
\end{proof}
\noindent
To show the same property for semi-eutactic points, a few lemmas are needed and we present them after the theorem.
\begin{theorem}[Semi-eutactic points]
For any semi-eutactic point $q\in\mathcal M$, there exist a neighborhood $V=V(q)$ of $q$ and $T_0=T_0(q)>0$ such that any $q'\in V$ is ordinary for $\syst$ for $T<T_0$.
\end{theorem}
\begin{proof}
% By continuity, let $V_q$ be a neighborhood of $q$ such that for any $q'\in V_q$: and let $C>0$ be a constant such that $Cr_p<\frac12$, such that for $q\in\overline{U}_p$, if we let $t=d(p,q)$:\\
% \\
% (1) there exists $0<\theta_1<\frac12\pi$ such that for any $\tau$, $C(\tau,\frac12\pi-\theta_1)\cap\{\nabla l_i\}\neq\emptyset$, and $l_i(q)>l_j(q)$ for $\nabla l_j(p)\in C(-\tau,\frac12\pi-\frac12\theta_1)$ and $\nabla l_i(q)\in\mathbb H(\tau)$.\\
% (2) $||\nabla l_i(q)-\nabla l_i(p)||<Ct$, and consequently for simplicity $\angle(\nabla l_i(q),\nabla l_i(p))<\frac12\min\{\theta_0,\frac12\pi-\theta_1\}$, and $\frac12<\frac{||\nabla l_i(q)||}{||\nabla l_i(p)||}<\frac32$,\\
% (3) $|l_i(q)-l_i(p)-t\langle\nabla l_i(p),\tau\rangle|<Ct^2$, and for $-\tau\in F_J$, $l_j(q)-l_j(p)<0,$\\
% (4) $||\nabla l_i(q)-\nabla l_i(p)-t\nabla^2l_i(\tau,\cdot)||<Ct^2$, and $t||\proj_{T_p^{\sys\perp}\mathcal T}\nabla^2l_i(\tau,\cdot)||-Ct^2>0$,\\
% (5) $\max\{l_i(q)\}<\min_{\gamma\not\in S(p)}{l_\gamma(q)}$.\\
% \\
% There exists $\tau$ such that $\langle\nabla l_j(q),\tau\rangle>\langle\nabla l_i(q),\tau\rangle=0$ for all $i\in I$ and $j\in J$. Within a neighborhood $V_q$ of $q$, $\langle\nabla l_j(q'),\tau\rangle>\langle\nabla l_i(q'),\tau\rangle$ for $q'\in V_q$ remains true. Specially, we can make $\langle\nabla l_j(q'),\tau\rangle-\langle\nabla l_i(q'),\tau\rangle>C$.
% \begin{align*}
%     \frac{||e^{-\frac1Tl_i(q')}\nabla l_i(q')||}{||e^{-\frac1Tl_j(q')}\nabla l_j(q')||}&=e^{-\frac1T(l_i(q')-l_j(q'))}\frac{||\nabla l_i(q')||}{||\nabla l_j(q')||}\\
%     &=e^{\frac1Tt(\langle\nabla l_j(q'),\tau\rangle-\langle\nabla l_i(q'),\tau\rangle)}\frac{||\nabla l_i(q')||}{||\nabla l_j(q')||}
% \end{align*}
% We make $\theta_0<\frac12\theta_2$ only for this proof if needed.\\
% \\
% If $\theta^\perp(\tau)\le\theta_0$,
% \begin{align*}
%     e^{-\frac1Tl_i(q)}&>e^{-\frac1T\sys(p)}e^{-\frac1T(t\langle\nabla l_i(p),\tau\rangle+Ct^2)}\\
%     &>e^{-\frac1T\sys(p)}e^{-\frac1T(t||\nabla l_i(p)||\sin\theta_0+Ct^2)}
% \end{align*}
% \begin{align*}
%     e^{-\frac1Tl_k(q)}&>e^{-\frac1T\sys(p)}e^{-\frac1T(t\langle\nabla l_k(p),\tau\rangle+Ct^2)}\\
%     &>e^{-\frac1T\sys(p)}e^{-\frac1T(t||\nabla l_k(p)||\sin\frac12\theta_2+Ct^2)}
% \end{align*}
% We consider the projection of $\sum_ie^{-\frac1Tl_i(q)}\nabla l_i(q)$ onto the minor subspace:\\
% For a single $\nabla l_i(q)$, we have
% \begin{align*}
%     ||\proj_{T_p^{\sys\perp}\mathcal T}\nabla l_i(q)||>t||\proj_{T_p^{\sys\perp}\mathcal T}\nabla^2l_i(\tau,\cdot)||-Ct^2,
% \end{align*}
% \begin{align*}
%     ||\proj_{T_p^{\sys\perp}\mathcal T}\nabla l_k(q)||>\frac12||\nabla l_k(p)||\sin\theta_2,
% \end{align*}
% Put all terms together,
% \begin{align*}
%     ||\sum_{I\cup K}e^{-\frac1Tl_i(q)}\nabla l_i(q)||&\ge||\proj_{T_p^{\sys\perp}\mathcal T}(\sum_ie^{-\frac1Tl_i(q)}\nabla l_i(q))||\\
%     &=\sum_{I\cup K}e^{-\frac1Tl_i(q)}||\proj_{T_p^{\sys\perp}\mathcal T}(\nabla l_i(q))||\\
%     &>\sum_Ie^{-\frac1T\sys(p)}e^{-(\rho||\nabla l_i(p)||\sin\theta_0+C\rho^2T)}(t||\proj_{T_p^{\sys\perp}\mathcal T}\nabla^2l_i(\tau,\cdot)||-Ct^2)\\
%     &+\sum_Ke^{-\frac1T\sys(p)}e^{-\frac1T(t||\nabla l_k(p)||\sin\frac12\theta_2+Ct^2)}\frac12||\nabla l_k(p)||\sin\theta_2\\
%     &>C_pe^{-\frac1T\sys(p)}.
% \end{align*}
% If $\theta^\perp(\tau)\ge\theta_0$, we consider the projection of $\sum_ie^{-\frac1Tl_i(q)}\nabla l_i(q)$ onto the major subspace. For any $\nabla l_j(p)\in C(-\tau,\frac12\pi-\theta_1)$,
% \begin{align*}
%     ||e^{-\frac1T(l_j(q)-sys(p))}\nabla l_j(q)||&>\frac12e^{-\frac1T(l_j(q)-l_j(p))}||\nabla l_j(p)||\\
%     &>\frac12e^{\frac1T(t\langle\nabla l_j,\tau\rangle-Ct^2)}||\nabla l_j(p)||\\
%     &>\frac12e^{\frac1T(D'\rho T-C(\rho T)^2)}||\nabla l_j(p)||>0
% \end{align*}
% % If we have any other $\nabla l_i(p)\in C(\tau,\frac12\pi+\frac12\theta_1)$, choose $0<\epsilon<\frac 1r$, where $r=\#S(p)$ as a reminder,
% % \begin{align*}
% %     \frac{||e^{-\frac1Tl_i(q)}\nabla l_i(q)||}{||e^{-\frac1Tl_j(q)}\nabla l_j(q)||}&<\frac{\frac32e^{-\frac1Tl_i(q)}||\nabla l_i(p)||}{\frac12e^{-\frac1Tl_j(q)}||\nabla l_j(p)||}\\
% %     &=3e^{-\frac1T(l_i(q)-l_j(q))}\frac{||\nabla l_i(p)||}{||\nabla l_j(p)||}\\
% %     &<3e^{\frac1T(-t(\langle\nabla l_i(p),\tau\rangle-\langle\nabla l_j(p),\tau\rangle)+2Ct^2)}\frac{||\nabla l_i(p)||}{||\nabla l_j(p)||}\\
% %     &<3e^{\frac1T(-D't+2Ct^2)}\frac{||\nabla l_i(p)||}{||\nabla l_j(p)||}\\
% %     &<3e^{\frac1T(-D'\rho T+2C(\rho T)^2)}\frac{||\nabla l_i(p)||}{||\nabla l_j(p)||}<\epsilon,
% % \end{align*}
% % when $T$ is small enough, where we modify $r_p$ to be the smaller one of $r_p$ and $\frac {D'}{C}$ in the last step to make the inequality hold.\\
% % Note that, if $\nabla l_i(p)\in C(-\tau,\frac12\pi-\frac12\theta_1)$, then $\nabla l_i(q)\in \mathbb H(\tau)$. Therefore,
% % \begin{align*}
% %     ||\sum_ie^{-\frac1Tl_i(q)}\nabla l_i(q)||&>||\sum_{\nabla l_i(p)\in\overline{C(-\tau,\frac12\pi-\frac12\theta_1)}}e^{-\frac1Tl_i(q)}\nabla l_i(q)||\\
% %     &-||\sum_{\nabla l_i(p)\in C(\tau,\frac12\pi+\frac12\theta_1)}e^{-\frac1Tl_i(q)}\nabla l_i(q)||\\
% %     &\ge||\langle\sum_{\nabla l_i(p)\in C(-\tau,\frac12\pi-\theta_1)}e^{-\frac1Tl_i(q)}\nabla l_i(q),\tau\rangle||\\
% %     &-||\sum_{\nabla l_i(p)\in C(\tau,\frac12\pi+\frac12\theta_1)}e^{-\frac1Tl_i(q)}\nabla l_i(q)||\\
% %     &\ge (1-r\epsilon)||e^{-\frac1T(l_j(q)-sys(p))}\nabla l_j(q)||>0
% % \end{align*}
By Lemma \ref{sesplit} below, we split the index set $I$ of $S(q)$ into a maximal eutactic index subset $I_{e}$ and the non-semi-eutactic index complement $I_{nse}$. Let $V(q)$ be the geodesic ball $B(q,t_0)$ for $t_0>0$ to be determined. Recall that $T^{\sys}_q\mathcal T=\spn_{I}\{\nabla l_i\}$ (we use $T^{\sys}$ for short). We let $T^{e}=\spn_{I_{e}}\{\nabla l_i\}$ and $T^{e\perp}=(T^e)^\perp$. Also recall that $\theta_0$ is chosen such that $\angle(\nabla^2l_i(\tau,\cdot),\tau)<\frac{\pi}{2}-2\theta_0$ for all $i$.\\
\\
By continuity, choose $\theta_3,\theta_4$ and $d$ small enough such that\\
(1) $\theta_3<\theta_0$ and $C_2\max\{\theta_3,C_1\theta_4\}<\theta_0$, where $C_1,C_2$ are fixed constants introduced in case 2(b),\\
(2) $\max_{i\in I_e,\angle(\tau,T^{e\perp})\le\theta_3}|\langle\nabla l_i,\tau\rangle|<d$,\\
(3) If $\langle\nabla l_k,\tau\rangle\ge -2d$ then $\angle(\nabla l_k,\tau)\le\frac{\pi}{2}+\theta_4$ for $k\in I_{nse}$,\\
and further conditions presented in the proof.
\\
% (1) $\angle(\nabla l_k(q),\nabla l_k(q'))<\theta_4$.\\
\\
Case 1 : When $\angle(\tau,T^{e\perp})\ge\theta_3$ (i.e., $\tau$ not perpendicular to $T^e$ with angle bound).\\
By the semi-eutactic condition, since $\tau\not\in T^{e\perp}$, we have that $\overline{\mathbb H(\tau)}\cap\{\nabla l_i(q)\}\neq\emptyset$ and $\mathbb H(-\tau)\cap\{\nabla l_i(q)\}\neq\emptyset$. In fact, over the compact region of candidate $\tau$'s in this case, there exists $D>0$ such that
% $$\min_{\nabla l_i(q)\in\overline{\mathbb H(\tau)}}\langle\nabla l_i(q),\tau\rangle>D \text{ and } \max_{\nabla l_i(q)\in\mathbb H(-\tau)}\langle\nabla l_i(q),\tau\rangle<-D.$$
$$\min_i\langle\nabla l_i(q),\tau\rangle<-D.$$
% Let $$J=\{j:\langle\nabla l_j(q),\tau\rangle=\min_i\langle\nabla l_i(q),\tau\rangle\}$$ and $$K=\{k:\langle\nabla l_k(q),\tau\rangle\ge-\frac12D\}.$$ Choose $t_0$ small enough for the radius of $V(q)$ such that\\
% (1) for any $i\in(J\cup K)^C$, $$\langle\nabla l_i(q'),\tau\rangle<0,$$
% (2) for any $j\in J$, $$\langle\nabla l_i(q'),\tau\rangle<-\frac12D,$$
% (3) for any $k\in K$, $$\langle\nabla l_i(q'),\tau\rangle>-\frac12D.$$
% \\
% \\
% For any $j\in J$ and $k\in K$,
% \begin{align*}
%     \frac{||e^{-\frac1Tl_i(q')}\nabla l_i(q')||}{||e^{-\frac1Tl_j(q')}\nabla l_j(q')||}&=e^{-\frac1T(l_i(q')-l_j(q'))}\frac{||\nabla l_i(q')||}{||\nabla l_j(q')||}\\
%     &=e^{\frac1T(t\langle\nabla l_j(q'),\tau\rangle-t\langle\nabla l_i(q'),\tau\rangle+O(t^2))}\frac{||\nabla l_i(q')||}{||\nabla l_j(q')||}\\
%     &<e^{\frac12\frac1TtD}\frac{||\nabla l_i(q')||}{||\nabla l_j(q')||}\\
%     &<e^{\frac12\frac1Tt_0D}\frac{||\nabla l_i(q')||}{||\nabla l_j(q')||}<\epsilon,
% \end{align*}
% \\
% \\
Let $$J=\{j:\langle\nabla l_j(q),\tau\rangle=\min_i\langle\nabla l_i(q),\tau\rangle\}$$ and $$K=\overline{\mathbb H(\tau)}.$$
For any $q'\in V(q)$, $j\in J$ and $k\in K$, we have
\begin{align*}
    \frac{||e^{-\frac1Tl_k(q')}\nabla l_k(q')||}{||e^{-\frac1Tl_j(q')}\nabla l_j(q')||}&=e^{-\frac1T(l_k(q')-l_j(q'))}\frac{||\nabla l_k(q')||}{||\nabla l_j(q')||}\\
    &=e^{\frac1T(t\langle\nabla l_j(q),\tau\rangle-t\langle\nabla l_k(q),\tau\rangle+O(t^2))}\frac{||\nabla l_k(q')||}{||\nabla l_j(q')||}\\
    &<e^{-\frac12\frac1Tt_0D}\frac{||\nabla l_k(q')||}{||\nabla l_j(q')||}<\epsilon,
\end{align*}
for $t_0$ small enough.\\
Pick a $j\in J$, and consider the sum over $K^C$,
\begin{align*}
    ||\sum_{K^C}e^{-\frac1Tl_j(q')}\nabla l_j(q')||&\ge-\langle\sum_{K^C}e^{-\frac1Tl_j(q')}\nabla l_j(q'),\tau\rangle\\
    &\ge-\langle\sum_Je^{-\frac1Tl_j(q')}\nabla l_j(q'),\tau\rangle\\
    &\ge De^{-\frac1Tl_j(q')}
\end{align*}
Put all terms together,
\begin{align*}
    ||\sum_{S(q)}e^{-\frac1Tl_i(q')}\nabla l_i(q')||&\ge||\sum_{K^C}e^{-\frac1Tl_i(q')}\nabla l_i(q')||\\
    &-||\sum_Ke^{-\frac1Tl_i(q')}\nabla l_i(q')||\\
    &\ge D\sum_Je^{-\frac1Tl_j(q')}-r\epsilon e^{-\frac1Tl_j(q')}||\nabla l_j(q')||\\
    &>(D-rM\epsilon)e^{-\frac1Tl_j(q')},
\end{align*}
where $M=\max_{i;q'}||\nabla l_i(q')||$. Note that by setting $t_0$ small, the sum above is $O(e^{-\frac1T(\sys(q)+\epsilon)})$, and the `tail', the sum over $S(q)^C$, is bounded by $e^{-\frac1T(\secsys(q)-\epsilon)}=o(e^{-\frac1T(\sys(q)+\epsilon)})$. Therefore, $\nabla\syst(q')\neq0$.\\
\\
% Case 2: When $\angle(\tau,T^{e\perp})\le\theta_3$ and $\angle(\tau,T^{\sys\perp})\le\theta_4$.\\
Case 2: When $\angle(\tau,T^{e\perp})\le\theta_3$ (i.e., when $\tau$ almost perpendicular to $T^e$).\\
Let $\mathbf C=\cap_{k\in I_{nse}}\mathbb H_{T^{\sys}}(\nabla l_k)=\{\tau:\langle\nabla l_k,\tau\rangle>0\text{\ for\ all\ }k\}$ be the polygonal cone in $T^{\sys}$ consisting of vectors with positive inner products with $\nabla l_k$ for all $k\in I_{nse}$, which is nonempty and convex by non-semi-eutactic condition. Note that $\mathbf C\times T^{\sys\perp}$ is the set of such vectors in the total tangent space. We fix a $v\in \mathbf C$ and make $t$ sufficiently small so that $\langle\nabla l_k(q'),v\rangle$ is positively bounded from below by continuity.\\
\\
- Case 2(a): When furthermore, $\angle(\nabla l_k,\tau)\le\frac{\pi}{2}+\theta_4$ for all $k\in I_{nse}$.\\
Let $J=\{j\in I_{nse}:\frac{\pi}{2}\le\angle(\nabla l_j,\tau)\le\frac{\pi}{2}+\theta_4\}\subset I_{nse}$ and $K:=I_{nse}\setminus J$ the complement.\\
\\
Let $T^{e\perp\sys}$ be the orthogonal complement of $T^e$ in $T^{\sys}$, then $$(\partial \mathbf C\times T^{\sys\perp})\cap T^{e\perp}=(\partial \mathbf C\cap T^{e\perp\sys})\times T^{\sys\perp},$$ which is trivial if and only if $\partial \mathbf C\cap T^{e\perp\sys}=0$ and $T^{\sys\perp}=0$ (and thus $T^{e\perp\sys}=T^{e\perp}$). In this case $J=\emptyset$ if $\theta_3$ and $\theta_4$ are small enough. Otherwise, $\angle(\partial \mathbf C,T^{e\perp\sys})\le\angle(\tau,\partial \mathbf C)+\angle(\tau,T^{e\perp})\le C_1\theta_4+\theta_3$ by Lemma \ref{convex proj}. Contradiction!
\\
\\
If $J=\emptyset$, then $\tau\in \mathbf C\times T^{\sys\perp}$. Note that $\mathbf C\cap T^{e\perp}$ is nonempty. Let $\tau'$ be the projection of $\tau$ onto $T^{e\perp}$, then $\tau'\in (\mathbf C\times T^{\sys\perp})\cap T^{e\perp}$ by convexity of $\mathbf C\times T^{\sys\perp}$ and $\angle(\tau,\tau')<\theta_3<\theta_0$. 
\\
\\
If $J\neq\emptyset$, let $\tau'$ be the projection of $\tau$ onto $(\mathbf C\times T^{\sys\perp})\cap T^{e\perp}$ which is nontrivial. By Lemma \ref{convex proj}, $\angle(\tau,\partial \mathbf C\times T^{\sys\perp})\le\angle(\tau,\partial \mathbf C)<C_1\theta_4$ and since $\angle(\tau,T^{e\perp})\le\theta_3$, by Lemma \ref{intersection angle}, $\angle(\tau,\tau')<C_2\max\{\theta_3,C_1\theta_4\}<\theta_0$.\\
\\
Therefore, for $i\in I_e\cup J$, since
$\nabla l_i(q)\perp\tau'$, $$\langle\nabla l_i(q'),\tau'\rangle=\langle\nabla l_i(q')-\nabla l_i(q),\tau'\rangle=\langle t\nabla^2l_i(\tau,\cdot)+O_i(t^2),\tau'\rangle.$$
Consider the angle, when $t$ is sufficiently small,
\begin{align*}
    \angle(t\nabla^2l_i(\tau,\cdot)+O_i(t^2),\tau')&\le\angle(t\nabla^2l_i(\tau,\cdot)+O_i(t^2),\tau)+\angle(\tau,\tau')\\
    &\le\frac{\pi}{2}-\frac32\theta_0+\theta_0=\frac{\pi}{2}-\frac12\theta_0,
\end{align*}
Thus, $\langle\nabla l_i(q'),\tau'\rangle>0$.\\
\\
For $k\in K$, note that $\langle\nabla l_k(q),\tau'\rangle>0$ since $\tau'\in \intr(\mathbf C\times T^{\sys\perp})$ if $J=\emptyset$ and $\tau'\in\partial \mathbf C\times T^{\sys\perp}$ if $J\neq\emptyset$, then
\begin{align*}
    \langle\nabla l_k(q'),\tau'\rangle &= \langle\nabla l_k(q),\tau'\rangle + \langle t\nabla^2 l_k(q)(\tau,\cdot)+O_k(t^2),\tau'\rangle>0
\end{align*}
% \begin{align*}
%     \angle(\nabla l_k(q'),\tau')&\le\angle(\nabla l_k(q'),\nabla l_k(q))+\angle(\nabla l_k(q),\tau')\\
%     &\le\frac12\theta_4+\frac{\pi}{2}-\theta_4=\frac{\pi}{2}-\frac12\theta_4.
% \end{align*}
where the second term is positive for the same reason as above.\\
Each term has a positive projection onto $\tau'$ and thus the sum is nonzero.
\\
\\
- Case 2(b): When furthermore, $\min_{k\in I_{nse}}\langle\nabla l_k,\tau\rangle\le -2d$.\\
Let $J\subset I_{nse}$ be the index set of the vectors that realize the minimum above.\\
For $i\in I_e$, note that $\langle\nabla l_j,\tau\rangle-\langle\nabla l_i,\tau\rangle\le -2d+d=-d$, then
\begin{align*}
    &\frac{||\proj_v(e^{-\frac1Tl_i(q')}\nabla l_i(q'))||}{||\proj_v(e^{-\frac1Tl_j(q')}\nabla l_j(q'))||}\\
    &=\frac{e^{-\frac1T(t\langle\nabla l_i,\tau\rangle+O_i(t^2))}} {e^{-\frac1T(t\langle\nabla l_j,\tau\rangle+O_j(t^2))} }\frac{||\proj_v(t\nabla^2l_i(q)(\tau,\cdot)+O_i(t^2))||}{||\proj_v(\nabla l_j(q'))||}\\
    &=te^{\frac1T(t\langle\nabla l_j,\tau\rangle-t\langle\nabla l_i,\tau\rangle+O_{ij}(t^2))}\frac{||\proj_v(\nabla^2l_i(q)(\tau,\cdot)+O_i(t))||}{||\proj_v(\nabla l_j(q'))||}\\
    &\le te^{-\frac12\frac1Ttd}\frac{||\proj_v(\nabla^2l_i(q)(\tau,\cdot)+O_{ij}(t))||}{||\proj_v(\nabla l_j(q'))||}
    <\epsilon,
\end{align*}
for $t$ sufficiently small.\\
By the choice of $v$, $\langle\nabla l_k(q'),v\rangle$ is bounded from below by a positive constant for all $k\in I_{nse}$. Therefore,
\begin{align*}
    ||\sum_Ie^{-\frac1Tl_i(q')}\nabla l_i(q')||&\ge||\proj_v\sum_{I_e\cup I_{nse}}e^{-\frac1Tl_i(q')}\nabla l_i(q')||\\
    &\ge||\proj_v\sum_{I_{nse}}e^{-\frac1Tl_j(q')}\nabla l_j(q')||\\
    &-||\proj_v\sum_{I_{e}}e^{-\frac1Tl_i(q')}\nabla l_i(q')||\\
    &\ge||\proj_v\sum_{J}e^{-\frac1Tl_j(q')}\nabla l_j(q')||\\
    &-||\proj_v\sum_{I_{e}}e^{-\frac1Tl_i(q')}\nabla l_i(q')||\\
    &>(1-r\epsilon)||e^{-\frac1Tl_j(q')}\proj_v(\nabla l_j(q'))||>0
\end{align*}
% Case 2 ($\tau\in \overline{U(T^{\sys\perp})}$): When $\angle(\tau,T^{e\perp})\le\theta_3$ and $\angle(\tau,T^e\oplus T^{\sys\perp})\le\theta_4$ (similar as above, this possibly does not happen).\\
% Taking the projection onto $T^{\sys\perp}$, we have
% \begin{align*}
%     \proj_{T^{\sys\perp}}\sum_ie^{-\frac1Tl_i(q')}\nabla l_i(q')=\proj_{T^{\sys\perp}}\sum_ie^{-\frac1Tl_i(q')}(\nabla^2l_i(q)+O(t^2))\neq0,
% \end{align*}
% as $\nabla l_i(q')\in\mathbb H(\proj_{\nabla l_i(q')}\tau)$.\\
% \\
% Case 3 ($\tau\in \overline{U(T^{e\perp})}\setminus U(T^{\sys\perp})$): When $\angle(\tau,T^{e\perp})\le\theta_3$ and $\angle(\tau,T^e\oplus T^{\sys\perp})\ge\theta_4$.\\
% % Let $d:=\max_{i\in I_e}|\langle\nabla l_i,\tau\rangle|$.
% \\
% By the semi-eutactic condition and maximality of $I_e$, $\{\nabla l_i\}_{I_{nse}}$ is properly contained in some half space of $T^{\sys}$ (not necessarily unique), and thus there exists a unit vector $v\in T^{e\perp\sys}$ such that $\langle\nabla l_j,v\rangle>0$ for all $j\in I_{nse}$.\\
% \\
% In this case, there exists $D>0$ (we may assume the same $D$ above by making it smaller if necessary) such that $$\langle\nabla l_j,\tau\rangle\ge D \text{\ or\ } \langle\nabla l_j,\tau\rangle\le -D.$$
% \\
% \\
% \\
% \\
% Case (3a): When $\langle\nabla l_i,\tau\rangle>?$ for all $i$.\\
% Let $\tau'$ be the projection of $\tau$ onto $T^{e\perp}$, then $\langle\nabla l_i,\tau'\rangle>\frac12?$. By definition, $\nabla^2l_i(\tau,\cdot)\in\mathbb H(\tau')$. Therefore, $\sum_Ie^{-\frac1Tl_i(q')}\nabla l_i(q')\in\mathbb H(\tau')$.
% \\
% \\
% \\
% \\
% \\
% \\
% Case (3b): When $\min_{k\in I_{nse}}\langle \nabla l_k,\tau\rangle\le-2d$.\\
% Let $J\subset I_{nse}$ be the index set of the vectors that realize the minimum above.\\
% % $$K_1=\{k:\langle\nabla l_k,\tau\rangle\ge-d\}\subset I_{nse}\setminus J$$
% % and
% % $$K_2=\{k:-d>\langle\nabla l_k,\tau\rangle>\min_{k\in I_{nse}}\langle \nabla l_k,\tau\rangle\}=I_{nse}\setminus(J\cup K_1).$$
% For $i\in I_e$,
% \begin{align*}
%     &\frac{||\proj_v(e^{-\frac1Tl_i(q')}\nabla l_i(q'))||}{||\proj_v(e^{-\frac1Tl_j(q')}\nabla l_j(q'))||}\\
%     &=\frac{e^{-\frac1T(t\langle\nabla l_i,\tau\rangle+\delta_i(t))}} {e^{-\frac1T(t\langle\nabla l_j,\tau\rangle+\delta_j(t))} }\frac{||\proj_v(t\nabla^2l_i(q')(\tau,\cdot)+\epsilon_i(t))||}{||\proj_v(\nabla l_j(q'))||}\\
%     &=te^{\frac1T(t\langle\nabla l_j,\tau\rangle-t\langle\nabla l_i,\tau\rangle+O(t^2))}\frac{||\proj_v(\nabla^2l_i(q')(\tau,\cdot)+O(t))||}{||\proj_v(\nabla l_j(q'))||}\\
%     &\le te^{-\frac12\frac1Ttd}\frac{||\proj_v(\nabla^2l_i(q')(\tau,\cdot)+O(t))||}{||\proj_v(\nabla l_j(q'))||}
%     <\epsilon,
% \end{align*}
% % For $k\in K_1$,
% % \begin{align*}
% %     &\frac{||\proj_{T^{e\perp\sys}}(e^{-\frac1Tl_k(q')}\nabla l_k(q'))||}{||\proj_{T^{e\perp\sys}}(e^{-\frac1Tl_j(q')}\nabla l_j(q'))||}\\
% %     &=e^{\frac1T(t\langle\nabla l_j,\tau\rangle-t\langle\nabla l_k,\tau\rangle+O(t^2))}\frac{||\proj_{T^{e\perp\sys}}(\nabla l_k(q'))||}{||\proj_{T^{e\perp\sys}}(\nabla l_j(q'))||}\\
% %     &\le e^{-\frac12\frac1Ttd}\frac{||\proj_{T^{e\perp\sys}}(\nabla l_k(q'))||}{||\proj_{T^{e\perp\sys}}(\nabla l_j(q'))||}
% %     <\epsilon,
% % \end{align*}
% Note $\langle\nabla l_k(q),v\rangle>0$ for all $k\in I_{nse}$. Therefore,
% \begin{align*}
%     ||\sum_Ie^{-\frac1Tl_i(q')}\nabla l_i(q')||&\ge||\proj_v\sum_{I_e\cup I_{nse}}e^{-\frac1Tl_i(q')}\nabla l_i(q')||\\
%     &\ge||\proj_v\sum_{I_{nse}}e^{-\frac1Tl_j(q')}\nabla l_j(q')||\\
%     &-||\proj_v\sum_{I_{e}}e^{-\frac1Tl_i(q')}\nabla l_i(q')||\\
%     &\ge||\proj_v\sum_{J}e^{-\frac1Tl_j(q')}\nabla l_j(q')||\\
%     &-||\proj_v\sum_{I_{e}}e^{-\frac1Tl_i(q')}\nabla l_i(q')||\\
%     &>(1-r\epsilon)||e^{-\frac1Tl_j(q')}\proj_v(\nabla l_j(q'))||>0
% \end{align*}
% \\
% \\
% \\
% \\
% \\
% Case (3c): When $\angle(\nabla l_j,\tau)<\theta_0$ for $j\in J\subset I_{nse}$ and $\langle\nabla l_k,\tau\rangle>d$ for $k\in K:=I_{nse}\setminus J$.\\
% \\
% Let $C=\cap_{k\in I_{nse}}\mathbb H_{T^{\sys}}(\nabla l_k)=\{\tau:\langle\nabla l_k,\tau\rangle>0\text{\ for\ all\ }i\}$ which is nonempty polygonal cone by non-semi-eutactic condition.\\
% Let $\tau'$ be the projection of $\tau$ onto $\partial C\cap T^{e\perp\sys}$ (more precisely $\cap_J\partial\mathbb H_{T^{\sys}}(\nabla l_j)\cap T^{e\perp\sys}$). Since $\angle(\tau,T^{e\perp\sys})<\theta_?$ and $\angle(\tau,\partial C)<\theta_?$, by lemma () $\angle(\tau,\tau')<2\theta_?$.\\
% $\proj_{\tau'}\nabla l_i=t\nabla^2l_i(\tau,\cdot)+O(t^2)>0$\\
% $\proj_{\tau'}\nabla l_j=t\nabla^2l_j(\tau,\cdot)+O(t^2)>0$\\
% $\proj_{\tau'}\nabla l_k>0$\\
% \\
% \\
% By definition, $\nabla^2l_i(q')\in\mathbb H(\tau)$ for $i\in I_e$, $\nabla^2l_j(q')\in\mathbb H(\tau)$ for $j\in J$. Therefore, $\sum_Ie^{-\frac1Tl_i(q')}\nabla l_i(q')\in\mathbb H(\tau)$.
% Case 3(a): When there is $j$ such that $\langle\nabla l_j,\tau\rangle\ge D$.\\
% Take the projection onto $v\in T^{e\perp\\sys}$:\\
% For $i\in I_{e}$, $\langle\nabla l_i(q'),v\rangle>0$, by definition of $\theta_3$.\\
% For the above $j\in I_{nse}$, $\langle\nabla l_i(q'),v\rangle>0$.\\
% Therefore,
% \begin{align*}
%     ||\sum_Ie^{-\frac1Tl_i(q')}\nabla l_i(q')||&\ge \langle\sum_Ie^{-\frac1Tl_i(q')}\nabla l_i(q'),v\rangle\\
%     &=\langle\sum_{I_{nse}}e^{-\frac1Tl_i(q')}\nabla l_i(q'),v\rangle\\
%     &+\langle\sum_{I_{e}}e^{-\frac1Tl_i(q')}\nabla l_i(q'),v\rangle\\
%     &\ge\langle e^{-\frac1Tl_j(q')}\nabla l_j(q'),v\rangle\\
%     &+\langle\sum_{I_{e}}e^{-\frac1Tl_i(q')}\nabla l_i(q'),v\rangle\\
%     &>0
% \end{align*}
% Case 3(b): When there is $j$ such that $\langle\nabla l_j,\tau\rangle\le -D$.\\
% For $i\in I_{e}$, and $j\in I_{nse}$,
% \begin{align*}
%     &\frac{||\proj_{T^{e\perp\sys}}(e^{-\frac1Tl_i(q')}\nabla l_i(q'))||}{||\proj_{T^{e\perp\sys}}(e^{-\frac1Tl_j(q')}\nabla l_j(q'))||}\\
%     &=\frac{e^{-\frac1T(t\langle\nabla l_i,\tau\rangle+\delta^i(t))}} {e^{-\frac1T(t\langle\nabla l_j,\tau\rangle+\delta^j(t))} }\frac{||\proj_{T^{e\perp\sys}}(t\nabla^2l_i(q')(\tau,\cdot)+\epsilon^i(t))||}{||\proj_{T^{e\perp\sys}}(\nabla l_j(q'))||}\\
%     &=te^{\frac1T(t\langle\nabla l_j,\tau\rangle-t\langle\nabla l_i,\tau\rangle+O(t^2))}\frac{||\proj_{T^{e\perp\sys}}(\nabla^2l_i(q')(\tau,\cdot)+O(t))||}{||\proj_{T^{e\perp\sys}}(\nabla l_j(q'))||}<\epsilon,
% \end{align*}
% as $\langle\nabla l_j,\tau\rangle<\langle\nabla l_i,\tau\rangle$ by definition of $\theta_4$.\\
% Put all terms together,
% \begin{align*}
%     ||\sum_Ie^{-\frac1Tl_i(q')}\nabla l_i(q')||&\ge||\proj_{T^{e\perp\sys}}\sum_Ie^{-\frac1Tl_i(q')}\nabla l_i(q')||\\
%     &\ge||\proj_{T^{e\perp\sys}}\sum_{I_{nse}}e^{-\frac1Tl_j(q')}\nabla l_j(q')||\\
%     &-||\proj_{T^{e\perp\sys}}\sum_{I_{e}}e^{-\frac1Tl_i(q')}\nabla l_i(q')||\\
%     &>(1-r\epsilon)||e^{-\frac1Tl_j(q')}\proj_{T^{e\perp\sys}}(\nabla l_j(q'))||>0
% \end{align*}
Each case corresponds to a compact region of the unit sphere, and the process in each case can be done continuously, hence theorem follows.
% \begin{align*}
%     \proj_{T^{e\perp\sys}}\sum_ie^{-\frac1Tl_i(q')}\nabla l_i(q')=\proj_{T^{e\perp\sys}}\sum_ie^{-\frac1Tl_i(q')}\nabla l_i(q')
% \end{align*}
% \begin{align*}
%     \proj_{T^{e\perp\sys}}e^{-\frac1Tl_i(q')}\nabla l_i(q')=te^{-\frac1T(t\langle\nabla l_i,\tau\rangle+\delta^i(t)}\proj_{T^{e\perp\sys}}(\nabla^2l_i(\tau,\cdot))
% \end{align*}
\end{proof}

\begin{lemma}
\label{sesplit}
Let $\{v_i\}_{i\in I}$ be a semi-eutactic set, then the index set $I$ can be uniquely split into a maximal eutactic index subset $I_e$ and the non-semi-eutactic index complement $I_{nse}$.
\end{lemma}
% \begin{lemma}
% Let $\{v_i\}\subset\mathbb R^n$ be a finite non-semi-eutactic set and $V=\spn\{v_i\}$ the spanned vector space. Then there exists $\theta>0$ such that $$\max_J\max_{k\in J^c}\angle(v_k,\partial(\cap_J\mathbb H(v_j)))\le\frac{\pi}{2}-\theta.$$
% \end{lemma}
\begin{proof}
Consider the convex hull of $\{v_i\}$, then the origin is on the boundary by definition. Let $P$ be the maximal face of the convex hull that the origin lives on. Let $I_{e}$ be the index set of the maximal subset whose convex hull is $P$, then any hyperplane perpendicular to $\spn\{v_i\}$ passing through $\{v_i\}_{I_e}$ separates all other vectors on one side.
\end{proof}
\begin{lemma}
Given a finite non-semi-eutactic set of vectors $\{v_k\}\subset\mathbb R^n$, then for any $\theta$ small, there exists $d>0$ such that for any unit vector $\tau$ at least one of the following is true:\\
(1) $\angle(v_k,\tau)\le\frac{\pi}{2}+\theta$ for all $k$,\\
(2) $\min_k\langle v_k,\tau\rangle\le-2d.$
\end{lemma}
\begin{proof}
    This is by continuity.
\end{proof}
\begin{lemma}
\label{intersection angle}
Let $V_1,V_2\subset\mathbb R^n$ be two linear subspaces with nontrivial intersection $V_0:=V_1\cap V_2$. Then there exists a constant $C$ depending on $\angle(V_1,V_2)$ such that for $\theta$ small enough, for any $v\in\mathbb R^n$ with $\angle(v,V_i)<\theta$ for $i=1,2$, we have $\angle(v,V_0)<C\theta$, i.e., $$C(V_1,\theta)\cap C(V_2,\theta)\subset C(V_0,C\theta).$$
\end{lemma}
\begin{proof}
It is obvious if one is a subspace of the other. Assume $\angle(V_1,V_2)>0$. Let $v_i$ be the projection of $v$ onto $V_i$, $i=0,1,2$ and $v'_i=v_i-v_0$, $i=1,2$. Now by the setup $\angle(V_1,V_2)=\min\{\angle(v'_1,v'_2),\pi-\angle(v'_1,v'_2)\}>0$ and $\angle(v,V_i)=\angle(v,v_i)$ for $i=0,1,2$, thus it suffices to prove the lemma in $\mathbb R^3$. Let $C_i$ be the great circle on $S^2$ spanned by $v_i$ and $v_0$, $i=1,2$, then $C_1\cap C_2=\pm v_0$, and $\min\{\angle(v'_1,v'_2),\pi-\angle(v'_1,v'_2)\}=\angle(C_1,C_2)$. Therefore, there exists $C>0$ depending on $\angle(C_1,C_2)$ such that $\angle(v,v_0)=d_{S^2}(v,v_0)< C\max_{i=1,2}{d_{S^2}(v,C_i)}=C\max_{i=1,2}{\angle(v,v_i)}<C\theta$ if $v$ is close enough to $C_1$ and $C_2$.
\end{proof}
\begin{lemma}
\label{convex proj}
Let $\{v_i\}_{i\in I}\subset\mathbb R^n$ be a finite non-semi-eutactic set of vectors, and $\mathbf C=\cap_i\mathbb H(v_i)$, then there exists $\theta$ small enough, such that for any vector $\tau\not\in \mathbf C$ with $\angle(v_i,\tau)\le\frac{\pi}{2}+\theta$ for all $i$, there exists a unique unit vector $u$ such that $\angle(u,\tau)=\min_{v\in\partial\mathbf C}\angle(v,\tau)$.
% Then it is valid to define the projection onto $\partial \mathbf C$ on vectors in $\{\tau\not\in \mathbf C \text{\ and\ } \angle(v_i,\tau)\le\frac{\pi}{2}+\theta \text{\ for\ all\ } i\}$ for $\theta$ small, i.e., vectors close enough to $\mathbf C$ from outside.
Furthermore, $\angle(u,\tau)<C\theta$ for constant $C$.
\end{lemma}
% \angle(\tau,\partial \mathbf C):=\min_{V=\text{\ a\ face\ of\ }\partial \mathbf C}\angle(\tau,V)<\theta
\begin{proof}
% Let $V\subset\mathbb R^n$ be a face of $\mathbf C$ such that $v\in V$ if and only if $\langle v_j,v\rangle=0$ for $j\in J$ and $\langle v_k,v\rangle>0$ for $k\in K:=J^c$. Let $v_1,v_2\in V$, for $0\le\lambda\le1$, by linearity, $\langle v_{j(k)},\lambda v_1+(1-\lambda)v_2\rangle=(>)0$, and thus $V$ is radial and convex. Consider $\angle:V\cap S^{n-1}\to\mathbb R^+$ measuring the angle between a vector and $\tau$. If $\tau\not\in V\cap S^{n-1}$, a positive minimum will be attained.
It is clear that $v\in \mathbf C$ if and only if $\langle v_i,v\rangle\ge0$ for all $i$, then for $v_1,v_2\in \mathbf C$, for $0\le\lambda\le1$, $\langle v_i,\lambda v_1+(1-\lambda)v_2\rangle\ge0$, which shows that $\mathbf C$ is radial and convex. Note that $\mathbf C$ is contained in some half space. Now consider the angle between a vector and $\tau$ as a function on $\mathbf C\cap S^{n-1}$. If $\tau\not\in \mathbf C\cap S^{n-1}$, $\angle(\cdot,\tau)=d_{S^{n-1}}(\cdot,\tau)$ is minimized over $\mathbf C\cap S^{n-1}$ by a unique vector $v\in\partial \mathbf C\cap S^{n-1}$ by convexity if $\tau$ is closed enough to $\partial \mathbf C$.\\
\\
The second part is due to Lemma \ref{intersection angle} and finiteness of faces.
\end{proof}
\begin{definition}
\label{convex proj def}
With the same settings in the lemma above, define $$\angle(\tau,\partial\mathbf C)=\angle(u,\tau) \text{ and\ } \proj_{\mathbf C}\tau=\proj_{u}\tau.$$
\end{definition}
% \begin{definition}
% Let $\mathbf C=\cap\mathbb H_i\neq\emptyset$ be the intersection of finitely many half spaces, if $\tau\not\in \mathbf C$ and $\angle(v_i,\tau)\le\frac{\pi}{2}+\theta$ for $\theta$ small, by lemma \ref{convex proj}, there exists $v_0$ that uniquely minimizes $\angle(v,\tau)$ over $\partial \mathbf C$. Define $\angle(\tau,\partial \mathbf C)=\angle(v_0,\tau)$ and $\proj_{\partial \mathbf C}\tau=\proj_{v_0}\tau$.
% \end{definition}
\begin{remark}
We can reduce the condition on $\mathbf C$ to any radial and convex subset for Lemma \ref{convex proj} and Definition \ref{convex proj def}.
\end{remark}
\begin{lemma}
Let $\mathbf C=\cap\mathbb H_i\neq\emptyset$ be the intersection of finitely many half spaces. Suppose $V$ is a linear subspace that has nontrivial intersection with $\mathbf C$, then for any $v\in\mathbb R^n$ and for $\theta$ small, if $\angle(v,\partial \mathbf C)<\theta$ and $\angle(v,V)<\theta$, then $\angle(v,\mathbf C\cap V)\le\angle(v,\partial \mathbf C\cap V)<C\theta$.
\end{lemma}
\begin{proof}
Let $\tau'$ be the projection of $\tau$ onto $\partial \mathbf C$ and $f$ be a face of $\partial \mathbf C$ such that $\tau'\in f$, then we claim $f\cap V\neq\emptyset$ if $\theta$ is chosen small enough. Otherwise, $\angle(f,V)\le\angle(v,f)+\angle(v,V)<2\theta$ which is a contradiction if $2\theta<\min_{f\in F}\angle(f,V)$, where $F=\{f, f \text{ is a face of }\partial \mathbf C \text{ and }f\cap V=\emptyset\}$. Note that $f\cap V$ is radial and convex, $\angle(\tau,\partial \mathbf C\cap V)\le\angle(\tau,f\cap V)\le\angle(\tau,\proj_{f\cap V}\tau')<C\theta$ by Lemma \ref{intersection angle}.
\end{proof}
% \begin{proof}
%     add: $\angle(v,\mathbf C\cap V)\le\angle(v,\partial \mathbf C\cap V)<C\theta$.
% \end{proof}

\section{Approaching the Boundary}
\label{boundary}
\begin{definition}
The $\epsilon$-thin part and $\epsilon$-thick part of the moduli space are defined as
    $$\mathcal M^{\le\epsilon}=\{X\in\mathcal M:\sys(X)\le\epsilon\},$$
    $$\mathcal M^{\ge\epsilon}=\{X\in\mathcal M:\sys(X)\ge\epsilon\}.$$
\end{definition}
\begin{theorem}[Points near the boundary]
There exist $\epsilon>0$ and $T_0>0$ such that any $q\in\mathcal M^{\le\epsilon}$ is an ordinary point for $\syst$ for all $T<T_0$. In fact, for any $\beta$ of length $\le\epsilon$, $\langle\nabla l_\beta,\nabla\syst\rangle>0$.
\end{theorem}
\begin{proof}
Let $q\in\mathcal M^{\le\epsilon}$ and $S(q)=\{\gamma_1,\cdots,\gamma_r\}$. By collar lemma or Corollary 4.1.2 in \cite{buser2010geometry}, $\gamma_i$'s are disjoint. Let $\gamma_I=\cup\gamma_i$. We have the Weil-Petersson pairing of $\nabla l_i$ with $\nabla l_\gamma$ for $\gamma$ classified by the following three types:\\
\\
Type 1: $\gamma=\gamma_j$.
$$\langle\nabla l_\gamma,\nabla l_i\rangle>||\nabla l_i||\delta_{ij}.$$
\\
Type 2: $\gamma\cap\gamma_I=\emptyset$.
$$\langle\nabla l_\gamma,\nabla l_i\rangle>0.$$
\\
Type 3: $\gamma\not\subset\gamma_I$, $\gamma\cap\gamma_I\neq\emptyset$.\\
By collar lemma, $l_\gamma>x(\epsilon)$, where $x(\epsilon)=2\arcsinh(\frac{1}{\sinh\frac{\epsilon}{2}})>-2\log\epsilon$.\\
\\
Therefore, choose any $i$,
\begin{align*}
    ||\sum_\gamma e^{-\frac1Tl_\gamma}\nabla l_\gamma||&\ge||\sum_{\gamma\cap\gamma_I=\emptyset\text{\ or\ }\gamma\subset\gamma_I} e^{-\frac1Tl_\gamma}\nabla l_\gamma||-||\sum_{\gamma\cap\gamma_I\neq\emptyset} e^{-\frac1Tl_\gamma}\nabla l_\gamma||\\
    &\ge e^{-\frac1Tl_i}||\nabla l_i||-Ce^{-\frac1T x(\epsilon)}\\
    &\ge \frac12\sqrt{\epsilon}e^{-\frac1T\epsilon}-Ce^{\frac2T\log\epsilon}
    % &\ge\sum_{\gamma\cap\gamma_I=\emptyset} e^{-\frac1Tl_\gamma}\langle\nabla l_\gamma,\nabla l_i\rangle -Ce^{\frac2T\log\epsilon}\\
    % &\ge\frac{4}{3\pi}\sum_{\gamma\cap\gamma_I=\emptyset} e^{-\frac1Tl_\gamma}\cosh^{-2}d(\gamma,\gamma_i) -Ce^{\frac2T\log\epsilon}
\end{align*}
In order for the lower bound above to be positive, it suffices to make $T$ satisfy
\begin{align*}
    T<\frac{-2\log\epsilon-\epsilon}{\log(2C)-\frac12\log\epsilon}
\end{align*}
Note that
\begin{align*}
    \frac{-2\log\epsilon-\epsilon}{\log(2C)-\frac12\log\epsilon}\to 4,
\end{align*}
as $\epsilon\to0^+$. Therefore, when $\epsilon<\epsilon_0$ small enough, a uniform upper bound $T_0$ for $T$ can be chosen such that for any $q\in\mathcal M^{\le\epsilon_0}$, $\syst(q)\neq0$ for all $T<T_0$.
% \\
% \\
% \\
% \\
% \\
% \begin{align*}
%     \langle\nabla l_i,\nabla l_j\rangle&=\frac{2}{\pi}\left(l_i\delta_{ij}+\sum_{\langle A\rangle\backslash\Gamma/\langle B\rangle} \left(u\log\frac{u+1}{u-1}-2\right)\right)\\
%     &>\frac{4}{3\pi}\cosh^{-2}(d(\gamma_i,\gamma_j))
% \end{align*}
% \\
% \\
% Let $\gamma_\alpha$ be a shortest closed geodesic on subsurface $q_i$\\
% $\langle\nabla l_\gamma,\nabla l_i\rangle$
% \\
% \\
% Let $\epsilon=2\arcsinh1$. For any $q$ that is $\epsilon$-close to $\partial\mathcal M$, $\sys(q)<\epsilon$. The shortest geodesics $\gamma_1,\cdots,\gamma_r$ are disjoint in this case by collar lemma, and $r\le 3g-3$. It is easy to see that $q$ is not semi-eutactic by expanding $\{\gamma_i\}$ to a pants decomposition.\\
% \\
% For any $\gamma_i,\gamma_j\in\{\gamma|l_\gamma<2\epsilon\}\supset S(q)$, $$\sinh(\frac{l_i}{2})\sinh(\frac{l_j}{2})<\sinh^2(\frac{\epsilon}{2})=1,$$
% hence, all curves are disjoint in $\{\gamma|l_\gamma<2\epsilon\}$.
% \\
% By (\ref{short}), there exists $C>0$ such that $\left|||\nabla l_i||-l_i^{\frac12}\right|<Cl_i^2$ and $0<\langle\nabla l_i,\nabla l_j\rangle<Cl_i^4$ for $i\neq j$. Then $\sum_ie^{-\frac1Tl_i}\nabla l_i\neq0$ for all $T$.
% \begin{align*}
%     ||\sum_\gamma e^{-\frac1Tl_\gamma}\nabla l_\gamma||&>||\sum_{l_\gamma<\epsilon}e^{-\frac1Tl_\gamma}\nabla l_\gamma||-||\sum_{l_\gamma\ge\epsilon}e^{-\frac1Tl_\gamma}\nabla l_\gamma||\\
%     &>e^{-\frac1T\sys(q)}||\nabla l_i||-?e^{-\frac1T\epsilon}\\
%     &>e^{-\frac1T}
% \end{align*}
\end{proof}
\begin{remark}
By critical points attracting property, all critical points of $\syst$ live in the thick part of the moduli space.
\end{remark}
\begin{remark}
\label{boundarydirection}
    The positive pairing in the theorem implies that when $T<T_0$, $\nabla\syst$ points transversely inward to the $\sys$-level sets $\mathcal M^{=\epsilon'}$ in $\mathcal M^{\le\epsilon}$. Therefore,
\end{remark}
\begin{corollary}
For $\epsilon$ small enough, $\mathcal M^{\ge\epsilon}$ is a deformation retract of $\mathcal M$.
\end{corollary}
% \\
% \\
% \\
% \\
% \begin{lemma}
% For any $\epsilon<2\arcsinh1$, $Crit(\sys)\subset\mathcal M^{\ge\epsilon}=\emptyset$.
% \end{lemma}
% \begin{proof}
% For any two different $\gamma_1,\gamma_2\in S(X)$ with $X\in\mathcal M^{\le\epsilon}$, by definition $l_1(X),l_2(X)\le\epsilon<2\arcsinh1$ and so $\sinh(\frac{l_1}{2})\sinh(\frac{l_2}{2})<\sinh^2(\frac{2\arcsinh1}{2})<1$. By collar lemma or Corollary 4.1.2 in [???], $\gamma_1$ and $\gamma_2$ are disjoint. Expand $S(X)$ into a pants decomposition, it is easy to see that $X$ is not semi-eutactic.
% \end{proof}
% \begin{corollary}
% There exists $\epsilon\le2\arcsinh 1$ such that $$Crit(\syst)\subset\mathcal M^{\ge\epsilon}.$$
% \end{corollary}
% \begin{proof}
% Let $X\in\mathcal M^{\le\epsilon}$, i.e., $\sys(X)\ge2\arcsinh 1$, then the shortest geodesics are disjoint by collar lemma. Expand $S(X)$ to a maximal set of disjoint geodesics, then by Fenchel-Nielsen coordinates, $X$ is not eutactic.
% \end{proof}
% \begin{corollary}
% If there is a decreasing series $T_n\to 0$ and critical points $p_{T_n}\to p$, then $p$ is a critical point of $\sys$.
% \end{corollary}
% \begin{corollary}
% There exists $\epsilon>0$, such that when $T<\epsilon$, $$d(Crit(\syst),Crit(\sys))<\rho T.$$
% \end{corollary}
% \begin{theorem}
% Let $p_T=u(t)$ where $u'(0)=\tau$ and write $\tau=\tau_1+\tau_2$, where $\tau_1\in T_p^{\sys}\mathcal T$ and $\tau_2\in T_p^{\sys\perp}\mathcal T$. then $||\tau_2||<C||\tau_1||$ for all $T$ sufficiently small.
% \end{theorem}
% \noindent\\
% \\
% \\
% \\
% \\
% \begin{theorem}
% (Existence) Suppose $p$ is a critical point for $\sys$, then for any neighborhood $U$ of $p$, there exists $\epsilon>0$ such that $\syst$ has a critical point in $U$ for $T\le\epsilon$.
% \end{theorem}
% \begin{theorem}
% The Hessian $H(\syst)$ is nondegenerate at critical points of $\syst$.
% \end{theorem}
% \begin{corollary}
% $\syst$ is an approximation of $\sys$ consisting of smooth Morse functions such that the critical points of $\syst$ .
% \end{corollary}
\section{Extension onto the Boundary}
\label{extension}
\noindent
Recall that the Deligne-Mumford boundary $\partial_{\textit{DM}}\mathcal M$ of the moduli space $\mathcal M(X)$ is the space of all nodal hyperbolic surfaces modeled on $X$. We may use $\partial\mathcal M$ for simplicity.
A stratum is a maximal connected subset in which points are homeomorphic to each other. The Deligne-Mumford compactification is the union $\overline{\mathcal M}=\mathcal M\cup\partial \mathcal M$.\\
\\
Let $X\in\partial\mathcal M$ with the pinched curve set $S$ and call a neighborhood \textit{stratifically\ closed} if the pinched curve set of any point is a subset of $S$. Expand $S$ to a pants decomposition $\bar{S}=\{\alpha_i\}$, then the Fenchel-Nielsen coordinates are given by the associated length and twist parameters $\{l_i,\tau_i\}$. For a small neighborhood $U$ of $X$ that is stratifically closed, let $(U,\{k_ie^{i\tau_i}\})$ be a chart, with $k_ie^{i\tau_i}=l_i^\chi e^{i\tau_i}$ being the transition maps on the overlap, where $\chi$ attains $\frac12$ if $\alpha_i\in S$ and 1 otherwise. We call such chart a \textit{nodal chart} at $X$. Different extensions are $C^\infty$-compatible by analytical compatibility of Fenchel-Nielsen coordinates given by different pants decompositions.\\
\\
Let $S=\{\beta_1,\cdots,\beta_s\}$ be a set of disjoint simple closed geodesics (not necessarily shortest) on $X$. As all $l_{\beta_i}\to 0$, by definition $X\to X_0\in\partial\mathcal M$ ending in the respective stratum. The limit nodal surface $X_0$ has a different topology and may not be connected away from the nodes.
% Let $\{X_i\}$ be all connected components of the result surface when curves in $S$ are pinched.
The tangent space of the compactification at $X_0\in\partial\mathcal M$ can be decomposed as $T(X_0)=T^{\textit{Str}}(X_0)\oplus T^{\textit{Nod}}(X_0)$, where $T^{\textit{Str}}(X_0)$ is given by the length and twist parameters for $\bar{S}\setminus S$ and $T^{\textit{Nod}}(X_0)$ is given by those for $S$.\\
% Note that $T^\textit{Str}(X)=\oplus_i T(X_i)$, where each summand is the tangent space of respective moduli space at $X_i$. We use $T(X_i)$ for both the tangent space of $\mathcal M(X_i)$ and the tangent subspace of $\overline{\mathcal M}(X)$ by abuse of notation.\\
\\
Under a nodal chart at some $X\in\partial\mathcal M$ with pinched curve set $S$ as above, the geodesic-length functions $l_{\beta_i}=k_{\beta_i}^2$ are $C^\infty$.\\
\\
To extend $\syst$ to $\partial\mathcal M$, note that for fixed $T>0$, $$e^{-\frac1Tl_{\beta_i}}\to 1 \text{ as } l_{\beta_i}\to 0.$$
For any $\gamma$ that crosses some $\beta_i$, since $\sinh\frac{l_{\beta_i}}{2}\sinh\frac{l_\gamma}{2}>2\arcsinh1$, $$l_\gamma\to\infty \text{ and } e^{-\frac1Tl_\gamma}\to 0 \text{ as } l_{\beta_i}\to 0.$$
\begin{definition}
    For $X\in\partial\mathcal M$, let 
    \begin{align*}
        \syst(X)=&-T\log\left(s+\sum_{\gamma \text{ s.c.g. on } X} e^{-\frac1Tl_\gamma(X)}\right)\\
        =&-T\log\left(s+\sum_i e^{-\frac1T\syst(X_i)}\right),
    \end{align*}
    where $s=\# S$.
\end{definition}
\noindent
It is known from previous discussion that $\syst$ is at least $C^2$ on each stratum. Note that an extension of the Weil-Petersson metric from $\mathcal M$ to the compactification $\overline{\mathcal M}$, that equals the Weil-Petersson metric when restricted on each stratum, does not exist. But given continuity of $\syst|_{\overline{\mathcal M}}$, following the stratum-wise gradient flow, we know that it attains its global minimum on maximally pinched surfaces, and therefore is positively bounded on $\partial M$. We write this as a lemma:
\begin{lemma}
\label{positivityofsyst}
    The function $\syst$ is bounded on $\overline{\mathcal M}$ by positive numbers.
\end{lemma}
% \begin{lemma}
%     With the notations above, if $\gamma$ is disjoint from $\cup\gamma_i$, then $\nabla l_\gamma$ converges into $T^{\textit{Str}}(X)$.
% \end{lemma}
% \begin{proof}
% Note that by Theorem \ref{norm}, $\langle\nabla l_\gamma,\nabla l_{\beta_i}\rangle\to0$, and by Theorem \ref{lengthtwist}, $\langle\nabla l_\gamma,\frac{\partial}{\partial\tau_{\beta_i}}\rangle=0$. Riera's formula guarantees continuity and Wolpert's upper bound by length implies convergence. Lemma follows from $J\frac{\partial}{\partial\tau_{\beta_i}}=\nabla l_{\beta_i}$.
% \end{proof}
\begin{lemma}
    For $T$ sufficiently small, $\syst:\overline{\mathcal M}\to\mathbb R$ is $C^2$, under the differential structure given by nodal charts.
\end{lemma}
\begin{proof}
    It is already shown that $\syst$ is $C^2$ on the interior and continuous globally by its definition. Let $X\to X_0\in\partial\mathcal M$ with pinched curve set $S=\{\beta_1,\cdots,\beta_s\}$. Recall that the first derivative is given by $$\syst'=\frac{\sum_\gamma e^{-\frac1Tl_\gamma}l_\gamma'}{\sum_\gamma e^{-\frac1Tl_\gamma}},$$ and the second derivative
\begin{align*}
\syst''= \frac1T\left(\frac{\sum_\gamma e^{-{\frac1T}l_\gamma}l_\gamma'}{\sum_\gamma e^{-{\frac1T}l_\gamma}}\right)^2-\frac1T\frac{\sum_\gamma e^{-{\frac1T}l_\gamma}(l_\gamma')^2}{\sum_\gamma e^{-{\frac1T}l_\gamma}} +\frac{\sum_\gamma e^{-{\frac1T}l_\gamma}l_\gamma''}{\sum_\gamma e^{-{\frac1T}l_\gamma}}.
\end{align*}
Thus it suffices to show that $e^{-\frac1T}l'$ and $e^{-\frac1T}l''$ converge when the base point approaches the boundary. We show that based on the following three types of geodesics:\\
\\
Type 1: $\gamma=\beta_i$.\\
It is clear by definition that $l=k_{\beta_i}^2$ is $C^\infty$.\\
\\
Type 2: $\gamma\cap\beta_i=\emptyset$.\\
We can express $l_\gamma=\text{tr}(\Pi A_i)$ by a tracking algorithm by unfolding the surface, where each $A_i$ is a rotation or translation matrix smooth in Fenchel-Nielsen coordinates and not involving $\tau_{\beta_i}$'s.\\
\\
Type 3: $\gamma\not\in\cup\beta_i, \gamma\cap(\cup\beta_i)\neq\emptyset$.\\
We have $l_\gamma\ge-2\log l_{\beta_i}$ for $\beta_i\cap\gamma\neq\emptyset$. Let $\gamma_i$'s be a few geodesics disjoint from $\cup \beta_i$, such that $\{\lambda_i=\nabla l_{\beta_i}^{\frac12},J\lambda_i,\nabla l_i\}$ is a local frame.
\\
For the first directional derivatives, note that
\begin{align*}
    e^{-\frac1Tl_\gamma}|\langle\nabla l_\gamma,\lambda_i\rangle|,e^{-\frac1Tl_\gamma}|\langle\nabla l_\gamma,J\lambda_i\rangle|&\le e^{-\frac1Tl}||\nabla l_\gamma||\cdot||\lambda_i||\\
    \le e^{-\frac1Tl_\gamma}(c(l_\gamma+l_\gamma^2e^{\frac{l_\gamma}{2}}))^{\frac12}\cdot c\to0,
\end{align*}
and
\begin{align*}
    e^{-\frac1Tl_\gamma}|\langle\nabla l_\gamma,\nabla l_i\rangle|&\le e^{-\frac1Tl}||\nabla l_\gamma||\cdot||\nabla l_i||\\
    &\le e^{-\frac1Tl_\gamma}(c(l_\gamma+l_\gamma^2e^{\frac{l_\gamma}{2}}))^{\frac12}\cdot (c(l_i+l_i^2e^{\frac{l_i}{2}}))^{\frac12}\to0.
\end{align*}
For the second directional derivatives, note that
\begin{align*}
    ||e^{-\frac1Tl_\gamma}XYl_\gamma||&= ||e^{-\frac1Tl_\gamma}(H(l_\gamma)(X,Y)+\langle D_XY,\nabla l_\gamma\rangle)||\\
    &\le e^{-\frac1Tl_\gamma}(||H(l_\gamma)(X,Y)||+||D_XY||\cdot||\nabla l_\gamma||)\\
    &\le e^{-\frac1Tl_\gamma}((4c(1+l_\gamma e^{\frac{l}{2}})+O(l_\gamma^3))||X||\cdot||Y||+\frac{c}{l_{\beta_i}^\frac12}\cdot||\nabla l_\gamma||)\to0,
\end{align*}
where $X,Y$ are two vector fields near $X_0$.\\
% in the local frame $\{\lambda_i,J\lambda_i,\nabla l_i\}$.\\
Therefore, $\sum_{\gamma\in \textit{Dt}(\gamma)}e^{\frac1Tl_\gamma}$ is $C^2$, where $\textit{Dt}(\gamma)$ is the set of $\gamma'$'s that are Dehn twist equivalent to $\gamma$ along $\cup\beta_i$.
% Note as $l_{\beta_i}\to 0$, $\nabla l_{\beta_i}\to0=\nabla (k_i^2)(X_0)$, and thus $e^{-\frac1Tl_{\beta_i}}\nabla l_{\beta_i}\to 0$, and for $\gamma$ crossing $\cup\beta_i$, $e^{-\frac1Tl_\gamma}\nabla l_\gamma\to 0$. Hence, the first derivative converges locally uniformly:
%     \begin{align*}
%     \nabla\syst(X)&=\frac{\sum_\gamma e^{-\frac1Tl_\gamma(X)}\nabla l_\gamma(X)}{\sum_\gamma e^{-\frac1Tl_\gamma(X)}}\\
%     &\to\frac{\sum_i e^{-\frac1T\syst(X_i)}\nabla \syst(X_i)}{s+\sum_i e^{-\frac1T\syst(X_i)}}
% \end{align*}
% Recall the second derivative
% \begin{align*}
% \tilde H_{\syst}= \frac1T\left(\frac{\sum_\gamma e^{-{\frac1T}l_\gamma}\nabla l_\gamma}{\sum_\gamma e^{-{\frac1T}l_\gamma}}\right)^2-\frac1T\frac{\sum_\gamma e^{-{\frac1T}l_\gamma}(\nabla l_\gamma)^2}{\sum_\gamma e^{-{\frac1T}l_\gamma}} +\frac{\sum_\gamma e^{-{\frac1T}l_\gamma}\nabla^2l_\gamma}{\sum_\gamma e^{-{\frac1T}l_\gamma}}.
% \end{align*}
% Note that as $l_{\beta_i}\to 0$, $$e^{-\frac1Tl_{\beta_i}}(\nabla l_{\beta_i})^2\to 0,$$
% and for $\gamma$ crossing $\cup\beta_i$, $$e^{-\frac1Tl_\gamma}(\nabla l_\gamma)^2\to 0 \text{ and } e^{-\frac1Tl_\gamma}\nabla^2 l_\gamma\to 0.$$
% Therefore,
% \begin{align*}
% \tilde H_{\syst}(X)&\to \frac1T\left(\frac{\sum_\gamma e^{-{\frac1T}l_\gamma}\nabla l_\gamma}{s+\sum_\gamma e^{-{\frac1T}l_\gamma}}\right)^2-\frac1T\frac{\sum_\gamma e^{-{\frac1T}l_\gamma}(\nabla l_\gamma)^2}{s+\sum_\gamma e^{-{\frac1T}l_\gamma}}\\
% & +\frac{\sum_i\nabla^2(k_{\beta_i}^2)+\sum_\gamma e^{-{\frac1T}l_\gamma}\nabla^2l_\gamma}{s+\sum_\gamma e^{-{\frac1T}l_\gamma}},
% \end{align*}
% where $\sum_\gamma$ above is the sum over all $\gamma$ disjoint from $\cup\gamma_i$.
\end{proof}
% \noindent
% Type 1: $\gamma=\beta_i$\\
% $l=k_{\beta_i}^2$ is $C^\infty$\\
% \\
% Type 2: $\gamma\cap\beta_i=\emptyset$\\
% $l=tr \Pi A_i$, where each $A_i$ is a rotation or translation matrix smooth in Fenchel-Nielsen coordinates.
% \\
% \\
% Type 3: $\gamma\cap\beta_i\neq\emptyset$\\
% $\frac{d}{dl_{\beta_i}}(e^{-\frac1Tl_\gamma})=-\frac1Te^{-\frac1Tl_\gamma}\frac{d}{dl _{\beta_i}}l_\gamma$
% \\
% \\
\noindent
% Critical points:\\
It is relatively clear about $\crit(\syst|_{\overline{\mathcal M}})\cap\mathcal M$. For critical points on the boundary $\partial\mathcal M$, they correspond exactly to the critical points for $\syst|_{\mathcal M(\mathcal S)}$. The $T^{\textit{Str}}$-directional derivatives are 0 since $\nabla\syst|_{\mathcal S}$ is parallel to $\nabla\syst|_{\mathcal M(\mathcal S)}$ under the identification $\mathcal S\leftrightarrow\mathcal M(\mathcal S)$ and so are $T^{\textit{Nod}}$-directional derivatives following Remark \ref{boundarydirection}.\\
\\
% By Remark \ref{boundarydirection}, for any $X$ on the boundary with pinched curve set as above, $\syst$ has derivative 0 in the nodal direction at the pinched surface $X$. This shows the inverse. Therefore, the critical points are 
% \\
% \begin{align*}
%     e^{-\frac1Tk^2}&\to1\\
%     -\frac1T e^{-\frac1Tk^2}\nabla (k^2)&\to0\\
%     \frac{1}{T^2}e^{-\frac1Tk^2}(\nabla(k^2))^2-\frac1Te^{-\frac1Tk^2}\nabla^2(k^2)&\to-\frac1T\nabla^2(k^2)\\
%     \nabla^2(k^2)&=2I
% \end{align*}
On a nodal chart $(U,\{k_ie^{i\tau_i}\})$, the second derivative matrix is
% Second derivative:
% \begin{align*}
% \tilde H_{\syst}(\tau,\tau)&=\frac{\sum_\gamma e^{-{\frac1T}l_\gamma}\nabla^2l_\gamma(\tau,\tau)}{\sum_\gamma e^{-{\frac1T}l_\gamma}}\ge0\\
% \end{align*}
\begin{align*}
\begin{pmatrix}
2I & 0\\
0 & \ast
\end{pmatrix},
\end{align*}
where $\ast$ is the second derivative matrix of $\syst$
restricted on $T^{\textit{Str}}$ that is nondegenerate. This shows that critical points on $\partial\mathcal M$ are nondegenerate.
\begin{theorem}
    $\syst:\overline{\mathcal M}\to\mathbb R$ is $C^2$-Morse with the given differential structure.
\end{theorem}
\noindent
Part (2) of the main theorem holds true for the extended $\syst$ for similar reasons as in Sections \ref{behavior}-\ref{boundary}. There we complete the proof of the main theorem.
% Application of Morse inequalities:????
% \cite{hepworth2009morse}
% \begin{theorem}
%     $$\sum(-1)^nC_n=\chi(\overline{\mathcal M})$$
%     $$\sum_{n=0}^m(-1)^{m-n}C_n\ge\sum_{n=0}^m(-1)^{m-n}b_n(\overline{\mathcal M})$$
% \end{theorem}
% Lower bound of index of critical points?
\section{Weil-Petersson Gradient Flow on $\overline{\mathcal M}_{1,1}$}
\label{case}
\noindent
As a $C^0$ function, the systole does not give a Weil-Petersson gradient vector field. Although we can consider the gradient `direction' instead, it turns out to be a discontinuous distribution in the tangent space. The $\syst$ functions are $C^2$ and thus it is possible to observe the structure of the base space through the gradient flow. Here we present an example.
\begin{theorem}
The Weil-Petersson gradient flow of $\sys_T$ on $\mathcal M_{1,1}$ when $T$ is sufficiently small:
\end{theorem}
% Figure environment removed
\noindent
The 3-punctured sphere is a 6-sheeted covering space of $\mathcal M_{1,1}$ that resolves the singularities. The red point is a triple point in the moduli space where $\syst$ attains its maximum, and the blue is a double point, see \cite{schaller1998geometry} for more on these surfaces. The punctures are 3-punctured spheres that form the Deligne-Mumford boundary.
% We construct $\mathcal T_{1,1}$ in the following way:\\
% \\
% Let $\Lambda_\tau=\mathbb Z+\mathbb Z\tau$, where $\tau\in\mathbb H$. For any torus $\mathbb C/\Lambda_\tau$, we remove the center of the parallelogram spanned by 1 and $\tau$ to get a once punctured torus, denoted by $S_\tau$. By Liouville's theorem, there exists a unique hyperbolic structure $m_\text{hyp}$ on $S_\tau$ that is conformal to the Euclidean metric. We denote $(S_\tau,m_\text{hyp})$ by simply $S_\tau$ by abuse of notation. As a result, $S_\tau$ respects the rotational and reflectional symmetries (if exist) of $\mathbb C/\Lambda_\tau$. The obtained map $\mathcal T_1\to\mathcal T_{1,1}$ is an isomorphism.\\
% \\
% Given that $\syst$ is a smooth Morse function, it is natural to study the geodesic flow, which does not exist for $\sys$ as it is not $C^1$. However, we can define the geodesic peudoflow for $\sys$ in the following way:
% \begin{definition}
% Let $f$ be a continuous function on a Riemannian manifold $M$, the gradient direction at $p$ is a vector in the unit tangent bundle $UT_pM$ such that $f$ increases fastest.
% \end{definition}
\pagebreak
\newpage
\section*{Table of notations}
\begin{table*}[htbp]
% \vspace{-1.5em}
\begin{center}
     \begin{tabular}{r c p{12cm} }
\toprule
&$J$ & multiindex, Weil-Petersson almost complex structure\\
&$K$ & multiindex\\
&$F_J$ & fan associated to $J$\\
&$v_J$ & projected average vector\\
&$c,C,\theta$ & constant\\
&$D$ & constant, Weil-Petersson connection\\
&$\gamma,\gamma_i$ & simple closed geodesic\\
&$\beta_i$ & pinched curve\\
&$l_\gamma,l_i$ & length function associated to $\gamma$ or $\gamma_i$\\
&$m,M$ & minimum, maximum of $\{||\nabla l_i||,||\nabla^2l_i(\tau,\cdot)\}_{i;\tau}$\\
&$\sys$ & systole function\\
&$\secsys$ & second systole function\\
&$\syst$ & $:=-T\log\sum_{\gamma \text{ simple closed geodesic on } X} e^{-\frac1Tl_\gamma(X)}$\\
&$S$ & pinched curve set\\
&$S(X)$ & set of shortest geodesics on a surface\\
&$s_X(L)$ & number of simple closed geodesics of length $\le L$ on $X$\\
&$r$ & $:=\#S(X)$, number of shortest geodesics\\
&$r_1,r_2$ & $\#$ of shortest geodesics with pairing with given vector $\ge0$, $<0$\\
&$I$ & index set for $S(X)$\\
&$I_{e}$ & subset of $I$ indexing the maximal eutactic subset\\
&$I_{nse}$ & complement of $I_{e}$ in $I$ when $X$ is semi-eutactic\\
&$\mathbf C$ & intersection of cones\\
&$\ex_p$ & exponential map at $p$\\
&$i$ & an automorphism of $T_p\mathcal T$\\
&$\pi$ & standard projection onto the unit sphere\\
&$\mathcal T,\mathcal M$ & Teichm\"uller space, moduli space\\
&$\overline{\mathcal M},\partial\mathcal M,\mathcal S$ & Deligne-Mumford compactification, D-M boundary, stratum\\
&$T^{\sys},T_p^{\sys}\mathcal T$ & major subspace\\
&$T^{\sys\perp},T_p^{\sys\perp}\mathcal T$ & minor subspace\\
&$T^{e}$ & subspace spanned by $\{\nabla l_i\}_{I_{e}}$\\
&$T^{e\perp\sys}$ & orthogonal complement of $T^{e}$ in $T^{\sys}$\\
&$\tilde\Omega_T,\Omega_T$ & see Definition \ref{setup}\\
&$\tilde\Phi_T,\Phi_T$ & see Section 6\\
&$\tilde\Psi_T,\Psi_T$ & see Definition \ref{ep2}\\
&$\Psi_1,\Psi_2$ & see Theorem \ref{existence2}\\
&$\tilde H_T, H_T$ & see Section 8\\


% \multicolumn{3}{c}{}\\
% \multicolumn{3}{c}{\underline{Decision Variables}}\\
% $y_{f}$ & $=$ & \left\{\begin{array}{rl}    1, &\mbox{ if Supplier located at site $f$ is open} \\
% 0, &\mbox{ otherwise} \end{array} \right.\\
% \bottomrule
    \end{tabular}
\end{center}
    \label{tab:TableOfNotationForMyResearch}
\end{table*}
% \begin{thebibliography}{9}
% \bibitem{texbook}
% Donald E. Knuth (1986) \emph{The \TeX{} Book}, Addison-Wesley Professional.
% \bibitem{lamport94}
% Leslie Lamport (1994) \emph{\LaTeX: a document preparation system}, Addison
% Wesley, Massachusetts, 2nd ed.
% \end{thebibliography}
% \\
\newpage
\bibliographystyle{alpha}
\bibliography{main}
% \\
% \\
% \\
% Biliography\\
% {[Kerckhoff]}\\
% {[Akrout]}\\
% {[Schmutz Schaller]}\\
% {[Mirzakhani]} Growth of the number of simple closed geodesics on hyperbolic surfaces\\
% ??? Shortening all the simple closed geodesics on surfaces with boundary,
% ATHANASE PAPADOPOULOS AND GUILLAUME THERET\\
% {[I. Rivin]}, Simple curves on surfaces, Geom. Dedicata 87 (2001), 345–360\\
% {[Wolpert]} Behavior of geodesic-length functions on Teichm\"uller space
% \\
% \\
% \\
% \\
% \\
% \\
% \\
% \\
% \\
% \\
% \\\begin{align*}
% & \langle\tau,\Psi_T(t,\tau)|_{S^{n-1}_{\rho T}}\rangle = \\
% & \sum_ie^{-\rho \langle\nabla l_i(p),\tau\rangle}(e^{-\frac12 \rho^2T \nabla^2l_i(\tau,\tau)}\langle\nabla l_i(p),\tau\rangle + \rho T\nabla^2l_i(p)(\tau,\tau))
% \end{align*}
% \\
% where 
% \\
% Define $E_T$ to be the (non standard) ellipsoid given by $||v||=T^{s(\theta(v))}$, where $\theta(v)$ is the angle between $v$ and $T_X^{\sys\perp}\mathcal T$, and $s(\cdot)$ is an increasing function that satisfies $s(0)=\frac13$, $s(\theta_0)=\frac 23$ and $s(\frac{\pi}{2})=1$.\\
% \\
% \\
% Restricted onto $E_T$, $\Psi_T(t,\tau)$ becomes 
% $$\Psi_T(t,\tau)|_{E_T}=\sum_ie^{-T^{s-1} \langle\nabla l_i(p),\tau\rangle}(e^{-\frac12 T^{2s-1} \nabla^2l_i(\tau,\tau)}\nabla l_i(p) +T^s\nabla^2l_i(p)(\tau,\cdot))$$
% \\
% \\
% When $\tau\in -F_{i_{j_1}\cdots i_{j_k}}\times T_X^{\sys\perp}\mathcal T$, \begin{align*}
% & \langle\tau,i\circ\Psi_T(t,\tau)|_{E_T}\rangle =\\
% & \sum_ie^{T^{s-1} \langle-\nabla l_i(p),\tau\rangle}(e^{-\frac12 T^{2s-1} \nabla^2l_i(\tau,\tau)}\langle-\nabla l_i(p),\tau\rangle +\langle i\circ T^s\nabla^2l_i(p)(\tau,\cdot),\tau\rangle) ???
% \end{align*}
% \\
% \\
% \\
% \\
%% Figure environment removed
% \\
% \\


%\citep{adams1995hitchhiker}

%\bibliographystyle{plain}
%\bibliography{references}
\end{document}