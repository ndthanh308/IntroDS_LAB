%Version 2.1 April 2023
% See section 11 of the User Manual for version history
%
%%%%%%%%%%%%%%%%%%%%%%%%%%%%%%%%%%%%%%%%%%%%%%%%%%%%%%%%%%%%%%%%%%%%%%
%%                                                                 %%
%% Please do not use \input{...} to include other tex files.       %%
%% Submit your LaTeX manuscript as one .tex document.              %%
%%                                                                 %%
%% All additional figures and files should be attached             %%
%% separately and not embedded in the \TeX\ document itself.       %%
%%                                                                 %%
%%%%%%%%%%%%%%%%%%%%%%%%%%%%%%%%%%%%%%%%%%%%%%%%%%%%%%%%%%%%%%%%%%%%%

%%\documentclass[referee,sn-basic]{sn-jnl}% referee option is meant for double line spacing

%%=======================================================%%
%% to print line numbers in the margin use lineno option %%
%%=======================================================%%

%%\documentclass[lineno,sn-basic]{sn-jnl}% Basic Springer Nature Reference Style/Chemistry Reference Style

%%======================================================%%
%% to compile with pdflatex/xelatex use pdflatex option %%
%%======================================================%%

%%\documentclass[pdflatex,sn-basic]{sn-jnl}% Basic Springer Nature Reference Style/Chemistry Reference Style


%%Note: the following reference styles support Namedate and Numbered referencing. By default the style follows the most common style. To switch between the options you can add or remove “Numbered” in the optional parenthesis. 
%%The option is available for: sn-basic.bst, sn-vancouver.bst, sn-chicago.bst, sn-mathphys.bst. %  
 
\documentclass[sn-nature]{sn-jnl}% Style for submissions to Nature Portfolio journals
%%\documentclass[sn-basic]{sn-jnl}% Basic Springer Nature Reference Style/Chemistry Reference Style
%%\documentclass[sn-mathphys,Numbered]{sn-jnl}% Math and Physical Sciences Reference Style
%%\documentclass[sn-aps]{sn-jnl}% American Physical Society (APS) Reference Style
%%\documentclass[sn-vancouver,Numbered]{sn-jnl}% Vancouver Reference Style
%%\documentclass[sn-apa]{sn-jnl}% APA Reference Style 
%%\documentclass[sn-chicago]{sn-jnl}% Chicago-based Humanities Reference Style
%%\documentclass[default]{sn-jnl}% Default
%%\documentclass[default,iicol]{sn-jnl}% Default with double column layout

%%%% Standard Packages
%%<additional latex packages if required can be included here>

\usepackage{graphicx}%
\usepackage{multirow}%
\usepackage{amsmath,amssymb,amsfonts}%
\usepackage{amsthm}%
\usepackage{mathrsfs}%
\usepackage[title]{appendix}%
\usepackage{xcolor}%
\usepackage{textcomp}%
\usepackage{manyfoot}%
\usepackage{booktabs}%
\usepackage{algorithm}%
\usepackage{algorithmicx}%
\usepackage{algpseudocode}%
\usepackage{listings}%

%%%%

%%%%%=============================================================================%%%%
%%%%  Remarks: This template is provided to aid authors with the preparation
%%%%  of original research articles intended for submission to journals published 
%%%%  by Springer Nature. The guidance has been prepared in partnership with 
%%%%  production teams to conform to Springer Nature technical requirements. 
%%%%  Editorial and presentation requirements differ among journal portfolios and 
%%%%  research disciplines. You may find sections in this template are irrelevant 
%%%%  to your work and are empowered to omit any such section if allowed by the 
%%%%  journal you intend to submit to. The submission guidelines and policies 
%%%%  of the journal take precedence. A detailed User Manual is available in the 
%%%%  template package for technical guidance.
%%%%%=============================================================================%%%%

%\jyear{2021}%

%% as per the requirement new theorem styles can be included as shown below
%\theoremstyle{thmstyleone}%
%\newtheorem{theorem}{Theorem}%  meant for continuous numbers
%%\newtheorem{theorem}{Theorem}[section]% meant for sectionwise numbers
%% optional argument [theorem] produces theorem numbering sequence instead of independent numbers for Proposition
%\newtheorem{proposition}[theorem]{Proposition}% 
%%\newtheorem{proposition}{Proposition}% to get separate numbers for theorem and proposition etc.

%\theoremstyle{thmstyletwo}%
%\newtheorem{example}{Example}%
%\newtheorem{remark}{Remark}%

%\theoremstyle{thmstylethree}%
%\newtheorem{definition}{Definition}%

\raggedbottom
%%\unnumbered% uncomment this for unnumbered level heads

\usepackage{subfig}
\usepackage{colortbl}
\definecolor{LightCyan}{rgb}{0.88,1,1}


\begin{document} 

\title[Article Title]{The Role of the IRA in Twitter during the 2016 US Presidential Election: Unveiling Amplification and Influence of Suspended Accounts}

%%=============================================================%%
%% Prefix	-> \pfx{Dr}
%% GivenName	-> \fnm{Joergen W.}
%% Particle	-> \spfx{van der} -> surname prefix
%% FamilyName	-> \sur{Ploeg}
%% Suffix	-> \sfx{IV}
%% NatureName	-> \tanm{Poet Laureate} -> Title after name
%% Degrees	-> \dgr{MSc, PhD}
%% \author*[1,2]{\pfx{Dr} \fnm{Joergen W.} \spfx{van der} \sur{Ploeg} \sfx{IV} \tanm{Poet Laureate} 
%%                 \dgr{MSc, PhD}}\email{iauthor@gmail.com}
%%=============================================================%%

\author*[1,5]{\fnm{Matteo} \sur{Serafino}}\email{m.serafi00@ccny.cuny.edu}

\author[2,5]{\fnm{Zhenkun} \sur{Zhou}}%\email{iauthor@gmail.com}

\author[3]{\fnm{Jos\'e} \sur{S. Andrade, Jr.}}%\email{iiauthor@gmail.com}
%\equalcont{These authors contributed equally to this work.}

\author[4]{\fnm{Alexandre} \sur{Bovet}}%\email{iiiauthor@gmail.com}
%\equalcont{These authors contributed equally to this work.} 

\author*[1]{\fnm{Hern\'an} \sur{A. Makse}}\email{hmakse@ccny.cuny.edu}
%\equalcont{These authors contributed equally to this work.}


\affil[1]{\orgdiv{Levich Institute and Physics Department}, \orgname{City College of New York}, \orgaddress{\city{New York}, \state{NY}, \country{USA}}}

\affil[2]{\orgdiv{School of Statistics}, \orgname{Capital University of Economics and Business}, \orgaddress{ \city{Beijing}, \country{China}}}

\affil[3]{\orgdiv{Physics Department}, \orgname{Universidade Federal do Cear\'a}, \orgaddress{\city{Fortaleza}, \state{Cear\'a}, \country{Brazil}}}

\affil[4]{\orgdiv{Department of Mathematics and Digital Society Initiative}, \orgname{ University of Zurich}, \orgaddress{\city{Zurich}, \country{Switzerland}}}

\affil[5]{These authors contributed equally: Matteo Serafino, Zhenkun Zhou}




%%==================================%%
%% sample for unstructured abstract %%
%%==================================%%

\abstract{The impact of the social media campaign conducted by the Internet Research Agency (IRA) during the 2016 U.S. presidential election continues to be a topic of ongoing debate. While it is widely acknowledged that the objective of this campaign was to support Donald Trump, the true extent of its influence on Twitter users remains uncertain. Previous research has primarily focused on analyzing the interactions between IRA users and the broader Twitter community to assess the campaign's impact. In this study, we propose an alternative perspective that suggests the existing approach may underestimate the true extent of the IRA campaign. Our analysis uncovers the presence of a notable group of suspended Twitter users, whose size surpasses the IRA user group size by a factor of 60. These suspended users exhibit close interactions with IRA accounts, suggesting potential collaboration or coordination. Notably, our findings reveal the significant role played by these previously unnoticed accounts in amplifying the impact of the IRA campaign, surpassing even the reach of the IRA accounts themselves by a factor of 10. In contrast to previous findings, our study reveals that the combined efforts of the Internet Research Agency (IRA) and the identified group of suspended Twitter accounts had a significant influence on individuals categorized as undecided or weak supporters, probably with the intention of swaying their opinions.}

%\abstract{The impact of the social media campaign by the Internet Research Agency (IRA) during the 2016 U.S. presidential election remains debated. While the campaign aimed to support Donald Trump, its influence on Twitter users remains uncertain. Previous research has primarily focused on analyzing the interactions between IRA users and the broader Twitter community to assess the campaign's impact. In this study, we propose an alternative perspective that suggests the existing approach may underestimate the true extent of the IRA campaign. Our analysis uncovers a notable group of suspended Twitter users, surpassing the IRA user group size by 60. These suspended users exhibit close interactions with IRA accounts, suggesting potential collaboration or coordination. Notably, our findings reveal the significant role played by these previously unnoticed accounts in amplifying the impact of the IRA campaign, surpassing even the reach of the IRA accounts themselves by a factor of 10. In contrast to previous findings, our study reveals that the combined efforts of the Internet Research Agency (IRA) and the identified group of suspended Twitter accounts had a significant influence on individuals categorized as undecided or weak supporters, probably with the intention of swaying their opinions.}

%%================================%%
%% Sample for structured abstract %%
%%================================%%

% \abstract{\textbf{Purpose:} The abstract serves both as a general introduction to the topic and as a brief, non-technical summary of the main results and their implications. The abstract must not include subheadings (unless expressly permitted in the journal's Instructions to Authors), equations or citations. As a guide the abstract should not exceed 200 words. Most journals do not set a hard limit however authors are advised to check the author instructions for the journal they are submitting to.
% 
% \textbf{Methods:} The abstract serves both as a general introduction to the topic and as a brief, non-technical summary of the main results and their implications. The abstract must not include subheadings (unless expressly permitted in the journal's Instructions to Authors), equations or citations. As a guide the abstract should not exceed 200 words. Most journals do not set a hard limit however authors are advised to check the author instructions for the journal they are submitting to.
% 
% \textbf{Results:} The abstract serves both as a general introduction to the topic and as a brief, non-technical summary of the main results and their implications. The abstract must not include subheadings (unless expressly permitted in the journal's Instructions to Authors), equations or citations. As a guide the abstract should not exceed 200 words. Most journals do not set a hard limit however authors are advised to check the author instructions for the journal they are submitting to.
% 
% \textbf{Conclusion:} The abstract serves both as a general introduction to the topic and as a brief, non-technical summary of the main results and their implications. The abstract must not include subheadings (unless expressly permitted in the journal's Instructions to Authors), equations or citations. As a guide the abstract should not exceed 200 words. Most journals do not set a hard limit however authors are advised to check the author instructions for the journal they are submitting to.}

\keywords{Social network, Disinformation, Election, Russian trolls}

%%\pacs[JEL Classification]{D8, H51}

%%\pacs[MSC Classification]{35A01, 65L10, 65L12, 65L20, 65L70}



\maketitle
\clearpage
\newpage
\section{Introduction}\label{sec1}
Social media platforms have become increasingly prominent in shaping
political events and social discussions. Political campaigns across
the globe are heavily reliant on social media platforms to communicate
with the masses and shape public opinion \cite{digrazia2013more, 
anstead2015social,bovet2018validation, ahmed20162014, majo2021role}. 
However, the rise of social media has also resulted in debates about 
their impact on society and the potential risks associated with their use.

Social media platforms, while holding the potential to facilitate 
communication and foster informed discussions, are also susceptible 
to the dissemination of misinformation and disinformation campaigns 
\cite{hegelich2016social, ratkiewicz2011detecting}. This issue extends
beyond politics and seeps into sensitive domains like public health,
as exemplified by the anti-vaccine movements during the COVID-19 
pandemic \cite{burki2020online}. Compelling evidence abounds, pointing
to the active exploitation of social media platforms by certain governments
to subvert domestic social movements and interfere in the democratic
elections of foreign adversaries \cite{tucker2017liberation}. Noteworthy
instances of such foreign interventions include the case of the 2017 
French presidential election \cite{ferrara2017disinformation} and 
the highly significant interference by the Internet Research Agency
(IRA: a Russian company engaged in online influence operations on 
behalf of Russian business and political interests) in the 2016 US
presidential election \cite{TheDisinformationReport, jamieson2020cyberwar}.

As outlined in the U.S. Special Counsel's report~\cite{mueller2019report}, 
the Internet Research Agency initiated Russian interference operations
as early as 2009. Their strategic approach involved the creation of social
network campaigns aimed at fueling and magnifying political and social
divisions within the United States \cite{mueller2019report,carroll2017st}. 
At the beginning of 2018, Twitter committed to the United States Congress and 
the public to provide regular updates and information regarding their investigation 
into foreign interference in U.S. political conversations on Twitter. In October
2018, Twitter openly released all the accounts and related content associated with
potential information operations they had found on Twitter since 2016. This 
dataset consists of more than three thousand accounts affiliated with the IRA. 
It contains more than 9 million tweets, including the earliest Twitter activity
of the accounts connected with these campaigns, dating back to 2009. The Twitter
corporation estimates that 9\% of the tweets from IRA accounts were election-related.

Since then, the number of works focusing on the role the IRA agency played in 
the 2016 US political campaign and social debates increased. In \cite{badawy2018analyzing}
A. Badawy {\it et al.} found that conservatives retweeted Russian trolls 
significantly more often than liberals and produced 36 times more tweets. 
Among the 5.7 million distinct users analyzed between September 16 and November 9,
2016, about 4.9\% and 6.2\% of liberal and conservative users, respectively,
were automated accounts (bots) used to share troll content. Text analysis of the
content shared by trolls reveals that they had a mostly conservative, pro-Trump agenda. 
In \cite{howard2018ira}, through extensive volumetric analysis, P. N. Howard 
{\it et al.} concluded that the Russian strategies targeted many communities within
the United States, particularly the most extreme conservatives and those with 
particular sensitivities to race and immigration. They found that IRA used a variety
of fake accounts to infiltrate political discussions in liberal and conservative
communities, including black activist communities, to exacerbate social 
divisions and influence the agenda. By combining network science and volumetric analysis, L. G. Stewart
{\it et al.} found that troll accounts shared content to polarized information
networks, likely accentuating disagreement and fostering division \cite{stewart2018examining}.
The conclusions above align with the findings of R. DiResta {\it et al.} 
\cite{diresta2018tactics}, who observed that the IRA campaign was designed to exploit societal fractures,
blur the lines between reality and fiction, and erode trust in media entities and the information
environment, in government, in each other, and in democracy itself. In their study 
on disinformation \cite{zannettou2019disinformation}, S. Zannettou {\it et al.} conducted an 
investigation into the behavioral differences between IRA and random Twitter users. The findings 
revealed that IRA users exhibit a higher tendency to disseminate content related to politics.
Additionally, IRA employed multiple identities throughout the lifespan  of their accounts and 
made deliberate efforts to amplify their impact on Twitter by increasing their number of followers.

The studies mentioned above aim to characterize the IRA campaign. 
In an attempt to evaluate the impact of the IRA campaign on Twitter users, 
C. Bail {\it et al.} conducted a study using a longitudinal survey that 
describes the attitudes and online behaviors of one thousand Republican
and Democratic Twitter users in late 2017 \cite{bail2020assessing}. Their findings
suggest that Russian trolls might have failed to sow discord because they mostly
interacted with those who were already highly polarized.
In \cite{grinberg2019fake}, N. Grinberg {\it et al.} demonstrated that exposure to fake
news content during the 2016 elections was typically concentrated among a small group
of users, particularly those who identify themselves as strong political partisans. If exposure 
to social media posts from Russian foreign influence accounts during the 2016 US election was 
similarly concentrated, their impact on changing candidate preferences may have
been minimal. In the attempt to verify this hypothesis, Eady {\it et al.} \cite{eady2023exposure}
combined US longitudinal survey data from over 1496 respondents with Twitter data. 
They found that exposure to the Russian foreign influence campaign was heavily
concentrated among a small fraction of users who identified themselves as Republicans. 
Moreover, they found no evidence of a significant relationship between exposure
to the campaign and changes in attitudes, polarization, or voting behavior in the 
2016 US election. 

%While this study provides valuable insights into the topic, it is crucial to 
%critically evaluate the methodologies employed and consider additional factors
%that may contribute to a comprehensive understanding of the issue at hand. 
%Firstly, although the authors made efforts to mitigate sampling bias, 
%it is important to acknowledge that the findings are based on a 
%a relatively small sample of fewer than two thousand individuals. 
%This sample size may not adequately represent the vast number of users
%actively engaged in political debates on social media platforms during 
%the political campaign.

By focusing on the interaction between Twitter users
and IRA users, the studies mentioned above suggest that the IRA 
campaign did not effectively influence Twitter users. 
However, social networks exhibit intricate dynamics that involve
connections beyond direct links, known as higher-order connections 
\cite{granovetter1973strength, newman2018networks, vosoughi2018spread}. 
These higher-order connections involve relationships mediated by other
individuals in the network, introducing a layer of indirect influence.
By solely examining direct contact between genuine users and IRA accounts,
the full picture of information dissemination may be incomplete. This 
becomes particularly important when considering the role of bots in
amplifying the reach of IRA content \cite{badawy2018analyzing}. The role 
of bot networks in political propaganda has also been observed \cite{caldarelli2020role},
emphasizing the need to consider an additional layer of indirect influence
when examining the impact of the IRA's social media campaigns.

%Recent research has highlighted the substantial impact of higher-order
%contacts on information diffusion and opinion formation in social media. 
%For example, studies analyzing the spread of false news on Twitter revealed
%that many users who propagated false information had a limited number of 
%followers and followed few accounts. However, they were often embedded
%within highly connected networks \cite{vosoughi2018spread}. This suggests 
%that analyzing direct connections alone may not capture the complete
%dynamics of information spread.

In addition, during the ``Twitter purge" in May 2018, Twitter 
suspended thousands of accounts, including those unrelated to 
the IRA \cite{bursztein2018quantifying, roth2018twitter}. Although
these accounts were classified as not of Russian provenance, no
investigations have specifically examined the relationships 
between the IRA and the other suspended accounts. While some 
studies have examined the impact of these suspended accounts 
and Twitter's countermeasures \cite{le2019postmortem}, their 
specific influence on shaping people's opinions during the 2016
US presidential election remains unclear.

This research contributes to the ongoing
discourse by investigating the potential role of suspended 
accounts in amplifying the IRA's disinformation campaign. By 
merging the IRA dataset with a comprehensive collection of 171
million tweets, we delve into the underlying structure and 
strategies employed by the IRA's political campaign. Our analysis uncovers
the presence of a group of suspended accounts, previously unnoticed by
prior studies, that demonstrate close connections to the IRA accounts.
In contrast to previous findings \cite{eady2023exposure}, our research
provides evidence that the collaboration between the IRA and these 
suspended users did indeed have an influence on regular users. Furthermore, 
our findings suggest that their impact extends beyond highly partisan 
individuals and targets undecided users and weak supporters who are
unsure about their voting choices.
 
This paper is organized as follows. In Section \ref{Section1}, 
we provide a characterization of users based 
on their account status and the content they share. Section \ref{Section2}
focuses on the analysis of the IRA ego network, which includes the IRA
nodes and their immediate contacts. Within this network, we identify a
group of suspended nodes that are closely associated with the IRA accounts.
Building upon these findings, we expand the analysis to include the contact 
of these suspended accounts in the IRA ego network. In Section \ref{Section4}, we
investigate the causal relationships between IRA, suspended users,
and the broader Twitter user population. Finally, we conclude the paper
with a comprehensive discussion and conclusion that encompasses the 
insights gained from our analyses.

\section{Comparing IRA, Suspended, and Active Accounts}
\label{Section1}

To characterize the role that IRA played during the 2016 US election 
on Twitter, we combine the IRA dataset with a dataset containing tweets
posted between June 1st and the election day, November 8th, 2016. The data
were collected continuously using the Twitter search API with the names 
of the two presidential candidates \cite{bovet2018validation,bovet2019influence, 
flamino2023political}. It is important to emphasize that the 2016 dataset 
was collected in real-time during the actual 2016 political campaign and was not obtained
using the historical Twitter API. This unique dataset contains valuable
information regarding IRA users and their posts, which cannot be accessed
through the Twitter historical API. It should be noted that Twitter removes all
the content shared by those accounts  that are suspended or not found, 
such as their posts, and associated account information. On Twitter,
a suspended account refers to an account that has been temporarily or permanently
disabled by Twitter due to a violation 
of its rules or policies. In contrast, a not found account is not deleted by Twitter 
but is no longer available because the user has chosen to delete or deactivate 
it. A not verified account on Twitter is an account that has not been officially
confirmed by Twitter. Verification is a process through which Twitter 
verifies the authenticity and identity of notable public figures, organizations,
or brands. On the other hand, a verified account on Twitter has undergone the 
verification process and has been confirmed by Twitter as an authentic
representation of a notable public figure, organization, or brand. Verified 
accounts are distinguished by a blue checkmark badge next to their username,
indicating their credibility and authenticity. Important to note that this is no longer the case (as of November 29th, 2023), as now anyone can buy the blue checkmark. 

The 2016 dataset consists of 171 million tweets sent by 11 million users.
To retrieve the account status of each user, we used the Twitter users API, as of October 2023.
It allows us to classify each account as suspended, not found, not verified,
or verified. Among the 11 million users, 73.8\% are not verified, 17.7\% are not
found, 7.7\% are suspended, and 0.8\% are verified. 

%To these four groups of accounts, 
%we add a further one, the IRA accounts, which are also suspended accounts. 

The misinformation campaign conducted by IRA started on April 24,
2009 the day of the creation of the first IRA account, and 
continued until April 3, 2018 (Twitter began closing IRA
accounts in 2017). Figure~\ref{fig:1}a shows the number of
accounts that the IRA created monthly to sustain its misinformation 
campaigns. This data is derived from the IRA dataset made available by Twitter.
Initially, the IRA produced them in small 
monthly amounts until reaching a peak of more than 500 accounts 
during August 2014. Figure~\ref{fig:1}b shows the evolution of the tweets
posted by IRA users grouped by languages. This data is also obtained from the IRA dataset
made available by Twitter. Among the roughly 9 million
tweets posted by the 3667 accounts, 54\% of the posts were 
in Russian, 36\% were in English, and the rest were in other languages.
As such, it is reasonable to suppose that IRA was conducting two 
parallel misinformation campaigns. A Russian campaign started at the
beginning of 2014, and a second campaign targeting U.S. audiences started
a few months later in the middle of 2014 \cite{howard2018ira}. From 2016,
IRA mainly targeted English-speaking audiences, producing more 
English than Russian posts (shadow area in Fig.~\ref{fig:1}b). From June 1st
to November 8th, 2016, 556 IRA accounts published 391680 tweets in English. 
According to \cite{howard2018ira}, the content of these tweets aimed to sow and
amplify political and social discord in the United States and manipulate the
2016 American presidential election. 

To understand the role that suspended, not found, not verified, verified, and IRA
accounts played in the spread of different types of news, we focus on tweets that 
contain at least one URL (Uniform Resource Locator) pointing to a news website
outside of Twitter. We classified URL links for outlets that mostly conform to 
professional standards of fact-based journalism in five news media categories:
Right, Right leaning, Center, Left leaning, and Left. The 
classifications rely on the website \href{https://www.allsides.com/unbiased-balanced-news}{allsides.com} 
(AS), followed by the bias classification from the website \href{https://mediabiasfactcheck.com/}
{mediabiasfactcheck.com} (MBFC) for outlets not listed in AS (both accessed on 7 January
2021 for the 2020 classification) \cite{bovet2018validation, bovet2019influence,flamino2023political,
grinberg2019fake}. We also include three
additional news media categories to include outlets that tend to disseminate
disinformation: Extreme bias right, Extreme bias left, and Fake news 
\cite{bovet2018validation,bovet2019influence,flamino2023political,
grinberg2019fake}. Websites in the fake news category have been flagged by 
fact-checking organizations as spreading fabricated news or conspiracy theories, 
while websites in the extremely biased categories have been flagged for reporting 
controversial information that distorts facts and may rely on propaganda,
decontextualized information or opinions misrepresented as facts. Supplementary 
Table 1 offers the list of news outlets per category considered in this work.

In the 2016 dataset, 2.3 million users shared 30.7 million tweets that contained 
URLs directing to news outlets. In the IRA dataset, 334 IRA accounts posted 23,806
tweets that included hyperlinks to news outlets. Figure~\ref{fig:2} shows the fraction
of tweets (panel a) and users (panel b) grouped by account types and news categories. 
Normalization is computed for each account type, meaning that, for example, the fraction 
of all the not verified users among the different categories sums up to one. 
Users are classified as being in the category in which they posted the most.

% with the other groups having
%a lower fraction ranging from  16\% to 21\% for center-related tweets and 16\% to 27\% for 
%left-related tweets.
As shown in Fig.~\ref{fig:2}, verified accounts have a significantly higher fraction of Center,
left-related (Left and Left leaning) tweets and the lowest fraction of right-related 
(Right, Right leaning, and Extreme biased right) tweets. Conversely, IRA, suspended and 
not found accounts have the highest fraction of right-related tweets. Notably, IRA accounts 
also show a high percentage of Center and Left leaning related news. Suspended 
accounts have the lowest fraction of left-related content and the highest fraction of 
fake-related content. As shown in Fig.~\ref{fig:2}b, the distribution of users among 
groups is more homogeneous, with verified and not verified accounts having the highest 
fraction of users in the Center and left-related categories and the lowest fractions 
of users in the right-related categories. The highest fraction of right-related users 
is held by suspended accounts, followed by not found and IRA accounts. Supplementary 
Tables 2 and 3 offer a full view of these percentages.

To assess potential statistical differences in news sharing among different account types,
we perform a contingency chi-square test \cite{cressie1984multinomial,lowry2014concepts}, 
with significance  threshold $\alpha=0.001$ and null hypothesis $H_{0}$: the two groups have no 
significant difference in the news they share. We focus on the tweets activity, 
and we test two groups per time, considering all the possible combinations (see 
Methods \ref{methods0}). For every combination of groups, the test 
rejects the null hypothesis, i.e., the share of news outlet type does depend on the
group we consider. The same result holds true when considering the number of users 
instead of the number of tweets per group.

%Besides misinformation and traditional news website, we also find that IRA users
%post around 27 thousand tweets (accounting for 49\% of the total tweets with an URL) 
%on local news websites without obvious political leanings 
%({\textcolor{blue}{see Table~1 in the Supplementary Material}}).
%98\% of them are original tweets sent by 49 IRA users, which leads to
%an average number of 530 original tweets per user. On the other hand, 
%non-IRA users post 243594 tweets (accounting for 7\% of the total) 
%related to local news, of which only 38\% of them are original tweets,
%with an average number of original tweets per user of around 3. 
%As such, IRA users are 200\% more active than non-IRA users in the
%sread of original content from local news when compared with non-IRA users.

%Another interesting difference between IRA and non-IRA accounts
%concern the type of Twitter clients they use to interact with the social platform.


Analyzing the clients can provide valuable insights into the origin
of tweets, particularly whether they are likely to be generated by bots. Third-party
clients, referred here as non official clients, encompass a range of applications, from 
those used by professionals to automate tasks (e.g., \href{https://ifttt.com/}{ifttt} or
\href{https://dlvrit.com/}{dlvrit}) to manually programmed bots. In Fig.~\ref{fig:3}, 
we present the percentage of  tweets shared through non official clients (panel a) and 
official clients (panel b) for each group in each media category. To ensure comparability, 
we normalize the percentages of tweets per account type by the total activity of the group, including both
official and non official clients. For a comprehensive list of official clients refer to Supplementary Table 4.

Interestingly, the fraction of tweets from verified accounts coming
from non official clients is the highest among all account types, accounting for 22.9\% 
of their total activity. When examining the most frequently used clients by verified accounts, we 
observe \href{https://www.hootsuite.com/}{Hootsuite} and \href{https://piano.io/it/
product/socialflow/}{Socialflow}, which are well-known applications used to automate
interactions with the Twitter ecosystem. These verified accounts 
belong to journalists or public figures who utilize such tools for their social media
activities. Suspended accounts rank second with 23.9\%, with \href{https://dlvrit.com/}{Dlvrit} 
as the most used client, followed closely by IRA accounts with 22.9\%, mostly
reliant on \href{https://twitterfeed.com/}{Twitterfeed} to perform automated activity.
Not verified and not found accounts have below 13\% non official clients usage,
with \href{https://twitterfeed.com/}{Twitterfeed} being the most used client.
It is noteworthy that non official clients were predominantly
used by verified accounts to spread Center, Left-leaning, and Left news, while IRA accounts
used non official clients mostly to spread right-related content, with a smaller 
percentage of Left leaning content. Suspended accounts used non official clients mainly
to disseminate Fake news, with a smaller percentage of Center and Left-leaning news.
For tweets posted by official clients, see Fig.~\ref{fig:3}b, we observed that verified
accounts focused mainly on Left leaning news. Not verified accounts have a distribution
similar to verified accounts. Suspended accounts mostly used official 
clients to spread Extreme bias right, Right, and Fake news. IRA and not found accounts 
used official clients similarly to not verified accounts, with higher percentages towards
Right and Extreme bias right news. A full view of the percentages can be found in Supplementary 
Tables 5 and 6. A contingency chi-square test between the distributions in Fig.~\ref{fig:3}a 
(considering two groups per time and considering all the combinations) rejects the null
hypothesis $H_0$, highlighting that different groups tend to behave differently in the usage
of non official clients. The same result holds true in the case of official clients.

These results show evidence of attempted interference by IRA accounts in the
2016 US presidential election. Specifically, we observed a significant increase 
in IRA-linked Twitter activity during the election year, with a focus on 
English-speaking audiences. Our analysis also revealed that IRA accounts 
demonstrated a preference toward right-related content, which was often shared
through non official clients. IRA accounts also show a preference for Left leaning 
content, which in most of the case is shared by means of official clients. 
Interestingly, we found that verified accounts, rather than IRA accounts, 
exhibited the highest use of non official clients to spread Left leaning content.
However, overall, the usage of non official clients across all groups was relatively
small, with automated activity ranging from 6\% to 30\%. Moreover, we found that
suspended accounts played a significant role in disseminating 
Fake, Extreme bias right, and Right news. In these cases, non official clients
were predominantly used to spread Fake news, while Extreme bias Right and Right 
news were more likely to be passed through official clients. 

\subsection{News Category Networks}

Although the findings above illustrate variations in political
orientation-related activity among suspended, not found, not verified, 
verified, and IRA accounts, additional investigations are necessary to 
evaluate the actual impact that each of these groups had on the 
dissemination of different types of news.
For this purpose, we built a retweet network for each news category. 
A link between two users occurs every time a user $u$ retweets a user's tweet $v$ 
that contains a URL linking to a website belonging to one of the news media categories.
The direction of the connection goes from $v$ to $u$, i.e., the direction of the
information flow between Twitter users. We do not include multiple links in the same 
direction between the same two users, nor do we include self-links.

Table~\ref{table:one} shows, for each retweet network, the number of nodes $N$, 
the number of edges $E$, 
and the average network degree. The degree of a node in a network is
defined as the number of edges connected to it. The out-degree of a 
node $u$, $ k_{out}^u $, represents the number of unique users who retweeted $u$. On the other
hand, the in-degree of a node $u$, $ k_{in}^u$, represents the number of users retweeted
by node $u$. It is worth noticing that, by construction, these networks are 
balanced directed networks, and as such, $ \langle k_{in} \rangle = \langle k_{out} \rangle =
\langle k \rangle /2$. 
%For convenience, we refer to the in, out average degree as $\langle k \rangle$.

Table~\ref{table:one} also shows the number of IRA users $N_{IRA}$ in each category, their average 
in-degree $ \langle k_{in} \rangle$, and the average out-degree  $ \langle k_{out} \rangle$.
Notice that in this case, the balanced directed network condition may not hold. To test
whether suspended, not found, verified, not verified, and IRA behave differently in terms 
of their in/out activity in each news category, we employ a two-sample Kolmogorov-Smirnov test \cite{hodges1958significance,mann1947test} (two-sided version, see Methods~\ref{methods2})
with null hypothesis $H_0$: the data are drawn from the same distribution.
We performed the test for each two-pair combination of the groups 
$(\overrightarrow{ k_{type} } ^i,\overrightarrow{ k_{type} } ^j)$, with $i,j \in $ 
(suspended, not found, verified, not verified, IRA), and $type \in$ (in, out). 
The $\overrightarrow{ k_{type}}$ vector contains the values of $\langle k_{type} \rangle$ 
for each category network.

To avoid sample bias, we randomly extracted the same amount as the number of 
IRA users for each group and category (making sure not to select the IRA users).
We average the in/out-degree over 1000 realizations. For each realization, sampling 
was without replacement. We refer to them as $(\overrightarrow{ k_{type}^s} ^i, 
\overrightarrow{ k_{type}^s } ^j)$, where the superscript $s$ indicates that the 
degree considered comes from the sampled nodes. Table~\ref{table:two} displays the sampled 
average degree for each group in each  news category, together with the standard error. 
Refer to Supplementary Table 7 for a view of the non-sampled case. The table 
demonstrates that verified accounts exhibit the highest out-degree among all categories and
groups, indicating a strong level of engagement. Notably, the Fake, Extreme bias right,
and Right categories receive approximately three times more retweets on average compared to the other categories. The IRA group follows 
closely as the second most retweeted, with the aforementioned categories receiving approximately
six times more retweets than the rest. In contrast, the suspended, not verified, and not found
accounts display considerably lower out-degree activity compared to the previous two groups. 
Conversely, suspended accounts show the highest in-degree, followed by the IRA, not found, 
not verified, and lastly, verified accounts.

Figures~\ref{fig:4}a and b show the results of the tests for the
out-degree and in-degree, respectively. We used a heatmap representation,
where the yellow color indicates the rejection of the null hypothesis $H_0$ 
in favor of the default two-sided alternative, suggesting that the data were
not drawn from the same distribution.

We always rejected the null hypothesis when comparing the out-degree activity of 
verified users with other groups, indicating that verified accounts behave differently.
IRA accounts exhibit different behavior from verified, not found, and not verified 
accounts but behave similarly to suspended accounts.
The in-degree results are less diverse (see Table~\ref{table:two}).
Verified accounts exhibit different behavior from suspended, not found, and 
not verified accounts, but similar to IRA users.

%The out-degree corresponds to the number of users who have retweeted content shared 
%by IRA users. The in-degree corresponds to the number of times IRA users retweeted 
%other users. The results above suggest that IRA users behave differently from 
%non-IRA users when retweeting other users. Moreover, IRA and Suspended accounts 
%are retweeted in a similar manner. However, when it comes to retweeting activity, 
%IRA accounts behave more like Verified accounts.

Although these findings provide valuable information regarding the retweet 
activity of each group, they do not provide a clear understanding of the
centrality each group holds. It is important to note that a high degree does not
necessarily indicate high centrality. Centrality in a network refers to the concept of 
identifying the most important nodes or actors in a network based on their position, 
influence, or connectivity \cite{bovet2021centralities}. Central nodes are those that have a 
significant impact on the network's structure and function. To measure the centrality of a node,
we used a directed version of the Collective  Influence (CI) algorithm \cite{morone2015influence}.
This algorithm considers influence as an emergent collective property, not as a local 
property like the node's degree, and has been shown to identify super-spreaders 
(users with high CI$_{out}$) and super-sinks (users with high CI$_{in}$) of information
in social networks \cite{bovet2019influence}. Here we try to understand how each 
type of user influenced the overall network and compare it with the IRA influence.
To this end and to avoid any sample bias, we implement a sampling strategy that 
randomly extracts for each news category and groups as many as the number of
IRA nodes. These nodes are then contracted into a single node incident to any edge 
that was incident to the original nodes, and finally, we compute the CI on the obtained 
super node. All edges between the selected nodes are removed.
See Methods~\ref{methods1}. 

Table~\ref{table:three} presents the rankings of the $\langle CI_{in} \rangle$ and
$\langle CI_{out} \rangle$ computed over 100 realizations for each account type and news media category. Higher $CI_{out}$  values indicate a greater ability to exert 
influence and impact the behavior or opinions of other users. Therefore, $CI_{out}$ 
can be used as a measure to identify influencers within the network. In contrast
to influencers, users with high values of $C_{in}$ represent individuals who 
frequently retweet but are not retweeted  by other users. These users act as sinkers in
the flow of information, receiving and disseminating content without actively 
gaining attention or amplification from others.

Verified users consistently occupy the top position as the most influential group
in each media category. Furthermore, verified users exhibit higher influence rankings,
particularly in the Extreme bias right, Right, and Center categories, with rankings 
above 38. Following verified users, the IRA accounts hold the second position with
rankings above 100 in the Fake, Extreme bias right, and Right categories. Among 
IRA accounts in Fake news is @TEN GOP, which claims to be the 
“Unofficial Twitter of Tennessee Republicans”. In the Extreme 
biased right network, we find @Pamela Moore13, that claimed to be 
a Texan Trump supporter \cite{mueller2019report}. The remaining groups do not 
play a significant role as influencers, as they occupy a ranking position above 1000.

%This means that
%while Suspended accounts engage in a high number of retweets, they themselves are
%not frequently retweeted by other users.
Suspended accounts have a significant presence as super sinks in the Fake, 
Extreme bias right, Right, Right-leaning, and Center categories. Following closely behind suspended 
accounts, we find the IRA accounts occupying a similar role. However, 
it is important to note that neither suspended nor IRA accounts play a 
significant super sink role in the left-related categories. For the remaining accounts, 
type their $\langle CI_{in} \rangle$ rankings surpass 200. Notably, in the Right leaning
category, both the not found and not verified accounts occupy prominent positions, 
ranking 14th and 22nd, respectively, which is higher than the ranking of the IRA accounts.

In order to gain a deeper understanding of user intent, 
our focus is directed toward the original tweets shared by individuals. 
Original tweets are the tweets created and posted by the users themselves.
They provide valuable insights into users' intentions, thoughts, and self-expression.
Through careful analysis of these original tweets, we can gain valuable insights
into the motivations, interests, and perspectives of users, enabling us to uncover
a wealth of information about their behaviors and preferences.

Figure~\ref{fig:5} shows the proportion of original tweets shared by each group 
and category. The normalization is over the total activity of each account type, 
meaning that the sum of the percentages per each account type represents the total 
fraction of original tweets. The group with the highest share of original tweets
is the verified one, with a value of 71.2\%. This group also shows the lowest share 
of original tweets linking to Fake news and the highest share of original tweets  
related to the Center, Left-Leaning, and Left categories.

IRA accounts instead show together with not found accounts, the lowest share of original 
tweets, with a percentage of 29.4\% and 27.3\%, respectively. Most of the original tweets
shared by IRA belong either to the Left leaning or Center categories.
Not found accounts have a more homogeneous distribution  of original tweets among the 
different categories. Suspended accounts, with 38.6\% of original tweets, show the highest
percentage of original tweets related to the Fake category and the Extreme 
bias right category. Not verified accounts (32\% of original tweets) show higher
percentages in the Center and Left leaning, similar to the verified accounts
but in a more homogeneous way.

Overall, the results in this section reveal an intriguing scenario. Verified accounts,
consisting of public users, hold high rankings as influencers and exhibit a higher
proportion of original tweets. Their low ranking as super sinks suggests that 
they primarily post original tweets that receive a significant number of retweets, 
but they do not tend to retweet other users frequently. This pattern aligns with their
status as important actors in the political landscape, and most of these influencers
lean towards the left-political orientation.

IRA accounts emerge as the second most influential group,
particularly excelling in the Fake and right-related categories. While their original 
tweets mostly focus on Center and Left-leaning content, Fig.~\ref{fig:2} indicates that
a majority of their tweets are right-related. This suggests their intention to discredit 
the left by creating tweets while endorsing the right through retweets. Notably,
IRA accounts also play a significant role as super sinks, especially in the right-related
categories, indicating their aim to inundate the platform with right-related retweets.

Conversely, suspended accounts hold the highest positions as a group of super sinks.
They primarily act as super sinks in the Fake and right-related categories, further 
indicating, similarly to IRA accounts, their objective to flood the social platform
with Fake and right-related retweets. Despite their extensive retweeting, these accounts 
do not receive the same level of retweet activity in return.

Interestingly, our analyses also reveal similar behaviors between IRA and
suspended accounts in terms of out-degree and in-degree activity. On the other hand,
not found and not verified users do not play significant roles as super-spreaders 
or super sinks in this context.

\section{IRA ego network}
\label{Section2}

If a cooperative relationship between suspended accounts and IRA
accounts were hypothetically discovered, it would indeed introduce a 
new dimension to the investigation. Such a scenario would 
require considering this new set of accounts when exploring the impact 
of the IRA campaigns on the 2016 US presidential election. However, it 
is important to note that while it is possible that a group of suspended 
accounts collaborated with the IRA, it is highly unlikely that all suspended 
accounts were involved in this direction. It is crucial to approach such
hypothetical scenarios with caution and thoroughly examine the available 
evidence to draw accurate conclusions. To test this hypothesis, we focus 
on the IRA ego network and conduct an in-depth study of its
community structure.

We construct separate networks for each of the four types of 
interactions: retweeting, mentioning, replying, and quoting. Each node in 
the network represents either an IRA or a non-IRA user. A link between two users 
occurs every time a user $u$ interacts with a user $v$ through a type of interaction.
The direction of the connection goes from $v$ to $u$, i.e., the
direction of the information flow. We summarize the key features of these networks in 
Table~\ref{table:four}. Retweeting and mentioning are the two most 
frequent types of interactions between IRA and non-IRA users. For each non-IRA user, 
we compute the number of times the account retweeted, 
mentioned, quoted, or replied to an IRA user, and vice-versa. We consider 
multiple interactions between two users; that is, networks are weighted by 
the number of times users interact. Doing so defines eight distributions: four
types of interactions and two directions of information flow (``in'' and ``out'').
For instance, the distribution of the number of retweets in the retweet network,
with the direction ``out'' lists how often each non-IRA user is retweeted by IRA users.
The ``in'' counterpart lists how often the non-IRA users retweet IRA users.

To test whether non-IRA users are more likely to interact with IRA users or vice-versa,
we perform a two-sample Kolmogorov-Smirnov test between the two distributions
extracted by each interaction network (version one-sided, see Methods \ref{methods2}). 
We set a level of 5\%, meaning that we will reject the null 
hypothesis and favor the alternative if the p-value is less than 0.05. 
The sample sizes for each interaction type are as follows: Retweet = 153,869,
Reply = 14,032, Mention = 70,418, and Quote = 18,842.

We could not reject $H_0$ for all interaction types but mentioning and quoting, 
which in turn means that non-IRA users are more likely to retweet and reply rather
than being retweeted or replied to. On the other hand, non-IRA users tend to be
mentioned or quoted more than mentioning and quoiting other users.

The results above suggest that retweets and mentions were the most common 
action between IRA and their first contacts, with non-IRA users being more inclined 
to retweet IRA content and being mentioned by IRA accounts. This behavior aligns with 
the assumption that users who want to establish connections and build relationships 
are more likely to mention other users, while those seeking information are more likely 
to retweet content from others. In other words, IRA accounts used mentions to
create relationships and expand their audience, while non-IRA users used IRA
content as an information source. 

Next, we analyze the interactions of users with different account types, such as verified, not verified, suspended, and not found. Specifically, we focus on the top 1000 users involved in each interaction type  (retweet, reply, quote, and tweet) and direction. For instance, in the case of retweet interaction, we consider the top 1000 (most retweeted) non-IRA users who were retweeted by IRA accounts in the ``out" direction. Similarly, in the ``in" direction, we examine the top 1000 (most retweeting) non-IRA users who retweeted IRA accounts.

{\bf The ``out'' direction}. Figure ~\ref{fig:6}a displays the distribution of the
users into the different account types. On average, 18.4\% of these users have 
verified accounts, 49.4\% are not verified, 18.9\% are suspended accounts, and 11.4\% are
not found accounts. Notably, among the verified accounts, we identified the
official profile of President Donald Trump and popular news outlets such as 
The Guardian and FOX NEWS. See Supplementary Tables 10, 11, 12, and 13 for a list of
the top 20 accounts.

{\bf The ``in'' direction}. Figure~\ref{fig:6}b, displays the distribution of the
users who interact with IRA into the different account types. None of the top 1000 
users who engage with IRA have verified accounts. On average, 45.1\% of them are not 
verified, 35.7\% is suspended, and 17.1\% is not found. See Supplementary Tables 14, 
15, 16, and 17 for a list of the top 20 accounts.

We compared these results with those obtained for users who did not interact with IRA users, 
as shown in Figs.~\ref{fig:6}c and d. To accomplish this, we constructed separate networks for 
each type of interaction: retweeting, mentioning, replying, and quoting. The methodology
used for building these networks is the same as explained for the IRA ego network
described earlier. However, in this case, we considered all the users who did not have 
direct interactions with IRA accounts. Figure~\ref{fig:6}c shows the results for 
the top 1000 non-IRA users retweeted, mentioned, quoted,
or replied to by non-IRA users, while Fig.~\ref{fig:6}d shows the results for the top 1000 non-IRA users that retweeted, mentioned, quoted, or replied to non-IRA users.

In terms of outward interactions, users non interacting with IRA are primarily 
engaged by verified, see Fig.~\ref{fig:6}c, while suspended and not found users
account for only 13.4\% of interactions. This percentage increases to 30.3\% for users 
interacting with IRA accounts, see Fig.~\ref{fig:6}a. The difference
is even more pronounced when examining inward interactions, as the proportion of the users 
interacting with not found and suspended accounts rise from an average of 35.6\% for
those users not interacting with IRA, see Fig.~\ref{fig:6}d, to 52.8\% for those interacting with
IRA, see Fig.~\ref{fig:6}a.

%As shown before, IRA accounts used mentions and quotes to increase their reachability. 
%This further layer of analyses suggests that the target of this operation was a
%consistent group of verified accounts. Mentioning Verified 
%accounts can be a strategy to enhance reachability due to their higher visibility
%and credibility among users. Verified accounts are typically associated with public
%figures, celebrities, organizations, or influential individuals who have undergone
%a verification process by the platform. By mentioning Verified accounts, IRA accounts
%may aim to associate themselves with well-known and trusted entities, thereby increasing
%the likelihood of their content being noticed and shared by a wider audience. Moreover, 
The results reveal that there is a higher probability of encountering suspended 
accounts within the IRA ego network compared to users who did not interact with IRA. 
This finding is intriguing as it suggests a potential association between IRA users and a
specific group of suspended accounts. The existence of this connection raises intriguing 
questions about the nature of their relationship and the possible implications it had 
during the period under study. Further investigation is needed to fully understand the
dynamics and implications of this association.

\subsection*{Ego polarization}
In this section, we delve deeper to uncover the inclination of users engaging with IRA 
towards a specific candidate. By exploring this additional layer,
we aim to reveal the extent to which these interactions potentially 
sway user sentiment and preferences, shedding light on the potential 
impact of IRA activities on favoring a particular candidate. 

Similarly to \cite{bovet2018validation}, we use a supervised classifier to classify each
tweet in favor of Donald Trump or Hillary Clinton. The training set was built using the
hashtag co-occurrences network to investigate Twitter users' opinions on the two presidential
candidates. We classified a user as a supporter of Trump if the number of her/his tweets 
supporting  Trump $N_{\textit{pro-T}}$ is greater than the number of tweets supporting Clinton
$N_{\textit{pro-C}}$. We define the support of a given user toward the candidates 
as $S=N_{\textit{pro-T}}- N_{\textit{pro-C}}$. If $S>0$, the user supports Trump. Otherwise, 
the user is likely to support Hillary. The highest the value of $S$ in absolute terms, the 
strongest the support. Considering all the users in the dataset, 65\% of them support Hillary 
Clinton while 28\% are in favor of Donald Trump (7\% are unclassified as they have the same
number of tweets in each camp)~\cite{bovet2018validation}. When considering only the users
interacting with IRA accounts, 25\% of the users are classified as Clinton supporters, and
72.6\% of the users are classified as Trump supporters. Figure~\ref{fig:8}a shows
the distribution of $S$ for the users interacting with IRA accounts (in red) and those
non interacting with IRA accounts (in blue).

%Fig. ~\ref{fig:dist_trump}b shows the vote intention for users interacting 
%with IRA (gray horizontal bar) and those who do not (purple horizontal bar). 
%In the former, 71.5\% of the users support Trump, while in the latter, 64.8\% 
%supports Hillary. The color of the link is according to the political preference
%of the users. We use Red for those users supporting Trump and blue for those 
%supporting Clinton. 

Figure~\ref{fig:8}a reveals an intriguing distinction in the distributions' tails. 
Specifically, there is a notable scarcity of strong supporters (high values of $S$
in absolute terms) among users who do not interact with IRA accounts. To estimate the 
power-law parameters of the distributions' tails, we use the Maximum Likelihood Estimation (MLE) method. The method 
aims to find the parameters that maximize the likelihood of observing the given 
data under the assumed power-law model \cite{alstott2014powerlaw}. While we used a 
power-law distribution to fit the data, other heavy tails may fit better. However, 
this investigation goes beyond the purpose of this study \cite{serafino2021true}.
Under the assumption of a power-law distribution for the right tails, we find that for users 
who do not interact with IRA, the best-fit results in a power law exponent $\alpha_{non-IRA}=3.05$,
while in the latter case, $\alpha_{IRA}=2.64$. When examining the left tail of the distribution, 
we observe a decrease in the power-law exponents: $\alpha_{non-IRA}$ is measured at 2.07, while $\alpha_{IRA}$ is found to be 1.55. In the case of the left tail, we fitted the absolute values of $S$.
These findings indicate that both users interacting with IRA and those
who do not show a greater presence of Trump supporters with high values of $S$ compared to
Clinton supporters. Moreover, the presence of users with high values of $S$ is more
pronounced when considering users who interact with IRA, as evidenced by the smaller
power-law exponents, resulting in slower decay of the tails.  


To differentiate between strong and weak supporters based on their $S$ values, 
we use the interquartile range (IQR) of $S$, as explained in the Methods \ref{methods3}.
Strong supporters are identified as users with $S$ values above $Q_3+1.5IQR$, where $Q_3$ 
represents the third quartile. Undecided users are those with $S=0$. \textcolor{blue}.
Therefore, we classified users into five distinct classes: strong Clinton supporters,
strong Trump supporters, weak Clinton supporters,  weak Trump supporters, and undecided users.

In the IRA ego network, we find that 8.1\% of the users are strong supporters 
of Trump, 4\% are strong supporters of Clinton, 64.5\% are weak
supporters of Trump, 21.1\% are weak supporters of Clinton, and 
the remaining 2.3\% of the users are categorized as undecided. These percentages 
show a different picture when considering users not interacting with IRA accounts. 
In particular, we find that 4.2\% of the users are strong supporters 
of Trump, 7.7\% are strong supporters of Clinton, 22.8\% are weak
supporters of Trump, 57.9\% are weak supporters of Clinton, and 
the remaining 7.4\% of the users are classified as undecided.

Figure~\ref{fig:9} illustrates the distribution of the verified, not verified, 
not found, suspended, and IRA users among the supporting classes, as defined earlier.
On the left side (Fig.~\ref{fig:9}a) are the users who interact with IRA accounts, 
while on the right side, Fig.~\ref{fig:9}b, are the users who do 
not interact with IRA accounts. The color of the links represents the supporting class:
red for weak Trump supporters, dark red for strong Trump supporters, 
blue for weak Clinton supporters, dark-blue for strong Clinton supporters,
and cyan for undecided users. The normalization  of the links is based on the number 
of users in each group, that is, the sum of all links
reaching a group at the bottom sum up to 100. Only values above 15\% are displayed 
in the plots. The values for each individual connection can be found in Tables~\ref{table:five} 
and \ref{table:six}. At the bottom of each group, we display 
its fraction in relation to the total number of users.

Among the users interacting with IRA accounts, approximately 37\%
are either suspended or not found. Approximately 58\% have not verified accounts, 
while the remaining users have verified accounts. On the other hand, considering
the users who do not interact with IRA accounts, the percentage of 
suspended and not found accounts for 25\% of the users. This is compensated 
by a larger proportion of not verified accounts, which account for 73.8\% of 
all the users not interacting with IRA.

The majority of not found, not verified, and suspended accounts interacting 
with IRA are classified as weak Trump supporters. Interestingly, the 
majority of IRA accounts are classified as weak Clinton supporters. 
Additionally, the majority of verified accounts are also classified as weak Clinton supporters. 
Considering the significant role played by the IRA in spreading information within
the fake/right categories, it is not surprising that more than 70\% of the users 
interacting with IRA accounts are classified as Trump supporters, whether weak
or strong. However, the presence of a substantial number of IRA accounts 
classified as weak Clinton supporters suggests a dual strategy 
employed by the IRA in their campaign. One aspect of this strategy involves 
reinforcing the opinions of users classified as strong Trump supporters.
On the other hand, another set of IRA accounts aims to expand their reach
within left-leaning accounts by mentioning verified accounts classified 
as weak and strong Trump supporters.

\subsection*{Community structure}

Starting from the four interaction networks of Table~\ref{table:four},
we build an aggregated network (referred to as IRA ego network or IRA aggregated ego network, interchangeably) by considering all types of interaction and 
removing self-loops. The resulting structure is a directed weighted network 
of 179,783 nodes (524 of which are IRA accounts) and  432,429 edges (see 
Table~\ref{table:four}). An edge connecting node $u$ with node $v$ 
means there was at least one type of interaction between them. The edge is 
weighted by the number of interactions among $u$ and $v$. The directions of
the links are according to the flow of information. 

Upon examining the presence of the top 100 influencers from each
news category defined in Section \ref{Section1} within the IRA ego network, we make
the following observations: 98 out of 100 influencers associated with fake content,
99 out of 100 influencers associated with extreme bias right content, 
98 out of 100 influencers associated with right content, 92 out of 100 influencers 
associated with right-leaning content, 95 out of 100 influencers associated with 
center-related content, 98 out of 100 influencers associated with left-leaning content, 
91 out of 100 influencers associated with left content, and 70 out of 100 influencers
associated with extreme biased left content are present in the network. This finding
strongly suggests that IRA accounts actively targeted and engaged with the
most influential users in an effort to amplify their own influence within the platform.
%It suggests a deliberate strategy to target and interact with prominent individuals 
%in order to amplify their messaging and potentially sway public opinion.

We perform multiscale community detection to the largest connected component
of the undirected weighted version of the aggregated network, which contains
99.9\% of the original nodes. To assess community stability, we utilize Markov
stability by examining the variation of information in the resulting partition
for different values of the resolution parameter \cite{delvenne2010stability,
pygenstability}. Figure~\ref{fig:10}a shows the results for 
Markov scale parameters $\log_{10}(t_{min})=-0.75$ and $\log_{10}(t_{max})=2$, 
and with the number of scale steps set to 30 \cite{pygenstability}.

The top panel of Fig.~\ref{fig:10}a shows the value of the 
optimized Generalized Markov stability ($MS$) function $Q_{gen}(H^*(t))$ 
together with the number of communities in the optimal partition.
The middle panel shows the Normalized Variation of Information
(hereafter, $NVI$) measures for the obtained partitions: $NVI(t)$ and
$NVI(t, t0)$ across scales. The bottom panel shows the automated scale
selection criterion, with basins corresponding to blocks in $NVI(t, t0)$ 
and robust scales identified as local minima of $NVI(t)$ within
each basin (purple dots).

The purple dots in Fig.\ref{fig:10}a indicate the 5 optimal partitions 
generated by the method, ranging from the finest 
with more communities to the coarsest with fewer communities. To determine
whether any of these partitions results in a community structure that 
distinguishes between right and left-related ideology, we examine the overlap 
of the community in each partition with the top 100 influencers for
each news category, as defined in Section \ref{Section1}. Table~\ref{table:seven} 
summarizes the characteristics of the communities with at least 10\% of
the nodes in the aggregated network. As indicated, optimal partition n°3, 
corresponding to the red circled point at the bottom of
Fig.~\ref{fig:10}a, unveils two prominent communities, namely, 
community 1 is predominantly composed of right-related influencers
(and thus, we will refer to it as the community right),
and community 2 primarily consisting of left-related influencers 
(and thus, we will refer to it as the community left). 
This partition is particularly intriguing as it supports the hypothesis 
that the IRA operated two separate micro-campaigns, one targeting the 
right and the other targeting the left. Indeed these communities reflect 
the polarization of users. In community right, 76.6\% of users are classified
as weak Trump supporters and 9.5\% as strong Trump supporters,
while community left has 70.2\% of weak Clinton supporters users and 13\% of
strong Clinton supporters users (see Table~\ref{table:eight}).

Figure~\ref{fig:11}a depicts the hashtags cloud associated with 
community right, which suggests that the community has a strong affinity 
toward Donald Trump, as evidenced by the prevalence of hashtags such as 
\#neverHillary, \#maga, \#trump, and \#benghazi. The \#benghazi hashtag is
significant because it refers to the Benghazi attacks, for which Republicans
have consistently blamed then-Secretary of State Hillary Clinton.
The word cloud in Fig.~\ref{fig:11}b corresponds to the community
left and reveals a strong preference for Hillary Clinton. This is evident
from the prominent presence of hashtags such as \#nevertrump, \#notmypresident,
\#imwithher, and \#hillaryclinton. Additionally, the word cloud includes hashtags
associated with the black community, such as \#blacklivesmatter and \#blacktwitter. 
While the Black Lives Matter movement does not align itself with any particular
political party, its activities and objectives frequently intersected with political
issues and policies that were in opposition to the ideologies espoused by Trump.
This observation aligns with other research findings that indicate the Internet
Research Agency (IRA) specifically targeted black communities \cite{howard2018ira}.

Another interesting difference between the two communities is their
account composition. In community right, more than 40\% of the users are either 
not found or suspended. This percentage decreases by half in the case of community left.
On the other hand, community left has a higher proportion of verified 
accounts, with 14.5\% of the nodes being verified, compared to only 1.7\% in
community right. While we cannot make definitive conclusions about not found accounts,
suspended accounts violated Twitter's policies, similar to IRA accounts.
Given the significant interactions between Suspended accounts and IRA accounts,
there appears to be a connection between IRA users and this specific
group of suspended accounts.

Figure~\ref{fig:12}a describes the flow of interactions between 
IRA users and verified, not verified, not found, suspended, and influencers
accounts. With influencer accounts, we refer to the top 100 influencers from 
the categories' networks present in the IRA ego network and introduced in Section
\ref{Section1}. The color of the links reflects the main polarization of the community. 
We used red (Republican) for the interactions in community right and blue (Democrat) 
for those in community left. Light colors indicate that the interactions go 
from IRA to non-IRA, while dark color has the opposite meaning. We normalized the weight of 
the links per community. We show the link weight with a value greater than 10\%.

In community right, around 21.7\%  of the interactions go from IRA to suspended 
accounts, 16.3\% from IRA to not found users, and 39.6\% from IRA to
not verified users. Approximately 5.4\% of interactions go from influencers to IRA, 
the same amount from suspended to IRA, and 7.6\% from not verified to IRA 
(see Table~\ref{table:eightb}). Community left presents with most of the connections
going from non-IRA to IRA users. In particular, 39.6\% of the connections go from 
not verified to IRA, 12.6\% from suspended to IRA, 24.4\% from verified to IRA, 
and 10\% from influencers to IRA (see Table~\ref{table:eightb}).

These results support our hypothesis that the IRA ego network consists
of two main communities: community right, which is right-oriented, and 
community left, which is left-oriented. The direction of the connections
reveals interesting characteristics of information flow within these
communities. In community right, the majority of connections are from 
non-IRA accounts (mostly classified as weak Trump supporters) to IRA users, 
indicating that IRA accounts acted as a source of information for
non-IRA accounts. In contrast, community left, which includes a significant 
number of verified users, exhibits connections predominantly from IRA accounts
to other user groups. It is reasonable to assume that IRA accounts
in this community aimed to penetrate left-related supporters
by mentioning verified users in order to increase their engagement.

Based on the observed patterns, we hypothesize that the group of suspended accounts
strongly interacting with IRA were either directly involved or collaborated with the accounts
associated with the IRA campaign. These accounts may have played a role in amplifying 
the campaign's influence and spreading misinformation. Considering the complex network of
interactions and the similarities in behavior between the IRA accounts and these 
suspended accounts, it is plausible that this undetected group contributed
to the dissemination of misleading information and manipulation of online 
discussions. To fully understand the impact of the IRA campaign and accurately assess 
its influence on the 2016 US presidential elections, it is crucial to consider
the potential involvement of this group of accounts. Failure to account
for their presence and actions could lead to incomplete or misleading conclusions
about the extent and effectiveness of the IRA's activities. 

\subsection*{Expanded ego network}
%By incorporating these suspended accounts into our analysis, we aim to 
%uncover their potential influence and shed light on their contributions
%to the overall dynamics of the campaign. This examination allows for a 
%more comprehensive understanding of the IRA campaign and its impact on 
%social media discourse. 
In the following section, we conduct an analysis that takes into
consideration a crucial aspect often overlooked
in previous studies: the presence of suspended accounts. Specifically, 
we explore the role of those suspended accounts that were involved
or collaborated with the accounts associated with the IRA campaign. To achieve this
goal, we extended the aggregated network from the previous section by incorporating
interactions involving this group of suspended nodes and all
other users. This network, similar to the previous section, was constructed based on
the four types of Twitter interactions. Supplementary Table 18 provides comprehensive
information regarding each interaction network.

It is worth noting that the number of suspended accounts
identified in the previous section (amounting to 30,622), closely related to IRA, is nearly 60 
times larger than the number of IRA accounts. This implies that if our hypothesis of 
collaboration between IRA and these suspended accounts holds true, previous studies
have based their findings on a user set that is 60 times smaller than the actual 
extent. This highlights the significance of considering the larger scope of 
interactions and collaborations when analyzing the impact and influence 
of IRA and its associated accounts. The inclusion of both IRA accounts and suspended accounts 
(referred to as IRA+S) significantly amplifies the influence of the IRA campaign,
increasing the nodes count from 179,783 in the IRA ego network to 1,723,477
in the expanded ego network. This expanded network exhibits 45 times more
connections than the aggregated ego network, with an average degree of $\langle k \rangle = 11$.
Among the 31,146 IRA+S nodes, the average out-degree $\langle k_{out} \rangle$ is 363.8, 
and the average in-degree $\langle k_{in} \rangle$ is 339.2. Similar to the IRA ego network,
retweeting and mentioning interactions remain the most common types of interactions in
the expanded network.

As for the case of the IRA ego network, to test whether non-IRA+S users
are more likely to interact with IRA+S nodes or vice-versa, we perform
a two-sample Kolmogorov-Smirnov test between the $k_{out}$
and $k_{in}$ distributions extracted by each interaction
network. We set a level of 5\%, meaning that
we will reject the null hypothesis and favor the alternative if the
p-value is less than 0.05. The sample sizes for each interaction type were 
as follows: Retweet = 1,394,371, Reply = 357,808, Mention = 468,319, and Quote = 149,797.
As in the IRA ego network, we could not reject $H_0$ for all interaction
types but mentioning and quoting, which in turn means that non-IRA+S users are
more likely to retweet and reply rather than being retweeted or replied to.
On the other hand, non-IRA+S users tend to be mentioned or quoted more than
mentioning and quoting IRA+S nodes. This finding again remarks on the similarity in
behavior between the suspended accounts and the IRA nodes. Furthermore, the significant 
difference in size between the expanded ego network and the IRA ego network (the
latter is 10 times smaller than the former) highlights the underestimated scale 
and impact of the IRA campaign.

Next, we conducted multi-scale community detection analyses and explored different 
parameter values to identify the optimal partition. The most suitable partition was defined 
as the one that maximizes the overlap between two communities in the expanded ego network
and the right and left communities from the IRA aggregated ego network. Among these partitions, 
we selected the one that ensures a clear distinction between the top 100 left and right
influencers within these communities.
By carefully selecting the partition that satisfies these criteria, we aimed to uncover
the expansions of communities right and left  (referred to as E-right and E-left, respectively) 
of the IRA aggregated ego network.

To achieve this, we used scale steps equal to 60 and we varied the Markov scale 
parameters between $\log_{10}(t_{min})=$-0.75 and  $\log_{10}(t_{max})=$1 \cite{pygenstability}.
The selected partition is shown in Fig.~\ref{fig:10}b, with the red circled
dot in the bottom panel representing the chosen optimal partition. In this
configuration, the overlap between the community right of the IRA ego network and 
community E-right is over 90\%, while the overlap between
community left  and community E-left is 85\%. As is shown in Table~\ref{table:nine},
community E-right mainly contained the top 100 influencers associated with the right
categories, while community E-left contained those associated with the left
categories. The  resulting partition also preserves the communities' 
polarization, as shown in Table~\ref{table:nine}, with the two expanded 
communities being mostly composed of supporters of Trump and Clinton. 

However, there are also significant differences between the two communities,
particularly regarding the flow of information between IRA+S nodes and all other nodes.
In community E-right, approximately 30.2\% of the interactions occur from IRA+S to 
non-IRA+S nodes, which is notably lower than the 77.6\% observed in the community right.
Conversely, about 57.3\% of the interactions in community E-right occur 
from non-IRA+S to IRA+S nodes, compared to the 22\% observed in the IRA ego network.
Similarly, in community E-left, there is an almost equal distribution of interactions 
between IRA+S and non-IRA+S nodes, with 48.8\% going from IRA+S to non-IRA+S nodes and 45.8\%
going in the opposite direction (see Table~\ref{table:nine}). These
findings suggest that IRA+S nodes in both communities have a more balanced 
interaction pattern with non-IRA+S nodes compared to the IRA ego network. It implies
that discerning between IRA+S and non-IRA+S users solely based on their connectivity
patterns is not sufficient. 
%While the presence of high out-degree accounts could serve
%as a  distinguishing factor in the IRA ego network, this criterion fails in the current 
%scenario.

Figure~\ref{fig:12}b describes the flow of interactions between IRA+S users
and verified, not verified, not found, suspended, and the top 100 influencers
from the news categories. The color of the links reflects the main 
polarization of the community. We used red (Republican) for the interactions in
community E-right and blue (Democrat) for those in the community E-left. Light colors 
indicate that the interactions go from IRA+S to non-IRA+S nodes, while dark color has
the opposite meaning. Links' normalization is computed per community. We show the links'
weight with a value greater than 10\%.

We notice that influencers and verified accounts rarely engaged with IRA+S nodes
but are often engaged by them. This is consistent with the results obtained for 
the IRA ego network and with the fact that both influencers and verified accounts do
not often engage with regular users. The role of the top 100 influencers in community E-right
changed from the case of the community right, with 31.25\% of interactions going from
the top 100 influencers to IRA+S nodes. This suggests that IRA+S nodes tend to 
interact with the top 100 influencers, probably as a strategy to increase their audience.
In the case of community E-left, this percentage did not suffer a substantial change from the 
case of community left, staying close to a value of 10\% (see Table~\ref{table:ten}). %Upon investigating the account status of the non-IRA+S nodes, 
%we discovered that 14\% of the users have a suspended account, 19.8\% are not found, 
%60.2\% are not verified, and 2.2\% have a verified account.
By examining the 
interaction between IRA+S nodes and suspended accounts within community E-right, 
we observed a significant decrease from approximately 28\% in the community right 
to less than 3\% in the community E-right. Similarly, the interaction 
between IRA+S nodes and suspended accounts in community E-left decreased from around 13\% 
to approximately 6\%. These results suggest that the suspended accounts not 
in the IRA+S group may not have strong connections or coordination 
with the IRA+S accounts, indicating a potential lack of collaboration between
the two groups (see Table~\ref{table:ten}).

Overall, these findings suggest that IRA+S nodes were effectively disguised among 
other users, making it challenging for an investigator to distinguish them solely
based on their activity. The results also shed light on the evolving dynamics 
and potential lack of coordinated behavior between the remaining suspended 
accounts (those not directly interacting with IRA) and the IRA+S nodes. 
This implies that the additional set of suspended accounts identified in the
expanded ego network may not be directly associated with the IRA campaign.

%Gaining a comprehensive understanding of the role that ego nodes played 
%in shaping Twitter discourse, as well as recognizing the potential
%consequences of not considering the groups of suspended accounts closely
%associated with them becomes of utmost importance.

\section{Contrasting causal network patterns: IRA nodes versus IRA+S nodes}
\label{Section4}

The results obtained from the expanded ego network provide evidence of potential 
collaboration between IRA accounts and the group of suspended nodes identified 
in the IRA aggregated ego network. It is plausible that these nodes aided the IRA
accounts in spreading misleading information with the aim of manipulating online
discussions to align with the Trump agenda (in our analysis, we refer to the combined group of 
IRA and this set of suspended nodes as IRA+S). As we showed, by considering the connections 
of these suspended nodes, the size of the IRA ego aggregated network (referred to as the
expanded ego network) increased tenfold, 
indicating that previous studies may have significantly underestimated the
impact of the IRA campaign. To measure the impact of the IRA campaign and compare our findings with
previous studies, we focus on two distinct scenarios. First, 
we examine the causal relationships between the IRA tweet activity and the activity of the 
supporting classes, namely weak Trump supporters, weak Clinton
supporters, strong Trump supporters, strong Clinton supporters, and undecided users.
In the second scenario, we shift our attention to the influence of IRA+S nodes on the same
classes of users. Any differences observed between the two scenarios will shed light on 
the significance of including the group of suspended accounts connected to the IRA, 
in order to accurately evaluate the impact of the IRA campaign.

We employ a multivariate Granger causal network reconstruction approach
to establish links between the activity of IRA (IRA+S) nodes and the supporting 
classes. This is achieved using the causal discovery 
algorithm \cite{spirtes2000causation,runge2012escaping,runge2019detecting}, 
which tests the independence of each pair of time series for several time 
lags conditioned on potential causal parents using a Partial Correlation
Independence test and it removes spurious correlations. We use the algorithm for
variable selection and perform a linear regression using 
only the true causal link discovered. We choose linear causal effects for their 
reliability and interpretability, which allows us to compare causal effects as 
first-order approximations, estimate the uncertainties of the model, and 
construct a causal-directed weighted network \cite{runge2015identifying}.
The causal effect between a time series $X^i$ and $X^j$ at a time delay $\tau$,
$I^{CE}_{i \rightarrow j}(\tau)$, is determined by the expected value of $X^j_t$ 
(in units of standard deviation) if $ x^i (t - \tau)$ is perturbed by one standard 
deviation \cite{runge2015identifying,bovet2019influence}. However, an assumption
of causal discovery is causal sufficiency,
which assumes that every common cause of any two or more variables
is present in the system \cite{spirtes2000causation}. In our case,
causal sufficiency is not satisfied because Twitter's activity is
only a part of a larger social system. Therefore, the term ``causal"
should be understood as relative to the system under study \cite{bovet2019influence}.

We created time series of Twitter activities by counting the number of
tweets posted by each node belonging to one of the supporting classes at a
15-minute resolution. We only consider users that belong to the verified and 
not verified classes, and only consider the tweets coming from official clients.
Instead, for the IRA (IRA+s) nodes, we consider all the tweets, no matter the clients.
To remove trend and circadian cycles from the time series, we utilized
the STL (seasonal trend decomposition procedure based on Loess) method \cite{cleveland1990stl},
which decomposes a time series into seasonal (in this case, daily),
trend, and remainder components. We used the residuals of the STL
filtering of the 15-minute tweet volume time series.

Tables~\ref{table:eleven} and \ref{table:twelve} present the causal relationships
among different groups in the two scenarios: one with only IRA nodes and the other
with IRA+S nodes. The direction of each link is from the column group to the row group.
For example, considering the strong Trump supporters, their causal effect on the weak
Clinton supporters is measured at 0.16 $\pm$ 0.011, as shown in Table~\ref{table:eleven}.
The blue entries in the tables represent the auto-correlation of each time series.
In both scenarios, the auto-correlations exhibit the strongest causal effects for all
time series, except for the undecided group.

To identify the most significant causal links, a threshold of 0.16 (0.20 for the
IRA+S scenario) was set on the causal relation, selecting connections that account
for 75\% of the total effect. These selected links are highlighted in bold in the
tables. Figures~\ref{fig:13}a and \ref{fig:13}b visualize the causal networks 
constructed using these connections. The nodes are colored as follows: dark 
red for strong Trump supporters, dark blue for strong Clinton supporters, light 
red for weak Trump supporters, light blue for weak Clinton supporters, orange 
for the IRA nodes, and gray for the undecided group. Arrows indicate the direction
of maximal causal effect ($\geq$0.16 and $\geq$ 0.20) between two activity time series. The
width of each arrow represents the strength of the causation, and the size of each 
node is proportional to the auto-correlation of each time series.

Figures~\ref{fig:13}a and b, illustrate contrasting scenarios 
in terms of the causal network structure when considering IRA nodes
alone versus IRA+S nodes. In Figure~\ref{fig:13}a, which represents the causal 
network considering IRA nodes only, the influence primarily flows from strong 
supporters of both Trump and Clinton to weak and strong supporters of
opposing political candidates. Additionally, weak supporters from both sides
play a role in influencing the undecided group, with weak Trump supporters
receiving support from strong Trump supporters in their efforts. Notably, IRA 
nodes do not play a significant role in this causal network, suggesting that 
they have limited influence on shaping Twitter discourse. This result aligns with
previous studies \cite{eady2023exposure,bail2020assessing}, which suggest that IRA 
accounts did not have a significant impact on shaping  the Twitter discourse or 
influencing Twitter users.

On the other hand, in Fig.~\ref{fig:13}b, which represents the causal network for IRA+S nodes, 
the structure shows substantial differences. IRA+S nodes take on a central role, acting as a bridge 
between strong Trump supporters and the weak and undecided supporters. Strong Trump supporters have a
causal effect on IRA+S nodes, which, in turn, have a causal influence on both weak supporters and 
the undecided group. Additionally, weak supporters continue to exert a causal effect on the 
undecided group. Interestingly, strong Trump supporters have a causal effect on strong Clinton
supporters, but not vice-versa.

These findings highlight the importance of considering the involvement of suspended 
accounts (IRA+S) in assessing the impact of the IRA campaign. While IRA nodes 
alone may have limited influence, the inclusion of IRA+S nodes reveals a more
intricate network structure and their significant impact on shaping Twitter discourse.

\section{Discussion}
\label{Section5}

By combining the IRA public dataset with a dataset of tweets collected during 
the five months preceding the 2016 presidential elections, we investigated the 
role that IRA played during this event, attempting to identify their strategies 
and to understand whether they influenced on individuals categorized as undecided or weak supporters.

Our analysis reveals that the IRA accounts played a central role in the 
right/fake-related categories while playing a marginal role in the left-related 
categories. This result agrees with current literature \cite{golovchenko2020cross,
linvill2019russians}, which claims that the IRA sought to support Donald Trump and
sow discord among the U.S. public. Further corroborating this conclusion is the fact
that the aggregated IRA ego network mainly consists of Trump supporters, as shown in 
Fig.~\ref{fig:9}. The users with the highest exposure to posts from IRA accounts
were likely to be favorable to Donald Trump and, thus, were less likely to require influencing,
as discussed in Section \ref{Section2}, {\bf Ego polarization}. 
Among these users, we found that there is a higher chance of finding strong Trump supporters,
the highly partisan Republicans, with respect to the case of users non-interacting with 
the IRA, in line with existing literature \cite{eady2023exposure}.

We found that interactions between IRA and non-IRA users mainly occurred by means
of retweeting and mentioning. In particular, we observed that IRA users tended 
to mention non-IRA users, probably associated with an attempt to increase their 
reachability, while non-IRA users tended to retweet IRA users by considering 
them as information sources. In the aggregated IRA ego network, we observed that
around 2\% of the total users were directly exposed to IRA content, consistent with
\cite{eady2023exposure}. A multiscale community detection on this network
unveiled two communities containing almost 90\% of the total nodes, 
reflecting the polarization of users and suggesting that the IRA campaign
consisted of two central cores. The biggest community (community right) aimed 
to support Trump, while the smallest (community left) directly interacted 
with Clinton supporters, probably in an attempt to sway their opinion 
(see Table~\ref{table:eight} and Section \ref{Section2}, {\bf Community structure}).

The presence of a significant number of fake, right-related top 100 influencers
from the news category networks within community right implies that
the information circulating in this community predominantly originates from 
those categories. However, it is noteworthy that the interactions between
the IRA and these influencers in the community represent only 5.4\% of the total
interactions in the community. Community left, on the other hand, shows
a higher share of interactions between the IRA accounts and the top 100 left-related 
influencers, as shown in Table~\ref{table:eight}. This suggests a deliberate effort
by the IRA campaign to establish connections and leverage the reach and influence of 
these influencers within that community. The absence of significant direct interactions
between the IRA accounts and top influencers in community right raises questions about
the strategy employed by the IRA campaign in this specific community. 

We found evidence that a group of suspended accounts by Twitter (10 times bigger than the number 
of IRA accounts), not directly associated with IRA, collaborated with the IRA
campaign, acting as intermediaries or amplifiers of their messages. These undetected accounts
played a crucial role in spreading IRA content and influencing users within community E-right,
while minimizing direct interactions between the IRA accounts and top influencers, 
as shown in Table~\ref{table:nine} and discussed in Section \ref{Section2}, {\bf Expanded ego netwwork}. 
By considering this group of suspended nodes, we found that the 
number of users exposed to IRA+S content increased from 2\% to approximately 16\%. 

While we cannot definitively determine if this group of accounts was an undetected 
part of the IRA campaign or a parallel campaign orchestrated by another entity,
we discovered evidence of significant interaction and similarities between 
these suspended accounts and IRA accounts, as discussed through Section~\ref{Section2}. 
Interestingly, recent insights \cite{theguardian} highlight that social media platforms 
only detect Russian operators of false social media accounts about 1\% of
the time. This detection rate aligns with the ratio we observed between 
IRA accounts and suspended accounts in our analyses, providing further 
support for the presence and influence of these accounts.

By considering the expanded ego network, which included the contact of the IRA+S 
nodes (IRA plus the suspended nodes found in the IRA ego network)
we found that the share of interactions between the IRA+s and the
other suspended accounts represent an insignificant amount of share 
in the overall set of interactions, suggesting that other suspended 
accounts were not directly related to the IRA campaign. Furthermore, 
this decline in the share of interactions with other suspended accounts was
accompanied by a corresponding increase in the interactions between the 
IRA+S accounts and the top influencers, as shown in Fig.~\ref{fig:12}b Table~\ref{table:ten}.

Next, we focus on the main controversial question: Did the IRA campaign
influence the 2016 U.S. presidential elections?  Our causal analyses reveal
that the group of IRA accounts did not have a significant impact in influencing 
the candidate's supporters, as shown in Figure \ref{fig:13}a and detailed in 
\cite{eady2023exposure}. However, the situation becomes more intricate when
we incorporate the suspended accounts closely associated with the IRA campaign, 
referred to as IRA+S. In this scenario, we discovered that IRA+S users wielded a 
substantial influence on individuals categorized as undecided or weak supporters,
potentially with the intention of swaying their opinions. This effect 
is graphically portrayed in Figure \ref{fig:13}b, illustrating the 
bridging effect that IRA+S nodes played between strong Trump supporters
and the group of weak supporters and undecided individuals. The IRA+S nodes
effectively transmit and amplify the messages and influence of 
the strong Trump supporters, as discussed in Section \ref{Section4}.

This comparison demonstrates the distinct influence and impact that
the inclusion of the IRA+S nodes has on the causal network structure, 
highlighting the importance of considering the collaboration between IRA 
and suspended accounts in assessing the influence of the IRA campaign 
on shaping Twitter discourse.

In addition to the specific case study discussed, this work highlights the
importance of not solely relying on information provided by social media
platforms. It emphasizes the need for independent analysis to uncover the
complex dynamics present in the social media landscape. Merely relying on
surface-level information and classifications may not provide a complete
understanding. Instead, a comprehensive investigation into the potential
relationships between collaborative account groups and other users is necessary.
By examining patterns and interactions among these accounts independently,
we can reveal more nuanced insights and gain a deeper understanding of the
intricate dynamics at play. 


\section*{Methods}
\subsection*{Sampled contingency chi-square test}
\label{methods0}
The contingency chi-square test \cite{cressie1984multinomial,lowry2014concepts} 
is a statistical method used to determine if there is a significant 
association between two categorical variables. It involves comparing 
the observed frequency of each combination of categories with the
expected frequency, assuming that the two variables are independent. 

In our case, the two categorical variables are $x \in $ (IRA, not verified, verified, 
suspended, not found) and $y \in $ (Fake, Extreme bias right, Right, Right leaning,
Center, Left, Left leaning, Extreme bias left).
Because IRA accounts generate less traffic than non-IRA accounts, we adopted a sampling 
strategy to compute the contingency chi-square test. We compute the test between each couple 
combination of $x$. We compare the p-value with $\alpha = 0.001$. If the p-value
is smaller than $\alpha$, we reject the null hypothesis; otherwise, $H_0$ holds true.
We remind that the null hypothesis $H_0$ states that there is no relationship or
dependency between $x$ and $y$. 

\subsection*{Two-sample Kolmogorov-Smirnov test}
\label{methods2}

\subsubsection*{Version one-sided}
To test whether non-IRA users are more likely to interact with IRA users or vice-versa,
we perform a two-sample Kolmogorov-Smirnov test between the two distributions
extracted by each interaction network. The null hypothesis, denoted as $H_0$,
assumes that the activity of a given user in each interaction type, 
represented by $x = (x_{out},x_{in})$, follows the condition $F_{out}(x) \geq
F_{in}(x)$ for every $x$. Here, $F_{out}(x)$ and $F_{in}(x)$ represent
the cumulative density functions (CDF) for the ``out'' and ``in'' directions, respectively.
The alternative hypothesis, on the other hand, suggests that $F_{out}(x) < 
F_{in}(x)$ for at least one $x$.

It is worth noting that these hypotheses describe the CDFs of the underlying 
distributions, not the observed data values. For example, suppose $x_{out} 
\sim F_{out}$ and $x_{in} \sim F_{in}$. If $F_{out}(x) > F_{in}(x)$ for
all $x$, the values $x_{out}$ tend  to be less than $x_{in}$. We set a level of 5\%, 
meaning that we will reject the null hypothesis and favor the alternative if the p-value
is less than 0.05. 

\subsubsection*{Version two-sided}
Similarly to the one-sided version, to test for differences in the in/out-degree activity of suspended, not found,
not verified, verified, and IRA accounts, we employed a Two-sample Kolmogorov-Smirnov
test with null hypothesis $H_0$: $F_i(x) = F_j(x)$ 
where $x =  k_{type}^s$. The superscript $s$ indicates that the degree considered comes from the 
sampled nodes, $i,j \in $  (suspended, not found, verified, not verified, IRA), and
$type \in$ (in, out). We set a level of 5\%, 
meaning that we will reject the null hypothesis and favor the alternative if
the p-value is less than 0.05.

\subsection*{Sampling approach for CI ranking}
\label{methods1}
To quantify the influence of Verified, Not Verified, Suspended, and Not Found users
in the retweet news category networks, in comparison to the influence of IRA accounts, 
we employed the following methodology. For each category:
\begin{itemize}
    \item For each group, we randomly select $n$ users.
    \item For each group, we contact the $n$ nodes into a single node incident to 
    any edge that was incident to the original nodes. All edges between the $n$ nodes are
    removed.
    \item We compute the CI$_{out}$ and CI$_{in}$ for the nodes in the newly aggregated
    network.  
    \item We repeat the procedure 100 times.
    \item We average the 100 realizations of CI ranking for each group.
\end{itemize}
We set the value of $n$ to the size of the smallest group in each category, 
which corresponds to the IRA accounts (refer to Table~\ref{table:one}, column $N_{IRA}$).

\subsection*{Supporting classes}
\label{methods3}

To distinguish between strong and weak supporters based on their $S$ values, we utilize 
the interquartile range (IQR) of $S$, defined as $IQR = Q_3 - Q_1$, where $Q_3$ represents
the third quartile and $Q_1$ represents the first quartile. In this analysis, a positive value of
$S$ indicates a likelihood of supporting Trump, while a negative value of $S$ suggests a preference for 
Clinton. The magnitude of $S$ quantifies the degree of support for a particular candidate.
Users who consistently retweet in favor of Trump referred to as strong supporters, exhibit 
higher $S$ values. Conversely, users with significantly negative $S$ values can be associated 
with strong supporters of Clinton. Users with $S=0$ are categorized as undecided since they 
display an equal number of tweets supporting both candidates.

In addition to the undecided category, we define four classes of supporters based on the
interquartile range (IQR) of $S$ values. For Trump supporters, the IQR is calculated over 
the values of $S>0$, while for Clinton supporters, the IQR is computed using the absolute
values of $S<0$. We identify weak Trump (Clinton) supporters as users whose $S$ values 
fall below $Q_3 + 1.5IQR$. On the other hand, 
strong Trump (Clinton) supporters are individuals whose $S$ values exceed $Q_3 + 1.5IQR$. 
This classification scheme allows us to distinguish between different levels of support.

Alternatively, we can consider the entire distribution of $S$ and define strong Trump
supporters as users with $S$ values above $Q_3 + 1.5IQR$. Similarly, strong Clinton 
supporters are those with $S$ values below $Q_1 - 1.5IQR$. Weak Trump supporters fall 
within the range $Q_1 - 1.5IQR \leq S \leq Q_3 + 1.5IQR$ with $S>0$, while weak Clinton
supporters fall within the same range but with $S<0$. This alternative classification,
in the case of users interacting with IRA, results in 12.7\% of users identified as strong
Trump supporters, 60\% as weak Trump supporters, 2.6\% as strong Clinton supporters, and 
22.5\% as weak Clinton supporters. These percentages slightly differ from the ones 
obtained using the other approach mentioned in the main paper. 



\section*{Data and code availability}
The Twitter data are provided according to its terms and are available at \url{https://osf.io/g4hws/} and \url{https://github.com/makselab/IRA-and-suspended-accounts.}.
Analytical codes are available in the same repositories. 

\section*{Author Contributions}
HAM, AB and MS conceived the research; HAM, AB and MS designed and supervised the
research; MS, AB and ZZ coordinated and supervised the analysis; MS and ZZ performed the analyses.
 MS and ZZ, JSA, AB, and HAM analyzed the results; MS and ZZ wrote the first draft. All authors
edited and approved the paper.

\section*{Author Declaration}
All authors declare no competing interests.

\section*{Acknowledgments}
HAM was supported by NSF-HNDS Award 2214217. JSA gratefully acknowledges the Brazilian agencies FUNCAP, CNPq and CAPES, the National Institute of Science and Technology for Complex Systems in Brazil, and the PRONEX-FUNCAP/CNPq Award PR2-0101-00050.01.00/15 for financial support.


\newpage
\clearpage

% Figure environment removed

% Figure environment removed


% Figure environment removed


% Figure environment removed


% Figure environment removed


% Figure environment removed


% Figure environment removed

% Figure environment removed


% Figure environment removed


% Figure environment removed


% Figure environment removed


% Figure environment removed


\clearpage

\begin{table}[!ht]
\center
\resizebox{\textwidth}{!}{\begin{tabular}{|l|r|r|r|r|r|r|l|}\hline
 & \multicolumn{3}{c|}{Full Network} & \multicolumn{3}{c|}{IRA} \\
 &  N  &  E  & $\langle k \rangle /2$ &  $N_{IRA}$ & $\langle k_{out}\rangle$  & $\langle k_{in} \rangle$ \\\hline
Fake               &       175,605 &  1,143,083 &    6.5 &      54 &    32.5 &    5.4 \\
Extreme bias right &       249,659 &  1,637,927 &    6.6 &      70 &    27.3 &    7.1 \\
Right              &       345,644 &  1,797,023 &    5.2 &      84 &    30.1 &    4.7 \\
Right leaning      &       216,026 &   495,307 &    2.3 &      67 &     6.3 &    1.7 \\
Center             &       864,733 &  2,501,037 &    2.9 &     163 &     4.6 &    2.6 \\
Left leaning       &      1,043,436 &  3,570,653 &    3.4 &     140 &     8.9 &    2.2 \\
Left               &       536,903 &  1,801,658 &    3.4 &     105 &     4.3 &    2.4 \\
Extreme bias Left  &        78,911 &   277,483 &    3.5 &      10 &     0.0 &    1.6 \\\hline
\end{tabular}}
\caption{{\bf Retweet categories' networks.} The table contains the characteristics of each of the eight 
retweet networks, such as the number of nodes $N$, the number of edges $E$, and the average degree
$\langle k_{in} \rangle = \langle k_{out} \rangle = \langle k\rangle/2 $. We also report the number
of IRA users in each retweet network $N_{IRA}$, as well as their average in-degree $ \langle k_{in} 
\rangle$ and out-degree $\langle k_{out} \rangle$.}
\label{table:one}
\end{table}

\begin{table}[!ht]
\center
\resizebox{\textwidth}{!}{\begin{tabular}{|l|r|r|r|r|r|r|r|r|r|r|r|l|}\hline
  & \multicolumn{2}{c|}{Suspended} & \multicolumn{2}{c|}{Not Found} & \multicolumn{2}{c|}{Verified} & \multicolumn{2}{c|}{Not Verified} & \multicolumn{2}{c|}{IRA} \\
  & $\langle k_{out}^s \rangle \pm \sigma / \sqrt{N}$  & $\langle k_{in}^s \rangle  \pm \sigma / \sqrt{N}$ & $\langle k_{out}^s\rangle  \pm  \sigma / \sqrt{N}$  & $\langle k_{in}^s \rangle  \pm \sigma / \sqrt{N}$ & $\langle k_{out}^s\rangle  \pm \sigma / \sqrt{N}$  & $\langle k_{in}^s \rangle  \pm \sigma / \sqrt{N}$ 
  & $\langle k_{out}^s\rangle  \pm \sigma / \sqrt{N}$  & $\langle k_{in}^s \rangle  \pm \sigma / \sqrt{N}$ & $\langle k_{out}^s\rangle$  & $\langle k_{in}^2 \rangle$ \\\hline
Fake               &      11.5 $\pm$ 0.79 &    9.6 $\pm$ 0.1 &       2.6 $\pm$ 0.22 &    6.0 $\pm$ 0.06 &    224.5 $\pm$ 8.56 &    1.7 $\pm$ 0.02 &          4.0 $\pm$ 0.86 &    5.5 $\pm$ 0.06 &    32.5 &    5.4 \\
Extreme bias right &       7.3 $\pm$ 0.48 &    8.7 $\pm$ 0.09 &       3.2 $\pm$ 0.23 &    6.5 $\pm$ 0.06 &    260.3 $\pm$ 8.67 &    2.2 $\pm$ 0.02 &          3.9 $\pm$ 0.63 &    5.8 $\pm$ 0.06 &    27.3 &    7.1 \\
Right              &       4.4 $\pm$ 0.28 &    7.4 $\pm$ 0.07 &       1.8 $\pm$ 0.2 &    5.2 $\pm$ 0.05 &    181.5 $\pm$ 7.56 &    2.1 $\pm$ 0.02 &          2.8 $\pm$ 0.22 &    4.6 $\pm$ 0.04 &    30.1 &    4.7 \\
Right leaning      &       1.9 $\pm$ 0.15 &    3.1 $\pm$ 0.02 &       0.7 $\pm$ 0.09 &    2.4 $\pm$ 0.01 &     49.2 $\pm$ 2.56 &    1.3 $\pm$ 0.01 &          0.9 $\pm$ 0.08 &    2.1 $\pm$ 0.01 &     6.3 &    1.7 \\
Center             &       1.3 $\pm$ 0.08 &    3.4 $\pm$ 0.02 &       0.7 $\pm$ 0.04 &    2.9 $\pm$ 0.02 &     81.1 $\pm$ 5.78 &    2.5 $\pm$ 0.01 &          0.7 $\pm$ 0.14 &    2.8 $\pm$ 0.02 &     4.6 &    2.6 \\
Left-leaning       &       1.2 $\pm$ 0.07 &    3.7 $\pm$ 0.03 &       0.9 $\pm$ 0.06 &    3.2 $\pm$ 0.03 &     84.0 $\pm$ 4.16 &    3.3 $\pm$ 0.02 &          1.2 $\pm$ 0.42 &    3.4 $\pm$ 0.03 &     8.9 &    2.2 \\
Left               &       1.6 $\pm$ 0.1 &    3.4 $\pm$ 0.04 &       1.2 $\pm$ 0.1 &    3.0 $\pm$ 0.03 &     67.1 $\pm$ 3.03 &    2.1 $\pm$ 0.02 &          2.4 $\pm$ 0.1 &    3.4 $\pm$ 0.04 &     4.3 &    2.4 \\
Extreme bias Left  &       1.5 $\pm$ 0.22 &    2.6 $\pm$ 0.08 &       3.4 $\pm$ 0.44 &    3.1 $\pm$ 0.07 &    95.0 $\pm$ 10.5 &    2.0 $\pm$ 0.2 &          3.6 $\pm$ 0.22 &    3.9 $\pm$ 0.09 &     0.0 &    1.6 \\\hline
\end{tabular}}
\caption{{\bf Sampled average degrees.} The table displays the average in and out-degree and the standard error per account type and category over $N=$1000 realizations. For each realization, we extract the same amount of IRA users belonging to that category and group (see Table~\ref{table:one}).
As such, IRA accounts do not have any standard error associated with them. The standard error is computed as the standard deviation over the 
$N$ realizations normalized over the square root of the number of realizations. }
\label{table:two}
\end{table}

\begin{table}[!ht]
\center
\resizebox{\textwidth}{!}{\begin{tabular}{|l|r|r|r|r|r|r|r|r|r|r|l|}\hline
{} & \multicolumn{5}{c|}{$\langle CI_{in}\rangle$} & \multicolumn{5}{c|}{$\langle CI_{out}\rangle$} \\
{} &   IRA & Suspended & Not Found & Verified & Not Verified &    IRA & Suspended & Not Found & Verified & Not Verified \\\hline
Fake               &   267 &        85 &       449 &    15,216 &          597 &    140 &      1,284 &      4,140 &       93 &         4,215 \\
Extreme bias right &   104 &        98 &       428 &    11,076 &          667 &    156 &      1,970 &      3,830 &       38 &         4,258 \\
Right              &    87 &        42 &       282 &     6,349 &          420 &    103 &      1,988 &      3,983 &       47 &         3,841 \\
Right leaning      &   101 &         3 &        14 &      345 &           22 &    206 &      2,226 &      4,156 &       57 &         3,591 \\
Center             &   196 &        72 &       182 &      699 &          202 &    440 &      4,501 &      6,367 &       45 &         6,329 \\
Left leaning       &  1,693 &       539 &       799 &     1,463 &          860 &    534 &      6,973 &      9,910 &       92 &         9,559 \\
Left               &  3,088 &       895 &      1,415 &     4,800 &         1,070 &   1,058 &      4,904 &      7,439 &      124 &         4,718 \\
Extreme bias Left  &  8,458 &     12,659 &      6,882 &    20,845 &         6,455 &  55,814 &      9,087 &      8,023 &     3,045 &         7,957 \\\hline
\end{tabular}}
\caption{{\bf Average CI ranking.} The table displays the average CI ranking of each group of users in each category, obtained as explained in Methods \ref{methods1}. Verified users consistently occupy the top position as the most influential group in each news media category. Following verified users, the IRA accounts hold the second position, with
rankings above 100 in the Fake, Extreme bias right, and Right categories. On the other hand, suspended accounts have a significant presence as super sinks in the Fake, Extreme
bias right, Right, Right-leaning, and Center categories. Following closely behind suspended accounts, we find the IRA accounts occupying a similar role. }
\label{table:three}
\end{table}

\begin{table}[!ht]
\center
\resizebox{\textwidth}{!}{\begin{tabular}{|l|c|c|c|c|c|l|}\hline
& \multicolumn{3}{c|}{Full networks} & \multicolumn{3}{c|}{IRA users}\\
           & N & E & $ \langle k \rangle$   & $N_{IRA}$ & $ \langle k_{out} \rangle$ & $\langle k_{in} \rangle$\\\hline
Retweeting &  154,366  &   360,265  &  2.3 &   497  &   468.4  & 262.0 \\
Mentioning &  70,926   &   197,644  &  2.8 &   508  &   353.1  & 41.7 \\
Replying   &  14,225   &   16,775   &  1.2 &   193  &   71.2   & 16.4 \\
Quoting    &  19,195   &   31,538   &  1.6 &   353  &   36.4   & 53.6 \\ \hline
Aggregated &  179,783  &   432,429  &  2.4 &   524  &   486.8  & 343.8\\\hline 
\end{tabular}}
\caption{{\bf Interactions ego networks.} The table contains information about each type of interaction network, as well as 
information about their aggregated version. We report the number of nodes $N$, edges $E$, the average degree $<k>$, 
and the number of IRA nodes $N_{IRA}$, with their in/out-degree. Retweeting and mentioning are the two most frequent types
of interactions between IRA and non-IRA users.  }
\label{table:four}
\end{table}

\begin{table}[!ht]
\center
\resizebox{\textwidth}{!}{
\begin{tabular}{|l|r|r|r|r|r|l|}\hline
 &    Strong Trump supporters  &  Weak Trump supporters &  Strong Clinton supporters & Weak Clinton supporters  &   Undecided \\\hline
Not Found    &   8.13 &  73.44 &   2.17 &  14.49 &  1.76 \\
Not Verified &   6.66 &  63.79 &   4.20 &  22.94 &  2.41 \\
Suspended    &  14.43 &  67.93 &   3.56 &  11.78 &  2.30 \\
Verified     &   0.25 &  15.84 &  11.63 &  68.95 &  3.33 \\
IRA          &   0.32 &  34.62 &   0.00 &  60.26 &  4.81 \\\hline
\end{tabular}}
\caption{{\bf Sankey connection for IRA ego network.} The table presents the distribution
of different user types, namely not found, not verified, suspended, verified, and IRA users,
among the various supporting classes. The percentages in the table are normalized per account type,
meaning that the sum of percentages of a given account type for each supporting class adds up to 100\%. These percentages are utilized as link weights in Fig.~\ref{fig:9}a.}
\label{table:five}
\end{table}


\begin{table}[!ht]
\center
\resizebox{\textwidth}{!}{
\begin{tabular}{|l|r|r|r|r|r|l|}\hline
  &   Strong Trump supporters &     Weak Trump supporters &  Strong Clinton supporters &     Weak Clinton supporters &   Undecided \\\hline
Not Found    &  5.12 &  23.69 &   7.10 &  56.53 &  7.56 \\
Not Verified &  3.56 &  22.15 &   7.73 &  59.37 &  7.18 \\
Suspended    &  7.91 &  27.71 &   7.02 &  47.66 &  9.70 \\
Verified     &  1.44 &  12.63 &  23.58 &  55.94 &  6.41 \\\hline
\end{tabular}}
\caption{{\bf Sankey connection for nodes not interacting with IRA.} The table presents the distribution
of different user types, namely not found, not verified, suspended, verified, and IRA users,
among the various supporting classes. The percentages in the table are normalized per account type,
meaning that the sum of percentages of a given account type for each supporting class adds up to 100\%. These percentages are utilized as link weights in Fig.~\ref{fig:9}b.}
\label{table:six}
\end{table}

\begin{table}[!ht]
\center
\resizebox{\textwidth}{!}{\begin{tabular}{|l|r|r|r|r|r|r|r|r|r|r|r|r|l|}\hline
   $\log_{10}{(t)}$ &  Optimal partition n° &  community &      N & Extreme bias right &  Fake &  Right &  Center &  Left leaning &  Left &  Right leaning &  Extreme bias Left  \\\hline
  0.10 &    1 &    1 & 109,676 &                   0 &     0 &      1 &       0 &             0 &     0 &              0 &                  4 \\
  0.10 &    1 &    2 &  22,281 &                   1 &     1 &      1 &      10 &            38 &    65 &             74 &                 53 \\
  0.48 &   2 &    1 & 119,761 &                   2 &     1 &      1 &       0 &             0 &     0 &              0 &                  4 \\
  0.48 &   2 &    2 &  31,017 &                   1 &     1 &      1 &      10 &            38 &    66 &             76 &                 53 \\
 \rowcolor{LightCyan} 
  1.05 &   3 &    1 & 138,685 &                  97 &    97 &     96 &      79 &            53 &    27 &             11 &                 13 \\
 \rowcolor{LightCyan} 
  1.05 &   3 &    2 &  31,734 &                   1 &     2 &      2 &      12 &            42 &    69 &             77 &                 52 \\
  1.52 &   4 &    1 & 170,310 &                  98 &    99 &     98 &      91 &            95 &    96 &             88 &                 65 \\
  1.71 &   5 &    1 & 179,519 &                  98 &    99 &     98 &      92 &            95 &    98 &             91 &                 70 \\\hline
\end{tabular}}
\caption{{\bf Partitions comparison.} The table shows the distribution of the top 100 influencers for
the news categories in the optimal partitions that were obtained by varying the time scale between -075 and 2,
and scale steps set to 30. We found 5 optimal partitions, as shown in Fig.~\ref{fig:10}a. We only show communities in each partition with at least 10\% of the total nodes in the aggregated network. When the number of influencers is zero, it means they are located in smaller communities of the same partition. As indicated by the cyan color, the partition
n°3, corresponding to the red circled point at the bottom of Fig.~\ref{fig:10}a, unveils two prominent communities, namely, community right is predominantly composed of right-related
influencers and community left primarily consisting of left-related influencers. }
\label{table:seven}
\end{table}

\begin{table}[!ht]
\centering
\begin{tabular}{|l|r|r|l|}\hline
  &  Community right & Community left \\\hline
n° nodes                 &  135,846 &   24,318 \\
n° IRA nodes             &     160 &     114 \\\hline
IRA  $\rightarrow$ non-IRA &      77.6 \% &       3.8 \% \\
non-IRA $\rightarrow$ IRA &      22.0 \% &      91.9 \% \\
IRA  $\rightarrow$ IRA     &       0.3 \% &       4.3 \% \\\hline
Undecided             &       1.8 \% &       4.1 \% \\
Strong Trump supporters            &       9.5 \% &       2.1 \% \\
Weak Trump supporters              &      76.6 \% &      10.5 \% \\
Strong Clinton supporters            &       1.8 \% &      13.0 \% \\
Weak Clinton supporters             &      10.3 \% &      70.2 \% \\\hline
Not Found      &      21.2 \%  &      11.6 \%  \\
Not Verified   &      56.6 \%  &      63.8 \%  \\
Suspended      &      20.3 \%  &       9.6 \%  \\
Verified       &       1.7 \%  &      14.5 \%  \\
IRA            &       0.1 \%  &       0.5 \%  \\\hline
\end{tabular}
\caption{{\bf IRA ego network: partition characteristics.} Characteristics of the 
community resulting from the red circled dot in Fig.~\ref{fig:10}a.
We display information for the communities with at least 10\% of the nodes of the overall
network. For each community, we report the number of nodes, the number of IRA accounts,
the share of supporting classes, the fraction of connection that goes from IRA to non-IRA,
from non-IRA to IRA, and between IRA, and the distribution of users among different groups.
In community right, 76.6\% of users are classified as weak Trump supporters and 9.5\% as strong 
Trump supporters, while community left has 70.2\% of weak Clinton supporters users and 13\% 
of strong Clinton supporters users. Another interesting difference between the two communities is their account com-
position. In community right, more than 40\% of the users are either not found or
suspended. This percentage decreases by half in the case of community left. On the
other hand, community left has a higher proportion of verified accounts, with 14.5\% of
the nodes being verified, compared to only 1.7\% in community right.}
\label{table:eight}
\end{table}

\begin{table}[!!ht]
\centering
\begin{tabular}{|l|r|r|l|}\hline
&    Community right (\%) &  Community left  (\%)\\\hline
$\rightarrow$Suspended &   21.74 &   0.39 \\
Suspended$\rightarrow$ &   5.37 &  12.55 \\
$\rightarrow$Not Found &  16.13 &   0.69 \\
Not Found$\rightarrow$ &   1.60 &   5.06 \\
$\rightarrow$Not Verified &   39.59 &   2.70 \\
Not Verified$\rightarrow$ &   7.59 &  39.58 \\
$\rightarrow$Verified &   0.04 &   0.01 \\
Verified$\rightarrow$ &  2.09 &  24.44 \\
$\rightarrow$Influencers & 0.11 &   0.00 \\
Influencers$\rightarrow$ & 5.40 &  10.25 \\
IRA-IRA & 0.34 &   4.32 \\\hline
\end{tabular}
\caption{{\bf IRA ego network: Interaction between IRA accounts and other users.}
Interaction (in \%) between IRA accounts and other groups of users. Breakdown by accounts type.
$\rightarrow$X indicated that the X-type account interacted with IRA while X$\rightarrow$
has an opposite meaning. Values are normalized over the total weighted edges in each community.
In the community right, around 21.7\% of the interactions go from IRA to suspended
accounts, 16.3\% from IRA to not found users, and 39.6\% from IRA to not verified users.
Approximately 5.4\% of interactions go from influencers to IRA, the same amount from
suspended to IRA, and 7.6\% from not verified to IRA. Community left
presents with most of the connections going from non-IRA to IRA users. In particular,
39.6\% of the connections go from not verified to IRA, 12.6\% from suspended to IRA,
24.4\% from verified to IRA, and 10\% from influencers to IRA. These percentages are utilized as link weights in Fig.~\ref{fig:12}a.}
\label{table:eightb}
\end{table}

\begin{table}[!ht]
\centering
\begin{tabular}{|l|r|r|l|}\hline
  &  Community E-right & Community E-left \\\hline
n° nodes                 &  718,825 &   314,559 \\
n° IRA nodes             &    25,565 &     4,359 \\\hline
IRA  $\rightarrow$ non-IRA &        30.2 \% &       48.8 \% \\
non-IRA $\rightarrow$ IRA &        57.3 \% &       45.8 \% \\
IRA  $\rightarrow$ IRA     &        12.5 \% &        5.4 \% \\\hline
Undecided              &       5.4 \% &        4.1 \% \\
Strong Trump supporters            &      10.2 \% &        2.6 \% \\
Weak Trump supporters            &      55.9 \% &       16.0 \% \\
Strong Clinton supporters             &      3.7 \% &       10.6 \% \\
Weak Clinton supporters              &       24.8 \% &       66.6 \% \\\hline
Not Found      &      19.8 \% &       15.9 \% \\
Not Verified   &       60.2 \% &       69.8 \% \\
Suspended      &     14.2 \% &        8.9 \% \\
Verified       &      2.2 \% &        4.0 \% \\
IRA            &        3.6  \% &        1.4 \% \\\hline
Fake               &  100 &   0 \\
Extreme bias right &  100 &   0 \\
Right              &  100 &   0 \\
Right leaning      &   95 &   5 \\
Center             &   67 &  33 \\
Left leaning       &   35 &  63 \\
Left               &   11 &  89 \\
Extreme bias Left  &    1 &  99 \\\hline
\end{tabular}
\caption{{\bf Expanded ego network: partition characteristics.} Characteristics of the 
community resulting from the red circled dot in Fig.~\ref{fig:10}b.
We display information for the communities with at least 10\% of the nodes of the overall
network. For each community, we report the number of nodes, the number of IRA accounts,
the share of supporting classes, the fraction of connection that goes from IRA to non-IRA,
from non-IRA to IRA, and between IRA, and the distribution of users among different groups.
We also show the number of top 100 influencers for each category. Community E-right mainly 
contained the top 100 influencers associated with the right categories, while community E-left 
contained those associated with the left categories. The communities also preserve the 
communities’ polarization, with the two expanded communities being mostly composed 
of supporters of Trump and Clinton. In community E-right, approximately 30.2\% of the interactions 
occur from IRA+S to non-IRA+S nodes, which is notably lower than the 77.6\% observed in the 
community right. Conversely, about 57.3\% of the interactions in community E-right occur from non-IRA+S
to IRA+S nodes, compared to the 22\% observed in the IRA aggregated ego network. Similarly, in
community E-left, there is an almost equal distribution of interactions between IRA+S and
non-IRA+S nodes, with 48.8\% going from IRA+S to non-IRA+S nodes and 45.8\%
going in the opposite direction.}
\label{table:nine}
\end{table}

\begin{table}[!!ht]
\centering
\begin{tabular}{|l|r|r|l|}\hline
&    Community E-right (\%) &  Community E-left  (\%)\\\hline
$\rightarrow$Suspended &     1.23 &   4.10 \\
Suspended$\rightarrow$ &   0.90 &   1.71 \\
$\rightarrow$Not Found & 8.34 &   7.38 \\
Not Found$\rightarrow$ &   4.56 &   4.05 \\
$\rightarrow$Not Verified &   20.44 &  36.84 \\
Not Verified$\rightarrow$ &   12.87 &  20.45 \\
$\rightarrow$Verified &    0.05 &   0.19 \\
Verified$\rightarrow$ &  7.73 &   9.67 \\
$\rightarrow$Influencers &  0.12 &   0.27 \\
Influencers$\rightarrow$ & 31.25 &   9.96 \\
IRA-IRA & 12.51 &   5.39 \\\hline
\end{tabular}
\caption{{\bf Expanded ego network: Interaction between ego nodes and non-ego users.}
Interaction direction in \% between ego users (IRA + Suspended accounts from IRA ego network) 
and other users. The breakdown is per account type and direction of the interaction. $\rightarrow$X indicated that the X-type accounts interacted with ego nodes, while X$\rightarrow$ has an opposite meaning. Values are normalized over the total weighted edges in each community. By examining the interaction between IRA+S
nodes and suspended accounts within community E-right, we observed a significant decrease
from approximately 28\% in the community right to less than 3\% in the
community E-right. Similarly, the interaction between IRA+S nodes and
suspended accounts in community E-left decreased from around 13\% to approximately 6\%. 
These percentages are utilized as link weights in Fig.~\ref{fig:12}b. }
\label{table:ten}
\end{table}


\begin{table}[!ht]
\resizebox{\textwidth}{!}{ 
\begin{tabular}{|l|r|r|r|r|r|r|l|}\hline
      &    IRA &       Weak Trump supporters &        Weak Clinton supporters  &    Strong Trump supporters   &    Strong Clinton supporters  &   Undecided    \\\hline
IRA   &   {\textcolor{blue}{ 0.87  $\pm$ 0.006}} &  0  &  0  &  0  & 0  & 0   \\
Weak Trump supporters    &   0  &   {\textcolor{blue}{0.54  $\pm$  0.0155}} &  0.08  $\pm$ 0.024 &  0.15  $\pm$ 0.015  &  {\bf 0.16  $\pm$ 0.022} &  0.05 $\pm$  0.019  \\
Weak Clinton supporters     &   0  &  0.1  $\pm$  0.021 &   {\textcolor{blue}{0.78  $\pm$ 0.014}} &  {\bf 0.16  $\pm$ 0.011}  &  0.13  $\pm$ 0.02  &  0.09  $\pm$  0.02   \\
Strong Trump supporters    &   0  &  0.01  $\pm$  0.004 &  0.03  $\pm$ 0.015 &   {\textcolor{blue}{0.47  $\pm$  0.011}}  &   {\bf  0.17  $\pm$ 0.015}  &  0.01  $\pm$  0.003  \\
Strong Clinton supporters   &   0  &   0.06  $\pm$ 0.018 &  0.09  $\pm$ 0.013 &  {\bf 0.16  $\pm$  0.012}  &   {\textcolor{blue}{0.75  $\pm$ 0.013}}  &  0.07  $\pm$ 0.016  \\
Undecided     &   0  &   {\bf 0.26  $\pm$ 0.017} &  {\bf0.34  $\pm$ 0.021} &  {\bf 0.16  $\pm$ 0.015}  &  0.14  $\pm$  0.02  &   {\textcolor{blue}{0.2  $\pm$ 0.019}} \\\hline
\end{tabular}}
\caption{{\bf Causal Links for the IRA case.} We show the value of the maximal causal effect,
$I_{i \to j}^{{\mathrm{CE, max}}} = {\mathrm{max}}_{0 < \tau \le \tau _{{\mathrm{max}}}}\left| {I_{i \to j}^{{\mathrm{CE}}}(\tau )} \right|$ between each pair (i, j) of activity time series, 
 where $\tau_{max}= 18 \times 15$min=4.5h is the maximal time 
lag considered, with standard errors. The arrows indicate the 
the direction of the causal effect. For each activity time series,
we indicate in bold the most important drivers of activity (excluding themselves).
In blue, we highlight the auto-correlation of each node. }
\label{table:eleven}
\end{table} 


\begin{table}[!ht]
\resizebox{\textwidth}{!}{ \begin{tabular}{|l|r|r|r|r|r|r|l|}\hline
      &   IRA+S &   Weak Trump supporters &   Weak Clinton supporters &   Strong Trump supporters   &    Strong Clinton supporters  &   Undecided \\\hline
IRA+S  &  {\textcolor{blue}{0.38 $\pm$ 0.012}}  &  0.05 $\pm$ 0.008  &  0.04 $\pm$ 0.012  &  {\bf 0.27 $\pm$ 0.018}  &  0.17 $\pm$ 0.020   &  0.03 $\pm$ 0.007  \\
Weak Trump supporters    &  {\bf 0.23 $\pm$ 0.027}  & { \textcolor{blue}{ 0.52 $\pm$ 0.015 }}  &  0.08 $\pm$ 0.021  &  0.10 $\pm$ 0.027  &  0.19 $\pm$ 0.023   &  0.05 $\pm$ 0.019  \\
Weak Clinton supporters    &   {\bf 0.29 $\pm$ 0.035}  &  0.06 $\pm$ 0.017  & { \textcolor{blue}{0.79 $\pm$ 0.014 }}  &  0.19 $\pm$ 0.037  &  0.12 $\pm$ 0.021   &  0.07 $\pm$ 0.019  \\
Strong Trump supporters      &   0.05 $\pm$ 0.009  &  0.03 $\pm$ 0.011  &  0.02 $\pm$ 0.012  & { \textcolor{blue}{0.50 $\pm$ 0.011 }}  &  0.15 $\pm$ 0.017   &  0.05 $\pm$ 0.012  \\
Strong Clinton supporters      &   {\bf 0.23 $\pm$ 0.030}  &  0.02 $\pm$ 0.010  &  0.04 $\pm$ 0.012  &  {\bf 0.20 $\pm$ 0.032}  & {\textcolor{blue}{ 0.77 $\pm$ 0.012 }}   &  0.04 $\pm$ 0.006  \\
Undecided    &  {\bf 0.24 $\pm$ 0.027}  & {\bf 0.25 $\pm$ 0.015 } &  {\bf 0.34 $\pm$ 0.021}  &  0.10 $\pm$ 0.027  &  0.16 $\pm$ 0.020   &  {\textcolor{blue}{ 0.20 $\pm$ 0.019 }}   \\\hline
\end{tabular}}
\caption{{\bf Causal Links for the IRA+S case.} We show the value of the maximal causal effect,
$I_{i \to j}^{{\mathrm{CE, max}}} = {\mathrm{max}}_{0 < \tau \le \tau _{{\mathrm{max}}}}\left| {I_{i \to j}^{{\mathrm{CE}}}(\tau )} \right|$ between each pair (i, j) of activity time series, 
 where $\tau_{max}= 18 \times 15$min=4.5h is the maximal time 
lag considered, with standard errors. The arrows indicate the 
the direction of the causal effect. For each activity time series,
we indicate in bold the three most important drivers of activity (excluding themselves).
In blue, we highlight the auto-correlation of each node. }
\label{table:twelve}
\end{table} 


\clearpage
\newpage
%\bibliographystyle{naturemag}
%\bibliography{bibliography}
%\bibliographystyle{elsarticle-num}
%\bibliography{bib-elite.bib}
\bibliography{sn-article}% common bib file
%% if required, the content of .bbl file can be included here once bbl is generated
%%\input sn-article.bbl


\end{document}
