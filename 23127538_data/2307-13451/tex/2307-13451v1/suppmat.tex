\documentclass[prl,superscriptaddress,amsmath,amssymb,onecolumn]{revtex4-1}
	\usepackage{natbib}
	\usepackage{lmodern}
	\usepackage{graphicx}% Include figure files
	%\usepackage{cases}
	\usepackage{mathtools}
	\usepackage{amsmath}
	\usepackage{amsfonts}
	\usepackage{amssymb}
	\usepackage{lipsum}
	\usepackage[export]{adjustbox}
	\usepackage{cancel}
	\usepackage{dcolumn}% Align table columns on decimal point
	\usepackage{bm}% bold math
	\usepackage{hyperref}% add hypertext capabilities
	\hypersetup{linktocpage,colorlinks,citecolor={blue},pdfdisplaydoctitle=true,pdfpagemode=UseOutlines,bookmarksnumbered=true}
	\usepackage{mathrsfs,dsfont}
	\usepackage{color}
	\usepackage{wrapfig}
	\usepackage{comment}
	\usepackage[normalem]{ulem}
        \usepackage{xr}
        \makeatletter
        \newcommand*{\addFileDependency}[1]{% argument=file name and extension
        \typeout{(#1)}
        \@addtofilelist{#1}
        \IfFileExists{#1}{}{\typeout{No file #1.}}
        }
        \makeatother
            \newcommand*{\myexternaldocument}[1]{%
            \externaldocument{#1}%
            \addFileDependency{#1.tex}%
           \addFileDependency{#1.aux}%
        }
        \myexternaldocument{main} 
	
	\newcommand{\bra}[1]{\langle #1 |} 
	\newcommand{\ket}[1]{| #1 \rangle } 
	\newcommand{\upd}{\mathrm{d}}
	\newcommand{\tr}{\mathrm{tr}}
	\newcommand{\ie}[0]{\textit{i.e.} }
	\newcommand{\eg}[0]{\textit{e.g.} }
	\newcommand{\xb}[0]{\mathbf{x}}
	\newcommand{\yb}[0]{\mathbf{y}}
	\newcommand{\im}{\mathbf{i}}
	
	\newcommand{\bigO}{\mathcal{O}}
	
	\newcommand\underrel[2]{\mathrel{\mathop{#2}\limits_{#1}}}
	
	
	\definecolor{cbl}{rgb}{0,0,1}
	\newcommand{\cbl}[1]{\textcolor{cbl}{#1}} 
	\definecolor{crd}{rgb}{1,0,0}
	\newcommand{\crd}[1]{\textcolor{crd}{#1}} 
	
	\newcommand{\dr}[1]{[#1]^{\mbox{\tiny (dr)}}}
	\newcommand{\Dr}[1]{[#1]^{\mbox{\tiny \bf(Dr)}}}
	
	
	
	\newcommand{\limtl}{\lim_{\mbox{\tiny T.L.}}}
	
	\newcommand{\ttt}{\boldsymbol{\tau}}
	
	\newcommand{\QQ}{Q}
	
	\newcommand{\qq}{q}
	
	%\renewcommand{\imath}{\mathbbit{i}}
	\renewcommand{\imath}{{i\mkern1mu}}
	
	
	
	
	
	
	\def\mno{\mu}
	\def\qhat{\hat{q}}
	
	\graphicspath{{SuppFigures/}}
	
	
	\def\Pc{P_0}
	\def\Psat{P_{\rm sat}}
	
	\def\tPsat{\tilde{P}_{\rm sat}}
	
	\def\nfull{m^{(\rm full)}}
	\def\nch{n^{\rm ch}}
	\def\nn{m}
	
	\def\BB{\mathcal{B}}
	\def\heaviside{\vartheta}
	
	\def\tbeta{\tilde{\beta}}
	
	
	\def\varmin{\chi_0}
	
	
	\setcounter{equation}{0}
	\setcounter{figure}{0}
	\renewcommand{\thetable}{S\arabic{table}}
	%\renewcommand{\theequation}{S\thesection.\arabic{equation}}
	\renewcommand{\theequation}{\thesection.\arabic{equation}}
	\renewcommand{\thefigure}{S\arabic{figure}}
	\setcounter{secnumdepth}{2}


\begin{document}
\onecolumngrid

\graphicspath{{SuppFigures/}}

\newpage 
	\setcounter{equation}{0}
	\begin{center}
		{\Large Supplementary Material \\
			Dynamic pore-network modeling of transition from viscous fingers to compact displacement during drainage
		}
	\end{center}
	%\tableofcontents
	%\begin{itemize}
	%\end{itemize}

\section{Dependence of the invasion pattern from the parameters}
In our work, we assumed that the viscosity ratio $M = \mu_N/\mu_W$ and the capillary number $\text{Ca}_P = (\Delta P/L_x) / (2\gamma/\overline{r})$ fully characterize the displacement pattern occurring in our system. In this Section of the Supplementary Material, we want to verify this assumption, showing that different simulations prepared setting the same values for $M$ and $\text{Ca}_P$ will return the same pattern. For each of the invasion patterns shown in Figure \ref{fig:varying_gamma}, different values for both the surface tension $\gamma$ and the global pressure drop $\Delta P$ were chosen, but such that $\text{Ca}_P$ is the same \footnote{The unit of measure of the physical quantities, not reported here, can be chosen arbitrarily, as long as the choice remain consistent with the fact that the global parameters introduced must be dimensionless.}. The patterns are very similar to each other, varying only by small details at the scale of few pores. In particular, the foam originates and propagates from the same transition height $\Lambda$. Analogously, Figure \ref{fig:varying_r} show three different patterns for which both the average radius $\langle r \rangle$ and $\Delta P/L_x$ are varied. Also in this case, identical values of $\text{Ca}_P$ leads to similar patterns. An analogue test for the viscosity ratio $M$ was done. In Figure \ref{fig:varying_mu} it is shown that a different choice for the wetting and non-wetting viscosities, $\mu_w$ and $\mu_n$, will not modify the macroscopic output, as long as their ratio is kept equal.

We also investigate the dependence of the invasion pattern from the width of the radii distribution. For different simulations, we generate the tubes radii from a uniform distribution
\begin{equation}
    \Pi(r) =\begin{dcases} 1/a & \text{if}\; r\in \left[ \overline{r} - a/2, \overline{r} + a/2 \right]\\
0 & \text{otherwise}
\end{dcases}\!,
\label{eq:Pi_uniform_supp}
\end{equation}
having the same $\overline{r}$ but different width $a$. The outputs are presented in Figure \ref{fig:varying_a}, where we observe $\Lambda$ to remain the same, while the number of finger and the compactness of the foam seem both to decrease as $a$ rises. A further analysis might characterize and quantify this relationship.

Finally, to look at the influence of the network size, in Figure \ref{fig:varying_L} we show some patterns obtained varying only the number of tubes per side, $N_x = N_y \equiv N$ in a square network. We observe the finger-foam transition to occur at approximately the same distance $\Lambda$ from the inlet, so the foam just propagates for longer distances when $N$ is bigger.

%Unless specified otherwise, in our numerical simulations we arbitrary set the link length, the surface tension and the viscosity of the wetting phase to, respectively, $l=1$, $\gamma = 25$ and $\mu_w = 10^{-3}$. The unit of measure of each of these three physical quantities can be chosen arbitrarily. Doing this, the unit of measure of every other physical quantity in this work can be obtained as a unique combination of these three ones.
%the which allows to a-dimenisonalize all the other physical quantities.

% Figure environment removed
% Figure environment removed
% Figure environment removed
% Figure environment removed
% Figure environment removed
\end{document}