\section{Experiments}\label{sec:experiments}

In this section we discuss the setup of our algorithms, and their results, on a range of case studies. The lower and upper bounds on stability, obtained through our algorithms as well as other existing algorithms, are summarized in Table \ref{tab:table1}. Table \ref{tab:table2} summarizes the run time of the algorithms on each instance. 

\subsection{Experimental setup}\label{ssec:setup}
All our experiments were run on recent commodity laptop hardware (Macbook Air 2022, M2 processor, 24GB RAM), using standard Python libraries (\texttt{numpy}, \texttt{gurobipy}).
We used the following implementations of the algorithms:
\begin{itemize}
    \item[ZAMinfluence \cite{broderick2020automatic}:] Our own implementation, since \cite{broderick2020automatic}'s is implemented in R. See \texttt{auditor\_tools}.
    \item[Greedy Heuristic \cite{kuschnig2021hidden}:] Our own implementation: we find some numerical instability in \cite{moitra2022provably}'s implementation of the greedy heuristic for ill-conditioned or rank-deficient regressions, arising from the use of \texttt{numpy.linalg.inv} for matrix inversions of ill-conditioned matrices, rather than using pseudoinverses via \texttt{numpy.linalg.pinv}.
    \item[\textsc{PartitionAndApprox, NetApprox} \cite{moitra2022provably}:] Implementation provided by \cite{moitra2022provably}.
    We run these algorithms only when we expect both of them to terminate within $5$ minutes. Because their running time scales exponentially with dimension, on our hardware this typically constrains them to $3$ dimensions or fewer.
    \item[Gurobi:] Our implementation, calling the Gurobi mathematical programming solver via \texttt{gurobipy}. We typically cut off the solver after $< 10$ seconds of solving time. Note that total running time is typically 0-5 minutes; in most cases this is dominated by the time to set up the mathematical program in Gurobi.
    % \item[Exact 2D Binary:] Our implementation.
    % \item[Exact Difference-in-Differences:] Our implementation.
    % \item[Spectral:] Our implementation.
\end{itemize}

For the Exact 2D Binary, the Exact Difference-in-Differences, and the Spectral algorithms we rely on our own implementation.

\subsection{Results}

\begin{table}[h!]
  \begin{center}
    \caption{Table of lower and upper bounds achieved by each algorithm. Cells left empty correspond to no nontrivial bound having been identified by the algorithm, whereas a dash (--) corresponds to the algorithm not being applicable to a given setting using a reasonable amount of time (e.g., MR22 exceeds our running time limits in high dimensions such as the study by \cite{eubank2022enfranchisement} and our exact algorithms only apply to particular instances). In the right-most column, $n$ denotes the number of samples and $d$ denotes the dimension of the samples, including intercept. (I.e. regression to find a slope and intercept with a single treatment variable has $d=2$.)}
    %\Dnote{Update these numbers!!}
    \label{tab:table1}
    \begin{tabular}{|l|c|c|r r||r r|c|c|}
    \hline
    &
    \multicolumn{1}{c|}{\cite{broderick2020automatic}}
    & \multicolumn{1}{c|}{
    \cite{kuschnig2021hidden}}
    & \multicolumn{2}{c||}{\cite{moitra2022provably}}
    & \multicolumn{2}{c|}{Gurobi}
    & \multicolumn{1}{c|}{Exact}
    & {Spectral}\\
    \hline
    Study/Instance (n,d) & UB  & UB &
    LB & UB &
    LB & UB &
     LB$=$UB&
    LB \\
    \hline
    Bosnia (1195,2) \cite{augsburg2015impacts} & 14 & 13 &
     & 14.8 &
    13 & 13 &
    13 & 3
     \\
    Ethiopia (3113,2) \cite{tarozzi2015impacts} & 1  & 1 &
     & 2 &
    1 & 1 &
    1 &
     \\
    India (6863,2) \cite{banerjee2015miracle} & 6  & 6 &
    4.6 & 5.7 &
    6 & 6 &
    6 & 2
     \\
    Mexico (16560,2) \cite{angelucci2015microcredit} & 1  & 1 &
     & 356 &
    1 & 1 &
    1 &
     \\
    Mongolia (961,2) \cite{attanasio2015impacts} & 16  & 15 &
    13.4 & 19.8 &
    15 & 15 &
    15 & 2
     \\
    Morocco (5498,2) \cite{crepon2015estimating} & 11  & 11 &
    10.4 & 10.5 &
    11 & 11 &
    11 & 2
     \\
    Philippines (1113,2) \cite{karlan2011microcredit} & 9  & 9 &
    7.8 & 9.9 &
    9 & 9 &
    9 &
     \\ \hline
    Min wage (384$\times$2,4) \cite{card1993minimum} & --
     & -- & --
     & -- &
      6 & 10 & 10
     & --
      \\ \hline
    Incarceration (504,48) \cite{eubank2022enfranchisement} & 33
     & 29 & --
     & -- & 
     & 28 & --
     & 
      \\ \hline
    GDP (3895,211)  \cite{martinez2022much}& 136
     & 110 & --
     & -- &  
     & 110  & --
     & 
      \\ \hline
      Synthetic 2D (100,2) & 
     & 63 & 60.2
     & 63 & 63
     & 63  & --
     & 19.5
      \\
      Synthetic 4D (1000,4) & 922 
     & 409 & --
     & 452 & 
     & 409  & --
     & 102
      \\ \hline
    \end{tabular}
  \end{center}
\end{table}

The first seven rows of Table \ref{tab:table1} consider the microcredit studies; as these are based on $X_i\in\{0,1\}$, our Algorithm \ref{alg:binary} obtains optimal results for these. Gurobi also finds optimal results, both for the fractional weights studied by \cite{moitra2022provably} and the integral weights we focus on. In two of the seven settings our results find a smaller set to flip the sign than that identified by ZAMinfluence \cite{broderick2020automatic}. In the other five, our results certify that their upper bound is indeed optimal. The reuslts of \cite{moitra2022provably} on this data do not provide comparably strong bounds, despite taking significantly longer to run, as displayed in the first row of Table \ref{tab:table2}. They only find nontrivial lower bounds on some of the instances and their upper bounds are often far weaker than those identified by ZAMinfluence \cite{broderick2020automatic}. One some runs, their algorithm identifies strong bounds that seemingly contradict the optimal exact solution (e.g., for India); this  reflects the difference in optimization problems, since \cite{moitra2022provably} solves the fractional problem, in which an objective of 8.2 is feasible (as identified by Gurobi, which solves the fractional instance to optimality); ZAMinfluence considers the integral version, for which 9 is the optimal solution (as certified by both Gurobi and our exact algorithm). The spectral algorithm obtains weak lower bounds for only some of these instances.

The next row, Minimum Wage, considers the difference-in-difference setting from Section \ref{ssec:diffs}. Here, we focus on a textbook example of difference-in-difference estimation, specifically \cite{card1993minimum}. We did not implement variants of ZAMinfluence or Moitra and Rohatgi's algorithms for this version of the problem, in which observations have to be dropped in pairs. However, we highlight that Gurobi cannot solve this instance to optimality with a 30-minute time limit, whereas our exact algorithm solves it in less than a second. 

Next, we consider the settings studied in \cite{eubank2022enfranchisement} and \cite{martinez2022much} with Zaminfluence \cite{broderick2020automatic}. Both of these settings are too high-dimensional for our exact algorithms to apply, or those of \cite{moitra2022provably} to converge in reasonable time, yet in both cases Gurobi (and our implementations of ZAMinfluence and the Greedy heuristic) finds significantly smaller subsets than those reported by the respective authors (28 compared to 97 and 2.8\% compared to 5.1\% --- we speculate that the large gap between our influence-based algorithms and those previously used arise from improved numerical stability in our implementations).

Finally, we consider two synthetic datasets.
The Synthetic 2D dataset consists of 100 samples $(X_i,Y_i)$, where $X_i \sim \mathcal{N}(0,1)$ and $Y_i = -2X_i + \epsilon_i$, with $\epsilon_i \sim \mathcal{N}(0,1)$; we consider the regression model $Y = X\beta + \alpha + \epsilon$; i.e. allowing for a fixed-effects/intercept term.
The Synthetic 4D dataset consists of 1000 samples $(X_i,Y_i)$ where $X_i \in \R^4$ has iid coordinates from $\mathcal{N}(0,1)$, and $Y_i = X_i(1) + X_i(2) + X_i(3) + X_i(4) + \epsilon_i$.
We consider the linear model $Y = X(1) \beta_1 + X(2) \beta_2 + X(3) \beta_3 + X(4) \beta_4 + \epsilon$, i.e. without a fixed-effects/intercept term.
The spectral algorithm only produces lower bounds on stability; it is the only algorithm among those we study to produce nontrivial lower bounds for the Synthetic 4D dataset, but performs comparatively poorly on the other datasets.
Gurobi is run with a 60 second cutoff on the synthetic datasets -- 30 seconds allotted to fractional solving, 30 to integer solving.
(Overall runtime is greater than 60 seconds because of the time required for Gurobi to set up the model.)


\begin{table}[h!]
  \begin{center}
    \caption{Algorithmic runtimes in seconds (rounded to the nearest integer and, in most nontrivial cases based on algorithmic parameters). Note that running time for Microcredit studies includes time to solve all 7 studies.}
    \label{tab:table2}
    \begin{tabular}{|l|c|c|r || r|c|c|}
    \hline
    Study/Instance &
    \multicolumn{1}{c|}{\cite{broderick2020automatic}}
    & \multicolumn{1}{c|}{
    \cite{kuschnig2021hidden}}
    & \multicolumn{1}{c||}{\cite{moitra2022provably}}
    & \multicolumn{1}{c|}{Gurobi}
    & \multicolumn{1}{c|}{Exact}
    & {Spectral}\\
    \hline
    Microcredit studies  & 5 & 8 & 3640
     & 
    151 & 0 & 0
     \\ \hline
    Min wage \cite{card1993minimum} & --
     & -- & --
     & 1950 &
     0 & --
     
      \\ \hline
    Incarceration \cite{eubank2022enfranchisement} & 0
     & 1 & --
     &  9 & --
     & 0
      \\ \hline
    GDP \cite{martinez2022much}& 50
     & 122 & --
     & 243 & --
     & 0
      \\ \hline
      Synthetic 2D & 0 
     & 0 & 25
     & 0  & --
     & 0
      \\
      Synthetic 4D & 2 
     & 7 &
    40 (no LB) & 61 & --
     & 0 
      \\ \hline
    \end{tabular}
  \end{center}
\end{table}
 

\paragraph{Boston Housing Data. }
As discussed above, Moitra and Rohatgi evaluate their algorithms on the well-known Boston housing dataset \cite{moitra2022provably}. For the 156 instances they consider, we display the results (lower and upper bounds) in Figure \ref{fig:bh}. To ensure a fair comparison, we set the parameters affecting the runtime for Gurobi and the \cite{moitra2022provably} algorithms so that they run in approximately the same time; in particular, for all these instances combined Gurobi took about 8 minutes whereas the \cite{moitra2022provably} algorithms took about 12 minutes. 

We first compare the upper bounds, comparing ours with the the ones identified by the Net-algorithm in \cite{moitra2022provably} and the ones identified through ZAMinfluence with resolving \cite{broderick2020automatic,kuschnig2021hidden}, as implemented by \cite{moitra2022provably}. Here we find that the Net algorithm of \cite{moitra2022provably} usually identifies similarly strong upper bounds to Gurobi. Though Gurobi identifies tighter upper bounds on about 99\% of instances, the difference is smaller than 5 on 99\% of instances. In contrast, ZAMinfluence (with or without resolving with either implementation) never identifies a tighter upper bound than Gurobi and is off by at least 20 on about 20\% of instances. Next, in Figure \ref{fig:bh2} we compare the 
lower bounds identified by \cite{moitra2022provably} and by Gurobi, noticing that Gurobi identifies stronger bounds on 93\% of the instances. Finally, in Figure \ref{fig:bh3} we plot the resulting optimality gaps across all instances. This comparison shows that Gurobi obtains tight bounds (within 1\%) on 92\% of the instances, and obtains a lower bound of at least 35\% of its upper bound on all instances. In contrast, \cite{moitra2022provably} does not obtain tight bounds (within 1\%) on 92\% of the instances and obtains a lower bound of at least 20\% of the upper bound on just 85\% of the instances.

% Figure environment removed



\subsection{Challenge Data}
In our accompanying replication package, we provide \texttt{csv} files for all the datasets above.
Three are designated as \emph{challenge datasets}: Synthetic 4D (\texttt{synthetic4d.csv}), Incarceration (\texttt{Eubank\_black\_perc.csv}), and GDP (\texttt{martinez.csv}).
As described in Table~\ref{tab:table1}, all our methods leave wide gaps between upper and lower bounds on these datasets. In particular, for Incarceration and GDP, we cannot identify any nontrivial lower bounds.
We believe that progress towards closing these gaps requires new algorithms that would constitute substantial steps toward practical  robustness auditing.

