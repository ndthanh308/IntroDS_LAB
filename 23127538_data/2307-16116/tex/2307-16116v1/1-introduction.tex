
\section{Introduction}
In recent years, many augmented reality (AR) sketching tools~\cite{arora2018symbiosissketch, gasques2019pintar, suzuki2020realitysketch, leiva2020pronto} have been explored in human-computer interaction (HCI) research, thanks to the proliferation of augmented and mixed reality (AR/MR) devices.
In contrast to traditional screen-based sketching interfaces such as \textit{SketchPad} ~\cite{sutherland1964sketchpad}, \textit{Pegasus}~\cite{igarashi1998pegasus}, \textit{Draco}~\cite{kazi2014draco}, and many others~\cite{bae2008ilovesketch, gross1996ambiguous, landay1995interactive}, \textit{AR sketching} enables the user to sketch and annotate directly onto a live or recorded real-world scene, opening up untapped opportunities for many applications, such as 3D design and prototyping (e.g., \textit{SymbiosisSketch}~\cite{arora2018symbiosissketch}, \textit{DesignAR}~\cite{reipschlager2019designar}), education (e.g., \textit{Augmented Body}~\cite{ferdous2019s}, \textit{PaperTrail}~\cite{rajaram2022paper}), collaboration (e.g., \textit{Vuforia Chalk}~\cite{vuforia-chalk}, \textit{PintAR}~\cite{gasques2019pintar}), and entertainment (e.g., \textit{Just a Line}~\cite{just-a-line}, \textit{DoodleLens}~\cite{doodlelens}).

While many existing AR sketching tools focus on \textit{static sketching}, in which sketched elements float in the mid-air without being animated, more recent works have explored \textit{dynamic sketching}, in which the sketched elements can animate and respond to real-world interactions~\cite{suzuki2020realitysketch, saquib2022graphiti, saquib2019interactive, kaimoto2022sketched, perlin2018chalktalkvrar}.
Such dynamic AR sketching tools allow more interactive and engaging experiences for various applications, however, these existing tools either require both preparation and pre-defined configuration\removed{ and preparation}, hindering real-time and improvisational exploration (e.g., \textit{Interactive Body-Driven Graphics}~\cite{saquib2019interactive}, \textit{ChalkTalk AR}~\cite{perlin2018chalktalkvrar}\removed{, Pronto~\cite{leiva2020pronto}}) or they do not support freehand drawing except for simple line-based geometries, significantly limiting the flexibility and generalizability of possible animations (e.g., \textit{RealitySketch}~\cite{suzuki2020realitysketch}, \textit{Graphiti}~\cite{saquib2022graphiti}, \textit{Sketched Reality}~\cite{kaimoto2022sketched}).

To address these limitations, we introduce \system{}, a mobile AR sketching tool that allows the user to embed responsive sketched animation to a live or recorded real-world scene through \textit{\textbf{freehand}} and \textbf{\textit{real-time}} sketching interactions (Figure~\ref{fig:teaser}).
The goal of \system{} is to enable more \textit{expressive} yet \textit{improvisational} AR sketched animations that can respond to and interact with the real world (Figure~\ref{fig:related-work}).
\removed{To this end, we introduce \system{}, an AR sketching tool that allows the user to embed responsive sketched animation to a live or recorded real-world scene through \textit{\textbf{freehand}} and \textit{\textbf{improvisational}} sketching interactions (Figure~\ref{fig:teaser}).}
With both expressive and improvisational capabilities, \system{} allows users to spontaneously blend digital sketches and physical motion in a flexible manner, without the need for prior preparation or planning. This enables interactive exploration and experimentation in real time, unlike prior work~\cite{saquib2019interactive, perlin2018chalktalkvrar}.
% the users can freely explore how to blend digital sketches into physical motions in more \textit{flexible} and \textit{spontaneous} ways. 
% about how to 

To design our system, we first collect and analyze 172 existing video-edited scribble animation examples to inform us of the possible design space of expressive and comprehensive sketched animations which respond to real-world interactions (Figure~\ref{fig:design-space}).
Based on the analysis, we identify the following six most common scribble animation effects: 1) \textit{\textbf{object binding}}: dynamically move sketched elements based on the body or object movement, 2) \textit{\textbf{flip-book animation}}: creating an animation based on multiple sketched frames, 3) \textit{\textbf{action trigger}}: one-time animation effect based on the specific action, 4) \textit{\textbf{particle effects}}: spawning many sketched elements around the object, 5) \textit{\textbf{motion trajectory}}: showing the motion path of objects, and 6) \textit{\textbf{contour highlight}}: line-based animation around the contour of a body or object.

Our main contribution is a set of sketching interaction techniques that allow the user to quickly create all of these expressive scribble animations through direct manipulation with the following three step workflow: 
1) \textit{object tracking}: the user specifies a visual entity (e.g., a physical object, a skeletal joint) to track in a live or recorded real-world scene;
2) \textit{sketch elements}: the user adds sketched objects with freehand drawing;
3) \textit{animate sketched elements:} the sketches respond to the world based on the above animation effects.
\removed{With this, \system{} lets the user sketch responsive AR animation in \textit{real-time} and \textit{improvisational} ways, without pre-defined programs or post-production.} 
With these range of techniques through a simple workflow, \system{} lets the user sketch responsive AR animation in \textit{real-time} and in \textit{flexible} ways, without prior preparation or planning.

We demonstrate applications of \system{} \removed{such as}including social media video creation, augmented classroom education, and storytelling.
We evaluate our system through two user studies: 1) a usability study with twenty participants, and 2) expert reviews with seven professional video creators and theatre professionals.
The study results confirm that our tool lowers the barrier to creating AR-based sketched animation by enabling the end-user to quickly draw expressive animation in engaging and intuitive ways.


In summary, this paper contributes:
\begin{enumerate}
\item A taxonomy of existing scribble animation effects that informs us of the possible design space of expressive AR sketched animation.
\item A set of techniques that enable authoring of expressive freehand AR sketched animations through improvisational sketching interactions.
\item An implementation, applications, and the user evaluation of \system{}.
\end{enumerate}


% \todo{improvisation: create and perform spontaneously or without preparation.}
% \begin{itemize}
%     \item Real-time
%     \item Mobile low cost system
%     \item Create animations in real world easily
%     \item freehand drawing
% \end{itemize}
