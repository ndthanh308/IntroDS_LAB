% Zach study incorporate (main points):
% reduce miscommunication
% concept development
% encourage discovery
% lowcost - esp for practice


\section{Usability Study}
\subsection{Method}
We evaluate our system with two user studies: a usability study and expert reviews. The first study's goal is to evaluate our system's usability.
To do so, we recruited twenty participants (14 male, 6 female, age: 19 - 29) from our local community.
The usability study consists of the following two tasks.

\subsubsection{Standardized Task} 
We first demonstrated all of the basic functionalities. We then showed an animation video (Figure~\ref{fig:teaser}) we created with our system. The video featured all the functionalities. The animation video had the following elements: 1) \textit{object binding} between hands and umbrella, 2) \textit{flip-book animation} for the triggered water splash, 3) \textit{action trigger} based on the stomp, 4) \textit{particle effects} with rain drops, 5) \textit{motion trajectory} of the hand movement, 6) \textit{contour highlight} of the whole body. 
We asked the participants to recreate this animation without instructing them how to make it.
The goal of this task is to let the user use all of the functionalities to measure each feature's usefulness and understandability.

\subsubsection{Exploration Task}
Once the user finished the identical task, we moved on to the exploration task. We asked the participants to create their own sketched animation from their imagination. 
For inspiration, participants were shown existing examples of scribble animations samples from the corpus we collected for our taxonomy analysis as well as many animation videos created by the authors. Each participant was asked to think of a scenario that involves one or more objects driven by a human action or object movement. Then one of the authors performed the actions for the participant while the participant added sketches on top of it by themselves. The goal of this task was to measure the expressiveness and flexibility of our system through open-ended animation scenarios.

\ \\
We designed these tasks as \textit{``usage evaluation''} based on the HCI toolkit evaluation strategy~\cite{ledo2018evaluation}.
Instead of a controlled experiment, we chose to conduct the usage evaluation because there was no clear baseline to compare with \system{}.
For example, existing video-editing tools have a completely different workflow, so we cannot compare them with our system. 
Therefore, for the usability study, we focused on the usage evaluation for the end user. 
On the other hand, we also conducted an expert review to compare our approach with existing video-editing or animation techniques from an expert's point of view. 

Both tasks were conducted in a research lab.
All of the participant's performances were recorded for objective measurements (e.g., task completion time).
After the session, we asked the participants to provide feedback through an online questionnaire. 
In total, the study took approximately 60 minutes, and participants were compensated \$10 CAD.

% Figure environment removed


% \todo{
%  Reviewer 2: Add more descriptions of each result
%  Reviewer 3: Second, related to goal of supporting expressiveness and improvisation, I felt the paper could have provided substantially more detail on the work produced by external participants using the system.
%  Reviewer 3: I felt the description of the creative outcomes should be focused on much more
%  Reviewer 3: How did participants approach the authoring process, in what ways did they use effects in combination or unexpected ways? What if any narratives emerged from the pieces?
% }
% Figure environment removed




\subsection{Results}
\subsubsection*{\textbf{1) Overall Experiences}}
Figure~\ref{fig:usability-result} summarizes the 7-point Likert questionnaire response of the usability study.
Overall, participants responded positively to the user experience (5.65/7). 
\textit{``P7: The entire system was quite neat and fun to use. When watching the examples, I imagined myself re-creating them.''}
The participants also responded positively about the interface and workflow.
From the questionnaire results, the participants found the system easy to understand (5.25/7), and from the comments, many participants mentioned that the interface was self-explanatory (P2, P5) and easy to learn (P8, P12).
\textit{``P5: The interface and the workflow were simple enough to learn that little explanation was required.'' }

Most participants did not have animation or video creation experience, but the participants felt that the system could be easy to use even for novice users. 
The participants generally agreed that the system would lower the barrier to creating AR animation (4.95/7).
\textit{``P15: It was very easy to pick up the tools and nuance of the system. Further, I can imagine many ways in which both amateurs and professionals can make interesting things using it.'' } 
For the standardized task, all participants could complete the given task without any explicit assistance from the researcher.
For the task completion time, the participants finished the standardized task on average in 3.25 min (Min: 1.48 min, Max: 8.2 min, SD: 1.58 min). 

While the participants did not see any issues with the workflow, some complained about the system's performance.
For example, the participants commented that the animation was not smooth (P14) and sometimes lagged (P9, P11).
Also, the participants commented that the tracking accuracy could be improved (P18). 
Additionally, the participants wished for finer-grained body and object tracking (P12, P19). 
% Reviewer 3: participants give comments
% Zhijie: "give" -> "gave".


\subsubsection*{\textbf{2) Quality and Expressiveness}}
One of the goals of our system is to increase the expressiveness and flexibility of sketched animations with freehand drawing.
From the questionnaire results, participants found the system flexible (5.45/7) and agreed that the quality and expressiveness of the created result are high (5.3/7).
For example, many participants found the system flexible (P2, P10, P18, P20) to create various animations,
\textit{``P20: Based off of what I tried to do, it has enough features to support different creative ideas.''} 
The participants also positively commented on the expressiveness of freehand drawing. 
\textit{``P7: As mentioned previously, I think once one has a better understanding of the system (more than ~20 mins), there are endless possibilities.''}
For the exploration task, the participants created various animations with 5.24 min on average (Min: 2.21 min, Max: 18.52 min, SD: 4.87 min) \added{and median 8.93 min}. 
Figure~\ref{fig:exploration-study} illustrates the example animations created by the participants. 
Among the 20 scenarios, we observed some common patterns such as dancing highlights, using superpowers, and equipping weapons.

\subsubsection*{\textbf{3) Creation Process and Improvisation}}
During the creation process for exploration tasks, we observed that most participants did not have a clear idea of what they wanted to create at first. Instead, they developed their narratives by drawing a simple shape and experimenting with supported animation features. For instance, in the case of the arrows and shield example (c), the participant first drew a simple arrow symbol and played with the emitting line. As they did so, the arrow began to look more like arrows from an enemy, which led to the story of a fighting soldier in the Middle Ages. In our study, we observed this type of ``unexpected creation'' happening with many of the examples, including a simple circle turning into a balloon (l), a centaur inspired by a sword and shield (e), and an energy blast derived from an unexpected action trigger of a simple line (h). While a few participants had a clear vision at the beginning, like body annotation for biology and anatomy education (d), overall, the participants appreciated the ability of improvisational and spontaneous exploration enabled by freehand drawing and interactive experiments.


\subsubsection*{\textbf{4) Animation Techniques}}
We also asked for user feedback for each animation feature. 
From the questionnaire responses, the participants positively responded to most of the features regarding the usefulness and understandability of the creation process for each feature: 
object binding (6.4/7, 6/7),
flipbook animation (5.55/7, 4.95/7),
action trigger (6.05/7, 5.45/7),
particle effects (5.25/7, 5.55/7),
motion trajectory (5.95/7, 6/7),
contour highlight (5.3/7, 6.45/7). 
We also counted the frequency of each animation technique used in the exploration study: object binding (14/20), flip-book animation (10/20), action trigger (9/20), particle effects (12/20), motion trajectory (7/20), contour lines (8/20). 
From our observations, most participants began by binding a simple shape and then gradually added more complex animations, such as emitting snowflakes, to refine their scenarios. Some participants chose not to use our contour highlight, particle effects, and motion trajectory features but instead used flip-book animations to achieve visually similar results by considering the depth effects.
From the questionnaire response and usage data, overall, the participants liked all six animation techniques.

The participants also provided comments on individual features.
\textit{``P15 on flip-book animation: This feature was particularly useful when paired with other features such as action-trigger. It is relatively simple to create with a few seconds of instruction.''} 
\textit{``P19 on action trigger: I really liked the action-trigger feature. I see this as being the most useful feature coupled with the frame drawing feature.''} 
\textit{``P4 on motion trajectory: cool feature, could be used in videos to make movements more fancy and interesting.''} 
\textit{``P17 on particle effects: Very creative feature for lasting animation like weather and environment effects.''}
\textit{``P12 on contour highlight: Easy to implement and useful to extract just the object you want to animate.''}

%Reviewer 3: typo when author is describing participant quote.
% Zhijie: I don't see where is wrong

\subsubsection*{\textbf{5) Potential Use Cases}}
The participants found \system{} useful for social media (6.1/7) and storytelling for children (6.2/7).
All participants saw potential benefits of \system{} for classroom teachings (5.35/7).
\removed{In particular, participants felt that \system{} provided an ``P3: Easy and engaging educational videos and lab tutorials can be created using \system{} which will be easy for teachers to create and helpful for students. Interactive and engaging experience for the students to visually see and involve in creating the abstractions of concepts''.}
In particular, the participants felt that \system{} provided an easy and engaging way to create educational videos and lab tutorials, which can be helpful for both teachers and students (P3). They also expressed interest in using \system{} for a variety of other applications, including presentations (P8), sports analysis (P15), and safety and training demonstrations (P17).

% Reviewer 3: I would have liked to hear more about what effects the experts experimented with in their interviews.
\section{Expert Review}
\subsection{Method}
We also conducted an expert review to gain in-depth feedback. We recruited seven experts mainly from two groups - video creator experts and experts with a background in theatre. 
E1: full-time animation artist who has 18K followers and 1.8M upvotes on TikTok, 
E2: professional video creator who has 100K followers on BiliBili,
E3: professional video creator who has 97K followers on YouTube,
E4: researcher in AR and animation creation fields in a renowned company,
E5: theatre performer with 10 years experience, also a Computer Science Ph.D. student with interests in Mixed Reality
E6: theatre performance director/creative head with 25+ years of experience
E7: theatre director and actor with 5 years of theatre experience

We interviewed video creators to qualitatively compare our system with existing tools. While for experts with a background in theatre, we evaluate our system for improvisation, storytelling, and AR prototyping perspectives. We recruited theatrical performers and directors strongly interested in creating AR-based live performances via university mailing lists. We conducted a workshop to let them use our system to explore and prototype their AR-based performance so that they have a better understanding of the different features our tool offers and the workflow that our tool can support. Designers and technical developers also joined the workshop \added{and all  participants used the tool in pairs or groups}, thus ensuring an opportunity for collaboration among professionals from different backgrounds and with different skill sets. We were specifically interested in evaluating the qualitative measurements of the improvisational ability of our system. After the workshop, we conducted in-person interviews with each participant to reflect on their prototyping and ideation experiences and gain insights into how they can use our system for augmented live performances and storytelling. The interview lasted approximately one hour for each expert, and we provided \$20 CAD for their participation.

\subsection{Results}
All experts were excited by the potential of our tool. They found the tool to be intuitive (E1, E3,E4), useful (E3, E5, E7), playful (E1, E6), compelling (E2), general purpose (E1), and versatile (E3). Overall, the experts found the system enabled a creative, improvisational, and expressive range of freehand scribble animations, which they highlight in the strengths and limitations of our system.

\subsubsection*{\textbf{1) Comparison to Current Practices and Workflows: }}
Experts found the authoring workflow to be intuitive. They thought the tool could greatly reduce the time and cost of adding embedded animations. \textit{``E2: It takes considerable time to add effects in Adobe AfterEffects than this system's process of sketching while recording''.} E4, an HCI expert, pointed out that \system{} has a unique proposition of real-time creation and interaction. \textit{E4: ``Current tools and research projects are either made for the pre-production or post-production stage of content creation.''}. In addition, experts found that the identified six animation techniques were extensive and covered the most commonly seen scribble effects (E1-E4). Theatre experts saw a similarity between the director-performer relationship in theatre and the animator-actor dynamic between tool users (E5-E7). For example, just like the director, the animator directs the actor and makes decisions about the scene, props, and story. Overall, experts thought that our tool could complement their existing workflows.

\subsubsection*{\textbf{2) Encourages Discovery and Ideation: }}
Experts shared that \system{} could be used for ideation as you can rapidly test, discard and recreate ideas (E1, E3, E4). 
\textit{E3:``I can see it as an AR prototyping tool, not for the final product but a quick lo-fi mockup which enables idea generation''.} Experts also found that our system's real-time feedback helps ensure that their ideas are technically feasible. \textit{E5:``We spend time coming up with gestures and actions only to find they are not being detected. The ability to interact in real-time allows us to make these beautiful discoveries because now you have some specifics available, instead of making whatever we like now and thinking about the feasibility later''}
Other than ideas for augmented performances, experts also found that the tool could be useful in traditional theatre. \textit{E7:``You can also sketch props, costumes, and set designs and even how they interact with each other, which could be especially useful in rehearsals.''} 

\subsubsection*{\textbf{3) Improves Communication and Collaboration:}}
An important aspect that experts highlighted was that it could help improve communication and collaboration, specifically by reducing miscommunication between teams. \textit{E6:`` ... and he and I never got to the same visual language. So I think a shared language like this would be really helpful.''} As E6 and E7 had been directors, they remarked that it would benefit directors as they often need to communicate with multiple teams with different skills and technical languages.

\subsubsection*{\textbf{4) Lightweight Design: }}
Experts appreciated that \system{} is lightweight with a mobile AR web-based interface and requires minimal setup.
Video creators appreciated that the system was quick and did not require expertise. \textit{``E5: I could imagine the world would become more interesting with \system{} on a mobile phone. I could just go out to the streets and shoot a movie with animations and effects''}. The experts with a theatre background also appreciated the low-cost setup (E5-E7) as they pointed out that small theatre groups often have low budgets (E6, E7). \textit{ E6:``We don't want to bring in a ton of projectors or control lighting and other tedious and costly arrangements. So making  it available on mobile and a wider audience is precisely why it is so exciting!''}. Experts also appreciated the freehand sketched modality of embedded animations as it gives more agency over the visuals, adds personality, and makes the video look authentic. \textit{E3:``I have worked with some designers who want to add sketches to their videos as it gives a lot of personality to the video and makes it authentic.}

% interesting - let the audience sketch props
% ``I can see use-cases where real-time feedback plays a crucial role like in sports analysis, live exercise annotations, teaching in a live classroom.'' (E3) - Overview/end


\subsubsection*{\textbf{Limitations and Future Suggestions}}
On the other hand, the experts also shared many limitations of \system{},  which were not identified in the usability study. For example, E2, E3, and E5 raised the concern that \system{} may not be suitable for those lacking sketching expertise.\textit{``E2: the prototype is only usable for artistically inclined users.''}. All the experts agreed that freehand drawing provides freedom and flexibility but requires the user to draw a new sketch every time they use it. Experts shared that the system needs a few modifications to replace video editing platforms (E1-E4). \textit{``E4: It does not have a video and audio track and other video editing functionalities''}. Also, although our system allows the user to erase the sketches in the drawing stage, the created effect cannot be removed once applied to the system. \textit{``E1: For a drawing system, an 'undo' button is essential.''} and \textit{``E2: I strongly request a fully functional 'undo' button.''}.

\subsubsection*{\textbf{Improvisational vs. Preparation-Based Authoring}}
Based on the rich insights we gained from the expert interviews, we reflected on the contrast between improvisational and preparation-based authoring systems. Improvisational authoring tools provide users with more flexibility during the creation process, allowing them to create and modify content in real time without the need for prior planning and preparation. This can be useful for quickly exploring different ideas and possibilities, particularly during the ideation phase. However, improvisational tools can result in lower production quality as a trade-off. Also, improvisational performances can be challenging to control and duplicate, resulting in undesired animation sequences.
In contrast, preparation-based tools give users more control over the final product but can be inflexible and time-consuming. Choosing between improvisational and preparation-based tools depends on the user's goals and needs. Future research could explore tools that allow users to switch between improvisational and preparation-based approaches or combine the best features of both in a single authoring tool.


% We also observed the participants' creation process for exploration tasks. 
% In general, most participants did not have a clear idea of what they wanted to create at first, but rather they developed their narratives by drawing a simple shape and experimenting with \system{}'s animation.
% For example, in (c) arrows and a shield example, the participant first drew a simple arrow symbol and played with the emitting line. As they did so, the arrow began to look more like actual arrows from an enemy, which led to the emergence of a whole story about the fighting soldier in the Middle Ages.
% In our study, we observed this ``unexpected discovery'' happening with many of the examples, thanks to the improvisational feature of \system{}, such as a simple circle leads the person with a balloon (l), a sword and a shield inspire the centaur (e), and a simple line and trigger action gives an idea of energy blast (h). 
% While a few participant had a clear idea at the very beginning, like body annotation for biology and anatomy education (d), in general, the participants appreciated the ability to spontaneously explore through freehand drawing and interactive experiments. 
% % come up with new ideas they may not have thought of before. 
% through playful experimentation. When sketches immediately start animating and responding, it

% offer more flexibility and control to the user, allowing them to create and modify content on the fly without the need for prior planning and preparation. This can be useful for quick exploration of different ideas, especially for the ideation and prototyping phase. On the other hand, it has a drawback in terms of the production quality, leading it to 
% but can lead to a less polished output. In comparison, preparation-based tools allow users to plan and prepare their content in advance, giving users more control over the final product. However, these tools can be inflexible and time-consuming. The choice between improvisational and preparation-based tools depends on the user's specific needs and goals. Future research could provide guidelines to help users choose between these tools and develop tools that allow users to switch between improvisational and preparation-based approaches or combine the best features of both approaches in a single authoring tool.
