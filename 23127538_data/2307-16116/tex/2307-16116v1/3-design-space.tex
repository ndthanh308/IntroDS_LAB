% Figure environment removed


\section{Design Space of Augmented and Responsive Sketched Animation}

\subsection{A Taxonomy Analysis of Scribble Animation Effects}
To better understand the possible design space of augmented and responsive sketched animation, we first collected and analyzed 172 existing video-edited scribble animation examples. 
\textit{Scribble animation effects} are a common video-editing technique to simply add scribble illustrations over a recorded video for each frame.
Because of the flexibility and aesthetics of the scribble effects, such animation techniques are often used in music videos~\cite{bruno-mars, bombay-bicycle-club}, commercial ads~\cite{microsoft, oreo, goldfish-grahams, budweiser}, trailer promotion videos~\cite{dave}, and presentation videos~\cite{ing-direct}.
Currently, these scribble animation effects are created by manual hand-drawing for each frame, but the resulting videos effectively showcase the potential animation techniques for embedded and responsive sketching.

\subsubsection{Corpus and Methodology}
We searched for video examples using popular video and image search platforms (e.g., Google Images, YouTube, Pinterest, Behance, Vimeo, TikTok) by using \textit{``scribble animation''}, \textit{``scribble effects''}, or \textit{``scribble animation effects''} as keywords.
We first collected 172 scribble animation videos, then filtered these videos by focusing only on those with embedded and responsive animations (i.e., remove those where the animation does not interact with or respond to real-world motion), which gave us a total of 120 videos. 
Given the collected videos, one of the authors took screenshots for representative examples for each video and conducted open coding to identify a first approximation of the dimensions by categorizing them. 
After this process, two authors used an online whiteboard (Miro board) to perform systematic coding with individual tagging. Finally, all authors reflected upon the design space and corresponding examples until all agreed on the consistency and comprehensiveness of the categorization.
While we did our best to cover the comprehensive and exhaustive design space, due to the nature of the manual search and taxonomy analysis, we do not argue that our proposed design space is the only way to categorize the existing embedded and responsive sketched animation. 
Rather, our goal is to identify common animation techniques to get insights for ourselves and the HCI community to explore the possibilities of sketched animation for AR. Our results and example videos will be also available on the website~\footnote{\href{ https://ilab.ucalgary.ca/realitycanvas}{ https://ilab.ucalgary.ca/realitycanvas}}.


\subsection{Six Common Animation Techniques}
\subsub{1) Object Binding}
\removed{First, }Object binding is a technique \removed{to bind}for binding sketched elements to a body or physical object, so that the sketches dynamically move according to the corresponding body or object movement.
Examples of object binding include a scribble wing for a human body, scribble glasses or hats for a human face, or scribble \removed{smoke}exhaust for a car. 
The attached sketched objects are mostly static in their shape, so that only the position or orientation changes based on the position of the bound object\removed{ position}.

\subsub{2) Flip-Book Animation}
Flip-book animation is a common animation technique that can be used for many different purposes. In scribble animation effects, many videos use flip-book animation for morphing the shape of the sketches or adding a dynamic effect to existing sketches.
Flip-book animation is often used to combine different animation techniques. 
For example, when combined with object binding, it can create animated object binding effects. 

\subsub{3) Action Trigger}
Action trigger is a one-time animation effect played when a certain action happens. 
For example, when two objects collide, the scribble animation can show\removed{s} a bumping effect to highlight the collision.
Alternatively, such actions can also be bound to different triggered actions.
Possible actions we have observed are hand-clapping, kicking, touching down, hand-waving, and stomping, among many others.

\subsub{4) Particle Effects}
Particle effects are a technique to create repeated animation with many spawned elements.
For example, some videos use scribble particle effects to effectively create raindrops, wind flow, or smoke animations.
In contrast to action-trigger, particle effects are \removed{supposed}intended to be repeated effects.


\subsub{5) Motion Trajectory}
Motion trajectories are used to highlight a certain movement of the body or object.
By showing the path trajectory, the video can make physical motion more visible and expressive.
Such motion trajectory is often used to make human actions stand out, \removed{like with}as in dancing and sports videos.
Most of the motion trajectories use a simple path animation that shows the afterimage of the body.
Some other examples create more expressive motion effects by using a morphing animation. 

% Figure environment removed

\subsub{6) Contour Highlight}
Finally, many scribble animations use an animated contour line to trace a body or object.
This allows a \removed{certain}selected object or body part to stand out by highlighting its contour.
When the body posture changes or the object moves, these contour lines keep following to fit the bound object. 
Most of the contour highlight lines are static, in which the entire contour is covered by a single line, but some examples leverage an animated line, in which a partial line moves across the contour of the object. 
