



\section{Related Work}

\subsection{Augmented Reality Sketching Tools}
With recent advances of AR technology, there is an increasing number of AR (or VR) sketching tools in both commercial products (e.g., \textit{Just a Line}~\cite{just-a-line}, \textit{TiltBrush}~\cite{tiltbrush}, \textit{Gravity Sketch}~\cite{gravity-sketch}, \textit{Vuforia Chalk AR}~\cite{vuforia-chalk}) as well as research prototypes (e.g., \textit{SymbiosisSketch}~\cite{arora2018symbiosissketch}, \textit{VRSketchIn}~\cite{drey2020vrsketchin}, \textit{PintAR}~\cite{gasques2019pintar}).
By leveraging the benefits of \textit{blending} virtual sketches into the physical world, these tools expand the real-world sketching experience for designing (e.g., \textit{SymbiosisSketch}~\cite{arora2018symbiosissketch}, \textit{Mobi3DSketch}~\cite{kwan2019mobi3dsketch}, \textit{DesignAR}~\cite{reipschlager2019designar}), AR prototyping (e.g., \textit{ProtoAR}~\cite{nebeling2018protoar}, \textit{360Proto}~\cite{nebeling2019360proto}), education (e.g., \textit{PaperTrail}~\cite{rajaram2022paper}, \textit{Augmented Body}~\cite{ferdous2019s}), collaboration (e.g., \textit{Vuforia Chalk}~\cite{vuforia-chalk}, \textit{PintAR}\cite{gasques2019pintar}), and entertainment (e.g., \textit{DoodleLens}~\cite{doodlelens}).

\removed{While most of these AR sketching tools only support \textit{static sketching}, in which sketched elements only stay in the mid-air without moving or animating,} 
To go beyond simple static AR sketching, more recent works have started exploring \textit{dynamic and responsive} sketched animation, in which the sketched elements can animate and respond to the corresponding real-world interaction.
For example, \textit{Interactive Body-driven Graphics}~\cite{saquib2019interactive} or \textit{ChalkTalk AR}~\cite{perlin2018chalktalkvrar} allows the creation of dynamic sketched animations for augmented storytelling.
\textit{RakugakiAR}~\cite{rakugakiar} allows the quick creation of animated sketched characters in AR.
Tools like \textit{Rapido}~\cite{leiva2021rapido} and \textit{Pronto}~\cite{leiva2020pronto} leverage expressive freehand sketched animations for video-based AR prototyping.
Moreover, \textit{RealitySketch}~\cite{suzuki2020realitysketch}, \textit{Reactile}~\cite{suzuki2018reactile}, \textit{Graphiti}~\cite{saquib2022graphiti}, and \textit{Sketched Reality}~\cite{kaimoto2022sketched} let the user bind sketched elements with a physical object so that the user can create responsive animations with real-time and improvisational sketching interactions.

However, these existing tools often require pre-defined configuration or preparation~\cite{saquib2019interactive, perlin2018chalktalkvrar, leiva2020pronto}.
For example, prior work~\cite{saquib2019interactive, perlin2018chalktalkvrar} distinguishes between a \textit{preparation phase} and \textit{interaction phase}, by requiring the user to prepare  assets~\cite{saquib2019interactive} or pre-programmed animations~\cite{perlin2018chalktalkvrar} prior to the performance, which hinders real-time and improvisational sketching exploration.
On the other hand, there are some real-time dynamic AR sketching tools~\cite{suzuki2020realitysketch, saquib2022graphiti, kaimoto2022sketched, suzuki2018reactile}, but they do not allow freehand drawing, limiting the flexibility and generalizability of possible animations.
In contrast, our tool enables both \textit{\textbf{freehand}} and \textit{\textbf{real-time}} AR sketched animation, allowing for more expressive, flexible, and improvisational animation authoring to augment live or recorded real-world interactions through AR.


% \todo{Fix the broken reference in Fig 2}
% Figure environment removed


% \todo{Reviewer 3:  Add/discuss the following work
% DataInk - Xia;  
% A Layered Authoring Tool for Stylized 3D animations - Ma;  
% Preserving Hand-Drawn Qualities in Audiovisual Performance Through Sketch-Based Interaction - Bourgault;

% Reviewer 3: James Patterson’s Norman: [https://presstube.com/hello/](https://presstube.com/hello/) (scroll down)
% Zhijie: I couldn't find the paper, link is broken.
% }

\subsection{Screen-based Sketched Animation Tools}
Apart from AR sketching tools, HCI research also has a long history of screen-based sketching tools (e.g., \textit{SketchPad}~\cite{sutherland1964sketchpad}, \textit{Flatland}~\cite{mynatt1999flatland}, \textit{Silk}~\cite{landay1996silk}, \textit{Teddy}~\cite{igarashi2006teddy}, 
\textit{Electronic Cocktail Napkin}~\cite{gross1996electronic}, 
\textit{ILoveSketch}~\cite{bae2008ilovesketch}).
In particular, recent works have explored real-time interactive sketched animations, which make static sketches dynamic and animated using the power of computation.
For example, \textit{Draco}~\cite{kazi2014draco} explores sketched animation through direct sketching interactions based on kinetic textures.
Such tools have democratized opportunities for animation authoring, which were previously only accessible to professional animators or video editors. 
These tools are used for many applications, such as math and science education (e.g., \textit{MathPad2}~\cite{laviola2004mathpad},  \textit{PhysInk}~\cite{scott2013physink}, \textit{Apparatus}~\cite{schachman2015apparatus}), storytelling (\textit{SketchStory}~\cite{lee2013sketchstory}, \textit{Stop Drawing Dead Fish}~\cite{victor2012stop}), data visualization\removed{s} (e.g., \textit{NpakinVis}~\cite{chao2010napkinvis}, \textit{Data Illustrator}~\cite{liu2018data}, \textit{Transmogrification}~\cite{brosz2013transmogrification}, \textit{DataInk}~\cite{xia2018dataink}), live music performance (e.g., \textit{Megafauna}~\cite{bourgault2021preserving}), and artistic animation authoring (e.g., \textit{Kitty}~\cite{kazi2014kitty}, \textit{Draco}~\cite{kazi2014draco}, \textit{Motion Amplifiers}~\cite{kazi2016motion}).
While these works have expanded possible sketched animation techniques, these interactions are only available on a computer screen. \added{VideoDoodles recently proposed a system to create hand-drawn animations on videos \cite{yu2023videodoodles}. However, their focus is also on video editing and their main contribution lies in a custom tracking algorithm for perspective deformations and occlusions. On the other hand, our focus is mainly on live AR scribble animation. Also, our main contribution lies in the taxonomy and design space, which has not been previously explored.}
Our goal is to bring these \textit{dynamic sketching} techniques into AR sketching tools by leveraging real-world interactions, as opposed to screen-based interactions. 
% This paper contributes to solving several technical and design challenges to make this happen in the real world.

\subsection{Authoring Tools for Augmented Effects}
Augmented animation effects --- adding visual effects to enhance live or recorded videos --- have been used in many videos on YouTube and TikTok.
Traditionally, creating these effects often requires extensive skills using professional tools such as Adobe Premiere Pro \removed{and}or Final Cut Pro.
For example, tools like \textit{Scribbl}~\cite{scribbl} also provide a means to create scribble animation effects for recorded videos, but these tools only support \textit{frame-by-frame} authoring without any object tracking or binding, which leads to significant time and effort to create animations, limiting real-time and improvisational exploration.
To fill this gap, HCI researchers have explored alternative authoring approaches for such augmented animation effects.
For example, \textit{PoseTween}~\cite{liu2020posetween} lets the user add virtual effects based on human movement to create augmented action videos.
SnapChat's \textit{Lens Studio}~\cite{lens-studio} also allows face augmentation with a simple authoring workflow.
\textit{RealityTalk}~\cite{liao2022realitytalk} also introduces an authoring tool to augment live presentations with embedded graphics.
These tools allow for the creation of impressive augmented animation for various applications, including live storytelling~\cite{saquib2019interactive, liao2022realitytalk}, entertainment ~\cite{liu2020posetween}, education ~\cite{gong2021holoboard}, collaborative discussions~\cite{dillenbourg2013design, kasahara2012second}, data visualization~\cite{chen2019marvist}, and sports training~\cite{chen2018computer, homecourt, sousa2016augmented}.
In these tools, however, the user needs to prepare \textit{assets} in advance to augment the real world. 
In contrast, \system{} lets the user draw these assets through sketching in real-time, which greatly enhances flexibility and improvisational interactions.
