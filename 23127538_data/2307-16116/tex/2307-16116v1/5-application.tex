% Figure environment removed

\section{Applications}

\subsection{Social Media Video Creation}
First, our system allows users to quickly create videos for social media platforms such as TikTok, YouTube, and Instagram. Currently, video creation and editing with these animation effects \removed{have a tedious workflow} require specific skills and non-trivial amounts of time. People can use our system to record their videos with animated effects much more easily. For example, the user can shoot a dancing or music video with real-time scribble animation effects, such as \textit{motion trajectory} of dancing movement, \textit{contour highlight} to make a person stand out, and \textit{action trigger} to highlight the claps or steps. 

\subsection{Augmented Storytelling}
Using our system, the user can also easily create augmented \added{stories}~\cite{saquib2019interactive, liao2022realitytalk}.
For example, by using animated \textit{object binding}, the user can augment the body with sketched illustrations and animation, like adding a Superman costume or pirate features. In this scenario, the user can sketch on another user’s performance, which can be streamed to a large screen in real time so that the performer can see the current performance. 
Such augmented storytelling would greatly expand, for example, children’s storytelling experience with dynamic effects.


% \todo{I think writing is sold, but the reviewers are not persuaded. Maybe creating system figures will help.}

\subsection{Classroom Education}
The user can also add dynamic visual annotations to a physical phenomenon to provide better explanations for science or physics classrooms. In this scenario, similar to the storytelling scenario, the teacher can stream the screen view to a large display to make the animation visible to the entire classroom. For example, by using \textit{particle effects}, the user can visualize the airflow \added{around} a levitated ping pong ball. The user can explain how water circulates on Earth by using an animated sketch of rain and clouds with \textit{particle effects} by augmenting a physical model such as terrains. Similar to \textit{RealitySketch}~\cite{suzuki2020realitysketch} and \textit{HoloBoard}~\cite{gong2021holoboard}, such augmentations can enhance the learning experience for classrooms. 

\subsection{Prototyping AR Applications}
Finally, the \system{} system can also be used to create an interactive and animated prototype for AR applications. Similar to existing AR prototyping tools~\cite{leiva2021rapido, leiva2020pronto, nebeling2019360proto, nebeling2018protoar}, embedded sketches allow the user to simulate AR experiences for low-fi prototypes quickly. For example, the user can quickly create an animated AR Christmas card using \textit{object binding}. With \textit{action trigger}, the user can create some interactive AR linked to an action, like the opening of the card. The user can also sketch the animation for augmented music visualization like AR Music Visualizer~\cite{tanprasert2022ar} to test how different effects can enhance the music experience. 
