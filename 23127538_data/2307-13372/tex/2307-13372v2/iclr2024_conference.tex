
\documentclass{article} % For LaTeX2e
\usepackage{iclr2024_conference,times}

% Optional math commands from https://github.com/goodfeli/dlbook_notation.
% %%%%% NEW MATH DEFINITIONS %%%%%
\newtheorem{property}{Property}
\newtheorem{definition}{Definition}
\newtheorem{theorem}{Theorem}
\newtheorem{lemma}{Lemma}
\newtheorem{corollary}{Corollary}
\DeclarePairedDelimiter\abs{\lvert}{\rvert}
\DeclarePairedDelimiter\norm{\lVert}{\rVert}
\makeatletter
\let\oldabs\abs
\def\abs{\@ifstar{\oldabs}{\oldabs*}}
\let\oldnorm\norm
\def\norm{\@ifstar{\oldnorm}{\oldnorm*}}
\makeatother

% Mark sections of captions for referring to divisions of figures
\newcommand{\figleft}{{\em (Left) }}
\newcommand{\figcenter}{{\em (Center) }}
\newcommand{\figright}{{\em (Right)}}
\newcommand{\figtop}{{\em (Top) }}
\newcommand{\figbottom}{{\em (Bottom) }}
\newcommand{\captiona}{{\em (a) }}
\newcommand{\captionb}{{\em (b) }}
\newcommand{\captionc}{{\em (c) }}
\newcommand{\captiond}{{\em (d) }}

% Highlight a newly defined term
\newcommand{\newterm}[1]{{\bf #1}}


\def\figref#1{figure~\ref{#1}}
\def\Figref#1{Figure~\ref{#1}}
\def\twofigref#1#2{figures \ref{#1} and \ref{#2}}
\def\quadfigref#1#2#3#4{figures \ref{#1}, \ref{#2}, \ref{#3} and \ref{#4}}
\def\secref#1{section~\ref{#1}}
\def\Secref#1{Section~\ref{#1}}
\def\twosecrefs#1#2{sections \ref{#1} and \ref{#2}}
\def\secrefs#1#2#3{sections \ref{#1}, \ref{#2} and \ref{#3}}
\def\eqref#1{equation~\ref{#1}}
\def\Eqref#1{Equation~\ref{#1}}
% A raw reference to an equation---avoid using if possible
\def\plaineqref#1{\ref{#1}}
% Reference to a chapter, lower-case.
\def\chapref#1{chapter~\ref{#1}}
% Reference to an equation, upper case.
\def\Chapref#1{Chapter~\ref{#1}}
% Reference to a range of chapters
\def\rangechapref#1#2{chapters\ref{#1}--\ref{#2}}
% Reference to an algorithm, lower-case.
\def\algref#1{algorithm~\ref{#1}}
% Reference to an algorithm, upper case.
\def\Algref#1{Algorithm~\ref{#1}}
\def\twoalgref#1#2{algorithms \ref{#1} and \ref{#2}}
\def\Twoalgref#1#2{Algorithms \ref{#1} and \ref{#2}}
% Reference to a part, lower case
\def\partref#1{part~\ref{#1}}
% Reference to a part, upper case
\def\Partref#1{Part~\ref{#1}}
\def\twopartref#1#2{parts \ref{#1} and \ref{#2}}

% Random variables
\def\reta{{\textnormal{$\eta$}}}
\def\ra{{\textnormal{a}}}

% Random vectors
\def\rvepsilon{{\mathbf{\epsilon}}}
\def\rvtheta{{\mathbf{\theta}}}
\def\rva{{\mathbf{a}}}

% Elements of random vectors
\def\erva{{\textnormal{a}}}
\def\ervb{{\textnormal{b}}}

% Random matrices
\def\rmA{{\mathbf{A}}}
\def\rmB{{\mathbf{B}}}

% Elements of random matrices
\def\ermA{{\textnormal{A}}}
\def\ermB{{\textnormal{B}}}

\def\fvec{{\mathbf{f}}}
\def\bff{{\mathbf{f}}}
\def\bfg{{\mathbf{g}}}
% Vectors
\def\vzero{{\bm{0}}}
\def\vone{{\bm{1}}}
\def\vmu{{\bm{\mu}}}
\def\vtheta{{\bm{\theta}}}
\def\va{{\bm{a}}}
\def\vb{{\bm{b}}}
\def\vc{{\bm{c}}}
\def\vd{{\bm{d}}}
\def\ve{{\bm{e}}}
\def\vf{{\bm{f}}}
\def\vg{{\bm{g}}}
\def\vh{{\bm{h}}}
\def\vi{{\bm{i}}}
\def\vj{{\bm{j}}}
\def\vk{{\bm{k}}}
\def\vl{{\bm{l}}}
\def\vm{{\bm{m}}}
\def\vn{{\bm{n}}}
\def\vo{{\bm{o}}}
\def\vp{{\bm{p}}}
\def\vq{{\bm{q}}}
\def\vr{{\bm{r}}}
\def\vs{{\bm{s}}}
\def\vt{{\bm{t}}}
\def\vu{{\bm{u}}}
\def\vv{{\bm{v}}}
\def\vw{{\bm{w}}}
\def\vx{{\bm{x}}}
\def\vy{{\bm{y}}}
\def\vz{{\bm{z}}}

% Matrix
\def\mA{{\bm{A}}}

% Tensor
\DeclareMathAlphabet{\mathsfit}{\encodingdefault}{\sfdefault}{m}{sl}
\SetMathAlphabet{\mathsfit}{bold}{\encodingdefault}{\sfdefault}{bx}{n}
\newcommand{\tens}[1]{\bm{\mathsfit{#1}}}
\def\tA{{\tens{A}}}
\def\tB{{\tens{B}}}
\def\tC{{\tens{C}}}
\def\tD{{\tens{D}}}
\def\tE{{\tens{E}}}
\def\tF{{\tens{F}}}
\def\tG{{\tens{G}}}
\def\tH{{\tens{H}}}
\def\tI{{\tens{I}}}
\def\tJ{{\tens{J}}}
\def\tK{{\tens{K}}}
\def\tL{{\tens{L}}}
\def\tM{{\tens{M}}}
\def\tN{{\tens{N}}}
\def\tO{{\tens{O}}}
\def\tP{{\tens{P}}}
\def\tQ{{\tens{Q}}}
\def\tR{{\tens{R}}}
\def\tS{{\tens{S}}}
\def\tT{{\tens{T}}}
\def\tU{{\tens{U}}}
\def\tV{{\tens{V}}}
\def\tW{{\tens{W}}}
\def\tX{{\tens{X}}}
\def\tY{{\tens{Y}}}
\def\tZ{{\tens{Z}}}


% Graph
\def\gA{{\mathcal{A}}}
\def\gB{{\mathcal{B}}}
\def\gC{{\mathcal{C}}}
\def\dataset{{\mathcal{D}}}
\def\gE{{\mathcal{E}}}
\def\gF{{\mathcal{F}}}
\def\fourier{{\mathcal{F}}}
\def\gG{{\mathcal{G}}}
\def\gH{{\mathcal{H}}}
\def\gI{{\mathcal{I}}}
\def\gJ{{\mathcal{J}}}
\def\gK{{\mathcal{K}}}
\def\gL{{\mathcal{L}}}
\def\loss{{\mathcal{L}}}
\def\gM{{\mathcal{M}}}
\def\gN{{\mathcal{N}}}
\def\normal{{\mathcal{N}}}
\def\gaussian{{\mathcal{N}}}
\def\gO{{\mathcal{O}}}
\def\gP{{\mathcal{P}}}
\def\gQ{{\mathcal{Q}}}
\def\gR{{\mathcal{R}}}
\def\gS{{\mathcal{S}}}
\def\gT{{\mathcal{T}}}
\def\gU{{\mathcal{U}}}
\def\uniform{{\mathcal{U}}}
\def\gV{{\mathcal{V}}}
\def\gW{{\mathcal{W}}}
\def\gX{{\mathcal{X}}}
\def\gY{{\mathcal{Y}}}
\def\gZ{{\mathcal{Z}}}

\def\algebra{{\mathscr{A}}}
\def\borel{{\mathscr{B}}}
\def\manifold{{\mathscr{M}}}

% Sets
\def\sA{{\mathbb{A}}}
\def\sB{{\mathbb{B}}}
\def\complex{{\mathbb{C}}}
\def\sD{{\mathbb{D}}}
\def\expectation{{\mathbb{E}}}
\newcommand{\E}{\mathbb{E}}
\def\sF{{\mathbb{F}}}
\def\sG{{\mathbb{G}}}
\def\sH{{\mathbb{H}}}
\def\sI{{\mathbb{I}}}
\def\sJ{{\mathbb{J}}}
\def\sK{{\mathbb{K}}}
\def\sL{{\mathbb{L}}}
\def\sM{{\mathbb{M}}}
\def\natural{{\mathbb{N}}}
\def\sO{{\mathbb{O}}}
\def\sP{{\mathbb{P}}}
\def\rational{{\mathbb{Q}}}
\def\real{{\mathbb{R}}}
\newcommand{\R}{\mathbb{R}}
\def\sS{{\mathbb{S}}}
\def\sphere{{\mathbb{S}}}
\def\sT{{\mathbb{T}}}
\def\sU{{\mathbb{U}}}
\def\sV{{\mathbb{V}}}
\def\sW{{\mathbb{W}}}
\def\sX{{\mathbb{X}}}
\def\sY{{\mathbb{Y}}}
\def\integer{{\mathbb{Z}}}
\def\indicator{{\mathbbm{1}}}

% Entries of a matrix
\def\emLambda{{\Lambda}}
\def\emA{{A}}
\def\emB{{B}}
\def\emC{{C}}
\def\emD{{D}}
\def\emE{{E}}
\def\emF{{F}}
\def\emG{{G}}
\def\emH{{H}}
\def\emI{{I}}
\def\emJ{{J}}
\def\emK{{K}}
\def\emL{{L}}
\def\emM{{M}}
\def\emN{{N}}
\def\emO{{O}}
\def\emP{{P}}
\def\emQ{{Q}}
\def\emR{{R}}
\def\emS{{S}}
\def\emT{{T}}
\def\emU{{U}}
\def\emV{{V}}
\def\emW{{W}}
\def\emX{{X}}
\def\emY{{Y}}
\def\emZ{{Z}}
\def\emSigma{{\Sigma}}

% entries of a tensor
% Same font as tensor, without \bm wrapper
\newcommand{\etens}[1]{\mathsfit{#1}}
\def\etLambda{{\etens{\Lambda}}}
\def\etA{{\etens{A}}}
\def\etB{{\etens{B}}}
\def\etC{{\etens{C}}}
\def\etD{{\etens{D}}}
\def\etE{{\etens{E}}}
\def\etF{{\etens{F}}}
\def\etG{{\etens{G}}}
\def\etH{{\etens{H}}}
\def\etI{{\etens{I}}}
\def\etJ{{\etens{J}}}
\def\etK{{\etens{K}}}
\def\etL{{\etens{L}}}
\def\etM{{\etens{M}}}
\def\etN{{\etens{N}}}
\def\etO{{\etens{O}}}
\def\etP{{\etens{P}}}
\def\etQ{{\etens{Q}}}
\def\etR{{\etens{R}}}
\def\etS{{\etens{S}}}
\def\etT{{\etens{T}}}
\def\etU{{\etens{U}}}
\def\etV{{\etens{V}}}
\def\etW{{\etens{W}}}
\def\etX{{\etens{X}}}
\def\etY{{\etens{Y}}}
\def\etZ{{\etens{Z}}}

\def\ceil#1{\lceil #1 \rceil}
\def\floor#1{\lfloor #1 \rfloor}
\def\eps{{\epsilon}}

\newcommand{\pder}[1]{\frac{\partial}{\partial #1}}

\newcommand{\half}{\frac{1}{2}}
\newcommand{\limNinf}{\lim_{N \to \infty}}
\newcommand{\limTzero}{\lim_{\tau \to 0}}


\newcommand{\cmark}{\ding{51}}
\newcommand{\xmark}{\ding{55}}

\newcommand{\layer}{\mathcal{H}}
\newcommand{\defeq}{\triangleq}
%\newcommand{\defeq}{vcentcolon=}
\newcommand{\domain}{\Omega}
\newcommand{\grad}{\nabla}

\newcommand{\cin}{c_{\rm{in}}}
\newcommand{\cout}{c_{\rm{out}}}
\newcommand{\intdomain}{\int_{\domain}}
\newcommand{\network}{\gT}
\newcommand{\subnet}{\gK}
\newcommand{\map}{\gR} %\gR

\newcommand{\innerproduct}[2]{\langle #1, #2 \rangle}
\newcommand{\mcsum}[1][j]{\frac{1}{N}\sum_{#1=1}^N}

\newcommand{\inrspace}[1][c]{\gF_{#1}}

\DeclareMathOperator*{\argmax}{arg\,max}
\DeclareMathOperator*{\argmin}{arg\,min}

\let\ab\allowbreak


\usepackage[utf8]{inputenc} % allow utf-8 input
\usepackage[T1]{fontenc}    % use 8-bit T1 fonts
\usepackage{hyperref}       % hyperlinks
\usepackage{url}            % simple URL typesetting
\usepackage{booktabs}       % professional-quality tables
\usepackage{amsfonts}       % blackboard math symbols
\usepackage{nicefrac}       % compact symbols for 1/2, etc.
\usepackage{microtype}      % microtypography
\usepackage{xcolor}         % colors
\usepackage{svg}
\usepackage{xspace}
% \usepackage{natbib}
\usepackage[export]{adjustbox}
\usepackage{printlen}
\uselengthunit{cm}
\usepackage{comment}
\usepackage{amsmath}
\usepackage{amsthm}
\usepackage{graphicx}
\usepackage{rotating}
\usepackage{float}
\usepackage{mathrsfs}
\usepackage{amssymb}
\usepackage{autobreak}
\usepackage{mathtools}
\usepackage{wrapfig}
\usepackage{bbm}
\usepackage{algorithm,algcompatible}
% \usepackage[capbesideposition=right]{floatrow}
% \usepackage[absolute,overlay]{textpos}
\usepackage[permil]{overpic}
% \usepackage[ruled,vlined]{algorithm2e}
% \usepackage{algorithmic}
\usepackage[noend]{algpseudocode}
\newcommand{\bigzero}{\mbox{\normalfont\Large\bfseries 0}}
% \usepackage{support_caption}
\usepackage{subcaption}
\usepackage{lipsum}
\usepackage[hang]{footmisc}
% \usepackage[heightadjust=all]{floatrow}
\usepackage{outlines}
% \usepackage{multicol}
\usepackage[capitalise]{cleveref}
\usepackage{tikz}
\usetikzlibrary{shapes,backgrounds,patterns}
\usepackage[absolute]{textpos}
\usepackage{thmtools} 
\usepackage{thm-restate}

\usepackage{thmtools} 
\usepackage{thm-restate}
\allowdisplaybreaks
\usepackage[toc,page,header]{appendix}
\usepackage{minitoc}
\usepackage{longtable,xtab,booktabs}
\declaretheorem[name=Theorem]{theorm} %,numberwithin=section

\declaretheorem[name=Corollary]{corolary} %,numberwithin=section

\declaretheorem[name=Lemma]{lema} %,numberwithin=section

\declaretheorem[name=Proposition]{prop}

\newsavebox{\algleft}
\newsavebox{\algright}
\newsavebox{\mdpfigright}

\algdef{SE}[DOWHILE]{Do}{doWhile}{\algorithmicdo}[1]{\algorithmicwhile\ #1}%

\DeclareMathOperator*{\argmax}{arg\,max}
\DeclareMathOperator*{\argmin}{arg\,min}
\newcommand\numberthis{\addtocounter{equation}{1}\tag{\theequation}}
\newcommand\calF{\mathcal{F}}
\newcommand\calG{\mathcal{G}}
\newcommand\calM{\mathcal{M}}
\newcommand\calV{\mathcal{V}}
\newcommand\calU{\mathcal{U}}
\newcommand\calW{\mathcal{W}}
\newcommand\calP{\mathcal{P}}
\newcommand\calD{\mathbb{D}}
%%%%%%%%%%%%%%%%%
%% macros introduced by Luke 
\newcommand\mydef[1]{{\bf\em #1}}
%%%%%%%%%%%%%%%%%

\newcommand{\numviparams}{{| \lambda |}}
\newcommand{\scoreaccvars}[1]{s_1^{#1}, \ldots, s_{\numviparams}^{#1}}
\newcommand{\scoreaccvar}[2]{s_{#1}^{#2}}
\newcommand{\isdeterm}[1]{\text{Deterministic}({#1})}


\newcommand{\expect}[1]{\mathbb{E}\left[{#1}\right]}
\newcommand{\var}[1]{\mathbb{V}\left[ {#1} \right]}
\newcommand{\expectdist}[2]{\mathbb{E}_{#1}\left[ {#2} \right]}
\newcommand{\vardist}[2]{\mathbb{V}_{#1}\left[ {#2} \right]}
\newcommand{\cov}[2]{\mathbb{C}\text{ov}[{#1}][{#2}]}
\newcommand{\covv}[1]{\mathbb{C}\text{ov}[{#1}]}
\newcommand{\corr}[1]{\mathbb{C}\text{orr}[{#1}]}

\newcommand{\fix}[1]{\mathit{fix}\left({#1}\right)}
\newcommand{\sbr}[1]{\left\llbracket {#1} \right\rrbracket}
\newcommand{\ctxtype}[3]{{#1} \cong_\text{ctx} {#2} : {#3}}
\newcommand{\bigstep}[3]{{#1} \Downarrow_{#2} {#3}}


% PCF types
\newcommand{\bool}{\mathit{bool}}
\newcommand{\nat}{\mathit{nat}}

\newcommand{\ctx}[1]{\mathcal{C}\left[ {#1}\right] }
\newcommand{\pcft}[1]{\text{PCF}_{#1}}

\newcommand{\nfl}{\mathbb{N}_\bot}
\newcommand{\bfl}{\mathbb{B}_\bot}

% PCF constructs
\newcommand{\succc}[1]{\mathbf{succ}({#1})}
\newcommand{\succcn}[2]{\mathbf{succ}^{#1}({#2})}
\newcommand{\zero}{\mathbf{0}}
\newcommand{\zerotest}[1]{\mathbf{zero}\left({#1}\right)}
\newcommand{\pred}[1]{\mathbf{pred}\left( {#1} \right)}
\newcommand{\predn}[2]{\mathbf{pred}^{#1}\left( {#2} \right)}
\def\solvable{\#}

\newcommand{\true}{\mathbf{true}}
\newcommand{\false}{\mathbf{false}}
\newcommand{\pcffix}[1]{\mathbf{fix}\left({#1}\right)}
\newcommand{\pcffn}[3]{\mathbf{fn}~{#1}:{#2}\mathpunct{.}{#3}}
\newcommand{\pairtype}[2]{{#1} * {#2}}
\newcommand{\pairexp}[2]{\mathbf{pair}({#1}, {#2})}
\newcommand{\leftexp}[1]{\mathbf{left}({#1})}
\newcommand{\rightexp}[1]{\mathbf{right}({#1})}

\newcommand{\RationalPos}{\mathbb{Q}^{+}}

\newcommand{\meas}[1]{\mathbb{M}\left( {#1} \right) }
\newcommand{\integ}[1]{\sbr{#1}_I}

\newcommand{\notbigstep}[2]{{#1}~\cancel{\Downarrow}_{#2}}
\newcommand{\subtrace}[3]{{#1}^{{#2} \ldots {#3}}}
\newcommand{\supp}[1]{\textsf{supp}\left({#1}\right)}
\newcommand{\dom}[1]{\textsf{Dom}\left({#1}\right)}
\newcommand{\suppk}[2]{\textsf{Supp}^{#1}\left({#2}\right)}
\newcommand{\tracespace}{\bigcup_{n \in \mathbb{N}}[0, 1]^n}
\newcommand{\generictracespace}{\mathbb{T}}
\newcommand{\nnreals}{\mathbb{R}_{\geq 0}}
\newcommand{\posreals}{\mathbb{R}_{> 0}}
\newcommand{\reals}{\mathbb{R}}

\newcommand{\unrollkM}[2]{\textsf{unroll}_{#1}\left({#2}\right)}
\newcommand{\nphmcint}[5]{\Psi_\textsf{NP}\left({#1}, {#2}, {#3}, {#4}, {#5}\right)}

%SPCF constructs
\newcommand{\spcfvalues}{\Lambda^0_v}

\newcommand{\prevalueM}[1]{\textsf{value}^{-1}_{#1}(\spcfvalues{})}
\newcommand{\num}[1]{\underline{#1}}

% \theoremstyle{definition}
% \newtheorem{thm}{Theorem}
% \newtheorem{lem}{Lemma}
% \newtheorem{defn}{Definition}
% \newtheorem{conj}{Conjecture}
% \newtheorem{prop}{Proposition}

%\theoremstyle{definition}
%\newtheorem{defn}{Definition}[section]
%\newtheorem{example}[defn]{Example}
%
%
%\theoremstyle{plain}
%\newtheorem{thm}{Theorem}[section]
%\newtheorem{lem}[thm]{Lemma}
%\newtheorem{cor}[thm]{Corollary}
%\newtheorem{conj}[thm]{Conjecture}
%\newtheorem{prop}[thm]{Proposition}
%\newtheorem{remark}[thm]{Remark}

%% Proofs
%\let\oldproof\proof
%\renewcommand{\proof}{\color{blue}\oldproof}


\definecolor{codegreen}{rgb}{0,0.6,0}
\definecolor{codegray}{rgb}{0.5,0.5,0.5}
\definecolor{codepurple}{rgb}{0.58,0,0.82}
\definecolor{backcolour}{rgb}{0.95,0.95,0.92}

\lstdefinestyle{myStyle}{
    belowcaptionskip=1\baselineskip,
    breaklines=true,
    frame=none,
    basicstyle=\footnotesize\ttfamily,
    keywordstyle=\bfseries\color{green!40!black},
    commentstyle=\itshape\color{purple!40!black},
    identifierstyle=\color{blue},
    backgroundcolor=\color{gray!10!white},
    %backgroundcolor=\color{backcolour}, 
    numberstyle=\tiny\color{codegray},
    stringstyle=\color{codepurple},
    breakatwhitespace=false,                          
    keepspaces=true,                 
    numbers=left,       
    numbersep=5pt,                  
    showspaces=false,                
    showstringspaces=false,
    showtabs=false,                  
    tabsize=2,
}

% argmin/argmax
\DeclareMathOperator*{\argmax}{arg\,max}
\DeclareMathOperator*{\argmin}{arg\,min}

% Concatenation of lists
\newcommand\doubleplus{+\kern-1.3ex+\kern0.8ex}

% Program configurations
\newcommand{\tuple}[1]{\ensuremath{\langle #1 \rangle}}
% Rule based definitions
\newcommand{\Rule}[4][]{\ensuremath{\inferrule*[lab={\hypertarget{#2}{(\TirName{#2})}},#1]{#3}{#4}}}

% Calligraphic symbols
\newcommand{\calI}{{\mathcal I}} 
\newcommand{\calT}{{\mathcal T}}

%  Macro for new Y operator.
\newcommand{\yBounded}[3]{\mu^{#1}_{#2}\rvert_{#3}}

%%%%%%%%%%%%%%%%%
 
%%%%%%%%%%%%%%%%%

\newcommand{\expv}{\mathbb{E}}

\newcommand{\combTr}[2]{\left[\begin{matrix}
		#1\\
		#2
	\end{matrix} \right]}

\newcommand{\exType}[2]{\left\{\begin{matrix}
		#1\\
		#2
	\end{matrix} \right\}}
\newcommand{\myint}[1]{ [#1]}
\newcommand{\Uniform}{\ensuremath{\mathrm{Uniform}}}
\newcommand{\Normal}{\ensuremath{\mathrm{normal}}}
\DeclareMathOperator{\abs}{abs}
\DeclareMathOperator{\pdf}{pdf}

\newcommand{\intConf}[1]{\lceil#1\rceil}
\newcommand{\tr}{\boldsymbol{t}}

\newcommand{\sample}{\tt{sample}}
%\newcommand{\fix}{\texttt{fix}}
%\newcommand{\num}[1]{\underline{#1}}
\newcommand{\myif}{\texttt{if}}
\newcommand{\mylet}{\texttt{let} \, }
\newcommand{\myin}{\, \texttt{in} \,}
\newcommand{\mythen}{\, \texttt{then} \,}
\newcommand{\myelse}{\, \texttt{else} \,}
\newcommand{\score}{\tt{score}}
\newcommand{\tick}{\tt{tick}}

\newcommand{\term}{\tt{term}}
\newcommand{\pv}{\mathbf{v}}
\newcommand{\rv}{\mathbf{r}}

\newcommand{\interval}{\mathfrak{I}}

\newcommand{\typeReal}{\textbf{\textsf{R}}}

\newcommand{\symbolInt}{\myint{\cdot}}

\newcommand{\LambdaInterval}{\Lambda_{\interval}}
\newcommand{\LambdaSymbolic}{\Lambda_{\text{sym}}}

\newcommand{\toIntervalTerm}[1]{#1^{2\interval}}

%Others
\newcommand{\Sset}{\mathbb{S}}
\newcommand{\Iset}{\mathbb{I}}
\newcommand{\Rset}{\mathbb{R}}
\newcommand{\Nset}{\mathbb{N}}
\newcommand{\Zset}{\mathbb{Z}}

\newcommand{\Term}{\mathbb{T}}
\newcommand{\prob}{\mathbb{P}}
\newcommand{\expt}{\mathbb{E}}


\newcommand{\Leb}{\tt{Leb}}
\newcommand{\Red}{\tt{Red}}
\newcommand{\cost}{\text{cost}}

%\newcommand{\intervalab}[2]{\underline{[#1,#2]}}
\newcommand{\intervalab}{\underline{[a,b]}}
\newcommand{\interI}{\mathcal{I}}
\newcommand{\trans}{\mathcal{T}}

\newcommand{\iv}{\mathbb{I}}

% Programming language constructs
\newcommand{\lit}[1]{\underline{#1}}
\newcommand{\letIn}[1]{\mathsf{let}\,{#1}\,\mathsf{in}\,}
\newcommand{\fixLam}[2]{\mu {#1} {#2}.}
\newcommand{\ifElse}[3]{\mathsf{if} (#1 \le \num{0}) \, {#2} \,\mathsf{else}\, {#3}}

%%Basic notions
\newcommand{\pspace}{(\Omega,\mathcal{F},\probm)}
\newcommand{\probm}{\mathbb{P}}
\newcommand{\condexpv}[2]{{\expt}{\left[{#1} \mid {#2}\right]}}

\newcommand{\stdConf}[1]{(#1)}
%\newcommand{\intConf}[1]{\lceil#1\rceil}
%\newcommand{\intConf}[1]{(#1)}
%\newcommand{\symConf}[1]{\langle\!\langle  #1 \rangle\!\rangle}
%\newcommand\symPath[1]{(#1)}
\newcommand{\symPath}[1]{\langle\!\langle  #1 \rangle\!\rangle}
\newcommand\symConf[1]{(#1)}

\newcommand{\ifSimple}[3]{\mathsf{if}(#1, #2, #3)}
%\newcommand{\ifElse}[3]{\mathsf{if} (#1 \le 0) \, \allowbreak {#2} \, \allowbreak \mathsf{else}\, {#3}}
%\newcommand{\ifElse}[3]{\ifSimple{#1}{#2}{#3}}

%\newcommand{\trace}{\mathsf{s}}
%
%\newcommand\defn[1]{{\bf \em #1}}
\newcommand{\traces}{\mathbb{T}}
%
%\newcommand{\stdConf}[1]{(#1)}
%%\newcommand{\intConf}[1]{\lceil#1\rceil}
%\newcommand{\intConf}[1]{(#1)}
%%\newcommand{\symConf}[1]{\langle\!\langle  #1 \rangle\!\rangle}
%%\newcommand\symPath[1]{(#1)}
%\newcommand{\symPath}[1]{\langle\!\langle  #1 \rangle\!\rangle}
%\newcommand\symConf[1]{(#1)}

\newcommand{\valueSem}[1]{\mathsf{val}_{#1}} % value (semantics)
\newcommand{\weightSem}[1]{\mathsf{wt}_{#1}} % weight (semantics)
\newcommand{\measureSem}[1]{\llbracket #1 \rrbracket}
\newcommand{\posterior}{\mathsf{posterior}}


%%%%%%%%%
% 
%%%%%%%%
\newcommand{\loc}{\ell}
\newcommand{\locs}{\mathit{L}}
\newcommand{\blocs}{\mathit{L}_{\mathrm{b}}}

\newcommand{\iflocs}{\mathit{L}_{\mathrm{if}}}
\newcommand{\looplocs}{\mathit{L}_{\mathrm{while}}}

\newcommand{\alocs}{\mathit{L}_{\mathrm{a}}}
\newcommand{\wlocs}{\mathit{L}_{\mathrm{w}}}
\newcommand{\rlocs}{\mathit{L}_{\mathrm{r}}}
\newcommand{\Alocs}[1]{\mathit{L}_{\mathrm{A}}^{\mathsf{#1}}}
\newcommand{\Dlocs}{\mathit{L}_{\mathrm{nd}}}
\newcommand{\transitions}{{\rightarrow}}

%%% 
\newcommand{\plocs}{\mathit{L}_{\mathrm{p}}}
\newcommand{\tlocs}{\mathit{L}_{\mathrm{t}}}

\newcommand{\lin}{\loc_\mathrm{init}}
\newcommand{\lout}{\loc_\mathrm{out}}
\newcommand{\val}[1]{\mbox{\sl Val}_{#1}}

\newcommand{\pvars}{V_\mathrm{p}}
\newcommand{\rvars}{V_{\mathrm{r}}}
\newcommand{\pre}{\mathrm{pre}}

\newcommand{\sle}{\sqsubseteq}
\newcommand{\sge}{\sqsupseteq}

\newcommand{\lfp}{\mathrm{lfp}}
\newcommand{\gfp}{\mathrm{gfp}}

\newcommand{\rdvarjdis}{\mathcal D}
\newcommand{\sampset}{\textit{supp}}

\newcommand{\upd}{\mbox{\sl upd}}
\newcommand{\wet}{\mbox{\sl wt}}
\newcommand{\transset}{\mathfrak T}
\newcommand{\valin}{\pv_{\mathrm{init}}}
\newcommand{\ret}{\mbox{\sl ret}}

\newcommand{\win}{w_{\mathrm{init}}}

\newcommand{\sampdpd}{\overline{\Upsilon}}

\newcommand{\outmap}{\text{O}}
\newcommand{\sat}[1]{\langle #1 \rangle}
\newcommand{\monoid}{\mbox{\sl Monoid}}
\newcommand{\handelmanformat}{(\dagger)}

\newcommand{\trunc}{\mathcal{B}}

\newcommand{\ewt}{\mbox{\sl ewt}}
\newcommand{\statemap}{\text{St}}

\newcommand{\valrd}{{\mathbf{r}}}
\newcommand{\frmloc}{\ell^{\mathrm{src}}}
\newcommand{\toloc}{\ell^{\mathrm{dst}}}

\newcommand{\monomials}{\mathbf{M}}
\allowdisplaybreaks
\makeatletter
\def\thanks#1{\protected@xdef\@thanks{\@thanks
        \protect\footnotetext{#1}}}
\makeatother

\captionsetup{skip=5.0pt} 
\title{Submodular Reinforcement Learning}


% Authors must not appear in the submitted version. They should be hidden
% as long as the \iclrfinalcopy macro remains commented out below.
% Non-anonymous submissions will be rejected without review.

\author{%
  Manish Prajapat\thanks{Code available at \url{https://github.com/manish-pra/non-additive-RL}}  \\
%   ETH AI Center\\
  ETH Zurich\\
  % \texttt{manishp@ethz.ch} \\
  % examples of more authors
   \And
   Mojm\'ir Mutn\'y \\
%   Dept. of Computer Science\\
% Department of Computer Science\\
   ETH Zurich \\
   % \texttt{mmutny@ethz.ch} \\
   \And
   Melanie N. Zeilinger \\
%   Institute of Dynamic Systems \& Control \\
% IDSC\\
   ETH Zurich \\
   % \texttt{mzeilinger@ethz.ch} \\
   \And
   Andreas Krause \\
%   Dept. of Computer Science \\
% Department of Computer Science\\
   ETH Zurich \\
   % \texttt{krausea@ethz.ch} \\
  % \And
  % Coauthor \\
  % Affiliation \\
  % Address \\
  % \texttt{email} \\
}

\newcommand{\remove}[1]{\textcolor{red}{#1}}
% \newcommand{\rev}[1]{\textcolor{blue}{#1}}
\newcommand{\rev}[1]{\textcolor{black}{#1}}

% The \author macro works with any number of authors. There are two commands
% used to separate the names and addresses of multiple authors: \And and \AND.
%
% Using \And between authors leaves it to \LaTeX{} to determine where to break
% the lines. Using \AND forces a linebreak at that point. So, if \LaTeX{}
% puts 3 of 4 authors names on the first line, and the last on the second
% line, try using \AND instead of \And before the third author name.

\newcommand{\fix}{\marginpar{FIX}}
\newcommand{\new}{\marginpar{NEW}}

\setlength{\footnotemargin}{2mm}

\iclrfinalcopy % Uncomment for camera-ready version, but NOT for submission.
\begin{document}

\maketitle


\begin{abstract}

The Fast Reciprocal Square Root Algorithm is a well-established approximation technique consisting of two stages: first, a coarse approximation is obtained by manipulating the bit pattern of the floating point argument using integer instructions, and second, the coarse result is refined through one or more steps, traditionally using Newtonian iteration but alternatively using improved expressions with carefully chosen numerical constants found by other authors. The algorithm was widely used before microprocessors carried built-in hardware support for computing reciprocal square roots. At the time of writing, however, there is in general no hardware acceleration for computing other fixed fractional powers. This paper generalises the algorithm to cater to all rational powers, and to support any polynomial degree(s) in the refinement step(s), and under the assumption of unlimited floating point precision provides a procedure which automatically constructs provably optimal constants in all of these cases. It is also shown that, under certain assumptions, the use of monic refinement polynomials yields results which are much better placed with respect to the cost/accuracy tradeoff than those obtained using general polynomials. Further extensions are also analysed, and several new best approximations are given.

\end{abstract}


\vspace{-4mm}
\section{Introduction}\vspace{-1mm}
\section{Introduction}
\label{sec:introduction}

The recent surge of Large Language Models (LLMs), such as GPT-3.5/4~\cite{bubeck_sparks_2023}, PaLM~\cite{chowdhery_palm_2022}, FLAN-T5~\cite{chung_scaling_2022}, and Alpaca~\cite{taori_stanford_2023}, has shown a promising trend of large pre-trained models to do a variety of tasks in a zero-shot setting (\ie without any new training data). Example tasks include question answering~\cite{omar2023chatgpt,robinson2023leveraging}, logic reasoning~\cite{wei_chain--thought_2023,zhou_least--most_2023}, machine translation~\cite{brants2007large,gulcehre2017integrating} \etc\ 
A number of experiments have revealed that, built on hundreds of billions of parameters, these LLMs have started to show the capability to understand the human common sense beneath the natural language and do proper reasoning and inference accordingly~\cite{bubeck_sparks_2023,nori_capabilities_2023}.

Among different applications, one particular question yet to be answered is how well LLMs can understand human mental health states through natural language.
Mental health problems represent a significant burden for individuals and societies worldwide.
A recent report suggested that more than 20\% of adults in the U.S. would experience at least one mental disorder in their lifetime~\cite{mental2022state} and 5.6\% of adults experienced a serious psychotic disorder that significantly impairs functioning~\cite{mental2023stats}. The global economy loses around \$1 trillion annually in productivity due to depression and anxiety alone~\cite{mentalcost2023}.

In the past decade, there has been a plethora of research in natural language processing (NLP) and computational social science on detecting mental health issues via online text data such as social media~(\eg \cite{guntuku_detecting_2017,eichstaedt2018facebook,coppersmith_clpsych_2015,de_choudhury_social_2013,de_choudhury_mental_2014}). However, most of these studies have focused on building domain-specific machine learning (ML) models (\ie one model for one particular task, such as stress detection~\cite{nijhawan2022stress,guntuku2019understanding}, depression prediction~\cite{eichstaedt2018facebook,tadesse2019detection,xu_leveraging_2019}, or suicide risk assessment~\cite{de_choudhury_discovering_2016,coppersmith2018natural}). Even for traditional pre-trained language models such as BERT, it needs to be finetuned for specific downstream tasks~\cite{devlin_bert_2019,liu_roberta_2019}.
Since natural language is a major component of mental health assessment and treatment~\cite{sharma2018mental,gkotsis2016language}, LLMs might be a potentially powerful tool to understand end-users' mental states based on the language users' wrote. These instruction-finetuned and general-purpose models can understand a variety of inputs and obviate the need to train multiple models for different tasks. Thus, we can envision using one LLM for a variety of mental-health-related tasks, such as multiple question-answering, reasoning, and inference.
Such a vision opens up a wide range of opportunities for UbiComp, Human-Computer Interaction (HCI), and mental health communities, such as online public health monitoring systems~\cite{patel2018psyheal,graham2019artificial}, intelligent assistants for mental counselors and supporters~\cite{sharma_towards_2021,sharma_humanai_2023}, mental-health-aware personal chatbots~\cite{abd2021perceptions,denecke2020mental}, to just name a few.
However, there is a lack of investigation into understanding, evaluating, and improving the capability of LLMs for mental health prediction tasks.

There are few very recent studies on the evaluation of LLMs (\eg ChatGPT) on mental-health-related tasks, most of which are in zero-shot settings with simple prompt engineering~\cite{yang_evaluations_2023,amin_will_2023,lamichhane_evaluation_2023}. Researchers have shown preliminary results that LLMs have some initial capability of predicting mental health disorders with natural language with some promising but still limited performance compared to state-of-the-art domain-specific NLP models~\cite{yang_evaluations_2023,lamichhane_evaluation_2023}.
This remaining gap is expected since existing general-purpose LLMs are not specifically trained on mental health tasks.
However, to achieve our vision of leveraging LLMs for mental health support and assistance, we need to answer the research question: \textbf{How to empower LLMs with more mental health domain knowledge and become an expert}?

We conducted a series of experiments with multiple LLMs, including Alpaca~\cite{noauthor_stanford_2023}, Alpaca-LoRA~\cite{hu_lora_2021}, and GPT-3.5~\cite{noauthor_introducing_2022}.
Considering the data availability, we focused on online social media data with high-quality human-generated mental health labels.
Our experiments contained three stages: (1) zero-shot prompting, where we experimented with various prompts related to mental health, (2) few-shot prompting, where we inserted examples into prompt inputs, and (3) instruction-finetuning, where we finetuned LLMs on multiple mental-health datasets with various tasks.

Our results indicate that zero-shot obtained promising but limited performance on multiple mental health prediction tasks across all models. GPT-3.5 had relatively better results since it has a larger scale. But their performance is still far from task-specific models. 
Meanwhile, providing a few shots in the prompt can improve the model performance to some extent ($\overline{\Delta}$ = 4.7\%), but the advantage is limited.
Finally and most importantly, we found that instruction-finetuning can significantly improve the model performance across multiple mental-health-related tasks at the same time. Our finetuned Alpaca, namely \textbf{Mental-Alpaca}, significantly outperforms the original GPT-3.5 ($\times$25 times of model size) by an average of 16.7\% on balance accuracy. 
Meanwhile, Mental-Alpaca can further perform on par with the task-specific state-of-the-art Mental-RoBERTa~\cite{ji_mentalbert_2021}. It is noteworthy that Mental-RoBERTa needs to be trained on each task individually, 
while our Mental-Alpaca can solve different tasks off the shelf. 
% We open-source our training code and model at [github link].
Our experiments present the first comprehensive evaluation of various techniques to enhance LLMs' capability in the mental health domain.

The contribution of our paper can be summarized as follows:
\begin{s_enumerate}
\item We present the first comprehensive evaluation of prompt engineering, few-shot, and finetuning techniques on multiple LLMs in the mental health domain.
\item With online social media data, our results reveal that finetuning on a variety of datasets can significantly improve LLM's capability on multiple mental-health-specific tasks simultaneously.
% We release our model \textbf{Mental-Alpaca} as the first open-source LLM targeted at mental health prediction tasks.
\item We provide a few technical guidelines for future researchers and developers on turning LLMs into experts in specific domains.
\end{s_enumerate}

%motivation - why is this good, and why classical RL cannot solve it. 
%related work 
% \mypar{Beyond Markovian \RL}
%We extend the modelling ability of classical reinforcement learning with Markov chain and Markovian rewards and consider a class of non-Markovian reward functions. 
Several prior works in \RL identify the deficiency in the modelling ability of classical Markovian rewards. This manifests itself especially when exploration is desired, e.g., when the transition dynamics are not completely known \cite{Tarbouriech2019, Hazan2019} or when the reward is not completely known \cite{lindner2021information, Belogolovsky2021}. While all these addresses in some aspect the shortcomings of Markovian rewards, they tend to focus on a specific aspect instead of postulating a new class of reward functions as we do in this work.

\mypar{Convex \RL} Convex \RL also seeks to optimize a family of non-additive rewards.  The goal is to find a policy that optimizes a convex function over the state visitation distribution (which averages over the randomness in the \mdp and the policies actions).  This framework has applications, e.g., in exploration and experimental design \citep{Hazan2019, Zahavy2021,Duff2002, Tarbouriech2019, tarbouriech2020active,Mutny2023}. %and  \citep{}
While sharing some motivating applications, convex and submodular \RL are rather different in nature.  Beyond the high-level distinction that convex and submodular function classes are complimentary, our non-additive (submodular) rewards are defined over the {\em actual sequence} of states visited by the policy, not its {\em average behaviour}. \citet{mutti2022challenging} points out that this results in substantial differences, noting the deficiency of convex \RL in modelling expected utilities of the form as in~\cref{eq: obj}, which we address in our work.

\mypar{Submodular Maximization}
Submodular functions are widely studied in combinatorial optimization and operations research and have found many applications in machine learning \citep{krause2014submodular,bilmes2022submodularity,tohidi20}.
%concepts popularized with 
The seminal work of  \citet{nemhauser1978analysis} shows that greedy algorithms enjoy a constant factor $1-1/e$ approximation for maximizing monotone submodular functions under cardinality constraints, which is information- and complexity- theoretically optimal \citep{Feige-no-other-effi-algo}. Beyond simpler cardinality (and matroid) constraints, more complex constraints have been considered: most relevant is the s-t-submodular orienteering problem \citep{chekuri_rg}, which aims to find an s-t-path in a graph of bounded length maximizing a submodular function of the visited nodes, and can be viewed as a special case of \subrl on deterministic SMDPs with deterministic starting state and hard constraint on the goal state.  It has been used as an abstraction for informative path planning \citep{Singh2009}. We generalize the setup and connect it with modern policy gradient techniques. 
Certain problems of the form considered in this work can satisfy a notion called \emph{adaptive submodularity}, which generalizes the greedy approximation guarantee over a set of policies \citep{golovin2011adaptive}. While adaptive submodularity allows capturing history-dependence, it fails to address complex constraints (such as those imposed by CMPs).

While submodularity is typically considered for discrete domains (i.e., for functions defined on $\{0,1\}^{|\V|}$, the concept can be generalized to continuous domains, e.g., $[0,1]^{|\V|}$ using notions such as DR-submodularity \citep{greedy-supermodular}. This notion forms a class of non-convex problems admitting provable approximation guarantees in polynomial time, which we exploit in Section~\ref{sec: theory}.

The problem of learning submodular functions has also been considered \citep{balcan2011learning}. \citet{dolhansky2016deep} introduce the class of deep submodular functions, neural network models guaranteed to yield functions that are submodular in their input.  These may be relevant for our setting when learning unknown rewards using function approximation, which is an interesting direction for future work.

%The bandit problem has been considered in the submodular setting as well. 
The submodular bandit problem is at the interface of learning and optimizing submodular functions \citep{streeter2008online, Chen2017InteractiveSB, YisongLSB}. Algorithms with no-regret (relative to the 1-1/e approximation) exist, whose performance can be improved by exploiting linearity \citep{YisongLSB} or smoothness \citep{Chen2017InteractiveSB} in the objective.  Our results in Section~\ref{sec: theory} can be viewed as addressing (a generalization of) the submodular stochastic bandit problem.  Exploiting further linearity or smoothness  to improve sample complexity is an interesting direction for future work.

\vspace{-2mm}
\section{Submodular RL: Preliminaries and Problem Statement}
\vspace{-2mm}
\label{sec: preliminaries}
\section{Preliminaries}
\label{sec:preliminaries}

% Figure environment removed
\noindent

\noindent
\textbf{Fuzz Driver Basics} \tab 
% key aspects of a fuzz driver
% The prerequisites can include the initialization of the target API arguments and the setup of the correct execution context.
The key components of a fuzz driver are illustrated in Figure~\ref{fig:fuzz-driver-internal}.
A typical fuzz driver includes at least three components: prerequisites initialization, execution, and post-cleaning of the target API, as mentioned in lines 3, 4, and 7.
In addition, there are three optional components listed in lines 2, 5, and 6 that can improve a driver's effectiveness.
The component mentioned in line 2 allows the driver to reject inputs with large sizes to reduce execution costs, or to split input data into several parts for testing multiple arguments of the target API.
The component mentioned in line 5 enables the driver to call additional APIs to trigger more program behaviors, which helps reveal more bugs.
Finally, the component mentioned in line 6 allows the driver to add semantic oracles to find more logical bugs.
These oracles are similar to \texttt{assert} statements used in unit tests, which abort execution when certain properties of the program are not satisfied.
Since the fuzz drivers are extensively executed with randomly mutated input data, there is a high requirement on the correctness and robustness of its API usage.
The incorrect or unrobust usage can lead to both false positives and negatives.
For instance, if a driver failed to feed the mutated data into the API, it can never find any bug inside the target.
Or if a driver passed an incorrectly initialized argument to the API, false crashes may be raised.
% In this paper, an effective fuzz driver represents the drivers which have correct API usage and produce no false positives.
% Precisely validating the effectiveness of fuzz drivers is crucial for evaluating fuzz driver generation methods.
% However, general validation techniques do not work well due to the diverse semantics on the API usages.

\noindent
\textbf{Minimum Requirements of an Effective Fuzz Driver} \tab 
% what is an effective fuzz driver
To be effective, a fuzz driver must correctly use the API and produce no false positive results.
The minimal requirements for an effective fuzz driver includes satisfying the necessary control flow dependencies and initializing the arguments correctly.
Argument initialization can be one of the following cases (in the order of simplicity):
\ding{182} \textbf{C1}: If the argument value is irrelevant or should be a naive value such as \texttt{0} or \texttt{NULL}, a variable is declared or a literal constant is used directly;
\ding{183} \textbf{C2}: If the argument is supposed to be a macro or a global variable that is already defined in common libraries or the target API's project, it is located and used;
\ding{184} \textbf{C3}: If creating the argument requires the use of common library APIs, such as creating a file and writing specific content, common practices are followed;
\ding{185} \textbf{C4}: If initializing the argument requires the output of other APIs within the project, those APIs are initialized first following the above initialization cases.

\noindent
\textbf{LLM-Based Code Generation} \tab 
LLMs provide a natural language interface that allows users to generate code through conversational queries and answers.
With this interface, code generation tasks can be completed more efficiently and with less technical expertise required.
In this study, prompt represents the content of a single query while the conversation represents one or more rounds of queries and answers sharing the same communication context.

% The type of prompt involved in the study is \textit{prefix prompt}~\cite{prompt-engineering-survey}, which expects the LLM to continue the content
% llm is query based, in a conversational style
% what is prompt, what is query, what is conversation
% we only consider the second form of prompt

\vspace{-1mm}
\section{Submodular RL: Theoretical Limits}\vspace{-1mm}
\label{sec: NP_hard}
\looseness -1 We first show that the \subrl problem is hard to approximate in general.  
In particular, we establish a lower bound that implies \subrl cannot be approximated up to any constant factor in polynomial time, even for {\em deterministic} submodular MDPs. 
We prove this by reducing our problem to a known hard-to-approximate problem -- the submodular orienteering problem (\sop) \citep{chekuri_rg}.  
Since we focus on {\em deterministic} \smdp's, according to Proposition~\ref{prop:deterministicmarkov} and Proposition~\ref{prop:restatedetMDP}, it suffices to consider deterministic, Markovian policies. We now formally state the inapproximability result, 
% \vspace{-5mm}
\begin{restatable*}{theorm}{restateinapprox}\label{thm: inappx}
Let \opt be the optimal value and $\gamma>0$. Even for deterministic \smdp's, the \subrl problem is hard to approximate within a factor of $\Omega(\log^{1-\gamma} \opt)$ unless $\NP \subseteq \ZTIME(n^{polylog(n)})$. 
\end{restatable*} %\vspace{-0.5em}
Thus, under common assumptions in complexity theory \citep{chekuri_rg,Halperin2003inapprox}, the \subrl problem cannot be approximated in general to better than logarithmic factors, i.e., no algorithm can guarantee $J(\pi) \geq \frac{\opt}{\log^{1-\gamma}\opt}$ for all input instances of \subrl. The proof is in \cref{apx: inapprox}.
The significance of this result extends beyond submodular RL. As \subrl falls within the broader category of general non-Markovian reward functions, \cref{thm: inappx} implies that problems involving general set functions are similarly inapproximable, limited to logarithmic factors.

Since our inapproximability result is worst-case in nature, it does not rule out that interesting \subrl problems remain practically solvable. In the next section, we introduce a general algorithm that is efficiently implementable, recovers constant factor approximation under assumptions (\cref{sec: theory}) and is empirically effective as shown in an extensive experimental study (\cref{sec: experiments}).

% Our inapproximability result, while worst-case in nature, does not rule out that interesting \subrl problems may be practically solvable. 


% for any \subrl problem. %Under assumptions, the algorithm can recover a constant factor approximation \cref{sec: theory} and later we demonstrate the effectiveness of the algorithm in an extensive experimental study involving submodular rewards \cref{sec: experiments}.

% We demonstrate the effectiveness of the algorithm and its utility in an extensive experimental study involving submodular rewards.

% \remove{Since the inapproximability result we present is worst-case in nature, it does not rule out that interesting \subrl problems remain practically solvable. In the next section, we introduce a general algorithm that can be efficiently implemented for any \subrl problem. In \cref{sec: experiments}, we demonstrate the effectiveness of the algorithm and its utility in an extensive experimental study involving submodular rewards. Additionally, in \cref{sec: theory}, we establish that our algorithm enjoys a constant factor approximation under stronger assumptions on the Markov chain.}





% introduce the problem - formally 
% state the hardness

% how to transition: In worst case, NP-hard to solve, but still practically solvable as its important. We provide provide geenral alg in the next section and then moving from wors-tscase we provide provable 1/2 approximation in polynomial time for a speicific class of MDPs. 
\savebox{\algleft}{
\begin{minipage}[t]{.60\textwidth}
\vspace{-10.5em}
\begin{algorithm}[H]
\caption{Submodular Policy Optimization (\subPO)}
\begin{algorithmic}[1]
% \SetAlgoLined
\State \textbf{Input:} $\smdp \M, \pi, N, B$
\For{epoch $k = 1:N$} 
% \textit{\textbf{(data collection)}}
\!\!\!\!\For{batch $b = 1:B$} 
\!\!\!\!\For{$h=0:H-1$} 
\State \!\!\!\!Sample $a_h \sim \pi(a_h|s_h)$, execute $a_h$
\State \!\!\!\!$D \!\leftarrow \!\{s_h,a_h, F(\traj_{0:h+1})\}$ \label{alg: sample_collect} 
\EndFor
\EndFor 
% \textit{\textbf{(Gradient estimator)}}
\State Estimate $\nabla_{\theta} J(\pi_{\theta})$ as per \cref{thm: PG} \label{alg: estimator} %, \textcolor{red}{Entropy Expl. $J(\pi) 
 % \leftarrow J(\pi) + \alpha H(\pi)$ }
% \State \textcolor{red}{Momentum $\nabla_{\pi} J(\pi_{k}) \! \leftarrow \! \beta \nabla_{\pi} J(\pi_{k-1})\! + (1-\beta)\nabla_{\pi} J(\pi)$, (try Adams?), Value net. critic $v_{\theta}(\traj_{0:i})$} \\
% \textit{\textbf{(Policy update)}}
% \State $\pi_{k+1} = \argmax_{\pi \in \Pi} \pi \cdot \nabla_{\pi} J(\pi_k)$
% \State $\pi_{k+1} = (1-\alpha_k)\pi_{k} + \alpha_k \pi_{k+1}$, $\alpha_k = \frac{2}{2+k}$
% \State \textcolor{red}{Projected gradient descent on simplex or GD + softmax for simplex constraint}
\State Update policy parameters $(\theta)$  using \cref{eqn: gd}
% \State $\theta \leftarrow \theta + \alpha \nabla_{\theta} J(\pi)$, using \cref{eqn: gd}
\EndFor
\end{algorithmic}
\label{alg:subrl}
\end{algorithm}
\end{minipage}}

\savebox{\figright}{
\begin{minipage}[t]{.35\textwidth}
\centering
    \scalebox{0.35}{


\tikzset{every picture/.style={line width=0.75pt}} %set default line width to 0.75pt        

\begin{tikzpicture}[x=0.75pt,y=0.75pt,yscale=-1,xscale=1]
%uncomment if require: \path (0,1596); %set diagram left start at 0, and has height of 1596

%Shape: Ellipse [id:dp22781965006960792] 
\draw   (271,1256) .. controls (271,1234.46) and (288.68,1217) .. (310.5,1217) .. controls (332.32,1217) and (350,1234.46) .. (350,1256) .. controls (350,1277.54) and (332.32,1295) .. (310.5,1295) .. controls (288.68,1295) and (271,1277.54) .. (271,1256) -- cycle ;
%Shape: Ellipse [id:dp0009770034873839428] 
\draw   (270,1098) .. controls (270,1076.46) and (287.68,1059) .. (309.5,1059) .. controls (331.32,1059) and (349,1076.46) .. (349,1098) .. controls (349,1119.54) and (331.32,1137) .. (309.5,1137) .. controls (287.68,1137) and (270,1119.54) .. (270,1098) -- cycle ;
%Curve Lines [id:da7351476657211149] 
\draw    (349,1098) .. controls (374.18,1090.4) and (453.95,1091.83) .. (475.16,1096.27) ;
\draw [shift={(478,1097)}, rotate = 197.88] [fill={rgb, 255:red, 0; green, 0; blue, 0 }  ][line width=0.08]  [draw opacity=0] (10.72,-5.15) -- (0,0) -- (10.72,5.15) -- (7.12,0) -- cycle    ;
%Curve Lines [id:da9803616319422612] 
\draw  [dash pattern={on 4.5pt off 4.5pt}]  (350,1256) .. controls (385.77,1254.04) and (500.77,1179.09) .. (516.65,1138.44) ;
\draw [shift={(517.5,1136)}, rotate = 106.7] [fill={rgb, 255:red, 0; green, 0; blue, 0 }  ][line width=0.08]  [draw opacity=0] (10.72,-5.15) -- (0,0) -- (10.72,5.15) -- (7.12,0) -- cycle    ;
%Curve Lines [id:da6979473775260718] 
\draw  [dash pattern={on 4.5pt off 4.5pt}]  (349,1098) .. controls (384.59,1096.05) and (453.92,1222.43) .. (476.37,1253.78) ;
\draw [shift={(478,1256)}, rotate = 232.79] [fill={rgb, 255:red, 0; green, 0; blue, 0 }  ][line width=0.08]  [draw opacity=0] (10.72,-5.15) -- (0,0) -- (10.72,5.15) -- (7.12,0) -- cycle    ;
%Shape: Ellipse [id:dp4069934712250871] 
\draw   (478,1256) .. controls (478,1234.46) and (495.68,1217) .. (517.5,1217) .. controls (539.32,1217) and (557,1234.46) .. (557,1256) .. controls (557,1277.54) and (539.32,1295) .. (517.5,1295) .. controls (495.68,1295) and (478,1277.54) .. (478,1256) -- cycle ;
%Shape: Ellipse [id:dp5197280881747808] 
\draw   (478,1097) .. controls (478,1075.46) and (495.68,1058) .. (517.5,1058) .. controls (539.32,1058) and (557,1075.46) .. (557,1097) .. controls (557,1118.54) and (539.32,1136) .. (517.5,1136) .. controls (495.68,1136) and (478,1118.54) .. (478,1097) -- cycle ;
%Shape: Ellipse [id:dp9694823829725758] 
\draw   (687,1256) .. controls (687,1234.46) and (704.68,1217) .. (726.5,1217) .. controls (748.32,1217) and (766,1234.46) .. (766,1256) .. controls (766,1277.54) and (748.32,1295) .. (726.5,1295) .. controls (704.68,1295) and (687,1277.54) .. (687,1256) -- cycle ;
%Shape: Ellipse [id:dp47684208247159465] 
\draw   (687,1097) .. controls (687,1075.46) and (704.68,1058) .. (726.5,1058) .. controls (748.32,1058) and (766,1075.46) .. (766,1097) .. controls (766,1118.54) and (748.32,1136) .. (726.5,1136) .. controls (704.68,1136) and (687,1118.54) .. (687,1097) -- cycle ;
%Curve Lines [id:da8926054440407609] 
\draw    (350,1256) .. controls (375.97,1248.16) and (451.4,1141.4) .. (476.53,1099.48) ;
\draw [shift={(478,1097)}, rotate = 120.43] [fill={rgb, 255:red, 0; green, 0; blue, 0 }  ][line width=0.08]  [draw opacity=0] (10.72,-5.15) -- (0,0) -- (10.72,5.15) -- (7.12,0) -- cycle    ;
%Curve Lines [id:da6909503346391981] 
\draw    (350,1256) .. controls (370.96,1261.85) and (441.83,1265.8) .. (475.49,1256.72) ;
\draw [shift={(478,1256)}, rotate = 162.9] [fill={rgb, 255:red, 0; green, 0; blue, 0 }  ][line width=0.08]  [draw opacity=0] (10.72,-5.15) -- (0,0) -- (10.72,5.15) -- (7.12,0) -- cycle    ;
%Curve Lines [id:da4227819479888133] 
\draw  [dash pattern={on 4.5pt off 4.5pt}]  (350,1256) .. controls (345.59,1305.98) and (499.17,1333.87) .. (516.6,1297.32) ;
\draw [shift={(517.5,1295)}, rotate = 106.7] [fill={rgb, 255:red, 0; green, 0; blue, 0 }  ][line width=0.08]  [draw opacity=0] (10.72,-5.15) -- (0,0) -- (10.72,5.15) -- (7.12,0) -- cycle    ;
%Curve Lines [id:da43040746641683647] 
\draw  [dash pattern={on 4.5pt off 4.5pt}]  (349,1098) .. controls (347.53,1047.04) and (495.4,1020.09) .. (516.38,1055.74) ;
\draw [shift={(517.5,1058)}, rotate = 247.69] [fill={rgb, 255:red, 0; green, 0; blue, 0 }  ][line width=0.08]  [draw opacity=0] (10.72,-5.15) -- (0,0) -- (10.72,5.15) -- (7.12,0) -- cycle    ;
%Curve Lines [id:da2169138512030282] 
\draw    (349,1098) .. controls (374.97,1090.16) and (492.66,1170.68) .. (516.18,1214.37) ;
\draw [shift={(517.5,1217)}, rotate = 245.06] [fill={rgb, 255:red, 0; green, 0; blue, 0 }  ][line width=0.08]  [draw opacity=0] (10.72,-5.15) -- (0,0) -- (10.72,5.15) -- (7.12,0) -- cycle    ;
%Curve Lines [id:da6364602537352781] 
\draw    (557,1100) .. controls (582.18,1092.4) and (662.85,1092.02) .. (684.15,1096.29) ;
\draw [shift={(687,1097)}, rotate = 197.88] [fill={rgb, 255:red, 0; green, 0; blue, 0 }  ][line width=0.08]  [draw opacity=0] (10.72,-5.15) -- (0,0) -- (10.72,5.15) -- (7.12,0) -- cycle    ;
%Curve Lines [id:da33346792042439244] 
\draw  [dash pattern={on 4.5pt off 4.5pt}]  (557,1256) .. controls (592.77,1254.04) and (709.69,1179.09) .. (725.65,1138.44) ;
\draw [shift={(726.5,1136)}, rotate = 106.7] [fill={rgb, 255:red, 0; green, 0; blue, 0 }  ][line width=0.08]  [draw opacity=0] (10.72,-5.15) -- (0,0) -- (10.72,5.15) -- (7.12,0) -- cycle    ;
%Curve Lines [id:da9081851392102287] 
\draw  [dash pattern={on 4.5pt off 4.5pt}]  (557,1100) .. controls (592.59,1098.05) and (662.87,1222.53) .. (685.37,1253.78) ;
\draw [shift={(687,1256)}, rotate = 232.79] [fill={rgb, 255:red, 0; green, 0; blue, 0 }  ][line width=0.08]  [draw opacity=0] (10.72,-5.15) -- (0,0) -- (10.72,5.15) -- (7.12,0) -- cycle    ;
%Curve Lines [id:da22645110117463774] 
\draw    (557,1256) .. controls (582.97,1248.16) and (660.32,1141.4) .. (685.52,1099.48) ;
\draw [shift={(687,1097)}, rotate = 120.43] [fill={rgb, 255:red, 0; green, 0; blue, 0 }  ][line width=0.08]  [draw opacity=0] (10.72,-5.15) -- (0,0) -- (10.72,5.15) -- (7.12,0) -- cycle    ;
%Curve Lines [id:da8221163768033788] 
\draw    (557,1256) .. controls (577.96,1261.85) and (650.73,1265.8) .. (684.48,1256.72) ;
\draw [shift={(687,1256)}, rotate = 162.9] [fill={rgb, 255:red, 0; green, 0; blue, 0 }  ][line width=0.08]  [draw opacity=0] (10.72,-5.15) -- (0,0) -- (10.72,5.15) -- (7.12,0) -- cycle    ;
%Curve Lines [id:da22332666192427086] 
\draw  [dash pattern={on 4.5pt off 4.5pt}]  (557,1256) .. controls (552.59,1305.98) and (708.09,1333.87) .. (725.6,1297.32) ;
\draw [shift={(726.5,1295)}, rotate = 106.7] [fill={rgb, 255:red, 0; green, 0; blue, 0 }  ][line width=0.08]  [draw opacity=0] (10.72,-5.15) -- (0,0) -- (10.72,5.15) -- (7.12,0) -- cycle    ;
%Curve Lines [id:da015628806951311303] 
\draw  [dash pattern={on 4.5pt off 4.5pt}]  (557,1100) .. controls (555.53,1049.04) and (704.36,1020.17) .. (725.38,1055.74) ;
\draw [shift={(726.5,1058)}, rotate = 247.69] [fill={rgb, 255:red, 0; green, 0; blue, 0 }  ][line width=0.08]  [draw opacity=0] (10.72,-5.15) -- (0,0) -- (10.72,5.15) -- (7.12,0) -- cycle    ;
%Curve Lines [id:da09275044993717585] 
\draw    (557,1100) .. controls (582.97,1092.16) and (701.62,1170.76) .. (725.18,1214.37) ;
\draw [shift={(726.5,1217)}, rotate = 245.06] [fill={rgb, 255:red, 0; green, 0; blue, 0 }  ][line width=0.08]  [draw opacity=0] (10.72,-5.15) -- (0,0) -- (10.72,5.15) -- (7.12,0) -- cycle    ;
%Straight Lines [id:da0927494444452106] 
\draw  [dash pattern={on 4.5pt off 4.5pt}]  (511,1386) -- (598.5,1386) ;
\draw [shift={(600.5,1386)}, rotate = 180] [color={rgb, 255:red, 0; green, 0; blue, 0 }  ][line width=0.75]    (10.93,-3.29) .. controls (6.95,-1.4) and (3.31,-0.3) .. (0,0) .. controls (3.31,0.3) and (6.95,1.4) .. (10.93,3.29)   ;
%Straight Lines [id:da08454871327533309] 
\draw    (613,1386) -- (700.5,1386) ;
\draw [shift={(702.5,1386)}, rotate = 180] [color={rgb, 255:red, 0; green, 0; blue, 0 }  ][line width=0.75]    (10.93,-3.29) .. controls (6.95,-1.4) and (3.31,-0.3) .. (0,0) .. controls (3.31,0.3) and (6.95,1.4) .. (10.93,3.29)   ;

% Text Node
\draw (247,1385) node [anchor=north west][inner sep=0.75pt]  [font=\huge] [align=left] {Transition probability};
% Text Node
\draw (356,1116) node [anchor=north west][inner sep=0.75pt]  [font=\huge] [align=left] {a\textsubscript{0}};
% Text Node
\draw (567,1206) node [anchor=north west][inner sep=0.75pt]  [font=\huge] [align=left] {a\textsubscript{0}};
% Text Node
\draw (543,1391) node [anchor=north west][inner sep=0.75pt]  [font=\huge] [align=left] { $\displaystyle \epsilon $};
% Text Node
\draw (618.75,1389) node [anchor=north west][inner sep=0.75pt]  [font=\huge] [align=left] { $\displaystyle 1\ -\ \epsilon $};
% Text Node
\draw (429,1100) node [anchor=north west][inner sep=0.75pt]  [font=\huge] [align=left] {a\textsubscript{0}};
% Text Node
\draw (361,1204) node [anchor=north west][inner sep=0.75pt]  [font=\huge] [align=left] {a\textsubscript{0}};
% Text Node
\draw (424,1267) node [anchor=north west][inner sep=0.75pt]  [font=\huge] [align=left] {a\textsubscript{1}};
% Text Node
\draw (710.4,1249.6) node [anchor=north west][inner sep=0.75pt]  [font=\huge] [align=left] {v\textsubscript{1}};
% Text Node
\draw (504,1174) node [anchor=north west][inner sep=0.75pt]  [font=\huge] [align=left] {a\textsubscript{1}};
% Text Node
\draw (379,1302) node [anchor=north west][inner sep=0.75pt]  [font=\huge] [align=left] {a\textsubscript{0}};
% Text Node
\draw (429,1055) node [anchor=north west][inner sep=0.75pt]  [font=\huge] [align=left] {a\textsubscript{1}};
% Text Node
\draw (497.2,1248.6) node [anchor=north west][inner sep=0.75pt]  [font=\huge] [align=left] {v\textsubscript{1}};
% Text Node
\draw (294.2,1249.8) node [anchor=north west][inner sep=0.75pt]  [font=\huge] [align=left] {v\textsubscript{1}};
% Text Node
\draw (707.2,1092.2) node [anchor=north west][inner sep=0.75pt]  [font=\huge] [align=left] {v\textsubscript{0}};
% Text Node
\draw (500.4,1092.6) node [anchor=north west][inner sep=0.75pt]  [font=\huge] [align=left] {v\textsubscript{0}};
% Text Node
\draw (291.8,1097.2) node [anchor=north west][inner sep=0.75pt]  [font=\huge] [align=left] {v\textsubscript{0}};
% Text Node
\draw (632,1098) node [anchor=north west][inner sep=0.75pt]  [font=\huge] [align=left] {a\textsubscript{0}};
% Text Node
\draw (629,1266) node [anchor=north west][inner sep=0.75pt]  [font=\huge] [align=left] {a\textsubscript{1}};
% Text Node
\draw (571,1120) node [anchor=north west][inner sep=0.75pt]  [font=\huge] [align=left] {a\textsubscript{0}};
% Text Node
\draw (584,1303) node [anchor=north west][inner sep=0.75pt]  [font=\huge] [align=left] {a\textsubscript{0}};
% Text Node
\draw (642,1052) node [anchor=north west][inner sep=0.75pt]  [font=\huge] [align=left] {a\textsubscript{1}};
% Text Node
\draw (285.6,1336.8) node [anchor=north west][inner sep=0.75pt]  [font=\huge] [align=left] {h=0};
% Text Node
\draw (705.4,1337.4) node [anchor=north west][inner sep=0.75pt]  [font=\huge] [align=left] {h=2};
% Text Node
\draw (495.4,1336.8) node [anchor=north west][inner sep=0.75pt]  [font=\huge] [align=left] {h=1};
% Text Node
\draw (713,1173) node [anchor=north west][inner sep=0.75pt]  [font=\huge] [align=left] {a\textsubscript{1}};


\end{tikzpicture}} 
\end{minipage}     }

% % Figure environment removed\vspace{-1.5mm}
\section{General Algorithm: Submodular Policy Optimization \subPOh} \vspace{-1.5mm}
\label{sec: algo}
\looseness -1 We now propose a practical algorithm for \subrl that can efficiently handle submodular rewards. The core of our approach follows a greedy gradient update on the policy $\pi$. As common in the modern \RL literature, we make use of approximation techniques for the policies to derive a method applicable to large state-action spaces. This means the policy $\pi_{\theta}(a|s)$\footnote{autoregressive policies (RNNs or transformers) can be used to capture history-dependence in the same algorithm} 
is parameterized by $\theta \in \Theta$ where $\Theta \subset \R^l$ is compact. In the case of tabular $\pi$, $\theta$ specifies an independent distribution over actions for each state.

\mypar{Approach} 
\looseness -1 The objective from \cref{eq: obj} can be equivalently formulated as $\theta^{\star} \in \argmax_{\theta \in \Theta} J(\pi_\theta)$ as $\theta$ indexes our policy class. Due to the nonlinearity of the parameterization, it is often not feasible to find a global optimum for the above problem. In practice, with appropriate initialization and hyperparameters, variants of gradient descent are known to perform well empirically for MDPs. Precisely, \vspace{-0.4em}
\begin{align}
    \theta \leftarrow \theta + \argmax_{\delta \theta: \delta \theta + \theta \in \Theta} \delta \theta^\top \nabla_{\theta} J(\pi_\theta) - \frac{1}{2 \alpha} \|\delta\theta\|^2. \label{eqn: gd} 
\end{align}
Various \PG methods arise with different methods for gradient estimation and applying regularization \citep{kakade2001natural,schulman2015trust,schulman2017proximal}. The key challenge to all of them is computation of the gradient $\nabla J(\pi_\theta)$. Below, we devise an unbiased gradient estimator for general non-additive functions.

\mypar{Gradient Estimator} As common in the policy gradient (\PG) literature, we can use the score function $g(\traj, \pi_\theta) \coloneqq \nabla_{\theta} (\log \prod\nolimits_{i=0}^{\horizon-1}\pi_{\theta}(a_i|s_i))$ to calculate the gradient $\nabla_{\theta} J$. Namely, Given an \mdp and the policy parameters $\theta$, \vspace{-0.4em} 
\begin{equation}  \label{prop: score}\nabla_{\theta} J(\pi_{\theta}) = \sum_{\traj} f(\traj; \pi_\theta) g(\traj, \pi_\theta) F(\traj).
\end{equation}
\looseness -1 As \cref{prop: score} shows, we do not require knowledge of the environment if sampled trajectories are available. It also does not require full observability of the states nor any structural assumption on the \mdp. On the other hand, the score gradients suffer from high variance due to sparsity induced by trajectory rewards \citep{FU_score,prajapat2021competitive,sutton2018reinforcement}. Hence, we take the \smdp structure into account to develop efficient algorithms. \vspace{-0.1em}

\emph{Marginal gain:} We define the marginal gain for a state $s$ in the trajectory $\traj_{0:j}$ up to horizon $j$ as \vspace{-0.25em}\[F(s|\traj_{0:j}) = F(\traj_{0:j}\cup \{s\}) - F(\traj_{0:j}).\vspace{-0.25em}\] Our approach aims to maximize the marginal gain associated with each action instead of maximizing state rewards. 
This approach shares similarities with the greedy algorithm commonly used in submodular maximization, which maximizes marginal gains and is known for its effectiveness. Moreover, decomposing the trajectory return into marginal gains and incorporating it in the policy gradient with suitable baselines \cite{greensmith2004variance} removes sparsity and thus helps to reduce variance. Inspired by the policy gradient method for additive rewards \citep{sutton1999policy,baxter2001infinite}, we propose the following for \smdp: %For $\traj_{l:l'}$, the events from time step $l$ to $l'$, we have:
\vspace{-0.1em}
\begin{restatable*}{theorm}{restatePG} \label{thm: PG} Given an \smdp and the policy parameters $\theta$, with any set function $\Obj$,\vspace{-0.5em}
    \begin{align}
        \nabla_{\theta} J(\pi_{\theta}) = \E_{\traj \sim f(\traj ; \pi_\theta)} \left[ \sum_{i=0}^{H-1} \nabla_{\theta} \log \pi_{\theta}(a_i|s_i) \left(\sum_{j=i}^{H-1}F(s_{j+1}|\traj_{0:j}) - b(\traj_{0:i})\right) \right] \label{eqn: grad_esti}
    \end{align}\vspace{-1.1em}
\end{restatable*} \vspace{-0.6em}
We use an importance sampling estimator (log trick) to obtain \cref{prop: score}. To reduce variance, we subtract a baseline $b(\traj_{0:i})$ from the score gradient, which can be a function of the past trajectory $\traj_{0:j}$. This incorporates the causality property in the estimator, ensuring that the action at timestep $j$ cannot affect previously observed states. After simplifying and considering marginals, we obtain \cref{thm: PG} (proof is in \cref{apx: PG}). This estimator assigns a higher weight to policies with high marginal gains and a lower weight to policies with low marginal gains. Empirically this performs very well (\cref{sec: experiments}). 

We can optimize this approach by using an appropriate baseline as a function of the history $\traj_{0:j}$, which leads to an actor-critic type method. The versatility of the approach is demonstrated by the fact that \cref{thm: PG} holds for any choice of baseline critic. We explain later in the experiments how to choose a baseline. One can perform a Monte Carlo estimate ~\citep{baird1993} or generalized advantage function (\GAE)~\citep{schulman2015highdimensional} to estimate returns based on the marginal gain.
% However, what the ideal critic function is for non-additive functions needs to be clarified, and we leave it as future work. 
To encourage exploration, similar to standard \PG, we can employ a soft policy update based on entropy penalization, resulting in diverse trajectory samples. Entropy penalization in \subrl can be thought of as the sum of modular and submodular rewards, which is a submodular function.

\mypar{Algorithm} The outline of the steps is given in \cref{alg:subrl}. We represent the agent by a stochastic policy parameterized by a neural network. 
% At each state, the agent uses a categorical distribution over the set of actions. We apply a softmax operation to the distribution to enforce the simplex constraint. 
The algorithm operates in epochs and assumes a way to generate samples from the environment, e.g., via a simulator. In each epoch, the agent recursively samples actions from its stochastic policy and applies them in the environment leading to a \emph{roll out} of the trajectory where it collects samples (\cref{alg: sample_collect}). We execute multiple ($B$) batches in each epoch for accurate gradient estimation.
To update the policy, we compute the estimator of the policy gradient as per \cref{thm: PG}, where we utilize marginal gains of the trajectory instead of immediate rewards as in  standard RL (\cref{alg: estimator}). Finally, we use stochastic gradient ascent to update the policy parameters.

\vspace{-2mm}
\section{Provable Guarantees in Simplified Settings}
\vspace{-2mm}
\label{sec: theory}

\looseness=-1
In general, \cref{sec: NP_hard} shows \subrl is NP-hard to approximate. A natural question is: Is it possible to do better under additional structural assumptions? 
In this section, we present one interesting case under which \subPO can be approximated to a constant factor via DR-submodular optimization.
\begin{definition}[DR submodularity and DR-property, \cite{bian2017continuous}] For $\X \subseteq \R^d$, a function $f: \X \to \R$ is DR-submodular (has the DR property) if $\forall \mathbf{a} \leq \mathbf{b} \in \X$, $\forall i \in [d]$, $\forall k \in \R_{+}$ s.t. $(ke_i + \mathbf{a})$ and $(k e_i + \mathbf{b})$ are still in $\X$, it holds, $f(k \mathbf{e}_i + \mathbf{a}) - f(\mathbf{a}) \geq f(k \mathbf{e}_i + \mathbf{b}) - f(\mathbf{b}). ~($Notation: $\mathbf{e}_i$ denotes $i^{th}$ basis vector and for any two vectors $\mathbf{a}, \mathbf{b}, \mathbf{a} \leq \mathbf{b}$ means $a_i \leq b_i \forall i\in [d])$
\end{definition}
Monotone DR-submodular functions form a family of generally non-convex functions, which can be approximately optimized.  As discussed below, gradient-based algorithms find constant factor approximations over general convex, downward-closed polytopes.

Under the following condition on the Markov chain, we can show that as long as the policy is parametrized in a particular way, the objective is indeed monotone DR-submodular. 
\begin{definition}[$\epsilon$-Bandit \smdp] \label{asm: bandit} An \smdp s.t. for any $v_j, v_k \in \V$, $j \neq k$, $\forall h \in [H]$ and $\forall v' \in \V$, $P_h(v_j|v',a_{j})= 1-\epsilon_h$, and $P_h(v_k|v',a_{j}) = \frac{\epsilon_h}{|\V|-1}$ for $\epsilon_h \in \left[0, \frac{|\V|}{|\V|+1}\right]$ is an $\epsilon$-Bandit \smdp.
\end{definition}
% Figure environment removed
% We call \mdp system that satisfies the above $\epsilon$-Bandit \smdp. 
This represents a "nearly deterministic" \mdp where there is a unique action for each state in the \mdp, which takes us to it with $1-\epsilon$ probability and with the rest, we end up in any other state \cref{fig:bandit_mdp}. 
% On top of this, we are allowed to shrink or increase the state space over the horizon indicated by changing state space $\S^h$. 
While limiting, it generalizes the bandit scenario, which would occur when $\epsilon = 0$. In the following, we consider a class of state-independent policies that can change in each horizon, denoting the horizon dependence with $\pi^h(a)$.
% In the following, we consider a class of state-independent policy that is allowed to change within each round, so we denote it with $\pi^h(a)$, where $\Delta^{|\A|-1}$ is a simplex constraint. 
We now formally establish the connection between \subrl and DR-submodularity,
\looseness=-1
% 
\begin{restatable*}{theorm}{restateDRsub} 
For horizon dependent policy $\pi$ parameterized as $\pi^h(a) \forall h \in [H]$ in an $\epsilon$-Bandit \smdp, and $F(\traj)$ is a monotone submodular function, then $J(\pi)$ is monotone DR-submodular.
\label{thm: dr_sub}
\end{restatable*}
The proof is in \cref{apx: dr_sub}. It builds on two steps; firstly, we use a  reparameterization trick to handle policy simplex constraints. We relax the equality constraints on $\pi$ to lie on a convex polytope $\mathcal{P} = \{ \pi^h(a) ~|~ 0 \leq \pi^h(a) \leq 1, 0 \leq \sum_{j, j \neq k} \pi^h(a_j) \leq 1, \forall k\in[|\A|], \forall h\in [H] \}$ and enforce the equality constraints directly in the objective \cref{eq: obj}. Secondly, under the assumptions of \cref{thm: dr_sub}, we show that the Hessian of $J(\pi)$ only has non-positive entries, which is an equivalent characterization of twice differentiable DR-submodular functions. Furthermore, the result can be generalized to accommodate state and action spaces that vary with horizons, although, for simplicity, we assumed fixed spaces. 

The convex polytope $\mathcal{P}$ belongs to a class of down-closed convex constraints.
\citet{bian2017guaranteed} proposes a modified \frank algorithm for DR-submodular maximization with down-closed constraints. This variant can achieve an $(1-1/e)$ approximation guarantee and has a sub-linear convergence rate. 
The algorithm proceeds as follows: the gradient oracle is the same as \cref{thm: PG}, while employing a tabular policy parameterization.
The polytopic constraints $\mathcal{P}$ are ensured through a \frank step, which involves solving a linear program over the policy domain. Finally, the policy is updated with a specific step size defined in \citep{bian2017guaranteed}.
Furthermore, \citet{hassani2017gradient} shows that any stationary point in the optimization landscape of DR-submodular maximization under general convex constraints is guaranteed to be $1/2$ optimal. Therefore, any gradient-based optimizer can be used for the $\epsilon-$ Bandit \smdp class, and will result in an $1/2$-optimal policy.
In \cref{sec: related_works}, we elaborate on how this setting generalizes previous works on submodular bandits.

\mypar{General \smdp} Although we cannot obtain a provable result for the general \smdp setting (\cref{thm: inappx}), we can show that the \subPO algorithm, under the assumption of modular rewards, reduces to standard \PG, and hence recovers the guarantees and benefits of the modular \PG. 
In particular, with tabular policy parameterization, under mild regularity assumptions, any stationary point of the modular \PG cost function is a global optimum \cite{bhandari2019global}. 
Moreover, interestingly, we can quantify the deviation of the submodular function from a modular function using the notion of curvature \citep{iyer2013fast}.

\emph{Curvature:} The notion of curvature reflects how much the marginal values $\Delta(v|A)$ can decrease as the function of A. The total curvature of $F$ is defined as, $c = 1 - \min_{A, j \notin A} \Delta(j|A)/F(j).$
Note that $c\in[0,1]$, and if $c=0$ then the marginal gain is independent of $A$ (i.e., $F$ is modular). With assumptions of bounded curvature, $c\in(0,1)$, we can guarantee the constant factor optimality for \subPO (proof in \cref{apx: proof_curvature}),
\begin{restatable*}{prop}{restateboundedC} \label{prop: bounded_C}
    Let \subPO under tabular policy parameterization converges to a policy $\pi$. For any monotone submodular function $\Obj$ with a bounded curvature $c\in(0,1)$, \subPO guarentees $J(\pi) \geq (1-c) J(\pi^{\star})$, where $\pi^{\star}$ is an optimal non-Markovian policy.
\end{restatable*}

\vspace{-1em}
% definitoion e-bandit mdp + discsussion 
% statement of the result 
% connection DR-submodualrity 
% FW connection 
\section{Related Work}
\vspace{-0.2 em}
\label{sec: related_works}
\mypar{Beyond Markovian \RL}
%We extend the modelling ability of classical reinforcement learning with Markov chain and Markovian rewards and consider a class of non-Markovian reward functions. 
Several prior works in \RL identify the deficiency in the modelling ability of classical Markovian rewards. This manifests itself especially when exploration is desired, e.g., when the transition dynamics are not completely known \cite{Tarbouriech2019, Hazan2019} or when the reward is not completely known \cite{lindner2021information, Belogolovsky2021}. While all these addresses in some aspect the shortcomings of Markovian rewards, they tend to focus on a specific aspect instead of postulating a new class of reward functions as we do in this work.

\mypar{Convex \RL} Convex \RL also seeks to optimize a family of non-additive rewards.  The goal is to find a policy that optimizes a convex function over the state visitation distribution (which averages over the randomness in the \mdp and the policies actions).  This framework has applications, e.g., in exploration and experimental design \citep{Hazan2019, Zahavy2021,Duff2002, Tarbouriech2019, tarbouriech2020active,Mutny2023}. %and  \citep{}
While sharing some motivating applications, convex and submodular \RL are rather different in nature.  Beyond the high-level distinction that convex and submodular function classes are complimentary, our non-additive (submodular) rewards are defined over the {\em actual sequence} of states visited by the policy, not its {\em average behaviour}. \citet{mutti2022challenging} points out that this results in substantial differences, noting the deficiency of convex \RL in modelling expected utilities of the form as in~\cref{eq: obj}, which we address in our work.

\mypar{Submodular Maximization}
Submodular functions are widely studied in combinatorial optimization and operations research and have found many applications in machine learning \citep{krause2014submodular,bilmes2022submodularity,tohidi20}.
%concepts popularized with 
The seminal work of  \citet{nemhauser1978analysis} shows that greedy algorithms enjoy a constant factor $1-1/e$ approximation for maximizing monotone submodular functions under cardinality constraints, which is information- and complexity- theoretically optimal \citep{Feige-no-other-effi-algo}. Beyond simpler cardinality (and matroid) constraints, more complex constraints have been considered: most relevant is the s-t-submodular orienteering problem \citep{chekuri_rg}, which aims to find an s-t-path in a graph of bounded length maximizing a submodular function of the visited nodes, and can be viewed as a special case of \subrl on deterministic SMDPs with deterministic starting state and hard constraint on the goal state.  It has been used as an abstraction for informative path planning \citep{Singh2009}. We generalize the setup and connect it with modern policy gradient techniques. 
Certain problems of the form considered in this work can satisfy a notion called \emph{adaptive submodularity}, which generalizes the greedy approximation guarantee over a set of policies \citep{golovin2011adaptive}. While adaptive submodularity allows capturing history-dependence, it fails to address complex constraints (such as those imposed by CMPs).

While submodularity is typically considered for discrete domains (i.e., for functions defined on $\{0,1\}^{|\V|}$, the concept can be generalized to continuous domains, e.g., $[0,1]^{|\V|}$ using notions such as DR-submodularity \citep{greedy-supermodular}. This notion forms a class of non-convex problems admitting provable approximation guarantees in polynomial time, which we exploit in Section~\ref{sec: theory}.

The problem of learning submodular functions has also been considered \citep{balcan2011learning}. \citet{dolhansky2016deep} introduce the class of deep submodular functions, neural network models guaranteed to yield functions that are submodular in their input.  These may be relevant for our setting when learning unknown rewards using function approximation, which is an interesting direction for future work.

%The bandit problem has been considered in the submodular setting as well. 
The submodular bandit problem is at the interface of learning and optimizing submodular functions \citep{streeter2008online, Chen2017InteractiveSB, YisongLSB}. Algorithms with no-regret (relative to the 1-1/e approximation) exist, whose performance can be improved by exploiting linearity \citep{YisongLSB} or smoothness \citep{Chen2017InteractiveSB} in the objective.  Our results in Section~\ref{sec: theory} can be viewed as addressing (a generalization of) the submodular stochastic bandit problem.  Exploiting further linearity or smoothness  to improve sample complexity is an interesting direction for future work.

\vspace{-6 mm}
\section{Experiments}
\vspace{-2 mm}
\label{sec: experiments}
% Figure environment removed
\looseness -1 We empirically study the performance of \subrl on multiple environments. They are i) Informative path planning, ii) Item collection, iii) Bayesian D-experimental design, iv) Building exploration, v) Car racing and vi) Mujoco-Ant. The environments involve discrete (i-iv) and continuous (v-vi) state-action spaces and capture a range of submodular rewards, illustrating the versatility of the framework.

The problem is challenging in two aspects: firstly, how to maximize submodular rewards, and secondly, how to maintain an effective state representation to enable history-dependent policies. Our experiments mainly focus on the first aspect and demonstrate that even with a simple Markovian state representation, by \emph{greedily maximizing marginal gains}, one can achieve good performance similar to the ideal case of non-Markovian representation in many environments. However, we do not claim that a Markovian state representation is sufficient in general. For instance, in the building exploration and item collection environments, Markovian policies are insufficient, and a history-dependent approach is necessary for further optimization. Natural avenues are to augment the state representation to incorporate additional information, e.g., based on domain knowledge, or to use a non-Markovian parametric policy class such as RNNs.
Exploring such representations is application specific, and beyond the scope of our work.

We consider two variants of \subPO: \subPOm and \subPOnm, corresponding to Markovian and non-Markovian policies, respectively. \subPOnm uses a stochastic policy that conditions an action on the history. We model the policy using a neural network that maps the history to a distribution over actions, whereas \subPOm maps the state to a distribution over actions. Disregarding sample complexity, we expect \subPOnm %to represents 
perform the best, % one can achieve 
since it can track the complete history. In our experiments, we always compare with  \emph{modular \RL} (\mrl), a baseline that represents standard RL, where the additive reward for any state $s$ is $F(\{s\})$. 
In \mrl, we maximize the undiscounted sum of additive rewards, whereas, in contrast, \subPO maximizes marginal gains. The rest of the process remains the same, i.e., we use the same policy gradient method.
We implemented all algorithms in Pytorch and will make the code and the videos public. We deploy Pytorch's automatic differentiation package to compute an unbiased gradient estimator. Experiment details and extended empirical analysis are in \cref{apx: experiments}. 
% For the first four environments, we consider a finite state action space where the agent takes a high-level action moving up, down, left, right or stay. 
Below we explain our observations for each environment: 

% \subsection{Finite state action domains} 
\mypar{Informative path planning} We simulate a bio-diversity monitoring task, where we aim to cover areas with a high density of gorilla nests with a quadrotor in the Kagwene Gorilla Sanctuary (\cref{fig: gorilla-env}). The quadrotor at location $s$ covers a limited sensing region around it, $\Discat[s]$. 
The quadrotor starts in a random initial state and follows deterministic dynamics. It is equipped with five discrete actions representing directions.
Let $\density:\Domain \to \R$ be the nest density obtained by fitting a smooth rate function \citep{mojmir-cox} over Gorilla nest counts \citep{gorilla-kagwene}. The objective function is given by $\Obj(\traj) = g(\sum_{s \in \traj} \Discat[s])$, where $g(V) = \sum_{v \in V} \density(v)$. As shown in \cref{fig: constant_H40}, we observe that \mrl repeatedly maximizes its modular reward and gets stuck at a high-density region, whereas \subPO achieves performance as good as \subPOnm while being more sample efficient.
To generalize the analysis, we replace the nest density with randomly generated synthetic multimodal functions and observe a similar trend (\cref{apx: experiments}).

\mypar{Item collection} \cref{fig: steiner_covering}, the environment consists of a grid with a group of items $\mathcal{G} = \{\textit{banana, apple, strawberries, watermelon}\}$ located at $g_i\subseteq \V$, $i \in \mathcal{G}$. We consider stochastic dynamics such that with probability 0.9, the action we take is executed, and with probability 0.1, a random action is executed \textit{(up, down, left, right, stay)}. 
The agent has to find a policy that generates trajectories $\traj$, which picks $d_i$ items from group $g_i$, for each $i$. Formally, the submodular reward function can be defined as $F(\traj) = \sum\nolimits_{i \in \mathcal{G}} \min(|\traj \cap g_i|, d_i)$. 
We performed the experiment with 10 different randomly generated environments and 20 runs in each. 
In this environment, the agent must keep track of items collected so far to optimize for future items. Hence as shown in \cref{fig: steiner_plot}, \subPOnm based on non-Markovian policy achieves good performance, and \subPOm achieves a slightly lower but yet comparable performance just by maximizing marginal gains.
% Figure environment removed

\mypar{Bayesian D-experimental design} In this experiment, we seek to estimate an a-priori unknown function $f$. The function $f$ is assumed to be regular enough to be modelled using Gaussian Processes. 
Where should we sample $f$ to estimate it as well as possible? Formally, our goal is to optimize over trajectories $\traj$ that provide maximum mutual information between $f$ and the observations $y_{\traj} = f_{\traj} + \epsilon_{\traj}$ at the points $\traj$. The mutual information is given by $I(y_{\traj};f) = H(y_{\traj}) - H(y_{\traj}|f)$, representing the reduction in uncertainty of $f$ after knowing $y_{\traj}$, where $H(y_{\traj})$ is entropy. We define the monotonic submodular function $F(\traj) = I(y_{\traj};f)$.  
The gorilla nest density $f$ is an a-priori unknown function. We generate 10 different environments by assuming random initialization and perform 20 runs on each to compute statistical confidence.  
In \cref{fig: entropy_H40}, we observe a similar trend that \mrl gets stuck at a high uncertainty region and cannot effectively optimize the information gained by the entire trajectory, whereas \subPOm achieves performance as good as \subPOnm while being very sample efficient due to the smaller search space of the Markovian policy class.


% Figure environment removed

\mypar{Building exploration} The environment consists of two rooms connected by a corridor. The agent at $s$ covers a nearby region $\Discat[s]$ around itself, marked as a green patch in \cref{fig: two_room}. The task is to find a trajectory $\traj$ that maximizes the submodular function $F(\traj) = |\cup_{s\in \traj}\Discat[s]|$. The agent starts in the corridor's middle and has deterministic dynamics. The horizon $H$ is just enough to cover both rooms.
Based on the time-augmented state space, there exists a deterministic Markovian policy that can solve the task. 
However, it is a challenging environment for exploration with Monte Carlo samples using Markovian policies. As shown in \cref{fig: plot_two_room}, \subPO achieves a sub-optimal solution of exploring primarily one side, whereas \subPOnm tracks the history and learns to explore the other room.

\textbf{Car Racing} is an interesting high-dimensional environment, with continuous state-action space, where a race car tries to finish the racing lap as fast as possible \citep{prajapat2021competitive}. The environment is accompanied by an important challenge of learning a policy to maneuver the car at the limit of handling. The track is challenging, consisting of 13 turns with different curvature (\cref{fig: car_track}). The car has a six-dimensional state space representing position and velocities. The control commands are two-dimensional, representing throttle and steering. Detailed information is in the \cref{apx: experiments}. The car is equipped with a camera and observes a patch around its state $s$ as $\Discat[s]$. The objective function is $\Obj(\traj) = |\cup_{s\in \traj}\Discat[s]|$. We set a finite horizon of 700. The \subPOnm will have a large state space of $700 \times 6$, which makes it difficult to train. 
For variance reduction, we use a baseline $b(s)$ in \cref{eqn: grad_esti} that estimates the cumulative sum of marginal gains. 
As shown in \cref{fig: plot_car_racing}, under the coverage-based reward formulation, the agent trained with \mrl tries to explore a little bit but gets stuck with a stationary action at the beginning of the episode to get a maximum modular reward. However, the \subPO agent tries to maximise the marginal again at each timestep and hence learns to drive on the race track (\href{https://youtu.be/jXp0QxIQ--E}{https://youtu.be/jXp0QxIQ--E}). 
Although it is possible to use alternative reward functions to train the car using standard \RL, the main objective of this study is to demonstrate \subPO on the continuous domains and how submodular functions in \RL can provide versatility to achieve surrogate goals.

\mypar{MuJoCo} \looseness -1 The MuJoCo ant task is a high-dimensional locomotion task, as depicted in \cref{fig: mujoco_ant}. The state space dimension is 30, containing information about the robot's global pose and the internal actuator's orientation. The control input dimension is 8, consisting of torque commands for each actuator. The Ant at any location $s$ covers locations in 2D space, $\Discat[s]$ and receives a reward based on it. The goal is to maximize $F(\traj) = |\cup_{s\in\traj} \Discat[s]|$. The results depicted in \cref{fig: plt_mujoco} demonstrate that \subPO maximizes marginal gain and learns to explore the environment, while \mrl learns to stay stationary, maximizing modular rewards. The environment carries the core challenge of continuous control and high-dimensional observation spaces. This experiment shows that \subPO can effectively scale to high-dimensional domains.

% 1. grid experiment 
% 2. steiner covering 
% 3. D-experimental design 
% 4. car example (no non-markovian result) 
% 5. mujoco (no non-markovian result) 
\vspace{-3 mm}
\section{Conclusion}
\vspace{-3 mm}
\section{Conclusion}

This paper investigated the $(\bm{\alpha}, \bm{\beta})$-proportionally fair normalized cut graph partitioning problem.
We proposed a novel algorithm, FNM, consisting of an extended spectral embedding method and a $k$-means-based rounding scheme to provide a node partitioning with a small Ncut value on a graph while strictly following the proportional fairness constraints.
The comprehensive experimental findings confirmed the superior performance of FNM in terms of partition quality, fairness, and efficiency.
In future work, we will generalize our algorithm to handle other notions of fairness, e.g., individual fairness~\cite{MahabadiV20, gupta2022consistency}, in graph partitioning problems.



\section*{Reproducibility Statement}
We have included all of the code and environments used in this study in the supplementary materials. These resources will be made open-source later on. The attached code contains a README.md file that provides comprehensive instructions for running the experiments. Furthermore, \cref{apx: experiments} contains additional emperical results and the parameters to reproduce the results. Regarding the theoretical results, all the proofs of the propositions and the theorems can be found in the appendix.

\subsubsection*{Acknowledgments}
This publication was made possible by an ETH AI Center doctoral fellowship to Manish Prajapat. We would like to thank Mohammad Reza Karimi, Pragnya Alatur and Riccardo De Santi for the insightful discussions. We thank Bhavya Sukhija and Alizée Pace for reviewing the manuscript.

The project has received funding from the European Research Council (ERC) under the European Union’s Horizon 2020 research and innovation program grant agreement No 815943 and the Swiss National Science Foundation under NCCR Automation grant agreement 51NF40 180545.


\bibliography{ref}
\bibliographystyle{iclr2024_conference}

%%%%%%%%%%%%%%%%%%%%%%%%%%%%%%%%%%%%%%%%%%%%%%%%%%%%%%%%%%%%
\newpage
% \section{Supplementary Material}
\appendix
\part{Appendix}
\section{Proof for Propositions}
\label{apx: deterministic_policy}
% For completion, we briefly prove the following propositions about the \mdp with any set function $F$.
\restatedetmarkov \vspace{-0.5em}
\begin{proof} The proof consists of two parts. First, we establish the existence of an optimal deterministic non-Markovian policy for a deterministic MDP $\mathcal{M}$ with non-Markovian rewards $F$. Second, we demonstrate that the trajectory generated by the optimal deterministic \nm policy $(\pi_\nm)$ in $\mathcal{M}$ can also be generated by a deterministic Markovian policy $(\pi_\mv)$. Since both policies yield the same trajectory, resulting in: $J(\pi_{\nm}) = J(\pi_{\mv})$. (Notably, it is sufficient to look at a value of the single induced trajectory since we consider a deterministic policy in a deterministic \mdp with a fixed initial state.)

\emph{i)} We can construct an extended MDP, denoted as $\mathcal{M}_e$, where the state space consists of trajectories $\tau_{0:h}$ for all time steps $h \in [H]$. In $\mathcal{M}_e$, the trajectory rewards $F$ are Markovian. Due to Markovian rewards in $\mathcal{M}_e$, there exists an optimal deterministic Markovian policy for $\mathcal{M}_e$ \citep{puterman_MDPbook} that, when projected back to $\mathcal{M}$, corresponds to a non-Markovian policy $(\pi_\nm)$.

\emph{ii)} Without loss of generality, let $\tau = ((s_0, a_0), (s_1, a_1), \dots, s_H)$ be a trajectory induced by $\pi_{\nm} \in \Pi_{\nm}$. If the states are augmented with time, the same trajectory $\tau$ is induced by a deterministic Markovian policy defined as $\pi_{\mv}(a_i|s_i) = 1$ and $\pi_{\mv}(a_j|s_i) = 0$ for $i \neq j$, $\forall i,j$. Due to the identical induced trajectory, we have $J(\pi_{\mv}) = J(\pi_{\nm})$.
\end{proof}

\restatedetMDP \vspace{-0.5em}
\begin{proof} 
Given any stochastic policy, it is possible to select a state and modify the action distribution at that state to a deterministic action such that the value of the new policy will be at least equal to the value of the original policy. Below we show a construction to ensure the monotonic improvement of the policy,

W.l.o.g., suppose there are two actions, denoted as $a_k$ and $a'_k$, available at a state $s_k$ (the argument remains applicable for any finite number of actions). Consider the objective $J(\pi)$ as shown below (for simplicity, we wrote only the trajectories affected with action at horizon $k$ at state $s_k$),
\begin{align*}
     J(\pi) &= \! \sum_{\traj \in \Gamma}  \mu(s_0) \!\! \left[ \prod_{i=0}^{H-1}  p^i(s_{i+1}|s_i,a_{i})  \pi^i(a_{i}|s_i) \right] \!\! F(\{(s_0,a_0), \hdots (s_k,a_{k}\!) \hdots s_H\}\!) \\
     \!\!\!\!\!\!&= \!\!\!\!\!\!\!\!\!\!\!\! \sum_{\traj \in \Gamma:(s_k,a_{k})\in\traj} \!\!\!\!\!\!\!\!\!\!\!\! \mu(s_0) \!\! \left[ \prod_{i=0, i \neq k}^{H-1}  p^i(s_{i+1}|s_i,a_{i})  \pi^i(a_{i}|s_i) \right] p^k(s_{k+1}|s_k,a_{k}) \underbrace{\pi^k(a_{k}|s_k)}_{x}\! F(\traj)\\
     \!\!\!\!\!\!&+ \!\!\!\!\!\!\!\!\!\!\!\! \sum_{\traj \in \Gamma:(s_k,a'_{k})\in\traj} \!\!\!\!\!\!\!\!\!\!\!\! \mu(s_0) \!\! \left[ \prod_{i=0, i \neq k}^{H-1}  p^i(s_{i+1}|s_i,a_{i})  \pi^i(a_{i}|s_i) \right] p^k(s_{k+1}|s_k,a'_{k}) \underbrace{\pi^k(a'_{k}|s_k)}_{y}\! F(\traj)\\
     &+ constant \tag{remaining terms, do not vary with $x$ and $y$}
\end{align*}
The objective is to maximize $J(\pi)$ subject to simplex constraints. The policy remains fixed for all states except $s_k$. It is important to note that $J$ is a linear function of variables $x$ and $y$. Given that $J$ is linear within the simplex constraints, the optimal solution lies on a vertex of the simplex $(\pi^k(a_{k}|s_k) + \pi^k(a'_{k}|s_k) = 1)$. This vertex represents a deterministic decision for state $s_k$.

We can define a new policy $\pi'$ based on $\pi$ by adjusting the action distribution at $s_k$ to the optimal deterministic action (either $a'_k$ or $a_k$). It is evident that $J(\pi') \geq J(\pi)$. This process can be repeated for all states. By starting with any optimal stochastic policy, we can obtain a deterministic policy that is at least as good as the original stochastic policy. This concludes the proof.
\end{proof}

\newpage
\section{Inapproximability Proof} \vspace{-0.5em}
\label{apx: inapprox}
First, we introduce a set of known hard problems that we will use to establish the hardness of \subrl.

\mypar{Group Steiner tree (\gst)} The input to the group Steiner problem is an edge-weighted graph $G=(V, E, l)$ and $k$ subsets of nodes $g_1, g_2, \hdots, g_k$ called \emph{groups}. Starting from a root node $r$, the goal in \gst is to find a minimum weight tree $T^{\star} = (V(T^{\star}), E(T^{\star}))$ in $G$ such that each group is visited at least once, i.e., $V(T^{\star}) \cap g_i \neq \emptyset, \forall i \in [k]$.

\mypar{Covering Steiner problem (\scp)} The input to \scp is an edge-weighted graph $G=(V, E, l)$ and $k$ subsets of groups $g_1, g_2, \hdots, g_k$, each group has a positive integer $d_i$ representing a minimum visiting requirement. Starting from a root node $r$, the goal in \scp is to find a minimum weight tree $T^{\star} = (V(T^{\star}), E(T^{\star}))$ in $G$ such that the tree covers at least $d_i$ nodes in group $g_i$, i.e., $|V(T^{\star}) \cap g_i| \geq d_i, \forall i \in [k]$. The \scp generalizes \gst problem to an arbitrary constraint $d_i$.

\mypar{Submodular Orienteering Problem (\sop)} In rooted \sop, we are given a root node $r\in V(T)$, and the goal is to find a \emph{walk} $\traj$ of length at most $B$ that maximizes some submodular function $F$ defined on the nodes of the underlying graph. 

\emph{Approximation ratio:} Let $x$ be an input instance of a maximization problem. The approximation ratio $\beta$ is defined as $\beta \geq \frac{\opt(x)}{\algo(x)}$, $\beta \geq 1$, where $\opt$ is the global optimal and $\algo$ is the value attained by the algorithm. The hardness lower bound and approximation upper bounds refer to the lower and upper bound on the approximation ratio $\beta$. 

% Figure environment removed

\restateinapprox \vspace{-1 em}
\begin{proof} 
We reduce \subrl to rooted \sop, demonstrating the inapproximability of \subrl. \cref{lem: rooted_sop} establishes the hardness of rooted \sop.

Given an instance of \sop with a graph $G=(V, E)$ and a root node $r \in V$, the goal is to find a walk $\traj$ that maximizes a submodular function $F(\traj)$, subject to a budget constraint $|\traj|\leq B$. This can be converted to a \smdp (input to \subrl), $\langle \S, \A, P, \rho, \Obj, H \rangle$ tuple in polynomial time as follows:

% To construct an \smdp  from the given \sop instance, we follow these steps in polynomial time:
\textit{i)} Set $\S \leftarrow H \times V$, $\A \leftarrow E$, $H \xleftarrow{} B$, $\rho(0,r) = 1$ and the submodular function $F$ remains unchanged.
\textit{ii)} Iterate over the edges $e \in E$. Let $e=(v',v)$, set $P((h+1, v')|(h,v),a) = 1$, for all $h \in [H-1]$ where action $a = e$. 

The solution to the \subrl problem is a policy $\pi$ that can be rolled out to obtain a solution for the \sop, which is also a polynomial time operation. By assuming the existence of a polynomial time algorithm for \subrl with an approximation ratio $\beta = o(\log^{1-\gamma} \opt)$, we can approximate \sop with $F(\traj) > \frac{F(\traj^{\star})}{\beta}$ (\cref{fig:SOPtoSubRL}). However, this contradicts the fact that rooted \sop cannot be approximated better than $\Omega(\log^{1-\gamma} \opt)$ (\cref{lem: rooted_sop}). Proved by contradiction. \end{proof} \vspace{-0.5em}


% , and the solution of \subrl, a policy $\pi$ can be rolled out to get a walk $\traj$. Assuming a polynomial time algorithm for \subrl with $\beta = o(\log^{1-\gamma} \opt)$, we can approximate \sop with $F(\traj) \geq \frac{\opt}{\beta}$. However, this contradicts the known lower bound of $\Omega(\log^{1-\gamma} \opt)$ for \sop \citep{chekuri_rg}.

% \begin{proof} 
% Following the reduction, as shown in \cref{fig:SOPtoSubRL}, we can use the \subrl algorithm to solve the \sop problem. Given an instance of the \sop, we can construct an \mdp for \subrl from a graph as follows. $(G=(V,E))$ by setting $\S \leftarrow V$, $(\A, \mathcal{P}) \leftarrow E$ and $H \xleftarrow{} B$ and the submodular function $F(.)$ stays the same (\cref{fig:SOPtoSubRL}, step a). The solution of the \subrl is a policy $(\pi)$ that can be rolled out to get a solution for the \sop(\cref{fig:SOPtoSubRL}, step c). By contradiction let's assume we have a polynomial time algorithm for \subrl with approximation ratio $\beta = o(\log^{1-\gamma} \opt)$(\cref{fig:SOPtoSubRL}, step b). Using this, \sop can be approximated with $f(\traj) \geq \frac{f(\traj^{\star})}{\beta}$.  This contradicts the fact that \sop can't be approximated better than $\Omega(\log^{1-\gamma} \opt)$(\cref{fig:SOPtoSubRL}, step d). Hence proved.
% \manish{Discussion on ZTIME quasi-polynomial   }


% \end{proof}

% In this section, we prove that our problem is not solvable beyond the log factors using the NP-Hardness of the Submodular orienteering problem. In SOP, given a graph $G=(V, E)$ and a budget constraint $B$, the agent needs to traverse a path $\traj$, that maximizes a coverage function within a budget constraint of $|\traj| \leq B$.

The results shows that there is no algorithm that can guarantee $J(\pi) \geq \frac{\opt}{\log^{1-\gamma}\opt}$ for all the input instances. As \opt increases with the input size of the problem, the ratio $\frac{1}{\log^{1-\gamma} \opt}$ degrades and hence no algorithm can approximate the problem up to any constant factor $c>0$. For the sake of completeness, we include Theorem 4.1 from \cite{chekuri_rg} and modify it for the rooted \sop. The reduction scheme is the same as \cite{chekuri_rg}. 
\begin{lemma}[Theorem 4.1 from \cite{chekuri_rg}] \label{lem: rooted_sop}
    The rooted submodular orienteering problem (SOP) in undirected graphs is hard to approximate within a factor of $\Omega(\log^{1-\gamma} \opt)$ unless $\NP \subseteq \ZTIME(n^{polylog(n)})$.
\end{lemma} \vspace{-0.5em}
\begin{proof} The group Steiner problem (\gst) is hard to approximate to within a factor of $\Omega(\log^{2-\gamma} \opt)$ unless \NP has quasi-polynomial time Las Vegas algorithms \citep{Halperin2003inapprox}. 
We reduce the problem of rooted \sop to \gst, proving the inapproximability of rooted \sop. This represents that if we have an efficient algorithm for \sop, then we can recover a solution for \gst by using the same \sop algorithm.

\mypar{Submodular function $F$} Given an \scp instance, define a submodular function $F(S) = \sum_{i=1}^{k} \min(d_i, |S\cap g_i|)$. $F$ is a monotone submodular set function. 

Consider an optimal solution of \scp as $T^{\star}$ of cost \opt.
% \manish{This cost can be integral weight or fraction.}.  
We can take an Euler tour of the tree $T^{\star}$ and obtain a tour from $r$ of length at most $2 \opt$ that covers all groups. %\manish{TODO: Draw a figure to explain this}

% Either explain how does an algorithm for \sop will solve \scp or how can a \scp problem will be casted to \sop problem.
\mypar{Reduction} We will reduce rooted \sop problem to the \scp (\scp generalises \gst with any $d_i$>0). Let's say we have an algorithm $\A$ for \sop with $\Omega(\log \sum_i d_i)$. ($\sum_i d_i$ is optimal value for \sop). 
In a single iteration, $\A$ will generate a walk that covers $f(V(T^{\star})/\log f(V(T^{\star}))$ of length $B$, which can be converted to a tour P of length at most $2B$.
We can remove the nodes in $P$ and reduce the coverage requirement of the groups that are partially covered and repeat the above procedure. Using \cref{lem: stand_set_cover_logk}, all groups will be covered up to the requisite amount in $\BigO(\log^2 \sum_i d_i)$ iterations. Combining all the tours yields a tree of length $\BigO(\log^2 \sum_i d_i ) B$ that is a "feasible solution" of the \scp.
% \manish{what does tree length mean? $\BigO(\log^2 \sum_i d_i ) 2B$?} 
$B$ can be evaluated using binary search and is within a constant factor of \opt. %(Think: If B = 1, then the method would not find a solution, if B = \opt then it should be enough? since optimal solution can't be of length more than \opt and \sop shall also return a solution within B) \manish{Ask Mojmir about binary search. B=K\opt, then in approximability ratio \opt will cancel out. This results in a constant factor in the approximability ratio.} 
When specialized to the \gst, i.e., $d_i=1$, the approximability ratio becomes $\BigO(\log^2 k)$. 

\mypar{Contradiction} Following the reduction above, assuming an algorithm $\A$ for \sop with an approximation ratio of $\log k$ results in $\BigO(\log^2 k)$ approximation ratio for \gst. Hence, an $\alpha=o(\log k)$ approximation algorithm for \sop will give an approximation of $\BigO(\alpha \log k)$ for \gst. But \gst is hard to approximate to within a factor of $\Omega(\log^{2-\gamma} \opt)$. Hence \sop is hard to approximate within a factor of $\Omega(\log^{1-\gamma} \opt)$.
% \manish{Give emphasis on $\alpha$-approximation for \sop in undirected graphs implies an $\BigO(\alpha \log k)$ approximation for the group Steiner problem in undirected }
% \begin{itemize}
%     \item overall explain how GST is solved using SOP
% \end{itemize}   
\end{proof}

\begin{lemma} \label{lem: stand_set_cover_logk}
    A algorithm $\A$ for \sop with approximability ratio $\Omega(\log k)$ can cover $k$ nodes after $\BigO(\log^2 k)$ iterations.
\end{lemma}\vspace{-0.5em}
\begin{proof} Let $L_{n}$ be the nodes available after the $n^{th}$ iteration with an algorithm having $\beta \geq \Omega(\log k)$.
    \begin{align*}
        L_0 &\xleftarrow[]{} k\\
        L_1 &\xleftarrow[]{} L_0 - \frac{L_0}{\log L_0} \\
        &\vdots\\
        L_{n+1} &\xleftarrow[]{} \underbrace{L_n - \frac{L_n}{\log L_n}}_{x - \frac{x}{\log x}} \\
    \end{align*} \vspace{-0.5em}
In the first iteration with $x=k$
\begin{align*}
    L_1 = x - \frac{x}{\log x} = x(1-\frac{1}{\log x}) \leq x e^{\frac{-1}{\log x}} = k e^{\frac{-1}{\log k}}.
\end{align*}
By definition, $L_1 \geq L_2 \geq \hdots L_n$, 
hence $L_i \leq k e^{-1/\log k} ~\forall i \in [1,n]$. \\
Nodes available after $n^{th}$ iteration : $L_0 (1-\frac{1}{\log L_0}) (1-\frac{1}{\log L_1}) \hdots (1-\frac{1}{\log L_n})$.

\begin{align*}
    L_0 (1-\frac{1}{\log L_0}) (1-\frac{1}{\log L_1}) \hdots (1-\frac{1}{\log L_n}) &\leq k \underbrace{e^{-1/\log k}\times e^{-1/\log k} \hdots e^{-1/\log k}}_{n~~times}\\
    &= k e^{-n/ \log k} 
\end{align*}

for $n > \log^2 k$, nodes available: $k e^{-n/\log k} < k e^{-\log^2 k/\log k} = k e^{- \log k} = 1$. Hence Proved.
% Hence with algorithm $\A$ having approximability ratio $\Omega(\log k)$, after $\BigO(\log^2 k)$ iterations no nodes are available.
\end{proof}

%The assumption "unless $\NP \subseteq \ZTIME(n^{polylog(n)})$" is a slightly stronger assumption than the common usual $\small{\textsc{P}} \neq \NP$, implying there is no deterministic algorithm in time $n^{polylog(n)}$ that can solve all the problems in $\NP$. However, this is also a widely 

% How critical is it that $F$ is a submodular function for \subPO ?

\section{Discussion}

\mypar{Submodularity} Since the algorithm and the theoretical hardness result readily extend to general set functions beyond submodular rewards, a natural question that arises is how critical is that $F$ is a submodular function and what can we say beyond submodular rewards? In this work, submodularity emerges in the lower bound (inapproximability hardness), implying that the problem is not just intractable but intractable even to approximate up to any constant factor. Additionally, it emerges in the upper bound of $1-c$ under curvature assumption and in the upper bound of $(1-1/e)$ under the simplified \smdp setting. There cannot exist an algorithm for bandit \smdp with better guarantees \citep{Feige-no-other-effi-algo}, and \subPO is able to achieve the optimal ratio $(1-1/e)$, thus utilising submodularity to provide intuition on why \subPO (a REINFORCE type strategy) is a right strategy.

Overall, submodularity lets us characterise the spectrum of the computational complexity of the SubRL framework, while some results, e.g., our algorithm \subPO, inapproximability hardness, naturally carry over to the general non-Markovain rewards beyond submodular $F$ (general non-additive reward function).

\mypar{Policy class} The restriction to Markovian policies in the theoretical limits section is mainly for emphasizing the “hardness result” even for the simple policy class, implying the source of hardness is not the representation of non-Markovian policy (which is an exponential object itself). The overall goal is to learn a policy that achieves a higher objective value; hence, we do not, in general, restrict it to the Markovian policy class. We treat the problem of learning state representation separately, which can be done, e.g. with RNN, and is an add-on to the \subPO, e.g. \subPOnm optimises in a non-Markovian policy class.

\mypar{Expressivity of rewards} The optimal policies for submodular rewards cannot be captured by the Markovian rewards in general since the optimal policies are non-Markovian. 
However, when the policy search is restricted to the Markovian class, the optimal policy is deterministic (\cref{prop:restatedetMDP}), and hence there exists a Markovian reward that would lead to the same optimal policy. But this does not help to solve the problem since finding such Markovian rewards has to be NP-hard to approximate \cref{thm: inappx}.

In contrast to finding such surrogate Markovian rewards, submodularity provides a natural way to capture the task. Moreover, since we do not restrict to the Markovian policies, given a policy class with compact history representation, \subPO can learn behaviours beyond the expressivity of the Markovian rewards \citep{abel2021expressivity}.

\mypar{Applications} Since submodularity is a natural characterization of diminishing returns, numerous tasks involving exploration or discouraging repeated actions can be captured via submodular functions. In addition to our experiments discussing experiment design, item collection and coverage objectives, the following \cref{tab:subrlapplications} provides a summary of problems that can be addressed with \subrl.

% \subsection{Application of \subrl}
\begin{table}[!h]
\hspace{-5em}
\setlength{\arrayrulewidth}{0.5mm}
\setlength{\tabcolsep}{0pt}
\renewcommand{\arraystretch}{1.6}
% \begin{tabular}{ | c{5em} | c{1em} | c{1em} | } 
\begin{tabular}{  c  c  c  } 
  \hline
 Tasks & Relevant works & Submodular reward function $F(\tau)$  \\ 
  \hline 
State entropy exploration  & \citep{Hazan2019} & $
        F(\tau) = \frac{-1}{|\tau|} \sum_{v \in \V} \mathbb{I}_{(v,\cdot)\in\tau} \log \frac{|\{t: (v,t)\in \tau \} |}{|\tau|} 
    $  \\[0.1em] 
D-Optimal Experimental Design &\citep{Mutny2023} &$F(\tau) = I(y_\tau;f)$, $I(y;f) = H(y_\tau) - H(y_\tau | f)$ \\
Steiner covering & \citep{chekuri_rg} & $F(\traj) = \sum\nolimits_{i \in \mathcal{G}} \min(|\traj \cap g_i|, d_i)$, pick $d_i$ items of group $g_i$  \\
% Goal reachability & &$|\tau \cap S_g|$ \\
State coverage functions & \citep{near_optimal_safe_cov}& $F(\tau) = \sum_{v \in \V} | \{ t \in [H]:(v,t)\in \tau \}|$, $F(\tau) =|\bigcup_{v\in \tau} D^v|$  \\
Weighted coverage function & \citep{karimi2017stochastic} & $F(\tau) = g(\bigcup_{s \in \tau} D^s)$, $g(V) = \sum_{v \in V} \rho(v)$ \\
Discourage repeated action/ & \citep{basu2019blocking} & \!\!\!\!\!\!$F(\tau) =|\bigcup_{s\in \tau} D^s|$, e.g., $s\!=\!(v,t)$ and \\[-0.4em]
(including coverage on Time) & &$\!D^s\!\!\coloneqq\!\{(v,t),\!(v,t+1),\!(v,t+2)\}$\\
Log determinant objectives & \citep{wang2020planning} & $F(\traj) = \log \det \left(\sum_{s\in \tau} F(\{s\}) + \lambda I \right)$\\
Facility location & \cite{krause2014submodular} & $F(\tau) = \sum_{i=1} \max_{j \in \tau} M_{i,j}$, $M_{ij} \geq 0$\\
% Exploration under time-varying process & $F(\tau) = \sum_{v \in V} \max \{ \alpha, \min\{ |S \cap S_v|, 1 \} \}, 0 \leq \alpha \leq 1$.\\
\hline
\end{tabular} 
\caption{ A few examples that can be tackled with submodular reinforcement learning framework}
\label{tab:subrlapplications}
\end{table}
\newpage
\section{Submodular policy optimization, \subPOh's policy gradient proof}
\label{apx: PG}
\restatePG
\begin{proof} The performance measure is given by,
\begin{align*}
   J(\pi_{\theta}) &= \E_{\traj \sim f(\traj ; \pi_{\theta})} [F(\traj)] = \sum_{\traj} f(\traj ; \pi_{\theta}) F(\traj)\\
\shortintertext{thus, the gradient with respect to $\theta$ is given by}
   \nabla_{\theta} J(\pi_{\theta}) &= \sum_{\traj} \nabla_{\theta} f(\traj ; \pi_{\theta}) F(\traj) 
\end{align*}
For any $p_{\theta}(\tau) \neq 0$ using log trick, $\nabla_{\theta} \log p_{\theta}(\tau) = \frac{\nabla_{\theta} p_{\theta}(\tau)}{p_{\theta}(\tau)}$ from standard calculus and the definition of $f(\tau;\pi_\theta)$ in \eq~\ref{eq: trajectory_distribution}, we can compute the gradient of the objective. Let us define $g(\traj; \pi_\theta) = \nabla_{\theta} (\log \prod_{i=0}^{\horizon-1}\pi_{\theta}(a_i|s_i))$ resulting in \cref{prop: score},
\begin{align*}
\nabla_{\theta} J(\pi_{\theta}) &= \sum_{\tau} f(\tau;\pi_\theta) \nabla_{\theta} (\log \prod_{i=0}^{\horizon-1}\pi_{\theta}(a_i|s_i)) F(\traj) = \sum_{\tau} f(\tau;\pi_\theta) g(\traj; \pi_\theta) F(\traj)\\
 &= \E_{\traj \sim f(\traj ; \pi_{\theta})} \left[ \left(\sum_{i=0}^{H-1} \nabla_{\theta} \log \pi_{\theta}(a_i|s_i) \right) F(\traj) \right] \numberthis\\
\end{align*}
Using marginal gain $F(s|\traj_{0:j}) = F(\traj_{0:j}\cup \{s\}) - F(\traj_{0:j})$ and telescopic sum $\sum_{j=0}^{H-1}F(s_{j+1}|\traj_{0:j}) = F(\traj) - F(s_0)$,
\begin{align*}
   \nabla_{\theta} J(\pi_{\theta}) &= \E_{\traj \sim f(\traj ; \pi_{\theta})} \left[ \left(\sum_{i=0}^{H-1} \nabla_{\theta} \log \pi_{\theta}(a_i|s_i) \right) \left(\sum_{j=0}^{H-1}F(s_{j+1}|\traj_{0:j}) + F(s_0)\right) \right] \\
    % \E\limits_{\traj \sim f(\traj ; \pi_{\theta})}  \left[  \left(\sum_{i=0}^{H-1} \nabla_{\theta} \log \pi_{\theta}(a_i|s_i) \right) \left( \sum_{j=0}^{H-1} F(s_{j+1}|\traj_{0:j}) + F(s_0) \right) \right] \\
    &= \E\limits_{\traj \sim f(\traj ; \pi_{\theta})} \left[ \sum\limits_{i=0}^{H-1} \nabla_{\theta} \log \pi_{\theta}(a_i|s_i) \left(\sum\limits_{j=0}^{H-1}F(s_{j+1}|\traj_{0:j}) + F(s_0) \right) \right]\\
    \intertext{For any function of partial trajectory up to $i$, $b'(\traj_{0:i})$, we have, $\sum\limits_{a_i}  \pi_{\theta}(a_i|s_i) \nabla_{\theta} \log \pi_{\theta}(a_i|s_i) b'(\traj_{0:i})$ $=  \sum\limits_{a_i} \nabla_\theta \pi_{\theta}(a_i|s_i) b'(\traj_{0:i}) = 0$. Thus one can subtract any history-dependent baseline without altering the gradient estimator,}
    &= \E\limits_{\traj \sim f(\traj ; \pi_{\theta})} \left[ \sum\limits_{i=0}^{H-1} \nabla_{\theta} \log \pi_{\theta}(a_i|s_i) \left(\sum\limits_{j=0}^{H-1}F(s_{j+1}|\traj_{0:j}) + F(s_0) - b'(\traj_{0:i}) \right) \right] \\
    &= \E_{\traj \sim f(\traj ; \pi_{\theta})} \left[ \sum_{i=0}^{H-1} \nabla_{\theta} \log \pi_{\theta}(a_i|s_i) \left(\sum_{j=i}^{H-1}F(s_{j+1}|\traj_{0:j})\right) \right]\\
    % &= \E_{\traj \sim f(\traj ; \pi_{\theta})} \left[ \sum_{i=0}^{H-1} \nabla_{\theta} \log \pi_{\theta}(a_i|s_i) \left(\sum_{j=i}^{H-1}F(s_{j+1}|\traj_{0:j})\right) \right]  \tag{Use Causality} \label{eqn: causality}\\
    \intertext{Finally, we can subtract a baseline again using a similar trick as above and we get the theorem statement:}
    &= \E_{\traj \sim f(\traj ; \pi_{\theta})} \left[ \sum_{i=0}^{H-1} \nabla_{\theta} \log \pi_{\theta}(a_i|s_i) \left(\sum_{j=i}^{H-1}F(s_{j+1}|\traj_{0:j}) - b(\traj_{0:i}) \right) \right]
\end{align*}
\end{proof}


\newpage
\section{Provable Guarantees in Simplified Settings}
\label{apx: dr_sub}
\subsection{DR-Submodularity Proof}
% Figure environment removed
\restateDRsub
\begin{proof}
$\epsilon$-Bandit \mdp considers \emph{state independent transitions} (only horizon dependent), i.e., $v_j, v_k \in \V$, $j \neq k$, $\forall h \in [H]$ and $\forall v' \in \V$, $P_h(v_j|v',a_{j})= 1-\epsilon_h$, and $P_h(v_k|v',a_{j}) = \frac{\epsilon_h}{|\V|-1}$ for $\epsilon_h \in \left[0, \frac{|\V|}{|\V|+1}\right]$. For simplicity, we consider a fixed size $\V$, but it can also vary with the horizon. To denote explicit dependent on the horizon, we use $P_h$ and $v$ instead of directly $s$.

Similarly, policy parameterization in \cref{thm: dr_sub} considers \emph{state independent policy} (only horizon dependent) $\pi^h(a|v') = \pi^h(a|v'') \forall h\in [H], \forall v',v'' \in \V$, in short notation we denote them as $\pi^h(a)$. 

Note that $\pi^{h}(a_{i}|v_i)$ corresponds to the self-loop actions ("stay"). In the proof, we reparameterize the probability of self-loop actions with that of other actions (i.e., $\pi^{h}(a_{i}|v_i) = 1 - \sum_{a \neq a_i}\pi^h(a|v_i)$), resulting in relaxation of the simplex constraint $\left(\sum_{a} \pi^h(a|v)=1, \forall h , v \to \sum_{a \neq a_i} \pi^h(a|v_i) \leq 1 \right)$. %of technique: we will substitute  and differentiate \cref{eq: obj} wrt $\pi^{h}(a_{i+1}|s_i)$. 

In the following, for ease of notation, we denote $v_i\coloneqq (i,v)$, in particular, $F((v'_i,a_i)_{i=0}^{H-1}, v_H) \coloneqq F((i,v',a_i)_{i=0}^{H-1}, (H,v))$. Consider the objective $J(\pi)$,
\begin{align*}
    J(\pi) &= \sum_{\traj \in \Gamma} \mu(v_0) \prod_{h=0}^{H-1} p_h(v_{h+1}|v_h,a_{h}) \pi^h(a_{h}|v_h) F((v_i,a_i)_{i=0}^{H-1}, v_H) \\ %\!\! F(\{(s_0,a_0), \hdots \underbrace{(s_i,a_{i}\!),(s_{i+1},a_{i+1}\!),(s_{i+2},a_{i+2}\!)}_{A} \hdots s_H\}\!) 
    &= \sum_{\traj \in \Gamma} \mu(v_0) \prod_{h=0}^{H-1} p_h(v_{h+1}|a_{h}) \pi^h(a_{h}) F((v_i,a_i)_{i=0}^{H-1},v_H) \tag{state independent assumptions}
\end{align*}
% Note in the equation above notation is overloaded, $v_h$ is a state at horizon $h$ and also a state where a unique action $a_h$ is a self-loop one. %$v_{\traj_h}?$

We show DR-submodularity by showing $\forall \pi \in \mP, \frac{\partial^2 J(\pi)}{\partial \pi' \partial \pi''} \leq 0, \forall \pi', \pi''\in \mP$. 
% For this, consider the gradient of the objective, (
We first reparameterize the self-loop actions (which bring the agent back to the same state) in $J(\pi)$ by substituting $\pi^{h}(a^l_{h}|v^l_h) = 1 - \sum_{a \neq a^l_h}\pi^h(a|v^l_h)$ in $J(\pi)$. Here $a^l$ is a looping action for state $v^l$ at horizon $h$.
% Consider $J(\pi)$ with all the self-loop actions (which bring back to the same state) being reparameterized as per $\pi^{h}(a^l_{h}|v^l_h) = 1 - \sum_{a \neq a^l_h}\pi^h(a|v^l_h)$. 

First, we prove monotonicity of $J(\pi)$ by showing $\frac{\partial J(\pi)}{\partial \pi^{h}(a'_{h})} \geq 0$. %We consider the terms in $\frac{\partial J(\pi)}{\partial \pi^{h}(a'_{h})}$ corresponding to the state $v^l$ and show that they are $\geq 0$. This can be sum over all the states to get $\frac{\partial J(\pi)}{\partial \pi^{h}(a'_{h})} \geq 0$.
\begin{align*}
    \frac{\partial J(\pi)}{\partial \pi^{h}(a'_{h})} = \sum_{v^l\in \V}\frac{\partial J_{v^l}(\pi)}{\partial \pi^{h}(a'_{h})}, \mathrm{where},
\end{align*}
\begin{align*}
% J(\pi) &= \sum_{\traj \in \Gamma} \mu(s_0) \left[\prod_{i=0}^{H-1} p^i(s_{i+1}|s_i,a_{i+1}) \pi^{i}(a_{i+1}|s_i) \right] F(\{(s_0,a_1), (s_1,a_2), \hdots s_H\})\\
& \frac{\partial J_{v^l}(\pi)}{\partial \pi^{h}(a'_{h})} = \\%\!\!\!\!\!\!\!\sum_{\substack{\traj \in \Gamma:(v_h,a'_{h})\in\traj,\\ \forall v \in |\V|, v \neq v^l}} \!\!\!\!\!\!\!\!\!\!\!\! \mu(v_0) \!\! \prod_{i=0}^{H-1} \!\! p_i(v_{i+1}|a_{i}) \!\! \left[\!\prod_{h\neq i}^{H-1}  \!\!\! \pi^h(a_{h}) \right] \!\! F(\traj) - \!\!\!\!\!\!\!\!\!\!\!\!\sum_{\substack{\traj \in \Gamma:(v_h,a^l_{h})\in\traj,\\ \forall v \in |\V|, v \neq v^l}} \!\!\!\!\!\!\!\!\!\!\!\! \mu(v_0) \!\! \prod_{i=0}^{H-1} \!\! p_i(v_{i+1}|a_{i}) \!\! \left[\!\prod_{h\neq i}^{H-1}  \!\!\! \pi^h(a_{h}) \right] \!\! F(\traj) \\ 
\!\!\!\!\!\!\!& \!\!\!\!\!\! \sum_{\traj \in \Gamma:(v^l_h,a'_{h})\in\traj} \!\!\!\!\!\!\!\!\!\!\!\! \mu(v_0) \!\! \prod_{i=0}^{H-1} \!\! p_i(v_{i+1}|a_{i}) \!\! \left[\!\prod_{h\neq i}^{H-1}  \!\!\! \pi^h(a_{h}) \right] \!\! F((v_0,a_0), \hdots \underbrace{(v^l_h,a'_{h}\!),(v'_{h+1},a_{h+1}\!),(v_{h+2},a_{h+2}\!)}_{A} \hdots v_H\!)\\
\!\!\!\!\!\!\!&- \!\!\!\!\!\!\!\!\!\!\!\! \sum_{\traj \in \Gamma:(v^l_h,a^l_{h})\in\traj} \!\!\!\!\!\!\!\!\!\!\!\! \mu(v_0) \!\! \prod_{i=0}^{H-1} p_i(v_{i+1}|a_{i}) \!\!\left[\!\prod_{h\neq i}^{H-1} \!\!\! \pi^h(a_{h}) \right] \!\! F((v_0,a_0), \hdots \underbrace{(v^l_h,a^l_{h}),(v^l_{h+1},a_{h+1}),(v_{h+2},a_{h+2})}_{B} \hdots v_H\!) 
\intertext{
We prove $\frac{\partial J_{v^l}(\pi)}{\partial \pi^{h}(a'_{h})} \geq 0$ for any $v^l$, which implies $\frac{\partial J(\pi)}{\partial \pi^{h}(a'_{h})}\geq0$. Note: For every trajectory in $E \coloneqq \{\traj \in \Gamma:(v^l_h, a'_{h})\in\traj\}$ ($1^{st}$ summation), we have a trajectory in $L \coloneqq \{\traj \in \Gamma:(v^l_h, a_{h})\in\traj\}$ ($2^{nd}$ summation) that differ only in $v'_{h+1}$ and $v^l_{h+1}$ and all other states are exactly same. For every trajectory in L, there is a trajectory in E with a higher value.


We define a short notation $f^{-}_{h} \coloneqq \mu(v_0) \prod\limits_{i \neq h}^{H-1} p_i(v_{i+1}|a_{i}) \pi^i(a_{i}) $, denotes trajectory distribution ignoring transition at $v_h$. }
&= \!\!\!\!\! \sum_{\traj \in \Gamma:(v^l_h,a'_{h})\in\traj} \!\!\!\!\!\!\!\!\! {f'}^{-}_{h} p_h(v'_{h+1}|a'_h) F((v_0,a_0), \hdots (v^l_h,a'_{h}),(v'_{h+1},a_{h+1}),(v_{h+2},a_{h+2}) \hdots v_H)\\
&- \!\!\!\!\! \sum_{\traj \in \Gamma:(v^l_h,a^l_{h})\in\traj} \!\!\!\!\!\!\!\!\! {f^l}^{-}_{h} p_h(v^l_{h+1}|a^l_{h})  F((v_0,a_0), \hdots (v^l_h,a^l_{h}),(v^l_{h+1},a_{h+1}),(v_{h+2},a_{h+2}) \hdots v_H) \numberthis \label{eqn: proof-cancel-out}

\intertext{
Note ${f'}^{-}_{h}$ and ${f^l}^{-}_{h}$ are equal due to state independent transition and policy.

Drop the actions (the function $F$ depends on the states) and let $R \coloneqq \traj \backslash v'$,}
 % only, we can drop actions and denote by $F'(.)$.}
&= \sum_{\traj \in \Gamma:(v^l_h,a'_{h})\in\traj} {f'}^{-}_{h} ( 1 - \epsilon_h - \frac{\epsilon_h}{|\V|-1}) \left( F( R \cup \{v'_{h+1}\}) - F( R)\right) \geq 0   \tag{since, $\epsilon_h \leq \frac{|\V|-1}{|\V|}$}
\end{align*}
The reason for $( 1 - \epsilon_h - \frac{\epsilon_h}{|\V|-1}):$ $(1-\epsilon_h) = p_h(v'_{h+1}|a'_{h})=p_h(v^l_{h+1}|a^l_{h})$ and $\frac{\epsilon_h}{|\V|-1} = p_h(v'_{h+1}|a^l_{h})= p_{h}(v^l_{h+1}|a'_{h})$.
In \cref{eqn: proof-cancel-out}, the two terms (corresponding to set L and E) are subtracted, with probability $(1-\epsilon_h)$ the first term will be larger since $F( R \cup \{v'_{h+1}\}) - F( R) \geq 0$ and with probability $\frac{\epsilon_h}{|\V|-1}$ the second term will be larger since the stochastic transition (e.g., looping action $a^l_h$ can jump to next state $v'_{h+1}$ and $a'_{h}$ stays to the same state $v^l_{h+1}$. This can happen with probability $\frac{\epsilon_h}{|\V|-1}$). In other stochastic transitions, in expectation, the two terms will sum to zero. 

In the above, we have proved the monotonicity of $J(\pi)$ given $F(\cdot)$ is a monotone function. We have,
\begin{align*}
       \frac{\partial J(\pi)}{\partial \pi^{h}(a'_{h})} &= \sum_{\traj} \mu(v_0) ( 1 - \frac{|\V|\epsilon_h}{|\V|-1}) \prod_{h=0}^{H-1} p_h(v_{h+1}|a_{h}) \left[\prod_{h\neq i}^{H-1} \pi^h(a_{h}) \right] \big( F( R \cup \{v'_{h+1}\}) - F( R)\big) \geq 0 
\end{align*}

To obtain the hessian terms, we can follow the same process as above at some horizon $g$ and state $a''$. 
Let $R \coloneqq \traj \backslash (v',v'')$, ($v''$ is the state corresponding to action $a''$),
\begin{align*}
    \frac{\partial^2 J(\pi)}{\partial \pi^g(a''_{g}) \partial \pi^{h}(a'_{h})} &= \sum_{\traj \in \Gamma} f'^{-}_{g,h} ( 1 - \frac{|\V|\epsilon_g}{|\V|-1}) ( 1 - \frac{|\V|\epsilon_h}{|\V|-1}) \\ 
    &\quad\quad\Big( F( R \cup \{v''_{g+1}, v'_{h+1}\}) - F( R \cup \{v'_{h+1}\}) - ( F( R \cup \{v''_{g+1}\})) - F(R)) \Big)\\
    &= \sum_{\traj \in \Gamma} f'^{-}_{g,h} ( 1 - \frac{|\V|\epsilon_g}{|\V|-1}) ( 1 - \frac{|\V|\epsilon_h}{|\V|-1}) \\
    &\quad \quad\bigg( F( \underbrace{R \cup \{v''_{g+1}\}}_{A} \cup \{v'_{h+1}\}) - F( \underbrace{R \cup \{v''_{g+1}\}}_{A}) - ( F( R \cup \{v'_{h+1}\}) - F(R)) \bigg) \\
    &\leq 0 \tag{ By submodularity of $F$, $R \subseteq A$, $\Delta_F(v'|A) \leq \Delta_F(v'|R)$ }
\end{align*}
Hence $J(\pi)$ is monotone DR-submodular.
\end{proof}


\subsection{Bounded Curvature}
\label{apx: proof_curvature}
\restateboundedC
\begin{proof} 
Consider the objectives $J(\pi)$ and $H(\pi)$ defined with submodular reward $F(\traj)$ and its corresponding modular rewards $F_m(\traj) = \sum_{s\in \traj} F(s)$ respectively,
\begin{align*}
    J(\pi) = \sum_{\traj} f(\traj; \pi) F(\traj),~~\mathrm{and}~~H(\pi) = \sum_{\traj} f(\traj; \pi) F_m(\traj).
\end{align*}
For any policy $\pi$, $J(\pi) \geq (1-c) H(\pi)$ using curvature definition. Consider $\nabla_{\pi} J(\pi)$,
\begin{align*}
    \nabla_{\pi} J(\pi) & = \sum_{\traj} \nabla_{\pi} f(\traj; \pi) F(\traj) \\
   & =  \sum_{\traj} \nabla_{\pi} f(\traj; \pi) \left( \sum_{i=0}^{|\traj|-1} F(s_{i+1}|\traj_{0:i}) + F(\{s_0\})\right) \\
   % &\geq \sum_{\traj} \nabla_{\pi} f(\traj; \pi) \left( \sum_{i=0}^{|\traj|-1} (1-c) F(\{s_{i+1}\}) + F(\{s_0\})\right)\\
   &\geq  \sum_{\traj} \nabla_{\pi} f(\traj; \pi) \left( \sum_{i=0}^{|\traj|} (1-c) F(\{s_{i}\})\right) \tag{using curvature definition}\\
   &=  (1-c) \sum_{\traj} \nabla_{\pi} f(\traj; \pi) F_m(\traj) = (1-c) \nabla_{\pi} H(\pi). 
\end{align*}
Similarly, since $F_m(\traj)\geq F(\traj)\geq (1-c)F_m(\traj)$, the following holds component-wise, 
\begin{align}
      \nabla_{\pi} H(\pi)|_{\pi = \pi'} \geq \nabla_{\pi} J(\pi)|_{\pi = \pi'} \geq  (1-c) \nabla_{\pi} H(\pi)|_{\pi = \pi'}  ~~\forall \pi'. \label{eg: gradient_sandwich}
\end{align}
At the convergence of \subPO, the stationary point $\pi$ satisfies, 
\begin{align*}
    &\max_{\pi' \in \Pi}\langle \nabla_{\pi} J(\pi), \pi'-\pi\rangle \leq 0 \\
    \implies &\max_{\pi' \in \Pi}\langle \nabla_{\pi} H(\pi), \pi'-\pi\rangle \leq 0. \tag{using \cref{eg: gradient_sandwich}, $c\neq1$}%\implies \max_{\pi' \in \kappa}\langle Q_\pi, \pi'-\pi\rangle_{\eta_\pi} \leq 0 
\end{align*}
Hence $\pi$ is also a stationary point for modular objective $H(\pi)$. 
Under mild regularity assumptions, any stationary point of the policy gradient cost function with modular rewards is a global optimum \citep{bhandari2019global}. Let $\pi$ be the policy where \subPO converges, then,
    \begin{align}
    J(\pi) = \sum_{\traj} f(\traj; \pi) F(\traj) &\geq (1-c) \sum_{\traj} f(\traj; \pi) F_m(\traj) \label{eq:curv_bound}\\
    &\geq (1-c) \sum_{\traj} f(\traj; \pi^{\star}) F_m(\traj) \label{eq:stationary_point}\\
    &\geq (1-c) \sum_{\traj} f(\traj; \pi^{\star}) F(\traj) = (1-c)J(\pi^{\star}) \label{eq:Fm_geq_F}
        % J(\pi) &\geq (1-c)J(\pi^\star)
    \end{align}
\cref{eq:curv_bound} follows using curvature definition for any policy $\pi$. \cref{eq:stationary_point} follows since $\pi$ is optimal of $H(\pi)$ where as $\pi^{\star}\in \Pi_{\nm}$ is optimal for $J(\pi)$. Finally, \cref{eq:Fm_geq_F} follows since $F_m(\traj) \geq F(\traj)$.
\end{proof}    

\newpage
\section{Experiments}
\section{Experiments}
% \haizhou{Follow the same way of introduction as we did in Section2.}
% \noindent In this section, we will introduce datasets and experimental setups that we used. Then we evaluate our method, other self-supervised methods, and supervised methods under different distribution shifts (\ie, concept shifts and covariate shifts) under common settings (\ie, transductive, inductive settings). It has to note that we focus on node-level tasks (\eg, node classification) in this work. As for graph-level tasks, we leave it as our future work and some simple experiments can be found in Appendix~\ref{app:graph_classification}. 
In this section, we first introduce the experimental setup including datasets, training, and evaluation protocol in Section~\ref{sec:dataset}~and~\ref{sec:unsupervised}. 
% Next, we present our experimental setup and conduct extensive experiments to evaluate our method in Section~\ref{sec:unsupervised}. 
We then perform an ablation study to demonstrate the effectiveness of each proposed component in Section~\ref{sec:ablation}. 
Additionally, we analyze the impact of important hyper-parameters in Section~\ref{sec:sensitivity}. 
Subsequently, we integrate our method with various encoding models, showcasing the model-agnostic nature of our recipe in Section~\ref{sec:other_models}. 
Finally, we provide some qualitative results such as feature visualization in Section~\ref{sec:vis}.
It is important to note that we focus on node-level tasks (\eg, node classification) in this work. As for graph-level tasks, we leave it as our future work, while some simple experiments are also provided in Appendix~\ref{app:graph_classification}.

\subsection{Datasets}\label{sec:dataset}
There exist some benchmarks for evaluating graph out-of-distribution generalization~\cite{good,ji2022drugood,gds}. 
Among them, GOOD~\cite{good} is the most representative and comprehensive benchmark that curates more diverse graph datasets with diverse tasks, including single/multi-task graph classification, graph regression, and node classification involving more distribution shifts (\ie, concept shifts and covariate shifts). Hence in this work, we follow the evaluation protocol proposed in \cite{good}. Furthermore, we validate the effectiveness of our method in the datasets (\ie, Amazon-Photo, Elliptic) that are used in EERM~\cite{eerm}. The statistics and detailed introduction to these datasets can be found in Table~\ref{tab:dataset} and Appendix~\ref{app:datasets}.

\begin{table*}[htp]
\caption{The descriptions of datasets. ``Domain-Level'' means splitting by graphs, ``Time-Aware'' denotes splitting according to chronological order.``Word'' and ``Degree'' represent splitting according to word diversity and node degree respectively. ``Language'' means splitting by user language, suggesting the prediction should not be impacted by the language the user use. ``University'' denotes splitting according to the domain university, implying that the prediction of webpages should be based on word contents and link connections rather than university features. ``Color'' means that nodes are split according to node differences in covariate shift and color-label correlations in concept shift.}
\label{tab:dataset}
\centering
\begin{tabular}{cccccccc}
\toprule
Datasets     & Network Type        & \#Nodes & \#Edges & \#Attributes &\#Classes& Train/Val/Test Split     & Metric   \\
% Cora         & Artificial Transformation & 2,703   &         &              &         &                      & Accuracy \\
Amazon-Photo\footnotemark
             & Co-purchasing network      & 7,650   & 119,081   & 755          & 10      & Domain-Level         & Accuracy \\
Elliptic\footnotemark  
             & Bitcoin transactions       & 203,769 & 234,355   & 165          & 2       & Time-Aware           & F1-Score \\
GOOD-Cora    & Scientific publications    & 19,793  & 126,842   & 8,710         & 70      & Word/Degree          & Accuracy \\
% GOOD-Arxiv   & arXiv papers               & 169,343 & 2,315,598 & 128          & 40      & Time/Degree          & Accuracy \\
GOOD-Twitch  & Gamer network              & 34,120  & 892,346   & 128          & 2       & Language             & ROC-AUC  \\
GOOD-CBAS    & A BA-house graph           & 700     & 3,962     & 4             & 4       & Color                & Accuracy \\
GOOD-WebKB   & Webpage network            & 617     & 1,138     & 1,703         & 5       & University           & Accuracy \\
\bottomrule
\end{tabular}
\end{table*}
\footnotetext[5]{This dataset is adopted from~\cite{yang2016revisiting}. \cite{eerm} constructs ten graphs with different environment id’s for each graph.} 
\footnotetext[6]{The original is available on \hyperlink{https://www.kaggle.com/ellipticco/elliptic-data-set}{https://www.kaggle.com/ellipticco/elliptic-data-set}}

\subsection{Unsupervised Representation Learning}\label{sec:unsupervised}
\subsubsection{Transductive Setting}~\label{sec:trans}
% \noindent\textbf{Baselines.}\quad We conduct experiments with 12 baselines which consist of three categories: supervised methods and self-supervised generative methods, self-supervised contrastive methods. Specifically, we compare with three supervised baselines: empirical risk minimization~(ERM)~\cite{erm}, invariant risk minimization (IRM)~\cite{irm}, and a recent proposed graph OOD method dubbed EERM~\cite{eerm}. We also compare various unsupervised node-level representation learning methods: three self-supervised generative methods including GAE~\cite{gae}, VGAE~\cite{gae}, GraphMAE~\cite{gmae} and seven self-supervised contrastive methods: DGI~\cite{dgi}, MVGRL~\cite{mvgrl}, GRACE~\cite{grace}, RoSA~\cite{rosa}, BGRL~\cite{bgrl}, COSTA~\cite{costa}, SwAV~\cite{swav}. The descriptions of these methods can be found in Appendix~\ref{app:baselines}.
In this subsection, we focus on validating our proposed algorithm under the transductive setting, where the test nodes will participate in message passing~\cite{gilmer2017neural} during training following~\cite{good}. 

\noindent\textbf{Baselines.} We conduct experiments with 12 baselines from three categories: (i)~supervised methods, including empirical risk minimization~(\textbf{ERM})~\cite{erm}, invariant risk minimization (\textbf{IRM})~\cite{irm}, and a recent proposed graph OOD method \textbf{EERM}~\cite{eerm}; (ii)~self-supervised generative methods including Graph Autoencoder (\textbf{GAE})~\cite{gae}, Variational Graph Autoencoder (\textbf{VGAE})~\cite{gae}, Self-Supervised Masked Graph Autoencoders (\textbf{GraphMAE})~\cite{gmae}; (iii)~self-supervised contrastive methods including Deep Graph Infomax (\textbf{DGI})~\cite{dgi}, Contrastive Multi-View Representation Learning on Graphs (\textbf{MVGRL})~\cite{mvgrl}, Deep Graph Contrastive Representation Learning (\textbf{GRACE})~\cite{grace}, A Robust Self-Aligned Framework for Node-Node Graph Contrastive Learning (\textbf{RoSA})~\cite{rosa}, Bootstrapped Representation Learning on Graphs (\textbf{BGRL})~\cite{bgrl}, Covariance-Preserving Feature Augmentation for Graph Contrastive Learning (\textbf{COSTA})~\cite{costa}, Unsupervised Learning of Visual Features by Contrasting Cluster Assignments (\textbf{SwAV})~\cite{swav}. The detailed descriptions of these baselines can be found in Appendix~\ref{app:baselines}.

\noindent\textbf{Experimental setup.} We use the same graph encoder across different datasets for a fair comparison following~\cite{good}. We use grid search to find other hyper-parameters (\eg, learning rate, epochs) for different methods. For all experiments, we select the best checkpoints for ID and OOD tests according to results on ID and OOD validation sets following~\cite{good}, respectively. Experimental details and hyper-parameter selections are provided in Appendix~\ref{app:hyper}. For evaluating unsupervised methods, a linear classifier will be built on the frozen trained encoder after finishing pre-training. The reported results are the mean performance with standard deviation after 10 runs following~\cite{good}.

\noindent\textbf{Analysis.}\quad Based on the experimental results listed in Table~\ref{tab:trans_concept} and \ref{tab:trans_covariate}, we can draw the following conclusions: firstly, we find strong self-supervised methods (\eg, GRACE, BGRL, COSTA) are more robust to distribution shifts (concept shift in Table~\ref{tab:trans_concept} and covariate shift in Table~\ref{tab:trans_covariate}) compared to supervised methods. For instance, on GOOD-CBAS and GOOD-WebKB datasets, GRACE surpasses the best supervised method by large margins (over 6\% absolute improvement). Interestingly, we find the methods designed for OOD generalization (\ie, IRM) and graph OOD generalization (\ie, EERM) do not attain superior performance than the standard ERM on most of the datasets. For example, EERM shows superior OOD performance compared to ERM in only one experiment, and IRM outperforms ERM in four out of ten experiments across the conducted evaluations. This phenomenon is also observed in \cite{good,ahuja2020empirical,rosenfeld2021risks}, showcasing the challenge of achieving invariant prediction in non-Euclidean graph settings. 

Furthermore, our method surpasses other SOTA self-supervised methods on the OOD test set of all datasets by a considerable margin while achieving comparable performance in the in-distribution test set. For instance, on small datasets such as GOOD-CBAS and GOOD-WebKB, our method outperforms GRACE\footnote{MARIO is built up on GRACE according to our recipe. So, we make a comparison with GRACE here.} by over 2\% absolute accuracy on the OOD test set. On larger datasets such as GOOD-Cora and GOOD-Twitch, our method still outperforms other methods which shows its superiority. For instance, under covariate shift, MARIO surpasses other methods by over 7\% absolute accuracy on the GOOD-Twitch OOD test set. These statistics prove the effectiveness of our design.


\begin{table*}[htp]
\caption{Experimental results of all methods under concept shift. The bold font means the top-1 performance and the underline represents the second performance across the unsupervised methods. 'ID' represents in-distribution test performance and 'OOD' means out-of-distribution test performance. (OOM: out-of-memory on a GPU with 24GB memory)}
\label{tab:trans_concept}
\centering
\scalebox{0.95}{
\begin{tabular}{l|cc|cc|cc|cc|cc}
\toprule
\toprule
\multirow{3}{*}{concept shift} & \multicolumn{4}{c|}{GOOD-Cora}                   & \multicolumn{2}{c|}{GOOD-CBAS} & \multicolumn{2}{c|}{GOOD-Twitch} & \multicolumn{2}{c}{GOOD-WebKB} \\
                           & \multicolumn{2}{c}{word} & \multicolumn{2}{c|}{degree}& \multicolumn{2}{c|}{color}    & \multicolumn{2}{c|}{language}   & \multicolumn{2}{c}{university} \\
                           & ID         & OOD         & ID          & OOD          & ID            & OOD           & ID             & OOD            & ID            & OOD            \\
\midrule
ERM                        & 66.38±0.45 & 64.44±0.18  & 68.60±0.40  & 60.76±0.34   & 89.79±1.39    & 83.43±1.19    & 80.80±1.00     & 56.92±0.92     & 62.67±1.53    & 26.33±1.09     \\
IRM                        & 66.42±0.41 & 64.29±0.31  & 68.57±0.35  & 61.45±0.24   & 89.64±1.21    & 82.29±1.14    & 78.87±1.04     & 59.30±1.79     & 62.67±1.10    & 26.88±1.42     \\
EERM                       & 65.10±0.44 & 62.45±0.19  & 66.95±0.44  & 56.58±0.25   & 79.07±2.12    & 64.50±1.01    & OOM            & OOM            & 62.50±2.01    & 28.07±3.23      \\
\midrule
% Random-Init                & 37.53±1.74 & 32.12±1.24  & 37.82±1.71  & 27.74±1.14   &               &               &                &                & 60.33±2.21    & 27.07±1.70     \\
GAE                        & 60.65±0.89 & 58.00±0.55  & 62.59±1.11  & 53.44±0.80   & 75.28±1.36    & 68.07±2.05    & 81.25±0.81     & 51.51±1.05     & 62.17±3.34    & 25.78±1.85     \\
VGAE                       & 63.19±0.53 & 60.35±0.47  & 61.65±0.66  & 54.28±0.28   & 76.50±0.50    & 59.07±0.56    & 80.46±0.53     & 55.56±4.53     & 62.50±2.38    & 24.40±2.57     \\
GraphMAE                   & \underline{66.44±0.46} & \underline{64.87±0.30}  & 67.95±0.46  & 59.41±0.39   & 89.14±0.89    & 82.93±0.93    & 80.05±0.64     & 59.38±1.49     & 61.83±3.37    & 29.27±2.15     \\
DGI                        & 63.33±0.56 & 60.71±0.49  & 65.93±1.02  & 55.83±0.53   & 91.22±1.47    & 85.00±1.66    & 80.05±0.87     & 59.16±1.88     & 61.83±2.83    & 28.63±1.92      \\
MVGRL                      & OOM        & OOM         & OOM         & OOM          & 88.57±1.15    & 76.50±1.17    & OOM            & OOM            & 62.00±3.79    & 28.26±4.20     \\
GRACE                      & 65.61±0.61 & 63.92±0.44  & \textbf{68.59±0.35}  & 60.15±0.45   & 92.00±1.39    & 88.64±0.67    & \textbf{83.43±0.63}     & \underline{60.45±1.46}     & 64.00±3.43    & \underline{34.86±3.43}  \\
RoSA                       & 64.06±0.67 & 62.44±0.39  & 67.07±0.65  & 57.68±0.44   & 90.78±2.27    & 85.93±2.14    & 82.39±0.42     & 57.45±2.16     & 64.17±4.10    & 32.20±2.15     \\
BGRL                       & 65.18±0.43 & 63.43±0.45  & 66.83±0.80  & 59.63±0.38   & 92.36±1.16    & 87.14±1.60    & 82.52±0.60     & 55.48±1.48     & 63.67±2.33    & 31.47±3.43     \\
COSTA                      & 65.05±0.80 & 62.37±0.45  & 66.76±0.87  & 55.73±0.36   & \underline{93.50±2.62}    & \underline{89.29±3.11}    & 83.15±0.30 & 55.03±3.22     & 61.66±2.58    & 32.39±2.13 \\
% ArCL                       &            &             & 67.64±0.57  & 59.71±0.44   &               &               &                &                & 65.00±3.94    & 35.41±1.97 \\      
SwAV                       & 62.22±0.53 & 59.79±0.53  & 64.65±0.94  & 55.06±0.39   & 89.00±0.79    & 81.72±0.66    & \underline{83.32±0.15}     & 59.69±1.97     & \underline{65.17±3.76}    & 29.36±2.01    \\
\midrule
MARIO                       & \textbf{67.11±0.46} & \textbf{65.28±0.34}  & \underline{68.46±0.40}  & \textbf{61.30±0.28}   & \textbf{94.36±1.21}    & \textbf{91.28±1.10}    & 82.31±0.54     & \textbf{63.33±1.72}     & \textbf{65.67±2.81}    & \textbf{37.15±2.37}     \\
\bottomrule
\end{tabular}}
\end{table*}

\begin{table*}[htp]
\caption{Experimental results of all methods under covariate shift. The bold font means the top-1 performance and the underline represents the second performance across the unsupervised methods. 'ID' represents in-distribution test performance and 'OOD' means out-of-distribution test performance. (OOM: out-of-memory on a GPU with 24GB memory)}
\label{tab:trans_covariate}
\centering
\scalebox{0.95}{
\begin{tabular}{l|cc|cc|cc|cc|cc}
\toprule
\toprule
\multirow{3}{*}{covariate shift} & \multicolumn{4}{c|}{GOOD-Cora}                                   & \multicolumn{2}{c|}{GOOD-CBAS} & \multicolumn{2}{c|}{GOOD-Twitch} & \multicolumn{2}{c}{GOOD-WebKB} \\
                           & \multicolumn{2}{c}{word} & \multicolumn{2}{c|}{degree}& \multicolumn{2}{c|}{color}    & \multicolumn{2}{c|}{language}   & \multicolumn{2}{c}{university} \\
                           & ID         & OOD         & ID          & OOD          & ID            & OOD           & ID             & OOD            & ID            & OOD            \\
\midrule
ERM                        & 70.50±0.41 & 64.69±0.33  & 72.46±0.49  & 55.53±0.50   & 92.00±3.08    & 77.57±1.29    & 70.98±0.41     & 49.35±5.09     & 39.34±1.79    & 14.52±3.14   \\
IRM                        & 70.48±0.26 & 64.53±0.57  & 71.98±0.34  & 53.72±0.46   & 90.86±2.41    & 78.86±1.67    & 69.81±0.95     & 49.11±2.82     & 38.52±3.30    & 13.97±2.80     \\
EERM                       & OOM        & OOM         & OOM         & OOM          & 65.00±2.57    & 57.43±3.60    & OOM            & OOM            & 46.07±4.55    & 27.40±7.65     \\
\midrule
GAE                        & 56.63±0.79 & 48.93±0.93  & 66.30±0.88  & 34.01±0.87   & 73.00±2.16    & 60.86±3.01    & 67.24±1.23     & 47.65±2.49     & 45.08±6.32    & 28.02±6.29    \\
VGAE                       & 62.02±0.66 & 54.12±0.86  & 69.41±0.57  & 44.20±1.29   & 62.29±2.04    & 63.29±1.11    & 66.99±1.43     & \underline{50.48±4.58}     & 48.85±4.68    & 20.87±6.69     \\
GraphMAE                   & 68.14±0.43 & 64.00±0.33  & \textbf{73.36±0.56}  & 53.75±0.55   & 67.28±3.03    & 67.28±1.49    & 68.84±1.20     & 48.02±2.79     & 48.03±4.34    & 30.00±8.09     \\
DGI                        & 60.85±0.75 & 57.03±0.67  & 68.97±0.41  & 41.75±0.88   & 69.57±4.09    & 59.71±3.43    & 68.43±1.05     & 44.83±1.61     & 48.52±5.04    & 21.11±7.50     \\
MVGRL                      & OOM        & OOM         & OOM         & OOM          & 65.00±1.94    & 64.15±0.77    & OOM            & OOM           & \textbf{54.10±5.39}    & 16.59±6.51     \\
GRACE                      & \underline{68.77±0.33} & \underline{64.21±0.41}  & 72.69±0.34  & \underline{56.10±0.63}   & \underline{93.57±1.83}    & \underline{89.29±3.40}    & \underline{71.12±0.87} & 46.21±1.54 & 49.67±5.82    & 28.10±4.68    \\
RoSA                       & 68.19±0.56 & 62.48±0.61  & 71.04±0.62  & 52.72±0.79   & 84.71±4.14    &79.14±3.51     & 70.58±0.36     & 45.83±1.72     & 52.30±4.24    & \underline{34.24±7.92}     \\
BGRL                       & 67.23±0.43 & 61.33±0.36  & 72.11±0.39  & 49.15±0.73   & 89.00±2.56    & 79.86±3.29    & \textbf{71.43±0.53}     & 43.86±0.94     & 51.80±5.55    & 30.32±7.61    \\
COSTA                      & 65.28±0.60 & 60.33±0.53  & 70.65±0.62  & 54.03±0.28   & 92.29±1.59    & 82.71±2.74    & 69.29±1.37     & 49.07±2.13     & 50.49±3.01    & 29.84±4.75   \\
SwAV                       & 63.29±1.01 & 56.98±0.94  & 70.27±0.73  & 43.00±0.52   & 89.57±1.12    & 81.43±1.69    & 69.19±0.93     & 49.37±2.96     & 49.84±4.82    & 30.55±6.72   \\
\midrule
MARIO                       & \textbf{69.99±0.54} & \textbf{65.06±0.34}  & \underline{72.73±0.43}  & \textbf{57.73±0.45}  & \textbf{94.57±2.46}    & \textbf{91.00±2.48}     & 68.31±0.78 & \textbf{57.37±1.37}     & \underline{53.94±3.23}    & \textbf{35.24±4.98}   \\
\bottomrule
\end{tabular}}

\end{table*}

\subsubsection{Inductive Setting}
In this subsection, we conduct experiments under the inductive settings, where the test nodes are kept unseen during training. This setting is more suitable for domain generalization.
% But we think it is more convincing that conduct experiments under inductive settings which means test nodes are unseen during training. This setting is more appropriate for domain generalization.

\noindent\textbf{Baselines:} For GOOD-WebKB and GOOD-CBAS datasets, we adopt ERM, IRM, GraphMAE, and GRACE as our baselines. And for Amazon-Photo and Elliptic datasets, we select ERM, EERM, and GRACE as our baselines.

\noindent\textbf{Experimental setup:} For GOOD-WebKB and GOOD-CBAS datasets, we use the same model configuration in Section~\ref{sec:trans}.
% Besides, we add experiments on Amazon-Photo dataset~\cite{yang2016revisiting} and Elliptic~\cite{elliptic} dataset in this subsection. 
For Amazon-Photo dataset~\cite{yang2016revisiting} and Elliptic~\cite{elliptic} dataset, they consist of many snapshots (training data and testing data use different snapshots) which are naturally inductive. For Amazon-Photo dataset, we use 2-layer GCN~\cite{gcn} as the encoder and for elliptic dataset, we use 5-layer GraphSAGE~\cite{sage} as encoder following~\cite{eerm}.

% Figure environment removed

\noindent\textbf{Analysis:}
According to Figure~\ref{fig:amazon},\ref{fig:elliptic},\ref{fig:ind_con},\ref{fig:ind_cov}, we can draw following conclusions:
firstly, based on Figure~\ref{fig:amazon}, it is evident that our method outperforms other representative supervised and self-supervised methods on all test graphs (T1$\sim$T8). This superiority is reflected in the larger median value of our method compared to others. For instance, MARIO achieves over a 3\% absolute improvement compared to ERM in terms of the mean value of eight median values. Additionally, our method demonstrates higher stability across different random initializations, as indicated by the closer proximity of the first and third quartile values to the median value~(\eg, the difference of first and third quartile values of ERM, EERM, GRACE and MARIO are 4.2, 3.3, 6.7 and 1.0 on T8 respectively which indicates MARIO is much more stable than other methods). Furthermore, our method exhibits consistent performance across different graphs (\eg, The standard deviation of median values on T1$\sim$T8 for ERM, EERM, GRACE, and MARIO are 0.4, 1.1, 1.2, and 0.3, respectively.), indicating its robustness to environmental variations and its ability to extract invariant features: $g(G^e) \approx g(G^{e'})$ for all $e, e' \in \mathcal{E}^\text{train}$. In summary, our method showcases enhanced OOD generalization capabilities.
% $g(G^e)g(G^e^\prime)$ where $any e, e^\prime in \mathcal{E}^{train}$

Secondly, from the results presented in Figure~\ref{fig:elliptic}, we can observe that our method averagely harvests 10.9\% absolute improvement over GRACE and 12.5\% absolute improvement over EERM in terms of F1 scores on Elliptic dataset. This demonstrates the effectiveness of our method in handling distribution shifts and improving performance compared to existing approaches. It is worth noting that GRACE's performance worsens over time, indicating its inability to handle distribution shifts effectively. In contrast, our method consistently achieves better F1 scores, except for T9, which is caused by the dark market shutdown occurred after T7~\cite{elliptic}. The emergence of such an event introduces significant variations in data distributions, which subsequently results in performance degradation for all methods. Indeed, this event serves as an unpredictable external factor that introduces significant challenges for models trained on limited training data. The results indicate that the performance heavily depends on available training data. Nonetheless, our approach outperforms other methods even in such an extreme case. This highlights the effectiveness of our method in addressing distribution shifts and improving generalization performance.

Finally, based on the observations from Figure~\ref{fig:ind_con} and Figure~\ref{fig:ind_cov} MARIO demonstrates the best performances on both ID and OOD test sets for GOOD-WebKB and GOOD-CBAS datasets, under both concept shift and covariate shift. Notably, MARIO outperforms other methods by more than 3\% and 10\% absolute improvement on GOOD-WebKB and GOOD-CBAS, respectively, under covariate shift. We can draw similar conclusions as discussed in Section~\ref{sec:trans}. Even under the inductive setting, our method continues to demonstrate excellent OOD generalization capabilities and achieves comparable or even improved in-distribution test performance. These statistical results further validate the effectiveness of our method in handling distribution shifts and enhancing generalization performance.

Overall, the observations we have made provide strong evidence of the great capacity of our method for handling distribution shifts, validating its effectiveness and potential for real-world applications.



% Figure environment removed

% Figure environment removed


% Figure environment removed


\subsection{Ablation Studies}\label{sec:ablation}
\noindent Table~\ref{tab:aba} provides a detailed analysis of the effect of each component according to our proposed recipe for improving OOD generalization in graph contrastive learning. Let's examine the different variants of our method and their impact on performance.
Specifically, MARIO~(w/o ad) represents MARIO without  adversarial augmentation. MARIO~(w/o cmi) denotes we only maximize the mutual information between positive pairs without considering conditional mutual information. MARIO~(w/o cmi, ad) means a vanilla graph contrastive method that is similar to GRACE. 

From Table~\ref{tab:aba}, we can find MARIO~(w/o cmi) lags far behind MARIO on OOD test set which demonstrates appropriately minimizing the redundant information (\ie, conditional mutual information) is essential to improve OOD generalization of GCL methods. And adversarial augmentation can also boost OOD generalization because it can approximately serve as a supermum operator to learn more invariant features  discussed in Section~\ref{sec:aug}. Based on the analysis of these variants, it is evident that the proposed improvements on data augmentation and contrastive loss in the recipe are both effective in enhancing graph OOD generalization. Each component contributes to the overall performance improvement, and their combination leads to a stronger self-supervised graph learner in terms of graph OOD generalization. 

In short, the findings from Table~\ref{tab:aba} support the rationale behind your proposed recipe and provide empirical evidence of the effectiveness of each proposed component. By incorporating these enhancements, our method achieves superior performance in handling distribution shifts and improving graph OOD generalization in graph contrastive learning.
\begin{table*}[htp]
\caption{Ablation studies for MARIO by masking each component.}
\label{tab:aba}
\centering
\scalebox{0.9}{
\begin{tabular}{l|cc|cc|cc|cc|cc}
\toprule
\toprule
\multirow{3}{*}{concept shift} & \multicolumn{4}{c|}{GOOD-Cora}                       & \multicolumn{2}{c|}{GOOD-CBAS} & \multicolumn{2}{c|}{GOOD-Twitch} & \multicolumn{2}{c}{GOOD-WebKB} \\
                           & \multicolumn{2}{c}{word} & \multicolumn{2}{c|}{degree}& \multicolumn{2}{c|}{color}    & \multicolumn{2}{c|}{language}   & \multicolumn{2}{c}{university} \\
                           & ID         & OOD         & ID          & OOD          & ID            & OOD           & ID             & OOD            & ID            & OOD            \\
\midrule
MARIO                      & \textbf{67.11±0.46} & \textbf{65.28±0.34}  & \textbf{68.46±0.40}  & \textbf{61.30±0.28}      & \textbf{94.36±1.21}  & \textbf{91.28±1.10}    & 82.31±0.54     & \textbf{63.33±1.72}     & \textbf{65.67±2.81}    & \textbf{37.15±2.37}     \\
MARIO(w/o ad)              & 66.23±0.53 & 64.02±0.18  & 67.88±0.38  & 60.46±0.29   & 93.21±1.25    & 90.29±0.91    & 82.42±0.73     & 60.50±1.02     & 64.83±2.83    & 36.51±3.25    \\
MARIO(w/o cmi)             & 65.32±0.60 & 63.51±0.32  & 68.14±0.32  & 61.19±0.34   & 94.15±1.23    & 90.57±1.96    & \textbf{82.51±0.56}     & 61.41±2.63     & 64.50±4.35    & 35.78±2.53     \\
MARIO(w/o cmi, ad)         & 64.67±0.55 & 63.11±0.32  & 67.95±0.65  & 60.01±0.57   & 93.36±1.66    & 89.64±1.73    & 81.90±0.75     & 60.12±1.60     & 64.17±3.67    & 34.13±2.38     \\
\bottomrule
\end{tabular}}
\end{table*}
% & 65.32±0.60 & 63.51±0.32 exchange 64.67±0.55 & 63.11±0.32
% 68.14±0.32       id ood test: 60.95±0.43       ood ood test: 61.19±0.34


\subsection{Sensitivity Analysis}\label{sec:sensitivity}
\noindent In this subsection, we will analyze some important hyper-parameters of our method. We conduct sensitivity analysis on GOOD-WebKB dataset with concept shift, we chose two sensitive hyper-parameters (\ie, the coefficient $\gamma$ of condition mutual information in Equation~\ref{equ:cmi} and the number of prototypes $|C|$ in Equation~\ref{equ:pq}). The coefficient of CMI range in $[0.001, 0.01, 0.1, 0.5, 1]$ and the number of prototypes $|C|$ ranges in $[10, 50, 100, 200, 300]$. From Figure~\ref{fig:sensitivity}, we can observe that $\gamma$ reaches 0.1 and $|C|$ reaches 100 or 200 can achieve the best OOD test accuracy. Both higher and lower values of $\gamma$ result in suboptimal performance. This finding aligns with previous research such as DIB~\cite{dib}, indicating that an appropriate compression level is crucial for achieving optimal performance. Extremely high or low compression values are not ideal. 

Regarding the number of prototypes $|C|$, based on the results shown in Figure~\ref{fig:sensitivity}, it is found that setting $|C|=100$ leads to the best performance in terms of OOD test accuracy. This choice provides a moderate number of pseudo labels, which is beneficial for the learning process. 

Based on the sensitivity analysis, we determined that setting $\gamma=0.1$ and $|C|=100$ on most datasets. These hyperparameter values strike a balance between compression level and the number of prototypes, resulting in improved graph OOD generalization.
% Figure environment removed


\subsection{Integrated with Other Models}\label{sec:other_models}
% Figure environment removed

\begin{table}[htp]
\caption{Results of different learning approaches with different encoding models (\ie, GCN, GraphSAGE, GAT).}
\label{tab:others}
\centering
\scalebox{0.9}{
\begin{tabular}{cc|cc|cc}
\toprule
\toprule
\multirow{3}{*}{Model}& \multirow{3}{*}{Method} & \multicolumn{2}{c|}{GOOD-CBAS} & \multicolumn{2}{c}{GOOD-WebKB} \\
                & & \multicolumn{2}{c|}{color}    & \multicolumn{2}{c}{university} \\
                &   & ID          & OOD         & ID          & OOD            \\
\midrule
\multirow{3}{*}{GCN} 
&ERM               & 89.79±1.39 & 83.43±1.19  &  62.67±1.53 & 26.33±1.09         \\
&GRACE             & 92.00±1.39 & 88.64±0.67  &  64.00±3.43 & 34.86±3.43        \\
&MARIO             & 94.36±1.21 & 91.28±1.10  &  65.67±2.81 & 37.15±2.37        \\ \bottomrule
\multirow{3}{*}{SAGE} 
&ERM               & 95.07±1.51 & 75.14±1.19  & 73.67±2.08  & 46.33±3.42       \\
&GRACE             & 95.29±1.11 & 74.43±2.36  & 70.50±5.06  & 49.54±3.83        \\
&MARIO             & 96.00±1.07 & 76.29±3.01  & 71.00±3.82  & 51.74±4.63        \\ \bottomrule
\multirow{3}{*}{GAT} 
&ERM               & 78.64±3.63 & 72.93±2.64  & 61.33±3.71  & 28.99±2.63        \\
&GRACE             & 84.57±1.79 & 78.36±1.60  & 59.50±2.36  & 35.78±3.26        \\
&MARIO             & 84.93±1.95 & 80.43±1.89  & 62.17±4.78  & 38.17±3.10        \\
\bottomrule
\end{tabular}}
\end{table}



\noindent In the subsection, we demonstrate the model-agnostic nature of the recipe by integrating it with various graph neural network (GNN) models, including GCN, GraphSAGE, and GAT.

From Table~\ref{tab:others}, it can be observed that regardless of the specific GNN model used as the encoder, our method consistently achieves the best performance on the OOD test set. This indicates the effectiveness and robustness of our method across different GNN models.
By achieving superior performance across different GNN models, MARIO demonstrates its versatility and ability to improve the OOD generalization of various graph neural models. This highlights the broad applicability and effectiveness of our recipe in enhancing the performance of different GNN encoders.

Furthermore, we integrate our recipe with other GCL methods in Appendix~\ref{app:other_methods}. The results demonstrate our recipe can boost the OOD generalization ability of various GCL methods which means our recipe can serve as a plug-in for many current classical GCL methods.

% Figure environment removed

\subsection{Visualization}\label{sec:vis}
\subsubsection{Metric Score Curves}
We present metric score curves for ERM and MARIO, including training, ID validation, ID testing, OOD validation, and OOD testing accuracy, in Figure~\ref{fig:curve2}. Notably, MARIO demonstrates superior convergence with approximately 10\% absolute improvement on the OOD test set compared to ERM. Furthermore, MARIO effectively narrows the performance gap between in-distribution and out-of-distribution performance, showcasing its efficacy in enhancing OOD generalization for graph data. More metric score curves can be found in Appendix~\ref{app:curves}.


\subsubsection{Feature Visualization}
In order to assess the quality of learned embeddings, we adopt t-SNE~\cite{tsne} to visualize the node embedding on GOOD-Cora dataset (concept shift in word domain) using random-init of GCN, EERM, GRACE, and MARIO, where different classes have different colors in Figure~\ref{fig:vis}. For clarity, we select eight classes with the largest number of nodes to enhance the informativeness and interpretability of the visualization. We can observe that the 2D projection of node embeddings learned by MARIO has a better separation of clusters, which indicates the model can help learn representative features for downstream tasks. It has to note that we depict both ID nodes and OOD nodes in the same figure. 

Besides, we also separately visualize ID nodes and OOD nodes in the different figures in the Appendix~\ref{app:feature}. And we can find MARIO performs a clearer separation of clusters whether on ID nodes or OOD nodes compared to other methods.




\end{document}
