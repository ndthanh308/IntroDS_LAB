\begin{abstract}
Normalized-cut graph partitioning aims to divide the set of nodes in a graph into $k$ disjoint clusters to minimize the fraction of the total edges between any cluster and all other clusters. In this paper, we consider a fair variant of the partitioning problem wherein nodes are characterized by a categorical sensitive attribute (e.g., \emph{gender} or \emph{race}) indicating membership to different demographic groups. Our goal is to ensure that each group is approximately proportionally represented in each cluster while minimizing the normalized cut value. To resolve this problem, we propose a two-phase spectral algorithm called FNM. In the first phase, we add an augmented Lagrangian term based on our fairness criteria to the objective function for obtaining a fairer spectral node embedding. Then, in the second phase, we design a rounding scheme to produce $k$ clusters from the fair embedding that effectively trades off fairness and partition quality. Through comprehensive experiments on nine benchmark datasets, we demonstrate the superior performance of FNM compared with three baseline methods.
\end{abstract}
