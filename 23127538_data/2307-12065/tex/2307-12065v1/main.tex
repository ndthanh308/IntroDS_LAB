\documentclass{article}

\usepackage{arxiv}

\usepackage[utf8]{inputenc}
\usepackage[T1]{fontenc}
\usepackage{hyperref}
\usepackage{url}
\usepackage{nicefrac}
\usepackage{microtype}
\usepackage{lipsum}

\usepackage{graphicx}
\usepackage{latexsym}

\usepackage{amsmath,amssymb,amsthm,amsfonts}
\usepackage{algorithm}
\usepackage{algorithmic}

\usepackage{bm}
\usepackage{xcolor}
\usepackage{colortbl}
\usepackage[caption=false]{subfig}
\usepackage{multirow}

\usepackage[square,numbers,sort]{natbib}
\bibliographystyle{abbrvnat}

\newtheorem{definition}{Definition}

\DeclareMathOperator*{\argmin}{arg\,min}
\DeclareMathOperator*{\argmax}{arg\,max}

\title{Spectral Normalized-Cut Graph Partitioning with Fairness Constraints}

\author{
  Jia Li, Yanhao Wang \\
  School of Data Science and Engineering\\
  East China Normal University\\
  Shanghai, China\\
  \texttt{jiali@stu.ecnu.edu.cn, yhwang@dase.ecnu.edu.cn} \\
  \And
  Arpit Merchant \\
  Department of Computer Science\\
  University of Helsinki\\
  Helsinki, Finland\\
  \texttt{arpit.merchant@helsinki.fi} \\
}

\begin{document}

\maketitle

\begin{abstract}

This paper presents a low-cost network architecture for training large language models (LLMs) at hyperscale. We study the optimal parallelization strategy of LLMs and propose a novel datacenter network design tailored to LLM's unique communication pattern. We show that LLM training generates sparse communication patterns in the network and, therefore, does not require any-to-any full-bisection network to complete efficiently. As a result, our design eliminates the spine layer in traditional GPU clusters. We name this design a \textit{Rail-only} network and demonstrate that it achieves the same training performance while reducing the network cost by 38\% to 77\% and network power consumption by 37\% to 75\% compared to a conventional GPU datacenter. Our architecture also supports Mixture-of-Expert (MoE) models with all-to-all communication through forwarding, with only 8.2\% to 11.2\% completion time overhead for all-to-all traffic. We study the failure robustness of Rail-only networks and provide insights into the performance impact of different network and training parameters. \looseness=-1


\end{abstract}



The meteoric advancement of machine learning and artificial intelligence technologies has enabled the construction of neural networks that effectively emulate the complex computations of the human brain. These deep learning models have found utility in a wide range of applications, such as computer vision, natural language processing, autonomous driving, and more. With the growing complexity and sophistication of these neural network models, the computational requirements, particularly for 32-bit operations, have exponentially increased. This heightened computational demand necessitates the exploration of more efficient alternatives, such as 16-bit operations.

However, the shift to 16-bit operations is riddled with challenges. A common standpoint within the research community argues that 16-bit operations are not ideally suited for neural network computations. This belief is mainly attributable to concerns related to numerical instability during the backpropagation phase, especially when popular optimizers like Adam are employed. This instability, more pronounced during the optimizer-mediated backpropagation process rather than forward propagation, can negatively impact the performance of 16-bit operations and compromise the functioning of the neural network model. Current optimizers predominantly operate on 32-bit precision. If these are deployed in a 16-bit environment without appropriate hyperparameter fine-tuning, the neural network models encounter difficulties during learning. This issue is particularly evident in backward propagation, which heavily relies on the optimizer. Confronted with these challenges, the objective of this research is to conduct an exhaustive investigation into the feasibility and implementation of 16-bit operations for neural network training. We propose and evaluate innovative strategies aimed at reducing the numerical instability encountered during the backpropagation phase under 16-bit environments. A significant focus of this paper is also dedicated to exploring the future possibilities of developing 16-bit based optimizers. One of the fundamental aims of this research is to adapt key optimizers such as Adam to prevent numerical instability, thereby facilitating efficient 16-bit computations. These newly enhanced optimizers are designed to not only address the issue of numerical instability but also leverage the computational advantages offered by 16-bit operations, all without compromising the overall performance of the neural network models. Through this research, our intention goes beyond improving the efficiency of neural network training; we also strive to validate the use of 16-bit operations as a dependable and efficient computational methodology in the domain of deep learning. We anticipate that our research will contribute to a shift in the prevalent perceptions about 16-bit operations and will foster further innovation in the field. Ultimately, we hope our findings will pave the way for a new era in deep learning research characterized by efficient, high-performance neural network models.

\subsection{Other Related Works}

\noindent\textbf{Min-Cut Graph Partitioning.}
Partitioning nodes in a graph into disjoint subsets to minimize an objective function, such as \emph{ratio cut} \cite{HagenK92}, \emph{normalized cut} \cite{ShiM00}, and \emph{Cheeger cut} \cite{Cheeger71}, is a fundamental combinatorial optimization problem.
Since all those graph partitioning problems are NP-hard \cite{HagenK92, ShiM00}, different heuristic algorithms were designed for them, among which spectral methods \cite{HagenK92, ShiM00, NgJW01, HanXN17} have attracted the most attention.
The basic idea of spectral methods is to relax the original integer minimization problems into continuous optimization problems, which are solved by computing the eigenvectors of the Laplacian matrix, and to find the partitioning using $k$-means or an alternative rounding method.
Other relaxation-based methods \cite{JiaDDX16, ChenHNHYH18, LiNL18} improved the efficiency over original spectral methods by avoiding eigendecomposition.
However, none of them incorporate the notion of \emph{fairness} into graph partitioning problems.

\smallskip\noindent\textbf{Fair Clustering.}
There has been rich literature on fair clustering algorithms.
Chierichetti \emph{et al.} \cite{Chierichetti0LV17} first introduced the notion of \emph{proportional fairness} for clustering and then proposed fairlet decomposition algorithms for fair $k$-median and $k$-center clustering in the case of two groups.
Following this work, the problem has been further generalized to handle more than two groups \cite{HuangJV19}, permit lower and upper bounds on the fraction of points from a group \cite{Bercea0KKRS019, BeraCFN19, Gupta_2023}, and allow probabilistic group memberships \cite{EsmaeiliBT020}.
In addition, more efficient fair clustering algorithms have been proposed based on faster fairlet computation \cite{BackursIOSVW19} or coresets \cite{HuangJV19, SchmidtSS19}.
Then, several studies \cite{DavidsonR20, ZikoYGA21, EsmaeiliBSD21} explored how to achieve better trade-offs between the fairness and clustering objectives.
Other fair variants of clustering problems focused on different fairness notions, including individual-level fairness \cite{MahabadiV20, NegahbaniC21, VakilianY22}, fair center selection \cite{KleindessnerAM19, ThejaswiOG21}, and minimax losses among groups \cite{MakarychevV21, GhadiriSV21, ChlamtacMV22}.
However, all the above methods are primarily designed for i.i.d.~data rather than graph data.
They cannot be directly applied to graph partitioning because graphs are high-dimensional and sparse, and they will incur huge computational costs and often return inferior results due to the curse of dimensionality.

There were also a few studies on fairness in spectral methods and other graph clustering problems.
In addition to \cite{KleindessnerSAM19, pmlr-v206-wang23h}, Gupta and Dukkipati \cite{gupta2022consistency} further investigated spectral clustering with individual fairness constraints.
Moreover, Ahmadian \emph{et al.} \cite{AhmadianE0M20}, Friggstad and Mousavi \cite{FriggstadM21a}, and Ahmadian and Negahbani \cite{abs-2206-05050} studied fair correlation clustering on signed graphs.
Anagnostopoulos \emph{et al.} \cite{Anagnostopoulos20} applied spectral methods to fair densest subgraph discovery on graphs.
These algorithms are interesting but not comparable to our algorithm.


\section{Problem Definition}
\label{sec-def}

\smallskip\noindent\textbf{Normalized-Cut Graph Partitioning.}
Define $[n] := \{1, 2,$ $\ldots, n\}$ for any positive integer $n$.
Let $G = (V, E)$ be an undirected graph, where $V = [n]$ is the set of $n$ nodes and $E \subseteq V \times V$ is the set of edges.
We use $ \bm{W} = (w_{ij})_{i,j \in [n]}$ to denote the adjacency matrix of $G$, where each entry $w_{ij} \geq 0$ is the weight of edge $(i,j)$ (always equal to $1$ for the unweighted case) if it exists or $0$ otherwise.
We consider $w_{ii} = 0$ for any $i \in [n]$.
The degree matrix $\bm{D} = (d_{i})_{i \in [n]}$ is a diagonal matrix with the degree $d_{i} = \sum_{j=1}^n w_{ij} \geq 0$ of each node $i$ on its diagonal.
Given an undirected graph $G$, an integer $k \geq 2$, we aim to find a partitioning $\mathcal{C} = \{C_1, \ldots, C_k\}$ of $V$ into $k$ disjoint clusters, i.e., $\bigcup_{l = 1}^{k} C_l = V$ and $C_l \cap C_{l'} = \emptyset$ for any $l \neq l' \in [k]$, to minimize the normalized cut \cite{YuS03} (Ncut) value as follows:
\begin{equation}\label{eq-Ncut}
  \mathsf{Ncut}(\mathcal{C}) := \sum_{l=1}^k \frac{\mathsf{cut}(C_l)}{\mathsf{vol}(C_l)} = \sum_{l=1}^k \frac{\sum_{i \in C_l, j \in V \setminus C_l} w_{ij}}{\sum_{i \in C_l, j \in V} w_{ij}}.
\end{equation}
The Ncut minimization problem is well-known as NP-hard \cite{ShiM00}.

\smallskip\noindent\textbf{Fairness Constraint.}
In the fair variant of graph partitioning, we consider that the node set $V$ consists of several demographic groups defined by a categorical sensitive attribute, e.g., gender or race.
Formally, suppose that $V$ is divided into $m$ disjoint groups indexed by $[m]$, and an indicator function $\phi: [n] \mapsto [m]$ maps each node $i \in [n]$ to the group $\phi(i)$ it belongs to.
Let $ V_c = \{ i \in [n] : \phi(i) = c \} $ be the subset of nodes from group $c$ in $V$.
We assume that $ \bigcup_{c = 1}^{m} V_c = V $ and $ V_c \cap V_{c'} = \emptyset, \forall c \neq c' \in [m]$.
For ease of presentation, we denote the group membership as an indicator matrix $\bm{M} \in \{0,1\}^{n \times m}$, where $\bm{M}_{i,c} = 1$ if $\phi(i) = c $ and $0$ otherwise.
We follow a notion of \emph{range-based proportional fairness} in \cite{BeraCFN19} to require that every demographic group is approximately proportionally represented in all the $k$ clusters.
We define the fairness constraint by two vectors $\bm{\alpha}, \bm{\beta} \in [0,1]^{m}$ that specify the upper and lower bounds $\alpha_c, \beta_c$ on the percentage of nodes from group $c$.
We say a partitioning $\mathcal{C}$ is $(\bm{\alpha}, \bm{\beta})$-\emph{proportionally fair} if $\beta_c \leq \frac{\lvert V_c \cap C_l \rvert}{\lvert C_l \rvert} \leq \alpha_c$ for any $C_l$ and $V_c$.
In practice, we parameterize $\bm{\alpha}, \bm{\beta}$ by a fairness variable $\sigma \in [0, 1]$ as $\alpha_c = \min\{r_c / (1 - \sigma), 1\}$ and $\beta_c = r_c \cdot (1 - \sigma)$, where $r_c = |V_c| / n$.\footnote{Any other parameterization scheme (e.g., $\beta_c = r_c \cdot (1 - \sigma)$ and $\alpha_c = \beta_c + \sigma$) is also compatible with our formulation as long as it guarantees that $\beta_c \leq \alpha_c$ and $\beta_c, \alpha_c \in [0, 1]$ for any $c \in [m]$.}
For example, if the percentage of females is 60\% in the population of all nodes, the fairness constraint requires that the percentage of females in each cluster should be between 48\% and 75\% when $\sigma = 0.2$.
The value of $\sigma$ can be interpreted as how loose the fairness constraint is, where $\sigma = 0$ corresponds to every group in each cluster having the same ratio as that group in the population, and $\sigma = 1$ corresponds to no fairness constraint at all.
Given all the above notions, we formally define the normalized-cut graph partitioning problem under $(\bm{\alpha}, \bm{\beta})$-proportional fairness as follows.
\begin{definition}\label{def-fnm}
  Given an undirected graph $G = (V, E)$, a set of $m$ groups $V_1, \ldots, V_m \subseteq V$, two fairness vectors $ \bm{\alpha}, \bm{\beta} \in [0,1]^m$, and an integer $k \geq 2$, find an $(\bm{\alpha}, \bm{\beta})$-proportionally fair partitioning $ \mathcal{C} = \{C_1, \ldots, C_k \} $ of $V$ into $k$ disjoint clusters such that $\mathsf{Ncut}(\mathcal{C})$ in Eq.~\ref{eq-Ncut} is minimized.
\end{definition}
The problem in Definition~\ref{def-fnm} is NP-hard since the vanilla Ncut minimization problem is its special case when $m = 1$.
Next, we will approach the problem by extending the spectral Ncut minimization algorithm.
Note that our method can be adapted to \emph{ratio cut} \cite{HagenK92} and any other cut measure with an equivalent spectral formulation.
This paper focuses on Ncut due to its prevalence and space limitations.


\looseness -1 We now propose a practical algorithm for \subrl that can efficiently handle submodular rewards. The core of our approach follows a greedy gradient update on the policy $\pi$. As common in the modern \RL literature, we make use of approximation techniques for the policies to derive a method applicable to large state-action spaces. This means the policy $\pi_{\theta}(a|s)$\footnote{autoregressive policies (RNNs or transformers) can be used to capture history-dependence in the same algorithm} 
is parameterized by $\theta \in \Theta$ where $\Theta \subset \R^l$ is compact. In the case of tabular $\pi$, $\theta$ specifies an independent distribution over actions for each state.

\mypar{Approach} 
\looseness -1 The objective from \cref{eq: obj} can be equivalently formulated as $\theta^{\star} \in \argmax_{\theta \in \Theta} J(\pi_\theta)$ as $\theta$ indexes our policy class. Due to the nonlinearity of the parameterization, it is often not feasible to find a global optimum for the above problem. In practice, with appropriate initialization and hyperparameters, variants of gradient descent are known to perform well empirically for MDPs. Precisely, \vspace{-0.4em}
\begin{align}
    \theta \leftarrow \theta + \argmax_{\delta \theta: \delta \theta + \theta \in \Theta} \delta \theta^\top \nabla_{\theta} J(\pi_\theta) - \frac{1}{2 \alpha} \|\delta\theta\|^2. \label{eqn: gd} 
\end{align}
Various \PG methods arise with different methods for gradient estimation and applying regularization \citep{kakade2001natural,schulman2015trust,schulman2017proximal}. The key challenge to all of them is computation of the gradient $\nabla J(\pi_\theta)$. Below, we devise an unbiased gradient estimator for general non-additive functions.

\mypar{Gradient Estimator} As common in the policy gradient (\PG) literature, we can use the score function $g(\traj, \pi_\theta) \coloneqq \nabla_{\theta} (\log \prod\nolimits_{i=0}^{\horizon-1}\pi_{\theta}(a_i|s_i))$ to calculate the gradient $\nabla_{\theta} J$. Namely, Given an \mdp and the policy parameters $\theta$, \vspace{-0.4em} 
\begin{equation}  \label{prop: score}\nabla_{\theta} J(\pi_{\theta}) = \sum_{\traj} f(\traj; \pi_\theta) g(\traj, \pi_\theta) F(\traj).
\end{equation}
\looseness -1 As \cref{prop: score} shows, we do not require knowledge of the environment if sampled trajectories are available. It also does not require full observability of the states nor any structural assumption on the \mdp. On the other hand, the score gradients suffer from high variance due to sparsity induced by trajectory rewards \citep{FU_score,prajapat2021competitive,sutton2018reinforcement}. Hence, we take the \smdp structure into account to develop efficient algorithms. \vspace{-0.1em}

\emph{Marginal gain:} We define the marginal gain for a state $s$ in the trajectory $\traj_{0:j}$ up to horizon $j$ as \vspace{-0.25em}\[F(s|\traj_{0:j}) = F(\traj_{0:j}\cup \{s\}) - F(\traj_{0:j}).\vspace{-0.25em}\] Our approach aims to maximize the marginal gain associated with each action instead of maximizing state rewards. 
This approach shares similarities with the greedy algorithm commonly used in submodular maximization, which maximizes marginal gains and is known for its effectiveness. Moreover, decomposing the trajectory return into marginal gains and incorporating it in the policy gradient with suitable baselines \cite{greensmith2004variance} removes sparsity and thus helps to reduce variance. Inspired by the policy gradient method for additive rewards \citep{sutton1999policy,baxter2001infinite}, we propose the following for \smdp: %For $\traj_{l:l'}$, the events from time step $l$ to $l'$, we have:
\vspace{-0.1em}
\begin{restatable*}{theorm}{restatePG} \label{thm: PG} Given an \smdp and the policy parameters $\theta$, with any set function $\Obj$,\vspace{-0.5em}
    \begin{align}
        \nabla_{\theta} J(\pi_{\theta}) = \E_{\traj \sim f(\traj ; \pi_\theta)} \left[ \sum_{i=0}^{H-1} \nabla_{\theta} \log \pi_{\theta}(a_i|s_i) \left(\sum_{j=i}^{H-1}F(s_{j+1}|\traj_{0:j}) - b(\traj_{0:i})\right) \right] \label{eqn: grad_esti}
    \end{align}\vspace{-1.1em}
\end{restatable*} \vspace{-0.6em}
We use an importance sampling estimator (log trick) to obtain \cref{prop: score}. To reduce variance, we subtract a baseline $b(\traj_{0:i})$ from the score gradient, which can be a function of the past trajectory $\traj_{0:j}$. This incorporates the causality property in the estimator, ensuring that the action at timestep $j$ cannot affect previously observed states. After simplifying and considering marginals, we obtain \cref{thm: PG} (proof is in \cref{apx: PG}). This estimator assigns a higher weight to policies with high marginal gains and a lower weight to policies with low marginal gains. Empirically this performs very well (\cref{sec: experiments}). 

We can optimize this approach by using an appropriate baseline as a function of the history $\traj_{0:j}$, which leads to an actor-critic type method. The versatility of the approach is demonstrated by the fact that \cref{thm: PG} holds for any choice of baseline critic. We explain later in the experiments how to choose a baseline. One can perform a Monte Carlo estimate ~\citep{baird1993} or generalized advantage function (\GAE)~\citep{schulman2015highdimensional} to estimate returns based on the marginal gain.
% However, what the ideal critic function is for non-additive functions needs to be clarified, and we leave it as future work. 
To encourage exploration, similar to standard \PG, we can employ a soft policy update based on entropy penalization, resulting in diverse trajectory samples. Entropy penalization in \subrl can be thought of as the sum of modular and submodular rewards, which is a submodular function.

\mypar{Algorithm} The outline of the steps is given in \cref{alg:subrl}. We represent the agent by a stochastic policy parameterized by a neural network. 
% At each state, the agent uses a categorical distribution over the set of actions. We apply a softmax operation to the distribution to enforce the simplex constraint. 
The algorithm operates in epochs and assumes a way to generate samples from the environment, e.g., via a simulator. In each epoch, the agent recursively samples actions from its stochastic policy and applies them in the environment leading to a \emph{roll out} of the trajectory where it collects samples (\cref{alg: sample_collect}). We execute multiple ($B$) batches in each epoch for accurate gradient estimation.
To update the policy, we compute the estimator of the policy gradient as per \cref{thm: PG}, where we utilize marginal gains of the trajectory instead of immediate rewards as in  standard RL (\cref{alg: estimator}). Finally, we use stochastic gradient ascent to update the policy parameters.



\section{Experiments}
\label{sec-exp}

In this section, we perform extensive empirical evaluations of our FNM algorithm.
We introduce our experimental setup in Section~\ref{setup} and describe our results in Section~\ref{results}.

\subsection{Experimental Setup}
\label{setup}

\noindent\textbf{Datasets.}
We use eight public real datasets with sensitive attributes and one synthetic dataset in the experiments.
\emph{Facebook}, \emph{LastFM}, \textit{Deezer}, \textit{Pokec-A}, \textit{Pokec-G} are all social networks;
\emph{DBLP} is a coauthor network;
\emph{German} and \emph{Credit} are similarity graphs created from i.i.d.~data;
and \emph{SBM} is generated from a stochastic block model with random groups.
If a graph is disconnected, we will extract and use its largest connected component.
Table~\ref{stats} summarizes the statistics of all processed datasets.
Detailed descriptions of the above datasets are provided in Appendix~\ref{app:dataset}.

\begin{table}[ht]
\footnotesize
\begin{center}
{\caption{Statistics of datasets in the experiments.}\label{stats}}
\begin{tabular}{|c|c|c|c|c|}
\hline
\textbf{Dataset} & $\lvert V \rvert$ & $\lvert E \rvert$ & \textbf{Sensitive Attribute} & $m$ \\
\hline
Facebook & 155 & 1,412 & gender & 2 \\
German & 1,000 & 21,742 & gender & 2 \\
SBM & 1,000 & 57,156 & -- & 5 \\
DBLP & 1,061 & 2,576 & continent & 3 \\
LastFM & 7,624 & 27,806 & country & 4 \\
Deezer & 28,281 & 92,752 & gender & 2 \\
Credit & 29,460 & 136,196 & education & 3 \\
Pokec-A & 1,097,077 & 10,792,894 & age & 4 \\
Pokec-G & 1,632,803 & 22,301,964 & gender & 2 \\
\hline
\end{tabular}
\end{center}
\end{table}
  

\smallskip\noindent\textbf{Baselines.}
We compare our FNM algorithm with the following three baseline methods for graph partitioning:
\emph{(i)} spectral clustering (SC) \cite{YuS03}, \emph{(ii)} fair spectral clustering (FSC) \cite{KleindessnerSAM19}, and \emph{(iii)} scalable fair spectral clustering (sFSC) \cite{pmlr-v206-wang23h}.
In the ablation study, we compare our range-based fair spectral embedding (rFSE) in Algorithm~\ref{alg1} with the following six node embeddings, i.e.,
spectral embedding (SE) \cite{YuS03}, fair spectral embedding (FSE) \cite{KleindessnerSAM19}, scalable fair spectral embedding (sFSE) \cite{pmlr-v206-wang23h}, DeepWalk (DW) \cite{PerozziAS14}, Node2Vec (N2V) \cite{GroverL16}, and FairWalk (FW) \cite{RahmanSBZ19};
and our range-based fair rounding (FR) in Algorithm~\ref{alg2} with four alternatives, i.e., $k$-means++ (KM+) \cite{ArthurV07}, $k$-means++ with reassignments (K+R), fair $k$-means (FK) \cite{BeraCFN19}, and solving $\mathrm{IP1}$ directly (IP).
Since the IP/LP solver fails to provide solutions in a reasonable time, FR, FK, and IP do not work on \textit{Pokec-A} and \textit{Pokec-G}, and we alternatively use K+R together with rFSE to obtain the results of FNM on both datasets.

\smallskip\noindent\textbf{Parameter Settings.}
For FNM,  $\bm{\alpha}, \bm{\beta}$ are parameterized by $\sigma \in [0,1]$ as per Section~\ref{sec-def}.
By default, we set $\sigma = 0.2$ and $0.8$ (resp.~the common 80\%-rule) to generate tight and loose fairness constraints.
For Algorithm~\ref{alg1}, we set $T_1 = 100$, $\varepsilon_1 = 10^{-6}$, $T_2 = 2,000$, $\tau = 10^{-3}$, and $\varepsilon_2 = 10^{-3}$.
We perform a grid search on $\xi \in \{2, 4, \ldots, 10\}$ and $\mu_0 \in \{10^{-4}, 10^{-2}, 10^0, 10^2\}$ and select the combination of $\xi, \mu_0$ achieving the lowest objective value for each experiment.
For Algorithm~\ref{alg2}, we set $T_0 = 100$, $T_3 = 10$, and $\varepsilon_3 = 10^{-4}$ for all experiments.
Further details of our parameter-tuning procedure are provided in Appendix~\ref{app:parameter}.
For the baselines, we follow the default parameters or use the recommended methods for parameter tuning as given in their original papers.

\smallskip\noindent\textbf{Evaluation Metrics.}
Each method is evaluated in three aspects.
First, we measure partition quality by the Ncut value in Eq.~\ref{eq-Ncut}.
Second, we adopt the notion of \emph{balance} in \cite{Chierichetti0LV17, BeraCFN19} as the metric for fairness.
Given a set $\mathcal{C} = \{C_1, \ldots, C_k\}$ of $k$ clusters and a set $\{V_1, \ldots, V_m\}$ of $m$ groups, the proportion of group $c$ in cluster $C_l$ is defined as $r_{cl} = \lvert C_l \cap V_c \rvert / \lvert C_l \rvert$.
Then, the \emph{balance} of $\mathcal{C}$ is defined by $\mathsf{balance}(\mathcal{C}) := \min_{c \in [m], l \in [k]} \min \big\{ r_c/r_{cl}, r_{cl}/r_c \big\}$, where $r_c = |V_c| / n$.
Higher \emph{balance} implies that the partitioning scheme is closer to being proportionally fair.
Balance also serves as an indicator of whether the fairness constraints parameterized by $\sigma$ are satisfied because $\mathcal{C}$ is $(\bm{\alpha}, \bm{\beta})$-proportionally fair iff $\mathsf{balance}(\mathcal{C}) \geq 1 - \sigma$.
Third, we use \emph{CPU time} to evaluate the efficiency of each method.

\smallskip\noindent\textbf{Implementation.}
We implement FNM in Python 3 and use Gurobi Optimizer to solve LP and IPs.
For each baseline, we either use a standard implementation in the SciPy library or the implementation published by the original authors.
The experiments were conducted on a desktop with an Intel Core i5-9500 processor @3.0GHz and 32GB RAM running Ubuntu 20.04.
Our code and data are published at \url{https://github.com/JiaLi2000/FNM}.

\subsection{Experimental Results}
\label{results}

\noindent\textbf{Overview.}
Table~\ref{default} presents the performance of different algorithms for normalized-cut graph partitioning with two fairness constraints parameterized by $\sigma = 0.8$ and $0.2$ when $k = 5$ on all nine datasets.

In terms of partition quality and fairness, the (unconstrained) SC mostly achieves the lowest Ncut values but fails to provide a fair partitioning when $\sigma = 0.8$ on four datasets while never meeting tighter fairness constraints when $\sigma = 0.2$.
Although FSC and sFSC provide more balanced partitions than SC in some cases, they still cannot guarantee the satisfaction of fairness constraints.
In addition, FSC does not return any results on medium and large graphs with over $10$k nodes due to huge memory consumption for eigendecomposition on dense matrices.
Next, we observe that FNM always provides fair partitioning schemes in all cases.
If unconstrained SC returns fair solutions when $\sigma = 0.8$, FNM will achieve nearly the same Ncut values.
Otherwise, the Ncut values of FNM will increase slightly to ensure fairness.
Moreover, the Ncut values of FNM for $\sigma = 0.2$ are significantly higher than those for $\sigma = 0.8$, which can be regarded as the \emph{price of fairness}.

In terms of time efficiency, FNM runs slower than SC as it is more time-consuming in embedding and rounding.
But FNM runs faster than FSC in most cases since it does not require eigendecomposition.
Compared to sFSC, which improves the scalability of FSC by avoiding eigendecomposition on dense matrices, FNM runs slower on smaller graphs due to a longer time for fair rounding but becomes faster on larger graphs owing to the efficiency improvements for fair embedding.
Finally, FNM is more efficient when $\sigma = 0.8$ than $\sigma = 0.2$ because of fewer iterations for convergence.

\begin{table}[t]
  \footnotesize
  \setlength\tabcolsep{1pt}
  \begin{center}
  {\caption{Performance of different algorithms for normalized-cut graph partitioning with $k = 5$ clusters. Cells in lighter and darker gray colors denote results satisfying looser ($\sigma = 0.8$) and tighter ($\sigma = 0.2$) fairness constraints, respectively. For the Ncut values, we highlight the best overall results on each dataset in \textbf{bold} font and \underline{underline} the best fair result when $\sigma = 0.8$. FSC is marked by ``--'' when it does not provide any solution due to huge memory consumption for eigendecomposition on dense matrices.}\label{default}}
  \begin{tabular}{|c|ccc|ccc|ccc|ccc|ccc|}
  \hline
  \multirow{2}{*}{\textbf{Dataset}} & \multicolumn{3}{c|}{\textbf{SC}} & \multicolumn{3}{c|}{\textbf{FSC}} & \multicolumn{3}{c|}{\textbf{sFSC}} & \multicolumn{3}{c|}{\textbf{FNM} ($\sigma = 0.8$)} & \multicolumn{3}{c|}{\textbf{FNM} ($\sigma = 0.2$)} \\
  \cline{2-16}
  & Ncut & Balance & Time (s) & Ncut & Balance & Time (s) & Ncut & Balance & Time (s) & Ncut & Balance & Time (s) & Ncut & Balance & Time (s) \\
  \hline
  Facebook & \cellcolor{gray!20}\textbf{1.378} & \cellcolor{gray!20}0.458 & \cellcolor{gray!20}0.233 & \cellcolor{gray!20}1.401 & \cellcolor{gray!20}0.623 & \cellcolor{gray!20}0.252 & \cellcolor{gray!20}1.401 & \cellcolor{gray!20}0.623 & \cellcolor{gray!20}0.193 & \cellcolor{gray!20}\textbf{1.378} & \cellcolor{gray!20}0.458 & \cellcolor{gray!20}0.225 & \cellcolor{gray!50}1.550 & \cellcolor{gray!50}0.81 & \cellcolor{gray!50}0.312 \\
  German   & \cellcolor{gray!20}\textbf{1.433} & \cellcolor{gray!20}0.211 & \cellcolor{gray!20}0.357 & \cellcolor{gray!20}1.442 & \cellcolor{gray!20}0.583 & \cellcolor{gray!20}0.817 & \cellcolor{gray!20}1.442 & \cellcolor{gray!20}0.583 & \cellcolor{gray!20}0.581 & \cellcolor{gray!20}\textbf{1.433} & \cellcolor{gray!20}0.211 & \cellcolor{gray!20}0.486 & \cellcolor{gray!50}1.498 & \cellcolor{gray!50}0.8 & \cellcolor{gray!50}2.841 \\
  SBM      & \cellcolor{gray!20}\textbf{2.542} & \cellcolor{gray!20}0.226 & \cellcolor{gray!20}0.375 & \cellcolor{gray!20}2.619 & \cellcolor{gray!20}0.245 & \cellcolor{gray!20}0.921 & \cellcolor{gray!20}2.619 & \cellcolor{gray!20}0.245 & \cellcolor{gray!20}0.703 & \cellcolor{gray!20}\textbf{2.542} & \cellcolor{gray!20}0.226 & \cellcolor{gray!20}0.660 & \cellcolor{gray!50}3.348 & \cellcolor{gray!50}0.81 & \cellcolor{gray!50}2.608 \\
  DBLP     & \textbf{0.022} & 0 & 0.317 & 0.024 & 0 & 0.568 & 0.024 & 0 & 0.435 & \cellcolor{gray!20}\underline{0.050} & \cellcolor{gray!20}0.2 & \cellcolor{gray!20}0.658 & \cellcolor{gray!50}0.269 & \cellcolor{gray!50}0.8 & \cellcolor{gray!50}0.698 \\
  LastFM   & \textbf{0.119} & 0 & 0.526 & 0.185 & 0 & 102.2 & 0.185 & 0 & 1.462 & \cellcolor{gray!20}\underline{0.265} & \cellcolor{gray!20}0.2 & \cellcolor{gray!20}4.624 & \cellcolor{gray!50}0.699 & \cellcolor{gray!50}0.8 & \cellcolor{gray!50}7.014 \\
  Deezer   & \cellcolor{gray!20}\textbf{0.038} & \cellcolor{gray!20}0.406 & \cellcolor{gray!20}2.989 & -- & -- & -- & \cellcolor{gray!20}0.040 & \cellcolor{gray!20}0.406 & \cellcolor{gray!20}8.246 & \cellcolor{gray!20}\textbf{0.038} & \cellcolor{gray!20}0.406 & \cellcolor{gray!20}6.465 & \cellcolor{gray!50}0.216 & \cellcolor{gray!50}0.81 & \cellcolor{gray!50}9.120 \\
  Credit   & \cellcolor{gray!20}0.035 & \cellcolor{gray!20}0.738 & \cellcolor{gray!20}9.097 & -- & -- & -- & \cellcolor{gray!20}0.035 & \cellcolor{gray!20}0.665 & \cellcolor{gray!20}41.28 & \cellcolor{gray!20}\textbf{0.034} & \cellcolor{gray!20}0.748 & \cellcolor{gray!20}10.74 & \cellcolor{gray!50}0.049 & \cellcolor{gray!50}0.8 & \cellcolor{gray!50}12.33 \\
  Pokec-A  & \textbf{0.077} & 0 & 241.5 & -- & -- & -- & \textbf{0.077} & 0 & 686.9 & \cellcolor{gray!20}\underline{0.450} & \cellcolor{gray!20}0.209 & \cellcolor{gray!20}270.6 & \cellcolor{gray!50}2.701 & \cellcolor{gray!50}0.8 & \cellcolor{gray!50}254.4 \\
  Pokec-G  & \textbf{0.070} & 0.141 & 445.8 & -- & -- & -- & \textbf{0.070} & 0.141 & 1024 & \cellcolor{gray!20}\underline{0.128} & \cellcolor{gray!20}0.254 & \cellcolor{gray!20}525.0 & \cellcolor{gray!50}1.330 & \cellcolor{gray!50}0.81 & \cellcolor{gray!50}644.8 \\
  \hline
  \end{tabular}
  \end{center}
\end{table}
  

\smallskip\noindent\textbf{Trade-off between Quality and Fairness.}
We present the performance of four algorithms with different fairness constraints parameterized by $\sigma = 0.1, 0.2, \ldots, 1$ in Figure~\ref{fig-sigma}.
We ignore $\sigma = 0$ since no solution may exist for indivisibility.
Since the results of SC, FSC, and sFSC are independent of $\sigma$, they are drawn as horizontal lines in the figure.
For FNM, as the value of $\sigma$ decreases (when the fairness constraints become looser), the Ncut and balance values also decrease.
When $\sigma = 1$ (no fairness constraint), FNM returns partitions of similar quality to SC.
To our best knowledge, FNM is the only known algorithm that achieves different trade-offs between partition quality (i.e., \emph{Ncut}) and fairness (i.e., \emph{balance}) w.r.t.~$\sigma$.

We illustrate the performance of four algorithms as a function of the numbers of clusters $k$ in Figure~\ref{fig-k}.
We vary $k$ from $2$ to $10$ on three smaller datasets and from $5$ to $50$ on two larger datasets.
For each algorithm, the Ncut value increases with $k$.
Meanwhile, the balance of each algorithm generally drops with increasing $k$, and FNM is the only algorithm that consistently achieves a balance of at least $1 - \sigma$, i.e., guaranteeing the fairness constraints.
Furthermore, FNM has comparable Ncut values to SC, FSC, and sFSC and higher balances on most datasets when $\sigma = 0.8$.
But on the \emph{LastFM} dataset, FNM has higher Ncut values than other algorithms when $\sigma = 0.8$ because it performs reassignments to ensure a balance of at least $0.2$.
When $\sigma = 0.2$, as FNM assigns more nodes to non-closest clusters for fairness than when $\sigma = 0.8$, it often has inferior partition quality, especially when $k$ is large.

Note that the results examining the partition quality by varying $\sigma$ and $k$ on the remaining four datasets, as well as the time efficiency by varying $\sigma$ and $k$ on all the nine datasets, are deferred to Appendix~\ref{app:additional}.

\input{figure_sigma}
\input{figure_k}

\smallskip\noindent\textbf{Ablation Study.}
In the ablation study, we first run each embedding method to obtain a $k$-dimensional node embedding (for $k = 5, 20$) on each graph and use the same fair rounding in Algorithm~\ref{alg2} to produce two fair partitioning schemes with $\sigma = 0.8, 0.2$ from node vectors.
SC, FSC, sFSC, and FNM are thus renamed SE, FSE, sFSE, and rFSE since we only test their spectral embedding performance.
We report the Ncut values on five small datasets (since FSE and FW cannot provide any result on four large datasets) in Table~\ref{tbl-ablation-embed}.
Our method, rFSE in Algorithm~\ref{alg1}, achieves the best or second-best partition quality among all methods in almost all cases.
Here, SE and FSE/sFSE are special cases of rFSE when $\sigma = 1$ (i.e., w/o fairness) and $0$ (i.e., with strict proportionality), respectively.
As such, SE performs closely to rFSE when $\sigma = 0.8$, and FSE/sFSE provides similar embeddings to rFSE when $\sigma = 0.2$.
Especially, sFSE slightly outperforms rFSE when $\sigma = 0.2$. Since sFSE and rFSE both provide embeddings with fractional fairness constraints, which are looser than the original integral ones, using a tighter (fractional) constraint than required (i.e., equal to $1 - \sigma$) in the embedding phase might help improve the overall performance after rounding.
The partition quality of deep learning-based node embeddings (with or without fairness) is much inferior to that of rFSE and other spectral embeddings since they are not designed for graph partitioning.

Then, we evaluate the performance of each rounding method on node vectors provided by rFSE on five small datasets for $k = 5, 20$ and $\sigma = 0.8, 0.2$ and present the Ncut values in Table~\ref{tbl-ablation-round}.
Our method, FR in Algorithm~\ref{alg2}, performs best among the four fair methods we compare.
We find that fair $k$-means for i.i.d.~data in \cite{BeraCFN19} may not be appropriate to round embedding vectors though it adopts the same fairness notion as ours.
Despite having the lowest Ncut values, $k$-means++ cannot produce fair partitions.
When $\sigma = 0.8$, FR provides partitions closer to $k$-means++ than when $\sigma = 0.2$ since fewer or no reassignments are required as fairness constraints are looser.

\begin{table}[t]
\footnotesize
\setlength\tabcolsep{1pt}
\begin{center}
{\caption{Performance of different embedding methods on partition quality (i.e., \emph{Ncut values}). Here, the best result(s) are highlighted in a \textbf{bold} font, and the second best result(s) are \underline{underlined}.}\label{tbl-ablation-embed}}
\begin{tabular}{|c|c|c|c|c|c|c|c|c|c|c|c|c|c|c|c|}
    \hline
    \multirow{2}{*}{\textbf{Dataset}} &
    \multirow{2}{*}{$k$} &
    \multicolumn{7}{c|}{$\sigma=0.8$} &
    \multicolumn{7}{c|}{$\sigma=0.2$} \\ \cline{3-16} 
    & & \textbf{SE} & \textbf{FSE} & \textbf{sFSE} & \textbf{DW} & \textbf{N2V} & \textbf{FW} & \textbf{rFSE ($^*$)} & \textbf{SE} & \textbf{FSE} & \textbf{sFSE} & \textbf{DW} & \textbf{N2V} & \textbf{FW} & \textbf{rFSE ($^*$)} \\ \hline
    \multirow{2}{*}{Facebook} & 5
    & \underline{1.378} & 1.401 & 1.401 & \textbf{1.374} & 1.438 & 1.444 & \underline{1.378}
    & 1.676 & \textbf{1.536} & \textbf{1.536} & 1.676 & 1.575 & 1.689 & \underline{1.546} \\ \cline{2-16} 
    & 20
    & 13.840 & 13.887 & 14.057 & \textbf{13.050} & \underline{13.230} & 14.825 & 13.753
    & 14.861 & 14.882 & 14.882 & \underline{14.541} & 14.708 & 15.461 & \textbf{14.346} \\ \hline
    \multirow{2}{*}{German} & 5
    & \textbf{1.433} & 1.442 & 1.442 & 1.500 & 1.492 & 1.492 & \textbf{1.433}
    & 1.537 & \textbf{1.471} & \textbf{1.471} & 1.519 & 1.502 & 1.505 & \underline{1.498} \\ \cline{2-16} 
    & 20
    & \underline{11.856} & 11.879 & 11.869 & 11.977 & 11.922 & 12.638 & \textbf{11.811}
    & 12.927 & 12.889 & \underline{12.884} & 12.954 & 13.012 & 13.059 & \textbf{12.852} \\ \hline
    \multirow{2}{*}{SBM} & 5
    & \textbf{2.542} & 2.619 & 2.619 & 2.585 & 2.807 & 2.941 & \textbf{2.542}
    & 3.377 & \textbf{3.345} & \textbf{3.345} & 3.490 & 3.509 & 3.515 & \underline{3.348} \\ \cline{2-16} 
    & 20
    & \underline{16.812} & 16.998 & 17.012 & 17.294 & 17.357 & 17.430 & \textbf{16.785}
    & 17.844 & 17.812 & \textbf{17.784} & 18.131 & 18.194 & 18.104 & \underline{17.799} \\ \hline
    \multirow{2}{*}{DBLP} & 5
    & \underline{0.050} & \textbf{0.032} & \textbf{0.032} & 0.929 & 1.002 & 0.423 & \underline{0.050}
    & 1.003 & \textbf{0.261} & \textbf{0.261} & 0.995 & 1.029 & 0.645 & \underline{0.269} \\ \cline{2-16} 
    & 20
    & 1.381 & \textbf{0.941} & \underline{0.984} & 7.560 & 6.875 & 3.793 & 1.166
    & 3.170 & 2.871 & \textbf{2.779} & 8.691 & 8.119 & 6.235 & \underline{2.787} \\ \hline
    \multirow{2}{*}{LastFM} & 5
    & \underline{0.294} & 0.453 & 0.453 & 0.846 & 0.754 & 1.392 & \textbf{0.265}
    & 0.908 & \textbf{0.677} & \textbf{0.677} & 1.704 & 1.667 & 1.900 & \underline{0.699} \\ \cline{2-16} 
    & 20
    & \underline{3.983} & 4.666 & 4.943 & 7.313 & 6.942 & 8.316 & \textbf{2.922}
    & \underline{7.367} & 7.699 & 7.680 & 10.698 & 10.612 & 11.607 & \textbf{7.080} \\ \hline
    \multicolumn{2}{|c|}{Avg.~Ranking}
    & \underline{2.2} & 2.8 & 3.1 & 4.1 & 4.4 & 5.5 & \textbf{1.6}
    & 4.0 & 2.2 & \textbf{1.6} & 4.8 & 4.8 & 5.4 & \underline{1.7} \\ \hline
\end{tabular}
\end{center}
\end{table}

\begin{table}[t]
\footnotesize
\setlength\tabcolsep{1pt}
\begin{center}
{\caption{Performance of different rounding methods on partition quality (i.e., \emph{Ncut values}). Here, the best result(s) are highlighted in a \textbf{bold} font, and the second best result(s) are \underline{underlined}. Note that the results of $k$-means++ do not satisfy the fairness constraints and thus are just presented to show the ``price of fairness'' in the rounding process.}\label{tbl-ablation-round}}
\begin{tabular}{|c|c|c|c|c|c|c|c|c|c|c|c|}
    \hline
    \multirow{2}{*}{\textbf{Dataset}} &
    \multirow{2}{*}{$k$} &
    \multicolumn{5}{c|}{$\sigma=0.8$} &
    \multicolumn{5}{c|}{$\sigma=0.2$} \\ \cline{3-12} 
    & & $k$-means++ & \textbf{K+R} & \textbf{FK} & \textbf{IP} & \textbf{FR ($^*$)} & $k$-means++ & \textbf{K+R} & \textbf{FK} & \textbf{IP} & \textbf{FR ($^*$)} \\ \hline
    \multirow{2}{*}{Facebook} & 5 & (1.378) & \textbf{1.378} & \textbf{1.378} & \textbf{1.378} & \textbf{1.378} & (1.381) & 1.640 & \underline{1.576} & 1.599 & \textbf{1.546} \\ \cline{2-12} 
    & 20 & (12.998) & 14.161 & 13.947 & \textbf{13.333} & \underline{13.753} & (12.586) & 14.991 & 14.887 & \underline{14.615} & \textbf{14.346} \\ \hline
    \multirow{2}{*}{German} & 5 & (1.433) & \textbf{1.433} & 1.442 & \textbf{1.433} & \textbf{1.433} & (1.479) & 1.526 & \textbf{1.498} & 1.499 & \textbf{1.498} \\ \cline{2-12} 
    & 20 & (11.629) & 11.995 & 11.948 & \underline{11.883} & \textbf{11.811} & (11.733) & 13.611 & 13.166 & \underline{13.069} & \textbf{12.852} \\ \hline
    \multirow{2}{*}{SBM} & 5 & (2.542) & \textbf{2.542} & \textbf{2.542} & \textbf{2.542} & \textbf{2.542} & (2.568) & \underline{3.350} & \underline{3.350} & \underline{3.350} & \textbf{3.348} \\ \cline{2-12} 
    & 20 & (16.767) & 16.839 & 16.834 & \underline{16.803} & \textbf{16.785} & (16.908) & 18.006 & \underline{17.926} & 17.889 & \textbf{17.799} \\ \hline
    \multirow{2}{*}{DBLP} & 5 & (0.022) & 0.062 & 0.067 & \textbf{0.049} & \underline{0.050} & (0.024) & \underline{0.466} & 0.521 & 0.500 & \textbf{0.269} \\ \cline{2-12} 
    & 20 & (0.400) & 1.381 & \underline{1.066} & \textbf{1.007} & 1.166 & (0.404) & 4.596 & 4.104 & \underline{3.452} & \textbf{2.787} \\ \hline
    \multirow{2}{*}{LastFM} & 5 & (0.152) & 0.320 & 0.272 & \underline{0.266} & \textbf{0.265} & (0.184) & 0.801 & 0.805 & \underline{0.791} & \textbf{0.699} \\ \cline{2-12} 
    & 20 & (2.005) & 5.751 & 5.139 & \underline{3.211} & \textbf{2.922} & (2.257) & 10.723 & 11.144 & \underline{9.911} & \textbf{7.080} \\ \hline
    \multicolumn{2}{|c|}{Avg.~Ranking} & -- & 2.9 & 2.8 & \textbf{1.4} & \textbf{1.4} & -- & 3.4 & 2.8 & \underline{2.4} & \textbf{1.0} \\ \hline
\end{tabular}
\end{center}
\end{table}


\section{Conclusion}

This paper investigated the $(\bm{\alpha}, \bm{\beta})$-proportionally fair normalized cut graph partitioning problem.
We proposed a novel algorithm, FNM, consisting of an extended spectral embedding method and a $k$-means-based rounding scheme to provide a node partitioning with a small Ncut value on a graph while strictly following the proportional fairness constraints.
The comprehensive experimental findings confirmed the superior performance of FNM in terms of partition quality, fairness, and efficiency.
In future work, we will generalize our algorithm to handle other notions of fairness, e.g., individual fairness~\cite{MahabadiV20, gupta2022consistency}, in graph partitioning problems.


\section*{Ethics Statement}
The implication of this work is that it enables us to find proportionally representative graph partitions. This is relevant for various real-world applications to reduce harmful biases of traditional algorithms for this problem. As such, we do not foresee situations in which our method may be directly misused. This is based on the assumption that the user aims to improve social outcomes rather than having a negative impact.

\section*{Acknowledgements}
This work was supported by the National Natural Science Foundation of China under grant no.~62202169. We would like to thank the anonymous reviewers for their comments, which helped improve this work considerably.

\bibliography{refs}

\begin{comment}
\section{System Architecture}
\label{appendix:architecture}
\system has a novel modularized system architecture with three key components: 
\emph{StreamManager}, 
\emph{TxnManager} and \emph{TxnScheduler}. 
These components are instantiated in each thread locally.
The execution outline of \system is presented in Algorithm~\ref{alg:algo}.
Transactional stream processing is continuous and potentially never ends (Line 1$\sim$8).
The dependency resolution and execution of state transactions are separated into two non-overlapping phases by punctuations~\cite{Tucker:2003:EPS:776752.776780} (Line 2 and 5), which guarantees that no subsequent input event will have a smaller timestamp. 
Effectively, a batch of state transactions is collected during the first phase, and processed during the second phase.

In the first phase (i.e., stream processing phase), 
the \emph{StreamManager} conducts preprocessing for every input event ($e$). Similar to some prior works~\cite{tstream}, state transactions may be issued but not immediately processed during preprocessing (Line 3).
The \emph{pre\_processing} and \emph{post\_processing} functions are exposed as APIs to users.
The \emph{TxnManager} handles dependency resolution (Line 4) among state transactions and insert decomposed operations to construct a \tpg. We discuss the detailed two-phase \tpg construction process in Section~\ref{subsec:construction}.

In the second phase  (i.e., transaction processing phase), 
the \emph{TxnManager} is first involved again to refine (Line 6) the constructed \tpg with further dependency resolution.
The \emph{TxnScheduler} 
schedules operations for concurrent execution based on the constructed \tpg according to the three dimensions of scheduling decisions (Line 7). 
In particular, a scheduling decision model $M$ is instantiated based on the constructed \tpg (Line 14).
\textbf{\circled{1}} Guided by $M$, execution threads adopt an exploration strategy (Section~\ref{subsec:explore}) to explore the constructed \tpg for operations available to be scheduled constrained by dependencies. 
\textbf{\circled{2}} 
During exploration, one or multiple operations may be treated as the 
% basic 
unit of scheduling (Section~\ref{subsec:granularity}). 
Subsequently, \textbf{\circled{3}} every thread executes operation(s) in the unit of scheduling with various abort handling mechanisms (Section~\ref{subsec:abort_handling}).
Only when state transactions are processed (i.e., committed or aborted) can the associated input events be postprocessed (Line 8) by the \emph{StreamManager} based on transaction processing results.
\end{comment}

\begin{comment}
\begin{algorithm}
\footnotesize
    \KwData{$e$ \tcp{Input event}}
    \KwData{$txn_{ts}$ \tcp{State transaction}}
    \KwData{$G$ \tcp{The currently constructed TPG}}
    \While{!finish processing of input streams}{
        \eIf(\tcp*[h]{Phase 1}){\text{$e$ is not a $punctuation$}}{
                $txn_{ts}$ $\gets$ PRE\_Processing($e$)\;
                \textbf{TPG\_Construction}($G$, $txn_{ts}$)\; 
          }(\tcp*[h]{Phase 2}){
                \textbf{TPG\_Refinement}($G$)\; 
                \textbf{TXN\_Scheduling}($G$)\; 
                POST\_Processing()\;
          }
    }
    
    \SetKwFunction{FMain}{TPG\_Construction}
    \SetKwProg{Fn}{Function}{:}{}
    \Fn{\FMain{$G$, $txn_{ts}$}}{
        $O_{1..k}$ $\gets$ \textbf{Partition} $txn_{ts}$\;
        \ForEach{\text{operation $O_{i}$ $\in$ $O_{1..k}$}}{
            \textbf{Identify} its \ld\;
            $G$ $\gets$ $G$ + $O_{i}$ \;
        }
    }
    \SetKwFunction{FMain}{TPG\_Refinement}
    \SetKwProg{Fn}{Function}{:}{}
    \Fn{\FMain{$G$}}{
        \ForEach{\text{vertex $e_{i}$ $\in$ $G$}}{
            \textbf{Identify} its \td, \pd\;
        }
    }
    
    \SetKwFunction{FMain}{TXN\_Scheduling}
    \SetKwProg{Fn}{Function}{:}{}
    \Fn{\FMain{$G$}}{
        $M$ $\gets$ Instantiated with $G$;\tcp{A decision model}
        \While{!finish scheduling of $G$
        }{
          \textbf{\circled{2}} $Scheduling Unit$ $\gets$ \textbf{\circled{1}} \emph{Explore}($G$, $M$)\; 
            \textbf{\circled{3}} \emph{Execute with Abort Handling} ($Scheduling Unit$)\; 
        }
    }
  \caption{Execution Outline of \system}
  \label{alg:algo}
\end{algorithm}
\end{comment}

\end{document}
