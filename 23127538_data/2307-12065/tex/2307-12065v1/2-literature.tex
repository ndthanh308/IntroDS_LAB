\subsection{Other Related Works}

\noindent\textbf{Min-Cut Graph Partitioning.}
Partitioning nodes in a graph into disjoint subsets to minimize an objective function, such as \emph{ratio cut} \cite{HagenK92}, \emph{normalized cut} \cite{ShiM00}, and \emph{Cheeger cut} \cite{Cheeger71}, is a fundamental combinatorial optimization problem.
Since all those graph partitioning problems are NP-hard \cite{HagenK92, ShiM00}, different heuristic algorithms were designed for them, among which spectral methods \cite{HagenK92, ShiM00, NgJW01, HanXN17} have attracted the most attention.
The basic idea of spectral methods is to relax the original integer minimization problems into continuous optimization problems, which are solved by computing the eigenvectors of the Laplacian matrix, and to find the partitioning using $k$-means or an alternative rounding method.
Other relaxation-based methods \cite{JiaDDX16, ChenHNHYH18, LiNL18} improved the efficiency over original spectral methods by avoiding eigendecomposition.
However, none of them incorporate the notion of \emph{fairness} into graph partitioning problems.

\smallskip\noindent\textbf{Fair Clustering.}
There has been rich literature on fair clustering algorithms.
Chierichetti \emph{et al.} \cite{Chierichetti0LV17} first introduced the notion of \emph{proportional fairness} for clustering and then proposed fairlet decomposition algorithms for fair $k$-median and $k$-center clustering in the case of two groups.
Following this work, the problem has been further generalized to handle more than two groups \cite{HuangJV19}, permit lower and upper bounds on the fraction of points from a group \cite{Bercea0KKRS019, BeraCFN19, Gupta_2023}, and allow probabilistic group memberships \cite{EsmaeiliBT020}.
In addition, more efficient fair clustering algorithms have been proposed based on faster fairlet computation \cite{BackursIOSVW19} or coresets \cite{HuangJV19, SchmidtSS19}.
Then, several studies \cite{DavidsonR20, ZikoYGA21, EsmaeiliBSD21} explored how to achieve better trade-offs between the fairness and clustering objectives.
Other fair variants of clustering problems focused on different fairness notions, including individual-level fairness \cite{MahabadiV20, NegahbaniC21, VakilianY22}, fair center selection \cite{KleindessnerAM19, ThejaswiOG21}, and minimax losses among groups \cite{MakarychevV21, GhadiriSV21, ChlamtacMV22}.
However, all the above methods are primarily designed for i.i.d.~data rather than graph data.
They cannot be directly applied to graph partitioning because graphs are high-dimensional and sparse, and they will incur huge computational costs and often return inferior results due to the curse of dimensionality.

There were also a few studies on fairness in spectral methods and other graph clustering problems.
In addition to \cite{KleindessnerSAM19, pmlr-v206-wang23h}, Gupta and Dukkipati \cite{gupta2022consistency} further investigated spectral clustering with individual fairness constraints.
Moreover, Ahmadian \emph{et al.} \cite{AhmadianE0M20}, Friggstad and Mousavi \cite{FriggstadM21a}, and Ahmadian and Negahbani \cite{abs-2206-05050} studied fair correlation clustering on signed graphs.
Anagnostopoulos \emph{et al.} \cite{Anagnostopoulos20} applied spectral methods to fair densest subgraph discovery on graphs.
These algorithms are interesting but not comparable to our algorithm.
