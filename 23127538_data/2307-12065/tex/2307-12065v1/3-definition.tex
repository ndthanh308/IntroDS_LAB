\section{Problem Definition}
\label{sec-def}

\smallskip\noindent\textbf{Normalized-Cut Graph Partitioning.}
Define $[n] := \{1, 2,$ $\ldots, n\}$ for any positive integer $n$.
Let $G = (V, E)$ be an undirected graph, where $V = [n]$ is the set of $n$ nodes and $E \subseteq V \times V$ is the set of edges.
We use $ \bm{W} = (w_{ij})_{i,j \in [n]}$ to denote the adjacency matrix of $G$, where each entry $w_{ij} \geq 0$ is the weight of edge $(i,j)$ (always equal to $1$ for the unweighted case) if it exists or $0$ otherwise.
We consider $w_{ii} = 0$ for any $i \in [n]$.
The degree matrix $\bm{D} = (d_{i})_{i \in [n]}$ is a diagonal matrix with the degree $d_{i} = \sum_{j=1}^n w_{ij} \geq 0$ of each node $i$ on its diagonal.
Given an undirected graph $G$, an integer $k \geq 2$, we aim to find a partitioning $\mathcal{C} = \{C_1, \ldots, C_k\}$ of $V$ into $k$ disjoint clusters, i.e., $\bigcup_{l = 1}^{k} C_l = V$ and $C_l \cap C_{l'} = \emptyset$ for any $l \neq l' \in [k]$, to minimize the normalized cut \cite{YuS03} (Ncut) value as follows:
\begin{equation}\label{eq-Ncut}
  \mathsf{Ncut}(\mathcal{C}) := \sum_{l=1}^k \frac{\mathsf{cut}(C_l)}{\mathsf{vol}(C_l)} = \sum_{l=1}^k \frac{\sum_{i \in C_l, j \in V \setminus C_l} w_{ij}}{\sum_{i \in C_l, j \in V} w_{ij}}.
\end{equation}
The Ncut minimization problem is well-known as NP-hard \cite{ShiM00}.

\smallskip\noindent\textbf{Fairness Constraint.}
In the fair variant of graph partitioning, we consider that the node set $V$ consists of several demographic groups defined by a categorical sensitive attribute, e.g., gender or race.
Formally, suppose that $V$ is divided into $m$ disjoint groups indexed by $[m]$, and an indicator function $\phi: [n] \mapsto [m]$ maps each node $i \in [n]$ to the group $\phi(i)$ it belongs to.
Let $ V_c = \{ i \in [n] : \phi(i) = c \} $ be the subset of nodes from group $c$ in $V$.
We assume that $ \bigcup_{c = 1}^{m} V_c = V $ and $ V_c \cap V_{c'} = \emptyset, \forall c \neq c' \in [m]$.
For ease of presentation, we denote the group membership as an indicator matrix $\bm{M} \in \{0,1\}^{n \times m}$, where $\bm{M}_{i,c} = 1$ if $\phi(i) = c $ and $0$ otherwise.
We follow a notion of \emph{range-based proportional fairness} in \cite{BeraCFN19} to require that every demographic group is approximately proportionally represented in all the $k$ clusters.
We define the fairness constraint by two vectors $\bm{\alpha}, \bm{\beta} \in [0,1]^{m}$ that specify the upper and lower bounds $\alpha_c, \beta_c$ on the percentage of nodes from group $c$.
We say a partitioning $\mathcal{C}$ is $(\bm{\alpha}, \bm{\beta})$-\emph{proportionally fair} if $\beta_c \leq \frac{\lvert V_c \cap C_l \rvert}{\lvert C_l \rvert} \leq \alpha_c$ for any $C_l$ and $V_c$.
In practice, we parameterize $\bm{\alpha}, \bm{\beta}$ by a fairness variable $\sigma \in [0, 1]$ as $\alpha_c = \min\{r_c / (1 - \sigma), 1\}$ and $\beta_c = r_c \cdot (1 - \sigma)$, where $r_c = |V_c| / n$.\footnote{Any other parameterization scheme (e.g., $\beta_c = r_c \cdot (1 - \sigma)$ and $\alpha_c = \beta_c + \sigma$) is also compatible with our formulation as long as it guarantees that $\beta_c \leq \alpha_c$ and $\beta_c, \alpha_c \in [0, 1]$ for any $c \in [m]$.}
For example, if the percentage of females is 60\% in the population of all nodes, the fairness constraint requires that the percentage of females in each cluster should be between 48\% and 75\% when $\sigma = 0.2$.
The value of $\sigma$ can be interpreted as how loose the fairness constraint is, where $\sigma = 0$ corresponds to every group in each cluster having the same ratio as that group in the population, and $\sigma = 1$ corresponds to no fairness constraint at all.
Given all the above notions, we formally define the normalized-cut graph partitioning problem under $(\bm{\alpha}, \bm{\beta})$-proportional fairness as follows.
\begin{definition}\label{def-fnm}
  Given an undirected graph $G = (V, E)$, a set of $m$ groups $V_1, \ldots, V_m \subseteq V$, two fairness vectors $ \bm{\alpha}, \bm{\beta} \in [0,1]^m$, and an integer $k \geq 2$, find an $(\bm{\alpha}, \bm{\beta})$-proportionally fair partitioning $ \mathcal{C} = \{C_1, \ldots, C_k \} $ of $V$ into $k$ disjoint clusters such that $\mathsf{Ncut}(\mathcal{C})$ in Eq.~\ref{eq-Ncut} is minimized.
\end{definition}
The problem in Definition~\ref{def-fnm} is NP-hard since the vanilla Ncut minimization problem is its special case when $m = 1$.
Next, we will approach the problem by extending the spectral Ncut minimization algorithm.
Note that our method can be adapted to \emph{ratio cut} \cite{HagenK92} and any other cut measure with an equivalent spectral formulation.
This paper focuses on Ncut due to its prevalence and space limitations.
