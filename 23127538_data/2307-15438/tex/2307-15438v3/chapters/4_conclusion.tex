%******************************************************************************%
%******************************    CONCLUSION   *******************************%
%******************************************************************************%
\section{Conclusion}
In this work, a novel temperature management system called APaTheCSys was assessed on real-world space equipment that was operating on the ground. APaTheCSys was able to sustain a simulated heavy load for 12 hours without surpassing an enforced temperature threshold by constantly modifying the available processor resources.

Power control in this study was limited to the CPUs' processing power, but it might also encompass additional functions like switching off other devices or altering their energy modes. 

Overall, the method could be suitable for small payloads from satellites, spacecraft, or even planetary probes on upcoming space exploration missions, provided they are capable of hosting onboard intelligence and have the ability to control hardware from a software interface. This aligns with other research on autonomous systems for collision avoidance, docking, and debris removal and constitutes a step toward fully autonomous robotic missions.