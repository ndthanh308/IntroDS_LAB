%******************************************************************************%
%*******************************    ABSTRACT   ********************************%
%******************************************************************************%
\begin{abstract}
In small satellites there is less room for heat control equipment, scientific instruments, and electronic components. Furthermore, the near proximity of electronic components makes power dissipation difficult, with the risk of not being able to control the temperature appropriately, reducing component lifetime and mission performance.
To address this challenge, taking advantage of the advent of increasing intelligence on board satellites, an autonomous thermal control tool that uses deep reinforcement learning is proposed for learning the thermal control policy onboard.
The tool was evaluated in a real space edge processing computer that will be used in a demonstration payload hosted in the International Space Station (ISS).
The experiment results show that the proposed framework is able to learn to control the payload processing power to maintain the temperature under operational ranges, complementing traditional thermal control systems.
\end{abstract}

%
\begin{keywords}
Smart thermal control, deep reinforcement learning, smallsat, spacecraft, onboard artificial intelligence.
\end{keywords}



%% Se podria haber hecho con la potencia consumida en vez de la temperatura, el problema es que la temperatura depende de otros factores externos y es lo que puede afectar realmente a los equipos.
%% Cuántos recursos necesita esto para funcionar? Corre en cpu con tales o cuales características.

%% Cambio de algoritmo. Por qué SAC y no PPO? (descarte de experiencia en escenarios en que no se puede generar tanta experiencia todo el tiempo.
%%Convergencia en poco tiempo
%% Funciona tanto en simulaciones como en un escenario real

