\section{Introduction}
In satellites, thermal control is a critical task that consists of maintaining the temperature of all its components within an acceptable range. Sensors, batteries, optics, joints, and bearings are some of the most affected components when the temperature goes below or above the operational range.

In small satellites, there is less room for heat control equipment, scientific instruments, and electronic components. In addition, the proximity of the electronic components makes it challenging to dissipate power. High temperatures do activate many degradation pathways that reduce component lifetime \cite{electronics8121423}.

Generally speaking, thermal control is achieved primarily by controlling heat fluxes, and the techniques for doing so are classified as active or passive thermal control systems. SmallSats typically favor passive systems over active ones because of their lower cost, volume, weight, and risk.
For components with more stringent temperature limitations or greater heat loads, active thermal management approaches, however, have proven to be more successful in maintaining closer temperature control. Unfortunately, due to power, mass, and volume requirements, the bulk of them are challenging to integrate into small satellites \cite{nasa}.

With the advent of onboard intelligence in SmallSats, it is increasingly feasible to assist tasks like navigation, calibration, or fault detection in place using tools like artificial intelligence which enable fast reaction and adaptation in an adverse environment like space. Because the heat created by the spacecraft is one of the primary contributors to the spacecraft's temperature, active management of it is a potential challenge to be tackled on board.

The spacecraft may learn a thermal control strategy by interacting with the environment. It might, among other things, turn on or off non-critical systems, alter clock speed, parallelize or serialize processing, halt or postpone processing, and so on to control power consumption and, as a result, adjust the emitted thermal radiation adaptively to different external situations (e.g., eclipses, direct sunlight, albedo variations).

It has been proposed in prior publications to employ model-based approaches like fuzzy \cite{wang2021research, dong2012fuzzy, 5365072} or Proportional Integral Derivative (PID) \cite{noauthor_2019-lb} controllers to accomplish adaptive heat regulation. 
These approaches aim to model thermal dynamics with simplified mathematical models, but they have some drawbacks. On the one hand, they struggle to accurately simulate the complexities of thermal dynamics and the various influencing factors \cite{gao2019energy}. On the other hand, fuzzy control systems require expert knowledge to define control rules, whereas PID requires manual adjustment of key parameters, which can be time-consuming and hinder adaptive online adjustments. In \cite{xiong2020intelligent}, an actor-critic deep reinforcement learning (DRL) algorithm is used to solve a live tuning problem for a PID controlling the heater voltage of a space telescope.

In this work, an on-board system called \emph{APaTheCSys} (\textbf{A}utonomous \textbf{Pa}yload \textbf{The}rmal \textbf{C}ontrol \textbf{Sys}tem) that uses DRL for helping in the thermal control of the payload is proposed, letting the agents to learn the thermal behavior of the payload on each scenario in a model-free manner. The system is evaluated in real hardware that has been included in the conceptual space edge computing mission called IMAGIN-e\footnote{IMAGIN-e is a demostration space edge computing payload that will be hosted in the International Space Station, carried out in a collaboration agreement by Thales Alenia Space and Microsoft}.

