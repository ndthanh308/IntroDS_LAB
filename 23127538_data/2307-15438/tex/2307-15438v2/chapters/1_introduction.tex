%******************************************************************************%
%*****************************    INTRODUCTION   ******************************%
%******************************************************************************%
\section{Introduction}
In satellites, thermal control is a critical task that consists of maintaining the temperature of all its components within an acceptable range. Sensors, batteries, optics, joints and bearings are some of the most affected components when the temperature goes below or above the operational range.

%Generally speaking, thermal control is achieved primarily by controlling heat fluxes, and the techniques for doing so are classified as active thermal control systems (which include electric heaters, louvers, cryocoolers, thermoelectric coolers, and fluid loops) and passive thermal control systems (which in turn include surface finishes, multi-layer insulation, thermal straps, sunshades, heat pipes, interface conductance and materials, and heat storage).%

In small satellites there is less room for heat control equipment, scientific instruments, and electronic components. In addition, the near proximity of the electronics makes it challenging to dissipate power. Despite the small form factor of these embedded devices, the integrated high-end cores make temperatures rise swiftly, reaching catastrophic levels in short periods of time if the cooling strategy is not able to dissipate enough heat. High temperatures do activate many degradation pathways that reduce component lifetime \cite{electronics8121423}.

SmallSats typically favor passive systems over active ones because of their lower cost, volume, weight, and risk.
For components with more stringent temperature limitations or greater heat loads, active thermal management approaches, however, have proven to be more successful in maintaining closer temperature control. Unfortunately, due to power, mass, and volume requirements, the bulk of them are challenging to integrate into small satellites \cite{nasa}.

With the advent of onboard intelligence in SmallSats, it is increasingly feasible to assist tasks like navigation, calibration or fault detection in place using tools like artificial intelligence which enable fast reaction and adaptation in an adverse environment like space. %As the heat generated by the spacecraft itself is one of the main contributors to the temperature of the spacecraft, the active control of it in relation to the other heat sources (solar heating reflected by the orbited planet, the solar heating directly received from the sun, and the infrared heating of the orbited planet), the spacecraft's heat dissipation, and the temperature limits of each component, is a candidate problem to be solved on-board.%
As the heat generated by the spacecraft itself is one of the main contributors to the temperature of the spacecraft, the active control of it is a candidate problem to be solved on-board. 

It has been proposed in prior publications to employ model-based approaches like fuzzy or Proportional Integral Derivative (PID) controllers to accomplish adaptive heat regulation. These approaches intend to model thermal dynamics using simplified mathematical models.
However, fuzzy and PID control systems struggle to accurately simulate the complexity of the thermal dynamics and the different affecting elements \cite{gao2019energy}.

Specifically, fuzzy control has been proposed for controlling thermoelectric modules \cite{wang2021research}, loop heat pipes and variable emittance radiators \cite{dong2012fuzzy} and MEMS (Micro Electro Mechanical Systems)-based \cite{5365072} thermal control systems. The drawback with fuzzy control systems is that they use rule-based control, requiring expert knowledge for defining those rules. 

PID thermal control has been commonly utilized in space telescopes but traditionally its key parameters, which affect the temperature control precision, are adjusted manually taking time and preventing from an adaptive online adjustment. In \cite{xiong2020intelligent}, the usage of an actor-critic deep reinforcement learning (DRL) for actively tuning the parameters of a PID for controlling the heater voltage of a space telescope is proposed.

The spacecraft may learn a thermal control strategy by interacting with the environment. It might, among other things, turn on or off non-critical systems, alter clock speed, parallelize or serialize processing, halt or postpone processing, and so on to control power consumption and, as a result, adjust the emitted thermal radiation adaptively to different external situations (e.g., eclipses, direct sun-light, albedo variations).

In this work, an on-board system called \emph{APaTheCSys} (\textbf{A}tonomous \textbf{Pa}yload \textbf{The}rmal \textbf{C}ontrol \textbf{Sys}tem) that uses deep reinforcement learning (DRL) for helping in the thermal control of the payload is proposed, letting the agents to learn on board the thermal behavior of the payload on each scenario in a model-free manner. 

The system is evaluated both in a naive simulation environment and in real hardware that will be included in the future space edge computing mission called IMAGIN-e\footnote{IMAGIN-e is a future space edge computing mission that will be hosted in the International Space Station, carried out in a collaboration agreement by Thales Alenia Space and Microsoft}.

%deficit of onboard intelligence \cite{mateo2022orbit}... Such onboard intelligence could help automate analysis in orbit