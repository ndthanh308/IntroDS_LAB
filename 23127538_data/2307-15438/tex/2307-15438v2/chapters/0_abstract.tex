%******************************************************************************%
%*******************************    ABSTRACT   ********************************%
%******************************************************************************%
\begin{abstract}
In small satellites there is less room for heat control equipment, scientific instruments, and electronic components. Furthermore, the near proximity of the electronics makes power dissipation difficult, with the risk of not being able to control the temperature appropriately, reducing component lifetime and mission performance.
To address this challenge, taking advantage of the advent of increasing intelligence on board satellites,  a deep reinforcement learning based framework that uses Soft Actor-Critic algorithm is proposed for learning the thermal control policy onboard.
The framework is evaluated both in a naive simulated environment and in a real space edge processing computer that will be shipped in the future IMAGIN-e mission and hosted in the ISS.
The experiment results show that the proposed framework is able to learn to control the payload processing power to maintain the temperature under operational ranges, complementing traditional thermal control systems.
\end{abstract}
%
\begin{keywords}
Smart thermal control, deep reinforcement learning, smallsat, spacecraft, onboard artificial intelligence.
\end{keywords}



%% Se podria haber hecho con la potencia consumida en vez de la temperatura, el problema es que la temperatura depende de otros factores externos y es lo que puede afectar realmente a los equipos.

%% 


