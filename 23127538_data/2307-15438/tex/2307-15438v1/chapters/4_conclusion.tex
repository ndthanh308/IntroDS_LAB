%******************************************************************************%
%******************************    CONCLUSION   *******************************%
%******************************************************************************%
\section{Conclusion}
In this study, a novel temperature control method called APaTheCSys was evaluated in real space equipment, operated under ground conditions. 
After an exploration of 5 hours, APaTheCSys was able to maintain a simulated high load for 12 hours without exceeding an imposed threshold of 55 degrees Celsius starting from an idle temperature of 50 degrees Celsius by adjusting the number of cpus available for processing and their frequency dynamically.
The gents must interact with the environment several times before coming up with a viable policy, particularly at the beginning, during the cold start, as the initial performance of agents on real hardware was not better than using a random policy. 

In this study, a condensed simulated environment was used for the initial validation of the algorithm. Future experimentation should address the utility of using these reduced environments to pretrain the policy to develop acceptable initial behavior, then allowing the agent to refine the solution in response to actual environmental conditions.

Since it is not possible to adjust the computing power to regulate the temperature of the equipment if no workload is present, other traditional thermal control systems should be used in conjunction to ensure the nominal temperature conditions of the equipment and avoid damage to sensitive components. In this context, APaTheCSys should be seen as an auxiliary thermal control method that maximizes the efficiency with which equipment is used in proportion to the dynamic conditions of the surrounding environment.

The experimentation was confined to controlling the processing power of the CPUs contained in the SoC used, but activities such as turning off other equipment or changing their energy modes might be included in the set of actions to regulate the temperature of the subsystems. However, the addition of too many dimensions to the action and state spaces increases complexity and might make it difficult for learning models to converge.

Overall, the method could be suitable for small payloads from satellites, spacecraft, or even planetary probes on upcoming space exploration missions, provided they are capable of hosting onboard intelligence and have the ability to control the hardware from a software interface. And this, like other active lines of research such as autonomous systems for collision avoidance, docking or active debris removal constitutes a further step towards autonomous robotic missions.

Finally, APaTheCSys will be evaluated under space conditions in the IMAGIN-e future space edge computing mission, that will be hosted in the ISS.
