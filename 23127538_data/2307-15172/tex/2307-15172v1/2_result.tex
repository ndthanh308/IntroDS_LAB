


\section*{Results}

We conducted a repeated-measures ANOVA to analyze the results of the aforementioned 12 sessions. To account for variability of participants' baseline performance, each metric mentioned below was normalized to have a zero mean and unit standard deviation within each participant.

\subsection*{Eyerofeedback Improves Human Attention}
\label{sec: R1}
In the attention task, participants were instructed to prioritize accuracy. Response time and the number of missed trials were used as metrics to quantify their performance. Our findings revealed that eyerofeedback significantly improved human attention compared to the control settings, particularly in more difficult (long duration) settings. 
Specifically, the ANOVA showed a significant main effect of feedback type on response time (F2,40 = 3.3135, P = 0.0466<0.05) (Fig.~\ref{f2}(b,c) and Appendix Fig.~S2). Post-hoc comparisons showed significantly faster response times with both the \textit{stationary} and \textit{filter} feedback in comparison to \textit{silence}.
%
There was also a significant interaction between feedback type and task duration (F2,40 = 3.8472, P = 0.0296<0.05), indicating larger differences between feedback conditions in the long duration trials compared to short.
In addition, we observed a 
%
significant difference on response time across feedback types during long duration with distraction (F2,40 = 3.466, P = 0.0409 < 0.05) and without distraction (F2,40 = 3.7076, P = 0.0333 < 0.05). 
Further pairwise comparisons within the long duration setting demonstrated significant differences in response time for 
%
\textit{stationary} feedback compared to \textit{silence} without external distraction (\textit{stationary}: -0.5468 ± 0.3768 (mean ± SD) vs. \textit{silence}: -0.1726 ± 0.5523, P = 0.0367 < 0.05), as well as for \textit{filter} feedback compared to \textit{silence} with external distraction (\textit{filter}: 1.1110 ± 0.4449 vs. \textit{silence}: 1.4790 ± 0.4096, P = 0.0243 < 0.05). These results indicate that eyerofeedback has the potential to accelerate human response time and enhance attention levels.

%

The ANOVA also revealed a significant main effect of feedback type on the number of missed trials (F2,40 = 4.3364, P = 0.0197<0.05). However, follow-up tests showed the effect of feedback on missed trials varied across conditions. Specifically, feedback type only impacted missed trials in the long duration trials with distraction (F2,40 = 3.6115, P = 0.0362<0.05). No significant differences between feedback types emerged for missed trials in the short duration trials or long duration without distraction (P > 0.05).
Furthermore, we did not find any significant differences in accuracy for feedback type across all four combinations of two duration $\times$ two distraction conditions (P > 0.05). This result is reasonable considering that participants were instructed to prioritize accuracy over response time during the task.



The varied effects of eyerofeedback across conditions may be attributed to differences in baseline task performance. Response times were faster and fewer trials were missed in the less demanding short duration condition compared to the long duration condition. Additionally, the presence of external distractions led to slower response times and more missed trials than no-distraction conditions. Thus, eyerofeedback conferred larger benefits when baseline attention performance was lower due to greater attentional demands from long durations or external distractions.
This interpretation is supported by the significant differences in response time between the two duration conditions 
%
(F1,20 = 118.1033, P < 0.001) and between the two distraction settings (F1,20 = 921.3792, P < 0.001). Furthermore, significant differences were observed in the number of missed trials between the two duration conditions (F1,20 = 19.0266, P = 0.0003 < 0.001) and between the two distraction settings (F1,20 = 9.967, P = 0.005 < 0.01). These findings substantiate our hypothesis that eyerofeedback holds promise to improve individuals' attention performance by reducing response time and the occurance of missed trials. Nonetheless, these benefits varied as a function of task demands, emerging to a greater extent when sustained attention was challenged by longer durations or external distractions. 


% Figure environment removed

\subsection*{Subjective Feelings Reveal Eyerofeedback Effect on Human Attention}
\label{sec: R2}


Subjective experiences were assessed via a questionnaire administered after each session (Fig.~\ref{f2}(d) and Appendix Fig.~S3).
Participants reported heightened attention to the task (Q1) and screen center (Q2) with \textit{filter} versus \textit{silence} feedback. This was evident in both the short no-distraction (Q1: F2,40 = 6.3132, P = 0.0041; Q2: F2,40 = 6.9047, P = 0.0027) and long distraction conditions (Q1: F2,40 = 5.6305, P = 0.007; Q2: F2,40 = 8.0503, P = 0.0012). Across all conditions, \textit{stationary} and \textit{filter} feedback enhanced perceived attention (Q4) compared to \textit{silence} (P < 0.001). However, eyerofeedback increased feelings of distraction (Q3) versus \textit{silence} (P < 0.001). Participants also indicated performance was impacted by eyerofeedback (Q5) and expressed desire to use it daily (Q6) (both P < 0.001). Together, these subjective reports offer compelling evidence that eyerofeedback heightens attention and gaze focus, thereby enhancing performance.



The participants' subjective ratings help explicate the differing attentional effects of \textit{stationary} versus \textit{filter} eyerofeedback across conditions. Without distraction, \textit{stationary} feedback heightened perceived attention compared to \textit{filter}. However, with distraction, \textit{filter} elicited higher attentional ratings than \textit{stationary}. This aligns with the objective performance data showing \textit{stationary} feedback expedited response times without distraction, while \textit{filter} feedback conferred benefits with distraction. The discrepancy suggests distractions may amplify \textit{stationary} feedback effects on anxiety, whereas \textit{filter}'s conditional tactile triggers may mitigate distraction-prompted anxiety. Supporting this interpretation, participants reported significantly greater belief in (Q4) and desire to use (Q6) \textit{filter} versus \textit{stationary} feedback (both P < 0.05). By alleviating anxiety, \textit{filter} eyerofeedback may optimize attentional modulation amidst distractions. These subjective insights complement the performance data in illuminating how adaptive tactile feedback principles can enhance attention regulation.

\subsection*{Gaze Behaviors Indicate Effective Attention Regulation with Eyerofeedback}
\label{sec: R3}
We employed gaze entropy, a commonly used metric in gaze data analysis \cite{shiferaw2019review}, to quantify the participants' attentional focus. Our experiments indicate that gaze entropy was significantly reduced by eyerofeedback versus \textit{silence} in the long duration conditions (Fig. \ref{f2}(e) and Appendix Fig. S4),  both with distraction (F2,40 = 17.6546, P < 0.0001) and without distraction (F2,40 = 14.2609, P < 0.0001).  
Pairwise comparisons within the long duration setting further revealed significant differences in gaze entropy among the three feedback types in both the no distraction condition (\textit{stationary}: 0.4472 ± 0.9277 (mean ± SD) vs. \textit{silence}: -0.4894 ± 1.0595, P = 0.0063 < 0.01; \textit{filter}: -1.2252 ± 0.7742 vs. \textit{silence}, P = 0.0389 < 0.05; \textit{filter} vs. \textit{stationary}, P < 0.0001) and the distraction condition (\textit{filter}: -0.5402 ± 0.6967 vs. \textit{stationary}: 0.7007 ± 0.5295, P < 0.0001; \textit{stationary} vs. \textit{silence}: -0.3951 ± 0.8576, P < 0.0001). In particular, gaze entropy significantly decreased with \textit{filter} feedback versus \textit{silence}.  Counterintuitively, gaze entropy increased with \textit{stationary} feedback. 
%
A possible explanation is that continuous tactile cues from \textit{stationary} feedback heighten anxiety and cognitive load since tactile stimuli always happen, which is supported by the higher distraction score felt by participants in \textit{stationary} feedback compared with \textit{filter} feedback in the previous subjective questionnaire.

On the other hand, no significant effects were observed with eyerofeedback (\textit{stationary} or \textit{filter}) in the short duration setting. The differential effects of eyerofeedback observed in long versus short durations were corroborated by the 
%
interaction between feedback type and duration (F2,40 = 10.1157, P = 0.0003 < 0.001). 
The entropy heatmap in Fig.~\ref{f2}(a) provides an intuitive visualization of these findings. 
By enhancing gaze awareness and regulation, eyerofeedback reduced gaze entropy, aligning with improved response times. \textit{filter} feedback optimally focused gaze, particularly amidst distractions, highlighting the benefits of adaptive tactile feedback principles.


The findings confirm our hypothesis that tactile-bodily gaze mapping via eyerofeedback enhances attention, particularly when demands on sustained focus are high. Eyerofeedback heightens gaze awareness and guides gaze directions, ameliorating attentional lapses. Notably, \textit{filter} feedback outperformed \textit{stationary} feedback in regulating attention. By adaptively triggering tactile cues, \textit{filter} eyerofeedback may mitigate extraneous cognitive load and anxiety associated with continuous \textit{stationary} feedback. Thus, incorporating adaptive principles into tactile biofeedback maximizes benefits on attentional modulation. Refer to Fig.~\ref{f3}, which visually demonstrates concentrated gaze fixations induced by \textit{filter} versus \textit{stationary} or \textit{silence} feedback. In conclusion, rendering gaze patterns on the body optimizes self-monitoring and control of attention through heightened gaze awareness.










