\dropcap{A}\textbf{ttention} is one of the most fundamental cognitive processes, yet sustaining focus over time poses a persistent challenge \cite{corbetta2002control}. 
To mitigate this challenge, prior research explored a variety of intervention technologies such as neurofeedback \cite{marzbani2016neurofeedback} and interoceptive illusion \cite{blanke2012multisensory}. These interventions are designed to enhance users' self-awareness of their behaviors or physiological signals, encompassing vital metrics like brain activity \cite{faller2019regulation} and heart rate \cite{iodice2019interoceptive}, thereby facilitating self-regulation. Nevertheless, the acquisition of such signals necessitates specialized apparatuses which are unwieldy for day-to-day routines. In addition, the signal variations may not provide immediate insights into users' cognitive states.

%In contrast, human eye movements or gaze patterns, renowned as pivotal indicators of human attention \cite{theeuwes2009interactions}, possess the unique ability to sensitively and directly reflect an individual's mental state. Eye gaze can be easily tracked by ubiquitous devices such as webcams in working or living enviornments. Consequently, the intuitive notion arises that enhanced 
Alternatively, awareness of one's gaze patterns can serve as a biofeedback signal \cite{gregori2016assessing}, readily available through ubiquitous webcams in working or living environment. For instance, when individuals are tasked with maintaining focus on the center of a screen to complete a given objective, there is a propensity for attentional lapses, resulting in inadvertent visual exploration \cite{unsworth2016pupillary}. In such scenarios, providing individuals with real-time feedback pertaining to their gaze patterns can effectively apprise them of their unconscious ocular movements, consequently redirecting their attention towards the designated zones on the screen \cite{toreini2020using}.  
% 
However, 
%delivering users' gaze patterns to themselves poses a significant challenge. 
%as eye movements are inherently two-dimensional signals that are difficult to map directly to acoustic signals for auditory feedback. 
%xyz: cite prior works (solved)
using the conventional auditory/visual biofeedback channels to deliver gaze signals, as proposed in prior works \cite{balgera2010synchronization}, may introduce additional distractions and cognitive load for individuals.


In this study, we propose a novel approach that utilizes tactile feedback to establish a mapping between gaze patterns and the human body. 
As illustrated in Fig.~\ref{f1}, individuals receive vibration-based tactile stimuli on either their wrists or ankles, depending on the direction of their gaze. For instance, if an individual stares at the upper left of the screen, she will receive a tactile feedback on the left wrist (see Materials and Methods).

Drawing inspiration from the concepts of neurofeedback \cite{marzbani2016neurofeedback} and interoceptive illusion \cite{blanke2012multisensory}, we put forth the hypothesis that \textit{this tactile bodily gaze mapping, referred to as \textbf{eyerofeedback}, has the potential to enhance user attention, even in the presence of distractions, by fostering heightened self-awareness of eye movements}.

To test our hypothesis, we recruited participants (N = 26) to finish an attention task (Three-Choice Vigilance Task: 3CVT) \cite{meghdadi2021eeg}, where participants were required to maintain their focus on the screen center and press corresponding keys on the keyboard when different shapes appeared. Each participant underwent 12 sessions (Fig.~\ref{f1}), which encompassed a combination of 3 feedback types (\textit{silence}, \textit{stationary}, \textit{filter}) $\times$ 2 duration settings (short/long) $\times$ 2 conditions (with/without distraction). This experimental design was based on our observations and existing literature \cite{mendoza2018effect}, which suggested that individuals may exhibit varying attentional performance in the presence or absence of external distractions and under different durations of attention. 
In the distraction condition, a movie \cite{WinNTBig} was presented in the background while participants completed the attention task, potentially inducing absent-mindedness earlier than in the no-distraction condition. The different durations refer to the varying time intervals at which the shapes appeared (short: 3-5 seconds, long: 25-35 seconds). A longer duration imposes a higher level of difficulty to the attention task \cite{posner1971components}. 
%as participants had to maintain their focus on the screen for a longer duration to detect the appearance of the shapes. 
Regarding the feedback types, ``\textit{silence}'' served as the control setting, where no tactile feedback was delivered. The ``\textit{stationary}'' condition delivers tactile feedback corresponding to the participants' real-time eye movement directions, thus implementing the concept of \textbf{eyerofeedback}. The ``\textit{filter}'' condition represents a variant of eyerofeedback, in which tactile stimuli were only triggered when participants' eye movement distance exceeded a certain threshold (see Materials and Methods).


Finally, our hypothesis posits that eyerofeedback exerts a substantial influence on user attention, particularly during long-duration sessions under distraction. In such sessions, individuals are more prone to experiencing lapses in attention, making eyerofeedback a crucial factor in regulating and redirecting their focus. We anticipate that the effects of eyerofeedback on attentional control will be notably pronounced under such circumstances.



% Figure environment removed