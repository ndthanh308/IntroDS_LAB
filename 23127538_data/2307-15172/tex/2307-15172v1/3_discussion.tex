% Figure environment removed


\section*{Discussion}
% (Findings overall) 
Our investigation examines the potential efficacy of rendering users' gaze patterns on the body to enhance self-awareness of gaze behaviors and redirect attention. 
%(Explanation 1 based on interoception) 
These findings align with the theory of interoceptive inference \cite{barrett2015interoceptive}, which posits that the human brain can both estimate and regulate critical homeostatic and physiological variables \cite{iodice2019interoceptive}. Within this framework, it is plausible that the brain estimates interoceptive physiological signals that individuals often disregard when not consciously attending to them. By providing real-time feedback of gaze behaviors, the brain's estimation process of these overlooked signals could be strengthened.
%(Explanation 2 based on neurofeedback)
Another plausible explanation draws upon neurofeedback theory, which utilizes online presentation of recorded electroencephalography (EEG) in audio/visual formats to assist individuals in consciously controlling their brain waves \cite{marzbani2016neurofeedback}. Analogously, eyerofeedback presents gaze patterns through tactile feedback to help individuals consciously regulate attentive behaviors.


% (Findings/Insights 1 and reasoning)
Our observations reveal varying effects of eyerofeedback on human attention under different conditions. In easy (short-duration) tasks, where individuals could consistently maintain focus, eyerofeedback did not significantly impact attention. However, in challenging (long-duration) tasks, where lapses in attention or difficulty sustaining focus over time may occur, eyerofeedback demonstrated an ability to redirect gaze and regulate attention. This was evidenced by faster response times, fewer missed stimuli, and decreased gaze entropy within our study. Furthermore, we found eyerofeedback's influence on attention depended on the applied threshold, particularly with external distractions. Specifically, without distractions, \textit{stationary} eyerofeedback exhibited superior attention regulation over \textit{filter} eyerofeedback. However, with distractions, \textit{filter} eyerofeedback became more effective. We posit distractions may induce anxiety \cite{denkova2010impact}, mitigated through \textit{filter} eyerofeedback's adaptive tactile control. Without distractions, \textit{stationary} eyerofeedback consistently stimulated arousal and attention. But individuals may have habituated to this stimulus during the distraction-filled scenario \cite{mehrabian1977questionnaire}, diminishing its regulatory capacity. In contrast, \textit{filter} eyerofeedback selectively provided reminders, preventing habituation.

% (Findings/Insights 2 and reasoning) 
Although eyerofeedback significantly influenced human attention only during long-duration tasks, participants reported noticeable subjective effects even in short-duration settings. This indicates eyerofeedback had a substantial impact on users' subjective experiences across all conditions, despite performance improvements being primarily observed in long-duration tasks.
One explanation may be that eyerofeedback effectively delivered regulatory signals to individuals in all scenarios. However, these signals may not have triggered immediate behavioral control in the brain within short timeframes. In other words, the brain potentially received the signals without generating sufficient action to control behaviors in the short-duration tasks \cite{merker2007consciousness}.
Alternatively, eyerofeedback could have impacted attention in short-duration tasks without reaching statistical significance. This may be because performance was already high at baseline \cite{broadhurst1957emotionality} in short-duration tasks. While present, enhancements from eyerofeedback may have been insufficiently large relative to baseline to achieve statistical significance, given the initially elevated performance levels.

% (Findings/Insights 3 and reasoning.) 

Furthermore, we discovered a synchronization effect between response time and gaze entropy. Significant effects on both metrics were observed solely in long-duration tasks, not short ones. This indicates eyerofeedback may enhance attention by regulating and redirecting gaze. Additionally, significant gaze entropy differences occurred only across feedback and duration conditions, unaffected by distractions. This implies that changes in gaze patterns resulted predominantly from eyerofeedback rather than distractions. Such robust evidence bolsters claims that eyerofeedback effectively modulates and guides attention.

% (Limitation 1,2,3 and open questions) 
One limitation of this study is the relatively small sample size, which restricts generalizability of findings to broader or specific demographic populations. Larger-scale studies would promote greater representation and generalizability across diverse groups. Another limitation involves the current tactile modality for eyerofeedback. Mapping continuous gaze to vibrations on discrete body parts may lose information about gaze smoothness. Intuitively, continuous tactile feedback could better represent gaze motions. However, implementing such a system poses challenges. Full-body wearables may be cumbersome, while smaller wrist devices could impede discerning continuous tactile stimuli \cite{zigler1926tactual}. Even solving these issues, continuous feedback might distract, adversely affecting attention \cite{harman1997distress}. Future work could explore whether continuous tactile mapping elicits different attentional effects compared to discrete stimuli \cite{evans1991tactile}. Additionally, our attention task required fixed screen center focus to measure attention. However, real-world scenarios often involve more complex visual search where eye movements are integral \cite{wolfe2017five}. Moreover, fixed gazing does not guarantee sustained attention, as absent-mindedness may still occur \cite{cheyne2006absent}. Thus, future studies should investigate tactile-bodily-gaze mapping in more intricate scenarios, potentially yielding further insights.

Although our study examined eyerofeedback's influence on attention, the underlying neural activities responding to tactile-bodily-gaze mapping remain unclear. Future work could utilize EEG \cite{teplan2002fundamentals} or fMRI \cite{logothetis2008we} to capture and analyze brain activity, enabling deeper interpretation of interoceptive signal processing and gaze/attention self-regulation by the brain. Incorporating neuroimaging would illuminate the neural foundations \cite{kosslyn2001neural} underlying eyerofeedback's efficacy, providing more comprehensive insight into its attentional impact.

% (Potential Impact and Example Application Scenarios)

Our findings may inspire new technologies enhancing self-awareness and addressing interoceptive processing issues like ADHD \cite{monastra2006electroencephalographic} and fatigue \cite{windthorst2017heart}. This work significantly contributes to biofeedback literature \cite{biofeedback2008biofeedback}, which has focused on neurofeedback \cite{hammond2011neurofeedback}. As a novel form of biofeedback, eyerofeedback provides distinct advantages over traditional neurofeedback. Notably, presenting intuitive, comprehensible gaze patterns surpasses neurofeedback's abstract brain visualizations \cite{kimmig2001relationship}. However, despite different implementations, both leverage bio-signals to foster self-awareness. Consequently, eyerofeedback could apply to various neurofeedback scenarios like cognitive training \cite{jiang2017tuning}, rehabilitation \cite{giggins2013biofeedback} and anxiety release \cite{micoulaud2021eeg}. Eyerofeedback's unique benefits and broad applicability make it a promising avenue for integration into interventions and practices improving self-awareness and addressing attention challenges.



