\documentclass[fleqn,10pt]{wlscirep}
\usepackage[utf8]{inputenc}
\usepackage[T1]{fontenc}
\title{Oculomotor trajectory mapping on body as an effective intervention to enhance attention}

\author[1,*]{Songlin Xu}
\author[1]{Xinyu Zhang}
\affil[1]{University of California San Diego}


\affil[*]{soxu@ucsd.edu}

% \affil[+]{these authors contributed equally to this work}

%\keywords{Keyword1, Keyword2, Keyword3}

\begin{abstract}
Increasing individuals' awareness of their body signals can lead to improved interoception, enabling the brain to estimate current body states more accurately and promptly. However, certain body signals, such as eye movements, often go unnoticed by individuals themselves. This study aimed to test the hypothesis that providing eye-movement-correlated tactile feedback on the body enhances individuals' awareness of their attentive states, subsequently improving attention. Our results demonstrate the effectiveness of such feedback in redirecting and enhancing attention, particularly in the presence of distractions during long-duration tasks. Additionally, we observed that people's gaze behaviors changed in response to the tactile feedback, suggesting an increased self-awareness of current eye movements and attentive states. Ultimately, these changes in gaze behaviors contribute to the modulation of attentive states. Our findings highlight the potential of eye-movement-correlated bodily tactile feedback to increase individuals' self-awareness of their eye movements and attentive states. By providing real-time feedback through tactile stimuli, we can actively engage individuals in regulating their attention and enhancing their overall performance.
\end{abstract}
\begin{document}

\flushbottom
\maketitle
% * <john.hammersley@gmail.com> 2015-02-09T12:07:31.197Z:
%
%  Click the title above to edit the author information and abstract
%
\thispagestyle{empty}



\section{Introduction}

% Figure environment removed

Reinforcement Learning from Human Feedback (RLHF) has recently been used to great effect to align pretrained large language models (LLMs) to human preferences, optimizing for desirable qualities like harmlessness and helpfulness~\citep{bai2022training} and achieving state-of-the-art results across a variety of natural language tasks~\citep{openai2023gpt4}. %RLHF approaches fundamentally rely on collecting pairs of LLM outputs $(o_1, o_2)$ from a shared prompt $p$, with a human indicating which output in each pair is better on a specified attribute.
% A fundamental component of RLHF is a preference model derived from human labels, typically formatted as pairs of LLM outputs $(o_1, o_2)$ generated from a shared prompt $p$.

A standard RLHF procedure fine-tunes an initial unaligned LLM using an RL algorithm such as PPO~\citep{schulman2017proximal}, optimizing the LLM to align with human preferences. %\violet{not sure whether we need to provide this detail in the intro, especially this has nothing to do with our contribution.} % i feel like this context is useful later when e.g. explaining that context distillation is SFT
RLHF is thus critically dependent on a reward model derived from human-labeled preferences, typically \textit{pairwise preferences} on LLM outputs $(o_1, o_2)$ generated from a shared prompt $p$. % and labeled by humans. 

However, collecting human pairwise preference data, especially high-quality data, may be expensive and time consuming at scale. To address this problem, approaches have been proposed to obtain labels without human annotation, such as Reinforcement Learning from AI Feedback (RLAIF) and context distillation. 

\iffalse
raising the question of whether we can generate high-quality data for RLHF without using human labeling. %accurately-labeled preference pairs $(o_1, o_2)$
%, motivating model alignment approaches that aim to generate accurately-labeled preference pairs $(o_1, o_2)$ without human involvement. 
Two major categories of such approaches are . 
\fi

RLAIF approaches (e.g.,~\citet{bai2022constitutional}) simulate human pairwise preferences by scoring $o_1$ and $o_2$ with an LLM (Figure \ref{fig:rlcd_differences} center); the scoring LLM is often the same as the one used to generate the original pairs $(o_1, o_2)$. Of course, the resulting LLM pairwise preferences will be somewhat noisier compared to human labels. However, this problem is exacerbated by using the same prompt $p$ to generate both $o_1$ and $o_2$, causing $o_1$ and $o_2$ to often be of very similar quality and thus hard to differentiate (e.g., Table~\ref{tab:rlaif_bad_example}). Consequently, training signal can be overwhelmed by label noise, yielding lower-quality preference data. 

% While it avoids human labeling efforts, it has weakness. First, LLM preference labels will naturally be somewhat noisier compared to human labels. Furthermore, since the same prompt $p$ is used to generate both $o_1$ and $o_2$, their quality is often very similar and hard to differentiate (See Table~\ref{tab:rlaif_bad_example}). As a result, training signals can be overwhelmed by label noise, yielding lower-quality preference data. 

Meanwhile, context distillation methods (e.g., \citet{sun2023principle}) create more training signal by modifying the initial prompt $p$. 
%to create more significant training signal. 
The modified prompt $p_+$ typically contains additional context encouraging a \textit{directional attribute change} in the output $o_+$ (Figure \ref{fig:rlcd_differences} right). However, context distillation methods only generate a single output $o_+$ per prompt $p_+$, which is then used for supervised fine-tuning, losing the pairwise preferences which help RLHF-style approaches to 
%rather than using a RLHF-style preference model to 
derive signal from the contrast between outputs. 
Multiple works have observed that RL approaches using preference models for pairwise preferences can substantially improve over supervised fine-tuning by itself when aligning LLMs~\citep{ouyang2022training,dubois2023alpacafarm}. 

% conduct alignment by running supervised fine-tuning on model outputs $o_+$ generated from a modified prompt $p_+$. $p_+$ typically contains additional context encouraging desirable attributes (Figure \ref{fig:rlcd_differences} right), such as in \citet{sun2023principle}. However, multiple works have observed that RLHF-style approaches can substantially improve over supervised fine-tuning by itself when aligning LLMs~\citep{ouyang2022training,dubois2023alpacafarm}. 

Therefore, while both RLAIF and context distillation approaches have already been successfully applied in practice to align language models, we posit that it may be even more effective to combine the key advantages of both. That is, we will use RL with \textit{pairwise preferences}, while also using modified prompts to encourage \textit{directional attribute change} in outputs. %In particular, we will adapt the RLAIF data generation process with two different prompts rather than a single $p$, modifying both prompts similarly to context distillation. %\violet{this motivation is a little unexciting. I think we can more specifically discuss the potential benefits of our approach, like the benefits from RL: exploration/data generation; benefits from contrast. I don't think we get too much benefits from context distillation since we switched to the RL framework.} 

Concretely, we propose \oursfull{} (\ours{}). 
\ours{} generates preference data as follows. Rather than producing two i.i.d.\ model outputs $(o_1, o_2)$ from the same prompt $p$ as in RLAIF, \ours{} creates two variations of $p$: a \textit{positive prompt} $p_+$ similar to context distillation which encourages directional change toward a desired attribute, and a \textit{negative prompt} $p_-$ which encourages directional change \textit{against} it (Figure \ref{fig:rlcd_differences} left). We then generate model outputs $(o_+, o_-)$ respectively, and automatically label $o_+$ as preferred---that is, \ours{} automatically ``generates'' pairwise preference labels by construction. %, without further post hoc labeling.\violet{should make it clearer that our approach `generates' labels by construction} 
We then follow the standard RL pipeline of training a preference model followed by PPO. 

Compared to RLAIF-generated preference pairs $(o_1, o_2)$ from the same input prompt $p$, there is typically a clearer difference in the quality of $o_+$ and $o_-$ generated using \ours{}'s directional prompts $p_+$ and $p_-$, which may result in less label noise. %which may result in better training signal for the preference model. 
That is, intuitively, \ours{} exchanges having examples be \textit{closer to the classification boundary} for much more \textit{accurate labels} on average. Compared to standard context distillation methods, on top of leveraging pairwise preferences for RL training, \ours{} can derive signal not only from the positive prompt $p_+$ which improves output quality, but also from the negative prompt $p_-$ which degrades it. %\ours{} is not learning to imitate $o_+$, but to distill the \textit{contrast} between $o_+$ and $o_-$. 
Positive outputs $o_+$ don't need to be perfect; they only need to contrast with $o_-$ on the desired attribute while otherwise following a similar style.

% \todo{discuss our method and why intuitively it may be better.}

We evaluate the practical effectiveness of \ours{} through both human and automatic evaluations on three tasks, aiming to improve the ability of LLaMA-7B~\citep{touvron2023llama} to generate harmless outputs, helpful outputs, and high-quality story outlines. %\ours{} outperforms both RLAIF and context distillation baselines in pairwise comparisons on 
As shown in Sec. \ref{sec:experiments}, \ours{} substantially outperforms both RLAIF and context distillation baselines in pairwise comparisons when simulating preference data with LLaMA-7B, while still performing equal or better when simulating with LLaMA-30B. 
%On all three tasks, \ours{} substantially outperforms both RLAIF and context distillation baselines in pairwise comparisons---by a margin of at least 9\% and often more than 30\%---validating our method's efficacy. 
We will release all code at a later date, although in any case \ours{} is fairly easy to implement by modifying any reference RLAIF codebase. %We release all code at \todo{github link}.



\section*{Results}

We conducted a repeated-measures ANOVA to analyze the results of the aforementioned 12 sessions. To account for variability of participants' baseline performance, each metric mentioned below was normalized to have a zero mean and unit standard deviation within each participant.

\subsection*{Eyerofeedback Improves Human Attention}
\label{sec: R1}
In the attention task, participants were instructed to prioritize accuracy. Response time and the number of missed trials were used as metrics to quantify their performance. Our findings revealed that eyerofeedback significantly improved human attention compared to the control settings, particularly in more difficult (long duration) settings. 
Specifically, the ANOVA showed a significant main effect of feedback type on response time (F2,40 = 3.3135, P = 0.0466<0.05) (Fig.~\ref{f2}(b,c) and Appendix Fig.~S2). Post-hoc comparisons showed significantly faster response times with both the \textit{stationary} and \textit{filter} feedback in comparison to \textit{silence}.
%
There was also a significant interaction between feedback type and task duration (F2,40 = 3.8472, P = 0.0296<0.05), indicating larger differences between feedback conditions in the long duration trials compared to short.
In addition, we observed a 
%
significant difference on response time across feedback types during long duration with distraction (F2,40 = 3.466, P = 0.0409 < 0.05) and without distraction (F2,40 = 3.7076, P = 0.0333 < 0.05). 
Further pairwise comparisons within the long duration setting demonstrated significant differences in response time for 
%
\textit{stationary} feedback compared to \textit{silence} without external distraction (\textit{stationary}: -0.5468 ± 0.3768 (mean ± SD) vs. \textit{silence}: -0.1726 ± 0.5523, P = 0.0367 < 0.05), as well as for \textit{filter} feedback compared to \textit{silence} with external distraction (\textit{filter}: 1.1110 ± 0.4449 vs. \textit{silence}: 1.4790 ± 0.4096, P = 0.0243 < 0.05). These results indicate that eyerofeedback has the potential to accelerate human response time and enhance attention levels.

%

The ANOVA also revealed a significant main effect of feedback type on the number of missed trials (F2,40 = 4.3364, P = 0.0197<0.05). However, follow-up tests showed the effect of feedback on missed trials varied across conditions. Specifically, feedback type only impacted missed trials in the long duration trials with distraction (F2,40 = 3.6115, P = 0.0362<0.05). No significant differences between feedback types emerged for missed trials in the short duration trials or long duration without distraction (P > 0.05).
Furthermore, we did not find any significant differences in accuracy for feedback type across all four combinations of two duration $\times$ two distraction conditions (P > 0.05). This result is reasonable considering that participants were instructed to prioritize accuracy over response time during the task.



The varied effects of eyerofeedback across conditions may be attributed to differences in baseline task performance. Response times were faster and fewer trials were missed in the less demanding short duration condition compared to the long duration condition. Additionally, the presence of external distractions led to slower response times and more missed trials than no-distraction conditions. Thus, eyerofeedback conferred larger benefits when baseline attention performance was lower due to greater attentional demands from long durations or external distractions.
This interpretation is supported by the significant differences in response time between the two duration conditions 
%
(F1,20 = 118.1033, P < 0.001) and between the two distraction settings (F1,20 = 921.3792, P < 0.001). Furthermore, significant differences were observed in the number of missed trials between the two duration conditions (F1,20 = 19.0266, P = 0.0003 < 0.001) and between the two distraction settings (F1,20 = 9.967, P = 0.005 < 0.01). These findings substantiate our hypothesis that eyerofeedback holds promise to improve individuals' attention performance by reducing response time and the occurance of missed trials. Nonetheless, these benefits varied as a function of task demands, emerging to a greater extent when sustained attention was challenged by longer durations or external distractions. 


% Figure environment removed

\subsection*{Subjective Feelings Reveal Eyerofeedback Effect on Human Attention}
\label{sec: R2}


Subjective experiences were assessed via a questionnaire administered after each session (Fig.~\ref{f2}(d) and Appendix Fig.~S3).
Participants reported heightened attention to the task (Q1) and screen center (Q2) with \textit{filter} versus \textit{silence} feedback. This was evident in both the short no-distraction (Q1: F2,40 = 6.3132, P = 0.0041; Q2: F2,40 = 6.9047, P = 0.0027) and long distraction conditions (Q1: F2,40 = 5.6305, P = 0.007; Q2: F2,40 = 8.0503, P = 0.0012). Across all conditions, \textit{stationary} and \textit{filter} feedback enhanced perceived attention (Q4) compared to \textit{silence} (P < 0.001). However, eyerofeedback increased feelings of distraction (Q3) versus \textit{silence} (P < 0.001). Participants also indicated performance was impacted by eyerofeedback (Q5) and expressed desire to use it daily (Q6) (both P < 0.001). Together, these subjective reports offer compelling evidence that eyerofeedback heightens attention and gaze focus, thereby enhancing performance.



The participants' subjective ratings help explicate the differing attentional effects of \textit{stationary} versus \textit{filter} eyerofeedback across conditions. Without distraction, \textit{stationary} feedback heightened perceived attention compared to \textit{filter}. However, with distraction, \textit{filter} elicited higher attentional ratings than \textit{stationary}. This aligns with the objective performance data showing \textit{stationary} feedback expedited response times without distraction, while \textit{filter} feedback conferred benefits with distraction. The discrepancy suggests distractions may amplify \textit{stationary} feedback effects on anxiety, whereas \textit{filter}'s conditional tactile triggers may mitigate distraction-prompted anxiety. Supporting this interpretation, participants reported significantly greater belief in (Q4) and desire to use (Q6) \textit{filter} versus \textit{stationary} feedback (both P < 0.05). By alleviating anxiety, \textit{filter} eyerofeedback may optimize attentional modulation amidst distractions. These subjective insights complement the performance data in illuminating how adaptive tactile feedback principles can enhance attention regulation.

\subsection*{Gaze Behaviors Indicate Effective Attention Regulation with Eyerofeedback}
\label{sec: R3}
We employed gaze entropy, a commonly used metric in gaze data analysis \cite{shiferaw2019review}, to quantify the participants' attentional focus. Our experiments indicate that gaze entropy was significantly reduced by eyerofeedback versus \textit{silence} in the long duration conditions (Fig. \ref{f2}(e) and Appendix Fig. S4),  both with distraction (F2,40 = 17.6546, P < 0.0001) and without distraction (F2,40 = 14.2609, P < 0.0001).  
Pairwise comparisons within the long duration setting further revealed significant differences in gaze entropy among the three feedback types in both the no distraction condition (\textit{stationary}: 0.4472 ± 0.9277 (mean ± SD) vs. \textit{silence}: -0.4894 ± 1.0595, P = 0.0063 < 0.01; \textit{filter}: -1.2252 ± 0.7742 vs. \textit{silence}, P = 0.0389 < 0.05; \textit{filter} vs. \textit{stationary}, P < 0.0001) and the distraction condition (\textit{filter}: -0.5402 ± 0.6967 vs. \textit{stationary}: 0.7007 ± 0.5295, P < 0.0001; \textit{stationary} vs. \textit{silence}: -0.3951 ± 0.8576, P < 0.0001). In particular, gaze entropy significantly decreased with \textit{filter} feedback versus \textit{silence}.  Counterintuitively, gaze entropy increased with \textit{stationary} feedback. 
%
A possible explanation is that continuous tactile cues from \textit{stationary} feedback heighten anxiety and cognitive load since tactile stimuli always happen, which is supported by the higher distraction score felt by participants in \textit{stationary} feedback compared with \textit{filter} feedback in the previous subjective questionnaire.

On the other hand, no significant effects were observed with eyerofeedback (\textit{stationary} or \textit{filter}) in the short duration setting. The differential effects of eyerofeedback observed in long versus short durations were corroborated by the 
%
interaction between feedback type and duration (F2,40 = 10.1157, P = 0.0003 < 0.001). 
The entropy heatmap in Fig.~\ref{f2}(a) provides an intuitive visualization of these findings. 
By enhancing gaze awareness and regulation, eyerofeedback reduced gaze entropy, aligning with improved response times. \textit{filter} feedback optimally focused gaze, particularly amidst distractions, highlighting the benefits of adaptive tactile feedback principles.


The findings confirm our hypothesis that tactile-bodily gaze mapping via eyerofeedback enhances attention, particularly when demands on sustained focus are high. Eyerofeedback heightens gaze awareness and guides gaze directions, ameliorating attentional lapses. Notably, \textit{filter} feedback outperformed \textit{stationary} feedback in regulating attention. By adaptively triggering tactile cues, \textit{filter} eyerofeedback may mitigate extraneous cognitive load and anxiety associated with continuous \textit{stationary} feedback. Thus, incorporating adaptive principles into tactile biofeedback maximizes benefits on attentional modulation. Refer to Fig.~\ref{f3}, which visually demonstrates concentrated gaze fixations induced by \textit{filter} versus \textit{stationary} or \textit{silence} feedback. In conclusion, rendering gaze patterns on the body optimizes self-monitoring and control of attention through heightened gaze awareness.











% Figure environment removed


\section*{Discussion}
% (Findings overall) 
Our investigation examines the potential efficacy of rendering users' gaze patterns on the body to enhance self-awareness of gaze behaviors and redirect attention. 
%(Explanation 1 based on interoception) 
These findings align with the theory of interoceptive inference \cite{barrett2015interoceptive}, which posits that the human brain can both estimate and regulate critical homeostatic and physiological variables \cite{iodice2019interoceptive}. Within this framework, it is plausible that the brain estimates interoceptive physiological signals that individuals often disregard when not consciously attending to them. By providing real-time feedback of gaze behaviors, the brain's estimation process of these overlooked signals could be strengthened.
%(Explanation 2 based on neurofeedback)
Another plausible explanation draws upon neurofeedback theory, which utilizes online presentation of recorded electroencephalography (EEG) in audio/visual formats to assist individuals in consciously controlling their brain waves \cite{marzbani2016neurofeedback}. Analogously, eyerofeedback presents gaze patterns through tactile feedback to help individuals consciously regulate attentive behaviors.


% (Findings/Insights 1 and reasoning)
Our observations reveal varying effects of eyerofeedback on human attention under different conditions. In easy (short-duration) tasks, where individuals could consistently maintain focus, eyerofeedback did not significantly impact attention. However, in challenging (long-duration) tasks, where lapses in attention or difficulty sustaining focus over time may occur, eyerofeedback demonstrated an ability to redirect gaze and regulate attention. This was evidenced by faster response times, fewer missed stimuli, and decreased gaze entropy within our study. Furthermore, we found eyerofeedback's influence on attention depended on the applied threshold, particularly with external distractions. Specifically, without distractions, \textit{stationary} eyerofeedback exhibited superior attention regulation over \textit{filter} eyerofeedback. However, with distractions, \textit{filter} eyerofeedback became more effective. We posit distractions may induce anxiety \cite{denkova2010impact}, mitigated through \textit{filter} eyerofeedback's adaptive tactile control. Without distractions, \textit{stationary} eyerofeedback consistently stimulated arousal and attention. But individuals may have habituated to this stimulus during the distraction-filled scenario \cite{mehrabian1977questionnaire}, diminishing its regulatory capacity. In contrast, \textit{filter} eyerofeedback selectively provided reminders, preventing habituation.

% (Findings/Insights 2 and reasoning) 
Although eyerofeedback significantly influenced human attention only during long-duration tasks, participants reported noticeable subjective effects even in short-duration settings. This indicates eyerofeedback had a substantial impact on users' subjective experiences across all conditions, despite performance improvements being primarily observed in long-duration tasks.
One explanation may be that eyerofeedback effectively delivered regulatory signals to individuals in all scenarios. However, these signals may not have triggered immediate behavioral control in the brain within short timeframes. In other words, the brain potentially received the signals without generating sufficient action to control behaviors in the short-duration tasks \cite{merker2007consciousness}.
Alternatively, eyerofeedback could have impacted attention in short-duration tasks without reaching statistical significance. This may be because performance was already high at baseline \cite{broadhurst1957emotionality} in short-duration tasks. While present, enhancements from eyerofeedback may have been insufficiently large relative to baseline to achieve statistical significance, given the initially elevated performance levels.

% (Findings/Insights 3 and reasoning.) 

Furthermore, we discovered a synchronization effect between response time and gaze entropy. Significant effects on both metrics were observed solely in long-duration tasks, not short ones. This indicates eyerofeedback may enhance attention by regulating and redirecting gaze. Additionally, significant gaze entropy differences occurred only across feedback and duration conditions, unaffected by distractions. This implies that changes in gaze patterns resulted predominantly from eyerofeedback rather than distractions. Such robust evidence bolsters claims that eyerofeedback effectively modulates and guides attention.

% (Limitation 1,2,3 and open questions) 
One limitation of this study is the relatively small sample size, which restricts generalizability of findings to broader or specific demographic populations. Larger-scale studies would promote greater representation and generalizability across diverse groups. Another limitation involves the current tactile modality for eyerofeedback. Mapping continuous gaze to vibrations on discrete body parts may lose information about gaze smoothness. Intuitively, continuous tactile feedback could better represent gaze motions. However, implementing such a system poses challenges. Full-body wearables may be cumbersome, while smaller wrist devices could impede discerning continuous tactile stimuli \cite{zigler1926tactual}. Even solving these issues, continuous feedback might distract, adversely affecting attention \cite{harman1997distress}. Future work could explore whether continuous tactile mapping elicits different attentional effects compared to discrete stimuli \cite{evans1991tactile}. Additionally, our attention task required fixed screen center focus to measure attention. However, real-world scenarios often involve more complex visual search where eye movements are integral \cite{wolfe2017five}. Moreover, fixed gazing does not guarantee sustained attention, as absent-mindedness may still occur \cite{cheyne2006absent}. Thus, future studies should investigate tactile-bodily-gaze mapping in more intricate scenarios, potentially yielding further insights.

Although our study examined eyerofeedback's influence on attention, the underlying neural activities responding to tactile-bodily-gaze mapping remain unclear. Future work could utilize EEG \cite{teplan2002fundamentals} or fMRI \cite{logothetis2008we} to capture and analyze brain activity, enabling deeper interpretation of interoceptive signal processing and gaze/attention self-regulation by the brain. Incorporating neuroimaging would illuminate the neural foundations \cite{kosslyn2001neural} underlying eyerofeedback's efficacy, providing more comprehensive insight into its attentional impact.

% (Potential Impact and Example Application Scenarios)

Our findings may inspire new technologies enhancing self-awareness and addressing interoceptive processing issues like ADHD \cite{monastra2006electroencephalographic} and fatigue \cite{windthorst2017heart}. This work significantly contributes to biofeedback literature \cite{biofeedback2008biofeedback}, which has focused on neurofeedback \cite{hammond2011neurofeedback}. As a novel form of biofeedback, eyerofeedback provides distinct advantages over traditional neurofeedback. Notably, presenting intuitive, comprehensible gaze patterns surpasses neurofeedback's abstract brain visualizations \cite{kimmig2001relationship}. However, despite different implementations, both leverage bio-signals to foster self-awareness. Consequently, eyerofeedback could apply to various neurofeedback scenarios like cognitive training \cite{jiang2017tuning}, rehabilitation \cite{giggins2013biofeedback} and anxiety release \cite{micoulaud2021eeg}. Eyerofeedback's unique benefits and broad applicability make it a promising avenue for integration into interventions and practices improving self-awareness and addressing attention challenges.





\matmethods{

\subsection*{Participants}
We recruited 26 adult participants from a local university (age: 23.5 ± 2.3 years; 6 females). All were right-handed with normal hearing and vision (with eyeglasses). Each participant received 10 USD compensation and was asked to wear headphones during the study. Three participants were excluded for not wearing headphones, and two were excluded due to incomplete data from technical issues, leaving 21 participants for analysis.
We used a within-subject design where each participant completed 12 randomized sessions combining 3 feedback types (\textit{silence}, \textit{stationary}, \textit{filter}) x 2 durations (short, long) x 2 conditions (with/without distraction). Session order randomization ensured variability. The University of California San Diego Institutional Review Board approved the study, and we obtained written informed consent from all participants beforehand.

\subsection*{Attention Task}
We employed a modified three-choice vigilance task (3CVT) to measure attention \cite{meghdadi2021eeg}. Participants were presented with three geometrical shapes: a target upward triangle, non-target downward triangle, and diamond distractor. Each of the ten trials per session displayed one shape followed by a random time interval. Shapes followed a 4:3:3 (target:non-target:distractor) ratio, with order shuffled to prevent bias.
We made two key modifications to the original 3CVT \cite{meghdadi2021eeg}. First, we introduced short (2-5 s) and long (25-35 s) intervals between shapes, manipulating difficulty. Longer durations required extended focus, thus increasing difficulty. Second, rather than random locations, shapes appeared at the center, enabling direct measurement of continuous attention and response times. Gaze deviations from center indicated lapsed attention, thus increasing response times. Participants pressed left/right arrows for target/non-target shapes, allowing the response time measurement.

\subsection*{Tactile Bodily Gaze Map}
The eyerofeedback system delivered vibratory stimuli to users' wrists and ankles based on eye movement directions. We divided the screen into four areas - Upper Left, Upper Right, Lower Left, Lower Right (Fig.~\ref{f1}) - establishing the following mapping: Upper Left $\rightarrow$ Left Wrist, Upper Right $\rightarrow$ Right Wrist, Lower Left $\rightarrow$ Left Ankle, Lower Right $\rightarrow$ Right Ankle. As users shifted gaze across areas, corresponding vibrations were triggered on their body. For instance, gazing upper left induced left wrist vibration, allowing users to sense their eye movements and regulate attention accordingly.

Vibratory tactile stimuli were delivered via 3D printed black wristbands housing vibration motors (Fig. S1 in Appendix). An Arduino Uno \cite{WinNT} controlled the 1 Hz vibration motors from PC commands. Adjustable wristbands using Velcro ensured comfortable wearing on both wrists and ankles.

Eye gaze data was collected using WebGazer \cite{papoutsaki2016webgazer}. Participants underwent pre-study calibration by clicking dots at various screen locations. Notably, we did not capture or save any face images during data collection. The gaze data contained timestamps for synchronization and (x,y) coordinates of eye movements on-screen.

\subsection*{Feedback Conditions}

We designed three feedback types: \textit{silence}, \textit{stationary}, and \textit{filter}. The \textit{silence} feedback served as the control condition with no tactile feedback. In the \textit{stationary} feedback condition, tactile feedback (eyerofeedback) was consistently provided to the users' entire body based on their real-time eye movements. The \textit{filter} feedback was a modified version of eyerofeedback, which was only triggered when users' eye movement distance exceeded a certain threshold. This decision was based on preliminary observations indicating that the \textit{stationary} feedback might introduce additional distractions to users, as tactile stimuli occurred continuously with any eye movement \cite{horvath2010distraction}. Consequently, the \textit{filter} feedback aimed to mitigate this potential distraction and warranted further investigation.
Specifically, we designated a centered sub-area spanning half the screen width/length (Fig.~\ref{f1}). Only eye movements beyond this sub-area's boundaries triggered tactile stimuli, allowing perception of gaze-pattern eyerofeedback.


\subsection*{Experiment Apparatus and Instruction}

The experimenter began by introducing the basic procedures of the study and addressing any questions or concerns participants had, ensuring their full comprehension of the information provided in the consent form. Subsequently, the experimenter assisted participants in wearing the four wristbands on their wrists and ankles. To minimize any additional pressure, the experimenter did not observe the participants or the screen during the formal study. However, if participants encountered any issues or had questions during the study, the experimenter was available to provide assistance.

Participants were informed that they would be completing an attention task and would receive tactile stimuli on their wrists and ankles, corresponding to their eye movements. They were instructed that the only way to reduce or eliminate the feedback was to maintain focus on the center of the screen. Prior to the formal study, participants were given an explanation of the attention task and provided with an opportunity to practice, allowing them to become familiar with the task.

After each session, participants were instructed to take a minimum of 1-minute rest and complete a questionnaire (Fig.~\ref{f2}) before proceeding to the next session. If participants felt fatigued from the previous study session, they were encouraged to take a longer rest period. Additionally, participants were required to perform a new eye tracking calibration after each session to minimize potential drift in WebGazer.

It is important to note that participants were instructed to wear the wristbands containing the vibration motors throughout the entire study, even during sessions when the feedback type was set to \textit{silence}. Finally, participants were instructed to prioritize accuracy in the attention task, ensuring they pressed the correct key on the keyboard when a shape appeared, and then attempting to respond as quickly as possible.


\subsection*{Data and Code Availability}

All data and codes needed to reproduce the findings and analysis are available at github: https://github.com/songlinxu/Eyerofeedback.
}

\showmatmethods{}



\bibliography{reference}




\end{document}