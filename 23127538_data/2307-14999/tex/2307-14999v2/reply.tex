\documentclass[10pt,a4paper]{article}
\usepackage{fullpage}
\usepackage[utf8]{inputenc}
\usepackage[english]{babel}
\usepackage{amsmath}
\usepackage{amsfonts}
\usepackage{amssymb}
\usepackage{graphicx}
\usepackage[shortlabels]{enumitem}

\title{\textbf{Reply to Referee Report} \\ on JHEP\_056P\_0823}
\author{Oscar Braun-White, Nigel Glover, Christian T Preuss}

\begin{document}
\maketitle

We are grateful to the referee for their review of our manuscript. While we believe our new subtraction framework to have a positive impact on the efficiency of numerical calculations at NNLO, we wish to emphasise that the improvements we primarily anticipate in the present paper are related to the complexity of the subtraction terms.

\par\medskip
To be specific, the antenna-subtraction scheme currently relies on antenna functions calculated from physical matrix elements. As such, these “OLD” antenna functions are ideal candidates for NNLO subtraction terms in two-parton processes. In previous works, they have been used for more complicated processes like $e^+e^-$ to three jets or di-jet production at hadron colliders. In these cases, the “OLD” antenna functions do not faithfully reflect the singularity structure of the matrix elements anymore and additional terms had to be introduced to remove spurious singularities in the subtraction term. The complexity to construct NNLO subtraction terms therefore increases with the multiplicity of the process, impeding applications of the antenna-subtraction scheme to more complicated processes in the future.

\par\medskip
The present paper intends to provide a new approach to the antenna-subtraction method, overcoming the hurdles faced with the “OLD” antenna functions. With the construction algorithm described in this and our companion paper, we aim at designing antenna functions in such a way that they facilitate a straightforward construction of NNLO subtraction terms without introducing spurious singularities. In particular, this will, in the future, allow for an automation of the antenna-subtraction method for higher multiplicity processes.

\par\medskip
We believe the subtraction scheme based on the new antenna functions to be more efficient in numerical calculations, because it is designed in such a way that additional terms for the subtraction of spurious singularities are avoided. This means that a smaller number of, potentially expensive, matrix elements has to be evaluated. The number of spurious singularities, and therefore additional matrix-element evaluations, increases with the parton multiplicity of the process. As such, a simple process like $e^+e^-$ to two jets will see no improvement, as the structure of the subtraction terms has not changed. The impact on processes like $e^+e^-$ to three jets will likely also be small.

\par\medskip
We have added two paragraphs to the manuscript pertaining to the above analysis. In the introduction, we now clarify for which class of processes we anticipate the method to provide significant improvements. In the conclusions, we now explicitly state $e^+e^-$ to four jets as a relevant future application of our new subtraction scheme.

\par\medskip
We hope that the referee agrees with our amendments and suggests the manuscript for publication in JHEP.

\end{document}
