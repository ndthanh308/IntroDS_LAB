\section{Values of mode parameters}
\label{app:mode_params}
The motional-mode frequencies and Lamb-Dicke parameters used in the simulations of Sec.~\ref{sec:results} are given in Table \ref{tab:valuesN3}. The values are obtained from a mode-structure model of equidistantly spaced ions trapped in an HOA2.0 trap~\cite{HOA}. 

\begin{table*}[ht]
\renewcommand*{\arraystretch}{1.5}
\begin{tabular}{ c | c | c | c }
\hline
& $p = 0$ & $p = 1$ & $p = 2$\\
\hline
\begin{tabular}{@{}c@{}}$\omega_p / 2\pi$ \\ (MHz)\end{tabular} & 
2.9574 & 3.0542 & 3.1222 \\
\hline
$\eta_{j=0,p}$ & $-0.0457$ & $0.0776$ & $0.0625$\\
 $\eta_{j=1,p}$ & $0.0909$ & $-2.77 \times 10^{-6}$ & $0.0629$ \\
$\eta_{j=2,p}$ & $-0.0457$ & $-0.0776$ & $0.0625$ \\
\hline
\end{tabular}
\caption{Values of mode parameters for $N = N' = 3$.}\label{tab:valuesN3}
\end{table*}

\section{Magnus Term Derivations}
\label{app:derivations}

In this section, we derive the Magnus terms and analyze their scaling trends with respect to parameters such as the average Rabi frequency $\bar{A}$, pulse length $\tau$, and detuning $\delta_p$ of the mode frequency. We start with (\ref{eq:HI}), the interaction Hamiltonian when the $j$-th ion is illuminated by a laser of pulse $g_j(t)$. As we consider the case where only one ion is illuminated at a time, we omit index $j$ for brevity. 

In order to analyze high-order terms and their scaling trends with respect to the parameters of our interest, we keep up to the second-order Taylor expansion term in expanding the exponential within (\ref{eq:HI})
\begin{align}
    \exp{i\sum_p \eta_{p} (\hat{a}_p e^{-i\omega_p t}+\hat{a}_p^{\dagger} e^{i\omega_p t})} \approx \hat{X}_0+\hat{X}_1+\hat{X}_2,
\end{align}
where $\eta_{p} := \eta_{j,p}$, and
\begin{align}
    &\hat{X}_0 = \hat{I},\\
    &\hat{X}_1(t) = i\sum_p \eta_{p}(\hat{a}_p e^{-i\omega_p t}+\hat{a}_p^{\dagger} e^{i\omega_p t}),\\
    &\hat{X}_2(t) = -\frac{1}{2}\left[
    \sum_p \eta_{p} (\hat{a}_p e^{-i\omega_p t}+\hat{a}_p^{\dagger} e^{i\omega_p t})
    \right]^2
\end{align}
are the zeroth-, first-, and second-order Taylor expansion terms in $\eta_{p}$. Here, $\hat{a}_p$ ($\hat{a}^{\dagger}_p$) is the annihilation (creation) operator on mode $p$.

We then write out the approximated full unitary corresponding to the Hamiltonian (\ref{eq:HI}) through Magnus expansion, keeping up to the second-order term 
\begin{align}
\label{eq:ap_uni}
\hat{U} \approx \exp{\hat{\Omega}_1+\hat{\Omega}_2}, \nonumber
\end{align}
where
\begin{align}
    \hat{\Omega}_1 &= -i\int_0^\tau dt \hat{H}_I(t) 
    \approx -i \hat{\sigma}^+ \int_0^{\tau}g(t) \left(1+\hat{X}_1(t)+\hat{X}_2(t) \right)dt - h.c. \\
    \hat{\Omega}_2 &= -\frac{1}{2}\int_0^\tau dt_1 \int_{0}^{t_1} dt_2 \left[ \hat{H}_I(t_1), \: \hat{H}_I(t_2) \right] \nonumber \\
    &\approx -\frac{1}{2}\int_0^\tau dt_1 \int_{0}^{t_1} dt_2 \left[ \hat{\sigma}^+ g(t_1)(1+\hat{X}_1(t_1)+\hat{X}_2(t_1))+h.c., \quad \hat{\sigma}^+ g(t_2)(1+\hat{X}_1(t_2)+\hat{X}_2(t_2))+h.c. \right]
\end{align}
are the first- and second-order Magnus terms. As a reminder, $\hat{\sigma}^+$ ($\hat{\sigma}^-$) is the raising (lowering) operator on the qubit. 

Here we assume the pulse to be in the form $g(t)=\sum_n A_n e^{-i\omega_n t}$ over a duration of pulse length $\tau$, where $\omega_n=\frac{2\pi n}{\tau}$ is the frequency of the pulse component with Rabi frequency $A_n$. We consider pulses with low power, i.e., $|A_n| \ll \omega_p, \omega_n$, such that perturbative equations apply. We are interested in the scaling trends of the qubit population and its error with respect to parameters such as $\alpha$, $\tau$, and $\delta_p$. Thus, we analyze the contributions to the qubit population from the Taylor expansion terms up to second order ($\hat{X}_0$, $\hat{X}_1$, and $\hat{X}_2$). We define
\begin{align}
    \hat{\Omega}_1^{(r)} &= -i \hat{\sigma}^+ \int_0^\tau g(t) \hat{X}_r(t) dt - h.c.,\\
    \hat{\Omega}_2^{(r,s)} &= -\frac{1}{2} \int_0^\tau dt_1 \int_{0}^{t_1} dt_2
    \left[ \hat{\sigma}^+ g(t_1) \hat{X}_r(t_1) + h.c., \:\: 
    \hat{\sigma}^+ g(t_2) \hat{X}_s(t_2) + h.c., \right],
\end{align}
such that $\hat{\Omega}_1 \approx \sum_{r=0}^2 \hat{\Omega}_1^{(r)}$ and $\hat{\Omega}_2 \approx \sum_{r=0}^2 \sum_{s=0}^2 \hat{\Omega}_2^{(r,s)}$.

\subsection{First-order Magnus terms}
\label{sec:appendix_first_order_magnus}
Here we evaluate the first-order Magnus term $\hat{\Omega}_1$. We consider pulses with low power ($|A_n| \ll \omega_p, \omega_n$) that are near-resonant to the blue-sideband frequencies; that is, $|A_n|$ is non-negligible only for $n$ such that $|\omega_p - \omega_n| \ll \omega_p$ for some $p$. Then, we obtain 
\begin{align}
    \int_0^\tau g(t)dt \approx 0 \quad &\Rightarrow \quad \hat{\Omega}_1^{(0)} \approx 0, \nonumber\\
    \int_0^\tau g(t) \hat{X}_2(t)dt \approx 0 \quad &\Rightarrow \quad \hat{\Omega}_1^{(2)} \approx 0, \nonumber
\end{align}
by the rotating wave approximation (RWA). Thus, $\hat{\Omega}_1 \approx \hat{\Omega}_1^{(1)}$, where (taking into account that small detuning $\omega_p \rightarrow \omega_p + \delta_p$ may occur to the mode frequency) $\hat{\Omega}_1^{(1)}$ is given by
\begin{align} \label{eq:Omega_1_1}
    \hat{\Omega}_1^{(1)} &= -i \hat{\sigma}^+ \int_0^\tau g(t) \hat{X}_1(t)dt - h.c. \nonumber \\
    &= \hat{\sigma}^+ \sum_p \eta_p \sum_n  A_n \int_0^\tau dt \left(\hat{a}_p e^{-i (\omega_p + \omega_n + \delta_p) t} + \hat{a}_p^\dagger e^{i (\omega_p - \omega_n + \delta_p) t} \right) - h.c. \nonumber \\
    &\approx \hat{\sigma}^+ \sum_p \eta_p \hat{a}_p^\dagger
    \sum_n  A_n \int_0^\tau dt e^{i (\omega_p - \omega_n + \delta_p) t} - h.c. \nonumber \\
    &= \sum_{p} \eta_p \Theta_{p}^{(1)} \hat{\sigma}^+ \hat{a}_p^{\dagger} - h.c.
\end{align}
Here we use the RWA from $|\eta_p A_n| \ll |\omega_p + \omega_n + \delta_p|$ and define $\Theta_p^{(1)} := \sum_n \Theta^{(1)}_{p,n}$, where
\begin{equation} \label{eq:theta_pn_1}
    \Theta^{(1)}_{p,n} := A_n \int_0^\tau dt e^{i(\omega_p - \omega_n + \delta_p) t} 
    = 2 A_n e^{i \omega_{p,n} \tau / 2} \frac{\sin(\omega_{p,n} \tau/2)}{\omega_{p,n}},
\end{equation}
and $\omega_{p,n}:=\omega_p-\omega_n+\delta_p$.

We discuss how $|\Theta_p^{(1)}|$ scales with various parameters. First, we consider the average Rabi frequency $\bar{A} := \sqrt{\sum_n |A_n|^2}$. From (\ref{eq:theta_pn_1}), $\alpha := |\Theta_{p^*}^{(1)}|$ is proportional to the average Rabi frequency $\bar{A}$. Indeed, when obtaining the pulse solution, $\alpha$ is the \textit{Rabi-frequency scaling factor} that is multiplied uniformly to the components $A_n$ of the pulse solution that is normalized such that $|\Theta_{p^*}^{(1)}|=1$; see (\ref{eq:normalizepulse}). For $p' \neq p^*$, $|\Theta_{p'}^{(1)}|$ is proportional to $\bar{A}$ as well.

Next, we consider the pulse length $\tau$. For the case of perfectly resonant square pulse $g(t) = \bar{A} \exp\{-i\omega_{p^*}t + i\phi\}$, we may take the limit $\omega_{p^*,n}\tau \rightarrow 0$ in (\ref{eq:theta_pn_1}) and obtain $|\Theta_{p^*}^{(1)}| =  \bar{A}\tau \propto \tau$. For shaped pulses, taking the limit $\omega_{p^*, n} \tau \rightarrow 0$ is not applicable for all $n$ with non-negligible $|A_n|$; however, we expect $|\Theta_{p^*}^{(1)}| \approx \bar{A} \tau$ for shaped pulses with similar spectrum of $|A_n|$ to square pulse (see moment-0 pulse of Fig.~\ref{fig:pulse_prof} for example). For $p' \neq p^*$, if we take the same limit $\omega_{p^*,n}\tau \rightarrow 0$ in (\ref{eq:theta_pn_1}), then $\omega_{p',n} \rightarrow \omega_{p'} - \omega_{p^*}$, so $|\Theta_{p'}^{(1)}|$ oscillates with angular frequency $(\omega_{p'} - \omega_{p^*})/2$ as $\tau$ is increased (unless $|\Theta_{p'}^{(1)}|$ is actively suppressed to zero by pulse shaping). This is shown by the dips in Fig.~\ref{fig:2}(b) for square pulses.

Finally, we consider the detuning $\delta_p$. Assuming small detuning $|\delta_p| \ll 1/\tau$, Taylor expansion of (\ref{eq:theta_pn_1}) with respect to $\delta_p$ includes non-zero terms for all orders of $\delta_p$. Thus, for a pulse with $K$-th order stabilization, the leading-order contribution of $\delta_p$ to $|\Theta_p^{(1)}|$ is proportional to $\delta_p^{K+1}$, where $K=0$ applies to square pulse or shaped pulse with no stabilization. 

\subsection{Second-order Magnus terms}
\label{sec:appendix_second_order_magnus}
Here we evaluate the second-order Magnus term $\hat{\Omega}_2$. Similarly to the case of $\hat{\Omega}_1$, we find that $\hat{\Omega}_2^{(r,s)}$ is negligibly small for $(r,s) = (0,0)$, $(0,1)$, $(1,0)$, $(0,2)$, and $(2,0)$ by the RWA. Thus, up to leading order, $\hat{\Omega}_2 \approx \hat{\Omega}_2^{(1,1)}$, where, in the presence of detuning $\omega_p \rightarrow \omega_p + \delta_p$,
\begin{align}
    \hat{\Omega}_2^{(1,1)}&= -\frac{1}{2} \int_0^\tau dt_1\int_0^{t_1}dt_2 
    \Big[ i \hat{\sigma}^+ g(t_1) \sum_p \eta_p (\hat{a}_p e^{-i (\omega_p + \delta_p) t_1} + \hat{a}_p^\dagger e^{i (\omega_p + \delta_p) t_1}) + h.c., \nonumber \\
    &\quad\quad\quad\quad\quad\quad\quad\quad\quad\quad
    i \hat{\sigma}^+ g(t_2) \sum_{p'} \eta_{p'} (\hat{a}_{p'} e^{-i (\omega_{p'} + \delta_{p'}) t_2} + \hat{a}_{p'}^\dagger e^{i (\omega_{p'} + \delta_{p'}) t_2}) + h.c. \Big] \nonumber\\
    &= \frac{1}{2} \sum_{p,p'} \eta_{p} \eta_{p'} \sum_{n,n'} A_n A_{n'}^*
    \int_0^\tau dt_1 \int_0^{t_1} dt_2 e^{i\omega_{p,n}t_1} e^{-i\omega_{p',n'}t_2} (\hat{\sigma}^z \hat{a}_p^{\dagger}\hat{a}_{p'}+\hat{\sigma}^-\hat{\sigma}^+ \delta_{p,p'}) - h.c. \nonumber\\
    &=\sum_{p,p'}\eta_{p}\eta_{p'} \Theta_{p,p'}^{(2)} (\hat{\sigma}^z \hat{a}_p^{\dagger}\hat{a}_{p'}+\hat{\sigma}^-\hat{\sigma}^+ \delta_{p,p'}) - h.c. 
    \label{eq:sec_order_term}
\end{align}
Here, $\hat{\sigma}^z$ is the Pauli-Z operator on the qubit, $\delta_{p,p'}$ is the Kronecker delta symbol, and $\Theta_{p,p'}^{(2)} := \sum_{n,n'} \Theta_{p,p',n,n'}^{(2)}$ where
\begin{align} \label{eq:theta_ppnn_2}
    \Theta_{p,p',n,n'}^{(2)} &:=  A_n A_{n'}^*
    \int_0^\tau dt_1 \int_0^{t_1} dt_2 e^{i\omega_{p,n}t_1} e^{-i\omega_{p',n'}t_2} \nonumber \\
    &= \frac{A_n A_{n'}^*}{2 \omega_{p,n}\omega_{p',n'}(\omega_{p,n}-\omega_{p',n'})} \left(\omega_{p',n'}-\omega_{p,n}e^{i(\omega_{p,n}-\omega_{p',n'})\tau}+(\omega_{p,n}-\omega_{p',n'})e^{i\omega_{p,n}\tau} \right). 
\end{align}

With all other parameters fixed, $|\Theta_{p,p'}^{(2)}|$ is proportional to $\bar{A}^2$ as $\Theta_{p,p',n,n'}^{(2)} \propto A_n A^*_{n'}$. This $\Theta_{p,p'}^{(2)}$ [$(p,p')\neq(p^*,p^*)$] is the second-order contribution to the CMC error for square pulses. Note that the first-order contribution $\Theta_{p'}^{(1)}$ ($p'\neq p^*$) is proportional to $\bar{A}$. This explains the dip in the population-error curve for square pulses in Fig.~\ref{fig:2}(a). As $\alpha \propto \bar{A}$ is increased from a small value (with fixed $\tau$), $|\Theta_{p,p'}^{(2)}|$ is initially smaller than $|\Theta_{p'}^{(1)}|$ but grows faster. At the point where the magnitude of $|\Theta_{p,p'}^{(2)}|$ roughly matches $|\Theta_{p'}^{(1)}|$, the effects of first- and second-order CMC to the population error may cancel each other out, which, as we pointed out in the main text, is a likely cause of the dip for square pulses in Fig.~\ref{fig:2}(a). On the other hand, for shaped pulses, $\Theta_{p,p'}^{(2)}$ is the leading-order contribution to the CMC error, as $|\Theta_{p'}^{(1)}|$ is suppressed to zero for all $p' \neq p^*$.

The scaling behavior of $|\Theta_{p,p'}^{(2)}|$ with respect to $\tau$ is rather complicated. We find three categories of mode pairs $(p,p')$ such that the pair(s) in the same category exhibit similar scaling behavior: (i) $(p=p^*, p'\neq p^*)$ or $(p \neq p^*, p'=p^*)$ or $(p, p'=p\neq p^*)$, (ii) $(p=p^*, p'=p^*)$, and (iii) $(p\neq p^*, p'\neq p^*)$ where $p \neq p'$. 

First, consider the case where $p=p^*$ and $p' \neq p^*$. Similarly to the first-order Magnus integral above, we may take the limit $\omega_{p^*,n}\tau \rightarrow 0$, which is valid for square pulse and shaped pulses of similar spectrum to square pulse. This leads to
\begin{equation} \label{eq:theta_ppnn_2_lim}
   \Theta_{p^*,p',n,n'}^{(2)} \rightarrow i A_n A^*_{n'} \left( 
    \frac{\tau}{2 \omega_{p',n'}} - e^{-i \omega_{p',n'}\tau/2} 
    \frac{\sin(\omega_{p',n'}\tau/2)}{\omega_{p',n'}^2}
   \right).
\end{equation}
It is reasonable to consider pulse lengths that are much longer than the inverse of the \textit{spacing} between the mode frequencies (albeit much shorter than the inverse of the mode-frequency value). In such case, for the types of pulses considered here, $\tau \gg |\omega_{p',n'}|^{-1}$ for all $n'$ such that $|A_{n'}|$ is non-negligible. Then, the first term, which is proportional to $\tau$, dominates the second term. Thus, for the types of pulses and the regime of pulse lengths considered here, $|\Theta_{p^*,p'}^{(2)}|$ is roughly proportional to $\tau$. Similar calculations show that $|\Theta_{p,p'}^{(2)}|$ is roughly proportional to $\tau$ when $p \neq p^*$, $p' = p^*$ and $p = p' \neq p^*$ as well. 

Second, consider $p=p'=p^*$. Then, from (\ref{eq:theta_ppnn_2_lim}), we take the limit $\omega_{p',n'}\tau \rightarrow 0$ again and obtain 
\begin{equation*}
   \Theta_{p^*,p^*,n,n'}^{(2)} \rightarrow - e^{-i \omega_{p',n'}\tau/4} A_n A^*_{n'} \frac{\tau^2}{4}.
\end{equation*}
Thus, $|\Theta_{p^*,p^*}^{(2)}|$ is roughly proportional to $\tau^2$.

Third, consider $p \neq p^*$ and $p' \neq p^*$ where $p \neq p'$. In such case, for all $n$ and $n'$ such that $|A_n|$ and $|A_{n'}|$ are both non-negligible, $\omega_{p,n}$, $\omega_{p',n'}$, and $\omega_{p,n} - \omega_{p',n'}$ in (\ref{eq:theta_ppnn_2}) are all nonzero. Thus, as $\tau$ is increased, we expect from (\ref{eq:theta_ppnn_2}) that $|\Theta_{p,p'}^{(2)}|$ oscillates with a roughly constant amplitude ($\propto \tau^0$). 

Figure~\ref{fig:sec_mag} plots the values of $|\eta_p \eta_{p'}\Theta_{p,p'}^{(2)}|$ of the moment-0 shaped pulses targeting mode $p^*=2$ for various pulse lengths. We emphasize that $\alpha$ is fixed to 1 in these plots. As $\alpha \approx \bar{A} \tau$ for moment-0 shaped pulses (see Appendix~\ref{sec:appendix_first_order_magnus}), the proportionality to $\tau^\beta$ when $\bar{A}$ is fixed shows up as the proportionality to $\tau^{\beta-2}$ in these plots, where the additional proportionality to $\tau^{-2}$ comes from $|\Theta_{p,p'}^{(2)}| \propto \bar{A}^2$. As expected from the above discussion, with fixed $\alpha$, $|\Theta_{p,p'}^{(2)}|$ is roughly proportional to $\tau^{\beta-2}$ where the closest integer to $\beta$ is 2 for $(p,p') = (2,2)$, 0 for $(p,p') = (0,1)$ and $(1,0)$, and 1 for all other $(p,p')$. Excluding $\Theta^{(2)}_{2,2}$ as it does not contribute to the CMC error, the largest second-order contribution to the CMC error comes from $\Theta^{(2)}_{2,1}$, whose magnitude is roughly proportional to $\tau^{\beta-2} = \tau^{-0.841}$. As explained above for the case where $p=p^*$ and $p' \neq p^*$, the value of $\beta$ is reasonably close to 1. 

% Figure environment removed

Finally, we consider the scaling trend with respect to $\delta_p$. Similarly to $\Theta_p^{(1)}$, in the presence of detuning $\delta_p$, there are contributions to $\Theta_{p,p'}^{(2)}$ from all orders of $\delta_p$. For the pulse-shaping method in the main text, only  $\Theta_p^{(1)}$ is stabilized with respect to $\delta_p$ and not $\Theta_{p,p'}^{(2)}$. Thus, the leading-order contribution of $\delta_p$ to $|\Theta_{p,p'}^{(2)}|$ is first order ($\propto \delta_p$). 


\subsection{Mangus terms and the qubit poulation}
\label{sec:appendix_population}

In this subsection, we explain how the leading-order Magnus terms affect the qubit population. Combined with the previous two subsections, this completes the discussion on how the qubit population and its error scale with respect to parameters such as $\alpha$, $\tau$, and $\delta_p$. 

First, we consider the first-order Magnus term in (\ref{eq:Omega_1_1}). To leading order in $\eta_p$, we may write $\exp(\hat{\Omega}_1) \approx \prod_p \exp(\hat{\Omega}_{1,p})$, where
\begin{equation}
    \hat{\Omega}_{1,p} = \eta_p \Theta_p^{(1)} \hat{\sigma}^+ \hat{a}_p^\dagger - h.c.
\end{equation}
Thus, the first-order contribution of mode $p$ to the qubit population is induced by the unitary $\exp(\hat{\Omega}_{1,p})$, followed by projection to the qubit subspace. In this paper, we set $|0,0\rangle$ as the initial state, where $|a,b\rangle$ is the composite state of the qubit state ($a=0,1$) and the motional Fock state ($b=0,1,2,...$). In such case, $\hat{\Omega}_{1,p}$ induces transition between only two states $|\tilde{0}\rangle := |0,0\rangle$ and $|\tilde{1}\rangle := |1,1\rangle$. Specifically, $\hat{\Omega}_{1,p} = \eta_p \Theta_p^{(1)} |\tilde{1} \rangle \langle \tilde{0}| - h.c. = i \eta_p |\Theta_p^{(1)}| (\sin \varphi \hat{\sigma}^{\tilde{x}} - \cos \varphi \hat{\sigma}^{\tilde{y}})$, where $\Theta_p^{(1)} = |\Theta_p^{(1)}| e^{i \varphi}$ and $\hat{\sigma}^{\tilde{x}}$ ($\hat{\sigma}^{\tilde{y}}$) is the Pauli-X (Y) operator of the qubit spanned by $|\tilde{0}\rangle$ and $|\tilde{1}\rangle$. Then, straightforward calculations show that
\begin{equation}
    \langle \tilde{1} | \exp(\hat{\Omega}_{1,p}) |\tilde{0}\rangle 
    = \langle \tilde{1} | \exp\left( i \eta_p |\Theta_p^{(1)}| (\sin \varphi \hat{\sigma}^{\tilde{x}} - \cos \varphi \hat{\sigma}^{\tilde{y}}) \right) |\tilde{0}\rangle 
    =  e^{i \varphi}
     \sin \left( \eta_p |\Theta_p^{(1)}| \right).
\end{equation}
In the perturbative regime $\eta_p |\Theta_p^{(1)}| \ll 1$ $\forall p$, when considering the contribution of mode $p$ to the qubit population, the qubit-state transition due to other modes can be ignored. Thus, we conclude that the magnitude of first-order contribution of mode $p$ to the qubit's bright-state ($|1\rangle$) population is approximately given by $\sin^2 ( |\eta_p \Theta_p^{(1)}| ) \approx ( |\eta_p \Theta_p^{(1)}| )^2$. The largest contribution comes from the target mode $p^*$, which gives the approximate bright-state population
\begin{align}
    \mathcal{P} \approx \sin^2 (\eta_{p^*} |\Theta_{p^*}^{(1)}|).
    \label{eq:pop}
\end{align}

Next, we consider the second-order Magnus term in (\ref{eq:sec_order_term}).
Similarly to above, we define
\begin{equation}
    \hat{\Omega}_{2,p,p'} = \eta_p \eta_{p'} \Theta^{(2)}_{p,p'} (\hat{\sigma}^z \hat{a}_p^\dagger \hat{a}_{p'} + \hat{\sigma}^- \hat{\sigma}^+ \delta_{p,p'}) - h.c.,
\end{equation}
and as a concrete example, we analyze the contribution of 
\begin{equation}
    \hat{\Omega}_{2,p^*,p^*}
    = 2i \eta_{p^*}^2 \Im[\Theta^{(2)}_{p^*,p^*}] (\hat{\sigma}^z \hat{a}_{p^*}^\dagger \hat{a}_{p^*} + \hat{\sigma}^- \hat{\sigma}^+)
\end{equation}
to the qubit population, which adds perturbatively to the contribution of $\hat{\Omega}_{1,p^*}$. An important observation is that in the subspace spanned by $|\tilde{0}\rangle$ and $|\tilde{1}\rangle$, $\hat{\sigma}^z \hat{a}_{p^*}^\dagger \hat{a}_{p^*} + \hat{\sigma}^- \hat{\sigma}^+ = |\tilde{0}\rangle \langle \tilde{0}| - |\tilde{1}\rangle \langle \tilde{1}| = \hat{\sigma}^{\tilde{z}}$. Again by straightforward calculations
\begin{align}
    \langle \tilde{1} | \exp(\hat{\Omega}_{1,p^*} + \hat{\Omega}_{2,p^*,p^*}) |\tilde{0}\rangle
    = \langle \tilde{1} | \exp \left( i B (\sin \varphi \hat{\sigma}^{\tilde{x}} - \cos \varphi \hat{\sigma}^{\tilde{y}}) + i C \hat{\sigma}^{\tilde{z}} \right)|\tilde{0}\rangle
    = \frac{Be^{i \varphi}}{\sqrt{B^2+C^2}} \sin \sqrt{B^2 + C^2},
\end{align}
where $B = \eta_{p^*} |\Theta_{p^*}^{(1)}|$ and $C = 2 \eta_{p^*}^2 \Im[\Theta_{p^*,p^*}^{(2)}]$. Therefore, the contribution of $\hat{\Omega}_{1,p^*} + \hat{\Omega}_{2,p^*,p^*}$ to the qubit population is given by $\frac{B^2}{B^2+C^2} \sin^2 \sqrt{B^2+C^2}$, which is approximately $B^2 - C^2$ in the perturbative regime where the expansion is performed with respect to $C/B \ll 1$ first and then $B \ll 1$. Compared to the case where the qubit-population transition is induced only by $\hat{\Omega}_{1,p^*}$, the addition of $\hat{\Omega}_{2,p^*,p^*}$ reduces the bright-state population by $C^2 = (\Im[\Theta^{(2)}_{p^*,p^*}])^2$. Other second-order Magnus terms $\hat{\Omega}_{2,p,p'}$ [$(p,p') \neq (p^*,p^*)$] are also expected to induce changes in the bright-state population that scale as $|\Theta^{(2)}_{p,p'}|^2$, as $\hat{\Omega}_{2,p,p'}$ is proportional to either $\hat{\sigma}^z$ (when $p \neq p'$) or $\hat{\sigma}^{\tilde{z}}$ (when $p = p'$).

Combined with the analysis in Appendix~\ref{sec:appendix_first_order_magnus} and \ref{sec:appendix_second_order_magnus}, this completes the discussion on how the first- and second-order contributions to the qubit population scale with respect to parameters such as $\alpha$, $\tau$, and $\delta_p$. For example, the scaling behaviors of $\mathcal{E} = |P-\mathcal{P}|/\mathcal{P}$ with respect to $\alpha$ and $\tau$ in Fig.~\ref{fig:2} is now fully understood. The denominator $\mathcal{P}$ in the $|\eta_{p^*} \alpha| \ll 1$ regime is approximately given by (\ref{eq:pop}) as $\sin^2(\eta_{p^*} \alpha) \approx (\eta_{p^*} \alpha)^2$, which is proportional to $\alpha^2$ in Fig.~\ref{fig:2}(a) and fixed to a constant in Fig.~\ref{fig:2}(b). For the numerator $|P - \mathcal{P}|$, the leading contribution is from $|\Theta_{p'}^{(1)}|^2$ ($|\Theta_{p^*,p'}^{(2)}|^2$) where $p' \neq p^*$ for square (shaped) pulses, which is roughly proportional to $\bar{A}^2 \tau^0$ ($\bar{A}^4 \tau^2$). Thus, in Fig.~\ref{fig:2}(a), for fixed $\tau \approx \alpha / \bar{A}$, $\mathcal{E}$ scales as $\bar{A}^2 / \alpha^2 \propto \alpha^0$ ($\bar{A}^4/\alpha^2 \propto \alpha^2$) for square (shaped) pulses. Also, in Fig.~\ref{fig:2}(b), for fixed $\alpha \approx \bar{A}\tau$, $\mathcal{E}$ scales as $\bar{A}^2 \tau^0 \propto \tau^{-2}$ ($\bar{A}^4 \tau^2 \propto \tau^{-2}$) for square (shaped) pulses. Similar arguments can be made for the scaling with respect to $\delta_p$. These scaling behaviors are important to understand the regime of parameters in which shaped pulses may allow more accurate mode characterization than square pulses. 


\subsection{Even- vs. odd-moment stabilization}
\label{app:even_odd_stabilization}
In Fig.~\ref{fig:3}(c) and (d), we observe that for integer $K$ and a wide range of pulse length $\tau$ and detuning $\delta$, moment-$2K$ pulses in general perform better than moment-$(2K+1)$ pulses. For example, while Fig.~\ref{fig:3}(b) clearly shows that moment-1 pulse suppresses $|\Theta_{p^*}^{(1)}(\delta) - \Theta_{p^*}^{(1)}(0)|$ compared to moment-0 pulse, Fig.~\ref{fig:3}(d) shows that moment-1 pulse has larger qubit-population error $\mathcal{E}(\delta)$ than moment-0 pulse for all plotted $\delta$. This appendix provides an explanation for this phenomenon. The key understanding is that while our pulse-shaping method for stabilization can suppress $|\Theta_p^{(1)}(\delta) - \Theta_p^{(1)}(0)|$, this is not equivalent to suppressing $| |\Theta_p^{(1)}(\delta)|^2 - |\Theta_p^{(1)}(0)|^2 |$, which is more relevant for the error in qubit population that is approximately given by $|\eta_{p^*} \Theta_{p^*}^{(1)}|^2$. 

For simplicity, we consider a resonant square pulse $g(t) = \bar{A} e^{-i\omega_{p^*}t + i\phi}$ as an example, and without losing generality we set $\phi=0$ as global phase of the pulse does not affect the qubit population. We evaluate $\Theta_{p^*}^{(1)}$ as this is the dominant contribution to the qubit population. From (\ref{eq:cmc}), we can easily find that $\partial^k \Theta_{p^*}^{(1)} / \partial \omega_{p^*}^k$ is real (imaginary) when $k$ is even (odd). In the presence of detuning $\omega_{p^*} \rightarrow \omega_{p^*} + \delta$, this leads to
\begin{equation}
    |\Theta_{p^*}^{(1)}(\delta)|^2 = 
    \left( \Theta_{p^*}^{(1)}(0) + \frac{\delta^2}{2!} \frac{\partial^2 \Theta_{p^*}^{(1)}}{\partial \omega_{p^*}^2} + \cdots \right)^2
    + \left( \delta \frac{\partial \Theta_{p^*}^{(1)}}{\partial \omega_{p^*}} + \frac{\delta^3}{3!} \frac{\partial^3 \Theta_{p^*}^{(1)}}{\partial \omega_{p^*}^3} + \cdots \right)^2,
\end{equation}
where all derivatives are evaluated at $\delta = 0$. Thus, the qubit-population error due to detuning $\delta$ is proportional to
\begin{equation}\label{eq:evenvsoddgeneral}
    |\Theta_{p^*}^{(1)}(\delta)|^2 - |\Theta_{p^*}^{(1)}(0)|^2
    = \sum_{k=1}^\infty \left[
    \frac{2}{(2k)!} \Theta_{p^*}^{(1)}(0) \frac{\partial^{2k} \Theta_{p^*}^{(1)}}{\partial \omega_{p^*}^{2k}}
    + \sum_{l=1}^{k-1} \frac{2}{l!(2k-l)!} \frac{\partial^{l} \Theta_{p^*}^{(1)}}{\partial \omega_{p^*}^{l}}\frac{\partial^{2k-l} \Theta_{p^*}^{(1)}}{\partial \omega_{p^*}^{2k-l}}
    + \left( \frac{1}{k!} \frac{\partial^k \Theta_{p^*}^{(1)}}{\partial \omega_{p^*}^k} \right)^2
    \right] \delta^{2k},
\end{equation}
which leads to
\begin{equation}\label{eq:evenvsodd}
    |\Theta_{p^*}^{(1)}(\delta)|^2 - |\Theta_{p^*}^{(1)}(0)|^2
    =  \left[ \Theta_{p^*}^{(1)}(0) \frac{\partial^{2} \Theta_{p^*}^{(1)}}{\partial \omega_{p^*}^{2}} + \left( \frac{\partial \Theta_{p^*}^{(1)}}{\partial \omega_{p^*}} \right)^2 \right] \delta^2 + \mathcal{O}(\delta^4)
\end{equation}

While (\ref{eq:evenvsodd}) is derived for square pulse, we argue that it is approximately valid for shaped pulses that have similar spectrum to square pulse, such as moment-0 and 1 pulses in Fig.~\ref{fig:pulse_prof}. In such case, (\ref{eq:evenvsodd}) explains why moment-1 pulse does not lead to smaller $\mathcal{E}(\delta)$ than moment-0 pulse. Moment-1 pulse only imposes $\partial \Theta_{p^*}^{(1)} / \partial \omega_{p^*} = 0$, which removes only the second term of the right-hand-side of (\ref{eq:evenvsodd}); the first term survives. Thus, the leading-order contribution of $\delta$ to $\mathcal{E}(\delta)$ is $\mathcal{O}(\delta^2)$ for both moment-0 and moment-1 pulses. At least second-order stabilization, which imposes $\partial^k \Theta_{p^*}^{(1)} / \partial \omega_{p^*}^k = 0$ for $k=1$ and $2$, is required to completely remove the $\mathcal{O}(\delta^2)$ contribution to $\mathcal{E}(\delta)$. As moment-1 pulses are more susceptible to the CMC than moment-0 pulses due to the larger average Rabi frequency $\bar{A}$ [see Fig.~\ref{fig:3}(a)], we expect moment-1 pulses to have larger qubit-population error than moment-0 pulses. 

The higher-order terms in (\ref{eq:evenvsoddgeneral}) suggest a more general claim that moment-$(2K+1)$ pulse does not outperform moment-$2K$ pulse in terms of smaller $\mathcal{E}(\delta)$. Indeed, Fig.~\ref{fig:4} shows that moment-3 pulses do not outperform moment-2 pulses for a significant range of $|\delta|$, especially for shorter pulse lengths. However, as shown in Fig.~\ref{fig:pulse_prof}, pulses with moment 2 or higher tend to have a spectrum that is substantially different from the resonant square pulse. Thus, for higher-moment pulses, $\mathcal{O}(\delta^{2K+1})$ terms may arise in the right-hand-side of (\ref{eq:evenvsoddgeneral}), which are completely removed by moment-$(2K+1)$ pulses but not by moment-$2K$ pulses. This is evidenced in the large-$|\delta|$ region of Fig.~\ref{fig:4}  where moment-3 pulses outperform moment-2 pulses. Whether moment-2 or moment-3 pulse leads to an overall better performance depends on the expected probability distribution of $\delta$ and the feasible range of pulse length $\tau$.

\section{Nulling Second-order Magnus Integrals}
\label{app:nulling_second_magnus}
To further minimize the CMC error, we describe our attempt at nulling the second-order Magnus integrals, thereby silencing the second-order Magnus terms in (\ref{eq:sec_order_term}). We adopt a similar approach to the pulse-shaping scheme in the main text that nulls unwanted first-order Magnus integrals. The scheme proposed here can be used in conjunction to the pulse-shaping scheme in the main text.

Specifically, we write an $N_{\rm{basis}}\times N_{\rm{basis}}$ matrix $\mathbf{Q}^{(p,p')}$ for each mode pair $(p,p')\neq (p^*,p^*)$ whose element is given by
\begin{equation*}
    Q_{n,n'}^{(p,p')} = \Theta^{(2)}_{p,p',n,n'}.
\end{equation*}
We then find the singular value decomposition of $|\mathbf{Q}^{(p,p')}|^2$ and select the set of singular vectors $S_{(p,p')}$ with singular values below a sufficiently small threshold of, for example, $10^{-10}$. The second-order Magnus integral $\Theta^{(2)}_{p,p'}$ is nulled by the pulse represented by a vector in $S_{(p,p')}$. We iterate this process over all mode-pairs except $(p,p')=(p^*,p^*)$ and find the overlap $\boldsymbol{\mathcal{S}}^{(2)}$ between the vector spaces $\mathbf{S}_{(p,p')}$ spanned by the vectors in $S_{(p,p')}$, i.e., $\boldsymbol{\mathcal{S}}^{(2)}:=\bigcap_{p,p',(p,p')\neq (p^*,p^*)} \mathbf{S}_{(p,p')}$. Finally, we find the overlap between $\boldsymbol{\mathcal{S}}^{(1)}$ nulling $\Theta^{(1)}_{p}$ as discussed in Sec.~\ref{sec:method} and $\boldsymbol{\mathcal{S}}^{(2)}$ nulling $\Theta^{(2)}_{p,p'}$, and proceed with the signal strength maximization over the set of vectors spanning the final overlapping space $\boldsymbol{\mathcal{S}}^{(1)}\cap \boldsymbol{\mathcal{S}}^{(2)}$. 

To find the overlap between two vector spaces, for example, $\boldsymbol{\mathcal{S}}^{(1)}$ and $\boldsymbol{\mathcal{S}}^{(2)}$, with spanning vectors $\mathcal{S}^{(1)}$ and $\mathcal{S}^{(2)}$, respectively, we find the nullspace of the enlarged matrix $U = \left(\hat{\mathcal{S}}^{(1)}|-\hat{\mathcal{S}}^{(2)}\right)$, where $\hat{\mathcal{S}}^{(1)}$ ($\hat{\mathcal{S}}^{(2)}$) is the matrix whose columns are the vectors in $\mathcal{S}^{(1)}$ ($\mathcal{S}^{(2)}$). Let's denote the vectors spanning the nullsapce of $U$ as $n_i^T=(n_i^{(1)} |\: n_i^{(2)})^T$. Then, the intersection space $\boldsymbol{\mathcal{S}}^{(1)}\cap \boldsymbol{\mathcal{S}}^{(2)}$ is spanned by the vectors $\hat{\mathcal{S}}^{(1)}n_i^{(1)}=\hat{\mathcal{S}}^{(2)}n_i^{(2)}$.

In practice, we found that the matrix $\mathbf{Q}^{(p,p')}$ does not have sufficiently small singular values when $p = p'$. Thus, our method above is applicable only when mode pairs ($p,p'$) such that $p \neq p'$ are considered. In Fig.~\ref{fig:sec_mag}, we show that for the case of a 3-ion chain and target mode $p^*=2$, the largest second-order contribution to the CMC error comes from $\Theta^{(2)}_{2,1}$. Thus, one might hope that additionally nulling the second-order Magnus integrals $\Theta^{(2)}_{p,p'}$ for $p \neq p'$ is sufficient for reducing the population error, compared to the pulse-shaping scheme in the main text that only nulls the first-order Magnus integrals. Unfortunately, it turns out that due to the increased power requirement for these additional nulling constraints, nulling $\Theta^{(2)}_{p,p'}$ for $p \neq p'$ results in increased magnitude of $\Theta^{(2)}_{p,p'}$ for $p = p'$, and thus no reduction in the qubit-population error. We expect similar results for all pulse lengths, as $|\Theta^{(2)}_{p^*,p'}|$ and $|\Theta^{(2)}_{p',p'}|$ ($p' \neq p^*$) exhibit similar scaling behavior with respect to $\tau$, as discussed in Appendix~\ref{sec:appendix_second_order_magnus}. 

\section{Comparison Error Landscape}
\label{app:ultimate_comparison}

In Fig.~\ref{fig:4}, we compared the performance of square and shaped pulses with different moments of stabilization in various regimes of pulse length $\tau$ and mode-frequency detuning $\delta$. Shaped pulses of different moments achieve the smallest qubit-population error $\mathcal{E}(\delta)$ compared to other pulses in different parameter regimes. Figure~\ref{fig:error_landscape} plots the values of $\mathcal{E}(\delta)$, providing a full comparison between square and shaped pulses of various moments of stabilization. 

% Figure environment removed

We first consider the case where $|\delta|$ is very small, such that the CMC is the dominant source of error $\mathcal{E}(\delta)$. In such case, as $\alpha$ is fixed and $\tau$ increases, the average Rabi frequency $\bar{A}$ decreases, which results in smaller effects of CMC and thus smaller $\mathcal{E}(\delta)$. This trend is observed in the narrow area at the horizontal center of each panel in Fig.~\ref{fig:error_landscape} as well as in Fig.~\ref{fig:2}(b). 

Next, we consider the case where $|\delta|$ is relatively large such that the detuning error dominates the CMC error. Here, an opposite trend emerges; as $\alpha$ is fixed and $\tau$ increases, $\mathcal{E}(\delta)$ increases. This is because as $\tau$ increases, a larger phase accumulates due to the detuning over the pulse duration. This trend is observed in the left and right ends of each panel in Fig.~\ref{fig:error_landscape} where $|\delta|$ is relatively large, as well as in Fig.~\ref{fig:3}(c). 

In the intermediate regime of $|\delta|$, effects of the CMC and detuning errors compete with each other. Given an expected range of detuning, or more specifically, the expected probability distribution of $\Pr(\delta)$, one can estimate the expected qubit-population error $\langle \mathcal{E} \rangle$ at each pulse length $\tau$ by integrating $\Pr(\delta)\mathcal{E}(\delta)$ over $\delta$. Then, the smallest $\langle \mathcal{E} \rangle$ will be achieved at an optimal $\tau$, where a transition from CMC-dominant regime ($\langle \mathcal{E} \rangle$ decreases as $\tau$ increases) to detuning-dominant regime ($\langle \mathcal{E} \rangle$ increases as $\tau$ increases) occurs. This transition is most clearly shown in Fig.~\ref{fig:error_landscape} for moment-2 and 3 shaped pulses and the plotted range of $\delta$, and also shown for lower-moment and square pulses when a narrower range of $\delta$ (e.g., $|\delta| \lesssim 2\pi \times 0.05 $ kHz) is considered. The optimal pulse length $\tau$ that achieves the smallest $\langle \mathcal{E} \rangle$ depends on the distribution $\Pr(\delta)$. This observation provides a general guideline for choosing the type of pulse and its length that enables as accurate mode characterization as possible. 

\section{Scalability}
\label{app:scalability}

For longer ion chains, silencing the CMC effect is expected to be more challenging. As the number of motional modes increases, more nulling constraints need to be satisfied when finding the pulse solution. Also, assuming equidistantly spaced ions, the spacings between neighboring mode frequencies are typically smaller for longer ion chains, as shown in Table~\ref{tab:modespacings}. With the more tightly-spaced motional mode frequencies, it is reasonable to expect that nulling $\Theta_{p}^{(1)}$ for all $p \neq p^*$ requires more pulse resources such as power and time duration. This brings two main concerns for the scalability of our pulse shaping scheme, namely the runtime and the power requirement. In this appendix, we address both of these concerns by solving for moment-0 shaped pulses for ion chains of length up to 7, demonstrating the scalability of our proposed pulse-shaping scheme. 

\begin{table*}[ht]
\renewcommand*{\arraystretch}{1.5}
\begin{tabular}{ c | c | c | c | c | c | c }
\hline
$N$ & $\Delta_{0,1}$ & $\Delta_{1,2}$ & $\Delta_{2,3}$ & $\Delta_{3,4}$ & $\Delta_{4,5}$ & $\Delta_{5,6}$\\
\hline
3 & 96.8 & 68.0 &  & & & \\
4 & 80.3 & 63.1 & 43.9 & & & \\
5 & 62.9 & 53.2 & 42.8 & 29.2 & & \\
6 & 53.9 & 48.2 & 41.6 & 34.4 & 21.8 & \\
7 & 43.6 & 41.5 & 38.4 & 34.1 & 29.2 & 15.9 \\
\hline
\end{tabular}
\caption{Spacings between neighboring mode frequencies $(\omega_{p+1}-\omega_{p})/2\pi$ are given in units of kHz, for various numbers of ions (modes) $N=N'$. All values of mode parameters are obtained by numerically
solving the normal modes of equidistantly spaced ions
trapped by a modelled potential of an HOA2.0 trap \cite{HOA}.} \label{tab:modespacings}
\end{table*}


First, we show how the runtime for finding the pulse solution scales with the number of ions (modes) $N=N'$. Specifically, we report the CPU runtime of our pulse solver on an Apple M1 Max chip. Although the number of nulling constraints increases linearly with $N$, we witness no increase in the runtime as $N$ is increased from 3 to 7, as shown in the left panel of Fig.~\ref{fig:scalability}. Instead, the runtime significantly increases with the pulse length $\tau$, which is proportional to the number of bases $N_{\rm{basis}}$ used. This is because $N_{\rm{basis}}$ determines the size of the matrix of which the nullspace vectors are calculated, which takes the longest runtime in our pulse-shaping scheme. We note that finding pulse solutions with length as long as $\tau = 2000$ $\mu$s only takes runtime less than a minute. As the runtime does not increase with $N$ (at least up to 7), we conclude that the runtime is unlikely to be a bottleneck for the scalability of our pulse-shaping scheme. 

Next, we discuss how the average Rabi frequency $\bar{A}$ of the shaped pulse scales with $N$. Somewhat surprisingly, the average Rabi frequency of moment-0 pulse does not change significantly as the number of modes is increased from 3 to 7, as shown in the right panel of Fig.~\ref{fig:scalability}. At least for the simulated numbers of modes and mode-frequency values, adding the nulling constraint $\Theta_{p}^{(1)}=0$ for $p \neq p^*$ does not incur additional power requirement, compared to square pulse with $\bar{A} = \alpha/\tau$. This is consistent with our previous observation in Fig.~\ref{fig:3}(a) that $\bar{A}$ is essentially the same for square and moment-0 pulses and increases only with larger moment of stabilization. The fact that suppressing the CMC effect is ``free'' in terms of power requirement makes our pulse-shaping scheme even more promising for application to longer ion chains. 

% Figure environment removed

