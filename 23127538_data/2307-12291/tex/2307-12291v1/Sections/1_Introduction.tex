%&latex
\section{Introduction}
% background 
Rendering free-viewpoint videos of dynamic human performers in high fidelity is vital for many applications such as  mixed reality, gaming, and telepresence. Recent works~\cite{peng2021neural,peng2021animatable,weng2022humannerf,te2022NeuralCapture} integrate the Neural Radiance Fields (NeRF)~\cite{mildenhall2021nerf} technology with parametric human prior models (\eg, SMPL~\cite{loper2015smpl}) for handling the dynamic human body and achieve fair novel view synthesis results. However, the tedious per-subject optimization and the requirement of dense training views largely hinder the application of such methods. Targeting these issues and inspired by the recent success of generalizable NeRF~\cite{yu2021pixelnerf,mvsnerf,wang2021ibrnet} on static scenes, the task of generalizable neural human rendering is proposed~\cite{kwon2021nhp}, which trains conditional NeRF across multi-view human videos, and can generalize to a new subject in a single feed-forward manner given sparse reference views as input. 

\input{./Figures/idea_illustration}

% the drawbacks of previous methos 
Previous methods for generalizable neural human rendering~\cite{chen2022gpnerf,kwon2021nhp} mainly employ the SparseConvNet (SPC)~\cite{liu2015sparse}-based  human representation (upper row of Fig. ~\ref{fig_idea}) which first project deep features from reference images onto the vertices of fitted SMPL and then diffuse them to nearby regions via SPC. The final representation is achieved via the trilinear sampling  in the discrete 3D feature volume. Such SPC-based representation mainly suffers from the following two aspects: (i) \textit{Volatile observation learning.} The SPC-based one optimizes under the observation space that contains varying poses. This leads to the pose misalignment during training and inference stages, and therefore limits the generalization ability. 
(ii) \textit{Limited local receptive fields.} 
As shown in Fig.~\ref{fig_idea}, due to the heavy self-occlusion of dynamic human bodies, the painted SMPL templates are usually incomplete. While, as a 3D convolution network, the limited local receptive fields of SPC make it sensitive to the incomplete input, especially when the occluded regions are large. 

% To address the aforementioned issues, 
To address the aforementioned issues, we propose to first process the painted SMPL with transformers under the \textit{static canonical space} to remove the pose misalignment between training and inference stages and capture the \textit{global relationships} between human parts. Then, a deformation from the canonical to the observation space is required to fetch the human representation of a query point (sampling points on rays) under the observation space. Finally, the fine-grained information directly achieved from the observation space should be further included to the coarse human representation to complement the details. 

% To address the aforementioned issues, we propose that processing the painted SMPL under the \textit{static canonical space} can remove the pose misalignment between training and inference stages, and the processed SMPL can be deformed back to the observation space for further diffusion. Also, the transformers~\cite{dosovitskiy2020ViT} is a good choice for capturing the \textit{global relationships} between human parts. 
% Finally, the fine-grained information directly achieved from the observation space should be further included to the coarse human representation to complement for the details. 


Motivated by this, we present the TransHuman, a brand-new framework that shows superior generalization ability with high efficiency. 
% , as shown by the lower part of Fig.~\ref{fig_idea}. 
% that optimizes under the \textit{canonical} space and processes the painted SMPL in a \textit{global} and \textit{continuous} manner, 
TransHuman is mainly composed of Transformer-based Human Encoding (TransHE), Deformable Partial Radiance Fields (DPaRF), and Fine-grained Detail Integration (FDI). (i) \textit{TransHE.} TransHE is a pipeline that processes the painted SMPL under the canonical space with transformers~\cite{dosovitskiy2020ViT}. The core of this pipeline includes a canonical body grouping strategy for the avoidance of semantic ambiguity, and a canonical learning scheme to ease the learning of global relationships. 
(ii) \textit{DPaRF.} DPaRF deforms the output tokens of TransHE from the canonical space to the observation space and gets a robust human representation for a query point from marched rays. As shown in Fig. \ref{fig_idea}, the main idea is to bind each token (representing a certain human part) with a radiance field whose partial coordinate system deforms as the pose changes, and the query point is encoded via the coordinates under the deformed partial coordinate systems.
(iii) \textit{FDI.} With TransHE and DPaRF, the human representation contains coarse information with human priors yet limited fine-grained details directly captured from the observation space. Therefore, similar to ~\cite{kwon2021nhp}, we propose to further integrate the detailed information from the pixel-aligned features at the guidance of the human representation. 


% Experiment and Summarization
Extensive experiments on ZJU-MoCap~\cite{peng2021neural} and H36M~\cite{ionescu2013human36m} demonstrate the superior generalization ability and high efficiency of TransHuman which attains a new state-of-the-art performance and outperforms previous methods by significant margins, \eg, $+2.20$ PSNR and $-45\%$ LPIPS on ZJU-MoCap~\cite{peng2021neural}  under the pose generalization setting. 

% that our TransHuman is both effective and efficient which outperforms previous methods by significant margins. 

% the superior generalization ability of TransHuman which outperforms previous methods by significant margins with high efficiency. 
% attains a new state-of-the-art performance .

Our contributions are summarized as follows:
\begin{itemize}
    \item We propose a brand-new framework TransHuman for the challenging generalizable neural human rendering task which attains a significantly new state-of-the-art performance with high efficiency.
    % which optimizes under the static canonical space and encodes the human representation in a global and continuous manner.  
    % \item To the best of our knowledge, we make the first attempt to explore the transformer technology around the painted SMPL. 
    % \item We attain a significantly new state-of-the-art performance with high efficiency.

    % $TransHE,DPaRF$
    \item We propose to process the painted SMPL under the canonical space to remove the pose misalignment during training and inference  stages and deform it back to the observation space via DPaRF for robust query point encoding. 
    
    \item To the best of our knowledge, we make the first attempt to explore the transformers technology around the painted SMPL for capturing the global relationships between human parts. 

    

    % \item We propose a brand-new framework TransHuman for the challenging generalizable neural human rendering task which attains a significantly new state-of-the-art performance with high efficiency.

    % \item We propose the TransHE that optimizes the 
    % % encodes the human representation in a global and continuous manner

    % \item To the best of our knowledge, we make the first attempt to explore the transformer technology around the surface of the parametric human body. 
    
\end{itemize}

% TODO: include the discussion about generalization ability 
% SPC learns the painted SMPL on observation space, which contains infinite poses, therefore . 
% While our TransHuman learns on the canonical space, and deform to obs space for application, the final continuous encoding is non-parametric. Therefore shows better generalizable ability.  

%%%%%%%%%%%%%%%%%%
\section{Secure Design of \puma}\label{sec:design}
In this section, we first present an overview of \puma, and present the protocols for secure $\gelu$ , $\softmax$, embedding, and $\layernorm$ used by \puma. Note that the linear layers such as matrix multiplication are straightforward in replicated secret sharing, so we mainly describe our protocols for non-linear layers in this manuscript.

\subsection{Overview of \puma}\label{sec:overview}
To achieve secure inference of Transformer models, \puma\ defines three kinds of roles: one model owner, one client, and three computing parties. The model owner and the client  provide their models or inputs to the computing parties (i.e., $P_0$, $P_1$, and $P_2$) in a secret-shared form, then the computing parties execute the MPC protocols and send the results back to the client. Note that the model owner and client can also act as one of the computing party, we describe them separately for generality. \eg, when the model owner acts as $P_0$, the client acts as  $P_1$, a third-party dealer acts as $P_2$, the system model becomes the same with \mpcformer~\citep{li2023mpcformer}.

During the secure inference process, a key invariant is maintained: For any layer, the computing parties always start with 2-out-of-3 replicated secret shares of the previous layer's output and the model weights, and end with 2-out-of-3 replicated secret shares of this layer's output. As the shares do not leak any information to each party, this ensures that the layers can be sequentially combined for arbitrary depths to obtain a secure computation scheme for any Transformer-based model.
%The main focus of \puma\ is to reduce the computation and communication costs between the computing parties while maintaining the desired level of security. 



\iffalse
\textbf{Threat Model.}
Following previous works~\citep{aby3,li2023mpcformer},
\puma\ resists a semi-honest (a.k.a., honest-but-curious) adversary in honest-majority~\citep{lindell2009proof}, where the adversary passively corrupts no more than one computing party. Such an adversary follows the protocol specification exactly, but may try to learn more information than permitted. Please note that \puma\ cannot protect against the extraction of information from the inference results, and the examination of mitigating solutions (\eg, differential privacy~\citep{abadi2016deep}) falls outside the scope of this study.
\fi 

\subsection{Protocol for Secure GeLU}\label{sec:gelu}
Most of the current approaches view the $\gelu$ function as a composition of smaller functions and try to optimize each piece of them, making them to miss the
chance of optimizing the private $\gelu$ as a whole. Given the $\gelu$ function:
\begin{equation}\label{eq:gelu}
\begin{split}
    \gelu(x) &= \frac{x}{2} \cdot \left(1 + \tanh \left( \sqrt{\frac{2}{\pi}} \cdot \left(x + 0.044715 \cdot x^3 \right) \right) \right)\\
    &\approx x\cdot \mathsf{sigmoid}(0.071355\cdot x^3 + 1.595769\cdot x) 
\end{split},
\end{equation}
these approaches~\citep{hao2022iron,characmpctranformer} focus either on designing efficient protocols for function $\tanh$
or using the existing MPC protocols of exponentiation and reciprocal for $\mathsf{sigmoid}$. 

However, none of current approaches have utilized the fact that $\gelu$ function is almost linear on the two sides (\ie, $\gelu(x)\approx 0$ for $x<-4$ and $\gelu(x)\approx x$ for $x>3$). 
Within the short interval $[-4,3]$ of $\gelu$,
we suggest a piece-wise approximation of low-degree polynomials is a more efficient and easy-to-implement choice for its secure protocol. Concretely, our piece-wise low-degree polynomials are shown as equation~(\ref{eq:geluapprox}):
\begin{equation}\label{eq:geluapprox}
\gelu(x)=
\begin{cases}
0, & x<-4 \\
F_0(x), & -4 \le x < -1.95 \\
F_1(x), & -1.95 \le x \le 3 \\
x, & x >3
\end{cases},
\end{equation}
where polynomials $F_0()$ and $F_1()$ are computed by library $\mathsf{numpy.ployfit}$\footnote{\url{https://numpy.org/doc/stable/reference/generated/numpy.polyfit.html}} as equation~(\ref{eq:f0f1}). Surprsingly, the above simple poly fit works very well and our $\mathsf{max\ error}< 0.01403$, $\mathsf{median\ error}< 4.41e-05$, and $\mathsf{mean\ error}< 0.00168$.
\begin{equation}\label{eq:f0f1}
\begin{cases}
F_0(x) &= -0.011034134030615728 x^3 -0.11807612951181953 x^2 \\
&- 0.42226581151983866 x -0.5054031199708174\\
F_1(x) &= 0.0018067462606141187x^6 -0.037688200365904236 x^4 \\
&+ 0.3603292692789629x^2 + 0.5x + 0.008526321541038084
\end{cases}
\end{equation}

Formally, given secret input $\share{x}$, our secure $\gelu$ protocol $\Pi_{\gelu}$ is constructed as algorithm~\ref{protocol:gelu}. 
\iffalse
\begin{itemize}
    \item The parties jointly compute
$\share{b_0}^2 = \Pi_{\mathsf{LT}}(\share{x}, 4)$,
$\share{b_1}^2 = \Pi_{\mathsf{LT}}(\share{x}, -1.95)$, and
$\share{b_2}^2 = \Pi_{\mathsf{LT}}(3, \share{x})$.

\item  Then, each $P_i$ locally compute
$\share{b_3}^2 = \share{b_1}^2 \oplus \share{b_2}^ \oplus 1$ and
$\share{b_4}^2 = \share{b_0}^2 \oplus \share{b_1}^2$

\item Finally, the parties compute and return 
$\share{b_2}^2 \cdot \share{x} + \share{b_4}^2 \cdot F_0(\share{x}) + \share{b_3}^2 \cdot F_1(\share{x})$, where polynomials $(F_0, F_1)$ can be computed easily using secure addition and multiplication (and its variants, \eg, secure square)~\citep{spu}. 
\end{itemize}
\fi 

\begin{algorithm}[tp]
\caption{Secure $\gelu$ Protocol $\Pi_{\mathsf{GeLU}}$}\label{protocol:gelu}
\begin{algorithmic}[1]
\REQUIRE
$P_i$ holds the 2-out-of-3 replicate secret share $\share{x}_i$ for $i\in \{0,1,2\}$ 
\ENSURE
$P_i$ gets the 2-out-of-3 replicate secret share $\share{y}_i$ for $i\in \{0,1,2\}$, where $y=\gelu(x)$.

\STATE $P_0$, $P_1$, and $P_2$ jointly compute
\begin{equation*}
\begin{split}
&\shareb{b_0} = \Pi_{\mathsf{LT}}(\share{x}, -4),~~~\vartriangleright b_0 = 1\{x<-4\}\\
&\shareb{b_1} = \Pi_{\mathsf{LT}}(\share{x}, -1.95),~~~\vartriangleright b_1 = 1\{x<-1.95\} \\
&\shareb{b_2} = \Pi_{\mathsf{LT}}(3, \share{x}),~~~~~~\vartriangleright b_2 = 1\{3<x\}
\end{split}
\end{equation*}
and compute 
$\shareb{z_0} = \shareb{b_0} \oplus \shareb{b_1}$,
$\shareb{z_1} = \shareb{b_1} \oplus \shareb{b_2} \oplus 1$, and $\shareb{z_2}=\shareb{b_2}$. Note that $z_0 = 1\{-4\le x < -1.95\}$, $z_1 = 1\{-1.95\le x\le 3\}$, and $z_2 = 1\{x>3\}$.

\STATE Jointly compute $\share{x^2} = \Pi_{\mathsf{Square}}(\share{x})$, $\share{x^3} = \Pi_{\mathsf{Mul}}(\share{x}, \share{x^2})$, $\share{x^4} = \Pi_{\mathsf{Square}}(\share{x^2})$, and $\share{x^6} = \Pi_{\mathsf{Square}}(\share{x^3})$.

\STATE Computing polynomials $\share{F_0(x)}$ and $\share{F_1(x)}$ based on $\{\share{x}, \share{x^2}, \share{x^3}, \share{x^4}, \share{x^6}\}$ as equation~(\ref{eq:geluapprox}) securely.


\RETURN$\share{y} = \Pi_{\mathsf{Mul_{BA}}}(\shareb{z_0}, \share{F_0(x)}) + \Pi_{\mathsf{Mul_{BA}}}(\shareb{z_1}, \share{F_1(x)})+\Pi_{\mathsf{Mul_{BA}}}(\shareb{z_2}, \share{x})$.

\end{algorithmic}
\end{algorithm}



\subsection{Protocol for Secure Softmax}\label{sec:secureatten}

In the function $\attention(\Q,\K,\V)=
\softmax(\Q \cdot \K^\mathsf{T} + \M) \cdot \V$, where $\M$ can be viewed as a bias matrix, the key challenge is computing function $\softmax$. For the sake of numerical stability, the $\softmax$ function is computed as
\begin{equation}\label{eq:softmax}
    \softmax(\x)[i]=\frac{\exp(\x[i] - \bar{x} - \epsilon)}{\sum_i \exp(\x[i] - \bar{x} - \epsilon)},
\end{equation}
where $\bar{x}$ is the maximum element of the input vector $\x$. 
For the normal plaintext softmax, $\epsilon=0$. For a two-dimension matrix, we apply equation~(\ref{eq:softmax}) to each of its row vector.

Formally, our detailed secure protocol  $\Pi_{\softmax}$ is illustrated in algorithm~\ref{protocol:softmax}, where we propose two optimizations:
\begin{itemize}
\item 
For the first optimization, we set $\epsilon$ in equation~\ref{eq:softmax} to a tiny and positive
value, e.g., $\epsilon =
10^{-6}$, so that the inputs to exponentiation
in equation~\ref{eq:softmax} are all negative. We exploit the negative operands
for acceleration. Particularly, we compute the exponentiation using the Taylor series~\citep{tan2021cryptgpu} with a simple clipping
\begin{equation}\label{eq:negexp}
\mathsf{negExp}(x) = \begin{cases}
    0, &x < T_{\exp} \\
    (1+\frac{x}{2^t})^{2^t}, &x\in [T_{\exp},0].
\end{cases}
\end{equation}
Indeed, we apply the less-than for the branch $x < T_{\exp}$
The division by $2^t$ can be achieved using
$\Pi_{\mathsf{Trunc}}^t$ since the input is already negative. Also, we can
compute the power-of-$2^t$ using $t$-step sequences of square function $\Pi_{\mathsf{square}}$ and $\Pi_{\mathsf{Trunc}}^f$. Suppose our MPC program uses
$18$-bit fixed-point precision. Then we set $T_{\exp}=-14$ given $\exp(-14) < 2^{-18}$, and empirically set $t = 5$.


\item 
Our second optimization is to reduce the number of divisions, which ultimately saves computation and communication costs.
To achieve this, for a vector $\x$ of size $n$, we have replaced the operation $\mathsf{Div}(\x, \mathsf{Broadcast}(y))$ with $\x \cdot  \mathsf{Broadcast}(\frac{1}{y})$, where $y=\sum_{i=1}^n\x[i]$. By making this replacement, we effectively reduce $n$ divisions to just one reciprocal operation and $n$ multiplications.
This optimization is particularly beneficial in the case of the $\softmax$ operation. The $\frac{1}{y}$ in the $\softmax$ operation is still large enough to maintain sufficient accuracy under fixed-point values. As a result, this optimization can significantly reduce the computational and communication costs while still providing accurate results.
\end{itemize}

\begin{algorithm}[tp]
\caption{Secure $\softmax$ Protocol $\Pi_{\softmax}$}\label{protocol:softmax}
\begin{algorithmic}[1]
\REQUIRE
$P_i$ holds the 2-out-of-3 replicate secret share $\share{\x}_i$ for $i\in \{0,1,2\}$, and $\x$ is a vector of size $n$. 
\ENSURE
$P_i$ gets the 2-out-of-3 replicate secret share $\share{\y}_i$ for $i\in \{0,1,2\}$, where $\y=\softmax(\x)$.

\STATE $P_0$, $P_1$, and $P_2$ jointly compute
$\shareb{\mathbf{b}} = \Pi_{\mathsf{LT}}(T_{\exp}, \share{\x})$ and the maximum $\share{\bar{x}} = \Pi_{\mathsf{Max}}(\share{\x})$.

\STATE Parties locally computes $\share{\hat{\x}} = \share{\x} - \share{\bar{x}} - \epsilon$, and jointly compute $\share{\z_0} = 1+  \Pi_{\mathsf{Trunc}}^t(\share{\hat{\x}})$.

\FOR{$j=1,2,\dots, t$}
\STATE $\share{\z_j} = \Pi_{\mathsf{Square}}(\share{\z_{j-1}})$.
\ENDFOR

\STATE Parties locally compute $\share{z} = \sum_{i=1}^n \share{\z[i]}$ and jointly compute $\share{1/z} = \Pi_{\mathsf{Recip}}(\share{z})$.

\STATE Parties jointly compute $\share{\z / z} = \Pi_{\mathsf{Mul}}(\share{\z}, \share{1/z})$

\RETURN $\share{\y} = \Pi_{\mathsf{Mul}_{\mathsf{BA}}}( \shareb{\mathbf{b}}, \share{\z / z})$.

\end{algorithmic}
\end{algorithm}

\subsection{Protocol for Secure Embedding}\label{sec:embed}


The current secure embedding procedure described in~\citep{li2023mpcformer} necessitates the client to  generate a one-hot vector using the token $\tokenid$ locally. This deviates from a plaintext Transformer workflow where the one-hot vector is generated inside the model. As a result, they have to carefully strip off the one-hot step from the pre-trained models, and add the step to the client side, which could be an obstacle for deployment. 



To address this issue, we propose a secure embedding design as follows. Assuming that the token $\tokenid\in [n]$ and all embedding vectors are denoted by $\E= (\e_1^T, \e_2^T, \dots, \e_n^T)$, the embedding can be formulated as $\e_{\tokenid} = \mathbf{E}[\tokenid]$. Given $(\tokenid, \E)$ are in secret-shared fashion, our secure embedding protocol $\Pi_{\mathsf{Embed}}$ works as follows:
\begin{itemize}
    \item The computing parties securely compute the one-hot vector $\shareb{\mathbf{o}}$ after receiving $\share{\tokenid}$ from the client. Specifically, $\shareb{\mathbf{o}[i]}=\Pi_{\mathsf{Eq}}(i,\share{\tokenid})$ for $i\in [n]$.
    \item The parties can compute the embedded vector via $\share{\e_{\tokenid}} = \Pi_{\mathsf{Mul_{BA}}}(\share{\E}, \shareb{\mathbf{o}})$, where  does not require secure truncation.
\end{itemize}
In this way, our $\Pi_{\mathsf{Embed}}$ does not require explicit modification of the workflow of plaintext Transformer models, at the cost of more $\Pi_{\mathsf{Eq}}$ and $\Pi_{\mathsf{Mul_{BA}}}$ operations. 



\subsection{Protocol for Secure LayerNorm}\label{sec:seclayernorm}
Recall that given a vector $\x$ of size $n$, $\layernorm(\x)[i] =  \gamma \cdot \frac{\x[i]-\mu}{\sqrt{\sigma}} + \beta$, where $(\gamma, \beta)$ are trained parameters, $\mu = \frac{\sum_{i=1}^n \x[i]}{n}$, and $\sigma = \sum_{i=1}^n (\x[i] - \mu)^2$. In MPC, the key challenge is the evaluation of the divide-square-root $\frac{\x[i]-\mu}{\sqrt{\sigma}}$ formula. To securely evaluate this formula, CrypTen sequentially executes the MPC protocols of square-root, reciprocal, and multiplication. However, we observe that $\frac{\x[i]-\mu}{\sqrt{\sigma}}$ is equal to $(\x[i]-\mu)\cdot \sigma^{-1/2}$. And in the MPC side, the costs of computing the inverse-square-root $\sigma^{-1/2}$ is similar to that of the square-root operation~\citep{rSqrt}. Besides, inspired by the second optimization of \S~\ref{sec:secureatten}, we can first compute $\sigma^{-1/2}$ and then $\mathsf{Broadcast}(\sigma^{-1/2})$ to support fast and secure $\layernorm(\x)$. And our formal protocol $\Pi_{\layernorm}$ is shown in algorithm~\ref{protocol:layernorm}.

\begin{algorithm}[tp]
\caption{Secure $\mathsf{LayerNorm}$ Protocol $\Pi_{\mathsf{LayerNorm}}$}\label{protocol:layernorm}
\begin{algorithmic}[1]
\REQUIRE
$P_i$ holds the 2-out-of-3 replicate secret share $\share{\x}_i$ for $i\in \{0,1,2\}$, and $\x$ is a vector of size $n$. 
\ENSURE
$P_i$ gets the 2-out-of-3 replicate secret share $\share{\y}_i$ for $i\in \{0,1,2\}$, where $\y=\mathsf{LayerNorm}(\x)$.

\STATE $P_0$, $P_1$, and $P_2$ compute $\share{\mu} = \frac{1}{n}\cdot \sum_{i=1}^n\share{\x[i]}$ and $\share{\sigma} = \sum_{i=1}^n \Pi_{\mathsf{Square}}(\share{\x} - \share{\mu})[i]$.

\STATE Parties jointly compute $\share{\sigma^{-1/2}} = \Pi_{\mathsf{rSqrt}}(\share{\sigma})$.

\STATE Parties jointly compute $\share{\mathbf{c}} = \Pi_{\mathsf{Mul}}((\share{\x} - \share{\mu}), \share{\sigma^{-1/2}})$

\RETURN $\share{\y} = \Pi_{\mathsf{Mul}}(\share{\gamma}, \share{\mathbf{c}}) + \share{\beta}$.

\end{algorithmic}
\end{algorithm}