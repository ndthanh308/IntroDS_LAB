\documentclass[letterpaper,conference]{IEEEtran}
\usepackage{cite}
\usepackage[pdftex]{graphicx}
 \graphicspath{{./img}}
 \DeclareGraphicsExtensions{.pdf,.jpeg,.png,.eps}

\usepackage{color}
\usepackage{xspace}
\usepackage{amssymb}

\usepackage{textcomp}
\usepackage{hyperref}


\usepackage[cmex10]{amsmath}
\interdisplaylinepenalty=2500

\def\BibTeX{{\rm B\kern-.05em{\sc i\kern-.025em b}\kern-.08em
    T\kern-.1667em\lower.7ex\hbox{E}\kern-.125emX}}

% *** SPECIALIZED LIST PACKAGES ***
%
\usepackage{algorithm, algpseudocode}


\usepackage{stfloats}



% correct bad hyphenation here
\hyphenation{op-tical net-works semi-conduc-tor}

\newcommand\calF{\mathcal{F}}
\newcommand\calG{\mathcal{G}}
\newcommand\calM{\mathcal{M}}
\newcommand\calV{\mathcal{V}}
\newcommand\calU{\mathcal{U}}
\newcommand\calW{\mathcal{W}}
\newcommand\calP{\mathcal{P}}
\newcommand\calD{\mathbb{D}}
%%%%%%%%%%%%%%%%%
%% macros introduced by Luke 
\newcommand\mydef[1]{{\bf\em #1}}
%%%%%%%%%%%%%%%%%

\newcommand{\numviparams}{{| \lambda |}}
\newcommand{\scoreaccvars}[1]{s_1^{#1}, \ldots, s_{\numviparams}^{#1}}
\newcommand{\scoreaccvar}[2]{s_{#1}^{#2}}
\newcommand{\isdeterm}[1]{\text{Deterministic}({#1})}


\newcommand{\expect}[1]{\mathbb{E}\left[{#1}\right]}
\newcommand{\var}[1]{\mathbb{V}\left[ {#1} \right]}
\newcommand{\expectdist}[2]{\mathbb{E}_{#1}\left[ {#2} \right]}
\newcommand{\vardist}[2]{\mathbb{V}_{#1}\left[ {#2} \right]}
\newcommand{\cov}[2]{\mathbb{C}\text{ov}[{#1}][{#2}]}
\newcommand{\covv}[1]{\mathbb{C}\text{ov}[{#1}]}
\newcommand{\corr}[1]{\mathbb{C}\text{orr}[{#1}]}

\newcommand{\fix}[1]{\mathit{fix}\left({#1}\right)}
\newcommand{\sbr}[1]{\left\llbracket {#1} \right\rrbracket}
\newcommand{\ctxtype}[3]{{#1} \cong_\text{ctx} {#2} : {#3}}
\newcommand{\bigstep}[3]{{#1} \Downarrow_{#2} {#3}}


% PCF types
\newcommand{\bool}{\mathit{bool}}
\newcommand{\nat}{\mathit{nat}}

\newcommand{\ctx}[1]{\mathcal{C}\left[ {#1}\right] }
\newcommand{\pcft}[1]{\text{PCF}_{#1}}

\newcommand{\nfl}{\mathbb{N}_\bot}
\newcommand{\bfl}{\mathbb{B}_\bot}

% PCF constructs
\newcommand{\succc}[1]{\mathbf{succ}({#1})}
\newcommand{\succcn}[2]{\mathbf{succ}^{#1}({#2})}
\newcommand{\zero}{\mathbf{0}}
\newcommand{\zerotest}[1]{\mathbf{zero}\left({#1}\right)}
\newcommand{\pred}[1]{\mathbf{pred}\left( {#1} \right)}
\newcommand{\predn}[2]{\mathbf{pred}^{#1}\left( {#2} \right)}
\def\solvable{\#}

\newcommand{\true}{\mathbf{true}}
\newcommand{\false}{\mathbf{false}}
\newcommand{\pcffix}[1]{\mathbf{fix}\left({#1}\right)}
\newcommand{\pcffn}[3]{\mathbf{fn}~{#1}:{#2}\mathpunct{.}{#3}}
\newcommand{\pairtype}[2]{{#1} * {#2}}
\newcommand{\pairexp}[2]{\mathbf{pair}({#1}, {#2})}
\newcommand{\leftexp}[1]{\mathbf{left}({#1})}
\newcommand{\rightexp}[1]{\mathbf{right}({#1})}

\newcommand{\RationalPos}{\mathbb{Q}^{+}}

\newcommand{\meas}[1]{\mathbb{M}\left( {#1} \right) }
\newcommand{\integ}[1]{\sbr{#1}_I}

\newcommand{\notbigstep}[2]{{#1}~\cancel{\Downarrow}_{#2}}
\newcommand{\subtrace}[3]{{#1}^{{#2} \ldots {#3}}}
\newcommand{\supp}[1]{\textsf{supp}\left({#1}\right)}
\newcommand{\dom}[1]{\textsf{Dom}\left({#1}\right)}
\newcommand{\suppk}[2]{\textsf{Supp}^{#1}\left({#2}\right)}
\newcommand{\tracespace}{\bigcup_{n \in \mathbb{N}}[0, 1]^n}
\newcommand{\generictracespace}{\mathbb{T}}
\newcommand{\nnreals}{\mathbb{R}_{\geq 0}}
\newcommand{\posreals}{\mathbb{R}_{> 0}}
\newcommand{\reals}{\mathbb{R}}

\newcommand{\unrollkM}[2]{\textsf{unroll}_{#1}\left({#2}\right)}
\newcommand{\nphmcint}[5]{\Psi_\textsf{NP}\left({#1}, {#2}, {#3}, {#4}, {#5}\right)}

%SPCF constructs
\newcommand{\spcfvalues}{\Lambda^0_v}

\newcommand{\prevalueM}[1]{\textsf{value}^{-1}_{#1}(\spcfvalues{})}
\newcommand{\num}[1]{\underline{#1}}

% \theoremstyle{definition}
% \newtheorem{thm}{Theorem}
% \newtheorem{lem}{Lemma}
% \newtheorem{defn}{Definition}
% \newtheorem{conj}{Conjecture}
% \newtheorem{prop}{Proposition}

%\theoremstyle{definition}
%\newtheorem{defn}{Definition}[section]
%\newtheorem{example}[defn]{Example}
%
%
%\theoremstyle{plain}
%\newtheorem{thm}{Theorem}[section]
%\newtheorem{lem}[thm]{Lemma}
%\newtheorem{cor}[thm]{Corollary}
%\newtheorem{conj}[thm]{Conjecture}
%\newtheorem{prop}[thm]{Proposition}
%\newtheorem{remark}[thm]{Remark}

%% Proofs
%\let\oldproof\proof
%\renewcommand{\proof}{\color{blue}\oldproof}


\definecolor{codegreen}{rgb}{0,0.6,0}
\definecolor{codegray}{rgb}{0.5,0.5,0.5}
\definecolor{codepurple}{rgb}{0.58,0,0.82}
\definecolor{backcolour}{rgb}{0.95,0.95,0.92}

\lstdefinestyle{myStyle}{
    belowcaptionskip=1\baselineskip,
    breaklines=true,
    frame=none,
    basicstyle=\footnotesize\ttfamily,
    keywordstyle=\bfseries\color{green!40!black},
    commentstyle=\itshape\color{purple!40!black},
    identifierstyle=\color{blue},
    backgroundcolor=\color{gray!10!white},
    %backgroundcolor=\color{backcolour}, 
    numberstyle=\tiny\color{codegray},
    stringstyle=\color{codepurple},
    breakatwhitespace=false,                          
    keepspaces=true,                 
    numbers=left,       
    numbersep=5pt,                  
    showspaces=false,                
    showstringspaces=false,
    showtabs=false,                  
    tabsize=2,
}

% argmin/argmax
\DeclareMathOperator*{\argmax}{arg\,max}
\DeclareMathOperator*{\argmin}{arg\,min}

% Concatenation of lists
\newcommand\doubleplus{+\kern-1.3ex+\kern0.8ex}

% Program configurations
\newcommand{\tuple}[1]{\ensuremath{\langle #1 \rangle}}
% Rule based definitions
\newcommand{\Rule}[4][]{\ensuremath{\inferrule*[lab={\hypertarget{#2}{(\TirName{#2})}},#1]{#3}{#4}}}

% Calligraphic symbols
\newcommand{\calI}{{\mathcal I}} 
\newcommand{\calT}{{\mathcal T}}

%  Macro for new Y operator.
\newcommand{\yBounded}[3]{\mu^{#1}_{#2}\rvert_{#3}}

%%%%%%%%%%%%%%%%%
 
%%%%%%%%%%%%%%%%%

\newcommand{\expv}{\mathbb{E}}

\newcommand{\combTr}[2]{\left[\begin{matrix}
		#1\\
		#2
	\end{matrix} \right]}

\newcommand{\exType}[2]{\left\{\begin{matrix}
		#1\\
		#2
	\end{matrix} \right\}}
\newcommand{\myint}[1]{ [#1]}
\newcommand{\Uniform}{\ensuremath{\mathrm{Uniform}}}
\newcommand{\Normal}{\ensuremath{\mathrm{normal}}}
\DeclareMathOperator{\abs}{abs}
\DeclareMathOperator{\pdf}{pdf}

\newcommand{\intConf}[1]{\lceil#1\rceil}
\newcommand{\tr}{\boldsymbol{t}}

\newcommand{\sample}{\tt{sample}}
%\newcommand{\fix}{\texttt{fix}}
%\newcommand{\num}[1]{\underline{#1}}
\newcommand{\myif}{\texttt{if}}
\newcommand{\mylet}{\texttt{let} \, }
\newcommand{\myin}{\, \texttt{in} \,}
\newcommand{\mythen}{\, \texttt{then} \,}
\newcommand{\myelse}{\, \texttt{else} \,}
\newcommand{\score}{\tt{score}}
\newcommand{\tick}{\tt{tick}}

\newcommand{\term}{\tt{term}}
\newcommand{\pv}{\mathbf{v}}
\newcommand{\rv}{\mathbf{r}}

\newcommand{\interval}{\mathfrak{I}}

\newcommand{\typeReal}{\textbf{\textsf{R}}}

\newcommand{\symbolInt}{\myint{\cdot}}

\newcommand{\LambdaInterval}{\Lambda_{\interval}}
\newcommand{\LambdaSymbolic}{\Lambda_{\text{sym}}}

\newcommand{\toIntervalTerm}[1]{#1^{2\interval}}

%Others
\newcommand{\Sset}{\mathbb{S}}
\newcommand{\Iset}{\mathbb{I}}
\newcommand{\Rset}{\mathbb{R}}
\newcommand{\Nset}{\mathbb{N}}
\newcommand{\Zset}{\mathbb{Z}}

\newcommand{\Term}{\mathbb{T}}
\newcommand{\prob}{\mathbb{P}}
\newcommand{\expt}{\mathbb{E}}


\newcommand{\Leb}{\tt{Leb}}
\newcommand{\Red}{\tt{Red}}
\newcommand{\cost}{\text{cost}}

%\newcommand{\intervalab}[2]{\underline{[#1,#2]}}
\newcommand{\intervalab}{\underline{[a,b]}}
\newcommand{\interI}{\mathcal{I}}
\newcommand{\trans}{\mathcal{T}}

\newcommand{\iv}{\mathbb{I}}

% Programming language constructs
\newcommand{\lit}[1]{\underline{#1}}
\newcommand{\letIn}[1]{\mathsf{let}\,{#1}\,\mathsf{in}\,}
\newcommand{\fixLam}[2]{\mu {#1} {#2}.}
\newcommand{\ifElse}[3]{\mathsf{if} (#1 \le \num{0}) \, {#2} \,\mathsf{else}\, {#3}}

%%Basic notions
\newcommand{\pspace}{(\Omega,\mathcal{F},\probm)}
\newcommand{\probm}{\mathbb{P}}
\newcommand{\condexpv}[2]{{\expt}{\left[{#1} \mid {#2}\right]}}

\newcommand{\stdConf}[1]{(#1)}
%\newcommand{\intConf}[1]{\lceil#1\rceil}
%\newcommand{\intConf}[1]{(#1)}
%\newcommand{\symConf}[1]{\langle\!\langle  #1 \rangle\!\rangle}
%\newcommand\symPath[1]{(#1)}
\newcommand{\symPath}[1]{\langle\!\langle  #1 \rangle\!\rangle}
\newcommand\symConf[1]{(#1)}

\newcommand{\ifSimple}[3]{\mathsf{if}(#1, #2, #3)}
%\newcommand{\ifElse}[3]{\mathsf{if} (#1 \le 0) \, \allowbreak {#2} \, \allowbreak \mathsf{else}\, {#3}}
%\newcommand{\ifElse}[3]{\ifSimple{#1}{#2}{#3}}

%\newcommand{\trace}{\mathsf{s}}
%
%\newcommand\defn[1]{{\bf \em #1}}
\newcommand{\traces}{\mathbb{T}}
%
%\newcommand{\stdConf}[1]{(#1)}
%%\newcommand{\intConf}[1]{\lceil#1\rceil}
%\newcommand{\intConf}[1]{(#1)}
%%\newcommand{\symConf}[1]{\langle\!\langle  #1 \rangle\!\rangle}
%%\newcommand\symPath[1]{(#1)}
%\newcommand{\symPath}[1]{\langle\!\langle  #1 \rangle\!\rangle}
%\newcommand\symConf[1]{(#1)}

\newcommand{\valueSem}[1]{\mathsf{val}_{#1}} % value (semantics)
\newcommand{\weightSem}[1]{\mathsf{wt}_{#1}} % weight (semantics)
\newcommand{\measureSem}[1]{\llbracket #1 \rrbracket}
\newcommand{\posterior}{\mathsf{posterior}}


%%%%%%%%%
% 
%%%%%%%%
\newcommand{\loc}{\ell}
\newcommand{\locs}{\mathit{L}}
\newcommand{\blocs}{\mathit{L}_{\mathrm{b}}}

\newcommand{\iflocs}{\mathit{L}_{\mathrm{if}}}
\newcommand{\looplocs}{\mathit{L}_{\mathrm{while}}}

\newcommand{\alocs}{\mathit{L}_{\mathrm{a}}}
\newcommand{\wlocs}{\mathit{L}_{\mathrm{w}}}
\newcommand{\rlocs}{\mathit{L}_{\mathrm{r}}}
\newcommand{\Alocs}[1]{\mathit{L}_{\mathrm{A}}^{\mathsf{#1}}}
\newcommand{\Dlocs}{\mathit{L}_{\mathrm{nd}}}
\newcommand{\transitions}{{\rightarrow}}

%%% 
\newcommand{\plocs}{\mathit{L}_{\mathrm{p}}}
\newcommand{\tlocs}{\mathit{L}_{\mathrm{t}}}

\newcommand{\lin}{\loc_\mathrm{init}}
\newcommand{\lout}{\loc_\mathrm{out}}
\newcommand{\val}[1]{\mbox{\sl Val}_{#1}}

\newcommand{\pvars}{V_\mathrm{p}}
\newcommand{\rvars}{V_{\mathrm{r}}}
\newcommand{\pre}{\mathrm{pre}}

\newcommand{\sle}{\sqsubseteq}
\newcommand{\sge}{\sqsupseteq}

\newcommand{\lfp}{\mathrm{lfp}}
\newcommand{\gfp}{\mathrm{gfp}}

\newcommand{\rdvarjdis}{\mathcal D}
\newcommand{\sampset}{\textit{supp}}

\newcommand{\upd}{\mbox{\sl upd}}
\newcommand{\wet}{\mbox{\sl wt}}
\newcommand{\transset}{\mathfrak T}
\newcommand{\valin}{\pv_{\mathrm{init}}}
\newcommand{\ret}{\mbox{\sl ret}}

\newcommand{\win}{w_{\mathrm{init}}}

\newcommand{\sampdpd}{\overline{\Upsilon}}

\newcommand{\outmap}{\text{O}}
\newcommand{\sat}[1]{\langle #1 \rangle}
\newcommand{\monoid}{\mbox{\sl Monoid}}
\newcommand{\handelmanformat}{(\dagger)}

\newcommand{\trunc}{\mathcal{B}}

\newcommand{\ewt}{\mbox{\sl ewt}}
\newcommand{\statemap}{\text{St}}

\newcommand{\valrd}{{\mathbf{r}}}
\newcommand{\frmloc}{\ell^{\mathrm{src}}}
\newcommand{\toloc}{\ell^{\mathrm{dst}}}

\newcommand{\monomials}{\mathbf{M}}
\begin{document}
\title{Reducing Latency of DAG-based Consensus in the Asynchronous Setting via the UTXO Model}



% author names and affiliations
% use a multiple column layout for up to three different
% affiliations
\author{\IEEEauthorblockN{Keyang Liu}
\IEEEauthorblockA{The University of Tokyo\\
stephenkobylky2022@g.ecc.u-tokyo.ac.jp}
\and
\IEEEauthorblockN{Maxim Jourenko}
\IEEEauthorblockA{Tokyo Institute of Technology\\
jourenko.m.ab@m.titech.ac.jp}
\and
\IEEEauthorblockN{Mario Larangeira}
\IEEEauthorblockA{Tokyo Institute of Technology/IOG\\
mario@c.titech.ac.jp}}

% conference papers do not typically use \thanks and this command
% is locked out in conference mode. If really needed, such as for
% the acknowledgment of grants, issue a \IEEEoverridecommandlockouts
% after \documentclass

% for over three affiliations, or if they all won't fit within the width
% of the page, use this alternative format:
%
%\author{\IEEEauthorblockN{Michael Shell\IEEEauthorrefmark{1},
%Homer Simpson\IEEEauthorrefmark{2},
%James Kirk\IEEEauthorrefmark{3},
%Montgomery Scott\IEEEauthorrefmark{3} and
%Eldon Tyrell\IEEEauthorrefmark{4}}
%\IEEEauthorblockA{\IEEEauthorrefmark{1}School of Electrical and Computer Engineering\\
%Georgia Institute of Technology,
%Atlanta, Georgia 30332--0250\\ Email: see http://www.michaelshell.org/contact.html}
%\IEEEauthorblockA{\IEEEauthorrefmark{2}Twentieth Century Fox, Springfield, USA\\
%Email: homer@thesimpsons.com}
%\IEEEauthorblockA{\IEEEauthorrefmark{3}Starfleet Academy, San Francisco, California 96678-2391\\
%Telephone: (800) 555--1212, Fax: (888) 555--1212}
%\IEEEauthorblockA{\IEEEauthorrefmark{4}Tyrell Inc., 123 Replicant Street, Los Angeles, California 90210--4321}}



\IEEEoverridecommandlockouts
%\makeatletter\def\@IEEEpubidpullup{6.5\baselineskip}\makeatother
%\IEEEpubid{\parbox{\columnwidth}{
%    Network and Distributed Systems Security (NDSS) Symposium 2021\\
%    21-24 February 2021, San Diego, CA, USA\\
%    ISBN 1-891562-66-5\\
%    https://dx.doi.org/10.14722/ndss.2021.23xxx\\
%    www.ndss-symposium.org
%}
%\hspace{\columnsep}\makebox[\columnwidth]{}}


% make the title area
\maketitle



\begin{abstract}
%\boldmath
DAG-based consensus has attracted significant interest due to its high throughput in asynchronous network settings. However, existing protocols such as DAG-rider (Keidar et al., PODC 2021) and ``Narwhal and Tusk'' (Danezis et al., Eurosys 2022) face two undesired practical issues: (1) high transaction latency and (2) high cost to verify transaction outcomes.

To address (1), this work introduces a novel commit rule based on the Unspent Transaction Output (UTXO) Data Model, which allows a node to predict the transaction results before triggering the commitment. We propose a new consensus algorithm named ``Board and Clerk'', which reduces the transaction latency by half for roughly 50\% of transactions. As the tolerance for faults escalates, more transactions can partake in this latency reduction.

In addition, we also propose the Hyper-Block Model with two flexible proposing strategies to tackle (2): blocking and non-blocking. Using our proposed strategies, each node first predicts the transaction results if its proposal is committed and packs this result as a commitment in its proposal. The hyper-block packs the signature of the proposal and the outputs of the consensus layer together in order to prove the transaction results.
\end{abstract}

% IEEEtran.cls defaults to using nonbold math in the Abstract.
% This preserves the distinction between vectors and scalars. However,
% if the conference you are submitting to favors bold math in the abstract,
% then you can use LaTeX's standard command \boldmath at the very start
% of the abstract to achieve this. Many IEEE journals/conferences frown on
% math in the abstract anyway.

\begin{IEEEkeywords}
Consensus, Byzantine Fault Tolerant, UTXO. 
\end{IEEEkeywords}



% For peer review papers, you can put extra information on the cover
% page as needed:
% \ifCLASSOPTIONpeerreview
% \begin{center} \bfseries EDICS Category: 3-BBND \end{center}
% \fi
%
% For peerreview papers, this IEEEtran command inserts a page break, and
% creates the second title. It will be ignored for other modes.
%%\IEEEpeerreviewmaketitle



\section{Introduction}
    \section{Introduction}
Current quantum hardware is unable to carry out universal quantum computations due to the buildup of errors that occur during the computation. 
The magnitude of the individual error is currently above the value that the Threshold Theorem requires in order to kick-start quantum error correction and fault-tolerant quantum computation~\cite[Section 10.6]{nielsen_chuang_2010}. 
Although the experimentally achieved fidelity rates are promising and the error bounds are inching closer to the required threshold, we will have to work for the foreseeable future with quantum hardware with errors that build-up during the computation.  This implies that we can only do a limited number of steps before the output of the computation has become completely uncorrelated with the intended one.

For fault-tolerant quantum computing, we repeat four steps: 
1) We apply a number of single and two-qubit quantum gates, in parallel whenever possible; 
2) We perform a syndrome measurement on a subset of the qubits; 
3) We perform fast classical computations to determine which errors have occurred and how to correct them; 
and, 4) We apply correction terms based on the classical computations.
We then repeat these four steps with a next sequence of gates. 
These four steps are essential to fault-tolerant quantum computing. 


The starting point of this work is to use the four steps outlined above, not to carry out error correction and fault-tolerant computation, but to enhance short, constant-depth, {\em uncorrected} quantum circuits that perform single qubit gates and {\em nearest-neighbor} two qubit gates. 
Since in the long run we will have to implement error-correction and fault-tolerant computation anyhow, and this is done by such a four-step process, why not make other use of this architecture? Moreover, on some of the quantum hardware platforms, these operations are already in place.
Embracing this idea we naturally arrive at the question: what is the computational power of \textit{low-depth} quantum-classical circuits organized as in the four steps outlined above? 
We thus investigate circuits that execute a small, ideally constant, number of stages, where at each stage we may apply, in parallel, single qubit gates and {\em nearest-neighbor} two qubit gates, followed by measurements, followed by low-depth classical computations of which the outcome can control quantum gates in later stages. 
It is not clear, at first, whether such circuits, especially with constant depth, can do anything remotely useful. 
But we will see that this is indeed the case: many quantum computations can be done by such circuits in constant depth. 
By parallelizing quantum computations in this way, we improve the overall computational capabilities of these circuits, as we do not incur errors on qubits that are idle, simply because qubits are not idle for a very long time. 
Furthermore, reducing the depth of quantum circuits, at the cost of increasing width, allows the circuit to be run faster even if errors occur.

The first usage of such a four-step layout, not to do error correction, but to perform computations, can be found in the paradigm of measurement-based quantum computing~\cite{gottesman1999demonstrating,raussendorf2001one,jozsa2006introduction,clark2007generalised}: 
A universal form of quantum computing where a quantum state is prepared and operations are performed by measuring qubits in different bases, depending on previous measurements and intermediate measurements.

\citeauthor{PhamSvore2013} were the first to formalize the four-step protocol for performing computations~\cite{PhamSvore2013}. They included specific hardware topologies by considering two-dimensional graphs for imposing constraints on qubit interactions. In their model, they develop circuits for particularly useful multi-qubit gates, including specifying costs in the width, number of qubits, depth, number of concurrent time steps, size, and total number of non-Identity operations.
As a result, they find an algorithm that factors integers in polylogarithmic depth.
\citeauthor{Browne:2011} showed that the main tool in the work by \citeauthor{PhamSvore2013}, the fan-out gate, can also be replaced by additional log-depth classical computations in the measurement-based quantum computing setting~\cite{Browne:2011}.

More recently, \citeauthor{Cirac:2021} introduced a scheme to implement unitary operations involving quantum circuits combined with Local Operations and Classical Communication ($\mathsf{LOCC}$) channels: $\mathsf{LOCC}$-assisted quantum circuits~\cite{Cirac:2021}. Similarly to the four-step scheme we just described, they allow for a short depth circuit to be run on the qubits, followed by one round of $\mathsf{LOCC}$, in which ancilla qubits are measured and local unitaries are applied based on the measurement outcomes. They show that in this model any 1D transitionally invariant matrix-product state (MPS) with fixed bond dimension is in the same phase of matter as the trivial state. Similar ideas can be found in~\cite{TVV_NonAbelianTopologicalOrder_2022, tantivasadakarn2021long}.

In this work, we introduce a new model, called \textit{Local Alternating Quantum-Classical Computations} ($\LAQCC$). In this model we alternate between running quantum circuits (constrained by locality), ending in the measurement of a subset of qubits, and fast classical computations based on the measurement results. The outcome of the classical computations are then used to control future quantum circuits. We allow for flexibility in this model, by giving different constraints to the power of both the quantum circuits and the classical circuits as well as the number of alternations between them. 
Most attention will be given to $\LAQCC$ containing quantum circuits of constant depth, classical circuits of logarithmic depth and at most a constant number of alternations between them. 
Any circuit constructed in this model is considered to be of constant depth. 
We restrict ourselves to logarithmic depth classical computations, as this is the first natural and non-trivial extension beyond constant-depth classical computations. 
Constant-depth classical computations do however also have an equivalent constant-depth quantum implementation.

The definition of $\LAQCC$ sharpens the original definition of \citeauthor{PhamSvore2013} by adding constraints to the intermediate classical computations. This allows us to bound the power of $\LAQCC$ from above. 

The main result of \citeauthor{Cirac:2021}, that 1D translational invariant MPS with fixed bond dimension can be prepared by $\mathsf{LOCC}$-assisted circuits, relies on local symmetries of the MPS. These symmetries allow them to prepare local states (on a constant number of qubits) and glue them together by doing one round of the appropriate entangling measurement and corrections, after which they run a round of local unitaries to get the desired result. This general scheme for preparing states that exhibit an MPS description with the appropriate local symmetries requires only geometrically local unitaries and one round of measurement and corrections an therefore is accessible in $\LAQCC$. Studying different local symmetries, known as Symmetry Protected Topological (SPT) phases of matter, to find measurement-based constant depth circuits for states is a broad ongoing field of research~\cite{TVV_NonAbelianTopologicalOrder_2022, tantivasadakarn2021long, smith2023deterministic}. 
All these schemes have a $\LAQCC$ implementation.

%$\LAQCC$-circuits also exist for general schemes of preparing local states, based on the local tensors, and gluing them together using one round of entangled measurement and corrections, based on the local symmetry. 
%The main result of \citeauthor{Cirac:2021}, that 1D translational invariant MPS with fixed bond dimension can be prepared by $\mathsf{LOCC}$-assisted circuits, relies heavily on local symmetries of the MPS and as a result also has an equivalent $\LAQCC$ implementation. 
%The corrections applied after the measurement round are local unitaries depending on the local symmetries of the MPS. 

 

%This general scheme of preparing local states, based on the local tensors, and gluing it together by doing one round of entangled measurement and corrections, based on the local symmetry, is accessible in $\LAQCC$.
Note however that \citeauthor{Cirac:2021} also suggest a circuit for the $W$-state.
This circuit uses sequentially and dependent measurement-based corrections of the ancilla qubits. 
These dependent measurements translate to sequential alternations between the quantum and classical circuits and therefore increase the total depth to linear depth, exceeding the constant-depth constraints imposed by $\LAQCC$-circuits. 

We study the power of the $\LAQCC$ model with respect to state preparation, showing that even with only constant quantum-depth and logarithmic classical depth it remains possible to prepare states with long-range entanglement.
Another surprising result is that it is unlikely that $\LAQCC$ circuits are classically simulatable. We show that any instantaneous quantum polynomial-time (IQP) circuit~\cite{Bremner2010,Shepherd2009} has an $\LAQCC$ implementation.
Classical simulation of IQP circuits implies the collapse of the polynomial hierarchy to the third level, which is not believed to be true~\cite{Bremner2017}. Therefore, we expect that $\LAQCC$ circuits are unlikely to be classically simulatable. We bound the power of $\LAQCC$ by showing that it is contained in $\QNC^1$, the class of polynomial-size, log-depth circuits.

Next, we also study the power that intermediate classical calculations can add to quantum computations, by considering a new model that alternates between polynomially many polynomial-depth quantum circuits and unbounded classical computations
We study this model by doing a complexity theoretical analysis, where we draw inspiration from the notions of complexity given by \citeauthor{RosenthalYuen:2022}, \citeauthor{MetgerYuen:2023}, and \citeauthor{Aaronson:2004}.
All three complexity notions are based on the notion of state preparation, instead of more traditional definition of complexity such as the decidability of a computational problem. 
The first two consider classes based on sequences of quantum states preparable by a polynomial-sized quantum circuit, where the circuits are uniformly generated by a computational class, for instance, the class $\mathsf{PSPACE}$, which results in the complexity class $\mathsf{StatePSPACE}$~\cite{RosenthalYuen:2022,MetgerYuen:2023}.
The third notion considers a relative complexity, where the complexity is measured between two given states, and is measured by the number of gates, from a given gate-set, required to transform one state in another state~\cite{Aaronson:2004}. 
For our definition of state preparation complexity, we drop the uniformity constraint from~\cite{RosenthalYuen:2022,MetgerYuen:2023} and define a class as $\mathsf{StateX}$, which refers to states preparable by circuits of type $\mathsf{X}$. 
As an example, if $\mathsf{X} = \QNC^0$, this results in the class $\mathsf{StateQNC^0}$, which is the set of states preparable from the $\ket{0}^n$ state by poly-size constant-depth circuits. 
This notion is similar to the relative complexity from~\cite{Aaronson:2004}, where one state is the  $\ket{0}^n$ state and instead of counting the number of gates we consider the set of states preparable by a fixed number of gates. Using this notion of complexity we show that any state preparable by an $\LAQCC^*$ circuit is also preparable by a $\mathsf{PostQPoly}$ circuit, the class of circuits of polynomial depth with an additional post-selection gate. 

All Clifford circuits have a constant-depth $\LAQCC$ implementation, implying that any stabilizer state can be implemented by a constant-depth $\LAQCC$ circuit, see Section~\ref{sec:clifford_circuits} for a proof of this statement. 
Efficient circuits for stabilizer states have been known already through measurement-based quantum computing. Therefore this paper focuses on the preparation of non-stabilizer states, and as a surprising result we find novel constant-depth protocols for four very natural classes of non-stabilizer states.
Despite the extensive research into these four classes of non-stabilizer states and the many applications of them, no efficient constant- or low-depth state preparation protocols are known yet. We specifically consider these four classes as they are all often used as initial states in other algorithms.

The first state is a uniform superposition over an arbitrary number of states. 
This state finds applications in many quantum algorithms, as they often start with a uniform superposition over multiple states. 
This superposition is often achieved by applying Hadamard gates to every qubit due to its simplicity to prepare. 
Yet, the analysis of many algorithms, such as Shor's algorithm~\cite{Shor:1997}, would benefit from a different initial superposition. 
The circuit to prepare the uniform superposition over an arbitrary number of states uses an exact version of Grover search as a subroutine, that turns a probabilistic circuit, with a known constant probability of success, into a deterministic circuit. 
We use the circuit for preparing a uniform superposition over an arbitrary number of states as a subroutine in the next two quantum state preparation protocols. 

The second state is the $W$-state, the uniform superposition over all computational basis states of Hamming-weight~$1$, a natural long-ranged entangled state that displays a fundamentally nonequivalent type of entanglement from the Greenberger–Horne–Zeilinger state~\cite{WState:2000}, for which $\LAQCC$-type constant-depth circuits were previously known~\cite{PhamSvore2013, Cirac:2021}. 
The $W$-state is often used as benchmark for new quantum hardware~\cite{Haffner2005,Neeley2010,GarciaPerez:2021}. 
A novel way to prepare the $W$-state therefore gives a new way to benchmark different quantum devices with each other. 
A circuit for preparing the $W$-state was given in~\cite{Cirac:2021}, but this implementation requires sequentially alternating measurements followed by local unitaries, which in the $\LAQCC$ model is not considered to be of constant depth. 
We improve this protocol by giving an $\LAQCC$ implementation of the $W$-state, based on a compress-uncompress method that links the one-hot and binary encoding of integers.

The third state considered is the Dicke state, a generalization of the $W$-state, a superposition over all computational basis states with Hamming-weight $k$~\cite{Dicke:1954}. 
Dicke states have relevance in various practical settings.
For instance, for quantum game theory~\cite{zdemir2007}, quantum storage~\cite{Bacon_Compress:2006,Plesch:2010}, quantum error correction~\cite{ouyang2014permutation}, quantum metrology~\cite{toth2012multipartite}, and quantum networking~\cite{prevedel2009experimental}. 
Dicke states have been used as a starting state for variational optimization algorithms, most notably Quantum Alternating Operator Ansatz (QAOA)~\cite{Hadfield2019}, to find solutions to problems such as Maximum k-vertex Cover~\cite{Brandhofer2022,cook2020quantum}.
The ground states of physical Hamiltonians describing one-dimensional chains tend to show a resemblance to Dicke states such as states resulting from the Bethe ansatz, making them an ideal starting state when investigating the ground state behavior of these Hamiltonians~\cite{TDL_BetheAnsatzDerivation:2010,B_ExcitedStateQuantumPhaseTransitions:2013,DickeTransitions:2021}. 
For instance, the algorithm by \citeauthor{van2021preparing}, who give an algorithm to prepare the Bethe ansatz eigenstates of the spin-1/2 XXZ spin chain, starts by first preparing a Dicke state~\cite{van2021preparing}. 
A Dicke-state preparation protocol based on the compress-uncompress methodology used in the $W$-state furthermore finds applications in entanglement distillation, where the entanglement of a large state is concentrated on only a few qubits. 
Efficient deterministic circuits for preparing Dicke states have been proposed by \citeauthor{bartschi2019deterministic}~\cite{bartschi2019deterministic, bartschi2022deterministic_short_depth}. 
They provide a quantum circuit of depth $\mathO(k \log(\frac{n}{k}))$, allowing arbitrary connectivity, to prepare a Dicke state, which they conjecture to be optimal when $k$ is constant. 
In this work, we provide a constant-depth $\LAQCC$ circuit below their conjectured bound already for constant $k$. 
However, this does not directly disprove their conjecture, as we allow for intermediate measurements and classical computations. 
More significantly, we even construct constant-depth $\LAQCC$ circuits for $k = \mathO(\sqrt{n})$ greatly improving their bound.
This construction extends the compress-uncompress method for the $W$-state combined with additional subroutines. 

We continue with a log-depth state preparation protocol for the Dicke-state for arbitrary $k$. 
This protocol implements an efficient transformation between the factoradic number representation and the combinatorial number representation of a positive integer. 
The combinatorial number representation relates directly to the Dicke state. 
The provided efficient transformation between number representation systems might be of independent interest. 

We conclude by modifying our protocol for preparing a Dicke-state to a protocol that prepares quantum many-body scar states in constant-depth. 
These states have low entanglement and longer coherence times than states with similar energy density.
These characteristics make many-body scar states interesting to analyze and relevant within physics.
Many-body scar states appear for instance in the AKLT model~\cite{AKLT:1987,MRBAR:2018,MRB:2018} and different spin models~\cite{SI:2019,MOBFR:2020}.
Known methods for preparing these states have polynomial-depth~\cite{Gustafson:2023}, whereas our circuit has constant depth. 

% We conclude by studying the power that intermediate classical calculations can add to quantum computations. 
% In this study, we define a new model that relaxes constant-depth quantum circuits to polynomial depth quantum circuits, log-depth classical calculations to unbounded classical computations and a constant number of alternations to a polynomial number of alternations. 
% We call this model $\LAQCC^*$. 
% We study this model by doing a complexity theoretical analysis, where we draw inspiration from the notions of complexity given by \citeauthor{RosenthalYuen:2022}, \citeauthor{MetgerYuen:2023}, and \citeauthor{Aaronson:2004}.
% All three complexity notions are based on the notion of state preparation, instead of more traditional definition of complexity such as the decidability of a computational problem. 
% The first two consider classes based on sequences of quantum states preparable by a polynomial-sized quantum circuit, where the circuits are uniformly generated by a computational class, for instance, the class $\mathsf{PSPACE}$, which results in the complexity class $\mathsf{StatePSPACE}$~\cite{RosenthalYuen:2022,MetgerYuen:2023}.
% The third notion considers a relative complexity, where the complexity is measured between two given states, and is measured by the number of gates, from a given gate-set, required to transform one state in another state~\cite{Aaronson:2004}. 
% For our definition of state preparation complexity, we drop the uniformity constraint from~\cite{RosenthalYuen:2022,MetgerYuen:2023} and define a class as $\mathsf{StateX}$, which refers to states preparable by circuits of type $\mathsf{X}$. 
% As an example, if $\mathsf{X} = \QNC^0$, this results in the class $\mathsf{StateQNC^0}$, which is the set of states preparable from the $\ket{0}^n$ state by poly-size constant-depth circuits. 
% This notion is similar to the relative complexity from~\cite{Aaronson:2004}, where one state is the  $\ket{0}^n$ state and instead of counting the number of gates we consider the set of states preparable by a fixed number of gates. Using this notion of complexity we show that any state preparable by an $\LAQCC^*$ circuit is also preparable by a $\mathsf{PostQPoly}$ circuit, the class of circuits of polynomial depth with an additional post-selection gate. 

\paragraph{Summary of results}
\begin{itemize}
    \item We give a new definition of a computational model that captures the power of the four step process: applying a constant number of layers of one- and two-qubit gates; performing a syndrome measurement; perform a fast classical computation determining corrections; apply corrections. We call this model \emph{Local Alternating Quantum Classical Computations}, or $\LAQCC$ for short. In this model we bound the allowed quantum operations, intermediate classical calculations, and number of rounds separately. In Section~\ref{sec:LAQCC_model} we define this model and give a list of operations based on results from literature contained in this computational model. In some of these operations we explicitly use that we allow for multiple, but at most constant, rounds  of corrections.
    \item  We show show that there exist $\LAQCC$ circuits that can not be weakly simulated in Section~\ref{sec:IQP_in_LAQCC}. We further show that for every $\LAQCC$ circuit there exists a $\QNC^1$ circuit simulating it perfectly, in Section~\ref{sec:LAQCC_in_QNC1}.
    \item We introduce a new type computational complexity for preparing states and show that the extension of $\LAQCC$ where we allow a polynomial number of rounds and unbounded classical computation, is contained in $\mathsf{PostQPoly}$, the class of polynomial circuits with post-selection, in Section~\ref{sec:Complexity results}.
    \item We show a protocol to prepare the uniform superposition state of size $q$ in $\LAQCC$ using $\mathO(\ceil{\log_2(q)}^2)$ qubits in Section~\ref{sec:superposition_modulo_q}. 
    \item We show a protocol to prepare the $W_n$ state in $\LAQCC$ using $\mathO(n\log(n))$ qubits in Section~\ref{sec:W_state_in_LAQCC}.
    \item We show two ways of preparing the Dicke-$(n,k)$ state. The first method is in $\LAQCC$, works up to $k = \mathO(\sqrt{n})$, uses $\mathO(n^2\log(n))$ qubits, and is found in Section~\ref{sec:dicke:small_k}. The second method is in $\LAQCC\text{-}\mathsf{LOG}$ (an extension of $\LAQCC$ allowing for logarithmic number of alterations instead of constant), works for any $k$, uses $\mathO(\text{poly}(n))$ qubits, and is found in Section~\ref{sec:Dicke_in_LAQCC_LOG}. 
    \item We extend on our $\LAQCC$ method of generating Dicke-$(n,k)$ states for $k = \mathO(\sqrt{n})$ and show a protocol to generate many-body scar states for a particular Hamiltonian in $\LAQCC$ (Section~\ref{sec:many_body_scar}). 
\end{itemize}
Summarized in a table, we provide the following state generation protocols:
\begin{table}[htb]
\centering
\begin{tabular}{l|l|l|l}
\textbf{State description} & \textbf{Width} & \textbf{Depth} & \textbf{Implementation}\\
\hline 
Uniform superposition mod $q$: $\frac{1}{\sqrt{q}} \sum_{i = 0}^{q-1}\ket{i}$ & $\mathO(\ceil{\log^2 q})$ & $\mathO(1)$ & Section~\ref{sec:superposition_modulo_q}\\

$W$-state: $\frac{1}{\sqrt{n}}\sum_{i = 0}^{n-1}\ket{e_i}$ & $\mathO(n \log n)$ & $\mathO(1)$ & Section~\ref{sec:W_state_in_LAQCC}\\

Dicke-$(n,k)$, $k = \mathO(\sqrt{n})$: $\binom{n}{k}^{-1/2}\sum_{x \in \{0,1\}^n: |x| = k} \ket{x}$ &  $\mathO(n^2\log n)$ & $\mathO(1)$ 
&Section~\ref{sec:dicke:small_k}\\

Dicke-$(n,k)$: $\binom{n}{k}^{-1/2}\sum_{x \in \{0,1\}^n: |x| = k} \ket{x}$ & $\mathO(\text{poly}(n))$ & $\mathO(\log n)$ &Section~\ref{sec:Dicke_in_LAQCC_LOG}\\

QMBS: $\ket{S_k} = \frac{1}{k! \sqrt{\mathcal N(n,k)}}(Q^\dagger)^k \ket{\Omega}$ &  $\mathO(n^2\log n)$ & $\mathO(1)$  &  Section~\ref{sec:many_body_scar}
\end{tabular}
\caption{Summary of state preparation protocols given in this paper.}
\label{tab:sate_prep}
\end{table}
In the entry for the quantum many-body scar state $Q$ denotes the raising operator and $\mathcal N(n,k)=\binom{n-k-1}{k}$. 
Section~\ref{sec:many_body_scar} will provide more details on the variables and the implementation. 

\paragraph{Organization of the paper}
\noindent We first introduce relevant preliminaries in Section~\ref{sec:preliminaries}. 
In Section~\ref{sec:LAQCC_model} we formally define the class of Local Alternating Quantum-Classical Computations ($\LAQCC$). We also show that any Clifford circuit can be implemented in constant depth $\LAQCC$ (a result based on a result from measurement-based quantum computing~\cite{jozsa2006introduction}). 
This result allows us to give many useful multi-qubit gates and routines in Section~\ref{sec:gates_created_in_LAQCC}. 
Beyond that we show that constant depth $\LAQCC$ circuits are contained in $\QNC^1$ and that any $\mathsf{IQP}$ circuit has an $\LAQCC$ implementation.
We conclude this section with an analysis of a more powerful instantiation of $\LAQCC$ and show an inclusion with respect to the class $\mathsf{PostQPoly}$, which is the class of circuits of polynomial depth with one additional post-selection gate. 
In Section~\ref{sec:state_prep_in_LAQCC} we give $\LAQCC$ circuit implementations for preparing the uniform superposition over an arbitrary number of states, the $W$-state and the Dicke state up to $k = \mathO(\sqrt{n})$. We furthermore give a log-depth circuit implementation for preparing the Dicke state for any $k$. We conclude by showing a $\LAQCC$ circuit for generating many body scar states of a particular type of Hamiltonian.



\section{Preliminaries} \label{sec:preliminaries}
    \section{Preliminaries}
In this section, we describe the necessary background for automated planning and the significance of the International Planning Competition. 

% \subsection{Ontology}
% A formal ontology is typically represented as a set of concepts, relations, and axioms. A concept represents a set of objects or entities that share common properties, while a relation represents a connection or association between two or more concepts. Axioms are statements that define the relationships between concepts and relations. It is a formal representation of knowledge that is designed to facilitate automated reasoning and information processing. It acts as a structured vocabulary that describes a domain and promotes interoperability, data integration, and communication between humans and machines. Formally, an ontology $O$ can be represented as a tuple $(C, R, A)$, where $C$ is the set of concepts, $R$ is the set of relations, and $A$ is the set of axioms. Each concept \textit{c} $\in$ $C$ can be represented as a set of attributes, denoted as $Att(c)$. Similarly, each relation \textit{r} $\in$ $R$ can be represented as a set of attributes, denoted as $Att(r)$.

% Ontology is a branch of philosophy that deals with the nature of existence and being. In the field of computer science, however, ontology refers to a formal representation of knowledge that is designed to facilitate automated reasoning and information processing. It is a structured vocabulary that describes a domain and promotes interoperability, data integration, and communication between humans and machines. Various tools and methodologies, including Protege and ontology editors, are available for ontology creation. Ontologies are increasingly important in artificial intelligence, knowledge engineering, and the semantic web, and researchers are exploring their potential in diverse domains and applications.

% Figure environment removed

\subsection{Automated Planning}

Automated planning, also known as AI planning, is the process of finding a sequence of actions that will transform an initial state of the world into a desired goal state \cite{ghallab2004automated}. It involves constructing a plan or a sequence of actions that will achieve a specified objective while respecting any constraints or limitations that may be present. Formally, automated planning can be defined as a tuple $(S, A, T, I, G)$, where:
\begin{itemize}
    \item $S$ is the set of possible states of the world
    \item $A$ is the set of possible actions that can be taken
    \item $T$ is the transition function that describes the effects of taking an action on the current state of the world
    \item $I$ is the initial state of the world
    \item $G$ is the desired goal state
\end{itemize}
Using this notation, the problem of automated planning can be framed as finding a sequence of actions $\prec a_1, a_2, ..., a_k\succ$ that will transform the initial state $I$ into the goal state $G$, while respecting any constraints or limitations on the actions. 
 % In automated planning, 
 A problem is defined in terms of a domain and a problem instance. The domain defines the possible actions that can be taken and the effects of each action, while the problem instance specifies the initial state of the world and the desired goal state. 
Various techniques can be used to solve the planning problem, such as search algorithms, constraint-based reasoning, and optimization methods. These techniques involve exploring the space of possible plans and selecting the one that satisfies the objective and any constraints. Figure \ref{fig:planning_bw} illustrates an automated planning scenario for the blocksworld domain, where an initial state can be transformed into a goal state by executing a sequence of actions.

% \noindent \textbf{Attributes modeled about a domain.}
%   %\noindent \textbf{Attributes modeled in a domain file}
%  \begin{enumerate}
%      \item \textbf{Requirements:} A list of requirements that the planner must satisfy in order to solve the domain. Requirements include durative actions, conditional effects, or negative preconditions. For example, in blocksworld domain with types involved, one of the requirements is \emph{typing}.
%     \item \textbf{Predicates:} Predicates are fundamental elements in the planning domain that define the properties of the world. They are used to describe the initial and goal states, as well as the preconditions and effects of actions. Predicates are usually defined as logical expressions over a set of variables, where each variable can take on a finite number of values. In the context of planning, predicates are typically used to represent facts about the world that can be true or false, such as the location of an object or the status of a machine. For example, in blocksworld domain, the predicate \verb|(on b1 b2)| could indicate that block 'b2' is on top of block 'b1'.
%      \item \textbf{Actions:} Actions are the basic units of change in the planning domain. They represent atomic operations that can be performed to transform the world from one state to another. Each action has a name, a set of parameters, preconditions that must be satisfied before the action can be executed, and effects that describe the changes that the action makes to the world. Actions can be used to model a wide variety of operations, ranging from simple movements or transformations to complex processes such as planning or decision-making. For example, in blocksworld domain, the action \verb|unstack b2 b1| can be used to unstack block 'b2' from block 'b1'. 
     
%      \item \textbf{Preconditions:} Preconditions are the conditions that must be true before an action can be executed. They are usually defined using predicates and can involve multiple variables. Preconditions can also be negative, which means that a certain condition must not be true for an action to be executed. In planning, preconditions ensure that actions are only executed when the necessary conditions have been met, such as ensuring that a machine is turned off before it is serviced. For example, in blocksworld domain, the action \verb|unstack b2 b1| has a precondition of \verb|(on b1 b2)|, meaning that for the action to be valid, the block 'b2' should be on top of block 'b1'.
     
%      \item \textbf{Effects:} Effects describe the changes that an action makes to the world. They are usually defined using predicates and can involve multiple variables. Effects can be positive, which means that a certain condition becomes true after the action is executed, or negative, which means that a certain condition becomes false after the action is executed. In the context of planning, effects are used to model the changes that result from executing an action, such as moving an object from one location to another or turning a machine on. For example, in blocksworld domain, when the action \verb|unstack b2 b1| is executed, one of its effect is \verb|(not (on b1 b2))|, indicating that block 'b2' is no longer on top of block 'b1'.
     
%      \item \textbf{Constants:} Constants are values that are fixed and do not change during the execution of the planning problem. They are used to represent objects or entities in the world that have a fixed value, such as the speed limit on a road. Constants can be used to simplify the planning problem by reducing the number of variables that need to be considered and by providing a fixed set of values that can be used in predicates and actions. For example, in blocksworld domain, the constant \emph{table} could represent the surface on which the blocks are initially placed.
     
%      \item \textbf{Types:} Types are used to classify objects or entities in the world based on their attributes or properties. They are used to define the domain of values that a variable can take on and can be used to constrain the values that are assigned to variables. In the context of planning, types are typically used to group related objects or entities together, such as cars or bicycles, and to specify the properties that are common to all members of a type, such as their color or size. For example, in blocksworld domain with types involved, one can represent the predicate as \verb|(on ?x - block ?y - block)| stating that the parameters in the predicate are of type \emph{block}.

%  \end{enumerate}


% ######### Shorter version for AI Planning preliminaries
% \subsection{Automated Planning}

% Automated planning, also known as AI planning, finds actions transforming an initial world state into a goal state \cite{ghallab2004automated}. It involves creating a plan, respecting constraints, defined as $(S, A, T, I, G)$ where $S$ is the world states set, $A$ is the actions set, $T$ is the state transition function, $I$ is the initial state, and $G$ is the goal state. The challenge is to find actions $\prec a_1, a_2, ..., a_k\succ$ converting $I$ to $G$ under constraints. 

% A problem has a domain (defining actions and effects) and an instance (specifying initial and goal states). Various techniques can be used to solve the planning problem, such as search algorithms, constraint-based reasoning, and optimization methods. These techniques involve exploring the space of possible plans and selecting the one that satisfies the objective and any constraints. Figure \ref{fig:planning_bw} illustrates an automated planning scenario for the blocksworld domain, where an initial state can be transformed into a goal state by executing a sequence of actions.

\noindent \textbf{Attributes modeled about a domain.}
 \begin{enumerate}
     \item \textbf{Requirements:} A list of requirements that the planner must satisfy to solve the given domain, e.g., \emph{typing} in blocksworld with types.
     \item \textbf{Predicates:} Define world properties, e.g., \verb|(on b1 b2)| in blocksworld.
     \item \textbf{Actions:} Units of change with preconditions and effects, e.g., \verb|unstack b2 b1| in blocksworld.
     \item \textbf{Preconditions:} Conditions for action execution, e.g., \verb|(on b1 b2)| for \\ \verb|unstack b2 b1|.
     \item \textbf{Effects:} Post-action world changes, e.g., \verb|(not (on b1 b2))| after \\ \verb|unstack b2 b1|.
     \item \textbf{Constants:} Fixed values, e.g., \emph{table} in blocksworld.
     \item \textbf{Types:} Classifications based on attributes, e.g., \\ \verb|(on ?x - block ?y - block)| in typed blocksworld.
 \end{enumerate}

\noindent \textbf{Attributes modeled about a problem instance from a domain.}
\begin{enumerate}
    \item \textbf{Name:} The name of the planning problem.
    \item \textbf{Domain:} The name of the planning domain that the problem belongs to.
    \item \textbf{Objects:} A list of objects that are present in the planning problem. Objects are typically defined in terms of their type and name. In the example shown in Figure \ref{fig:planning_bw}, objects are b1, b2, and b3.
    \item \textbf{Initial State:} A description of the initial state of the world, including the values of all relevant predicates. Figure \ref{fig:planning_bw} represents an example initial state.
    \item \textbf{Goal State:} A description of the desired goal state of the world, including the values of all relevant predicates. Figure \ref{fig:planning_bw} represents an example goal state.
\end{enumerate}

% \vspace{2cm}
\subsection{International Planning Competition (IPC)}

% IPC serves as a significant means of assessing and comparing various planning systems. By presenting new planners and benchmark problems each year, the competitions aim to stimulate the advancement of new planning methodologies and reflect current trends and challenges in the field. The competition comprises multiple tracks, each covering various planning problems such as classical, temporal, and probabilistic planning. These tracks include benchmark problems that evaluate the performance of planners concerning parameters such as plan quality, plan length, and run time. The results of these competitions provide insights into the current state-of-the-art in planning and help identify the strengths and weaknesses of different planning systems. IPC can serve as an excellent starting point for building a planning-related ontology as the benchmark problems used in these competitions can provide a comprehensive overview of the domain and the types of problems that planners need to solve. 

IPC is pivotal for evaluating and contrasting planning systems. Introducing new planners and benchmarks, it promotes innovative planning methodologies and reflects the field's evolving challenges. The competition has multiple tracks, such as classical and probabilistic planning, with benchmarks assessing plan quality, length, and run time. IPC results offer a glimpse into the latest in planning, highlighting system pros and cons. The benchmarks from IPC are ideal for crafting a planning-related ontology, encapsulating the domain's breadth and planners' challenges.


\section{Board and Clerk}\label{sec:txorder}
   Before describing our protocol, we present an initial intuition regarding the interplay between the DAG data structure and the UTXO Model.


\subsection{Intuition: DAG meets UTXO}

Our first proposal is a new transaction commit rule that compares the ordering decided by the random leader in Tusk. The UTXO Model is designed to support parallelism, so we do not need a strict sequential order of all transactions. Considering that workers can ensure internal validity beforehand, the consensus algorithm only needs to identify double-spending transactions and agree on which transactions are invalid. 

Currently, the existing DAG-based consensus algorithms rely on deterministic DAG traversal algorithms to determine the transaction order and the external validity. However, the proposer of the leader proposal often decides the order almost randomly based on how it arranged the proposals from the previous round. This approach requires significant work to determine the transaction's external validity. 

Therefore, we propose a new commitment rule that does not rely on the order of transactions in the batches, making the results of transactions more predictable before their commitment. Consequently, any node can be confident about the results of most transactions (except double-spending transactions) after a certain number of rounds, which enables them to process the following transactions and, ultimately, reduce the actual latency of transaction commitment. 


\subsection{Transaction States}

We use the transaction state to represent its lifecycle in memory. The states can be:

\begin{itemize}
    \item \textbf{Verified}: A received transaction that has passed the internal validity check by a worker;
    
    \item \textbf{Submitted}: A verified transaction included in at least one single proposal;
    
    \item \textbf{Fast-committed}:  A submitted transaction that will be committed in the future with a confirmed result;
    
    \item \textbf{Committed}: A submitted transaction that the consensus algorithm has committed.
\end{itemize}

We use the fast-committed state to process transactions and apply changes to the state before committing them via the consensus algorithm. The fast-committed transactions use a shorter internal path to execute and enable the node to verify its following transactions. This feature is especially advantageous when the consensus algorithm takes a long time to select appropriate leaders for committing the DAG. 

\subsection{The Novel Commit Rule}

A local node must determine whether a transaction will succeed or fail to enable a fast-committed channel. Therefore, a procedure to decide the external validity  of a transaction is required to reduce uncertainty when extending the DAG. Hence we have made the following two changes to the proposal flow compared to Narwhal and Tusk:
\begin{enumerate}
    \item We require that any valid batch of transactions does not contain conflicting transactions. A batch is a wrapper of transactions that are submitted through a worker. It is reasonable to assume that the worker has verified all transactions in the batch, ensuring that transactions spending the same UTXO are not kept. It should be noted that different batches may have conflicting transactions since they may come from different workers.

    \item We require that a node's proposal for round $r$ always includes its proposal for round $r-1$. This requirement ensures the basic consistency of the node's proposal. If a node's proposal at round $r$ is committed, so is its proposal at round $r-1$. This significantly improves the reliability of the projection of the external validity decision.
\end{enumerate}
Now, we provide some definitions describing our approach to determine external validity.

\begin{definition}
For a sub-DAG that begins with a node's proposal at round $r$ containing a transaction $tx$ for the first time, we say that the node votes for $tx$ at round $r$.
\end{definition}

A node generally votes for a transaction in one of two cases: (1) its proposal includes the transaction, or (2) its proposal links to a proposal from another node that already voted for the transaction. The vote is only kept for the minimum round. To be noticed, a node may vote for conflicting transactions at the same round; in this case, its vote is counted for both transactions. Figure~\ref{fig_TX} shows some typical cases for voting. 

% Figure environment removed

\begin{definition}~\label{def:ord}
For any set of transactions, we define an ordering relationship $Ord$ such that there is only one stable sort of the set under $Ord$.
\label{def_tx_ord}
\end{definition}
A straightforward example is ordering according to the hexadecimal representation of transaction IDs.

\begin{definition}[External Validity-Commit Rule]
For all transactions inside a sub-DAG to be committed by the consensus algorithm at even rounds, a transaction is successful only if more nodes have voted for it before any other transactions conflict. In other words, let $tx.I$ be the inputs of $tx$ and $\mathsf{TX_S}$ be the set of all submitted transactions. We commit $tx$ if:

%$$\forall tx'\; s.t.\; tx'.I \cap tx. I \neq \phi.$$
$$\forall tx' \in \mathsf{TX_S} \; : \; tx'.I \cap tx.I = \varnothing.$$
Furthermore, it is also committed if
\begin{equation}
    |\{\text{nodes vote tx first}\}| > |\{\text{nodes vote }tx'\text{ first}\}|,\nonumber
\end{equation}
or otherwise if
\begin{equation}
\begin{split}
  & tx >_{Ord} tx', \\
  & |\{\text{nodes vote tx first}\}| = |\{\text{nodes vote tx' first}\}|.\nonumber \\
\end{split}
\end{equation}
\label{def_tx_rule}
\end{definition}

%Given a sub-DAG, 
Definition~\ref{def_tx_rule} allows for a deterministic outcome of the transaction's external validity. Transactions without conflicts are committed directly. For transactions with conflicts, we commit according to the votes from nodes, and any tie will be resolved by the deterministic order $Ord$ (given by Definition~\ref{def:ord}).

Compared to~\cite{Danezis_Kokoris-Kogias_Sonnino_Spiegelman_2022}, we only change the commit rule. Hence our protocol inherits the security claim of Tusk. Intuitively, we did not change the consensus rule at the proposal level, eventually committing the same proposals as Tusk. Since Tusk guarantees that all honest participants commit the same sequence of leaders, the underlying transaction sets and votes are consistent. Therefore, our commit rule does not affect the consistency of the consensus outputs. 

However, this new commit rule enables the node to pre-compute the votes for each transaction before the consensus decides the leader. We present Theorem~\ref{The_SuccessCase} in the center of our proposal correctness. Here, a node observes a transaction when a proposal that contains that transaction has been added to its DAG.
\begin{theorem}
    \label{The_SuccessCase}
    When a node added a proposal $p$ to its DAG at an odd round $r$, if a transaction $tx$ has more than $2f+1$ votes from nodes and no valid conflicting transactions were observed, then $tx$ will eventually be committed as a successful transaction according to Definition~\ref{def_tx_rule}.
\end{theorem}

\begin{IEEEproof}
    Recall that Definition~\ref{def_tx_rule} states that a transaction that has been included in more nodes earlier than other conflicting transactions will be committed, and the consensus layer commits a leader proposal only if $f+1$ proposals have included it in the next odd round.

    Firstly, we demonstrate that the transaction will eventually be committed. As there are at least $2f+1$ nodes whose proposals indicate they have voted for the transaction $tx$, any proposal at round $r+1$ will contain the transaction. Thus, $tx$ will be committed by some proposal eventually.
    
    Next, we show this transaction will be committed successfully. If any proposal is to be committed before round $r$, it must have $f+1$ subsequent proposals, and thus, it will be in the sub-DAG starting from $p$. Hence, the node will observe any valid conflicting transaction that can be committed before round $r$. If a proposal $p'$ after round $r$ is selected as a leader, then its sub-DAG will have at least $2f+1$ proposals from round $r$, which implies that at most $f$ nodes will vote for conflicting transactions while at least $f+1$ will vote for $tx$. According to our commit rule, $p'$ can only commit $tx$.
\end{IEEEproof}

A proposal in an even round cannot trigger fast-commit since its vote on that round may not impact the leader's choice in the same round. The leader may encounter a tied vote in its sub-DAG case and commit a different transaction. According to Theorem~\ref{The_SuccessCase}, we can define the fast commit rule.

\begin{definition}[Fast Commit Rule]
    A transaction is fast committed with a successful result at an odd round if $2f+1$ nodes have voted it and no valid conflicting transactions are observed.
    \label{def_fast_commit}
\end{definition}

This fast commit rule is derived from our new commit rule and Theorem~\ref{The_SuccessCase}, which does not affect the security claim of Tusk while reducing the average transaction latency.

\subsection{Our Construction: Board and Clerk} 

We propose a new consensus algorithm called ``Board and Clerk" to replace Tusk for applying our new commit rule. At a high level, Board decides on transaction results and fast-commit transactions, and Clerk agrees on the formal commitment. The algorithm operates as follows: After Narwhal commits a proposal, Board counts the votes from that proposal, keeps it in storage, and fast commits transactions if possible. Once the algorithm reaches the commit rule (\ie when processing a proposal of even round $r>4$), Clerk uses a global coin to select the leader. If the leader is valid according to the DAG, Clerk will output all uncommitted proposals from the leader's sub-dag, and Board can compute the transaction result accordingly for commitment.

Board stores all the vote results and conflict information. More specifically, Board stores a map of \{transaction Id: vote record\} and a map of \{batch: vote record\}. Each vote record is a hash map between \{node Id: round\} that records the round number that a node voted for the element (transaction or batch). We use Board.AddTxVotes, and Board.AddBatchVotes functions to denote updating the vote (insert the round for the node Id). 

Board provides two open APIs: first, when a new proposal is added to the DAG, Board counts votes for the author node. Since proposals are added to the DAG after its ancestors, the first vote from a node to an element is always smaller than the later votes from that node to the same element. Algorithm~\ref{Alg_Count_Vote} highlights the flow for counting transactions' votes, which triggers the fast commitment channel.

\begin{algorithm}[htbp]
	\caption{Algorithm for counting transaction votes}
	\label{Alg_Count_Vote}
	\begin{algorithmic}[1]
		\Require Proposal P
        
        \Function{Board.Process}{P}
        
        \State Initiate list B
        \State P.round $\rightarrow$ round
        \State P.author $\rightarrow$ voter
        \For{ batch in P.batches}
                \If{!Board.HaveSeen(batch)} 
                    \State add the batch in B
                \EndIf
        \EndFor

        \For{ batch in B}  %T}
            \For{ tx in batch}
                \State Board.Add(tx.txos, tx)
                %\State Record the TXO info to the board
                \If{Any TXO has more than one tx}
                    \State Add tx to conflicted set
                \EndIf
            \EndFor
            \State Board.AddTxVotes(batch.txs, voter, round)
        \EndFor

        \Comment{Counting Votes for sub-dag proposal}
        \State P.parents $\rightarrow$ C
        
        \While{C is not empty} 
            \State C.Pop() $\rightarrow$ parent
            \State Board.AddBatchVotes(parent.batches, voter, round)
            \If{any vote changed}
                \Comment{new batch}
                \State C.Concat(parent.parents)
                \State Board.AddTxVotes(batch.txs, voter, round)
                %\State Update votes for transactions in cert.
            \EndIf
        \EndWhile

        \If{round is odd}
            \Comment{Fast Commit}
            \State Filter out transactions that are not conflicted and have more than 2f+1 votes.
            \State Fast commit those transactions
            \State Mark those transactions for fast commitment
        \EndIf
        \EndFunction

\end{algorithmic}
\end{algorithm}

To aid the decision process of transaction states, we utilize internal states. To achieve this, we require the storage of node votes for different batches and transactions, with the added benefit of serving as a shortcut to avoid duplicated batches. Additionally, we must maintain records of conflicting transactions and the existing TXO-TX pairs, thus preventing the missing of any vital information regarding conflicts. To ensure smooth operation, we also use a cache to store transactions fast-committed to reject later conflicting transactions.

The second API for Board is triggered when Clerk commits a sub-DAG by the consensus, resulting in Board committing all transactions in the sub-DAG as required. Since the consensus algorithm only decides the leader for the previous round, we cannot directly use the latest vote counts, which include votes that are not to be committed. To avoid recounting votes, we define a frontier for each proposal.

\begin{definition}
A frontier $F$ of a proposal $p$ is a map between node Ids and their latest proposal round in the sub-DAG that begins with $p$.
\end{definition}

We can remove the vote records after the corresponding round in the frontier since the current commitment will not include those vote records. The flow for counting transaction results is outlined in Algorithm~\ref{Alg_TX_Result}.

\begin{algorithm*}[htbp]
	\caption{Algorithm for getting transaction results of a sub-DAG}
	\label{Alg_TX_Result}
	\begin{algorithmic}[1]
		\Require A list of proposals P
            \Ensure A set of successful transactions ST and a set of failed transactions FT
            
        \Function{Board.Commit}{P}
        \State Init empty ST, FT
        \State Set T as a list of all uncommitted TXs in the sub-DAG of P 
        
        \State Board.ConflictedTX() $\longrightarrow$ CT
        \Comment{CT is the dictionary of conflicted TXO and its conflicted transaction pair according to TXO-TXs}
        \State T.IntersectWith(CT.values) $\longrightarrow$ TxToProcess
        \If{ TxToProcess = $\phi$}
            \Comment{We can directly commit all as success}
            \State ST.Add(T)
            \State Return ST, FT
        \Else
            \State ST.Add(T.Except(TxToProcess))
        \EndIf

        \State Set M as a map of $<$TXO,TX$>$ with values from TxToProcess
        \State M.sortbyKey()
        \Comment{use a stable sort}
        \State Compute frontier $F(P)$
        \For{(txo,txs) in M}
            \State Prune transaction vote records by F and remove vote records from nodes that are not earliest.
            \State Sort txs with vote counts and $Ord$
            \State Push the first tx in txs to ST and the remaining to FT
            \For{ (txo', txs') in M}
            \Comment{We need to check if other conflicted sets have been affected by this result}
                \If{tx contains txo'}
                    \State Push other tx' in txs' to FT
                \EndIf
                \State txs' = txs'.disjoint(txs)
            \EndFor
        \EndFor

        \State Fast Commit other transactions according to ST, FT
        \Comment{Some transactions might meet the fast commit rule by now, (\eg, the conflicted $tx$ is failed) or it is not processed now but doomed to fail (\eg, the conflicted $tx$ is succeeded now)}
        \State Board.CleanUp
        
 	\State Return ST, FT
  \EndFunction

\end{algorithmic}
 
\end{algorithm*}



\section{Hyper-Block Model}\label{sec:hyperblock}
   Our second proposal is a Hyper-Block Model that allows for generating proofs for transaction results. Our solution adheres to the architecture demonstrated in Figure~\ref{fig_NT} and integrates the fast-commit rules from Definition~\ref{def_fast_commit}.

At a high level, a hyper-block consists of three parts: 1) proof of leadership, 2) a commitment to batches in the sub-DAG, and 3) a commitment to the failed transaction set. The consortium will sign all contents to enable any client to use the hyper-block and verify whether its transaction is correct. The commitment of the set can be any proper cryptographic aggregator that satisfies the requirements. Specifically, each proof of transaction result contains three parts. 
\begin{itemize} 
    \item \textbf{Block membership}: Proof that the transaction is in a batch contained by a hyper-block; 
    
    \item \textbf{Result Proof}: A demonstration of whether the transaction is a member or non-member of the corresponding failed transaction set in the hyper-block; 
    
    \item \textbf{Leadership Proof}: Proof of leadership for the protocol. %{\color{red}\textit{Leadership Proof} might be briefer.}
\end{itemize}

The primary technical challenge arises from preparing the hyper-block without disrupting the consensus protocol, which only decides the leader proposal index. To overcome this problem, we need to accomplish two tasks: Firstly, we need to obtain the transaction results to commit the failed transactions. Secondly, we must ensure that byzantine nodes cannot forge a hyper-block that validates a failed transaction.

We propose two strategies to achieve this: 1) A blocking strategy that synchronizes the DAG and consensus layers to execute a process similar to a view-change in other BFT protocols, and 2) a non-blocking strategy that commits all possible and valid output paths through the consensus algorithm.

Here we list some sub-functions we use in our blocking and non-blocking algorithms, along with implementation remarks: 
\begin{itemize}
    \item MakeProposal: Take a round $r$ as inputs and return a proposal $p$ that patches at least $2f+1$ proposals from the previous round and a bundle of batches received from its workers; 
    
    \item TakeLeader: Take a round $r$ as inputs, await the consensus layer output the leader of round $r$. If the leader is not in the DAG, recursively outputs the latest leader in the DAG;
    
    \item SimulateCommit: Take a proposal and the previous leader as input. It outputs a list of committed batches and a set of failed transaction Ids. The flow is almost the same as Algorithm~\ref{Alg_TX_Result} without side effects, thus should be implemented as a function of Board;
    
    \item CommitList: Take a list as input and return one value as output commitment. It can be implemented as a cryptographic accumulator that supports both membership and non-membership proof;

    \item AttachWitness: Including some witnesses to a proposal $p$ for acquiring signatures.
\end{itemize}

Now, we thoroughly detail both strategies.

\subsection{The Blocking Strategy}
The blocking strategy modifies the clerk consensus algorithm and proposes rules. For the consensus algorithm, if the first proposal of an odd round of $r\geq 5$ is added to the DAG, the consensus layer will automatically trigger the protocol to select the leader for the round $r-1$. %{\color{red}Specify which consensus algorithm is modified.}

Secondly, when the proposer initiates a proposal for any even round of $r+1 > 4$, it will first collect batches and wrap them properly. Next, the proposer will block the proposal and await the selection of a leader from round $r-1$. After the proposer becomes aware of the selection of the round leader, it will simulate the transaction results of its proposal as if this proposal is committed to the future leader. This simulation will compute the corresponding failed transaction commitment and batch set commitment. Subsequently, the two commitments are wrapped together with the proposal and dispatched to the broadcast phase. The flow for the proposing process is presented in Algorithm~\ref{Alg_Proposing}. 



 \begin{algorithm}[htbp]
 	\caption{Blocking algorithm for proposing}
 	\label{Alg_Proposing}
 	\begin{algorithmic}[1]
 		\Require Round $r$
 		\Ensure Proposal $p$ for round $r$
 		
            \Function{Proposer.Proposing}{$r$}
 		\State MakeProposal($r$) $\longrightarrow p$
            \State TakeLeader($r-2$).await() $\longrightarrow L$
            \State Board.SimulateCommimt($p, L$) $\longrightarrow (B,F)$
 		\State CommitList($F$) $\longrightarrow wf$
            \State CommitList($B$) $\longrightarrow wb$
 		\State AttachWitness($p$,$wb,wf$)
 		\State Return $p$
            \EndFunction
 	\end{algorithmic}
 \end{algorithm}

Upon receiving the proposal from the other nodes, the initial action is to verify the DAG history and confirm the correct calculation of the failed transaction commitment. Confirmation is done by signing the proposal ID, the commitment of failed transactions, and the commitment of batch set. The signed information is used to generate the certificate of the proposal in the DAG. Once the proposal is successfully committed to a leader, each node independently assembles the hyper block by fetching this signature and the output of the consensus layer.

Our blocking strategy does not affect the security claims of the Tusk algorithm, as the election process is independent. It should be noted that this process triggers at least $2f+1$ simulations during the RBCs phases of each round. The result can be cached for the formal commit process.  

\begin{remark}
The counting process can be accelerated for efficiency by utilizing dynamic programming methods to cache the votes for conflicted transactions (M in line 13 of Algorithm~\ref{Alg_TX_Result}). The cached outcome can later be removed after the commitment of corresponding leaders.
\end{remark}

\subsection{The Non-Blocking Strategy}

In an asynchronous network setting, waiting for the consensus layer results in the blocking strategy may not be tolerable. Therefore, we propose a non-blocking strategy to minimize the potential delays due to the network.

We use a commit and prove flow technique to create the hyper block in the non-blocking strategy. The proposer maintains a leader DAG by the present DAG. The difference is that the leader DAG comprises proposals only for even rounds (\ie, rounds for selecting leaders). Every two proposals are connected only if they are connected in the DAG. Moreover, whenever a leader is chosen for a round $r$, the proposals other than the selected one are removed from the leader DAG instantaneously. The leader DAG can be seen as comprising all the probable leader proposals of the current DAG.

In the leader DAG, we store batch and failed transaction commitments. Each proposal has a vector for these two commitments; every position in the vector reflects a path in the leader DAG from the current proposal to the previously committed leader (if the leader is not in the DAG), given that the corresponding proposals on the path are chosen as the leaders for the corresponding rounds. 

Figure~\ref{fig_leader_dag} shows an instance of a leader DAG and its verification path based on the consensus layer results. When proposing for round 10, we assume the leaders of round 4,6 are known(checked), and the leader of round 8 is unknown (question marked). The new proposal at round 10 has four possible cases for the round leader at round 8. It will create corresponding commits and form them as a vector for proposing. For proposals at round 8, which already know the leader of round 6 (checked), Node 2 will contain a commitment for it and be used by the proposer in round 10. On the other hand, Node 3 has not yet seen proposals 2-6, so its commitment to the older leader 1-4 (checked) will be used for proposing. If the 2-8 is the eventual new leader, the black line represents the commitments that all users eventually use.

% Figure environment removed

We require that a leader DAG supports the following functions: First, Candidates($r$) returns a list of candidates leaders for round $r$. If the leader is not selected, it should be all proposals in round $r$ plus Candidate($r-2$) in a recursive way. Second, Add($p, V$) adds the proposal $p$ to the leader DAG with a vector $V$ of commitment of batches and transactions. In addition, the leader DAG should also support some clean-up functions to remove non-leader or committed proposals accordingly.

The proposal contains the vector created according to the paths of s leader DAG. After receiving a proposal, a node verifies the vector and then commits it as a Merkle tree, sending the root's signature as evidence back to the proposer. This evidence, along with the consensus layer output, will eventually be used to verify the correctness of the batch commitment and the failed transactions commitment made by the node if this proposal is the leading one in the future. Algorithm~\ref{Alg_Proposing_NB} outlines the proposal procedure in the non-blocking strategy.

 \begin{algorithm}[htb]
 	\caption{Non-Blocking algorithm for proposing}
 	\label{Alg_Proposing_NB}
 	\begin{algorithmic}[1]
 		\Require Round $r$, Leader DAG $LD$
 		\Ensure Proposal $p$ for round $r$
 		\Function{Proposer.Proposing}{$r,LD$}
 		\State MakeProposal($r$) $\longrightarrow p$
            \State Initiate a vector $V$
            \For{ $L$ in $LD$.candidates($r-2$)}
                \State Board.SimulateCommit($p,L$) $\longrightarrow (B,F)$
     		\State CommitList($F$) $\longrightarrow wf$
                \State CommitList($B$) $\longrightarrow wb$
                \State $V$.Push($wf,wb$)
            \EndFor
            \State $LD$.Add($p,V$)
 		\State AttachWitness($p$,$V$) 
 		\State Return $p$
        \EndFunction
 	\end{algorithmic}
 \end{algorithm}

Unlike the blocking strategy, the non-blocking strategy involves a polynomial scale of possible paths to commit. However, in practice, if the leader of round $r-4$ is evident by round $r$, the proposer needs to calculate at most $2f+2$ possibilities for its vector, given that each proposal's commitment at round $r-2$ is already confirmed. Thus, the total computation cost for each leader round is at most $O(n^2)$ in this case.

\begin{remark}
    Practically, we can hybrid non-blocking and blocking strategies. We can safely choose the non-blocking strategy when there are no or limited conflict transactions. Once the conflicted transaction increase, a blocking strategy can cool down the proposing phase and reduce the computation overhead.
\end{remark}

\section{Implementation and Evaluation}\label{sec:implementation}
    %-------------------------------------------------------------------------------
\section{Implementation} \label{imple}
%-------------------------------------------------------------------------------

% Figure environment removed

\sys's implementation consists of (\romannumerber{1}) a fully-functional \sys switch prototype which implements the overall in-fabric logic; and (\romannumerber{2}) a set of software APIs exposed to applications. Our prototype is built upon a FPGA-assisted commodity switch while a P4 programmable switch implementation is also provided.

%\sys's implementation consists of (\romannumerber{1}) a FPGA-based fully-functional prototype which implements the self-defined switch logic and (\romannumerber{2}) a set of software APIs exposed to multicast applications. \sys's switch logic can also be implemented on the Tofino P4 switch, as described in $\S$\ref{dis}.

\parab{FPGA-based prototype.} We implement the group registration, data packet duplication, header modification, and feedback aggregation logics on an FPGA board. The board is equipped with a commodity FPGA chip~\cite{ultrascale} and four 100Gbps Ethernet interfaces. The FPGA resource utilization is shown in Table~\ref{tab:overhead}. We build our testbed with the FPGA board, a commodity Ethernet switch, and four servers, as illustrated in Fig.~\ref{fig:fpgaprototype}. Each server is equipped with a commodity RNIC. The FPGA board and four RNICs are connected to the commodity switch through 100Gbps Ethernet interfaces. 

The commodity switch is configured by Access Control List (ACL) to route the servers' multicast traffic to the FPGA board. The FPGA board identifies the multicast data (ACK\footnote{In Fig.~\ref{fig:fpgaprototype}, we use ACK to represent all types of feedback.}) packets through the specific packet header by \textit{Parser} and \textit{Arbiter}. The data (ACK) packets will be duplicated (aggregated) by \textit{Duplicator} (\textit{ACK Aggregator}). The resulting packets will be pushed in \textit{Queue System}, waiting for the \textit{Multiplexer} to schedule in case for queue competition. Finally, the duplicated data (aggregated ACK) packets are sent back to the commodity switch. During processing, the \textit{Multicast Forwarding Table} is accessed when needed. %\todo{describe Fig. 8} Note those resulting packets' destination IP would be unicast IPs, so the switch routes them as normal unicast packets.
%

\parab{P4-based implementation.}
\sys in-fabric logic can be implemented on the P4 switch as well with some special handling. For the one-to-many data forwarding, P4 switch duplicates packets at the Traffic Manager (TM). The extended table states in Fig.~\ref{fig:table} are stored in the egress pipeline, indexed by <GroupID, EgressPort>. Because the lookup key in P4 switch is at most 32bits, we can use the least significant 24bits of GroupIP plus the 8bits port number as the real index. 

For the many-to-one feedback aggregation, there are two challenges due to the limited capability of P4 switch. Commodity P4 switch switch contains many stages, each with minimal computation capability and independent memory. Firstly, a single stage cannot support the \textit{wrapped-around} PSN comparison. To handle it, we simplify the standard PSN comparison to match the stage's capability, resulting in a tighter PSN space reduced from $2^{23}$ to $2^{22}$. Secondly, a single stage cannot iterate the entire table entries and find the minimum PSN. To handle it, we leverage multiple stages, each responsible for partial entries. Thus, the maximum number of entries supported in each multicast group is limited by the total stage number.

Besides, the P4 switch lacks the computation capability to recalculate the Invariant Cyclic Redundancy Checksum (ICRC) for the modified (aggregated) data (ACK) packets. As a result, those packets would violate the ICRC validation and be discarded at the receiver. This is why we choose the FPGA-based prototype to evaluate \sys in this work. However, a recent work shows that some RNICs provide the ability to bypass ICRC validation~\cite{switchML}. 
%Although this can work, it's a compromised method having security risks. 
%So we select the more integrated FPGA-based prototype to evaluate in $\S$\ref{eva}. Other operators can choose their preferred implementation based on their requirements.

%\begin{algorithm}[t]
%	\caption{Update PSN record and find PSN minimum in P4}\label{alg:psncomp}
%	\begin{algorithmic}[1]
%		%\Function{Generation}{}
%		\State $ack.psn, ack.port\gets $ the PSN and port of ACK packet
%		\State $rec.psn, rec.port\gets$ the PSN and port recorded
%		\State $isTrigger\gets$ whether the packet is a trigger packet
%		%\State $last\_ack\_psn\gets$ last aggregated ACK's psn
%		%\State $min\_port\gets$ port with minimum $ack\_psn$ last time
%		\State min.psn = ack.psn;
%		\State \textcolor{purple}{// every stage compare the PSN}
%		\If{$ack.psn > rec.psn$ or $(ack.psn <= 24'b3fffff$ and $rec.psn >= 24'b600000)$}
%			\State min.psn = rec.psn;
%			\If{rec.port == ack.port}
%				\State rec.psn = ack.psn;
%			\EndIf
%		\EndIf
%		\State \textcolor{purple}{// last stage write the min PSN back}
%		\If{isTrigger} 
%			\State ack.psn = min.psn;
%			\State Forward ACK.
%		\EndIf
%	\end{algorithmic}
%\end{algorithm}

\parab{Software APIs.} We provide various communication libraries and middleboxes for \sys multicast support. Take the commonly-used OpenMPI as an example, we modify the OpenMPI (v4.1.1)~\cite{openmpi} and UCX (v2.3)~\cite{ucx} to adapt to \sys's design, as shown in Fig.~\ref{fig:fpgaprototype}. Specifically, we add a new implementation of $MPI\_Bcast$ and modify UCX for multicast QPs creation and data transmission. When the new $MPI\_Bcast$ is called, the MPI process calls the UCX to establish QPs for multicast. Multicast members exchange their QPs information, and the handshake starts, as described in Appendix \ref{apx:regis}. Once the multicast group is successfully established, the UCX finally calls the RDMA primitives defined in the well-known \textit{libibverbs}~\cite{libibverbs} to transmit data. The software modifications at the end-host are transparent to the upper-layer applications and don't require any RNIC or RDMA driver modification.
%\parab{Coalescence of unicast and multicast}
%When we design \sys, there is a question in our mind: \textit{which is better, maintaining unicast and multicast transports separately at end-host, or utilizing the in-network support to enabling them to match the same transport?} Because of the long-standing resource limit in RNICs and the emeging trend of shifting appropriate computation task to programmability network, we believe the latter is the correct selection.

\begin{table}[t]
	\small
    \centering
%	\begin{center}
%    \begin{tabular}{l|c|c|c}
    \begin{tabular}{|p{0.2\linewidth}|p{0.18\linewidth}|p{0.18\linewidth}|p{0.18\linewidth}|}
    \hline
    \textbf{Resource} & \hfil \textbf{LUT} & \hfil \textbf{Register} & \hfil \textbf{BRAM} \\
    \hline
   	\textbf{Usage} & \hfil 53169 & \hfil 15391 & \hfil 188 \\
    \hline
    \end{tabular}
%    \end{center}
    \caption{Resource usage of the \sys in-fabric logic.}
    \label{tab:overhead}
    \vspace{-0.25cm}
\end{table}

%\parab{Resource overhead}  Note that the size of multicast forwarding table is determined by the number of ports of the switch and doesn't scale up with the multicast group size. 2.7MB memory can support upto 1K multicast groups, which is satisfied in datacenter. We provide a detailed calculation of the maxmum group support in Appendix \ref{apx:cal}. 
    
\section{Conclusion}\label{sec:conclusion}
    \section{Conclusion and Future Work}
In this work, I design corruption-robust algorithms for the Lipschitz contextual search problem. I present the \emph{agnostic checking} technique and demonstrate its effectiveness in designing corruption-robust algorithms. There are several open problems for future research. First, in the algorithm I propose for pricing loss, the schedule for agnostic checks is fixed upfront. Can the learner design an adaptive checking schedule for the pricing loss? Second, this work assumes the learner has knowledge of the Lipschitz constant $L$. Can the learner design efficient no-regret algorithms without knowledge of $L$? 


%\section*{Acknowledgment}
%
%
%The authors would like to thank...
%
%




\bibliographystyle{IEEEtranS}
\bibliography{bib/reference,bib/abbrev0,bib/crypto}
\end{document}


