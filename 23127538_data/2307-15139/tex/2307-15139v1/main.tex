\documentclass[10pt,twocolumn,letterpaper]{article}
\usepackage{iccv}
\usepackage{times}
\usepackage{amsmath}
\usepackage{amssymb}
\usepackage{arydshln}
\usepackage{bm}
\usepackage{boldline}
\usepackage{booktabs}
\usepackage{color}
\usepackage{enumitem}
\usepackage{float}
\usepackage{graphicx}
\usepackage{multirow}
\usepackage{nicefrac}
\usepackage{subfig}
\usepackage[accsupp]{axessibility}
\usepackage[pagebackref=true,breaklinks=true,colorlinks,bookmarks=false]{hyperref}

% Support for easy cross-referencing
\usepackage[capitalize]{cleveref}
\crefname{section}{Sec.}{Secs.}
\Crefname{section}{Section}{Sections}
\Crefname{table}{Table}{Tables}
\crefname{table}{Tab.}{Tabs.}

\makeatletter
\renewcommand{\paragraph}{%
  \@startsection{paragraph}{4}%
  {\z@}{0.25em}{-1em}%
  {\normalfont\normalsize\bfseries}%
}
\makeatother

\iccvfinalcopy
\def\iccvPaperID{1726}
\ificcvfinal\pagestyle{empty}\fi

\makeatletter
\def\input@path{{.}{..}}
%or: \def\input@path{{/path/to/folder/}{/path/to/other/folder/}}
\makeatother
\graphicspath{{.}{..}}

\def\mname{CVQ-VAE\xspace}
\newcommand{\todo}[1]{{\color{red}{[ToDo:#1]}}}

\title{Online Clustered Codebook}

\author{
Chuanxia Zheng \quad Andrea Vedaldi \\[0.3em]
Visual Geometry Group, University of Oxford \\
{\tt\small \{cxzheng, vedaldi\}@robots.ox.ac.uk}}

\begin{document}
\maketitle 
\ificcvfinal\thispagestyle{empty}\fi

\begin{abstract}
Vector Quantisation (VQ) is experiencing a comeback in machine learning, where it is increasingly used in representation learning.
However, optimizing the codevectors in existing VQ-VAE is not entirely trivial.
A problem is codebook collapse, where only a small subset of codevectors receive gradients useful for their optimisation, whereas a majority of them simply ``dies off'' and is never updated or used.
This limits the effectiveness of VQ for learning larger codebooks in complex computer vision tasks that require high-capacity representations.
In this paper, we present a simple alternative method for online codebook learning, Clustering VQ-VAE (\mname).
Our approach selects encoded features as anchors to update the ``dead'' codevectors, while optimising the codebooks which are alive via the original loss.
This strategy brings unused codevectors closer in distribution to the encoded features, increasing the likelihood of being chosen and optimized.
We extensively validate the generalization capability of our quantiser on various datasets, tasks (\eg reconstruction and generation), and architectures (\eg VQ-VAE, VQGAN, LDM).
\mname can be easily integrated into the existing models with just a few lines of code.
\end{abstract}

% Figure environment removed

\section{Introduction}

Vector Quantisation (VQ)~\cite{1162229} is a basic building block of many machine learning techniques.
It is often used to help learning unsupervised representations for vision and language tasks, including data compression~\cite{agustsson2017soft,williams2020hierarchical,takida2022sq}, recognition~\cite{maodiscrete,baobeit,yu2022vectorquantized,liu2022cross,li2022unimo}, and generation~\cite{van2017neural,razavi2019generating,esser2021taming,rombach2022high,zhengmovq,sargent2023vq3d,sanghi2023sketch}.
VQ quantises continuous feature vectors into a discrete space by embedding them to the closest vectors in a codebook of representatives or codevectors.
Quantisation has been shown to simplify optimization problems by reducing a continuous search space to a discrete one.
%to finding a soluin a restricted discrete space.

Despite its success, VQ has some drawbacks when applied to deep networks~\cite{van2017neural}.
One of them is that quantisation stops gradients from back-propagating to the codevectors.
This has been linked to \emph{codebook collapse}~\cite{takida2022sq}, which means that only a small subset of active codevectors are optimized alongside the learnable features, while the majority of them \emph{are not used at all} (see the \textcolor[RGB]{100,210,180}{green} ``dead'' points in \cref{fig:intro_mov}(a)).
As a result, many recent methods~\cite{esser2021taming,esser2021imagebart,yu2022vectorquantized,rombach2022high,chang2022maskgit,zhengmovq} fail to utilise the full expressive power of a codebook due to the low codevector utilisation, especially when the codebook size is large.
This significantly limits VQ's effectiveness.
%  to explore large codebooks for complex modalities.

To tackle this issue, we propose a new alternative quantiser called \emph{Clustering VQ-VAE} (\mname).
We observe that classical clustering algorithms, such as refined initialization $k$-means~\cite{bradley1998refining} and $k$-means++~\cite{arthur2007k}, use a dynamic cluster initialization approach.
For example, $k$-means++ randomly selects a data point as the first cluster centre, and then chooses the next new centre based on a weighted probability calculated from the distance to the previous centres.
Analogously, \mname \emph{dynamically} initializes unoptimized codebooks by resampling them from the learned features (\cref{fig:method}).
This simple approach can avoid codebook collapse and significantly enhance the usage of larger codebooks by enabling optimization of all codevectors (achieving $100\%$ codebook utilisation in \cref{fig:intro_mov}{(c)}).

While \mname is inspired by previous dynamic cluster initialization techniques~\cite{bradley1998refining,arthur2007k}, its implementation in deep networks requires careful consideration.
Unlike traditional clustering algorithms~\cite{lloyd1982least,bradley1998refining,hamerly2002alternatives,arthur2007k} where source data points are fixed, in deep networks features and their corresponding codevectors are mutually and incrementally optimized.
Thus, simply sampling codevectors from a single snapshot of features would not work well because any mini-batch used for learning \emph{cannot} capture the true data distribution, as demonstrated in our offline version in \cref{tab:rule}.
To fix this issue, we propose to \emph{compute running averages} of the encoded features across different training mini-batches and use these to improve the dynamic reinitialization of the collapsed codevectors.
This operation is similar to an online feature clustering method that calculates average features across different training iterations (\cref{fig:method}).
While this may seem a minor change, it leads to a very significant improvement in terms of performance (\cref{fig:intro_mov}{(e)}).

% Without bells and whistles,
As a result of these changes, \mname significantly outperforms the previous models VQ-VAE~\cite{van2017neural} and SQ-VAE~\cite{takida2022sq} on various datasets under the same setting, and with no other changes except for swapping in the new quantiser.
Moreover, we conduct thorough ablation experiments on variants of the method to demonstrate the effectiveness of our design and analyse the importance of various design factors.
Finally, we incorporate \mname into large models (\eg VQ-GAN~\cite{esser2021taming} and LDM~\cite{rombach2022high}) to further demonstrate its generality and potential in various applications.

\section{Related Works}

VQ-VAE~\cite{van2017neural} learns to quantise the continuous features into a discrete space using a restricted number of codebook vectors.
By clustering features in the latent space, VQ-VAE can automatically learn a crucially compact representation and store the domain information in the decoder that does not require supervision.
This discrete representation has been applied to various downstream tasks, including image generation~\cite{razavi2019generating,yu2022vectorquantized,chang2022maskgit,lee2022autoregressive,hu2022global}, image-to-image translation~\cite{esser2021taming,ramesh2021zero,esser2021imagebart,rombach2022high}, text-to-image synthesis~\cite{ramesh2021zero,ding2021cogview,ramesh2022hierarchical,hu2022unified}, conditional video generation~\cite{rakhimov2021latent,wu2022nuwa,yan2021videogpt}, image completion~\cite{esser2021taming,esser2021imagebart,zheng2022high}, recognition~\cite{maodiscrete,baobeit,yu2022vectorquantized,liu2022cross,li2022unimo} and 3D reconstruction~\cite{mittal2022autosdf,sargent2023vq3d,sanghi2023sketch}.

Among them, VQ-GAN~\cite{esser2021taming}, ViT-VQGAN~\cite{yu2022vectorquantized}, RQ-VAE~\cite{lee2022autoregressive}, and MoVQ~\cite{zheng2022high} aim to train a better discrete representation through deeper network architectures, additional loss functions, multichannel or higher resolution representations. However, none of them tackle the \emph{codebook collapse} issue for the unoptimized ``dead'' point.

To address this issue, additional training heuristics are proposed in recent works. 
SQ-VAE~\cite{takida2022sq} improves VQ-VAE with stochastic quantisation and a trainable posterior categorical distribution. 
VQ-WAE~\cite{vuong2023vector} builds upon SQ-VAE by directly encouraging the discrete representation to be a uniform distribution via a \emph{Wasserstein} distance. 
The most related works are HVQ-VAE~\cite{williams2020hierarchical} and Jukebox~\cite{dhariwal2020jukebox} that use \emph{codebook reset} to randomly reinitialize unused or low-used codebook entries. 
However, they only assign a single sampled anchor to each unoptimized codevector. 
In contrast, our \mname considers the changing of features in deep networks and designs an online clustering algorithm by running average updates across the training mini-batch. 
Additionally, our work bridges codebook reset in Jukebox for music generation to the more general class of running average updates that are applicable to image compression and generation problems in computer vision.

% Figure environment removed

\section{Method}%
\label{s:method}

VQ is in the context of unsupervised representation learning.
Our main goal is to learn a discrete codebook that efficiently utilizes \emph{all codebook entries within it}.
To achieve this, our quantisation method, as illustrated in \cref{fig:method}, is conceptually similar to VQ-VAE~\cite{van2017neural}, except that our codevectors are \emph{dynamically initialized} rather than being sampled from a \emph{fixed} uniform or Gaussian distribution.
In the following sections, we provide a general overview of VQ (\cref{sec:method_back}), followed by %several specific instantiations of
our proposed \mname (\cref{sec:mth_online}).

\subsection{Background: VQ-VAE}%
\label{sec:method_back}

Given a high dimensional image $x\in\mathbb{R}^{H\times W\times c}$, VQ-VAE~\cite{van2017neural} learns to embed it with low dimensional codevectors $z_q\in\mathbb{R}^{h \times w\times n_q}$, where $n_q$ is the dimensionality of the vectors in the codebook.
Then, the feature tensor can be equivalently described as a compact representation with $h\times w$ indices corresponding to the codebook entries $z_q$.
This is done via an autoencoder
\begin{equation}\label{eq:vq-vae}
    \hat{x}
    = \mathcal{G}_\theta(z_q)
    = \mathcal{G}_\theta(\mathbf{q}(\hat{z}))
    = \mathcal{G}_\theta(\mathbf{q}(\mathcal{E}_\phi(x))).
\end{equation}
Here $\mathcal{E}_\phi$ and $\mathcal{G}_\theta$ refer to the encoder and decoder, respectively.
The encoder embeds images into the continuous latent space, while the decoder inversely maps the latent vectors back to the original image.
$\mathbf{q}(\cdot)$ is a quantisation operation that maps the continuous encoded observations $\hat{z}$ into the discrete space by looking up the closest codebook entry $e_k$ for each grid feature $\hat{z}_i$ using the following equation:
\begin{equation}\label{eq:quant}
    z_{q_{i}}
    = \mathbf{q}(\hat{z}_{i})
    = e_k,\quad\text{where}\quad
    k
    = \underset{e_k\in\mathcal{Z}}{\operatorname{argmin}}\lVert\hat{z}_{i}-e_k\rVert,
\end{equation}
where $\mathcal{Z}=\{e_k\}_{k=1}^K$ is the codebook that consists of $K$ entries $e_k\in\mathbb{R}^{n_q}$ with dimensionality $n_q$.
During training, the encoder $\mathcal{E}_\phi$, decoder $\mathcal{G}_\theta$ and codebook $\mathcal{Z}$ are jointly optimized by minimizing the following objective:
\begin{equation}\label{eq:vae_loss}
    \mathcal{L}
    = \lVert x - \hat{x} \rVert_2^2 + \lVert\mathrm{sg}[\mathcal{E}_\psi(x)] - z_q\rVert_2^2 + \beta\lVert\mathcal{E}_\psi(x) - \mathrm{sg}[z_q]\rVert_2^2,
\end{equation}
where $\mathrm{sg}$ denotes a stop-gradient operator, and $\beta$ is the hyperparameter for the last term \emph{commitment loss}.
The first term is known as \emph{reconstruction loss}, which measures the difference between the observed $x$ and the reconstructed $\hat{x}$.
The second term is the \emph{codebook loss}, which encourages the codevectors to be close to the encoded features.
In practice, the codebook $\mathcal{Z}$ is optimized using either the \emph{codebook loss}~\cite{van2017neural} or using an exponential moving average (EMA)~\cite{razavi2019generating}.
However, these methods work only for the active codevectors, \emph{leaving the ``dead'' ones unoptimized}.

\subsection{Clustering VQ-VAE (CVQ-VAE)}%
\label{sec:mth_online}

The choice of initial points is a crucial aspect of unsupervised codebook learning.
Classical clustering methods like refined $k$-means~\cite{bradley1998refining} and $k$-means++~\cite{arthur2007k} are \emph{dynamically-initialized}, which means that each new clustering centre is initialized based on previously calculated distance or points.
This leads to a more robust and effective clustering result, as reported in comparative studies~\cite{celebi2013comparative}.

Analogously, we build a \emph{dynamically-initialized} vector quantized codebook in deep networks. % to avoid codebook collapse.
However, unlike traditional clustering settings, the data points, \ie the encoded features $\hat{z}$ in the deep network, are also updated during training instead of being fixed.
Therefore, a dynamical initialization strategy should take into account the changing feature representations during training.

\paragraph{Running average updates.}

To build the online initialization for a codebook, we start by accumulatively counting the average usage of codevectors in each training mini-batch:
\begin{equation}\label{eq:count}
    N_k^{(t)} 
    = N_k^{(t-1)} \cdot \gamma + \frac{n_k^{(t)}}{Bhw} \cdot (1-\gamma),
\end{equation}
where $n_k^{(t)}$ is the number of encoded features in a training mini-batch that will be quantised to the closest codebook entry $e_k$, and $Bhw$ denotes the number of features on Batch, height, and width.
$\gamma$ is a decay hyperparameter with a value in $(0,1)$ (default $\gamma=0.99$).
$N_k^{(0)}$ is initially set as zero.

We then select a subset $\bar{\mathcal{Z}}$ with $K$ vectors from the encoded features $\hat{z}$, which we denote as \textbf{anchors}.
Instead of directly using the anchors to reinitialize the unoptimized codevectors, we expect that \emph{codevectors that are less-used or unused should be modified more than frequently used ones}.
To achieve this goal, we compute a decay value $a_k^{(t)}$ for each entry $e_k$ using the accumulative average usage $N_k^{(t)}$ and reinitialize the features as follows:
\begin{gather}\label{eq:update}
    \alpha_k^{(t)} 
    = \exp^{-N_k^{(t)} K\,\frac{10}{1-\gamma}-\epsilon}, \\
    e_k^{(t)} 
    = e_k^{(t-1)} \cdot (1-\alpha_k^{(t)}) + \hat{z}_k^{(t)} \cdot \alpha_k^{(t)},
\end{gather}
where $\epsilon$ is a small constant to ensure the entries are assigned with the average values of features along different mini-batches, and $\hat{z}_k^{(t)}\in{\bar{\mathcal{Z}}}^{K\times z_q}$ is the sampled anchor.

This running average operation differs from the exponential moving average (EMA) used in VQ-VAE~\cite{razavi2019generating}.
Our equation is applied to reinitialize unused or low-used codevectors, instead of updating the active ones.
Furthermore, our decay parameter in \cref{eq:update} is computed based on the average usage, which is \emph{not a pre-defined hyperparameter}.

\paragraph{Choice of the anchors.} Next, we describe several versions of the anchor sampling methods.
Interestingly, experimental results (\cref{tab:rec_ab_featm}) show that our online version is \emph{not} sensitive to these choices.
However, the different anchor sampling methods have a direct impact on the \emph{offline} version, suggesting that our running average updates behaviour is the primary reason for the observed improvements.
\begin{itemize}[itemsep=0pt]
    \vspace{-0.1cm}
    \item \textbf{Random.} Following the codebook reset~\cite{dhariwal2020jukebox,williams2020hierarchical}, a natural choice of anchors is randomly sampled from the encoded features.
    \item \textbf{Unique.} To avoid repeated anchors, a random permutation of integers within the number of features ($Bhw$) is performed. Then, we select the first $K$ features.
    \item \textbf{Closest.} A simple way is inversely looking up the closest features of each entry, \ie $i = \underset{\hat{z}_i\in\mathcal{E}_\phi(x)}{\operatorname{argmin}}\lVert\hat{z}_{i}-e_k\rVert$.
    \item \textbf{Probabilistic random.} We can also sample anchors based on the distance $D_{i,k}$ between the codevectors and the encoded features.
    In this paper, we consider the probability $p=\frac{\exp{(-D_{i,k})}}{\sum_{i=1}^{Bhw}\exp{(-D_{i,k})}}$.
\end{itemize}

% In a training mini-batch, the number of features may be smaller than the size of the codebook. 
% To avoid many code entries being reinitialized with the same anchor, we introduce a feature pool inspired by the image pool used in CycleGAN~\cite{zhu2017unpaired}.
% Specifically, we store a set of encoded features from previous batches into a pool, and then update it with the latest features and sample anchors from this pool.

\paragraph{Contrastive loss.}

We further introduce a contrastive loss
$
-\log
\frac
{e^{sim(e_k,\hat{z}_i^+)/\tau}}
{\sum_{i=1}^N e^{sim(e_k,\hat{z}_i^-)/\tau}}
$
to encourage sparsity in the codebook.
In particular, for each codevector $e_k$, we directly select the closest feature $\hat{z}_i^+$ as the positive pair and sample other farther features $\hat{z}_i^-$ as negative pairs using the $D_{i,k}$.
% By adding this contrastive loss, we further slightly improve the performance (\cref{tab:rule} $\mathbb{D}$).

\paragraph{Relation to prior work.}

To mitigate the codebook collapse issue, several methods have been proposed, like 
%exponential moving average in VQ-VAE-2~\cite{razavi2019generating} and 
normalized codevectors in ViT-VQGAN~\cite{yu2022vectorquantized}.
However, these methods only optimize the \emph{active} entries, rather than the entire codebook.
Recently, SQ-VAE~\cite{takida2022sq}, SeQ-GAN~\cite{gu2022rethinking}, and VQ-WAE~\cite{vuong2023vector} assume that the codebook follows a fixed distribution.
Although these methods achieve high perplexity, the reconstruction quality is \emph{not} always improved (\cref{tab:rec_ab}). 
The most relevant work to ours is codebook reset, which randomly reinitializes the unused or low-used codevectors to high-usage ones~\cite{williams2020hierarchical} or encoder outputs~\cite{dhariwal2020jukebox}.
However, these methods rely only on a temporary single value for initialization and miss the opportunity of exploiting online clustering across different training steps.

\section{Experiments: Image Quantisation}%
\label{sec:exp}

\subsection{Experimental Details}

\paragraph{Implementation.} \mname can be easily implemented by a few lines of code in Pytorch, where the gradient for the selected codevectors is preserved. The code is available at \href{https://github.com/lyndonzheng/CVQ-VAE}{https://github.com/lyndonzheng/CVQ-VAE}.

Our implementation is built upon existing network architectures.
We set all hype-parameters following the original code, except that we replace these quantisers with our online clustering codebook.
In particular, we first demonstrate our assumption on small datasets with the officially released VQ-VAE~\cite{van2017neural} implementation~\footnote{\href{https://github.com/deepmind/sonnet/blob/v2/sonnet/src/nets/vqvae.py}{https://github.com/deepmind/sonnet/blob/v2/sonnet/src/nets/vqvae.py}}$^,$\footnote{\href{https://github.com/deepmind/sonnet/blob/v1/sonnet/examples/vqvae_example.ipynb}{https://github.com/deepmind/sonnet/blob/v1/sonnet/examples/vqvae\_
example.ipynb}}. 
Then, we verify the generality of our quantiser on large datasets using the officially released VQ-GAN~\cite{esser2021taming} architecture~\footnote{\href{https://github.com/CompVis/taming-transformers}{https://github.com/CompVis/taming-transformers}}.

% Figure environment removed

\paragraph{Datasets.}

We evaluated the proposed quantiser on various datasets, including MNIST~\cite{lecun1998gradient}, CIFAR10~\cite{krizhevsky2009learning}, Fashion MNIST~\cite{xiao2017online}, and the higher-resolution FFHQ~\cite{karras2019style} and the large ImageNet~\cite{deng2009imagenet}.

\paragraph{Metrics.}

Following existing works~\cite{esser2021taming,zhengmovq,gu2022rethinking}, we evaluated the image quality between reconstructed and original images on different scales, including patch-level structure similarity index (SSIM), feature-level Learned Perceptual Image Patch Similarity (LPIPS)~\cite{zhang2018unreasonable}, and dataset-level Fr\'{e}chet Inception Distance (FID)~\cite{heusel2017gans}.
We also report the perplexity score for the codebook ablation study as in SQ-VAE~\cite{takida2022sq} and VQ-WAE~\cite{vuong2023vector}.
It is defined as
$
e^{-\sum_{k=1}^K p_{e_k}\log p_{e_k}}
$,
where $p_{e_k}=\frac{n_k}{\sum_{i=i}^K n_k}$, and $n_k$ is the number of encoded features associated with codevector $e_k$.

\subsection{Main Results}

\begin{table}[tb!]
    \centering
    \renewcommand{\arraystretch}{1.0}
    \setlength\tabcolsep{5pt}
    \begin{tabular}{@{}l c ccc @{}}
    \toprule
    \textbf{Method} & \textbf{Dataset} & SSIM $\uparrow$ & LPIPS $\downarrow$ & rFID $\downarrow$ \\
    \midrule
    VQ-VAE~\cite{van2017neural} & \multirow{4}{*}{MNIST}& 0.9777 & 0.0282 & 3.43 \\
    HVQ-VAE~\cite{williams2020hierarchical} & & 0.9790 & 0.0270 & 3.17 \\
    SQ-VAE~\cite{takida2022sq} & & 0.9819 & 0.0256 & 3.05 \\
    \cdashline{1-5}
    \textbf{\mname} & & \textbf{0.9833} & \textbf{0.0222} & \textbf{1.80} \\
    \midrule
    VQ-VAE~\cite{van2017neural} & \multirow{4}{*}{CIFAR10}& 0.8595 & 0.2504 & 39.67 \\
    HVQ-VAE~\cite{williams2020hierarchical} & & 0.8553 & 0.2553 & 41.08 \\
    SQ-VAE~\cite{takida2022sq} & & 0.8779 & 0.2333 & 37.92 \\
    \cdashline{1-5}
    \textbf{\mname} & & \textbf{0.8978} & \textbf{0.1883} & \textbf{24.73} \\
    \bottomrule
    \end{tabular}
    \caption{\textbf{Reconstruction results} on the validation sets of MNIST (10,000 images) and CIFAR10 (10,000 images). All models are trained with the same experimental settings, except for the different quantisers.}%
    \label{tab:rec_quant}
\end{table}

\begin{table}[tb!]
    \centering
    \renewcommand{\arraystretch}{1.0}
    \setlength\tabcolsep{2pt}
    \begin{tabular}{@{}l c cccc @{}}
    \toprule
         \textbf{Method} &  \textbf{Dataset} & $\mathcal{S}$ $\downarrow$ & $\mathcal{K}$ $\downarrow$ & Usage $\uparrow$ & rFID $\downarrow$ \\
    \midrule
    VQGAN~\cite{esser2021taming} & \multirow{6}{*}{\rotatebox{90}{FFHQ}}& 16$^2$& 1024& $42\%$& 4.42\\
    ViT-VQGAN~\cite{yu2022vectorquantized} & & 32$^2$& 8192& --- & 3.13\\
    RQ-VAE~\cite{lee2022autoregressive} & & 16$^2\times$4 & 2048 & --- & 3.88 \\
    MoVQ~\cite{zhengmovq} & & 16$^2\times$4 & 1024 & $56\%$& 2.26$^*$ \\
    SeQ-GAN~\cite{gu2022rethinking} & & 16$^2$& 1024 & $100\%$& 3.12 \\
    \cdashline{1-6}
    \textbf{\mname} (ours) & & 16$^2$& 1024 & $100\%$ & \underline{2.80}\\
    \textbf{\mname} (ours) & & 16$^2\times$4 & 1024 & $100\%$ & \textbf{2.03}\\
    \midrule
    VQGAN~\cite{esser2021taming} & \multirow{6}{*}{\rotatebox{90}{ImageNet}}& 16$^2$& 1024 & $44\%$ & 7.94 \\
    ViT-VQGAN~\cite{yu2022vectorquantized} & & 32$^2$& 8192 & $96\%$& \underline{1.28} \\
    RQ-VAE~\cite{lee2022autoregressive} & & 8$^2\times$16& 16384 & --- & 1.83\\
    MoVQ~\cite{zhengmovq} & & 16$^2\times$4& 1024 & $63\%$& \textbf{1.12} \\
    SeQ-GAN~\cite{gu2022rethinking} & & 16$^2$ & 1024 & $100\%$ & 1.99\\
    \cdashline{1-6}
    \textbf{\mname} (ours) & & 16$^2$& 1024 & $100\%$ & 1.57 \\
    \textbf{\mname} (ours) & & 16$^2\times$4 & 1024 & $100\%$ & 1.20$^*$ \\
    \bottomrule
    \end{tabular}
    \caption{\textbf{Reconstruction results} on validation sets of ImageNet (50,000 images) and FFHQ (10,000 images). $\mathcal{S}$ denotes the latent size of encoded features, and $\mathcal{K}$ is the number of codevectors in the codebook. Usage indicates how many entries in a codebook are used during the quantisation on the validation set. More evaluation metrics are reported in Appendix Table \ref{tab:app_rec_sota}.}%
    \label{tab:rec_sota}
\end{table}

\begin{table*}[tb!]
    \centering
    \renewcommand{\arraystretch}{1.0}
    \setlength\tabcolsep{3pt}
    \begin{tabular}{@{}ll cccc cccc cccc@{}}
    \toprule
         &  \multirow{2}{*}{\textbf{Method}} & \multicolumn{3}{c}{\textbf{MNIST} (28$\times$28)} && \multicolumn{3}{c}{\textbf{CFAIR10} (32$\times$32)} && \multicolumn{3}{c}{\textbf{Fashion MNIST} (28$\times$28)}\\
         \cline{3-5}\cline{7-9}\cline{11-13}
         &  & SSIM $\uparrow$ & LPIPS $\downarrow$ & rFID $\downarrow$ && SSIM $\uparrow$ & LPIPS $\downarrow$ & rFID $\downarrow$ && SSIM $\uparrow$ & LPIPS $\downarrow$ & rFID $\downarrow$\\
    \midrule
    $\mathbb{A}$ & Baseline VQ-VAE~\cite{van2017neural}$_{\text{\scriptsize{NeurIPS'2017}}}$ & 0.9777 & 0.0282 & 3.43 && 0.8595 & 0.2504 & 39.67 && 0.9140 & 0.0801 & 12.73 \\
    \cdashline{1-13}
    $\mathbb{B}$ & + Cosine distance & 0.9791 & 0.0266 & 3.06 && 0.8709 & 0.2303 & 35.14 && 0.9160 & 0.0764 & 11.40 \\
    $\mathbb{C}$ & + Anchor initialization (offline) &  0.9810 & 0.0253 & 2.78 && 0.8829 & 0.2132 & 31.10 && 0.9145 & 0.0773 & 11.92 \\
    $\mathbb{D}$ & + Anchor initialization (online) & 0.9823 & 0.0236 & 2.23 && \textbf{0.8991} & 0.1897 & 26.62  && \textbf{0.9254} & \textbf{0.0683} & 9.27 \\
    $\mathbb{E}$ & + Contrastive loss &  \textbf{0.9833} & \textbf{0.0222} & \textbf{1.80} && 0.8978 & \textbf{0.1883} & \textbf{24.73} && 0.9233 & 0.0693 & \textbf{8.85}\\
    \bottomrule
    \end{tabular}
    \vspace{-0.1cm}
    \caption{\textbf{Results on various settings.} We report patch-level SSIM, feature-level LPIPS, and dataset-level FID\@. All evaluation metrics are reported in Appendix Table \ref{tab:app_rule}.}%
    \vspace{-0.3cm}
    \label{tab:rule}
\end{table*}

\paragraph{Quantitative Results:} We first evaluated our \mname and various quantisers, including VQ-VAE~\cite{van2017neural}$_{\text{\scriptsize{NeurIPS'2017}}}$, HVQ-VAE~\cite{williams2020hierarchical}$_{\text{\scriptsize{NeurIPS'2020}}}$, and SQ-VAE~\cite{takida2022sq}$_{\text{\scriptsize{ICML'2022}}}$, under the identical experimental settings in \cref{tab:rec_quant}. 
All instantiations of our model outperform the baseline variants of previous state-of-the-art models. 
Although the latest SQ-VAE~\cite{takida2022sq} optimizes all code entries by explicitly enforcing the codebook to be a defined distribution, this assumption may not hold for all datasets.
For instance, code entries that respond to the background elements like sky and ground should take more count than code entries that represent specific objects, such as vehicle wheels.
In contrast, our quantiser only encourages all code entries to be optimized, leaving the association to be automatically learned.

Then, we compared our \mname with the state-of-the-art methods, including VQGAN \cite{esser2021taming}$_{\text{\scriptsize{CVPR'2021}}}$, ViT-VQGAN \cite{yu2022vectorquantized}$_{\text{\scriptsize{ICLR'2022}}}$, RQ-VAE \cite{lee2022autoregressive}$_{\text{\scriptsize{CVPR'2022}}}$, and MoVQ \cite{zhang2018unreasonable}$_{\text{\scriptsize{NeurIPS'2022}}}$ for the task of reconstruction.
\Cref{tab:rec_sota} shows quantitative results on two large datasets.
Under the same compression ratio (768$\times$, \ie 256$\times$256$\times$3$\to$16$\times$16), our model significantly outperforms the state-of-the-art models, including the baseline VQGAN~\cite{esser2021taming} and the concurrent SeQ-GAN~\cite{gu2022rethinking}.
Interestingly, on the FFHQ dataset, our model even outperforms ViT-VQGAN~\cite{yu2022vectorquantized} and RQ-VAE~\cite{lee2022autoregressive}, which utilize 4$\times$ tokens for the representation.
This suggests that the high usage of codevectors is significant for maintaining information during data compression.
Additionally, we also run $4\times$ tokens experiments, as in MoVQ~\cite{zhengmovq}. The \mname further achieves a relative 10.1\% improvement.
Although our 4$\times$ version shows a slightly lower rFID score than the MoVQ~\cite{zhengmovq} on ImageNet dataset (1.12 \emph{vs.} 1.20), we achieve better performance on other metrics (as shown in Appendix \cref{tab:app_rec_sota}).

\begin{table*}[tb!]
    \centering
    \subfloat[
    \textbf{Codebook size}. The blobs' size is proportional to the number of codebook vectors \{32, 64, 128, 256, 512,  1024\}. The larger size naturally leads to better results in our \mname.
    \label{tab:rec_ab_codesize}
    ]{
    \centering
    \begin{minipage}{0.30\linewidth}
    {
    \begin{center}
        % Figure removed
        % Figure removed
    \end{center}
    }
    \end{minipage}
    }
    \hspace{0.5em}
    \subfloat[
    \textbf{Codebook dimensionality}. The blob's size refers to the dimensionality of codebook vectors \{4,8,16,32,64,128\}. The higher dimensionality does not ensure a better representation.
    \label{tab:rec_ab_codedim}
    ]{
    \centering
    \begin{minipage}{0.30\linewidth}
    {
    \begin{center}
        % Figure removed
        % Figure removed
    \end{center}
    }
    \end{minipage}
    }\hspace{0.5em}
    \subfloat[
    \textbf{Anchor sampling methods.}
    The choice of anchor sampling method has a significant impact on offline (one-time) feature initialization, while the online clustered method is robust for various samplings.%
    \label{tab:rec_ab_featm}
    ]{
    \centering
    \begin{minipage}{0.34\linewidth}
    {
    \begin{center}
        \renewcommand{\arraystretch}{1.15}
        \setlength\tabcolsep{3pt}
        \begin{tabular}{@{}l ccc  @{}}
        \toprule
        \multirow{2}{*}{\textbf{Method}} & \multirow{2}{*}{\textbf{Dataset}}& \multicolumn{2}{c}{rFID$\downarrow$}\\
        \cline{3-4}
        & & (offline) & (online) \\
        \midrule
        random & \multirow{4}{*}{MNIST} & 3.20 & 2.27 \\
        unique &&  2.84 & 2.24 \\
        probability &&  2.78 & \textbf{2.23} \\
        closest &&  \textbf{2.51} & 2.59 \\
        \midrule
        random & \multirow{4}{*}{CIFAR10}&  34.49 & 26.04 \\
        unique &&  36.99 & 26.02 \\
        probability &&  \textbf{31.10} & 26.62 \\
        closest &&  32.31 & \textbf{25.99} \\
        \bottomrule
        \end{tabular}
    \end{center}
    }
    \end{minipage}
    }
    \\
    \subfloat[
    \textbf{Codebook reinitialization methods.} In previous works~\cite{williams2020hierarchical,dhariwal2020jukebox}, each code entry is associated only with a single feature.
    \label{tab:rec_ab_reset}
    ]{
    \centering
    \begin{minipage}{\linewidth}
    {
    \begin{center}
        \renewcommand{\arraystretch}{1.2}
        \setlength\tabcolsep{5pt}
        \begin{tabular}{@{}l cccc cccc cccc @{}}
        \toprule
        \multirow{2}{*}{\textbf{Methods}}& \multicolumn{3}{c}{\textbf{MNIST}{ (28$\times$28)}} && \multicolumn{3}{c}{\textbf{CIFAR10}{ (32$\times$32)}} && \multicolumn{3}{c}{\textbf{FFHQ} (256$\times$256)} \\
        \cline{2-4}\cline{6-8}\cline{10-12}
         & SSIM $\uparrow$ & LPIPS $\downarrow$ & rFID$\downarrow$ && SSIM $\uparrow$ & LPIPS $\downarrow$ & rFID$\downarrow$&& SSIM $\uparrow$ & LPIPS $\downarrow$ & rFID$\downarrow$ \\
        \midrule
        near codevectors~\cite{williams2020hierarchical}  & 0.9790 & 0.0270 & 3.17 && 0.8553 & 0.2553 & 41.08 && 0.7282 & 0.1085 & 4.31 \\
        hard encoded features~\cite{dhariwal2020jukebox} & 0.9814 & 0.0243 & 2.25 && 0.8988 & 0.1978 & 29.16 && 0.7646 & 0.0870 & 3.91 \\
        running average (ours) & \textbf{0.9823} & \textbf{0.0236} & \textbf{2.23} && \textbf{0.8991} & \textbf{0.1897} & \textbf{26.62} && \textbf{0.8193} & \textbf{0.0603} & \textbf{2.94} \\
        \bottomrule
        \end{tabular}
    \end{center}
    }
    \end{minipage}
    }
    \caption{\textbf{Ablations} for \mname on image quantisation. We mainly train on MNIST and CIFAR10 training set, and evaluate on the validation set unless otherwise noted.}%
    \label{tab:rec_ab}
\end{table*}

\paragraph{Qualitative Results:}

The qualitative comparisons are presented in \cref{fig:rec}.
Our model achieves superior results even under challenging conditions.
Compared to the baseline model VQGAN~\cite{esser2021taming}, our \mname provides higher-quality reconstructed images that retain much more details.
In particular, VQGAN struggles with reconstructing abundant scene elements, as evidenced by the artifacts on the bowls.
In contrast, our \mname shows no such artifacts.
These fine-grained details are crucial for downstream generation-related tasks, such as generation, completion, and translation~\cite{zhengmovq}.

\subsection{Ablation Experiments}

We ran a number of ablations to analyse the effects of core factors in codebook learning.
Results are reported in \cref{tab:rule,tab:rec_ab,tab:app_rule,tab:app_rec_ab_featm}.

\paragraph{Core Factors.}

We evaluated core components in our redesigned online clustering quantiser in \cref{tab:rule}, which shows that the new quantiser considerably enhances the reconstruction quality.
We started by implementing the baseline configuration ($\mathbb{A}$) from VQ-VAE~\cite{van2017neural}.
Next, we explored different distance metrics, which are used to look up the closest entry for each encoded feature.
We found that using cosine similarity ($\mathbb{B}$) improved performance on some datasets, which is consistent with the findings in previous works such as ViT-VQGAN~\cite{yu2022vectorquantized}.
In configuration ($\mathbb{C}$), we reinitialized the unoptimized code entries with the selected anchors, but only in the first training batch, which we refer to as the \emph{offline} version.
This improved the usage of the codebook, resulting in slightly better gains.
Significantly, when we applied the proposed \emph{running average updates} across different training mini-batches in configuration ($\mathbb{D}$), the performance on all metrics in various datasets improved substantially.
This suggests that our proposed online clustering is significant for handling the changing encoded feature representation in deep networks.
Finally, we naturally introduced contrastive loss to each entry based on its similarity to features ($\mathbb{E}$), which further improved the results.

\paragraph{Codebook Size.}

VQ embeds the continuous features into a discrete space with a finite size, \ie $K$ codebook entries.
The codebook size has significant effects on traditional clustering. In \cref{tab:rec_ab_codesize}, we showed the performances of various quantisers with different codebook sizes.
Our \mname benefits greatly from a larger number of codebook entries, while SQ-VAE~\cite{takida2022sq} shows smaller improvements.
It is worth noting that \emph{not} all quantizers automatically benefit from a larger codebook size, such as VQ-VAE's performance on the CIFAR10 dataset shown in \cref{tab:rec_ab_codesize} (bottom).

\paragraph{Perplexity \emph{vs.} rFID.}

Recent concurrent studies~\cite{takida2022sq,gu2022rethinking,vuong2023vector} have explicitly promoted a large perplexity by optimizing a perplexity-related loss.
However, as illustrated in \cref{tab:rec_ab_codesize}, a larger perplexity does \emph{not} always guarantee a lower rFID\@.
This suggests that a uniform distribution of perplexity, represented by the highest score, may not be the optimal solution for the codebook.

\paragraph{Codebook Dimensionality.}

\Cref{tab:rec_ab_codedim} presents the results on various codebook dimensionalities.
Interestingly, the performance of the quantizers \emph{does not exhibit a straightforward positive correlation} with the number of codebook dimensionality.
In fact, some smaller codebook dimensionalities yield better results than larger ones, indicating that the choice of codebook dimensionality should be carefully considered depending on the specific application and dataset.
Based on this observation, a low-dimensional codebook can be employed to represent images and used in downstream tasks, as demonstrated in the latent diffusion model (LDM)~\cite{rombach2022high}.
The relevant downstream applications on generation can be found in \cref{sec:app}.

\paragraph{Anchor Sampling Methods.}

An evaluation of various \emph{anchor sampling methods} is reported in \cref{tab:rec_ab_featm}.
The results indicate that the \emph{offline} version with only one reinitialization is highly sensitive to the anchor sampling methods.
Interestingly, the random, unique, closest, and probabilistic random versions perform similarly for \emph{online} version, up to some random variations (rFID from 2.23 to 2.59 on MNIST, and from 25.99 to 26.62 on CIFAR10).
As discussed in \cref{sec:mth_online}, different anchor sampling methods have significant effects on traditional clustering~\cite{bradley1998refining,hamerly2002alternatives}.
However, our experiments demonstrate that the codebook reinitialization needs to consider the fact that \emph{the encoded features change along with the deep network is trained}.
The results highlight the effectiveness of our \emph{online} version with the running average updates, which is insensitive to the different instantiations.

\paragraph{Reinitialization Methods.}

Some latest works~\cite{williams2020hierarchical,dhariwal2020jukebox} also consider updating the unoptimized codevectors, called \emph{codebook reset}.
In \cref{tab:rec_ab_reset}, we compare these methods with VQ-VAE's architecture~\cite{van2017neural} under the same experimental setting, except for the different quantisers.
As discussed in \cref{sec:mth_online}, HVQ-VAE~\cite{williams2020hierarchical} resets the low usage codevectors using the high usage ones, which learns a narrow space codebook, resulting in limited improvement.
The hard encoded features presented in~\cite{dhariwal2020jukebox} achieve better results (3.17$\to$2.25, 41.08$\to$29.16, and 4.31$\to$3.91) than the HVQ-VAE~\cite{williams2020hierarchical} by adding noise signal to ensure the independent anchors for each codebook entry.
In contrast, our \mname calculates the running average updates, resulting in a significant improvement.
This further suggests that the online clustering centre along with the different training mini-batches is crucial for proper codebook reinitialization.

\section{Experiments: Applications}%
\label{sec:app}

% Figure environment removed

Except for data compression, our \mname can also be easily applied to downstream tasks, such as generation and completion.
Following existing works~\cite{esser2021taming,rombach2022high,zhengmovq}, we conduct a simple experiment to verify the effectiveness of the proposed quantisers.
Although this simple yet effective quantiser can be applied to more applications, it is beyond the main scope of this paper.

\begin{table}[tb!]
    \centering
    \setlength\tabcolsep{5pt}
        \begin{tabular}{@{}l cc @{}}
        \toprule
         \multirow{2}{*}{\textbf{Methods}} &  \multicolumn{2}{c}{FID$\downarrow$}\\
         \cline{2-3}
         & Churches & Bedrooms \\
         \midrule
         StyleGAN~\cite{karras2019style} & 4.21 & 2.35\\
         DDPM~\cite{ho2020denoising} & 7.89 & 4.90 \\
         ImageBART~\cite{esser2021imagebart} & 7.32 & 5.51 \\
         Projected-GAN~\cite{sauer2021projected} & \textbf{1.59} & \textbf{1.52} \\
         \midrule
         LDM~\cite{rombach2022high}-8$^*$ & 4.02 & - \\
         LDM~\cite{rombach2022high}-4 & - & 2.95 \\
         \midrule
         LDM~\cite{rombach2022high}-8 (reproduced) & 4.15 & 3.57 \\
         \mname-LDM~\cite{rombach2022high}-8 & 3.86 & 3.02 \\
         \bottomrule
    \end{tabular}
    \caption{\textbf{Quantitative comparisons on unconditional image generation.} The better quantiser can improve the generation quality without modifying the training settings in the second stage. $^*$: trained in $KL$-regularized latent space, instead of the VQ discrete space.}%
    \label{tab:ge}
\end{table}

\begin{table}[tb!]
    \centering
    % \renewcommand{\arraystretch}{0.9}
    \setlength\tabcolsep{5pt}
    \begin{tabular}{@{}l ccc cc@{}}
    \toprule
     \multirow{2}{*}{\textbf{Model}} & \multicolumn{2}{c}{\textbf{FFHQ}} && \multicolumn{2}{c}{\textbf{ImageNet}} \\
     \cline{2-3}\cline{5-6}
     & Steps & FID$\downarrow$ &&  Steps & FID$\downarrow$ \\
    \midrule
     % VQGAN [11]$_{\text{\scriptsize{CVPR'2021}}}$  & 256 & 11.4 && 256 & 15.78 \\  
     RQVAE [22]$_{\text{\scriptsize{CVPR'2022}}}$  & 256 & 10.38 && 1024 & 7.55 \\
     MoVQ [44]$_{\text{\scriptsize{NeurIPS'2022}}}$   & 1024 & 8.52 && 1024 & 7.13 \\
     SQ-VAE [33]$_{\text{\scriptsize{ICML'2022}}}$   & 200 & 5.17 && 250 & 9.31 \\
     % SeQGAN [13]$_{\text{\scriptsize{arXiv'2022}}}$   & 12 & 3.62 && 12 & 4.55 \\
     % \cdashline{1-6}
     LDM-4 [31]$_{\text{\scriptsize{CVPR'2022}}}$  & 200 & 4.98 && 250 & 10.56 \\
     \textbf{CVQ-VAE} (ours) & 200 & \textbf{4.46} && 250 &  \textbf{6.87} \\
    \bottomrule
    \end{tabular}
    \caption{Quantitative results for unconditional generation on FFHQ and \emph{class}-conditional generation on ImageNet.}
    \label{tab:ge-c}
\end{table}

\paragraph{Implementation Details.}

We made minor modifications to the baseline LDM~\cite{rombach2022high} system when adapting it with our quantiser for the downstream tasks.
We first replace the original quantiser from VQGAN~\cite{esser2021taming} with our proposed \mname's quantiser.
Then, we trained the models on LSUN~\cite{yu2015lsun} and ImageNet~\cite{deng2009imagenet} for generation (8$\times$ downsampling).
Following the setting in LDM~\cite{rombach2022high}, we set the sampling step as 200 during the inference. %, and on ImageNet~\cite{deng2009imagenet} for completion (16$\times$ downsampling).

\subsection{Unconditional Generation}

\Cref{tab:ge,tab:ge-c} compares our proposed \mname to the state-of-the-art methods on LSUN and ImageNet datasets for unconditional and \emph{class}-conditional image generation.
The results show that our model consistently improves the generated image quality under the sample compression ratio, as in the reconstruction.
This confirms the advantages of using a better codebook for downstream tasks.
Our \mname also outperforms the LDM-8$^*$ that is trained with $KL$-regularized latent space, indicating that exploring a better discrete codebook is worth pursuing for unsupervised representation learning.
Our \mname also achieves comparable results to LDM-4 (3.02 \emph{vs.} 2.95), whereas the LDM-4 uses a 4$\times$ higher resolution representation, requiring more computational costs.

Example results are presented in \cref{fig:ge}.
As we can see, even with 8$\times$ downsampling, the proposed \mname is still able to generate reasonable structures for these complicated scenes with various instances.
Although there are artifacts on windows in the two scenarios, the other high-frequency details are realistic, such as the sheet on the bed.

% % Figure environment removed
% \subsection{Image Completion}
% For image editing, the model is expected to generate plausibly reasonable content for the missing or masked holes, while preserving the original visible information as much as possible.

\section{Conclusion and Limitation}

We have introduced \mname, a novel codebook reinitialization method that tackles the \emph{codebook collapse} issue by assigning the online clustered anchors to unoptimized code entries.
Our proposed quantiser is a simple yet effective solution that can be integrated into many existing architectures for representation learning.
Experimental results show that our \mname significantly outperforms the state-of-the-art VQ models on image modeling, yet without increasing computational cost and latent size.
We hope this new plug-and-play quantiser will become an important component of future vector methods that use VQ in their learned architecture.
% quantised codebook learning.

\paragraph{Ethics.}

We use the MNIST, Fashion-MNIST, CIFAR10, LSUN, and ImageNet datasets in a manner compatible with their terms.
While some of these images contain personal information (\eg, faces) collected without consent, algorithms in this research do not extract biometric information.
For further details on ethics, data protection, and copyright please see \url{https://www.robots.ox.ac.uk/~vedaldi/research/union/ethics.html}.

\paragraph{Acknowledgements.}

This research is supported by ERC-CoG UNION 101001212.

{\small \bibliographystyle{ieee_fullname} \bibliography{egbib}}

\beginsupplement
\begin{refsection}

\section{Pseudocode Example of Cumulative Disruption Algorithm} \label{sec:psuedocode}

For readers seeking a succinct code-like description of our cumulative disruption curve algorithm, we have included \cref{lst:psuedocode}.

\begin{lstlisting}[label=lst:psuedocode, language=Python, caption=Pseudocode for disruption algorithm]
disruption = []
for c in communities:
    remaining = 0
    original = 0
    removeCommunity(c)
    for user in users:
        if degree(user) > 0:
            remaining += degree(user)
            original += originalDegree(user)
    disruption += [1 - (remaining / original)]
\end{lstlisting}

Note that when calculating disruption on large networks, it is much more efficient to cache the size of the smallest community that each user participates in. We can then sort all users by the order in which they will be removed, and avoid computationally expensive references to a graph or adjacency matrix for each removal-step in the algorithm.

\section{Applications to Unipartite Networks} \label{sec:unipartite}

Our influence metric is intended for settings with clearly defined communities. For example, participation in subreddits, membership on a Mastodon server, or committing to a software code repository, all discretely identify users as members of those explicitly-bounded groups. However, network data is often presented in a unipartite configuration such as users following other users. If it is still desirable to delineate communities and measure their influence in these settings, then they can be converted into compatible bipartite networks using the following procedure:

\begin{enumerate}
    \item Apply a context-appropriate community detection algorithm to label each user as belonging to one community

    \item Create a vertex for each community

    \item Replace all user-user edges with user-to-community edges, where the edge weight is equal to the number of unipartite edges each user had to other nodes in that community

    \item Apply our influence metric to the resulting bipartite graph
\end{enumerate}

An example of this procedure is illustrated in \cref{fig:unipartite}, using a unipartite Watts-Strogatz small-world network (100 nodes, 5 neighbors, rewiring probability of 5\%), and label-propagation for community detection. The unipartite graph is shown in the top-left with community labels visualized with color. It is converted to a bipartite representation shown in the upper-right, and the effect of removing each community is illustrated in the bottom frame.

% Figure environment removed






\section{Calculating the Area Under the Disruption Curve} \label{sec:auc_explanation}

For \cref{fig:real_networks_auc,fig:toy_networks_auc,fig:assortativity_auc} we use the area under the disruption curve as a single-variable summary of how centralized a network is around its largest communities. To calculate the AUC, we use a trapezoidal approximation in logarithmic space.

We chose a trapezoidal approximation to calculate the area even with limited sample points from real-world networks. Integration is possible for purely analytic disruption curve simulations as in \cref{sec:analytic_simulations}, but this is not feasible for our non-Erd\H{o}s-R\'{e}nyi networks, so we use a trapezoidal approximation for all synthetic networks for consistency.

We measure the AUC in logarithmic space, because measuring in linear space would heavily weight the influence of the smallest communities that are removed last, and our primary interest is in examining the influence of the largest communities on the broader population. 

\section{Synthetic Network Topology Details} \label{sec:toy_examples}

We measure centralization on a variety of synthetic networks introduced in \cref{sec:disruption_toy}. In this section, we include further description and visualization of the synthetic networks used.

Bipartite Near-Star networks are analogous to a unipartite star network with duplicate edges, but in a bipartite setting. Starting with a unipartite star, replace each edge from the hub to a leaf with a two-path from the hub community to a new ``user" vertex, to the leaf community. Duplicate edges from the unipartite hub to leaves are converted into multiple users that share a community, and serve to break ties when pruning communities for disruption curves. This is illustrated in \cref{fig:star}.

% Figure environment removed

For our ``Powerlaw" networks we follow a bipartite configuration model. We first create vertices representing the desired number of communities and users. We then draw from a powerlaw distribution with an assigned $\gamma$ exponent, and assign the drawn degree to each community. Then, we create a corresponding number of edges, wiring each community to users drawn uniformly at random without replacement. This yields networks where communities follow a powerlaw degree distribution, while users follow a normal degree distribution.

Bipartite community-user networks can be visualized in a flat plane, as in \cref{fig:centralization-pl}, or as a multi-layer graph, as in \cref{fig:pl-toy}. A multi-layer representation may be beneficial for representing inter-community relationships that are not explained by shared users, such as Mastodon federation agreements, or shared moderator staff in two subverses. However, these multiplex relationships were deemed out-of-scope for our current work.

% Figure environment removed




\begin{comment}
  #data structure for the dispersion metric
  D = np.zeros(nm)

  #calculate dispersion
  cumu_sum=0
  for n in np.arange(0,nm):
    cumu_sum += n*pn[n]
    #calculate U_n
    if pn[n]==0:
      continue
    Pnpm = Pnm[n,:]/np.sum(Pnm[n,:])
    U=0
    for m in np.arange(0,mm):
      if(np.sum(Pnm[:,m])>0):
        Pnmp = Pnm[:,m]/np.sum(Pnm[:,m])
        prob = np.sum(Pnmp[n:-1])
        U+=Pnpm[m]*prob**(m-1)
    D[n] = n*pn[n]*(1-U)/(cumu_sum-n*pn[n]*U)
\end{comment}

\section{Mathematical Analysis of Disruption in Random Networks} \label{sec:analytic_simulations}

We here calculate the disruption curves for random bipartite networks parameterized by their joint-degree distribution. This approach therefore fixes the distribution $\lbrace g_m \rbrace$ of communities $m$ per user, the distribution $\lbrace p_n \rbrace$ of community size $n$, and the joint-distribution $P_{n,m}$ for the degree of the node and community involved in a random bipartite link. Beyond these constraints, the networks are fully random but allow us to explore the role of heterogeneous connectivity at the user and community level as well as the impact of correlations between both levels.

We wish to calculate the disruption $D(n)$ involved when removing communities of size $n'<n$ in these random networks. By definition of the bipartite network, we know that $np_n$ edges are removed when removing communities of size $n$. Once again, we define disruption as the fraction of \textit{remaining} edges disrupted by communities of size $n$ during the pruning process. It is thus given by the number of edges that belong to communities of size $n$ minus the fraction $u_n$ of those that are the sole edge of the corresponding users (since these users are removed in the pruning) divided by the number of edges belonging to communities of size equal or smaller than $n$ minus the $u_nnp_n$ users removed. We write:

\vspace{2em}
\begin{equation}
    D(n) = \frac{
            \eqnmarkbox[NavyBlue]{bigedges}{np_n}
            -
            \eqnmarkbox[OliveGreen]{prunededges}{u_nnp_n}
        }{
            \eqnmarkbox[WildStrawberry]{remainingedges}{\sum_{n'\leq n}n'p_{n'}}
            -
            \eqnmarkbox[OliveGreen]{prunededges2}{u_nnp_n}
        } \; .
    % Here's Laurent's original expression
    %D(n) = \frac{np_n-u_nnp_n}{-u_nnp_n + \sum_{n'\leq n}n'p_{n'}} \; .
\end{equation}
\annotate[yshift=1em]{above,left}{bigedges}{Edges to comms. of size n}
\annotate[yshift=1em]{above,right}{prunededges}{Edges to removed users}
%\annotate[yshift=-1em]{below,right}{prunededges2}{Edges for removed users}
\annotate[yshift=-0.5em]{below}{remainingedges}{Edges to comms. n or smaller}
\vspace{2em}

The quantity $u_n$ can also be defined as the probability that a random user of a community of size $n$ has no community smaller than $n$. It can therefore be calculated like so:

\vspace{1em}
\begin{equation}
    u_n = \mathlarger{\sum}_m 
        \eqnmarkbox[NavyBlue]{users_in_n_with_m}{\frac{P_{n,m}}{\sum_{m'}P_{n,m'}}}
        \left(
            \eqnmarkbox[OliveGreen]{users_with_m_larger_than_n}{\frac{\sum_{n'\geq n} P_{n',m}}{\sum_{n'}P_{n',m}}}
        \right)^{m-1} \; .
    %u_n = \mathlarger{\sum}_m \frac{P_{n,m}}{\sum_{m'}P_{n,m}} \left(\frac{\sum_{n'\geq n} P_{n',m}}{\sum_{n'}P_{n',m}}\right)^{m-1} \; .
    \label{eq:un}
\end{equation}
\annotate[yshift=1em]{above,right}{users_in_n_with_m}{Fraction of users in comm. \\ \sffamily \footnotesize size n that have m edges}
\annotate[yshift=-0.5em]{below,left}{users_with_m_larger_than_n}{Fraction of users with m edges\\ \sffamily \footnotesize in comms. larger than size n}
\vspace{2.5em}

In the previous equation, we sum over every possible type of node in a community of size $n$, which will have a number of \textit{other} communities $m-1$ proportional to $P_{n,m}$, and ask for all of these communities to be larger or equal to $n$, which will be proportional to the sum of $P_{n',m}$ over all $n'$ larger or equal to $n$. Normalizing the probabilities appropriately yields Eq. (\ref{eq:un}) as written.

Note that these equations assume that edges are unweighted, and that there are no duplicate edges, which is what we expect from an infinite random simple graph. In our real-world data sets there are often duplicate edges (for example, one user following several different users on a Mastodon instance), which we compress to weighted edges for convenience.

Despite this difference between the analytical expression and real socio-technical networks, the analysis of random infinite graphs can be useful to test how disruption is impacted by simple network statistics such as degree distributions or correlations in the joint community-user degree matrix $P_{n,m}$. 

In a simple experiment, we create a random Erd\H{o}s-R\'{e}nyi-like bipartite network and correlated equivalent networks with the same degree distributions and variable community-user degree matrices $P_{n,m}$. The random network has a simple $P^{\textrm{rand}}_{n,m} \propto np_n mg_m$ (normalized) which we can modify manually. To do so, we calculate the maximally correlated $P^{\textrm{max}}_{n,m}$ by assigning users with highest degrees $m_{\textrm{max}}$ to the largest communities available before doing the same to users with the next higher degree and so on all the way down. We can do the same to calculate $P^{\textrm{min}}_{n,m}$ by assigning users with the lowest degree to the largest communities and working our way up in the user degree distribution. We can then create arbitrary community-user degree matrix $P_{n,m}$ by interpolating between linearly with $(1-\rho) P^{\textrm{rand}}_{n,m} + \rho P^{\textrm{max}}_{n,m}$ or $(1-\rho) P^{\textrm{rand}}_{n,m} + \rho P^{\textrm{min}}_{n,m}$.

Our results are shown in \cref{fig:assortivity_random_networks}. We find that positive user-community degree correlations increase disruption and therefore \textit{centralizes} the resulting socio-technical network. Conversely, negative correlations decreases correlations and \textit{decentralizes} the network. That being said, the relative effect of correlations is relatively small as the networks are still otherwise completely random.

% Figure environment removed

\section{Further Analysis of Assortativity} \label{sec:supplemental_assortativity}

There are multiple interpretations of degree assortativity in a bipartite setting. The linear correlation between user degrees and community degrees measures whether high-degree users are likely to be connected to high-degree communities. In our network definitions edges represent activity, like follow relationships or participation in conversations, so this measures whether active users are likely to be connected to communities with lots of activity. However, a second metric of interest is whether large communities are likely to be connected to other large communities, or in other words, the  assortativity of a unipartite-projected community-community graph. This can also be broken into two sub-cases: assortativity of community size (do communities with many users share users with other high-population communities), and assortativity of degree (do communities with lots of activity share users with other high-activity communities).

These three notions of assortativity are not independent; we might expect that users with lots of activity are active in communities with high populations, and may act as bridges between multiple communities with high activity and high population. However, the three metrics are not guaranteed to correlate and should be measured separately.

While rewiring to promote user-community degree assortativity, we also plotted the changes in community-community degree assortativity, shown in \cref{fig:assortivity_user_vs_community}. Strikingly, the community assortativity \textit{decreases} as we rewire to promote user assortativity. This is because as we rewire edges to focus user connections on the largest communities we implicitly decrease the number of edges between communities. This also matches the changes in disruption in \cref{fig:assortativity_auc}: increasing assortativity may reconnect large and insular communities with the rest of the network, briefly increasing their influence, but continued assortativity rewiring also cuts bridges to and between smaller communities, yielding a sparse network that is far less centralized.

% Figure environment removed

To further explore the relationship between these types of assortativity, we also rewired networks in the reverse direction: for randomly selected pairs of edges, we rewired those edges to \textit{decrease} user to community activity assortativity. We have plotted the change in disruption curves (\cref{fig:disassortative_auc}) and correlation between assortativity metrics (\cref{fig:disassortivity_user_vs_community}). In most networks, decreasing activity assortativity lowers centralization, although the effect diminishes as the network topology more closely approximates a random network. The one exception is the Penumbra; this network has such sparse inter-community connections that any perturbation of edges increases the cross-community links and therefore \textit{increases} centralization.

% Figure environment removed

% Figure environment removed

\section{Cumulative Impact on Giant Component Size} \label{sec:giant_components}

Some readers may be interested in how removing large communities influences the giant component size on each network. This is closely related to the cumulative population size in the top sub-plots of \cref{fig:real_networks_size_comparison} and \cref{fig:toy_networks_size_comparison}. Intuition suggests that the size of the giant component will be inversely proportional to the number of cumulative communities removed; as more large communities are pruned, the giant component should shrink. This relationship holds so long as the remaining communities are interlinked, but falters once a ``bridge" community is removed and the giant component splinters. Therefore, sparsely connected networks where bridges are more prominent will have a chaotic giant component size, while more densely connected networks will present a smooth curve until most communities are pruned. This relationship is illustrated in \cref{fig:real_giant_component}. Most curves are smooth until the tail of the distribution, with two notable exceptions: Voat's giant component changes once the largest insular communities are removed (see \cref{fig:voat_render}), and the Penumbra's curve is much ``spikier" as a result of its highly sparse structure.

% Figure environment removed

Measuring the change in giant component size captures some of the same features as our disruption metric. In particular, removing large insular communities may not change the giant component size if the community is completely isolated from the giant component, so this captures some aspect of both the size and topological role of a community. However, the impact of a community is boolean: if it touches the giant component, then removing the community will shrink the giant component by the size of that community. There is no distinction between a minimally integrated and tightly integrated community. Measuring the impact of a community in terms of fraction of edges severed, rather than component vertex size, offers finer insight into the interplay between size distribution and network structure.



\section{Comparison to Network Bottlenecking} \label{sec:cheeger}

The Cheeger number \cite{cheeger} is a single-valued metric representing how large of a ``bottleneck" inhibits conductance across a graph. It is typically written as:

\vspace{2em}
\begin{equation}
    h(G) = \min \left\{
        \frac{
                \eqnmarkbox[NavyBlue]{cheeger_crossedges}{|\partial A|}
            }{
                \eqnmarkbox[OliveGreen]{cheeger_alledges}{|A|}
            }
        : \eqnmarkbox[WildStrawberry]{cheeger_subset}{A \subseteq V(G)}, 
        \eqnmarkbox[Plum]{cheeger_bounds}{0 < |A| \leq \frac{1}{2} |V(G)|}
    \right\} 
\end{equation}
\annotate[yshift=1.2em]{above}{cheeger_crossedges}{Edges crossing the boundary of A}
\annotate[yshift=-0.2em]{below}{cheeger_alledges}{All edges in+across A}
\annotate[yshift=0.8em]{above}{cheeger_subset}{A is a subset of vertices of G}
\annotate[yshift=-2em]{below,left}{cheeger_bounds}{A contains at most half of all vertices}
\vspace{2em}

Our measurement of how much a community influences a larger population, and the Cheeger measurement of whether a community is a ``bottleneck" bear some conceptual similarities. Therefore, we compare our metric to the Cheeger number in two ways. First, we create a ``local Cheeger number," following an identical equation $\frac{|\partial A|}{|A|}$, but where $A$ is defined as the set of communities we are pruning, rather than via a global search. Second, we estimate bounds on the global Cheeger value of the graph. Since evaluating the graph conductance of all possible subsets of vertices is an NP-hard problem \cite{kaibel2004expansion}, it is impractical to directly measure the Cheeger constant on most large graphs. Fortunately, the Cheeger inequality offers upper and lower bounds on the Cheeger number based on the second eigenvalue of the normalized Laplacian of the adjacency matrix of G as follows:

$$\lambda_2/2 \leq h(G) \leq \sqrt{2\lambda_2}$$

Since they are sparse, these bounds can be calculated even on large real-world datasets. 
Unfortunately, in our tests the bounds are quite wide (see \cref{fig:cheeger}), limiting the utility of this approximation. We have plotted a comparison of the ``local" Cheeger number, bounds of the global Cheeger number, and our disruption metric, for a variety of simulated networks.

% Figure environment removed

\printbibliography[heading=subbibliography]
\end{refsection}


\end{document}