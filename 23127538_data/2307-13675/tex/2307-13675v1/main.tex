\documentclass{amsart}
\usepackage[utf8]{inputenc}
\usepackage{amssymb,amsmath,enumerate,amsthm}
\usepackage{colonequals}
\usepackage{stmaryrd}

\makeatletter
\@namedef{subjclassname@2020}{\textup{2020} Mathematics Subject Classification}
\makeatother

%\usepackage{showkeys}

\usepackage[colorlinks]{hyperref}
\hypersetup{
 colorlinks=true,
 linkcolor=blue,
 filecolor=magenta, 
 urlcolor=cyan,
}

\PassOptionsToPackage{hyphens}{url}\usepackage{hyperref}

\newcommand{\pat}[1]{{\color{green} #1}}
%\newcommand{\example}[1]{{\color{blue} #1}}

\usepackage[capitalise]{cleveref}
\crefformat{equation}{(#2#1#3)}
\crefrangeformat{equation}{(#3#1#4--#5#2#6)}
\crefformat{enumi}{(#2#1#3)}
\crefrangeformat{enumi}{(#3#1#4--#5#2#6)}

\newcommand{\creflastconjunction}{, and\nobreakspace}

\usepackage[pagewise]{lineno}
\overfullrule = 100pt
\let\oldequation\equation
\let\oldendequation\endequation
\renewenvironment{equation}{\linenomathNonumbers\oldequation}{\oldendequation\endlinenomath}
\expandafter\let\expandafter\oldequationstar\csname equation*\endcsname
\expandafter\let\expandafter\oldendequationstar\csname endequation*\endcsname
\renewenvironment{equation*}{\linenomathNonumbers\oldequationstar}{\oldendequationstar\endlinenomath}
\let\oldalign\align
\let\oldendalign\endalign
\renewenvironment{align}{\linenomathNonumbers\oldalign}{\oldendalign\endlinenomath}
\expandafter\let\expandafter\oldalignstar\csname align*\endcsname
\expandafter\let\expandafter\oldendalignstar\csname endalign*\endcsname
\renewenvironment{align*}{\linenomathNonumbers\oldalignstar}{\oldendalignstar\endlinenomath}

%%% TIKZ-CD --------------------------------------------------
\usepackage{tikz-cd}

%%% MATHSYMBOLS ----------------------------------------------
\makeatletter

\makeatother

%%% THEOREM STYLES --------------------------------------------------

\newcounter{intro}
\newtheorem{introthm}[intro]{Theorem}
\renewcommand{\theintro}{\Alph{intro}}
\newtheorem{introcor}[intro]{Corollary}
\renewcommand{\theintro}{\Alph{intro}}
\newtheorem{goal}[intro]{Goal}

\newcounter{result}
\newtheorem{theorem}[result]{Theorem}
\newtheorem{proposition}[result]{Proposition}
\newtheorem{lemma}[result]{Lemma}
\newtheorem{corollary}[result]{Corollary}
\newtheorem{example}[result]{Example}
\newtheorem{definition}[result]{Definition}
\newtheorem{chunk}[result]{Chunk}
\newtheorem{convention}[result]{Convention}
\newtheorem{problem}[result]{Problem}
\newtheorem{conjecture}[result]{Conjecture}
\newtheorem{remark}[result]{Remark}
\newtheorem{question}[result]{Question}
\newtheorem{construction}[result]{Construction}
\newtheorem{notation}[result]{Notation}

\newtheorem*{ack}{Acknowledgements}

\setcounter{secnumdepth}{2}
\numberwithin{result}{section}
\numberwithin{equation}{subsection}

%%% DOCUMENT INFORMATION --------------------------------------------------
\title[Generation \& Module categories]{Strong generation \& (Co)ghost index for module categories}

\author[P.~Lank]{Pat Lank}
\address{P.~Lank,
Department of Mathematics,
University of South Carolina, 
Columbia, SC 29208,
U.S.A.}
\email{plankmathematics@gmail.com}

\date{\today}

\keywords{derived categories, strong generator, module categories, coghost lemma, frobenius pushforward, local-to-global principle}

\subjclass[2020]{13D09, 13D05 13D02, 13F40, 18G80} 

\begin{document}

\begin{abstract}
    This work is concerned with both strong generation and (co)ghost index in the module category of a commutative noetherian ring. A sufficiency criterion is established for such rings to admit strong generators in their module category, and as a consequence, it answers affirmatively to a question of Iyengar and Takahashi. Moreover, it is shown that any noetherian quasi-excellent ring of finite krull dimension admits strong generators in their module category. Lastly, a local-to-global principle is established for (co)ghost index in the module category, and explicit computations are made.
\end{abstract}

\maketitle
\setcounter{tocdepth}{1}
\tableofcontents

%%%%%%%%%%%%%%%%%%%%%%%%%%%%%%%%%%%%%%%
\section{Introduction}
\label{sec:intro}
%%%%%%%%%%%%%%%%%%%%%%%%%%%%%%%%%%%%%%%

The study of homological properties of modules has served as a fruitful endeavour for understanding ring-theoretic properties. For instance, if $R$ is a commutative noetherian regular local ring of finite krull dimension $d$, then every finitely generated $R$-module admits a finite free resolution of length at most $d$. However, any such free resolution compactifies rich data describing how the module is constructed. If viewed from another perspective, then this is equivalent to saying any module can be \textit{built} from $R$ by using a finite combination of cokernels of monomorphisms, kernels of epimorphisms, and summands of finite direct sums. It is the latter mode of thought that has been fruitful in recent decades to enriching the apprehension for homological properties of a ring.

This work is concerned with notions of strong generation in the category $\operatorname{mod} R$ of finitely generated modules over a commutative noetherian ring $R$. By \textit{generation}, it is meant that there is an algorithmic style of computation which realizes any object in $\operatorname{mod}R$ as an output. There is a closely linked story to this regarding generation of the bounded derived category $D^b (\operatorname{mod}R)$. Introduced in \cite[$\S 2$]{BVdB:2003}, an object in $D^b (\operatorname{mod}R)$ is a \textit{classical generator} when every object in $D^b (\operatorname{mod}R)$ can be built from $G$ using a finite combination of finite direct sums, cones, and shifts. If the number of steps to finitely build any object $E\in D^b (\operatorname{mod}R)$ from $G$ can be done so uniformly, then $G$ is said to be \textit{strong}. 

The beauty of generation in $D^b (\operatorname{mod}R)$ is that it compartmentalizes the homological information into minuscule chunks. This means any object in $D^b (\operatorname{mod}R)$ can be realized as an output of a computation involving only three elementary operations applied to a single object. These techniques can be applied in the global setting for the derived category of bounded complexes with coherent cohomology $D^b (\operatorname{coh}X)$ over a noetherian scheme. Recently, it was shown if $X$ is a quasi-compact quasi-excellent separated scheme of finite krull dimension, then $D^b (\operatorname{coh}X)$ admits a strong generator (see \cite{Aoki:2021}). This enriches the work of \cite{BVdB:2003}, \cite{rouquier:2008}, \cite{Iyengar/Takahashi:2016}, \cite{KMVdB:2011}, \cite{Lunts:2010}, and \cite{Neeman:2021}. Applied to the affine setting, the result ensures $D^b (\operatorname{mod}R)$ admits strong generators whenever $R$ is a noetherian quasi-excellent ring of finite krull dimension. 

The scope of this work is an import of strong generation $D^b (\operatorname{mod}R)$ to the module-theoretic context. A notion of strong generators in $\operatorname{mod}R$ was introduced in \cite[$\S 4$]{Iyengar/Takahashi:2016}, but there are variations that could also be considered (i.e. \cite{Dao/Takahashi:2014,Beligiannis:2008}). An issue that arises is the non-existence of a shift operation for which triangulated categories enjoy, but the two algorithms to finitely build objects spelled out in \cite[$\S 4$]{Iyengar/Takahashi:2016} capture the intrinsic operations of abelian categories. The first uses extensions of objects, whereas the other refines this by allowing additionally kernels of epimorphisms, and cokernels of monomorphisms. These are discussed in greater detail within Section~\ref{sec:thick_subcategories}.

Given a full subcategory $\mathcal{S} \subseteq \operatorname{mod}R$, let $|\mathcal{S}|_1$ denote the smallest full subcategory of direct summands of finite direct sums of objects in $\mathcal{S}$. For $n\geq 2$, $|\mathcal{S}|_n$ consists of objects which are direct summands of objects $E$ appearing in a short exact sequence
\begin{displaymath}
    0 \to A \to E \to B \to 0
\end{displaymath}
where $A\in |\mathcal{S}|_{n-1},B\in |\mathcal{S}|_1$. An object $G\in \operatorname{mod}R$ is said to be a \textit{strong generator} when $R$ is a direct summand of $G$ and there exists $n,s\geq 0$ such that for any $E\in \operatorname{mod}R$ the $s$-syzygy $\Omega^s_R (E)\in |G|_n$. For instance, equicharacteristic excellent local rings or algebras which are essentially of finite type over a field admit strong generators in $\operatorname{mod}R$ (see \cite[$\S 5$]{Iyengar/Takahashi:2016}). This brings the attention to the first result which establishes a sufficient criterion for the existence of strong generators in $\operatorname{mod}R$. 

\begin{introthm}\label{introthm:strong_gen}
    Suppose $R$ is a noetherian ring with finite krull dimension. If $D^b (\operatorname{mod}R/\mathfrak{p})$ has finite rouquier dimension for every $\mathfrak{p}\in \operatorname{Spec}(R)$, then $\operatorname{mod}R/\mathfrak{p}$ admits a strong generator for every $\mathfrak{p}\in \operatorname{Spec}(R)$, and in particular, $\operatorname{mod}R$ admits a strong generator.
\end{introthm}

Theorem~\ref{introthm:strong_gen} should be compared to \cite[Theorem 1.1]{Iyengar/Takahashi:2019} where the existence of classical generators can be detected via a similar hypothesis to Theorem~\ref{introthm:strong_gen}. That is, $\operatorname{mod}R/\mathfrak{p}$ admits classical generators for all $\mathfrak{p}\in \operatorname{Spec}(R)$ if, and only if, $D^b (\operatorname{mod}R/\mathfrak{p})$ admits classical generators for all $\mathfrak{p}\in \operatorname{Spec}(R)$. However, regarding strong generation, \cite[Question 5.5]{Iyengar/Takahashi:2016} asks the following:

\begin{question}\label{q:Iyengar/Takahashi}
    If $R$ is an excellent noetherian ring, then does $\operatorname{mod}R$ admit strong generators?
\end{question}    

Theorem~\ref{introthm:strong_gen} not only responds affirmatively to Question~\ref{q:Iyengar/Takahashi}, but enlarges the class of noetherian rings for which strong generators exist.

\begin{introcor}\label{introcor:quasi_excellent_strong_gen}
    For any noetherian quasi-excellent ring $R$ of finite krull dimension, $\operatorname{mod}R$ admits a strong generator.
\end{introcor}

The story for strong generation of a noetherian quasi-excellent ring of finite krull dimension is rich as it is accessible in both the module category $\operatorname{mod}R$ and bounded derived category $D^b (\operatorname{mod}R)$. Ultimately, this provides new tools for studying the homological algebra of $R$. An upshot to being a strong generator in $\operatorname{mod}R$ is that it ensures the object is a strong generator in $D^b (\operatorname{mod}R)$ (see \cite[Lemma 7.1]{Iyengar/Takahashi:2016}). The proof of Theorem~\ref{introthm:strong_gen} is essentially an adaptation of \cite[Theorem 5.1]{Iyengar/Takahashi:2016} where the existence of strong generation in $D^b (\operatorname{mod}R)$ is invoked for noetherian quasi-excellent ring of finite krull dimensions. By allowing more operations to use in computations for finitely building objects from one another, this leads to studying subcategories which are closed under this process of generating.

A full triangulated subcategory $\mathcal{T} \subseteq D^b (\operatorname{mod}R)$ is \textit{thick} when it is closed under direct summands; and a full additive subcategory $\mathcal{S} \subseteq \operatorname{mod}R$ is \textit{thick} when it is closed under direct summands and if for every short exact sequence
\begin{displaymath}
    0 \to A \to B \to C \to 0
\end{displaymath}
with two of the three belonging to $\mathcal{S}$, then so does the third. Observe that a thick subcategory in $D^b (\operatorname{mod}R)$ is closed under the operations of direct summands, cones, and shifts whereas a thick subcategory in $\operatorname{mod}R$ is closed under direct summands, kernels of epimorphisms, cokernels of monomorphisms, and extensions. Both notions of thick capture subcategories being closed under operations that can be utilized to finitely build other objects from one another. Moreover, there is a bijection between thick subcategories of $D^b (\operatorname{mod}R)$ containing $R$ and thick subcategories of $\operatorname{mod}R$ containing all projective $R$-modules, and this is discussed further in Section~\ref{sec:thick_subcategories_module_category} (or \cite[Theorem 1]{Krause/Stevenson:2013}).

Given any collection of objects $\mathcal{G}\subseteq \operatorname{mod}R$, the smallest thick subcategory containing $\mathcal{G}$ exists, and is denoted by $\operatorname{thick}_{\operatorname{mod}R}(\mathcal{G})$. Fortunately, this subcategory comes with a filtration
\begin{displaymath}
    \begin{aligned}
        \{0\}=:&\operatorname{thick}_{\operatorname{mod}R}^0(\mathcal{G})\subseteq \operatorname{thick}_{\operatorname{mod}R}^1 (\mathcal{G})\subseteq \cdots \\&\subseteq \operatorname{thick}_{\operatorname{mod}R}^j(\mathcal{G}) \subseteq \cdots \\&\subseteq \cup^\infty_{n=0} \operatorname{thick}_{\operatorname{mod}R}^n (\mathcal{G}) = \operatorname{thick}_{\operatorname{mod}R}(\mathcal{G}).
    \end{aligned}
\end{displaymath}
This filtration keeps track of the number of steps required any object which may be finitely built from $\mathcal{G}$. For objects $E,G\in \operatorname{mod}R$ and integers $n,s\geq 0$ such that $\Omega^s_R (E)\in |G|_n$, it follows that $E\in \operatorname{thick}_{\operatorname{mod}R}^{n+s}(R \oplus G)$ which shows a relation of generating objects up to syzygys and on the nose (see \cite[Corollary 4.6]{Iyengar/Takahashi:2016}). This hints to another form of generation that is of interest: determine objects $N\in \operatorname{mod}R$ such that $\operatorname{thick}_{\operatorname{mod}R}^n(N)=\operatorname{mod}R$ for some $n\geq 0$. 

Luckily, a consequence of Theorem~\ref{introthm:strong_gen} is the possibility of explicitly writing out such objects under a mild hypothesis. Suppose $R$ is a commutative noetherian ring of prime characteristic $p$. If the frobenius morphism $F\colon R \to R$ (i.e. $r\mapsto r^p$) is finite, then it is excellent (see \cite[Theorem 2.5]{Kunz:1976}), and so strong generators in $\operatorname{mod}R$ exist by Theorem~\ref{introthm:strong_gen}. For instance, any essentially of finite type algebra over a perfect field of prime characteristic satisfies the condition of the frobenius morphism being finite. Recently, it has been shown that $F_\ast^e R$ is a strong generator in $D^b(\operatorname{mod}R)$ for $e \gg 0$ (see \cite[Corollary 3.9]{BILMP:2023}). In light of this fact, subsequently the next result naturally spurs.

\begin{introcor}\label{introcor:F_finite}
    Suppose $R$ is a noetherian ring of prime characteristic $p$. If the frobenius morphism $F \colon R \to R$ (i.e. $r\mapsto r^p$) is finite, then there exists $e,n\geq 0$ such that
    \begin{displaymath}
        \operatorname{thick}_{\operatorname{mod}R}^n (R\oplus F_\ast^e R) = \operatorname{mod}R
    \end{displaymath}
    where $F_\ast^e R$ denotes $R$ viewed as an $R$-module via restriction of the $e$-composition of $F \colon R \to R$.
\end{introcor} 

In a closely related direction, this work exhibits a local-to-global principle for (co)ghost index in module categories. For triangulated categories, \cite{OS:2012,Beligiannis:2008} established a connection between how many steps it takes to build on object to the vanishing of the composition for a certain class maps called (co)ghosts. A local-to-global principle for such vanishing conditions was established for $D^b (\operatorname{mod}R)$ in \cite{Letz:2021}. Following \cite{Beligiannis:2008}, (co)ghost maps in $\operatorname{mod}R$ can be made sensed of. Choose $E,G\in \operatorname{mod}R$. If $G\in \operatorname{mod}R$ and $f:E \to D$ is a map, then $f$ is \textit{$G$-ghost} when the induced map 
\begin{displaymath}
    \operatorname{Hom}_R (X,f): \operatorname{Hom}_R (X,E) \to \operatorname{Hom}_R (X,D)
\end{displaymath}
is zero for each $X \in |G|_1$. For $f:E \to D$, it is an \textit{$n$-fold $G$-ghost} when it may be written as a composition of $n$ $G$-ghost maps. If $E,G \in \mathcal{A}$, then the \textit{ghost index} of $E$ with respect to $G$ is the smallest $n\geq 0$ such that any $n$-fold $G$-ghost map from $E$ vanishes, and is denoted $\operatorname{gin}^G_\mathcal{A} (E)$. By dualizing everything mentioned, this yields a notion of \textit{coghost index}. The next main result of this work establishes a local-to-global principle.

\begin{introthm}
    If $E,G\in \operatorname{mod}R$, then the following are equivalent:
    \begin{enumerate}
        \item $\operatorname{gin}^G_{\operatorname{mod}R} (E)<\infty$;
        \item $\operatorname{gin}^{G_\mathfrak{m}}_{\operatorname{mod}R_\mathfrak{m}} (E_\mathfrak{m}) < \infty$ for all $\mathfrak{m}\in \operatorname{mSpec}(R)$;
        \item $\operatorname{gin}^{G_\mathfrak{p}}_{\operatorname{mod}R_\mathfrak{p}} (E_\mathfrak{p}) < \infty$ for all $\mathfrak{p}\in \operatorname{Spec}(R)$.
    \end{enumerate}
    A similar result holds for coghost index $\operatorname{cogin}^G_{\operatorname{mod}R} (E)$.
\end{introthm}

To motivate these values, observe that $\Omega^s_R (\operatorname{mod}R) \subseteq |G|_n$ implies
\begin{displaymath}
    \operatorname{gin}_{\operatorname{mod}R}^G (\Omega^s_R (E)),\operatorname{cogin}_{\operatorname{mod}R}^G (\Omega^s_R (E)) \geq n
\end{displaymath}
for each $E\in \operatorname{mod}R$. Hence, this produces lower bounds for the parameters $n,s$ of a strong generator in $\operatorname{mod}R$.

\begin{ack}
    The author would like to thank discussions with Srikanth B. Iyengar, Janina Letz, and Ryo Takahashi for their wonderfully valuable comments and perspectives shared that led to improvement on a draft of this manuscript.
\end{ack}

%%%%%%%%%%%%%%%%%%%%%%%%%%%%%%%%%%%%%%%
\section{Thick subcategories}
\label{sec:thick_subcategories}
%%%%%%%%%%%%%%%%%%%%%%%%%%%%%%%%%%%%%%%

This section recalls necessary content needed for generation in module categories and its corresponding bounded derived category. All rings considered are assumed to be commutative unital, and in the background a noetherian ring $R$ is fixed. Given a category $\mathcal{C}$ and subcategory $\mathcal{S} \subseteq\mathcal{S}$, $\mathcal{S}$ is said to be \textbf{strictly full} when it is a full subcategory closed under isomorphisms. 

%%%%%%%%%%%%%%%%%%%%%%%%%%%%%%%%%%%%%%%
\subsection{Algebra}
\label{sec:algebra}
%%%%%%%%%%%%%%%%%%%%%%%%%%%%%%%%%%%%%%%

To start working, the concourse that sets any stage for which this work takes place on needs to be specified. The category of finitely generated $R$-modules is denoted by $\operatorname{mod}R$, and its bounded derived category by $D^b (\operatorname{mod}R)$. If $D(R)$ is the unbounded derived category of complexes of $R$-modules (not necessarily finitely generated), then $D^b (\operatorname{mod}R)$ is a strictly full triangulated subcategory of $D(R)$. These two categories are the central themes for the results presented in Section~\ref{sec:strong_gen_excl} and Section~\ref{sec:co_ghost_mod_cats}. For the noetherian ring $R$, there exists a finite chain of ideals
\begin{displaymath}
    (0)=I_0 \subseteq I_1 \subseteq \cdots \subseteq R
\end{displaymath}
such that $I_{j+1}/I_j \cong R/\mathfrak{p}$ for some $\mathfrak{p}\in \operatorname{Spec}(R)$, and the smallest such value is $\lambda(R)$ (see \cite[Theorem 6.4]{Matsumura:1989}). Given a subcategory $\mathcal{S}\subseteq \operatorname{mod}R$, $\operatorname{add}\mathcal{S}$ denotes the strictly full subcategory of $\operatorname{mod}R$ consisting of direct summands of finite direct sums of copies of objects in $\mathcal{S}$. Perhaps familiar, for any $E\in \operatorname{mod}R$, an $n$-th syzygy of $E$ is denoted $\Omega^n_R (E)$. 

%%%%%%%%%%%%%%%%%%%%%%%%%%%%%%%%%%%%%%%
\subsection{Thick subcategories in $D^b(\operatorname{mod}R)$}
\label{sec:thick_subcategories_derived_category}
%%%%%%%%%%%%%%%%%%%%%%%%%%%%%%%%%%%%%%%

A triangulated category enjoys three defining features: cones of maps, shifts of objects, and finite coproducts. If retracts of finite coproducts are included, then there is a process which finitely builds objects from one another, and this section discusses how to do so utilizing these operations. The primary source of reference are \cite{BVdB:2003, Krause:2022, Huybrechts:2006}, and it should be noted that most of this discussion holds in a more general setting.

\begin{definition}\label{def:thick_subcategory_derived_category}
    A strictly full triangulated subcategory $\mathcal{S}\subseteq D^b(\operatorname{mod}R)$ is said to be \textbf{thick} when it is closed under direct summands.
\end{definition}

Definition~\ref{def:thick_subcategory_derived_category} captures special subcategories of $D^b(\operatorname{mod}R)$ which are closed under cones of maps, shifts of objects, and retracts of finite coproducts. For any $G\in D^b(\operatorname{mod}R)$, the smallest thick subcategory of $D^b(\operatorname{mod}R)$ containing $G$ is denoted $\operatorname{thick}_{D(R)} (G)$. This category always exists, and it can be filtered in the following way by inductively defining certain strictly full additive subcategories in $D^b(\operatorname{mod}R)$. Let $\operatorname{thick}^0_{D(R)} (G) := \{0\}$, and $\operatorname{thick}^1_{D(R)} (G)$ consist of objects which are direct summands of objects of the form $\oplus_{n\in \mathbb{Z}} G^{\oplus r_n}[n]$ whose differential is zero and $r_n\not=0$ for only finitely many integers. For $n\geq 2$, $\operatorname{thick}^n_{D(R)} (G)$ consists of direct summands of objects $E$ fitting into a distinguished triangle
\begin{displaymath}
    A \to E \to B \to A[1]
\end{displaymath}
where $A\in \operatorname{thick}^{n-1}_{D(R)} (G)$, $B\in \operatorname{thick}^1_{D(R)} (G)$. This yields an ascending chain of additive subcategories
\begin{displaymath}
    \{0\}=\operatorname{thick}^0_{D(R)} (G) \subseteq \operatorname{thick}^1_{D(R)} (G) \subseteq \cdots \operatorname{thick}^n_{D(R)} (G) \subseteq \cdots \subseteq \operatorname{thick}_{D(R)} (G)
\end{displaymath}
where $\cup^\infty_{n=0} \operatorname{thick}^n_{D(R)} (G) = \operatorname{thick}_{D(R)} (G)$. This filtration of $\operatorname{thick}_{D(R)} (G)$ is keeping track of the number of steps needed to finitely build an object from $G$.

\begin{definition}
    Given $G\in D^b (\operatorname{mod}R)$, if $\operatorname{thick}_{D(R)} (G) = D^b (\operatorname{mod}R)$, then $G$ is called a \textbf{classical generator}. Additionally, when there exists $n\geq 0$ such that $\operatorname{thick}^{n+1}_{D(R)} (G) = D^b (\operatorname{mod}R)$, then $G$ is called a \textbf{strong generator} and the smallest such value $n$ is its \textbf{generation time}, which is denoted $\operatorname{gen.time}(G)$.
\end{definition}

\begin{example}
    If $R$ is a quasi-excellent noetherian ring of finite krull dimension, then $D^b (\operatorname{mod}R)$ admit strong generators (see \cite[Main Theorem]{Aoki:2021}). For instance, if $R$ is regular, then $R$ finitely builds every $E \in D^b (\operatorname{mod}R)$ in $\dim R + 1$ steps.
\end{example}

\begin{definition}
    If $D^b(\operatorname{mod}R)$ admits a strong generator, then its \textbf{rouquier dimension} is the minimum over all possible generation times, and it is denoted $\dim D^b(\operatorname{mod}R)$.
\end{definition} 

\begin{example}
    If $R$ is a reduced noetherian ring, then 
    \begin{displaymath}
        \dim R -1 \leq \dim D^b(\operatorname{mod}R).
    \end{displaymath}
    This is \cite[Corollary 6.6]{Aihara/Takahashi:2015}.
\end{example}

%%%%%%%%%%%%%%%%%%%%%%%%%%%%%%%%%%%%%%%
\subsection{Thick subcategories in $\operatorname{mod}R$}
\label{sec:thick_subcategories_module_category}
%%%%%%%%%%%%%%%%%%%%%%%%%%%%%%%%%%%%%%%

An abelian category enjoys three important features: kernels of epimorphisms, cokernels of monomorphism, and finite coproducts of objects. These operations are the ingredients to describing an algorithm which finitely builds objects in $\operatorname{mod}R$ from one another in a similar fashion to what is presented in Section~\ref{sec:thick_subcategories_derived_category}. The primary source of reference is \cite[$\S 4$]{Iyengar/Takahashi:2016}.

\begin{definition}\label{def:thick_subcategory_module_category}
    A strictly full additive subcategory $\mathcal{S}\subseteq \operatorname{mod}R$ is \textbf{thick} when it is closed under direct summands and satisfies the two-out-of-three property, i.e. for any short exact sequence
    \begin{displaymath}
        0 \to A \to B \to C \to 0,
    \end{displaymath}
    if two of the three belong to $\mathcal{S}$, then so does the third.
\end{definition}

Any subcategory in $\operatorname{mod}R$ which satisfies Definition~\ref{def:thick_subcategory_module_category} is closed under the operations of kernels of epimorphisms, cokernels of monomorphism, and finite coproducts of objects. Given a subset $\mathcal{S}\subseteq \operatorname{mod}R$, the smallest thick subcategory in $\operatorname{mod}R$ containing $\mathcal{S}$ is denoted by $\operatorname{thick}_{\operatorname{mod}R} (\mathcal{S})$. This category always exists, and it is possible to decompose the information it contains in a precise way as follows. Let $\mathcal{S}\subseteq \operatorname{mod}R$. Set $\operatorname{thick}_{\operatorname{mod}R}^1(\mathcal{S}) := \operatorname{add}\mathcal{S}$ and $\operatorname{thick}_{\operatorname{mod}R}^0(\mathcal{S}) := \{ 0\}$. For $n\geq 2$, let $\operatorname{thick}_{\operatorname{mod}R}^n(\mathcal{S})$ be the strictly full subcategory of $\operatorname{mod}R$ whose objects are direct summands of an object fitting into a short exact sequence
\begin{displaymath}
    0 \to X \to Y \to Z \to 0
\end{displaymath}
where amongst the other two, one belongs to $\operatorname{thick}_{\operatorname{mod}R}^{n-1}(\mathcal{S})$ and the other in $\operatorname{thick}_{\operatorname{mod}R}^1(\mathcal{S})$. This forms an ascending chain of subcategories in $\operatorname{mod}R$:
\begin{displaymath}
    \operatorname{thick}_{\operatorname{mod}R}^0 (\mathcal{S})\subseteq \operatorname{thick}_{\operatorname{mod}R}^1 (\mathcal{S}) \subseteq \operatorname{thick}_{\operatorname{mod}R}^2(\mathcal{S})\cdots.
\end{displaymath}
It can be verified that this yields a filtration of the smallest thick subcategory containing $\mathcal{S}$ in $\operatorname{mod}R$,
\begin{displaymath}
    \operatorname{thick}_{\operatorname{mod}R} (\mathcal{S}) =  \cup^\infty_{n=0} \operatorname{thick}_{\operatorname{mod}R}^n (\mathcal{S}).
\end{displaymath}

There is a closely linked connection between thick subcategories containing all projective $R$-modules in $\operatorname{mod}R$ and thick subcategories of $D^b (\operatorname{mod}R)$ containing all bounded complexes finitely built by $R$.

\begin{lemma}\label{lem:thick_subcategory_krause_stevenson}
    There exists a bijection between thick subcategories of $\operatorname{mod}R$ containing all projective objects and thick subcategories of $D^b (\operatorname{mod}R)$ which contain all bounded complexes consisting of projective objects:
    \begin{displaymath}
    \begin{tikzcd}
	{\bigg \{ \begin{array}{l}     \textrm{thick subcategories }\ \\  \mathcal{S}\subseteq \operatorname{mod}R    \textrm{ where } R\in \mathcal{S}   \end{array} \bigg \}} && {\bigg \{ \begin{array}{l}     \textrm{thick subcategories }\ \\  \mathcal{T}\subseteq D^b(\operatorname{mod}R)    \textrm{ where } R\in \mathcal{T}   \end{array} \bigg \}}
	\arrow["{\mathcal{S}\mapsto \operatorname{thick}_{D(R)} (\mathcal{S})}", shift left=6, from=1-1, to=1-3]
	\arrow["{\mathcal{T}\cap \operatorname{mod}R \mapsfrom   \mathcal{T}}", shift left=6, from=1-3, to=1-1]
\end{tikzcd}
 \end{displaymath}
\end{lemma}

\begin{proof}
    This is \cite[Theorem 1]{Krause/Stevenson:2013}.
\end{proof}

%%%%%%%%%%%%%%%%%%%%%%%%%%%%%%%%%%%%%%%
\subsection{Generation in $\operatorname{mod}R$}
\label{sec:extension_construction}
%%%%%%%%%%%%%%%%%%%%%%%%%%%%%%%%%%%%%%%

The process discussed in Section~\ref{sec:thick_subcategories_module_category} describes a way to finitely build objects from one another using kernels of injective maps, cokernels of surjective maps, and retracts of finite direct sums. However, if instead only extensions of objects and retracts of finite direct sums were allowed, then a restricted analog of this process emerges. The corresponding refinement will serve as the backbone for a modified algorithm where objects are built up to syzygys as opposed to on the nose. Before doing so, there are constructions in order and the primary source is \cite[$\S 4$]{Iyengar/Takahashi:2016},

As opposed to thick subcategories in Definition~\ref{def:thick_subcategory_module_category}, there is a directed family of additive subcategories in $\operatorname{mod}R$ which aid in bookkeeping the number of extensions to finitely build objects. For any subcategory $\mathcal{S}\subseteq \operatorname{mod}R$, the strictly full subcategory on objects which are direct summands of finite direct sums of objects in $\mathcal{S}$ is denoted $\operatorname{add}(\mathcal{S})$. Utilizing extensions of objects in $\operatorname{add}(\mathcal{S})$, inductively construct an ascending chain of subcategories of $\operatorname{mod}R$ as follows (see \cite[Definition 4.1]{Iyengar/Takahashi:2016} for details). Set $|\mathcal{S}|_1 := \operatorname{add}\mathcal{S}$ and $|\mathcal{S}|_0:= 0$. For $n\geq 2$, let $|\mathcal{S}|_n$ be the strictly full subcategory of $\operatorname{mod}R$ whose objects $M$ fit into a short exact sequence
\begin{displaymath}
    0 \to Y \to M\oplus W \to X \to 0
\end{displaymath}
where $Y\in |\mathcal{S}|_{n-1}$ and $X \in |\mathcal{S}|_1$. These subcategories form an ascending chain:
\begin{equation}\label{eq:extension_subcategories_filtration}
    |\mathcal{S}|_1 \subseteq |\mathcal{S}|_2 \subseteq \cdots.
\end{equation}
If $|\mathcal{S}|_a \star |\mathcal{S}|_b$ denotes the full subcategory whose objects $E$ fit into a short exact sequence
\begin{displaymath}
    0 \to X \to E \oplus E' \to Y \to 0
\end{displaymath}
where $X \in |\mathcal{S}|_a$ and $Y \in |\mathcal{S}|_b$, then $|\mathcal{S}|_a \star |\mathcal{S}|_b = |\mathcal{S}|_{a+b}$ (see \cite[$\S 5$]{Dao/Takahashi:2014}).

After introducing the filtration in Equation~\ref{eq:extension_subcategories_filtration}, the necessary components have been set for the refined algorithm of finitely building objects using only extensions of objects and retracts of finite coproducts.

\begin{definition}\label{def:strong_generator_module_category}
    If $G\in \operatorname{mod}R$, then it is a \textbf{strong generator} when $R$ is a direct summand of $G$ and $\Omega^s_R (\operatorname{mod}R)\subseteq |G|_n$ for some $s,n\geq 0$ where
    \begin{displaymath}
    \Omega_R^s (\operatorname{mod}R) := \{ \Omega^s_R (M) : M\in \operatorname{mod}R \}.
\end{displaymath}
\end{definition}

There is an intimate relationship between the algorithm discussed in Section~\ref{sec:thick_subcategories_module_category} and Definition~\ref{def:strong_generator_module_category}.

\begin{lemma}
    If $G\in \operatorname{mod}R$, and $s,n\geq 0$, then
    \begin{itemize}
        \item $|G|_n \subseteq \operatorname{thick}_{\operatorname{mod}R}^n (G)$;
        \item $\Omega_R^s (\operatorname{mod}R) \subseteq |G|_n \implies \operatorname{thick}_{\operatorname{mod}R}^{s+n} (R\oplus G) = \operatorname{mod}R$.
    \end{itemize}
\end{lemma}

\begin{proof}
    This is \cite[Proposition 4.5 \& Corollary 4.6]{Iyengar/Takahashi:2016}.
\end{proof}

The following are instances where strong generators in $\operatorname{mod}R$ exist, and being able to expand what is known serves as a motivating objective of this work.

\begin{example}\label{ex:Iyengar/Takahashi_strong_gen_examples}
Let $R$ be a noetherian ring of finite krull dimension $d$.
\begin{itemize}
    \item If $R$ is an artinian ring, then $\operatorname{mod}R \subseteq |R\oplus R/J(R)|_{\ell\ell(R)}$ where $J(R)$ is the jacobson radical of $R$ and $\ell\ell(R)$ is the loewy length.
    \item If $R$ is an excellent equicharacteristic local ring, then $\operatorname{mod}R$ admits a strong generator (see \cite[Theorem 5.3]{Iyengar/Takahashi:2016}).
    \item If $R$ is essentially of finite type over a field, then $\operatorname{mod}R$ admits a strong generator (see \cite[Theorem 5.4]{Iyengar/Takahashi:2016}).
\end{itemize}
\end{example}

%%%%%%%%%%%%%%%%%%%%%%%%%%%%%%%%%%%%%%%
\section{Strong generators}
\label{sec:strong_gen_excl}
%%%%%%%%%%%%%%%%%%%%%%%%%%%%%%%%%%%%%%%

This section answers Question~\ref{q:Iyengar/Takahashi} affirmatively, and some of its consequences are studied. In doing so, a connection between strong generators in $\operatorname{mod}R$ and $D^b (\operatorname{mod}R)$ is established, which yields a larger class of rings whose module categories admit strong generators than predicted by Question~\ref{q:Iyengar/Takahashi}. For instance, any noetherian quasi-excellent ring of finite krull dimension satisfy such conditions (see Corollary~\ref{cor:quasi_excellent_strong_gen}). The following main result of this section is stated below, and its proof will be broken into smaller statements. 

\begin{theorem}\label{thm:strong_generation}
    For a noetherian ring $R$ with finite krull dimension, the following are equivalent:
    \begin{itemize}
        \item $\operatorname{mod}R/\mathfrak{p}$ admits a strong generator for every $\mathfrak{p}\in \operatorname{Spec}(R)$;
        \item $D^b (\operatorname{mod}R/\mathfrak{p})$ has finite rouquier dimension for every $\mathfrak{p}\in \operatorname{Spec}(R)$.
    \end{itemize}
    If any one of these conditions are satisfied, then both $\operatorname{mod}R$ and $D^b (\operatorname{mod}R)$ admit strong generators.
\end{theorem}

\begin{proof}
    For the first claim, the forward direction is Lemma~\ref{lem:strong_generation_implies_finite_rouquier_dimension}, whereas its converse is Proposition~\ref{prop:integral_quotients_strong_generation_implies_strong_generation}. By Proposition~\ref{prop:strong_generation_via_minimal_primes} and Remark~\ref{rmk:strong_generator_to_derived_category}, the last claim holds.
\end{proof}

Theorem~\ref{thm:strong_generation} should be compared to \cite[Theorem 1.1]{Iyengar/Takahashi:2019} where the existence of classical generators can be detected via a similar hypothesis. That is, $\operatorname{mod}R/\mathfrak{p}$ admits a classical generator for all $\mathfrak{p}\in\operatorname{Spec}(R)$ if, and only if, $D^b(\operatorname{mod}R/\mathfrak{p})$ admits a classical generator for all $\mathfrak{p}\in\operatorname{Spec}(R)$. As previously mentioned, the proof of Theorem~\ref{thm:strong_generation} will be accomplished in steps, but before doing so an important digression is made with cohomology annihilator ideals.

%%%%%%%%%%%%%%%%%%%%%%%%%%%%%%%%%%%%%%%
\subsection{Cohomology annihilator ideals}
\label{sec:cohomology_annihilator_ideals}
%%%%%%%%%%%%%%%%%%%%%%%%%%%%%%%%%%%%%%%

For each integer $n\geq0$, consider the following ideal of $R$:
\begin{displaymath}
    \operatorname{ca}^n (R):= \operatorname{ann}_{R^c}\operatorname{Ext}^{\geq n}_R (\operatorname{mod}R,\operatorname{mod}R).
\end{displaymath}
The \textbf{cohomology annihilator (ideal)} of $R$ is given by
\begin{displaymath}
    \operatorname{ca}(R) := \cup^\infty_{n=0}\operatorname{ca}^n (R).
\end{displaymath}
These ideals are closely related to rouquier dimension of $\dim D^b (\operatorname{mod}R)$, see \cite{Iyengar/Takahashi:2016} and \cite{BHST:2016}. For instance, \cite[Theorem 5.3]{Iyengar/Takahashi:2016} establishes a proof for Question~\ref{q:Iyengar/Takahashi} in the equicharacteristic excellent local ring case using a nonvanishing of cohomology annihilator ideals. These ideals will be vital for proving Theorem~\ref{thm:strong_generation}, and useful statements regarding their relevance to this problem are stated below.

\begin{remark}\label{rmk:rouquier_dim_cohomology_annihilator_nonzero}
    If $R$ is a noetherian integral domain such that $D^b (\operatorname{mod}R)$ has finite rouquier dimension $r$, then the $(r+1)$-th cohogomology annihilator ideal $\operatorname{ca}^{r+1} (R)$ is nonzero. This is \cite[Theorem 5]{Elagin/Lunts:2018}, and note a triangulated category being \textbf{regular} means the existence of strong generators.
\end{remark}

\begin{remark}\label{rmk:regular_element_syzygy_quotient}
    Let $R$ be a noetherian ring, and set $x\in R$ to be an $R$-regular element. If $M\in \operatorname{mod}R$, then there exists an $R$-module isomorphism for all $n\geq 0$
    \begin{displaymath}
        \Omega_{R/xR}^n (\Omega^1_R (M)/ x\Omega^1_R (M)) \cong \Omega_R^{n+1} (M)/ x \Omega_R^{n+1} (M).
    \end{displaymath}
    This is \cite[Lemma 5.6]{Dao/Takahashi:2014}.
\end{remark}

\begin{remark}\label{rmk:ses_cohomology_annihilator}
    Let $R$ be a noetherian ring, and $M\in \operatorname{mod}R$. If $a\in R$ annihilates $\operatorname{Ext}_R^1(M,\Omega^1_R (M))$, then there exists a short exact sequence in $\operatorname{mod}R$,
    \begin{displaymath}
        0 \to (0:_M a) \to M \oplus \Omega^1_R (M) \to \Omega_R (M/a M) \to 0.
    \end{displaymath}
    This is \cite[Remark 2.12]{Iyengar/Takahashi:2016}.
\end{remark}

\begin{example}\label{ex:Dieterich_hypersurface_cohomology_annihilator}
    Let $k$ be a field, and $R=k \llbracket  x_0,\ldots,x_n \rrbracket  $. If $f\in R$, then 
    \begin{displaymath}
        (\frac{\partial f}{\partial x_0}, \ldots, \frac{\partial f}{\partial x_n})\subseteq \operatorname{ca}^{d+1} (R/f R).
    \end{displaymath}
    This is \cite[Proposition 18]{Dieterich/1987}.
\end{example}

\begin{example}\label{ex:BHST_completion_cohomology_annihilator}
    If $(R,\mathfrak{m})$ is a noetherian local ring of finite krull dimension and $n\geq 0$, then $\operatorname{ca}^n (\widehat{R})\cap R\subseteq \operatorname{ca}^n (R)$ where $\widehat{R}$ denotes the $\mathfrak{m}$-adic completion of $R$. This is \cite[Theorem 4.5.1]{BHST:2016}. 
\end{example}

%%%%%%%%%%%%%%%%%%%%%%%%%%%%%%%%%%%%%%%
\subsection{Results}
\label{sec:strong_generators_results}
%%%%%%%%%%%%%%%%%%%%%%%%%%%%%%%%%%%%%%%

After discussing cohomology annihilator ideals in Section~\ref{sec:cohomology_annihilator_ideals}, the strategy for developing a proof of Theorem~\ref{thm:strong_generation} commences. The first step is relating the content of Section~\ref{sec:thick_subcategories} which introduced various ways of how objects can be finitely built from one another in $\operatorname{mod}R$ and $D^b (\operatorname{mod}R)$.

\begin{remark}\label{rmk:strong_generator_to_derived_category}
    Let $R$ be a noetherian ring. If there exists $G\in \operatorname{mod}R$ and $n,s\geq 0$ such that $\Omega^s_R (\operatorname{mod}R)\subseteq |G|_n$, then
    \begin{displaymath}
        \operatorname{mod}R = \operatorname{thick}_{\operatorname{mod}R}^{n+s} (R\oplus G)
    \end{displaymath}
    and
    \begin{displaymath}
        D^b (\operatorname{mod}R) = \operatorname{thick}_{D(R)}^{2(n+s)} (R\oplus G).
    \end{displaymath}
    These are respectively \cite[Corollary 4.6]{Iyengar/Takahashi:2016} and \cite[Lemma 7.1]{Iyengar/Takahashi:2016}.
\end{remark}

\begin{lemma}\label{lem:strong_generation_implies_finite_rouquier_dimension}
    Consider a noetherian ring $R$ with finite krull dimension. If $\operatorname{mod}R/\mathfrak{p}$ admits a strong generator for every $\mathfrak{p}\in \operatorname{Spec}(R)$, then $D^b (\operatorname{mod}R/\mathfrak{p})$ has finite rouquier dimension for every $\mathfrak{p}\in \operatorname{Spec}(R)$.
\end{lemma}

\begin{proof}
    Choose $\mathfrak{p}\in \operatorname{Spec}(R)$. If $G\in \operatorname{mod}R/\mathfrak{p}$ is a strong generator, then there exists $n,s\geq 0$ such that $\Omega^s_{R/\mathfrak{p}}\subseteq |G|_n$. By Remark~\ref{rmk:strong_generator_to_derived_category}, $D^b(\operatorname{mod}R/\mathfrak{p})$ admits a strong generator, and so it has finite rouquier dimension.
\end{proof}

\begin{lemma}\label{lem:ext_prop}
    If $E\in |G|_s$, then $|E|_n \subseteq|G|_{ns}$ for all $n\geq 1$.
\end{lemma}

\begin{proof}
    This is shown by induction on $n$. There exists a short exact sequence
    \begin{displaymath}
        0 \to A \to E \oplus M \to B \to 0
    \end{displaymath}
    where $A \in |G|_{s-1}$ and $B\in |G|_1$. By taking direct sums of this short exact sequence, it verifies $(E\oplus M)^{\oplus \alpha}\in |G|_s$ for all $\alpha>0$. Hence, $|E|_1 \subseteq|G|_s$, and this establishes the base case. Assume the claim holds for all $1\leq j \leq n$. If $S\in |E|_n$, then there exists a short exact sequence
    \begin{displaymath}
        0 \to X \to S \oplus N \to Y \to 0
    \end{displaymath}
    where $X \in |E|_n$ and $Y\in |E|_1$. The induction step ensures $X \in |G|_{ns}$ and $Y\in |G|_s$, which implies $S\in |G|_{(n+1)s}$ as desired.
\end{proof}

\begin{remark}\label{rmk:ses_to_syzygy_ses}
    Let $R$ be a noetherian ring. If there exists a short exact sequence
    \begin{displaymath}
        0 \to L \to M \to N \to 0
    \end{displaymath}
    in $\operatorname{mod}R$ and $n\geq 0$, then there exists a short exact sequence
    \begin{displaymath}
        0 \to \Omega^n_R (L) \to \Omega^n_R(M) \to \Omega^n_R(N) \to 0
    \end{displaymath}
    in $\operatorname{mod}R$. This is \cite[Remark 2.2]{Iyengar/Takahashi:2016}. Choose an ideal $I\subseteq R$, and let $M\in \operatorname{mod}R/I$. There exists a short exact sequence
    \begin{displaymath}
        0 \to E \to \Omega_R^1 (M) \to \Omega^1_{R/I}(M) \to 0
    \end{displaymath}
    where $E\in \operatorname{add}(I)$. Furthermore, for each $n\geq 0$,
    \begin{displaymath}
        \Omega_R^n (M) \in |\Omega^n_{R/I} (M) \oplus \big( \oplus^{n-1}_{i=0}\Omega_R^i (I)\big)|_{n+1}.
    \end{displaymath}
    This is \cite[Proposition 5.3]{Dao/Takahashi:2014}.
\end{remark}

\begin{proposition}\label{prop:strong_generation_via_minimal_primes}
    Let $R$ be a noetherian ring with finite krull dimension. If $\operatorname{mod}R/\mathfrak{p}$ admits a strong generator for every $\mathfrak{p}\in \operatorname{Spec}(R)$, then $\operatorname{mod}R$ admits a strong generator.
\end{proposition}

\begin{proof}
    If $R$ is an integral domain, then there exists nothing to show, so assume it is not so. There exists a chain of ideals
    \begin{displaymath}
        (0)=I_0 \subseteq I_1 \subseteq \cdots \subseteq I_m = R
    \end{displaymath}
    such that $I_{j+1}/I_j \cong R/\mathfrak{p}_j$ for some $\mathfrak{p}_j\in\operatorname{Spec}(R)$ and $m= \lambda(R)$. If $R$ is not an integral domain, then $m>0$. The hypothesis ensures for each $1\leq j \leq m$ there exists $n_j, s_j\geq 0$ and $G_j \in \operatorname{mod}R/\mathfrak{p}_j$ such that $\Omega^{s_j}_{R/\mathfrak{p}_j} (\operatorname{mod}R/\mathfrak{p}_j)\subseteq |G_j|_{n_j}$. For any $M\in \operatorname{mod}R$ and $0 \leq j \leq m-1$, there exists a short exact sequence
    \begin{displaymath}
        0 \to I_j M \to I_{j+1} M \to I_{j+1}M/I_j M \to 0.
    \end{displaymath}
    If $s:=\max\{s_1,\ldots,s_m\}$, then for each $1\leq j \leq m$, 
    \begin{displaymath}
        \Omega^s_{R/\mathfrak{p}_j} (\operatorname{mod}R/\mathfrak{p}_j)\subseteq |\Omega^{s-s_j}_{R/\mathfrak{p}_j} G_j|_{n_j}.
    \end{displaymath}
    If $n:=1+ \max\{ n_1,\ldots,n_m\}$, then $|\Omega^{s-s_j}_{R/\mathfrak{p}_j} G_j|_{n_j} \subseteq |\Omega^{s-s_j}_{R/\mathfrak{p}_j} G_j|_n$. Observe there is a uniform choice of $n,s$ as in for each $1\leq j \leq m$, $\Omega^s_{R/\mathfrak{p}_j} (\operatorname{mod}R/\mathfrak{p}_j)\subseteq  |\Omega^{s-s_j}_{R/\mathfrak{p}_j} G_j|_n$. By abuse of notation, identify $G_j$ with $\Omega^{s-s_j}_{R/\mathfrak{p}_j} G_j$. For some choice of an $s$-th syzygy, there exists a short exact sequence via Remark~\ref{rmk:ses_to_syzygy_ses}
    \begin{displaymath}
        0 \to \Omega_R^s (I_j M) \to \Omega_R^s ( I_{j+1} M) \to \Omega_R^s ( I_{j+1}M/I_j M) \to 0.
    \end{displaymath}
    By Remark~\ref{rmk:ses_to_syzygy_ses} and Lemma~\ref{lem:ext_prop}, it can be verified for each $0\leq j \leq m-1$,
    \begin{displaymath}
        \begin{aligned}
            \Omega^s_R (I_{j+1} M/I_j M) &\in |\Omega^s_{R/\mathfrak{p}_j}(I_{j+1} M/I_j M) \oplus \big( \bigoplus_{i=0}^{s-1} \Omega_R^i (\mathfrak{p}_j) \big)|_{s+1} \\&\subseteq |G_j \oplus \big( \bigoplus_{i=0}^{s-1} \Omega_R^i (\mathfrak{p}_j) \big)|_{n(s+1)}.
        \end{aligned}
    \end{displaymath}
    Another application of Lemma~\ref{lem:ext_prop} with an inductive argument exhibits
    \begin{displaymath}
        \Omega^s_R (I_{j+1} M) \in |\bigoplus^j_{i=0} \big(G_i \oplus (\bigoplus_{l=0}^{s-1} \Omega_R^l (\mathfrak{p}_i)) \big)|_{ n(s+1)(j+1)},
    \end{displaymath}
    and after working up the ladder of extensions,
    \begin{displaymath}
        \Omega^s_R (\operatorname{mod}R) \subseteq |\bigoplus^m_{i=1} \big(G_i \oplus (\bigoplus_{l=0}^{s-1} \Omega_R^l (\mathfrak{p}_i)) \big)|_{ m n (s+1)}.
    \end{displaymath}
    Therefore, it follows that
    \begin{displaymath}
        R\oplus \bigoplus^m_{i=1} \big(G_i \oplus (\bigoplus_{l=0}^{s-1} \Omega_R^l (\mathfrak{p}_i)) \big)
    \end{displaymath}
    is a strong generator for $\operatorname{mod}R$.
\end{proof}

\begin{proposition}\label{prop:integral_quotients_strong_generation_implies_strong_generation}
    Consider a noetherian ring $R$ with finite krull dimension. If $D^b (\operatorname{mod}R/\mathfrak{p})$ has finite rouquier dimension for every $\mathfrak{p}\in \operatorname{Spec}(R)$, then $\operatorname{mod}R/\mathfrak{p}$ admits a strong generator for every $\mathfrak{p}\in \operatorname{Spec}(R)$.
\end{proposition}

\begin{proof}
    The claim will be proved by induction on krull dimension. That is, if $R$ a noetherian ring of finite krull dimension satisfying the hypothesis $D^b (\operatorname{mod}R/\mathfrak{p})$ has finite rouquier dimension for every $\mathfrak{p}\in \operatorname{Spec}(R)$, then $\operatorname{mod}R/\mathfrak{p}$ admits a strong generator for every $\mathfrak{p}\in \operatorname{Spec}(R)$.

    If $\dim R = 0$, then $R$ is artinian. For each $\mathfrak{p}\in \operatorname{Spec}(R)$, $R/\mathfrak{p}$ is a field, and this ensures $\operatorname{mod}R/\mathfrak{p}$ is strongly generated by $R/\mathfrak{p}$. Note that $D^b (\operatorname{mod}R/\mathfrak{p})$ is also strongly generated by $R/\mathfrak{p}$. Hence, the base case holds. Assume there exists $d>0$ such that the claim holds for any noetherian ring $S$ with krull dimension strictly less than $d$ satisfying the hypothesis $D^b (\operatorname{mod}S/\mathfrak{p})$ has finite rouquier dimension for every $\mathfrak{p}\in \operatorname{Spec}(S)$.

    Suppose $R$ is a noetherian ring of krull dimension $d$ satisfying the hypothesis $D^b (\operatorname{mod}R/\mathfrak{p})$ has finite rouquier dimension for every $\mathfrak{p}\in \operatorname{Spec}(R)$. For each $\mathfrak{p}\in \operatorname{Spec}(R)$, there exists $\mathfrak{q}\in \operatorname{min}(R)$ such that $\mathfrak{q}\subseteq \mathfrak{p}$. This makes it possible to reduce to the case where $R$ is an integral domain whose krull dimension is $d$, which will be assumed for the remainder of the proof. If $R$ is an integral domain, then the hypothesis ensures $D^b (\operatorname{mod}R)$ has finite rouquier dimension $r$, and from Remark~\ref{rmk:rouquier_dim_cohomology_annihilator_nonzero}, $\operatorname{ca}^{r+1}(R) \not= (0)$. Choose $a\in \operatorname{ca}^{r+1}(R)$ nonzero, and reduce to the case $r$ is not a unit. There is a bijection between the prime ideals of $R/a R$ and all $\mathfrak{p}\in \operatorname{Spec}(R)$ such that $f\in \mathfrak{p}$. For any such prime ideal $\mathfrak{p}$ containing $f$, $D^b (\operatorname{mod}R/\mathfrak{p})$ has finite rouquier dimension, and so it admits strong generations via the induction hypothesis as $\dim R/ a R<d$. Furthermore, Proposition~\ref{prop:strong_generation_via_minimal_primes} ensures there exists $G\in \operatorname{mod}R/aR$ and $n,s\geq 0$ such that $\Omega_{R/aR}^s (\operatorname{mod}R/aR) \subseteq |G|_n$. Fix $M\in \operatorname{mod}R$. It is enough to show $\Omega_R^{r+s+1} (M) \in |\Omega^1_R (G)|_n$. Set $N=\Omega^{r+s+1}_R (M)$. If $R$ is an integral domain, then via Remark~\ref{rmk:regular_element_syzygy_quotient} there exists an isomorphism
    \begin{displaymath}
        N/aN = \Omega^{r+s}_{R/aR} (\Omega^1_R (M)/ a \Omega^1_R (M)).
    \end{displaymath}
    Now restriction of scalars ensures $N/aN \in |G|_n$, and so $\Omega_R^1 (N/aN) \in |\Omega_R^1 (G)|_n$. For each $L\in \operatorname{mod}R$ there exists an isomorphism
    \begin{displaymath}
        \operatorname{Ext}_R^1( N, L) \cong \operatorname{Ext}^{r+1}_R ( \Omega_R^{s+1} M,L),
    \end{displaymath}
    and so $\operatorname{Ext}_R^1( N, L)$ is annihilated by $a$. Note that $N$ is at least a first syzygy of $M$, and $a$ is a nonzero divisor on $N$ as $R$ is a domain. By Remark~\ref{rmk:ses_cohomology_annihilator}, $N$ is a direct summand of $\Omega_R^1 (N/aN)$. This implies $\Omega_R^{r + s+1} (M) \in |\Omega_R^1 (G)|_n$, and hence, establishes that $R\oplus \Omega^1_R (G)$ is a strong generator for $\operatorname{mod}R$.
\end{proof}

\begin{remark}
    It is worthwhile to note that there is a resemblance of the proof of Proposition~\ref{prop:integral_quotients_strong_generation_implies_strong_generation} and that of \cite[Theorem 5.1]{Iyengar/Takahashi:2016}. The interesting component is that this strategy is an honest adaptation to the context where the rouquier dimension of an integral domain is finite, and Remark~\ref{rmk:rouquier_dim_cohomology_annihilator_nonzero} ensures nonvanishing of the cohomology annihilator ideal. This yields enough control on a suitable choice of quotient ring to construct a strong generator in $\operatorname{mod}R$.
\end{remark}

\begin{corollary}\label{cor:quasi_excellent_strong_gen}
    For any noetherian quasi-excellent ring $R$ of finite krull dimension, $\operatorname{mod}R$ admits a strong generator. In particular, Question~\ref{q:Iyengar/Takahashi} holds true.
\end{corollary}

\begin{proof}
    If $R$ is quasi-excellent, then any essentially $R$-algebra of finite type is as well. By \cite[Main Theorem]{Aoki:2021}, it was shown that any noetherian quasi-excellent of finite krull dimension has finite rouquier dimension, and so appealing to Theorem~\ref{thm:strong_generation} furnishes the proof of the first claim. Lastly, any excellent ring is quasi-excellent of finite krull dimension, so the second claim follows.
\end{proof}

\begin{corollary}
    A noetherian $1$-dimensional $J\textrm{-}2$ ring $R$ has finite rouquier dimension if, and only if, $\operatorname{mod}R$ admits a strong generator.
\end{corollary}

\begin{proof}
    If $\operatorname{mod}R$ admits a strong generator, then Remark~\ref{rmk:strong_generator_to_derived_category} ensures $R$ has finite rouquier dimension, the other direction needs to be checked. First, suppose that $R$ is an integral domain. If $R$ has finite rouquier dimension $r$, then Remark~\ref{rmk:rouquier_dim_cohomology_annihilator_nonzero} ensures $\operatorname{ca}^{r+1}(R)$. Choosing $a\in \operatorname{ca}^{r+1}(R)$, the proof strategy for Proposition~\ref{prop:integral_quotients_strong_generation_implies_strong_generation} ensures there exists strong generators in $\operatorname{mod}R$ as $R/aR$ is artinian, and such rings admit strong generators in their module categories. To complete the proof, the non-integral domain case follows by appealing to Proposition~\ref{prop:strong_generation_via_minimal_primes}.
\end{proof}

\begin{remark}\label{rmk:thick_strong_generation}
    In the notation for the proof of Proposition~\ref{prop:integral_quotients_strong_generation_implies_strong_generation}, if $R$ is a noetherian quasi-excellent ring of finite krull dimension, then Remark~\ref{rmk:strong_generator_to_derived_category} ensures
    \begin{displaymath}
        \operatorname{mod}R = \operatorname{thick}^{s+\lambda(R) n (s+1)}_{\operatorname{mod}R} \big( R\oplus \bigoplus_{\mathfrak{p}\in \operatorname{min}(R)} (G(\mathfrak{p}) \oplus \mathfrak{p}) \big).
    \end{displaymath}
    This follows from the fact that $\Omega_R^i (R/\mathfrak{p})\in \operatorname{thick}_{\operatorname{mod}R} (R\oplus \mathfrak{p})$. Indeed, the short exact sequence
    \begin{displaymath}
        0 \to \mathfrak{p} \to R \to R/\mathfrak{p} \to 0
    \end{displaymath}
    implies $R/\mathfrak{p} \in \operatorname{thick}^2_{\operatorname{mod}R} (R \oplus \mathfrak{p})$, and an inductive argument will show for each $i\geq 2$ that $\Omega^i_R (R/\mathfrak{p})\in  \operatorname{thick}^i_{\operatorname{mod}R} (R \oplus \mathfrak{p})$.
\end{remark}

\begin{corollary}
    Suppose $R$ is a noetherian quasi-excellent ring of finite krull dimension. If $G\in \operatorname{mod}R$ satisfies $\operatorname{thick}_{\operatorname{mod}R}^n (G)= \operatorname{mod}R$ for some $n\geq 0$, then $G$ strongly generates $D^b (\operatorname{mod}R)$. On the other hand, if $E\in D^b (\operatorname{mod}R)$ strongly generates, then there exists $n\geq 0$ such that 
    \begin{displaymath}
        \operatorname{thick}_{\operatorname{mod}R}^n \big( R\oplus (\bigoplus_{j=1}^c H^{i_j} (E))\big) = \operatorname{mod}R
    \end{displaymath}
    where $i_j$ corresponds to each $H^i (E)\not=0$.
\end{corollary}

\begin{proof}
    The first claim is Remark~\ref{rmk:strong_generator_to_derived_category}, so only the second needs to be verified. If $E\in D^b (\operatorname{mod}R)$ strongly generates, then $\bigoplus_{j=1}^c H^{i_j} (E)$ does as well. Furthermore, Lemma~\ref{lem:thick_subcategory_krause_stevenson} ensures
    \begin{displaymath}
        \operatorname{thick}_{\operatorname{mod}R} \big(R\oplus (\bigoplus_{j=1}^c H^{i_j} (E))\big) = \operatorname{mod}R.
    \end{displaymath}
    From Corollary~\ref{cor:quasi_excellent_strong_gen}, there exists $G\in \operatorname{mod}R$ and $n,s\geq 0$ such that $\Omega^s_R (\operatorname{mod}R)\subseteq |G|_n$. Hence, it follows that
    \begin{displaymath}
        \operatorname{thick}_{\operatorname{mod}R}^{n+s} (R\oplus G) = \operatorname{mod}R.
    \end{displaymath}
    Choosing $t\geq 0$ such that $G\in \operatorname{thick}_{\operatorname{mod}R}^t \big(R\oplus (\bigoplus_{j=1}^c H^{i_j} (E))\big) $, appealing to Lemma~\ref{lem:ext_prop} furnishes the proof.
\end{proof}

\begin{corollary}
    If $R$ is a noetherian ring of finite krull dimension such that $R/\mathfrak{p}$ is regular for each $\mathfrak{p}\in\operatorname{min}(R)$, then 
    \begin{displaymath}
        \operatorname{thick}_{\operatorname{mod}R}^{1+ d + \lambda(R)(d + 2)} ( R \oplus (\bigoplus_{\mathfrak{p}\in \operatorname{min}(R)} R/\mathfrak{p})) = \operatorname{mod}R.
    \end{displaymath}
\end{corollary}

\begin{proof}
    For each $\mathfrak{p}\in\operatorname{min}(R)$, $R/\mathfrak{p}$ being regular ensures $\Omega^{\dim R/\mathfrak{p} +1}_{R/\mathfrak{p}} \subseteq |R/\mathfrak{p}|_0$. By Remark~\ref{rmk:thick_strong_generation}, the claim follows immediately.
\end{proof}

\begin{example}
    Let $R$ be a $1$-dimensional integral domain which is essentially of finite type over a field $k$. If $R$ is non-regular, then $\operatorname{ca}^3 (R) \not =0$ via Example~\ref{ex:Iyengar/Takahashi_strong_gen_examples}. For $a \in \operatorname{ca}^3 (R)$ nonzero nonunit, the proof of Proposition~\ref{prop:strong_generation_via_minimal_primes} promises
    \begin{displaymath}
        \Omega^4_R (\operatorname{mod}R) \subseteq |\Omega^1_R (\overline{R}/J(\overline{R}))|_{\ell\ell(\overline{R})}
    \end{displaymath}
    where $\overline{R}=R/(a)$.
\end{example}

\begin{example}\label{ex:1_dim_int_dom_strong_gen}
    Suppose that $(R,\mathfrak{m},k)$ is a noetherian quasi-excellent integral domain local ring of krull dimension one. Choose $r\geq 0$ such that there exists a nonzero nonunit element $a\in \operatorname{ca}^r (R)\not=0$. The proof of Proposition~\ref{prop:strong_generation_via_minimal_primes} ensures $\Omega^{r+1}_{R} (\operatorname{mod}R) \subseteq |\mathfrak{m}|_{\ell\ell(R/aR)}$. Additionally, if $R$ is equicharacteristic, then $r=3$ can be taken via Example~\ref{ex:Iyengar/Takahashi_strong_gen_examples}.
\end{example}

\begin{example}
    The following is a non-integral domain case. Consider the ring $R=\mathbb{F}_5 [x,y]_{(x,y)}/(x^{10} + y^{10})$. This is a $1$-dimensional non-integral domain. Its minimal primes are the ideals $(x\pm 2y)$, and $R/(x \pm 2y)$ are regular $1$-dimensional noetherian local rings. In particular, $\Omega^2_{R/(x \pm 2y)} (\operatorname{mod}R/(x \pm 2y))\subseteq |R/(x \pm 2y)|_1$, and so from the proof of Proposition~\ref{prop:strong_generation_via_minimal_primes},
    \begin{displaymath}
        \Omega^3_R (\operatorname{mod}R) \subseteq |R/(x+2y) \oplus R/(x -2y) \oplus (x+2y) \oplus (x-2y)|_{12}.
    \end{displaymath}
    Note that $\lambda(R) \leq 3$.
\end{example}

\begin{remark}\label{sec:F_finite_rings}
    Suppose $R$ is a noetherian ring of prime characteristic. If the frobenius morphism $F\colon R \to R$ is finite, then $R$ is said to be \textbf{$F$-finite}, i.e. $F_\ast R \in \operatorname{mod}R$ where $F_\ast R$ denotes restriction of scalars along $F\colon R \to R$. For instance, any algebra that is essentially of finite type over a perfect field of prime characteristic is $F$-finite. Additionally, if the map $F \colon R \to R$ splits in $\operatorname{mod}R$, then $R$ is said to be \textbf{frobenius split}, or \textbf{$F$-split}. For instance, given  $f\in k \llbracket  x_1,\ldots,x_n \rrbracket  =\colon S$ where $k$ is a perfect field of prime characteristic, then \cite[Proposition 1.7]{Fedder:1983} yields a useful computational strategy to determine whether or not $S/(f)$ is $F$-split.
\end{remark}

\begin{remark}
    If $R$ is a noetherian $F$-finite ring of positive depth, then for all $e\geq 1$ there does not exist an $n\geq 0$ such that $|F_\ast^e R|_n = \operatorname{mod}R$. If $E\in |F_\ast^e R|_n$, then by \cite[Proposition 1.12.1]{Takahashi:2010}, 
    \begin{displaymath}
        \operatorname{depth}(E) \geq \operatorname{depth}(F_\ast^e R).
    \end{displaymath}
    This ensures that no module with depth less than that of $\operatorname{depth}(R)$ can be contained in $|F_\ast^e R|_n$ for some $n$.
    %In \cite[Remark 1.9]{Takahashi:2010}, $\operatorname{ext}^{n+1} (F_\ast^e R) = |F_\ast^e R|_n$ relative to our notation and counting conventions. 
\end{remark}

\begin{proposition}\label{prop:F_finite_mod_cat}
    If $R$ is an $F$-finite noetherian ring, then there exists $e,n\geq 0$ such that
    \begin{displaymath}
        \operatorname{thick}_{\operatorname{mod}R}^n (R\oplus F_\ast^e R) = \operatorname{mod}R.
    \end{displaymath}
\end{proposition}

\begin{proof}
    By \cite[Corollary 3.9]{BILMP:2023}, there exists $e \geq 0$ such that $F_\ast^e R$ is a strong generator for $D^b (\operatorname{mod}R)$. Furthermore, Lemma~\ref{lem:thick_subcategory_krause_stevenson} ensures that 
    \begin{displaymath}
        \operatorname{thick}_{\operatorname{mod}R} (R\oplus F_\ast^e R) = \operatorname{mod}R.
    \end{displaymath}
    If $R$ is $F$-finite, then it is excellent via \cite[Theorem 2.5]{Kunz:1976}. Appealing Corollary~\ref{cor:quasi_excellent_strong_gen}, there exists $G\in \operatorname{mod}R$ and $n,s\geq 0$ such that $\Omega^s_R (\operatorname{mod}R)\subseteq |G|_n$. Now this ensures $\operatorname{thick}_{\operatorname{mod}R}^{n+s}(R\oplus G) = \operatorname{mod}R$. Hence, there exists $t \geq 0$ such that $R\oplus G\in \operatorname{thick}_{\operatorname{mod}R}^t (R\oplus F_\ast^e R)$, and Lemma~\ref{lem:ext_prop} implies
    \begin{displaymath}
        \operatorname{mod}R = \operatorname{thick}_{\operatorname{mod}R}^{n+s}(R\oplus G) \subseteq \operatorname{thick}_{\operatorname{mod}R}^{t(n+s}) (R\oplus F_\ast^e R).
    \end{displaymath}
\end{proof}

\begin{example}
    If $R$ is $F$-split, then Proposition~\ref{prop:F_finite_mod_cat} ensures there exists $e,n > 0$ such that
    \begin{displaymath}
        \operatorname{mod}R = \operatorname{thick}_{\operatorname{mod}R}^n (F_\ast^e R).
    \end{displaymath}
    The reader is encouraged to see \cite{Hochster:2013} for instances of $F$-split noetherian rings.
\end{example}

\begin{example}
    If $R$ is a locally complete intersection $F$-finite noetherian ring, then \cite[Theorem 6.3]{BILMP:2023} ensures $F_\ast R$ strongly generates $D^b (\operatorname{mod} R)$, and so Proposition~\ref{prop:F_finite_mod_cat} promises an $n\geq 0 $ such that
    \begin{displaymath}
        \operatorname{mod}R = \operatorname{thick}_{\operatorname{mod}R}^n (R \oplus F_\ast R).
    \end{displaymath}
\end{example}

\begin{example}
    Suppose $(R,\mathfrak{m},k)$ is a noetherian $F$-finite local ring. If $R_\mathfrak{p}$ is dominant (see \cite[Definition 5.1]{Takahashi:2022}) for each $\mathfrak{p}\in \operatorname{Spec}(R)$, then \cite[Theorem 10.10]{Takahashi:2022} and Proposition~\ref{prop:F_finite_mod_cat} ensures there exists an $n\geq 0 $ such that
    \begin{displaymath}
        \operatorname{mod}R = \operatorname{thick}_{\operatorname{mod}R}^n (R \oplus F_\ast R).
    \end{displaymath}
\end{example}

%%%%%%%%%%%%%%%%%%%%%%%%%%%%%%%%%%%%%%%
\section{Approximations}
\label{sec:coghost_and_approximations}
%%%%%%%%%%%%%%%%%%%%%%%%%%%%%%%%%%%%%%%

This section covers necessary content regarding approximations and (co)ghost maps in abelian category. Consider an abelian category $\mathcal{A}$, and let $f\colon E \to G$ be a map in $\mathcal{A}$. By pre- and post-composing with $f$, this induces a map on hom-sets, and studying the behavior of such leads to values called \textit{(co)ghost index}. Interestingly enough, these values are linked to the number of extensions needed to finitely build objects in the sense of Section~\ref{sec:extension_construction}. The primary reference in this section is \cite{Beligiannis:2008}. In the background, let $\mathcal{X}$ be an additive subcategory. 

\begin{definition}\label{def:right_approximation}
    A map $f \colon X \to E$ in $\mathcal{A}$ with $X\in \mathcal{X}$ is called a \textbf{right $\mathcal{X}$-approximation} of $E$ if for every $f' \colon X' \to E$ with $X'\in \mathcal{X}$ there exists a  $g:X' \to X$ such that $f' = f \circ g$. If every $E\in \mathcal{A}$ admits a right $\mathcal{X}$-approximation, then $\mathcal{X}$ is \textbf{contravariantly finite}.
\end{definition}

There are natural instances where Definition~\ref{def:right_approximation} is satisfied in practice. For instance, any $E\in \operatorname{mod}R$ with $R$ a noetherian ring, $\operatorname{add}(E)$ is a contravariantly finite subcategory in $\operatorname{mod}R$ (see \cite[Example 12.1]{Takahashi:2021}). 

Now, to relate right $\mathcal{X}$-approximations to finitely building objects via extensions and retracts of finite coproducts, this leads to studying a particular property regarding maps induced on hom-sets. For $G\in \mathcal{A}$, $\operatorname{Fac}(G)$ denotes the set of all $E\in \mathcal{A}$ such that there exists epimorphisms $T \to E$ where $T\in \operatorname{add}(G)$. 

\begin{definition}\label{def:ghost_maps}
    If $G\in \mathcal{A}$ and $f \colon E \to D$ is a map, then $f$ is \textbf{$G$-ghost} when the induced map 
    \begin{displaymath}
        \operatorname{Hom}_\mathcal{A} (X,f): \operatorname{Hom}_\mathcal{A} (X,E) \to \operatorname{Hom}_\mathcal{A} (X,D)
    \end{displaymath}
    is zero for each $X \in \operatorname{add}(G)$. For $f:E \to D$, it is an \textbf{$n$-fold $G$-ghost} when it may be written as a composition of $n$ $G$-ghost maps.  If $E,G \in \mathcal{A}$, then the \textbf{ghost index} of $E$ with respect to $G$ is the smallest $n\geq 0$ such that any $n$-fold $G$-ghost map from $E$ vanishes, and is denoted $\operatorname{gin}^G_\mathcal{A} (E)$.
\end{definition} 

The following result establishes a link between finitely building objects in the sense of Section~\ref{sec:extension_construction} with Definition~\ref{def:ghost_maps}.

\begin{lemma}[Ghost lemma for abelian categories]\label{lem:ghost_lemma_abelian}
    Suppose $E,G\in \mathcal{A}$.
    \begin{itemize}
        \item If $E\in |\operatorname{Fac}(G)|_n$, then $\operatorname{gin}^G_\mathcal{A} (E) \leq n$.
        \item If $\operatorname{add}(G)$ is contravariantly finite, then the following are equivalent:
        \begin{enumerate}
            \item $E\in |\operatorname{Fac}(G)|_n$;
            \item $\operatorname{gin}^G_\mathcal{A} (E) \leq n$.
        \end{enumerate}
    \end{itemize}
\end{lemma}

\begin{proof}
    This is \cite[Lemma 1.3]{Beligiannis:2008}.
\end{proof}

The ideas presented so far can also be applied to the opposite category $\mathcal{A}^{op}$, which leads to a dual notion of Definition~\ref{def:right_approximation}, Definition~\ref{def:ghost_maps}, and Lemma~\ref{lem:ghost_lemma_abelian}.

\begin{definition}
    A map $f \colon E \to X$ in $\mathcal{A}$ with $X$ belonging to $\mathcal{X}$ is called a \textbf{left $\mathcal{X}$-approximation} of $E$ if for every $f' \colon E\to X'$ with $X'\in \mathcal{X}$ there exists a  $g:X \to X'$ such that $f' = g \circ g$. If every $E\in \mathcal{A}$ admits a left $\mathcal{X}$-approximation, then $\mathcal{X}$ is \textbf{covariantly finite}.
\end{definition}

For instance, $E\in \operatorname{mod}R$ with $R$ a noetherian ring, $\operatorname{add}(E)$ is a covariantly finite subcategory in $\operatorname{mod}R$ (see \cite[Example 12.1]{Takahashi:2021}). For $G\in \mathcal{A}$, $\operatorname{Sub}(G)$ denotes the set of all $E\in \mathcal{A}$ such that there exists an monomorphisms $T \to E$ where $T\in \operatorname{add}(G)$.

\begin{definition}
    If $G\in \mathcal{A}$ and $f:E \to D$ is a map, then $f$ is \textbf{$G$-coghost} when the induced map 
    \begin{displaymath}
        \operatorname{Hom}_\mathcal{A} (f,X): \operatorname{Hom}_\mathcal{A} (D,X) \to \operatorname{Hom}_\mathcal{A} (E,X)
    \end{displaymath}
    is zero for each $X \in \operatorname{add}(G)$. For $f:E \to D$, it is an \textbf{$n$-fold $G$-coghost} when it may be written as a composition of $n$ $G$-coghost maps. If $E,G \in \mathcal{A}$, then the \textbf{coghost index} of $D$ with respect to $G$ is the smallest $n\geq 0$ such that any $n$-fold $G$-coghost map to $D$ vanishes, and is denoted $\operatorname{cogin}^G_\mathcal{A} (D)$.
\end{definition}

\begin{lemma}[Coghost lemma for abelian categories]\label{lem:coghost_lemma_abelian}
    Suppose $E,G\in \mathcal{A}$.
    \begin{itemize}
        \item If $E\in |\operatorname{Sub}(G)|_n$, then $\operatorname{cogin}^G_\mathcal{A} (E) \leq n$.
        \item If $\operatorname{Sub}(G)$ is covariantly finite, then the following are equivalent:
        \begin{enumerate}
            \item $E\in |\operatorname{Sub}(G)|_n$;
            \item $\operatorname{cogin}^G_\mathcal{A} (E) \leq n$.
        \end{enumerate}
    \end{itemize}
\end{lemma}

\begin{proof}
    This follows by applying Lemma~\ref{lem:ghost_lemma_abelian} to the opposite category $\mathcal{A}^{op}$, but a proof is provided.  Note that $\operatorname{Sub}(G)$ in $\mathcal{A}$ coincides with $\operatorname{Fac}(G)$ in $\mathcal{A}^{op}$.

    If $E \in |\operatorname{Sub}(G)|_1$, then for any $G$-coghost map $f:A \to E$ the induced map on hom-sets
    \begin{displaymath}
        \operatorname{Hom}_\mathcal{A} (f,X): \operatorname{Hom}_\mathcal{A} (E,E) \to \operatorname{Hom}_\mathcal{A} (E,X)
    \end{displaymath}
    vanishes. Hence, $f\circ 1_E = 0$, and so $f=0$. Let $E \in |\operatorname{Sub}(G)|_2$. There exists a short exact sequence
    \begin{displaymath}
        0 \to A \xrightarrow{\alpha} E \xrightarrow{\beta} B \to 0
    \end{displaymath} 
    where $A \in |\operatorname{Sub}(G)|_1$ and $B \in |\operatorname{Sub}(G)|_1$. Consider a composition of $G$-coghost maps,
    \begin{displaymath}
        X_2 \xrightarrow{f_2} X_1 \xrightarrow{f_1} X_0 := E.
    \end{displaymath}
    There exists monomorphisms $e_0:A \to T_0$ and $e_1: B \to T_1$ where $T_1,T_2 \in \operatorname{add}(G)$. If $f_1$ is $G$-coghost, then $e_1 \circ \beta \circ f_1= 0$, and as $e_1$ is a monomorphism, $\beta \circ f_1 =0$. But this means there exists an $h: A \to X_0$ such that $f_1 = \alpha \circ h$. There exists the following diagram,
    \begin{displaymath}
        \begin{tikzcd}
    	    & {T_0} && {T_1} \\
    	    0 & A & E & B & 0 \\
    	    && {X_1} \\
    	    && {X_2}
    	    \arrow[from=2-1, to=2-2]
    	    \arrow["\alpha", from=2-2, to=2-3]
    	    \arrow["\beta", from=2-3, to=2-4]
    	    \arrow[from=2-4, to=2-5]
            \arrow["{e_0}", hook, from=2-2, to=1-2]
            \arrow["{e_1}", hook, from=2-4, to=1-4]
            \arrow["{f_1}", from=3-3, to=2-3]
            \arrow["{f_2}", from=4-3, to=3-3]
            \arrow["h", dashed, from=3-3, to=2-2]
        \end{tikzcd}
    \end{displaymath}
    As $f_2$ is $G$-coghost, $e_0 \circ h \circ f_2 =0$, and $e_0$ being a monomorphism ensures $h \circ f_2= 0$. Hence, $f_1 \circ f_2 = f_1 \alpha \circ h \circ f_2 = 0$. The remaining cases follow by induction.

    Now, the last claim is verified. Suppose that $\operatorname{add}(G)$ is covariantly finite in $\mathcal{A}$. From the first claim, it suffices to check that $\operatorname{cogin}^G_\mathcal{A} (E) \leq n$ implies $E\in |\operatorname{Sub}(G)|_n$. If $\operatorname{cogin}^G_\mathcal{A} (E) =1$, then $E\in |\operatorname{Sub}(G)|_1$. Indeed, let $f:E \to X_1$ be a left $G$-approximation with $X\in \operatorname{add}(G)$. There exists an exact sequence
    \begin{displaymath}
        0 \to \ker (f) \xrightarrow{i_1} E \xrightarrow{f} X_1.
    \end{displaymath}
    As $f$ is a left $G$-approximation, $i$ is $G$-coghost, and so $i_1=0$. This ensures $f$ is a monomorphism, and so $E \in |\operatorname{Sub}(G)|_1$. Suppose now $\operatorname{cogin}^G_\mathcal{A} (E) =2$, and let $f$ be a left $G$-approximation as above. Choose a left $G$-approximation $g: \operatorname{ker}(f) \to X_2$ with $X_2\ in \operatorname{add}(G)$, and consider diagram
    \begin{displaymath}
        \begin{tikzcd}
            && {X_2} & {X_1} \\
            0 & {\ker (f_2)} & {\ker(f_1)} & E \\
            && 0
            \arrow["{i_1}", hook, from=2-3, to=2-4]
            \arrow["{f_1}", from=2-4, to=1-4]
            \arrow["{i_2}", hook, from=2-2, to=2-3]
             \arrow["{f_2}", from=2-3, to=1-3]
            \arrow[from=2-1, to=2-2]
            \arrow[from=3-3, to=2-3]
        \end{tikzcd}
    \end{displaymath}
    It can be checked that $i_1 \circ i_2$ is $G$-coghost, so $i_2 \circ i_1=0$. As $i_1$ is a monomorphism, $i_2=0$, and so $\ker(f_1)\in |\operatorname{Sub}(G)|_1$. From the short exact sequence
    \begin{displaymath}
        0 \to \ker(f_1) \to E \to \operatorname{im}(f_1) \to 0,
    \end{displaymath}
    it follows that $E\in |\operatorname{Sub}(G)|_2$. By proceeding in a similar fashion, the proof follows by induction on $n$.
\end{proof}

%%%%%%%%%%%%%%%%%%%%%%%%%%%%%%%%%%%%%%%
\section{(Co)ghost index}
\label{sec:co_ghost_mod_cats}
%%%%%%%%%%%%%%%%%%%%%%%%%%%%%%%%%%%%%%%

In this section, it will be shown (co)ghost index for the category of finitely generated modules over a noetherian ring satisfies a local-to-global principle a la \cite{Letz:2021}, and a number of examples will be computed in the $F$-finite case (see Remark~\ref{sec:F_finite_rings}). The main result of this section is the following statement.

\begin{theorem}\label{thm:ghost_index_local_global}
    Let $R$ be a noetherian rng. If $E,G\in \operatorname{mod}R$, then the following are equivalent:
    \begin{enumerate}
        \item $\operatorname{gin}^G_{\operatorname{mod}R} (E)<\infty$;
        \item $\operatorname{gin}^{G_\mathfrak{m}}_{\operatorname{mod}R_\mathfrak{m}} (E_\mathfrak{m}) < \infty$ for all $\mathfrak{m}\in \operatorname{mSpec}(R)$;
        \item $\operatorname{gin}^{G_\mathfrak{p}}_{\operatorname{mod}R_\mathfrak{p}} (E_\mathfrak{p}) < \infty$ for all $\mathfrak{p}\in \operatorname{Spec}(R)$.
    \end{enumerate}
\end{theorem}

To motivate these values, observe when given a strong generator $G\in \operatorname{mod}R$ satisfying $\Omega^s_R (\operatorname{mod}R) \subseteq |G|_n$ this implies
\begin{displaymath}
    \operatorname{gin}_{\operatorname{mod}R}^G (\Omega^s_R (E)),\operatorname{cogin}_{\operatorname{mod}R}^G (\Omega^s_R (E)) \leq n
\end{displaymath}
for each $E\in \operatorname{mod}R$. Indeed, it follows from Section~\ref{sec:coghost_and_approximations}, and hence, this produces lower bounds for the parameters $n,s$ of a strong generator in $\operatorname{mod}R$. To start, an understanding as to how (co)ghost maps behaves under localization is crucial.

\begin{lemma}\label{lem:localizing_(co)ghost_maps}
    Let $R$ be a noetherian ring, and $\mathfrak{p}\in \operatorname{Spec}(R)$. If $f \colon E \to M$ is a $G$-(co)ghost map in $\operatorname{mod}R$, then $f_\mathfrak{p} \colon E_\mathfrak{p} \to M_\mathfrak{p}$ is $G_\mathfrak{p}$-(co)ghost map in $\operatorname{mod}R_\mathfrak{p}$.
\end{lemma}

\begin{proof}
    It suffices to only check the ghost case as coghost follows similarly. Since $f$ is $G$-ghost, the induced map
    \begin{displaymath}
        \operatorname{Hom}_R (G,f) \colon \operatorname{Hom}_R (G,E) \to \operatorname{Hom}_R (G,M)
    \end{displaymath}
    vanishes. As $R$ is noetherian and localization is flat, there exists an $R_\mathfrak{p}$-module isomorphism
    \begin{displaymath}
        \operatorname{Hom}_{R_\mathfrak{p}} (K_\mathfrak{p},L_\mathfrak{p}) \cong \big(\operatorname{Hom}_R (K,L)\big)_\mathfrak{p}.
    \end{displaymath}
    for all $K,L\in \operatorname{mod}R$. Hence, after localizing, the induced map
    \begin{displaymath}
        \operatorname{Hom}_{R_\mathfrak{p}} (G_\mathfrak{p},f_\mathfrak{p}) \colon \operatorname{Hom}_{R_\mathfrak{p}} (G_\mathfrak{p},E_\mathfrak{p}) \to \operatorname{Hom}_{R_\mathfrak{p}} (G_\mathfrak{p},M_\mathfrak{p})
    \end{displaymath}
    vanishes. Thus, $f_p\colon E_\mathfrak{p} \to M_\mathfrak{p}$ is $G_\mathfrak{p}$-coghost in $\operatorname{mod}R_\mathfrak{p}$ as desired.
\end{proof}

\begin{lemma}\label{lem:ghost_local_global_principle}
    Let $R$ be a noetherian ring. If $E,G\in \operatorname{mod}R$, then
    \begin{displaymath}
        \begin{aligned}
            \operatorname{gin}^G_{\operatorname{mod}R} (E) &= \sup\{ \operatorname{gin}^{G_\mathfrak{m}}_{\operatorname{mod}R_\mathfrak{m}} (E_\mathfrak{m}) : \mathfrak{m}\in \operatorname{mSpec}(R)\} \\&= \sup\{ \operatorname{gin}^{G_\mathfrak{p}}_{\operatorname{mod}R_\mathfrak{p}} (E_\mathfrak{p}) : \mathfrak{p}\in \operatorname{Spec}(R)\}
        \end{aligned}
    \end{displaymath}
    A similar result holds for $\operatorname{cogin}^G_{\operatorname{mod}R} (E)$
\end{lemma}

\begin{proof}
    Let $f \colon E \to B$ be an $n$-fold $G$-ghost map. After localizing at a maximal ideal $\mathfrak{m}\subseteq R$, Lemma~\ref{lem:localizing_(co)ghost_maps} promises that $f_\mathfrak{m} \colon E_\mathfrak{m} \to B_\mathfrak{m}$ is an $n$-fold $G_\mathfrak{m}$-ghost map in $\operatorname{mod}R_\mathfrak{m}$. If $f_\mathfrak{m}=0$ for all $\mathfrak{m}\in \operatorname{mSpec}(R)$, then $f=0$. This shows that
    \begin{displaymath}
        \begin{aligned}
            \operatorname{gin}^G_{\operatorname{mod}R} (E) &\leq \sup \{ \operatorname{gin}^{G_\mathfrak{m}}_{\operatorname{mod}R_\mathfrak{m}}  (E_\mathfrak{m}) : \mathfrak{m} \in \operatorname{mSpec}(R)\} \\&\leq \sup \{ \operatorname{gin}^{G_\mathfrak{p}}_{\operatorname{mod}R_\mathfrak{p}}  (E_\mathfrak{p}) : \mathfrak{p} \in \operatorname{Spec}(R)\},
        \end{aligned}
    \end{displaymath}
    where the second follows analogously.

    %Localization at the maximal ideal preverses the ghost map as our noetherian ring is noetherian, the modules are f.g., and localization is flat. The second part comes from showing the image vanishes.

    By Lemma~\ref{lem:ghost_lemma_abelian}, $\operatorname{gin}^G_{\operatorname{mod}R} (E) \leq n$ if, and only if, $E\in |\operatorname{Fac}(G)|_n$. If $E\in |\operatorname{Fac}(G)|_n$, then for each $\mathfrak{p}\in \operatorname{Spec}(R)$, $E_\mathfrak{p} \in |\operatorname{Fac}(G_\mathfrak{p})|_n$, so $\operatorname{gin}^{G_\mathfrak{p}}_{\operatorname{mod}R_\mathfrak{p}} (E_\mathfrak{p}) \leq n$. Hence, this shows 
    \begin{displaymath}
        \sup \{ \operatorname{gin}^{G_\mathfrak{p}}_{\operatorname{mod}R_\mathfrak{p}}  (E_\mathfrak{p}) : \mathfrak{p} \in \operatorname{Spec}(R)\} \leq \operatorname{gin}^G_{\operatorname{mod}R} (E).
    \end{displaymath}

    The claim for coghost index $\operatorname{gin}^G_{\operatorname{mod}R} (E)$ comes by applying a similar argument above and using Lemma~\ref{lem:coghost_lemma_abelian}.
\end{proof}

\begin{lemma}\label{lem:approximation_localizing}
    Let $R$ be a noetherian ring, and choose $E,G\in \operatorname{mod}R$. If $f\colon X \to E$ is a right $G$-approximation and $\mathfrak{p}\in \operatorname{Spec}(R)$, then $f_\mathfrak{p} \colon X_\mathfrak{p} \to G_\mathfrak{p}$ is a right $G_\mathfrak{p}$ approximation in $\operatorname{mod}R_\mathfrak{p}$. A similar statement holds for localizing left $G$-approximations.
\end{lemma}

\begin{proof}
    Consider a morphism $h:X' \to E_\mathfrak{p}$ in $\operatorname{mod}R_\mathfrak{p}$ where $X'\in \operatorname{add}(G_\mathfrak{p})$. There exists an $\alpha\geq 1$ such that $X'$ is a direct summand of $G_\mathfrak{p}^{\oplus \alpha}$. Hence, there exists an epimorphism $\pi\colon G^{\oplus \alpha}_\mathfrak{p} \to X'$ and monomorphism $i\colon X' \to G^{\oplus \alpha}_\mathfrak{p}$ such that $\pi \circ i = 1_{X'}$. As $R$ is noetherian and localization is flat, there exists an $R_\mathfrak{p}$-module isomorphism
    \begin{displaymath}
        \operatorname{Hom}_{R_\mathfrak{p}} (K_\mathfrak{p},L_\mathfrak{p}) \cong \big(\operatorname{Hom}_R (K,L)\big)_\mathfrak{p}.
    \end{displaymath}
    for all $K,L\in \operatorname{mod}R$. Choose $\overline{X'}\in \operatorname{mod}R$ such that $\overline{X'}_\mathfrak{p}=X'$. There exists a map $\overline{\pi} \colon G^{\oplus \alpha} \to \overline{X'}$ such that $\overline{\pi}_\mathfrak{p}=\pi$. Furthermore, there is a map $\overline{h} \colon \overline{X'} \to E$ such that $\overline{h}_\mathfrak{p} = h$. Now this gives a map $\overline{h} \circ \overline{\pi}\colon G^{\oplus \alpha}\to E$. As $f$ is a right $G$-approximation, there exists $g \colon G^{\oplus \alpha} \to X$ such that $\overline{h} \circ \overline{\pi} = f \circ g$. After localizing at $\mathfrak{p}$, there exists a diagram
    \begin{displaymath}
        \begin{tikzcd}
            {X_\mathfrak{p}} && E \\
            {G^{\oplus \alpha}_\mathfrak{p}} && {X'} & {}
            \arrow["{h_\mathfrak{p}}", from=2-1, to=1-1]
            \arrow["{f_\mathfrak{p}}", from=1-1, to=1-3]
            \arrow["i"', shift right=1, from=2-3, to=2-1]
            \arrow["h"', from=2-3, to=1-3]
            \arrow["\pi"', shift right=1, from=2-1, to=2-3]
        \end{tikzcd}
    \end{displaymath}
    in $\operatorname{mod}R_\mathfrak{p}$ for which the outside square is commutative. The proof is complete if the inside square commutes. Notice that
    \begin{displaymath}
        \begin{aligned}
            (f_\mathfrak{p} \circ g_\mathfrak{p}) \circ i &= (\overline{h} \circ \overline{\pi})_\mathfrak{p} \circ i \\&= (h \circ \pi) \circ i \\&= h \circ (\pi \circ i) \\&= h.
        \end{aligned}
    \end{displaymath}
    Therefore, this gives the desired morphism $g_\mathfrak{p} \circ i \colon X' \to X$ to show $f_\mathfrak{p}$ is a right $G_\mathfrak{p}$-approximation in $\operatorname{mod}R_\mathfrak{p}$.
\end{proof}

\begin{proof}[Proof of Theorem~\ref{thm:ghost_index_local_global}]
    From Lemma~\ref{lem:ghost_local_global_principle}, $(1)\implies (2) \implies (3)$ and so, $(3)\implies (1)$ needs to be checked. Define a diagram of the form
    \[
    \begin{tikzcd}
        E_0 \arrow{r}{\pi_0} & E_1 \arrow{r}{\pi_1}  & \cdots \\
        A_0 \arrow{u}{f_0} \arrow{r}{h_0} & A_1 \arrow{u}{f_1} \arrow{r}{h_0} & \cdots 
    \end{tikzcd}
    \]
    to be \textit{admissible} when each $\pi_j \colon E_j \to E_{j+1}$ are $G$-ghost maps with $E_0 \colon =E$, $f_j  \colon A_j \to E_j$ a right $G$-approximation
    satisfying the property that $\pi_j$ is a cokernel map of $f_j$ and $f_{j+1} \circ h_j = \pi_j \circ f_ j$. Note this diagram is not necessarily commutative, only its internal squares are. First, it is shown that an admissible diagram always exists for a pair of modules $E,G\in \operatorname{mod}R$. Indeed, as $\operatorname{add}(G)$ is contravariantly finite, there exists a right $G$-approximation $f_0 \colon A_0 \to E$ for $A_0 \in \operatorname{add}(G)$. Let $E_1 := \operatorname{coker}(f_0)$. Then $\pi_0 \colon E_0 \to E_1$ is $G$-ghost, and this can be seen from the fact that $f_0 \colon A_0 \to E_0$ is a right $G$-approximation. Choose a right $G$-approximation $f_1 \colon A_1 \to E_1$. There exists $h_0 \colon A_0 \to A_1$ such that $f_1 \circ h_0 = \pi_0 \circ f_0$. By iterating this process, there exists admissible diagrams.

    Next, it is shown that for any admissible diagram of $E,G \in \operatorname{mod}R$,
    \begin{displaymath}
        \operatorname{gin}_{\operatorname{mod}R}^G (E) = \inf\{ n\geq 0: \pi_n \circ \cdots \circ \pi_0 = 0\}.
    \end{displaymath}
    Since each $\pi_j$ are $G$-ghost,
    \begin{displaymath}
        \operatorname{gin}_{\operatorname{mod}R}^G (E) \geq \inf\{ n\geq 0: \pi_n \circ \cdots \circ \pi_0 = 0\}.
    \end{displaymath}
    Assume there exists $n\geq 0$ such that $\pi_n \circ \cdots \circ \pi_0 = 0$. This is a composition of surjective maps, so $E_{n+1}=\operatorname{im}(\pi_n) = 0$. Hence, $\pi_n =0$ and so, $E_n \in \operatorname{Fac}(G)$. There exists a short exact sequence
    \begin{displaymath}
        0 \to \operatorname{im}(f_{n-1}) \to E_{n-1} \to E_n \to 0,
    \end{displaymath}
    and $\operatorname{im}(f_{n-1})\in \operatorname{Fac}(G)$. Continuing to work backwards, it follows that $E\in |\operatorname{Fac}(G)|_n$, which shows $\operatorname{gin}_{\operatorname{mod}R}^G (E) \leq n$.

    Localizing any admissible diagram of $E,G$ in $\operatorname{mod}R$ at a prime ideal $\mathfrak{p}$ yields an admissible diagram of $E_\mathfrak{p},G_\mathfrak{p}$ in $\operatorname{mod}R_\mathfrak{p}$. Indeed, this follows from Lemma~\ref{lem:localizing_(co)ghost_maps} and Lemma~\ref{lem:approximation_localizing}. Fix an admissible diagram $(E_j,A_j,\pi_j)$ of $E,G$. It will be shown that
    \begin{displaymath}
        \{\mathfrak{p}: \operatorname{gin}_{\operatorname{mod}R_\mathfrak{p}}^{G_\mathfrak{p}} (E_\mathfrak{p}) \leq n \} = \operatorname{Spec}(R) \setminus \operatorname{supp} \operatorname{im} (\pi_n \circ \cdots \pi_0).
    \end{displaymath}
    If $\operatorname{gin}_{\operatorname{mod}R_\mathfrak{p}}^{G_\mathfrak{p}} (E_\mathfrak{p}) \leq n$, then $(\pi_n \circ \cdots \pi_0)_\mathfrak{p} =0$ as $(\pi_j)_\mathfrak{p}$ are $G_\mathfrak{p}$-ghost, so $\mathfrak{p} \not\in \operatorname{supp} \operatorname{im} (\pi_n \circ \cdots \pi_0)$. On the other hand, if $ \mathfrak{p} \in \operatorname{Spec}(R) \setminus \operatorname{supp} \operatorname{im} (\pi_n \circ \cdots \pi_0)$, then the work above ensures $\operatorname{gin}_{\operatorname{mod}R_\mathfrak{p}}^{G_\mathfrak{p}} (E_\mathfrak{p}) \leq n$ as localization of admissible diagrams remain admissible.

    Since the sets $V_n \colon = \operatorname{Spec}(R) \setminus \operatorname{supp} \operatorname{im} (\pi_n \circ \cdots \pi_0)$ are zariski open and $V_n \subseteq V_{n+1}$, noetherianity of $R$ promises there exists $N\geq 0$ such that $V_N = V_{N+1} = \cdots$. Assuming $(3)$ holds, for each $\mathfrak{p}\in \operatorname{Spec}(R)$ there exists $n$ such that $\mathfrak{p}\in V_n$. This implies
    \begin{displaymath}
        \operatorname{gin}_{\operatorname{mod}R_\mathfrak{p}}^{G_\mathfrak{p}} (E_\mathfrak{p}) \leq N;
    \end{displaymath}
    showing that $(1)$ holds.
\end{proof}

\begin{remark}
    In a suitable sense, admissible diagrams in the proof of Theorem~\ref{thm:ghost_index_local_global} are a module-theoretic analog of the adams resolution in a triangulated category (see \cite[Definition 3.4.1]{Letz:2020}). The commutativity of the internal squares of admissible diagrams were not used in the proof.
\end{remark}

\begin{theorem}\label{thm:coghost_index_local_global}
    Let $R$ be a noetherian ring. If $E,G\in \operatorname{mod}R$, then the following are equivalent:
    \begin{enumerate}
        \item $\operatorname{cogin}^G_{\operatorname{mod}R} (E)<\infty$;
        \item $\operatorname{cogin}^{G_\mathfrak{m}}_{\operatorname{mod}R_\mathfrak{m}} (E_\mathfrak{m}) < \infty$ for all $\mathfrak{m}\in \operatorname{mSpec}(R)$;
        \item $\operatorname{cogin}^{G_\mathfrak{p}}_{\operatorname{mod}R_\mathfrak{p}} (E_\mathfrak{p}) < \infty$ for all $\mathfrak{p}\in \operatorname{Spec}(R)$.
    \end{enumerate}
\end{theorem}

\begin{proof}
    This argument is essentially the dualization of that in Theorem~\ref{thm:ghost_index_local_global}.
\end{proof}

\begin{example}\label{ex:artinian_first_frob_push}
    If $(R,\mathfrak{m},k)$ is an $F$-finite artinian local ring, then $\operatorname{gin}^{F_\ast R} (E) < \infty$ for all $E\in \operatorname{mod}R$. Indeed, there exists a surjection $R \to k$, and $\operatorname{gin}^k (E) \leq \ell\ell(R)$. Then $k^{\oplus \alpha} \cong F_\ast k \in \operatorname{Fac}(F_\ast R)$, and so $\operatorname{mod}R = |\operatorname{Fac}(F_\ast R)|_{\ell\ell(R)}$. If $\ell\ell(R)>1$, then $\operatorname{mod}R \not= |\operatorname{Fac}(F_\ast R)|_1$ as this would imply $R$ is a direct summand of $F_\ast R$, and hence $R$ would be reduced.
\end{example}

\begin{example}
    Let $R$ be a $1$-dimensional noetherian $F$-finite integral domain. Example~\ref{ex:Iyengar/Takahashi_strong_gen_examples} verifies $\operatorname{ca}^3 (R)\not=0$, and so there exists a nonzero divisor $a\in \operatorname{ca}^3 (R)$. Note that $R/a R$ is artinian. Choose $e \geq 0$ such that $F_\ast^e (R/ aR) \cong k^{\oplus \alpha}$ in $\operatorname{mod} R/a R$. Now, restriction of scalars confirms $F_\ast^e (R/ aR) \in \operatorname{Fac}(F_\ast^e R)$, and the proof of Proposition~\ref{prop:strong_generation_via_minimal_primes} exhibits $\Omega^4_R(\operatorname{mod}R) \subseteq |\Omega^1_R (k)|_{\ell\ell(R/a R)}$. Putting this together, 
    \begin{displaymath}
        \Omega^4_R(\operatorname{mod}R) \subseteq |\operatorname{Fac}\big(\Omega^1_R (F_\ast^e R)\big)|_{\ell\ell(R/a R)}.
    \end{displaymath}    
    However, as $\Omega_R^1 (F_\ast^e R) \in \operatorname{Fac}(\Omega^1_R (F_\ast R))$, it follows  that
    \begin{displaymath}
        \Omega^4_R(\operatorname{mod}R) \subseteq |\operatorname{Fac}\big(\Omega^1_R (F_\ast R)\big)|_{\ell\ell(R/a R)}.
    \end{displaymath}
    This bound can be sharpened even in the case $R$ is not frobenius split. For instance, let $k$ be an $F$-finite field of characteristic five. Consider $R=k \llbracket  x,y \rrbracket  /(x^2  - y^3)$. This is a one-dimensional local integral domain that is not frobenius split. By Example~\ref{ex:Dieterich_hypersurface_cohomology_annihilator}, the elements $2x, 3y^2$ belong to $\operatorname{ca}^2 (R)$. From the previous discussion, if $a=x$, then $\ell\ell(R/xR)=3$, and so 
    \begin{displaymath}
        \operatorname{gin}_{\operatorname{mod}R}^{\Omega^1_R (F_\ast R)} (\Omega^3_R(E)) \leq 3
    \end{displaymath}
    for all $E\in \operatorname{mod}R$.
\end{example}

\begin{remark}
    In general, for some $G\in \operatorname{mod}R$ and $n\geq 0$ it cannot be expected $|\operatorname{Fac}(G)|_n = \operatorname{mod}R$ implies $|G|_N = \operatorname{mod}R$ for some $N\geq 0$. Let $R$ be a non-regular noetherian ring. Note that $|\operatorname{Fac}(R)|_1 = \operatorname{mod}R$, but $|R|_n = \operatorname{mod}R$ would imply $R$ strongly generates $D^b (\operatorname{mod}R)$, which cannot happen unless $R$ is regular. Furthermore, this exhibits that (co)ghost index in $\operatorname{mod}R$ does not necessarily coincide with level in $D^b (\operatorname{mod}R)$.
\end{remark}

\begin{example}
    The following produces a family of examples where both the coghost index and krull dimension can be made arbitrarily large. Let $k$ be an $F$-finite field of prime characteristic $p$. Consider the ring 
    \begin{displaymath}
        R=k[x_1,\ldots,x_n]/(x_1^{pm},\ldots,x_j^{pm})
    \end{displaymath}
    for some $m\geq 0$. A direct calculation will show that $F_\ast R$ is a direct sum of copies of $S\colon =k[x_1,\ldots,x_n]/(x_1^m,\ldots,x_n^m)$. Consider the composition of maps
    \begin{displaymath}
        R \xrightarrow{x_1^m} R \xrightarrow{x_2^m} \cdots \xrightarrow{x_{n-1}^m} R \xrightarrow{x_j^m} R,
    \end{displaymath}
    i.e. the endomorphism $(x_1\cdots x_j)^m \colon R \to R$. This is a nonzero map, but its induced map on $\operatorname{Hom}(R, F_\ast R)$ vanishes, and hence, implying that $\operatorname{cogin}_{\operatorname{mod}R}^{F_\ast R} (R) \geq j$.
\end{example}

\begin{proposition}
    Let $k$ be an $F$-finite field of prime characteristic $p$. If $I\subseteq k \llbracket  x_1,\ldots,x_n \rrbracket  $ is a monomial ideal generated by a regular sequence of monomials of the form $f_1^{pm}, \ldots, f_c^{pm}$ for some $m\geq 1$, then $\operatorname{cogin}_{\operatorname{mod}R}^{F_\ast R} (R) \geq c$.
\end{proposition}

\begin{proof}
    A direct calculation shows that $F_\ast R$ is a direct sum of copies of 
    \begin{displaymath}
        k \llbracket  x_1,\ldots,x_n \rrbracket  /(f_1^m,\ldots,f_c^m).
    \end{displaymath}
     Consider the composition of maps
    \begin{displaymath}
        R \xrightarrow{f_1^m} R \xrightarrow{f_2^m} \cdots \xrightarrow{f_{c-1}^m} R \xrightarrow{f_c^m} R,
    \end{displaymath}
    i.e. the endomorphism $(f_1\cdots f_c)^m \colon R \to R$. This is a nonzero map, but its induced map on $\operatorname{Hom}(R, F_\ast R)$ vanishes, and hence, implying that $\operatorname{cogin}_{\operatorname{mod}R}^{F_\ast R} (R) \geq c$.
\end{proof}

\begin{proposition}
    Choose an integer $e\geq 1$. If $R$ is a noetherian $F$-finite ring and for each $N$ there exists $f_1,\ldots f_N \in \operatorname{ann}_R (F_\ast^e R)$ such that $f_1 \cdots f_N \not =0$, then $F_\ast^e R$ does not strongly generate $D^b (\operatorname{mod}R)$.
\end{proposition}

\begin{proof}
    If $f\in \operatorname{ann}_R (F_\ast^e R)$, then the multiplication map $f \colon F_\ast^e R \to F_\ast^e R$ vanishes in $\operatorname{mod}R$. Furthermore, by viewing $f$ as an endomorphism on $R$, it is $F_\ast^e R$-coghost in $\operatorname{mod}R$. Hence, if interpreted within this context, the hypothesis ensures for each $N\geq 1$ there exists a nonzero $N$-fold $F_\ast^e R$-coghost endomorphism on $R$ in $\operatorname{mod}R$, and so $\operatorname{cogin}_{\operatorname{mod}R}^{F_\ast^e R} (R)=\infty$. But $R$ is projective, so $\operatorname{Ext}^n_R (R, F_\ast^e R) =0$ when $n\not=0$. Hence, any $N$-fold $F_\ast^e R$-coghost map in $\operatorname{mod}R$ is such in $D^b (\operatorname{mod}R)$, and this completes the proof.
\end{proof}

\begin{corollary}
    Let $R$ be a noetherian $F$-finite ring. If there exists $f_1,\ldots,f_n\in \operatorname{ann}_R (F_\ast^e R)$ whose product is nonzero, then 
    \begin{displaymath}
        \operatorname{gen.time}(F_\ast^e R) \geq n.
    \end{displaymath}
\end{corollary}

\begin{proof}
    As $R$ is projective, each $f_j$ is an $F_\ast^e R$-coghost map in $D^b (\operatorname{mod}R)$. Since the product is nonzero, 
    \begin{displaymath}
        n \leq \operatorname{cogin}_{\operatorname{mod}R}^{F_\ast^e R} (R) \leq \operatorname{cogin}_{D^b(\operatorname{mod}R)}^{F_\ast^e R} (R)
    \end{displaymath}
    Note this inequality holds as $\operatorname{Ext}^n_R (R, F_\ast^e R) =0$ when $n\not=0$, and so $F_\ast^e R$-coghost endomorphisms of $R$ in $\operatorname{mod}R$ extend to such in $D^b (\operatorname{mod}R)$.
\end{proof}

\bibliographystyle{amsplain}
\bibliography{mainbib}

\end{document}
