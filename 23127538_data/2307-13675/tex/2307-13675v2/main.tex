\documentclass{amsart}
\usepackage[utf8]{inputenc}
\usepackage{amssymb,amsmath,enumerate,amsthm,amsfonts, cite, colonequals, stmaryrd}

\makeatletter
\@namedef{subjclassname@2020}{\textup{2020} Mathematics Subject Classification}
\makeatother

%\usepackage{showkeys}

\usepackage[colorlinks]{hyperref}
\hypersetup{
 colorlinks=true,
 linkcolor=blue,
 filecolor=magenta, 
 urlcolor=cyan,
}

\PassOptionsToPackage{hyphens}{url}\usepackage{hyperref}

\newcommand{\pat}[1]{{\color{green} #1}}

\usepackage[capitalise]{cleveref}
\crefformat{equation}{(#2#1#3)}
\crefrangeformat{equation}{(#3#1#4--#5#2#6)}
\crefformat{enumi}{(#2#1#3)}
\crefrangeformat{enumi}{(#3#1#4--#5#2#6)}

\newcommand{\creflastconjunction}{, and\nobreakspace}

\usepackage[pagewise]{lineno}
\overfullrule = 100pt
\let\oldequation\equation
\let\oldendequation\endequation
\renewenvironment{equation}{\linenomathNonumbers\oldequation}{\oldendequation\endlinenomath}
\expandafter\let\expandafter\oldequationstar\csname equation*\endcsname
\expandafter\let\expandafter\oldendequationstar\csname endequation*\endcsname
\renewenvironment{equation*}{\linenomathNonumbers\oldequationstar}{\oldendequationstar\endlinenomath}
\let\oldalign\align
\let\oldendalign\endalign
\renewenvironment{align}{\linenomathNonumbers\oldalign}{\oldendalign\endlinenomath}
\expandafter\let\expandafter\oldalignstar\csname align*\endcsname
\expandafter\let\expandafter\oldendalignstar\csname endalign*\endcsname
\renewenvironment{align*}{\linenomathNonumbers\oldalignstar}{\oldendalignstar\endlinenomath}

%%% TIKZ-CD --------------------------------------------------
\usepackage{tikz-cd}

%%% MATHSYMBOLS ----------------------------------------------
\makeatletter
\makeatother

\newcommand{\com}[1]{{\color{magenta} #1}}   
\newcommand{\old}[1]{{\color{red} #1}}
\newcommand{\new}[1]{{\color{blue} #1}}

\newcounter{intro}
\newcounter{result}
\newcounter{bigresult}

\newtheorem{introthm}[intro]{Theorem}
\renewcommand{\theintro}{\Alph{intro}}
\newtheorem{introcor}[intro]{Corollary}
\renewcommand{\theintro}{\Alph{intro}}
\newtheorem{introconjecture}[intro]{Conjecture}
\renewcommand{\theintro}{\Alph{intro}}
\newtheorem{introquestion}[intro]{Question}
\renewcommand{\theintro}{\Alph{intro}}
\newtheorem{introprop}[intro]{Proposition}
\renewcommand{\theintro}{\Alph{intro}}

\theoremstyle{theorem}
\newtheorem{theorem}[result]{Theorem}
\newtheorem{lemma}[result]{Lemma}
\newtheorem{proposition}[result]{Proposition}
\newtheorem{corollary}[result]{Corollary}
\newtheorem{Theorem}[bigresult]{Theorem}
\newtheorem{Proposition}[bigresult]{Proposition}
\newtheorem{Corollary}[bigresult]{Corollary}

\theoremstyle{definition}
\newtheorem{definition}[result]{Definition}
\newtheorem{example}[result]{Example}
\newtheorem{chunk}[result]{}
\newtheorem{Example}[bigresult]{Example}

\theoremstyle{remark}
\newtheorem{remark}[result]{Remark}
\newtheorem{conjecture}[result]{Conjecture}
\newtheorem{construction}[result]{Construction}
\newtheorem{convention}[result]{Convention}
\newtheorem{notation}[result]{Notation}
\newtheorem{question}[result]{Question}
\newtheorem{problem}[result]{Problem}
\newtheorem{setup}[result]{Setup}
\newtheorem{Question}[bigresult]{Question}
\newtheorem{Remark}[bigresult]{Remark}

\newtheorem*{ack}{Acknowledgements}

\setcounter{secnumdepth}{2}
\numberwithin{equation}{subsection}
\numberwithin{result}{subsection}
\numberwithin{bigresult}{section}

%%% DOCUMENT INFORMATION --------------------------------------------------
\title[Generation \& Module categories]{Strong generation and (co)ghost index for module categories}

\author[S.~Dey]{Souvik Dey}
\address{S.~Dey,
Faculty of Mathematics and Physics,
Department of Algebra,
Charles University, 
Sokolovsk\'{a} 83, 186 75 Praha, 
Czech Republic}
\email{souvik.dey@matfyz.cuni.cz}

\author[P.~Lank]{Pat Lank}
\address{P.~Lank,
Department of Mathematics,
University of South Carolina, 
Columbia, SC 29208,
U.S.A.}
\email{plankmathematics@gmail.com}

\author[R.~Takahashi]{Ryo Takahashi}
\address{R.~Takahashi,
Graduate School of Mathematics,
Nagoya University,
\break Furocho, Chikusaku, Nagoya 464-8602, Japan}
\email{takahashi@math.nagoya-u.ac.jp}

\date{\today}

\keywords{derived categories, strong generator, module categories, Rouquier dimension, Frobenius pushforward, singularity category, maximal Cohen-Macaulay modules, coghost lemma, local-to-global principle}

\subjclass[2020]{13D09 (primary), 13C60, 13C14, 13D05, 13D02, 13F40, 18G80} 

\begin{document}

\begin{abstract}
    This work focuses on notions of generation and (co)ghost index in the module category of a Noetherian commutative ring. In particular, a sufficiency condition is established for the existence of strong generators in the module category of a Noetherian ring. As a result, this provides an affirmative answer to a question posed by Iyengar and Takahashi. Furthermore, the techniques developed provide upper bounds on the Rouquier dimension and rank respectively for the singularity category and category of maximal Cohen-Macaulay modules. Additionally, a local-to-global principle for (co)ghost index in the module category is investigated, and explicit computations are provided.
\end{abstract}

\maketitle
\setcounter{tocdepth}{1}
%\tableofcontents

%%%%%%%%%%%%%%%%%%%%%%%%%%%%%%%%%%%%%%%
\section{Introduction}
\label{sec:intro}
%%%%%%%%%%%%%%%%%%%%%%%%%%%%%%%%%%%%%%%

All rings considered are commutative and unital. The category of finitely generated modules over a Noetherian  ring $R$, which is denoted $\operatorname{mod}R$, encodes important homological information capturing interesting ring theoretic properties. For instance, every finitely generated $R$-module $M$ of finite projective dimension admits a finite projective resolution. This can be interpreted as $M$ being \textit{finitely built} from $R$ by taking finitely many cokernels of monomorphisms and direct sums of $R$. However, if $M$ does not have finite projective dimension, then $M$ is not necessarily finitely built from $R$ in this fashion. Hence, one may wonder whether or not there is a similar process to finitely build objects from one another.

This is a problem that is well-suited to be studied in the derived category $D^b (\operatorname{mod}R)$ of bounded complexes of $R$-modules with finitely generated cohomology. The additive category $D^b (\operatorname{mod}R)$ is triangulated, so it closed under cones, shifts, and direct summands. These three operations serve as tools describing a process for finitely building objects from one another.

Let $\mathcal{T}$ be a triangulated category. The following was introduced in \cite{BVdB:2003,ABIM:2010,Rouquier:2008}. An object $G$ in $\mathcal{T}$ is said to be a \textit{classical generator} if every object $E$ in $\mathcal{T}$ can be finitely built from $G$ by taking cones, shifts, and direct summands. The minimal number of cones needed to finitely build $E$ from $G$ is called the \textit{level} of $E$ with respect to $G$ and is denoted $\operatorname{level}_\mathcal{T}^G (E)$. If the maximum of $\operatorname{level}_\mathcal{T}^G (E)$ is finite as one ranges over all objects $E$ in $\mathcal{T}$, then $G$ is called a \textit{strong generator} and this value is called its \textit{generation time}. The \textit{Rouquier dimension} of $\mathcal{T}$ is the minimal generation time amongst all strong generators, and is denoted $\dim \mathcal{T}$.

The study of Rouquier dimension has provided a flavorful palette of research in recent years for algebraic geometry and commutative algebra. For instance, it is conjectured that the Krull dimension of a smooth projective variety $X$ over a field is the Rouquier dimension of the derived category of bounded complexes with coherent cohomology $D^b(\operatorname{coh}X)$ \cite{Orlov:2009}. This open problem has been confirmed for various types of varieties: smooth proper curves \cite{Orlov:2009}; Del Pezzo surfaces, Fano threefolds of type $V_5$ or $V_{22}$ \cite{Ballard/Favero:2012}; projective spaces, quadrics, Grassmannians \cite{Rouquier:2008}; a product of two Fermat elliptic curves, a Fermat $K3$ surface \cite{Ballard/Favero/Katzarkov:2014}; and toric varieties \cite{Hanlon/Hicks/Lazarev:2023}. However, if the variety is not smooth, then the story for Rouquier dimension of $D^b (\operatorname{coh}X)$ is not as well understood when $X$ is singular. If $X$ is a singular projective curve, then there exist upper bounds \cite{Buban/Drozd:2011,Burban/Drozd:2017} on $\dim D^b (\operatorname{coh}X)$. More generally, one can utilize a resolution of singularities $\pi \colon \widetilde{X} \to X$ to bound $\dim D^b (\operatorname{coh}X)$ in terms of a controlled multiple of $\dim D^b(\operatorname{coh}\widetilde{X})$ \cite{Lank:2023}. These results are geometric in flavor and our work explores methods which yield upper bounds for Rouquier dimension in the algebraic context. 

A classical example for our context is a Noetherian regular ring $R$ of finite Krull dimension. It turns out that $R$ is a strong generator for $D^b (\operatorname{mod}R)$ whose generation time is at most $\dim R$ \cite{Christensen:1998}. In fact, $R$ is regular, if and only if, $R$ is a strong generator for $D^b (\operatorname{mod}R)$ \cite{Rouquier:2008}. This tells us that strong generators can be utilized to characterize regularity as a statement in $D^b (\operatorname{mod}R)$. Another instance where a ring theoretic property can be captured is in \cite{Pollitz:2019, Pollitz:2021,Briggs/Grifo/Pollitz:2022,Dwyer/Greenlees/Iyengar:2006} where it was shown that a commutative Noetherian ring has locally complete intersection singularities if, and only if, every nonzero object in its bounded derived category finitely builds a perfect complex. 

Recently, \cite{Aoki:2021} exhibited that for any Noetherian quasi-excellent ring $R$ of finite Krull dimension, the category $D^b (\operatorname{mod}R)$ admits a strong generator. Consequently, this tells us that their existence is known for a large number of cases, but to explicitly write such objects down is a difficult task. This work tackles the task of concretely describing strong generators in $D^b (\operatorname{mod}R)$ by utilizing a notion of strong generation in the module category. Let $R$ be a Noetherian  ring. A subcategory $\mathcal{S}$ in $\operatorname{mod}R$ is said to be \textit{thick} when it is closed under cokernels of monomoprhims, kernels of epimorphisms, and direct summands. Given any object $G$ in $\operatorname{mod}R$, one can consider the smallest thick subcategory of $\operatorname{mod}R$ containing $G$, which is denoted $\operatorname{thick}_{\operatorname{mod}R} (G)$. If $G$ is an object of $\operatorname{mod}R$ satisfying $\operatorname{thick}_{\operatorname{mod}R} (G) = \operatorname{mod}R$, then $G$ is called a \textit{classical generator}. Recall that $R$ is said to be \textit{$J\textrm{-}2$} when every module-finite $R$-algebra has an open regular locus \cite[$\S 32$]{Matsumura:1989}. In \cite{Iyengar/Takahashi:2019}, it was shown that $R$ is $J\textrm{-}2$ if, and only if, $\operatorname{mod}S$ admits a classical generator for all module-finite $R$-algebras $S$ if, and only if, $D^b(\operatorname{mod}S)$ admits a classical generator for all module-finite $R$-algebras $S$. Once more, this exhibits that a ring theoretic property can be captured as a statement about generation of modules or bounded complexes thereof.

There is a notion of strong generation in $\operatorname{mod}R$, and it is the primary focus of our work. An object $G$ in $\operatorname{mod} R$ is said to be a \textit{strong generator} when there exist an $n, s\geq 0$ such that the $s$-th syzygy of any object in $\operatorname{mod} R$ can be finitely built using only extensions and direct summands of $G$. For instance, equicharacteristic excellent local rings or algebras essentially of finite type over a field admit strong generators in $\operatorname{mod} R$ \cite{Iyengar/Takahashi:2016}. A strong generator differs from a classical generator in the way for which objects can be built from one another, i.e. up to a syzygy versus on the nose. However, these two types of objects are closely related, and, in fact, a strong generator is a classical generator, but not necessarily conversely. 

In many cases, the existence of strong generators in $\operatorname{mod}R$ have been detected by the nontriviality of a particular ideal in $R$ \cite{Iyengar/Takahashi:2016}. The \textit{$n$-th cohomology annihilator ideal} of $R$, denoted $\operatorname{ca}^n (R)$, is the intersection of all annihilator ideals for $\operatorname{Ext}^s_R (M,N)$ where $M,N$ range over all objects in $\operatorname{mod}R$ and $s\geq n$. They form a tower of ideals $\operatorname{ca}^n (R) \subseteq \operatorname{ca}^{n+1} (R)$, and their union forms an ideal $\operatorname{ca} (R)$. This has been extensively studied in \cite{Iyengar/Takahashi:2016,BHST:2016}. The nontriviality of $\operatorname{ca}^s (R)$ has been linked to the existence of strong generators in $\operatorname{mod}R$. For instance, \cite{Iyengar/Takahashi:2016} showed that Noetherian equicharacteristic excellent local rings have nonzero cohomology annihilator ideals, and as a consequence their module categories have strong generators. In light of such results, \cite[Question 5.5]{Iyengar/Takahashi:2016} poses the following regarding the existence of strong generators in module categories for a larger class of rings.

\begin{introquestion}\label{q:Iyengar/Takahashi}
    If $R$ is a Noetherian excellent ring of finite Krull dimension, then does $\operatorname{mod}R$ admit strong generators?
\end{introquestion} 
 
This brings attention to the first result, which establishes a sufficient condition for the existence of strong generators in $\operatorname{mod} R$,which appears later as Theorem~\ref{thm:strong_generation}.

\begin{introthm}\label{introthm:strong_gen}
    For a Noetherian ring $R$ with finite Krull dimension, the following are equivalent:
    \begin{enumerate}
        \item $\operatorname{ca}(R/\mathfrak{p})\neq 0$ for every $\mathfrak{p}\in \operatorname{Spec}(R)$;
        \item $\operatorname{mod}R/\mathfrak{p}$ admits a strong generator for every $\mathfrak{p}\in \operatorname{Spec}(R)$;
        \item $D^b (\operatorname{mod}R/\mathfrak{p})$ has finite Rouquier dimension for every $\mathfrak{p}\in \operatorname{Spec}(R)$.
    \end{enumerate}
    If any one of these conditions is satisfied, then both $\operatorname{mod}R/I$ and $D^b (\operatorname{mod}R/I)$ admit strong generators where $I$ is any ideal of $R$.
\end{introthm}   

An upshot of Theorem~\ref{introthm:strong_gen} is that it answers Question~\ref{q:Iyengar/Takahashi} affirmatively and for a larger class of rings. The following result appears later as Corollary~\ref{cor:quasi_excellent_strong_gen}.

\begin{introcor}\label{introcor:quasi_excellent_strong_gen}
    For any Noetherian quasi-excellent ring $R$ of finite Krull dimension, $\operatorname{mod}R$ admits a strong generator. In particular, Question~\ref{q:Iyengar/Takahashi} has an affirmative answer.
\end{introcor}

Corollary~\ref{introcor:quasi_excellent_strong_gen} allows us to explicitly find an object in $\operatorname{mod}R$ that can finitely build any object in $\operatorname{mod}R$ using only cokernels of monomorphisms, kernels of epimorphisms, and direct summands. Suppose $R$ is a Noetherian ring of prime characteristic $p$. Recall the Frobenius morphism $F\colon R \to R$ is the ring homomorphism given by $r\mapsto r^p$. If $F\colon R \to R$ is finite, then we say $R$ is \textit{$F$-finite}. This class of rings are excellent \cite{Kunz:1976}. For example, any essentially of finite type algebra over a perfect field of prime characteristic are $F$-finite. Recently, \cite{BILMP:2023} showed that $F_\ast^e R$ is a strong generator for $D^b (\operatorname{mod}R)$ when $e \gg 0$ and $F_\ast^e R$ is $R$ viewed as a module of itself via restriction of scalars along the $e$-iterate of $F\colon R \to R$. The following result is a module-theoretic analog of \cite{BILMP:2023}, and appears as Corollary~\ref{cor:f_finite_strong_gen_module_cat}.

\begin{introcor}
        If $R$ is an $F$-finite Noetherian ring, then for each $e \gg 0$ there exist $t,m,c\geq 0$ such that 
    \begin{displaymath}
        \operatorname{mod} R=\operatorname{thick}_{\operatorname{mod}R}^m \bigg(R \oplus \big( \bigoplus^c_{i=0} \Omega_R^i (F_\ast^e R) \big) \bigg).
    \end{displaymath}
\end{introcor}

An application of interest coming from Theorem~\ref{introthm:strong_gen} is that it yields upper bounds on the Rouquier dimension for the singularity category of $R$. Recall the singularity category of $R$ is the Verdier quotient $D_{\textrm{sg}}(R) :=D^b(\operatorname{mod}R)/\operatorname{perf}R$ where $\operatorname{perf}R$ denotes the full subcategory of objects in $D^b (\operatorname{mod}R)$ which may be finitely built by $R$ \cite{Buchweitz:2021}. See Section~\ref{sec:singularity_category_bounds} for background. Note that $D_{\textrm{sg}}(R)$ has a triangulated structure induced from $D^b (\operatorname{mod}R)$. For any subcategory $\mathcal{S}$ in $\operatorname{mod}R$, there is an invariant called the \textit{radius of $\mathcal{S}$} which roughly speaking, is defined as the least number of extensions required to finitely build every object in $\mathcal{S}$ from a fixed choice of object in $\operatorname{mod}R$. This value is denoted $\operatorname{radius}\mathcal{S}$, and was first introduced in \cite{Dao/Takahashi:2014}. Let $\Omega^n_R (\mathcal{S})$ denote the collection of objects of the form $\Omega_R^n (E)$ for each object $E$ in $\mathcal{S}$. The next result states an upper bound on Rouquier dimension of $D_{\textrm{sg}}(R)$, which is Proposition~\ref{prop:singularity_category_bounds_via_radius}.

\begin{introprop}\label{introprop:singularity_category_bounds_via_radius}
    If $R$ is a Noetherian ring and $\Omega^n_R (\operatorname{mod} R)$ has finite radius for some $n\geq 0$, then one has the following upper bound on the Rouquier dimension of the singularity category:
    \begin{displaymath}
        \dim D_{\textrm{sg}}(R) \leq 
        \underset{n > 0}{\inf}\{ \operatorname{radius} \Omega^n_R (\operatorname{mod} R)\} .
    \end{displaymath}
\end{introprop}

Proposition~\ref{introprop:singularity_category_bounds_via_radius} presents a strengthening to \cite[Lemma 7.1.(1)]{Iyengar/Takahashi:2016} where upper bounds on the Rouquier dimension of $D_{\textrm{sg}}(R)$ are established from the two parameters of a strong generator in $\operatorname{mod}R$. Moreover, it is possible to establish bounds on rank of the category $\operatorname{CM}(R)$ of maximal Cohen-Macaulay modules (see Section~\ref{sec:MCM_category}). Recall the \textit{rank} of a subcategory $\mathcal{S}$ in $\operatorname{mod}R$ is an invariant that is closely related to the size of $\mathcal{S}$, and it measures the number of extensions to finitely build all objects in $\mathcal{S}$ using every syzygy of an object $G$ in $\operatorname{mod}R$. This was introduced in \cite{Dao/Takahashi:2014}, and further details are presented in Section~\ref{sec:generation_via_syzygys}. The next result establishes upper bound on rank for $\operatorname{CM}(R)$. This appears later as Proposition~\ref{prop:CM_category_bounds}.

\begin{introprop}\label{introprop:CM_category_bounds}
    Let $R$ be a Cohen-Macaulay local ring of dimension $d$ admitting a canonical module.  If $R/\mathfrak{p}$ is Cohen-Macaulay for every $\mathfrak{p} \in \operatorname{Min}(R)$, then \begin{displaymath}
        \operatorname{rank}\operatorname{CM}(R)\leq 2(d+1)^2 \left(\sum_{\mathfrak{p}\in \operatorname{Min}(R)} \ell\ell(R_{\mathfrak{p}})\right)(s+1)-1
    \end{displaymath} where $s:=\underset{\mathfrak{p} \in \operatorname{Min}(R)}{\sup} \{\operatorname{size}\operatorname{CM}(R/\mathfrak{p}) \}$.  
\end{introprop}

Another objective of this work is to develop techniques for studying values that are related to the parameters describing how many syzygys and extensions are required for an object to be a strong generator in the module category. Let $R$ be   Noetherian ring. Recall that the level of an object $E$ with respect to an object $G$ in $D^b (\operatorname{mod}R)$ measures the number of required cones to finitely build the former from the latter. Level has connections to other familiar invariants. For instance, if $M$ is a finitely generated module, then level of $M$ with respect to $R$ is the projective dimension of $M$ \cite{Christensen:1998}. 

For triangulated categories, \cite{Kelly:1965, OS:2012, Beligiannis:2008} established a connection between how many cones it takes to finitely build an object to the vanishing of the composition for a certain class maps called (co)ghosts. This provides an effective method to compute level. Let $E,G$ be a pair of objects in $D^b (\operatorname{mod}R)$. A map in $D^b (\operatorname{mod}R)$ is \textit{$G$-coghost} if it vanishes post-composition with any shift of $G$. The \textit{coghost index} of $E$ with respect to $G$, denoted by $\operatorname{cogin}_R^G (E)$, is the minimal integer $n$ such that any $n$-fold composition of $G$-coghost maps ending at $E$ vanish. It was shown in \cite{OS:2012} that $\operatorname{level}_R^G (E) = \operatorname{cogin}^G_R (E)$.

Following \cite{Beligiannis:2008}, (co)ghost maps in $\operatorname{mod}R$ can be made sensed of. Consider a pair of objects $E,G$ in $\operatorname{mod}R$. A map in $\operatorname{mod}R$ is \textit{$G$-ghost} if it vanishes pre-composition with $G$. The \textit{ghost index} of $E$ with respect to $G$ is the minimal integer $n$ such that any $n$-fold composition of $G$-coghost maps starting at $E$ vanishes, and is denoted $\operatorname{gin}^G_{\operatorname{mod}R} (E)$. By dualizing everything mentioned, this yields a notion of \textit{coghost index}. These values are closely related to the parameters of a strong generator for modules categories. Indeed, if $G$ is a strong generator for $\operatorname{mod}R$ such that the $s$-th syzygy of any object in $\operatorname{mod}R$ can be finitely built from $G$ using up to $n$ extensions, then the (co)ghost index with respect to $G$ is at most $n$. See Section~\ref{sec:coghost_and_approximations} for details on background.

The behavior for (co)ghost index in $D^b (\operatorname{mod}R)$ under ring theoretic operations such as localization has been studied \cite{Letz:2021, Liu/Pollitz:2023}. For instance, given any pair of objects $E,G$ in $D^b (\operatorname{mod}R)$, \cite{Letz:2021} exhibited that $\operatorname{level}^G_R (E) < \infty$ if, and only if, $\operatorname{level}_{R_\mathfrak{p}}^{G_\mathfrak{p}} (E_\mathfrak{p})<\infty$ for all prime ideals $\mathfrak{p}$ in $R$. This a local-to-global principle for level in $D^b (\operatorname{mod}R)$, and motivates the next result of this work which establishes a local-to-global principle for (co)ghost index in module categories. This appears later as Theorem~\ref{thm:ghost_index_local_global}.

\begin{introthm}\label{introthm:ghost_index_local_global}
    Let $R$ be a Noetherian ring. If $E,G\in \operatorname{mod}R$, then the following are equivalent:
    \begin{enumerate}
        \item $\operatorname{gin}^G_{\operatorname{mod}R} (E)<\infty$;
        \item $\operatorname{gin}^{G_\mathfrak{m}}_{\operatorname{mod}R_\mathfrak{m}} (E_\mathfrak{m}) < \infty$ for all $\mathfrak{m}\in \operatorname{mSpec}(R)$;
        \item $\operatorname{gin}^{G_\mathfrak{p}}_{\operatorname{mod}R_\mathfrak{p}} (E_\mathfrak{p}) < \infty$ for all $\mathfrak{p}\in \operatorname{Spec}(R)$.
    \end{enumerate}
    A similar statement holds for coghost index.
\end{introthm}

There are interesting applications of Theorem~\ref{introthm:ghost_index_local_global} for when $R$ is $F$-finite. For instance, there are lower bounds on generation time for $F_\ast^e R$ in the $D^b (\operatorname{mod}R)$ when $R$ is $F$-finite, cf. Corollary~\ref{cor:f_finite_strong_gen_module_cat}. Another is the number of iterates of $F\colon R \to R$ needed to ensure $F_\ast^e R$ is a strong generator for $D^b (\operatorname{mod}R)$, cf. Proposition~\ref{prop:f_finite_non_generator}. In Section~\ref{sec:co_ghost_mod_cats_results}, there are explicit examples computed for coghost index for a pair of objects. 

\begin{ack}
    The authors would like to thank discussions with Srikanth Iyengar, Janina Letz, and Josh Pollitz for their wonderfully valuable comments and perspectives shared that led to improvements on this manuscript.
\end{ack}

%%%%%%%%%%%%%%%%%%%%%%%%%%%%%%%%%%%%%%%
\section{Thick subcategories \& approximations}
\label{sec:thick_subcategories}
%%%%%%%%%%%%%%%%%%%%%%%%%%%%%%%%%%%%%%%

This section recalls background for generation in module categories and its corresponding bounded derived category.

%%%%%%%%%%%%%%%%%%%%%%%%%%%%%%%%%%%%%%%
\subsection{Notation and conventions}
\label{sec:notation_convention}
%%%%%%%%%%%%%%%%%%%%%%%%%%%%%%%%%%%%%%%

Throughout this section, we fix a Noetherian ring $R$. All rings are assumed to be commutative and unital. The category of finitely generated $R$-modules is denoted by $\operatorname{mod}R$, its bounded derived category by $D^b (\operatorname{mod}R)$, and the unbounded derived category of complexes of $R$-modules by $D(R)$. Note that $D^b (\operatorname{mod}R)$ is a full triangulated subcategory of $D(R)$. Given a subcategory $\mathcal{S}$ of $\operatorname{mod}R$, the full subcategory of $\operatorname{mod}R$ consisting of direct summands of finite direct sums of objects in $\mathcal{S}$ is denoted $\operatorname{add}\mathcal{S}$. Recall a \textbf{syzygy} of $M\in \operatorname{mod}R$ is the kernel of an epimorphism $\pi \colon P \to M$ where $P$ is a finitely generated proective $R$-module. This is wel-defined up to stable isomorphism. For any $M\in \operatorname{mod}R$ and $n \geq 0$, an $n$-th syzygy of $M$ is denoted $\Omega^n_R (M)$ and is defined as follows. Set $\Omega^0_R (M):= M$. If $n=1$, then $\Omega^1_R (M)$ is a syzygy of $M$. For $n>1$, then $\Omega^n_R (M)$ is a syzygy of $\Omega^{n-1}_R (M)$.

%%%%%%%%%%%%%%%%%%%%%%%%%%%%%%%%%%%%%%%
\subsection{Thick subcategories in $D^b(\operatorname{mod}R)$}
\label{sec:thick_subcategories_derived_category}
%%%%%%%%%%%%%%%%%%%%%%%%%%%%%%%%%%%%%%%

This section discusses thick subcategories and generation in $D^b (\operatorname{mod}R)$. The primary sources of reference are \cite{BVdB:2003, Krause:2022}. It should be noted that much of this discussion holds more generally in triangulated categories.

\begin{chunk}\label{c:thick_subcategories}
    A full triangulated subcategory $\mathcal{S}$ of $D(R)$ is said to be \textbf{thick} when it is closed under direct summands. These subcategories are closed under cones, shifts, and direct summands. For example, $D^b(\operatorname{mod}R)$ is a thick subcategory of $D(R)$. We are primarily interested in thick subcategories of $D^b (\operatorname{mod}R)$. For any $G\in D^b(\operatorname{mod}R)$, the smallest thick subcategory of $D^b(\operatorname{mod}R)$ containing $G$ is denoted $\operatorname{thick}_{D(R)} (G)$. This category always exists, and it can be filtered in the following way by inductively defining certain full additive subcategories in $D^b(\operatorname{mod}R)$. Let $\operatorname{thick}^0_{D(R)} (G)$ be the full subcategory consisting of all objects isomorphic to the zero object. Define $\operatorname{thick}^1_{D(R)} (G)$ to be the full subcategory consisting of bounded complexes which are direct summands of objects of the form $\bigoplus_{s\in \mathbb{Z}} G^{\oplus r_s}[s]$ whose differential is zero. If $n\geq 2$, then $\operatorname{thick}^n_{D(R)} (G)$ is the full subcategory consisting of objects which are direct summands of objects $E$ fitting into a distinguished triangle
    \begin{displaymath}
        A \to E \to B \to A[1]
    \end{displaymath}
    where $A\in \operatorname{thick}^{n-1}_{D(R)} (G)$ and $B\in \operatorname{thick}^1_{D(R)} (G)$. This yields an ascending filtration of additive subcategories:
    \begin{displaymath}
        \begin{aligned}
            \operatorname{thick}^0_{D(R)}  & (G) \subseteq \operatorname{thick}^1_{D(R)} (G) \subseteq \cdots \\& \subseteq \operatorname{thick}^n_{D(R)} (G) \subseteq \cdots \subseteq \bigcup^\infty_{n=0} \operatorname{thick}^n_{D(R)} (G)= \operatorname{thick}_{D(R)} (G).
        \end{aligned}
    \end{displaymath}
    If $E\in \operatorname{thick}_{D(R)} (G)$, then we say that $G$ \textbf{finitely builds} $E$. This filtration keeps track of the minimal number of required cones required to finitely build an object from $G$.
\end{chunk}

\begin{chunk}\label{c:generator_derived_category}
    An object $G$ in $D^b (\operatorname{mod}R)$ is a \textbf{classical generator} if $\operatorname{thick}_{D(R)} (G) = D^b (\operatorname{mod}R)$. Additionally, $G$ is called a \textbf{strong generator} if there exists $n\geq 0$ such that $\operatorname{thick}^{n+1}_{D(R)} (G) = D^b (\operatorname{mod}R)$, and the smallest such value $n$ is its \textbf{generation time}, which is denoted $\operatorname{gen.time}(G)$. The \textbf{Rouquier dimension} of $D^b (\operatorname{mod}R)$ is the minimum over all possible generation times, and it is denoted $\dim D^b(\operatorname{mod}R)$.
\end{chunk}

\begin{example}
    If $R$ is a quasi-excellent ring of finite Krull dimension, then $D^b (\operatorname{mod}R)$ admit strong generators \cite[Main Theorem]{Aoki:2021}. For instance, if $R$ is regular of finite Krull dimension, then $R$ finitely builds every $E \in D^b (\operatorname{mod}R)$ in at most $\dim R + 1$ cones \cite{Christensen:1998}.
\end{example}

\begin{lemma}\label{lem:reduced_rouquier_bound}
    If $R$ is a reduced ring, then $\dim R -1 \leq \dim D^b(\operatorname{mod}R)$.
\end{lemma}

\begin{proof}
    This is a direct consequence of \cite[Lemma 4.2(3), Corollary 6.6]{Aihara/Takahashi:2015} coupled with the observation that localization of a reduced ring is again reduced.  
\end{proof}   

%%%%%%%%%%%%%%%%%%%%%%%%%%%%%%%%%%%%%%%
\subsection{Classical generation in $\operatorname{mod}R$}
\label{sec:thick_subcategories_module_category}
%%%%%%%%%%%%%%%%%%%%%%%%%%%%%%%%%%%%%%%

This section discusses classical generation and thick subcategories in $\operatorname{mod}R$. The primary source of reference is \cite[$\S 4$]{Iyengar/Takahashi:2016}.

\begin{chunk}\label{c:thick_subcategory_module_category}
    A full additive subcategory $\mathcal{S}\subseteq \operatorname{mod}R$ is \textbf{thick} when it is closed under direct summands and for any short exact sequence
    \begin{displaymath}
        0 \to A \to B \to C \to 0,
    \end{displaymath}
    if two of the three belong to $\mathcal{S}$, then so does the third. This notion is also found in \cite{Krause/Stevenson:2013}. These subcategories are closed under kernels of epimorphisms, cokernels of monomorphism, and direct summands of objects. Given $\mathcal{S}\subseteq \operatorname{mod}R$, the smallest thick subcategory in $\operatorname{mod}R$ containing $\mathcal{S}$ is denoted by $\operatorname{thick}_{\operatorname{mod}R} (\mathcal{S})$. An object $G\in \operatorname{mod}R$ is said to be a \textbf{classical generator} when $\operatorname{thick}_{\operatorname{mod}R} (G) = \operatorname{mod}R$. 
\end{chunk}

\begin{example}\label{ex:f_finite_ring}
    Suppose $R$ is a Noetherian ring of prime characteristic $p$ and of finite Krull dimension. Recall the Frobenius morphism $F\colon R \to R$ is given by $r\mapsto r^p$. If $F\colon R \to R$ is a finite ring homomorphism, then $R$ is said to be \textbf{$F$-finite}. That is, $F_\ast R \in \operatorname{mod}R$ where $F_\ast R$ denotes restriction of scalars along $F\colon R \to R$. For instance, any ring that is essentially of finite type over a perfect field of prime characteristic is $F$-finite. These rings are excellent \cite[Theorem 2.5]{Kunz:1976}. In \cite[Corollary 3.9]{BILMP:2023} it was shown there exists $e >0$ such that $F_\ast^e R$ is a strong generator for $D^b (\operatorname{mod}R)$. By \cite[Theorem 1]{Krause/Stevenson:2013} (see also \cite[Theorem 10.5]{Takahashi:2023}), it can be checked that $R\oplus F_\ast^e R$ is a classical generator for $\operatorname{mod}R$.
\end{example}

\begin{chunk}
    Suppose $\mathcal{S}$ is a subcategory of $\operatorname{mod}R$. One has an ascending chain of additive subcategories for $\operatorname{thick}_{\operatorname{mod}R} (\mathcal{S})$ which keeps track of finitely building objects from $\mathcal{S}$ using cokernels of mononomorphisms, kernels of epimorphisms, extensions, and direct summands. Let $\operatorname{thick}_{\operatorname{mod}R}^0(\mathcal{S})$ be the full subcategory consisting of objects isomorphic to the zero module, and $\operatorname{thick}_{\operatorname{mod}R}^1(\mathcal{S}) := \operatorname{add}\mathcal{S}$. If $n\geq 2$, then $\operatorname{thick}_{\operatorname{mod}R}^n(\mathcal{S})$ is the full subcategory of $\operatorname{mod}R$ whose objects are direct summands of an object fitting into a short exact sequence
    \begin{displaymath}
        0 \to X \to Y \to Z \to 0
    \end{displaymath}
    where amongst the other two, one belongs to $\operatorname{thick}_{\operatorname{mod}R}^{n-1}(\mathcal{S})$ and the other in $\operatorname{thick}_{\operatorname{mod}R}^1(\mathcal{S})$. This forms an ascending chain of subcategories in $\operatorname{mod}R$:
    \begin{displaymath}
        \operatorname{thick}_{\operatorname{mod}R}^0 (\mathcal{S}) \subseteq \operatorname{thick}_{\operatorname{mod}R}^1 (\mathcal{S}) \subseteq \operatorname{thick}_{\operatorname{mod}R}^2(\mathcal{S}) \subseteq \cdots \subseteq \operatorname{thick}_{\operatorname{mod}R}(\mathcal{S}).
    \end{displaymath}
\end{chunk}

%%%%%%%%%%%%%%%%%%%%%%%%%%%%%%%%%%%%%%%
\subsection{Strong generation in $\operatorname{mod}R$}
\label{sec:extension_construction}
%%%%%%%%%%%%%%%%%%%%%%%%%%%%%%%%%%%%%%%

This section deals with a notion of strong generation in $\operatorname{mod}R$, and the primary source is \cite[$\S 4$]{Iyengar/Takahashi:2016}.

\begin{chunk}
    Let $\mathcal{S}$ be a collection of objects in $\operatorname{mod}R$. Inductively construct an ascending chain of additive subcategories of $\operatorname{mod}R$ as follows. Let $|\mathcal{S}|_0$ be the full subcategory consisting of objects isomorphic to the zero module, and set $|\mathcal{S}|_1 := \operatorname{add}\mathcal{S}$. If $n\geq 2$, then $|\mathcal{S}|_n$ denotes the full subcategory of $\operatorname{mod}R$ whose objects $M$ fit into a short exact sequence
    \begin{displaymath}
        0 \to Y \to M\oplus W \to X \to 0
    \end{displaymath}
    where amongst $X$ and $Y$, one belongs to $|\mathcal{S}|_{n-1}$ and the other in $|\mathcal{S}|_1$. These subcategories form an ascending chain:
    \begin{equation}\label{eq:extension_subcategories_filtration}
        |\mathcal{S}|_1 \subseteq |\mathcal{S}|_2 \subseteq \cdots.
    \end{equation}
    If $|\mathcal{S}|_a \star |\mathcal{S}|_b$ denotes the full subcategory whose objects $E$ fit into a short exact sequence
    \begin{displaymath}
        0 \to X \to E \oplus E' \to Y \to 0
    \end{displaymath}
    where $X \in |\mathcal{S}|_a$ and $Y \in |\mathcal{S}|_b$, then $|\mathcal{S}|_a \star |\mathcal{S}|_b = |\mathcal{S}|_{a+b}$. For further details, see  \cite[$\S 5$]{Dao/Takahashi:2014}.
\end{chunk}

\begin{chunk}\label{c:strong_generator_module_category}
    If $G\in \operatorname{mod}R$, then it is a \textbf{strong generator} when $R$ is a direct summand of $G$ and $\Omega^s_R (\operatorname{mod}R)\subseteq |G|_n$ for some $s,n\geq 0$ where
    \begin{displaymath}
    \Omega_R^s (\operatorname{mod}R) := \{ \Omega^s_R (M) : M\in \operatorname{mod}R \}.
\end{displaymath}
\end{chunk}

\begin{lemma}\label{lem:thick_categories_module_property}
    If $G\in \operatorname{mod}R$, and $s,n\geq 0$, then
    \begin{enumerate}
        \item $|G|_n \subseteq \operatorname{thick}_{\operatorname{mod}R}^n (G)$;
        \item $\Omega_R^s (\operatorname{mod}R) \subseteq |G|_n \implies \operatorname{thick}_{\operatorname{mod}R}^{s+n} (R\oplus G) = \operatorname{mod}R$.
    \end{enumerate}
\end{lemma}

\begin{proof}
    This is \cite[Proposition 4.5 \& Corollary 4.6]{Iyengar/Takahashi:2016}. 
\end{proof}

\begin{example}\label{ex:Iyengar/Takahashi_strong_gen_examples}
    Suppose $R$ has finite Krull dimension $d$.
    \begin{itemize}
        \item If $R$ is an Artinian ring, then $\operatorname{mod}R \subseteq |R\oplus R/J(R)|_{\ell\ell(R)}$ where $J(R)$ is the Jacobson radical of $R$ and $\ell\ell(R)$ is the Loewy length.
        \item If $R$ is an excellent equicharacteristic local ring, then $\operatorname{mod}R$ admits a strong generator \cite[Theorem 5.3]{Iyengar/Takahashi:2016}.
        \item If $R$ is essentially of finite type over a field, then $\operatorname{mod}R$ admits a strong generator \cite[Theorem 5.4]{Iyengar/Takahashi:2016}.
    \end{itemize}
\end{example}

\begin{lemma}\label{lem:strong_generation_module_to_derived}
    Suppose $G\in \operatorname{mod}R$.
    \begin{itemize}
        \item If there exist $n,s\geq 0$ such that $\Omega_R^s (\operatorname{mod}R) \subseteq |G|_n$, then $\operatorname{thick}_{D(R)}^{2(n+s)} (G) = D^b (\operatorname{mod}R)$.
        \item If there exists $n\geq 0$ such that $\operatorname{thick}_{\operatorname{mod}R}^n (G)= \operatorname{mod}R$, then $\operatorname{thick}_{D(R)}^{2n} (G) = D^b (\operatorname{mod}R)$.
    \end{itemize}
\end{lemma}

\begin{proof}
    This is Lemma~\ref{lem:thick_categories_module_property} coupled with \cite[Lemma 7.1]{Iyengar/Takahashi:2016}.
\end{proof}

%%%%%%%%%%%%%%%%%%%%%%%%%%%%%%%%%%%%%%%
\subsection{Generation via syzygys}
\label{sec:generation_via_syzygys}
%%%%%%%%%%%%%%%%%%%%%%%%%%%%%%%%%%%%%%%

This section describes another form of generation that will be of use for Section~\ref{sec:applications}. The primary source of reference is \cite{Dao/Takahashi:2014}. 

\begin{chunk}
    If $\mathcal{S}$ is a subcategory of $\operatorname{mod}R$, then $\operatorname{size}\mathcal{S}$ (respectively $\operatorname{rank} \mathcal{S}$) denotes the smallest integer $n\geq 0$ such that $\mathcal{S}\subseteq |G|_{n+1}$ (respectively $\mathcal{S}=|G|_{n+1}$) for some $G\in \operatorname{mod}R$. This is \cite[Definition 5.2]{Dao/Takahashi:2014}. It can checked that $\operatorname{size} \mathcal{S}\leq \operatorname{rank} \mathcal{S}$. Furthermore, $\operatorname{mod} R$ has a strong generator if, and only if, $\operatorname{size} \Omega^s_R (\operatorname{mod}R)<\infty$ for some $s\geq 0$.
\end{chunk} 

\begin{chunk}
    Suppose $\mathcal{S}$ is a subcategory of $\operatorname{mod}R$. The additive closure in $\operatorname{mod}R$ of objects of the form $\Omega^n_R (E)$ for $E\in \mathcal{S}$ is denoted by $[ \mathcal{S} ]$. Inductively define the following additive subcategories of $\operatorname{mod}R$. Let $[\mathcal{S}]_0$ consist of objects isomorphic to the zero module, and $[\mathcal{S}]_1=[\mathcal{S}]$. If $n\geq 2$, then $[\mathcal{S}]_n$ is the additive closure consisting of all syzygys of objects $E$ in $\operatorname{mod}R$ which fit into a short exact sequence
    \begin{displaymath}
        0 \to A \to E \to B \to 0
    \end{displaymath}
    such that $A\in [\mathcal{S}]_{n-1}$ and $B\in [\mathcal{S}]_1$. If $n\geq 0$, then the \textbf{ball of radius $n$ centered at $\mathcal{S}$} is the additive subcategory $[\mathcal{S}]_n$. If $\mathcal{S}$ consists of only one object, then this is written as $[G]_n$.
\end{chunk}

\begin{chunk}
    Given a subcategory $\mathcal{S}$ in $\operatorname{mod}R$, the \textbf{radius} of $\mathcal{S}$ is denoted $\operatorname{radius}\mathcal{S}$ and is the minimal integer $n\geq 0$ such that there exists a ball of radius $n + 1$ centered at a module containing $\mathcal{S}$.
\end{chunk}

%%%%%%%%%%%%%%%%%%%%%%%%%%%%%%%%%%%%%%%
\subsection{Approximations}
\label{sec:coghost_and_approximations}
%%%%%%%%%%%%%%%%%%%%%%%%%%%%%%%%%%%%%%%

This section discusses approximations and (co)ghost maps in an abelian category. The primary reference is \cite{Beligiannis:2008}.

\begin{chunk}
    Consider an abelian category $\mathcal{A}$, and let $f\colon E \to G$ be a map in $\mathcal{A}$. Observe that pre- and post-composing with $f$ induces a map on hom-sets. Let $\mathcal{X}$ be an additive subcategory of $\mathcal{A}$. A map $f \colon X \to E$ in $\mathcal{A}$ with $X\in \mathcal{X}$ is called a \textbf{right $\mathcal{X}$-approximation} of $E$ if for every $f' \colon X' \to E$ with $X'\in \mathcal{X}$ there exists $g \colon X' \to X$ such that $f' = f \circ g$. If every $E\in \mathcal{A}$ admits a right $\mathcal{X}$-approximation, then $\mathcal{X}$ is \textbf{contravariantly finite}. 
\end{chunk}

\begin{example}
    For any $E\in \operatorname{mod}R$, $\operatorname{add}(E)$ is a contravariantly finite subcategory in $\operatorname{mod}R$ \cite[Example 12]{Takahashi:2021}.
\end{example}

\begin{chunk}
    Right $\mathcal{X}$-approximations are related to finitely building objects via extensions and direct summands. For $G\in \mathcal{A}$, $\operatorname{Fac}(G)$ denotes the set of all $E\in \mathcal{A}$ such that there exists an epimorphism $T \to E$ for some $T\in \operatorname{add}(G)$. If $G\in \mathcal{A}$ and $f \colon E \to D$ is a map, then $f$ is \textbf{$G$-ghost} when the induced map 
    \begin{displaymath}
        \operatorname{Hom}_\mathcal{A} (X,f)\colon  \operatorname{Hom}_\mathcal{A} (X,E) \to \operatorname{Hom}_\mathcal{A} (X,D)
    \end{displaymath}
    is zero for each $X \in \operatorname{add}(G)$. For $f\colon E \to D$, it is an \textbf{$n$-fold $G$-ghost} when it may be written as a composition of $n$ $G$-ghost maps.  If $E,G \in \mathcal{A}$, then the \textbf{ghost index} of $E$ with respect to $G$ is the smallest $n\geq 0$ such that any $n$-fold $G$-ghost map from $E$ vanishes and is denoted $\operatorname{gin}^G_\mathcal{A} (E)$.
\end{chunk} 

\begin{lemma}[Ghost lemma for abelian categories]\label{lem:ghost_lemma_abelian}
    Suppose $E,G\in \mathcal{A}$.
    \begin{itemize}
        \item If $E\in |\operatorname{Fac}(G)|_n$, then $\operatorname{gin}^G_\mathcal{A} (E) \leq n$.
        \item If $\operatorname{add}(G)$ is contravariantly finite, then the following are equivalent:
        \begin{enumerate}
            \item $E\in |\operatorname{Fac}(G)|_n$;
            \item $\operatorname{gin}^G_\mathcal{A} (E) \leq n$.
        \end{enumerate}
    \end{itemize}
\end{lemma}

\begin{proof}
    This is \cite[Lemma 1.3]{Beligiannis:2008}.
\end{proof}

\begin{remark}
    The ideas presented so far can also be applied to the opposite category $\mathcal{A}^{op}$, which leads to dualizations of the aforementioned concepts: \textit{left $\mathcal{X}$-approximations, covariant finiteness, $G$-coghost, $n$-fold $G$-coghost maps, coghost index, and  a coghost lemma for abelian categories.}
\end{remark}

%%%%%%%%%%%%%%%%%%%%%%%%%%%%%%%%%%%%%%%
\section{Strong generators}
\label{sec:strong_gen_excl}
%%%%%%%%%%%%%%%%%%%%%%%%%%%%%%%%%%%%%%%

This section establishes a connection between nonvanishing of the cohomology annihilator ideal $\operatorname{ca}(R)$ for a Noetherian ring $R$ and the existence of strong generators in both $\operatorname{mod}R$, $D^b (\operatorname{mod}R)$. The following main result is stated below.

\begin{Theorem}\label{thm:strong_generation}
    For a Noetherian ring $R$ with finite Krull dimension, the following are equivalent:
    \begin{enumerate}
        \item $\operatorname{ca}(R/\mathfrak{p})\neq 0$ for every $\mathfrak{p}\in \operatorname{Spec}(R)$;
        \item $\operatorname{mod}R/\mathfrak{p}$ admits a strong generator for every $\mathfrak{p}\in \operatorname{Spec}(R)$;
        \item $D^b (\operatorname{mod}R/\mathfrak{p})$ has finite Rouquier dimension for every $\mathfrak{p}\in \operatorname{Spec}(R)$.
    \end{enumerate}
    If any one of these conditions is satisfied, then both $\operatorname{mod}R/I$ and $D^b (\operatorname{mod}R/I)$ admit strong generators where $I$ is any ideal of $R$.
\end{Theorem}

The proof of Theorem~\ref{thm:strong_generation} will be accomplished in steps, but before doing so an important digression is made with cohomology annihilator ideals.

%%%%%%%%%%%%%%%%%%%%%%%%%%%%%%%%%%%%%%%
\subsection{Cohomology annihilator ideals}
\label{sec:cohomology_annihilator_ideals}
%%%%%%%%%%%%%%%%%%%%%%%%%%%%%%%%%%%%%%%

Let $R$ be a Noetherian ring. For each $n\geq0$, consider the following ideal of $R$:
\begin{displaymath}
    \operatorname{ca}^n (R):= \bigcup_{j=n}^\infty \bigcup_{E,E^\prime\in \operatorname{mod}R} \operatorname{ann}_R\operatorname{Ext}^j_R (E,E^\prime).
\end{displaymath}
The \textbf{cohomology annihilator (ideal)} of $R$ is given by $\operatorname{ca}(R) := \cup^\infty_{n=0}\operatorname{ca}^n (R)$. These ideals are closely related to the Rouquier dimension of $D^b (\operatorname{mod}R)$, see \cite{Iyengar/Takahashi:2016} and \cite{BHST:2016}. For instance, \cite[Theorem 5.3]{Iyengar/Takahashi:2016} establishes a proof for Question~\ref{q:Iyengar/Takahashi} in the equicharacteristic excellent local ring case via the nonvanishing of cohomology annihilator ideals. These ideals will be vital for proving Theorem~\ref{thm:strong_generation}, and useful results regarding their relevance to this problem are stated below.

\begin{remark}\label{rmk:Rouquier_dim_cohomology_annihilator_nonzero}
    If $R$ is an integral domain and $D^b (\operatorname{mod}R)$ has finite Rouquier dimension $r$, then the $(r+1)$-th cohomology annihilator ideal $\operatorname{ca}^{r+1} (R)$ is nonzero. This is \cite[Theorem 5]{Elagin/Lunts:2018}.
\end{remark}

\begin{remark}\label{rmk:regular_element_syzygy_quotient}
    Choose an $R$-regular element $x\in R$. If $M\in \operatorname{mod}R$, then one has an $R$-module isomorphism for all $n\geq 0$:
    \begin{displaymath}
        \Omega_{R/xR}^n \big(\Omega^1_R (M)/ x\Omega^1_R (M)\big) \cong \Omega_R^{n+1} (M)/ x \Omega_R^{n+1} (M).
    \end{displaymath}
    This is \cite[Lemma 5.6]{Dao/Takahashi:2014}.
\end{remark}

\begin{remark}\label{rmk:ses_cohomology_annihilator}
    Suppose that $M$ belongs to $\operatorname{mod}R$. If an element $a$ in $R$ annihilates $\operatorname{Ext}_R^1(M,\Omega^1_R (M))$, then there exists a short exact sequence in $\operatorname{mod}R$:
    \begin{displaymath}
        0 \to (0:_M a) \to M \oplus \Omega^1_R (M) \to \Omega_R (M/a M) \to 0.
    \end{displaymath}
    This is \cite[Remark 2.12]{Iyengar/Takahashi:2016}.
\end{remark}

\begin{example}\label{ex:Dieterich_hypersurface_cohomology_annihilator}
    Let $k$ be a field, and $R=k \llbracket  x_0,\ldots,x_n \rrbracket  $. If $f\in R$, then 
    \begin{displaymath}
        (\frac{\partial f}{\partial x_0}, \ldots, \frac{\partial f}{\partial x_n})\subseteq \operatorname{ca}^{d+1} (R/f R).
    \end{displaymath}
    This is \cite[Proposition 18]{Dieterich/1987}.
\end{example}

\begin{example}\label{ex:BHST_completion_cohomology_annihilator}
    If $(R,\mathfrak{m})$ is a Noetherian local ring of finite Krull dimension and $n\geq 0$, then $\operatorname{ca}^n (\widehat{R})\cap R\subseteq \operatorname{ca}^n (R)$ where $\widehat{R}$ denotes the $\mathfrak{m}$-adic completion of $R$. This is \cite[Theorem 4.5.1]{BHST:2016}. 
\end{example}

%%%%%%%%%%%%%%%%%%%%%%%%%%%%%%%%%%%%%%%
\subsection{Results}
\label{sec:strong_generators_results}
%%%%%%%%%%%%%%%%%%%%%%%%%%%%%%%%%%%%%%%

After discussing cohomology annihilator ideals in Section~\ref{sec:cohomology_annihilator_ideals}, the strategy for developing a proof of Theorem~\ref{thm:strong_generation} begins. The first step towards the existence of strong generators in the module category is by reducing the problem to a question about minimal prime ideals. Let $R$ be Noetherian ring.

\begin{remark}
    There is a finite chain of ideals
    \begin{displaymath}
        (0)=I_0 \subseteq I_1 \subseteq \cdots \subseteq R
    \end{displaymath}
    such that $I_{j+1}/I_j \cong R/\mathfrak{p}$ for some $\mathfrak{p}\in \operatorname{Spec}(R)$. The smallest such value for which one ha such a filtration is denoted by $\theta(R)$. This is \cite[Theorem 6.4]{Matsumura:1989}.
\end{remark}
    
\begin{lemma}\label{lem:radius_subcategory}
    If $I$ is an ideal of $R$ and $n \geq 0$, then 
    \begin{displaymath}
        \operatorname{size}\Omega^n_R (\operatorname{mod}R/I)\leq (n+1) \big(\operatorname{size}\Omega^n_{R/I} (\operatorname{mod}R/I) +1 \big) -1.
    \end{displaymath}
\end{lemma}  

\begin{proof}
    This is \cite[Corollary 5.5]{Dao/Takahashi:2014}.
\end{proof}

\begin{proposition}\label{prop:minimal_primes_strong_generators} 
    If $\operatorname{mod}R/\mathfrak{p}$ has a strong generator for every $\mathfrak{p}\in \operatorname{Min}(R)$, then for some $G\in \operatorname{mod}R$ and $n,s\geq 0$ one has $\Omega^n_R(\operatorname{mod}R)\subseteq |G|_{\theta(R) n(s+1)}$.
\end{proposition} 

\begin{proof}  
    There is a filtration 
    \begin{displaymath}
        0=I_0 \subseteq I_1\subseteq \cdots \subseteq I_m =R
    \end{displaymath}
    by ideals such that $I_i/I_{i-1}\cong R/\mathfrak{q}_i$ where
    $\mathfrak{q}_i\in \operatorname{Spec}(R)$ and $m:= \theta(R)$. For each $1\leq i\leq m$, we can
    choose $\mathfrak{p}_i\in \operatorname{Min}(R)$ such that
    $\mathfrak{p}_i\subseteq \mathfrak{q}_i$. By setting $G_i:=G(\mathfrak{p}_i)$, the hypothesis ensures there exist $n_i,s_i\geq 0$ such that $\Omega^{s_i}_{R/\mathfrak{p}_i}(\operatorname{mod}R/\mathfrak{p}_i)\subseteq |G_i|_{n_i}$. To make these parameters uniform, let 
    \begin{displaymath}
        \begin{aligned}
            &s:=\sup\{ s_i : 1\leq i\leq m\}, \\& n:= \sup\{ n_i :1\leq i \leq m\}.
        \end{aligned}
    \end{displaymath}
    For each $1\leq i \leq m$,
    \begin{displaymath}
        \operatorname{size}\Omega^s_{R/\mathfrak{p}_i}(\operatorname{mod}R/\mathfrak{p}_i) \leq n-1.
    \end{displaymath}
    Hence, Lemma~\ref{lem:radius_subcategory} promises
    \begin{displaymath}
        \operatorname{size} \Omega^s_{R}(\operatorname{mod}R/\mathfrak{p}_i) <
        n(s+1).
    \end{displaymath}
    Furthermore, for each $1\leq i \leq m$ there exists $H_i\in \operatorname{mod} R$ such that 
    \begin{displaymath}
        \Omega^s_{R}(\operatorname{mod}R/\mathfrak{p}_i)\subseteq |H_i|_{n(s+1)}.
    \end{displaymath}
    For any $M\in \operatorname{mod}R$, there is a filtration 
    \begin{displaymath}
        0=I_0 M\subseteq I_1 M\subseteq \cdots \subseteq I_m M=M
    \end{displaymath}
    which gives rise to short exact sequences
    \begin{displaymath}
    0\to I_{i-1}M\to I_i M\to I_iM/I_{i-1}M\to 0,
    \end{displaymath}
    and these yield additionally short exact sequences 
    \begin{equation}\label{eq:syzygy_ses_module}
        0\to \Omega^n_R (I_{i-1}M)\to \Omega^n_R (I_i M) \to \Omega^n_R(I_i M/I_{i-1}M)\to 0.
    \end{equation} 
    From the chain of inclusions $\mathfrak{p}_i I_i M\subseteq \mathfrak{q}_i I_i M\subseteq I_{i-1} M$, it follows that $I_iM/I_{i-1} M\in \operatorname{mod}R/\mathfrak{p}_i$. By iterating the fact that $\Omega^n_R(\operatorname{mod}R/\mathfrak{p}_i)$ is contained in $|H_i|_{n(s+1)}$ for each $1\leq i\leq m$, the desired claim follows. Indeed, $I_1 M\in \operatorname{mod}R/\mathfrak{p}_1$ implies $\Omega^s_{R} (I_1 M)\in |H_1|_{n(s+1)}$. Similarly, $I_2M/I_1M\in \operatorname{mod} R/\mathfrak{p}_2$ implies $\Omega^s_R (I_2M/I_1M) \in |H_2|_{n(s+1)}$. If $i=2$, then Equation~\ref{eq:syzygy_ses_module} ensures $\Omega^n_R (I_2 M)\in |H_1\oplus H_2|_{2n(s+1)}$. If this process continues, then 
    \begin{displaymath}
        \Omega^s_R (M)=\Omega^n_R(I_m M)\in |\bigoplus_{i=1}^m H_i|_{mn(s+1)}.
    \end{displaymath}
    However, $M\in \operatorname{mod}R$ was arbitrary, and so this completes the proof.
\end{proof}

\begin{corollary}\label{cor:strong_generator_parameter_bounds_minimal_primes}
    If for each $\mathfrak{p}\in \operatorname{Min}(R)$ there exist $n(\mathfrak{p}),s(\mathfrak{p})\geq 0$ and $G(\mathfrak{p})\in \operatorname{mod}R/\mathfrak{p}$ such that $\Omega^{s(\mathfrak{p})}_{R/\mathfrak{p}}(\operatorname{mod}R/\mathfrak{p})\subseteq |G(\mathfrak{p})|_{n(\mathfrak{p})}$, then
    \begin{displaymath}
        \Omega^s_R (\operatorname{mod}R) \subseteq |\bigoplus^m_{i=1} \big(G_i \oplus (\bigoplus_{l=0}^{s-1} \Omega_R^l (\mathfrak{p}_i)) \big)|_{\theta(R) n (s+1)}
    \end{displaymath}
    where $n=\underset{\mathfrak{p}\in \operatorname{Min}(R)}{\sup}\{n(\mathfrak{p}) \}$ and $s=\underset{\mathfrak{p}\in \operatorname{Min}(R)}{\sup} \{ s(\mathfrak{p})\}$.
\end{corollary}

\begin{proof}
    This is immediate from the proof of Proposition~\ref{prop:minimal_primes_strong_generators}.
\end{proof}

\begin{lemma}\label{lem:lifting_strong_generation_from_quotient}
    Consider integers $i\geq 1$ and $t\geq 0$. If there is a non-zerodivisor $a\in \operatorname{ca}^{i+1}(R)$ and $G\in \operatorname{mod}R/a R$ such that $\Omega^{i-1}_{R/aR}(\operatorname{mod}R/aR)$ is contained in $|G|_t$,
    then $\Omega^i_R(\operatorname{mod}R)$ is contained in $|\Omega_R (G)|_t$. 
\end{lemma}

\begin{proof} 
    Let $M\in \operatorname{mod} R$, and set $N:=\Omega_R^i (M)$. By Remark~\ref{rmk:regular_element_syzygy_quotient} and the hypothesis, we have
    \begin{displaymath}
        N/aN\cong \Omega^{i-1}_{R/aR}\left(\Omega_R^1 (M)/a\Omega_R^1 (M)\right)\in | G|_t.
    \end{displaymath}
    Restriction of scalars ensures $G$ is an $R$-module, and so it follows that $\Omega_R^1 (N/aN)\in |\Omega_R^1 (G)|_t$. If $i\geq 1$, then the element $a$ is a non-zerodivisor on $N$ as well. Furthermore, as $a\in \operatorname{ca}^{i+1}(R)$, this element annihilates 
    \begin{displaymath} 
        \operatorname{Ext}^{i+1}_R(M,\Omega^{i+1}_R( M))\cong \operatorname{Ext}^1_R(\Omega^i_R (M), \Omega^{i+1}_R (M))\cong \operatorname{Ext}^1_R(N,\Omega_R^1 (N)).
    \end{displaymath}
    Therefore, $N=\Omega^i_R (M)$ is a direct summand of $\Omega_R^1 (N/aN)$ via Remark~\ref{rmk:ses_cohomology_annihilator}, and so $\Omega^i_R( M)\in |\Omega_R^1 (G)|_t$.
\end{proof}

\begin{remark}
    Suppose $R$ is a Noetherian ring satisfying the property $\operatorname{ca}(R/\mathfrak{p})\neq 0$ for every $\mathfrak{p}\in \operatorname{Spec}(R)$. It is not evident that there exists a uniform $s$ such that $\operatorname{ca}^s(R/\mathfrak{p})\neq 0$ for every $\mathfrak{p}\in \operatorname{Spec}(R)$, which is a crucial requirement of \cite[Theorem 5.1]{Iyengar/Takahashi:2016}.
\end{remark}

\begin{theorem}\label{thm:cohomology_annihilator_to_strong_module}
    Suppose $R$ has finite Krull dimension. If the ideal $\operatorname{ca}(R/\mathfrak{p})$ is nonzero for every $\mathfrak{p}\in \operatorname{Spec}(R)$, then $\operatorname{mod} R$ has a strong generator.  
\end{theorem}

\begin{proof}
    This is demonstrated via induction on $\dim R$. If $\dim R=0$, then $\operatorname{mod} R=|R/J(R)|_l$, where $l$ is the Loewy length of the Jacobson radical $J(R)$, see Example~\ref{ex:Iyengar/Takahashi_strong_gen_examples}. It may be assumed $\dim R\geq 1$.  

    First, consider the case where $R$ is an integral domain. By hypothesis, there is a nonzero element $a \in \operatorname{ca}(R)$. Note that every prime ideal of $R/aR$ can be expressed in the form $\mathfrak{p}/aR$ for some $\mathfrak{p}\in \operatorname{Spec}(R)$. Hence, the hypothesis guarantees $\operatorname{ca}((R/aR)/(\mathfrak{p}/aR))=\operatorname{ca}(R/\mathfrak{p})$ is nonzero. Since $\dim (R/aR)<\dim R$, the induction hypothesis tells us $\operatorname{mod}R/a R$ has a strong generator. By Lemma~\ref{lem:lifting_strong_generation_from_quotient}, $\operatorname{mod} R$ has a strong generator, and this finishes the domain case of the inductive step.  

    Consider the non-integral domain case, and fix $\mathfrak{q}\in \operatorname{Spec}(R)$. Since each prime ideal of the quotient ring $R/\mathfrak{q}$ is of the form $\mathfrak{p}/\mathfrak{q}$ for some $\mathfrak{p}\in \operatorname{Spec}(R)$ containing $\mathfrak{q}$, the hypothesis promises $\operatorname{ca}((R/\mathfrak{q})/(\mathfrak{p}/\mathfrak{q}))=\operatorname{ca}(R/\mathfrak{p})\not= 0$. Furthermore, as $\dim(R/\mathfrak{q})\leq \dim R$ and $R/\mathfrak{q}$ is a domain, the integral domain case for the inductive step coupled with the induction hypothesis ensures $\operatorname{mod}R/\mathfrak{q}$ has a strong generator. However, this holds for every $\mathfrak{q}\in \operatorname{Spec}(R)$, so $\operatorname{mod}R/\mathfrak{q}$ has a strong generator for all $\mathfrak{q}\in \operatorname{Spec}(R)$. Hence, $\operatorname{mod} R$ has a strong generator by Proposition~\ref{prop:minimal_primes_strong_generators}. Therefore, this furnishes the inductive step of the proof.
\end{proof}   

\begin{corollary}\label{cor:cohomology_annihilator_to_strong_generator_quotient}
    Suppose $R$ has finite Krull dimension. If $\operatorname{ca}(R/\mathfrak{p})\neq 0$ for all $\mathfrak{p}\in \operatorname{Spec}(R)$, then $\operatorname{mod}R/I$ has a strong generator for every ideal $I$ of $R$.  
\end{corollary}

\begin{proof}
    Let $I$ be an ideal of $R$. Every prime ideal of $R/I$ is of the form $\mathfrak{p}/I$ for some $\mathfrak{p}\in \operatorname{Spec}(R)$ containing $I$, and moreover $(R/I)/(\mathfrak{p}/I)\cong R/\mathfrak{p}$. Hence, $\operatorname{ca}((R/I)/(\mathfrak{p}/I))=\operatorname{ca}(R/\mathfrak{p})$ is non-zero from hypothesis, and so, $\operatorname{mod}R/I$ has a strong generator by Theorem~\ref{thm:cohomology_annihilator_to_strong_module}.  
\end{proof}  

\begin{corollary}\label{cor:finite_Rouquier_dimension_quotient_implies_strong_generator}
    Suppose $R$ has finite Krull dimension. If there exists a strong generator in $D^b ( \operatorname{mod} R/\mathfrak{p})$ for all $\mathfrak{p}\in \operatorname{Spec}(R)$, then $\operatorname{mod}R/I$ has a strong generator for every ideal $I$ of $R$.     
\end{corollary}   

\begin{proof}
    By Remark~\ref{rmk:Rouquier_dim_cohomology_annihilator_nonzero}, $\operatorname{ca}(R/\mathfrak{p})\neq 0$ for every $\mathfrak{p}\in \operatorname{Spec}(R)$, and Corollary ~\ref{cor:cohomology_annihilator_to_strong_generator_quotient} furnishes the claim.  
\end{proof}

\begin{corollary}\label{cor:strong_generation_cohomology_ideal_relationship}
    Suppose $R$ has finite Krull dimension.
    \begin{enumerate}
        \item If $\operatorname{ca}(R/\mathfrak{p})\neq 0$  for all $\mathfrak{p}\in \operatorname{Spec}(R)$, then $\operatorname{mod} R/I$ has a strong generator for every ideal $I$ of $R$.
        \item If $D^b ( \operatorname{mod} R/\mathfrak{p})$ has strong generator for all $\mathfrak{p}\in \operatorname{Spec}(R)$, then $\operatorname{mod} R/I$ has strong generator for every ideal $I$ of $R$.  
        \item If $\operatorname{ca}(R/\mathfrak{p})\neq 0$  for all $\mathfrak{p}\in \operatorname{Spec}(R)$, then $\operatorname{ca}(R/I)$ contains a non-zero-divisor for every radical ideal $I$ of $R$. 
    \end{enumerate}
\end{corollary}

\begin{proof}
    Let $I$ be an ideal of $R$. Every prime ideal of $R/I$ is of the form $\mathfrak{p}/I$ for some $\mathfrak{p}\in \operatorname{Spec}(R)$ containing $I$, and moreover, $(R/I)/(\mathfrak{p}/I)\cong R/\mathfrak{p}$. Now $\operatorname{mod}R/I$ has a strong generator by Corollary~\ref{cor:finite_Rouquier_dimension_quotient_implies_strong_generator}, and so $(1)$ follows. To show $(2)$, it can be deduced by Remark~\ref{rmk:Rouquier_dim_cohomology_annihilator_nonzero} that $\operatorname{ca}(R/\mathfrak{p})\neq 0$ for every $\mathfrak{p}\in \operatorname{Spec}(R)$, and appealing to $(1)$ establishes the desired result. Lastly, if $\operatorname{ca}(R/\mathfrak{p})\neq 0$  for all $\mathfrak{p}\in \operatorname{Spec}(R)$, then $(1)$ implies $\operatorname{mod}R/I$ has a strong generator for every radical ideal $I$ of $R$. Since $R/I$ is a reduced ring, $(3)$ can be deduced by Lemma~\ref{lem:thick_categories_module_property} and \cite[Corollary 6.8]{Iyengar/Takahashi:2016}
\end{proof}

\begin{proof}[Proof of Theorem~\ref{thm:strong_generation}]
    This follows from previous work: $(1)\implies (2)$ by Corollary~\ref{cor:strong_generation_cohomology_ideal_relationship}; $(2)\implies (3)$ by Lemma~\ref{lem:strong_generation_module_to_derived} and Corollary~\ref{cor:strong_generation_cohomology_ideal_relationship}; $(3)\implies (1)$ by Remark~\ref{rmk:Rouquier_dim_cohomology_annihilator_nonzero} and Corollary~\ref{cor:strong_generation_cohomology_ideal_relationship}.
\end{proof}

%%%%%%%%%%%%%%%%%%%%%%%%%%%%%%%%%%%%%%%
\section{Applications}
\label{sec:applications}
%%%%%%%%%%%%%%%%%%%%%%%%%%%%%%%%%%%%%%%

This section studies consequences to Theorem ~\ref{thm:strong_generation}, and the first result answers Question~\ref{q:Iyengar/Takahashi} affirmatively with a larger scope.

\begin{Corollary}\label{cor:quasi_excellent_strong_gen}
    For any Noetherian quasi-excellent ring $R$ of finite Krull dimension, $\operatorname{mod}R$ admits a strong generator. In particular, Question~\ref{q:Iyengar/Takahashi} has an affirmative answer.
\end{Corollary}

An upshot to Corollary~\ref{cor:quasi_excellent_strong_gen} is that it exhibits module categories admit a strong generator for many familiar rings in commutative algebra. Furthermore, if our ring is $F$-finite, then we can explicitly write examples of an object which finitely build any finitely generated $R$-module using kernels of epimorphism, cokernel of a monomorphism, extensions, and direct summands. 

\begin{Corollary}\label{cor:f_finite_strong_gen_module_cat}
    If $R$ is an $F$-finite Noetherian ring, then for each $e \gg 0$ there exist $t,m,c\geq 0$ such that 
    \begin{displaymath}
        \operatorname{mod} R=\operatorname{thick}_{\operatorname{mod}R}^m \bigg(R \oplus \big( \bigoplus^c_{i=0} \Omega_R^i (F_\ast^e R) \big) \bigg).
    \end{displaymath}
\end{Corollary}

In another direction, the following result bounds the Rouquier dimension of the singularity category $D_{\textrm{sg}}(R)$ of a Noetherian ring $R$ in terms of radius of the module category. See Section~\ref{sec:singularity_category_bounds} for background.

\begin{Proposition}\label{prop:singularity_category_bounds_via_radius}
    If $R$ is a Noetherian ring and $\Omega^n_R (\operatorname{mod} R)$ has finite radius for some $n\geq 0$, then one has the following upper bound on the Rouquier dimension of the singularity category:
    \begin{displaymath}
        \dim D_{\textrm{sg}}(R) \leq 
        \underset{n > 0}{\inf}\{ \operatorname{radius} \Omega^n_R (\operatorname{mod} R)\} .
    \end{displaymath}
\end{Proposition}

Proposition~\ref{prop:singularity_category_bounds_via_radius} presents a strengthening to \cite[Lemma 7.1.(1)]{Iyengar/Takahashi:2016} where upper bounds on the Rouquier dimension of $D_{\textrm{sg}}(R)$ are established from the two parameters of a strong generator in $\operatorname{mod}R$. Moreover, it is possible to establish bounds on rank of the category of maximal Cohen-Macaulay modules, which is denoted $\operatorname{CM}(R)$. See Section~\ref{sec:MCM_category} for further details.

\begin{Proposition}\label{prop:CM_category_bounds}
    Let $R$ be a Cohen-Macaulay local ring of dimension $d$ admitting a canonical module.  If $R/\mathfrak{p}$ is Cohen-Macaulay for every $\mathfrak{p} \in \operatorname{Min}(R)$, then \begin{displaymath}
        \operatorname{rank}\operatorname{CM}(R)\leq 2(d+1)^2 \left(\sum_{\mathfrak{p}\in \operatorname{Min}(R)} \ell\ell(R_{\mathfrak{p}})\right)(s+1)-1
    \end{displaymath} where $s:=\underset{\mathfrak{p} \in \operatorname{Min}(R)}{\sup} \{\operatorname{size}\operatorname{CM}(R/\mathfrak{p}) \}$.  
    
    %If $R/\mathfrak{p}$ is Cohen-Macaulay and $\operatorname{size} \operatorname{CM}(R/\mathfrak{p})<\infty$ for every $\mathfrak{p} \in \operatorname{Min}$, then $\operatorname{rank} \operatorname{CM}(R)<\infty$. 
\end{Proposition}

%%%%%%%%%%%%%%%%%%%%%%%%%%%%%%%%%%%%%%%
\subsection{Examples}
\label{sec:strong_generation_examples}
%%%%%%%%%%%%%%%%%%%%%%%%%%%%%%%%%%%%%%%

This section answers positively to Question~\ref{q:Iyengar/Takahashi}, and explicit computations are made for the two parameters of a strong generator in the module category. 

\begin{proof}[Proof of Corollary~\ref{cor:quasi_excellent_strong_gen}]
    If $R$ is quasi-excellent, then any essentially $R$-algebra of finite type is as well. By \cite[Main Theorem]{Aoki:2021}, it was shown that any Noetherian quasi-excellent ring of finite Krull dimension has finite Rouquier dimension, and so appealing to Theorem~\ref{thm:strong_generation} furnishes the proof of the first claim. Lastly, any excellent ring is quasi-excellent, so the second claim follows.
\end{proof}

\begin{corollary}
    A Noetherian $1$-dimensional $J\textrm{-}2$ ring $R$ has finite Rouquier dimension if, and only if, $\operatorname{mod}R$ admits a strong generator.
\end{corollary}

\begin{proof}
    If $\operatorname{mod}R$ admits a strong generator, then Lemma~\ref{lem:strong_generation_module_to_derived} ensures $R$ has finite Rouquier dimension, and so the other direction needs to be checked. First, suppose that $R$ is an integral domain. If $R$ has finite Rouquier dimension $r$, then Remark~\ref{rmk:Rouquier_dim_cohomology_annihilator_nonzero} ensures $\operatorname{ca}^{r+1}(R)$. Choosing $a\in \operatorname{ca}^{r+1}(R)$, Lemma~\ref{lem:lifting_strong_generation_from_quotient} tells us there exists a strong generator in $\operatorname{mod}R$ as $R/aR$ is Artinian, and such rings admit strong generators in their module categories. To complete the proof, the non-integral domain case follows by appealing to Proposition~\ref{prop:minimal_primes_strong_generators}.
\end{proof}

\begin{example}
    Let $R$ be a $1$-dimensional integral domain which is essentially of finite type over a field $k$. If $R$ is not regular, then $\operatorname{ca}^3 (R) \not =0$ via Example~\ref{ex:Iyengar/Takahashi_strong_gen_examples}. For $a \in \operatorname{ca}^3 (R)$ nonzero nonunit, Lemma~\ref{lem:lifting_strong_generation_from_quotient} promises
    \begin{displaymath}
        \Omega^4_R (\operatorname{mod}R) \subseteq |\Omega^1_R (\overline{R}/J(\overline{R}))|_{\ell\ell(\overline{R})}
    \end{displaymath}
    where $\overline{R}=R/(a)$.
\end{example}

\begin{example}\label{ex:1_dim_int_dom_strong_gen}
    Suppose that $(R,\mathfrak{m},k)$ is a Noetherian quasi-excellent  local integral domain of Krull dimension one. Choose $r\geq 0$ such that there exists a nonzero nonunit element $a\in \operatorname{ca}^r (R)\not=0$. Lemma~\ref{lem:lifting_strong_generation_from_quotient} ensures $\Omega^{r+1}_{R} (\operatorname{mod}R) \subseteq |\mathfrak{m}|_{\ell\ell(R/aR)}$. Additionally, if $R$ is equicharacteristic, then $r=3$ can be taken via Example~\ref{ex:Iyengar/Takahashi_strong_gen_examples}.
\end{example}

\begin{example}
    The following is a non-integral domain case. Consider the ring $R=\mathbb{F}_5 [x,y]_{(x,y)}/(x^{10} + y^{10})$. This is a $1$-dimensional ring which is not an integral domain. Its minimal primes are the ideals $(x\pm 2y)$, and $R/(x \pm 2y)$ are regular $1$-dimensional Noetherian local rings. In particular, $\Omega^2_{R/(x \pm 2y)} (\operatorname{mod}R/(x \pm 2y))\subseteq |R/(x \pm 2y)|_1$, and so Lemma~\ref{lem:lifting_strong_generation_from_quotient} promises
    \begin{displaymath}
        \Omega^3_R (\operatorname{mod}R) \subseteq |R/(x+2y) \oplus R/(x -2y) \oplus (x+2y) \oplus (x-2y)|_{12}.
    \end{displaymath}
\end{example}

\begin{remark}\label{rmk:thick_strong_generation}
    Suppose $R$ is a Noetherian ring of finite Krull dimension satisfying any of the equivalent conditions in Theorem~\ref{thm:strong_generation}. By keeping the notation of Corollary~\ref{cor:strong_generator_parameter_bounds_minimal_primes} and appealing to Lemma~\ref{lem:strong_generation_module_to_derived}, we see that
    \begin{displaymath}
        \operatorname{mod}R = \operatorname{thick}^{1 + s+\theta(R) n (s+1)}_{\operatorname{mod}R} \big( R\oplus \bigoplus_{\mathfrak{p}\in \operatorname{Min}(R)} (G(\mathfrak{p}) \oplus \mathfrak{p}) \big).
    \end{displaymath}
    This follows from the fact that $\Omega_R^i (R/\mathfrak{p})\in \operatorname{thick}_{\operatorname{mod}R}^{i+1} (R\oplus \mathfrak{p})$. Indeed, the short exact sequence
    \begin{displaymath}
        0 \to \mathfrak{p} \to R \to R/\mathfrak{p} \to 0
    \end{displaymath}
    implies $R/\mathfrak{p} \in \operatorname{thick}^2_{\operatorname{mod}R} (R \oplus \mathfrak{p})$, and an inductive argument will show for each $i\geq 2$ that $\Omega^i_R (R/\mathfrak{p})\in  \operatorname{thick}^i_{\operatorname{mod}R} (R \oplus \mathfrak{p})$.
\end{remark}

\begin{corollary}\label{cor:min_primes_regular_module_category_bound}
    If $R$ is a Noetherian ring of finite Krull dimension $d$ such that $R/\mathfrak{p}$ is regular for each $\mathfrak{p}\in\operatorname{Min}(R)$, then 
    \begin{displaymath}
        \operatorname{thick}_{\operatorname{mod}R}^{1+ d + \theta(R)(d + 2)} \big( R \oplus (\bigoplus_{\mathfrak{p}\in \operatorname{Min}(R)} R/\mathfrak{p})\big) = \operatorname{mod}R.
    \end{displaymath}
\end{corollary}

\begin{proof}
    For each $\mathfrak{p}\in\operatorname{Min}(R)$, $R/\mathfrak{p}$ being regular ensures $\Omega^{\dim R/\mathfrak{p} +1}_{R/\mathfrak{p}}(\operatorname{mod}R/\mathfrak{p}) \subseteq |R/\mathfrak{p}|_0$. By Remark~\ref{rmk:thick_strong_generation}, the claim follows immediately.
\end{proof}

\begin{example}
    Let $k$ be a field and $n\geq 0$. Suppose $R$ is either $k[ x_1,\ldots,x_n ]$ or $k\llbracket x_1,\ldots, x_n \rrbracket$. If $I$ is a monomial ideal in $R$, then each minimal prime is an ideal of the form $(x_{i_1},\ldots,x_{i_c})$, and their associated quotient rings are regular. Corollary~\ref{cor:min_primes_regular_module_category_bound} tells us that
    \begin{displaymath}
        \operatorname{thick}_{\operatorname{mod}R/I}^{1+ d + \theta(R/I)(d + 2)} \big( R/I \oplus (\bigoplus_{\mathfrak{p}\in \operatorname{Min}(R/I)} R/\mathfrak{p}) \big) = \operatorname{mod}R/I
    \end{displaymath}
    where $d:= \dim R/I$. This should be compared to \cite[Corollary 3.11]{Aihara/Takahashi:2015}.
\end{example}

\begin{lemma}\label{lem:isolated_singularity_lemma}
     Let $R$ be a Noetherian ring, and choose $n \geq 0$. 
    \begin{enumerate}[\rm(1)]
        \item Suppose there is an exact sequence 
        \begin{displaymath}
            0\to M \to C_0\to \cdots\to C_{n-1}\to N\to 0
        \end{displaymath}
        in $\operatorname{mod} R$. Then 
        \begin{displaymath}
            \Omega^n_R (N)\in |M\oplus (\bigoplus_{i=0}^{n-1}\Omega_R^{i+1}(C_i))|_{n+1}.
        \end{displaymath}
    
        \item If $\mathbf{x}:=x_1,\ldots,x_n$ is an $M$-regular sequence in $R$, then 
        \begin{displaymath}
            \Omega^n_R(M/\mathbf x M)\in |\bigoplus_{i=0}^n \Omega^i_R (M)|_{n+1}.
        \end{displaymath}
    
        \item If $(R,\mathfrak m,k)$ is an $F$-finite local ring of depth $t$, then for all $e\gg 0$:
        \begin{displaymath}
            \Omega^t_R (k) \in |R \oplus (\bigoplus_{i=0}^t \Omega^i_R (F^e_* R))|_{t+1}.
        \end{displaymath}
    \end{enumerate}
\end{lemma}

\begin{proof}
    (1) Let us induct on $n$, and note the case where $n=0$ is obvious. If $n=1$, then the short exact sequence
    \begin{displaymath}
        0 \to M \to C_0 \to N \to 0
    \end{displaymath}
    yields another short exact sequence
    \begin{displaymath}
        0\to \Omega_R (C_0) \to \Omega_R (N) \to M\to 0    
    \end{displaymath}
    by repeatedly applying \cite[Proposition 2.2(1)]{Dao/Takahashi:2015b}. Hence, the object $\Omega_R (N)$ belongs to $|M\oplus \Omega_R (C_0)|_2$. 
    
    Assume $n\geq 2$. There are exact sequences 
    \begin{displaymath}
        \begin{aligned}
            & 0\to M \to C_0\to \cdots \to C_{n-2}\to L\to 0,
            \\& 0\to L \to C_{n-1}\to N\to 0.
        \end{aligned}
    \end{displaymath}
    The first exact sequence, in view of induction hypothesis, tells us that $\Omega_R^{n-1} (L)$ is in $|M\oplus(\bigoplus_{i=0}^{n-2} \Omega_R^{i+1}(C_i))|_n$. The short exact sequence 
    \begin{displaymath}
        0\to L \to C_{n-1}\to N\to 0
    \end{displaymath}
    yields another short exact sequece,
    \begin{displaymath}
        0\to \Omega_R (C_{n-1})\to \Omega_R (N) \to L \to 0    
    \end{displaymath}
    by repeatedly applying \cite[Proposition 2.2(1)]{Dao/Takahashi:2015b}. Furthermore, another application of \cite[Proposition 2.2(1)]{Dao/Takahashi:2015b} tells us  
    \begin{displaymath}
        0\to \Omega^n_R (C_{n-1}) \to \Omega^n_R (N) \to \Omega^{n-1}_R  (L) \to 0.    
    \end{displaymath}
    Thus, $\Omega_R^n (N)$ is in $|M\oplus \Omega_R^n (C_{n-1})\oplus(\oplus_{i=0}^{n-2} \Omega_R^{i+1}(C_i))|_{n+1}$, finishing the inductive step. 

    (2) As $\mathbf x :=x_1,\ldots,x_n$ is $M$-regular, so we have an exact sequence arising from the Koszul complex
    \begin{displaymath}
        0\to M \to L_0\to \cdots \to L_{n-1}\to M/\mathbf x M \to 0
    \end{displaymath}
    where each $L_i$ is a finite direct sum of copies of $M$. Hence, $\Omega^n_R(M/\mathbf x M)$ belongs to $|\bigoplus_{i=0}^n \Omega^i_R (M)|_{n+1}$ by  (1).  

    (3) By \cite[Corollary 3.3]{Takahashi/Yoshino:2004}, for every $e\gg 0$, there is an $F^e_\ast R$-regular sequence $\mathbf x :=x_1,\ldots,x_t$ such that $k$ is a direct summand of $F^e_\ast R/(\mathbf x)$. Hence, $\Omega^t_R (k) \in |R \oplus (\bigoplus_{i=0}^t \Omega^i_R (F^e_\ast R))|_{t+1}$ by (2) and remembering that syzygies are defined up to free summands. 
\end{proof} 

\begin{proposition}\label{prop:f_finite_isolated singularity_module_category}
    If $(R, \mathfrak{m}, k)$ be an $F$-finite local ring with isolated singularity of Krull dimension $d$, then there exist $s,r\geq 0$ such that $\Omega_R^s(\operatorname{mod} R)\subseteq |R \oplus (\bigoplus_{i=0}^{d} \Omega_R^i (F^e_\ast R))|_r$ for all $e\gg 0$. Moreover, if $R$ is Gorenstein, then there exists $r\geq 0$ such that $\operatorname{CM}(R)=|R \oplus (\bigoplus_{i=0}^{d} \Omega_R^i (F^e_\ast R))|_r$ for all $e\gg 0$.  
\end{proposition}   

\begin{proof}
    Let $t=\operatorname{depth } R$. As $R$ is $F$-finite, it is excellent, and hence $\operatorname{mod} R$ has a strong generator by Corollary~\ref{cor:quasi_excellent_strong_gen}. That is, $\Omega_R^n(\operatorname{mod} R)$ has finite size, and so finite radius for some $n$. By \cite[Theorem 3.2]{BHST:2016}, we get $\Omega_R^n(\operatorname{mod} R)\subseteq|\bigoplus_{i=0}^d\Omega^i_R (k)|_b$ for some $b\geq 1$. So, $\Omega^{n+t}_R(\operatorname{mod} R)\subseteq|\bigoplus_{i=0}^d\Omega^{i+t}_R (k)|_b$. Now, $\Omega_R^{i+t}(k)$ is locally free on punctured spectrum, and has depth at least $t$ for each $i\geq 0$. By \cite[Theorem 4.1]{BHST:2016}, $G:=\bigoplus_{i=0}^d\Omega^{i+t}_R (k) \in |\bigoplus_{i=t}^d \Omega_R^i(k))|$. Hence, \cite[Remark 2.11]{BHST:2016} ensures we have $G\in |\bigoplus_{i=t}^d \Omega_R^i(k)|_v$ for some $v$. Thus, $\Omega^{n+t}_R(\operatorname{mod} R)\subseteq|G|_b\subseteq |\bigoplus_{i=t}^d \Omega_R^i(k)|_{bv}$. Since for all $e\gg 0$ we have $\Omega^t_R (k) \in |R \oplus (\bigoplus_{i=0}^t \Omega^i_R (F^e_\ast R))|_{t+1}$ by Lemma \ref{lem:isolated_singularity_lemma}, so $\Omega_R^j(k)\in |R \oplus (\bigoplus_{i=j-t}^j \Omega^i_R (F^e_\ast R))|_{t+1}$ for all $j\geq t$. Thus, for some $r$, we have
    \begin{displaymath}
        \Omega^{n+t}_R(\operatorname{mod} R)\subseteq |\bigoplus_{i=t}^d \Omega_R^i(k)|_{bv}\subseteq |R\oplus(\bigoplus_{i=0}^d \Omega^i_R(F^e_\ast R))|_{r}.
    \end{displaymath}
    
    %%%%%%Since for all $e\gg 0$ we have $\Omega^t_R (k) \in |R \oplus (\oplus_{i=0}^t \Omega^i_R (F^e_\ast R))|_{t+1}$ by Lemma \ref{lem:isolated_singularity_lemma}, this concludes the proof of the inclusion in the statement. 

    Lastly, assume $R$ is Gorenstein. Then remembering that $\operatorname{CM}(R)=\Omega^n_R(\operatorname{mod} R)$ for all $n\gg 0$, we get $\operatorname{CM}(R)\subseteq |R \oplus (\bigoplus_{i=0}^{d} \Omega_R^i (F^e_\ast R))|_r$ for all $e\gg 0$, and the other inclusion is obvious because $F^e_\ast R \in \operatorname{CM}(R)$ for all $e$.   
\end{proof} 

%%%%%%%%%%%%%%%%%%%%%%%%%%%%%%%%%%%%%%%
\subsection{Large thickenings in $\operatorname{mod}R$}
\label{sec:large_thickenings}
%%%%%%%%%%%%%%%%%%%%%%%%%%%%%%%%%%%%%%%

This section introduces another filtration for thick subcategories in module categories. Let $R$ be a Noetherian ring, and choose $G\in \operatorname{mod}R$. Recall there is an ascending chain of additive subcategories:
\begin{displaymath}
    \operatorname{thick}_{\operatorname{mod}R}^0 (G) \subseteq \operatorname{thick}_{\operatorname{mod}R}^1 (G) \subseteq \operatorname{thick}_{\operatorname{mod}R}^2(G) \subseteq \cdots \subseteq \operatorname{thick}_{\operatorname{mod}R}(G).
\end{displaymath}
It is not clear whether or not the following equality holds:
\begin{displaymath}
    \bigcup^\infty_{i=0} \operatorname{thick}_{\operatorname{mod}R}^i (G) = \operatorname{thick}_{\operatorname{mod}R} (G).
\end{displaymath}
However, under a mild hypothesis, there is another filtration whose union coincides with $\operatorname{thick}_{\operatorname{mod}R} (G)$ under a mild hypothesis. This occurs when strong generators in $\operatorname{mod}R$ exist.

\begin{definition}
    Let $G\in \operatorname{mod}R$. For $n=1$, let $\operatorname{th}_{\operatorname{mod}R}^1 (G):=\operatorname{add} (G)$. For $n>1$, define $\operatorname{th}_{\operatorname{mod}R}^n (G)$ to be the full subcategory of $\operatorname{mod} R$ consisting of modules $M$ such that there exists an exact sequence
    \begin{displaymath}
        0 \to A \to B \to C \to 0
    \end{displaymath}
    such that $D_1$ and $D_2$ belong to $\operatorname{th}_{\operatorname{mod}R}^{n-1} (G)$ and $M$ is a direct summand of $D_3$ where $\{D_1,D_2,D_3\}=\{A,B,C\}$.
\end{definition}

\begin{lemma}\label{lem:large_thickenings_filtration}
    If $G\in \operatorname{mod}R$, then 
    \begin{displaymath}
        \bigcup_{n=0}^\infty \operatorname{th}_{\operatorname{mod}R}^n (G) = \operatorname{thick}_{\operatorname{mod}R} (G).
    \end{displaymath}
\end{lemma}

\begin{proof}
    First, one shows that $\operatorname{th}_{\operatorname{mod}R}^n (G) \subseteq \operatorname{thick}_{\operatorname{mod}R} (G)$ for each $n\geq 0$. This may be done by induction on $n$. To show the reverse inclusion, it suffices to check that $\bigcup_{n=0}^\infty \operatorname{th}_{\operatorname{mod}R}^n (G)$ is thick subcategory in $\operatorname{mod}R$.
\end{proof}

\begin{lemma}\label{lem:large_thickening_to_extension_decompositions}
    Let $G,M\in \operatorname{mod}R$, and choose $n\geq 0$. If $M$ belongs to $\operatorname{th}_{\operatorname{mod}R}^{n+1}(G)$, then $\Omega_R^n (M)$ belongs to $|C|_m$ where $C=R\oplus\bigoplus_{i=0}^{2n}\Omega_R^i (G)$ and $m=2^n$.
\end{lemma}

\begin{proof}
    We use induction on $n$. The assertion is clear if $n=0$. Let $n>0$.
    There is an exact sequence
    \begin{displaymath}
        0 \to A \to B \to C \to 0
    \end{displaymath}
    such that $D_1$ and $D_2$ belong to $\operatorname{th}_{\operatorname{mod}R}^n (G)$ and $M$ is a direct summand of $D_3$
    where $\{D_1,D_2,D_3\}=\{A,B,C\}$. The induction hypothesis implies that
    $\Omega_R^{n-1}(D_1)$ and $\Omega_R^{n-1}(D_2)$ are in $|E|_m$, where
    $E=R\oplus\bigoplus_{i=0}^{2n-2}\Omega_R^i (G)$ and $m=2^{n-1}$.

    (1) Suppose $(D_1,D_2,D_3)=(C,A,B)$. Then there is an exact sequence
    \begin{displaymath}
        0 \to \Omega_R^{n-1}(D_2) \to \Omega_R^{n-1}(D_3) \to \Omega_R^{n-1}(D_1) \to 0
    \end{displaymath}
    up to projective summands. Hence, $\Omega_R^{n-1}(M)$ belongs to $|E|_{2m}$.
    Therefore, $\Omega_R^n (M)$ belongs to $|\Omega_R (E)|_{2m}$.

    (2) Suppose $(D_1,D_2,D_3)=(B,C,A)$. Then there is an exact sequence
    \begin{displaymath}
        0 \to \Omega_R^{n-1}(D_3) \to \Omega_R^{n-1}(D_1) \to \Omega_R^{n-1}(D_2) \to 0
    \end{displaymath}
    up to projective summands, which induces an exact sequence
    \begin{displaymath}
        0 \to \Omega_R^n (D_2) \to \Omega_R^{n-1}(D_3) \to \Omega_R^{n-1}(D_1) \to 0
    \end{displaymath}
    up to projective summands. As $\Omega_R^{n-1} (D_2)$ is in $|E|_m$, we have
    $\Omega^n_R (D_2)$ is in $|\Omega_R^1 (E)|_m$. Hence, $\Omega_R^{n-1}(M)$ belongs to $|E\oplus\Omega_R^1 (E)|_{2m}$. Therefore, $\Omega^n_R (M)$ belongs to $|\Omega_R^1 (E)\oplus\Omega_R^2(E)|_{2m}$.

    (3) Suppose $(D_1,D_2,D_3)=(A,B,C)$. Then there is an exact sequence
    \begin{displaymath}
        0 \to \Omega_R^{n-1} (D_1) \to \Omega_R^{n-1} (D_2) \to \Omega_R^{n-1} (D_3) \to 0
    \end{displaymath}
    up to projective summands, which induces an exact sequence
    \begin{displaymath}
        0 \to \Omega_R^n (D_3) \to \Omega_R^{n-1} (D_1) \to \Omega_R^{n-1} (D_2) \to 0
    \end{displaymath}
    up to projective summands. However, this further induces an exact sequence
    \begin{displaymath}
        0 \to \Omega_R^n (D_2) \to \Omega_R^n (D_3) \to \Omega_R^{n-1} (D_1) \to 0
    \end{displaymath}
    up to projective summands. As $\Omega_R^{n-1} (D_2)$ is in $|E|_m$, we have
    $\Omega_R^n (D_2)$ is in $|\Omega_R^1 (E)|_m$. Hence, $\Omega_R^n (M)$ belongs to $|E\oplus\Omega_R (E)|_{2m}$.

    Consequently, $\Omega_R^n (M)$ belongs to $|E\oplus\Omega_R^1
    (E)\oplus\Omega_R^2(E)|_{2m}=|F|_{2m}$, where $F=R\oplus\bigoplus_{i=0}^{2n}\Omega_R^i (G)$ and $2m=2^n$.
\end{proof}

\begin{proposition}\label{prop:large_thickening_to_syzygy}
    Let $G,M\in \operatorname{mod}R$.
    \begin{enumerate}
        \item For $n=1,2$ one has $\operatorname{thick}_{\operatorname{mod}R}^n (G)=\operatorname{th}_{\operatorname{mod}R}^n (G)$, and $\operatorname{thick}_{\operatorname{mod}R}^n (G) \subseteq \operatorname{th}_{\operatorname{mod}R}^n (G)$ when $n>2$.
        \item Let $n\geq 0$ be an integer. Put $m=2^n$ and $C=R\oplus(\bigoplus_{i=0}^{2n}\Omega_R^i (G))$. Consider the following three conditions.
        \begin{enumerate}
            \item The module $M$ belongs to $\operatorname{th}_{\operatorname{mod}R}^{n+1} (G)$.
            \item The module $\Omega_R^n (M)$ belongs to $|C|_m$.
            \item The module $M$ belongs to $\operatorname{thick}^{m+n}_{\operatorname{mod}R}(R\oplus C)$.
        \end{enumerate}
        Then the implications $(a) \implies (b) \implies (c)$ hold.
    \end{enumerate}
\end{proposition}

\begin{proof}
    (1) The assertion is immediate from the definitions of $\operatorname{thick}_{\operatorname{mod}R}^n (G)$ and $\operatorname{th}_{\operatorname{mod}R}^n (G)$.

    (2) Lemma~\ref{lem:large_thickening_to_extension_decompositions} ensures $(a)$ implies $(b)$. Let us show that $(b)$ implies $(c)$. Assume $\Omega_R^n (M)$ belongs to $|C|_m$. There is an exact sequence 
    \begin{displaymath}
    0 \to \Omega_R^n (M) \to P \to \Omega_R^{n-1} (M) \to 0.
    \end{displaymath}
    where $P$ is a projective $R$-module. We have that $\Omega_R^n (M)$ is in $\operatorname{thick}_{\operatorname{mod}R}^m(R\oplus C)$ and $R^{\oplus r}$ is in $\operatorname{add}(R\oplus C)$. Hence, $\Omega_R^{n-1} (M)$ is in $\operatorname{thick}_{\operatorname{mod}R}^{m+1}(R\oplus C)$. It may be verfied that $M\in \operatorname{thick}_{\operatorname{mod}R}^{m+n}(R\oplus C)$ by induction on $n$.
\end{proof}

\begin{corollary}
    The following are equivalent:
    \begin{enumerate}
        \item $\operatorname{mod} R=\operatorname{th}_{\operatorname{mod}R}^n(G)$ for some $G\in \operatorname{mod}R$ and $n\geq 0$;
        \item $\operatorname{mod} R=\operatorname{thick}_{\operatorname{mod}R}^m (C)$ for some $C\in \operatorname{mod}R$ and $m\geq 0$.
    \end{enumerate}
\end{corollary}

\begin{proof}
    $(1)\implies (2)$ by Proposition~\ref{prop:large_thickening_to_syzygy}, and $(2)\implies (1)$ comes by construction.
\end{proof}

\begin{proof}[Proof of Corollary~\ref{cor:f_finite_strong_gen_module_cat}]
    By Example~\ref{ex:f_finite_ring}, we know that $R\oplus F_\ast^e R$ is a classical generator for $\operatorname{mod}R$. However, Corollary~\ref{cor:quasi_excellent_strong_gen} promises that there exist $G\in \operatorname{mod}R$ and $n,s \geq 0$ such that $\Omega^s_R (\operatorname{mod}R) \subseteq |G|_n$. Lemma~\ref{lem:thick_categories_module_property} tells us that $\operatorname{mod} R=\operatorname{thick}_{\operatorname{mod}R}^{n+s} (G)$. Furthermore, Lemma~\ref{lem:large_thickenings_filtration} implies there exists $l\geq 0$ such that $G\in\operatorname{th}_{\operatorname{mod}R}^l (R\oplus F_\ast^e R)$. Therefore, the desired claim follows by appealing to Proposition~\ref{prop:large_thickening_to_syzygy}.
\end{proof}

%%%%%%%%%%%%%%%%%%%%%%%%%%%%%%%%%%%%%%%
\subsection{Singularity categories}
\label{sec:singularity_category_bounds}
%%%%%%%%%%%%%%%%%%%%%%%%%%%%%%%%%%%%%%%

This section establishes an upper bound on the singularity category of a Noetherian ring using radius in the module category. Let $R$ be a Noetherian ring. The reader is encouraged to look at Section~\ref{sec:generation_via_syzygys} for background regarding radius in $\operatorname{mod}R$.

\begin{chunk}
    Recall an object $P\in D^b (\operatorname{mod}R)$ is said to be \textbf{perfect} when $P$ is finitely built by $R$. The full subcategory of perfect complexes in $D^b (\operatorname{mod}R)$ is denoted $\operatorname{perf}R$. Introduced in \cite{Buchweitz:2021}, the \textbf{singularity category} of $R$ is defined to be the Verdier quotient 
    \begin{displaymath}
        D_{\textrm{sg}}(R) := D^b (\operatorname{mod}R) / \operatorname{perf}R.
    \end{displaymath}
    This category has a triangulated structure induced from $D^b (\operatorname{mod}R)$, and it captures homological properties of $R$. For instance, $R$ is regular if, and only if, $D_{\textrm{sg}}(R)$ is trivial. There is a chain of exact functors
    \begin{displaymath}
        \operatorname{mod}R \to D^b(\operatorname{mod}R) \to D_{\textrm{sg}}(R)
    \end{displaymath}
    where $E\in \operatorname{mod}R$ is assigned the image of $E[0]$ in the Verdier quotient $D_{\textrm{sg}}(R)$.
\end{chunk}

\begin{chunk}\label{c:singularity_triangle}
    Suppose $E\in D^b (\operatorname{mod}R)$, and let $s\geq \sup\{i : H^i (E)\not=0\}$. There exists a distinguished triangle
    \begin{equation}\label{eq:singularity_triangle}
        P \to E \to M[s] \to P[1]
    \end{equation}
    where $P$ is a perfect complex and $M\in \operatorname{mod}R$. This may be constructed by replacing $E$ with a suitable projective resolution such that each $E^i$ is a finitely generated projective $R$-module and $E^j=0$ whenever $j < \inf\{i : H^i (E)\not=0 \}$. The truncated subcomplex $E_{<s}$ of $E$ is perfect, and the quotient complex $E/E_{<s}$ has homology only in one degree, and so $E/E_{<s} \cong H^s (E/E_{<s})[s]$. This gives Equation~\ref{eq:singularity_triangle}. In particular, for each $E\in \operatorname{mod}R$ and $n\in\mathbb{N}$, it follows that $E[-n]$ is isomorphic to $\Omega_R^n (E)$ in $D_{\textrm{sg}}(R)$. For further details, see \cite[Lemma 2.4]{Dao/Takahashi:2015} or \cite[Remark 2.2]{Iyengar/Takahashi:2019}.
\end{chunk}

\begin{proof}[Proof of Proposition~\ref{prop:singularity_category_bounds_via_radius}]
    Choose an integer $n$ such that $\Omega^n_R (\operatorname{mod} R)$ has finite radius. This ensures $\Omega^n_R (\operatorname{mod} R)$ is contained in $[G]_s$ for some $s\geq 0$ and $G\in \operatorname{mod}R$. It follows that $[G]_s$ is contained in $\operatorname{thick}_{D_{\textrm{sg}} (R)}^s (G)$. Hence, $\Omega^n_R (\operatorname{mod} R)$ is contained in $\operatorname{thick}_{D_{\textrm{sg}} (R)}^s (G)$, and as arbitrary shifts are allowed in $\operatorname{thick}_{D_{\textrm{sg}} (R)}^s (G)$, we have $\operatorname{mod}R$ is contained in $\operatorname{thick}_{D_{\textrm{sg}} (R)}^s (G)$. The desired upper bound on Rouquier dimension follows immediately from \ref{c:singularity_triangle}.
\end{proof}

%%%%%%%%%%%%%%%%%%%%%%%%%%%%%%%%%%%%%%%
\subsection{Bounds on size}
\label{sec:bounds_on_size}
%%%%%%%%%%%%%%%%%%%%%%%%%%%%%%%%%%%%%%%

In what follows is a version of Proposition~\ref{prop:minimal_primes_strong_generators} for a class of rings where the invariant $\theta(R)$ can be replaced with a tangible value. Recall a module $M$ is said to satisfy \textbf{Serre's condition $(S_n)$}, where $n\geq 0$, if for all $\mathfrak{p} \in \operatorname{Spec}(R)$:
\begin{displaymath}
    \operatorname{depth}_{R_{\mathfrak{p}}} M_{\mathfrak{p}} \geq \inf\{n,\dim R_{\mathfrak{p}}\}.
\end{displaymath}
It is clear that $M$ satisfies $(S_1)$ if, and only if, $\operatorname{Ass}_R(M)\subseteq \operatorname{Min}(R)$. For a Noetherian local ring $(R,\mathfrak{m})$ and $M\in \operatorname{mod} R$, the \textbf{Loewy length of $M$} is given by:
\begin{displaymath}
    \ell\ell(M):=\inf\{n\geq 0: \mathfrak{m}^n M=0\}.    
\end{displaymath}
Observe that $\ell\ell(M)<\infty$ if, and only if, $M$ has finite length $\ell(M)$. Moreover, it is always the case that $\ell\ell(M)\leq\ell(M)$. 

\begin{lemma}\label{lem:s_1}
    Suppose $R$ is a Noetherian ring and $M\in \operatorname{mod}R$.
    \begin{enumerate}[\rm(1)]
        \item If $\mathfrak{p}\in \operatorname{Spec}(R)$ and $N$ is the kernel of the localization map $M\to M_{\mathfrak{p}}$, then $\operatorname{Ass}_R(N)\subseteq \operatorname{Ass}_R(M)\setminus\{\mathfrak{p}\}$. Moreover, if $\mathfrak{p}\in \operatorname{Min}(R)$, then  $\operatorname{Ass}_R(M/N)\subseteq \{\mathfrak{p} \}$.
        \item If $M$ satisfies $(S_1)$, then there exists a filtration 
        \begin{displaymath}
            0=M_0\subseteq M_1\subseteq \cdots \subseteq M_l=M,
        \end{displaymath}
        where $l\leq \sum_{\mathfrak{p}\in \operatorname{Ass}_R(M)} \ell\ell(R_{\mathfrak{p}})$, and for every $i$, there exists $\mathfrak{p}_i\in \operatorname{Min}(R)$ such that $M_i/M_{i-1}\in \operatorname{mod}R/\mathfrak{p}_i$.  
        \item If $R$ is local and $M$ satisfies $(S_1)$, then a filtration as in (2) can be chosen such that $M_i/M_{i-1}$ has positive depth for each $i$.
    \end{enumerate}  
\end{lemma}

\begin{proof}
    (1) The kernel can be expressed as follows: 
    \begin{displaymath}
        N=\{x\in M: \dfrac x 1=0 \in M_{\mathfrak{p}}\}=\{x\in M: \exists s\in R\setminus \mathfrak{p} \textrm{ s.t. } sx=0\}
    \end{displaymath}
    If $N\subseteq M$, then $\operatorname{Ass}_R(N)\subseteq \operatorname{Ass}_R(M)$. Since $\mathfrak{p} \in \operatorname{Ass}_R(N)$, there exists $x\in N$ such that $\mathfrak{p}=\operatorname{ann}_R(x)$. But $x\in N$ also implies $s\in \operatorname{ann}_R(x)$ for some $s\in R\setminus \mathfrak{p}$, which is a contradiction. Hence, $\mathfrak{p} \notin \operatorname{Ass}_R(N)$ and this establishes $\operatorname{Ass}_R(N)\subseteq \operatorname{Ass}_R(M)\setminus\{\mathfrak{p}\}$. To check the second claim, suppose $\mathfrak{p}\in \operatorname{Min}(R)$. Since $M/N$ embeds into $M_{\mathfrak{p}}$ and $R_{\mathfrak{p}}$ is an Artinian local ring, we have:
    \begin{displaymath}
        \begin{aligned}
            \operatorname{Ass}_R(M/N)&\subseteq \operatorname{Ass}_R(M_{\mathfrak{p}}) \\&=\operatorname{Ass}_{R_{\mathfrak{p}}}(M_{\mathfrak{p}})\cap \operatorname{Spec}(R)\\&\subseteq \operatorname{Spec}(R_{\mathfrak{p}})\cap\operatorname{Spec}(R) \\&=\{\mathfrak{p} R_{\mathfrak{p}}\}\cap\operatorname{Spec}(R)\\&=\{\mathfrak{p}\}.
        \end{aligned}
    \end{displaymath}

    (2) This will be shown by induction on the cardinality $|\operatorname{Ass}_R(M)|$. Since $M= (0)$ is equivalent to $|\operatorname{Ass}_R(M)|=0$, the case where $|\operatorname{Ass}_R(M)|=1$ may be considered, say $\operatorname{Ass}_R(M)=\{\mathfrak{q}\}$. As $M$ satisfies $(S_1)$, $\mathfrak{q} \in \operatorname{Min}(R)$ ensures $R_{\mathfrak{q}}$ is Artinian local ring. Set $l:=\ell\ell(R_{\mathfrak{q}})$. First, it is checked $\mathfrak{q}^l M=(0)$. Indeed, if not, then the chain of inclusions
    \begin{displaymath}
        \emptyset \neq \operatorname{Ass}_R(\mathfrak{q}^l M)\subseteq \operatorname{Ass}_R(M)=\{\mathfrak{q}\}    
    \end{displaymath}
    implies $\{\mathfrak{q}\}=\operatorname{Ass}_R(\mathfrak{q}^l M)$, and so this guarantees $\mathfrak{q}^lM_{\mathfrak{q}}\neq (0)$, contradicting $\mathfrak{q}^lR_{\mathfrak{q}}=(0)$. Hence, this exhibits $\mathfrak{q}^l M=(0)$. There is the desired filtration, 
    \begin{displaymath}
        0=\mathfrak{q}^l M\subseteq \mathfrak{q}^{l-1}M\subseteq \cdots \subseteq \mathfrak{q}^0M=M.   
    \end{displaymath}
    Assume that $\operatorname{Ass}_R(M)=\{\mathfrak{q}_1,\cdots,\mathfrak{q}_t\}$. Since $M$ satisfies $(S_1)$, $\mathfrak{q}_i\in \operatorname{Min}(R)$. Let $N$ be the kernel of the localization map $M\to M_{\mathfrak{q}_1}$. By$(1)$, there exist two chains of inclusions
    \begin{displaymath}
        \begin{aligned}
            &\operatorname{Ass}_R(N)\subseteq \{\mathfrak{q}_2,\cdots,\mathfrak{q}_t\}\subseteq \operatorname{Min}(R),
            \\& \operatorname{Ass}_R(M/N)\subseteq \{\mathfrak{q}_1\}\subseteq \operatorname{Min}(R).
        \end{aligned}
    \end{displaymath}
    Hence, both $N$ and $M/N$ satisfy $(S_1)$. By the induction hypothesis, one has filtrations 
    \begin{displaymath}
         0=N_0\subseteq N_1\subseteq \cdots \subseteq N_l=N
    \end{displaymath}
    and
    \begin{displaymath}
        0=\dfrac{M_0'}{N}\subseteq \dfrac{M_1'}{N}\subseteq \cdots \subseteq \dfrac{M_j'}{N}=M/N,
    \end{displaymath}
    where
    \begin{enumerate}
        \item $l\leq \sum_{\mathfrak{q}\in \operatorname{Ass}_R(N)} \ell\ell(R_{\mathfrak{q}}) \leq \sum_{i=2}^t\ell\ell(R_{\mathfrak{q}_i})$,
        \item $j\leq \sum_{\mathfrak{q} \in \operatorname{Ass}_R(M/N)}\ell\ell(R_{\mathfrak{q}})\leq \ell\ell(R_{\mathfrak{q}_1})$.
    \end{enumerate}
    Furthermore, for every $i$ there exists $\mathfrak{p}_i\in \operatorname{Min}(R)$ such that $N_i/N_{i-1}\in \operatorname{mod}R/\mathfrak{p}_i$, and similarly 
    \begin{displaymath}
        M'_i/M'_{i-1}\cong \dfrac{M_i'/N}{M'_{i-1}/N}\in \operatorname{mod}R/\mathfrak{p}_i'
    \end{displaymath}
    where $\mathfrak{p}'_i\in \operatorname{Min}(R)$. There is a filtration with required factor modules and has $l+j+1$ many submodules in-between
    \begin{displaymath}
        0=N_0\subseteq N_1\subseteq \cdots \subseteq N_l=N=M'_0\subseteq M'_1\subseteq \cdots\subseteq M'_j=M    
    \end{displaymath}
    satisfying
    \begin{displaymath}
        l+j\leq \sum_{i=2}^t\ell\ell(R_{\mathfrak{q}_i})+\ell\ell(R_{\mathfrak{q}_1})=\sum_{\mathfrak{p}\in \operatorname{Ass}_R(M)} \ell\ell(R_{\mathfrak{p}}).
    \end{displaymath}   

    (3) This follows by applying the relevant part of the proof of \cite[Theorem 5.1]{kawa} and using induction similar to (2).   
\end{proof}

\begin{proposition}\label{329}
    Suppose $R$ is a Noetherian ring satisfying $(S_1)$. If there exists $n\geq 0$ such that $\operatorname{size} \Omega^n_{R/\mathfrak{p}} (\operatorname{mod}R/\mathfrak{p})<\infty$ for every $\mathfrak{p}\in \operatorname{Min}(R)$, then 
    \begin{displaymath}
        \operatorname{size} \Omega^{n+1}_R( \operatorname{mod}R )\leq \left(\sum_{\mathfrak{p}\in \operatorname{Min}} \ell\ell(R_{\mathfrak{p}})\right)(n+1)(s+1)-1
    \end{displaymath}
    where $s:=\underset{\mathfrak{p} \in \operatorname{Min}(R)}{\sup} \{\operatorname{size} \Omega^n_{R/\mathfrak{p}} (\operatorname{mod}R/\mathfrak{p}) \}$.     
\end{proposition}    

\begin{proof}
    Fix $X\in \operatorname{mod}R$, and set $M=\Omega_R (X)$. By Lemma~\ref{lem:radius_subcategory}, \begin{displaymath}
        \operatorname{size} \Omega^n_R (\operatorname{mod}R/\mathfrak{p})<(n+1)(s+1)
    \end{displaymath}
    for every $\mathfrak{p} \in \operatorname{Min}(R)$. Since $R$ satisfies $(S_1)$, $M=\Omega_R (X)$ satisfies $(S_1)$. By Lemma \ref{lem:s_1}, there is a filtration 
    \begin{displaymath}
        0=M_0\subseteq M_1\subseteq \cdots \subseteq M_l=M,
    \end{displaymath}
    where 
    \begin{displaymath}
        l\leq \sum_{\mathfrak{p}\in \operatorname{Ass}_R(M)} \ell\ell(R_{\mathfrak{p}})\leq \sum_{\mathfrak{p}\in \operatorname{Min}(R)} \ell\ell(R_{\mathfrak{p}}),
    \end{displaymath}and for every $i$, there exists $\mathfrak{p}_i\in \operatorname{Min}(R)$ such that $M_i/M_{i-1}\in \operatorname{mod}R/\mathfrak{p}_i$.  For every $i$ there exists $H_i\in \operatorname{mod} R$ such that $\Omega^n_R (\operatorname{mod}R/\mathfrak{p}_i)\subseteq |H_i|_{(s+1)(n+1)}$. %Put $r:=\sum_{\mathfrak{p}\in \operatorname{Min}} \ell\ell(R_{\mathfrak{p}})$.

    There is a short exact sequences $0\to M_{i-1}\to M_i\to M_i/M_{i-1}\to 0$ giving rise to short exact sequences
    \begin{displaymath}
        0\to \Omega^n_R (M_{i-1})\to \Omega^n_R (M_i)\to \Omega^n_R(M_i/M_{i-1})\to 0.
    \end{displaymath}
    Since $M_1 \in \operatorname{mod}R/\mathfrak{p}_1$, it follows that $\Omega^n_{R} (M_1 )\in |H_1|_{(n+1)(s+1)}$, and similarly, $M_2/M_1\in \operatorname{mod} R/\mathfrak{p}_2$ ensures $\Omega^n_R (M_2/M_1) \in |H_2|_{(n+1)(s+1)}$. For $i=2$, the short exact sequence above shows $\Omega^n_R (M_2 )\in |H_1\oplus H_2|_{2(n+1)(s+1)}$. If this process is continued, then 
    \begin{displaymath}
        \Omega^{n+1}_R (X)\cong \Omega^n_R (M)\cong \Omega^n_R(M_l)\in |\bigoplus_{i=1}^l H_i|_{l(n+1)(s+1)}.
    \end{displaymath}
    As $X$ is arbitrary and $l\leq \sum_{\mathfrak{p}\in \operatorname{Min}(R)} \ell\ell(R_{\mathfrak{p}})$, \begin{displaymath}
        \operatorname{size} \Omega^{n+1}_R (\operatorname{mod}R)\leq \left(\sum_{\mathfrak{p}\in \operatorname{Min}(R)} \ell\ell(R_{\mathfrak{p}})\right)(n+1)(s+1)-1.
    \end{displaymath} 
\end{proof}  

\begin{remark}
    It is worth noting that if $R$ is an unmixed local ring (i.e. $\dim(R)=\dim(R/\mathfrak{p})$ for all $\mathfrak{p} \in \operatorname{Ass}(R)$), then $R$ satisfies $(S_1)$ and 
    \begin{displaymath}
    \sum_{\mathfrak{p}\in \operatorname{Min}(R)} \ell\ell(R_{\mathfrak{p}})\leq \sum_{\mathfrak{p}\in \operatorname{Min}(R)} \ell(R_{\mathfrak{p}})e(R/\mathfrak{p})=e(R)    
    \end{displaymath}
   where $e(-)$ denotes Hilbert-Samuel multiplicity, and for the last equality above, see \cite[Corollary 4.7.8]{Bruns/Herzog:1998}.
\end{remark}  

%%%%%%%%%%%%%%%%%%%%%%%%%%%%%%%%%%%%%%%
\subsection{Maximal Cohen-Macaulay modules}
\label{sec:MCM_category}
%%%%%%%%%%%%%%%%%%%%%%%%%%%%%%%%%%%%%%%

Let $R$ be a Cohen-Macaulay local ring. Recall $M\in \operatorname{mod}R$ is said to be \textbf{maximal Cohen-Macaulay} when the depth of $M$ is at least $\dim R$. In general, when $R$ is not necessarily local, $M\in \operatorname{mod}R$ is said to be \textbf{maximal Cohen-Macaulay} if $M_{\mathfrak{p}}$ is  \textbf{maximal Cohen-Macaulay} over $R_{\mathfrak p}$ for every prime ideal $\mathfrak p$ of $R$. The full subcategory of maximal Cohen-Macaulay modules in $\operatorname{mod}R$ is denoted $\operatorname{CM}(R)$. Note that if $M\in \operatorname{CM}(R)$, then $M$ satisfies $(S_n)$ for all $n\geq 0$.   

Next, an analogue of Proposition \ref{prop:minimal_primes_strong_generators} is proven for $\operatorname{CM}(R)$ when $R$ is a Cohen-Macaulay local ring. The following is a necessary result in doing so.

\begin{proposition}\label{sizecm}
    Let $R$ be a Cohen-Macaulay ring of finite Krull dimension $d$.
    \begin{enumerate}[\rm(1)]
        \item For each $s \geq d$, one has $\operatorname{size} \Omega^s_R(\operatorname{mod} R)\leq \operatorname{size} \operatorname{CM}(R)$. In particular, if $\operatorname{size} \operatorname{CM}(R)$ is finite, then $\operatorname{mod} R$ has a strong generator.
        \item Additionally, suppose $R$ is a local ring admitting a canonical module $\omega$. For every integer $n\geq 0$,
        \begin{displaymath}
            \begin{aligned}
                \operatorname{rank} \operatorname{CM}(R)&\leq (d+n+1)(\operatorname{size} \Omega^n_R (\operatorname{CM}(R))+1)-1 \\&\leq (d+n+1)(\operatorname{size} \Omega^n_R (\operatorname{mod}R)+1)-1
            \end{aligned}
        \end{displaymath}
        In particular, $\operatorname{mod} R$ has a strong generator if, and only if, $\operatorname{size} \operatorname{CM}(R)<\infty$ if, and only if, $\operatorname{rank} \operatorname{CM}(R)<\infty$.   
    \end{enumerate} 
\end{proposition} 

\begin{proof} 
    (1) If $\Omega^s_R(\operatorname{mod} R)\subseteq \operatorname{CM}(R)$ for every $s\geq d$, then 
    \begin{displaymath}
        \operatorname{size} \Omega^s_R(\operatorname{mod} R)\leq \operatorname{size} \operatorname{CM}(R),
    \end{displaymath}
    and so $(1)$ follows. 

    (2) Given any $M\in \operatorname{mod}{R}$, let $M^{\dagger}:= \operatorname{Hom}_R (M,\omega)$. It is enough to prove the first inequality as the second inequality is obvious. The $n=0$ case of the first inequality is precisely \cite[Proposition 5.10(1)]{Dao/Takahashi:2014}, so it may be assumed $n\geq 1$. If  $\operatorname{size} \Omega^n_R (\operatorname{CM}(R))=\infty$, then there is nothing to prove, so we may also assume $\operatorname{size} \Omega^n_R (\operatorname{CM}(R))<\infty$. Let $M\in \operatorname{CM}(R)$. There is an exact sequence 
    \begin{displaymath}
        0\to N\to F_{d+n-1}\to \cdots \to F_0\to M^{\dagger}\to 0,
    \end{displaymath}
    where $F_i$ s are free modules, and $N\in \operatorname{CM} (R)$. Dualizing by $\omega$ and remembering $\operatorname{Ext}_R^{>0}(M^{\dagger},\omega)=0$ and $M^{\dagger \dagger}\cong M$, there is an exact sequence
    \begin{displaymath}
        0\to M\to \omega^{\oplus b_0}\to \cdots\to \omega^{\oplus b_{d+n-1}}\to L\to 0
    \end{displaymath}
    where $b_i$ s are positive integers, and $L\cong N^{\dagger}\in \operatorname{CM}(R)$. By \cite[Lemma 5.8]{Dao/Takahashi:2014}, $M\in |\Omega^{d+n}_R( L) \oplus W|_{d+n+1}$, where $W=\bigoplus_{j=0}^{d+n-1}\omega^{\oplus b_j} \in \operatorname{CM}(R)$. Setting $s=\operatorname{size} \Omega^n_R (\operatorname{CM}(R))$, there exists $G\in \operatorname{mod} R$ such that $\Omega^n_R (L)\in |G|_{s+1}$, and so, $\Omega^{d+n}_R (L)\in |\Omega^d_R (G)|_{s+1}$. Thus, 
    \begin{displaymath}
        M\in |\Omega^{d+n}_R (L) \oplus W|_{d+n+1}\subseteq |\Omega^d_R (G)\oplus W|_{(s+1)(d+n+1)}.
    \end{displaymath}
    As $M\in \operatorname{CM}(R)$ is arbitrary, this promises
    \begin{displaymath}
        \operatorname{CM}(R)\subseteq |\Omega^d_R (G)\oplus W|_{(s+1)(d+n+1)}.
    \end{displaymath}
    Since $\Omega^d_R (G)\oplus W\in \operatorname{CM}(R)$ and $\operatorname{CM}(R)$ is closed under direct summands and extensions, 
    \begin{displaymath}
        |\Omega^d_R (G)\oplus W|_{(s+1)(d+n+1)}\subseteq \operatorname{CM}(R).
    \end{displaymath}
    Thus, $\operatorname{CM}(R)=|\Omega^d_R (G)\oplus W|_{(s+1)(d+n+1)}$, and the desired upper bound follows.

    (3) The last part about strong generator follows from combining with (1).   
\end{proof}     

\begin{proof}[Proof of Proposition~\ref{prop:CM_category_bounds}]
    It may be assumed $s<\infty$. Notice that $d=\dim(R/\mathfrak p)$ for every $\mathfrak p\in \operatorname{Min}(R)$. Furthermore, via Proposition \ref{sizecm}(1), $\operatorname{size} \Omega^d_{R/\mathfrak p}(\operatorname{mod}R/\mathfrak{p}))\leq s$ for every $\mathfrak p\in \operatorname{Min}(R)$. By Proposition \ref{329},
    \begin{displaymath}
        \operatorname{size} \Omega^{d+1}_R (\operatorname{mod}R )\leq \left(\sum_{\mathfrak{p}\in \operatorname{Min}(R)} \ell\ell(R_{\mathfrak{p}})\right)(d+1)(s+1)-1.
    \end{displaymath}
    Thus, Proposition \ref{sizecm} (with $n=d+1$) ensures, 
    \begin{displaymath}
        \operatorname{rank}\operatorname{CM}(R)\leq 2(d+1)^2\left(\sum_{\mathfrak{p}\in \operatorname{Min}(R)} \ell\ell(R_{\mathfrak{p}})\right)(s+1)-1.
    \end{displaymath}
\end{proof} 

An application of Lemma \ref{lem:s_1}(3) furnishes a stronger result for one-dimensional Cohen-Macaulay local rings.

\begin{proposition}
    If $R$ is a Cohen-Macaulay local ring of dimension $1$, then
    \begin{displaymath}
        \operatorname{size} \operatorname{CM}(R)\leq \left(\sum_{\mathfrak{p}\in \operatorname{Min}(R)} \ell\ell(R_{\mathfrak{p}})\right)(s+1)-1.
    \end{displaymath} 
    where $s:=\sup \{\operatorname{size}\operatorname{CM}(R/\mathfrak{p}) : \mathfrak{p} \in \operatorname{Min}(R)\}$. Moreover, if $R$ admits a canonical module, then 
    \begin{displaymath}
        \operatorname{rank} \operatorname{CM}(R)\leq 2\left(\sum_{\mathfrak{p}\in \operatorname{Min}(R)} \ell\ell(R_{\mathfrak{p}})\right)(s+1)-1.
    \end{displaymath}

%\begin{displaymath}
 %   \operatorname{rank} \operatorname{CM}(R)\leq 2\left(\sum_{\mathfrak{p}\in \operatorname{Min}(R)} \ell\ell(R_{\mathfrak{p}})\right)(s+1)-1. \end{displaymath}
\end{proposition}   

\begin{proof}
    It may be assumed $s$ is finite. Since $R$ is a one-dimensional Cohen-Macaulay local ring, so is $R/\mathfrak p$ for every $\mathfrak{p} \in \operatorname{Min}(R)$, and so every $R/\mathfrak p$-module of positive depth belongs to $\operatorname{CM}(R/\mathfrak{p})$. 
    
    Choose $M\in \operatorname{CM}(R)$. By Lemma \ref{lem:s_1}(3), there is a filtration 
    \begin{displaymath}
        0=M_0\subseteq M_1\subseteq \cdots \subseteq M_l=M,
    \end{displaymath}
    where 
    \begin{displaymath}
        l\leq \sum_{\mathfrak{p}\in \operatorname{Ass}_R(M)} \ell\ell(R_{\mathfrak{p}})\leq \sum_{\mathfrak{p}\in \operatorname{Min}(R)} \ell\ell(R_{\mathfrak{p}}),
    \end{displaymath}and for every $i$, there exists $\mathfrak{p}_i\in \operatorname{Min}(R)$ such that $M_i/M_{i-1}\in \operatorname{CM}(R/\mathfrak{p}_i)$. For every $i$ there exists $H_i\in \operatorname{mod}R/\mathfrak{p}\subseteq \operatorname{mod}R$ such that $ \operatorname{CM}(R/\mathfrak{p}_i)\subseteq |H_i|_{s+1}$. For each $i$, there is a short exact sequences $0\to M_{i-1}\to M_i\to M_i/M_{i-1}\to 0$. Since $M_1 \in \operatorname{CM}(R/\mathfrak{p}_1)$, it follows that $M_1\in |H_1|_{s+1}$, and so, $M_2/M_1\in \operatorname{CM} (R/\mathfrak{p}_2)$ implies $M_2/M_1 \in |H_2|_{s+1}$. For $i=2$, the short exact sequence above shows $M_2 \in |H_1\oplus H_2|_{2(s+1)}$. If this process is continued, then 
    \begin{displaymath}
         M\cong M_l\in |\bigoplus_{i=1}^l H_i|_{l(s+1)}.
    \end{displaymath}
    As $M\in \operatorname{CM}(R)$ is arbitrary and $l\leq \sum_{\mathfrak{p}\in \operatorname{Min}(R)} \ell\ell(R_{\mathfrak{p}})$, it follows that
    \begin{displaymath}
        \operatorname{size} \operatorname{CM}(R)\leq \left(\sum_{\mathfrak{p}\in \operatorname{Min}(R)} \ell\ell(R_{\mathfrak{p}})\right)(s+1)-1.
    \end{displaymath} 
    The rank inequality follows from Proposition \ref{sizecm}(2).
\end{proof}   

%%%%%%%%%%%%%%%%%%%%%%%%%%%%%%%%%%%%%%%
\section{(Co)ghost index}
\label{sec:co_ghost_mod_cats}
%%%%%%%%%%%%%%%%%%%%%%%%%%%%%%%%%%%%%%%

In this section, it will be shown (co)ghost index for the category of finitely generated modules over a Noetherian ring satisfies a local-to-global principle a la \cite{Letz:2021}, and a number of examples will be computed in the $F$-finite case. The main result of this section is the following statement.

\begin{Theorem}\label{thm:ghost_index_local_global}
    Let $R$ be a Noetherian ring. If $E,G\in \operatorname{mod}R$, then the following are equivalent:
    \begin{enumerate}
        \item $\operatorname{gin}^G_{\operatorname{mod}R} (E)<\infty$;
        \item $\operatorname{gin}^{G_\mathfrak{m}}_{\operatorname{mod}R_\mathfrak{m}} (E_\mathfrak{m}) < \infty$ for all $\mathfrak{m}\in \operatorname{mSpec}(R)$;
        \item $\operatorname{gin}^{G_\mathfrak{p}}_{\operatorname{mod}R_\mathfrak{p}} (E_\mathfrak{p}) < \infty$ for all $\mathfrak{p}\in \operatorname{Spec}(R)$.
    \end{enumerate}
    A similar statement holds for coghost index.
\end{Theorem}

These values are closely related to the parameters of a strong generator for modules categories. Indeed, if $G\in \operatorname{mod}R$ is a strong generator such that $\Omega^s_R (\operatorname{mod}R) \subseteq |G|_n$, then $\Omega^s_R (E) \in |G|_n$ for all $E\in \operatorname{mod}R$. Furthermore, Lemma~\ref{lem:ghost_lemma_abelian} ensures
\begin{displaymath}
    \operatorname{gin}_{\operatorname{mod}R}^G (\Omega^s_R (E)) \leq n,
\end{displaymath}
and a similar bound holds for $\operatorname{cogin}_{\operatorname{mod}R}^G (\Omega^s_R (E))$. 

%%%%%%%%%%%%%%%%%%%%%%%%%%%%%%%%%%%%%%%
\subsection{Results}
\label{sec:co_ghost_mod_cats_results}
%%%%%%%%%%%%%%%%%%%%%%%%%%%%%%%%%%%%%%%

To start proving of the Theorem~\ref{thm:ghost_index_local_global}, an understanding as to how (co)ghost maps behaves under localization is crucial.

\begin{lemma}\label{lem:localizing_(co)ghost_maps}
    Let $R$ be a Noetherian ring, and $\mathfrak{p}\in \operatorname{Spec}(R)$. If $f \colon E \to M$ is a $G$-(co)ghost map in $\operatorname{mod}R$, then $f_\mathfrak{p} \colon E_\mathfrak{p} \to M_\mathfrak{p}$ is $G_\mathfrak{p}$-(co)ghost map in $\operatorname{mod}R_\mathfrak{p}$.
\end{lemma}

\begin{proof}
    It suffices to only check the ghost case as coghost follows similarly. Since $f$ is $G$-ghost, the induced map
    \begin{displaymath}
        \operatorname{Hom}_R (G,f) \colon \operatorname{Hom}_R (G,E) \to \operatorname{Hom}_R (G,M)
    \end{displaymath}
    vanishes. As $R$ is Noetherian and localization is flat, there is an $R_\mathfrak{p}$-module isomorphism for all $K,L\in \operatorname{mod}R$:
    \begin{displaymath}
        \operatorname{Hom}_{R_\mathfrak{p}} (K_\mathfrak{p},L_\mathfrak{p}) \cong \big(\operatorname{Hom}_R (K,L)\big)_\mathfrak{p}.
    \end{displaymath}
    Hence, after localizing, the induced map
    \begin{displaymath}
        \operatorname{Hom}_{R_\mathfrak{p}} (G_\mathfrak{p},f_\mathfrak{p}) \colon \operatorname{Hom}_{R_\mathfrak{p}} (G_\mathfrak{p},E_\mathfrak{p}) \to \operatorname{Hom}_{R_\mathfrak{p}} (G_\mathfrak{p},M_\mathfrak{p})
    \end{displaymath}
    vanishes. Thus, $f_p\colon E_\mathfrak{p} \to M_\mathfrak{p}$ is $G_\mathfrak{p}$-coghost in $\operatorname{mod}R_\mathfrak{p}$ as desired.
\end{proof}

\begin{lemma}\label{lem:ghost_local_global_principle}
    Let $R$ be a Noetherian ring. If $E,G\in \operatorname{mod}R$, then
    \begin{displaymath}
        \begin{aligned}
            \operatorname{gin}^G_{\operatorname{mod}R} (E) &= \sup\{ \operatorname{gin}^{G_\mathfrak{m}}_{\operatorname{mod}R_\mathfrak{m}} (E_\mathfrak{m}) : \mathfrak{m}\in \operatorname{mSpec}(R)\} \\&= \sup\{ \operatorname{gin}^{G_\mathfrak{p}}_{\operatorname{mod}R_\mathfrak{p}} (E_\mathfrak{p}) : \mathfrak{p}\in \operatorname{Spec}(R)\}
        \end{aligned}
    \end{displaymath}
    A similar result holds for $\operatorname{cogin}^G_{\operatorname{mod}R} (E)$
\end{lemma}

\begin{proof}
    Let $f \colon E \to B$ be an $n$-fold $G$-ghost map. After localizing at a maximal ideal $\mathfrak{m}\subseteq R$, Lemma~\ref{lem:localizing_(co)ghost_maps} promises that $f_\mathfrak{m} \colon E_\mathfrak{m} \to B_\mathfrak{m}$ is an $n$-fold $G_\mathfrak{m}$-ghost map in $\operatorname{mod}R_\mathfrak{m}$. If $f_\mathfrak{m}=0$ for all $\mathfrak{m}\in \operatorname{mSpec}(R)$, then $f=0$. This shows that
    \begin{displaymath}
        \begin{aligned}
            \operatorname{gin}^G_{\operatorname{mod}R} (E) &\leq \sup\{ \operatorname{gin}^{G_\mathfrak{m}}_{\operatorname{mod}R_\mathfrak{m}}  (E_\mathfrak{m}) : \mathfrak{m} \in \operatorname{mSpec}(R)\} \\&\leq \sup\{ \operatorname{gin}^{G_\mathfrak{p}}_{\operatorname{mod}R_\mathfrak{p}}  (E_\mathfrak{p}) : \mathfrak{p} \in \operatorname{Spec}(R)\},
        \end{aligned}
    \end{displaymath}
    where the second follows from the fact $\operatorname{mSpec}(R) \subseteq \operatorname{Spec}(R)$.

    %Localization at the maximal ideal preverses the ghost map as our Noetherian ring is Noetherian, the modules are f.g., and localization is flat. The second part comes from showing the image vanishes.

    By Lemma~\ref{lem:ghost_lemma_abelian}, $\operatorname{gin}^G_{\operatorname{mod}R} (E) \leq n$ if, and only if, $E\in |\operatorname{Fac}(G)|_n$. If $E\in |\operatorname{Fac}(G)|_n$, then for each $\mathfrak{p}\in \operatorname{Spec}(R)$, $E_\mathfrak{p} \in |\operatorname{Fac}(G_\mathfrak{p})|_n$, and so $\operatorname{gin}^{G_\mathfrak{p}}_{\operatorname{mod}R_\mathfrak{p}} (E_\mathfrak{p}) \leq n$. Hence, this shows 
    \begin{displaymath}
        \sup\{ \operatorname{gin}^{G_\mathfrak{p}}_{\operatorname{mod}R_\mathfrak{p}}  (E_\mathfrak{p}) : \mathfrak{p} \in \operatorname{Spec}(R)\} \leq \operatorname{gin}^G_{\operatorname{mod}R} (E).
    \end{displaymath}

    The claim for coghost index $\operatorname{gin}^G_{\operatorname{mod}R} (E)$ comes by applying a similar argument above and using the dual of Lemma~\ref{lem:ghost_lemma_abelian}.
\end{proof}

\begin{lemma}\label{lem:approximation_localizing}
    Let $R$ be a Noetherian ring, and choose $E,G\in \operatorname{mod}R$. If $f\colon X \to E$ is a right $G$-approximation and $\mathfrak{p}\in \operatorname{Spec}(R)$, then $f_\mathfrak{p} \colon X_\mathfrak{p} \to G_\mathfrak{p}$ is a right $G_\mathfrak{p}$ approximation in $\operatorname{mod}R_\mathfrak{p}$. A similar statement holds for localizing left $G$-approximations.
\end{lemma}

\begin{proof}
    Consider a morphism $h:X' \to E_\mathfrak{p}$ in $\operatorname{mod}R_\mathfrak{p}$ where $X'\in \operatorname{add}(G_\mathfrak{p})$. There exists an $\alpha\geq 1$ such that $X'$ is a direct summand of $G_\mathfrak{p}^{\oplus \alpha}$. Hence, there exists an epimorphism $\pi\colon G^{\oplus \alpha}_\mathfrak{p} \to X'$ and monomorphism $i\colon X' \to G^{\oplus \alpha}_\mathfrak{p}$ such that $\pi \circ i = 1_{X'}$. As $R$ is Noetherian and localization is flat, there is an $R_\mathfrak{p}$-module isomorphism for all $K,L\in \operatorname{mod}R$:
    \begin{displaymath}
        \operatorname{Hom}_{R_\mathfrak{p}} (K_\mathfrak{p},L_\mathfrak{p}) \cong \big(\operatorname{Hom}_R (K,L)\big)_\mathfrak{p}.
    \end{displaymath}
    Choose $\overline{X'}\in \operatorname{mod}R$ such that $\overline{X'}_\mathfrak{p}=X'$. There exists a map $\overline{\pi} \colon G^{\oplus \alpha} \to \overline{X'}$ such that $\overline{\pi}_\mathfrak{p}=\pi$. Furthermore, there is a map $\overline{h} \colon \overline{X'} \to E$ such that $\overline{h}_\mathfrak{p} = h$. Now this gives a map $\overline{h} \circ \overline{\pi}\colon G^{\oplus \alpha}\to E$. As $f$ is a right $G$-approximation, there exists $g \colon G^{\oplus \alpha} \to X$ such that $\overline{h} \circ \overline{\pi} = f \circ g$. After localizing at $\mathfrak{p}$, there is a diagram
    \begin{displaymath}
        \begin{tikzcd}
            {X_\mathfrak{p}} && E \\
            {G^{\oplus \alpha}_\mathfrak{p}} && {X'} & {}
            \arrow["{h_\mathfrak{p}}", from=2-1, to=1-1]
            \arrow["{f_\mathfrak{p}}", from=1-1, to=1-3]
            \arrow["i"', shift right=1, from=2-3, to=2-1]
            \arrow["h"', from=2-3, to=1-3]
            \arrow["\pi"', shift right=1, from=2-1, to=2-3]
        \end{tikzcd}
    \end{displaymath}
    in $\operatorname{mod}R_\mathfrak{p}$ for which the outside square is commutative. The proof is complete if the inside square commutes. Notice that
    \begin{displaymath}
        \begin{aligned}
            (f_\mathfrak{p} \circ g_\mathfrak{p}) \circ i &= (\overline{h} \circ \overline{\pi})_\mathfrak{p} \circ i \\&= (h \circ \pi) \circ i \\&= h \circ (\pi \circ i) \\&= h.
        \end{aligned}
    \end{displaymath}
    Therefore, this gives the desired morphism $g_\mathfrak{p} \circ i \colon X' \to X$ to show $f_\mathfrak{p}$ is a right $G_\mathfrak{p}$-approximation in $\operatorname{mod}R_\mathfrak{p}$.
\end{proof}

\begin{proof}[Proof of Theorem~\ref{thm:ghost_index_local_global}]
    From Lemma~\ref{lem:ghost_local_global_principle}, $(1)\implies (2) \implies (3)$ and so, $(3)\implies (1)$ needs to be checked. Define a diagram of the form
    \[
    \begin{tikzcd}
        E_0 \arrow{r}{\pi_0} & E_1 \arrow{r}{\pi_1}  & \cdots \\
        A_0 \arrow{u}{f_0} \arrow{r}{h_0} & A_1 \arrow{u}{f_1} \arrow{r}{h_0} & \cdots 
    \end{tikzcd}
    \]
    to be \textit{admissible} when each $\pi_j \colon E_j \to E_{j+1}$ are $G$-ghost maps with $E_0 :=E$, $f_j  \colon A_j \to E_j$ a right $G$-approximation satisfying the property that $\pi_j$ is a cokernel map of $f_j$ and $f_{j+1} \circ h_j = \pi_j \circ f_ j$. First, it is shown that an admissible diagram always exist in $\operatorname{mod}R$. Indeed, as $\operatorname{add}(G)$ is contravariantly finite, there exists a right $G$-approximation $f_0 \colon A_0 \to E$ for $A_0 \in \operatorname{add}(G)$. Let $E_1 := \operatorname{coker}(f_0)$. Then $\pi_0 \colon E_0 \to E_1$ is $G$-ghost as $f_0 \colon A_0 \to E_0$ is a right $G$-approximation. Choose a right $G$-approximation $f_1 \colon A_1 \to E_1$. There exists $h_0 \colon A_0 \to A_1$ such that $f_1 \circ h_0 = \pi_0 \circ f_0$. By iterating this process, there exist admissible diagrams.

    Next, it is shown that for any admissible diagram of $E,G \in \operatorname{mod}R$,
    \begin{displaymath}
        \operatorname{gin}_{\operatorname{mod}R}^G (E) = \inf\{ n\geq 0: \pi_n \circ \cdots \circ \pi_0 = 0\}.
    \end{displaymath}
    Since each $\pi_j$ are $G$-ghost,
    \begin{displaymath}
        \operatorname{gin}_{\operatorname{mod}R}^G (E) \geq \inf\{ n\geq 0: \pi_n \circ \cdots \circ \pi_0 = 0\}.
    \end{displaymath}
    Assume there exists $n\geq 0$ such that $\pi_n \circ \cdots \circ \pi_0 = 0$. This is a composition of surjective maps, so $E_{n+1}=\operatorname{im}(\pi_n) = 0$. Hence $\pi_n =0$, and so, $E_n \in \operatorname{Fac}(G)$. There is a short exact sequence
    \begin{displaymath}
        0 \to \operatorname{im}(f_{n-1}) \to E_{n-1} \to E_n \to 0
    \end{displaymath}
    where $\operatorname{im}(f_{n-1})\in \operatorname{Fac}(G)$. Continuing to work backwards, it follows that $E\in |\operatorname{Fac}(G)|_n$, which shows $\operatorname{gin}_{\operatorname{mod}R}^G (E) \leq n$.

    Localizing any admissible diagram of $E,G$ in $\operatorname{mod}R$ at a prime ideal $\mathfrak{p}$ yields an admissible diagram of $E_\mathfrak{p},G_\mathfrak{p}$ in $\operatorname{mod}R_\mathfrak{p}$. Indeed, this follows from Lemma~\ref{lem:localizing_(co)ghost_maps} and Lemma~\ref{lem:approximation_localizing}. Fix an admissible diagram $(E_j,A_j,\pi_j)$ of $E,G$. It will be shown that
    \begin{displaymath}
        \{\mathfrak{p}: \operatorname{gin}_{\operatorname{mod}R_\mathfrak{p}}^{G_\mathfrak{p}} (E_\mathfrak{p}) \leq n \} = \operatorname{Spec}(R) \setminus \operatorname{supp} \operatorname{im} (\pi_n \circ \cdots \circ \pi_0).
    \end{displaymath}
    If $\operatorname{gin}_{\operatorname{mod}R_\mathfrak{p}}^{G_\mathfrak{p}} (E_\mathfrak{p}) \leq n$, then $(\pi_n \circ \cdots \circ \pi_0)_\mathfrak{p} =0$ as $(\pi_j)_\mathfrak{p}$ are $G_\mathfrak{p}$-ghost, so $\mathfrak{p} \not\in \operatorname{supp} \operatorname{im} (\pi_n \circ \cdots \circ \pi_0)$. On the other hand, if $ \mathfrak{p} \in \operatorname{Spec}(R) \setminus \operatorname{supp} \operatorname{im} (\pi_n \circ \cdots \circ \pi_0)$, then the work above ensures $\operatorname{gin}_{\operatorname{mod}R_\mathfrak{p}}^{G_\mathfrak{p}} (E_\mathfrak{p}) \leq n$ as localization of admissible diagrams remain admissible.

    Since the sets $V_n \colon = \operatorname{Spec}(R) \setminus \operatorname{supp} \operatorname{im} (\pi_n \circ \cdots \circ \pi_0)$ are Zariski open and $V_n \subseteq V_{n+1}$, there exists $N\geq 0$ such that $V_N = V_{N+1} = \cdots$. Assuming $(3)$ holds, for each $\mathfrak{p}\in \operatorname{Spec}(R)$ there exists $n$ such that $\mathfrak{p}\in V_n$. This implies $\operatorname{gin}_{\operatorname{mod}R_\mathfrak{p}}^{G_\mathfrak{p}} (E_\mathfrak{p}) \leq N$; showing that $(1)$ holds.
\end{proof}

\begin{remark}
    In a suitable sense, admissible diagrams in the proof of Theorem~\ref{thm:ghost_index_local_global} are a module-theoretic analog of the adams resolution in a triangulated category (see \cite[Definition 3.4.1]{Letz:2020}).
\end{remark}

\begin{example}\label{ex:Artinian_first_frob_push}
    If $(R,\mathfrak{m},k)$ is an $F$-finite Artinian local ring, then $\operatorname{gin}^{F_\ast R}_{\operatorname{mod}R} (E) < \infty$ for all $E\in \operatorname{mod}R$. Indeed, there exists a surjection $R \to k$, and $\operatorname{gin}^k_{\operatorname{mod}R} (E) \leq \ell\ell(R)$. Then $k^{\oplus \alpha} \cong F_\ast k \in \operatorname{Fac}(F_\ast R)$, and so $\operatorname{mod}R = |\operatorname{Fac}(F_\ast R)|_{\ell\ell(R)}$. If $\ell\ell(R)>1$, then $\operatorname{mod}R \not= |\operatorname{Fac}(F_\ast R)|_1$ as this would imply $R$ is a direct summand of $F_\ast R$, and hence $R$ would be reduced.
\end{example}

\begin{remark}\label{sec:F_split_rings}
    Let $R$ be a Noetherian $F$-finite ring. if the map $F \colon R \to R$ splits in $\operatorname{mod}R$, then $R$ is said to be \textbf{Frobenius split}, or \textbf{$F$-split}. For instance, given  $f\in k \llbracket  x_1,\ldots,x_n \rrbracket  =\colon S$ where $k$ is a perfect field of prime characteristic, then \cite[Proposition 1.7]{Fedder:1983} yields a useful computational strategy to determine whether or not $S/(f)$ is $F$-split.
\end{remark}

\begin{example}
    Let $R$ be a $1$-dimensional Noetherian $F$-finite integral domain. Example~\ref{ex:Iyengar/Takahashi_strong_gen_examples} verifies $\operatorname{ca}^3 (R)\not=0$, and so there exists a nonzero divisor $a\in \operatorname{ca}^3 (R)$. Note that $R/a R$ is Artinian. Choose $e \geq 0$ such that $F_\ast^e (R/ aR) \cong k^{\oplus \alpha}$ in $\operatorname{mod} R/a R$. Now, restriction of scalars confirms $F_\ast^e (R/ aR) \in \operatorname{Fac}(F_\ast^e R)$, and Lemma~\ref{lem:lifting_strong_generation_from_quotient} exhibits $\Omega^4_R(\operatorname{mod}R) \subseteq |\Omega^1_R (k)|_{\ell\ell(R/a R)}$. Putting this together, 
    \begin{displaymath}
        \Omega^4_R(\operatorname{mod}R) \subseteq |\operatorname{Fac}\big(\Omega^1_R (F_\ast^e R)\big)|_{\ell\ell(R/a R)}.
    \end{displaymath}    
    However, as $\Omega_R^1 (F_\ast^e R) \in \operatorname{Fac}(\Omega^1_R (F_\ast R))$, it follows  that
    \begin{displaymath}
        \Omega^4_R(\operatorname{mod}R) \subseteq |\operatorname{Fac}\big(\Omega^1_R (F_\ast R)\big)|_{\ell\ell(R/a R)}.
    \end{displaymath}
    This bound can be sharpened even in the case $R$ is not Frobenius split. For instance, let $k$ be an $F$-finite field of characteristic five. Consider $R=k \llbracket  x,y \rrbracket  /(x^2  - y^3)$. This is a one-dimensional local integral domain that is not Frobenius split. By Example~\ref{ex:Dieterich_hypersurface_cohomology_annihilator}, the elements $2x, 3y^2$ belong to $\operatorname{ca}^2 (R)$. From the previous discussion, if $a=x$, then $\ell\ell(R/xR)=3$, and so 
    \begin{displaymath}
        \operatorname{gin}_{\operatorname{mod}R}^{\Omega^1_R (F_\ast R)} (\Omega^3_R(E)) \leq 3
    \end{displaymath}
    for all $E\in \operatorname{mod}R$.
\end{example}

\begin{remark}
    In general, for some $G\in \operatorname{mod}R$ and $n\geq 0$ it cannot be expected $|\operatorname{Fac}(G)|_n = \operatorname{mod}R$ implies $|G|_N = \operatorname{mod}R$ for some $N\geq 0$. Let $R$ be a non-regular Noetherian ring. Note that $|\operatorname{Fac}(R)|_1 = \operatorname{mod}R$, but $|R|_n = \operatorname{mod}R$ would imply $R$ strongly generates $D^b (\operatorname{mod}R)$, which cannot happen unless $R$ is regular. Furthermore, this exhibits that (co)ghost index in $\operatorname{mod}R$ does not necessarily coincide with level in $D^b (\operatorname{mod}R)$.
\end{remark}

\begin{example}
    The following produces a family of examples where both the coghost index and Krull dimension can be made arbitrarily large. Let $k$ be an $F$-finite field of prime characteristic $p$. Consider the ring 
    \begin{displaymath}
        R=k[x_1,\ldots,x_n]/(x_1^{pm},\ldots,x_j^{pm})
    \end{displaymath}
    for some $m\geq 0$. A direct calculation will show that $F_\ast R$ is a direct sum of copies of $S\colon =k[x_1,\ldots,x_n]/(x_1^m,\ldots,x_n^m)$. Consider the composition of maps
    \begin{displaymath}
        R \xrightarrow{x_1^m} R \xrightarrow{x_2^m} \cdots \xrightarrow{x_{n-1}^m} R \xrightarrow{x_j^m} R,
    \end{displaymath}
    i.e. the endomorphism $(x_1\cdots x_j)^m \colon R \to R$. This is a nonzero map, but its induced map on $\operatorname{Hom}(R, F_\ast R)$ vanishes, and hence, implying that $\operatorname{cogin}_{\operatorname{mod}R}^{F_\ast R} (R) \geq j$.
\end{example}

\begin{proposition}
    Let $k$ be an $F$-finite field of prime characteristic $p$. If $I$ is an ideal of $k \llbracket  x_1,\ldots,x_n \rrbracket  $ that is generated by a regular sequence of monomials of the form $f_1^{pm}, \ldots, f_c^{pm}$ for some $m\geq 1$, then $\operatorname{cogin}_{\operatorname{mod}R}^{F_\ast R} (R) \geq c$.
\end{proposition}

\begin{proof}
    A direct calculation shows that $F_\ast R$ is a direct sum of copies of 
    \begin{displaymath}
        k \llbracket  x_1,\ldots,x_n \rrbracket  /(f_1^m,\ldots,f_c^m).
    \end{displaymath}
     Consider the composition of maps
    \begin{displaymath}
        R \xrightarrow{f_1^m} R \xrightarrow{f_2^m} \cdots \xrightarrow{f_{c-1}^m} R \xrightarrow{f_c^m} R,
    \end{displaymath}
    i.e. the endomorphism $(f_1\cdots f_c)^m \colon R \to R$. This is a nonzero map, but its induced map on $\operatorname{Hom}(R, F_\ast R)$ vanishes, and hence, implying that $\operatorname{cogin}_{\operatorname{mod}R}^{F_\ast R} (R) \geq c$.
\end{proof}

\begin{proposition}\label{prop:f_finite_non_generator}
    Choose an integer $e\geq 1$. If $R$ is a Noetherian $F$-finite ring and for each $N$ there exists $f_1,\ldots f_N \in \operatorname{ann}_R (F_\ast^e R)$ such that $f_1 \cdots f_N \not =0$, then $F_\ast^e R$ is not a strong generator for $D^b (\operatorname{mod}R)$.
\end{proposition}

\begin{proof}
    If $f\in \operatorname{ann}_R (F_\ast^e R)$, then the multiplication map $f \colon F_\ast^e R \to F_\ast^e R$ vanishes in $\operatorname{mod}R$. Furthermore, by viewing $f$ as an endomorphism on $R$, it is $F_\ast^e R$-coghost in $\operatorname{mod}R$. Hence, if interpreted within this context, the hypothesis ensures for each $N\geq 1$ there exists a nonzero $N$-fold $F_\ast^e R$-coghost endomorphism on $R$ in $\operatorname{mod}R$, and so $\operatorname{cogin}_{\operatorname{mod}R}^{F_\ast^e R} (R)=\infty$. But $R$ is projective, so $\operatorname{Ext}^n_R (R, F_\ast^e R) =0$ when $n\not=0$. Hence, any $N$-fold $F_\ast^e R$-coghost map in $\operatorname{mod}R$ is such in $D^b (\operatorname{mod}R)$, and this completes the proof.
\end{proof}

\begin{corollary}\label{cor:f_finite_generation_time_lower_bound}
    Let $R$ be a Noetherian $F$-finite ring. If there exist $f_1,\ldots,f_n\in \operatorname{ann}_R (F_\ast^e R)$ whose product is nonzero, then $\operatorname{gen.time}(G)\geq n$.
\end{corollary}

\begin{proof}
    As $R$ is projective, each $f_j$ is an $F_\ast^e R$-coghost map in $D^b (\operatorname{mod}R)$. Since the product is nonzero, 
    \begin{displaymath}
        n \leq \operatorname{cogin}_{\operatorname{mod}R}^{F_\ast^e R} (R) \leq \operatorname{cogin}_{D^b(\operatorname{mod}R)}^{F_\ast^e R} (R)
    \end{displaymath}
    Note this inequality holds as $\operatorname{Ext}^n_R (R, F_\ast^e R) =0$ when $n\not=0$, and so $F_\ast^e R$-coghost endomorphisms of $R$ in $\operatorname{mod}R$ extend to such in $D^b (\operatorname{mod}R)$.
\end{proof}

\bibliographystyle{plain}
\bibliography{mainbib}

\end{document}
