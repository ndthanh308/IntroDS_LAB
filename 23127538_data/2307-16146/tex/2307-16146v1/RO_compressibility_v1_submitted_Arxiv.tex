\documentclass[11pt,a4paper]{article}%\documentclass[bibliography=totocnumbered,listof=totocnumbered,openany,12pt]{scrbook}
%\usepackage[margin=1.0in]{geometry}
\usepackage[makeroom]{cancel}
\usepackage{bm}
\usepackage{amsmath}
\usepackage{amsfonts}
\usepackage{amssymb}
\usepackage{graphicx}
%\usepackage{frac}
\usepackage{authblk}
\usepackage{cite}
\usepackage[left=2.4cm,top=2.4cm,right=2.4cm,bottom=2.4cm,nohead]{geometry}
\DeclareGraphicsExtensions{.png,.jpg,.pdf}
\usepackage{epstopdf}
\usepackage{float}
\usepackage{fouriernc} 
\usepackage{enumitem}
\usepackage[utf8]{inputenc}
\usepackage[english]{babel}
\usepackage{mathtools}
\usepackage{amscd}
\usepackage{mathrsfs}
\usepackage{amsthm}
\usepackage{siunitx}%For SI units
\usepackage{caption}
\usepackage{subcaption}
\setlist{nolistsep,leftmargin=*}
\renewcommand{\baselinestretch}{1.1}
\setlength{\parskip}{0.2em}
\usepackage{mhchem} 
\usepackage{cleveref} 
\usepackage{xcolor}
\usepackage{soul}
\usepackage{mdframed}
\usepackage{framed}
\usepackage{nicefrac}
%\usepackage{lineno}
%\linenumbers
\usepackage{longtable}
\DeclareMathAlphabet{\mathpzc}{OT1}{pzc}{m}{it}

\title{Expansion and compression of polymeric materials: Implications for reverse osmosis membrane performance}

\makeatletter

\renewcommand\AB@authnote[1]{\textsuperscript{\normalfont#1}}
\renewcommand\Authsep{ }
\renewcommand\Authands{ and }

\makeatother
\renewcommand\Affilfont{\small}
		
\author[1]{P.M. Biesheuvel} 
\author[2]{M. Elimelech}


\affil[1]{Wetsus, European Centre of Excellence for Sustainable Water Technology,  Leeuwarden, The~Netherlands.}
\affil[2]{Department of Chemical and Environmental Engineering, Yale University, New Haven, USA.}

\date{} 

\newcommand{\s}[1]{\mathrm{_{#1}}}
\newcommand{\temphl}[1]{{\color{black} #1}} % change red to black

\newcommand{\WR}{WR}%{\textsc{wr}}

\newcommand{\dotorspace}{\;}%{\textsc{wr}}

\begin{document}

\maketitle

\begin{abstract}

We set up a theory for the expansion and compression of porous materials such as cross-linked polymer networks. The theory includes volume exclusion, affinity with the solvent, and finite stretching of the polymer chains. We extend this equilibrium theory to the case that a pressure is applied across a thin layer of such a material, for instance a membrane, and liquid flows across this layer. The theory describes how in the direction of liquid flow the membrane is increasingly compacted (becomes less porous), and the more so at higher applied pressures. For thin membranes used in reverse osmosis, a method to desalinate water, we calculate that for commercially available membranes the porosity decreases only very little across the membrane, in line with experimental observations that for pure solvents up to high pressures water flux is proportional with pressure.

\end{abstract} 

\section{Introduction}

Elastic porous materials, for instance made of cross-linked polymer networks, play an important role in many applications including tissue engineering % hydrogels for soil improvement, HPLC columns, size exclusion membranes used in biological studies, 
and membranes for water desalination~\cite{Horkay_2007,Davenport_2020}. In this work we focus on polymeric materials that form cross-linked networks with open spaces, i.e., pores or free volumes, in between the %strands, the 
chains of the  polymer network. The degree of swelling can be expressed as a porosity, which is the volume fraction that is open and accessible for solvent and solutes such as ions, and depends on many properties of the polymer network
such as cross-linking degree, elasticity, and polymer-solvent interaction. % (which for water is described as the degree of hydrophilicity/hydrophobicity). 
When the polymer is charged, electrostatic (Donnan) effects also play a role, but these effects are not discussed in the present work. 

The classical treatment is based on the Flory-Rehner theory~\cite{Flory_Rehner_1943}, which %heuristically 
combines lattice-based 
Flory-Huggins polymer theory with a linear theory that assumes weak chain stretching. The weak stretching assumption may not be valid in practice, resulting in an overestimate of the size of the particle, while the lattice-based model is difficult to combine with a calculation of hydrodynamics and fluid flow through the porous material. It also underestimates polymer volumetric interactions and thus leads to an overestimate of the density of the material. %The derivation in ref.~\cite{Flory_1943} is difficult to follow, but the derivation in ref..  
In the present work we set up an accurate theory for polymer expansion (swelling) and compression (compaction), including volume exclusion by the polymer chains based on an accurate equation of state, a solvent-polymer interaction energy, and non-linear theory of chain elasticity that includes finite stretching of the cross-linked polymer network. We extend this equilibrium model to the situation that solvent flows across a thin layer of such a porous material when a pressure difference is applied and we include expressions for fluid-polymer friction that depend on the density of the material. One type of such a porous layer are membranes, and these are applied in many separation processes in industry and water treatment, both for organic solvents and aqueous solutions. Calculations show how in the direction of solvent flow (towards lower pressures), the material becomes more dense, i.e., the porosity goes down. The findings of this work have important implications for understanding the mechanisms of water and salt transport in reverse osmosis membranes, which are widely used in water desalination.

\section{Theory for equilibrium swelling of cross-linked polymer networks}

A cross-linked polymer material % (in this work also called membrane, network, or polymer film) 
has a porosity, \textit{p}, or, vice-versa, a polymer volume fraction, $\phi$. These two volume fractions add up to one, thus $p+\phi=1$. The volume fraction of polymer, $\phi$, is the fraction of the total volume of a material that is occupied by polymer chains, and $p$ is the volume fraction filled %The open volume, i.e., the pores, or porosity, is filled 
with solvent (and if present, solutes).\footnote{In this paper the terms solvent, liquid, and fluid, all have the same meaning.} When the material is compressed, porosity goes down, and when it swells, porosity goes up. Because the total amount of polymer is fixed, a change in porosity of a sample submerged in liquid implies that for some time liquid flows into or out of the material, because the pore phase (porosity) is always filled with solvent.

The polymer network consists of nodes (also called cross-linkages, or branching points), with polymer chains (also called sub-chains, strings, or strands) in between the nodes. These chains oppose being stretched, i.e., a force is needed to stretch them, and progressively more force is needed to stretch them further. We assume the material only contains one type of chain, of one length, so at the same macroscopic location all chains are stretched equally. When we apply pressure, and fluid flows through a layer of this material (for instance, a membrane), we include how on the downstream side the membrane is supported against a rigid structure. That structure is also porous, i.e., it lets water pass, but keeps the membrane in place. 

To describe polymer expansion and compression, we must define several parameters.  
First of all, there are \textit{N} polymer chains per unit total volume, each with contour length $\ell$, which is the length, or distance, along the contour, or backbone, of the polymer chain from one node to the next. Thus $N\!\cdot\!\ell$ is a `polymer length density'. On both its ends, each chain is terminated by nodes, where a chain is connected to two %or three 
other chains, in this way forming a cross-linked network. In part of the theory the chains are described as strings of touching beads, each of size $\sigma\s{b}$. The volume of each bead is $v\s{b}=\pi/6\cdot \sigma_\text{b}^3$ (in $\text{m}^3$). This size can be chosen such that per unit chain length, following the contour of the chain, this theoretical chain, consisting of a string of beads, has the same volume (per unit chain length) as that of a polymer chain envision as a tube or cylinder. This approach leads to a highly exact equation-of-state for polymer solutions, brushes, and gels, in alignment with results of MD simulations~\cite{Biesheuvel_2008,Spruijt_2014,Spruijt_2015,Biesheuvel_Dykstra_2020}. With a polymer fraction $\phi$, the concentration of these beads is $c\s{b}=\phi/v\s{b}$. The various parameter just introduced are related by $N \cdot\ell = c\s{b} \cdot \sigma\s{b}$. 

When all polymer chains are stretched to the maximum, the material has the lowest possible density, and this volume fraction is $\phi\s{min}$. This minimum density depends on the architecture of the network, the contour length of the chains, and the polymer volume per unit chain length. The minimum density $\phi\s{min}$ can be anywhere between a fraction of a percent for highly charged hydrogels, to $\sim\!60$~\%~for a compact reverse osmosis water desalination membrane. 

If a porous material is free to expand and contract in all directions, then any relative volumetric change leads to a relative change in any distance measure that is one-third the volumetric change. This is similar to Eq.~(2) in ref.~\cite{Flory_Rehner_1943} where the extension of chains and polymer volume fraction are related by the same proportionality. But when the material is constricted in some direction, then the expansion in the other directions is not one-third of the volume change, but a higher number. For instance, if the porous material is a thin layer tightly bound to a rigid support layer, then it can only expand in the direction away from this support layer. Any relative volume change is now equal to the relative change in any distance measure in the direction of free expansion. We introduce the parameter $\alpha$ which describes this dimensionality of the geometry under study, ranging from $\alpha\!=\! 1$ for expansion in only one free direction, to $\alpha \! = \! 3$ for free expansion in three directions. We define the chain stretching degree, $\widetilde{x}$, as the actual distance between chain end-points (nodes), $x$, divided by the contour length, $\ell$, i.e., $\widetilde{x}=x/\ell$, and we then arrive at 
%
\begin{equation}
{\widetilde{x}}^{\text{~}\alpha} =\phi\s{min} / \phi 
\label{eq_x_to_eta_min}
\end{equation} 
%
a result we will extensively use later on.\footnote{For free expansion in three dimensions, what we wrote is correct, that all distances in a material increase relatively by one-third of how the volume changes. However, for $\alpha\!=\!1$ or $\alpha\!=\!2$, this is only an approximation. As an example, for $\alpha\!=\!1$ a material can only expand in one specific direction, \textit{x}, and is constrained in the other directions (for instance because the material is a thin layer with large area that by chemical bonds is firmly attached to a supporting structure), and only chains oriented in the \textit{x}-direction expand relatively by the same extent as the macroscopic expansion of the layer. However, most chains are oriented in other directions, and they will expand less. Thus we arrive at a distribution of stretching degrees. This distribution can be included in the analysis but in the present work we neglect this complicating factor. It is also assumed in our theory that all chains are equally long (along their backbone), and that there are no topological constrictions (entanglements) in the network, but all chains can expand or contract freely without such restrictions.} When the chains are fully stretched and thus $\widetilde{x}\!=\! 1$, the distance between nodes, \textit{x}, equals the contour length, $\ell$. In this section we first discuss the theory that we propose, and at the end of this section describe the Flory-Rehner theory based on the explanation in ref.~\cite{Doi_1996}.

To calculate the density of a porous cross-linked polymer material, we must analyze several pressures, and at mechanical equilibrium they add up to zero. These pressures are due to volumetric interactions between polymer segments, $\Pi\s{exc}$, where index `exc' refers to `excess' or `excluded volume', are due to a polymer-solvent interaction, $\Pi\s{aff}$, and are due to a pressure exerted on the network because of elasticity, $\Pi\s{el}$.
In the absence of solutes (such as ions) in the solvent, the pressure due to volumetric interactions between the polymer chains is~\cite{Biesheuvel_2008, Spruijt_2014, Spruijt_2015, Biesheuvel_Dykstra_2020}
%
\begin{equation}
\Pi\s{exc} \cdot \frac{v\s{b}}{k\s{B}T} = \frac{\phi^2\left(3-\phi^2\right)}{\left(1-\phi\right)^3}  \sim 3 \phi^2 + 9 \phi^3 + 17 \phi^4 + \dots
\label{eq_bmcsl_sb}
\end{equation}
%
where $k\s{B} T$ is the thermal energy, which at room temperature is $k\s{B}T \! = \!  4.11\cdot 10^{-21}$~J. Eq.~\eqref{eq_bmcsl_sb} is a very successful modification of the Carnahan-Starling equation-of-state for hard sphere mixtures, extended to the situation that the spheres in such a mixture form long or connected chains~\cite{Biesheuvel_2008,Spruijt_2014,Spruijt_2015,Biesheuvel_Dykstra_2020}. The polymer also has chemical interactions with the solvent, i.e., an affinity, and when water is the solvent this is often described by the terminology of hydrophilicity and hydrophobicity. Its contribution to the %chemical potential of polymer, this energy scales with $\phi$, and as contribution to the 
osmotic pressure is given by~\cite{Biesheuvel_2008, Spruijt_2014, Spruijt_2015, Biesheuvel_Dykstra_2020}
%
\begin{equation}
\Pi\s{aff} \cdot \frac{v\s{b}}{k\s{B}T} = - \chi \phi^2
\label{eq_Pi_affinity}
\end{equation} 
%
where we make use of the index `aff' for affinity. The interaction parameter $\chi$ is positive when solvent-polymer interaction is unfavourable (and the two materials prefer being demixed, and a polymer network would not swell), %i.e., in the case of water as solvent, the material is called hydrophobic. V
and vice-versa, a negative value of $\chi$ implies %the material is hydrophilic (if water is the solvent), and water and 
that solvent and polymer like to be mixed on the molecular scale. 

At equilibrium (i.e., without an external force and in the absence of water flowing through the material), these two forces are exactly compensated by a contractive elastic pressure, which we discuss next. This pressure originates from the fact that a polymer chain consists of many segments that we assume are freely jointed. These are also called Kuhn segments. The angle/orientation of each segment is unrelated to that of its neighbours, leading a configuration similar to a random walk in three dimensions. With one chain end fixed at a certain position, $\mathcal{O}$, and the other chain end located somewhere along a line $\mathscr{L}$ that starts in $\mathcal{O}$, the number of configurations to find the other end at a certain distance along this line $\mathscr{L}$ decreases with \textit{x},  making these configuration less likely. This translates into a contractive force between the two chain ends (between the nodes) which pulls the entire network closer. 
As we discuss next, this force increases with distance \textit{x}, and diverges when \textit{x} approaches the contour length, $\ell$.

The Langevin equation describes the force, $\widetilde{f}$, by which the two chain ends pull on each other as function of stretching degree, $\widetilde{x}$, and is given by~\cite{Doi_1996,Biesheuvel_2004,Jedynak_2015}
%
\begin{equation}
\widetilde{x}=\frac{1}{k \, \widetilde{f}}- \frac{1}{\tanh \left( k \, \widetilde{f} \right)}
\label{eq_Langevin}
\end{equation}
%
where $k$ is the aforementioned Kuhn length. A shorter Kuhn length implies that we have more Kuhn segments in a given chain, and thus the polymer material is more difficult to stretch. For low $\widetilde{x}$, thus low force $\widetilde{f}$, Eq.~\eqref{eq_Langevin} simplifies to %an expression for linear stretching, given by % I removed Hookean, 
%
\begin{equation}
- k \widetilde{f} = - \frac{k \, f}{k\s{B}T} = 3  \cdot \frac{x}{\ell}  = 3  \cdot  \widetilde{x} 
\label{eq_Hooke}
\end{equation}
%
with the minus-sign indicating a contractive force. There are no exact solutions to the inversion of Eq.~\eqref{eq_Langevin} (expressing force explicitly as function of stretching degree) but the ``Cohen exact Padé approximation [3/2]'' has the prefactors up to $\widetilde{x}^5$ correct, resulting in~\cite{Jedynak_2015}
%
\begin{equation}
 - {k \, \widetilde{f}} \sim \widetilde{x} \, \frac{3 - \frac{36}{35} \widetilde{x}^2 }{1- \frac{33}{35} \widetilde{x}^2 } \sim 3 \, \widetilde{x} + \frac{9}{5} \, \widetilde{x}^3  +  \frac{297}{175} \, \widetilde{x}^5 + \mathcal{O}\left(\widetilde{x}^7 \right) 
\label{eq_Pade}
\end{equation}
%
with the prefactor for $\widetilde{x}^7$ an underestimate of the exact result by 10\%, and by 20\% for $\widetilde{x}^9$. The ``rounded'' version of this equation is
%
\begin{equation}
 - {k \, \widetilde{f}} \sim \widetilde{x} \, \frac{3 - \widetilde{x}^2 }{1-  \widetilde{x}^2 }= 3 \, \widetilde{x} + 2 \,  \widetilde{x}^3  +  2 \, \widetilde{x}^5 + \mathcal{O}\left(\widetilde{x}^7 \right) 
\label{eq_Pade_rounded}
\end{equation}
%
which overestimates the prefactors from $\widetilde{x}^3$ onward, by 10\% for $\widetilde{x}^3$, 18\% for  $\widetilde{x}^5$, and 25\% for $\widetilde{x}^7$, so also this approximation is not that far off.

The limit of Eq.~\eqref{eq_Langevin} for $\widetilde{x} \rightarrow 1$ is~\cite{Naji_2003}
%
\begin{equation}
- k \widetilde{f} = \frac{1}{1-\widetilde{x}}
\label{eq_Langevin_high_stretching_limit}
\end{equation}
%
and this equation is very close to Eq.~\eqref{eq_Langevin} for stretching degrees beyond 50\%. 

The elastic energy of a single chain is 
%
\begin{equation}
U\s{el} =  - \int_0^x f \text{d} x
\label{eq_definition_elastic_energy}
\end{equation}
%
which in case of the weak stretching regime, Eq.~\eqref{eq_Hooke}, %for the Hookean term 
results in
%
\begin{equation}
U\s{el} = k\s{B}T \cdot \frac{\ell}{k} \cdot \frac{3}{2} \cdot \widetilde{x}^2 \, .
\label{eq_Hookean_energy}
\end{equation}
%
When we also include the higher order terms, we have~\cite{Biesheuvel_2004}
%
\begin{equation}
{U\s{el}} = {k\s{B}T} \cdot \frac{\ell}{k} \cdot \sum_i a_i \cdot \widetilde{x}^{2 i}
\label{eq_elastic_higher_order} 
\end{equation}
%
where the counter \textit{i} runs from 1 to $\infty$, and where the prefactors up to $\widetilde{x}^8$ are $a_1=3/2$, $a_2=9/20$, $a_3=99/350$, and $a_4=1539/7000$. 

\noindent For high stretching ($\widetilde{x}>0.5$), Eq.~\eqref{eq_Langevin_high_stretching_limit} can be integrated, which leads to~\cite{Naji_2003, Biesheuvel_2004}
%
\begin{equation}
U\s{el}=- \, k\s{B}T\cdot \frac{\ell}{k} \cdot \left(\ln\left(1-\widetilde{x} \right)+f_1\right)
\end{equation}
%
where the factor $f_1$ is given by $f_1 \sim 0.307$.

To go from an energy per chain to an elastic pressure, $\Pi\s{el}$, we must evaluate
%
\begin{equation}
\Pi\s{el} = - \frac{\partial \, V \! N U\s{el}  }{\partial \, V} = - V \! N \,   \frac{\partial U\s{el}}{\partial V}
\label{eq_osm_pr_definition}
\end{equation}
%
where $V$ is volume. We can treat the term $V \! N$ as a constant because the differentiation is performed under the constraint of a constant amount of polymer (in the changing volume), and the amount of polymer is proportional to $V \! N$. The elastic pressure, $\Pi\s{el}$, is a force that pulls the material closer together, i.e., it is a contractive force, and thus it is a negative quantity. 

To solve Eq.~\eqref{eq_osm_pr_definition}, we use the approach discussed before Eq.~\eqref{eq_x_to_eta_min}, that any relative change in \textit{V} equals %$\text{d} \ln V$ is 
the geometry factor $\alpha$ times the relative change in any linear dimension $x$, thus $\text{d}\ln V = \alpha \cdot \text{d}\ln x$,  with $\alpha \! = \! 1$ when the layer is rigidly attached to a support layer and can only expand in the direction away from that support layer, and $\alpha \! = \! 3$ is for a material that is free to expand in all directions. Including that conversion, %Eq.~\eqref{eq_x_to_eta_min}, 
Eq.~\eqref{eq_osm_pr_definition} changes to
%
\begin{equation}
\Pi\s{el}  = - N \cdot \frac{x}{\alpha} \cdot \frac{\text{d} U\s{el}}{\text{d} x}  = - N \cdot \frac{\widetilde{x}}{\alpha} \cdot \frac{\text{d} U\s{el}}{\text{d} \widetilde{x}} =  N \cdot \frac{x}{\alpha} \cdot f = \frac{\sigma\s{b}}{k} \cdot \frac{\widetilde{x}}{\alpha} \cdot \frac{\phi}{v\s{b}} \cdot k\widetilde{f} \cdot k\s{B}T
\label{eq_osm_pr_definition_2}
\end{equation}
%
where we also included Eq.~\eqref{eq_definition_elastic_energy}. Thus the elastic pressure is the contractive force per chain, \textit{f}, times the number of chains per unit volume, \textit{N}, and times the stretching degree, \textit{x}, and divided by the geometry factor $\alpha$. This is a very interesting result, that $\Pi\s{el}$ is also proportional to stretching degree, but not a function of the energy of stretching, $U\s{el}$.\footnote{For free expansion in three dimensions ($\alpha\!=\!3$), if we expand the exact result for $\Pi\s{el}$ we obtain the result that the first term (starting at high $\phi$) is proportional to $\phi$ to the power 1/3, as also discussed in ref.~\cite{Katchalsky_1954}. Also in ref.~\cite{Katchalsky_1954}, % (again for expansion in three dimensions) 
higher order terms can be derived from a footnote on p.~23, and are $\phi$ to the power -1/3, -1, etc. This is also what we find for $\alpha\!=\!3$. Instead, for $\alpha\!=\!1$, Taylor expansion of the elastic pressure leads to terms $\phi$ to the power -1, -3, -5, etc.} 

For low levels of stretching, we can implement Eqs.~\eqref{eq_x_to_eta_min}~and~\eqref{eq_Hooke} in Eq.~\eqref{eq_osm_pr_definition_2}, which leads to
%
\begin{equation}
\frac{\Pi\s{el}}{k\s{B}T} = - N \cdot \frac{\ell}{k} \cdot \frac{3}{\alpha} \cdot \widetilde{x}^2 = - N \cdot \frac{\ell}{k} \cdot \frac{3}{\alpha} \cdot \left(\frac{\phi\s{min}}{\phi}\right)^{2/\alpha}  = - \frac{1}{v\s{b}} \cdot \frac{\sigma\s{b}}{k} \cdot \frac{3}{\alpha} \cdot \phi\s{min}^{2/\alpha} \cdot \phi^{1-2/\alpha} 
\label{eq_osm_pr_hookean}
\end{equation}
%
where we made use of $N  \ell = \sigma\s{b}  c\s{b}$ and $ c\s{b} = \phi / v\s{b}$. For expansion in all directions ($\alpha\!=\!3$), Eq.~(15) in ref.~\cite{Katchalsky_1954} gives $\Pi\s{el}$ as function of particle volume to the power --1/3, in line with Eq.~\eqref{eq_osm_pr_hookean} where $\Pi\s{el}$ depends on $\phi$ to the power 1/3 (for $\alpha\!=\!3$). 

In general, at each position in a porous material, we have the force balance~\cite{Van_der_Sman_2015}
%
\begin{equation}
P^\text{c} = \Pi\s{exc} + \Pi\s{aff} + \Pi\s{el} 
\label{eq_polymer_force_balance}
\end{equation}
%
where $P^\text{c}$ is the compression pressure. When the material is changing its shape or volume in time (it is shrinking or swelling), or when there are flows through the material, or when it is pushed against a surface, in all these cases $P^\text{c}$ will not be zero. But when the shape of the material is unchanging and there are no flows or external forces, then $P^\text{c}=0$, and we have the force balance for mechanical equilibrium
%
\begin{equation}
\Pi\s{exc} + \Pi\s{aff} + \Pi\s{el} = 0 \, .
\label{eq_polymer_force_balance_equilibrium}
\end{equation}
%
\noindent In this case, inside the material volume exclusion and affinity are %internally 
exactly compensated by the elastic force that pulls the material inward. Eq.~\eqref{eq_polymer_force_balance_equilibrium} is different from the criterion for equilibrium of swelling used by Flory and Rehner~\cite{Flory_Rehner_1943}, that ``At equilibrium with pure solvent $\Delta \overline{F}_1=0$.'' That we must evaluate a balance of pressures, with the pressure obtained from a differentiation of free energy with volume, is expressed in Eqs.~(8)--(13) in ref.~\cite{Katchalsky_1954}, with reference to J. Frenkel (1938). 

A force balance for low stretching degree is obtained from combination of Eqs.~\eqref{eq_bmcsl_sb},~\eqref{eq_Pi_affinity},~and~\eqref{eq_polymer_force_balance_equilibrium}, which results in
%
\begin{equation}
{\phi^2\left(3-\phi^2\right)}{\left(1-\phi\right)^{-3}} - \chi \phi^2 - {3} \cdot {\alpha}^{-1} \cdot \sigma\s{b}\cdot{k}^{-1} \cdot \phi\s{min}^{2/\alpha} \cdot \phi^{1-2/\alpha}       = 0
\label{eq_force_balance}
\end{equation}
%
where we implemented the moderate stretching contribution to the elastic energy given by Eq.~\eqref{eq_osm_pr_hookean}. For any value of $\sigma\s{b}$, $\phi\s{min}$, and \textit{k}, Eq.~\eqref{eq_force_balance} calculates the equilibrium density of the porous material. For a highly porous material (low $\phi$) and free expansion in all directions ($\alpha\!=\!3$), we can solve Eq.~\eqref{eq_force_balance} and obtain the result that the volume of a particle \textit{V} (inversely proportional to the polymer density $\phi$) has a dependence on Kuhn length and interaction energy $\chi$ according to
%
\begin{equation}
V \propto  \left\{ {k} \cdot \left(3-\chi\right) \right\}^{3/5} 
\label{eq_three_five_limit}
\end{equation}
%
where the $\propto$-sign refers to the left side of the equation being proportional to that on the right side. Eq.~\eqref{eq_three_five_limit} shows that --all other things equal-- the material swells %(volume \textit{V} goes up) 
when the Kuhn length $k$ increases, and when the solvent-polymer interaction parameter $\chi$ decreases (become less positive, or more negative), i.e., the material is more solvo-philic. Note that in this simplified model, $\chi$ cannot be larger than $\chi\!=\! 3$, but in the general model we discuss next, this limitation is not there.

In a more general model based on Eq.~\eqref{eq_polymer_force_balance_equilibrium} we no longer use the moderate stretching limit, but use the exact Langevin equation for $\widetilde{f}$. The force balance is now a set of equations that must be solved simultaneously, given by 
%
\begin{equation}
\frac{\phi^2\left(3-\phi^2\right)}{\left(1-\phi\right)^{3}} - \chi \cdot \phi^2 + \frac{\sigma\s{b}}{k} \cdot \phi \cdot \frac{\widetilde{x}}{\alpha} \cdot 
k\widetilde{f} = 0  \hspace{2mm}, \hspace{2mm} \widetilde{x} =\frac{1}{k  \widetilde{f}}- \frac{1}{\tanh \left( k  \widetilde{f} \right)} \hspace{2mm}, \hspace{2mm} \widetilde{x}^{\text{~}\alpha} = \frac{\phi\s{min}}{\phi}  \,  .
\label{eq_force_balance_gen}
\end{equation}

\noindent The three unknowns in these three equations are the polymer volume fraction $\phi$, the elastic force times Kuhn length, $k\widetilde{f}$, and the stretching degree $\widetilde{x}$. Input parameters in the equations are the attraction parameter $\chi$, the geometry factor $\alpha$, the ratio of bead size over Kuhn length $\sigma\s{b}/k$, and the minimum polymer density $\phi\s{min}$. It is possible to write Eq.~\eqref{eq_force_balance_gen} as an % single 
implicit equation with $\phi$ the only unknown, after which $\widetilde{x}$ and $k\widetilde{f}$ follow directly, but that is a very unwieldy expression. Eq.~\eqref{eq_force_balance_gen} includes finite chain stretching, an accurate equation-of-state for polymer volume exclusion, and it can be used for linear ($\alpha \! = \! 1$), areal ($\alpha\! = \! 2$), and volumetric expansion ($\alpha \! = \! 3$). %

In Fig.~\ref{fig_1} we show how the porosity, $p$, of the polymer ($p=1-\phi$) increases when the material becomes more solvo-philic (attraction parameter $\chi$ more negative, which is to the right in the figure). On the very right we reach the maximum porosity and thus all polymer chains in the material are fully stretched, which is the situation we reach when the solvent-polymer interaction is very favourable. We then reach the set value for the minimum polymer volume fraction, which in this calculation is $\phi\s{min}=0.10$, and thus the maximum porosity is $p\s{max} = 0.90$. Starting left, when the material is solvo-phobic and in a collapsed state, when we make the material less solvo-phobic and move to the right in the figure ($-\chi$ increasing), at some point the material starts to expand from a porosity around 30 vol\%, to quickly reach porosities near 90 vol\%.

% Figure environment removed

\begin{framed}
\underline{Flory-Rehner (FR) theory}. A very clear explanation of the FR theory for the swelling of an hydrogel particle is presented in section 3.4 of the book by Doi and See~\cite{Doi_1996}. The balance of forces is that of elasticity and of mixing. The force due to mixing is a combination of the statistics of lattice occupation (volume exclusion), and of solvent-polymer interaction. The pressure due to mixing is given by Eq.~(2.25) in ref.~\cite{Doi_1996} (see also, for instance Eq.~(2) in ref.~\cite{Van_der_Sman_2015})
%
\begin{equation}
\Pi\s{mix}= - \frac{k\s{B}T}{v\s{c}} \left[ \ln\left(1-\phi\right)+\phi+\chi\phi^2\right] \sim \frac{k\s{B}T}{v\s{c}} \left[ \left( \frac{1}{2}-\chi \right)\phi^2 +\frac{1}{3} \phi^3 + \dots \right]
\label{eq_Pi_mix_Doi}
\end{equation}
%
where we included $N \rightarrow \infty$, and where $v\s{c}$ is the volume of a lattice site.

\noindent For moderate stretching the elastic energy of a certain gel particle is given by Eq.~(3.70) in ref.~\cite{Doi_1996}
%
\begin{equation}
A\s{el,FR}= \frac{3}{2} \cdot n\s{c} \cdot k\s{B}T \cdot \left(\frac{\phi_0}{\phi}\right)^{2/3}
\label{eq_Doi_A_el}
\end{equation}
%
where $n\s{c}$ is the number of chains in the particle, and $\phi_0$ is the polymer volume fraction `before expansion'. The dependence on a density \textit{before expansion} raises many questions: how can we know this density? %but this is incorrect and makes no sense. What is then this polymer density before expansion? 
Because to calculate this $\phi_0$, we would need the theory that we are presenting here, which would then require another $\phi_0$, etc. Luckily, this problem will not arise, because the correct expression for the elastic energy in this limit is that this $\phi_0$ must be the minimum polymer density, $\phi\s{min}$, which is found at \textit{maximum} expansion, when all chains are stretched completely. %$x \rightarrow \ell$, thus $\widetilde{x}\rightarrow 1$. 
Indeed, when we combine Eq.~\eqref{eq_x_to_eta_min} and Eq.~\eqref{eq_Hookean_energy}, and multiply by $n\s{c}$, we obtain
%
\begin{equation}
A\s{el} = \frac{3}{2} \cdot n\s{c} \cdot k\s{B}T \cdot \frac{\ell}{k} \cdot \left(\frac{\phi\s{min}}{\phi_0}\right)^{2/3}
\label{eq_A_el_comp_DOI}
\end{equation}
%
which differs from Eq.~\eqref{eq_Doi_A_el} in the aforementioned difference between $\phi_0$ and $\phi\s{min}$, and in a term $\ell / k$ which is absent in Eq.~\eqref{eq_Doi_A_el}. 

\noindent If we derive an elastic pressure based on Eq.~\eqref{eq_Doi_A_el}, making use of $\Pi= - \partial A\s{el} / \partial V$ (while $n\s{c}$ is taken as a constant) with $V \phi = V_0 \phi_0 $, we obtain 
%
\begin{equation}
\Pi\s{el,FR}= - \frac{n\s{c}}{V_0} \cdot k\s{B}T \cdot \left(\frac{\phi}{\phi_0} \right)^{1/3}
\label{eq_Pi_el_Doi}
\end{equation}
%
whereas Eq.~\eqref{eq_A_el_comp_DOI} leads to
%
\begin{equation}
\Pi\s{el} = - \frac{n\s{c}}{V\s{max}} \cdot \frac{\ell}{k} \cdot {k\s{B}T}  \cdot \left(\frac{\phi}{\phi\s{min}}\right)^{1/3} 
\label{eq_Pi_el_comp_DOI}
\end{equation}
%
where we replaced $\phi\s{min}/v\s{b}$, which is the polymer bead concentration at minimum density, with $n\s{b} / V\s{max}$, where $n\s{b}$ is the total number of beads in a certain particle. For this $n\s{b}$ we have the equality $n\s{b}=n\s{c} \cdot \ell / \sigma\s{b}$, which is also implemented in Eq.~\eqref{eq_Pi_el_comp_DOI}. So we have the same dependence on polymer density to the power 1/3 in Eqs.~\eqref{eq_Pi_el_Doi} and~\eqref{eq_Pi_el_comp_DOI}, but at three points there are differences in the two prefactors. 

\noindent In ref.~\cite{Doi_1996}, Eqs.~\eqref{eq_Pi_mix_Doi} and~\eqref{eq_Pi_el_Doi} are now added together and this total pressure set to zero, which results in Eq.~(3.74) in ref.~\cite{Doi_1996}, which is the classical Flory-Rehner result
%
\begin{equation}
\phi + \ln\left(1-\phi\right) +  \chi \phi^2 + \nu\s{c} \left(\frac{\phi}{\phi_0}\right)^{1/3} = 0
\label{eq_mech_balance_Doi}
\end{equation}
% 
where the prefactor is $\nu\s{c}= n\s{c}/V_0\cdot v\s{c}$, with $v\s{c}$ the volume per lattice site, see Eq.~(3.72) in~ref.~\cite{Doi_1996}. 

If we combine Eq.~\eqref{eq_Pi_mix_Doi} with the correct expression for elastic pressure, Eq.~\eqref{eq_Pi_el_comp_DOI}, we obtain an improved Flory-Rehner equation, given by
%
\begin{equation}
\phi + \ln\left(1-\phi\right) +  \chi \phi^2 + \nu_\text{c}^* \left(\frac{\phi}{\phi\s{min}}\right)^{1/3} = 0
\label{eq_mech_balance_FR_improved}
\end{equation}
% 
which is different from Eq.~\eqref{eq_mech_balance_Doi} because we do not use the polymer density \textit{before expansion}, $\phi_0$, but the polymer density at \textit{maximum expansion} (when all chains are stretched completely), $\phi\s{min}$, and the new prefactor also includes a dependence on Kuhn length, \textit{k}, and chain length, $\ell$, resulting in $\nu_\text{c}^* = n\s{c} / V\s{max} \cdot \ell / k \cdot v\s{c} $.

But a further improvement of FR-theory would be to replace the lattice statistics by an accurate equation of state for long polymer chains, and thus use the last part of Eq.~\eqref{eq_osm_pr_hookean} for $\Pi\s{el}$ which then results in Eq.~\eqref{eq_force_balance} for $\alpha\!=\!3$, and thus we arrive at
%
\begin{equation}
- \phi^2\cdot \frac{3-\phi^2}{\left(1-\phi\right)^3} + \chi \phi^2 +  \nu_\text{c}^{\dagger} \cdot \left( \frac{\phi}{\phi\s{min}}  \right)^{1/3} = 0
\label{eq_mech_balance_FR_improved_again}
\end{equation}
%
where $\nu_\text{c}^{\dagger} =  n\s{c} / V\s{max} \cdot \ell / k \cdot v\s{b} $. 

Finally, we can remove the limitation of considering only weakly stretched chains, by replacing $\Pi\s{el}$ from Eq.~\eqref{eq_Pi_el_comp_DOI} by an improved expression for the elastic pressure, for which we use here the rounded Padé (rP) approximation, Eq.~\eqref{eq_Pade_rounded}, which accurately describes the elastic force from low and moderate stretching to the infinite force at full stretching. In this case the elastic pressure becomes
%
\begin{equation}
\Pi\s{el,rP} = - \frac{n\s{c}}{V\s{max}} \cdot \frac{\ell}{k} \cdot {k\s{B}T}  \cdot \left(\frac{\phi}{\phi\s{min}}\right)^{1/3} \cdot \frac{1- \nicefrac{1}{3}\cdot \left(\phi\s{min} / \phi  \right)^{2/3}}{1-\left(\phi\s{min} / \phi  \right)^{2/3}}
\label{eq_Pi_el_comp_DOI_rounded_pade}
\end{equation}
%
and the further improved FR-theory is now
%
\begin{equation}
- \phi^2\cdot \frac{3-\phi^2}{\left(1-\phi\right)^3} + \chi \phi^2 +  \nu_\text{c}^{\dagger} \cdot \left( \frac{\phi}{\phi\s{min}}  \right)^{1/3} \cdot \frac{1- \nicefrac{1}{3}\cdot \left(\phi\s{min} / \phi  \right)^{2/3}}{1-\left(\phi\s{min} / \phi  \right)^{2/3}} = 0 \, .
\label{eq_mech_balance_FR_improved_again_and_again}
\end{equation}


\end{framed}

\section{Theory for compression of a porous membrane}

Having set up a theory for the equilibrium swelling of a cross-linked polymer network as discussed in the previous section, we now continue to describe how this equilibrium profile changes if we apply pressure to the fluid phases outside the material. %In this section we assume the liquid is water. 
Of course nothing happens when the material, which we assume is a membrane from this point onward, is placed in a volume of liquid that is equally pressurized on all sides. Instead, we only have an effect of pressure when on one side of the membrane the liquid is pressurized % to a high pressure, relative to a lower pressure 
to a different value than on the other side. Because the membrane is porous and we have continuous open pathways, % for water flow, 
fluid will flow across the material from the side of high hydrostatic pressure to the side of low pressure. Thus inside the membrane the hydrostatic pressure gradually decreases in a direction \textit{x}. At the same time the compression pressure, which is the pressure by which external forces push on the polymer, will go up in that same \textit{x}-direction, and thus the polymer network is increasingly compacted. We present theory to calculate by how much the material is compacted, dependent on where we are within the layer. % (because how would it otherwise ha

After a change in applied pressure (the pressure between the two liquid phases on the two sides of the membrane), for a period of time there are changes in the polymer density profile, and fluid is flowing in or out of the material. After some time all these changes have ceased and there is no further time-dependence of the polymer density profile, $\phi\left(x\right)$, and we reach steady state. This is the situation we will consider. In steady state we have a constant volumetric solvent flux, $J\s{s}$, i.e., $J\s{s}$ is the same at each cross-section in the membrane. According to Darcy's equation, solvent flux relates to the local hydrostatic pressure gradient according to
%
\begin{equation}
J\s{s} = - \frac{k_i}{\mu} \frac{\partial P^\text{h}}{\partial x}
\label{eq_Darcy}
\end{equation}
%
where $\mu$ is the viscosity of the solvent (unit Pa.s), and $x$ is the coordinate across the membrane. The permeability is $k_i$ (with unit m\textsuperscript{2}) and we use the Carman-Kozeny equation to describe the dependence of $k_i$ on porosity, \textit{p}. The Carman-Kozeny equation is based on the concept that the material is made up of a large assembly of spherical particles, and is given by
%
\begin{equation}
k_i =  \frac{1}{180} \cdot \frac{p^3}{\left(1-p\right)^2}\cdot \sigma_\s{CK}^2 = \frac{1}{180} \cdot \left(1 +  2  p + 3 p^2  + \dots   \right) \cdot p^3 \cdot \sigma_\s{CK}^2 
\label{eq_Carman}
\end{equation}
%
where $\sigma_\s{CK}$ is the size of the spheres used to describe the permeability of the porous medium, conceptually comparable to the $\sigma\s{b}$ introduced in the previous section, but we do not have to set these two parameters to the same numerical value. In Ch.~8 of ref.~\cite{Biesheuvel_Dykstra_2020} it was concluded that Eq.~\eqref{eq_Carman} is accurate in the range of porosities $0\!<\!p\!<\!0.35$. A modification of Eq.~\eqref{eq_Carman} was proposed there that is accurate up to $p\!=\!0.93$, by replacing one of the $\left(1\!-\!p\right)$-terms in Eq.~\eqref{eq_Carman} by $\left(1\!-\nicefrac{4}{5} {p}\right)$. However, in the present work we use the original Carman-Kozeny equation. Note that $J\s{s}$ is the solvent flux, or velocity, defined per unit projected area, i.e., a superficial velocity. It is not an interstitial (interpore) velocity, $v\s{s,int}$, which --for the same $J\s{s}$-- increases when porosity is lower, because \mbox{$J\s{s} = p \, v\s{s,int} $}. 

The compression pressure relates to the hydrostatic pressure according to 
%
\begin{equation}
\frac{\partial P^\text{h}}{\partial x}+\frac{\partial P^\text{c}}{\partial x}=0
\label{overall_pressure_balance_1}
\end{equation}
%
which can be integrated, and then expresses that the decrease in hydrostatic pressure from $x\!=\!0$ (the upstream, high pressure, side of the membrane) equals the compression pressure, because at $x\!=\!0$ we have $P^\text{c}\!=\!0$. Indeed, the outside surface of the material facing the high pressure liquid phase, is not influenced by an applied pressure; the compression pressure is still zero. There is the high hydrostatic pressure but that acts on it just as much from the outside as 1~$\AA$ inside the material, so a high hydrostatic pressure as such does not compact a liquid-filled material. 

At the upstream side of the material the material is not compacted because we have zero compression pressure, and then the calculation of the last section applies, which then results in $\left.\phi\right|_{x=0}=\phi\s{eq}$, where index `eq' refers to the mechanical equilibrium condition that applies on the upstream side of the membrane. The deeper we go into the membrane (following the coordinate axis \textit{x}), the material is increasingly compacted. This is because polymer material located between 0 and \textit{x} pushes on the polymer at position \textit{x}, and this force increases with \textit{x} because solvent flows through the polymer network and exerts a drag force on it. Thus, in the direction of solvent flow the compression pressure increases and the polymer is increasingly compressed.  

So how does knowledge of the profile of $P^\text{c}\left(x\right)$ lead to the polymer density profile, $\phi\left(x\right)$, and porosity profile, $p\left(x\right)$? To this end we solve the force balance Eq.~\eqref{eq_polymer_force_balance} at each position \textit{x} in the membrane, together with Eqs.~\eqref{eq_force_balance_gen}--%Eqs.~\eqref{eq_Darcy}--
\eqref{overall_pressure_balance_1}. 

Finally we must consider that the calculation is for a given amount of polymer, so starting with an membrane at equilibrium with density $\phi\s{eq}$ and thickness $L\s{eq}$, this total amount is conserved when the material is compressed, and thus in the calculation we have the constraint
%
\begin{equation}
\int_0^L \phi  \,\text{d}x=\phi\s{eq} \, L\s{eq}
\end{equation}
%
where \textit{L} is the thickness of the layer during operation, when the material is being compressed.

We make a calculation for transport through a membrane of the material that we discussed in Fig.~\ref{fig_1}, and we assume $\chi \!= \! 10$, for which the equilibrium porosity is $p\s{eq}\!=\!0.645$, and thus $\phi\s{eq} \! = \! 0.355$. 
Calculation results are presented in Fig.~\ref{fig_2} as function of position in the membrane after the membrane is compressed to a thickness of $L=200$~$\mu$m. Before compression it had a thickness of $L\s{eq} = 356\text{~}\mu$m. The sphere diameter for the Carman-Kozeny equation is set to the same value as the bead size used in the force balance, Eq.~\eqref{eq_force_balance_gen}, thus $\sigma_\s{CK}=\sigma\s{b}$. We apply a solvent flux across the membrane of $J\s{s}=10$~L/m\textsuperscript{2}/hr (LMH) and viscosity is $\mu\!=\! 1$~mPa.s. The calculation clearly shows that porosity decreases from the high pressure side to the low pressure side of the membrane, while the compression pressure strongly increases toward the downstream side of the layer. As can be noted from the pressures involved, this calculation output is not likely found in any experimental system: we are not going to push solvent through such a membrane at these high pressures. The question is whether interesting density profiles can be encountered for more realistic experimental conditions.

% Figure environment removed

To resolve that question, we make calculations for a realistic reverse osmosis (RO) membrane. We varied parameter settings in a large range but arrived at similar results in each case. The result we present is for a membrane of thickness $L\s{eq}\!=\!150$~nm. Compared to the earlier calculations, we increase the interaction energy to $\chi\!=\!30$ (more solvo-phobic), and use a minimum polymer density of $\phi\s{min}\!=\!0.40$, but otherwise keep all parameters the same. The equilibrium polymer density is now $\eta\s{eq}\!=\!0.572$. The solvent permeability for this condition is $A\!=\! 3.20$~LMH/bar, a very realistic number for brackish water RO membranes. We report in Fig.~\ref{fig_3} results for a realistic transmembrane solvent flux of $J\s{s}\!=\!50$~LMH. If the membrane would not be compressed, the required pressure is then $\Delta P^\text{h}\!=\! 15.63$~bar. We find in this calculation a slight decrease of porosity from~0.428 to~0.426. Because of this slight compression, the applied pressure is not~15.63 bar, but $\Delta P^\text{h}\!=\!15.77$~bar. These result point to the fact that the very thin membranes (selective layer) that are presently applied in the RO process for water desalination do not compress much, in line with results reported in ref.~\cite{Davenport_2020}. This conclusion is in line with analysis of multiple data sets of water flux against pressure at many salt concentrations, that for the same membrane can always be described by one unique value of water permeability~\textit{A}~\cite{Biesheuvel_2023}. If the membrane would significantly compact at higher pressures, a good fit of these data would not be possible with a constant value of \textit{A}. Thus, we expect that for the very thin RO membranes currently in use, there is not a significant change of porosity across the thickness of the membrane.

% Figure environment removed


\section{Conclusions}

We presented a novel theory for the equilibrium swelling of a cross-linked porous  polymer material, and extended it to describe the density profile when solvent is pushed through a planar layer of such a material, i.e., a membrane. The theory includes polymer volume self-exclusion, polymer-solvent affinity, and the elastic stretching forces because of the cross-linking of the polymer network. When solvent is pushed through a membrane, a compression pressure develops, increasing in the direction of water flow, which compresses the polymer network. When compressed, the porosity goes down and solvent-polymer friction up, and thus the compression pressure increases even faster, leading to even more compression. All of these effects are included in the theory. 

A calculation for conditions representative of state-of-the-art membranes for water desalination using reverse osmosis (RO) predicts a very small decrease in porosity. Thus, the flow of water across an RO membrane does not have a marked effect on the porosity of such an ultrathin membrane. This is in line with observations that water flux and pressure are proportional for RO membranes up to very high pressures such as 50 bar. Thus, the presented theory predicts that across very thin RO membranes a profile in porosity develops, but the change in porosity across this layer is too small to have an effect on the pressure-driven flow of water across such membranes. 


\begin{thebibliography}{100}

\bibitem{Horkay_2007}
F. Horkay and G.B. McKenna, ``Polymer Networks and Gels,'' Ch.~29 in \textit{Physical Properties of Polymers Handbook}, J.E. Mark (Ed.) Springer (2007).

\bibitem{Davenport_2020}
D.M. Davenport, C.L. Ritt, R. Verbeke, M. Dickmann, W. Egger, I.F.J. Vankelecom, and  M. Elimelech, ``Thin film composite membrane compaction in high-pressure reverse osmosis,'' \textit{J. Membrane Sci.} \textbf{610}, 118268 (2020).

\bibitem{Flory_Rehner_1943} P.J. Flory and J. Rehner, Jr., ``Statistical mechanics of cross-linked polymer networks,'' \textit{J. Chem. Phys.} \textbf{11}, 521--526 (1943).

\bibitem{Biesheuvel_2008}
P.M. Biesheuvel, W.M. de Vos, and V.M. Amoskov, ``Semianalytical continuum model for nondilute neutral and charged brushes including finite stretching,'' \textit{Macromolecules} \textbf{41}, 6254--6259 (2008).

\bibitem{Spruijt_2014} 
E. Spruijt and P.M. Biesheuvel, ``Sedimentation dynamics and equilibrium profiles in multicomponent mixtures of colloidal particles,'' \textit{J. Phys. Condens. Matter }\textbf{26}, 075101 (2014).

\bibitem{Spruijt_2015}
E. Spruijt, P.M. Biesheuvel, and W.M. de Vos, ``Adsorption of charged and neutral polymer chains on silica surfaces: The role of electrostatics, volume exclusion, and hydrogen bonding,'' \textit{Phys. Rev. E} \textbf{91}, 012601 (2015).

\bibitem{Biesheuvel_Dykstra_2020} 
P.M. Biesheuvel and J.E. Dykstra, \textit{Physics of Electrochemical Processes},  ISBN:9789090332581 (2020). % http://www.physicsofelectrochemicalprocesses.com (2020).

\bibitem{Doi_1996}
M. Doi and H. See, \textit{Introduction to Polymer Physics}, Clarendon Press, Oxford %, United Kingdom 
(1996).

\bibitem{Biesheuvel_2004} 
P.M. Biesheuvel, ``Ionizable polyelectrolyte brushes: brush height and electrosteric interaction,'' \textit{J. Colloid Interface Sci.} \textbf{275}, 97--106 (2004).

\bibitem{Jedynak_2015} 
R. Jedynak, ``Approximation of the inverse Langevin function revisited,'' \textit{Rheol. Acta} \textbf{54}, 29--39 (2015).

%\bibitem{Jia_2021}
%D Jia and M. Muthukumar, ``Theory of Charged Gels: Swelling, Elasticity, and Dynamics,'' \textit{Gels} \textbf{7}, 49 (2021).

\bibitem{Naji_2003} 
A. Naji, R.R. Netz, C. Seidel, ``Non-linear Osmotic Brush Regime: Simulations and mean-field theory,'' \textit{Eur. Phys. J. E} \textbf{12}, 223--237 (2003).

\bibitem{Van_der_Sman_2015}
R.G.M. van der Sman, ``Biopolymer gel swelling analysed with scaling laws and Flory--Rehner theory,'' \textit{Food Hydrocolloids} \textbf{48}, 94--101 (2015).

\bibitem{Katchalsky_1954} A. Katchalsky, ``Polyelectrolyte Gels,'' \textit{Progress in Biophysics and Biophysical Chemistry} \textbf{4}, 1--59 (1954).

\bibitem{Biesheuvel_2023} 
P.M. Biesheuvel, S.B. Rutten, I.I. Ryzhkov, S. Porada, and M. Elimelech, ``Theory for salt transport in charged reverse osmosis membranes: Novel analytical equations for desalination performance and experimental validation,'' \textit{Desalination} \textbf{557}, 116580 (2023).

%\bibitem{Biesheuvel_2022} P.M. Biesheuvel, S. Porada, M. Elimelech, and J.E. Dykstra, ``Tutorial review of reverse osmosis and electrodialysis,'' \textit{J. Membr. Sci.} \textbf{647}, 120221 (2022). % arXiv:2110.07506 (2021).

%\bibitem{Biesheuvel_2023} P.M. Biesheuvel, ``Tutorial on the chemical potential of ions in water and porous materials: electrical double layer theory and the influence of ion volume and ion-ion electrostatic interactions,'' ArXiv:2212.07851 (2023).

%\bibitem{Biesheuvel_2020}  P.M. Biesheuvel, ``The activity coefficient of \textit{z}:1 ionic solutions scales with the cube root of salt concentration,'' Arxiv:2012.12194 (2020).

%\bibitem{Castano_2022} S. Casta{\~n}o Osorio, P.M. Biesheuvel, E. Spruijt, J.E. Dykstra, and A. van der Wal, ``Modeling micropollutant removal by nanofiltration and reverse osmosis membranes: considerations and challenges,'' \textit{Water Research} \textbf{225}, 119130 (2022).

%\bibitem{Einstein} A. Einstein, ``Investigations on the theory of brownian movement,'' Dover (1956). Ed. R. F{\"u}rth, A.D. Cowper. Reproduction of translation into English published in 1926. Collection of papers published between 1905 and 1908.

%\bibitem{De_Groot_1951} S.R. de Groot,~\textit{Thermodynamics of irreversible processes}, North-Holland Publishing Company, Amsterdam (1951). 
	
%\bibitem{Sonin_1976} A.A. Sonin, ``Osmosis and ion transport in charged porous membranes: A macroscopic, mechanistic model,'' in \textit{Charged Gels and Membranes I}, E. Sélégny (Ed.), D. Reidel, Dordrecht, pp. 255--265 (1976).

%\bibitem{Biesheuvel_2011} P.M. Biesheuvel, ``Two-fluid model for the simultaneous flow of colloids and fluids in porous media,'' \textit{J. Colloid Interface Sci.} \textbf{355}, 389--395 (2011).

\end{thebibliography}


	
\end{document}

