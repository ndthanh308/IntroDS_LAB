\section{Related Work}
% please introduce the limitation of existing work here
In this seecion, we review and contrast related work on OOD detection in machine learning.

%%%% unsupervised and supervised OOD detection?
\noindent
\textbf{Single-Modal OOD Detection.} There exists a plethora of works on single-modal OOD detection~\cite{salehiunified} for machine learning. They can be generally categorized into three types: (i) vision OOD detection, (ii) text OOD detection, and (iii) time-series OOD detection~\cite{du2022unknown}. For vision OOD detection, various methods have been developed, including softmax confidence score~\cite{devries2018learning,hein2019relu,hendrycksbaseline,huang2021mos}, energy-based score~\cite{yang2021generalized,sun2021react,sun2022dice}, distance-based method~\cite{sun2022knn,ren2021simple,podolskiy2021revisiting,techapanurak2020hyperparameter,zaeemzadeh2021out,van2020uncertainty}, generative models~\cite{li2022out,xiao2020likelihood}. For instance, Liu et al.~\cite{liu2020energy} proposed an energy-based OOD detection method with theoretical analysis. Sun et al.~\cite{sun2022knn} developed nearest neighbors to improve the flexibility and
generality of OOD detection. For text OOD detection, pre-trained language models~\cite{podolskiy2021revisiting,zhou2021contrastive} are commonly used due to their robustness in identifying OOD samples in natural languages. Other methods, such as data augmentation~\cite{zhan2021out} and contrastive learning~\cite{zhou2022knn,zhou2021contrastive}, have also been developed for OOD detection. Furthermore, some researchers have focused on OOD detection in time-series data~\cite{romero2022outlier,kaur2022codit,georgescu2021anomaly}, where several ML models have been developed for video anomaly detection~\cite{georgescu2021anomaly,wang2018abnormal}. Wang et al.~\cite{wang2018abnormal} combined LSTM with CNN to improve anomaly detection using a spatio-temporal auto-encoder. Li et al.~\cite{li2022context} leveraged generative models to predict middle frames based on past and future frames. However, these methods mainly detect OOD samples using unimodal data, such as images and text. In contrast, we develop a general-purpose model that combines multi-modal data, such as images and textual information, to enhance the performance of OOD detection.


\noindent
\textbf{Multi-Modal OOD Detection.} Some studies \cite{sun2020real, wang2021radar} have adopted multi-modal data to improve the OOD detection accuracy based on deep neural networks (DNNs). Wang et al. \cite{wang2021radar} proposed a multi-modal transformer network that combines Radar and LiDAR data to detect radar ghost targets. Ji et al. \cite{ji2020multi} developed a supervised VAE (SVAE) model that integrates sensor data of multiple modalities to detect an anomalous operation mode of the car. To improve the accuracy of detecting abnormal driving segments, Qiu et al. \cite{qiu2022unsupervised} developed an unsupervised contrastive approach that uses generative adversarial networks to extract latent features from five modalities. More recently, CLIP-based methods~\cite{ming2022delving,esmaeilpour2022zero,fort2021exploring} have been developed to detect OOD samples. Esmaeilpour et al.~\cite{esmaeilpour2022zero} designed a zero-shot OOD detection model based on pre-trained CLIP~\cite{radford2021learning} to detect and generate candidate labels for test images of unknown classes. However, this method heavily relies on a set of candidate labels. To overcome this issue, Ming et al.~\cite{ming2022delving} developed a zero-shot OOD detection method, called Maximum Concept Matching (MCM), based on pre-train CLIP model. While MCM has shown good performance on multi-modal OOD detection, it can only detect visual OOD samples rather than for both an image and its corresponding texture description. Hence, it is not applicable to other OOD scenarios we are exploring in this work, such as scenario 1.

Different from prior works, we develop a general-purpose multi-modal OOD detector that can identify OOD samples arising from three different scenarios in a fine-grained manner. Our proposed method leverages both weakly supervised learning and contrastive learning for improving OOD detection.