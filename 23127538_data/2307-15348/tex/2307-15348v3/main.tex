%%%%%%%% ICML 2024 EXAMPLE LATEX SUBMISSION FILE %%%%%%%%%%%%%%%%%

\documentclass{article}

% Recommended, but optional, packages for figures and better typesetting:
\usepackage{microtype}
\usepackage{graphicx}
\usepackage{subfigure}
\usepackage{booktabs} % for professional tables

% hyperref makes hyperlinks in the resulting PDF.
% If your build breaks (sometimes temporarily if a hyperlink spans a page)
% please comment out the following usepackage line and replace
% \usepackage{icml2024} with \usepackage[nohyperref]{icml2024} above.
\usepackage{hyperref}


% Attempt to make hyperref and algorithmic work together better:
\newcommand{\theHalgorithm}{\arabic{algorithm}}

% Use the following line for the initial blind version submitted for review:
% \usepackage{icml2024}

% If accepted, instead use the following line for the camera-ready submission:
\usepackage[accepted]{icml2024}

% For theorems and such
\usepackage{amsmath}
\usepackage{amssymb}
\usepackage{mathtools}
\usepackage{amsthm}

% if you use cleveref..
\usepackage[capitalize,noabbrev]{cleveref}

%%%%%%%%%%%%%%%%%%%%%%%%%%%%%%%%
% THEOREMS
%%%%%%%%%%%%%%%%%%%%%%%%%%%%%%%%
\theoremstyle{plain}
\newtheorem{theorem}{Theorem}[section]
\newtheorem{proposition}[theorem]{Proposition}
\newtheorem{lemma}[theorem]{Lemma}
\newtheorem{corollary}[theorem]{Corollary}
\theoremstyle{definition}
\newtheorem{definition}[theorem]{Definition}
\newtheorem{assumption}[theorem]{Assumption}
\theoremstyle{remark}
\newtheorem{remark}[theorem]{Remark}


\icmltitlerunning{The curse of isotropy: from principal components to principal subspaces}

\newcommand{\R}{\mathbb{R}}
\renewcommand{\O}{\mathcal{O}}
\newcommand{\T}{^\top}
\newcommand{\N}[1]{\mathcal{N}\left(#1\right)}
\newcommand{\lrp}[1]{\left(#1\right)}
\newcommand{\lrb}[1]{\left[#1\right]}
\newcommand{\lrs}[1]{\left\{#1\right\}}
\newcommand{\diag}[1]{\operatorname{diag}\left(#1\right)}
\newcommand{\tr}[1]{\operatorname{tr}\left(#1\right)}
\DeclareMathOperator*{\argmin}{arg\,min}
\newcommand{\BIC}{\operatorname{BIC}}
\newcommand{\Skew}{\operatorname{Skew}}


\begin{document}

\twocolumn[
\icmltitle{The curse of isotropy: from principal components to principal subspaces}

\icmlsetsymbol{equal}{*}

\begin{icmlauthorlist}
\icmlauthor{Tom Szwagier}{yyy}
\icmlauthor{Xavier Pennec}{yyy}
\end{icmlauthorlist}

\icmlaffiliation{yyy}{Inria -- Université Côte D’Azur, Sophia-Antipolis, France}

\icmlcorrespondingauthor{Tom Szwagier}{tom.szwagier@inria.fr}

\icmlkeywords{Principal Component Analysis, Isotropy, Interpretability, Parsimonious Models, Flag Manifolds}

\vskip 0.3in
]

\printAffiliationsAndNotice{}  % leave blank if no need to mention equal contribution

\begin{abstract}
This paper raises an important issue about the interpretation of principal component analysis. The \textit{curse of isotropy} states that a covariance matrix with repeated eigenvalues yields rotation-invariant eigenvectors. In other words, principal components associated with equal eigenvalues show large intersample variability and are arbitrary combinations of potentially more interpretable components. However, empirical eigenvalues are never exactly equal in practice due to sampling errors. Therefore, most users overlook the problem. In this paper, we propose to identify datasets that are likely to suffer from the curse of isotropy by introducing a generative Gaussian model with repeated eigenvalues and comparing it to traditional models via the principle of parsimony. This yields an \textit{explicit} criterion to detect the curse of isotropy in practice. We notably argue that in a dataset with 1000 samples, all the eigenvalue pairs with a relative eigengap lower than 21\% should be assumed equal. This demonstrates that the curse of isotropy cannot be overlooked. In this context, we propose to transition from fuzzy principal components to much-more-interpretable principal subspaces. The final methodology---\textit{principal subspace analysis}---is extremely simple and shows promising results on a variety of datasets from different fields.
\end{abstract}

The problem of the presence or absence of phase transition is central in statistical mechanics. To prove the existence of phase transition, the standard idea is to define a notion of contour and use \textit{Peierls' argument} \cite{Peierls.1936}. In the usual Ising model \cite{Ising_25}, particles of the system interact only with their nearest-neighbors. On ferromagnetic long-range Ising models \cite{Anderson_Yuval_69}, there is interaction between each pair of spins in the lattice. The Hamiltonian of the model is given formally by
\begin{equation*}
    H(\sigma) = - \sum_{x,y\in \Z^d}J_{xy}\sigma_x\sigma_y,
\end{equation*}
where $J_{xy}=J|x-y|^{-\alpha}$, $J>0$, $\alpha > d$. It is well-known that the Peierls' argument in dimension 2 implies phase transition for Ising models with nearest-neighbors or long-range interactions when $d\geq 2$, using correlation inequalities. For the unidimensional lattice, it was known that short-range models do not present phase transition. In the long-range case, a different behavior was expected depending on the exponent $\alpha$ (see \cite{Kac_Thompson_69}), but the problem was challenging since contours were first created as multidimensional objects.

In dimension $d=1$, phase transition was proved first in 1969 by Dyson \cite{Dyson.69}, for $\alpha \in (1,2)$, by proving phase transition in an auxiliary model and then using correlation inequalities. In 1982, Fr{\"o}hlich and Spencer \cite{Frohlich.Spencer.82} introduced a notion of one-dimensional contours and then applied the Peierls' argument to show phase transition for the critical value $\alpha = 2$. These contours were inspired by the multiscale techniques previously introduced to study the Berezinskii-Kosterlitz-Thouless transition in two-dimensional continuous spin systems \cite{FS81}. Later, Cassandro, Ferrari, Merola and Presutti  \cite{Cassandro.05} extended the contour argument previously available for $\alpha=2$ to exponents $\alpha\in (3-\frac{\ln 3}{\ln 2}, 2)$, with the additional restriction that the nearest-neighbor interaction is strong, i.e.,  ${J(1)\gg 1}$; this restriction was removed for a subclass of interactions in \cite{Bissacot.Endo.18}. Further results were obtained using contour arguments, such as the decay of correlations, cluster expansions, phase transition with random interactions, etc; some references with these results are \cite{ Cassandro.Merola.Picco.17, Cassandro.Merola.Picco.Rozikov.14, Imbrie.82, Imbrie.Newman.88, Johansson.91}. 

In the multidimensional setting ($d\geq 2$), Ginibre, Grossmann, and Ruelle, in \cite{Ginibre.Grossmann.Ruelle.66}, proved the phase transition for $\alpha > d+1$, using an enhanced version of Peierls' argument and the usual contours. Park proposed a different notion of contour for long-range systems in \cite{Park.88.I, Park.88.II}, extending the Pirogov-Sinai theory available for short-range interactions assuming $\alpha > 3d+1$, although he can also consider Potts models with his methods. Some results in the literature suggest that truly long-range effects appear only when $d < \alpha \leq d+1$, see for instance, \cite{Biskup_Chayes_Kivelson_07}. Recently, Affonso, Bissacot, Endo and Handa \cite{Affonso.2021}, inspired by the ideas from Fr{\"o}hlich and Spencer in \cite{FS81, Frohlich.Spencer.82}, introduced a version of multiscale multidimensional contour and proved phase transition by a contour argument in the whole region $\alpha > d$. They can consider long-range Ising models with deterministic decaying fields, first introduced in the context of nearest-neighbor interactions in \cite{Bissacot_Cioletti_10}. For these models, the lack of analyticity of the free energy does not imply phase transition since these models have the same free energy as the models with zero field. It is expected that fields decaying slowly imply uniqueness. In this setting, a contour argument is useful for proofs of phase transitions as well for uniqueness, some papers with models with deterministic decaying fields are \cite{Aoun_Ott_Velenik_23, Bissacot_Cass_Cio_Pres_15, Bissacot.Endo.18, Cioletti_Vila_2016}.

The Random Field Ising model (RFIM) \cite{Imry.Ma.75} is the nearest-neighbor Ising model with an additional external field acting on each site $(h_x)_{x\in\Z^d}$ that is a family of i.i.d. Gaussian random variable with mean 0 and variance 1. Formally, the Hamiltonian of the model is given by
\begin{equation*}
    H(\sigma) = - \sum_{\substack{x,y\in \Z^d \\|x-y|=1}}J\sigma_x\sigma_y  - \varepsilon\sum_{x\in\Z^d}h_x\sigma_x,
\end{equation*}
where $J>0$, $\varepsilon>0$, $\alpha > d$ and $d \geq 1$. A detailed account of the history of the phase transition problem for this model, as well as detailed proofs, was given in \cite{Bovier.06}. Here we present a brief overview.

During the 1980s, the question of the specific dimension where phase transition for the RFIM should happen attracted much attention and was a topic of heated debate. Two convincing arguments were dividing the physics community. One of them, due to Imry and Ma \cite{Imry.Ma.75}, was a non-rigorous application of the Peierls' argument together with the use of the isoperimetric inequality. The key idea of Peierls' argument is to define a notion of contour and calculate the energy cost of "erasing" each contour, i.e., the energy cost of flipping all spins inside the contour. When there is no external field, that energy necessary to flip the spins in a region $A\subset \Z^d$ is of the order of the boundary $|\partial A|$. When we add an external field, we get an extra cost depending on this field. Imry and Ma argued that this cost should be approximately $\sqrt{|A|}$, which is smaller than $|\partial A|$ for all regions only when $d\geq 3$, so this should be the region where phase transition occurs. The other argument, due to Parisi and Sourlas \cite{Parisi.Sourlas.79}, based on dimensional reduction, predicted that the $d$-dimensional RFIM would behave like the $d-2$-dimensional nearest-neighbor Ising model, therefore presenting phase transition only when $d\geq 4$. 

The question was settled by two celebrated papers showing that Imry and Ma's prediction was correct. First, in 1988, Bricmont and Kupiainen \cite{Bricmont.Kupiainen.88} showed that there is phase transition almost surely in $d\geq3$, for low temperatures and variance $\varepsilon$ small enough. Their proof uses a rigorous renormalization group analysis for the short-range case and it is considered involved. Still, they claimed that the result works for any model with a suitable contour representation and centered sub-gaussian external field. Later on, Aizenman and Wehr \cite{Aizenman.Wehr.90} proved uniqueness for $d\leq 2$. For detailed proofs of these results, we refer the reader to \cite{Bovier.06} (see also \cite{Berretti.85, Camia.18, Frohlich.Imbre.84,  Klein.Masooman.97} for more uniqueness results). 

Recently, Ding and Zhuang, see \cite{Ding2021}, provided a simpler proof of the phase transition, not using RGM. And in  \cite{Ding.Liu.Xia.22}, Ding, Liu and Xia proved that if $\beta_c(d)$ is the critical inverse of the temperature of the Ising model with no field, for all $\beta>\beta_c(d)$ there exists a critical value $\varepsilon_0(d, \beta)$ such that the RFIM with $\varepsilon \leq \varepsilon_0$ presents phase transition. 

In the present paper, we are considering a long-range Ising model with a random field, whose Hamiltonian is given formally by
\begin{equation*}
    H(\sigma) = - \sum_{x,y\in \Z^d}J_{xy}\sigma_x\sigma_y - \varepsilon\sum_{x\in\Z^d}h_x\sigma_x,
\end{equation*}
where $J_{xy}=J|x-y|^{-\alpha}$, $J, \varepsilon>0$, $\alpha > d$ and $h_x\in\mathbb{R}$, $d\geq 3$.
Until now, the only known result in the long-range setting is for the one-dimensional long-range Ising model with a random field, by Cassandro, Orlandi, and Picco \cite{Cassandro.Picco.09}. They used the contours of \cite{Cassandro.05} to show the phase transition for the model when $\alpha\in (3-\frac{\ln 3}{\ln 2}, \frac{3}{2})$, under the assumption $J(1) \gg 1$. We stress that, as remarked by Aizenman, Greenblatt, and Lebowitz \cite{Aizenman_Greenblatt_Lebowitz_2012}, although their argument does not work for the whole region for the exponent $\alpha$, the phase transition holds for values close to the critical value $\alpha=3/2$, since by the Aizenman-Wehr theorem we know that there is uniqueness for $\alpha>3/2$.

The argument from Ding and Zhuang in \cite{Ding2021}, for $d\geq3$, involves controlling the probability of a bad event, which is closely related to controlling the quantity $$\sup_{\substack{0\in A\subset\Z^d \\ A \text{ connected }}}\frac{\sum_{x\in A}h_x}{|\partial A|},$$ known as the greedy animal lattice normalized by the boundary. The greedy animal lattice normalized by the size, instead of the boundary, was extensively studied for general distributions of $(h_x)_{x\in\Z^d}$, see \cite{Cox_Gandolfi_Griffin_Kesten_93, Gandolfi_Kesten_94, Hammond_06, Martin_02}. When we normalize by the boundary, an argument by Fisher, Fr\"{o}hlich and Spencer \cite{FFS84} shows that the expected value of the greedy animal lattice is constant. In dimension $d=2$, the expected value is not finite, see \cite{Ding.Wirth.20}. The supremum is taken over connected regions containing the origin since the interiors of the usual Peierls contours are of this form.


For the long-range model, the interior of contours is not necessarily connected. In fact, long-range contours may have considerably large diameters with respect to their size, so their interiors can be very sparse. To avoid this, we define contours, strongly inspired by the $(M,a,r)$-partition in \cite{Affonso.2021}, using a multiscaled procedure that assures that the contours have no cluster with small density.  With them, we generalize the arguments by Fisher-Fr\"{o}hlich-Spencer \cite{FFS84}, and prove that the expected value of the greedy animal lattice is constant, even considering regions not necessarily connected in the supremum. Then, we prove the phase transition for $d\geq 3$. The main result of this paper is the following.
\begin{theorem*}Given $d\geq 3$, $\alpha>d$, there exists $\beta_c\coloneqq\beta(d, \alpha)$ and $\varepsilon_c\coloneqq\varepsilon(d, \alpha)$ such that, for $\beta >\beta_c$ and $\varepsilon\leq \varepsilon_c$, the extremal Gibbs measures $\mu_{\beta, \varepsilon}^+$ and $\mu_{\beta, \varepsilon}^-$ are distinct, that is, $\mu_{\beta, \varepsilon}^+ \neq \mu_{\beta, \varepsilon}^-$ $\mathbb{P}$-almost surely. Therefore the long-range random field Ising model presents phase transition.
\end{theorem*}

This paper is divided as follows. In Section 2, we define the model and the contours, and suitable generalizations to the constructions in \cite{Affonso.2021} are introduced.  In Section 3, we define two bad events of the external field and prove that they occur with a small probability.  In Section 4, we present the proof of the phase transition.
\section{The curse of isotropy}

Let us consider a dataset sampled independently from a two-dimensional \textit{isotropic} Gaussian distribution. This implies that the eigenvalues of the population covariance matrix are equal. The sample covariance matrix, however, is an approximation of the population covariance matrix, whose accuracy improves with the number of observed samples~\cite{tyler_asymptotic_1981}. Notably, the empirical eigenvalues are almost surely distinct (cf. Thm~\ref{appthm:PSA}). 
Therefore, PCA outputs the unique eigenvectors (up to sign) associated with each eigenvalue.
If we repeat this experiment several times independently and plot the principal components, we get Fig.~\ref{fig:isotropy}.
As we can see, the principal components are evenly spread in all directions---i.e. isotropically.  We call this phenomenon \textit{the curse of isotropy}. It is a curse since it yields principal components with high intersample variability and without any preferred direction. The observed components could therefore be random combinations of \textit{actually} interpretable components.

A legitimate question might then be: \textit{why (and when) should we assume that a given dataset has been sampled from a Gaussian distribution with repeated eigenvalues?} 
The Gaussian assumption is notably justified by the central limit theorem, the entropy maximization and the attractive computational properties that make Gaussian distributions the cornerstone of machine learning generative models~\cite{bishop_pattern_2006}.
Now, regarding the multiple-eigenvalue assumption, we have to go back to one of the founding principles of modeling that is the \textit{law of parsimony}, also known as \textit{Occam's razor}: ``The simplest explanation is usually the best one''. This principle is particularly applied in statistical modeling, where the limited number of observed samples makes overparameterized models overfitting~\cite{myung_counting_2000}. Notably, covariance matrices (which have $\mathcal{O}(p^2)$ parameters) can almost never be correctly estimated in practice, especially in high dimensions. Therefore, more parsimonious models have to be considered, like isotropic Gaussians (which have $1$ parameter---the variance), where all the covariance eigenvalues are equal. 
In the following, we show that a Gaussian model with \textit{repeated eigenvalues}, i.e. isotropic in some multidimensional eigenspaces, has less parameters than one with \textit{distinct eigenvalues} and therefore provides a simpler explanation of the data. Then, using parsimonious model selection criteria such as the BIC, we are able to decide which eigenvalues should be assumed equal.
\section{Identifying the curse of isotropy}
In order to spot the curse of isotropy, we go through the lens of statistical modeling and introduce the PSA generative model. This model assumes a Gaussian distribution with repeated covariance eigenvalues. 
It enjoys an explicit maximum likelihood estimate with a rich geometry enabling effective model selection.


\subsection{PSA model}
Let ${\gamma} := (\gamma_1, \dots, \gamma_d)$ be a \textit{composition} of a positive integer $p$---i.e. a sequence of positive integers that sums up to $p$.
We define the PSA model of \emph{type} ${\gamma}$ as the family of Gaussian distributions $~{p(x | \mu, \Sigma) := \mathcal{N}(x | \mu, \Sigma)}$, where $~{\mu \in \R^p}$ is a mean vector and $~{\Sigma = \sum_{k=1}^d \lambda_k Q_k {Q_k}\T\in S_p^{++}}$ is a covariance matrix with repeated eigenvalues $~{\lambda_1 > \dots > \lambda_d > 0}$ of respective multiplicity $\gamma_1, \dots, \gamma_d$ and associated eigenspaces $\mathrm{Im}(Q_1), \dots, \mathrm{Im}(Q_d)$.
These distributions can be rewritten as a (linear-Gaussian) latent variable generative model
\begin{equation}\label{eq:PSA_model}
{x} = \sum_{k=1}^{d-1} \sigma_k {Q}_k {z}_k + {\mu} + {\epsilon},
\end{equation}
where $~{\sigma_1 > \dots > \sigma_{d-1} > 0}$ are decreasing scaling factors,
${Q}_k \in \R^{p \times \gamma_k}$ are mutually-orthogonal orthonormal $\gamma_k$-frames, $~{{z}_k \sim \N{{0}, {I}_{\gamma_k}}}$ are independent latent variables and $~{{\epsilon} \sim \N{{0}, \sigma^2 {I}_{p}}}$ is an isotropic Gaussian noise. 
An illustration of the generative model is provided in Fig.~\ref{fig:PSA}. 
PPCA and IPPCA models can then be reinterpreted as PSA models, of respective types $~{{\gamma} = (1, \dots, 1, p - q)}$ and ${\gamma} = (q, p - q)$, where $q < p$ is the intrinsic dimension (cf. Sec.~\ref{appsec:PSA}).
% Figure environment removed



\subsection{Geometry and inference}
From a geometric point of view, the fitted density is isotropic on a sequence of mutually-orthogonal subspaces $\operatorname{Im}(Q_1) \perp \dots \perp \operatorname{Im}(Q_{d})$ of respective dimensions $\gamma_1, \dots, \gamma_d$.
Such a sequence is called a \emph{flag} of linear subspaces of \emph{type} ${\gamma}$.
Therefore, flags of type $\gamma$---which are diffeomorphic to $\O(p) / (\O(\gamma_1) \times \dots \times \O(\gamma_d))$~\cite{arnold_modes_1972, ye_optimization_2022}---naturally parameterize PSA models. 
Consequently, Stiefel manifolds and Grassmannians---which are particular cases of flag manifolds---respectively parameterize PPCA and IPPCA models (cf. Sec.~\ref{appsec:PSA}).
The remaining model parameters are the subspace variances $(\lambda_1, \dots, \lambda_d) \in \R^{d}$ and the mean ${\mu} \in \R^p$.
Thus, the \textit{complexity} (dimension of the parameter space) of the PSA model of type ${\gamma}$ is
\begin{equation}\label{eq:kappa}
    \kappa({\gamma}) := p + d + \frac{p(p-1)}{2} - \sum_{k=1}^{d} \frac {\gamma_k (\gamma_k - 1)} {2}.
\end{equation}
We can notably see that the decrease in model complexity is quadratic in the number of equalized eigenvalues.

One of the strength of the PSA models is that their maximum likelihood estimate is \textit{explicit}, similarly to PPCA and IPPCA. In a nutshell, we show in Thm.~\ref{appthm:PSA} that the most likely mean vector $\mu$ is the \textit{empirical mean}, the most likely variances $\lambda_1, \dots, \lambda_d$ are the \textit{block-averaged sample eigenvalues} according to the type $\gamma$, and the most likely flag $(\operatorname{Im}(Q_1), \dots, \operatorname{Im}(Q_{d}))$ is the sequence of mutually-orthogonal subspaces spanned by the associated eigenvectors. This yields the following expression for the maximum likelihood
\begin{equation}\label{eq:PSA_ML}
    \ln \hat{\mathcal{L}} (\gamma) = -\frac n 2 \left(p \ln(2\pi) + \sum_{k=1}^d \gamma_k \ln{\overline{L_k}} + p\right).
\end{equation}

\subsection{Identifying the curse of isotropy in practice}
The Bayesian information criterion~\cite{schwarz_estimating_1978} is defined as 
\begin{equation}\label{eq:BIC}
    \operatorname{BIC} (\gamma) := \kappa (\gamma) \ln n - 2 \ln \hat{\mathcal{L}} (\gamma).
\end{equation}
It is a widely-used model selection criterion, making a tradeoff between model complexity and goodness-of-fit, to prevent from overfitting given the number of observed samples. The formula results from an asymptotic approximation of the Bayesian model evidence. Given a dataset, one can compare the BIC of a PSA model with repeated eigenvalues to the BIC of a PSA model with distinct eigenvalues. The model with the lowest BIC is selected over the other one. 


As discussed previously, two adjacent sample eigenvalues with a relatively small gap may be prone to isotropic PC variability. 
To identify such situations where the curse of isotropy may arise, we compare a \textit{full} covariance model $\gamma = (1, \dots, 1)$ with an \textit{equalized} covariance model $~{\gamma' = (1, \dots, 1, 2, 1, \dots, 1)}$ where eigenvalues $j$ and $j+1$ are assumed equal.
Denoting $\delta_j := \frac{\ell_{j} - \ell_{j+1}}{\ell_j}$ the \emph{relative eigengap} between the two sample eigenvalues, we show in Sec.~\ref{appsec:MS} that
\begin{equation}\label{eq:releigengap_threshold}
    \mathrm{BIC}(\gamma') < \mathrm{BIC}(\gamma) \iff \frac{\delta_j}{2} < 1 - n^{\frac2n} + n^{\frac1n}\sqrt{n^{\frac2n} - 1}.
\end{equation}
This condition---independent of $p$---is illustrated in Fig.~\ref{fig:BIC_eigengap}. 
% Figure environment removed
We notably deduce by substitution that for $n = 1000$ samples, all the adjacent sample eigenvalues with a relative eigengap lower than $\delta = 21\%$ should be assumed equal. In other words, given two sample eigenvalues of respective magnitude $1$ and $0.8$, one needs \textit{at least} $1000$ samples to overcome the curse of isotropy. \textit{This is rarely the case in practice.} To illustrate this, we test the condition~\eqref{eq:releigengap_threshold} on many classical datasets from the UCI Machine Learning Repository (cf. Sec.~\ref{appsec:data}), with $n/p$ ratios ranging from $10$ to $10^4$.
For each dataset, we report the pairs of adjacent eigenvalues that are below the relative eigengap threshold in Fig.~\ref{fig:releigengap_UCI}.
% Figure environment removed
The outcomes are striking: all datasets but one have some eigenvalue pairs below the threshold. This does not only concern the smallest eigenvalues---which are usually tossed away because considered as noise---but also the highest ones---which are usually interpreted by applied scientists.
This shows that the curse of isotropy is not a negligible phenomenon at all and that particular care should be taken before interpreting the principal components.
Note that~\eqref{eq:releigengap_threshold} involves the \textit{relative} eigengap between adjacent eigenvalues and not the \textit{absolute} one, meaning that an exponentially-decreasing sample eigenvalue profile can actually highly suffer from the curse of isotropy. In other words, PSA models are not just suited to piecewise-constant-like sample covariance profiles.

The BIC is known for its tendency to select underparameterized models~\cite{bishop_pattern_2006}. Therefore, we also investigate in Sec.~\ref{appsec:MS} the eigenvalue-equalization guideline under other model selection criteria like the Akaike information criterion (AIC)~\cite{akaike_new_1974} and under empirical models~\cite{north_sampling_1982}. We get relative eigengaps around $10-20\%$ for $n=1000$, and experimental results substantiating the curse of isotropy's importance.


\subsection{Stratification and efficient model selection}
We now explicit the stratified structure of PSA models and show how it enables to design efficient model selection strategies to choose which groups of eigenvalues to equalize. More details are given in Sec.~\ref{appsec:MS}.


The space of symmetric matrices can be stratified according to the sequence of eigenvalue multiplicities \cite{arnold_modes_1972,groisser_geometric_2017,breiding_geometry_2018}. This implies that the PSA models in dimension $p$ form a stratified exponential family~\cite{geiger_stratified_2001} of cardinal $2^{p-1}$, partially-ordered~\cite{taeb_model_2024} by the stratum-inclusion relation.
We illustrate the family of $5$-dimensional PSA models in Fig.~\ref{fig:hasse_complexity}.
% Figure environment removed

In order to prevent from greedily exploring the whole family for model selection, we propose a simple yet efficient model selection technique based on the stratified structure of this family.
The \textit{hierarchical clustering strategy} consists in performing a hierarchical clustering of the sample eigenvalues, based on chosen \textit{pairwise distance} (e.g. the relative eigengap $\delta_j = \frac{\ell_{j} - \ell_{j+1}}{\ell_j}$) and \textit{cluster-linkage criterion} (e.g. single-linkage). This strategy yields a hierarchical subfamily of $p$ models with decreasing complexity, from which we can more efficiently select the model minimizing the BIC. We prove the \textit{asymptotic consistency} of the hierarchical clustering strategy in Prop~\ref{appprop:hierarchical_heuristic}, as well as introduce other strategies.
\section{From principal components to principal subspaces}

To summarize the previous section, parsimonious considerations invite us to block-average eigenvalues whose relative gaps are close---given the number of observed samples. The associated PSA model is now parameterized with \textit{eigenspaces} instead of individual \textit{eigenvectors} and we are therefore facing the curse of isotropy.
In this section, we propose to actually take advantage of the curse of isotropy and improve data interpretability by transitioning from \textit{principal components} to \textit{principal subspaces}.

A first idea, rather \textit{qualitative}, is to work at the subspace level and generate samples from the multidimensional principal subspaces via Eq.~\eqref{eq:PSA_model} (cf. Fig.~\ref{fig:PSA}). Those samples might have common characteristics like low-frequencies or invariances for images~\cite{hyvarinen_emergence_2000}.
Since there is an isotropic Gaussian variability, one can also uniformly discretize the unit sphere included in the principal subspace---especially in 2D and 3D---to help visualization. Other ideas involve orthogonal projections of explainable variability modes and correlation between subspace-projected dataset and co-variables.

A second idea, rather \textit{quantitative}, is to look for rotations of principal components inside their principal subspace in order to increase interpretability. 
Indeed, as explained previously, the curse of isotropy might cause principal components to be rotated versions of more interpretable components.
\textit{Varimax} rotation~\cite{kaiser_varimax_1958} (on the components and not the projected data as sometimes done~\cite{rohe_vintage_2023}) enables for instance to get rotated components with sparse loadings.
Many other criteria can be considered depending on the data type, notably entropy, structured sparsity~\cite{jenatton_structured_2010} or total variation for images.
\section{Experimental Evaluations}\label{sec:experiment}

\textbf{Implementation.}
We implement \puma\ on top of SecretFlow~\citep{spu} in \textrm{C++} and Python. SecretFlow compiles a high-level Flax code to secure computation protocols, which are then executed by our designed cryptographic backends, and we encode the floating-ponit values as $64$-bit integers in ring $\mathbb{Z}_{2^{64}}$ with $18$-bit fractional part. 
Our experiments are run on 3 Alibaba Cloud ecs.g7.8xlarge servers with 32 vCPU and 128GB RAM each. The CPU model is Intel Xeon(Ice Lake) Platinum 8369B CPU @ 2.70GHz. We evaluate \puma\ on Ubuntu 20.04.6 LTS with Linux kernel 5.4.0-144-generic. Our bandwidth is about 5Gbps and round trip time is about 1ms. %\cheng{Describe fixed point parameters: scale, share bits.}

\textbf{Models \& Datasets.}
We evaluate \puma\ on seven NLP models: Bert-Base, Roberta-Base, and Bert-Large~\citep{bert}; GPT2-Base, GPT2-Medium, and GPT2-Large~\citep{gpt}; and LLaMA-7B~\citep{touvron2023llama}. We measure the Bert performance for three NLP tasks over the datasets of Corpus of Linguistic Acceptability (CoLA), Recognizing Textual Entailment (RTE), Stanford Question Answering Dataset (QNLI) from GLUE benchmarks~\citep{wang2018glue}, and GPT2 performance on Wikitext-103 V1~\citep{merity2016pointer}.

\textbf{Baseline.}
We compare \puma\ to the most similar prior work \mpcformer~\citep{li2023mpcformer}. But for fair comparison, we have the following considerations:
\romannumeral1) As \mpcformer\ neither supports loading pretrained transformer models nor implements LayerNorm faithfully\footnote{ As \mpcformer~does not support loading pre-trained Transformer models, we did an experiment in plaintext Bert-Base that replaced LayerNorm with BatchNorm  as \mpcformer~did. This  resulted in a significant drop in the MCC score for CoLA task from $0.616$ to $-0.020$. On the contrary, \puma~achieves an MCC score of $0.613$. }, we cannot achieve meaningful secure inference results using their framework.
Therefore, we compare our secure Transformer models inference performance to that of plaintext (floating-point) to show our precision guarantee.
\romannumeral2) \mpcformer\ with \textit{Quad} approximations (for both $\gelu$ and $\softmax$) requires retraining the  modified models. As \puma\ does not require retraining, we compare our cost to that of \mpcformer\ without \textit{Quad} approximations. Also, we re-run \mpcformer~in our environment.



\subsection{Precision}\label{sec:accuracy}

% Figure environment removed

%\begin{table}
\centering
\caption{Performance on GLUE benchmark of Bert-Base, Roberta-Base, and Bert-Large on CoLA, RTE, and QNLI, Matthews correlation is reported for CoLA. Accuracy is reported for other datasets.}\label{table:bertacc}
\begin{tabular}{c|ccc|ccc|ccc}
\hline \hline
 Model & \multicolumn{3}{c|}{Bert-Base} & \multicolumn{3}{c|}{Roberta-Base} & \multicolumn{3}{c}{Bert-Large} \\ \hline
 TASK & CoLA & RTE & QNLI & CoLA & RTE & QNLI & CoLA & RTE & QNLI \\ \hline
CPU & $0.616$     & $0.700$      & $0.916$     & $0.629$ & $0.805$ & $0.920$  & $0.686$   & $0.755$ & $0.922$ \\
\puma   & $0.613$     & $0.700$     & $0.916$     & $0.618$ & $0.805$ & $0.918$ & $0.690$ & $0.747$ & $0.918$ \\ \hline \hline
\end{tabular}
\end{table}

\begin{table}[]
    \centering
    \caption{Perplexity of GPT2-Base, GPT2-Medium, and GPT2-Large on Wikitext-103 V1.}
    \label{tab:gpot2ppl}
    \begin{tabular}{c|c|c|c}
    \hline \hline
      Model & GPT2-Base & GPT2-Medium & GPT2-Large \\ \hline
      CPU & $16.284$ & $12.536$ & $10.142$ \\
      \puma & $16.284$ & $12.540$ & $10.161$ \\
      \hline \hline
    \end{tabular}
    
\end{table}

We compare our secure model 
inference performance to that of plaintext (floating-point) in Figure~\ref{fig:performance} to show our precision guarantee.

In Figure~\ref{fig:bert-base}-\ref{fig:bert-large}, we show the Matthews correlation/accuracy of plaintext and \puma\ on the Bert-Base, Roberta-base, and Bert-Large. We observe that the accuracy achieved by \puma~ matches the accuracy of the plaintext Flax code. Specifically, the accuracy difference does
not exceed $0.011$ over all datasets. 

Moreover, in Figure~\ref{fig:gpt2}, we also compare our perplexity on dataset Wikitext-103 V1 with the plaintext baseline on models GPT2-Base, GPT2-Medium, and GPT2-Large. The results are similar and the perplexity differences do not exceed $0.02$ over all models.

The above accuracy and perplexity advantages experimentally validate that our protocols are numerically precise. 

\subsection{Inference cost}\label{sec:efficiency}
\begin{table}[h]
    \centering
    \caption{Costs of Bert-Base, Roberta-Base, and Bert-Large for one sentence of length $128$. Time is in seconds and Communication (Comm. for short) is in GB, which is the same for the following tables.}\label{tab:costbert}
    \begin{tabular}{c|cc|cc|cc}
    \hline \hline
       Model & \multicolumn{2}{c|}{Bert-Base} & \multicolumn{2}{c|}{Roberta-Base} & \multicolumn{2}{c}{Bert-Large} \\ \hline
       Costs & Time & Comm. & Time & Comm. & Time & Comm. \\ \hline
       \mpcformer & $55.320$ & $12.089$ & $57.256$ & $12.373$ & $141.222$ & $32.577$ \\
       \puma & $33.913$ & $10.773$ & $41.641$ & $11.463$ & $73.720$ & $27.246$ \\
       \cellcolor{mygray} Improv. & \cellcolor{mygray} $1.631\times$ & \cellcolor{mygray} $1.122\times$ & \cellcolor{mygray} $1.375\times$ & \cellcolor{mygray} $1.079\times$ & \cellcolor{mygray} $1.916\times$ & \cellcolor{mygray} $1.195\times$ \\
       \hline \hline
    \end{tabular}
    \vspace{-0.2cm}
\end{table}

\begin{table}[]
    \centering
    \caption{Costs of GPT2-Base, GPT2-Medium, and GPT2-Large. The input sentence is of length $32$, all of the costs are for generating $1$ token.}\label{tab:costgpt2}
    \begin{tabular}{c|cc|cc|cc}
    \hline \hline
       Model & \multicolumn{2}{c|}{GPT2-Base} & \multicolumn{2}{c|}{GPT2-Medium} & \multicolumn{2}{c}{GPT2-Large} \\ \hline
       Costs & Time & Comm. & Time & Comm. & Time & Comm. \\ \hline
       \mpcformer & $34.889$ & $4.999$ & $73.078$ & $11.766$ & $129.095$ & $22.522$  \\
       \puma & $15.506$ & $3.774$ & $30.272$ & $7.059$ & $54.154$ & $11.952$ \\
       \cellcolor{mygray} Improv. & \cellcolor{mygray} $2.250\times$ & \cellcolor{mygray} $1.325\times$ & \cellcolor{mygray} $2.414\times$ & \cellcolor{mygray} $1.667\times$ & \cellcolor{mygray} $2.383\times$ & \cellcolor{mygray} $1.884\times$ \\
       \hline \hline
    \end{tabular}
    \vspace{-0.2cm}
\end{table}

In this subsection, we compare \puma's inference cost to that of \mpcformer. 
We evaluate  three Bert models (Bert-Base, Roberta-Base, and Bert-Large) and three GPT2 models (GPT2-Base, GPT2-Medium, and GPT2-Large).
The costs are for processing one input sentence: \romannumeral1) For Bert models the input sentence is of length $128$. \romannumeral2) GPT2 models input one length-32 sentence and generate $1$ new word. 

On the 3 Bert models in Table~\ref{tab:costbert}, \puma\ is  $1.375\sim 1.916\times$ faster than  \mpcformer, and is $1.079\sim 1.195\times$ more communication-efficient. For the GPT2 models in Table~\ref{tab:costgpt2}, \puma\ is $2.250\sim 2.414\times$ faster than \mpcformer, and is $1.325\sim 1.884\times$ more communication-efficient. 
    
We observe that \puma's improvements increase as the model size grows, particularly for the GPT2 models. This trend is because our specialized optimizations are more effective when processing large-scale evaluations.



\subsection{Scalability}\label{sec:scala}

In this subsection, we measure the costs of evaluating \puma\ on Bert-Base and GPT2-Base models for varying-length inputs, and varying-length outputs (only for GPT2-Base). We also compare our costs to those of \mpcformer~to demonstrate our improvements.





\begin{table}[]
    \centering
    \caption{Costs of Bert-Base and GPT2-Base for different input length (denoted as \#Input). The input lengths for Bert-Base and GPT2-Base are respective $\{64, 128, 256, 512\}$ and $\{16, 32, 64, 128\}$. GPT2-Base generates $1$ token.}\label{tab:costbertinput}
    \begin{tabular}{cc|cc|cc|cc|cc}
    \hline \hline
       \multicolumn{2}{c|}{\#Input} & \multicolumn{2}{c|}{$64 / 16$} & \multicolumn{2}{c|}{$128 / 32$} & \multicolumn{2}{c|}{$256 / 64$} & \multicolumn{2}{c}{$512 / 128$}  \\ \hline
       \multicolumn{2}{c|}{Costs} & Time & Comm. & Time & Comm. & Time & Comm. & Time & Comm. \\ \hline
       \multirow{3}{*}{Bert}& \mpcformer & $46.428$ & $4.750$ & $85.887$ & $9.673$ & $196.372$ & $23.443$ & $582.787$ & $68.069$ \\
       & \puma & $24.345$ & $1.627$ & $42.525$ & $3.591$ & $87.561$ & $8.668$ & $212.600$ & $23.439$\\
       & \cellcolor{mygray} Improv. & \cellcolor{mygray} $1.907\times$ & \cellcolor{mygray} $2.919\times$ & \cellcolor{mygray} $2.020\times$ & \cellcolor{mygray} $2.694\times$ & \cellcolor{mygray} $2.243\times$ & \cellcolor{mygray} $2.705\times$ & \cellcolor{mygray} $2.741\times$ & \cellcolor{mygray} $2.904$ \\
       \hline
       \multirow{3}{*}{GPT2}& \mpcformer & $34.522$ & $3.767$ & $42.615$ & $4.516$ & $60.451$ & $6.281$ & $105.028$ & $11.225$  \\
       & \puma & $20.692$ & $0.625$ & $29.248$ & $1.258$ & $40.968$ & $2.607$ & $74.529$ & $5.611$\\
       &\cellcolor{mygray} Improv. & \cellcolor{mygray} $1.668\times$ & \cellcolor{mygray} $6.027\times$ & \cellcolor{mygray} $1.457\times$ & \cellcolor{mygray} $3.590\times$ & \cellcolor{mygray} $1.476\times$ & \cellcolor{mygray} $2.409\times$ & \cellcolor{mygray} $1.409\times$ & \cellcolor{mygray} $2.001\times$\\
       \hline \hline
    \end{tabular}
\end{table}
\textbf{Input Length Evaluation.}
Table~\ref{tab:costbertinput} shows our costs on varying-length inputs, we evaluate Bert-Base on the inputs of length $\{64, 128, 256, 512\}$, and GPT2-Base on the inputs of length $\{16, 32, 64, 128\}$.
For Bert-Base, \puma\ is $1.720\sim 2.282\times$ faster, and for GPT2-Base, \puma\ is $1.550\sim 2.686\times$ faster. Unlike the observations in Section~\ref{sec:efficiency}, our efficiency gains decrease with increasing input sizes in GPT2, and \puma\ requires more communication when the input length is greater than 64. This phenomenon is attributed to the interesting fact: To directly support pre-trained plaintext models, \puma\ strictly follows the plaintext model format that only accept token ids as input, so \puma\ has to compute the one-hot vectors from token ids in an MPC way. On the other hand, \mpcformer\ uses modified models that accept one-hot vectors as input, so the one-hot function could be computed at the client side in plaintext. Nevertheless, \puma\ remains faster than \mpcformer.

%\begin{table}[]
    \centering
    \caption{Costs of GPT2-small for generating different output tokens (denoted as \#Output), the input length is set as $32$.}\label{tab:costgpt2tokens}
    \begin{tabular}{c|cc|cc|cc|cc}
    \hline \hline
       \#Output & \multicolumn{2}{c|}{2} & \multicolumn{2}{c|}{4} & \multicolumn{2}{c|}{8} & \multicolumn{2}{c}{16}  \\ \hline
       Costs & Time & Comm. & Time & Comm. & Time & Comm. & Time & Comm. \\ \hline
       \mpcformer & $72.833$ & $7.676$ & $132.644$ & $13.998$ & $252.796$ & $26.648$ & $494.509$ & $51.972$ \\
       \puma & $53.191$ & $2.549$ & $111.457$ & $5.167$ & $215.352$ & $11.115$ & $457.994$ & $24.917$ \\
       Improv. & $1.369\times$ & $3.011\times$ & $1.190\times$ & $2.709\times$ & $1.174\times$ & $2.397\times$ & $1.080\times$ & $2.086\times$ \\
       \hline \hline
    \end{tabular}
\end{table}

\begin{wrapfigure}{r}{0.4\textwidth}
    % Figure removed
    \caption{Runtime of GPT2-Base for generating different number of output tokens, the input length is of length $32$.} 
    \label{fig:gptwoutcosts}
\end{wrapfigure}

\textbf{Output Length Evaluation.}
Fig~\ref{fig:gptwoutcosts} presents our costs on varying-length outputs for GPT2-Base, and compares our costs to those of \mpcformer. Our improvements in runtime range from $1.279\sim 2.700\times$ respectively.
As more output tokens are generated, both costs increase in a linear way, this is because each output token must be input back into the model to generate the next token, increasing the required one-hot embedding costs. We should emphasize
again that although the time costs might be close for long outputs, \puma\ could achieve a similar accuracy as plaintext models while \mpcformer\  could not. 


\begin{table}[]
    \centering
    \caption{Costs of the secure inference of LLaMA-7B, \#Input denotes the length of input sentence and \#Output denotes the number of generated tokens.}\label{tab:llama7b}
    \begin{tabular}{c|cc|cc|cc}
    \hline \hline
       (\#Input, \#Output) & \multicolumn{2}{c|}{$(4,1)$} & \multicolumn{2}{c|}{$(8,1)$} & \multicolumn{2}{c}{$(8,2)$} \\ \hline
       Costs & Time & Comm. & Time & Comm. & Time & Comm. \\ \hline
       \puma & $122.004$ & $0.907$ & $200.473$ & $1.794$ & $364.527$ & $3.857$ \\
       \hline \hline
    \end{tabular}
    \vspace{-0.2cm}
\end{table}

\textbf{Scale to LLaMA-7B in Five Minutes.}
We evaluated the large language model LLaMA-7B using \puma\ under 3 Alibaba Cloud
ecs.r7.32xlarge servers, each has 128 threads and 1TB RAM, with 20GB bandwidth, 0.06ms round-trip-time. 
As shown in Table~\ref{tab:llama7b}, \puma\ can support the secure inference of large language model LLaMA-7B with reasonable costs. For example, given an input sentence of 8 tokens, \puma\ can output one token in around $346.126$ seconds with communication costs of $1.865$ GB. To our knowledge, this is the first time that LLaMA-7B has been evaluated using MPC.


%Llama-7B, LAN=(20GB, 0.06ms), 128 threads, input length=8, output=1 token, costs: 346.126s, 2002213760 bytes
\section{Related works}
In the climate research community, a celebrated work~\cite{north_sampling_1982}---often cited as \textit{North's rule-of-thumb}---warns scientists against close eigenvalues in the Karhunen-Loève expansion of a meteorological field. Indeed, the associated principal components---referred to as \textit{empirical orthogonal functions} (EOF)---suffer from large sampling errors, which is very problematic due to the key role EOF's play in this field for exploratory data analysis. The authors provide a perturbation-theoretical rule-of-thumb to decide which eigenvalues form \textit{degenerate multiplets}. The rule as stated in the paper is quite vague, however we are able (cf. Sec.~\ref{appsec:MS}) to reformulate its practical software implementation as a relative eigengap threshold and to compare it to our criterion~\eqref{eq:releigengap_threshold}. We show that this threshold is much lower than ours (e.g. $8.6\%$ instead of $21\%$ for $1000$ samples), therefore our result has a much larger impact on the practical methodology of PCA.

More broadly, several works have mentioned close-eigenvalues in PCA or in general symmetric matrices. 
A paper from Jolliffe~\cite{jolliffe_rotation_1989} shows the advantages of factor rotation inside subspaces spanned by principal components with close eigenvalues for structured data. 
Eigenvalue equality has also been studied formally in the context of oscillatory systems~\cite{arnold_modes_1972,lazutkin_kam_1993,gershkovich_problem_2004} diffusion tensor imaging~\cite{groisser_geometric_2017}, spectral geometry~\cite{besson_multiplicy_1988}, spectral shape analysis~\cite{lombaert_diffeomorphic_2013}, statistical tests~\cite{anderson_asymptotic_1963,tyler_asymptotic_1981,rabenoro_geometric_2024} etc.

Finally, the use of flags for statistical analysis has been particularly well illustrated with the example of \textit{independent subspace analysis}~\cite{hyvarinen_emergence_2000}, from which the name of our model is drawn. The authors notice the emergence of phase and shift-invariant features by maximizing the independence between the norms of projections of samples into so-called \textit{independent feature subspaces}. The learning algorithm is later recast as an optimization problem on flag manifolds~\cite{nishimori_riemannian_2006}. 
Flags also implicitly arise in general subspace methods under the name \textit{mutually orthogonal subspaces}, like in the mutually-orthogonal class-subspaces of Watanabe and Pakvasa~\cite{watanabe_subspace_1973} and the adaptive-subspace self-organizing maps of Kohonen~\cite{kohonen_emergence_1996}.
More recently, PCA was also reformulated as an optimization problem on flag manifolds~\cite{pennec_barycentric_2018}, raising perspectives for multilevel data analysis on manifolds.
\section{Conclusion}
\label{sec:conclusion}


We presented DCA, an algorithm to address disparity in outcomes of ranking processes using compensatory bonus points. We showed that DCA, by relying on a sampling-based approach, successfully reduces disparity in a wide range of settings, while being significantly more efficient than state-of-the-art approaches, running in sub-linear time. This makes DCA a good candidate for iterative processes that would allow users to identify the ranking function that best fits their needs while checking for its fairness impacts and the required compensatory bonus points.  


Our approach relies on the use of compensatory bonus points, a departure from previous work, which has mostly focused on modifying the ranking function directly, or on the use of quotas. A significant advantage of compensatory bonus points is that they are transparent, interpretable, and easily explainable to all stakeholders.



\section*{Materials and methods}
The following self-contained appendix formally introduces the principal subspace analysis model, derives all the results from scratch with proofs, details the associated methodology, and provides additional evidence that the curse of isotropy is an important phenomenon that should not be overlooked in practice.

\section*{Acknowledgements}
This work was supported by the ERC grant \#786854 G-Statistics from the European Research Council under the European Union’s Horizon 2020 research and innovation program and by the French government through the 3IA Côte d’Azur Investments ANR-19-P3IA-0002 managed by the National Research Agency.

% \bibliography{ref}
% \bibliographystyle{icml2024}


%%%%%%%%%%%%%%%%%%%%%%%%%%%%%%%%%%%%%%%%%%%%%%%%%%%%%%%%%%%%%%%%%%%%%%%%%%%%%%%
%%%%%%%%%%%%%%%%%%%%%%%%%%%%%%%%%%%%%%%%%%%%%%%%%%%%%%%%%%%%%%%%%%%%%%%%%%%%%%%
% APPENDIX
%%%%%%%%%%%%%%%%%%%%%%%%%%%%%%%%%%%%%%%%%%%%%%%%%%%%%%%%%%%%%%%%%%%%%%%%%%%%%%%
%%%%%%%%%%%%%%%%%%%%%%%%%%%%%%%%%%%%%%%%%%%%%%%%%%%%%%%%%%%%%%%%%%%%%%%%%%%%%%%
\newpage
\appendix
\onecolumn
%\section*{Appendix}
\paragraph{Abstract}
This appendix investigates a general family of covariance models with repeated eigenvalues extending probabilistic principal component analysis (PPCA). A geometric interpretation shows that these models are parameterized by flag manifolds and stratify the space of covariance matrices according to the sequence of eigenvalue multiplicities. The subsequent analysis sheds light on PPCA and answers an important question on the practical identifiability of individual eigenvectors. It notably shows that one rarely has enough samples to fit a covariance model with distinct eigenvalues and that block-averaging the adjacent sample eigenvalues with small gaps achieves a better complexity/goodness-of-fit tradeoff.

\section{Reminders on probabilistic principal component analysis}
Principal component analysis (PCA) is a ubiquitous tool in statistics, which however lacks a probabilistic formulation.
Such a framework can indeed be useful in a variety of contexts like decision-making, generative modeling, missing data and model selection.
The Probabilistic PCA model of~\citet{tipping_probabilistic_1999} circumvents this issue, and we describe it in this section.

\subsection{Model}
Let $\lrp{{x}_i}_{i=1}^n$ be a $p$-dimensional dataset and $q \in \lrb{0, p-1}$ a lower dimension. In PPCA, the observed data is assumed to stem from a $q$-dimensional latent variable via a linear-Gaussian model

\begin{equation}\label{appeq:PPCA_model}
    {x} = {W} {z} + {\mu} + {\epsilon},
\end{equation}
with ${z} \sim \N{0, {I}_q}$, ${W} \in \R^{p \times q}$, ${\mu} \in \R^p$, ${\epsilon} \sim \N{0, \sigma^2 {I}_p}$ and $\sigma^2 > 0$. 

Through classical probability theory, one can show that the observed data is modeled as following a multivariate Gaussian distribution
\begin{equation}
    {x} \sim \N{{\mu}, {W} {W}\T + \sigma^2 {I}_p}.
\end{equation}
An analysis of the covariance matrix reveals that the distribution is actually anisotropic on the first $q$ dimensions and isotropic on the remaining $p - q$ ones. Hence there is an implicit constraint on the covariance model of the data, which is that the lowest $p - q$ eigenvalues are assumed to be all equal.


\subsection{Maximum likelihood}
The PPCA model parameters are the shift ${\mu}$, the linear map ${W}$ and the noise factor $\sigma^2$. Let some observed dataset $\lrp{{x}_i}_{i=1}^n$, $\overline {{x}} := \frac 1 n \sum_{i=1}^n {x}_i$ its mean and $~{{S} := \sum_{j=1}^p \ell_j {v}_j {{v}_j}\T}$ its sample covariance matrix, with its eigenvalues $\ell_1 \geq \dots \geq \ell_p \geq 0$ and associated eigenvectors ${v}_1 \perp \dots \perp {v}_p$. One can explicitly infer the parameters that are the most likely to have generated these data using maximum likelihood estimation.
It is shown in the original PPCA paper that the most likely shift is the empirical mean, the most likely linear map is the composition of a scaling by the $q$ highest eigenvalues ${L}_q:=\diag{\ell_1, \dots, \ell_q}$ (up to the noise) and an orthogonal transformation by the associated $q$ eigenvectors ${V}_q:=\lrb{{{v}}_1|\dots|{{v}}_q}$, and finally the most likely noise factor is the average of the $p - q$ discarded eigenvalues
\begin{equation}\label{appeq:PPCA_ML}
\hat{{\mu}} = \overline {{x}} , \hspace*{15mm}
\hat{{W}} = {V}_q \lrp{{L}_q - \hat{\sigma}^2 {I}_q}^{\frac 1 2} , \hspace*{15mm}
\hat{\sigma}^2 = \frac 1 {p - q} \sum_{j=q+1}^p \ell_j.
\end{equation}
One can then easily express the maximum log-likelihood
\begin{equation}
    \ln \hat{\mathcal{L}}(q) := -\frac n 2 \lrp{p \ln(2\pi) + \sum_{j=1}^q \ln{\ell_j} + (p - q) \ln\lrp{\frac 1 {p - q} \sum_{j=q+1}^p \ell_j} + p}.
\end{equation}

\subsection{Parsimony and model selection}
The previously described PPCA is already a parsimonious statistical model. Indeed, it not only makes the assumption that the observed data follows a multivariate Gaussian distribution, which is the entropy-maximizing distribution at a fixed mean and covariance, but it also reduces the number of covariance parameters by constraining the last $p-q$ eigenvalues to be equal.
The covariance matrix $\Sigma := {W} {W}\T + \sigma^2 {I}_p$ is parameterized by ${W} \in \R^{p\times q}$ and $\sigma^2$. It is shown in the original PPCA paper to have $\kappa(q) := p q - \frac{q (q-1)}{2} + 1$ free parameters---the removal of $\frac{q (q-1)}{2}$ parameters being due to the rotational-invariance of the latent variable $z\in\R^q$. Although not evident at first sight with this expression of $\kappa$, we have a drop of complexity---with respect to the full covariance model which is of dimension $\frac {p(p+1)}{2}$---due to the equality constraint on the low eigenvalues, and the number of parameters decreases along with $q$.
As discussed in the next section, we can give an insightful geometric interpretation to the number of free parameters in the PPCA model using Stiefel manifolds.

For a given data dimension $p$, a PPCA model is indexed by its latent variable dimension $q \in \lrb{0, p-1}$. The process of model selection then consists in comparing different PPCA models and choosing the one that optimizes a criterion, like the Bayesian information criterion (BIC) or more PPCA-oriented ones like Bayesian PCA~\citep{bishop_bayesian_1998} or Minka's criterion~\cite{minka_automatic_2000}. They often rely on a tradeoff between goodness-of-fit (via maximum likelihood) and complexity (via the number of parameters), weighted by the number of samples.

\subsection{Isotropic PPCA}
Isotropic PPCA (IPPCA) is an even more constrained covariance model with only two distinct eigenvalues. For $a > b$ and ${U} \in \R^{p \times q}$ such that ${U}\T {U} = {I}_q$, one defines it as
\begin{equation}
    \Sigma := \lrp{a - b} {U} {U}\T + b {I}_p.
\end{equation}
Such a parsimonious model is shown to be efficient in high-dimensional classification problems~\cite{bouveyron_high-dimensional_2007}.
The authors derive the maximum likelihood of such a model, which is highly related to the one of PPCA, where this time the $q$ first sample covariance eigenvalues are also averaged to fit the model. They also show that the maximum likelihood criterion alone is surprisingly asymptotically consistent for selecting the true intrinsic dimension under the assumptions of IPPCA.
\section{Identifying the curse of isotropy}
In order to spot the curse of isotropy, we go through the lens of statistical modeling and introduce the PSA generative model. This model assumes a Gaussian distribution with repeated covariance eigenvalues. 
It enjoys an explicit maximum likelihood estimate with a rich geometry enabling effective model selection.


\subsection{PSA model}
Let ${\gamma} := (\gamma_1, \dots, \gamma_d)$ be a \textit{composition} of a positive integer $p$---i.e. a sequence of positive integers that sums up to $p$.
We define the PSA model of \emph{type} ${\gamma}$ as the family of Gaussian distributions $~{p(x | \mu, \Sigma) := \mathcal{N}(x | \mu, \Sigma)}$, where $~{\mu \in \R^p}$ is a mean vector and $~{\Sigma = \sum_{k=1}^d \lambda_k Q_k {Q_k}\T\in S_p^{++}}$ is a covariance matrix with repeated eigenvalues $~{\lambda_1 > \dots > \lambda_d > 0}$ of respective multiplicity $\gamma_1, \dots, \gamma_d$ and associated eigenspaces $\mathrm{Im}(Q_1), \dots, \mathrm{Im}(Q_d)$.
These distributions can be rewritten as a (linear-Gaussian) latent variable generative model
\begin{equation}\label{eq:PSA_model}
{x} = \sum_{k=1}^{d-1} \sigma_k {Q}_k {z}_k + {\mu} + {\epsilon},
\end{equation}
where $~{\sigma_1 > \dots > \sigma_{d-1} > 0}$ are decreasing scaling factors,
${Q}_k \in \R^{p \times \gamma_k}$ are mutually-orthogonal orthonormal $\gamma_k$-frames, $~{{z}_k \sim \N{{0}, {I}_{\gamma_k}}}$ are independent latent variables and $~{{\epsilon} \sim \N{{0}, \sigma^2 {I}_{p}}}$ is an isotropic Gaussian noise. 
An illustration of the generative model is provided in Fig.~\ref{fig:PSA}. 
PPCA and IPPCA models can then be reinterpreted as PSA models, of respective types $~{{\gamma} = (1, \dots, 1, p - q)}$ and ${\gamma} = (q, p - q)$, where $q < p$ is the intrinsic dimension (cf. Sec.~\ref{appsec:PSA}).
% Figure environment removed



\subsection{Geometry and inference}
From a geometric point of view, the fitted density is isotropic on a sequence of mutually-orthogonal subspaces $\operatorname{Im}(Q_1) \perp \dots \perp \operatorname{Im}(Q_{d})$ of respective dimensions $\gamma_1, \dots, \gamma_d$.
Such a sequence is called a \emph{flag} of linear subspaces of \emph{type} ${\gamma}$.
Therefore, flags of type $\gamma$---which are diffeomorphic to $\O(p) / (\O(\gamma_1) \times \dots \times \O(\gamma_d))$~\cite{arnold_modes_1972, ye_optimization_2022}---naturally parameterize PSA models. 
Consequently, Stiefel manifolds and Grassmannians---which are particular cases of flag manifolds---respectively parameterize PPCA and IPPCA models (cf. Sec.~\ref{appsec:PSA}).
The remaining model parameters are the subspace variances $(\lambda_1, \dots, \lambda_d) \in \R^{d}$ and the mean ${\mu} \in \R^p$.
Thus, the \textit{complexity} (dimension of the parameter space) of the PSA model of type ${\gamma}$ is
\begin{equation}\label{eq:kappa}
    \kappa({\gamma}) := p + d + \frac{p(p-1)}{2} - \sum_{k=1}^{d} \frac {\gamma_k (\gamma_k - 1)} {2}.
\end{equation}
We can notably see that the decrease in model complexity is quadratic in the number of equalized eigenvalues.

One of the strength of the PSA models is that their maximum likelihood estimate is \textit{explicit}, similarly to PPCA and IPPCA. In a nutshell, we show in Thm.~\ref{appthm:PSA} that the most likely mean vector $\mu$ is the \textit{empirical mean}, the most likely variances $\lambda_1, \dots, \lambda_d$ are the \textit{block-averaged sample eigenvalues} according to the type $\gamma$, and the most likely flag $(\operatorname{Im}(Q_1), \dots, \operatorname{Im}(Q_{d}))$ is the sequence of mutually-orthogonal subspaces spanned by the associated eigenvectors. This yields the following expression for the maximum likelihood
\begin{equation}\label{eq:PSA_ML}
    \ln \hat{\mathcal{L}} (\gamma) = -\frac n 2 \left(p \ln(2\pi) + \sum_{k=1}^d \gamma_k \ln{\overline{L_k}} + p\right).
\end{equation}

\subsection{Identifying the curse of isotropy in practice}
The Bayesian information criterion~\cite{schwarz_estimating_1978} is defined as 
\begin{equation}\label{eq:BIC}
    \operatorname{BIC} (\gamma) := \kappa (\gamma) \ln n - 2 \ln \hat{\mathcal{L}} (\gamma).
\end{equation}
It is a widely-used model selection criterion, making a tradeoff between model complexity and goodness-of-fit, to prevent from overfitting given the number of observed samples. The formula results from an asymptotic approximation of the Bayesian model evidence. Given a dataset, one can compare the BIC of a PSA model with repeated eigenvalues to the BIC of a PSA model with distinct eigenvalues. The model with the lowest BIC is selected over the other one. 


As discussed previously, two adjacent sample eigenvalues with a relatively small gap may be prone to isotropic PC variability. 
To identify such situations where the curse of isotropy may arise, we compare a \textit{full} covariance model $\gamma = (1, \dots, 1)$ with an \textit{equalized} covariance model $~{\gamma' = (1, \dots, 1, 2, 1, \dots, 1)}$ where eigenvalues $j$ and $j+1$ are assumed equal.
Denoting $\delta_j := \frac{\ell_{j} - \ell_{j+1}}{\ell_j}$ the \emph{relative eigengap} between the two sample eigenvalues, we show in Sec.~\ref{appsec:MS} that
\begin{equation}\label{eq:releigengap_threshold}
    \mathrm{BIC}(\gamma') < \mathrm{BIC}(\gamma) \iff \frac{\delta_j}{2} < 1 - n^{\frac2n} + n^{\frac1n}\sqrt{n^{\frac2n} - 1}.
\end{equation}
This condition---independent of $p$---is illustrated in Fig.~\ref{fig:BIC_eigengap}. 
% Figure environment removed
We notably deduce by substitution that for $n = 1000$ samples, all the adjacent sample eigenvalues with a relative eigengap lower than $\delta = 21\%$ should be assumed equal. In other words, given two sample eigenvalues of respective magnitude $1$ and $0.8$, one needs \textit{at least} $1000$ samples to overcome the curse of isotropy. \textit{This is rarely the case in practice.} To illustrate this, we test the condition~\eqref{eq:releigengap_threshold} on many classical datasets from the UCI Machine Learning Repository (cf. Sec.~\ref{appsec:data}), with $n/p$ ratios ranging from $10$ to $10^4$.
For each dataset, we report the pairs of adjacent eigenvalues that are below the relative eigengap threshold in Fig.~\ref{fig:releigengap_UCI}.
% Figure environment removed
The outcomes are striking: all datasets but one have some eigenvalue pairs below the threshold. This does not only concern the smallest eigenvalues---which are usually tossed away because considered as noise---but also the highest ones---which are usually interpreted by applied scientists.
This shows that the curse of isotropy is not a negligible phenomenon at all and that particular care should be taken before interpreting the principal components.
Note that~\eqref{eq:releigengap_threshold} involves the \textit{relative} eigengap between adjacent eigenvalues and not the \textit{absolute} one, meaning that an exponentially-decreasing sample eigenvalue profile can actually highly suffer from the curse of isotropy. In other words, PSA models are not just suited to piecewise-constant-like sample covariance profiles.

The BIC is known for its tendency to select underparameterized models~\cite{bishop_pattern_2006}. Therefore, we also investigate in Sec.~\ref{appsec:MS} the eigenvalue-equalization guideline under other model selection criteria like the Akaike information criterion (AIC)~\cite{akaike_new_1974} and under empirical models~\cite{north_sampling_1982}. We get relative eigengaps around $10-20\%$ for $n=1000$, and experimental results substantiating the curse of isotropy's importance.


\subsection{Stratification and efficient model selection}
We now explicit the stratified structure of PSA models and show how it enables to design efficient model selection strategies to choose which groups of eigenvalues to equalize. More details are given in Sec.~\ref{appsec:MS}.


The space of symmetric matrices can be stratified according to the sequence of eigenvalue multiplicities \cite{arnold_modes_1972,groisser_geometric_2017,breiding_geometry_2018}. This implies that the PSA models in dimension $p$ form a stratified exponential family~\cite{geiger_stratified_2001} of cardinal $2^{p-1}$, partially-ordered~\cite{taeb_model_2024} by the stratum-inclusion relation.
We illustrate the family of $5$-dimensional PSA models in Fig.~\ref{fig:hasse_complexity}.
% Figure environment removed

In order to prevent from greedily exploring the whole family for model selection, we propose a simple yet efficient model selection technique based on the stratified structure of this family.
The \textit{hierarchical clustering strategy} consists in performing a hierarchical clustering of the sample eigenvalues, based on chosen \textit{pairwise distance} (e.g. the relative eigengap $\delta_j = \frac{\ell_{j} - \ell_{j+1}}{\ell_j}$) and \textit{cluster-linkage criterion} (e.g. single-linkage). This strategy yields a hierarchical subfamily of $p$ models with decreasing complexity, from which we can more efficiently select the model minimizing the BIC. We prove the \textit{asymptotic consistency} of the hierarchical clustering strategy in Prop~\ref{appprop:hierarchical_heuristic}, as well as introduce other strategies.
\section{Model selection}\label{appsec:MS}
As discussed previously, sample covariance matrices almost surely have distinct eigenvalues. This makes the full covariance model the most likely to have generated some observed data.
However, it does not mean that the true parameters---that are the eigenvectors and the eigenvalues---can be individually precisely inferred, especially in the small-data regime.
Hence, one can wonder if a covariance model with repeated eigenvalues and multidimensional eigenspaces would not be more robust.
The results of the previous section enable us to provide a possible answer, through PSA model selection. 
First, we study the inference of two adjacent eigenvalues and their associated eigenvectors. We show that when the relative eigengap is small and the number of samples is limited, one should prefer a PSA model with repeated eigenvalues---i.e. block-average the eigenvalues and gather the associated eigenvectors in a multidimensional eigenspace.
Second, to extend this result to more than two eigenvalues, we develop a general model selection framework based on the stratified structure of PSA models.


\subsection{Bayesian information criterion}
The Bayesian information criterion (BIC) is defined as 
\begin{equation}\label{appeq:BIC}
    \operatorname{BIC}(\gamma) = \kappa(\gamma) \ln n - 2 \ln \hat{\mathcal{L}}(\gamma),
\end{equation}
where $\kappa$ is the number of free parameters~\eqref{appeq:PSA_kappa} and $\ln \hat{\mathcal{L}}$ is the maximum log-likelihood~\eqref{appeq:PSA_ML}.
It is a widely-used model selection criterion, making a tradeoff between model complexity $\kappa$ and goodness-of-fit $\hat{\mathcal{L}}$. The formula results from an asymptotic approximation of the model evidence.
In this section, we use the BIC for PSA model selection. The model with lowest BIC is considered as the best model. In the two-eigenvalue case, we get an explicit criterion based on eigenvalue gaps to decide if we must assume that they are equal, and in the more general case, we propose efficient model comparison strategies. We also investigate other model selection criteria than the BIC for completeness in this section, and get similar conclusions.

\subsection{The two-eigenvalue case}
In order to better understand the dynamics of PSA model selection, we lead the experiment of quantifying the BIC variation induced by the equalization of two adjacent eigenvalues. More precisely and without loss of generality, we compare the BIC of a \emph{full covariance model} ${\gamma} = \lrp{1, \dots, 1}$ to the one of an \emph{equalized covariance model} ${\gamma}' = \lrp{1 \dots 1, 2, 1 \dots 1}$, where the eigenvalue $\lambda_j$ has multiplicity $2$.


\begin{theorem}
\label{appthm:releigengap}
Let $\lrp{x_i}_{i=1}^n$ be a $p$-dimensional dataset with $n$ samples, $\ell_j \geq \ell_{j+1}$ two adjacent sample eigenvalues and $\delta_j = \frac{\ell_{j} - \ell_{j+1}}{\ell_j}$ be their \emph{relative eigengap}. 
If
\begin{equation}\label{appeq:releigengap_BIC}
    \delta_j < 2\lrp{1 - n^{\frac 2 n} + n^{\frac 1 n}\sqrt{n^{\frac 2 n} - 1}},
\end{equation}
then the equalized covariance model has a lower BIC than the full one.
\end{theorem}
\begin{proof}
Since $n$ and $p$ are constant within model selection, the BIC can be rewritten (up to constant terms and factors) as
\begin{equation}\label{appeq:BIC_simpl}
\operatorname{BIC} (\gamma) := \lrp{d - \sum_{k=1}^d \frac{\gamma_k (\gamma_k - 1)} {2}} \frac{\ln{n}}{n} + \sum_{k=1}^d \gamma_k \ln{\overline{L_k}}.
\end{equation}
We compare the BIC of the full covariance model ${\gamma} = \lrp{1, \dots, 1}$ to the one of the equalized covariance model ${\gamma}' = \lrp{1, \dots, 1, 2, 1, \dots 1}$ where the $j$-th eigenvalue has been equalized with the $j+1$-th. This boils down to studying the sign of the function $\Delta \BIC = \BIC({{\gamma}}) - \BIC({{\gamma}}')$. One gets  % may
\begin{align}
\Delta \BIC &= p \frac{\ln n}{n} + \sum_{k=1}^p \ln \ell_k - \lrp{p - 2} \frac{\ln n}{n} - \sum_{k \notin \lrs{j, j+1}} \ln \ell_k - 2 \ln\lrp{\frac{\ell_j + \ell_{j+1}} 2},\\
&= 2\frac{\ln n}{n} + \ln \ell_j + \ln \ell_{j+1} - 2 \ln\lrp{\frac{\ell_j + \ell_{j+1}} 2},\\
&= 2\frac{\ln n}{n} + \ln \ell_j + \ln \lrp{\ell_j \lrp{1 - \delta_j}} - 2 \ln\lrp{\frac{\ell_j \lrp{2 - \delta_j}} 2},\\
&= 2\frac{\ln n}{n} + \ln{\lrp{1 - \delta_j}} - 2 \ln\lrp{1 - \frac{\delta_j} 2},\\
&= 2\frac{\ln n}{n} - \ln\lrp{\frac{\lrp{1 - \frac{\delta_j} 2}^2}{1 - \delta_j}}.
\end{align}
Hence, one has
\begin{equation}
\Delta \BIC = 0 \iff \exp\lrp{2\frac{\ln n}{n}} = \frac{\lrp{1 - \frac{\delta_j} 2}^2}{1 - \delta_j} \iff \frac {\delta_j^2} 4 - \lrp{1 - \exp\lrp{2 \frac{\ln n} n}} \delta_j + 1 - \exp\lrp{2 \frac{\ln n} n} = 0.    
\end{equation}
It is a polynomial equation whose positive solution is unique when $n \geq 1$ and is
\begin{equation}
\delta(n) =  2 - 2 \exp\lrp{2 \frac{\ln n} n} + 2\sqrt{\exp\lrp{4 \frac{\ln n} n} - \exp\lrp{2 \frac{\ln n} n}}.
\end{equation}
\end{proof}

\subsection{Comparison with North's rule-of-thumb}
A rule-of-thumb for determining which sample eigenvalue pairs might lead to large PC sampling error is proposed in~\citet{north_sampling_1982}.
The authors show that the asymptotic sampling error of a population eigenvalue $\lambda$ is $\Delta\lambda := \lambda(\frac2n)^{\frac12}$ in the Gaussian setting. North's rule-of-thumb (NRT) states that when one population eigenvalue's sampling error is comparable to or larger than its distance to an adjacent eigenvalue, then the PC's sampling error is comparable to the associated adjacent PC.
Note that this is not an explicit rule (compared to our relative eigengap threshold~\eqref{appeq:releigengap_BIC}) since one has to choose the level of uncertainty, and---most of all---it is based on the \textit{true} eigenvalues (on which the confidence intervals are based) which are unknown.
However, this rule has been applied in many contexts and it is commonly implemented in the following way~\cite{sinkr}. 
For each sample eigenvalue pair $\ell_{j} \geq \ell_{j+1}$, compute the 1 sigma error intervals $I_{j} = [\ell_{j} - \ell_{j}\sqrt{\frac2n}, \ell_{j} + \ell_{j}\sqrt{\frac2n}]$ and $I_{j+1} = [\ell_{j+1} - \ell_{j+1}\sqrt{\frac2n}, \ell_{j+1} + \ell_{j+1}\sqrt{\frac2n}]$. If $I_j\cap I_{j+1} \neq \emptyset$, then the associated principal components suffer from large sampling errors and might be random mixtures of the true eigenvectors. We reformulate it as a relative eigengap threshold.
\begin{proposition}\label{appprop:releigengap_North}
North's rule-of-thumb (as implemented in practice) boils down to the relative eigengap threshold
\begin{equation}\label{appeq:releigengap_North}
    \delta_j \leq \frac{2\sqrt{\frac2n}}{1+\sqrt{\frac2n}}.
\end{equation}
\end{proposition}
\begin{proof}
The sampling error interval overlap condition writes as
\begin{align}
    \ell_{j} - \sqrt{\frac2n} \ell_{j} \leq \ell_{j+1} + \sqrt{\frac2n} \ell_{j+1} & \iff \frac{\ell_j - \ell_{j+1}}{\ell_j} \leq \sqrt{\frac2n} \lrp{1 + \frac{\ell_{j+1}}{\ell_j}},\\
    & \iff \frac{\ell_j - \ell_{j+1}}{\ell_j} \leq \sqrt{\frac2n} \lrp{2 - \frac{\ell_j - \ell_{j+1}}{\ell_j}},\\
    & \iff \frac{\ell_j - \ell_{j+1}}{\ell_j} \leq \frac{2 \sqrt{\frac2n}}{1 + \sqrt{\frac2n}}.
\end{align}
\end{proof}
\noindent This threshold is reported in Fig.~\ref{appfig:releigengap_curves}, under the name NRT-1 (for 1 sigma sampling errors). We also report North's rule-of-thumb for 2 sigma sampling errors (NRT-2), yielding a relative eigengap threshold of $\frac{4 \sqrt{\frac2n}}{1 + 2\sqrt{\frac2n}}$.
% Figure environment removed
We see that the relative eigengap NRT-1 is much smaller than ours (e.g. $8.6\%$ instead of $21\%$ for $1000$ samples). Therefore, although warning scientists about close sample eigenvalues in principal component analysis, North's rule-of-thumb largely overlooks the curse of isotropy compared to our method.
To see the practical effect of this lower threshold, we test this condition on the same real datasets as in Fig.~\ref{fig:releigengap_UCI}. The results are in Fig.~\ref{appfig:releigengaps_real}.
% Figure environment removed
We can see that the curse of isotropy remains a nonnegligible phenomenon with North's rule, even though it is less marked than with the BIC.
We think that North's rule (as implemented in practice) underestimates the phenomenon, notably because it uses 1 sigma uncertainties and since it is based on sample eigenvalues instead of true eigenvalues in the implementations. We recall that 1 sigma uncertainties (NRT-1) correspond to $68\%$ error bars while 2 sigma uncertainties (NRT-2) correspond to $95\%$ error bars and yield a relative eigengap threshold of $16\%$, which is much closer to our results with the BIC.
An interesting perspective would be to consider our guideline instead of the less-impactful North's rule in seminal climate science papers which made some conclusions out of possibly degenerate principal components.


\subsection{Comparison with other model selection criteria}
Although being widely used in model selection, the BIC is well-known for its heavy complexity penalization, tending to select over-parsimonious models~\cite{burnham_model_2004}. Another widely-used criterion is the Akaike information criterion~\cite{akaike_new_1974}. It is defined as
\begin{equation}\label{appeq:AIC}
    \mathrm{AIC}(\gamma) = 2 \kappa(\gamma) - 2 \ln \hat{\mathcal{L}}(\gamma)
\end{equation}
where $\kappa$ is the number of free parameters~\eqref{appeq:PSA_kappa} and $\ln \hat{\mathcal{L}}$ is the maximum log-likelihood~\eqref{appeq:PSA_ML}.
Comparing an equalized covariance model to one with distinct eigenvalues like in Thm.~\ref{appthm:releigengap} but this time using the AIC yields another relative eigengap condition.
\begin{proposition}\label{appprop:releigengap_AIC}
Let $\lrp{x_i}_{i=1}^n$ be a $p$-dimensional dataset with $n$ samples, $\ell_j \geq \ell_{j+1}$ two adjacent sample eigenvalues and $\delta_j = \frac{\ell_{j} - \ell_{j+1}}{\ell_j}$ their \emph{relative eigengap}. 
If
\begin{equation}\label{appeq:releigengap_AIC}
    \delta_j < 2\lrp{1 - e^{\frac4n} + e^{\frac2n}\sqrt{e^{\frac4n} - 1}}
\end{equation} 
then the equalized covariance model has a lower AIC than the full one.
\end{proposition}
\begin{proof}
The proof is essentially the same as the one of Thm.~\ref{appthm:releigengap}.
Since $n$ and $p$ are constant within model selection, the AIC can be rewritten (up to constant terms and factors) as
\begin{equation}\label{appeq:AIC_simpl}
\operatorname{AIC} (\gamma) := \lrp{d - \sum_{k=1}^d \frac{\gamma_k (\gamma_k - 1)} {2}} \frac{2}{n} + \sum_{k=1}^d \gamma_k \ln{\overline{L_k}}
\end{equation}
Replacing $\frac{\ln n}{n}$ with $\frac{2}{n}$ in the proof of Thm.~\ref{appthm:releigengap}, we finally get the result that $\delta(n) =  2 - 2 \exp\lrp{\frac{4} n} + 2\sqrt{\exp\lrp{\frac{8} n} - \exp\lrp{\frac{4} n}}$.
\end{proof}
\noindent This threshold is reported in Fig.~\ref{appfig:releigengap_curves}. We see that this relative eigengap is smaller than ours~\eqref{appeq:releigengap_BIC} (e.g. $12\%$ instead of $21\%$ for $1000$ samples), but higher than North's rule~\eqref{appeq:releigengap_North}. This result is interesting since AIC is known for tending to select overparameterized models, especially for small sample sizes~\cite{burnham_model_2004} (cf. next paragraph). 
Despite this, the relative eigengap condition with AIC is more impactful than North's rule.
To see the practical effect of the AIC threshold of~\eqref{appeq:releigengap_AIC}, we also report the relative eigengap condition on real datasets in Fig.~\ref{appfig:releigengaps_real}.
We see that many eigenvalue pairs should be assumed equal---slightly less than with BIC. Therefore, even with another model selection criterion, the curse of isotropy is still a nonnegligible phenomenon in real datasets, and the principal subspace analysis methodology enables to leverage it to improve interpretability.


Additionally, we provide a relative eigengap condition for the AICc~\cite{hurvich_regression_1989}, which is a small-sample correction to the AIC. In practice, the AICc is advised over the AIC for $n/\kappa < 40$~\cite{burnham_model_2004}.
The AICc is defined as
\begin{equation}\label{appeq:AICc}
    \mathrm{AICc}(\gamma) = 2 \kappa(\gamma) {\frac{n}{n-\kappa(\gamma)-1}} - 2 \ln \hat{\mathcal{L}}(\gamma)
\end{equation}
where $\kappa$ is the number of free parameters~\eqref{appeq:PSA_kappa} and $\ln \hat{\mathcal{L}}$ is the maximum log-likelihood~\eqref{appeq:PSA_ML}.
One can see that this corrected criterion converges asymptotically to the AIC. 
Comparing an equalized covariance model to one with distinct eigenvalues like in Thm.~\ref{appthm:releigengap} but this time with the AICc yields the following relative eigengap condition.
\begin{proposition}\label{appprop:releigengap_AICc}
Let $\lrp{x_i}_{i=1}^n$ be a $p$-dimensional dataset with $n > \frac{p(p+3)}{2} + 1$ samples, $\ell_j \geq \ell_{j+1}$ two adjacent sample eigenvalues, $\delta_j = \frac{\ell_{j} - \ell_{j+1}}{\ell_j}$ their \emph{relative eigengap} and $\varphi = \frac{4n-4}{\lrp{n - \frac{p(p+3)}{2}}^2 - 1}$.
If
\begin{equation}\label{appeq:releigengap_AICc}
    \delta_j < 2\lrp{1 - e^{\varphi} + e^{\frac{\varphi}{2}}\sqrt{e^{\varphi} - 1}}
\end{equation} 
then the equalized covariance model has a lower AICc than the full one.
\end{proposition}
\begin{proof}
The proof is essentially the same as in Thm~\ref{appthm:releigengap} and Prop.~\ref{appprop:releigengap_AIC}.
Since $n$ and $p$ are constant within model selection, the AICc can be rewritten (up to constant terms and factors) as
\begin{equation}\label{appeq:AICc_simpl}
\operatorname{AICc} (\gamma) := \frac{2\kappa(\gamma)}{n - \kappa(\gamma) - 1} + \sum_{k=1}^d \gamma_k \ln{\overline{L_k}}
\end{equation}
We compare the AICc of the full covariance model ${\gamma} = \lrp{1, \dots, 1}$ to the one of the equalized covariance model ${\gamma}' = \lrp{1, \dots, 1, 2, 1, \dots 1}$ where the $j$-th eigenvalue has been equalized with the $j+1$-th. This boils down to studying the sign of the function $\Delta \operatorname{AICc} = \operatorname{AICc}({{\gamma}}) - \operatorname{AICc}({{\gamma}}')$. One gets
\begin{align}
\Delta \operatorname{AICc} &= \frac{p(p+3)}{n - \frac{p(p+3)}{2} - 1} - \frac{p(p+3) - 4}{n - \lrp{\frac{p(p+3)}{2} - 2} - 1} + \ln \ell_j + \ln \ell_{j+1} - 2 \ln\lrp{\frac{\ell_j + \ell_{j+1}} 2}\\
&= \frac{4n-4}{\lrp{n-\frac{p(p+3)}{2}}^2 - 1} + \ln \ell_j + \ln \ell_{j+1} - 2 \ln\lrp{\frac{\ell_j + \ell_{j+1}} 2}
\end{align}
Replacing $2\frac{\ln n}{n}$ with $\varphi = \frac{4n-4}{\lrp{n-\frac{p(p+3)}{2}}^2 - 1}$ in the proof of Thm.~\ref{appthm:releigengap}, we finally get the result that $\delta(n) =  2 - 2 \exp\lrp{\varphi} + 2\sqrt{\exp\lrp{2\varphi} - \exp\lrp{\varphi}}$.
\end{proof}
\noindent Contrary to the other criteria (Thm~\ref{appthm:releigengap}, Prop.~\ref{appprop:releigengap_North} and Prop.~\ref{appprop:releigengap_AIC}), this threshold depends on the dimension $p$. Therefore, we plot it for several $p$ in Fig.~\ref{appfig:releigengap_curves}. 
We can see that this relative eigengap converges to the AIC for large $n$, but is higher than the one with the BIC~\eqref{appeq:releigengap_BIC} when the number of samples is close to the number of model parameters.
We also test this condition on the same real datasets as in Fig.~\ref{fig:releigengap_UCI} and report the results in Fig.~\ref{appfig:releigengaps_real}.
We see that many eigenvalue pairs are ill-defined, especially in high-dimensional datasets where those are even more numerous than with the BIC.


\subsection{Efficient model selection}
Given a dimension $p$, PPCA has $p$ models, ranging from the isotropic Gaussian ($q=0$) to the full covariance model ($q=p-1$). We can naturally equip the set of PPCA models with the  \emph{less-than-or-equal} relation $\leq$ on the latent variable dimension $q$, which makes it a totally ordered set. The complexity of the model then increases with $q$.

The characterization of the PSA family structure is a bit more technical, as it requires to study the hierarchy of types, involving the concept of integer composition. Fortunately, this analysis can be lifted to the stratification of symmetric matrices according to the multiplicities of the eigenvalues, which is already well-known~\cite{arnold_modes_1972,groisser_geometric_2017,breiding_geometry_2018}. Therefore, without proof, we can state the following result.

\begin{proposition}\label{appprop:pos}
The family of $p$-dimensional PSA models induces a stratification of the space of symmetric positive-definite (SPD) matrices $S_p^{++}$ according to the type ${\gamma}$.
The refinement relation $\preceq$ makes it a partially ordered set of cardinal $2^{p-1}$.
\end{proposition}

\noindent Hence the set of PSA models at a given data dimension can be represented using a Hasse diagram, as done in Fig.~\ref{fig:hasse_complexity}.
We see that PSA contains PPCA, IPPCA, and many new models. 
PSA therefore has the advantage of possibly providing more adapted models than PPCA and IPPCA, but also the drawback of requiring more comparisons for model selection. 
In high dimension this becomes quickly computationally heavy, therefore we need to define strategies for selecting only a few number of models to compare. The previously derived partial order $\preceq$ on the set of PSA models allows simple efficient strategies for model selection. In the following subsubsections, we detail those strategies and prove additional properties.

\subsubsection{Relative eigengap threshold clustering of eigenvalues}
The \textit{relative eigengap threshold strategy} consists in clustering the eigenvalues whose relative eigengap $\delta_j := \frac{\ell_{j} - \ell_{j+1}}{\ell_j}$ is below a given threshold, e.g. the one of Thm.~\ref{appthm:releigengap}. This clustering uniquely determines a PSA type $\gamma$, from which we apply maximum likelihood estimation, i.e. we block-average the corresponding eigenvalue clusters.
This rule is extremely simple but it may select overly parsimonious models, since distant eigenvalues may end up in the same cluster by propagation. Therefore, we provide a more-advanced strategy in the following subsubsection.


\subsubsection{Hierarchical clustering of eigenvalues}
In this strategy, the subset of candidate models is generated by the \emph{hierarchical clustering} of the sample eigenvalues. The general principle of hierarchical clustering is to agglomerate one by one the eigenvalues into clusters, thanks to a so-called \emph{cluster-linkage criterion}, which is a measure of dissimilarity between clusters.
More precisely, here we choose a \textit{continuous} pairwise distance $\delta$ between adjacent eigenvalues (such as the relative eigengap defined in Thm.~\ref{appthm:releigengap}), and a linkage criterion $\Delta$ between eigenvalue clusters, making sense with respect to our model selection problem (such as the single-linkage criterion $\Delta(\Lambda_1, \Lambda_2) = \min_{\ell_1, \ell_2 \in \Lambda_1 \times \Lambda_2} \delta(\ell_1, \ell_2)$ or the centroid-linkage criterion $\Delta(\Lambda_1, \Lambda_2) = \delta(\overline{\Lambda_1}, \overline{\Lambda_2})$). 
The method is detailed in Algorithm~\ref{appalg:hierarchical} and illustrated in Figure~\ref{appfig:hierarchical_clustering}.
\begin{algorithm}[H]
   \caption{Hierarchical clustering strategy for PSA model selection}
   \label{appalg:hierarchical}
\begin{algorithmic}
   \STATE {\bfseries Input:} $\ell_1 \geq \dots \geq \ell_p, \Delta$ \hfill sample eigenvalues and distance
   \STATE {\bfseries Output:} $\lrp{{\gamma}^{t}}_{t=1}^p$ \hfill hierarchical subfamily of PSA models
   \STATE ${\gamma}^{1} \gets \lrp{1, \dots, 1}, \quad {{\Lambda}}^{1} \gets \lrp{\lrs{\ell_1}, \dots, \lrs{\ell_p}}$ \hfill initialize with full covariance model
   \FOR{$t = 1 \dots p-1$}
    \STATE $\Delta^{t} \gets \lrp{\Delta({\Lambda}^{t}_1, {\Lambda}^{t}_{2}), \dots, \Delta({\Lambda}^{t}_{p-t}, {\Lambda}^{t}_{p-t+1})}$ \hfill {compute distances between adjacent clusters}
    \STATE $k^{t} \gets \argmin \Delta^{t}$ \hfill {find clusters with minimal distance}
    \STATE ${{\Lambda}}^{t+1} \gets ({\Lambda}^t_1, \dots, {\Lambda}^t_{{k^t}-1}, {\Lambda}^t_{k^t} \cup {\Lambda}^t_{{k^t}+1}, {\Lambda}^t_{{k^t}+2}, \dots, {\Lambda}^t_d)$ % \overline{{{\lambda}}^{{\gamma}^{t+1}}}$
    \hfill {merge the two clusters of eigenvalues}
    \STATE ${\gamma}^{t+1} \gets ({\gamma}^t_1, \dots, {\gamma}^t_{{k^t}-1}, {\gamma}^t_{k^t} + {\gamma}^t_{{k^t}+1}, {\gamma}^t_{{k^t}+2}, \dots, {\gamma}^t_d)$ \hfill {update the model type}
\ENDFOR
\end{algorithmic}
\end{algorithm}
% Figure environment removed
\noindent The hierarchical clustering strategy creates a \emph{trajectory} $({\gamma}^t)_{t=1}^p$ in the Hasse diagram of PSA models (cf. Fig.~\ref{fig:hasse_complexity}). The sequence starts from ${\gamma}^1 = \lrp{1, \dots, 1}$, the full covariance model, in which each eigenvalue is in its own cluster. Then, one by one, the eigenvalues that are the closest in terms of distance $\Delta$ are agglomerated, and the inter-cluster distances are updated. The algorithm ends when one reaches the isotropic covariance model, ${\gamma}^p = \lrp{p}$, in which all the eigenvalues are in the same cluster. This corresponds to an \textit{agglomerative} approach in the hierarchical clustering vocabulary, in opposition to a \textit{divisive} approach, that we could similarly develop for this strategy.

The hierarchical clustering strategy hence generates a subfamily of $p$ models that can be then compared within a classical model selection framework. In order to assess the quality of such a strategy, we show the following consistency result.

\begin{proposition}[Asymptotic consistency of the hierarchical clustering strategy]\label{appprop:hierarchical_heuristic}
The hierarchical clustering strategy generates a subfamily of PSA models that almost surely contains the true PSA model for $n$ large enough.
\end{proposition}
\begin{proof}
Let us assume that the true generative model is stratified with type ${{\gamma}} \in \mathcal{C}(p)$. 
We can then write the population covariance matrix as ${{\Sigma}} = \sum_{k=1}^{d} \lambda_k {{Q}}_k {{{Q}}_k}\T$ with $~{\lambda_1 > \dots > \lambda_{d} > 0}$ and ${{Q}} := \lrb{{{Q}}_1|\dots|{{Q}}_{d}} \in \O(p)$. 
Let $n$ be the number of independent samples and ${{S}}_n := \sum_{j=1}^{p} \ell_j({{S}}_n) {{v}}_j({{S}}_n) {{{v}}_j({{S}}_n)}\T$ with $\ell_1 \geq \dots \geq \ell_p$ and ${{V}} := \lrb{{{v}}_1|\dots|{{v}}_p} \in \O(p)$. 
According to Tyler (1981), Lemma~2.1~(i), one then has almost surely, as $n$ goes to infinity, $\ell_j({{S}}_n) \to \lambda_{\phi_{{{\gamma}}}(j)}$, where $\phi_{{{\gamma}}}$ is the ${{\gamma}}$-composition function.
Hence for $n$ large enough, by continuity of the distance function $\Delta$, the gaps between eigenvalues in the same part of the ${{\gamma}}$-composition will be arbitrarily close to $0$, while the other will be arbitrarily close to the true values $\lrs{\Delta\lrp{\lambda_k, \lambda_{k+1}}, k \in \lrb{1, d-1}}$, which are all positive.
Hence the hierarchical clustering method will first agglomerate the eigenvalues that are in the same part of ${{\gamma}}$, and second the distinct blocks, by increasing order of pairwise distance. The last model of the first phase will be exactly the true model.
\end{proof}


\noindent Hence, the hierarchical clustering strategy generates a hierarchical subfamily of models of decreasing complexities, including the true PSA model for $n$ large enough. The true model can be then recovered using asymptotically consistent model selection criteria on the subfamily.
We now propose a second strategy that is not hierarchical but instead makes a prior assumption on the model complexity and then selects the one that has the maximum likelihood among all the candidates.

\subsubsection{Prior on the number of distinct eigenvalues}
In this strategy, we perform model selection at a given level of the Hasse diagram (cf. Fig.~\ref{fig:hasse_complexity}). More precisely, we consider as candidates only the models that have a given type length $d$, like done in IPPCA with $d=2$.
The type-length prior strategy reduces the search space like the previous strategy, this time to $\binom{p-1}{d-1}$ models. In contrast to the hierarchical clustering strategy which creates a hierarchy of models with decreasing complexity, we here rather fix the complexity range of the candidate models, by working on one floor of the Hasse diagram, and then try to find the model of best fit.

Just like in the hierarchical clustering strategy, we could use the BIC to choose the best model among this reduced family.
For completeness, we provide an additional criterion that is nothing but the maximum likelihood itself. 
We indeed manage to extend to PSA the surprising result from~\citet{bouveyron_intrinsic_2011} stating that the maximum likelihood criterion alone asymptotically consistently finds the true intrinsic dimension within the IPPCA setting. 
Intuitively, this can be explained by the fact that we a priori fix the complexity of the candidate models and therefore we can focus on the other side of the weighing scale that is the goodness of fit.
As this criterion empirically yields competitive results with respect to other classical model selection criteria in the large sample, low signal-to-noise ratio regime, we expect it to be of interest in PSA as well.
\begin{proposition}[Asymptotic consistency of the maximum likelihood for fixed $d$]\label{appprop:fixed_length_heuristic}
If the true PSA model has $d$ distinct eigenvalues, then maximum likelihood model selection within the subfamily of PSA models of type-length $d$ almost surely recovers the true model for $n$ large enough.
\end{proposition}
\begin{proof}
Let us assume that the true generative model is stratified with type $~{{{\gamma}}^*:=\lrp{\gamma_1^*, \dots, \gamma_d^*}}$, of length $d$, and let $\lambda_1 > \dots > \lambda_{d} > 0$ be the eigenvalues of the associated population covariance matrix.
Then, similarly as in the previous proof, almost surely, asymptotically, the sample covariance matrix eigenvalues are the ones of the population covariance matrix.
Hence, for any PSA model of type ${{\gamma}} := \lrp{\gamma_1, \dots, \gamma_d}$, the maximum likelihood writes
\begin{equation}
\ln{\hat{\mathcal{L}}} \sim -\frac n 2 \lrp{p \ln 2\pi + \sum_{k=1}^{d} \gamma_k \ln \lrp{\frac{1}{\gamma_k}\sum_{j \in \phi_{{{\gamma}}}^{-1} \lrs{k}} \lambda_{\phi_{{{\gamma}}^*}(j)}}}.
\end{equation}
As $n$ and $p$ are fixed when we compare the models, they do not intervene in the model selection. Hence, the search of the optimal model in terms of maximum likelihood boils down to the following problem 
\begin{equation}
    \argmin_{\substack{{{\gamma}} \in \mathcal{C}(p)\\ \#{{{\gamma}}}=d}}
    \sum_{k=1}^{d} \gamma_k \ln \lrp{\frac{1}{\gamma_k}\sum_{j \in \phi_{{{\gamma}}}^{-1} \lrs{k}} \lambda_{\phi_{{{\gamma}}^*}(j)}} := f({{\gamma}}).
\end{equation}
One has $f({{\gamma}}) = \sum_{k=1}^{d} \gamma_k \ln (\frac{1}{\gamma_k}\sum_{k'=1}^{d}
c_{kk'} \lambda_{k'})$, where $c_{kk'}$ is the cardinal of the intersection of the $k$-th part of ${{\gamma}}$ with the $k'$-th part of ${{\gamma}}^*$.
Then, by definition, one has $\sum_{k'=1}^{d} c_{kk'} = \gamma_k$ and $\sum_{k=1}^{d} c_{kk'} = {{\gamma}}^*_{k'}$. Hence, using Jensen's inequality,
\begin{equation}
f({{\gamma}}) \geq \sum_{k=1}^{d} \gamma_k \lrp{\sum_{k'=1}^{d} \frac{c_{kk'}}{\gamma_k}\ln \lambda_{k'}} = \sum_{k,k'=1}^{d} c_{kk'} \ln \lambda_{k'} = \sum_{k'=1}^{d} {{\gamma}}^*_{k'} \ln \lambda_{k'} = f({{\gamma}}^*).
\end{equation}
To conclude, asymptotically, ${{\gamma}}^*$-PSA is the most likely model. Hence, the maximum likelihood criterion alone finds the true model among the family of PSA models with the same type length.
\end{proof}


\noindent Hence we derived three simple strategies for model selection, taking into account the structure of the PSA models family. 
\begin{remark}
Many variants can be adopted depending on the problem at hand. For instance if the noise is known, or assumed with some explained variance ratio rules, one can first search for the associated intrinsic dimension $q$ like in classical PCA, and then try to equalize some of the $q$ first eigenvalues by optimizing the model selection criterion over the subfamily of models whose $p - q$ last eigenvalues are all equal.
\end{remark}
\begin{remark}
In high dimensions, some eigenvalues might be very small or even null. The case of small positive eigenvalues may yield high relative eigengaps in the last eigenvalue pairs---therefore PSA model selection tends to separate those eigenvalues---whereas those are traditionally considered as noise. The case of null eigenvalues yields undefined PSA models. To circumvent those two issues, a classical trick is the one of \textit{covariance regularization}, consisting in adding a small constant to all the covariance eigenvalues. This somewhat boils down to adding an isotropic Gaussian noise to the data. This has notably the effect of diminishing the relative eigengaps, especially for the small positive or null eigenvalues. Another possibility would be to constrain the model types to have at least the last $p - q$ eigenvalues equal, where $q$ is chosen sufficiently small such that the first $q$ eigenvalues are large enough. This is left for future research.
\end{remark}
\section{Experimental Evaluations}\label{sec:experiment}

\textbf{Implementation.}
We implement \puma\ on top of SecretFlow~\citep{spu} in \textrm{C++} and Python. SecretFlow compiles a high-level Flax code to secure computation protocols, which are then executed by our designed cryptographic backends, and we encode the floating-ponit values as $64$-bit integers in ring $\mathbb{Z}_{2^{64}}$ with $18$-bit fractional part. 
Our experiments are run on 3 Alibaba Cloud ecs.g7.8xlarge servers with 32 vCPU and 128GB RAM each. The CPU model is Intel Xeon(Ice Lake) Platinum 8369B CPU @ 2.70GHz. We evaluate \puma\ on Ubuntu 20.04.6 LTS with Linux kernel 5.4.0-144-generic. Our bandwidth is about 5Gbps and round trip time is about 1ms. %\cheng{Describe fixed point parameters: scale, share bits.}

\textbf{Models \& Datasets.}
We evaluate \puma\ on seven NLP models: Bert-Base, Roberta-Base, and Bert-Large~\citep{bert}; GPT2-Base, GPT2-Medium, and GPT2-Large~\citep{gpt}; and LLaMA-7B~\citep{touvron2023llama}. We measure the Bert performance for three NLP tasks over the datasets of Corpus of Linguistic Acceptability (CoLA), Recognizing Textual Entailment (RTE), Stanford Question Answering Dataset (QNLI) from GLUE benchmarks~\citep{wang2018glue}, and GPT2 performance on Wikitext-103 V1~\citep{merity2016pointer}.

\textbf{Baseline.}
We compare \puma\ to the most similar prior work \mpcformer~\citep{li2023mpcformer}. But for fair comparison, we have the following considerations:
\romannumeral1) As \mpcformer\ neither supports loading pretrained transformer models nor implements LayerNorm faithfully\footnote{ As \mpcformer~does not support loading pre-trained Transformer models, we did an experiment in plaintext Bert-Base that replaced LayerNorm with BatchNorm  as \mpcformer~did. This  resulted in a significant drop in the MCC score for CoLA task from $0.616$ to $-0.020$. On the contrary, \puma~achieves an MCC score of $0.613$. }, we cannot achieve meaningful secure inference results using their framework.
Therefore, we compare our secure Transformer models inference performance to that of plaintext (floating-point) to show our precision guarantee.
\romannumeral2) \mpcformer\ with \textit{Quad} approximations (for both $\gelu$ and $\softmax$) requires retraining the  modified models. As \puma\ does not require retraining, we compare our cost to that of \mpcformer\ without \textit{Quad} approximations. Also, we re-run \mpcformer~in our environment.



\subsection{Precision}\label{sec:accuracy}

% Figure environment removed

%\begin{table}
\centering
\caption{Performance on GLUE benchmark of Bert-Base, Roberta-Base, and Bert-Large on CoLA, RTE, and QNLI, Matthews correlation is reported for CoLA. Accuracy is reported for other datasets.}\label{table:bertacc}
\begin{tabular}{c|ccc|ccc|ccc}
\hline \hline
 Model & \multicolumn{3}{c|}{Bert-Base} & \multicolumn{3}{c|}{Roberta-Base} & \multicolumn{3}{c}{Bert-Large} \\ \hline
 TASK & CoLA & RTE & QNLI & CoLA & RTE & QNLI & CoLA & RTE & QNLI \\ \hline
CPU & $0.616$     & $0.700$      & $0.916$     & $0.629$ & $0.805$ & $0.920$  & $0.686$   & $0.755$ & $0.922$ \\
\puma   & $0.613$     & $0.700$     & $0.916$     & $0.618$ & $0.805$ & $0.918$ & $0.690$ & $0.747$ & $0.918$ \\ \hline \hline
\end{tabular}
\end{table}

\begin{table}[]
    \centering
    \caption{Perplexity of GPT2-Base, GPT2-Medium, and GPT2-Large on Wikitext-103 V1.}
    \label{tab:gpot2ppl}
    \begin{tabular}{c|c|c|c}
    \hline \hline
      Model & GPT2-Base & GPT2-Medium & GPT2-Large \\ \hline
      CPU & $16.284$ & $12.536$ & $10.142$ \\
      \puma & $16.284$ & $12.540$ & $10.161$ \\
      \hline \hline
    \end{tabular}
    
\end{table}

We compare our secure model 
inference performance to that of plaintext (floating-point) in Figure~\ref{fig:performance} to show our precision guarantee.

In Figure~\ref{fig:bert-base}-\ref{fig:bert-large}, we show the Matthews correlation/accuracy of plaintext and \puma\ on the Bert-Base, Roberta-base, and Bert-Large. We observe that the accuracy achieved by \puma~ matches the accuracy of the plaintext Flax code. Specifically, the accuracy difference does
not exceed $0.011$ over all datasets. 

Moreover, in Figure~\ref{fig:gpt2}, we also compare our perplexity on dataset Wikitext-103 V1 with the plaintext baseline on models GPT2-Base, GPT2-Medium, and GPT2-Large. The results are similar and the perplexity differences do not exceed $0.02$ over all models.

The above accuracy and perplexity advantages experimentally validate that our protocols are numerically precise. 

\subsection{Inference cost}\label{sec:efficiency}
\begin{table}[h]
    \centering
    \caption{Costs of Bert-Base, Roberta-Base, and Bert-Large for one sentence of length $128$. Time is in seconds and Communication (Comm. for short) is in GB, which is the same for the following tables.}\label{tab:costbert}
    \begin{tabular}{c|cc|cc|cc}
    \hline \hline
       Model & \multicolumn{2}{c|}{Bert-Base} & \multicolumn{2}{c|}{Roberta-Base} & \multicolumn{2}{c}{Bert-Large} \\ \hline
       Costs & Time & Comm. & Time & Comm. & Time & Comm. \\ \hline
       \mpcformer & $55.320$ & $12.089$ & $57.256$ & $12.373$ & $141.222$ & $32.577$ \\
       \puma & $33.913$ & $10.773$ & $41.641$ & $11.463$ & $73.720$ & $27.246$ \\
       \cellcolor{mygray} Improv. & \cellcolor{mygray} $1.631\times$ & \cellcolor{mygray} $1.122\times$ & \cellcolor{mygray} $1.375\times$ & \cellcolor{mygray} $1.079\times$ & \cellcolor{mygray} $1.916\times$ & \cellcolor{mygray} $1.195\times$ \\
       \hline \hline
    \end{tabular}
    \vspace{-0.2cm}
\end{table}

\begin{table}[]
    \centering
    \caption{Costs of GPT2-Base, GPT2-Medium, and GPT2-Large. The input sentence is of length $32$, all of the costs are for generating $1$ token.}\label{tab:costgpt2}
    \begin{tabular}{c|cc|cc|cc}
    \hline \hline
       Model & \multicolumn{2}{c|}{GPT2-Base} & \multicolumn{2}{c|}{GPT2-Medium} & \multicolumn{2}{c}{GPT2-Large} \\ \hline
       Costs & Time & Comm. & Time & Comm. & Time & Comm. \\ \hline
       \mpcformer & $34.889$ & $4.999$ & $73.078$ & $11.766$ & $129.095$ & $22.522$  \\
       \puma & $15.506$ & $3.774$ & $30.272$ & $7.059$ & $54.154$ & $11.952$ \\
       \cellcolor{mygray} Improv. & \cellcolor{mygray} $2.250\times$ & \cellcolor{mygray} $1.325\times$ & \cellcolor{mygray} $2.414\times$ & \cellcolor{mygray} $1.667\times$ & \cellcolor{mygray} $2.383\times$ & \cellcolor{mygray} $1.884\times$ \\
       \hline \hline
    \end{tabular}
    \vspace{-0.2cm}
\end{table}

In this subsection, we compare \puma's inference cost to that of \mpcformer. 
We evaluate  three Bert models (Bert-Base, Roberta-Base, and Bert-Large) and three GPT2 models (GPT2-Base, GPT2-Medium, and GPT2-Large).
The costs are for processing one input sentence: \romannumeral1) For Bert models the input sentence is of length $128$. \romannumeral2) GPT2 models input one length-32 sentence and generate $1$ new word. 

On the 3 Bert models in Table~\ref{tab:costbert}, \puma\ is  $1.375\sim 1.916\times$ faster than  \mpcformer, and is $1.079\sim 1.195\times$ more communication-efficient. For the GPT2 models in Table~\ref{tab:costgpt2}, \puma\ is $2.250\sim 2.414\times$ faster than \mpcformer, and is $1.325\sim 1.884\times$ more communication-efficient. 
    
We observe that \puma's improvements increase as the model size grows, particularly for the GPT2 models. This trend is because our specialized optimizations are more effective when processing large-scale evaluations.



\subsection{Scalability}\label{sec:scala}

In this subsection, we measure the costs of evaluating \puma\ on Bert-Base and GPT2-Base models for varying-length inputs, and varying-length outputs (only for GPT2-Base). We also compare our costs to those of \mpcformer~to demonstrate our improvements.





\begin{table}[]
    \centering
    \caption{Costs of Bert-Base and GPT2-Base for different input length (denoted as \#Input). The input lengths for Bert-Base and GPT2-Base are respective $\{64, 128, 256, 512\}$ and $\{16, 32, 64, 128\}$. GPT2-Base generates $1$ token.}\label{tab:costbertinput}
    \begin{tabular}{cc|cc|cc|cc|cc}
    \hline \hline
       \multicolumn{2}{c|}{\#Input} & \multicolumn{2}{c|}{$64 / 16$} & \multicolumn{2}{c|}{$128 / 32$} & \multicolumn{2}{c|}{$256 / 64$} & \multicolumn{2}{c}{$512 / 128$}  \\ \hline
       \multicolumn{2}{c|}{Costs} & Time & Comm. & Time & Comm. & Time & Comm. & Time & Comm. \\ \hline
       \multirow{3}{*}{Bert}& \mpcformer & $46.428$ & $4.750$ & $85.887$ & $9.673$ & $196.372$ & $23.443$ & $582.787$ & $68.069$ \\
       & \puma & $24.345$ & $1.627$ & $42.525$ & $3.591$ & $87.561$ & $8.668$ & $212.600$ & $23.439$\\
       & \cellcolor{mygray} Improv. & \cellcolor{mygray} $1.907\times$ & \cellcolor{mygray} $2.919\times$ & \cellcolor{mygray} $2.020\times$ & \cellcolor{mygray} $2.694\times$ & \cellcolor{mygray} $2.243\times$ & \cellcolor{mygray} $2.705\times$ & \cellcolor{mygray} $2.741\times$ & \cellcolor{mygray} $2.904$ \\
       \hline
       \multirow{3}{*}{GPT2}& \mpcformer & $34.522$ & $3.767$ & $42.615$ & $4.516$ & $60.451$ & $6.281$ & $105.028$ & $11.225$  \\
       & \puma & $20.692$ & $0.625$ & $29.248$ & $1.258$ & $40.968$ & $2.607$ & $74.529$ & $5.611$\\
       &\cellcolor{mygray} Improv. & \cellcolor{mygray} $1.668\times$ & \cellcolor{mygray} $6.027\times$ & \cellcolor{mygray} $1.457\times$ & \cellcolor{mygray} $3.590\times$ & \cellcolor{mygray} $1.476\times$ & \cellcolor{mygray} $2.409\times$ & \cellcolor{mygray} $1.409\times$ & \cellcolor{mygray} $2.001\times$\\
       \hline \hline
    \end{tabular}
\end{table}
\textbf{Input Length Evaluation.}
Table~\ref{tab:costbertinput} shows our costs on varying-length inputs, we evaluate Bert-Base on the inputs of length $\{64, 128, 256, 512\}$, and GPT2-Base on the inputs of length $\{16, 32, 64, 128\}$.
For Bert-Base, \puma\ is $1.720\sim 2.282\times$ faster, and for GPT2-Base, \puma\ is $1.550\sim 2.686\times$ faster. Unlike the observations in Section~\ref{sec:efficiency}, our efficiency gains decrease with increasing input sizes in GPT2, and \puma\ requires more communication when the input length is greater than 64. This phenomenon is attributed to the interesting fact: To directly support pre-trained plaintext models, \puma\ strictly follows the plaintext model format that only accept token ids as input, so \puma\ has to compute the one-hot vectors from token ids in an MPC way. On the other hand, \mpcformer\ uses modified models that accept one-hot vectors as input, so the one-hot function could be computed at the client side in plaintext. Nevertheless, \puma\ remains faster than \mpcformer.

%\begin{table}[]
    \centering
    \caption{Costs of GPT2-small for generating different output tokens (denoted as \#Output), the input length is set as $32$.}\label{tab:costgpt2tokens}
    \begin{tabular}{c|cc|cc|cc|cc}
    \hline \hline
       \#Output & \multicolumn{2}{c|}{2} & \multicolumn{2}{c|}{4} & \multicolumn{2}{c|}{8} & \multicolumn{2}{c}{16}  \\ \hline
       Costs & Time & Comm. & Time & Comm. & Time & Comm. & Time & Comm. \\ \hline
       \mpcformer & $72.833$ & $7.676$ & $132.644$ & $13.998$ & $252.796$ & $26.648$ & $494.509$ & $51.972$ \\
       \puma & $53.191$ & $2.549$ & $111.457$ & $5.167$ & $215.352$ & $11.115$ & $457.994$ & $24.917$ \\
       Improv. & $1.369\times$ & $3.011\times$ & $1.190\times$ & $2.709\times$ & $1.174\times$ & $2.397\times$ & $1.080\times$ & $2.086\times$ \\
       \hline \hline
    \end{tabular}
\end{table}

\begin{wrapfigure}{r}{0.4\textwidth}
    % Figure removed
    \caption{Runtime of GPT2-Base for generating different number of output tokens, the input length is of length $32$.} 
    \label{fig:gptwoutcosts}
\end{wrapfigure}

\textbf{Output Length Evaluation.}
Fig~\ref{fig:gptwoutcosts} presents our costs on varying-length outputs for GPT2-Base, and compares our costs to those of \mpcformer. Our improvements in runtime range from $1.279\sim 2.700\times$ respectively.
As more output tokens are generated, both costs increase in a linear way, this is because each output token must be input back into the model to generate the next token, increasing the required one-hot embedding costs. We should emphasize
again that although the time costs might be close for long outputs, \puma\ could achieve a similar accuracy as plaintext models while \mpcformer\  could not. 


\begin{table}[]
    \centering
    \caption{Costs of the secure inference of LLaMA-7B, \#Input denotes the length of input sentence and \#Output denotes the number of generated tokens.}\label{tab:llama7b}
    \begin{tabular}{c|cc|cc|cc}
    \hline \hline
       (\#Input, \#Output) & \multicolumn{2}{c|}{$(4,1)$} & \multicolumn{2}{c|}{$(8,1)$} & \multicolumn{2}{c}{$(8,2)$} \\ \hline
       Costs & Time & Comm. & Time & Comm. & Time & Comm. \\ \hline
       \puma & $122.004$ & $0.907$ & $200.473$ & $1.794$ & $364.527$ & $3.857$ \\
       \hline \hline
    \end{tabular}
    \vspace{-0.2cm}
\end{table}

\textbf{Scale to LLaMA-7B in Five Minutes.}
We evaluated the large language model LLaMA-7B using \puma\ under 3 Alibaba Cloud
ecs.r7.32xlarge servers, each has 128 threads and 1TB RAM, with 20GB bandwidth, 0.06ms round-trip-time. 
As shown in Table~\ref{tab:llama7b}, \puma\ can support the secure inference of large language model LLaMA-7B with reasonable costs. For example, given an input sentence of 8 tokens, \puma\ can output one token in around $346.126$ seconds with communication costs of $1.865$ GB. To our knowledge, this is the first time that LLaMA-7B has been evaluated using MPC.


%Llama-7B, LAN=(20GB, 0.06ms), 128 threads, input length=8, output=1 token, costs: 346.126s, 2002213760 bytes
\section{Information about datasets}\label{appsec:data}
In this section, we give a few more details about the data used for the experiments.

\subsection{Natural image patches}
In this experiment, we consider 10 flower images from the ImageNet database~\cite{deng_imagenet_2009}. Those were downloaded from Kaggle (\url{https://www.kaggle.com/datasets/prasunroy/natural-images}) and extracted from the folder \texttt{natural\_images/flower/} from \texttt{flower\_0000.jpg} up to \texttt{flower\_0009.jpg}.

\subsection{Eigenfaces}
In this experiment, we consider 31 digital images from the CMU Face Images database~\cite{mitchell_cmu_1997}. Those were downloaded from Kaggle (\url{https://www.kaggle.com/datasets/raviprakash22/cmu-face-images}) and extracted from the folder \texttt{faces/faces/choon}. We only extracted the $(60, 64)$ images, corresponding to all the files ending with \texttt{\_2.pgm}.

\subsection{Structured data}
For the structured data experiment (cf. Fig.~\ref{fig:exp_rotation}) and the relative eigengap tables (cf. Fig.~\ref{fig:releigengap_UCI} and Fig.~\ref{appfig:releigengaps_real}), we consider data from the UCI Machine Learning Repository (\url{https://archive.ics.uci.edu/}): Ionosphere~\cite{sigillito_ionosphere_1989}, Wine~\cite{aeberhard_wine_1991}, Wisconsin~\cite{wolberg_breast_1995}, Glass~\cite{german_glass_1987}, Iris~\cite{fisher_iris_1936}, Spambase~\cite{hopkins_spambase_1999}, Digits~\cite{alpaydin_optical_1998}, Covertype~\cite{blackard_covertype_1998}.
%%%%%%%%%%%%%%%%%%%%%%%%%%%%%%%%%%%%%%%%%%%%%%%%%%%%%%%%%%%%%%%%%%%%%%%%%%%%%%%
%%%%%%%%%%%%%%%%%%%%%%%%%%%%%%%%%%%%%%%%%%%%%%%%%%%%%%%%%%%%%%%%%%%%%%%%%%%%%%%

\bibliography{ref}
\bibliographystyle{icml2024}

\end{document}


% This document was modified from the file originally made available by
% Pat Langley and Andrea Danyluk for ICML-2K. This version was created
% by Iain Murray in 2018, and modified by Alexandre Bouchard in
% 2019 and 2021 and by Csaba Szepesvari, Gang Niu and Sivan Sabato in 2022.
% Modified again in 2023 and 2024 by Sivan Sabato and Jonathan Scarlett.
% Previous contributors include Dan Roy, Lise Getoor and Tobias
% Scheffer, which was slightly modified from the 2010 version by
% Thorsten Joachims & Johannes Fuernkranz, slightly modified from the
% 2009 version by Kiri Wagstaff and Sam Roweis's 2008 version, which is
% slightly modified from Prasad Tadepalli's 2007 version which is a
% lightly changed version of the previous year's version by Andrew
% Moore, which was in turn edited from those of Kristian Kersting and
% Codrina Lauth. Alex Smola contributed to the algorithmic style files.
