\documentclass[letterpaper, 10 pt, conference]{ieeeconf}  % Comment this line out if you need a4paper

%\documentclass[a4paper, 10pt, conference]{ieeeconf}      % Use this line for a4 paper

\IEEEoverridecommandlockouts                              % This command is only needed if 
                                                          % you want to use the \thanks command

\overrideIEEEmargins                                      % Needed to meet printer requirements.

\usepackage{graphics} % for pdf, bitmapped graphics files
\usepackage{epsfig} % for postscript graphics files
\usepackage{mathptmx} % assumes new font selection scheme installed
\usepackage{times} % assumes new font selection scheme installed
\usepackage{amsmath} % assumes amsmath package installed
\usepackage{amssymb}  % assumes amsmath package installed
\usepackage{multirow}
\usepackage{color,soul}
\usepackage{algorithm}
\usepackage{siunitx}
\usepackage{algpseudocode}
% \usepackage{nidanfloat} % may not work for submission

\usepackage{hyperref}
%\usepackage[natbib=true]{biblatex}
\usepackage[comma,numbers]{natbib}
\renewcommand*{\bibfont}{\footnotesize}


\bibliographystyle{IEEEtran}

\addtolength{\topmargin}{4pt}

\title{\LARGE \bf
Authoring and Operating Humanoid Behaviors On the Fly using Coactive Design Principles
}

\author{Duncan Calvert$^{1,2}$,
Dexton Anderson$^{1}$,
Tomasz Bialek$^{1}$,
Stephen McCrory$^{1,2}$, \\
%Bhavyansh Mishra$^{1,2}$,
%Sylvain Bertrand$^{1}$,
%Matt Johnson$^{1}$,
Luigi Penco$^{1}$,
Jerry Pratt$^{1,2,3}$
and Robert Griffin$^{1,2}$% <-this % stops a space
\thanks{This work was funded through ONR Grant N00014-19-1-2023, NASA Grant No. 80NSSC20M0197, and ARL Cooperative Agreement W911NF-21-2-0241.}% <-this % stops a space
\thanks{$^{1}$The authors are with the Florida Institute for Human and Machine Cognition, 40 S Alcaniz St, Pensacola, FL 32502, United States}%
\thanks{$^{2}$The author are with the University of West Florida, 11000 University Pkwy, Pensacola, FL 32514, United States}%
\thanks{$^{3}$The author is with Figure AI, Inc., Sunnyvale, CA, United States}%
\thanks{Email : \url{{dcalvert, danderson, tbialek, smccrory, lpenco, jpratt, rgriffin}@ihmc.org}
}} 
%
\usepackage{amsmath}
\usepackage{mathtools}
\usepackage{thmtools}
\usepackage{cancel}
\usepackage{wrapfig}


\newcommand{\cmark}{\ding{51}}%
\newcommand{\xmark}{\ding{55}}%

\usepackage{tcolorbox}
\tcbset{boxsep=0mm,boxrule=0pt,colframe=white,arc=0mm,left=0.5mm,right=0.5mm}

\newcommand\SC{\mathcal{S}}
\newcommand{\antonio}[1]{{\color{magenta} Antonio: ``#1''}}

\DeclareMathOperator*{\argmax}{arg\,max}
\DeclareMathOperator*{\argmin}{arg\,min}

\newcommand{\theHalgorithm}{\arabic{algorithm}}

\usepackage[capitalize,noabbrev]{cleveref}


\DeclareMathOperator{\LRU}{LRU}

\newcommand\A{\mathbf{A}}
\newcommand\V{\mathbf{V}}
\newcommand\B{\mathbf{B}}
\newcommand\C{\mathbb{C}}
\newcommand\Exp{\mathbb{E}}
\newcommand\R{\mathbb{R}}
\newcommand\calM{\mathcal{M}}
\newcommand\calR{\mathcal{R}}
\newcommand\rank{\operatorname{rank}}
\newcommand\eps{\varepsilon}
\newcommand\h{h}
\newcommand\bound{b}
\DeclareMathOperator{\tr}{tr}
\DeclareMathOperator{\vect}{vec}
\DeclareMathOperator{\diag}{diag}




\begin{document}

\maketitle
\thispagestyle{empty}
\pagestyle{empty}

\begin{abstract}
Humanoid robots have the potential to  perform useful tasks in a world built for humans.
However, communicating intention and teaming with a humanoid robot is a multi-faceted and complex problem.
In this paper, we tackle the problems associated with quickly and interactively authoring new robot behavior that works on real hardware.
We bring the powerful concepts of Affordance Templates and Coactive Design methodology to this problem to attempt to solve and explain it.
In our approach we use interactive stance and hand pose goals along with other types of actions to author humanoid robot behavior on the fly.
% We then describe how Coactive Design is applied during the authoring process and provide interdependence analysis charts.
We then describe how our operator interface works to author behaviors on the fly and provide interdependence analysis charts for task approach and door opening.
We present timings from real robot performances for traversing a push door and doing a pick and place task on our Nadia humanoid robot.
%We approach this with an application of core computer science concepts and disciplined software engineering which enables us to experiment atop %fundamental constraints and open up opportunities to remove artificial ones.
%We present in detail our approach, execution, and results on real humanoid robot hardware.

\end{abstract}

\section{Introduction}
\label{introduction}

% OVERALL PURPOSE
The humanoid form has uniquely diverse mobility and manipulation capabilities that drive its suitability as the embodiment of a general purpose robot.
This has lead to the pursuit of building humanoid robots to perform useful tasks in spaces designed for humans.
%We believe that the humanoid form may be optimal as a general purpose utility because of it's uniquely superior mobility and manipulation abilities.
There are a number of promising humanoid robot platforms in the world today\cite{bostondynamics2023atlas, promat2023recap}, but humanoid robots that are economical and general purpose are likely still a decade or more away.
%However, they are not yet profitable or general purpose and the field of humanoid robotics has never moved at a faster pace.
One component that is missing is the ability to quickly and effectively get the robot to perform useful tasks with a minimal amount of human supervision, which we refer to as behavior authoring.
This paper explores techniques that build upon known, useful principles in the literature in an effort to nudge the state of the art of behavior authoring on humanoid robots forward.

% HIGH LEVEL FOCUS, STANCE, AND POINT
%In this paper, we focus on a search for better ways to work with them.
Behavior \textit{authoring} is the process in which a human operator assembles a system of actionable instructions for the robot to execute a task.
When authoring is concluded, the robot should possess the ability to perform that task in an automatic fashion with high reliability.
%In this work, we specifically focus on the speed at which an operator can author new robot behavior that works on real hardware.
In this work, we focus on an interface for authoring behavior ``on the fly" and the interdependence between the operator and robot during the behavior authoring process.
We use the phrase ``on the fly" in this context to mean that the operator is able to create, modify and execute task components while the robot is powered on and in the field.
% creates new tasks and executes parts of them and replays parts of them while tweaking them.

% Figure environment removed

Our system is a new implementation inspired by affordance template architecture in which we have strived to provide a foundation for providing powerful interdependence during operation.
% is qualified using Coactive Design principles and interdependence analysis, and is built from the ground up through disciplined software engineering.
We try to tighten the feedback loop of experimentation and validation in the authoring process, facilitating authoring speed and enabling exploration of a larger set of possible solutions in a realtime setting.
% and unlocking freedom of expression as a pathway for finding unexpected solutions.
% We also show how our system has an ever-growing list of missing and desirable features and how interdependence analysis is used to unearth and prioritize them.
We present interdependence analysis charts for two common actions: task approach and door handle manipulation.
This analysis allows us to formally zoom in on the interactions between the robot and the operator during the authoring process.
We demonstrate the potential of this framework by performing and presenting successful trials on real robot hardware.
%Instead of side-stepping them, we embrace and highlight them.
%Unearthing and prioritizing potential improvements is a key strength of Coactive Design.
In this paper, we present the following contributions:
\begin{enumerate}
    \item A description of key interface elements and how they help the operator interact with the robot.
    \item Discussion of how our operator interface works to author behavior on the fly.
%    \item A methodology for designing and analyzing a system for rapidly authoring humanoid robot behaviors.
    \item Interdependence analysis charts for task approach and door opening.
    \item Timings of real robot authoring and subsequent automatic execution for pick and place and door traversal.
\end{enumerate}


\section{Humanoid Behavior Authoring}

We abstract humanoid behavior into two primary categories: mobility and manipulation.
\textit{Mobility} is getting somewhere and \textit{manipulation} is doing something once you get there.
We build a hierarchy of abstractions as we dive into each.

Mobility itself has several fields of research that comprise it, including collision-free path planning\cite{Lin_2021}, contact sequence planning\cite{Griffin_2019}, dynamic motion planning\cite{Egle_2022}, and balance control\cite{Koolen_2016}.
% whole body control, balance control, gait planning, footstep planning, navigation, multi-contact planning, and collision-free traversal.
Manipulation also has entire fields of research associated with it, which include inverse kinematics\cite{Beeson_2015}, collision-free trajectory planning\cite{MoveIt_2019}, grasp generation\cite{mousavian20196dof}, and semantic planning\cite{driess2023palme}.
%Mobility and manipulation are very closely coupled and in the real world depend on perception in the real world, 
Additionally, for autonomy, mobility and manipulation depend on perception, which is a mature research field in its own right. Topics in perception include computer vision, sensor design, semantic segmentation, object pose estimation, SLAM, and scene graphs.

In this work, we focus on navigation through operator placed stance pose goals and assume that sufficient planning and control is available for execution.
Likewise, for manipulation we focus only on inverse kinematics to achieve hand poses and ``open" and ``close" hand configurations.
For perception, we provide the operator with a view of a colored point cloud and reference frames for virtual scene graph objects which are detected using fiducial markers.
We chose these fundamental elements to provide a basic framework and basis for future expansion.

The authoring process should allow the operator to quickly construct new behavior that can later be run in an automatic but optionally supervised mode that executes at near human speed.
The key and qualitative and quantitative measurements of value are:
\begin{enumerate}
  \item The time taken by the operator to author a given task.
  \item The complexity and usefulness of the task.
  \item The degree of human assistance required during execution after authoring.
  \item The robustness and reliability or success rate of the behavior.
\end{enumerate}

In this paper we focus on two basic behaviors of humanoid robots that are fundamental for doing useful work: door traversal and pick and place.

\subsection{Door Traversal}

Door traversals illustrate complexity in mobility.
Traversing doors is not in itself useful, but a means to accomplishing something else.
It is also a task that is uniquely suited to the humanoid form -- most doors are designed for humans.
Doors have handles that are relatively high and designed for human hands to manipulate.
It is also beneficial to use two arms to traverse a door: one to pull or push it open and the other to keep it open.
Door frames are relatively narrow spaces which could require a humanoid to turn sideways where large wheeled or multi-legged bases cannot.
In this paper we focus on a common type of door with a lever handle on one side, a hinge on the opposite side, that swings open only one way, and does not have a automatic closer.
For this type, a push side traversal is generally comprised of the following parts:
\begin{enumerate}
  \item Approach the door near enough to reach the handle.
  \item Grasp and turn or push on the handle enough to disengage the latch.
  \item Push the door open.
%  \item If there is a spring and damper, hold the door open with the hand or arm nearest the hinge.
  \item Walk through the door.
\end{enumerate}

Traversing the door from the pull side is more complex because you must avoid the door swinging towards you \textit{and} temporarily hold the door open with the hand or arm opposite the hinge side first before transferring that role to the other hand or arm.

\subsection{Pick and Place}

A pick a place task is among the simplest of useful manipulation tasks.
We do not add any further complexity to this task in this work, keeping it to a set of straightforward steps:
\begin{enumerate}
  \item Identify the pose of the object.
  \item Approach near enough to reach the object.
  \item Grasp and lift up the object.
  \item Place the object somewhere else.
\end{enumerate}

% Some things have natural Robustness, like when you don't have to be that precise.
% When failing to pull the handle is fine -- it springs back.
% In the nominal case for most tasks under consideration, behaviors will execute automatically without the need for operator intervention.
% Manipulation parts: home robot, stance pose, hand configuration, grasp approach, pre-grasp, grasp, manipulate, release grasp, grasp departure, home robot

\section{Related Work}

% AFFORDANCE TEMPLATES
The Affordance Template (AT) framework\cite{Hart_2014, Hart_2015, Hart_2022} is a primary inspiration for this work.
It provides an integrated environment for authoring templates for tasks and provides a general definition language for robot-agnostic manipulation.
Examples of ATs include interactive solutions for humanoid robots to pick and place items, operate industrial valves, and open a car door to retrieve an object.
ATs provide support for advanced planning such as stance and grasp generation, navigation and motion planning, and motion primitives to abstract common physical manipulation interactions.
Work on ATs have not explored on the fly authoring.
%There are two primary differences with our approach and ATs as published in the literature.
%First, our approach reimplements ATs outside of the ROS ecosystem and, secondly, we directly explore on the fly authoring.

%\hl{Rethink Intera} Nah

% MIT Director?
MIT's Director, developed for the DARPA Robotics Challenge (DRC), used affordance template concepts, a rebuilt framework to integrate robot autonomy components, and a task execution framework\cite{Marion_2017}.
It was used very successfully to score 16 points in the finals.
It includes an operator in the loop pipeline for executing actions, an intuitive 3D scene with interactive widgets, and a way to make custom panels of widgets.
An embedded Python programming environment was used to write task scripts, however, the authoring process is not detailed.

% mc_rtc?
mc\_rtc is an integrated framework for managing robot behavior that supports bringing your own robot model as a URDF.
It allows the user to write behaviors that can be run on real robot and simulation using the same interface\cite{Singh_2023}.
It has been used to get humanoid robots walking up stairs, driving vehicles, and performing industrial manipulation tasks.
The framework allows the developer to programmatically construct finite state machines that can then be operated by the user interface.
However, mc\_rtc does not have an interactive behavior authoring interface.

% Figure environment removed

% COACTIVE DESIGN AND INTERDEPENDENCE ANALYSIS
The Coactive Design method\cite{Johnson_2014} is an iterative process comprised of three main processes: an identification process, a selection and implementation process, and an evaluation of change process.
The most complex process is in the identification process in which requirements, alternatives, and interdependence relationships are explored.
A set of desired interdependence relationships are determined and selected for implementation.
The result is then evaluated using human feedback and performance analysis.
This method was used for the design and development of the operator interface used by IHMC in the 2015 DRC\cite{Johnson_2014}.
A later analysis details how that methodology led to success in the competition\cite{Johnson_2017}.
The car egress shown in \autoref{fig:drc_egress} used a scripting engine which allowed the operator to cycle through a sequence of predefined actions.
This scripting engine is a precursor to the presented work in this paper, which has been rewritten with heavy reference to the original and applies lessons learned at a base architectural level.

%\subsection{IHMC's Humanoid Teleoperation Interface}
%\hl{Discuss interactable footstep placement, stance pose control ring, VR support, kinematics streaming, etc.}

\section{Authoring Interface}

%We take a vertically integrated software approach which allows us to experiment at the level of fundamental constraints and %affords us opportunities to remove artificial ones at any level.
%We feel that frameworks like ROS\cite{Macenski_2022} and Unity\cite{unity_website}, often used for this type of work, are %too all-consuming.
%We find that these frameworks impose artificial ``lines in the sand" with respect to what is possible to assemble %computationally.
%The tradeoff is that our approach does, however, rely on disciplined software development practices and competence in %fundamental computer science concepts.
%% performed by talented engineers who are
%Attacking problems in computer graphics, multi-threading, data structures, algorithms, and GPU computing has been necessary %to implement this work.

Our interface is designed to put the operator and robot in a situation where they are engaged in rich interaction with data.
It is an environment in which behaviors can be created from scratch, existing behaviors modified, and end-to-end tested while the robot is powered on and performing action in the field.
We refer to this as ``on the fly" authoring because it can be used to accomplish and automate tasks as they are encountered.
Achieving this requires elements to have observability, predictablilty, and directability.
%Coative design has three main phases:
%Coactive design is building these properties into UI elements during the design phase.
%and as long as they require only the collection of available abilities present in our system.
% Design centers on managing constraints and we are able to design with relationship to fundamental constraints rather than artificial ones.


% INTERACTION MEDIUM
A 3D scene is the central focus, where behavior keyframes and widgets are laid out in world space, as seen in \autoref{fig:push_door_behavior_layout}.
The user can orbit the camera and move the focus point with the mouse and keyboard.
Visualizing behavior data in 3D allows the operator to inspect spatial relationships which is a fundamental part of verification.
% The operator attains situational awareness using 3D viewport to identify robot and the task locations in the world and observes the spatial relationship between them.
If this information was not visualized, the operator would be forced to doubt the validity and intention of the behavior.
They would need to hold in mind questions like ``Is the next hand pose where I think it is?" and ``Where does the robot end up at the end of the task?".
Sometimes, there are bugs and issues where the task actions would be completely somewhere else in the world.
With this approach, it is intuitive for the operator to verify the alignment of task actions to the task.

% Questions like "will the robot walk anywhere during this?", "how high/low are the arms going?", "how far is the robot intending to reach?", and "what objects will be interacted with?" can be answered by observing the 3D when a behavior is selected for execution.


% Figure environment removed

% Figure environment removed

Interacting with parts of a behavior in a 3D setting provides important context when overlayed with a model of the environment.
For example, authoring a hand pose in code would have a name, description, and numbers associated with it, but seeing the pose and hand mesh graphic in 3D shows the proximity to other things in the environment.
The hand pose may be colliding with the surface of a table, or it might be very near a wall, which may be relevant to the planning complexity of the surrounding action keyframes.
The operator will notice these 3D spatial relationships and constraints and may choose to modify the behavior based on it.


% The operator uses a colored point cloud or other kind of map to interpret the robot's surrounding environment.
% If the robot is not close to the task or cannot see the task, the operator directly commands the robot to roughly approach the task.
% The operator does a search for the task or uses context clues to know where the task is.
% The planning for how to get to the task will be done by the operator manually.
% Enter stance pose editing mode by either selecting an existing stance pose or clicking a button.
% The behavior performance is managed by the same suite of tools used for authoring.
% The operation of humanoid robots has a gradient.
% The operator might be directly controlling the position of a robot limb using a motion tracking system, joystick, or mouse and keyboard.
% In this mode the robot 
% Save and load behaviors to and from JSON files. Relevance?
% This allows to have many different versions and new authors do not need to change source code to create new versions of behaviors, modify existing ones, or maintain their own.
% Anyone has a chance at modifying the behavior as it is entirely authored from a UI. You do not have to be a robotics software engineer.
% To communicate intent and get a robot to do what the operator wants, such as where the operator wants a hand to be, there are various modes with which to do that.
% One mode is to carefully line of a pose relative to the current hand position or specified in 3D space world coordinates and execute it.
% This is often done at a low frequency and each pose is crafted by the operator either by hardcoded numbers or in a 3D graphics environment by using a 6 degree-of-freedom pose editor, often called a gizmo.
% Another common mode is where the robot is setup to continuously track the operator's movement, tracking poses provided via a motion capture system attached to the operator.
% In this paper, we will be dealing with the first mode but instead allowing the operator to align poses to task objects so they can be reused.

Our model of the environment consists of colored point clouds and predefined detected objects such as doors and cups as shown in \autoref{fig:can_of_soup_detection}.
These get rendered in the 3D scene for the operator to view from different perspectives.
% To model the environment, we render predefined detected objects, colored point clouds, height map terrain visualization, and planar regions in the scene.
Detected objects have reference frames which can be used to specify action poses with respect to that object.
% An example can be seen in \autoref{fig:can_of_soup_detection}.
Tasks are defined with respect to predefined task frames such that the behavior can occur anywhere in the world that task is encountered.

% Figure environment removed

To keep the operator in the loop and able to respond to more scenarios, direct teleoperation tools are kept readily available in the same application.
The tools include include manual footstep placement and upper body kinematic streaming using virtual reality controllers.
They can be used to take over task performance, recover from failures, or avoid damage to the robot by getting it out of tricky situations.

% The same UI program is used for both the authoring and performance phases as there is a gradient from authoring to highly automatic performance.
%That same program also includes a suite of direct teleoperation tools which includes joystick walking and realtime upper body human pose retargeting using virtual reality controllers.
%It is useful to have them on hand for recovering from or handling situations for experimentation, to avoid breaking the robot, or getting into a certain configuration for behavior continuation.
%This allows the operator to quickly and naturally switch between direct teleoperation in it's most powerful form and behavior authoring. 
%This setup is referred to as a ``human in the loop" architecture where the operator can step in and take the place of automated planners or pause and edit an automatically executing behavior.
% We have built this behavior authoring system using some components that already existed and have been available to our operators for use in more basic teleoperation.

To create, edit, and execute actions, an action sequence editor is used, shown in \autoref{fig:sequence_editor}.
It represents a linear sequence of execution.
%Our action sequence editor can be seen in \autoref{fig:sequence_editor}.
%It is used to add, edit, reorder, and delete actions representing a linear sequence of execution.
Actions can be added through the buttons at the bottom and tuned using cooresponding panels of widgets specific to that action.
The action panels are lined up in order from top to bottom.
The action panels can be expanded and minimized, shown in the figure minimized.
Each action is given a hand written human-readable description, which is important for remembering the semantic context of that action.

% Figure environment removed

% Figure environment removed

% Figure environment removed

% Show keyframe spline with numbers on it.
% Scanned environment, detected object poses

% Figure environment removed

To specify a stance pose goal, we use a ring graphic with footstep outlines in the middle as shown in \autoref{fig:stance_pose_action}.
It is used to specify where the robot should be standing at some phase in the task.
% The stance pose action shown in \autoref{fig:stance_pose_action} is used to specify where the robot should be standing at some phase in the task.
It can be translated on it's X-Y plane by dragging the ring with the left mouse button and yawed by dragging on the ring with the right mouse button.
The red and green arrows are used to specify which way the X forward and Y left axes are facing and to quickly reorient the pose.
Clicking the arrows will orient the ring to face that direction.
A tooltip gives more information about the stance pose when the mouse hovers it.
A panel is also provided which can be used to adjust parent reference frame, swing and transfer duration of the planned steps\cite{Griffin_2019}, and the poses of the stance steps individually in order to achieve a staggered pose.

% Figure environment removed

To specify a hand pose goal, we use a semi-transparent model of the hand as shown in \autoref{fig:hand_pose_action}.
It is used to specify where the hand should be in at some point during the behavior.
% The hand pose action shown in \autoref{fig:hand_pose_action} is used to specify a goal for the hand to be in at some point during the behavior.
When selected, the pose gizmo shown in \autoref{fig:pose_gizmo} appears around it and can be used to control X, Y, Z translation and yaw, pitch, and roll rotation.
When fine or absolute adjustment is needed, a context menu can be used by right clicking the gizmo.
%The corresponding configuration panel is shown in \autoref{fig:hand_pose_panel}.

% % Figure environment removed

When the robot needs to walk through a tight space like a door frame, we want to pull the arms into a specific configuration for that.
The arms configuration for that on Nadia is shown in \autoref{fig:collision_avoidance_arms}.
Joint angle based arm configurations can be more reliably executed because they don't rely on the inverse kinematics solver to give a consistent solution for all joints along the arm.
This is important for pulling the arms in because the configuration is carefully tuned for the arms to take up as little space as possible.
% Specific arm configurations can be more reliably executed through the specification of joint angles rather than relying on the inverse kinematics solver to give a consistent solution for all joints along the arm.
The context menu available by right clicking on an interactive hand in the 3D scene is used to gather specific values.
% and the panel shown in % \autoref{fig:arm_joint_angles_panel} is used to specify them as an action.
% Lastly, notice that there is no parent frame for a joint angles action.
% There is no parent frame for a joint angles action.

% % Figure environment removed

% Mouse hover tooltips, clicking. Show some figures.
% Using gizmos to modify poses
% Automatic mode, execution stops when success conditions not met.

% Maximizing Authoring Speed

% SIMULATION
% Talk about benefits of simulation, if the same behavior can run in simulation as on the real robot.
% What are the constraints? Probably can't tune for the real robot in simulation

\section{Interdependence Analysis}

% The Coactive Design methodology focuses on maximizing three pillars of process design: observability, predictability, and directability.
% We use these pillars to think about our system design and in this paper provide interdependence analysis charts for various components.
In this section we zoom in on the interdependence relationships between the operator and the robot for two key sub-tasks: approaching a task and opening a door.
The purpose of interdependence analysis is to identify key parts of a process where the human and the robot are interacting and depending on each other.
It also used as a tool to find key shortcomings in the system by analyzing the reliability and capability of task components with relation to their dependencies.
% Using charts, we illustrate the process of authoring key actions with the robot.
The leaf nodes are color coded to show where things can go wrong and where they usually go right.
Using green, yellow, orange, and red we show the gradient from ``sufficient" to ``not functional".
Arrows are drawn to indicate dependencies and the flow of information.
Where a dependency is yellow, orange, or red are where improvements to the system's reliability and capability can be made.
Green indicates that something generally works well.
Yellow indicates that the reliability or capability is less than 100\%.
Orange indicates that the reliability or capability is not sufficient, causing dependencies to suffer.
Red indicates missing functionality or something that never works.

% Our authoring is a user interface designed to provide a rapid feedback cycle while creating new behavior and modifying existing behavior.
% This provides an "on the fly" setting for creating new behavior that can be done at any time during robot operation.
% The behaviors are immediately executable.
% The behaviors can be verified by immediately executing actions and qualifying the real world result.
% The performances of parts of the behavior during authoring are representative of their performance when being executed end-to-end.
% During on the fly authoring, when an action fails it can put the robot and environment a state which is not immediately reversible.
% Therefore, the operator must reset the state by hand and/or have the robot reset the state.
% Sometimes, playing back action keyframes in reverse or merely executing and earlier keyframe directly will work.
% For example,  when doing a pick and place task, the hand approach might knock the target object over on approach.
% A can when knocked over will likely roll away or off the table, which is a situation that the operator must reset manually.
% An object like a cube may be robust to such a disturbance and the grasp approach action may be tried again.
% When authoring a door traversal behavior, the robot may attempt to pull the lever but not turn it enough.
% In this case executing the grasp approach hand pose and trying the grasp again may be a valid strategy and not require a state reset.
% Sometimes going backwards would be a whole new behavior.
% Requirement: Human must be there to reset the environment.

\subsection{A Task Approach Scenario}

% Figure environment removed

Our interdependence analysis chart for authoring the approach of a manipulation task can be seen in \autoref{fig:task_approach_ia}.
It is consists of a table with columns for phase, operator, and robot and colored cells that represent the components.
The first phase is observing the situation and estimating a good approach location for the robot.
The operator uses the colored point cloud, robot state visualization (which is just a graphic of the robot at some pose and configuration), and ghost objects that are detected by the perception sensors.
However, the colored point cloud is not very detailed.
Our point cloud does not currently use filters over time or construct an accurate mesh of the environment -- it shows a relatively noisy scattering of points.
Because of this, its difficult to tell where things are, and introduces uncertainty for the operator in estimating a good approach.
When approaching a task, you don't want the robot to bump into a table or wall or disturb items that you may want to grasp.
We show in red that there are two currently missing functionalities that would improve on this.
If we rendered a previewed robot at the goal position, the operator could check visually for collisions.
Another goal of task approaches in general is to arrive at a stance such that the robot can reach what it needs to without having to readjust.
We do not currently have a reachability analysis tool to help with that, so we also show it in red.

When the operator is happy with the candidate stance pose they adjusted with the control ring, they command the robot to walk there.
Planning footsteps is usually not an issue here so we show it as green.
Because we are uncertain if the arriving at the stance pose will result in a collision, the robot may bump into the task and fall.
Once the walking has completed, the operator looks at the 3D scene view to qualify the result.
They may also choose to look at the robot and environment directly (in the case of local operation).
The operator may choose to further adjust the stance goal, command it, and repeat until the result is sufficient.

\subsection{Opening a Door}


\autoref{fig:door_latch_ia} shows an interdependence analysis chart for opening a door with a lever handle.
This task is aligned with a lever handle which is preregistered as a scene object with a static transform relative to an ArUco marker.
We show that this is somewhat problematic as we have had difficulty getting a good measurement of that transform.
%and also getting accurate ArUco marker detection poses.
The operator can see a 2D camera video feed from the robot's head but that is not sufficient to estimate a pose with accuracy.
In this case, the operator was forced to directly observe the scene because of these shortcomings in dependencies.
Then, the operator adjusts the grasp hand pose while estimating the feasibility that the motion will successfully disengage the latch.
The operator cannot know this will work, so we color it orange.
The operator commands the action and the robot applies forces on the world and may push through the lever and forward enough to open the door.
After this, the operator must classify the new state as successful or failed.
In the case of failure, we illustrate that the operator may alternatively approach the robot and push on the lever and robot's hand a bit to get a sense for the forces involved and adjust the hand motion accordingly.
The operator will iterate on this process until success is achieved.

% \subsection{Interdependence of Pick and Place Task}

\section{Results}

In \autoref{tab:pick_and_place_authoring}, we show the key timestamps in authoring a pick and place action sequence on the real robot.
Each timestamp after the first represents the time in which that action was completed.
%This was the first time we ever tried authoring a manipulation sequence on the fly on the real robot.
%Importantly, in this run, the hands were not reliabily opening and closing, requiring operator intervention in the form of repeatedly clicking the open or close command for up to 10 seconds.
In this run the operator took 38 minutes to author actions on the fly such that the robot had picked up the can, stepped to the side, and placed it down again during the authoring process itself.
\autoref{tab:pick_and_place_execution} shows the timestamps for re-executing the authored behavior step-by-step while supervising the robot's progress.
The execution was around 14 times faster than the authoring.
\autoref{tab:push_door_traversal_authoring} shows the timestamps of the authoring process of a push door traversal which took 33 minutes in which the robot performed the task succesfully during authoring.
\autoref{tab:push_door_traversal_automatic} shows the timestamps of executing the push door traversal in fully automatic mode in which the operator did not intervene after the initial command.

\begin{table}
\caption{Authoring pick and place of a can of soup.}
\centering
\begin{tabular}{l l}
 \hline
 Time (m) & Action authored \\
 \hline
 0 & Create new action sequence \\
 2 & Rough manual table approach \\
 3 & Approach table \\
 4 & Right hand approaches can \\
 11 & Pre-grasp pose sufficient \\
 12 & Robot grasps can of soup \\
 13 & Robot lifts can of soup \\
 22 & Pull arm back action setup \\
 23 & Side step action executed \\
 28 & Put arm forward \\
 32 & Set down can of soup \\
 36 & Release grasp \\
 37 & Pull arm back \\
 38 & Back away from table \\
\label{tab:pick_and_place_authoring}
\end{tabular}
\end{table}

\begin{table}
\caption{Step-by-step supervised execution of picking and placing can of soup.}
\centering
\begin{tabular}{l l}
 \hline
 Time (m:s) & Action completed \\
 \hline
 0:00 & Begin approach \\
 0:11 & Approach table \\
 0:14 & Right hand approaches can \\
 0:53 & Pre-grasp hand pose \\
 0:58 & Grasp can of soup \\
 1:00 & Pull back hand with can of soup \\
 1:16 & Step to the side \\
 1:20 & Set down can \\
 1:36 & Release grasp on can \\
 1:46 & Back away from task \\
\label{tab:pick_and_place_execution}
\end{tabular}
\end{table}


\begin{table}
\caption{Authoring push door traversal.}
\centering
\begin{tabular}{l l}
 \hline
 Time (m) & Action authored \\
 \hline
  0 & Create new action sequence \\
  2 & Approach door \\
  5 & Right hand approaches handle \\
  7 & Pre-grasp hand pose \\
  9 & First handle turn contact \\
 14 & Latch disengaged \\
 20 & Door pushed open with right hand \\
 24 & Door pushed open more with left hand \\
 25 & Door pushed open all the way with left hand \\
 26 & Arms in collision avoidance configuration \\
 32 & Step forward a little \\
 33 & Walk through the door frame \\
\label{tab:push_door_traversal_authoring}
\vspace{-3mm}
\end{tabular}
\end{table}

\begin{table}
\caption{Fully automatic push door traversal.}
\centering
\begin{tabular}{l l}
 \hline
 Time (m:s) & Action completed \\
 \hline
 0:00 & Begin approach \\
 0:07 & Approach door \\
 0:11 & Right hande approaches handle \\
 0:12 & Pre-grasp \\
 0:14 & Door opened \\
 0:20 & Push door all the way open with left hand \\
 0:24 & Arms in collision avoidance configuration \\
 0:35 & Finish walking through door frame \\
\label{tab:push_door_traversal_automatic}
\vspace{-6mm}
\end{tabular}
\end{table}

% Figure environment removed

\section{Discussion and Future Work}

We think that this work provides a base for expansion.
There are a lot of known improvements to the user interface details that would make the authoring process faster.
%In our trials there was a bug in the hand software causing the hands to take up to 10 seconds to open or close which reduced the speed and reliability of the behaviors.
%In addition, hardware downtime is one of the biggest developmental slow downs, as many issues only appear when running the full setup on the real robot.
One of the most significant developmental difficulties is that many issues only appear when running the full setup on the real robot, thus making it difficult to document and address issues.

%There are three main pillars of natural extension to this work: incorporating more advanced perception and planning, increasing the usefulness of performed tasks, and making measured improvements of reliability, and repeatability.

% We expect the speed of authoring to improve dramatically as the presented results were only the first attempts.
There are algorithms for stance and grasp generation available as detailed in \cite{Hart_2022} which could be used to automate task approach, considering reachabilty, and to plan grasps on modelled objects.
Today's neural net assisted algorithms\cite{tremblay2018deep}\cite{Lin_2022} could enable the detection of a wide variety of objects and remove the need for fiducial markers.
Having an expansive library of detectable objects would open up the space for autonomous behavior.
%Inclusion of tools for perceptual registration\cite{Hagenow_2022} could improve the flexibility to act on a wider variety of objects.
%Also, if the perception system could estimate the pose objects directly with camera images, we could avoid using fiducial markers and avoid depending on the error-prone measurement of transforms to the markers.
%There is off the shelf technology available now to tackle this problem that we are currently exploring.\cite{Lin_2022}
There is a growing area of research termed ``affordance primitives"\cite{Pettinger_2020}\cite{Pettinger_2022} which are a way to model actions with constrained movement, such as turning a valve or closing a drawer.
Affordance primitives have the potential for force and perceptual feedback control during the manipulation task.
In general, the authoring process should evolve into higher level abstractions, such as specifying the desired poses of objects rather than the movement of the robot to get them there.

% For something like a push door traversal, we have had the authoring time be around 45 minutes and the resulting automatic execution be under 40 seconds.
% We need measurements of repeatability for several tasks, which will allow the scope of this paper to expand to that, which will sell it a lot better.
% The primary metric of success for this work is if an operator can put together a behavior for a task in under and hour and using the real robot to author it, then taking steps back and running that behavior to perform the task automatically 5 times in a row.
% When automatically executed, the tasks should approach being 1/10 the speed it would take a human or faster.
% In a previously unseen environment, be able to author a new behavior on the fly and be able to execute that task with a high degree of autonomy after that.
% Undo and redo
% Actions where mass of manipulated object is considered
% Two-handed manipulation tasks
% Tasks where the robot is exerting large forces on the world with feedback.
% Action sequences which are dynamically moving and do not pass through statically stable configurations nominally.

\section{Conclusion}

In this work, we developed a new interface for authoring humanoid behaviors with the ability to do the authoring on the fly.
We detailed several interactive interface elements and how they are used in the authoring process.
We introduced interdependence analysis charts for a few key tasks that humanoid robots encounter.
Finally, we demonstrated this approach by conducting two experiments on real robot hardware and provided the timings of the authoring process and execution processes.

\subsection{Acknowledgements}

We would like to thank Stephen Hart, Matt Johnson, William Howell and the team at IHMC Robotics, without whom this work would not have been possible.

\subsection{Source Code and Media}

Our implementation X and the associated modules discussed in this paper can be found on our GitHub at \url{https://github.com/ihmcrobotics}.
The accompanying video can be found at \url{https://youtu.be/Wcq6QqB4g40}.


% eventually replace this with generated final bbl maybe?
\bibliography{mybib}

\end{document}
