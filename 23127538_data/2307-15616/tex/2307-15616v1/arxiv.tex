\documentclass[11pt]{article}

\usepackage{natbib}
 % \NatBibNumeric
 \def\bibfont{\small}%
 \def\bibsep{\smallskipamount}%
 \def\bibhang{24pt}%
 \def\BIBand{and}%
 \def\newblock{\ }%
 \bibpunct[, ]{[}{]}{,}{n}{}{,}%

\usepackage{amsmath}
\usepackage{amssymb}
\usepackage{amsfonts}
\usepackage{mathrsfs}
\usepackage{graphicx}
\usepackage{color,xcolor}
\usepackage{float}
\usepackage{subfig}
\usepackage[normalem]{ulem}
\usepackage[linktocpage,colorlinks,linkcolor=blue,anchorcolor=blue, citecolor=blue,urlcolor=blue]{hyperref}
\usepackage{pdfpages}
%\usepackage[linktocpage,pagebackref,colorlinks,linkcolor=blue,anchorcolor=blue, citecolor=blue,urlcolor=blue]{hyperref}
\usepackage{enumitem}
\usepackage{multirow}
\usepackage{makecell}
\usepackage{breakcites}
\usepackage{bbm}
\usepackage{wrapfig}
% \usepackage[title]{appendix}

\topmargin      0.0truein
\headheight     0.0truein
\headsep        0.0truein
\textheight     9.0truein
\textwidth      6.5truein
\oddsidemargin  0.0truein
\evensidemargin 0.0truein
\renewcommand{\baselinestretch}{1.0}

\newtheorem{theorem}{Theorem}[section]
\newtheorem{lemma}[theorem]{Lemma}
\newtheorem{proposition}[theorem]{Proposition}
\newtheorem{corollary}[theorem]{Corollary}
\newtheorem{claim}[theorem]{Claim}
\newtheorem{conjecture}[theorem]{Conjecture}
\newtheorem{hypothesis}[theorem]{Hypothesis}
\newtheorem{assumption}[theorem]{Assumption}
% \theoremstyle{EX}
\newtheorem{remark}[theorem]{Remark}
\newtheorem{example}[theorem]{Example}
\newtheorem{problem}[theorem]{Problem}
\newtheorem{definition}[theorem]{Definition}
\newtheorem{question}[theorem]{Question}
\newtheorem{answer}[theorem]{Answer}
\newtheorem{exercise}[theorem]{Exercise}
\newtheorem{algorithm}[theorem]{Algorithm}

\def\bd{\boldsymbol}
\def\pn{\par\smallskip\noindent}
\def\proof{\pn {\em Proof.}}
\def\endproof{\hfill $\Box$ \vskip 0.2truein}

\newenvironment{myproof}[1] {\pn {\em Proof of {#1}.}}{\hfill $\Box$ \vskip 0.2truein}

\newcommand{\BD}{\mathbb{D}}
\newcommand{\BE}{\mathbb{E}}
\newcommand{\BH}{\mathbb{H}}
\newcommand{\BX}{\mathbb{X}}
\newcommand{\BY}{\mathbb{Y}}
\newcommand{\BZ}{\mathbb{Z}}
\newcommand{\BS}{\mathbb{S}}
\newcommand{\BT}{\mathbb{T}}
\newcommand{\BN}{\mathbb{N}}
\newcommand{\BI}{\mathbb{I}}
\newcommand{\R}{\mathbb{R}}
\newcommand{\BK}{\mathbb{K}}
\newcommand{\BQ}{\mathbb{Q}}

\newcommand{\A}{\mathcal{A}}
\newcommand{\B}{\mathcal{B}}
\newcommand{\OO}{\mathcal{O}}
\newcommand{\F}{\mathcal{F}}
\newcommand{\E}{\mathcal{E}}
\newcommand{\G}{\mathcal{G}}
\newcommand{\X}{\mathcal{X}}
\newcommand{\Y}{\mathcal{Y}}
\newcommand{\Z}{\mathcal{Z}}
\newcommand{\TT}{\mathcal{T}}
\newcommand{\U}{\mathcal{U}}
\newcommand{\V}{\mathcal{V}}

\newcommand{\ba}{\boldsymbol{a}}
\newcommand{\bb}{\boldsymbol{b}}
\newcommand{\be}{\boldsymbol{e}}
\newcommand{\bx}{\boldsymbol{x}}
\newcommand{\by}{\boldsymbol{y}}
\newcommand{\bz}{\boldsymbol{z}}
\newcommand{\bu}{\boldsymbol{u}}
\newcommand{\bv}{\boldsymbol{v}}
\newcommand{\bw}{\boldsymbol{w}}
\newcommand{\bn}{\boldsymbol{n}}
\newcommand{\bt}{\boldsymbol{t}}
\newcommand{\bh}{\boldsymbol{h}}
\newcommand{\bp}{\boldsymbol{p}}

\newcommand{\T}{\textnormal{T}}
\newcommand{\st}{\textnormal{s.t.}}
\newcommand{\tr}{\textnormal{tr}\,}
\newcommand{\Mat}{M}
\newcommand{\vct}{\textnormal{vec}}
\newcommand{\conv}{\textnormal{conv}\,}
\newcommand{\Prob}{\textnormal{Prob}\,}
\newcommand{\rank}{\textnormal{rank}\,}
\newcommand{\sign}{\textnormal{sign}\,}
\newcommand{\diag}{\textnormal{diag}\,}
\newcommand{\Diag}{D}

\usepackage{booktabs}
\usepackage{graphicx}
\usepackage{stmaryrd}
\usepackage{mathtools}
% \DeclarePairedDelimiter\ceil{\lceil}{\rceil}
% \DeclarePairedDelimiter\floor{\lfloor}{\rfloor}
\usepackage{multicol}

\usepackage{algorithm}
\usepackage{algorithmic}
\renewcommand{\algorithmiccomment}[1]{\hfill{$\triangleright$~#1}}
\newcommand{\algorithmautorefname}{Algorithm}
\renewcommand{\algorithmicrequire}{\textbf{Input:}}
\renewcommand{\algorithmicensure}{\textbf{Output:}}

\makeatletter
\newcommand{\pushright}[1]{\ifmeasuring@#1\else\omit\hfill$\displaystyle#1$\fi\ignorespaces}
\newcommand{\pushleft}[1]{\ifmeasuring@#1\else\omit$\displaystyle#1$\hfill\fi\ignorespaces}
\makeatother

%% tikz macros
\usepackage{tikz}
\usetikzlibrary{calc}
\usepackage{pgfplots}
\usepackage{xxcolor}
\pgfplotsset{compat=1.16}
\usepgfplotslibrary{fillbetween}
% Declare nice sphere shading: http://tex.stackexchange.com/a/54239/12440
\pgfdeclareradialshading[tikz@ball]{ball}{\pgfqpoint{0bp}{0bp}}{%
 color(0bp)=(tikz@ball!0!white);
 color(7bp)=(tikz@ball!0!white);
 color(15bp)=(tikz@ball!70!black);
 color(20bp)=(black!70);
 color(30bp)=(black!70)}
\makeatother

% Style to set TikZ camera angle, like PGFPlots `view`
\tikzset{viewport/.style 2 args={
    x={({cos(-#1)*1cm},{sin(-#1)*sin(#2)*1cm})},
    y={({-sin(-#1)*1cm},{cos(-#1)*sin(#2)*1cm})},
    z={(0,{cos(#2)*1cm})}
}}

% Styles to plot only points that are before or behind the sphere.
\pgfplotsset{only foreground/.style={
    restrict expr to domain={rawx*\CameraX + rawy*\CameraY + rawz*\CameraZ}{-0.05:100},
}}
\pgfplotsset{only background/.style={
    restrict expr to domain={rawx*\CameraX + rawy*\CameraY + rawz*\CameraZ}{-100:0.05}
}}

% Automatically plot transparent lines in background and solid lines in foreground
\def\addFGBGplot[#1]#2;{
    \addplot3[#1,only background, opacity=0.25] #2;
    \addplot3[#1,only foreground] #2;
}

\newcommand{\ViewAzimuth}{-20}
\newcommand{\ViewElevation}{15}
%%
\usetikzlibrary{3d}
\usetikzlibrary{calc}
\usetikzlibrary{babel} % for issues with some babel packages

% isometric axes
\pgfmathsetmacro\xx{1/sqrt(2)}
\pgfmathsetmacro\xy{1/sqrt(6)}
\pgfmathsetmacro\zy{sqrt(2/3)}


\newcommand{\downimplies}{\rotatebox[origin=c]{-90}{$\implies$}}

\newcommand{\upimplies}{\rotatebox[origin=c]{-90}{$\impliedby$}}

\begin{document}

\title{$\ell_p$-sphere covering and approximating nuclear $p$-norm}

\author{
Jiewen GUAN
\thanks{Research Institute for Interdisciplinary Sciences, School of Information Management and Engineering, Shanghai University of Finance and Economics, Shanghai 200433, China. Email: seemjwguan@gmail.com}
    \and
Simai HE
\thanks{Research Institute for Interdisciplinary Sciences, School of Information Management and Engineering, Shanghai University of Finance and Economics, Shanghai 200433, China. Email: simaihe@mail.shufe.edu.cn}
    \and
Bo JIANG
\thanks{Research Institute for Interdisciplinary Sciences, School of Information Management and Engineering, Shanghai University of Finance and Economics, Shanghai 200433, China. Email: isyebojiang@gmail.com}
    \and
Zhening LI
\thanks{School of Mathematics and Physics, University of Portsmouth, Portsmouth PO1 3HF, United Kingdom. Email: zheningli@gmail.com}
}

\date{\today}

\maketitle

\begin{abstract}
The spectral $p$-norm and nuclear $p$-norm of matrices and tensors appear in various applications albeit both are NP-hard to compute. The former sets a foundation of $\ell_p$-sphere constrained polynomial optimization problems and the latter has been found in many rank minimization problems in machine learning. We study approximation algorithms of the tensor nuclear $p$-norm with an aim to establish the approximation bound matching the best one of its dual norm, the tensor spectral $p$-norm. Driven by the application of sphere covering to approximate both tensor spectral and nuclear norms ($p=2$), we propose several types of hitting sets that approximately represent $\ell_p$-sphere with adjustable parameters for different levels of approximations and cardinalities, providing an independent toolbox for decision making on $\ell_p$-spheres. Using the idea in robust optimization and second-order cone programming, we obtain the first polynomial-time algorithm with an $\Omega(1)$-approximation bound for the computation of the matrix nuclear $p$-norm when $p\in(2,\infty)$ is a rational, paving a way for applications in modeling with the matrix nuclear $p$-norm. These two new results enable us to propose various polynomial-time approximation algorithms for the computation of the tensor nuclear $p$-norm using tensor partitions, convex optimization and duality theory, attaining the same approximation bound to the best one of the tensor spectral $p$-norm. We believe the ideas of $\ell_p$-sphere covering with its applications in approximating nuclear $p$-norm would be useful to tackle optimization problems on other sets such as the binary hypercube with its applications in graph theory and neural networks, the nonnegative sphere with its applications in copositive programming and nonnegative matrix factorization.

\vspace{0.25cm}

\noindent {\bf Keywords:} optimization on $\ell_p$-sphere, nuclear $p$-norm, semidefinite programming, $\ell_p$-sphere covering, approximation algorithm, spectral $p$-norm, polynomial optimization

\vspace{0.25cm}

\noindent {\bf Mathematics Subject Classification:} 15A60, % Norms of matrices, numerical range, applications of functional analysis to matrix theory 
90C59, % Approximation methods and heuristics in mathematical programming
52C17, % Packing and covering in n dimensions
68Q17 % Computational difficulty of problems (lower bounds, completeness, difficulty of approximation, etc.)

\end{abstract}


%\section{Introduction.}\label{intro} %%1.
%\subsection{Duality and the classical EOQ problem.}\label{class-EOQ} %% 1.1.
%\subsection{Outline.}\label{outline1} %% 1.2.
%\subsubsection{Cyclic schedules for the general deterministic SMDP.}
%  \label{cyclic-schedules} %% 1.2.1
%\section{Problem description.}\label{problemdescription} %% 2.

\section{Introduction}\label{sec:introduction}

% Tensors are ubiquitous in modern data analytics, and their inherent structural properties can oftentimes lead to conducive consequences in many realistic scenarios, such as unsupervised feature selection~\cite{chen2022unsupervised}, principal component analysis~\cite{jiang2015tensor}, etc. An important and essential research object in the field of tensor computation and analysis is the tensor nuclear norm, which is critically important for 
% % many applications including but not limited to 
% tensor completion~\cite{yuan2016tensor}, and also has close connections to some physical concepts in quantum mechanics such the separability of bipartite states~\cite[Section~3]{friedland2014computational}. Unfortunately, the tensor nuclear norm is NP-hard to compute in general~\cite{friedland2018nuclear}. (In fact, ``most tensor problems are NP-hard''~\cite{hillar2013most}). Therefore, in the last ten years, various efforts have been paid by researchers to approximate it in polynomial time or find its tight bounds. E.g.,~\cite{hu2015relations} studied the relations between the nuclear norms of a third-order tensor and its matrix flattenings.~\cite{li2016bounds} studied various bounds on the tensor spectral and nuclear norms through the technique of regular tensor partition. Recently, by virtue of the sphere covering technique,~\cite{he2023approx} established a polynomial-time approximation ratio of $\Omega\left(\prod_{k=1}^{d-2}\sqrt{\frac{\ln{n_k}}{n_k}}\right)$ for the nuclear norm of tensors in $\R^{n_1\times n_2\times \dots\times n_d}$, which matches the best-known one of its dual norm, the tensor spectral norm~\cite[Theorem~4]{so2011deterministic},
% % polynomial optimization on the Cartesian product of the unit Euclidean spheres
% % (and equivalently~\cite{hou2014hardness}, the tensor spectral $p$-norm), 
% and thus ``bridges the gap between the primal and dual''. Closely related to the tensor nuclear norm is the tensor nuclear $p$-norm\footnote{Unless explicitly stated otherwise, we assume $p\in(2,\infty)$ and $q$ as the conjugate of $p$ throughout the paper.}, which is a natural generalization of the former. Therefore, the investigation of the tensor nuclear $p$-norm is also of critical importance.

Decision making on spheres has been one of the important topics in mathematical optimization. It includes an extensive list of both theoretical and practical applications in areas of study such as biomedical engineering~\cite{BJVS07}, computational anatomy~\cite{FVJ09}, numerical multilinear algebra~\cite{Q05}, quantum mechanics~\cite{DLMO07}, solid mechanics~\cite{HDQ09}, signal processing~\cite{WV10}, tensor decompositions~\cite{KB09}, to name a few. Taking one of its simple models, $\min\{f(\bx):\|\bx\|_2=1,\,\bx\in\R^n\}$ where $f(\bx)$ is a multivariate polynomial function, this includes particular cases such as Cauchy-Schwarz inequality for a linear objective, extreme matrix eigenvalues for a quadratic form, trust region subproblem for a generic quadratic function~\cite{Y00}, as well as deciding the nonnegativity of a homogeneous polynomial~\cite{R00}. However, even in this simple form the problem can be very difficult since minimizing a cubic form on the unit sphere is already NP-hard~\cite{N03}. This fact has resulted a lot of research works~\cite{he2010approximation,so2011deterministic,zhangqy12,libook12,zhou12, he2014probability} on polynomial-time approximation algorithms for sphere constrained polynomial optimization in the recent decade. All these works essentially rely on a fundamental problem which is to maximize a multilinear form on Cartesian products of unit spheres. Taking a trilinear form
$\TT(\bx,\by,\bz) = \sum_{i=1}^{n}\sum_{j=1}^{n}\sum_{k=1}^{n} t_{ijk} x_iy_jz_k$ where $\TT=(t_{ijk})\in\R^{n\times n\times n}$ is a tensor of order three as a typical example, the optimization model is
\begin{equation}\label{eq:3tnorm}
\|\TT\|_{\sigma}= \max \left\{\TT(\bx,\by,\bz): \|\bx\|_2=\|\by\|_2=\|\bz\|_2=1,\,\bx,\by,\bz\in\R^n\right\},
\end{equation}
and the optimal value $\|\TT\|_{\sigma}$ is known as the spectral norm of $\TT$. The model~\eqref{eq:3tnorm} is NP-hard~\cite{he2010approximation}. The best-known approximation bound obtained by a polynomial-time algorithm is $\Omega(\sqrt{\ln n/n})$~\cite{so2011deterministic,he2014probability,he2023approx}.

The dual norm problem to~\eqref{eq:3tnorm} is
\begin{equation}\label{eq:3nnorm}
\|\TT\|_{*}= \max \left\{\left\langle\TT,\X\right\rangle: \|\X\|_\sigma\le 1,\,\X\in\R^{n\times n\times n}\right\},
\end{equation}
and the optimal value $\|\TT\|_*$ is known as the nuclear norm of $\TT$. The nuclear norm is much more widely used in practical applications as the convex surrogate of the rank function, no matter for matrices~\cite{RFP10} or tensors~\cite{yuan2016tensor}. The model~\eqref{eq:3nnorm} is also NP-hard~\cite{friedland2018nuclear}. It seems irrelevant to decision making on spheres but it is actually highly related. By carefully choosing vectors on the unit sphere and using semidefinite programs with duality theory~\cite{hu2022complexity}, He et al.~\cite{he2023approx} obtained the best-known approximation bound $\Omega(\sqrt{\ln n/n})$ for~\eqref{eq:3nnorm}. The bound improved the previous best one $\Omega(\sqrt{1/n})$ and matched the best one $\Omega(\sqrt{\ln n/n})$ for~\eqref{eq:3tnorm}, hence bridging the gap between the primal and dual. The key idea of the algorithm in~\cite{he2023approx} is to explicitly construct a number of spherical caps to cover the unit sphere. It provides a useful tool and opens a new door to deal with decision making on spheres.

The ideas of constructions for sphere covering and the applications to~\eqref{eq:3tnorm} and~\eqref{eq:3nnorm} in~\cite{he2023approx} motivate us to study decision making on other important sets. An immediate one that generalizes the unit sphere is the $\ell_p$-sphere defined by $\|\bx\|_p=1$ for $p\in[1,\infty]$. It includes an important case of the unit box constraint when $p=\infty$. The example models~\eqref{eq:3tnorm} and~\eqref{eq:3nnorm} are then generalized to the spectral $p$-norm problem
\begin{equation}\label{eq:3tpnorm}
\|\TT\|_{p_\sigma}= \max \left\{\TT(\bx,\by,\bz): \|\bx\|_p=\|\by\|_p=\|\bz\|_p=1,\,\bx,\by,\bz\in\R^n\right\}
\end{equation}
and the nuclear $p$-norm problem
\begin{equation}\label{eq:3npnorm}
\|\TT\|_{p_*}= \max \left\{\left\langle\TT,\X\right\rangle: \|\X\|_{p_\sigma}\le 1,\,\X\in\R^{n\times n\times n}\right\},
\end{equation}
respectively. They are often found in many practical applications and in modeling various tensor optimization models. In fact, many labeling problems in pattern recognition and image processing can be modeled by maximizing a certain polynomial function on the $\ell_p$-sphere~\cite{BBP98}. Moreover, local optimal solutions of~\eqref{eq:3tpnorm} correspond to $\ell_p$-singular vectors of a tensor~\cite{lim2005singular} that have been extensively studied in the spectral theory of tensors and play an important role in signal processing, automatic control and data analysis. The nuclear $p$-norm~\eqref{eq:3npnorm} has found successful applications in temporal knowledge base completion~\cite{lacroix2018canonical, LacroixOU20}. In particular, under the nontemporal setting~\cite{lacroix2018canonical}, the nuclear $3$-norm is adopted as a regularizer for the binary tensor completion problem to simultaneously reduce the intrinsic complexity of the learned model and induce coefficient separability. This leads to great performance improvement and optimization convenience under stochastic gradient descent. Similarly, under the temporal setting~\cite{LacroixOU20}, the nuclear $3$-norm and $4$-norm are respectively adopted as two different regularizers for binary tensor completion and they both experimentally exhibit salient performance promotion over other comparative regularizers.

In fact, Hou and So~\cite{hou2014hardness} studied various $\ell_p$-sphere constrained polynomial optimization problems including~\eqref{eq:3tpnorm} as a typical example. In particular, they showed that~\eqref{eq:3tpnorm} is NP-hard when $p\in(2,\infty)$ and proposed a deterministic polynomial-time algorithm with an approximation bound $\Omega(\sqrt[p]{\ln n}/\sqrt{n})$, which can be improved to $\Omega(\sqrt{\ln n/n})$ with the help of randomization. These bounds remain currently the best when $p\in(2,\infty)$. The dual problem~\eqref{eq:3npnorm} is also NP-hard, evidenced by the complexity of duality~\cite{friedland2016computational}. However, to the best of our knowledge, the only known polynomial-time approximation bound of~\eqref{eq:3npnorm} in the literature is $\Omega(1/\sqrt[q]{n^2})$~\cite[Proposition 4.3]{chen2020tensor} where $\frac{1}{p}+\frac{1}{q}=1$. It was obtained by a simple partition of vectors and is much worse than that of~\eqref{eq:3tpnorm}. An improved approximation bound $\Omega(1/\sqrt[q]{n})$ of~\eqref{eq:3npnorm} can be obtained by a partition of matrices or matrix unfolding in~\cite{chen2020tensor} but the methods rely on efficient computation of the matrix nuclear $p$-norm who itself is already NP-hard~\cite[Section~7]{friedland2018nuclear}. In fact, the bound $\Omega(1/\sqrt[q]{n})$ is made possible when $p\in(2,\infty)$ with the help of an $\Omega(1)$-approximation bound of the matrix nuclear $p$-norm developed in Section~\ref{sec:matrix-p-norm}. Well, this is still worse than that of~\eqref{eq:3tpnorm} when $p\in(2,\infty)$, no matter the deterministic one $\Omega (\sqrt[p]{\ln n}/\sqrt{n})$ or the randomized one $\Omega(\sqrt{\ln n/n})$. This naturally motivates the following question: Can the idea of sphere covering be applied to develop polynomial-time algorithms for~\eqref{eq:3npnorm} with approximation bounds matching currently the best one for its dual problem~\eqref{eq:3tpnorm}?

%  In fact, even the computation of the matrix nuclear $p$-norm is proven to be NP-hard~\cite[Section~7]{friedland2018nuclear}, which, combined with the fact that the nuclear $p$-norms of a matrix and its canonical embeddings to higher-order tensors by appending zeroes are identical~\cite[Theorem~4.6]{chen2020tensor}, implies the previous statement. Therefore, we are mainly interested in polynomial-time approximation to the tensor nuclear $p$-norm. However, to the best of our knowledge, no work regarding this topic can be found in the literature, except~\cite{li2020norm} and~\cite{chen2020tensor}. In the former work, a norm compression inequality for the tensor nuclear $p$-norm was conjectured (see~\cite[(7)]{li2020norm}). In the later work, various bounds on the tensor spectral and nuclear $p$-norms were obtained via the technique of arbitrary tensor partition, and it established a polynomial-time approximation ratio of $\Omega\left(\sqrt[\leftroot{-2}\uproot{2}q]{\prod_{k=1}^{d-2}\frac{1}{n_k}}\right)$ for the tensor nuclear $p$-norm, which was achieved by performing matrix unfolding or slicing of tensors\footnote{We remark that although the original approximation in that work is not polynomial-time computable, since it involves computing the matrix nuclear $p$-norms of the matricizatons or slices of a tensor, which is NP-hard as aforementioned, the matrix nuclear $p$-norm can in fact be approximated to within a constant factor in polynomial time by utilizing the techniques to be developed in Section~\ref{sec:relax-tensor-p-norm} in this paper,
% % we can bypass this dilemma to prove the claim. This is because it is known that the dual norm of the matrix nuclear $p$-norm, the matrix spectral $p$-norm, or equivalently the $p\rightarrow q$ norm, can be approximated to within a constant factor in polynomial time through convex optimization~\cite[Proposition~3]{hou2014hardness}, which combined with~\cite[Corollary~11]{friedland2016computational}, implies that the matrix nuclear $p$-norm is also polynomial-time approximable to the same extent, i.e., within a constant factor, and therefore further implies that the tensor nuclear $p$-norm can indeed be approximated to within the claimed factor in polynomial time.
% and therefore we can still compute an alternative approximation based on the original one for the tensor nuclear $p$-norm in polynomial time with the same approximation ratio. 
% % In fact, an explicit constant-factor polynomial-time approximation algorithm for the matrix nuclear $p$-norm will be clear after all developments in Section~\ref{sec:relax-tensor-p-norm} have been established.
% }. However, such a ratio still shows room for improvement. This is because it is known that the tensor spectral $p$-norm can be approximated to within a factor of $\Omega\left(\prod_{k=1}^{d-2} \frac{\sqrt[\leftroot{-2}\uproot{2}p]{\ln{n_k}}}{\sqrt{n_k}}\right)$ (resp. $\Omega\left(\prod_{k=1}^{d-2} \sqrt{\frac{\ln{n_k}}{n_k}}\right)$) by a deterministic (resp. randomized) algorithm in polynomial time, as it is in essence an $\ell_p$-spherically-constrained multilinear maximization problem~\cite[Theorems~7 and~8]{hou2014hardness}, and hence it must be the case that the tensor nuclear $p$-norm, which is its dual norm~\cite[Lemma~2.5]{chen2020tensor}, can also be approximated to within the same factor in polynomial time~\cite[Corollary~11]{friedland2016computational}.

In this paper, we answer the above question affirmatively by proposing easily implementable polynomial-time approximation algorithms for both the tensor spectral $p$-norm (the general version of~\eqref{eq:3tpnorm}) and the tensor nuclear $p$-norm (the general version of~\eqref{eq:3npnorm}) with the same best approximation bounds, in a short word, the $\ell_p$-sphere generalization of the $\ell_2$-sphere in~\cite{he2023approx}. These extensions are highly nontrivial and many challenges need to be tackled in order to achieve the goal. As a reward, our study also leads to some new tools and byproducts that would be beneficial to the community. In the following, we elaborate the difficulties conquered in this paper.

The first difficulty lies in the NP-hardness to compute the matrix nuclear $p$-norm that sets a foundation to the approximation of the tensor nuclear $p$-norm, in contrast to the matrix nuclear norm ($p=2$) that can be computed easily as the sum of its singular values. The research of approximating the matrix nuclear $p$-norm was almost blank although there are quite a few works on approximating the matrix spectral $p$-norm~\cite{N00,BN01,steinberg2005computation,hou2014hardness} albeit it is also NP-hard~\cite[Section~2.3]{steinberg2005computation}. To tackle this difficulty, we first apply a convex optimization model that approximates the matrix spectral $p$-norm within a constant factor~\cite[Proposition~3]{hou2014hardness} to the model~\eqref{eq:3npnorm} but it results the optimal value of the optimization model appearing in the constraints. We then apply the techniques developed in robust optimization~\cite{ben2009book,goldfarb2003robust} to compute the dual formulation of the convex optimization model that makes~\eqref{eq:3npnorm} explicit via a strong duality. In a final step, we reformulate some seemingly hard constrains to second-order cone (SOC) constraints. These enable us to (approximately) transfer~\eqref{eq:3npnorm} to a linear conic optimization model with semidefinite constraints and SOC constraints. As a result, the optimal value of the conic optimization model approximates the matrix nuclear $p$-norm within a constant factor when $p\in(2,\infty)$.

The second difficulty is that $\ell_p$-sphere covering cannot be easily extended from $\ell_2$-sphere covering as the former is not self-dual. Specifically, the goal is to find a hitting set $\BH^n$ consisting of a polynomial number of vectors on $\ell_p$-sphere in $\R^n$ such that $\max\{\bz^{\T}\bx:\bz\in\BH^n\}\ge\tau$ holds for any $\|\bx\|_q=1$. The $\tau$ is called the covering ratio that is aimed to be the largest possible. Therefore, covering $\ell_q$-sphere needs to analyze vectors on $\ell_p$-sphere. However, the generalization of $\ell_2$-sphere covering in~\cite{he2023approx} that enjoys the best covering ratio $\Omega(\sqrt{\ln n/n})$ can only achieve a covering ratio $\Omega(\sqrt[q]{\ln{n}/n})$, which becomes worse for the important case $p\in(2,\infty)$ (the condition that the matrix nuclear $p$-norm can be approximated) and not helpful to obtain the best approximation bound. To further improve the covering ratio, we notice an idea in computational geometry~\cite{brieden1998approximation} that an extended Hadamard transform (a linear transformation obtained by replacing $1$ with identity matrix in a Hadamard matrix)~\cite{syl1867,SY92} can enlarge $\ell_2$-norm while keeping $\ell_p$-norm unchanged for some vectors of our interest. This refinement sets a foundation to improve the covering ratio to  $\Omega(\sqrt[p]{\ln{n}}/\sqrt{n})$ which matches the best deterministic approximation bound of~\eqref{eq:3tpnorm}. Finally, we construct a randomized hitting set of $\ell_p$-sphere whose covering ratio further improves to $\Omega(\sqrt{\ln{n}/n})$ which matches the best randomized approximation bound of~\eqref{eq:3tpnorm}. As a result, we propose various constructions of hitting sets of $\ell_p$-sphere covering that are of independent interest and have direct applications to $\ell_p$-sphere decision making such as polynomial optimization on $\ell_p$-spheres.

With the aforementioned new results, we are able to put all the pieces together to derive various polynomial-time algorithms to approximate the tensor nuclear $p$-norm when $p\in(2,\infty)$. In particular, we first adopt the tensor partition technique~\cite{chen2020tensor} that helps to obtain an $\Omega(1/\sqrt[q]{n})$-approximation bound of~\eqref{eq:3npnorm}, an improvement from the only known one $\Omega(1/\sqrt[q]{n^2})$ in the literature. We then apply the hitting set relaxation approach that was originally proposed in~\cite{hu2022complexity} to improve the approximation bound of~\eqref{eq:3npnorm} to $\Omega(\sqrt[q]{\ln{n}/n})$ and $\Omega (\sqrt[p]{\ln n}/\sqrt{n})$ based on two different hitting sets of $\ell_p$-sphere. The latter bound matches the best approximation bound of~\eqref{eq:3tpnorm} by a deterministic algorithm. Finally, with the help of a randomized hitting set of $\ell_p$-sphere and a careful treatment of probability argument, we obtain the best approximation bound $\Omega(\sqrt{\ln n/n})$ of~\eqref{eq:3npnorm} that matches the best randomized one og~\eqref{eq:3tpnorm}, hence bridging the gap between the primal and dual perfectly.

We summarize our main contributions in the paper below.
\begin{itemize}
\item We obtain the first polynomial-time algorithm with $\Omega(1)$-approximation bound for the computation of the matrix nuclear $p$-norm when $p\in(2,\infty)$. This paves a way for applications in modeling with the matrix nuclear $p$-norm.
\item We propose several types of hitting sets of $\ell_p$-sphere with varying cardinalities and covering ratios. This provides an independent tool for decision making on $\ell_p$-spheres. 
\item We propose various polynomial-time algorithms for the computation of the tensor nuclear $p$-norm when $p\in(2,\infty)$ with the best approximation bound matching the best one for the tensor spectral $p$-norm.
\end{itemize}

The rest of the paper is organized as follows. We first introduce many uniform notations, formal definitions and their optimization models, as well as some basic properties in Section~\ref{sec:notation}. We then study the computation of the matrix nuclear $p$-norm in Section~\ref{sec:matrix-p-norm} by proposing a convex optimization model whose optimal value approximates the matrix nuclear $p$-norm within a constant factor. The construction of various hitting sets of $\ell_p$-sphere covering is discussed in Section~\ref{sec:hitting-sets}. In Section~\ref{sec:algorithms}, we study the design and analysis of several deterministic and randomized polynomial-time algorithms to approximate the tensor nuclear $p$-norm. Finally, we conclude this paper and propose some future research in Section~\ref{sec:conclusion}.


\section{Notations and preliminaries}\label{sec:notation}

\subsection{Uniform notations}

Throughout this paper we uniformly adopt lowercase letters (e.g., $x$), boldface lowercase letters (e.g., $\bx=(x_i)$), capital letters (e.g., $X=(x_{ij})$), and calligraphic letters (e.g., $\X=(x_{i_1i_2\dots i_d})$) to denote scalars, vectors, matrices, and higher-order (order three or more) tensors, respectively. Lowercase Greek letters are usually denoted for constants and $\delta$'s, specifically, for universal constants. Denote $\R^{n_1\times n_2\times\dots\times n_d}$ to be the space of real tensors of order $d$ with dimension $n_1\times n_2\times\dots\times n_d$. The same notation applies to a vector space and a matrix space when $d=1$ and $d=2$, respectively. Denote $\BN$ to be the set of positive integers and $\BQ$ to be the set of rationals. In particular, all blackboard bold capital letters denote sets, such as $\R^n$, the standard basis $\BE^n:=\{\be_1,\be_2,\dots,\be_n\}$ of $\R^n$, the $\ell_p$-sphere $\BS^n_p:=\{\bx\in\R^n:\|\bx\|_p=1\}$. The superscript $n$ of a set, specifically, indicates that the concerned set is a subset of $\R^n$. 

For a vector $\bx\in\R^n$, denote $\|\bx\|_p:=(\sum_{i=1}^n |x_i|^p)^{1/p}$ to be the $\ell_p$-norm for $p\in[1,\infty]$ and $|\bx|\in\R^n$ to be the vector taking its absolute values element-wisely. We denote $D(\bx)\in\R^{n\times n}$ to be the matrix whose diagonal vector is $\bx$ and off-diagonal entries are zeros. For a square matrix $X\in\R^{n\times n}$, we denote $D(X)\in\R^{n\times n}$ to be the matrix by keeping only the diagonal vector of $X$, i.e., replacing off-diagonal entries with zeros. The $n$-dimensional all-zero vector is denoted by $\bd{0}_n$ and all-one vector by $\bd{1}_n$. The $n\times n$ identity matrix is denoted by $I_n$ and a zero matrix is denoted by $O$. The subscript of these special vectors and matrices is often omitted as long as there is no ambiguity. For symmetric matrices $A,B\in\R^{n\times n}$, $A\succeq O$ indicates that $A$ is positive semidefinite and $A\succeq B$ indicates that $A-B\succeq O$. The Frobenius inner product between two tensors $\U,\V\in\R^{n_1\times n_2\times\dots\times n_d}$ is defined as
\[
\langle\U, \V\rangle := \sum_{i_1=1}^{n_1}\sum_{i_2=1}^{n_2} \dots\sum_{i_d=1}^{n_d} u_{i_1i_2\dots i_d} v_{i_1i_2\dots i_d}.
\]
Its induced Frobenius norm is naturally defined as $\|\TT\|:=\sqrt{\langle\TT,\TT\rangle}$. The two terms automatically apply to tensors of order two (matrices) and tensors of order one (vectors) as well. This is the conventional norm (a norm without a subscript) used throughout the paper.

Three vector operations are used frequently, namely the outer product $\otimes$, the Kronecker product $\boxtimes$, and appending vectors $\vee$. In particular, if $\bx\in\R^{n_1}$ and $\by\in\R^{n_2}$, then
\begin{align*}
\bx\otimes\by&=\bx\by^{\T}\in\R^{n_1\times n_2} \\
\bx\boxtimes\by&=(x_1\by^{\T},x_2\by^{\T},\dots,x_{n_1}\by^{\T})^{\T}\in\R^{n_1n_2} \\
\bx\vee\by&=(x_1,x_2,\dots,x_{n_1},y_1,y_2,\dots,y_{n_2})^{\T}\in\R^{n_1+n_2}.
\end{align*}
These three operators also apply to vector sets via element-wise operations.

The notion $\Omega(f(n))$ means the same order of magnitude to $f(n)$, i.e., there exist positive constants $\alpha,\beta$ and $n_0$ such that $\alpha f(n)\le\Omega(f(n))\le\beta f(n)$ for all $n\ge n_0$. As a convention, the notation $O(f(n))$ means at most the same order of magnitude to $f(n)$.
  
 %The induced $p$-norm of a matrix $A\in\R^{m\times n}$ is defined as $\|A\|_p:=\max \left\{\|A \bx\|_p:\bx\in\BS_p^n\right\}$, and in the special case $p=\infty$ we have $\|A\|_{\infty}=\max _{1 \le i \le m} \sum_{j=1}^n\left|a_{i j}\right|$, while the $p\rightarrow q$ norm of the matrix is defined to be $\|A\|_{p \rightarrow q}:=\max \left\{\|A \bx\|_q:\bx\in\BS_p^n\right\}$, which also equals the spectral $p$-norm of $A$ as per Definition~\ref{def:spectral} below. % For a third-order tensor $\TT\in\R^{n_1\times n_2\times n_3}$, we use $\TT_i^{(1)}\in\R^{n_2\times n_3}$ to denote its $i$th horizontal slice, i.e., the matrix obtained by fixing the mode-$1$ index of $\TT$ to $i$. Denote $\BN$ and $\mathbb{Z}$ to be the set of positive integers and all integers, respectively. 

% The epigraph of a function $f:\R^n\rightarrow\R$ is denoted by $\operatorname{epi}f:=\{(\bx,\tau)\in\R^n\times\R:f(\bx)\le\tau\}$. We use $\operatorname{poly}(n)$ (resp. $\operatorname{exp}(n)$) to denote some polynomial (resp. exponential) function of $n$. We use $\lfloor\cdot\rfloor$ and $\lceil\cdot\rceil$ to denote the floor and ceil functions, 

% We also slightly abuse our notations to use calligraphic letters again and blackboard bold capital letters interchangeably to denote general sets (e.g., $\mathcal{C},\BH\subseteq\R^n$). The complement of a set $\mathcal{C}\subseteq\R^n$ is denoted by $\mathcal{C}^\complement$, i.e., $\mathcal{C}^\complement=\R^n\setminus\mathcal{C}$. The polar of a set $\mathcal{C}\subseteq\R^n$ is denoted by $\mathcal{C}^\circ := \left\{ \by \in \R^n: \langle \bx,\by \rangle \le 1 , \,\forall \bx \in \mathcal{C}\right\}$.
    % The polytope generated by a finite set of points $\BH^n\subseteq\BS_p^n$ is defined by $\mathcal{P}(\BH)=\bigcap_{\bu\in\BH^n}\left\{\by:\by^\T\bu\le 1\right\}$.
    % We denote the indicator function of a set $\mathcal{C}\subseteq\R^n$ as 
    % $$
    % \delta_0_\mathcal{C}(\bx):=\begin{cases}
    %     0 & \text{if $\bx\in\mathcal{C}$,}\\
    %     \infty & \text{otherwise,}
    % \end{cases}
    % $$ 
    % which maps from $\R^n$ to $\{0,\infty\}$.
    % \begin{align*}
    % \delta_0_\mathcal{C}:\bx&\mapsto\begin{cases}
    %     0 & \text{if $\bx\in\mathcal{C}$,}\\
    %     \infty & \text{otherwise,}
    % \end{cases}\\
    % \R^n &\rightarrow\{0,\infty\}.
    % \end{align*}

    % Finally, for any function $g:\R_+\rightarrow\R_+$, we denote
    % $$
    %  O(g(n)):=\left\{f(n) : \exists c,n_0>0, \text{ s.t. }  f(n) \le c g(n),\,\forall n \ge n_0\right\},
    % $$
    % $$
    % \Omega(g(n)):=\left\{f(n): \exists \delta_1, \delta_2, n_0>0, \text{ s.t. } \delta_1 g(n) \le f(n) \le \delta_2 g(n),\,\forall n \ge n_0\right\},
    % $$
    % and 
    % $$
    % \Theta(g(n)):=\left\{f(n) : \exists c,n_0>0, \text{ s.t. } c g(n) \le f(n),\,\forall n \ge n_0\right\},
    % $$
    % which are in comply with the conventions in~\cite{he2023approx,hou2014hardness}. In plain English, $f(n)\in\Omega(g(n))$ indicates that they are of the same order of growth, while $f(n)\in O(g(n))$ (resp. $f(n)\in\Theta(g(n))$) means that $f(n)$ is dominated by (resp. dominates) $g(n)$. 
    % \begin{align*}
    % \sign:x&\mapsto\begin{cases}
    %     1 & \text{if $x\ge 0$,}\\
    %     -1 & \text{otherwise,}
    % \end{cases}\\
    %     \R&\rightarrow \{1,-1\}.
    % \end{align*} 

\subsection{Spectral $p$-norm and nuclear $p$-norm}\label{sec:norm}

Given any tensor $\TT\in\R^{n_1 \times n_2 \times \dots \times n_d}$ and constant $p\in[1,\infty]$, the spectral $p$-norm~\cite{lim2005singular} of $\TT$ is defined as
\begin{equation}\label{def:spectral}
    \|\TT\|_{p_\sigma}:=\max \left\{\left\langle\TT, \bx_1 \otimes \bx_2 \otimes \dots \otimes \bx_d\right\rangle:\|\bx_k\|_p=1,\, k=1,2,\dots, d\right\}.
\end{equation}
This includes the matrix spectral $p$-norm for $d=2$ as its special case. In particular, the spectral $p$-norm reduces to the spectral norm when $p=2$.

A tensor $\TT\in\R^{n_1\times n_2\times\dots\times n_d}$ has $d$ modes, namely $1,2,\dots,d$. Fixing every mode index to a fixed value except the mode-$k$ index will result a vector in $\R^{n_k}$, called a mode-$k$ fiber. Fixing every mode index except two will result a matrix. In particular, fixing only one, say the mode-$k$ index to $i$ where $1\le i\le n_k$ will result a tensor of order $d-1$ in $\R^{n_1\times\dots \times n_{k-1}\times n_{k+1}\times\dots\times n_d}$. We call it the $i$th mode-$k$ slice, denoted by $\TT^{(k)}_i$. The mode-$k$ product of $\TT$ with a vector $\bx\in\R^{n_k}$ is denoted by
$$
\TT\times_k\bx:=\sum_{i=1}^{n_k}x_i\TT^{(k)}_i\in \R^{n_1\times\dots \times n_{k-1}\times n_{k+1}\times\dots\times n_d}.
$$
As a consequence, mode products with more vectors are obtained one by one, e.g.,
$$
\TT \times_1 \bx \times_2 \by =  (\TT \times_2 \by) \times_1 \bx = (\TT \times_1 \bx) \times_1 \by,
$$
where $\times_1 \by$ in the last equality is used instead of $\times_2 \by$ as mode $2$ of $\TT$ becomes mode $1$ of $\TT \times_1 \bx$. Mode products with $d-2$ vectors result a matrix, and with one more product result a vector. In particu;ar, one has
$$\label{eq:multilinear}
\langle \TT, \bx_1 \otimes\bx_2\otimes \dots \otimes \bx_d \rangle
= \langle \TT \times_1 \bx_1, \bx_2 \otimes \dots \otimes \bx_d \rangle=
\dots 
=    \TT \times_1 \bx_1 \times_2 \bx_2 \dots \times_d \bx_d
%=\sum_{i_1=1}^{n_1}\sum_{i_2=1}^{n_2}\dots\sum_{i_d=1}^{n_d}t_{i_1i_2\dots i_d}x^1_{i_1}x^2_{i_2}\dots x^d_{i_d},
$$
which is a multilinear form of $(\bx_1,\bx_2,\dots,\bx_d)$. By multilinearity, it means a linear form of $\bx_j$ by fixing all $\bx_k$'s but $\bx_j$ for every $j=1,2,\dots,d$. Therefore, the tensor spectral $p$-norm is to maximize a multilinear form over the Cartesian product of $\ell_p$-spheres. It was shown that the tensor spectral $p$-norm is equal to the largest $\ell_p$-singular value of the tensor~\cite[Proposition~1]{lim2005singular}. Mode product of a tensor with a vector on $\ell_p$-sphere will decrease the spectral $p$-norm in the weak sense.
\begin{lemma}\label{thm:contraction}
If $\TT\in\R^{n_1\times n_2\times\dots\times n_d}$, $p\in[1,\infty]$ and $\|\bx\|_p=1$, then $\|\TT\times_k \bx\|_{p_\sigma}\le\|\TT\|_{p_\sigma}$ for any $k$.
\end{lemma}
The proof can be easily obtained by comparing feasibility with optimality to the optimization model~\eqref{def:spectral} and noticing that
$\langle \TT, \bx_1 \otimes\bx_2\otimes \dots \otimes \bx_d \rangle
= \langle \TT \times_k \bx_k, \bx_1 \otimes \dots\otimes \bx_{k-1}\otimes \bx_{k+1} \otimes\dots\otimes \bx_d \rangle.$

% We use the shorthand notation
% $$
% \TT(\bx^1, \bx^2, \dots, \bx^d):=\langle\TT, \bx^1 \otimes \bx^2 \otimes \dots \otimes \bx^d\rangle=\sum_{i_1=1}^{n_1} \sum_{i_2=1}^{n_2} \dots \sum_{i_d=1}^{n_d} t_{i_1 i_2 \dots i_d}x^1_{i_1}x^2_{i_2} \dots x^d_{i_d},
% $$
% to denote the multilinear form of $(\bx^1,\bx^2,\dots,\bx^d)$. 
% It is easy to see that, if some vector, say $\bx_1$, is missing from the input of the mapping $\TT$, then the resulting quantity becomes a vector $\TT\left(\bullet, \bx_2, \bx_3, \dots, \bx_d\right) \in \R^{n_1}$, and similar applies to higher-order cases. 

The nuclear $p$-norm~\cite{friedland2018nuclear} of $\TT\in\R^{n_1\times n_2\times\dots\times n_d}$ is defined as
\begin{equation}\label{def:nuclear}
    \|\TT\|_{p_*}:=\min \left\{\sum_{i=1}^r\left|\lambda_i\right|: \TT=\sum_{i=1}^r \lambda_i\, \bx_i^1 \otimes \bx_i^2 \otimes \dots \otimes \bx_i^d,\, \|\bx_i^k\|_p=1 \text{ for all } k \text{ and } i,\,  r\in\BN \right\}.
\end{equation}
This also includes the matrix nuclear $p$-norm for $d=2$ and the nuclear norm for $p=2$ as its special cases. Similar to the well-known duality between the spectral norm and nuclear norm, the nuclear $p$-norm does be the dual norm of the spectral $p$-norm, and vise versa.
\begin{lemma}[{\cite[Lemma 2.5]{chen2020tensor}}]\label{lma:norm-duality}
    For any $\TT\in\R^{n_1\times n_2\times\dots\times n_d}$ and $p\in[1,\infty]$, it holds that 
    $$
    \|\TT\|_{p_\sigma} =\max\{\langle\TT, \mathcal{Z}\rangle: \|\Z\|_{p_*} \le 1\}   
    \text{ and }
    \|\TT\|_{p_*} =\max\{\langle\TT, \Z\rangle:\|\Z\|_{p_\sigma} \le 1\}.
    $$
\end{lemma}

For computational complexity of the spectral $p$-norm and nuclear $p$-norm, the beautiful duality result by Friedland and Lim~\cite{friedland2016computational} states that if one of them can be computed in polynomial time then the other also can, and if one of them is NP-hard to compute then the other is also NP-hard. For $d=1$, the vector case is trivial since the spectral $p$-norm becomes the $\ell_q$-norm where $\frac{1}{p}+\frac{1}{q}=1$ and nuclear $p$-norm becomes the $\ell_p$-norm. For $d=2$, the matrix spectral $p$-norm and nuclear $p$-norm can be computed in polynomial time when $p=1,2,\infty$ and are NP-hard when $p\in(2,\infty)$, while it remains unknown for the rest of $p$~\cite{steinberg2005computation}. However, for $d\ge3$, everything becomes NP-hard or unknown; see~\cite[Proposition~1]{hou2014hardness} and~\cite[Section~7]{friedland2018nuclear} for details. 

\subsection{Hitting set for $\ell_p$-sphere covering}\label{sec:lp-sphere-covering}

Throughout this paper, $p\in[1,\infty]$ is a given constant and $q\in[1,\infty]$ satisfies $\frac{1}{p}+\frac{1}{q}=1$. A set of vectors on $\ell_p$-sphere, $\BH^n=\{\bv_j \in \BS_p^n: j=1,2, \dots, m\}$, is called a hitting set of $\BS_p^n$ with hitting ratio $\tau\ge0$ and cardinality $m\in\BN$ if
$$
\bigcup_{j=1}^m \left\{\bx\in\BS_q^n:\bx^{\T}\bv_j\ge\tau\right\}=\BS_q^n.
$$
For simplicity, we usually call this $\BH^n$ a $\tau$-hitting set. Denote all the hitting sets of $\BS_p^n$ with hitting ratio at least $\tau$ and cardinality at most $m$ to be $\BT^n_p(\tau, m)$, i.e.,
$$
\BT^n_p(\tau, m):=\left\{\BH^n \subseteq \BS_p^n: \BH^n \text{ is a }  \tau \text {-hitting set and }|\BH^n|\le m\right\}.
$$
From the geometric point of view, a hitting set $\BH^n$ of $\BS_p^n$ is a representation of $\BS_p^n$ and approximately covers $\BS_q^n$, in the sense that any $\bx\in\BS_q^n$ is approximately covered by some $\bv_j\in\BH^n$ with $\bx^{\T}\bv_j\ge\tau$. By H\"{o}lder's inequality, $\bx^{\T}\bv_j\le1$ for $\bx\in\BS^n_p$ and $\bv_j\in\BS^n_q$ and so the best possible $\tau$ is obviously one when the hitting set is $\BS^n_p$ itself. The larger the $\tau$, the better coverage to $\BS_q^n$.
% Closely related to hitting sets on $\BS_p^n$ is the concept of $\epsilon$-nets on $\BS_p^n$, whose formal definition is given as below. 
% \begin{definition}\textnormal{\cite[Definition~4.2.1]{vershynin2018high}}
%     Given $\epsilon>0$, then a subset $\mathcal{N}\subseteq\BS_p^n$ is called an $\epsilon$-net if for any $\bx\in\BS_p^n$, there is some $\by\in\mathcal{N}$, s.t. $\|\by-\bx\|\le\epsilon$.
% \end{definition}
% In words, an $\epsilon$-net is a subset of the unit $\ell_p$-sphere s.t. any other point on the sphere has a distance at most $\epsilon$ to it. 
% At this place, we also make a remark on a caveat that is closely related to the hitting ratio of a hitting set. For two vectors $\bx,\by\in\BS^n$, the larger their inner product $\by^\T\bx$, the smaller the angle between them, necessarily, and vice versa. This is mainly due to the self-polarity of $\mathbb{B}^n$. However, on $\BS_p^n$, whose corresponding unit ball is not self-polar (i.e., $\mathbb{B}_p^n\ne\left(\mathbb{B}_p^n\right)^\circ=\mathbb{B}_q^n$), things become much different, and the above property does not necessarily hold in general. Indeed, for a pair of vectors $(\bx,\by)\in\BS_p^n\times\BS_q^n$, even if the  is tight on them, i.e., they have the largest possible inner product, they are not necessarily parallel, and in fact, they can even be nearly orthogonal in extreme cases such as $(\bx,\by):=(\bd{1}_n,\be_1)\in\BS_\infty^n\times \BS_1^n$. As we shall see, this peculiarity will pose some new challenges when we design randomized hitting sets of $\BS_p^n$ in Section~\ref{sec:rand-hitting-set}.

We end this section with the norm equivalence property between $\ell_p$-norms that is frequently used in the paper.
\begin{lemma}\label{lma:lp-norm-equiv}
 % If $1\le r\le p\le\infty$, then $\|\bx\|_{p}\le\|\bx\|_{r}\le n^{\frac{1}{r}-\frac{1}{p}}\|\bx\|_{p}$ for any $\bx\in\R^n$.
 If $1\le r\le p\le\infty$, then $\|\bx\|_{p}\le\|\bx\|_{r}\le n^{1/r-1/p}\|\bx\|_{p}$ for any $\bx\in\R^n$.
\end{lemma}


\section{Approximating matrix nuclear $p$-norm}\label{sec:matrix-p-norm}

This section is devoted to a polynomial-time method that approximates the matrix nuclear $p$-norm within a factor of $1/\delta_G >\frac{2}{\pi}\ln(1+\sqrt{2}) > 0.561$ when $p\in\BQ\cap(2,\infty)$, where $\delta_G$ %$<\frac{\pi}{2\ln(1+\sqrt{2})}<1.783$ 
is the well-known Grothendieck constant~\cite{Grothendieck}.

\subsection{From matrix spectral $p$-norm to matrix nuclear $p$-norm}\label{sec:kG-matrix-nuclear-p-norm}

As mentioned in Section~\ref{sec:norm}, both the matrix spectral $p$-norm and nuclear $p$-norm are NP-hard to compute when $p\in(2,\infty)$. While the approximation of the matrix nuclear $p$-norm is almost blank in the literature, the approximation of the matrix spectral $p$-norm has been well studied under an equivalent concept called the $p\rightarrow q$ norm. In particular, via a convex optimization relaxation, Nesterov~\cite{N00} showed that the matrix spectral $p$-norm can be approximated within a factor of $\frac{2\sqrt{3}}{\pi}-\frac{2}{3}>0.435$. Ben-Tal and Nemirovski~\cite{BN01} and Steinberg~\cite{steinberg2005computation} gave a better analysis of Nesterov's relaxation method but their approximation bound is better only for certain $p$'s and certain dimensions of the matrix. Hou and So~\cite{hou2014hardness} provided the best approximation bound $1/\delta_G>0.561$ that uniformly beats all the above. We quote their result that is used in our analysis for the approximation of the nuclear $p$-norm.
\begin{lemma}[{\cite[Proposition~3]{hou2014hardness}}]\label{lma:vecp}
For any matrix $A\in\R^{m\times n}$ and constant $p\in\BQ\cap(2,\infty)$,
% \begin{equation}\label{opt:vecp-original}
% \operatorname{vec}_p(A)= \begin{cases}\max \left\{\langle A', X\rangle: \sum_{i=1}^m\left|x_{i i}\right|^{p/2} \le 1,\, \sum_{j=m+1}^{m+n}\left|x_{j j}\right|^{p/2} \le 1,\, X \succeq O\right\} & \text { for } p \in(2, \infty), \\ \max \left\{\langle A', X\rangle: \underset{1 \le i \le m}{\max}\left|x_{i i}\right| \le 1,\, \underset{m+1 \le j \le m+n}{\max}\left|x_{j j}\right| \le 1,\, X \succeq O\right\} & \text { for } p=\infty ,\end{cases}
% \end{equation}
\begin{equation}\label{opt:vecp-original}
\|A\|_{p_v}:= \max \left\{\left\langle \begin{pmatrix}
O & A/2 \\
A^\T/2 & O
\end{pmatrix}, X\right\rangle: \sum_{i=1}^m\left|x_{i i}\right|^{p/2} \le 1,\, \sum_{i=m+1}^{m+n}\left|x_{ii}\right|^{p/2} \le 1,\, X \succeq O\right\}
\end{equation}
% where $B=\begin{pmatrix}
% O & B/2 \\
% B^\T/2 & O
% \end{pmatrix}\in\R^{(m+n)\times(m+n)}$, 
satisfies $\|A\|_{p_v}/\delta_G  \le \|A\|_{p_\sigma} \le \|A\|_{p_v}$, where $\delta_G$ is the Grothendieck constant.
\end{lemma}

% \begin{enumerate}
%     \item \textbf{Relaxing (\ref{eq:nuclear-pnorm-pqnorm}) within a constant factor by using a polynomial-time computable relaxation of the matrix spectral $p$-norm (Section~\ref{sec:kG-matrix-nuclear-p-norm}).} Although the matrix spectral $p$-norm is NP-hard to compute as aforementioned, fortunately,~\cite[Proposition~3]{hou2014hardness} shows that, actually, it can be well approximated by a polynomial-time solvable convex optimization problem, as follows. As a result, by replacing the constraint $\|\Z\times_1\bx\|_{p_\sigma}\le 1$ by $\| \Z\times_1\bx\|_{p_v})\le 1$ in (\ref{eq:nuclear-pnorm-pqnorm}), we shall hopefully obtain a constant-factor relaxation of (\ref{eq:nuclear-pnorm-pqnorm}) while getting rid of the NP-hardness induced by the previously-used matrix spectral $p$-norm.
% % while making the function involved in the constraints is now polynomial-time computable.
% We stress here that, in the sense of approximation, the two problems are still equivalent.\label{enum:pt1}

%     \item \textbf{Transforming the resulting constraints involving maximization into regular ones by studying the dual of $\operatorname{vec}_p(\cdot)$ (Section~\ref{sec:dual-vecp}).} It should be noted that, the characterization of $\operatorname{vec}_p(\cdot)$ (i.e., (\ref{opt:vecp-original})) is in essence a convex maximization problem, which means that the constraint $\| \Z\times_1\bx\|_{p_v}\le 1$ is actually equivalent to a `for all' argument, which is still hard to be handled due to its enumerative nature. Fortunately, the duality does help in this case. In fact, once the strong duality holds for $\operatorname{vec}_p(\cdot)$ and its dual, then we can further equivalently use its dual, which is essentially a minimization problem and actually equivalent to a `there exist' argument that can be easily handled, to replace itself in the constraints without any loss.
%     This nice strategy originates from the field of robust optimization~\cite{goldfarb2003robust,huang2021projection}.

%     \item \textbf{Establishing the SDP-representability of both the dual of $\operatorname{vec}_p(\cdot)$ and consequently the relaxed problem (Section~\ref{sec:SDP-relax-nuclear-p-norm}).} Although the difficulty of the approximation problem has been greatly reduced through the above two processes, there is still a remaining tricky problem to be addressed: It is not clear if the dual of $\operatorname{vec}_p(\cdot)$ can be equivalently posed as some catalogue problems such as an SDP, as only in this case can the aforementioned relaxed problem admit an SDP characterization but with uncountably many semidefinite constraints, and if it can really be, how. As we will see soon, the objective function of the dual of $\operatorname{vec}_p(\cdot)$ is rather complicated, which reflects the difficulty of the problem. Nevertheless, we will utilize techniques in modern convex optimization developed by Ben-Tal and Nemirovski~\cite[Section~3]{ben2001lectures} to carefully address this issue.
% \end{enumerate}

The quantity $\|A\|_{p_v}$ defines a matrix norm that can be solved to arbitrary accuracy in polynomial time using, e.g., the ellipsoid method~\cite{grotschelbook}. In order to approximate the matrix nuclear $p$-norm using the duality in Lemma~\ref{lma:norm-duality}, $\|A\|_{p_*}=\max\{\langle A, Z\rangle:\|Z\|_{p_\sigma} \le 1\}$,
one is naturally suggested using $\|\cdot\|_{p_v}$ to replace $\|\cdot\|_{p_\sigma}$ that is NP-hard to compute. %, i.e.,
% \begin{equation}\label{eq:matrix-nuclear-pnorm-vecp}
% \max\left\{\langle A,Z\rangle:\|Z\|_{p_v}\le 1\right\}.
% \end{equation}
In fact, we have the following observation. 
\begin{lemma}\label{prop:matrix-equi-pq-vecp}
For any matrix $A\in\R^{m\times n}$ and constant $p \in\BQ\cap(2, \infty)$,
\begin{align}\label{eq:matrix-nuclear-pnorm-pqnorm}
    \|A\|_{p_*}=\max\left\{\langle A, Z\rangle:\|Z\|_{p_\sigma} \le 1\right\},
\end{align}
and
\begin{equation}\label{eq:matrix-nuclear-pnorm-vecp}
\|A\|_{p_u}:=\max\left\{\langle A,Z\rangle:\|Z\|_{p_v}\le 1\right\}
\end{equation}
are equivalent in the sense that $\|A\|_{p_u}\le \|A\|_{p_*}\le \delta_G \|A\|_{p_u}$.
\end{lemma}
\begin{proof}
   We see from Lemma~\ref{lma:vecp} that $\|Z\|_{p_v}\le 1$ implies $\|Z\|_{p_\sigma}\le 1$. Therefore,~\eqref{eq:matrix-nuclear-pnorm-pqnorm} is a relaxation of~\eqref{eq:matrix-nuclear-pnorm-vecp}, implying that $\|A\|_{p_u}\le \|A\|_{p_*}$.
   
   On the other hand, let $Y$ be an optimal solution of~\eqref{eq:matrix-nuclear-pnorm-pqnorm}, i.e., $\|A\|_{p_*}=\langle A,Y\rangle$ and $\|Y\|_{p_\sigma}\le1$. By Lemma~\ref{lma:vecp} again, one has 
   $$
   \|Y/\delta_G\|_{p_v}=\|Y\|_{p_v}/\delta_G\le  \|Y\|_{p_\sigma}\le1,
   $$
   implying that $Y/\delta_G$ is a feasible solution to~\eqref{eq:matrix-nuclear-pnorm-vecp}. Therefore, $\|A\|_{p_u}\ge \langle A,Y/\delta_G\rangle=\|A\|_{p_*}/\delta_G$, proving the upper bound.
\end{proof}

For the purpose of obtaining a constant approximation bound of the matrix nuclear $p$-norm, it suffices to study~\eqref{eq:matrix-nuclear-pnorm-vecp} because of Lemma~\ref{prop:matrix-equi-pq-vecp}. Reformulate~\eqref{eq:matrix-nuclear-pnorm-vecp} in a more explicit way, we have
\begin{equation}\label{eq:matrix-nuclear-pnorm-vecp-essence}
\|A\|_{p_u}=\max \left\{\left\langle A,Z\right\rangle: \left\langle \begin{pmatrix}
O & Z/2 \\
Z^\T/2 & O
\end{pmatrix}, X\right\rangle\le1 \text{ for all } X\in\mathbb{M}^{(m+n)\times (m+n)}\right\},
\end{equation}
where
$$
\mathbb{M}^{(m+n)\times (m+n)}:=\left\{X\in\R^{(m+n)\times(m+n)}: \sum_{i=1}^m\left|x_{i i}\right|^{p/2} \le 1,\, \sum_{i=m+1}^{m+n}\left|x_{ii}\right|^{p/2} \le 1,\, X \succeq O\right\}.
$$
The maximization formulation of $\|A\|_{p_v}$ in~\eqref{opt:vecp-original} makes~\eqref{eq:matrix-nuclear-pnorm-vecp-essence} difficult to handle because of an infinite number of constraints in $X\in\mathbb{M}^{(m+n)\times (m+n)}$. Inspired by a technique in robust optimization~\cite{ben2009book, goldfarb2003robust}, the constraint can be completely explicit if one can equivalently transform the maximization problem to a minimization one. This motivates us to study the dual problem of~\eqref{opt:vecp-original}.

\subsection{Equivalent formulations for $\|\cdot\|_{p_v}$ and $\|\cdot\|_{p_u}$}\label{sec:dual-vecp} 

We first derive an equivalent formulation of $\|\cdot\|_{p_v}$ via the dual problem of~\eqref{opt:vecp-original}. 
% , which can help us transform the irregular constraint $\|Z\|_{p_v}\le 1$ into a regular one.

% For simplifying the computations of our derivations, we first make some equivalent reformulations to the primal characterization of $\operatorname{vec}_p(\cdot)$, i.e., (\ref{opt:vecp-original}), as detailed in the following lemma.
% \begin{lemma}\label{lma:equi-vecp}
% (\ref{opt:vecp-original}) can be equivalently written as
% \begin{equation}\label{opt:vecp-finite}
%     \begin{aligned}
%     & \max
%     & &  \langle B, X-\Diag(X)\rangle \\
%     & \st
%     & & \sum_{i=1}^m |x_{i i}|^{p/2} \le 1,\, \sum_{j=m+1}^{m+n}|x_{j j}|^{p/2} \le 1, \\
%     \span&& X \succeq O,\, \underset{1\le k\le m+n}{\min}x_{kk}\ge 0.
%     \end{aligned}
% \end{equation}
% % for $p\in(2,\infty)$.
% % , and
% % \begin{equation}\label{opt:vecp-infinite}
% %     \begin{aligned}
% %     & \max
% %     & &  \langle B, X-\Diag(X)\rangle \\
% %     & \st
% %     & & \underset{1 \le i \le m}{\max}|x_{i i}| \le 1, \underset{m+1 \le j \le m+n}{\max}|x_{j j}| \le 1, \\
% %     \span&& X \succeq O,\, \underset{1\le k\le m+n}{\min}x_{kk}\ge 0,
% %     \end{aligned}
% % \end{equation}
% % for $p=\infty$,
% % $$
% % \hspace{-2cm}
% % \operatorname{vec}_p(B)=
% % \begin{cases}\max \left\{\left\langle B, X-\Diag(X)\right\rangle: \sum_{i=1}^m |x_{i i}|^{p/2} \le 1,\, \sum_{j=m+1}^{m+n}|x_{j j}|^{p/2} \le 1,\, X \succeq O,\, \underset{1\le k\le m+n}{\min}x_{kk}\ge 0\right\} & \text { for } p \in(2, \infty), \\ \max \left\{\left\langle B, X-\Diag(X)\right\rangle: \underset{1 \le i \le m}{\max}|x_{i i}| \le 1, \underset{m+1 \le j \le m+n}{\max}|x_{j j}| \le 1, X \succeq O,\, \underset{1\le k\le m+n}{\min}x_{kk}\ge 0\right\} & \text { for } p=\infty ,\end{cases}
% % $$
% \end{lemma}
% % The proof is nearly trivial once one notices that $\Diag(B)=O$. It is remarked that, the above characterization is still a convex optimization problem.

% \begin{proof}
%     % We only focus on the case for $p\in(2, \infty)$ as the remaining case is similar.
%     By the structure of $B$, we see that $\Diag(B)=O$. Therefore, the objective function $\langle B, X\rangle=\langle B, X-\Diag(X)\rangle$.
% %     Besides, if some optimal solution $\Y$ strictly satisfies, say, the first inequality, i.e. $\sum_{i=1}^m\left|y_{i i}\right|^{p/2} < 1$, then we can generate another matrix $\Y+\begin{pmatrix}
% % \epsilon I_m & O \\
% % O & O
% % \end{pmatrix}$ for some suitably chosen $\epsilon>0$ to fulfill the equality, and the resulting matrix is still positive semidefinite and the objective function value is not changed.
% Besides, since $X\succeq O$, all its diagonal elements are nonnegative, and therefore
% % we can remove the absolute mapping in constraints and
% the additional constraint $\underset{1\le k\le m+n}{\min}x_{kk}\ge 0$ will not influence optimal solutions.
% \end{proof}

% % As (\ref{opt:vecp-finite}) and (\ref{opt:vecp-infinite}) are totally the same as (\ref{opt:vecp-original}), we next derive the dual formulations of (\ref{opt:vecp-finite}) and (\ref{opt:vecp-infinite}) instead.

% As (\ref{opt:vecp-finite}) is totally the same as (\ref{opt:vecp-original}), we next derive the dual formulation of (\ref{opt:vecp-finite}) instead.

% % Since the diagonal entries of $B$ are all zero, we see that the. Moreover, since $X\succeq O$, we in addition have $|x_{j j}|$

% % Let us split $X=X_1+X_2$, where $X_1$ is its diagonal part while $X_2$ is the remaining part, and split similarly for $B=B_1+B_2$ and $ V= V_1+ V_2$. Since $X\succeq O$, we see that $X_1=|X_1|$. Moreover, by its structure, we have $B_1=O$ and thus $B_2=B$.

\begin{proposition}\label{lma:dual-vecp}
    The dual problem of~\eqref{opt:vecp-original} is
    \begin{equation}\label{opt:dual-vecp-finite}
    \begin{array}{lll}
    \|A\|_{p_v}= &\min & u_1+ u_2+\theta_p\sum_{i=1}^{m+n} t_i    \\
    &\st   & {v_i}^{p/(p-2)}\le t_i {u_1}^{2/(p-2)}\quad i=1,2,\dots, m \\
    && {v_i}^{p/(p-2)}\le t_i {u_2}^{2/(p-2)}\quad i=m+1,m+2,\dots, m+n\\ 
    &&  u_1\ge0, \,u_2\ge 0, \,   \bt\ge {\bf 0},\, \Diag(\bv)\succeq \begin{pmatrix} O & A/2 \\ A^\T/2 & O \end{pmatrix},
    \end{array}
    \end{equation}
    % \begin{equation}\label{opt:dual-vecp-finite}
    % \begin{aligned}
    % & \min
    % & &   u_1+ u_2+\underbrace{\left(\left(\frac{2}{p}\right)^{2/(p-2)}-\left(\frac{2}{p}\right)^{p/(p-2)}\right)}_{=:\theta_p}\left(\sum_{i=1}^m g_p(v_i,u_1)+\sum_{j=m+1}^{m+n} g_p( v_j, u_2)\right)=: h_p\left( u_1, u_2,\bv,\bt\right)\\
    % & \st
    % & &  u_1\ge0, \,u_2\ge 0,\,\Diag(\bv)\succeq B,
    % \end{aligned}
    % \end{equation}
    % for $p\in(2,\infty)$,
    % while the dual formulation of (\ref{opt:vecp-infinite}) is
    % \begin{equation}\label{opt:dual-vecp-infinite}
    % \begin{aligned}
    % & \min
    % & &  \sum_{i=1}^{m+n} v_i=:\zeta_\infty\left(\bv\right)\\
    % & \st
    % & & \Diag(\bv)\succeq B.
    % \end{aligned}
    % \end{equation}
    % for $p=\infty$.
    where $\theta_p:=(2/p)^{2/(p-2)}-(2/p)^{p/(p-2)}>0$ when $p\in(2,\infty)$.
%    Besides, strong duality holds for (\ref{opt:vecp-original}) and (\ref{opt:dual-vecp-finite}).
    % for both cases,
    % i.e.,
    % we actually have that
    % $\operatorname{vec}_p(B)=\min_{ u_1\ge0, \,u_2\ge 0,\,\Diag(\bv)\succeq B}\allowbreak h_p\left( u_1, u_2,\bv,\bt\right)$.
    % for $p\in(2,\infty)$, and $\operatorname{vec}_p(B)=\min_{\Diag(\bv)\succeq B}\zeta_\infty\left(\bv\right)$ for $p=\infty$.
\end{proposition}
\begin{proof}
Denote $B=\begin{pmatrix}O & A/2 \\A^\T/2 & O\end{pmatrix}$. The Lagrangian function associated with the maximization problem~\eqref{opt:vecp-original} is
\begin{equation*}
\begin{aligned}
    &~~~f(X, u_1, u_2, V)\\
    &=\langle B, X\rangle+ u_1 \left(1-\sum_{i=1}^m |x_{i i}|^{p/2}\right) +  u_2\left(1-\sum_{j=m+1}^{m+n}|x_{j j}|^{p/2}\right) + \left\langle  V, X\right\rangle
    % + \delta_0_{\left\{\underset{1\le k\le m+n}{\min}x_{kk}\ge 0\right\}}(X)
    \\&=u_1+u_2-\sum_{i=1}^m u_1|x_{i i}|^{p/2}-\sum_{j=m+1}^{m+n}u_2|x_{ii}|^{p/2}+ \left\langle B+ V, X\right\rangle
    \\&=u_1+u_2-\sum_{i=1}^m u_1|x_{i i}|^{p/2}+\sum_{i=1}^{m} v_{i i}x_{i i}-\sum_{j=m+1}^{m+n}u_2|x_{ii}|^{p/2}+ \sum_{i=m+1}^{m+n} v_{ii}x_{ii} + \left\langle B+ V-\Diag(V), X-\Diag(X)\right\rangle,
    % +\delta_0_{\left\{\underset{1\le k\le m+n}{\min}x_{kk}\ge 0\right\}}(X),
\end{aligned}
\end{equation*}
where $V\succeq O$ is the multiplier associated with the constraint $X\succeq O$ and $u_1\ge0$ and $u_2\ge 0$ are the multipliers associated with the constraints $\sum_{i=1}^m |x_{i i}|^{p/2}\le 1$ and $\sum_{i=m+1}^{m+n}|x_{i}|^{p/2}\le 1$, respectively. The last equality holds because $\Diag(B)=O$. It is purposely written as the sum of two parts with the first part involving only the diagonal entries of $X$ while the second part (the inner product) involving only the off-diagonal entries of $X$.

Let us analyze $\max\left\{f(X, u_1, u_2, V):X\in\R^{(m+n)\times(m+n)}\right\}$ based on the two parts in the last equality. First we must have $B+V-\Diag(V)=O$ since $X$ is a free matrix variable and any off-diagonal entry of $X$ appears only in $\left\langle B+ V-\Diag(V), X-\Diag(X)\right\rangle$. It remains to deal with the part involving the diagonal entries of $X$. In fact, this part enjoys a separable structure for all $x_{ii}$'s, each of which appears in a same type of subproblem $\max\left\{-u |x|^{p/2}+vx:x\in\R\right\}$ with $u,v\ge0$. This is because of $u_1\ge0, \,u_2\ge 0$ and $V\succeq O$.

Let us consider the following cases for $\max\left\{-u |x|^{p/2}+vx:x\in\R\right\}$ with $u,v\ge0$.
\begin{itemize}
    % \item If $ u_1< 0$, then clearly $\inf\left\{ u_1 |x_{i i}|^{p/2}-  v_{i i}x_{i i}:x_{ii}\in\R\right\}=-\infty$ since $ v_{ii}\in\R$. Similar analysis applies to other $ v_{jj}$'s and $ u_2$.
    \item If $v=0$ and $u\ge0$, then $\max\left\{-u|x|^{p/2}+vx:x\in\R\right\}=0$.
    \item If $v>0$ and $u=0$, then $\max\left\{-u |x|^{p/2}+vx:x\in\R\right\}=+\infty$.
    \item If $v>0$ and $u>0$, by noticing that an optimal $x$ must be nonnegative and $p\in(2,\infty)$, it can be calculated that 
    $$
    \max\left\{-u |x|^{p/2}+vx:x\in\R\right\}=\left(\left(\frac{2}{p}\right)^{2/(p-2)}-\left(\frac{2}{p}\right)^{p/(p-2)}\right)\frac{v^{p/(p-2)}}{u^{2/(p-2)}}=\frac{\theta_pv^{p/(p-2)}}{u^{2/(p-2)}}.
    $$
\end{itemize}
To summarize, we have
$$
\max\left\{-u |x|^{p/2}+vx:x\in\R\right\}=\theta_pg_p(u,v),
$$
where
\begin{align*}
     g_p(u,v)&=\begin{cases}
    \frac{ v^{p/(p-2)}}{ u^{2/(p-2)}} &u> 0 \\
    0 & u=v=0 \\
    +\infty & u=0, v>0.
    \end{cases}
%   \R_+^2&\rightarrow\R_+\cup\{\infty\}.
\end{align*}

With the above calculation, we obtain the following dual formulation of~\eqref{opt:vecp-original} by minimizing $\max\left\{f(X, u_1, u_2, V):X\in\R^{(m+n)\times(m+n)}\right\}$ over $u_1,u_2\ge0$ and $V\succeq O$
    \begin{equation} \label{eq:dualproof}
    \begin{array}{ll}
  \min &u_1+ u_2+\theta_p\sum_{i=1}^m g_p(u_1, v_{ii})+\theta_p\sum_{i=m+1}^{m+n} g_p(u_2, v_{ii})\\
 \st
    &  B+ V-\Diag( V)=O \\
    &  u_1\ge0, \,u_2\ge 0,\,  V\succeq O.
  %  &  u_1\ge0, \,u_2\ge 0,\, V\succeq O,\,B+ V-\Diag( V)=O \\
    \end{array}
    \end{equation}

It remains a couple of treatments in order to make the above formulation simpler and explicit. First, the off-diagonal part of $V$ must be $-B$ since $\Diag(B)=O$ and $B+V-\Diag(V)=O$ in the constraint. $V\succeq O$ then becomes $\Diag(V)-B\succeq O$ and this makes the off-diagonal entries of $V$ disappeared completely in~\eqref{eq:dualproof}. We use a new notation $v_i$ to replace $v_{ii}$ for $i=1,2,\dots,m+n$ and denote $\bv\in\R^{m+n}$ be the vector consisting of these $v_i$'s. Therefore, $B+ V-\Diag( V)=O$ and $V\succeq O$ are equivalent to $\Diag(\bv)-B\succeq O$. % i.e., $\Diag(\bv)\succeq B$.
Next, as $\theta_p>0$ and~\eqref{eq:dualproof} is a minimization problem, we may let $g_p(u_1,v_{i})=t_i$ for $i=1,2,\dots,m$ and $g_p(u_2,v_{i})=t_i$ for $i=m+1,m+2,\dots,m+n$ in the objective function and add additional constraints $g_p(u_j,v_{i})\le t_i$ for corresponding $j$ and $i$. An important observation is that 
$$
\min\left\{ t: g_p(u,v)\le t \right\} = \min\left\{ t: v^{p/(p-2)} \le t u^{2/(p-2)},\,t\ge0 \right\} \text{ when }u,v\ge0.
$$
With these two observations,~\eqref{eq:dualproof} can be equivalently formulated to
% Besides, we claim that the above problem can be further equivalently simplified to
% \begin{equation*}
%     \begin{aligned}
%     & \min
%     & &   u_1+ u_2+\left(\left(\frac{2}{p}\right)^{2/(p-2)}-\left(\frac{2}{p}\right)^{p/(p-2)}\right)\left(\sum_{i=1}^m g_p(v_i,u_1)+\sum_{j=m+1}^{m+n} g_p( v_j, u_2)\right)\\
%     & \st
%     & &  u_1\ge0, \,u_2\ge 0,\,\Diag(\bv)\succeq B.
%     \end{aligned}
% \end{equation*}
% This is because on one hand, the simplified problem is clearly a relaxation of the original one, while on the other hand, if $( u_1, u_2,\bv)$ is optimal to the simplified problem, then $( u_1, u_2,\Diag(\bv)-B)$ is feasible to the original one, and the corresponding objective function value remains unchanged since it is clearly only relevant to the diagonal part of $ V$ (recall $B$ has a zero diagonal). 
% Moreover, it is easy to verify that
% $$
% \left\{(v,u,t)\in\R_+^3:g_p(v,u)\le t\right\}=\left\{(v,u,t)\in\R_+^3:v^{p/(p-2)}\le t u^{2/(p-2)}\right\}.
% $$
    \begin{equation*} 
    \begin{array}{ll}
  \min &u_1+ u_2+\theta_p\sum_{i=1}^{m+n} t_i\\
    \st   & {v_i}^{p/(p-2)}\le t_i {u_1}^{2/(p-2)}\quad i=1,2,\dots, m \\
    & {v_i}^{p/(p-2)}\le t_i {u_2}^{2/(p-2)}\quad i=m+1,m+2,\dots, m+n\\ 
    &  u_1\ge0, \,u_2\ge 0,\, \bt\ge{\bf 0},\, \Diag(\bv)-B\succeq O.
    \end{array}
    \end{equation*}

Finally, to show that the objective value of the above optimization problem is equal to $\|A\|_{p_v}$, we only need to show the strong duality. In fact, by letting $Y=\frac{1}{2}\begin{pmatrix}m^{-2/p}I_m & O \\
O & n^{-2/p}I_n
\end{pmatrix}$, one has $\sum_{i=1}^m |y_{i i}|^{p/2} = \sum_{i=m+1}^{m+n}|y_{ii}|^{p/2}= (\frac{1}{2})^{p/2} < 1$ and $Y\succ O$, i.e., $Y$ is strictly feasible to~\eqref{opt:vecp-original}. This means that the Slater's condition holds for the convex optimization problem~\eqref{opt:vecp-original} and hence the strong duality holds~\cite[Section~5.2.3]{boyd2004convex}.
% In a similar vein, we can also easily deduce the dual formulation for $\operatorname{vec}_p(B)$ when $p=\infty$, which is
%     \begin{equation*}
%     \begin{aligned}
%     & \min
%     & &  \sum_{i=1}^{m+n} v_i\\
%     & \st
%     & & \Diag(\bv)\succeq B,
%     \end{aligned}
%     \end{equation*}
%     and check that the strong duality holds in this case as well. We leave the details for this case as a routine exercise to interested readers.
\end{proof}

% From now on, to simplify notation, we denote $\theta_p:=\left(\left(\frac{2}{p}\right)^{p/(p-2)}-\left(\frac{2}{p}\right)^{2/(p-2)}\right)$.

% It is remarked that, by letting $p\rightarrow\infty$, the first formulation tends to the second, which shows some `continuity'.
% As our main task is to transform the constraint $\| \Z\times_1\bx\|_{p_v}\le 1$ in (\ref{eq:matrix-nuclear-pnorm-vecp}) into something like a set of semidefinite constraints with which the resulting problem (if no other annoying obstacles admit) can be solved in polynomial time, we only focus on the case where $p\in (2, \infty)$ in the sequel as the dual formulation for (\ref{opt:vecp-infinite}), i.e. (\ref{opt:dual-vecp-infinite}), is already an SDP, clearly.
% Equipped with Proposition~\ref{lma:dual-vecp}, we are now able to equivalently write (\ref{eq:matrix-nuclear-pnorm-vecp}) as
% As discussed previously, to design approximation algorithms for $\|\TT\|_{p_*}$, it suffices to design approximation algorithms for (\ref{eq:matrix-nuclear-pnorm-vecp}).
% the following optimization problem
% \begin{align*}
%     \max\left\{\langle\TT,\Z\rangle:\| \Z\times_1\bx\|_{p_v}\le 1 \text{ for all } \bx\in\BS_p^{n_1}\right\}.
% \end{align*}
% As a result of Proposition~\ref{lma:dual-vecp}, we see that (\ref{eq:matrix-nuclear-pnorm-vecp}) is further equivalent to the following simpler problem

Armed with Proposition~\ref{lma:dual-vecp}, the infinite number of constraints in~\eqref{eq:matrix-nuclear-pnorm-vecp-essence} hold if and only if there exists one feasible solution in~\eqref{eq:dualproof} satisfying this constraint. As a result,
we are able to 
equivalently reformulate $\|A\|_{p_u}$, i.e., either~\eqref{eq:matrix-nuclear-pnorm-vecp} or~\eqref{eq:matrix-nuclear-pnorm-vecp-essence}, to a completely explicit optimization problem.
% Recall that (\ref{eq:matrix-nuclear-pnorm-vecp}) and (\ref{eq:matrix-nuclear-pnorm-vecp-essence}) are the same thing.
% \begin{align*}
%     \max\left\{\langle\TT,\Z\rangle:\min\left\{ h_p\left( u_1, u_2,\bv,\bt\right): u_1\ge0, \,u_2\ge 0,\,\Diag(\bv)\succeq\frac{1}{2}\begin{pmatrix}
%     O_{n_2} & \Z\times_1\bx \\
%     \Z\times_1\bx^\T & O_{n_3}
%     \end{pmatrix}\right\}\le 1 \text{ for all } \bx\in\BS_p^{n_1}\right\}.
% \end{align*}
\begin{corollary}\label{cor:pnorm-simpler}
For any matrix $A\in\R^{m\times n}$ and constant $p \in\BQ\cap(2, \infty)$,
 \begin{equation}\label{eq:matrix-nuclear-pnorm-vecp-dual}
    \begin{array}{lll}
    \|A\|_{p_u}= &\max & \langle A,Z\rangle    \\
    &\st   &u_1+ u_2+\theta_p\sum_{i=1}^{m+n} t_i\le 1\\
    && {v_i}^{p/(p-2)}\le t_i {u_1}^{2/(p-2)}\quad i=1,2,\dots, m \\
    && {v_i}^{p/(p-2)}\le t_i {u_2}^{2/(p-2)}\quad i=m+1,m+2,\dots, m+n\\ 
    &&  u_1\ge0, \,u_2\ge 0, \,   \bt\ge {\bf 0},\, \Diag(\bv)\succeq \begin{pmatrix}
O & Z/2 \\
Z^\T/2 & O
\end{pmatrix}.
    \end{array}
    \end{equation}
\end{corollary}

Problem~\eqref{eq:matrix-nuclear-pnorm-vecp-dual} greatly relieves the unpleasant constraint $X\in\mathbb{M}^{(m+n)\times (m+n)}$ in~\eqref{eq:matrix-nuclear-pnorm-vecp-essence}. However, it is still not perfect because of a series of nonlinear constraints of the type ${v}^{p/(p-2)}\le t {u}^{2/(p-2)}$ albeit the remaining parts are either linear constraint or semidefinite constraint. In fact, $\{(u,v,t)\in\R_+^3:v^{p/(p-2)}\le t u^{2/(p-2)}\}$ admits an SOC representation, shown in the next subsection.

% \begin{itemize}
%     \item \textbf{How can the constraint $ h_p\left( u_1, u_2,\bv,\bt\right)\le 1$ be handled?} By its definition in (\ref{opt:dual-vecp-finite}), $ h_p\left( u_1, u_2,\bv,\bt\right)$ is a rather complicated function, as its major component $ g_p( v, u)$ is almost a quotient of $ v$ and $ u$. At a first glance, it seems impossible to equivalently write $ h_p\left( u_1, u_2,\bv,\bt\right)\le 1$ as some catalogue constraints such as SOC or semidefinite ones.
%     % \item \textbf{How to deal with the uncountability of $\BS_p^{n_1}$?} Clearly, there is no way to enumerate all points on $\BS_p^{n_1}$ in polynomial time since they are uncountable. Therefore, even checking the feasibility of the last constraint in (\ref{eq:matrix-nuclear-pnorm-vecp-dual}) is not tractable, let alone solving an optimization problem involving it as only a part of the constraints.
% \end{itemize}

% Therefore, (\ref{eq:matrix-nuclear-pnorm-vecp-dual}) is still a rather challenging problem. To find a breakthrough of it, we shall perform some reformulations. Recall from optimization theory a useful trick~\cite{ben2001lectures}: We can always pass from an optimization problem to an equivalent one with a linear objective function. By using this trick, (\ref{opt:dual-vecp-finite}) is equivalent to
% \begin{equation*}\label{opt:dual-vecp-finite-reform}
%     \begin{aligned}
%     & \min
%     & &   u_1+ u_2+\left(\left(\frac{2}{p}\right)^{2/(p-2)}-\left(\frac{2}{p}\right)^{p/(p-2)}\right)\sum_{i=1}^{m+n} t_i\\
%     & \st
%     & &  u_1\ge0, \,u_2\ge 0,\,\Diag(\bv)\succeq B,\\
%     \span &&(v_i,u_1,t_i)\in\operatorname{epi} g_p  \text{ for all }  i\in[m],\\
%     \span &&( v_j, u_2,t_j)\in\operatorname{epi} g_p  \text{ for all }  j\in [m+n]\setminus[m].
%     \end{aligned}
% \end{equation*}

% Our main task in the next subsection is to establish that,
% However, as established in the next subsection,

% \subsection{The SOC-representability of $\{(v,u,t)\in\R_+^3:v^{p/(p-2)}\le t u^{2/(p-2)}\}$ and its implication.}
\subsection{Conic optimization model for $\|\cdot\|_{p_u}$}\label{sec:SDP-relax-nuclear-p-norm}
% Our first question is if the dual formulation of $\operatorname{vec}_p(B)$ can be cast as an SDP.
% In this subsection, we establish the results mentioned in the title
% the SOC-representability of $\{(v,u,t)\in\R_+^3:v^{p/(p-2)}\le t u^{2/(p-2)}\}$ and consequently the SDP-representability of (\ref{opt:dual-vecp-finite}) and (\ref{eq:matrix-nuclear-pnorm-vecp-dual})

% that will be used in the subsequent analysis.
Let us start to deal with the constraint $v^{p/(p-2)}\le t u^{2/(p-2)}$. We first quote a result from a book by Ben-Tal and Nemirovski~\cite{ben2001lectures} together with a detailed construction. % that was not provided in the book.
\begin{lemma}[{\cite[Example~3.11]{ben2001lectures}}]\label{lma:P-SOC} For any $k\in\BN$, the set
$$
%\BK^{2^k+1}:=
\left\{\bx \in \R^{2^k}_+,\,y\in\R_+ : %\bx\ge{\bf 0}, 
y \le\left(\prod_{i=1}^{2^k}x_i\right)^{2^{-k}}\right\}
% \BK^{2^k+1}:=\left\{\bx \in \R^{2^k}_+,y\in\R_+ : %\bx\ge{\bf 0}, 
% y \le\sqrt[2^k]{\prod_{i=1}^{2^k}x_i}\right\}
$$
is SOC representable. In particular, when $k\ge2$, by defining additional variables $\bz^i\in\R^{2^i}_+$ for $i=1,2,\dots,k-1$, this set is the projection to the first $2^k+1$ entries of the following set defined by
% $$
% \begin{cases}
%     \bx_i\ge{\bf 0} ,\, \left\|\begin{pmatrix}
%     x_{i+1,j} \\ \frac{1}{2}(x_{i,2j-1}-x_{i,2j})
% \end{pmatrix}\right\|_2\le\frac{1}{2}(x_{i,2j-1}+x_{i,2j})\quad i=0,1,\dots,k-1,\, j=1,2,\dots 2^{k-i}\\
% y\le x_{k,1}.
% \end{cases}
% $$
\begin{equation} \label{eq:SOCA1P}
\begin{cases}
\left\|\left(
    y, \frac{1}{2}(z^1_1-z^1_2)
\right)\right\|_2\le\frac{1}{2}(z^1_1+z^1_2)\\
\left\|\left(
    z^{i}_{j}, \frac{1}{2}(z^{i+1}_{2j-1}-z^{i+1}_{2j})
\right)\right\|_2\le\frac{1}{2}(z^{i+1}_{2j-1}+z^{i+1}_{2j})& i=1,2,\dots,k-2,\, j=1,2,\dots 2^{i}\\
\left\|\left(
    z^{k-1}_{j}, \frac{1}{2}(x_{2j-1}-x_{2j})
\right)\right\|_2\le\frac{1}{2}(x_{2j-1}+x_{2j})& j=1,2,\dots, 2^{k-1}\\
\bx\ge{\bf 0},\,y\ge0,\,\bz^i\ge{\bf 0} & i=1,2,\dots, k-1.
\end{cases}
\end{equation}
\end{lemma}

We now provide an SOC representation of $\{(u,v,t)\in\R_+^3:v^{p/(p-2)}\le t u^{2/(p-2)}\}$.
\begin{proposition}\label{prop:SOC-A-1P}
For any constant $p\in\BQ\cap(2,\infty)$, denote $1/p=a/b$ with $a,b\in\BN$ being mutually prime and let $k=\lceil\log_2 (b+1)\rceil$. It holds that
\begin{align}
\BK^3_p:=&\{(u,v,t)\in\R_+^3:v^{p/(p-2)}\le t u^{2/(p-2)}\}  \nonumber \\
=&\{(u,v,t)\in\R^3: (u,v,t,\bz^1,\bz^2,\dots,\bz^{k-1}) \text{ satisfies~\eqref{eq:SOC-A-1P}},\,\bz^i\in\R^{2^i} \text{ for } i=1,2,\dots,k-1 \}, \label{eq:J3p}
\end{align}
where%~\eqref{eq:SOC-A-1P} is formulated as
\begin{equation} \label{eq:SOC-A-1P}
\begin{cases}
\left\|\left(
    v, \frac{1}{2}(z^1_1-z^1_2)
\right)\right\|_2\le\frac{1}{2}(z^1_1+z^1_2)\\
\left\|\left(
    z^{i}_{j}, \frac{1}{2}(z^{i+1}_{2j-1}-z^{i+1}_{2j})
\right)\right\|_2\le\frac{1}{2}(z^{i+1}_{2j-1}+z^{i+1}_{2j})& i=1,2,\dots,k-2,\, j=1,2,\dots 2^{i}\\
z^{k-1}_{j}\le u& j=1,2,\dots, a\\
z^{k-1}_{j}\le t& j=a+1,a+2,\dots, \lfloor\frac{b}{2}\rfloor\\
\left\|\left(
    z^{k-1}_{j}, \frac{1}{2}(t-v)
\right)\right\|_2\le\frac{1}{2}(t+v)& j=\lfloor\frac{b}{2}\rfloor+1,\lfloor\frac{b}{2}\rfloor+2,\dots,\lceil\frac{b}{2}\rceil\\
z^{k-1}_{j}\le v& j=\lceil\frac{b}{2}\rceil+1,\lceil\frac{b}{2}\rceil+2,\dots, 2^{k-1}\\
\bz^i\ge{\bf 0} & i=1,2,\dots, k-1\\
u\ge 0,\,v\ge0,\,t\ge0.
\end{cases}
\end{equation}
\end{proposition}
\begin{proof}
Consider a linear mapping $f_{p}:\R^3\rightarrow\R^{2^k+1}$ where
\begin{align*}
f_p(u,v,t) =(\underbrace{ u, u,\dots, u}_{2a}, \underbrace{t,t, \dots, t}_{b-2a}, \underbrace{v,  v, \dots,  v}_{2^k-b}, v).
\end{align*}
Noticing that $2a>0$, $b-2a>0$ as $p\in(2,\infty)$ and $2^k-b>0$, it can be calculated that
%$$\BK^3_p=\{(u,v,t)\in\R_+^3:v^{p/(p-2)}\le t u^{2/(p-2)}\}=\{(u,v,t)\in\R_+^3:v\le (u^{2a}t^{b-2a}v^{2^k-b})^{2^{-k}}\}.$$
$$ v^{p/(p-2)}\le t u^{2/(p-2)}  \Longleftrightarrow v\le (u^{2a}t^{b-2a}v^{2^k-b})^{2^{-k}}  \text{ when } u,v,t\ge0.$$

If $(u,v,t)\in\BK^3_p$, then $f_p(u,v,t)\in \{\bx \in \R^{2^k}_+,\, y\in\R_+ : %\bx\ge{\bf 0}, 
y \le(\prod_{i=1}^{2^k}x_i)^{2^{-k}}\}$. It is straightforward to verify that~\eqref{eq:SOCA1P} becomes exactly~\eqref{eq:SOC-A-1P} for $f_p(u,v,t)$. By applying Lemma~\ref{lma:P-SOC}, there must exist $\bz^i\in\R^{2^i}_+$ for $i=1,2,\dots,k-1$ such that $(u,v,t,\bz^1,\bz^2,\dots,\bz^{k-1}) \text{ satisfies~\eqref{eq:SOC-A-1P}}$.

On the other hand, if $(u,v,t)$ belongs to the RHS of~\eqref{eq:J3p}, then $(u,v,t,\bz^1,\bz^2,\dots,\bz^{k-1})$ satisfies~\eqref{eq:SOC-A-1P}. Noticing that $\|(w, \frac{1}{2}(x-y))\|_2\le\frac{1}{2}(x+y)$ is equivalent to $w\le\sqrt{xy}$ when $w,x,y\ge0$, it is easy to verify recursively from~\eqref{eq:SOC-A-1P} that
$$
v\le \left(\prod_{j=1}^2z^1_j\right)^{2^{-1}} \le \left(\prod_{j=1}^4z^2_j\right)^{2^{-2}} \le \dots \le \left(\prod_{j=1}^{2^{k-1}}z^{k-1}_{j}\right)^{2^{-(k-1)}}\le (u^{2a}t^{b-2a}v^{2^k-b})^{2^{-k}}.
$$
Therefore, $v^{p/(p-2)}\le t u^{2/(p-2)}$, implying that $(u,v,t)\in\BK^3_p$.
\end{proof}
%
% On one hand, one gas $\{(v,u,t)\in\R_+^3:v^{p/(p-2)}\le t u^{2/(p-2)}\}$ is the preimage of $\BK^{2^\eta+1}$ under the linear mapping since
% \begin{align*}
% m_{p}^{-1}(\BK^{2^\eta+1})=\left\{\left( v, u,t\right)\in\R^3:m_{p}\left( v, u,t\right)\in\BK^{2^\eta+1}\right\}&=\left\{\left( v, u,t\right)\in\R_+^3: v\le( u^{2\nu} t^{\mu-2\nu}  v^{2^\eta -\mu})^{1/2^\eta}\right\}
% \\&=\left\{(v,u,t)\in\R_+^3:v^{p/(p-2)}\le t u^{2/(p-2)}\right\}.
% % \\&=\left\{\left( v, u,t\right)\in\R_+\times\R_{++}\times\R_+:\frac{ v^{p/(p-2)}}{ u^{2/(p-2)}}\le t\right\}\bigcup\left(\{0\}\times\{0\}\times\R_+\right)=\operatorname{epi} g_p.
% \end{align*}
%
% On the other hand, it is easy to see that for affine maps $f_i:\R^{n+k}\rightarrow\R^{s_i}$ and $g_i:\R^{n+k}\rightarrow\R$, $i\in[n]$, if the system of SOC inequalities $\{\|f_i(\by,\bu)\|_2\le g_i(\by,\bu)\}_{i=1}^n$ with design variables $\by\in\R^n$ and additional variables $\bu\in\R^k$ is an SOC representation for a set $\mathbb{O}^n\subseteq\R^{n}$, then $\{\|f_i(h(\bx),\bu)\|_2\le g_i(h(\bx),\bu)\}_{i=1}^n$ with design variables $\bx\in\R^m$ and additional variables $\bu\in\R^k$ is also an SOC representation for $h^{-1}(\mathbb{O}^n)=\{\bx\in\R^m:h(\bx)\in\mathbb{O}^n\}$ for any affine map $h:\R^m\rightarrow\R^n$. Therefore, by the previous fact and Lemma~\ref{lma:P-SOC}, it is straightforward (but with some tedious case discussions) to show that the following system of SOC inequalities in the design variables $( v, u,t)\in\R^3$ and additional variables $\left(x_{j,i}:j\in [\eta],\,i\in[2^{\eta-j}]\right)\in\R^{2^\eta -1}$ is an SOC representation for $m_{p}^{-1}(\BK^{2^\eta+1})$:
% \begin{equation}
% \label{eq:SOC-A-1P}
% \begin{cases}
%     x_{1,i}\ge 0,\,
% %     \left\|\begin{pmatrix}
% %     x_{1,i} \\ 0
% % \end{pmatrix}\right\|_2\le u
% |x_{1,i}|\le u, & \text{for $i=1,2,\dots,\nu$},\\
%
% x_{1,i}\ge 0,\,
% % \left\|\begin{pmatrix}
% %     x_{1,i} \\ 0
% % \end{pmatrix}\right\|_2\le t
% |x_{1,i}|\le t, & \text{for $i=\nu+1,\dots,\lfloor{\mu}/{2}\rfloor$},\\
%
% x_{1,i}\ge 0,\, \left\|\begin{pmatrix}
%     x_{1,i} \\ \frac{t- v}{2}
% \end{pmatrix}\right\|_2\le \frac{t+ v}{2}, & \text{for $i=
% % 2^{\eta-1}-\lceil(2^\eta -\mu)/2\rceil+1
% \lfloor{\mu}/{2}\rfloor+1,\dots,\lceil \mu/2\rceil$},\\

% x_{1,i}\ge 0,\,
% % \left\|\begin{pmatrix}
% %     x_{1,i} \\ 0
% % \end{pmatrix}\right\|_2\le  v
% |x_{1,i}|\le v, & \text{for $i=
% % 2^{\eta-1}-\lfloor(2^\eta -\mu)/2\rfloor+1
% \lceil \mu/2\rceil+1,\dots,2^{\eta-1}$},\\
% x_{j,i}\ge 0,\, \left\|\begin{pmatrix}
%     x_{j,i} \\ \frac{x_{j-1,2i-1}-x_{j-1,2i}}{2}
% \end{pmatrix}\right\|_2\le\frac{x_{j-1,2i-1}+x_{j-1,2i}}{2}, & \text{for $j=2,\dots,\eta$ and $i\in[2^{\eta-j}]$},\\
%  v\le x_{\eta,1},
% \end{cases}
% \end{equation}
%
% \begin{equation}\label{eq:SOC-A-1P}
% \begin{aligned}
% &\left(\bigcap_{i\in[n]}\left\{
% x_{1,i}\ge 0,\, \left\|\begin{pmatrix}
%     x_{1,i} \\ 0
% \end{pmatrix}\right\|_2\le\mu\right\}\right)\bigcap\left(\bigcap_{i=n+1}^{\lfloor{m}/{2}\rfloor}\left\{
% x_{1,i}\ge 0,\, \left\|\begin{pmatrix}
%     x_{1,i} \\ 0
% \end{pmatrix}\right\|_2\le t\right\}\right)\bigcap
% \\&\left(\bigcap_{i=2^{\eta-1}-\lceil(2^\eta -m)/2\rceil+1}^{\lceil m/2\rceil}\left\{
% x_{1,i}\ge 0,\, \left\|\begin{pmatrix}
%     x_{1,i} \\ \frac{t-\lambda}{2}
% \end{pmatrix}\right\|_2\le \frac{t+\lambda}{2}\right\}\right)\bigcap
% \\&\left(\bigcap_{i=2^{\eta-1}-\lfloor(2^\eta -m)/2\rfloor+1}^{2^{\eta-1}}\left\{
% x_{1,i}\ge 0,\, \left\|\begin{pmatrix}
%     x_{1,i} \\ 0
% \end{pmatrix}\right\|_2\le \lambda\right\}\right)\bigcap
% \\&\left(\bigcap_{j=2}^\eta\bigcap_{i\in[2^{\eta-j}]}\left\{x_{j,i}\ge 0,\, \left\|\begin{pmatrix}
%     x_{j,i} \\ \frac{x_{j-1,2i-1}-x_{j-1,2i}}{2}
% \end{pmatrix}\right\|_2\le\frac{x_{j-1,2i-1}+x_{j-1,2i}}{2}\right\}\right)\bigcap\left\{
% \lambda\le x_{\eta,1}\right\},
% \end{aligned}
% \end{equation}
%
% By Lemma~\ref{lma:P-SOC}, we have that for any $\eta\in\BN$, the set $\BK^{2^\eta+1}$ is SOC-representable.
% (and we denote the SOC representation of it as $\mathcal{R}_\BK^{2^\eta+1}$).
% Therefore, due to the linearity of $\mathcal{A}_{p}$, we have that $\operatorname{epi} g_p=\BK^3_p$ is SOC-representable as well, as desired.
% Let $\eta$ be the smallest number s.t. $2^\eta\ge p$. Consider the affine mapping
% \begin{align*}
% \mathcal{A}:\R^3&\rightarrow\R^{2^\eta +1}\\
% \left(\lambda,\mu,t\right) &\mapsto(\mu,\mu, \underbrace{t,t, \dots, t}_{p-2\text{ times}}, \underbrace{\lambda, \lambda, \dots, \lambda}_{2^{\eta}-p\text{ times}},\lambda).
% \end{align*}
% \end{proof}

% The above scheme can only deal with integer values of $p$. How to deal with rational values? Let $p=m/n$ where $m,n\in\BN$ and $m\ge 2n$, and let $\ell$ be the smallest that $2^\ell\ge m$. Then, the epigraph of $\frac{\lambda^{p/(p-2)}}{\mu^{2/(p-2)}}=\frac{\lambda^{m/(m-2n)}}{\mu^{2n/(m-2n)}}$ is the preimage of $\BK^{2^\ell+1}$ under the affine mapping
% \begin{align*}
% \mathcal{A}_{p}:\R^3&\rightarrow\R^{2^\ell +1}\\
% \left(\lambda,\mu,t\right) &\mapsto(\underbrace{\mu,\mu,\dots,\mu}_{2n\text{ times}}, \underbrace{t,t, \dots, t}_{m-2n\text{ times}}, \underbrace{\lambda, \lambda, \dots, \lambda}_{2^{\ell}-m\text{ times}},\lambda).
% \end{align*}


% Explicitly write out the SOC representation: Denote the matrix associating with $\mathcal{A}$ as $A$. Denote the SOC representation of $\BK^{2^\ell+1}$ as $\BK^{2^\ell+1}$.
% (x_1,x_2, \dots, x_{2^\ell}, t)

% In the sequel, we denote
% \begin{equation}\label{eq:J3p}
% \BK^3_p:=\left\{( v, u,t)\in\R^3:\exists \left(x_{j,i}\right)_{j\in [\eta],\,i\in[2^{\eta-j}]}\in\R^{2^\eta -1} \text{ s.t. } ( v, u,t)\vee\left(x_{j,i}\right)_{j\in [\eta],\,i\in[2^{\eta-j}]} \text{ satisfies~(\ref{eq:SOC-A-1P})} \right\},
% \end{equation}
% i.e., $\BK^3_p$ is the canonical projection of the solution set of (\ref{eq:SOC-A-1P}) onto its first three coordinates. 

With the SOC representation of $\BK^3_p$ in Proposition~\ref{prop:SOC-A-1P} in an explicit way, we provide semidefinite optimization models for both $\|A\|_{p_v}$ and $\|A\|_{p_u}$.
\begin{corollary}\label{lma:dual-SDP}
For any matrix $A\in\R^{m\times n}$ and constant $p\in\BQ\cap(2,\infty)$,
    \begin{equation}\label{cor:final-spectral-p-norm}
    \begin{array}{lll}
    \|A\|_{p_v}= &\min & u_1+ u_2+\theta_p\sum_{i=1}^{m+n} t_i    \\
    &\st   & (v_i, u_1,t_i)\in\BK^3_p\quad i=1,2,\dots, m \\
    && (v_i, u_2,t_i)\in\BK^3_p\quad i=m+1,m+2,\dots, m+n\\
    && %  u_1\ge0, \,u_2\ge 0, \,   \bt\ge {\bf 0},\, 
    \Diag(\bv)\succeq \begin{pmatrix} O & A/2 \\A^\T/2 & O\end{pmatrix}
    \end{array}
    \end{equation}
and
\begin{equation}\label{cor:final-nuclear-p-norm}
    \begin{array}{lll}
    \|A\|_{p_u}= &\max & \langle A,Z\rangle    \\
    &\st   &u_1+ u_2+\theta_p\sum_{i=1}^{m+n} t_i\le 1\\
    && (v_i, u_1,t_i)\in\BK^3_p\quad i=1,2,\dots, m \\
    && (v_i, u_2,t_i)\in\BK^3_p\quad i=m+1,m+2,\dots, m+n\\ 
    && %  u_1\ge0, \,u_2\ge 0, \,   \bt\ge {\bf 0},\, 
    \Diag(\bv)\succeq \begin{pmatrix} O & Z/2 \\Z^\T/2 & O\end{pmatrix},
    \end{array}
    \end{equation}
where $\BK^3_p$ is defined in~\eqref{eq:J3p}.
\end{corollary}

As the given constant $\frac{1}{p}=\frac{a}{b}$ has $a,b\in\BN$ mutually prime and $2a<b$, the number of variables and the number of constraints describing $\BK^3_p$ in~\eqref{eq:J3p} are $O(b)$ and $O(b\ln b)$, respectively. The number of variables in~\eqref{cor:final-nuclear-p-norm} is $O(bm+bn+mn)$ while there are $O((m+n)b\ln b)$ linear or simple SOC constraints together with one positive semidefinite constraint. Anyway,~\eqref{cor:final-nuclear-p-norm} can be solved to arbitrary accuracy in polynomial time using, e.g., an interior-point method. In fact,~\eqref{cor:final-spectral-p-norm} provides a much better formulation of $\|\cdot\|_{p_v}$ than~\eqref{opt:vecp-original}. By combining Lemma~\ref{lma:vecp}, Lemma~\ref{prop:matrix-equi-pq-vecp} and Corollary~\ref{lma:dual-SDP}, we conclude this section with the following important result in matrix theory.
\begin{theorem}\label{thm:KG-matrix-nuclear-pnorm}
For any matrix $A\in\R^{m\times n}$ and constant $p \in\BQ\cap(2, \infty)$, $\|A\|_{p_u}$ can be obtained by the optimal value of~\eqref{cor:final-nuclear-p-norm} such that
$$
\|A\|_{p_u}\le \|A\|_{p_*}\le {\delta_G}\|A\|_{p_u}.
$$
Moreover, any optimal solution $Z\in\R^{m\times n}$ of~\eqref{cor:final-nuclear-p-norm} is an approximate dual certificate of $\|A\|_{p_*}$ in the sense that
$$
\|Z\|_{p_\sigma}\le 1 \text{ and } \langle A,Z\rangle\ge \|A\|_{p_*}/\delta_G.
$$
\end{theorem}
\begin{proof}
The first statement is an immediate consequence of Lemma~\ref{prop:matrix-equi-pq-vecp} and the second part of Corollary~\ref{lma:dual-SDP}. For the second statement, an optimal solution of~\eqref{cor:final-nuclear-p-norm} exists since the primal problem~\eqref{opt:vecp-original} is bounded and strictly feasible, mentioned at the end of the proof of Proposition~\ref{lma:dual-vecp}. According to~\eqref{cor:final-spectral-p-norm}, any optimal $Z$ of~\eqref{cor:final-nuclear-p-norm} satisfies $\|Z\|_{p_v}\le1$. Therefore, $\|Z\|_{p_\sigma}\le\|Z\|_{p_v}\le1$ by Lemma~\ref{lma:vecp} and $ \langle A,Z\rangle=\|A\|_{p_u}\ge \|A\|_{p_*}/\delta_G$.
\end{proof}
%     \begin{equation*}
%     \begin{aligned}
%     & \max
%     & & \langle A,Z\rangle \\
%     & \st
%     & &  u_1+ u_2+\theta_p\sum_{i=1}^{n_1+n_2}t_i\le 1, \\ \span &&  u_1\ge 0,\, u_2\ge 0,\\
%     \span &&\Diag(\bv)\succeq\frac{1}{2}\begin{pmatrix}
%     O & Z \\
%     Z^\T & O
%     \end{pmatrix},
%     % \\ \span && (v_i,u_1,t_i)\in\BK^3_p \text{ for all }  i\in[n_1],
%     % \\ \span && ( v_j, u_2,t_j)\in\BK^3_p \text{ for all }  j\in[n_1+n_2]\setminus [n_1].
% %%%%%%%%%%%%%%%%%%%%%%%%%%%%%%%%%%%%%%%%%%%%%%%%%%%%%%
%     \\ \span && x^{i_1}_{1,k}\ge 0,\, \left\|\begin{pmatrix}
%     x^{i_1}_{1,k} \\ 0
% \end{pmatrix}\right\|_2\le u_1,\, \text{for $k\in[n]$},\, i_1\in[n_1],
% \\ \span && x^{i_1}_{1,k}\ge 0,\, \left\|\begin{pmatrix}
%     x^{i_1}_{1,k} \\ 0
% \end{pmatrix}\right\|_2\le t_{i_1},\, \text{for $k=n+1,\dots,\lfloor{m}/{2}\rfloor$},\, i_1\in[n_1],
% \\ \span && x^{i_1}_{1,k}\ge 0,\, \left\|\begin{pmatrix}
%     x^{i_1}_{1,k} \\ \frac{t_{i_1}-v_{i_1}}{2}
% \end{pmatrix}\right\|_2\le \frac{t_{i_1}+v_{i_1}}{2},\, \text{for $k=2^{\eta-1}-\lceil(2^\eta -m)/2\rceil+1,\dots,\lceil m/2\rceil$},\, i_1\in[n_1],
% \\ \span && x^{i_1}_{1,k}\ge 0,\, \left\|\begin{pmatrix}
%     x^{i_1}_{1,k} \\ 0
% \end{pmatrix}\right\|_2\le v_{i_1},\, \text{for $k=2^{\eta-1}-\lfloor(2^\eta -m)/2\rfloor+1,\dots,2^{\eta-1}$},\, i_1\in[n_1],
% \\ \span && x^{i_1}_{j,k}\ge 0,\, \left\|\begin{pmatrix}
%     x^{i_1}_{j,k} \\ \frac{x^{i_1}_{j-1,2k-1}-x^{i_1}_{j-1,2k}}{2}
% \end{pmatrix}\right\|_2\le\frac{x^{i_1}_{j-1,2k-1}+x^{i_1}_{j-1,2k}}{2},\, \text{for $j=2,\dots,\eta$ and $k\in[2^{\eta-j}]$},\, i_1\in[n_1],
% \\ \span && v_{i_1}\le x^{i_1}_{\eta,1},\, i_1\in[n_1],
% %%%%%%%%%%%%%%%%%%%%%%%%%%%%%%%%%%%%%%%%%%%%%%%%%%%%%%
% \\ \span && x^{i_2}_{1,k}\ge 0,\, \left\|\begin{pmatrix}
%     x^{i_2}_{1,k} \\ 0
% \end{pmatrix}\right\|_2\le u_2,\, \text{for $k\in[n]$},\, i_2\in[n_1+n_2]\setminus [n_1],
% \\ \span && x^{i_2}_{1,k}\ge 0,\, \left\|\begin{pmatrix}
%     x^{i_2}_{1,k} \\ 0
% \end{pmatrix}\right\|_2\le t_{i_2},\, \text{for $k=n+1,\dots,\lfloor{m}/{2}\rfloor$},\, i_2\in[n_1+n_2]\setminus [n_1],
% \\ \span && x^{i_2}_{1,k}\ge 0,\, \left\|\begin{pmatrix}
%     x^{i_2}_{1,k} \\ \frac{t_{i_2}-v_{i_2}}{2}
% \end{pmatrix}\right\|_2\le \frac{t_{i_2}+v_{i_2}}{2},\, \text{for $k=2^{\eta-1}-\lceil(2^\eta -m)/2\rceil+1,\dots,\lceil m/2\rceil$},\, i_2\in[n_1+n_2]\setminus [n_1],
% \\ \span && x^{i_2}_{1,k}\ge 0,\, \left\|\begin{pmatrix}
%     x^{i_2}_{1,k} \\ 0
% \end{pmatrix}\right\|_2\le v_{i_2},\, \text{for $k=2^{\eta-1}-\lfloor(2^\eta -m)/2\rfloor+1,\dots,2^{\eta-1}$},\, i_2\in[n_1+n_2]\setminus [n_1],
% \\ \span && x^{i_2}_{j,k}\ge 0,\, \left\|\begin{pmatrix}
%     x^{i_2}_{j,k} \\ \frac{x^{i_2}_{j-1,2k-1}-x^{i_2}_{j-1,2k}}{2}
% \end{pmatrix}\right\|_2\le\frac{x^{i_2}_{j-1,2k-1}+x^{i_2}_{j-1,2k}}{2},\, \text{for $j=2,\dots,\eta$ and $k\in[2^{\eta-j}]$},\, i_2\in[n_1+n_2]\setminus [n_1],
% \\ \span && v_{i_2}\le x^{i_2}_{\eta,1},\, i_2\in[n_1+n_2]\setminus [n_1].
%     \end{aligned}
% \end{equation*}


\section{$\ell_p$-sphere covering}\label{sec:hitting-sets}

This section is devoted to explicit constructions of hitting sets to approximately cover $\BS^n_p$ in $\R^n$. In particular, we focus on hitting sets that are aimed to maximize the hitting ratio while keeping the cardinality bounded by a polynomial function of $n$, called polynomial cardinality. This is important to the approximation schemes of the tensor nuclear $p$-norm to be run in polynomial time in Section~\ref{sec:algorithms}. Some other hitting sets constructed as byproducts have independent interest.

When $p=2$, the Euclidean sphere covering is well studied in computational geometry since the pioneering work of Rogers~\cite{R58}; see~\cite{he2023approx} and references therein. In particular, several hitting sets with the hitting ratio $\Omega(\sqrt{\ln{n}/{n}})$ were explicitly constructed in~\cite{he2023approx}. However, little was known for $\ell_p$-sphere. This section can be taken as the $\ell_p$-generalization of the work in~\cite{he2023approx}. Unlike the Euclidean sphere which is self-dual, $\ell_p$-sphere makes the study more difficult and the covering results are slightly worse than that of the Euclidean sphere. We are left with an open problem to explicitly construct a deterministic hitting set with hitting ratio $\Omega(\sqrt{\ln{n}/{n}})$ while its existence can be shown; see Section~\ref{sec:rand-hitting-set}. Most results developed in this section apply for $p\in[2,\infty)$ and some apply for $p\in(1,2)$ as well. For ease of reference, we summarize the specifications of our main constructions below in Table~\ref{tab:hitting-sets}.

\begin{table}[!ht]
\centering
\caption{Hitting sets of $\BS_p^n$ with polynomial cardinality %{, where $\frac{1}{p} + \frac{1}{q} =1$} % (d.u.b. means deterministic upper bound)
}
\label{tab:hitting-sets}
%\resizebox{\textwidth}{!}{
\begin{tabular}{|lc|ccc|c|}
\hline
Hitting sets  & $p$ & Hitting ratio  & Cardinality   & Type          & Reference                             \\ \hline
$\BH_{1}^n(\alpha,\beta)$  & $(1,\infty)$                                      & $\mu_{\alpha,\beta}\sqrt[\leftroot{-2}\uproot{2}q]{\frac{\ln{n}}{n+\ln{n}}}$                                       & $n^{\ln{\nu_{\alpha,\beta}}}\left(\frac{\nu_{\alpha,\beta}n}{\ln n}+1\right)$                               & Deterministic & Corollary~\ref{cor:lnn1q}       \\
$\BH_{2}^n(\alpha,\beta)$ & $[2,\infty)$& $\frac{\mu_{\alpha,\beta}\sqrt[\leftroot{-2}\uproot{2}p]{\ln n}}{\sqrt{2n}}$ & $\frac{\nu_{\alpha,\beta}n^{2\ln{\nu_{\alpha,\beta}}+1}}{\ln{n}}$                                   & Deterministic & Corollary~\ref{thm:h2} \\
$\BH_{3}^n(\epsilon)$          & $[2,\infty)$                                          & $\sqrt{\frac{\delta_0 \ln{n}}{2n}}$ & $\left\lceil\delta_3n^{\delta_2}\left( \left(\frac{1}{2}+\frac{1}{q}\right)n \ln n+\ln \frac{1}{\epsilon}\right)\right\rceil$ & Randomized & Theorem~\ref{thm:rand-hitting-set}    \\ \hline
% D.u.b.                              & $\Omega(\sqrt[q]{\ln n/n})$                                             & $\Omega(\sqrt{\ln{n}/{n}})$                                                                      & $O\left(\operatorname{poly}(n)\right)$                                                                                           & Deterministic           & Section~\ref{sec:rand-hitting-set}    \\ \hline
\end{tabular}
%}
\end{table}

\subsection{$\Omega(\sqrt[q]{\ln n/n})$-hitting sets of $\BS_p^n$}\label{sec:worst-hitting} 

We start with the best hitting set of $\BS_2^n$ recently proposed by He et al.~\cite{he2023approx}. In particular, our first construction generalizes the $\Omega(\sqrt{\ln n/n})$-hitting set~\cite[Theorem 2.14]{he2023approx} of $\BS_2^n$ to $\Omega(\sqrt[q]{\ln n/n})$-hitting sets of $\BS_p^n$ when $p\in(1,\infty)$. They are the only hitting sets covering the range of $p\in (1,2)$ in the paper.

An important preparation in the construction is an $\Omega(1)$-hitting set of $\BS_p^n$ although its cardinality has to be exponential in $n$. To this end, we adapt the construction in~\cite[Algorithm 2.1]{he2023approx} for $\BS^n_2$ to a one for $\BS_p^n$, i.e., the hitting set $\BH_{H}^n(\alpha, \beta)$ shown in Algorithm~\ref{alg:alg1}.

\begin{algorithm}[!h]\caption{$\BH_{H}^n(\alpha, \beta)$: $\Omega(1)$-hitting sets of $\BS_p^n$ with $p\in(1,\infty)$}
\begin{algorithmic}[1]
\REQUIRE A dimension $n\in\BN$, a constant $p\in(1,\infty)$ and two parameters $\alpha\ge 1$ and $\beta\ge\alpha+1$.
\ENSURE An $\Omega(1)$-hitting set of $\BS_p^n$.
\STATE Let $m=\left\lceil\log_\beta \alpha n \right\rceil$ and partition $\BI:=\{1,2,\dots,n\}$ into %disjoint subsets 
$\{\BI_1,\BI_2,\dots,\BI_m\}$ such that
\begin{equation}\label{eq:partition}
\BI=\bigcup_{j=1}^m\BI_j,\,\BI_i\bigcap\BI_j=\emptyset \text{ for } i\neq j,\,|\BI_1|=n-\sum_{j=2}^m|\BI_j| \text{ and } |\BI_j|=\left\lfloor \frac{\alpha n}{\beta^{j-1}} \right\rfloor
\text{ for } j=2,3\dots, m.
\end{equation}
\STATE Generate a set of vectors
$$\BX^n=\bigcup_{\{\BI_1,\BI_2,\dots,\BI_m\}\text{ satisfies~\eqref{eq:partition}}}\left\{\bz\in\R^n: z_i\in\left\{\pm1,\pm \beta^{\frac{j-1}{p}}\right\} \text{ if } i\in\BI_j \text{ for } j=1,2,\dots,m \right\}.$$
% \STATE Let $m:=\left\lceil\log _\beta \alpha n\right\rceil$ and generate a set of partitions of $\BI:=[n]$ as
% $$
% \mathbb{L}:=\left\{\left\{\BI_1, \BI_2, \dots, \BI_m\right\}\subseteq\BI:\left|\BI_1\right|=n-\sum_{k=2}^m\left|\BI_k\right|,\,\left|\BI_k\right|=\left\lfloor\frac{\alpha n}{\beta^{k-1}}\right\rfloor \text { for } k=2,3, \dots, m,\,\bigcup_{i\ne j}\left(\BI_i\cap\BI_j\right)=\varnothing\right\};
% $$
% \STATE Generate a set of vectors
% $$
% \BX^n:=\bigcup_{\left\{\BI_1, \BI_2, \dots, \BI_m\right\}\in\mathbb{L}}\left\{\bz \in \R^n: z_i \in\left\{\pm 1, \pm\beta^{\frac{k-1}{p}}\right\} \text { if } i \in \BI_k \text { for } k\in[m]\right\};
% $$
\RETURN $\BH_{H}^n(\alpha, \beta):=\{\bz/\|\bz\|_p\in\BS_p^n: \bz \in \BX^n\}$.
\end{algorithmic}\label{alg:alg1}
\end{algorithm}

Algorithm~\ref{alg:alg1} definitely generates a nonempty hitting set since there exists some $\{\BI_1,\BI_2,\dots,\BI_m\}$ satisfying~\eqref{eq:partition}. In particular, 
$$
\sum_{j=2}^m |\BI_j| \le \sum_{j=2}^m \frac{\alpha n}{\beta^{j-1}} =\frac{\alpha n}{\beta-1}-\frac{\alpha n}{\beta^{m-1}(\beta-1)}\le\frac{\alpha n}{\beta-1}\le n,
$$
implying that $|\BI_1|=n-\sum_{j=2}^m |\BI_j|\ge0$. We now show that $\BH_{H}^n(\alpha, \beta)$ is an $\Omega(1)$-hitting set of $\BS_p^n$.
\begin{proposition}\label{thm:HE}
It follows that
$$
\BH_{H}^n(\alpha, \beta)\in\allowbreak\BT_p^n\left(\mu_{\alpha,\beta},{\nu_{\alpha,\beta}}^n\right),
$$
where
$$\mu_{\alpha,\beta}:=\sqrt[\leftroot{-2}\uproot{2}p]{\frac{\alpha}{\beta(\alpha+1)}}\left(1-\frac{1}{\alpha}\right) \text{ and } \nu_{\alpha,\beta}:=2^{\frac{\beta+\alpha-1}{\beta-1}} \alpha^{-\frac{\alpha}{\beta-1}}\beta^{\frac{\alpha\beta}{(\beta-1)^2}}\left(\frac{\beta-1}{\beta-\alpha-1}\right)^{\frac{\beta-\alpha-1}{\beta-1}}+o(1).$$
% $$
% \BH_{H}^n(\alpha, \beta)\in\allowbreak\BT_p^n\left(\underbrace{\allowbreak\sqrt[\leftroot{-2}\uproot{2}p]{\frac{\alpha}{\beta(\alpha+1)}}\left(1-\frac{1}{\alpha}\right)}_{=:\mu_{\alpha,\beta}},\underbrace{\left(2^{\frac{\beta+\alpha-1}{\beta-1}} \alpha^{-\frac{\alpha}{\beta-1}}\beta^{\frac{\alpha\beta}{(\beta-1)^2}}\left(\frac{\beta-1}{\beta-\alpha-1}\right)^{\frac{\beta-\alpha-1}{\beta-1}}+o(1)\right)^n}_{=:\left(\nu_{\alpha,\beta}\right)^n}\right).
% $$
% $$
% \BH_{H}^n(\alpha,\gamma+1)\in\BT_p\left(n, \frac{\alpha-1}{\sqrt{\alpha(\alpha+1)(\gamma+1)}},\left(2^{\frac{\gamma+\alpha}{\gamma}} \alpha^{-\frac{\alpha}{\gamma}}(\gamma+1)^{\frac{\alpha(\gamma+1)}{\gamma^2}}\left(\frac{\gamma}{\gamma-\alpha}\right)^{\frac{\gamma-\alpha}{\gamma}}+o(1)\right)^n\right).
% $$
\end{proposition}
\begin{proof}
    The upper bound ${\nu_{\alpha,\beta}}^n$ of the cardinality was proved exactly in~\cite[Proposition~2.9]{he2023approx}. We only estimate the lower bound, $\mu_{\alpha,\beta}$, for the hitting ratio.
    
    For any given $\bx\in\BS_q^n$, denote the index sets
    $$
    \BD_0(\bx)=\left\{i \in \BI:\left|x_i\right| \le \frac{1}{\sqrt[\leftroot{-2}\uproot{2}q]{\alpha n}}\right\}
    \text { and }
    \BD_j(\bx)=\left\{i \in \BI: \sqrt[\leftroot{-2}\uproot{2}q]{\frac{\beta^{j-1}}{\alpha n}}<\left|x_i\right| \le \sqrt[\leftroot{-2}\uproot{2}q]{\frac{\beta^j}{\alpha n}}\right\} \text{ for } j=1,2,\dots,m.
    $$
    Notice that $|x_i|\le 1\le \sqrt[q]{\beta^m/(\alpha n)}$ and so $\{\BD_0(\bx),\BD_1(\bx),\dots,\BD_m(\bx)\}$ is a partition of $\BI$. Besides, it is obvious that for $j\ge2$,
$$
\frac{\beta^{j-1}}{\alpha n}  |\BD_j(\bx)|= \sum_{i \in \BD_j(\bx)}\frac{\beta^{j-1}}{\alpha n} < \sum_{i \in \BD_j(\bx)}|x_i|^q \le 1.
$$
This implies that $|\BD_j(\bx)| < \alpha n/\beta^{j-1}$, i.e., $|\BD_j(\bx)| \le\lfloor\alpha n/\beta^{j-1}\rfloor$ for $j=2,3, \dots, m$. Hence, there exists a partition $\{\BI_1,\BI_2,\dots,\BI_m\}$ of $\BI$ satisfying~\eqref{eq:partition} such that $\BD_j(\bx)\subseteq \BI_j$ for $j=2,3, \dots, m$. Furthermore, we may find a vector $\bz\in\BX^n$ such that
    $$
    z_i= \begin{cases}\sign(x_i) & i \in \BD_0(\bx)\cup \BD_1(\bx) \\ \sign(x_i) \beta^{\frac{j-1}{p}} & i \in \BD_j(\bx) \text { for } j=2,3,\dots, m,\end{cases}
    $$
    where the $\sign$ function takes $1$ for nonnegative reals and $-1$ for negative reals.
    
    We can now estimate $\bz^{\T}\bx$ and $\|\bz\|_p$. First of all, we have
    $$
    \sum_{i \in \BD_0(x)} |x_i|^q \le \sum_{i \in \BD_0(x)} \frac{1}{\alpha n} \le \frac{1}{\alpha} \text{ and }\sum_{j=1}^m \sum_{i \in \BD_j(x)} |x_i|^q = \sum_{i\in\BI} |x_i|^q -  \sum_{i \in \BD_0(\bx)} |x_i|^q \ge 1-\frac{1}{\alpha}.
    $$
    Next, we have
    $$
    \sum_{i \in \BD_0(\bx)} |z_i|^p=\left|\BD_0(\bx)\right| \le n \text{ and }\sum_{j=1}^m \sum_{i \in \BD_j(\bx)} |z_i|^p=\sum_{k=1}^m \sum_{i \in \BD_j(\bx)} \beta^{j-1}<\sum_{k=1}^m \sum_{i \in \BD_j(\bx)} \alpha n |x_i|^q \le \alpha n,
    $$
    implying that $\|\bz\|_p^p \le(\alpha+1) n$ by summing up the two inequalities. Lastly, noticing that $\sign(z_i)=\sign(x_i)$ for every $i$, we have
    $$
    \bz^{\T} \bx \ge \sum_{j=1}^m \sum_{i \in \BD_j(\bx)} \beta^{\frac{j-1}{p}}\left|x_i\right| \ge \sum_{j=1}^m \sum_{i \in \BD_j(\bx)} \sqrt[\leftroot{-2}\uproot{2}p]{\frac{\alpha n}{\beta}}\left|x_i\right|^q \ge \sqrt[\leftroot{-2}\uproot{2}p]{\frac{\alpha n}{\beta}}\left(1-\frac{1}{\alpha}\right),
    $$
    where the second inequality follows from $\left|x_i\right| \le \sqrt[q]{\beta^j/(\alpha n)}$ in defining $\BD_j(\bx)$. Therefore, we conclude that the vector $\bz/\|\bz\|_p\in\BH_{H}^n(\alpha, \beta)$ satisfies
    $$
    \bx^{\T} \frac{\bz}{\|\bz\|_p} \ge \sqrt[\leftroot{-2}\uproot{2}p]{\frac{\alpha n}{\beta}}\left(1-\frac{1}{\alpha}\right)\cdot \frac{1}{\sqrt[\leftroot{-2}\uproot{2}p]{(\alpha+1)n}} = 
    \sqrt[\leftroot{-2}\uproot{2}p]{\frac{\alpha}{\beta(\alpha+1)}}\left(1-\frac{1}{\alpha}\right),
    $$
    proving the desired lower bound $\mu_{\alpha,\beta}$.
\end{proof}

% It is remarked that, compared with the construction by~\cite{brieden2001deterministic} (i.e., $\BC^n(\gamma)$ returned by Algorithm~\ref{alg:lp-fptas}), our construction has two advantages:
% \begin{itemize}
%     \item Our construction has a smaller cardinality bound. As can be easily seen from Proposition~\ref{thm:HE}, $|\BH_{H}^n|$ can at best be bounded by $(2+\epsilon)^n$ for any $\epsilon>0$ (achieved when $\beta$ is large enough), but even when $\BC^n(\gamma)$ has a hitting ratio of $0$, its cardinality bound is still $\left(3\max\left\{1,\sqrt{\frac{2\pi}{p}}\right\}e^{p/12}\right)^n$.
%     \item Our construction has a $p$-free cardinality bound. Due to the term `$e^{p/12}$' in the cardinality bound of $\BC^n(\gamma)$, the bound tends to $\infty$ as  $p\rightarrow\infty$ even for fixed $n$. On the contrary, the cardinality bound of our $\BH_{H}^n(\alpha, \beta)$ is totally $p$-free.
% \end{itemize}

We remark that Proposition~\ref{thm:HE} actually improves an existing $\Omega(1)$-hitting set due to Brieden et al.~\cite{brieden2001deterministic}, at least from numerical examples to be shown later. In particular, by letting $
\BY^n(\gamma)=\{\bz\in\BZ^n: 0<\|\bz\|_p\le \gamma n^{1/p}\}$ with a parameter $\gamma>1$, the hitting set 
$$\BH_B^n(\gamma):=\left\{\frac{\bz}{\|\bz\|_p}\in\BS_p^n: \bz \in \BY^n(\gamma)\right\}$$
that was implicitly given in the proof of~\cite[Lemma~3.7]{brieden2001deterministic} satisfies that
\begin{equation}\label{eq:bupper}
\BH_B^n(\gamma)\in\BT_p^n\left(1-\frac{1}{\gamma}, \left(e^{\frac{p}{12}}(2\gamma+1)\max\left\{1,\sqrt{\frac{2\pi}{p}}\right\}\right)^n\right).
\end{equation}

% \begin{algorithm}[!h]
% \caption{$\BH_{B}^n(\gamma)$: $\Omega(1)$-hitting sets of $\BS_p^n$ with $p\in(1,\infty)$}
% \begin{algorithmic}[1]
% \REQUIRE A dimension $n\in\BN$, a constant $p\in(1,\infty)$ and a parameter $\gamma>1$.
% \ENSURE An $\Omega(1)$-hitting set of $\BS_p^n$.
% \STATE Generate a set of vectors
% % $$
% % \BY^n:=\mathbb{Z}^n \bigcap\left(\gamma n^{1 / p} \mathbb{B}_{p}^n \setminus\{\bd{0}_n\}\right);
% % $$
% $$
% \BY^n:=\left\{\bz\in\BZ^n: 0<\|\bz\|_p\le \gamma n^{\frac{1}{p}}\right\}.
% $$
% \RETURN $\BH_B^n(\gamma):=\{\bz/\|\bz\|_p\in\BS_p^n: \bz \in \BY^n\}$.
% \end{algorithmic}
% \label{alg:lp-fptas}
% \end{algorithm}
% \begin{lemma}[{\cite[Lemma~3.7]{brieden2001deterministic}}]\label{lma:lp-grid}
% It holds that, the output of Algorithm~\ref{alg:lp-fptas}
% $$
% \BC^n(\gamma)\in\BT_p^n\left(1-\frac{1}{\gamma}, \left((2\gamma+1)\max\left\{1,\sqrt{\frac{2\pi}{p}}\right\}e^{p/12}\right)^n\right).
% $$
% \end{lemma}

$\BH_B^n(\gamma)$ is essentially based on grid sampling in an $\ell_p$-ball while $\BH_{H}^n(\alpha,\beta)$ provides a more refined selection that has an obvious advantage in terms of the cardinality if both attain the same hitting ratio. As an example, let us choose $p=6$ which is the global minimizer of the above upper~\eqref{eq:bupper} for $|\BH_B^n(\gamma)|$ and let $\beta=\alpha+1$ for $\BH_{H}^n(\alpha,\beta)$ which maximizes $\mu_{\alpha,\beta}$ for fixed $\alpha$. By setting the hitting ratio $1-\frac{1}{\gamma}=(1-\frac{1}{\alpha})\sqrt[6]{\frac{\alpha}{(\alpha+1)^2}}$, we have the following upper bounds of cardinalities
$$
|\BH_B^n(\gamma)|^{\frac{1}{n}} \le %e^{\frac{1}{6}}
\sqrt{\frac{e \pi}{3}}\left(\frac{2}{1-\left(1-\frac{1}{\alpha}\right)\sqrt[\leftroot{-2}\uproot{2}6]{\frac{\alpha}{(\alpha+1)^2}}}+1\right)
\text{ and }
|\BH_{H}^n(\alpha,\alpha+1)|^{\frac{1}{n}} \le \frac{4 (\alpha+1)^{\frac{\alpha+1}{\alpha}}}{\alpha}.
$$
\begin{wrapfigure}[16]{r}{0.42\linewidth}
    \centering
    % Figure removed
    \caption{Comparison of upper bounds of $|\BH_B^n(\gamma)|$ and $|\BH_{H}^n(\alpha,\alpha+1)|$ when $p=6$.}
    \label{fig:bvh}
\end{wrapfigure}
The comparison of the two upper bounds with respect to the same hitting ratio is shown in Figure~\ref{fig:bvh}.
% With the above preparations at hand, we are now at a place to introduce the detailed experimental procedure. Specifically, we plot the curves regarding the change of the above two bases in relation to the common hitting ratio $\sqrt[\leftroot{-2}\uproot{2}6]{\frac{\alpha}{(\alpha+1)^2}}\left(1-\frac{1}{\alpha}\right)$ by sampling $\alpha\in[8.583,1,000]$ with $1,000$ sampled points of equal step size, where $8.583$ is chosen due to the fact that it is the (numerical) maximizer of $\sqrt[\leftroot{-2}\uproot{2}6]{\frac{\alpha}{(\alpha+1)^2}}\left(1-\frac{1}{\alpha}\right)$, which yields the largest-possible hitting ratio for $\BH_{H}^n(\alpha,\beta)$.
It verifies a clear lower cardinality for $\BH_{H}^n$ compared to that for $\BH_B^n$. Moreover, $|\BH_{H}^n(\alpha,\beta)|$ is invariant to $p$ from Proposition~\ref{thm:HE} while $|\BH_B^n(\gamma)|$ attains its minimum when $p=6$. If $p$ deviates more from $6$, the gap between the two cardinalities may be even larger. On the other hand, $\BH_B^n(\gamma)$ is able to obtain a close-to-one hitting ratio that $\BH_{H}^n(\alpha,\beta)$ cannot achieve, but with a price of large cardinality.

Before proceeding to construct $\Omega(\sqrt[q]{\ln n/n})$-hitting sets of $\BS_p^n$, we need to establish two basic properties of hitting sets of $\BS_p^n$. They are straightforwardly generalized from~\cite[Lemma~2.12]{he2023approx} and~\cite[Lemma~2.13]{he2023approx} for hitting sets of $\BS_2^n$, the Euclidean sphere. We omit the proofs as they are simple and similar to that for $\BS_2^n$.

\begin{lemma}\label{lma:kronecker-hitting}
For any $p\in(1,\infty)$, if a hitting set $\BH^{n_1} \in \BT^{n_1}_p(\tau, m)$ with $\tau\ge0$, then 
$$\BE^{n_2} \boxtimes \BH^{n_1} \in \BT^{n_1 n_2}_p\left(\frac{\tau}{\sqrt[\leftroot{-2}\uproot{2}q]{n_2}}, m n_2\right).$$
\end{lemma}

\begin{lemma}\label{lma:appending-hitting}
For any $p\in(1,\infty)$, if two hitting sets $\BH^{n_1} \in \BT^{n_1}_p\left(\tau_1, m_1\right)$ and $\BH^{n_2} \in \BT^{n_2}_p\left(\tau_2, m_2\right)$ with $\tau_1,\tau_2>0$, then
$$
\left(\BH^{n_1} \vee \mathbf{0}_{n_2}\right) \bigcup\left(\mathbf{0}_{n_1} \vee \BH^{n_2}\right) \in \BT^{n_1+n_2}_p\left(\frac{\tau_1 \tau_2}{\sqrt[\leftroot{-2}\uproot{2}q]{{\tau_1}^q+{\tau_2}^q}}, m_1+m_2\right).
$$
\end{lemma}

With all the preparation ready, in particular the $\Omega(1)$-hitting sets of $\BS^n_p$, we are able to construct $\Omega(\sqrt[q]{\ln n/n})$-hitting sets of $\BS_p^n$ with polynomial cardinality. The construction is based on the Kronecker product with the standard basis of a Euclidean space.
\begin{corollary}\label{cor:lnn1q}
Given an integer $n\ge 2$, let $n_1=\lceil\ln n\rceil$, $n_2=\lfloor \frac{n}{n_1}\rfloor$ and $n_3 = n-n_1 n_2$.
For any $p\in(1,\infty)$, $\alpha\ge1$ and $\beta\ge\alpha+1$, one has %$\BH_{1}^n(\alpha,\beta):=\left(\left(\BE^{n_2} \boxtimes \BH_H^{n_1}(\alpha, \beta)\right) \vee \mathbf{0}_{n_3}\right) \cup\left(\mathbf{0}_{n_1 n_2} \vee \BH_H^{n_3}(\alpha, \beta)\right)$ satisfies
        \begin{align*}
        \BH_{1}^n(\alpha,\beta)&:=\left(\left(\BE^{n_2} \boxtimes \BH_H^{n_1}(\alpha, \beta)\right) \vee \mathbf{0}_{n_3}\right) \bigcup\left(\mathbf{0}_{n_1 n_2} \vee \BH_H^{n_3}(\alpha, \beta)\right) \\
        &\in \BT_p^n\left(\mu_{\alpha,\beta}\sqrt[\leftroot{-2}\uproot{2}q]{\frac{\ln{n}}{n+\ln{n}}}, n^{\ln{\nu_{\alpha,\beta}}}\left(\frac{\nu_{\alpha,\beta}n}{\ln n}+1\right)\right).
        % $$
        \end{align*}
\end{corollary}
% \begin{proof}
% \end{proof}

The above result is an immediate consequence of Proposition~\ref{thm:HE}, Lemma~\ref{lma:kronecker-hitting} and Lemma~\ref{lma:appending-hitting}. % The proof is similar to that of~\cite[Theorem 2.13]{he2023approx} for $\BS_2^n$ and is left to interested readers. 
We remark that the hitting ratio $\Omega(\sqrt[q]{\ln n/n})$ for $\BS_p^n$ is the largest possible by a deterministic hitting set with polynomial cardinality when $p\in(1,2]$; see~\cite[Theorem~3.2]{brieden1998approximation}.

\subsection{$\Omega(\sqrt[p]{\ln{n}}/\sqrt{n})$-hitting sets of $\BS_p^n$}\label{sec:brieden-hitting-set}

Although $\BH_{1}^n(\alpha,\beta)$ attains the best hitting ratio $\Omega(\sqrt[q]{\ln n/n})$ for a hitting set with polynomial cardinality when $p\in(1,2]$, it is not the best when $p\in(2,\infty)$. In this subsection we propose an improved one only when $p\in[2,\infty)$.

% REMARK: We need a Hadamard matrix~\cite{SY92} in $\R^{m\times m}$ with entries $1$ been replaced by $I_n$ and $-1$ been replaced by $-I_n$, leading to a matrix in $\R^{mn\times mn}$. As long as this kinds of Hadamard matrix in $\R^{m\times m}$ exists for $m$ being a generalized one that can tend to infinity, it will work for us. It was conjectured that $m=4k$ exists while it remains 12 unfounded example for $k\le 500$; see latest development on a book by~\cite{SY20}. Wallis (Seberry) proved that there exists an Hadamard matrix of order $2^tn$ for $t\ge\lfloor 2\log_2 (n-3)\rfloor +1$ where $n>3$ in an integer. For our purpose, we have a simple construction for $m=2^k$.

To begin with, we introduce Walsh-Hadamard transform~\cite{YH97} that plays a key role of improvement in the construction based on Kronecker product with the standard basis of a Euclidean space. Walsh-Hadamard transform is a class of matrices defined recursively via Kronecker products:
$$I_{1,0}:=I_1 \text{ and } I_{1,k}:=\begin{pmatrix}1&1\\ 1&-1\end{pmatrix} \boxtimes I_{1,k-1}=\begin{pmatrix}I_{1,k-1}&I_{1,k-1}\\ I_{1,k-1}&-I_{1,k-1}\end{pmatrix}\in\R^{2^k\times 2^k} \text{ for } k=1,2,\dots$$
Sometimes there is a normalization factor $2^{-k/2}$ with $I_{1,k}$ and this makes its eigenvalues to be $\pm1$. Without the normalization, the entries of these matrices are either $1$ or $-1$. For our purpose, we extend Walsh-Hadamard transform by replacing $1$ with $I_m$ and $-1$ with $-I_m$, i.e.,
\begin{equation}\label{eq:hmatrix}
  I_{m,0}:=I_m \text{ and } I_{m,k}:=\begin{pmatrix}1&1\\ 1&-1\end{pmatrix}\boxtimes I_{m,k-1}=\begin{pmatrix}I_{m,k-1}&I_{m,k-1}\\ I_{m,k-1}&-I_{m,k-1}\end{pmatrix}\in\R^{2^km\times 2^km} \text{ for } k=1,2,\dots  
\end{equation}
$I_{m,k}$ is symmetric and nonsingular with its inverse given by $2^{-k} I_{m,k}$. Its eigenvalues are $\pm2^{-k/2}$. In fact, upon scaling by the normalization factor $2^{-k/2}$, it actually becomes orthogonal. In particular, the following properties of $I_{m,k}$ are essential to the construction of an improved hitting set.
\begin{lemma}\label{lma:parallelogram}
    For any integer $k\ge0$ and $\bx\in\R^{2^km}$, one has $\|I_{m,k}\bx\|_2=2^{k/2}\|\bx\|_2$. If further $\bx=\be_i\boxtimes \by$ where $\be_i\in\R^{2^k}$ and $\by\in\R^{m}$, then $\|I_{m,k}\bx\|_p=2^{k/p}\|\bx\|_p$.
\end{lemma}
\begin{proof}
    We prove the first statement by induction. The base case $k=0$ is obvious. Assuming that the statement holds for $k$, we have for any $\bx=\bx_1\vee\bx_2\in\R^{2^{k+1}m}$ with $\bx_1,\bx_2\in\R^{2^km}$ that
    \begin{align*}
    {\|I_{m,k+1}\bx\|_2}^2&={\left\|\begin{pmatrix}I_{m,k} & I_{m,k} \\ I_{m,k} & -I_{m,k}\end{pmatrix}\binom{\bx_1}{\bx_2}\right\|_2}^2\\
    &={\|I_{m,k}\bx_1+I_{m,k}\bx_2\|_2}^2+{\|I_{m,k}\bx_1-I_{m,k}\bx_2\|_2}^2    \\
    &=2\left({\|I_{m,k}\bx_1\|_2}^2+{\|I_{m,k}\bx_2\|_2}^2\right)\\
    &=2\left(2^{k} {\|\bx_1\|_2}^2+2^{k} {\|\bx_2\|_2}^2\right)\\
    &=2^{k+1}{\|\bx\|_2}^2,
    \end{align*}
    where the second-to-last equality is due to the induction assumption.

    For the second statement, we observe from~\eqref{eq:hmatrix} that the matrix $I_{m,k}$ can be partitioned into $2^k\times 2^k$ block matrices with each block being either $I_m$ or $-I_m$ since the entries of Walsh-Hadamard transform are $\pm1$. We also observe that the vector $\be_i\boxtimes \by$ can be partitioned into $2^k$ block vectors in $\R^m$ with the $i$th block being $\by$ and others being zero vectors. Therefore, the vector $I_{m,k}(\be_i\boxtimes \by)\in\R^{2^km}$ can be partitioned into $2^k$ block vectors in $\R^m$ with each block being either $\by$ or $-\by$, implying that
    $$
    \|I_{m,k}\bx\|_p=\|I_{m,k}(\be_i\boxtimes \by)\|_p=\left({2^k{\|\by\|_p}^p}\right)^{1/p}=2^{k/p}\|\by\|_p=2^{k/p}\|\bx\|_p.
    $$   
\end{proof}

Denote $J=2^{-k/p}I_{m,k}$. If $\bx=\be_i\boxtimes \by$, Lemma~\ref{lma:parallelogram} tells us that $\|J\bx\|_p=\|\bx\|_p$ but $\|J\bx\|_2=2^{k/2-k/p}\|\bx\|_2$ is actually enlarged when $p\in(2,\infty)$. Moreover, the property is independent on $m$. This is how the improvement can be made by applying $J$ from the construction $\BH_{1}^n(\alpha,\beta)$ when $p\in(2,\infty)$.

\begin{lemma}\label{lma:2km}
For any $p\in[2,\infty)$, if a hitting set $\BH^{m} \in \BT_p^m(\tau, m_0)$, then 
$$2^{-\frac{k}{p}}I_{m,k}\big(\BE^{2^k} \boxtimes \BH^{m}\big) \in \BT_p^{2^km}\big(2^{-\frac{k}{2}} m^{\frac{1}{2}-\frac{1}{q}}\tau,2^k m_0\big).$$
\end{lemma}
\begin{proof}
  Given a vector $2^{-k/p}I_{m,k}(\be_i\boxtimes\by)$ in $2^{-k/p}I_{m,k}(\BE^{2^k} \boxtimes \BH^{m})$, $\|2^{-k/p}I_{m,k}(\be_i\boxtimes \by)\|_p=1$ has been verified by the second statement of Lemma~\ref{lma:parallelogram}. The upper bound of the cardinality, $2^km_0$, is also obvious. It suffices to show the lower bound of the hitting ratio.
    
    % We first show that $2^{-k/p}I_{m,k}\left(\BE^{2^k} \boxtimes \BH^{n}\right)\subseteq\BS_p^{2^k n}$. Let $\by\in 2^{-\frac{k}{p}}I_{m,k}\left(\BE^{2^k} \boxtimes \BH^{n}\right)$. Then, $\by=2^{-\frac{k}{p}}I_{m,k}\bz$ for some $\bz\in\BE^{2^k} \boxtimes \BH^{n}$. Without loss of generality, we may let $\bz=\bw\vee\bd{0}_{(2^k-1)n}$, where $\bw\in\BH^n$. Then, it is computed that $\by=2^{-\frac{k}{p}}\bw^{\vee 2^k}$, and thus $\|\by\|_p=2^{-\frac{k}{p}}2^{k/p}\|\bw\|_p=1$, as desired. Other cases can be handled in a nearly identical way.
    % Besides, by noticing possible overlaps, we have that
    % $
    % \left|2^{-\frac{k}{p}}I_{m,k}\left(\BE^{2^k} \boxtimes \BH^{n}\right)\right|\le\left|\BE^{2^k} \boxtimes \BH^{n}\right|\le 2^k m.
    % $

    For any $\bx\in\BS_q^{2^km}$ with $q\in(1,2]$, denote $2^{-k/p}I_{m,k}\bx=\bz_1\vee\bz_2\vee\dots\vee\bz_{2^k}$ where $\bz_i\in\R^m$ for $i=1,2,\dots,2^k$. By the first statement of Lemma~\ref{lma:parallelogram} and the bounds between $\ell_p$-norms (Lemma~\ref{lma:lp-norm-equiv}), we have
    $$
    2^{\frac{k}{2}}\max_{1\le i\le 2^k}\|\bz_i\|_2 \ge \left(\sum_{i=1}^{2^k}\|\bz_i\|_2^2\right)^{1/2}
    = \|2^{-\frac{k}{p}}I_{m,k}\bx\|_2 = 2^{\frac{k}{2}-\frac{k}{p}}\|\bx\|_2\ge 2^{\frac{k}{2}-\frac{k}{p}} (2^km)^{\frac{1}{2}-\frac{1}{q}} \|\bx\|_q = m^{\frac{1}{2}-\frac{1}{q}}.
    $$   
    There exists some $j$ such that $\|\bz_j\|_2\ge 2^{-k/2}m^{1/2-1/q}$.
    %$\|\bz_j\|_q^2\ge2^{-k}(m^{\frac{1}{2}-\frac{1}{q}})^2$, i.e., $\|\bz_j\|_q\ge2^{-\frac{k}{2}}m^{\frac{1}{2}-\frac{1}{q}}$.
    Moreover, there also exists a vector $\bw\in\BH^m$ such that $\bw^{\T}(\bz_j/\|\bz_j\|_q)\ge\tau$. As a result, we find a vector $2^{-k/p}I_{m,k}(\be_j\boxtimes \bw)\in 2^{-k/p}I_{m,k}(\BE^{2^k} \boxtimes \BH^{m})$ such that
    $$
    \bx^{\T}2^{-\frac{k}{p}}I_{m,k}(\be_j\boxtimes \bw)=\big(2^{-\frac{k}{p}}I_{m,k}\bx\big)^{\T}(\be_j\boxtimes \bw)=\bz_j^{\T}\bw\ge \tau \|\bz_j\|_q\ge  \tau \|\bz_j\|_2\ge 2^{-\frac{k}{2}}m^{\frac{1}{2}-\frac{1}{q}} \tau,
    $$
    where the first equality holds because $I_{m,k}$ is symmetric.
    %
    % let us partition $\bx$ to be $\bigvee_{i=1}^{2^k}\bx_i$ where each $\bx_i\in\R^n$ as well as $I_{m,k}\bx$ to be $\bigvee_{i=1}^{2^k}\by_i$ where each $\by_i\in\R^n$. Then we have that
    % $$
    % \begin{aligned}
    %     \sum_{i=1}^{2^k}\|\by_i\|_q^2\ge\sum_{i=1}^{2^k}\|\by_i\|_2^2=2^k \sum_{i=1}^{2^k}\|\bx_i\|_2^2\ge 2^k n^{1-2/q}\sum_{i=1}^{2^k}\|\bx_i\|_q^2 \ge 2^k 2^{k(1-2/q)} n^{1-2/q}\|\bx\|_q^2=2^{k(2-2/q)} n^{1-2/q},
    % \end{aligned}
    % $$
    % where the first equality follows from Lemma~\ref{lma:parallelogram} and the second inequality follows from Lemma~\ref{lma:lp-norm-equiv}.
    %     $$
    % \big\langle \bx,2^{-\frac{k}{p}}I_{m,k}\big(\be_{i}\boxtimes\bh\big)\big\rangle=2^{-\frac{k}{p}}\big\langle \by_{i_*},\bh\big\rangle\ge 2^{-\frac{k}{p}}\|\by_{i_*}\|_q \tau\ge 2^{-\frac{k}{p}} 2^{k(1/2-1/q)} n^{1/2-1/q} \tau=\frac{ \tau}{2^{k/2}n^{1/q-1/2}},
    % $$
    %     Therefore, one has (recall the meaning of $\by_i$)
    % $$
    % \min_{\bx\in\BS_q^{2^k n}}\max_{i\in[2^k]}\|\by_i\|_q\ge 2^{k(1/2-1/q)} n^{1/2-1/q}.
    % $$
    % As a result, for any $\bx\in\BS_q^{2^k n}$, there is some $\bh\in\BH^n$, s.t. for $i_*=\arg\max_{i\in[2^k]}\|\by_i\|_q$.
\end{proof}

If we merely adopt $\BE^{2^k} \boxtimes \BH^{m}$ without applying $2^{-k/p}I_{m,k}$, then the hitting ratio is $2^{-k/q}\tau$ by Lemma~\ref{lma:kronecker-hitting}. Therefore, applying $2^{-k/p}I_{m,k}$ improves the ratio with a factor of $2^{k/q-k/2}m^{1/2-1/q}=(2^k/m)^{1/q-1/2}$, which is indeed an improvement if $2^k>m$ for $q\in(1,2)$.

Suppose now we have $2^k\approx n/ \ln n$ and $m\approx\ln n$. Applying Lemma~\ref{lma:2km} with an $\Omega(1)$-hitting set $\BH^m$, the hitting ratio of $2^{-k/p}I_{m,k}(\BE^{2^k} \boxtimes \BH^{m})$ is then
$2^{-k/2} m^{1/2-1/q}\Omega(1)
=\Omega(\sqrt[p]{\ln{n}}/\sqrt{n})$, a clear improvement from the previous $\Omega(\sqrt[q]{\ln n/n})$-hitting sets in Section~\ref{sec:worst-hitting} when $p\in(2,\infty)$. Therefore, we only need to handle a bit of details on the integerity as $2^k$ can never be $n/\ln n$. Appending a short vector such as Lemma~\ref{lma:appending-hitting} is not doable in this case as we may need to append a vector of dimension $2^k\approx n/\ln n$. Instead, we can apply cutting from a longer vector, essentially the following monotonicity.
\begin{lemma}\label{lma:cut}
For any $p\in(1,\infty)$ and $n_1,n_2\in\BN$ with $n_1\le n_2$, if a hitting set $\BH^{n_2} \in \BT^{n_2}_p\left(\tau, m\right)$ with $\tau>0$, then
$$
\BH^{n_1}(\BH^{n_2}):=\left\{\frac{\bz}{\|\bz\|_p}:\bz\in\R^{n_1},\,\bz\neq {\bf 0},\,\bz\vee\by\in\BH^{n_2}\text{ for some }\by\in\R^{n_2-n_1}\right\}\in \BT^{n_1}_p\left(\tau, m\right).
$$
\end{lemma}
\begin{proof}
    For any $\bx\in\BS_q^{n_1}$, it is obvious that $\bx\vee{\bf 0}_{n_2-n_1}\in\BS_q^{n_2}$. There is a vector $\bz\vee\by\in\BH^{n_2}$ with $\bz\in\R^{n_1}$ and $\by\in\R^{n_2-n_1}$ such that $(\bx\vee{\bf 0})^{\T} (\bz\vee\by) \ge \tau$, i.e., $\bx^{\T}\bz \ge \tau$. By that $\bz\neq {\bf 0}$ as $\tau>0$, we find a vector $\bz/\|\bz\|_p\in\BH^{n_1}(\BH^{n_2})$ such that 
    $$
    \bx^{\T}\frac{\bz}{\|\bz\|_p}\ge\frac{\tau}{\|\bz\|_p}\ge\frac{\tau}{\|\bz\vee\by\|_p} = \tau.
    $$
    On the other hand, it is easy to see that $|\BH^{n_1}(\BH^{n_2})|\le|\BH^{n_2}|\le m$.
\end{proof}

We conclude this subsection by proposing the following hitting set with an improved hitting ratio when $p\in(2,\infty)$ as an immediate consequence of Lemma~\ref{lma:2km}, Proposition~\ref{thm:HE} and Lemma~\ref{lma:cut}.
\begin{corollary}\label{thm:h2}
Given an integer $n\ge 2$, let $k=\lfloor\log_2\frac{n}{\ln n}\rfloor$ and $m=\lceil2^{-k}n\rceil$.
For any $p\in[2,\infty)$, $\alpha\ge1$ and $\beta\ge\alpha+1$, one has %$\BH_{1}^n(\alpha,\beta):=\left(\left(\BE^{n_2} \boxtimes \BH_H^{n_1}(\alpha, \beta)\right) \vee \mathbf{0}_{n_3}\right) \cup\left(\mathbf{0}_{n_1 n_2} \vee \BH_H^{n_3}(\alpha, \beta)\right)$ satisfies
        \begin{align*}
        \BH_{2}^n(\alpha,\beta)&:=\BH^{n}\left(2^{-\frac{k}{p}}I_{m,k}\big(\BE^{2^k}\boxtimes \BH_{H}^m(\alpha, \beta) \big)\right)
\in \BT_p^n\left(\frac{\mu_{\alpha,\beta}\sqrt[\leftroot{-2}\uproot{2}p]{\ln n}}{\sqrt{2n}}, \frac{\nu_{\alpha,\beta}n^{2\ln{\nu_{\alpha,\beta}}+1}}{\ln{n}}\right).
        \end{align*}
\end{corollary}
\begin{proof}
   We first observe the following bounds
   $$\frac{n}{2\ln n}< 2^k\le \frac{n}{\ln n},\,\ln n\le 2^{-k}n\le m<2^{-k}n+1 \text{ and } n\le 2^km< n+2^k\le 2n.$$
      By Lemma~\ref{lma:2km} and Proposition~\ref{thm:HE}, $2^{-k/p}I_{m,k}(\BE^{2^k}\boxtimes \BH_{H}^m(\alpha, \beta))$ is a hitting set of $\BS^{2^km}_p$ with the hitting ratio 
    $$
    2^{-\frac{k}{2}} m^{\frac{1}{2}-\frac{1}{q}} \mu_{\alpha,\beta}=(2^km)^{-\frac{1}{2}}m^{\frac{1}{p}}\mu_{\alpha,\beta}
    \ge\frac{\mu_{\alpha,\beta}\sqrt[\leftroot{-2}\uproot{2}p]{\ln n}}{\sqrt{2n}}
    $$
    and the cardinality no more than 
    $$2^k{\nu_{\alpha,\beta}}^m \le \frac{n}{\ln{n}}{\nu_{\alpha,\beta}}^{2^{-k}n+1}\le\frac{n}{\ln{n}}{\nu_{\alpha,\beta}}^{2\ln{n}+1}=\frac{\nu_{\alpha,\beta}n^{2\ln{\nu_{\alpha,\beta}}+1}}{\ln{n}},$$
    where the last inequality follows from $\frac{n}{2\ln n}<2^k$. The conclusion then follows by Lemma~\ref{lma:cut} since $n\le 2^km$.
\end{proof}

% \begin{proposition}\label{prop:divides}
%     For any $p\in(2,\infty)$, given a number of dimension $n\ge 2$, let $m:=\lceil \ln{n}\rceil$. Assume $m$ divides $n$. Then, by letting $s:=\lfloor \log_2 \left(\frac{n}{m}+1\right)\rfloor$, $\frac{n}{m}$ can be written as $\sum_{k=0}^{s} b_k 2^k$ where $ b_k\in\{0,1\}$ and thus $n$ can be written as $\sum_{k=0}^{s} b_k 2^k m$, and we further have that
%     \begin{equation*}
%     % \begin{aligned}
%     \BH_{\heartsuit}^n(\alpha,\beta):=\bigcup_{k\in\left\{i\in\{0\}\cup[s]: b_i=1\right\}}\Bigg(\bd{0}_{\sum_{i=0}^{k-1} b_i 2^i m} \bigvee \left(2^{-\frac{k}{p}}I_{2^k, m}\left(\BE^{2^k} \boxtimes \BH_H^{m}(\alpha,\beta)\right)\right)\bigvee \bd{0}_{\sum_{i=k+1}^{s} b_i 2^i m }\Bigg),
%     % \end{aligned}
%     \end{equation*}
%     satisfies
%     $$
%     \BH_{\heartsuit}^n(\alpha,\beta)\in\BT_p^n\left(\frac{\mu_{\alpha,\beta}}{\left(\frac{2^{q/2}}{2^{q/2}-1}\right)^{\frac{1}{q}}}\frac{(\ln{n})^{1/p}}{(n+\ln{n}+1)^{1/2}},\nu_{\alpha,\beta} n^{\ln{\nu_{\alpha,\beta}}}\frac{n}{\ln{n}}\right).
%     $$
% \end{proposition}

% We remark that, in the binary factorization $n=\sum_{k=0}^{s} b_k 2^k m$, the undesired case where $2^k<m$ happens only if $k\in O\left(\ln{\ln{n}}\right)$ since $m=\lceil \ln{n}\rceil$, which is small enough to be negligible compared to $s=\lfloor \log_2 \left(\frac{n}{m}+1\right)\rfloor$ which is of order $\Omega\left(\ln{\left(\frac{n}{\ln{n}}\right)}\right)$. This simple observation explains how the enhancement is led by the above proposition.

% \begin{proof}
%     To begin with, the binary representability of $\frac{n}{m}$ can be easily verified by definition of $s$.

%     Besides, we have $\BH_{2}^n(\alpha,\beta)\subseteq\BS_p^n$ since $2^{-k/p}I_{2^k, m}\left(\BE^{2^k} \boxtimes \BH_H^{m}(\alpha,\beta)\right)\subseteq\BS_{p}^{2^k m}$ (cf. Lemma~\ref{lma:2km}).

%     Moreover, by noticing possible overlaps, we have that
%     \begin{align*}
%     |\BH_{2}^n(\alpha,\beta)|&=\sum_{k=0}^{s}  b_k 2^k \left(\nu_{\alpha,\beta}\right)^m\le \left(\nu_{\alpha,\beta}\right)^m \frac{n}{m}
%     \le \left(\nu_{\alpha,\beta}\right)^{\ln{n}+1}\frac{n}{\ln{n}}=\nu_{\alpha,\beta} n^{\ln{\nu_{\alpha,\beta}}}\frac{n}{\ln{n}}.
%     \end{align*}

%     We next claim that for any $\by\in\BS_q^n$ partitioned by $\bigvee_{k\in\left\{i\in\{0\}\cup[s]: b_i=1\right\}}\by_k$ where $\by_k\in\R^{2^k m}$, we have
%     $$
%     \min_{\by\in\BS_q^n}\max_{k\in\left\{i\in\{0\}\cup[s]: b_i=1\right\}}\|\by_k\|_q\ge 2^{k/2}\left(\frac{2^{q/2}-1}{2^{q/2}}\right)^{\frac{1}{q}}\sqrt{\frac{m}{n+m}}.
%     $$
%     If this is not the case, then we have that
%     $$
%     \begin{aligned}
%         \|\by\|_q^q&=\sum_{k=0}^{s}\|\by_k\|_q^q<\sum_{k=0}^{s}\left(2^{k/2}\left(\frac{2^{q/2}-1}{2^{q/2}}\right)^{\frac{1}{q}}\sqrt{\frac{m}{n+m}}\right)^q=\frac{2^{q/2}-1}{2^{q/2}}\left(\frac{m}{n+m}\right)^{q/2}\sum_{k=0}^{s}\left(2^{q/2}\right)^k\\&=\frac{2^{q/2}-1}{2^{q/2}}\left(\frac{m}{n+m}\right)^{q/2}\frac{\left(2^{s+1}\right)^{q/2}-1}{2^{q/2}-1}\le \frac{2^{q/2}-1}{2^{q/2}}\left(\frac{m}{n+m}\right)^{q/2}\frac{2^{q/2}}{2^{q/2}-1}\left(2^{s}\right)^{q/2}\le 1,
%     \end{aligned}
%     $$
%     a contradiction. The desired result thus follows by a similar argument as in the proof of Lemma~\ref{lma:2km}.
% \end{proof}
% 
% Combining Proposition~\ref{prop:divides} with Lemma~\ref{lma:appending-hitting} yields the main result in this subsection as follows.
% \begin{corollary}
%     For any $p\in(2,\infty)$, given a number of dimension $n\ge 2$, let $m:=\lceil \ln{n}\rceil$, $h:=\lfloor\frac{n}{m}\rfloor$, $r:=n-hm\in\{0,1,\dots,m-1\}$, and $s:=\lfloor \log_2 \left(\frac{n}{m}+1\right)\rfloor$. Then, $h$ can be written as $\sum_{k=0}^{s} b_k 2^k$ where $ b_k\in\{0,1\}$ and thus $n$ can be written as $\sum_{k=0}^{s} b_k 2^k m + r$, and we further have that
%     \begin{equation*}
%     \begin{aligned}
%     \BH_{2}^n(\alpha,\beta):=\left(\BH_{\heartsuit}^{hm}(\alpha,\beta)\bigvee \bd{0}_{r}\right)\bigcup\left(\bd{0}_{hm}\bigvee\BH_H^{r}(\alpha,\beta)\right),
%     \end{aligned}
%     \end{equation*}
%     satisfies
%     $$
%     \BH_{2}^n(\alpha,\beta)\in\BT_p^n\left(\frac{\mu_{\alpha,\beta}}{\left(\frac{2^{1+q/2}}{2^{q/2}-1}\right)^{\frac{1}{q}}}\frac{(\ln{n})^{1/p}}{(n+\ln{n}+1)^{1/2}},\nu_{\alpha,\beta} n^{\ln{\nu_{\alpha,\beta}}}\left(\frac{n}{\ln{n}}+1\right)\right).
%     $$
% \end{corollary}

% \subsection{$O\left(\operatorname{poly}(n)\right)$-sized $\Omega\left(\frac{\sqrt[\leftroot{-2}\uproot{2}p]{\ln{n}}}{\sqrt{n}}\right)$-hitting sets of $\BS_p^n$.}\label{sec:brieden-hitting-set}
% Although the above $\Omega(\sqrt[q]{\ln n/n})$-hitting sets are useful,
% % already have the capability to improve the best-known polynomial-time approximation ratio for the tensor nuclear $p$-norm,
% they are still not capable to be applied to attain the best-known one for its dual norm, and we will thus make a further improvement in this subsection to bridge this gap by designing $O\left(\operatorname{poly}(n)\right)$-sized $\Omega\left(\frac{\sqrt[\leftroot{-2}\uproot{2}p]{\ln{n}}}{\sqrt{n}}\right)$-hitting sets of $\BS_p^n$, which suffice. Before we present the construction, we first discuss an important connection between $\ell_p$-sphere covering and polytope approximation to its dual ball as a necessary preparation.
% % , which will be helpful and useful when we subsequently construct hitting sets.

% Every hitting set, say $\BH^n\in\BT_p^n(\tau,m)$, naturally induces an outer polytope approximation $\bigcap_{\bu\in\BH^n}\allowbreak\left\{\bx\in\R^n:\bu^\T\bx\le 1\right\}$ to the dual unit ball $\mathbb{B}_q^n$, since by polarity
% % and the envelope representation of closed convex sets~\cite[Theorem~6.20]{rockafellar2009variational}
% we have that
% \begin{equation}\label{eq:outer-polytope}
% \mathbb{B}_q^n=\left(\mathbb{B}_p^n\right)^\circ=\bigcap_{\bu\in\mathbb{B}_p^n}\allowbreak\left\{\bx\in\R^n:\bu^\T\bx\le 1\right\}=\bigcap_{\bu\in\BS_p^n}\left\{\bx\in\R^n:\bu^\T\bx\le 1\right\}\subseteq\bigcap_{\bu\in\BH^n}\allowbreak\left\{\bx\in\R^n:\bu^\T\bx\le 1\right\}.
% \end{equation}
% From this point of view, it is natural to ask, does there exist any intrinsic connection between the hitting ratio of a hitting set and the approximation ratio of the polytope approximation it induces? The following proposition explicitly reveals this connection.

% \begin{proposition}\label{lma:conn-hitapp}
% For any $p\in(1,\infty)$, let $\BH^n\subseteq\BS_p^n$ be an arbitrary set with finite cardinality, and assume $\tau>0$, $m\in\BN$. Then, the following statements are equivalent:
% \begin{itemize}
%     \item $\BH^n\in\BT_p^n(\tau,m)$, i.e., $\min_{\|\bx\|_q=1}\max_{\bu\in\BH^n}\bu^\T\bx\ge\tau$;
%     \item $\mathbb{B}_q^n\subseteq\bigcap_{\bu\in\BH^n}\left\{\bx\in\R^n:\bu^\T\bx\le 1\right\}\subseteq\frac{1}{\tau}\mathbb{B}_q^n$.
% \end{itemize}
% \end{proposition}

% It should be noted that, the first inclusion of the second statement has nothing to do with the first statement, and it has already been explained in (\ref{eq:outer-polytope}).
% % We thus omit its proof.

% \begin{proof}
% We first establish the `$\downimplies$' direction. The first statement is equivalent, in set-theoretical language, to $\BS_q^n\subseteq\bigcup_{\bu\in\BH^n}\left\{\bx\in\R^n:\bu^\T\bx\ge\tau\right\}$. Moreover, it is easy to see that the above inclusion further implies that $\left(\mathbb{B}_q^n\right)^\complement\subseteq\bigcup_{\bu\in\BH^n}\left\{\bx\in\R^n:\bu^\T\bx>\tau\right\}=\tau\bigcup_{\bu\in\BH^n}\left\{\bx\in\R^n:\bu^\T\bx> 1\right\}$, since for all $\bx\in\left(\mathbb{B}_q^n\right)^\complement$, $\|\bx\|_q>1$, which directly implies the second statement (by taking complements to both sides). On the other hand, for the `$\upimplies$' direction, by reversing the argument above, we see that $\left(\mathbb{B}_q^n\right)^\complement\subseteq\bigcup_{\bu\in\BH^n}\left\{\bx\in\R^n:\bu^\T\bx>\tau\right\}$. Consider, for any $\epsilon>0$ and $\bx\in\BS_q^n$, the vector $(1+\epsilon)\bx\in\left(\mathbb{B}_q^n\right)^\complement$. Then, by the above inclusion we have that for any $\bx\in\BS_q^n$, there exists some $\bu\in\BH^n$, s.t. $\langle(1+\epsilon)\bx,\bu\rangle>\tau$, i.e., $\bx^\T\bu>\frac{\tau}{1+\epsilon}$, for any $\epsilon>0$. By letting $\epsilon\downarrow 0$, we get $\bx^\T\bu\ge\tau$, which completes the proof.
%     % it remains to see that all the above arguments can be reversed.
% \end{proof}
% % Recall Proposition~\ref{lma:conn-hitapp}, which established the connection between sphere covering and polytope approximation.
% % Before giving the construction, we first introduce a useful lemma.
% As a result, equipped with Proposition~\ref{lma:conn-hitapp}, we can carry over existing results on polytope approximation in~\cite{brieden2001deterministic}\footnote{These results, after some equivalent reformulations and transcriptions in our notations and language, are listed in Appendix~\ref{sec:brieden-lemmas}.} to construct $O\left(\operatorname{poly}(n)\right)$-sized $\Omega\left(\frac{\sqrt[\leftroot{-2}\uproot{2}p]{\ln{n}}}{\sqrt{n}}\right)$-hitting sets of $\BS_p^n$.
% %
% % : Given a polytope which well approximates the unit $\ell_p$-ball, the set collecting all its vertices is a hitting set with a good hitting ratio.
% To begin with, we first convert~\cite[Lemma~3.12]{brieden2001deterministic} (i.e., Lemma~\ref{lma:lma3.12} in Appendix~\ref{sec:brieden-lemmas}) into a result of hitting sets as below through the lens of Proposition~\ref{lma:conn-hitapp}, which essentially states that in some specific dimensions, the hitting ratio of hitting sets in the form of $\BE^{n_2} \boxtimes \BH^{n_1}$ can be further improved through a linear transformation, provided that $n_2\ge n_1$.

% \begin{lemma}\label{lma:2km}
% Let $k\in\BN\cup\{0\}$ be a constant, and matrices $I_{m,k}\in\R^{2^k n\times 2^k n}$ be defined recursively as
% $$
% I_{m,0}=I,\text{ and }I_{m, k+1}=\begin{pmatrix}I_{m,k} & I_{m,k} \\ I_{m,k} & -I_{m,k}\end{pmatrix}.
% $$
% For any $p\in(2,\infty)$, if a hitting set $\BH^{n} \in \BT_p^n\left(\tau, m\right)$, then $2^{-k/p}I_{m,k}\left(\BE^{2^k} \boxtimes \BH^{n}\right) \in \BT^{2^k n}_p\Big(\frac{\tau}{2^{k/2} n^{1/q - 1/2}}, \allowbreak 2^k m\Big)$.
% \end{lemma}

% We remark that the matrix $I_{m,k}$ defined above enjoys many nice properties. For example, it is easy to see that $I_{m,k}$ is symmetric and invertible, with its inverse given by $2^{-k} I_{m,k}$. Besides, upon scaling, $I_{m,k}$ is also an orthogonal matrix, which represents a rotational or reflectional transformation. All the above properties can be easily verified by some inductive arguments. Moreover, it is also known that its eigenvalues only range in $\left\{-\sqrt{2^k},\sqrt{2^k}\right\}$~\cite{brieden2001deterministic}.

% \begin{proof}
%     We will first show that $2^{-k/p}I_{m,k}\left(\BE^{2^k} \boxtimes \BH^{n}\right)\subseteq\BS_p^{2^k n}$. Then, by Proposition~\ref{lma:conn-hitapp}, to show the desired statement, it remains to show that
%     $$
%     \bigcap_{\bu\in 2^{-\frac{k}{p}}I_{m,k}\left(\BE^{2^k} \boxtimes \BH^{n}\right)}\left\{\bx\in\R^{2^k n}:\bu^\T\bx\le 1\right\}\subseteq \frac{2^{k/2} n^{1/q - 1/2}}{\tau}\mathbb{B}_q^{2^k n}.
%     $$
%     Let $\by\in 2^{-\frac{k}{p}}I_{m,k}\left(\BE^{2^k} \boxtimes \BH^{n}\right)$. Then, $\by=2^{-\frac{k}{p}}I_{m,k}\bz$ for some $\bz\in\BE^{2^k} \boxtimes \BH^{n}$. Without loss of generality, we let $\bz=\bw\vee\bd{0}_{(2^k-1)n}$, where $\bw\in\BH^n$. Then, it is computed that $\by=2^{-\frac{k}{p}}\bw^{\vee 2^k}$, and thus $\|\by\|_p=2^{-\frac{k}{p}}2^{k/p}\|\bw\|_p=1$, as desired. Other cases can be handled in a nearly identical way, and we thus omit the details here to avoid the heavy symbols that will be incurred. Next, because $\by=2^{-\frac{k}{p}}I_{m,k}\bx$ if and only if $\bx=2^{-k/q}I_{m,k}\by$ due to the relation $I_{m,k}^{-1}=2^{-k} I_{m,k}$, as previously remarked, we have that
%     \begin{align*}
%     &\bigcap_{\bu\in 2^{-\frac{k}{p}}I_{m,k}\left(\BE^{2^k} \boxtimes \BH^{n}\right)}\left\{\bx\in\R^{2^k n}:\bu^\T\bx\le 1\right\}=\bigcap_{\bu\in\BE^{2^k} \boxtimes \BH^{n}}\left\{\bx\in\R^{2^k n}:\left(2^{-\frac{k}{p}}I_{m,k}\bu\right)^\T\bx\le 1\right\}\\=&\bigcap_{\bu\in\BE^{2^k} \boxtimes \BH^{n}}\left\{\bx\in\R^{2^k n}:\bu^\T\left(2^{-\frac{k}{p}}I_{m,k}\bx\right)\le 1\right\}=2^{-k/q}I_{m,k}\bigcap_{\bu\in\BE^{2^k} \boxtimes \BH^{n}}\left\{\bx\in\R^{2^k n}:\bu^\T\bx\le 1\right\},
%     \end{align*}
%     where the last equality follows from the fact that the set $\left\{\bx\in\R^{2^k n}:\bu^\T\bx\le 1\right\}$ describes a closed half-space, which is closed and convex, the invertibility of $I_{m,k}$ again, and~\cite[Theorem~5]{kushnir2020linear} \footnote{This seemingly-na\"{\i}ve equality is actually far from trivial. Essentially, it is a question about the relation between the linear transformation of the intersection of two closed convex sets and the intersection of their images under the same linear transformation. See the recent paper~\cite{kushnir2020linear} for a comprehensive treatment to this type of problems.}. Now, for any $\by\in\bigcap_{\bu\in 2^{-\frac{k}{p}}I_{m,k}\left(\BE^{2^k} \boxtimes \BH^{n}\right)}\allowbreak\left\{\bx\in\R^{2^k n}:\bu^\T\bx\le 1\right\}$, there exists some $\bz\in\bigcap_{\bu\in\left(\BE^{2^k} \boxtimes \BH^{n}\right)}\allowbreak\left\{\bx\in\R^{2^k n}:\bu^\T\bx\le 1\right\}=\left(\bigcap_{\bu\in\BH^{n}}\left\{\bx\in\R^n:\bu^\T\bx\le 1\right\}\right)^{2^k}$, i.e., $\bz=\bigvee_{i=1}^{2^k}\bz_i\in\R^{2^k n}$ where $\bz_i\in\bigcap_{\bu\in\BH^{n}}\allowbreak\left\{\bx\in\R^n:\bu^\T\bx\le 1\right\}$ for each $i$,
%     % \footnote{This is viewed as performing Cartesian product $2^k$ times. By the way, the dimension of $\bx$ in the last expression is $n$, which is different from the others.},
%     s.t. $\by=2^{-k/q}I_{m,k}\bz$, which further implies that
%     \begin{align*}
%     \|\by\|_q=\|2^{-k/q}I_{m,k}\bz\|_q&\le 2^{-k/q}(2^k n)^{1/q - 1/2}\|I_{m,k}\|_2\left\|\bigvee_{i=1}^{2^k}\bz_i\right\|_2\\&\le n^{1/q - 1/2}\left(\sum_{i=1}^{2^k}\left(\frac{1}{\tau}\right)^2\right)^{1/2}\le \frac{2^{k/2} n^{1/q - 1/2}}{\tau}.
%     % \le 2^{-k/q}\|H\|_q\|\bz\|_q=2^{-k/q} 2^{k/q}\left(\sum_{i=1}^{2^k}\|\bz_i\|_q^q\right)^{\frac{1}{q}}\le\left(\sum_{i=1}^{2^k}\left(\frac{1}{\tau}\right)^q\right)^{\frac{1}{q}}=\frac{2^{k/q}}{\tau},
%     \end{align*}
%     where the first inequality follows from Lemma~\ref{lma:lp-norm-equiv}, while the second is due to the spectrum of $I_{m,k}$ as previously remarked, the fact that $\BH^{n}$ is a $\tau$-hitting set of $\BS_p^n$, Proposition~\ref{lma:conn-hitapp}, and the monotonicity of $\ell_p$-norms, which completes the proof.
% \end{proof}

% It is clear from Lemma~\ref{lma:2km} that provided $2^k\ge n$, the linear transformation $2^{-k/p}I_{m,k}$ indeed helps in improving the hitting ratio of hitting sets in the form of $\BE^{2^k} \boxtimes \BH^{n}$, as the hitting ratio of $\BE^{2^k} \boxtimes \BH^{n}$ is only $\frac{\tau}{2^{k/q}}$, by Lemma~\ref{lma:kronecker-hitting}. We also remark that, in the last chain of equalities and inequalities in the proof of Lemma~\ref{lma:2km}, the equivalence between $\ell_q$ and $\ell_2$ norms, i.e., a special case of Lemma~\ref{lma:lp-norm-equiv}, has been used. To explain why this extra step is needed instead of directly considering the induced $q$-norm of $I_{m,k}$ and to deliver some useful insights, we next present a lemma which fully characterizes the induced $q$-norm of $I_{m,k}$.

% \begin{lemma}\label{lma:induced-qnorm}
%     It holds that
%     $$
%     \|I_{m,k}\|_q=\begin{cases}
%         2^{k/p} & \text{if $q>2$,}\\
%         2^{k/q} & \text{if $1<q<2$.}
%     \end{cases}
%     $$
% \end{lemma}

% \begin{proof}
% We divide the deductions into the following two cases.

% % \begin{enumerate}[label=\uline{\Rseries Case~\arabic*:}, leftmargin=*]
% %     \item
% % \end{enumerate}

% % \noindent\uline{\Rseries Case 1:}

% % \noindent\uline{\Rseries Case 2:}

% \begin{itemize}
%     \item If $q>2$, then by the log-convexity of $\ell_p$-norms, we have that
%     $$
%     \|I_{m,k}\|_q\le\|I_{m,k}\|_2^{2/q}\|I_{m,k}\|_\infty^{1-2/q}=(2^{k/2})^{2/q} (2^k)^{1-2/q}=2^{k/p},
%     $$
%     where the first equality follows from the spectrum of $I_{m,k}$ and its special structure that each row of it has an absolute sum of $2^k$. On the other hand, let $\bx:=\left(2^{-k/q}\be_1^n\right)^{\vee 2^k}\in\R^{2^k n}$, we see that $\|\bx\|_q=\sqrt[\leftroot{-2}\uproot{2}q]{2^k(2^{-k/q})^q}=1$ and $I_{m,k}\bx=2^k 2^{-k/q}\be_1^{2^k n}=2^{k/p}\be_1^{2^k n}$, and thus $\|I_{m,k}\|_q\ge\frac{\|I_{m,k}\bx\|_q}{\|\bx\|_q}=2^{k/p}$, which completes the first case.
%     \item If $q<2$, then similarly, by the log-convexity of $\ell_p$-norms, we have that
%     $$
%     \|I_{m,k}\|_q\le\|I_{m,k}\|_1^{2/q -1}\|I_{m,k}\|_2^{2-2/q}=(2^k)^{2/q -1}(2^{k/2})^{2- 2/q} =2^{k/q}.
%     $$
%     Moreover, let $\bx:=\be_1^{2^k n}$, we see that $\|\bx\|_q=1$ and $I_{m,k}\bx=\left(\be_1^n\right)^{\vee 2^k}$, and therefore $\|I_{m,k}\|_q\ge\frac{\|I_{m,k}\bx\|_q}{\|\bx\|_q}=2^{k/q}$, which completes the second case.
% \end{itemize}
% Combining the above analysis yields the desired result.
% \end{proof}

% As we can easily see from Lemma~\ref{lma:induced-qnorm}, the induced $q$-norm of $I_{m,k}$ is first decreasing and then increasing w.r.t. $q$, and its minimum is attained at $q=2$. Therefore, when we need to bound some quantity related to the induced $q$-norm of $I_{m,k}$, we shall first make some efforts to transform it into something related to the induced $2$-norm of $I_{m,k}$ so as to obtain a tighter bound. This explains the functionality of the extra step mentioned above.

% % Remark:

% % \begin{conjecture}
% %     It holds that for $\by=\bigvee_{i=1}^{2^k}\by_i$ where $\by_i\in\R^{n}$ for each $i=\in[2^k]$,
% %     $$
% %     \|I_{m,k}\by\|_q\le 2^{k/q} \left\|\bigvee_{i=1}^{2^k}\|\by_i\|_q\right\|_2.
% %     $$
% % \end{conjecture}

% % Thoughts:
% % \begin{itemize}
% %     \item Numerical evidence suggest that the above is true.
% %     \item When $k=1$ and $q=2$, the above reduces to the parallelogram identity.
% %     \item When $k=1$ and $1<q<2$, the above is related to the so-called Clarkson's inequality in $\ell_p$-space theory.
% %     \item The above has an expansion
% %     $$
% %     \left(\sum_{i=1}^{2^k}\left\|\sum_{j=1}^{2^k}\epsilon_{ij}\by_j\right\|_q^q\right)^{\frac{1}{q}}\le 2^{k/q} \left(\sum_{i=1}^{2^k}\|\by_i\|_q^2\right)^{1/2}.
% %     $$
% %     where $\epsilon_{ij}$ is the $(i,j)$-th element of the Hadamard matrix. This expansion reminds us with the generalized Clarkson's inequality.
% %     % In particular, the generalized Clarkson's inequality,
% % \end{itemize}

% However, although the matrix $I_{m,k}$ is conducive for further improving some specific hitting sets, we observe that it unfortunately only exists in dimensions that are some multiples of powers of $2$, which are quite scarce in reality, and thus Lemma~\ref{lma:2km} is not as general as we want. Fortunately, the binary representation trick can be applied here to factorize any dimension into a sum of powers of $2$, and as a result, the hitting set in each factored subspace can thus be successfully enhanced through the power of Lemma~\ref{lma:2km}. This leads to the following main theorem of this section, which is converted from~\cite[Lemma~3.13]{brieden2001deterministic} (i.e., Lemma~\ref{lma:lma3.13} in Appendix~\ref{sec:brieden-lemmas}) by virtue of Proposition~\ref{lma:conn-hitapp}, and explicitly presents a construction for $O\left(\operatorname{poly}(n)\right)$-sized $\Omega\left(\frac{\sqrt[\leftroot{-2}\uproot{2}p]{\ln{n}}}{\sqrt{n}}\right)$-hitting sets of $\BS_p^n$, which have the capability to help bridge the gap between the primal and dual.

% \begin{theorem}
%     For any $p\in(2,\infty)$, given a number of dimension $n\ge 2$, let $m:=\lceil \ln{n}\rceil$, write $n=hm+r$ by the quotient-remainder theorem where $h\in\BN$ and $r\in\{0,1,\dots,m-1\}$, and let $l:=\lfloor \log_2 \left(\frac{n}{m}+1\right)\rfloor$. Then, $h$ can be written as $\sum_{k=0}^{s} b_k 2^k$ where $ b_k\in\{0,1\}$ and thus $n$ can be written as $\sum_{k=0}^{s} b_k 2^k m + r$, and we further have that
%     \begin{equation*}
%     \begin{aligned}
%     \BH_{2}^n(\alpha,\beta)&:=&&\bigcup_{k\in\left\{i\in\{0\}\cup[s]: b_i=1\right\}}\Bigg(\bd{0}_{\sum_{i=0}^{k-1} b_i 2^i m} \bigvee \left(2^{-\frac{k}{p}}I_{2^k, m}\left(\BE^{2^k} \boxtimes \BH_H^{m}(\alpha,\beta)\right)\right)\bigvee \bd{0}_{\sum_{i=k+1}^{s} b_i 2^i m +r}\Bigg)\\\bigcup\left(\bd{0}_{hm}\bigvee\BH_H^{r}(\alpha,\beta)\right)\in\BT_p^n\left(\frac{\mu_{\alpha,\beta}}{\left(\frac{2^{q/2}}{2^{q/2}-1}\right)^{\frac{1}{q}}}\frac{(\ln{n})^{1/p}}{(n+\ln{n}+1)^{1/2}},\nu_{\alpha,\beta} n^{\ln{\nu_{\alpha,\beta}}}\left(\frac{n}{\ln{n}}+1\right)\right).\span\span\span
%     \end{aligned}
%     \end{equation*}
% \end{theorem}

% We remark that, in the binary factorization $n=\sum_{k=0}^{s} b_k 2^k m + r$, the undesired case where $2^k<m$ happens only if $k\in O\left(\ln{\ln{n}}\right)$ since $m=\lceil \ln{n}\rceil$, which is small enough to be negligible compared to $s=\lfloor \log_2 \left(\frac{n}{m}+1\right)\rfloor$, the number of bits used to represent $h$ binarily, which is of order $\Omega\left(\ln{\left(\frac{n}{\ln{n}}\right)}\right)$. This simple observation explains how the enhancement is led by the above theorem.

% % , since the binary representation of any $c\in\BN$ has $\lfloor\log_2 c\rfloor+1$ bits,

% \begin{proof}
%     We first verify the representability of $h$ by $l+1$ binary bits. Since $n=hm+r$ and $r\in\{0,1,\dots,m-1\}$, we have that $\frac{n}{m}\in[h,h+1)$. By the definition of $s$, we have that $\sum_{k=0}^{s} b_k 2^k$ can at most represent $\sum_{k=0}^{s} 2^k=2^{s+1}-1\ge\frac{n}{m}+1-1=\frac{n}{m}\ge h$, and thus $h$ can indeed be represented as a binary code of length $l+1$, as desired. Besides, it is easy to see that $\BH_{2}^n(\alpha,\beta)\subseteq\BS_p^n$, since $2^{-k/p}I_{2^k, m}\left(\BE^{2^k} \boxtimes \BH_H^{m}(\alpha,\beta)\right)\subseteq\BS_{p}^{2^k m}$, as previously established in Lemma~\ref{lma:2km}, and $\BH_H^{r}(\alpha,\beta)\subseteq\BS_p^r$, by its definition. Moreover, it is also easy to compute that the cardinality of the constructed hitting set satisfies
%     \begin{align*}
%     |\BH_{2}^n(\alpha,\beta)|&=\sum_{k=0}^l  b_k 2^k \left(\nu_{\alpha,\beta}\right)^m + \left(\nu_{\alpha,\beta}\right)^r\le(h+1)\left(\nu_{\alpha,\beta}\right)^m
%     \\&\le \left(\frac{n}{\ln{n}}+1\right)\left(\nu_{\alpha,\beta}\right)^{\ln{n}+1}=\nu_{\alpha,\beta} n^{\ln{\nu_{\alpha,\beta}}}\left(\frac{n}{\ln{n}}+1\right),
%     \end{align*}
%     as desired. Therefore, all it remains is to show the desired hitting ratio of $\BH_{2}^n(\alpha,\beta)$. By Proposition~\ref{lma:conn-hitapp}, it is equivalent to show that
%     $$
%     \bigcap_{\bu\in \BH_{2}^n(\alpha,\beta)}\{\bx\in\R^n:\bu^\T\bx\le 1\}\subseteq \frac{\left(\frac{2^{q/2}}{2^{q/2}-1}\right)^{\frac{1}{q}}}{\mu_{\alpha,\beta}}\frac{(n+\ln{n}+1)^{1/2}}{(\ln{n})^{1/p}}\mathbb{B}_q^n.
%     $$
%     Let
%     \begin{equation*}
%     \begin{aligned}
%     &\by\in\bigcap_{\bu\in \BH_{2}^n(\alpha,\beta)}\{\bx\in\R^n:\bu^\T\bx\le 1\}\\
%     &=\left(\bigtimes_{k\in\left\{i\in\{0\}\cup[s]: b_i=1\right\}}\bigcap_{\bu\in 2^{-\frac{k}{p}}I_{2^k, m}\left(\BE^{2^k} \boxtimes \BH_H^{m}(\alpha,\beta)\right)}\{\bx\in\R^{2^k m}:\bu^\T\bx\le 1\}\right)\bigtimes\hspace{-1em}\bigcap_{\bu\in \BH_H^r(\alpha,\beta)}\hspace{-1em}\{\bx\in\R^r:\bu^\T\bx\le 1\},
%     \end{aligned}
%     \end{equation*}
%     i.e., $\by=\left(\bigvee_{k\in\left\{i\in\{0\}\cup[s]: b_i=1\right\}}\bz_k\right)\vee \bw$ for some $\bz_k\in\bigcap_{\bu\in 2^{-\frac{k}{p}}I_{2^k, m}\left(\BE^{2^k} \boxtimes \BH_H^{m}(\alpha,\beta)\right)}\{\bx\in\R^{2^k m}:\bu^\T\bx\allowbreak\le\allowbreak 1\}$ and $\bw\in\bigcap_{\bu\in \BH_H^r(\alpha,\beta)}\{\bx\in\R^r:\bu^\T\bx\le 1\}$.
%     Then, by Lemma~\ref{lma:2km}, Proposition~\ref{thm:HE} and again Proposition~\ref{lma:conn-hitapp},
%     % and Proposition~\ref{lma:conn-hitapp},
%     we have that
%     % $\by=\left(\bigvee_{k\in\left\{i\in\{0\}\cup[s]: b_i=1\right\}} 2^{-\frac{k}{p}}I_{2^k, m}\bz_k\right)\bigvee\bw$ where $\bz_k\in\left(\BE^{2^k} \boxtimes \BH_H^{m}(\alpha,\beta)\right)$ and $\bw\in\BH_H^{r}(\alpha,\beta)$.
%     % Therefore, we see that
%     \begin{align*}
%         \|\by\|_q^q=\sum_{k\in\left\{i\in\{0\}\cup[s]: b_i=1\right\}}\left\|\bz_k\right\|_q^q+\|\bw\|_q^q&\le\sum_{k=0}^{s}\left(\frac{2^{k/2}m^{1/q - 1/2}}{\mu_{\alpha,\beta}}\right)^q + \frac{1}{\mu_{\alpha,\beta}^q}\\
%         &\le\left(\frac{m^{1/q - 1/2}}{\mu_{\alpha,\beta}}\right)^q\left(\sum_{k=0}^{s}\left(2^{q/2}\right)^k + 1\right)
%         % =\left(\frac{m^{1/q - 1/2}}{\mu_{\alpha,\beta}}\right)^q\frac{\left(2^{q/2}\right)^{s+1}-1}{2^{q/2}-1}\\
%         % &
%         \\
%         &\le\left(\frac{m^{1/q - 1/2}}{\mu_{\alpha,\beta}}\right)^q\frac{\left(2^{s+1}\right)^{q/2}}{2^{q/2}-1}\\
%         &\le\left(\frac{m^{1/q - 1/2}}{\mu_{\alpha,\beta}}\right)^q\frac{2^{q/2}}{2^{q/2}-1}\left(\frac{n}{m}+1\right)^{q/2},
%     \end{align*}
%     where the second, third and last inequalities follow from the facts that $m\ge 1$ and $2^{q/2}-1\le 1$, and the definition of $s$, respectively. Now, recall that $m=\lceil\ln{n}\rceil$, we have that
%     $$
%     \|\by\|_q\le\frac{\left(\frac{2^{q/2}}{2^{q/2}-1}\right)^{\frac{1}{q}}}{\mu_{\alpha,\beta}}\frac{(n+m)^{1/2}}{m^{1/p}}\le\frac{\left(\frac{2^{q/2}}{2^{q/2}-1}\right)^{\frac{1}{q}}}{\mu_{\alpha,\beta}}\frac{(n+\ln{n}+1)^{1/2}}{(\ln{n})^{1/p}},
%     $$
%     as desired, which completes the proof.
% \end{proof}


% \subsection{\text{$O\left(\operatorname{poly}(n)\right)$-sized hitting sets of $\BS_p^n$ ($p\in\BQ\cap(2,\infty)$) have hitting ratio at most $\Omega(\sqrt{\ln{n}/{n}})$}.}

\subsection{Randomized $\Omega(\sqrt{\ln n/n})$-hitting sets of $\BS_p^n$}\label{sec:rand-hitting-set}

The $\Omega(\sqrt[p]{\ln{n}}/\sqrt{n})$-hitting sets in Section~\ref{sec:brieden-hitting-set} already significantly increased the $\Omega(\sqrt[q]{\ln{n}/n})$ one in Section~\ref{sec:worst-hitting} when $p\in(2,\infty)$. However, there is still room to be improved with the help of randomization. In particular, we provide randomized $\Omega(\sqrt{\ln{n}/{n}})$-hitting sets of $\BS_p^n$. % As we shall remark later, this is the largest possible hitting ratio with polynomial cardinality when $p\in[2,\infty)$.

To begin with, we need a couple of technical results. One is a property of a grid-based hitting set of $\BS_p^n$ (Lemma~\ref{lma:lq-enet}) and the other is a lower bound of the inner product between a random vector on $\BS_p^n$ and a vector on $\BS_q^n$ (Lemma~\ref{lma:lp-sampling}).

% This result will be established by adapting the methodology introduced in~\cite[Section~2.1]{he2023approx}, which was originally designed for $\BS^n$, to $\BS_p^n$. 
%
% % \subsection{$O\left(\operatorname{poly}(n)\right)$-sized randomized $\Omega(\sqrt{\ln{n}/{n}})$-hitting set of $\BS_p^n$.}

% The general methodology introduced in~\cite[Section~2.1]{he2023approx} to construct randomized hitting sets of $\BS^n$ is two-step, as briefly described below.
% \begin{enumerate}
%     \item Construct a grid set $\BY^n\subseteq\BS^n$ to deterministically ensure that, for any $\bx\in\BS^n$, there is some $\by\in\BY^n$, s.t. $\by^\T\bx$ is close to $1$.
%     % to well approximate the whole $\BS^n$.
%     \item Randomly sample a sufficiently large number of vectors on $\BS^n$ (denoted as $\BH^n$) to ensure that, for any $\by\in\BY^n$, there is some $\bz\in\BH^n$, s.t. $\bz^\T\by\ge\tau$ for some desired $\tau$ with high probability.
%     % well approximate the constructed grid set.
% \end{enumerate}
% As a result, since $\by^\T\bx$ is close to $1$, while $\bz^\T\by\ge\tau$ with high probability, by some transitivity arguments, it can be favorably concluded that $\bz^\T\bx\ge\rho\tau$
% % for any $(\bz,\bx)\in\BH^n\times\BS^n$
% with high probability, where $\rho\in(0,1)$ is some constant, and $\BH^n$ thereupon serves as a desired randomized hitting set. To adapt the above methodology to $\BS_p^n$, a very natural and intuitive extension is as follows.
% \begin{enumerate}
%     \item Construct a grid set $\BY_p^n\subseteq\BS_p^n$ (say, by using $\BC^n(\gamma)$ returned by Algorithm~\ref{alg:lp-fptas} in Appendix~\ref{sec:grid-point}) to deterministically ensure that, for any $\bx\in\BS_q^n$, there is some $\bv\in\BY_p^n$, s.t. $\bv^\T\bx$ is close to $1$, and in turn construct another grid set $\BY_q^n\subseteq\BS_q^n$ to deterministically ensure that, for any $\bv\in\BY_p^n$, there is some $\bu\in\BY_q^n$, s.t. $\bu^\T\bv$ is close to $1$ as well.
%     % to well approximate the whole $\BS^n$.
%     \item Randomly sample a sufficiently large number of vectors on $\BS_p^n$ (denoted as $\BH^n$) to ensure that, for any $\bu\in\BY_q^n$, there is some $\bz\in\BH^n$, s.t. $\bz^\T\bu\ge\tau$ for some desired $\tau$ with high probability.
%     % well approximate the constructed grid set.
% \end{enumerate}
% Therefore, since $\bv^\T\bx$ and $\bu^\T\bv$ are both close to $1$, while $\bz^\T\bu\ge\tau$ with high probability, we should probably obtain a similar conclusion that $\bz^\T\bx\ge\rho\tau$ for some constant $\rho\in(0,1)$ with high probability as well. However, such an argument is invalid, unfortunately, as explained below. Recall that as we remarked in Section~\ref{sec:lp-sphere-covering}, on $\BS_p^n\times\BS_q^n$, even if two vectors have the largest-possible inner product, they are not necessarily parallel in general, and they can even be nearly orthogonal. As a result, even if $\bv^\T\bx$ and $\bu^\T\bv$ are close to $1$, and
% $\bz^\T\bu\ge\tau$ with high probability, $\bz^\T\bx$ can even be negative with high probability, unfavorably.
% % Therefore, existing hitting sets of $\BS_q^n$
% % \footnote{Note that the focus on $\BS_q^n$ is deliberate, as will be clear from the proof of Theorem~\ref{thm:rand-hitting-set} below.}
% % that can achieve an arbitrarily large hitting ratio, such as , are unable to be directly used for the first (extended) sub-procedure above, as it is unclear if it can be guaranteed that for any vector on $\BS_q^n$, there is always some vector in the hitting sets s.t. they can have a large enough inner product.
% This undesiredness shows that the above natural extension is in fact infeasible, and further entails new techniques to be developed for constructing randomized hitting sets of $\BS_p^n$.

% It should be realized that, our initial intention of constructing two different grid sets on $\BS_p^n$ and $\BS_q^n$ respectively in the above procedure is to use them as bridges to conclude $\bz^\T\bx\ge\rho\tau$ from $\bz^\T\bu\ge\tau$. However, if we can show that it is possible to directly construct a grid set on $\BS_q^n$, say $\BY^n$, s.t. for any $\bx\in\BS_q^n$, there is some $\by\in\BY^n$, s.t. $\bz^\T\bx$ for any $\bz\in\BS_p^n$ can be well approximated by $\bz^\T\by$, then the intermediate grid sets (i.e., $\BY_p^n$ and $\BY_q^n$ above) are no longer required to be constructed since we can directly conclude from $\bz^\T\by$ is large enough that $\bz^\T\bx$ will also be large enough due to the approximation, and the non-transitivity issue can thus be bypassed. This motivates us to show the following lemma, which implies that, such a grid set can indeed be constructed.
% % \min_{\by\in\BY^n}\max_{\bz\in\BS_p^n}
% % \min_{\bx\in\BS_q^n}\max_{\bz\in\BS_p^n}

\begin{lemma}\label{lma:lq-enet}
    For any $p\in[1,\infty]$ and $m,n\in\BN$, one has $|\BH^n_G(m)|\le(2m+1)^n$ where
    $$
    \BH^n_G(m):=\left\{\frac{\bw}{\|\bw\|_p}\in\BS_p^n:\bw\neq\bd{0},\, w_i\in\left\{0,\pm\frac{1}{m},\pm\frac{2}{m},\dots,\pm 1\right\}\text{ for } i=1,2,\dots, n\right\}.
    $$
    For any $\bx\in\BS_p^n$, there exists $\by\in\BH^n_G(m)$ such that for any $\bz\in\R^n$,
    \begin{align*}
        \by^\T\bz-\frac{\|\bz\|_1}{m} &\le \bx^\T\bz \le \left(1+\frac{n^{1/p}}{m}\right)\by^\T\bz+\frac{\|\bz\|_1}{m} &\text{if }\by^\T\bz\ge 0,\\
        \left(1+\frac{n^{1/p}}{m}\right)\by^\T\bz-\frac{\|\bz\|_1}{m}&\le\bx^\T\bz\le \by^\T\bz+\frac{\|\bz\|_1}{m} &\text{if }\by^\T\bz< 0.
    \end{align*}
\end{lemma}
\begin{proof}
    $|\BH^n_G(m)|\le(2m+1)^n$ obviously holds from its definition. For any $\bx\in\BS_p^n$, we choose $w_i\in\{0,\pm\frac{1}{m},\pm\frac{2}{m},\dots,\pm 1\}$ to be the closest to $x_i$ satisfying $|x_i|\le|w_i|$, i.e., for $i=1,2,\dots,n$
    $$
    w_i=\begin{cases}
        \frac{\lceil m x_i\rceil}{m} & \text{if } x_i\ge 0 \\
        \frac{\lfloor m x_i\rfloor}{m} &  \text{if } x_i< 0.
    \end{cases}
    $$
    By noticing that $0\le |w_i|-|x_i|\le \frac{1}{m}$, one has 
    \begin{equation}\label{eq:bound-y}
    1=\|\bx\|_p\le\|\bw\|_p=\left\||\bw|\right\|_p\le\left\||\bx|+\frac{1}{m}\bd{1}\right\|_p\le\left\||\bx|\right\|_p+\frac{1}{m}\left\|\bd{1}\right\|_p=1+\frac{n^{1/p}}{m}.
    \end{equation}
    Since $|w_i-x_i|\le \frac{1}{m}$, for any $\bz\in\R^n$,
    $$
    \left|\bw^\T\bz-\bx^\T\bz\right| \le \sum_{i=1}^n |w_i-x_i|\cdot |z_i| \le \sum_{i=1}^n \frac{1}{m} |z_i| = \frac{\|\bz\|_1}{m}.
    $$
    Let $\by=\bw/\|\bw\|_p\in\BS_p^n$ and the above leads to
    $$
    \|\bw\|_p\by^\T\bz-\frac{\|\bz\|_1}{m}\le\bx^\T\bz\le \|\bw\|_p\by^\T\bz+\frac{\|\bz\|_1}{m}.
    $$
    The desired bounds of $\bx^\T\bz$ can be obtained immediately based on the sign of $\by^\T\bz$ by noticing the lower and upper bounds of $\|\bw\|_p$ in~\eqref{eq:bound-y}.
\end{proof}

% Notably, as can be easily seen, if we restrict $\bz\in\BS_p^n$ and let $m\rightarrow\infty$ in Lemma~\ref{lma:lq-enet}, the error between $\bx^\T\bz$ and $\bz^\T\bz$ will vanish, which shows that $\BH^n_G(m)$ exactly fulfills our desired requirements.
% Therefore, Lemma~\ref{lma:lq-enet} successfully bypasses the aforementioned non-transitivity issue and can thus be used to serve our purposes.
% which makes it possible to be used for constructing randomized hitting sets of $\BS_p^n$.
% Therefore, based on Lemma~\ref{lma:lq-enet}, we can design another feasible and promising procedure for constructing randomized hitting sets of $\BS_p^n$, as detailed below.
% Based on Lemma~\ref{lma:lq-enet}, the general methodology for constructing $O\left(\operatorname{poly}(n)\right)$-sized randomized $\Omega(\sqrt{\ln{n}/{n}})$-hitting sets of $\BS_p^n$ is outlined as follows.
% \begin{enumerate}
%     \item Construct a grid set $\BH^n_G(m)\subseteq\BS_q^n$ to deterministically ensure that, for any $\bx\in\BS_q^n$, there is some $\by\in\BH^n_G(m)$, s.t. $\bz^\T\bx$ for any $\bz\in\BS_p^n$ can be well approximated by $\bz^\T\by$.
%     \item Randomly sample a sufficiently large number of vectors on $\BS_p^n$ (denote as $\BH^n$) to ensure that, for any $\by\in\BH^n_G(m)$, there is some $\bz\in\BH^n$, s.t. $\bz^\T\by\ge\tau$ for some desired $\tau$ with high probability.
%     % well approximate the constructed grid set.
% \end{enumerate}

% the first sub-procedure mentioned above.
% which will be much more clear in the proof of Theorem~\ref{thm:rand-hitting-set} below.
% Clearly, no transitivity argument will be needed in the above procedure with the help of Lemma~\ref{lma:lq-enet}, and as a result, we shall hopefully obtain $\bz^\T\bx\ge\rho\tau$ eventually for any $(\bz,\bx)\in\BH^n\times\BS^n$ for some constant $\rho\in(0,1)$ with high probability after the procedure.

% The other vital result is a nontrivial lower bound for random sampling on $\BS_p^n$.
\begin{lemma}[{\cite[Lemma 3.3]{khot2008linear}}]\label{lma:lp-sampling}
    For any $p\in(2,\infty)$ and integer $n\ge2$, let $\by\in\R^n$ whose entries are i.i.d. random variables with the probability density function $p e^{-|x|^p}/\int_0^{\infty} 2t^{x-1} e^{-t} dt$. There exist universal constants $\delta_0,\delta_1,\delta_2>0$ such that for any $\bz\in\BS_q^n$
    $$
    \Prob \left\{\frac{\bz^\T \by}{\|\by\|_p} \ge \sqrt{\frac{\delta_0 \ln{n}}{n}}\right\} \ge \frac{\delta_1}{n^{\delta_2}}.
    $$
\end{lemma}

We are ready to present randomized $\Omega(\sqrt{\ln{n}/{n}})$-hitting sets of $\BS_p^n$. To simplify the language, we call the vector $\by/\|\by\|_p\in\BS^n_p$ to be the even distribution on $\ell_p$-sphere, where the entries of $\by\in\R^n$ are i.i.d. random variables with the probability density function $p e^{-|x|^p}/\int_0^{\infty} 2t^{x-1} e^{-t} dt$.
\begin{theorem}\label{thm:rand-hitting-set}
For any $p\in(2,\infty)$ and $\epsilon\in(0,1)$, there exist universal constants $\delta_0,\delta_2,\delta_3>0$, such that
$$
\BH_{3}^n(\epsilon):=\left\{\bz_i \text{ is i.i.d. even on $\BS_p^n$ for }i=1,2,\dots, \left\lceil\delta_3n^{\delta_2}\left( \left(\frac{1}{2}+\frac{1}{q}\right)n \ln n+\ln \frac{1}{\epsilon}\right)\right\rceil \right\}
$$
satisfies
$$\Prob\left\{\BH_{3}^n(\epsilon)\in\BT_p^n\left(\sqrt{\frac{\delta_0 \ln{n}}{2n}}, \left\lceil\delta_3n^{\delta_2}\left( \left(\frac{1}{2}+\frac{1}{q}\right)n \ln n+\ln \frac{1}{\epsilon}\right)\right\rceil\right)\right\}\ge1-\epsilon.
$$
\end{theorem}
% Our general construction idea is as follows: We use Algorithm~\ref{alg:lp-fptas} to generate two grid point sets on $\BS_p^n$ and $\BS_q^n$, respectively, both having constant hitting ratio and exponential-sized cardinality. Besides, we then randomly sample polynomially-many points on $\BS_p^n$ according to the distribution introduced in Lemma~\ref{lma:lp-sampling}. Intuitively, the points randomly generated should have a $\Omega(\sqrt{\ln{n}/{n}})$ hitting ratio with those grid points on $\BS_q^n$ by Lemma~\ref{lma:lp-sampling}, but the grid points on $\BS_q^n$ have been proven (in Lemma~\ref{lma:lp-grid}) to have constant hitting ratio with any point on $\BS_p^n$, and as a special case, any grid point on $\BS_p^n$, and in a similar vein the grid points on $\BS_p^n$ have also been proven to have constant hitting ratio with any point on $\BS_p^n$. Combining these pieces shall provide what we want.
\begin{proof}
The proof consists two parts. We first provide sufficient conditions for $\BH_3^n(\epsilon)$ to cover $\BS^n_q$ with high probability by using $\BH^n_G(m)$ as an intermediary. We then analyze the parameters to make sure the probability lower bound and the upper bound of $|\BH_{3}^n(\epsilon)|$ are achieved uniformly.

First, we show that $\BH_3^n(\epsilon)$ is close enough to $\BH^n_G(m)$ with high probability, i.e.,
    \begin{equation}\label{eq:halfway}
    \Prob\left\{\min_{\by\in\BH^n_G(m)}\max_{\bz\in\BH_{3}^n(\epsilon)}\bz^\T\by \ge \sqrt{\frac{\delta_0 \ln{n}}{n}}\right\} \ge 1-\epsilon.
    \end{equation}
    To see why, we have that the probability of its complement
    \begin{align*}
    \Prob\left\{\min_{\by\in\BH^n_G(m)}\max_{\bz\in\BH_{3}^n(\epsilon)}\bz^\T\by<\sqrt{\frac{\delta_0 \ln{n}}{n}}\right\}
    &=\Prob\left\{\bigcup_{\by\in\BH^n_G(m)}\bigcap_{\bz\in\BH_{3}^n(\epsilon)}\left\{\bz^\T\by<\sqrt{\frac{\delta_0 \ln{n}}{n}}\right\}\right\}\\
    &\le \sum_{\by\in\BH^n_G(m)}\Prob\left\{\bigcap_{\bz\in\BH_{3}^n(\epsilon)}\left\{\bz^\T\by<\sqrt{\frac{\delta_0 \ln{n}}{n}}\right\}\right\}\\
    &= \sum_{\by\in\BH^n_G(m)}\prod_{\bz\in\BH_{3}^n(\epsilon)}\Prob\left\{\bz^\T\by<\sqrt{\frac{\delta_0 \ln{n}}{n}}\right\}
    \\&\le (2m+1)^n \left(1-\frac{\delta_1}{n^{\delta_2}}\right)^{|\BH_{3}^n(\epsilon)|},
    \end{align*}
    where the first inequality follows from the subadditivity of probability measure, the second equality is due to the independence of $\bz$'s, and the last inequality follows from Lemma~\ref{lma:lq-enet} and Lemma~\ref{lma:lp-sampling}. To ensure the upper bound $(2m+1)^n (1-\frac{\delta_1}{n^{\delta_2}})^{|\BH_{3}^n(\epsilon)|}\le\epsilon$, it is not difficult to verify that it suffices to have 
    \begin{equation}\label{eq:bh3card}
    |\BH_{3}^n(\epsilon)|\ge\frac{n^{\delta_2}}{\delta_1}\left(n\ln{(2m+1)}+\ln{\frac{1}{\epsilon}}\right).
    \end{equation}
    Therefore, for any $\bx\in\BS_q^n$, there exists $\by\in\BH^n_G(m)$ by Lemma~\ref{lma:lq-enet}, and then there exists $\bz\in\BH_{3}^n(\epsilon)$ with probability at least $1-\epsilon$ by~\eqref{eq:halfway}, such that
    % based on the event that $\min_{\by\in\BH^n_G(m)}\max_{\bz\in\BH_{3}^n(\epsilon)}\bz^\T\by\ge\sqrt{\frac{\delta_0 \ln{n}}{n}}$
    $$
    \bz^\T\bx\ge \bz^\T\by - \frac{\|\bz\|_1}{m} \ge \sqrt{\frac{\delta_0 \ln{n}}{n}} - \frac{n^{1/q}\|\bz\|_p}{m} =\sqrt{\frac{\delta_0 \ln{n}}{n}} - \frac{n^{1/q}}{m},
    $$
    where the second inequality is due to the bounds between $\ell_p$-norms (Lemma~\ref{lma:lp-norm-equiv}). To ensure the lower bound $\sqrt{\frac{\delta_0 \ln{n}}{n}} - \frac{n^{1/q}}{m}\ge \sqrt{\frac{\delta_0 \ln{n}}{2n}}$, the hitting ratio, we need to have $\frac{n^{1/q}}{m} \le (\sqrt{2}-1)\sqrt{\frac{\delta_0 \ln{n}}{2n}}$. In a word, it suffices to have
    \begin{equation}\label{eq:bh3lower}
     2m+1\ge\sqrt{\frac{47}{\delta_0 \ln{n}}}n^{\frac{1}{q} + \frac{1}{2}} + 1.
    \end{equation}

    Let us carefully define $m$ for every $n$ in order to satisfy both~\eqref{eq:bh3card} and~\eqref{eq:bh3lower}. Let $n_0\ge2$ be the smallest one such that $n\ge \sqrt{\frac{47}{\delta_0 \ln{n}}}n + 3$ for all $n\ge n_0$ and this $n_0$ depends only on $\delta_0$. We then let $\delta\ge1$ be the smallest one such that $n^{\delta}\ge \sqrt{\frac{47}{\delta_0 \ln{n}}}n + 3$ for all $2\le n\le n_0$ and this $\delta$ also depends only on $\delta_0$. Therefore, we uniformly have $n^{\delta}\ge \sqrt{\frac{47}{\delta_0 \ln{n}}}n + 3$ for any $n\ge2$. Since $\delta\ge1$ and $\frac{1}{q}-\frac{1}{2}\ge0$, one has
    $$n^{(\frac{1}{q} + \frac{1}{2})\delta} = n^{(\frac{1}{q} - \frac{1}{2})\delta}  n^\delta \ge n^{\frac{1}{q} - \frac{1}{2}}\left( \sqrt{\frac{47}{\delta_0 \ln{n}}}n + 3\right) \ge  \sqrt{\frac{47}{\delta_0 \ln{n}}}n^{\frac{1}{q} + \frac{1}{2}} + 3.$$
    To summarize, if we define $m=\lfloor\frac{1}{2}(n^{\delta/q+\delta/2}-1)\rfloor$, then 
    $$ n^{(\frac{1}{q} + \frac{1}{2})\delta} \ge 2m+1 \ge n^{(\frac{1}{q} + \frac{1}{2})\delta} -2
      \ge\sqrt{\frac{47}{\delta_0 \ln{n}}}n^{\frac{1}{q} + \frac{1}{2}} + 1.
    $$
    This shows that~\eqref{eq:bh3lower} holds for any $n\ge2$. In the meantime, if we let $\delta_3=\delta/\delta_1$ that depends only on $\delta_0$ and $\delta_1$ and is thus universal, we have 
    $$
    |\BH_{3}^n(\epsilon)| %= \delta_3n^{\delta_2}\left( \left(\frac{1}{2}+\frac{1}{q}\right)n \ln n+\ln \frac{1}{\epsilon}\right)
    \ge \frac{\delta}{\delta_1} n^{\delta_2}\left( \left(\frac{1}{2}+\frac{1}{q}\right)n \ln n+\ln \frac{1}{\epsilon}\right)
    = \frac{n^{\delta_2}}{\delta_1}\left(\delta n\ln n^{\frac{1}{2}+\frac{1}{q}}+\delta\ln{\frac{1}{\epsilon}}\right)
    \ge \frac{n^{\delta_2}}{\delta_1}\left(n\ln{(2m+1)}+\ln{\frac{1}{\epsilon}}\right),
    $$
    ensuring the validity of~\eqref{eq:bh3card}.
    %
    % Let us now define $\kappa_0=\max_{1\le n\le n_0-1}\frac{\ln(2m_n+1)}{\ln\left(2n^{1/q+1/2}+1\right)}$ that depends only on $n_0$ and so depends only on $\delta_0$ as well. It suffices to show that the lower bound of $|\BH_{3}^n(\epsilon)|$ in~\eqref{eq:bh3card}. This is because for $n\ge n_0$, due to $\kappa_0\ge 1$ by its definition and the setting of $m_n=m=n^{1/q+1/2}$ in this case, the probability bound automatically fulfills, while for $n<n_0$, on the other hand, we still have that
    % \begin{align*}
    % \left\lceil\kappa_0\frac{n^{\delta_2}}{\delta_1}\left(n\ln{(2n^{\frac{1}{q} + \frac{1}{2}}+1)}+\ln{\frac{1}{\epsilon}}\right)\right\rceil &\ge\left\lceil\frac{\ln{(2m_n+1)}}{\ln{(2n^{\frac{1}{q} + \frac{1}{2}}+1)}}\frac{n^{\delta_2}}{\delta_1}\left(n\ln{(2n^{\frac{1}{q} + \frac{1}{2}}+1)}+\ln{\frac{1}{\epsilon}}\right)\right\rceil\\
    % &\ge\frac{n^{\delta_2}}{\delta_1}\left(n\ln{(2m_n+1)}+\ln{\frac{1}{\epsilon}}\right),
    % \end{align*}
    % which implies that the probability bound is fulfilled as well.
\end{proof}

% \begin{lemma}\textnormal{\cite[Theorem~2.7]{brieden2001deterministic}}
%     Let all convex bodies in $\R^n$ be presented by strong optimization oracles, and $q\in(1,2)$ be a constant. If $\operatorname{Alg}$ is a randomized algorithm that uses a polynomial number of oracle calls to compute an approximation $\operatorname{Alg}(\BK^n)$ for each convex body $\BK^n\subseteq\R^n$, then there exists a constant $c>0$ s.t., in every dimension $n$, there exists a convex body $\BK^n_0\subseteq\R^n$, for which
%     $$
%     \Prob\left(\left\{c\left(\sqrt{\frac{\ln{n}}{n}}\right)\operatorname{diam}(\BK^n_0)\le\operatorname{Alg}(\BK^n_0)\le\operatorname{diam}(\BK^n_0)\right\}\right)\le\frac{1}{4}.
%     $$
% \end{lemma}

Theorem~\ref{thm:rand-hitting-set} not only provides a simple construction via randomization but also trivially implies the existence of $\Omega(\sqrt{\ln n/n})$-hitting sets of $\BS_p^n$. However, we are currently unable to explicitly construct a deterministic one when $p\in(2,\infty)$. As we shall see in Section~\ref{sec:algorithms}, a deterministic $\Omega(\sqrt{\ln n/n})$-hitting set can be used to improve the best-known approximation bound by a deterministic polynomial-time algorithm for both the tensor spectral $p$-norm and nuclear $p$-norm.

We remark that the hitting ratio $\Omega(\sqrt{\ln n/n})$ for $\BS_p^n$ is the largest possible by a deterministic hitting set with polynomial cardinality when $p\in[2,\infty)$; see~\cite[Theorem~3.2]{brieden1998approximation}. Therefore, with the $\Omega(\sqrt[q]{\ln n/n})$-hitting set $\BH_{1}^n(\alpha,\beta)$ that attains the largest hitting ratio with polynomial cardinality when $p\in(1,2]$, this section almost provides a complete story under the deterministic framework, except explicit construction of deterministic $\Omega(\sqrt{\ln n/n})$-hitting sets when $p\in(2,\infty)$.

% We define the $\ell_q$ diameter of a convex body (i.e., a compact convex set with non-empty interior) $\mathcal{K}\subseteq\R^n$ to be $\operatorname{diam}_q(\mathcal{K}):=\max_{\bx,\by\in\mathcal{K}}\|\bx-\by\|_q$. 
     % \begin{lemma}[{\cite[Theorem~3.2]{brieden1998approximation} (see also~\cite[Theorem~3.2.1]{brieden2001deterministic})}]
% Let $\BK^n\subseteq\R^n$ be a convex body with a strong optimization oracle. Then for any $q\in(1,2)$, the accuracy of the approximation for $\operatorname{diam}_{q}(\BK^n)$ delivered by any deterministic oracle-polynomial-time algorithm is at most $\Omega(\sqrt{\ln{n}/{n}})$.
% % $$
% % \left[\Theta\left(\sqrt{\frac{\ln{n}}{n}}\right)\operatorname{diam}_q(\BK^n),\, \operatorname{diam}_q(\BK^n)\right].
% % $$
% \end{lemma}
% With the above inapproximability result at hand, we are now able to establish a hitting ratio upper bound for $O\left(\operatorname{poly}(n)\right)$-sized hitting sets of $\BS_p^n$.
% \begin{proposition}\label{lma:hitting-value-upper-bound}
% For any $p\in(2,\infty)$, the hitting ratio of any $O\left(\operatorname{poly}(n)\right)$-sized hitting set of $\BS_p^n$ is at most $\Omega(\sqrt{\ln{n}/{n}})$.
% \end{proposition}
% \begin{proof}
%     Assume to the contrary that there does exist some hitting set $\BH^n\in\BT_p^n\left(\tau, O\left(\operatorname{poly}(n)\right)\right)$ for which $\tau\in\Theta\left(\sqrt{\frac{\ln{n}}{n}}\right)\setminus\Omega\left(\sqrt{\frac{\ln{n}}{n}}\right)$.
%     % Let $q$ be the conjugate of $p$, which clearly satisfies $q\in\BQ\cap(1,2)$.
%     Then, for any convex body $\BK^n\subseteq\R^n$, let $\by_0,\bz_0\in\BK^n$ be s.t. $\|\by_0-\bz_0\|_q=\operatorname{diam}_q(\BK^n)>0$, we have that there exists some $\bx_0\in\BH^n$, s.t. $\left\langle\bx_0,\frac{\by_0-\bz_0}{\|\by_0-\bz_0\|_q}\right\rangle\ge\tau$, which implies that
%     \begin{align*}
%     \operatorname{diam}_q(\BK^n)&=\underset{\by,\bz\in\BK^n}{\max}\|\by-\bz\|_q=\underset{\bx\in\BS_p^n}{\max}\underset{\by,\bz\in\BK^n}{\max}\langle\bx,\by-\bz\rangle\ge \underset{\bx\in\BH^n}{\max}\underset{\by,\bz\in\BK^n}{\max}\langle\bx,\by-\bz\rangle\\&=\underset{\bx\in\BH^n}{\max}\left(\underset{\by\in\BK^n}{\max}\langle\bx,\by\rangle+\underset{\bz\in\BK^n}{\max}\langle-\bx,\bz\rangle\right)
%     \ge\langle\bx_0,\by_0-\bz_0\rangle\ge \tau\|\by_0-\bz_0\|_q=\tau\operatorname{diam}_q(\BK^n).
%     \end{align*}
%     Therefore, we have that there is a deterministic oracle-polynomial-time algorithm, i.e.,
%     $$
%     \operatorname{Alg}(\BK^n):=\underset{\bx\in\BH^n}{\max}\left(\underset{\by\in\BK^n}{\max}\langle\bx,\by\rangle+\underset{\bz\in\BK^n}{\max}\langle-\bx,\bz\rangle\right),
%     $$
%     which requires $2|\BH^n|\in O\left(\operatorname{poly}(n)\right)$ strong optimization oracle calls, that can approximate the $\ell_q$ diameters of convex bodies with an accuracy of $\tau\in\Theta\left(\sqrt{\frac{\ln{n}}{n}}\right)\setminus\Omega\left(\sqrt{\frac{\ln{n}}{n}}\right)$, a contradiction.
%     % for $1 \le p \le q \le \infty$,
%     % $$
%     % \|x\|_q \le\|x\|_p \le n^{1 / p-1 / q}\|x\|_q
%     % $$
% \end{proof}

% In a similar vein, by using other parts of~\cite[Theorem~3.2]{brieden1998approximation}, one can also show a dual version of Proposition~\ref{lma:hitting-value-upper-bound}, which is listed as follows without proof.
% \begin{proposition}\label{lma:hitting-value-upper-bound-dual}
% For any $p\in(1,2)$, the hitting ratio of any $O\left(\operatorname{poly}(n)\right)$-sized hitting set of $\BS_p^n$ is at most $\Omega(\sqrt[q]{\ln n/n})$.
% \end{proposition}


\section{Approximating tensor nuclear $p$-norm}\label{sec:algorithms}

As an application of the results in Section~\ref{sec:matrix-p-norm} and Section~\ref{sec:hitting-sets}, this section is devoted to the design and analysis of polynomial-time algorithms to approximate the tensor nuclear $p$-norm. As mentioned in the introduction, such results are almost blank in the literature mainly because of the lack of good approximation of the matrix nuclear $p$-norm. Armed with the $\Omega(1)$-approximation bound of the matrix nuclear $p$-norm developed in Section~\ref{sec:matrix-p-norm}, we provide an overview of approximating the tensor nuclear $p$-norms using existing tools as well as what we have developed in Section~\ref{sec:hitting-sets}, from a basic approach to the best approximation. Here in this section, we consider the tensor space $\R^{n_1\times n_2\times \dots \times n_d}$ of order $d \ge 3$ and assume without loss of generality that $2 \leq n_1 \leq n_2 \leq \dots \leq n_d$. % Moreover, for simplicity, we use $\operatorname{Alg}_k(\xi_1,\xi_2,\dots,\xi_n)$ to denote the return value of Algorithm~$k$ by calling it with inputs $\xi_1,\xi_2,\dots,\xi_n$.

Before discussing the tensor nuclear $p$-norm, let us first propose very simple algorithms to approximate the tensor spectral $p$-norm, as an immediate application of $\ell_p$-sphere covering in Section~\ref{sec:hitting-sets}.

\begin{algorithm}[!h]
\begin{algorithmic}[1]
\REQUIRE A tensor $\TT\in\R^{n_1 \times n_2 \times \dots \times n_d}$, a constant $p\in\BQ\cap(2,\infty)$ and $d-2$ hitting sets $\BH^{n_k}\in\BT^{n_k}_p(\tau_k, O({n_k}^{\alpha_k}))$ for $k=1,2,\dots,d-2$.
\ENSURE An approximation of $\|\TT\|_{p_\sigma}$.
\STATE Applying~\eqref{cor:final-spectral-p-norm} to compute
\begin{equation*}
u = \max\left\{\left\|\TT\times_1\bx_1\dots\times_{d-2}\bx_{d-2}\right\|_{p_v} :
\bx_k\in \BH^{n_k},\,k=1,2,\dots, d-2\right\};
\end{equation*}
\RETURN $u/\delta_G$.
\end{algorithmic}
\caption{Approximating the tensor spectral $p$-norm based on $\ell_p$-sphere covering}
\label{alg:alg0}
\end{algorithm}

\begin{theorem}\label{thm:snorm}
For any $\TT\in\R^{n_1 \times n_2 \dots \times n_d}$, $p\in \BQ\cap(2,\infty)$ and hitting sets $\BH^{n_k}\in\BT^{n_k}_p(\tau_k, O({n_k}^{\alpha_k}))$ for $k=1,2,\dots,d-2$,
Algorithm~\ref{alg:alg0} is a deterministic polynomial-time algorithm whose output $\operatorname{Alg}_{\,\ref{alg:alg0}}(\|\TT\|_{p_\sigma})$ satisfies
$$
\left(\frac{1}{\delta_G}\prod_{k=1}^{d-2}\tau_k\right)\|\TT\|_{p_\sigma}\le \operatorname{Alg}_{\,\ref{alg:alg0}}(\|\TT\|_{p_\sigma}) \le \|\TT\|_{p_\sigma}.
$$
\end{theorem}
\begin{proof}
Denote $(\by_1,\by_2,\dots,\by_d)$ to be an optimal solution of~\eqref{def:spectral}, i.e., $\langle\TT,\by_1\otimes\by_2\otimes \dots\otimes\by_d\rangle=\|\TT\|_{p_\sigma}$ with $\|\by_k\|_p=1$ for $k=1,2,\dots,d$.
For the vector $\bv_1=\TT\times_2\by_2\dots\times_{d}\by_{d}\in\R^{n_1}$, either $\|\bv_1\|_q=0$ or there exists $\bz_1\in\BH^{n_1}$ such that $\bz_1^{\T}\bv_1/\|\bv_1\|_q\ge\tau_1$. In any case, one has $\bz_1^{\T} \bv_1 \ge \tau_1 \|\bv_1\|_q$ and
\[
\langle\TT,\bz_1\otimes\by_2\otimes \dots\otimes\by_d\rangle= \bz_1^{\T} \bv_1 \ge \tau_1 \|\bv_1\|_q
\ge \tau_1 \by_1^{\T}\bv_1=\tau_1 \langle\TT,\by_1\otimes\by_2\otimes \dots\otimes\by_d\rangle,
\]
where the last inequality follows from H\"{o}lder's inequality and $\|\by_1\|_p=1$. Similarly, for every $k=2,3\dots,d-2$ that are chosen one by one increasingly, there exists $\bz_k\in\BH^{n_k}$ such that
\[
\langle \TT, \bz_1\otimes\dots\otimes\bz_k\otimes\by_{k+1}\otimes\dots\otimes\by_{d}\rangle = \bz_k^{\T} \bv_k \ge \tau_k \|\bv_k\|_q 
\ge \tau_k \by_k^{\T}\bv_k= \tau_k \langle \TT, \bz_1\otimes\dots\otimes\bz_{k-1}\otimes\by_{k}\otimes\dots\otimes\by_{d}\rangle,
\]
where $\bv_k=\TT\times_1\bz_1\dots\times_{k-1}\bz_{k-1}\times_{k+1}\by_{k+1}\dots\times_d\by_{d}\in\R^{n_k}$. By applying the above inequalities recursively, one has
\[
\langle \TT, \bz_1\otimes\dots\otimes\bz_{d-2}\otimes\by_{d-1}\otimes\by_{d}\rangle\ge \left(\prod_{k=1}^{d-2}\tau_k\right) \langle\TT,\by_1\otimes\by_2\otimes \dots\otimes\by_d\rangle  = \left(\prod_{k=1}^{d-2}\tau_k\right) \|\TT\|_{p_\sigma}.
\]
Therefore, by Lemma~\ref{lma:vecp}, the $u$ in Algorithm~\ref{alg:alg0} satisfies
\begin{align*}
u& \ge \left\|\TT\times_1\bz_1\dots\times_{d-2}\bz_{d-2}\right\|_{p_v}\\
&\ge \left\|\TT\times_1\bz_1\dots\times_{d-2}\bz_{d-2}\right\|_{p_\sigma} \\
&= \max\left\{\left\langle\TT\times_1\bz_1\dots\times_{d-2}\bz_{d-2},\bx_{d-1}\otimes\bx_{d-2} \right\rangle: \|\bx_{d-1}\|_p=\|\bx_{d-2}\|_p=1\right\} \\
&\ge\langle \TT, \bz_1\otimes\dots\otimes\bz_{d-2}\otimes\by_{d-1}\otimes\by_{d}\rangle \\
&\ge \left(\prod_{k=1}^{d-2}\tau_k\right) \|\TT\|_{p_\sigma},
\end{align*}
implying that $u/\delta_G\ge (\prod_{k=1}^{d-2}\tau_k) \|\TT\|_{p_\sigma}/\delta_G$. On the other hand, one also has
\begin{align*}
\frac{u}{\delta_G}& = \max\left\{\frac{1}{\delta_G}\left\|\TT\times_1\bx_1\dots\times_{d-2}\bx_{d-2}\right\|_{p_v}:
\bx_k\in \BH^{n_k},\,k=1,2,\dots, d-2\right\}\\
&\le \max\left\{\left\|\TT\times_1\bx_1\dots\times_{d-2}\bx_{d-2}\right\|_{p_\sigma}:
\bx_k\in \BH^{n_k},\,k=1,2,\dots, d-2\right\} \\
&\le  \|\TT\|_{p_\sigma},
\end{align*}
where the last inequality follows by applying Lemma~\ref{thm:contraction} $d-2$ times.
\end{proof}

By applying deterministic hitting sets $\BH_{2}^n(\alpha,\beta)$ with hitting ratio $\Omega(\sqrt[p]{\ln{n}}/\sqrt{n})$ to Algorithm~\ref{alg:alg0}, the tensor spectral $p$-norm can be approximated within a bound of $\Omega(\prod_{k=1}^{d-2} \sqrt[p]{\ln{n_k}}/\sqrt{n_k})$ and by applying randomized hitting sets $\BH_{3}^n(\epsilon)$ with hitting ratio $\Omega(\sqrt{\ln n/n})$, the approximation bound can be improved to 
$\Omega(\prod_{k=1}^{d-2} \sqrt{\ln{n_k}/n_k})$ when $p\in\BQ\cap(2,\infty)$. Both bounds are exactly the same to the best-known ones by Hou and So~\cite[Theorem~7]{hou2014hardness}. However, Algorithm~\ref{alg:alg0} is much simpler than that in~\cite{hou2014hardness}.

We remark that approximation of the tensor spectral $p$-norm serves a foundation to many $\ell_p$-sphere or $\ell_p$-ball constrained polynomial optimization problems~\cite{he2010approximation,so2011deterministic,hou2014hardness}. Hence, Algorithm~\ref{alg:alg0} provides a new tool to study approximation algorithms of these polynomial optimizations. We leave these to interested readers since our focus is on the tensor nuclear $p$-norm. We now summarize approximation methods of the tensor nuclear $p$-norm in this section in Table~\ref{tab:alg-ratios} before the detailed discussion.
\begin{table}[!ht]
\centering
\caption{Approximation methods of the nuclear $p$-norm of a tensor $\TT\in\R^{n_1\times n_2\dots\times n_d}$}
\label{tab:alg-ratios}
%\resizebox{\textwidth}{!}{%
\begin{tabular}{|lc|ccc|}
\hline
Algorithm &$p$ & Approximation bound    & Approach    & Type        \\ \hline
\cite[Proposition 4.3]{chen2020tensor} & $(1,\infty)$ & $\prod_{k=1}^{d-1} \frac{1}{\sqrt[\leftroot{-2}\uproot{2}q]{n_k}}$                                 &Partition to vectors  & Deterministic  \\
Algorithm~\ref{alg:matricization}  & $\BQ\cap(2,\infty)$& $\frac{1}{\delta_G}\prod_{k=1}^{d-2} \frac{1}{\sqrt[\leftroot{-2}\uproot{2}q]{n_k}}$    &Matrix unfolding   & Deterministic \\
Algorithm~\ref{alg:partition} &  $\BQ\cap(2,\infty)$  & $\frac{1}{\delta_G}\prod_{k=1}^{d-2} \frac{1}{\sqrt[\leftroot{-2}\uproot{2}q]{n_k}}$                                 &Partition to matrices & Deterministic \\
Algorithm~\ref{alg:alg3} with $\BH_{1}^{n_k}$  & $\BQ\cap(2,\infty)$ & $\Omega\left(\prod_{k=1}^{d-2} \sqrt[\leftroot{-2}\uproot{2}q]{\frac{\ln n_k}{n_k}}\right)$                        & $\ell_p$-sphere covering     & Deterministic \\
Algorithm~\ref{alg:alg3} with $\BH_{2}^{n_k}$  & $\BQ\cap(2,\infty)$ & $\Omega\left(\prod_{k=1}^{d-2} \frac{\sqrt[\leftroot{-2}\uproot{2}p]{\ln{n_k}}}{\sqrt{n_k}}\right)$ & $\ell_p$-sphere covering & Deterministic \\
Algorithm~\ref{alg:rand-alg}   & $\BQ\cap(2,\infty)$ & $\Omega\left(\prod_{k=1}^{d-2} \sqrt{\frac{\ln{n_k}}{n_k}}\right)$    & $\ell_p$-sphere covering                                                    & Randomized  \\ \hline
\end{tabular}%
%}
\end{table}
% Before presenting our algorithms, we first give the formal definition of approximation ratio for tensor norm approximation, which is necessary to measure the quality of the approximations.
% \begin{definition}\textnormal{\cite[Definition~3.1]{he2023approx}}
%     We say an algorithm $\operatorname{Alg}$ is a polynomial-time approximation algorithm for the tensor norm $\mathcal{N}(\cdot):\R^{n_1\times n_2\times\dots\times n_d}\rightarrow\R_+$ with approximation ratio of $\tau\in(0,1]$, if for any instance $\TT\in\R^{n_1\times n_2\times\dots\times n_d}$ the algorithm returns a quantity $\operatorname{Alg}(\TT)\in\R_+$ that satisfies $\tau\mathcal{N}(\TT)\le \operatorname{Alg}(\TT) \le\mathcal{N}(\TT)$.
% \end{definition}

\subsection{Deterministic algorithms via tensor unfolding and partition}\label{sec:alg1}

We first introduce the only known approximation bound of the tensor nuclear $p$-norm via vector fibers. It is essentially the follower result by Chen and Li~\cite{chen2020tensor}.
\begin{lemma}[{\cite[Proposition 4.3]{chen2020tensor}}]\label{thm:vectorbound}
For any $\TT\in\R^{n_1\times n_2\times \dots\times n_d}$ and $p\in[1,\infty]$, denote $\bt_{i_1i_2\dots i_{d-1}}\in\R^{n_d}$ to be the mode-$d$ fiber of $\TT$ by fixing the mode-$k$ index to be $i_k$ for $k=1,2,\dots,d-1$. One has
\begin{equation}
  \label{eq:vectorbound}
  %\max\left\{\|\bt_{i_1i_2\dots i_{d-1}}\|_q: 1\le i_k \le n_k,\, k=1,2,\dots,d-1\right\}\le \|\TT\|_{p_\sigma} \le 
  \|\TT\|_p:= \left(\sum_{i_1=1}^{n_1}\sum_{i_2=1}^{n_2}\dots\sum_{i_{d}=1}^{n_{d}}\left|t_{i_1i_2\dots i_{d}}\right|^p\right)^{1/p}
  \le \|\TT\|_{p_*} \le \sum_{i_1=1}^{n_1}\sum_{i_2=1}^{n_2}\dots\sum_{i_{d-1}=1}^{n_{d-1}}\|\bt_{i_1i_2\dots i_{d-1}}\|_p.
\end{equation}
\end{lemma}
It is obvious that the lower and upper bounds of~\eqref{eq:vectorbound} can be computed in polynomial time. Noticing that $$\|\TT\|_p= \left(\sum_{i_1=1}^{n_1}\sum_{i_2=1}^{n_2}\dots\sum_{i_{d-1}=1}^{n_{d-1}}{\|\bt_{i_1i_2\dots i_{d-1}}\|_p}^p\right)^{1/p},
$$
the lower and upper bounds are actually $\ell_p$-norm and $\ell_1$-norm of an $\prod_{k=1}^{d-1}n_k$-dimensional vector, respectively. By the bounds between $\ell_p$-norms (Lemma~\ref{lma:lp-norm-equiv}), $\|\TT\|_p$ provides an approximation bound of $1/\prod_{k=1}^{d-1}\sqrt[q]{n_k}$ for $\|\TT\|_{p_*}$ when $p\in[1,\infty]$. The result is simple but is currently the only possible method when $p\in(1,2)$.

To proceed with better approximation bounds, we have to deal with the matrix nuclear $p$-norms. In fact, Chen and Li~\cite{chen2020tensor} did provide better approximation frameworks of the tensor nuclear $p$-norm via tensor unfoldings and partitions.
\begin{lemma}[{\cite[Theorem 4.7]{chen2020tensor}}]\label{lma:chen2020}
  Let $\TT\in\R^{n_1\times n_2\times \dots\times n_d}$ and $p\in[1,\infty]$. Let $\left\{\BI_1,\BI_2\right\}$ be a partition of $\{1,2,\dots,d\}$ and choose any $i\in\BI_1$ and $j\in\BI_2$. Denote $\Mat(\TT)$ to be the matrix unfolding of $\TT$ by combining modes of $\BI_1$ into the row index and modes of $\BI_2$ into the column index, i.e., a $(\prod_{k\in\BI_1} n_k) \times (\prod_{k\in\BI_2} n_k)$ matrix. Consider the set of matrix slices of $\TT$ obtained by fixing every mode index except modes $i$ and $j$, i.e., a set of $\prod_{1\le k\le d,\,k\neq i,j}n_k$ number of $n_i\times n_j$ matrices and denote $\bt_{p_*}\in\R^{\prod_{1\le k\le d,\,k\neq i,j}n_k}$ to be the vector whose entries are the nuclear $p$-norms of this set of matrix slices. One has
  \begin{equation} \label{eq:matbound}
    \left\|\bt_{p_*}\right\|_p
    \le \|\Mat(\TT)\|_{p_*}
    \le \|\TT\|_{p_*}
    % \le \left\|\bt_{p_*}\right\|_1
    \le \left\| \bt_{p_*} \right\|_p \prod_{1\le k\le d,\,k\neq i,j}{n_k}^{\frac{1}{q}}
    \le \|\Mat(\TT)\|_{p_*} \prod_{1\le k\le d,\,k\neq i,j}{n_k}^{\frac{1}{q}}.
  \end{equation}
\end{lemma}

Although the above bounds are tighter than~\eqref{eq:vectorbound}, they cannot be computed in polynomial time because computing the matrix nuclear $p$-norm is NP-hard. Well, the $\Omega(1)$-approximation bound of the matrix nuclear $p$-norm developed in Section~\ref{sec:matrix-p-norm} enlightens the bounds in~\eqref{eq:matbound}. We now propose two polynomial-time algorithms to approximate the tensor nuclear $p$-norm. The first algorithm is fairly straightforward, i.e., computing $\|\cdot\|_{p_u}$-norm of a proper matrix unfolding of $\TT$.
\begin{algorithm}[!h]
\begin{algorithmic}[1]
\REQUIRE A tensor $\TT\in\R^{n_1 \times n_2 \times \dots \times n_d}$, a constant $p\in\BQ\cap(2,\infty)$ and a partition $\{\BI_1,\BI_2\}$ of $\{1,2,\dots,d\}$ with $d-1\in\BI_1$ and $d\in\BI_2$.
\ENSURE An approximation of $\|\TT\|_{p_*}$.
\STATE Unfold the tensor $\TT$ to $\Mat(\TT)\in\R^{\left(\prod_{k\in\BI_1} n_k\right) \times \left(\prod_{k\in\BI_2} n_k\right)}$ by combining modes of $\BI_1$ into the row index and modes of $\BI_2$ into the column index;
\STATE Compute $\|\Mat(\TT)\|_{p_u}$, i.e., the optimal value of the following SDP
\begin{equation*}\label{opt:d-tensor-matric}
    \begin{array}{ll}
    \max & \langle \Mat(\TT),Z\rangle    \\
    \st   &u_1+ u_2+\theta_p\sum_{i=1}^{\prod_{k\in\BI_1} n_k + \prod_{k\in\BI_2}n_k} t_i\le 1\\
    & (v_i, u_1,t_i)\in\BK^3_p\quad i=1,2,\dots, \prod_{k\in\BI_1} n_k \\%,\,\forall\, i\in[m],
    & (v_i, u_2,t_i)\in\BK^3_p\quad i=\prod_{k\in\BI_1} n_k+1,\prod_{k\in\BI_1} n_k+2,\dots, \prod_{k\in\BI_1} n_k + \prod_{k\in\BI_2}n_k\\ %,\,\forall\, i\in[m+n]\setminus[m].
    & %  u_1\ge0, \,u_2\ge 0, \,   \bt\ge {\bf 0},\, 
    \Diag(\bv)\succeq \begin{pmatrix} O & Z/2 \\Z^\T/2 & O\end{pmatrix};
    \end{array}
    \end{equation*} 
% \begin{equation*}\label{opt:d-tensor-matric}
%     \begin{aligned}
%     & \max
%     & & \langle \Mat(\TT),Z\rangle \\
%     & \st
%     & &  u_1+ u_2+\theta_p\sum_{i=1}^{\prod_{k\in\BI_1} n_k + \prod_{k\in\BI_2} n_k}t_i\le 1, \\ \span &&  u_1\ge 0,\, u_2\ge 0,\\
%     \span &&\Diag(\bv)\succeq\frac{1}{2}\begin{pmatrix}
%     O & Z \\
%     Z^\T & O
%     \end{pmatrix},
%     \\ \span && (v_i,u_1,t_i)\in\BK^3_p \text{ for all }  i\in\left[\prod_{k\in\BI_1} n_k\right],
%     \\ \span && (v_i,u_2,t_i)\in\BK^3_p \text{ for all }  i\in\left[\prod_{k\in\BI_1} n_k + \prod_{k\in\BI_2} n_k\right]\setminus \left[\prod_{k\in\BI_1} n_k\right],
%     \end{aligned}
% \end{equation*}
% and denote the optimal objective function value as $u$;
\RETURN $\|\Mat(\TT)\|_{p_u}$.
\end{algorithmic}
\caption{Approximating the tensor nuclear $p$-norm based on matrix unfolding}
\label{alg:matricization}
\end{algorithm}

We see from~\eqref{eq:matbound} that $\|\Mat(\TT)\|_{p_*}$ is already a better approximation of $\|\TT\|_{p_*}$ with an approximation bound $1/\prod_{1\le k\le d,\,k\neq i,j}\sqrt[q]{n_k}$. By choosing the largest two $n_k$'s, i.e., $n_{d-1}$ and $n_d$ as in Algorithm~\ref{alg:matricization}, the bound attains the best one $1/\prod_{k=1}^{d-2}\sqrt[q]{n_k}$. Combining it with the polynomial-time method to approximate $\|\Mat(\TT)\|_{p_*}$ using $\|\Mat(\TT)\|_{p_u}$ with approximation bound $1/\delta_G$ (Theorem~\ref{thm:KG-matrix-nuclear-pnorm}), Algorithm~\ref{alg:matricization} immediately leads to the following approximation bound of $\|\TT\|_{p_*}$.
\begin{proposition}
  For any $\TT\in\R^{n_1 \times n_2 \times \dots \times n_d}$, $p\in \BQ\cap(2,\infty)$ and partition $\{\BI_1,\BI_2\}$ of $\{1,2,\dots,d\}$ with $d-1\in\BI_1$ and $d\in\BI_2$, Algorithm~\ref{alg:matricization} is a deterministic polynomial-time algorithm whose output $\operatorname{Alg}_{\,\ref{alg:matricization}}(\|\TT\|_{p_*})$ satisfies
$$
\left(\frac{1}{\delta_G}\prod_{k=1}^{d-2} \frac{1}{\sqrt[\leftroot{-2}\uproot{2}q]{n_k}}\right)\|\TT\|_{p_*}\le \operatorname{Alg}_{\,\ref{alg:matricization}}(\|\TT\|_{p_*})\le \|\TT\|_{p_*}.
$$
\end{proposition}

The second approach to approximate $\|\TT\|_{p_*}$ is using the quantity $\|\bt_{p_*}\|_p$ in~\eqref{eq:matbound}. The main effort is to compute entries of $\bt_{p_*}$. Every entry is the nuclear $p$-norm of a matrix and we use $\|\cdot\|_{p_u}$ to approximate it.
\begin{algorithm}[!h]
\begin{algorithmic}[1]
\REQUIRE A tensor $\TT\in\R^{n_1 \times n_2 \times \dots \times n_d}$ and a constant $p\in\BQ\cap(2,\infty)$.
\ENSURE An approximation of $\|\TT\|_{p_*}$.
\STATE Partition $\TT$ by fixing every mode index except modes $d-1$ and $d$ to obtain a $\prod_{k=1}^{d-2}n_k$ number of $n_{d-1}\times n_d$ matrices, namely $\{T_1,T_2,\dots,T_{\prod_{k=1}^{d-2}n_k}\}\subseteq\R^{n_{d-1}\times n_d}$;
\STATE For every $i=1,2,\dots,\prod_{k=1}^{d-2}n_k$, compute $u_i=\|T_i\|_{p_u}$, i.e., the optimal value of the following SDP
    \begin{equation*} \label{opt:d-tensor-partition}
    \begin{array}{ll}
    \max & \langle T_i,Z\rangle    \\
    \st   &u_1+ u_2+\theta_p\sum_{i=1}^{n_{d-1}+n_d} t_i\le 1\\
    & (v_i, u_1,t_i)\in\BK^3_p\quad i=1,2,\dots, n_{d-1} \\
    & (v_i, u_2,t_i)\in\BK^3_p\quad i=n_{d-1}+1,n_{d-1}+2,\dots, n_{d-1} + n_d\\
    & %  u_1\ge0, \,u_2\ge 0, \,   \bt\ge {\bf 0},\, 
    \Diag(\bv)\succeq \begin{pmatrix} O & Z/2 \\Z^\T/2 & O\end{pmatrix};
    \end{array}
    \end{equation*} 
% \begin{equation*}\label{opt:d-tensor-partition}\boxed{
%     \begin{aligned}
%     & \max
%     & & \langle T_i,Z\rangle \\
%     & \st
%     & &  u_1+ u_2+\theta_p\sum_{i=1}^{n_{d-1}+n_d}t_i\le 1, \\ \span &&  u_1\ge 0,\, u_2\ge 0,\\
    % \span &&\Diag(\bv)\succeq\frac{1}{2}\begin{pmatrix}
    % O & Z \\
    % Z^\T & O
    % \end{pmatrix},
%     \\ \span && (v_i,u_1,t_i)\in\BK^3_p \text{ for all }  i\in[n_{d-1}],
%     \\ \span && (v_i,u_2,t_i)\in\BK^3_p \text{ for all }  i\in[n_{d-1}+n_d]\setminus [n_d];
%     \end{aligned}}
% \end{equation*}
\RETURN $\|(u_1,u_2,\dots,u_{\prod_{k=1}^{d-2}n_k})\|_p$.
\end{algorithmic}
\caption{Approximating the tensor nuclear $p$-norm based on matrix partition}
\label{alg:partition}
\end{algorithm}

We have exactly the same theoretical approximation bound to that of Algorithm~\ref{alg:matricization}.
\begin{proposition}
  For any $\TT\in\R^{n_1 \times n_2 \times \dots \times n_d}$ and $p\in \BQ\cap(2,\infty)$, Algorithm~\ref{alg:partition} is a deterministic polynomial-time algorithm whose output $\operatorname{Alg}_{\,\ref{alg:partition}}(\|\TT\|_{p_*})$ satisfies
$$
\left(\frac{1}{\delta_G}\prod_{k=1}^{d-2} \frac{1}{\sqrt[\leftroot{-2}\uproot{2}q]{n_k}}\right)\|\TT\|_{p_*}\le \operatorname{Alg}_{\,\ref{alg:partition}}(\|\TT\|_{p_*})\le \|\TT\|_{p_*}.
$$
\end{proposition}
\begin{proof}
    Denote $\bu=(u_1,u_2,\dots,u_{\prod_{k=1}^{d-2}n_k})^{\T}$. By Theorem~\ref{thm:KG-matrix-nuclear-pnorm}, one has $\|T_i\|_{p_*}/\delta_G\le u_i=\|T\|_{p_u} \le \|T_i\|_{p_*}$ for $i=1,2,\dots,\prod_{k=1}^{d-2}n_k$. This implies that $\|\bt_{p_*}\|_p/\delta_G\le\|\bu\|_p\le\|\bt_{p_*}\|_p$. The result follows immediately from the bounds $\left\|\bt_{p_*}\right\|_p   \le \|\TT\|_{p_*} \le \left\| \bt_{p_*} \right\|_p \prod_{k=1}^{d-2} {n_k}^{1/q}$ in~\eqref{eq:matbound}.
\end{proof}

Both Algorithm~\ref{alg:matricization} and Algorithm~\ref{alg:partition} have the same theoretical performance guarantee but their complexities are different. In particular, Algorithm~\ref{alg:matricization} needs to solve only one SDP with dimension $O(\prod_{k=1}^d n_k)$ while Algorithm~\ref{alg:partition} needs to solve $\prod_{k=1}^{d-2}n_k$ number of SDPs each with dimension $O(n_{d-1} n_d)$. Therefore, Algorithm~\ref{alg:partition} runs faster than Algorithm~\ref{alg:matricization} in general, especially when $d$ is large. 
% Besides, we also remark that, no hitting set has been used in Algorithm~\ref{alg:matricization} and Algorithm~\ref{alg:partition} above.

\subsection{Deterministic approximation via $\ell_p$-sphere covering}\label{sec:alg2}

The approximation bounds of the tensor nuclear $p$-norm obtained in Section~\ref{sec:alg1} cannot be improved via manipulating matrices as they are restricted by the bounds in Lemma~\ref{lma:chen2020}. In order to make improvement so as to match the best approximation bound for the tensor spectral $p$-norm, we now apply the $\ell_p$-sphere covering developed in Section~\ref{sec:hitting-sets}. It also needs to adopt certain reformulation and convex optimization proposed in~\cite{hu2022complexity} with some extra treatments. 

% introduced in~\cite{he2023approx} based on the developments in Section~\ref{sec:matrix-p-norm} and Section~\ref{sec:hitting-sets}.
% For illustration, let us begin with a third-order tensor $\TT\in\R^{n_1\times n_2\times n_3}$. In~\cite[Section~3.2]{he2023approx}, the computation of the tensor nuclear norm was first equivalently reformulated as an SDP but with unaccountably many semidefinite constraints through the following chain of equalities
% $$
% \begin{aligned}
% \|\TT\|_* & =\max \left\{\langle\TT, \mathcal{Z}\rangle:\|\mathcal{Z}\|_\sigma \le 1\right\} \\
% & =\max \left\{\langle\TT, \mathcal{Z}\rangle: \mathcal{Z}(\bx, \by, \bz) \le 1 \text{ for all } \bx\in\BS^{n_1},\,\by\in\BS^{n_2},\,\bz\in\BS^{n_3}\right\} \\
% & =\max \left\{\langle\TT, \mathcal{Z}\rangle: \| \mathcal{Z}(\bx, \bullet, \bullet) \|_\sigma \le 1 \text{ for all } \bx\in\BS^{n_1}\right\} \\
% & =\max \left\{\langle\TT, \mathcal{Z}\rangle: \left[\begin{array}{cc}
% I & \mathcal{Z}(\bx, \bullet, \bullet) \\
% \mathcal{Z}(\bx, \bullet, \bullet)^\T & I
% \end{array}\right] \succeq O \text{ for all } \bx\in\BS^{n_1}\right\},
% \end{aligned}
% $$

% We first derive an SDP relaxation framework for the tensor nuclear $p$-norm that imitates the counterpart for the tensor nuclear norm 
% where the first equality is exactly due to the duality between the two norms, i.e., Lemma~\ref{lma:norm-duality} with $p=2$, the second and third follow from the definition of the tensor spectral norm, i.e., ~\ref{def:spectral} with $p=2$, while the last follows from the fact that $\| \mathcal{Z}(\bx, \bullet, \bullet) \|_\sigma \le 1 $ if and only if $ I \succeq \mathcal{Z}(\bx, \bullet, \bullet) \mathcal{Z}(\bx, \bullet, \bullet)^\T$ and the Schur complement.
% Due to the hardness of enumerating all points on the unit Euclidean sphere, the above problem was then relaxed by using some carefully-constructed and well-behaved hitting sets of $\BS^n$.

% Naturally, to go one step further as for the tensor nuclear norm, we shall prove a higher-order generalization of Proposition~\ref{prop:matrix-equi-pq-vecp}. Notice that all we need to additionally do besides almost identically carrying over the proof of Proposition~\ref{prop:matrix-equi-pq-vecp} to here is to show the compactness of $\bigcap_{\bx\in\BS_p^{n_1}}\left\{\Z:\|\Z\times_1\bx\|_{p_\sigma}\le 1\right\}$, as only equipped with this can we then guarantee that the optimality of (\ref{eq:nuclear-pnorm-pqnorm}) can indeed be attained by some tensor. This is detailed in the following lemma.
% %     \begin{lemma}\label{lma:vec-compact}
% %         The set $\bigcap_{\bx\in\BS_p^{n_1}}\left\{\Z:\|\Z\times_1\bx\|_{p_\sigma}\le 1\right\}$ is compact.
% %     \end{lemma}
% %     \begin{proof}
% %         On one hand, it is easy to see that the set is equal to
% %         $$        \bigcap_{\bx\in\BS_p^{n_1}}\left(\Z\mapsto\|\Z\times_1\bx\|_{p_\sigma}\right)^{-1}\left((-\infty,1]\right),
% %         $$
% %         and thus is clearly closed due to the continuity of the mapping $\Z\mapsto\|\Z\times_1\bx\|_{p_\sigma}$ and the closedness of $(-\infty,1]$. On the other hand, assume the set is not bounded, i.e., there exists a sequence of tensors $\{\Z_k\}_{k\in\BN}$ belonging to the set for which $\lim_{k\rightarrow\infty}\|\Z_k\|_{\textnormal{F}}=\infty$. However, by letting $\bx=\be_i$ for each $i$, we see that each horizontal slice $(\Z_k)^{(1)}_i$ of any $\Z_k$ has matrix spectral $p$-norm no more than $1$, and since in finite-dimensional vector spaces all norms are equivalent, we see that $\|(\Z_k)^{(1)}_i\|_{\textnormal{F}}$ is thus bounded, say by a constant $\beta$. Therefore, by the triangle inequality we obtain that $\|\Z_k\|_{\textnormal{F}}\le\sum_{i=1}^{n_1}\|(\Z_k)^{(1)}_i\|_{\textnormal{F}}\le \beta n_1$ for all $k\in\BN$, a contradiction. The result then follows from the Heine-Borel theorem
% %     \end{proof}
% % With Lemma~\ref{lma:vec-compact} at hand, the following proposition is straightforward.
% \begin{proposition}\label{prop:equi-pq-vecp}
% For $p \in\BQ\cap(2, \infty)$, the optimization problems
% $$
% \max\left\{\langle\TT,\Z\rangle:\|\Z\times_1\bx\|_{p_\sigma}\le 1 \text{ for all } \bx\in\BS_p^{n_1}\right\}=:v_1^*,
% $$
% and
% \begin{equation}\label{eq:nuclear-pnorm-vecp}
% \max\left\{\langle\TT,\Z\rangle:\| \Z\times_1\bx\|_{p_v}\le 1 \text{ for all } \bx\in\BS_p^{n_1}\right\}=:v_2^*,
% \end{equation}
% are equivalent in the sense that $v_2^*\le v_1^*\le \delta_G v_2^*$, where $\delta_G$ is the Grothendieck constant.
% \end{proposition}

To explain the main idea, let us first focus on tensors of order three, i.e., $\TT\in\R^{n_1\times n_2\times n_3}$. By the duality in Lemma~\ref{lma:norm-duality}, one has
\begin{align}\label{eq:nuclear-pnorm-pqnorm}
    \|\TT\|_{p_*}&=\max \left\{\langle\TT, \mathcal{Z}\rangle:\|\mathcal{Z}\|_{p_\sigma} \le 1\right\}\notag
    \\&=\max\left\{\langle\TT,\Z\rangle:\langle\Z,\bx\otimes\by\otimes\bz\rangle\le 1 \text{ for all } \bx\in\BS_p^{n_1},\,\by\in\BS_p^{n_2},\,\bz\in\BS_p^{n_3}\right\}\notag
    \\&=\max\left\{\langle\TT,\Z\rangle:\langle\Z\times_1\bx,\by\otimes\bz\rangle\le 1 \text{ for all } \bx\in\BS_p^{n_1},\,\by\in\BS_p^{n_2},\,\bz\in\BS_p^{n_3}\right\}\notag
    \\&=\max\left\{\langle\TT,\Z\rangle:\|\Z\times_1\bx\|_{p_\sigma}\le 1\text{ for all }\bx\in\BS_p^{n_1}\right\}.
\end{align}
By noticing that $\|A\|_{p_v}/\delta_G  \le \|A\|_{p_\sigma} \le \|A\|_{p_v}$ in Lemma~\ref{prop:matrix-equi-pq-vecp}, we see that 
$$\max\left\{\langle\TT,\Z\rangle:\|\Z\times_1\bx\|_{p_v}\le 1\text{ for all }\bx\in\BS_p^{n_1}\right\}$$
becomes a restriction of~\eqref{eq:nuclear-pnorm-pqnorm} and
$$\max\left\{\langle\TT,\Z\rangle:\|\Z\times_1\bx\|_{p_v}/\delta_G\le 1\text{ for all }\bx\in\BS_p^{n_1}\right\}=\delta_G\max\left\{\langle\TT,\Z\rangle:\|\Z\times_1\bx\|_{p_v}\le 1\text{ for all }\bx\in\BS_p^{n_1}\right\}$$
becomes a relaxation of~\eqref{eq:nuclear-pnorm-pqnorm}. Therefore, we obtain
\begin{equation}\label{eq:equi-pq-vecp}
\|\TT\|_{p_*}/\delta_G
\le \max\left\{\langle\TT,\Z\rangle:\|\Z\times_1\bx\|_{p_v}\le 1 \text{ for all }\bx\in\BS_p^{n_1}\right\} \le \|\TT\|_{p_*}.
%\max\left\{\langle\TT,\Z\rangle:\|\Z\times_1\bx\|_{p_v}\le 1 \text{ for all }\bx\in\BS_p^{n_1}\right\} \le \|\TT\|_{p_*}\le  \delta_G\max\left\{\langle\TT,\Z\rangle:\|\Z\times_1\bx\|_{p_v}\le 1\text{ for all }\bx\in\BS_p^{n_1}\right\}.
\end{equation}
It suffices to focus on $\max\{\langle\TT,\Z\rangle:\|\Z\times_1\bx\|_{p_v}\le 1 \text{ for all }\bx\in\BS_p^{n_1}\}$. In fact, by the duality theory in Corollary~\ref{lma:dual-SDP}, it is easy to verify that
% \begin{equation}\label{eq:nuclear-pnorm-SDP}
%     \begin{array}{ll}
% \max & \langle\TT,\Z\rangle \\
% \st  &  u_1+ u_2+\theta_p\sum_{i=1}^{n_2+n_3} t_i\le 1    \\
% & (v_i, u_1,t_i)\in\BK^3_p\quad i=1,2,\dots,n_2 \\
% & (v_i, u_2,t_i)\in\BK^3_p\quad i=n_2+1,n_2+2,\dots, n_2+n_3\\
% & %  u_1\ge0, \,u_2\ge 0, \,   \bt\ge {\bf 0},\, 
%     \Diag(\bv)\succeq \begin{pmatrix} O & \Z\times_1\bx/2 \\(\Z\times_1\bx)^\T/2 & O\end{pmatrix} \text{ for all }  \bx\in\BS_p^{n_1}.
%     \end{array}
%     \end{equation}
\begin{equation}\label{eq:nuclear-pnorm-SDP}
    \begin{array}{lllll}
\max & \langle\TT,\Z\rangle &= &\max & \langle\TT,\Z\rangle \\
\st  &\|\Z\times_1\bx\|_{p_v}\le 1 \quad \bx\in\BS_p^{n_1} & &\st  &  u_1^{\bx}+ u_2^{\bx}+\theta_p\sum_{i=1}^{n_2+n_3} t_i^{\bx}\le1 \quad  \bx\in\BS_p^{n_1}  \\
&& && (v_i^{\bx}, u_1^{\bx},t_i^{\bx})\in\BK^3_p\quad \bx\in\BS_p^{n_1},\,i=1,2,\dots,n_2 \\
&& && (v_i^{\bx}, u_2^{\bx},t_i^{\bx})\in\BK^3_p\quad \bx\in\BS_p^{n_1},\, i=n_2+1,n_2+2,\dots, n_2+n_3\\
&& && %  u_1^{\bx}\ge0, \,u_2^{\bx}\ge 0, \,   \bt\ge {\bf 0},\, 
    \Diag(\bv^{\bx})\succeq \begin{pmatrix} O & \Z\times_1\bx/2 \\(\Z\times_1\bx)^\T/2 & O\end{pmatrix} \quad \bx\in\BS_p^{n_1}.
    \end{array}
    \end{equation}
% \begin{equation}\label{eq:nuclear-pnorm-SDP}
%     \begin{array}{lllll}
%  \max & \langle\TT,\Z\rangle &= &\max & \langle\TT,\Z\rangle \\
%  \st  &\|\Z\times_1\bx\|_{p_v}\le 1 \text{ for all }\bx\in\BS_p^{n_1} & &\st  &  u_1+ u_2+\theta_p\sum_{i=1}^{n_2+n_3} t_i\le 1    \\
%    && && (v_i, u_1,t_i)\in\BK^3_p\quad i=1,2,\dots,n_2 \\
%    && && (v_i, u_2,t_i)\in\BK^3_p\quad i=n_2+1,n_2+2,\dots, n_2+n_3\\
%    && && %  u_1\ge0, \,u_2\ge 0, \,   \bt\ge {\bf 0},\, 
%     \Diag(\bv)\succeq \begin{pmatrix} O & \Z\times_1\bx/2 \\(\Z\times_1\bx)^\T/2 & O\end{pmatrix} \text{ for all }  \bx\in\BS_p^{n_1}.
%     \end{array}
%     \end{equation}
Everything looks doable except the infinite number of constraints brought by $\bx\in\BS_p^{n_1}$. % Every vector $\bx\in\BS_p^{n_1}$ generates a constraint and this is obviously impossible to be solved. 
Now, the hitting sets developed in Section~\ref{sec:hitting-sets} are helpful. This is essentially the idea in~\cite[Section~3.2]{he2023approx}, relaxing~\eqref{eq:nuclear-pnorm-SDP} by replacing $\BS_p^{n_1}$ with a hitting set that consists of a polynomial number of vectors. We summarize the polynomial-time algorithm to approximate $\|\TT\|_{p_*}$ as well as its theoretical guarantee based on $\ell_p$-sphere covering below.

\begin{algorithm}[!h]
\begin{algorithmic}[1]
\REQUIRE A tensor $\TT\in\R^{n_1\times n_2\times n_3}$, a constant $p\in\BQ\cap(2,\infty)$ and a hitting set $\BH^{n_1}\in\BT^{n_1}_p\left(\tau, O({n_1}^{\alpha_1})\right)$.
\ENSURE An approximation of $\|\TT\|_{p_*}$.
\STATE Solve the following SDP
\begin{equation}\label{opt:three-tensor-sdp}
    \begin{array}{ll}
\max & \langle\TT,\Z\rangle \\
\st  &  u_1^{\bx}+ u_2^{\bx}+\theta_p\sum_{i=1}^{n_2+n_3} t_i^{\bx}\le1 \quad  \bx\in\BH^{n_1}  \\
& (v_i^{\bx}, u_1^{\bx},t_i^{\bx})\in\BK^3_p\quad \bx\in\BH^{n_1},\,i=1,2,\dots,n_2 \\
& (v_i^{\bx}, u_2^{\bx},t_i^{\bx})\in\BK^3_p\quad \bx\in\BH^{n_1},\, i=n_2+1,n_2+2,\dots, n_2+n_3\\
& %  u_1^{\bx}\ge0, \,u_2^{\bx}\ge 0, \,   \bt\ge {\bf 0},\, 
    \Diag(\bv^{\bx})\succeq \begin{pmatrix} O & \Z\times_1\bx/2 \\(\Z\times_1\bx)^\T/2 & O\end{pmatrix} \quad \bx\in\BH^{n_1}.
    \end{array}
%     \begin{array}{ll}
% \max & \langle\TT,\Z\rangle \\
% \st  &  u_1+ u_2+\theta_p\sum_{i=1}^{n_2+n_3} t_i\le 1    \\
% & (v_i, u_1,t_i)\in\BK^3_p\quad i=1,2,\dots,n_2 \\
% & (v_i, u_2,t_i)\in\BK^3_p\quad i=n_2+1,n_2+2,\dots, n_2+n_3\\
% & %  u_1\ge0, \,u_2\ge 0, \,   \bt\ge {\bf 0},\, 
%     \Diag(\bv)\succeq \begin{pmatrix} O & \Z\times_1\bx/2 \\(\Z\times_1\bx)^\T/2 & O\end{pmatrix} \text{ for all }  \bx\in\BH^{n_1};
%     \end{array}
\end{equation}
to obtain its optimal value $u$;
\RETURN $\tau u$.
\end{algorithmic}
\caption{Approximating the tensor nuclear $p$-norm based on $\ell_p$-sphere covering}
\label{alg:alg2}
\end{algorithm}

\begin{theorem}\label{thm:approx-ratio-three} For any $\TT\in\R^{n_1 \times n_2 \times n_3}$, $p\in \BQ\cap(2,\infty)$ and hitting set $\BH^{n_1}\in\BT^{n_1}_p\left(\tau, O({n_1}^{\alpha_1})\right)$,
Algorithm~\ref{alg:alg2} is a deterministic polynomial-time algorithm whose output $\operatorname{Alg}_{\,\ref{alg:alg2}}(\|\TT\|_{p_*})$ satisfies
$$
\frac{\tau}{\delta_G}\|\TT\|_{p_*}\le \operatorname{Alg}_{\,\ref{alg:alg2}}(\|\TT\|_{p_*})\le \|\TT\|_{p_*}.
$$
% approximates $\|\TT\|_{p_*}$ with a worst-case approximation ratio $\frac{\tau}{\delta_G}$.
% $\Omega\left(\sqrt[\leftroot{-2}\uproot{2}q]{\prod_{k=1}^{d-2} \frac{\ln n_k}{n_k}}\right)$
% that returns a scalar $\xi$ with the following property
\end{theorem}
\begin{proof}
  Since the cardinality of $\BH^{n_1}$ is $O({n_1}^{\alpha_1})$, the SDP in~\eqref{opt:three-tensor-sdp} does have a polynomial number of constraints and so the algorithm runs in polynomial time. To show the approximation bounds, let $\Y$ be an optimal solution of~\eqref{opt:three-tensor-sdp}.

    First, the lower bound can be shown by
    $$
    \|\TT\|_{p_*}/ \delta_G\le 
    \max\left\{\langle\TT,\Z\rangle:\|\Z\times_1\bx\|_{p_v}\le 1 \text{ for all } \bx\in\BS_p^{n_1}\right\}
    \le\langle\TT,\Y\rangle=\operatorname{Alg}_{\,\ref{alg:alg2}}(\|\TT\|_{p_*})/\tau,
    $$
    where the first inequality is due to the lower bound of~\eqref{eq:equi-pq-vecp} while the second inequality is due to~\eqref{eq:nuclear-pnorm-SDP} and that~\eqref{opt:three-tensor-sdp} is a relaxation of~\eqref{eq:nuclear-pnorm-SDP}.
    
    Next, as $\BH^{n_1}$ is a $\tau$-hitting set, we have that
    \begin{align*}
    \max_{\bx \in \BH^{n_1},\,\by\in\BS_p^{n_2},\,\bz\in\BS_p^{n_3}} \langle\Y\times_1\bx,\by\otimes\bz\rangle
    &=\max_{\by\in\BS_p^{n_2},\,\bz\in\BS_p^{n_3}} \max_{\bx \in \BH^{n_1}} \left\langle\Y\times_2\by\times_3\bz,\bx\right\rangle\\
    &\ge\max_{\by\in\BS_p^{n_2},\,\bz\in\BS_p^{n_3}} \tau\|\Y\times_2\by\times_3\bz\|_q\\
    &=\tau\max_{\bx\in\BS_p^{n_1},\,\by\in\BS_p^{n_2},\,\bz\in\BS_p^{n_3}} \left\langle\Y\times_2\by\times_3\bz,\bx\right\rangle\\
    &=\tau\|\Y\|_{p_\sigma}.
    \end{align*}
    By the feasibility of $\Y$ to~\eqref{opt:three-tensor-sdp} and Corollary~\ref{lma:dual-SDP}, we have that $\|\Y\times_1\bx\|_{p_v}\le 1$ for all $\bx\in\BH^{n_1}$. Therefore,
    $$
    \|\tau \Y\|_{p_\sigma} \le \max_{\bx \in \BH^{n_1},\by\in\BS_p^{n_2},\bz\in\BS_p^{n_3}} \langle\Y\times_1\bx,\by\otimes\bz\rangle=\max _{\bx \in \BH^{n_1}}\|\Y\times_1\bx\|_{p_\sigma} \le \max _{\bx \in \BH^{n_1}}\|\Y\times_1\bx\|_{p_v} \le 1,
    $$
    implying that $\tau \Y$ is feasible to $\max \left\{\langle\TT, \mathcal{Z}\rangle:\|\mathcal{Z}\|_{p_\sigma} \le 1\right\}=\|\TT\|_{p_*}$. Finally, we have that
    $$
    \|\TT\|_{p_*}=\max \left\{\langle\TT, \mathcal{Z}\rangle:\|\mathcal{Z}\|_{p_\sigma} \le 1\right\} \ge\langle\TT, \tau \Y\rangle=\operatorname{Alg}_{\,\ref{alg:alg2}}(\|\TT\|_{p_*}),
    $$
    implying the upper bound.
\end{proof}

It is straightforward to generalize Algorithm~\ref{alg:alg2} to approximate the nuclear $p$-norm of higher-order tensors; see Algorithm~\ref{alg:alg3}. The theoretical approximation bound is summarized in Theorem~\ref{thm:ratio-higher-order}, whose proof is similar to that of Theorem~\ref{thm:approx-ratio-three} and is thus omitted.
\begin{algorithm}[!h]
\begin{algorithmic}[1]
\REQUIRE A tensor $\TT\in\R^{n_1 \times n_2 \times \dots \times n_d}$, a constant $p\in\BQ\cap(2,\infty)$ and $d-2$ hitting sets $\BH^{n_k}\in\BT^{n_k}_p(\tau_k, O({n_k}^{\alpha_k}))$ for $k=1,2,\dots,d-2$.
\ENSURE An approximation of $\|\TT\|_{p_*}$.
\STATE Solve the following SDP
% \begin{equation*}
%     \begin{array}{ll}
% \max & \langle\TT,\Z\rangle \\
% \st  &  u_1+ u_2+\theta_p\sum_{i=1}^{n_{d-1}+n_d} t_i\le 1    \\
% & (v_i, u_1,t_i)\in\BK^3_p\quad i=1,2,\dots,n_{d-1} \\
% & (v_i, u_2,t_i)\in\BK^3_p\quad i=n_{d-1}+1,n_{d-1}+2,\dots, n_{d-1}+n_d\\
% & %  u_1\ge0, \,u_2\ge 0, \,   \bt\ge {\bf 0},\, 
%     \Diag(\bv)\succeq \frac{1}{2}\begin{pmatrix} O & \Z\times_1\bx_1\dots\times_{d-2}\bx_{d-2} \\(\Z\times_1\bx_1\dots\times_{d-2}\bx_{d-2})^\T & O\end{pmatrix} \text{ for all }  \bx_1\in\BH^{n_1},\dots, \bx_{d-2}\in\BH^{n_{d-2}}
%     \end{array}
% \end{equation*}
\begin{equation*}
    \begin{array}{ll}
\max & \langle\TT,\Z\rangle \\
\st  &  u_1^{\bx}+ u_2^{\bx}+\theta_p\sum_{i=1}^{n_{d-1}+n_d} t_i^{\bx}\le 1   \quad  \bx\in\BH^{n_1}\vee\BH^{n_2}\vee\dots\vee\BH^{n_{d-2}} \\
& (v_i^{\bx}, u_1^{\bx},t_i^{\bx})\in\BK^3_p\quad \bx\in\BH^{n_1}\vee\BH^{n_2}\vee\dots\vee\BH^{n_{d-2}},\, i=1,2,\dots,n_{d-1} \\
& (v_i^{\bx}, u_2^{\bx},t_i^{\bx})\in\BK^3_p\quad \bx\in\BH^{n_1}\vee\BH^{n_2}\vee\dots\vee\BH^{n_{d-2}},\, i=n_{d-1}+1,n_{d-1}+2,\dots, n_{d-1}+n_d\\
& %  u_1^{\bx}\ge0, \,u_2^{\bx}\ge 0, \,   \bt\ge {\bf 0},\, 
    \Diag(\bv)\succeq \frac{1}{2}\begin{pmatrix} O & \Z\times_1\bx_1\dots\times_{d-2}\bx_{d-2} \\(\Z\times_1\bx_1\dots\times_{d-2}\bx_{d-2})^\T & O\end{pmatrix} \quad  \bx\in\BH^{n_1}\vee\BH^{n_2}\vee\dots\vee\BH^{n_{d-2}}\\
&\bx=\bx_1\vee\bx_2\vee\dots\vee\bx_{d-2}
    \end{array}
\end{equation*}
to obtain its optimal value $u$;
\RETURN $u\prod_{k=1}^{d-2} \tau_k$.
\end{algorithmic}
\caption{Approximating the tensor nuclear $p$-norm based on $\ell_p$-sphere covering}
\label{alg:alg3}
\end{algorithm}

\begin{theorem}\label{thm:ratio-higher-order}
For any $\TT\in\R^{n_1 \times n_2 \dots \times n_d}$, $p\in \BQ\cap(2,\infty)$ and hitting sets $\BH^{n_k}\in\BT^{n_k}_p(\tau_k, O({n_k}^{\alpha_k}))$ for $k=1,2,\dots,d-2$,
Algorithm~\ref{alg:alg3} is a deterministic polynomial-time algorithm whose output $\operatorname{Alg}_{\,\ref{alg:alg3}}(\|\TT\|_{p_*})$ satisfies
$$
\left(\frac{1}{\delta_G}\prod_{k=1}^{d-2}\tau_k\right)\|\TT\|_{p_*}\le \operatorname{Alg}_{\,\ref{alg:alg3}}(\|\TT\|_{p_*}) \le \|\TT\|_{p_*}.
$$
\end{theorem}

As a remark, the cardinality of any hitting set in Algorithm~\ref{alg:alg3} must be a polynomial function of the problem dimension in order to ensure the algorithm running in polynomial time. By directly applying Theorem~\ref{thm:ratio-higher-order} with the hitting sets $\BH_{1}^n(\alpha,\beta)$ in Section~\ref{sec:worst-hitting} and  $\BH_{2}^n(\alpha,\beta)$ in Section~\ref{sec:brieden-hitting-set}, we have the following improved approximation bounds for the tensor nuclear $p$-norm.

\begin{corollary}\label{cor:nonbridge-gap-determ}
By choosing $\BH^{n_k}=\BH_{1}^{n_k}(\alpha,\beta)$ for $k=1,2,\dots,d-2$ in Algorithm~\ref{alg:alg3}, we have
$$
\Omega\left(\prod_{k=1}^{d-2} \sqrt[\leftroot{-2}\uproot{2}q]{\frac{\ln n_k}{n_k}}\right)\|\TT\|_{p_*}\le \operatorname{Alg}_{\,\ref{alg:alg3}}(\|\TT\|_{p_*}) \le \|\TT\|_{p_*}
$$
for any $\TT\in\R^{n_1 \times n_2 \times\dots \times n_d}$ and $p\in \BQ\cap(2,\infty)$.
\end{corollary}

\begin{corollary}\label{cor:bridge-gap-determ}
By choosing $\BH^{n_k}=\BH_{2}^{n_k}(\alpha,\beta)$ for $k=1,2,\dots,d-2$ in Algorithm~\ref{alg:alg3}, we have
$$
\Omega\left(\prod_{k=1}^{d-2} \frac{\sqrt[\leftroot{-2}\uproot{2}p]{\ln{n_k}}}{\sqrt{n_k}}\right)\|\TT\|_{p_*}\le \operatorname{Alg}_{\,\ref{alg:alg3}}(\|\TT\|_{p_*}) \le \|\TT\|_{p_*}
$$
for any $\TT\in\R^{n_1 \times n_2 \times\dots \times n_d}$ and $p\in \BQ\cap(2,\infty)$.
\end{corollary}

In terms of the approximation bounds, Corollary~\ref{cor:bridge-gap-determ} improves Corollary~\ref{cor:nonbridge-gap-determ} while the latter improves the bounds by manipulating matrices in Section~\ref{sec:alg1}. %All these algorithms run in polynomial time deterministically. 
Well, the most significant result is that the best bound, $\Omega(\prod_{k=1}^{d-2} \sqrt[p]{\ln{n_k}}/\sqrt{n_k})$ in Corollary~\ref{cor:bridge-gap-determ}, matches the best-known approximation bound of its dual norm (the tensor spectral $p$-norm)~\cite[Theorem~7]{hou2014hardness} by a deterministic polynomial-time algorithm. 
% Besides, it is clear that the above ratios $\Omega\left(\prod_{k=1}^{d-2} \sqrt[\leftroot{-2}\uproot{2}q]{\frac{\ln n_k}{n_k}}\right)$ and $\Omega\left(\prod_{k=1}^{d-2} \frac{\sqrt[\leftroot{-2}\uproot{2}p]{\ln{n_k}}}{\sqrt{n_k}}\right)$ substantially improve over the (common) one obtained in Section~\ref{sec:alg1}, i.e., $\Omega\left(\prod_{k=1}^{d-2} \sqrt[\leftroot{-2}\uproot{2}q]{\frac{1}{n_k}}\right)$.

\subsection{Randomized approximations via $\ell_p$-sphere covering}\label{sec:alg3}

As shown in Section~\ref{sec:rand-hitting-set}, with the help of randomization, the randomized hitting set $\BH_3^n(\epsilon)$ enjoys the best possible hitting ratio. This can also be applied to Algorithm~\ref{alg:alg3} to improve the approximation bound of the tensor nuclear $p$-norm. For every $\ell_p$-sphere, one can sample a sufficiently large number of vectors on $\BS_p^n$ to construct a hitting set with high probability by Theorem~\ref{thm:rand-hitting-set}. Once all $d-2$ hitting sets are constructed successfully, they can be applied to Algorithm~\ref{alg:alg3} directly. However, we need to take a little bit care of the overall probability. As $d-2$ hitting sets need to be input for Algorithm~\ref{alg:alg3} independently, if the success rate for every randomized hitting set is set as $1-\epsilon$, then the overall success rate would be $(1-\epsilon)^{d-2}\ge1-\epsilon(d-2)$. Therefore, in order to make the overall success rate to be $1-\epsilon$, it suffices to set the success rate of each hitting set to be $1-\frac{\epsilon}{d-2}$. Here are the randomized algorithm and its theoretical guarantee. 

\begin{algorithm}[!h]
\begin{algorithmic}[1]
\REQUIRE A tensor $\TT\in\R^{n_1\times n_2\times \dots\times n_d}$, a constant $p\in\BQ\cap(2,\infty)$ and a tolerance $\epsilon\in(0,1)$.
\ENSURE An approximation of $\|\TT\|_{p_*}$ with probability at least $1-\epsilon$.
\STATE For every $k=1,2,\dots,d-2$, generate $\lceil\delta_3{n_k}^{\delta_2}( (\frac{1}{2}+\frac{1}{q}){n_k} \ln {n_k}+\ln \frac{d-2}{\epsilon})\rceil$ samples of random vectors independently and evenly on $\BS^{n_k}_p$ to form a hitting set $\BH_3^{n_k}(\frac{d-2}{\epsilon})$.
% \STATE Invoke Algorithm~\ref{alg:alg3} with input $\left(\TT,\{\BH_{3}^{n_i}(\epsilon)\}_{i=1}^{d-2}\right)$, and denote the returned value as $\lambda$;
\RETURN $\operatorname{Alg}_{\,\ref{alg:alg3}}(\|\TT\|_{p_*})$ with $\BH^{n_k}=\BH_3^{n_k}(\frac{d-2}{\epsilon})$ for $k=1,2,\dots,d-2$.
\end{algorithmic}
\caption{Approximating the tensor nuclear $p$-norm based on randomized $\ell_p$-sphere covering}
\label{alg:rand-alg}
\end{algorithm}

\begin{theorem}
    For any $\TT\in\R^{n_1 \times n_2 \dots \times n_d}$, $p\in\BQ\cap(2,\infty)$ and $\epsilon\in(0,1)$, Algorithm~\ref{alg:rand-alg} is a randomized polynomial-time algorithm whose output $\operatorname{Alg}_{\,\ref{alg:rand-alg}}(\|\TT\|_{p_*})$ satisfies
    $$
    \Omega\left(\prod_{k=1}^{d-2} \sqrt{\frac{\ln{n_k}}{n_k}}\right)\|\TT\|_{p_*} \le \operatorname{Alg}_{\,\ref{alg:rand-alg}}(\|\TT\|_{p_*})
    \text{ and }
    \Prob\left\{\operatorname{Alg}_{\,\ref{alg:rand-alg}}(\|\TT\|_{p_*})\le\|\TT\|_{p_*}\right\}\ge 1-\epsilon.
    $$
\end{theorem}

The approximation bound, $\Omega(\prod_{k=1}^{d-2} \sqrt{\ln{n_k}/n_k})$, matches the best-known approximation bound for the tensor spectral $p$-norm by a randomized algorithm~\cite[Theorem~8]{hou2014hardness}. Together with the approximation bound in Corollary~\ref{cor:bridge-gap-determ}, the approximation bounds of the tensor nuclear $p$-norm derived in this paper, exactly match that of the tensor spectral $p$-norm, no matter by deterministic polynomial-time algorithms or randomized ones.


\section{Concluding remarks}\label{sec:conclusion}

We study approximation algorithms of the tensor nuclear $p$-norm with an aim to establish the approximation bound matching the best one of its dual norm, the tensor spectral $p$-norm. Driven by the application of sphere covering to approximate tensor spectral and nuclear norms, we propose several types of hitting sets that approximately represent $\ell_p$-sphere with varying cardinalities and covering ratios, providing an independent toolbox for decision making on $\ell_p$-spheres. Using the idea in robust optimization and second-order cone programming, we obtain the first polynomial-time algorithm with an $\Omega(1)$-approximation bound for the computation of the matrix nuclear $p$-norm when $p\in\BQ\cap(2,\infty)$, paving a way for applications in modeling with the matrix nuclear $p$-norm. These two new results enable us to propose various polynomial-time algorithms for the computation of the tensor nuclear $p$-norm with the best approximation bound being the same to the best one of the tensor spectral $p$-norm, no matter for deterministic algorithms or randomized ones. In the meantime, our study opens up a challenge problem on how to explicitly construct a deterministic hitting set of $\BS^n_p$ with covering ratio $\Omega(\sqrt{\ln{n}/{n}})$ when $p\in(2,\infty)$.

% The tensor nuclear $p$-norm is an important research object in tensor computation and analysis, and it has found successful and effective applications in some machine learning problems. Due to its NP-hard computational nature, the study on its polynomial-time approximation is necessary. However, to the best of our knowledge, this topic is still very under-explored, and the best-known polynomial-time approximation ratio for the tensor nuclear $p$-norm is still much inferior to that for its dual norm, the tensor spectral $p$-norm, which shows room for improvement. In this paper, we design deterministic and randomized polynomial-time approximation algorithms for the tensor nuclear $p$-norm based on two key techniques introduced below, with more importance and emphasis associated to the former, to bridge this gap. The first technique is a comprehensive and careful treatment to the $p\rightarrow q$ norm involved in the dual formulation of the tensor nuclear $p$-norm and a series of ensuing problems whose ultimate goal is to derive a constant-factor semidefinite program relaxation of the tensor nuclear $p$-norm but with uncountably many semidefinite constraints, and techniques in robust optimization as well as those in modern convex optimization developed by Ben-Tal and Nemirovski are utilized therein. The second technique concerns the uncountability issue appeared above which is actually caused by the uncountable cardinality of the unit $\ell_p$-sphere and studies the $\ell_p$-sphere covering problem, namely the design of hitting sets of the unit $\ell_p$-sphere (i.e., a set of representative points on the sphere, roughly speaking), 
%  As main products, we construct deterministic and randomized polynomial-sized hitting sets of the unit $\ell_p$-sphere with their hitting ratios having the capability to help bridge the aforementioned gap.
 % % Specifically, we first equivalently rewrite the constraints of the dual formulation of the tensor nuclear $p$-norm to be uncountable $p\rightarrow q$ norm inequalities. Afterwards, we utilize techniques developed in robust optimization to transform the reformulated approximation problem into a semidefinite program (SDP) but with uncountably many semidefinite constraints. Furthermore, with the help of existing results on polytope approximation and probability bounds for $\ell_p$-sphere vector sampling, we explicitly give or construct deterministic and randomized hitting sets to make the above SDP with only polynomially-many semidefinite constraints, which becomes polynomial-time solvable, with some sacrifice embodied by the approximation ratios of the resulting algorithms. 
%  As a result of the above developments, deterministic and randomized polynomial-time approximation algorithms for the tensor nuclear $p$-norm can then be naturally derived, and we further show their approximation ratios match the best-known ones of the tensor spectral $p$-norm, its dual norm, as desired.
% $\ell_p$-spherically-constrained 
% polynomial optimization (and as a special case, the tensor spectral $p$-norm), as desired. 
% We believe our methodology (resp. results) shall deliver useful and far-reaching insights (resp. have further applications) for other related problems.


% By equivalently reducing the problem to a matrix spectral $p$-norm-constrained linear maximization problem where (for a third-order tensor) each constraint is associated to a point on the unit $\ell_p$-sphere, we were then left to derive a constant-factor SDP relaxation but with uncountably many semidefinite constraints for it while addressing the uncountability issue, where the former was resolved by first replacing the matrix spectral $p$-norm constraints by $\operatorname{vec}_p(\cdot)$ constraints in the above problem which resolves the NP-hardness induced by the matrix spectral $p$-norm and only yields a constant-factor relaxation but is not in any catalogue characterization yet and then utilizing techniques in robust optimization~\cite{goldfarb2003robust,huang2021projection} as well as those in modern convex optimization developed by Ben-Tal and Nemirovski~\cite[Section~3]{ben2001lectures} to carefully study the dual formulation and strong duality of $\operatorname{vec}_p(\cdot)$ and the SDP-representability of its dual as well as the relaxed problem, while the latter was tackled by constructing and utilizing $O\left(\operatorname{poly}(n)\right)$-sized hitting sets of $\BS_p^n$ with powerful hitting ratios. 

The idea of covering and its applications may lead to some interesting future work. One important case that was not discussed in the paper is $p=\infty$ which corresponds to the so-called $\infty \mapsto 1$ norm~\cite{he2013approximation} of a tensor and directly relates to binary constrained polynomial optimization. It is still not known how to construct deterministic hitting set to cover $\{-1,1\}^n$ with polynomial cardinality and best possible covering ratio. A positive answer could result a wide range of applications such as graph theory, neural networks, error-correcting codes; see~\cite[Section~1]{he2013approximation} and the references therein. 

A generalization of the spectral and nuclear $p$-norms is the spectral and nuclear $\bp$-norms where $\bp\in\R^d$ and $\ell_p$-spheres are replaced by $\ell_{p_k}$-spheres for different $p_k$'s; see~\cite{lim2005singular,li2020norm}. Although the constructed hitting sets of $\BS_p^n$ can still be used, the approximation of the matrix $(p_1,p_2)$-spectral norms remains mostly unknown except for some special $(p_1,p_2)$, the same to the matrix $(p_1,p_2)$-nuclear norms. If they can be approximated within a bound of $\Omega(1)$, then a deterministic $\Omega(\prod_{k=1}^{d-2} \sqrt[p_k]{\ln{n_k}}/\sqrt{n_k})$-approximation bound and a randomized $\Omega(\prod_{k=1}^{d-2} \sqrt{\ln{n_k}/n_k})$-approximation bound can be achieved.

Covering a subset of a sphere instead of the whole one may be of particular interest. One immediate case that has wider applications is the nonnegative sphere, $\BS^n_2\cap\R^n_+$, which only occupies $1/2^n$ of the whole sphere. The best possible hitting ratio with polynomial cardinality remains unknown and should be better than $\Omega(\sqrt{\ln{n}/{n}})$. Some probabilistic evidence hints that the best possible hitting ratio might be even $\Omega(1)$, somewhat surprising. A theoretical justification would have a profound influence to optimization problems such as copositive programming and nonnegative matrix factorization. 


% Appendix here
% Options are (1) APPENDIX (with or without general title) or
%             (2) APPENDICES (if it has more than one unrelated sections)
% Outcomment the appropriate case if necessary
%
% \begin{APPENDIX}{<Title of the Appendix>}
% \end{APPENDIX}
%


% \begin{APPENDICES}
% \section{Another scheme to construct $O\left(\operatorname{exp}(n)\right)$-sized $\Omega(1)$-hitting sets of $\BS_p^n$.}\label{sec:grid-point}
% The paper~\cite[Lemma~3.7]{brieden2001deterministic} has already suggested a scheme to construct $O\left(\operatorname{exp}(n)\right)$-sized $\Omega(1)$-hitting sets of $\BS_p^n$ by sampling grid points on $\BS_p^n$, which is detailed in Algorithm~\ref{alg:lp-fptas}.


% \begin{algorithm}[!h]
% \begin{algorithmic}[1]
% \REQUIRE A number of dimension $n$, a parameter $\gamma>1$, and a constant $p\in(1,\infty)$.
% \ENSURE A hitting set of $\BS_p^n$ with exponential cardinality and constant hitting ratio.
% \STATE Generate a set of vectors
% $$
% \BY^n:=\mathbb{Z}^n \bigcap\left(\gamma n^{1 / p} \mathbb{B}_{p}^n \setminus\{\bd{0}_n\}\right);
% $$
% \RETURN $\BC^n(\gamma):=\left\{\frac{\bx}{\|\bx\|_p}\in\BS_p^n:\bx\in\BY^n\right\}$.
% \end{algorithmic}
% \caption{Constructing another exponential-sized hitting set of $\BS_p^n$ with constant hitting ratio}
% \label{alg:lp-fptas}
% \end{algorithm}

% Regarding Algorithm~\ref{alg:lp-fptas}, the following theoretical guarantee has been established in~\cite[Lemma~3.7]{brieden2001deterministic}.
% \begin{lemma}[{\cite[Lemma~3.7]{brieden2001deterministic}}]\label{lma:lp-grid}
% It holds that, the output of Algorithm~\ref{alg:lp-fptas}
% $$
% \BC^n(\gamma)\in\BT_p^n\left(1-\frac{1}{\gamma}, \left((2\gamma+1)\max\left\{1,\sqrt{\frac{2\pi}{p}}\right\}e^{p/12}\right)^n\right).
% $$
%     % $$
%     % \min _{\|\bx\|_q=1} \max _{\bu \in \BC^n( \gamma)} \bu^{\T} \bx \ge 1-\frac{1}{\gamma}.
%     % $$
% \end{lemma}

% For the completeness of the paper and a better understanding of the whole scheme, we restate the proof of the lemma in our language and notations again, as follows.
% \begin{proof}
%     For any $\bx\in\BS_q^n$, we define a vector $\bu$ of the same dimension whose $i$th element is given by
%     $$
%     u_i=
%     \begin{cases}
%         \lfloor\gamma n^{1/p}|x_i|^{q-1}\rfloor & \text{ if } x_i\ge 0,\\
%         \lceil-\gamma n^{1/p}|x_i|^{q-1}\rceil & \text{ otherwise.}
%     \end{cases}
%     $$
%     Then, clearly, $\bu\in\mathbb{Z}^n$, and
%     $$
%     \|\bu\|_p\le\sqrt[\leftroot{-2}\uproot{2}p]{\sum_{i=1}^n \gamma^p n |x_i|^{(q-1)p}}=\sqrt[\leftroot{-2}\uproot{2}p]{\sum_{i=1}^n \gamma^p n |x_i|^{q}}=\gamma n^{1/p},
%     $$
%     which means that $\bu\in\gamma n^{1 / p} \mathbb{B}_{p}^n $. Next, we claim that $\bu\neq\bd{0}_n$. Assume this is not the case, then we have that $\gamma n^{1/p}|x_i|^{q-1}< 1$ for each $i$ by the definition of $\bu$, which, combined with the assumption that $\bx\in\BS_q^n$, implies that
%     $$
%     \sum_{i=1}^n\left(\gamma n^{1/p}|x_i|^{q-1}\right)^p< n, \text{ i.e., } \gamma<1,
%     $$
%     a contradiction.
%     Thence, we conclude that $\bu\in\gamma n^{1 / p} \mathbb{B}_{p}^n \setminus \{\bd{0}_n\}$ actually, and thus $\frac{\bu}{\|\bu\|_p}\in\BC^n(\gamma)$. We next show that $\left\langle\frac{\bu}{\|\bu\|_p},\bx\right\rangle\ge 1-\frac{1}{\gamma}$. Indeed, it is computed that,
%     \begin{align*}
%         \bu^\T\bx=\sum_{i=1}^n u_i x_i\ge\sum_{i=1}^n\left(\gamma n^{1/p}|x_i|^{q-1}-1\right)|x_i|=\gamma n^{1/p}-\|\bx\|_1\ge\gamma n^{1/p}- n^{(q-1)/q}\|\bx\|_q=(\gamma-1)n^{1/p},
%     \end{align*}
%     where the last inequality follows from Lemma~\ref{lma:lp-norm-equiv}. Therefore, we have that
%     $$
%     \left\langle\frac{\bu}{\|\bu\|_p},\bx\right\rangle\ge\frac{(\gamma-1)n^{1/p}}{\gamma n^{1/p}}=\frac{\gamma-1}{\gamma},
%     $$
%     as desired. All it remains is to estimate the cardinality of $\BC^n(\gamma)$, or equivalently, $\BY^n$. To begin with, for all $\bu\in\BY^n\subseteq\mathbb{Z}^n$, we associate it with a unit cube $\left\{\by\in\R^n:\|\by-\bu\|_\infty\le\frac{1}{2}\right\}$. Since $\BY^n$ is a subset of the integer lattice, we have that for any $\bu_i,\bu_j\in\BY^n$ with $\bu_i\ne\bu_j$, $\left\{\by\in\R^n:\|\by-\bu_i\|_\infty\le\frac{1}{2}\right\}\allowbreak\bigcap\allowbreak\left\{\by\in\R^n:\|\by-\bu_j\|_\infty\le\frac{1}{2}\right\}=\varnothing$, and therefore the cardinality of $\BY^n$ is, in digit, equal to the volume of $\bigcup_{\bu\in\BY^n}\left\{\by\in\R^n:\|\by-\bu\|_\infty\le\frac{1}{2}\right\}$. Simple computations show that, for any $\bu\in\BY^n$ and $\by\in\left\{\by\in\R^n:\|\by-\bu\|_\infty\le\frac{1}{2}\right\}$, we have that
%     $$
%     \|\by\|_p=\|\by-\bu+\bu\|_p\le\|\bu\|_p+\|\by-\bu\|_p\le\gamma n^{1/p}+n^{1/p}\|\by-\bu\|_\infty\le \left(\gamma+\frac{1}{2}\right) n^{1/p},
%     $$
%     where in the second inequality we have used again Lemma~\ref{lma:lp-norm-equiv}, and therefore
%     $$\
%     \bigcup_{\bu\in\BY^n}\left\{\by\in\R^n:\|\by-\bu\|_\infty\le\frac{1}{2}\right\}\subseteq \left(\gamma+\frac{1}{2}\right) n^{1/p}\mathbb{B}_p^n.
%     $$
%     It turns out from the above inclusion that, $\operatorname{vol}\left(\left(\gamma+\frac{1}{2}\right) n^{1/p}\mathbb{B}_p^n\right)$ provides a proper upper bound for the cardinality of $\BY^n$. However, the precise computation of $\operatorname{vol}\left(\left(\gamma+\frac{1}{2}\right) n^{1/p}\mathbb{B}_p^n\right)$ can be rather hard. Therefore, we next derive a tight estimation for $\operatorname{vol}\left(\left(\gamma+\frac{1}{2}\right) n^{1/p}\mathbb{B}_p^n\right)$ with the help of Stirling's approximation. It is well known that the volume of an $n$-dimensional unit $\ell_p$-ball is $2^n \frac{\left(\Gamma\left(1+\frac{1}{p}\right)\right)^n}{\Gamma\left(1+\frac{n}{p}\right)}$, therefore
%     $$
%     \operatorname{vol}\left(\left(\gamma+\frac{1}{2}\right) n^{1/p}\mathbb{B}_p^n\right)=\left(\gamma+\frac{1}{2}\right)^n n^{n/p }2^n \frac{\left(\Gamma\left(1+\frac{1}{p}\right)\right)^n}{\Gamma\left(1+\frac{n}{p}\right)}.
%     $$
%     By a specific version of Stirling's approximation, i.e.,
%     $$
%     \sqrt{2 \pi} y^{y+1 / 2} e^{-y}<\Gamma(1+y)<\sqrt{2 \pi} y^{y+1 / 2} e^{-y+1 /12 y},
%     $$
%     for any $y>0$, we have that
%     $$
%     \Gamma\left(1+\frac{n}{p}\right)>\sqrt{2 \pi}\left(\frac{n}{p}\right)^{n / p+1 / 2} e^{-n / p},
%     $$
%     as well as
%     $$
%     \left(\Gamma\left(1+\frac{1}{p}\right)\right)^n<(2 \pi)^{n / 2}\left(\frac{1}{p}\right)^{n / p + n / 2} e^{-n / p+n p / 12},
%     $$
%     which together imply that
%     $$
%     \operatorname{vol}\left(\left(\gamma+\frac{1}{2}\right) n^{1/p}\mathbb{B}_p^n\right)<(2\gamma+1)^n\frac{\left(\sqrt{\frac{2\pi}{p}}\right)^{n-1}\left(e^{p/12}\right)^n}{n^{1/2}}\le\left((2\gamma+1)\max\left\{1,\sqrt{\frac{2\pi}{p}}\right\}e^{p/12}\right)^n,
%     $$
%     and this completes the proof.
% \end{proof}

% \section{Existing results on polytope approximation in~\cite{brieden2001deterministic}.}\label{sec:brieden-lemmas}
% In this section, we list existing results on polytope approximation introduced in~\cite{brieden2001deterministic}, which were previously used in Section~\ref{sec:brieden-hitting-set} to construct $O\left(\operatorname{poly}(n)\right)$-sized $\Omega\left(\frac{\sqrt[\leftroot{-2}\uproot{2}p]{\ln{n}}}{\sqrt{n}}\right)$-hitting sets of $\BS_p^n$, i.e., $\BH_{2}^n(\alpha,\beta)$, for the completeness of the paper. Instead of directly copying their results here, we rearrange and rephrase these results in our notations and language again for offering better clarity, understandability, and more insights, and also make some conducive simplifications so as to better illustrate their main ideas and reveal their underlying nature. We also remark here that, although the statements of these results in the original paper~\cite{brieden2001deterministic} are existential, after a closer inspection, they are in fact constructive.

% \begin{lemma}[{\cite[Lemma~3.12]{brieden2001deterministic}}]\label{lma:lma3.12}
%     Let $k\in\BN\cup\{0\}$ be a constant, and matrices $I_{m,k}\in\R^{2^k n\times 2^k n}$ be defined recursively as
% $$
% I_{m,0}=I,\text{ and }I_{m, k+1}=\begin{pmatrix}I_{m,k} & I_{m,k} \\ I_{m,k} & -I_{m,k}\end{pmatrix}.
% $$
% Then for any $\gamma>1$, the polytope $\BK^{2^k n}:=\bigcap_{\bu\in 2^{-\frac{k}{p}}I_{m,k}\left(\BE^{2^k} \boxtimes \BC^n(\gamma)\right)}\{\bx\in\R^{2^k n}:\bu^\T\bx\le 1\}$ satisfies $\mathbb{B}_q^{2^k n}\subseteq \BK^{2^k n}\subseteq \left(\frac{\gamma}{\gamma-1}\right)2^{k/2} n^{1/q - 1/2}\mathbb{B}_q^{2^k n}$.
% \end{lemma}

% \begin{lemma}[{\cite[Lemma~3.13]{brieden2001deterministic}}]\label{lma:lma3.13}
%     Given a number of dimension $n\ge 2$, let $m=:\lceil \ln{n}\rceil$, write $n=qm+r$ by the quotient-remainder theorem where $q\in\BN$ and $r\in\{0,1,\dots,m-1\}$, and let $l:=\lfloor \log_2 \left(\frac{n}{m}+1\right)\rfloor$. Then, $q$ can be written as $\sum_{k=0}^{s} b_k 2^k$ where $ b_k\in\{0,1\}$ and thus $n$ can be written as $\sum_{k=0}^{s} b_k 2^k m + r$, and we have that the polytope
%     \begin{align*}
%     \BK^{2^l+1}:=\left(\bigtimes_{k\in\left\{i\in\{0\}\cup[s]: b_i=1\right\}}\bigcap_{\bu\in 2^{-\frac{k}{p}}I_{2^k, m}\left(\BE^{2^k} \boxtimes \BC^m(\gamma)\right)}\{\bx\in\R^{2^k m}:\bu^\T\bx\le 1\}\right)\bigtimes\hspace{-0.5em}\bigcap_{\bu\in \BC^r(\gamma)}\hspace{-0.5em}\{\bx\in\R^r:\bu^\T\bx\le 1\},
%     \end{align*}
%     satisfies $\mathbb{B}_q^n\subseteq\BK^{2^l+1}\subseteq \frac{\gamma}{\gamma-1}\left(\frac{2^{q/2}}{2^{q/2}-1}\right)^{\frac{1}{q}}\frac{(n+\ln{n}+1)^{1/2}}{(\ln{n})^{1/p}}\mathbb{B}_q^n$.
% \end{lemma}
% \end{APPENDICES}


\section*{Acknowledgments.} This research is partially supported by the National Natural Science Foundation of China (Grants 72192832, 72171141, 72150001, 71771141, 71825003, and 11831002) and by the Program for Innovative Research Team of Shanghai University of Finance and Economics.


% \begin{thebibliography}{99}\itemsep0pt

% \bibitem{YH97} R. K. Rao Yarlagadda , John E. Hershey, 
% Hadamard Matrix Analysis and Synthesis: With Applications to Communications and Signal/Image Processing

% \bibitem{syl1867} J. J. Sylvester. Thoughts on inverse orthogonal matrices, simultaneous sign-successions, and tessellated pavements in two or more colours, with applications to Newton's rule, ornamental tile work, and the theory of numbers. Philos. Magazine, 34 (1867), 461-475.

% \bibitem{N00} Nesterov, Yu. 2000. Global Quadratic Optimization via Conic Relaxation. H. Wolkowicz, R. Saigal, L. Vandenberghe, eds., Handbook of Semidefinite Programming: Theory, Algorithms, and Applications, International Series in Operations Research and Management Science, vol. 27. Kluwer Academic Publishers, Boston, Massachusetts, 363–387

% \bibitem{BN01} Ben-Tal, A., A. Nemirovski. 2001. On Approximating Matrix Norms. Manuscript.

% \bibitem{S05} Steinberg, D. 2005. Computation of Matrix Norms with Applications to Robust Optimization. Master’s thesis, Technion—Israel Institute of Technology, Technion City, Haifa 32000, Israel.

% \bibitem{BJVS07} A. Barmpoutis, B. Jian, B. C. Vemuri, and T. M. Shepherd, Symmetric Positive 4th Order Tensors \& Their Estimation from Diffusion Weighted MRI, Proceedings of the 20th International Conference on Information Processing in Medical Imaging, 308-319, 2007. 

% \bibitem{BBP98} 
% Baratchart, L., M. Berthod, L. Pottier. 1998. Optimization of Positive Generalized Polynomials under $l^p$ Constraints. J. Convex Anal. 5(2) 353-379.

% \bibitem{RFP10}
% Guaranteed Minimum-Rank Solutions of Linear Matrix Equations via Nuclear Norm Minimization

% \bibitem{YZ16} 
% M. Yuan and C.-H. Zhang, {\em On tensor completion via nuclear norm minimization}, Foundations of Computational Mathematics, 16, 1031--1068, 2016.

% \bibitem{Y00}
% Y.-X. Yuan, A review of trust region algorithms for optimization, ICIAM 99 (Edinburgh), Oxford Univ. Press, Oxford, 2000, pp. 271-282. MR1824450

% \bibitem{R00}
% B.Reznick, Some Concrete Aspects of Hilbert’s 17th Problem, Real Algebraic Geometry and Ordered Structures, Contemporary Mathematics, Volume 253, American Mathematical Society, Providence (2000)

% \bibitem{HJL22} H Hu, B Jiang, Z Li, Complexity and computation for the spectral norm and nuclear norm of order three tensors with one fixed dimension, arXiv:2212.14775

% \bibitem{HDQ09}
% Han D, Dai HH, Qi L (2009) Conditions for strong ellipticity of anisotropic elastic materials. J Elast 97(1):1-13

% \bibitem{WV10}
% Weiland S, van Belzen F (2010) Singular value decompositions and low rank approximations of tensors. IEEE Trans Signal Process 58(3):1171-1182

% \bibitem{FVJ09}
% Fletcher PT, Venkatasubramanian S, Joshi S (2009) The geometric median on Riemannian manifolds with application to robust atlas estimation. NeuroImage 45:S143-S152

% \bibitem{DLMO07}
% Dahl G, Leinaas JM, Myrheim J, Ovrum E (2007) A tensor product matrix approximation problem in quantum physics. Linear Algebra Appl 420(2-3):711-725

% \bibitem{Q05}
% Eigenvalues of a real supersymmetric tensor, Liqun Qi

% \bibitem{KB09}
% Tensor decompositions and applications.

% \bibitem{N03}
% Nesterov, Yu.: Random walk in a simplex and quadratic optimization over convex polytopes. CORE Discussion Paper. UCL, Louvain-la-Neuve, Belgium (2003)

% \bibitem{zhou12}
% Zhou, G., L. Caccetta, K. L. Teo, and S. Y. Wu. 2012. "Nonnegative polynomial optimization over unit spheres and convex programming relaxations." SIAM Journal on Optimization 22: 987-1008.

% \bibitem{zhangqy12}
% X. Zhang, L. Qi, and Y. Ye, The cubic spherical optimization problems, Mathematics of Computation, 81, 279, 1513-1525, 2012.

% \bibitem{SY20} Hadamard Matrices: Constructions using Number Theory and Algebra
% Author(s):Jennifer Seberry, Mieko Yamada
% First published:6 August 2020

% \bibitem{R58} C. A. Rogers, The packing of equal spheres, Proceedings of the London Mathematical Society, 713 s3-8, 609–620, 1958

% \bibitem{libook12}
% Z. Li, S. He, and S. Zhang, Approximation Methods for Polynomial Optimization: Models, Algorithms, and Applications, Springer, New York, 2012.

% \bibitem{SY92} Seberry, J. R. and Yamada, M.. Hadamard matrices, sequences, and block designs. Contemporary Design Theory-A Collection of Surveys (North Holland, New York, 1992), 431–560

% \bibitem{Grothendieck} Braverman, M., K. Makarychev, Y. Makarychev, A. Naor. 2011. The Grothendieck Constant is Strictly Smaller than Krivine's Bound. Proc. 52nd Annual IEEE Sympos. Foundations Comput. Sci. (FOCS 2011). 453–462.

% \bibitem{HJL22} H. Hu, B. Jiang, and Z. Li, {\em Complexity and computation for the spectral norm and nuclear norm of order three tensors with one fixed dimension}, arXiv:2212.14775, 2022.

% \end{thebibliography}


\bibliographystyle{abbrv}
\bibliography{references}{}

\end{document}
