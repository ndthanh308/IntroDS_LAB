\documentclass[%
%reprint,
%superscriptaddress,
%groupedaddress,
%unsortedaddress,
%runinaddress,
%frontmatterverbose, 
 preprint,
%preprintnumbers,
%nofootinbib,
%nobibnotes,
%bibnotes,
 amsmath,
 amssymb,
 aps, 
%prl,
%pra,
%prb,
%rmp,
%prstab,
%prstper,
%floatfix,
]{revtex4-2}

% \usepackage{CJK}
\usepackage{graphicx}
\usepackage{caption}
% \usepackage{subcaption}
\usepackage{xcolor}

\usepackage{amsmath}
\DeclareFontFamily{U}{mathb}{}
\DeclareFontShape{U}{mathb}{m}{n}{
  <-5.5> mathb5
  <5.5-6.5> mathb6
  <6.5-7.5> mathb7
  <7.5-8.5> mathb8
  <8.5-9.5> mathb9
  <9.5-11.5> mathb10
  <11.5-> mathb12
}{}
\DeclareSymbolFont{mathb}{U}{mathb}{m}{n}
\DeclareMathSymbol{\ulsh}{3}{mathb}{"E8}
\DeclareMathSymbol{\ursh}{3}{mathb}{"E9}
\DeclareMathSymbol{\dlsh}{3}{mathb}{"EA}
\DeclareMathSymbol{\drsh}{3}{mathb}{"EB}

\newcommand{\com}[1]{{\color{red}#1}}

% \usepackage{newtxmath}
\bibliographystyle{apsrev4-2}

\begin{document}


Dear editor and referees,\\

The manuscript “Near-wall depletion and layering affect contact line friction of multicomponent liquids” has been revised according to the review comments we have received. We would like to thank the referees for expressing very pertinent critique points. We hope you will find the manuscript improved.

We list the major and the minor changes to the manuscript below.\\

Major changes:
\begin{itemize}
    \item In section IV.A, we have expanded the paragraph that discusses the calculation of surface tension to include a comparison between molecular dynamics simulations and experimental results. Furthermore, we have compared the results for the SPC/E water model (originally presented in the manuscript) with new ones for the TIP4P/2005 model.
    \item Section V.B has been thoroughly reviewed. Parts of the text have been rephrased in order to clarify the rationale of the correction procedure. New references have been added. Additionally, a table containing the parameters used to evaluate equation 12 has been included. We have added a new appendix, ``Appendix D: Interfacial friction'', to provide supplementary details on the scaling correction.
    \item A new section, titled ``V.C. Comparison with experimental studies'', has been added to the manuscript. In this section, we offer our interpretation of the discrepancies found in both experimental studies and between experiments and molecular dynamics simulations, in regards to the estimation of contact line friction.
\end{itemize}
Minor changes:
\begin{itemize}
    \item In the introduction, some of the articles mentioned in the comments of and in the reply to referee \#1 are now cited.
    \item The point made by referee \#2 in regards to aqueous glycerol density has been clarified in section IV.A.
    \item The point made by referee \#2 in regards to the fit shown in figure 3 has been clarified in section IV.B and in the caption of figure 3.
    \item The typo mentioned by referee \#1 has been corrected.
    \item In appendix A, the point made by referee \#2 in regards to the solid substrate has been clarified.
    \item Also in appendix A, a mention to the effects of the cutoff distance of the Lennard-Jones potential has been included.
    \item References [50] to [55] have been fixed and now include the links to the Zenodo datasets, as requested by referee \#2.
    \item Some of the results (figures and tables) attached to the replies to the referees have been added to the supplementary information.
\end{itemize}

In the revised manuscript, the parts reviewed in relation to the comments of referee \#1 are marked in {\color{red}red}, while the ones marked in {\color{blue}blue} refers to the comments of referee \#2.\\

Kind regards,\\

Michele Pellegrino

Berk Hess

\end{document}