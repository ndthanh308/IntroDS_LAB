% mnras_template.tex 
%
% LaTeX template for creating an MNRAS paper
%
% v3.0 released 14 May 2015
% (version numbers match those of mnras.cls)
%
% Copyright (C) Royal Astronomical Society 2015
% Authors:
% Keith T. Smith (Royal Astronomical Society)

% Change log
%
% v3.0 May 2015
%    Renamed to match the new package name
%    Version number matches mnras.cls
%    A few minor tweaks to wording
% v1.0 September 2013
%    Beta testing only - never publicly released
%    First version: a simple (ish) template for creating an MNRAS paper

%%%%%%%%%%%%%%%%%%%%%%%%%%%%%%%%%%%%%%%%%%%%%%%%%%
% Basic setup. Most papers should leave these options alone.
\documentclass[fleqn,usenatbib]{mnras}
\usepackage{float}
% MNRAS is set in Times font. If you don't have this installed (most LaTeX
% installations will be fine) or prefer the old Computer Modern fonts, comment
% out the following line
\usepackage{newtxtext,newtxmath}
% Depending on your LaTeX fonts installation, you might get better results with one of these:
%\usepackage{mathptmx}
%\usepackage{txfonts}
\usepackage{afterpage}
\usepackage{placeins}
% Use vector fonts, so it zooms properly in on-screen viewing software
% Don't change these lines unless you know what you are doing
\usepackage[T1]{fontenc}
\usepackage{xcolor}
% Allow "Thomas van Noord" and "Simon de Laguarde" and alike to be sorted by "N" and "L" etc. in the bibliography.
% Write the name in the bibliography as "\VAN{Noord}{Van}{van} Noord, Thomas"
\DeclareRobustCommand{\VAN}[3]{#2}
\let\VANthebibliography\thebibliography
\def\thebibliography{\DeclareRobustCommand{\VAN}[3]{##3}\VANthebibliography}


\usepackage{orcidlink}
%%%%% AUTHORS - PLACE YOUR OWN PACKAGES HERE %%%%%

% Only include extra packages if you really need them. Common packages are:
\usepackage{graphicx}	% Including figure files
\usepackage{amsmath}	% Advanced maths commands
% \usepackage{amssymb}	% Extra maths symbols
\newcommand{\red}[1]{\textcolor{red}{#1}}

%%%%%%%%%%%%%%%%%%%%%%%%%%%%%%%%%%%%%%%%%%%%%%%%%%

%%%%% AUTHORS - PLACE YOUR OWN COMMANDS HERE %%%%%

% Please keep new commands to a minimum, and use \newcommand not \def to avoid
% overwriting existing commands. Example:
%\newcommand{\pcm}{\,cm$^{-2}$}	% per cm-squared

%%%%%%%%%%%%%%%%%%%%%%%%%%%%%%%%%%%%%%%%%%%%%%%%%%

%%%%%%%%%%%%%%%%%%% TITLE PAGE %%%%%%%%%%%%%%%%%%%

% Title of the paper, and the short title which is used in the headers.
% Keep the title short and informative.
%\title[QPOs in MAXI J1816--195]{Timing and Spectral Analysis of  the Newly Discovered Millisecond Pulsar MAXI J1816--195}

\title[ The strong \textasciitilde2.5 Hz modulation in MAXI J1816--195]{Detection of a strong \textasciitilde2.5 Hz modulation in the Newly Discovered Millisecond Pulsar MAXI J1816--195}

% The list of authors, and the short list Twhich is used in the headers.
% If you need two or more lines of authors, add an extra line using \newauthor

\author[Panping Li et al.]{
P. P. Li,$^{1,2}$ L. Tao,$^{1}$\thanks{E-mail: taolian@ihep.ac.cn} L. Zhang,$^{1}$ Q. C. Bu~\orcidlink{0000-0001-5238-3988},$^{3}$ J. L. Qu,$^{1}$ L. Ji~\orcidlink{0000-0001-9599-7285},$^{4}$ P. J. Wang,$^{1,2}$ Y. P. Chen~\orcidlink{0000-0001-8768-3294},$^{1}$  S. Zhang,$^{1}$
\newauthor{R. C. Ma,$^{1, 2}$ Z. X. Yang,$^{1, 2}$ W. T. Ye,$^{1, 2}$ S. J. Zhao,$^{1, 2}$ Q. C. Zhao,$^{1, 2}$ Y. Huang,$^{1}$ X. Ma,$^{1}$  E. L. Qiao,$^{5}$ } 
\newauthor{S. M. Jia,$^{1}$ S. N. Zhang$^{1}$}
\\
% List of institutions
$^{1}$Key Laboratory of Particle Astrophysics, Institute of High Energy Physics, Chinese Academy of Sciences, 100049 Beijing, China\\
$^{2}$Uinversity of Chinese Academy of Sciences, Chinese Academy of Sciences, 100049 Beijing, China\\
$^{3}$Institut f\"ur Astronomie und Astrophysik, Kepler Center for Astro and Particle Physics, Eberhard Karls Universit\"at, Sand 1, D-72076 T\"ubingen, Germany\\
$^{4}$School of Physics and Astronomy, Sun Yat-sen University, Zhuhai, 519082, People’s Republic of China\\
$^{5}$National Astronomical Observatories, Chinese Academy of Sciences, Beijing 100101, China}



% These dates will be filled out by the publisher
\date{Accepted XXX. Received YYY; in original form ZZZ}

% Enter the current year, for the copyright statements etc.
\pubyear{2015}

% Don't change these lines
\begin{document}
\label{firstpage}
\pagerange{\pageref{firstpage}--\pageref{lastpage}}
\maketitle

% Abstract of the paper
\begin{abstract}
MAXI J1816--195 is a newly discovered accreting millisecond X-ray pulsar that went outburst in June 2022. Through timing analysis with \textit{NICER} and \textit{NuSTAR} observations, we find a transient modulation at \textasciitilde2.5 Hz during the decay period of MAXI J1816--195. The modulation is strongly correlated with a spectral hardening, and its fractional rms amplitude increases with energy. These results suggest that the modulation is likely to be produced in an unstable corona. In addition, the presence of the modulation during thermonuclear bursts indicates that it may originate from a disk-corona where the optical depth is likely the main factor affecting the modulation, rather than temperature. Moreover, we find significant reflection features in the spectra observed simultaneously by \textit{NICER} and \textit{NuSTAR}, including a relativistically broadened Fe-K line around 6--7\,keV, and a Compton hump in the 10--30\,keV energy band. The radius of the inner disc is constrained to be $R_{\rm in}$ = (1.04--1.23)\,$R_{\rm ISCO}$ based on reflection modeling of the broadband spectra. Assuming that the inner disc is truncated at the magnetosphere radius, we estimate that the magnetic field strength is $\leq 4.67 \times 10^{8}\,\rm G$. 
\end{abstract}

% Select between one and six entries from the list of approved keywords.
% Don't make up new ones.
\begin{keywords}
 accretion, accretion disk --  X-rays: binaries -- X-rays: individual (MAXI J1816–195)
\end{keywords}

%%%%%%%%%%%%%%%%%%%%%%%%%%%%%%%%%%%%%%%%%%%%%%%%%%

%%%%%%%%%%%%%%%%% BODY OF PAPER %%%%%%%%%%%%%%%%%%

\section{Introduction}

 Low-mass X-ray binary systems (LMXBs) consist of a compact object, which is either a black hole (BH) or a neutron star (NS), and a companion star (mass $< 1\ M_{\odot}$) which emits X-ray radiation resulting from the release of gravitational potential energy from the fall of accreting material via Roche-lobe \citep[]{1975van}. A NS-LMXB is usually accompanied by an old neutron star with a weak magnetic field. Thus, the behavior of NS-LMXBs is similar to that of BH-LMXBs in terms of variability and state transition. However, the presence of rigid surfaces and natural magnetic fields in NS-LMXBs, as well as smaller gravitational forces, distinguish them from BH-LMXBs (see \cite{2013Matsuoka} and references therein).

Accreting millisecond X-ray pulsars (AMXPs) are a special subclass of NS-LMXBs with spin frequencies of several hundred hertz (see reviews by \cite{2020Di}). Since the discovery of the first accretion millisecond pulsar, SAX J1808.4--3658, by \textit{RXTE} in 1998 \citep[]{1998Wijnands}, a total of 24 accretion millisecond pulsars have been discovered \citep[]{2022Marino}, with an average of one discovery per year. Most recently, two accretion millisecond pulsars, MAXI J1816--195 \citep[]{2022Negoro} and MAXI J1957+032 \citep[]{2022ATelNg}, were discovered in June 2022. All AMXPs are transients, usually with X-ray outbursts recurring every 2--4 yrs. The outburst duration is typically a couple of weeks, reaching luminosities of $10^{36 - 37}\ \rm ergs\ \rm s^{-1}$ in the X-ray band \citep[]{2020Kuiper}. As with some other NS-LMXBs, the spectra of AMXPs in outbursts generally include a thermal component emitted by an accretion disc, the NS surface, or a boundary layer, and a high-energy nonthermal component produced by the hot corona of tens of keV via the scattering of soft photons. The majority of AMXPs also contain a reflection component of the hard Comptonized photons illuminating the cold accretion disk, consisting of several emission lines and a broad hump of 10--30\,keV \citep{2013Papitto,2016Pintore,2022Marino}. The Fe-K line is the most prominent feature in the reflection spectrum, which in turn provides constraints on the structure of the inner disc and inclination \citep[]{1989Fabian}.


In addition, the X-ray emission from AMXPs also shows rich variability. These various timing features in the power-density spectra (PDS) of AMXPs are commonly present as broad-band noise continuum and sharp peaks called quasi-periodic oscillations (QPOs). The study of QPOs or similar modulations reveals a rich phenomenology which has helped us to understand the accretion physics of compact objects \citep[e.g.][]{2021Belloni}. The kHz QPO has been found in many millisecond pulsars, such as SAX J1808.4--3658 \citep[]{2003Wijnands}, XTE J1807--294 \citep[]{2005Linares}, IGR J17511--305 \citep[]{2011Kalamkar} and Aql X--1 \citep[]{2008Barret}. Sometimes two kHz QPOs are observed simultaneously and the frequency difference between the two QPOs ($\Delta v$) is related to the NS spin frequency $v_{\rm spin}$, i.e. $\Delta v=v_{\rm spin}$ \citep[]{2005Linares} or $\Delta v=v_{\rm spin}/2$ \citep[]{2011Kalamkar}, which is consistent with the predictions of the relativistic precession model \cite{1999Stella} and  spin-resonance model \cite{2003Lamb}.

%The study of QPOs in power density spectra (PDS) of LMXBs reveals a rich phenomenology which has helped us to understand the physics of accretion disks and compact objects \citep[]{2021Belloni}. The kHz QPO has been tested in multiple sources from the 24-millisecond pulsars discovered so far, e.g. SAX J1808.4-3658 \citep[]{2003Wijnands}, XTE J1807-294 \citep[]{2005Linares}, IGR J17511-305 \citep[]{2011Kalamkar} and Aql X-1 \citep[]{2008Barret}. Generally, two kHz QPOs are observed simultaneously and the kHz QPO separation frequency $\Delta v$ is related to neutron-star spin frequency $v_{\rm spin}$, e.g. $\Delta v=v_{\rm spin}$ \citep[]{2005Linares} or $\Delta v=v_{\rm spin}/2$ \citep[]{2011Kalamkar}, which is consistent with some predictions of the sonic-point spin resonance and relativistic-resonance models \citep[]{2021Belloni}. 


Compared with kHz QPOs, the detection of QPOs or modulations at low frequencies in AMXPs is relatively scarce. The 1\,Hz modulations of SAX J1808.4--3658 \citep{patruno2009} and NGC 6440 X--2 \citep{patruno2013} are widely known, which are interpreted as variability being caused by disk availability when the inner edge of the accretion disk is close to the co-rotation radius \citep{patruno2013}. In addition, the mHz QPOs are found in several sources, such as the \textasciitilde 6--7\,mHz QPO in Aql X--1  \citep[]{2001Revnivtsev} and the \textasciitilde 8\,mHz QPO in IGR J00291+5934 \citep[]{2017Ferrigno}. The power spectra of these sources are more complicated, especially at low frequencies. According to the characteristic frequencies of different components in the power spectra and the correlations between these characteristic frequencies, these components are generally identified as break, low-frequency QPO, harmonic of the low-frequency QPO, hump, hectoHz QPO, lower kHz QPO, or upper kHz QPO \citep[]{2005Straaten}. Moreover, similar to atoll sources, the correlations between the characteristic frequencies for AMXPs usually follow the WK \citep[]{1999Wijnands} and PBK \citep[]{1999Psaltis} relations, but being scaled by a constant factor when comparing with the atoll sources \citep[]{2005Linares,2015Bult,2017Doesburgh}. 
%The complexity of the power spectra may account for the fewer studies on the low-frequency QPO of AMXPs.


%Compared with kHz QPOs, the detection of QPOs at low frequencies in AMXPs is relatively scarce. The 1\,Hz flares of SAX J1808.4-3658 and NGC 6440 X-2 are widely known, and \citet{patruno2009} and \citet{patruno2013} interpreted that the variability was caused by disk availability when the inner edge of the accretion disk is close to the corotation radius. There are also some mHz QPOs found, such as the \textasciitilde 6-7\,mHz QPO of Aql X-1  \citep[]{2001Revnivtsev} and the \textasciitilde 8\,mHz QPO of IGR J00291+5934 \citep[]{2017Ferrigno}. The power spectral components of these sources are more complex, especially at low frequencies. According to the location of components in the power spectrum and the correlations between characteristic frequencies of power spectral components, features in power spectral  are generally identified as break, low-frequency QPO, harmonic of the low-frequency QPO, hump, hectoHz, lower kHz QPO, or upper kHz QPO \citep[]{2005Straaten}. These power spectral components of AMXPs, like other atoll sources, follow the WK and PBK relations, although AMXPs tend to be shifted in characteristic frequencies of the power spectral components by a constant factor \citep[]{2005Linares,2015Bult,2017Doesburgh}. This is why there are fewer separate studies on the low-frequency QPO of AMXPs.

The low-frequency behaviors, especially QPOs or modulations, are important probes to understand the accretion variations in the outer regions and large timescales. Their studies will provide a comprehensive understanding of the physical processes of accretion in AMXPs, on the basis of high-frequency behaviors. However, as mentioned above, the low-frequency QPOs or modulations are only observed in a few AMXPs, thus it is very important to obtain new observational samples and perform in-depth studies. 

MAXI J1816--195 is a new accreting millisecond X-ray pulsar discovered by \textit{MAXI}/GSC on June 07, 2022 \citep[]{2022Negoro}. Some X-ray satellites perform follow-up observations. With \textit{NICER} observations, a pulsation of 528\,Hz \citep[]{2022BultA} and an orbital period of 17402.27\,s are found \citep[]{2022BultB}. Moreover, \textit{NuSTAR} observed this source on June 23, 2022 with a total exposure time of 40\,ks \citep[]{2022Chauhan}. By analyzing this \textit{NuSTAR} observation and the simultaneous \textit{NICER} observation, \cite{2022Mandal} studied the evolution of the type-I burst profile with flux and energy. \textit{Insight}-HXMT also observed this source during the outburst \citep[]{2022Li}, and interestingly, 73 thermonuclear X-ray bursts have been detected. \cite{2022Chen} selected 66 bursts with similar burst profiles and intensities, and found significant hard X-ray shortages during the bursts due to a burst cooling effect. Moreover, \cite{2023Li} reported the pulse profile and the spin evolution using the \textit{Insight}-HXMT observations, revealing the detection of X-ray pulsations with energies up to $\sim$95\,keV, and found that the pulse profiles were characterized by a truncated Fourier series with two harmonics.

%A broad iron line of reflection was discovered in the joint Insight-\textit{HXMT}/\textit{NICER} spectra of persistent emission. There are no reports of MAXI J1816--195 persistent spectral time analysis, such as the search for quasi-periodic oscillations (QPO), mainly exploring this source thermonuclear X-ray bursts \citep[]{2022Mandal,2022Chen,2022Bult}. 

%\cite{2022Mandal} combined with \textit{NICER}'s simultaneous detection to analyze the evolution of burst profile with flux and energy. Insight-\textit{HXMT} also intensively detected MAXI J1816--195 until its quiescent state \citep[]{2022Li}, where 73 thermonuclear X-ray bursts have been detected. Sixty-six of the bursts with similar contours and intensity were analyzed by \cite{2022Chen} and found significant hard X-ray shortages during the burst, which was interpreted as a burst cooling corona. A broad iron line of reflection was discovered in the joint Insight-\textit{HXMT}/\textit{NICER} spectra of persistent emission. There are no reports of MAXI J1816--195 persistent spectral time analysis, such as the search for quasi-periodic oscillations (QPO), mainly exploring this source thermonuclear X-ray bursts \citep[]{2022Mandal,2022Chen,2022Bult}. 


In this paper, we will study the timing and spectral properties of MAXI J1816--195 using data from \textit{NuSTAR} and \textit{NICER} observations, and focus on the behaviors of the \textasciitilde2.5\,Hz modulation. The paper is structured as follows. Observation and data reduction are presented in Section~\ref{sec:2}, and the results are summarised in Section~\ref{sec:3}. The discussion and conclusions are presented in Section~\ref{sec:4}.



%In this paper, we study the timing and spectral properties of MAXI J1816--195 using data from \textit{NuSTAR} and \textit{NICER} observatories. We generated power spectra for 26 \textit{NICER}'s ObsIDs, covering the peak and the decay phase of the outburst, and searched for potential oscillation signals in 9 of them, which were also detected in \textit{NuSTAR}. We further analyzed the strength of this QPO and the rms and lag spectra. In addition, we found a Fe K emission lines of 6-7 kev and Compton hump in the broadband spectra of \textit{NICER} and \textit{NuSTAR}, which can be well described by the self-consistent relativistic reflection mode. The structure of this article is as follows: Section~\ref{sec:2} presents the observation and data reduction. Section~\ref{sec:3} summarises the results of timing and spectral analysis. The discussion and conclusions are presented in Section~\ref{sec:4}.



% Figure environment removed



\section{Observations and Data Reduction}
\label{sec:2}
%Normally the next section describes the techniques the authors used.
%It is frequently split into subsections, such as Section~\ref{sec:maths} below.




%by here
\subsection{\textit{NICER}}
\label{sec:2.1} % used for referring to this section from elsewhere

The Neutron star Interior Composition Explorer (\textit{NICER}) is an instrument onboard the International Space Station (ISS) designed to study neutron stars in the 0.2--12 keV X-ray band \citep[]{2016SPIE}. \textit{NICER}’s public observations of MAXI J1816–195 from June 8 (ObsID 5202820102) to July 1 (ObsID 5533012301), 2022, are used in this work (Table~\ref{tab:obsid_1}). These data are processed with {\tt HEASoft v 6.30.1} and the \textit{NICER} Data Analysis Software ({\tt NICERDAS v10}) with Calibration Database ({\tt CALDB vXTI20210707}). All standard calibration and screening criteria are applied to produce clean event files from the {\tt nicerl2} tool. We use the task {\tt barycorr} to apply barycentric corrections for each clean event file. Based on the timing windows of 100 s before and after the burst peak, the event files during the burst and the non-burst (i.e. persistent emission) are extracted separately.


We choose to analyze the spectrum of ObsID 5533011601, which is synchronized with \textit{NuSTAR} (ObsID 90801315001). The total persistent spectra and background spectra are generated by the tool {\tt nibackgen3C50} \citep[]{2022Remillard}. The ancillary response files (ARFs; detectors 14 and 34 are removed) and response matrix files (RMFs) are extracted respectively by using the {\tt nicerarf} and {\tt nicermf}. Next, we use the {\tt ftgrouppha} \citep[]{2016Kaastra} command to rebin the spectrum with groupscale to 30. Spectral data are analyzed using {\tt XSPEC v12.12.1}, the system error is set to 1\%, and the energy range is limited to 0.5--10.0 keV, which was recommended by \textit{NICER} Calibration Recommendations \footnote{\url{https://heasarc.gsfc.nasa.gov/docs/nicer/analysis_threads/cal-recommend/}}.

%Simple mathematics can be inserted into the flow of the text e.g. $2\times3=6$
%or $v=220$\,km\,s$^{-1}$, but more complicated expressions should be entered
%as a numbered equation:

%\begin{equation}
 %   x=\frac{-b\pm\sqrt{b^2-4ac}}{2a}.
%	\label{eq:quadratic}
%\end{equation}

%Refer back to them as e.g. equation~(\ref{eq:quadratic}).



\subsection{\textit{NuSTAR}}
\label{sec:2.2}

Between June 23 and 24, 2022, the Nuclear Spectroscopic Telescope Array (\textit{NuSTAR}) observed MAXI J1816--195 (ObsID 90801315001) with a total exposure of 40\,ks. \textit{NuSTAR}’s focal plane consists of two independent solid-state detectors (FPMA and FPMB) operating in 3--79\,kev \citep[]{2013Harrison}. The data from the \textit{NuSTAR} detectors (FPMA and FPMB) are calibrated and screened to get clean events using the {\tt nupipeline} routine of \textit{NuSTAR} data analysis software ({\tt NustarDAS v2.1.2}) with {\tt CALDB v20220706}, which is distributed with {\tt HEASoft v6.30.1}. Since the burst peak count rate exceeds 100 counts s$^{-1}$, the keyword STATUEXPR is set to be "STATUS==b0000xxx00xx0000". We use {\tt DS9} to produce separately the source (a $210^{\prime \prime}$ circular region) and background (a $110^{\prime \prime}$ circular region) extraction region files, and then apply the {\tt nuproducts} command for extracting the source spectra, light-curve, and instrument responses. The barycenter corrected cleaned event file is got by setting the parameter "write\_baryevtfile=yes" and "barycorr=yes". We also alter GTI files to separate the burst and persistent X-ray emission from MAXI J1816--195. The persistent emission is rebinned with a minimum of 50 counts per energy bin and limited to the 3--40\,keV energy range for the poor signal-to-noise ratio at high energies.



% Figure environment removed


% Figure environment removed

% Figure environment removed


%Figures and tables should be placed at logical positions in the text. Don't
%worry about the exact layout, which will be handled by the publishers.

%Figures are referred to as e.g. Fig.~\ref{fig:example_figure}, and tables as
%e.g. Table~\ref{tab:example_table}.



% Example table
%\begin{table}
%	\centering
%	\caption{This is an example table. Captions appear above each table.
%	Remember to define the quantities, symbols and units used.}
%	\label{tab:example_table}
%	\begin{tabular}{lccr} % four columns, alignment for each
%		\hline
%		A & B & C & D\\
%		\hline
%		1 & 2 & 3 & 4\\
%		2 & 4 & 6 & 8\\
%		3 & 5 & 7 & 9\\
%		\hline
%	\end{tabular}
%\end{table}

\begin{table}
    \centering
    \caption{\textit{NICER} and \textit{NuSTAR} Observations of MAXI J1816--195 used in this paper. The observations with the \textasciitilde2.5\,Hz modulation are highlighted in bold text. Simultaneous observations by \textit{NICER} and \textit{NuSTAR} are marked in blue.}
	\label{tab:obsid_1}
    \begin{tabular}{@{}ccccc@{}}
    \hline
        Instrument & ObsIDs & Start time & Exposure time & Number of \\&&(MJD) & (s)& type-I bursts \\ \hline
        \textit{NICER} & 5202820102 & 59738.32 & 1210 & 0 \\ 
        \textit{NICER} & 5533010101 & 59738.77 & 2325 & 1 \\ 
        NICER & 5533010102 & 59739.35 & 3127 & 1 \\ 
        \textit{NICER} & 5533010103 & 59740.02 & 9177 & 2 \\ 
        \textit{NICER} & 5533010104 & 59741.61 & 2201 & 1 \\ 
        \textit{NICER} & 5533010105 & 59742.25 & 3869 & 1 \\ 
        \textit{NICER} & 5533010106 & 59743.09 & 10797 & 1 \\ 
        \textit{NICER} & 5533010107 & 59744.01 & 3441 & 0 \\ 
        \textit{NICER} & 5533010801 & 59746.14 & 4987 & 0 \\ 
        \textbf{\textit{NICER}} &  \textbf{5533010901} & \ \textbf{59747.16} &  \textbf{5099} &  \textbf{3} \\ 
        \textit{NICER} & 5533010108 & 59747.82 & 675 & 0 \\ 
        \ \textbf{\textit{NICER}} &  \textbf{5533010902} &  \textbf{59748.01} &  \textbf{1443} &  \textbf{0} \\ 
         \textbf{\textit{NICER}} &  \textbf{5533011001} &  \textbf{59748.15} &  \textbf{5504} &  \textbf{0} \\ 
         \textbf{\textit{NICER}} &  \textbf{5533011101} &  \textbf{59749.18} &  \textbf{2493} &  \textbf{0} \\ 
         \textbf{\textit{NICER}} &  \textbf{5533011201} &  \textbf{59750.13} &  \textbf{2330} &  \textbf{0} \\ 
         \textbf{\textit{NICER}} &  \textbf{5533011301} &  \textbf{59751.75} &  \textbf{1988} &  \textbf{1} \\ 
         \textbf{\textit{NICER}} &  \textbf{5533011401} &  \textbf{59752.65} &  \textbf{1712} &  \textbf{0} \\ 
         \textbf{\textit{NICER}} &  \textbf{5533011501} &  \textbf{59753.30} &  \textbf{5213} &  \textbf{0} \\
         \textbf{\color{blue}{ \textit{NuSTAR}}} &  \textbf{\color{blue}{90801315001}} &  \textbf{\color{blue}{59753.45}} &  \textbf{\color{blue}{35699}} &  \textbf{\color{blue}{4}} \\
         \textbf{\textit{\color{blue}{NICER}}} &  \textbf{\color{blue}{5533011601}} &  \textbf{\color{blue}{59754.15}} &  \textbf{\color{blue}{2373}} &  \textbf{\color{blue}{2}} \\ 
        \textit{NICER} & 5533011502 & 59754.92 & 613 & 0 \\ 
        \textit{NICER} & 5533011503 & 59755.24 & 1263 & 0 \\ 
        \textit{NICER} & 5533011701 & 59755.50 & 1653 & 0 \\ 
        \textit{NICER} & 5533011801 & 59756.02 & 1236 & 0 \\ 
        \textit{NICER} & 5533012101 & 59759.29 & 1131 & 0 \\ 
        \textit{NICER} & 5533012201 & 59760.72 & 2516 & 0 \\ 
        \textit{NICER} & 5533012301 & 59761.16 & 3146 & 0 \\ 
          \hline
    \end{tabular}
    %\tablecomments{The QPOs analyzed in this paper are highlighted in bold text.}
    %\footnote{The QPOs analyzed in this paper are highlighted in bold text.}
\end{table}


\subsection{\textit{Insight}-HXMT} 
\label{sec:2.3}
The hard X-ray Modulation Telescope (HXMT, also named \textit{Insight}-HXMT) is China’s first X-ray astronomy satellite with a broad energy in 1--250\,keV \citep[]{zhang2020}. Following the trigger of \textit{MAXI}, \textit{Insight}-HXMT began monitoring MAXI J1816--195 on June 8, 2022, and observations stopped on July 5, 2022. These observations covered the peak and the decay phase of the outburst. All observations are analyzed using the \textit{Insight}-HXMT processing software {\tt HXMTDAS v2.05}\footnote{\url{http://hxmtweb.ihep.ac.cn/software.jhtml}} according to standard processing procedures\footnote{\url{http://hxmtweb.ihep.ac.cn/SoftDoc/648.jhtml}}. Then, we generate light curves for the energy ranges of 2--10 keV (LE), 10--30 keV (ME), and 30--100 keV (HE), which do not include data between 30 s before the burst until 70 s after the burst. In addition, the arrival time of photons in the clean event file is corrected using the {\tt hxbary} tool. 
%In addition, the arrival time of photons in the clean event file is corrected using the {\tt hxbary} tool for subsequent PDS generation with {\tt POWSPEC}. 










\section{Analysis and Results}
\label{sec:3}
\subsection{Light Curve, Hardness Ratio, and HIDs}
\label{sec:3.1}
The 2--10\,keV light curves for \textit{MAXI}/GSC\footnote{\url{http://maxi.riken.jp/top/index.html}}, \textit{NICER}, and \textit{Insight}-HXMT/LE observations are shown in the first three panels on the left of Fig.~\ref{fig:hr_lc_1}. The X-ray bursts have been removed from the \textit{NICER} and \textit{Insight}-HXMT/LE data. The light curves of the three satellites are consistent with regards to the evolution of the outburst from MJD 59738 to MJD 59761, which cover the peak and the decay phase of the outburst from MAXI J1816–195. The highest count rate of this source is \textasciitilde 480 counts s$^{-1}$, observed by \textit{NICER} on June 10, 2022, and then gradually decayed to \textasciitilde 15 counts s$^{-1}$ (MJD 59761). It should be noted that around MJD 59760, the count rates of HXMT are approximately 30\% of the peak fluxes, while the count rates in the MAXI and NICER data are less than 10\%. This is because HXMT is a collimated telescope with a large field of view, and there may be some contamination sources present within the field of view. The yellow shade of the left image of Fig.~\ref{fig:hr_lc_1} indicates the \textit{NuSTAR}'s observation interval for MAXI J1816–195. From this figure, it is clear that \textit{NuSTAR} and \textit{NICER} have a quasi-simultaneous observation on 2022 June 24 during the outburst decay phase. This 5\,s binned light curve of 3--79\,keV using \textit{NuSTAR}/FPMA is shown in the first panel on the right image of Fig.~\ref{fig:hr_lc_1}, which shows four bursts. The average count of persistent emissions is \textasciitilde 50 counts s$^{-1}$, and the count rate at bursts is as high as 26 times the average count. Its subgraph shows the burst profile with a 5\,s binned light curve and a hardness ratio. 

We define a hardness ratio (HR) from the ratio of the 4–10\,keV to the 2--4\,keV count rates in \textit{NICER}, and a hardness ratio for the 6--20\,keV to 3--6\,keV rates in \textit{NuSTAR}. The last panel on the left image of Fig.~\ref{fig:hr_lc_1} shows the HR evolution of the outburst using \textit{NICER} data. The HR first gradually decreases with the rise of the outburst count rates, maintains a relatively stable ratio of \textasciitilde 0.25 after reaching the peak, and then has a hardening bulge process at the red triangle mark (corresponding to the ObsIDs where the \textasciitilde2.5\,Hz modulation are detected; see detailed analysis in Section~\ref{sec:3.2}), and finally continues to harden as the outburst decays. In the subgraph on the right image of Fig.~\ref{fig:hr_lc_1}, during the decay phase of the burst, low-energy photons exhibit a long tail due to slow decay. The HR decreases and then returns to pre-burst levels.

Combining the intensity and HR, we see a counterclockwise evolution in Fig.~\ref{fig:hid_2}. For comparison, we also plot the hardness-intensity diagram (HID) of \textit{MAXI}/GSC and \textit{Insight}-HXMT/LE in the same energy band. As mentioned above, although the contamination sources may have some impact on the HXMT measurements in Fig.~\ref{fig:hid_2}, the evolutionary trend still holds. Noticeably, the high-statistics NICER data reveals a distinct hardened bulge at the red circle mark, which corresponds to the detection of the \textasciitilde2.5\,Hz modulation.





% Figure environment removed



% Figure environment removed


% Figure environment removed

\begin{table*}
    %\tablewidth{\textwidth}
    \centering
    \caption{Multiple Gaussian fitting parameters for the PDS from \textit{NICER} and \textit{NuSTAR}. $\nu$, rms and Q represent the centroid frequency, fractional rms amplitude and quality factor, respectively. Subscript numbers indicate different Gaussian components.} 
    \renewcommand\arraystretch{1.37}
    \setlength{\tabcolsep}{1.5mm}{
	\label{tab:obsid_qpo_2}
    \begin{tabular}{@{}cccccccccccccc@{}}
    \hline
        ObsID & $\nu$$_{0}$ & rms$_{0}$ (\%) & Q$_{0}$  & $\nu$$_{1}$ & rms$_{1}$ (\%) & Q$_{1}$  & $\nu$$_{2}$ & rms$_{2}$ (\%) & Q$_{2}$  &$\chi^2$/d.o.f \\ \hline
        5533010901 & $2.7^{+0.2}_{-0.2}$ & $11.2^{+1.9}_{-1.7}$ & $2.3^{+0.4}_{-0.3}$  & $5.3^{+0.9}_{-2.4}$ & $10.4^{+2.2}_{-2.9}$ & $2.7^{+0.8}_{-1.5}$  & $10.7^{+2.7}_{-2.9}$ & $7.7^{+5.2}_{-1.7}$ & $2.4^{+0.6}_{-0.5}$  & 83/86\\ 
        5533010902 & $2.3^{+0.1}_{-0.1}$ & $12.5^{+2.1}_{-1.8}$ & $2.9^{+0.4}_{-0.4}$  & $3.8^{+1.0}_{-0.4}$ & $12.2^{+1.9}_{-2.1}$ & $3.0^{+1.0}_{-0.7}$  & $6.7^{+1.0}_{-0.8}$ & $12.9^{+1.3}_{-1.4}$ & $1.7^{+1.0}_{-1.0}$  &132/101\\ 
        5533011001 & $2.2^{+0.1}_{-0.2}$ & $13.9^{+1.8}_{-1.8}$ & $2.6^{+0.4}_{-0.4}$  & $4.1^{+0.4}_{-0.4}$ & $14.2^{+1.8}_{-1.8}$ & $2.8^{+0.5}_{-0.5}$  & $7.4^{+1.1}_{-0.9}$ & $12.3^{+1.8}_{-1.6}$ & $2.2^{+0.8}_{-0.8}$  &126/86\\ 
        5533011101 & $3.6^{+0.2}_{-0.1}$ & $11.9^{+1.3}_{-1.3}$ & $2.1^{+0.2}_{-0.3}$  & $7.3^{+1.1}_{-1.1}$ & $11.4^{+1.6}_{-1.5}$ & $1.9^{+0.4}_{-0.4}$  & ~.~.~. & ~.~.~. & ~.~.~.  &107/101\\ 
        5533011201 & $3.4^{+0.5}_{-0.3}$ & $11.9^{+2.7}_{-3.1}$ & $2.2^{+0.5}_{-0.5}$  & $6.4^{+4.8}_{-1.4}$ & $9.8^{+3.8}_{-3.6}$ & $2.8^{+6.1}_{-1.5}$  & $11.9^{+3.6}_{-4.1}$ & $7.1^{+3.7}_{-2.1}$ & $2.5^{+1.3}_{-1.1}$  &91/101\\ 
        5533011301 & $2.9^{+0.1}_{-0.1}$ & $13.6^{+1.1}_{-1.1}$ & $2.5^{+0.2}_{-0.2}$  & $5.6^{+0.6}_{-0.6}$ & $10.8^{+1.4}_{-1.4}$ & $2.6^{+0.5}_{-0.5}$  & $12.9^{+4.2}_{-3.6}$ & $7.4^{+2.1}_{-1.6}$ & $1.8^{+1.5}_{-1.5}$  &129/101\\ 
        5533011401 & $3.3^{+0.2}_{-0.1}$ & $12.2^{+1.1}_{-1.2}$ & $2.3^{+0.3}_{-0.3}$  & $7.3^{+2.5}_{-1.1}$ & $9.5^{+1.6}_{-1.4}$ & $2.5^{+0.7}_{-0.7}$  & ~.~.~. & ~.~.~. & ~.~.~.  &118/101\\ 
        5533011501 & $2.6^{+0.1}_{-0.1}$ & $21.4^{+2.4}_{-2.7}$ & $2.4^{+0.2}_{-0.3}$  & $4.6^{+6.4}_{-1.9}$ & $15.2^{+3.9}_{-3.6}$ & $2.6^{+3.6}_{-1.3}$  & $8.5^{+2.5}_{-2.5}$ & $13.4^{+4.1}_{-2.8}$ & $1.8^{+0.5}_{-0.5}$  &83/86\\ 
        5533011601 & $2.2^{+0.1}_{-0.1}$ & $15.6^{+1.5}_{-1.6}$ & $2.6^{+0.3}_{-0.3}$  & $4.3^{+0.6}_{-0.5}$ & $12.4^{+2.0}_{-2.2}$ & $2.7^{+0.6}_{-0.7}$  & $8.6^{+2.1}_{-2.1}$ & $8.7^{+2.6}_{-1.9}$ & $2.3^{+0.4}_{-0.3}$  &90/101\\ 
        90801315001 & $2.3^{+0.1}_{-0.1}$ & $32.1^{+1.8}_{-1.9}$ & $2.6^{+0.2}_{-0.2}$  & $4.4^{+0.4}_{-0.4}$ & $26.7^{+2.5}_{-2.5}$ & $2.4^{+0.4}_{-0.4}$  & $8.7^{+1.7}_{-1.5}$ & $22.1^{+3.41}_{-2.8}$ & $1.7^{+1.0}_{-1.0}$  &193/119\\ \hline
    \end{tabular}}
\end{table*}

\subsection{Timing Analysis}
\label{sec:3.2}
For timing analysis, the power density spectrum (PDS) calculated using {\tt POWSPEC} is written to text files. These files can be used as the input to {\tt flx2xsp} to create a "spectrum" file and diagonal response to read into {\tt XSPEC} for fitting (see \cite{2012Ingram} and references therein). In addition, \textit{NuSTAR} is primarily used for spectroscopy observation of dim sources, and there is a dead time of \textasciitilde 2.5\,ms for bright sources. This dead time can cause the resulting PDS to be severely distorted, so we used Fourier Amplitude Difference correction \citep[]{2018Bachetti} in {\tt Stingray} \citep[]{2022matteo} to correct the PDS. The error by $err=power/\sqrt N$, $N$ is the number of power averaged in each bin, which is added to each power and then imported {\tt XSPEC} for fitting.


We initially search for coherent signals by producing a PDS from the \textit{NICER} observation in the 0.5--12.0\,keV energy band. All individual power spectral estimates are averaged per observation (the type-I burst was removed) using 64\,s long segment and 1/8192\,s time bins, corresponding to a frequency resolution of 1/64\,Hz and a Nyquist frequency of 4096\,Hz. We perform Leahy normalization on the PDS \citep[]{1983Leahy} in order to assess the quality of the data depending on whether the poisson noise is 2. Ultimately, nine out of the twenty-six \textit{NICER}’s ObsIDs used in this paper show a strong and broad peak around 2.5\,Hz and do not observe any kHz QPOs. in the PDS (see  Table~\ref{tab:obsid_qpo_2} and Fig.~\ref{fig:PDS_3}). Such a signal is also found in \textit{NuSTAR}’s dead time corrected data in Fig.~\ref{fig:PDS_3}. Since the source is a bit dim, we did not find this peak in the PDS from \textit{Insight}-HXMT observations. The presence of a wide feature in the signal may suggest that the peaks evolve over time, so we generate dynamic power spectra for these observations with 0.03125 s bins, 16 s segments. As shown in Fig.~\ref{fig:PDS_lc_5}, the modulation is intermittent, and the central frequency of the modulation remains constant. We remove segments with no signal from each observation to reduce this intermittent broadening of the peak. We then regenerate PDSs, setting 1/512\,s time bins and applying Miyamoto normalization \citep[]{miyamoto1992}, to focus on peaks between 0.1--100\,Hz. We follow \citet{patruno2009} and \citet{patruno2013}'s analysis of SAX J1808.4--3658 and NGC 6440 X--2, and model these features in the PDS with two or three Gaussian functions in {\tt XSPEC} because the peak is quite broad and has a decline at lower frequencies. To assess the need for a third Gaussian component, we simulate $10^{5}$ spectra with the {\tt simftest} script, and if the significance of the third Gaussian exceeds 3$\sigma$, we include it in the fitting of the PDS. The results are shown in  Table~\ref{tab:obsid_qpo_2} and Fig.~\ref{fig:PDS_3}, a weak and insignificant broad component is observed around 40 Hz. This feature was detected in other observations, we thus do not give it much emphasis. 

The approximate harmonic relationships between the centroid frequencies of multiple Gaussians are $\nu_{1}$/$\nu_{0}$=$1.9\pm{0.1}$ and $\nu_{2}$/$\nu_{0}$=$3.4\pm{0.1}$, as shown in Fig.~\ref{fig:f2_f1_4}. The ratios based on $\nu_{1}$, however, gives $\nu_{0}$/$\nu_{1}$=$0.530\pm{0.01}$ and $\nu_{2}$/$\nu_{1}$=$1.88\pm{0.05}$, suggesting that the third component is also the fourth harmonics of the $\nu_{0}$ (or the second harmonics of $\nu_{1}$). This finding bears some resemblance to the centroid frequency ratios of simultaneous low-frequency QPOs in low-mass X-ray binaries 
\citep[e.g.][]{2020Doesburgh,2021Fei}. However, in their PDSs, both the fundamental and harmonic frequendcies are comparatively narrower, making them easier to be distinguished. In the case of MAXI J1816--195, these components are broader and weaker. Therefore, it is challenging to determine if all three components originate from genuine physical processes, requiring further observational research. Thus, this paper primarily focuses on discussing the strongest component ($\nu_{0}$), hereafter \textasciitilde2.5\,Hz modulation. Its quality factor ($Q$ = $\nu_{0}$/$FWHM$) is $> 2$. 
%The maximum significance is $8.74 \sigma$, defined as the integral of the Gaussian divided by its error.
%and the fitting results may be coincidental to some extent


%it is possible that the resonance frequency does not actually exist and is simply a coincidental artifact of the fitting process caused by the wide bandwidth of the signal. 
We calculate the fractional rms amplitude of this \textasciitilde2.5\,Hz modulation in order to study the strength of the low-frequency X-ray variability. The fractional rms amplitude of the Miyamoto normalized power density spectrum is defined as \citep[]{1989van,belloni1990}:

\begin{equation}
    rms=\sqrt{\int_{}^{} P(v) d v}.
	\label{eq:rms}
\end{equation}

%Taking into account the background contribution, the RMS is counted as $RMS=rms \times(S+B) / S$, where $S$ and $B$ are the source and background count rates, respectively.

%Taking into account the background contribution, the fractional rms amplitude is calculated as $rms=\sqrt{P} \times(S+B) / S$, where $S$ and $B$ are the source and background count rates, respectively. $P$ is the power normalized according to  \cite{belloni1990}. 

We did not consider the background contribution because it is very low for both \textit{NICER} and \textit{NuSTAR}. As shown in Fig.~\ref{fig:f_rmd_lc_6}, the fractional rms of the \textit{NICER}’s nine observations are basically the same, and there is little correlation between the frequency, the fractional rms, and the source count rate. However, when considering short time scales ($<$1 Day), the presence of this modulation appears to be correlated with both the count rates and HRs, as demonstrated in Fig.~\ref{fig:PDS_lc_5}. Specifically, when the \textasciitilde2.5\,Hz modulation is observed, both the count rates and HRs exhibit higher values.

We also investigate the energy dependence of the \textasciitilde2.5\,Hz modulation rms, as well as the time lag between the energies. According to the equal count rate, each observation of \textit{NICER} is divided into six energy bands: 0.5--1.5\,keV, 1.5--1.8\,keV, 1.8--2.1\,keV, 2.1--2.6\,keV, 2.6--3.4\,keV, and 3.4--12.0\,keV. \textit{NuSTAR} is in four energy bands: 3.0--4.6\,keV, 4.6--6.4\,keV, 6.4--9.5\,keV, 9.5--79.0\,keV. The fractional rms amplitude increases with energy from \textasciitilde 12\% at 2\,keV to \textasciitilde 30\% at 8\,keV in Fig.~\ref{fig:rms_energy_7}. Above 5\,keV, the fractional rms remains approximately constant. It should be reminded that Fig.~\ref{fig:rms_energy_7} does not give the results of the low energy bands (0.5--1.5\,keV and 1.5--1.8\,keV) of \textit{NICER}. This is due to the very low significance of the modulation signals (Fig.~\ref{fig:PDS1.5}), which results in substantial uncertainties in the measurements.
%It should be reminded that Fig.~\ref{fig:rms_energy_7} does not give the results of the 0.5--1.5\,keV, 1.5--1.8\,keV low energy band of \textit{NICER}, since their power spectra are very flat without any fluctuation. 

Following standard techniques in \citet{uttley2014x}, we initially calculate the lag-frequency spectrum using the AveragedCrossspectrum object in {\tt Stingray}. The upper panel in Fig.~\ref{fig:time_lag_8} corresponds to ObsID 5533011601 as an example, where the reference energy band is taken to be 1.8--2.1\,keV. The energy dependence of the time lags for all \textit{NICER}’s ObsIDs with the \textasciitilde2.5\,Hz modulation is obtained by averaging the lags within the FWHM of the \textasciitilde2.5\,Hz modulation in the lower panel of Fig.~\ref{fig:time_lag_8}. This result indicates that no time lags are detected.


All of the above analyses are based on the MAXI J1816--195 persistent emission spectra and the type-I burst data are removed. In order to investigate whether there is a modulation signal during the burst, we generate dynamic power spectra by taking 300\,s of data starting 100\,s before the burst. We also extracted the first 50\,s data since the start of the burst to generate PDS. ObsID 5533011301 is shown as an example in the upper panel of Fig.~\ref{fig:dps_burst_9}, and it is not hard to see that the low-frequency modulation is not affected during the burst. The modulation in the subplot of the upper panel of Fig.~\ref{fig:dps_burst_9} is not smooth due to averaging together power spectra by using just three segments of data, but it does not affect the overall results. However, there is a serious noise at lower frequencies, which we conjecture is due to the Fourier transform being unfriendly to unstable signals, see \citet{polikar1996wavelet}. To prove our idea, we use an Inverse-Gamma distribution to simulate the type-I burst profile, followed by a sinusoidal function of 2.5\,Hz and Poisson noise. This simulated signal and its dynamic power spectrum are plotted in the lower panel of Fig.~\ref{fig:dps_burst_9}. The low-frequency strong noise corresponding to the position of the simulated burst on the figure proves our idea. In summary, the \textasciitilde2.5\,Hz modulation exists during the burst.



% Figure environment removed

% Figure environment removed


% Figure environment removed

% Figure environment removed




\subsection{Spectral Analysis}
\label{sec:3.3}
In this work, all spectra are fitted within {\tt XSPEC} v12.12.1. Abundances were set to WILM \citep[]{2000Wilms}, and cross-sections to VERN \citep[]{1996Verner}.

We conduct a persistent spectral analysis of MAXI J1816--195 using simultaneous observations from \textit{NICER} in the 0.5--10.0\,keV range and \textit{NuSTAR} (FPMA and FPMB) in the 3.0--40.0\,keV range on MJD 59754). We attempt to fit the data for periods with and without \textasciitilde2.5\,Hz modulation using the same model (model 1, as described below), but no significant variations in the spectral parameters were found. It is worth noting that for the spectra with modulation, the values of parameter $\Gamma$ = $1.90^{+0.01}_{-0.03}$ and $kT_{\rm e}$ = $9.39^{+1.99}_{-0.87}$\,keV, while for the spectra without modulation, the values of $\Gamma$ = $1.94\pm{0.01}$ and $kT_{\rm e}$ = $11.94^{+2.52}_{-1.23}$\,keV. However, considering the uncertainties, we cannot conclude that the spectra during the modulation period are harder with high significance. Therefore, we also plot the data ratios of periods with and without the \textasciitilde2.5\,Hz modulation from the \textit{NuSTAR}/FPMA observation in Fig~\ref{fig:ratio}. The ratios greater than one and slightly steeper with increasing energy are consistent with the timing analysis mentioned in Section~\ref{sec:3.2}, indicating that the fluxes become larger and the spectra are slightly harder when the \textasciitilde2.5\,Hz modulation is detected. But since the ratios do not exhibit a very strong dependence on energy, to enhance the signal-to-noise ratio, we subsequently studied the persistent emission of the entire \textit{NuSTAR} observation.

%We conduct a persistent spectral analysis of MAXI J1816--195 using simultaneous observations from \textit{NICER} in the 0.5--10.0\,keV range and \textit{NuSTAR} (FPMA and FPMB) in the 3.0--40.0\,keV range on June 24. Fig~\ref{fig:ratio} shows the energy ratio with periods of the \textasciitilde2.5\,Hz modulation and periods without the \textasciitilde2.5\,Hz modulation from the \textit{NuSTAR}/FPMA observation. Ratios greater than one and steeper with increasing energy are consistent with the Section~\ref{sec:3.2} timing analysis, i.e., the fluxes become stronger and the spectra become slightly harder when the \textasciitilde2.5\,Hz modulation are detected. In addition, we have fitted data for periods with and without \textasciitilde2.5\,Hz modulation using the same model, and no significant variations in the spectral parameters between the two periods have been found, so we next study the persistent emission of \textit{NuSTAR}'s entire observation to enhance the signal-to-noise ratio.


Following the procedures of \citet{2022Chen}, \citet{2022Mandal} and \citet{2022Bult}, we initially try to fit the joint \textit{NICER} and \textit{NuSTAR} spectra by an absorbed disk black-body ({\tt diskbb}) plus a thermally Comptonised continuum ({\tt nthcomp}) with seed photons from the accretion disc. A constant multiplication factor is included to account for calibration differences between different telescopes. Residuals with the broad emission feature in 6--7\,keV and excess in 10--30\,keV suggest a possible emission line from Fe-K and a Compton hump from the reflection of hard X-rays by the cool accretion disc. These features are evident in Fig.~\ref{fig:residuals_10} (upper panel). We therefore proceed by modeling our data with the self-consistent reflection model {\tt relxill v2.2} \footnote{\url{http://www.sternwarte.uni-erlangen.de/~dauser/research/relxill/index.html}}. The {\tt relxillCP} model, which describes the reflection using {\tt nthcomp} as an illuminating continuum, is used. We have assumed an unbroken emissivity profile ($\propto$ $r^{\rm -q}$) with a fixed slope of $q = 3$ \citep[]{2012Wilkins}. We also have fixed the outer radius $R_{\rm out}$ =1000\,$R_{\rm g}$, as the sensitivity of the reflection fit decreases with increasing outer disc radius, and set a redshift of 0. From measurements of the NS spin frequency with 528\,Hz \citep[]{2022Bult}, the dimensionless spin parameter $a = 0.248$  has been approximated using the relation $a = 0.47/P_{\rm ms}$ \citep[]{2000Braje}, where $P_{\rm ms}$ is the spin period in ms. We notice that the large residuals using the {\tt const}$\times${\tt tbabs}$\times${\tt (relxillcp}$+${\tt diskbb)} model with a $\chi^2$/dof = 1884/1683 = 1.12 are still at soft energies, as shown in the lower panel of Fig.~\ref{fig:residuals_10}; there are two edge-like shapes near \textasciitilde 0.9 and \textasciitilde 1.8 keV. These remaining features maybe come from \textit{NICER}’s calibration systematics\footnote{\url{https://heasarc.gsfc.nasa.gov/docs/nicer/analysis_threads/arf-rmf/}}\footnote{\url{https://heasarc.gsfc.nasa.gov/docs/nicer/analysis_threads/plot-ratio/}} (see e.g. \cite{2020Ludlam,2021Ludlam}). Therefore, two {\tt edge} models are added in the fitting, and the overall model becomes {\tt const}$\times${\tt edge}$\times${\tt edge}$\times${\tt tbabs}$\times${\tt (relxillcp}$+${\tt diskbb)}, hereafter Model 1. The addition of the two {\tt edge} model improves the overall fit significantly to $\chi^2$/dof = 1737/1678 = 1.03.

\begin{table}
	%\tablewidth{\textwidth}
	\centering
	\caption{Best-fitting spectral parameters of the \textit{NICER} and \textit{NuSTAR} observations of MAXI J1816--195 using Model 1: {\tt const}$\times${\tt edge}$\times${\tt edge}$\times${\tt tbabs}$\times${\tt (relxillcp}$+${\tt diskbb)}. Uncertainties are given at 90\%.}
	\renewcommand\arraystretch{1.36}
	\begin{tabular}{ccc}
			\hline
			Model & Parament (unit) & Value \\
			\hline
			TBABS & $N_{\rm H}\ (\times10^{22}\ {\rm cm^{-2}})$ & $2.27^{+0.04}_{-0.04}$ \\
			EDGE & $E_{\rm edge,1}  $ (keV)& $1.84^{+0.04}_{-0.04}$\\
			EDGE & $E_{\rm edge,2}$ (keV) & $0.90^{+0.02}_{-0.03}$\\
			DISKBB & $kT_{\rm in}\ ({\rm keV})$  & $0.55^{+0.01}_{-0.01}$  \\
			   & $N_{\rm diskbb}$  &$916^{+74}_{-62}$ \\
			  
			RELXILLCP & $\Gamma$ & $1.96^{+0.02}_{-0.01}$ \\
			 & $kT_{\rm e}\ ({\rm keV})$ & $10.06^{+0.40}_{-0.35}$ \\
			 & $R_{\rm in}$\ $(\times R_{\rm ISCO})$ & $1.14^{+0.09}_{-0.10}$\\
			 & $a$ & 0.248\ (fixed)\\
			 & Refl\_frac & $0.18^{+0.04}_{-0.02}$  \\
			 & $i\ (^{\circ})$ & $<13.23$  \\
			 & $\log\xi$ & $3.28^{+0.16}_{-0.25}$  \\
			 & $\log N\ ({\rm cm^{-3}})$ & $17.95^{+0.67}_{-0.83}$\\
			 & $A_{\rm Fe}$\ $(\times \rm solar)$ & $2.13^{+0.73}_{-0.84}$ \\
			 & Norm\ ($\times10^{-4}$) & $55^{+2}_{-3}$\\
			\hline
			CFLUX & $F_{\rm total}\ (\times10^{-9}\ {\rm ergs\ \rm s^{-1}\ \rm cm^{-2}})$ &$5.42^{+0.07}_{-0.07}$\\
			 (0.5--79\,keV)&  $F_{\rm diskbb}\ (\times10^{-9}\ {\rm ergs\ \rm s^{-1}\ \rm cm^{-2}})$ & $1.36^{+0.06}_{-0.06}$\\
			 & $F_{\rm relxillcp}\ (\times10^{-9}\ {\rm ergs\ \rm s^{-1}\ \rm cm^{-2}})$ &$4.05^{+0.02}_{-0.02}$\\
			\hline
			 & $\chi^2$/d.o.f & 1.03\ (1738/1678) \\
			\hline
	\end{tabular}
	\label{tab:3}
\end{table}

Regrettably, the downside of RELXILLCP is the fixed seed photon temperature of 0.01\,keV. In an attempt to mitigate this limitation, we explored the use of the {\tt nthratio} model \footnote{\url{https://github.com/garciafederico/nthratio}}. However, the revised parameters obtained from this model did not significantly impact the subsequent discussions. Moreover, considering the self-consistency of the reflection model, we decided not to include this modification in the paper. We look forward to further improvements and optimizations of future models.

The best-fitting parameters are listed in Table~\ref{tab:3} and the persistent spectra are shown in Fig.~\ref{fig:spectrum_11}. We obtain a hydrogen column density ($N_{\rm H}$) of $(2.27\pm{0.04}) \times 10^{22}$\ ${\rm cm^{-2}}$, an inner disc temperature of $kT_{\rm in}$ = $0.55\pm{0.01}$\,keV, a power-law photon index of $\Gamma$ = $1.96^{+0.02}_{-0.01}$, and the electron temperature $kT_{\rm e}$ = $10.06^{+0.40}_{-0.35}$\,keV. The iron abundance $A_{\rm Fe}$ in solar abundance and the ionization parameter $\log\xi$ of the disc are $2.13^{+0.73}_{-0.84}$ and $3.28^{+0.16}_{-0.25}$, respectively, which is consistent with other NS LMXBs (see e.g. \citet{2019Sharma}). It predicts a small inner disc radius of $R_{\rm in}$ = (1.04--1.23)\,$R_{\rm ISCO}$ from the reflection spectra. For a spinning NS with $a = 0.248$, $R_{\rm ISCO}$ can be approximated using $R_{\rm ISCO}$ = 6$R_{\rm g}$(1--0.54$a$) \citep[]{1998Miller}. An inner disc radius of $R_{\rm in}$ = (5.41--6.40)\,$R_{\rm g}$ = (11.20--13.24)\,km is measured for Model 1 ($R_{\rm g} = GM/c^{2}$ is the gravitational radii, which is 2.07\,km for M = 1.4\,$M_{\rm \odot}$). This value is roughly consistent with the inner disc radius ($<19$\,km) inferred from the {\tt diskbb} normalization $N_{\rm diskbb}$ = $(R_{\rm in,diskbb}/D_{\rm 10})^{2} \cos{i}$, where the upper limit of the distance is assumed to be 6.3\,kpc \citep[]{2022Chen} and the lower limit of inclination is $0^{\circ}$. If we consider the emission from the disk to be Comptonized by the corona, the inner radius of the disk would be slightly larger \citep{1998Kubota}. In addition, our spectral fits can only give the upper limit on inclination $< 13.23^{\circ}$ at $3\sigma$ confidence level. We compute $\chi^{2}$ for inclination using steppar in {\tt XSPEC}. The variation of $\chi^{2}$ of the fit versus the inclination ($i$) is shown in Fig.~\ref{fig:inclination_12}.


%the upper limit on inner disc radius $R_{\rm in,diskbb}$\textasciitilde 19\,km implied by the normalization $N_{\rm diskbb}$ = $(R_{\rm in,diskbb}/D_{\rm 10})^{2} \cos{i}$ (with $R_{\rm in,diskbb}$ in km and $D_{\rm 10}$ the distance in units of 10\,kpc) of the disc blackbody component assuming $D_{\rm 10}$ = 6.3\,kpc \citep[]{2022Chen} and $i$ = $0^{\circ}$. 



 In addition, by considering that the presence of a hard surface of a neutron star might imply a potential blackbody component, we also test the fits by adding a {\tt bbodyrad} to Model 1, which we define as Model 2 ({\tt const}$\times${\tt edge}$\times${\tt edge}$\times${\tt tbabs}$\times${\tt (relxillcp}$+${\tt bbodyrad}$+${\tt diskbb)}).
 We find that the blackbody component is not significant with a temperature $kT_{\rm bb}$ = 2.05\,keV and a radius $R_{\rm bb}$ = 0.52\,km, and the values of other parameters do not change (Table~\ref{tab:4}). Moreover, $\chi^{2}$ is just improved from 1737 to 1717. The weak blackbody component is similar to the 2017 outburst of IGR J16597--3704 \citep[]{2018Sanna} and the 2019 outburst of SAX J1808.4--3658 \citep[]{2022Sharma}. Therefore, the following discussion is based on the parameters obtained from Model 1. 
 
 
 
 %the blackbody component in the spectrum is not prominent, which is similar to the 2017 outburst of IGR J16597--3704 \citep[]{2018Sanna} and the 2019 outburst of SAX J1808.4--3658 \citep[]{2022Sharma}. The following discussion is based on the parameters obtained from model 1.
 
 
 %We add a {\tt bbodyrad} to Model 1, hereafter Model 2. We obtain a blackbody temperature $kT_{\rm bb}$ = 2.05\,keV and a radius $R_{\rm bb}$ = 0.52\,km, and $\chi^{2}$ is raised from 1737 to 1717, just marginally improve the fitting. Hence, the blackbody component in the spectrum is not prominent, which is similar to the 2017 outburst of IGR J16597--3704 \citep[]{2018Sanna} and the 2019 outburst of SAX J1808.4--3658 \citep[]{2022Sharma}. The following discussion is based on the parameters obtained from model 1.


%It can be seen from the data in Table~\ref{tab:4} that Model 2 does not limit the parameters very well. Except for the inclination angle $i$, both $R_{\rm in}$ and $\log N$ are also provided as upper limits. 




%In section 3.2, the case where the \textasciitilde 2.5\,Hz QPO is not affected by bursts prefers the disk corona model, and \citet{2022Chen} makes the same view about MAXI J1816--195.  So we attempted to use {\tt relxilllpcp} with the lamp post geometry to fit the joint \textit{NICER} and \textit{NuSTAR} spectrum. The primary spectrum of {\tt relxilllpcp} is {\tt nthcomp} and coincides with {\tt relxillcp}. 





% Figure environment removed

% Figure environment removed


			
\begin{table}
	%\tablewidth{\textwidth}
	\centering
	\caption{Best-fitting spectral parameters of the \textit{NICER} and \textit{NuSTAR} observations using Model 2: {\tt const}$\times${\tt edge}$\times${\tt edge}$\times${\tt tbabs}$\times${\tt (relxillcp}$+${\tt bbodyrad}$+${\tt diskbb)}. Uncertainties are given at 90\%.}
	\renewcommand\arraystretch{1.36}
	%\setlength{\tabcolsep}{7mm}
	\begin{tabular}{ccc}
			\hline
			Model & Parament (unit) & Value \\
			\hline
			TBABS & $N_{\rm H}\ (\times10^{22}\ {\rm cm^{-2}})$ & $2.20^{+0.04}_{-0.06}$ \\
			EDGE & $E_{\rm edge,1} $(keV)& $1.85^{+0.03}_{-0.03}$\\
			EDGE & $E_{\rm edge,2} $ (keV) & $0.88^{+0.03}_{-0.03}$\\
			DISKBB & $kT_{\rm in}\ ({\rm keV})$  & $0.59^{+0.02}_{-0.02}$  \\
			   & $N_{\rm diskbb}$  &$736^{+66}_{-70}$ \\
			BBODYRAD & $kT_{\rm bb}\ ({\rm keV})$& $2.05^{+0.57}_{-0.25}$\\
			 &$N_{\rm bb}$&$0.69^{+1.12}_{-0.22}$\\
			RELXILLCP & $\Gamma$ & $1.88^{+0.04}_{-0.08}$ \\
			 & $kT_{\rm e}\ ({\rm keV})$ & $10.27^{+0.57}_{-0.56}$ \\
			 & $R_{\rm in}$\ $(\times R_{\rm ISCO})$ & $< 1.12$\\
			 & $a$ & 0.248\ (fixed)\\
			 & Refl\_frac & $0.38^{+0.13}_{-0.20}$  \\
			 & $i\ (^{\circ})$ & $< 13.81$  \\
			 & $\log\xi$ & $3.64^{+0.36}_{-0.27}$ \\
			 & $\log N\ ({\rm cm^{-3}})$ & $< 18.12$\\
			 & $A_{\rm Fe}$\ $(\times \rm solar)$ & $2.27^{+4.41}_{-1.14}$ \\
			 & Norm\ ($\times10^{-4}$) & $38^{+3}_{-13}$\\
			\hline
			 & $\chi^2$/d.o.f & 1.02\ (1717/1676) \\
			\hline
	\end{tabular}
	\label{tab:4}
\end{table}



\section{Discussion}
\label{sec:4}
In this paper, we have analyzed the spectral and timing properties of the newly discovered millisecond pulsar MAXI J1816--195 using \textit{NICER} and \textit{NuSTAR}, and find a strong low-frequency modulation in the 2--4\,Hz range during the decay stage of the outburst. The modulation is well described by multiple Gaussians. The strongest component is the main subject, called \textasciitilde2.5\,Hz modulation. The broadband X-ray spectra of MAXI J1816--195 were well-fit by a combination of the reflection model and a disk black-body component over the range of 0.5 to 40\,keV.

\subsection{The properties of MAXI J1816--195 from the reflection model}
The persistent spectra of MAXI J1816--195 during simultaneous observations of \textit{NICER} and \textit{NuSTAR} are well-fitted by Model 1 ({\tt const}$\times${\tt edge}$\times${\tt edge}$\times${\tt tbabs}$\times${\tt (relxillcp}$+${\tt diskbb)}). According to Model 1, the flux in 0.5--79\,keV is $(5.42\pm{0.07})\times10^{-9}\ {\rm ergs\ \rm s^{-1}\ \rm cm^{-2}}$, corresponding to 14.3\%$L_{\rm Edd}$ at a distance of 6.3\,kpc for $L_{\rm Edd}$ = $1.8\times10^{38}\ {\rm ergs\ \rm s^{-1}}$ assuming a neutron star mass of 1.4 $M_{\odot}$. Equation (2) in \citet{2008Galloway} is used to estimate the mass accretion rate of MAXI J1816--195. We assume $c_{\rm bol} = 1.38$, $1+z = 1.31$ for a NS with mass ($M_{\rm NS}$) $1.4\,M_{\odot}$ and radius ($R_{\rm NS}$) 10\,km. We then obtain an upper limit for mass accretion rate of $3.96 \times 10^{-9}\,M_{\odot}\ y^{-1}$ for $D = 6.3\,\rm kpc$.


 %\red{The reflection model shows that the inner edge of the accretion disc extends inwards to $R_{\rm in} = 5.41-6.40,R_{\rm g}$ (11.20--13.24,km), which is 
 %not expected for AMXPs with low accretion rates. In fact, \cite{2015Degenaar} discovered that the disk is near the surface of the neutron star at low accretion rates in 4U 1608--52.} 
 %Furthermore, \cite{cackett2010} studied the relationship between the inner disk radius and Eddington fraction for a sample of 10 NSs, and found that there is no apparent systematic trend with luminosity and no clear difference between atoll sources, Z sources, and accreting millisecond pulsars. The same result was also seen by \cite{2017Ludlam}. %However, the inner disc radius of $R_{\rm in}$\textasciitilde 19\,km is implied by the normalization of the {\tt diskbb} component in Model 1. Such a difference may be not outrageous given that the distance we are using is an upper limit. 

The reflection model shows that the inner edge of the accretion disc extends inwards to $R_{\rm in} = 5.41-6.40\,R_{\rm g}$ (11.20--13.24\,km), which is near the surface of the neutron star. If the inner disc is truncated at the magnetosphere radius, we used Equation (1) of \citet{2009Cackett} to calculate the upper limit of the magnetic field strength of the NS:
\begin{equation}
\begin{aligned}
\mu = & 3.5 \times 10^{23} k_{\rm A}^{-7 / 4} x^{7 / 4}\left(\frac{M}{1.4 }\right)^2 \\
& \times\left(\frac{f_{\rm a n g}}{\eta} \frac{F_{\rm b o l}}{10^{-9}\ {\rm ergs}\ {\rm cm}^{-2}\ \rm{s}^{-1}}\right)^{1 / 2} \frac{D}{3.5\ {\rm kpc}}\ {\rm G}\ {\rm cm}^3
\label{eq:flied}
\end{aligned}
\end{equation}
where $k_{\rm A}$ is the coefficient depending on the conversion from spherical to disk accretion, $f_{\rm ang}$ is the anisotropy correction factor \citep[]{2009Ibragimov}, and $\eta$ is the accretion efficiency in the Schwarzschild metric. The bolometric flux of $F_{\rm b o l} = 7.32 \times 10^{-9}\ \rm ergs\ \rm cm^{-2}\ \rm s^{-1}$  is estimated by extrapolating the spectral fit over the 0.1--100\,keV range. We set $k_{\rm A} = 0.5$, $f_{\rm ang} = 1$, and $\eta = 0.1$ for an NS of mass 1.4\,$M_{\rm \odot}$ and radius 10\,km based on \citet{2009Cackett}. $x$ is obtained from $R_{\rm in} = xGM/c^{2}$. If we take the constraint of $x = 6.4$ and $D = 6.3\,\rm kpc$, the magnetic field strength is $B \leq 4.67 \times 10^{8}\,\rm G$, which is similar with other AMXPs, and consistent with the calculation in \cite{2023Li} by assuming that the spin-up of the pulsar is solely induced by the angular momentum transferred from the accreted material to the NS.

%. \red{This magnetic field value is consistent with the calculation by \cite{2023Li}, which assumes that the spin-up of the pulsar is solely induced by the angular momentum transferred from the accreted material to the NS.}

\subsection{The \textasciitilde2.5\,Hz modulation generation mechanism}

%As shown in Fig.~\ref{fig:PDS_3}, we find that MAXI J1816--195 has a broad oscillation at low frequencies and does not find any kHz profile. This low-frequency oscillation can be fitted by Gaussian, as is the case with SAX J1808.4--3658 and NGC 6440 X--2. It is worth noting that the low-frequency power spectrum of this source does not have as many complex components as other AMXPs or atoll sources, and can be fitted only by gaussian, and their center frequency is approximately multiple. This implies a relatively stable coupling between the neutron star and the accretion flow. However, the relationship between the center frequencies does not rule out the possibility of coincidence, which needs to be verified in more studies in the future. From the Table.~\ref{tab:obsid_qpo_2}, we can see that the Q factor of "fundamental frequency" is greater than 2 and has a high significance. Thus, we call it 2.5\,Hz QPO, and then further study it separately.


%Although both are low-frequency QPOs present in AMXPs, MAXI J1816--195 is significantly different from Aql X--1 and IGR J00291+5934. The rms of MAXI J1816--195 increases with energy, but both of them are displayed in soft X-ray with RMS falling strongly with energy. Another difference is the connection with burst, the oscillation of Aql X--1 disappeared after the burst suggesting that QPO could be related to the marginally stable burning in the neutron star envelope. So their mechanism is different. The QPO amplitudes of MAXI J0911--655 show an increase as a function of energy, but no QPOs can be resolved for the 10--30 keV energy band \citep[]{2017BultB}. SAX J1808.4--3658 and NGC 6440 X--2 sources's results are most similar to the characteristics of MAXI J1816--195, and they interpret the 1\,Hz modulation as hydrodynamic disk instabilities when the system is entering the propeller regime. Although the power spectrum of MAXI J1816--195 in this paper is very similar to that of the two accretion millisecond pulsars, and the energy dependence of the QPO amplitude trend is consistent, the innermost radius of the accretion disk of MAXI J1816--195 is very small, which does not meet the conditions of the propeller stage \citep[]{1993Spruit}. So, the trapped disk instability model cannot explain the \textasciitilde 2.5\,Hz QPO we found.
As shown in Fig.~\ref{fig:PDS_3}, we find that MAXI J1816--195 has a broad peak at low frequencies in the PDS and does not find any kHz QPO. The PDS of normal atoll sources is a more complex at low frequencies (< 200\,Hz) characterized by broadband-limited noise and broad Lorentzian components \citep[]{2002Belloni}. In the case of MAXI J1816--195, the PDS shows a peak that, although not as sharp as the classic QPO, is concentrated in the 0.1--10\,Hz range, with Poisson noise dominating the rest of the frequency range. Similar peaks in the PDS have also been observed in SAX J1808.4--3658 and NGC 6440 X--2, and the Gaussian component with the highest peak power in the PDS has been identified as a QPO. Additionally, \cite{Di_Salvo_2001} found evidence that a component can sometimes switch from being a broad noise-like component into a QPO or vice versa. Therefore, we conclude that the strong \textasciitilde2.5\,Hz modulation observed in MAXI J1816--195 is most likely a low-frequency QPO rather than a broadband-limited noise.

%The \textasciitilde2.5\,Hz modulation of MAXI J1816--195 is perhaps observed in such a conversion.
%The \textasciitilde2.5\,Hz modulation of MAXI J1816--195 may also exhibit as a clear QPO under a particular condition, which needs to be tested by future observations. 

%Although the Gaussian components have a narrower width (Q\textasciitilde2) than the typical zero-centered low-frequency Lorentzians observed in all power spectra, it is not as sharp as the QPO. Similar peaks in the PDS have also been observed in the PDS of SAX J1808.4--3658 and NGC 6440 X--2, and the Gaussian component with the highest peak power in the PDS has been identified as a QPO. Additionally, \cite{Di_Salvo_2001} found evidence that a component can sometimes switch from being a clear band-limited noise component into a clear QPO or vice versa. Therefore, we conclude that the strong \textasciitilde2.5\,Hz modulation observed in MAXI J1816--195 is most likely a low-frequency QPO rather than a broadband-limited noise.
%Although the accreting millisecond pulsars MAXI J1816-195, Aql X-1, and IGR J00291+5934 all exhibit low-frequency QPOs, there are significant differences in the characteristics of these QPOs.

However, the observational properties of this \textasciitilde2.5\,Hz modulation differ from those of the low-frequency QPOs observed in most AMXPs. The rms of MAXI J1816--195 increases with energy, but the QPOs of Aql X-1 and IGR J00291+5934 are displayed in soft X-rays with fractional rms falling strongly with energy. Another difference is the connection with the burst; the QPOs of Aql X--1 disappeared after the burst, suggesting that QPOs could be related to the stable burning in the neutron star envelope. The burst does not affect the \textasciitilde2.5\,Hz modulation of MAXI 1816--195. Thus, implying a different mechanism for this \textasciitilde2.5\,Hz modulation. The QPO amplitudes of MAXI J0911--655 increase with energy, but no QPOs can be resolved for the 10--30\,keV energy band \citep[]{2017BultB}. Instead, only a single broad noise term is present. Results of SAX J1808.4--3658 and NGC 6440 X--2 are most similar to the characteristics of MAXI J1816--195. The QPO frequencies are very close to the frequency of MAXI J1816--195 and the energy dependence of the QPO amplitude also increases with energy. \citet{patruno2009} and \citet{patruno2013} interpreted the modulation as hydrodynamic disk instabilities when the system moves into the propeller regime at very low $\dot{M}_{\rm \odot}$. In the propeller regime, the Keplerian velocity in the innermost region of the accretion disk is slower than the rotational velocity of the neutron star magnetosphere. While the innermost radius of the accretion disk of MAXI J1816--195 is very small, it does not meet the conditions of the propeller stage (the corotation radius is \textasciitilde 26\,km for MAXI J1816--195). Moreover, we do not observe flux drop in the light curve. So, the trapped disk instability model cannot explain the \textasciitilde2.5\,Hz modulation we found.


 %In addition to kHz QPO, low-frequency quasi-periodic oscillations have also been found in millisecond pulsars. Aql X-1 \citep[]{2001Revnivtsev} and IGR J00291+5934 \citep[]{2017Ferrigno} found QPOs of \textasciitilde 6-7\,mHz and \textasciitilde 8\,mHz, respectively. Unlike our work, both were displayed in soft X-ray with RMS falling strongly with energy. This oscillation of Aql X-1 disappeared after the burst suggesting that QPO could be related to the marginally stable burning in the neutron star envelope. However, this situation is very hard to investigate for IGR J00291+5934 due to lack of observational data. \citet{2017Ferrigno} suggest that the X-ray variability and \textasciitilde 8\,mHz QPO of IGR J00291+5934 in 2015 are likely produced by a heartbeat-like mechanism in black hole binaries. 
 
%\citet{patruno2009} studied systematically the five outbursts of SAX J1808.4-3658 and found the 1\,Hz modulation with an RMS increasing strongly with energy in the 2002 and 2005 outbursts. This phenomenon was also seen in NGC 6440 X-2. The mechanism of the 1\,Hz modulation is still unclear, \citet{patruno2013} interpreted that the variability was caused by disk availability when the inner edge of the accretion disk is close to the corotation radius. Although the power spectrum of MAXI J1816--195 in this paper is very similar to that of the two accretion millisecond pulsars, and the energy dependence of the QPO amplitude trend is consistent, the innermost radius of the accretion disk of MAXI J1816--195 is very small, which does not meet the conditions of the propeller stage \citep[]{1993Spruit}. So, the trapped disk instability model cannot explain the \textasciitilde 2.5\,Hz QPO we found.

The fractional rms of the \textasciitilde2.5\,Hz modulation in MAXI J1816--195 increased with energy from \textasciitilde 12\% at 2\,keV to \textasciitilde 30\% at 8\,keV (see Fig.~\ref{fig:rms_energy_7}), while at higher energies it still remained at a high value. Additionally, given that there are insignificant oscillations below 2\,keV, and that the appearance of \textasciitilde2.5\,Hz modulation (see Fig.~\ref{fig:hid_2} and Fig.~\ref{fig:PDS_lc_5}) is strongly correlated with a harder spectrum, it is suggested that this \textasciitilde2.5\,Hz modulation may be produced by the hard component, perhaps from an unstable corona. From Fig.~\ref{fig:dps_burst_9}, it can be seen that the modulation is almost unaffected by the burst, which suggests that the location where the modulation is generated should be relatively far from the neutron star, tending towards a disk-corona. If assuming a lamppost geometry of the corona, via fitting the spectra with the {\tt relxilllpcp} model, the value of corona height ($h$) is several $R_{\rm g}$ (see Table~\ref{tab:app} in the appendix), providing further evidence for the disk-corona model. This is because it would be challenging for a lamppost corona to remain unaffected at such a low height during the burst. In addition, \cite{2022Chen} mentioned that MAXI J1816--195 was significantly cooled during the burst, by considering that the photons in the 30--100\,keV range during the burst peak accounted for only \textasciitilde 30\% of the persistent flux, which indicates that temperature change of the corona is not the main factor accounting for the \textasciitilde2.5\,Hz modulation generation. \cite{2022Bellavita} used a variable Comptonization model ({\tt vKompth}) to explore the behavior of the spectra by independently varying the corona parameters. From their Figure 3, it can also be seen that changes in the electron temperature have little effect on the amplitude of the \textasciitilde2.5\,Hz modulation above 2\,keV, while the photon index (with the electron temperature fixed), which represents the optical depth, is the main parameter affecting the amplitude of the \textasciitilde2.5\,Hz modulation. Therefore, the \textasciitilde2.5\,Hz modulation of MAXI J1816--195 is likely caused by oscillations in the optical depth of the corona, either caused by the variations of the plasma density or by the corona size, or both. Together with the slight increase in intensity for periods with the \textasciitilde 2.5\,Hz modulation (Fig.~\ref{fig:PDS_lc_5}) and the intermittent phenomenon of the \textasciitilde2.5\,Hz modulation, we suggest that the oscillations in the optical depth are further likely to be related to the accretion variability of the neutron star.


%indicate a possible dependence of the QPO generation on the accretion process of neutron stars.


%This needs to be confirmed by more samples or simulations in the future. The slight increase in intensity (Fig.~\ref{fig:PDS_lc_5}) and  the intermittent phenomenon of \textasciitilde 2.5\,Hz QPO may indicate a possible dependence of the QPO generation on the accretion process of neutron stars.



%The value of corona height ($h$) listed in Table~\ref{tab:app} in the appendix, which was obtained by fitting the spectra using the {\tt relxilllpcp} model, is several times $R_{\rm g}$, providing further evidence for the disk-corona model. 

%The slight increase in intensity in Fig.~\ref{fig:PDS_lc_5} is most likely the result of this instability. And the intermittent phenomenon of \textasciitilde 2.5\,Hz Qpo may have some dependence on the accretion process of neutron stars. 

%The RMS of the \textasciitilde 2.5\,Hz QPO in MAXI J 1816--195 increased with energy from \textasciitilde 12 percent at 2\,keV to \textasciitilde 30 percent at 8\,keV in Fig.~\ref{fig:rms_energy_7}, while at higher energies it still remained a high value. Additionally, given that there are no oscillations below 2\,keV, and that the appearance of QPOs shown in Fig.~\ref{fig:hid_2} and Fig.~\ref{fig:PDS_lc_5} is strongly correlated with a harder spectrum, it is suggested that this \textasciitilde 2.5\,Hz QPO be produced from unstable coronas. The idea of quasi-periodic oscillations from accretion disk coronas was first proposed by \cite{1986Boyle}, who argued that optical depth regulated by the incident flux was linked to the QPO frequency. To explain the low-frequency QPO was observed to increase with increasing intensity in GX5--1 \citep[]{1985van}, Sco X--I \citep[]{1986Priedhorsky} and Cyg X--2 \citep[]{1986Hasinger}. Although the frequency and intensity of this paper remain relatively stable during the QPO process, the possibility of this model is not excluded. The model also predicts that when the source intensity is constant, the temperature of the electrons in the corona changes and the frequency may change. If an oscillation in the temperature of the corona contributed to this QPO, \cite{2022Chen} found that the cooling of the corona during MAXI J 1816--195 burst, contradicts the fact that \textasciitilde 2.5\,Hz QPO was not affected by the burst. However, considering the oscillation of an extended corona at a distant radius from a compact object, it is not impossible to see a QPO at such a low inclination. 


%Based on the rms and lag spectra of the QPOs in LMXBs, \cite{1998Lee}, \cite{2001Lee} and \cite{2014Kumar} proposed the Comptonization model, which is assumed any QPO can be described as an oscillation of the time-averaged spectrum, that is coupled to oscillations of other thermodynamic properties of the system, such as the coronal electron density, the corona temperature and seed-photon source temperature, etc. They explored how X-ray spectral variations depend on coronal properties and on the nature of the underlying driving modulation, and successfully explained kHz oscillations in NS-LMXBs. Although this model was originally used to explain kHz QPOs in neutron stars, \cite{2021Karpouzas}, \cite{2021Garc} and \cite{2022Bellavita} successfully applied the model to the time lags and rms of low-frequency QPOs in BH-LMXBs. All of the above results prove the significance of a variable corona arose from an oscillation in the temperature of the corona and seed photon, the coronal electron density or optical depth, but also superposition of multiple parameter oscillations for quasi-periodic oscillations in LMXBs. The slight increase in intensity in Fig.~\ref{fig:PDS_lc_5} is most likely the result of this thermal instability. And the intermittent phenomenon of \textasciitilde 2.5\,Hz Qpo may have some dependence on the accretion process of neutron stars. 





\section*{Acknowledgements}

We thank the anonymous referee for useful comments that have improved the paper. This work made use of the data and software from the High Energy Astrophysics Science Archive Research Center (HEASARC), provided by NASA’s Goddard Space Flight Center,and the \textit{Insight}-HXMT mission, funded by China National Space Administration (CNSA) and the Chinese Academy of Sciences (CAS). This work is supported by the National Key R\&D Program of China (2021YFA0718500). We acknowledge funding support from the National Natural Science Foundation of China (NSFC) under grant No. 12122306, No. U2038102, No. U2031205, No. U2038104, No. U1838201, No. U1838108, No. 12173103, the CAS Pioneer Hundred Talent Program Y8291130K2 and the Scientific and technological innovation project of IHEP Y7515570U1. Thanks to Thomas Dauser and Javier Garcia for their guidance on reflection models.


%%%%%%%%%%%%%%%%%%%%%%%%%%%%%%%%%%%%%%%%%%%%%%%%%%
\section*{Data Availability}

 
The data used for this article are publicly available in the High Energy Astrophysics Science Archive Research Centre (HEASARC) at \url{https://heasarc.gsfc.nasa.gov/cgi-bin/W3Browse/w3browse.pl}





%%%%%%%%%%%%%%%%%%%% REFERENCES %%%%%%%%%%%%%%%%%%

% The best way to enter references is to use BibTeX:

\bibliographystyle{mnras}
\bibliography{example} % if your bibtex file is called example.bib


% Alternatively you could enter them by hand, like this:
% This method is tedious and prone to error if you have lots of references
%\begin{thebibliography}{99}
%\bibitem[\protect\citeauthoryear{Author}{2012}]{Author2012}
%Author A.~N., 2013, Journal of Improbable Astronomy, 1, 1
%\bibitem[\protect\citeauthoryear{Others}{2013}]{Others2013}
%Others S., 2012, Journal of Interesting Stuff, 17, 198
%\end{thebibliography}

%%%%%%%%%%%%%%%%%%%%%%%%%%%%%%%%%%%%%%%%%%%%%%%%%%

%%%%%%%%%%%%%%%%% APPENDICES %%%%%%%%%%%%%%%%%%%%%

\FloatBarrier
\appendix
\section{Figures}
\label{A}
% Figure environment removed



\section{{\tt relxilllpCp} model fitting results}
\label{B}

We fit the broadband spectra of MAXI J1816--195 using {\tt relxilllpcp} with a lamp post geometry, and we fixed the energy of the two edges. The inclination angle is also fixed at 10. This does not affect the values of other spectral parameters. The best spectral parameter results using the {\tt relxilllpCp} model are presented in Table~\ref{tab:app} of the appendix.
%Since the paper mainly focuses on the disk-corona model,

\begin{table}

 %\tablewidth{\textwidth}
 \centering
 \caption{Best-fitting spectral parameters of the \textit{NICER} and \textit{NuSTAR} observations of MAXI J1816--195 using Model 3: {\tt const}$\times${\tt edge}$\times${\tt edge}$\times${\tt tbabs}$\times${\tt (relxilllpCp}$+${\tt diskbb)}. Uncertainties are at 90\%.}
 \renewcommand\arraystretch{1.36}
 %\setlength{\tabcolsep}{7mm}
 \begin{tabular}{ccc}
   \hline
   Model & Parament (unit) & Value \\
   \hline
   TBABS & $N_{\rm H}\ (\times10^{22}\ {\rm cm^{-2}})$ & $2.28^{+0.02}_{-0.02}$ \\
   EDGE & $E_{\rm edge,1} $(keV)& $1.84$\ (fixed)\\
   EDGE & $E_{\rm edge,2} $ (keV) & $0.90$\ (fixed)\\
   DISKBB & $kT_{\rm in}\ ({\rm keV})$  & $0.558^{+0.002}_{-0.003}$  \\
      & $N_{\rm diskbb}$  &$870^{+37}_{-22}$ \\
   
   RELXILLLPCP & $\Gamma$ & $1.93^{+0.02}_{-0.01}$\\
    & $kT_{\rm e}\ ({\rm keV})$ & $27.84^{+0.64}_{-0.52}$\\
    & $R_{\rm in}$\ $(\times R_{\rm ISCO})$ & $1.11^{+0.04}_{-0.04}$\\
              & $h$\ ($R_{\rm g}$)&$< 3.2394$\\
    & $a$ & 0.248\ (fixed)\\
    & Refl\_frac & $0.21^{+0.01}_{-0.02}$  \\
    & $i\ (^{\circ})$ & $10$\ (fixed)  \\
    & $\log\xi$ & $3.01^{+0.09}_{-0.14}$ \\
    & $\log N\ ({\rm cm^{-3}})$ & $18.97^{+0.09}_{-0.36}$\\
    & $A_{\rm Fe}$\ $(\times \rm solar)$ & $2.52^{+0.18}_{-0.16}$ \\
    & Norm & $0.162^{+0.007}_{-0.003}$\\
   \hline
    & $\chi^2$/d.o.f & 1.03\ (1726/1680) \\
   \hline
 \end{tabular}
 \label{tab:app}
\end{table}


%%%%%%%%%%%%%%%%%%%%%%%%%%%%%%%%%%%%%%%%%%%%%%%%%%


% Don't change these lines
\bsp	% typesetting comment
\label{lastpage}
\end{document}

% End of mnras_template.tex

