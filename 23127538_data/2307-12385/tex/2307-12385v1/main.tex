\documentclass[10pt,aps,prl,nofootinbib,superscriptaddress,twocolumn,preprintnumbers,balancelastpage]{revtex4-1}
%===============================================================================================================
\usepackage{amsmath,amssymb,amsthm,amsopn}
\usepackage{graphicx}
\usepackage[T1]{fontenc} 
\usepackage{tikz-feynman}
\usepackage{epsfig}
\usepackage{psfrag}
\usepackage[colorlinks=true,allcolors=blue!65]{hyperref}
\usepackage{multirow}
\usepackage{empheq}
\usepackage{slashed}
\usepackage{graphicx}
\usepackage{url}
\usepackage{subfigure}
\usepackage{textcomp}
\usepackage{bm}
\usepackage{dcolumn}
\usepackage{color,xcolor}
\usepackage{ulem}
\usepackage{cancel}
\usepackage{braket}
\usepackage[title]{appendix}
\usepackage{lipsum}
\usepackage{mathptmx}
%===============================================================================================================
\DeclareMathAlphabet{\mathcal}{OMS}{cmsy}{m}{n}

\def\red#1{\textcolor{red}{#1}}
\def\blue#1{\textcolor{blue}{#1}}
\def\rb#1{\textcolor{black}{#1}}
\def\rs#1{\textcolor{red}{\cancel{#1}}}
\def\mn#1{\marginpar[\tiny{\rr{#1}}]{\tiny{\rr{#1}}}}
\def\xo#1{\rr{{\xout{{#1}}}}}
\def\xc#1{\rr{{\cancel{#1}}}}

\def\h{\hat}
\def\d{\delta}
\def\L{\Lambda}
\def\ep{\epsilon\!\!\!/}
\def\comment#1{}
\def\bra#1{\mathinner{\langle{#1}|}}
\def\ket#1{\mathinner{|{#1}\rangle}}
\def\Bra#1{\left<#1\right|}
\def\Ket#1{\left|#1\right>}
\def\Imm#1{\Im\mathrm{m}\{#1\}}

\def\beq{\begin{equation}}
\def\eeq{\end{equation}}
\def\bea{\begin{eqnarray}}
\def\eea{\end{eqnarray}}
\def\h{\hat}
\def\d{\delta}
\def\L{\Lambda}
\def\ep{\epsilon\!\!\!/}

\newcommand{\mac}[1]{{\color{red!80!blue}\textsc{#1}}}
\newcommand{\mat}[1]{{\color{red!80!yellow}#1}}

%===============================================================================================================
\begin{document}

\title{A Heavy QCD Axion model in Light of Pulsar Timing Arrays}

\author{Moslem Ahmadvand}
\email[]{ahmadvand@ipm.ir}
\affiliation{School of Particles and Accelerators,  Institute Research \\ in Fundamental Sciences (IPM)
P. O. Box 19395-5531, Tehran, Iran}
\author{Ligong Bian}
\email[]{lgbycl@cqu.edu.cn}
\affiliation{Department of Physics and Chongqing Key Laboratory for Strongly Coupled Physics,  \\ Chongqing University, Chongqing 401331, China}
\affiliation{Center for High Energy Physics, Peking University, Beijing 100871, China}
\author{ Soroush Shakeri}
\email[]{s.shakeri@iut.ac.ir}
\affiliation{Department of Physics, Isfahan University of Technology, Isfahan 84156-83111, Iran}
\affiliation{ICRANet, Piazza della Repubblica 10, I-65122 Pescara, Italy}


%\author{She-Sheng Xue}
%\email[]{xue@icra.it}
%\affiliation{ICRANet Piazzale della Repubblica, 10 -65122, Pescara, Italy, \\ Physics Department, University of Rome La Sapienza, \\ P.le Aldo Moro 5, I–00185 Rome, Italy}





% e-mail addresses: one for each author, in the same order as the authors

%In this paper, we delve into the recent discoveries and groundbreaking results brought forth by the Nanograv collaboration in the domain of axion research.%


\date{\today}

\begin{abstract}
Recently, pulsar timing array (PTA) experiments reported the observation of a stochastic gravitational wave (GW) background in the nanohertz range frequency band. We show that such signal can be originated from a cosmological first-order phase transition (PT) within a well-motivated heavy (visible) QCD axion model. Considering the Peccei-Quinn symmetry breaking at the TeV scale in the scenario, we find a supercooled PT, in the parameter space of the model, prolonging the PT with the reheating temperature at the GeV scale.

\end{abstract}

\maketitle
\section{I. Introduction}
Gravitational wave (GW) experiments have provided new paths to explore the Universe and examine models of physics. Recently, PTA experiments, including the North American Nanohertz Observatory for Gravitational Waves (NANOGrav) \cite{NANOGrav:2023gor}, the European Pulsar Timing Array (EPTA) \cite{Antoniadis:2023ott} along with Parkes Pulsar Timing Array (PPTA) \cite{Reardon:2023gzh} and Chinese Pulsar Timing Array (CPTA) \cite{Xu:2023wog} %Indian Pulsar Timing Array (InPTA) \cite{bibid} 
 released their latest results observing significant evidence of a GW following the Hellings-Downs pattern \cite{1983ApJ...265L..39H} in the angular cross-correlation of pulsar timing residuals, supporting a stochastic gravitational background (SGWB).

Various sources can be the origin of these signals, for instance supermassive black hole binary mergers have been proposed as an astrophysical candidate \cite{NANOGrav:2023hvm,Ellis:2023dgf,Bi:2023tib}. However, such stochastic signals may be come from a relic cosmic background similar to the cosmic microwave background (CMB). In addition, cosmological sources such as first-order phase transition (FOPT) around the QCD scale \cite{Ellis:2023tsl,Ratzinger:2020koh,Addazi:2023jvg,Ghosh:2023aum,Rezapour:2020mvi,Rezapour:2022iqq,Neronov:2020qrl,Ahmadvand:2017tue,Ahmadvand:2017xrw,Athron:2023mer}, cosmic strings \cite{Blasi:2020mfx,Ellis:2020ena,Buchmuller:2020lbh,Lazarides:2023ksx,Wang:2023len,Bian:2022tju}, domain walls \cite{Bai:2023cqj,Kitajima:2023cek,Blasi:2023sej,Li:2023yzq,Bian:2022qbh}, primordial black holes \cite{Depta:2023qst,Gouttenoire:2023nzr,Franciolini:2023pbf,Guo:2023hyp,Gouttenoire:2023nzr} and inflation \cite{Firouzjahi:2023ahg,Vagnozzi:2020gtf,Ashoorioon:2022raz,Niu:2023bsr,Servant:2023mwt} may fit the recent data better \cite{NANOGrav:2023gor}. There are also some recent discussions about the lake of strong preferences for any specific SGWB sources with the current data based on Bayesian analysis \cite{Bian:2023dnv}. 


An intriguing candidate that we propose in this paper is a supercooling PT which ends around the $100\,\mathrm{MeV}$ scale and such a PT can naturally be accommodated in the context of heavy QCD axion models \cite{Rubakov:1997vp,DiLuzio:2020wdo}. A cosmic FOPT may be accomplished due to a spontaneous broken symmetry and the process is accompanied by the nucleation of bubbles separating true and false vacua. In a supercooled PT, the vacuum energy is dominated and the Universe remains in the false vacuum for a long period and supercools, increasing the PT duration \cite{Ellis:2019oqb,Brdar:2018num,Kobakhidze:2017mru,VonHarling:2019rgb,Athron:2023mer,Yang:2023qlf}.   

In this work, we consider a scenario within heavy axion models which are very appealing in that not only the strong CP problem \cite{Peccei:1977hh} can be addressed, but also the models have richer phenomenology \cite{Berezhiani:2000gh,Dimopoulos:2016lvn} relative to the invisible ones \cite{Zhitnitsky:1980tq,Dine:1981rt,Kim:1979if,Shifman:1979if}. Moreover, the relation of the axion mass, $m_a$, and its decay constant, $f_a$ is modified so that larger $m_a$ and lower $f_a$ are allowed. These models are also motivated in connection with the Peccei-Quinn (PQ) quality problem \cite{Georgi:1981pu,Kamionkowski:1992mf,Holman:1992us}.

We explore a supercooled PT with a U(1) PQ symmetry breaking at the TeV scale. Based on the nearly conformal dynamics of the PQ scalar field, we analytically find important parameters encoding the GW spectrum and show that the corresponding signals can be well fitted to the PTA data. 




\section{II. PQ phase transition in the model}

The CP-violating $\bar{\theta}$ parameter in the strong interactions is experimentally bounded $\bar{\theta}\lesssim 10^{-10} $ \cite{Baker:2006ts} and the nature of this smallness, known as the strong CP problem, is a big puzzle in particle physics. One of the interesting solutions to this problem is based on a $U(1)$ global symmetry, $U(1)_{\mathrm{PQ}}$, first proposed by Peccei and Quinn \cite{Peccei:1977hh}. At energies higher than the electroweak (EW) scale, the symmetry is spontaneously broken and the axion as the pseudo Nambu-Goldstone boson is generated. The interaction of the axion with gluons at the QCD scale, $ (a/f_a +\bar{\theta})G\widetilde{G} $, and as a result the vacuum expectation value of the axion can cancel the $\bar{\theta}$ term \cite{Peccei:2006as}. Considering the astrophysical constraints on the axion decay constant, $10^8\,\mathrm{GeV}\lesssim f_a \lesssim 10^{17}\,\mathrm{GeV}$ \cite{Raffelt:2006cw,Arvanitaki:2009fg,Shakeri:2022usk}, there are two classes of models, known as invisible axion models \cite{Zhitnitsky:1980tq,Dine:1981rt,Kim:1979if,Shifman:1979if}. However, as mentioned before, $f_a$ can be lowered, $f_a\sim (1-100)\,\mathrm{TeV}$ in the co-called visible axion models, still addressing the strong CP problem. In this sense, there are diverse scenarios, e.g. enlarging the QCD color gauge group $SU(3+\mathcal{N})$ \cite{Gherghetta:2016fhp} or assuming some hidden sector with a confining scale larger than the QCD cut-off $\Lambda_{\mathrm{QCD}}$ such that additional terms are supplemented to the axion potential and change the axion mass \cite{Berezhiani:2000gh,Fukuda:2015ana}. In the case of adding a hidden sector, one can consider a whole copy of the standard model (SM), with $SM\times SM'$ gauge group, and particles of the two sectors can interact gravitationally or by some very feebly couplings. Furthermore, a $Z_2$ mirror symmetry between  particles of the two sectors can be imposed and soft terms breaking the symmetry can induce $\Lambda'_{\mathrm{QCD}}\gg \Lambda_{\mathrm{QCD}}$ \cite{Berezhiani:2000gh,Fukuda:2015ana}.

Here, we consider a DFSZ axion model \cite{Zhitnitsky:1980tq,Dine:1981rt}, containing a gauge-singlet PQ scalar field, $\Phi$, and two Higgs doublets under $SU(2)_L$, $H_u$ and $H_d$, which is supplemented with its hidden copy. Indeed, a theory with $SM\times SM'$ gauge group and each sector interacting with the PQ scalar (see appendix A) and has the same PQ symmetry.\footnote{An analogous procedure can also be applied to a KSVZ axion model \cite{Kim:1979if,Shifman:1979if}.} Therefore, the axion couples to both QCD sectors
\begin{equation}
\frac{a}{f_a}\left(G\widetilde{G}+G'\widetilde{G}'\right),
\end{equation}
and its mass would be $m_a\sim \Lambda_{\mathrm{QCD}}'^2/f_a$ \cite{Berezhiani:2000gh,Fukuda:2015ana} with $f_a\sim \mathrm{TeV}$.
In this work, we do not go through axion interactions and focus on the UV theory and study the PQ PT associated with the PQ symmetry breaking.
We explore a supercooled PQ phase transition along the direction of PQ scalar dynamic, $\langle\Phi\rangle= v_{\phi}/\sqrt{2} $. We consider the case where mass parameters are small, $ \mu_d^2, \mu_u^2, \lambda_{\phi}f^2\ll f^2$ (see Appendix A) with the similar condition in the hidden sector, and main contributions to the potential arises from the one-loop Coleman-Weinberg (CW) quantum correction \cite{Coleman:1973jx} and the leading contribution from thermal corrections \cite{Quiros:1999jp}, thereby masses and the symmetry breaking scale are generated radiatively \cite{Gildener:1976ih}.

At the zero temperature limit, quantum corrections (see Appendix B) contribute to the potential that is approximately scale invariant as
\begin{equation}
V=\left(A+B\ln\frac{\phi^2}{f_a^2}\right)\phi^4\equiv \lambda(\phi)\phi^4. 
\end{equation}
where
\begin{equation}
	A=\frac{1}{64\pi^2}\left(\kappa_{u}^2\left(\ln(\kappa_{u})-\frac{3}{2}\right)+\kappa_u'^{2}\left(\ln(\kappa'_{u})-\frac{3}{2}\right)\right),
\end{equation}
\begin{equation}
	B=\frac{1}{64\pi^2}\left(\kappa_u^2+\kappa_u'^{2}\right).
\end{equation}
For the sake of simplicity, we only considered $\kappa_u$ and $\kappa_u'$ couplings which are determined at the vacuum and are then evolved to a scale $\phi$ through the beta function (see Appendix B). Thus, at the non-zero vacuum, $ v_{\phi}(T=0)\equiv f_a $, from $(dV/d\phi) |_{f_a}=0$ we find
\begin{equation}\label{min}
-4\lambda=\frac{d\lambda}{d\ln \phi},~~~~~\Delta V\equiv V(0)-V(f_a)=\frac{Bf_a^4}{2}=\frac{(\bar{\kappa}_u^2+\bar{\kappa}_u'^{2})f_a^4}{128\pi^2}.
\end{equation}
Considering thermal corrections (see Appendix A) and high temperatures, the origin point would be the minimal, and the potential would be
\begin{equation}\label{poten}
V=DT^2\phi^2+\lambda\phi^4+\cdots,  
\end{equation}
where 
\begin{equation}
D=\frac{1}{24}\left(\kappa_u+\kappa_u'\right).  
\end{equation}
As temperature goes down, negative values of $\lambda$ can change the potential slope and induce a bump so that at the critical temperature, $T_c$, two degenerate states, $\Delta V(T)= V(0, T)-V(v_{\phi}(T), T)=0$, indicating a FOPT, can be generated due to these values of $\lambda$, Fig.\ (\ref{fig1}). At temperatures below $T_c$, the vacuum $v_{\phi}(T)$ is the favorable one and at some temperature $T_n$ bubbles of the new phase are nucleated. In fact, bubble nucleation occurs when the bubble formation probability per unit Hubble space-time volume, which is proportional to $ \exp \left(-S_3(T)/T\right)$ where $S_3(T)$ is the bounce action quantifying the tunneling process, would be of the order of one. As a result, we can find the nucleation temperature $T_n$ via the following relation \cite{Linde:1981zj,Arnold:1991cv}
\begin{equation}\label{nuc}
	\frac{S_3(T_n)}{T_n}\sim 4\ln\Big(\frac{T_n}{H(T_n)}\Big), 
\end{equation}
where the Hubble parameter $H$ is given by
\begin{equation}
	H^2=\frac{1}{3 M_{\mathrm{Pl}}^2}\left(\xi T^4+\Delta V(T)\right),
\end{equation}
where the reduced Planck mass is $ M_{\mathrm{pl}}\simeq 2.43\times 10^{18}\,\mathrm{GeV}$, $\xi= \pi^2g_*/30$, and $g_*\simeq 107$ is the effective number of relativistic degrees of freedom, assuming that it does not change during the PT. At low temperatures, $T\ll T_c$, it is expected $v_{\phi}(T)=f_a$, and $\Delta V(T)$ is well approximated by $\Delta V$. Furthermore, in this limit, the bounce action can be well approximated by \cite{Witten:1980ez}
\begin{equation}
	\frac{S_3}{T} \simeq -18.9 \frac{\sqrt{2 D}}{\lambda(T)} \simeq -5.4 \frac{\sqrt{\kappa_u+\kappa_u'}}{\lambda(T)}   
\end{equation} 
and hence we can find the nucleation temperature as
\begin{equation}\label{nuc}
	4\ln\Big(\frac{\sqrt{3}M_{\mathrm{Pl}} T_n}{\sqrt{\xi T_n^4+\Delta V}}\Big)\simeq -5.4 \frac{\sqrt{\kappa_u+\kappa_u'}}{\lambda(T)}.   
\end{equation}
At the time when the transition completes, from the conservation of energy, one obtains $\rho_R(T_*)=\rho_R(T_p)+\Delta V $ \cite{Ellis:2019oqb}  where $\rho_R(T)=\xi T^4 $ is the radiation energy and $T_p$ is the percolation temperature. Assuming $T_p\simeq T_n$, the reheating temperature can be computed by
\begin{equation}
T_*^4=\frac{\Delta V }{\xi} +T_n^4. 
\end{equation}
For the case of fast reheating $H_*\equiv H(T_*)\simeq H(T_n)$. 
Another important quantity in characterizing the generated GWs is the inverse of PT duration calculated by the following relation
\begin{equation}\label{beta}
\frac{\beta}{H_*}=T_n\frac{d}{dT}\left(\frac{S_3(T)}{T}\right)\Bigg |_{T_n}\simeq -\frac{S_3(T_n)}{T_n}\frac{\beta_{\lambda}(T_n)}{\lambda(T_n)}, 
\end{equation}
where $(d\lambda(T)/d\ln T)|_{T_n}=\beta_{\lambda}(T_n)$. From Eq.\ (\ref{poten}) and $ V(\phi_0, T)=0$, for $T\ll T_c$, and $ \phi_0\sim T$ we obtain $\lambda(T_n)\sim A+B\ln (T_n/f_a) $. Therefore, expressing  $-\lambda(T_n)/\beta_{\lambda}(T_n)\sim \gamma\ln (f_a/T_n)$, we find
\begin{equation}\label{beta}
\frac{\beta}{H_*}\simeq \frac{4}{\gamma\ln (f_a/T_n)}\ln\Big(\frac{\sqrt{3}M_{\mathrm{Pl}} T_n}{\sqrt{\xi T_n^4+\Delta V}}\Big). 
\end{equation}
Expecting $\beta/H_*\sim \mathcal{O}(1-10)$ for the case of supercooling, we fix the parameter $\gamma$ with the data. 

The strength of the supercooled PT is also obtained by
\begin{equation}\label{al}
	\alpha =\frac{\Delta V}{\rho_R(T_n)}.
\end{equation}
In the next section, calculating the aforementioned quantities, we obtain the GW energy density spectrum.  
% Figure environment removed

\section{III. SGWB signal}
In this section, based on the obtained PT quantities, we calculate the GW spectrum of the supercooled PT. In this sense, sources contributing to the GW production are bubble wall collisions and fluid motions. Relying on the recent numerical computation \cite{Lewicki:2022pdb}, the present-day GW energy density spectrum is obtained by 
\begin{equation}
	h^2\Omega _{\mathrm{GW}}(f)=1.67\times 10^{-5}\Big(\frac{H_*}{\beta}\Big)^{2}\Big(\frac{\kappa \alpha}{1+\alpha}\Big)^2 \Big(\frac{100}{g_*}\Big)^{\frac{1}{3}}S(f),
\end{equation}
where the spectral shape of the GW is obtained by
\begin{equation}\label{spec}
	S(f)=\frac{\bar{A}(a+b)^c}{\left[b\left(\frac{f}{f_{p,0}}\right)^{-\frac{a}{c}}+a\left(\frac{f}{f_{p,0}}\right)^{\frac{b}{c}}\right]^c},
\end{equation}
and $\kappa$ is the fraction of the vacuum energy converted to the kinetic
energy of bubble walls, $\kappa_{\mathrm{vac}}\simeq 1/(1+5/(\beta R_{\mathrm{eq}}))$, and fluid motions $\kappa_{\mathrm{f}}=1-\kappa_{\mathrm{vac}}$ \cite{Lewicki:2022pdb}. In our case with $\alpha\gg 1 $, bubbles collide in the vacuum and the bubble wall velocity can reach the speed of light, thus we consider $\kappa_{\mathrm{vac}}\simeq 1$ corresponding to $\beta R_{\mathrm{eq}}\gg 1$.
Therefore, according to the table I of Ref. \cite{Lewicki:2022pdb}, corresponding fit parameters of the spectral function appropriate for a global broken symmetry, $T_{rr}\propto R^{-2}$ (where $T_{rr} $ is the maximum of the stress-energy tensor and $R$ is the bubble radius), would be as $\bar{A}\simeq 5.93\times 10^{-2}$, $a\simeq 1.03$, $b\simeq 1.84$, and , $c\simeq 1.45$, whereas for the envelop approximation $\bar{A}\simeq 3.78\times 10^{-2}$, $a\simeq 3.08$, $b\simeq 0.98$, and , $c\simeq 1.91$.

The present red-shifted peak frequency is given by
\begin{equation}
	f_{p,0}=16.5\times 10^{-6}[\mathrm{Hz}] \Big(\frac{f_p}{\beta}\Big)\Big(\frac{\beta}{H_*}\Big)\Big(\frac{T_*}{100~\mathrm{GeV}}\Big)\Big(\frac{g_*}{100}\Big)^{\frac{1}{6}},
\end{equation}
where for the case of $T_{rr}\propto R^{-2}$, $f_p/\beta \simeq 0.64/(2\pi)$ and for the envelope approximation $f_p/\beta \simeq 1.33/(2\pi)$ \cite{Lewicki:2022pdb}.

As can be seen from  Fig.\ (\ref{fig2}), using datasets of NANOGrav 15-yr, EPTA release-2, and PPTA DR3, for $f_a=1\,\mathrm{TeV}$, and some representative values of parameters, $\bar{\kappa}_u=\bar{\kappa}'_{u}=0.0001$, $\kappa_u=\kappa'_{u}=0.001$ and $\lambda(T)=-0.0015$, which correspond to $T_n=100\,\mathrm{MeV}$, $T_*=1\,\mathrm{GeV}$, $\alpha=8995$, and $\beta/H_*=17, 3.4, 1.7$ with the corresponding parameter $\gamma=1, 5, 10$ respectively, the GW signals can be consistent very well with the PTA data from the bubble collision spectra for both the spectral shape of $T_{rr}\propto R^{-2}$ (solid line)  and for the envelope (dashed line). Remarkably, we see that the high frequency range of the GW signals with the envelope approximation falls within the sensitivity range of future space-based GW experiments such as LISA, DECIGO, and BBO and can be probed by these detectors. 

Furthermore, using the PTA data, we perform a best fit analysis over a range of $ \gamma$ parameter ($\beta/H_*$) through $\chi^2$ test based on the following relation 
\begin{equation}
\chi^{2}=\sum_{i=1}^{N} \frac{\left(\log_{10}\Omega_{\mathrm{th}}h^{2}-\log_{10}\Omega_{\mathrm{exp}}h^{2}\right)^{2}}{2\bar{S_{i}}^{2}},
\end{equation}
where $\Omega_{\mathrm{th}}h^2$ denotes the GW predicted by the model, $\Omega_{\mathrm{exp}}h^2$ represents the observed GW signal by PTA experiments, and $\bar{S_{i}}$ is  the deviation from the midpoint value of each data point in $\log_{10}\Omega_{\mathrm{exp}}h^2$ within the uncertainty range. As shown in Figs.\ (\ref{fig4}, \ref{fig5}) the best fit point of the GW energy spectrum taking into account all data sets derived from  $T_{rr}\propto R^{-2}$ case for the aggregated data set is $\gamma=3.28$ ($\beta/H_*=5.1$) and for the envelope case, the best fit point would be $\gamma=13.3$ ($\beta/H_*=1.3$).

\section{IV. Discussion}

Another PQ PT consequence of the model is that the symmetry breaking may induce similar or different EW symmetry breaking scale in the ordinary and hidden sector. For instance, as for  $v_{\mathrm{EW}}'$, this can be realized by these terms of the Lagrangian $\kappa'_{u} h_u^2 f_a^2+\lambda'_u h_u^4$, expressed in terms of the electrically neutral component of the fields, and hence $v_{\mathrm{EW}}'^2\simeq \kappa'_{u} f_a^2/\lambda'_u$. Setting $\kappa'_{u}=0.0001$ and $\lambda'_{u}=0.1$, we obtain $v_{\mathrm{EW}}'=1\,\mathrm{GeV}$. Even for these values of $v_{\mathrm{EW}}'$ its QCD scale could be $\Lambda'_{\mathrm{QCD}}\gtrsim v'_{\mathrm{EW}}$ \cite{Fukuda:2015ana}. Consequently, in this case, we obtain the axion mass $m_a\gtrsim 1\,\mathrm{MeV}$, while considering $ v'_{\mathrm{EW}}\simeq v_{\mathrm{EW}}<f_a$ and assuming $\Lambda'_{\mathrm{QCD}}\lesssim v_{\mathrm{EW}} $, one finds $m_a \lesssim  10\,\mathrm{GeV}$.

Heavy axions may interact with visible and hidden (dark) photons 
\begin{equation}
\frac{a}{f_a}\left(F\widetilde{F}+F'\widetilde{F}'\right),
\end{equation}
so that the axion decay rate would be
\begin{equation}
\Gamma(a \rightarrow \gamma \gamma)=\frac{g_{a \gamma}^2 m_a^3}{64 \pi}, \quad \Gamma\left(a \rightarrow \gamma^{\prime} \gamma^{\prime}\right)=\frac{g_{a \gamma}^{\prime 2} m_a^3}{64 \pi}
\end{equation}
where $g_{a\gamma}$ and $g_{a \gamma}^{\prime}$ are proportional to $E/N$, where $E$ and $N$ are the electromagnetic and QCD anomaly coefficients, respectively, \cite{DiLuzio:2020wdo}. Indeed, Decaying to these photons contributes to the effective number of relativistic species which is strongly constrained by CMB measurements, $N_{\mathrm{eff}}=2.96^{+0.34}_{-0.33}$ \cite{Planck:2018vyg}, putting the most stringent constraints on the heavy QCD axion mass. According to the analysis of Ref.\ \cite{Dunsky:2022uoq}, for $E/N=8/3$ which may be the case of a DFSZ model  \cite{Ahmadvand:2021vxs}, $m_a<100\,\mathrm{MeV}$ would be excluded. However, lower masses, $1\,\mathrm{MeV}<m_a<100\,\mathrm{MeV}$ where the dark photon contribution to $N_{\mathrm{eff}}$ compensates neutrino dilution from axion decays to photons, are allowed for $E/N=1/3$. Complementary limits may be provided by collider experiments \cite{Bertholet:2021hjl}.

Moreover, the heavy axion cannot be a dark matter candidate. However, some stable particles in the hidden sector may contribute to the dark matter relic abundance. 
We leave detailed calculations in connection with axion interactions and possible dark matter candidates in the presented scenario to a future work.



% Figure environment removed



%The GW energy spectra derived from different models employing the best-fit parameters from current new datasets of PPTA, EPTA and NANOGrav. The violin plots represent the first five fre- quency bins of the corresponding datasets.
% Figure environment removed

% Figure environment removed


\section{V. Conclusion}
With regard to the recently detected GW signals by PTA experiments, we have proposed a heavy QCD axion scenario, with a PQ symmetry breaking scale around the TeV scale, based on a (DFSZ) axion model which is supplemented with its hidden copy. We investigated a supercooled PQ FOPT which is derived by CW quantum corrections and analytically found important PT quantities, including the reheating temperature, the inverse duration of the PT, and the strength of the PT. We have shown that within the parameter space of the model the generated GWs from the bubble wall collisions can be consistently fitted with the recently observed PTA data. Furthermore, it is shown that the high frequency range of such GW signals can be probed by future space-based GW detectors. 

In addition, Heavy axion models are well-motivated scenarios allowing $(m_a, f_a)$ relation to be relaxed, still addressing the strong CP problem. It is shown that considering the EW symmetry breaking scale in the ordinary and hidden sectors induced after the PQ symmetry breaking, a range of axion masses, $1\,\mathrm{MeV}\lesssim m_a \lesssim 10\,\mathrm{GeV}$ can be obtained. Nevertheless, due to the dark photon contribution to $N_{\mathrm{eff}}$, viable regions of the mass space are limited via CMB observation. 

As a result, interpreting the recent GW data, the model provides a setup which can be probed by future GW detectors, CMB telescopes and collider experiments.


\section{ACKNOWLEDGMENTS}

SS thank Fazlollah Hajkarim and Seyed Mohammad Mahdi Sanagostar for helpful discussions about data analysis. This work is supported in part by the National Key Research and Development Program of China under Grant No. 2021YFC2203004, and the National Natural Science Foundation of China (NSFC) under Grants No. 12075041 and No. 12147102. 

\section{Appendix A: The scalar potential}\label{ap}
The tree level scalar potential with the PQ symmetry is given by \cite{Ahmadvand:2021vxs}
\begin{align} \label{pot}	
	&V(\Phi, H_u, H_d) = \lambda_{\phi} \left(|\Phi|^{2} -f^{2}\right)^{2} + \left|H_{d}\right|^{2} \left(\kappa_{d}\, |\Phi|^{2} -\mu_{d}^{2}\right)\nonumber\\ &+ \left|H_{u}\right|^{2} \left(\kappa_{u}\, |\Phi|^{2} - \mu_{u}^{2}\right) -\left(\kappa_{ud}\, \Phi H_{u}^{\dagger}\, H_{d} + \text{H.c.}\right)\nonumber\\
	& + \lambda_{d} \left|H_{d}\right|^{4} + \lambda_{u} \left|H_{u}\right|^{4} + \lambda_{ud} \left(\left|H_{u}\right|^{2}\left|H_{d}\right|^2-\left|H_{u}^{\dagger} H_{d}\right|^{2} \right).
\end{align}
We also assume analogous potential with $H'_{d}$ and $H'_{u}$ from the hidden sector. 

\section{Appendix B: Quantum and thermal corrections}\label{app}

The one-loop Coleman-Weinberg quantum correction is given by 
\begin{equation}
	\begin{aligned}
		V_{\mathrm{CW}}\left(\phi\right)&=\sum_{i}(-1)^{F_{b/f}} g_{i} \frac{m_{i}^{4}\left(\phi\right)}{64 \pi^{2}}\left[\ln \left(\frac{m_{i}^{2}\left(\phi\right)}{\Lambda^{2}}\right)-c_{i}\right]
	\end{aligned}
\end{equation}
where $ F_{b/f}=1(0)$ for fermions (bosons), $ g_i$ is the number of degrees of freedom for a given field, and $c_i=3/2(5/2)$ for scalars and fermions (vectors).

For our model, considering only $\kappa_{u}$ and $\kappa'_{u}$ couplings, $m^2(\phi)=\kappa_{u}\phi^2$ (and the same with $\kappa'_{u}$), one can write

\begin{equation}
	V_{\mathrm{CW}}=\left(A+B\ln\frac{\phi^2}{\Lambda^2}\right)\phi^4\equiv \lambda(\phi)\phi^4,
\end{equation}
where
\begin{equation}
	A=\frac{1}{64\pi^2}\left(\kappa_{u}^2\left(\ln(\kappa_{u})-\frac{3}{2}\right)+\kappa_u'^{2}\left(\ln(\kappa'_{u})-\frac{3}{2}\right)\right),
\end{equation}
\begin{equation}
B=\frac{1}{64\pi^2}\left(\kappa_u^2+\kappa_u'^{2}\right).
\end{equation}

The running couplings at the scale $\phi$ are also obtained by the leading-order beta function
\begin{equation}
\frac{d\kappa_u}{d\ln\phi}\simeq \frac{\kappa_u^2}{16\pi^2}+\frac{\kappa_u \kappa_u^{\prime}}{8\pi^2}
\end{equation}
\begin{equation}
\frac{d\kappa_u^{\prime}}{d\ln\phi}\simeq \frac{\kappa_u'^{2}}{16\pi^2}+\frac{\kappa_u \kappa_u^{\prime}}{8\pi^2}
\end{equation}
The one-loop thermal correction is expressed as
\begin{equation}
	V_{T}\left(\phi, T\right)=\sum_{i}(-1)^{F_{b/f}} g_{i} \frac{T^{4}}{2 \pi^{2}} J_{b / f}\left[\frac{m_{i}^{2}\left(\phi\right)}{T^{2}}\right]
\end{equation}
where 
\begin{equation}
	J_{b/f}\left(y^{2}\right)=\int_{0}^{\infty} d x\, x^{2} \ln \left[1 \mp e^{-\sqrt{x^{2}+y^{2}}}\right]
\end{equation}
and these thermal functions in the high temperature limit is given by
\begin{align}\label{therm}
	J_{b}\left(\frac{m^{2}}{T^{2}}\right)&=-\frac{\pi^{4}}{45}+\frac{\pi^{2}}{12}\left(\frac{m}{T}\right)^{2}-\frac{\pi}{6}\left(\frac{m^{2}}{T^{2}}\right)^{3 / 2}\nonumber\\&-\frac{1}{32}\left(\frac{m}{T}\right)^{4} \ln \left(\frac{m^{2}}{a_{b} T^{2}}\right)+\cdots
\end{align}
\begin{equation}
	J_{f}\left(\frac{m^{2}}{T^{2}}\right)=\frac{7 \pi^{4}}{360}-\frac{\pi^{2}}{24}\left(\frac{m}{T}\right)^{2}-\frac{1}{32}\left(\frac{m}{T}\right)^{4} \ln \left(\frac{m^{2}}{a_{f} T^{2}}\right)+\cdots
\end{equation}
where $ \ln(a_b)=5.4076 $ and $ \ln(a_f)=2.6351 $. Thus, in the high temperature limit one has
\begin{equation}
	V_{T}\left(\phi, T\right)=\frac{1}{24}\left(\kappa_u+\kappa_u'\right)T^2\phi^2+\cdots.
\end{equation} 








\bibliography{references.bib}



%%%%%%%%%% Merge with supplemental materials 

%\begin{equation*}
%\mathcal{P}(q_x,L,\ell) = \Big[\frac{\ell q_{x}\cos(\frac{L}{2\ell})\sin(\frac{Lq_{x}}{2}) -\sin(\frac{L}{2\ell})\cos(\frac{Lq_{x}}{2})}{\ell^{2}q_{x}^{2}-1}\Big]^2.
%\end{equation*}
 




\end{document}

