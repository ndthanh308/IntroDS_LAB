
\section{Attack Taxonomy}
\label{sec:attack-framework}

In this section, we present a comprehensive taxonomy of wallet attack vectors, systematically examining the methods, techniques, and targeted components involved. Building on our generalised wallet mechanisms and threat model taxonomy, we outline a broad spectrum of attacks, as illustrated in \autoref{fig:wallet-attacks}. These attacks are categorised according to the specific functions and components targeted within the wallet infrastructure (see \autoref{sec:wallet_mechanism}) and the threats exploited (see \autoref{sec:threat_class}). We further incorporate the infrastructure layer of our design taxonomy to capture the multi-layered nature of these threats, as summarised in \autoref{tab:attack_vectors}.





\subsection{Network}
\label{sec:network-attacks} 

\subsubsection{Connection Hijack}
\label{sec:mitm}

These attacks aim to compromise the communication channel between wallets and other network participants using \acf{mitm} attacks to intercept and modify the \textcolor{teal}{\textit{txn}} message generated by \hyperref[algo:transaction-signing]{Algorithm 2}. Various types of \acs{mitm} attacks include Rogue \acs{ap} \cite{Hu2021SecurityCountermeasures}, \acs{dns} spoofing \cite{Ahmed2017MitigatingNetworking, Al-Mashhadi2020ASystems}, \acs{ip} spoofing \cite{shrivas2020disruptive}, and \acf{bgp} hijacking \cite{ekparinya2018impact}, as shown in \autoref{tab:attack_vectors}. Hardware wallets are vulnerable to these attacks if the online wallet client (see \autoref{sec:hardware-wallets}) is compromised. Ledger has previously reported susceptibility to \acs{mitm} attacks.

The Rogue \acf{ap} vector functions through unauthorised WiFi hotspots that can intercept transactions by exploiting the \textcolor{orange}{\textit{txnInit}} function. This allows an attacker to modify \textcolor{teal}{\textit{state\_trans\_info}} before blockchain forwarding, potentially redirecting funds to a different address than the recipient's address \cite{Hu2021SecurityCountermeasures}. The \acf{dns} spoofing vector occurs when a \acs{dns} resolver, which translates human-readable domain names into IP addresses, is compromised \cite{pillai2019smart}. This leads to fraudulent cryptocurrency service website redirection. One notable example is the 2017 EtherDelta DNS hijack, where attackers altered \acs{dns} records to redirect users to a phishing clone \cite{CryptocurrencyScheme}. An attacker can also execute a \acf{bgp} hijacking attack that maliciously advertises false \acs{bgp} routes to divert traffic intended for legitimate blockchain nodes (see \hyperref[algo:transaction-broadcast]{Algorithm 3}) or wallet \acs{api} endpoints \cite{ekparinya2018impact}. The MyEtherWallet attacker employed the \acs{bgp} hijacking vector \cite{myetherwallet}. Another connection hijack avenue, the \acf{arp} spoofing vector, is initiated when attackers broadcast fraudulent \acs{arp} messages across a local network. This links their MAC address with the \acs{ip} address of a legitimate network host to redirect the user's transaction data generated in \hyperref[algo:transaction-signing]{Algorithm 2} \cite{Hu2021SecurityCountermeasures, ekparinya2018impact}.

% recipient's address (\teal{$\alpha_R$})


\subsubsection{Service Denial}
\label{sec:dos}

This is executed using adversary-controlled devices to orchestrate \acf{ddos} attacks which overwhelm the network infrastructure with an excessive volume of requests, causing a decline or cessation of wallet operations (see \autoref{sec:wallet_mechanism}) \cite{ChandanProtectiveCryptocoin}. These attacks often target the \acf{icmp}, \acf{tcp} handshake mechanism, and other network infrastructure \cite{chaganti2022comprehensive}. One common medium of conducting a \acs{ddos} attack is through botnets, which involves an adversary using a network of computers \cite{krombholz2015advanced}.

The \acf{icmp} flooding vector overloads a wallet with network requests (\acs{icmp} echo request packets) at a rate exceeding the processing capacity. This results in a decline or cessation of transaction management operations (see \autoref{sec:transaction_management}) \cite{chaganti2022comprehensive}. An adversary may also disrupt the wallet network by exploiting the \acf{tcp} handshake mechanism, which establishes a connection between the wallet application and its servers, through \acf{syn} attacks \cite{chaganti2022comprehensive}.


% Figure environment removed


\subsection{Application}
\label{sec:application-attacks}

\subsubsection{Malware Execution}
\label{sec:malware}

This attack intrusively exploits system vulnerabilities to steal transaction data, the \textcolor{teal}{\textit{sk}} and password credentials, or to manipulate wallet operations as described in \autoref{sec:wallet_mechanism}. Malware threatens the wallet mechanism by replacing the \textcolor{teal}{\textit{recipient\_address}} via clipboard hijackers \cite{li2020android} or through input monitoring via keyloggers \cite{balakrishnan2023analysis} and other spyware types \cite{weichbroth2023security, ferdous2023review}. Hardware wallets are also vulnerable to clipboard hijack attacks \cite{ivanov2021ethclipper, Akter2023AChallenges}; malware can be injected through interactions between the wallet and removable media such as USB drives \cite{guri2018beatcoin}. 

Malware can also be engineered to monitor user actions and retrieve the user's password (\textcolor{teal}{\textit{pw}}) or private key (\textcolor{teal}{\textit{sk}}) \cite{weichbroth2023security, ferdous2023review}. Spyware includes keyloggers which can track every keystroke executed on an infected wallet device to steal confidential data \cite{Shaikh2022SurveyExchanges, balakrishnan2023analysis}. The custodial wallet Cashaa \cite{CoinTelegraph} and non-custodial wallets BitKeep \cite{CertiKIncidents} and Bittensor \cite{Explained:2024} have previously been exploited by malware-based vectors. Malware can also be combined with other attack methods, such as social engineering or privilege escalation, to achieve hacks, as noted in the ByBit case (see \autoref{sec:bybit_case} and \autoref{fig:lossMonthly}).


% Figure environment removed

\subsubsection{Social Engineering}
\label{sec:social}
These attacks aim to manipulate the user to divulge confidential data. Phishing attacks, for instance, aim to deceive wallet users into revealing \textcolor{teal}{\textit{sk}} or \textcolor{teal}{\textit{pw}} by mimicking legitimate services. Once successful, attackers can leverage additional vectors to gain unauthorised access \cite{krombholz2015advanced}.

Notably, malware delivered through phishing, such as Pink Drainer, Monkey Drainer, Venom Drainer, and Inferno, has been particularly effective against non-custodial wallets (see \autoref{tab:attack-incidents}). Phishing attacks have also been effective against custodial wallets \cite{HTXReport, HackScience} and notable individuals \cite{Explained:2024}. Adversaries have also exploited third-party dependencies by targeting their personnel, thereby extending the reach of social engineering campaigns \cite{bybit}. 

Telegram-embedded wallets heighten social-engineering exposure. Coordinated Telegram bots and rogue Mini App have drained millions from users \cite{slowmisttron,dailycoin}. Address-poisoning adds yet another twist: attackers inject look-alike addresses into a victim’s history so that a routine copy-and-paste transaction quietly redirects funds \cite{MetaMaskScam}.


\subsubsection{Privilege Escalation}
\label{sec:privilege}

These attacks aim to circumvent standard access controls to acquire elevated permissions. In Android root privilege attacks, the adversary can gain unauthorised root access to mobile wallets through vulnerabilities in the \acf{os} \cite{he2020security}. Another OS-related attack, Android USB debugging \cite{he2020security}, exploits \acf{os} vulnerabilities in mobile devices by wireless debugging, using a computer connected to the same network. Following this, the adversary gains unrestricted access to manipulate the execution flow of the wallet and capture \textcolor{teal}{\textit{sk}}, \textcolor{teal}{\textit{rdm\_seed}}, and other sensitive data \cite{he2020security}. 


\begin{table}[!htbp]
\centering
\tiny
\setlength{\tabcolsep}{2.1pt}
\renewcommand{\arraystretch}{0.9}
\begin{tabular}{lllrll}
\toprule
\textbf{Name} & \textbf{Custody Design} & \textbf{Date} & \textbf{Loss (\$)} & \textbf{Attack Category} & \textbf{Attack Name} \\
\midrule
ByBit \cite{bybit}  
  & Custodial  
  & %21/02/2025 
    2025-02 
  & 1,500M  
  & Application  
  & Logic Exploitation \\

US Govt. \cite{Decrypt}  
  & Non-Custodial  
  & %25/10/2024 
    2024-10 
  & 50M  
  & –  
  & – \\

BigX \cite{Explained:2024}  
  & Custodial  
  & %20/09/2024 
    2024-09 
  & 52M  
  & –  
  & – \\

Indodax \cite{IndonesianTRX}  
  & Custodial  
  & %11/09/2024 
    2024-09 
  & 22M  
  & –  
  & – \\

WazirX \cite{Explained:2024g}  
  & Custodial  
  & %18/07/2024 
    2024-07 
  & 235M  
  & Application  
  & Logic Exploitation \\

Bittensor \cite{Explained:2024}  
  & Non-Custodial  
  & %02/07/2024 
    2024-07 
  & 8M  
  & Application  
  & Malware \\

BTCTurk \cite{Explained:2024}  
  & Custodial  
  & %23/06/2024 
    2024-06 
  & 55M  
  & –  
  & – \\

Loopring \cite{Explained:2024}  
  & Non-Custodial  
  & %09/06/2024 
    2024-06 
  & 5M  
  & Authentication  
  & Identity Spoofing\textsuperscript{*} \\

Lykke \cite{CoinTelegraph}  
  & Custodial  
  & %04/06/2024 
    2024-06 
  & 22M  
  & –  
  & – \\

DMM Bitcoin \cite{Explained:2024}  
  & Custodial  
  & %31/05/2024 
    2024-05 
  & 305M  
  & –  
  & – \\

Axie Co-Founder \cite{Decrypt}  
  & Non-Custodial  
  & %23/02/2024 
    2024-02 
  & 10M  
  & –  
  & – \\

Fixed Float \cite{Explained:2024}  
  & Custodial  
  & %16/02/2024 
    2024-02 
  & 26.1M  
  & –  
  & – \\

kirilm.eth \cite{Explained:2024}  
  & Non-Custodial  
  & %16/02/2024 
    2024-02 
  & 5.1M  
  & Application  
  & Phishing \\

Ripple Co-Founder \cite{RippleMillion}  
  & Non-Custodial  
  & %30/01/2024 
    2024-01 
  & 112.5M  
  & –  
  & – \\

HTX (Huobi) \cite{HTXReport}  
  & Custodial  
  & %22/11/2023 
    2023-11 
  & 13.6M  
  & –  
  & \teal{\textit{sk}} Compromise\textsuperscript{*} \\

Pink Drainer \cite{RektREKT}  
  & Non-Custodial  
  & %16/11/2023 
    2023-11 
  & 12M  
  & Application  
  & Phishing, Malware \\

Monkey Drainer \cite{RektREKT}  
  & Non-Custodial  
  & %16/11/2023 
    2023-11 
  & 16M  
  & Application  
  & Phishing, Malware \\

Venom Drainer \cite{RektREKT}  
  & Non-Custodial  
  & %16/11/2023 
    2023-11 
  & 27M  
  & Application  
  & Phishing, Malware \\

Infarno \cite{infarno}  
  & Non-Custodial  
  & %16/11/2023 
    2023-11 
  & 66M  
  & Application  
  & Phishing, Malware \\

Poloniex \cite{RektREKT}  
  & Custodial  
  & %10/11/2023 
    2023-11 
  & 126M  
  & –  
  & \teal{\textit{sk}} Compromise\textsuperscript{*} \\

Lastpass \cite{RektREKT}  
  & Non-Custodial  
  & %31/10/2023 
    2023-10 
  & 37M  
  & Authentication  
  & – \\

Fantom Fdn. \cite{AnalysisMedium}  
  & Non-Custodial  
  & %18/10/2023 
    2023-10 
  & 7M  
  & –  
  & – \\

HTX (Huobi) \cite{HTXReport}  
  & Custodial  
  & %25/09/2023 
    2023-09 
  & 8M  
  & Application  
  & Phishing \\

Fake Voucher \cite{RektREKT}  
  & Non-Custodial  
  & %20/09/2023 
    2023-09 
  & 4.5M  
  & Application  
  & Phishing \\

Remitano \cite{RektREKT}  
  & Custodial  
  & %15/09/2023 
    2023-09 
  & 2.7M  
  & Application  
  & – \\

CoinEx \cite{CoinTelegraph}  
  & Custodial  
  & %12/09/2023 
    2023-09 
  & 55M  
  & –  
  & \teal{\textit{sk}} Compromise\textsuperscript{*} \\

Monero \cite{MoneroFlash}  
  & Non-Custodial  
  & %01/09/2023 
    2023-09 
  & 0.5M  
  & –  
  & – \\

AlphaPo \cite{RektREKT}  
  & Custodial  
  & %26/07/2023 
    2023-07 
  & 60M  
  & –  
  & \teal{\textit{sk}} Compromise\textsuperscript{*} \\

Atomic Wallet \cite{CoinTelegraph}  
  & Non-Custodial  
  & %03/06/2023 
    2023-06 
  & 100M  
  & –  
  & – \\

Bitrue \cite{Explained:2024}  
  & Custodial  
  & %14/04/2023 
    2023-04 
  & 23M  
  & –  
  & \teal{\textit{sk}} Compromise\textsuperscript{*} \\

GDAC \cite{CoinTelegraph}  
  & Custodial  
  & %09/04/2023 
    2023-04 
  & 13M  
  & –  
  & \teal{\textit{sk}} Compromise\textsuperscript{*} \\

MyAlgo \cite{CoinTelegraph}  
  & Non-Custodial  
  & %27/02/2023 
    2023-02 
  & 9.2M  
  & –  
  & – \\

BitKeep \cite{CertiKIncidents}  
  & Non-Custodial  
  & %26/12/2022 
    2022-12 
  & 8M  
  & Application  
  & Phishing, Malware \\

FTX \cite{FTXMistake}  
  & Custodial  
  & %12/11/2022 
    2022-11 
  & 450M  
  & Authentication  
  & Sim Swap Attack \\

Deribit \cite{CryptoWithdrawals}  
  & Custodial  
  & %01/11/2022 
    2022-11 
  & 28M  
  & Application  
  & – \\

Wintermute \cite{TheMedium}  
  & Custodial  
  & %20/09/2022 
    2022-09 
  & 160M  
  & Authentication  
  & Brute force \\

Slope \cite{CoinTelegraph}  
  & Non-Custodial  
  & %02/08/2022 
    2022-08 
  & 8M  
  & Storage and Memory  
  & – \\

MetaMask \cite{CertiKIncidents}  
  & Non-Custodial  
  & %17/04/2022 
    2022-04 
  & 0.65M  
  & Authentication  
  & Phishing \\

Crypto.com \cite{Explained:2024}  
  & Custodial  
  & %17/01/2022 
    2022-01 
  & 30M  
  & Authentication  
  & – \\

Lympo \cite{CoinTelegraph}  
  & Custodial  
  & %10/01/2022 
    2022-01 
  & 18.7M  
  & –  
  & – \\

LCX \cite{LookingHacken}  
  & Custodial  
  & %08/01/2022 
    2022-01 
  & 8M  
  & –  
  & \teal{\textit{sk}} Compromise\textsuperscript{*} \\

Vulcan Forged \cite{VulcanHack}  
  & Non-Custodial  
  & %13/12/2021 
    2021-12 
  & 140M  
  & Application  
  & \teal{\textit{sk}} Compromise\textsuperscript{*} \\

BitMart \cite{HackScience}  
  & Custodial  
  & %05/12/2021 
    2021-12 
  & 196M  
  & Application  
  & Phishing \\

Liquid \cite{HackBreach}  
  & Custodial  
  & %19/08/2021 
    2021-08 
  & 90M  
  & Application  
  & \teal{\textit{sk}} Compromise\textsuperscript{*} \\

Roll \cite{CoinDesk}  
  & Custodial  
  & %14/03/2021 
    2021-03 
  & 5.7M  
  & Application  
  & \teal{\textit{sk}} Compromise\textsuperscript{*} \\

MetaMask \cite{Explained:2024}  
  & Non-Custodial  
  & %14/12/2020 
    2020-12 
  & 8M  
  & –  
  & – \\

KuCoin \cite{kucoinNew}  
  & Custodial  
  & %25/09/2020 
    2020-09 
  & 275M  
  & Application  
  & \teal{\textit{sk}} Compromise\textsuperscript{*} \\

Cashaa \cite{CoinTelegraph}  
  & Custodial  
  & %11/07/2020 
    2020-07 
  & 3.1M  
  & Application  
  & Malware \\

Trinity Wallet \cite{IOTA:Wallet}  
  & Non-Custodial  
  & %12/02/2020 
    2020-02 
  & 2.3M  
  & Application  
  & – \\

Altsbit \cite{AltsbitZDNET}  
  & Custodial  
  & %05/02/2020 
    2020-02 
  & 72.5M  
  & Application  
  & – \\

Upbit \cite{UpbitMedium}  
  & Custodial  
  & %26/11/2019 
    2019-11 
  & 49M  
  & Application  
  & Phishing, Malware \\

Bitpoint \cite{BitPointMedium}  
  & Custodial  
  & %11/07/2019 
    2019-07 
  & 36.5M  
  & –  
  & – \\

Vindax \cite{VinDAXBlock}  
  & Custodial  
  & %05/11/2019 
    2019-11 
  & 0.5M  
  & –  
  & – \\

Bitrue \cite{CryptoNews}  
  & Custodial  
  & %27/06/2019 
    2019-06 
  & 4.5M  
  & Authentication  
  & – \\

Gatehub \cite{OverviewMedium}  
  & Custodial  
  & %06/06/2019 
    2019-06 
  & 9.5M  
  & –  
  & – \\

Binance Exchange \cite{binanceNew}  
  & Custodial  
  & %07/05/2019 
    2019-05 
  & 40M  
  & Unknown  
  & – \\

Bithumb \cite{CoinDesk}  
  & Custodial  
  & %29/03/2019 
    2019-03 
  & 13M  
  & Other  
  & Insider Job \\

Coinbene \cite{CoinTelegraph}  
  & Custodial  
  & %25/03/2019 
    2019-03 
  & 99M  
  & –  
  & – \\

DragonEX \cite{CoinDesk}  
  & Custodial  
  & %24/03/2019 
    2019-03 
  & 1M  
  & Application  
  & – \\

Cryptopia \cite{HowHacken}  
  & Custodial  
  & %01/02/2019 
    2019-02 
  & 16M  
  & –  
  & \teal{\textit{sk}} Compromise\textsuperscript{*} \\

LocalBitcoins \cite{CoinDesk}  
  & Custodial  
  & %26/01/2019 
    2019-01 
  & 0.02M  
  & Application  
  & Phishing \\

Electrum \cite{DeepSwig}  
  & Non-Custodial  
  & %21/12/2018 
    2018-12 
  & 0.75M  
  & Application  
  & Phishing \\

Maplechange \cite{MapleChangeInvestorPlace}  
  & Custodial  
  & %28/10/2018 
    2018-10 
  & 6M  
  & –  
  & – \\

Zaif \cite{CoinDesk}  
  & Custodial  
  & %14/09/2018 
    2018-09 
  & 100M  
  & –  
  & – \\

Coinrail \cite{CoinDesk}  
  & Custodial  
  & %10/06/2018 
    2018-06 
  & 40M  
  & –  
  & – \\

MyEtherWallet \cite{myetherwallet}  
  & Non-Custodial  
  & %24/04/2018 
    2018-04 
  & 0.15M  
  & Network  
  & \acs{bgp} Hijacking \\

Gate.io \cite{ZachXBTWraps}  
  & Custodial  
  & %18/04/2018 
    2018-04 
  & 234M  
  & –  
  & – \\

CoinSecure \cite{CoinDesk}  
  & Custodial  
  & %13/04/2018 
    2018-04 
  & 3.5M  
  & Other  
  & Insider Job \\

Bitgrail \cite{BitGrailCoinMarketCap}  
  & Custodial  
  & %10/02/2018 
    2018-02 
  & 146M  
  & Other  
  & Insider Job \\

CoinCheck \cite{TheHack}  
  & Custodial  
  & %27/01/2018 
    2018-01 
  & 560M  
  & –  
  & – \\

BlackWallet \cite{BlackWalletFault}  
  & Non-Custodial  
  & %15/01/2018 
    2018-01 
  & 0.4M  
  & Network  
  & \acs{dns} Spoofing \\

EtherDelta \cite{CryptocurrencyScheme}  
  & Custodial  
  & %20/12/2017 
    2017-12 
  & 1.4M  
  & Network  
  & \acs{dns} Spoofing \\

Parity \cite{palladino2017parity}  
  & Non-Custodial  
  & %19/07/2017 
    2017-07 
  & 30M  
  & Application  
  & Logic Exploitation \\

Yapizon \cite{CoinTelegraph}  
  & Custodial  
  & %22/04/2017 
    2017-04 
  & 5.3M  
  & –  
  & – \\

Bitfinex \cite{CoinDesk}  
  & Custodial  
  & %02/08/2016 
    2016-08 
  & 623M  
  & Application  
  & – \\

Gatecoin \cite{CoinDesk}  
  & Custodial  
  & %09/05/2016 
    2016-05 
  & 2.1M  
  & –  
  & – \\

Shapeshift \cite{LootingShapeShift}  
  & Custodial  
  & %07/04/2016 
    2016-04 
  & 0.23M  
  & Other  
  & Insider Job \\

Bitstamp \cite{DetailsRevealed}  
  & Custodial  
  & %11/12/2015 
    2015-12 
  & 5M  
  & Application  
  & Phishing \\

BTER \cite{CoinDesk}  
  & Custodial  
  & %15/08/2015 
    2015-08 
  & 1.65M  
  & Application  
  & – \\

Mintpal \cite{RememberingLedger}  
  & Custodial  
  & %13/07/2014 
    2014-07 
  & 2M  
  & Other  
  & Insider Job \\

Poloniex \cite{PoloniexHack}  
  & Custodial  
  & %04/03/2014 
    2014-03 
  & 0.05M  
  & Application  
  & – \\

Mt. Gox \cite{mtgox_hack}  
  & Custodial  
  & %24/02/2014 
    2014-02 
  & 460M  
  & –  
  & – \\

Bitcash \cite{CzechEmptied}  
  & Custodial  
  & %11/11/2013 
    2013-11 
  & 0.1M  
  & Application  
  & Phishing \\

Bitfloor \cite{HackSecurityWeek}  
  & Custodial  
  & %12/09/2012 
    2012-09 
  & 0.25M  
  & Application  
  & \teal{\textit{sk}} Compromise\textsuperscript{*} \\

Bitcoinica \cite{ExchangeStolen}  
  & Custodial  
  & %01/03/2012 
    2012-03 
  & 0.09M  
  & Application  
  & \teal{\textit{sk}} Compromise\textsuperscript{*} \\

\midrule
\textbf{Summary:}
  & \textbf{85 incidents}
  & \textbf{2012–2025}
  & \textbf{6.98B}
  &  
  &  
\\
\bottomrule
\end{tabular}
\caption{Wallet attack incidents in the industry. We retrieve 85 notable attack incidents involving both custodial and non-custodial wallets. Several attack methods remain unknown (–) or undetailed, we indicate undetailed incidents with \textsuperscript{*}.}
\label{tab:attack-incidents}
\end{table}



\subsubsection{Logic Exploitation}
\label{sec:logic_expl}

Logic flow exploitation encompasses several wallet types and involves identifying and exploiting flaws in the programming logic of a wallet mechanism (\autoref{sec:wallet_mechanism}) to gain unauthorised access or manipulate wallet functions \cite{Parisi2023WalletSecurity}. Notable incidents include WazirX (2024), where investigators linked the drain to a malicious Safe module that slipped through the upgrade mechanism and rewired the wallet via \texttt{DELEGATECALL} \cite{Explained:2024}. In ByBit (2025), attackers pushed a forged implementation contract into the exchange's cold-wallet proxy, overwriting storage and seizing ownership by abusing Safe's upgrade path \cite{bybit_certik} (see \autoref{sec:bybit_case}). The classic Parity library bug (2017) involved an uninitialised contract that allowed the adversary to gain ownership and drain multi-sig wallets \cite{palladino2017parity}. These cases map to two recurrent sub-patterns: \begin{enumerate*}[label=(\arabic*)] \item upgrade-path hijack, where the authorised proxy-upgrade or module-installation channel is abused to introduce attacker-controlled logic (ByBit, WazirX); and \item constructor hijack, where the \texttt{init} function is left callable after deployment (Parity). \end{enumerate*} 


\subsection{Authentication}
\label{sec:auth-attacks}

\subsubsection{Credential Cracking}
\label{sec:cred-crack}

This category of attacks systematically attempts different credential values to bypass the authentication mechanism. Brute-force attacks involve an adversary systematically trying all possible character combinations to bypass the authentication function and decrypt \textcolor{teal}{\textit{enc\_sk}}. If successful, the adversary can create malicious transactions using \hyperref[algo:transaction-signing]{Algorithm 2} \cite{Kiktenko2019DetectingWallets}. Dictionary attacks, on the other hand, leverage commonly used words to predict \textcolor{teal}{\textit{rdm\_seed}} phrases for access. Unlike brute-force attacks that exhaust all possible combinations, dictionary attacks are computationally less demanding, and their success rate increases with the use of leaked password datasets \cite{Uddin2021Horus:Wallets, praitheeshan2019security}.

\subsubsection{Identity Spoofing}
\label{sec:iden-spoof}

For enhanced \acs{kek} security, wallets leverage supplementary user authentication methods, such as user biometrics and \acf{2fa} implementations. 


The identity spoofing attack method bypasses these verification mechanisms (see \hyperref[algo:cryptocurrency-wallet]{Algorithm 1}) by impersonating the user to decrypt \textcolor{teal}{\textit{enc\_sk}} and authorise malicious transactions. In fake biometric attacks, an adversary employs synthetic or reconstructed biometric data to achieve this goal \cite{galbally2013image}. To circumvent SMS-based \acs{2fa}, an adversary can also use SIM swap attacks, which execute the transfer of the user's phone number to an adversary-controlled mobile device \cite{Kim2022ACountermeasures}. Mobile wallets, smart contract wallets and other infrastructures that integrate SMS-based \acs{2fa} or biometric verification can be vulnerable to these attacks (see \autoref{tab:attack_vectors}).

\subsection{Storage and Memory}
\label{sec:physical-attacks}

\subsubsection{Physical Tampering}
\label{sec:tam-per}

These primarily involve physically altering a wallet’s hardware to bypass security protections. In evil maid attacks, the attacker physically modifies the unencrypted storage of a device to capture credentials or manipulate the system \cite{altuwaijri2020android}. In contrast, microscopy attacks use advanced techniques, such as electron microscopy, to examine the microelectronic components of a wallet. These attacks can extract critical data or identify vulnerabilities, often without altering the hardware itself \cite{courbon2016reverse}.

\subsubsection{Fault Injection}
\label{sec:fau-inj}

These attacks manipulate the wallet's components by forcing an erroneous system state to bypass the security mechanisms \cite{Akter2023AChallenges}. For instance, fault injection attacks on hardware wallets often exploit vulnerabilities in volatile memory (such as \acs{sram}) by manipulating environmental factors. Data remanence vulnerabilities in the Trezor wallet have been exploited to demonstrate these attacks \cite{trezor_memory, trezor_medium}. Fault injection attacks on smart contracts have also been shown in the literature \cite{hajdu2020using}.

\subsubsection{Other Non-Invasive Techniques}
\label{sec:non-inv-man}

Other non-invasive storage and memory attacks exist which are not based on fault injection methods. In cold boot attacks, the attacker executes a cold restart on the wallet device to exploit the data remanence properties of volatile memory, such as \acf{dram} and \acf{sram}, to retrieve sensitive data \cite{Shaikh2022SurveyExchanges}. Similarly, \acs{puf} attacks exploit the unique characteristics of hardware defence implementations known as \acf{puf}. These implementations have challenge-response functionality that exhibits physical unclonability \cite{Garcia-Bosque2020IntroductionApplications, wang2024efficient}.


% % cold boot
% In this attack, an adversary with physical access exploits the data remanence properties of \acs{ram} i.e. \acf{dram} and \acf{sram} in some wallet devices to retrieve encryption key (\teal{$\rho_r$}), passwords (\teal{$pwd$}), or other sensitive data from memory after a cold restart \cite{shaikh2022survey}. 


\subsection{Cryptanalysis}
\label{sec:cryptanalysis-analysis}

% Side-Channel Analysis

\subsubsection{Side-Channel Analysis}
\label{sec:side-channel}

Non-invasive key extraction attacks on cryptographic functions, including timing and power \acf{sca}, are executed by exploiting side channels. These attacks exploit leakages in behaviours exhibited by cryptographic functions (see \autoref{sec:wallet_mechanism}) through side channels to measure and extract values such as time and power  \cite{Shaikh2022SurveyExchanges, Park2023}. Timing-based \acs{sca} measures the cryptographic function execution time. Successful implementation of a timing-based side-channel attack has been demonstrated on a Trezor One hardware wallet \cite{kocher1996timing}. Power-based \acs{sca} analyses the cryptographic function's power trace, including the hash function. \acs{sca} on the hash function has been utilised to extract the \textcolor{teal}{\textit{rdm\_seed}} \cite{Park2024CloningFunction}.

\subsubsection{Direct Exploitation}
\label{sec:impl-exp}

These attacks directly target implementation errors within the cryptographic surface area. Weak signature (\teal{$\sigma$}) attacks, for example, target weaknesses in the signing algorithm due to improper implementation, weak or outdated cryptographic algorithms or errors in encryption logic \cite{Rokhjavan2023SecuringWallets}. In addition, an adversary can exploit vulnerabilities in \hyperref[algo:transaction-signing]{Algorithm 2} by reusing a nonce during transaction authorisation \cite{brengel2018identifying}. Such reuse can compromise the security of wallets by resulting in \textcolor{teal}{\textit{sk}} leakage \cite{Ko2020PrivateSignatures}.

