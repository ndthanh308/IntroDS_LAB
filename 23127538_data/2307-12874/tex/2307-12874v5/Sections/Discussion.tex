\section{Discussion}
\label{sec:discussion}

\subsection{Limitations}

One significant limitation of our study is the quality and completeness of the data available on wallet attacks. As highlighted, many recorded incidents from custodial and non-custodial wallet providers contain a high degree of uncertainty regarding attack vectors (see \autoref{tab:attack-incidents}). This ambiguity restricts our capability to perform detailed quantitative analyses of wallet attacks, thereby limiting the precision of our analysis.


\subsection{Future Work}

To address these limitations, we propose the following research directions to improve wallet security:



\subsubsection{Enhanced Transaction Validation Measures} Our study highlights the uncertainty in recorded attack vectors, underscoring the need for enhanced transaction validation approaches. Advanced validation methods, such as independent transaction hash verification and proactive policy enforcement through on-chain gatekeeping, should be explored to improve transaction data clarity and reliability. Furthermore, integrating hardware wallets capable of clear-signing raw transaction parameters will significantly mitigate risks associated with deceptive UI interactions and unauthorised operational logic.

\subsubsection{Addressing Signature Verification Logic Flaws} Given the prevalence of signature verification logic flaws across wallet architectures, targeted research is crucial for developing secure and robust signature verification frameworks. Future work should prioritise the formal verification of signature verification algorithms, exploring cryptographic approaches specifically designed to mitigate known logic vulnerabilities. This will directly enhance the integrity and trustworthiness of wallet systems.

\subsubsection{Development of Reactive Defence Mechanisms} Our study identified a substantial gap in reactive security measures, with an evident imbalance favouring preventive strategies. Future research should emphasise the development of advanced reactive mitigation strategies, including real-time anomaly detection, responsive incident management protocols, and automated recovery frameworks tailored explicitly for wallet incidents. Enhancing reactive defence capabilities will substantially improve resilience and responsiveness to evolving threat vectors. By addressing these targeted research areas informed by our identified limitations, the community can significantly advance wallet security practices. This will lead to improved theoretical understanding and enhanced practical outcomes.


