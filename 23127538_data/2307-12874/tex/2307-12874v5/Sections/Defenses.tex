
\section{Security Measures}
\label{sec:defense-strategies}

This section builds upon the framework outlined in \autoref{sec:attack-framework} by presenting mitigation approaches against wallet attacks. We aim to examine defence mechanisms for each identified attack vector affecting wallets.

\subsection{Network}
\label{sec:net-def}

Suspicious network activity can be detected through machine learning techniques, including anomaly detection models \cite{kapoor2021ransomware} and classification algorithms \cite{balakrishnan2023analysis}. Additionally, dynamic network parameter adjustments \cite{Girdler2021ImplementingAddresses} and other intrusion detection mechanisms \cite{guri2018beatcoin, zimba2019cryptojacking} further contribute to identifying such anomalies. 


\begin{landscape}
% \vspace*{\fill}
% \begin{table}[!htbp] !p
% \begin{sidewaystable*}[!htbp]
% \begin{table*}[!p]
\begin{table}[!htbp]
% \thispagestyle{empty} 
\centering
\caption{
Three-level attack classification showing gap analysis, threat occurrences, adversary's target and mapping to possible security measures (\autoref{sec:defense-strategies}). The \enquote{Gaps} summary shows that academic literature covers 24 of the 28 enumerated attack vectors (86\%), whereas publicly reported incidents cover 9 vectors (32\%). Notable incident percentages are calculated from a total of 85 reported industry incidents (see \autoref{tab:attack-incidents}). Symbols: ( \smallfullcirc : include, \smallhalfcirc : part-inclusion (influenced by other factors), \smallemptycirc : not include) }
\label{tab:attack_vectors}
\tiny
% \renewcommand{\arraystretch}{1}
% \setlength{\tabcolsep}{1pt}
\renewcommand{\arraystretch}{1}
\setlength{\tabcolsep}{1.5pt} 
% \setlength{\tabcolsep}{1pt} 
\resizebox{\linewidth}{!}
{
\begin{tabular}{llllccccccccccccccccccccccccccccccccccccccccccccccccl}
% llllccccccccccccccccccccccccccccccccccccccccccccccccccccccccccccccccccccl
\toprule
\multicolumn{1}{c}{\textbf{Category}} &
  \multicolumn{1}{c}{\multirow{1}{*}{\textbf{\hyperref[sec:attack-framework]{ \textbf{Method}}}}} &
  \multicolumn{1}{c}{\multirow{1}{*}{\textbf{\hyperref[sec:attack-framework]{\textbf{Vector}}}}} &
  \multicolumn{21}{c}{\textbf{\hyperref[sec:threat_framework]{Threat}}} &
  % \multicolumn{1}{p{2.15cm}}{\textbf{Description}} &
  \multicolumn{16}{c}{\textbf{\hyperref[sec:wallet_mechanism]{Target}}} & 
  % \multicolumn{3}{c}{\textbf{\hyperref[sec:threat_class]{Threat}}} &
  \multicolumn{3}{c}{\textbf{\hyperref[sec:adversary_goal]{Goal}}} &
  \multicolumn{5}{c}{\textbf{\hyperref[sec:infrastructure]{Infrastructure}}} &
  \multicolumn{2}{c}{\textbf{\hyperref[sec:attacks_discussion]{Gaps}}} &
  \multicolumn{1}{c}{\textbf{\hyperref[sec:defense-strategies]{Possible Defence}}} \\ 
  % \multicolumn{25}{c}{\textbf{\hyperref[sec:defense-strategies]{Possible Defence Methods}}} \\ 
\cmidrule(lr){4-40} 
% \cmidrule(lr){45-47}
% \multicolumn{1}{c}{\multirow{7}{*}{\rotatebox[origin=l]{90}{\textbf{Attack Category}}}} 
% \multicolumn{1}{c}{\textbf{Category}}
&
  \multicolumn{1}{c}{} &
  \multicolumn{1}{c}{} &
  \multicolumn{12}{c}{} &
  % \multicolumn{12}{c}{\textbf{Mechanism Vuln.}} &
  \multicolumn{5}{c}{} &
  % \multicolumn{5}{c}{\textbf{Syst. Vuln.}} &
  \multicolumn{2}{c}{} &
  % \multicolumn{2}{c}{\textbf{Ex.}} &
  \multicolumn{2}{c}{} &
  % \multicolumn{2}{c}{\textbf{In.}} &
  \multicolumn{7}{c}{\textbf{Data}} &
  \multicolumn{6}{c}{\textbf{Mechanism}} &
  \multicolumn{3}{c}{\textbf{Other}} &
  % \multicolumn{3}{c}{} &
  \multicolumn{3}{c}{} &
  \multicolumn{1}{l}{} &
  \multicolumn{4}{l}{} &
  \multicolumn{2}{l}{} &
  \multicolumn{1}{l}{} 
  % \multicolumn{8}{c}{\textbf{Auth. Bypass}} &
  % \multicolumn{2}{c}{\textbf{Disrp.}} &
  % \multicolumn{4}{c}{\textbf{Intrusion}} &
  % \multicolumn{5}{c}{\textbf{Alter.}} &
  % \multicolumn{7}{c}{\textbf{Extraction}}
  &
   \\
% \cmidrule(lr){4-15} 
% \cmidrule(lr){16-20} 
% \cmidrule(lr){21-22} 
% \cmidrule(lr){23-24} 
\cmidrule(lr){25-31} \cmidrule(lr){32-37} \cmidrule(lr){38-40} 

% \cmidrule(lr){45-53} \cmidrule(lr){54-55} \cmidrule(lr){56-59} \cmidrule(lr){60-64} \cmidrule(lr){65-72}
\multicolumn{1}{c}{} &
  \multicolumn{1}{c}{} &
  \multicolumn{1}{c}{} &
\multicolumn{1}{c}{\rotatebox[origin=l]{90}{Predictable \acs{rng} \cite{brengel2018identifying, cve_31290, cve_23660}}} &
\multicolumn{1}{c}{\rotatebox[origin=l]{90}{Inadequate Authentication \cite{Uddin2021Horus:Wallets}}} &
\multicolumn{1}{c}{\rotatebox[origin=l]{90}{Inadequate Encryption \cite{cve_15947}}} &
\multicolumn{1}{c}{\rotatebox[origin=l]{90}{Application Logic Flaw \cite{Destefanis2018SmartEngineering, Parisi2023WalletSecurity, oren2023fireblocks}}} &
\multicolumn{1}{c}{\rotatebox[origin=l]{90}{Low-strength Passwords \cite{Kiktenko2019DetectingWallets, volety2019cracking}}} &
\multicolumn{1}{c}{\rotatebox[origin=l]{90}{Data Leakage \cite{cve_14353, cve_14354, KrakenBlog}}} &
\multicolumn{1}{c}{\rotatebox[origin=l]{90}{Data Remanence \cite{trezor_memory, trezor_medium}}} &
\multicolumn{1}{c}{\rotatebox[origin=l]{90}{Data Manipulation \cite{trezor_memory, trezor_medium}}} &
\multicolumn{1}{c}{\rotatebox[origin=l]{90}{Insecure Boot Environment \cite{Shaikh2022SurveyExchanges}}} &
\multicolumn{1}{c}{\rotatebox[origin=l]{90}{Microelectronic Component Exposure \cite{courbon2016reverse}}} &
\multicolumn{1}{c}{\rotatebox[origin=l]{90}{Weak Signature \cite{Rokhjavan2023SecuringWallets}}} &
\multicolumn{1}{c}{\rotatebox[origin=l]{90}{Inadequate Signature Verification \cite{cve_14199, tymokhanov2021alpha}}} &
\multicolumn{1}{c}{\rotatebox[origin=l]{90}{Insecure Permissions \cite{cve_32969, halborn_vuln}}} &
\multicolumn{1}{c}{\rotatebox[origin=l]{90}{Library Vulnerability \cite{bitcore_lib, Ledger2023SecurityReport} }} &
\multicolumn{1}{c}{\rotatebox[origin=l]{90}{\acs{os} Vulnerabilities \cite{he2020security}}} &
\multicolumn{1}{c}{\rotatebox[origin=l]{90}{Coding Errors \cite{Parisi2023WalletSecurity}}} &
\multicolumn{1}{c}{\rotatebox[origin=l]{90}{Insec. Network \cite{cve_33297, cve_14198, cve_17144}}} &
\multicolumn{1}{c}{\rotatebox[origin=l]{90}{Insec. User Interactions \cite{ZengoZengo, thodex}}} & 
\multicolumn{1}{c}{\rotatebox[origin=l]{90}{Comp. Provider \cite{CoinTelegraph2022SlopeAttack}}} &
\multicolumn{1}{c}{\rotatebox[origin=l]{90}{Malicious Insider \cite{decrypt_ftx}}} &
\multicolumn{1}{c}{\rotatebox[origin=l]{90}{Compromised Insider \cite{Ledger2023SecurityReport}}} &
\multicolumn{1}{c}{\rotatebox[origin=l]{90}{Private Key (\teal{$sk$})}} &
\multicolumn{1}{c}{\rotatebox[origin=l]{90}{Signature (\teal{$\sigma$})}} &
\multicolumn{1}{c}{\rotatebox[origin=l]{90}{Mnemonics (\teal{$rdm\_seed$})}} &
\multicolumn{1}{c}{\rotatebox[origin=l]{90}{\acs{kek} or Password (\teal{$pw$})}} &
\multicolumn{1}{c}{\rotatebox[origin=l]{90}{Memory}} &
\multicolumn{1}{c}{\rotatebox[origin=l]{90}{State Trans. Info.}} &
\multicolumn{1}{c}{\rotatebox[origin=l]{90}{Nonce}} &
\multicolumn{1}{c}{\rotatebox[origin=l]{90}{KeyGen}} &
\multicolumn{1}{c}{\rotatebox[origin=l]{90}{UserAuth}} &
\multicolumn{1}{c}{\rotatebox[origin=l]{90}{KeyStore}} &
\multicolumn{1}{c}{\rotatebox[origin=l]{90}{TxnInit}} &
\multicolumn{1}{c}{\rotatebox[origin=l]{90}{TxnSign}} &
\multicolumn{1}{c}{\rotatebox[origin=l]{90}{TxnVer}} &
  \multicolumn{1}{c}{\rotatebox[origin=l]{90}{Service Provider}} &
  % \multicolumn{1}{c}{\rotatebox[origin=l]{90}{Network Connection}} &
  \multicolumn{1}{c}{\rotatebox[origin=l]{90}{Operating System}} &
  \multicolumn{1}{c}{\rotatebox[origin=l]{90}{Wallet User}} &
  % \multicolumn{1}{c}{\rotatebox[origin=l]{90}{System}} &
  % \multicolumn{1}{c}{\rotatebox[origin=l]{90}{External}} &
  % \multicolumn{1}{c}{\rotatebox[origin=l]{90}{Insider}} &
  \multicolumn{1}{c}{\rotatebox[origin=l]{90}{Transaction Alteration}} &
  \multicolumn{1}{c}{\rotatebox[origin=l]{90}{Credential Compromise}} &
  \multicolumn{1}{c}{\rotatebox[origin=l]{90}{Network Disruption}} &
  % \multicolumn{1}{c}{\rotatebox[origin=l]{90}{Software}} &
  \multicolumn{1}{c}{\rotatebox[origin=l]{90}{Desktop Wallet}} &
  \multicolumn{1}{c}{\rotatebox[origin=l]{90}{Browser Wallet}} &
  \multicolumn{1}{c}{\rotatebox[origin=l]{90}{Mobile Wallet}} &
  \multicolumn{1}{c}{\rotatebox[origin=l]{90}{Smart Wallet}} &
  \multicolumn{1}{c}{\rotatebox[origin=l]{90}{Hardware Wallet}} &
  \multicolumn{1}{c}{\rotatebox[origin=l]{90}{Academic Papers No. (\%)}} &
  \multicolumn{1}{c}{\rotatebox[origin=l]{90}{Notable Incidents No. (\%)}} &
  
  % \multicolumn{1}{c}{\rotatebox[origin=l]{90}{Custom Keyboard Functions \cite{aldawood2020advanced}}} &
  % \multicolumn{1}{c}{\rotatebox[origin=l]{90}{Access Control Restrictions \cite{li2020android}}} &
  % \multicolumn{1}{c}{\rotatebox[origin=l]{90}{Enhanced Network Authentication \cite{Cai2014ResearchNetwork}}} &
  % \multicolumn{1}{c}{\rotatebox[origin=l]{90}{Multi-factor Authentication \cite{Aratani2015AuthenticationChannel}}} &
  % \multicolumn{1}{c}{\rotatebox[origin=l]{90}{Advanced Passwords \cite{aldawood2020advanced}}} &
  % \multicolumn{1}{c}{\rotatebox[origin=l]{90}{Liveness Assessment \cite{galbally2013image}}} &
  % \multicolumn{1}{c}{\rotatebox[origin=l]{90}{\acf{mpc} \cite{Lindell2020SecureComputation}}} & 
  % \multicolumn{1}{c}{\rotatebox[origin=l]{90}{Multi-sig. Implementation \cite{bip11}}} &
  % \multicolumn{1}{c}{\rotatebox[origin=l]{90}{Traffic Mitigation \cite{liu2018deep}}} &
  % \multicolumn{1}{c}{\rotatebox[origin=l]{90}{Reset TCP Conn. \cite{sathwara2017distributed}}} &
  % \multicolumn{1}{c}{\rotatebox[origin=l]{90}{Intrusion Detection \cite{zimba2019cryptojacking}}} &
  % \multicolumn{1}{c}{\rotatebox[origin=l]{90}{IP Verification \& Monitoring \cite{Bhirud2011LightPrevention}}} &
  % \multicolumn{1}{c}{\rotatebox[origin=l]{90}{Anti-Malware \cite{ferdous2023review}}} &
  %  \multicolumn{1}{c}{\rotatebox[origin=l]{90}{WebApp Firewalls \cite{ahmed2017mitigating}}} &
  %  \multicolumn{1}{c}{\rotatebox[origin=l]{90}{Alt. Prevention Features \cite{li2020android}}} &
  %  \multicolumn{1}{c}{\rotatebox[origin=l]{90}{Code Obfuscation \cite{indusface}}} &
  %  \multicolumn{1}{c}{\rotatebox[origin=l]{90}{Cryptographic Verification \cite{Tirronen2018StoppingData}}} &
  %  \multicolumn{1}{c}{\rotatebox[origin=l]{90}{Runtime Protection \cite{qi2012spad}}} &
  %  \multicolumn{1}{c}{\rotatebox[origin=l]{90}{Algorithmmic Fault Detection \cite{breier2022practical}}} &
  %  \multicolumn{1}{c}{\rotatebox[origin=l]{90}{\acl{puf}\cite{hu2020overview, Urien2021InnovativeWallets}}} & 
  %  \multicolumn{1}{c}{\rotatebox[origin=l]{90}{Deterministic Nonce Selection \cite{brengel2018identifying}}} &
  %  \multicolumn{1}{c}{\rotatebox[origin=l]{90}{Algorithmic Memory Erase \cite{seol2019amnesiac}}} &
  %  \multicolumn{1}{c}{\rotatebox[origin=l]{90}{Memory \& Cache Data Split \cite{Gupta2019ImpactSecurity}}} &
  %  \multicolumn{1}{c}{\rotatebox[origin=l]{90}{Supplementary Storage \cite{altuwaijri2020android}}} &
  %  \multicolumn{1}{c}{\rotatebox[origin=l]{90}{Secure Cryptographic Schemes \cite{brengel2018identifying}}} &
  %  \multicolumn{1}{c}{\rotatebox[origin=l]{90}{Correlation Elimination Sounds \cite{Park2023, Park2024CloningFunction}}} &
  %  &
   \\
   \addlinespace[2ex] % Add space before the second top rule
   \toprule
\multirow{6}{*}{Network} &
  \multirow{4}{*}{\hyperref[sec:dos]{Connection Hijack}} &
  Rogue AP \cite{Hu2021SecurityCountermeasures}  

  % {\smallemptycirc} &
  % {\smallfullcirc} &
  &
  {\smallemptycirc} &
  % -- rng
  {\smallemptycirc} &
  % -- inadequ auth
  {\smallemptycirc} &
  % -- inadequ encry
  {\smallemptycirc} &
  % -- Appl. Logic Flaw
  {\smallemptycirc} &
  % -- Low-strength pwds
  {\smallemptycirc} &
  % -- Data Leakage
  {\smallemptycirc} &
  % -- Data Remanence
  {\smallemptycirc} &
  % -- Data Remanence
  {\smallemptycirc} &
  % -- Insec. Boot Environ.
  {\smallemptycirc} &
  % -- Micro-electr. Exposure
  {\smallemptycirc} &
  % -- Weak Signature
  {\smallemptycirc} &
  % -- Inadeq. Sig. Verif. 
  % -- SYSTEM
  {\smallemptycirc} &
  % -- Insec. Permissions
  {\smallemptycirc} &
  % -- Library Vulnerability
  {\smallemptycirc} &
  % -- OS Vulnerabilities
  {\smallemptycirc} &
  % -- Coding Errors
  {\smallfullcirc} &
  % -- Insec. Network
  %  -- SYSTEM
  {\smallfullcirc} &
  % -- Insec. User Interactions
  {\smallfullcirc} &
  % -- Comp. Provider
  % -- EXT
  {\smallemptycirc} &
  % -- Malicious Insider
  {\smallemptycirc} &
  % -- Insider Compromise
  % -- INSIDER
  {\smallemptycirc} &
  {\smallemptycirc} &
  {\smallemptycirc} &
  {\smallemptycirc} &
  {\smallemptycirc} &
  {\smallfullcirc} &
  {\smallemptycirc} &
  {\smallemptycirc} &
  {\smallemptycirc} &
  {\smallemptycirc} &
  {\smallfullcirc} &
  {\smallemptycirc} &
  % data --
  {\smallfullcirc} &
  {\smallfullcirc} &
  {\smallemptycirc} &
  {\smallfullcirc} &
  {\smallfullcirc} &
  {\smallemptycirc} &
  % mech --
  {\smallemptycirc} &
  % {\smallfullcirc} &
  % {\smallfullcirc} &
  {\smallfullcirc} &
  {\smallfullcirc} &
  {\smallfullcirc} &
  {\smallfullcirc} &
  {\smallfullcirc} &
\cellcolor{g2}{$1$} &
\cellcolor{g0}{$0$} &
  % other --
  % -- &
  % -- &
  % -- &
  % {\smallemptycirc} &
  % {\smallemptycirc} &
  % {\smallfullcirc} &
  % % goal --
  % {\smallemptycirc} &
  % {\smallemptycirc} &
  % % tax --
  % {\smallemptycirc} &
  % {\smallemptycirc} &
  % {\smallemptycirc} &
  % {\smallemptycirc} &
  % {\smallemptycirc} &
  % {\smallfullcirc} &
  % {\smallemptycirc} &
  % {\smallemptycirc} &
  % % auth --
  % {\smallemptycirc} &
  % {\smallemptycirc} &
  % % disr --
  % {\smallemptycirc} &
  % {\smallemptycirc} &
  % {\smallemptycirc} &
  % {\smallemptycirc} &
  % {\smallemptycirc} &
  % % intru --
  % {\smallemptycirc} &
  % {\smallemptycirc} &
  % {\smallemptycirc} &
  % {\smallemptycirc} &
  % {\smallemptycirc} &
  % {\smallemptycirc} 
  \cite{Cai2014ResearchNetwork, zimba2019cryptojacking} 
  \\
 &
   &
  DNS Spoofing \cite{pillai2019smart, Al-Mashhadi2020ASystems} &
  {\smallemptycirc} &
  % -- rng
  {\smallemptycirc} &
  % -- inadequ auth
  {\smallemptycirc} &
  % -- inadequ encry
  {\smallemptycirc} &
  % -- Appl. Logic Flaw
  {\smallemptycirc} &
  % -- Low-strength pwds
  {\smallemptycirc} &
  % -- Data Leakage
  {\smallemptycirc} &
  % -- Data Remanence
  {\smallemptycirc} &
  % -- Data Remanence
  {\smallemptycirc} &
  % -- Insec. Boot Environ.
  {\smallemptycirc} &
  % -- Micro-electr. Exposure
  {\smallemptycirc} &
  % -- Weak Signature
  {\smallemptycirc} &
  % -- Inadeq. Sig. Verif. 
  % -- SYSTEM
  {\smallemptycirc} &
  % -- Insec. Permissions
  {\smallemptycirc} &
  % -- Library Vulnerability
  {\smallemptycirc} &
  % -- OS Vulnerabilities
  {\smallemptycirc} &
  % -- Coding Errors
  {\smallfullcirc} &
  % -- Insec. Network
  %  -- SYSTEM
  {\smallfullcirc} &
  % -- Insec. User Interactions
  {\smallfullcirc} &
  % -- Comp. Provider
  % -- EXT
  {\smallemptycirc} &
  % -- Malicious Insider
  {\smallemptycirc} &
  % -- Insider Compromise
  % -- INSIDER
  {\smallemptycirc} &
  {\smallemptycirc} &
  {\smallemptycirc} &
  {\smallemptycirc} &
  {\smallemptycirc} &
  {\smallfullcirc} &
  {\smallemptycirc} &
  % data --
  {\smallemptycirc} &
  {\smallemptycirc} &
  {\smallemptycirc} &
  {\smallfullcirc} &
  {\smallemptycirc} &
  {\smallfullcirc} &
  % data --
  {\smallfullcirc} &
  % {\smallfullcirc} &
  {\smallemptycirc} &
  {\smallfullcirc} &
  % data --
  % -- &
  % -- &
  % -- &
  {\smallfullcirc} &
  {\smallemptycirc} &
  {\smallemptycirc} &
  % data --
  % {\smallfullcirc} &
  {\smallfullcirc} &
  {\smallfullcirc} &
  {\smallfullcirc} &
  {\smallfullcirc} &
  {\smallfullcirc} &
  \cellcolor{g4}{$2$} &
\cellcolor{r3}{$3$} &

% 83 total attack incidents
  
  % data --
  
  % {\smallemptycirc} &
  % {\smallemptycirc} &
  % {\smallfullcirc} &
  % {\smallemptycirc} &
  % {\smallemptycirc} &
  % {\smallemptycirc} &
  % {\smallemptycirc} &
  % {\smallemptycirc} &
  % % data --
  % {\smallemptycirc} &
  % {\smallemptycirc} &
  % % data --
  % {\smallfullcirc} &
  % {\smallemptycirc} &
  % {\smallemptycirc} &
  % {\smallfullcirc} &
  % % data --
  % {\smallemptycirc} &
  % {\smallemptycirc} &
  % {\smallemptycirc} &
  % {\smallemptycirc} &
  % {\smallemptycirc} &
  % {\smallemptycirc} &
  % {\smallemptycirc} &
  % {\smallemptycirc} &
  % {\smallemptycirc} &
  % {\smallemptycirc} &
  % {\smallemptycirc} &
  % {\smallemptycirc} &
  \cite{Ahmed2017MitigatingNetworking, Cai2014ResearchNetwork, zimba2019cryptojacking} 
   \\
 & 
   &
  IP Spoofing \cite{shrivas2020disruptive} &
  {\smallemptycirc} &
  % -- rng
  {\smallemptycirc} &
  % -- inadequ auth
  {\smallemptycirc} &
  % -- inadequ encry
  {\smallemptycirc} &
  % -- Appl. Logic Flaw
  {\smallemptycirc} &
  % -- Low-strength pwds
  {\smallemptycirc} &
  % -- Data Leakage
  {\smallemptycirc} &
  % -- Data Remanence
  {\smallemptycirc} &
  % -- Data Remanence
  {\smallemptycirc} &
  % -- Insec. Boot Environ.
  {\smallemptycirc} &
  % -- Micro-electr. Exposure
  {\smallemptycirc} &
  % -- Weak Signature
  {\smallemptycirc} &
  % -- Inadeq. Sig. Verif. 
  % -- SYSTEM
  {\smallemptycirc} &
  % -- Insec. Permissions
  {\smallemptycirc} &
  % -- Library Vulnerability
  {\smallemptycirc} &
  % -- OS Vulnerabilities
  {\smallemptycirc} &
  % -- Coding Errors
  {\smallfullcirc} &
  % -- Insec. Network
  %  -- SYSTEM
  {\smallfullcirc} &
  % -- Insec. User Interactions
  {\smallfullcirc} &
  % -- Comp. Provider
  % -- EXT
  {\smallemptycirc} &
  % -- Malicious Insider
  {\smallemptycirc} &
  % -- Insider Compromise
  % -- INSIDER
  {\smallemptycirc} &
  {\smallemptycirc} &
  {\smallemptycirc} &
  {\smallemptycirc} &
  {\smallemptycirc} &
  {\smallfullcirc} &
  {\smallemptycirc} &
  % data --
  {\smallemptycirc} &
  {\smallemptycirc} &
  {\smallemptycirc} &
  {\smallfullcirc} &
  {\smallemptycirc} &
  {\smallfullcirc} & 
  {\smallfullcirc} &
  % {\smallemptycirc} &
  {\smallemptycirc} &
  {\smallfullcirc} &
  % -- &
  % -- &
  % -- &
  {\smallfullcirc} &
  {\smallemptycirc} &
  {\smallemptycirc} &
  % {\smallfullcirc} &
  {\smallfullcirc} &
  {\smallfullcirc} &
  {\smallfullcirc} &
  {\smallfullcirc} &
  {\smallfullcirc} &
  \cellcolor{g2}{$1$} &
\cellcolor{g0}{$0$} &

  
  % {\smallemptycirc} &
  % {\smallemptycirc} &
  % {\smallfullcirc} &
  % {\smallemptycirc} &
  % {\smallemptycirc} &
  % {\smallemptycirc} &
  % {\smallemptycirc} &
  % {\smallemptycirc} &
  % {\smallemptycirc} &
  % {\smallemptycirc} &
  % {\smallfullcirc} &
  % {\smallfullcirc} &
  % {\smallemptycirc} &
  % {\smallemptycirc} &
  % {\smallemptycirc} &
  % {\smallemptycirc} &
  % {\smallemptycirc} &
  % {\smallemptycirc} &
  % {\smallemptycirc} &
  % {\smallemptycirc} &
  % {\smallemptycirc} &
  % {\smallemptycirc} &
  % {\smallemptycirc} &
  % {\smallemptycirc} &
  % {\smallemptycirc} &
  % {\smallemptycirc} &
  \cite{Bhirud2011LightPrevention, Cai2014ResearchNetwork, zimba2019cryptojacking} 
   \\
%    & 
%    &
%   \acs{icmp} Redirection \cite{Feng2023Man-in-the-middleRedirects} &
%   {\smallemptycirc} &
%   % -- rng
%   {\smallemptycirc} &
%   % -- inadequ auth
%   {\smallemptycirc} &
%   % -- inadequ encry
%   {\smallemptycirc} &
%   % -- Appl. Logic Flaw
%   {\smallemptycirc} &
%   % -- Low-strength pwds
%   {\smallemptycirc} &
%   % -- Data Leakage
%   {\smallemptycirc} &
%   % -- Data Remanence
%   {\smallemptycirc} &
%   % -- Data Remanence
%   {\smallemptycirc} &
%   % -- Insec. Boot Environ.
%   {\smallemptycirc} &
%   % -- Micro-electr. Exposure
%   {\smallemptycirc} &
%   % -- Weak Signature
%   {\smallemptycirc} &
%   % -- Inadeq. Sig. Verif. 
%   % -- SYSTEM
%   {\smallemptycirc} &
%   % -- Insec. Permissions
%   {\smallemptycirc} &
%   % -- Library Vulnerability
%   {\smallemptycirc} &
%   % -- OS Vulnerabilities
%   {\smallemptycirc} &
%   % -- Coding Errors
%   {\smallfullcirc} &
%   % -- Insec. Network
%   %  -- SYSTEM
%   {\smallemptycirc} &
%   % -- Insec. User Interactions
%   {\smallemptycirc} &
%   % -- Comp. Provider
%   % -- EXT
%   {\smallemptycirc} &
%   % -- Malicious Insider
%   {\smallemptycirc} &
%   % -- Insider Compromise
%   % -- INSIDER
%   {\smallemptycirc} &
%   {\smallemptycirc} &
%   {\smallemptycirc} &
%   {\smallemptycirc} &
%   {\smallemptycirc} &
%   {\smallfullcirc} &
%   {\smallemptycirc} &
%   % data --
%   {\smallemptycirc} &
%   {\smallemptycirc} &
%   {\smallemptycirc} &
%   {\smallfullcirc} &
%   {\smallemptycirc} &
%   {\smallemptycirc} & 
%   {\smallfullcirc} &
%   % {\smallemptycirc} &
%   {\smallemptycirc} &
%   {\smallfullcirc} &
%   % -- &
%   % -- &
%   % -- &
%   {\smallfullcirc} &
%   {\smallemptycirc} &
%   {\smallemptycirc} &
%   % {\smallfullcirc} &
%   {\smallfullcirc} &
%   {\smallfullcirc} &
%   {\smallfullcirc} &
%   {\smallfullcirc} &
%   {\smallfullcirc} &
%   \cellcolor{g2}{$1$($3\%$)} &
% \cellcolor{g0}{$0$($0\%$)} &
%   \cite{Feng2023Man-in-the-middleRedirects} 
%    \\
&
&
\acs{bgp} Hijacking \cite{ekparinya2018impact} &
{\smallemptycirc} & % RNG
{\smallemptycirc} & % Inadequate Authentication
{\smallemptycirc} & % Inadequate Encryption
{\smallemptycirc} & % Application Logic Flaw
{\smallemptycirc} & % Low-strength Passwords
{\smallemptycirc} & % Data Leakage
{\smallemptycirc} & % Data Remanence
{\smallemptycirc} & % Data Manipulation
{\smallemptycirc} & % Insecure Boot Environment
{\smallemptycirc} & % Microelectronic Component Exposure
{\smallemptycirc} & % Weak Signature
{\smallemptycirc} & % Inadequate Signature Verification
{\smallemptycirc} & % Insecure Permissions
{\smallemptycirc} & % Library Vulnerability
{\smallemptycirc} & % OS Vulnerabilities
{\smallemptycirc} & % Coding Errors
{\smallfullcirc} & % Insecure Network
{\smallfullcirc} & % Insecure User Interactions
{\smallfullcirc} & % Compromised Provider
{\smallemptycirc} & % Malicious Insider
{\smallemptycirc} & % Compromised Insider
{\smallemptycirc} & % Private Key
{\smallemptycirc} & % Signature
{\smallemptycirc} & % Mnemonics
{\smallemptycirc} & % KEK or Password (pw)
{\smallemptycirc} & % Memory
{\smallfullcirc} & % State Transition Info
{\smallemptycirc} & % Nonce
{\smallemptycirc} & % KeyGen
{\smallemptycirc} & % UserAuth
{\smallemptycirc} & % KeyStore
{\smallfullcirc} & % CreateTxn
{\smallemptycirc} & % TnxSign
{\smallfullcirc} & % TnxVer
{\smallfullcirc} & % Service Provider
{\smallemptycirc} & % Operating System
{\smallfullcirc} & % Wallet User
{\smallfullcirc} & % Transaction Alteration
{\smallemptycirc} & % Credential Compromise
{\smallemptycirc} & % Network Disruption
{\smallfullcirc} & % Desktop Wallet
{\smallfullcirc} & % Browser Wallet
{\smallfullcirc} & % Mobile Wallet
{\smallfullcirc} & % Smart Wallet
{\smallfullcirc} & % Hardware Wallet
\cellcolor{g2}{$1$} & % Academic Papers
\cellcolor{r2}{$1$} & % Notable Incidents
\cite{ekparinya2018impact} % Possible Defence
\\
 &
  \multirow{2}{*}{\hyperref[sec:mitm]{Service Denial}} &
  \acs{icmp} Flooding \cite{chaganti2022comprehensive, chaganti2022role} &
  {\smallemptycirc} &
  % -- rng
  {\smallemptycirc} &
  % -- inadequ auth
  {\smallemptycirc} &
  % -- inadequ encry
  {\smallemptycirc} &
  % -- Appl. Logic Flaw
  {\smallemptycirc} &
  % -- Low-strength pwds
  {\smallemptycirc} &
  % -- Data Leakage
  {\smallemptycirc} &
  % -- Data Remanence
  {\smallemptycirc} &
  % -- Data Remanence
  {\smallemptycirc} &
  % -- Insec. Boot Environ.
  {\smallemptycirc} &
  % -- Micro-electr. Exposure
  {\smallemptycirc} &
  % -- Weak Signature
  {\smallemptycirc} &
  % -- Inadeq. Sig. Verif. 
  % -- SYSTEM
  {\smallemptycirc} &
  % -- Insec. Permissions
  {\smallemptycirc} &
  % -- Library Vulnerability
  {\smallemptycirc} &
  % -- OS Vulnerabilities
  {\smallemptycirc} &
  % -- Coding Errors
  {\smallfullcirc} &
  % -- Insec. Network
  %  -- SYSTEM
  {\smallemptycirc} &
  % -- Insec. User Interactions
  {\smallemptycirc} &
  % -- Comp. Provider
  % -- EXT
  {\smallemptycirc} &
  % -- Malicious Insider
  {\smallemptycirc} &
  % -- Insider Compromise
  % -- INSIDER
  {\smallemptycirc} &
  {\smallemptycirc} &
  {\smallemptycirc} &
  {\smallemptycirc} &
  {\smallemptycirc} &
  {\smallfullcirc} &
  {\smallemptycirc} &
  {\smallemptycirc} &
  {\smallemptycirc} &
  % changed back to empty circle as user authentication does not rely on internet in most cases
  %  {\smallhalfcirc} &

  {\smallemptycirc} &
  {\smallemptycirc} &
  {\smallemptycirc} &
  {\smallhalfcirc} & 
  {\smallfullcirc} &
  % {\smallhalfcirc} &
  {\smallemptycirc} &
  {\smallemptycirc} &
  % -- &
  % -- &
  % -- &
  {\smallemptycirc} &
  {\smallemptycirc} &
  {\smallfullcirc} &
  % {\smallfullcirc} &
  {\smallfullcirc} &
  {\smallfullcirc} &
  {\smallfullcirc} &
  {\smallfullcirc} &
  {\smallfullcirc} &
 \cellcolor{g4}{$2$} &
\cellcolor{g0}{$0$} &

  
  % {\smallemptycirc} &
  % {\smallemptycirc} &
  % {\smallfullcirc} &
  % {\smallemptycirc} &
  % {\smallemptycirc} &
  % {\smallemptycirc} &
  % {\smallemptycirc} &
  % {\smallemptycirc} &
  % {\smallfullcirc} &
  % {\smallemptycirc} &
  % {\smallfullcirc} &
  % {\smallfullcirc} &
  % {\smallemptycirc} &
  % {\smallemptycirc} &
  % {\smallemptycirc} &
  % {\smallemptycirc} &
  % {\smallemptycirc} &
  % {\smallemptycirc} &
  % {\smallemptycirc} &
  % {\smallemptycirc} &
  % {\smallemptycirc} &
  % {\smallemptycirc} &
  % {\smallemptycirc} &
  % {\smallemptycirc} &
  % {\smallemptycirc} &
  % {\smallemptycirc} &
  \cite{liu2018deep, Bhirud2011LightPrevention} 
   \\
 &
   &
  TCP SYN Flooding \cite{chaganti2022comprehensive} &
  {\smallemptycirc} &
  % -- rng
  {\smallemptycirc} &
  % -- inadequ auth
  {\smallemptycirc} &
  % -- inadequ encry
  {\smallemptycirc} &
  % -- Appl. Logic Flaw
  {\smallemptycirc} &
  % -- Low-strength pwds
  {\smallemptycirc} &
  % -- Data Leakage
  {\smallemptycirc} &
  % -- Data Remanence
  {\smallemptycirc} &
  % -- Data Remanence
  {\smallemptycirc} &
  % -- Insec. Boot Environ.
  {\smallemptycirc} &
  % -- Micro-electr. Exposure
  {\smallemptycirc} &
  % -- Weak Signature
  {\smallemptycirc} &
  % -- Inadeq. Sig. Verif. 
  % -- SYSTEM
  {\smallemptycirc} &
  % -- Insec. Permissions
  {\smallemptycirc} &
  % -- Library Vulnerability
  {\smallemptycirc} &
  % -- OS Vulnerabilities
  {\smallemptycirc} &
  % -- Coding Errors
  {\smallfullcirc} &
  % -- Insec. Network
  %  -- SYSTEM
  {\smallemptycirc} &
  % -- Insec. User Interactions
  {\smallemptycirc} &
  % -- Comp. Provider
  % -- EXT
  {\smallemptycirc} &
  % -- Malicious Insider
  {\smallemptycirc} &
  % -- Insider Compromise
  % -- INSIDER
 {\smallemptycirc} &
  {\smallemptycirc} &
  {\smallemptycirc} &
  {\smallemptycirc} &
  {\smallemptycirc} &
 {\smallfullcirc} &
  {\smallemptycirc} &
  {\smallemptycirc} &
  {\smallemptycirc} &
  {\smallemptycirc} &
  {\smallemptycirc} &
  {\smallemptycirc} &
  {\smallhalfcirc} & 
  {\smallfullcirc} &
  % {\smallhalfcirc} &
  {\smallemptycirc} &
  {\smallemptycirc} &
  % -- &
  % -- &
  % -- &
  {\smallemptycirc} &
  {\smallemptycirc} &
  {\smallfullcirc} &
  % {\smallfullcirc} &
  {\smallfullcirc} &
  {\smallfullcirc} &
  {\smallfullcirc} &
  {\smallfullcirc} &
  {\smallfullcirc} &
  \cellcolor{g2}{$1$} &
\cellcolor{g0}{$0$} &
  
  % {\smallemptycirc} &
  % {\smallemptycirc} &
  % {\smallfullcirc} &
  % {\smallemptycirc} &
  % {\smallemptycirc} &
  % {\smallemptycirc} &
  % {\smallemptycirc} &
  % {\smallemptycirc} &
  % {\smallfullcirc} &
  % {\smallfullcirc} &
  % {\smallfullcirc} &
  % {\smallfullcirc} &
  % {\smallemptycirc} &
  % {\smallemptycirc} &
  % {\smallemptycirc} &
  % {\smallemptycirc} &
  % {\smallemptycirc} &
  % {\smallemptycirc} &
  % {\smallemptycirc} &
  % {\smallemptycirc} &
  % {\smallemptycirc} &
  % {\smallemptycirc} &
  % {\smallemptycirc} &
  % {\smallemptycirc} &
  % {\smallemptycirc} &
  % {\smallemptycirc} &
  \cite{sathwara2017distributed, liu2018deep} 
  % Bhirud2011LightPrevention, Cai2014ResearchNetwork, zimba2019cryptojacking
   \\
  \midrule
\multirow{8}{*}{Application} &
  \multirow{2}{*}{\hyperref[sec:malware]{Malware Execution}} &
  Clipboard Hijack \cite{ivanov2021ethclipper, Kim2018RiskThreats, li2020android} &
  {\smallemptycirc} &
  % -- rng
  {\smallemptycirc} &
  % -- inadequ auth
  {\smallfullcirc} &
  % -- inadequ encry
  {\smallemptycirc} &
  % -- Appl. Logic Flaw
  {\smallemptycirc} &
  % -- Low-strength pwds
  {\smallemptycirc} &
  % -- Data Leakage
  {\smallemptycirc} &
  % -- Data Remanence
  {\smallfullcirc} &
  % -- Data Manipulation
  {\smallemptycirc} &
  % -- Insec. Boot Environ.
  {\smallemptycirc} &
  % -- Micro-electr. Exposure
  {\smallemptycirc} &
  % -- Weak Signature
  {\smallemptycirc} &
  % -- Inadeq. Sig. Verif. 
  % -- SYSTEM
  {\smallfullcirc} &
  % -- Insec. Permissions
  {\smallemptycirc} &
  % -- Library Vulnerability
  {\smallfullcirc} &
  % -- OS Vulnerabilities
  {\smallemptycirc} &
  % -- Coding Errors
  {\smallemptycirc} &
  % -- Insec. Network
  %  -- SYSTEM
  {\smallfullcirc} &
  % -- Insec. User Interactions
  {\smallemptycirc} &
  % -- Comp. Provider
  % -- EXT
  {\smallemptycirc} &
  % -- Malicious Insider
  {\smallemptycirc} &
  % -- Insider Compromise
  % -- INSIDER
  {\smallemptycirc} &
  {\smallemptycirc} &
  {\smallemptycirc} &
  {\smallemptycirc} &
  {\smallemptycirc} &
  {\smallfullcirc} &
  {\smallemptycirc} &
  {\smallemptycirc} &
  {\smallemptycirc} &
  {\smallemptycirc} &
  {\smallfullcirc} &
  {\smallemptycirc} &
  {\smallemptycirc} &
  {\smallemptycirc} &
  % {\smallemptycirc} &
  {\smallfullcirc} &
  {\smallfullcirc} &
  % -- &
  % -- &
  % -- &
  {\smallfullcirc} &
  {\smallemptycirc} &
  {\smallemptycirc} &
  % {\smallfullcirc} &
  {\smallfullcirc} &
  {\smallfullcirc} &
  {\smallfullcirc} &
  {\smallfullcirc} &
  {\smallfullcirc} &
  \cellcolor{g6}{$3$} &
\cellcolor{r4}\multirow{-1}{*}{$8$} &

  
  % {\smallemptycirc} &
  % {\smallemptycirc} &
  % {\smallemptycirc} &
  % {\smallemptycirc} &
  % {\smallemptycirc} &
  % {\smallemptycirc} &
  % {\smallemptycirc} &
  % {\smallemptycirc} &
  % {\smallemptycirc} &
  % {\smallemptycirc} &
  % {\smallemptycirc} &
  % {\smallemptycirc} &
  % {\smallfullcirc} &
  % {\smallemptycirc} &
  % {\smallfullcirc} &
  % {\smallemptycirc} &
  % {\smallemptycirc} &
  % {\smallemptycirc} &
  % {\smallemptycirc} &
  % {\smallemptycirc} &
  % {\smallemptycirc} &
  % {\smallemptycirc} &
  % {\smallemptycirc} &
  % {\smallemptycirc} &
  % {\smallemptycirc} &
  % {\smallemptycirc} &
  \cite{ferdous2023review, li2020android}
   \\
 &
   &
  Spyware \cite{weichbroth2023security, aldawood2020advanced} &
  {\smallemptycirc} &
  % -- rng
  {\smallemptycirc} &
  % -- inadequ auth
  {\smallfullcirc} &
  % -- inadequ encry
  {\smallemptycirc} &
  % -- Appl. Logic Flaw
  {\smallemptycirc} &
  % -- Low-strength pwds
  {\smallfullcirc} &
  % -- Data Leakage
  {\smallemptycirc} &
  % -- Data Remanence
  {\smallemptycirc} &
  % -- Data Remanence
  {\smallemptycirc} &
  % -- Insec. Boot Environ.
  {\smallemptycirc} &
  % -- Micro-electr. Exposure
  {\smallemptycirc} &
  % -- Weak Signature
  {\smallemptycirc} &
  % -- Inadeq. Sig. Verif. 
  % -- SYSTEM
  {\smallfullcirc} &
  % -- Insec. Permissions
  {\smallemptycirc} &
  % -- Library Vulnerability
  {\smallfullcirc} &
  % -- OS Vulnerabilities
  {\smallemptycirc} &
  % -- Coding Errors
  {\smallemptycirc} &
  % -- Insec. Network
  %  -- SYSTEM
  {\smallfullcirc} &
  % -- Insec. User Interactions
  {\smallemptycirc} &
  % -- Comp. Provider
  % -- EXT
  {\smallemptycirc} &
  % -- Malicious Insider
  {\smallemptycirc} &
  % -- Insider Compromise
  % -- INSIDER
  {\smallfullcirc} &
  {\smallemptycirc} &
  {\smallfullcirc} &
  {\smallfullcirc} &
  {\smallemptycirc} &
  {\smallemptycirc} &
  {\smallemptycirc} &
  {\smallfullcirc} &
  {\smallemptycirc} &
  {\smallfullcirc} &
  {\smallemptycirc} &
  {\smallemptycirc} &
  {\smallemptycirc} &
  {\smallemptycirc} &
  % {\smallemptycirc} &
  {\smallemptycirc} &
  {\smallfullcirc} &
  % -- &
  % -- &
  % -- &
  {\smallemptycirc} &
  {\smallfullcirc} &
  {\smallemptycirc} &
  % {\smallfullcirc} &
  {\smallfullcirc} &
  {\smallfullcirc} &
  {\smallfullcirc} &
  {\smallemptycirc} &
  {\smallemptycirc} &
  \cellcolor{g4}{$2$} &
  \cellcolor{r4}{} &
  % \multicolumn{1}{c}{} &
% \cellcolor{r4}{$8$($9\%$)} &
% \cellcolor{r4}{}

  
  % {\smallemptycirc} &
  % {\smallemptycirc} &
  % {\smallemptycirc} &
  % {\smallemptycirc} &
  % {\smallemptycirc} &
  % {\smallemptycirc} &
  % {\smallemptycirc} &
  % {\smallemptycirc} &
  % {\smallemptycirc} &
  % {\smallemptycirc} &
  % {\smallemptycirc} &
  % {\smallemptycirc} &
  % {\smallfullcirc} &
  % {\smallemptycirc} &
  % {\smallemptycirc} &
  % {\smallemptycirc} &
  % {\smallemptycirc} &
  % {\smallemptycirc} &
  % {\smallemptycirc} &
  % {\smallemptycirc} &
  % {\smallemptycirc} &
  % {\smallemptycirc} &
  % {\smallemptycirc} &
  % {\smallemptycirc} &
  % {\smallemptycirc} &
  % {\smallemptycirc} &
  % Anti-Malware Software \cite{ferdous2023review} 
  \cite{ferdous2023review}
   \\
 % &
 %   &
 %  Ransomware \cite{conti2018economic, robinson2022new} &
 %  {\smallemptycirc} &
 %  % -- rng
 %  {\smallemptycirc} &
 %  % -- inadequ auth
 %  {\smallfullcirc} &
 %  % -- inadequ encry
 %  {\smallemptycirc} &
 %  % -- Appl. Logic Flaw
 %  {\smallemptycirc} &
 %  % -- Low-strength pwds
 %  {\smallfullcirc} &
 %  % -- Data Leakage
 %  {\smallemptycirc} &
 %  % -- Data Remanence
 %  {\smallfullcirc} &
 %  % -- Data Remanence
 %  {\smallemptycirc} &
 %  % -- Insec. Boot Environ.
 %  {\smallemptycirc} &
 %  % -- Micro-electr. Exposure
 %  {\smallemptycirc} &
 %  % -- Weak Signature
 %  {\smallemptycirc} &
 %  % -- Inadeq. Sig. Verif. 
 %  % -- SYSTEM
 %  {\smallfullcirc} &
 %  % -- Insec. Permissions
 %  {\smallemptycirc} &
 %  % -- Library Vulnerability
 %  {\smallfullcirc} &
 %  % -- OS Vulnerabilities
 %  {\smallemptycirc} &
 %  % -- Coding Errors
 %  {\smallemptycirc} &
 %  % -- Insec. Network
 %  %  -- SYSTEM
 %  {\smallfullcirc} &
 %  % -- Insec. User Interactions
 %  {\smallemptycirc} &
 %  % -- Comp. Provider
 %  % -- EXT
 %  {\smallemptycirc} &
 %  % -- Malicious Insider
 %  {\smallemptycirc} &
 %  % -- Insider Compromise
 %  % -- INSIDER
 %  {\smallemptycirc} &
 %  {\smallemptycirc} &
 %  {\smallemptycirc} &
 %  {\smallemptycirc} &
 %  {\smallemptycirc} &
 %  {\smallemptycirc} &
 %  {\smallemptycirc} &
 %  {\smallemptycirc} &
 %  {\smallemptycirc} &
 %  {\smallemptycirc} &
 %  {\smallemptycirc} &
 %  {\smallemptycirc} &
 %  {\smallemptycirc} &
 %  {\smallemptycirc} &
 %  % {\smallemptycirc} &
 %  {\smallemptycirc} &
 %  {\smallfullcirc} &
 %  % -- &
 %  % -- &
 %  % -- &
 %  {\smallemptycirc} &
 %  {\smallfullcirc} &
 %  {\smallemptycirc} &
 %  % {\smallfullcirc} &
 %  {\smallemptycirc} &
 %  {\smallfullcirc} &
 %  {\smallemptycirc} &
 %  {\smallemptycirc} &
 %  {\smallfullcirc} &

  
 %  % {\smallemptycirc} &
 %  % {\smallemptycirc} &
 %  % {\smallemptycirc} &
 %  % {\smallemptycirc} &
 %  % {\smallemptycirc} &
 %  % {\smallemptycirc} &
 %  % {\smallemptycirc} &
 %  % {\smallemptycirc} &
 %  % {\smallemptycirc} &
 %  % {\smallemptycirc} &
 %  % {\smallemptycirc} &
 %  % {\smallemptycirc} &
 %  % {\smallfullcirc} &
 %  % {\smallemptycirc} &
 %  % {\smallemptycirc} &
 %  % {\smallemptycirc} &
 %  % {\smallemptycirc} &
 %  % {\smallemptycirc} &
 %  % {\smallemptycirc} &
 %  % {\smallemptycirc} &
 %  % {\smallemptycirc} &
 %  % {\smallemptycirc} &
 %  % {\smallemptycirc} &
 %  % {\smallemptycirc} &
 %  % {\smallemptycirc} &
 %  % {\smallemptycirc} &
 %  % Anti-Malware Software \cite{ferdous2023review} 
 %  \cite{ferdous2023review}
 %   \\
 &
  \multirow{2}{*}{\hyperref[sec:logic_expl]{Logic Exploitation}} &
  Constructor Hijack \cite{palladino2017parity} &
 {\smallemptycirc}&{\smallemptycirc}&{\smallemptycirc}&{\smallfullcirc}&{\smallemptycirc}&
 {\smallemptycirc}&{\smallemptycirc}&{\smallemptycirc}&{\smallemptycirc}&{\smallemptycirc}&
 {\smallemptycirc}&{\smallemptycirc}&{\smallfullcirc}&{\smallfullcirc}&{\smallemptycirc}&{\smallfullcirc}&
 {\smallemptycirc}&{\smallemptycirc}&{\smallemptycirc}&{\smallemptycirc}&{\smallemptycirc}&
 {\smallemptycirc}&{\smallemptycirc}&{\smallemptycirc}&{\smallemptycirc}&{\smallemptycirc}&
 {\smallemptycirc}&{\smallemptycirc}&{\smallemptycirc}&{\smallfullcirc}&{\smallemptycirc}&
 {\smallfullcirc}&{\smallemptycirc}&{\smallemptycirc}&{\smallemptycirc}&{\smallemptycirc}&
 {\smallemptycirc}&{\smallemptycirc}&{\smallemptycirc}&{\smallemptycirc}&{\smallemptycirc}&{\smallemptycirc}&{\smallemptycirc}&{\smallfullcirc}&{\smallemptycirc}&
 \cellcolor{g0}{$0$}&\cellcolor{r2}{$1$}&\cite{palladino2017parity}
   \\
 &
   &
  Upgrade-Path Hijack \cite{bybit_certik}  &
 {\smallemptycirc}&{\smallfullcirc}&{\smallemptycirc}&{\smallfullcirc}&{\smallemptycirc}&
 {\smallemptycirc}&{\smallemptycirc}&{\smallemptycirc}&{\smallemptycirc}&{\smallemptycirc}&
 {\smallemptycirc}&{\smallemptycirc}&{\smallemptycirc}&{\smallfullcirc}&{\smallemptycirc}&{\smallemptycirc}&
 {\smallemptycirc}&{\smallfullcirc}&{\smallfullcirc}&{\smallemptycirc}&{\smallemptycirc}&
 {\smallemptycirc}&{\smallemptycirc}&{\smallemptycirc}&{\smallemptycirc}&{\smallemptycirc}&
 {\smallfullcirc}&{\smallemptycirc}&{\smallfullcirc}&{\smallemptycirc}&{\smallemptycirc}&
 {\smallfullcirc}&{\smallemptycirc}&{\smallemptycirc}&{\smallemptycirc}&{\smallemptycirc}&
 {\smallfullcirc}&{\smallemptycirc}&{\smallfullcirc}&{\smallemptycirc}&{\smallemptycirc}&{\smallemptycirc}&{\smallemptycirc}&{\smallfullcirc}&{\smallfullcirc}&
 \cellcolor{g0}{$0$}&\cellcolor{r3}{$2$}&\cite{bybit_certik}
   \\ 
    &
  \multirow{2}{*}{\hyperref[sec:privilege]{Privilege Escalation}} &
  Android Root Privilege \cite{he2020security} &
  {\smallemptycirc} &
  % -- rng
  {\smallemptycirc} &
  % -- inadequ auth
  {\smallemptycirc} &
  % -- inadequ encry
  {\smallemptycirc} &
  % -- Appl. Logic Flaw
  {\smallemptycirc} &
  % -- Low-strength pwds
  {\smallemptycirc} &
  % -- Data Leakage
  {\smallemptycirc} &
  % -- Data Remanence
  {\smallemptycirc} &
  % -- Data Remanence
  {\smallemptycirc} &
  % -- Insec. Boot Environ.
  {\smallemptycirc} &
  % -- Micro-electr. Exposure
  {\smallemptycirc} &
  % -- Weak Signature
  {\smallemptycirc} &
  % -- Inadeq. Sig. Verif. 
  % -- SYSTEM
  {\smallfullcirc} &
  % -- Insec. Permissions
  {\smallemptycirc} &
  % -- Library Vulnerability
  {\smallfullcirc} &
  % -- OS Vulnerabilities
  {\smallemptycirc} &
  % -- Coding Errors
  {\smallemptycirc} &
  % -- Insec. Network
  %  -- SYSTEM
  {\smallemptycirc} &
  % -- Insec. User Interactions
  {\smallemptycirc} &
  % -- Comp. Provider
  % -- EXT
  {\smallemptycirc} &
  % -- Malicious Insider
  {\smallemptycirc} &
  % -- Insider Compromise
  % -- INSIDER
  {\smallfullcirc} &
  {\smallemptycirc} &
  {\smallfullcirc} &
  {\smallemptycirc} &
  {\smallemptycirc} &
  {\smallemptycirc} &
  {\smallemptycirc} &
  {\smallfullcirc} &
  {\smallemptycirc} &
  {\smallfullcirc} &
  {\smallemptycirc} &
  {\smallemptycirc} &
  {\smallemptycirc} &
  {\smallemptycirc} &
  % {\smallemptycirc} &
  {\smallfullcirc} &
  {\smallemptycirc} &
  % -- &
  % -- &
  % -- &
  {\smallemptycirc} &
  {\smallfullcirc} &
  {\smallemptycirc} &
  % {\smallfullcirc} &
  {\smallemptycirc} &
  {\smallemptycirc} &
  {\smallfullcirc} &
  {\smallemptycirc} &
  {\smallemptycirc} &
  \cellcolor{g2}{$1$} &
\cellcolor{r0}{$0$} &

  
  % {\smallemptycirc} &
  % {\smallemptycirc} &
  % {\smallemptycirc} &
  % {\smallemptycirc} &
  % {\smallemptycirc} &
  % {\smallemptycirc} &
  % {\smallemptycirc} &
  % {\smallemptycirc} &
  % {\smallemptycirc} &
  % {\smallemptycirc} &
  % {\smallemptycirc} &
  % {\smallemptycirc} &
  % {\smallemptycirc} &
  % {\smallemptycirc} &
  % {\smallemptycirc} &
  % {\smallfullcirc} &
  % {\smallemptycirc} &
  % {\smallemptycirc} &
  % {\smallemptycirc} &
  % {\smallemptycirc} &
  % {\smallemptycirc} &
  % {\smallemptycirc} &
  % {\smallemptycirc} &
  % {\smallemptycirc} &
  % {\smallemptycirc} &
  % {\smallemptycirc} &
  % Code Obfuscation 
  \cite{indusface} 
   \\
 &
   &
  Android USB Debugging \cite{he2020security} &
  {\smallemptycirc} &
  % -- rng
  {\smallemptycirc} &
  % -- inadequ auth
  {\smallemptycirc} &
  % -- inadequ encry
  {\smallemptycirc} &
  % -- Appl. Logic Flaw
  {\smallemptycirc} &
  % -- Low-strength pwds
  {\smallemptycirc} &
  % -- Data Leakage
  {\smallemptycirc} &
  % -- Data Remanence
  {\smallemptycirc} &
  % -- Data Remanence
  {\smallemptycirc} &
  % -- Insec. Boot Environ.
  {\smallemptycirc} &
  % -- Micro-electr. Exposure
  {\smallemptycirc} &
  % -- Weak Signature
  {\smallemptycirc} &
  % -- Inadeq. Sig. Verif. 
  % -- SYSTEM
  {\smallfullcirc} &
  % -- Insec. Permissions
  {\smallemptycirc} &
  % -- Library Vulnerability
  {\smallfullcirc} &
  % -- OS Vulnerabilities
  {\smallemptycirc} &
  % -- Coding Errors
  {\smallemptycirc} &
  % -- Insec. Network
  %  -- SYSTEM
  {\smallemptycirc} &
  % -- Insec. User Interactions
  {\smallemptycirc} &
  % -- Comp. Provider
  % -- EXT
  {\smallemptycirc} &
  % -- Malicious Insider
  {\smallemptycirc} &
  % -- Insider Compromise
  % -- INSIDER
  {\smallemptycirc} &
  {\smallemptycirc} &
  {\smallfullcirc} &
  {\smallfullcirc} &
  {\smallemptycirc} &
  {\smallemptycirc} &
  {\smallemptycirc} &
  {\smallfullcirc} &
  {\smallfullcirc} &
  {\smallemptycirc} &
  {\smallemptycirc} &
  {\smallemptycirc} &
  {\smallemptycirc} &
  {\smallemptycirc} &
  % {\smallemptycirc} &
  {\smallfullcirc} &
  {\smallemptycirc} &
  % -- &
  % -- &
  % -- &
  {\smallemptycirc} &
  {\smallfullcirc} &
  {\smallemptycirc} &
  % {\smallfullcirc} &
  {\smallemptycirc} &
  {\smallemptycirc} &
  {\smallfullcirc} &
  {\smallemptycirc} &
  {\smallemptycirc} &
  \cellcolor{g2}{$1$} &
\cellcolor{r0}{$0$} &

  
  % {\smallemptycirc} &
  % {\smallfullcirc} &
  % {\smallemptycirc} &
  % {\smallemptycirc} &
  % {\smallemptycirc} &
  % {\smallemptycirc} &
  % {\smallemptycirc} &
  % {\smallemptycirc} &
  % {\smallemptycirc} &
  % {\smallemptycirc} &
  % {\smallemptycirc} &
  % {\smallemptycirc} &
  % {\smallemptycirc} &
  % {\smallemptycirc} &
  % {\smallemptycirc} &
  % {\smallemptycirc} &
  % {\smallemptycirc} &
  % {\smallfullcirc} &
  % {\smallemptycirc} &
  % {\smallemptycirc} &
  % {\smallemptycirc} &
  % {\smallemptycirc} &
  % {\smallemptycirc} &
  % {\smallemptycirc} &
  % {\smallemptycirc} &
  % {\smallemptycirc} &
  % Access Control Restrictions** 
  \cite{qi2012spad, li2020android}
   \\ 
  % \cite{Tirronen2018StoppingData, indusface}
   &
\multirow{2}{*}{\hyperref[sec:social]{Social Engineering}} &
   Phishing \cite{andryukhin2019phishing} & 
  {\smallemptycirc} &
  % -- rng
  {\smallemptycirc} &
  % -- inadequ auth
  {\smallemptycirc} &
  % -- inadequ encry
  {\smallemptycirc} &
  % -- Appl. Logic Flaw
  {\smallemptycirc} &
  % -- Low-strength pwds
  {\smallemptycirc} &
  % -- Data Leakage
  {\smallemptycirc} &
  % -- Data Remanence
  {\smallemptycirc} &
  % -- Data Remanence
  {\smallemptycirc} &
  % -- Insec. Boot Environ.
  {\smallemptycirc} &
  % -- Micro-electr. Exposure
  {\smallemptycirc} &
  % -- Weak Signature
  {\smallemptycirc} &
  % -- Inadeq. Sig. Verif. 
  % -- SYSTEM
  {\smallemptycirc} &
  % -- Insec. Permissions
  {\smallemptycirc} &
  % -- Library Vulnerability
  {\smallemptycirc} &
  % -- OS Vulnerabilities
  {\smallemptycirc} &
  % -- Coding Errors
  {\smallemptycirc} &
  % -- Insec. Network
  %  -- SYSTEM
  {\smallfullcirc} &
  % -- Insec. User Interactions
  {\smallfullcirc} &
  % -- Comp. Provider
  % -- EXT
  {\smallfullcirc} &
  % -- Malicious Insider
  {\smallfullcirc} &
  % -- Insider Compromise
  % -- INSIDER
  {\smallfullcirc} &
  {\smallemptycirc} &
  {\smallfullcirc} &
  {\smallfullcirc} &
  {\smallemptycirc} &
  {\smallemptycirc} &
  {\smallemptycirc} &
  {\smallemptycirc} &
  {\smallfullcirc} &
  {\smallemptycirc} &
  {\smallemptycirc} &
  {\smallemptycirc} &
  {\smallemptycirc} &
  {\smallemptycirc} &
  % {\smallemptycirc} &
  {\smallemptycirc} &
  {\smallfullcirc} &
  % -- &
  % -- &
  % -- &
  {\smallemptycirc} &
  {\smallfullcirc} &
  {\smallemptycirc} &
  % {\smallfullcirc} &
  {\smallfullcirc} &
  {\smallfullcirc} &
  {\smallfullcirc} &
  {\smallfullcirc} &
  {\smallfullcirc} &
    \cellcolor{g2}{$1$} &
\cellcolor{r5}{$15$} &

  
  % {\smallemptycirc} &
  % {\smallemptycirc} &
  % {\smallemptycirc} &
  % {\smallfullcirc} &
  % {\smallemptycirc} &
  % {\smallemptycirc} &
  % {\smallfullcirc} &
  % {\smallfullcirc} &
  % {\smallemptycirc} &
  % {\smallemptycirc} &
  % {\smallemptycirc} &
  % {\smallemptycirc} &
  % {\smallemptycirc} &
  % {\smallemptycirc} &
  % {\smallemptycirc} &
  % {\smallemptycirc} &
  % {\smallemptycirc} &
  % {\smallemptycirc} &
  % {\smallemptycirc} &
  % {\smallemptycirc} &
  % {\smallemptycirc} &
  % {\smallemptycirc} &
  % {\smallemptycirc} &
  % {\smallemptycirc} &
  % {\smallemptycirc} &
  % {\smallemptycirc} &
   % Multi-factor Authentication 
   \cite{Aratani2015AuthenticationChannel, bip11, Lindell2020SecureComputation} 
   \\
   &
&
   Address Poisoning \cite{MetaMaskScam} & 
  {\smallemptycirc} &
  % -- rng
  {\smallemptycirc} &
  % -- inadequ auth
  {\smallemptycirc} &
  % -- inadequ encry
  {\smallemptycirc} &
  % -- Appl. Logic Flaw
  {\smallemptycirc} &
  % -- Low-strength pwds
  {\smallemptycirc} &
  % -- Data Leakage
  {\smallemptycirc} &
  % -- Data Remanence
  {\smallemptycirc} &
  % -- Data Remanence
  {\smallemptycirc} &
  % -- Insec. Boot Environ.
  {\smallemptycirc} &
  % -- Micro-electr. Exposure
  {\smallemptycirc} &
  % -- Weak Signature
  {\smallemptycirc} &
  % -- Inadeq. Sig. Verif. 
  % -- SYSTEM
  {\smallemptycirc} &
  % -- Insec. Permissions
  {\smallemptycirc} &
  % -- Library Vulnerability
  {\smallemptycirc} &
  % -- OS Vulnerabilities
  {\smallemptycirc} &
  % -- Coding Errors
  {\smallemptycirc} &
  % -- Insec. Network
  %  -- SYSTEM
  {\smallfullcirc} &
  % -- Insec. User Interactions
  {\smallemptycirc} &
  % -- Comp. Provider
  % -- EXT
  {\smallemptycirc} &
  % -- Malicious Insider
  {\smallemptycirc} &
  % -- Insider Compromise
  % -- INSIDER
  {\smallemptycirc} &
  {\smallemptycirc} &
  {\smallemptycirc} &
  {\smallemptycirc} &
  {\smallemptycirc} &
  {\smallfullcirc} &
  {\smallemptycirc} &
  {\smallemptycirc} &
  {\smallemptycirc} &
  {\smallemptycirc} &
  {\smallfullcirc} &
  {\smallemptycirc} &
  {\smallemptycirc} &
  {\smallemptycirc} &
  % {\smallemptycirc} &
  {\smallemptycirc} &
  {\smallfullcirc} &
  % -- &
  % -- &
  % -- &
  {\smallemptycirc} &
  {\smallemptycirc} &
  {\smallemptycirc} &
  % {\smallfullcirc} &
  {\smallfullcirc} &
  {\smallfullcirc} &
  {\smallfullcirc} &
  {\smallfullcirc} &
  {\smallfullcirc} &
  \cellcolor{g0}{$0$} &
\cellcolor{r2}{$1$} &

  
  % {\smallemptycirc} &
  % {\smallemptycirc} &
  % {\smallemptycirc} &
  % {\smallfullcirc} &
  % {\smallemptycirc} &
  % {\smallemptycirc} &
  % {\smallfullcirc} &
  % {\smallfullcirc} &
  % {\smallemptycirc} &
  % {\smallemptycirc} &
  % {\smallemptycirc} &
  % {\smallemptycirc} &
  % {\smallemptycirc} &
  % {\smallemptycirc} &
  % {\smallemptycirc} &
  % {\smallemptycirc} &
  % {\smallemptycirc} &
  % {\smallemptycirc} &
  % {\smallemptycirc} &
  % {\smallemptycirc} &
  % {\smallemptycirc} &
  % {\smallemptycirc} &
  % {\smallemptycirc} &
  % {\smallemptycirc} &
  % {\smallemptycirc} &
  % {\smallemptycirc} &
   % Address White-listing 
   % Industry Defence
   % Destination address management tools such as address whitelisitng 
   \cite{ManageAddresses} 
   \\
 % &
 % &
 % Honeypot Scams *** \cite{Torres2019TheContracts} & 
 %   {\smallemptycirc} &
 %  % -- rng
 %  {\smallemptycirc} &
 %  % -- inadequ auth
 %  {\smallemptycirc} &
 %  % -- inadequ encry
 %  {\smallemptycirc} &
 %  % -- Appl. Logic Flaw
 %  {\smallemptycirc} &
 %  % -- Low-strength pwds
 %  {\smallemptycirc} &
 %  % -- Data Leakage
 %  {\smallemptycirc} &
 %  % -- Data Remanence
 %  {\smallemptycirc} &
 %  % -- Data Remanence
 %  {\smallemptycirc} &
 %  % -- Insec. Boot Environ.
 %  {\smallemptycirc} &
 %  % -- Micro-electr. Exposure
 %  {\smallemptycirc} &
 %  % -- Weak Signature
 %  {\smallemptycirc} &
 %  % -- Inadeq. Sig. Verif. 
 %  % -- SYSTEM
 %  {\smallemptycirc} &
 %  % -- Insec. Permissions
 %  {\smallemptycirc} &
 %  % -- Library Vulnerability
 %  {\smallemptycirc} &
 %  % -- OS Vulnerabilities
 %  {\smallemptycirc} &
 %  % -- Coding Errors
 %  {\smallemptycirc} &
 %  % -- Insec. Network
 %  %  -- SYSTEM
 %  {\smallfullcirc} &
 %  % -- Insec. User Interactions
 %  {\smallemptycirc} &
 %  % -- Comp. Provider
 %  % -- EXT
 %  {\smallemptycirc} &
 %  % -- Malicious Insider
 %  {\smallemptycirc} &
 %  % -- Insider Compromise
 %  % -- INSIDER
 %  {\smallemptycirc} &
 %  {\smallemptycirc} &
 %  {\smallemptycirc} &
 %  {\smallemptycirc} &
 %  {\smallemptycirc} &
 %  {\smallemptycirc} &
 %  {\smallemptycirc} &
 %  {\smallemptycirc} &
 %  {\smallemptycirc} &
 %  {\smallemptycirc} &
 %  {\smallemptycirc} &
 %  {\smallemptycirc} &
 %  {\smallemptycirc} &
 %  {\smallemptycirc} &
 %  % {\smallemptycirc} &
 %  {\smallemptycirc} &
 %  {\smallemptycirc} &
 %  % -- &
 %  % -- &
 %  % -- &
 %  {\smallemptycirc} &
 %  {\smallemptycirc} &
 %  {\smallemptycirc} &
 %  % {\smallfullcirc} &
 %  {\smallfullcirc} &
 %  {\smallfullcirc} &
 %  {\smallfullcirc} &
 %  {\smallfullcirc} &
 %  {\smallfullcirc} &
  
 %  % {\smallfullcirc} &
 %  % {\smallemptycirc} &
 %  % {\smallemptycirc} &
 %  % {\smallemptycirc} &
 %  % {\smallemptycirc} &
 %  % {\smallemptycirc} &
 %  % {\smallemptycirc} &
 %  % {\smallemptycirc} &
 %  % {\smallemptycirc} &
 %  % {\smallemptycirc} &
 %  % {\smallemptycirc} &
 %  % {\smallemptycirc} &
 %  % {\smallemptycirc} &
 %  % {\smallemptycirc} &
 %  % {\smallemptycirc} &
 %  % {\smallemptycirc} &
 %  % {\smallemptycirc} &
 %  % {\smallemptycirc} &
 %  % {\smallemptycirc} &
 %  % {\smallemptycirc} &
 %  % {\smallemptycirc} &
 %  % {\smallemptycirc} &
 %  % {\smallemptycirc} &
 %  % {\smallemptycirc} &
 %  % {\smallemptycirc} &
 %  % {\smallemptycirc} &
 %  % Honeypot Detection Technique
 %  \cite{Torres2019TheContracts}
 %  \\
\midrule
\multirow{4}{*}{Authentication} &
 \multirow{2}{*}{\hyperref[sec:cred-crack]{Credential Cracking}} &
  Brute-force \cite{Kiktenko2019DetectingWallets, volety2019cracking, Byun2024AAttacks} & 
{\smallemptycirc} &
  % -- rng
  {\smallfullcirc} &
  % -- inadequ auth
  {\smallemptycirc} &
  % -- inadequ encry
  {\smallemptycirc} &
  % -- Appl. Logic Flaw
  {\smallfullcirc} &
  % -- Low-strength pwds
  {\smallemptycirc} &
  % -- Data Leakage
  {\smallemptycirc} &
  % -- Data Remanence
  {\smallemptycirc} &
  % -- Data Remanence
  {\smallemptycirc} &
  % -- Insec. Boot Environ.
  {\smallemptycirc} &
  % -- Micro-electr. Exposure
  {\smallemptycirc} &
  % -- Weak Signature
  {\smallemptycirc} &
  % -- Inadeq. Sig. Verif. 
  % -- SYSTEM
  {\smallemptycirc} &
  % -- Insec. Permissions
  {\smallemptycirc} &
  % -- Library Vulnerability
  {\smallemptycirc} &
  % -- OS Vulnerabilities
  {\smallemptycirc} &
  % -- Coding Errors
  {\smallemptycirc} &
  % -- Insec. Network
  %  -- SYSTEM
  {\smallemptycirc} &
  % -- Insec. User Interactions
  {\smallemptycirc} &
  % -- Comp. Provider
  % -- EXT
  {\smallemptycirc} &
  % -- Malicious Insider
  {\smallemptycirc} &
  % -- Insider Compromise
  % -- INSIDER
  {\smallemptycirc} &
  {\smallemptycirc} &
  {\smallfullcirc} &
  {\smallfullcirc} &
  {\smallemptycirc} &
  {\smallemptycirc} &
  {\smallemptycirc} &
  {\smallemptycirc} &
  {\smallfullcirc} &
  {\smallemptycirc} &
  {\smallemptycirc} &
  {\smallemptycirc} &
  {\smallemptycirc} &
  {\smallemptycirc} &
  % {\smallemptycirc} &
  {\smallemptycirc} &
  {\smallemptycirc} &
  % -- &
  % -- &
  % -- &
  {\smallemptycirc} &
  {\smallfullcirc} &
  {\smallemptycirc} &
  % {\smallfullcirc} &
  {\smallfullcirc} &
  {\smallfullcirc} &
  {\smallfullcirc} &
  {\smallemptycirc} &
  {\smallfullcirc} &
   \cellcolor{g6}{$3$} &
\cellcolor{r0}{$0$} &
  
  % {\smallemptycirc} &
  % {\smallemptycirc} &
  % {\smallemptycirc} &
  % {\smallfullcirc} &
  % {\smallfullcirc} &
  % {\smallemptycirc} &
  % {\smallfullcirc} &
  % {\smallfullcirc} &
  % {\smallemptycirc} &
  % {\smallemptycirc} &
  % {\smallemptycirc} &
  % {\smallemptycirc} &
  % {\smallemptycirc} &
  % {\smallemptycirc} &
  % {\smallemptycirc} &
  % {\smallemptycirc} &
  % {\smallemptycirc} &
  % {\smallemptycirc} &
  % {\smallemptycirc} &
  % {\smallemptycirc} &
  % {\smallemptycirc} &
  % {\smallemptycirc} &
  % {\smallemptycirc} &
  % {\smallemptycirc} &
  % {\smallemptycirc} &
  % {\smallemptycirc} &
  % Advanced Passwords 
  \cite{Kiktenko2019DetectingWallets, Byun2024AAttacks}
   \\
 &
 &
 Dictionary \cite{praitheeshan2019security, Uddin2021Horus:Wallets} & 
   {\smallemptycirc} &
  % -- rng
  {\smallfullcirc} &
  % -- inadequ auth
  {\smallemptycirc} &
  % -- inadequ encry
  {\smallemptycirc} &
  % -- Appl. Logic Flaw
  {\smallemptycirc} &
  % -- Low-strength pwds
  {\smallfullcirc} &
  % -- Data Leakage
  {\smallemptycirc} &
  % -- Data Remanence
  {\smallemptycirc} &
  % -- Data Remanence
  {\smallemptycirc} &
  % -- Insec. Boot Environ.
  {\smallemptycirc} &
  % -- Micro-electr. Exposure
  {\smallemptycirc} &
  % -- Weak Signature
  {\smallemptycirc} &
  % -- Inadeq. Sig. Verif. 
  % -- SYSTEM
  {\smallemptycirc} &
  % -- Insec. Permissions
  {\smallemptycirc} &
  % -- Library Vulnerability
  {\smallemptycirc} &
  % -- OS Vulnerabilities
  {\smallemptycirc} &
  % -- Coding Errors
  {\smallemptycirc} &
  % -- Insec. Network
  %  -- SYSTEM
  {\smallemptycirc} &
  % -- Insec. User Interactions
  {\smallemptycirc} &
  % -- Comp. Provider
  % -- EXT
  {\smallemptycirc} &
  % -- Malicious Insider
  {\smallemptycirc} &
  % -- Insider Compromise
  % -- INSIDER
  {\smallemptycirc} &
  {\smallemptycirc} &
  {\smallfullcirc} &
  {\smallemptycirc} &
  {\smallemptycirc} &
  {\smallemptycirc} &
  {\smallemptycirc} &
  {\smallemptycirc} &
  {\smallfullcirc} &
  {\smallemptycirc} &
  {\smallemptycirc} &
  {\smallemptycirc} &
  {\smallemptycirc} &
  {\smallemptycirc} &
  % {\smallemptycirc} &
  {\smallemptycirc} &
  {\smallemptycirc} &
  % -- &
  % -- &
  % -- &
  {\smallemptycirc} &
  {\smallfullcirc} &
  {\smallemptycirc} &
  % {\smallfullcirc} &
  {\smallfullcirc} &
  {\smallfullcirc} &
  {\smallfullcirc} &
  {\smallemptycirc} &
  {\smallfullcirc} &
   \cellcolor{g4}{$2$} &
\cellcolor{r0}{$0$} &
  
  % {\smallfullcirc} &
  % {\smallemptycirc} &
  % {\smallemptycirc} &
  % {\smallemptycirc} &
  % {\smallemptycirc} &
  % {\smallemptycirc} &
  % {\smallemptycirc} &
  % {\smallemptycirc} &
  % {\smallemptycirc} &
  % {\smallemptycirc} &
  % {\smallemptycirc} &
  % {\smallemptycirc} &
  % {\smallemptycirc} &
  % {\smallemptycirc} &
  % {\smallemptycirc} &
  % {\smallemptycirc} &
  % {\smallemptycirc} &
  % {\smallemptycirc} &
  % {\smallemptycirc} &
  % {\smallemptycirc} &
  % {\smallemptycirc} &
  % {\smallemptycirc} &
  % {\smallemptycirc} &
  % {\smallemptycirc} &
  % {\smallemptycirc} &
  % {\smallemptycirc} &
  % Custom Keyboard Functions
  \cite{aldawood2020advanced}
  \\
&
 \multirow{2}{*}{\hyperref[sec:iden-spoof]{Identity Spoofing}} &
  Fake Biometrics \cite{galbally2013image} & 
   {\smallemptycirc} &
  % -- rng
  {\smallfullcirc} &
  % -- inadequ auth
  {\smallemptycirc} &
  % -- inadequ encry
  {\smallemptycirc} &
  % -- Appl. Logic Flaw
  {\smallemptycirc} &
  % -- Low-strength pwds
  {\smallemptycirc} &
  % -- Data Leakage
  {\smallemptycirc} &
  % -- Data Remanence
  {\smallemptycirc} &
  % -- Data Remanence
  {\smallemptycirc} &
  % -- Insec. Boot Environ.
  {\smallemptycirc} &
  % -- Micro-electr. Exposure
  {\smallemptycirc} &
  % -- Weak Signature
  {\smallemptycirc} &
  % -- Inadeq. Sig. Verif. 
  % -- SYSTEM
  {\smallemptycirc} &
  % -- Insec. Permissions
  {\smallemptycirc} &
  % -- Library Vulnerability
  {\smallemptycirc} &
  % -- OS Vulnerabilities
  {\smallemptycirc} &
  % -- Coding Errors
  {\smallemptycirc} &
  % -- Insec. Network
  %  -- SYSTEM
  {\smallemptycirc} &
  % -- Insec. User Interactions
  {\smallemptycirc} &
  % -- Comp. Provider
  % -- EXT
  {\smallemptycirc} &
  % -- Malicious Insider
  {\smallemptycirc} &
  % -- Insider Compromise
  % -- INSIDER
  {\smallemptycirc} &
  {\smallemptycirc} &
  {\smallemptycirc} &
  {\smallhalfcirc} &
  {\smallemptycirc} &
  {\smallemptycirc} &
  {\smallemptycirc} &
  {\smallemptycirc} &
  {\smallfullcirc} &
  {\smallemptycirc} &
  {\smallemptycirc} &
  {\smallemptycirc} &
  {\smallemptycirc} &
  {\smallemptycirc} &
  % {\smallemptycirc} &
  {\smallemptycirc} &
  {\smallemptycirc} &
  % -- &
  % -- &
  % -- &
  {\smallemptycirc} &
  {\smallfullcirc} &
  {\smallemptycirc} &
  % {\smallfullcirc} &
  {\smallemptycirc} &
  {\smallemptycirc} &
  {\smallfullcirc} &
  {\smallfullcirc} &
  {\smallemptycirc} &
   \cellcolor{g2}{$1$} &
\cellcolor{r0}{$0$} &

  
  % {\smallemptycirc} &
  % {\smallemptycirc} &
  % {\smallemptycirc} &
  % {\smallfullcirc} &
  % {\smallemptycirc} &
  % {\smallfullcirc} &
  % {\smallfullcirc} &
  % {\smallfullcirc} &
  % {\smallemptycirc} &
  % {\smallemptycirc} &
  % {\smallemptycirc} &
  % {\smallemptycirc} &
  % {\smallemptycirc} &
  % {\smallemptycirc} &
  % {\smallemptycirc} &
  % {\smallemptycirc} &
  % {\smallemptycirc} &
  % {\smallemptycirc} &
  % {\smallemptycirc} &
  % {\smallemptycirc} &
  % {\smallemptycirc} &
  % {\smallemptycirc} &
  % {\smallemptycirc} &
  % {\smallemptycirc} &
  % {\smallemptycirc} &
  % {\smallemptycirc} &
  % Liveness Assessment Features
  \cite{galbally2013image} 
   \\
 &
 &
 SIM Swap \cite{Kim2022ACountermeasures} & 
   {\smallemptycirc} &
  % -- rng
  {\smallfullcirc} &
  % -- inadequ auth
  {\smallemptycirc} &
  % -- inadequ encry
  {\smallemptycirc} &
  % -- Appl. Logic Flaw
  {\smallemptycirc} &
  % -- Low-strength pwds
  {\smallemptycirc} &
  % -- Data Leakage
  {\smallemptycirc} &
  % -- Data Remanence
  {\smallemptycirc} &
  % -- Data Remanence
  {\smallemptycirc} &
  % -- Insec. Boot Environ.
  {\smallemptycirc} &
  % -- Micro-electr. Exposure
  {\smallemptycirc} &
  % -- Weak Signature
  {\smallemptycirc} &
  % -- Inadeq. Sig. Verif. 
  % -- SYSTEM
  {\smallemptycirc} &
  % -- Insec. Permissions
  {\smallemptycirc} &
  % -- Library Vulnerability
  {\smallemptycirc} &
  % -- OS Vulnerabilities
  {\smallemptycirc} &
  % -- Coding Errors
  {\smallemptycirc} &
  % -- Insec. Network
  %  -- SYSTEM
  {\smallemptycirc} &
  % -- Insec. User Interactions
  {\smallemptycirc} &
  % -- Comp. Provider
  % -- EXT
  {\smallemptycirc} &
  % -- Malicious Insider
  {\smallemptycirc} &
  % -- Insider Compromise
  % -- INSIDER
  {\smallemptycirc} &
  {\smallemptycirc} &
  {\smallemptycirc} &
  {\smallhalfcirc} &
  {\smallemptycirc} &
  {\smallemptycirc} &
  {\smallemptycirc} &
  {\smallemptycirc} &
  {\smallfullcirc} &
  {\smallemptycirc} &
  {\smallemptycirc} &
  {\smallemptycirc} &
  {\smallemptycirc} &
  {\smallemptycirc} &
  % {\smallemptycirc} &
  {\smallemptycirc} &
  {\smallemptycirc} &
  % -- &
  % -- &
  % -- &
  {\smallemptycirc} &
  {\smallfullcirc} &
  {\smallemptycirc} &
  % {\smallfullcirc} &
  {\smallemptycirc} &
  {\smallemptycirc} &
  {\smallfullcirc} &
  {\smallfullcirc} &
  {\smallemptycirc} &
   \cellcolor{g0}{$0$} &
\cellcolor{r2}{$1$} &
  
  % {\smallfullcirc} &
  % {\smallemptycirc} &
  % {\smallemptycirc} &
  % {\smallemptycirc} &
  % {\smallemptycirc} &
  % {\smallemptycirc} &
  % {\smallemptycirc} &
  % {\smallemptycirc} &
  % {\smallemptycirc} &
  % {\smallemptycirc} &
  % {\smallemptycirc} &
  % {\smallemptycirc} &
  % {\smallemptycirc} &
  % {\smallemptycirc} &
  % {\smallemptycirc} &
  % {\smallemptycirc} &
  % {\smallemptycirc} &
  % {\smallemptycirc} &
  % {\smallemptycirc} &
  % {\smallemptycirc} &
  % {\smallemptycirc} &
  % {\smallemptycirc} &
  % {\smallemptycirc} &
  % {\smallemptycirc} &
  % {\smallemptycirc} &
  % {\smallemptycirc} &
  % Custom Keyboard Functions
  \cite{Kim2022ACountermeasures}
  \\
\midrule
\multirow{5}{*}{Storage}
   &
\hyperref[sec:fau-inj]{Fault Injection} &
  Fault Injection Attacks \cite{Akter2023AChallenges, hajdu2020using} &{\smallemptycirc} &
  % -- rng
  {\smallemptycirc} &
  % -- inadequ auth
  {\smallemptycirc} &
  % -- inadequ encry
  {\smallemptycirc} &
  % -- Appl. Logic Flaw
  {\smallemptycirc} &
  % -- Low-strength pwds
  {\smallemptycirc} &
  % -- Data Leakage
  {\smallfullcirc} &
  % -- Data Remanence
  {\smallfullcirc} &
  % -- Data Remanence
  {\smallemptycirc} &
  % -- Insec. Boot Environ.
  {\smallemptycirc} &
  % -- Micro-electr. Exposure
  {\smallemptycirc} &
  % -- Weak Signature
  {\smallemptycirc} &
  % -- Inadeq. Sig. Verif. 
  % -- SYSTEM
  {\smallemptycirc} &
  % -- Insec. Permissions
  {\smallemptycirc} &
  % -- Library Vulnerability
  {\smallemptycirc} &
  % -- OS Vulnerabilities
  {\smallemptycirc} &
  % -- Coding Errors
  {\smallemptycirc} &
  % -- Insec. Network
  %  -- SYSTEM
  {\smallemptycirc} &
  % -- Insec. User Interactions
  {\smallemptycirc} &
  % -- Comp. Provider
  % -- EXT
  {\smallemptycirc} &
  % -- Malicious Insider
  {\smallemptycirc} &
  % -- Insider Compromise
  % -- INSIDER
  {\smallfullcirc} &
  {\smallemptycirc} &
  {\smallemptycirc} &
  {\smallemptycirc} &
  {\smallfullcirc} &
  {\smallemptycirc} &
  {\smallemptycirc} &
  {\smallemptycirc} &
  {\smallemptycirc} &
  {\smallfullcirc} &
  {\smallemptycirc} &
  {\smallemptycirc} &
  {\smallemptycirc} &
  {\smallemptycirc} &
  % {\smallemptycirc} &
  {\smallemptycirc} &
  {\smallemptycirc} &
  % -- &
  % -- &
  % -- &
  {\smallemptycirc} &
  {\smallfullcirc} &
  {\smallemptycirc} &
  % {\smallfullcirc} &
  {\smallfullcirc} &
  {\smallfullcirc} &
  {\smallfullcirc} &
  {\smallfullcirc} &
  {\smallfullcirc} &
   \cellcolor{g4}{$2$} &
\cellcolor{r0}{$0$} &

  
  % {\smallemptycirc} &
  % {\smallemptycirc} &
  % {\smallemptycirc} &
  % {\smallemptycirc} &
  % {\smallemptycirc} &
  % {\smallemptycirc} &
  % {\smallemptycirc} &
  % {\smallemptycirc} &
  % {\smallemptycirc} &
  % {\smallemptycirc} &
  % {\smallemptycirc} &
  % {\smallemptycirc} &
  % {\smallemptycirc} &
  % {\smallemptycirc} &
  % {\smallemptycirc} &
  % {\smallemptycirc} &
  % {\smallemptycirc} &
  % {\smallemptycirc} &
  % {\smallemptycirc} &
  % {\smallemptycirc} &
  % {\smallemptycirc} &
  % {\smallemptycirc} &
  % {\smallemptycirc} &
  % {\smallfullcirc} &
  % {\smallemptycirc} &
  % {\smallemptycirc} &
  % Algorithmic Fault Detection
  \cite{Shuvo2023AAttacks, breier2022practical} 
   \\
 &
 \multirow{2}{*}{\hyperref[sec:tam-per]{Physical Tampering}} &
  Evil Maid \cite{Shaikh2022SurveyExchanges} &
{\smallemptycirc} &
  % -- rng
  {\smallfullcirc} &
  % -- inadequ auth
  {\smallemptycirc} &
  % -- inadequ encry
  {\smallemptycirc} &
  % -- Appl. Logic Flaw
  {\smallemptycirc} &
  % -- Low-strength pwds
  {\smallemptycirc} &
  % -- Data Leakage
  {\smallemptycirc} &
  % -- Data Remanence
  {\smallemptycirc} &
  % -- Data Remanence
  {\smallfullcirc} &
  % -- Insec. Boot Environ.
  {\smallemptycirc} &
  % -- Micro-electr. Exposure
  {\smallemptycirc} &
  % -- Weak Signature
  {\smallemptycirc} &
  % -- Inadeq. Sig. Verif. 
  % -- SYSTEM
  {\smallemptycirc} &
  % -- Insec. Permissions
  {\smallemptycirc} &
  % -- Library Vulnerability
  {\smallemptycirc} &
  % -- OS Vulnerabilities
  {\smallemptycirc} &
  % -- Coding Errors
  {\smallemptycirc} &
  % -- Insec. Network
  %  -- SYSTEM
  {\smallfullcirc} &
  % -- Insec. User Interactions
  {\smallemptycirc} &
  % -- Comp. Provider
  % -- EXT
  {\smallemptycirc} &
  % -- Malicious Insider
  {\smallemptycirc} &
  % -- Insider Compromise
  % -- INSIDER
  {\smallemptycirc} &
  {\smallemptycirc} &
  {\smallemptycirc} &
  {\smallfullcirc} &
  {\smallemptycirc} &
  {\smallemptycirc} &
  {\smallemptycirc} &
  {\smallemptycirc} &
  {\smallfullcirc} &
  {\smallemptycirc} &
  {\smallemptycirc} &
  {\smallemptycirc} &
  {\smallemptycirc} &
  {\smallemptycirc} &
  % {\smallemptycirc} &
  {\smallemptycirc} &
  {\smallemptycirc} &
  % -- &
  % -- &
  % -- &
  {\smallfullcirc} &
  {\smallfullcirc} &
  {\smallemptycirc} &
  % {\smallfullcirc} &
  {\smallemptycirc} &
  {\smallemptycirc} &
  {\smallemptycirc} &
  {\smallemptycirc} &
  {\smallfullcirc} &
   \cellcolor{g2}{$1$} &
\cellcolor{r0}{$0$} &

  
  % {\smallemptycirc} &
  % {\smallemptycirc} &
  % {\smallemptycirc} &
  % {\smallfullcirc} &
  % {\smallemptycirc} &
  % {\smallemptycirc} &
  % {\smallemptycirc} &
  % {\smallemptycirc} &
  % {\smallemptycirc} &
  % {\smallemptycirc} &
  % {\smallemptycirc} &
  % {\smallemptycirc} &
  % {\smallemptycirc} &
  % {\smallemptycirc} &
  % {\smallemptycirc} &
  % {\smallemptycirc} &
  % {\smallemptycirc} &
  % {\smallemptycirc} &
  % {\smallemptycirc} &
  % {\smallemptycirc} &
  % {\smallemptycirc} &
  % {\smallemptycirc} &
  % {\smallemptycirc} &
  % {\smallemptycirc} &
  % {\smallemptycirc} &
  % {\smallemptycirc} &
  % Multi-factor Authentication
  \cite{Aratani2015AuthenticationChannel} 
   \\
 &
 &
  Microscopy \cite{courbon2016reverse} &
   % \multicolumn{2}{c}{\hyperref[sec:microscopy]{Microscopy} \cite{courbon2016reverse}}  & 
   {\smallemptycirc} &
  % -- rng
  {\smallemptycirc} &
  % -- inadequ auth
  {\smallemptycirc} &
  % -- inadequ encry
  {\smallemptycirc} &
  % -- Appl. Logic Flaw
  {\smallemptycirc} &
  % -- Low-strength pwds
  {\smallemptycirc} &
  % -- Data Leakage
  {\smallemptycirc} &
  % -- Data Remanence
  {\smallemptycirc} &
  % -- Data Remanence
  {\smallemptycirc} &
  % -- Insec. Boot Environ.
  {\smallfullcirc} &
  % -- Micro-electr. Exposure
  {\smallemptycirc} &
  % -- Weak Signature
  {\smallemptycirc} &
  % -- Inadeq. Sig. Verif. 
  % -- SYSTEM
  {\smallemptycirc} &
  % -- Insec. Permissions
  {\smallemptycirc} &
  % -- Library Vulnerability
  {\smallemptycirc} &
  % -- OS Vulnerabilities
  {\smallemptycirc} &
  % -- Coding Errors
  {\smallemptycirc} &
  % -- Insec. Network
  %  -- SYSTEM
  {\smallemptycirc} &
  % -- Insec. User Interactions
  {\smallemptycirc} &
  % -- Comp. Provider
  % -- EXT
  {\smallemptycirc} &
  % -- Malicious Insider
  {\smallemptycirc} &
  % -- Insider Compromise
  % -- INSIDER
  {\smallemptycirc} &
  {\smallemptycirc} &
  {\smallemptycirc} &
  {\smallemptycirc} &
  {\smallfullcirc} &
  {\smallemptycirc} &
  {\smallemptycirc} &
  {\smallemptycirc} &
  {\smallemptycirc} &
  {\smallfullcirc} &
  {\smallemptycirc} &
  {\smallemptycirc} &
  {\smallemptycirc} &
  {\smallemptycirc} &
  % {\smallemptycirc} &
  {\smallemptycirc} &
  {\smallemptycirc} &
  % -- &
  % -- &
  % -- &
  {\smallemptycirc} &
  {\smallfullcirc} &
  {\smallemptycirc} &
  % {\smallemptycirc} &
  {\smallemptycirc} &
  {\smallemptycirc} &
  {\smallemptycirc} &
  {\smallemptycirc} &
  {\smallfullcirc} &
   \cellcolor{g2}{$1$} &
\cellcolor{r2}{$1$} &

% ** need to cite the source from industry but it was ledger or trezor

  
  % {\smallemptycirc} &
  % {\smallemptycirc} &
  % {\smallemptycirc} &
  % {\smallemptycirc} &
  % {\smallemptycirc} &
  % {\smallemptycirc} &
  % {\smallemptycirc} &
  % {\smallemptycirc} &
  % {\smallemptycirc} &
  % {\smallemptycirc} &
  % {\smallemptycirc} &
  % {\smallemptycirc} &
  % {\smallemptycirc} &
  % {\smallemptycirc} &
  % {\smallemptycirc} &
  % {\smallemptycirc} &
  % {\smallemptycirc} &
  % {\smallemptycirc} &
  % {\smallemptycirc} &
  % {\smallfullcirc} &
  % {\smallemptycirc} &
  % {\smallemptycirc} &
  % {\smallemptycirc} &
  % {\smallemptycirc} &
  % {\smallemptycirc} &
  % {\smallemptycirc} &
  % \acf{puf} 
  \cite{hu2020overview, Urien2021InnovativeWallets} 
   \\
   &
\multirow{2}{*}{\hyperref[sec:non-inv-man]{Non-invasive Manip.}} &
  Cold Boot Attack \cite{Shaikh2022SurveyExchanges} &
{\smallemptycirc} &
  % -- rng
  {\smallemptycirc} &
  % -- inadequ auth
  {\smallemptycirc} &
  % -- inadequ encry
  {\smallfullcirc} &
  % -- Appl. Logic Flaw
  {\smallemptycirc} &
  % -- Low-strength pwds
  {\smallemptycirc} &
  % -- Data Leakage
  {\smallemptycirc} &
  % -- Data Remanence
  {\smallemptycirc} &
  % -- Data Remanence
  {\smallfullcirc} &
  % -- Insec. Boot Environ.
  {\smallemptycirc} &
  % -- Micro-electr. Exposure
  {\smallemptycirc} &
  % -- Weak Signature
  {\smallemptycirc} &
  % -- Inadeq. Sig. Verif. 
  % -- SYSTEM
  {\smallfullcirc} &
  % -- Insec. Permissions
  {\smallemptycirc} &
  % -- Library Vulnerability
  {\smallemptycirc} &
  % -- OS Vulnerabilities
  {\smallemptycirc} &
  % -- Coding Errors
  {\smallemptycirc} &
  % -- Insec. Network
  %  -- SYSTEM
  {\smallemptycirc} &
  % -- Insec. User Interactions
  {\smallemptycirc} &
  % -- Comp. Provider
  % -- EXT
  {\smallemptycirc} &
  % -- Malicious Insider
  {\smallemptycirc} &
  % -- Insider Compromise
  % -- INSIDER
  {\smallemptycirc} &
  {\smallemptycirc} &
  {\smallemptycirc} &
  {\smallemptycirc} &
  {\smallfullcirc} &
  {\smallemptycirc} &
  {\smallemptycirc} &
  {\smallemptycirc} &
  {\smallemptycirc} &
  {\smallfullcirc} &
  {\smallemptycirc} &
  {\smallemptycirc} &
  {\smallemptycirc} &
  {\smallemptycirc} &
  % {\smallemptycirc} &
  {\smallemptycirc} &
  {\smallemptycirc} &
  % -- &
  % -- &
  % -- &
  {\smallemptycirc} &
  {\smallfullcirc} &
  {\smallemptycirc} &
  % {\smallfullcirc} &
  {\smallfullcirc} &
  {\smallfullcirc} &
  {\smallfullcirc} &
  {\smallemptycirc} &
  {\smallfullcirc} & 
   \cellcolor{g2}{$1$} &
\cellcolor{r0}{$0$} &

  
  % {\smallemptycirc} &
  % {\smallemptycirc} &
  % {\smallemptycirc} &
  % {\smallemptycirc} &
  % {\smallemptycirc} &
  % {\smallemptycirc} &
  % {\smallemptycirc} &
  % {\smallemptycirc} &
  % {\smallemptycirc} &
  % {\smallemptycirc} &
  % {\smallfullcirc} &
  % {\smallemptycirc} &
  % {\smallemptycirc} &
  % {\smallemptycirc} &
  % {\smallemptycirc} &
  % {\smallemptycirc} &
  % {\smallemptycirc} &
  % {\smallemptycirc} &
  % {\smallemptycirc} &
  % {\smallemptycirc} &
  % {\smallemptycirc} &
  % {\smallfullcirc} &
  % {\smallemptycirc} &
  % {\smallemptycirc} &
  % {\smallemptycirc} &
  % {\smallemptycirc} &
  % Supplementary Storage
  \cite{altuwaijri2020android} 
   \\
 &
 &
 \acs{puf} Attacks \cite{wang2024efficient} &
   {\smallfullcirc} &
  % -- rng
  {\smallfullcirc} &
  % -- inadequ auth
  {\smallemptycirc} &
  % -- inadequ encry
  {\smallemptycirc} &
  % -- Appl. Logic Flaw
  {\smallemptycirc} &
  % -- Low-strength pwds
  {\smallemptycirc} &
  % -- Data Leakage
  {\smallemptycirc} &
  % -- Data Remanence
  {\smallemptycirc} &
  % -- Data Remanence
  {\smallemptycirc} &
  % -- Insec. Boot Environ.
  {\smallfullcirc} &
  % -- Micro-electr. Exposure
  {\smallemptycirc} &
  % -- Weak Signature
  {\smallemptycirc} &
  % -- Inadeq. Sig. Verif. 
  % -- SYSTEM
  {\smallemptycirc} &
  % -- Insec. Permissions
  {\smallemptycirc} &
  % -- Library Vulnerability
  {\smallemptycirc} &
  % -- OS Vulnerabilities
  {\smallemptycirc} &
  % -- Coding Errors
  {\smallemptycirc} &
  % -- Insec. Network
  %  -- SYSTEM
  {\smallemptycirc} &
  % -- Insec. User Interactions
  {\smallemptycirc} &
  % -- Comp. Provider
  % -- EXT
  {\smallemptycirc} &
  % -- Malicious Insider
  {\smallemptycirc} &
  % -- Insider Compromise
  % -- INSIDER
  {\smallemptycirc} &
  {\smallemptycirc} &
  {\smallemptycirc} &
  {\smallemptycirc} &
  {\smallfullcirc} &
  {\smallemptycirc} &
  {\smallemptycirc} &
  {\smallemptycirc} &
  {\smallemptycirc} &
  {\smallfullcirc} &
  {\smallemptycirc} &
  {\smallemptycirc} &
  {\smallemptycirc} &
  {\smallemptycirc} &
  % {\smallemptycirc} &
  {\smallemptycirc} &
  {\smallemptycirc} &
  % -- &
  % -- &
  % -- &
  {\smallemptycirc} &
  {\smallfullcirc} &
  {\smallemptycirc} &
  % {\smallemptycirc} &
  {\smallemptycirc} &
  {\smallemptycirc} &
  {\smallemptycirc} &
  {\smallemptycirc} &
  {\smallfullcirc} &
   \cellcolor{g2}{$1$} &
\cellcolor{r0}{$0$} &

  
  % {\smallemptycirc} &
  % {\smallemptycirc} &
  % {\smallemptycirc} &
  % {\smallemptycirc} &
  % {\smallemptycirc} &
  % {\smallemptycirc} &
  % {\smallemptycirc} &
  % {\smallemptycirc} &
  % {\smallemptycirc} &
  % {\smallemptycirc} &
  % {\smallemptycirc} &
  % {\smallemptycirc} &
  % {\smallemptycirc} &
  % {\smallemptycirc} &
  % {\smallemptycirc} &
  % {\smallemptycirc} &
  % {\smallemptycirc} &
  % {\smallemptycirc} &
  % {\smallemptycirc} &
  % {\smallemptycirc} &
  % {\smallemptycirc} &
  % {\smallemptycirc} &
  % {\smallemptycirc} &
  % {\smallemptycirc} &
  % {\smallemptycirc} &
  % {\smallfullcirc} &
  % \acf{puf} 
  \cite{Park2023, Park2024CloningFunction}
   \\
\midrule
\multirow{5}{*}{Cryptanalysis} 
&
\multirow{3}{*}{\hyperref[sec:side-channel]{Side-channel Analysis}}
&
Timing-based \cite{kocher1996timing}
&
   {\smallemptycirc} &
  % -- rng
  {\smallemptycirc} &
  % -- inadequ auth
  {\smallemptycirc} &
  % -- inadequ encry
  {\smallemptycirc} &
  % -- Appl. Logic Flaw
  {\smallemptycirc} &
  % -- Low-strength pwds
  {\smallfullcirc} &
  % -- Data Leakage
  {\smallemptycirc} &
  % -- Data Remanence
  {\smallemptycirc} &
  % -- Data Remanence
  {\smallemptycirc} &
  % -- Insec. Boot Environ.
  {\smallemptycirc} &
  % -- Micro-electr. Exposure
  {\smallemptycirc} &
  % -- Weak Signature
  {\smallemptycirc} &
  % -- Inadeq. Sig. Verif. 
  % -- SYSTEM
  {\smallemptycirc} &
  % -- Insec. Permissions
  {\smallemptycirc} &
  % -- Library Vulnerability
  {\smallemptycirc} &
  % -- OS Vulnerabilities
  {\smallemptycirc} &
  % -- Coding Errors
  {\smallemptycirc} &
  % -- Insec. Network
  %  -- SYSTEM
  {\smallemptycirc} &
  % -- Insec. User Interactions
  {\smallemptycirc} &
  % -- Comp. Provider
  % -- EXT
  {\smallemptycirc} &
  % -- Malicious Insider
  {\smallemptycirc} &
  % -- Insider Compromise
  % -- INSIDER
  {\smallfullcirc} &
  {\smallemptycirc} &
  {\smallemptycirc} &
  {\smallemptycirc} &
  {\smallemptycirc} &
  {\smallemptycirc} &
  {\smallemptycirc} &
  {\smallfullcirc} &
  {\smallemptycirc} &
  {\smallemptycirc} &
  {\smallemptycirc} &
  {\smallfullcirc} &
  {\smallfullcirc} &
  {\smallemptycirc} &
  % {\smallemptycirc} &
  {\smallemptycirc} &
  {\smallemptycirc} &
  % -- &
  % -- &
  % -- &
  {\smallemptycirc} &
  {\smallfullcirc} &
  {\smallemptycirc} &
  % {\smallfullcirc} &
  {\smallfullcirc} &
  {\smallfullcirc} &
  {\smallfullcirc} &
  {\smallemptycirc} &
  {\smallfullcirc} &
   \cellcolor{g2}{$1$} &
\cellcolor{r0}{$0$} &

  
  % {\smallemptycirc} &
  % {\smallemptycirc} &
  % {\smallemptycirc} &
  % {\smallemptycirc} &
  % {\smallemptycirc} &
  % {\smallemptycirc} &
  % {\smallemptycirc} &
  % {\smallemptycirc} &
  % {\smallemptycirc} &
  % {\smallemptycirc} &
  % {\smallemptycirc} &
  % {\smallemptycirc} &
  % {\smallemptycirc} &
  % {\smallemptycirc} &
  % {\smallemptycirc} &
  % {\smallemptycirc} &
  % {\smallemptycirc} &
  % {\smallemptycirc} &
  % {\smallemptycirc} &
  % {\smallemptycirc} &
  % {\smallemptycirc} &
  % {\smallemptycirc} &
  % {\smallfullcirc} &
  % {\smallemptycirc} &
  % {\smallemptycirc} &
  % {\smallemptycirc} &
  % Memory and Cache Data Split 
  \cite{Akter2023AChallenges, Gupta2019ImpactSecurity} 
   \\
&
&
Power on Crypt. Algo. \cite{Park2023}
&
   {\smallemptycirc} &
  % -- rng
  {\smallemptycirc} &
  % -- inadequ auth
  {\smallemptycirc} &
  % -- inadequ encry
  {\smallemptycirc} &
  % -- Appl. Logic Flaw
  {\smallemptycirc} &
  % -- Low-strength pwds
  {\smallfullcirc} &
  % -- Data Leakage
  {\smallemptycirc} &
  % -- Data Remanence
  {\smallemptycirc} &
  % -- Data Remanence
  {\smallemptycirc} &
  % -- Insec. Boot Environ.
  {\smallemptycirc} &
  % -- Micro-electr. Exposure
  {\smallemptycirc} &
  % -- Weak Signature
  {\smallemptycirc} &
  % -- Inadeq. Sig. Verif. 
  % -- SYSTEM
  {\smallemptycirc} &
  % -- Insec. Permissions
  {\smallemptycirc} &
  % -- Library Vulnerability
  {\smallemptycirc} &
  % -- OS Vulnerabilities
  {\smallemptycirc} &
  % -- Coding Errors
  {\smallemptycirc} &
  % -- Insec. Network
  %  -- SYSTEM
  {\smallemptycirc} &
  % -- Insec. User Interactions
  {\smallemptycirc} &
  % -- Comp. Provider
  % -- EXT
  {\smallemptycirc} &
  % -- Malicious Insider
  {\smallemptycirc} &
  % -- Insider Compromise
  % -- INSIDER
  {\smallfullcirc} &
  {\smallemptycirc} &
  {\smallemptycirc} &
  {\smallemptycirc} &
  {\smallemptycirc} &
  {\smallemptycirc} &
  {\smallemptycirc} &
  {\smallfullcirc} &
  {\smallemptycirc} &
  {\smallemptycirc} &
  {\smallemptycirc} &
  {\smallfullcirc} &
  {\smallfullcirc} &
  {\smallemptycirc} &
  % {\smallemptycirc} &
  {\smallemptycirc} &
  {\smallemptycirc} &
  % -- &
  % -- &
  % -- &
  {\smallemptycirc} &
  {\smallfullcirc} &
  {\smallemptycirc} &
  % {\smallfullcirc} &
  {\smallfullcirc} &
  {\smallfullcirc} &
  {\smallfullcirc} &
  {\smallemptycirc} &
  {\smallfullcirc} &
  \cellcolor{g2}{$1$} &
\cellcolor{r0}{$0$} &

  
  % {\smallemptycirc} &
  % {\smallemptycirc} &
  % {\smallemptycirc} &
  % {\smallemptycirc} &
  % {\smallemptycirc} &
  % {\smallemptycirc} &
  % {\smallemptycirc} &
  % {\smallemptycirc} &
  % {\smallemptycirc} &
  % {\smallemptycirc} &
  % {\smallemptycirc} &
  % {\smallemptycirc} &
  % {\smallemptycirc} &
  % {\smallemptycirc} &
  % {\smallemptycirc} &
  % {\smallemptycirc} &
  % {\smallemptycirc} &
  % {\smallemptycirc} &
  % {\smallemptycirc} &
  % {\smallemptycirc} &
  % {\smallemptycirc} &
  % {\smallemptycirc} &
  % {\smallfullcirc} &
  % {\smallemptycirc} &
  % {\smallemptycirc} &
  % {\smallfullcirc} &
  % Memory and Cache Data Split 
  \cite{Akter2023AChallenges, Gupta2019ImpactSecurity} 
   \\
&
   &
  Power on Hash \cite{Park2024CloningFunction} 
 &
   {\smallemptycirc} &
  % -- rng
  {\smallemptycirc} &
  % -- inadequ auth
  {\smallemptycirc} &
  % -- inadequ encry
  {\smallemptycirc} &
  % -- Appl. Logic Flaw
  {\smallemptycirc} &
  % -- Low-strength pwds
  {\smallfullcirc} &
  % -- Data Leakage
  {\smallemptycirc} &
  % -- Data Remanence
  {\smallemptycirc} &
  % -- Data Remanence
  {\smallemptycirc} &
  % -- Insec. Boot Environ.
  {\smallemptycirc} &
  % -- Micro-electr. Exposure
  {\smallemptycirc} &
  % -- Weak Signature
  {\smallemptycirc} &
  % -- Inadeq. Sig. Verif. 
  % -- SYSTEM
  {\smallemptycirc} &
  % -- Insec. Permissions
  {\smallemptycirc} &
  % -- Library Vulnerability
  {\smallemptycirc} &
  % -- OS Vulnerabilities
  {\smallemptycirc} &
  % -- Coding Errors
  {\smallemptycirc} &
  % -- Insec. Network
  %  -- SYSTEM
  {\smallemptycirc} &
  % -- Insec. User Interactions
  {\smallemptycirc} &
  % -- Comp. Provider
  % -- EXT
  {\smallemptycirc} &
  % -- Malicious Insider
  {\smallemptycirc} &
  % -- Insider Compromise
  % -- INSIDER
  {\smallemptycirc} &
  {\smallemptycirc} &
  {\smallfullcirc} &
  {\smallemptycirc} &
  {\smallemptycirc} &
  {\smallemptycirc} &
  {\smallemptycirc} &
  {\smallemptycirc} &
  {\smallemptycirc} &
  {\smallemptycirc} &
  {\smallemptycirc} &
  {\smallfullcirc} &
  {\smallemptycirc} &
  {\smallemptycirc} &
  % {\smallemptycirc} &
  {\smallemptycirc} &
  {\smallemptycirc} &
  % -- &
  % -- &
  % -- &
  {\smallemptycirc} &
  {\smallfullcirc} &
  {\smallemptycirc} &
  % {\smallfullcirc} &
  {\smallfullcirc} &
  {\smallfullcirc} &
  {\smallfullcirc} &
  {\smallemptycirc} &
  {\smallfullcirc} &
  \cellcolor{g2}{$1$} &
\cellcolor{r0}{$0$} &

  
  % {\smallemptycirc} &
  % {\smallemptycirc} &
  % {\smallemptycirc} &
  % {\smallemptycirc} &
  % {\smallemptycirc} &
  % {\smallemptycirc} &
  % {\smallemptycirc} &
  % {\smallemptycirc} &
  % {\smallemptycirc} &
  % {\smallemptycirc} &
  % {\smallemptycirc} &
  % {\smallemptycirc} &
  % {\smallemptycirc} &
  % {\smallemptycirc} &
  % {\smallemptycirc} &
  % {\smallemptycirc} &
  % {\smallemptycirc} &
  % {\smallemptycirc} &
  % {\smallemptycirc} &
  % {\smallemptycirc} &
  % {\smallemptycirc} &
  % {\smallemptycirc} &
  % {\smallfullcirc} &
  % {\smallemptycirc} &
  % {\smallemptycirc} &
  % {\smallfullcirc} &
  % Memory and Cache Data Split 
  \cite{Akter2023AChallenges, Gupta2019ImpactSecurity} 
   \\
 &
\multirow{2}{*}{\hyperref[sec:impl-exp]{Direct Exploitation}}
&
 Weak Signature  \cite{Rokhjavan2023SecuringWallets}
&
   {\smallfullcirc} &
  % -- rng
  {\smallemptycirc} &
  % -- inadequ auth
  {\smallemptycirc} &
  % -- inadequ encry
  {\smallemptycirc} &
  % -- Appl. Logic Flaw
  {\smallemptycirc} &
  % -- Low-strength pwds
  {\smallemptycirc} &
  % -- Data Leakage
  {\smallemptycirc} &
  % -- Data Remanence
  {\smallemptycirc} &
  % -- Data Remanence
  {\smallemptycirc} &
  % -- Insec. Boot Environ.
  {\smallemptycirc} &
  % -- Micro-electr. Exposure
  {\smallfullcirc} &
  % -- Weak Signature
  {\smallfullcirc} &
  % -- Inadeq. Sig. Verif. 
  % -- SYSTEM
  {\smallemptycirc} &
  % -- Insec. Permissions
  {\smallfullcirc} &
  % -- Library Vulnerability
  {\smallemptycirc} &
  % -- OS Vulnerabilities
  {\smallfullcirc} &
  % -- Coding Errors
  {\smallemptycirc} &
  % -- Insec. Network
  %  -- SYSTEM
  {\smallemptycirc} &
  % -- Insec. User Interactions
  {\smallemptycirc} &
  % -- Comp. Provider
  % -- EXT
  {\smallemptycirc} &
  % -- Malicious Insider
  {\smallemptycirc} &
  % -- Insider Compromise
  % -- INSIDER
  {\smallemptycirc} &
  {\smallfullcirc} &
  {\smallemptycirc} &
  {\smallemptycirc} &
  {\smallemptycirc} &
  {\smallemptycirc} &
  {\smallemptycirc} &
  {\smallemptycirc} &
  {\smallemptycirc} &
  {\smallemptycirc} &
  {\smallemptycirc} &
  {\smallfullcirc} &
  {\smallemptycirc} &
  {\smallemptycirc} &
  % {\smallemptycirc} &
  {\smallemptycirc} &
  {\smallemptycirc} &
  % -- &
  % -- &
  % -- &
  {\smallemptycirc} &
  {\smallfullcirc} &
  {\smallemptycirc} &
  % {\smallfullcirc} &
  {\smallfullcirc} &
  {\smallfullcirc} &
  {\smallfullcirc} &
  {\smallemptycirc} &
  {\smallfullcirc} & 
  \cellcolor{g2}{$1$} &
\cellcolor{r0}{$0$} &

  
  % {\smallemptycirc} &
  % {\smallemptycirc} &
  % {\smallemptycirc} &
  % {\smallemptycirc} &
  % {\smallemptycirc} &
  % {\smallemptycirc} &
  % {\smallemptycirc} &
  % {\smallemptycirc} &
  % {\smallemptycirc} &
  % {\smallemptycirc} &
  % {\smallemptycirc} &
  % {\smallemptycirc} &
  % {\smallemptycirc} &
  % {\smallemptycirc} &
  % {\smallemptycirc} &
  % {\smallemptycirc} &
  % {\smallemptycirc} &
  % {\smallemptycirc} &
  % {\smallemptycirc} &
  % {\smallemptycirc} &
  % {\smallemptycirc} &
  % {\smallemptycirc} &
  % {\smallemptycirc} &
  % {\smallemptycirc} &
  % {\smallfullcirc} &
  % {\smallemptycirc}
  % Secure Cryptograpphic Schemes 
  \cite{brengel2018identifying}
   \\
 &
 &
 Nonce Reuse \cite{brengel2018identifying}
&
  % -- rng
  {\smallemptycirc} &
  % -- inadequ auth
  {\smallemptycirc} &
  % -- inadequ encry
  {\smallemptycirc} &
  % -- Appl. Logic Flaw
  {\smallemptycirc} &
  % -- Low-strength pwds
  {\smallemptycirc} &
  % -- Data Leakage
  {\smallemptycirc} &
  % -- Data Remanence
  {\smallemptycirc} &
  % -- Data Remanence
  {\smallemptycirc} &
  % -- Insec. Boot Environ.
  {\smallemptycirc} &
  % -- Micro-electr. Exposure
  {\smallemptycirc} &
  % -- Weak Signature
  {\smallemptycirc} &
  % -- Inadeq. Sig. Verif. 
  % -- SYSTEM
  {\smallemptycirc} &
  % -- Insec. Permissions
  {\smallemptycirc} &
  % -- Library Vulnerability
  {\smallemptycirc} &
  % -- OS Vulnerabilities
  {\smallemptycirc} &
  % -- Coding Errors
  {\smallemptycirc} &
  % -- Insec. Network
  %  -- SYSTEM
  {\smallemptycirc} &
  % -- Insec. User Interactions
  {\smallemptycirc} &
  % -- Comp. Provider
  % -- EXT
  {\smallemptycirc} &
  % -- Malicious Insider
  {\smallemptycirc} &
  % -- Insider Compromise
  % -- INSIDER
  {\smallemptycirc} &
  {\smallemptycirc} &
  {\smallemptycirc} &
  {\smallemptycirc} &
  {\smallemptycirc} &
  {\smallemptycirc} &
  {\smallemptycirc} &
  {\smallfullcirc} &
  % --
  {\smallemptycirc} &
  {\smallemptycirc} &
  {\smallemptycirc} &
  {\smallemptycirc} &
  {\smallfullcirc} &
  {\smallemptycirc} &
  % --
  {\smallemptycirc} &
  {\smallemptycirc} &
  {\smallemptycirc} &
  % --
  {\smallemptycirc} &
  {\smallfullcirc} &
  {\smallemptycirc} &
   % --
  % {\smallfullcirc} &
  {\smallfullcirc} &
  {\smallfullcirc} &
  {\smallfullcirc} &
  {\smallemptycirc} &
  {\smallfullcirc} &
  \cellcolor{g2}{$1$} &
\cellcolor{r0}{$0$} &
  %  % --
  % {\smallemptycirc} &
  % {\smallemptycirc} &
  % {\smallemptycirc} &
  % {\smallemptycirc} &
  % {\smallemptycirc} &
  % {\smallemptycirc} &
  % {\smallemptycirc} &
  % {\smallemptycirc} &
  %  % --
  % {\smallemptycirc} &
  % {\smallemptycirc} &
  %  % --
  % {\smallemptycirc} &
  % {\smallemptycirc} &
  % {\smallemptycirc} &
  % {\smallemptycirc} &
  %  % --
  % {\smallemptycirc} &
  % {\smallemptycirc} &
  % {\smallemptycirc} &
  % {\smallemptycirc} &
  % {\smallemptycirc} &
  %  % --
  % {\smallemptycirc} &
  % {\smallfullcirc} &
  % {\smallemptycirc} &
  % {\smallemptycirc} &
  % {\smallemptycirc} &
  % {\smallemptycirc} &
  % {\smallemptycirc}
  % Deterministic Nonce Selection 
  \cite{brengel2018identifying} 
   \\
\midrule
% \multicolumn{3}{c}{}  &
%   &
% \multicolumn{44}{r}{Undetailed  }  &
%   $0$($0\%$) &
% \cellcolor{r6}{$51$($62\%$)} &
%    \\
\multicolumn{2}{c}{Summary}  
  &
\multicolumn{1}{c}{28 Attack Vectors}  
  &
\multicolumn{36}{c}{}  
  &
 \multicolumn{9}{r}{Attack Vectors Occurrence  }  &
  \cellcolor{g6}{$24$($86\%$)} &
\cellcolor{g2}{$9$($32\%$)} &
% (\hyperref[tab:attack-incidents]{\cellcolor{r0}{$83$($100\%$)}}) 
% $29$ Methods
   \\
\bottomrule
\end{tabular}}
\end{table}
\clearpage
\end{landscape}

To mitigate these attacks, wallets can adopt network security protocols that validate and authenticate IP addresses \cite{rengarajan2016secure} and incorporate additional security layers within the wallet's network to prevent potential \teal{$txn$} modification attempts by adversaries \cite{Cai2014ResearchNetwork}. To limit or prevent \acf{ddos} attacks, wallets must distinguish malicious and authentic network traffic using classifiers such as the decision tree algorithm \cite{khan2019adaptive} and reinforcement learning approaches to analyse patterns in network data \cite{liu2018deep}. Another mitigation approach involves analysing the network for unusual patterns, such as repeated request attempts from the same \acs{ip} address \cite{sathwara2017distributed}.

\subsection{Application}
\label{sec:app-def}

To mitigate the risk of message alteration by clipboard hijackers, wallets can employ features such as NFC and two-dimensional codes to prevent recipient address modification during transaction creation \cite{li2020android}. From a user perspective, human-readable addresses such as \acs{ens} \cite{ENS2024EthereumService} aid in detecting address tampering, though they have certain security vulnerabilities \cite{Xia2022ChallengesENS}. Wallets can also prevent system behaviour modifications by addressing specific attack vectors. Attack vectors that attempt these modifications by targeting vulnerabilities in the \acs{os} can be mitigated by employing code obfuscation \cite{indusface} and runtime protection mechanisms \cite{qi2012spad}. Furthermore, by enforcing \acf{cfi} measures, wallets can ensure that control flow cannot be hijacked to deviate from intended control flow paths for malicious transactions \cite{Creech2017NewMitigation}. 

\subsection{Authentication}
\label{sec:auth-def}

Wallets can incorporate features either as direct protection against specific attack methods or as general authentication bypass protection. By directly integrating improved functionalities to obstruct access to predictive text data, wallets can prevent dictionary attacks \cite{Uddin2021Horus:Wallets}. Additionally, to prevent brute-force attacks, only complex passwords should be allowed in the initialisation stage  \cite{praitheeshan2019security}. Biometric falsifying attacks can be prevented by incorporating liveness detection features in wallets \cite{galbally2013image}.

To prevent single points of failure, wallets can enhance authentication levels (\autoref{sec:design-authen}) through \acs{mfa}, \acf{mpc} \cite{Lindell2020SecureComputation} and multi-signatory features such as \acs{bip}-11's M-of-N standard  \cite{bip11} (\autoref{sec:design-distr}). To mitigate social engineering attacks, wallets can incorporate phishing-resistant \acs{mfa} techniques such as FIDO2 \cite{Wang2021OnAttacks}. This feature enables communication with the original wallet website to verify authenticity before allowing access to the wallet \cite{fido2}.

\subsection{Storage and Memory}
\label{sec:sto-def}

An effective defence method against these attacks involves incorporating \acf{puf} to generate cryptographic keys on demand, without storing \teal{$sk$} on the wallet's chip. This method also prevents microscopy attacks, some other physical tampering attacks, and side-channel attacks (see \autoref{sec:crypt-def}) \cite{Urien2021InnovativeWallets, Park2024CloningFunction}. Physical tampering through the evil maid attack can be limited by implementing trusted boot mechanisms \cite{Tereshkin2010EvilEncryption}. Possible mitigations against non-invasive manipulation, such as the cold boot attack, involve adopting features which algorithmically clear the wallet's memory following intrusion \cite{seol2019amnesiac}. For example, Ledger has introduced a secure layer which detects chip intrusion and erases \teal{$sk$} following extraction attempts \cite{ledgerwallet}.

\begin{table*}[!h]
\centering
\renewcommand{\arraystretch}{1.1}
\setlength{\tabcolsep}{1.5pt} % Adjust the column separation space here
\footnotesize % or \scriptsize, \tiny, etc.
\resizebox{1.0\textwidth}{!}{
\begin{tabular}{llcccccccccccccccccccccccccccccc}
\toprule
\vspace{1pt} 
& \multicolumn{31}{c}{\textbf{Possible Defence Methods}}
\vspace{1pt} 
\\
\multicolumn{2}{c}{\textbf{ Classification}} 
& \rotatebox[origin=c]{90}{\cite{Cai2014ResearchNetwork}} % Network Authentication Tool
& \rotatebox[origin=c]{90}{\cite{Ahmed2017MitigatingNetworking}} % Web App Firewalls
& \rotatebox[origin=c]{90}{\cite{Bhirud2011LightPrevention}} % Dynamic IP Verification
& \rotatebox[origin=c]{90}{\cite{liu2018deep}} % Agent-based Traffic Mitigation
& \rotatebox[origin=c]{90}{\cite{sathwara2017distributed}} % Reset TCP Connections
& \rotatebox[origin=c]{90}{\cite{li2020android}} % Alteration Prevention Features | Access Control Restrictions ***
& \rotatebox[origin=c]{90}{\cite{ferdous2023review}} % Anti-malware Software
& \rotatebox[origin=c]{90}{\cite{indusface}} % Code Obfuscation
& \rotatebox[origin=c]{90}{\cite{Tirronen2018StoppingData}} % Cryptographic Code Verification
& \rotatebox[origin=c]{90}{\cite{Aratani2015AuthenticationChannel}} % Multi-factor Authentication
& \rotatebox[origin=c]{90}{\cite{aldawood2020advanced}} % Advanced Password Selection | Custom Keyboard Functions
& \rotatebox[origin=c]{90}{\cite{galbally2013image}} % Liveness Assessment Features
& \rotatebox[origin=c]{90}{\cite{altuwaijri2020android}} % Supplementary Storage
& \rotatebox[origin=c]{90}{\cite{breier2022practical}} % Algorithmic Fault Detection
& \rotatebox[origin=c]{90}{\cite{Urien2021InnovativeWallets}} % PUF
& \rotatebox[origin=c]{90}{\cite{Gupta2019ImpactSecurity}} % Memory and Cache Data Split
& \rotatebox[origin=c]{90}{\cite{brengel2018identifying}} % Secure Cryptographic Schemes | Deterministic Nonce Selection
& \rotatebox[origin=c]{90}{\cite{Park2024CloningFunction}} % Correlation Elimination
& \rotatebox[origin=c]{90}{\cite{Akter2023AChallenges}} % Correlation Elimination
& \rotatebox[origin=c]{90}{\cite{Lindell2020SecureComputation}} % MPC
& \rotatebox[origin=c]{90}{\cite{bip11}} % Multi-sig
& \rotatebox[origin=c]{90}{\cite{Park2023}} % Correlation sounds
& \rotatebox[origin=c]{90}{\cite{Feng2023Man-in-the-middleRedirects}} % Mitm mitation
& \rotatebox[origin=c]{90}{\cite{Kim2022ACountermeasures}} % MPC
& \rotatebox[origin=c]{90}{\cite{Shuvo2023AAttacks}} % algorithmic fault detection
& \rotatebox[origin=c]{90}{\cite{zimba2019cryptojacking}} % Intrusion Detection
& \rotatebox[origin=c]{90}{\cite{qi2012spad}} % Runtime Protection
& \rotatebox[origin=c]{90}{\cite{ManageAddresses}} % manage destination address
& \rotatebox[origin=c]{90}{\cite{hu2020overview}} % Physical Unclonable Functions (PUFs)
& \# (\%)
\vspace{1.5pt} 
\\
\midrule
\multirow{3}{*}{Precautionary} &
\rotatebox[origin=c]{0}{Prevention} & {\smallemptycirc} & {\smallemptycirc} & {\smallemptycirc} & {\smallemptycirc} & {\smallemptycirc} & {\smallfullcirc} & {\smallemptycirc} & {\smallemptycirc} & {\smallfullcirc} & {\smallemptycirc} & {\smallemptycirc} & {\smallemptycirc} & {\smallemptycirc} & {\smallemptycirc} & {\smallemptycirc} & {\smallemptycirc} & {\smallfullcirc} & {\smallemptycirc} & {\smallemptycirc} & {\smallemptycirc} & {\smallemptycirc} & {\smallemptycirc} & {\smallemptycirc} & {\smallemptycirc} & {\smallemptycirc} & {\smallemptycirc} & {\smallemptycirc} & {\smallemptycirc} & {\smallemptycirc} & \cellcolor{g2}{$3$($10\%$)} \\
& \rotatebox[origin=c]{0}{Protection} & {\smallfullcirc} & {\smallfullcirc} & {\smallfullcirc} & {\smallemptycirc} & {\smallemptycirc} & {\smallfullcirc} & {\smallfullcirc} & {\smallfullcirc} & {\smallemptycirc} & {\smallfullcirc} & {\smallfullcirc} & {\smallfullcirc} & {\smallemptycirc} & {\smallemptycirc} & {\smallfullcirc} & {\smallfullcirc} & {\smallemptycirc} & {\smallfullcirc} & {\smallfullcirc} & {\smallemptycirc} & {\smallemptycirc} & {\smallfullcirc} & {\smallfullcirc} & {\smallemptycirc} & {\smallemptycirc} & {\smallemptycirc} & {\smallfullcirc} & {\smallemptycirc} & {\smallfullcirc} &  \cellcolor{g6}{$17$($58\%$)} \\
& \rotatebox[origin=c]{0}{Limitation} & {\smallemptycirc} & {\smallemptycirc} & {\smallemptycirc} & {\smallfullcirc} & {\smallemptycirc} & {\smallemptycirc} & {\smallemptycirc} & {\smallemptycirc} & {\smallemptycirc} & {\smallemptycirc} & {\smallemptycirc} & {\smallemptycirc} & {\smallfullcirc} & {\smallemptycirc} & {\smallemptycirc} & {\smallemptycirc} & {\smallemptycirc} & {\smallemptycirc} & {\smallemptycirc} & {\smallfullcirc} & {\smallfullcirc} & {\smallemptycirc} & {\smallemptycirc} & {\smallfullcirc} & {\smallemptycirc} & {\smallemptycirc} & {\smallemptycirc} & {\smallfullcirc} & {\smallemptycirc} & \cellcolor{g3}{$6$($21\%$)} \\
\midrule
\multirow{3}{*}{Remedial} & \rotatebox[origin=c]{0}{Detection} & {\smallemptycirc} & {\smallemptycirc} & {\smallemptycirc} & {\smallfullcirc} & {\smallemptycirc} & {\smallemptycirc} & {\smallemptycirc} & {\smallemptycirc} & {\smallemptycirc} & {\smallemptycirc} & {\smallemptycirc} & {\smallemptycirc} & {\smallemptycirc} & {\smallfullcirc} & {\smallemptycirc} & {\smallemptycirc} & {\smallemptycirc} & {\smallemptycirc} & {\smallemptycirc} & {\smallemptycirc} & {\smallemptycirc} & {\smallemptycirc} & {\smallfullcirc} & {\smallemptycirc} & {\smallfullcirc} & {\smallfullcirc} & {\smallemptycirc} & {\smallemptycirc} & {\smallemptycirc} & \cellcolor{g3}{$5$($17\%$)} \\
& \rotatebox[origin=c]{0}{Response} & {\smallemptycirc} & {\smallemptycirc} & {\smallemptycirc} & {\smallfullcirc} & {\smallemptycirc} & {\smallemptycirc} & {\smallemptycirc} & {\smallemptycirc} & {\smallemptycirc} & {\smallemptycirc} & {\smallemptycirc} & {\smallemptycirc} & {\smallemptycirc} & {\smallemptycirc} & {\smallemptycirc} & {\smallemptycirc} & {\smallemptycirc} & {\smallemptycirc} & {\smallemptycirc} & {\smallemptycirc} & {\smallemptycirc} & {\smallemptycirc} & {\smallemptycirc} & {\smallemptycirc} & {\smallemptycirc} & {\smallemptycirc} & {\smallemptycirc} & {\smallemptycirc} & {\smallemptycirc} & \cellcolor{g1}{$1$($3\%$)} \\
& \rotatebox[origin=c]{0}{Recovery} & {\smallemptycirc} & {\smallemptycirc} & {\smallemptycirc} & {\smallemptycirc} & {\smallfullcirc} & {\smallemptycirc} & {\smallemptycirc} & {\smallemptycirc} & {\smallemptycirc} & {\smallemptycirc} & {\smallemptycirc} & {\smallemptycirc} & {\smallemptycirc} & {\smallemptycirc} & {\smallemptycirc} & {\smallemptycirc}  & {\smallemptycirc} & {\smallemptycirc} & {\smallemptycirc} & {\smallemptycirc} & {\smallemptycirc} & {\smallemptycirc} & {\smallemptycirc} & {\smallemptycirc} & {\smallemptycirc} & {\smallemptycirc} & {\smallemptycirc} & {\smallemptycirc} & {\smallemptycirc} & \cellcolor{g1}{$1$($3\%$)}
\vspace{1pt}
\\
\midrule 
\multicolumn{3}{c}{Summary}  &
\multicolumn{8}{c}{Precautionary:  \cellcolor{g6}{$26$($89\%$)}}  &
\multicolumn{8}{c}{Remedial: 
 \cellcolor{g3}{$7$($24\%$)}}  &
% \multicolumn{7}{c}{}  &
\multicolumn{11}{r}{Total Unique Methods    }  &
\multicolumn{1}{c}{}  &
\cellcolor{g0}{$29$($100\%$)} 
\vspace{1pt} 
   \\
\bottomrule
\end{tabular}
}
\vspace{1ex} % Add space before the caption
\caption{Defence methods categorised by type showing classification frequency (\#) and percentage (\%). Precautionary methods proactively prevent attacks; remedial methods provide attack detection, response, or data recovery.}
\label{tab:defence_methods}
\end{table*}




% need to add in brute force defence methods 
% to this table - i.e. table 5
% and also table 4
% \cite{Kiktenko2019DetectingWallets, volety2019cracking, Byun2024AAttacks}



  % {\smallfullcirc} &
  % {\smallemptycirc} &


\subsection{Cryptanalysis}
\label{sec:crypt-def}

Exploiting cryptographic vulnerabilities can lead to \teal{$sk$} extraction. Attacks that aim to exploit weak cryptographic signatures (\teal{$\sigma$}) can be counteracted by employing stronger hashing algorithms \cite{Rokhjavan2023SecuringWallets}, while deterministic \teal{$nonce$} selection prevents nonce reuse attacks \cite{brengel2018identifying}. Non-invasive attacks on cryptographic functions, including timing and power \acs{sca}, are executed by exploiting side channels. Effective prevention methods include data leakage protection and disguising data access patterns as noise injection \cite{Akter2023AChallenges, Lou2021ACryptography, Ali2023CharacterizationHardware, Park2024CloningFunction}. These affect the adversary's ability to interpret leaked information effectively \cite{Mosquera2023GuardAttacks}. 
