\section{Related Works}
\label{sec:related-work}

\subsection{Key Management}
\label{sec:key-management}

Several studies have explored key management mechanisms. Courtois and Mercer \cite{Courtois2017StealthSystems} compare key management solutions with a focus on stealth addresses. Mangipudi et al. \cite{Mangipudi2023UncoveringCrypto-Wallets} investigate key management from the wallet users' perspective. He et al. \cite{He2018AScheme} propose a secure key management scheme based on semi-trusted social networks. Di Angelo and Salzer \cite{di2020characteristics} analyse the functionality of smart contracts for key management through transaction data. Most recently, Chatzigiannis et al. \cite{chaganti2022comprehensive} propose a framework that formally evaluates hybrid recovery setups, highlighting key-management choices. Our study adopts a threat-centric view, mapping each key management technique in our multi-dimensional design taxonomy to specific attacks.

\subsection{Wallet Taxonomy}
\label{sec:wallet-taxonomy}

Prior research has proposed various methods to classify key management mechanisms \cite{bonneau2015sok, eskandari2018first, karantias2020sok, Homoliak2024SoK:Factors}.
Early wallet taxonomies by Bonneau et al. \cite{bonneau2015sok} and Eskandari et al. \cite{eskandari2018first} survey key management techniques such as password-protected files, paper-based methods, \acf{hsm} systems, password-derived wallets, and hosted services. However, this classification was confined to a single axis of key storage. Karantias \cite{karantias2020sok} contributes a protocol-centric taxonomy, examining light, full, and superlight clients and evaluating performance and security trade-offs. However, this approach does not extend to design elements such as key recovery methods or smart contract wallets, and lacks a mapping to threats and attack methods. Homoliak et al. \cite{Homoliak2024SoK:Factors} introduce an authentication-focused classification, examining k-factor and threshold-based co-signing solutions. While this emphasises the importance of multi-factor authentication in wallets, it only examines one of the several design elements we analyse.


By contrast, our taxonomy unifies multiple design dimensions into one integrative framework. These dimensions include custody model, key distribution, infrastructure (software or hardware), authentication, authorisation policies, and user recovery mechanisms. We include hardware wallets, exchange-based custodial solutions, shared-custodial implementations, non-custodial wallets, \acf{mpc} wallets, and smart contract wallets in a consistent scheme. This approach bridges the gap between academic and industry viewpoints.


\subsection{Wallet Attack and Security}
\label{sec:wallet-security}


A broad line of work surveys blockchain vulnerabilities and defences \cite{Li2020ASystems, Guo2022ASecurity, Chen2020ADefenses, zhou2023sok}. Chen et al. \cite{Chen2020ADefenses} focus on Ethereum’s protocol-layer issues. Researchers also analyse specific wallet mechanisms; in particular, HSM-focused defence studies \cite{Shbair2021HSM-basedBlockchain, Gotte2021TechAttacks}. Additional studies investigate specific vectors such as phishing \cite{andryukhin2019phishing} and desktop-wallet RPC pitfalls \cite{bui2019pitfalls}. Others scope security across wallet types \cite{Das2019AWallets}, access key management impacts \cite{Eyal2022OnDesign}, and review attacks and defences in academia \cite{Houy2023}.

Our work differs by adopting a multi-layered defence perspective and incorporating real-world incident analysis to evaluate how design choices influence attacks. This approach bridges academic models with industry practice.


\begin{table}[!ht]
  \centering
  \renewcommand{\arraystretch}{1.2}
  \resizebox{\linewidth}{!}{%
    \begin{tabular}{r*{6}{c}*{4}{c}*{4}{c}}
      \toprule
      & \multicolumn{6}{c}{\textbf{Subjects Covered}}
      & \multicolumn{4}{c}{\textbf{Methodology}}
      & \multicolumn{4}{c}{\textbf{Scope}}
      \\
      \cmidrule(lr){2-7} \cmidrule(lr){8-11} \cmidrule(lr){12-15}
      \textbf{Reference}
      & \rot[90]{Key Cryptography}
      & \rot[90]{Key Management}
      & \rot[90]{Key Recovery}
      & \rot[90]{Attack Methods}
      & \rot[90]{Security Measures}
      & \rot[90]{Privacy Techniques}
      & \rot[90]{Literature}
      & \rot[90]{Taxonomisation}
      & \rot[90]{Analysis}
      & \rot[90]{Case Study}
      & \rot[90]{Wallet Software}
      & \rot[90]{Wallet Hardware}
      & \rot[90]{Smart Contract Wallet}
      & \rot[90]{Blockchain Network} \\
      \midrule
      This Study
      & \CIRCLE & \CIRCLE & \CIRCLE & \CIRCLE & \CIRCLE & \Circle
      & \CIRCLE & \CIRCLE & \CIRCLE & \CIRCLE
      & \CIRCLE  & \CIRCLE  & \CIRCLE  & \Circle \\
      \cite{bonneau2015sok}
      & \CIRCLE & \CIRCLE & \Circle  & \Circle  & \CIRCLE & \CIRCLE
      & \CIRCLE & \CIRCLE & \CIRCLE & \Circle
      & \CIRCLE  & \Circle   & \Circle   & \CIRCLE \\
      \cite{eskandari2018first}
      & \Circle   & \CIRCLE  & \CIRCLE  & \Circle   & \CIRCLE  & \Circle
      & \CIRCLE  & \CIRCLE  & \CIRCLE  & \Circle
      & \CIRCLE  & \Circle   & \Circle   & \Circle \\
      \cite{karantias2020sok}
      & \Circle   & \Circle   & \Circle   & \Circle   & \CIRCLE  & \CIRCLE
      & \CIRCLE  & \CIRCLE  & \CIRCLE  & \Circle
      & \CIRCLE  & \Circle   & \Circle   & \Circle \\
      \cite{Homoliak2020SmartOTPs:Wallets}
      & \Circle   & \CIRCLE  & \Circle   & \CIRCLE  & \CIRCLE  & \Circle
      & \CIRCLE  & \CIRCLE  & \CIRCLE  & \Circle
      & \CIRCLE  & \CIRCLE  & \Circle   & \CIRCLE \\
      \cite{Houy2023}
      & \Circle   & \CIRCLE  & \CIRCLE  & \CIRCLE  & \CIRCLE  & \CIRCLE
      & \CIRCLE  & \CIRCLE  & \CIRCLE  & \Circle
      & \CIRCLE  & \CIRCLE  & \Circle   & \CIRCLE \\
      \cite{suratkar2020cryptocurrency}
      & \CIRCLE & \CIRCLE & \CIRCLE & \Circle  & \Circle  & \Circle
      & \CIRCLE & \CIRCLE & \Circle   & \Circle
      & \CIRCLE  & \Circle   & \Circle   & \Circle \\
      \cite{bui2019pitfalls}
      & \Circle  & \Circle  & \Circle  & \CIRCLE & \CIRCLE & \Circle
      & \CIRCLE & \CIRCLE & \CIRCLE  & \Circle
      & \CIRCLE  & \Circle   & \Circle   & \Circle \\
      \cite{zaghloul2020bitcoin}
      & \Circle  & \Circle  & \Circle  & \Circle  & \CIRCLE & \CIRCLE
      & \CIRCLE & \CIRCLE & \CIRCLE  & \Circle
      & \CIRCLE  & \Circle   & \Circle   & \CIRCLE \\
      \cite{li2020android}
      & \Circle  & \Circle  & \Circle  & \CIRCLE & \CIRCLE & \Circle
      & \CIRCLE & \Circle   & \CIRCLE  & \Circle
      & \CIRCLE  & \Circle   & \Circle   & \Circle \\
      \cite{Dai2018SBLWT:Trustzone}
      & \CIRCLE & \CIRCLE & \Circle  & \CIRCLE & \CIRCLE & \Circle
      & \Circle  & \CIRCLE  & \CIRCLE  & \Circle
      & \CIRCLE  & \Circle   & \Circle   & \Circle \\
      \cite{volety2019cracking}
      & \Circle  & \Circle  & \Circle  & \CIRCLE & \CIRCLE & \Circle
      & \CIRCLE & \Circle   & \CIRCLE  & \Circle
      & \CIRCLE  & \Circle   & \Circle   & \Circle \\
      \cite{8966739}
      & \CIRCLE & \CIRCLE & \CIRCLE & \CIRCLE & \CIRCLE & \CIRCLE
      & \CIRCLE & \Circle   & \CIRCLE  & \Circle
      & \Circle   & \CIRCLE   & \Circle   & \Circle \\
      \cite{rezaeighaleh2020improving}
      & \CIRCLE & \CIRCLE & \CIRCLE & \Circle  & \CIRCLE & \Circle
      & \CIRCLE & \CIRCLE & \CIRCLE  & \Circle
      & \Circle   & \CIRCLE   & \Circle   & \Circle \\
      \cite{Urien2021InnovativeWallets}
      & \Circle  & \Circle  & \Circle  & \CIRCLE & \CIRCLE & \Circle
      & \CIRCLE & \CIRCLE & \Circle   & \Circle
      & \Circle   & \CIRCLE   & \Circle   & \Circle \\
      \cite{Rezaeighaleh2020MultilayeredWallet}
      & \Circle  & \Circle  & \CIRCLE & \Circle  & \Circle  & \CIRCLE
      & \CIRCLE & \CIRCLE & \Circle   & \Circle
      & \Circle   & \CIRCLE   & \Circle   & \Circle \\
      \cite{rezaeighaleh2019new}
      & \CIRCLE & \CIRCLE & \CIRCLE & \Circle  & \CIRCLE & \Circle
      & \CIRCLE & \Circle   & \CIRCLE  & \Circle
      & \Circle   & \CIRCLE   & \Circle   & \Circle \\
      \cite{di2020characteristics}
      & \Circle  & \Circle  & \Circle  & \Circle  & \Circle  & \Circle
      & \CIRCLE & \CIRCLE & \CIRCLE  & \Circle
      & \Circle   & \Circle    & \CIRCLE   & \Circle \\
% ----------- VERIFIED & CORRECTED ROWS ----------------------------------
\cite{Homoliak2024SoK:Factors}
& \Circle & \CIRCLE & \Circle & \Circle & \CIRCLE & \Circle
& \CIRCLE & \CIRCLE & \CIRCLE & \Circle
& \CIRCLE & \CIRCLE & \Circle & \Circle \\

\cite{andryukhin2019phishing}
& \Circle & \Circle & \Circle & \CIRCLE & \CIRCLE & \Circle
& \CIRCLE & \Circle & \CIRCLE & \Circle
& \Circle & \Circle & \Circle & \Circle \\

\cite{Chen2020ADefenses}
& \Circle & \Circle & \Circle & \CIRCLE & \CIRCLE & \Circle
& \CIRCLE & \CIRCLE & \CIRCLE & \Circle
& \Circle & \Circle & \Circle & \CIRCLE \\

\cite{Courtois2017StealthSystems}
& \CIRCLE & \CIRCLE & \Circle & \CIRCLE & \CIRCLE & \CIRCLE
& \CIRCLE & \Circle & \CIRCLE & \Circle
& \CIRCLE & \Circle & \Circle & \CIRCLE \\

\cite{Das2019AWallets}
& \CIRCLE & \CIRCLE & \Circle & \Circle & \CIRCLE & \Circle
& \Circle & \Circle & \CIRCLE & \Circle
& \CIRCLE & \Circle & \Circle & \Circle \\

\cite{Eyal2022OnDesign}
& \Circle & \CIRCLE & \Circle & \Circle & \CIRCLE & \Circle
& \Circle & \CIRCLE & \CIRCLE & \Circle
& \CIRCLE & \CIRCLE & \Circle & \Circle \\

\cite{Gotte2021TechAttacks}
& \CIRCLE & \CIRCLE & \Circle & \CIRCLE & \CIRCLE & \Circle
& \Circle & \Circle & \CIRCLE & \Circle
& \Circle & \CIRCLE & \Circle & \Circle \\

\cite{Guo2022ASecurity}
& \Circle & \CIRCLE & \Circle & \CIRCLE & \CIRCLE & \Circle
& \CIRCLE & \CIRCLE & \CIRCLE & \Circle
& \CIRCLE & \CIRCLE & \Circle & \CIRCLE \\

\cite{He2018AScheme}
& \Circle & \CIRCLE & \CIRCLE & \Circle & \CIRCLE & \Circle
& \Circle & \Circle & \CIRCLE & \Circle
& \CIRCLE & \Circle & \Circle & \Circle \\

\cite{Houy2023}
& \CIRCLE & \CIRCLE & \CIRCLE & \CIRCLE & \CIRCLE & \Circle
& \CIRCLE & \CIRCLE & \CIRCLE & \Circle
& \CIRCLE & \CIRCLE & \Circle & \CIRCLE \\

\cite{Li2020ASystems}
& \Circle & \CIRCLE & \Circle & \CIRCLE & \CIRCLE & \Circle
& \Circle & \Circle & \CIRCLE & \Circle
& \CIRCLE & \Circle & \Circle & \Circle \\

\cite{Mangipudi2023UncoveringCrypto-Wallets}
& \Circle & \CIRCLE & \Circle & \Circle & \Circle & \Circle
& \Circle & \Circle & \CIRCLE & \CIRCLE
& \CIRCLE & \Circle & \Circle & \Circle \\

\cite{Shbair2021HSM-basedBlockchain}
& \CIRCLE & \CIRCLE & \Circle & \CIRCLE & \CIRCLE & \Circle
& \Circle & \Circle & \CIRCLE & \CIRCLE
& \Circle & \CIRCLE & \Circle & \Circle \\

\cite{zhou2023sok}
& \Circle & \Circle & \Circle & \CIRCLE & \CIRCLE & \Circle
& \CIRCLE & \CIRCLE & \CIRCLE & \Circle
& \Circle & \Circle & \CIRCLE & \CIRCLE \\

\cite{chatzigiannis2025composability}
& \Circle & \CIRCLE & \CIRCLE & \CIRCLE & \CIRCLE & \Circle
& \Circle & \Circle & \CIRCLE & \CIRCLE
& \CIRCLE & \Circle & \CIRCLE & \Circle \\

% ------------------------------------------------------------------------

% -------------------------------------------------------------------------

    % \cite{Homoliak2024SoK:Factors}
    %   & \Circle  & \Circle  & \Circle  & \Circle  & \Circle  & \Circle
    %   & \Circle & \Circle & \Circle  & \Circle
    %   & \Circle   & \Circle    & \Circle   & \Circle \\
    %       \cite{andryukhin2019phishing}
    %   & \Circle  & \Circle  & \Circle  & \Circle  & \Circle  & \Circle
    %   & \Circle & \Circle & \Circle  & \Circle
    %   & \Circle   & \Circle    & \Circle   & \Circle \\
      
    %       \cite{Chen2020ADefenses}
    %   & \Circle  & \Circle  & \Circle  & \Circle  & \Circle  & \Circle
    %   & \Circle & \Circle & \Circle  & \Circle
    %   & \Circle   & \Circle    & \Circle   & \Circle \\
    %       \cite{Courtois2017StealthSystems}
    %   & \Circle  & \Circle  & \Circle  & \Circle  & \Circle  & \Circle
    %   & \Circle & \Circle & \Circle  & \Circle
    %   & \Circle   & \Circle    & \Circle   & \Circle \\
      
    %       \cite{Das2019AWallets}
    %   & \Circle  & \Circle  & \Circle  & \Circle  & \Circle  & \Circle
    %   & \Circle & \Circle & \Circle  & \Circle
    %   & \Circle   & \Circle    & \Circle   & \Circle \\
      
    %       \cite{Eyal2022OnDesign}
    %   & \Circle  & \Circle  & \Circle  & \Circle  & \Circle  & \Circle
    %   & \Circle & \Circle & \Circle  & \Circle
    %   & \Circle   & \Circle    & \Circle   & \Circle \\
      
    %       \cite{Gotte2021TechAttacks}
    %   & \Circle  & \Circle  & \Circle  & \Circle  & \Circle  & \Circle
    %   & \Circle & \Circle & \Circle  & \Circle
    %   & \Circle   & \Circle    & \Circle   & \Circle \\
      
    %       \cite{Guo2022ASecurity}
    %   & \Circle  & \Circle  & \Circle  & \Circle  & \Circle  & \Circle
    %   & \Circle & \Circle & \Circle  & \Circle
    %   & \Circle   & \Circle    & \Circle   & \Circle \\
      
    %       \cite{He2018AScheme}
    %   & \Circle  & \Circle  & \Circle  & \Circle  & \Circle  & \Circle
    %   & \Circle & \Circle & \Circle  & \Circle
    %   & \Circle   & \Circle    & \Circle   & \Circle \\
      
    %       \cite{Houy2023}
    %   & \Circle  & \Circle  & \Circle  & \Circle  & \Circle  & \Circle
    %   & \Circle & \Circle & \Circle  & \Circle
    %   & \Circle   & \Circle    & \Circle   & \Circle \\
      
    %       \cite{Li2020ASystems}
    %   & \Circle  & \Circle  & \Circle  & \Circle  & \Circle  & \Circle
    %   & \Circle & \Circle & \Circle  & \Circle
    %   & \Circle   & \Circle    & \Circle   & \Circle \\
    %       \cite{Mangipudi2023UncoveringCrypto-Wallets}
    %   & \Circle  & \Circle  & \Circle  & \Circle  & \Circle  & \Circle
    %   & \Circle & \Circle & \Circle  & \Circle
    %   & \Circle   & \Circle    & \Circle   & \Circle \\
      
    %       \cite{Shbair2021HSM-basedBlockchain}
    %   & \Circle  & \Circle  & \Circle  & \Circle  & \Circle  & \Circle
    %   & \Circle & \Circle & \Circle  & \Circle
    %   & \Circle   & \Circle    & \Circle   & \Circle \\
      
    %       \cite{zhou2023sok}
    %   & \Circle  & \Circle  & \Circle  & \Circle  & \Circle  & \Circle
    %   & \Circle & \Circle & \Circle  & \Circle
    %   & \Circle   & \Circle    & \Circle   & \Circle \\
      \bottomrule
    \end{tabular}%
  }
  \caption{Overview of related works. (\CIRCLE: include, \Circle: not include)}
  \label{Literature-Gap-Table-1}
\end{table}


\subsection{Addressing Literature Gaps}
\label{sec:gaps-in-literature}

Despite various studies on specific wallet types, mechanisms, and attack vectors, there is a lack of comprehensive examination spanning wallet design taxonomy, attack methods, incident analysis, security measures, and case studies, as shown in \autoref{Literature-Gap-Table-1}. Moreover, our design taxonomy is mapped with a detailed threat model and defence strategies, allowing a systematic evaluation of each design's security trade-offs. This comprehensive coverage and empirical attack data distinguish our work from prior classification-focused surveys. Our study bridges this gap, providing a holistic understanding crucial for advancing wallet security.


