
\section{Case Studies}
\label{sec:case_study}

In this section, we present detailed case studies of notable wallet security breaches. We apply our wallet design taxonomy (\autoref{sec:wallet-taxonomy}), threat model (\autoref{sec:threat_framework}), and attack taxonomy (\autoref{sec:attack-framework}). Each case study systematically analyses the wallet's architecture, identifies exploited vulnerabilities, and explores the sequence of attack events. We conclude each study with recommended and implemented security measures.

\subsection{Case Study: ByBit Custodial Wallet Hack}
\label{sec:bybit_case}

In February 2025, ByBit experienced a significant security breach that resulted in a loss of approximately \$1.5 billion in Ethereum, marking the largest cryptocurrency theft to date \cite{bybit}. This sophisticated attack aligns with the attack vectors outlined by our taxonomy. We provide a detailed analysis below using our frameworks for design classification, threat assessment, attack sequence analysis, and mitigation strategies.

\subsubsection{Design}
\label{sec:bybit_mech}

Using our design taxonomy in \autoref{sec:wallet-taxonomy}, we analyse the ByBit wallet design below:


\begin{itemize}
    \item \textbf{Custody:} ByBit maintained full custody of user funds, with users relinquishing \textcolor{teal}{\textit{sk}} control to the exchange. This particular case pertains to the \textcolor{teal}{\textit{sk}}, which controlled the Ethereum assets of the exchange. 
    \item \textbf{Infrastructure:} 
    ByBit employed a multi-faceted infrastructure design, integrating hardware wallets with a smart contract-enabled proxy architecture. The primary proxy contract delegated logic execution to a separate implementation contract via \texttt{delegateCall}. It stored the implementation contract's address in storage slot 0 to facilitate future upgrades \cite{bybit_secux}. However, the design did not enforce strict access controls on this critical operation. This became a key factor exploited in the attack, as described in the threat analysis (see \autoref{sec:bybit_dep}).


    \item \textbf{Distribution:} \textcolor{teal}{\textit{sk}} management was distributed securely with authorisation rights shared among multiple private key (\textcolor{teal}{\textit{sk}}) holders in the multi-sig scheme across different hardware devices. The multi-signature scheme prevented unilateral transactions, mandating consensus among multiple trusted individuals. 
    \item \textbf{Authorisation:} Transactions were generated via Safe's web interface. Signers reviewed transaction details on the web user interface and hardware wallet screens. Only after confirmation on their Ledger hardware wallet devices were transactions broadcast to the blockchain.
    \item \textbf{Validation:} After obtaining the necessary approvals, transactions underwent validation to ensure compliance with ByBit's internal security policies. This included verifying adherence to address whitelisting protocols and transfer limits. The multi-sig smart contract enforced these policies by executing transactions only when the requisite number of valid signatures was present.
\end{itemize}


\subsubsection{Threats and Dependencies}
\label{sec:bybit_dep}

ByBit’s security architecture relied significantly on several interconnected elements, including the Safe user interface, which proved vulnerable to the adversaries' attempts. We outline the threats, which were exploited by the adversary inline with our threat model below:

\begin{itemize}
    \item \textbf{Insecure Interaction:} Insecure interactions resulted in the system's exposure to threats. The adversary likely exploited these interactions to achieve infiltration of the Safe developer's machine \cite{bybit_certik}. 
    \item \textbf{Application Provider Compromise:} ByBit's operational security was heavily dependent on the integrity and security posture of third-party service providers, in this case, Safe’s web interface.
    \item \textbf{Data Misrepresentation:} The adversary compromised the accuracy and reliability of transaction data presented to authorised signers through Safe's user interface. This highlighted a critical vulnerability in wallet user interfaces.
    \item \textbf{Application Logic Flaw:} The infrastructure design permitted unrestricted use of the \texttt{delegateCall} instruction, allowing malicious actors to overwrite critical storage slots. Specifically, the attackers exploited the ability to overwrite the logic pointer stored in storage slot 0, leading to unauthorised control of the proxy's logic \cite{bybit_certik}. This violated the principle of least privilege and directly facilitated the privilege escalation step of the attack.
    \item \textbf{Blind Signing:} ByBit's reliance on hardware wallet confirmation processes did not sufficiently address the blind signing risk. Signers assumed the hardware wallet displays were a trustworthy verification source and approved transactions without explicit visibility into critical transaction metadata. This included \texttt{delegateCall} operations and underlying implementation changes. 

\end{itemize}



\subsubsection{Adversary Goal and Capabilities}
\label{sec:bybit_cap}

\textcolor{teal}{\textit{A}} aimed to gain unauthorised rights by masking adversary-created transactions as benign. The capabilities of \textcolor{teal}{\textit{A}} significantly evolved during the attack as extended knowledge was gained, starting from restricted external knowledge and progressing to insider-level knowledge and access:

\begin{itemize}
    \item \textbf{Initial Phase:} \textcolor{teal}{\textit{A}} remotely exploited publicly accessible information to exploit Safe developer interactions and gain restricted internal access.
    \item \textbf{Intermediate Phase:} Having achieved insider-level knowledge and privileges following a successful repository compromise, \textcolor{teal}{\textit{A}} could inject malicious software into operational components of the wallet software.
    \item \textbf{Final Phase:} \textcolor{teal}{\textit{A}} could exploit application logic to deceive \textcolor{teal}{\textit{sk}} holders, achieving credential compromise. Subsequently, \textcolor{teal}{\textit{A}} gained full wallet control and authorisation rights.
\end{itemize}

\subsubsection{Attack Sequence}
\label{sec:bybit_att}

The ByBit incident represents a sophisticated combination of several coordinated attack vectors identified in our Application threats taxonomy:

\begin{itemize}
    \item \textbf{Social Engineering:} A phishing attack method enabled the execution of subsequent attack vectors. Social engineering and malware were combined to compromise ByBit, as seen in past incidents (e.g., BitKeep \cite{CertiKIncidents}, Upbit \cite{UpbitMedium}, and wallet drainers \cite{RektREKT}). This gave the adversary direct access to Safe's front-end code repository, highlighting the importance of secure developer environments.
    \item \textbf{Malware Execution:} The compromised machine enabled the injection of malicious JavaScript into Safe's front-end code, targeting the transaction approval interface. The malware modified the transaction data displayed to \textcolor{teal}{\textit{sk}} holders. While legitimate transaction details were displayed in the Safe wallet user interface, the data sent to the hardware wallet was altered.
    \item \textbf{Privilege Escalation:} 
    The approved transaction altered the smart contract's logic. The attackers exploited storage slot hijacking by crafting a transaction that used \texttt{delegateCall} to execute a spoofing contract. This contract’s \texttt{transfer()} function wrote the attacker’s malicious implementation address to storage slot 0 via the \acf{evm} \texttt{SSTORE} opcode, overwriting the proxy’s logic pointer. With the proxy now delegating to the attacker’s contract, all subsequent transactions executed attacker-controlled code in the proxy’s context, granting full authorisation rights.


\end{itemize}

\subsubsection{Security Measures}
\label{sec:bybit_def}

Before the breach, ByBit used a layered security model: most funds were in a Safe contract, private keys on six Ledger devices, requiring 4-of-6 multi-sig. These measures were bypassed. After the incident, industry experts highlighted the following additional controls:

\begin{itemize}
    \item \textbf{Independent Transaction Hash Verification:} The use of tools such as \texttt{safe-tx-hashes} to independently verify transaction hashes against on-chain data mitigates the risk of UI-level deception \cite{bybit_cyfrin}. By enabling signers to cross-reference actual transaction payloads outside of potentially compromised interfaces, this approach detects malicious operations such as unauthorised \texttt{delegateCall} or logic pointer overwrites before execution.
    \item \textbf{Transaction Policy Enforcement via On-Chain Gatekeeping:} Preventative solutions such as Halborn’s Seraph simulate signed transactions before execution and block operations that violate predefined organisational policies \cite{bybit}. In the context of the ByBit attack, this approach could have flagged and halted the unauthorised upgrade triggered by the malicious \texttt{delegateCall}, enforcing a secondary layer of validation beyond signer intent.
    \item \textbf{Hardware Wallet Clear-Signing:} Require devices that support the on-device display of the complete destination, value, function selector, and raw calldata (clear-signing) before approval. This enables signers can independently verify every field and avoid hash-only blind signing, a weakness exploited in the ByBit breach \cite{bybit_secux}.    
    \item \textbf{Wallet Auditing:} Conducting regular audits focusing on storage layout consistency and \texttt{delegateCall} whitelisting and other wallet-related code is pertinent \cite{bybit_slowmist}
\end{itemize}


\subsection{Case Study: Slope Non-Custodial Wallet Hack} \label{sec:slope_case}

In August 2022, Slope Wallet experienced a severe security incident, resulting in the compromise of over 9,200 user wallets on the Solana blockchain and a loss of approximately \$4.1 million in SOL and USDC \cite{CoinTelegraph2022SlopeAttack}. We provide a detailed analysis below using our frameworks for design classification, threat assessment, attack sequence analysis, and implemented security measures.

\subsubsection{Design} \label{sec:slope_design}

Applying our design taxonomy, we analyse the Slope wallet design below:

\begin{itemize} 
\item \textbf{Custody:} Slope utilised a non-custodial model where users retained complete control over the private key (\textcolor{teal}{\textit{sk}}). This case pertains to the management and leakage of the user's private key.
\item \textbf{Infrastructure:} Slope used a mobile software wallet that relied on a self-hosted Sentry monitoring stack \cite{cyberintel_slope, fyeo_slope}. This setup collected application data for debugging but inadvertently logged sensitive information due to a faulty logging function.
\item \textbf{Distribution:} Slope used a single-distribution model, with all cryptographic operations and storage conducted solely on the user’s mobile device. No advanced key distribution methods, such as MPC or multi-signature schemes, were integrated. 
\item \textbf{Authorisation and Validation:} The standard Solana \textit{Ed25519} signature flow was executed locally on the user device. Transaction broadcasting was performed via Slope's own \acs{rpc} endpoints.
 \end{itemize}

\subsubsection{Threats and Dependencies} \label{sec:slope_dep}

Slope’s security architecture relied heavily on interconnected dependencies, particularly its integrated application-monitoring stack, as detailed below:
\begin{itemize} \item \textbf{Application‑Monitoring Dependency:} Slope utilised an on-premise implementation of the Sentry SDK, designed to assist developers in debugging. A single improperly added \texttt{toString()} method circumvented built-in security filters, resulting in sensitive wallet private keys being unintentionally logged in plaintext \cite{cyberintel_slope}.
\item \textbf{Data Leakage:} Multiple defensive measures were used (collection filtering, \acf{tls} certificate pinning, database encryption at rest). However, collection filtering and database encryption were disabled, causing plaintext private keys to be stored in the database.
\item \textbf{Third‑Party Supply‑Chain Threat:} Slope employed a self-hosted version of the third-party monitoring solution (Sentry), inheriting risks associated with configuration drift, patch management latency, and internal operational errors. This on-premise deployment introduced vulnerabilities typically mitigated by a SaaS-managed setup.
\item \textbf{Insecure User Interaction:} Users continued to interact with wallets whose keys had potentially been exfiltrated. No built-in key-rotation prompt existed. \end{itemize}

\subsubsection{Adversary Goal and Capabilities} \label{sec:slope_cap}

The adversary, \textcolor{teal}{\textit{A}}, aimed primarily for credential compromise, specifically targeting the user's private key (\textcolor{teal}{\textit{sk}}). The capabilities leveraged by \textcolor{teal}{\textit{A}} included:

\begin{itemize} 
\item \textbf{Initial Phase:} \textcolor{teal}{\textit{A}} used knowledge of Slope’s logging vulnerability (via reverse engineering or insider information) to target the timeframe and method to extract logged private keys. 
\item \textbf{Intermediate Phase:} \textcolor{teal}{\textit{A}} employed remote network access, either directly to the internal database or by intercepting \acs{tls} traffic prior to 18 July 2022. This remote capability allowed the extraction of plaintext private keys despite the security measures initially in place.
\item \textbf{Final Phase:} \textcolor{teal}{\textit{A}} employed legitimate wallet signing authority using stolen keys and subsequently drained user funds directly via standard blockchain transactions without triggering conventional anomaly detection.
\end{itemize}

\subsubsection{Attack Sequence} \label{sec:slope_att}

In this incident, the adversary employed the logic exploitation vector to compromise credentials, as summarised below:

\begin{itemize} 
\item \textbf{Logic Bug Introduction:} Slope utilised a helper function (\texttt{toString()}) to streamline debugging, unintentionally bypassing established security filters. This bug directly caused private keys to enter plaintext logging pipelines.
\item \textbf{Data Pipeline Restart:} Slope utilised Kafka for data processing. After restarting, Kafka inadvertently flushed cached logs containing private keys in plaintext format directly into a PostgreSQL database.
\item \textbf{Log Exfiltration:} \textcolor{teal}{\textit{A}} accessed the misconfigured Sentry instance and retrieved the plaintext seed phrases, fully compromising user private keys.
\item \textbf{Wallet draining:} \textcolor{teal}{\textit{A}} utilised legitimate signing authority gained from compromised private keys and drained assets from 9,229 wallet addresses within seven hours.

\end{itemize}

\subsubsection{Security Measures
} 
\label{sec:slope_def}

Following the Slope Wallet breach, the development team initiated several immediate reactive security measures. The team promptly disabled the self-hosted Sentry server within 15 minutes of identifying the vulnerability and advised users to transfer their assets to new wallets \cite{dailycoin_slope}. Additionally, audits conducted by \href{https://osec.io/}{OtterSec} and \href{https://www.slowmist.com/}{Slowmist} confirmed that sensitive data, including private keys, had been inadvertently logged \cite{ottersec_slope, zellic_slope}. In response, Slope removed all sensitive logging functionalities and implemented a \texttt{beforeSend} whitelist to filter out confidential information \cite{slope_statement}. 

To prevent such incidents in the future, it is crucial to ensure that application monitoring tools, such as Sentry, are meticulously configured to exclude sensitive data from logs. This involves implementing stringent data scrubbing protocols and avoiding the logging of private keys or seed phrases. Proper calibration of these safeguards is essential for preserving the confidentiality and integrity of user credential data.



