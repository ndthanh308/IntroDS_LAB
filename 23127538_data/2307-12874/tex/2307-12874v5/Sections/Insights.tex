\section{Insights}
\label{sec:insights}

We discuss insights on design, threats, attack methods, and security measures from academic papers, industry incidents, and case studies below:

\subsection{Influence of Design on Threats}
\label{sec:threats_dis_influence}

Despite a wide range of security setups, we observe that the majority of the design combinations of existing wallets surveyed have been threatened by multiple vulnerabilities, as shown in \autoref{tab:wlt._taxonomy}. This is due to similar implementations i.e., the use of replicated libraries and commonly integrated implementation proposals (e.g., \acs{eip}-4337). We also observe that some wallets have had numerous vulnerabilities discovered in industry and academia. Most notably, Ledger and Trezor have several data remanence, data manipulation and insecure cryptographic vulnerabilities. Furthermore, in mapping vulnerabilities to attacks, we observe that some vulnerabilities can lead to numerous attack vectors as shown in \autoref{fig:wallet-mapping}. These include inadequate authentication, insecure permissions, insecure user interactions, and particularly data leakage. The Slope Wallet incident exemplifies this, where an improperly configured debug logging mechanism led directly to private key leakage.

\subsection{High Occurrence of Signature Verification Logic Flaws}
\label{sec:sig_verif_flaw}

We observe that signature verification logic flaws account for the most vulnerability occurrences in various wallets surveyed, constituting 19\%. Another interesting observation is the occurrence of this vulnerability in three diverse wallet security enhancement architectures, namely hardware, smart contract and \acs{mpc} wallets \cite{cve_14199, fireblocks_23, AccountMedium, UncoveringVulnerability}.


\subsection{Gap Analysis on Wallet Threats}

Conducting a gap analysis across industry and academic reports is difficult because many incidents do not disclose precise attack methods. We generally observe a high correlation between identified threats in industry and academia, except for insider and external threats. Specifically, in the following threats: malicious insider, compromised insider and compromised service provider threats. Although several custodial designs have been proposed by academia along with threat models, an investigation into the potential external threats and attacks in custodial setups would be highly beneficial for the industry. Notably, most industry attacks target exchanges and other custodial setups, as large funds are concentrated within a few wallet addresses. Additionally, research into these areas will also be pertinent due to the fact that wallet designs are gradually evolving into shared-custodial or other setups which require authentication from a centralised party (e.g., passkey, \acs{2fa}).

To address the gaps identified in \autoref{tab:threat_capability}, we propose the following measures:  
\begin{itemize}
    \item \textbf{Responsible Disclosure Policies:} Create a standardised incident template for responsible disclosure of wallet-related incidents. This could employ a uniform reporting format for exchanges and custodians to use when disclosing incidents, enabling both industry and academic audiences to analyse them consistently. A notable example in industry is Immunefi’s vulnerability disclosure platform \cite{immunefi2024}.
    
    \item \textbf{Public-Private Collaborations:} Formalise partnerships between exchanges, blockchain security firms, and academic institutions to analyse incident data. Successful models exist, including as IC3 and Chainlink partnership \cite{ic3} and the Stanford Centre for Blockchain Research’s industry partnerships \cite{stanford_cbr}.

       \item \textbf{Open-source Incident Registry:} Develop an open repository where vetted blockchain incident post-mortems can be deposited by operators and accessed by researchers, policymakers, and other exchanges. An existing example is the SlowMist Hacked incident archive \cite{slowmist_hacked}.


\end{itemize}


\subsection{Difference in Academia and Notable Industry Incidents}

Identifying attack vectors within the industry remains challenging, as sources often lack specificity. Notable attack vectors are significantly less clear (46\% unknown) and show a lower spread compared to attacks described in the literature (see \autoref{tab:attack_vectors}). This might be attributed to a lack of detailed post-mortem analysis in several incidents and an adversary's tendency to prioritise cost-effective methods. Academia, on the other hand, shows a high percentage (93\%) and spreads across various attack methods. Our case study on the ByBit incident also exemplifies the complexity of real-world incidents compared to academic models. While academic literature often isolates attack vectors, the ByBit incident involved a multi-stage, multi-vector attack with a chain of sub-goals linked to the main goal of \teal{$sk$} compromise.

\subsection{High-Risk Third-Party Dependencies}

The ByBit attack highlights a critical systemic risk in modern wallet architectures: third-party dependencies can nullify even highly secure solutions. Despite ByBit’s use of hardware wallets, multi-sig authorisation, and transaction policies, its reliance on Safe’s third-party UI created a single point of failure. Similarly, Slope Wallet's reliance on a self-hosted instance of a third-party monitoring solution (Sentry) introduced vulnerabilities due to misconfiguration and operational errors. This further underscores how third-party integrations significantly impact wallet security. This demonstrates that wallet security inherits the weakest link in dependency chains. To mitigate these risks, wallets must adopt resilient architectures and proactively manage third-party risks through multi-layered audits and adversarial scenario modelling.



\subsection{Comparison of Custodial and Non-Custodial Attacks}
Our incident analysis reveals that custodial wallets and non-custodial accounts for 70\% and 30\% of attacks,  respectively.  Additionally, unknown methods are significantly higher in custodial wallets (50\%) than in non-custodial wallets (36\%). Incidents show a high degree of similarity between custodial and non-custodial attacks. For instance, in comparison to other attacks, phishing attacks account for a relatively high percentage of both custodial (10\%) and non-custodial (36\%) wallets, especially factoring in the number of unknown attacks. 

\subsection{High Malware and Phishing Attack Occurrence}

We also find that application attacks account for a significant percentage of incident occurrences (43\%), with 34\% in custodial wallets and 48\% in non-custodial wallets. Our data also indicates that malware and phishing attacks are the most common attack vectors, accounting for 10\% and 18\% of total incidents, respectively. We also find that phishing-malware attacks constitute 48\% of total non-custodial wallet attacks.

\subsection{Limitations of Security Measures}
\label{sec:def_dis_attacks}


The majority of defence implementations in academia are particularly tailored to specific advanced attacks such as \acs{puf} for microscopic attacks, correlation elimination sounds for non-invasive side channels, and \acs{puf} attacks. Despite this, academia does not account for sophisticated attacks, which may leverage multiple attack vectors. Furthermore, distributed architectures prevalent in the industry are insufficient if dependencies remain centralised. The ByBit breach demonstrates that security measures must extend to third-party components, requiring redundant safeguards such as on-chain transaction simulation to detect UI spoofing or logic hijacking. In addition, the Slope Wallet incident demonstrates how inadequate configuration of application monitoring tools can undermine otherwise secure implementations, highlighting the need for strict data scrubbing and monitoring configurations.

\subsection{Comparison of Precautionary and Remedial Defence Methods}
\label{sec:def_dis_attacks}

Our study presents defence methods applicable to various attack vectors, with the majority offering either precautionary or remedial strategies, as illustrated in \autoref{tab:defence_methods}. Notably, precautionary defences significantly outnumber remedial approaches, comprising roughly 89\% of all methods observed. Within the precautionary category, protection-focused implementations are the most prevalent, accounting for 58\%. Among remedial defences, detection methods are the most common at 17\%, while response and recovery measures each represent a mere 3\%. This disparity highlights a critical gap in reactive mitigation techniques, indicating a potential area for further development in response and recovery-focused defences.


