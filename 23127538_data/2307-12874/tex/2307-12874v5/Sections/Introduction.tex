\section{Introduction}
\label{sec:introduction}



Pioneered by Bitcoin \cite{NakamotoBitcoin:System}, peer-to-peer transactions have evolved into a digital ecosystem of decentralised financial applications on the blockchain. Building on this foundation with self-executing smart contracts on blockchain networks such as Ethereum, \ac{defi} protocols enable decentralised lending \cite{arora2024secplf}, exchanges \cite{xu2023sok}, derivatives \cite{luo2025piercing}, insurance \cite{cousaert2022token}, and numerous other financial applications \cite{cousaert2022sok, mita2019stablecoin, luo2025llm}. As the user-facing component, wallets intricately trigger various transactions.

A wallet is a transaction-facilitating tool that manages user authentication to enable digital signing of transactions. It broadcasts these messages to a blockchain network to confirm their validity. When initiating a transaction, wallets use a private key to sign and broadcast the signature to the blockchain network \cite{khan2022gas}. Private key security is therefore critical, as incidents such as the Mt. Gox exchange attack (850,000 BTC) have resulted in significant financial losses for individual users and entities relying on the service \cite{mtgox_hack}. Additional attacks on KuCoin \cite{kucoinNew}, Vulcan Forged \cite{VulcanHack}, Infarno \cite{infarno}, WazirX \cite{Explained:2024g}, and ByBit \cite{bybit_certik} have demonstrated that both custodial and non-custodial wallets present attractive targets.

This paper introduces a novel multi-dimensional cryptocurrency wallet taxonomy that extends beyond earlier approaches by covering both legacy and emerging wallets. The taxonomy reveals how specific design decisions correlate with known threat occurrences (\autoref{sec:wallet-taxonomy}). We systematise threats (\autoref{sec:threat_framework}) and attacks (\autoref{sec:attack-framework}), which enables us to suggest potential defence strategies (\autoref{sec:defense-strategies}). We then discuss our analysis of design elements, attack vectors, and defence types in \autoref{sec:insights}. In summary, our contributions are as follows:
\begin{itemize}
\item \textbf{Wallet Design Taxonomy:} We provide a taxonomy to analyse the design of various existing wallet types and propose new wallet designs. We also outline the threats to existing wallet designs based on our threat model.
\item \textbf{Wallet Attacks Framework:} We systematise and analyse various attack methods, techniques and targets in literature. We then analyse 85 notable wallet incidents between 2012 and 2025 and investigate the attack gaps between academia and industry.
\item \textbf{Defence Strategies:} We recommend defence methods based on the overall mitigation approach, incorporating both proactive and reactive approaches. We also analyse the influence of defence methods in mitigating attacks.
\end{itemize}

To facilitate independent verification, all datasets and code used in this study are publicly available.\footnote{GitHub repository at \url{https://github.com/xujiahuayz/crypto-wallets}.}


