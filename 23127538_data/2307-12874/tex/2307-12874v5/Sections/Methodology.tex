\section{Methodology}
\label{sec:methodology}

Our methodology systematically bridges academic research and industry practice by analysing cryptocurrency wallet security across four axes: design, vulnerabilities, attacks, and defence measures. 


% Specifically, we follow a structured analytical framework consisting of: 
% \begin{enumerate*}[label=(\arabic*)]
%   \item \textit{Design and Threat Analysis}, 
%   \item \textit{Attack Methods}, 
%   \item \textit{Security Measures} and 
%   \item \textit{Case Studies}.
% \end{enumerate*}

\subsection{Procedure}


\subsubsection{Design Taxonomy and Vulnerability}

Our wallet design survey is structured as follows. We first perform reverse-engineering of specific vulnerable wallets to map vulnerabilities explicitly to underlying design features. Unique wallet features relevant to security were carefully documented and compared across wallet categories. The results of this analysis are summarised systematically in our wallet taxonomy table (\autoref{tab:wlt._taxonomy}), enabling structured comparisons and insight into security-usability trade-offs.

\subsubsection{Attack Methods}

Following our design and threat analysis, we examine wallet attack methods in both academia and industry through a three-phase process. First, we conduct a comprehensive review of academic literature and industry incidents. We examine 33 peer-reviewed papers alongside 85 real-world incidents (2012–2025) documented in grey literature sources such as \href{https://rekt.news/}{Rekt News} and \href{https://www.slowmist.com/}{Slowmist}. 


To expand the reviewed literature scope, we conduct forward and backward reference searches. Following this, we categorise attacks using a three-tier framework to establish clarity and consistency. Attacks are classified hierarchically by their mechanism-centric goal (e.g. bypass the authentication mechanism), method (e.g., credential cracking), and vector (e.g., dictionary attack). We analyse industry incidents and identify patterns related to our design taxonomy or attack categorisation. Lastly, we perform a gap analysis to evaluate the alignment between academic research and industry practices.

\subsubsection{Security Measures}

Our security measures analysis begins by identifying proposed and implemented defensive strategies documented within the 33 academic papers focused on attack methods. We employ forward and backward reference searches to expand the scope of our reviewed literature to 61 unique references, retrieving an additional 28 academic papers. In addition, we consult grey literature sources on security measures. Each security measure is mapped to an identified wallet attack vector and classified based on the approach (e.g. proactive or reactive). 

\subsubsection{Case Studies}
We conduct in-depth case studies to illustrate the practical application of our framework. We systematically select representative wallet incidents based on severity and distinctiveness. Each case study follows a structured approach: 
\begin{enumerate*}[label=(\arabic*)]
  \item describing the wallet’s design using our taxonomy, 
  \item detailing exploited vulnerabilities and threats, 
  \item outlining the adversary’s goals, capabilities, and attack sequences and 
  \item recommending security measures.
\end{enumerate*}
By integrating these real-world examples, we provide actionable insights into the interplay of wallet design, threats, and mitigation strategies.


\begin{algorithm}[!b]
    \caption{Transaction broadcast}
    \label{algo:transaction-broadcast}
    \begin{algorithmic}[1]  % This enables line numbering
        \State \textbf{Input:} \textcolor{teal}{$\sigma$}: \textcolor{olive}{bytes}, \textcolor{teal}{\textit{pk}}: \textcolor{olive}{hex}

        % \Comment{\textcolor{gray}{// Broadcast the transaction to network for verification}}
        \State \textcolor{teal}{\textit{verified}} = \textcolor{orange}{\textit{txnVer}}(\textcolor{teal}{$\sigma$}, \textcolor{teal}{\textit{pk}})
        \State \textcolor{orange}{\textit{assert}}(\textcolor{teal}{\textit{verified}}, \enquote{transaction failed})
        \State \textcolor{orange}{\textit{broadcast}}(\textcolor{teal}{$\sigma$}, \textcolor{teal}{\textit{pk}})

    \end{algorithmic}
\end{algorithm}

\subsection{Data Sources}

We sourced design variation, vulnerability, attacks and defence methods data from the following:

\begin{itemize}
    \item \textbf{\acs{cve} Database:} We query the \href{https://www.cve.org/}{\acf{cve}} databases to retrieve previously identified wallet vulnerabilities.
    \item \textbf{Academic Papers:} We systematically retrieve academic papers, which serve as the primary data source for a range of wallet attack vectors and defence implementations.
    \item \textbf{Grey Literature:} We discover incidents on custodial and non-custodial wallets between 2012 and 2025, with most sources from \href{https://rekt.news/}{Rekt News} and \href{https://www.slowmist.com/}{Slowmist}. Grey literature is also employed to retrieve additional vulnerabilities and security measures.
\end{itemize}

% Figure environment removed

% \pagebreak

\subsection{Inclusion Criteria}

Our resulting data conformed to the inclusion criteria below:

\begin{itemize}
    \item 
    \textbf{General Scope:} We limit our scope to exclude attacks on the blockchain protocol and on \ac{defi} protocols from our discussion or analysis.
    \item 
    \textbf{Vulnerability Inclusion:} We include wallet solutions with at least one \href{https://www.cve.org/}{\acs{cve}} or previously detected vulnerability from searches.
    \item \textbf{Design Inclusion:} We include wallets with previously identified vulnerabilities, as well as those with significant user bases (such as MetaMask, Trust Wallet) or \acf{aum} (centralised exchanges such as Coinbase Exchange, Binance Exchange) and wallets with novel features (Argent, Safe (previously Gnosis Safe), ZenGo and Ngrave).
    \item \textbf{Attack and Defence Inclusion:} We include only attack methods and defence implementations, which can be mapped to key components within the underlying mechanism.
    \item \textbf{Case Studies Inclusion:} We include two notable incidents exhibiting: one custodial breach with the largest recorded monetary loss and one non‑custodial compromise affecting the widest user base. These also provide comparative coverage of attacks against smart contract, hardware, and mobile wallet infrastructures.
\end{itemize}


