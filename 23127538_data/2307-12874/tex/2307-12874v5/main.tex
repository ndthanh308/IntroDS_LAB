% \documentclass[conference,compsoc]{IEEEtran}

\documentclass[a4paper,fleqn]{cas-dc}

% \usepackage[utf8]{inputenc} % allow utf-8 input
%\usepackage[T1]{fontenc}    % use 8-bit T1 fonts
%\usepackage{hyperref}       % hyperlinks
\usepackage{url}            % simple URL typesetting
\usepackage{booktabs}       % professional-quality tables
\usepackage{multirow}    
\usepackage{amsfonts}       % blackboard math symbols
\usepackage{nicefrac}       % compact symbols for 1/2, etc.
\usepackage{microtype}      % microtypogrhy
% \usepackage{natbib}
\usepackage{enumerate}
%\usepackage{enumitem}
\usepackage{hhline}
\usepackage{makecell}
\usepackage{pifont}

% use Times
%\usepackage{times}
% For figures
\usepackage{graphicx} % more modern
%\usepackage{epsfig} % less modern
%\usepackage{subfigure}
\usepackage{caption}
\usepackage{subcaption}
% For citations
\usepackage{amsmath}
\usepackage{amsthm}
\usepackage{amssymb}
\usepackage{tikz}
\usepackage{xcolor}
\usetikzlibrary{arrows}

\allowdisplaybreaks

%for fonts
\usepackage{mathrsfs}

% For algorithms
\usepackage{algorithm}
\usepackage{algorithmic}
% \usepackage{algpseudocode}
% \usepackage[noend]{algpseudocode}
\usepackage{hyperref}
\usepackage{bm}
%\usepackage{todonotes}

%For theorems
\allowdisplaybreaks

%for convinience
\newcommand{\RR}{\mathbb{R}}
\newcommand{\vct}{\boldsymbol }
%\newcommand{\mat}{\mathbf}
\newcommand{\rnd}{\mathsf}
\newcommand{\ud}{\mathrm d}
\newcommand{\nml}{\mathcal{N}}
\newcommand{\loss}{\mathcal{L}}
\newcommand{\hinge}{\mathcal{R}}
\newcommand{\kl}{\mathrm{KL}}
\newcommand{\cov}{\mathrm{cov}}
\newcommand{\dir}{\mathrm{Dir}}
\newcommand{\mult}{\mathrm{Mult}}
\newcommand{\err}{\mathrm{err}}
\newcommand{\sgn}{\mathrm{sgn}}
%\renewcommand{\span}{\mathrm{span}}
% \newcommand{\argmin}{\mathrm{argmin}}
% \newcommand{\argmax}{\mathrm{argmax}}
\newcommand{\poly}{\mathrm{poly}}
% \newcommand{\rank}{\mathrm{rank}}
% \newcommand{\conv}{\mathrm{conv}}
%\newcommand{\E}{\mathbb{E}}
% \newcommand{\diag}{\mat{diag}}
\newcommand{\acc}{\mathrm{acc}}

\newcommand{\labs}{\left\vert}
\newcommand{\rabs}{\right\vert}
\newcommand{\lnorm}{\left\Vert}
\newcommand{\rnorm}{\right\Vert}

\newcommand{\aff}{\mathrm{aff}}
% \newcommand{\range}{\mathrm{Range}}
\newcommand{\Sgn}{\mathrm{sign}}

\newcommand{\hit}{\mathrm{hit}}
\newcommand{\cross}{\mathrm{cross}}
\newcommand{\Left}{\mathrm{left}}
\newcommand{\Right}{\mathrm{right}}
\newcommand{\Mid}{\mathrm{mid}}
\newcommand{\bern}{\mathrm{Bernoulli}}
\newcommand{\ols}{\mathrm{ols}}
\newcommand{\tr}{\operatorname{tr}}
\newcommand{\opt}{\mathrm{opt}}
%\newcommand{\ridge}{\mathrm{ridge}}
\newcommand{\unif}{\mathrm{Unif}}
\newcommand{\Image}{\mathrm{im}}
\newcommand{\Kernel}{\mathrm{ker}}
\newcommand{\supp}{\mathrm{supp}}
\newcommand{\pred}{\mathrm{pred}}
\newcommand{\distequal}{\stackrel{\mathbf{P}}{=}}
%\newcommand{\gege}{\textcircled{1}}
\newcommand{\gege}{{A(\vect{w},\vect{w}_*)}}
\newcommand{\gele}{{A(\vect{w},-\vect{w}_*)}}
\newcommand{\lele}{{A(-\vect{w},-\vect{w}_*)}}
\newcommand{\lege}{{A(-\vect{w},\vect{w}_*)}}
\newcommand{\firstlayer}{\mathbf{W}}
\newcommand{\firstlayerWN}{v}
\newcommand{\secondlayer}{a}
\newcommand{\inputvar}{\vect{x}}
\newcommand{\anglemat}{\mathbf{\Phi}}
\newcommand{\holder}{H\"{o}lder }
\newcommand{\real}{\mathbb{R}}
\newcommand{\approxerr}{\delta}

\def\R{\mathbb{R}}
\def\Z{\mathbb{Z}}
\def\cA{\mathcal{A}}
\def\cB{\mathcal{B}}
\def\cD{\mathcal{D}}
\def\cE{\mathcal{E}}
\def\cF{\mathcal{F}}
\def\cG{\mathcal{G}}
\def\cH{\mathcal{H}}
\def\cS{\mathcal{S}}
\def\cI{\mathcal{I}}
\def\cL{\mathcal{L}}
\def\cM{\mathcal{M}}
\def\cN{\mathcal{N}}
\def\cP{\mathcal{P}}
\def\cS{\mathcal{S}}
\def\cT{\mathcal{T}}
\def\cV{\mathcal{V}}
\def\cW{\mathcal{W}}
\def\cZ{\mathcal{Z}}
\def\SS{\mathbb{S}}
\def\NN{\mathbb{N}}
\def\bP{\mathbf{P}}
\def\TV{\mathrm{TV}}
\def\MSE{\mathrm{MSE}}

\def\vw{\mathbf{w}}
\def\va{\mathbf{a}}
\def\vZ{\mathbf{Z}}

\newcommand{\mat}[1]{#1}
\newcommand{\vect}[1]{#1}
\newcommand{\norm}[1]{\left\|#1\right\|}
\newcommand{\normop}[1]{\left\|#1\right\|_{\mathrm{op}}}
\newcommand{\simplex}{\triangle}
\newcommand{\abs}[1]{\left|#1\right|}
\newcommand{\expect}{\mathbb{E}}
\newcommand{\prob}{\mathbb{P}}
\newcommand{\proj}{\gP}
% \newcommand{\prox}[2]{\textbf{Prox}_{#1}\left\{#2\right\}}
\newcommand{\event}[1]{\mathscr{#1}}
\newcommand{\set}[1]{#1}
\newcommand{\diff}{\text{d}}
\newcommand{\difference}{\triangle}
\newcommand{\inputdist}{\mathcal{Z}}
\newcommand{\indict}{\mathbb{I}}
\newcommand{\rotmat}{\mathbf{R}}
\newcommand{\normalize}[1]{\overline{#1}}
\newcommand{\vectorize}[1]{\text{vec}\left(#1\right)}
\newcommand{\vclass}{\mathcal{G}}
\newcommand{\pclass}{\Pi}
\newcommand{\qclass}{\mathcal{Q}}
\newcommand{\rclass}{\mathcal{R}}
\newcommand{\classComplexity}[2]{N_{class}(#1,#2)}
\newcommand{\cclass}{\mathcal{F}}
\newcommand{\gclass}{\mathcal{G}}
\newcommand{\pthres}{p_{thres}}
\newcommand{\ethres}{\epsilon_{thres}}
\newcommand{\eclass}{\epsilon_{class}}
\newcommand{\states}{\mathcal{S}}
\newcommand{\trans}{P}
\newcommand{\lowprobstate}{\psi}
\newcommand{\actions}{\mathcal{A}}
\newcommand{\contexts}{\mathcal{X}}
\newcommand{\edges}{\mathcal{E}}
\newcommand{\variance}{\text{Var}}
\newcommand{\params}{\vect{w}}

\newcommand{\relu}[1]{\sigma\left(#1\right)}
\newcommand{\reluder}[1]{\sigma'\left(#1\right)}
\newcommand{\act}[1]{\sigma\left(#1\right)}

\newtheorem{thm}{Theorem}
% \newtheorem{thm}{Theorem}
\newtheorem{lem}{Lemma}
% Thm -> corollary 
\newtheorem{cor}{Corollary}
\newtheorem{prop}{Proposition}
\newtheorem{asmp}{Assumption}
\newtheorem{defn}{Definition}
\newtheorem{oracle}{Oracle}
\newtheorem{fact}{Fact}
\newtheorem{conj}{Conjecture}
\newtheorem{rem}{Remark}
\newtheorem{example}{Example}
\newtheorem{condition}{Condition}
\newtheorem{exercise}{Exercise}
\newtheorem{mess}{Message}
\newtheorem{claim}{Claim}
\newtheorem{ec}{Empirical Conclusion}






\usepackage[capitalize,noabbrev]{cleveref}
% \usepackage{cleveref}
\crefname{thm}{Theorem}{Theorems}
\crefname{lem}{Lemma}{Lemmas}
\crefname{cor}{Corollary}{Corollaries}
\crefname{prop}{Proposition}{Propositions}
\crefname{asmp}{Assumption}{Assumptions}
\crefname{defn}{Definition}{Definitions}
\crefname{oracle}{Oracle}{Oracles}
\crefname{fact}{Fact}{Facts}
\crefname{conj}{Conjecture}{Conjectures}
\crefname{rem}{Remark}{Remarks}
\crefname{claim}{Claim}{Claims}
\crefname{ec}{Empirical Observation}{Empirical Observations}


\renewcommand{\algorithmicrequire}{\textbf{Input:}}
\renewcommand{\algorithmicensure}{\textbf{Output:}}


\definecolor{red}{rgb}{1, 0, 0}
\newcommand{\RED}[1]{{\color{red} #1}}

\definecolor{green}{rgb}{0, 1, 0}
\definecolor{darkgreen}{rgb}{0.0, 0.2, 0.13}
\definecolor{darkseagreen}{rgb}{0.56, 0.74, 0.56}
\definecolor{officegreen}{rgb}{0.0, 0.5, 0.0}


\newcommand{\GREEN}[1]{{\color{green} #1}}

\definecolor{blue}{rgb}{0, 0, 1}
\newcommand{\BLUE}[1]{{\color{blue} #1}}

\definecolor{orange}{rgb}{1, 0.4, 0.0}
\newcommand{\ORANGE}[1]{{\color{orange} #1}}



% \journal{Computer Networks}

% \date{}

\begin{acronym}
    \acro{SSIM}{Structural Similarity}
    \acro{CNN}{Convolutional Neural Network}
    \acro{BCE}{Binary Cross-Entropy}
    \acro{MSE}{Mean Squared Error}
    \acro{GAN}{Generative Adversarial Network}
    \acro{MI}{Mutual Information}
    \acro{EMD}{Earth Mover's Distance}
    \acro{AUROC}{Area Under the Receiver Operator Characteristic Curve}
    \acro{RMSE}{Root Mean Square Error}
    \acro{ETCS}{European Train Control System}
    \acro{AE}{Auto-Encoder}
    \acro{Lidar}{Light detection and ranging}
\end{acronym}

\begin{document}
\let\WriteBookmarks\relax
\def\floatpagepagefraction{1}
\def\textpagefraction{.001}

% \begin{frontmatter}

%don't want date printed
% \date{}

\shorttitle{SoK: Design, Vulnerabilities, and Security Measures of Cryptocurrency Wallets}   

\shortauthors{Erinle et~al.}


\title [mode = title]{SoK: Design, Vulnerabilities, and Security Measures of Cryptocurrency Wallets} 

\author[1,2]{Yimika Erinle}
[
% type=editor,
%                         auid=000,bioid=1,    role=Researcher,
                        orcid=0009-0003-9151-9114          
                        ]
% \cormark[1]
% \fnmark[1]
% \ead{jkk@example.in}
% \ead[url]{www.jkkrishnan.in}

\credit{Conceptualization of this study, Methodology, Software}

%\address[1]{, Street 129, 1043 NX Amsterdam, The Netherlands}
\affiliation[1]{organization={
DLT Science Foundation},
% Department of Physics, J.K. Institute of Science},
                % addressline=
                % {Jawahar Nagar}, 
                % city={London},
              citysep={}, % Uncomment if no comma needed between city and postcode
                postcode={WC2H 2JQ}, 
                state={London},
                country={United Kingdom}
                }
                
\author[3]{Yathin Kethepalli}
[
% type=editor,
%                         auid=000,bioid=1,    role=Researcher,
                        orcid=0000-0002-1018-5251          
                        ]
% [style=chinese]


\author[4]{Yebo Feng}
[
% type=editor,
%                         auid=000,bioid=1,    role=Researcher,
                        orcid=0000-0002-7235-2377          
                        ]

% [%
%    role=Co-ordinator,
%    suffix=Jr,
%    ]
% \fnmark[2]
% \ead{wjh@example.org}
% \ead[URL]{https://www.university.org}

\credit{Data curation, Writing - Original draft preparation}

\affiliation[2]{organization={Centre for Blockchain Technologies, University College London},
% ,
%                 addressline={Street 29}, 
                postcode={WCIE 6BT}, 
                postcodesep={}, 
                city={London},
                country={United Kingdom}      
                }

\author[1,2]{Jiahua Xu}
[
% type=editor,
%                         auid=000,bioid=1,    role=Researcher,
                        orcid=0000-0002-3993-5263          
                        ]

% \cormark[2]
% \fnmark[1,3]
% \ead{t.rafeeq@example.in}
% \ead[URL]{www.campus.in}

\affiliation[3]{organization={GlueX Protocol},
% ,
%                 addressline={Street 15}, 
%                 city={Jabaldesh},
                postcode={2595VG}, 
                state={The Hague}, 
                country={Netherlands}
                }

\affiliation[4]{organization={Nanyang Technological University},
% ,
%                 addressline={Street 15}, 
%                 city={Jabaldesh},
                postcode={639798}, 
%                 state={Orissa}, 
                country={Singapore}
                }

% \title{SoK: Design, Vulnerabilities, and Security Measures of Cryptocurrency Wallets}


% \author{}

% \author{Yimika Erinle $^{1,2}$*, Yathin Kethepalli $^{3}$, Yebo Feng $^{1,2}$ and Jiahua Xu $^{1,2}$}

% \affiliation{organization={},%Department and Organization
%             addressline={}, 
%             city={},
%             postcode={}, 
%             state={},
%             country={}}

% \title{\Large \bf SoK: Design, Vulnerabilities, and Security Measures of Cryptocurrency Wallets}


% \author{
% \IEEEauthorblockN{
% Yimika Erinle\IEEEauthorrefmark{5} \IEEEauthorrefmark{4}  
% Yathin Kethepalli\IEEEauthorrefmark{2} \IEEEauthorrefmark{4}
% Yebo Feng\IEEEauthorrefmark{3}, 
% and 
% Jiahua Xu\IEEEauthorrefmark{5} \IEEEauthorrefmark{4}
% }

% \IEEEauthorblockA{
% \IEEEauthorrefmark{5}University College London,
% \IEEEauthorrefmark{2}GlueX Protocol,
% \IEEEauthorrefmark{3}Nanyang Technological University,
% \IEEEauthorrefmark{4}DLT Science Foundation
% }
% }

\begin{abstract}
With the advent of decentralised digital currencies powered by blockchain technology, a new era of peer-to-peer transactions has commenced. The rapid growth of the cryptocurrency economy has led to the increased use of transaction-enabling wallets, making them a focal point for security risks. As the frequency of wallet-related incidents rises, there is a critical need for a systematic approach to measure and evaluate these attacks, drawing lessons from past incidents to enhance wallet security.

In response, we introduce a multi-dimensional design taxonomy for legacy and emerging wallets. We classify existing industry wallets based on this taxonomy, identify previously occurring vulnerabilities and discuss the security implications of design decisions. We also systematise threats to the wallet mechanism and analyse the adversary's goals, capabilities and required knowledge. We present a multi-layered attack framework and investigate 85 incidents between 2012 and 2025, accounting for a total of \$6.98B. Following this, we classify defence implementations for these attacks on the precautionary and remedial axes. We map the mechanism and design decisions to vulnerabilities, attacks, and possible defence methods to discuss various insights. 

% \noindent\texttt{\textbackslash begin{abstract}} \dots 
% \texttt{\textbackslash end{abstract}} and
% \verb+\begin{keyword}+ \verb+...+ \verb+\end{keyword}+ 
% which
% contain the abstract and keywords respectively. 

% \noindent Each keyword shall be separated by a \verb+\sep+ command.

  % \label{sec:abstract}
With the advent of decentralised digital currencies powered by blockchain technology, a new era of peer-to-peer transactions has commenced. The rapid growth of the cryptocurrency economy has led to increased use of transaction-enabling wallets, making them a focal point for security risks. As the frequency of wallet-related incidents rises, there is a critical need for a systematic approach to measure and evaluate these attacks, drawing lessons from past incidents to enhance wallet security.

In response, we introduce a multi-dimensional design taxonomy for existing and novel wallets with various design decisions. We classify existing industry wallets based on this taxonomy, identify previously occurring vulnerabilities and discuss the security implications of design decisions. We also systematise threats to the wallet mechanism and analyse the adversary's goals, capabilities and required knowledge. We present a multi-layered attack framework and investigate 84 incidents between 2012 and 2024, accounting for \$5.4B. Following this, we classify defence implementations for these attacks on the precautionary and remedial axes. We map the mechanism and design decisions to vulnerabilities, attacks, and possible defence methods to discuss various insights. 

\end{abstract}


\begin{keywords}
Cryptocurrency Wallet \sep Attacks \sep Defences \sep Key Management \sep Wallet Security \sep Wallet Design
\end{keywords}

\maketitle
% \tableofcontents

% \end{frontmatter}

% \begin{IEEEkeywords}
% Cryptocurrency Wallet, Attacks, Defences, Key Management, Wallet Security, Wallet Design.
% \end{IEEEkeywords}


The problem of the presence or absence of phase transition is central in statistical mechanics. To prove the existence of phase transition, the standard idea is to define a notion of contour and use \textit{Peierls' argument} \cite{Peierls.1936}. In the usual Ising model \cite{Ising_25}, particles of the system interact only with their nearest-neighbors. On ferromagnetic long-range Ising models \cite{Anderson_Yuval_69}, there is interaction between each pair of spins in the lattice. The Hamiltonian of the model is given formally by
\begin{equation*}
    H(\sigma) = - \sum_{x,y\in \Z^d}J_{xy}\sigma_x\sigma_y,
\end{equation*}
where $J_{xy}=J|x-y|^{-\alpha}$, $J>0$, $\alpha > d$. It is well-known that the Peierls' argument in dimension 2 implies phase transition for Ising models with nearest-neighbors or long-range interactions when $d\geq 2$, using correlation inequalities. For the unidimensional lattice, it was known that short-range models do not present phase transition. In the long-range case, a different behavior was expected depending on the exponent $\alpha$ (see \cite{Kac_Thompson_69}), but the problem was challenging since contours were first created as multidimensional objects.

In dimension $d=1$, phase transition was proved first in 1969 by Dyson \cite{Dyson.69}, for $\alpha \in (1,2)$, by proving phase transition in an auxiliary model and then using correlation inequalities. In 1982, Fr{\"o}hlich and Spencer \cite{Frohlich.Spencer.82} introduced a notion of one-dimensional contours and then applied the Peierls' argument to show phase transition for the critical value $\alpha = 2$. These contours were inspired by the multiscale techniques previously introduced to study the Berezinskii-Kosterlitz-Thouless transition in two-dimensional continuous spin systems \cite{FS81}. Later, Cassandro, Ferrari, Merola and Presutti  \cite{Cassandro.05} extended the contour argument previously available for $\alpha=2$ to exponents $\alpha\in (3-\frac{\ln 3}{\ln 2}, 2)$, with the additional restriction that the nearest-neighbor interaction is strong, i.e.,  ${J(1)\gg 1}$; this restriction was removed for a subclass of interactions in \cite{Bissacot.Endo.18}. Further results were obtained using contour arguments, such as the decay of correlations, cluster expansions, phase transition with random interactions, etc; some references with these results are \cite{ Cassandro.Merola.Picco.17, Cassandro.Merola.Picco.Rozikov.14, Imbrie.82, Imbrie.Newman.88, Johansson.91}. 

In the multidimensional setting ($d\geq 2$), Ginibre, Grossmann, and Ruelle, in \cite{Ginibre.Grossmann.Ruelle.66}, proved the phase transition for $\alpha > d+1$, using an enhanced version of Peierls' argument and the usual contours. Park proposed a different notion of contour for long-range systems in \cite{Park.88.I, Park.88.II}, extending the Pirogov-Sinai theory available for short-range interactions assuming $\alpha > 3d+1$, although he can also consider Potts models with his methods. Some results in the literature suggest that truly long-range effects appear only when $d < \alpha \leq d+1$, see for instance, \cite{Biskup_Chayes_Kivelson_07}. Recently, Affonso, Bissacot, Endo and Handa \cite{Affonso.2021}, inspired by the ideas from Fr{\"o}hlich and Spencer in \cite{FS81, Frohlich.Spencer.82}, introduced a version of multiscale multidimensional contour and proved phase transition by a contour argument in the whole region $\alpha > d$. They can consider long-range Ising models with deterministic decaying fields, first introduced in the context of nearest-neighbor interactions in \cite{Bissacot_Cioletti_10}. For these models, the lack of analyticity of the free energy does not imply phase transition since these models have the same free energy as the models with zero field. It is expected that fields decaying slowly imply uniqueness. In this setting, a contour argument is useful for proofs of phase transitions as well for uniqueness, some papers with models with deterministic decaying fields are \cite{Aoun_Ott_Velenik_23, Bissacot_Cass_Cio_Pres_15, Bissacot.Endo.18, Cioletti_Vila_2016}.

The Random Field Ising model (RFIM) \cite{Imry.Ma.75} is the nearest-neighbor Ising model with an additional external field acting on each site $(h_x)_{x\in\Z^d}$ that is a family of i.i.d. Gaussian random variable with mean 0 and variance 1. Formally, the Hamiltonian of the model is given by
\begin{equation*}
    H(\sigma) = - \sum_{\substack{x,y\in \Z^d \\|x-y|=1}}J\sigma_x\sigma_y  - \varepsilon\sum_{x\in\Z^d}h_x\sigma_x,
\end{equation*}
where $J>0$, $\varepsilon>0$, $\alpha > d$ and $d \geq 1$. A detailed account of the history of the phase transition problem for this model, as well as detailed proofs, was given in \cite{Bovier.06}. Here we present a brief overview.

During the 1980s, the question of the specific dimension where phase transition for the RFIM should happen attracted much attention and was a topic of heated debate. Two convincing arguments were dividing the physics community. One of them, due to Imry and Ma \cite{Imry.Ma.75}, was a non-rigorous application of the Peierls' argument together with the use of the isoperimetric inequality. The key idea of Peierls' argument is to define a notion of contour and calculate the energy cost of "erasing" each contour, i.e., the energy cost of flipping all spins inside the contour. When there is no external field, that energy necessary to flip the spins in a region $A\subset \Z^d$ is of the order of the boundary $|\partial A|$. When we add an external field, we get an extra cost depending on this field. Imry and Ma argued that this cost should be approximately $\sqrt{|A|}$, which is smaller than $|\partial A|$ for all regions only when $d\geq 3$, so this should be the region where phase transition occurs. The other argument, due to Parisi and Sourlas \cite{Parisi.Sourlas.79}, based on dimensional reduction, predicted that the $d$-dimensional RFIM would behave like the $d-2$-dimensional nearest-neighbor Ising model, therefore presenting phase transition only when $d\geq 4$. 

The question was settled by two celebrated papers showing that Imry and Ma's prediction was correct. First, in 1988, Bricmont and Kupiainen \cite{Bricmont.Kupiainen.88} showed that there is phase transition almost surely in $d\geq3$, for low temperatures and variance $\varepsilon$ small enough. Their proof uses a rigorous renormalization group analysis for the short-range case and it is considered involved. Still, they claimed that the result works for any model with a suitable contour representation and centered sub-gaussian external field. Later on, Aizenman and Wehr \cite{Aizenman.Wehr.90} proved uniqueness for $d\leq 2$. For detailed proofs of these results, we refer the reader to \cite{Bovier.06} (see also \cite{Berretti.85, Camia.18, Frohlich.Imbre.84,  Klein.Masooman.97} for more uniqueness results). 

Recently, Ding and Zhuang, see \cite{Ding2021}, provided a simpler proof of the phase transition, not using RGM. And in  \cite{Ding.Liu.Xia.22}, Ding, Liu and Xia proved that if $\beta_c(d)$ is the critical inverse of the temperature of the Ising model with no field, for all $\beta>\beta_c(d)$ there exists a critical value $\varepsilon_0(d, \beta)$ such that the RFIM with $\varepsilon \leq \varepsilon_0$ presents phase transition. 

In the present paper, we are considering a long-range Ising model with a random field, whose Hamiltonian is given formally by
\begin{equation*}
    H(\sigma) = - \sum_{x,y\in \Z^d}J_{xy}\sigma_x\sigma_y - \varepsilon\sum_{x\in\Z^d}h_x\sigma_x,
\end{equation*}
where $J_{xy}=J|x-y|^{-\alpha}$, $J, \varepsilon>0$, $\alpha > d$ and $h_x\in\mathbb{R}$, $d\geq 3$.
Until now, the only known result in the long-range setting is for the one-dimensional long-range Ising model with a random field, by Cassandro, Orlandi, and Picco \cite{Cassandro.Picco.09}. They used the contours of \cite{Cassandro.05} to show the phase transition for the model when $\alpha\in (3-\frac{\ln 3}{\ln 2}, \frac{3}{2})$, under the assumption $J(1) \gg 1$. We stress that, as remarked by Aizenman, Greenblatt, and Lebowitz \cite{Aizenman_Greenblatt_Lebowitz_2012}, although their argument does not work for the whole region for the exponent $\alpha$, the phase transition holds for values close to the critical value $\alpha=3/2$, since by the Aizenman-Wehr theorem we know that there is uniqueness for $\alpha>3/2$.

The argument from Ding and Zhuang in \cite{Ding2021}, for $d\geq3$, involves controlling the probability of a bad event, which is closely related to controlling the quantity $$\sup_{\substack{0\in A\subset\Z^d \\ A \text{ connected }}}\frac{\sum_{x\in A}h_x}{|\partial A|},$$ known as the greedy animal lattice normalized by the boundary. The greedy animal lattice normalized by the size, instead of the boundary, was extensively studied for general distributions of $(h_x)_{x\in\Z^d}$, see \cite{Cox_Gandolfi_Griffin_Kesten_93, Gandolfi_Kesten_94, Hammond_06, Martin_02}. When we normalize by the boundary, an argument by Fisher, Fr\"{o}hlich and Spencer \cite{FFS84} shows that the expected value of the greedy animal lattice is constant. In dimension $d=2$, the expected value is not finite, see \cite{Ding.Wirth.20}. The supremum is taken over connected regions containing the origin since the interiors of the usual Peierls contours are of this form.


For the long-range model, the interior of contours is not necessarily connected. In fact, long-range contours may have considerably large diameters with respect to their size, so their interiors can be very sparse. To avoid this, we define contours, strongly inspired by the $(M,a,r)$-partition in \cite{Affonso.2021}, using a multiscaled procedure that assures that the contours have no cluster with small density.  With them, we generalize the arguments by Fisher-Fr\"{o}hlich-Spencer \cite{FFS84}, and prove that the expected value of the greedy animal lattice is constant, even considering regions not necessarily connected in the supremum. Then, we prove the phase transition for $d\geq 3$. The main result of this paper is the following.
\begin{theorem*}Given $d\geq 3$, $\alpha>d$, there exists $\beta_c\coloneqq\beta(d, \alpha)$ and $\varepsilon_c\coloneqq\varepsilon(d, \alpha)$ such that, for $\beta >\beta_c$ and $\varepsilon\leq \varepsilon_c$, the extremal Gibbs measures $\mu_{\beta, \varepsilon}^+$ and $\mu_{\beta, \varepsilon}^-$ are distinct, that is, $\mu_{\beta, \varepsilon}^+ \neq \mu_{\beta, \varepsilon}^-$ $\mathbb{P}$-almost surely. Therefore the long-range random field Ising model presents phase transition.
\end{theorem*}

This paper is divided as follows. In Section 2, we define the model and the contours, and suitable generalizations to the constructions in \cite{Affonso.2021} are introduced.  In Section 3, we define two bad events of the external field and prove that they occur with a small probability.  In Section 4, we present the proof of the phase transition.

\section{Related Works}
\label{sec:related-work}

% \autoref{Literature-Gap-Table-1} outlines studies investigating crypto wallets and related mechanisms. 

% \begin{table}[!ht]
  \centering
  \renewcommand{\arraystretch}{1.2}
  \resizebox{\linewidth}{!}{%
    \begin{tabular}{r*{6}{c}*{4}{c}*{4}{c}}
      \toprule
      & \multicolumn{6}{c}{\textbf{Subjects Covered}}
      & \multicolumn{4}{c}{\textbf{Methodology}}
      & \multicolumn{4}{c}{\textbf{Scope}}
      \\
      \cmidrule(lr){2-7} \cmidrule(lr){8-11} \cmidrule(lr){12-15}
      \textbf{Reference}
      & \rot[90]{Key Cryptography}
      & \rot[90]{Key Management}
      & \rot[90]{Key Recovery}
      & \rot[90]{Attack Methods}
      & \rot[90]{Security Measures}
      & \rot[90]{Privacy Techniques}
      & \rot[90]{Literature}
      & \rot[90]{Taxonomisation}
      & \rot[90]{Analysis}
      & \rot[90]{Case Study}
      & \rot[90]{Wallet Software}
      & \rot[90]{Wallet Hardware}
      & \rot[90]{Smart Contract Wallet}
      & \rot[90]{Blockchain Network} \\
      \midrule
      This Study
      & \CIRCLE & \CIRCLE & \CIRCLE & \CIRCLE & \CIRCLE & \Circle
      & \CIRCLE & \CIRCLE & \CIRCLE & \CIRCLE
      & \CIRCLE  & \CIRCLE  & \CIRCLE  & \Circle \\
      \cite{bonneau2015sok}
      & \CIRCLE & \CIRCLE & \Circle  & \Circle  & \CIRCLE & \CIRCLE
      & \CIRCLE & \CIRCLE & \CIRCLE & \Circle
      & \CIRCLE  & \Circle   & \Circle   & \CIRCLE \\
      \cite{eskandari2018first}
      & \Circle   & \CIRCLE  & \CIRCLE  & \Circle   & \CIRCLE  & \Circle
      & \CIRCLE  & \CIRCLE  & \CIRCLE  & \Circle
      & \CIRCLE  & \Circle   & \Circle   & \Circle \\
      \cite{karantias2020sok}
      & \Circle   & \Circle   & \Circle   & \Circle   & \CIRCLE  & \CIRCLE
      & \CIRCLE  & \CIRCLE  & \CIRCLE  & \Circle
      & \CIRCLE  & \Circle   & \Circle   & \Circle \\
      \cite{Homoliak2020SmartOTPs:Wallets}
      & \Circle   & \CIRCLE  & \Circle   & \CIRCLE  & \CIRCLE  & \Circle
      & \CIRCLE  & \CIRCLE  & \CIRCLE  & \Circle
      & \CIRCLE  & \CIRCLE  & \Circle   & \CIRCLE \\
      \cite{Houy2023}
      & \Circle   & \CIRCLE  & \CIRCLE  & \CIRCLE  & \CIRCLE  & \CIRCLE
      & \CIRCLE  & \CIRCLE  & \CIRCLE  & \Circle
      & \CIRCLE  & \CIRCLE  & \Circle   & \CIRCLE \\
      \cite{suratkar2020cryptocurrency}
      & \CIRCLE & \CIRCLE & \CIRCLE & \Circle  & \Circle  & \Circle
      & \CIRCLE & \CIRCLE & \Circle   & \Circle
      & \CIRCLE  & \Circle   & \Circle   & \Circle \\
      \cite{bui2019pitfalls}
      & \Circle  & \Circle  & \Circle  & \CIRCLE & \CIRCLE & \Circle
      & \CIRCLE & \CIRCLE & \CIRCLE  & \Circle
      & \CIRCLE  & \Circle   & \Circle   & \Circle \\
      \cite{zaghloul2020bitcoin}
      & \Circle  & \Circle  & \Circle  & \Circle  & \CIRCLE & \CIRCLE
      & \CIRCLE & \CIRCLE & \CIRCLE  & \Circle
      & \CIRCLE  & \Circle   & \Circle   & \CIRCLE \\
      \cite{li2020android}
      & \Circle  & \Circle  & \Circle  & \CIRCLE & \CIRCLE & \Circle
      & \CIRCLE & \Circle   & \CIRCLE  & \Circle
      & \CIRCLE  & \Circle   & \Circle   & \Circle \\
      \cite{Dai2018SBLWT:Trustzone}
      & \CIRCLE & \CIRCLE & \Circle  & \CIRCLE & \CIRCLE & \Circle
      & \Circle  & \CIRCLE  & \CIRCLE  & \Circle
      & \CIRCLE  & \Circle   & \Circle   & \Circle \\
      \cite{volety2019cracking}
      & \Circle  & \Circle  & \Circle  & \CIRCLE & \CIRCLE & \Circle
      & \CIRCLE & \Circle   & \CIRCLE  & \Circle
      & \CIRCLE  & \Circle   & \Circle   & \Circle \\
      \cite{8966739}
      & \CIRCLE & \CIRCLE & \CIRCLE & \CIRCLE & \CIRCLE & \CIRCLE
      & \CIRCLE & \Circle   & \CIRCLE  & \Circle
      & \Circle   & \CIRCLE   & \Circle   & \Circle \\
      \cite{rezaeighaleh2020improving}
      & \CIRCLE & \CIRCLE & \CIRCLE & \Circle  & \CIRCLE & \Circle
      & \CIRCLE & \CIRCLE & \CIRCLE  & \Circle
      & \Circle   & \CIRCLE   & \Circle   & \Circle \\
      \cite{Urien2021InnovativeWallets}
      & \Circle  & \Circle  & \Circle  & \CIRCLE & \CIRCLE & \Circle
      & \CIRCLE & \CIRCLE & \Circle   & \Circle
      & \Circle   & \CIRCLE   & \Circle   & \Circle \\
      \cite{Rezaeighaleh2020MultilayeredWallet}
      & \Circle  & \Circle  & \CIRCLE & \Circle  & \Circle  & \CIRCLE
      & \CIRCLE & \CIRCLE & \Circle   & \Circle
      & \Circle   & \CIRCLE   & \Circle   & \Circle \\
      \cite{rezaeighaleh2019new}
      & \CIRCLE & \CIRCLE & \CIRCLE & \Circle  & \CIRCLE & \Circle
      & \CIRCLE & \Circle   & \CIRCLE  & \Circle
      & \Circle   & \CIRCLE   & \Circle   & \Circle \\
      \cite{di2020characteristics}
      & \Circle  & \Circle  & \Circle  & \Circle  & \Circle  & \Circle
      & \CIRCLE & \CIRCLE & \CIRCLE  & \Circle
      & \Circle   & \Circle    & \CIRCLE   & \Circle \\
% ----------- VERIFIED & CORRECTED ROWS ----------------------------------
\cite{Homoliak2024SoK:Factors}
& \Circle & \CIRCLE & \Circle & \Circle & \CIRCLE & \Circle
& \CIRCLE & \CIRCLE & \CIRCLE & \Circle
& \CIRCLE & \CIRCLE & \Circle & \Circle \\

\cite{andryukhin2019phishing}
& \Circle & \Circle & \Circle & \CIRCLE & \CIRCLE & \Circle
& \CIRCLE & \Circle & \CIRCLE & \Circle
& \Circle & \Circle & \Circle & \Circle \\

\cite{Chen2020ADefenses}
& \Circle & \Circle & \Circle & \CIRCLE & \CIRCLE & \Circle
& \CIRCLE & \CIRCLE & \CIRCLE & \Circle
& \Circle & \Circle & \Circle & \CIRCLE \\

\cite{Courtois2017StealthSystems}
& \CIRCLE & \CIRCLE & \Circle & \CIRCLE & \CIRCLE & \CIRCLE
& \CIRCLE & \Circle & \CIRCLE & \Circle
& \CIRCLE & \Circle & \Circle & \CIRCLE \\

\cite{Das2019AWallets}
& \CIRCLE & \CIRCLE & \Circle & \Circle & \CIRCLE & \Circle
& \Circle & \Circle & \CIRCLE & \Circle
& \CIRCLE & \Circle & \Circle & \Circle \\

\cite{Eyal2022OnDesign}
& \Circle & \CIRCLE & \Circle & \Circle & \CIRCLE & \Circle
& \Circle & \CIRCLE & \CIRCLE & \Circle
& \CIRCLE & \CIRCLE & \Circle & \Circle \\

\cite{Gotte2021TechAttacks}
& \CIRCLE & \CIRCLE & \Circle & \CIRCLE & \CIRCLE & \Circle
& \Circle & \Circle & \CIRCLE & \Circle
& \Circle & \CIRCLE & \Circle & \Circle \\

\cite{Guo2022ASecurity}
& \Circle & \CIRCLE & \Circle & \CIRCLE & \CIRCLE & \Circle
& \CIRCLE & \CIRCLE & \CIRCLE & \Circle
& \CIRCLE & \CIRCLE & \Circle & \CIRCLE \\

\cite{He2018AScheme}
& \Circle & \CIRCLE & \CIRCLE & \Circle & \CIRCLE & \Circle
& \Circle & \Circle & \CIRCLE & \Circle
& \CIRCLE & \Circle & \Circle & \Circle \\

\cite{Houy2023}
& \CIRCLE & \CIRCLE & \CIRCLE & \CIRCLE & \CIRCLE & \Circle
& \CIRCLE & \CIRCLE & \CIRCLE & \Circle
& \CIRCLE & \CIRCLE & \Circle & \CIRCLE \\

\cite{Li2020ASystems}
& \Circle & \CIRCLE & \Circle & \CIRCLE & \CIRCLE & \Circle
& \Circle & \Circle & \CIRCLE & \Circle
& \CIRCLE & \Circle & \Circle & \Circle \\

\cite{Mangipudi2023UncoveringCrypto-Wallets}
& \Circle & \CIRCLE & \Circle & \Circle & \Circle & \Circle
& \Circle & \Circle & \CIRCLE & \CIRCLE
& \CIRCLE & \Circle & \Circle & \Circle \\

\cite{Shbair2021HSM-basedBlockchain}
& \CIRCLE & \CIRCLE & \Circle & \CIRCLE & \CIRCLE & \Circle
& \Circle & \Circle & \CIRCLE & \CIRCLE
& \Circle & \CIRCLE & \Circle & \Circle \\

\cite{zhou2023sok}
& \Circle & \Circle & \Circle & \CIRCLE & \CIRCLE & \Circle
& \CIRCLE & \CIRCLE & \CIRCLE & \Circle
& \Circle & \Circle & \CIRCLE & \CIRCLE \\

\cite{chatzigiannis2025composability}
& \Circle & \CIRCLE & \CIRCLE & \CIRCLE & \CIRCLE & \Circle
& \Circle & \Circle & \CIRCLE & \CIRCLE
& \CIRCLE & \Circle & \CIRCLE & \Circle \\

% ------------------------------------------------------------------------

% -------------------------------------------------------------------------

    % \cite{Homoliak2024SoK:Factors}
    %   & \Circle  & \Circle  & \Circle  & \Circle  & \Circle  & \Circle
    %   & \Circle & \Circle & \Circle  & \Circle
    %   & \Circle   & \Circle    & \Circle   & \Circle \\
    %       \cite{andryukhin2019phishing}
    %   & \Circle  & \Circle  & \Circle  & \Circle  & \Circle  & \Circle
    %   & \Circle & \Circle & \Circle  & \Circle
    %   & \Circle   & \Circle    & \Circle   & \Circle \\
      
    %       \cite{Chen2020ADefenses}
    %   & \Circle  & \Circle  & \Circle  & \Circle  & \Circle  & \Circle
    %   & \Circle & \Circle & \Circle  & \Circle
    %   & \Circle   & \Circle    & \Circle   & \Circle \\
    %       \cite{Courtois2017StealthSystems}
    %   & \Circle  & \Circle  & \Circle  & \Circle  & \Circle  & \Circle
    %   & \Circle & \Circle & \Circle  & \Circle
    %   & \Circle   & \Circle    & \Circle   & \Circle \\
      
    %       \cite{Das2019AWallets}
    %   & \Circle  & \Circle  & \Circle  & \Circle  & \Circle  & \Circle
    %   & \Circle & \Circle & \Circle  & \Circle
    %   & \Circle   & \Circle    & \Circle   & \Circle \\
      
    %       \cite{Eyal2022OnDesign}
    %   & \Circle  & \Circle  & \Circle  & \Circle  & \Circle  & \Circle
    %   & \Circle & \Circle & \Circle  & \Circle
    %   & \Circle   & \Circle    & \Circle   & \Circle \\
      
    %       \cite{Gotte2021TechAttacks}
    %   & \Circle  & \Circle  & \Circle  & \Circle  & \Circle  & \Circle
    %   & \Circle & \Circle & \Circle  & \Circle
    %   & \Circle   & \Circle    & \Circle   & \Circle \\
      
    %       \cite{Guo2022ASecurity}
    %   & \Circle  & \Circle  & \Circle  & \Circle  & \Circle  & \Circle
    %   & \Circle & \Circle & \Circle  & \Circle
    %   & \Circle   & \Circle    & \Circle   & \Circle \\
      
    %       \cite{He2018AScheme}
    %   & \Circle  & \Circle  & \Circle  & \Circle  & \Circle  & \Circle
    %   & \Circle & \Circle & \Circle  & \Circle
    %   & \Circle   & \Circle    & \Circle   & \Circle \\
      
    %       \cite{Houy2023}
    %   & \Circle  & \Circle  & \Circle  & \Circle  & \Circle  & \Circle
    %   & \Circle & \Circle & \Circle  & \Circle
    %   & \Circle   & \Circle    & \Circle   & \Circle \\
      
    %       \cite{Li2020ASystems}
    %   & \Circle  & \Circle  & \Circle  & \Circle  & \Circle  & \Circle
    %   & \Circle & \Circle & \Circle  & \Circle
    %   & \Circle   & \Circle    & \Circle   & \Circle \\
    %       \cite{Mangipudi2023UncoveringCrypto-Wallets}
    %   & \Circle  & \Circle  & \Circle  & \Circle  & \Circle  & \Circle
    %   & \Circle & \Circle & \Circle  & \Circle
    %   & \Circle   & \Circle    & \Circle   & \Circle \\
      
    %       \cite{Shbair2021HSM-basedBlockchain}
    %   & \Circle  & \Circle  & \Circle  & \Circle  & \Circle  & \Circle
    %   & \Circle & \Circle & \Circle  & \Circle
    %   & \Circle   & \Circle    & \Circle   & \Circle \\
      
    %       \cite{zhou2023sok}
    %   & \Circle  & \Circle  & \Circle  & \Circle  & \Circle  & \Circle
    %   & \Circle & \Circle & \Circle  & \Circle
    %   & \Circle   & \Circle    & \Circle   & \Circle \\
      \bottomrule
    \end{tabular}%
  }
  \caption{Overview of related works. (\CIRCLE: include, \Circle: not include)}
  \label{Literature-Gap-Table-1}
\end{table}


% To the best of our knowledge, our work is the first systematisation of knowledge focused on wallet attacks and defence methods. we outline how our work is differentiated from the existing one below.

\subsection{Key Management}
\label{sec:key-management}

Several studies have explored key management mechanisms. Courtois and Mercer \cite{Courtois2017StealthSystems} compare key management solutions with a focus on stealth addresses. Mangipudi et al. \cite{Mangipudi2023UncoveringCrypto-Wallets} investigate key management from the wallet users' perspective. He et al. \cite{He2018AScheme} propose a secure key management scheme based on semi-trusted social networks. Di Angelo and Salzer \cite{di2020characteristics} analyse the functionality of smart contracts for key management through transaction data. Our study differs by focusing on attacks and defence methods for key management mechanisms and wallet taxonomy.

% Additional studies have proposed key management designs. Khan et al. \cite{8966739} suggest using QR codes for transaction authentication between hardware and software wallets. Homoliak et al. \cite{homoliak2018air} propose a self-sovereign smart contract key management framework. Unlike these studies, we examine various key management designs and their security from a blockchain-agnostic viewpoint.

\subsection{Wallet Attack and Security}
\label{sec:wallet-security}

% zamani2020security Urien2021,

Various studies have analysed blockchain systems' security and vulnerabilities \cite{Li2020ASystems, Guo2022ASecurity, Chen2020ADefenses}. For instance, Chen et al. \cite{Chen2020ADefenses} focus on Ethereum's vulnerabilities and defence mechanisms. Our work differs by focusing on wallet security, categorised under external auxiliary services, rather than blockchain layers. The security of specific wallets has also been explored \cite{Shbair2021HSM-basedBlockchain, Gotte2021TechAttacks}. Götte and Scheuermann \cite{Gotte2021TechAttacks} propose defences for Hardware Security Modules against physical attacks. Our study takes a multi-layered approach (see \autoref{sec:defense-strategies}) to analyse a wide range of wallet attacks.

Specific attack vectors have been investigated as well \cite{andryukhin2019phishing, bui2019pitfalls}. Andryukhin \cite{andryukhin2019phishing} evaluates phishing attacks and proposes prevention mechanisms. Bui et al. \cite{bui2019pitfalls} examine security vulnerabilities in the \acs{rpc} of desktop wallets. Our work covers a broader scope of attacks compared to these studies. While some studies have explored security across various wallet types, the scope and depth vary. Das et al. \cite{Das2019AWallets} propose a security model for hot/cold wallets. Our research extends beyond hot/cold wallets, employing a detailed taxonomy and analysing operational mechanisms, bridging the gap between academia and industry. Eyal \cite{Eyal2022OnDesign} evaluates the impact of key management on wallet security. Houy et al. \cite{Houy2023} conduct a literature review of wallet attacks and defences, however, does not include theoretical or empirical evaluations.

\subsection{Addressing Literature Gaps}
\label{sec:gaps-in-literature}

Despite various studies on specific wallet types, mechanisms, and attack vectors, there is a lack of a comprehensive examination spanning wallet design taxonomy, mechanisms, attack analysis, and security measures. Our study bridges this gap, providing a holistic understanding crucial for advancing wallet security.

% Old Literature Review

% \autoref{Literature-Gap-Table-1} outlines studies that investigate crypto wallets and their related mechanisms. Our work is the first systematisation of knowledge focused on the attack and security of wallets.

% \subsection{Key Management}
% \label{sec:key-management}

% Various studies have investigated key management mechanisms and schemes. Courtois and Mercer \cite{courtois2017stealth} provide an overview and compare key management solutions, with a focus on stealth addresses. Mangipudi et al. \cite{mangipudi2022uncovering} investigate key management design from the perspective of wallet users. He et al \cite{he2018social} propose a highly effective and secure key management scheme based on semi-trusted social networks. In addition, Di Angelo and Salzer \cite{di2020characteristics} investigate the functionality of smart contracts for key management by analysing transaction data. We differ from these studies as we explore attacks and defence methods for key management mechanisms, as well as the wallet mechanisms and taxonomy.
 
%  Key management design has also been an area of interest in some studies. Khan et al. \cite{8966739} propose an architectural design for a Bitcoin wallet by utilising QR codes to authenticate transactions between hardware and software wallets. Homoliak et al \cite{homoliak2018air} propose a unique self-sovereign smart contract key management framework. We can also be differentiated from these studies by the broad spectrum we cover, which includes examining various types of key management designs and their related security. Other studies have surveyed key management solutions for Bitcoin \cite{eskandari2018first, pal2021key}. Our work is not restricted to any blockchain network and analyses key management methods and security from a blockchain-agnostic viewpoint.

% % \begin{table}[!ht]
  \centering
  \renewcommand{\arraystretch}{1.2}
  \resizebox{\linewidth}{!}{%
    \begin{tabular}{r*{6}{c}*{4}{c}*{4}{c}}
      \toprule
      & \multicolumn{6}{c}{\textbf{Subjects Covered}}
      & \multicolumn{4}{c}{\textbf{Methodology}}
      & \multicolumn{4}{c}{\textbf{Scope}}
      \\
      \cmidrule(lr){2-7} \cmidrule(lr){8-11} \cmidrule(lr){12-15}
      \textbf{Reference}
      & \rot[90]{Key Cryptography}
      & \rot[90]{Key Management}
      & \rot[90]{Key Recovery}
      & \rot[90]{Attack Methods}
      & \rot[90]{Security Measures}
      & \rot[90]{Privacy Techniques}
      & \rot[90]{Literature}
      & \rot[90]{Taxonomisation}
      & \rot[90]{Analysis}
      & \rot[90]{Case Study}
      & \rot[90]{Wallet Software}
      & \rot[90]{Wallet Hardware}
      & \rot[90]{Smart Contract Wallet}
      & \rot[90]{Blockchain Network} \\
      \midrule
      This Study
      & \CIRCLE & \CIRCLE & \CIRCLE & \CIRCLE & \CIRCLE & \Circle
      & \CIRCLE & \CIRCLE & \CIRCLE & \CIRCLE
      & \CIRCLE  & \CIRCLE  & \CIRCLE  & \Circle \\
      \cite{bonneau2015sok}
      & \CIRCLE & \CIRCLE & \Circle  & \Circle  & \CIRCLE & \CIRCLE
      & \CIRCLE & \CIRCLE & \CIRCLE & \Circle
      & \CIRCLE  & \Circle   & \Circle   & \CIRCLE \\
      \cite{eskandari2018first}
      & \Circle   & \CIRCLE  & \CIRCLE  & \Circle   & \CIRCLE  & \Circle
      & \CIRCLE  & \CIRCLE  & \CIRCLE  & \Circle
      & \CIRCLE  & \Circle   & \Circle   & \Circle \\
      \cite{karantias2020sok}
      & \Circle   & \Circle   & \Circle   & \Circle   & \CIRCLE  & \CIRCLE
      & \CIRCLE  & \CIRCLE  & \CIRCLE  & \Circle
      & \CIRCLE  & \Circle   & \Circle   & \Circle \\
      \cite{Homoliak2020SmartOTPs:Wallets}
      & \Circle   & \CIRCLE  & \Circle   & \CIRCLE  & \CIRCLE  & \Circle
      & \CIRCLE  & \CIRCLE  & \CIRCLE  & \Circle
      & \CIRCLE  & \CIRCLE  & \Circle   & \CIRCLE \\
      \cite{Houy2023}
      & \Circle   & \CIRCLE  & \CIRCLE  & \CIRCLE  & \CIRCLE  & \CIRCLE
      & \CIRCLE  & \CIRCLE  & \CIRCLE  & \Circle
      & \CIRCLE  & \CIRCLE  & \Circle   & \CIRCLE \\
      \cite{suratkar2020cryptocurrency}
      & \CIRCLE & \CIRCLE & \CIRCLE & \Circle  & \Circle  & \Circle
      & \CIRCLE & \CIRCLE & \Circle   & \Circle
      & \CIRCLE  & \Circle   & \Circle   & \Circle \\
      \cite{bui2019pitfalls}
      & \Circle  & \Circle  & \Circle  & \CIRCLE & \CIRCLE & \Circle
      & \CIRCLE & \CIRCLE & \CIRCLE  & \Circle
      & \CIRCLE  & \Circle   & \Circle   & \Circle \\
      \cite{zaghloul2020bitcoin}
      & \Circle  & \Circle  & \Circle  & \Circle  & \CIRCLE & \CIRCLE
      & \CIRCLE & \CIRCLE & \CIRCLE  & \Circle
      & \CIRCLE  & \Circle   & \Circle   & \CIRCLE \\
      \cite{li2020android}
      & \Circle  & \Circle  & \Circle  & \CIRCLE & \CIRCLE & \Circle
      & \CIRCLE & \Circle   & \CIRCLE  & \Circle
      & \CIRCLE  & \Circle   & \Circle   & \Circle \\
      \cite{Dai2018SBLWT:Trustzone}
      & \CIRCLE & \CIRCLE & \Circle  & \CIRCLE & \CIRCLE & \Circle
      & \Circle  & \CIRCLE  & \CIRCLE  & \Circle
      & \CIRCLE  & \Circle   & \Circle   & \Circle \\
      \cite{volety2019cracking}
      & \Circle  & \Circle  & \Circle  & \CIRCLE & \CIRCLE & \Circle
      & \CIRCLE & \Circle   & \CIRCLE  & \Circle
      & \CIRCLE  & \Circle   & \Circle   & \Circle \\
      \cite{8966739}
      & \CIRCLE & \CIRCLE & \CIRCLE & \CIRCLE & \CIRCLE & \CIRCLE
      & \CIRCLE & \Circle   & \CIRCLE  & \Circle
      & \Circle   & \CIRCLE   & \Circle   & \Circle \\
      \cite{rezaeighaleh2020improving}
      & \CIRCLE & \CIRCLE & \CIRCLE & \Circle  & \CIRCLE & \Circle
      & \CIRCLE & \CIRCLE & \CIRCLE  & \Circle
      & \Circle   & \CIRCLE   & \Circle   & \Circle \\
      \cite{Urien2021InnovativeWallets}
      & \Circle  & \Circle  & \Circle  & \CIRCLE & \CIRCLE & \Circle
      & \CIRCLE & \CIRCLE & \Circle   & \Circle
      & \Circle   & \CIRCLE   & \Circle   & \Circle \\
      \cite{Rezaeighaleh2020MultilayeredWallet}
      & \Circle  & \Circle  & \CIRCLE & \Circle  & \Circle  & \CIRCLE
      & \CIRCLE & \CIRCLE & \Circle   & \Circle
      & \Circle   & \CIRCLE   & \Circle   & \Circle \\
      \cite{rezaeighaleh2019new}
      & \CIRCLE & \CIRCLE & \CIRCLE & \Circle  & \CIRCLE & \Circle
      & \CIRCLE & \Circle   & \CIRCLE  & \Circle
      & \Circle   & \CIRCLE   & \Circle   & \Circle \\
      \cite{di2020characteristics}
      & \Circle  & \Circle  & \Circle  & \Circle  & \Circle  & \Circle
      & \CIRCLE & \CIRCLE & \CIRCLE  & \Circle
      & \Circle   & \Circle    & \CIRCLE   & \Circle \\
% ----------- VERIFIED & CORRECTED ROWS ----------------------------------
\cite{Homoliak2024SoK:Factors}
& \Circle & \CIRCLE & \Circle & \Circle & \CIRCLE & \Circle
& \CIRCLE & \CIRCLE & \CIRCLE & \Circle
& \CIRCLE & \CIRCLE & \Circle & \Circle \\

\cite{andryukhin2019phishing}
& \Circle & \Circle & \Circle & \CIRCLE & \CIRCLE & \Circle
& \CIRCLE & \Circle & \CIRCLE & \Circle
& \Circle & \Circle & \Circle & \Circle \\

\cite{Chen2020ADefenses}
& \Circle & \Circle & \Circle & \CIRCLE & \CIRCLE & \Circle
& \CIRCLE & \CIRCLE & \CIRCLE & \Circle
& \Circle & \Circle & \Circle & \CIRCLE \\

\cite{Courtois2017StealthSystems}
& \CIRCLE & \CIRCLE & \Circle & \CIRCLE & \CIRCLE & \CIRCLE
& \CIRCLE & \Circle & \CIRCLE & \Circle
& \CIRCLE & \Circle & \Circle & \CIRCLE \\

\cite{Das2019AWallets}
& \CIRCLE & \CIRCLE & \Circle & \Circle & \CIRCLE & \Circle
& \Circle & \Circle & \CIRCLE & \Circle
& \CIRCLE & \Circle & \Circle & \Circle \\

\cite{Eyal2022OnDesign}
& \Circle & \CIRCLE & \Circle & \Circle & \CIRCLE & \Circle
& \Circle & \CIRCLE & \CIRCLE & \Circle
& \CIRCLE & \CIRCLE & \Circle & \Circle \\

\cite{Gotte2021TechAttacks}
& \CIRCLE & \CIRCLE & \Circle & \CIRCLE & \CIRCLE & \Circle
& \Circle & \Circle & \CIRCLE & \Circle
& \Circle & \CIRCLE & \Circle & \Circle \\

\cite{Guo2022ASecurity}
& \Circle & \CIRCLE & \Circle & \CIRCLE & \CIRCLE & \Circle
& \CIRCLE & \CIRCLE & \CIRCLE & \Circle
& \CIRCLE & \CIRCLE & \Circle & \CIRCLE \\

\cite{He2018AScheme}
& \Circle & \CIRCLE & \CIRCLE & \Circle & \CIRCLE & \Circle
& \Circle & \Circle & \CIRCLE & \Circle
& \CIRCLE & \Circle & \Circle & \Circle \\

\cite{Houy2023}
& \CIRCLE & \CIRCLE & \CIRCLE & \CIRCLE & \CIRCLE & \Circle
& \CIRCLE & \CIRCLE & \CIRCLE & \Circle
& \CIRCLE & \CIRCLE & \Circle & \CIRCLE \\

\cite{Li2020ASystems}
& \Circle & \CIRCLE & \Circle & \CIRCLE & \CIRCLE & \Circle
& \Circle & \Circle & \CIRCLE & \Circle
& \CIRCLE & \Circle & \Circle & \Circle \\

\cite{Mangipudi2023UncoveringCrypto-Wallets}
& \Circle & \CIRCLE & \Circle & \Circle & \Circle & \Circle
& \Circle & \Circle & \CIRCLE & \CIRCLE
& \CIRCLE & \Circle & \Circle & \Circle \\

\cite{Shbair2021HSM-basedBlockchain}
& \CIRCLE & \CIRCLE & \Circle & \CIRCLE & \CIRCLE & \Circle
& \Circle & \Circle & \CIRCLE & \CIRCLE
& \Circle & \CIRCLE & \Circle & \Circle \\

\cite{zhou2023sok}
& \Circle & \Circle & \Circle & \CIRCLE & \CIRCLE & \Circle
& \CIRCLE & \CIRCLE & \CIRCLE & \Circle
& \Circle & \Circle & \CIRCLE & \CIRCLE \\

\cite{chatzigiannis2025composability}
& \Circle & \CIRCLE & \CIRCLE & \CIRCLE & \CIRCLE & \Circle
& \Circle & \Circle & \CIRCLE & \CIRCLE
& \CIRCLE & \Circle & \CIRCLE & \Circle \\

% ------------------------------------------------------------------------

% -------------------------------------------------------------------------

    % \cite{Homoliak2024SoK:Factors}
    %   & \Circle  & \Circle  & \Circle  & \Circle  & \Circle  & \Circle
    %   & \Circle & \Circle & \Circle  & \Circle
    %   & \Circle   & \Circle    & \Circle   & \Circle \\
    %       \cite{andryukhin2019phishing}
    %   & \Circle  & \Circle  & \Circle  & \Circle  & \Circle  & \Circle
    %   & \Circle & \Circle & \Circle  & \Circle
    %   & \Circle   & \Circle    & \Circle   & \Circle \\
      
    %       \cite{Chen2020ADefenses}
    %   & \Circle  & \Circle  & \Circle  & \Circle  & \Circle  & \Circle
    %   & \Circle & \Circle & \Circle  & \Circle
    %   & \Circle   & \Circle    & \Circle   & \Circle \\
    %       \cite{Courtois2017StealthSystems}
    %   & \Circle  & \Circle  & \Circle  & \Circle  & \Circle  & \Circle
    %   & \Circle & \Circle & \Circle  & \Circle
    %   & \Circle   & \Circle    & \Circle   & \Circle \\
      
    %       \cite{Das2019AWallets}
    %   & \Circle  & \Circle  & \Circle  & \Circle  & \Circle  & \Circle
    %   & \Circle & \Circle & \Circle  & \Circle
    %   & \Circle   & \Circle    & \Circle   & \Circle \\
      
    %       \cite{Eyal2022OnDesign}
    %   & \Circle  & \Circle  & \Circle  & \Circle  & \Circle  & \Circle
    %   & \Circle & \Circle & \Circle  & \Circle
    %   & \Circle   & \Circle    & \Circle   & \Circle \\
      
    %       \cite{Gotte2021TechAttacks}
    %   & \Circle  & \Circle  & \Circle  & \Circle  & \Circle  & \Circle
    %   & \Circle & \Circle & \Circle  & \Circle
    %   & \Circle   & \Circle    & \Circle   & \Circle \\
      
    %       \cite{Guo2022ASecurity}
    %   & \Circle  & \Circle  & \Circle  & \Circle  & \Circle  & \Circle
    %   & \Circle & \Circle & \Circle  & \Circle
    %   & \Circle   & \Circle    & \Circle   & \Circle \\
      
    %       \cite{He2018AScheme}
    %   & \Circle  & \Circle  & \Circle  & \Circle  & \Circle  & \Circle
    %   & \Circle & \Circle & \Circle  & \Circle
    %   & \Circle   & \Circle    & \Circle   & \Circle \\
      
    %       \cite{Houy2023}
    %   & \Circle  & \Circle  & \Circle  & \Circle  & \Circle  & \Circle
    %   & \Circle & \Circle & \Circle  & \Circle
    %   & \Circle   & \Circle    & \Circle   & \Circle \\
      
    %       \cite{Li2020ASystems}
    %   & \Circle  & \Circle  & \Circle  & \Circle  & \Circle  & \Circle
    %   & \Circle & \Circle & \Circle  & \Circle
    %   & \Circle   & \Circle    & \Circle   & \Circle \\
    %       \cite{Mangipudi2023UncoveringCrypto-Wallets}
    %   & \Circle  & \Circle  & \Circle  & \Circle  & \Circle  & \Circle
    %   & \Circle & \Circle & \Circle  & \Circle
    %   & \Circle   & \Circle    & \Circle   & \Circle \\
      
    %       \cite{Shbair2021HSM-basedBlockchain}
    %   & \Circle  & \Circle  & \Circle  & \Circle  & \Circle  & \Circle
    %   & \Circle & \Circle & \Circle  & \Circle
    %   & \Circle   & \Circle    & \Circle   & \Circle \\
      
    %       \cite{zhou2023sok}
    %   & \Circle  & \Circle  & \Circle  & \Circle  & \Circle  & \Circle
    %   & \Circle & \Circle & \Circle  & \Circle
    %   & \Circle   & \Circle    & \Circle   & \Circle \\
      \bottomrule
    \end{tabular}%
  }
  \caption{Overview of related works. (\CIRCLE: include, \Circle: not include)}
  \label{Literature-Gap-Table-1}
\end{table}

 
% \subsection{Wallet Attack and Security}
% \label{sec:wallet-security}

% Various studies have rigorously analysed blockchain systems' overall security and vulnerability \cite{li2020survey, guo2022survey, zamani2020security, Urien2021, chen2020survey, zhou2023sok}. For instance, Chen et al. \cite{chen2020survey} provide a security analysis of the Ethereum blockchain by investigating the vulnerabilities, attacks, and defence mechanisms. We can be differentiated from these, as our work focuses on securing wallets which are categorised under the external auxiliary services \cite{zhou2023sok}, as opposed to any of the blockchain layers.

% The security of specific wallets within our taxonomy \autoref{sec:wallet-taxonomy} has also been an area of interest \cite{shbair2021hsm, gotte2021tech}. Götte and Scheuermann \cite{gotte2021tech} propose a defence method for Hardware Security Modules against advanced physical attacks. Our study analyses wallet security from a multi-layered approach (see \autoref{sec:defense-strategies}) and encompasses a wide range of attacks on wallets.

% % specific attacks on wallets
% Specific attack vectors have also been investigated \cite{andryukhin2019phishing, andryukhin2019phishing}. Andryukhin \cite{andryukhin2019phishing} evaluates phishing attacks on wallets and proposed prevention mechanisms. Bui et al. \cite{bui2019pitfalls} investigate the security vulnerabilities of the \acf{rpc} in desktop wallets. Our work analyses a broader scope of attacks compared to the focus of these papers.

% While a handful of studies have explored the security dimensions across various wallet types, the scope and depth of these analyses vary. Das et al \cite{10.1145/3319535.3354236} propose a comprehensive security model for deterministic hot/cold wallets and develop secure wallet schemes within these models. In contrast, our research broadens the wallet security analysis beyond the hot/cold wallet paradigm. We employ a multi-dimensional taxonomy of which the hot/cold dimension includes, as well as analyse operational mechanisms and investigate the gap between academia and industry. Eyal \cite{eyal2022cryptocurrency} evaluates how the number and type of keys managed affect the overall security of a wallet.
% Our research is differentiated through its detailed taxonomy, operational mechanisms, and empirical analysis. Although, Houy et al. \cite{houy2023security} conduct a systematic literature review of attacks and defence mechanisms of wallets, their investigation stops short of incorporating theoretical or empirical evaluations. 

% \subsection{Addressing Literature Gaps}
% \label{sec:gaps-in-literature}

% While various studies have explored specific wallet types, their mechanisms, and associated attack vectors, the literature lacks a unified, comprehensive examination that spans the full spectrum of wallet taxonomy, mechanisms, attack analysis, and security measures. This omission represents a significant gap, as a holistic understanding of these aspects is crucial for advancing the security of wallets. Our study bridges this gap and demonstrates the subject areas we cover in comparison to other studies as shown in \autoref{Literature-Gap-Table-1}.

\section{Generalised Wallet Mechanism}
\label{sec:wallet_mechanism}

\begin{definition}[Cryptocurrency Wallet]
 A wallet is a system that typically generates a private key (\teal{$sk$}) and securely stores it in an encrypted form, enabling an authenticated owner to sign transactions that are broadcast to the blockchain.
  \end{definition}

% \begin{algorithm}
%     \label{algo:cryptocurrency-wallet}
%     \SetAlgoLined
%     \KwIn{\teal{$rdm\_seed$}: \orange{bin}, \teal{$pw$}: \orange{str}}
%     \teal{$sk$} = \olive{genPrivateKey}(\teal{$rdm\_seed$})\\
%     \teal{$pk$} = \olive{genPublicKey}({\teal{$sk$})\\
%     \teal{$enc\_sk$} = \olive{encrypt}(\teal{$sk$}, \teal{$pw$})\\
%     \teal{$address$} = \hyperref[Hashing-Table-1]{\olive{hash}}(\teal{$pk$})\\
%     \teal{$nonce$} = \olive{0}}
%     \caption{Wallet initalisation}
% \end{algorithm}



% \begin{algorithm}[H] % Specify options here instead of in the package
%     \label{algo:cryptocurrency-wallet}
%     \SetAlgoNlRelativeSize{-1} % Adjust line number size if needed
%     \KwIn{\textcolor{teal}{$rdm\_seed$}: \textcolor{orange}{bin}, \textcolor{teal}{$pw$}: \textcolor{orange}{str}}
%     \textcolor{teal}{$sk$} = \textcolor{olive}{genPrivateKey}(\textcolor{teal}{$rdm\_seed$})\\
%     \textcolor{teal}{$pk$} = \textcolor{olive}{genPublicKey}(\textcolor{teal}{$sk$})\\
%     \textcolor{teal}{$enc\_sk$} = \textcolor{olive}{encrypt}(\textcolor{teal}{$sk$}, \textcolor{teal}{$pw$})\\
%     \textcolor{teal}{$address$} = \hyperref[Hashing-Table-1]{\textcolor{olive}{hash}}(\textcolor{teal}{$pk$})\\
%     \textcolor{teal}{$nonce$} = \textcolor{olive}{0}
%     \caption{Wallet initialization}
% \end{algorithm}


\begin{algorithm}
    \caption{Wallet initialisation}
    \label{algo:cryptocurrency-wallet}
    \begin{algorithmic}[1]  % This enables line numbering
        \State \textbf{Input:} \textcolor{teal}{$rdm\_seed$}: \textcolor{orange}{bin}, \textcolor{teal}{$pw$}: \textcolor{orange}{str}
        \State \textcolor{teal}{$sk$} = \textcolor{olive}{genPrivateKey}(\textcolor{teal}{$rdm\_seed$})
        \State \textcolor{teal}{$pk$} = \textcolor{olive}{genPublicKey}(\textcolor{teal}{$sk$})
        \State \textcolor{teal}{$enc\_sk$} = \textcolor{olive}{encrypt}(\textcolor{teal}{$sk$}, \textcolor{teal}{$pw$})
        \State \textcolor{teal}{$address$} = \textcolor{olive}{hash}(\textcolor{teal}{$pk$})
        \State \textcolor{teal}{$nonce$} = \textcolor{olive}{0}
    \end{algorithmic}
\end{algorithm}



\begin{definition}[Transaction]
\label{sec:def:tnx}
A transaction (\teal{$txn$}) is a structured message created by a wallet that enables state change executions on the blockchain. These state changes include token transfer transactions and smart contract transactions. 
\end{definition}

% \subsection{Wallet Mechanism}
% \label{sec:wallet_mechanism}

% In this section, we describe the major operations within the wallet mechanism including key generation, key storage and transaction management as shown in \autoref{fig:wallet-mechanism}.

\subsection{Key Generation}
\label{sec:key_generation}

\autoref{fig:wallet-mechanism} shows the operations within the wallet mechanisms. The process typically begins with \teal{$sk$} generation using a random seed (\teal{$rdm\_seed$}). Subsequently, the public key (\teal{$pk$}) is derived from \teal{$sk$} using the asymmetric key algorithm specific to the blockchain in use. For instance, Solana utilises the ed25519 curve for key generation, while Ethereum and Bitcoin use the secp256k1 curve. Once the key pair is generated and \teal{$pk$} is obtained, the wallet generates the address (\teal{$address$}) using a hash algorithm on \teal{$pk$}. \teal{$address$} serves as a public identifier for the wallet which shows user transactions on the respective blockchain and is used to retrieve state changes including nonce (\teal{$nonce$}) via a \acf{rpc} to the blockchain. \teal{$nonce$}, initially set to zero, acts as a transaction index, ensuring the sequential ordering of transactions from the wallet.

% was in text before
% For example, Ethereum and Bitcoin use the Keccak-256 and Hash-160 algorithms \cite{jiasong2023digital}, respectively, to generate \teal{$address$} from \teal{$pk$}.

% Figure environment removed

\subsection{Key Storage}
\label{sec:key-storage}
\teal{$sk$} is stored and encrypted using a \acf{kek} which we simply refer to as password (\teal{$pw$}) as shown in \hyperref[algo:cryptocurrency-wallet]{Algorithm 1} following its generation. This encrypted private key (\teal{$enc\_sk$}) remains secure during storage, with \teal{$pw$} serving as the means to decrypt and utilise \teal{$sk$} for transactions utilising symmetric key algorithms. Secure \teal{$sk$} storage is governed by the interplay of several factors: the key management infrastructure (see \autoref{sec:infrastructure}), representing the medium where \teal{$sk$} resides, the controlling entity (see \autoref{sec:design-cust}), which denotes the entity responsible for managing and safeguarding \teal{$sk$} and several other design factors described in \autoref{sec:wallet-taxonomy}. 

% \begin{algorithm}
%     \label{algo:transaction-signing}
%     \SetAlgoLined
%     \KwIn{\teal{$nonce$}: \orange{int}, \teal{$state\_trans\_info$}: \orange{str}, \teal{$enc\_sk$}: \orange{bytes}, \teal{$pwd$}: \orange{str}}
%     \KwOut{\teal{$\sigma$}: \orange{bytes}}

%     \teal{$nonce$} += 1\\
%     % \gray{// Create the transaction object} \\
%     % \teal{$txn$} = \{\quotes{nonce}: \teal{$nonce$},
%     % \quotes{data}: \teal{$txn\_instruction$}, \quotes{txn fee}: \teal{$txn\_fee$}\}\\
%     % \gray{// Compute the hash of the transaction data} \\
%     \teal{$txn$} = \olive{createTxn}(\teal{$state\_trans\_info$}, \teal{$nonce$})\\
%     \teal{$txn\_hash$} = \olive{hash}(\teal{$txn$})\\
%     % \gray{// Sign the transaction hash with private key}
%     \teal{$sk$} = \olive{decrypt}(\teal{$enc\_sk$}, \teal{$pwd$})\\
%     \teal{$\sigma$} = \olive{sign}(\teal{$txn\_hash$}, \teal{$sk$})\\

%     \KwRet{\teal{$\sigma$}}\\

%     \caption{Transaction generation}
% \end{algorithm}




\begin{algorithm}
    \caption{Transaction Generation}
    \label{algo:transaction-signing}
    \begin{algorithmic}[1]  % This enables line numbering
        \State \textbf{Input:} \textcolor{teal}{$nonce$}: \textcolor{orange}{int}, \textcolor{teal}{$state\_trans\_info$}: \textcolor{orange}{str}, \textcolor{teal}{$enc\_sk$}: \textcolor{orange}{bytes}, \textcolor{teal}{$pwd$}: \textcolor{orange}{str}
        \State \textbf{Output:} \textcolor{teal}{$\sigma$}: \textcolor{orange}{bytes}

        \State \textcolor{teal}{$nonce$} += 1
        % \Comment{\textcolor{gray}{// Create the transaction object}}
        \State \textcolor{teal}{$txn$} = \textcolor{olive}{createTxn}(\textcolor{teal}{$state\_trans\_info$}, \textcolor{teal}{$nonce$})
        \State \textcolor{teal}{$txn\_hash$} = \textcolor{olive}{hash}(\textcolor{teal}{$txn$})
        % \Comment{\textcolor{gray}{// Sign the transaction hash with private key}}
        \State \textcolor{teal}{$sk$} = \textcolor{olive}{decrypt}(\textcolor{teal}{$enc\_sk$}, \textcolor{teal}{$pwd$})
        \State \textcolor{teal}{$\sigma$} = \textcolor{olive}{sign}(\textcolor{teal}{$txn\_hash$}, \textcolor{teal}{$sk$})

        \State \textbf{return:} \textcolor{teal}{$\sigma$}  % This adds a numbered return line
    \end{algorithmic}
\end{algorithm}


\subsection{Transaction Management}
\label{sec:transaction_management}

\subsubsection{Transaction Generation}
\label{sec:transaction_signing}
This begins with transaction message creation (\teal{$txn$}) by inputting the state transition information (\teal{$\it state\_trans\_info$}). The message (\teal{$txn$}) is then hashed to produce the transaction hash (\teal{$txn\_hash$}). Following transaction creation, the sender proceeds to sign the transaction and provides \teal{$pw$} to decrypt the private key. The signing algorithm takes the decrypted private key (\teal{$sk$}) and \teal{$txn\_hash$} as inputs to generate the signature (\teal{$\sigma$}), which authorises the transaction (see \hyperref[algo:transaction-signing]{Algorithm 2}).

% During this interaction, the sender provides details of the transaction message including the \teal{$recipient\_addr$}, which could be the token \teal{$recipient\_addr$} or the \teal{$contract\_addr$} included in the specific \teal{$txn\_instruction$}. The \teal{$txn\_instruction$} generated by the user depends on the transaction type. For instance, a token transfer instruction specifies the number and amount of tokens to be sent while a smart contract method call specifies the function to be executed along with any necessary parameters that dictate the operation's details.

% Upon initiation, the \teal{$txn\_fee$} is calculated to determine the total cost, including any transaction amount, if applicable. The transaction can proceed only if the \teal{$sender\_balance$} exceeds this total cost. This ensures that the sender has adequate funds to cover both any potential transfer amounts and the necessary txn fee, confirming resource availability for execution.

% , such as the number of tokens (\teal{$Q$}) to be transferred out to the recipient's address (\teal{$\alpha_R$}), the nonce (\teal{$N$}), and smart contract data (\teal{$D$}) required for accessing smart contract methods.

% The signature (\teal{$\sigma$}) is added to the transaction data to create a signed transaction object (\teal{$\tau_{sign}$}), which includes both the transaction data (\teal{$\tau$}) and the signature (\teal{$sign$}). The signature authorises the user-defined instruction.

\subsubsection{Transaction Broadcast}
\label{sec:transaction_broadcast}
\teal{$\sigma$} is verified using the public key to assert its validity as shown in \hyperref[algo:transaction-broadcast]{Algorithm 3}. If \teal{$\sigma$} is invalid, the transaction is rejected and not processed further. Conversely, if \teal{$\sigma$} is valid, the transaction is broadcast to the blockchain. 


\begin{algorithm}
    \caption{Transaction broadcast}
    \label{algo:transaction-broadcast}
    \begin{algorithmic}[1]  % This enables line numbering
        \State \textbf{Input:} \textcolor{teal}{$\sigma$}: \textcolor{orange}{str}, \textcolor{teal}{$pk$}: \textcolor{orange}{hex}

        % \Comment{\textcolor{gray}{// Broadcast the transaction to network for verification}}
        \State \textcolor{teal}{$verified$} = \textcolor{olive}{verify}(\textcolor{teal}{$\sigma$}, \textcolor{teal}{$sender\textunderscore pub\textunderscore key$})
        \State \textcolor{olive}{assert}(\textcolor{teal}{$verified$}, \enquote{transaction failed})
        \State \textcolor{olive}{broadcast}(\textcolor{teal}{$\sigma$}, \textcolor{teal}{$sender\textunderscore pub\textunderscore key$})

    \end{algorithmic}
\end{algorithm}



\section{Methodology}
\label{sec:method}
% Figure environment removed


%\vspace*{0.2cm}\noindent\textbf{Problem formulation.} 
% Let $O$ be the set of Objects,  $S$ be the set of
%  States and $\textit{I}$ the set of
%   Images which consists of the disjoint sets ${I^{S}}$ and ${I^{U}}$ that are used during the training and testing phase respectively. 
%    Each image $i_{k} \in \textit{I}$  contains an object $o_{i} \in O$ which  is situated in a state $s_{j} \in S$. The OSC task deals with the yielding  of a  predicted state label $sp_{j} \in S$ for an image $i_{k} \in {I^U}$ that has been given as an input. In the zero-shot variation of OSC, ${S^{S}} 
%   \not \supseteq {S^{U}}$, i.e. some of the states contained in the testing images do not appear in the training images. 
% Let $O$ denote a set of objects, 
% \textcolor{red}{$S^S$ as the set of known object states found in the training images, $S^U$ as the test, yet unknown, object state labels} and $I$ the set of images, which is partitioned into the training set $I^T$ and the test set $I^U$. 
% % Each image $i \in I$ contains an object $o \in O$ in a state $s \in S$. 
% \textcolor{red}{Each image $i \in I^T$ contains an object $o \in O$ in a state $s \in S^S$, while an image $i \in I^U$ contains an object $o \in O$ in any state $s \in S = \{S^S \cup S^U\}$. }
% The goal of OSC is to predict the state $s \in S$, given the object $o$ in $i \in I^U$. In the zero-shot variation of OSC, the set of states observed in the test images $S^U$ is not a subset of the set of states observed in the training images $S^S$, i.e., there exists some states in the test image set that do not appear in the training set. Furthermore, the task should be addressed in an object-agnostic manner, i.e. no information concerning the object classes is to be utilized explicitly.  However,  although the set of object classes does not directly affect the task of OaSC, its size is proportional to the complexity of the problem. 
% The workflow of the proposed method is shown in \autoref{fig:pipeline}.

Let $O$ denote a set of objects, $S$ denote the set of states and $I$ denote the set of images, which is partitioned into the training set $I^T$ and the testing set $I^U$. Each image $i \in I$ contains an object $o \in O$ in a state $s \in S$. 
The goal of OSC is to predict the state $s \in S$, given the object $o$ in $i \in I^U$. In the zero-shot variation of OSC, the set of states observed in the test images $S^U$ is not a subset of the set of states observed in the training images $S^S$, i.e., there exists some states in the test image set that do not appear in the training set. Furthermore, the task should be addressed in an object-agnostic manner, i.e. no information concerning the object classes is to be utilized explicitly.  However,  although the set of object classes does not directly affect the task of OaSC, its size is proportional to the complexity of the problem. 
The workflow of the proposed method is shown in \autoref{fig:pipeline}.




% and will be analyzed in the following sections.

% \vspace*{0.2cm}\noindent\textbf{Approach.}

\subsection{Overview}
% Our method is inspired by works that address the problem of zero-shot object classification \cite{}. The main idea behind this line of work is that the necessary information for the classification of the unseen classes can be found in a Knowledge Graph (KG) if processed appropriately by a Graph Neural Network (GNN). Obviously, the most crucial component of this approach lies in the combination of the visual information stemming from the training images and referring to the seen classes with the semantic information stemming from the KG  and referring to the unseen classes.

We are inspired by prior research on zero-shot object classification and leverage the potential of KGs and GNNs to classify previously unseen objects~\cite{Kampffmeyer2019,nayak:tmlr22}. 
The core idea is that semantic information that is stored in the KG can be used by GNNs to learn graph embeddings that can be utilized jointly with visual information extracted from training images. 
This enables the model to generalize to new object classes by leveraging the semantic and contextual information encoded in the graph embeddings of the KG.

% More in detail, the GNN architecture is adopted to the architecture of the Classifier  that is used for the training on seen classes, the GNN last layer has the same size  with the Classifier last layer. This way the GNN can produce semantic embedding features that correspond to all the classes, both seen and unseen, that will be encountered during the inference. These embedding features  replace the last layer of the Classifier. Holding this layer fixed, the body of the Classifier is then fine-tuned with the training images.

GNNs are designed to operate on graph-structured data, such as KGs~\cite{kipf2016semi,Monka2022}. KGs are typically represented as labeled multi-graphs, where nodes correspond to entities, and edges represent entity relationships. GNNs process this graph by iteratively aggregating information from neighboring nodes, using neural network-based operations.

At each iteration, a GNN receives a feature vector for each graph node, which is initially set to the node's embedding vector. Then, the GNN performs a message-passing step that aggregates information from neighboring nodes, based on the edge weights and the features of the nodes. This message-passing operation can be formulated as a neural network layer, which applies a learnable function to the features of the neighboring nodes and returns an aggregated message for each node. After the message-passing step, the GNN updates the node features by applying a learnable transformation that takes into account the original features of the node and the received messages from its neighbors. This updated feature vector is then passed to the next iteration of the message-passing step. The process continues until a fixed number of epochs or convergence.
%%%AAA: Endexetai na mas rethrown gia tis times aytwn twn parametrwn?
% KP edw anaferetai genika mia diadikasia GNN training. Na anaferoume edw times parametrwn h sto 4 - see implementation details ?

The proposed method leverages GNN training using a visual classifier that is trained on seen state classes as supervision. In particular, the last layer of the GNN is designed to have the same size as the last layer of the classifier. This enables the GNN to generate semantic embedding features that correspond to all classes, including both seen and unseen classes that will be encountered during inference. Subsequently, the semantic embedding features replace the last layer of the classifier while this layer is kept fixed. The body of the classifier is then fine-tuned with the training images to optimize the overall model for state recognition.

% \vspace*{0.2cm}\noindent\textbf{GNN Details.} 
Overall, we experimented with four different model architectures and opted for the Transformer Graph Convolutional network (Tr-GCN)~\cite{nayak:tmlr22}. Further details are provided in Section~\ref{sec:abl} and the supplementary material of this work. 
The Tr-GCN mode is capable of combining input sets non-linearly by utilizing multilayer perceptrons and self-attention. Tr-GCN refers to an inductive model that can learn node representations by aggregating local neighborhood features allowing the trained model to make predictions on new graph structures without retraining. 
We leverage the aforementioned property of the Tr-GCN to train a permutation invariant non-linear aggregator that captures the intricate structure of a common sense knowledge graph. 
% , rendering it well-suited for zero-shot learning. 
% It is worth noting that a similar network architecture has been effectively employed for zero-shot object classification~\cite{nayak:tmlr22}.

% A critical aspect of the proposed method involves calibrating the weights of the GNN in a manner that its predictions in the semantic space are useful for the classifier deployed in the visual space. To accomplish this, we adopt an approach based on prior research \cite{Kampffmeyer2019, Wang2018b, nayak:tmlr22} that involves learning the semantic class representations by minimizing the L2 distance between the learned class representations and the weights of a fully connected layer in a ResNet classifier pre-trained on the ILSVRC 2012 dataset \cite{russakovsky2015imagenet}. Once the class representations are learned, we fix them and fine-tune the ResNet backbone using the training images from the dataset.




% \vspace*{0.2cm}\noindent\textbf{Building of the KG.}
% The KG is created by the querying  of a common sense repository. The repositories that we are ConceptNet \cite{} and WordNet\cite{}. The procedure takes place as follows. Initially we create a set of nodes that correspond to the target stace classes. Subsequently, the repository is queried for each of these nodes and its neighbours in the repository of  added to the KG if  certain criteria are met (see ablation section for more details). This procedure is repeated for the newly added nodes and henceforth until a number of hops has been reached.  

\subsection{The proposed OaSC approach}
\label{sec:pipeline}
Overall, the proposed method consists of four stages, as shown in \autoref{fig:pipeline}: (1) construction of the KG, (2) GNN training and learning of semantic graph embeddings, (3) fine-tuning of the visual classifier and (4) deployment of the fine-tuned state classifier.

\vspace*{0.0cm}\noindent\textbf{Construction of the KG (Stage 1)}:
To create the KG, we query a common sense repository to compile a generic solution and to avoid the construction of a task-specific KG, tailored to the entities at hand and their relationships. First, a set of nodes that correspond to the words of the target state classes $S^U$ and $S^S$ is generated. Then, we query the repository for each of these nodes and add their neighbors in the KG, if they meet specific criteria (see also Section~\ref{sec:abl}). This process is repeated for the newly added nodes until a specified number of node hops is reached.

This technique for building a generic KG offers several advantages in comparison to other problem-specific approaches. First, it allows the same KG to be used for different variations of the task. It also enables transfer learning since KGs can be reused to tackle other related problems. Moreover, the construction of such a KG does not rely on expert knowledge. Besides, the structured representation of relationships between entities and concepts that KGs provide can be leveraged to generate robust embeddings for zero-shot learning.
% which is expensive and time-consuming.  
The trade-off is that such KGs are prone to noisy information in the used repositories. 

% In comparison, language models, such as BERT~\cite{devlin2018bert}, often rely on large amounts of unstructured text data to generate embeddings. While language models are highly effective at capturing semantic relationships between words and phrases, they can also be prone to create associations between concepts that are not actually related. This can lead to noisy or unreliable embeddings, which can in turn degrade the performance of zero-shot learning models. By contrast, the structured nature of KGs allows for more accurate and precise capture of relationships between entities and concepts, leading to more robust embeddings that can improve the accuracy and reliability of zero-shot learning models~\cite{brown2020language}.


\vspace*{0.0cm}\noindent\textbf{Computation of  Graph Embeddings (Stage 2)}:
% Given the KG constructed in Stage 1, a word features embedding matrix corresponding to the KG nodes is created by utilizing the pre-computed word features of GloVe~\cite{pennington2014glove}. 
% % Subsequently,  random walks are performed in the KG and a sample of neighbors for each node is obtained.
% By taking the word features embedding matrix, the KG topology, and a target node as inputs, the GNN estimates the node's embeddings: the features of the node and its neighbors are  aggregated and submitted to a series of convolutions and pooling operations before the  output is produced in the form of a feature vector, the length of which is tailored to be the same as the size dimension of the last layer of a ResNet-101 classifier. 
% This procedure is repeated for all KG nodes and results in the computation of the semantic embeddings for all target state classes with each embedding being a feature vector of length equal to 2048. By combining these embeddings for the \mathcal{d} target classes  a  $ d \times 2048$ features matrix is created which serves as the last layer of a CNN classifier that is utilized during Stages 3 and 4.
% which serves as the last layer of a CNN classifier that is utilized during Stages 3 and 4 .
% We employ an established approach~\cite{Kampffmeyer2019, Wang2018b} that involves training of a transformer-based Graph Convolutional Model using graph embeddings of a set of semantic entities acquired by a common sense repository by minimizing the L2 distance between the learned class representations and the weights of a fully connected layer in a ResNet classifier, pre-trained on the ILSVRC 2012 dataset~\cite{russakovsky2015imagenet}, ensuring that the semantic class representations are meaningfully embedded.
We employ an established approach~\cite{Kampffmeyer2019, Wang2018b} that involves the training of a transformer-based Graph Convolutional Network (GCN)
 \textcolor{black}{ that utilizes a KG as input  %Training is performed %using features of a set of semantic entities acquired by a common sense repository, \textcolor{red}{(e.g. the ConceptNet, CSKG, or other)}  
 and generates an embedding vector for each node of the  KG. %. For the production of the embeddings vectors the GCM employs a sequence of transformations to the semantic features that correspond to the concepts linked to each node.
This process defines pre-computed GloVe word, i.e. semantic features~\cite{pennington2014glove}, for the KG nodes with each node representing a concept class.
% To compute node embeddings, the GNN is applied to encode the KG topology and the word feature embedding matrix. 
The GNN  aggregates each node's and its neighbors' features through a sequence of convolutions and pooling operations. %This results in the generation of a feature vector having a length equal to the dimension of the last layer in a visual CNN-based classifier that is instantiated using a ResNet-101 model. 
%By pre-training the visual classifier in a set of target classes 
The visual classifier is pre-trained on a set of target classes and using the weights of its fully connected layer, the GCN learns to produce visual feature representations, i.e. visual embeddings,  corresponding to the concept classes of the KG`s nodes.}
\textcolor{black}{
Formally,  the training involves the minimization of the L2 distance   $\mathcal{L_G}$ between the generated visual embeddings and the ground truth visual embeddings stemming from the visual classifier.} 
\textcolor{black}{In notation, 
\begin{equation}
\mathcal{L_G} = \frac{1}{2N} \sum_{n \in N} \sum_{p \in P} (W_{n,p} - \tildea{W}_{n,p})^2,
 \end{equation}
where $\tildea{W} \in \mathbb{R}^{|N|xP}$ denotes the weights of the GCN for the set of known concept classes $N$ and the dimensionality $P$ of the weight vector. Similar to~\cite{Kampffmeyer2019}, the ground truth weights, denoted as $W \in \mathbb{R}^{|N|xP}$, are obtained by extracting the last layer weights of a pre-trained CNN.}
% This process is repeated for all KG nodes corresponding to $S^U$ and $S^S$, generating semantic graph embeddings for all target state classes. 
%Each embedding comes in the form of a feature vector of length 2048. 


%By combining these embeddings for the $d$ target classes, a  $d \times 2048$ features matrix is defined that is integrated as the final layer of the visual CNN-based classifier that is employed in Stages~3-4.
%A critical aspect of this process is adjusting the GNN weights to align its predictions with the semantic space. This ensures that the semantic embeddings effectively aid the classifier used in Stages 3 and 4, operating in the visual space. 



\textcolor{black}{ 
The KG  given as an input to the GCN model is a hierarchical graph created for the requirements of the   ILSVRC 2012 dataset~\cite{russakovsky2015imagenet} and represents the WordNet hierarchical structure of the $1,000$ classes comprising the dataset. These 1,000 concept labels constitute the set of classes upon which the visual classifier used for the extraction of the ground truth visual embeddings is pre-trained.
}
% A critical aspect of this process is adjusting the GNN weights to align its predictions with the semantic space. This ensures that the semantic embeddings effectively aid the classifier used in Stages 3 and 4, operating in the visual space. 
% The concepts  that are used for the training refer to a set of 1K object classes of the ILSVRC 2012 dataset~\cite{russakovsky2015imagenet}, while the pre-trained ResNet101-based classifier is used for supervision to ensure that the GNN outputs, thus the semantic object class representations, are meaningfully embedded into the visual feature space. 
After the training is completed, the GCN model is employed to process the KG (constructed in Stage 1) and generate visual embeddings for the KG nodes that correspond to the object state classes,  by taking as input the  KG that was constructed during Stage 1. Each embedding comes in the form of a feature vector of length 2048, i.e. dimension of the last layer of the  pre-trained visual CNN-based classifier.
By combining these embeddings for the $d$ target classes, a  $d \times 2048$ features matrix is defined that is integrated as the final layer of the visual CNN-based classifier that is employed in Stages~3-4.
% First, using their pre-computed GloVe word features~\cite{pennington2014glove}, a matrix of word, i.e. semantic, features embeddings is defined for each of the KG nodes.
% % To compute node embeddings, the GNN is applied to encode the KG topology and the word feature embedding matrix. 
% Subsequently,  the GNN takes as input every target node that corresponds to any class in $S^U$ and $S^S$ and aggregates the features about the node and its neighbors through a sequence of convolutions and pooling operations. This results in the generation of a feature vector having a length equal to the dimension of the last layer in the visual CNN-based classifier that is instantiated using a ResNet-101 model.
% % This process is repeated for all KG nodes corresponding to $S^U$ and $S^S$, generating semantic graph embeddings for all target state classes. 
% Each embedding comes in the form of a feature vector of length 2048. By combining these embeddings for the $d$ target classes, a  $d \times 2048$ features matrix is defined that is integrated as the final layer of the visual CNN-based classifier that is employed in Stages~3-4.
% A critical aspect of this process is adjusting the GNN weights to align its predictions with the semantic space. This ensures that the semantic embeddings effectively aid the classifier used in Stages 3 and 4, operating in the visual space. 

% A critical aspect of this procedure involves calibrating the weights of the GNN to embed its predictions in the semantic space, i.e. semantic embeddings, are useful for the classifier deployed in the visual space during Stages 3 and 4. To accomplish this, we adopt an approach based on prior research~\cite{Kampffmeyer2019, Wang2018b} that involves learning the semantic class representations by minimizing the L2 distance between the learned class representations and the weights of a fully connected layer in a ResNet classifier pre-trained on the ILSVRC 2012 dataset~\cite{russakovsky2015imagenet}.  

\vspace*{0.0cm}\noindent\textbf{Fine-tuning of the Visual Classifier (Stage 3)}:
The estimated semantic embeddings are integrated into a visual CNN classifier that relies on the ResNet backbone and is initially pre-trained for object classification. The embeddings serve as the final layer of the network, encapsulating the representations essential for predicting the train state classes $S^S$. To enable this adaptation, the visual classifier undergoes re-training, specifically tailored to the classification of the train classes. 
During this fine-tuning process, input images $I^T$ contain states sourced exclusively from the training set $S^S$, i.e. ``seen states''. The primary objective is to harness the classifier capabilities to classify these familiar states, accurately. Notably, fine-tuning involves keeping the weights of the last layer fixed, safeguarding the integrity of the acquired semantic representations from Stage 2. Consequently, adjustments are only applied to the weights of preceding layers to ensure they effectively match the ``frozen'' last-layer weights.
% Apart from this detail, the procedure takes place in the same manner as the training of a CNN classifier.
% in every training epoch a loss is computed the value of which guides the update of all layers weights except the last one. 
Following the notation introduced 
\textcolor{black}{in the beginning of Section~\ref{sec:method}, the loss function is defined as:}
\begin{equation}
% \mathcal{L} = -\sum_{i \in S^{S}} y_i \cdot \log(P(y=i|X))
\mathcal{L_V} = -\sum_{s \in S^S, i \in I^{T}} y_s \cdot \log(P(s|i)),
 \end{equation}
\textcolor{black}{for the predicted \textit{$y_s$} state label in the \textit{$S^S$} set of state labels. $P(s|i)$ denotes the probability of state label \textit{s} based on the softmax vector given an image \textit{i} from the $I^T$ training set.}

\noindent\textbf{Zero-shot OaSC (Stage 4)}:
Upon the completion of fine-tuning, the visual state classifier can be utilized for  prediction by choosing the most likely class
\begin{equation}
% \^y = \arg\max_{i \in S} \left( P(y=i|X) \right)
\hat{y} = \arg\max_{s \in S^U i \in I^{U}} \left( P(s|i) \right),
\end{equation}
\textcolor{black}{where $I^U$ denotes the test image set and $S^U$ the test state classes respectively.} 
We highlight that the classifier is well-suited for predicting either only unseen classes, i.e. zero-shot classification, or both seen and unseen classes, i.e. generalized zero-shot classification.
\vspace{-.15cm}
% \subsection{Pipeline}

% Overall, the pipeline of our method consists of four stages (\autoref{fig:pipeline}}). During \textbf{Stage 1}, the KG is constructed.

% \vspace*{0.2cm}\noindent\textbf{Construction of the KG (Stage 1)}:
% The KG creation process involves querying a common sense repository to enable generalization instead of creating a custom KG tailored to specific entities and relationships. Initially, nodes corresponding to the target state classes are generated. The repository is then queried for each node, and neighbors meeting specific criteria are added to the knowledge graph. This process continues for the newly added nodes until a specified number of hops is reached. More details can be found in the ablation section.


% \vspace*{0.2cm}\noindent\textbf{Computation of semantic embeddings (Stage 2)}:


% \vspace*{0.2cm}\noindent\textbf{Finetuning of the Classfier (Stage 3)}:

% \vspace*{0.2cm}\noindent\textbf{Deployment  (Stage 4)}:

\section{Design Decisions}
\label{sec:wallet-taxonomy}


We propose a design framework for developing wallets that integrates traditional models and recent advancements. To develop this framework, we analyse various designs of wallets within the industry. We also identify known vulnerabilities and previous attacks associated with these wallets, as summarised in \autoref{tab:wlt._taxonomy}. 

% Lastly, we provide a security and privacy evaluation of individual wallet design factors. 

% This evaluation does not serve as a method to aggregate individual factors to evaluate the overall privacy of a wallet.

% Additionally, we examine several academic wallet solutions to ensure applicability. Based on this analysis, we categorise existing wallets according to design factors such as infrastructure, custody, initialisation, distribution, authentication, authorisation, validation, and recovery. Furthermore, we identify known vulnerabilities and previous attacks associated with these wallets, as summarised in \autoref{tab:wlt._taxonomy}. 

% what methodology was used to come to this design taxonomy
% how did you select the wallets in the table

\subsection{Infrastructure}
\label{sec:infrastructure}

This design factor is centred on the private key (\teal{$sk$}) or transaction management infrastructure (see \autoref{sec:wallet_mechanism}) the controlling entity employs. 

% For ease of illustration, we also represent custodial wallet designs as desktop, browser, and mobile; however, this is merely to demonstrate user options as the exchange manages \teal{$sk$} by design.

\subsubsection{Software Wallets}
\label{sec:software-wallets} 

% The first wallet application Bitcoin Core, as well as Electrum, Exodus and Atomic wallets, are examples notable examples.

% However, integration with simplified payment verification (SPV) or a centralised server enables efficient transaction verification \cite{biryukov2019security}. 

Software wallets are applications that manage private keys (\teal{$sk$}) or transaction authorisation conditions within a software environment. Existing software infrastructure designs include desktop, browser, mobile and smart contract wallets. Desktop wallets are installed on computers and typically store \teal{$sk$} on a local file in the computer's file system of the software environment. Browser wallets present an alternative setup, as programs are installed or built into the web browser and credentials are typically stored in the browser's local storage \cite{2024MetaMaskWallet}. Two existing designs are browser extensions such as Metamask and Phantom and built-in browser-native such as Brave \cite{Brave2023BraveBrave}. Another prevalent wallet type is the mobile wallet which is installed on devices with limited computing power and storage capability in comparison with PCs. Mobile wallets also typically store \teal{$sk$} locally and can enhance security with mobile \acs{os} integrations such as the Android Keystore and iOS Keychain \cite{keystore}. However, should vulnerabilities be present in the operating system \autoref{sec:threat_class}, there exists susceptibility to specific attacks that exploit these weaknesses (see \autoref{sec:privilege}). Metamask, Phantom, Brave and Coinbase wallets are available as mobile wallets.

To mitigate the risk of \teal{$sk$} and \teal{$rdm\_seed$} loss, smart contract wallets (e.g. Argent and Safe) are deployed on the blockchain to abstract typical \teal{$sk$} management (see \autoref{sec:wallet_mechanism}) and create advanced transaction functions such as multi-factor authentication, ownership assignments, spending limits, and recovery mechanisms, often through integration with centralised or decentralised relayers \cite{di2020characteristics, erc4337}. Despite these advanced capabilities, these wallets are susceptible to specific vulnerabilities due to the immutable nature of blockchain. Flawed implementation and access control in parity wallet resulted in significant financial losses \cite{palladino2017parity}.

% % Figure environment removed

% scale used to be = 0.11

% \begin{itemize}
%     \item \textbf{Definition}
%     \item \textbf{Mechanism}
%     \item \textbf{Security Features}
%     \item \textbf{Vulnerabilities}
%     \item \textbf{Example}
% \end{itemize}

% accessible on browsers or browser extensions

% Unlike desktop wallets, these wallets offer the advantage of accessibility to cryptocurrencies across multiple devices to enhance user flexibility and minimise the risks associated with reliance on a single device \cite{consensys2}. To achieve this, MetaMask stores \teal{$sk$}, seed phrases, and \teal{$pw$} locally in the browser's data cache. This data is encrypted and can be accessed when \teal{$pw$} is input. However, despite this, some web wallets such as Bitstamp are known to store data on third-party servers.

% These wallets typically store \teal{$sk$} locally and can enhance security with mobile \acs{os} integrations such as the Android Keystore and iOS Keychain, which ensure secure management and isolation from other applications \cite{keystore}.

% \subsubsection{Paper Wallets}
% \label{sec:paper-wallets} 

% Paper wallets were introduced to the broader crypto community through platforms such as Bitaddress.org by Peter Kroll in 2011 \cite{paperwallet}. As one of the earliest forms of offline storage, these wallets emerged following the development of desktop wallets, aiming to mitigate online threats. \teal{$sk$} and \teal{$pub\_key$} pairs are generated, as detailed in \hyperref[algo:cryptocurrency-wallet]{Algorithm} \autoref{algo:cryptocurrency-wallet}, then printed on paper, often encoded as QR codes or mnemonics, to ensure secure offline storage. Unlike typical key storage algorithms that require user authentication, paper wallets simplify the security process by eliminating digital storage \cite{GkaniatsouAndrianaandArapinis2017, bitcoinwallet}. However, this method requires manual input for transaction signing, sacrificing some convenience for increased security.

% Over time, the concept of physical wallets evolved, with notable innovations such as the Casascius coins (similar to paper wallets but the \teal{$secret\_keys$} are concealed under a tamper-resistant hologram), introduced by Mike Caldwell in 2011 \cite{adrianne2013casascius}. Nonetheless, regulatory measures, particularly the FinCEN's order in 2013, significantly curtailed the production of these funded bitcoin tokens \cite{ahonen2016encyclopedia}. 

% The popularity of paper wallets has also waned currently due to their inherent limitations, such as promoting address reuse, reliance on centralised validation, and the risks associated with managing raw private keys and executing transactions. Modern alternatives, including seed phrases and deterministic wallets, have been developed to offer more secure and user-friendly solutions. These advancements address the drawbacks of traditional paper wallets and mitigate the risks associated with their use, marking a significant evolution in cryptocurrency storage solutions.

% \subsubsection{Brain Wallets}
% \label{sec:brain-wallets} 

% As the cryptocurrency ecosystem developed, brain wallets emerged around 2012 as another alternative method of key storage, relying entirely on the cognitive ability of individuals to memorise a passphrase or seed phrase \cite{vasek2017bitcoin}. By eliminating the need for physical storage, these wallets place the security of the funds entirely in the owner's memory.

% However, the simplicity of brain wallets comes with considerable risks. If the passphrase, which acts as the private key, is forgotten or inadequately complex, the funds are irrevocably lost or susceptible to brute-force attacks (see \autoref{sec:brute-force}). The security of brain wallets is wholly dependent on the user's ability to create and recall a sufficiently random and complex passphrase. Despite these vulnerabilities, brain wallets provide a unique method of managing and spending funds securely by importing private keys into wallet clients when necessary for transactions \cite{brain}.

% To address these security issues, recent innovations have sought to enhance the usability and safety of brain wallets. In 2019, a semi-custodial brain wallet was developed, which simplifies the user's burden by only requiring them to remember a username and password, while a server assists in a key generation without storing the complete key \cite{aman2019zerowallet}. This model has been further enhanced with post-quantum cryptography to fortify the security of brain wallets, offering a more robust solution that attempts to balance the ease of use with enhanced security measures \cite{kethepalli2023reinforcing}.

\subsubsection{Hardware Wallets}
\label{sec:hardware-wallets}

Hardware wallets typically involve \teal{$sk$} management within a \acf{se} (e.g. microcontroller or smart card), to protect against tampering and facilitate the execution of cryptographic operations, such as transaction signing (see \autoref{sec:wallet_mechanism}). Isolated in design with no internet connectivity functionality, their mechanism operates by performing all cryptographic operations on an offline hardware device and typically requires a distinct online device to create and broadcast transactions \cite{ledgeracademy}. The connection between both devices can be achieved by Bluetooth (e.g. Ledger), USB (e.g. Trezor), NFC (e.g. Tangem) and QR codes (e.g. Ngrave). Despite these implementations, hardware wallets have been liable to supply chain \cite{ledger_vuln}, software \cite{Cointelegraph2023LedgerRedefined, Ledger2018FirmwareFixed} and other vulnerabilities \cite{CoinDesk2018SecurityAntennae, Freemindtronic2023LedgerHackers}.

% \autoref{sec:threat_class} classifies threats and vulnerabilities in the wallet system. 


% Hardware wallets exist in air-gapped such as Ngrave, USB or Bluetooth such as Ledger and Trezor, and Tangen.

% Hardware wallet design focuses on the physical form of the offline hardware, the internet connectivity method it employs and the mechanism design. Examples of hardware wallets include

% Table to show

% exist in air-gapped such as, USB or Bluetooth such as or smart card

% Hardware wallets exist in various forms, including USB devices, air-gapped setups, and smart cards as shown in \autoref{fig:wallet-taxonomy}. 

% \begin{itemize}
%     \item \textbf{Definition}
%     \item \textbf{Mechanism}
%     \item \textbf{Security Features}
%     \item \textbf{Vulnerabilities}
%     \item \textbf{Example}
% \end{itemize}

% The transaction process in hardware wallets involves two critical stages: transaction creation and transaction transmission \cite{ledgeracademy}. The transaction creation stage begins when the user initiates a transaction through a client interface, such as a wallet or web application. Created transaction details are then securely communicated to the hardware wallet, typically via a physical connection or a secure wireless connection such as Bluetooth. The user then reviews and confirms these details on the device’s secure display, ensuring that the transaction data has not been tampered with, even if the computer or smartphone is compromised \cite{ivanov2021ethclipper}.

% Once confirmed, the hardware wallet cryptographically signs the transaction within its secure environment, utilising a non-extractable private key that never leaves the device \cite{ivanov2021ethclipper}. The signed transaction is then transmitted to the client software, which broadcasts it to the blockchain network. 


% ----
% was commented out before
% This design effectively safeguards against malware, phishing attacks, and other common vectors of cyber theft, ensuring that the transaction signature can only occur with the user's explicit approval \cite{arapinis2019formal}.
% ----

% Designed to operate in potentially malicious environments, hardware wallets can function offline, minimising threats posed by online attacks. The devices require physical user interaction for each transaction, embedding the user in the verification loop and establishing a trusted pathway for confirming transaction authenticity \cite{arapinis2019formal}. This hands-on verification process is critical, especially when the client device might be under the control of an attacker \autoref{sec:attack-framework}.


% ----
% was commented out before
% % Figure environment removed
% ----

% Despite early challenges related to transaction speed and dependency on the physical device, recent advancements have significantly broadened the capabilities of hardware wallets. Modern versions now support multiple currencies, can interact directly with decentralised applications (DApps), and seamlessly integrate with smartphone applications, enhancing user convenience and extending their utility beyond simple storage \cite{8966739}.


% The foundational concept of smart contract wallets stems from the \acf{eip} 86 introduced in 2016. This proposal aimed to enhance the flexibility of signature (\teal{$\sigma$}) verification and facilitate the creation of account contracts \cite{eip86}. The concept was further developed and standardised with the introduction of \acs{eip}-4337 in 2021, which provides a more decentralised framework than earlier smart contract wallet implementations like Argent and Safe \cite{EIP4337}. 

% These wallets also incorporate multi-factor authentication, multi-signature authorisation, and seamless interaction with various blockchain functions and external libraries securely \cite{di2020wallet}.

% Smart contract wallets now represent one of the latest technological advancements in the wallet landscape, enabling highly flexible and secure transaction management (see \autoref{sec:transaction_management} by utilising smart contracts to allow operations to be processed directly on the blockchain. 


% without private key generation (see \hyperref[algo:key-generation]{Algorithm} \autoref{algo:key-generation}) or storage (see \hyperref[algo:key-storage]{Algorithm} \autoref{algo:key-storage})

% Furthermore, the complexity of setting up and managing these wallets, along with the associated gas fees for deployment and operation can pose barriers to entry for some users, potentially limiting their widespread adoption.

\subsection{Custody}
\label{sec:design-cust}

The degree of \teal{$sk$} control by an entity or between one or more entities defines custody design. Custody setups include custodial, non-custodial and semi-custodial.

\subsubsection{Custodial}
\label{sec:custodial-wallets}

\teal{$sk$} is stored by a trusted custodian (e.g. Coinbase, Binance, Kraken) who signs user-initiated transactions in this model. The user relinquishes \teal{$sk$} security to the custodian who fully controls the wallet operations (see \autoref{sec:wallet_mechanism}, while the user solely crafts transaction messages. Although most of the design factors for custodial wallets are not disclosed (see \autoref{tab:wlt._taxonomy}), a classification of their design can be conducted using our framework. Two notable design variations exist: an omnibus setup, where the custodian aggregates and controls all users' funds under a few shared addresses, without a one-to-one correspondence between user accounts and addresses; and a segregated setup, where each user is assigned a unique blockchain address, with the custodian retaining control of the associated private keys (\teal{$sk$}) \cite{chalkias2022broken}.



% Custodial wallets, such Kraken, Coinbase Exchange, Bitbuy, and Xapo \cite{chalkias2022broken} facilitate an indirect blockchain interaction through a central authority which stores the private key and signs transactions on behalf of the user. The wallet initialisation (see \hyperref[algo:cryptocurrency-wallet]{Wallet Initialisation Algorithm}) begins as the user creates an account using email/username and \teal{$pw$}, and the \teal{$sk$} is generated by the custodian. The custodian typically employs a combination of hot and cold private key storage \cite{sans2022decentralized}. While users cannot directly access the private key, they can initiate custodian-signed transactions by providing their credentials (see \hyperref[algo:transaction-signing]{Transaction Generation Algorithm}). The custodian matches the transaction with peers to execute it (see \hyperref[algo:transaction-broadcast]{Transaction Broadcast Algorithm}). This model offers convenience, especially for novices or frequent traders, but it comes with security risks due to its centralised management structure, as user data is stored on servers. 

% The security and reliability of these wallets depend on the trustworthiness of the managing organisation's infrastructure. Furthermore, custodial wallets are subject to stringent regulatory oversight, including compliance with KYC and AML guidelines, which can impact user privacy \cite{vadlamani2023bridging}. Currently, custodial wallets provide functionality exclusively for managing tokens, not decentralised applications.

\subsubsection{Non-Custodial}
\label{sec:non-custodial-wallets}

In non-custodial wallet architectures, (e.g. Metamask, Phantom, Ledger) the user does not relinquish any control to any custodian party. Instead, a direct interaction between the user and the blockchain network exists in these setups with the user in full control of \teal{$sk$}, to facilitate all the wallet operations (see \autoref{sec:wallet_mechanism}). With full autonomy, the user is solely responsible for securing \teal{$sk$} and is more susceptible to insecure user interaction threats as well as other vulnerabilities (see \autoref{sec:threat_class}) and attacks such as social engineering attacks and malware-based attacks (see \autoref{sec:application-attacks}) which aim to exploit user negligence. While non-custodial wallets are expected to not have credential control, a few incidents in the past (e.g. Slope Wallet \cite{CoinTelegraph}) have resulted in \teal{$sk$} compromise due to poor implementation practices, insecure storage of sensitive information, or inadvertent leaks \cite{CoinTelegraph2022SlopeAttack}.


% (\teal{$enc\_secret\_key$}) are employed to store the private keys, with the user setting up a password for protection \cite{goodell2021development}. Users can access their stored private keys by decrypting (\teal{$dec\_secret\_key$}) them with the password to initiate transactions. They create transactions, sign them with their private keys (see \hyperref[algo:transaction-signing]{Transaction Generation Algorithm}), and broadcast them to the network (see \hyperref[algo:transaction-broadcast]{Transaction Broadcast Algorithm}). Non-custodial wallets, exemplified by MetaMask, Safe, and Ledger, are favoured by users who prioritise asset control, privacy, and security. The autonomy of non-custodial wallets comes with the responsibility of securing and managing private keys, posing a risk of irreversible loss if keys are misplaced or inadequately backed up.


\subsubsection{Shared-Custodial}
\label{sec:semi-custodial-wallets}

Shared-custodial wallets strike a balance between custodial and non-custodial models by enabling joint control of the secret key (\teal{$sk$}) between a user and a custodian. In this setup, the \teal{$sk$} is split or distributed across two or more parties, allowing the user to delegate a degree of transaction authorisation rights and trust to the custodian. This arrangement provides both parties with partial control over the wallet's signing and recovery operations. As a result, even if one party's security is compromised, the risk of a complete \teal{$sk$} compromise is mitigated. For example, Zengo’s operational model implements shared custody with \acf{mpc} by storing one part of the \teal{$sk$} on Zengo's centralised server, while the other part remains on the user's device \cite{zengo_rec}. Other shared custodian models are discussed in \autoref{sec:design-distr}.

% The degrees of control can be .

% Despite challenges in user experience, technological advancements are improving accessibility. Non-custodial wallets are less regulated than their custodial counterparts and offer a more flexible operational environment. Additionally, non-custodial crypto wallets can also be used to interact with smart contracts, providing users with a broader range of functionalities \cite{bowler2023non}.


% \subsection{Internet Connectivity}
% \label{sec:internet-connectivity}

% The internet connectivity dimension plays a crucial role in the security and accessibility of crypto wallets. 

% \subsubsection{Hot Wallets}
% \label{sec:hot-wallets}

% Hot wallets, characterised by their constant connectivity to the internet, play a crucial role in the cryptocurrency ecosystem due to their convenience and ease of access \cite{khanum2022exposure}. These wallets, including software wallets (see \autoref{sec:software-wallets}) such as Bitbuy, MetaMask, and smart contract wallets (see \autoref{sec:smart-contract-wallets}) like Safe and Argent, provide users with the ability to perform transactions quickly and efficiently. This feature makes them particularly suitable for frequent traders and those who need regular access to their digital assets. However, this constant online presence inherently increases the vulnerability of hot wallets to various cyber threats \autoref{sec:attack-framework}. 

% Despite these risks, hot wallets continue to evolve, with recent advancements focusing on enhancing security through measures such as biometric authentication and improved encryption methods.

% \subsubsection{Cold Wallets}
% \label{sec:cold-wallets}

% Cold wallets operate entirely offline and are considered to be the secured crypto storage mechanisms \cite{8966739}. These wallets include hardware wallets such as Ledger and Anchorage, as well as paper wallets, which are particularly adept at safeguarding large amounts of cryptocurrency from online threats. Software wallets managed by a few exchanges store \teal{$sk$} isolated from the internet, to mitigate the risk of cyber attacks. However, this security comes at the cost of reduced accessibility and convenience. Users of cold wallets must consider the risks associated with physical damage, loss, or theft, necessitating careful planning for secure storage and backup. 

% Despite these challenges, technological developments in cold wallets are gradually enhancing their user-friendliness, making them more accessible to a broader audience.

\subsection{Initialisation}
\label{sec:design-init}

This pertains to the creation of the wallet through \teal{$sk$} generation (see \autoref{sec:key_generation}) or contract deployment. During initialisation in smart contract wallets, user account contracts are created typically by interactions made by the relayer. In conventional wallets, the \teal{$sk$} generation scheme can be non-deterministic, deterministic, or hierarchical deterministic, depending on the degree of randomness and flexibility required. Another interesting design option is the key derivation factor (KDF) choice. Typically, most wallets (e.g. Ledger \cite{ledger_seed}) employ \acf{pbkdf}, however, novel research into threshold \acf{mfkdf} construction could influence current cryptographic designs \cite{NairMulti-FactorManagement, nair2023decentralizing}. While this improves security, more processing time and power may be required to generate the derived key \cite{trezor_memory}.

% add a little bit of maths notation.

% Another design factor is the distribution during key generation. 

% \subsubsection{Non-Deterministic}

% \subsubsection{Deterministic}

% \subsubsection{Hierarchical Deterministic (HD)}
% \label{sec:bip-32}
% Bitcoin Improvement Proposal 32 (BIP-32) \cite{bip32} introduces a system for creating Bitcoin wallets in a structured, hierarchical way. This system uses complex mathematical problems and secure hashing algorithms (specifically, elliptic curve discrete logarithm problem and HMAC-SHA512) to ensure that the wallet and its sub-wallets (or keys) are very secure. Each \quotes{child} key, or sub-wallet, is generated in a way that makes it extremely difficult to trace back to the \quotes{parent} key, adding an extra layer of security.




% \begin{table*}[!htbp]
\centering
\renewcommand{\arraystretch}{1.1}
\setlength{\tabcolsep}{1.25pt} % Adjust the column separation space here
\tiny
\begin{tabular}{llcccccccccccccccccccccccccccccccccccccccccccccccccccccccccccc}
\toprule
% \multicolumn{1}{c}{} &
  \multicolumn{1}{c}{\textbf{Name}} &
  \multicolumn{1}{c}{\textbf{{\hyperref[fig:wallet-evolution]{Est.}}}} &
  \multicolumn{3}{c}{\textbf{{\hyperref[sec:design-cust]{Cust.}}}} &
  \multicolumn{8}{c}{\textbf{{\hyperref[sec:infrastructure]{Infrastructure}}}} &
  \multicolumn{4}{c}{\textbf{{\hyperref[sec:design-init]{Init.}}}} &
  \multicolumn{3}{c}{\textbf{{\hyperref[sec:design-distr]{Distr.}}}} &
  \multicolumn{3}{c}{\textbf{{\hyperref[sec:design-author]{Authoris.}}}} &
  \multicolumn{3}{c}{\textbf{{\hyperref[sec:design-val]{Valid.}}}} &
  \multicolumn{5}{c}{\textbf{{\hyperref[sec:design-authen]{Authentication}}}} &
  \multicolumn{4}{c}{\textbf{{\hyperref[sec:design-rec]{Recovery}}}} &
  \multicolumn{2}{c}{\textbf{{\hyperref[sec:design-rec]{Trans.}}}} &
  \multicolumn{9}{c}{\textbf{{\hyperref[sec:design-rec]{Agnosticism}}}} &
  \multicolumn{15}{c}{\textbf{{\hyperref[sec:threat_framework]{Threat Occurrences}}}} 
  % \multicolumn{2}{c}{\textbf{{\hyperref[sec:attack-framework]{Atk.}}}} &
  \\ 
  \cmidrule(lr){6-13} \cmidrule(lr){14-17} 
  \cmidrule(lr){18-20} \cmidrule(lr){21-23} \cmidrule(lr){24-26} \cmidrule(lr){27-31} \cmidrule(lr){32-35} \cmidrule(lr){36-37} \cmidrule(lr){38-46} \cmidrule(lr){47-61}
  % \multicolumn{1}{c}{} &
  \multicolumn{1}{c}{} &
  \multicolumn{1}{c}{} &
  \multicolumn{3}{c}{} &
  \multicolumn{4}{c}{\textbf{Software}} &
  \multicolumn{4}{c}{\textbf{Hardware}} &
  \multicolumn{3}{c}{\textbf{}} &
  \multicolumn{1}{c}{\textbf{}} &
  \multicolumn{1}{c}{\textbf{}} &
    % \multicolumn{1}{c}{\textbf{Sgl.}} &
  \multicolumn{2}{c}{\textbf{}} &
    % \multicolumn{2}{c}{\textbf{Multi.}} &
  \multicolumn{2}{c}{\textbf{}} &
    % \multicolumn{2}{c}{\textbf{User}} &
  \multicolumn{1}{c}{\textbf{}} &
    % \multicolumn{1}{c}{\textbf{RL}} &
  \multicolumn{3}{c}{} &
  \multicolumn{5}{c}{} &
  \multicolumn{4}{c}{} &
  \multicolumn{2}{c}{} &
  \multicolumn{9}{c}{} &
  \multicolumn{15}{c}{} &
  % \rotatebox[origin=l]{90}{\cellcolor{r6}{$0\%$}} &
  % \rotatebox[origin=l]{90}{\cellcolor{r4}{$0\%$}} &
  % \rotatebox[origin=l]{90}{\cellcolor{r1}{$0\%$}} &
  % \rotatebox[origin=l]{90}{\cellcolor{r2}{$0\%$}} &
  % \rotatebox[origin=l]{90}{\cellcolor{r5}{$0\%$}} &
  % \rotatebox[origin=l]{90}{\cellcolor{r3}{$0\%$}} &
  % \rotatebox[origin=l]{90}{\cellcolor{r2}{$0\%$}} &
  % \rotatebox[origin=l]{90}{\cellcolor{r4}{$0\%$}} &
  % \rotatebox[origin=l]{90}{\cellcolor{r1}{$0\%$}} &
  % \rotatebox[origin=l]{90}{\cellcolor{r2}{$0\%$}} &
  % \rotatebox[origin=l]{90}{\cellcolor{r3}{$0\%$}} &
  % \rotatebox[origin=l]{90}{\cellcolor{r3}{$0\%$}} &
  % \rotatebox[origin=l]{90}{\cellcolor{r5}{$0\%$}} &
  % \rotatebox[origin=l]{90}{\cellcolor{r2}{$0\%$}} &
  % \rotatebox[origin=l]{90}{\cellcolor{r4}{$0\%$}} &
  \multicolumn{1}{c}{} 
  
  \\
  \cmidrule(lr){6-9} \cmidrule(lr){10-13} 
  % \cmidrule(lr){19-19} \cmidrule(lr){20-21}
 %  \multicolumn{1}{c}{\multirow{-3}{*}{\rotatebox[origin=l]{90}{\textbf{}}}}
 % &
   &
   \multicolumn{1}{c}{} &
   \rotatebox[origin=l]{90}{Non-Custodial} &
  \rotatebox[origin=l]{90}{Shared-Custodial} &
  \rotatebox[origin=l]{90}{Custodial} &
  \rotatebox[origin=l]{90}{Desktop} &
  \rotatebox[origin=l]{90}{Browser} &
  \rotatebox[origin=l]{90}{Mobile} &
  \rotatebox[origin=l]{90}{Smart} &
  \rotatebox[origin=l]{90}{USB} &
  \rotatebox[origin=l]{90}{Bluetooth} &
  \rotatebox[origin=l]{90}{NFC} &
  \rotatebox[origin=l]{90}{QR Code} &
  \rotatebox[origin=l]{90}{Non-Deterministic} &
  \rotatebox[origin=l]{90}{Deterministic (Non-HD)} &
  \rotatebox[origin=l]{90}{\acf{hd}} &
   \rotatebox[origin=l]{90}{Account Contract} &
  \rotatebox[origin=l]{90}{Single Distributed} &
  \rotatebox[origin=l]{90}{Multi-Sig} &
  \rotatebox[origin=l]{90}{\acf{mpc}} &
  \rotatebox[origin=l]{90}{Single SK} &
  \rotatebox[origin=l]{90}{Multiple SK} &
  \rotatebox[origin=l]{90}{Relayer} &
  \rotatebox[origin=l]{90}{Single PK Validation} &
  \rotatebox[origin=l]{90}{Multiple PK Validation} &
  \rotatebox[origin=l]{90}{Contract Validation} &
  \rotatebox[origin=l]{90}{PW/PIN} &
  \rotatebox[origin=l]{90}{2FA} &
  \rotatebox[origin=l]{90}{U2F} &
  \rotatebox[origin=l]{90}{Passkey} &
  \rotatebox[origin=l]{90}{Biometric} &
  \rotatebox[origin=l]{90}{12W Seed} &
  \rotatebox[origin=l]{90}{24W Seed} &
  \rotatebox[origin=l]{90}{Social} &
  \rotatebox[origin=l]{90}{DeRec} &
  \rotatebox[origin=l]{90}{Open-Source} &
  \rotatebox[origin=l]{90}{Closed-Source} &
  \rotatebox[origin=l]{90}{BTC} &
  \rotatebox[origin=l]{90}{ETH} &
  \rotatebox[origin=l]{90}{POLY} &
  \rotatebox[origin=l]{90}{BNB} &
  \rotatebox[origin=l]{90}{XRP} &
  \rotatebox[origin=l]{90}{HBAR} &
  \rotatebox[origin=l]{90}{SOL} &
  \rotatebox[origin=l]{90}{ADA} &
  \rotatebox[origin=l]{90}{AVAX} &
  \rotatebox[origin=l]{90}{Inadequate Encryption \cite{cve_15947, cve_37192}} &
  \rotatebox[origin=l]{90}{Insecure Network \cite{cve_33297, cve_14198, cve_17144}} &
  \rotatebox[origin=l]{90}{Library Vulnerability \cite{bitcore_lib, Ledger2023SecurityReport} } &
  \rotatebox[origin=l]{90}{Insecure Permission \cite{cve_32969, halborn_vuln}} &
  \rotatebox[origin=l]{90}{Predictable RNG \cite{cve_31290, cve_23660}} &
  % cve_14199,  tymokhanov2021alpha, fireblocks_23, chainlight
  % \cite{fireblocks_23, chainlight}}
  \rotatebox[origin=l]{90}{Sig. Verif. Logic Flaw \cite{cve_14199, fireblocks_23, AccountMedium, UncoveringVulnerability}} &
  \rotatebox[origin=l]{90}{Side-channel Leakage \cite{cve_14353, cve_14354, KrakenBlog}} &
  \rotatebox[origin=l]{90}{Data Remanence \cite{trezor_memory, trezor_medium}} &
  \rotatebox[origin=l]{90}{Data Manipulation \cite{trezor_memory, trezor_medium}} &
  \rotatebox[origin=l]{90}{Insecure Interactions \cite{ZengoZengo, thodex}} &
  \rotatebox[origin=l]{90}{Inadequate Authentication \cite{open_zeppelin}} &
  \rotatebox[origin=l]{90}{Input Validation Logic Flaw \cite{immunefi}} &
  \rotatebox[origin=l]{90}{Recovery Logic Flaw \cite{cve_15302}} &
  \multicolumn{1}{c}{\rotatebox[origin=l]{90}{Provider Compromise \cite{CoinTelegraph2022SlopeAttack}}} &
  \multicolumn{1}{c}{\rotatebox[origin=l]{90}{Insider Compromise \cite{Ledger2023SecurityReport}}} &
  % \# (\& \%)
  \multicolumn{1}{c}{\rotatebox[origin=l]{90}{Threat \# (\& \%)}} 
  % &
  % \multicolumn{1}{c}{\rotatebox[origin=l]{90}{Attacks \# (\& \%)}}
   \\
\midrule
% \multirow{19}{*}{\rotatebox[origin=l]{90}{Non-Custodial}} 
% & 
Bitcoin Core & 2009 & {\fullcirc} & {\emptycirc} & {\emptycirc} & {\fullcirc} & {\emptycirc} & {\emptycirc} & {\emptycirc} & {\emptycirc} & {\emptycirc} & {\emptycirc} & {\emptycirc} & {\fullcirc} & {\emptycirc} & {\fullcirc} & {\emptycirc} & {\fullcirc} & {\emptycirc} & {\emptycirc} & {\fullcirc} & {\emptycirc} & {\emptycirc}  & {\fullcirc} & {\emptycirc} & {\emptycirc} & {\fullcirc} & {\emptycirc} & {\emptycirc} & {\emptycirc} & {\emptycirc} & {\emptycirc} & {\emptycirc} & {\emptycirc} & {\emptycirc} & {\fullcirc} & {\emptycirc} & {\fullcirc} & {\emptycirc} & {\emptycirc} & {\emptycirc} & {\emptycirc} & {\emptycirc} & {\emptycirc} & {\emptycirc} & {\emptycirc} & {\fullcirc} & {\fullcirc} & {\fullcirc} & {\emptycirc} & {\emptycirc} & {\emptycirc} & {\emptycirc} & {\emptycirc} & {\emptycirc} & {\emptycirc} & {\emptycirc} & {\emptycirc} & {\emptycirc} & {\emptycirc} & {\emptycirc} & \cellcolor{o3}{$3$($20\%$)}

% &  \cellcolor{r6}{$0\%$}   
\\ 
% \cellcolor{g6}{$21$($49\%$)}
Electrum & 2011 & {\fullcirc} & {\emptycirc} & {\emptycirc} & {\fullcirc} & {\emptycirc} & {\emptycirc} & {\emptycirc} & {\emptycirc} & {\emptycirc} & {\emptycirc} & {\emptycirc} & {\fullcirc} & {\emptycirc} & {\fullcirc} & {\emptycirc} & {\fullcirc} & {\fullcirc} & {\emptycirc}  & {\fullcirc} & {\fullcirc} & {\emptycirc} & {\fullcirc} & {\fullcirc} & {\emptycirc} & {\fullcirc} & {\fullcirc} & {\emptycirc} & {\emptycirc} & {\emptycirc} & {\fullcirc} & {\emptycirc} & {\emptycirc} & {\emptycirc} & {\fullcirc} & {\emptycirc} & {\fullcirc} & {\emptycirc} & {\emptycirc} & {\emptycirc} & {\emptycirc} & {\emptycirc} & {\emptycirc} & {\emptycirc} & {\emptycirc} & {\emptycirc} & {\emptycirc} & {\emptycirc} & {\emptycirc} & {\emptycirc} & {\emptycirc} & {\emptycirc} & {\emptycirc} & {\emptycirc} & {\emptycirc} & {\emptycirc} & {\fullcirc} & {\emptycirc} & {\emptycirc} & {\emptycirc} & \cellcolor{o0}{$1$($7\%$)} 
% & \cellcolor{r2}{$0\%$}  
\\ 
Coinbase Ex. & 2012  & {\emptycirc} & {\emptycirc} & {\fullcirc} & {\emptycirc} & {\fullcirc} & {\fullcirc} & {\emptycirc} & {\emptycirc} & {\emptycirc} & {\emptycirc} & {\emptycirc} & {\emptycirc} & {\emptycirc} & {\emptycirc} & {\emptycirc} & {\emptycirc} & {\emptycirc} & {\emptycirc} & {\emptycirc} & {\emptycirc} & {\emptycirc} & {\emptycirc} & {\emptycirc} & {\emptycirc} & {\emptycirc} & {\emptycirc} & {\emptycirc} & {\emptycirc} & {\emptycirc} & {\emptycirc} & {\emptycirc} & {\emptycirc} & {\emptycirc} & {\emptycirc} & {\fullcirc} & {\fullcirc} & {\fullcirc} & {\fullcirc} & {\emptycirc} & {\fullcirc} & {\fullcirc} & {\fullcirc} & {\fullcirc} & {\fullcirc} & {\emptycirc} & {\emptycirc} & {\emptycirc} & {\emptycirc} & {\emptycirc} & {\emptycirc} & {\emptycirc} & {\emptycirc} & {\emptycirc} & {\emptycirc} & {\emptycirc} & {\emptycirc} & {\emptycirc} & {\emptycirc} & {\emptycirc} & $0$($0\%$)
% & \cellcolor{r0}{$0\%$}  
\\ 
% & 8.8M m*
% found out Trezor has multi-sig - i.e 2-of-3 need to reconfirm if it is 2 hardware devices or if there is a smart contract element
Trezor  & 2013 & {\fullcirc} & {\emptycirc} & {\emptycirc} & {\emptycirc} & {\emptycirc} & {\emptycirc} & {\emptycirc} & {\fullcirc} & {\emptycirc} & {\emptycirc} & {\emptycirc} & {\emptycirc} & {\emptycirc} & {\fullcirc} & {\emptycirc} & {\fullcirc} & {\fullcirc} & {\emptycirc} & {\fullcirc} & {\fullcirc} & {\emptycirc} & {\fullcirc} & {\fullcirc} & {\emptycirc} & {\fullcirc} & {\emptycirc} & {\fullcirc} & {\emptycirc} & {\emptycirc} & {\fullcirc} & {\fullcirc} & {\emptycirc} & {\emptycirc} & {\fullcirc} & {\emptycirc} & {\fullcirc} & {\fullcirc} & {\fullcirc} & {\fullcirc} & {\fullcirc} & {\emptycirc} & {\fullcirc} & {\fullcirc} & {\fullcirc} & {\emptycirc} & {\emptycirc} & {\emptycirc} & {\emptycirc} & {\emptycirc} & {\fullcirc} & {\fullcirc} & {\fullcirc} & {\fullcirc} & {\fullcirc} & {\emptycirc} & {\emptycirc} & {\emptycirc} & {\emptycirc} & {\emptycirc} & \cellcolor{o5}{$5$($33\%$})
% & \cellcolor{r4}{$0\%$}    
\\ 
% & 4
% & 2M
eToro & 2013 & {\emptycirc} & {\emptycirc} & {\fullcirc} & {\emptycirc} & {\fullcirc} & {\fullcirc} & {\emptycirc} & {\emptycirc} & {\emptycirc} & {\emptycirc} & {\emptycirc} & {\emptycirc} & {\emptycirc} & {\emptycirc} & {\emptycirc} & {\emptycirc} & {\emptycirc} & {\emptycirc} & {\emptycirc}  & {\emptycirc} & {\emptycirc} & {\emptycirc} & {\emptycirc} & {\emptycirc} & {\emptycirc} & {\emptycirc} & {\emptycirc} & {\emptycirc} & {\emptycirc} & {\emptycirc} & {\emptycirc} & {\emptycirc} & {\emptycirc} & {\emptycirc} & {\fullcirc} & {\fullcirc} & {\fullcirc} & {\fullcirc} & {\fullcirc} & {\fullcirc} & {\fullcirc} & {\fullcirc} & {\fullcirc} & {\fullcirc} & {\emptycirc} & {\emptycirc} & {\emptycirc} & {\emptycirc} & {\emptycirc} & {\emptycirc} & {\emptycirc} & {\emptycirc} & {\emptycirc} & {\emptycirc} & {\emptycirc} & {\emptycirc} & {\emptycirc} & {\emptycirc} & {\emptycirc} & $0$($0\%$)
% & \cellcolor{r2}{$0\%$}  
\\ 
% & 33M
Kraken Ex. & 2013 & {\emptycirc} & {\emptycirc} & {\fullcirc} & {\emptycirc} & {\fullcirc} & {\fullcirc} & {\emptycirc} & {\emptycirc} & {\emptycirc} & {\emptycirc} & {\emptycirc} & {\emptycirc}  & {\emptycirc} & {\emptycirc} & {\emptycirc} & {\emptycirc} & {\emptycirc}  & {\emptycirc} & {\emptycirc} & {\emptycirc} & {\emptycirc} & {\emptycirc} & {\emptycirc} & {\emptycirc} & {\emptycirc} & {\emptycirc} & {\emptycirc} & {\emptycirc} & {\emptycirc} & {\emptycirc} & {\emptycirc} & {\emptycirc} & {\emptycirc} & {\emptycirc} & {\fullcirc} & {\fullcirc} & {\fullcirc} & {\fullcirc} & {\emptycirc} & {\fullcirc} & {\emptycirc} & {\fullcirc} & {\fullcirc} & {\fullcirc} & {\emptycirc} & {\emptycirc} & {\emptycirc} & {\emptycirc} & {\emptycirc} & {\emptycirc} & {\emptycirc} & {\emptycirc} & {\emptycirc} & {\emptycirc} & {\emptycirc} & {\emptycirc} & {\emptycirc} & {\emptycirc} & {\emptycirc} & {$0$($0\%$)} 
% & \cellcolor{r3}{$0\%$}  
\\ 
Ledger & 2014 & {\fullcirc} & {\emptycirc} & {\emptycirc} & {\emptycirc} & {\emptycirc} & {\emptycirc} & {\emptycirc} & {\fullcirc} & {\fullcirc} & {\emptycirc} & {\emptycirc} & {\emptycirc} & {\emptycirc} & {\fullcirc} & {\emptycirc} & {\fullcirc} & {\emptycirc} & {\emptycirc} & {\fullcirc} & {\emptycirc} & {\emptycirc} & {\fullcirc} & {\emptycirc} & {\emptycirc} & {\fullcirc} & {\emptycirc} & {\fullcirc} & {\emptycirc} & {\emptycirc} & {\emptycirc} & {\fullcirc} & {\emptycirc} & {\emptycirc} & {\halfcirc} & {\emptycirc} & {\fullcirc} & {\fullcirc} & {\fullcirc} & {\fullcirc} & {\fullcirc} & {\fullcirc} & {\fullcirc} & {\fullcirc} & {\fullcirc} & {\emptycirc} & {\emptycirc} & {\fullcirc} & {\emptycirc} & {\emptycirc} & {\emptycirc} & {\fullcirc} & {\emptycirc} & {\emptycirc} & {\fullcirc} & {\emptycirc} & {\emptycirc} & {\emptycirc} & {\emptycirc} & {\fullcirc} & \cellcolor{o4}{$4$($27\%$)}
% & \cellcolor{r6}{$0\%$}  
\\ 
% & 6M
% & software open source - firmware closed source
Gemini & 2014 & {\emptycirc} & {\emptycirc} & {\fullcirc} & {\emptycirc} & {\fullcirc} & {\fullcirc} & {\emptycirc} & {\emptycirc} & {\emptycirc} & {\emptycirc} & {\emptycirc} & {\emptycirc} & {\emptycirc} & {\emptycirc} & {\emptycirc} & {\emptycirc} & {\emptycirc} & {\emptycirc} & {\emptycirc} & {\emptycirc} & {\emptycirc} & {\emptycirc} & {\emptycirc} & {\emptycirc} & {\emptycirc} & {\emptycirc} & {\emptycirc} & {\emptycirc} & {\emptycirc} & {\emptycirc} & {\emptycirc} & {\emptycirc} & {\emptycirc} & {\emptycirc} & {\fullcirc} & {\fullcirc} & {\fullcirc} & {\fullcirc} & {\emptycirc} & {\fullcirc} & {\emptycirc} & {\fullcirc} & {\emptycirc} & {\fullcirc} & {\emptycirc} & {\emptycirc} & {\emptycirc} & {\emptycirc} & {\emptycirc} & {\emptycirc} & {\emptycirc} & {\emptycirc} & {\emptycirc} & {\emptycirc} & {\emptycirc} & {\emptycirc} & {\emptycirc} & {\emptycirc} & {\emptycirc} & $0$($0\%$)
% & \cellcolor{r3}{$0\%$}  
\\
Metamask & 2016 & {\fullcirc} & {\emptycirc} & {\emptycirc} & {\emptycirc} & {\fullcirc} & {\fullcirc} & {\emptycirc} & {\emptycirc} & {\emptycirc} & {\emptycirc} & {\emptycirc} & {\emptycirc} & {\emptycirc} & {\fullcirc} & {\emptycirc} & {\fullcirc} & {\emptycirc} & {\emptycirc} & {\fullcirc} & {\emptycirc} & {\emptycirc} & {\fullcirc} & {\emptycirc} & {\emptycirc} & {\fullcirc} & {\emptycirc} & {\emptycirc} & {\emptycirc} & {\fullcirc} & {\fullcirc} & {\emptycirc} & {\emptycirc} & {\emptycirc} & {\fullcirc} & {\emptycirc} & {\emptycirc} & {\fullcirc} & {\fullcirc} & {\fullcirc} & {\emptycirc} & {\fullcirc} & {\emptycirc} & {\emptycirc} & {\fullcirc} & {\emptycirc} & {\emptycirc} & {\emptycirc} & {\fullcirc} & {\emptycirc} & {\emptycirc} & {\emptycirc} & {\emptycirc} & {\emptycirc} & {\emptycirc} & {\emptycirc} & {\emptycirc} & {\emptycirc} & {\emptycirc} & {\emptycirc} & \cellcolor{o0}{$1$($7\%$}) 
% & \cellcolor{r1}{$0\%$}  
\\ 
% & 30M m*
Bitbuy &  2016 & {\emptycirc} & {\emptycirc} & {\fullcirc} & {\emptycirc} & {\fullcirc} & {\fullcirc} & {\emptycirc} & {\emptycirc} & {\emptycirc} & {\emptycirc} & {\emptycirc} & {\emptycirc} & {\emptycirc} & {\emptycirc} & {\emptycirc} & {\emptycirc} & {\emptycirc} & {\emptycirc} & {\emptycirc} & {\emptycirc} & {\emptycirc} & {\emptycirc} & {\emptycirc} & {\emptycirc} & {\emptycirc} & {\emptycirc} & {\emptycirc} & {\emptycirc} & {\emptycirc} & {\emptycirc} & {\emptycirc} & {\emptycirc} & {\emptycirc} & {\emptycirc} & {\fullcirc} & {\fullcirc} & {\fullcirc} & {\fullcirc} & {\emptycirc} & {\fullcirc} & {\fullcirc} & {\fullcirc} & {\fullcirc} & {\fullcirc} & {\emptycirc} & {\emptycirc} & {\emptycirc} & {\emptycirc} & {\emptycirc} & {\emptycirc} & {\emptycirc} & {\emptycirc} & {\emptycirc} & {\emptycirc} & {\emptycirc} & {\emptycirc} & {\emptycirc} & {\emptycirc} & {\emptycirc} & $0$($0\%$)
% & \cellcolor{r3}{$0\%$}  
\\ 
% & 0.45M
Exodus & 2016 & {\fullcirc} & {\emptycirc} & {\emptycirc} & {\fullcirc} & {\fullcirc} & {\fullcirc} & {\emptycirc} & {\emptycirc} & {\emptycirc} & {\emptycirc} & {\emptycirc} & {\emptycirc} & {\emptycirc} & {\fullcirc} & {\emptycirc} & {\fullcirc} & {\emptycirc} & {\fullcirc} & {\fullcirc} & {\emptycirc} & {\emptycirc} & {\fullcirc} & {\emptycirc} & {\emptycirc} & {\fullcirc} & {\emptycirc} & {\emptycirc} & {\fullcirc} & {\fullcirc} & {\fullcirc} & {\emptycirc} & {\emptycirc} & {\emptycirc} & {\emptycirc} & {\fullcirc} & {\fullcirc} & {\fullcirc} & {\fullcirc} & {\fullcirc} & {\fullcirc} & {\fullcirc} & {\fullcirc} & {\fullcirc} & {\fullcirc} & {\emptycirc} & {\emptycirc} & {\emptycirc} & {\emptycirc} & {\emptycirc} & {\emptycirc} & {\emptycirc} &  {\emptycirc} & {\emptycirc} & {\fullcirc} & {\emptycirc} & {\emptycirc} & {\emptycirc} & {\emptycirc} & {\emptycirc} & \cellcolor{o0}{$1$($7\%$)} 
% & \cellcolor{r5}{$0\%$}   
\\ 
% & 0.8M m*
Binance Ex. & 2017 & {\emptycirc} & {\emptycirc} & {\fullcirc} & {\fullcirc} & {\fullcirc} & {\fullcirc} & {\emptycirc} & {\emptycirc} & {\emptycirc} & {\emptycirc} & {\emptycirc} & {\emptycirc} & {\emptycirc} & {\emptycirc} & {\emptycirc} & {\emptycirc} & {\emptycirc} & {\emptycirc} & {\emptycirc} & {\emptycirc} & {\emptycirc} & {\emptycirc} & {\emptycirc} & {\emptycirc} & {\emptycirc} & {\emptycirc} & {\emptycirc} & {\emptycirc} & {\emptycirc} & {\emptycirc} & {\emptycirc} & {\emptycirc} & {\emptycirc} & {\emptycirc} & {\fullcirc} & {\fullcirc} & {\fullcirc} & {\fullcirc} & {\fullcirc} & {\fullcirc} & {\fullcirc} & {\fullcirc} & {\fullcirc} & {\fullcirc} & {\emptycirc} & {\emptycirc} & {\emptycirc} & {\emptycirc} & {\emptycirc} & {\emptycirc} & {\emptycirc} & {\emptycirc} & {\emptycirc} & {\emptycirc} & {\emptycirc} & {\emptycirc} & {\emptycirc} & {\emptycirc} & {\emptycirc} & $0$($0\%$))
% & \cellcolor{r2}{$0\%$}  
\\ 
% & 200M
Trust Wlt. & 2017 & {\fullcirc} & {\emptycirc} & {\emptycirc} & {\emptycirc} & {\fullcirc} & {\fullcirc} & {\emptycirc} & {\emptycirc} & {\emptycirc} & {\emptycirc} & {\emptycirc} & {\emptycirc} & {\emptycirc} & {\fullcirc} & {\emptycirc} & {\fullcirc} & {\emptycirc} & {\halfcirc} & {\fullcirc} & {\emptycirc} & {\emptycirc} & {\fullcirc} & {\emptycirc} & {\emptycirc} & {\fullcirc} & {\emptycirc} & {\emptycirc} & {\emptycirc} & {\fullcirc} & {\fullcirc} & {\emptycirc} & {\emptycirc} & {\emptycirc}  & {\fullcirc} & {\emptycirc} & {\fullcirc} & {\fullcirc} & {\fullcirc} & {\fullcirc} & {\fullcirc} & {\emptycirc} & {\fullcirc} & {\fullcirc} & {\fullcirc} & {\emptycirc} & {\emptycirc} & {\emptycirc} & {\emptycirc} & {\fullcirc} & {\emptycirc} & {\emptycirc} & {\emptycirc} & {\emptycirc} &  {\emptycirc} & {\emptycirc} & {\emptycirc} & {\emptycirc} & {\emptycirc} & {\emptycirc} & \cellcolor{o0}{$1$($7\%$)} 
% & \cellcolor{r1}{$0\%$}  
\\ 
% & 2
% & 130M
Argent & 2017 & {\fullcirc} & {\emptycirc} & {\emptycirc} & {\emptycirc} & {\fullcirc} & {\fullcirc} & {\fullcirc} & {\emptycirc} & {\emptycirc} & {\emptycirc} & {\emptycirc} & {\emptycirc} & {\fullcirc} & {\emptycirc} & {\fullcirc} & {\emptycirc} & {\fullcirc} & {\emptycirc} & {\emptycirc} & {\fullcirc} & {\fullcirc} & {\emptycirc} & {\emptycirc} & {\fullcirc} & {\emptycirc} & {\emptycirc} & {\emptycirc} & {\fullcirc} & {\emptycirc} & {\emptycirc} & {\emptycirc} & {\fullcirc} & {\emptycirc} & {\fullcirc} & {\emptycirc} & {\emptycirc} & {\fullcirc} & {\fullcirc} & {\emptycirc} & {\emptycirc} & {\emptycirc} & {\emptycirc} & {\emptycirc} & {\emptycirc} & {\emptycirc} & {\emptycirc} & {\emptycirc} & {\emptycirc} & {\emptycirc} & {\fullcirc} & {\emptycirc} & {\emptycirc} & {\emptycirc} & {\emptycirc} & {\emptycirc} & {\emptycirc} & {\fullcirc} & {\emptycirc} & {\emptycirc} & \cellcolor{o2}{$2$($13\%$)} 
% & \cellcolor{r2}{$0\%$}   
\\ 
CoinEx & 2017 & {\emptycirc} & {\emptycirc} & {\fullcirc} & {\emptycirc} & {\fullcirc} & {\fullcirc} & {\emptycirc} & {\emptycirc} & {\emptycirc} & {\emptycirc} & {\emptycirc} & {\emptycirc} & {\emptycirc} & {\emptycirc} & {\emptycirc} & {\emptycirc} & {\emptycirc} & {\emptycirc} & {\emptycirc} & {\emptycirc} & {\emptycirc} & {\emptycirc} & {\emptycirc} & {\emptycirc} & {\emptycirc} & {\emptycirc} & {\emptycirc} & {\emptycirc} & {\emptycirc} & {\emptycirc} & {\emptycirc} & {\emptycirc} & {\emptycirc} & {\emptycirc} & {\fullcirc} & {\fullcirc} & {\fullcirc} & {\fullcirc} & {\fullcirc} & {\fullcirc} & {\fullcirc} & {\fullcirc} & {\fullcirc} & {\fullcirc} & {\emptycirc} & {\emptycirc} & {\emptycirc} & {\emptycirc} & {\emptycirc} & {\emptycirc} & {\emptycirc} & {\emptycirc} & {\emptycirc} & {\emptycirc} & {\emptycirc} & {\emptycirc} & {\emptycirc} & {\emptycirc} & {\emptycirc} & $0$($0\%$))
% & \cellcolor{r2}{$0\%$}  
\\ 
% \FilledCircle
 % & 5M 
Safe (Gnosis) & 2017 & {\fullcirc} & {\emptycirc} & {\emptycirc} & {\emptycirc} & {\emptycirc} & {\fullcirc} & {\fullcirc} & {\emptycirc} & {\emptycirc} & {\emptycirc} & {\emptycirc} & {\emptycirc} & {\fullcirc} & {\emptycirc} & {\fullcirc} & {\emptycirc} & {\fullcirc} & {\emptycirc} & {\emptycirc} & {\fullcirc} & {\fullcirc} & {\emptycirc} & {\emptycirc} & {\fullcirc} & {\emptycirc} & {\emptycirc} & {\emptycirc} & {\fullcirc} & {\emptycirc} & {\emptycirc} & {\emptycirc} & {\fullcirc} & {\emptycirc} &  {\fullcirc} & {\emptycirc} & {\emptycirc} & {\fullcirc} & {\emptycirc} & {\emptycirc} & {\emptycirc} & {\emptycirc} & {\emptycirc} & {\emptycirc} & {\emptycirc} & {\emptycirc} & {\emptycirc} & {\emptycirc} & {\emptycirc} & {\emptycirc} & {\fullcirc} & {\emptycirc} & {\emptycirc} & {\emptycirc} & {\emptycirc} & {\fullcirc} & {\emptycirc} & {\emptycirc} & {\emptycirc} & {\emptycirc} & \cellcolor{o2}{$2$($13\%$)} 
% & \cellcolor{r2}{$0\%$}   
\\ 
% & 1.6M m*
Atomic & 2017 & {\fullcirc} & {\emptycirc} & {\emptycirc} & {\fullcirc} & {\emptycirc} & {\fullcirc} & {\emptycirc} & {\emptycirc} & {\emptycirc} & {\emptycirc} & {\emptycirc} & {\emptycirc} & {\emptycirc} & {\emptycirc} & {\fullcirc} & {\fullcirc} & {\emptycirc} & {\emptycirc} & {\fullcirc} & {\emptycirc} & {\emptycirc} & {\fullcirc} & {\emptycirc} & {\emptycirc} & {\fullcirc} & {\emptycirc} & {\emptycirc} & {\emptycirc} & {\emptycirc} & {\fullcirc} & {\emptycirc} & {\emptycirc} & {\emptycirc} & {\emptycirc} & {\fullcirc} & {\fullcirc} & {\fullcirc} & {\fullcirc} & {\fullcirc} & {\fullcirc} & {\fullcirc} & {\fullcirc} & {\fullcirc} & {\fullcirc} & {\emptycirc} & {\emptycirc} & {\emptycirc} & {\emptycirc} & {\fullcirc} & {\fullcirc} & {\emptycirc} & {\emptycirc} & {\emptycirc} & {\emptycirc} & {\emptycirc} & {\emptycirc} &  {\emptycirc} & {\emptycirc} & {\emptycirc} & \cellcolor{o2}{$2$($13\%$)} 
% & \cellcolor{r3}{$0\%$}  
\\
% & 10M
Tangem & 2017 & {\fullcirc} & {\emptycirc} & {\emptycirc} & {\emptycirc} & {\emptycirc} & {\emptycirc} & {\emptycirc} & {\emptycirc} & {\emptycirc} & {\fullcirc} & {\emptycirc} & {\emptycirc} & {\emptycirc} & {\fullcirc} & {\emptycirc} & {\fullcirc} & {\emptycirc} & {\emptycirc} & {\fullcirc} & {\emptycirc} & {\emptycirc} & {\fullcirc} & {\emptycirc} & {\emptycirc} & {\fullcirc} & {\emptycirc} & {\emptycirc} & {\emptycirc} & {\fullcirc} & {\fullcirc} & {\fullcirc} & {\emptycirc} & {\emptycirc} & {\fullcirc} & {\emptycirc} & {\fullcirc} & {\fullcirc} & {\emptycirc} & {\fullcirc} & {\fullcirc} & {\emptycirc} & {\fullcirc} & {\emptycirc} & {\fullcirc} & {\emptycirc} & {\emptycirc} & {\emptycirc} & {\emptycirc} & {\emptycirc} & {\emptycirc} & {\emptycirc} & {\emptycirc} & {\emptycirc} & {\emptycirc} & {\emptycirc} & {\emptycirc} & {\emptycirc} & {\emptycirc} & {\emptycirc} & $0$($0\%$)
% & \cellcolor{r0}{$0\%$}  
\\
Ngrave & 2018 & {\fullcirc} & {\emptycirc} & {\emptycirc} & {\emptycirc} & {\emptycirc} & {\emptycirc} & {\emptycirc} & {\emptycirc} & {\emptycirc} & {\emptycirc} & {\fullcirc} & {\emptycirc} & {\emptycirc} & {\fullcirc} & {\emptycirc} & {\fullcirc} & {\emptycirc} & {\emptycirc} & {\fullcirc} & {\emptycirc} & {\emptycirc} & {\fullcirc} & {\emptycirc} & {\emptycirc} & {\fullcirc} & {\emptycirc} & {\emptycirc} & {\emptycirc} & {\fullcirc} & {\emptycirc} & {\fullcirc} & {\emptycirc} & {\emptycirc} & {\emptycirc} & {\fullcirc} & {\fullcirc} & {\fullcirc} & {\emptycirc} & {\fullcirc} & {\fullcirc} & {\emptycirc} & {\fullcirc} & {\emptycirc} & {\fullcirc} & {\emptycirc} & {\emptycirc} & {\emptycirc} & {\emptycirc} & {\emptycirc} & {\emptycirc} & {\emptycirc} & {\emptycirc} & {\emptycirc} & {\emptycirc} & {\emptycirc} & {\emptycirc} & {\emptycirc} & {\emptycirc} & {\emptycirc} & $0$($0\%$)
% & \cellcolor{r0}{$0\%$}   
\\ 
Zengo & 2018 & {\emptycirc} & {\fullcirc} & {\emptycirc} & {\emptycirc} & {\emptycirc} & {\fullcirc} & {\emptycirc} & {\emptycirc} & {\emptycirc} & {\emptycirc} & {\emptycirc} & {\emptycirc} & {\fullcirc} & {\emptycirc} & {\fullcirc} & {\emptycirc} & {\emptycirc} & {\fullcirc} & {\fullcirc} & {\emptycirc} & {\emptycirc} & {\fullcirc} & {\emptycirc} & {\emptycirc} & {\emptycirc} & {\fullcirc} & {\emptycirc} & {\emptycirc} & {\fullcirc} & {\emptycirc} & {\emptycirc} & {\emptycirc} & {\emptycirc} & {\fullcirc} & {\emptycirc} & {\fullcirc} & {\fullcirc} & {\fullcirc} & {\fullcirc} & {\emptycirc} & {\emptycirc} & {\emptycirc} & {\emptycirc} & {\emptycirc} & {\emptycirc} & {\emptycirc} & {\emptycirc} & {\emptycirc} & {\emptycirc} & {\fullcirc} & {\emptycirc} & {\emptycirc} & {\emptycirc} & {\emptycirc} & {\emptycirc}  & {\emptycirc} & {\emptycirc} & {\emptycirc} & {\emptycirc} & \cellcolor{o1}{$1$($7\%$)}
% & \cellcolor{r1}{$0\%$}  
\\ 
% & 1m
% Need to confirm coinbase wallet because it seems it has some smart features but it also has seed phrase
% Whats the difference between passkey and biometrics
Coinbase Wlt  & 2019 & {\fullcirc} & {\emptycirc} & {\emptycirc} & {\emptycirc} & {\fullcirc} & {\fullcirc} & {\fullcirc} & {\emptycirc} & {\emptycirc} & {\emptycirc} & {\emptycirc} & {\emptycirc} & {\emptycirc} & {\emptycirc} & {\fullcirc} & {\fullcirc} & {\emptycirc} & {\emptycirc} & {\fullcirc} & {\emptycirc} & {\fullcirc} & {\emptycirc} & {\emptycirc} & {\fullcirc} & {\emptycirc} & {\emptycirc} & {\emptycirc} & {\fullcirc} & {\emptycirc} & {\fullcirc} & {\emptycirc} & {\fullcirc} & {\emptycirc} & {\emptycirc} & {\fullcirc} & {\fullcirc} & {\fullcirc} & {\fullcirc} & {\fullcirc} & {\fullcirc} & {\emptycirc} & {\fullcirc} & {\fullcirc} & {\fullcirc} & {\emptycirc} & {\emptycirc} & {\emptycirc} & {\emptycirc} & {\emptycirc} & {\emptycirc} & {\emptycirc} & {\emptycirc} & {\emptycirc} & {\fullcirc} & {\emptycirc} & {\emptycirc} & {\emptycirc} & {\emptycirc} & {\emptycirc} & \cellcolor{o1}{$1$($7\%$)} 
% & \cellcolor{r0}{$0\%$}  
\\ 
Biconomy & 2019 & {\fullcirc} & {\emptycirc} & {\emptycirc} &  {\emptycirc} & {\emptycirc} & {\emptycirc} & {\fullcirc} & {\emptycirc} & {\emptycirc} & {\emptycirc} & {\emptycirc} & {\emptycirc} & {\emptycirc} & {\emptycirc} & {\fullcirc} & {\fullcirc} & {\emptycirc} & {\emptycirc}  & {\fullcirc} & {\emptycirc} & {\fullcirc} & {\emptycirc} & {\emptycirc} & {\fullcirc} & {\emptycirc} & {\emptycirc} & {\emptycirc} & {\fullcirc} & {\emptycirc} & {\emptycirc} & {\emptycirc} & {\fullcirc} & {\emptycirc} & {\fullcirc} & {\emptycirc} & {\emptycirc} & {\fullcirc} & {\fullcirc} & {\fullcirc} & {\emptycirc} & {\emptycirc} & {\emptycirc} & {\emptycirc} & {\fullcirc} & {\emptycirc} & {\emptycirc} & {\emptycirc} & {\emptycirc} & {\emptycirc} & {\fullcirc} & {\emptycirc} & {\emptycirc} & {\emptycirc} & {\emptycirc} & {\emptycirc} & {\emptycirc} & {\emptycirc} & {\emptycirc} & {\emptycirc} & \cellcolor{o1}{$1$($7\%$)}  
% & \cellcolor{r2}{$0\%$}  
\\ 
% & 5M 
Web3Auth & 2020 & {\emptycirc} & {\fullcirc} & {\emptycirc} & {\emptycirc} & {\emptycirc} & {\fullcirc} & {\emptycirc} & {\emptycirc} & {\emptycirc} & {\emptycirc} & {\emptycirc} & {\emptycirc} & {\fullcirc} & {\emptycirc} & {\fullcirc} & {\emptycirc} & {\emptycirc} & {\fullcirc} & {\emptycirc} & {\emptycirc} & {\fullcirc} & {\emptycirc} & {\emptycirc} & {\fullcirc} & {\emptycirc} & {\emptycirc} & {\fullcirc} & {\fullcirc} & {\emptycirc} & {\emptycirc} & {\emptycirc} & {\fullcirc} & {\emptycirc} & {\fullcirc} & {\emptycirc} & {\emptycirc} & {\fullcirc} & {\fullcirc} & {\fullcirc} & {\emptycirc} & {\emptycirc} & {\emptycirc} & {\emptycirc} & {\fullcirc} & {\emptycirc} & {\emptycirc} & {\emptycirc} & {\emptycirc} & {\emptycirc} & {\emptycirc} & {\emptycirc} & {\emptycirc} & {\emptycirc} & {\emptycirc} & {\fullcirc} & {\emptycirc} & {\emptycirc} & {\emptycirc} & {\emptycirc} & \cellcolor{o1}{$1$($7\%$)}  
% & \cellcolor{r2}{$0\%$}  
\\ 
Brave & 2021 & {\fullcirc} & {\emptycirc} & {\emptycirc} & {\emptycirc} & {\fullcirc} & {\fullcirc} & {\emptycirc} & {\emptycirc} & {\emptycirc} & {\emptycirc} & {\emptycirc} & {\emptycirc} & {\emptycirc} & {\fullcirc} & {\emptycirc} & {\fullcirc} & {\emptycirc} & {\emptycirc} & {\fullcirc} & {\emptycirc} & {\emptycirc} & {\fullcirc} & {\emptycirc} & {\emptycirc} & {\fullcirc} & {\emptycirc} & {\emptycirc} & {\emptycirc} & {\fullcirc} & {\fullcirc} & {\emptycirc} & {\emptycirc} & {\emptycirc} & {\fullcirc} & {\emptycirc} & {\fullcirc} & {\fullcirc} & {\fullcirc} & {\emptycirc} & {\emptycirc} & {\emptycirc} & {\fullcirc} & {\emptycirc} & {\emptycirc} & {\emptycirc} & {\fullcirc} & {\emptycirc} & {\fullcirc} & {\emptycirc} & {\emptycirc} & {\emptycirc} & {\emptycirc} & {\emptycirc} & {\emptycirc} & {\emptycirc} & {\emptycirc} & {\emptycirc} & {\emptycirc} & {\emptycirc} & \cellcolor{o3}{$2$($13\%$)}  
% & \cellcolor{r2}{$0\%$}  
\\ 
% & 70M m*
Phantom & 2021 & {\fullcirc} & {\emptycirc} & {\emptycirc} & {\emptycirc} & {\fullcirc} & {\fullcirc} & {\emptycirc} & {\emptycirc} & {\emptycirc} & {\emptycirc} & {\emptycirc} & {\emptycirc} & {\emptycirc} & {\fullcirc} & {\emptycirc} & {\fullcirc} & {\emptycirc} & {\emptycirc} & {\fullcirc} & {\emptycirc} & {\emptycirc} & {\fullcirc} & {\emptycirc} & {\emptycirc} & {\fullcirc} & {\emptycirc} & {\emptycirc} & {\emptycirc} & {\fullcirc} & {\fullcirc} & {\fullcirc} & {\emptycirc} & {\emptycirc} & {\emptycirc} & {\fullcirc} & {\fullcirc} & {\fullcirc} & {\fullcirc} & {\emptycirc} & {\emptycirc} & {\emptycirc} & {\fullcirc} & {\emptycirc} & {\emptycirc} & {\emptycirc} & {\fullcirc} & {\emptycirc} & {\fullcirc} & {\emptycirc} & {\emptycirc} & {\emptycirc} & {\emptycirc} & {\emptycirc} & {\emptycirc} & {\emptycirc} & {\emptycirc} & {\emptycirc} & {\emptycirc} & {\emptycirc} & \cellcolor{o3}{$2$($13\%$)}  
% & \cellcolor{r2}{$0\%$}  
\\ 
% & 7M m* 
Slope & 2021 & {\fullcirc} & {\emptycirc} & {\emptycirc} & {\emptycirc} & {\fullcirc} & {\fullcirc} & {\emptycirc} & {\emptycirc} & {\emptycirc} & {\emptycirc} & {\emptycirc} & {\emptycirc} & {\emptycirc} & {\fullcirc} & {\emptycirc} & {\fullcirc} & {\emptycirc} & {\emptycirc} & {\fullcirc} & {\emptycirc} & {\emptycirc} & {\fullcirc} & {\emptycirc} & {\emptycirc} & {\fullcirc} & {\emptycirc} & {\emptycirc} & {\emptycirc} & {\fullcirc} & {\fullcirc} & {\emptycirc} & {\emptycirc} & {\emptycirc} & {\fullcirc} & {\emptycirc} & {\emptycirc} & {\fullcirc} & {\emptycirc} & {\fullcirc} & {\emptycirc} & {\emptycirc} & {\fullcirc} & {\emptycirc} & {\emptycirc} & {\fullcirc} & {\emptycirc} & {\emptycirc} & {\emptycirc} & {\emptycirc} & {\emptycirc} & {\emptycirc} & {\emptycirc} & {\emptycirc} & {\emptycirc} & {\emptycirc} & {\emptycirc} & {\emptycirc} & {\fullcirc}  & {\emptycirc} & \cellcolor{o3}{$2$($13\%$)} 
% & \cellcolor{r1}{$0\%$}  
\\ 
HashPack  & 2021 & {\fullcirc} & {\emptycirc} & {\emptycirc} & {\emptycirc} & {\fullcirc} & {\fullcirc} & {\emptycirc} & {\emptycirc} & {\emptycirc} & {\emptycirc} & {\emptycirc} & {\emptycirc} & {\emptycirc} & {\fullcirc} & {\emptycirc} & {\fullcirc} & {\emptycirc} & {\emptycirc} & {\fullcirc} & {\emptycirc} & {\emptycirc} & {\fullcirc} & {\emptycirc} & {\emptycirc} & {\fullcirc} & {\emptycirc} & {\emptycirc} & {\emptycirc} & {\fullcirc} & {\fullcirc} & {\emptycirc} & {\emptycirc} & {\fullcirc} & {\emptycirc} & {\fullcirc} & {\emptycirc} & {\emptycirc} & {\emptycirc} & {\emptycirc} & {\emptycirc} & {\emptycirc} & {\emptycirc} & {\emptycirc} & {\emptycirc} & {\emptycirc} & {\emptycirc} & {\emptycirc} & {\emptycirc} & {\emptycirc} & {\emptycirc} & {\emptycirc} & {\emptycirc} & {\emptycirc} & {\emptycirc} & {\emptycirc} & {\emptycirc} & {\emptycirc} & {\emptycirc} & {\emptycirc} & $0$($0\%$)
% & \cellcolor{r0}{$0\%$}  
\\ 
Binance Web3 & 2023 & {\emptycirc} & {\fullcirc} & {\emptycirc} & {\emptycirc} & {\emptycirc} & {\fullcirc} & {\emptycirc} & {\emptycirc} & {\emptycirc} & {\emptycirc} & {\emptycirc} & {\emptycirc} & {\fullcirc} & {\emptycirc} & {\fullcirc} & {\emptycirc} & {\emptycirc} & {\fullcirc} & {\fullcirc} & {\emptycirc} & {\emptycirc} & {\fullcirc} & {\emptycirc} & {\emptycirc} & {\emptycirc} & {\emptycirc} & {\emptycirc} & {\fullcirc} & {\fullcirc} & {\emptycirc} & {\emptycirc} & {\emptycirc} & {\emptycirc} & {\fullcirc} & {\emptycirc} & {\emptycirc} & {\fullcirc} & {\fullcirc} & {\fullcirc} & {\emptycirc} & {\emptycirc} & {\fullcirc} & {\emptycirc} & {\fullcirc} & {\emptycirc} & {\emptycirc} & {\emptycirc} & {\emptycirc} & {\emptycirc} & {\fullcirc} & {\emptycirc} & {\emptycirc} & {\emptycirc} & {\emptycirc} & {\emptycirc} & {\emptycirc} & {\emptycirc} & {\emptycirc} & {\emptycirc} & \cellcolor{o1}{$1$($7\%$)} 
% & \cellcolor{r1}{$0\%$}  
\\ 
Kraken Wlt. & 2024 & {\fullcirc} & {\emptycirc} & {\emptycirc} & {\emptycirc} & {\emptycirc} & {\fullcirc} & {\emptycirc} & {\emptycirc} & {\emptycirc} & {\emptycirc} & {\emptycirc} & {\emptycirc} & {\emptycirc} & {\fullcirc} & {\fullcirc} & {\fullcirc} & {\emptycirc} & {\emptycirc} & {\fullcirc} & {\emptycirc} & {\emptycirc} & {\fullcirc} & {\emptycirc} & {\emptycirc} & {\emptycirc} & {\emptycirc} & {\emptycirc} & {\fullcirc} & {\fullcirc} & {\fullcirc} & {\emptycirc} & {\emptycirc} & {\emptycirc} & {\fullcirc} & {\emptycirc} & {\fullcirc} & {\fullcirc} & {\fullcirc} & {\emptycirc} & {\emptycirc} & {\emptycirc} & {\fullcirc} & {\emptycirc} & {\emptycirc} & {\emptycirc} & {\emptycirc} & {\emptycirc} & {\emptycirc} & {\emptycirc} & {\emptycirc} & {\emptycirc} & {\emptycirc} & {\emptycirc} & {\emptycirc} & {\emptycirc} & {\emptycirc} & {\emptycirc} & {\emptycirc} & {\emptycirc} & $0$($0\%$)
% & \cellcolor{r1}{$0\%$}  
\\ 
\midrule
\multicolumn{3}{c}{\textbf{Summary}} &
\multicolumn{17}{c}{\textbf{Highest Occurrence: Signature Verification Logic Flaw}} &
\multicolumn{5}{c}{\cellcolor{o3}{$7$($21\%$)}} &
\multicolumn{20}{c}{} &
\multicolumn{16}{r}{\textbf{Total Vulnerabilities Detected in All Wallets}} &
$33$($100\%$)  
% \cellcolor{o0}{$33$($100\%$)} 

 \\ 
% \midrule
% \multirow{7}{*}{\rotatebox[origin=l]{90}{Custodial}} 
% &  
% \multirow{-7}{*}{\rotatebox[origin=l]{90}{Custodial}}
% & 
% {llccccccccccccccccccccccccccccccccccccccccccccccccccccccccccc}
% \multicolumn{5}{l}{} &
%   \multicolumn{5}{l}{} &
%   \multicolumn{5}{l}{} &
%   \multicolumn{5}{l}{} &
%   \multicolumn{5}{c}{} &
%   \multicolumn{5}{l}{} &
%   \multicolumn{5}{l}{} &
%   \multicolumn{5}{l}{} &
%    \multicolumn{5}{c}{\textbf{{Vulnerabilities No \& \%}}} &
%    \cellcolor{g6}{($0\%$)} &
% \cellcolor{g6}{($0\%$)} &
% \cellcolor{g6}{($0\%$)} &
% \cellcolor{g6}{($0\%$)} &
% \cellcolor{g6}{($0\%$)} &
%   \cellcolor{g6}{($0\%$)} &
% \cellcolor{g6}{($0\%$)} &
% \cellcolor{g6}{($0\%$)} &
% \cellcolor{g6}{($0\%$)} &
% \cellcolor{g6}{($0\%$)} &
%   \cellcolor{g6}{($0\%$)} &
% \cellcolor{g6}{($0\%$)} &
% \cellcolor{g6}{($0\%$)} &
% \cellcolor{g6}{($0\%$)} &
% \cellcolor{g6}{($0\%$)} 
% \\
\bottomrule
\end{tabular}
\vspace{1ex} % Add space before the caption
\caption{Industry Wallet design variations and identified threats. ( \fullcirc : include, \halfcirc : part-inclusion, \emptycirc : not include)
}
\label{tab:wlt._taxonomy}
\end{table*}



\subsection{Distribution}
\label{sec:design-distr}

This is the degree of authorisation (see \autoref{sec:design-author}) or \teal{$sk$} distribution between storage mechanisms.  Single or variations of shared authorisation between multiple user devices, multiple users or a user and a custodian (see \autoref{sec:design-cust} are observable setups. Single setups allow for sole authorisation by a user or custodian while authorisation is distributed in the shared setup to avoid a single point of failure. Multi-distributed designs typically exist in two forms; smart wallet-enabled multi-sig (on-chain multi-sig) and threshold \acs{mpc}. On-chain multi-sig typically have authorisation dispersed between multiple private keys \teal{$sk$}, while \acs{mpc} wallets divide a single \teal{$sk$} into \enquote{key shares} which are then distributed \cite{bip11, Lindell2020SecureComputation}. Design flexibility in some \acs{mpc} wallets also allows for a hierarchical sub-shard distribution (e.g. Web3Auth) if necessary \cite{web3_auth}. While both offer authorisation distribution, trade-offs exist between the two (see \autoref{sec:design-author} \& \autoref{sec:design-val}).

\subsection{Authentication}
\label{sec:design-authen}

We define authentication as the process of verifying the legitimate wallet owner before granting access, either by decrypting \teal{$sk$} with the \acs{kek} (see \autoref{sec:key-storage}) or by employing other methods defined within the underlying logic. Existing authentication methods include single-factor (\teal{$pw$} or \teal{$PIN$}), multi-factor authentication and novel password-abstracted authentication methods such as passkey enabled by smart contract or MPC wallets. For instance, the Binance Web3 MPC wallet splits cryptographic key shards between the user, a cloud provider (e.g., iCloud or Google Drive), and Binance itself, requiring user authentication to retrieve at least two of the three shards to approve transactions \cite{Binance2023EmbracingWallet}.

% would be also interesting to also talk about authentication between the client and the server, semi-custodian methods of authentication - authentication is very concerned with control ie custodians and non-custodians -- how is the user authentication before the private key is decrypted or accessed or before an authorisation is possible. 

% would be interesting to look at the operations of
% mpc wallets 
% tangem wallets - is very interesting too

% Transaction authorisation can be designed to either require a single or multiple signature, alternatively, a relayer can trigger the process, following approval on a user's behalf.

% transaction authorisation is triggered on the user's behalf by a relayer.

% or a more complex architecture where by a relayer signs transactions on behalf of users.

% in the context of wallet design 


\subsection{Authorisation}
\label{sec:design-author}

Authorisation in the context of wallets is defined as a direct or indirect confirmation of a state change transaction (see \autoref{sec:def:tnx}) by a single signature or multiple signatures. An indirect authorisation is executed via a centralised or decentralised relayer's signature who signs on behalf of a user (e,g, ERC-4337 architecture \cite{erc4337}). \acs{mpc} key shards produce a single signature, while distributed among various parties with individual public addresses hidden. Multi-sig smart wallets demonstrate authorisation through multiple signatures, each associated with an individual public address, which does not enhance privacy since all involved addresses are visible on the blockchain. ERC-4337-enabled smart contract wallets employ a relayer (bundler) to aggregate multiple users' state transfer messages into a single authorised transition. Other factors which influence the authorisation setup include the signature scheme choice.

% requires multiple private keys \teal{$sk$} to sign a transaction 

% Smart multi-sig traditionally required multiple private keys \teal{$sk$} to sign a transaction 


% Rather than safeguarding a single \teal{$sk$}, \ac{mpc} wallets divide \teal{$sk$} into \enquote{shards} or \enquote{key shares} and distribute them among multiple parties, which are then used to sign the transaction. 

% With custom logic and recent developments, smart contract wallets can sign messages 

% mpc blockchain agnostic

% mpc is private

% lower fees - faster transactions

% multi-sig is no protocol agnostic

% multi-sig is not operational flexible


\subsection{Validation}
\label{sec:design-val}
Transaction validation is typically referred to as authentication against the blockchain using the user's \teal{$pk$} \cite{Homoliak2024SoK:Factors, Homoliak2020SmartOTPs:Wallets}. In addition to single distributed wallets, \acs{mpc} wallet also produces a single \teal{$pk$} from key shards, which can be employed to validate the transaction. On the other hand, native multi-sig wallets validate each party's public key. ERC-4337 allows more flexible validation variations, as an EntryPoint contract validates and executes state changes sent by authenticated users \cite{erc4337}. Additionally, recent developments (ERC-1271 \cite{Ethereum2018ERC-1271:Contracts} \& ERC-6492 \cite{Ethereum2023ERC-6492:Contracts}) have enabled standardised and improved signature validation methods for smart contracts. 

% % Figure environment removed


\subsection{Recovery and Other Design Factors}
\label{sec:design-rec}

Recovery serves as a method to retrieve \teal{$sk$} or lost transaction authorisation rights and typically follows the initialisation (see \autoref{sec:design-init}) and the distribution \autoref{sec:design-distr} setup selected. Single-distributed wallets are generally recovered using one method such as \teal{$rdm\_seed$}, while multi-distributed recovery varies based on the implementation. Recovery has different cost implications in smart contract wallets and \acs{mpc} wallets. \acs{mpc} wallets are recovered off-chain and have no costs, while Smart contract wallets (e.g. Coinbase Smart Wallet) generally require you to pay a network for account recovery. However, a smart contract wallet, Argent circumvents this by offering users off-chain recovery \cite{argent_rec}. 

\autoref{tab:wlt._taxonomy} shows other design factors such as transparency and agnosticism. The underlying mechanism of existing hardware, software, non-custodial and semi-custodial wallets often function in degrees of transparency. While open-source models benefit from public audits, open knowledge of mechanisms can provide an advantage to an adversary. Blockchain agnosticism is another important factor. Integration with multiple blockchain networks defines blockchain-agnosticism. As blockchains often operate as fragmented systems, heterogeneous designs foster enhanced interoperability.

% \begin{itemize}
%     \item Bloom Filter Multi-Party Computation Paper \cite{Han2021AnFilter}
%     \item Hardware and Smart Contract Paper \cite{Homoliak2020SmartOTPs:Wallets}
%     \item A Novel Cryptocurrency Wallet Management Scheme Based on Decentralized Multi-Constrained Derangement \cite{He2019ADerangement}
%     % \item New Secure Approach to Backup Cryptocurrency Wallets \cite{rezaeighaleh2019new}
%     \item SBLWT: A Secure Blockchain Lightweight Wallet Based on Trustzone \cite{Dai2018SBLWT:Trustzone}
%     \item A Social-Network-Based Cryptocurrency Wallet-Management Scheme \cite{He2018AScheme}
%     \item A Secure and Flexible Blockchain-Based Offline Payment Protocol \cite{Jie2024AProtocol}
%     \item Multilayered Defense-in-Depth Architecture for Cryptocurrency Wallet \cite{Rezaeighaleh2020MultilayeredWallet}
%     \item Deterministic Sub-Wallet for Cryptocurrencies \cite{Rezaeighaleh2019DeterministicCryptocurrencies}
%     \item An Anti-Quantum Transaction Authentication Approach in Blockchain \cite{Yin2017AnBlockchain}
%     \item CryptoVault - A Secure Hardware Wallet for Decentralized Key Management \cite{Lehto2021}
%     \item Trustzone-based secure lightweight wallet for hyperledger fabric \cite{Dai2021Trustzone-basedFabric}
%     \item Secure wallet-assisted offline bitcoin payments with double-spender revocation \cite{Dai2021Trustzone-basedFabric}
%     \item Enhancing Cold Wallet Security with Native Multi-Signature schemes in Centralized Exchanges \cite{Ebrahimi2021EnhancingExchanges}
%     \item A Two-Party Hierarchical Deterministic Wallets in Practice \cite{Chuang2023APractice}
%     \item Shared-Custodial Password-Authenticated Deterministic Wallets \cite{Das2024Shared-CustodialWallets}
%     % this paper's related work also shows shared custodian settings of  a client and a server jointly generating a key
% \end{itemize}





% \subsection{Security Design Evaluation}
% \label{sec:design-eval}

% We propose a framework to qualitatively evaluate the security of existing wallets on individual design factors as shown in \autoref{tab:design_eval}. We denote the wallet user by \teal{$U$} and the adversary by \teal{$A$}.

% \subsubsection{Custody*}
% \label{sec:design-eval-cust}

% The threat is 
% While the security architecture of custodial 
% The entity authorised to control \teal{$sk$}. 

% \begin{itemize}
%     \item \textbf{High Security:}
%     \item \textbf{Medium Security:}
%     \item \textbf{Low Security:} 
% \end{itemize}

% \subsubsection{Infrastructure}
% \label{sec:design-eval-infra}

% We define this by the degree to which the attack surface area reduces or increases due to inherent vulnerabilities and defence implementations within the infrastructure. All wallet types, with the exception of smart contract wallets, possess traditional cryptographic vulnerabilities. The browser wallet, for instance, exposes \teal{$sk$} to new infrastructural vulnerabilities without providing any notable defence implementations. On the other hand, the hardware wallet provides specific defence implementations that reduce the attack surface area of \teal{$A$}.


% % which is locally stored 

% \begin{itemize}
%     \item \textbf{Typical Traditional Vulnerabilities =:} 
%     \item \textbf{Inherent Infrastructural Vulnerabilities -:}
%     \item \textbf{Inherent Infrastructural Security +:} 
% \end{itemize}


% approach -- 
% 1. define typical wallet vulnerabilities i.e. common vulnerabilities on most wallets i.e cryptographic vulnerabilities common in desktop, browser, mobile and hardware

% 2. define inherent infrastructural vulnerabilities i.e. new vulnerabilities that exist as a result of deployment on a given infrastructure i.e. browser, or smart contracts

% 3. inherent infrastructural security i.e. design in infrastructure which limits attack surface to a degree. 

% 4. classification as high medium or low:
% low: browser: vulnerability to both traditional vulnerabilities -- introduced new vulnerabilities based on the infrastructure -- infrastructure does not provide a significant defence implementation
% medium: desktop: vulnerability to both traditional vulnerabilities -- introduced new vulnerabilities based on the infrastructure -- infrastructure does not provide a significant defence implementation
% high: hardware: vulnerability to traditional -- introduced no or a few vulnerabilities -- infrastructure provides significant defence

% need to decide where to place mobile and smart contract
% -- both are likely to be placed in medium because -- mobile would inherent some vulnerabilities i.e. in OS e.g. android OS vulnerabilities but might also inherent some security features -- smart contract on the other hand removes traditional vulnerabilities but introduces new vulnerabilities -- and also new forms of security. 

% to attacks which leverage the nature

% \begin{itemize}
%     \item \textbf{High Security:} Hardware Airgapped \& Hardware USB
%     \item \textbf{Medium Security:} Desktop Wallet \& Mobile
%     \item \textbf{Low Security:} Browser Wallets
% \end{itemize}

% \subsubsection{Initialisation***}
% \label{sec:design-eval-init}

% We define the initialisation score by determinants which ensure a higher or lower degree of security during this stage. For instance, \acs{dkg} offers a higher level than non-distributed key generation. 

% \begin{itemize}
%     \item \textbf{High Security:}
%     \item \textbf{Medium Security:}
%     \item \textbf{Low Security:} 
% \end{itemize}



% \begin{table}[!h]
\centering
\setlength{\tabcolsep}{2.5pt} % Adjust the column separation space here
\begin{tabular}{llllllll}
\toprule
\multirow{2}{*}{\textbf{Factors}} & \multirow{2}{*}{\textbf{Designs}} & \multicolumn{3}{c}{\textbf{Security}} & \multicolumn{3}{c}{\textbf{Privacy}} \\
\cmidrule(lr){3-5} \cmidrule(lr){6-8}
& & \multicolumn{1}{c}{\rotatebox[origin=c]{90}{Low}} & \multicolumn{1}{c}{\rotatebox[origin=c]{90}{Medium}} & \multicolumn{1}{c}{\rotatebox[origin=c]{90}{High}} & \multicolumn{1}{c}{\rotatebox[origin=c]{90}{Low}} & \multicolumn{1}{c}{\rotatebox[origin=c]{90}{Medium}} & \multicolumn{1}{c}{\rotatebox[origin=c]{90}{High}} \\
\addlinespace[1ex]
\toprule
\multirow{5}{*}{Infrastructure} & Desktop Wallet & \emptycirc & \fullcirc & \emptycirc & \emptycirc & \emptycirc & \fullcirc \\
& Browser Wallet & \fullcirc & \emptycirc & \emptycirc & \fullcirc & \emptycirc & \emptycirc \\
& Mobile Wallet & \emptycirc & \fullcirc & \emptycirc & \emptycirc & \emptycirc & \emptycirc \\
& Hardware Wallet & \emptycirc & \emptycirc & \fullcirc & \emptycirc & \emptycirc & \emptycirc \\
& Smart Wallet & \emptycirc & \fullcirc & \emptycirc & \fullcirc & \emptycirc & \emptycirc \\
\midrule
\multirow{3}{*}{Distribution} & Single & \fullcirc & \emptycirc & \emptycirc & \fullcirc & \emptycirc & \emptycirc \\
& Multi-sig & \emptycirc & \emptycirc & \fullcirc & \fullcirc & \emptycirc & \emptycirc \\
& MPC & \emptycirc & \emptycirc & \fullcirc & \emptycirc & \fullcirc & \emptycirc \\
\midrule
\multirow{3}{*}{Authentication} & None & \fullcirc & \emptycirc & \emptycirc & \emptycirc & \emptycirc & \emptycirc \\
& Single & \emptycirc & \fullcirc & \emptycirc & \emptycirc & \emptycirc & \emptycirc \\
& MFA & \emptycirc & \emptycirc & \fullcirc & \emptycirc & \emptycirc & \emptycirc \\
\midrule
\multirow{3}{*}{Recovery} & File & \emptycirc & \emptycirc & \emptycirc & \emptycirc & \emptycirc & \emptycirc \\
& Seed & \emptycirc & \emptycirc & \emptycirc & \emptycirc & \emptycirc & \emptycirc \\
& Social & \emptycirc & \emptycirc & \emptycirc & \emptycirc & \emptycirc & \emptycirc \\
\midrule
\multirow{3}{*}{Custody} & Non-Custodial & \emptycirc & \emptycirc & \emptycirc & \emptycirc & \fullcirc & \emptycirc \\
& Shared-Custodial & \emptycirc & \emptycirc & \emptycirc & \emptycirc & \emptycirc & \emptycirc \\
& Custodial & \emptycirc & \fullcirc & \emptycirc & \emptycirc & \fullcirc & \emptycirc \\
\midrule
\multirow{3}{*}{Initialisation} & Non-Deterministic & \emptycirc & \emptycirc & \emptycirc & \emptycirc & \emptycirc & \emptycirc \\
& Deterministic & \emptycirc & \emptycirc & \emptycirc & \emptycirc & \emptycirc & \emptycirc \\
& HD & \emptycirc & \emptycirc & \emptycirc & \emptycirc & \emptycirc & \fullcirc \\
\bottomrule
\end{tabular}
\vspace{1ex} % Add space before the caption
\caption{Factors categorised by design, security, and privacy}
\label{tab:design_eval}
\end{table}



% \subsubsection{Distribution}
% \label{sec:design-eval-dist}

% We define the distribution by the number of parties to which \teal{$sk$}  is distributed and the relationship between parties i.e. parties which share common themes such as organisation, which may aid the adversary.

% \begin{itemize}
%     \item \textbf{High Security:}
%     \item \textbf{Medium Security:}
%     \item \textbf{Low Security:} Non-distributed
% \end{itemize}


% \paragraph{Number of Parties}
% \label{sec:design-eval}

% \paragraph{Controlling Entity}
% \label{sec:design-eval}

% \subsubsection{Authentication}
% \label{sec:design-eval-authen}

% We define sub-metrics by the level or levels of authentication required before access to \teal{$sk$} or the respective account(s).

% \begin{itemize}
%     \item \textbf{High Security:} MFA
%     \item \textbf{Medium Security:} 2FA
%     \item \textbf{Low Security:} Single authentication
% \end{itemize}


% \subsubsection{Authorisation***}
% \label{sec:design-eval-author}

% \subsubsection{Validation*}
% \label{sec:design-eval-val}

% We define this metric by the security of the validation system. For instance, validation in wallets ranges from cryptographic validation to pre-defined logic validation.

% \begin{itemize}
%     \item \textbf{High Security:} Cryptographically Validated
%     \item \textbf{Medium Security:} 
%     \item \textbf{Low Security:} Contract Validation
% \end{itemize}

% \subsubsection{Recovery}
% \label{sec:design-eval-rec}

% We define this metric by the degree of vulnerability of the recovery method to attacks.

% \begin{itemize}
%     \item \textbf{High Security:} No Seed Phrase / Social Recovery
%     \item \textbf{Medium Security:} Seed Phrase
%     \item \textbf{Low Security:} Files or Duplicated Private Key
% \end{itemize}


% --
% just commented out 
% privacy evaluation 

% \subsection{Privacy Design Evaluation}
% \label{sec:design-eval}

% We also map individual wallet design considerations to privacy implications on \teal{$U$}. 

% \subsubsection{Custody*}
% \label{sec:design-eval-cust}

% \begin{itemize}
%     \item \textbf{High Privacy:} Non-Custodial 
%     \item \textbf{Medium Privacy:} Shared-Custodial
%     \item \textbf{Low Privacy:} Custodial
% \end{itemize}

% \subsubsection{Infrastructure}
% \label{sec:design-eval-infra}

% The browser wallet exposes \teal{$sk$} which is locally stored 

% \begin{itemize}
%     \item \textbf{High Privacy:} Desktop
%     \item \textbf{Medium Privacy:} Hardware
%     \item \textbf{Low Privacy:} Browser Mobile Smart Contract
%     % Hardware Airgapped \& Hardware USB \& Desktop Wallet \& Mobile
% \end{itemize}

% \subsubsection{Initialisation}
% \label{sec:design-eval-init}

% \begin{itemize}
%     \item \textbf{High Privacy:} \acs{hd}
%     \item \textbf{Medium Privacy:}
%     \item \textbf{Low Privacy:} Non-Deterministic
% \end{itemize}

% \subsubsection{Distribution}
% \label{sec:design-eval-dist}

% \begin{itemize}
%     \item \textbf{High Privacy:} \acs{mpc}
%     \item \textbf{Medium Privacy:}
%     \item \textbf{Low Privacy:} Multi-sig
% \end{itemize}

% \subsubsection{Validation*}
% \label{sec:design-eval-val}

% Transaction validation design decisions influence the degree of privacy in wallets.

% \begin{itemize}
%     \item \textbf{High Privacy: Full Node Validation} 
%     \item \textbf{Medium Privacy: Some Light Client} 
%     \item \textbf{Low Privacy: Servers} 
% \end{itemize}

% \input{Exhibits/Types- Academia}


\subsection{Discussion}
\label{sec:tax_discussion}
\subsubsection{Insight 1: Infrastructure Evolution}

The key management infrastructure dimension in our taxonomy has been a product of evolution influenced by two major factors; security and functionality, as shown in \autoref{fig:wallet-evolution}.

% Figure environment removed

\begin{table*}[!htbp]
\centering
\renewcommand{\arraystretch}{1.1}
\setlength{\tabcolsep}{1.25pt} % Adjust the column separation space here
\tiny
\begin{tabular}{llcccccccccccccccccccccccccccccccccccccccccccccccccccccccccccc}
\toprule
% \multicolumn{1}{c}{} &
  \multicolumn{1}{c}{\textbf{Name}} &
  \multicolumn{1}{c}{\textbf{{\hyperref[fig:wallet-evolution]{Est.}}}} &
  \multicolumn{3}{c}{\textbf{{\hyperref[sec:design-cust]{Cust.}}}} &
  \multicolumn{8}{c}{\textbf{{\hyperref[sec:infrastructure]{Infrastructure}}}} &
  \multicolumn{4}{c}{\textbf{{\hyperref[sec:design-init]{Init.}}}} &
  \multicolumn{3}{c}{\textbf{{\hyperref[sec:design-distr]{Distr.}}}} &
  \multicolumn{3}{c}{\textbf{{\hyperref[sec:design-author]{Authoris.}}}} &
  \multicolumn{3}{c}{\textbf{{\hyperref[sec:design-val]{Valid.}}}} &
  \multicolumn{5}{c}{\textbf{{\hyperref[sec:design-authen]{Authentication}}}} &
  \multicolumn{4}{c}{\textbf{{\hyperref[sec:design-rec]{Recovery}}}} &
  \multicolumn{2}{c}{\textbf{{\hyperref[sec:design-rec]{Trans.}}}} &
  \multicolumn{9}{c}{\textbf{{\hyperref[sec:design-rec]{Agnosticism}}}} &
  \multicolumn{15}{c}{\textbf{{\hyperref[sec:threat_framework]{Threat Occurrences}}}} 
  % \multicolumn{2}{c}{\textbf{{\hyperref[sec:attack-framework]{Atk.}}}} &
  \\ 
  \cmidrule(lr){6-13} \cmidrule(lr){14-17} 
  \cmidrule(lr){18-20} \cmidrule(lr){21-23} \cmidrule(lr){24-26} \cmidrule(lr){27-31} \cmidrule(lr){32-35} \cmidrule(lr){36-37} \cmidrule(lr){38-46} \cmidrule(lr){47-61}
  % \multicolumn{1}{c}{} &
  \multicolumn{1}{c}{} &
  \multicolumn{1}{c}{} &
  \multicolumn{3}{c}{} &
  \multicolumn{4}{c}{\textbf{Software}} &
  \multicolumn{4}{c}{\textbf{Hardware}} &
  \multicolumn{3}{c}{\textbf{}} &
  \multicolumn{1}{c}{\textbf{}} &
  \multicolumn{1}{c}{\textbf{}} &
    % \multicolumn{1}{c}{\textbf{Sgl.}} &
  \multicolumn{2}{c}{\textbf{}} &
    % \multicolumn{2}{c}{\textbf{Multi.}} &
  \multicolumn{2}{c}{\textbf{}} &
    % \multicolumn{2}{c}{\textbf{User}} &
  \multicolumn{1}{c}{\textbf{}} &
    % \multicolumn{1}{c}{\textbf{RL}} &
  \multicolumn{3}{c}{} &
  \multicolumn{5}{c}{} &
  \multicolumn{4}{c}{} &
  \multicolumn{2}{c}{} &
  \multicolumn{9}{c}{} &
  \multicolumn{15}{c}{} &
  % \rotatebox[origin=l]{90}{\cellcolor{r6}{$0\%$}} &
  % \rotatebox[origin=l]{90}{\cellcolor{r4}{$0\%$}} &
  % \rotatebox[origin=l]{90}{\cellcolor{r1}{$0\%$}} &
  % \rotatebox[origin=l]{90}{\cellcolor{r2}{$0\%$}} &
  % \rotatebox[origin=l]{90}{\cellcolor{r5}{$0\%$}} &
  % \rotatebox[origin=l]{90}{\cellcolor{r3}{$0\%$}} &
  % \rotatebox[origin=l]{90}{\cellcolor{r2}{$0\%$}} &
  % \rotatebox[origin=l]{90}{\cellcolor{r4}{$0\%$}} &
  % \rotatebox[origin=l]{90}{\cellcolor{r1}{$0\%$}} &
  % \rotatebox[origin=l]{90}{\cellcolor{r2}{$0\%$}} &
  % \rotatebox[origin=l]{90}{\cellcolor{r3}{$0\%$}} &
  % \rotatebox[origin=l]{90}{\cellcolor{r3}{$0\%$}} &
  % \rotatebox[origin=l]{90}{\cellcolor{r5}{$0\%$}} &
  % \rotatebox[origin=l]{90}{\cellcolor{r2}{$0\%$}} &
  % \rotatebox[origin=l]{90}{\cellcolor{r4}{$0\%$}} &
  \multicolumn{1}{c}{} 
  
  \\
  \cmidrule(lr){6-9} \cmidrule(lr){10-13} 
  % \cmidrule(lr){19-19} \cmidrule(lr){20-21}
 %  \multicolumn{1}{c}{\multirow{-3}{*}{\rotatebox[origin=l]{90}{\textbf{}}}}
 % &
   &
   \multicolumn{1}{c}{} &
   \rotatebox[origin=l]{90}{Non-Custodial} &
  \rotatebox[origin=l]{90}{Shared-Custodial} &
  \rotatebox[origin=l]{90}{Custodial} &
  \rotatebox[origin=l]{90}{Desktop} &
  \rotatebox[origin=l]{90}{Browser} &
  \rotatebox[origin=l]{90}{Mobile} &
  \rotatebox[origin=l]{90}{Smart} &
  \rotatebox[origin=l]{90}{USB} &
  \rotatebox[origin=l]{90}{Bluetooth} &
  \rotatebox[origin=l]{90}{NFC} &
  \rotatebox[origin=l]{90}{QR Code} &
  \rotatebox[origin=l]{90}{Non-Deterministic} &
  \rotatebox[origin=l]{90}{Deterministic (Non-HD)} &
  \rotatebox[origin=l]{90}{\acf{hd}} &
   \rotatebox[origin=l]{90}{Account Contract} &
  \rotatebox[origin=l]{90}{Single Distributed} &
  \rotatebox[origin=l]{90}{Multi-Sig} &
  \rotatebox[origin=l]{90}{\acf{mpc}} &
  \rotatebox[origin=l]{90}{Single SK} &
  \rotatebox[origin=l]{90}{Multiple SK} &
  \rotatebox[origin=l]{90}{Relayer} &
  \rotatebox[origin=l]{90}{Single PK Validation} &
  \rotatebox[origin=l]{90}{Multiple PK Validation} &
  \rotatebox[origin=l]{90}{Contract Validation} &
  \rotatebox[origin=l]{90}{PW/PIN} &
  \rotatebox[origin=l]{90}{2FA} &
  \rotatebox[origin=l]{90}{U2F} &
  \rotatebox[origin=l]{90}{Passkey} &
  \rotatebox[origin=l]{90}{Biometric} &
  \rotatebox[origin=l]{90}{12W Seed} &
  \rotatebox[origin=l]{90}{24W Seed} &
  \rotatebox[origin=l]{90}{Social} &
  \rotatebox[origin=l]{90}{DeRec} &
  \rotatebox[origin=l]{90}{Open-Source} &
  \rotatebox[origin=l]{90}{Closed-Source} &
  \rotatebox[origin=l]{90}{BTC} &
  \rotatebox[origin=l]{90}{ETH} &
  \rotatebox[origin=l]{90}{POLY} &
  \rotatebox[origin=l]{90}{BNB} &
  \rotatebox[origin=l]{90}{XRP} &
  \rotatebox[origin=l]{90}{HBAR} &
  \rotatebox[origin=l]{90}{SOL} &
  \rotatebox[origin=l]{90}{ADA} &
  \rotatebox[origin=l]{90}{AVAX} &
  \rotatebox[origin=l]{90}{Inadequate Encryption \cite{cve_15947, cve_37192}} &
  \rotatebox[origin=l]{90}{Insecure Network \cite{cve_33297, cve_14198, cve_17144}} &
  \rotatebox[origin=l]{90}{Library Vulnerability \cite{bitcore_lib, Ledger2023SecurityReport} } &
  \rotatebox[origin=l]{90}{Insecure Permission \cite{cve_32969, halborn_vuln}} &
  \rotatebox[origin=l]{90}{Predictable RNG \cite{cve_31290, cve_23660}} &
  % cve_14199,  tymokhanov2021alpha, fireblocks_23, chainlight
  % \cite{fireblocks_23, chainlight}}
  \rotatebox[origin=l]{90}{Sig. Verif. Logic Flaw \cite{cve_14199, fireblocks_23, AccountMedium, UncoveringVulnerability}} &
  \rotatebox[origin=l]{90}{Side-channel Leakage \cite{cve_14353, cve_14354, KrakenBlog}} &
  \rotatebox[origin=l]{90}{Data Remanence \cite{trezor_memory, trezor_medium}} &
  \rotatebox[origin=l]{90}{Data Manipulation \cite{trezor_memory, trezor_medium}} &
  \rotatebox[origin=l]{90}{Insecure Interactions \cite{ZengoZengo, thodex}} &
  \rotatebox[origin=l]{90}{Inadequate Authentication \cite{open_zeppelin}} &
  \rotatebox[origin=l]{90}{Input Validation Logic Flaw \cite{immunefi}} &
  \rotatebox[origin=l]{90}{Recovery Logic Flaw \cite{cve_15302}} &
  \multicolumn{1}{c}{\rotatebox[origin=l]{90}{Provider Compromise \cite{CoinTelegraph2022SlopeAttack}}} &
  \multicolumn{1}{c}{\rotatebox[origin=l]{90}{Insider Compromise \cite{Ledger2023SecurityReport}}} &
  % \# (\& \%)
  \multicolumn{1}{c}{\rotatebox[origin=l]{90}{Threat \# (\& \%)}} 
  % &
  % \multicolumn{1}{c}{\rotatebox[origin=l]{90}{Attacks \# (\& \%)}}
   \\
\midrule
% \multirow{19}{*}{\rotatebox[origin=l]{90}{Non-Custodial}} 
% & 
Bitcoin Core & 2009 & {\fullcirc} & {\emptycirc} & {\emptycirc} & {\fullcirc} & {\emptycirc} & {\emptycirc} & {\emptycirc} & {\emptycirc} & {\emptycirc} & {\emptycirc} & {\emptycirc} & {\fullcirc} & {\emptycirc} & {\fullcirc} & {\emptycirc} & {\fullcirc} & {\emptycirc} & {\emptycirc} & {\fullcirc} & {\emptycirc} & {\emptycirc}  & {\fullcirc} & {\emptycirc} & {\emptycirc} & {\fullcirc} & {\emptycirc} & {\emptycirc} & {\emptycirc} & {\emptycirc} & {\emptycirc} & {\emptycirc} & {\emptycirc} & {\emptycirc} & {\fullcirc} & {\emptycirc} & {\fullcirc} & {\emptycirc} & {\emptycirc} & {\emptycirc} & {\emptycirc} & {\emptycirc} & {\emptycirc} & {\emptycirc} & {\emptycirc} & {\fullcirc} & {\fullcirc} & {\fullcirc} & {\emptycirc} & {\emptycirc} & {\emptycirc} & {\emptycirc} & {\emptycirc} & {\emptycirc} & {\emptycirc} & {\emptycirc} & {\emptycirc} & {\emptycirc} & {\emptycirc} & {\emptycirc} & \cellcolor{o3}{$3$($20\%$)}

% &  \cellcolor{r6}{$0\%$}   
\\ 
% \cellcolor{g6}{$21$($49\%$)}
Electrum & 2011 & {\fullcirc} & {\emptycirc} & {\emptycirc} & {\fullcirc} & {\emptycirc} & {\emptycirc} & {\emptycirc} & {\emptycirc} & {\emptycirc} & {\emptycirc} & {\emptycirc} & {\fullcirc} & {\emptycirc} & {\fullcirc} & {\emptycirc} & {\fullcirc} & {\fullcirc} & {\emptycirc}  & {\fullcirc} & {\fullcirc} & {\emptycirc} & {\fullcirc} & {\fullcirc} & {\emptycirc} & {\fullcirc} & {\fullcirc} & {\emptycirc} & {\emptycirc} & {\emptycirc} & {\fullcirc} & {\emptycirc} & {\emptycirc} & {\emptycirc} & {\fullcirc} & {\emptycirc} & {\fullcirc} & {\emptycirc} & {\emptycirc} & {\emptycirc} & {\emptycirc} & {\emptycirc} & {\emptycirc} & {\emptycirc} & {\emptycirc} & {\emptycirc} & {\emptycirc} & {\emptycirc} & {\emptycirc} & {\emptycirc} & {\emptycirc} & {\emptycirc} & {\emptycirc} & {\emptycirc} & {\emptycirc} & {\emptycirc} & {\fullcirc} & {\emptycirc} & {\emptycirc} & {\emptycirc} & \cellcolor{o0}{$1$($7\%$)} 
% & \cellcolor{r2}{$0\%$}  
\\ 
Coinbase Ex. & 2012  & {\emptycirc} & {\emptycirc} & {\fullcirc} & {\emptycirc} & {\fullcirc} & {\fullcirc} & {\emptycirc} & {\emptycirc} & {\emptycirc} & {\emptycirc} & {\emptycirc} & {\emptycirc} & {\emptycirc} & {\emptycirc} & {\emptycirc} & {\emptycirc} & {\emptycirc} & {\emptycirc} & {\emptycirc} & {\emptycirc} & {\emptycirc} & {\emptycirc} & {\emptycirc} & {\emptycirc} & {\emptycirc} & {\emptycirc} & {\emptycirc} & {\emptycirc} & {\emptycirc} & {\emptycirc} & {\emptycirc} & {\emptycirc} & {\emptycirc} & {\emptycirc} & {\fullcirc} & {\fullcirc} & {\fullcirc} & {\fullcirc} & {\emptycirc} & {\fullcirc} & {\fullcirc} & {\fullcirc} & {\fullcirc} & {\fullcirc} & {\emptycirc} & {\emptycirc} & {\emptycirc} & {\emptycirc} & {\emptycirc} & {\emptycirc} & {\emptycirc} & {\emptycirc} & {\emptycirc} & {\emptycirc} & {\emptycirc} & {\emptycirc} & {\emptycirc} & {\emptycirc} & {\emptycirc} & $0$($0\%$)
% & \cellcolor{r0}{$0\%$}  
\\ 
% & 8.8M m*
% found out Trezor has multi-sig - i.e 2-of-3 need to reconfirm if it is 2 hardware devices or if there is a smart contract element
Trezor  & 2013 & {\fullcirc} & {\emptycirc} & {\emptycirc} & {\emptycirc} & {\emptycirc} & {\emptycirc} & {\emptycirc} & {\fullcirc} & {\emptycirc} & {\emptycirc} & {\emptycirc} & {\emptycirc} & {\emptycirc} & {\fullcirc} & {\emptycirc} & {\fullcirc} & {\fullcirc} & {\emptycirc} & {\fullcirc} & {\fullcirc} & {\emptycirc} & {\fullcirc} & {\fullcirc} & {\emptycirc} & {\fullcirc} & {\emptycirc} & {\fullcirc} & {\emptycirc} & {\emptycirc} & {\fullcirc} & {\fullcirc} & {\emptycirc} & {\emptycirc} & {\fullcirc} & {\emptycirc} & {\fullcirc} & {\fullcirc} & {\fullcirc} & {\fullcirc} & {\fullcirc} & {\emptycirc} & {\fullcirc} & {\fullcirc} & {\fullcirc} & {\emptycirc} & {\emptycirc} & {\emptycirc} & {\emptycirc} & {\emptycirc} & {\fullcirc} & {\fullcirc} & {\fullcirc} & {\fullcirc} & {\fullcirc} & {\emptycirc} & {\emptycirc} & {\emptycirc} & {\emptycirc} & {\emptycirc} & \cellcolor{o5}{$5$($33\%$})
% & \cellcolor{r4}{$0\%$}    
\\ 
% & 4
% & 2M
eToro & 2013 & {\emptycirc} & {\emptycirc} & {\fullcirc} & {\emptycirc} & {\fullcirc} & {\fullcirc} & {\emptycirc} & {\emptycirc} & {\emptycirc} & {\emptycirc} & {\emptycirc} & {\emptycirc} & {\emptycirc} & {\emptycirc} & {\emptycirc} & {\emptycirc} & {\emptycirc} & {\emptycirc} & {\emptycirc}  & {\emptycirc} & {\emptycirc} & {\emptycirc} & {\emptycirc} & {\emptycirc} & {\emptycirc} & {\emptycirc} & {\emptycirc} & {\emptycirc} & {\emptycirc} & {\emptycirc} & {\emptycirc} & {\emptycirc} & {\emptycirc} & {\emptycirc} & {\fullcirc} & {\fullcirc} & {\fullcirc} & {\fullcirc} & {\fullcirc} & {\fullcirc} & {\fullcirc} & {\fullcirc} & {\fullcirc} & {\fullcirc} & {\emptycirc} & {\emptycirc} & {\emptycirc} & {\emptycirc} & {\emptycirc} & {\emptycirc} & {\emptycirc} & {\emptycirc} & {\emptycirc} & {\emptycirc} & {\emptycirc} & {\emptycirc} & {\emptycirc} & {\emptycirc} & {\emptycirc} & $0$($0\%$)
% & \cellcolor{r2}{$0\%$}  
\\ 
% & 33M
Kraken Ex. & 2013 & {\emptycirc} & {\emptycirc} & {\fullcirc} & {\emptycirc} & {\fullcirc} & {\fullcirc} & {\emptycirc} & {\emptycirc} & {\emptycirc} & {\emptycirc} & {\emptycirc} & {\emptycirc}  & {\emptycirc} & {\emptycirc} & {\emptycirc} & {\emptycirc} & {\emptycirc}  & {\emptycirc} & {\emptycirc} & {\emptycirc} & {\emptycirc} & {\emptycirc} & {\emptycirc} & {\emptycirc} & {\emptycirc} & {\emptycirc} & {\emptycirc} & {\emptycirc} & {\emptycirc} & {\emptycirc} & {\emptycirc} & {\emptycirc} & {\emptycirc} & {\emptycirc} & {\fullcirc} & {\fullcirc} & {\fullcirc} & {\fullcirc} & {\emptycirc} & {\fullcirc} & {\emptycirc} & {\fullcirc} & {\fullcirc} & {\fullcirc} & {\emptycirc} & {\emptycirc} & {\emptycirc} & {\emptycirc} & {\emptycirc} & {\emptycirc} & {\emptycirc} & {\emptycirc} & {\emptycirc} & {\emptycirc} & {\emptycirc} & {\emptycirc} & {\emptycirc} & {\emptycirc} & {\emptycirc} & {$0$($0\%$)} 
% & \cellcolor{r3}{$0\%$}  
\\ 
Ledger & 2014 & {\fullcirc} & {\emptycirc} & {\emptycirc} & {\emptycirc} & {\emptycirc} & {\emptycirc} & {\emptycirc} & {\fullcirc} & {\fullcirc} & {\emptycirc} & {\emptycirc} & {\emptycirc} & {\emptycirc} & {\fullcirc} & {\emptycirc} & {\fullcirc} & {\emptycirc} & {\emptycirc} & {\fullcirc} & {\emptycirc} & {\emptycirc} & {\fullcirc} & {\emptycirc} & {\emptycirc} & {\fullcirc} & {\emptycirc} & {\fullcirc} & {\emptycirc} & {\emptycirc} & {\emptycirc} & {\fullcirc} & {\emptycirc} & {\emptycirc} & {\halfcirc} & {\emptycirc} & {\fullcirc} & {\fullcirc} & {\fullcirc} & {\fullcirc} & {\fullcirc} & {\fullcirc} & {\fullcirc} & {\fullcirc} & {\fullcirc} & {\emptycirc} & {\emptycirc} & {\fullcirc} & {\emptycirc} & {\emptycirc} & {\emptycirc} & {\fullcirc} & {\emptycirc} & {\emptycirc} & {\fullcirc} & {\emptycirc} & {\emptycirc} & {\emptycirc} & {\emptycirc} & {\fullcirc} & \cellcolor{o4}{$4$($27\%$)}
% & \cellcolor{r6}{$0\%$}  
\\ 
% & 6M
% & software open source - firmware closed source
Gemini & 2014 & {\emptycirc} & {\emptycirc} & {\fullcirc} & {\emptycirc} & {\fullcirc} & {\fullcirc} & {\emptycirc} & {\emptycirc} & {\emptycirc} & {\emptycirc} & {\emptycirc} & {\emptycirc} & {\emptycirc} & {\emptycirc} & {\emptycirc} & {\emptycirc} & {\emptycirc} & {\emptycirc} & {\emptycirc} & {\emptycirc} & {\emptycirc} & {\emptycirc} & {\emptycirc} & {\emptycirc} & {\emptycirc} & {\emptycirc} & {\emptycirc} & {\emptycirc} & {\emptycirc} & {\emptycirc} & {\emptycirc} & {\emptycirc} & {\emptycirc} & {\emptycirc} & {\fullcirc} & {\fullcirc} & {\fullcirc} & {\fullcirc} & {\emptycirc} & {\fullcirc} & {\emptycirc} & {\fullcirc} & {\emptycirc} & {\fullcirc} & {\emptycirc} & {\emptycirc} & {\emptycirc} & {\emptycirc} & {\emptycirc} & {\emptycirc} & {\emptycirc} & {\emptycirc} & {\emptycirc} & {\emptycirc} & {\emptycirc} & {\emptycirc} & {\emptycirc} & {\emptycirc} & {\emptycirc} & $0$($0\%$)
% & \cellcolor{r3}{$0\%$}  
\\
Metamask & 2016 & {\fullcirc} & {\emptycirc} & {\emptycirc} & {\emptycirc} & {\fullcirc} & {\fullcirc} & {\emptycirc} & {\emptycirc} & {\emptycirc} & {\emptycirc} & {\emptycirc} & {\emptycirc} & {\emptycirc} & {\fullcirc} & {\emptycirc} & {\fullcirc} & {\emptycirc} & {\emptycirc} & {\fullcirc} & {\emptycirc} & {\emptycirc} & {\fullcirc} & {\emptycirc} & {\emptycirc} & {\fullcirc} & {\emptycirc} & {\emptycirc} & {\emptycirc} & {\fullcirc} & {\fullcirc} & {\emptycirc} & {\emptycirc} & {\emptycirc} & {\fullcirc} & {\emptycirc} & {\emptycirc} & {\fullcirc} & {\fullcirc} & {\fullcirc} & {\emptycirc} & {\fullcirc} & {\emptycirc} & {\emptycirc} & {\fullcirc} & {\emptycirc} & {\emptycirc} & {\emptycirc} & {\fullcirc} & {\emptycirc} & {\emptycirc} & {\emptycirc} & {\emptycirc} & {\emptycirc} & {\emptycirc} & {\emptycirc} & {\emptycirc} & {\emptycirc} & {\emptycirc} & {\emptycirc} & \cellcolor{o0}{$1$($7\%$}) 
% & \cellcolor{r1}{$0\%$}  
\\ 
% & 30M m*
Bitbuy &  2016 & {\emptycirc} & {\emptycirc} & {\fullcirc} & {\emptycirc} & {\fullcirc} & {\fullcirc} & {\emptycirc} & {\emptycirc} & {\emptycirc} & {\emptycirc} & {\emptycirc} & {\emptycirc} & {\emptycirc} & {\emptycirc} & {\emptycirc} & {\emptycirc} & {\emptycirc} & {\emptycirc} & {\emptycirc} & {\emptycirc} & {\emptycirc} & {\emptycirc} & {\emptycirc} & {\emptycirc} & {\emptycirc} & {\emptycirc} & {\emptycirc} & {\emptycirc} & {\emptycirc} & {\emptycirc} & {\emptycirc} & {\emptycirc} & {\emptycirc} & {\emptycirc} & {\fullcirc} & {\fullcirc} & {\fullcirc} & {\fullcirc} & {\emptycirc} & {\fullcirc} & {\fullcirc} & {\fullcirc} & {\fullcirc} & {\fullcirc} & {\emptycirc} & {\emptycirc} & {\emptycirc} & {\emptycirc} & {\emptycirc} & {\emptycirc} & {\emptycirc} & {\emptycirc} & {\emptycirc} & {\emptycirc} & {\emptycirc} & {\emptycirc} & {\emptycirc} & {\emptycirc} & {\emptycirc} & $0$($0\%$)
% & \cellcolor{r3}{$0\%$}  
\\ 
% & 0.45M
Exodus & 2016 & {\fullcirc} & {\emptycirc} & {\emptycirc} & {\fullcirc} & {\fullcirc} & {\fullcirc} & {\emptycirc} & {\emptycirc} & {\emptycirc} & {\emptycirc} & {\emptycirc} & {\emptycirc} & {\emptycirc} & {\fullcirc} & {\emptycirc} & {\fullcirc} & {\emptycirc} & {\fullcirc} & {\fullcirc} & {\emptycirc} & {\emptycirc} & {\fullcirc} & {\emptycirc} & {\emptycirc} & {\fullcirc} & {\emptycirc} & {\emptycirc} & {\fullcirc} & {\fullcirc} & {\fullcirc} & {\emptycirc} & {\emptycirc} & {\emptycirc} & {\emptycirc} & {\fullcirc} & {\fullcirc} & {\fullcirc} & {\fullcirc} & {\fullcirc} & {\fullcirc} & {\fullcirc} & {\fullcirc} & {\fullcirc} & {\fullcirc} & {\emptycirc} & {\emptycirc} & {\emptycirc} & {\emptycirc} & {\emptycirc} & {\emptycirc} & {\emptycirc} &  {\emptycirc} & {\emptycirc} & {\fullcirc} & {\emptycirc} & {\emptycirc} & {\emptycirc} & {\emptycirc} & {\emptycirc} & \cellcolor{o0}{$1$($7\%$)} 
% & \cellcolor{r5}{$0\%$}   
\\ 
% & 0.8M m*
Binance Ex. & 2017 & {\emptycirc} & {\emptycirc} & {\fullcirc} & {\fullcirc} & {\fullcirc} & {\fullcirc} & {\emptycirc} & {\emptycirc} & {\emptycirc} & {\emptycirc} & {\emptycirc} & {\emptycirc} & {\emptycirc} & {\emptycirc} & {\emptycirc} & {\emptycirc} & {\emptycirc} & {\emptycirc} & {\emptycirc} & {\emptycirc} & {\emptycirc} & {\emptycirc} & {\emptycirc} & {\emptycirc} & {\emptycirc} & {\emptycirc} & {\emptycirc} & {\emptycirc} & {\emptycirc} & {\emptycirc} & {\emptycirc} & {\emptycirc} & {\emptycirc} & {\emptycirc} & {\fullcirc} & {\fullcirc} & {\fullcirc} & {\fullcirc} & {\fullcirc} & {\fullcirc} & {\fullcirc} & {\fullcirc} & {\fullcirc} & {\fullcirc} & {\emptycirc} & {\emptycirc} & {\emptycirc} & {\emptycirc} & {\emptycirc} & {\emptycirc} & {\emptycirc} & {\emptycirc} & {\emptycirc} & {\emptycirc} & {\emptycirc} & {\emptycirc} & {\emptycirc} & {\emptycirc} & {\emptycirc} & $0$($0\%$))
% & \cellcolor{r2}{$0\%$}  
\\ 
% & 200M
Trust Wlt. & 2017 & {\fullcirc} & {\emptycirc} & {\emptycirc} & {\emptycirc} & {\fullcirc} & {\fullcirc} & {\emptycirc} & {\emptycirc} & {\emptycirc} & {\emptycirc} & {\emptycirc} & {\emptycirc} & {\emptycirc} & {\fullcirc} & {\emptycirc} & {\fullcirc} & {\emptycirc} & {\halfcirc} & {\fullcirc} & {\emptycirc} & {\emptycirc} & {\fullcirc} & {\emptycirc} & {\emptycirc} & {\fullcirc} & {\emptycirc} & {\emptycirc} & {\emptycirc} & {\fullcirc} & {\fullcirc} & {\emptycirc} & {\emptycirc} & {\emptycirc}  & {\fullcirc} & {\emptycirc} & {\fullcirc} & {\fullcirc} & {\fullcirc} & {\fullcirc} & {\fullcirc} & {\emptycirc} & {\fullcirc} & {\fullcirc} & {\fullcirc} & {\emptycirc} & {\emptycirc} & {\emptycirc} & {\emptycirc} & {\fullcirc} & {\emptycirc} & {\emptycirc} & {\emptycirc} & {\emptycirc} &  {\emptycirc} & {\emptycirc} & {\emptycirc} & {\emptycirc} & {\emptycirc} & {\emptycirc} & \cellcolor{o0}{$1$($7\%$)} 
% & \cellcolor{r1}{$0\%$}  
\\ 
% & 2
% & 130M
Argent & 2017 & {\fullcirc} & {\emptycirc} & {\emptycirc} & {\emptycirc} & {\fullcirc} & {\fullcirc} & {\fullcirc} & {\emptycirc} & {\emptycirc} & {\emptycirc} & {\emptycirc} & {\emptycirc} & {\fullcirc} & {\emptycirc} & {\fullcirc} & {\emptycirc} & {\fullcirc} & {\emptycirc} & {\emptycirc} & {\fullcirc} & {\fullcirc} & {\emptycirc} & {\emptycirc} & {\fullcirc} & {\emptycirc} & {\emptycirc} & {\emptycirc} & {\fullcirc} & {\emptycirc} & {\emptycirc} & {\emptycirc} & {\fullcirc} & {\emptycirc} & {\fullcirc} & {\emptycirc} & {\emptycirc} & {\fullcirc} & {\fullcirc} & {\emptycirc} & {\emptycirc} & {\emptycirc} & {\emptycirc} & {\emptycirc} & {\emptycirc} & {\emptycirc} & {\emptycirc} & {\emptycirc} & {\emptycirc} & {\emptycirc} & {\fullcirc} & {\emptycirc} & {\emptycirc} & {\emptycirc} & {\emptycirc} & {\emptycirc} & {\emptycirc} & {\fullcirc} & {\emptycirc} & {\emptycirc} & \cellcolor{o2}{$2$($13\%$)} 
% & \cellcolor{r2}{$0\%$}   
\\ 
CoinEx & 2017 & {\emptycirc} & {\emptycirc} & {\fullcirc} & {\emptycirc} & {\fullcirc} & {\fullcirc} & {\emptycirc} & {\emptycirc} & {\emptycirc} & {\emptycirc} & {\emptycirc} & {\emptycirc} & {\emptycirc} & {\emptycirc} & {\emptycirc} & {\emptycirc} & {\emptycirc} & {\emptycirc} & {\emptycirc} & {\emptycirc} & {\emptycirc} & {\emptycirc} & {\emptycirc} & {\emptycirc} & {\emptycirc} & {\emptycirc} & {\emptycirc} & {\emptycirc} & {\emptycirc} & {\emptycirc} & {\emptycirc} & {\emptycirc} & {\emptycirc} & {\emptycirc} & {\fullcirc} & {\fullcirc} & {\fullcirc} & {\fullcirc} & {\fullcirc} & {\fullcirc} & {\fullcirc} & {\fullcirc} & {\fullcirc} & {\fullcirc} & {\emptycirc} & {\emptycirc} & {\emptycirc} & {\emptycirc} & {\emptycirc} & {\emptycirc} & {\emptycirc} & {\emptycirc} & {\emptycirc} & {\emptycirc} & {\emptycirc} & {\emptycirc} & {\emptycirc} & {\emptycirc} & {\emptycirc} & $0$($0\%$))
% & \cellcolor{r2}{$0\%$}  
\\ 
% \FilledCircle
 % & 5M 
Safe (Gnosis) & 2017 & {\fullcirc} & {\emptycirc} & {\emptycirc} & {\emptycirc} & {\emptycirc} & {\fullcirc} & {\fullcirc} & {\emptycirc} & {\emptycirc} & {\emptycirc} & {\emptycirc} & {\emptycirc} & {\fullcirc} & {\emptycirc} & {\fullcirc} & {\emptycirc} & {\fullcirc} & {\emptycirc} & {\emptycirc} & {\fullcirc} & {\fullcirc} & {\emptycirc} & {\emptycirc} & {\fullcirc} & {\emptycirc} & {\emptycirc} & {\emptycirc} & {\fullcirc} & {\emptycirc} & {\emptycirc} & {\emptycirc} & {\fullcirc} & {\emptycirc} &  {\fullcirc} & {\emptycirc} & {\emptycirc} & {\fullcirc} & {\emptycirc} & {\emptycirc} & {\emptycirc} & {\emptycirc} & {\emptycirc} & {\emptycirc} & {\emptycirc} & {\emptycirc} & {\emptycirc} & {\emptycirc} & {\emptycirc} & {\emptycirc} & {\fullcirc} & {\emptycirc} & {\emptycirc} & {\emptycirc} & {\emptycirc} & {\fullcirc} & {\emptycirc} & {\emptycirc} & {\emptycirc} & {\emptycirc} & \cellcolor{o2}{$2$($13\%$)} 
% & \cellcolor{r2}{$0\%$}   
\\ 
% & 1.6M m*
Atomic & 2017 & {\fullcirc} & {\emptycirc} & {\emptycirc} & {\fullcirc} & {\emptycirc} & {\fullcirc} & {\emptycirc} & {\emptycirc} & {\emptycirc} & {\emptycirc} & {\emptycirc} & {\emptycirc} & {\emptycirc} & {\emptycirc} & {\fullcirc} & {\fullcirc} & {\emptycirc} & {\emptycirc} & {\fullcirc} & {\emptycirc} & {\emptycirc} & {\fullcirc} & {\emptycirc} & {\emptycirc} & {\fullcirc} & {\emptycirc} & {\emptycirc} & {\emptycirc} & {\emptycirc} & {\fullcirc} & {\emptycirc} & {\emptycirc} & {\emptycirc} & {\emptycirc} & {\fullcirc} & {\fullcirc} & {\fullcirc} & {\fullcirc} & {\fullcirc} & {\fullcirc} & {\fullcirc} & {\fullcirc} & {\fullcirc} & {\fullcirc} & {\emptycirc} & {\emptycirc} & {\emptycirc} & {\emptycirc} & {\fullcirc} & {\fullcirc} & {\emptycirc} & {\emptycirc} & {\emptycirc} & {\emptycirc} & {\emptycirc} & {\emptycirc} &  {\emptycirc} & {\emptycirc} & {\emptycirc} & \cellcolor{o2}{$2$($13\%$)} 
% & \cellcolor{r3}{$0\%$}  
\\
% & 10M
Tangem & 2017 & {\fullcirc} & {\emptycirc} & {\emptycirc} & {\emptycirc} & {\emptycirc} & {\emptycirc} & {\emptycirc} & {\emptycirc} & {\emptycirc} & {\fullcirc} & {\emptycirc} & {\emptycirc} & {\emptycirc} & {\fullcirc} & {\emptycirc} & {\fullcirc} & {\emptycirc} & {\emptycirc} & {\fullcirc} & {\emptycirc} & {\emptycirc} & {\fullcirc} & {\emptycirc} & {\emptycirc} & {\fullcirc} & {\emptycirc} & {\emptycirc} & {\emptycirc} & {\fullcirc} & {\fullcirc} & {\fullcirc} & {\emptycirc} & {\emptycirc} & {\fullcirc} & {\emptycirc} & {\fullcirc} & {\fullcirc} & {\emptycirc} & {\fullcirc} & {\fullcirc} & {\emptycirc} & {\fullcirc} & {\emptycirc} & {\fullcirc} & {\emptycirc} & {\emptycirc} & {\emptycirc} & {\emptycirc} & {\emptycirc} & {\emptycirc} & {\emptycirc} & {\emptycirc} & {\emptycirc} & {\emptycirc} & {\emptycirc} & {\emptycirc} & {\emptycirc} & {\emptycirc} & {\emptycirc} & $0$($0\%$)
% & \cellcolor{r0}{$0\%$}  
\\
Ngrave & 2018 & {\fullcirc} & {\emptycirc} & {\emptycirc} & {\emptycirc} & {\emptycirc} & {\emptycirc} & {\emptycirc} & {\emptycirc} & {\emptycirc} & {\emptycirc} & {\fullcirc} & {\emptycirc} & {\emptycirc} & {\fullcirc} & {\emptycirc} & {\fullcirc} & {\emptycirc} & {\emptycirc} & {\fullcirc} & {\emptycirc} & {\emptycirc} & {\fullcirc} & {\emptycirc} & {\emptycirc} & {\fullcirc} & {\emptycirc} & {\emptycirc} & {\emptycirc} & {\fullcirc} & {\emptycirc} & {\fullcirc} & {\emptycirc} & {\emptycirc} & {\emptycirc} & {\fullcirc} & {\fullcirc} & {\fullcirc} & {\emptycirc} & {\fullcirc} & {\fullcirc} & {\emptycirc} & {\fullcirc} & {\emptycirc} & {\fullcirc} & {\emptycirc} & {\emptycirc} & {\emptycirc} & {\emptycirc} & {\emptycirc} & {\emptycirc} & {\emptycirc} & {\emptycirc} & {\emptycirc} & {\emptycirc} & {\emptycirc} & {\emptycirc} & {\emptycirc} & {\emptycirc} & {\emptycirc} & $0$($0\%$)
% & \cellcolor{r0}{$0\%$}   
\\ 
Zengo & 2018 & {\emptycirc} & {\fullcirc} & {\emptycirc} & {\emptycirc} & {\emptycirc} & {\fullcirc} & {\emptycirc} & {\emptycirc} & {\emptycirc} & {\emptycirc} & {\emptycirc} & {\emptycirc} & {\fullcirc} & {\emptycirc} & {\fullcirc} & {\emptycirc} & {\emptycirc} & {\fullcirc} & {\fullcirc} & {\emptycirc} & {\emptycirc} & {\fullcirc} & {\emptycirc} & {\emptycirc} & {\emptycirc} & {\fullcirc} & {\emptycirc} & {\emptycirc} & {\fullcirc} & {\emptycirc} & {\emptycirc} & {\emptycirc} & {\emptycirc} & {\fullcirc} & {\emptycirc} & {\fullcirc} & {\fullcirc} & {\fullcirc} & {\fullcirc} & {\emptycirc} & {\emptycirc} & {\emptycirc} & {\emptycirc} & {\emptycirc} & {\emptycirc} & {\emptycirc} & {\emptycirc} & {\emptycirc} & {\emptycirc} & {\fullcirc} & {\emptycirc} & {\emptycirc} & {\emptycirc} & {\emptycirc} & {\emptycirc}  & {\emptycirc} & {\emptycirc} & {\emptycirc} & {\emptycirc} & \cellcolor{o1}{$1$($7\%$)}
% & \cellcolor{r1}{$0\%$}  
\\ 
% & 1m
% Need to confirm coinbase wallet because it seems it has some smart features but it also has seed phrase
% Whats the difference between passkey and biometrics
Coinbase Wlt  & 2019 & {\fullcirc} & {\emptycirc} & {\emptycirc} & {\emptycirc} & {\fullcirc} & {\fullcirc} & {\fullcirc} & {\emptycirc} & {\emptycirc} & {\emptycirc} & {\emptycirc} & {\emptycirc} & {\emptycirc} & {\emptycirc} & {\fullcirc} & {\fullcirc} & {\emptycirc} & {\emptycirc} & {\fullcirc} & {\emptycirc} & {\fullcirc} & {\emptycirc} & {\emptycirc} & {\fullcirc} & {\emptycirc} & {\emptycirc} & {\emptycirc} & {\fullcirc} & {\emptycirc} & {\fullcirc} & {\emptycirc} & {\fullcirc} & {\emptycirc} & {\emptycirc} & {\fullcirc} & {\fullcirc} & {\fullcirc} & {\fullcirc} & {\fullcirc} & {\fullcirc} & {\emptycirc} & {\fullcirc} & {\fullcirc} & {\fullcirc} & {\emptycirc} & {\emptycirc} & {\emptycirc} & {\emptycirc} & {\emptycirc} & {\emptycirc} & {\emptycirc} & {\emptycirc} & {\emptycirc} & {\fullcirc} & {\emptycirc} & {\emptycirc} & {\emptycirc} & {\emptycirc} & {\emptycirc} & \cellcolor{o1}{$1$($7\%$)} 
% & \cellcolor{r0}{$0\%$}  
\\ 
Biconomy & 2019 & {\fullcirc} & {\emptycirc} & {\emptycirc} &  {\emptycirc} & {\emptycirc} & {\emptycirc} & {\fullcirc} & {\emptycirc} & {\emptycirc} & {\emptycirc} & {\emptycirc} & {\emptycirc} & {\emptycirc} & {\emptycirc} & {\fullcirc} & {\fullcirc} & {\emptycirc} & {\emptycirc}  & {\fullcirc} & {\emptycirc} & {\fullcirc} & {\emptycirc} & {\emptycirc} & {\fullcirc} & {\emptycirc} & {\emptycirc} & {\emptycirc} & {\fullcirc} & {\emptycirc} & {\emptycirc} & {\emptycirc} & {\fullcirc} & {\emptycirc} & {\fullcirc} & {\emptycirc} & {\emptycirc} & {\fullcirc} & {\fullcirc} & {\fullcirc} & {\emptycirc} & {\emptycirc} & {\emptycirc} & {\emptycirc} & {\fullcirc} & {\emptycirc} & {\emptycirc} & {\emptycirc} & {\emptycirc} & {\emptycirc} & {\fullcirc} & {\emptycirc} & {\emptycirc} & {\emptycirc} & {\emptycirc} & {\emptycirc} & {\emptycirc} & {\emptycirc} & {\emptycirc} & {\emptycirc} & \cellcolor{o1}{$1$($7\%$)}  
% & \cellcolor{r2}{$0\%$}  
\\ 
% & 5M 
Web3Auth & 2020 & {\emptycirc} & {\fullcirc} & {\emptycirc} & {\emptycirc} & {\emptycirc} & {\fullcirc} & {\emptycirc} & {\emptycirc} & {\emptycirc} & {\emptycirc} & {\emptycirc} & {\emptycirc} & {\fullcirc} & {\emptycirc} & {\fullcirc} & {\emptycirc} & {\emptycirc} & {\fullcirc} & {\emptycirc} & {\emptycirc} & {\fullcirc} & {\emptycirc} & {\emptycirc} & {\fullcirc} & {\emptycirc} & {\emptycirc} & {\fullcirc} & {\fullcirc} & {\emptycirc} & {\emptycirc} & {\emptycirc} & {\fullcirc} & {\emptycirc} & {\fullcirc} & {\emptycirc} & {\emptycirc} & {\fullcirc} & {\fullcirc} & {\fullcirc} & {\emptycirc} & {\emptycirc} & {\emptycirc} & {\emptycirc} & {\fullcirc} & {\emptycirc} & {\emptycirc} & {\emptycirc} & {\emptycirc} & {\emptycirc} & {\emptycirc} & {\emptycirc} & {\emptycirc} & {\emptycirc} & {\emptycirc} & {\fullcirc} & {\emptycirc} & {\emptycirc} & {\emptycirc} & {\emptycirc} & \cellcolor{o1}{$1$($7\%$)}  
% & \cellcolor{r2}{$0\%$}  
\\ 
Brave & 2021 & {\fullcirc} & {\emptycirc} & {\emptycirc} & {\emptycirc} & {\fullcirc} & {\fullcirc} & {\emptycirc} & {\emptycirc} & {\emptycirc} & {\emptycirc} & {\emptycirc} & {\emptycirc} & {\emptycirc} & {\fullcirc} & {\emptycirc} & {\fullcirc} & {\emptycirc} & {\emptycirc} & {\fullcirc} & {\emptycirc} & {\emptycirc} & {\fullcirc} & {\emptycirc} & {\emptycirc} & {\fullcirc} & {\emptycirc} & {\emptycirc} & {\emptycirc} & {\fullcirc} & {\fullcirc} & {\emptycirc} & {\emptycirc} & {\emptycirc} & {\fullcirc} & {\emptycirc} & {\fullcirc} & {\fullcirc} & {\fullcirc} & {\emptycirc} & {\emptycirc} & {\emptycirc} & {\fullcirc} & {\emptycirc} & {\emptycirc} & {\emptycirc} & {\fullcirc} & {\emptycirc} & {\fullcirc} & {\emptycirc} & {\emptycirc} & {\emptycirc} & {\emptycirc} & {\emptycirc} & {\emptycirc} & {\emptycirc} & {\emptycirc} & {\emptycirc} & {\emptycirc} & {\emptycirc} & \cellcolor{o3}{$2$($13\%$)}  
% & \cellcolor{r2}{$0\%$}  
\\ 
% & 70M m*
Phantom & 2021 & {\fullcirc} & {\emptycirc} & {\emptycirc} & {\emptycirc} & {\fullcirc} & {\fullcirc} & {\emptycirc} & {\emptycirc} & {\emptycirc} & {\emptycirc} & {\emptycirc} & {\emptycirc} & {\emptycirc} & {\fullcirc} & {\emptycirc} & {\fullcirc} & {\emptycirc} & {\emptycirc} & {\fullcirc} & {\emptycirc} & {\emptycirc} & {\fullcirc} & {\emptycirc} & {\emptycirc} & {\fullcirc} & {\emptycirc} & {\emptycirc} & {\emptycirc} & {\fullcirc} & {\fullcirc} & {\fullcirc} & {\emptycirc} & {\emptycirc} & {\emptycirc} & {\fullcirc} & {\fullcirc} & {\fullcirc} & {\fullcirc} & {\emptycirc} & {\emptycirc} & {\emptycirc} & {\fullcirc} & {\emptycirc} & {\emptycirc} & {\emptycirc} & {\fullcirc} & {\emptycirc} & {\fullcirc} & {\emptycirc} & {\emptycirc} & {\emptycirc} & {\emptycirc} & {\emptycirc} & {\emptycirc} & {\emptycirc} & {\emptycirc} & {\emptycirc} & {\emptycirc} & {\emptycirc} & \cellcolor{o3}{$2$($13\%$)}  
% & \cellcolor{r2}{$0\%$}  
\\ 
% & 7M m* 
Slope & 2021 & {\fullcirc} & {\emptycirc} & {\emptycirc} & {\emptycirc} & {\fullcirc} & {\fullcirc} & {\emptycirc} & {\emptycirc} & {\emptycirc} & {\emptycirc} & {\emptycirc} & {\emptycirc} & {\emptycirc} & {\fullcirc} & {\emptycirc} & {\fullcirc} & {\emptycirc} & {\emptycirc} & {\fullcirc} & {\emptycirc} & {\emptycirc} & {\fullcirc} & {\emptycirc} & {\emptycirc} & {\fullcirc} & {\emptycirc} & {\emptycirc} & {\emptycirc} & {\fullcirc} & {\fullcirc} & {\emptycirc} & {\emptycirc} & {\emptycirc} & {\fullcirc} & {\emptycirc} & {\emptycirc} & {\fullcirc} & {\emptycirc} & {\fullcirc} & {\emptycirc} & {\emptycirc} & {\fullcirc} & {\emptycirc} & {\emptycirc} & {\fullcirc} & {\emptycirc} & {\emptycirc} & {\emptycirc} & {\emptycirc} & {\emptycirc} & {\emptycirc} & {\emptycirc} & {\emptycirc} & {\emptycirc} & {\emptycirc} & {\emptycirc} & {\emptycirc} & {\fullcirc}  & {\emptycirc} & \cellcolor{o3}{$2$($13\%$)} 
% & \cellcolor{r1}{$0\%$}  
\\ 
HashPack  & 2021 & {\fullcirc} & {\emptycirc} & {\emptycirc} & {\emptycirc} & {\fullcirc} & {\fullcirc} & {\emptycirc} & {\emptycirc} & {\emptycirc} & {\emptycirc} & {\emptycirc} & {\emptycirc} & {\emptycirc} & {\fullcirc} & {\emptycirc} & {\fullcirc} & {\emptycirc} & {\emptycirc} & {\fullcirc} & {\emptycirc} & {\emptycirc} & {\fullcirc} & {\emptycirc} & {\emptycirc} & {\fullcirc} & {\emptycirc} & {\emptycirc} & {\emptycirc} & {\fullcirc} & {\fullcirc} & {\emptycirc} & {\emptycirc} & {\fullcirc} & {\emptycirc} & {\fullcirc} & {\emptycirc} & {\emptycirc} & {\emptycirc} & {\emptycirc} & {\emptycirc} & {\emptycirc} & {\emptycirc} & {\emptycirc} & {\emptycirc} & {\emptycirc} & {\emptycirc} & {\emptycirc} & {\emptycirc} & {\emptycirc} & {\emptycirc} & {\emptycirc} & {\emptycirc} & {\emptycirc} & {\emptycirc} & {\emptycirc} & {\emptycirc} & {\emptycirc} & {\emptycirc} & {\emptycirc} & $0$($0\%$)
% & \cellcolor{r0}{$0\%$}  
\\ 
Binance Web3 & 2023 & {\emptycirc} & {\fullcirc} & {\emptycirc} & {\emptycirc} & {\emptycirc} & {\fullcirc} & {\emptycirc} & {\emptycirc} & {\emptycirc} & {\emptycirc} & {\emptycirc} & {\emptycirc} & {\fullcirc} & {\emptycirc} & {\fullcirc} & {\emptycirc} & {\emptycirc} & {\fullcirc} & {\fullcirc} & {\emptycirc} & {\emptycirc} & {\fullcirc} & {\emptycirc} & {\emptycirc} & {\emptycirc} & {\emptycirc} & {\emptycirc} & {\fullcirc} & {\fullcirc} & {\emptycirc} & {\emptycirc} & {\emptycirc} & {\emptycirc} & {\fullcirc} & {\emptycirc} & {\emptycirc} & {\fullcirc} & {\fullcirc} & {\fullcirc} & {\emptycirc} & {\emptycirc} & {\fullcirc} & {\emptycirc} & {\fullcirc} & {\emptycirc} & {\emptycirc} & {\emptycirc} & {\emptycirc} & {\emptycirc} & {\fullcirc} & {\emptycirc} & {\emptycirc} & {\emptycirc} & {\emptycirc} & {\emptycirc} & {\emptycirc} & {\emptycirc} & {\emptycirc} & {\emptycirc} & \cellcolor{o1}{$1$($7\%$)} 
% & \cellcolor{r1}{$0\%$}  
\\ 
Kraken Wlt. & 2024 & {\fullcirc} & {\emptycirc} & {\emptycirc} & {\emptycirc} & {\emptycirc} & {\fullcirc} & {\emptycirc} & {\emptycirc} & {\emptycirc} & {\emptycirc} & {\emptycirc} & {\emptycirc} & {\emptycirc} & {\fullcirc} & {\fullcirc} & {\fullcirc} & {\emptycirc} & {\emptycirc} & {\fullcirc} & {\emptycirc} & {\emptycirc} & {\fullcirc} & {\emptycirc} & {\emptycirc} & {\emptycirc} & {\emptycirc} & {\emptycirc} & {\fullcirc} & {\fullcirc} & {\fullcirc} & {\emptycirc} & {\emptycirc} & {\emptycirc} & {\fullcirc} & {\emptycirc} & {\fullcirc} & {\fullcirc} & {\fullcirc} & {\emptycirc} & {\emptycirc} & {\emptycirc} & {\fullcirc} & {\emptycirc} & {\emptycirc} & {\emptycirc} & {\emptycirc} & {\emptycirc} & {\emptycirc} & {\emptycirc} & {\emptycirc} & {\emptycirc} & {\emptycirc} & {\emptycirc} & {\emptycirc} & {\emptycirc} & {\emptycirc} & {\emptycirc} & {\emptycirc} & {\emptycirc} & $0$($0\%$)
% & \cellcolor{r1}{$0\%$}  
\\ 
\midrule
\multicolumn{3}{c}{\textbf{Summary}} &
\multicolumn{17}{c}{\textbf{Highest Occurrence: Signature Verification Logic Flaw}} &
\multicolumn{5}{c}{\cellcolor{o3}{$7$($21\%$)}} &
\multicolumn{20}{c}{} &
\multicolumn{16}{r}{\textbf{Total Vulnerabilities Detected in All Wallets}} &
$33$($100\%$)  
% \cellcolor{o0}{$33$($100\%$)} 

 \\ 
% \midrule
% \multirow{7}{*}{\rotatebox[origin=l]{90}{Custodial}} 
% &  
% \multirow{-7}{*}{\rotatebox[origin=l]{90}{Custodial}}
% & 
% {llccccccccccccccccccccccccccccccccccccccccccccccccccccccccccc}
% \multicolumn{5}{l}{} &
%   \multicolumn{5}{l}{} &
%   \multicolumn{5}{l}{} &
%   \multicolumn{5}{l}{} &
%   \multicolumn{5}{c}{} &
%   \multicolumn{5}{l}{} &
%   \multicolumn{5}{l}{} &
%   \multicolumn{5}{l}{} &
%    \multicolumn{5}{c}{\textbf{{Vulnerabilities No \& \%}}} &
%    \cellcolor{g6}{($0\%$)} &
% \cellcolor{g6}{($0\%$)} &
% \cellcolor{g6}{($0\%$)} &
% \cellcolor{g6}{($0\%$)} &
% \cellcolor{g6}{($0\%$)} &
%   \cellcolor{g6}{($0\%$)} &
% \cellcolor{g6}{($0\%$)} &
% \cellcolor{g6}{($0\%$)} &
% \cellcolor{g6}{($0\%$)} &
% \cellcolor{g6}{($0\%$)} &
%   \cellcolor{g6}{($0\%$)} &
% \cellcolor{g6}{($0\%$)} &
% \cellcolor{g6}{($0\%$)} &
% \cellcolor{g6}{($0\%$)} &
% \cellcolor{g6}{($0\%$)} 
% \\
\bottomrule
\end{tabular}
\vspace{1ex} % Add space before the caption
\caption{Industry Wallet design variations and identified threats. ( \fullcirc : include, \halfcirc : part-inclusion, \emptycirc : not include)
}
\label{tab:wlt._taxonomy}
\end{table*}


\paragraph{\textbf{Security-Focused Evolution}} 
The infrastructural evolution of wallets with a focus on security has been a response to the inherent vulnerabilities associated with software-based systems. This led to the development of hardware wallets as well as paper and brain key storage mediums, which introduce an offline component into traditional wallet architectures, effectively reducing the attack vectors associated with internet connectivity. 

% --
% just commented out 

% Initial forms such as paper and brain wallets, though highly secure due to their offline nature, were not without drawbacks. These wallets compromise on user convenience to improve security. Paper wallets necessitated secure physical storage, while brain wallets relied on the user's ability to memorise and safeguard complex passphrases. The advent of hardware wallets marked a significant milestone, offering an engaging user experience while maintaining the core principle of offline key storage through highly secure, physical devices. Looking forward, we anticipate an improved hardware wallet functionality without compromising its foundational security features.


\paragraph{\textbf{Functionality-Focused Evolution}} 

The drive towards improved functionality has resulted in the development of web, mobile, and smart contract wallets. These wallets marked a notable shift towards enhanced flexibility and user convenience. Web and mobile wallets introduced the ability to manage cryptocurrencies across various platforms, while smart contract wallets further expanded wallet capabilities through advanced and flexible transaction management. 


% Major catalysts for evolution on this axis have been improvement proposals on Bitcoin and Ethereum.

% As the cryptocurrency ecosystem developed, brain wallets emerged around 2012 as another alternative method of key storage, relying entirely on the cognitive ability of individuals to memorise a passphrase or seed phrase \cite{vasek2017bitcoin}. By eliminating the need for physical storage, these wallets place the security of the funds entirely in the owner's memory. However, the simplicity of brain wallets comes with considerable risks. If the passphrase, which acts as the private key, is forgotten or inadequately complex, the funds are irrevocably lost or susceptible to brute-force attacks (see \autoref{sec:brute-force}). The security of brain wallets is wholly dependent on the user's ability to create and recall a sufficiently random and complex passphrase. Despite these vulnerabilities, brain wallets provide a unique method of managing and spending funds securely by importing private keys into wallet clients when necessary for transactions \cite{brain}. To address these security issues, recent innovations have sought to enhance the usability and safety of brain wallets. In 2019, a semi-custodial brain wallet was developed, which simplifies the user's burden by only requiring them to remember a username and password, while a server assists in a key generation without storing the complete key \cite{aman2019zerowallet}. This model has been further enhanced with post-quantum cryptography to fortify the security of brain wallets, offering a more robust solution that attempts to balance the ease of use with enhanced security measures \cite{kethepalli2023reinforcing}.

\subsubsection{Insight 2: Nuanced Wallet Designs}
\label{sec:hybrid-multi-tier-wallets}

% % \subsubsection{Design Decisions}
% % \label{sec:design-decisions}

% The traditional classification of wallets into \enquote{hot} and \enquote{cold} categories based solely on internet connectivity is overly simplistic and fails to capture the nuanced operational mechanisms of modern wallets. While hardware wallets, often labelled as \enquote{cold}, generate \teal{$pk$} and sign transactions internally on an offline component, interaction with an online environment to broadcast transactions to the blockchain network is required for state change executions. This is consistent with how we describe the mechanism of a wallet in \autoref{sec:formalisation}.

We propose a more nuanced framework that considers internet connectivity as an additional factor across various phases of the wallet design. By incorporating connectivity as a dynamic attribute rather than a fixed binary state, we can more accurately assess a wallet's security complexity. Our design taxonomy also aids in the creation of more nuanced wallet solutions, as trade-offs exist within initialisation, distribution, authorisation, validation, authentication and recovery design factors. Therefore, expanding the design spectrum that can be streamlined to meet institutional and retail clients' requirements. We discuss the influence of design on threats in \autoref{sec:threats_dis_influence}.






% \subsubsection{Design Decisions}
% \label{sec:design-decisions}

% \subsubsection{Implication of Design Choice}
% \label{sec:existing-vulnerabilityy}


% have a look at TEE design paper for inspirartion

% The differentiation between cold and hot wallets should not exist, when there are both online and offline operations in 

% There should exist a more

% For instance, a mobile wallet can operate as a hardware wallet if the private key is stored in the secure enclave chip.


% Hardware and Software

% Custodial and Non-custodial

% Cold and Hot




% this was part of the text before but its a bit too long needs to be rewritten concisely

% Furthermore, a hybrid wallet can be formed by integrating functionalities from different types of wallets, effectively amalgamating the accessibility of an online wallet with the robust security features of a hardware wallet. These amalgamations broaden the options available to users and empower them to tailor their crypto wallets to suit their specific needs, striking an optimal balance between accessibility and security \cite{biernacki2021comparative}.

% \cite{kethepalli2023reinforcing}



\begin{table*}[!h]
\centering
\renewcommand{\arraystretch}{1.1}
\setlength{\tabcolsep}{1.7pt} % Adjust the column separation space here
% \tiny
\begin{tabular}{llcccccccc@{\hspace{6pt}}l@{\hspace{3pt}}cccccc} % Changed column specifier 'c' to 'l' for left alignment
\toprule
\vspace{0.2pt} 
 & 
\multicolumn{1}{c}{\textbf{Threat}} &
\multicolumn{2}{c}{\textbf{Gap}} &
\multicolumn{6}{c}{\textbf{Target}} & % Higher-level category for KeyGen to TxnVer
\multicolumn{1}{c@{\hspace{10pt}}}{\textbf{Adversary's (\teal{$A$}) Capability Summary}} 
& \multicolumn{3}{c}{\textbf{Knwl.}}
& \multicolumn{2}{c}{\textbf{Acc.}} % Higher-level category for access levels
\vspace{0.5pt} 
\\
% \vspace{0.5pt} 
\cmidrule(lr){3-4} \cmidrule(lr){5-10} \cmidrule(lr){12-14} \cmidrule(lr){15-16}
\vspace{0.5pt} 
% Lines below the higher-level categories to separate the sub-columns
  \rotatebox{90}{\textbf{Category}}
  &
  &
  \rotatebox{90}{Academia} &
  \rotatebox{90}{Incidents} &
  \rotatebox{90}{KeyGen} &
  \rotatebox{90}{CreateTxn} &
  \rotatebox{90}{Auth} &
  \rotatebox{90}{KeyStore} &
  \rotatebox{90}{TxnSign} &
  \rotatebox{90}{TxnVer} &
   & 
  \rotatebox{90}{Public} &
  \rotatebox{90}{Restricted} &
  \rotatebox{90}{Insider} &
  \rotatebox{90}{Remote} &
  \rotatebox{90}{Physical}  \\
\midrule
  \multirow{3}{*}{\rotatebox{90}{Net.}} & Insecure Network Channel \cite{cve_33297, cve_14198, cve_17144} &
  {\fullcirc} &
  {\fullcirc} &
  {\emptycirc} &
  {\fullcirc} &
  {\emptycirc} &
  {\emptycirc} &
  {\emptycirc} &
  {\emptycirc} &
  % \teal{$A$} can 
  Exploit network to intercept or alter communications. &
  {\fullcirc} &
  {\emptycirc} &
  {\emptycirc} &
  {\fullcirc} &
  {\emptycirc} \\
  &
  Compromised Network Protocol \cite{Hu2021SecurityCountermeasures}  &
  {\fullcirc} &
  {\emptycirc} &
  {\emptycirc} &
  {\emptycirc} &
  {\fullcirc} &
  {\emptycirc} &
  {\emptycirc} &
  {\emptycirc} &
  % \teal{$A$} can 
  Exploit network protocol to intercept transactions. &
  {\fullcirc} &
  {\emptycirc} &
  {\emptycirc} &
  {\fullcirc} &
  {\emptycirc} \\
  %   &
  % Network Connection Exploitation  &
  % {\emptycirc} &
  % {\emptycirc} &
  % {\emptycirc} &
  % {\emptycirc} &
  % {\emptycirc} &
  % {\emptycirc} &
  % {\emptycirc} &
  % {\emptycirc} &
  % % \teal{$A$} can 
  % Exploit network providers to intercept transactions. &
  % {\emptycirc} &
  % {\emptycirc} &
  % {\emptycirc} &
  % {\emptycirc} &
  % {\emptycirc} \\
% \midrule
\multirow{6}{*}{\rotatebox{90}{App.}} & Application Logic Flaw \cite{Parisi2023WalletSecurity, oren2023fireblocks} &
  {\fullcirc} &
  {\fullcirc} &
  {\emptycirc} &
  {\emptycirc} &
  {\fullcirc} &
  {\emptycirc} &
  {\emptycirc} &
  {\emptycirc} &
  % \teal{$A$} can 
  Exploit the programming logic of functions. &
  {\fullcirc} &
  {\emptycirc} &
  {\emptycirc} &
  {\fullcirc} &
  {\emptycirc} \\
&
  \acs{os} Vulnerabilities \cite{he2020security} &
  {\fullcirc} &
  {\fullcirc} &
  {\emptycirc} &
  {\emptycirc} &
  {\emptycirc} &
  {\fullcirc} &
  {\emptycirc} &
  {\emptycirc} &
  % \teal{$A$} can 
  Exploit \acs{os} (see \autoref{sec:privilege}) to bypass security. &
  {\emptycirc} &
  {\fullcirc} &
  {\emptycirc} &
  {\fullcirc} &
  {\emptycirc} \\
 &
  Library Vulnerability \cite{bitcore_lib, Ledger2023SecurityReport}  &
  {\fullcirc} &
  {\fullcirc} &
  {\fullcirc} &
  {\emptycirc} &
  {\emptycirc} &
  {\emptycirc} &
  {\fullcirc} &
  {\fullcirc} &
  % \teal{$A$} can 
  Exploit vulnerabilities in third-party libraries. &
  {\fullcirc} &
  {\fullcirc} &
  {\emptycirc} &
  {\fullcirc} &
  {\emptycirc} \\
 &
 Insecure Permissions \cite{cve_32969, halborn_vuln} &
 {\fullcirc} &
  {\fullcirc} &
  {\emptycirc} &
  {\fullcirc} &
  {\fullcirc} &
  {\emptycirc} &
  {\emptycirc} &
  {\emptycirc} &
  % \teal{$A$} can 
  Make unauthorised changes in the system.
   & 
  {\emptycirc} &
  {\fullcirc} &
  {\emptycirc} &
  {\fullcirc} &
  {\emptycirc} \\
 &
  Coding Errors \cite{Parisi2023WalletSecurity} &
  {\fullcirc} &
  {\fullcirc} &
  {\emptycirc} &
  {\fullcirc} &
  {\fullcirc} &
  {\emptycirc} &
  {\emptycirc} &
  {\emptycirc} &
  % \teal{$A$} can e
  Exploit coding errors to bypass security. &
  {\fullcirc} &
  {\emptycirc} &
  {\emptycirc} &
  {\fullcirc} &
  {\emptycirc} \\
&
  Insecure Interaction \cite{ZengoZengo} &
  {\fullcirc} &
  {\fullcirc} &
  {\emptycirc} &
  {\fullcirc} &
  {\emptycirc} &
  {\emptycirc} &
  {\emptycirc} &
  {\emptycirc} &
  % \teal{$A$} can 
  Exploit users through application layer interactions. &
  {\fullcirc} &
  {\fullcirc} &
  {\fullcirc} &
  {\fullcirc} &
  {\emptycirc} \\
% \midrule
% \multirow{11}{*}{\rotatebox{90}{Mechanism Vuln.}}
\multirow{2}{*}{\rotatebox{90}{Au.}} &
  Inadeq. Authentication \cite{Uddin2021Horus:Wallets} &
  {\fullcirc} &
  {\fullcirc} &
  {\emptycirc} &
  {\emptycirc} &
  {\fullcirc} &
  {\emptycirc} &
  {\emptycirc} &
  {\emptycirc} &
  % \teal{$A$} can 
  Attempt to bypass the authentication mechanism. &
  {\fullcirc} &
  {\fullcirc} &
  {\emptycirc} &
  {\fullcirc} &
  {\fullcirc} \\
 &
  Low-strength Password \cite{Kiktenko2019DetectingWallets, volety2019cracking} &
 {\fullcirc} &
  {\fullcirc} &
  {\emptycirc} &
  {\emptycirc} &
  {\fullcirc} &
  {\emptycirc} &
  {\emptycirc} &
  {\emptycirc} &
  % \teal{$A$} can 
  Attempt possible \teal{$pw$} combinations to decrypt \teal{$sk$}. &
  {\fullcirc} &
  {\emptycirc} &
  {\emptycirc} &
  {\fullcirc} &
  {\emptycirc} \\
 \multirow{7}{*}{\rotatebox{90}{Sto.}} &
  Insecure Boot Environment \cite{Shaikh2022SurveyExchanges} &
  {\fullcirc} &
  {\emptycirc} &
  {\emptycirc} &
  {\emptycirc} &
  {\emptycirc} &
  {\fullcirc} &
  {\emptycirc} &
  {\emptycirc} &
  % \teal{$A$} can 
  Exploit an insecure boot to execute code. &
  {\emptycirc} &
  {\fullcirc} &
  {\emptycirc} &
  {\emptycirc} &
  {\fullcirc} \\
&
  Inadequate Encryption \cite{cve_15947, CoinTelegraph2022SlopeAttack} & 
  {\fullcirc} &
  {\fullcirc} &
  {\fullcirc} &
  {\emptycirc} &
  {\emptycirc} &
  {\fullcirc} &
  {\fullcirc} &
  {\emptycirc} &
  % \teal{$A$} can 
  Access credentials stored unencrypted. &
  {\emptycirc} &
  {\fullcirc} &
  {\fullcirc} &
  {\fullcirc} &
  {\fullcirc} \\
 &
  Data Remanence \cite{trezor_memory, trezor_medium} &
  {\fullcirc} &
  {\fullcirc} &
  {\emptycirc} &
  {\emptycirc} &
  {\emptycirc} &
  {\fullcirc} &
  {\emptycirc} &
  {\emptycirc} &
  % \teal{$A$} can 
  Exploit remanence in memory to extract info. &
  {\emptycirc} &
  {\fullcirc} &
  {\emptycirc} &
  {\emptycirc} &
  {\fullcirc} \\
 &
 Data Manipulation \cite{trezor_memory, trezor_medium} &
  {\fullcirc} &
  {\fullcirc} &
  {\emptycirc} &
  {\emptycirc} &
  {\emptycirc} &
  {\fullcirc} &
  {\emptycirc} &
  {\emptycirc} &
  % \teal{$A$} can 
  Manipulate or tamper with data. &
  {\emptycirc} &
  {\fullcirc} &
  {\emptycirc} &
  {\fullcirc} &
  {\fullcirc} \\
 &
  Micro-electrical Exposure \cite{courbon2016reverse} &
  {\fullcirc} &
  {\fullcirc} &
  {\emptycirc} &
  {\emptycirc} &
  {\emptycirc} &
  {\fullcirc} &
  {\emptycirc} &
  {\emptycirc} &
  % \teal{$A$} can 
  Tamper with micro-electrical components. &
  {\emptycirc} &
  {\fullcirc} &
  {\emptycirc} &
  {\emptycirc} &
  {\fullcirc} \\
  &
  Storage Provider Compromise \cite{CoinTelegraph2022SlopeAttack} &
  {\emptycirc} &
  {\fullcirc} &
  {\emptycirc} &
  {\emptycirc} &
  {\emptycirc} &
  {\fullcirc} &
  {\emptycirc} &
  {\emptycirc} &
  % \teal{$A$} can 
  Exploit external providers for indirect access. &
  {\emptycirc} &
  {\fullcirc} &
  {\fullcirc} &
  {\fullcirc} &
  {\emptycirc} \\
% \midrule
\multirow{3}{*}{\rotatebox{90}{Cry.}} &
 Predictable \acs{rng} \cite{cve_31290, cve_23660} &
  {\fullcirc} &
  {\fullcirc} &
  {\fullcirc} &
  {\emptycirc} &
  {\emptycirc} &
  {\emptycirc} &
  {\emptycirc} &
  {\emptycirc} &
  % \teal{$A$} can 
  Predict or reproduce \acs{rng} outputs. &
  {\fullcirc} &
  {\fullcirc} &
  {\emptycirc} &
  {\fullcirc} &
  {\emptycirc} \\
 &
  Weak Signature \cite{Rokhjavan2023SecuringWallets} &
  {\fullcirc} &
  {\fullcirc} &
  {\emptycirc} &
  {\emptycirc} &
  {\emptycirc} &
  {\emptycirc} &
  {\fullcirc} &
  {\fullcirc} &
  % \teal{$A$} can 
  Attempt to create malicious transactions. 
  % forge transaction signatures due to weak algorithms.
  &
  {\fullcirc} &
  {\fullcirc} &
  {\emptycirc} &
  {\fullcirc} &
  {\emptycirc} \\
 &
  Side-channel Leakage \cite{cve_14353, cve_14354, KrakenBlog} &
  {\fullcirc} &
  {\fullcirc} &
  {\emptycirc} &
  {\emptycirc} &
  {\emptycirc} &
  {\fullcirc} &
  {\emptycirc} &
  {\emptycirc} &
  % \teal{$A$} can 
  Exploit side-channel leakages in the system. &
  {\fullcirc} &
  {\fullcirc} &
  {\emptycirc} &
  {\fullcirc} &
  {\fullcirc} \\
 % &
 % Inadequate Signature Verification \cite{cve_14199, tymokhanov2021alpha} &
 % {\fullcirc} &
 %  {\fullcirc} &
 %  {\emptycirc} &
 %  {\emptycirc} &
 %  {\emptycirc} &
 %  {\emptycirc} &
 %  {\emptycirc} &
 %  {\fullcirc} &
 %  % \teal{$A$} can
 %  Exploit inadequate verification functions. &
 %  {\fullcirc} &
 %  {\fullcirc} &
 %  {\emptycirc} &
 %  {\fullcirc} &
 %  {\emptycirc} \\
% \midrule
\multirow{2}{*}{\rotatebox{90}{Oth.}} &
   Insider Collusion \cite{decrypt_ftx} &
  {\emptycirc} &
  {\fullcirc} &
  {\emptycirc} &
  {\emptycirc} &
  {\fullcirc} &
  {\fullcirc} &
  {\emptycirc} &
  {\emptycirc} &
  % \teal{$A$} can 
  Act malicious as an insider or insider group colluding. &
  {\emptycirc} &
  {\emptycirc} &
  {\fullcirc} &
  {\fullcirc} &
  {\fullcirc} \\
   &
Insider Compromise \cite{Ledger2023SecurityReport} &
  {\emptycirc} &
  {\fullcirc} &
  {\emptycirc} &
  {\emptycirc} &
  {\fullcirc} &
  {\fullcirc} &
  {\emptycirc} &
  {\emptycirc} &
  % \teal{$A$} can 
  Exploit insider information to bypass security. &
  {\emptycirc} &
  {\emptycirc} &
  {\fullcirc} &
  {\fullcirc} &
  {\fullcirc} \\
\bottomrule
\end{tabular}
\vspace{1ex} % Add space before the caption
\caption{Threat and capability classification on wallet mechanism}
\label{tab:threat_capability}
\end{table*}


% Weak Signature
% Insec. Boot Environ.
% App Logic Flaw
% Low pwds
% Micro-electr. Exposure
% OS Vulnerabilities
% Coding Errors

% Weak Signature \cite{rokhjavan2023securing}
% Insec. Boot Environ. \cite{shaikh2022survey}
% App Logic Flaw \cite{destefanis2018smart, parisi2023wallet, oren2023fireblocks}
% Low pwds \cite{kiktenko2019detecting, volety2019cracking}
% Micro-electr. Exposure  \cite{courbon2016reverse}
% OS Vulnerabilities  \cite{he2020security}
% Coding Errors  \cite{parisi2023wallet}

\section{Threat Model}
\label{sec:threat_framework}


We analyse threats to the wallet mechanism, considering adversary goals, knowledge, and capabilities. Using our design taxonomy (\autoref{tab:wlt._taxonomy}), we also identify industry threats and highlight gaps between industry and academia (\autoref{tab:threat_capability}).

\subsection{Classification}
\label{sec:threat_class}

Our threat classification is structured around distinct operations within the wallet mechanism across three stages: wallet initialisation, transaction generation, and transaction broadcast. Threats to the system can be categorised into five areas: network, authentication, application, storage and memory, and cryptanalysis.

\subsubsection{Network}
\label{sec:vuln_mech}

The wallet communicates with the blockchain to retrieve and broadcast \textcolor{teal}{\textit{state\_trans\_info}} using internet protocols. The network enables the secure transmission of messages within and outside of the system. Vulnerabilities in the communication channels can be targeted, as shown in \autoref{tab:attack_vectors}. Service providers in the network can also be compromised, rendering messages vulnerable to interception and alteration.

\subsubsection{Application}
\label{sec:vuln_mech}

Wallets rely on application libraries \cite{Hu2021SecurityCountermeasures}, and operating systems \cite{he2020security, li2020android}, which may possess vulnerabilities that the adversary can exploit. Vulnerabilities in these systems include application logic vulnerabilities such as key recovery \cite{cve_15302}, signature verification \cite{cve_14199}, and input validation \cite{immunefi} flaws, which can result in privilege escalation. Additionally, malware exposure \cite{balakrishnan2023analysis, li2020android}, insecure third-party interactions \cite{ZengoZengo, thodex}, and user negligence \cite{weichbroth2023security} can threaten the security of the \textcolor{teal}{\textit{sk}}, \textcolor{teal}{\textit{rdm\_seed}}, or \textcolor{teal}{\textit{pw}}. For instance, projects integrating TON wallets have experienced silent exfiltration of mnemonic phrases via malicious application libraries \cite{sockettonpkg, ghsadvisory}. Web3 wallets embedded in social platforms amplify supply-chain risk. Malicious npm modules impersonating TON SDKs (e.g., \texttt{@ton-wallet/create}) execute clipboard-sniffers that forward seed phrases to attacker-controlled Telegram bots \cite{sockettonpkg,beincrypto2024socket}. Since wallet logic is tightly coupled to chatbot \acs{api}s, a single rogue Mini App link can invoke in-chat transaction authorisation by users, as seen in the June 2025 phishing wave \cite{slowmisttron}.


% Figure environment removed

\subsubsection{Authentication}
\label{sec:vuln_mech}

Authentication is a critical process in modern wallets, as only an authorised owner can decrypt an encrypted private key (\textcolor{teal}{\textit{enc\_sk}}) and sign transactions (refer to the \textcolor{orange}{\textit{encrypt}} and \textcolor{orange}{\textit{decrypt}} functions in \hyperref[algo:cryptocurrency-wallet]{Algorithm 1} and \hyperref[algo:transaction-signing]{Algorithm 2}, respectively). Authentication attacks aim to compromise the wallet function that verifies the user's identity, thereby gaining unauthorised access to wallets. The authentication functions, which handle the encryption and decryption of the \textcolor{teal}{\textit{enc\_sk}}, can be vulnerable to insecure boot environments \cite{Shaikh2022SurveyExchanges} and single-factor authentication methods and low-strength passwords (\textcolor{teal}{\textit{pw}}).


\subsubsection{Storage and Memory}
\label{sec:vuln_phy}

Data stored can be vulnerable to threats of extraction, manipulation and disruption. Exploitation of the wallet's storage mechanism (see \autoref{sec:key-storage}) can lead to the compromise of \textcolor{teal}{\textit{sk}}, \textcolor{teal}{\textit{rdm\_seed}} or \textcolor{teal}{\textit{pw}}. Storage mechanism vulnerabilities include data remanence \cite{Shaikh2022SurveyExchanges}, unencrypted data \cite{breier2022practical, robinson2022new}, and physical security vulnerabilities \cite{courbon2016reverse} that can be exploited by the adversary.

\subsubsection{Cryptanalysis}
\label{sec:vuln_mech}

Cryptographic vulnerabilities may exist in the signature scheme (\textcolor{orange}{\textit{keyGen}}, \textcolor{orange}{\textit{txnSign}}, \textcolor{orange}{\textit{txnVer}}) as a result of the direct implementation or unintended data leakages from side channels. These vulnerabilities include hash function vulnerabilities \cite{shrivas2020disruptive}, weak signatures (\teal{$\sigma$}) \cite{Rokhjavan2023SecuringWallets}, predictable \acf{rng} \cite{brengel2018identifying}, and data leakages from side-channels \cite{Park2023, kocher1996timing}. 


\subsubsection{Other Threats}
\label{sec:vuln_ext}

Threats can occur via other avenues, such as an insider who may have access to transactional information, user credentials and other security details. These can arise from insiders acting maliciously or by exploitation through coercion or social engineering methods. Custodial (\autoref{sec:custodial-wallets}) and Shared-custodial (\autoref{sec:semi-custodial-wallets}) architectures are more vulnerable to these threats due to their more centralised architecture. Non-custodial setups (see \autoref{sec:non-custodial-wallets}) may be vulnerable if third-party services are employed for functionalities such as \textcolor{teal}{\textit{pw}} management or if inadequate access controls are relied upon (e.g., Ledger incident \cite{zerocap}).


\subsection{Adversary's Goals}
\label{sec:adversary_goal}

 We define an adversary, \textcolor{teal}{\textit{A}}, who aims to exploit threats described above to trigger unauthorised transactions to an adversary-controlled wallet address or disrupt operations. The major goals of \textcolor{teal}{\textit{A}} include:
\begin{itemize}
    \item \textbf{Credential Compromise:} \textcolor{teal}{\textit{A}} aims to compromise \textcolor{teal}{\textit{sk}}, \textcolor{teal}{\textit{rdm\_seed}} and \textcolor{teal}{\textit{pw}} by exploiting wallet mechanism vulnerabilities or user-interactions.
    \item \textbf{State Transition Information Manipulation:} \textcolor{teal}{\textit{A}} aims to modify the \textcolor{teal}{\textit{state\_trans\_info}} created by the user such as \textcolor{teal}{\textit{recipient\_address}}. Following this, \textcolor{teal}{\textit{A}} deceives the user into signing the transaction. \textcolor{teal}{\textit{A}} may also manipulate the \textcolor{teal}{\textit{state\_trans\_info}} displayed on the wallet interface
\end{itemize}

\subsection{Adversary's Capabilities}
\label{sec:adversary_cap}


\autoref{tab:threat_capability} details the various capabilities of \textcolor{teal}{\textit{A}}, illustrating how identified vulnerabilities can be exploited to achieve an objective with various degrees of knowledge and access. \textcolor{teal}{\textit{A}} can possess public, restricted and insider knowledge. Public knowledge includes information that is openly accessible to anyone, such as open-source code, publicly available audit reports, discussions in open forums, websites, and applications. Restricted knowledge refers to information that is not readily accessible to the public and often requires specific roles, permissions, or effort to obtain. Information that is only accessible to individuals within an organisation is defined as insider knowledge, particularly in setups where custodians have some level of authorisation (\autoref{sec:design-cust}). \textcolor{teal}{\textit{A}} can also execute several attack capabilities remotely or physically. 

% crypto-wallets-mapping.drawio-2.pdf


% VERY IMPORTANT HOW STATE CHANGE IN TERMS OF PRIVILEGE CAN CHANGE FOR AN ATTACKER
% --
% Dynamic Capabilities: Attackers often move between categories through the course of an attack (e.g., an external attacker becomes an insider by compromising credentials). The model could benefit from incorporating these dynamic aspects, perhaps by outlining common pathways between categories or stages in a typical attack lifecycle.
% --

% \subsection{Insights}
% \label{sec:threat_discussion}

% \subsubsection{Insight 1: Influence of Design on Threats}
% \label{sec:threats_dis_influence}

% Despite a wide range of security setups, we observe that the majority of the design combinations of existing wallets surveyed, including desktop, browser, hardware, mobile, smart and \acs{mpc} wallets have been threatened by multiple vulnerabilities, as shown in \autoref{tab:wlt._taxonomy}. This is due to similar implementations i.e., the use of replicated libraries and commonly integrated implementation proposals (e.g. ERC-4337). We also observe that some wallets have had numerous vulnerabilities discovered in industry and academia. Most notably, Ledger and Trezor have several data remanence, data manipulation and insecure cryptographic vulnerabilities. Furthermore, in mapping vulnerabilities to attacks, we observe that some vulnerabilities can lead to numerous attack vectors as shown in \autoref{fig:wallet-mapping}. These include inadequate authentication, data leakage, insecure permission and insecure user interactions. 

% % % Figure environment removed


% % Data remanence vulnerabilities were found in Trezor whereby an attacker with physical access could extract the seed from the device \acs{ram} \cite{trezor_memory}


% \subsubsection{Insight 2: Signature Verification Logic Flaw Occurrence}
% \label{sec:threats_dis_influence}

% We observe that signature verification logic flaws account for the most vulnerability occurrences in various wallets surveyed constituting 21\%. Another interesting observation is the occurrence of this vulnerability in three diverse wallet security enhancement architectures, namely hardware, smart contract and \acs{mpc} wallets \cite{cve_14199, fireblocks_23, AccountMedium, UncoveringVulnerability}.


% \subsubsection{Insight 3: Gap Analysis on Wallet Threats}
% While a gap analysis on executed attacks in industry and academia proves difficult to conduct accurately due to the lack of known industry attack methods, we analyse the gaps in vulnerabilities and threats. We generally observe a high correlation between identified threats in industry and academia, except for insider and external threats. Specifically, in the following threats: malicious insider, compromised insider and compromised service provider threats. Although, there are several custodial designs brought forward by academia with threat models, an investigation into the possible external threats and attacks in custodial setups would be very beneficial for the industry. Notably, most industry attacks target exchanges and other custodial setups, as large funds are concentrated within a few wallet addresses. Additionally, research into these areas will also be pertinent because, wallet designs are gradually evolving into shared-custodial or other setups which require authentication from a centralised party (e.g. passkey, 2FA).



% Threats which are more concurrent on custodial wallet designs are not investigated as much in academia. 
% T However, % Mechanism and system-related vulnerabilities are frequently explored in both academia and industry, which demonstrates 


% \paragraph{Adversary's Access}
% \label{sec:adversary_access}


% Key Shard Secret An adversary may have the secrets of one \textcolor{teal}{\textit{sk}} shard.
% Physical Device LocationThe location of the hardware device may be known to the adversary.

% \subsection{Adversary's Access}
% \label{sec:adversary_access}



% The primary objective of the adversary is to trigger unauthorised transactions to an adversary-controlled wallet address using a stolen or leaked \textcolor{teal}{\textit{sk}} or by altering \teal{$tnx$} messages. This can be conducted by:
% \begin{itemize}
%     \item \textbf{Compromising the Private Key:} The adversary seeks to gain unauthorised access to the private key.
%     \item \textbf{Tampering with Transaction Messages:} The adversary aims to intercept and modify transaction messages submitted by the user, potentially altering transaction details and outcomes.
%     \item \textbf{Exploiting Cryptographic Algorithms:} The adversary attempts to exploit vulnerabilities in the cryptographic algorithms used by the wallet mechanism.
% \end{itemize}


% \subsection{Adversarial Capabilities}

% 

\begin{table}[!ht]
    \centering
    \begin{tabular}{p{0.25cm}p{7.15cm}}
        \toprule
         & \textbf{Description} \\
        \toprule
        \multirow{3}{*}{\rotatebox[origin=c]{90}{NET}} & A can compromise network service providers. \\
        & A can intercept network communication channels. \\
        & A can create malicious connection channels. \\
        \midrule
        \multirow{7}{*}{\rotatebox[origin=c]{90}{APP}} & A can monitor the user's actions to retrieve credentials. \\
        & A can replace the user-specified address. \\
        & A can mimic legitimate services to extract passwords or secret keys. \\
        & A can exploit vulnerabilities to achieve elevated permissions. \\
        & A can compromise third-party services to gain access to the wallet. \\

        \midrule
        \multirow{5}{*}{\rotatebox[origin=c]{90}{AUT}} & A can attempt all possible password combinations to decrypt secret key. \\
        & A can attempt to predict mnemonics. \\
        & A can attempt to replicate the verification method to bypass the authentication mechanism. \\
        \midrule
        \multirow{5}{*}{\rotatebox[origin=c]{90}{STO}} &  A can access sensitive information stored unencrypted. \\
        & A can manipulate or tamper with physical memory to gain access \\
        & A can exploit data remanence in the wallet's memory to retrieve sensitive \\
        \midrule
        \multirow{5}{*}{\rotatebox[origin=c]{90}{CRY}} &  A can exploit vulnerabilities in the cryptographic algorithms to derive secret keys or compromise the integrity of cryptographic values. \\
        & A can exploit behavioural leakages in cryptographic functions to extract sensitive information. \\
        \bottomrule
    \end{tabular}
    \caption{Capability Description Table}
    \label{tab:capability}
\end{table}


% $C_2^{\text{NET}}$

% \subsubsection{Network}

% \begin{itemize}
%     \item A can compromise network service providers.
%     \item A can intercept network communication channels.
%     \item A can create malicious connection channels. 
% \end{itemize}

% The attacker can eavesdrop on the network traffic to intercept and modify communication within the wallet system \cite{hu2021security}. Additionally, the adversary can gain control of trusted components on the network layer with which the wallet interacts.

% \subsubsection{Application}

% \begin{itemize}
%     \item \textbf{Compromise of the Private Key:} The adversary may compromise the \textcolor{teal}{\textit{sk}} by theft or leakage.
%     \item \textbf{Alteration of the State Transition Information:} The adversary may to intercept and modify the \teal{$state\_trans\_info$} created by the user.
%     % \item \textbf{Exploiting Cryptographic Algorithms:} The adversary attempts to exploit vulnerabilities in the cryptographic algorithms used by the wallet mechanism.
% \end{itemize}

% \subsubsection{Authentication}

% \begin{itemize}
%     \item \textbf{Compromise of the Private Key:} The adversary may compromise the \textcolor{teal}{\textit{sk}} by theft or leakage.
%     \item \textbf{Alteration of the State Transition Information:} The adversary may to intercept and modify the \teal{$state\_trans\_info$} created by the user.
%     % \item \textbf{Exploiting Cryptographic Algorithms:} The adversary attempts to exploit vulnerabilities in the cryptographic algorithms used by the wallet mechanism.
% \end{itemize}

% \subsubsection{Storage}

% \begin{itemize}
%     \item \textbf{Compromise of the Private Key:} The adversary may compromise the \textcolor{teal}{\textit{sk}} by theft or leakage.
%     \item \textbf{Alteration of the State Transition Information:} The adversary may to intercept and modify the \teal{$state\_trans\_info$} created by the user.
%     % \item \textbf{Exploiting Cryptographic Algorithms:} The adversary attempts to exploit vulnerabilities in the cryptographic algorithms used by the wallet mechanism.
% \end{itemize}

% \subsubsection{Cryptographic}

% \begin{itemize}
%     \item \textbf{Compromise of the Private Key:} The adversary may compromise the \textcolor{teal}{\textit{sk}} by theft or leakage.
%     \item \textbf{Alteration of the State Transition Information:} The adversary may to intercept and modify the \teal{$state\_trans\_info$} created by the user.
%     % \item \textbf{Exploiting Cryptographic Algorithms:} The adversary attempts to exploit vulnerabilities in the cryptographic algorithms used by the wallet mechanism.
% \end{itemize}

% \subsection{Threat Model}
% \label{sec:threat_model} 

% \subsubsection{System Vulnerabilities} 
% \label{sec:incident}

% \paragraph{Key Generation Algorithm}
% This component can be vulnerable to attacks if insufficient randomness is generated. 

% \paragraph{Transaction Signing Algorithm} 

% \paragraph{Transaction Verification Algorithm} 

% \paragraph{Transaction Creation Algorithm} 

% \paragraph{Private Key Storage Medium} 

% \paragraph{User Authentication Algorithm}
% This component is vulnerable to exploits



% These events are illicit or malevolent actions undertaken to undermine blockchain wallets' confidentiality, availability, or integrity. Based on the framework provided, we classify these attacks into two major types:

% \paragraph{Direct Attacks} 
% These are incidents whereby an unauthorised entity conducts malicious activity by attempting to gain sensitive information owned by a user or the application system. Such attacks often target the cryptographic underpinnings of a wallet, such as attempts to decrypt private keys, attacks on key management mechanisms, or employing manipulative tactics such as social engineering to mislead users into divulging their credentials or compromising their security. 

% Private Keys, mnemonic phrases, passphrases and passwords are targeted in these attacks.
% \paragraph{Indirect Attacks} 
% These incidents, when executed in isolation, disrupt the wallet operations without directly leading to financial losses by exploiting vulnerabilities in the system's intercommunication or dependency structures. We include these attacks due to the weakening of a wallet's overall security landscape.

% \subsubsection{Attacker’s Objective}
% We classify the objective of the attacker into four major categories:
% \paragraph{Monetary Gain} The most direct and common motivation for wallet attacks. Adversaries with this objective aim to extract funds from compromised wallets by decrypting private keys, exploiting software vulnerabilities, or via other direct attack vectors. 
% \paragraph{Operational Disruption} As in \autoref{sec:incident}, adversaries might not always seek financial gain. Instead, some aim to disrupt the operations of a wallet, primarily through indirect attacks. By destabilising the wallet's operations, adversaries can attempt other, more significant attacks.
% \paragraph{Privacy Compromise} These less common motivation focuses on de-anonymising the wallet user.
% \paragraph{Information Theft} Personal information, contact details and transactions can also be targeted by attackers in an attempt to undergo further attacks.

% \paragraph{Network-related Capabilities}
% Regarding network-related capabilities, an attacker possesses multiple avenues of potential disruption. Firstly, they can control the system’s network, allowing for significant influence over its operations. Secondly, they can intercept and modify the network's traffic, which can manipulate the data being sent or received. Lastly, the attacker can directly disrupt a system’s network, causing interruptions or complete outages.
% \paragraph{Application-related Capabilities}
% Concerning application-related capabilities, attackers have a wide range of techniques at their disposal. They can manipulate or deceive users into revealing their private keys, putting personal assets at risk. In addition, they can extract sensitive information from devices, potentially leading to further breaches. Vulnerabilities in the application library can also be exploited, as can storage vulnerabilities within devices. Moreover, these adversaries have the know-how to deploy malicious software directly into the application, adding another layer of potential threat.
% \paragraph{Storage-related Capabilities}
% Attacks can exploit vulnerabilities in various storage types, including physical devices and cloud storage systems, employing tactics such as unauthorised access, data tampering, and other breaches to jeopardise data integrity and recovery efforts.
% \paragraph{Authentication-related Capabilities}
% Here, an attacker can capture and alter data during its transmission. An attacker can methodically attempt to derive or decrypt a private key.
% \paragraph{Blockchain-related Capabilities}
% In this case, an attacker can exploit a wide range of vulnerabilities inherent to smart contract wallets.



% --
% Our system model categorises all prominent wallet attacks into four layers: application, network, storage and authentication. Application attacks on crypto wallets encompass software vulnerabilities and manipulative tactics, targeting user interactions, and software libraries to compromise assets and sensitive information. Network attacks target the communication between different consensus and connectivity layers of blockchain wallet networks, encompassing threats such as routing manipulation, de-anonymisation techniques, \acf{mitm} intrusions, \acf{dns} exploits, and \acf{dos} disruptions. Authentication attacks target the private key or sensitive information directly to gain unauthorised access to the accounts of users.

% --
% \subsection{Adversary's Objectives}

% The major goal of the adversary is three-fold; first, the adversary attempts to compromise the security of the private key, second, the adversary attempts to modify transaction messages inputted by the user and thirdly, the adversary attempts to compromise the cryptographic algorithms of the wallet mechanism.

\section{Attack Taxonomy}
\label{sec:attack-framework}

In this section, we present a comprehensive taxonomy of wallet attack vectors, systematically examining the methods, techniques, and targeted components involved. Building on our generalized wallet mechanisms and threat model taxonomy, we outline a broad spectrum of attacks, as illustrated in \autoref{fig:wallet-attacks}. These attacks are categorized based on the specific functions and components they target within the wallet infrastructure (see \autoref{sec:wallet_mechanism}) and are classified according to our threat model (see \autoref{sec:threat_class}). We further incorporate the infrastructure layer of our design taxonomy to capture the multi-layered nature of these threats, as summarized in \autoref{tab:attack_vectors}. To construct this taxonomy, we analysed data from academic literature and notable industry incidents from 2012 to 2024, each varying in severity and financial impact (see \autoref{fig:lossMonthly}).

% \subsection{Methodology}
% \label{sec:methodology-attacks}

% % Figure environment removed


\subsection{Network Attacks}
\label{sec:network-attacks} 

\subsubsection{Connection Hijack}
\label{sec:mitm}
These attacks aim to compromise the communication channel between wallets and other network participants using \acs{mitm} attacks to intercept and modify the \teal{$txn$} message generated by \hyperref[algo:transaction-signing]{Algorithm 3}. Various types of \acs{mitm} include Rogue \acs{ap} \cite{Hu2021SecurityCountermeasures}, \acs{dns} spoofing \cite{Ahmed2017MitigatingNetworking, Al-Mashhadi2020ASystems}, \acs{ip} spoofing \cite{shrivas2020disruptive} and \acf{icmp} redirection \cite{Feng2023Man-in-the-middleRedirects} as shown in \autoref{tab:attack_vectors}. Any software which allows users to manage or import the private key is vulnerable to these attacks. For example, EtherDelta, a \acs{dex} which allows users to import \teal{$sk$} was a victim of a \acs{mitm} attack following a \acs{dns} server compromise. Hardware wallets are also vulnerable to these attacks if the online wallet client (see \autoref{sec:hardware-wallets}) is compromised. Ledger has previously reported susceptibility to \acs{mitm} attacks.  

\subsubsection{Service Denial}
\label{sec:dos}

This is executed using adversary-controlled devices to orchestrate \acf{ddos} attacks which overwhelm the network infrastructure with an excessive volume of requests causing a decline or cessation of the wallet operations (see \autoref{sec:wallet_mechanism}) \cite{ChandanProtectiveCryptocoin}. These attacks often target the \acf{icmp}, \acf{tcp} handshake mechanism and other network infrastructure \cite{chaganti2022comprehensive}. One common medium of conducting a \acs{ddos} attack is through botnets, which involves an adversary using a network of computers \cite{krombholz2015advanced}. 


% The transaction management operations (see \autoref{sec:transaction_management}) in software wallets and the transaction broadcast function in hardware wallets are affected by this attack.


% % Figure environment removed

\subsection{Application Attacks}
\label{sec:application-attacks}

\subsubsection{Malware Execution}
\label{sec:malware}

This intrusively exploits system vulnerabilities to steal transaction data, the \teal{$sk$} and password credentials, or to manipulate wallet operations as described in \autoref{sec:wallet_mechanism}. Malware threatens the wallet mechanism by replacing the \teal{$recipient\_address$} via a clipboard hijacker \cite{li2020android} or input monitoring via keyloggers \cite{balakrishnan2023analysis} and other spyware types \cite{weichbroth2023security, ferdous2023review}. Hardware wallets are also vulnerable to clipboard hijack attacks \cite{ivanov2021ethclipper, Akter2023AChallenges}; malware can be injected through interactions between the wallet and removable media such as USB drives \cite{guri2018beatcoin}. Several studies have investigated malware execution on hardware wallets.

% \paragraph{Clipboard Hijacker}
% \label{sec:clipboard}

% This attack involves the use of malware to exploit the \enquote{ClipboardManager} component of a wallet device to monitor the contents of a user’s clipboard and steal the \teal{$pwd$} or replace the \teal{$recipient\_addr$} with the adversary's \cite{ulqinaku2019scan, li2020android}.

% This attack exploits the fact wallet addresses are difficult to type out.

% \paragraph{Keylogger}
% \label{sec:keylogger}

% Keyloggers can track every keystroke executed on an infected wallet device to steal password details (\teal{$pwd$}) and other confidential data \cite{balakrishnan2023analysis}.

% These programmes can intercept data transmitted by application programming interfaces (APIs), which enable other apps to accept keyboard input, despite not having administrator privileges.Software keyloggers are far more prevalent since they are much simpler to introduce and install on victims' devices. Software keyloggers do not pose a hazard to the systems they infect, in contrast to other types of malwares. In actuality, keyloggers are designed to function invisibly, sniffing out keystrokes while the machine is still in use. Keyloggers are undoubtedly a menace to consumers, even if they don't damage the hardware, particularly when they take private information.

% \paragraph{Ransomware}
% \label{sec:ransom}

% This malware specifically targets the file system, where the adversary encrypts data to deny user access (see \autoref{sec:key-storage}) to the wallet \cite{conti2018economic}. After successful infiltration, the adversary encrypts files on the wallet device and extorts a ransom in exchange for access to the encrypted data \cite{kapoor2021ransomware}.  

% \paragraph{Spyware}
% \label{sec:spyware}

% This is malware engineered to monitor the user's actions and retrieve confidential data such as the user's \teal{$pwd$} or the \teal{$sk$} \cite{weichbroth2023security, ferdous2023review}. It includes keyloggers which track every keystroke executed on an infected wallet device to steal confidential data \cite{shaikh2022survey, balakrishnan2023analysis}.

% \paragraph{Malicious Files}
% need access to the reference to write on this 


% \paragraph{Supply Chain Attack}
% \label{sec:supply}


% Supply Chain Attacks compromise wallet security by embedding malicious code during software or hardware development phases \cite{robinson2022new}. Such breaches can fundamentally undermine the integrity of key generation and transaction management mechanisms (\autoref{sec:transaction_management}), exposing sensitive operations to the adversary. An example of this attack occurred in the BitKeep wallet hack, where the compromise of the application update mechanism resulted in the hijack of the non-custodial wallet's (see \autoref{sec:non-custodial-wallets}) website and the loss of funds for users \cite{certik_2022}.

% \paragraph{Removable Media Infection}
% \label{sec:removable}

% This attack method capitalises on the interactions between hardware wallets described in \autoref{sec:hardware-wallets} and removable media, such as USB drives, to inject malware during critical processes like transaction signing (\autoref{sec:transaction_signing}) \cite{guri2018beatcoin}. By compromising these removable devices, adversaries can disrupt the secure execution of transaction creation (\autoref{sec:transaction_signing}).

\subsubsection{Social Engineering}
\label{sec:social}
These attacks aim to manipulate the user into divulging confidential data. Phishing attacks, for instance, aim to deceive wallet users into revealing \teal{$sk$} or \teal{$pw$} by mimicking legitimate services. If successful, the adversary can use supplementary attack vectors to gain unauthorised access \cite{krombholz2015advanced}. Instances where adversaries have employed phishing to deliver malware include the Pink Drainer, Monkey Drainer, Venom Drainer and Inferno attacks \autoref{tab:attack-incidents}. 



% automatic Github
% crypto-wallets-mapping.drawio.pdf
% scale=0.80

% crypto-wallets-mapping.drawio.drawio.pdf
% scale=0.40


\subsubsection{Privilege Escalation}
\label{sec:privilege}

These attacks aim to circumvent standard access controls to acquire elevated permissions. In the Android root privilege attack, the adversary can gain unauthorised root access to mobile wallets via vulnerabilities in the \acf{os} \cite{he2020security}. Another OS-related attack, Android USB debugging \cite{he2020security}, exploits \acf{os} vulnerabilities in mobile devices by wireless debugging, using a computer connected to the same network. Following this, the adversary gains unrestricted access to manipulate the execution flow of the wallet and capture \teal{$sk$}, \teal{$rdm\_seed$} and other sensitive data \cite{he2020security}. Logic Flow Exploitation encompasses several wallet types and involves the identification and exploitation of flaws in the programming logic of a wallet mechanism (\autoref{sec:wallet_mechanism}) to gain unauthorised access or manipulate wallet functions \cite{Parisi2023WalletSecurity}. The WazirX and parity wallet attacks are notable examples of this attack \cite{palladino2017parity}.

% This attack exploits the Android’s Linux-based operating system vulnerabilities of mobile wallets to gain unauthorised administrative control; therefore, giving the adversary unrestricted permission \cite{he2020security}. With escalated access, the adversary can manipulate the wallet function and steal the private key.


\subsection{Authentication Attacks}
\label{sec:auth-attacks}

% % Figure environment removed

\subsubsection{Credential Cracking}
\label{sec:cred-crack}

This category of attacks systematically attempts different credential values to bypass the authentication mechanism. Brute force attacks involve an adversary systematically trying all possible character combinations to bypass the authentication function and decrypt the \teal{$sk$}. If successful, the adversary can create malicious transactions using the \hyperref[algo:transaction-signing]{Algorithm 3} \cite{Kiktenko2019DetectingWallets}. Dictionary attacks, on the other hand, leverage commonly used words to predict \teal{$rdm\_seed$} phrases or passphrases for access. Unlike brute force attacks that exhaust all possible combinations, dictionary attacks are computationally less demanding \cite{Uddin2021Horus:Wallets}. Their success rate increases with the use of leaked password datasets \cite{praitheeshan2019security}.



% Prevents the figure from moving too far from the table
% \FloatBarrier

% \clearpage


% These attacks require substantial computational resources and may utilize multiple machines operating in parallel, making the process highly time-consuming \cite{volety2019cracking}

% \subsubsection{Dictionary Attacks}
% \label{sec:dictionary}

% \subsubsection{Code Reuse}
% \label{sec:code-reuse}

% These attacks manipulate the intended functionality of software components to alter execution flow and bypass security measures without necessarily injecting new code \cite{palladino2017parity}. An adversary can exploit existing code or logic constructs in the wallet mechanism to perform unauthorised actions \cite{bletsch2011jump}.

% A likely method of attack for the Gatecoin incident

% the malicious external party involved in this breach, managed to alter our system so that ETH deposit transfers by-passed the multi-sig cold storage and went directly to the hot wallet during the breach period

  
 % By leveraging inherent flaws, misconfigurations, or oversight in the system's design or implementation, attackers orchestrate unintended behaviour that can compromise security, integrity, or availability of the targeted system.



% \subsubsection{Evil Maid Attack}
% \label{sec:evil-maid}


% This attack leverages areas of a wallet device's storage that remain unencrypted, enabling attackers who gain physical access to substitute parts of the system with a maliciously modified version, which eventually leads to bypassing user authentication (\autoref{sec:key-storage}) \cite{altuwaijri2020android}. When the wallet owner next accesses their device, the tampered system could capture and leak their credentials to the attacker. Following the owner's next wallet use, the manipulated system can acquire and transfer credentials to the attacker.

\subsubsection{Identity Spoofing}
\label{sec:iden-spoof}

These involve an adversary's impersonating the user's identity to bypass the user verification mechanism and decrypt \teal{$sk$}. These include fake biometric attacks \cite{galbally2013image} which provide synthetic or reconstructed biometric data, and SIM swap attacks \cite{Kim2022ACountermeasures} which aim to bypass SMS-based 2FA and other identify spoofing attacks.

% \subsubsection{Communication Manipulation}
% \label{sec:fake-biometrics}

% This involves the adversary's attempts to bypass the authentication mechanisms by exploiting weaknesses in communication. This includes replay attacks, where the adversary captures a valid authentication message and resends it to the system to gain unauthorised access. 

% Resilience Against Replay Attacks. In our context au- thentication is tied to a specific bitstring τ, e.g., we want to make sure that an unusual transaction τ is initiated by the legitimate user. It is therefore important to ensure that previous successful authentications for a string τ cannot be mauled into successful authentications for a different string τ′, without the knowledge of the secret.


% Encryption of sensitive data. Once the session key is established slightly modified versions of the four Ledger protocols (Alive, Login, Setup, and Pay- ment) can be executed. The four new protocols are derived from the original Ledger protocols as follows. First a session identifier is established for each execution of each of these protocols. This will be generated dongle side, and transmitted to the API in plaintext. The session identifier does not need to be confidential, but will need to be fresh and generated by the dongle to avoid replay attacks. Then dongle and API execute the original protocol but encrypting under the current session key the sensitive data identified previously (Table 5). The computed ciphertexts will all include the established session identifier. A Message Authentication Code (MAC) is further computed and concatenated to the chiphertext. The other party will then be able to decrypt and verify the encrypted parts.


% exploiting the lack of time-sensitive or unique identifiers that prevent the reuse of such data.

% In the context of wallets, communication manipulation can target the exchange of authentication credentials or transaction data between the wallet and the blockchain network or a connected server. Attackers may intercept and replay valid authentication tokens or transaction requests, tricking the wallet into granting unauthorized access or executing unintended transactions. By leveraging flaws in session handling, non-unique tokens, or the absence of adequate cryptographic protections, attackers can effectively impersonate the legitimate user or gain control over their assets. This category includes attacks like replay attacks, where the adversary captures a valid authentication or transaction message and resends it to the system to gain unauthorized access, exploiting the lack of time-sensitive or unique identifiers that prevent the reuse of such data.

% \subsubsection{Physical Access}

% \subsubsection{Shoulder Surfing}

% \label{sec:shoulder-surfing}
% This is an attack that involves secretly monitoring users' input wallet credentials such as \teal{$recipient\_addr$}, \teal{$sk$} or \teal{$pwd$} to steal these credentials and gain unauthorised access to their crypto-wallets \cite{yasin2019contemplating}.

% \subsubsection{Fault Injection}
% \label{sec:fault-inj}

% These attacks manipulate the wallet's physical components and induce hardware errors to exploit vulnerabilities and access sensitive data \cite{hajdu2020using}. By manipulating the hardware's physical state, adversaries can induce several errors within computational processes to force it into an erroneous state and bypass the security mechanisms (\autoref{sec:wallet_mechanism}).

% This attack induces errors in a system to expose and exploit vulnerabilities, using software, hardware, or environmental methods. It aims to compromise a system's security or functionality, enabling unauthorized access or causing system failures. 

% In this attack, an adversary with physical access exploits the data remanence properties of \acs{ram} i.e. \acf{dram} and \acf{sram} in some wallet devices to retrieve \teal{$enc\_secret\_key$}, \teal{$pwd$}, or other sensitive data from memory after a cold restart \cite{shaikh2022survey}. 

\subsection{Storage \& Memory Attacks}
\label{sec:physical-attacks}

\subsubsection{Physical Tampering}
\label{sec:tam-per}

These primarily involve physically altering a wallet’s hardware to bypass security protections. In an evil maid attack, the attacker physically modifies the unencrypted storage of an unattended device to capture credentials or manipulate the system \cite{altuwaijri2020android}. In contrast, microscopy attacks use advanced techniques, such as electron microscopy, to examine the microelectronic components of a wallet and extract critical data or identify vulnerabilities, often without altering the hardware itself \cite{courbon2016reverse}.

\begin{table}[!htbp]
\centering
\tiny
\setlength{\tabcolsep}{2.1pt}
\renewcommand{\arraystretch}{0.9}
\begin{tabular}{lllrll}
\toprule
\textbf{Name} & \textbf{Custody Design} & \textbf{Date} & \textbf{Loss (\$)} & \textbf{Attack Category} & \textbf{Attack Name} \\
\midrule
ByBit \cite{bybit}  
  & Custodial  
  & %21/02/2025 
    2025-02 
  & 1,500M  
  & Application  
  & Logic Exploitation \\

US Govt. \cite{Decrypt}  
  & Non-Custodial  
  & %25/10/2024 
    2024-10 
  & 50M  
  & –  
  & – \\

BigX \cite{Explained:2024}  
  & Custodial  
  & %20/09/2024 
    2024-09 
  & 52M  
  & –  
  & – \\

Indodax \cite{IndonesianTRX}  
  & Custodial  
  & %11/09/2024 
    2024-09 
  & 22M  
  & –  
  & – \\

WazirX \cite{Explained:2024g}  
  & Custodial  
  & %18/07/2024 
    2024-07 
  & 235M  
  & Application  
  & Logic Exploitation \\

Bittensor \cite{Explained:2024}  
  & Non-Custodial  
  & %02/07/2024 
    2024-07 
  & 8M  
  & Application  
  & Malware \\

BTCTurk \cite{Explained:2024}  
  & Custodial  
  & %23/06/2024 
    2024-06 
  & 55M  
  & –  
  & – \\

Loopring \cite{Explained:2024}  
  & Non-Custodial  
  & %09/06/2024 
    2024-06 
  & 5M  
  & Authentication  
  & Identity Spoofing\textsuperscript{*} \\

Lykke \cite{CoinTelegraph}  
  & Custodial  
  & %04/06/2024 
    2024-06 
  & 22M  
  & –  
  & – \\

DMM Bitcoin \cite{Explained:2024}  
  & Custodial  
  & %31/05/2024 
    2024-05 
  & 305M  
  & –  
  & – \\

Axie Co-Founder \cite{Decrypt}  
  & Non-Custodial  
  & %23/02/2024 
    2024-02 
  & 10M  
  & –  
  & – \\

Fixed Float \cite{Explained:2024}  
  & Custodial  
  & %16/02/2024 
    2024-02 
  & 26.1M  
  & –  
  & – \\

kirilm.eth \cite{Explained:2024}  
  & Non-Custodial  
  & %16/02/2024 
    2024-02 
  & 5.1M  
  & Application  
  & Phishing \\

Ripple Co-Founder \cite{RippleMillion}  
  & Non-Custodial  
  & %30/01/2024 
    2024-01 
  & 112.5M  
  & –  
  & – \\

HTX (Huobi) \cite{HTXReport}  
  & Custodial  
  & %22/11/2023 
    2023-11 
  & 13.6M  
  & –  
  & \teal{\textit{sk}} Compromise\textsuperscript{*} \\

Pink Drainer \cite{RektREKT}  
  & Non-Custodial  
  & %16/11/2023 
    2023-11 
  & 12M  
  & Application  
  & Phishing, Malware \\

Monkey Drainer \cite{RektREKT}  
  & Non-Custodial  
  & %16/11/2023 
    2023-11 
  & 16M  
  & Application  
  & Phishing, Malware \\

Venom Drainer \cite{RektREKT}  
  & Non-Custodial  
  & %16/11/2023 
    2023-11 
  & 27M  
  & Application  
  & Phishing, Malware \\

Infarno \cite{infarno}  
  & Non-Custodial  
  & %16/11/2023 
    2023-11 
  & 66M  
  & Application  
  & Phishing, Malware \\

Poloniex \cite{RektREKT}  
  & Custodial  
  & %10/11/2023 
    2023-11 
  & 126M  
  & –  
  & \teal{\textit{sk}} Compromise\textsuperscript{*} \\

Lastpass \cite{RektREKT}  
  & Non-Custodial  
  & %31/10/2023 
    2023-10 
  & 37M  
  & Authentication  
  & – \\

Fantom Fdn. \cite{AnalysisMedium}  
  & Non-Custodial  
  & %18/10/2023 
    2023-10 
  & 7M  
  & –  
  & – \\

HTX (Huobi) \cite{HTXReport}  
  & Custodial  
  & %25/09/2023 
    2023-09 
  & 8M  
  & Application  
  & Phishing \\

Fake Voucher \cite{RektREKT}  
  & Non-Custodial  
  & %20/09/2023 
    2023-09 
  & 4.5M  
  & Application  
  & Phishing \\

Remitano \cite{RektREKT}  
  & Custodial  
  & %15/09/2023 
    2023-09 
  & 2.7M  
  & Application  
  & – \\

CoinEx \cite{CoinTelegraph}  
  & Custodial  
  & %12/09/2023 
    2023-09 
  & 55M  
  & –  
  & \teal{\textit{sk}} Compromise\textsuperscript{*} \\

Monero \cite{MoneroFlash}  
  & Non-Custodial  
  & %01/09/2023 
    2023-09 
  & 0.5M  
  & –  
  & – \\

AlphaPo \cite{RektREKT}  
  & Custodial  
  & %26/07/2023 
    2023-07 
  & 60M  
  & –  
  & \teal{\textit{sk}} Compromise\textsuperscript{*} \\

Atomic Wallet \cite{CoinTelegraph}  
  & Non-Custodial  
  & %03/06/2023 
    2023-06 
  & 100M  
  & –  
  & – \\

Bitrue \cite{Explained:2024}  
  & Custodial  
  & %14/04/2023 
    2023-04 
  & 23M  
  & –  
  & \teal{\textit{sk}} Compromise\textsuperscript{*} \\

GDAC \cite{CoinTelegraph}  
  & Custodial  
  & %09/04/2023 
    2023-04 
  & 13M  
  & –  
  & \teal{\textit{sk}} Compromise\textsuperscript{*} \\

MyAlgo \cite{CoinTelegraph}  
  & Non-Custodial  
  & %27/02/2023 
    2023-02 
  & 9.2M  
  & –  
  & – \\

BitKeep \cite{CertiKIncidents}  
  & Non-Custodial  
  & %26/12/2022 
    2022-12 
  & 8M  
  & Application  
  & Phishing, Malware \\

FTX \cite{FTXMistake}  
  & Custodial  
  & %12/11/2022 
    2022-11 
  & 450M  
  & Authentication  
  & Sim Swap Attack \\

Deribit \cite{CryptoWithdrawals}  
  & Custodial  
  & %01/11/2022 
    2022-11 
  & 28M  
  & Application  
  & – \\

Wintermute \cite{TheMedium}  
  & Custodial  
  & %20/09/2022 
    2022-09 
  & 160M  
  & Authentication  
  & Brute force \\

Slope \cite{CoinTelegraph}  
  & Non-Custodial  
  & %02/08/2022 
    2022-08 
  & 8M  
  & Storage and Memory  
  & – \\

MetaMask \cite{CertiKIncidents}  
  & Non-Custodial  
  & %17/04/2022 
    2022-04 
  & 0.65M  
  & Authentication  
  & Phishing \\

Crypto.com \cite{Explained:2024}  
  & Custodial  
  & %17/01/2022 
    2022-01 
  & 30M  
  & Authentication  
  & – \\

Lympo \cite{CoinTelegraph}  
  & Custodial  
  & %10/01/2022 
    2022-01 
  & 18.7M  
  & –  
  & – \\

LCX \cite{LookingHacken}  
  & Custodial  
  & %08/01/2022 
    2022-01 
  & 8M  
  & –  
  & \teal{\textit{sk}} Compromise\textsuperscript{*} \\

Vulcan Forged \cite{VulcanHack}  
  & Non-Custodial  
  & %13/12/2021 
    2021-12 
  & 140M  
  & Application  
  & \teal{\textit{sk}} Compromise\textsuperscript{*} \\

BitMart \cite{HackScience}  
  & Custodial  
  & %05/12/2021 
    2021-12 
  & 196M  
  & Application  
  & Phishing \\

Liquid \cite{HackBreach}  
  & Custodial  
  & %19/08/2021 
    2021-08 
  & 90M  
  & Application  
  & \teal{\textit{sk}} Compromise\textsuperscript{*} \\

Roll \cite{CoinDesk}  
  & Custodial  
  & %14/03/2021 
    2021-03 
  & 5.7M  
  & Application  
  & \teal{\textit{sk}} Compromise\textsuperscript{*} \\

MetaMask \cite{Explained:2024}  
  & Non-Custodial  
  & %14/12/2020 
    2020-12 
  & 8M  
  & –  
  & – \\

KuCoin \cite{kucoinNew}  
  & Custodial  
  & %25/09/2020 
    2020-09 
  & 275M  
  & Application  
  & \teal{\textit{sk}} Compromise\textsuperscript{*} \\

Cashaa \cite{CoinTelegraph}  
  & Custodial  
  & %11/07/2020 
    2020-07 
  & 3.1M  
  & Application  
  & Malware \\

Trinity Wallet \cite{IOTA:Wallet}  
  & Non-Custodial  
  & %12/02/2020 
    2020-02 
  & 2.3M  
  & Application  
  & – \\

Altsbit \cite{AltsbitZDNET}  
  & Custodial  
  & %05/02/2020 
    2020-02 
  & 72.5M  
  & Application  
  & – \\

Upbit \cite{UpbitMedium}  
  & Custodial  
  & %26/11/2019 
    2019-11 
  & 49M  
  & Application  
  & Phishing, Malware \\

Bitpoint \cite{BitPointMedium}  
  & Custodial  
  & %11/07/2019 
    2019-07 
  & 36.5M  
  & –  
  & – \\

Vindax \cite{VinDAXBlock}  
  & Custodial  
  & %05/11/2019 
    2019-11 
  & 0.5M  
  & –  
  & – \\

Bitrue \cite{CryptoNews}  
  & Custodial  
  & %27/06/2019 
    2019-06 
  & 4.5M  
  & Authentication  
  & – \\

Gatehub \cite{OverviewMedium}  
  & Custodial  
  & %06/06/2019 
    2019-06 
  & 9.5M  
  & –  
  & – \\

Binance Exchange \cite{binanceNew}  
  & Custodial  
  & %07/05/2019 
    2019-05 
  & 40M  
  & Unknown  
  & – \\

Bithumb \cite{CoinDesk}  
  & Custodial  
  & %29/03/2019 
    2019-03 
  & 13M  
  & Other  
  & Insider Job \\

Coinbene \cite{CoinTelegraph}  
  & Custodial  
  & %25/03/2019 
    2019-03 
  & 99M  
  & –  
  & – \\

DragonEX \cite{CoinDesk}  
  & Custodial  
  & %24/03/2019 
    2019-03 
  & 1M  
  & Application  
  & – \\

Cryptopia \cite{HowHacken}  
  & Custodial  
  & %01/02/2019 
    2019-02 
  & 16M  
  & –  
  & \teal{\textit{sk}} Compromise\textsuperscript{*} \\

LocalBitcoins \cite{CoinDesk}  
  & Custodial  
  & %26/01/2019 
    2019-01 
  & 0.02M  
  & Application  
  & Phishing \\

Electrum \cite{DeepSwig}  
  & Non-Custodial  
  & %21/12/2018 
    2018-12 
  & 0.75M  
  & Application  
  & Phishing \\

Maplechange \cite{MapleChangeInvestorPlace}  
  & Custodial  
  & %28/10/2018 
    2018-10 
  & 6M  
  & –  
  & – \\

Zaif \cite{CoinDesk}  
  & Custodial  
  & %14/09/2018 
    2018-09 
  & 100M  
  & –  
  & – \\

Coinrail \cite{CoinDesk}  
  & Custodial  
  & %10/06/2018 
    2018-06 
  & 40M  
  & –  
  & – \\

MyEtherWallet \cite{myetherwallet}  
  & Non-Custodial  
  & %24/04/2018 
    2018-04 
  & 0.15M  
  & Network  
  & \acs{bgp} Hijacking \\

Gate.io \cite{ZachXBTWraps}  
  & Custodial  
  & %18/04/2018 
    2018-04 
  & 234M  
  & –  
  & – \\

CoinSecure \cite{CoinDesk}  
  & Custodial  
  & %13/04/2018 
    2018-04 
  & 3.5M  
  & Other  
  & Insider Job \\

Bitgrail \cite{BitGrailCoinMarketCap}  
  & Custodial  
  & %10/02/2018 
    2018-02 
  & 146M  
  & Other  
  & Insider Job \\

CoinCheck \cite{TheHack}  
  & Custodial  
  & %27/01/2018 
    2018-01 
  & 560M  
  & –  
  & – \\

BlackWallet \cite{BlackWalletFault}  
  & Non-Custodial  
  & %15/01/2018 
    2018-01 
  & 0.4M  
  & Network  
  & \acs{dns} Spoofing \\

EtherDelta \cite{CryptocurrencyScheme}  
  & Custodial  
  & %20/12/2017 
    2017-12 
  & 1.4M  
  & Network  
  & \acs{dns} Spoofing \\

Parity \cite{palladino2017parity}  
  & Non-Custodial  
  & %19/07/2017 
    2017-07 
  & 30M  
  & Application  
  & Logic Exploitation \\

Yapizon \cite{CoinTelegraph}  
  & Custodial  
  & %22/04/2017 
    2017-04 
  & 5.3M  
  & –  
  & – \\

Bitfinex \cite{CoinDesk}  
  & Custodial  
  & %02/08/2016 
    2016-08 
  & 623M  
  & Application  
  & – \\

Gatecoin \cite{CoinDesk}  
  & Custodial  
  & %09/05/2016 
    2016-05 
  & 2.1M  
  & –  
  & – \\

Shapeshift \cite{LootingShapeShift}  
  & Custodial  
  & %07/04/2016 
    2016-04 
  & 0.23M  
  & Other  
  & Insider Job \\

Bitstamp \cite{DetailsRevealed}  
  & Custodial  
  & %11/12/2015 
    2015-12 
  & 5M  
  & Application  
  & Phishing \\

BTER \cite{CoinDesk}  
  & Custodial  
  & %15/08/2015 
    2015-08 
  & 1.65M  
  & Application  
  & – \\

Mintpal \cite{RememberingLedger}  
  & Custodial  
  & %13/07/2014 
    2014-07 
  & 2M  
  & Other  
  & Insider Job \\

Poloniex \cite{PoloniexHack}  
  & Custodial  
  & %04/03/2014 
    2014-03 
  & 0.05M  
  & Application  
  & – \\

Mt. Gox \cite{mtgox_hack}  
  & Custodial  
  & %24/02/2014 
    2014-02 
  & 460M  
  & –  
  & – \\

Bitcash \cite{CzechEmptied}  
  & Custodial  
  & %11/11/2013 
    2013-11 
  & 0.1M  
  & Application  
  & Phishing \\

Bitfloor \cite{HackSecurityWeek}  
  & Custodial  
  & %12/09/2012 
    2012-09 
  & 0.25M  
  & Application  
  & \teal{\textit{sk}} Compromise\textsuperscript{*} \\

Bitcoinica \cite{ExchangeStolen}  
  & Custodial  
  & %01/03/2012 
    2012-03 
  & 0.09M  
  & Application  
  & \teal{\textit{sk}} Compromise\textsuperscript{*} \\

\midrule
\textbf{Summary:}
  & \textbf{85 incidents}
  & \textbf{2012–2025}
  & \textbf{6.98B}
  &  
  &  
\\
\bottomrule
\end{tabular}
\caption{Wallet attack incidents in the industry. We retrieve 85 notable attack incidents involving both custodial and non-custodial wallets. Several attack methods remain unknown (–) or undetailed, we indicate undetailed incidents with \textsuperscript{*}.}
\label{tab:attack-incidents}
\end{table}


\begin{landscape}
% \vspace*{\fill}
% \begin{table}[!htbp] !p
% \begin{sidewaystable*}[!htbp]
% \begin{table*}[!p]
\begin{table}[!htbp]
% \thispagestyle{empty} 
\centering
\caption{
Three-level attack classification showing gap analysis, threat occurrences, adversary's target and mapping to possible security measures (\autoref{sec:defense-strategies}). The \enquote{Gaps} summary shows that academic literature covers 24 of the 28 enumerated attack vectors (86\%), whereas publicly reported incidents cover 9 vectors (32\%). Notable incident percentages are calculated from a total of 85 reported industry incidents (see \autoref{tab:attack-incidents}). Symbols: ( \smallfullcirc : include, \smallhalfcirc : part-inclusion (influenced by other factors), \smallemptycirc : not include) }
\label{tab:attack_vectors}
\tiny
% \renewcommand{\arraystretch}{1}
% \setlength{\tabcolsep}{1pt}
\renewcommand{\arraystretch}{1}
\setlength{\tabcolsep}{1.5pt} 
% \setlength{\tabcolsep}{1pt} 
\resizebox{\linewidth}{!}
{
\begin{tabular}{llllccccccccccccccccccccccccccccccccccccccccccccccccl}
% llllccccccccccccccccccccccccccccccccccccccccccccccccccccccccccccccccccccl
\toprule
\multicolumn{1}{c}{\textbf{Category}} &
  \multicolumn{1}{c}{\multirow{1}{*}{\textbf{\hyperref[sec:attack-framework]{ \textbf{Method}}}}} &
  \multicolumn{1}{c}{\multirow{1}{*}{\textbf{\hyperref[sec:attack-framework]{\textbf{Vector}}}}} &
  \multicolumn{21}{c}{\textbf{\hyperref[sec:threat_framework]{Threat}}} &
  % \multicolumn{1}{p{2.15cm}}{\textbf{Description}} &
  \multicolumn{16}{c}{\textbf{\hyperref[sec:wallet_mechanism]{Target}}} & 
  % \multicolumn{3}{c}{\textbf{\hyperref[sec:threat_class]{Threat}}} &
  \multicolumn{3}{c}{\textbf{\hyperref[sec:adversary_goal]{Goal}}} &
  \multicolumn{5}{c}{\textbf{\hyperref[sec:infrastructure]{Infrastructure}}} &
  \multicolumn{2}{c}{\textbf{\hyperref[sec:attacks_discussion]{Gaps}}} &
  \multicolumn{1}{c}{\textbf{\hyperref[sec:defense-strategies]{Possible Defence}}} \\ 
  % \multicolumn{25}{c}{\textbf{\hyperref[sec:defense-strategies]{Possible Defence Methods}}} \\ 
\cmidrule(lr){4-40} 
% \cmidrule(lr){45-47}
% \multicolumn{1}{c}{\multirow{7}{*}{\rotatebox[origin=l]{90}{\textbf{Attack Category}}}} 
% \multicolumn{1}{c}{\textbf{Category}}
&
  \multicolumn{1}{c}{} &
  \multicolumn{1}{c}{} &
  \multicolumn{12}{c}{} &
  % \multicolumn{12}{c}{\textbf{Mechanism Vuln.}} &
  \multicolumn{5}{c}{} &
  % \multicolumn{5}{c}{\textbf{Syst. Vuln.}} &
  \multicolumn{2}{c}{} &
  % \multicolumn{2}{c}{\textbf{Ex.}} &
  \multicolumn{2}{c}{} &
  % \multicolumn{2}{c}{\textbf{In.}} &
  \multicolumn{7}{c}{\textbf{Data}} &
  \multicolumn{6}{c}{\textbf{Mechanism}} &
  \multicolumn{3}{c}{\textbf{Other}} &
  % \multicolumn{3}{c}{} &
  \multicolumn{3}{c}{} &
  \multicolumn{1}{l}{} &
  \multicolumn{4}{l}{} &
  \multicolumn{2}{l}{} &
  \multicolumn{1}{l}{} 
  % \multicolumn{8}{c}{\textbf{Auth. Bypass}} &
  % \multicolumn{2}{c}{\textbf{Disrp.}} &
  % \multicolumn{4}{c}{\textbf{Intrusion}} &
  % \multicolumn{5}{c}{\textbf{Alter.}} &
  % \multicolumn{7}{c}{\textbf{Extraction}}
  &
   \\
% \cmidrule(lr){4-15} 
% \cmidrule(lr){16-20} 
% \cmidrule(lr){21-22} 
% \cmidrule(lr){23-24} 
\cmidrule(lr){25-31} \cmidrule(lr){32-37} \cmidrule(lr){38-40} 

% \cmidrule(lr){45-53} \cmidrule(lr){54-55} \cmidrule(lr){56-59} \cmidrule(lr){60-64} \cmidrule(lr){65-72}
\multicolumn{1}{c}{} &
  \multicolumn{1}{c}{} &
  \multicolumn{1}{c}{} &
\multicolumn{1}{c}{\rotatebox[origin=l]{90}{Predictable \acs{rng} \cite{brengel2018identifying, cve_31290, cve_23660}}} &
\multicolumn{1}{c}{\rotatebox[origin=l]{90}{Inadequate Authentication \cite{Uddin2021Horus:Wallets}}} &
\multicolumn{1}{c}{\rotatebox[origin=l]{90}{Inadequate Encryption \cite{cve_15947}}} &
\multicolumn{1}{c}{\rotatebox[origin=l]{90}{Application Logic Flaw \cite{Destefanis2018SmartEngineering, Parisi2023WalletSecurity, oren2023fireblocks}}} &
\multicolumn{1}{c}{\rotatebox[origin=l]{90}{Low-strength Passwords \cite{Kiktenko2019DetectingWallets, volety2019cracking}}} &
\multicolumn{1}{c}{\rotatebox[origin=l]{90}{Data Leakage \cite{cve_14353, cve_14354, KrakenBlog}}} &
\multicolumn{1}{c}{\rotatebox[origin=l]{90}{Data Remanence \cite{trezor_memory, trezor_medium}}} &
\multicolumn{1}{c}{\rotatebox[origin=l]{90}{Data Manipulation \cite{trezor_memory, trezor_medium}}} &
\multicolumn{1}{c}{\rotatebox[origin=l]{90}{Insecure Boot Environment \cite{Shaikh2022SurveyExchanges}}} &
\multicolumn{1}{c}{\rotatebox[origin=l]{90}{Microelectronic Component Exposure \cite{courbon2016reverse}}} &
\multicolumn{1}{c}{\rotatebox[origin=l]{90}{Weak Signature \cite{Rokhjavan2023SecuringWallets}}} &
\multicolumn{1}{c}{\rotatebox[origin=l]{90}{Inadequate Signature Verification \cite{cve_14199, tymokhanov2021alpha}}} &
\multicolumn{1}{c}{\rotatebox[origin=l]{90}{Insecure Permissions \cite{cve_32969, halborn_vuln}}} &
\multicolumn{1}{c}{\rotatebox[origin=l]{90}{Library Vulnerability \cite{bitcore_lib, Ledger2023SecurityReport} }} &
\multicolumn{1}{c}{\rotatebox[origin=l]{90}{\acs{os} Vulnerabilities \cite{he2020security}}} &
\multicolumn{1}{c}{\rotatebox[origin=l]{90}{Coding Errors \cite{Parisi2023WalletSecurity}}} &
\multicolumn{1}{c}{\rotatebox[origin=l]{90}{Insec. Network \cite{cve_33297, cve_14198, cve_17144}}} &
\multicolumn{1}{c}{\rotatebox[origin=l]{90}{Insec. User Interactions \cite{ZengoZengo, thodex}}} & 
\multicolumn{1}{c}{\rotatebox[origin=l]{90}{Comp. Provider \cite{CoinTelegraph2022SlopeAttack}}} &
\multicolumn{1}{c}{\rotatebox[origin=l]{90}{Malicious Insider \cite{decrypt_ftx}}} &
\multicolumn{1}{c}{\rotatebox[origin=l]{90}{Compromised Insider \cite{Ledger2023SecurityReport}}} &
\multicolumn{1}{c}{\rotatebox[origin=l]{90}{Private Key (\teal{$sk$})}} &
\multicolumn{1}{c}{\rotatebox[origin=l]{90}{Signature (\teal{$\sigma$})}} &
\multicolumn{1}{c}{\rotatebox[origin=l]{90}{Mnemonics (\teal{$rdm\_seed$})}} &
\multicolumn{1}{c}{\rotatebox[origin=l]{90}{\acs{kek} or Password (\teal{$pw$})}} &
\multicolumn{1}{c}{\rotatebox[origin=l]{90}{Memory}} &
\multicolumn{1}{c}{\rotatebox[origin=l]{90}{State Trans. Info.}} &
\multicolumn{1}{c}{\rotatebox[origin=l]{90}{Nonce}} &
\multicolumn{1}{c}{\rotatebox[origin=l]{90}{KeyGen}} &
\multicolumn{1}{c}{\rotatebox[origin=l]{90}{UserAuth}} &
\multicolumn{1}{c}{\rotatebox[origin=l]{90}{KeyStore}} &
\multicolumn{1}{c}{\rotatebox[origin=l]{90}{TxnInit}} &
\multicolumn{1}{c}{\rotatebox[origin=l]{90}{TxnSign}} &
\multicolumn{1}{c}{\rotatebox[origin=l]{90}{TxnVer}} &
  \multicolumn{1}{c}{\rotatebox[origin=l]{90}{Service Provider}} &
  % \multicolumn{1}{c}{\rotatebox[origin=l]{90}{Network Connection}} &
  \multicolumn{1}{c}{\rotatebox[origin=l]{90}{Operating System}} &
  \multicolumn{1}{c}{\rotatebox[origin=l]{90}{Wallet User}} &
  % \multicolumn{1}{c}{\rotatebox[origin=l]{90}{System}} &
  % \multicolumn{1}{c}{\rotatebox[origin=l]{90}{External}} &
  % \multicolumn{1}{c}{\rotatebox[origin=l]{90}{Insider}} &
  \multicolumn{1}{c}{\rotatebox[origin=l]{90}{Transaction Alteration}} &
  \multicolumn{1}{c}{\rotatebox[origin=l]{90}{Credential Compromise}} &
  \multicolumn{1}{c}{\rotatebox[origin=l]{90}{Network Disruption}} &
  % \multicolumn{1}{c}{\rotatebox[origin=l]{90}{Software}} &
  \multicolumn{1}{c}{\rotatebox[origin=l]{90}{Desktop Wallet}} &
  \multicolumn{1}{c}{\rotatebox[origin=l]{90}{Browser Wallet}} &
  \multicolumn{1}{c}{\rotatebox[origin=l]{90}{Mobile Wallet}} &
  \multicolumn{1}{c}{\rotatebox[origin=l]{90}{Smart Wallet}} &
  \multicolumn{1}{c}{\rotatebox[origin=l]{90}{Hardware Wallet}} &
  \multicolumn{1}{c}{\rotatebox[origin=l]{90}{Academic Papers No. (\%)}} &
  \multicolumn{1}{c}{\rotatebox[origin=l]{90}{Notable Incidents No. (\%)}} &
  
  % \multicolumn{1}{c}{\rotatebox[origin=l]{90}{Custom Keyboard Functions \cite{aldawood2020advanced}}} &
  % \multicolumn{1}{c}{\rotatebox[origin=l]{90}{Access Control Restrictions \cite{li2020android}}} &
  % \multicolumn{1}{c}{\rotatebox[origin=l]{90}{Enhanced Network Authentication \cite{Cai2014ResearchNetwork}}} &
  % \multicolumn{1}{c}{\rotatebox[origin=l]{90}{Multi-factor Authentication \cite{Aratani2015AuthenticationChannel}}} &
  % \multicolumn{1}{c}{\rotatebox[origin=l]{90}{Advanced Passwords \cite{aldawood2020advanced}}} &
  % \multicolumn{1}{c}{\rotatebox[origin=l]{90}{Liveness Assessment \cite{galbally2013image}}} &
  % \multicolumn{1}{c}{\rotatebox[origin=l]{90}{\acf{mpc} \cite{Lindell2020SecureComputation}}} & 
  % \multicolumn{1}{c}{\rotatebox[origin=l]{90}{Multi-sig. Implementation \cite{bip11}}} &
  % \multicolumn{1}{c}{\rotatebox[origin=l]{90}{Traffic Mitigation \cite{liu2018deep}}} &
  % \multicolumn{1}{c}{\rotatebox[origin=l]{90}{Reset TCP Conn. \cite{sathwara2017distributed}}} &
  % \multicolumn{1}{c}{\rotatebox[origin=l]{90}{Intrusion Detection \cite{zimba2019cryptojacking}}} &
  % \multicolumn{1}{c}{\rotatebox[origin=l]{90}{IP Verification \& Monitoring \cite{Bhirud2011LightPrevention}}} &
  % \multicolumn{1}{c}{\rotatebox[origin=l]{90}{Anti-Malware \cite{ferdous2023review}}} &
  %  \multicolumn{1}{c}{\rotatebox[origin=l]{90}{WebApp Firewalls \cite{ahmed2017mitigating}}} &
  %  \multicolumn{1}{c}{\rotatebox[origin=l]{90}{Alt. Prevention Features \cite{li2020android}}} &
  %  \multicolumn{1}{c}{\rotatebox[origin=l]{90}{Code Obfuscation \cite{indusface}}} &
  %  \multicolumn{1}{c}{\rotatebox[origin=l]{90}{Cryptographic Verification \cite{Tirronen2018StoppingData}}} &
  %  \multicolumn{1}{c}{\rotatebox[origin=l]{90}{Runtime Protection \cite{qi2012spad}}} &
  %  \multicolumn{1}{c}{\rotatebox[origin=l]{90}{Algorithmmic Fault Detection \cite{breier2022practical}}} &
  %  \multicolumn{1}{c}{\rotatebox[origin=l]{90}{\acl{puf}\cite{hu2020overview, Urien2021InnovativeWallets}}} & 
  %  \multicolumn{1}{c}{\rotatebox[origin=l]{90}{Deterministic Nonce Selection \cite{brengel2018identifying}}} &
  %  \multicolumn{1}{c}{\rotatebox[origin=l]{90}{Algorithmic Memory Erase \cite{seol2019amnesiac}}} &
  %  \multicolumn{1}{c}{\rotatebox[origin=l]{90}{Memory \& Cache Data Split \cite{Gupta2019ImpactSecurity}}} &
  %  \multicolumn{1}{c}{\rotatebox[origin=l]{90}{Supplementary Storage \cite{altuwaijri2020android}}} &
  %  \multicolumn{1}{c}{\rotatebox[origin=l]{90}{Secure Cryptographic Schemes \cite{brengel2018identifying}}} &
  %  \multicolumn{1}{c}{\rotatebox[origin=l]{90}{Correlation Elimination Sounds \cite{Park2023, Park2024CloningFunction}}} &
  %  &
   \\
   \addlinespace[2ex] % Add space before the second top rule
   \toprule
\multirow{6}{*}{Network} &
  \multirow{4}{*}{\hyperref[sec:dos]{Connection Hijack}} &
  Rogue AP \cite{Hu2021SecurityCountermeasures}  

  % {\smallemptycirc} &
  % {\smallfullcirc} &
  &
  {\smallemptycirc} &
  % -- rng
  {\smallemptycirc} &
  % -- inadequ auth
  {\smallemptycirc} &
  % -- inadequ encry
  {\smallemptycirc} &
  % -- Appl. Logic Flaw
  {\smallemptycirc} &
  % -- Low-strength pwds
  {\smallemptycirc} &
  % -- Data Leakage
  {\smallemptycirc} &
  % -- Data Remanence
  {\smallemptycirc} &
  % -- Data Remanence
  {\smallemptycirc} &
  % -- Insec. Boot Environ.
  {\smallemptycirc} &
  % -- Micro-electr. Exposure
  {\smallemptycirc} &
  % -- Weak Signature
  {\smallemptycirc} &
  % -- Inadeq. Sig. Verif. 
  % -- SYSTEM
  {\smallemptycirc} &
  % -- Insec. Permissions
  {\smallemptycirc} &
  % -- Library Vulnerability
  {\smallemptycirc} &
  % -- OS Vulnerabilities
  {\smallemptycirc} &
  % -- Coding Errors
  {\smallfullcirc} &
  % -- Insec. Network
  %  -- SYSTEM
  {\smallfullcirc} &
  % -- Insec. User Interactions
  {\smallfullcirc} &
  % -- Comp. Provider
  % -- EXT
  {\smallemptycirc} &
  % -- Malicious Insider
  {\smallemptycirc} &
  % -- Insider Compromise
  % -- INSIDER
  {\smallemptycirc} &
  {\smallemptycirc} &
  {\smallemptycirc} &
  {\smallemptycirc} &
  {\smallemptycirc} &
  {\smallfullcirc} &
  {\smallemptycirc} &
  {\smallemptycirc} &
  {\smallemptycirc} &
  {\smallemptycirc} &
  {\smallfullcirc} &
  {\smallemptycirc} &
  % data --
  {\smallfullcirc} &
  {\smallfullcirc} &
  {\smallemptycirc} &
  {\smallfullcirc} &
  {\smallfullcirc} &
  {\smallemptycirc} &
  % mech --
  {\smallemptycirc} &
  % {\smallfullcirc} &
  % {\smallfullcirc} &
  {\smallfullcirc} &
  {\smallfullcirc} &
  {\smallfullcirc} &
  {\smallfullcirc} &
  {\smallfullcirc} &
\cellcolor{g2}{$1$} &
\cellcolor{g0}{$0$} &
  % other --
  % -- &
  % -- &
  % -- &
  % {\smallemptycirc} &
  % {\smallemptycirc} &
  % {\smallfullcirc} &
  % % goal --
  % {\smallemptycirc} &
  % {\smallemptycirc} &
  % % tax --
  % {\smallemptycirc} &
  % {\smallemptycirc} &
  % {\smallemptycirc} &
  % {\smallemptycirc} &
  % {\smallemptycirc} &
  % {\smallfullcirc} &
  % {\smallemptycirc} &
  % {\smallemptycirc} &
  % % auth --
  % {\smallemptycirc} &
  % {\smallemptycirc} &
  % % disr --
  % {\smallemptycirc} &
  % {\smallemptycirc} &
  % {\smallemptycirc} &
  % {\smallemptycirc} &
  % {\smallemptycirc} &
  % % intru --
  % {\smallemptycirc} &
  % {\smallemptycirc} &
  % {\smallemptycirc} &
  % {\smallemptycirc} &
  % {\smallemptycirc} &
  % {\smallemptycirc} 
  \cite{Cai2014ResearchNetwork, zimba2019cryptojacking} 
  \\
 &
   &
  DNS Spoofing \cite{pillai2019smart, Al-Mashhadi2020ASystems} &
  {\smallemptycirc} &
  % -- rng
  {\smallemptycirc} &
  % -- inadequ auth
  {\smallemptycirc} &
  % -- inadequ encry
  {\smallemptycirc} &
  % -- Appl. Logic Flaw
  {\smallemptycirc} &
  % -- Low-strength pwds
  {\smallemptycirc} &
  % -- Data Leakage
  {\smallemptycirc} &
  % -- Data Remanence
  {\smallemptycirc} &
  % -- Data Remanence
  {\smallemptycirc} &
  % -- Insec. Boot Environ.
  {\smallemptycirc} &
  % -- Micro-electr. Exposure
  {\smallemptycirc} &
  % -- Weak Signature
  {\smallemptycirc} &
  % -- Inadeq. Sig. Verif. 
  % -- SYSTEM
  {\smallemptycirc} &
  % -- Insec. Permissions
  {\smallemptycirc} &
  % -- Library Vulnerability
  {\smallemptycirc} &
  % -- OS Vulnerabilities
  {\smallemptycirc} &
  % -- Coding Errors
  {\smallfullcirc} &
  % -- Insec. Network
  %  -- SYSTEM
  {\smallfullcirc} &
  % -- Insec. User Interactions
  {\smallfullcirc} &
  % -- Comp. Provider
  % -- EXT
  {\smallemptycirc} &
  % -- Malicious Insider
  {\smallemptycirc} &
  % -- Insider Compromise
  % -- INSIDER
  {\smallemptycirc} &
  {\smallemptycirc} &
  {\smallemptycirc} &
  {\smallemptycirc} &
  {\smallemptycirc} &
  {\smallfullcirc} &
  {\smallemptycirc} &
  % data --
  {\smallemptycirc} &
  {\smallemptycirc} &
  {\smallemptycirc} &
  {\smallfullcirc} &
  {\smallemptycirc} &
  {\smallfullcirc} &
  % data --
  {\smallfullcirc} &
  % {\smallfullcirc} &
  {\smallemptycirc} &
  {\smallfullcirc} &
  % data --
  % -- &
  % -- &
  % -- &
  {\smallfullcirc} &
  {\smallemptycirc} &
  {\smallemptycirc} &
  % data --
  % {\smallfullcirc} &
  {\smallfullcirc} &
  {\smallfullcirc} &
  {\smallfullcirc} &
  {\smallfullcirc} &
  {\smallfullcirc} &
  \cellcolor{g4}{$2$} &
\cellcolor{r3}{$3$} &

% 83 total attack incidents
  
  % data --
  
  % {\smallemptycirc} &
  % {\smallemptycirc} &
  % {\smallfullcirc} &
  % {\smallemptycirc} &
  % {\smallemptycirc} &
  % {\smallemptycirc} &
  % {\smallemptycirc} &
  % {\smallemptycirc} &
  % % data --
  % {\smallemptycirc} &
  % {\smallemptycirc} &
  % % data --
  % {\smallfullcirc} &
  % {\smallemptycirc} &
  % {\smallemptycirc} &
  % {\smallfullcirc} &
  % % data --
  % {\smallemptycirc} &
  % {\smallemptycirc} &
  % {\smallemptycirc} &
  % {\smallemptycirc} &
  % {\smallemptycirc} &
  % {\smallemptycirc} &
  % {\smallemptycirc} &
  % {\smallemptycirc} &
  % {\smallemptycirc} &
  % {\smallemptycirc} &
  % {\smallemptycirc} &
  % {\smallemptycirc} &
  \cite{Ahmed2017MitigatingNetworking, Cai2014ResearchNetwork, zimba2019cryptojacking} 
   \\
 & 
   &
  IP Spoofing \cite{shrivas2020disruptive} &
  {\smallemptycirc} &
  % -- rng
  {\smallemptycirc} &
  % -- inadequ auth
  {\smallemptycirc} &
  % -- inadequ encry
  {\smallemptycirc} &
  % -- Appl. Logic Flaw
  {\smallemptycirc} &
  % -- Low-strength pwds
  {\smallemptycirc} &
  % -- Data Leakage
  {\smallemptycirc} &
  % -- Data Remanence
  {\smallemptycirc} &
  % -- Data Remanence
  {\smallemptycirc} &
  % -- Insec. Boot Environ.
  {\smallemptycirc} &
  % -- Micro-electr. Exposure
  {\smallemptycirc} &
  % -- Weak Signature
  {\smallemptycirc} &
  % -- Inadeq. Sig. Verif. 
  % -- SYSTEM
  {\smallemptycirc} &
  % -- Insec. Permissions
  {\smallemptycirc} &
  % -- Library Vulnerability
  {\smallemptycirc} &
  % -- OS Vulnerabilities
  {\smallemptycirc} &
  % -- Coding Errors
  {\smallfullcirc} &
  % -- Insec. Network
  %  -- SYSTEM
  {\smallfullcirc} &
  % -- Insec. User Interactions
  {\smallfullcirc} &
  % -- Comp. Provider
  % -- EXT
  {\smallemptycirc} &
  % -- Malicious Insider
  {\smallemptycirc} &
  % -- Insider Compromise
  % -- INSIDER
  {\smallemptycirc} &
  {\smallemptycirc} &
  {\smallemptycirc} &
  {\smallemptycirc} &
  {\smallemptycirc} &
  {\smallfullcirc} &
  {\smallemptycirc} &
  % data --
  {\smallemptycirc} &
  {\smallemptycirc} &
  {\smallemptycirc} &
  {\smallfullcirc} &
  {\smallemptycirc} &
  {\smallfullcirc} & 
  {\smallfullcirc} &
  % {\smallemptycirc} &
  {\smallemptycirc} &
  {\smallfullcirc} &
  % -- &
  % -- &
  % -- &
  {\smallfullcirc} &
  {\smallemptycirc} &
  {\smallemptycirc} &
  % {\smallfullcirc} &
  {\smallfullcirc} &
  {\smallfullcirc} &
  {\smallfullcirc} &
  {\smallfullcirc} &
  {\smallfullcirc} &
  \cellcolor{g2}{$1$} &
\cellcolor{g0}{$0$} &

  
  % {\smallemptycirc} &
  % {\smallemptycirc} &
  % {\smallfullcirc} &
  % {\smallemptycirc} &
  % {\smallemptycirc} &
  % {\smallemptycirc} &
  % {\smallemptycirc} &
  % {\smallemptycirc} &
  % {\smallemptycirc} &
  % {\smallemptycirc} &
  % {\smallfullcirc} &
  % {\smallfullcirc} &
  % {\smallemptycirc} &
  % {\smallemptycirc} &
  % {\smallemptycirc} &
  % {\smallemptycirc} &
  % {\smallemptycirc} &
  % {\smallemptycirc} &
  % {\smallemptycirc} &
  % {\smallemptycirc} &
  % {\smallemptycirc} &
  % {\smallemptycirc} &
  % {\smallemptycirc} &
  % {\smallemptycirc} &
  % {\smallemptycirc} &
  % {\smallemptycirc} &
  \cite{Bhirud2011LightPrevention, Cai2014ResearchNetwork, zimba2019cryptojacking} 
   \\
%    & 
%    &
%   \acs{icmp} Redirection \cite{Feng2023Man-in-the-middleRedirects} &
%   {\smallemptycirc} &
%   % -- rng
%   {\smallemptycirc} &
%   % -- inadequ auth
%   {\smallemptycirc} &
%   % -- inadequ encry
%   {\smallemptycirc} &
%   % -- Appl. Logic Flaw
%   {\smallemptycirc} &
%   % -- Low-strength pwds
%   {\smallemptycirc} &
%   % -- Data Leakage
%   {\smallemptycirc} &
%   % -- Data Remanence
%   {\smallemptycirc} &
%   % -- Data Remanence
%   {\smallemptycirc} &
%   % -- Insec. Boot Environ.
%   {\smallemptycirc} &
%   % -- Micro-electr. Exposure
%   {\smallemptycirc} &
%   % -- Weak Signature
%   {\smallemptycirc} &
%   % -- Inadeq. Sig. Verif. 
%   % -- SYSTEM
%   {\smallemptycirc} &
%   % -- Insec. Permissions
%   {\smallemptycirc} &
%   % -- Library Vulnerability
%   {\smallemptycirc} &
%   % -- OS Vulnerabilities
%   {\smallemptycirc} &
%   % -- Coding Errors
%   {\smallfullcirc} &
%   % -- Insec. Network
%   %  -- SYSTEM
%   {\smallemptycirc} &
%   % -- Insec. User Interactions
%   {\smallemptycirc} &
%   % -- Comp. Provider
%   % -- EXT
%   {\smallemptycirc} &
%   % -- Malicious Insider
%   {\smallemptycirc} &
%   % -- Insider Compromise
%   % -- INSIDER
%   {\smallemptycirc} &
%   {\smallemptycirc} &
%   {\smallemptycirc} &
%   {\smallemptycirc} &
%   {\smallemptycirc} &
%   {\smallfullcirc} &
%   {\smallemptycirc} &
%   % data --
%   {\smallemptycirc} &
%   {\smallemptycirc} &
%   {\smallemptycirc} &
%   {\smallfullcirc} &
%   {\smallemptycirc} &
%   {\smallemptycirc} & 
%   {\smallfullcirc} &
%   % {\smallemptycirc} &
%   {\smallemptycirc} &
%   {\smallfullcirc} &
%   % -- &
%   % -- &
%   % -- &
%   {\smallfullcirc} &
%   {\smallemptycirc} &
%   {\smallemptycirc} &
%   % {\smallfullcirc} &
%   {\smallfullcirc} &
%   {\smallfullcirc} &
%   {\smallfullcirc} &
%   {\smallfullcirc} &
%   {\smallfullcirc} &
%   \cellcolor{g2}{$1$($3\%$)} &
% \cellcolor{g0}{$0$($0\%$)} &
%   \cite{Feng2023Man-in-the-middleRedirects} 
%    \\
&
&
\acs{bgp} Hijacking \cite{ekparinya2018impact} &
{\smallemptycirc} & % RNG
{\smallemptycirc} & % Inadequate Authentication
{\smallemptycirc} & % Inadequate Encryption
{\smallemptycirc} & % Application Logic Flaw
{\smallemptycirc} & % Low-strength Passwords
{\smallemptycirc} & % Data Leakage
{\smallemptycirc} & % Data Remanence
{\smallemptycirc} & % Data Manipulation
{\smallemptycirc} & % Insecure Boot Environment
{\smallemptycirc} & % Microelectronic Component Exposure
{\smallemptycirc} & % Weak Signature
{\smallemptycirc} & % Inadequate Signature Verification
{\smallemptycirc} & % Insecure Permissions
{\smallemptycirc} & % Library Vulnerability
{\smallemptycirc} & % OS Vulnerabilities
{\smallemptycirc} & % Coding Errors
{\smallfullcirc} & % Insecure Network
{\smallfullcirc} & % Insecure User Interactions
{\smallfullcirc} & % Compromised Provider
{\smallemptycirc} & % Malicious Insider
{\smallemptycirc} & % Compromised Insider
{\smallemptycirc} & % Private Key
{\smallemptycirc} & % Signature
{\smallemptycirc} & % Mnemonics
{\smallemptycirc} & % KEK or Password (pw)
{\smallemptycirc} & % Memory
{\smallfullcirc} & % State Transition Info
{\smallemptycirc} & % Nonce
{\smallemptycirc} & % KeyGen
{\smallemptycirc} & % UserAuth
{\smallemptycirc} & % KeyStore
{\smallfullcirc} & % CreateTxn
{\smallemptycirc} & % TnxSign
{\smallfullcirc} & % TnxVer
{\smallfullcirc} & % Service Provider
{\smallemptycirc} & % Operating System
{\smallfullcirc} & % Wallet User
{\smallfullcirc} & % Transaction Alteration
{\smallemptycirc} & % Credential Compromise
{\smallemptycirc} & % Network Disruption
{\smallfullcirc} & % Desktop Wallet
{\smallfullcirc} & % Browser Wallet
{\smallfullcirc} & % Mobile Wallet
{\smallfullcirc} & % Smart Wallet
{\smallfullcirc} & % Hardware Wallet
\cellcolor{g2}{$1$} & % Academic Papers
\cellcolor{r2}{$1$} & % Notable Incidents
\cite{ekparinya2018impact} % Possible Defence
\\
 &
  \multirow{2}{*}{\hyperref[sec:mitm]{Service Denial}} &
  \acs{icmp} Flooding \cite{chaganti2022comprehensive, chaganti2022role} &
  {\smallemptycirc} &
  % -- rng
  {\smallemptycirc} &
  % -- inadequ auth
  {\smallemptycirc} &
  % -- inadequ encry
  {\smallemptycirc} &
  % -- Appl. Logic Flaw
  {\smallemptycirc} &
  % -- Low-strength pwds
  {\smallemptycirc} &
  % -- Data Leakage
  {\smallemptycirc} &
  % -- Data Remanence
  {\smallemptycirc} &
  % -- Data Remanence
  {\smallemptycirc} &
  % -- Insec. Boot Environ.
  {\smallemptycirc} &
  % -- Micro-electr. Exposure
  {\smallemptycirc} &
  % -- Weak Signature
  {\smallemptycirc} &
  % -- Inadeq. Sig. Verif. 
  % -- SYSTEM
  {\smallemptycirc} &
  % -- Insec. Permissions
  {\smallemptycirc} &
  % -- Library Vulnerability
  {\smallemptycirc} &
  % -- OS Vulnerabilities
  {\smallemptycirc} &
  % -- Coding Errors
  {\smallfullcirc} &
  % -- Insec. Network
  %  -- SYSTEM
  {\smallemptycirc} &
  % -- Insec. User Interactions
  {\smallemptycirc} &
  % -- Comp. Provider
  % -- EXT
  {\smallemptycirc} &
  % -- Malicious Insider
  {\smallemptycirc} &
  % -- Insider Compromise
  % -- INSIDER
  {\smallemptycirc} &
  {\smallemptycirc} &
  {\smallemptycirc} &
  {\smallemptycirc} &
  {\smallemptycirc} &
  {\smallfullcirc} &
  {\smallemptycirc} &
  {\smallemptycirc} &
  {\smallemptycirc} &
  % changed back to empty circle as user authentication does not rely on internet in most cases
  %  {\smallhalfcirc} &

  {\smallemptycirc} &
  {\smallemptycirc} &
  {\smallemptycirc} &
  {\smallhalfcirc} & 
  {\smallfullcirc} &
  % {\smallhalfcirc} &
  {\smallemptycirc} &
  {\smallemptycirc} &
  % -- &
  % -- &
  % -- &
  {\smallemptycirc} &
  {\smallemptycirc} &
  {\smallfullcirc} &
  % {\smallfullcirc} &
  {\smallfullcirc} &
  {\smallfullcirc} &
  {\smallfullcirc} &
  {\smallfullcirc} &
  {\smallfullcirc} &
 \cellcolor{g4}{$2$} &
\cellcolor{g0}{$0$} &

  
  % {\smallemptycirc} &
  % {\smallemptycirc} &
  % {\smallfullcirc} &
  % {\smallemptycirc} &
  % {\smallemptycirc} &
  % {\smallemptycirc} &
  % {\smallemptycirc} &
  % {\smallemptycirc} &
  % {\smallfullcirc} &
  % {\smallemptycirc} &
  % {\smallfullcirc} &
  % {\smallfullcirc} &
  % {\smallemptycirc} &
  % {\smallemptycirc} &
  % {\smallemptycirc} &
  % {\smallemptycirc} &
  % {\smallemptycirc} &
  % {\smallemptycirc} &
  % {\smallemptycirc} &
  % {\smallemptycirc} &
  % {\smallemptycirc} &
  % {\smallemptycirc} &
  % {\smallemptycirc} &
  % {\smallemptycirc} &
  % {\smallemptycirc} &
  % {\smallemptycirc} &
  \cite{liu2018deep, Bhirud2011LightPrevention} 
   \\
 &
   &
  TCP SYN Flooding \cite{chaganti2022comprehensive} &
  {\smallemptycirc} &
  % -- rng
  {\smallemptycirc} &
  % -- inadequ auth
  {\smallemptycirc} &
  % -- inadequ encry
  {\smallemptycirc} &
  % -- Appl. Logic Flaw
  {\smallemptycirc} &
  % -- Low-strength pwds
  {\smallemptycirc} &
  % -- Data Leakage
  {\smallemptycirc} &
  % -- Data Remanence
  {\smallemptycirc} &
  % -- Data Remanence
  {\smallemptycirc} &
  % -- Insec. Boot Environ.
  {\smallemptycirc} &
  % -- Micro-electr. Exposure
  {\smallemptycirc} &
  % -- Weak Signature
  {\smallemptycirc} &
  % -- Inadeq. Sig. Verif. 
  % -- SYSTEM
  {\smallemptycirc} &
  % -- Insec. Permissions
  {\smallemptycirc} &
  % -- Library Vulnerability
  {\smallemptycirc} &
  % -- OS Vulnerabilities
  {\smallemptycirc} &
  % -- Coding Errors
  {\smallfullcirc} &
  % -- Insec. Network
  %  -- SYSTEM
  {\smallemptycirc} &
  % -- Insec. User Interactions
  {\smallemptycirc} &
  % -- Comp. Provider
  % -- EXT
  {\smallemptycirc} &
  % -- Malicious Insider
  {\smallemptycirc} &
  % -- Insider Compromise
  % -- INSIDER
 {\smallemptycirc} &
  {\smallemptycirc} &
  {\smallemptycirc} &
  {\smallemptycirc} &
  {\smallemptycirc} &
 {\smallfullcirc} &
  {\smallemptycirc} &
  {\smallemptycirc} &
  {\smallemptycirc} &
  {\smallemptycirc} &
  {\smallemptycirc} &
  {\smallemptycirc} &
  {\smallhalfcirc} & 
  {\smallfullcirc} &
  % {\smallhalfcirc} &
  {\smallemptycirc} &
  {\smallemptycirc} &
  % -- &
  % -- &
  % -- &
  {\smallemptycirc} &
  {\smallemptycirc} &
  {\smallfullcirc} &
  % {\smallfullcirc} &
  {\smallfullcirc} &
  {\smallfullcirc} &
  {\smallfullcirc} &
  {\smallfullcirc} &
  {\smallfullcirc} &
  \cellcolor{g2}{$1$} &
\cellcolor{g0}{$0$} &
  
  % {\smallemptycirc} &
  % {\smallemptycirc} &
  % {\smallfullcirc} &
  % {\smallemptycirc} &
  % {\smallemptycirc} &
  % {\smallemptycirc} &
  % {\smallemptycirc} &
  % {\smallemptycirc} &
  % {\smallfullcirc} &
  % {\smallfullcirc} &
  % {\smallfullcirc} &
  % {\smallfullcirc} &
  % {\smallemptycirc} &
  % {\smallemptycirc} &
  % {\smallemptycirc} &
  % {\smallemptycirc} &
  % {\smallemptycirc} &
  % {\smallemptycirc} &
  % {\smallemptycirc} &
  % {\smallemptycirc} &
  % {\smallemptycirc} &
  % {\smallemptycirc} &
  % {\smallemptycirc} &
  % {\smallemptycirc} &
  % {\smallemptycirc} &
  % {\smallemptycirc} &
  \cite{sathwara2017distributed, liu2018deep} 
  % Bhirud2011LightPrevention, Cai2014ResearchNetwork, zimba2019cryptojacking
   \\
  \midrule
\multirow{8}{*}{Application} &
  \multirow{2}{*}{\hyperref[sec:malware]{Malware Execution}} &
  Clipboard Hijack \cite{ivanov2021ethclipper, Kim2018RiskThreats, li2020android} &
  {\smallemptycirc} &
  % -- rng
  {\smallemptycirc} &
  % -- inadequ auth
  {\smallfullcirc} &
  % -- inadequ encry
  {\smallemptycirc} &
  % -- Appl. Logic Flaw
  {\smallemptycirc} &
  % -- Low-strength pwds
  {\smallemptycirc} &
  % -- Data Leakage
  {\smallemptycirc} &
  % -- Data Remanence
  {\smallfullcirc} &
  % -- Data Manipulation
  {\smallemptycirc} &
  % -- Insec. Boot Environ.
  {\smallemptycirc} &
  % -- Micro-electr. Exposure
  {\smallemptycirc} &
  % -- Weak Signature
  {\smallemptycirc} &
  % -- Inadeq. Sig. Verif. 
  % -- SYSTEM
  {\smallfullcirc} &
  % -- Insec. Permissions
  {\smallemptycirc} &
  % -- Library Vulnerability
  {\smallfullcirc} &
  % -- OS Vulnerabilities
  {\smallemptycirc} &
  % -- Coding Errors
  {\smallemptycirc} &
  % -- Insec. Network
  %  -- SYSTEM
  {\smallfullcirc} &
  % -- Insec. User Interactions
  {\smallemptycirc} &
  % -- Comp. Provider
  % -- EXT
  {\smallemptycirc} &
  % -- Malicious Insider
  {\smallemptycirc} &
  % -- Insider Compromise
  % -- INSIDER
  {\smallemptycirc} &
  {\smallemptycirc} &
  {\smallemptycirc} &
  {\smallemptycirc} &
  {\smallemptycirc} &
  {\smallfullcirc} &
  {\smallemptycirc} &
  {\smallemptycirc} &
  {\smallemptycirc} &
  {\smallemptycirc} &
  {\smallfullcirc} &
  {\smallemptycirc} &
  {\smallemptycirc} &
  {\smallemptycirc} &
  % {\smallemptycirc} &
  {\smallfullcirc} &
  {\smallfullcirc} &
  % -- &
  % -- &
  % -- &
  {\smallfullcirc} &
  {\smallemptycirc} &
  {\smallemptycirc} &
  % {\smallfullcirc} &
  {\smallfullcirc} &
  {\smallfullcirc} &
  {\smallfullcirc} &
  {\smallfullcirc} &
  {\smallfullcirc} &
  \cellcolor{g6}{$3$} &
\cellcolor{r4}\multirow{-1}{*}{$8$} &

  
  % {\smallemptycirc} &
  % {\smallemptycirc} &
  % {\smallemptycirc} &
  % {\smallemptycirc} &
  % {\smallemptycirc} &
  % {\smallemptycirc} &
  % {\smallemptycirc} &
  % {\smallemptycirc} &
  % {\smallemptycirc} &
  % {\smallemptycirc} &
  % {\smallemptycirc} &
  % {\smallemptycirc} &
  % {\smallfullcirc} &
  % {\smallemptycirc} &
  % {\smallfullcirc} &
  % {\smallemptycirc} &
  % {\smallemptycirc} &
  % {\smallemptycirc} &
  % {\smallemptycirc} &
  % {\smallemptycirc} &
  % {\smallemptycirc} &
  % {\smallemptycirc} &
  % {\smallemptycirc} &
  % {\smallemptycirc} &
  % {\smallemptycirc} &
  % {\smallemptycirc} &
  \cite{ferdous2023review, li2020android}
   \\
 &
   &
  Spyware \cite{weichbroth2023security, aldawood2020advanced} &
  {\smallemptycirc} &
  % -- rng
  {\smallemptycirc} &
  % -- inadequ auth
  {\smallfullcirc} &
  % -- inadequ encry
  {\smallemptycirc} &
  % -- Appl. Logic Flaw
  {\smallemptycirc} &
  % -- Low-strength pwds
  {\smallfullcirc} &
  % -- Data Leakage
  {\smallemptycirc} &
  % -- Data Remanence
  {\smallemptycirc} &
  % -- Data Remanence
  {\smallemptycirc} &
  % -- Insec. Boot Environ.
  {\smallemptycirc} &
  % -- Micro-electr. Exposure
  {\smallemptycirc} &
  % -- Weak Signature
  {\smallemptycirc} &
  % -- Inadeq. Sig. Verif. 
  % -- SYSTEM
  {\smallfullcirc} &
  % -- Insec. Permissions
  {\smallemptycirc} &
  % -- Library Vulnerability
  {\smallfullcirc} &
  % -- OS Vulnerabilities
  {\smallemptycirc} &
  % -- Coding Errors
  {\smallemptycirc} &
  % -- Insec. Network
  %  -- SYSTEM
  {\smallfullcirc} &
  % -- Insec. User Interactions
  {\smallemptycirc} &
  % -- Comp. Provider
  % -- EXT
  {\smallemptycirc} &
  % -- Malicious Insider
  {\smallemptycirc} &
  % -- Insider Compromise
  % -- INSIDER
  {\smallfullcirc} &
  {\smallemptycirc} &
  {\smallfullcirc} &
  {\smallfullcirc} &
  {\smallemptycirc} &
  {\smallemptycirc} &
  {\smallemptycirc} &
  {\smallfullcirc} &
  {\smallemptycirc} &
  {\smallfullcirc} &
  {\smallemptycirc} &
  {\smallemptycirc} &
  {\smallemptycirc} &
  {\smallemptycirc} &
  % {\smallemptycirc} &
  {\smallemptycirc} &
  {\smallfullcirc} &
  % -- &
  % -- &
  % -- &
  {\smallemptycirc} &
  {\smallfullcirc} &
  {\smallemptycirc} &
  % {\smallfullcirc} &
  {\smallfullcirc} &
  {\smallfullcirc} &
  {\smallfullcirc} &
  {\smallemptycirc} &
  {\smallemptycirc} &
  \cellcolor{g4}{$2$} &
  \cellcolor{r4}{} &
  % \multicolumn{1}{c}{} &
% \cellcolor{r4}{$8$($9\%$)} &
% \cellcolor{r4}{}

  
  % {\smallemptycirc} &
  % {\smallemptycirc} &
  % {\smallemptycirc} &
  % {\smallemptycirc} &
  % {\smallemptycirc} &
  % {\smallemptycirc} &
  % {\smallemptycirc} &
  % {\smallemptycirc} &
  % {\smallemptycirc} &
  % {\smallemptycirc} &
  % {\smallemptycirc} &
  % {\smallemptycirc} &
  % {\smallfullcirc} &
  % {\smallemptycirc} &
  % {\smallemptycirc} &
  % {\smallemptycirc} &
  % {\smallemptycirc} &
  % {\smallemptycirc} &
  % {\smallemptycirc} &
  % {\smallemptycirc} &
  % {\smallemptycirc} &
  % {\smallemptycirc} &
  % {\smallemptycirc} &
  % {\smallemptycirc} &
  % {\smallemptycirc} &
  % {\smallemptycirc} &
  % Anti-Malware Software \cite{ferdous2023review} 
  \cite{ferdous2023review}
   \\
 % &
 %   &
 %  Ransomware \cite{conti2018economic, robinson2022new} &
 %  {\smallemptycirc} &
 %  % -- rng
 %  {\smallemptycirc} &
 %  % -- inadequ auth
 %  {\smallfullcirc} &
 %  % -- inadequ encry
 %  {\smallemptycirc} &
 %  % -- Appl. Logic Flaw
 %  {\smallemptycirc} &
 %  % -- Low-strength pwds
 %  {\smallfullcirc} &
 %  % -- Data Leakage
 %  {\smallemptycirc} &
 %  % -- Data Remanence
 %  {\smallfullcirc} &
 %  % -- Data Remanence
 %  {\smallemptycirc} &
 %  % -- Insec. Boot Environ.
 %  {\smallemptycirc} &
 %  % -- Micro-electr. Exposure
 %  {\smallemptycirc} &
 %  % -- Weak Signature
 %  {\smallemptycirc} &
 %  % -- Inadeq. Sig. Verif. 
 %  % -- SYSTEM
 %  {\smallfullcirc} &
 %  % -- Insec. Permissions
 %  {\smallemptycirc} &
 %  % -- Library Vulnerability
 %  {\smallfullcirc} &
 %  % -- OS Vulnerabilities
 %  {\smallemptycirc} &
 %  % -- Coding Errors
 %  {\smallemptycirc} &
 %  % -- Insec. Network
 %  %  -- SYSTEM
 %  {\smallfullcirc} &
 %  % -- Insec. User Interactions
 %  {\smallemptycirc} &
 %  % -- Comp. Provider
 %  % -- EXT
 %  {\smallemptycirc} &
 %  % -- Malicious Insider
 %  {\smallemptycirc} &
 %  % -- Insider Compromise
 %  % -- INSIDER
 %  {\smallemptycirc} &
 %  {\smallemptycirc} &
 %  {\smallemptycirc} &
 %  {\smallemptycirc} &
 %  {\smallemptycirc} &
 %  {\smallemptycirc} &
 %  {\smallemptycirc} &
 %  {\smallemptycirc} &
 %  {\smallemptycirc} &
 %  {\smallemptycirc} &
 %  {\smallemptycirc} &
 %  {\smallemptycirc} &
 %  {\smallemptycirc} &
 %  {\smallemptycirc} &
 %  % {\smallemptycirc} &
 %  {\smallemptycirc} &
 %  {\smallfullcirc} &
 %  % -- &
 %  % -- &
 %  % -- &
 %  {\smallemptycirc} &
 %  {\smallfullcirc} &
 %  {\smallemptycirc} &
 %  % {\smallfullcirc} &
 %  {\smallemptycirc} &
 %  {\smallfullcirc} &
 %  {\smallemptycirc} &
 %  {\smallemptycirc} &
 %  {\smallfullcirc} &

  
 %  % {\smallemptycirc} &
 %  % {\smallemptycirc} &
 %  % {\smallemptycirc} &
 %  % {\smallemptycirc} &
 %  % {\smallemptycirc} &
 %  % {\smallemptycirc} &
 %  % {\smallemptycirc} &
 %  % {\smallemptycirc} &
 %  % {\smallemptycirc} &
 %  % {\smallemptycirc} &
 %  % {\smallemptycirc} &
 %  % {\smallemptycirc} &
 %  % {\smallfullcirc} &
 %  % {\smallemptycirc} &
 %  % {\smallemptycirc} &
 %  % {\smallemptycirc} &
 %  % {\smallemptycirc} &
 %  % {\smallemptycirc} &
 %  % {\smallemptycirc} &
 %  % {\smallemptycirc} &
 %  % {\smallemptycirc} &
 %  % {\smallemptycirc} &
 %  % {\smallemptycirc} &
 %  % {\smallemptycirc} &
 %  % {\smallemptycirc} &
 %  % {\smallemptycirc} &
 %  % Anti-Malware Software \cite{ferdous2023review} 
 %  \cite{ferdous2023review}
 %   \\
 &
  \multirow{2}{*}{\hyperref[sec:logic_expl]{Logic Exploitation}} &
  Constructor Hijack \cite{palladino2017parity} &
 {\smallemptycirc}&{\smallemptycirc}&{\smallemptycirc}&{\smallfullcirc}&{\smallemptycirc}&
 {\smallemptycirc}&{\smallemptycirc}&{\smallemptycirc}&{\smallemptycirc}&{\smallemptycirc}&
 {\smallemptycirc}&{\smallemptycirc}&{\smallfullcirc}&{\smallfullcirc}&{\smallemptycirc}&{\smallfullcirc}&
 {\smallemptycirc}&{\smallemptycirc}&{\smallemptycirc}&{\smallemptycirc}&{\smallemptycirc}&
 {\smallemptycirc}&{\smallemptycirc}&{\smallemptycirc}&{\smallemptycirc}&{\smallemptycirc}&
 {\smallemptycirc}&{\smallemptycirc}&{\smallemptycirc}&{\smallfullcirc}&{\smallemptycirc}&
 {\smallfullcirc}&{\smallemptycirc}&{\smallemptycirc}&{\smallemptycirc}&{\smallemptycirc}&
 {\smallemptycirc}&{\smallemptycirc}&{\smallemptycirc}&{\smallemptycirc}&{\smallemptycirc}&{\smallemptycirc}&{\smallemptycirc}&{\smallfullcirc}&{\smallemptycirc}&
 \cellcolor{g0}{$0$}&\cellcolor{r2}{$1$}&\cite{palladino2017parity}
   \\
 &
   &
  Upgrade-Path Hijack \cite{bybit_certik}  &
 {\smallemptycirc}&{\smallfullcirc}&{\smallemptycirc}&{\smallfullcirc}&{\smallemptycirc}&
 {\smallemptycirc}&{\smallemptycirc}&{\smallemptycirc}&{\smallemptycirc}&{\smallemptycirc}&
 {\smallemptycirc}&{\smallemptycirc}&{\smallemptycirc}&{\smallfullcirc}&{\smallemptycirc}&{\smallemptycirc}&
 {\smallemptycirc}&{\smallfullcirc}&{\smallfullcirc}&{\smallemptycirc}&{\smallemptycirc}&
 {\smallemptycirc}&{\smallemptycirc}&{\smallemptycirc}&{\smallemptycirc}&{\smallemptycirc}&
 {\smallfullcirc}&{\smallemptycirc}&{\smallfullcirc}&{\smallemptycirc}&{\smallemptycirc}&
 {\smallfullcirc}&{\smallemptycirc}&{\smallemptycirc}&{\smallemptycirc}&{\smallemptycirc}&
 {\smallfullcirc}&{\smallemptycirc}&{\smallfullcirc}&{\smallemptycirc}&{\smallemptycirc}&{\smallemptycirc}&{\smallemptycirc}&{\smallfullcirc}&{\smallfullcirc}&
 \cellcolor{g0}{$0$}&\cellcolor{r3}{$2$}&\cite{bybit_certik}
   \\ 
    &
  \multirow{2}{*}{\hyperref[sec:privilege]{Privilege Escalation}} &
  Android Root Privilege \cite{he2020security} &
  {\smallemptycirc} &
  % -- rng
  {\smallemptycirc} &
  % -- inadequ auth
  {\smallemptycirc} &
  % -- inadequ encry
  {\smallemptycirc} &
  % -- Appl. Logic Flaw
  {\smallemptycirc} &
  % -- Low-strength pwds
  {\smallemptycirc} &
  % -- Data Leakage
  {\smallemptycirc} &
  % -- Data Remanence
  {\smallemptycirc} &
  % -- Data Remanence
  {\smallemptycirc} &
  % -- Insec. Boot Environ.
  {\smallemptycirc} &
  % -- Micro-electr. Exposure
  {\smallemptycirc} &
  % -- Weak Signature
  {\smallemptycirc} &
  % -- Inadeq. Sig. Verif. 
  % -- SYSTEM
  {\smallfullcirc} &
  % -- Insec. Permissions
  {\smallemptycirc} &
  % -- Library Vulnerability
  {\smallfullcirc} &
  % -- OS Vulnerabilities
  {\smallemptycirc} &
  % -- Coding Errors
  {\smallemptycirc} &
  % -- Insec. Network
  %  -- SYSTEM
  {\smallemptycirc} &
  % -- Insec. User Interactions
  {\smallemptycirc} &
  % -- Comp. Provider
  % -- EXT
  {\smallemptycirc} &
  % -- Malicious Insider
  {\smallemptycirc} &
  % -- Insider Compromise
  % -- INSIDER
  {\smallfullcirc} &
  {\smallemptycirc} &
  {\smallfullcirc} &
  {\smallemptycirc} &
  {\smallemptycirc} &
  {\smallemptycirc} &
  {\smallemptycirc} &
  {\smallfullcirc} &
  {\smallemptycirc} &
  {\smallfullcirc} &
  {\smallemptycirc} &
  {\smallemptycirc} &
  {\smallemptycirc} &
  {\smallemptycirc} &
  % {\smallemptycirc} &
  {\smallfullcirc} &
  {\smallemptycirc} &
  % -- &
  % -- &
  % -- &
  {\smallemptycirc} &
  {\smallfullcirc} &
  {\smallemptycirc} &
  % {\smallfullcirc} &
  {\smallemptycirc} &
  {\smallemptycirc} &
  {\smallfullcirc} &
  {\smallemptycirc} &
  {\smallemptycirc} &
  \cellcolor{g2}{$1$} &
\cellcolor{r0}{$0$} &

  
  % {\smallemptycirc} &
  % {\smallemptycirc} &
  % {\smallemptycirc} &
  % {\smallemptycirc} &
  % {\smallemptycirc} &
  % {\smallemptycirc} &
  % {\smallemptycirc} &
  % {\smallemptycirc} &
  % {\smallemptycirc} &
  % {\smallemptycirc} &
  % {\smallemptycirc} &
  % {\smallemptycirc} &
  % {\smallemptycirc} &
  % {\smallemptycirc} &
  % {\smallemptycirc} &
  % {\smallfullcirc} &
  % {\smallemptycirc} &
  % {\smallemptycirc} &
  % {\smallemptycirc} &
  % {\smallemptycirc} &
  % {\smallemptycirc} &
  % {\smallemptycirc} &
  % {\smallemptycirc} &
  % {\smallemptycirc} &
  % {\smallemptycirc} &
  % {\smallemptycirc} &
  % Code Obfuscation 
  \cite{indusface} 
   \\
 &
   &
  Android USB Debugging \cite{he2020security} &
  {\smallemptycirc} &
  % -- rng
  {\smallemptycirc} &
  % -- inadequ auth
  {\smallemptycirc} &
  % -- inadequ encry
  {\smallemptycirc} &
  % -- Appl. Logic Flaw
  {\smallemptycirc} &
  % -- Low-strength pwds
  {\smallemptycirc} &
  % -- Data Leakage
  {\smallemptycirc} &
  % -- Data Remanence
  {\smallemptycirc} &
  % -- Data Remanence
  {\smallemptycirc} &
  % -- Insec. Boot Environ.
  {\smallemptycirc} &
  % -- Micro-electr. Exposure
  {\smallemptycirc} &
  % -- Weak Signature
  {\smallemptycirc} &
  % -- Inadeq. Sig. Verif. 
  % -- SYSTEM
  {\smallfullcirc} &
  % -- Insec. Permissions
  {\smallemptycirc} &
  % -- Library Vulnerability
  {\smallfullcirc} &
  % -- OS Vulnerabilities
  {\smallemptycirc} &
  % -- Coding Errors
  {\smallemptycirc} &
  % -- Insec. Network
  %  -- SYSTEM
  {\smallemptycirc} &
  % -- Insec. User Interactions
  {\smallemptycirc} &
  % -- Comp. Provider
  % -- EXT
  {\smallemptycirc} &
  % -- Malicious Insider
  {\smallemptycirc} &
  % -- Insider Compromise
  % -- INSIDER
  {\smallemptycirc} &
  {\smallemptycirc} &
  {\smallfullcirc} &
  {\smallfullcirc} &
  {\smallemptycirc} &
  {\smallemptycirc} &
  {\smallemptycirc} &
  {\smallfullcirc} &
  {\smallfullcirc} &
  {\smallemptycirc} &
  {\smallemptycirc} &
  {\smallemptycirc} &
  {\smallemptycirc} &
  {\smallemptycirc} &
  % {\smallemptycirc} &
  {\smallfullcirc} &
  {\smallemptycirc} &
  % -- &
  % -- &
  % -- &
  {\smallemptycirc} &
  {\smallfullcirc} &
  {\smallemptycirc} &
  % {\smallfullcirc} &
  {\smallemptycirc} &
  {\smallemptycirc} &
  {\smallfullcirc} &
  {\smallemptycirc} &
  {\smallemptycirc} &
  \cellcolor{g2}{$1$} &
\cellcolor{r0}{$0$} &

  
  % {\smallemptycirc} &
  % {\smallfullcirc} &
  % {\smallemptycirc} &
  % {\smallemptycirc} &
  % {\smallemptycirc} &
  % {\smallemptycirc} &
  % {\smallemptycirc} &
  % {\smallemptycirc} &
  % {\smallemptycirc} &
  % {\smallemptycirc} &
  % {\smallemptycirc} &
  % {\smallemptycirc} &
  % {\smallemptycirc} &
  % {\smallemptycirc} &
  % {\smallemptycirc} &
  % {\smallemptycirc} &
  % {\smallemptycirc} &
  % {\smallfullcirc} &
  % {\smallemptycirc} &
  % {\smallemptycirc} &
  % {\smallemptycirc} &
  % {\smallemptycirc} &
  % {\smallemptycirc} &
  % {\smallemptycirc} &
  % {\smallemptycirc} &
  % {\smallemptycirc} &
  % Access Control Restrictions** 
  \cite{qi2012spad, li2020android}
   \\ 
  % \cite{Tirronen2018StoppingData, indusface}
   &
\multirow{2}{*}{\hyperref[sec:social]{Social Engineering}} &
   Phishing \cite{andryukhin2019phishing} & 
  {\smallemptycirc} &
  % -- rng
  {\smallemptycirc} &
  % -- inadequ auth
  {\smallemptycirc} &
  % -- inadequ encry
  {\smallemptycirc} &
  % -- Appl. Logic Flaw
  {\smallemptycirc} &
  % -- Low-strength pwds
  {\smallemptycirc} &
  % -- Data Leakage
  {\smallemptycirc} &
  % -- Data Remanence
  {\smallemptycirc} &
  % -- Data Remanence
  {\smallemptycirc} &
  % -- Insec. Boot Environ.
  {\smallemptycirc} &
  % -- Micro-electr. Exposure
  {\smallemptycirc} &
  % -- Weak Signature
  {\smallemptycirc} &
  % -- Inadeq. Sig. Verif. 
  % -- SYSTEM
  {\smallemptycirc} &
  % -- Insec. Permissions
  {\smallemptycirc} &
  % -- Library Vulnerability
  {\smallemptycirc} &
  % -- OS Vulnerabilities
  {\smallemptycirc} &
  % -- Coding Errors
  {\smallemptycirc} &
  % -- Insec. Network
  %  -- SYSTEM
  {\smallfullcirc} &
  % -- Insec. User Interactions
  {\smallfullcirc} &
  % -- Comp. Provider
  % -- EXT
  {\smallfullcirc} &
  % -- Malicious Insider
  {\smallfullcirc} &
  % -- Insider Compromise
  % -- INSIDER
  {\smallfullcirc} &
  {\smallemptycirc} &
  {\smallfullcirc} &
  {\smallfullcirc} &
  {\smallemptycirc} &
  {\smallemptycirc} &
  {\smallemptycirc} &
  {\smallemptycirc} &
  {\smallfullcirc} &
  {\smallemptycirc} &
  {\smallemptycirc} &
  {\smallemptycirc} &
  {\smallemptycirc} &
  {\smallemptycirc} &
  % {\smallemptycirc} &
  {\smallemptycirc} &
  {\smallfullcirc} &
  % -- &
  % -- &
  % -- &
  {\smallemptycirc} &
  {\smallfullcirc} &
  {\smallemptycirc} &
  % {\smallfullcirc} &
  {\smallfullcirc} &
  {\smallfullcirc} &
  {\smallfullcirc} &
  {\smallfullcirc} &
  {\smallfullcirc} &
    \cellcolor{g2}{$1$} &
\cellcolor{r5}{$15$} &

  
  % {\smallemptycirc} &
  % {\smallemptycirc} &
  % {\smallemptycirc} &
  % {\smallfullcirc} &
  % {\smallemptycirc} &
  % {\smallemptycirc} &
  % {\smallfullcirc} &
  % {\smallfullcirc} &
  % {\smallemptycirc} &
  % {\smallemptycirc} &
  % {\smallemptycirc} &
  % {\smallemptycirc} &
  % {\smallemptycirc} &
  % {\smallemptycirc} &
  % {\smallemptycirc} &
  % {\smallemptycirc} &
  % {\smallemptycirc} &
  % {\smallemptycirc} &
  % {\smallemptycirc} &
  % {\smallemptycirc} &
  % {\smallemptycirc} &
  % {\smallemptycirc} &
  % {\smallemptycirc} &
  % {\smallemptycirc} &
  % {\smallemptycirc} &
  % {\smallemptycirc} &
   % Multi-factor Authentication 
   \cite{Aratani2015AuthenticationChannel, bip11, Lindell2020SecureComputation} 
   \\
   &
&
   Address Poisoning \cite{MetaMaskScam} & 
  {\smallemptycirc} &
  % -- rng
  {\smallemptycirc} &
  % -- inadequ auth
  {\smallemptycirc} &
  % -- inadequ encry
  {\smallemptycirc} &
  % -- Appl. Logic Flaw
  {\smallemptycirc} &
  % -- Low-strength pwds
  {\smallemptycirc} &
  % -- Data Leakage
  {\smallemptycirc} &
  % -- Data Remanence
  {\smallemptycirc} &
  % -- Data Remanence
  {\smallemptycirc} &
  % -- Insec. Boot Environ.
  {\smallemptycirc} &
  % -- Micro-electr. Exposure
  {\smallemptycirc} &
  % -- Weak Signature
  {\smallemptycirc} &
  % -- Inadeq. Sig. Verif. 
  % -- SYSTEM
  {\smallemptycirc} &
  % -- Insec. Permissions
  {\smallemptycirc} &
  % -- Library Vulnerability
  {\smallemptycirc} &
  % -- OS Vulnerabilities
  {\smallemptycirc} &
  % -- Coding Errors
  {\smallemptycirc} &
  % -- Insec. Network
  %  -- SYSTEM
  {\smallfullcirc} &
  % -- Insec. User Interactions
  {\smallemptycirc} &
  % -- Comp. Provider
  % -- EXT
  {\smallemptycirc} &
  % -- Malicious Insider
  {\smallemptycirc} &
  % -- Insider Compromise
  % -- INSIDER
  {\smallemptycirc} &
  {\smallemptycirc} &
  {\smallemptycirc} &
  {\smallemptycirc} &
  {\smallemptycirc} &
  {\smallfullcirc} &
  {\smallemptycirc} &
  {\smallemptycirc} &
  {\smallemptycirc} &
  {\smallemptycirc} &
  {\smallfullcirc} &
  {\smallemptycirc} &
  {\smallemptycirc} &
  {\smallemptycirc} &
  % {\smallemptycirc} &
  {\smallemptycirc} &
  {\smallfullcirc} &
  % -- &
  % -- &
  % -- &
  {\smallemptycirc} &
  {\smallemptycirc} &
  {\smallemptycirc} &
  % {\smallfullcirc} &
  {\smallfullcirc} &
  {\smallfullcirc} &
  {\smallfullcirc} &
  {\smallfullcirc} &
  {\smallfullcirc} &
  \cellcolor{g0}{$0$} &
\cellcolor{r2}{$1$} &

  
  % {\smallemptycirc} &
  % {\smallemptycirc} &
  % {\smallemptycirc} &
  % {\smallfullcirc} &
  % {\smallemptycirc} &
  % {\smallemptycirc} &
  % {\smallfullcirc} &
  % {\smallfullcirc} &
  % {\smallemptycirc} &
  % {\smallemptycirc} &
  % {\smallemptycirc} &
  % {\smallemptycirc} &
  % {\smallemptycirc} &
  % {\smallemptycirc} &
  % {\smallemptycirc} &
  % {\smallemptycirc} &
  % {\smallemptycirc} &
  % {\smallemptycirc} &
  % {\smallemptycirc} &
  % {\smallemptycirc} &
  % {\smallemptycirc} &
  % {\smallemptycirc} &
  % {\smallemptycirc} &
  % {\smallemptycirc} &
  % {\smallemptycirc} &
  % {\smallemptycirc} &
   % Address White-listing 
   % Industry Defence
   % Destination address management tools such as address whitelisitng 
   \cite{ManageAddresses} 
   \\
 % &
 % &
 % Honeypot Scams *** \cite{Torres2019TheContracts} & 
 %   {\smallemptycirc} &
 %  % -- rng
 %  {\smallemptycirc} &
 %  % -- inadequ auth
 %  {\smallemptycirc} &
 %  % -- inadequ encry
 %  {\smallemptycirc} &
 %  % -- Appl. Logic Flaw
 %  {\smallemptycirc} &
 %  % -- Low-strength pwds
 %  {\smallemptycirc} &
 %  % -- Data Leakage
 %  {\smallemptycirc} &
 %  % -- Data Remanence
 %  {\smallemptycirc} &
 %  % -- Data Remanence
 %  {\smallemptycirc} &
 %  % -- Insec. Boot Environ.
 %  {\smallemptycirc} &
 %  % -- Micro-electr. Exposure
 %  {\smallemptycirc} &
 %  % -- Weak Signature
 %  {\smallemptycirc} &
 %  % -- Inadeq. Sig. Verif. 
 %  % -- SYSTEM
 %  {\smallemptycirc} &
 %  % -- Insec. Permissions
 %  {\smallemptycirc} &
 %  % -- Library Vulnerability
 %  {\smallemptycirc} &
 %  % -- OS Vulnerabilities
 %  {\smallemptycirc} &
 %  % -- Coding Errors
 %  {\smallemptycirc} &
 %  % -- Insec. Network
 %  %  -- SYSTEM
 %  {\smallfullcirc} &
 %  % -- Insec. User Interactions
 %  {\smallemptycirc} &
 %  % -- Comp. Provider
 %  % -- EXT
 %  {\smallemptycirc} &
 %  % -- Malicious Insider
 %  {\smallemptycirc} &
 %  % -- Insider Compromise
 %  % -- INSIDER
 %  {\smallemptycirc} &
 %  {\smallemptycirc} &
 %  {\smallemptycirc} &
 %  {\smallemptycirc} &
 %  {\smallemptycirc} &
 %  {\smallemptycirc} &
 %  {\smallemptycirc} &
 %  {\smallemptycirc} &
 %  {\smallemptycirc} &
 %  {\smallemptycirc} &
 %  {\smallemptycirc} &
 %  {\smallemptycirc} &
 %  {\smallemptycirc} &
 %  {\smallemptycirc} &
 %  % {\smallemptycirc} &
 %  {\smallemptycirc} &
 %  {\smallemptycirc} &
 %  % -- &
 %  % -- &
 %  % -- &
 %  {\smallemptycirc} &
 %  {\smallemptycirc} &
 %  {\smallemptycirc} &
 %  % {\smallfullcirc} &
 %  {\smallfullcirc} &
 %  {\smallfullcirc} &
 %  {\smallfullcirc} &
 %  {\smallfullcirc} &
 %  {\smallfullcirc} &
  
 %  % {\smallfullcirc} &
 %  % {\smallemptycirc} &
 %  % {\smallemptycirc} &
 %  % {\smallemptycirc} &
 %  % {\smallemptycirc} &
 %  % {\smallemptycirc} &
 %  % {\smallemptycirc} &
 %  % {\smallemptycirc} &
 %  % {\smallemptycirc} &
 %  % {\smallemptycirc} &
 %  % {\smallemptycirc} &
 %  % {\smallemptycirc} &
 %  % {\smallemptycirc} &
 %  % {\smallemptycirc} &
 %  % {\smallemptycirc} &
 %  % {\smallemptycirc} &
 %  % {\smallemptycirc} &
 %  % {\smallemptycirc} &
 %  % {\smallemptycirc} &
 %  % {\smallemptycirc} &
 %  % {\smallemptycirc} &
 %  % {\smallemptycirc} &
 %  % {\smallemptycirc} &
 %  % {\smallemptycirc} &
 %  % {\smallemptycirc} &
 %  % {\smallemptycirc} &
 %  % Honeypot Detection Technique
 %  \cite{Torres2019TheContracts}
 %  \\
\midrule
\multirow{4}{*}{Authentication} &
 \multirow{2}{*}{\hyperref[sec:cred-crack]{Credential Cracking}} &
  Brute-force \cite{Kiktenko2019DetectingWallets, volety2019cracking, Byun2024AAttacks} & 
{\smallemptycirc} &
  % -- rng
  {\smallfullcirc} &
  % -- inadequ auth
  {\smallemptycirc} &
  % -- inadequ encry
  {\smallemptycirc} &
  % -- Appl. Logic Flaw
  {\smallfullcirc} &
  % -- Low-strength pwds
  {\smallemptycirc} &
  % -- Data Leakage
  {\smallemptycirc} &
  % -- Data Remanence
  {\smallemptycirc} &
  % -- Data Remanence
  {\smallemptycirc} &
  % -- Insec. Boot Environ.
  {\smallemptycirc} &
  % -- Micro-electr. Exposure
  {\smallemptycirc} &
  % -- Weak Signature
  {\smallemptycirc} &
  % -- Inadeq. Sig. Verif. 
  % -- SYSTEM
  {\smallemptycirc} &
  % -- Insec. Permissions
  {\smallemptycirc} &
  % -- Library Vulnerability
  {\smallemptycirc} &
  % -- OS Vulnerabilities
  {\smallemptycirc} &
  % -- Coding Errors
  {\smallemptycirc} &
  % -- Insec. Network
  %  -- SYSTEM
  {\smallemptycirc} &
  % -- Insec. User Interactions
  {\smallemptycirc} &
  % -- Comp. Provider
  % -- EXT
  {\smallemptycirc} &
  % -- Malicious Insider
  {\smallemptycirc} &
  % -- Insider Compromise
  % -- INSIDER
  {\smallemptycirc} &
  {\smallemptycirc} &
  {\smallfullcirc} &
  {\smallfullcirc} &
  {\smallemptycirc} &
  {\smallemptycirc} &
  {\smallemptycirc} &
  {\smallemptycirc} &
  {\smallfullcirc} &
  {\smallemptycirc} &
  {\smallemptycirc} &
  {\smallemptycirc} &
  {\smallemptycirc} &
  {\smallemptycirc} &
  % {\smallemptycirc} &
  {\smallemptycirc} &
  {\smallemptycirc} &
  % -- &
  % -- &
  % -- &
  {\smallemptycirc} &
  {\smallfullcirc} &
  {\smallemptycirc} &
  % {\smallfullcirc} &
  {\smallfullcirc} &
  {\smallfullcirc} &
  {\smallfullcirc} &
  {\smallemptycirc} &
  {\smallfullcirc} &
   \cellcolor{g6}{$3$} &
\cellcolor{r0}{$0$} &
  
  % {\smallemptycirc} &
  % {\smallemptycirc} &
  % {\smallemptycirc} &
  % {\smallfullcirc} &
  % {\smallfullcirc} &
  % {\smallemptycirc} &
  % {\smallfullcirc} &
  % {\smallfullcirc} &
  % {\smallemptycirc} &
  % {\smallemptycirc} &
  % {\smallemptycirc} &
  % {\smallemptycirc} &
  % {\smallemptycirc} &
  % {\smallemptycirc} &
  % {\smallemptycirc} &
  % {\smallemptycirc} &
  % {\smallemptycirc} &
  % {\smallemptycirc} &
  % {\smallemptycirc} &
  % {\smallemptycirc} &
  % {\smallemptycirc} &
  % {\smallemptycirc} &
  % {\smallemptycirc} &
  % {\smallemptycirc} &
  % {\smallemptycirc} &
  % {\smallemptycirc} &
  % Advanced Passwords 
  \cite{Kiktenko2019DetectingWallets, Byun2024AAttacks}
   \\
 &
 &
 Dictionary \cite{praitheeshan2019security, Uddin2021Horus:Wallets} & 
   {\smallemptycirc} &
  % -- rng
  {\smallfullcirc} &
  % -- inadequ auth
  {\smallemptycirc} &
  % -- inadequ encry
  {\smallemptycirc} &
  % -- Appl. Logic Flaw
  {\smallemptycirc} &
  % -- Low-strength pwds
  {\smallfullcirc} &
  % -- Data Leakage
  {\smallemptycirc} &
  % -- Data Remanence
  {\smallemptycirc} &
  % -- Data Remanence
  {\smallemptycirc} &
  % -- Insec. Boot Environ.
  {\smallemptycirc} &
  % -- Micro-electr. Exposure
  {\smallemptycirc} &
  % -- Weak Signature
  {\smallemptycirc} &
  % -- Inadeq. Sig. Verif. 
  % -- SYSTEM
  {\smallemptycirc} &
  % -- Insec. Permissions
  {\smallemptycirc} &
  % -- Library Vulnerability
  {\smallemptycirc} &
  % -- OS Vulnerabilities
  {\smallemptycirc} &
  % -- Coding Errors
  {\smallemptycirc} &
  % -- Insec. Network
  %  -- SYSTEM
  {\smallemptycirc} &
  % -- Insec. User Interactions
  {\smallemptycirc} &
  % -- Comp. Provider
  % -- EXT
  {\smallemptycirc} &
  % -- Malicious Insider
  {\smallemptycirc} &
  % -- Insider Compromise
  % -- INSIDER
  {\smallemptycirc} &
  {\smallemptycirc} &
  {\smallfullcirc} &
  {\smallemptycirc} &
  {\smallemptycirc} &
  {\smallemptycirc} &
  {\smallemptycirc} &
  {\smallemptycirc} &
  {\smallfullcirc} &
  {\smallemptycirc} &
  {\smallemptycirc} &
  {\smallemptycirc} &
  {\smallemptycirc} &
  {\smallemptycirc} &
  % {\smallemptycirc} &
  {\smallemptycirc} &
  {\smallemptycirc} &
  % -- &
  % -- &
  % -- &
  {\smallemptycirc} &
  {\smallfullcirc} &
  {\smallemptycirc} &
  % {\smallfullcirc} &
  {\smallfullcirc} &
  {\smallfullcirc} &
  {\smallfullcirc} &
  {\smallemptycirc} &
  {\smallfullcirc} &
   \cellcolor{g4}{$2$} &
\cellcolor{r0}{$0$} &
  
  % {\smallfullcirc} &
  % {\smallemptycirc} &
  % {\smallemptycirc} &
  % {\smallemptycirc} &
  % {\smallemptycirc} &
  % {\smallemptycirc} &
  % {\smallemptycirc} &
  % {\smallemptycirc} &
  % {\smallemptycirc} &
  % {\smallemptycirc} &
  % {\smallemptycirc} &
  % {\smallemptycirc} &
  % {\smallemptycirc} &
  % {\smallemptycirc} &
  % {\smallemptycirc} &
  % {\smallemptycirc} &
  % {\smallemptycirc} &
  % {\smallemptycirc} &
  % {\smallemptycirc} &
  % {\smallemptycirc} &
  % {\smallemptycirc} &
  % {\smallemptycirc} &
  % {\smallemptycirc} &
  % {\smallemptycirc} &
  % {\smallemptycirc} &
  % {\smallemptycirc} &
  % Custom Keyboard Functions
  \cite{aldawood2020advanced}
  \\
&
 \multirow{2}{*}{\hyperref[sec:iden-spoof]{Identity Spoofing}} &
  Fake Biometrics \cite{galbally2013image} & 
   {\smallemptycirc} &
  % -- rng
  {\smallfullcirc} &
  % -- inadequ auth
  {\smallemptycirc} &
  % -- inadequ encry
  {\smallemptycirc} &
  % -- Appl. Logic Flaw
  {\smallemptycirc} &
  % -- Low-strength pwds
  {\smallemptycirc} &
  % -- Data Leakage
  {\smallemptycirc} &
  % -- Data Remanence
  {\smallemptycirc} &
  % -- Data Remanence
  {\smallemptycirc} &
  % -- Insec. Boot Environ.
  {\smallemptycirc} &
  % -- Micro-electr. Exposure
  {\smallemptycirc} &
  % -- Weak Signature
  {\smallemptycirc} &
  % -- Inadeq. Sig. Verif. 
  % -- SYSTEM
  {\smallemptycirc} &
  % -- Insec. Permissions
  {\smallemptycirc} &
  % -- Library Vulnerability
  {\smallemptycirc} &
  % -- OS Vulnerabilities
  {\smallemptycirc} &
  % -- Coding Errors
  {\smallemptycirc} &
  % -- Insec. Network
  %  -- SYSTEM
  {\smallemptycirc} &
  % -- Insec. User Interactions
  {\smallemptycirc} &
  % -- Comp. Provider
  % -- EXT
  {\smallemptycirc} &
  % -- Malicious Insider
  {\smallemptycirc} &
  % -- Insider Compromise
  % -- INSIDER
  {\smallemptycirc} &
  {\smallemptycirc} &
  {\smallemptycirc} &
  {\smallhalfcirc} &
  {\smallemptycirc} &
  {\smallemptycirc} &
  {\smallemptycirc} &
  {\smallemptycirc} &
  {\smallfullcirc} &
  {\smallemptycirc} &
  {\smallemptycirc} &
  {\smallemptycirc} &
  {\smallemptycirc} &
  {\smallemptycirc} &
  % {\smallemptycirc} &
  {\smallemptycirc} &
  {\smallemptycirc} &
  % -- &
  % -- &
  % -- &
  {\smallemptycirc} &
  {\smallfullcirc} &
  {\smallemptycirc} &
  % {\smallfullcirc} &
  {\smallemptycirc} &
  {\smallemptycirc} &
  {\smallfullcirc} &
  {\smallfullcirc} &
  {\smallemptycirc} &
   \cellcolor{g2}{$1$} &
\cellcolor{r0}{$0$} &

  
  % {\smallemptycirc} &
  % {\smallemptycirc} &
  % {\smallemptycirc} &
  % {\smallfullcirc} &
  % {\smallemptycirc} &
  % {\smallfullcirc} &
  % {\smallfullcirc} &
  % {\smallfullcirc} &
  % {\smallemptycirc} &
  % {\smallemptycirc} &
  % {\smallemptycirc} &
  % {\smallemptycirc} &
  % {\smallemptycirc} &
  % {\smallemptycirc} &
  % {\smallemptycirc} &
  % {\smallemptycirc} &
  % {\smallemptycirc} &
  % {\smallemptycirc} &
  % {\smallemptycirc} &
  % {\smallemptycirc} &
  % {\smallemptycirc} &
  % {\smallemptycirc} &
  % {\smallemptycirc} &
  % {\smallemptycirc} &
  % {\smallemptycirc} &
  % {\smallemptycirc} &
  % Liveness Assessment Features
  \cite{galbally2013image} 
   \\
 &
 &
 SIM Swap \cite{Kim2022ACountermeasures} & 
   {\smallemptycirc} &
  % -- rng
  {\smallfullcirc} &
  % -- inadequ auth
  {\smallemptycirc} &
  % -- inadequ encry
  {\smallemptycirc} &
  % -- Appl. Logic Flaw
  {\smallemptycirc} &
  % -- Low-strength pwds
  {\smallemptycirc} &
  % -- Data Leakage
  {\smallemptycirc} &
  % -- Data Remanence
  {\smallemptycirc} &
  % -- Data Remanence
  {\smallemptycirc} &
  % -- Insec. Boot Environ.
  {\smallemptycirc} &
  % -- Micro-electr. Exposure
  {\smallemptycirc} &
  % -- Weak Signature
  {\smallemptycirc} &
  % -- Inadeq. Sig. Verif. 
  % -- SYSTEM
  {\smallemptycirc} &
  % -- Insec. Permissions
  {\smallemptycirc} &
  % -- Library Vulnerability
  {\smallemptycirc} &
  % -- OS Vulnerabilities
  {\smallemptycirc} &
  % -- Coding Errors
  {\smallemptycirc} &
  % -- Insec. Network
  %  -- SYSTEM
  {\smallemptycirc} &
  % -- Insec. User Interactions
  {\smallemptycirc} &
  % -- Comp. Provider
  % -- EXT
  {\smallemptycirc} &
  % -- Malicious Insider
  {\smallemptycirc} &
  % -- Insider Compromise
  % -- INSIDER
  {\smallemptycirc} &
  {\smallemptycirc} &
  {\smallemptycirc} &
  {\smallhalfcirc} &
  {\smallemptycirc} &
  {\smallemptycirc} &
  {\smallemptycirc} &
  {\smallemptycirc} &
  {\smallfullcirc} &
  {\smallemptycirc} &
  {\smallemptycirc} &
  {\smallemptycirc} &
  {\smallemptycirc} &
  {\smallemptycirc} &
  % {\smallemptycirc} &
  {\smallemptycirc} &
  {\smallemptycirc} &
  % -- &
  % -- &
  % -- &
  {\smallemptycirc} &
  {\smallfullcirc} &
  {\smallemptycirc} &
  % {\smallfullcirc} &
  {\smallemptycirc} &
  {\smallemptycirc} &
  {\smallfullcirc} &
  {\smallfullcirc} &
  {\smallemptycirc} &
   \cellcolor{g0}{$0$} &
\cellcolor{r2}{$1$} &
  
  % {\smallfullcirc} &
  % {\smallemptycirc} &
  % {\smallemptycirc} &
  % {\smallemptycirc} &
  % {\smallemptycirc} &
  % {\smallemptycirc} &
  % {\smallemptycirc} &
  % {\smallemptycirc} &
  % {\smallemptycirc} &
  % {\smallemptycirc} &
  % {\smallemptycirc} &
  % {\smallemptycirc} &
  % {\smallemptycirc} &
  % {\smallemptycirc} &
  % {\smallemptycirc} &
  % {\smallemptycirc} &
  % {\smallemptycirc} &
  % {\smallemptycirc} &
  % {\smallemptycirc} &
  % {\smallemptycirc} &
  % {\smallemptycirc} &
  % {\smallemptycirc} &
  % {\smallemptycirc} &
  % {\smallemptycirc} &
  % {\smallemptycirc} &
  % {\smallemptycirc} &
  % Custom Keyboard Functions
  \cite{Kim2022ACountermeasures}
  \\
\midrule
\multirow{5}{*}{Storage}
   &
\hyperref[sec:fau-inj]{Fault Injection} &
  Fault Injection Attacks \cite{Akter2023AChallenges, hajdu2020using} &{\smallemptycirc} &
  % -- rng
  {\smallemptycirc} &
  % -- inadequ auth
  {\smallemptycirc} &
  % -- inadequ encry
  {\smallemptycirc} &
  % -- Appl. Logic Flaw
  {\smallemptycirc} &
  % -- Low-strength pwds
  {\smallemptycirc} &
  % -- Data Leakage
  {\smallfullcirc} &
  % -- Data Remanence
  {\smallfullcirc} &
  % -- Data Remanence
  {\smallemptycirc} &
  % -- Insec. Boot Environ.
  {\smallemptycirc} &
  % -- Micro-electr. Exposure
  {\smallemptycirc} &
  % -- Weak Signature
  {\smallemptycirc} &
  % -- Inadeq. Sig. Verif. 
  % -- SYSTEM
  {\smallemptycirc} &
  % -- Insec. Permissions
  {\smallemptycirc} &
  % -- Library Vulnerability
  {\smallemptycirc} &
  % -- OS Vulnerabilities
  {\smallemptycirc} &
  % -- Coding Errors
  {\smallemptycirc} &
  % -- Insec. Network
  %  -- SYSTEM
  {\smallemptycirc} &
  % -- Insec. User Interactions
  {\smallemptycirc} &
  % -- Comp. Provider
  % -- EXT
  {\smallemptycirc} &
  % -- Malicious Insider
  {\smallemptycirc} &
  % -- Insider Compromise
  % -- INSIDER
  {\smallfullcirc} &
  {\smallemptycirc} &
  {\smallemptycirc} &
  {\smallemptycirc} &
  {\smallfullcirc} &
  {\smallemptycirc} &
  {\smallemptycirc} &
  {\smallemptycirc} &
  {\smallemptycirc} &
  {\smallfullcirc} &
  {\smallemptycirc} &
  {\smallemptycirc} &
  {\smallemptycirc} &
  {\smallemptycirc} &
  % {\smallemptycirc} &
  {\smallemptycirc} &
  {\smallemptycirc} &
  % -- &
  % -- &
  % -- &
  {\smallemptycirc} &
  {\smallfullcirc} &
  {\smallemptycirc} &
  % {\smallfullcirc} &
  {\smallfullcirc} &
  {\smallfullcirc} &
  {\smallfullcirc} &
  {\smallfullcirc} &
  {\smallfullcirc} &
   \cellcolor{g4}{$2$} &
\cellcolor{r0}{$0$} &

  
  % {\smallemptycirc} &
  % {\smallemptycirc} &
  % {\smallemptycirc} &
  % {\smallemptycirc} &
  % {\smallemptycirc} &
  % {\smallemptycirc} &
  % {\smallemptycirc} &
  % {\smallemptycirc} &
  % {\smallemptycirc} &
  % {\smallemptycirc} &
  % {\smallemptycirc} &
  % {\smallemptycirc} &
  % {\smallemptycirc} &
  % {\smallemptycirc} &
  % {\smallemptycirc} &
  % {\smallemptycirc} &
  % {\smallemptycirc} &
  % {\smallemptycirc} &
  % {\smallemptycirc} &
  % {\smallemptycirc} &
  % {\smallemptycirc} &
  % {\smallemptycirc} &
  % {\smallemptycirc} &
  % {\smallfullcirc} &
  % {\smallemptycirc} &
  % {\smallemptycirc} &
  % Algorithmic Fault Detection
  \cite{Shuvo2023AAttacks, breier2022practical} 
   \\
 &
 \multirow{2}{*}{\hyperref[sec:tam-per]{Physical Tampering}} &
  Evil Maid \cite{Shaikh2022SurveyExchanges} &
{\smallemptycirc} &
  % -- rng
  {\smallfullcirc} &
  % -- inadequ auth
  {\smallemptycirc} &
  % -- inadequ encry
  {\smallemptycirc} &
  % -- Appl. Logic Flaw
  {\smallemptycirc} &
  % -- Low-strength pwds
  {\smallemptycirc} &
  % -- Data Leakage
  {\smallemptycirc} &
  % -- Data Remanence
  {\smallemptycirc} &
  % -- Data Remanence
  {\smallfullcirc} &
  % -- Insec. Boot Environ.
  {\smallemptycirc} &
  % -- Micro-electr. Exposure
  {\smallemptycirc} &
  % -- Weak Signature
  {\smallemptycirc} &
  % -- Inadeq. Sig. Verif. 
  % -- SYSTEM
  {\smallemptycirc} &
  % -- Insec. Permissions
  {\smallemptycirc} &
  % -- Library Vulnerability
  {\smallemptycirc} &
  % -- OS Vulnerabilities
  {\smallemptycirc} &
  % -- Coding Errors
  {\smallemptycirc} &
  % -- Insec. Network
  %  -- SYSTEM
  {\smallfullcirc} &
  % -- Insec. User Interactions
  {\smallemptycirc} &
  % -- Comp. Provider
  % -- EXT
  {\smallemptycirc} &
  % -- Malicious Insider
  {\smallemptycirc} &
  % -- Insider Compromise
  % -- INSIDER
  {\smallemptycirc} &
  {\smallemptycirc} &
  {\smallemptycirc} &
  {\smallfullcirc} &
  {\smallemptycirc} &
  {\smallemptycirc} &
  {\smallemptycirc} &
  {\smallemptycirc} &
  {\smallfullcirc} &
  {\smallemptycirc} &
  {\smallemptycirc} &
  {\smallemptycirc} &
  {\smallemptycirc} &
  {\smallemptycirc} &
  % {\smallemptycirc} &
  {\smallemptycirc} &
  {\smallemptycirc} &
  % -- &
  % -- &
  % -- &
  {\smallfullcirc} &
  {\smallfullcirc} &
  {\smallemptycirc} &
  % {\smallfullcirc} &
  {\smallemptycirc} &
  {\smallemptycirc} &
  {\smallemptycirc} &
  {\smallemptycirc} &
  {\smallfullcirc} &
   \cellcolor{g2}{$1$} &
\cellcolor{r0}{$0$} &

  
  % {\smallemptycirc} &
  % {\smallemptycirc} &
  % {\smallemptycirc} &
  % {\smallfullcirc} &
  % {\smallemptycirc} &
  % {\smallemptycirc} &
  % {\smallemptycirc} &
  % {\smallemptycirc} &
  % {\smallemptycirc} &
  % {\smallemptycirc} &
  % {\smallemptycirc} &
  % {\smallemptycirc} &
  % {\smallemptycirc} &
  % {\smallemptycirc} &
  % {\smallemptycirc} &
  % {\smallemptycirc} &
  % {\smallemptycirc} &
  % {\smallemptycirc} &
  % {\smallemptycirc} &
  % {\smallemptycirc} &
  % {\smallemptycirc} &
  % {\smallemptycirc} &
  % {\smallemptycirc} &
  % {\smallemptycirc} &
  % {\smallemptycirc} &
  % {\smallemptycirc} &
  % Multi-factor Authentication
  \cite{Aratani2015AuthenticationChannel} 
   \\
 &
 &
  Microscopy \cite{courbon2016reverse} &
   % \multicolumn{2}{c}{\hyperref[sec:microscopy]{Microscopy} \cite{courbon2016reverse}}  & 
   {\smallemptycirc} &
  % -- rng
  {\smallemptycirc} &
  % -- inadequ auth
  {\smallemptycirc} &
  % -- inadequ encry
  {\smallemptycirc} &
  % -- Appl. Logic Flaw
  {\smallemptycirc} &
  % -- Low-strength pwds
  {\smallemptycirc} &
  % -- Data Leakage
  {\smallemptycirc} &
  % -- Data Remanence
  {\smallemptycirc} &
  % -- Data Remanence
  {\smallemptycirc} &
  % -- Insec. Boot Environ.
  {\smallfullcirc} &
  % -- Micro-electr. Exposure
  {\smallemptycirc} &
  % -- Weak Signature
  {\smallemptycirc} &
  % -- Inadeq. Sig. Verif. 
  % -- SYSTEM
  {\smallemptycirc} &
  % -- Insec. Permissions
  {\smallemptycirc} &
  % -- Library Vulnerability
  {\smallemptycirc} &
  % -- OS Vulnerabilities
  {\smallemptycirc} &
  % -- Coding Errors
  {\smallemptycirc} &
  % -- Insec. Network
  %  -- SYSTEM
  {\smallemptycirc} &
  % -- Insec. User Interactions
  {\smallemptycirc} &
  % -- Comp. Provider
  % -- EXT
  {\smallemptycirc} &
  % -- Malicious Insider
  {\smallemptycirc} &
  % -- Insider Compromise
  % -- INSIDER
  {\smallemptycirc} &
  {\smallemptycirc} &
  {\smallemptycirc} &
  {\smallemptycirc} &
  {\smallfullcirc} &
  {\smallemptycirc} &
  {\smallemptycirc} &
  {\smallemptycirc} &
  {\smallemptycirc} &
  {\smallfullcirc} &
  {\smallemptycirc} &
  {\smallemptycirc} &
  {\smallemptycirc} &
  {\smallemptycirc} &
  % {\smallemptycirc} &
  {\smallemptycirc} &
  {\smallemptycirc} &
  % -- &
  % -- &
  % -- &
  {\smallemptycirc} &
  {\smallfullcirc} &
  {\smallemptycirc} &
  % {\smallemptycirc} &
  {\smallemptycirc} &
  {\smallemptycirc} &
  {\smallemptycirc} &
  {\smallemptycirc} &
  {\smallfullcirc} &
   \cellcolor{g2}{$1$} &
\cellcolor{r2}{$1$} &

% ** need to cite the source from industry but it was ledger or trezor

  
  % {\smallemptycirc} &
  % {\smallemptycirc} &
  % {\smallemptycirc} &
  % {\smallemptycirc} &
  % {\smallemptycirc} &
  % {\smallemptycirc} &
  % {\smallemptycirc} &
  % {\smallemptycirc} &
  % {\smallemptycirc} &
  % {\smallemptycirc} &
  % {\smallemptycirc} &
  % {\smallemptycirc} &
  % {\smallemptycirc} &
  % {\smallemptycirc} &
  % {\smallemptycirc} &
  % {\smallemptycirc} &
  % {\smallemptycirc} &
  % {\smallemptycirc} &
  % {\smallemptycirc} &
  % {\smallfullcirc} &
  % {\smallemptycirc} &
  % {\smallemptycirc} &
  % {\smallemptycirc} &
  % {\smallemptycirc} &
  % {\smallemptycirc} &
  % {\smallemptycirc} &
  % \acf{puf} 
  \cite{hu2020overview, Urien2021InnovativeWallets} 
   \\
   &
\multirow{2}{*}{\hyperref[sec:non-inv-man]{Non-invasive Manip.}} &
  Cold Boot Attack \cite{Shaikh2022SurveyExchanges} &
{\smallemptycirc} &
  % -- rng
  {\smallemptycirc} &
  % -- inadequ auth
  {\smallemptycirc} &
  % -- inadequ encry
  {\smallfullcirc} &
  % -- Appl. Logic Flaw
  {\smallemptycirc} &
  % -- Low-strength pwds
  {\smallemptycirc} &
  % -- Data Leakage
  {\smallemptycirc} &
  % -- Data Remanence
  {\smallemptycirc} &
  % -- Data Remanence
  {\smallfullcirc} &
  % -- Insec. Boot Environ.
  {\smallemptycirc} &
  % -- Micro-electr. Exposure
  {\smallemptycirc} &
  % -- Weak Signature
  {\smallemptycirc} &
  % -- Inadeq. Sig. Verif. 
  % -- SYSTEM
  {\smallfullcirc} &
  % -- Insec. Permissions
  {\smallemptycirc} &
  % -- Library Vulnerability
  {\smallemptycirc} &
  % -- OS Vulnerabilities
  {\smallemptycirc} &
  % -- Coding Errors
  {\smallemptycirc} &
  % -- Insec. Network
  %  -- SYSTEM
  {\smallemptycirc} &
  % -- Insec. User Interactions
  {\smallemptycirc} &
  % -- Comp. Provider
  % -- EXT
  {\smallemptycirc} &
  % -- Malicious Insider
  {\smallemptycirc} &
  % -- Insider Compromise
  % -- INSIDER
  {\smallemptycirc} &
  {\smallemptycirc} &
  {\smallemptycirc} &
  {\smallemptycirc} &
  {\smallfullcirc} &
  {\smallemptycirc} &
  {\smallemptycirc} &
  {\smallemptycirc} &
  {\smallemptycirc} &
  {\smallfullcirc} &
  {\smallemptycirc} &
  {\smallemptycirc} &
  {\smallemptycirc} &
  {\smallemptycirc} &
  % {\smallemptycirc} &
  {\smallemptycirc} &
  {\smallemptycirc} &
  % -- &
  % -- &
  % -- &
  {\smallemptycirc} &
  {\smallfullcirc} &
  {\smallemptycirc} &
  % {\smallfullcirc} &
  {\smallfullcirc} &
  {\smallfullcirc} &
  {\smallfullcirc} &
  {\smallemptycirc} &
  {\smallfullcirc} & 
   \cellcolor{g2}{$1$} &
\cellcolor{r0}{$0$} &

  
  % {\smallemptycirc} &
  % {\smallemptycirc} &
  % {\smallemptycirc} &
  % {\smallemptycirc} &
  % {\smallemptycirc} &
  % {\smallemptycirc} &
  % {\smallemptycirc} &
  % {\smallemptycirc} &
  % {\smallemptycirc} &
  % {\smallemptycirc} &
  % {\smallfullcirc} &
  % {\smallemptycirc} &
  % {\smallemptycirc} &
  % {\smallemptycirc} &
  % {\smallemptycirc} &
  % {\smallemptycirc} &
  % {\smallemptycirc} &
  % {\smallemptycirc} &
  % {\smallemptycirc} &
  % {\smallemptycirc} &
  % {\smallemptycirc} &
  % {\smallfullcirc} &
  % {\smallemptycirc} &
  % {\smallemptycirc} &
  % {\smallemptycirc} &
  % {\smallemptycirc} &
  % Supplementary Storage
  \cite{altuwaijri2020android} 
   \\
 &
 &
 \acs{puf} Attacks \cite{wang2024efficient} &
   {\smallfullcirc} &
  % -- rng
  {\smallfullcirc} &
  % -- inadequ auth
  {\smallemptycirc} &
  % -- inadequ encry
  {\smallemptycirc} &
  % -- Appl. Logic Flaw
  {\smallemptycirc} &
  % -- Low-strength pwds
  {\smallemptycirc} &
  % -- Data Leakage
  {\smallemptycirc} &
  % -- Data Remanence
  {\smallemptycirc} &
  % -- Data Remanence
  {\smallemptycirc} &
  % -- Insec. Boot Environ.
  {\smallfullcirc} &
  % -- Micro-electr. Exposure
  {\smallemptycirc} &
  % -- Weak Signature
  {\smallemptycirc} &
  % -- Inadeq. Sig. Verif. 
  % -- SYSTEM
  {\smallemptycirc} &
  % -- Insec. Permissions
  {\smallemptycirc} &
  % -- Library Vulnerability
  {\smallemptycirc} &
  % -- OS Vulnerabilities
  {\smallemptycirc} &
  % -- Coding Errors
  {\smallemptycirc} &
  % -- Insec. Network
  %  -- SYSTEM
  {\smallemptycirc} &
  % -- Insec. User Interactions
  {\smallemptycirc} &
  % -- Comp. Provider
  % -- EXT
  {\smallemptycirc} &
  % -- Malicious Insider
  {\smallemptycirc} &
  % -- Insider Compromise
  % -- INSIDER
  {\smallemptycirc} &
  {\smallemptycirc} &
  {\smallemptycirc} &
  {\smallemptycirc} &
  {\smallfullcirc} &
  {\smallemptycirc} &
  {\smallemptycirc} &
  {\smallemptycirc} &
  {\smallemptycirc} &
  {\smallfullcirc} &
  {\smallemptycirc} &
  {\smallemptycirc} &
  {\smallemptycirc} &
  {\smallemptycirc} &
  % {\smallemptycirc} &
  {\smallemptycirc} &
  {\smallemptycirc} &
  % -- &
  % -- &
  % -- &
  {\smallemptycirc} &
  {\smallfullcirc} &
  {\smallemptycirc} &
  % {\smallemptycirc} &
  {\smallemptycirc} &
  {\smallemptycirc} &
  {\smallemptycirc} &
  {\smallemptycirc} &
  {\smallfullcirc} &
   \cellcolor{g2}{$1$} &
\cellcolor{r0}{$0$} &

  
  % {\smallemptycirc} &
  % {\smallemptycirc} &
  % {\smallemptycirc} &
  % {\smallemptycirc} &
  % {\smallemptycirc} &
  % {\smallemptycirc} &
  % {\smallemptycirc} &
  % {\smallemptycirc} &
  % {\smallemptycirc} &
  % {\smallemptycirc} &
  % {\smallemptycirc} &
  % {\smallemptycirc} &
  % {\smallemptycirc} &
  % {\smallemptycirc} &
  % {\smallemptycirc} &
  % {\smallemptycirc} &
  % {\smallemptycirc} &
  % {\smallemptycirc} &
  % {\smallemptycirc} &
  % {\smallemptycirc} &
  % {\smallemptycirc} &
  % {\smallemptycirc} &
  % {\smallemptycirc} &
  % {\smallemptycirc} &
  % {\smallemptycirc} &
  % {\smallfullcirc} &
  % \acf{puf} 
  \cite{Park2023, Park2024CloningFunction}
   \\
\midrule
\multirow{5}{*}{Cryptanalysis} 
&
\multirow{3}{*}{\hyperref[sec:side-channel]{Side-channel Analysis}}
&
Timing-based \cite{kocher1996timing}
&
   {\smallemptycirc} &
  % -- rng
  {\smallemptycirc} &
  % -- inadequ auth
  {\smallemptycirc} &
  % -- inadequ encry
  {\smallemptycirc} &
  % -- Appl. Logic Flaw
  {\smallemptycirc} &
  % -- Low-strength pwds
  {\smallfullcirc} &
  % -- Data Leakage
  {\smallemptycirc} &
  % -- Data Remanence
  {\smallemptycirc} &
  % -- Data Remanence
  {\smallemptycirc} &
  % -- Insec. Boot Environ.
  {\smallemptycirc} &
  % -- Micro-electr. Exposure
  {\smallemptycirc} &
  % -- Weak Signature
  {\smallemptycirc} &
  % -- Inadeq. Sig. Verif. 
  % -- SYSTEM
  {\smallemptycirc} &
  % -- Insec. Permissions
  {\smallemptycirc} &
  % -- Library Vulnerability
  {\smallemptycirc} &
  % -- OS Vulnerabilities
  {\smallemptycirc} &
  % -- Coding Errors
  {\smallemptycirc} &
  % -- Insec. Network
  %  -- SYSTEM
  {\smallemptycirc} &
  % -- Insec. User Interactions
  {\smallemptycirc} &
  % -- Comp. Provider
  % -- EXT
  {\smallemptycirc} &
  % -- Malicious Insider
  {\smallemptycirc} &
  % -- Insider Compromise
  % -- INSIDER
  {\smallfullcirc} &
  {\smallemptycirc} &
  {\smallemptycirc} &
  {\smallemptycirc} &
  {\smallemptycirc} &
  {\smallemptycirc} &
  {\smallemptycirc} &
  {\smallfullcirc} &
  {\smallemptycirc} &
  {\smallemptycirc} &
  {\smallemptycirc} &
  {\smallfullcirc} &
  {\smallfullcirc} &
  {\smallemptycirc} &
  % {\smallemptycirc} &
  {\smallemptycirc} &
  {\smallemptycirc} &
  % -- &
  % -- &
  % -- &
  {\smallemptycirc} &
  {\smallfullcirc} &
  {\smallemptycirc} &
  % {\smallfullcirc} &
  {\smallfullcirc} &
  {\smallfullcirc} &
  {\smallfullcirc} &
  {\smallemptycirc} &
  {\smallfullcirc} &
   \cellcolor{g2}{$1$} &
\cellcolor{r0}{$0$} &

  
  % {\smallemptycirc} &
  % {\smallemptycirc} &
  % {\smallemptycirc} &
  % {\smallemptycirc} &
  % {\smallemptycirc} &
  % {\smallemptycirc} &
  % {\smallemptycirc} &
  % {\smallemptycirc} &
  % {\smallemptycirc} &
  % {\smallemptycirc} &
  % {\smallemptycirc} &
  % {\smallemptycirc} &
  % {\smallemptycirc} &
  % {\smallemptycirc} &
  % {\smallemptycirc} &
  % {\smallemptycirc} &
  % {\smallemptycirc} &
  % {\smallemptycirc} &
  % {\smallemptycirc} &
  % {\smallemptycirc} &
  % {\smallemptycirc} &
  % {\smallemptycirc} &
  % {\smallfullcirc} &
  % {\smallemptycirc} &
  % {\smallemptycirc} &
  % {\smallemptycirc} &
  % Memory and Cache Data Split 
  \cite{Akter2023AChallenges, Gupta2019ImpactSecurity} 
   \\
&
&
Power on Crypt. Algo. \cite{Park2023}
&
   {\smallemptycirc} &
  % -- rng
  {\smallemptycirc} &
  % -- inadequ auth
  {\smallemptycirc} &
  % -- inadequ encry
  {\smallemptycirc} &
  % -- Appl. Logic Flaw
  {\smallemptycirc} &
  % -- Low-strength pwds
  {\smallfullcirc} &
  % -- Data Leakage
  {\smallemptycirc} &
  % -- Data Remanence
  {\smallemptycirc} &
  % -- Data Remanence
  {\smallemptycirc} &
  % -- Insec. Boot Environ.
  {\smallemptycirc} &
  % -- Micro-electr. Exposure
  {\smallemptycirc} &
  % -- Weak Signature
  {\smallemptycirc} &
  % -- Inadeq. Sig. Verif. 
  % -- SYSTEM
  {\smallemptycirc} &
  % -- Insec. Permissions
  {\smallemptycirc} &
  % -- Library Vulnerability
  {\smallemptycirc} &
  % -- OS Vulnerabilities
  {\smallemptycirc} &
  % -- Coding Errors
  {\smallemptycirc} &
  % -- Insec. Network
  %  -- SYSTEM
  {\smallemptycirc} &
  % -- Insec. User Interactions
  {\smallemptycirc} &
  % -- Comp. Provider
  % -- EXT
  {\smallemptycirc} &
  % -- Malicious Insider
  {\smallemptycirc} &
  % -- Insider Compromise
  % -- INSIDER
  {\smallfullcirc} &
  {\smallemptycirc} &
  {\smallemptycirc} &
  {\smallemptycirc} &
  {\smallemptycirc} &
  {\smallemptycirc} &
  {\smallemptycirc} &
  {\smallfullcirc} &
  {\smallemptycirc} &
  {\smallemptycirc} &
  {\smallemptycirc} &
  {\smallfullcirc} &
  {\smallfullcirc} &
  {\smallemptycirc} &
  % {\smallemptycirc} &
  {\smallemptycirc} &
  {\smallemptycirc} &
  % -- &
  % -- &
  % -- &
  {\smallemptycirc} &
  {\smallfullcirc} &
  {\smallemptycirc} &
  % {\smallfullcirc} &
  {\smallfullcirc} &
  {\smallfullcirc} &
  {\smallfullcirc} &
  {\smallemptycirc} &
  {\smallfullcirc} &
  \cellcolor{g2}{$1$} &
\cellcolor{r0}{$0$} &

  
  % {\smallemptycirc} &
  % {\smallemptycirc} &
  % {\smallemptycirc} &
  % {\smallemptycirc} &
  % {\smallemptycirc} &
  % {\smallemptycirc} &
  % {\smallemptycirc} &
  % {\smallemptycirc} &
  % {\smallemptycirc} &
  % {\smallemptycirc} &
  % {\smallemptycirc} &
  % {\smallemptycirc} &
  % {\smallemptycirc} &
  % {\smallemptycirc} &
  % {\smallemptycirc} &
  % {\smallemptycirc} &
  % {\smallemptycirc} &
  % {\smallemptycirc} &
  % {\smallemptycirc} &
  % {\smallemptycirc} &
  % {\smallemptycirc} &
  % {\smallemptycirc} &
  % {\smallfullcirc} &
  % {\smallemptycirc} &
  % {\smallemptycirc} &
  % {\smallfullcirc} &
  % Memory and Cache Data Split 
  \cite{Akter2023AChallenges, Gupta2019ImpactSecurity} 
   \\
&
   &
  Power on Hash \cite{Park2024CloningFunction} 
 &
   {\smallemptycirc} &
  % -- rng
  {\smallemptycirc} &
  % -- inadequ auth
  {\smallemptycirc} &
  % -- inadequ encry
  {\smallemptycirc} &
  % -- Appl. Logic Flaw
  {\smallemptycirc} &
  % -- Low-strength pwds
  {\smallfullcirc} &
  % -- Data Leakage
  {\smallemptycirc} &
  % -- Data Remanence
  {\smallemptycirc} &
  % -- Data Remanence
  {\smallemptycirc} &
  % -- Insec. Boot Environ.
  {\smallemptycirc} &
  % -- Micro-electr. Exposure
  {\smallemptycirc} &
  % -- Weak Signature
  {\smallemptycirc} &
  % -- Inadeq. Sig. Verif. 
  % -- SYSTEM
  {\smallemptycirc} &
  % -- Insec. Permissions
  {\smallemptycirc} &
  % -- Library Vulnerability
  {\smallemptycirc} &
  % -- OS Vulnerabilities
  {\smallemptycirc} &
  % -- Coding Errors
  {\smallemptycirc} &
  % -- Insec. Network
  %  -- SYSTEM
  {\smallemptycirc} &
  % -- Insec. User Interactions
  {\smallemptycirc} &
  % -- Comp. Provider
  % -- EXT
  {\smallemptycirc} &
  % -- Malicious Insider
  {\smallemptycirc} &
  % -- Insider Compromise
  % -- INSIDER
  {\smallemptycirc} &
  {\smallemptycirc} &
  {\smallfullcirc} &
  {\smallemptycirc} &
  {\smallemptycirc} &
  {\smallemptycirc} &
  {\smallemptycirc} &
  {\smallemptycirc} &
  {\smallemptycirc} &
  {\smallemptycirc} &
  {\smallemptycirc} &
  {\smallfullcirc} &
  {\smallemptycirc} &
  {\smallemptycirc} &
  % {\smallemptycirc} &
  {\smallemptycirc} &
  {\smallemptycirc} &
  % -- &
  % -- &
  % -- &
  {\smallemptycirc} &
  {\smallfullcirc} &
  {\smallemptycirc} &
  % {\smallfullcirc} &
  {\smallfullcirc} &
  {\smallfullcirc} &
  {\smallfullcirc} &
  {\smallemptycirc} &
  {\smallfullcirc} &
  \cellcolor{g2}{$1$} &
\cellcolor{r0}{$0$} &

  
  % {\smallemptycirc} &
  % {\smallemptycirc} &
  % {\smallemptycirc} &
  % {\smallemptycirc} &
  % {\smallemptycirc} &
  % {\smallemptycirc} &
  % {\smallemptycirc} &
  % {\smallemptycirc} &
  % {\smallemptycirc} &
  % {\smallemptycirc} &
  % {\smallemptycirc} &
  % {\smallemptycirc} &
  % {\smallemptycirc} &
  % {\smallemptycirc} &
  % {\smallemptycirc} &
  % {\smallemptycirc} &
  % {\smallemptycirc} &
  % {\smallemptycirc} &
  % {\smallemptycirc} &
  % {\smallemptycirc} &
  % {\smallemptycirc} &
  % {\smallemptycirc} &
  % {\smallfullcirc} &
  % {\smallemptycirc} &
  % {\smallemptycirc} &
  % {\smallfullcirc} &
  % Memory and Cache Data Split 
  \cite{Akter2023AChallenges, Gupta2019ImpactSecurity} 
   \\
 &
\multirow{2}{*}{\hyperref[sec:impl-exp]{Direct Exploitation}}
&
 Weak Signature  \cite{Rokhjavan2023SecuringWallets}
&
   {\smallfullcirc} &
  % -- rng
  {\smallemptycirc} &
  % -- inadequ auth
  {\smallemptycirc} &
  % -- inadequ encry
  {\smallemptycirc} &
  % -- Appl. Logic Flaw
  {\smallemptycirc} &
  % -- Low-strength pwds
  {\smallemptycirc} &
  % -- Data Leakage
  {\smallemptycirc} &
  % -- Data Remanence
  {\smallemptycirc} &
  % -- Data Remanence
  {\smallemptycirc} &
  % -- Insec. Boot Environ.
  {\smallemptycirc} &
  % -- Micro-electr. Exposure
  {\smallfullcirc} &
  % -- Weak Signature
  {\smallfullcirc} &
  % -- Inadeq. Sig. Verif. 
  % -- SYSTEM
  {\smallemptycirc} &
  % -- Insec. Permissions
  {\smallfullcirc} &
  % -- Library Vulnerability
  {\smallemptycirc} &
  % -- OS Vulnerabilities
  {\smallfullcirc} &
  % -- Coding Errors
  {\smallemptycirc} &
  % -- Insec. Network
  %  -- SYSTEM
  {\smallemptycirc} &
  % -- Insec. User Interactions
  {\smallemptycirc} &
  % -- Comp. Provider
  % -- EXT
  {\smallemptycirc} &
  % -- Malicious Insider
  {\smallemptycirc} &
  % -- Insider Compromise
  % -- INSIDER
  {\smallemptycirc} &
  {\smallfullcirc} &
  {\smallemptycirc} &
  {\smallemptycirc} &
  {\smallemptycirc} &
  {\smallemptycirc} &
  {\smallemptycirc} &
  {\smallemptycirc} &
  {\smallemptycirc} &
  {\smallemptycirc} &
  {\smallemptycirc} &
  {\smallfullcirc} &
  {\smallemptycirc} &
  {\smallemptycirc} &
  % {\smallemptycirc} &
  {\smallemptycirc} &
  {\smallemptycirc} &
  % -- &
  % -- &
  % -- &
  {\smallemptycirc} &
  {\smallfullcirc} &
  {\smallemptycirc} &
  % {\smallfullcirc} &
  {\smallfullcirc} &
  {\smallfullcirc} &
  {\smallfullcirc} &
  {\smallemptycirc} &
  {\smallfullcirc} & 
  \cellcolor{g2}{$1$} &
\cellcolor{r0}{$0$} &

  
  % {\smallemptycirc} &
  % {\smallemptycirc} &
  % {\smallemptycirc} &
  % {\smallemptycirc} &
  % {\smallemptycirc} &
  % {\smallemptycirc} &
  % {\smallemptycirc} &
  % {\smallemptycirc} &
  % {\smallemptycirc} &
  % {\smallemptycirc} &
  % {\smallemptycirc} &
  % {\smallemptycirc} &
  % {\smallemptycirc} &
  % {\smallemptycirc} &
  % {\smallemptycirc} &
  % {\smallemptycirc} &
  % {\smallemptycirc} &
  % {\smallemptycirc} &
  % {\smallemptycirc} &
  % {\smallemptycirc} &
  % {\smallemptycirc} &
  % {\smallemptycirc} &
  % {\smallemptycirc} &
  % {\smallemptycirc} &
  % {\smallfullcirc} &
  % {\smallemptycirc}
  % Secure Cryptograpphic Schemes 
  \cite{brengel2018identifying}
   \\
 &
 &
 Nonce Reuse \cite{brengel2018identifying}
&
  % -- rng
  {\smallemptycirc} &
  % -- inadequ auth
  {\smallemptycirc} &
  % -- inadequ encry
  {\smallemptycirc} &
  % -- Appl. Logic Flaw
  {\smallemptycirc} &
  % -- Low-strength pwds
  {\smallemptycirc} &
  % -- Data Leakage
  {\smallemptycirc} &
  % -- Data Remanence
  {\smallemptycirc} &
  % -- Data Remanence
  {\smallemptycirc} &
  % -- Insec. Boot Environ.
  {\smallemptycirc} &
  % -- Micro-electr. Exposure
  {\smallemptycirc} &
  % -- Weak Signature
  {\smallemptycirc} &
  % -- Inadeq. Sig. Verif. 
  % -- SYSTEM
  {\smallemptycirc} &
  % -- Insec. Permissions
  {\smallemptycirc} &
  % -- Library Vulnerability
  {\smallemptycirc} &
  % -- OS Vulnerabilities
  {\smallemptycirc} &
  % -- Coding Errors
  {\smallemptycirc} &
  % -- Insec. Network
  %  -- SYSTEM
  {\smallemptycirc} &
  % -- Insec. User Interactions
  {\smallemptycirc} &
  % -- Comp. Provider
  % -- EXT
  {\smallemptycirc} &
  % -- Malicious Insider
  {\smallemptycirc} &
  % -- Insider Compromise
  % -- INSIDER
  {\smallemptycirc} &
  {\smallemptycirc} &
  {\smallemptycirc} &
  {\smallemptycirc} &
  {\smallemptycirc} &
  {\smallemptycirc} &
  {\smallemptycirc} &
  {\smallfullcirc} &
  % --
  {\smallemptycirc} &
  {\smallemptycirc} &
  {\smallemptycirc} &
  {\smallemptycirc} &
  {\smallfullcirc} &
  {\smallemptycirc} &
  % --
  {\smallemptycirc} &
  {\smallemptycirc} &
  {\smallemptycirc} &
  % --
  {\smallemptycirc} &
  {\smallfullcirc} &
  {\smallemptycirc} &
   % --
  % {\smallfullcirc} &
  {\smallfullcirc} &
  {\smallfullcirc} &
  {\smallfullcirc} &
  {\smallemptycirc} &
  {\smallfullcirc} &
  \cellcolor{g2}{$1$} &
\cellcolor{r0}{$0$} &
  %  % --
  % {\smallemptycirc} &
  % {\smallemptycirc} &
  % {\smallemptycirc} &
  % {\smallemptycirc} &
  % {\smallemptycirc} &
  % {\smallemptycirc} &
  % {\smallemptycirc} &
  % {\smallemptycirc} &
  %  % --
  % {\smallemptycirc} &
  % {\smallemptycirc} &
  %  % --
  % {\smallemptycirc} &
  % {\smallemptycirc} &
  % {\smallemptycirc} &
  % {\smallemptycirc} &
  %  % --
  % {\smallemptycirc} &
  % {\smallemptycirc} &
  % {\smallemptycirc} &
  % {\smallemptycirc} &
  % {\smallemptycirc} &
  %  % --
  % {\smallemptycirc} &
  % {\smallfullcirc} &
  % {\smallemptycirc} &
  % {\smallemptycirc} &
  % {\smallemptycirc} &
  % {\smallemptycirc} &
  % {\smallemptycirc}
  % Deterministic Nonce Selection 
  \cite{brengel2018identifying} 
   \\
\midrule
% \multicolumn{3}{c}{}  &
%   &
% \multicolumn{44}{r}{Undetailed  }  &
%   $0$($0\%$) &
% \cellcolor{r6}{$51$($62\%$)} &
%    \\
\multicolumn{2}{c}{Summary}  
  &
\multicolumn{1}{c}{28 Attack Vectors}  
  &
\multicolumn{36}{c}{}  
  &
 \multicolumn{9}{r}{Attack Vectors Occurrence  }  &
  \cellcolor{g6}{$24$($86\%$)} &
\cellcolor{g2}{$9$($32\%$)} &
% (\hyperref[tab:attack-incidents]{\cellcolor{r0}{$83$($100\%$)}}) 
% $29$ Methods
   \\
\bottomrule
\end{tabular}}
\end{table}
% % Figure environment removed
% \begin{table*}[!htbp]
\centering
\renewcommand{\arraystretch}{1.1}
\setlength{\tabcolsep}{1.25pt} % Adjust the column separation space here
\tiny
\begin{tabular}{llcccccccccccccccccccccccccccccccccccccccccccccccccccccccccccc}
\toprule
% \multicolumn{1}{c}{} &
  \multicolumn{1}{c}{\textbf{Name}} &
  \multicolumn{1}{c}{\textbf{{\hyperref[fig:wallet-evolution]{Est.}}}} &
  \multicolumn{3}{c}{\textbf{{\hyperref[sec:design-cust]{Cust.}}}} &
  \multicolumn{8}{c}{\textbf{{\hyperref[sec:infrastructure]{Infrastructure}}}} &
  \multicolumn{4}{c}{\textbf{{\hyperref[sec:design-init]{Init.}}}} &
  \multicolumn{3}{c}{\textbf{{\hyperref[sec:design-distr]{Distr.}}}} &
  \multicolumn{3}{c}{\textbf{{\hyperref[sec:design-author]{Authoris.}}}} &
  \multicolumn{3}{c}{\textbf{{\hyperref[sec:design-val]{Valid.}}}} &
  \multicolumn{5}{c}{\textbf{{\hyperref[sec:design-authen]{Authentication}}}} &
  \multicolumn{4}{c}{\textbf{{\hyperref[sec:design-rec]{Recovery}}}} &
  \multicolumn{2}{c}{\textbf{{\hyperref[sec:design-rec]{Trans.}}}} &
  \multicolumn{9}{c}{\textbf{{\hyperref[sec:design-rec]{Agnosticism}}}} &
  \multicolumn{15}{c}{\textbf{{\hyperref[sec:threat_framework]{Threat Occurrences}}}} 
  % \multicolumn{2}{c}{\textbf{{\hyperref[sec:attack-framework]{Atk.}}}} &
  \\ 
  \cmidrule(lr){6-13} \cmidrule(lr){14-17} 
  \cmidrule(lr){18-20} \cmidrule(lr){21-23} \cmidrule(lr){24-26} \cmidrule(lr){27-31} \cmidrule(lr){32-35} \cmidrule(lr){36-37} \cmidrule(lr){38-46} \cmidrule(lr){47-61}
  % \multicolumn{1}{c}{} &
  \multicolumn{1}{c}{} &
  \multicolumn{1}{c}{} &
  \multicolumn{3}{c}{} &
  \multicolumn{4}{c}{\textbf{Software}} &
  \multicolumn{4}{c}{\textbf{Hardware}} &
  \multicolumn{3}{c}{\textbf{}} &
  \multicolumn{1}{c}{\textbf{}} &
  \multicolumn{1}{c}{\textbf{}} &
    % \multicolumn{1}{c}{\textbf{Sgl.}} &
  \multicolumn{2}{c}{\textbf{}} &
    % \multicolumn{2}{c}{\textbf{Multi.}} &
  \multicolumn{2}{c}{\textbf{}} &
    % \multicolumn{2}{c}{\textbf{User}} &
  \multicolumn{1}{c}{\textbf{}} &
    % \multicolumn{1}{c}{\textbf{RL}} &
  \multicolumn{3}{c}{} &
  \multicolumn{5}{c}{} &
  \multicolumn{4}{c}{} &
  \multicolumn{2}{c}{} &
  \multicolumn{9}{c}{} &
  \multicolumn{15}{c}{} &
  % \rotatebox[origin=l]{90}{\cellcolor{r6}{$0\%$}} &
  % \rotatebox[origin=l]{90}{\cellcolor{r4}{$0\%$}} &
  % \rotatebox[origin=l]{90}{\cellcolor{r1}{$0\%$}} &
  % \rotatebox[origin=l]{90}{\cellcolor{r2}{$0\%$}} &
  % \rotatebox[origin=l]{90}{\cellcolor{r5}{$0\%$}} &
  % \rotatebox[origin=l]{90}{\cellcolor{r3}{$0\%$}} &
  % \rotatebox[origin=l]{90}{\cellcolor{r2}{$0\%$}} &
  % \rotatebox[origin=l]{90}{\cellcolor{r4}{$0\%$}} &
  % \rotatebox[origin=l]{90}{\cellcolor{r1}{$0\%$}} &
  % \rotatebox[origin=l]{90}{\cellcolor{r2}{$0\%$}} &
  % \rotatebox[origin=l]{90}{\cellcolor{r3}{$0\%$}} &
  % \rotatebox[origin=l]{90}{\cellcolor{r3}{$0\%$}} &
  % \rotatebox[origin=l]{90}{\cellcolor{r5}{$0\%$}} &
  % \rotatebox[origin=l]{90}{\cellcolor{r2}{$0\%$}} &
  % \rotatebox[origin=l]{90}{\cellcolor{r4}{$0\%$}} &
  \multicolumn{1}{c}{} 
  
  \\
  \cmidrule(lr){6-9} \cmidrule(lr){10-13} 
  % \cmidrule(lr){19-19} \cmidrule(lr){20-21}
 %  \multicolumn{1}{c}{\multirow{-3}{*}{\rotatebox[origin=l]{90}{\textbf{}}}}
 % &
   &
   \multicolumn{1}{c}{} &
   \rotatebox[origin=l]{90}{Non-Custodial} &
  \rotatebox[origin=l]{90}{Shared-Custodial} &
  \rotatebox[origin=l]{90}{Custodial} &
  \rotatebox[origin=l]{90}{Desktop} &
  \rotatebox[origin=l]{90}{Browser} &
  \rotatebox[origin=l]{90}{Mobile} &
  \rotatebox[origin=l]{90}{Smart} &
  \rotatebox[origin=l]{90}{USB} &
  \rotatebox[origin=l]{90}{Bluetooth} &
  \rotatebox[origin=l]{90}{NFC} &
  \rotatebox[origin=l]{90}{QR Code} &
  \rotatebox[origin=l]{90}{Non-Deterministic} &
  \rotatebox[origin=l]{90}{Deterministic (Non-HD)} &
  \rotatebox[origin=l]{90}{\acf{hd}} &
   \rotatebox[origin=l]{90}{Account Contract} &
  \rotatebox[origin=l]{90}{Single Distributed} &
  \rotatebox[origin=l]{90}{Multi-Sig} &
  \rotatebox[origin=l]{90}{\acf{mpc}} &
  \rotatebox[origin=l]{90}{Single SK} &
  \rotatebox[origin=l]{90}{Multiple SK} &
  \rotatebox[origin=l]{90}{Relayer} &
  \rotatebox[origin=l]{90}{Single PK Validation} &
  \rotatebox[origin=l]{90}{Multiple PK Validation} &
  \rotatebox[origin=l]{90}{Contract Validation} &
  \rotatebox[origin=l]{90}{PW/PIN} &
  \rotatebox[origin=l]{90}{2FA} &
  \rotatebox[origin=l]{90}{U2F} &
  \rotatebox[origin=l]{90}{Passkey} &
  \rotatebox[origin=l]{90}{Biometric} &
  \rotatebox[origin=l]{90}{12W Seed} &
  \rotatebox[origin=l]{90}{24W Seed} &
  \rotatebox[origin=l]{90}{Social} &
  \rotatebox[origin=l]{90}{DeRec} &
  \rotatebox[origin=l]{90}{Open-Source} &
  \rotatebox[origin=l]{90}{Closed-Source} &
  \rotatebox[origin=l]{90}{BTC} &
  \rotatebox[origin=l]{90}{ETH} &
  \rotatebox[origin=l]{90}{POLY} &
  \rotatebox[origin=l]{90}{BNB} &
  \rotatebox[origin=l]{90}{XRP} &
  \rotatebox[origin=l]{90}{HBAR} &
  \rotatebox[origin=l]{90}{SOL} &
  \rotatebox[origin=l]{90}{ADA} &
  \rotatebox[origin=l]{90}{AVAX} &
  \rotatebox[origin=l]{90}{Inadequate Encryption \cite{cve_15947, cve_37192}} &
  \rotatebox[origin=l]{90}{Insecure Network \cite{cve_33297, cve_14198, cve_17144}} &
  \rotatebox[origin=l]{90}{Library Vulnerability \cite{bitcore_lib, Ledger2023SecurityReport} } &
  \rotatebox[origin=l]{90}{Insecure Permission \cite{cve_32969, halborn_vuln}} &
  \rotatebox[origin=l]{90}{Predictable RNG \cite{cve_31290, cve_23660}} &
  % cve_14199,  tymokhanov2021alpha, fireblocks_23, chainlight
  % \cite{fireblocks_23, chainlight}}
  \rotatebox[origin=l]{90}{Sig. Verif. Logic Flaw \cite{cve_14199, fireblocks_23, AccountMedium, UncoveringVulnerability}} &
  \rotatebox[origin=l]{90}{Side-channel Leakage \cite{cve_14353, cve_14354, KrakenBlog}} &
  \rotatebox[origin=l]{90}{Data Remanence \cite{trezor_memory, trezor_medium}} &
  \rotatebox[origin=l]{90}{Data Manipulation \cite{trezor_memory, trezor_medium}} &
  \rotatebox[origin=l]{90}{Insecure Interactions \cite{ZengoZengo, thodex}} &
  \rotatebox[origin=l]{90}{Inadequate Authentication \cite{open_zeppelin}} &
  \rotatebox[origin=l]{90}{Input Validation Logic Flaw \cite{immunefi}} &
  \rotatebox[origin=l]{90}{Recovery Logic Flaw \cite{cve_15302}} &
  \multicolumn{1}{c}{\rotatebox[origin=l]{90}{Provider Compromise \cite{CoinTelegraph2022SlopeAttack}}} &
  \multicolumn{1}{c}{\rotatebox[origin=l]{90}{Insider Compromise \cite{Ledger2023SecurityReport}}} &
  % \# (\& \%)
  \multicolumn{1}{c}{\rotatebox[origin=l]{90}{Threat \# (\& \%)}} 
  % &
  % \multicolumn{1}{c}{\rotatebox[origin=l]{90}{Attacks \# (\& \%)}}
   \\
\midrule
% \multirow{19}{*}{\rotatebox[origin=l]{90}{Non-Custodial}} 
% & 
Bitcoin Core & 2009 & {\fullcirc} & {\emptycirc} & {\emptycirc} & {\fullcirc} & {\emptycirc} & {\emptycirc} & {\emptycirc} & {\emptycirc} & {\emptycirc} & {\emptycirc} & {\emptycirc} & {\fullcirc} & {\emptycirc} & {\fullcirc} & {\emptycirc} & {\fullcirc} & {\emptycirc} & {\emptycirc} & {\fullcirc} & {\emptycirc} & {\emptycirc}  & {\fullcirc} & {\emptycirc} & {\emptycirc} & {\fullcirc} & {\emptycirc} & {\emptycirc} & {\emptycirc} & {\emptycirc} & {\emptycirc} & {\emptycirc} & {\emptycirc} & {\emptycirc} & {\fullcirc} & {\emptycirc} & {\fullcirc} & {\emptycirc} & {\emptycirc} & {\emptycirc} & {\emptycirc} & {\emptycirc} & {\emptycirc} & {\emptycirc} & {\emptycirc} & {\fullcirc} & {\fullcirc} & {\fullcirc} & {\emptycirc} & {\emptycirc} & {\emptycirc} & {\emptycirc} & {\emptycirc} & {\emptycirc} & {\emptycirc} & {\emptycirc} & {\emptycirc} & {\emptycirc} & {\emptycirc} & {\emptycirc} & \cellcolor{o3}{$3$($20\%$)}

% &  \cellcolor{r6}{$0\%$}   
\\ 
% \cellcolor{g6}{$21$($49\%$)}
Electrum & 2011 & {\fullcirc} & {\emptycirc} & {\emptycirc} & {\fullcirc} & {\emptycirc} & {\emptycirc} & {\emptycirc} & {\emptycirc} & {\emptycirc} & {\emptycirc} & {\emptycirc} & {\fullcirc} & {\emptycirc} & {\fullcirc} & {\emptycirc} & {\fullcirc} & {\fullcirc} & {\emptycirc}  & {\fullcirc} & {\fullcirc} & {\emptycirc} & {\fullcirc} & {\fullcirc} & {\emptycirc} & {\fullcirc} & {\fullcirc} & {\emptycirc} & {\emptycirc} & {\emptycirc} & {\fullcirc} & {\emptycirc} & {\emptycirc} & {\emptycirc} & {\fullcirc} & {\emptycirc} & {\fullcirc} & {\emptycirc} & {\emptycirc} & {\emptycirc} & {\emptycirc} & {\emptycirc} & {\emptycirc} & {\emptycirc} & {\emptycirc} & {\emptycirc} & {\emptycirc} & {\emptycirc} & {\emptycirc} & {\emptycirc} & {\emptycirc} & {\emptycirc} & {\emptycirc} & {\emptycirc} & {\emptycirc} & {\emptycirc} & {\fullcirc} & {\emptycirc} & {\emptycirc} & {\emptycirc} & \cellcolor{o0}{$1$($7\%$)} 
% & \cellcolor{r2}{$0\%$}  
\\ 
Coinbase Ex. & 2012  & {\emptycirc} & {\emptycirc} & {\fullcirc} & {\emptycirc} & {\fullcirc} & {\fullcirc} & {\emptycirc} & {\emptycirc} & {\emptycirc} & {\emptycirc} & {\emptycirc} & {\emptycirc} & {\emptycirc} & {\emptycirc} & {\emptycirc} & {\emptycirc} & {\emptycirc} & {\emptycirc} & {\emptycirc} & {\emptycirc} & {\emptycirc} & {\emptycirc} & {\emptycirc} & {\emptycirc} & {\emptycirc} & {\emptycirc} & {\emptycirc} & {\emptycirc} & {\emptycirc} & {\emptycirc} & {\emptycirc} & {\emptycirc} & {\emptycirc} & {\emptycirc} & {\fullcirc} & {\fullcirc} & {\fullcirc} & {\fullcirc} & {\emptycirc} & {\fullcirc} & {\fullcirc} & {\fullcirc} & {\fullcirc} & {\fullcirc} & {\emptycirc} & {\emptycirc} & {\emptycirc} & {\emptycirc} & {\emptycirc} & {\emptycirc} & {\emptycirc} & {\emptycirc} & {\emptycirc} & {\emptycirc} & {\emptycirc} & {\emptycirc} & {\emptycirc} & {\emptycirc} & {\emptycirc} & $0$($0\%$)
% & \cellcolor{r0}{$0\%$}  
\\ 
% & 8.8M m*
% found out Trezor has multi-sig - i.e 2-of-3 need to reconfirm if it is 2 hardware devices or if there is a smart contract element
Trezor  & 2013 & {\fullcirc} & {\emptycirc} & {\emptycirc} & {\emptycirc} & {\emptycirc} & {\emptycirc} & {\emptycirc} & {\fullcirc} & {\emptycirc} & {\emptycirc} & {\emptycirc} & {\emptycirc} & {\emptycirc} & {\fullcirc} & {\emptycirc} & {\fullcirc} & {\fullcirc} & {\emptycirc} & {\fullcirc} & {\fullcirc} & {\emptycirc} & {\fullcirc} & {\fullcirc} & {\emptycirc} & {\fullcirc} & {\emptycirc} & {\fullcirc} & {\emptycirc} & {\emptycirc} & {\fullcirc} & {\fullcirc} & {\emptycirc} & {\emptycirc} & {\fullcirc} & {\emptycirc} & {\fullcirc} & {\fullcirc} & {\fullcirc} & {\fullcirc} & {\fullcirc} & {\emptycirc} & {\fullcirc} & {\fullcirc} & {\fullcirc} & {\emptycirc} & {\emptycirc} & {\emptycirc} & {\emptycirc} & {\emptycirc} & {\fullcirc} & {\fullcirc} & {\fullcirc} & {\fullcirc} & {\fullcirc} & {\emptycirc} & {\emptycirc} & {\emptycirc} & {\emptycirc} & {\emptycirc} & \cellcolor{o5}{$5$($33\%$})
% & \cellcolor{r4}{$0\%$}    
\\ 
% & 4
% & 2M
eToro & 2013 & {\emptycirc} & {\emptycirc} & {\fullcirc} & {\emptycirc} & {\fullcirc} & {\fullcirc} & {\emptycirc} & {\emptycirc} & {\emptycirc} & {\emptycirc} & {\emptycirc} & {\emptycirc} & {\emptycirc} & {\emptycirc} & {\emptycirc} & {\emptycirc} & {\emptycirc} & {\emptycirc} & {\emptycirc}  & {\emptycirc} & {\emptycirc} & {\emptycirc} & {\emptycirc} & {\emptycirc} & {\emptycirc} & {\emptycirc} & {\emptycirc} & {\emptycirc} & {\emptycirc} & {\emptycirc} & {\emptycirc} & {\emptycirc} & {\emptycirc} & {\emptycirc} & {\fullcirc} & {\fullcirc} & {\fullcirc} & {\fullcirc} & {\fullcirc} & {\fullcirc} & {\fullcirc} & {\fullcirc} & {\fullcirc} & {\fullcirc} & {\emptycirc} & {\emptycirc} & {\emptycirc} & {\emptycirc} & {\emptycirc} & {\emptycirc} & {\emptycirc} & {\emptycirc} & {\emptycirc} & {\emptycirc} & {\emptycirc} & {\emptycirc} & {\emptycirc} & {\emptycirc} & {\emptycirc} & $0$($0\%$)
% & \cellcolor{r2}{$0\%$}  
\\ 
% & 33M
Kraken Ex. & 2013 & {\emptycirc} & {\emptycirc} & {\fullcirc} & {\emptycirc} & {\fullcirc} & {\fullcirc} & {\emptycirc} & {\emptycirc} & {\emptycirc} & {\emptycirc} & {\emptycirc} & {\emptycirc}  & {\emptycirc} & {\emptycirc} & {\emptycirc} & {\emptycirc} & {\emptycirc}  & {\emptycirc} & {\emptycirc} & {\emptycirc} & {\emptycirc} & {\emptycirc} & {\emptycirc} & {\emptycirc} & {\emptycirc} & {\emptycirc} & {\emptycirc} & {\emptycirc} & {\emptycirc} & {\emptycirc} & {\emptycirc} & {\emptycirc} & {\emptycirc} & {\emptycirc} & {\fullcirc} & {\fullcirc} & {\fullcirc} & {\fullcirc} & {\emptycirc} & {\fullcirc} & {\emptycirc} & {\fullcirc} & {\fullcirc} & {\fullcirc} & {\emptycirc} & {\emptycirc} & {\emptycirc} & {\emptycirc} & {\emptycirc} & {\emptycirc} & {\emptycirc} & {\emptycirc} & {\emptycirc} & {\emptycirc} & {\emptycirc} & {\emptycirc} & {\emptycirc} & {\emptycirc} & {\emptycirc} & {$0$($0\%$)} 
% & \cellcolor{r3}{$0\%$}  
\\ 
Ledger & 2014 & {\fullcirc} & {\emptycirc} & {\emptycirc} & {\emptycirc} & {\emptycirc} & {\emptycirc} & {\emptycirc} & {\fullcirc} & {\fullcirc} & {\emptycirc} & {\emptycirc} & {\emptycirc} & {\emptycirc} & {\fullcirc} & {\emptycirc} & {\fullcirc} & {\emptycirc} & {\emptycirc} & {\fullcirc} & {\emptycirc} & {\emptycirc} & {\fullcirc} & {\emptycirc} & {\emptycirc} & {\fullcirc} & {\emptycirc} & {\fullcirc} & {\emptycirc} & {\emptycirc} & {\emptycirc} & {\fullcirc} & {\emptycirc} & {\emptycirc} & {\halfcirc} & {\emptycirc} & {\fullcirc} & {\fullcirc} & {\fullcirc} & {\fullcirc} & {\fullcirc} & {\fullcirc} & {\fullcirc} & {\fullcirc} & {\fullcirc} & {\emptycirc} & {\emptycirc} & {\fullcirc} & {\emptycirc} & {\emptycirc} & {\emptycirc} & {\fullcirc} & {\emptycirc} & {\emptycirc} & {\fullcirc} & {\emptycirc} & {\emptycirc} & {\emptycirc} & {\emptycirc} & {\fullcirc} & \cellcolor{o4}{$4$($27\%$)}
% & \cellcolor{r6}{$0\%$}  
\\ 
% & 6M
% & software open source - firmware closed source
Gemini & 2014 & {\emptycirc} & {\emptycirc} & {\fullcirc} & {\emptycirc} & {\fullcirc} & {\fullcirc} & {\emptycirc} & {\emptycirc} & {\emptycirc} & {\emptycirc} & {\emptycirc} & {\emptycirc} & {\emptycirc} & {\emptycirc} & {\emptycirc} & {\emptycirc} & {\emptycirc} & {\emptycirc} & {\emptycirc} & {\emptycirc} & {\emptycirc} & {\emptycirc} & {\emptycirc} & {\emptycirc} & {\emptycirc} & {\emptycirc} & {\emptycirc} & {\emptycirc} & {\emptycirc} & {\emptycirc} & {\emptycirc} & {\emptycirc} & {\emptycirc} & {\emptycirc} & {\fullcirc} & {\fullcirc} & {\fullcirc} & {\fullcirc} & {\emptycirc} & {\fullcirc} & {\emptycirc} & {\fullcirc} & {\emptycirc} & {\fullcirc} & {\emptycirc} & {\emptycirc} & {\emptycirc} & {\emptycirc} & {\emptycirc} & {\emptycirc} & {\emptycirc} & {\emptycirc} & {\emptycirc} & {\emptycirc} & {\emptycirc} & {\emptycirc} & {\emptycirc} & {\emptycirc} & {\emptycirc} & $0$($0\%$)
% & \cellcolor{r3}{$0\%$}  
\\
Metamask & 2016 & {\fullcirc} & {\emptycirc} & {\emptycirc} & {\emptycirc} & {\fullcirc} & {\fullcirc} & {\emptycirc} & {\emptycirc} & {\emptycirc} & {\emptycirc} & {\emptycirc} & {\emptycirc} & {\emptycirc} & {\fullcirc} & {\emptycirc} & {\fullcirc} & {\emptycirc} & {\emptycirc} & {\fullcirc} & {\emptycirc} & {\emptycirc} & {\fullcirc} & {\emptycirc} & {\emptycirc} & {\fullcirc} & {\emptycirc} & {\emptycirc} & {\emptycirc} & {\fullcirc} & {\fullcirc} & {\emptycirc} & {\emptycirc} & {\emptycirc} & {\fullcirc} & {\emptycirc} & {\emptycirc} & {\fullcirc} & {\fullcirc} & {\fullcirc} & {\emptycirc} & {\fullcirc} & {\emptycirc} & {\emptycirc} & {\fullcirc} & {\emptycirc} & {\emptycirc} & {\emptycirc} & {\fullcirc} & {\emptycirc} & {\emptycirc} & {\emptycirc} & {\emptycirc} & {\emptycirc} & {\emptycirc} & {\emptycirc} & {\emptycirc} & {\emptycirc} & {\emptycirc} & {\emptycirc} & \cellcolor{o0}{$1$($7\%$}) 
% & \cellcolor{r1}{$0\%$}  
\\ 
% & 30M m*
Bitbuy &  2016 & {\emptycirc} & {\emptycirc} & {\fullcirc} & {\emptycirc} & {\fullcirc} & {\fullcirc} & {\emptycirc} & {\emptycirc} & {\emptycirc} & {\emptycirc} & {\emptycirc} & {\emptycirc} & {\emptycirc} & {\emptycirc} & {\emptycirc} & {\emptycirc} & {\emptycirc} & {\emptycirc} & {\emptycirc} & {\emptycirc} & {\emptycirc} & {\emptycirc} & {\emptycirc} & {\emptycirc} & {\emptycirc} & {\emptycirc} & {\emptycirc} & {\emptycirc} & {\emptycirc} & {\emptycirc} & {\emptycirc} & {\emptycirc} & {\emptycirc} & {\emptycirc} & {\fullcirc} & {\fullcirc} & {\fullcirc} & {\fullcirc} & {\emptycirc} & {\fullcirc} & {\fullcirc} & {\fullcirc} & {\fullcirc} & {\fullcirc} & {\emptycirc} & {\emptycirc} & {\emptycirc} & {\emptycirc} & {\emptycirc} & {\emptycirc} & {\emptycirc} & {\emptycirc} & {\emptycirc} & {\emptycirc} & {\emptycirc} & {\emptycirc} & {\emptycirc} & {\emptycirc} & {\emptycirc} & $0$($0\%$)
% & \cellcolor{r3}{$0\%$}  
\\ 
% & 0.45M
Exodus & 2016 & {\fullcirc} & {\emptycirc} & {\emptycirc} & {\fullcirc} & {\fullcirc} & {\fullcirc} & {\emptycirc} & {\emptycirc} & {\emptycirc} & {\emptycirc} & {\emptycirc} & {\emptycirc} & {\emptycirc} & {\fullcirc} & {\emptycirc} & {\fullcirc} & {\emptycirc} & {\fullcirc} & {\fullcirc} & {\emptycirc} & {\emptycirc} & {\fullcirc} & {\emptycirc} & {\emptycirc} & {\fullcirc} & {\emptycirc} & {\emptycirc} & {\fullcirc} & {\fullcirc} & {\fullcirc} & {\emptycirc} & {\emptycirc} & {\emptycirc} & {\emptycirc} & {\fullcirc} & {\fullcirc} & {\fullcirc} & {\fullcirc} & {\fullcirc} & {\fullcirc} & {\fullcirc} & {\fullcirc} & {\fullcirc} & {\fullcirc} & {\emptycirc} & {\emptycirc} & {\emptycirc} & {\emptycirc} & {\emptycirc} & {\emptycirc} & {\emptycirc} &  {\emptycirc} & {\emptycirc} & {\fullcirc} & {\emptycirc} & {\emptycirc} & {\emptycirc} & {\emptycirc} & {\emptycirc} & \cellcolor{o0}{$1$($7\%$)} 
% & \cellcolor{r5}{$0\%$}   
\\ 
% & 0.8M m*
Binance Ex. & 2017 & {\emptycirc} & {\emptycirc} & {\fullcirc} & {\fullcirc} & {\fullcirc} & {\fullcirc} & {\emptycirc} & {\emptycirc} & {\emptycirc} & {\emptycirc} & {\emptycirc} & {\emptycirc} & {\emptycirc} & {\emptycirc} & {\emptycirc} & {\emptycirc} & {\emptycirc} & {\emptycirc} & {\emptycirc} & {\emptycirc} & {\emptycirc} & {\emptycirc} & {\emptycirc} & {\emptycirc} & {\emptycirc} & {\emptycirc} & {\emptycirc} & {\emptycirc} & {\emptycirc} & {\emptycirc} & {\emptycirc} & {\emptycirc} & {\emptycirc} & {\emptycirc} & {\fullcirc} & {\fullcirc} & {\fullcirc} & {\fullcirc} & {\fullcirc} & {\fullcirc} & {\fullcirc} & {\fullcirc} & {\fullcirc} & {\fullcirc} & {\emptycirc} & {\emptycirc} & {\emptycirc} & {\emptycirc} & {\emptycirc} & {\emptycirc} & {\emptycirc} & {\emptycirc} & {\emptycirc} & {\emptycirc} & {\emptycirc} & {\emptycirc} & {\emptycirc} & {\emptycirc} & {\emptycirc} & $0$($0\%$))
% & \cellcolor{r2}{$0\%$}  
\\ 
% & 200M
Trust Wlt. & 2017 & {\fullcirc} & {\emptycirc} & {\emptycirc} & {\emptycirc} & {\fullcirc} & {\fullcirc} & {\emptycirc} & {\emptycirc} & {\emptycirc} & {\emptycirc} & {\emptycirc} & {\emptycirc} & {\emptycirc} & {\fullcirc} & {\emptycirc} & {\fullcirc} & {\emptycirc} & {\halfcirc} & {\fullcirc} & {\emptycirc} & {\emptycirc} & {\fullcirc} & {\emptycirc} & {\emptycirc} & {\fullcirc} & {\emptycirc} & {\emptycirc} & {\emptycirc} & {\fullcirc} & {\fullcirc} & {\emptycirc} & {\emptycirc} & {\emptycirc}  & {\fullcirc} & {\emptycirc} & {\fullcirc} & {\fullcirc} & {\fullcirc} & {\fullcirc} & {\fullcirc} & {\emptycirc} & {\fullcirc} & {\fullcirc} & {\fullcirc} & {\emptycirc} & {\emptycirc} & {\emptycirc} & {\emptycirc} & {\fullcirc} & {\emptycirc} & {\emptycirc} & {\emptycirc} & {\emptycirc} &  {\emptycirc} & {\emptycirc} & {\emptycirc} & {\emptycirc} & {\emptycirc} & {\emptycirc} & \cellcolor{o0}{$1$($7\%$)} 
% & \cellcolor{r1}{$0\%$}  
\\ 
% & 2
% & 130M
Argent & 2017 & {\fullcirc} & {\emptycirc} & {\emptycirc} & {\emptycirc} & {\fullcirc} & {\fullcirc} & {\fullcirc} & {\emptycirc} & {\emptycirc} & {\emptycirc} & {\emptycirc} & {\emptycirc} & {\fullcirc} & {\emptycirc} & {\fullcirc} & {\emptycirc} & {\fullcirc} & {\emptycirc} & {\emptycirc} & {\fullcirc} & {\fullcirc} & {\emptycirc} & {\emptycirc} & {\fullcirc} & {\emptycirc} & {\emptycirc} & {\emptycirc} & {\fullcirc} & {\emptycirc} & {\emptycirc} & {\emptycirc} & {\fullcirc} & {\emptycirc} & {\fullcirc} & {\emptycirc} & {\emptycirc} & {\fullcirc} & {\fullcirc} & {\emptycirc} & {\emptycirc} & {\emptycirc} & {\emptycirc} & {\emptycirc} & {\emptycirc} & {\emptycirc} & {\emptycirc} & {\emptycirc} & {\emptycirc} & {\emptycirc} & {\fullcirc} & {\emptycirc} & {\emptycirc} & {\emptycirc} & {\emptycirc} & {\emptycirc} & {\emptycirc} & {\fullcirc} & {\emptycirc} & {\emptycirc} & \cellcolor{o2}{$2$($13\%$)} 
% & \cellcolor{r2}{$0\%$}   
\\ 
CoinEx & 2017 & {\emptycirc} & {\emptycirc} & {\fullcirc} & {\emptycirc} & {\fullcirc} & {\fullcirc} & {\emptycirc} & {\emptycirc} & {\emptycirc} & {\emptycirc} & {\emptycirc} & {\emptycirc} & {\emptycirc} & {\emptycirc} & {\emptycirc} & {\emptycirc} & {\emptycirc} & {\emptycirc} & {\emptycirc} & {\emptycirc} & {\emptycirc} & {\emptycirc} & {\emptycirc} & {\emptycirc} & {\emptycirc} & {\emptycirc} & {\emptycirc} & {\emptycirc} & {\emptycirc} & {\emptycirc} & {\emptycirc} & {\emptycirc} & {\emptycirc} & {\emptycirc} & {\fullcirc} & {\fullcirc} & {\fullcirc} & {\fullcirc} & {\fullcirc} & {\fullcirc} & {\fullcirc} & {\fullcirc} & {\fullcirc} & {\fullcirc} & {\emptycirc} & {\emptycirc} & {\emptycirc} & {\emptycirc} & {\emptycirc} & {\emptycirc} & {\emptycirc} & {\emptycirc} & {\emptycirc} & {\emptycirc} & {\emptycirc} & {\emptycirc} & {\emptycirc} & {\emptycirc} & {\emptycirc} & $0$($0\%$))
% & \cellcolor{r2}{$0\%$}  
\\ 
% \FilledCircle
 % & 5M 
Safe (Gnosis) & 2017 & {\fullcirc} & {\emptycirc} & {\emptycirc} & {\emptycirc} & {\emptycirc} & {\fullcirc} & {\fullcirc} & {\emptycirc} & {\emptycirc} & {\emptycirc} & {\emptycirc} & {\emptycirc} & {\fullcirc} & {\emptycirc} & {\fullcirc} & {\emptycirc} & {\fullcirc} & {\emptycirc} & {\emptycirc} & {\fullcirc} & {\fullcirc} & {\emptycirc} & {\emptycirc} & {\fullcirc} & {\emptycirc} & {\emptycirc} & {\emptycirc} & {\fullcirc} & {\emptycirc} & {\emptycirc} & {\emptycirc} & {\fullcirc} & {\emptycirc} &  {\fullcirc} & {\emptycirc} & {\emptycirc} & {\fullcirc} & {\emptycirc} & {\emptycirc} & {\emptycirc} & {\emptycirc} & {\emptycirc} & {\emptycirc} & {\emptycirc} & {\emptycirc} & {\emptycirc} & {\emptycirc} & {\emptycirc} & {\emptycirc} & {\fullcirc} & {\emptycirc} & {\emptycirc} & {\emptycirc} & {\emptycirc} & {\fullcirc} & {\emptycirc} & {\emptycirc} & {\emptycirc} & {\emptycirc} & \cellcolor{o2}{$2$($13\%$)} 
% & \cellcolor{r2}{$0\%$}   
\\ 
% & 1.6M m*
Atomic & 2017 & {\fullcirc} & {\emptycirc} & {\emptycirc} & {\fullcirc} & {\emptycirc} & {\fullcirc} & {\emptycirc} & {\emptycirc} & {\emptycirc} & {\emptycirc} & {\emptycirc} & {\emptycirc} & {\emptycirc} & {\emptycirc} & {\fullcirc} & {\fullcirc} & {\emptycirc} & {\emptycirc} & {\fullcirc} & {\emptycirc} & {\emptycirc} & {\fullcirc} & {\emptycirc} & {\emptycirc} & {\fullcirc} & {\emptycirc} & {\emptycirc} & {\emptycirc} & {\emptycirc} & {\fullcirc} & {\emptycirc} & {\emptycirc} & {\emptycirc} & {\emptycirc} & {\fullcirc} & {\fullcirc} & {\fullcirc} & {\fullcirc} & {\fullcirc} & {\fullcirc} & {\fullcirc} & {\fullcirc} & {\fullcirc} & {\fullcirc} & {\emptycirc} & {\emptycirc} & {\emptycirc} & {\emptycirc} & {\fullcirc} & {\fullcirc} & {\emptycirc} & {\emptycirc} & {\emptycirc} & {\emptycirc} & {\emptycirc} & {\emptycirc} &  {\emptycirc} & {\emptycirc} & {\emptycirc} & \cellcolor{o2}{$2$($13\%$)} 
% & \cellcolor{r3}{$0\%$}  
\\
% & 10M
Tangem & 2017 & {\fullcirc} & {\emptycirc} & {\emptycirc} & {\emptycirc} & {\emptycirc} & {\emptycirc} & {\emptycirc} & {\emptycirc} & {\emptycirc} & {\fullcirc} & {\emptycirc} & {\emptycirc} & {\emptycirc} & {\fullcirc} & {\emptycirc} & {\fullcirc} & {\emptycirc} & {\emptycirc} & {\fullcirc} & {\emptycirc} & {\emptycirc} & {\fullcirc} & {\emptycirc} & {\emptycirc} & {\fullcirc} & {\emptycirc} & {\emptycirc} & {\emptycirc} & {\fullcirc} & {\fullcirc} & {\fullcirc} & {\emptycirc} & {\emptycirc} & {\fullcirc} & {\emptycirc} & {\fullcirc} & {\fullcirc} & {\emptycirc} & {\fullcirc} & {\fullcirc} & {\emptycirc} & {\fullcirc} & {\emptycirc} & {\fullcirc} & {\emptycirc} & {\emptycirc} & {\emptycirc} & {\emptycirc} & {\emptycirc} & {\emptycirc} & {\emptycirc} & {\emptycirc} & {\emptycirc} & {\emptycirc} & {\emptycirc} & {\emptycirc} & {\emptycirc} & {\emptycirc} & {\emptycirc} & $0$($0\%$)
% & \cellcolor{r0}{$0\%$}  
\\
Ngrave & 2018 & {\fullcirc} & {\emptycirc} & {\emptycirc} & {\emptycirc} & {\emptycirc} & {\emptycirc} & {\emptycirc} & {\emptycirc} & {\emptycirc} & {\emptycirc} & {\fullcirc} & {\emptycirc} & {\emptycirc} & {\fullcirc} & {\emptycirc} & {\fullcirc} & {\emptycirc} & {\emptycirc} & {\fullcirc} & {\emptycirc} & {\emptycirc} & {\fullcirc} & {\emptycirc} & {\emptycirc} & {\fullcirc} & {\emptycirc} & {\emptycirc} & {\emptycirc} & {\fullcirc} & {\emptycirc} & {\fullcirc} & {\emptycirc} & {\emptycirc} & {\emptycirc} & {\fullcirc} & {\fullcirc} & {\fullcirc} & {\emptycirc} & {\fullcirc} & {\fullcirc} & {\emptycirc} & {\fullcirc} & {\emptycirc} & {\fullcirc} & {\emptycirc} & {\emptycirc} & {\emptycirc} & {\emptycirc} & {\emptycirc} & {\emptycirc} & {\emptycirc} & {\emptycirc} & {\emptycirc} & {\emptycirc} & {\emptycirc} & {\emptycirc} & {\emptycirc} & {\emptycirc} & {\emptycirc} & $0$($0\%$)
% & \cellcolor{r0}{$0\%$}   
\\ 
Zengo & 2018 & {\emptycirc} & {\fullcirc} & {\emptycirc} & {\emptycirc} & {\emptycirc} & {\fullcirc} & {\emptycirc} & {\emptycirc} & {\emptycirc} & {\emptycirc} & {\emptycirc} & {\emptycirc} & {\fullcirc} & {\emptycirc} & {\fullcirc} & {\emptycirc} & {\emptycirc} & {\fullcirc} & {\fullcirc} & {\emptycirc} & {\emptycirc} & {\fullcirc} & {\emptycirc} & {\emptycirc} & {\emptycirc} & {\fullcirc} & {\emptycirc} & {\emptycirc} & {\fullcirc} & {\emptycirc} & {\emptycirc} & {\emptycirc} & {\emptycirc} & {\fullcirc} & {\emptycirc} & {\fullcirc} & {\fullcirc} & {\fullcirc} & {\fullcirc} & {\emptycirc} & {\emptycirc} & {\emptycirc} & {\emptycirc} & {\emptycirc} & {\emptycirc} & {\emptycirc} & {\emptycirc} & {\emptycirc} & {\emptycirc} & {\fullcirc} & {\emptycirc} & {\emptycirc} & {\emptycirc} & {\emptycirc} & {\emptycirc}  & {\emptycirc} & {\emptycirc} & {\emptycirc} & {\emptycirc} & \cellcolor{o1}{$1$($7\%$)}
% & \cellcolor{r1}{$0\%$}  
\\ 
% & 1m
% Need to confirm coinbase wallet because it seems it has some smart features but it also has seed phrase
% Whats the difference between passkey and biometrics
Coinbase Wlt  & 2019 & {\fullcirc} & {\emptycirc} & {\emptycirc} & {\emptycirc} & {\fullcirc} & {\fullcirc} & {\fullcirc} & {\emptycirc} & {\emptycirc} & {\emptycirc} & {\emptycirc} & {\emptycirc} & {\emptycirc} & {\emptycirc} & {\fullcirc} & {\fullcirc} & {\emptycirc} & {\emptycirc} & {\fullcirc} & {\emptycirc} & {\fullcirc} & {\emptycirc} & {\emptycirc} & {\fullcirc} & {\emptycirc} & {\emptycirc} & {\emptycirc} & {\fullcirc} & {\emptycirc} & {\fullcirc} & {\emptycirc} & {\fullcirc} & {\emptycirc} & {\emptycirc} & {\fullcirc} & {\fullcirc} & {\fullcirc} & {\fullcirc} & {\fullcirc} & {\fullcirc} & {\emptycirc} & {\fullcirc} & {\fullcirc} & {\fullcirc} & {\emptycirc} & {\emptycirc} & {\emptycirc} & {\emptycirc} & {\emptycirc} & {\emptycirc} & {\emptycirc} & {\emptycirc} & {\emptycirc} & {\fullcirc} & {\emptycirc} & {\emptycirc} & {\emptycirc} & {\emptycirc} & {\emptycirc} & \cellcolor{o1}{$1$($7\%$)} 
% & \cellcolor{r0}{$0\%$}  
\\ 
Biconomy & 2019 & {\fullcirc} & {\emptycirc} & {\emptycirc} &  {\emptycirc} & {\emptycirc} & {\emptycirc} & {\fullcirc} & {\emptycirc} & {\emptycirc} & {\emptycirc} & {\emptycirc} & {\emptycirc} & {\emptycirc} & {\emptycirc} & {\fullcirc} & {\fullcirc} & {\emptycirc} & {\emptycirc}  & {\fullcirc} & {\emptycirc} & {\fullcirc} & {\emptycirc} & {\emptycirc} & {\fullcirc} & {\emptycirc} & {\emptycirc} & {\emptycirc} & {\fullcirc} & {\emptycirc} & {\emptycirc} & {\emptycirc} & {\fullcirc} & {\emptycirc} & {\fullcirc} & {\emptycirc} & {\emptycirc} & {\fullcirc} & {\fullcirc} & {\fullcirc} & {\emptycirc} & {\emptycirc} & {\emptycirc} & {\emptycirc} & {\fullcirc} & {\emptycirc} & {\emptycirc} & {\emptycirc} & {\emptycirc} & {\emptycirc} & {\fullcirc} & {\emptycirc} & {\emptycirc} & {\emptycirc} & {\emptycirc} & {\emptycirc} & {\emptycirc} & {\emptycirc} & {\emptycirc} & {\emptycirc} & \cellcolor{o1}{$1$($7\%$)}  
% & \cellcolor{r2}{$0\%$}  
\\ 
% & 5M 
Web3Auth & 2020 & {\emptycirc} & {\fullcirc} & {\emptycirc} & {\emptycirc} & {\emptycirc} & {\fullcirc} & {\emptycirc} & {\emptycirc} & {\emptycirc} & {\emptycirc} & {\emptycirc} & {\emptycirc} & {\fullcirc} & {\emptycirc} & {\fullcirc} & {\emptycirc} & {\emptycirc} & {\fullcirc} & {\emptycirc} & {\emptycirc} & {\fullcirc} & {\emptycirc} & {\emptycirc} & {\fullcirc} & {\emptycirc} & {\emptycirc} & {\fullcirc} & {\fullcirc} & {\emptycirc} & {\emptycirc} & {\emptycirc} & {\fullcirc} & {\emptycirc} & {\fullcirc} & {\emptycirc} & {\emptycirc} & {\fullcirc} & {\fullcirc} & {\fullcirc} & {\emptycirc} & {\emptycirc} & {\emptycirc} & {\emptycirc} & {\fullcirc} & {\emptycirc} & {\emptycirc} & {\emptycirc} & {\emptycirc} & {\emptycirc} & {\emptycirc} & {\emptycirc} & {\emptycirc} & {\emptycirc} & {\emptycirc} & {\fullcirc} & {\emptycirc} & {\emptycirc} & {\emptycirc} & {\emptycirc} & \cellcolor{o1}{$1$($7\%$)}  
% & \cellcolor{r2}{$0\%$}  
\\ 
Brave & 2021 & {\fullcirc} & {\emptycirc} & {\emptycirc} & {\emptycirc} & {\fullcirc} & {\fullcirc} & {\emptycirc} & {\emptycirc} & {\emptycirc} & {\emptycirc} & {\emptycirc} & {\emptycirc} & {\emptycirc} & {\fullcirc} & {\emptycirc} & {\fullcirc} & {\emptycirc} & {\emptycirc} & {\fullcirc} & {\emptycirc} & {\emptycirc} & {\fullcirc} & {\emptycirc} & {\emptycirc} & {\fullcirc} & {\emptycirc} & {\emptycirc} & {\emptycirc} & {\fullcirc} & {\fullcirc} & {\emptycirc} & {\emptycirc} & {\emptycirc} & {\fullcirc} & {\emptycirc} & {\fullcirc} & {\fullcirc} & {\fullcirc} & {\emptycirc} & {\emptycirc} & {\emptycirc} & {\fullcirc} & {\emptycirc} & {\emptycirc} & {\emptycirc} & {\fullcirc} & {\emptycirc} & {\fullcirc} & {\emptycirc} & {\emptycirc} & {\emptycirc} & {\emptycirc} & {\emptycirc} & {\emptycirc} & {\emptycirc} & {\emptycirc} & {\emptycirc} & {\emptycirc} & {\emptycirc} & \cellcolor{o3}{$2$($13\%$)}  
% & \cellcolor{r2}{$0\%$}  
\\ 
% & 70M m*
Phantom & 2021 & {\fullcirc} & {\emptycirc} & {\emptycirc} & {\emptycirc} & {\fullcirc} & {\fullcirc} & {\emptycirc} & {\emptycirc} & {\emptycirc} & {\emptycirc} & {\emptycirc} & {\emptycirc} & {\emptycirc} & {\fullcirc} & {\emptycirc} & {\fullcirc} & {\emptycirc} & {\emptycirc} & {\fullcirc} & {\emptycirc} & {\emptycirc} & {\fullcirc} & {\emptycirc} & {\emptycirc} & {\fullcirc} & {\emptycirc} & {\emptycirc} & {\emptycirc} & {\fullcirc} & {\fullcirc} & {\fullcirc} & {\emptycirc} & {\emptycirc} & {\emptycirc} & {\fullcirc} & {\fullcirc} & {\fullcirc} & {\fullcirc} & {\emptycirc} & {\emptycirc} & {\emptycirc} & {\fullcirc} & {\emptycirc} & {\emptycirc} & {\emptycirc} & {\fullcirc} & {\emptycirc} & {\fullcirc} & {\emptycirc} & {\emptycirc} & {\emptycirc} & {\emptycirc} & {\emptycirc} & {\emptycirc} & {\emptycirc} & {\emptycirc} & {\emptycirc} & {\emptycirc} & {\emptycirc} & \cellcolor{o3}{$2$($13\%$)}  
% & \cellcolor{r2}{$0\%$}  
\\ 
% & 7M m* 
Slope & 2021 & {\fullcirc} & {\emptycirc} & {\emptycirc} & {\emptycirc} & {\fullcirc} & {\fullcirc} & {\emptycirc} & {\emptycirc} & {\emptycirc} & {\emptycirc} & {\emptycirc} & {\emptycirc} & {\emptycirc} & {\fullcirc} & {\emptycirc} & {\fullcirc} & {\emptycirc} & {\emptycirc} & {\fullcirc} & {\emptycirc} & {\emptycirc} & {\fullcirc} & {\emptycirc} & {\emptycirc} & {\fullcirc} & {\emptycirc} & {\emptycirc} & {\emptycirc} & {\fullcirc} & {\fullcirc} & {\emptycirc} & {\emptycirc} & {\emptycirc} & {\fullcirc} & {\emptycirc} & {\emptycirc} & {\fullcirc} & {\emptycirc} & {\fullcirc} & {\emptycirc} & {\emptycirc} & {\fullcirc} & {\emptycirc} & {\emptycirc} & {\fullcirc} & {\emptycirc} & {\emptycirc} & {\emptycirc} & {\emptycirc} & {\emptycirc} & {\emptycirc} & {\emptycirc} & {\emptycirc} & {\emptycirc} & {\emptycirc} & {\emptycirc} & {\emptycirc} & {\fullcirc}  & {\emptycirc} & \cellcolor{o3}{$2$($13\%$)} 
% & \cellcolor{r1}{$0\%$}  
\\ 
HashPack  & 2021 & {\fullcirc} & {\emptycirc} & {\emptycirc} & {\emptycirc} & {\fullcirc} & {\fullcirc} & {\emptycirc} & {\emptycirc} & {\emptycirc} & {\emptycirc} & {\emptycirc} & {\emptycirc} & {\emptycirc} & {\fullcirc} & {\emptycirc} & {\fullcirc} & {\emptycirc} & {\emptycirc} & {\fullcirc} & {\emptycirc} & {\emptycirc} & {\fullcirc} & {\emptycirc} & {\emptycirc} & {\fullcirc} & {\emptycirc} & {\emptycirc} & {\emptycirc} & {\fullcirc} & {\fullcirc} & {\emptycirc} & {\emptycirc} & {\fullcirc} & {\emptycirc} & {\fullcirc} & {\emptycirc} & {\emptycirc} & {\emptycirc} & {\emptycirc} & {\emptycirc} & {\emptycirc} & {\emptycirc} & {\emptycirc} & {\emptycirc} & {\emptycirc} & {\emptycirc} & {\emptycirc} & {\emptycirc} & {\emptycirc} & {\emptycirc} & {\emptycirc} & {\emptycirc} & {\emptycirc} & {\emptycirc} & {\emptycirc} & {\emptycirc} & {\emptycirc} & {\emptycirc} & {\emptycirc} & $0$($0\%$)
% & \cellcolor{r0}{$0\%$}  
\\ 
Binance Web3 & 2023 & {\emptycirc} & {\fullcirc} & {\emptycirc} & {\emptycirc} & {\emptycirc} & {\fullcirc} & {\emptycirc} & {\emptycirc} & {\emptycirc} & {\emptycirc} & {\emptycirc} & {\emptycirc} & {\fullcirc} & {\emptycirc} & {\fullcirc} & {\emptycirc} & {\emptycirc} & {\fullcirc} & {\fullcirc} & {\emptycirc} & {\emptycirc} & {\fullcirc} & {\emptycirc} & {\emptycirc} & {\emptycirc} & {\emptycirc} & {\emptycirc} & {\fullcirc} & {\fullcirc} & {\emptycirc} & {\emptycirc} & {\emptycirc} & {\emptycirc} & {\fullcirc} & {\emptycirc} & {\emptycirc} & {\fullcirc} & {\fullcirc} & {\fullcirc} & {\emptycirc} & {\emptycirc} & {\fullcirc} & {\emptycirc} & {\fullcirc} & {\emptycirc} & {\emptycirc} & {\emptycirc} & {\emptycirc} & {\emptycirc} & {\fullcirc} & {\emptycirc} & {\emptycirc} & {\emptycirc} & {\emptycirc} & {\emptycirc} & {\emptycirc} & {\emptycirc} & {\emptycirc} & {\emptycirc} & \cellcolor{o1}{$1$($7\%$)} 
% & \cellcolor{r1}{$0\%$}  
\\ 
Kraken Wlt. & 2024 & {\fullcirc} & {\emptycirc} & {\emptycirc} & {\emptycirc} & {\emptycirc} & {\fullcirc} & {\emptycirc} & {\emptycirc} & {\emptycirc} & {\emptycirc} & {\emptycirc} & {\emptycirc} & {\emptycirc} & {\fullcirc} & {\fullcirc} & {\fullcirc} & {\emptycirc} & {\emptycirc} & {\fullcirc} & {\emptycirc} & {\emptycirc} & {\fullcirc} & {\emptycirc} & {\emptycirc} & {\emptycirc} & {\emptycirc} & {\emptycirc} & {\fullcirc} & {\fullcirc} & {\fullcirc} & {\emptycirc} & {\emptycirc} & {\emptycirc} & {\fullcirc} & {\emptycirc} & {\fullcirc} & {\fullcirc} & {\fullcirc} & {\emptycirc} & {\emptycirc} & {\emptycirc} & {\fullcirc} & {\emptycirc} & {\emptycirc} & {\emptycirc} & {\emptycirc} & {\emptycirc} & {\emptycirc} & {\emptycirc} & {\emptycirc} & {\emptycirc} & {\emptycirc} & {\emptycirc} & {\emptycirc} & {\emptycirc} & {\emptycirc} & {\emptycirc} & {\emptycirc} & {\emptycirc} & $0$($0\%$)
% & \cellcolor{r1}{$0\%$}  
\\ 
\midrule
\multicolumn{3}{c}{\textbf{Summary}} &
\multicolumn{17}{c}{\textbf{Highest Occurrence: Signature Verification Logic Flaw}} &
\multicolumn{5}{c}{\cellcolor{o3}{$7$($21\%$)}} &
\multicolumn{20}{c}{} &
\multicolumn{16}{r}{\textbf{Total Vulnerabilities Detected in All Wallets}} &
$33$($100\%$)  
% \cellcolor{o0}{$33$($100\%$)} 

 \\ 
% \midrule
% \multirow{7}{*}{\rotatebox[origin=l]{90}{Custodial}} 
% &  
% \multirow{-7}{*}{\rotatebox[origin=l]{90}{Custodial}}
% & 
% {llccccccccccccccccccccccccccccccccccccccccccccccccccccccccccc}
% \multicolumn{5}{l}{} &
%   \multicolumn{5}{l}{} &
%   \multicolumn{5}{l}{} &
%   \multicolumn{5}{l}{} &
%   \multicolumn{5}{c}{} &
%   \multicolumn{5}{l}{} &
%   \multicolumn{5}{l}{} &
%   \multicolumn{5}{l}{} &
%    \multicolumn{5}{c}{\textbf{{Vulnerabilities No \& \%}}} &
%    \cellcolor{g6}{($0\%$)} &
% \cellcolor{g6}{($0\%$)} &
% \cellcolor{g6}{($0\%$)} &
% \cellcolor{g6}{($0\%$)} &
% \cellcolor{g6}{($0\%$)} &
%   \cellcolor{g6}{($0\%$)} &
% \cellcolor{g6}{($0\%$)} &
% \cellcolor{g6}{($0\%$)} &
% \cellcolor{g6}{($0\%$)} &
% \cellcolor{g6}{($0\%$)} &
%   \cellcolor{g6}{($0\%$)} &
% \cellcolor{g6}{($0\%$)} &
% \cellcolor{g6}{($0\%$)} &
% \cellcolor{g6}{($0\%$)} &
% \cellcolor{g6}{($0\%$)} 
% \\
\bottomrule
\end{tabular}
\vspace{1ex} % Add space before the caption
\caption{Industry Wallet design variations and identified threats. ( \fullcirc : include, \halfcirc : part-inclusion, \emptycirc : not include)
}
\label{tab:wlt._taxonomy}
\end{table*}

\end{landscape}

\subsubsection{Fault Injection}
\label{sec:fau-inj}

These attacks manipulate the wallet's components by forcing an erroneous system state to bypass the security mechanisms \cite{Akter2023AChallenges}.

 For instance, fault injection attacks on hardware wallets often exploit vulnerabilities in volatile memory (such as \acs{sram}) by manipulating environmental factors. Data remanence vulnerabilities in the Trezor wallet have been exploited to demonstrate these attacks \cite{trezor_memory, trezor_medium}. Fault injection attacks on smart contracts have also been shown in the literature \cite{hajdu2020using}.




\subsubsection{Other Non-Invasive Techniques}
\label{sec:non-inv-man}

Other non-invasive storage/memory attacks exist which are not based on fault injection methods. In a Cold Boot Attack, the attacker executes a cold restart on the wallet device to exploit the data remanence properties of volatile memory, such as DRAM and SRAM to retrieve sensitive data \cite{Shaikh2022SurveyExchanges}. Similarly, \acs{puf} attacks exploit the unique characteristics of hardware defence implementations known as \acf{puf} (see \autoref{sec:def_dis_def_vuln}), which have challenge-response functionality that exhibits physical unclonability \cite{Garcia-Bosque2020IntroductionApplications, wang2024efficient}. 


% \paragraph{Cold Boot Attack}
% \label{sec:cold-boot}

% This involves the adversary performing a cold restart on the wallet device to exploit the data remanence properties of \acs{ram} i.e. \acf{dram} and \acf{sram} and retrieve sensitive data \cite{shaikh2022survey}. Data remanence properties in Trezor's \acs{sram} chips have resulted in attack vulnerabilities \cite{trezor_medium, trezor_memory}. 

% \paragraph{\acf{puf} Attacks}
% \label{sec:puf}



% Attackers aim to uncover the physical structures and data storage mechanisms to reveal the \teal{$sk$}, algorithmic secrets, or other sensitive data integral to securing the wallet’s mechanism (see \autoref{sec:formalisation}). 


% These attacks involve exploiting unintended information leakage, such as power consumption measurements, to extract sensitive information like private keys \cite{san2019practical}. These attacks can be conducted using techniques like Single Trace Power Analysis and High-Correlation Analysis on cryptographic operations \cite{park2023stealing}. The attacker leverages the leaked information to compromise the security of the hardware wallet (\autoref{sec:hardware-wallets}). Side channel attacks can be carried out with minimal setup and rely on analyzing power traces or other unintended leakages \cite{gentilal2017trustzone}.


% Side-channel attacks exploit the physical emanations from a wallet's hardware during cryptographic operations, circumventing the mathematical security of algorithms detailed in \autoref{sec:formalisation}. Park et al. (2023) illustrate this through a sophisticated method that extracts private keys by analyzing a single trace of elliptic curve scalar multiplication, demonstrating the subtlety and efficiency of these attacks without needing device profiling. The vast array of leakage sources, including but not limited to power consumption patterns, timing discrepancies, and additional observable data, extend the attack surface significantly, as underscored by Lou et al. (2021) and Ali et al. (2023). These sources offer attackers multiple vectors for extracting sensitive data such as private keys (\teal{$\rho_r$}), highlighting the extensive and varied nature of side-channel threats.


% These attacks exploit systemic vulnerabilities, focusing on the 'side' or peripheral aspects of security mechanisms rather than confronting the cryptographic algorithms directly.


\subsection{Cryptanalysis Attacks}
\label{sec:cryptanalysis-analysis}

\subsubsection{Side-channel Analysis}
\label{sec:side-channel}

Non-invasive key extraction attacks on cryptographic functions including timing and power \acs{sca} are executed by exploiting side channels. These exploit leakages in behaviours exhibited by cryptographic functions (see \autoref{sec:wallet_mechanism}) through side-channels to measure and extract values such as time and power  \cite{Shaikh2022SurveyExchanges, Park2023}. Timing-based \acs{sca} measures the cryptographic function execution time. Successful implementation of a timing-based side-channel attack has been demonstrated on a Trezor One hardware wallet, \cite{kocher1996timing}. Power-based \acs{sca} analyses the cryptographic function's power trace, including the hash function. \acs{sca} on the hash function has been utilised to extract the \teal{$rdm\_seed$} \cite{Park2024CloningFunction}.

\subsubsection{Direct Exploitation}
\label{sec:impl-exp}

These attacks directly target implementation errors within the cryptographic surface area. Weak signature (\teal{$\sigma$}) attacks, for example, target weaknesses in the signing algorithm due to improper implementation, weak or outdated cryptographic algorithms or errors in encryption logic. \cite{Rokhjavan2023SecuringWallets}.  In addition, an adversary can exploit vulnerabilities in the \hyperref[algo:transaction-signing]{Algorithm 3} by reusing a nonce during transactions authorisation \cite{brengel2018identifying}. Such reuse can compromise the security of wallets by resulting in \teal{$sk$} leakage \cite{Ko2020PrivateSignatures}.

\subsection{Discussion}
\label{sec:attacks_discussion}

\subsubsection{Insight 1: Difference in Academia and Notable Industry Incidents}

Identifying attack vectors within the industry remains challenging, as sources often lack specificity. Notable attack vectors are significantly less clear (46\% unknown) and show a lower spread when compared to attacks described in the literature. This might be attributed to a lack of detailed post-mortem analysis in several incidents and a tendency for an adversary to prioritise cost-effective methods. Academia, on the other hand, shows a high percentage (93\%) and spread on various attack methods.

\subsubsection{Insight 2: Comparison of Custodial and Non-Custodial Attacks}
Custodial wallets and non-custodial accounts for 70\% and 30\% of attacks respectively.  Additionally, unknown methods are significantly higher in custodial wallets (50\%) than in non-custodial wallets (36\%). Incidents show a high degree of similarity between custodial and non-custodial attacks. For instance, in comparison to other attacks phishing attacks account for a relatively high percentage of both custodial (10\%) and non-custodial (36\%) wallets, especially factoring in the number of unknown attacks. 

\subsubsection{Insight 3: High Malware \& Phishing Attack Occurrence}

Application attacks account for a significant percentage of incident occurrences (43\%) with 34\% in custodial wallets and 48\% in non-custodial wallets. Our data also indicates that malware and phishing attacks are the most common attack vectors, accounting for 8\% and 18\% of incidents respectively. We also find phishing-malware attacks constitute 48\% of total non-custodial wallet attacks.



% \subsubsection{Insight 1: Difference in Custodial and Non-Custodial Attack Methods}

% \subsubsection{Difference in Attack Methods for Custodial and Non-Custodial Wallets}

% The contrasting occurrence of attack methods in custodial and non-custodial wallets underscores the varying security challenges faced by each type. The data indicates a crucial need for targeted security enhancements to suit custodial and non-custodial wallet architectures, in addition to employing security measures applicable to both.

% \subsubsection{Attack Insight 3: Difference in Custodial and Non-Custodial Attack Methods}

% \subsubsection{Attack Insight 2: Low Percentage of Authentication Attacks in Non-Custodial Wallets}

% Authentication-based attacks account for a relatively small proportion of the total incidents and funds lost in non-custodial wallets, as illustrated in \autoref{fig:custody-pie}. This likely stems from the inherent structure of non-custodial wallets, which grant users complete control over their private keys and seed phrases. Additionally, attackers may prefer targeting other vulnerabilities within non-custodial systems given the potential user negligence, shifting their focus away from direct authentication attacks.

% \subsubsection{Insight 3: High Percentage of Application Attacks in Non-Custodial Wallets}

% Application attacks constitute a significant portion of the security breaches observed in non-custodial wallets, as documented in \autoref{fig:custody-pie}. This is primarily attributable to the open design of the software environments that non-custodial wallets operate within. Many non-custodial wallets are built on platforms that permit third-party integrations and extensions, which, while enhancing functionality, also increase the attack surface. 


% was commented out in old version 
% These applications, often involving smart contracts or decentralised applications (dApps), are exposed to a range of vulnerabilities from coding errors to flawed logic, making them prime targets for attackers. Additionally, the autonomous nature of non-custodial wallets means that users must rely on their judgement or third-party tools to verify the security of the applications they interact with. 


% \subsubsection{Insight 4: Prevalence of Malware Phishing Attacks}

% What percentage of malware attacks are malware phishing?
% What wallets are malware phishing attacks more common on?

% Most wallet attacks were identified only after unauthorised fund transfers had been made using compromised private keys. This indicates significant deficiencies in the existing detection mechanisms within their security operations. Therefore, this demonstrates the need for more effective intrusion detection mechanisms within wallets.

% \subsubsection{Challenge 1: Inaccessibility of Root Causes in Industry} 

% A challenge encountered in our research was the lack of quality data on wallet attacks, with several root causes unknown. With more quality data, it would have been possible to map components in wallets with vulnerabilities, attack methods, vectors and root causes. 

% This attack targets vulnerabilities in the \hyperref[algo:key-generation]{key generation algorithm} or \hyperref[algo:transaction-signing]{signing algorithm} due to improper implementation, 

% such as insufficient validation mechanisms or encryption key size checks to compromise the signature integrity \cite{rokhjavan2023securing}. 

% The vulnerabilities may specifically be a product of weak or outdated cryptographic algorithms (detailed in \autoref{Encryption-Table-1}) 


% and errors in encryption logic.

% \subsection{Industry Incidents Analysis}
% \label{sec:attacks_analysis}


% This section analyses real-world wallet attacks, distinguishing between wallet types, attack targets and attack categories. 

% % % methodology

% The goal of our empirical analysis is to identify patterns which have emerged in previous real-world wallet attacks.

% The data we gather is employed for analysis purposes in \autoref{sec:incident-analysis}.
% Our dataset omits: \begin{enumerate*}
%   \item Incidents on \ac{defi} protocol mechanisms and \ac{defi} or \ac{dao} treasuries.
%   \item Research papers focused on wallets and cryptography with no attacks described.
% \end{enumerate*}

% \paragraph{Academic Papers}
% \label{sec:papers}

% We identify 63 academic papers, and 18 detailed security incident reports, focusing on several different wallet attacks. Our approach involves crawling  Google Scholar and conducting backwards and forward reference searches to identify further pertinent studies. We retrieve attacks on the various types of wallets in our taxonomy including software, hardware, brain, paper, and smart contract wallets (see \autoref{attack-vectors}). 

% \paragraph{Industry Incidents}
% \label{sec:incidents}
% Our data contains 69 real-world attacks on cryptocurrency wallets, which occurred between March 2012 and November 2023. We gather incidents from several sources including DeFiLama, Slowmist, and Rekt News. 



% % % analysis -- incident analysis

% \autoref{fig:attack-frequency} presents the correlation between the frequency of real-world attack incidents on cryptocurrency wallets and the total funds lost annually. Notably, the accumulated losses from these thefts exceed USD 4.5 billion. 2018 was particularly severe, with reported losses surpassing USD 1 billion. Frequently, these incidents involved the compromise of private keys; however, the precise methods utilised often remain undisclosed.



% % % analysis -- taxonomy analysis

% \autoref{fig:custody-method} shows the amount lost for custodial and non-custodial wallets grouped by attack methods. Non-custodial wallets account for one-third of wallets attacked, however, these wallet hacks are only one-tenth of total hacks at 439 million USD. On the other hand, custodial wallets account for 4.1 billion USD, which is 90\% of the total amount lost. This shows the average attack on custodial wallets (\$80 million USD) exceeds that of their non-custodial (\$25 million USD) counterparts. Additionally, nearly all wallets in our dataset are hot. With the exception of the FTX cold wallet incident which claimed a staggering \$400 million USD of user fund managed by the exchange.



% % % analysis -- previously commented out  --  taxonomy analysis

% Consistent with \autoref{sec:attack-framework}, we examine the frequency and impact of different attack methods as depicted in \autoref{fig:attack-method-frequency}. Application-based attacks are the most prevalent, constituting 42\% of all recorded attacks and accounting for 35\% of the total funds lost. Authentication attacks, while less frequent, are almost equally costly, causing 33\% of the financial losses from only 23\% of the incidents. Network attacks are markedly rarer, representing less than 3\% of both incidents and associated financial losses. In about 30\% of cases, the specific methods compromising private keys remain unidentified by wallet operators.

% In further detail, our analysis differentiates the impact of attack methods on custodial and non-custodial wallets as illustrated in \autoref{fig:custody-pie}. For custodial wallets, authentication attacks incur 35.61\% of the total funds lost, occurring in 25.49\% of the incidents. This suggests a high financial impact relative to their frequency. We also observe the prevalence of application attacks which are 33.09\% of the funds lost and 37.25\% of incidents, making them the most common attack vector in custodial settings. 

% Conversely, We find that non-custodial wallets exhibit a different pattern. Application attacks dominate, accounting for 62.99\% of the funds lost and 55.56\% of incidents, underscoring their critical role in non-custodial wallet security. Authentication attacks, while less damaging financially at 10.40\% of the funds lost, still represent 16.67\% of incidents, suggesting their relatively frequent exploitation. Network attacks remain minimal, both in frequency and impact, with just 0.03\% of funds lost and 5.56\% of incidents reported. Unknown attack methods also play a significant role, with 26.58\% of the funds lost and 22.22\% of incidents, reflecting the ongoing challenges in identifying and mitigating these attacks. Unknown methods comprise 22\% and 35\% of the non-custodial and custodial incidents respectively.




% % % analysis -- previously commented out  --  analysis on attack vectors

% \autoref{fig:wallet_attacks_bar} illustrates the frequency of wallet attack vectors. 

% Identifying attack vectors proves even more difficult as these are less clarified in sources than methods, as explained in \autoref{sec:challenges}. Despite more than 50\% of attack vectors being unknown, our data reveals some attack vectors. Notably, malware and phishing attacks are the most prevalent, with 20\% and 23\% of incidents respectively.

% Other significant vectors include storage exploits and SIM swap attacks, both notable for their impact on security, constituting 26.47\% and 25.89\% of the attack vectors. While storage exploits represent a considerable percentage of funds lost, they only account for 2.94\% of incidents, suggesting high effectiveness where used. Similarly, SIM swap attacks though high in funds claimed as a method only occur once. 

% Insider jobs and third-party breaches also contribute significantly to the attack landscape, indicating potential threats beyond the internal wallet mechanisms. Less common but still notable are server attacks, \acs{dns} hijacks, brute force, and API attacks, underscoring the diverse methods attackers employ to exploit wallet vulnerabilities.



% Website, storage and server attacks are categorised under wallet infrastructure attacks.


% Comparison of Total Funds Lost by Attack 

% \subsection{Industry Incidents Analysis}
% % \label{sec:attacks_analysis}




% was commented out in old version 
% This challenge compelled us to derive attack paths aligned with various attacker objectives as shown in \autoref{sec:attack-tree}. Our approach sheds light on potential vulnerabilities and attack vectors extend beyond this analysis.

% \subsubsection{Challenge 2: Clarity of Attack Vectors}
% \label{sec:method-inclarity}

% Many recorded incidents from exchanges or non-custodial wallet providers show a high degree of uncertainty in the reporting of attack vectors. 30\% of all incidents attack methods were unknown. This ambiguity often gives rise to various hypotheses from sources regarding the exact nature of the attacks. For instance, the atomic wallet hack of USD 100 million was said to be a result of one of four probable causes which include a \acf{mitm} attack, a malware code injection or an infrastructure breach \cite{cointele_atomic}. 

% was commented out in old version 
% This uncertainty from sources complicates the analysis, making it challenging to gain insights into attack vectors.



\section{Defence Methods}
\label{sec:defense-strategies}

This section builds upon the framework outlined in \autoref{sec:attack-framework} by presenting mitigation approaches against wallet attacks. We aim to examine defence mechanisms for each identified attack vector affecting wallets.


% We conduct forward reference searches on key articles to track subsequent papers that cite initial studies and utilise Google Scholar's advanced search to propose mitigation. 


\subsection{Defence against Network Attacks}
\label{sec:net-def}

Suspicious network activity can be detected through machine learning techniques, including anomaly detection models \cite{kapoor2021ransomware} and classification algorithms \cite{balakrishnan2023analysis}. Additionally, dynamic network parameter adjustments \cite{Girdler2021ImplementingAddresses} and the implementation of other intrusion detection mechanisms \cite{guri2018beatcoin, zimba2019cryptojacking} further contribute to identifying such anomalies. To mitigate these attacks, wallets can adopt network security protocols that validate and authenticate IP addresses \cite{rengarajan2016secure}, and incorporate additional security layers within the wallet's network to prevent potential \teal{$txn$} modification attempts by adversaries \cite{Cai2014ResearchNetwork}.


In limiting or preventing \acf{ddos} attacks, malicious and authentic network traffic needs to be distinguished by using classifiers such as the decision tree algorithm \cite{khan2019adaptive} and reinforcement learning approaches to analyse patterns in network data \cite{liu2018deep}. Another mitigation approach is analysing the network for unusual patterns, such as repeated request attempts from the same \acs{ip} address \cite{sathwara2017distributed}.

% % Enhanced defence mechanisms against network attacks in cryptocurrency wallets include multifaceted approaches. The integration of hot and cold wallets, as introduced by Jasem et al. \cite{jasem2021enhancement}, addresses deanonymization risks by expanding key space and generating unique keys for each transaction. Security in RPC communications is fortified by recommendations from Bui et al. \cite{bui2019pitfalls}, involving TLS usage and architecture shifts towards inter-process communication to mitigate impersonation threats. Privacy-enhancing network policies like Dandelion, proposed by Venkatakrishnan et al. \cite{bojja2017dandelion}, obscure user identities in the Bitcoin network by mixing messages, while wallet applications like Darkwallet, Wasabi Wallet, and Samourai Wallet implement stealth addresses, blind signatures, and dummy addresses for increased transaction anonymity (Bergman et al. \cite{bergman2021revealing}). These innovations collectively strengthen wallet security and user privacy.

% Suspicious network activity can be identified using machine learning algorithms \cite{ahmed2017mitigating}, dynamically adjusting network parameters \cite{girdler2021implementing} and employing other intrusion detection methods \cite{zimba2019cryptojacking}. To mitigate these attacks, wallets can implement network security protocols that validate and authenticate IP \cite{rengarajan2016secure}, and also add a layer of security implemented in the wallet's network to hinder the adversary's \teal{$txn$} modification attempt \cite{cai2014research}.

% % To mitigate this threat where unauthorised WiFi hotspots intercept and potentially alter transaction details (\autoref{sec:rogue-ap}), an extra layer of security can be implemented within the wallet network, hindering the adversary's \teal{$txn$} modification attempt \cite{cai2014research}.

% % Cai et al \cite{cai2014research} propose a two-factor-based dynamic password technology designed to enhance network security against rogue AP attacks.

% % In response to web-based infections (\autoref{sec:web-infection}), wallet applications should incorporate \acf{ids} that can identify and alert suspicious activities \cite{zimba2019cryptojacking}. These systems should be configured to monitor for unusual network traffic and script executions that may indicate a web-based infection attempt.

% % To prevent a MAC-\acs{ip} address linkage by an adversary through \acs{arp} spoofing, the wallet network's operating parameters can be dynamically adjusted to detect malicious network traffic \cite{girdler2021implementing}.

% % The redirection of the wallet devices to fraudulent websites through the compromise of the \acs{dns} revolver \autoref{sec:dns-spoofing} can be counteracted by employing machine learning algorithms to identify suspicious network activities \cite{ahmed2017mitigating}.

% % To counteract this attack, wallets can implement network security protocols that validate and authenticate IP to block unauthorised access attempts \cite{rengarajan2016secure}. This \acs{ip} authentication mitigation ensures that only traffic from verified and trustworthy IP addresses can interact with the wallet.

% This attack can be mitigated by distinguishing between malicious and authentic network traffic. Two notable traffic distinguishing techniques are using classifiers such as the decision tree algorithm \cite{khan2019adaptive} and using reinforcement learning approaches to analyse patterns in network data \cite{liu2018deep}. Another mitigation approach is analysing the network for unusual patterns, such as repeated request attempts from the same \acs{ip} address \cite{sathwara2017distributed}.

% % \acs{dos} attacks delivered via botnet attacks can be detected by filtering the network traffic to identify potential bot activity and using classifiers such as the decision tree algorithm to distinguish between benign and malicious traffic \cite{khan2019adaptive}.

% % To prevent downtime and stop excessive traffic from an adversary executing this attack, wallets can use reinforcement learning approaches to analyse patterns in network data to stop malicious traffic while allowing authentic network traffic \cite{liu2018deep}.

% % An effective strategy for mitigating the \acs{tcp} \acs{syn} flooding attack which limits the wallet's operation (\autoref{sec:wallet_mechanism}) involves network analysis for unusual patterns, such as repeated request attempts from the same \acs{ip} address. Following this, the server responds with \acs{tcp}-RST packets to halt these attempts, to prevent network downtime \cite{sathwara2017distributed}.

\subsection{Defence against Application Attacks}
\label{sec:app-def}

To mitigate the risk of message alteration by clipboard hijackers, features such as NFC, and two-dimensional codes

\begin{table*}[!h]
\centering
\renewcommand{\arraystretch}{1.1}
\setlength{\tabcolsep}{1.5pt} % Adjust the column separation space here
\footnotesize % or \scriptsize, \tiny, etc.
\resizebox{1.0\textwidth}{!}{
\begin{tabular}{llcccccccccccccccccccccccccccccc}
\toprule
\vspace{1pt} 
& \multicolumn{31}{c}{\textbf{Possible Defence Methods}}
\vspace{1pt} 
\\
\multicolumn{2}{c}{\textbf{ Classification}} 
& \rotatebox[origin=c]{90}{\cite{Cai2014ResearchNetwork}} % Network Authentication Tool
& \rotatebox[origin=c]{90}{\cite{Ahmed2017MitigatingNetworking}} % Web App Firewalls
& \rotatebox[origin=c]{90}{\cite{Bhirud2011LightPrevention}} % Dynamic IP Verification
& \rotatebox[origin=c]{90}{\cite{liu2018deep}} % Agent-based Traffic Mitigation
& \rotatebox[origin=c]{90}{\cite{sathwara2017distributed}} % Reset TCP Connections
& \rotatebox[origin=c]{90}{\cite{li2020android}} % Alteration Prevention Features | Access Control Restrictions ***
& \rotatebox[origin=c]{90}{\cite{ferdous2023review}} % Anti-malware Software
& \rotatebox[origin=c]{90}{\cite{indusface}} % Code Obfuscation
& \rotatebox[origin=c]{90}{\cite{Tirronen2018StoppingData}} % Cryptographic Code Verification
& \rotatebox[origin=c]{90}{\cite{Aratani2015AuthenticationChannel}} % Multi-factor Authentication
& \rotatebox[origin=c]{90}{\cite{aldawood2020advanced}} % Advanced Password Selection | Custom Keyboard Functions
& \rotatebox[origin=c]{90}{\cite{galbally2013image}} % Liveness Assessment Features
& \rotatebox[origin=c]{90}{\cite{altuwaijri2020android}} % Supplementary Storage
& \rotatebox[origin=c]{90}{\cite{breier2022practical}} % Algorithmic Fault Detection
& \rotatebox[origin=c]{90}{\cite{Urien2021InnovativeWallets}} % PUF
& \rotatebox[origin=c]{90}{\cite{Gupta2019ImpactSecurity}} % Memory and Cache Data Split
& \rotatebox[origin=c]{90}{\cite{brengel2018identifying}} % Secure Cryptographic Schemes | Deterministic Nonce Selection
& \rotatebox[origin=c]{90}{\cite{Park2024CloningFunction}} % Correlation Elimination
& \rotatebox[origin=c]{90}{\cite{Akter2023AChallenges}} % Correlation Elimination
& \rotatebox[origin=c]{90}{\cite{Lindell2020SecureComputation}} % MPC
& \rotatebox[origin=c]{90}{\cite{bip11}} % Multi-sig
& \rotatebox[origin=c]{90}{\cite{Park2023}} % Correlation sounds
& \rotatebox[origin=c]{90}{\cite{Feng2023Man-in-the-middleRedirects}} % Mitm mitation
& \rotatebox[origin=c]{90}{\cite{Kim2022ACountermeasures}} % MPC
& \rotatebox[origin=c]{90}{\cite{Shuvo2023AAttacks}} % algorithmic fault detection
& \rotatebox[origin=c]{90}{\cite{zimba2019cryptojacking}} % Intrusion Detection
& \rotatebox[origin=c]{90}{\cite{qi2012spad}} % Runtime Protection
& \rotatebox[origin=c]{90}{\cite{ManageAddresses}} % manage destination address
& \rotatebox[origin=c]{90}{\cite{hu2020overview}} % Physical Unclonable Functions (PUFs)
& \# (\%)
\vspace{1.5pt} 
\\
\midrule
\multirow{3}{*}{Precautionary} &
\rotatebox[origin=c]{0}{Prevention} & {\smallemptycirc} & {\smallemptycirc} & {\smallemptycirc} & {\smallemptycirc} & {\smallemptycirc} & {\smallfullcirc} & {\smallemptycirc} & {\smallemptycirc} & {\smallfullcirc} & {\smallemptycirc} & {\smallemptycirc} & {\smallemptycirc} & {\smallemptycirc} & {\smallemptycirc} & {\smallemptycirc} & {\smallemptycirc} & {\smallfullcirc} & {\smallemptycirc} & {\smallemptycirc} & {\smallemptycirc} & {\smallemptycirc} & {\smallemptycirc} & {\smallemptycirc} & {\smallemptycirc} & {\smallemptycirc} & {\smallemptycirc} & {\smallemptycirc} & {\smallemptycirc} & {\smallemptycirc} & \cellcolor{g2}{$3$($10\%$)} \\
& \rotatebox[origin=c]{0}{Protection} & {\smallfullcirc} & {\smallfullcirc} & {\smallfullcirc} & {\smallemptycirc} & {\smallemptycirc} & {\smallfullcirc} & {\smallfullcirc} & {\smallfullcirc} & {\smallemptycirc} & {\smallfullcirc} & {\smallfullcirc} & {\smallfullcirc} & {\smallemptycirc} & {\smallemptycirc} & {\smallfullcirc} & {\smallfullcirc} & {\smallemptycirc} & {\smallfullcirc} & {\smallfullcirc} & {\smallemptycirc} & {\smallemptycirc} & {\smallfullcirc} & {\smallfullcirc} & {\smallemptycirc} & {\smallemptycirc} & {\smallemptycirc} & {\smallfullcirc} & {\smallemptycirc} & {\smallfullcirc} &  \cellcolor{g6}{$17$($58\%$)} \\
& \rotatebox[origin=c]{0}{Limitation} & {\smallemptycirc} & {\smallemptycirc} & {\smallemptycirc} & {\smallfullcirc} & {\smallemptycirc} & {\smallemptycirc} & {\smallemptycirc} & {\smallemptycirc} & {\smallemptycirc} & {\smallemptycirc} & {\smallemptycirc} & {\smallemptycirc} & {\smallfullcirc} & {\smallemptycirc} & {\smallemptycirc} & {\smallemptycirc} & {\smallemptycirc} & {\smallemptycirc} & {\smallemptycirc} & {\smallfullcirc} & {\smallfullcirc} & {\smallemptycirc} & {\smallemptycirc} & {\smallfullcirc} & {\smallemptycirc} & {\smallemptycirc} & {\smallemptycirc} & {\smallfullcirc} & {\smallemptycirc} & \cellcolor{g3}{$6$($21\%$)} \\
\midrule
\multirow{3}{*}{Remedial} & \rotatebox[origin=c]{0}{Detection} & {\smallemptycirc} & {\smallemptycirc} & {\smallemptycirc} & {\smallfullcirc} & {\smallemptycirc} & {\smallemptycirc} & {\smallemptycirc} & {\smallemptycirc} & {\smallemptycirc} & {\smallemptycirc} & {\smallemptycirc} & {\smallemptycirc} & {\smallemptycirc} & {\smallfullcirc} & {\smallemptycirc} & {\smallemptycirc} & {\smallemptycirc} & {\smallemptycirc} & {\smallemptycirc} & {\smallemptycirc} & {\smallemptycirc} & {\smallemptycirc} & {\smallfullcirc} & {\smallemptycirc} & {\smallfullcirc} & {\smallfullcirc} & {\smallemptycirc} & {\smallemptycirc} & {\smallemptycirc} & \cellcolor{g3}{$5$($17\%$)} \\
& \rotatebox[origin=c]{0}{Response} & {\smallemptycirc} & {\smallemptycirc} & {\smallemptycirc} & {\smallfullcirc} & {\smallemptycirc} & {\smallemptycirc} & {\smallemptycirc} & {\smallemptycirc} & {\smallemptycirc} & {\smallemptycirc} & {\smallemptycirc} & {\smallemptycirc} & {\smallemptycirc} & {\smallemptycirc} & {\smallemptycirc} & {\smallemptycirc} & {\smallemptycirc} & {\smallemptycirc} & {\smallemptycirc} & {\smallemptycirc} & {\smallemptycirc} & {\smallemptycirc} & {\smallemptycirc} & {\smallemptycirc} & {\smallemptycirc} & {\smallemptycirc} & {\smallemptycirc} & {\smallemptycirc} & {\smallemptycirc} & \cellcolor{g1}{$1$($3\%$)} \\
& \rotatebox[origin=c]{0}{Recovery} & {\smallemptycirc} & {\smallemptycirc} & {\smallemptycirc} & {\smallemptycirc} & {\smallfullcirc} & {\smallemptycirc} & {\smallemptycirc} & {\smallemptycirc} & {\smallemptycirc} & {\smallemptycirc} & {\smallemptycirc} & {\smallemptycirc} & {\smallemptycirc} & {\smallemptycirc} & {\smallemptycirc} & {\smallemptycirc}  & {\smallemptycirc} & {\smallemptycirc} & {\smallemptycirc} & {\smallemptycirc} & {\smallemptycirc} & {\smallemptycirc} & {\smallemptycirc} & {\smallemptycirc} & {\smallemptycirc} & {\smallemptycirc} & {\smallemptycirc} & {\smallemptycirc} & {\smallemptycirc} & \cellcolor{g1}{$1$($3\%$)}
\vspace{1pt}
\\
\midrule 
\multicolumn{3}{c}{Summary}  &
\multicolumn{8}{c}{Precautionary:  \cellcolor{g6}{$26$($89\%$)}}  &
\multicolumn{8}{c}{Remedial: 
 \cellcolor{g3}{$7$($24\%$)}}  &
% \multicolumn{7}{c}{}  &
\multicolumn{11}{r}{Total Unique Methods    }  &
\multicolumn{1}{c}{}  &
\cellcolor{g0}{$29$($100\%$)} 
\vspace{1pt} 
   \\
\bottomrule
\end{tabular}
}
\vspace{1ex} % Add space before the caption
\caption{Defence methods categorised by type showing classification frequency (\#) and percentage (\%). Precautionary methods proactively prevent attacks; remedial methods provide attack detection, response, or data recovery.}
\label{tab:defence_methods}
\end{table*}




% need to add in brute force defence methods 
% to this table - i.e. table 5
% and also table 4
% \cite{Kiktenko2019DetectingWallets, volety2019cracking, Byun2024AAttacks}



  % {\smallfullcirc} &
  % {\smallemptycirc} &
can be employed to prevent modification of the \teal{$recipient\_address$} during transaction creation \cite{li2020android}. From a user perspective, Human-readable addresses such as \acs{ens} \cite{ENS2024EthereumService} aid in detecting address tampering, though they have certain security vulnerabilities \cite{Xia2022ChallengesENS}. System behaviour modifications can be prevented by addressing specific attack vectors. Attack vectors which attempt these by targeting vulnerabilities in the operating system can be mitigated by employing code obfuscation \cite{indusface} and runtime protection mechanisms \cite{qi2012spad}. Furthermore, by enforcing \acf{cfi} measures, wallets can ensure that the control flow hijacked to deviate from the intended control flow paths for malicious transactions cannot be executed \cite{Creech2017NewMitigation}. 


% Application-based intrusion software, such as malware, can be detected through effective analysis of system interactions, network communications, and behavioural patterns. Additionally, employing machine learning classifiers \cite{balakrishnan2023analysis}, anomaly detection models \cite{kapoor2021ransomware} and intrusion detection systems \cite{guri2018beatcoin} have emerged as effective detection methods. Following detection, the malware file can be tracked and neutralised.


% \subsubsection{Malware}
% \label{sec:def-malware}

% Detecting this attack effectively requires the analysis of malware behaviours, system interactions, and network communications.
% Employing machine learning classifiers \cite{balakrishnan2023analysis}, anomaly detection models \cite{kapoor2021ransomware} and intrusion detection systems \cite{guri2018beatcoin} have emerged as effective detection methods. Following detection, the malware file can be tracked and neutralised. 
% % Clipboard hijacking malware can be significantly limited by implementing human-readable address names.

% % \paragraph{Clipboard Hijacker}
% % \label{sec:def-clipboard}

% % Wallet can address this attack by employing novel features which prevent copying and pasting the \teal{$recipient\_addr$} during the transaction creation stage (\autoref{sec:transaction_signing}) \cite{li2020android}. These address replacement features include NFC, two-dimensional codes and human-readable address names. 

% % \paragraph{Keylogger}
% % \label{sec:def-keylogger}

% % Keyloggers, known for their hidden operations, present detection challenges. Despite this, employing advanced machine learning classifiers has emerged as a potent strategy for identifying and neutralising such threats\cite{balakrishnan2023analysis}.

% % \paragraph{Ransomware}
% % \label{sec:def-ransomware}

% % Detecting ransomware effectively requires the analysis of malware behaviours, system interactions, and network communications. Advanced machine learning techniques, such as anomaly detection models and supervised learning algorithms play a critical role in identifying potential ransomware activities \cite{kapoor2021ransomware}. However, while detection is crucial, secure backup and restore implementations prevent attacks. 


% % This approach ensures that, in the event of encryption by ransomware, affected files can be promptly and effortlessly restored.

% % \paragraph{Spyware}
% % \label{sec:def-spyware}
% % Spyware is particularly hard to detect as its hidden operations present challenges. Despite this, employing machine learning classifiers has emerged as an effective detection \cite{balakrishnan2023analysis}. Following detection, the spyware file can be tracked and neutralised.

% % \paragraph{Supply Chain Attack}
% % \label{sec:def-supply}

% % To enhance resilience against these attacks, emphasising early detection and response is crucial. This involves monitoring network data flows to identify unauthorised data access or unusual communication patterns that deviate from established norms \cite{wang2021feasibility}. By comparing current traffic behaviours against a pre-determined baseline of typical network activity, any unusual data movements or suspected malicious interactions can be identified.

% % -- extended explanation 
% % This method relies on the consistent aspects of traffic data, allowing for the detection of anomalies, even in communications that are encrypted, to track and uncover the mechanisms of such attacks effectively.

% % \paragraph{Phishing}
% % \label{sec:def-phishing}


% % To enhance security against phishing, wallets should incorporate phishing-resistant multi-factor authentication (MFA) techniques such as FIDO2 \cite{wang2021feasibility}. This feature prevents phishing by communicating with the original wallet website to verify the authenticity of the illegitimate one before allowing access to the wallet \cite{fido2}. Therefore, in scenarios where users are attack victims, the adversary will not bypass the other authentication methods. Additionally, users should verify the URLs of wallet browsers and exchange services before using them \cite{weichbroth2023security}.

% % \paragraph{Removable Media Infection}
% % \label{sec:def-removable}

% % Malware delivered to hardware wallets via removable media can be addressed by implementing intrusion detection and prevention systems \cite{guri2018beatcoin}. Furthermore, anti-virus programs can be installed on these wallet devices. 

% \subsubsection{Social Engineering}
% \label{sec:soc-engineering}
% \paragraph{Phishing}
% \label{sec:def-phishing}

% To enhance security against phishing, wallets should incorporate phishing-resistant multi-factor authentication (MFA) techniques such as FIDO2 \cite{wang2021feasibility}. This feature prevents phishing by communicating with the original wallet website to verify the authenticity of the illegitimate one before allowing access to the wallet \cite{fido2}. Additionally, users should verify the URLs of wallet browsers and exchange services before using them \cite{weichbroth2023security}.

% \subsubsection{Privilege Escalation}
% \label{sec:def-privilege}

% \paragraph{Android Root Exploitation}
% \label{sec:def-android-root}

% To safeguard mobile wallets from this attack, employing code obfuscation is essential. This technique scrambles the application's code without altering its functionality \cite{indusface}. 

% % Consequently, adversaries are unable to understand the application's inner workings, providing a robust defence against unauthorised access and the exploitation of sensitive data.

% \paragraph{Debugger}
% \label{sec:def-debugger}

% Implementing a monitoring feature to manage and restrict debugging settings mitigates this attack \cite{li2020android}. Additionally, employing runtime protection mechanisms helps detect and respond to unauthorised debugging attempts \cite{qi2012spad}.

% \paragraph{Logic Flow Exploitation}
% \label{sec:def-logic-flow}

% By enforcing \acf{cfi} measures, wallets can ensure that malicious transactions cannot be executed or control flow hijacked to deviate from the intended control flow paths \cite{creech2017new}. 

% % CFI works by inserting checks before indirect control flow transfers (like indirect function calls and returns) during program execution. These checks validate that the destination of the control flow transfer is within the set of allowed targets as defined by the CFG. If an attempted control flow transfer does not correspond to an allowed path in the CFG, it is considered an attack attempt, and the execution can be halted or redirected to a safe state.


\subsection{Defence against Authentication Attacks}
\label{sec:auth-def}

Wallets can either incorporate features as direct protection against specific attack methods or incorporate general authentication bypass features. By directly integrating improved functionalities to obstruct access to predictive text data, wallets can prevent the dictionary attack \cite{Uddin2021Horus:Wallets}. Additionally, to prevent brute force attacks, only complex passwords should be allowed in the initialisation stage  \cite{praitheeshan2019security}. Biometric falsifying attacks can be prevented by incorporating liveness detection features in wallets \cite{galbally2013image}.

To prevent single points of failure, wallets can enhance authentication levels (\autoref{sec:design-authen}) through \acf{mfa}, \acf{mpc} \cite{Lindell2020SecureComputation} and multi-signatory features such as BIP-11's M-of-N standard  \cite{bip11} (\autoref{sec:design-distr}). To mitigate social engineering attacks, for example, wallets can incorporate phishing-resistant multi-factor authentication (MFA) techniques such as FIDO2 \cite{Wang2021OnAttacks}. This feature enables communication with the original wallet website to verify the authenticity of the illegitimate one before allowing access to the wallet \cite{fido2}. 

\subsection{Defence against Storage and Memory Attacks}
\label{sec:sto-def}

An effective defence method against these attacks involves incorporating \acf{puf} to generate cryptographic keys on-demand, without storing \teal{$sk$} on the wallet's chip. This method also prevents microscopy attacks, some other physical tampering attacks and side-channel attacks (see \autoref{sec:crypt-def}) \cite{Urien2021InnovativeWallets, Park2024CloningFunction}. Physical tampering through the evil maid attack can be limited by implementing trusted boot mechanisms \cite{Tereshkin2010EvilEncryption}. Possible mitigations against non-invasive manipulation such as the cold boot attack involve adopting features which algorithmically clear the wallet's memory following intrusion \cite{seol2019amnesiac}. For example, Ledger has introduced a secure layer which detects chip intrusion and erases \teal{$sk$} following extraction attempts \cite{ledgerwallet}.

% % Wallets can implement trusted boot mechanisms to counteract this attack. Trusted boot rigorously verifies the integrity of software components during the boot process, allowing only those validated by their cryptographic signatures to execute \cite{tereshkin2010evil}. This effectively shields the device from unauthorised changes that could otherwise exploit the system to retrieve credentials. 

% To shield the device from unauthorised changes, wallets could implement trusted boot mechanisms, which verify the integrity of components during the boot process, allowing only those validated by their cryptographic signatures to execute \cite{tereshkin2010evil}.

\subsection{Defence against Cryptanalysis Attacks}
\label{sec:crypt-def}

The exploitation of cryptographic vulnerabilities can lead to \teal{$sk$} extraction. Attacks that aim to exploit weak cryptographic signatures (\teal{$\sigma$}), for instance, can be counteracted by employing stronger hashing algorithms \cite{Rokhjavan2023SecuringWallets}, while deterministic \teal{$nonce$} selection prevents nonce reuse attacks \cite{brengel2018identifying}. Non-invasive attacks on cryptographic functions including timing and power \acs{sca} are executed by exploiting side channels. Effective prevention methods include data leakage protection and data access patterns disguised as noise injection \cite{Akter2023AChallenges, Lou2021ACryptography, Ali2023CharacterizationHardware, Park2024CloningFunction}. These disrupt the adversary's ability to interpret leaked information effectively \cite{Mosquera2023GuardAttacks}. 

% Another method of disrupting the adversary for non-invasive \acs{sca} power attacks is by introducing sounds to disrupt the correlation between power consumption and extractable data \cite{Park2024CloningFunction}.


% \subsection{Intrusion Attacks}
% \label{sec:int-def}

% These defence methods are aimed at detecting network, application or physical intrusion attempts by the adversary. 

% Network Intrusion

% Suspicious network activity can be identified using machine learning algorithms \cite{Ahmed2017MitigatingNetworking}, dynamically adjusting network parameters \cite{Girdler2021ImplementingAddresses} and employing other intrusion detection methods \cite{zimba2019cryptojacking}. To mitigate these attacks, wallets can implement network security protocols that validate and authenticate IP \cite{rengarajan2016secure}, and also add a layer of security implemented in the wallet's network to hinder the adversary's \teal{$txn$} modification attempt \cite{Cai2014ResearchNetwork}. Additionally, employing machine learning classifiers \cite{balakrishnan2023analysis}, anomaly detection models \cite{kapoor2021ransomware} and intrusion detection systems \cite{guri2018beatcoin} have emerged as effective detection methods.

% girdler2021implementing
% cai2014research

% Application Intrusion

% Application-based intrusion software, such as malware, can be detected through effective analysis of system interactions, network communications, and behavioural patterns. Additionally, employing machine learning classifiers \cite{balakrishnan2023analysis}, anomaly detection models \cite{kapoor2021ransomware} and intrusion detection systems \cite{guri2018beatcoin} have emerged as effective detection methods. Following detection, the malware file can be tracked and neutralised. Additionally, physical tampering attacks such as the row hammer attack can be detected and mitigated using machine learning techniques \cite{joardar2022learning}.


% Physical Intrusion



% \subsection{Alteration Defence}
% \label{sec:alt-def}

% These defence methods cover a wide range of attacks aimed at altering the transaction message or system behaviour. To prevent message alteration by clipboard hijacker, features such as NFC, and two-dimensional codes which prevent \teal{$recipient\_addr$} change during transaction creation can be introduced \cite{li2020android}. Additionally, the use of human-readable addresses enables the user to notice address modifications. 

% Modifications to the system behaviour can be prevented by addressing specific attack vectors. Attack vectors which attempt these modifications by targeting vulnerabilities in the operating system can be prevented by employing code obfuscation \cite{indusface} and runtime protection mechanisms \cite{qi2012spad}. Furthermore, by enforcing \acf{cfi} measures, wallets can ensure that the control flow hijacked to deviate from the intended control flow paths for malicious transactions cannot be executed \cite{Creech2017NewMitigation}. 


% is essential. This technique scrambles the application's code without altering its functionality \cite{indusface}. 

% Additionally, employing runtime protection mechanisms helps detect and respond to unauthorised debugging attempts \cite{qi2012spad}.





% \subsubsection{Access Control Restrictions}
% \label{sec:def-mitm}

% \subsection{Authentication Bypass Defence}
% \label{sec:auth-def}

% These defence methods aim to prevent adversaries from decrypting private keys or bypassing authentication mechanisms by eliminating single points of failure and mitigating potential threats. 


% Wallets can either incorporate features as direct protection against specific attack methods or incorporate general authentication bypass features. By directly integrating improved functionalities to obstruct access to predictive text data, wallets can prevent the dictionary attack \cite{Uddin2021Horus:Wallets}. Additionally, to prevent brute force attacks, only complex passwords should be allowed in the initialisation stage  \cite{praitheeshan2019security}. Biometric falsifying attacks can be prevented by incorporating liveness detection features in wallets \cite{galbally2013image}.

% To prevent single points of failure, wallets can enhance authentication levels (\autoref{sec:design-authen}) through \acf{mfa}, \acf{mpc} \cite{Lindell2020SecureComputation} and multi-signatory features such as BIP-11's M-of-N standard  \cite{bip11} (\autoref{sec:design-distr}). To mitigate social engineering attacks, for example, wallets can incorporate phishing-resistant multi-factor authentication (MFA) techniques such as FIDO2 \cite{Wang2021OnAttacks}. This feature enables communication with the original wallet website to verify the authenticity of the illegitimate one before allowing access to the wallet \cite{fido2}. 

% To shield the device from unauthorised changes, wallets could implement trusted boot mechanisms, which verify the integrity of components during the boot process, allowing only those validated by their cryptographic signatures to execute \cite{tereshkin2010evil}.


% \subsection{Extraction Defence}
% \label{sec:ext-def}

% These defence methods are aimed at preventing invasive and non-invasive forms of private key extraction by the adversary, as these attacks target vulnerabilities in the wallet mechanism.

% mosquera2023guard
% park2024cloning

% \subsection{Disruption Defence}
% \label{sec:dis-def}

% \subsubsection{The Influence of Design on Defence}

% \paragraph{Hierarchical Deterministic Wallets}

% \paragraph{Distributed Wallets}

% \paragraph{Contract Validation Vulnerabilities}

% \paragraph{2FA \& Trusted Entities}

\subsection{Discussion}
\label{sec:def_discussion}

\subsubsection{Insight 1: Mitigations Against Multiple Attack Vectors}
\label{sec:def_dis_attacks}

We observe that design plays a critical role in enhancing defence mechanisms. For example, distributed architectures, such as \acs{mpc} and multi-signature functionalities in smart contract wallets, and multi-factor authentication, limit or protect against several attack vectors. On the other hand, the majority of defence implementations are particularly tailored to specific advanced attacks such as \acs{puf} for microscopic attacks, correlation elimination sounds for non-invasive side channels, and \acs{puf} attacks. These demonstrate the variety of defence strategies. 

\subsubsection{Insight 2: Comparison of Precautionary and Remedial Defence Methods}
\label{sec:def_dis_attacks}

Our study presents defence methods applicable to various attack vectors, with the majority offering either precautionary or remedial strategies, as illustrated in Table~\ref{tab:defence_methods}. Notably, precautionary defences significantly outnumber remedial approaches, comprising roughly 89\% of all methods observed. Within the precautionary category, protection-focused implementations are the most prevalent, accounting for 58\%. Among remedial defences, detection methods are the most common at 17\%, while response and recovery measures each represent a mere 3\%. This disparity highlights a critical gap in reactive mitigation techniques, indicating a potential area for further development in response and recovery-focused defences.


% While we provide defence methods for all attack vectors in our study, the majority of defence implementations offer either precautionary or remedial strategies as shown in \autoref{tab:defence_methods}, except for implementations which offer both \cite{liu2018deep}. We also observe that most of the defence implementations focus on precautionary rather than remedial techniques, with the former accounting for approximately 89\% of the total. Additionally, in analysing the sub-classes, protection-based implementations have the highest occurrence at 58 \%  Redemial sub-classes, detection accounts for 17\%, while response and recover account for 3\% each. The low percentage of response and recovery implementations demonstrates a lack of reactive mitigation methods.  


\subsubsection{Insight 3: Vulnerabilities in Defence Methods}
\label{sec:def_dis_def_vuln}

An interesting observation is the occurrence of targeted attacks and vulnerabilities in defence implementations. For instance, \acs{puf} effectively mitigates against the microscopy attack and other invasive hardware-based attacks. However, specific attack vectors in the literature exist against this protection mechanism. Furthermore, several vulnerabilities which enable \teal{$sk$} derivation from a single shard exist in \acs{mpc} wallets \cite{cve_12118}.

% just uncommented out  ****
% \subsubsection{Challenge 1: Inadequate Detection Mechanisms in Industry}

% \subsubsection{Effectiveness of Defence Methods}

% \subsubsection{Gap Analysis}

% \subsection{Defence against Network Attacks}
% \label{sec:def-network}

% % Enhanced defence mechanisms against network attacks in cryptocurrency wallets include multifaceted approaches. The integration of hot and cold wallets, as introduced by Jasem et al. \cite{jasem2021enhancement}, addresses deanonymization risks by expanding key space and generating unique keys for each transaction. Security in RPC communications is fortified by recommendations from Bui et al. \cite{bui2019pitfalls}, involving TLS usage and architecture shifts towards inter-process communication to mitigate impersonation threats. Privacy-enhancing network policies like Dandelion, proposed by Venkatakrishnan et al. \cite{bojja2017dandelion}, obscure user identities in the Bitcoin network by mixing messages, while wallet applications like Darkwallet, Wasabi Wallet, and Samourai Wallet implement stealth addresses, blind signatures, and dummy addresses for increased transaction anonymity (Bergman et al. \cite{bergman2021revealing}). These innovations collectively strengthen wallet security and user privacy.

% Suspicious network activity can be identified using machine learning algorithms \cite{ahmed2017mitigating}, dynamically adjusting network parameters \cite{girdler2021implementing} and employing other intrusion detection methods \cite{zimba2019cryptojacking}. To mitigate these attacks, wallets can implement network security protocols that validate and authenticate IP \cite{rengarajan2016secure}, and also add a layer of security implemented in the wallet's network to hinder the adversary's \teal{$txn$} modification attempt \cite{cai2014research}.

% % To mitigate this threat where unauthorised WiFi hotspots intercept and potentially alter transaction details (\autoref{sec:rogue-ap}), an extra layer of security can be implemented within the wallet network, hindering the adversary's \teal{$txn$} modification attempt \cite{cai2014research}.

% % Cai et al \cite{cai2014research} propose a two-factor-based dynamic password technology designed to enhance network security against rogue AP attacks.

% % In response to web-based infections (\autoref{sec:web-infection}), wallet applications should incorporate \acf{ids} that can identify and alert suspicious activities \cite{zimba2019cryptojacking}. These systems should be configured to monitor for unusual network traffic and script executions that may indicate a web-based infection attempt.

% % To prevent a MAC-\acs{ip} address linkage by an adversary through \acs{arp} spoofing, the wallet network's operating parameters can be dynamically adjusted to detect malicious network traffic \cite{girdler2021implementing}.

% % The redirection of the wallet devices to fraudulent websites through the compromise of the \acs{dns} revolver \autoref{sec:dns-spoofing} can be counteracted by employing machine learning algorithms to identify suspicious network activities \cite{ahmed2017mitigating}.

% % To counteract this attack, wallets can implement network security protocols that validate and authenticate IP to block unauthorised access attempts \cite{rengarajan2016secure}. This \acs{ip} authentication mitigation ensures that only traffic from verified and trustworthy IP addresses can interact with the wallet.

% This attack can be mitigated by distinguishing between malicious and authentic network traffic. Two notable traffic distinguishing techniques are using classifiers such as the decision tree algorithm \cite{khan2019adaptive} and using reinforcement learning approaches to analyse patterns in network data \cite{liu2018deep}. Another mitigation approach is analysing the network for unusual patterns, such as repeated request attempts from the same \acs{ip} address \cite{sathwara2017distributed}.

% % \acs{dos} attacks delivered via botnet attacks can be detected by filtering the network traffic to identify potential bot activity and using classifiers such as the decision tree algorithm to distinguish between benign and malicious traffic \cite{khan2019adaptive}.

% % To prevent downtime and stop excessive traffic from an adversary executing this attack, wallets can use reinforcement learning approaches to analyse patterns in network data to stop malicious traffic while allowing authentic network traffic \cite{liu2018deep}.

% % An effective strategy for mitigating the \acs{tcp} \acs{syn} flooding attack which limits the wallet's operation (\autoref{sec:wallet_mechanism}) involves network analysis for unusual patterns, such as repeated request attempts from the same \acs{ip} address. Following this, the server responds with \acs{tcp}-RST packets to halt these attempts, to prevent network downtime \cite{sathwara2017distributed}.


% \subsubsection{\acf{mitm}}
% \label{sec:def-mitm}

% Suspicious network activity can be identified using machine learning algorithms \cite{ahmed2017mitigating}, dynamically adjusting network parameters \cite{girdler2021implementing} and employing other intrusion detection methods \cite{zimba2019cryptojacking}. To mitigate these attacks, wallets can implement network security protocols that validate and authenticate IP \cite{rengarajan2016secure}, and also add a layer of security implemented in the wallet's network to hinder the adversary's \teal{$txn$} modification attempt \cite{cai2014research}.

% % \paragraph{Rogue \acf{ap}} 
% % \label{sec:def-rogue-ap}
% % To mitigate this threat where unauthorised WiFi hotspots intercept and potentially alter transaction details (\autoref{sec:rogue-ap}), an extra layer of security can be implemented within the wallet network, hindering the adversary's \teal{$txn$} modification attempt \cite{cai2014research}. 

% % Cai et al \cite{cai2014research} propose a two-factor-based dynamic password technology designed to enhance network security against rogue AP attacks.

% % \paragraph{Web-based Infection}
% % \label{sec:def-web-infection}

% % In response to web-based infections (\autoref{sec:web-infection}), wallet applications should incorporate \acf{ids} that can identify and alert suspicious activities \cite{zimba2019cryptojacking}. These systems should be configured to monitor for unusual network traffic and script executions that may indicate a web-based infection attempt.

% % \paragraph{\acs{arp} Spoofing}
% % \label{sec:def-arp-spoofing}

% % To prevent a MAC-\acs{ip} address linkage by an adversary through \acs{arp} spoofing, the wallet network's operating parameters can be dynamically adjusted to detect malicious network traffic \cite{girdler2021implementing}.

% % \paragraph{\acf{dns} Spoofing}
% % \label{sec:def-dns-spoofing}

% % The redirection of the wallet devices to fraudulent websites through the compromise of the \acs{dns} revolver \autoref{sec:dns-spoofing} can be counteracted by employing machine learning algorithms to identify suspicious network activities \cite{ahmed2017mitigating}.

% % \paragraph{\acf{ip} Spoofing}
% % \label{sec:def-ip-spoofing}

% % To counteract this attack, wallets can implement network security protocols that validate and authenticate IP to block unauthorised access attempts \cite{rengarajan2016secure}. This \acs{ip} authentication mitigation ensures that only traffic from verified and trustworthy IP addresses can interact with the wallet.

% \subsubsection{\acf{dos}}
% \label{sec:def-dos}

% This attack can be mitigated by distinguishing between malicious and authentic network traffic. Two notable traffic distinguishing techniques are using classifiers such as the decision tree algorithm \cite{khan2019adaptive} and using reinforcement learning approaches to analyse patterns in network data \cite{liu2018deep}. Another mitigation approach is analysing the network for unusual patterns, such as repeated request attempts from the same \acs{ip} address \cite{sathwara2017distributed}.

% % \acs{dos} attacks delivered via botnet attacks can be detected by filtering the network traffic to identify potential bot activity and using classifiers such as the decision tree algorithm to distinguish between benign and malicious traffic \cite{khan2019adaptive}.

% % To prevent downtime

% % \paragraph{\acs{icmp} Flooding}
% % \label{sec:def-icmp-flooding}

% % To prevent downtime and stop excessive traffic from an adversary executing this attack, wallets can use reinforcement learning approaches to analyse patterns in network data to stop malicious traffic while allowing authentic network traffic \cite{liu2018deep}.

% % \paragraph{\acs{tcp} \acs{syn} Flooding}
% % \label{sec:def-tcp-flooding}

% % An effective strategy for mitigating the \acs{tcp} \acs{syn} flooding attack which limits the wallet's operation (\autoref{sec:wallet_mechanism}) involves network analysis for unusual patterns, such as repeated request attempts from the same \acs{ip} address. Following this, the server responds with \acs{tcp}-RST packets to halt these attempts, to prevent network downtime \cite{sathwara2017distributed}.

% \subsection{Defence against Application Attacks}

% % Barber et al. \cite{barber2012bitter} address Bitcoin vulnerabilities by promoting threshold cryptography to split and secure private keys, and introducing super wallets for dividing assets across devices. Android's touch filter mechanism, though effective against clickjacking, is limited against scan-and-pay attacks, leading to suggestions by Ulqinaku et al. \cite{ulqinaku2019scan} for OS modifications for better recognition of malicious overlays. Rezaeighaleh et al. \cite{rezaeighaleh2019new} present a layered wallet model, enhancing security by separating the storage and transaction functionalities. Homoliak et al. \cite{homoliak2020security} propose a secure Ethereum transaction system involving authenticators and smart contracts. The Bitcoin Security Rectifier by Hu et al. \cite{hu2020securing} aims to improve BitcoinJ library security. Lastly, Takahashi et al. \cite{takahashi2019multiple} advocate for a detailed analysis of apps using a blend of static, dynamic, and semantic methods.

% \subsubsection{Malware}
% \label{sec:def-malware}

% Detecting this attack effectively requires the analysis of malware behaviours, system interactions, and network communications.
% Employing machine learning classifiers \cite{balakrishnan2023analysis}, anomaly detection models \cite{kapoor2021ransomware} and intrusion detection systems \cite{guri2018beatcoin} have emerged as effective detection methods. Following detection, the malware file can be tracked and neutralised. 
% % Clipboard hijacking malware can be significantly limited by implementing human-readable address names.

% % \paragraph{Clipboard Hijacker}
% % \label{sec:def-clipboard}

% % Wallet can address this attack by employing novel features which prevent copying and pasting the \teal{$recipient\_addr$} during the transaction creation stage (\autoref{sec:transaction_signing}) \cite{li2020android}. These address replacement features include NFC, two-dimensional codes and human-readable address names. 

% % \paragraph{Keylogger}
% % \label{sec:def-keylogger}

% % Keyloggers, known for their hidden operations, present detection challenges. Despite this, employing advanced machine learning classifiers has emerged as a potent strategy for identifying and neutralising such threats\cite{balakrishnan2023analysis}.

% % \paragraph{Ransomware}
% % \label{sec:def-ransomware}

% % Detecting ransomware effectively requires the analysis of malware behaviours, system interactions, and network communications. Advanced machine learning techniques, such as anomaly detection models and supervised learning algorithms play a critical role in identifying potential ransomware activities \cite{kapoor2021ransomware}. However, while detection is crucial, secure backup and restore implementations prevent attacks. 


% % This approach ensures that, in the event of encryption by ransomware, affected files can be promptly and effortlessly restored.

% % \paragraph{Spyware}
% % \label{sec:def-spyware}
% % Spyware is particularly hard to detect as its hidden operations present challenges. Despite this, employing machine learning classifiers has emerged as an effective detection \cite{balakrishnan2023analysis}. Following detection, the spyware file can be tracked and neutralised.

% % \paragraph{Supply Chain Attack}
% % \label{sec:def-supply}

% % To enhance resilience against these attacks, emphasising early detection and response is crucial. This involves monitoring network data flows to identify unauthorised data access or unusual communication patterns that deviate from established norms \cite{wang2021feasibility}. By comparing current traffic behaviours against a pre-determined baseline of typical network activity, any unusual data movements or suspected malicious interactions can be identified.

% % -- extended explanation 
% % This method relies on the consistent aspects of traffic data, allowing for the detection of anomalies, even in communications that are encrypted, to track and uncover the mechanisms of such attacks effectively.

% % \paragraph{Phishing}
% % \label{sec:def-phishing}


% % To enhance security against phishing, wallets should incorporate phishing-resistant multi-factor authentication (MFA) techniques such as FIDO2 \cite{wang2021feasibility}. This feature prevents phishing by communicating with the original wallet website to verify the authenticity of the illegitimate one before allowing access to the wallet \cite{fido2}. Therefore, in scenarios where users are attack victims, the adversary will not bypass the other authentication methods. Additionally, users should verify the URLs of wallet browsers and exchange services before using them \cite{weichbroth2023security}.

% % \paragraph{Removable Media Infection}
% % \label{sec:def-removable}

% % Malware delivered to hardware wallets via removable media can be addressed by implementing intrusion detection and prevention systems \cite{guri2018beatcoin}. Furthermore, anti-virus programs can be installed on these wallet devices. 

% \subsubsection{Social Engineering}
% \label{sec:soc-engineering}
% \paragraph{Phishing}
% \label{sec:def-phishing}

% To enhance security against phishing, wallets should incorporate phishing-resistant multi-factor authentication (MFA) techniques such as FIDO2 \cite{wang2021feasibility}. This feature prevents phishing by communicating with the original wallet website to verify the authenticity of the illegitimate one before allowing access to the wallet \cite{fido2}. Additionally, users should verify the URLs of wallet browsers and exchange services before using them \cite{weichbroth2023security}.

% \subsubsection{Privilege Escalation}
% \label{sec:def-privilege}

% \paragraph{Android Root Exploitation}
% \label{sec:def-android-root}

% To safeguard mobile wallets from this attack, employing code obfuscation is essential. This technique scrambles the application's code without altering its functionality \cite{indusface}. 

% % Consequently, adversaries are unable to understand the application's inner workings, providing a robust defence against unauthorised access and the exploitation of sensitive data.

% \paragraph{Debugger}
% \label{sec:def-debugger}

% Implementing a monitoring feature to manage and restrict debugging settings mitigates this attack \cite{li2020android}. Additionally, employing runtime protection mechanisms helps detect and respond to unauthorised debugging attempts \cite{qi2012spad}.

% \paragraph{Logic Flow Exploitation}
% \label{sec:def-logic-flow}

% By enforcing \acf{cfi} measures, wallets can ensure that malicious transactions cannot be executed or control flow hijacked to deviate from the intended control flow paths \cite{creech2017new}. 

% % CFI works by inserting checks before indirect control flow transfers (like indirect function calls and returns) during program execution. These checks validate that the destination of the control flow transfer is within the set of allowed targets as defined by the CFG. If an attempted control flow transfer does not correspond to an allowed path in the CFG, it is considered an attack attempt, and the execution can be halted or redirected to a safe state.

% \subsection{Defence against Authentication Attacks}

% % Authentication-based attacks can be generally counteracted by employing transaction management (see \autoref{sec:transaction_management}) security features which require more than one party to sign transactions and reduce a single point of failure. An example of such security features includes multi-signatory implementation and \acf{mpc} \cite{lindell2020secure}.

% % Multi-signature features reduce the single point of failure. Bitfinex, a cryptocurrency exchange has been vulnerable to a multi-signatory implementation whereby 2 out of the 3 keys required to sign a transaction were on one device \cite{protos}.

% % Khan et al. \cite{8966739} propose using complex passwords to increase cracking difficulty, limiting login attempts, and implementing Captchas to prevent automated access. Bulut et al. \cite{bulut2020security} emphasize the need for two-factor authentication and complex passwords against brute-force and dictionary attacks. Jasim et al. \cite{jasim2019enhancing} improve Brainwallet security by adding entropy to the master seed. Hu et al. \cite{hu2020securing} introduce a continuous verification method using mouse behavior biometrics. The BioWallet \cite{benli2017biowallet} utilizes a two-step authentication process, combining fingerprint verification with traditional password mechanisms, enhancing security and ensuring user identity validation for each transaction. Gentile et al. \cite{gentilal2017trustzone} advocate for a Trusted Execution Environment, segregating wallet layers to secure sensitive data and operations, thereby ensuring robust protection against unauthorised access and data manipulation.

% \subsubsection{Brute Force}
% \label{sec:def-brute-force}

% To reduce the possibility of an adversary bypassing the wallet authentication described in \autoref{sec:key-storage} using this attack vector, users should adopt more complex passwords \cite{praitheeshan2019attainable}.

% % -- Yathin's Old Version
% % These attacks on crypto-wallets involve systematically trying all possible combinations of characters or words to crack the passphrase or access the wallet \cite{rezaeighaleh2019new}. In this offline approach, the attacker uses a large dataset of words and generates a dictionary file for the brute force attack. The process is time-consuming and requires significant computational resources, often utilizing multiple machines or virtual machines operating in parallel \cite{volety2019cracking}. By attempting to crack the wallet using various combinations, the attacker identifies valid combinations that could potentially access the wallet. However, the success rate decreases as the number of combinations increases. The enormous number of possible combinations, especially when dealing with long passphrases or extensive wordlists, makes brute force attacks highly time-consuming and practically infeasible \cite{vasek2017bitcoin, praitheeshan2019attainable}.

% \subsubsection{Dictionary Attacks}
% \label{sec:def-dictionary}

% Similar to brute force, this attack can be neutralised by adopting advanced similar methods such as complex passphrases and mnemonic phrases, which significantly reduce the attacker's success chances \cite{aldawood2020advanced} or by integrating improved functionalities within devices to obstruct access to predictive text data \cite{uddin2021horus}.

% % These attack shares the same goal as brute force attacks and can be effectively neutralised by adopting advanced similar methods such as complex passphrases and mnemonic phrases. These measures significantly diminish the attack's success probability by complicating predictable patterns \cite{aldawood2020advanced}. Additionally, integrating custom keyboard functionalities within wallet devices obstructs the attackers' access to predictive text data, thereby, enhancing security against these attacks \cite{uddin2021horus}.

% % -- Yathin's Old Version
% % This exploits the user dictionary feature present in keyboard apps to predict the mnemonic phrase or passphrase used in the mobile wallet (see \autoref{sec:mobile-wallets}). Wallet apps typically rely on the default keyboard, which uses a user dictionary for predictive text inputs. The mnemonic phrase consists of common English words, and the information regarding these words is saved in the user dictionary. An attacker app with virtual keyboard permission can access the dictionary and extract frequency information of typed words to predict the mnemonic phrase. Only a small percentage of wallet apps (6\%) have implemented custom keyboards to protect against dictionary attacks \cite{uddin2021horus}. The effectiveness of dictionary attacks can be enhanced by building dictionaries based on leaked password datasets and analyzing the frequency of occurrence of passwords \cite{praitheeshan2019attainable}. Smarter guessing techniques can also increase the success rate \cite{holmes2023framework}.

% % \subsubsection{Code Reuse}
% % \label{sec:def-code-reuse}

% % An effective way of counteracting these attacks is by using \acf{cfi} which prevents the adversary from altering the execution path of a wallet and ensures only legitimate sequences of are executed \cite{burow2017control}. Additionally, formal verification can be used to check the correctness of control flows \cite{liu2019survey}. 

%  % An adversary can exploit existing code or logic constructs of a wallet to perform unauthorised actions or achieve malicious outcomes \cite{bletsch2011jump}. These attacks exploit the intended functionality of software components to alter execution flow and bypass security measures without necessarily injecting new code \cite{palladino2017parity}.

%  \subsubsection{Fake Biometrics}
% \label{sec:def-fake-biometrics}

% To prevent synthetic biometric data, biometric authentication security can be enhanced by incorporating liveness detection features \cite{galbally2013image}.


% % Wallet devices can enhance the security of biometric authentication by using liveness detection features. This feature enables a liveness assessment of the user to decrypt the \teal{$sk$} in the wallet storage mechanism to prevent synthetic or replicated biometric data \cite{galbally2013image}.


 
 
% \subsubsection{Evil Maid Attack}
% \label{sec:def-evil-maid}

% % Wallets can implement trusted boot mechanisms to counteract this attack. Trusted boot rigorously verifies the integrity of software components during the boot process, allowing only those validated by their cryptographic signatures to execute \cite{tereshkin2010evil}. This effectively shields the device from unauthorised changes that could otherwise exploit the system to retrieve credentials. 

% To shield the device from unauthorised changes, wallets could implement trusted boot mechanisms, which verify the integrity of components during the boot process, allowing only those validated by their cryptographic signatures to execute \cite{tereshkin2010evil}.

% % \subsubsection{Shoulder Surfing}
% % \label{sec:def-shoulder-surfing}

% % Shoulder surfing and similarly related observation attacks can be mitigated by employing \acf{mfa}. \acs{mfa} requires verification beyond what the adversary can see, such as one-time codes sent to the user's device, to significantly reduce the attack's effectiveness \cite{aratani2015authentication}.

% \subsection{Defence against Storage \& Memory Attacks}

% % \subsubsection{Fault Injection}
% % \label{sec:def-fault-inj}

% % A dual-layered defence strategy can used against fault injection attacks by combining formal verification and runtime protection mechanisms to provide prior and real-time mitigation \cite{hajdu2020using}. Before code deployment, formal verification proves the correctness of the wallet's code, as described in \autoref{sec:wallet_init}. Runtime protection mechanisms, on the other hand, monitor the system's operation in real time to detect and mitigate any attempted attacks or anomalies that occur post-deployment. 


% % additional information
% % This combination not only validates the integrity of the wallet's software from the outset but also offers continuous safeguarding against emerging threats, thereby maintaining the blockchain system's reliability and security against fault injection exploits [Á. Hajdu et al., 2020]

% % can also potentially add self-debugging 
% % found here --
% % Tightly-coupled Self-debugging Software Protection
% %  (Abrath et al., 2016)

% \subsubsection{Cold Boot Attack}
% \label{sec:def-cold-boot}

% Employing features which clear the wallet's memory following sensing physical tampering can prevent this attack \cite{seol2019amnesiac}. For example, Ledger has introduced a secure layer which detects chip intrusion and erases the private key following extraction attempts \cite{ledgerwallet}.

% \subsubsection{Row Hammer Attack}
% \label{sec:def-row-hammer}

% To prevent this, wallets can randomly exchange memory rows to protect the \acs{dram} against attacks \cite{saileshwar2022randomized}. Additionally, row swaps can be detected and mitigated using machine learning techniques \cite{joardar2022learning}.

% \subsubsection{Microscopy}
% \label{sec:def-microscopy}

% Incorporating \acf{puf} which generates cryptographic keys on-demand, without storing the \teal{$sk$} on the wallet's chip prevents this attack as well as other invasive key extraction attacks \cite{urien2021innovative}.

% % Given that microscopy attacks are a form of invasive attack, incorporating \acf{puf} is highly effective. \acs{puf} harness the inherent, unique properties of semiconductor devices to generate cryptographic keys on-demand, without ever storing these keys on the chip \cite{urien2021innovative}. This method significantly reduces the risk of key extraction via invasive techniques. Therefore, an adversary who aims to conduct a microscopy attack will find it impossible to retrieve the \teal{$sk$}.

% % \subsubsection{Probing}
% % \label{sec:def-probing}

% % Develop more resilient cryptographic devices \cite{wei2015vulnerability}


% \subsection{Defence against Cryptanalysis Attacks}


% \subsubsection{Weak Signature Exploit}
% \label{sec:def-weak-sig}

% % This attack targets vulnerabilities in the \hyperref[algo:key-generation]{key generation algorithm} or \hyperref[algo:transaction-signing]{signing algorithm} due to improper implementation \cite{rokhjavan2023securing}. The vulnerabilities may specifically be a product of weak or outdated cryptographic algorithms (detailed in \autoref{Encryption-Table-1}) or errors in encryption logic.

% Attacks that aim to exploit weak cryptographic signatures can be counteracted by employing stronger hashing algorithms \cite{rokhjavan2023securing}.

% \subsubsection{Nonce Reuse}
% \label{sec:def-nonse-reuse}

% Mitigation against nonce reuse attacks can be implemented by ensuring the deterministic selection of \teal{$nonce$} to prevent private key leakage \cite{brengel2018identifying}.

% \subsubsection{Side Channel Attacks}
% \label{sec:def-side-channel}

% Protection of data leakage points and disguising data access patterns prevent micro-architectural and timing-based side-channel attacks respectively \cite{lou2021survey, ali2023characterization}. These disrupt the attackers' ability to interpret leaked information effectively \cite{mosquera2023guard}.


% % The broad spectrum of information leakage, including power consumption patterns, timing discrepancies, and other observable data, presents significant obstacles to defending wallets \cite{lou2021survey, ali2023characterization}. For micro-architectural vulnerabilities, proactive identification and protection of potential leakage points are crucial \cite{lou2021survey}. In addition, mitigating data access attacks involves strategies to obscure the timing of operations, safeguarding against attacks that leverage time discrepancies \cite{ali2023characterization}. Disguising data access patterns also serves as a critical strategy in thwarting time-based side-channel attacks, thereby disrupting attackers' ability to interpret leaked information effectively \cite{mosquera2023guard}



% Other Features

% \subsubsection{Multi-Factor Key Derivation Function (MFKDF)}
% \label{sec:mfkdf}
% The MFKDF \cite{nair2023multi, nair2023decentralizing} is an advanced method for creating secure cryptographic keys, similar to complex passwords, using multiple types of security measures. This method is more secure and versatile compared to traditional approaches as it can integrate various authentication factors like one-time passwords, and tokens from devices like YubiKey, and others. A significant advantage of this method is that it moves away from the conventional use of mnemonics typically used in crypto wallets. By doing so, it greatly enhances the user experience, making it both simpler and more intuitive. Additionally, MFKDF empowers users with the ability to recover their accounts on their own, without needing a central master key. This self-service recovery is facilitated through a secret-sharing approach, where the key is divided into parts and only a certain number (K out of N) of these parts are needed to reconstruct the full key. This combination of ease of use, security, and user-friendly account recovery marks a significant improvement in managing access to crypto wallets.

% Ensuring the use of robust, up-to-date cryptographic algorithms and practices, proper key management, and thorough validation of signatures to resist exploitation

% it's critical to enhance both algorithmic integrity and implementation security. As outlined by Rokhjavan (2023), ensuring rigorous validation within key generation and signing algorithms, along with enforcing appropriate encryption key sizes, are fundamental steps towards securing signatures. 

% Rokhjavan (2023) advocates for rigorous validation mechanisms within key generation and signing processes, emphasizing the need for robust encryption key size checks to preserve signature integrity.

% Using more secure and tested cryptographic schemes


% Upholding cryptographic standards and best practices

% Cortez et al. \cite{cortez2020cryptanalysis} advocate for employing strong cryptographic algorithms with proven security properties. Prajapat et al. \cite{prajapat2016avk} highlight the importance of robust key management practices, with keys generated, stored, and used securely. Prajapat and Thakur \cite{prajapat2015various} suggest designing wallets to resist common cryptanalytic attacks by using sufficiently long and complex keys. Aciiçmez et al. \cite{aciiccmez2007micro} recommend employing techniques like threshold cryptography to split and secure private keys. Tomassini and Perrenoud \cite{tomassini2001cryptography} advise conducting regular security audits and cryptographic analysis of the wallet's implementation to identify and mitigate potential vulnerabilities.






% ---




% \subsection{Old Defence Features}
% \label{sec:wallet-features}

% % \subsection{Other Wallet Defence Features}
% % \label{sec:wallet-features}

% \subsubsection{M-of-N Standard Transactions}
% \label{sec:bip-11}
% Bitcoin Improvement Proposal 11 (BIP-11) \cite{bip11} introduces a method for Bitcoin transactions that requires multiple signatures before the transaction can be completed. Consider a collective decision-making process, wherein a specified subset of individuals (M) from a larger group (N) is required to reach a consensus before initiating any action. 

% % However, the initial key from which all others are derived (the root seed) must be very well protected, as its security is crucial for the entire wallet structure. This approach allows for a flexible and secure way of managing wallets, but users must be careful to safeguard their root seeds.

% % \subsubsection{Mnemonic Code for HD Wallets}
% % \label{sec:bip-39}
% % Bitcoin Improvement Proposal 39 (BIP-39) \cite{bip39} introduces a method for creating easy-to-remember recovery phrases for Bitcoin wallets. These phrases, made up of a series of simple words, are based on strong cryptographic principles. The words are chosen from a specific list to ensure security and ease of use. The idea is to make these phrases both random enough to be secure and simple enough for people to remember and write down. To add extra security, a technique called \quotes{salting} and a process called PBKDF2 (Password Based Key Derivation Function 2 is a cryptography function using hashing messages and values several times to produce keys.) are used, which make it harder for attackers to guess these phrases. The security of these phrases also depends on how carefully users handle them – if someone uses them carelessly, they can still be vulnerable to attacks. Overall, BIP-39 aims to provide a balance between memorability, ease of writing down, and randomness to create secure backup phrases for wallet users.

% % \subsubsection{Multi-Account Hierarchy}
% % \label{sec:bip-44}
% % Bitcoin Improvement Proposal 44 (BIP-44) \cite{bip44} builds on the ideas of BIP-32, which introduced a hierarchical structure for Bitcoin wallets. BIP-44 takes this a step further by organizing these wallets into separate accounts, ensuring that the information or assets in one account don't accidentally get mixed up with another. It uses special techniques like \quotes{account discovery} and \quotes{gap limits} to manage these accounts efficiently and securely. These features make it easier for users to have multiple wallets for different purposes, all under one main account, without compromising on security. However, just like in BIP-32, the main concern is still about protecting the initial key (root seed) from which all other wallets are derived. While BIP-44 is great for organizing multiple wallets and sharing them selectively, the overall security also depends on how the user manages their main key.

% \subsubsection{Multi-party computation (MPC)}
% \label{sec:mpc-wallets}
% MPC wallets \cite{canetti2020uc} represents a new and advanced way to make crypto wallets more secure, especially in how they handle private keys (the critical component for accessing and using cryptocurrencies). MPC works by using sophisticated cryptographic methods that allow several people or entities to work together on a calculation or process without having to share their individual pieces of information. When applied to crypto wallets, MPC splits the control of the private key among multiple parties. This means that no single person or entity holds the entire key, making it much harder for hackers to gain access to the wallet. The advantage of this approach is that it reduces the risk that comes from having all the security depend on just one point (like a single password or key), significantly enhancing overall security.

% \subsubsection{Account Abstraction Using Alt Mempool}
% \label{sec:erc-4337}
% Ethereum Request for Comments 4337 (ERC-4337) \cite{erc4337} is a proposal that aims to make Ethereum accounts more flexible and secure without needing to change the core rules of the Ethereum network. It introduces a new system that works on top of the existing network, using a two-step process (validate and execute) for transactions. This approach makes transactions more efficient and helps prevent certain types of attacks where transaction fees can be manipulated. ERC-4337 also includes measures to ensure that the execution of transactions is consistent with their validation, and it allows for more versatile ways of managing transaction orders than the traditional method. 




\section{Case Studies}
\label{sec:case_study}

In this section, we present detailed case studies of notable wallet security breaches. We apply our wallet design taxonomy (\autoref{sec:wallet-taxonomy}), threat model (\autoref{sec:threat_framework}), and attack taxonomy (\autoref{sec:attack-framework}). Each case study systematically analyses the wallet's architecture, identifies exploited vulnerabilities, and explores the sequence of attack events. We conclude each study with recommended and implemented security measures.

\subsection{Case Study: ByBit Custodial Wallet Hack}
\label{sec:bybit_case}

In February 2025, ByBit experienced a significant security breach that resulted in a loss of approximately \$1.5 billion in Ethereum, marking the largest cryptocurrency theft to date \cite{bybit}. This sophisticated attack aligns with the attack vectors outlined by our taxonomy. We provide a detailed analysis below using our frameworks for design classification, threat assessment, attack sequence analysis, and mitigation strategies.

\subsubsection{Design}
\label{sec:bybit_mech}

Using our design taxonomy in \autoref{sec:wallet-taxonomy}, we analyse the ByBit wallet design below:


\begin{itemize}
    \item \textbf{Custody:} ByBit maintained full custody of user funds, with users relinquishing \textcolor{teal}{\textit{sk}} control to the exchange. This particular case pertains to the \textcolor{teal}{\textit{sk}}, which controlled the Ethereum assets of the exchange. 
    \item \textbf{Infrastructure:} 
    ByBit employed a multi-faceted infrastructure design, integrating hardware wallets with a smart contract-enabled proxy architecture. The primary proxy contract delegated logic execution to a separate implementation contract via \texttt{delegateCall}. It stored the implementation contract's address in storage slot 0 to facilitate future upgrades \cite{bybit_secux}. However, the design did not enforce strict access controls on this critical operation. This became a key factor exploited in the attack, as described in the threat analysis (see \autoref{sec:bybit_dep}).


    \item \textbf{Distribution:} \textcolor{teal}{\textit{sk}} management was distributed securely with authorisation rights shared among multiple private key (\textcolor{teal}{\textit{sk}}) holders in the multi-sig scheme across different hardware devices. The multi-signature scheme prevented unilateral transactions, mandating consensus among multiple trusted individuals. 
    \item \textbf{Authorisation:} Transactions were generated via Safe's web interface. Signers reviewed transaction details on the web user interface and hardware wallet screens. Only after confirmation on their Ledger hardware wallet devices were transactions broadcast to the blockchain.
    \item \textbf{Validation:} After obtaining the necessary approvals, transactions underwent validation to ensure compliance with ByBit's internal security policies. This included verifying adherence to address whitelisting protocols and transfer limits. The multi-sig smart contract enforced these policies by executing transactions only when the requisite number of valid signatures was present.
\end{itemize}


\subsubsection{Threats and Dependencies}
\label{sec:bybit_dep}

ByBit’s security architecture relied significantly on several interconnected elements, including the Safe user interface, which proved vulnerable to the adversaries' attempts. We outline the threats, which were exploited by the adversary inline with our threat model below:

\begin{itemize}
    \item \textbf{Insecure Interaction:} Insecure interactions resulted in the system's exposure to threats. The adversary likely exploited these interactions to achieve infiltration of the Safe developer's machine \cite{bybit_certik}. 
    \item \textbf{Application Provider Compromise:} ByBit's operational security was heavily dependent on the integrity and security posture of third-party service providers, in this case, Safe’s web interface.
    \item \textbf{Data Misrepresentation:} The adversary compromised the accuracy and reliability of transaction data presented to authorised signers through Safe's user interface. This highlighted a critical vulnerability in wallet user interfaces.
    \item \textbf{Application Logic Flaw:} The infrastructure design permitted unrestricted use of the \texttt{delegateCall} instruction, allowing malicious actors to overwrite critical storage slots. Specifically, the attackers exploited the ability to overwrite the logic pointer stored in storage slot 0, leading to unauthorised control of the proxy's logic \cite{bybit_certik}. This violated the principle of least privilege and directly facilitated the privilege escalation step of the attack.
    \item \textbf{Blind Signing:} ByBit's reliance on hardware wallet confirmation processes did not sufficiently address the blind signing risk. Signers assumed the hardware wallet displays were a trustworthy verification source and approved transactions without explicit visibility into critical transaction metadata. This included \texttt{delegateCall} operations and underlying implementation changes. 

\end{itemize}



\subsubsection{Adversary Goal and Capabilities}
\label{sec:bybit_cap}

\textcolor{teal}{\textit{A}} aimed to gain unauthorised rights by masking adversary-created transactions as benign. The capabilities of \textcolor{teal}{\textit{A}} significantly evolved during the attack as extended knowledge was gained, starting from restricted external knowledge and progressing to insider-level knowledge and access:

\begin{itemize}
    \item \textbf{Initial Phase:} \textcolor{teal}{\textit{A}} remotely exploited publicly accessible information to exploit Safe developer interactions and gain restricted internal access.
    \item \textbf{Intermediate Phase:} Having achieved insider-level knowledge and privileges following a successful repository compromise, \textcolor{teal}{\textit{A}} could inject malicious software into operational components of the wallet software.
    \item \textbf{Final Phase:} \textcolor{teal}{\textit{A}} could exploit application logic to deceive \textcolor{teal}{\textit{sk}} holders, achieving credential compromise. Subsequently, \textcolor{teal}{\textit{A}} gained full wallet control and authorisation rights.
\end{itemize}

\subsubsection{Attack Sequence}
\label{sec:bybit_att}

The ByBit incident represents a sophisticated combination of several coordinated attack vectors identified in our Application threats taxonomy:

\begin{itemize}
    \item \textbf{Social Engineering:} A phishing attack method enabled the execution of subsequent attack vectors. Social engineering and malware were combined to compromise ByBit, as seen in past incidents (e.g., BitKeep \cite{CertiKIncidents}, Upbit \cite{UpbitMedium}, and wallet drainers \cite{RektREKT}). This gave the adversary direct access to Safe's front-end code repository, highlighting the importance of secure developer environments.
    \item \textbf{Malware Execution:} The compromised machine enabled the injection of malicious JavaScript into Safe's front-end code, targeting the transaction approval interface. The malware modified the transaction data displayed to \textcolor{teal}{\textit{sk}} holders. While legitimate transaction details were displayed in the Safe wallet user interface, the data sent to the hardware wallet was altered.
    \item \textbf{Privilege Escalation:} 
    The approved transaction altered the smart contract's logic. The attackers exploited storage slot hijacking by crafting a transaction that used \texttt{delegateCall} to execute a spoofing contract. This contract’s \texttt{transfer()} function wrote the attacker’s malicious implementation address to storage slot 0 via the \acf{evm} \texttt{SSTORE} opcode, overwriting the proxy’s logic pointer. With the proxy now delegating to the attacker’s contract, all subsequent transactions executed attacker-controlled code in the proxy’s context, granting full authorisation rights.


\end{itemize}

\subsubsection{Security Measures}
\label{sec:bybit_def}

Before the breach, ByBit used a layered security model: most funds were in a Safe contract, private keys on six Ledger devices, requiring 4-of-6 multi-sig. These measures were bypassed. After the incident, industry experts highlighted the following additional controls:

\begin{itemize}
    \item \textbf{Independent Transaction Hash Verification:} The use of tools such as \texttt{safe-tx-hashes} to independently verify transaction hashes against on-chain data mitigates the risk of UI-level deception \cite{bybit_cyfrin}. By enabling signers to cross-reference actual transaction payloads outside of potentially compromised interfaces, this approach detects malicious operations such as unauthorised \texttt{delegateCall} or logic pointer overwrites before execution.
    \item \textbf{Transaction Policy Enforcement via On-Chain Gatekeeping:} Preventative solutions such as Halborn’s Seraph simulate signed transactions before execution and block operations that violate predefined organisational policies \cite{bybit}. In the context of the ByBit attack, this approach could have flagged and halted the unauthorised upgrade triggered by the malicious \texttt{delegateCall}, enforcing a secondary layer of validation beyond signer intent.
    \item \textbf{Hardware Wallet Clear-Signing:} Require devices that support the on-device display of the complete destination, value, function selector, and raw calldata (clear-signing) before approval. This enables signers can independently verify every field and avoid hash-only blind signing, a weakness exploited in the ByBit breach \cite{bybit_secux}.    
    \item \textbf{Wallet Auditing:} Conducting regular audits focusing on storage layout consistency and \texttt{delegateCall} whitelisting and other wallet-related code is pertinent \cite{bybit_slowmist}
\end{itemize}


\subsection{Case Study: Slope Non-Custodial Wallet Hack} \label{sec:slope_case}

In August 2022, Slope Wallet experienced a severe security incident, resulting in the compromise of over 9,200 user wallets on the Solana blockchain and a loss of approximately \$4.1 million in SOL and USDC \cite{CoinTelegraph2022SlopeAttack}. We provide a detailed analysis below using our frameworks for design classification, threat assessment, attack sequence analysis, and implemented security measures.

\subsubsection{Design} \label{sec:slope_design}

Applying our design taxonomy, we analyse the Slope wallet design below:

\begin{itemize} 
\item \textbf{Custody:} Slope utilised a non-custodial model where users retained complete control over the private key (\textcolor{teal}{\textit{sk}}). This case pertains to the management and leakage of the user's private key.
\item \textbf{Infrastructure:} Slope used a mobile software wallet that relied on a self-hosted Sentry monitoring stack \cite{cyberintel_slope, fyeo_slope}. This setup collected application data for debugging but inadvertently logged sensitive information due to a faulty logging function.
\item \textbf{Distribution:} Slope used a single-distribution model, with all cryptographic operations and storage conducted solely on the user’s mobile device. No advanced key distribution methods, such as MPC or multi-signature schemes, were integrated. 
\item \textbf{Authorisation and Validation:} The standard Solana \textit{Ed25519} signature flow was executed locally on the user device. Transaction broadcasting was performed via Slope's own \acs{rpc} endpoints.
 \end{itemize}

\subsubsection{Threats and Dependencies} \label{sec:slope_dep}

Slope’s security architecture relied heavily on interconnected dependencies, particularly its integrated application-monitoring stack, as detailed below:
\begin{itemize} \item \textbf{Application‑Monitoring Dependency:} Slope utilised an on-premise implementation of the Sentry SDK, designed to assist developers in debugging. A single improperly added \texttt{toString()} method circumvented built-in security filters, resulting in sensitive wallet private keys being unintentionally logged in plaintext \cite{cyberintel_slope}.
\item \textbf{Data Leakage:} Multiple defensive measures were used (collection filtering, \acf{tls} certificate pinning, database encryption at rest). However, collection filtering and database encryption were disabled, causing plaintext private keys to be stored in the database.
\item \textbf{Third‑Party Supply‑Chain Threat:} Slope employed a self-hosted version of the third-party monitoring solution (Sentry), inheriting risks associated with configuration drift, patch management latency, and internal operational errors. This on-premise deployment introduced vulnerabilities typically mitigated by a SaaS-managed setup.
\item \textbf{Insecure User Interaction:} Users continued to interact with wallets whose keys had potentially been exfiltrated. No built-in key-rotation prompt existed. \end{itemize}

\subsubsection{Adversary Goal and Capabilities} \label{sec:slope_cap}

The adversary, \textcolor{teal}{\textit{A}}, aimed primarily for credential compromise, specifically targeting the user's private key (\textcolor{teal}{\textit{sk}}). The capabilities leveraged by \textcolor{teal}{\textit{A}} included:

\begin{itemize} 
\item \textbf{Initial Phase:} \textcolor{teal}{\textit{A}} used knowledge of Slope’s logging vulnerability (via reverse engineering or insider information) to target the timeframe and method to extract logged private keys. 
\item \textbf{Intermediate Phase:} \textcolor{teal}{\textit{A}} employed remote network access, either directly to the internal database or by intercepting \acs{tls} traffic prior to 18 July 2022. This remote capability allowed the extraction of plaintext private keys despite the security measures initially in place.
\item \textbf{Final Phase:} \textcolor{teal}{\textit{A}} employed legitimate wallet signing authority using stolen keys and subsequently drained user funds directly via standard blockchain transactions without triggering conventional anomaly detection.
\end{itemize}

\subsubsection{Attack Sequence} \label{sec:slope_att}

In this incident, the adversary employed the logic exploitation vector to compromise credentials, as summarised below:

\begin{itemize} 
\item \textbf{Logic Bug Introduction:} Slope utilised a helper function (\texttt{toString()}) to streamline debugging, unintentionally bypassing established security filters. This bug directly caused private keys to enter plaintext logging pipelines.
\item \textbf{Data Pipeline Restart:} Slope utilised Kafka for data processing. After restarting, Kafka inadvertently flushed cached logs containing private keys in plaintext format directly into a PostgreSQL database.
\item \textbf{Log Exfiltration:} \textcolor{teal}{\textit{A}} accessed the misconfigured Sentry instance and retrieved the plaintext seed phrases, fully compromising user private keys.
\item \textbf{Wallet draining:} \textcolor{teal}{\textit{A}} utilised legitimate signing authority gained from compromised private keys and drained assets from 9,229 wallet addresses within seven hours.

\end{itemize}

\subsubsection{Security Measures
} 
\label{sec:slope_def}

Following the Slope Wallet breach, the development team initiated several immediate reactive security measures. The team promptly disabled the self-hosted Sentry server within 15 minutes of identifying the vulnerability and advised users to transfer their assets to new wallets \cite{dailycoin_slope}. Additionally, audits conducted by \href{https://osec.io/}{OtterSec} and \href{https://www.slowmist.com/}{Slowmist} confirmed that sensitive data, including private keys, had been inadvertently logged \cite{ottersec_slope, zellic_slope}. In response, Slope removed all sensitive logging functionalities and implemented a \texttt{beforeSend} whitelist to filter out confidential information \cite{slope_statement}. 

To prevent such incidents in the future, it is crucial to ensure that application monitoring tools, such as Sentry, are meticulously configured to exclude sensitive data from logs. This involves implementing stringent data scrubbing protocols and avoiding the logging of private keys or seed phrases. Proper calibration of these safeguards is essential for preserving the confidentiality and integrity of user credential data.





\section{Insights}
\label{sec:insights}

We discuss insights on design, threats, attack methods, and security measures from academic papers, industry incidents, and case studies below:

\subsection{Influence of Design on Threats}
\label{sec:threats_dis_influence}

Despite a wide range of security setups, we observe that the majority of the design combinations of existing wallets surveyed have been threatened by multiple vulnerabilities, as shown in \autoref{tab:wlt._taxonomy}. This is due to similar implementations i.e., the use of replicated libraries and commonly integrated implementation proposals (e.g., \acs{eip}-4337). We also observe that some wallets have had numerous vulnerabilities discovered in industry and academia. Most notably, Ledger and Trezor have several data remanence, data manipulation and insecure cryptographic vulnerabilities. Furthermore, in mapping vulnerabilities to attacks, we observe that some vulnerabilities can lead to numerous attack vectors as shown in \autoref{fig:wallet-mapping}. These include inadequate authentication, insecure permissions, insecure user interactions, and particularly data leakage. The Slope Wallet incident exemplifies this, where an improperly configured debug logging mechanism led directly to private key leakage.

\subsection{High Occurrence of Signature Verification Logic Flaws}
\label{sec:sig_verif_flaw}

We observe that signature verification logic flaws account for the most vulnerability occurrences in various wallets surveyed, constituting 19\%. Another interesting observation is the occurrence of this vulnerability in three diverse wallet security enhancement architectures, namely hardware, smart contract and \acs{mpc} wallets \cite{cve_14199, fireblocks_23, AccountMedium, UncoveringVulnerability}.


\subsection{Gap Analysis on Wallet Threats}

Conducting a gap analysis across industry and academic reports is difficult because many incidents do not disclose precise attack methods. We generally observe a high correlation between identified threats in industry and academia, except for insider and external threats. Specifically, in the following threats: malicious insider, compromised insider and compromised service provider threats. Although several custodial designs have been proposed by academia along with threat models, an investigation into the potential external threats and attacks in custodial setups would be highly beneficial for the industry. Notably, most industry attacks target exchanges and other custodial setups, as large funds are concentrated within a few wallet addresses. Additionally, research into these areas will also be pertinent due to the fact that wallet designs are gradually evolving into shared-custodial or other setups which require authentication from a centralised party (e.g., passkey, \acs{2fa}).

To address the gaps identified in \autoref{tab:threat_capability}, we propose the following measures:  
\begin{itemize}
    \item \textbf{Responsible Disclosure Policies:} Create a standardised incident template for responsible disclosure of wallet-related incidents. This could employ a uniform reporting format for exchanges and custodians to use when disclosing incidents, enabling both industry and academic audiences to analyse them consistently. A notable example in industry is Immunefi’s vulnerability disclosure platform \cite{immunefi2024}.
    
    \item \textbf{Public-Private Collaborations:} Formalise partnerships between exchanges, blockchain security firms, and academic institutions to analyse incident data. Successful models exist, including as IC3 and Chainlink partnership \cite{ic3} and the Stanford Centre for Blockchain Research’s industry partnerships \cite{stanford_cbr}.

       \item \textbf{Open-source Incident Registry:} Develop an open repository where vetted blockchain incident post-mortems can be deposited by operators and accessed by researchers, policymakers, and other exchanges. An existing example is the SlowMist Hacked incident archive \cite{slowmist_hacked}.


\end{itemize}


\subsection{Difference in Academia and Notable Industry Incidents}

Identifying attack vectors within the industry remains challenging, as sources often lack specificity. Notable attack vectors are significantly less clear (46\% unknown) and show a lower spread compared to attacks described in the literature (see \autoref{tab:attack_vectors}). This might be attributed to a lack of detailed post-mortem analysis in several incidents and an adversary's tendency to prioritise cost-effective methods. Academia, on the other hand, shows a high percentage (93\%) and spreads across various attack methods. Our case study on the ByBit incident also exemplifies the complexity of real-world incidents compared to academic models. While academic literature often isolates attack vectors, the ByBit incident involved a multi-stage, multi-vector attack with a chain of sub-goals linked to the main goal of \teal{$sk$} compromise.

\subsection{High-Risk Third-Party Dependencies}

The ByBit attack highlights a critical systemic risk in modern wallet architectures: third-party dependencies can nullify even highly secure solutions. Despite ByBit’s use of hardware wallets, multi-sig authorisation, and transaction policies, its reliance on Safe’s third-party UI created a single point of failure. Similarly, Slope Wallet's reliance on a self-hosted instance of a third-party monitoring solution (Sentry) introduced vulnerabilities due to misconfiguration and operational errors. This further underscores how third-party integrations significantly impact wallet security. This demonstrates that wallet security inherits the weakest link in dependency chains. To mitigate these risks, wallets must adopt resilient architectures and proactively manage third-party risks through multi-layered audits and adversarial scenario modelling.



\subsection{Comparison of Custodial and Non-Custodial Attacks}
Our incident analysis reveals that custodial wallets and non-custodial accounts for 70\% and 30\% of attacks,  respectively.  Additionally, unknown methods are significantly higher in custodial wallets (50\%) than in non-custodial wallets (36\%). Incidents show a high degree of similarity between custodial and non-custodial attacks. For instance, in comparison to other attacks, phishing attacks account for a relatively high percentage of both custodial (10\%) and non-custodial (36\%) wallets, especially factoring in the number of unknown attacks. 

\subsection{High Malware and Phishing Attack Occurrence}

We also find that application attacks account for a significant percentage of incident occurrences (43\%), with 34\% in custodial wallets and 48\% in non-custodial wallets. Our data also indicates that malware and phishing attacks are the most common attack vectors, accounting for 10\% and 18\% of total incidents, respectively. We also find that phishing-malware attacks constitute 48\% of total non-custodial wallet attacks.

\subsection{Limitations of Security Measures}
\label{sec:def_dis_attacks}


The majority of defence implementations in academia are particularly tailored to specific advanced attacks such as \acs{puf} for microscopic attacks, correlation elimination sounds for non-invasive side channels, and \acs{puf} attacks. Despite this, academia does not account for sophisticated attacks, which may leverage multiple attack vectors. Furthermore, distributed architectures prevalent in the industry are insufficient if dependencies remain centralised. The ByBit breach demonstrates that security measures must extend to third-party components, requiring redundant safeguards such as on-chain transaction simulation to detect UI spoofing or logic hijacking. In addition, the Slope Wallet incident demonstrates how inadequate configuration of application monitoring tools can undermine otherwise secure implementations, highlighting the need for strict data scrubbing and monitoring configurations.

\subsection{Comparison of Precautionary and Remedial Defence Methods}
\label{sec:def_dis_attacks}

Our study presents defence methods applicable to various attack vectors, with the majority offering either precautionary or remedial strategies, as illustrated in \autoref{tab:defence_methods}. Notably, precautionary defences significantly outnumber remedial approaches, comprising roughly 89\% of all methods observed. Within the precautionary category, protection-focused implementations are the most prevalent, accounting for 58\%. Among remedial defences, detection methods are the most common at 17\%, while response and recovery measures each represent a mere 3\%. This disparity highlights a critical gap in reactive mitigation techniques, indicating a potential area for further development in response and recovery-focused defences.




\section{Conclusion}
\label{sec:conclusion}


We presented DCA, an algorithm to address disparity in outcomes of ranking processes using compensatory bonus points. We showed that DCA, by relying on a sampling-based approach, successfully reduces disparity in a wide range of settings, while being significantly more efficient than state-of-the-art approaches, running in sub-linear time. This makes DCA a good candidate for iterative processes that would allow users to identify the ranking function that best fits their needs while checking for its fairness impacts and the required compensatory bonus points.  


Our approach relies on the use of compensatory bonus points, a departure from previous work, which has mostly focused on modifying the ranking function directly, or on the use of quotas. A significant advantage of compensatory bonus points is that they are transparent, interpretable, and easily explainable to all stakeholders.



\section{Conclusion}\label{sec:conclusion}

This paper presents our empirical domain knowledge distillation framework using ChatGPT and discusses our observations from the framework application experiments in the autonomous driving domain. The key finding is that: 1) with proper design of prompt engineering and execution flow, fully automated domain knowledge (in the ontology format) distillation is possible. However, due to the randomness in the response and the butterfly effect, the quality of fully automated distillation results is not guaranteed. To address this, we develop a web-based assistant to enable manual supervision and early intervention at runtime. We hope our findings and tools inspire future research toward revolutionizing the engineering processes of knowledge-based systems across domains.

% \begin{acronym}
    \acro{SSIM}{Structural Similarity}
    \acro{CNN}{Convolutional Neural Network}
    \acro{BCE}{Binary Cross-Entropy}
    \acro{MSE}{Mean Squared Error}
    \acro{GAN}{Generative Adversarial Network}
    \acro{MI}{Mutual Information}
    \acro{EMD}{Earth Mover's Distance}
    \acro{AUROC}{Area Under the Receiver Operator Characteristic Curve}
    \acro{RMSE}{Root Mean Square Error}
    \acro{ETCS}{European Train Control System}
    \acro{AE}{Auto-Encoder}
    \acro{Lidar}{Light detection and ranging}
\end{acronym}

% \printacronyms[title=ACRONYMS, name=ACRONYMS, heading=none]

% \printacronyms[title=ACRONYMS, heading=section*]


\section*{ACRONYMS}
\addcontentsline{toc}{section}{ACRONYMS}
\printacronyms[heading=none]

\section*{Acknowledgements}


The authors would like to thank Liyi Zhou and Zhipeng Wang for their in-depth review and constructive discussions during the study. This research is based upon work partially supported by the University Blockchain Research Initiative (UBRI)~\cite{feng2022university}. Any opinions, findings, and conclusions or recommendations expressed in this material are those of the authors and do not necessarily reflect the views of Ripple.


% \bibliographystyle{IEEEtran}
\bibliographystyle{cas-model2-names}
% \bibliographystyle{elsarticle-harv}
\bibliography{main}

% \appendices
\section{The Proof of Proposition \ref{prop2}}
\label{appa}
For the jointly Gaussian random vectors $\bm{x}$ and $\bm{y}$, we have
\begin{equation}
\begin{aligned}
&    \left[\begin{matrix}\bm{x}\\\bm{y}\\\end{matrix}\right] \sim \mathcal{N}\left(\left[\begin{matrix}\bm{\mu}_x\\\bm{\mu}_y\\\end{matrix}\right],\left[\begin{matrix}A&C\\C^T&B\\\end{matrix}\right]\right) \\
& = \mathcal{N}\left(\left[\begin{matrix}\bm{\mu}_x\\\bm{\mu}_y\\\end{matrix}\right],\left[\begin{matrix}\widetilde{A}&\widetilde{C}\\{\widetilde{C}}^T&B\\\end{matrix}\right]^{-1}\right)
\end{aligned}
\end{equation}
then the marginal and conditional distribution of $\bm{x}$ are shown as follows according to \cite{williams2006gaussian}.
\begin{equation}
    \bm{x} \sim \mathcal{N}\left(\bm{\mu}_x,A\right)
\end{equation}
% and
\begin{equation}
\label{app2-1}
    \bm{x}|\bm{y} \sim \mathcal{N}\left(\bm{\mu}_x+CB^{-1}\left(\bm{y}-\bm{\mu}_y\right),A-CB^{-1}C^T\right)
\end{equation}
% or
\begin{equation}
\label{app2-2}
    \bm{x}|\bm{y} \sim \mathcal{N}\left(\bm{\mu}_x-{\widetilde{A}}^{-1}\widetilde{C}\left(\bm{y}-\bm{\mu}_y\right),{\widetilde{A}}^{-1}\right)
\end{equation}

Thus, \textbf{Proposition \ref{prop2}} is proved.










\section{The Proof of Proposition \ref{prop3}}
\label{appb}
The product of two Gaussian distributions is represented as
\begin{equation}
\mathcal{N}\left(\bm{x}\middle|\bm{a},A\right)\mathcal{N}\left(\bm{x}\middle|\bm{b},B\right)=Z^{-1}\mathcal{N}\left(\bm{x}\middle|\bm{c},C\right)
\end{equation}
where
\begin{equation}
\label{app4}
    \bm{c}=C\left(A^{-1}\bm{a}+B^{-1}\bm{b}\right)
\end{equation}
\begin{equation}
\label{app5}
    C=\left(A^{-1}+B^{-1}\right)^{-1}
\end{equation}
\begin{equation}
\label{app6}
    Z^{-1}=\left(2\pi\right)^{-\frac{D}{2}}\left|A+B\right|^{-\frac{1}{2}}\exp{\left(-\frac{\left(\bm{a}-\bm{b}\right)^T\left(\bm{a}-\bm{b}\right)}{2\left(A+B\right)}\right)}
\end{equation}

Thus, through multiplying the cavity distribution by $t_i$ from (\ref{11}), \textbf{Proposition \ref{prop3}} is proved.


\section{The Proof of Proposition \ref{prop4}}
\label{appc}
Consider
\begin{equation}
\label{app7}
Z=\int_{-\infty}^{\infty}{\Phi\left(\frac{x-m}{v}\right)\mathcal{N}(x|\mu,\sigma^2)dx}
\end{equation}
% where
% \begin{equation}
%     \Phi\left(x\right)=\int_{-\infty}^{x}{\mathcal{N}\left(y\right)dy}
% \end{equation}
When $v>0$, by combining$ z=y-x+\mu-m$ and $w=x-\mu$ we can get
\begin{equation}
\begin{aligned}
& Z_{v>0}=\frac{\int_{-\infty}^{\infty}\int_{-\infty}^{x}\exp{\left(-\frac{\left(y-m\right)^2}{2v^2}-\frac{\left(x-\mu\right)^2}{2\sigma^2}\right)}}{2\pi\sigma v}dydx \\
& =\frac{\int_{-\infty}^{\mu-m}\int_{-\infty}^{\infty}\exp{\left(-\frac{\left(z+w\right)^2}{2v^2}-\frac{w^2}{2\sigma^2}\right)}}{2\pi\sigma v}dwdz
\end{aligned}
\end{equation}
% and
\begin{equation}
\begin{aligned}
& Z_{v>0} \\
& =\frac{\int_{-\infty}^{\mu-m}\int_{-\infty}^{\infty}\exp{\left(-\frac{1}{2}\left[\begin{matrix}w\\z\\\end{matrix}\right]^T\left[\begin{matrix}\frac{1}{v^2}+\frac{1}{\sigma^2}&\frac{1}{v^2}\\\frac{1}{v^2}&\frac{1}{v^2}\\\end{matrix}\right]\left[\begin{matrix}w\\z\\\end{matrix}\right]\right)}}{2\pi\sigma v}dwdz \\
& =\int_{-\infty}^{\mu-m}\int_{-\infty}^{\infty}\mathcal{N}\left(\left[\begin{matrix}w\\z\\\end{matrix}\right]|\mathbf{0},\left[\begin{matrix}\sigma^2&-\sigma^2\\-\sigma^2&v^2+\sigma^2\\\end{matrix}\right]\right)dwdz
\end{aligned}
\end{equation}
According to (\ref{app2-1}) and (\ref{app2-2}), we can get
\begin{equation}
\label{app11}
    Z_{v>0}=\frac{\int_{-\infty}^{\mu-m}\exp{\left(-\frac{z^2}{2\left(v^2+\sigma^2\right)}\right)}dz}{\sqrt{2\pi(v^2+\sigma^2)}}=\Phi\left(\frac{\mu-m}{\sqrt{v^2+\sigma^2}}\right)
\end{equation}
When $v<0$, by combining $\Phi\left(-z\right)=1-\Phi\left(z\right)$ and (\ref{app7}),
% we can obtain
\begin{equation}
\label{app12}
Z_{v<0}=1-\Phi\left(\frac{\mu-m}{\sqrt{v^2+\sigma^2}}\right)=\Phi\left(-\frac{\mu-m}{\sqrt{v^2+\sigma^2}}\right)
\end{equation}

By collecting (\ref{app11}) and (\ref{app12}), we can get
\begin{equation}
\label{app13}
Z=\int\Phi\left(\frac{x-m}{v}\right)\mathcal{N}\left(x\middle|\mu,\sigma^2\right)dx=\Phi\left(z\right)
\end{equation}
where $z=\frac{\mu-m}{v\sqrt{1+\sigma^2/v^2}} (v\neq0)$. 
% We aim to get the moments of
% \begin{equation}
% q\left(x\right)=Z^{-1}\Phi\left(\frac{x-m}{v}\right)\mathcal{N}\left(x\middle|\mu,\sigma^2\right)
% \end{equation}
By differentiating with respect to $\mu$ on (\ref{app13}), we can obtain
\begin{equation}
\begin{aligned}
& \frac{\partial Z}{\partial\mu}=\int{\frac{x-\mu}{\sigma^2}\Phi\left(\frac{x-m}{v}\right)}\mathcal{N}\left(x\middle|\mu,\sigma^2\right)dx =\frac{\partial}{\partial\mu}\Phi\left(z\right) \\
& \Longleftrightarrow \frac{1}{\sigma^2}\int x\Phi\left(\frac{x-m}{v}\right)\mathcal{N}\left(x\middle|\mu,\sigma^2\right)dx-\frac{\mu Z}{\sigma^2} \\
& =\frac{\mathcal{N}(z)}{v\sqrt{1+\sigma^2/v^2}}
\end{aligned}
\end{equation}
where $\partial\Phi\left(z\right)/\partial\mu=\mathcal{N}(z)\partial z/\partial\mu$ is utilized. Multiplying through by $\sigma^2/Z$, (\ref{app16}) is obtained.
\begin{equation}
\label{app16}
\mathbb{E}_q\left[x\right]=\mu+\frac{\sigma^2\mathcal{N}\left(z\right)}{\Phi\left(z\right)v\sqrt{1+\frac{\sigma^2}{v^2}}}
\end{equation}
Similarly, we can obtain the second moment as
\begin{equation}
\label{app17}
\begin{aligned}
 & \frac{\partial^2Z}{\partial\mu^2} \\
 & =\int{[\frac{x^2}{\sigma^4}-\frac{2\mu x}{\sigma^4}+\frac{\mu^2}{\sigma^4}-\frac{1}{\sigma^2}] \Phi\left(\frac{x-m}{v}\right)\mathcal{N}\left(x\middle|\mu,\sigma^2\right)} dx  \\
 & =-\frac{z\mathcal{N}(z)}{v^2+\sigma^2} \Longleftrightarrow \\
 & \mathbb{E}_q\left[x^2\right]=2\mu\mathbb{E}_q\left[x\right]-\mu^2+\sigma^2-\frac{\sigma^4z\mathcal{N}\left(z\right)}{\Phi\left(z\right)\left(v^2+\sigma^2\right)}
\end{aligned}
\end{equation}
By combining (\ref{app16}) and (\ref{app17}), we can get
\begin{equation}
\begin{aligned}
& \mathbb{E}_q\left[{(x-\mathbb{E}_q\left[x\right])}^2\right]=\mathbb{E}_q\left[x^2\right]-\mathbb{E}_q[x]^2 \\
& =\sigma^2-\frac{\sigma^4\mathcal{N}\left(z\right)}{\left(v^2+\sigma^2\right)\Phi\left(z\right)}\left(z+\frac{\mathcal{N}\left(z\right)}{\Phi\left(z\right)}\right)
\end{aligned}
\end{equation}

Thus, \textbf{Proposition \ref{prop4}} is proved.

\section{The Proof of Proposition \ref{prop5}}
\label{appd}
We can obtain (\ref{19-1}), (\ref{19-2}), and (\ref{19-3}) according to (\ref{app4}), (\ref{app5}), and (\ref{app6}). Hence, \textbf{Proposition \ref{prop5}} is proved.



\section{The Proof of Proposition \ref{prop6}}
\label{appe}
The approximated mean for $f_\ast$ can be denoted as
\begin{equation}
\begin{aligned}
& \mathbb{E}_q\left[f_\ast|X,\bm{y},\bm{x}_\ast\right]=\bm{k}_\ast^TK^{-1}\bm{\mu} \\
& =\bm{k}_\ast^TK^{-1}\left(K^{-1}+{\widetilde{\Sigma}}^{-1}\right)^{-1}{\widetilde{\Sigma}}^{-1}\widetilde{\bm{\mu}} \\
& =\bm{k}_\ast^T\left(K+\widetilde{\Sigma}\right)^{-1}\widetilde{\bm{\mu}}
\end{aligned}
\end{equation}

The variance of $f_\ast|(X,\bm{y})$ under the Gaussian approximation can be denoted as
\begin{equation}
\begin{aligned}
& \mathbb{V}_q\left[f_\ast\middle| X,\bm{y},\bm{x}_\ast\right] = \mathbb{E}_{p(f_\ast|X,\bm{x}_\ast,\bm{f})} {f_\ast-\mathbb{E}[f_\ast|X,\bm{x}_\ast,\bm{f}]}^2 \\
& =k\left(\bm{x}_\ast,\bm{x}_\ast\right)-\bm{k}_\ast^TK^{-1}\bm{k}_\ast+\bm{k}_\ast^TK^{-1}\left(K^{-1}+\widetilde{\Sigma}\right)^{-1}K^{-1}\bm{k}_\ast \\
& =k\left(\bm{x}_\ast,\bm{x}_\ast\right)-\bm{k}_\ast^T\left(K^{-1}+\widetilde{\Sigma}\right)^{-1}\bm{k}_\ast
\end{aligned}
\end{equation}

Then, we can obtain
\begin{equation}
\begin{aligned}
& q\left(y_\ast\middle| X,\bm{y},\bm{x}_\ast\right)=\mathbb{E}_q\left[\pi_\ast|X,\bm{y},\bm{x}_\ast\right] \\
& =\int\Phi\left(f_\ast\right)q\left(f_\ast\middle| X,\bm{y},\bm{x}_\ast\right)df_\ast
\end{aligned}
\end{equation}

According to (\ref{app11}), we can obtain
\begin{equation}
\label{app22}
\begin{aligned}
& q\left(y_\ast\middle| X,\bm{y},\bm{x}_\ast\right) \\
& =\Phi\left(\frac{\bm{k}_\ast^T\left(K+\widetilde{\Sigma}\right)^{-1}\widetilde{\bm{\mu}}}{\sqrt{1+k\left(\bm{x}_\ast,\bm{x}_\ast\right)-\bm{k}_\ast^T\left(K+\widetilde{\Sigma}\right)^{-1}\bm{k}_\ast}}\right)
\end{aligned}
\end{equation}

By combining (\ref{13}) and (\ref{app22}), \textbf{Proposition \ref{prop6}} is proved.




\section{The Proof of Proposition \ref{prop7}}
\label{appf}
Given $f_s$ and $f_\ast$, $y_s$ and $y_\ast$ are conditionally independent. Hence, $p\left(y_s,y_\ast\middle|\bm{x}_s,\bm{x}_\ast\right)$ can be represented as
\begin{equation}
\begin{aligned}
& p\left(y_s=1,y_\ast=1\middle|\bm{x}_s,\bm{x}_\ast\right) \\
& =\iint{\Phi\left(f_s\right)\Phi\left(f_\ast\right)\phi\left(f_s,f_\ast\middle|\mu_{s\ast},\Sigma_{s\ast}\right)}df_sdf_\ast \\
& =\iint{\Phi\left(f_\ast\right)\phi\left(f_\ast\middle|{\widetilde{\mu}}_\ast\left(f_s\right),{\widetilde{\sigma}}_{\ast\ast}\right)df_\ast\Phi\left(f_s\right)}\phi\left(f_s\middle|\mu_s,\sigma_{ss}\right)df_s \\
& =\int\Phi\left(\frac{{\widetilde{\mu}}_\ast\left(f_s\right)}{\sqrt{{\widetilde{\sigma}}_{\ast\ast}+1}}\right)\Phi\left(f_s\right)\phi\left(f_s\middle|\mu_s,\sigma_{ss}\right)df_s
\end{aligned}
\end{equation}

Hence, \textbf{Proposition \ref{prop7}} is proved.

% \section{The Proof of Lemma \ref{lem}}
% \label{appg}
% \begin{equation}
% \begin{aligned}
% & R_e=\frac{1}{N_a}\sum_{n=1}^{N_a}\mathbb{I}\left(\bm{L}_n \neq \bm{Y}_n\right) \\
% & =\displaystyle\frac{FA+FL}{TL+TA+FL+FA} \\
% & =\displaystyle\frac{1}{\displaystyle\frac{TL+TA+FL+FA}{FA+FL}} \\
% & =\displaystyle\frac{1}{1+\displaystyle\frac{TL+TA}{FA+FL}} \\
% & =\displaystyle\frac{1}{1+\displaystyle\frac{\displaystyle\frac{TL}{TA}+1}{\displaystyle\frac{FA}{TA}+\displaystyle\frac{FL}{TA}}} \\
% & =\frac{1}{1+\displaystyle\frac{\displaystyle\frac{TL}{TA}+1}{\displaystyle\frac{1}{P_{md}-1}+\displaystyle\frac{1}{P_{fa}-1}}}
% \end{aligned}
% \end{equation}

% Hence, \textbf{Lemma \ref{lem}} is proved.

\end{document}

