\section{Empirical Analysis}
\label{sec:empirical-analysis}

This section analyses real-world wallet attacks, distinguishing between wallet types, attack targets and attack categories. 

% Figure environment removed

\subsection{Methodology}
\label{sec:methodology-empirical}


The goal of our empirical analysis is to identify patterns which have emerged in previous real-world wallet attacks.

\subsubsection{Data Collection}
\label{sec:data_collection}

The data we gather is employed for analysis purposes in \autoref{sec:incident-analysis}.
Our dataset omits: \begin{enumerate*}
  \item Incidents on \ac{defi} protocol mechanisms and \ac{defi} or \ac{dao} treasuries.
  \item Research papers focused on wallets and cryptography with no attacks described.
\end{enumerate*}

\paragraph{Academic Papers}
\label{sec:papers}

We identify 63 academic papers, and 18 detailed security incident reports, focusing on several different wallet attacks. Our approach involves crawling  Google Scholar and conducting backwards and forward reference searches to identify further pertinent studies. We retrieve attacks on the various types of wallets in our taxonomy including software, hardware, brain, paper, and smart contract wallets (see \autoref{attack-vectors}). 

\paragraph{Industry Incidents}
\label{sec:incidents}
Our data contains 69 real-world attacks on cryptocurrency wallets, which occurred between March 2012 and November 2023. We gather incidents from several sources including DeFiLama, Slowmist, and Rekt News. 

\subsection{Incident Analysis}
\label{sec:incident-analysis}

\autoref{fig:attack-frequency} presents the correlation between the frequency of real-world attack incidents on cryptocurrency wallets and the total funds lost annually. Notably, the accumulated losses from these thefts exceed USD 4.5 billion. 2018 was particularly severe, with reported losses surpassing USD 1 billion. Frequently, these incidents involved the compromise of private keys; however, the precise methods utilised often remain undisclosed.


\subsection{Wallet Taxonomy Analysis}
\label{sec:taxonomy-analysis}

\autoref{fig:custody-method} shows the amount lost for custodial and non-custodial wallets grouped by attack methods. Non-custodial wallets account for one-third of wallets attacked, however, these wallet hacks are only one-tenth of total hacks at 439 million USD. On the other hand, custodial wallets account for 4.1 billion USD, which is 90\% of the total amount lost. This shows the average attack on custodial wallets (\$80 million USD) exceeds that of their non-custodial (\$25 million USD) counterparts. Additionally, nearly all wallets in our dataset are hot. With the exception of the FTX cold wallet incident which claimed a staggering \$400 million USD of user fund managed by the exchange.



% \subsection{Analysis of Attack Methods}
% \label{sec:incident-analysis}

% Consistent with \autoref{sec:attack-framework}, we examine the frequency and impact of different attack methods as depicted in \autoref{fig:attack-method-frequency}. Application-based attacks are the most prevalent, constituting 42\% of all recorded attacks and accounting for 35\% of the total funds lost. Authentication attacks, while less frequent, are almost equally costly, causing 33\% of the financial losses from only 23\% of the incidents. Network attacks are markedly rarer, representing less than 3\% of both incidents and associated financial losses. In about 30\% of cases, the specific methods compromising private keys remain unidentified by wallet operators.

% In further detail, our analysis differentiates the impact of attack methods on custodial and non-custodial wallets as illustrated in \autoref{fig:custody-pie}. For custodial wallets, authentication attacks incur 35.61\% of the total funds lost, occurring in 25.49\% of the incidents. This suggests a high financial impact relative to their frequency. We also observe the prevalence of application attacks which are 33.09\% of the funds lost and 37.25\% of incidents, making them the most common attack vector in custodial settings. 

% Conversely, We find that non-custodial wallets exhibit a different pattern. Application attacks dominate, accounting for 62.99\% of the funds lost and 55.56\% of incidents, underscoring their critical role in non-custodial wallet security. Authentication attacks, while less damaging financially at 10.40\% of the funds lost, still represent 16.67\% of incidents, suggesting their relatively frequent exploitation. Network attacks remain minimal, both in frequency and impact, with just 0.03\% of funds lost and 5.56\% of incidents reported. Unknown attack methods also play a significant role, with 26.58\% of the funds lost and 22.22\% of incidents, reflecting the ongoing challenges in identifying and mitigating these attacks. Unknown methods comprise 22\% and 35\% of the non-custodial and custodial incidents respectively.

% \subsection{Analysis of Attack Vectors}
% \label{sec:incident-analysis}

% \autoref{fig:wallet_attacks_bar} illustrates the frequency of wallet attack vectors. 

% Identifying attack vectors proves even more difficult as these are less clarified in sources than methods, as explained in \autoref{sec:challenges}. Despite more than 50\% of attack vectors being unknown, our data reveals some attack vectors. Notably, malware and phishing attacks are the most prevalent, with 20\% and 23\% of incidents respectively.

% Other significant vectors include storage exploits and SIM swap attacks, both notable for their impact on security, constituting 26.47\% and 25.89\% of the attack vectors. While storage exploits represent a considerable percentage of funds lost, they only account for 2.94\% of incidents, suggesting high effectiveness where used. Similarly, SIM swap attacks though high in funds claimed as a method only occur once. 

% Insider jobs and third-party breaches also contribute significantly to the attack landscape, indicating potential threats beyond the internal wallet mechanisms. Less common but still notable are server attacks, \acs{dns} hijacks, brute force, and API attacks, underscoring the diverse methods attackers employ to exploit wallet vulnerabilities.



% Website, storage and server attacks are categorised under wallet infrastructure attacks.


% Comparison of Total Funds Lost by Attack 


% Figure environment removed


% Figure environment removed

% % Figure environment removed

% % Figure environment removed

% \paragraph{The Goal of Network Attacks}

\subsection{Discussion}
\subsubsection{Insights}

\paragraph{Difference in Attack Methods for Custodial and Non-Custodial Wallets}

The contrasting occurrence of attack methods in custodial and non-custodial wallets underscores the varying security challenges faced by each type. The data indicates a crucial need for targeted security enhancements to suit custodial and non-custodial wallet architectures, in addition to employing security measures applicable to both.

\paragraph{Low Percentage of Authentication-based Attacks in Non-Custodial Wallets}

Authentication-based attacks account for a relatively small proportion of the total incidents and funds lost in non-custodial wallets, as illustrated in \autoref{fig:custody-pie}. This likely stems from the inherent structure of non-custodial wallets, which grant users complete control over their private keys and seed phrases. Additionally, attackers may prefer targeting other vulnerabilities within non-custodial systems given the potential user negligence, shifting their focus away from direct authentication attacks.

\paragraph{High Percentage of Application Attacks in Non-Custodial Wallets}

Application attacks constitute a significant portion of the security breaches observed in non-custodial wallets, as documented in \autoref{fig:custody-pie}. This is primarily attributable to the open design of the software environments that non-custodial wallets operate within. Many non-custodial wallets are built on platforms that permit third-party integrations and extensions, which, while enhancing functionality, also increase the attack surface. 

% These applications, often involving smart contracts or decentralised applications (dApps), are exposed to a range of vulnerabilities from coding errors to flawed logic, making them prime targets for attackers. Additionally, the autonomous nature of non-custodial wallets means that users must rely on their judgement or third-party tools to verify the security of the applications they interact with. 


% \paragraph{The Prevalence of Malware Phishing Attacks}

% What percentage of malware attacks are malware phishing?
% What wallets are malware phishing attacks more common on?

% This reliance can lead to gaps in security, especially if users are not technically adept. Attackers exploit these vulnerabilities to orchestrate phishing attacks, inject malicious code, or execute other harmful exploits that lead to the loss of funds. The high value and liquidity associated with such wallets further incentivize attackers to focus their efforts on these application-level weaknesses.


\paragraph{Inadequate Detection Mechanisms}

Most wallet attacks were identified only after unauthorised fund transfers had been made using compromised private keys. This indicates significant deficiencies in the existing detection mechanisms within their security operations. Therefore, this demonstrates the need for more effective intrusion detection mechanisms within wallets.


\subsubsection{Challenges}
\label{sec:challenges}
\paragraphtitle{Inaccessibility of Root Causes} A challenge encountered in our research was the lack of quality data on wallet attacks, with several root causes unknown. With more quality data, it would have been possible to map components in wallets with vulnerabilities, attack methods, vectors and root causes. 

% This challenge compelled us to derive attack paths aligned with various attacker objectives as shown in \autoref{sec:attack-tree}. Our approach sheds light on potential vulnerabilities and attack vectors extend beyond this analysis.

\paragraphtitle{Clarity of Attack Vectors}
\label{sec:method-inclarity}


Many recorded incidents from exchanges or non-custodial wallet providers show a high degree of uncertainty in the reporting of attack vectors. 30\% of all incidents attack methods were unknown. This ambiguity often gives rise to various hypotheses from sources regarding the exact nature of the attacks. For instance, the atomic wallet hack of USD 100 million was said to be a result of one of four probable causes which include a \acf{mitm} attack, a malware code injection or an infrastructure breach \cite{cointele_atomic}. 

% This uncertainty from sources complicates the analysis, making it challenging to gain insights into attack vectors.














