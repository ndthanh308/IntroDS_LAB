\section{Wallet Features}
\label{sec:wallet-features}

\subsection{M-of-N Standard Transactions}
\label{sec:bip-11}
Bitcoin Improvement Proposal 11 (BIP-11) \cite{bip11} introduces a method for Bitcoin transactions that requires multiple signatures before the transaction can be completed. Consider as collective decision-making process, wherein a specified subset of individuals (M) from a larger group (N) is required to reach a consensus before initiating any action. This method is particularly useful for increasing security in situations like escrow services, where trust and verification are crucial. However, this approach still depends on the existing security measures in Bitcoin and has some vulnerabilities. For instance, the keys used for signing need to be kept very secure to prevent misuse, and there's a risk that the signatures could be tampered with in a way that doesn't completely invalidate the transaction. Also, as the number of required signatures increases, the transaction becomes more costly and slower. While BIP-11 effectively allows multiple people to jointly authorize Bitcoin transactions, further improvements could make this process more secure and efficient.

Multi-signature features need to reduce the single point of failure. Bitfinex, a cryptocurrency exchange has been vulnerable to a multo-signatory implementation whereby 2 out of the 3 keys required to sign a transaction were on one device \cite{protos}.

\subsection{Hierarchical Deterministic (HD) Wallets}
\label{sec:bip-32}
Bitcoin Improvement Proposal 32 (BIP-32) \cite{bip32} introduces a system for creating Bitcoin wallets in a structured, hierarchical way. Imagine a family tree, where each branch can create its own sub-branches. This system uses complex mathematical problems and secure hashing algorithms (specifically, elliptic curve discrete logarithm problem and HMAC-SHA512) to ensure that the wallet and its sub-wallets (or keys) are very secure. Each \quotes{child} key, or sub-wallet, is generated in a way that makes it extremely difficult to trace back to the \quotes{parent} key, adding an extra layer of security. The system includes methods to keep the growth of the blockchain in check and to prevent any security risks that might arise from generating too many keys. However, the initial key from which all others are derived (the root seed) must be very well protected, as its security is crucial for the entire wallet structure. This approach allows for a flexible and secure way of managing wallets, but users must be careful to safeguard their root seeds.

\subsection{Mnemonic Code for HD Wallets}
\label{sec:bip-39}
Bitcoin Improvement Proposal 39 (BIP-39) \cite{bip39} introduces a method for creating easy-to-remember recovery phrases for Bitcoin wallets. These phrases, made up of a series of simple words, are based on strong cryptographic principles. The words are chosen from a specific list to ensure security and ease of use. The idea is to make these phrases both random enough to be secure and simple enough for people to remember and write down. To add extra security, a technique called \quotes{salting} and a process called PBKDF2 (Password Based Key Derivation Function 2 is a cryptography function using hashing messages and values several times to produce keys.) are used, which make it harder for attackers to guess these phrases. The security of these phrases also depends on how carefully users handle them – if someone uses them carelessly, they can still be vulnerable to attacks. Overall, BIP-39 aims to provide a balance between memorability, ease of writing down, and randomness to create secure backup phrases for wallet users.

\subsection{Multi-Account Hierarchy}
\label{sec:bip-44}
Bitcoin Improvement Proposal 44 (BIP-44) \cite{bip44} builds on the ideas of BIP-32, which introduced a hierarchical structure for Bitcoin wallets. BIP-44 takes this a step further by organizing these wallets into separate accounts, ensuring that the information or assets in one account don't accidentally get mixed up with another. It uses special techniques like \quotes{account discovery} and \quotes{gap limits} to manage these accounts efficiently and securely. These features make it easier for users to have multiple wallets for different purposes, all under one main account, without compromising on security. However, just like in BIP-32, the main concern is still about protecting the initial key (root seed) from which all other wallets are derived. While BIP-44 is great for organizing multiple wallets and sharing them selectively, the overall security also depends on how the user manages their main key.

\subsection{Multi-party computation (MPC)}
\label{sec:mpc-wallets}
MPC wallets \cite{canetti2020uc} represent a new and advanced way to make crypto wallets more secure, especially in how they handle private keys (the critical component for accessing and using cryptocurrencies). MPC works by using sophisticated cryptographic methods that allow several people or entities to work together on a calculation or process without having to share their individual pieces of information. When applied to crypto wallets, MPC splits the control of the private key among multiple parties. This means that no single person or entity holds the entire key, making it much harder for hackers to gain access to the wallet. The advantage of this approach is that it reduces the risk that comes from having all the security depend on just one point (like a single password or key), significantly enhancing overall security.

\subsection{Account Abstraction Using Alt Mempool}
\label{sec:erc-4337}
Ethereum Request for Comments 4337 (ERC-4337) \cite{erc4337} is a proposal that aims to make Ethereum accounts more flexible and secure without needing to change the core rules of the Ethereum network. It introduces a new system that works on top of the existing network, using a two-step process (validate and execute) for transactions. This approach makes transactions more efficient and helps prevent certain types of attacks where transaction fees can be manipulated. ERC-4337 also includes measures to ensure that the execution of transactions is consistent with their validation, and it allows for more versatile ways of managing transaction order than the traditional method. The goal of this proposal is to enhance the way Ethereum accounts work, especially in terms of security, without altering the fundamental rules of the Ethereum network. However, the responsibility for managing the security of private keys (the essential keys used to access and control Ethereum accounts) still lies with the users.


\subsection{Multi-Factor Key Derivation Function (MFKDF)}
\label{sec:mfkdf}
The MFKDF \cite{nair2023multi, nair2023decentralizing} is an advanced method for creating secure cryptographic keys, similar to complex passwords, using multiple types of security measures. This method is more secure and versatile compared to traditional approaches as it can integrate various authentication factors like one-time passwords, tokens from devices like YubiKey, and others. A significant advantage of this method is that it moves away from the conventional use of mnemonics typically used in crypto wallets. By doing so, it greatly enhances the user experience, making it both simpler and more intuitive. Additionally, MFKDF empowers users with the ability to recover their accounts on their own, without needing a central master key. This self-service recovery is facilitated through a secret-sharing approach, where the key is divided into parts and only a certain number (K out of N) of these parts are needed to reconstruct the full key. This combination of ease of use, security, and user-friendly account recovery marks a significant improvement in managing access to crypto wallets.