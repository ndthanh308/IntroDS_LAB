\section{Conclusion}
\label{sec:conclusion}

% Wallets will only grow in significance and functionality as the next millions of users are on-boarded into blockchain technology. An expectation is the functionality of wallets expanding to transaction authorisations related to non-monetary applications such as identity verification, smart property management, and decentralised voting systems. With this in mind, wallet security will be even more critical.

This paper analyses the design, threats, attack vectors, and defence strategies of cryptocurrency wallets. We introduce a multi-dimensional taxonomy of wallets, providing a framework to understand the intricate security landscape encompassing various wallet types. By systematising attack vectors, we provide a framework which applies to various wallet types. We examine 84 notable incidents accounting for more than \$5.4B. We go beyond this, to propose possible mitigation strategies for all attack vectors based on this framework. By mapping the wallet mechanism to design decisions, threats, attack methods and defence implementations, we discuss the interplay between dimensions. We also investigate industry incidents in compare these with academia.

% We first formalise a generalised wallet mechanism to understand the functionality of wallets.

% removed page limitation
% As the cryptocurrency landscape continues to evolve, we believe this research can assist with providing knowledge of diverse attack vectors and defence strategies applicable to wallets to guide academic and industry efforts towards developing more secure wallet solutions. 


% In future, it is important to conduct in-depth security assessments of key recovery mechanisms to prevent potential vulnerabilities and advocate for focused research on iOS-based wallet applications, which have been largely overlooked in the existing literature. 

