\section{Related Works}
\label{sec:related-work}

% \autoref{Literature-Gap-Table-1} outlines studies investigating crypto wallets and related mechanisms. 

% \begin{table}[!ht]
  \centering
  \renewcommand{\arraystretch}{1.2}
  \resizebox{\linewidth}{!}{%
    \begin{tabular}{r*{6}{c}*{4}{c}*{4}{c}}
      \toprule
      & \multicolumn{6}{c}{\textbf{Subjects Covered}}
      & \multicolumn{4}{c}{\textbf{Methodology}}
      & \multicolumn{4}{c}{\textbf{Scope}}
      \\
      \cmidrule(lr){2-7} \cmidrule(lr){8-11} \cmidrule(lr){12-15}
      \textbf{Reference}
      & \rot[90]{Key Cryptography}
      & \rot[90]{Key Management}
      & \rot[90]{Key Recovery}
      & \rot[90]{Attack Methods}
      & \rot[90]{Security Measures}
      & \rot[90]{Privacy Techniques}
      & \rot[90]{Literature}
      & \rot[90]{Taxonomisation}
      & \rot[90]{Analysis}
      & \rot[90]{Case Study}
      & \rot[90]{Wallet Software}
      & \rot[90]{Wallet Hardware}
      & \rot[90]{Smart Contract Wallet}
      & \rot[90]{Blockchain Network} \\
      \midrule
      This Study
      & \CIRCLE & \CIRCLE & \CIRCLE & \CIRCLE & \CIRCLE & \Circle
      & \CIRCLE & \CIRCLE & \CIRCLE & \CIRCLE
      & \CIRCLE  & \CIRCLE  & \CIRCLE  & \Circle \\
      \cite{bonneau2015sok}
      & \CIRCLE & \CIRCLE & \Circle  & \Circle  & \CIRCLE & \CIRCLE
      & \CIRCLE & \CIRCLE & \CIRCLE & \Circle
      & \CIRCLE  & \Circle   & \Circle   & \CIRCLE \\
      \cite{eskandari2018first}
      & \Circle   & \CIRCLE  & \CIRCLE  & \Circle   & \CIRCLE  & \Circle
      & \CIRCLE  & \CIRCLE  & \CIRCLE  & \Circle
      & \CIRCLE  & \Circle   & \Circle   & \Circle \\
      \cite{karantias2020sok}
      & \Circle   & \Circle   & \Circle   & \Circle   & \CIRCLE  & \CIRCLE
      & \CIRCLE  & \CIRCLE  & \CIRCLE  & \Circle
      & \CIRCLE  & \Circle   & \Circle   & \Circle \\
      \cite{Homoliak2020SmartOTPs:Wallets}
      & \Circle   & \CIRCLE  & \Circle   & \CIRCLE  & \CIRCLE  & \Circle
      & \CIRCLE  & \CIRCLE  & \CIRCLE  & \Circle
      & \CIRCLE  & \CIRCLE  & \Circle   & \CIRCLE \\
      \cite{Houy2023}
      & \Circle   & \CIRCLE  & \CIRCLE  & \CIRCLE  & \CIRCLE  & \CIRCLE
      & \CIRCLE  & \CIRCLE  & \CIRCLE  & \Circle
      & \CIRCLE  & \CIRCLE  & \Circle   & \CIRCLE \\
      \cite{suratkar2020cryptocurrency}
      & \CIRCLE & \CIRCLE & \CIRCLE & \Circle  & \Circle  & \Circle
      & \CIRCLE & \CIRCLE & \Circle   & \Circle
      & \CIRCLE  & \Circle   & \Circle   & \Circle \\
      \cite{bui2019pitfalls}
      & \Circle  & \Circle  & \Circle  & \CIRCLE & \CIRCLE & \Circle
      & \CIRCLE & \CIRCLE & \CIRCLE  & \Circle
      & \CIRCLE  & \Circle   & \Circle   & \Circle \\
      \cite{zaghloul2020bitcoin}
      & \Circle  & \Circle  & \Circle  & \Circle  & \CIRCLE & \CIRCLE
      & \CIRCLE & \CIRCLE & \CIRCLE  & \Circle
      & \CIRCLE  & \Circle   & \Circle   & \CIRCLE \\
      \cite{li2020android}
      & \Circle  & \Circle  & \Circle  & \CIRCLE & \CIRCLE & \Circle
      & \CIRCLE & \Circle   & \CIRCLE  & \Circle
      & \CIRCLE  & \Circle   & \Circle   & \Circle \\
      \cite{Dai2018SBLWT:Trustzone}
      & \CIRCLE & \CIRCLE & \Circle  & \CIRCLE & \CIRCLE & \Circle
      & \Circle  & \CIRCLE  & \CIRCLE  & \Circle
      & \CIRCLE  & \Circle   & \Circle   & \Circle \\
      \cite{volety2019cracking}
      & \Circle  & \Circle  & \Circle  & \CIRCLE & \CIRCLE & \Circle
      & \CIRCLE & \Circle   & \CIRCLE  & \Circle
      & \CIRCLE  & \Circle   & \Circle   & \Circle \\
      \cite{8966739}
      & \CIRCLE & \CIRCLE & \CIRCLE & \CIRCLE & \CIRCLE & \CIRCLE
      & \CIRCLE & \Circle   & \CIRCLE  & \Circle
      & \Circle   & \CIRCLE   & \Circle   & \Circle \\
      \cite{rezaeighaleh2020improving}
      & \CIRCLE & \CIRCLE & \CIRCLE & \Circle  & \CIRCLE & \Circle
      & \CIRCLE & \CIRCLE & \CIRCLE  & \Circle
      & \Circle   & \CIRCLE   & \Circle   & \Circle \\
      \cite{Urien2021InnovativeWallets}
      & \Circle  & \Circle  & \Circle  & \CIRCLE & \CIRCLE & \Circle
      & \CIRCLE & \CIRCLE & \Circle   & \Circle
      & \Circle   & \CIRCLE   & \Circle   & \Circle \\
      \cite{Rezaeighaleh2020MultilayeredWallet}
      & \Circle  & \Circle  & \CIRCLE & \Circle  & \Circle  & \CIRCLE
      & \CIRCLE & \CIRCLE & \Circle   & \Circle
      & \Circle   & \CIRCLE   & \Circle   & \Circle \\
      \cite{rezaeighaleh2019new}
      & \CIRCLE & \CIRCLE & \CIRCLE & \Circle  & \CIRCLE & \Circle
      & \CIRCLE & \Circle   & \CIRCLE  & \Circle
      & \Circle   & \CIRCLE   & \Circle   & \Circle \\
      \cite{di2020characteristics}
      & \Circle  & \Circle  & \Circle  & \Circle  & \Circle  & \Circle
      & \CIRCLE & \CIRCLE & \CIRCLE  & \Circle
      & \Circle   & \Circle    & \CIRCLE   & \Circle \\
% ----------- VERIFIED & CORRECTED ROWS ----------------------------------
\cite{Homoliak2024SoK:Factors}
& \Circle & \CIRCLE & \Circle & \Circle & \CIRCLE & \Circle
& \CIRCLE & \CIRCLE & \CIRCLE & \Circle
& \CIRCLE & \CIRCLE & \Circle & \Circle \\

\cite{andryukhin2019phishing}
& \Circle & \Circle & \Circle & \CIRCLE & \CIRCLE & \Circle
& \CIRCLE & \Circle & \CIRCLE & \Circle
& \Circle & \Circle & \Circle & \Circle \\

\cite{Chen2020ADefenses}
& \Circle & \Circle & \Circle & \CIRCLE & \CIRCLE & \Circle
& \CIRCLE & \CIRCLE & \CIRCLE & \Circle
& \Circle & \Circle & \Circle & \CIRCLE \\

\cite{Courtois2017StealthSystems}
& \CIRCLE & \CIRCLE & \Circle & \CIRCLE & \CIRCLE & \CIRCLE
& \CIRCLE & \Circle & \CIRCLE & \Circle
& \CIRCLE & \Circle & \Circle & \CIRCLE \\

\cite{Das2019AWallets}
& \CIRCLE & \CIRCLE & \Circle & \Circle & \CIRCLE & \Circle
& \Circle & \Circle & \CIRCLE & \Circle
& \CIRCLE & \Circle & \Circle & \Circle \\

\cite{Eyal2022OnDesign}
& \Circle & \CIRCLE & \Circle & \Circle & \CIRCLE & \Circle
& \Circle & \CIRCLE & \CIRCLE & \Circle
& \CIRCLE & \CIRCLE & \Circle & \Circle \\

\cite{Gotte2021TechAttacks}
& \CIRCLE & \CIRCLE & \Circle & \CIRCLE & \CIRCLE & \Circle
& \Circle & \Circle & \CIRCLE & \Circle
& \Circle & \CIRCLE & \Circle & \Circle \\

\cite{Guo2022ASecurity}
& \Circle & \CIRCLE & \Circle & \CIRCLE & \CIRCLE & \Circle
& \CIRCLE & \CIRCLE & \CIRCLE & \Circle
& \CIRCLE & \CIRCLE & \Circle & \CIRCLE \\

\cite{He2018AScheme}
& \Circle & \CIRCLE & \CIRCLE & \Circle & \CIRCLE & \Circle
& \Circle & \Circle & \CIRCLE & \Circle
& \CIRCLE & \Circle & \Circle & \Circle \\

\cite{Houy2023}
& \CIRCLE & \CIRCLE & \CIRCLE & \CIRCLE & \CIRCLE & \Circle
& \CIRCLE & \CIRCLE & \CIRCLE & \Circle
& \CIRCLE & \CIRCLE & \Circle & \CIRCLE \\

\cite{Li2020ASystems}
& \Circle & \CIRCLE & \Circle & \CIRCLE & \CIRCLE & \Circle
& \Circle & \Circle & \CIRCLE & \Circle
& \CIRCLE & \Circle & \Circle & \Circle \\

\cite{Mangipudi2023UncoveringCrypto-Wallets}
& \Circle & \CIRCLE & \Circle & \Circle & \Circle & \Circle
& \Circle & \Circle & \CIRCLE & \CIRCLE
& \CIRCLE & \Circle & \Circle & \Circle \\

\cite{Shbair2021HSM-basedBlockchain}
& \CIRCLE & \CIRCLE & \Circle & \CIRCLE & \CIRCLE & \Circle
& \Circle & \Circle & \CIRCLE & \CIRCLE
& \Circle & \CIRCLE & \Circle & \Circle \\

\cite{zhou2023sok}
& \Circle & \Circle & \Circle & \CIRCLE & \CIRCLE & \Circle
& \CIRCLE & \CIRCLE & \CIRCLE & \Circle
& \Circle & \Circle & \CIRCLE & \CIRCLE \\

\cite{chatzigiannis2025composability}
& \Circle & \CIRCLE & \CIRCLE & \CIRCLE & \CIRCLE & \Circle
& \Circle & \Circle & \CIRCLE & \CIRCLE
& \CIRCLE & \Circle & \CIRCLE & \Circle \\

% ------------------------------------------------------------------------

% -------------------------------------------------------------------------

    % \cite{Homoliak2024SoK:Factors}
    %   & \Circle  & \Circle  & \Circle  & \Circle  & \Circle  & \Circle
    %   & \Circle & \Circle & \Circle  & \Circle
    %   & \Circle   & \Circle    & \Circle   & \Circle \\
    %       \cite{andryukhin2019phishing}
    %   & \Circle  & \Circle  & \Circle  & \Circle  & \Circle  & \Circle
    %   & \Circle & \Circle & \Circle  & \Circle
    %   & \Circle   & \Circle    & \Circle   & \Circle \\
      
    %       \cite{Chen2020ADefenses}
    %   & \Circle  & \Circle  & \Circle  & \Circle  & \Circle  & \Circle
    %   & \Circle & \Circle & \Circle  & \Circle
    %   & \Circle   & \Circle    & \Circle   & \Circle \\
    %       \cite{Courtois2017StealthSystems}
    %   & \Circle  & \Circle  & \Circle  & \Circle  & \Circle  & \Circle
    %   & \Circle & \Circle & \Circle  & \Circle
    %   & \Circle   & \Circle    & \Circle   & \Circle \\
      
    %       \cite{Das2019AWallets}
    %   & \Circle  & \Circle  & \Circle  & \Circle  & \Circle  & \Circle
    %   & \Circle & \Circle & \Circle  & \Circle
    %   & \Circle   & \Circle    & \Circle   & \Circle \\
      
    %       \cite{Eyal2022OnDesign}
    %   & \Circle  & \Circle  & \Circle  & \Circle  & \Circle  & \Circle
    %   & \Circle & \Circle & \Circle  & \Circle
    %   & \Circle   & \Circle    & \Circle   & \Circle \\
      
    %       \cite{Gotte2021TechAttacks}
    %   & \Circle  & \Circle  & \Circle  & \Circle  & \Circle  & \Circle
    %   & \Circle & \Circle & \Circle  & \Circle
    %   & \Circle   & \Circle    & \Circle   & \Circle \\
      
    %       \cite{Guo2022ASecurity}
    %   & \Circle  & \Circle  & \Circle  & \Circle  & \Circle  & \Circle
    %   & \Circle & \Circle & \Circle  & \Circle
    %   & \Circle   & \Circle    & \Circle   & \Circle \\
      
    %       \cite{He2018AScheme}
    %   & \Circle  & \Circle  & \Circle  & \Circle  & \Circle  & \Circle
    %   & \Circle & \Circle & \Circle  & \Circle
    %   & \Circle   & \Circle    & \Circle   & \Circle \\
      
    %       \cite{Houy2023}
    %   & \Circle  & \Circle  & \Circle  & \Circle  & \Circle  & \Circle
    %   & \Circle & \Circle & \Circle  & \Circle
    %   & \Circle   & \Circle    & \Circle   & \Circle \\
      
    %       \cite{Li2020ASystems}
    %   & \Circle  & \Circle  & \Circle  & \Circle  & \Circle  & \Circle
    %   & \Circle & \Circle & \Circle  & \Circle
    %   & \Circle   & \Circle    & \Circle   & \Circle \\
    %       \cite{Mangipudi2023UncoveringCrypto-Wallets}
    %   & \Circle  & \Circle  & \Circle  & \Circle  & \Circle  & \Circle
    %   & \Circle & \Circle & \Circle  & \Circle
    %   & \Circle   & \Circle    & \Circle   & \Circle \\
      
    %       \cite{Shbair2021HSM-basedBlockchain}
    %   & \Circle  & \Circle  & \Circle  & \Circle  & \Circle  & \Circle
    %   & \Circle & \Circle & \Circle  & \Circle
    %   & \Circle   & \Circle    & \Circle   & \Circle \\
      
    %       \cite{zhou2023sok}
    %   & \Circle  & \Circle  & \Circle  & \Circle  & \Circle  & \Circle
    %   & \Circle & \Circle & \Circle  & \Circle
    %   & \Circle   & \Circle    & \Circle   & \Circle \\
      \bottomrule
    \end{tabular}%
  }
  \caption{Overview of related works. (\CIRCLE: include, \Circle: not include)}
  \label{Literature-Gap-Table-1}
\end{table}


% To the best of our knowledge, our work is the first systematisation of knowledge focused on wallet attacks and defence methods. we outline how our work is differentiated from the existing one below.

\subsection{Key Management}
\label{sec:key-management}

Several studies have explored key management mechanisms. Courtois and Mercer \cite{Courtois2017StealthSystems} compare key management solutions with a focus on stealth addresses. Mangipudi et al. \cite{Mangipudi2023UncoveringCrypto-Wallets} investigate key management from the wallet users' perspective. He et al. \cite{He2018AScheme} propose a secure key management scheme based on semi-trusted social networks. Di Angelo and Salzer \cite{di2020characteristics} analyse the functionality of smart contracts for key management through transaction data. Our study differs by focusing on attacks and defence methods for key management mechanisms and wallet taxonomy.

% Additional studies have proposed key management designs. Khan et al. \cite{8966739} suggest using QR codes for transaction authentication between hardware and software wallets. Homoliak et al. \cite{homoliak2018air} propose a self-sovereign smart contract key management framework. Unlike these studies, we examine various key management designs and their security from a blockchain-agnostic viewpoint.

\subsection{Wallet Attack and Security}
\label{sec:wallet-security}

% zamani2020security Urien2021,

Various studies have analysed blockchain systems' security and vulnerabilities \cite{Li2020ASystems, Guo2022ASecurity, Chen2020ADefenses}. For instance, Chen et al. \cite{Chen2020ADefenses} focus on Ethereum's vulnerabilities and defence mechanisms. Our work differs by focusing on wallet security, categorised under external auxiliary services, rather than blockchain layers. The security of specific wallets has also been explored \cite{Shbair2021HSM-basedBlockchain, Gotte2021TechAttacks}. Götte and Scheuermann \cite{Gotte2021TechAttacks} propose defences for Hardware Security Modules against physical attacks. Our study takes a multi-layered approach (see \autoref{sec:defense-strategies}) to analyse a wide range of wallet attacks.

Specific attack vectors have been investigated as well \cite{andryukhin2019phishing, bui2019pitfalls}. Andryukhin \cite{andryukhin2019phishing} evaluates phishing attacks and proposes prevention mechanisms. Bui et al. \cite{bui2019pitfalls} examine security vulnerabilities in the \acs{rpc} of desktop wallets. Our work covers a broader scope of attacks compared to these studies. While some studies have explored security across various wallet types, the scope and depth vary. Das et al. \cite{Das2019AWallets} propose a security model for hot/cold wallets. Our research extends beyond hot/cold wallets, employing a detailed taxonomy and analysing operational mechanisms, bridging the gap between academia and industry. Eyal \cite{Eyal2022OnDesign} evaluates the impact of key management on wallet security. Houy et al. \cite{Houy2023} conduct a literature review of wallet attacks and defences, however, does not include theoretical or empirical evaluations.

\subsection{Addressing Literature Gaps}
\label{sec:gaps-in-literature}

Despite various studies on specific wallet types, mechanisms, and attack vectors, there is a lack of a comprehensive examination spanning wallet design taxonomy, mechanisms, attack analysis, and security measures. Our study bridges this gap, providing a holistic understanding crucial for advancing wallet security.

% Old Literature Review

% \autoref{Literature-Gap-Table-1} outlines studies that investigate crypto wallets and their related mechanisms. Our work is the first systematisation of knowledge focused on the attack and security of wallets.

% \subsection{Key Management}
% \label{sec:key-management}

% Various studies have investigated key management mechanisms and schemes. Courtois and Mercer \cite{courtois2017stealth} provide an overview and compare key management solutions, with a focus on stealth addresses. Mangipudi et al. \cite{mangipudi2022uncovering} investigate key management design from the perspective of wallet users. He et al \cite{he2018social} propose a highly effective and secure key management scheme based on semi-trusted social networks. In addition, Di Angelo and Salzer \cite{di2020characteristics} investigate the functionality of smart contracts for key management by analysing transaction data. We differ from these studies as we explore attacks and defence methods for key management mechanisms, as well as the wallet mechanisms and taxonomy.
 
%  Key management design has also been an area of interest in some studies. Khan et al. \cite{8966739} propose an architectural design for a Bitcoin wallet by utilising QR codes to authenticate transactions between hardware and software wallets. Homoliak et al \cite{homoliak2018air} propose a unique self-sovereign smart contract key management framework. We can also be differentiated from these studies by the broad spectrum we cover, which includes examining various types of key management designs and their related security. Other studies have surveyed key management solutions for Bitcoin \cite{eskandari2018first, pal2021key}. Our work is not restricted to any blockchain network and analyses key management methods and security from a blockchain-agnostic viewpoint.

% % \begin{table}[!ht]
  \centering
  \renewcommand{\arraystretch}{1.2}
  \resizebox{\linewidth}{!}{%
    \begin{tabular}{r*{6}{c}*{4}{c}*{4}{c}}
      \toprule
      & \multicolumn{6}{c}{\textbf{Subjects Covered}}
      & \multicolumn{4}{c}{\textbf{Methodology}}
      & \multicolumn{4}{c}{\textbf{Scope}}
      \\
      \cmidrule(lr){2-7} \cmidrule(lr){8-11} \cmidrule(lr){12-15}
      \textbf{Reference}
      & \rot[90]{Key Cryptography}
      & \rot[90]{Key Management}
      & \rot[90]{Key Recovery}
      & \rot[90]{Attack Methods}
      & \rot[90]{Security Measures}
      & \rot[90]{Privacy Techniques}
      & \rot[90]{Literature}
      & \rot[90]{Taxonomisation}
      & \rot[90]{Analysis}
      & \rot[90]{Case Study}
      & \rot[90]{Wallet Software}
      & \rot[90]{Wallet Hardware}
      & \rot[90]{Smart Contract Wallet}
      & \rot[90]{Blockchain Network} \\
      \midrule
      This Study
      & \CIRCLE & \CIRCLE & \CIRCLE & \CIRCLE & \CIRCLE & \Circle
      & \CIRCLE & \CIRCLE & \CIRCLE & \CIRCLE
      & \CIRCLE  & \CIRCLE  & \CIRCLE  & \Circle \\
      \cite{bonneau2015sok}
      & \CIRCLE & \CIRCLE & \Circle  & \Circle  & \CIRCLE & \CIRCLE
      & \CIRCLE & \CIRCLE & \CIRCLE & \Circle
      & \CIRCLE  & \Circle   & \Circle   & \CIRCLE \\
      \cite{eskandari2018first}
      & \Circle   & \CIRCLE  & \CIRCLE  & \Circle   & \CIRCLE  & \Circle
      & \CIRCLE  & \CIRCLE  & \CIRCLE  & \Circle
      & \CIRCLE  & \Circle   & \Circle   & \Circle \\
      \cite{karantias2020sok}
      & \Circle   & \Circle   & \Circle   & \Circle   & \CIRCLE  & \CIRCLE
      & \CIRCLE  & \CIRCLE  & \CIRCLE  & \Circle
      & \CIRCLE  & \Circle   & \Circle   & \Circle \\
      \cite{Homoliak2020SmartOTPs:Wallets}
      & \Circle   & \CIRCLE  & \Circle   & \CIRCLE  & \CIRCLE  & \Circle
      & \CIRCLE  & \CIRCLE  & \CIRCLE  & \Circle
      & \CIRCLE  & \CIRCLE  & \Circle   & \CIRCLE \\
      \cite{Houy2023}
      & \Circle   & \CIRCLE  & \CIRCLE  & \CIRCLE  & \CIRCLE  & \CIRCLE
      & \CIRCLE  & \CIRCLE  & \CIRCLE  & \Circle
      & \CIRCLE  & \CIRCLE  & \Circle   & \CIRCLE \\
      \cite{suratkar2020cryptocurrency}
      & \CIRCLE & \CIRCLE & \CIRCLE & \Circle  & \Circle  & \Circle
      & \CIRCLE & \CIRCLE & \Circle   & \Circle
      & \CIRCLE  & \Circle   & \Circle   & \Circle \\
      \cite{bui2019pitfalls}
      & \Circle  & \Circle  & \Circle  & \CIRCLE & \CIRCLE & \Circle
      & \CIRCLE & \CIRCLE & \CIRCLE  & \Circle
      & \CIRCLE  & \Circle   & \Circle   & \Circle \\
      \cite{zaghloul2020bitcoin}
      & \Circle  & \Circle  & \Circle  & \Circle  & \CIRCLE & \CIRCLE
      & \CIRCLE & \CIRCLE & \CIRCLE  & \Circle
      & \CIRCLE  & \Circle   & \Circle   & \CIRCLE \\
      \cite{li2020android}
      & \Circle  & \Circle  & \Circle  & \CIRCLE & \CIRCLE & \Circle
      & \CIRCLE & \Circle   & \CIRCLE  & \Circle
      & \CIRCLE  & \Circle   & \Circle   & \Circle \\
      \cite{Dai2018SBLWT:Trustzone}
      & \CIRCLE & \CIRCLE & \Circle  & \CIRCLE & \CIRCLE & \Circle
      & \Circle  & \CIRCLE  & \CIRCLE  & \Circle
      & \CIRCLE  & \Circle   & \Circle   & \Circle \\
      \cite{volety2019cracking}
      & \Circle  & \Circle  & \Circle  & \CIRCLE & \CIRCLE & \Circle
      & \CIRCLE & \Circle   & \CIRCLE  & \Circle
      & \CIRCLE  & \Circle   & \Circle   & \Circle \\
      \cite{8966739}
      & \CIRCLE & \CIRCLE & \CIRCLE & \CIRCLE & \CIRCLE & \CIRCLE
      & \CIRCLE & \Circle   & \CIRCLE  & \Circle
      & \Circle   & \CIRCLE   & \Circle   & \Circle \\
      \cite{rezaeighaleh2020improving}
      & \CIRCLE & \CIRCLE & \CIRCLE & \Circle  & \CIRCLE & \Circle
      & \CIRCLE & \CIRCLE & \CIRCLE  & \Circle
      & \Circle   & \CIRCLE   & \Circle   & \Circle \\
      \cite{Urien2021InnovativeWallets}
      & \Circle  & \Circle  & \Circle  & \CIRCLE & \CIRCLE & \Circle
      & \CIRCLE & \CIRCLE & \Circle   & \Circle
      & \Circle   & \CIRCLE   & \Circle   & \Circle \\
      \cite{Rezaeighaleh2020MultilayeredWallet}
      & \Circle  & \Circle  & \CIRCLE & \Circle  & \Circle  & \CIRCLE
      & \CIRCLE & \CIRCLE & \Circle   & \Circle
      & \Circle   & \CIRCLE   & \Circle   & \Circle \\
      \cite{rezaeighaleh2019new}
      & \CIRCLE & \CIRCLE & \CIRCLE & \Circle  & \CIRCLE & \Circle
      & \CIRCLE & \Circle   & \CIRCLE  & \Circle
      & \Circle   & \CIRCLE   & \Circle   & \Circle \\
      \cite{di2020characteristics}
      & \Circle  & \Circle  & \Circle  & \Circle  & \Circle  & \Circle
      & \CIRCLE & \CIRCLE & \CIRCLE  & \Circle
      & \Circle   & \Circle    & \CIRCLE   & \Circle \\
% ----------- VERIFIED & CORRECTED ROWS ----------------------------------
\cite{Homoliak2024SoK:Factors}
& \Circle & \CIRCLE & \Circle & \Circle & \CIRCLE & \Circle
& \CIRCLE & \CIRCLE & \CIRCLE & \Circle
& \CIRCLE & \CIRCLE & \Circle & \Circle \\

\cite{andryukhin2019phishing}
& \Circle & \Circle & \Circle & \CIRCLE & \CIRCLE & \Circle
& \CIRCLE & \Circle & \CIRCLE & \Circle
& \Circle & \Circle & \Circle & \Circle \\

\cite{Chen2020ADefenses}
& \Circle & \Circle & \Circle & \CIRCLE & \CIRCLE & \Circle
& \CIRCLE & \CIRCLE & \CIRCLE & \Circle
& \Circle & \Circle & \Circle & \CIRCLE \\

\cite{Courtois2017StealthSystems}
& \CIRCLE & \CIRCLE & \Circle & \CIRCLE & \CIRCLE & \CIRCLE
& \CIRCLE & \Circle & \CIRCLE & \Circle
& \CIRCLE & \Circle & \Circle & \CIRCLE \\

\cite{Das2019AWallets}
& \CIRCLE & \CIRCLE & \Circle & \Circle & \CIRCLE & \Circle
& \Circle & \Circle & \CIRCLE & \Circle
& \CIRCLE & \Circle & \Circle & \Circle \\

\cite{Eyal2022OnDesign}
& \Circle & \CIRCLE & \Circle & \Circle & \CIRCLE & \Circle
& \Circle & \CIRCLE & \CIRCLE & \Circle
& \CIRCLE & \CIRCLE & \Circle & \Circle \\

\cite{Gotte2021TechAttacks}
& \CIRCLE & \CIRCLE & \Circle & \CIRCLE & \CIRCLE & \Circle
& \Circle & \Circle & \CIRCLE & \Circle
& \Circle & \CIRCLE & \Circle & \Circle \\

\cite{Guo2022ASecurity}
& \Circle & \CIRCLE & \Circle & \CIRCLE & \CIRCLE & \Circle
& \CIRCLE & \CIRCLE & \CIRCLE & \Circle
& \CIRCLE & \CIRCLE & \Circle & \CIRCLE \\

\cite{He2018AScheme}
& \Circle & \CIRCLE & \CIRCLE & \Circle & \CIRCLE & \Circle
& \Circle & \Circle & \CIRCLE & \Circle
& \CIRCLE & \Circle & \Circle & \Circle \\

\cite{Houy2023}
& \CIRCLE & \CIRCLE & \CIRCLE & \CIRCLE & \CIRCLE & \Circle
& \CIRCLE & \CIRCLE & \CIRCLE & \Circle
& \CIRCLE & \CIRCLE & \Circle & \CIRCLE \\

\cite{Li2020ASystems}
& \Circle & \CIRCLE & \Circle & \CIRCLE & \CIRCLE & \Circle
& \Circle & \Circle & \CIRCLE & \Circle
& \CIRCLE & \Circle & \Circle & \Circle \\

\cite{Mangipudi2023UncoveringCrypto-Wallets}
& \Circle & \CIRCLE & \Circle & \Circle & \Circle & \Circle
& \Circle & \Circle & \CIRCLE & \CIRCLE
& \CIRCLE & \Circle & \Circle & \Circle \\

\cite{Shbair2021HSM-basedBlockchain}
& \CIRCLE & \CIRCLE & \Circle & \CIRCLE & \CIRCLE & \Circle
& \Circle & \Circle & \CIRCLE & \CIRCLE
& \Circle & \CIRCLE & \Circle & \Circle \\

\cite{zhou2023sok}
& \Circle & \Circle & \Circle & \CIRCLE & \CIRCLE & \Circle
& \CIRCLE & \CIRCLE & \CIRCLE & \Circle
& \Circle & \Circle & \CIRCLE & \CIRCLE \\

\cite{chatzigiannis2025composability}
& \Circle & \CIRCLE & \CIRCLE & \CIRCLE & \CIRCLE & \Circle
& \Circle & \Circle & \CIRCLE & \CIRCLE
& \CIRCLE & \Circle & \CIRCLE & \Circle \\

% ------------------------------------------------------------------------

% -------------------------------------------------------------------------

    % \cite{Homoliak2024SoK:Factors}
    %   & \Circle  & \Circle  & \Circle  & \Circle  & \Circle  & \Circle
    %   & \Circle & \Circle & \Circle  & \Circle
    %   & \Circle   & \Circle    & \Circle   & \Circle \\
    %       \cite{andryukhin2019phishing}
    %   & \Circle  & \Circle  & \Circle  & \Circle  & \Circle  & \Circle
    %   & \Circle & \Circle & \Circle  & \Circle
    %   & \Circle   & \Circle    & \Circle   & \Circle \\
      
    %       \cite{Chen2020ADefenses}
    %   & \Circle  & \Circle  & \Circle  & \Circle  & \Circle  & \Circle
    %   & \Circle & \Circle & \Circle  & \Circle
    %   & \Circle   & \Circle    & \Circle   & \Circle \\
    %       \cite{Courtois2017StealthSystems}
    %   & \Circle  & \Circle  & \Circle  & \Circle  & \Circle  & \Circle
    %   & \Circle & \Circle & \Circle  & \Circle
    %   & \Circle   & \Circle    & \Circle   & \Circle \\
      
    %       \cite{Das2019AWallets}
    %   & \Circle  & \Circle  & \Circle  & \Circle  & \Circle  & \Circle
    %   & \Circle & \Circle & \Circle  & \Circle
    %   & \Circle   & \Circle    & \Circle   & \Circle \\
      
    %       \cite{Eyal2022OnDesign}
    %   & \Circle  & \Circle  & \Circle  & \Circle  & \Circle  & \Circle
    %   & \Circle & \Circle & \Circle  & \Circle
    %   & \Circle   & \Circle    & \Circle   & \Circle \\
      
    %       \cite{Gotte2021TechAttacks}
    %   & \Circle  & \Circle  & \Circle  & \Circle  & \Circle  & \Circle
    %   & \Circle & \Circle & \Circle  & \Circle
    %   & \Circle   & \Circle    & \Circle   & \Circle \\
      
    %       \cite{Guo2022ASecurity}
    %   & \Circle  & \Circle  & \Circle  & \Circle  & \Circle  & \Circle
    %   & \Circle & \Circle & \Circle  & \Circle
    %   & \Circle   & \Circle    & \Circle   & \Circle \\
      
    %       \cite{He2018AScheme}
    %   & \Circle  & \Circle  & \Circle  & \Circle  & \Circle  & \Circle
    %   & \Circle & \Circle & \Circle  & \Circle
    %   & \Circle   & \Circle    & \Circle   & \Circle \\
      
    %       \cite{Houy2023}
    %   & \Circle  & \Circle  & \Circle  & \Circle  & \Circle  & \Circle
    %   & \Circle & \Circle & \Circle  & \Circle
    %   & \Circle   & \Circle    & \Circle   & \Circle \\
      
    %       \cite{Li2020ASystems}
    %   & \Circle  & \Circle  & \Circle  & \Circle  & \Circle  & \Circle
    %   & \Circle & \Circle & \Circle  & \Circle
    %   & \Circle   & \Circle    & \Circle   & \Circle \\
    %       \cite{Mangipudi2023UncoveringCrypto-Wallets}
    %   & \Circle  & \Circle  & \Circle  & \Circle  & \Circle  & \Circle
    %   & \Circle & \Circle & \Circle  & \Circle
    %   & \Circle   & \Circle    & \Circle   & \Circle \\
      
    %       \cite{Shbair2021HSM-basedBlockchain}
    %   & \Circle  & \Circle  & \Circle  & \Circle  & \Circle  & \Circle
    %   & \Circle & \Circle & \Circle  & \Circle
    %   & \Circle   & \Circle    & \Circle   & \Circle \\
      
    %       \cite{zhou2023sok}
    %   & \Circle  & \Circle  & \Circle  & \Circle  & \Circle  & \Circle
    %   & \Circle & \Circle & \Circle  & \Circle
    %   & \Circle   & \Circle    & \Circle   & \Circle \\
      \bottomrule
    \end{tabular}%
  }
  \caption{Overview of related works. (\CIRCLE: include, \Circle: not include)}
  \label{Literature-Gap-Table-1}
\end{table}

 
% \subsection{Wallet Attack and Security}
% \label{sec:wallet-security}

% Various studies have rigorously analysed blockchain systems' overall security and vulnerability \cite{li2020survey, guo2022survey, zamani2020security, Urien2021, chen2020survey, zhou2023sok}. For instance, Chen et al. \cite{chen2020survey} provide a security analysis of the Ethereum blockchain by investigating the vulnerabilities, attacks, and defence mechanisms. We can be differentiated from these, as our work focuses on securing wallets which are categorised under the external auxiliary services \cite{zhou2023sok}, as opposed to any of the blockchain layers.

% The security of specific wallets within our taxonomy \autoref{sec:wallet-taxonomy} has also been an area of interest \cite{shbair2021hsm, gotte2021tech}. Götte and Scheuermann \cite{gotte2021tech} propose a defence method for Hardware Security Modules against advanced physical attacks. Our study analyses wallet security from a multi-layered approach (see \autoref{sec:defense-strategies}) and encompasses a wide range of attacks on wallets.

% % specific attacks on wallets
% Specific attack vectors have also been investigated \cite{andryukhin2019phishing, andryukhin2019phishing}. Andryukhin \cite{andryukhin2019phishing} evaluates phishing attacks on wallets and proposed prevention mechanisms. Bui et al. \cite{bui2019pitfalls} investigate the security vulnerabilities of the \acf{rpc} in desktop wallets. Our work analyses a broader scope of attacks compared to the focus of these papers.

% While a handful of studies have explored the security dimensions across various wallet types, the scope and depth of these analyses vary. Das et al \cite{10.1145/3319535.3354236} propose a comprehensive security model for deterministic hot/cold wallets and develop secure wallet schemes within these models. In contrast, our research broadens the wallet security analysis beyond the hot/cold wallet paradigm. We employ a multi-dimensional taxonomy of which the hot/cold dimension includes, as well as analyse operational mechanisms and investigate the gap between academia and industry. Eyal \cite{eyal2022cryptocurrency} evaluates how the number and type of keys managed affect the overall security of a wallet.
% Our research is differentiated through its detailed taxonomy, operational mechanisms, and empirical analysis. Although, Houy et al. \cite{houy2023security} conduct a systematic literature review of attacks and defence mechanisms of wallets, their investigation stops short of incorporating theoretical or empirical evaluations. 

% \subsection{Addressing Literature Gaps}
% \label{sec:gaps-in-literature}

% While various studies have explored specific wallet types, their mechanisms, and associated attack vectors, the literature lacks a unified, comprehensive examination that spans the full spectrum of wallet taxonomy, mechanisms, attack analysis, and security measures. This omission represents a significant gap, as a holistic understanding of these aspects is crucial for advancing the security of wallets. Our study bridges this gap and demonstrates the subject areas we cover in comparison to other studies as shown in \autoref{Literature-Gap-Table-1}.