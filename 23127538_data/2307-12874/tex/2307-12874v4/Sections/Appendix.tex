% this was commented out before`

% \onecolumn
\appendix
\section{Appendix}
\label{sec:appendix}

% \subsection{Ethical Considerations and Compliance with Open Science Policy}

% In this study, we proactively considered the ethical implications related to our research on cryptocurrency wallet security and the potential negative outcomes of our research. Our work adheres to all ethical standards and does not involve human participants directly.

% Furthermore, by systematising wallet vulnerabilities, threats, and defence mechanisms, we aim to enhance wallet security rather than inadvertently expose systems to new risks. To mitigate any potential harm, such as the misuse of identified vulnerabilities by malicious actors, we have ensured that sensitive data about specific attack vectors and implementation details of wallet designs are not disclosed in a manner that could enable exploitation. Additionally, our analysis focuses on existing wallet attack incidents, relying on publicly available data, to avoid violating user privacy or exposing sensitive information. In line with the open research policy at USENIX, all materials employed in this study will be made available to the larger public.


% \subsection{Figures}

% Below in \autoref{fig:hardware-mechanism}, we illustrate the difference between generalised software and hardware wallets based on our mechanism (see \autoref{fig:wallet-mechanism}).

% % Figure environment removed

% ----


% \subsection{Tables}

% Attached on the following pages are two tables on wallet attack incidents and institutional custody providers respectively.

% \paragraph{Table 5: Wallet Attack Incidents}
% \autoref{tab:attack-incidents} provides an overview of wallet attack incidents, between March 2012 and November 2023. As shown in the table, there is a lack of clarity on several attacks.

% \paragraph{Table 6: Institutional Custody Providers}

% \autoref{tab:instit2} shows institutional custody providers, who provide private keys management services for other organisations.


\begin{table}[!htbp]
\centering
\tiny
\setlength{\tabcolsep}{2.1pt}
\renewcommand{\arraystretch}{0.9}
\begin{tabular}{lllrll}
\toprule
\textbf{Name} & \textbf{Custody Design} & \textbf{Date} & \textbf{Loss (\$)} & \textbf{Attack Category} & \textbf{Attack Name} \\
\midrule
ByBit \cite{bybit}  
  & Custodial  
  & %21/02/2025 
    2025-02 
  & 1,500M  
  & Application  
  & Logic Exploitation \\

US Govt. \cite{Decrypt}  
  & Non-Custodial  
  & %25/10/2024 
    2024-10 
  & 50M  
  & –  
  & – \\

BigX \cite{Explained:2024}  
  & Custodial  
  & %20/09/2024 
    2024-09 
  & 52M  
  & –  
  & – \\

Indodax \cite{IndonesianTRX}  
  & Custodial  
  & %11/09/2024 
    2024-09 
  & 22M  
  & –  
  & – \\

WazirX \cite{Explained:2024g}  
  & Custodial  
  & %18/07/2024 
    2024-07 
  & 235M  
  & Application  
  & Logic Exploitation \\

Bittensor \cite{Explained:2024}  
  & Non-Custodial  
  & %02/07/2024 
    2024-07 
  & 8M  
  & Application  
  & Malware \\

BTCTurk \cite{Explained:2024}  
  & Custodial  
  & %23/06/2024 
    2024-06 
  & 55M  
  & –  
  & – \\

Loopring \cite{Explained:2024}  
  & Non-Custodial  
  & %09/06/2024 
    2024-06 
  & 5M  
  & Authentication  
  & Identity Spoofing\textsuperscript{*} \\

Lykke \cite{CoinTelegraph}  
  & Custodial  
  & %04/06/2024 
    2024-06 
  & 22M  
  & –  
  & – \\

DMM Bitcoin \cite{Explained:2024}  
  & Custodial  
  & %31/05/2024 
    2024-05 
  & 305M  
  & –  
  & – \\

Axie Co-Founder \cite{Decrypt}  
  & Non-Custodial  
  & %23/02/2024 
    2024-02 
  & 10M  
  & –  
  & – \\

Fixed Float \cite{Explained:2024}  
  & Custodial  
  & %16/02/2024 
    2024-02 
  & 26.1M  
  & –  
  & – \\

kirilm.eth \cite{Explained:2024}  
  & Non-Custodial  
  & %16/02/2024 
    2024-02 
  & 5.1M  
  & Application  
  & Phishing \\

Ripple Co-Founder \cite{RippleMillion}  
  & Non-Custodial  
  & %30/01/2024 
    2024-01 
  & 112.5M  
  & –  
  & – \\

HTX (Huobi) \cite{HTXReport}  
  & Custodial  
  & %22/11/2023 
    2023-11 
  & 13.6M  
  & –  
  & \teal{\textit{sk}} Compromise\textsuperscript{*} \\

Pink Drainer \cite{RektREKT}  
  & Non-Custodial  
  & %16/11/2023 
    2023-11 
  & 12M  
  & Application  
  & Phishing, Malware \\

Monkey Drainer \cite{RektREKT}  
  & Non-Custodial  
  & %16/11/2023 
    2023-11 
  & 16M  
  & Application  
  & Phishing, Malware \\

Venom Drainer \cite{RektREKT}  
  & Non-Custodial  
  & %16/11/2023 
    2023-11 
  & 27M  
  & Application  
  & Phishing, Malware \\

Infarno \cite{infarno}  
  & Non-Custodial  
  & %16/11/2023 
    2023-11 
  & 66M  
  & Application  
  & Phishing, Malware \\

Poloniex \cite{RektREKT}  
  & Custodial  
  & %10/11/2023 
    2023-11 
  & 126M  
  & –  
  & \teal{\textit{sk}} Compromise\textsuperscript{*} \\

Lastpass \cite{RektREKT}  
  & Non-Custodial  
  & %31/10/2023 
    2023-10 
  & 37M  
  & Authentication  
  & – \\

Fantom Fdn. \cite{AnalysisMedium}  
  & Non-Custodial  
  & %18/10/2023 
    2023-10 
  & 7M  
  & –  
  & – \\

HTX (Huobi) \cite{HTXReport}  
  & Custodial  
  & %25/09/2023 
    2023-09 
  & 8M  
  & Application  
  & Phishing \\

Fake Voucher \cite{RektREKT}  
  & Non-Custodial  
  & %20/09/2023 
    2023-09 
  & 4.5M  
  & Application  
  & Phishing \\

Remitano \cite{RektREKT}  
  & Custodial  
  & %15/09/2023 
    2023-09 
  & 2.7M  
  & Application  
  & – \\

CoinEx \cite{CoinTelegraph}  
  & Custodial  
  & %12/09/2023 
    2023-09 
  & 55M  
  & –  
  & \teal{\textit{sk}} Compromise\textsuperscript{*} \\

Monero \cite{MoneroFlash}  
  & Non-Custodial  
  & %01/09/2023 
    2023-09 
  & 0.5M  
  & –  
  & – \\

AlphaPo \cite{RektREKT}  
  & Custodial  
  & %26/07/2023 
    2023-07 
  & 60M  
  & –  
  & \teal{\textit{sk}} Compromise\textsuperscript{*} \\

Atomic Wallet \cite{CoinTelegraph}  
  & Non-Custodial  
  & %03/06/2023 
    2023-06 
  & 100M  
  & –  
  & – \\

Bitrue \cite{Explained:2024}  
  & Custodial  
  & %14/04/2023 
    2023-04 
  & 23M  
  & –  
  & \teal{\textit{sk}} Compromise\textsuperscript{*} \\

GDAC \cite{CoinTelegraph}  
  & Custodial  
  & %09/04/2023 
    2023-04 
  & 13M  
  & –  
  & \teal{\textit{sk}} Compromise\textsuperscript{*} \\

MyAlgo \cite{CoinTelegraph}  
  & Non-Custodial  
  & %27/02/2023 
    2023-02 
  & 9.2M  
  & –  
  & – \\

BitKeep \cite{CertiKIncidents}  
  & Non-Custodial  
  & %26/12/2022 
    2022-12 
  & 8M  
  & Application  
  & Phishing, Malware \\

FTX \cite{FTXMistake}  
  & Custodial  
  & %12/11/2022 
    2022-11 
  & 450M  
  & Authentication  
  & Sim Swap Attack \\

Deribit \cite{CryptoWithdrawals}  
  & Custodial  
  & %01/11/2022 
    2022-11 
  & 28M  
  & Application  
  & – \\

Wintermute \cite{TheMedium}  
  & Custodial  
  & %20/09/2022 
    2022-09 
  & 160M  
  & Authentication  
  & Brute force \\

Slope \cite{CoinTelegraph}  
  & Non-Custodial  
  & %02/08/2022 
    2022-08 
  & 8M  
  & Storage and Memory  
  & – \\

MetaMask \cite{CertiKIncidents}  
  & Non-Custodial  
  & %17/04/2022 
    2022-04 
  & 0.65M  
  & Authentication  
  & Phishing \\

Crypto.com \cite{Explained:2024}  
  & Custodial  
  & %17/01/2022 
    2022-01 
  & 30M  
  & Authentication  
  & – \\

Lympo \cite{CoinTelegraph}  
  & Custodial  
  & %10/01/2022 
    2022-01 
  & 18.7M  
  & –  
  & – \\

LCX \cite{LookingHacken}  
  & Custodial  
  & %08/01/2022 
    2022-01 
  & 8M  
  & –  
  & \teal{\textit{sk}} Compromise\textsuperscript{*} \\

Vulcan Forged \cite{VulcanHack}  
  & Non-Custodial  
  & %13/12/2021 
    2021-12 
  & 140M  
  & Application  
  & \teal{\textit{sk}} Compromise\textsuperscript{*} \\

BitMart \cite{HackScience}  
  & Custodial  
  & %05/12/2021 
    2021-12 
  & 196M  
  & Application  
  & Phishing \\

Liquid \cite{HackBreach}  
  & Custodial  
  & %19/08/2021 
    2021-08 
  & 90M  
  & Application  
  & \teal{\textit{sk}} Compromise\textsuperscript{*} \\

Roll \cite{CoinDesk}  
  & Custodial  
  & %14/03/2021 
    2021-03 
  & 5.7M  
  & Application  
  & \teal{\textit{sk}} Compromise\textsuperscript{*} \\

MetaMask \cite{Explained:2024}  
  & Non-Custodial  
  & %14/12/2020 
    2020-12 
  & 8M  
  & –  
  & – \\

KuCoin \cite{kucoinNew}  
  & Custodial  
  & %25/09/2020 
    2020-09 
  & 275M  
  & Application  
  & \teal{\textit{sk}} Compromise\textsuperscript{*} \\

Cashaa \cite{CoinTelegraph}  
  & Custodial  
  & %11/07/2020 
    2020-07 
  & 3.1M  
  & Application  
  & Malware \\

Trinity Wallet \cite{IOTA:Wallet}  
  & Non-Custodial  
  & %12/02/2020 
    2020-02 
  & 2.3M  
  & Application  
  & – \\

Altsbit \cite{AltsbitZDNET}  
  & Custodial  
  & %05/02/2020 
    2020-02 
  & 72.5M  
  & Application  
  & – \\

Upbit \cite{UpbitMedium}  
  & Custodial  
  & %26/11/2019 
    2019-11 
  & 49M  
  & Application  
  & Phishing, Malware \\

Bitpoint \cite{BitPointMedium}  
  & Custodial  
  & %11/07/2019 
    2019-07 
  & 36.5M  
  & –  
  & – \\

Vindax \cite{VinDAXBlock}  
  & Custodial  
  & %05/11/2019 
    2019-11 
  & 0.5M  
  & –  
  & – \\

Bitrue \cite{CryptoNews}  
  & Custodial  
  & %27/06/2019 
    2019-06 
  & 4.5M  
  & Authentication  
  & – \\

Gatehub \cite{OverviewMedium}  
  & Custodial  
  & %06/06/2019 
    2019-06 
  & 9.5M  
  & –  
  & – \\

Binance Exchange \cite{binanceNew}  
  & Custodial  
  & %07/05/2019 
    2019-05 
  & 40M  
  & Unknown  
  & – \\

Bithumb \cite{CoinDesk}  
  & Custodial  
  & %29/03/2019 
    2019-03 
  & 13M  
  & Other  
  & Insider Job \\

Coinbene \cite{CoinTelegraph}  
  & Custodial  
  & %25/03/2019 
    2019-03 
  & 99M  
  & –  
  & – \\

DragonEX \cite{CoinDesk}  
  & Custodial  
  & %24/03/2019 
    2019-03 
  & 1M  
  & Application  
  & – \\

Cryptopia \cite{HowHacken}  
  & Custodial  
  & %01/02/2019 
    2019-02 
  & 16M  
  & –  
  & \teal{\textit{sk}} Compromise\textsuperscript{*} \\

LocalBitcoins \cite{CoinDesk}  
  & Custodial  
  & %26/01/2019 
    2019-01 
  & 0.02M  
  & Application  
  & Phishing \\

Electrum \cite{DeepSwig}  
  & Non-Custodial  
  & %21/12/2018 
    2018-12 
  & 0.75M  
  & Application  
  & Phishing \\

Maplechange \cite{MapleChangeInvestorPlace}  
  & Custodial  
  & %28/10/2018 
    2018-10 
  & 6M  
  & –  
  & – \\

Zaif \cite{CoinDesk}  
  & Custodial  
  & %14/09/2018 
    2018-09 
  & 100M  
  & –  
  & – \\

Coinrail \cite{CoinDesk}  
  & Custodial  
  & %10/06/2018 
    2018-06 
  & 40M  
  & –  
  & – \\

MyEtherWallet \cite{myetherwallet}  
  & Non-Custodial  
  & %24/04/2018 
    2018-04 
  & 0.15M  
  & Network  
  & \acs{bgp} Hijacking \\

Gate.io \cite{ZachXBTWraps}  
  & Custodial  
  & %18/04/2018 
    2018-04 
  & 234M  
  & –  
  & – \\

CoinSecure \cite{CoinDesk}  
  & Custodial  
  & %13/04/2018 
    2018-04 
  & 3.5M  
  & Other  
  & Insider Job \\

Bitgrail \cite{BitGrailCoinMarketCap}  
  & Custodial  
  & %10/02/2018 
    2018-02 
  & 146M  
  & Other  
  & Insider Job \\

CoinCheck \cite{TheHack}  
  & Custodial  
  & %27/01/2018 
    2018-01 
  & 560M  
  & –  
  & – \\

BlackWallet \cite{BlackWalletFault}  
  & Non-Custodial  
  & %15/01/2018 
    2018-01 
  & 0.4M  
  & Network  
  & \acs{dns} Spoofing \\

EtherDelta \cite{CryptocurrencyScheme}  
  & Custodial  
  & %20/12/2017 
    2017-12 
  & 1.4M  
  & Network  
  & \acs{dns} Spoofing \\

Parity \cite{palladino2017parity}  
  & Non-Custodial  
  & %19/07/2017 
    2017-07 
  & 30M  
  & Application  
  & Logic Exploitation \\

Yapizon \cite{CoinTelegraph}  
  & Custodial  
  & %22/04/2017 
    2017-04 
  & 5.3M  
  & –  
  & – \\

Bitfinex \cite{CoinDesk}  
  & Custodial  
  & %02/08/2016 
    2016-08 
  & 623M  
  & Application  
  & – \\

Gatecoin \cite{CoinDesk}  
  & Custodial  
  & %09/05/2016 
    2016-05 
  & 2.1M  
  & –  
  & – \\

Shapeshift \cite{LootingShapeShift}  
  & Custodial  
  & %07/04/2016 
    2016-04 
  & 0.23M  
  & Other  
  & Insider Job \\

Bitstamp \cite{DetailsRevealed}  
  & Custodial  
  & %11/12/2015 
    2015-12 
  & 5M  
  & Application  
  & Phishing \\

BTER \cite{CoinDesk}  
  & Custodial  
  & %15/08/2015 
    2015-08 
  & 1.65M  
  & Application  
  & – \\

Mintpal \cite{RememberingLedger}  
  & Custodial  
  & %13/07/2014 
    2014-07 
  & 2M  
  & Other  
  & Insider Job \\

Poloniex \cite{PoloniexHack}  
  & Custodial  
  & %04/03/2014 
    2014-03 
  & 0.05M  
  & Application  
  & – \\

Mt. Gox \cite{mtgox_hack}  
  & Custodial  
  & %24/02/2014 
    2014-02 
  & 460M  
  & –  
  & – \\

Bitcash \cite{CzechEmptied}  
  & Custodial  
  & %11/11/2013 
    2013-11 
  & 0.1M  
  & Application  
  & Phishing \\

Bitfloor \cite{HackSecurityWeek}  
  & Custodial  
  & %12/09/2012 
    2012-09 
  & 0.25M  
  & Application  
  & \teal{\textit{sk}} Compromise\textsuperscript{*} \\

Bitcoinica \cite{ExchangeStolen}  
  & Custodial  
  & %01/03/2012 
    2012-03 
  & 0.09M  
  & Application  
  & \teal{\textit{sk}} Compromise\textsuperscript{*} \\

\midrule
\textbf{Summary:}
  & \textbf{85 incidents}
  & \textbf{2012–2025}
  & \textbf{6.98B}
  &  
  &  
\\
\bottomrule
\end{tabular}
\caption{Wallet attack incidents in the industry. We retrieve 85 notable attack incidents involving both custodial and non-custodial wallets. Several attack methods remain unknown (–) or undetailed, we indicate undetailed incidents with \textsuperscript{*}.}
\label{tab:attack-incidents}
\end{table}

% \begin{table}
\caption{Survey of Institutional Custody Providers: \ac{mpc} - Multi-Party Computation, M-Sig - Multi-sig, HSM - Hardware Security Module, BTC  - Bitcoin, ETH - Ethereum, ADA - Cardano, SOL - Solana, XRP - Ripple, XLM - Stellar, HBAR - Hedera. ({\fullcirc}: include, {\emptycirc}: not include)}
\begin{center}
\tiny
\renewcommand{\arraystretch}{1.3}
\resizebox{1\textwidth}{!}{
        {\scriptsize
\begin{tabular}{l|ccc|ccccccc}
\toprule
\multicolumn{1}{l}{Name} & \multicolumn{3}{c}{Key Management}
                                      & \multicolumn{7}{c}{Chains}                                                                                                      \\
\midrule
               & \ac{mpc} & M-Sig & HSM & BTC & ETH & ADA & SOL & XRP & XLM & HBAR                           \\
\hline
Copper          &    {\fullcirc}   &    {\emptycirc}   &      {\emptycirc}  & {\fullcirc} & {\fullcirc} & {\fullcirc} & {\fullcirc} & {\fullcirc} & {\fullcirc}    & {\emptycirc}                                \\
Gemini Custody          &    {\fullcirc}   &   {\emptycirc}    &   {\emptycirc}   & {\fullcirc} & {\fullcirc} &  {\emptycirc} & {\fullcirc} & {\emptycirc} & {\emptycirc}    &       {\emptycirc}         \\
Fireblocks     &   {\fullcirc}    &   {\emptycirc}    &    {\emptycirc}  & {\fullcirc} & {\fullcirc} &  {\fullcirc} & {\fullcirc} & {\fullcirc} & {\fullcirc}           &  {\fullcirc}    \\
BitGo        &    {\emptycirc}   &   {\fullcirc}    &      {\emptycirc}    & {\fullcirc} & {\fullcirc} &  {\fullcirc}  & {\fullcirc}  & {\fullcirc} & {\fullcirc}    &     {\fullcirc}                             \\
Genesis Custody        &   {\fullcirc}    &   {\emptycirc}    &     {\emptycirc}    & {\fullcirc} & {\fullcirc} &  {\emptycirc}  & {\emptycirc}  & {\fullcirc} & {\fullcirc}       &     {\emptycirc}      \\
Metaco    &    {\fullcirc}   &   {\emptycirc}    &     {\fullcirc}  & {\fullcirc} & {\fullcirc} &  {\fullcirc}    & {\fullcirc}    & {\fullcirc} &       {\fullcirc}       &    {\emptycirc}      \\
Tangany       &   {\emptycirc}    &    {\emptycirc}   &           {\fullcirc}   & {\fullcirc} & {\fullcirc} &  {\emptycirc}  & {\emptycirc}  & {\emptycirc}  & {\emptycirc}      &    {\emptycirc}           \\
Prosegur Crypto &    {\fullcirc}   &   {\fullcirc}    &   {\emptycirc}  & {\fullcirc} & {\fullcirc} &  {\emptycirc}  & {\emptycirc}  & {\fullcirc} & {\emptycirc}       &     {\emptycirc}         \\
DLT Finance     &   {\fullcirc}    &    {\emptycirc}   &      {\emptycirc}    & {\fullcirc} & {\fullcirc} &  {\emptycirc}  & {\emptycirc}  & {\fullcirc} & {\fullcirc}     &     {\emptycirc}        \\
Hex Trust       &   {\emptycirc}    &   {\emptycirc}    &           {\fullcirc}   & {\fullcirc} & {\fullcirc} & {\fullcirc} & {\emptycirc}  & {\fullcirc} & {\fullcirc}      &     {\fullcirc}       \\
Onchain Custodian         &   {\fullcirc}    &   {\fullcirc}    &     {\emptycirc}    & {\fullcirc} & {\fullcirc} &  {\emptycirc}  & {\emptycirc}  & {\emptycirc}  &        {\emptycirc}    &    {\fullcirc}                              \\
BitPanda   &   {\emptycirc}    &    {\fullcirc}   &     {\fullcirc}   & {\fullcirc} & {\fullcirc} & {\fullcirc} & {\fullcirc} & {\fullcirc} & {\fullcirc}        &     {\emptycirc}                          \\
GK8     &    {\fullcirc}   &   {\emptycirc}    &     {\emptycirc}   & {\fullcirc} & {\fullcirc} &  {\fullcirc} & {\fullcirc} & {\fullcirc} & {\fullcirc}         &   {\fullcirc}                          \\
KNØX             &    {\emptycirc}   &   {\fullcirc}    &     {\fullcirc}  & {\fullcirc} & {\emptycirc}  &  {\emptycirc}  & {\emptycirc}  & {\emptycirc}  & {\emptycirc}  &    {\emptycirc}                           \\
DigiVault         &   {\emptycirc}    &   {\fullcirc}    &      {\fullcirc}   & {\fullcirc} & {\fullcirc} &  {\emptycirc}  & {\emptycirc}  & {\fullcirc} & {\emptycirc}    &        {\emptycirc}        \\
Cybavo              &    {\fullcirc}   &    {\emptycirc}   &       {\emptycirc}  & {\fullcirc} & {\fullcirc} &  {\fullcirc} & {\fullcirc} & {\fullcirc} &   {\fullcirc}    &       {\emptycirc}                          \\
Bitcoin Suisse        &   {\emptycirc}    &  {\fullcirc}     &       {\fullcirc}   & {\fullcirc} & {\fullcirc} & {\fullcirc} & {\fullcirc} & {\fullcirc} & {\emptycirc}        &     {\emptycirc}      \\
Cobo         &    {\fullcirc}   &   {\emptycirc}    &             {\fullcirc}  & {\fullcirc} & {\fullcirc} & {\fullcirc} & {\fullcirc} & {\fullcirc} & {\fullcirc}   &    {\emptycirc}
\\
\bottomrule
\end{tabular}
}}
\end{center}
\label{tab:instit2}
\end{table}




















% \paragraph{Table 6: Business Data of Wallets}
% \autoref{CryptoWallets-Table-1} provides a business metrcis comparison of various non-custodial crypto wallets.


















% \subsection{Severity Calculation}
% \label{appendix:severity-calculation}
% We calculate the severity of each threat similar to Zhang et al. \cite{zhang2017conditional} in \autoref{tab:prob}. We define a function in \autoref{eq:probability-formula} to calculate the severity (S) of occurrence of a threat (T) using CVSS Exploitability and CC Attack Potential metric values. 

% \begin{table}[!htbp] % Adjust placement specifiers to be consistent
\centering
\caption{CVSS Exploitability Metrics \\
Access Vector (AV) quantifies the ease of access to the vulnerability, considering the proximity of the threat actor to the target. Access Complexity (AC) gauges the sophistication of techniques required for exploitation. Privileges Required (PR) assesses the permission level necessary for an exploit. User Interaction (UI) measures whether the exploitation process needs human participation.}
\label{tab:cvss}
\tiny
\renewcommand{\arraystretch}{0.8} % Adjust space between rows
\resizebox{\textwidth}{!}{%
\begin{tabular}{cccccccccc}
\toprule % Replaces \hline for a top border with better spacing
\textbf{Method}                       & \textbf{CVSS Exploitability}       & \textbf{Network} & \textbf{Adjacent}  & \textbf{Local} & \textbf{Physical} & \textbf{None} & \textbf{Low} & \textbf{High} & \textbf{Required} \\
\midrule % Replaces \hline for middle borders with better spacing
\multirow{4}{*}{Likelihood} & Access Vector (AV)       & 0.85         & 0.62         & 0.55       & 0.2          & -        & -       & -        & -            \\ 
                            & Access Complexity (AC)   & -            & -            & -          & -            & -        & 0.77    & 0.44     & -            \\ 
                            & Privileges Required (PR) & -            & -            & -          & -            & 0.85     & 0.62    & 0.27     & -            \\ 
                            & User Interaction (UI)    & -            & -            & -          & -            & 0.85     & -       & -        & 0.62         \\ 
\bottomrule % Replaces \hline for a bottom border with better spacing
\end{tabular}%
}
\end{table}
% % Table 5
\begin{table*}[!htbp] % Adjust placement specifiers to be consistent
\centering
\caption{CC Attack Potential Metrics \\ Elapsed Time (ET) is a temporal aspect, indicating the viability of an attack from immediate (None) to one requiring extensive preparation (over ten years). Equipment (E) considers the tools at the attacker’s disposal.}
\label{tab:cc}
\tiny
\renewcommand{\arraystretch}{0.10} % Adjust space between rows
\resizebox{\textwidth}{!}{%
\begin{tabular}{@{}c*{5}{c}@{}}
\toprule
\textbf{CC Attack Potential}                                                              & \multicolumn{5}{c}{\textbf{Metric Value}} \\
\midrule
\multirow{2}{*}{Elapsed Time (ET)} & None       & $\leq$ 6 Months  & $\leq$ 5 Years & $\leq$ 10 Years & $\ge$ 10 Years \\ 
                                    & 0.85          & 0.78                 & 0.42                & 0.05                & 0                   \\ 
\addlinespace
\multirow{2}{*}{Equipment (E)}      & Standard & Specialised     & Bespoke         & \multicolumn{2}{c}{Multiple Bespoke} \\ 
                                    & 0.85          & 0.47                 & 0.35                & \multicolumn{2}{c}{0.3}                   \\
\bottomrule
\end{tabular}%
}
\end{table*}
% \begin{table}[!htbp]
\centering
\caption{Severity of Attack Vectors using CVSS and CC Metrics}
\label{tab:prob}
\tiny
% \setlength{\tabcolsep}{1.5pt} % Reduce space between columns
\renewcommand{\arraystretch}{0.76} % Adjust space between rows
\resizebox{\textwidth}{!}{%
\begin{tabular}{
    p{2cm}
    >{\centering\arraybackslash}p{0.5cm}
    >{\centering\arraybackslash}p{0.5cm}
    >{\centering\arraybackslash}p{0.5cm}
    >{\centering\arraybackslash}p{0.6cm}
    >{\centering\arraybackslash}p{0.4cm}
    >{\centering\arraybackslash}p{0.4cm}
    c}
\toprule
\multirow{2}{*}{\textbf{Attack Vector}} & \multicolumn{4}{c}{\textbf{CVSS Exploitability}} & \multicolumn{2}{c}{\textbf{CC Attack Potential}} & \multirow{2}{*}{\textbf{S(T)}} \\ 
                                & AV & AC & PR & UI & ET & E &  \\ 
\midrule
Rogue AP                        & 0.62 & 0.77 & 0.62 & 0.62 & 0.78 & 0.47 & 0.067276 \\
% ARP Spoofing                    & 0.62 & 0.44 & 0.85 & 0.85 & 0.85 & 0.47 & 0.078741 \\
DNS Spoofing                    & 0.85 & 0.44 & 0.85 & 0.85 & 0.78 & 0.47 & 0.099061 \\
IP Spoofing                     & 0.85 & 0.44 & 0.85 & 0.85 & 0.78 & 0.47 & 0.099061 \\
ICMP Flooding                   & 0.85 & 0.44 & 0.85 & 0.85 & 0.85 & 0.47 & 0.107951 \\
TCP SYN Flooding                & 0.85 & 0.44 & 0.85 & 0.85 & 0.85 & 0.47 & 0.107951 \\
Malware                         & 0.85 & 0.77 & 0.85 & 0.62 & 0.85 & 0.85 & 0.249206 \\
Android Root Exploitation       & 0.20 & 0.44 & 0.27 & 0.62 & 0.78 & 0.85 & 0.009767 \\
USB Debugging                        & 0.55 & 0.44 & 0.27 & 0.62 & 0.05 & 0.35 & 0.000709 \\
Logic Flow Exploitation         & 0.55 & 0.44 & 0.27 & 0.62 & 0.78 & 0.47 & 0.014851 \\
Phishing                        & 0.85 & 0.77 & 0.62 & 0.62 & 0.78 & 0.47 & 0.092233 \\
Brute-force                     & 0.20 & 0.77 & 0.62 & 0.85 & 0.78 & 0.85 & 0.053808 \\
Fake Biometrics                 & 0.20 & 0.44 & 0.85 & 0.62 & 0.42 & 0.35 & 0.006817 \\
Cold Boot Attack                 & 0.20 & 0.44 & 0.85 & 0.85 & 0.42 & 0.35 & 0.009346 \\
Row Hammer Attack               & 0.55 & 0.44 & 0.27 & 0.85 & 0.42 & 0.35 & 0.008164 \\
% % Botnet                          & 0.85 & 0.44 & 0.85 & 0.85 & 0.78 & 0.35 & 0.073769 \\
% Supply Chain Attack             & 0.20 & 0.44 & 0.27 & 0.62 & 0.42 & 0.30 & 0.001856 \\
% Keylogger                       & 0.55 & 0.77 & 0.62 & 0.62 & 0.78 & 0.85 & 0.107932 \\
% Removable Media Infection       & 0.20 & 0.77 & 0.27 & 0.62 & 0.85 & 0.85 & 0.018626 \\
% Code Reuse                      & 0.55 & 0.44 & 0.62 & 0.85 & 0.78 & 0.85 & 0.084555 \\
% Shoulder Surfing                & 0.20 & 0.77 & 0.85 & 0.62 & 0.85 & 0.85 & 0.058637 \\
% Fault Injection                 & 0.55 & 0.44 & 0.27 & 0.85 & 0.05 & 0.35 & 0.000972 \\
Side-channel Attack             & 0.55 & 0.44 & 0.85 & 0.85 & 0.42 & 0.47 & 0.034514 \\
Signature Exploitation     & 0.55 & 0.44 & 0.85 & 0.85 & 0.42 & 0.35 & 0.025702 \\
Nonce Reuse                     & 0.55 & 0.44 & 0.85 & 0.85 & 0.42 & 0.35 & 0.025702 \\
% Social Engineering              & 0.85 & 0.77 & 0.62 & 0.62 & 0.78 & 0.85 & 0.166804 \\
% Network Packet Sniffer          & 0.85 & 0.44 & 0.85 & 0.85 & 0.85 & 0.85 & 0.195230 \\
% Disk Formatting                 & 0.55 & 0.77 & 0.27 & 0.62 & 0.85 & 0.85 & 0.051221 \\

\bottomrule
\end{tabular}%
}
\end{table}


% \begin{tabular}{p{2.5cm}|p{0.35cm}p{0.35cm}p{0.35cm}p{0.35cm}p{0.35cm}p{0.35cm}c}

% & Attack Vector & Attack Complexity & Privilege Required & User Interaction & Elapsed Time & Equipment 


% Rogue AP                        & {\makecell{A\\0.62}} & {\makecell{L\\0.77}} & {\makecell{L\\0.62}} & {\makecell{R\\0.62}} & {\makecell{Si\\0.78}} & {\makecell{Sp\\0.47}} & 0.067276 \\
% ARP Spoofing                    & {\makecell{A\\0.62}} & {\makecell{H\\0.44}} & {\makecell{N\\0.85}} & {\makecell{N\\0.85}} & {\makecell{N\\0.85}}  & {\makecell{Sp\\0.47}} & 0.078741 \\
% DNS Spoofing                    & {\makecell{Ne\\0.85}} & {\makecell{H\\0.44}} & {\makecell{N\\0.85}} & {\makecell{N\\0.85}} & {\makecell{Si\\0.78}} & {\makecell{Sp\\0.47}} & 0.099061 \\
% IP Spoofing                     & {\makecell{Ne\\0.85}} & {\makecell{H\\0.44}} & {\makecell{N\\0.85}} & {\makecell{N\\0.85}} & {\makecell{Si\\0.78}} & {\makecell{Sp\\0.47}} & 0.099061 \\
% ICMP Flooding                   & {\makecell{Ne\\0.85}} & {\makecell{H\\0.44}} & {\makecell{N\\0.85}} & {\makecell{N\\0.85}} & {\makecell{N\\0.85}}  & {\makecell{Sp\\0.47}} & 0.107951 \\
% TCP SYN Flooding                & {\makecell{Ne\\0.85}} & {\makecell{H\\0.44}} & {\makecell{N\\0.85}} & {\makecell{N\\0.85}} & {\makecell{N\\0.85}}  & {\makecell{Sp\\0.47}} & 0.107951 \\
% Botnet                          & \makecell{Ne\\0.85} & \makecell{H\\0.44} & \makecell{N\\0.85} & \makecell{N\\0.85} & \makecell{Si\\0.78} & \makecell{B\\0.35} & 0.073769 \\
% Supply Chain Attack             & \makecell{P\\0.2} & \makecell{H\\0.44} & \makecell{H\\0.27} & \makecell{R\\0.62} & \makecell{Fi\\0.42} & \makecell{MB\\0.3} & 0.001856 \\
% Keylogger                       & \makecell{Lo\\0.55} & \makecell{L\\0.77} & \makecell{L\\0.62} & \makecell{R\\0.62} & \makecell{Si\\0.78} & \makecell{St\\0.85} & 0.107932 \\
% Phishing                        & \makecell{Ne\\0.85} & \makecell{L\\0.77} & \makecell{L\\0.62} & \makecell{R\\0.62} & \makecell{Si\\0.78} & \makecell{Sp\\0.47} & 0.092233 \\
% Removable Media Infection       & \makecell{P\\0.2} & \makecell{L\\0.77} & \makecell{H\\0.27} & \makecell{R\\0.62} & \makecell{N\\0.85} & \makecell{St\\0.85} & 0.018626 \\
% Android Root Exploitation       & \makecell{P\\0.2} & \makecell{H\\0.44} & \makecell{H\\0.27} & \makecell{R\\0.62} & \makecell{Si\\0.78} & \makecell{St\\0.85} & 0.009767 \\
% Logic Flow Exploitation         & \makecell{Lo\\0.55} & \makecell{H\\0.44} & \makecell{H\\0.27} & \makecell{R\\0.62} & \makecell{Si\\0.78} & \makecell{Sp\\0.47} & 0.014851 \\
% Row Hammer Attack               & \makecell{Lo\\0.55} & \makecell{H\\0.44} & \makecell{H\\0.27} & \makecell{N\\0.85} & \makecell{Fi\\0.42} & \makecell{B\\0.35} & 0.008164 \\
% Brute-force                     & \makecell{P\\0.2} & \makecell{L\\0.77} & \makecell{L\\0.62} & \makecell{N\\0.85} & \makecell{Si\\0.78} & \makecell{St\\0.85} & 0.053808 \\
% Code Reuse                      & \makecell{Lo\\0.55} & \makecell{H\\0.44} & \makecell{L\\0.62} & \makecell{N\\0.85} & \makecell{Si\\0.78} & \makecell{St\\0.85} & 0.084555 \\
% Fake Biometrics                 & \makecell{P\\0.2} & \makecell{H\\0.44} & \makecell{N\\0.85} & \makecell{R\\0.62} & \makecell{Fi\\0.42} & \makecell{B\\0.35} & 0.006817 \\
% Shoulder Surfing                & \makecell{P\\0.2} & \makecell{L\\0.77} & \makecell{N\\0.85} & \makecell{R\\0.62} & \makecell{N\\0.85} & \makecell{St\\0.85} & 0.058637 \\
% Fault Injection                 & \makecell{Lo\\0.55} & \makecell{H\\0.44} & \makecell{H\\0.27} & \makecell{N\\0.85} & \makecell{T\\0.05} & \makecell{B\\0.35} & 0.000972 \\
% Debugger                        & \makecell{Lo\\0.55} & \makecell{H\\0.44} & \makecell{H\\0.27} & \makecell{R\\0.62} & \makecell{T\\0.05} & \makecell{B\\0.35} & 0.000709 \\
% Coldboot Attack                 & \makecell{P\\0.2} & \makecell{H\\0.44} & \makecell{N\\0.85} & \makecell{N\\0.85} & \makecell{Fi\\0.42} & \makecell{B\\0.35} & 0.009346 \\
% Side-channel Attack             & \makecell{Lo\\0.55} & \makecell{H\\0.44} & \makecell{N\\0.85} & \makecell{N\\0.85} & \makecell{Fi\\0.42} & \makecell{Sp\\0.47} & 0.034514 \\
% Weak Signature Exploitation     & \makecell{Lo\\0.55} & \makecell{H\\0.44} & \makecell{N\\0.85} & \makecell{N\\0.85} & \makecell{Fi\\0.42} & \makecell{B\\0.35} & 0.025702 \\
% Nonce Reuse                     & \makecell{Lo\\0.55} & \makecell{H\\0.44} & \makecell{N\\0.85} & \makecell{N\\0.85} & \makecell{Fi\\0.42} & \makecell{B\\0.35} & 0.025702 \\
% Social Engineering              & \makecell{Ne\\0.85} & \makecell{L\\0.77} & \makecell{L\\0.62} & \makecell{R\\0.62} & \makecell{Si\\0.78} & \makecell{St\\0.85} & 0.166804 \\
% Network Packet Sniffer          & \makecell{Ne\\0.85} & \makecell{H\\0.44} & \makecell{N\\0.85} & \makecell{N\\0.85} & \makecell{N\\0.85} & \makecell{St\\0.85} & 0.195230 \\
% Malware                         & \makecell{Ne\\0.85} & \makecell{L\\0.77} & \makecell{N\\0.85} & \makecell{R\\0.62} & \makecell{N\\0.85} & \makecell{St\\0.85} & 0.249206 \\
% Disk Formatting                 & \makecell{Lo\\0.55} & \makecell{L\\0.77} & \makecell{H\\0.27} & \makecell{N\\0.85} & \makecell{N\\0.85} & \makecell{St\\0.85} & 0.051221 \\


% -- Old Format --
% Rogue AP                        & A & L & L & R & Si & Sp & \multirow{2}{*}{0.067276} \\
%                                 & 0.62 & 0.77 & 0.62 & 0.62 & 0.78 & 0.47 &  \\
% ARP Spoofing                    & A & H & N & N & N & Sp & \multirow{2}{*}{0.078741} \\
%                                 & 0.62 & 0.44 & 0.85 & 0.85 & 0.85 & 0.47 &  \\
% DNS Spoofing                    & Ne & H & N & N & Si & Sp & \multirow{2}{*}{0.099061} \\
%                                 & 0.85 & 0.44 & 0.85 & 0.85 & 0.78 & 0.47 &  \\
% IP Spoofing                     & Ne & H & N & N & Si & Sp & \multirow{2}{*}{0.099061} \\
%                                 & 0.85 & 0.44 & 0.85 & 0.85 & 0.78 & 0.47 &  \\
% ICMP Flooding                   & Ne & H & N & N & N & Sp & \multirow{2}{*}{0.107951} \\
%                                 & 0.85 & 0.44 & 0.85 & 0.85 & 0.85 & 0.47 &  \\
% TCP SYN Flooding                & Ne & H & N & N & N & Sp & \multirow{2}{*}{0.107951} \\
%                                 & 0.85 & 0.44 & 0.85 & 0.85 & 0.85 & 0.47 &  \\
% Botnet                          & Ne & H & N & N & Si & B & \multirow{2}{*}{0.073769} \\
%                                 & 0.85 & 0.44 & 0.85 & 0.85 & 0.78 & 0.35 &  \\
% Supply Chain Attack             & P & H & H & R & Fi & MB & \multirow{2}{*}{0.001856} \\
%                                 & 0.2 & 0.44 & 0.27 & 0.62 & 0.42 & 0.3 &  \\
% Keylogger                       & Lo & L & L & R & Si & St & \multirow{2}{*}{0.107932} \\
%                                 & 0.55 & 0.77 & 0.62 & 0.62 & 0.78 & 0.85 &  \\
% Phishing                        & Ne & L & L & R & Si & Sp & \multirow{2}{*}{0.092233} \\
%                                 & 0.85 & 0.77 & 0.62 & 0.62 & 0.78 & 0.47 &  \\
% Removable Media Infection       & P & L & H & R & N & St & \multirow{2}{*}{0.018626} \\
%                                 & 0.2 & 0.77 & 0.27 & 0.62 & 0.85 & 0.85 &  \\
% Android Root Exploitation       & P & H & H & R & Si & St & \multirow{2}{*}{0.009767} \\
%                                 & 0.2 & 0.44 & 0.27 & 0.62 & 0.78 & 0.85 &  \\
% Logic Flow Exploitation         & Lo & H & H & R & Si & Sp & \multirow{2}{*}{0.014851} \\
%                                 & 0.55 & 0.44 & 0.27 & 0.62 & 0.78 & 0.47 &  \\
% Row Hammer Attack               & Lo & H & H & N & Fi & B & \multirow{2}{*}{0.008164} \\
%                                 & 0.55 & 0.44 & 0.27 & 0.85 & 0.42 & 0.35 &  \\
% Brute-force                     & P & L & L & N & Si & St & \multirow{2}{*}{0.053808} \\
%                                 & 0.2 & 0.77 & 0.62 & 0.85 & 0.78 & 0.85 &  \\
% Code Reuse                      & Lo & H & L & N & Si & St & \multirow{2}{*}{0.084555} \\
%                                 & 0.55 & 0.44 & 0.62 & 0.85 & 0.78 & 0.85 &  \\
% Fake Biometrics                 & P & H & N & R & Fi & B & \multirow{2}{*}{0.006817} \\
%                                 & 0.2 & 0.44 & 0.85 & 0.62 & 0.42 & 0.35 &  \\
% Shoulder Surfing                & P & L & N & R & N & St & \multirow{2}{*}{0.058637} \\
%                                 & 0.2 & 0.77 & 0.85 & 0.62 & 0.85 & 0.85 &  \\
% Fault Injection                 & Lo & H & H & N & T & B & \multirow{2}{*}{0.000972} \\
%                                 & 0.55 & 0.44 & 0.27 & 0.85 & 0.05 & 0.35 &  \\
% Debugger                        & Lo & H & H & R & T & B & \multirow{2}{*}{0.000709} \\
%                                 & 0.55 & 0.44 & 0.27 & 0.62 & 0.05 & 0.35 &  \\
% Coldboot Attack                 & P & H & N & N & Fi & B & \multirow{2}{*}{0.009346} \\
%                                 & 0.2 & 0.44 & 0.85 & 0.85 & 0.42 & 0.35 &  \\
% Side-channel Attack             & Lo & H & N & N & Fi & Sp & \multirow{2}{*}{0.034514} \\
%                                 & 0.55 & 0.44 & 0.85 & 0.85 & 0.42 & 0.47 &  \\
% Signature Exploitation     & Lo & H & N & N & Fi & B & \multirow{2}{*}{0.025702} \\
%                                 & 0.55 & 0.44 & 0.85 & 0.85 & 0.42 & 0.35 &  \\
% Nonce Reuse                     & Lo & H & N & N & Fi & B & \multirow{2}{*}{0.025702} \\
%                                 & 0.55 & 0.44 & 0.85 & 0.85 & 0.42 & 0.35 &  \\
% Social Engineering              & Ne & L & L & R & Si & St & \multirow{2}{*}{0.166804} \\
%                                 & 0.85 & 0.77 & 0.62 & 0.62 & 0.78 & 0.85 &  \\
% Network Packet Sniffer          & Ne & H & N & N & N & St & \multirow{2}{*}{0.195230} \\
%                                 & 0.85 & 0.44 & 0.85 & 0.85 & 0.85 & 0.85 &  \\
% Malware                         & Ne & L & N & R & N & St & \multirow{2}{*}{0.249206} \\
%                                 & 0.85 & 0.77 & 0.85 & 0.62 & 0.85 & 0.85 &  \\
% Disk Formatting                 & Lo & L & H & N & N & St & \multirow{2}{*}{0.051221} \\
%                                 & 0.55 & 0.77 & 0.27 & 0.62 & 0.85 & 0.85 &  \\

% Wallets
% \begin{landscape}
%     \begin{table*}[h]
	\caption{Comparision of various cryptocurrency wallets (Non-Custodial). (\CIRCLE: include, \Circle: not include)}
	\label{CryptoWallets-Table-1}
	\tiny
	\renewcommand{\arraystretch}{2.5}
	\resizebox{1.35\textwidth}{!}{
		\begin{tabular}{l|l|cccll|ccccccccc|cl|ll|cccc|lc}
			\hline
			\multirow{2}{*}{Wallet} & \multirow{2}{*}{Wallet Type} & \multicolumn{5}{c}{Security}                                                   & \multicolumn{9}{c}{Cryptocurrency   Supported}                                              & \multicolumn{2}{c}{Software} & \multicolumn{2}{c}{Revenue Model}                      & \multicolumn{4}{c}{Payment Methods}                & \multicolumn{2}{c}{Company's Info} \\ \cline{3-26}
			                                                                                 &          & 2FA     & Multi-Sig & Pin Code & Backup                          & Private Key & BTC         & ETH     & XRP     & HBAR                                   & SOL      & ERC20   & Legacy Addr & SegWit  & Bech32          & Fee Control & Transparency  & Base Price & Service Fee                               & Wire Transfer                                                                  & Card     & PayPal  & Cryptocurrency & Origin      & Year                   \\ \hline
			\href{https://shop.ledger.com/products/ledger-nano-x}{Ledger Nano X}             & Hardware & \CIRCLE & \Circle   & \CIRCLE  & 24-word                         & You         & \CIRCLE     & \CIRCLE & \CIRCLE & \CIRCLE                                & \CIRCLE  & \CIRCLE & \CIRCLE & \CIRCLE & \CIRCLE         & \CIRCLE & Open-Source   & 119 EUR    & N/A                                       & \CIRCLE                                                                        & \CIRCLE  & \CIRCLE & \CIRCLE & France      & 2014                   \\
			\href{https://trezor.io/trezor-model-t}{Trezor Model T}                          & Hardware & \CIRCLE & \CIRCLE   & \CIRCLE  & 12-word                         & You         & \CIRCLE     & \CIRCLE & \Circle & \Circle                                & \Circle  & \CIRCLE & \CIRCLE & \CIRCLE & \CIRCLE         & \CIRCLE & Open-Source   & 169.99 USD & N/A                                       & \Circle                                                                        & \CIRCLE  & \Circle & \CIRCLE & Prague      & 2017                   \\
			\href{https://shop.ledger.com/products/ledger-nano-s-plus}{Ledger Nano S Plus}   & Hardware & \CIRCLE & \Circle   & \CIRCLE  & 24-word                         & You         & \CIRCLE     & \CIRCLE & \CIRCLE & \CIRCLE                                & \CIRCLE  & \CIRCLE & \CIRCLE & \CIRCLE & \CIRCLE         & \CIRCLE & Open-Source   & 59 EUR     & N/A                                       & \Circle                                                                        & \CIRCLE  & \CIRCLE & \CIRCLE & France      & 2014                   \\
			\href{https://www.coinbase.com/wallet}{Coinbase Wallet}                          & Software & \CIRCLE & \CIRCLE   & \CIRCLE  & Seed Phrase                     & You         & \CIRCLE     & \CIRCLE & \Circle & \CIRCLE                                & \CIRCLE  & \CIRCLE & \CIRCLE & \CIRCLE & \CIRCLE         & \CIRCLE & Open-Source   & 0 USD      & N/A                                       & \CIRCLE                                                                        & \CIRCLE  & \CIRCLE & \CIRCLE & USA         & 2018                   \\
			\href{https://www.binance.com/en-IN/web3wallet}{Binance Wallet}                  & Software & \CIRCLE & \CIRCLE   & \CIRCLE  & Seed Phrase                     & Wallet      & \CIRCLE     & \CIRCLE & \CIRCLE & \CIRCLE                                & \CIRCLE  & \CIRCLE & \CIRCLE & \CIRCLE & \CIRCLE         & \CIRCLE & Close-Source  & 0 USD      & N/A                                       & \CIRCLE                                                                        & \CIRCLE  & \CIRCLE & \CIRCLE & Malta       & 2017                   \\
			\href{https://www.keepkey.com/}{KeepKey}                                         & Hardware & \CIRCLE & \Circle   & \CIRCLE  & 24-word                         & You         & \CIRCLE     & \CIRCLE & \CIRCLE & \Circle                                & \Circle  & \CIRCLE & \CIRCLE & \CIRCLE & \CIRCLE         & \CIRCLE & Open-Source   & 49 USD     & N/A                                       & \CIRCLE                                                                        & \CIRCLE  & \CIRCLE & \CIRCLE & USA         & 2014                   \\
			\href{https://robinhood.com/us/en/support/articles/robinhood-wallet/}{Robinhood} & Software & \CIRCLE & \CIRCLE   & \CIRCLE  & Seed Phrase                     & Wallet      & \CIRCLE     & \CIRCLE & \Circle & \Circle                                & \Circle  & \Circle & \Circle & \Circle & \Circle         & \CIRCLE & Open-Source   & 0 USD      & N/A                                       & \CIRCLE                                                                        & \CIRCLE  & \Circle & \Circle & USA         & 2013                   \\
			\href{https://zengo.com/}{ZenGo}                                                 & Software & \CIRCLE & \Circle   & \CIRCLE  & Google, iCloud                  & Partial     & \CIRCLE     & \CIRCLE & \Circle & \Circle                                & \CIRCLE  & \CIRCLE & \CIRCLE & \CIRCLE & \CIRCLE         & \CIRCLE & Open-Source   & 0 USD      & Trading 0.75\%,   Withdrawal N/A          & \CIRCLE                                                                        & \CIRCLE  & \CIRCLE & \Circle & Israel      & 2018                   \\
			\href{https://electrum.org/}{Electrum}                                           & Software & \CIRCLE & \CIRCLE   & \CIRCLE  & Seed Phrase                     & You         & \CIRCLE     & \Circle & \Circle & \Circle                                & \Circle  & \CIRCLE & \CIRCLE & \CIRCLE & \CIRCLE         & \CIRCLE & Open-Source   & 0          & Fee adjustment   setting                  & \CIRCLE                                                                        & \CIRCLE  & \CIRCLE & \CIRCLE & EU          & 2011                   \\
			\href{https://metamask.io/}{MetaMask}                                            & Software & \CIRCLE & \CIRCLE   & \CIRCLE  & Seed Phrase                     & You         & \Circle     & \CIRCLE & \Circle & \Circle                                & \Circle  & \CIRCLE & \CIRCLE & \CIRCLE & \CIRCLE         & \CIRCLE & Open-Source   & 0 USD      & Trading 0.3\% -   0.875\%, Withdrawal N/A & \Circle                                                                        & \Circle  & \CIRCLE & \Circle & USA         & 2016                   \\
			\href{https://www.exodus.com/}{Exodus}                                           & Software & \Circle & \Circle   & \CIRCLE  & 12-word                         & You         & \CIRCLE     & \CIRCLE & \CIRCLE & \CIRCLE                                & \CIRCLE  & \CIRCLE & \CIRCLE & \CIRCLE & \CIRCLE         & \CIRCLE & Open-Source   & 0 USD      & Trading 0, Withdrawal   0.1\%             & \CIRCLE                                                                        & \CIRCLE  & \CIRCLE & \CIRCLE & USA         & 2015                   \\
			\href{https://atomicwallet.io/bread-wallet}{BRD}                                 & Software & \Circle & \Circle   & \CIRCLE  & 12-word                         & You         & \CIRCLE     & \CIRCLE & \CIRCLE & \CIRCLE                                & \Circle  & \CIRCLE & \CIRCLE & \CIRCLE & \CIRCLE         & \CIRCLE & Open-Source   & 0 USD      & Trading 0, Withdrawal   N/A               & \CIRCLE                                                                        & \CIRCLE  & \Circle & \Circle & Switzerland & 2015                   \\
			\href{https://www.myetherwallet.com/}{MyEtherWallet}                             & Software & \Circle & \CIRCLE   & \CIRCLE  & 24 words                        & You         & \Circle     & \CIRCLE & \Circle & \Circle                                & \Circle  & \CIRCLE & \CIRCLE & \CIRCLE & \Circle         & \CIRCLE & Open-Source   & 0 USD      & Trading 0, Withdrawal   N/A               & \CIRCLE                                                                        & \CIRCLE  & \CIRCLE & \CIRCLE & Switzerland & 2015                   \\
			\href{https://edge.app/}{Edge}                                                   & Software & \CIRCLE & \CIRCLE   & \CIRCLE  & Seed Phrase                     & You         & \CIRCLE     & \CIRCLE & \CIRCLE & \CIRCLE                                & \CIRCLE  & \CIRCLE & \Circle & \CIRCLE & \CIRCLE         & \CIRCLE & Open-Source   & 0 USD      & 4-6\%                                     & \CIRCLE                                                                        & \CIRCLE  & \Circle & \CIRCLE & USA         & 2014                   \\
			\href{https://myjaxx.io/\#/intro/create-wallet}{Jaxx}                             & Software & \CIRCLE & \CIRCLE   & \CIRCLE  & 12-word                         & You         & \CIRCLE     & \CIRCLE & \CIRCLE & \Circle                                & \Circle  & \CIRCLE & \CIRCLE & \CIRCLE & \CIRCLE         & \CIRCLE & Open-Source   & 0 USD      & Automatic fee   adjustment                & \CIRCLE                                                                        & \CIRCLE  & \CIRCLE & \Circle & Canada      & 2014                   \\
			\href{https://wallet.cex.io/}{Cex Wallet}                                        & Software & \CIRCLE & \CIRCLE   & \CIRCLE  & Google   Authenticator          & Wallet      & \CIRCLE     & \CIRCLE & \CIRCLE & \Circle                                & \CIRCLE  & \CIRCLE & \Circle & \CIRCLE & \Circle         & \CIRCLE & Open-Source   & 0 USD      & Trading 0.25\%,   Withdrawal up to 3\%    & \CIRCLE                                                                        & \CIRCLE  & \Circle & \CIRCLE & UK          & 2013                   \\
			\href{https://bitpay.com/wallet/}{Bitpay}                                        & Software & \Circle & \CIRCLE   & \CIRCLE  & Seed Phrase                     & You         & \CIRCLE     & \CIRCLE & \CIRCLE & \Circle                                & \Circle  & \CIRCLE & \Circle & \CIRCLE & \CIRCLE         & \CIRCLE & Open-Source   & 0 USD      & N/A                                       & \CIRCLE                                                                        & \CIRCLE  & \CIRCLE & \CIRCLE & USA         & 2015                   \\
			\href{https://wallet.mycelium.com/}{MyCelium}                                    & Software & \CIRCLE & \CIRCLE   & \CIRCLE  & 12-word                         & You         & \CIRCLE     & \CIRCLE & \Circle & \Circle                                & \Circle  & \Circle & \Circle & \CIRCLE & \CIRCLE         & \CIRCLE & Open-Source   & 0  USD     & N/A                                       & \CIRCLE                                                                        & \CIRCLE  & \CIRCLE & \CIRCLE & Austria     & 2008                   \\
			\href{https://gatehub.net/}{Gatehub}                                             & Software & \CIRCLE & \Circle   & \Circle  & 2FA Backup Code                 & You         & \CIRCLE     & \CIRCLE & \CIRCLE & \Circle                                & \Circle  & \Circle & \Circle & \Circle & \Circle         & \CIRCLE & Open-Source   & 0 USD      & Trading 0.25\%,   Withdrawal 0.001 BTC    & \CIRCLE                                                                        & \Circle  & \Circle & \CIRCLE & UK          & 2014                   \\
			\href{https://play.google.com/store/apps/details?id=com.kucoin.wallet}{KuCoin}   & Software & \CIRCLE & \CIRCLE   & \Circle  & Google   Authenticator          & Wallet      & \CIRCLE     & \CIRCLE & \CIRCLE & \CIRCLE                                & \CIRCLE  & \CIRCLE & \Circle & \CIRCLE & \CIRCLE         & \CIRCLE & Open-Source   & 0 USD      & Trading 0.1\%,   Withdrawal 0.0005 BTC    & \CIRCLE                                                                        & \CIRCLE  & \CIRCLE & \Circle & Singapore   & 2017                   \\
			\href{https://www.blockchain.com/wallet}{Blockchain}                             & Software & \CIRCLE & \CIRCLE   & \CIRCLE  & Seed Phrase                     & Wallet      & \CIRCLE     & \CIRCLE & \CIRCLE & \Circle                                & \CIRCLE  & \Circle & \Circle & \Circle & \Circle         & \CIRCLE & Open-Source   & 0 USD      & Trading 0.4\%,   Withdrawal \$25-\$30     & \CIRCLE                                                                        & \CIRCLE  & \Circle & \CIRCLE & Luxembourg  & 2011                   \\
			\href{https://guarda.com/}{Guarda}                                               & Software & \CIRCLE & \CIRCLE   & \CIRCLE  & Seed Phrase                     & You         & \CIRCLE     & \CIRCLE & \CIRCLE & \CIRCLE                                & \CIRCLE  & \CIRCLE & \CIRCLE & \CIRCLE & \CIRCLE         & \CIRCLE & Open-Source   & 0 USD      & 3.5\%                                     & \CIRCLE                                                                        & \CIRCLE  & \Circle & \Circle & Tallin      & 2017                   \\
			\href{https://www.bitcoin.com/}{Bitcoin.com}                                     & Software & \Circle & \CIRCLE   & \CIRCLE  & 12-word                         & You         & \CIRCLE     & \Circle & \Circle & \Circle                                & \Circle  & \CIRCLE & \CIRCLE & \CIRCLE & N/A             & \CIRCLE & Open-Source   & 0 USD      & Trading 0.2\%,   Withdrawal 0.00005 BTC   & \CIRCLE                                                                        & \CIRCLE  & \CIRCLE & \Circle & Nevis       & 2015                   \\
			\href{https://freewallet.org/crypto-wallet}{Freewallet}                          & Software & \CIRCLE & \CIRCLE   & \CIRCLE  & Seed Phrase                     & Wallet      & \CIRCLE     & \CIRCLE & \CIRCLE & \Circle                                & \Circle  & \CIRCLE & \Circle & \CIRCLE & \CIRCLE         & \CIRCLE & Open-Source   & 0 USD      & Trading 0.5\%,   Withdrawal N/A           & \CIRCLE                                                                        & \CIRCLE  & \Circle & \Circle & Estonia     & 2016                   \\
			\href{https://www.coolwallet.io/product/coolwallet/}{CoolWallet S}               & Hardware & \CIRCLE & \CIRCLE   & \CIRCLE  & Seed Phrase                     & You         & \CIRCLE     & \CIRCLE & \CIRCLE & \Circle                                & \CIRCLE  & \CIRCLE & \CIRCLE & \CIRCLE & \CIRCLE         & \CIRCLE & Closed-Source & 99 USD     & N/A                                       & \CIRCLE                                                                        & \CIRCLE  & \CIRCLE & \CIRCLE & Taiwan      & 2018                   \\
			\href{https://atomicwallet.io/}{Atomic}                                          & Software & \CIRCLE & \CIRCLE   & \CIRCLE  & 12-word                         & You         & \CIRCLE     & \CIRCLE & \CIRCLE & \CIRCLE                                & \CIRCLE  & \CIRCLE & \CIRCLE & \CIRCLE & \CIRCLE         & \CIRCLE & Open-Source   & 0 USD      & Trading 2\%, Withdrawal   N/A             & \CIRCLE                                                                        & \CIRCLE  & \CIRCLE & \CIRCLE & Tallin      & 2017                   \\
			\href{https://bc-vault.com/}{BC Vault}                                           & Hardware & \CIRCLE & \CIRCLE   & \CIRCLE  & MicroSD Card, Printed   QR Code & You         & \CIRCLE     & \CIRCLE & \CIRCLE & \Circle                                & \CIRCLE  & \CIRCLE & \CIRCLE & \CIRCLE & \CIRCLE         & \CIRCLE & Open-Source   & 180 USD    & N/A                                       & \CIRCLE                                                                        & \CIRCLE  & \Circle & \CIRCLE & Slovenia    & 2018                   \\
			\href{http://secalot.com/}{Secalot}                                              & Hardware & \CIRCLE & \CIRCLE   & \CIRCLE  & 24-word                         & You         & \CIRCLE     & \CIRCLE & \CIRCLE & \Circle                                & \Circle  & \CIRCLE & \CIRCLE & \CIRCLE & \Circle         & \CIRCLE & Open-Source   & 50 USD     & N/A                                       & \CIRCLE                                                                        & \CIRCLE  & \CIRCLE & \CIRCLE & France      & 2017                   \\ \hline
			
		\end{tabular}
    }
\end{table*}
% \end{landscape}

% The table highlights the variety and severity of attacks, ranging from phishing and malware to private key compromises and insider jobs, affecting both custodial and non-custodial wallets.




