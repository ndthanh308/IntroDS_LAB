\section{Methodology}
\label{sec:methodology}

The methodology of this research is driven by the primary goal of systematically analyzing the security vulnerabilities and attack incidents associated with cryptocurrency wallets. By examining both academic research and real-world attack cases, we aim to bridge the gap between theoretical security measures and practical challenges faced in the industry. This comprehensive approach allows us to identify key areas of concern, understand the nature and impact of various attack vectors, and explore effective countermeasures to enhance wallet security. 

\subsection{System Model}
\label{sec:system_model}

Our attacks framework categorizes all prominent wallet attacks into five categories: Network, Application, Authentication, Physical, and Cryptanalysis as shown in \autoref{fig:attack-vectors}. Network attacks target the communication between a wallet and the blockchain network, encompassing threats such as MiTM and DoS. Application attacks encompass malware, privilege escalation, and software vulnerabilities, targeting user interactions, and software libraries to compromise assets and sensitive information. Authentication attacks target the private key or sensitive information directly to gain unauthorized access to the accounts of users. Physical attacks target the hardware components of wallets, such as theft or tampering with physical devices to extract data. Cryptanalysis attacks involve the mathematical analysis of the cryptographic algorithms used within wallets to find vulnerabilities that can be exploited to gain access to the wallet, thereby compromising the integrity and confidentiality of wallet data.

% Figure environment removed

\subsection{Threat Model and Analysis}
As the specific methodologies employed in various attacks on crypto wallets remain elusive, it is imperative to establish a robust and systematic approach to analyze these threats. Our proposed methodology for evaluating the threats to a crypto wallet system is structured into two distinct but interconnected stages: threat modeling and threat analysis. 

% This comprehensive risk assessment process is illustrated in the accompanying \autoref{fig:methodology}.

% % Figure environment removed


The first stage, threat model (see \autoref{sec:threat-model}), involves representing the crypto wallet system through a detailed data flow diagram (see \autoref{fig:dfd}). This diagrammatic representation serves as a foundation for identifying potential threats, utilizing the STRIDE analysis framework in \autoref{sec:stride-technique}. A key component of this stage is the development of an attack tree in \autoref{sec:attack-tree}, which outlines possible threat scenarios in a hierarchical structure.

The second stage, threat analysis (see \autoref{sec:threat-analysis}), takes the findings from the threat modeling stage and translates them into a Fault Tree Analysis (see \autoref{fig:fault-tree-analysis}). This analysis delves deeper into the potential vulnerabilities within the system. To quantify and prioritize these risks, we employ the Common Vulnerability Scoring System (CVSS) and Common Criteria (CC) Attack Potential (see \autoref{sec:probability-calculation}), which provides a standardized way to assess the severity of security vulnerabilities. By applying CVSS and CC Attack Potential, we aim to pinpoint the most vulnerable aspects of crypto wallets, thereby enabling targeted and effective security enhancements. This two-stage methodology not only identifies existing threats but also anticipates emerging risks, ensuring that the crypto wallet system remains resilient against a constantly evolving landscape of digital threats.