\section{Discussion}
\label{sec:discussion}

\subsection{Limitations}

% A limitation of our study is the limited availability of high-quality data regarding attacks on wallets, with many incidents lacking detailed analysis of their root causes. This deficiency in detailed information restricts our ability to map wallet components to specific vulnerabilities.

% In addition, reports from exchanges and wallet providers used to conduct empirical analysis in \autoref{sec:empirical-analysis} often contain high uncertainty about the exact nature of attack vectors. Notably, about 30\% documented incidents fail to specify the attack methods employed. This lack of clarity leads to some hypotheses regarding the precise nature of attacks. 

% Furthermore, our research employs a systematic approach to categorise attack occurrences related to specific wallet types, documenting only those attacks that are explicitly mentioned by studies for each wallet type as shown in \autoref{attack-vectors}. There may be possible attacks not identified by thew study. 

One limitation of our study is the lack of quality data on wallet attacks, we observe that many recorded incidents from exchanges and non-custodial wallet providers show a high degree of uncertainty (see \autoref{tab:attack-incidents}) in the reporting of attack vectors. This ambiguity makes it difficult to conduct a quantitative attack analysis. In addition, our study encompasses a wide spectrum of attacks documented both in academic literature and observed in industry practice, however, we do explore these attacks in exhaustive detail. Despite these limitations, our findings provide valuable insights into the design, vulnerabilities, attack vectors and defence implementations associated with different wallet types. 



% This evaluation does not serve as a method to aggregate individual factors to evaluate the overall privacy of a wallet.

% Hardware wallets, for instance, deserve a highly specific of attacks and defence implementations. 

% For instance, the atomic wallet hack of USD 100 million was said to be a result of one of four probable causes which include a \acf{mitm} attack, a malware code injection or an infrastructure breach \cite{cointele_atomic}. 

\subsection{Future Work}

Given the number of hardware-specific wallet attacks and defence implementations, we believe a systematisation of hardware wallet attacks would be an interesting area for future research. Furthermore, an evaluation specifically on various key recovery mechanisms and security across different wallet types can be conducted in the future. 


% Several areas warrant further investigation to improve the security frameworks of wallets. Firstly, addressing the existing gap in research concerning vulnerabilities specific to iOS-based wallet applications is important. With much of the existing literature skewed towards Android-based applications, a detailed security analysis of iOS wallets is crucial. Such research would help to identify and mitigate potential security risks and weaknesses inherent to the iOS ecosystem.


% \subsubsection{Key Encryption Key combined with Secret Key Distribution}
% \label{sec:hybrid-multi-tier-wallets}

% Given the spectrum of accessibility and security, a range of interesting design choices can be made. An interesting solution could potentially combine MFKDF with threshold \acs{mpc} or contract-based multi-sigs to require more than one password-derived   (see )





