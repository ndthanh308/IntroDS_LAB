\section{Introduction}
\label{sec:introduction}


Pioneered by Bitcoin \cite{NakamotoBitcoin:System}, peer-to-peer transactions have evolved into a digital ecosystem of decentralised financial applications on the blockchain. By building on this with self-executing smart contracts on blockchain networks such as Ethereum, \ac{defi} protocols allow decentralised lending, exchanges, derivatives and a growing number of financial applications. The digital authorisation of these transactions is intricately facilitated by a wallet.

A wallet is a transaction-facilitating tool that manages user authentication to enable the digital signing of transactions and broadcasts these messages to a blockchain network to confirm their validity. When initiating a transaction, wallets use a private key to sign and broadcast the signature to the blockchain network \cite{khan2022gas}. Therefore, private key security is critical and cannot be overstated, as incidents such as the Mt. Gox exchange attack (850,000 BTC) have resulted in significant financial losses, affecting individual users and various entities relying on the service \cite{mtgox_hack}. Other attack incidents on KuCoin, Vulcan Forged, Infarno and WazirX have demonstrated the attractiveness of both custodial and non-custodial wallets \cite{TheChainalysis, VulcanHack, Explained:2024, RektREKT}.


% -- NEED TO FIND STATEMENT OF TOTAL WALLET LOSSES SINCE 2011 OR TOTAL CRYPTO LOSSES SINCE --
% -- PREVIOUS STATEMENT --
% The importance of security cannot be overstated, as reported losses in the crypto ecosystem have exceeded a staggering \$16.7 billion since 2011 \cite{CrystalBlockchain}. 

This paper assesses the security of cryptocurrency wallets by analysing their design, associated threats, attacks, and possible defences. We introduce a design framework applicable to all traditional and modern wallets (\autoref{sec:wallet-taxonomy}). Following this, we systematise threats (\autoref{sec:threat_framework}) and attacks (\autoref{sec:attack-framework}), which enables us to suggest potential defence strategies (\autoref{sec:defense-strategies}). We then discuss our analysis of design elements (\autoref{sec:tax_discussion}), attack vectors (\autoref{sec:attacks_discussion}), and defence types (\autoref{sec:def_discussion}). In summary, our contributions are as follows: 
\begin{itemize}
    \item \textbf{Taxonomy of Wallet Design Framework:} We provide a framework to analyse the design of various existing wallet types and propose new wallet designs. We also outline the threats to existing wallet designs based on our threat model.
    \item \textbf{Wallet Attacks Framework:} We systematise and analyse various attacks' methods, techniques and targets in literature. We then analyse 84 notable wallet incidents between 2012 and 2024 and investigate the attack gaps between academia and industry.
    \item \textbf{Defence Strategies:} We suggest defence methods based on the overall mitigation approach, incorporating both proactive and reactive approaches. We also analyse the influence of defence methods in mitigating attacks.
\end{itemize}


% such as authentication bypass, intrusion, alteration, extraction and disruption prevention

