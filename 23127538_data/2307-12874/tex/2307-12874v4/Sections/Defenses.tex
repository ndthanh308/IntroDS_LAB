\section{Defence Methods}
\label{sec:defense-strategies}

This section builds upon the framework outlined in \autoref{sec:attack-framework} by presenting mitigation approaches against wallet attacks. We aim to examine defence mechanisms for each identified attack vector affecting wallets.


% We conduct forward reference searches on key articles to track subsequent papers that cite initial studies and utilise Google Scholar's advanced search to propose mitigation. 


\subsection{Defence against Network Attacks}
\label{sec:net-def}

Suspicious network activity can be detected through machine learning techniques, including anomaly detection models \cite{kapoor2021ransomware} and classification algorithms \cite{balakrishnan2023analysis}. Additionally, dynamic network parameter adjustments \cite{Girdler2021ImplementingAddresses} and the implementation of other intrusion detection mechanisms \cite{guri2018beatcoin, zimba2019cryptojacking} further contribute to identifying such anomalies. To mitigate these attacks, wallets can adopt network security protocols that validate and authenticate IP addresses \cite{rengarajan2016secure}, and incorporate additional security layers within the wallet's network to prevent potential \teal{$txn$} modification attempts by adversaries \cite{Cai2014ResearchNetwork}.


In limiting or preventing \acf{ddos} attacks, malicious and authentic network traffic needs to be distinguished by using classifiers such as the decision tree algorithm \cite{khan2019adaptive} and reinforcement learning approaches to analyse patterns in network data \cite{liu2018deep}. Another mitigation approach is analysing the network for unusual patterns, such as repeated request attempts from the same \acs{ip} address \cite{sathwara2017distributed}.

% % Enhanced defence mechanisms against network attacks in cryptocurrency wallets include multifaceted approaches. The integration of hot and cold wallets, as introduced by Jasem et al. \cite{jasem2021enhancement}, addresses deanonymization risks by expanding key space and generating unique keys for each transaction. Security in RPC communications is fortified by recommendations from Bui et al. \cite{bui2019pitfalls}, involving TLS usage and architecture shifts towards inter-process communication to mitigate impersonation threats. Privacy-enhancing network policies like Dandelion, proposed by Venkatakrishnan et al. \cite{bojja2017dandelion}, obscure user identities in the Bitcoin network by mixing messages, while wallet applications like Darkwallet, Wasabi Wallet, and Samourai Wallet implement stealth addresses, blind signatures, and dummy addresses for increased transaction anonymity (Bergman et al. \cite{bergman2021revealing}). These innovations collectively strengthen wallet security and user privacy.

% Suspicious network activity can be identified using machine learning algorithms \cite{ahmed2017mitigating}, dynamically adjusting network parameters \cite{girdler2021implementing} and employing other intrusion detection methods \cite{zimba2019cryptojacking}. To mitigate these attacks, wallets can implement network security protocols that validate and authenticate IP \cite{rengarajan2016secure}, and also add a layer of security implemented in the wallet's network to hinder the adversary's \teal{$txn$} modification attempt \cite{cai2014research}.

% % To mitigate this threat where unauthorised WiFi hotspots intercept and potentially alter transaction details (\autoref{sec:rogue-ap}), an extra layer of security can be implemented within the wallet network, hindering the adversary's \teal{$txn$} modification attempt \cite{cai2014research}.

% % Cai et al \cite{cai2014research} propose a two-factor-based dynamic password technology designed to enhance network security against rogue AP attacks.

% % In response to web-based infections (\autoref{sec:web-infection}), wallet applications should incorporate \acf{ids} that can identify and alert suspicious activities \cite{zimba2019cryptojacking}. These systems should be configured to monitor for unusual network traffic and script executions that may indicate a web-based infection attempt.

% % To prevent a MAC-\acs{ip} address linkage by an adversary through \acs{arp} spoofing, the wallet network's operating parameters can be dynamically adjusted to detect malicious network traffic \cite{girdler2021implementing}.

% % The redirection of the wallet devices to fraudulent websites through the compromise of the \acs{dns} revolver \autoref{sec:dns-spoofing} can be counteracted by employing machine learning algorithms to identify suspicious network activities \cite{ahmed2017mitigating}.

% % To counteract this attack, wallets can implement network security protocols that validate and authenticate IP to block unauthorised access attempts \cite{rengarajan2016secure}. This \acs{ip} authentication mitigation ensures that only traffic from verified and trustworthy IP addresses can interact with the wallet.

% This attack can be mitigated by distinguishing between malicious and authentic network traffic. Two notable traffic distinguishing techniques are using classifiers such as the decision tree algorithm \cite{khan2019adaptive} and using reinforcement learning approaches to analyse patterns in network data \cite{liu2018deep}. Another mitigation approach is analysing the network for unusual patterns, such as repeated request attempts from the same \acs{ip} address \cite{sathwara2017distributed}.

% % \acs{dos} attacks delivered via botnet attacks can be detected by filtering the network traffic to identify potential bot activity and using classifiers such as the decision tree algorithm to distinguish between benign and malicious traffic \cite{khan2019adaptive}.

% % To prevent downtime and stop excessive traffic from an adversary executing this attack, wallets can use reinforcement learning approaches to analyse patterns in network data to stop malicious traffic while allowing authentic network traffic \cite{liu2018deep}.

% % An effective strategy for mitigating the \acs{tcp} \acs{syn} flooding attack which limits the wallet's operation (\autoref{sec:wallet_mechanism}) involves network analysis for unusual patterns, such as repeated request attempts from the same \acs{ip} address. Following this, the server responds with \acs{tcp}-RST packets to halt these attempts, to prevent network downtime \cite{sathwara2017distributed}.

\subsection{Defence against Application Attacks}
\label{sec:app-def}

To mitigate the risk of message alteration by clipboard hijackers, features such as NFC, and two-dimensional codes

\begin{table*}[!h]
\centering
\renewcommand{\arraystretch}{1.1}
\setlength{\tabcolsep}{1.5pt} % Adjust the column separation space here
\footnotesize % or \scriptsize, \tiny, etc.
\resizebox{1.0\textwidth}{!}{
\begin{tabular}{llcccccccccccccccccccccccccccccc}
\toprule
\vspace{1pt} 
& \multicolumn{31}{c}{\textbf{Possible Defence Methods}}
\vspace{1pt} 
\\
\multicolumn{2}{c}{\textbf{ Classification}} 
& \rotatebox[origin=c]{90}{\cite{Cai2014ResearchNetwork}} % Network Authentication Tool
& \rotatebox[origin=c]{90}{\cite{Ahmed2017MitigatingNetworking}} % Web App Firewalls
& \rotatebox[origin=c]{90}{\cite{Bhirud2011LightPrevention}} % Dynamic IP Verification
& \rotatebox[origin=c]{90}{\cite{liu2018deep}} % Agent-based Traffic Mitigation
& \rotatebox[origin=c]{90}{\cite{sathwara2017distributed}} % Reset TCP Connections
& \rotatebox[origin=c]{90}{\cite{li2020android}} % Alteration Prevention Features | Access Control Restrictions ***
& \rotatebox[origin=c]{90}{\cite{ferdous2023review}} % Anti-malware Software
& \rotatebox[origin=c]{90}{\cite{indusface}} % Code Obfuscation
& \rotatebox[origin=c]{90}{\cite{Tirronen2018StoppingData}} % Cryptographic Code Verification
& \rotatebox[origin=c]{90}{\cite{Aratani2015AuthenticationChannel}} % Multi-factor Authentication
& \rotatebox[origin=c]{90}{\cite{aldawood2020advanced}} % Advanced Password Selection | Custom Keyboard Functions
& \rotatebox[origin=c]{90}{\cite{galbally2013image}} % Liveness Assessment Features
& \rotatebox[origin=c]{90}{\cite{altuwaijri2020android}} % Supplementary Storage
& \rotatebox[origin=c]{90}{\cite{breier2022practical}} % Algorithmic Fault Detection
& \rotatebox[origin=c]{90}{\cite{Urien2021InnovativeWallets}} % PUF
& \rotatebox[origin=c]{90}{\cite{Gupta2019ImpactSecurity}} % Memory and Cache Data Split
& \rotatebox[origin=c]{90}{\cite{brengel2018identifying}} % Secure Cryptographic Schemes | Deterministic Nonce Selection
& \rotatebox[origin=c]{90}{\cite{Park2024CloningFunction}} % Correlation Elimination
& \rotatebox[origin=c]{90}{\cite{Akter2023AChallenges}} % Correlation Elimination
& \rotatebox[origin=c]{90}{\cite{Lindell2020SecureComputation}} % MPC
& \rotatebox[origin=c]{90}{\cite{bip11}} % Multi-sig
& \rotatebox[origin=c]{90}{\cite{Park2023}} % Correlation sounds
& \rotatebox[origin=c]{90}{\cite{Feng2023Man-in-the-middleRedirects}} % Mitm mitation
& \rotatebox[origin=c]{90}{\cite{Kim2022ACountermeasures}} % MPC
& \rotatebox[origin=c]{90}{\cite{Shuvo2023AAttacks}} % algorithmic fault detection
& \rotatebox[origin=c]{90}{\cite{zimba2019cryptojacking}} % Intrusion Detection
& \rotatebox[origin=c]{90}{\cite{qi2012spad}} % Runtime Protection
& \rotatebox[origin=c]{90}{\cite{ManageAddresses}} % manage destination address
& \rotatebox[origin=c]{90}{\cite{hu2020overview}} % Physical Unclonable Functions (PUFs)
& \# (\%)
\vspace{1.5pt} 
\\
\midrule
\multirow{3}{*}{Precautionary} &
\rotatebox[origin=c]{0}{Prevention} & {\smallemptycirc} & {\smallemptycirc} & {\smallemptycirc} & {\smallemptycirc} & {\smallemptycirc} & {\smallfullcirc} & {\smallemptycirc} & {\smallemptycirc} & {\smallfullcirc} & {\smallemptycirc} & {\smallemptycirc} & {\smallemptycirc} & {\smallemptycirc} & {\smallemptycirc} & {\smallemptycirc} & {\smallemptycirc} & {\smallfullcirc} & {\smallemptycirc} & {\smallemptycirc} & {\smallemptycirc} & {\smallemptycirc} & {\smallemptycirc} & {\smallemptycirc} & {\smallemptycirc} & {\smallemptycirc} & {\smallemptycirc} & {\smallemptycirc} & {\smallemptycirc} & {\smallemptycirc} & \cellcolor{g2}{$3$($10\%$)} \\
& \rotatebox[origin=c]{0}{Protection} & {\smallfullcirc} & {\smallfullcirc} & {\smallfullcirc} & {\smallemptycirc} & {\smallemptycirc} & {\smallfullcirc} & {\smallfullcirc} & {\smallfullcirc} & {\smallemptycirc} & {\smallfullcirc} & {\smallfullcirc} & {\smallfullcirc} & {\smallemptycirc} & {\smallemptycirc} & {\smallfullcirc} & {\smallfullcirc} & {\smallemptycirc} & {\smallfullcirc} & {\smallfullcirc} & {\smallemptycirc} & {\smallemptycirc} & {\smallfullcirc} & {\smallfullcirc} & {\smallemptycirc} & {\smallemptycirc} & {\smallemptycirc} & {\smallfullcirc} & {\smallemptycirc} & {\smallfullcirc} &  \cellcolor{g6}{$17$($58\%$)} \\
& \rotatebox[origin=c]{0}{Limitation} & {\smallemptycirc} & {\smallemptycirc} & {\smallemptycirc} & {\smallfullcirc} & {\smallemptycirc} & {\smallemptycirc} & {\smallemptycirc} & {\smallemptycirc} & {\smallemptycirc} & {\smallemptycirc} & {\smallemptycirc} & {\smallemptycirc} & {\smallfullcirc} & {\smallemptycirc} & {\smallemptycirc} & {\smallemptycirc} & {\smallemptycirc} & {\smallemptycirc} & {\smallemptycirc} & {\smallfullcirc} & {\smallfullcirc} & {\smallemptycirc} & {\smallemptycirc} & {\smallfullcirc} & {\smallemptycirc} & {\smallemptycirc} & {\smallemptycirc} & {\smallfullcirc} & {\smallemptycirc} & \cellcolor{g3}{$6$($21\%$)} \\
\midrule
\multirow{3}{*}{Remedial} & \rotatebox[origin=c]{0}{Detection} & {\smallemptycirc} & {\smallemptycirc} & {\smallemptycirc} & {\smallfullcirc} & {\smallemptycirc} & {\smallemptycirc} & {\smallemptycirc} & {\smallemptycirc} & {\smallemptycirc} & {\smallemptycirc} & {\smallemptycirc} & {\smallemptycirc} & {\smallemptycirc} & {\smallfullcirc} & {\smallemptycirc} & {\smallemptycirc} & {\smallemptycirc} & {\smallemptycirc} & {\smallemptycirc} & {\smallemptycirc} & {\smallemptycirc} & {\smallemptycirc} & {\smallfullcirc} & {\smallemptycirc} & {\smallfullcirc} & {\smallfullcirc} & {\smallemptycirc} & {\smallemptycirc} & {\smallemptycirc} & \cellcolor{g3}{$5$($17\%$)} \\
& \rotatebox[origin=c]{0}{Response} & {\smallemptycirc} & {\smallemptycirc} & {\smallemptycirc} & {\smallfullcirc} & {\smallemptycirc} & {\smallemptycirc} & {\smallemptycirc} & {\smallemptycirc} & {\smallemptycirc} & {\smallemptycirc} & {\smallemptycirc} & {\smallemptycirc} & {\smallemptycirc} & {\smallemptycirc} & {\smallemptycirc} & {\smallemptycirc} & {\smallemptycirc} & {\smallemptycirc} & {\smallemptycirc} & {\smallemptycirc} & {\smallemptycirc} & {\smallemptycirc} & {\smallemptycirc} & {\smallemptycirc} & {\smallemptycirc} & {\smallemptycirc} & {\smallemptycirc} & {\smallemptycirc} & {\smallemptycirc} & \cellcolor{g1}{$1$($3\%$)} \\
& \rotatebox[origin=c]{0}{Recovery} & {\smallemptycirc} & {\smallemptycirc} & {\smallemptycirc} & {\smallemptycirc} & {\smallfullcirc} & {\smallemptycirc} & {\smallemptycirc} & {\smallemptycirc} & {\smallemptycirc} & {\smallemptycirc} & {\smallemptycirc} & {\smallemptycirc} & {\smallemptycirc} & {\smallemptycirc} & {\smallemptycirc} & {\smallemptycirc}  & {\smallemptycirc} & {\smallemptycirc} & {\smallemptycirc} & {\smallemptycirc} & {\smallemptycirc} & {\smallemptycirc} & {\smallemptycirc} & {\smallemptycirc} & {\smallemptycirc} & {\smallemptycirc} & {\smallemptycirc} & {\smallemptycirc} & {\smallemptycirc} & \cellcolor{g1}{$1$($3\%$)}
\vspace{1pt}
\\
\midrule 
\multicolumn{3}{c}{Summary}  &
\multicolumn{8}{c}{Precautionary:  \cellcolor{g6}{$26$($89\%$)}}  &
\multicolumn{8}{c}{Remedial: 
 \cellcolor{g3}{$7$($24\%$)}}  &
% \multicolumn{7}{c}{}  &
\multicolumn{11}{r}{Total Unique Methods    }  &
\multicolumn{1}{c}{}  &
\cellcolor{g0}{$29$($100\%$)} 
\vspace{1pt} 
   \\
\bottomrule
\end{tabular}
}
\vspace{1ex} % Add space before the caption
\caption{Defence methods categorised by type showing classification frequency (\#) and percentage (\%). Precautionary methods proactively prevent attacks; remedial methods provide attack detection, response, or data recovery.}
\label{tab:defence_methods}
\end{table*}




% need to add in brute force defence methods 
% to this table - i.e. table 5
% and also table 4
% \cite{Kiktenko2019DetectingWallets, volety2019cracking, Byun2024AAttacks}



  % {\smallfullcirc} &
  % {\smallemptycirc} &
can be employed to prevent modification of the \teal{$recipient\_address$} during transaction creation \cite{li2020android}. From a user perspective, Human-readable addresses such as \acs{ens} \cite{ENS2024EthereumService} aid in detecting address tampering, though they have certain security vulnerabilities \cite{Xia2022ChallengesENS}. System behaviour modifications can be prevented by addressing specific attack vectors. Attack vectors which attempt these by targeting vulnerabilities in the operating system can be mitigated by employing code obfuscation \cite{indusface} and runtime protection mechanisms \cite{qi2012spad}. Furthermore, by enforcing \acf{cfi} measures, wallets can ensure that the control flow hijacked to deviate from the intended control flow paths for malicious transactions cannot be executed \cite{Creech2017NewMitigation}. 


% Application-based intrusion software, such as malware, can be detected through effective analysis of system interactions, network communications, and behavioural patterns. Additionally, employing machine learning classifiers \cite{balakrishnan2023analysis}, anomaly detection models \cite{kapoor2021ransomware} and intrusion detection systems \cite{guri2018beatcoin} have emerged as effective detection methods. Following detection, the malware file can be tracked and neutralised.


% \subsubsection{Malware}
% \label{sec:def-malware}

% Detecting this attack effectively requires the analysis of malware behaviours, system interactions, and network communications.
% Employing machine learning classifiers \cite{balakrishnan2023analysis}, anomaly detection models \cite{kapoor2021ransomware} and intrusion detection systems \cite{guri2018beatcoin} have emerged as effective detection methods. Following detection, the malware file can be tracked and neutralised. 
% % Clipboard hijacking malware can be significantly limited by implementing human-readable address names.

% % \paragraph{Clipboard Hijacker}
% % \label{sec:def-clipboard}

% % Wallet can address this attack by employing novel features which prevent copying and pasting the \teal{$recipient\_addr$} during the transaction creation stage (\autoref{sec:transaction_signing}) \cite{li2020android}. These address replacement features include NFC, two-dimensional codes and human-readable address names. 

% % \paragraph{Keylogger}
% % \label{sec:def-keylogger}

% % Keyloggers, known for their hidden operations, present detection challenges. Despite this, employing advanced machine learning classifiers has emerged as a potent strategy for identifying and neutralising such threats\cite{balakrishnan2023analysis}.

% % \paragraph{Ransomware}
% % \label{sec:def-ransomware}

% % Detecting ransomware effectively requires the analysis of malware behaviours, system interactions, and network communications. Advanced machine learning techniques, such as anomaly detection models and supervised learning algorithms play a critical role in identifying potential ransomware activities \cite{kapoor2021ransomware}. However, while detection is crucial, secure backup and restore implementations prevent attacks. 


% % This approach ensures that, in the event of encryption by ransomware, affected files can be promptly and effortlessly restored.

% % \paragraph{Spyware}
% % \label{sec:def-spyware}
% % Spyware is particularly hard to detect as its hidden operations present challenges. Despite this, employing machine learning classifiers has emerged as an effective detection \cite{balakrishnan2023analysis}. Following detection, the spyware file can be tracked and neutralised.

% % \paragraph{Supply Chain Attack}
% % \label{sec:def-supply}

% % To enhance resilience against these attacks, emphasising early detection and response is crucial. This involves monitoring network data flows to identify unauthorised data access or unusual communication patterns that deviate from established norms \cite{wang2021feasibility}. By comparing current traffic behaviours against a pre-determined baseline of typical network activity, any unusual data movements or suspected malicious interactions can be identified.

% % -- extended explanation 
% % This method relies on the consistent aspects of traffic data, allowing for the detection of anomalies, even in communications that are encrypted, to track and uncover the mechanisms of such attacks effectively.

% % \paragraph{Phishing}
% % \label{sec:def-phishing}


% % To enhance security against phishing, wallets should incorporate phishing-resistant multi-factor authentication (MFA) techniques such as FIDO2 \cite{wang2021feasibility}. This feature prevents phishing by communicating with the original wallet website to verify the authenticity of the illegitimate one before allowing access to the wallet \cite{fido2}. Therefore, in scenarios where users are attack victims, the adversary will not bypass the other authentication methods. Additionally, users should verify the URLs of wallet browsers and exchange services before using them \cite{weichbroth2023security}.

% % \paragraph{Removable Media Infection}
% % \label{sec:def-removable}

% % Malware delivered to hardware wallets via removable media can be addressed by implementing intrusion detection and prevention systems \cite{guri2018beatcoin}. Furthermore, anti-virus programs can be installed on these wallet devices. 

% \subsubsection{Social Engineering}
% \label{sec:soc-engineering}
% \paragraph{Phishing}
% \label{sec:def-phishing}

% To enhance security against phishing, wallets should incorporate phishing-resistant multi-factor authentication (MFA) techniques such as FIDO2 \cite{wang2021feasibility}. This feature prevents phishing by communicating with the original wallet website to verify the authenticity of the illegitimate one before allowing access to the wallet \cite{fido2}. Additionally, users should verify the URLs of wallet browsers and exchange services before using them \cite{weichbroth2023security}.

% \subsubsection{Privilege Escalation}
% \label{sec:def-privilege}

% \paragraph{Android Root Exploitation}
% \label{sec:def-android-root}

% To safeguard mobile wallets from this attack, employing code obfuscation is essential. This technique scrambles the application's code without altering its functionality \cite{indusface}. 

% % Consequently, adversaries are unable to understand the application's inner workings, providing a robust defence against unauthorised access and the exploitation of sensitive data.

% \paragraph{Debugger}
% \label{sec:def-debugger}

% Implementing a monitoring feature to manage and restrict debugging settings mitigates this attack \cite{li2020android}. Additionally, employing runtime protection mechanisms helps detect and respond to unauthorised debugging attempts \cite{qi2012spad}.

% \paragraph{Logic Flow Exploitation}
% \label{sec:def-logic-flow}

% By enforcing \acf{cfi} measures, wallets can ensure that malicious transactions cannot be executed or control flow hijacked to deviate from the intended control flow paths \cite{creech2017new}. 

% % CFI works by inserting checks before indirect control flow transfers (like indirect function calls and returns) during program execution. These checks validate that the destination of the control flow transfer is within the set of allowed targets as defined by the CFG. If an attempted control flow transfer does not correspond to an allowed path in the CFG, it is considered an attack attempt, and the execution can be halted or redirected to a safe state.


\subsection{Defence against Authentication Attacks}
\label{sec:auth-def}

Wallets can either incorporate features as direct protection against specific attack methods or incorporate general authentication bypass features. By directly integrating improved functionalities to obstruct access to predictive text data, wallets can prevent the dictionary attack \cite{Uddin2021Horus:Wallets}. Additionally, to prevent brute force attacks, only complex passwords should be allowed in the initialisation stage  \cite{praitheeshan2019security}. Biometric falsifying attacks can be prevented by incorporating liveness detection features in wallets \cite{galbally2013image}.

To prevent single points of failure, wallets can enhance authentication levels (\autoref{sec:design-authen}) through \acf{mfa}, \acf{mpc} \cite{Lindell2020SecureComputation} and multi-signatory features such as BIP-11's M-of-N standard  \cite{bip11} (\autoref{sec:design-distr}). To mitigate social engineering attacks, for example, wallets can incorporate phishing-resistant multi-factor authentication (MFA) techniques such as FIDO2 \cite{Wang2021OnAttacks}. This feature enables communication with the original wallet website to verify the authenticity of the illegitimate one before allowing access to the wallet \cite{fido2}. 

\subsection{Defence against Storage and Memory Attacks}
\label{sec:sto-def}

An effective defence method against these attacks involves incorporating \acf{puf} to generate cryptographic keys on-demand, without storing \teal{$sk$} on the wallet's chip. This method also prevents microscopy attacks, some other physical tampering attacks and side-channel attacks (see \autoref{sec:crypt-def}) \cite{Urien2021InnovativeWallets, Park2024CloningFunction}. Physical tampering through the evil maid attack can be limited by implementing trusted boot mechanisms \cite{Tereshkin2010EvilEncryption}. Possible mitigations against non-invasive manipulation such as the cold boot attack involve adopting features which algorithmically clear the wallet's memory following intrusion \cite{seol2019amnesiac}. For example, Ledger has introduced a secure layer which detects chip intrusion and erases \teal{$sk$} following extraction attempts \cite{ledgerwallet}.

% % Wallets can implement trusted boot mechanisms to counteract this attack. Trusted boot rigorously verifies the integrity of software components during the boot process, allowing only those validated by their cryptographic signatures to execute \cite{tereshkin2010evil}. This effectively shields the device from unauthorised changes that could otherwise exploit the system to retrieve credentials. 

% To shield the device from unauthorised changes, wallets could implement trusted boot mechanisms, which verify the integrity of components during the boot process, allowing only those validated by their cryptographic signatures to execute \cite{tereshkin2010evil}.

\subsection{Defence against Cryptanalysis Attacks}
\label{sec:crypt-def}

The exploitation of cryptographic vulnerabilities can lead to \teal{$sk$} extraction. Attacks that aim to exploit weak cryptographic signatures (\teal{$\sigma$}), for instance, can be counteracted by employing stronger hashing algorithms \cite{Rokhjavan2023SecuringWallets}, while deterministic \teal{$nonce$} selection prevents nonce reuse attacks \cite{brengel2018identifying}. Non-invasive attacks on cryptographic functions including timing and power \acs{sca} are executed by exploiting side channels. Effective prevention methods include data leakage protection and data access patterns disguised as noise injection \cite{Akter2023AChallenges, Lou2021ACryptography, Ali2023CharacterizationHardware, Park2024CloningFunction}. These disrupt the adversary's ability to interpret leaked information effectively \cite{Mosquera2023GuardAttacks}. 

% Another method of disrupting the adversary for non-invasive \acs{sca} power attacks is by introducing sounds to disrupt the correlation between power consumption and extractable data \cite{Park2024CloningFunction}.


% \subsection{Intrusion Attacks}
% \label{sec:int-def}

% These defence methods are aimed at detecting network, application or physical intrusion attempts by the adversary. 

% Network Intrusion

% Suspicious network activity can be identified using machine learning algorithms \cite{Ahmed2017MitigatingNetworking}, dynamically adjusting network parameters \cite{Girdler2021ImplementingAddresses} and employing other intrusion detection methods \cite{zimba2019cryptojacking}. To mitigate these attacks, wallets can implement network security protocols that validate and authenticate IP \cite{rengarajan2016secure}, and also add a layer of security implemented in the wallet's network to hinder the adversary's \teal{$txn$} modification attempt \cite{Cai2014ResearchNetwork}. Additionally, employing machine learning classifiers \cite{balakrishnan2023analysis}, anomaly detection models \cite{kapoor2021ransomware} and intrusion detection systems \cite{guri2018beatcoin} have emerged as effective detection methods.

% girdler2021implementing
% cai2014research

% Application Intrusion

% Application-based intrusion software, such as malware, can be detected through effective analysis of system interactions, network communications, and behavioural patterns. Additionally, employing machine learning classifiers \cite{balakrishnan2023analysis}, anomaly detection models \cite{kapoor2021ransomware} and intrusion detection systems \cite{guri2018beatcoin} have emerged as effective detection methods. Following detection, the malware file can be tracked and neutralised. Additionally, physical tampering attacks such as the row hammer attack can be detected and mitigated using machine learning techniques \cite{joardar2022learning}.


% Physical Intrusion



% \subsection{Alteration Defence}
% \label{sec:alt-def}

% These defence methods cover a wide range of attacks aimed at altering the transaction message or system behaviour. To prevent message alteration by clipboard hijacker, features such as NFC, and two-dimensional codes which prevent \teal{$recipient\_addr$} change during transaction creation can be introduced \cite{li2020android}. Additionally, the use of human-readable addresses enables the user to notice address modifications. 

% Modifications to the system behaviour can be prevented by addressing specific attack vectors. Attack vectors which attempt these modifications by targeting vulnerabilities in the operating system can be prevented by employing code obfuscation \cite{indusface} and runtime protection mechanisms \cite{qi2012spad}. Furthermore, by enforcing \acf{cfi} measures, wallets can ensure that the control flow hijacked to deviate from the intended control flow paths for malicious transactions cannot be executed \cite{Creech2017NewMitigation}. 


% is essential. This technique scrambles the application's code without altering its functionality \cite{indusface}. 

% Additionally, employing runtime protection mechanisms helps detect and respond to unauthorised debugging attempts \cite{qi2012spad}.





% \subsubsection{Access Control Restrictions}
% \label{sec:def-mitm}

% \subsection{Authentication Bypass Defence}
% \label{sec:auth-def}

% These defence methods aim to prevent adversaries from decrypting private keys or bypassing authentication mechanisms by eliminating single points of failure and mitigating potential threats. 


% Wallets can either incorporate features as direct protection against specific attack methods or incorporate general authentication bypass features. By directly integrating improved functionalities to obstruct access to predictive text data, wallets can prevent the dictionary attack \cite{Uddin2021Horus:Wallets}. Additionally, to prevent brute force attacks, only complex passwords should be allowed in the initialisation stage  \cite{praitheeshan2019security}. Biometric falsifying attacks can be prevented by incorporating liveness detection features in wallets \cite{galbally2013image}.

% To prevent single points of failure, wallets can enhance authentication levels (\autoref{sec:design-authen}) through \acf{mfa}, \acf{mpc} \cite{Lindell2020SecureComputation} and multi-signatory features such as BIP-11's M-of-N standard  \cite{bip11} (\autoref{sec:design-distr}). To mitigate social engineering attacks, for example, wallets can incorporate phishing-resistant multi-factor authentication (MFA) techniques such as FIDO2 \cite{Wang2021OnAttacks}. This feature enables communication with the original wallet website to verify the authenticity of the illegitimate one before allowing access to the wallet \cite{fido2}. 

% To shield the device from unauthorised changes, wallets could implement trusted boot mechanisms, which verify the integrity of components during the boot process, allowing only those validated by their cryptographic signatures to execute \cite{tereshkin2010evil}.


% \subsection{Extraction Defence}
% \label{sec:ext-def}

% These defence methods are aimed at preventing invasive and non-invasive forms of private key extraction by the adversary, as these attacks target vulnerabilities in the wallet mechanism.

% mosquera2023guard
% park2024cloning

% \subsection{Disruption Defence}
% \label{sec:dis-def}

% \subsubsection{The Influence of Design on Defence}

% \paragraph{Hierarchical Deterministic Wallets}

% \paragraph{Distributed Wallets}

% \paragraph{Contract Validation Vulnerabilities}

% \paragraph{2FA \& Trusted Entities}

\subsection{Discussion}
\label{sec:def_discussion}

\subsubsection{Insight 1: Mitigations Against Multiple Attack Vectors}
\label{sec:def_dis_attacks}

We observe that design plays a critical role in enhancing defence mechanisms. For example, distributed architectures, such as \acs{mpc} and multi-signature functionalities in smart contract wallets, and multi-factor authentication, limit or protect against several attack vectors. On the other hand, the majority of defence implementations are particularly tailored to specific advanced attacks such as \acs{puf} for microscopic attacks, correlation elimination sounds for non-invasive side channels, and \acs{puf} attacks. These demonstrate the variety of defence strategies. 

\subsubsection{Insight 2: Comparison of Precautionary and Remedial Defence Methods}
\label{sec:def_dis_attacks}

Our study presents defence methods applicable to various attack vectors, with the majority offering either precautionary or remedial strategies, as illustrated in Table~\ref{tab:defence_methods}. Notably, precautionary defences significantly outnumber remedial approaches, comprising roughly 89\% of all methods observed. Within the precautionary category, protection-focused implementations are the most prevalent, accounting for 58\%. Among remedial defences, detection methods are the most common at 17\%, while response and recovery measures each represent a mere 3\%. This disparity highlights a critical gap in reactive mitigation techniques, indicating a potential area for further development in response and recovery-focused defences.


% While we provide defence methods for all attack vectors in our study, the majority of defence implementations offer either precautionary or remedial strategies as shown in \autoref{tab:defence_methods}, except for implementations which offer both \cite{liu2018deep}. We also observe that most of the defence implementations focus on precautionary rather than remedial techniques, with the former accounting for approximately 89\% of the total. Additionally, in analysing the sub-classes, protection-based implementations have the highest occurrence at 58 \%  Redemial sub-classes, detection accounts for 17\%, while response and recover account for 3\% each. The low percentage of response and recovery implementations demonstrates a lack of reactive mitigation methods.  


\subsubsection{Insight 3: Vulnerabilities in Defence Methods}
\label{sec:def_dis_def_vuln}

An interesting observation is the occurrence of targeted attacks and vulnerabilities in defence implementations. For instance, \acs{puf} effectively mitigates against the microscopy attack and other invasive hardware-based attacks. However, specific attack vectors in the literature exist against this protection mechanism. Furthermore, several vulnerabilities which enable \teal{$sk$} derivation from a single shard exist in \acs{mpc} wallets \cite{cve_12118}.

% just uncommented out  ****
% \subsubsection{Challenge 1: Inadequate Detection Mechanisms in Industry}

% \subsubsection{Effectiveness of Defence Methods}

% \subsubsection{Gap Analysis}

% \subsection{Defence against Network Attacks}
% \label{sec:def-network}

% % Enhanced defence mechanisms against network attacks in cryptocurrency wallets include multifaceted approaches. The integration of hot and cold wallets, as introduced by Jasem et al. \cite{jasem2021enhancement}, addresses deanonymization risks by expanding key space and generating unique keys for each transaction. Security in RPC communications is fortified by recommendations from Bui et al. \cite{bui2019pitfalls}, involving TLS usage and architecture shifts towards inter-process communication to mitigate impersonation threats. Privacy-enhancing network policies like Dandelion, proposed by Venkatakrishnan et al. \cite{bojja2017dandelion}, obscure user identities in the Bitcoin network by mixing messages, while wallet applications like Darkwallet, Wasabi Wallet, and Samourai Wallet implement stealth addresses, blind signatures, and dummy addresses for increased transaction anonymity (Bergman et al. \cite{bergman2021revealing}). These innovations collectively strengthen wallet security and user privacy.

% Suspicious network activity can be identified using machine learning algorithms \cite{ahmed2017mitigating}, dynamically adjusting network parameters \cite{girdler2021implementing} and employing other intrusion detection methods \cite{zimba2019cryptojacking}. To mitigate these attacks, wallets can implement network security protocols that validate and authenticate IP \cite{rengarajan2016secure}, and also add a layer of security implemented in the wallet's network to hinder the adversary's \teal{$txn$} modification attempt \cite{cai2014research}.

% % To mitigate this threat where unauthorised WiFi hotspots intercept and potentially alter transaction details (\autoref{sec:rogue-ap}), an extra layer of security can be implemented within the wallet network, hindering the adversary's \teal{$txn$} modification attempt \cite{cai2014research}.

% % Cai et al \cite{cai2014research} propose a two-factor-based dynamic password technology designed to enhance network security against rogue AP attacks.

% % In response to web-based infections (\autoref{sec:web-infection}), wallet applications should incorporate \acf{ids} that can identify and alert suspicious activities \cite{zimba2019cryptojacking}. These systems should be configured to monitor for unusual network traffic and script executions that may indicate a web-based infection attempt.

% % To prevent a MAC-\acs{ip} address linkage by an adversary through \acs{arp} spoofing, the wallet network's operating parameters can be dynamically adjusted to detect malicious network traffic \cite{girdler2021implementing}.

% % The redirection of the wallet devices to fraudulent websites through the compromise of the \acs{dns} revolver \autoref{sec:dns-spoofing} can be counteracted by employing machine learning algorithms to identify suspicious network activities \cite{ahmed2017mitigating}.

% % To counteract this attack, wallets can implement network security protocols that validate and authenticate IP to block unauthorised access attempts \cite{rengarajan2016secure}. This \acs{ip} authentication mitigation ensures that only traffic from verified and trustworthy IP addresses can interact with the wallet.

% This attack can be mitigated by distinguishing between malicious and authentic network traffic. Two notable traffic distinguishing techniques are using classifiers such as the decision tree algorithm \cite{khan2019adaptive} and using reinforcement learning approaches to analyse patterns in network data \cite{liu2018deep}. Another mitigation approach is analysing the network for unusual patterns, such as repeated request attempts from the same \acs{ip} address \cite{sathwara2017distributed}.

% % \acs{dos} attacks delivered via botnet attacks can be detected by filtering the network traffic to identify potential bot activity and using classifiers such as the decision tree algorithm to distinguish between benign and malicious traffic \cite{khan2019adaptive}.

% % To prevent downtime and stop excessive traffic from an adversary executing this attack, wallets can use reinforcement learning approaches to analyse patterns in network data to stop malicious traffic while allowing authentic network traffic \cite{liu2018deep}.

% % An effective strategy for mitigating the \acs{tcp} \acs{syn} flooding attack which limits the wallet's operation (\autoref{sec:wallet_mechanism}) involves network analysis for unusual patterns, such as repeated request attempts from the same \acs{ip} address. Following this, the server responds with \acs{tcp}-RST packets to halt these attempts, to prevent network downtime \cite{sathwara2017distributed}.


% \subsubsection{\acf{mitm}}
% \label{sec:def-mitm}

% Suspicious network activity can be identified using machine learning algorithms \cite{ahmed2017mitigating}, dynamically adjusting network parameters \cite{girdler2021implementing} and employing other intrusion detection methods \cite{zimba2019cryptojacking}. To mitigate these attacks, wallets can implement network security protocols that validate and authenticate IP \cite{rengarajan2016secure}, and also add a layer of security implemented in the wallet's network to hinder the adversary's \teal{$txn$} modification attempt \cite{cai2014research}.

% % \paragraph{Rogue \acf{ap}} 
% % \label{sec:def-rogue-ap}
% % To mitigate this threat where unauthorised WiFi hotspots intercept and potentially alter transaction details (\autoref{sec:rogue-ap}), an extra layer of security can be implemented within the wallet network, hindering the adversary's \teal{$txn$} modification attempt \cite{cai2014research}. 

% % Cai et al \cite{cai2014research} propose a two-factor-based dynamic password technology designed to enhance network security against rogue AP attacks.

% % \paragraph{Web-based Infection}
% % \label{sec:def-web-infection}

% % In response to web-based infections (\autoref{sec:web-infection}), wallet applications should incorporate \acf{ids} that can identify and alert suspicious activities \cite{zimba2019cryptojacking}. These systems should be configured to monitor for unusual network traffic and script executions that may indicate a web-based infection attempt.

% % \paragraph{\acs{arp} Spoofing}
% % \label{sec:def-arp-spoofing}

% % To prevent a MAC-\acs{ip} address linkage by an adversary through \acs{arp} spoofing, the wallet network's operating parameters can be dynamically adjusted to detect malicious network traffic \cite{girdler2021implementing}.

% % \paragraph{\acf{dns} Spoofing}
% % \label{sec:def-dns-spoofing}

% % The redirection of the wallet devices to fraudulent websites through the compromise of the \acs{dns} revolver \autoref{sec:dns-spoofing} can be counteracted by employing machine learning algorithms to identify suspicious network activities \cite{ahmed2017mitigating}.

% % \paragraph{\acf{ip} Spoofing}
% % \label{sec:def-ip-spoofing}

% % To counteract this attack, wallets can implement network security protocols that validate and authenticate IP to block unauthorised access attempts \cite{rengarajan2016secure}. This \acs{ip} authentication mitigation ensures that only traffic from verified and trustworthy IP addresses can interact with the wallet.

% \subsubsection{\acf{dos}}
% \label{sec:def-dos}

% This attack can be mitigated by distinguishing between malicious and authentic network traffic. Two notable traffic distinguishing techniques are using classifiers such as the decision tree algorithm \cite{khan2019adaptive} and using reinforcement learning approaches to analyse patterns in network data \cite{liu2018deep}. Another mitigation approach is analysing the network for unusual patterns, such as repeated request attempts from the same \acs{ip} address \cite{sathwara2017distributed}.

% % \acs{dos} attacks delivered via botnet attacks can be detected by filtering the network traffic to identify potential bot activity and using classifiers such as the decision tree algorithm to distinguish between benign and malicious traffic \cite{khan2019adaptive}.

% % To prevent downtime

% % \paragraph{\acs{icmp} Flooding}
% % \label{sec:def-icmp-flooding}

% % To prevent downtime and stop excessive traffic from an adversary executing this attack, wallets can use reinforcement learning approaches to analyse patterns in network data to stop malicious traffic while allowing authentic network traffic \cite{liu2018deep}.

% % \paragraph{\acs{tcp} \acs{syn} Flooding}
% % \label{sec:def-tcp-flooding}

% % An effective strategy for mitigating the \acs{tcp} \acs{syn} flooding attack which limits the wallet's operation (\autoref{sec:wallet_mechanism}) involves network analysis for unusual patterns, such as repeated request attempts from the same \acs{ip} address. Following this, the server responds with \acs{tcp}-RST packets to halt these attempts, to prevent network downtime \cite{sathwara2017distributed}.

% \subsection{Defence against Application Attacks}

% % Barber et al. \cite{barber2012bitter} address Bitcoin vulnerabilities by promoting threshold cryptography to split and secure private keys, and introducing super wallets for dividing assets across devices. Android's touch filter mechanism, though effective against clickjacking, is limited against scan-and-pay attacks, leading to suggestions by Ulqinaku et al. \cite{ulqinaku2019scan} for OS modifications for better recognition of malicious overlays. Rezaeighaleh et al. \cite{rezaeighaleh2019new} present a layered wallet model, enhancing security by separating the storage and transaction functionalities. Homoliak et al. \cite{homoliak2020security} propose a secure Ethereum transaction system involving authenticators and smart contracts. The Bitcoin Security Rectifier by Hu et al. \cite{hu2020securing} aims to improve BitcoinJ library security. Lastly, Takahashi et al. \cite{takahashi2019multiple} advocate for a detailed analysis of apps using a blend of static, dynamic, and semantic methods.

% \subsubsection{Malware}
% \label{sec:def-malware}

% Detecting this attack effectively requires the analysis of malware behaviours, system interactions, and network communications.
% Employing machine learning classifiers \cite{balakrishnan2023analysis}, anomaly detection models \cite{kapoor2021ransomware} and intrusion detection systems \cite{guri2018beatcoin} have emerged as effective detection methods. Following detection, the malware file can be tracked and neutralised. 
% % Clipboard hijacking malware can be significantly limited by implementing human-readable address names.

% % \paragraph{Clipboard Hijacker}
% % \label{sec:def-clipboard}

% % Wallet can address this attack by employing novel features which prevent copying and pasting the \teal{$recipient\_addr$} during the transaction creation stage (\autoref{sec:transaction_signing}) \cite{li2020android}. These address replacement features include NFC, two-dimensional codes and human-readable address names. 

% % \paragraph{Keylogger}
% % \label{sec:def-keylogger}

% % Keyloggers, known for their hidden operations, present detection challenges. Despite this, employing advanced machine learning classifiers has emerged as a potent strategy for identifying and neutralising such threats\cite{balakrishnan2023analysis}.

% % \paragraph{Ransomware}
% % \label{sec:def-ransomware}

% % Detecting ransomware effectively requires the analysis of malware behaviours, system interactions, and network communications. Advanced machine learning techniques, such as anomaly detection models and supervised learning algorithms play a critical role in identifying potential ransomware activities \cite{kapoor2021ransomware}. However, while detection is crucial, secure backup and restore implementations prevent attacks. 


% % This approach ensures that, in the event of encryption by ransomware, affected files can be promptly and effortlessly restored.

% % \paragraph{Spyware}
% % \label{sec:def-spyware}
% % Spyware is particularly hard to detect as its hidden operations present challenges. Despite this, employing machine learning classifiers has emerged as an effective detection \cite{balakrishnan2023analysis}. Following detection, the spyware file can be tracked and neutralised.

% % \paragraph{Supply Chain Attack}
% % \label{sec:def-supply}

% % To enhance resilience against these attacks, emphasising early detection and response is crucial. This involves monitoring network data flows to identify unauthorised data access or unusual communication patterns that deviate from established norms \cite{wang2021feasibility}. By comparing current traffic behaviours against a pre-determined baseline of typical network activity, any unusual data movements or suspected malicious interactions can be identified.

% % -- extended explanation 
% % This method relies on the consistent aspects of traffic data, allowing for the detection of anomalies, even in communications that are encrypted, to track and uncover the mechanisms of such attacks effectively.

% % \paragraph{Phishing}
% % \label{sec:def-phishing}


% % To enhance security against phishing, wallets should incorporate phishing-resistant multi-factor authentication (MFA) techniques such as FIDO2 \cite{wang2021feasibility}. This feature prevents phishing by communicating with the original wallet website to verify the authenticity of the illegitimate one before allowing access to the wallet \cite{fido2}. Therefore, in scenarios where users are attack victims, the adversary will not bypass the other authentication methods. Additionally, users should verify the URLs of wallet browsers and exchange services before using them \cite{weichbroth2023security}.

% % \paragraph{Removable Media Infection}
% % \label{sec:def-removable}

% % Malware delivered to hardware wallets via removable media can be addressed by implementing intrusion detection and prevention systems \cite{guri2018beatcoin}. Furthermore, anti-virus programs can be installed on these wallet devices. 

% \subsubsection{Social Engineering}
% \label{sec:soc-engineering}
% \paragraph{Phishing}
% \label{sec:def-phishing}

% To enhance security against phishing, wallets should incorporate phishing-resistant multi-factor authentication (MFA) techniques such as FIDO2 \cite{wang2021feasibility}. This feature prevents phishing by communicating with the original wallet website to verify the authenticity of the illegitimate one before allowing access to the wallet \cite{fido2}. Additionally, users should verify the URLs of wallet browsers and exchange services before using them \cite{weichbroth2023security}.

% \subsubsection{Privilege Escalation}
% \label{sec:def-privilege}

% \paragraph{Android Root Exploitation}
% \label{sec:def-android-root}

% To safeguard mobile wallets from this attack, employing code obfuscation is essential. This technique scrambles the application's code without altering its functionality \cite{indusface}. 

% % Consequently, adversaries are unable to understand the application's inner workings, providing a robust defence against unauthorised access and the exploitation of sensitive data.

% \paragraph{Debugger}
% \label{sec:def-debugger}

% Implementing a monitoring feature to manage and restrict debugging settings mitigates this attack \cite{li2020android}. Additionally, employing runtime protection mechanisms helps detect and respond to unauthorised debugging attempts \cite{qi2012spad}.

% \paragraph{Logic Flow Exploitation}
% \label{sec:def-logic-flow}

% By enforcing \acf{cfi} measures, wallets can ensure that malicious transactions cannot be executed or control flow hijacked to deviate from the intended control flow paths \cite{creech2017new}. 

% % CFI works by inserting checks before indirect control flow transfers (like indirect function calls and returns) during program execution. These checks validate that the destination of the control flow transfer is within the set of allowed targets as defined by the CFG. If an attempted control flow transfer does not correspond to an allowed path in the CFG, it is considered an attack attempt, and the execution can be halted or redirected to a safe state.

% \subsection{Defence against Authentication Attacks}

% % Authentication-based attacks can be generally counteracted by employing transaction management (see \autoref{sec:transaction_management}) security features which require more than one party to sign transactions and reduce a single point of failure. An example of such security features includes multi-signatory implementation and \acf{mpc} \cite{lindell2020secure}.

% % Multi-signature features reduce the single point of failure. Bitfinex, a cryptocurrency exchange has been vulnerable to a multi-signatory implementation whereby 2 out of the 3 keys required to sign a transaction were on one device \cite{protos}.

% % Khan et al. \cite{8966739} propose using complex passwords to increase cracking difficulty, limiting login attempts, and implementing Captchas to prevent automated access. Bulut et al. \cite{bulut2020security} emphasize the need for two-factor authentication and complex passwords against brute-force and dictionary attacks. Jasim et al. \cite{jasim2019enhancing} improve Brainwallet security by adding entropy to the master seed. Hu et al. \cite{hu2020securing} introduce a continuous verification method using mouse behavior biometrics. The BioWallet \cite{benli2017biowallet} utilizes a two-step authentication process, combining fingerprint verification with traditional password mechanisms, enhancing security and ensuring user identity validation for each transaction. Gentile et al. \cite{gentilal2017trustzone} advocate for a Trusted Execution Environment, segregating wallet layers to secure sensitive data and operations, thereby ensuring robust protection against unauthorised access and data manipulation.

% \subsubsection{Brute Force}
% \label{sec:def-brute-force}

% To reduce the possibility of an adversary bypassing the wallet authentication described in \autoref{sec:key-storage} using this attack vector, users should adopt more complex passwords \cite{praitheeshan2019attainable}.

% % -- Yathin's Old Version
% % These attacks on crypto-wallets involve systematically trying all possible combinations of characters or words to crack the passphrase or access the wallet \cite{rezaeighaleh2019new}. In this offline approach, the attacker uses a large dataset of words and generates a dictionary file for the brute force attack. The process is time-consuming and requires significant computational resources, often utilizing multiple machines or virtual machines operating in parallel \cite{volety2019cracking}. By attempting to crack the wallet using various combinations, the attacker identifies valid combinations that could potentially access the wallet. However, the success rate decreases as the number of combinations increases. The enormous number of possible combinations, especially when dealing with long passphrases or extensive wordlists, makes brute force attacks highly time-consuming and practically infeasible \cite{vasek2017bitcoin, praitheeshan2019attainable}.

% \subsubsection{Dictionary Attacks}
% \label{sec:def-dictionary}

% Similar to brute force, this attack can be neutralised by adopting advanced similar methods such as complex passphrases and mnemonic phrases, which significantly reduce the attacker's success chances \cite{aldawood2020advanced} or by integrating improved functionalities within devices to obstruct access to predictive text data \cite{uddin2021horus}.

% % These attack shares the same goal as brute force attacks and can be effectively neutralised by adopting advanced similar methods such as complex passphrases and mnemonic phrases. These measures significantly diminish the attack's success probability by complicating predictable patterns \cite{aldawood2020advanced}. Additionally, integrating custom keyboard functionalities within wallet devices obstructs the attackers' access to predictive text data, thereby, enhancing security against these attacks \cite{uddin2021horus}.

% % -- Yathin's Old Version
% % This exploits the user dictionary feature present in keyboard apps to predict the mnemonic phrase or passphrase used in the mobile wallet (see \autoref{sec:mobile-wallets}). Wallet apps typically rely on the default keyboard, which uses a user dictionary for predictive text inputs. The mnemonic phrase consists of common English words, and the information regarding these words is saved in the user dictionary. An attacker app with virtual keyboard permission can access the dictionary and extract frequency information of typed words to predict the mnemonic phrase. Only a small percentage of wallet apps (6\%) have implemented custom keyboards to protect against dictionary attacks \cite{uddin2021horus}. The effectiveness of dictionary attacks can be enhanced by building dictionaries based on leaked password datasets and analyzing the frequency of occurrence of passwords \cite{praitheeshan2019attainable}. Smarter guessing techniques can also increase the success rate \cite{holmes2023framework}.

% % \subsubsection{Code Reuse}
% % \label{sec:def-code-reuse}

% % An effective way of counteracting these attacks is by using \acf{cfi} which prevents the adversary from altering the execution path of a wallet and ensures only legitimate sequences of are executed \cite{burow2017control}. Additionally, formal verification can be used to check the correctness of control flows \cite{liu2019survey}. 

%  % An adversary can exploit existing code or logic constructs of a wallet to perform unauthorised actions or achieve malicious outcomes \cite{bletsch2011jump}. These attacks exploit the intended functionality of software components to alter execution flow and bypass security measures without necessarily injecting new code \cite{palladino2017parity}.

%  \subsubsection{Fake Biometrics}
% \label{sec:def-fake-biometrics}

% To prevent synthetic biometric data, biometric authentication security can be enhanced by incorporating liveness detection features \cite{galbally2013image}.


% % Wallet devices can enhance the security of biometric authentication by using liveness detection features. This feature enables a liveness assessment of the user to decrypt the \teal{$sk$} in the wallet storage mechanism to prevent synthetic or replicated biometric data \cite{galbally2013image}.


 
 
% \subsubsection{Evil Maid Attack}
% \label{sec:def-evil-maid}

% % Wallets can implement trusted boot mechanisms to counteract this attack. Trusted boot rigorously verifies the integrity of software components during the boot process, allowing only those validated by their cryptographic signatures to execute \cite{tereshkin2010evil}. This effectively shields the device from unauthorised changes that could otherwise exploit the system to retrieve credentials. 

% To shield the device from unauthorised changes, wallets could implement trusted boot mechanisms, which verify the integrity of components during the boot process, allowing only those validated by their cryptographic signatures to execute \cite{tereshkin2010evil}.

% % \subsubsection{Shoulder Surfing}
% % \label{sec:def-shoulder-surfing}

% % Shoulder surfing and similarly related observation attacks can be mitigated by employing \acf{mfa}. \acs{mfa} requires verification beyond what the adversary can see, such as one-time codes sent to the user's device, to significantly reduce the attack's effectiveness \cite{aratani2015authentication}.

% \subsection{Defence against Storage \& Memory Attacks}

% % \subsubsection{Fault Injection}
% % \label{sec:def-fault-inj}

% % A dual-layered defence strategy can used against fault injection attacks by combining formal verification and runtime protection mechanisms to provide prior and real-time mitigation \cite{hajdu2020using}. Before code deployment, formal verification proves the correctness of the wallet's code, as described in \autoref{sec:wallet_init}. Runtime protection mechanisms, on the other hand, monitor the system's operation in real time to detect and mitigate any attempted attacks or anomalies that occur post-deployment. 


% % additional information
% % This combination not only validates the integrity of the wallet's software from the outset but also offers continuous safeguarding against emerging threats, thereby maintaining the blockchain system's reliability and security against fault injection exploits [Á. Hajdu et al., 2020]

% % can also potentially add self-debugging 
% % found here --
% % Tightly-coupled Self-debugging Software Protection
% %  (Abrath et al., 2016)

% \subsubsection{Cold Boot Attack}
% \label{sec:def-cold-boot}

% Employing features which clear the wallet's memory following sensing physical tampering can prevent this attack \cite{seol2019amnesiac}. For example, Ledger has introduced a secure layer which detects chip intrusion and erases the private key following extraction attempts \cite{ledgerwallet}.

% \subsubsection{Row Hammer Attack}
% \label{sec:def-row-hammer}

% To prevent this, wallets can randomly exchange memory rows to protect the \acs{dram} against attacks \cite{saileshwar2022randomized}. Additionally, row swaps can be detected and mitigated using machine learning techniques \cite{joardar2022learning}.

% \subsubsection{Microscopy}
% \label{sec:def-microscopy}

% Incorporating \acf{puf} which generates cryptographic keys on-demand, without storing the \teal{$sk$} on the wallet's chip prevents this attack as well as other invasive key extraction attacks \cite{urien2021innovative}.

% % Given that microscopy attacks are a form of invasive attack, incorporating \acf{puf} is highly effective. \acs{puf} harness the inherent, unique properties of semiconductor devices to generate cryptographic keys on-demand, without ever storing these keys on the chip \cite{urien2021innovative}. This method significantly reduces the risk of key extraction via invasive techniques. Therefore, an adversary who aims to conduct a microscopy attack will find it impossible to retrieve the \teal{$sk$}.

% % \subsubsection{Probing}
% % \label{sec:def-probing}

% % Develop more resilient cryptographic devices \cite{wei2015vulnerability}


% \subsection{Defence against Cryptanalysis Attacks}


% \subsubsection{Weak Signature Exploit}
% \label{sec:def-weak-sig}

% % This attack targets vulnerabilities in the \hyperref[algo:key-generation]{key generation algorithm} or \hyperref[algo:transaction-signing]{signing algorithm} due to improper implementation \cite{rokhjavan2023securing}. The vulnerabilities may specifically be a product of weak or outdated cryptographic algorithms (detailed in \autoref{Encryption-Table-1}) or errors in encryption logic.

% Attacks that aim to exploit weak cryptographic signatures can be counteracted by employing stronger hashing algorithms \cite{rokhjavan2023securing}.

% \subsubsection{Nonce Reuse}
% \label{sec:def-nonse-reuse}

% Mitigation against nonce reuse attacks can be implemented by ensuring the deterministic selection of \teal{$nonce$} to prevent private key leakage \cite{brengel2018identifying}.

% \subsubsection{Side Channel Attacks}
% \label{sec:def-side-channel}

% Protection of data leakage points and disguising data access patterns prevent micro-architectural and timing-based side-channel attacks respectively \cite{lou2021survey, ali2023characterization}. These disrupt the attackers' ability to interpret leaked information effectively \cite{mosquera2023guard}.


% % The broad spectrum of information leakage, including power consumption patterns, timing discrepancies, and other observable data, presents significant obstacles to defending wallets \cite{lou2021survey, ali2023characterization}. For micro-architectural vulnerabilities, proactive identification and protection of potential leakage points are crucial \cite{lou2021survey}. In addition, mitigating data access attacks involves strategies to obscure the timing of operations, safeguarding against attacks that leverage time discrepancies \cite{ali2023characterization}. Disguising data access patterns also serves as a critical strategy in thwarting time-based side-channel attacks, thereby disrupting attackers' ability to interpret leaked information effectively \cite{mosquera2023guard}



% Other Features

% \subsubsection{Multi-Factor Key Derivation Function (MFKDF)}
% \label{sec:mfkdf}
% The MFKDF \cite{nair2023multi, nair2023decentralizing} is an advanced method for creating secure cryptographic keys, similar to complex passwords, using multiple types of security measures. This method is more secure and versatile compared to traditional approaches as it can integrate various authentication factors like one-time passwords, and tokens from devices like YubiKey, and others. A significant advantage of this method is that it moves away from the conventional use of mnemonics typically used in crypto wallets. By doing so, it greatly enhances the user experience, making it both simpler and more intuitive. Additionally, MFKDF empowers users with the ability to recover their accounts on their own, without needing a central master key. This self-service recovery is facilitated through a secret-sharing approach, where the key is divided into parts and only a certain number (K out of N) of these parts are needed to reconstruct the full key. This combination of ease of use, security, and user-friendly account recovery marks a significant improvement in managing access to crypto wallets.

% Ensuring the use of robust, up-to-date cryptographic algorithms and practices, proper key management, and thorough validation of signatures to resist exploitation

% it's critical to enhance both algorithmic integrity and implementation security. As outlined by Rokhjavan (2023), ensuring rigorous validation within key generation and signing algorithms, along with enforcing appropriate encryption key sizes, are fundamental steps towards securing signatures. 

% Rokhjavan (2023) advocates for rigorous validation mechanisms within key generation and signing processes, emphasizing the need for robust encryption key size checks to preserve signature integrity.

% Using more secure and tested cryptographic schemes


% Upholding cryptographic standards and best practices

% Cortez et al. \cite{cortez2020cryptanalysis} advocate for employing strong cryptographic algorithms with proven security properties. Prajapat et al. \cite{prajapat2016avk} highlight the importance of robust key management practices, with keys generated, stored, and used securely. Prajapat and Thakur \cite{prajapat2015various} suggest designing wallets to resist common cryptanalytic attacks by using sufficiently long and complex keys. Aciiçmez et al. \cite{aciiccmez2007micro} recommend employing techniques like threshold cryptography to split and secure private keys. Tomassini and Perrenoud \cite{tomassini2001cryptography} advise conducting regular security audits and cryptographic analysis of the wallet's implementation to identify and mitigate potential vulnerabilities.






% ---




% \subsection{Old Defence Features}
% \label{sec:wallet-features}

% % \subsection{Other Wallet Defence Features}
% % \label{sec:wallet-features}

% \subsubsection{M-of-N Standard Transactions}
% \label{sec:bip-11}
% Bitcoin Improvement Proposal 11 (BIP-11) \cite{bip11} introduces a method for Bitcoin transactions that requires multiple signatures before the transaction can be completed. Consider a collective decision-making process, wherein a specified subset of individuals (M) from a larger group (N) is required to reach a consensus before initiating any action. 

% % However, the initial key from which all others are derived (the root seed) must be very well protected, as its security is crucial for the entire wallet structure. This approach allows for a flexible and secure way of managing wallets, but users must be careful to safeguard their root seeds.

% % \subsubsection{Mnemonic Code for HD Wallets}
% % \label{sec:bip-39}
% % Bitcoin Improvement Proposal 39 (BIP-39) \cite{bip39} introduces a method for creating easy-to-remember recovery phrases for Bitcoin wallets. These phrases, made up of a series of simple words, are based on strong cryptographic principles. The words are chosen from a specific list to ensure security and ease of use. The idea is to make these phrases both random enough to be secure and simple enough for people to remember and write down. To add extra security, a technique called \quotes{salting} and a process called PBKDF2 (Password Based Key Derivation Function 2 is a cryptography function using hashing messages and values several times to produce keys.) are used, which make it harder for attackers to guess these phrases. The security of these phrases also depends on how carefully users handle them – if someone uses them carelessly, they can still be vulnerable to attacks. Overall, BIP-39 aims to provide a balance between memorability, ease of writing down, and randomness to create secure backup phrases for wallet users.

% % \subsubsection{Multi-Account Hierarchy}
% % \label{sec:bip-44}
% % Bitcoin Improvement Proposal 44 (BIP-44) \cite{bip44} builds on the ideas of BIP-32, which introduced a hierarchical structure for Bitcoin wallets. BIP-44 takes this a step further by organizing these wallets into separate accounts, ensuring that the information or assets in one account don't accidentally get mixed up with another. It uses special techniques like \quotes{account discovery} and \quotes{gap limits} to manage these accounts efficiently and securely. These features make it easier for users to have multiple wallets for different purposes, all under one main account, without compromising on security. However, just like in BIP-32, the main concern is still about protecting the initial key (root seed) from which all other wallets are derived. While BIP-44 is great for organizing multiple wallets and sharing them selectively, the overall security also depends on how the user manages their main key.

% \subsubsection{Multi-party computation (MPC)}
% \label{sec:mpc-wallets}
% MPC wallets \cite{canetti2020uc} represents a new and advanced way to make crypto wallets more secure, especially in how they handle private keys (the critical component for accessing and using cryptocurrencies). MPC works by using sophisticated cryptographic methods that allow several people or entities to work together on a calculation or process without having to share their individual pieces of information. When applied to crypto wallets, MPC splits the control of the private key among multiple parties. This means that no single person or entity holds the entire key, making it much harder for hackers to gain access to the wallet. The advantage of this approach is that it reduces the risk that comes from having all the security depend on just one point (like a single password or key), significantly enhancing overall security.

% \subsubsection{Account Abstraction Using Alt Mempool}
% \label{sec:erc-4337}
% Ethereum Request for Comments 4337 (ERC-4337) \cite{erc4337} is a proposal that aims to make Ethereum accounts more flexible and secure without needing to change the core rules of the Ethereum network. It introduces a new system that works on top of the existing network, using a two-step process (validate and execute) for transactions. This approach makes transactions more efficient and helps prevent certain types of attacks where transaction fees can be manipulated. ERC-4337 also includes measures to ensure that the execution of transactions is consistent with their validation, and it allows for more versatile ways of managing transaction orders than the traditional method. 

