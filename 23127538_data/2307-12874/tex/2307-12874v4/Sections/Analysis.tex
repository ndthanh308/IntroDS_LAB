\section{Analysis}
\label{sec:analysis}


% Figure environment removed

% This section introduces the analysis carried out on our dataset.

\subsection{Wallet Attack Frequency}
\label{sec:frquency}

\autoref{fig:incidents_freq} illustrates the correlation between the amount of wallet attack incidents and the aggregated funds lost monthly. According to our data. The total amount of funds recorded stolen by malicious actors equates to \$2.436 Billion USD, with an average of \$56 Million USD per attack. We also observe a sharp rise in November 2023 with over \$250 Million USD stolen across 6 wallets. A staggering amount of \$1.75 Billion was lost without knowledge of how malicious actors infiltrated the system as also shown in \autoref{tab:incidents3}, a more concise version of this incident table by wallet types is shown in \autoref{Security-Threats-Table-1}.

\subsection{Wallet Attack Categories}
\label{sec:types}
\autoref{fig:incidents_freq2} demonstrates the frequency of wallet attacks per wallet attack category in line with the system framework introduced in \autoref{sec:system_model}. 48\% of attacks in our dataset have insufficient information on the methods or layers in which the malicious actor utilised to steal credentials or retrieve assets. Furthermore, the major methods by which bad actors gain access to wallet credentials are application and storage. Out of the known attacks, these account for 59\% and 28\% respectively. Evidently, network attacks were not a common method as these are non-monetary in nature as discussed in \autoref{sec:threat_model}. 


\subsection{Incident Gap Analysis}
\label{sec:gap}

We also investigate the discrepancies between wallet attacks across academic research and industry incidents as shown in \autoref{fig:attack-classification}. While papers are well spread out across main attack categories and sub-sections, industry incidents do not occur in several sub-categories. However, some attacks such as faulty libraries, programming errors, malware, clipboard, side channel and \ac{dns} are not commonly investigated by academia. Notably, these attacks do not occur frequently in the industry as well. Social engineering and key storage are the most frequent means of attack for malicious actors in the industry. From our findings, academia and industry mirror each other relatively well, as they share the most and least investigated areas