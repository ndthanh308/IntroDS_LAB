\section{Attacks and Vulnerabilities}
\label{sec:security-analysis}

\autoref{Security-Threats-Table-1} provides a comprehensive overview of significant attacks on crypto-wallets and exchanges, resulting in substantial financial losses. Through an extensive literature review, we have systematically organized the identified attacks into four distinct categories. These categories, which are detailed in \autoref{attack-classification}, include Network (\autoref{sec:network}), Application (\autoref{sec:application}), Blockchain (\autoref{sec:blockchain}), Authentication (\autoref{sec:authentication}).

% Figure environment removed

\begin{table*}[!h]
\centering
\caption[Survey of Wallet Attacks]{Survey of wallet attacks and their causes}
\label{Security-Threats-Table-1}
\tiny
\renewcommand{\arraystretch}{1.1}
\resizebox{1\textwidth}{!}{
\begin{tabular}{l|lllllll}
\hline
Name & Occurrence & Funds & Date & Attack Type & Cause & Impact & Mitigation Strategies \\ \hline
Slope \cite{slope} & Solana Wallet & \$8,000,000 & 02-08-2022 & - & Seed phrase compromise & Loss of funds & Update software, secure seed phrase \\
Vulcan Forged \cite{vulcan} & Polygon Wallet & \$140,000,000 & 13-12-2021 & Server Exploitation & Private key compromise & Massive financial loss & Secure infrastructure, multi-factor auth \\
KuCoin \cite{kucoin} & Exchange Wallet & \$285,000,000 & 26-09-2020 & Advanced Persistent Threat & Private key compromise & Large-scale theft & Enhanced security, audits \\
Lympo \cite{lympo} & NFT Platform & \$18,700,000 & 10-01-2022 & - & - & Unknown & Investigate, strengthen security \\
Binance \cite{binance} & Exchange Wallet & \$41,000,000 & 07-05-2019 & Phishing attacks and viruses & API keys, 2FA codes hacked & Temporary disruption & Awareness, strong authentication \\
BitMart \cite{bitmart} & Exchange Wallet & \$196,000,000 & 05-12-2021 & - & Private Key Theft & Huge financial loss & Enhanced security, monitoring \\
Liquid \cite{liquid} & Exchange Wallet & \$90,000,000 & 19-08-2020 & - & Disputed & Financial loss & Investigate, improve security \\
Coincheck \cite{coincheck} & Exchange Wallet & \$534,000,000 & 26-01-2018 & Phishing attacks & Phishing attacks & Massive financial loss & Strengthen verification, security measures \\
Premint \cite{premint} & NFT Platform & \$375,000 & 17-07-2022 & Phishing attacks & Phishing attacks & Loss of funds & User education, improve security \\
Mt. Gox \cite{mtgox} & Exchange Wallet & \$460,000,000 & 19-06-2011 & Unauthorized Access & Unknown & Massive financial loss & Strengthen security, regular audits \\
Bitstamp \cite{bitstamp} & Exchange Wallet & \$5,000,000 & 04-01-2015 & Advanced Persistent Threat & Compromised Server & Temporary disruption & Regular updates, security assessments \\
GateHub \cite{gatehub} & Web Wallet & \$10,000,000 & 06-06-2019 & Phishing & Stolen Credentials & Financial loss & User education, strong authentication \\
Coincheck \cite{coincheck} & Exchange Wallet & \$534,000,000 & 26-01-2018 & Unauthorized Access & Vulnerability Exploitation & Massive financial loss & Regular assessments, security patches \\
MyEtherWallet \cite{myetherwallet} & Web Wallet & \$152,000 & 24-04-2017 & \ac{mitm} & Compromised \ac{dns} Server & Financial loss & Verify, secure connections \\
Bitfinex \cite{bitfinex} & Exchange Wallet & \$71,000,000 & 02-08-2016 & Denial of Service & D\ac{dos} Attack & Temporary disruption & DDoS protection, redundant infrastructure \\
Electrum \cite{electrum} & Desktop Wallet & \$22,000,000 & 27-12-2018 & Malicious Application & Fake Wallet Update & Financial loss & Trusted sources, verify checksums \\
Coinomi \cite{coinomi} & Desktop Wallet & \$70,000 & 27-02-2019 & \ac{mitm} & Compromised Wi-Fi Network & Financial loss & Secure Wi-Fi, enable encryption \\
Ledger \cite{ledger} & Hardware Wallet & \$1,500,000 & 25-06-2020 & Data Breach & Customer data leak & Compromised privacy & Strengthen security, enhanced encryption \\ \hline
\end{tabular}
}
\end{table*}
\begin{table*}[!ht]
	\centering
	\caption{Classification of wallet attacks and vulnerabilities using academic literature}
	\label{attack-classification}
	\Large
	\renewcommand{\arraystretch}{1.3}
	\resizebox{1\linewidth}{!}
	{
        \begin{tabular}{|l|ll|ccccc|cc|}
        \hline
        \multicolumn{1}{|c|}{\multirow{2}{*}{\textbf{Category}}} & \multicolumn{2}{c|}{\multirow{2}{*}{\textbf{Attacks}}}                              & \multicolumn{5}{c|}{\textbf{Types   of Wallets}}                                                                                                                                                                                                                                                                                                                                                                                                                                                                                                                                                                                                                                                                                                                                                                                                                                                                   & \multicolumn{2}{c|}{Gap Analysis}                              \\ \cline{4-10} 
\multicolumn{1}{|c|}{}                                   & \multicolumn{2}{c|}{}                                                               & \multicolumn{1}{c|}{Software}                                                                                                                                                                                                                                                                                                                                                                                                                                  & \multicolumn{1}{c|}{Hardware}                                                                                                                                               & \multicolumn{1}{c|}{Paper}                                                           & \multicolumn{1}{c|}{Brain}                                          & Smart Contract                                                                                         & \multicolumn{1}{l|}{Papers}   & \multicolumn{1}{l|}{Incidents} \\ \hline
\multirow{9}{*}{Network}                                 & \multicolumn{2}{l|}{Deanonymization}                                                & \multicolumn{1}{c|}{\cite{conti2018survey,   teomete2019anonymous, Biryukov2019SecurityZcash, androulaki2013evaluating,   khalilov2018survey, biryukov2014deanonymisation, neudecker2017could,   bergman2021revealing, aghaee2021identifying, bojja2017dandelion,   jasem2021enhancement, suratkar2020cryptocurrency}}                                                                                                                        & \multicolumn{1}{c|}{-}                                                                                                                                                      & \multicolumn{1}{c|}{-}                                                               & \multicolumn{1}{c|}{-}                                              & -                                                                                                      & \multicolumn{1}{c|}{\cellcolor[HTML]{26d701}11(12\%)} & -                              \\ \cline{2-10} 
                                                         & \multicolumn{2}{l|}{Man-in-The-Middle}                                              & \multicolumn{1}{c|}{\cite{garba2018analysis,   8613949, Biryukov2019SecurityZcash, rezaeighaleh2020multilayered}}                                                                                                                                                                                                                                                                                                                             & \multicolumn{1}{c|}{\cite{almutairi2019usability,   rezaeighaleh2019deterministic, rezaeighaleh2019new, 8500144, 9220841,   rezaeighaleh2020multilayered}} & \multicolumn{1}{c|}{-}                                                               & \multicolumn{1}{c|}{-}                                              & -                                                                                                      & \multicolumn{1}{c|}{\cellcolor[HTML]{4ded30}10(11\%)} & -                              \\ \cline{2-10} 
                                                         & \multicolumn{1}{l|}{\multirow{3}{*}{Domain   Name System}} & DNS Hijacking          & \multicolumn{2}{c|}{\cite{ahmadjee2022study,   sander2022quantitative, mandiant2022, cisa2019, talosintel2019, shaikhsurvey,   AGCH, KTSSH}}                                                                                                                                                                                                                                                                                                                                                                                                                                                                                & \multicolumn{1}{c|}{-}                                                               & \multicolumn{1}{c|}{-}                                              & -                                                                                                      & \multicolumn{1}{c|}{\cellcolor[HTML]{4ded30}8(9\%)}   & -                              \\ \cline{3-10} 
                                                         & \multicolumn{1}{l|}{}                                      & DNS Tunneling          & \multicolumn{2}{c|}{\cite{Banoth2023, ALIR}}                                                                                                                                                                                                                                                                                                                                                                                                                                                                                                                                                                                & \multicolumn{1}{c|}{-}                                                               & \multicolumn{1}{c|}{-}                                              & -                                                                                                      & \multicolumn{1}{c|}{\cellcolor[HTML]{B7FFBF}2(2\%)}   & -                              \\ \cline{3-10} 
                                                         & \multicolumn{1}{l|}{}                                      & DNS Cache Poisoning    & \multicolumn{2}{c|}{\cite{homoliak2019security,   son2010hitchhiker}}                                                                                                                                                                                                                                                                                                                                                                                                                                                                                                                                                       & \multicolumn{1}{c|}{-}                                                               & \multicolumn{1}{c|}{-}                                              & -                                                                                                      & \multicolumn{1}{c|}{\cellcolor[HTML]{B7FFBF}2(2\%)}   & -                              \\ \cline{2-10} 
                                                         & \multicolumn{1}{l|}{\multirow{4}{*}{Denial-of-Service}}    & Distributed DoS        & \multicolumn{2}{c|}{\multirow{3}{*}{\cite{AGGARWAL2021399,   wang2006effectiveness, feder2017impact, raikwar2021attacks,   begum2020blockchain, dasgupta2019survey, homoliak2019security,   homoliak2020security, MRRR2021, RMMDD, SGAP}}}                                                                                                                                                                                                                                                                                                                                                                                  & \multicolumn{1}{c|}{-}                                                               & \multicolumn{1}{c|}{-}                                              & -                                                                                                      & \multicolumn{1}{c|}{\cellcolor[HTML]{26d701}11(12\%)} & \cellcolor[HTML]{b96666}1(2\%)                         \\ \cline{3-3} \cline{6-10} 
                                                         & \multicolumn{1}{l|}{}                                      & DoS on Connectivity    & \multicolumn{2}{c|}{}                                                                                                                                                                                                                                                                                                                                                                                                                                                                                                                                                                                                                        & \multicolumn{1}{c|}{-}                                                               & \multicolumn{1}{c|}{-}                                              & -                                                                                                      & \multicolumn{1}{c|}{\cellcolor[HTML]{26d701}11(12\%)} & -                              \\ \cline{3-3} \cline{6-10} 
                                                         & \multicolumn{1}{l|}{}                                      & DoS on Local Resources & \multicolumn{2}{c|}{}                                                                                                                                                                                                                                                                                                                                                                                                                                                                                                                                                                                                                        & \multicolumn{1}{c|}{-}                                                               & \multicolumn{1}{c|}{-}                                              & -                                                                                                      & \multicolumn{1}{c|}{\cellcolor[HTML]{26d701}11(12\%)} & -                              \\ \cline{3-10} 
                                                         & \multicolumn{1}{l|}{}                                      & DoS on Smart Contract  & \multicolumn{1}{c|}{-}                                                                                                                                                                                                                                                                                                                                                                                                                                         & \multicolumn{1}{c|}{-}                                                                                                                                                      & \multicolumn{1}{c|}{-}                                                               & \multicolumn{1}{c|}{-}                                              & \cite{samreen2021smartscan,   krupp2018teether}                                       & \multicolumn{1}{c|}{\cellcolor[HTML]{B7FFBF}2(2\%)}   & -                              \\ \hline
\multirow{8}{*}{Application}                             & \multicolumn{2}{l|}{Social Engineering}                                             & \multicolumn{1}{c|}{\cite{astrakhantseva2021cryptocurrency,   krombholz2015advanced, binti2023blockchain, weichbroth2023security,   horch2022adversary, bhusal2021systematic, andryukhin2019phishing,   weber2020exploiting, parthy2019identification, yasin2019contemplating,   breda2017social}}                                                                                                                                            & \multicolumn{1}{c|}{\cite{krombholz2015advanced,   weichbroth2023security}}                                                                                & \multicolumn{1}{c|}{}                                                                & \multicolumn{1}{c|}{\cite{weichbroth2023security}} & \cite{ivanov2021targeting,   horch2022adversary}                                      & \multicolumn{1}{c|}{\cellcolor[HTML]{5dbb63}16(18\%)} & \cellcolor[HTML]{a23333}7(9\%)                         \\ \cline{2-10} 
                                                         & \multicolumn{2}{l|}{Rooting   and Debugging}                                        & \multicolumn{1}{c|}{\cite{zaghloul2020bitcoin,   volety2019cracking, uddin2021horus, Sai2019PrivacyApplications,   montanez2014investigation, 9143206, haigh2019if, li2020android}}                                                                                                                                                                                                                                                           & \multicolumn{1}{c|}{\cite{guri2018beatcoin}}                                                                                                               & \multicolumn{1}{c|}{-}                                                               & \multicolumn{1}{c|}{-}                                              & -                                                                                                      & \multicolumn{1}{c|}{\cellcolor[HTML]{4ded30}9(10\%)}  & -                              \\ \cline{2-10} 
                                                         & \multicolumn{2}{l|}{Faulty   Libraries}                                             & \multicolumn{1}{c|}{\cite{hu2021security,   tschannen2020evaluation, kaushal2017evolution}}                                                                                                                                                                                                                                                                                                                                                   & \multicolumn{1}{c|}{-}                                                                                                                                                      & \multicolumn{1}{c|}{-}                                                               & \multicolumn{1}{c|}{-}                                              & -                                                                                                      & \multicolumn{1}{c|}{\cellcolor[HTML]{B7FFBF}3(3\%)}   & -                              \\ \cline{2-10} 
                                                         & \multicolumn{1}{l|}{\multirow{3}{*}{Programming   Errors}} & Re-entrancy            & \multicolumn{1}{c|}{-}                                                                                                                                                                                                                                                                                                                                                                                                                                         & \multicolumn{1}{c|}{-}                                                                                                                                                      & \multicolumn{1}{c|}{-}                                                               & \multicolumn{1}{c|}{-}                                              & \cite{saad2019exploring,   liu2018reguard}                                            & \multicolumn{1}{c|}{\cellcolor[HTML]{B7FFBF}2(2\%)}   & -                              \\ \cline{3-10} 
                                                         & \multicolumn{1}{l|}{}                                      & Permissioning Error    & \multicolumn{1}{c|}{-}                                                                                                                                                                                                                                                                                                                                                                                                                                         & \multicolumn{1}{c|}{-}                                                                                                                                                      & \multicolumn{1}{c|}{-}                                                               & \multicolumn{1}{c|}{-}                                              & \cite{breidenbach2017depth,   kalra2018zeus, praitheeshan2019security}                & \multicolumn{1}{c|}{\cellcolor[HTML]{B7FFBF}3(3\%)}   & -                              \\ \cline{3-10} 
                                                         & \multicolumn{1}{l|}{}                                      & Mishandled Exception   & \multicolumn{1}{c|}{-}                                                                                                                                                                                                                                                                                                                                                                                                                                         & \multicolumn{1}{c|}{-}                                                                                                                                                      & \multicolumn{1}{c|}{-}                                                               & \multicolumn{1}{c|}{-}                                              & \cite{voskobojnikov2021u,   mahmood2022cybersecurity, guo2022survey, kustov2022three} & \multicolumn{1}{c|}{\cellcolor[HTML]{95f985}4(4\%)}   & -                              \\ \cline{2-10} 
                                                         & \multicolumn{1}{l|}{\multirow{2}{*}{Malware}}              & Ransomware             & \multicolumn{1}{c|}{\cite{koerhuis2020forensic,   weichbroth2023security}}                                                                                                                                                                                                                                                                                                                                                                    & \multicolumn{1}{c|}{-}                                                                                                                                                      & \multicolumn{1}{c|}{-}                                                               & \multicolumn{1}{c|}{-}                                              & -                                                                                                      & \multicolumn{1}{c|}{\cellcolor[HTML]{B7FFBF}2(2\%)}   & \cellcolor[HTML]{ae4d4d}5(5\%)                         \\ \cline{3-10} 
                                                         & \multicolumn{1}{l|}{}                                      & Trojan                 & \multicolumn{1}{c|}{\cite{haigh2019if,   binti2023blockchain, weichbroth2023security, krombholz2015advanced,   binti2023blockchain}}                                                                                                                                                                                                                                                                                                          & \multicolumn{1}{c|}{-}                                                                                                                                                      & \multicolumn{1}{c|}{-}                                                               & \multicolumn{1}{c|}{-}                                              & -                                                                                                      & \multicolumn{1}{c|}{\cellcolor[HTML]{95f985}4(4\%)}   & \cellcolor[HTML]{ae4d4d}5(5\%)                         \\ \hline
\multirow{3}{*}{Storage}                                 & \multicolumn{2}{l|}{Memory}                                                         & \multicolumn{1}{c|}{\cite{uddin2021horus,   koerhuis2020forensic, montanez2014investigation, dai2018sblwt,   van2017process, haigh2019if, volety2019cracking, zollner2019automated}}                                                                                                                                                                                                                                                          & \multicolumn{1}{c|}{\cite{ivanov2021ethclipper}}                                                                                                           & \multicolumn{1}{c|}{-}                                                               & \multicolumn{1}{c|}{-}                                              & -                                                                                                      & \multicolumn{1}{c|}{\cellcolor[HTML]{4ded30}9(10\%)}  & -                              \\ \cline{2-10} 
                                                         & \multicolumn{2}{l|}{Key Storage}                                                    & \multicolumn{1}{c|}{\cite{ko2020private,   bulut2020security, uddin2021horus, koerhuis2020forensic, Zhu2017py,   eskandari2018first, barber2012bitter, montanez2014investigation,   gentilal2017trustzone, Biryukov2019SecurityZcash, dai2018sblwt,   van2017process, Sai2019PrivacyApplications, Soltani2019zb, haigh2019if,   brengel2018identifying, Zhang2020ni, suratkar2020cryptocurrency,   Rakdej2019ri, praitheeshan2019attainable}} & \multicolumn{1}{c|}{\cite{bulut2020security,   eskandari2018first, Rakdej2019ri, suratkar2020cryptocurrency}}                                              & \multicolumn{1}{c|}{\cite{bulut2020security,   eskandari2018first}} & \multicolumn{1}{c|}{\cite{Rakdej2019ri}}           & -                                                                                                      & \multicolumn{1}{c|}{\cellcolor[HTML]{00c301}27(30\%)} & \cellcolor[HTML]{971919}8(18\%)                        \\ \cline{2-10} 
                                                         & \multicolumn{2}{l|}{Clipboard}                                                      & \multicolumn{1}{c|}{\cite{ulqinaku2019scan,   kim2018risk, li2020android}}                                                                                                                                                                                                                                                                                                                                                                    & \multicolumn{1}{c|}{-}                                                                                                                                                      & \multicolumn{1}{c|}{-}                                                               & \multicolumn{1}{c|}{-}                                              & -                                                                                                      & \multicolumn{1}{c|}{\cellcolor[HTML]{B7FFBF}3(3\%)}   & -                              \\ \hline
\multirow{3}{*}{Authentication}                          & \multicolumn{2}{l|}{Side Channel}                                                   & \multicolumn{1}{c|}{-}                                                                                                                                                                                                                                                                                                                                                                                                                                         & \multicolumn{1}{c|}{\cite{san2019practical,   park2023stealing, gentilal2017trustzone}}                                                                    & \multicolumn{1}{c|}{-}                                                               & \multicolumn{1}{c|}{-}                                              & -                                                                                                      & \multicolumn{1}{c|}{\cellcolor[HTML]{B7FFBF}3(3\%)}   & -                              \\ \cline{2-10} 
                                                         & \multicolumn{2}{l|}{Brute   Force}                                                  & \multicolumn{1}{c|}{\cite{volety2019cracking,   praitheeshan2019attainable}}                                                                                                                                                                                                                                                                                                                                                                  & \multicolumn{1}{c|}{\cite{rezaeighaleh2019new,   rezaeighaleh2020multilayered}}                                                                            & \multicolumn{1}{c|}{-}                                                               & \multicolumn{1}{c|}{\cite{vasek2017bitcoin}}       & -                                                                                                      & \multicolumn{1}{c|}{\cellcolor[HTML]{95f985}5(5\%)}   & \cellcolor[HTML]{b96666}1(2\%)                         \\ \cline{2-10} 
                                                         & \multicolumn{2}{l|}{Dictionary}                                                     & \multicolumn{1}{c|}{\cite{gentilal2017trustzone,   praitheeshan2019attainable, uddin2021horus, holmes2023framework}}                                                                                                                                                                                                                                                                                                                          & \multicolumn{1}{c|}{\cite{holmes2023framework}}                                                                                                            & \multicolumn{1}{c|}{-}                                                               & \multicolumn{1}{c|}{\cite{holmes2023framework}}    & -                                                                                                      & \multicolumn{1}{c|}{\cellcolor[HTML]{95f985}6(6\%)}   & -                              \\ \hline
\end{tabular}}
\end{table*}

\subsection{Network}
\label{sec:network}

\old{
\subsubsection{Routing}
\label{sec:routing}
It represents a critical area of concern as it targets the network-level communication and consensus mechanisms that govern blockchain networks. The two significant routing attacks in crypto-wallets and blockchain networks are the Partitioning Attack and the Delay Attack \cite{zaghloul2020bitcoin, conti2018survey, apostolaki2017hijacking, navamani2023review, ramos2021great}.
\paragraphtitle{Partitioning Attack}
\label{sec:partitioning}
This aims to split a blockchain network into disjoint groups, effectively isolating certain nodes or groups of nodes from the rest of the network. By doing so, the attacker gains control over a subset of the network and can manipulate transactions within that isolated portion. This attack has the potential to facilitate double-spending attacks, where the same cryptocurrency units are spent multiple times.
\paragraphtitle{Delay Attack}
This exploits the inherent delay in blockchain network communication to deceive nodes and introduce false information. By deliberately delaying the propagation of blocks or transactions, the attacker creates a window of opportunity to mine blocks without competition, maximizing their chances of receiving mining rewards. This attack capitalizes on the time-sensitive nature of blockchain consensus and exposes vulnerabilities in the network's resiliency.
}


\subsubsection{Deanonymization}
\label{sec:deanonymization}
This attack aims to uncover the identities of users within a network. Clustering techniques are employed to achieve this goal, allowing researchers to analyze wallet applications and protocols. Biryukov and Tikhomirov effectively cluster smaller networks such as the Bitcoin test network and Zcash by analyzing propagation times between nodes. However, clustering larger networks, such as the Bitcoin mainnet, requires substantial computing power, presenting challenges due to resource limitations \cite{Biryukov2019SecurityZcash}.
\paragraphtitle{Clustering Techniques}
\label{sec:clustering}
Reid et al. introduce a technique based on transaction graph analysis, leveraging the publicly accessible links between senders and receivers in the blockchain. This approach creates a network of identities and groups them together, facilitating the identification of commonly used transactions. Using financial and customer transaction histories, Bitcoin addresses can be traced if an individual's identity is disclosed \cite{reid2013analysis}.
Meiklejohn et al. propose a method that assumes combining Bitcoins from different addresses implies the same user controls those addresses. This approach focuses on the consolidation of Bitcoins and identifies addresses belonging to the same user based on this behavior \cite{meiklejohn2013fistful}.
Another clustering approach utilizes website analytic cookies to cross-match information on the blockchain and locate Bitcoin transactions. This technique enables the linkage of transactions to specific users through data integration \cite{goldfeder2017cookie}.
\paragraphtitle{Law Enforcement}
\label{sec:law-enforcement}
Teomete et al. highlight clustering techniques that can be employed by law enforcement agencies. By injecting mining software into a network, traffic can be monitored. Multiple validators receiving data from the same IP address are assumed to pertain to the same user. Linking IP addresses to real-life identities and locations can facilitate deanonymization \cite{teomete2019anonymous, koshy2014analysis}.

Deanonymization attacks pose a significant challenge to user privacy in blockchain networks. Clustering techniques, such as transaction graph analysis and IP address linkage, have been successful in identifying users and their associated Bitcoin addresses. Law enforcement agencies can utilize these methods to uncover the cryptocurrency accounts of suspects. These findings emphasize the need for improved privacy measures in blockchain systems to protect user identities and transactions \cite{khalilov2018survey, neudecker2017could, biryukov2014deanonymisation, androulaki2013evaluating, bergman2021revealing}.


\subsubsection{\acf{mitm}}
\label{sec:mitm}
This attack is a significant security concern for software wallets (\autoref{sec:software-wallets}), such as BitcoinWallet, Bither, DashWallet, and SimpleBitcoin. These attacks exploit vulnerabilities in the connection between the wallet and external sources, potentially allowing attackers to manipulate Bitcoin addresses and divert funds. In the case of Java programming, the X509TrustManager class vulnerability can be exploited to intercept connections by accepting unauthenticated certificates. BitcoinWallet and DashWallet rely on Electrum servers, which introduce a potential risk if the connections lack proper encryption and authentication. Bither utilizes its own API URLs and blockchain.info as a block explorer, which also presents a vulnerability if the connections are compromised \cite{garba2018analysis, 8613949, Biryukov2019SecurityZcash}.

The integrity of \ac{nfc} transactions (see \autoref{sec:nfc-wallets}) is also at risk due to potential \ac{mitm} attacks. These attacks involve the use of two cards: a legitimate and authorized card accepted by the system and an adversary-controlled card acting as a clone of a victim's card or a stolen card. By positioning the \ac{mitm} card between the legitimate card and the terminal, the attacker can intercept communication and successfully complete the transaction. The attacker uses data from the legitimate card during the Card Authentication phase and responds with the appropriate cryptogram during Transaction Authorization. The terminal forwards the cryptogram to the payment card association, which often does not apply further verification but sends it to the card-issuing bank. As the \ac{mitm} card possesses valid credentials, the issuer bank sends a response, leading to successful transaction \cite{9220841}.

To evaluate the usability of hardware wallets (\autoref{sec:hardware-wallets}) like KeepKey in the context of \ac{mitm} attacks, a study \cite{almutairi2019usability} employs a scripted experimental procedure. Participants are introduced to the KeepKey hardware wallet system and trained on sending bitcoins while being warned about the risk of sending funds to the wrong address. The study assesses participants' ability to detect potential MitM attacks by tasking them with refunding bitcoin amounts to customers under various attack conditions.


\subsubsection{\acf{dns}}
\label{sec:dns}
\paragraphtitle{\ac{dns} Hijacking}
\label{sec:dns-hijacking}
It represents a malicious technique where attackers manipulate the \ac{dns} to redirect users to fraudulent websites. It can be carried out by influencing user workstations or compromising \ac{dns} servers. A malware is used to modify the name servers configured on a user's workstation, redirecting \ac{dns} requests to malicious servers. Alternatively, attackers can hack and manipulate \ac{dns} servers to send incorrect responses. In both cases, users are directed to the attacker's site, enabling them to engage in activities such as phishing (see \autoref{sec:phishing}) and malware distribution (\autoref{sec:malware}). This type of attack can also be referred to as DNS spoofing. There have been instances of \ac{dns} hijacking attacks targeting web and desktop wallet (see \autoref{sec:software-wallets}) users. FireEye discovered a large-scale DNS hijacking effort in 2019 that successfully targeted victims worldwide \cite{mandiant2022, cisa2019, KTSSH}. In response, the US Cybersecurity and Infrastructure Security Agency issued emergency directives to mitigate \ac{dns} hijacking risks. The Sea Turtle campaign, a state-sponsored attack, also employs \ac{dns} hijacking to capture user credentials from numerous companies across multiple countries \cite{talosintel2019, AGCH}.
\paragraphtitle{\ac{dns} Tunneling}
\label{sec:dns-tunneling}
In this type of attack, \ac{dns} queries and responses are manipulated to carry malicious payloads, unauthorized access attempts, command and control instructions, or even bidirectional protocols like SSH. \ac{dns} traffic is often considered safe and goes unnoticed, providing an opportunity for attackers to establish covert communication channels. Point of sale (PoS) malware, such as MULTIGRAIN, is one example of client malware that utilizes \ac{dns} tunneling. Additionally, remote backdoors like \ac{dns}Messenger and \ac{dns}pionage are known to exploit this technique. Open-source \ac{dns} tunneling tools like Heyoka, dnscat2, and iodine further exemplify the implementation of such attacks \cite{Banoth2023, ALIR}.
\paragraphtitle{\ac{dns} Cache Poisoning}
\label{sec:dns-cache-poisoning}
It occurs when a \ac{dns} resolver, due to a vulnerability, accepts an invalid resource record, resulting in the cache being \quotes{poisoned} with incorrect data. This can be achieved by an intruder utilizing a long \acf{ttl} value that prolongs the retention of the malicious data in the resolver's cache. The impact of cache poisoning is similar to \ac{dns} hijacking, as users can be directed to fraudulent websites or manipulated resources posing a threat to software wallets (\autoref{sec:software-wallets}) \cite{son2010hitchhiker}.


\subsubsection{\acf{dos}}
\label{sec:dos}
\paragraphtitle{D\ac{dos}}
\label{sec:ddos}
These attacks are a concerning threat to the security of crypto-wallets. Although not exclusive to blockchain networks, these attacks can be specifically targeted towards blockchain and asset exchange networks with certain modifications. In this attack, the attacker utilizes a network of compromised devices to overwhelm the targeted network with an excessive volume of requests, resulting in a significant degradation of the network's capacity to handle legitimate traffic. An instance of such an attack occurred during the early stages of Bitcoin Gold, a Bitcoin fork, where the network was subjected to a massive D\ac{dos} attack, experiencing an overwhelming influx of 10 million false requests per minute. These D\ac{dos} attacks can disrupt the functioning of crypto-wallets, hampering users' ability to access and transact with their digital assets \cite{feder2017impact, MRRR2021, RMMDD, SGAP}.
\paragraphtitle{\ac{dos} on Connectivity}
\label{sec:dos-connectivity}
Attacks targeting the connectivity of nodes can have significant implications for the security of crypto-wallets. Such attacks can result in a loss of consensus power, preventing consensus nodes from being rewarded and disrupting blockchain-dependent services for validating nodes. This type of attack aims to disrupt the connectivity between nodes, causing a breakdown in the consensus mechanism and impeding the normal functioning of the blockchain network. A \ac{dos} attack on connectivity can hinder users' ability to perform transactions, access their digital assets, and engage with the blockchain ecosystem effectively \cite{homoliak2020security}.
\paragraphtitle{\ac{dos} on Local Resources}
\label{sec:dos-local-resources}
Attacks targeting local resources, such as memory and storage, can have a significant impact on the functionality of crypto-wallets. These attacks aim to degrade the peering and consensus capabilities of nodes within the network. One such attack is flooding the network with low-fee transactions, also known as penny-flooding. This flood of transactions can cause depletion of the memory pool, leading to system crashes and disruptions in the wallet's operations. To mitigate such attacks, measures can be taken within crypto-wallets, such as raising the minimum transaction fee and implementing rate limits on the number of transactions \cite{homoliak2019security}.
\paragraphtitle{\ac{dos} on Smart Contracts}
\label{dos-smart-contract}
Attacks can also be targeted towards smart contracts (\autoref{sec:smart-contract-wallets}) with the purpose of disrupting, suspending, or freezing the execution of the contract, or even compromising the logic that underlies the contract itself. These kinds of attacks can render the contract invalid and impede its ability to function as it was intended to. A \ac{dos} vulnerability in an Ethereum contract might result in the malicious manipulation or uncontrolled consumption of resources, which would lead to an enormous amount of gas being used. The repercussions may be severe, leading to monetary losses and exposing users of applications as well as the applications themselves to danger \cite{samreen2021smartscan, krupp2018teether}.


\subsection{Application}
\label{sec:application}

\subsubsection{Social Engineering}
\label{sec:social-engineering}
Social engineering attacks leverage human vulnerabilities, such as trust and curiosity, to manipulate crypto-wallet users into complying with seemingly legitimate requests, catching them off guard, or emotionally manipulating them. The lack of security awareness among many crypto-wallet users further exacerbates their susceptibility to these tactics. Additionally, the high stakes associated with crypto-wallets, as they store valuable assets, make them attractive targets for attackers, leading users to potentially make impulsive decisions when they perceive their funds to be at risk. The consequences of key loss due to successful social engineering attacks are dire, as users can experience irreversible loss of funds if they lose access to their private keys or seed phrases. Adding to the challenge, attackers continuously enhance their techniques, making social engineering attacks increasingly sophisticated, convincing, and difficult to detect, further emphasizing the critical importance of cybersecurity measures to protect crypto-wallet users and their assets in this evolving landscape.
\paragraphtitle{Impersonation}
\label{sec:impersonation}
This involves the adversary assuming a false identity, such as a trusted authority, to gain the trust of the crypto-wallet user. By impersonating a legitimate entity, the attacker aims to deceive the user into revealing sensitive information or granting access to their crypto-wallet. This attack can be carried out in person, over the phone, or through electronic communication \cite{parthy2019identification, yasin2019contemplating, breda2017social}.
\paragraphtitle{Shoulder Surfing}
\label{sec:shoulder-surfing}
This refers to the act of covertly observing a user as they enter their wallet credentials, such as private keys or passwords. The attacker may physically stand behind the user while they access their wallet on a computer or mobile device, or they may observe from a distance in public spaces. The goal is to obtain the user's confidential information to gain unauthorized access to their crypto-wallet \cite{parthy2019identification, yasin2019contemplating, breda2017social}.
\paragraphtitle{Eavesdropping}
\label{sec:eavesdropping}
This involves the unauthorized interception of private conversations or communications between the user and trusted parties. Attackers may exploit vulnerable communication channels, such as unsecured email or telephone lines, to listen in on discussions related to the user's crypto-wallet, including key management or authentication processes \cite{breda2017social}.
\paragraphtitle{Phishing}
\label{sec:phishing}
This is a prevalent and insidious attack that targets crypto-wallet users through electronic communication channels. In this context, attackers craft fraudulent emails, messages, or websites that appear to be from legitimate sources like crypto exchanges or wallet providers. The aim is to trick users into disclosing their wallet credentials, private keys, or other sensitive information. The sophistication of phishing attacks can vary, ranging from basic emails to highly convincing replicas of official websites \cite{parthy2019identification, yasin2019contemplating, breda2017social}.
\paragraphtitle{Baiting}
\label{sec:baiting}
This involves enticing users with physical or digital bait that contains malicious software. For instance, attackers might leave infected USB drives in public places or send them directly to targeted individuals, hoping that the user will plug the USB drive into their computer, thereby unknowingly installing malware that compromises their wallet \cite{parthy2019identification, yasin2019contemplating, breda2017social}.
\paragraphtitle{Pop-up Window}
\label{sec:pop-up-window}
In this attack, users are presented with persistent notifications claiming various issues with their crypto-wallet, such as a connection loss or malware detection. The goal is to prompt the user to interact with the pop-up, leading them to unintentionally execute a malicious program that steals their login information or private keys \cite{parthy2019identification}.

\subsubsection{Rooting and Debugging}
\label{sec:rooting-debugging}
Attackers can utilize artifact analysis techniques to scan the memory of mobile devices and extract sensitive information \cite{li2020android}. Studies have shown the feasibility of extracting PIN codes from popular mobile wallet (see \autoref{sec:mobile-wallets}) applications \cite{9143206}, obtaining sensitive data using \acf{adb} on rooted devices \cite{haigh2019if}, performing brute-force attacks (\autoref{sec:brute-force}) after gaining access through debugging mode \cite{volety2019cracking}, and extracting information about wallet applications on both Android and iOS devices \cite{montanez2014investigation}. In an analysis of 311 Android wallet apps, it was found that 111 apps store key-related information in plain text, only 20 apps implement their own keyboards to prevent mnemonic passphrase prediction, and just 70 apps check for device rooting, leading to security risks \cite{uddin2021horus}.

Certain features in the Android platform, while beneficial to developers and users, can be exploited by attackers, posing risks to the security of cryptocurrency systems on Android \cite{9143206}. The security of crypto-wallets is crucial as they contain sensitive information like private keys necessary for fund access and transfers. Two identified attacks that compromise wallet security involve capturing sensitive information from the user's screen and capturing user input through USB debugging. The former entails creating an accessibility service to extract displayed text and search for predefined sensitive data, successfully accessing sensitive information in apps like imToken and Huobi Global Wallet \cite{SHPS}. The latter attack exploits USB debugging to acquire user input events.

% \input{Exhibits/Vulnerabilities-Table-8.tex}

\subsubsection{Faulty Libraries}
\label{sec:faulty-libraries}
Vulnerable libraries pose a significant risk in crypto-wallets. The improper use of the libbitcoin library, according to Tschannen et al., can result in flaws and security vulnerabilities, including zero-day attacks \cite{tschannen2020evaluation}. Hu et al. identified a vulnerability in the BitcoinJ client library used by wallet applications like Bitcoin Wallet and Mycelium, which allows for manipulation of connections and transactions \cite{hu2021security}. Few software wallets (\autoref{sec:software-wallets}), such as Bitcoin Wallet and Coinbase, also violate the decentralization principle by relying on the provider's server instead of directly joining the network \cite{hu2021security}.

\subsubsection{Malware}
\label{sec:malware}
A vulnerability in mobile wallets (\autoref{sec:mobile-wallets}), involves exploiting the Accessibility Mode feature \cite{li2020android}. Attackers can request the \quotes{Bind Accessibility Service} permission, granting access to GUI events and displayed information on the screen, except for password textboxes. This allows backdooring of Android's security model \cite{leguesse2020reducing}. Attackers can upload a seemingly harmless app to the PlayStore, which requests accessibility mode permission. The app tricks users into enabling installation from third-party sources and downloading a malicious app with marked permissions. Enabled accessibility features combined with third-party keyboards can capture sensitive information by listening to click events \cite{li2020android}.


\subsubsection{Storage}
\label{sec:storage}
\paragraphtitle{Memory}
\label{sec:memory}
Attacks targeting computer memory can expose sensitive information, posing a significant risk to desktop wallet (see \autoref{sec:desktop-wallets}) security. By analyzing the computer's RAM, critical data such as private keys, PINs, and mnemonic passphrases can be extracted and exploited. Researchers have demonstrated the extraction of valuable artifacts from web applications, including wallet URLs, mnemonic passphrases, and wallet IDs. In-depth investigations focusing on privacy-oriented cryptocurrencies like Monero and Verge have successfully retrieved wallet passphrases, transaction history, and associated addresses through meticulous memory and disk analysis techniques \cite{haigh2019if, uddin2021horus, zollner2019automated, koerhuis2020forensic}.

\paragraphtitle{Key Storage}
\label{sec:key-storage}
Brain wallets (\autoref{sec:brain-wallets}) have emerged as a convenient solution, using a passphrase of 12 or 24 words stored in memory instead of a physical device. However, this approach introduces a vulnerability, as attackers can guess the passphrase without restrictions, potentially gaining control over the wallet. A study by Vasek et al. found that the majority of brain wallets were emptied within 24 hours \cite{vasek2017bitcoin}. In memory analysis, the improper storage of sensitive information, such as private keys, is a major concern. Accessing this data requires physical access to the device and it being powered on. However, extracting all data from memory is not always feasible without user or root access, depending on the application.
Attackers also exploit Android's storage division into internal and external storage \cite{li2020android}. Pop-up windows or mimicking older Android versions bypass runtime permission to access external storage, where backup files may contain unencrypted sensitive information like transaction IDs \cite{uddin2021horus}. \quotes{System Alert Window} permission allows tampering with the UI, capturing passwords, or transaction credentials. Modifying the clipboard intercepts copied addresses, allowing attackers to replace them with their own \cite{kim2018risk}. Additionally, overlay attacks using fake QR codes exploit the scan-and-pay mechanism \cite{ulqinaku2019scan}.

\paragraphtitle{Clipboard}
\label{sec:clipboard}
Web wallets (see \autoref{sec:web-wallets}) can leave behind artifacts even after removal, posing security risks \cite{zollner2019automated}. Inappropriately stored credentials, plaintext transactions, and contact lists raise privacy concerns \cite{haigh2019if, van2017process}. Publicly shared debugging logs on platforms like Pastebin can expose sensitive information \cite{brengel2018identifying}. Few wallets have vulnerabilities like hard-coded encryption keys and inadequate protection of sensitive data \cite{haigh2019if, Sai2019PrivacyApplications}. Secure key generation is crucial, as nonce reuse incidents have occurred due to weak random number generators or faulty implementations \cite{brengel2018identifying, ko2020private}.

\subsection{Blockchain}
\label{sec:blockchain}

\subsubsection{Programming Errors}
\label{sec:programming-language}

\paragraphtitle{Mishandled Exceptions}
\label{sec:mishandled-exceptions}
Smart contracts often require calling another contract to fulfill specific functionalities. These calls can be made either by sending instructions or directly invoking a contract's method with a reference to the contract's name. In the callee contract, exceptions can be raised, leading the callee contract to terminate and revert its state while returning a false value to the caller contract. These exceptions may occur for a variety of reasons, such as running out of gas before the operation is finished, exceeding the call stack limit, unanticipated system problems in the callee node, and more. To ascertain whether the call was successfully completed or not, the caller contract must explicitly examine the return value, and the callee contract must propagate any exceptions to the caller. However, occasionally there may be inconsistent exception propagation mechanisms, which pose dangers to real-world transactions \cite{atzei2017survey, ghaleb2018addressing, atzei2017survey, luu2016making, kalra2018zeus, delmolino2016step}.

\paragraphtitle{Integer Overflow/ Underflow}
\label{sec:integer-overflow-underflow}
It is a critical vulnerability that can affect smart contract crypto-wallets. In Solidity, the integer type uint256 has a maximum size of 256 bits. When the value of an integer variable reaches its maximum value (2\textsuperscript{256} - 1), adding an additional integer 1 to the variable will cause it to automatically wrap around and be reset to zero. This behavior can be exploited by hackers who target these variables in smart contracts. They attempt to manipulate the value of integers by continuously increasing or decreasing them until they reach the maximum or minimum value, respectively. Such manipulation can lead to unexpected and unintended consequences, potentially resulting in financial losses or unauthorized access to assets within the crypto-wallet. It is crucial to implement robust input validation, range checking, and careful handling of arithmetic operations involving integers in smart contract crypto-wallets to mitigate the risks associated with integer overflow/underflow vulnerabilities \cite{singh2020blockchain, chen2019exploiting, mik2017smart, tsankov2018securify, brent2018vandal, kalra2018zeus}.

\subsubsection{Temporal Dependency}
\label{sec:temporal-dependency}

\paragraphtitle{Re-entrancy}
\label{sec:re-entrancy}
This vulnerability occurs when a malicious contract repeatedly calls back into the calling contract before the initial function invocation is completed. This recursive nature of the call allows the attacker to execute multiple repetitive withdrawals without affecting their balance. An example of this vulnerability is the \acf{dao} attack, where an attacker exploited the re-entrancy vulnerability to steal \$60 million. In Ethereum smart contracts, there is an unnamed fallback function that is invoked when a contract calls the fallback function of another contract or the same contract to transfer Ether. The code inside the fallback function will keep running until the remaining gas is used up if it includes harmful code and there is no gas restriction for the call method invocation. This vulnerability poses a serious attack vector in smart contract wallets and is still a potential risk for new contracts \cite{saad2019exploring, liao2019soliaudit, min2019blockchain, liu2018reguard}.

\paragraphtitle{Timestamp Dependency}
\label{sec:timestamp-dependency}
It serves as an initial condition for executing critical operations, typically set to the system time of the miner's local computer or server. When a block is mined, the miner generates the timestamp for that block. It is important to note that the timestamps of different blocks can vary by approximately 900 seconds. After establishing the legitimacy of a new block, the miner determines whether its timestamp is greater than the timestamp of the preceding block and whether it is less than 900 seconds from the date of the new block. Because of the miners' latitude in choosing the block timestamp, malicious or hostile miners have a chance to influence the results of smart contracts that depend on timestamps. The miner can alter the timestamp by a few seconds to sway the result in their favor if a contract depends on the current time, starting time, and ending time depending on the block's timestamp \cite{alharby2018blockchain, wohrer2018design, bartoletti2017empirical, atzei2017survey, luu2016making}.

\paragraphtitle{Transaction Ordering}
\label{sec:transaction-ordering}
A block consists of a set of transactions, and the blockchain state undergoes multiple updates during each epoch. The state of a smart contract is determined by the values of its fields and the current balance. When a user initiates a transaction to invoke a smart contract in the network, there is no guarantee that the transaction will run in the same state as when the contract was initially initialized. The actual state of the smart contract becomes unpredictable to any user when it is called by their transaction \cite{natoli2016blockchain, wang2019detecting, di2019survey, luu2016making, atzei2017survey, cong2019blockchain, li2020survey}.

\subsubsection{Access Control}
\label{sec:access-control}

\paragraphtitle{Sequential Execution}
\label{sec:sequential-execution}
In blockchain networks, the execution of transactions on smart contracts follows a sequential order determined by the consensus mechanism. The consensus technique is used to arrange the requests for smart contract invocations throughout this sequential execution, and the contracts are carried out uniformly across all nodes. Sequential execution, however, has several performance restrictions and disadvantages in blockchain-based systems. One of the most significant challenges is the effect on the effective throughput of the blockchain application, as sequential operations increase latency and become a performance bottleneck. Malicious users can exploit this by introducing smart contracts with lengthy execution times, thereby delaying subsequent transaction traffic and undermining network performance. The sequential execution of smart contracts restricts the number of contracts that can be executed per second, limiting the overall performance and scalability of the network \cite{androulaki2018hyperledger, vukolic2017rethinking, alharby2018blockchain, gao2017scalable, yu2017smart, dickerson2017adding, sergey2017concurrent, wohrer2018smart}.

\paragraphtitle{No Restricted Write}
\label{sec:no-restricted-write}
Write operation can be performed on storage variables without any restricted conditions or authorization checks. This vulnerability allows attackers to exploit the contract by manipulating the storage and performing unauthorized operations \cite{nikolic2018finding}. The absence of restricted write conditions can lead to severe security breaches, as in the case of the Parity multisig wallet hack. In this incident, the attacker was able to set the ownership of the wallet library without any conditions or proper authorization checks, resulting in unauthorized control over the wallet and potentially compromising the funds held within \cite{breidenbach2017depth, kalra2018zeus}.

\paragraphtitle{Function Visibility}
\label{sec:function-visibility}
The visibility specifiers in Solidity are crucial in determining the accessibility and exposure of functions within the contract. These visibility specifiers, including external, public, internal, and private, define who can interact with the functions and how they can be accessed. Incorrect usage or misconfiguration of these visibility specifiers can introduce vulnerabilities to the smart contract, making it an attractive target for attackers. The visibility specifiers play a vital role in controlling the visibility and permissions associated with critical functions in the crypto-wallet smart contract, such as balance management, transaction handling, and asset transfers \cite{wohrer2018design, lin2017survey}.

\subsubsection{Resource Mismanagement}
\label{sec:resource-mismanagement}

\paragraphtitle{Destroyable Contract}
\label{sec:destroyable-contract}
A contract that can be terminated or killed by executing a self-destruct function. This function can be triggered by an external user account or another smart contract and is typically used by the contract's owner in response to an attack or emergency situation. The self-destruct function should be designed to validate the user executing it, allowing only legitimate owners to invoke the kill method. By incorporating this functionality, destroyable smart contracts provide a mechanism to shut down and remove compromised or vulnerable contracts from the blockchain network, ensuring the security and integrity of the crypto-wallet ecosystem. This feature allows contract owners to take swift action when necessary, mitigating risks and protecting user assets within the crypto-wallet from potential threats \cite{nikolic2018finding}.

\paragraphtitle{Gas Costly Pattern}
\label{sec:gas-costly-pattern}
It refers to specific coding patterns implemented in Solidity code that result in higher gas consumption during the execution of each instruction \cite{chen2017under}. These patterns have been identified as being more expensive in terms of gas usage, potentially leading to higher transaction fees and reduced efficiency in contract execution \cite{destefanis2018smart}. The presence of gas costly patterns in contract code can impact the overall cost-effectiveness and performance of smart contract crypto-wallets \cite{wang2019vultron, ghaleb2018addressing}.

\subsubsection{Economic Explotation}
\label{sec:economic-exploitation}

\paragraphtitle{Unsecured Balance}
\label{sec:unsecured-balance}
The balance of a contract is exposed and susceptible to being drained by an anonymous caller. This vulnerability arises from an inadequate access control mechanism for the balance variable and constructor functions, or from improperly updating the balance after invoking a call instruction to transfer funds to another contract or arbitrary user. In such cases, the contract lacks the necessary security measures to protect its balance from unauthorized access or manipulation. This vulnerability poses a significant risk to the assets held within the smart contract crypto-wallet, as attackers can exploit the weakness to drain the contract's balance and potentially cause financial losses for users \cite{nikolic2018finding, brent2018vandal}.

\paragraphtitle{Greedy Contract}
\label{sec:greedy-contract}
It refers to a contract that remains active and continuously locks the Ether balance due to the inability to access external library contracts for fund transfers or sending funds. These contracts are labeled as greedy contracts according to \cite{nikolic2018finding}. The term \enquote{greedy} implies that these contracts retain funds without releasing or allowing their transfer to other addresses. If the library contracts that are called by these contracts are intentionally or accidentally terminated or destructed by an arbitrary user, the contracts that rely on these external library functions become greedy contracts themselves. The attackers exploited this vulnerability in the Parity Multisig wallets by claiming ownership of the wallet library contracts and subsequently destructing them, resulting in the freezing of funds in the affected wallet contracts \cite{breidenbach2017depth}.

\subsection{Authentication}
\label{sec:authentication}
\subsubsection{Side Channel Attacks}
\label{sec:side-channel}
These attacks involve exploiting unintended information leakage, such as power consumption measurements, to extract sensitive information like private keys \cite{san2019practical}. These attacks can be conducted using techniques like Single Trace Power Analysis and High-Correlation Analysis on cryptographic operations \cite{park2023stealing}. The attacker leverages the leaked information to compromise the security of the hardware wallet (\autoref{sec:hardware-wallets}). Side channel attacks can be carried out with minimal setup and rely on analyzing power traces or other unintended leakages \cite{gentilal2017trustzone}.

\subsubsection{Brute Force}
\label{sec:brute-force}
These attacks on crypto-wallets involve systematically trying all possible combinations of characters or words to crack the passphrase or access the wallet \cite{rezaeighaleh2019new}. In this offline approach, the attacker uses a large dataset of words and generates a dictionary file for the brute force attack. The process is time-consuming and requires significant computational resources, often utilizing multiple machines or virtual machines operating in parallel \cite{volety2019cracking}. By attempting to crack the wallet using various combinations, the attacker identifies valid combinations that could potentially access the wallet. However, the success rate decreases as the number of combinations increases. The enormous number of possible combinations, especially when dealing with long passphrases or extensive wordlists, makes brute force attacks highly time-consuming and practically infeasible \cite{vasek2017bitcoin, praitheeshan2019attainable}.

\subsubsection{Dictionary Attacks}
\label{sec:dictionary}
This exploits the user dictionary feature present in keyboard apps to predict the mnemonic phrase or passphrase used in the mobile wallet (see \autoref{sec:mobile-wallets}). Wallet apps typically rely on the default keyboard, which uses a user dictionary for predictive text inputs. The mnemonic phrase consists of common English words, and the information regarding these words is saved in the user dictionary. An attacker app with virtual keyboard permission can access the dictionary and extract frequency information of typed words to predict the mnemonic phrase. Only a small percentage of wallet apps (6\%) have implemented custom keyboards to protect against dictionary attacks \cite{uddin2021horus}. The effectiveness of dictionary attacks can be enhanced by building dictionaries based on leaked password datasets and analyzing the frequency of occurrence of passwords \cite{praitheeshan2019attainable}. Smarter guessing techniques can also increase the success rate \cite{holmes2023framework}.
