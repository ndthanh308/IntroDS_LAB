\section{Cryptography Primitives}
\label{sec:wallet-infrastructure}

Cryptography plays a crucial role in ensuring the security of crypto-wallets. Four main types of cryptographic techniques used are asymmetric cryptography, digital signatures, hash functions, and \acf{ecc}. \old{\cite{ASCCB}}
In a blockchain system, users have a pair of cryptographic keys, a public key, and a private key. The public key is used to receive transactions, while the private key is used to sign transactions and prove ownership of the public key\old{ \cite{SSCC, AAPK}}.
When a user sends a transaction on a blockchain network, they create a digital signature using their private key. The digital signature is then verified using their public key, ensuring that the transaction is legitimate and was indeed initiated by the user who owns the public key\old{ \cite{CDSEB, GSDSB, KMPKPK}}.

\subsection{Asymmetric Key Cryptography} \old{Asymmetric key encryption is a method of encrypting and decrypting data using a pair of keys: a public key that is used for encryption, and a private key that is used for decryption as shown in  \autoref{Asymmetric-Cryptography}. This type of encryption is also known as public key encryption. The security of asymmetric key encryption relies on the computational infeasibility of deriving the private key from the public key \cite{DRASE}. Asymmetric key cryptography in crypto-wallets involves generating a unique key pair (private key and public key) for each user, deriving wallet addresses from the public key, creating and verifying digital signatures for transaction authenticity, enabling secure and confidential transactions through encryption, and ensuring key management and access control for secure storage and verification of private keys. Examples of asymmetric key encryption algorithms that are commonly used in crypto-wallets include RSA and \ac{ecdsa}. RSA is one of the most widely used asymmetric key encryption algorithms, while ECDSA is a more efficient and secure alternative that is based on \ac{ecc}\old{ \cite{JTER}} as shown in \autoref{Encryption-Table-1} with comparison of 50 encryption algorithms.}

Asymmetric cryptography, known as public-key cryptography, is a cryptographic method that utilizes a pair of keys: a public key and a private key. \old{as shown in \autoref{Asymmetric-Cryptography}}

% % Figure environment removed

\subsubsection{Public and Private keys}
Public and private keys are integral components of asymmetric cryptography. Each user in a blockchain network possesses a unique key pair, consisting of a public key and a private key.

\paragraphtitle{Public key}

The public key is intended for public distribution and is used for encryption. It is derived from the private key through a complex mathematical algorithm. The public key can be shared with others, allowing them to encrypt messages or data that can only be decrypted by the corresponding private key.

\paragraphtitle{Private key}
The private key is kept secret and securely stored by the user. It is used for decryption and digital signature generation. The private key is used to decrypt messages or data that have been encrypted with the corresponding public key. It is also used to generate digital signatures, which provide authentication and non-repudiation in blockchain transactions.

\subsubsection{Key Generation}
This process of involves creation of a unique key pair for each user in a blockchain network. This is typically achieved through sophisticated mathematical algorithms. The generation of a secure key pair relies on randomness and computational complexity to ensure the uniqueness and strength of the keys.

% Asymmetric cryptography plays a crucial role in blockchain wallets, offering several key functionalities:
\paragraphtitle{Wallet Address Derivation}
Blockchain wallets utilize the public key to derive unique wallet addresses. These addresses serve as identifiers for receiving digital assets. Wallet addresses are generated by applying a hash function to the public key, ensuring the privacy and security of the underlying public key.

\paragraphtitle{Digital Signature Creation}
To verify the authenticity and integrity of transactions, blockchain wallets use the private key to generate digital signatures. Digital signatures are created by applying cryptographic algorithms to the transaction data, combined with the private key. These signatures serve as proof of ownership and ensure that the transactions are not tampered with during transmission.

\paragraphtitle{Secure and Confidential Transactions}
Asymmetric cryptography enables secure and confidential transactions within blockchain wallets. The public key of the recipient is used for encryption, ensuring that only the intended recipient can decrypt and access the transmitted data. This protects the confidentiality of the transaction details, including the recipient's wallet address and the transferred assets.

\subsection{Digital Signatures}
Digital signatures are a method of adding a digital signature to a piece of data to verify its authenticity as shown in \autoref{Digital-Signature}. They are based on the use of asymmetric key and are generated using the private key of the signer\old{ \cite{GIDSB, BCCS}}. Digital signature algorithms that are used in crypto-wallets include RSA and ECDSA.
% \cite{JTER}

% % Figure environment removed

\subsubsection{Role of Digital Signatures in Crypto-wallets}
Digital signatures play a critical role in ensuring the integrity, authenticity, and non-repudiation of transactions within crypto-wallets. They provide a mechanism for verifying the identity of the sender and confirming that the transaction data has not been tampered with during transmission or storage. By using asymmetric key encryption algorithms, digital signatures establish trust and security in the blockchain ecosystem.

\subsubsection{Transaction Signing Process}
\paragraphtitle{Combining transaction data and private key}
To create a digital signature, the transaction data, including the relevant details such as sender, recipient, amount, additional information, is combined with the private key of the sender. This combination ensures that the signature is unique to the specific transaction and the associated private key.

\paragraphtitle{Generating a Digital Signature}
Using the combined transaction data and private key, a digital signature is generated. This process involves applying a cryptographic algorithm, such as RSA or ECDSA, to create a mathematical representation of the transaction data that is unique to the signer's private key. The resulting digital signature serves as proof of authenticity and integrity for the transaction.

\subsubsection{Signature Verification Process}
The verification of digital signatures within blockchain wallets is crucial for ensuring the validity and trustworthiness of transactions.

\paragraphtitle{Using the Public key to Verify Signatures}
To verify the digital signature, the recipient or any other party can utilize the public key associated with the sender's wallet. The digital signature is decrypted using the public key, and the resulting decrypted data is compared to the original transaction data. If the two match, the signature is considered valid, indicating that the transaction has not been modified and was indeed signed by the private key holder.

\paragraphtitle{Ensuring Transaction Integrity and Authenticity}
The verification of digital signatures provides strong assurance regarding the integrity and authenticity of transactions within blockchain wallets. By confirming that the signature is valid and matches the transaction data, recipients can trust that the transaction has not been tampered with or falsified. This verification process helps prevent fraud, ensures non-repudiation, and enhances the overall security and reliability of the blockchain network.

\subsection{Hash Functions}
Hash functions are a method of generating a fixed-size output, known as a hash or hash code, from an input of any size \old{as shown in \autoref{Hashing}}. These are designed to be one-way functions, meaning that it is computationally infeasible to derive the original input from the generated hash\old{ \cite{ALHF}}. They are widely used in wallets to generate unique addresses, ensure data integrity, securely store and verify passwords, and generate message authentication codes. They enhance security and privacy within blockchain wallet operations. Examples of hash functions that are commonly used in crypto-wallets include Secure Hash Algorithm 2 (SHA-2) and Secure Hash Algorithm 3 (SHA-3) as shown in \autoref{Hashing-Table-1} with comparison of 24 hashing algorithms.
% \cite{DRS256}

% % Figure environment removed

\subsubsection{Generating Wallet Addresses}
One key application of hash functions in wallets is the generation of unique wallet addresses. Wallet addresses are derived from the public key associated with the wallet using a hash function. This process ensures that each address is unique and can be used to receive digital assets. By applying a hash function to the public key, a fixed-size output is obtained, which serves as the wallet address. This address can be safely shared with others for transactions.

\subsubsection{Transaction and Block Identifiers}
Hash functions are utilized to create unique identifiers for transactions and blocks within the blockchain. Each transaction or block is assigned a hash, which acts as a unique fingerprint. This hash is computed by applying a hash function to the relevant data, such as transaction details or block contents. These identifiers enable efficient identification, verification, and retrieval of specific transactions or blocks within the blockchain network.

\subsection{\ac{ecc}}

\ac{ec} is a cryptographic approach based on the algebraic structure of elliptic curves. These curves are defined by equations of the form $y^2 = x^3 + ax + b \pmod{p}$, where $p$ is a prime number. \ac{ecc} operations involve mathematical operations such as addition, subtraction, multiplication, and division within a finite field. The \ac{ec} structure provides the foundation for the security and efficiency of \ac{ecc} algorithms.

\subsubsection{Advantages of \ac{ecc} in Crypto-wallets}

\paragraphtitle{Enhanced Security} \ac{ecc} provides a higher level of security compared to traditional cryptographic algorithms, such as RSA, for the same key length. This enables the use of shorter key lengths, reducing computational overhead and storage requirements while maintaining robust security.

\paragraphtitle{Efficient Key Generation} \ac{ecc} operations involve smaller key sizes and faster computation compared to other asymmetric cryptographic algorithms. This Efficiency is crucial in resource-constrained environments such as blockchain networks, where speed and scalability are paramount.

\paragraphtitle{Compact Signatures and Key Sizes} \ac{ecc} produces shorter digital signatures and key sizes compared to other algorithms, resulting in reduced bandwidth and storage requirements. This is particularly beneficial in blockchain wallets, where transaction data and storage space are limited.

\paragraphtitle{Forward Secrecy} \ac{ecc} ensures that the compromise of a private key cannot be used to decrypt past transactions or compromise future transactions. This property enhances the long-term security of blockchain wallets.

\subsubsection{Generating \ac{ecc}-based Key Pairs}

\paragraphtitle{Curve Parameter Selection} Various elliptic curves with different parameters, such as $a$ and $b$, are available for \ac{ecc}. Commonly used curves include secp256k1 \old{(parameters as shown in \autoref{curve-paramaters})}, defined in the Standards for Efficient Cryptography, which is employed in many blockchain networks.

% % Figure environment removed

\paragraphtitle{Private Key Generation} A random private key is generated, typically represented as a fixed-length binary number. The private key is securely generated using a cryptographically secure random number generator. It should remain confidential and never be shared.

\paragraphtitle{Public Key Computation} The corresponding public key is derived from the private key through \ac{ec} point multiplication. This operation involves performing mathematical calculations on the \ac{ec} using the private key as a scalar. The resulting point on the curve represents the public key.

\paragraphtitle{Key Validation} Both the private and public keys undergo validation processes to ensure their correctness and adherence to specified standards. This validation helps prevent key-related issues that could compromise the security and functionality of the blockchain wallet.