\section{Conclusion}
\label{sec:conclusion}


In this comprehensive study, we conducted a multi-faceted examination of cryptocurrency wallets. Our research began with the formulation of an exhaustive taxonomy of crypto-wallets, shedding light on their distinctive characteristics. Following this, we deconstructed prevalent vulnerabilities and associated attack vectors, taxonomising attack based on exiting literature. We alsl
oanalyzed extant defense mechanisms and mitigation strategies targeted at neutralizing these threats. In addition to this, we conducted a comparative study of diverse wallet types, summarizing their respective security attributes and inherent risks. We also underscored the rising significance of semi-custodial wallets and quantum-resistant crypto-wallets in the dynamic landscape of digital asset security.

Overall, this research represents a concerted effort to deepen the understanding of wallet security and, thereby, contribute to the advancement of a resilient and secure cryptocurrency ecosystem.

% \old{
%     Our research provides a comprehensive analysis of crypto-wallets, focusing on their design, vulnerabilities, and defense mechanisms. We explored the cryptographic algorithms used in building wallets and categorized them based on funds controlability, internet connectivity, and infrastructure. Our primary focus was on identifying and analyzing various attacks that target crypto-wallets, as well as the available defense mechanisms.

% 	Our findings highlight that wallet attacks targeting the application software are the most prevalent from the user's perspective. To enhance security, users are advised to choose reputable wallet applications and consider using multiple wallets, combining offline and online options. We have also determined that paper wallets and brain wallets offer the highest level of security for storing cryptocurrencies. However, they require users to manage their own private keys and take full responsibility for their wallet security. Hardware wallets, while providing a good balance between security and convenience, may present usability challenges for some users. On the other hand, smart contract wallets offer ease of use but are prone to coding errors, including vulnerabilities such as re-entrancy attacks, transaction ordering issues, gas cost inefficiencies, and the risk of being a destroyable contract.

% 	We recognize the emerging importance of semi-custodial wallets, which strike a balance between user control and platform convenience. Exploring the security aspects of these wallets and developing robust defense mechanisms specific to their architecture will be crucial. Additionally, given the growing concerns regarding the potential threat posed by quantum computers to current cryptographic algorithms, investigating and developing post-quantum secured crypto-wallets will be of paramount importance in ensuring the long-term security of digital assets.
%     }

% Our research provides a comprehensive analysis of crypto-wallets, focusing on their design, vulnerabilities, and defense mechanisms. By categorizing wallets based on funds controlability, internet connectivity, and infrastructure, we shed light on the diverse cryptographic algorithms utilized in their construction. Our findings underscore the prevalence of attacks targeting wallet application software from the user's perspective, emphasizing the importance of selecting reputable wallet applications and utilizing a combination of offline and online wallets for enhanced security.

% Additionally, our investigation reveals that paper wallets and brain wallets offer the highest level of security for storing cryptocurrencies, albeit demanding users to manage their private keys independently. Hardware wallets strike a good balance between security and convenience but may pose usability challenges for some users. Meanwhile, smart contract wallets offer ease of use, but they are susceptible to coding errors and vulnerabilities. Furthermore, the emerging significance of semi-custodial wallets requires further exploration to develop specific defense mechanisms tailored to their architecture. In light of potential quantum computing threats, we advocate researching and developing post-quantum secured crypto-wallets to ensure the long-term security of digital assets. To fortify the digital vaults of crypto-wallets comprehensively, future research should address vulnerabilities unique to iOS-based wallet applications and conduct thorough security assessments of key recovery mechanisms. By advancing our understanding of wallet security and developing targeted defense measures, we can contribute to a more secure and resilient cryptocurrency ecosystem, instilling confidence among users and promoting wider adoption in the decentralized financial landscape.