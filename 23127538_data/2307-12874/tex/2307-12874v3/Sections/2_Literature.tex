\section{Related Works}
\label{sec:related-work}

Over the years, various research studies are conducted to investigate diverse aspects of crypto-wallets. These studies examine general and specific aspects of the subject, highlighting the multidimensional nature of research in this domain.

Bonneau et al. \cite{bonneau2015sok} perform an extensive examination of the difficulties confronted by Bitcoin and other cryptocurrencies. Their research covers a broad scope of subjects, including new payment protocols and easy-to-use key management systems. Although they discuss vulnerabilities and security measures, their primary concentration is on wallet applications pertaining to key management. Our present work builds on Bonneau et al.'s study by condensing the current vulnerabilities linked with wallets and offering related countermeasures. Our research strives to illuminate the classification of crypto-wallets, security and privacy issues like deanonymization attacks, and defense mechanism.

These studies can be broadly classified into three categories: key management, wallet design, and wallet security.

\subsection{Key Management}
\label{sec:key-management}

Crypto-wallets are applications that facilitate the generation of transactions to send, receive, and store cryptographic keys \cite{di2020characteristics}. The security of a user's assets depends on the ability to safeguard the private key stored on the wallet. Therefore, key management is an essential aspect of wallets and is the focus of several studies in the literature \cite{mangipudi2022uncovering, he2018social, pal2021key, he2019novel, courtois2017stealth, Belchior2022, chen2020survey, suratkar2020cryptocurrency}.

Key management methods can vary, and keys can be managed on a range of digital and physical applications, and may be stored on multiple devices. Mangipudi et al. \cite{mangipudi2022uncovering} discuss key management from the perspective of devices, grouping wallets into single-device and multi-device wallets, based on their ability to distribute the risk associated with private key compromise across multiple devices. Pal et al. \cite{pal2021key} expand on this concept by discussing key management through storing the key on the cloud.

Other studies investigate key management from the perspective of a single asset \cite{eskandari2018first}, hardware devices known as Hardware Security Modules \cite{gotte2021tech, shbair2021hsm}, and innovative methods of key recovery \cite{he2019novel}. These studies highlight the importance of key management in crypto-wallets and the various methods that can be used to secure private keys.


\subsection{Wallet Design}
\label{sec:wallet-design}

Several studies focus on the design of crypto-wallets, with a specific focus on wallet architecture. Khadzhi et al. \cite{khadzhi2020method} analyze transaction activity on hardware wallets, while Khan et al. \cite{8966739} propose an architectural design combination of wallets by utilizing QR codes to authenticate transactions between hardware and software wallets. Eyal \cite{eyal2022cryptocurrency} provides unique insights on wallet design by evaluating the failure probability of wallets from a security perspective. Other studies investigate the design of Ethereum wallets \cite{di2020characteristics, homoliak2018air}.

One unique architectural concept identified in the literature is the main and sub-wallet setup, which can be likened to a savings and a current account for blockchain wallets. Rezaeighaleh and Zou \cite{rezaeighaleh2019deterministic} discuss the concept of sub-wallets and main-wallets in deterministic wallets and implement a prototype in a hardware wallet. The mechanism employed in this study is found to be invulnerable to classical wallet attacks, such as \acf{mitm} attacks, where a hacker replaces a malicious address with a user's address.


\subsection{Wallet Security}
\label{sec:wallet-security}

Various studies have rigorously analyzed the overall security and vulnerability of blockchain systems \cite{chen2020survey, li2020survey, guo2022survey, zamani2020security, Urien2021}. Chen et al. \cite{chen2020survey} provide a security analysis of the Ethereum blockchain by investigating the vulnerabilities, attacks, and defense mechanisms. This study also highlights vulnerabilities and attacks on each blockchain layer.

Building on the foundation established by earlier studies, other researchers have examined different facets of wallet security. Homoliak et al. \cite{homoliak2018air} analyze attacks on crypto-wallet applications and propose defense mechanisms to limit wallet vulnerabilities. The study suggests the implementation of air-gapped hardware wallet solutions as a defense mechanism.


Furthermore, Bui et al. \cite{bui2019pitfalls} focus on wallet attack scenarios and defense mechanisms on the server side, specifically addressing man-in-the-middle attacks. Phishing attacks, another type of wallet attack, are discussed in another study with proposed prevention mechanisms \cite{andryukhin2019phishing}. 

\begin{table}[ht]
	\centering
	\caption{Overview of related works. KC - Key Cryptography, KM - Key Mechanism, KR - Key Recovery, WA - Wallet Attack, WS - Wallet Security, WP - Wallet Privacy, DLT - Distributed, MECH - Mechanical, Ledger Technology, LIT - Literature, TAX - Taxonomization, ANAL - Analysis, SC - Smart contract. (\CIRCLE: include, \Circle: not include)}
	\label{Literature-Gap-Table-1}
        \renewcommand{\arraystretch}{1.3}
	\resizebox{\linewidth}{!}{%
		\begin{tabular}{|c|c|ccc|ccc|cc|ccc|}
		\hline
		\multirow{2}{*}{Cat.} & \multirow{2}{*}{Ref.} & \multicolumn{3}{c|}{Cryptography} & \multicolumn{3}{c|}{Security} & \multicolumn{2}{c|}{Virtual Machine} & \multicolumn{3}{c|}{Methodology} \\ \cline{3-13} 
		& & \multicolumn{1}{c|}{KC} & \multicolumn{1}{c|}{KM} & \multicolumn{1}{c|}{KR} & \multicolumn{1}{c|}{WA} & \multicolumn{1}{c|}{WS} & \multicolumn{1}{c|}{WP} & \multicolumn{1}{c|}{DLT} & \multicolumn{1}{c|}{MECH} & \multicolumn{1}{c|}{LIT} & \multicolumn{1}{c|}{TAX} & ANAL \\ \hline
        \multicolumn{2}{|c|}{This Study} & \CIRCLE & \CIRCLE & \CIRCLE & \CIRCLE & \CIRCLE & \CIRCLE & \CIRCLE & \CIRCLE & \CIRCLE & \CIRCLE & \CIRCLE \\ \hline
        % Software
		\multirow{9}{*}{\rot[90]{Software}} & \cite{suratkar2020cryptocurrency}    & \CIRCLE & \CIRCLE & \CIRCLE & \Circle & \Circle & \Circle & \Circle & \Circle & \CIRCLE & \CIRCLE & \Circle \\
		& \cite{bui2019pitfalls}               & \Circle & \Circle & \Circle & \CIRCLE & \CIRCLE & \Circle & \Circle & \Circle & \CIRCLE & \CIRCLE & \CIRCLE \\
		& \cite{ur2019trust}                   & \Circle & \Circle & \Circle & \Circle & \CIRCLE & \CIRCLE & \CIRCLE & \CIRCLE & \CIRCLE & \CIRCLE & \CIRCLE \\
		& \cite{zaghloul2020bitcoin}           & \Circle & \Circle & \Circle & \Circle & \CIRCLE & \CIRCLE & \Circle & \CIRCLE & \CIRCLE & \CIRCLE & \CIRCLE \\
		& \cite{cryptoeprint:2020/868}         & \Circle & \Circle & \Circle & \Circle & \CIRCLE & \CIRCLE & \Circle & \CIRCLE & \CIRCLE & \Circle & \CIRCLE \\
		& \cite{li2020android}                 & \Circle & \Circle & \Circle & \CIRCLE & \CIRCLE & \Circle & \Circle & \Circle & \CIRCLE & \Circle & \CIRCLE \\
		& \cite{benli2017biowallet}            & \Circle & \Circle & \Circle & \Circle & \CIRCLE & \Circle & \Circle & \Circle & \CIRCLE & \Circle & \CIRCLE \\
		& \cite{dai2018sblwt}                  & \CIRCLE & \CIRCLE & \Circle & \CIRCLE & \CIRCLE & \Circle & \CIRCLE & \CIRCLE & \Circle & \CIRCLE & \CIRCLE \\
		& \cite{volety2019cracking}            & \Circle & \Circle & \Circle & \CIRCLE & \CIRCLE & \Circle & \Circle & \Circle & \CIRCLE & \Circle & \CIRCLE \\ \hline
        % Hardware
		\multirow{6}{*}{\rot[90]{Hardware}} & \cite{8966739}                       & \CIRCLE & \CIRCLE & \CIRCLE & \CIRCLE & \CIRCLE & \CIRCLE & \Circle & \Circle & \CIRCLE & \Circle & \CIRCLE \\
		& \cite{rezaeighaleh2019deterministic} & \CIRCLE & \CIRCLE & \CIRCLE & \Circle & \CIRCLE & \Circle & \Circle & \CIRCLE & \CIRCLE & \CIRCLE & \CIRCLE \\
		& \cite{Urien2021}                     & \Circle & \Circle & \Circle & \CIRCLE & \CIRCLE & \Circle & \Circle & \Circle & \CIRCLE & \CIRCLE & \Circle \\
		& \cite{rezaeighaleh2020multilayered}  & \Circle & \Circle & \CIRCLE & \Circle & \Circle & \CIRCLE & \Circle & \Circle & \CIRCLE & \CIRCLE & \Circle \\
		& \cite{rezaeighaleh2019new}           & \CIRCLE & \CIRCLE & \CIRCLE & \Circle & \CIRCLE & \Circle & \Circle & \CIRCLE & \CIRCLE & \Circle & \CIRCLE \\
		& \cite{rezaeighaleh2020improving}     & \CIRCLE & \CIRCLE & \CIRCLE & \CIRCLE & \CIRCLE & \CIRCLE & \Circle & \CIRCLE & \CIRCLE & \CIRCLE & \CIRCLE \\ \hline
        % smart contract 
		\multirow{2}{*}{\rot[90]{SC}} & \cite{di2020characteristics}         & \Circle & \Circle & \Circle & \Circle & \Circle & \Circle & \Circle & \Circle & \CIRCLE & \CIRCLE & \CIRCLE \\
		& \cite{homoliak2018air}               & \CIRCLE & \CIRCLE & \CIRCLE & \Circle & \CIRCLE & \CIRCLE & \Circle & \Circle & \CIRCLE & \CIRCLE & \CIRCLE \\ \hline
        % key management
		\multirow{8}{*}{\rot[90]{Key Management}} & \cite{pal2021key}                    & \Circle & \CIRCLE & \Circle & \Circle & \CIRCLE & \Circle & \CIRCLE & \Circle & \Circle & \Circle & \Circle \\
		& \cite{he2019novel}                   & \CIRCLE & \Circle & \CIRCLE & \CIRCLE & \CIRCLE & \Circle & \Circle & \Circle & \CIRCLE & \Circle & \CIRCLE \\
		& \cite{courtois2017stealth}           & \CIRCLE & \CIRCLE & \Circle & \Circle & \CIRCLE & \CIRCLE & \Circle & \CIRCLE & \CIRCLE & \Circle & \Circle \\
		& \cite{eskandari2018first}            & \Circle & \CIRCLE & \CIRCLE & \Circle & \CIRCLE & \Circle & \Circle & \Circle & \CIRCLE & \CIRCLE & \Circle \\
		& \cite{eyal2022cryptocurrency}        & \Circle & \CIRCLE & \Circle & \Circle & \Circle & \Circle & \Circle & \Circle & \CIRCLE & \Circle & \CIRCLE \\
		& \cite{banupriya2021privacy}          & \CIRCLE & \CIRCLE & \Circle & \Circle & \CIRCLE & \CIRCLE & \Circle & \CIRCLE & \CIRCLE & \Circle & \CIRCLE \\
		& \cite{DiLuzio2020Arcula:Blockchains} & \CIRCLE & \CIRCLE & \CIRCLE & \CIRCLE & \CIRCLE & \Circle & \Circle & \Circle & \CIRCLE & \Circle & \CIRCLE \\
		& \cite{10.1145/3319535.3354236}       & \CIRCLE & \CIRCLE & \CIRCLE & \CIRCLE & \CIRCLE & \Circle & \Circle & \Circle & \CIRCLE & \CIRCLE & \CIRCLE \\ \hline
        % Blockchain
		\multirow{4}{*}{\rot[90]{Blockchain}} & \cite{he2018social}                  & \CIRCLE & \CIRCLE & \Circle & \CIRCLE & \CIRCLE & \Circle & \Circle & \CIRCLE & \CIRCLE & \CIRCLE & \CIRCLE \\
		& \cite{ferdous2020blockchain}         & \Circle & \Circle & \Circle & \Circle & \CIRCLE & \Circle & \CIRCLE & \Circle & \CIRCLE & \CIRCLE & \Circle \\
		& \cite{nelaturu2022review}            & \CIRCLE & \Circle & \Circle & \Circle & \CIRCLE & \CIRCLE & \CIRCLE & \CIRCLE & \CIRCLE & \CIRCLE & \CIRCLE \\
		& \cite{paik2019analysis}              & \CIRCLE & \CIRCLE & \Circle & \Circle & \CIRCLE & \Circle & \Circle & \Circle & \CIRCLE & \CIRCLE & \Circle \\ \hline
        % Others
		\multirow{5}{*}{\rot[90]{Others}} & \cite{mangipudi2022uncovering}       & \CIRCLE & \CIRCLE & \CIRCLE & \Circle & \CIRCLE & \CIRCLE & \Circle & \CIRCLE & \CIRCLE & \CIRCLE & \CIRCLE \\
		& \cite{caldarelli2021wrapping}        & \Circle & \Circle & \Circle & \Circle & \Circle & \Circle & \Circle & \Circle & \CIRCLE & \CIRCLE & \Circle \\
		& \cite{vanstone1997elliptic}          & \CIRCLE & \Circle & \Circle & \Circle & \CIRCLE & \CIRCLE & \Circle & \Circle & \CIRCLE & \Circle & \Circle \\
		& \cite{albayati2021study}             & \Circle & \Circle & \Circle & \Circle & \Circle & \Circle & \Circle & \Circle & \CIRCLE & \Circle & \CIRCLE \\
		& \cite{rajasekaran2022comprehensive}  & \Circle & \Circle & \Circle & \Circle & \Circle & \CIRCLE & \CIRCLE & \Circle & \Circle & \Circle & \Circle \\
			\hline
		\end{tabular}%
	}
\end{table}


\subsection{Addressing Literature Gaps}
\label{sec:gaps-in-literature}

This study bridges the gaps in current literature by meticulously examining the internal mechanisms of various wallet types, while conducting an in-depth analysis of different attack strategies and defense implementations, thus accommodating the discrepancies among different wallet categories. Additionally, as shown in \autoref{Literature-Gap-Table-1}, this study covers a wider range of subjects compared to existing literature, including key cryptography, key management, wallet attack, and wallet security. By providing a comprehensive examination of these subjects, this study aims to provide a more in-depth understanding of the various aspects of crypto-wallets and how they can be designed and implemented to provide secure and efficient solutions for managing private keys and facilitating transactions.
The comparison table further emphasizes the significance of this research by providing a clear summary of the key findings and the unique contributions of this study.
