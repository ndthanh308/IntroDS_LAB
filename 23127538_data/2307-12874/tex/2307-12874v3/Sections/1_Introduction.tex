\section{Introduction}
\label{sec:introduction}

\old{
    The widespread adoption of cryptocurrencies has ushered in a paradigm shift in the global financial system, enabling decentralized peer-to-peer transactions. With the growing acceptance and usage of cryptocurrencies, the importance of secure crypto-wallets has become increasingly paramount. A cryptocurrency wallet plays a critical role in facilitating transaction management on the blockchain and securely storing private keys associated with digital assets. However, the security of wallets remains a significant concern, with reported losses exceeding \$16.7 billion since 2011 \cite{CrystalBlockchain}. The wide array of wallet types available, each with its intricate specifications, poses a significant challenge for users in selecting a secure and suitable option. This challenge arises from a lack of comprehensive information regarding the vulnerabilities inherent in these applications, hindering users' ability to make informed decisions. The presence of a vulnerability within a wallet can result in the partial or complete loss of the monetary value associated with the tokens it holds.

	When the media reports on wallet breaches and the subsequent loss of assets, the impact is not limited to individual users but extends to a larger group of users who rely on the same exchange service. Media coverage often emphasizes the institutional repercussions on the exchange, overshadowing the individual users affected. A notable example illustrating this phenomenon is the infamous attack on the Mt. Gox exchange, which resulted in the theft of approximately 850,000 Bitcoins \cite{mtgox}. However, Mt. Gox is not an isolated case, as other exchanges like NiceHash, YouBit, and Coincheck have also experienced similar incidents \cite{castonguay2020digital, mccorry2018preventing}. These exchanges remain attractive targets for attackers due to the potential for substantial illicit gains \cite{feder2017impact}.

	Crypto-wallets, with their complex and multifaceted architecture, surpass the functionality of applications focused solely on credential and key management, such as password managers. This complexity provides fertile ground for potential attackers. In addition to storing login credentials, crypto-wallets serve as banking applications, adding further layers of complexity. Another enticing aspect for malicious actors is the wallets' commitment to preserving user anonymity. Thus, it is essential to analyze crypto-wallets in isolation to uncover all potential vulnerabilities comprehensively. To achieve this, we commence by reviewing relevant literature on key management, wallet design, wallet security, and addressing gaps in the field \autoref{sec:related-work}. Next, we delve into the necessary cryptography primitives for crypto-wallets in the context of their underlying infrastructure \autoref{sec:wallet-infrastructure}. Subsequently, we present a detailed taxonomy of different types of crypto-wallets to facilitate a clearer understanding of their unique characteristics \autoref{sec:wallet-taxonomy}. The core focus of our research lies in the security analysis of crypto-wallets, where we explore common vulnerabilities and attack vectors from both theoretical and practical perspectives \autoref{sec:security-analysis}. Furthermore, we provide an overview of existing defenses and mitigation strategies to combat these threats \autoref{sec:defense}. \old{Finally, we discuss recent wallet developments and propose potential directions for future investigation \autoref{sec:discussion}.}
 }

Blockchain, the distributed ledger technology that enables transparent and immutable recording of transactions without a central authority, gives rise to \acf{defi}, an ecosystem of financial applications aimed at replacing traditional intermediaries like banks and brokers. By utilizing self-executing smart contracts on blockchain networks like Ethereum, \ac{defi} protocols allow for decentralized lending, trading, derivatives, and more peer-to-peer financial activities. The composability, accessibility, transparency, and censorship-resistance of \ac{defi} attract billions in crypto asset investments, pointing to the disruptive potential of non-hierarchical finance.

Crypto-wallets are software programs or hardware devices that allow users to manage, receive, and transmit digital assets on a blockchain network. They contain public and private keys which respectively resemble a bank account number and a PIN code to access the assets. Wallets interact with the blockchain to monitor balances and broadcast valid transactions using the keys. With the rise of \ac{defi}, wallets become gateways to participate in activities like lending, borrowing, trading, and more on platforms like Uniswap and Compound. They are essential tools to hold the keys that help us access blockchain networks and permissionlessly transact value or utilize \ac{defi} functionalities enabled by smart contracts.

The importance of robust wallet security cannot be overstated, as reported losses in the cryptocurrency space have exceeded a staggering \$16.7 billion since 2011 \cite{CrystalBlockchain}. This alarming statistic underscores the seriousness of the wallet security problem. Incidents of wallet breaches and cyberattacks on exchanges, such as the Mt. Gox attack that resulted in the theft of approximately 850,000 Bitcoins \cite{mtgox}, have severe consequences, affecting not only individual users but also communities relying on the exchange service. Examples like NiceHash, YouBit, and Coincheck demonstrate the attractiveness of exchanges as lucrative targets for attackers \cite{castonguay2020digital, mccorry2018preventing, feder2017impact}.

The immutability of transactions on public blockchain networks poses severe security risks for crypto-wallets. Once transactions are validated and added to the chain, it is cryptographically infeasible to reverse any accidental or malicious transfers from a wallet. Private keys, in particular, must be heavily safeguarded, as their loss or theft enables full control and depletion of a wallet's funds irreversibly. Sophisticated hackers are also constantly developing strategies like phishing sites, keyloggers, and clipboard hijacking to steal keys. The complexity of key generation, wallet software vulnerabilities, user carelessness, and lack of recourse after theft provide fertile ground for security breaches. While solutions like multi-signature wallets, offline cold storage, and decentralized custody services help, cryptocurrency wallet security remains a constant challenge without centralized oversight. The irreversible nature of errors and thefts on public blockchains compounds the risks.

This paper provides a holistic analysis of cryptocurrency wallet security, spanning vulnerabilities, attacks, and defense strategies across wallet types, architectures, and blockchain interactions. The key contributions include constructing a taxonomy of wallets based on custody, connectivity, and infrastructure; systematically evaluating network, application, blockchain, and authentication threats; analyzing attacks like phishing, side-channels, and re-entrancy; reviewing hashing, encryption, multi-signature, air-gapped, and biometric defense mechanisms; and compiling comparative tables of algorithms, wallets, and exchanges. By consolidating this multidimensional information, the research significantly advances the understanding of wallet security challenges and equips developers, researchers, and users with actionable knowledge to design, evaluate, and utilize wallets more securely. The comprehensive analysis and insights contribute to building robustness and trust in the cryptocurrency ecosystem.

The paper proceeds as follows, \autoref{sec:related-work} commences by reviewing relevant literature on key management, wallet design, wallet security, and addressing gaps in the field. Next, we present a detailed taxonomy of different types of crypto-wallets to facilitate a clearer understanding of their unique characteristics in \autoref{sec:wallet-taxonomy}. Subsequently, we delve into exploring common vulnerabilities and attack vectors from both theoretical and practical perspectives in \autoref{sec:security-analysis}. Furthermore, we provide an overview of existing defenses and mitigation strategies to combat these threats in \autoref{sec:defense}. Finally, we discuss proposing potential directions for future investigation in \autoref{sec:discussion} and the conclude in \autoref{sec:conclusion}.
