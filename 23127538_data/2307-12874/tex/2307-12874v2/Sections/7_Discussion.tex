\section{Future Works}
\label{sec:discussion}

Several critical avenues merit exploration to bolster the security of crypto-wallets. Firstly, addressing the existing gap in research concerning vulnerabilities specific to iOS-based wallet applications is paramount. As the majority of the literature focuses on Android-based applications, a comprehensive analysis of iOS wallets is imperative to uncover potential security risks and weaknesses unique to the iOS environment. This research should encompass in-depth investigations into cryptographic algorithms, key management practices, and iOS-specific attack vectors. By understanding and mitigating vulnerabilities in iOS wallets, researchers can contribute to the development of targeted defense mechanisms to ensure a robust and secure experience for cryptocurrency users on both Android and iOS platforms.

Secondly, conducting thorough security assessments of key recovery mechanisms in various types of crypto-wallets is crucial. While these mechanisms are designed to provide users with backup options in case of lost or compromised private keys, they may also introduce potential security vulnerabilities. Researchers should examine the effectiveness and resilience of key recovery processes, identifying any potential weak points or exploitable loopholes that could compromise the security of users' digital assets. By improving and fortifying key recovery mechanisms, developers can offer users enhanced peace of mind and protection against the risk of irreversible asset loss, fostering greater confidence in the usage and adoption of crypto-wallets.