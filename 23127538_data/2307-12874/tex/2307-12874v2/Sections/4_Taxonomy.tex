\section{Wallet Taxonomy}
\label{sec:wallet-taxonomy}

Cryptocurrency wallets can be classified into several categories based on various characteristics, such as the storage location of private keys, the connection to the internet, and the underlying technology \cite{suratkar2020cryptocurrency}. However, in practice, these categories are often combined to create more nuanced wallet solutions, as shown in \autoref{fig:wallet-taxonomy}. One common approach involves the combination of a hot wallet and an online wallet, enabling users to have quick and convenient access to their funds for frequent transactions. Conversely, a cold wallet, renowned for its heightened security, can be further fortified by utilizing a hardware wallet, which stores private keys offline, thus adding an extra layer of protection against potential hacking attempts. Furthermore, a hybrid wallet can be formed by integrating functionalities from different types of wallets, effectively amalgamating the accessibility of an online wallet with the robust security features of a hardware wallet. These amalgamations not only broaden the range of options available to users but also empower them to tailor their crypto-wallets to suit their specific needs, striking an optimal balance between accessibility and security \cite{biernacki2021comparative}.

% Figure environment removed

% \subsection{Classification of Wallets Based on the Level of Control and Responsibility of Funds}
\subsection{Funds Controlability}
\label{sec:funds-controlability}

\subsubsection{Custodial Wallets}
\label{sec:custodial-wallets}
These are hosted and managed by a third-party service provider that retains control over users' private keys and funds. Users must trust the provider to securely store assets and execute transactions on their behalf. These wallets prioritize convenience and usability, offering features like customer support and insurance, but they introduce counterparty risk as users relinquish sole control of their cryptocurrency. The centralized management of keys and lack of user self-custody conflict with the decentralized nature of cryptocurrencies. However, custodial services can implement institutional-grade security practices such as cold storage and encryption that may exceed typical individual user capabilities \cite{chalkias2022broken}. Despite the implementation of institutional-grade security measures, it's essential to recognize that certain risks such as \ac{mitm} attack (\autoref{sec:mitm}), \ac{dns} (\autoref{sec:dns}), \acf{dos} (\autoref{sec:dos}), social engineering (\autoref{sec:social-engineering}), and brute force (\autoref{sec:brute-force}) persists. To evaluate the risks and benefits of utilizing a custodial wallet provider, it is essential to conduct proper due diligence regarding governance, audits, and insurance coverage.

\subsubsection{Non-Custodial Wallets}
\label{sec:non-custodial-wallets}
These provide users with exclusive access and control over their private keys, aligning with cryptocurrency principles of autonomy and self-sovereignty. Users independently manage keys without relying on any intermediary. This approach places greater responsibility on users to adopt robust security practices to safeguard keys and assets. Non-custodial wallets empower users to fully control their cryptocurrency, interact with decentralized networks, and avoid counterparty risks. However, poor key management by individuals can lead to irreversible loss or theft. Moreover, they are susceptible to to phishing (see \autoref{sec:phishing}), malware (\autoref{sec:malware}), dictionary attacks (\autoref{sec:dictionary}) and network attacks (\autoref{sec:network}) such as deanonymization (\autoref{sec:deanonymization}), \ac{mitm} (\autoref{sec:mitm}), \ac{dos} (\autoref{sec:dos}). Best practices include careful generation and storage of keys, multi-signature configurations, and education on threats like phishing. With proper precautions, non-custodial wallets enable users to fully benefit from the decentralization and independence of cryptocurrency \cite{bowler2023non}.


% \subsection{Classification of Wallets Based on Connection to Internet and Storage of Private keys}
\subsection{Internet Connectivity}
\label{sec:internet-connectivity}

\subsubsection{Hot Wallets}
\label{sec:hot-wallets}
Internet-connected wallets are used for online transactions, offering convenient access from devices like computers, smartphones, or tablets, allowing users to manage their digital assets in real-time. However, being constantly connected to the internet makes them more vulnerable to network (\autoref{sec:network}) and application attacks (\autoref{sec:application}). These are commonly used for everyday transactions, provide quick and seamless fund access. To mitigate risks, users must implement strong security measures such as using strong passwords, enabling two-factor authentication, and regularly updating their software which are exhaustively covered in \autoref{sec:defense} \cite{khanum2022exposure}.

\subsubsection{Cold Wallets}
\label{sec:cold-wallets}
Offline wallets prioritize the security of users' digital assets by keeping private keys and sensitive information isolated from online threats. These can be hardware devices or physical mediums like specialized hardware wallets, paper wallets, or offline storage solutions. Storing private keys offline significantly reduces the risk of unauthorized access, hacking, and malware attacks. While these require additional steps for transactions, they are preferred by users who prioritize long-term storage and security of their cryptocurrency holdings. They are ideal for storing large amounts of cryptocurrencies or for a \quotes{store and hold} investment strategy, ensuring the protection of valuable digital assets \cite{8966739}. Amidst the advantages of cold wallets, they are susceptible to attacks such as \ac{mitm} (\autoref{sec:mitm}), social engineering (\autoref{sec:social-engineering}) and side channel (\autoref{sec:side-channel}).


% \subsection{Classification of Wallets Based on Infrastructure}
\subsection{Infrastructure}
\label{sec:infrastructure}

\subsubsection{Software Wallets}
\label{sec:software-wallets}
They are digital applications that store private keys and allow users to manage their assets \cite{suratkar2020cryptocurrency}. They can be classified further based on the type of device they are installed on into the following categories:

\paragraphtitle{Web Wallets}
\label{sec:web-wallets}
Accessed through a web browser, these wallets are convenient to use as they can be accessed from any device with an internet connection.

\paragraphtitle{Mobile Wallets}
\label{sec:mobile-wallets}
These are software applications that users can install on their smartphones, providing convenience and accessibility from anywhere. However, they are considered less secure than desktop wallets.

\paragraphtitle{Desktop Wallets}
\label{sec:desktop-wallets}
Installed on a computer, these software programs are only accessible from the device on which they are installed, making them more secure than web and mobile wallets. However, they are vulnerable to malware and hackers, necessitating regular updates with the latest security patches.

\paragraphtitle{Bot Wallets}
\label{sec:bot-wallets}
These wallets are integrated into chatting applications, such as the Telegram Wallet Bot, revolutionizing in-app payments by seamlessly enabling users to send and receive cryptocurrency transactions. Telegram empowers its users with convenient and accessible cryptocurrency transactions, supporting currencies like Bitcoin, Tether, and TON, eliminating the need for third-party gateways.

\subsubsection{Hardware Wallets}
\label{sec:hardware-wallets}
They are physical devices that store private keys and allow users to manage their cryptocurrency assets, offering advanced security measures, such as a secure element and PIN protection \cite{10.1145/3319535.3354236}.

\paragraphtitle{Smart Card}
\label{sec:smart-card}
These wallets leverage the secure micro-controller and cryptographic capabilities of card technology to store private keys and perform cryptographic operations. User authentication is often required through PIN codes or biometric verification.

\paragraphtitle{USB-based Wallets}
\label{sec:usb-wallets}
These wallets are small devices that connect to a computer via a USB port. They feature a secure micro-controller that stores the private keys and facilitates secure transactions. USB-based hardware wallets often come with a built-in screen and buttons for user interaction, ensuring secure verification of transactions.

\paragraphtitle{\ac{nfc}-enabled Wallets}
\label{sec:nfc-wallets}
These wallets allow users to securely manage their digital assets using a mobile device. They leverage \ac{nfc} to establish a secure connection with a mobile device for initiating transactions and verifying information, providing an additional layer of convenience and portability for users.

\paragraphtitle{Air-gapped Wallets}
\label{sec:air-gapped-wallets}
These wallets are designed to operate completely offline, disconnected from any network or internet connectivity. They prioritize security by isolating the private keys from potential online threats. Offline hardware wallets typically involve generating private keys on an offline device and signing transactions offline for enhanced protection against hacking attempts.

% \cite{JFJM}
\subsubsection{Paper Wallets}
\label{sec:paper-wallets}
These involve printing out the private keys and storing them in a secure location, offering advanced security measures such as offline storage \cite{GkaniatsouAndrianaandArapinis2017}. However, they may be more difficult to use than other types of wallets.

\subsubsection{Brain Wallets}
\label{sec:brain-wallets}
These require users to memorize a seed phrase to store cryptocurrency. If the seed phrase is forgotten or in case of permanent incapacitation, the cryptocurrency becomes irretrievably lost. While not recommended due to the fallibility of human memory, brain wallets can be valuable in exceptional circumstances, such as fleeing as a refugee with minimal possessions \cite{vasek2017bitcoin}.

\subsubsection{Smart Contract Wallets}
\label{sec:smart-contract-wallets}
These are non-custodial web3 wallets that employ smart contracts to manage assets, offering enhanced security and advanced customization features. These wallets enable functionalities such as recoverable wallets, signless transactions, and batched transactions, which are not possible with traditional crypto-wallets \cite{di2020wallet}.

\subsection{Wallet Features}
\label{sec:wallet-features}

\subsubsection{M-of-N Standard Transactions}
\label{sec:bip-11}
BIP-11 \cite{bip11} proposes the use of the OP\_CHECKMULTISIG opcode to enable multi-signature transactions requiring M-of-N signatures for validity. This provides increased security for use cases like escrow services by making forgeries more complex. However, BIP-11 remains reliant on the underlying security of ECDSA signatures used in Bitcoin. It also remains vulnerable to key leakage, so signers must protect keys. Additionally, scriptSig malleability allows signatures to be changed without full invalidation. BIP-11's linear scaling of signatures increases costs and latency. The initial limit of 3 signatures was reasonable based on client limitations at the time, but restricted flexibility. While BIP-11 straightforwardly achieves its primary goal of enabling multi-signature transactions, enhancements to signature schemes, opcodes, and replay protection could provide better properties.

\subsubsection{Hierarchical Deterministic (HD) Wallets}
\label{sec:bip-32}
BIP-32 \cite{bip32} proposes hierarchical deterministic wallets using parent and child key derivation based on the hardness of the elliptic curve discrete logarithm problem and infeasibility of reversing HMAC-SHA512. Additional entropy from the 256-bit chain code improves security beyond key derivation alone. Hardened keys prevent parent discovery even given child private keys. Validity of public keys is ensured through proper serialization encoding. Restricting maximum child key depth limits blockchain growth. The gap limit and account discovery help prevent leakage from unchecked derivation. Together, these mechanisms provide robust hierarchical key generation with flexible structures while maintaining interoperability. However, protection of the root seed and leakage risks remain a concern.

\subsubsection{Mnemonic Code for HD Wallets}
\label{sec:bip-39}
BIP-39 \cite{bip39} presents a standardized approach for human-readable mnemonic recovery phrases based on cryptographic entropy, checksums to enforce validity, and carefully constructed word lists. Password salting and iteration with PBKDF2 increase brute force resistance compared to raw entropy. However, risks from poor user practices remain, as mnemonic extension attacks are still possible. BIP-39 successfully balances memorability, transcribability, and randomness to achieve user-friendly, secure seed backup phrases.

\subsubsection{Multi-Account Hierarchy for Deterministic Wallets}
\label{sec:bip-44}
BIP-44 \cite{bip44} utilizes the hardened hierarchical derivation properties of BIP-32 to securely separate accounts and prevent leakage between them. Additional mechanisms like account discovery, gap limits, and child key fingerprints optimize the structure further. However, the root seed protection and leakage issues inherent in BIP-32 remain. While BIP-44 successfully achieves structured, selective sharing across multiple accounts, risks from poor user practices persist.

\subsubsection{Account Abstraction Using Alt Mempool}
\label{sec:erc-4337}
ERC-4337 \cite{erc4337} introduces account abstraction by using a layer above the Ethereum consensus protocol to avoid changes. The two-phase validate and execute model improves efficiency and prevents fee-griefing attacks. Simulation and opcode banning prevent differences between validation and execution. Flexible nonces support custom schemes compared to simple sequences. Requiring bonds and reputation for paymasters prevents abuse. It achieves the goal of Ethereum account abstraction with improved security properties compared to raw transactions while avoiding consensus changes. However, key management and seed protection remain a user responsibility.
