
Graph reduction through maximal weighted matching has two main drawbacks within the GNN framework. First, a maximal matching may produce many vertices not adjacent to the set of selected edges. Such vertices are just copied to the next level which induce a low  decimation ratio (lower than $50\%$). Given that, the number of layers of a GNN is usually fixed, this last drawback may produce a graph with an insufficient level of abstraction at the final layer of the GNN. Secondly, only the score of the selected edges are used to compute the reduced attributes. This last point reduces the number of scores used for the back-propagation  and hence the quality of the learned similarity measures. 

As in the previous section, let us denote by $\mathcal{J}^{(l)}$ the maximal weighted matching defined at layer $l$. By definition of a maximal weighted matching, each vertex not incident to $\mathcal{J}^{(l)}$ is adjacent to at least one vertex which is incident to $\mathcal{J}^{(l)}$. Following~\cite{haxhimusa2007structurally}, we increase the decimation ratio, by attaching isolated vertices to contracted ones. This operation is performed by selecting for each isolated vertex $u$ the edge $e_{uv}$ such that $s_{uv}$ is maximal and $v$ is incident to $\mathcal{J}^{(l)}$. 

This operation provides a spanning forest of $\graph{l}$ composed of isolated edges, trees of depth one (called stars) with one central vertex and paths of length 3. This last type of tree corresponds to a sequence of $4$ vertices with strong similarities between any pair of adjacent vertices along the paths. However, merging all $4$ vertices into a single one, suppose implicitly to apply twice an hypothesis on the transitivity of our similarity measure: more precisely the fact that the two extremities of the paths are similar is not explicitly encoded by our selection of edges. In order to avoid such assumption we remove the central edge of such paths from the selection in order to obtain two isolated edges (see Figures~\ref{fig:MIS_d} to~\ref{fig:MIS_f}).

Let us denote by $\mathcal{J'}^{(l)}$ the resulting set of selected edges which forms a spanning forest of $\graph{l}$ composed of isolated edges and stars.  Concerning the definition of $\matrice{S}{l}$, we apply the same procedure than in the previous section for isolated edges. For stars, we select the central vertex as the surviving vertex. Let us denote by $u$ such a star's center. We then set $\matrice{S}{l}_{uu}=\frac{1}{2}$ and $\matrice{S}{l}_{vu}=\frac{1}{2M}s_{uv}$ for any $v$ such that $e_{uv}\in \mathcal{J'}^{(l)}$ where $M$ is a normalizing factor defined as: $M=\sum_{v|e_{uv}\in \mathcal{J'}^{(l)}} s_{uv}$.  The computation of the attributes of the reduced graph using equation~\ref{eq:attributes_reduction} and matrix $\matrice{S}{l}$ is equivalent to compute for each star's center $u$, the sum, weighted by the score, of the edges' attributes (equation~\ref{eq:edge_attribute}) incident to $u$ and belonging to $\mathcal{J'}^{(l)}$:
\begin{equation}
    x^{(l+1)}_u=\frac{1}{\sum_{v |e_{uv}\in \mathcal{J'}^{l}} s_{uv}}
    \sum_{v |e_{uv}\in \mathcal{J'}^{l}} s_{uv}x_{uv}^{(l)} 
    \label{eq:attributes_MIESCut}
\end{equation}
