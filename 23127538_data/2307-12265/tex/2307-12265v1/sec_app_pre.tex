\section{Proofs for Section~\ref{sec:prelim}}


\PropImplicationSystem*
\begin{proof}
%\todo[inline]{M: todo sketch proof}
Point 1. Suppose that, for every $w\in \Wmc$, $\alpha,\beta\subseteq \Wmc$, we have: ($\mathbf{M}$-condition) $\alpha\in \Nmc_{i}(w)$ and $\alpha\subseteq\beta$ implies $\beta\in \Nmc_{i}(w)$; and ($\mathbf{Q}$-condition) $\Wmc \not \in \Nmc_{i}(w)$. This means that $\alpha \not \in \Nmc_{i}(w)$, for every $\alpha \subseteq \Wmc$, i.e., $\Nmc_{i}(w) = \emptyset$.
Thus, every condition, except for the $\mathbf{N}$-condition, is satisfied by $\Nmc_{i}$.

Point 2.
%Item $(i)$ follows from the fact that $\Nmc_{i}(w)=\emptyset$ 
%for every $w\in \Wmc$, as already stated in Point 1.
%Suppose~$(i)$ that, for every $w\in \Wmc$, $\alpha,\beta\subseteq \Wmc$, we have: ($\mathbf{M}$-condition) $\alpha\in \Nmc_{i}(w)$ and $\alpha\subseteq\beta$ implies $\beta\in \Nmc_{i}(w)$; and ($\mathbf{Q}$-condition) $\Wmc \not \in \Nmc_{i}(w)$. This means that $\alpha \not \in \Nmc_{i}(w)$, for every $\alpha \subseteq \Wmc$, hence in particular  ($\mathbf{P}$-condition) $\emptyset \not \in \Nmc_{i}(w)$.
%
Suppose~$(i)$ that, for every $w\in \Wmc$, $\alpha,\beta\subseteq \Wmc$, we have: ($\mathbf{M}$-condition) $\alpha\in \Nmc_{i}(w)$ and $\alpha\subseteq\beta$ implies $\beta\in \Nmc_{i}(w)$; and ($\mathbf{D}$-condition) $\alpha \in \Nmc_{i}(w)$ implies $\Wmc \setminus \alpha \not \in \Nmc_{i}(w)$.
%Since $\emptyset \subseteq \beta$, for every $\beta\subseteq \Wmc$, we have that
Towards a contradiction, suppose that $\emptyset \in \Nmc_{i}(w)$. Then, we have $\Wmc = \Wmc \setminus \emptyset \not \in \Nmc_{i}(w)$ as well. By contraposition, this implies in particular that $\emptyset \not \in \Nmc_{i}(w)$, a contradiction. Thus, ($\mathbf{P}$-condition) $\emptyset \not \in \Nmc_{i}(w)$

Moreover, suppose~$(ii)$ that, for every $\alpha\subseteq \Wmc$, we have: ($\mathbf{N}$-condition) $\Wmc \in \Nmc_{i}(w)$, i.e., $\Wmc \setminus \emptyset \in \Nmc_{i}(w)$; and ($\mathbf{D}$-condition) $\alpha \in \Nmc_{i}(w)$ implies $\Wmc \setminus \alpha \not \in \Nmc_{i}(w)$. By contraposition, we obtain that ($\mathbf{P}$-condition) $\emptyset \not \in  \Nmc_{i}(w)$.

Finally, suppose~$(iii)$ that, for every $w\in \Wmc$, $\alpha,\beta\subseteq \Wmc$, we have: ($\mathbf{T}$-condition) $\alpha\in \Nmc_{i}(w)$ implies $w \in \alpha$. Then, we have ($\mathbf{P}$-condition) $\emptyset \not \in \Nmc_{i}(w)$, for otherwise we would get a contradiction.

Point 3.
%By Point~1, if $(i)$ we have the $\mathbf{MQ}$-condition
%then the $\mathbf{D}$-condition is trivially satisfied. %\nb{O:added}
Suppose~$(i)$ that, for every $w\in \Wmc$, $\alpha,\beta\subseteq \Wmc$, we have: ($\mathbf{C}$-condition)
$\alpha\in \Nmc_{i}(w)$ and $\beta\in \Nmc_{i}(w)$ implies $\alpha\cap\beta\in \Nmc_{i}(w)$; and ($\mathbf{P}$-condition) $\emptyset \not \in  \Nmc_{i}(w)$. Given $\alpha \in \Nmc_{i}(w)$, suppose towards a contradiction that $\Wmc \setminus \alpha \in \Nmc_{i}(w)$. From this, we obtain that $\emptyset = \alpha \cap (\Wmc \setminus \alpha) \in \Nmc_{i}(w)$, a contradiction. Thus, we have that ($\mathbf{D}$-condition) $\alpha \in \Nmc_{i}(w)$ implies $\Wmc \setminus \alpha \not \in \Nmc_{i}(w)$.

Moreover, suppose~$(ii)$ that, for every $w\in \Wmc$, $\alpha,\beta\subseteq \Wmc$, we have: ($\mathbf{T}$-condition) $\alpha\in \Nmc_{i}(w)$ implies $w \in \alpha$. Consider $\alpha\in \Nmc_{i}(w)$ and suppose, towards a contradiction, that also $\Wmc \setminus \alpha \in \Nmc_{i}(w)$. We obtain that $w \in \alpha$ 
%\nb{O: fixed mistake, it was $w \in \Wmc$}
and $w \in \Wmc \setminus \alpha$ as well, a contradiction. Hence, we have that ($\mathbf{D}$-condition) $\alpha \in \Nmc_{i}(w)$ implies $\Wmc \setminus \alpha \not \in \Nmc_{i}(w)$.

Point 4. Straightforward, because otherwise we immediately have a contradiction.
% for every $w\in \Wmc$, $\alpha,\beta\subseteq \Wmc$,
%%
%\begin{description}
%	\item[\textnormal{\emph{$E$-condition}:}] $\Nmc_{i}$ is a neighbourhood function (always true);
%	\item[\textnormal{\emph{$M$-condition} (\emph{supplementation}):}] $\alpha\in \Nmc_{i}(w)$ and $\alpha\subseteq\beta$ implies $\beta\in \Nmc_{i}(w)$;
%	\item[\textnormal{\emph{$C$-condition} (\emph{closure under intersection}):}] $\alpha\in \Nmc_{i}(w)$ and $\beta\in \Nmc_{i}(w)$ implies $\alpha\cap\beta\in \Nmc_{i}(w)$;
%	\item[\textnormal{\emph{$N$-condition} (\emph{containment of unit}):}] $\Wmc \in \Nmc_{i}(w)$;
%	\item[\textnormal{\emph{$P$-condition}:}] $\emptyset \not \in \Nmc_{i}(w)$;
%		\item[\textnormal{\emph{$Q$-condition}:}] $\Wmc \not \in \Nmc_{i}(w)$;
%			\item[\textnormal{\emph{$D$-condition}:}]  $\alpha \in \Nmc_{i}(w)$ implies $\Wmc \setminus \alpha \not \in \Nmc_{i}(w)$;
%	\item[\textnormal{\emph{$T$-condition}:}] $\alpha \in \Nmc_{i}(w)$ implies $w \in \alpha$.
%\end{description}
\end{proof}

%\todo[inline]{M: add
%\\
%$T \to D$ \\ $T \to P$ \\ $N + Q \to \bot$ \\ $N + D \to P$ \\ $M + D \to P$ \\ $M + Q \to P$ \\ $M + Q \to \Nmc_{i}(w) = \emptyset$ \\ $C + P \to C +  D$ \\ \ldots}
%\begin{corollary}
%	The MQ-condition implies the P-condition.
%\end{corollary}
















\PropCorresp*
\begin{proof}
%\todo{M: todo fix proof}
Here we present a proof  only for $\MLALC{n}$ concept inclusions %\nb{O: concept inclusions?}
 and only for the basic principles $L \in \{ \mathbf{E, M, C, N, P, Q, D, T} \}$.
For $\MLALC{n}$ formulas, the proof is similar to the case of propositional non-normal modal logics (see e.g.~\cite{Pac}).
More complex principles  (e.g. $\mathbf{EMCN}$) %$L' \in \mathsf{Pantheon}$, 
%the corresponding results 
can be obtained by suitably combining the basic principles. %$L$.
%
%
%\emph{Point~1.}
%
%In the following, 
%Let $L \in \{ E, M, C, N, P, Q, D, T \}$ and 
Let $\Fmc = (\Wmc, \{ \Nmc_{i} \}_{i \in J})$ be a neighbourhood frame and let 
$L$ be as above.
%\nb{O: checking}
\begin{description}
	\item[\textnormal{$\mathit{L = \mathbf{E}}$.}]
%	[\textnormal{($\mathit{E}$-\emph{principle})}]
	If $C \equiv D$ is valid on $\Fmc$, then for all
	$\Mmc = (\Fmc, \Imc)$
%	$\Mmc = (\Fmc, \Delta, \Imc)$
	based on $\Fmc$, and all $w$ in $\Mmc$, we have 
$C^{\Int_w}  = D^{\Int_w}$.
Thus, for all $d \in \Delta_{w}$, $\llbracket C \rrbracket^{\Mmc}_{d} = \llbracket D \rrbracket^{\Mmc}_{d}$.
So for all $v\in\W$, $\llbracket C \rrbracket^{\Mmc}_{d} \in \Nmc_{i}(v)$ iff $\llbracket D \rrbracket^{\Mmc}_{d} \in \Nmc_{i}(v)$,
which implies $d\in (\B_{i} C)^{\Int_v}$ iff $d\in(\B_{i} D)^{\Int_v}$,
that is $(\B_{i} C)^{\Int_v} = (\B_{i} D)^{\Int_v}$.
Then, $\B_{i} C \equiv \B_{i} D$ is valid on $\Fmc$, for all $i\in J$.
%
	\item[\textnormal{$\mathit{L = \mathbf{M}}$.}]
%	($\mathit{M}$-\emph{principle})
From right to left,
assume that $\Fmc$ is supplemented and $C \sqsubseteq D$ is valid on $\Fmc$.
Then, for all
$\Mmc = (\Fmc, \Imc)$
%$\Mmc = (\Fmc, \Delta, \Imc)$
based on $\Fmc$, and all $w$ in $\Mmc$, we have 
$C^{\Int_w} \subseteq D^{\Int_w}$.
Thus, for all $d\in\Delta_{w}$, $\llbracket C \rrbracket^{\Mmc}_{d} \subseteq \llbracket D \rrbracket^{\Mmc}_{d}$.
By supplementation we have that, for all $v\in\W$, $\llbracket C \rrbracket^{\Mmc}_{d} \in \Nmc_{i}(v)$ implies $\llbracket D \rrbracket^{\Mmc}_{d} \in \Nmc_{i}(v)$.
So $d\in (\B_{i} C)^{\Int_v}$ implies $d\in(\B_{i} D)^{\Int_v}$,
that is $(\B_{i} C)^{\Int_v} \subseteq (\B_{i} D)^{\Int_v}$.
Then $\B_{i} C \sqsubseteq \B_{i} D$ is valid on $\Fmc$.
% 
For the left-to-right direction,
assume that $\Fmc$ is not supplemented. Then 
there are $w\in\Wmc$, $\alpha,\beta\subseteq\Wmc$ such that $\alpha\subseteq\beta$, $\alpha\in\Nmc_{i}(w)$ and $\beta\notin\Nmc_{i}(w)$.
We define over $\Fmc$ the model
$\Mmc= ( \Fmc, \Int )$,
%$\Mmc= ( \Fmc, \Delta, \Int )$,
where
$\Delta_{w} = \{d\}$, for every $w \in \Wmc$,
%$\Delta = \{d\}$,
and the interpretation of two concept names $A, B \in \NC$ is defined as follows:
$d\in A^{\Int_v}$ iff $v\in\alpha$, and
$d\in B^{\Int_v}$ iff $v\in\beta$
(and defined arbitrarily on all other symbols in $( \NC \cup \NR ) \setminus \{ A, B \}$).
%
As a consequence, we have that $\llbracket A \rrbracket_d^{\Mmc} = \alpha$ and $\llbracket B \rrbracket_d^{\Mmc} = \beta$,
which implies $\llbracket A \rrbracket_d^{\Mmc} = \llbracket A \rrbracket_d^{\Mmc} \cap \llbracket B \rrbracket_d^{\Mmc} = \llbracket A \sqcap B \rrbracket_d^{\Mmc}$.
%\nb{T: Do we need to show explicitly that $\llbracket A \rrbracket_d^{\Mmc} \cap \llbracket B \rrbracket_d^{\Mmc} = \llbracket A \sqcap B \rrbracket_d^{\Mmc}$? \\ M: i don't think so}
Thus $\llbracket A \sqcap B \rrbracket_d^{\Mmc}\in\Nmc_{i}(w)$ and $\llbracket B \rrbracket_d^{\Mmc}\notin \Nmc_{i}(w)$. 
By definition we have $d\in(\B_{i}(A\sqcap B))^{\Int_w}$ and $d\notin(\B_{i} B)^{\Int_w}$.
%
	\item[\textnormal{$\mathit{L = \mathbf{C}}$.}]
%	($\mathit{C}$-\emph{principle})
From right-to-left,
assume that $\Fmc$ is closed under intersection. 
Moreover, let
$\Mmc = (\Fmc, \Imc)$
%$\Mmc = (\Fmc, \Delta, \Imc)$
be a model based on $\Fmc$, with $w$ world of $\Mmc$, and $d \in \Delta$
such that $d\in(\B_{i} C \sqcap \B_{i} D)^{\Int_w}$.
Thus $d\in(\B_{i} C)^{\Int_w}$ and $d\in(\B_{i} D)^{\Int_w}$,
that is $[C]_d^{\Mmc}, [D]_d^{\Mmc} \in\Nmc_{i}(w)$.
By closure under intersection, $[C]_d^{\Mmc} \cap [D]_d^{\Mmc} =  [C\sqcap D]_d^{\Mmc}\in\Nmc_{i}(w)$.
Then $d\in(\B_{i} ( C \sqcap D))^{\Int_w}$.
%
For the left-to-right direction,
assume that $\Fmc$ is not closed under intersection.
Then, there are $w\in\Wmc$, $\alpha,\beta\subseteq\Wmc$ such that $\alpha, \beta\in\Nmc_{i}(w)$ and $\alpha\cap\beta\notin\Nmc_{i}(w)$.
We define over $\Fmc$ the model
$\Mmc = ( \Fmc, \Int )$,
%$\Mmc = ( \Fmc, \{d\}, \Int )$,
where
$\Delta_{w} = \{d\}$, for all $w \in \Wmc$,
%$\Delta = \{d\}$,
and the interpretation of two concept names $A, B \in \NC$ is defined as follows:
$d\in A^{\Int_v}$ iff $v\in\alpha$, and
$d\in B^{\Int_v}$ iff $v\in\beta$
(and defined arbitrarily on all other symbols in $( \NC \cup \NR ) \setminus \{ A, B \}$).
%
We have $\llbracket A \rrbracket_d^{\Mmc} = \alpha$ and $\llbracket B \rrbracket_d^{\Mmc} = \beta$, which implies
$d\in(\B_{i} A)^{\Int_w}$ and $d\in(\B_{i} B)^{\Int_w}$.
Moreover, $\llbracket A \sqcap B \rrbracket_d^{\Mmc} = \llbracket A \rrbracket_d^{\Mmc} \cap \llbracket B \rrbracket_d^{\Mmc} = \alpha\cap\beta \notin\Nmc_{i}(w)$.
Thus $d\notin(\B_{i}(A\sqcap B))^{\Int_w}$.
%
	\item[\textnormal{$\mathit{L = \mathbf{N}}$.}]
%($\mathit{N}$-\emph{principle})
From right-to-left,
assume that $\Fmc$ contains the unit and $\top \sqsubseteq C$ is valid on $\Fmc$.
Then for all
$\Mmc = (\Fmc, \Imc)$
%$\Mmc = (\Fmc, \Delta, \Imc)$
based on $\Fmc$, and all $w$ in $\Mmc$, we have 
$C^{\Int_w} = \Delta_{w}$. 
As a consequence, for all
$d\in\Delta_{w}$,
%$d\in\Delta$,
$\llbracket C \rrbracket^{\Mmc}_{d} = \W$.
By the property of containing the unit we have that, for all $v\in\W$, $\llbracket C \rrbracket^{\Mmc}_{d} \in \Nmc_{i}(v)$.
So $d\in (\B_{i} C)^{\Int_v}$ for all
$d\in\Delta_{v}$, 
%$d\in\Delta$, 
that is, $\top \sqs \B_{i} C$ is valid on $\Fmc$.
% 
For the left-to-right direction,
%other direction,
assume that $\Fmc$ does not contain the unit, i.e.,
there is $w\in\Wmc$ such that $\W\notin\Nmc_{i}(w)$.
Then, for all models
$\Mmc= ( \Fmc, \Int )$
%$\Mmc= ( \Fmc, \Delta, \Int )$
based on $\Fmc$,
all $w \in \Wmc$,
and all
$d\in\Delta_{w}$,
%$d\in\Delta$,
%it holds the following.
%By definition,
we have $d\in\top^{\Int_w}$.
Moreover, 
since $d\in (\B_{i}\top)^{\Int_w}$ iff $\llbracket \top \rrbracket_d^{\Mmc}\in\Nmc_{i}(w)$ iff $\W\in\Nmc_{i}(w)$,
we also have $d\notin (\B_{i}\top)^{\Int_w}$.
%
	\item[\textnormal{$\mathit{L = \mathbf{P}}$.}]
          %
          From right-to-left, assume that $\Fmc = ( \Wmc, \{\Nmc_i \}_{i \in J})$ satisfies the $\mathbf{P}$-condition. I.e., for all $w \in \Wmc$ and $i \in J$, $\emptyset \not\in \Nmc_i(w)$.
          Then, for all models
          $\Mmc= ( \Fmc, \Int )$
%       $\Mmc= ( \Fmc, \Delta, \Int )$
          based on $\Fmc$,
          all $w \in \Wmc$,
          and all
          $d\in\Delta_{w}$,
%          $d\in\Delta$,
          we have $\emptyset = \llbracket \bot \rrbracket^\Mmc_d \not\in \Nmc_i(w)$. So $d \not\in (\Box_i\bot)^{\Int_w}$, or equivalently $d \in (\lnot \Box_i\bot)^{\Int_w}$. Also,
          $d \in \Delta_{w} = \top^{\Int_w}$.
%          $d \in \Delta = \top^{\Int_w}$.
          So $\top^{\Int_w} \subseteq (\lnot \Box_i\bot)^{\Int_w}$. Then $\Mmc, w \models \top \sqsubseteq \lnot \Box_i \bot$. Hence $\top \sqsubseteq \lnot \Box_i \bot$ is valid on $\Fmc$.
          %
          For the
          left-to-right direction,
%          other direction,
          assume that $\Fmc$ does not satisfy the $P$-condition for some $i \in J$. This means that there is $w \in \Wmc$ such that $\emptyset \in \Nmc_i(w)$. So there exists a model
             $\Mmc= ( \Fmc, \Int )$
%          $\Mmc= ( \Fmc, \Delta, \Int )$
          based on $\Fmc$,
          a $w \in \Wmc$, and a
          $d\in\Delta_{w}$,
%          $d\in\Delta$,
          such that $\emptyset = \llbracket \bot \rrbracket^\Mmc_d \in \Nmc_i(w)$. So $d \in (\Box_i \bot)^{\Int_w}$, or equivalently $d \not\in (\lnot\Box_i \bot)^{\Int_w}$. But $d \in \Delta_{w} = \top^{\Int_w}$. So $\top^{\Int_w} \not\subseteq (\lnot\Box_i \bot)^{\Int_w}$. Then $\Mmc, w \not\models \top \sqsubseteq \lnot\Box_i\bot$. Hence $\top \sqsubseteq \lnot \Box_i \bot$ is not valid on $\Fmc$.

	\item[\textnormal{$\mathit{L = \mathbf{Q}}$.}]
          %
          From right to left, assume that $\Fmc = ( \Wmc, \{\Nmc_i \}_{i \in J})$ satisfies the $\mathbf{Q}$-condition. I.e., for all $w \in \Wmc$ and $i \in J$, $\Wmc \not\in \Nmc_i(w)$.
          Then, for all models
          $\Mmc= ( \Fmc, \Int )$
%          $\Mmc= ( \Fmc, \Delta, \Int )$
          based on $\Fmc$,
          all $w \in \Wmc$,
          and all
          $d\in\Delta_{w}$,
%          $d\in\Delta$,
          we have $\Wmc = \llbracket \top \rrbracket^\Mmc_d \not\in \Nmc_i(w)$. So $d \not\in (\Box_i\top)^{\Int_w}$, or equivalently $d \in (\lnot \Box_i\top)^{\Int_w}$. Also $d \in \Delta = \top^{\Int_w}$. So $\top^{\Int_w} \subseteq (\lnot \Box_i\top)^{\Int_w}$. Then $\Mmc, w \models \top \sqsubseteq \lnot \Box_i \top$. Hence $\top \sqsubseteq \lnot \Box_i \top$ is valid on $\Fmc$.
          %
          For the other direction, assume that $\Fmc$ does not satisfy the $\mathbf{Q}$-condition for some $i \in J$. This means that there is $w \in \Wmc$ such that $\Wmc \in \Nmc_i(w)$.
          So there exists a model
          $\Mmc= ( \Fmc, \Int )$
%          $\Mmc= ( \Fmc, \Delta, \Int )$
          based on $\Fmc$,
          a $w \in \Wmc$,
          and a
               $d\in\Delta_{w}$,
%          $d\in\Delta$,
          such that $\Wmc = \llbracket \top \rrbracket^\Mmc_d \in \Nmc_i(w)$. So $d \in (\Box_i \top)^{\Int_w}$, or equivalently $d \not\in (\lnot\Box_i \top)^{\Int_w}$. But $d \in \Delta = \top^{\Int_w}$. So $\top^{\Int_w} \not\subseteq (\lnot\Box_i \top)^{\Int_w}$. Then $\Mmc, w \not\models \top \sqsubseteq \lnot\Box_i\top$. Hence $\top \sqsubseteq \lnot \Box_i \top$ is not valid on $\Fmc$.
          
	\item[\textnormal{$\mathit{L = \mathbf{D}}$.}]
          %
          From right-to-left, assume that $\Fmc  = ( \Wmc, \{\Nmc_i \}_{i \in J})$ satisfies the $\mathbf{D}$-condition. Moreover, let
          $\Mmc= ( \Fmc, \Int )$
%          $\Mmc= ( \Fmc, \Delta, \Int )$
          based on $\Fmc$, with a world $w \in \Wmc$, and
          $d \in \Delta_{w}$,
%          $d \in \Delta$,
          a concept $C$, and $i \in J$, all arbitrarily chosen. Suppose that $d \in (\Box_i C)^{\Int_w}$. It means that $[C]^\Mmc_d \in \Nmc_i(w)$. By the $\mathbf{D}$-condition, $\Wmc \setminus [C]^\Mmc_d \not\in \Nmc_i(w)$. Equivalently, $[\lnot C]^\Mmc_d \not\in \Nmc_i(w)$. This means that $d \not\in (\Box_i\lnot C)^{\Int_w}$, or equivalently, that $d \in (\lnot\Box_i\lnot C)^{\Int_w}$. So $(\Box_iC)^{\Int_w} \subseteq (\lnot \Box_i\lnot C)^{\Int_w}$. Then $\Mmc,w \models \Box_i C \sqsubseteq \Diamond_i C$. Hence, $\Box_i C \sqsubseteq \Diamond_i C$ is valid on $\Fmc$.
          %
          For the left-to-right direction, assume that $\Fmc$ does not satisfy the $\mathbf{D}$-condition for $i \in J$. So there is a $w \in \Wmc$, such that, for some $\alpha \subseteq \Wmc$, we have $\alpha \in \Nmc_i(w)$ and $\Wmc \setminus \alpha \in \Nmc_i(w)$.
          We define
            $\Mmc= (\Fmc, \Int )$
%          $\Mmc= (\Fmc, \Delta, \Int )$
          based on $\Fmc$,
          where,
          for any $v \in \Wmc$,
          $\Delta_{v} = \{d\}$,
%          $\Delta = \{d\}$,
          and $A^{\Int_v} = \{d\}$ iff $v \in \alpha$ (and defined arbitrarily on all other symbols in $( \NC \cup \NR ) \setminus \{A\}$). We have $\llbracket A \rrbracket^\Mmc_d = \{v \in \Wmc \mid d \in A^{\Int_w}\} = \alpha$, and $\llbracket \lnot A \rrbracket^\Mmc_d = \Wmc \setminus \alpha$. So $d \in (\Box_i A)^{\Int_w}$ and $d \in (\Box_i \lnot A)^{\Int_w}$, and then $d \not \in (\lnot \Box_i \lnot A)^{\Int_w}$. So $(\Box_iA)^{\Int_w} \not \subseteq (\lnot \Box_i \lnot A)^{\Int_w}$. This means that, $\Mmc, w \not\models \Box_i A \sqsubseteq \Diamond_i A$. Hence, $\Box_i C \sqsubseteq \Diamond_i C$ is not valid on $\Fmc$.
          

	\item[\textnormal{$\mathit{L = \mathbf{T}}$.}]
          %
          From right-to-left, assume that $\Fmc = ( \Wmc, \{\Nmc_i \}_{i \in J})$ satisfies the $\mathbf{T}$-condition. Moreover, let
          $\Mmc= ( \Fmc, \Int )$
%          $\Mmc= ( \Fmc, \Delta, \Int )$
          based on $\Fmc$, with a world $w \in \Wmc$, and
          $d \in \Delta_{w}$,
%          $d \in \Delta$,
          a concept $C$, and $i \in J$, all arbitrarily chosen. Suppose that $d \in (\Box_i C)^{\Int_w}$. It means that $\llbracket C \rrbracket^\Mmc_d \in \Nmc_i(w)$. By the $\mathbf{T}$-condition, $w \in \llbracket C \rrbracket^\Mmc_d$. So $d \in C^{\Int_w}$. So $(\Box_i C)^{\Int_w} \subseteq C^{\Int_w}$. Then $\Mmc, w \models \Box_i C \sqsubseteq C$. Hence, $\Box_i C \sqsubseteq C$ is valid on $\Fmc$.
          %
          For the left-to-right direction, assume that $\Fmc$ does not satisfy the $\mathbf{T}$-condition for $i \in J$. So there is a $w \in \Wmc$, such that for some $\alpha \subseteq \Wmc$ we have $\alpha \in \Nmc_i(w)$ and $w \not\in \alpha$. 
          We define
          $\Mmc= (\Fmc, \Int )$
%          $\Mmc= (\Fmc, \Delta, \Int )$
          based on $\Fmc$, where
          $\Delta_{v} = \{d\}$, for any $v \in \Wmc$,
%          $\Delta = \{d\}$,
          and $A^{\Int_v} = \{d\}$ iff $v \in \alpha$ (and defined arbitrarily on all other symbols in $( \NC \cup \NR ) \setminus \{A\}$). We have $\llbracket A \rrbracket^\Mmc_d = \{v \in \Wmc \mid d \in A^{\Int_w}\} = \alpha$. So $d \in (\Box_i A)^{\Int_w}$. Since $w \not \in \alpha$, $w \in \Wmc \setminus \alpha$. That is, $w \in \Wmc \setminus \llbracket A \rrbracket^\Mmc_d$, or equivalently $w \in \llbracket \lnot A \rrbracket^\Mmc_d$. Then, $d \in (\lnot A)^{\Int_w}$, or equivalently $d \not \in (A)^{\Int_w}$. So, 
%          $(\lnot A)^{\Int_w} \not\subseteq A^{\Int_w}$, 
$(\Box_i A)^{\Int_w} \not\subseteq A^{\Int_w}$,
%\nb{O: fixed problem here} 
          meaning that  $\Mmc, w \not\models \Box_i A \sqsubseteq A$. Hence, $\Box_i C \sqsubseteq C$ is not valid on $\Fmc$, as required.
\qedhere
\end{description}
%\nb{T: Work in progress}
\end{proof}




















\PropValid*
\begin{proof}
\emph{Point~1.}
{{The $(\Rightarrow)$ direction follows from the proof of Proposition~\ref{prop:corresp}.
To see that the $(\Leftarrow)$ direction does not hold in general,
we provide the following counterexample
showing that the $\mathbf{T}$-principle holds in a model that does not satisfy the $\mathbf{T}$-condition.
Consider 
$\Mmc = ( \Wmc, \{\Nmc_i \}_{i \in J}, \Int)$,
where 
\begin{itemize}
\item $\Wmc = \{w, v\}$;
\item
$\Nmc_{i}(w) = \{\{v\},\Wmc\}$ and
$\Nmc_{i}(v) = \{\{w\},\Wmc\}$,
 for $i \in J$;
%\item $\Delta_{w} = \Delta_{v} = \{d\}$;
%\item $\Imc_{w} = \Imc_{v}$.
\item $\Imc_{w} = \Imc_{v}$, with $\Delta_{w} = \Delta_{v} = \{d\}$.
\end{itemize}
$\Mmc$ does not satisfy the $\mathbf{T}$-condition,
since $\{v\} \in \Nmc_{i}(w)$ but $w \notin \{v\}$.
We show that the $\mathbf{T}$-principle holds in $\Mmc$.

\begin{claim}
For all concepts $C$, % every concept $C$, 
$\llbracket C \rrbracket^{\Mmc}_{d} = \emptyset$ or
$\llbracket C \rrbracket^{\Mmc}_{d} = \Wmc$. % \{w,v\}$.
\end{claim}
\begin{proof}[Proof of Claim]
By induction on the construction of $C$.

For the base case $C = A \in \NC$,
it follows from the definition that either
$d \in A^{\Imc_{w}}$ and $d \in A^{\Imc_{v}}$,
hence 
$\llbracket A \rrbracket^{\Mmc}_{d} = \Wmc$, % \{w,v\}$,
or
$d \notin A^{\Imc_{w}}$ and $d \notin A^{\Imc_{v}}$,
hence 
$\llbracket A \rrbracket^{\Mmc}_{d} = \emptyset$.

We now show the inductive cases.
For $C = \lnot D, D \sqcap E$, the proof is immediate by applying the induction hypothesis.

For $C = \exists \role.D$, the proof follows by the application of the induction hypothesis
and the fact that $r^{\Imc_{w}} = r^{\Imc_{v}}$.

For $C = \B_{i} D$, by induction hypothesis
$\llbracket D \rrbracket^{\Mmc}_{d} = \emptyset$ or
$\llbracket D \rrbracket^{\Mmc}_{d} = \Wmc$. %\{w,v\}$.
In the first case,
$\llbracket D \rrbracket^{\Mmc}_{d} \notin \Nmc_{i}(w)$
and
$\llbracket D \rrbracket^{\Mmc}_{d} \notin \Nmc_{i}(v)$,
hence $w \notin \llbracket \B_{i} D \rrbracket^{\Mmc}_{d}$
and $v \notin \llbracket \B_{i} D \rrbracket^{\Mmc}_{d}$,
thus $ \llbracket \B_{i} D \rrbracket^{\Mmc}_{d} = \emptyset$.
In the second case,
$\llbracket D \rrbracket^{\Mmc}_{d} \in \Nmc_{i}(w)$
and
$\llbracket D \rrbracket^{\Mmc}_{d} \in \Nmc_{i}(v)$,
hence $w \in \llbracket \B_{i} D \rrbracket^{\Mmc}_{d}$
and $v \in \llbracket \B_{i} D \rrbracket^{\Mmc}_{d}$,
thus $ \llbracket \B_{i} D \rrbracket^{\Mmc}_{d} = \Wmc$. % \{w,v\}$.
\end{proof}

Now, given a concept $C$, suppose that $d \in (\B_{i} C)^{\Imc_{w}}$,
that is, $\llbracket C \rrbracket^{\Mmc}_{d} \in \Nmc_{i}(w)$.
By the claim, $\llbracket C \rrbracket^{\Mmc}_{d} = \emptyset$
or $\llbracket C \rrbracket^{\Mmc}_{d} = \Wmc$.
Since $\emptyset \notin \Nmc_{i}(w)$,
we have that $\llbracket C \rrbracket^{\Mmc}_{d} = \Wmc$,
and thus $w \in \llbracket C \rrbracket^{\Mmc}_{d}$.
This means that $d \in C^{\Imc_{w}}$.
By the same argument, we can show that 
$d \in (\B_{i} C)^{\Imc_{v}}$ implies
$d \in C^{\Imc_{v}}$.
Therefore $\Mmc \models \B_{i} C \sqsubseteq C$.
%for any concept $C$.
Similarly we can prove that $\Mmc \models \Box_{i} \varphi \to \varphi$,
for any formula $\varphi$.
}}



\emph{Point~2.}
Let $\Fmf = (W, \{ R_{i} \}_{i \in J})$ be a relational frame, and let $\Mmf = (F, \Delta, I)$ be a relational model based on $\Fmf$.

%\todo{M: add $\mathbf{E}$-principle}

($\mathbf{E}$-principle) Follows directly from the ($\mathbf{M}$-principle) case below.

($\mathbf{M}$-principle) Assume $C \sqsubseteq D$ valid in $\Mmf$. Then $C^{I_w} \subseteq D^{I_w}$, for all $w \in W$. Now, suppose that $d\in (\B_{i} C)^{I_{w}}$, for $d \in \Delta$ and $w \in W$.
For all $v \in W$, $w \relations_{i} v$ implies $d\in C^{I_{v}}$,
hence $d\in D^{I_{v}}$.
Therefore, $d\in (\B_{i} D)^{I_{w}}$.
%and so $\B_{i} C \sqs \B_{i} D$ is valid in $\Mmf$.
%

($\mathbf{C}$-principle) Assume $d\in(\B_{i} C \sqcap \B_{i} D)^{I_{w}}$, that is, $d\in(\B_{i} C)^{I_{w}}$ and $d\in(\B_{i} D)^{I_{w}}$.
Then, for all $v \in W$, $w\relations_{i} v$ implies $d\in C^{I_{v}}$ and $d\in D^{I_{v}}$,
that is $d\in (C\sqcap D)^{I_{v}}$.
Therefore, $d\in(\B_{i} ( C \sqcap D))^{I_{w}}$.
%

($\mathbf{N}$-principle) Assume $\top \sqsubseteq C$ valid in $\Mmf$.
Then, for all $w \in W$,  $C^{I_w} =\Delta$.
Thus, for all $d\in\Delta$ and all $v \in W$, $w \relations_{i} v$ implies $d\in C^{I_{v}}$.
In conclusion, for all $d \in \Delta$, $d\in (\B_{i} C)^{I_{w}}$. 
%Therefore $\top \sqsubseteq \B_{i} C$ is valid in $\Mmf$.

In conclusion, the
$\mathbf{E}$-,
$\mathbf{M}$-, $\mathbf{C}$-, and $\mathbf{N}$-principle
%monotonicity, agglomeration, and necessitation
hold in $M$, and hence in $F$.

{{
($\mathbf{D}$-principle and $\mathbf{P}$-principle are equivalent)
It is easy to see that both the $\mathbf{D}$- and the $\mathbf{P}$-principle
hold if and only if 
$R_{i}$ is serial, for all $i \in J$
(that is, for all $w \in W$, there is $v\in W$ such that $w R_i v$).


($\mathbf{Q}$-principle does not hold)
The $\mathbf{Q}$-principle is incompatible with the $\mathbf{N}$-principle, whcih holds in relational models.
Indeed, if both principles hold in the relational model $\Mmf$,
then $\Mmf\models\top\sqsubseteq \Box_i\top\sqcap\lnot\Box_i\top$,
against the fact that $\ALC$ domains are non-empty.
}}
%\todo{T: added short proofs for D, P, Q \\ M: thanks}
\end{proof}
