\section{Description Logics of Agency and Ability}


\subsection{DL extension of Elgesem's logic}

\subsubsection{Syntax}


An $AL^{n}_{\ALC}$ \emph{concept} is an expression of the form
\[
C ::= A \mid \lnot C \mid C \sqcap C \mid \exists \role.C \mid \mathbb{B}_{i} C \mid \mathbb{C}_{i} C,
\]
where $A \in \NC$, $\role \in \NR$, while $\mathbb{B}_{i}$ and $\mathbb{C}_{i}$, with
$i \in J = \{ 1, \ldots, n \}$,
are \emph{agency} and \emph{ability} \emph{modalities}, respectively.

An \emph{$AL^{n}_{\ALC}$ formula}
takes the form
\[
\p ::= \pi \mid \neg \p \mid \p \land \p \mid \mathbb{B}_{i} \p \mid \mathbb{C}_{i} \p,
\]
where
$\pi$ is an $AL^{n}_{\ALC}$ atom
and
$i \in J$.





\subsubsection{Semantics}


An \emph{agency neighbourhood frame}, or simply \emph{frame}, is a pair $\Fmc = ( \Wmc, \{\Nmc^{\mathbb{B}}_i , \Nmc^{\mathbb{C}}_i \}_{i \in J})$,
where $\Wmc$ is a non-empty set of worlds and, for each $i \in J = \{1, \ldots, n\}$, $\Nmc^{\mathbb{B}}_{i}$ and $\Nmc^{\mathbb{C}}_{i}$ are neighbourhood functions satisfying the following conditions, for every
$i \in J$, $w \in \Wmc$, and $\alpha, \beta \subseteq \Wmc$:
%
\begin{itemize}
	\item $\Wmc \not \in \Nmc^{\mathbb{C}}_i(w)$;
	\item $\emptyset \not \in \Nmc^{\mathbb{C}}_i(w)$;
	\item if $\alpha\in \Nmc^{\mathbb{B}}_i(w)$ and $\beta\in \Nmc^{\mathbb{B}}_i(w)$, then $\alpha\cap\beta\in \Nmc^{\mathbb{B}}_i(w)$;
	\item if $\alpha \in \Nmc^{\mathbb{B}}_i(w)$, then $w \in \alpha$;
	\item $ \Nmc^{\mathbb{B}}_i(w) \subseteq  \Nmc^{\mathbb{C}}_i(w)$.
\end{itemize}

An \emph{$AL^{n}_{\ALC}$ varying} or \emph{constant domain neighbourhood model},
%or simply \emph{model},
based on an agency neighbourhood frame frame $\Fmc$, is a pair
$\Mmc = (\Fmc, \Int)$,
where
$\Fmc = (\Wmc, \{\Nmc^{\mathbb{B}}_{i}, \Nmc^{\mathbb{C}}_{i} \}_{i \in J})$ is an agency neighbourhood frame
and $\Imc$ is
defined as in Section~\ref{sec:prelim}, for the case of varying or constant domain neighbourhood models, respectively.
%a function
%associating with every $w \in \Wmc$ an \emph{$\ALC$ interpretation}
%$\Imc_{w} = (\Delta_{w}, \cdot^{\Imc_{w}})$,
%with non-empty \emph{domain} $\Delta_{w}$,
%and where $\cdot^{\Imc_{w}}$ is a function such that:
%for all $A \in \NC$, $A^{\Imc_{w}} \subseteq \Delta_{w}$;
%for all $\role \in \NR$, $\role^{\Imc_{w}} \subseteq \Delta_{w} {\times} \Delta_{w}$;
%for all $a \in \NI$, $a^{\Imc_{w}} \in \Delta$, and for all $u, v \in \reldomain$, $a^{\Imc_{u}}= a^{\Imc_{v}}$(denoted by $a^{\Imc}$).
%An \MLALC{n} \emph{constant domain neighbourhood model}
%is defined in the same way, except that, for all $w,w'\in\Wmc$,
%we have that $\Delta_{w}=\Delta_{w'}$.


\subsection{DL extension of Troquard's logic}


Given a finite set of \emph{agents} $J = \{ 1, \ldots, n \}$, we define $CL^{n}_{\ALC}$ \emph{concepts} and \emph{formulas} analogously to $AL^{n}_{\ALC}$, but with agency modalities $\mathbb{B}_{g}$ and ability modalities $\mathbb{C}_{g}$ indexed by a \emph{coalition} $g \subseteq J$.


A \emph{$CL^{n}_{\ALC}$ varying} or \emph{constant domain neighbourhood model} $\Mmc = (\Fmc, \Int)$ is defined analogously to the $AL^{n}_{\ALC}$ case, based on a \emph{coalitional agency neighbourhood frame} $\Fmc = ( \Wmc, \{\Nmc^{\mathbb{B}}_{g}, \Nmc^{\mathbb{C}}_{g} \}_{g \subseteq J})$.




