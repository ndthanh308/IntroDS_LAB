%%%%%%%%%%%%%%%%%%%%%%%%%%%%%%%%%%%%%%%%%%%%%%%%%%%%%%%%%%%%%%%%%%%%%%
%\subsection{Reductions to Satisfiability on Relational Models}
%\subsection{Reductions to $\K^{m}_{\ALC}$ Formula Satisfiability on Relational Models}
%\subsection{Reductions to Normal Modal Description Logics}
%
\label{sec:relation}

%\todo[inline]{M: add reduction of $\EMC^{n}_{\ALC}$ to $\K^{n}_{\ALC}$?}
	
We now study the complexity of the formula satisfiability problem in $\EnALC{n}$ and $\MnALC{n}$
on constant domain neighbourhood models.
%This result is then lowered %to $\ExpTime$ 
%for fragments, denoted by $\EALCg$  and $\MALCg$, in which the modal operators are applied \emph{globally}, i.e., over \ALC axioms only.
%%% OLD VERSION
%At the propositional level,
%%(multi-modal)
%logics $\Ebf^{n}$ and $\Mbf^{n}$ have both been used as a basis for weak deontic systems~\cite{AngEtAl,Che} (although $\Mbf^{n}$ suffers from several problems discussed in Section~\ref{sec:problem}), as well as to interpret praxeological operators, such as `agent $i$ has the ability to bring about $\p$'~\cite{Bro,Pac}.
%Moreover, $\Mbf^{n}$ has been combined with $\ALC$, as a basis for further coalition logic 
%extensions of description logic languages~\cite{SeyJam,SeyJam1}, and $\Ebf^{n}$ modal operators have been applied 
%over $\ALC$ axioms to formalise reasoning about agents' intentions~\cite{ErdSey} (however, without 
%establishing tight complexity results).
%%\nb{O: added}
%In this section we study the complexity of the formula satisfiability problem in $\EnALC{n}$ and $\MnALC{n}$. 
%This result is then lowered %to $\ExpTime$ 
%for fragments of these logics in which the modal operators are applied only over \ALC axioms.
%These fragments, denoted by $\EALCg$  and $\MALCg$,
%%\nb{M: Change notation?}
%are called \emph{global}.
%\nb{M: Change terminology/notation? It is fine with me as it is, but we don't have space for multimodal index $n$}
%\nb{M: mention papers~\cite{SeyErd, ErdSey}?}
%\paragraph{{\bf Satisfiability in $\EnALC{n}$ and $\MnALC{n}$.}}
We provide a $\NExpTime$ upper bound for satisfiability in $\EnALC{n}$ and $\MnALC{n}$ by using a reduction, lifted from the propositional case, to multi-modal $\KnALC{m}$.
The translation $\cdot\tr$ from \MLALC{n} to \MLALC{3n} is defined as~\cite{KraWol,GasHer}: 
	$A\tr = A$,
	$(\lnot C)\tr = \lnot C\tr$,
	$(C \sqcap D)\tr = C\tr \sqcap D\tr$,
	$(\exists \role.C)\tr = \exists \role.C\tr$;
$(C(a))\tr = C\tr(a)$,
$(r(a,b))\tr = r(a,b)$,
$(C \sqsubseteq D)\tr = C\tr \sqsubseteq D\tr$,
$(\lnot \psi)\tr  = \lnot \psi\tr$,
$(\psi \land \chi)\tr  = \psi\tr \land \chi\tr$;
$(\B_{i} \gamma)\tr = \D_{i_{1}} (\B_{i_{2}} \gamma\tr \circ \B_{i_{3}} \lnot \gamma\tr)$;
%
%\[
%%\begin{equation*}
%\begin{aligned}[c]
%	A\tr & = A \\
%	(\lnot C)\tr & = \lnot C\tr \\
%	(C \sqcap D)\tr & = C\tr \sqcap D\tr,\\
%	(\exists \role.C)\tr & = \exists \role.C\tr, \\
%\end{aligned}
%\qquad\qquad
%\begin{aligned}[c]
%(C(a))\tr & = C\tr(a) \\
%(r(a,b))\tr & = r(a,b), \\
%(C \sqsubseteq D)\tr & = C\tr \sqsubseteq D\tr, \\
%(\lnot \psi)\tr & = \lnot \psi\tr, \\
%(\psi \land \chi)\tr & = \psi\tr \land \chi\tr
%%			A\tr & = A, \\
%%			(\lnot C)\tr & = \lnot C\tr, \\
%%			(C \sqcap D)\tr & = C\tr \sqcap D\tr, \\
%%			(\exists \role.C)\tr & = \exists \role.C\tr, \\
%%			(C(a))\tr & = C\tr(a), \\
%%			(r(a,b))\tr & = r(a,b), \\
%%			(C \sqsubseteq D)\tr & = C\tr \sqsubseteq D\tr, \\
%%			(\lnot \gamma)\tr & = \lnot \gamma\tr, \\
%%			(\gamma \circ \delta)\tr & = \gamma\tr \circ \delta\tr, \\
%%			(\B_{i} \gamma)\tr & = \D_{i_{1}} (\B_{i_{2}} \gamma\tr \circ \B_{i_{3}} \lnot \gamma\tr)
%		\end{aligned}
%%\end{equation*}
%\]
%\[
%(\B_{i} \gamma)\tr = \D_{i_{1}} (\B_{i_{2}} \gamma\tr \circ \B_{i_{3}} \lnot \gamma\tr)
%\]
%\begin{align*}
%			A\tr & = A, \\
%%			(\lnot C)\tr & = \lnot C\tr, \\
%%			(C \sqcap D)\tr & = C\tr \sqcap D\tr, \\
%			(\exists \role.C)\tr & = \exists \role.C\tr, \\
%			(C(a))\tr & = C\tr(a), \\
%			(r(a,b))\tr & = r(a,b), \\
%			(C \sqsubseteq D)\tr & = C\tr \sqsubseteq D\tr, \\
%			(\lnot \gamma)\tr & = \lnot \gamma\tr, \\
%			(\gamma \circ \delta)\tr & = \gamma\tr \circ \delta\tr, \\
%			(\B_{i} \gamma)\tr & = \D_{i_{1}} (\B_{i_{2}} \gamma\tr \circ \B_{i_{3}} \lnot \gamma\tr)
%		\end{align*}
%%		\begin{align*}
%%			A\tr & = A, \\
%%			(\exists \role.C)\tr & = \exists \role.C\tr, \\
%%			(C \sqs D)\tr & = C\tr \sqs D\tr, \\
%%			(\vartheta)\tr & = \vartheta, \\
%%			(\lnot \gamma)\tr & = \lnot \gamma\tr, \\
%%			(\gamma \circ \delta)\tr & = \gamma\tr \circ \delta\tr, \\
%%			(\B_{i} \gamma)\tr & = \D_{i_{1}} (\B_{i_{2}} \gamma\tr \circ \B_{i_{3}} \lnot \gamma\tr)
%%		\end{align*}
where $A \in \NC$, $\role\in\NR$,
%$\vartheta$ is an assertion,
$\gamma$ is either an $\MLALC{n}$ concept or formula,
and $\circ \in \{ \sqcap, \land \}$ accordingly.
%
Using this translation, one can show that 
 satisfiability on neighbourhood models
%\nb{M: added}
is reducible to %the formula 
satisfiability on the
relational
%normal modal
models~\cite{KraWol,GasHer}.
%It follows from the reduction, given in Theorem~\ref{theor:classicalred}, 
%of the formula satisfiability problem for $\EALC$ to the same problem for $\KthreeALC$, which 
Since satisfiability in $\KnALC{3n}$ constant domain relational models is
$\NExpTime$-complete~\cite[Theorem 15.15]{GabEtAl03},
 we obtain the following complexity result.

\begin{restatable}{theorem}{Theoremcomplealc}\label{theor:complealc}
Satisfiability in $\EnALC{n}$  on constant domain neighbourhood models is decidable in $\NExpTime$.
\end{restatable}
%



%%% SKETCH
%\begin{proof}[Sketch]
%Let $\p$ be an \MLALC{n} formula s.t.
%$\Mmc, w \mdl \p$, for some \Nmodel $\Mmc = (\Fmc, \Delta, \Int)$
%and some $w$ in $\Fmc = (\Wmc, \{ \Nmc_{i} \}_{i \in [1, n]})$.
%We define an \Rframe
%$\Fmf = (W, \{ R_{i_{j}} \}_{i \in [1,n], j \in [1, 3]})$
%and an $\MLALC{3n}$ \Rmodel
%$\Mmf = (\Fmf, \Delta, I)$
%s.t.:
%	\begin{itemize}
%		\item $W = \{ (w, 0) \mid w \in \Wmc \} \cup \{ (\alpha, 1) \mid \alpha \in \bigcup_{v \in \Wmc} \Nmc_{i}(v) \}$
%		\item $\relations_{i_1} = \{ ((w, 0), (\alpha, 1)) \mid \alpha \in \Nmc_{i}(w)\}$;
%		\item $\relations_{i_2} = \{ ((\alpha, 1), (w, 0)) \mid w \in \alpha \}$
%		\item $\relations_{i_3} = \{  ((\alpha, 1), (w, 0)) \mid w \not \in \alpha \}$ 
%		\item for every $(w, 0) \in W$, $I(w, 0) = \Int(w)$; for every $(\alpha, 1) \in W$, $X^{I(\alpha, 1)} = \eset$, for all $X \in \NC \cup \NR$, and $a^{I(\alpha, 1)} = a^{\Int}$, for all $a \in \NI$.
%	\end{itemize}
%	%
%The pairs $(w, 0), (\alpha, 1)$ are used to ensure that $W$ is the disjoint union of the sets of worlds $w$ and subsets $\alpha$ of $\Wmc$.
%%
%By induction on concept and formulas occurring in $\p$,
%one can show that $\Mmf, (w, 0) \mdl \p\tr$.
%Conversely, given a $\MLALC{3n}$ formula $\p\tr$ s.t.
%$\Mmf, w \mdl \p\tr$,
%for some $\MLALC{3n}$ R-model
%$\Mmf = (\Fmf, \Delta, I)$
%based on
%$\Fmf = (W, \{ \relations_{i_{j}} \}_{i \in [1, n], j \in [1, 3]})$,
%and some
%$w \in W$,
%we define a $\MLnALC{n}$ N-model
%$\Mmc = (\Fmc, \Delta, \Int)$
%based on
%$\Fmc = (\Wmc, \{ \Nmc_{i} \}_{i \in [1, n]})$
%s.t.
%$\Wmc = W$,
%and for all $w \in W$:
%\begin{itemize}
%			\item $\alpha \in \Nmc_{i}(w)$ iff there is $v \in W$ s.t. $w \relations_{i_1} v$ and: $(i)$ for all $u \in W$, $v \relations_{i_2} u \Rightarrow u \in \alpha$, and $(ii)$ for all $u \in W$, $v \relations_{i_3} u \Rightarrow u \not \in \alpha$;
%			\item $\Int(w) = I(w)$.
%\end{itemize}
%Again, by induction, we obtain that $\Mmc, w \mdl \p$.
%\qed
%\end{proof}
 

%\subsection{The Monotonic Case}
The translation $\cdot\ttr$ from $\MLALC{n}$ to $\MLALC{2n}$ is defined as $\cdot\tr$ on all concepts and formulas, except for the modalised concepts or formulas $\gamma$:
%\begin{gather*}
	$(\B_{i} \gamma)\ttr = \D_{i_1} \B_{i_2} \gamma\ttr$.
%\end{gather*}
%\begin{definition}\label{def:monotonicmodel}
%\nb{M: Togliere, cfr. Sec. 3}
%A \e{supplemented $N$-frame}
%$\Fmc = (\Wmc, N)$
%is a $N$-frame such that for all
%$w \in \Wmc$ and all $\alpha, \beta \sbs \Wmc$,
%if
%$\alpha \in \Nmc(w)$ and $\alpha \sbs \beta$,
%then
%$\beta \in \Nmc(w)$.
%A \emph{supplemented $N$-model} is a $N$-model based on a supplemented $N$-frame.
%\end{definition}
%\begin{proof}(Sketch)
%\nb{M: Merge with next Th?}
%\end{proof}
%\noindent
We obtain an upper bound analogous to the one for $\EnALC{n}$ by a reduction %, shown in Theorem~\ref{theor:monotonicred}, 
of the formula satisfiability problem for $\MnALC{n}$
%\nb{O: typo here? \\ M: sure, thanks}
to the $\KnALC{2n}$ one~\cite{KraWol,GasHer,GabEtAl03}.
%which is $\NExpTime$-complete~\cite[Theorem 15.15]{GabEtAl}. 

\begin{restatable}{theorem}{Theoremcomplmalc}\label{theor:complmalc}
Satisfiability in $\MnALC{n}$   on constant domain neighbourhood models is decidable in $\NExpTime$.
\end{restatable}
%



%%% SKETCH
%\begin{proof}[Sketch]
%The proof is similar to the one of Theorem~\ref{theor:complealc}.
%%
%Given an \Nmodel based on a supplemented \Nframe satisfying an \MLALC{n} formula $\p$, we define an $\MLnALC{2n}$ \Rmodel satisfying $\p\ttr$ as above, by using relations $\relations_{i_1}$ and $\relations_{i_2}$ only.
%%
%To prove the inductive step for modalised formulas $\Box_{i} \psi$ occurring in $\p$, we use the fact that, in N-models $\Mmc$ based on supplemented N-frames $\Fmc = (\Wmc, \{ \Nmc_{i} \}_{i \in [1, n]})$,
%$\Mmc, w \mdl \B_{i} \psi$
%%$[ \p ]^{\Mmc}  \in \Nmc_{i}(w)$,
%is equivalent to:
%there is $\alpha \in \Nmc_{i}(w)$ s.t. $ \alpha \sbs [ \psi ]^{\Mmc}$.
%%
%Conversely, given a $\MLnALC{2n}$ \Rmodel
%$\Mmf = (\Fmf, \Delta, I)$
%based on
%$\Fmf = (W, \{ \relations_{i_{j}} \}_{i \in [1, n], j \in [1, 2]})$
%and satisfying $\p\ttr$, we define a \MLALC{n} \Nmodel
%$\Mmc = (\Fmc, \Delta, \Int)$
%based on
%$\Fmc = (\Wmc, \{ \Nmc_{i} \}_{i \in [1, n]})$
%s.t. $\Wmc = W$ and, for all $w \in W$: $\Int(w) =I(w)$;
%$\alpha \in \Nmc_{i}(w)$ iff there is $v \in W$ s.t. $w \relations_{i_1} v$ and for all $u \in W$, $v \relations_{i_2} u \Rightarrow u \in \alpha$.
%The \Nframe $\Fmc$ so defined is supplemented:
%for all $w \in W$, if $\alpha \in \Nmc_{i}(w)$ and $\alpha \sbs \beta \sbs W$, then there is $v \in W$ s.t. $w \relations_{i_1} v$ and for all $u \in W$, $v \relations_{i_2} u \Rightarrow u \in \beta$, i.e., $\beta \in \Nmc_{i}(w)$.
%Moreover, by induction, we have that $\Mmc$ satisfies $\p$.
%\qed
%\end{proof}


%%%%%%%%%%%%%%%%%%%%%%%%%%%%%%%%%%%%%%%%%%%%%%%%%%%%%%%%%%%%%%%%%%%%%%
\endinput
