\section{Discussion}
\label{sec:discuss}

We investigated reasoning in non-normal modal description logics,
%After providing motivations and preliminaries for these logics, we have focused on the following two aspects.
focussing on:
$(i)$
%terminating, sound and complete
tableaux algorithms to check satisfiability of multi-modal description logics formulas in varying domain neighbourhood models based on classes of frames
for
%that characterise
39 different
non-normal
systems;
%relevant to agency, epistemic, and deontic scenarios;
%(based on conditions on neighbourhood frames that extend the \emph{classical cube}~\cite{LelPim19}, obtained by combinations of the $\mathbf{E}$-, $\mathbf{M}$-, $\mathbf{C}$-, and $\mathbf{N}$-conditions, with the $\mathbf{T}$-, $\mathbf{D}$-, $\mathbf{P}$-, and $\mathbf{Q}$-conditions, which are relevant to agency, epistemic, and deontic scenarios);
$(ii)$ complexity of satisfiability restricted to fragments with modal operators applied only over formulas,
%(thus without modalised concepts)
%and over neighbourhood models with varying domains;
and interpreted on varying domain models;
$(iii)$ preliminary reduction of formula satisfiability for two non-normal modal description logics to satisfiability in the standard relational semantics on a constant domain.
We now discuss possible future work.
%
%We now discuss future research directions in connection with relevant related work.
%

First, we intend to devise tableaux for formula satisfiability on neighbourhood models with constant domain, by solving the problem of newly introduced variables that do not occur in other previously expanded labelled constraints systems.
For instance, by applying the $\mathbf{M}^{n}_{\ALC}$-rules to the $n$-labelled constraint system $S_{n} = \{ n : \Diamond_{i} \exists r. A(x), \Box_{i} \lnot A(x) \}$, we get the $m$-labelled constraint system $S_{m} = \{ m : \exists r. A(x),  m : \lnot A(x), m : r(x,y), m : A(y) \}$.
The fresh variable $y$ in $S_{m}$ does not allow for the direct extraction of a constant domain model,
as no object in the domain of the world associated with $S_{n}$ would be capable of representing $y$ correctly.
%
An alternative approach involves \emph{quasimodels}~\cite{GabEtAl03}, to characterise satisfiability on constant domain
%neighbourhood
models in terms of structures representing ``abstractions'' of the actual models of a formula.
Objects across worlds can be represented by means of \emph{runs}, i.e., functions to guarantee
%that they do not violate
their modal properties and the constant domain assumption.
A similar strategy is presented in~\cite{SeyErd09,SeyJam09,SeyJam10},
where the definition of runs (which is not carried out in detail) involves the introduction of suitable world ``copies''.
%However, such a definition is not fully carried out.
We conjecture that a quasimodel-based approach with \emph{marked variables}, as
illustrated in~\cite{GabEtAl03}, can also be adopted to solve the constant domain model extraction issue.

%%% PREVIOUS VERSION
%First, we would like to adapt our tableau algorithms to check formula satisfiability on neighbourhood models with constant domain.
%This
%%Such an adaptation requires
%requires to address the
%%problem of the
%introduction of fresh variables
%%introduced in a certain labelled constraint system
%that do not occur
%%not occurring
%in other previously expanded labelled constraints systems.
%For instance, by applying the $\mathbf{M}^{n}_{\ALC}$-rules to the $n$-labelled constraint system $S_{n} = \{ n : \Diamond_{i} \exists r. A(x), \Box_{i} \lnot A(x) \}$, we get the $m$-labelled constraint system $S_{m} = \{ m : \exists r. A(x),  m : \lnot A(x), m : r(x,y), m : A(y) \}$.
%The
%%introduction of the
%fresh
%variable $y$ in $S_{m}$ does not allow us to directly extract a model with constant domain,
%%from a completion set with labelled constraint systems of this kind,
%since there would be no object in the domain of the world associated with $S_{n}$ capable of representing $y$ correctly.
%%the variable $y$ introduced in $S_{m}$.
%%
%A possible solution could involve suitably defined \emph{quasimodels}~\cite{GabEtAl03}, to equivalently characterise satisfiability on constant domain neighbourhood models in terms of structures representing ``abstractions'' of the actual models of a formula.
%The representation of objects across worlds would be via suitably defined functions, called \emph{runs}, to guarantee that they do not violate their modal properties and the constant domain assumption.
%%These notions could then be used in the soundness proof of the tableau algorithms, where one starts from from a complete and clash-free completion set to construct a quasimodel for a formula (in place of a concrete model), in turn implying its satisfiability.
%A similar approach is followed by~\cite{SeyErd09,SeyJam09,SeyJam10}
%with suitable ``copies'' of worlds introduced to address the problem of the definition of runs.
%%representing the behaviour of domain objects across worlds.
%In these works, however,
%such a definition is not carried out in detail.
%%it is not made explicit how such a definition should be carried out in detail.
%We conjecture that a quasimodel-based approach with \emph{marked variables}, as
%illustrated by~\cite{GabEtAl03},
%%~\shortcite{GabEtAl03},
%can be
%%fruitfully
%adopted to solve the constant domain model extraction issue.
%%from a complete and clash-free completion set for a formula.



Moreover, we aim at tight complexities for $\LnALC$ satisfiability, both in varying and in constant domain 
%neighbourhood
models.
%This problem requires in particular to develop proof strategies for the lower bounds.
While $\ALC$ formula satisfiability is $\ExpTime$-complete, it is unclear whether the upper bound for $\LnALC$ on varying or constant domain
neighbourhood
models can be improved to $\ExpTime$-membership, for any $\Lvar \in \Log$.
%Note that,
%It has to be noted that,
At the propositional level, the formula satisfiability problem for the systems based on the $L$-condition, with $\mathbf{C} \not \in L$, is $\NP$-complete, rising to $\PSpace$ if the $\mathbf{C}$-condition is included~\cite{Var2}.
%At the propositional level, the formula satisfiability problem for the systems based on combinations of the $\mathbf{E}$-, $\mathbf{M}$-, and $\mathbf{N}$-conditions is
%%known to be
%$\NP$-complete, with a rise to $\PSpace$-completeness for systems respecting the $\mathbf{C}$-condition~\cite{Var2}.
For normal modal description logics, instead, the (tight) $\NExpTime$-hardness results are based on complexity proofs of \emph{product logics} over relational product frames~\cite{GabEtAl03}, and cannot be immediately adapted to neighbourhood semantics, where an analogous notion of product is not yet well understood.
Nonetheless, we conjecture that the $\NExpTime$-hardness known for,
%some normal modal DLs with relational semantics,
e.g., $\mathbf{K}_\mathcal{ALC}$ on constant domain relational models, also holds in the neighbourhood case, at least in presence of the $\mathbf{C}$-condition.

%%% PREVIOUS VERSION
%Moreover, we aim at tight
%complexity results
%for $\LnALC$ formula satisfiability, both in varying and in constant domain neighbourhood models.
%%This problem requires in particular to develop proof strategies for the lower bounds.
%It is known that $\ALC$ formula satisfiability is $\ExpTime$-complete.
%However, we do not know whether the upper bound for $\LnALC$ formula satisfiability problem on varying or constant domain neighbourhood models can be improved to $\ExpTime$-membership, for any $\Lvar \in \Log$.
%%Note that,
%%It has to be noted that,
%At the propositional level, the formula satisfiability problem for the systems based on the $L$-condition, with $\mathbf{C} \not \in L$ , is $\NP$-complete, rising to $\PSpace$-completeness if the $\mathbf{C}$-condition is included~\cite{Var2}.
%%At the propositional level, the formula satisfiability problem for the systems based on combinations of the $\mathbf{E}$-, $\mathbf{M}$-, and $\mathbf{N}$-conditions is
%%%known to be
%%$\NP$-complete, with a rise to $\PSpace$-completeness for systems respecting the $\mathbf{C}$-condition~\cite{Var2}.



Finally, we plan to study: non-normal modal description logics in \emph{coalitional} and \emph{strategic} settings~\cite{Pau,Tro,SeyJam09}, with an interplay between abilities and powers of \emph{groups} of agents, rather than single ones;
%Further dimensions to explore concern:
additional description logics constructs (e.g. \emph{nominals}, \emph{inverse roles}, or \emph{number restrictions}~\cite{BaaEtAl17}); and \emph{interactions between modalities}, with axioms expressing e.g. that an agent \emph{can do} anything they \emph{actually do}, by means of formulas of the form $\mathbb{D}_{i}C \sqsubseteq \mathbb{C}_{i}C$ or $\mathbb{D}_{i}\varphi \to \mathbb{C}_{i}\varphi$.




%\todo[inline,caption={}]{
%M: todo add Long-term todos
%\begin{itemize}
%	\item Tableaux for interacting modalities + coalitions + other DL constructs (e.g. $\mathcal{ALCOIQ}$)
%	\item Tableaux for constant domains (marked variables strategy?)
%	\item (Un-)decidability with global roles?
%	\item Matching upper and lower bounds for satisfiability ($\ExpTime$-membership from Donini and Massacci $\ALC$ tableaux? $\NExpTime$-hardness from products?)
%\end{itemize}
%}






%\section{Conclusion}
%\label{sec:conc}
%
%
%We have investigated reasoning in non-normal modal description logics,
%%After providing motivations and preliminaries for these logics, we have focused on the following two aspects.
%focussing on the following three aspects.
%First, we have introduced terminating, sound and complete tableaux algorithms to check satisfiability of multi-modal description logics formulas in varying domain neighbourhood models based on classes of frames that characterise 39 different non-normal systems.
%We considered conditions on neighbourhood frames that extend the \emph{classical cube}~\cite{LelPim19}, obtained by combinations of the $\mathbf{E}$-, $\mathbf{M}$-, $\mathbf{C}$-, and $\mathbf{N}$-conditions, with the $\mathbf{T}$-, $\mathbf{D}$-, $\mathbf{P}$-, and $\mathbf{Q}$-conditions, which are relevant to agency, epistemic, and deontic scenarios.
%We have then studied the complexity of satisfiability restricted to fragments where modal operators can be applied to formulas only (thus without modalised concepts) and interpreted on neighbourhood models with  varying domains.
%Finally, we have moved first steps towards constant domains, providing a reduction of formula satisfiability for two non-normal modal description logics to satisfiability in the standard relational semantics on a constant domain.



