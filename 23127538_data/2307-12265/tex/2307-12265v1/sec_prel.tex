%%%%%%%%%%%%%%%%%%%%%%%%%%%%%%%%%%%%%%%%%%%%%%%%%%%%%%%%%%%%%%%%%%%%%%
\section{Preliminaries}
\label{sec:prelim}

Here we introduce
%both non-normal and normal
modal description logics, first presenting their syntax, and then their semantics based on neighbourhood and relational models, respectively.
Finally, we introduce the family of frame conditions here considered. 

\subsection{Syntax}
%\paragraph{Syntax}

Let \NC, \NR and \NI be countably infinite and pairwise disjoint 
sets of \emph{concept}, \emph{role}, and \emph{individual names}, respectively.
%
An $\MLALC{n}$ \emph{concept} is an expression of the form
$
C ::= A \mid \lnot C \mid C \sqcap C \mid \exists \role.C \mid \B_{i} C,
$
where $A \in \NC$, $\role \in \NR$, and $\B_{i}$ such that
$i \in J = \{ 1, \ldots, n \}$.
%
%are
%modal operators.
%called \emph{boxes}.
%
An \emph{\MLALC{n} atom} is
%either
a \emph{concept inclusion} (\emph{CI}) of
the form $(C \sqsubseteq D)$, or an \emph{assertion} of the form $C(a)$ or
$\role(a,b)$, with $C, D$ \MLALC{n} concepts,
%$A \in \NC$,
$\role \in \NR$, and $a, b \in \NI$.
%
%
An \emph{\MLALC{n} formula}
has the form
%is an expression of the form
$
\varphi::= \pi \mid \neg \varphi\mid \varphi\land \varphi\mid \B_{i} \p,
$
where
$\pi$ is an \MLALC{n} atom
and
$i \in J$.
We use the following standard definitions for concepts:
%O: it is a bit strange to define abbrev here and then later in the paper say we use them as prim 
$\forall \role.C :=  \lnot \exists \role.\lnot C$;
$(C \sqcup D) :=  \lnot(\lnot C \sqcap \lnot D)$;
%$(C \Rightarrow D) := (\lnot C \sqcup D)$;
%$(C \Leftrightarrow D) := (C \Rightarrow D) \sqcap (D \Rightarrow C)$;
$\bot := A \sqcap \lnot A$,
$\top := A \sqcup \lnot A$
(for an arbitrarily fixed $A \in \NC$);
%$\top :=  \lnot \bot$;
%}}
$\D_{i} C := \lnot \B_{i} \lnot C$.
%($\D_{i}$ are called \emph{diamonds}).
%\nb{M: $\D$ primitivo?}
Concepts of the form $\B_{i} C$, $\D_{i} C$ are \emph{modalised concepts}.
Analogous conventions
%also
hold for formulas,
writing
$C \equiv D$ for $(C \sqsubseteq D) \land (D \sqsubseteq C)$
and setting
$\mathsf{false} := (\top \sqsubseteq \bot)$, $\mathsf{true} := (\bot \sqsubseteq \top)$.
%



%%%%%%%%%%%%%%%%%%%%%%%%%%%%%%%
\subsection{Semantics}\label{sec:sem}

%\nb{M: todo add brief intro}
%
%%%%%%%%%%%%%%%%%%%%%%%%%%%%%
%We
%now define 
%the neighbourhood and relational semantics.
We now define neighbourhood semantics, which (as already mentioned) can be seen as a generalisation of the relational semantics, introduced immediately after.

\subsubsection{Neighbourhood Semantics}
%\subsection{Neighbourhood Semantics for Propositional Modal Logic}
%\nb{M: Changed! neighbourhood frame $\to$ multimodal neighbourhood frame}
%\paragraph{Neighbourhood Semantics}
 A \emph{neighbourhood frame}, or simply \emph{frame},
%shortened to \emph{\Rframe} when $n$ is clear from the context)
is a pair
%\nb{M: Aggiunto $\Fmc$}
$\Fmc = ( \Wmc, \{\Nmc_i \}_{i \in J})$,
where 
$\Wmc$ is a non-empty set
of \emph{worlds}
and
%$\Nmc$ is a function associating to
$\Nmc_{i} \colon \W \rightarrow 2^{2^{\Wmc}}$ is   a \emph{neighbourhood function}, for each
\emph{agent}
%$1 \leq i \leq n$, $\Nmc_{i}$
$i \in J = \{1, \ldots, n\}$. %\nb{O: rephrased}
%$\N$ is a function $\W \longrightarrow 2^{2^{\Wmc}}$, called \emph{neighbourhood function}.
%
%
%\label{sec:semantics}
An \emph{\MLALC{n} varying domain neighbourhood model}, or simply \emph{model}, based on a neighbourhood frame $\Fmc$ is a pair
$\Mmc = (\Fmc, \Int)$,
where
$\Fmc = (\Wmc,  \{\Nmc_i \}_{i \in J})$ is a neighbourhood frame
%neighbourhood frame,
%$\Delta$ is a non-empty set called \emph{domain of $\Mmc$},
and $\Imc$ is a function associating with every $w \in \Wmc$ an \emph{$\ALC$ interpretation}
%(or \emph{model})
$\Imc_{w} = (\Delta_{w}, \cdot^{\Imc_{w}})$,
with non-empty \emph{domain} $\Delta_{w}$,
%so that $\Delta = \bigcup_{w \in \Wmc} \Delta_{w}$,
and where $\cdot^{\Imc_{w}}$ is a function such that:
for all $A \in \NC$, $A^{\Imc_{w}} \subseteq \Delta_{w}$;
for all $\role \in \NR$, $\role^{\Imc_{w}} \subseteq \Delta_{w} {\times} \Delta_{w}$;
for all $a \in \NI$, $a^{\Imc_{w}} \in \Delta_{w}$.
%%and for all $u, v \in \reldomain$, $a^{\Imc_{u}}= a^{\Imc_{v}}$(denoted by $a^{\Imc}$).
An \MLALC{n} \emph{constant domain neighbourhood model}
is defined in the same way, except that, for all $w,w'\in\Wmc$,
we have that $\Delta_{w}=\Delta_{w'}$
and, for all $u, v \in \Wmc$,
%\nb{O: \Wmc? M: thanks}
we require $a^{\Imc_{u}}= a^{\Imc_{v}}$ (denoted by $a^{\Imc}$), that is, individual names are \emph{rigid designators}.
%\todo{M: added}
We often write $\Mmc = (\Fmc, \Delta, \Imc)$ to denote a constant domain neighbourhood model $\Mmc = (\Fmc, \Imc)$ with domain $\Delta = \Delta_{w}$, for every $w \in \Wmc$.
%\nb{O: changed; M: thanks}
%
%A model $\M = \langle \Fmc, \Delta_{w}, \Imc \rangle$ is \emph{supplemented, closed under intersection, or contains the unit}
%\nb{M: Used? Check}
%if $\Fmc$ is supplemented, closed under intersection, or contains the unit, respectively.
%
%The latter condition means that the individual names are treated as \emph{rigid designators}.
%Moreover, since the domain $\Delta_{w}$ is shared by all worlds $w \in \Wmc$, we say that we accept the \emph{constant domain assumption}.
%
Given a model $\Mmc = (\Fmc, \Int)$ and a world $w \in \Wmc$ of $\Fmc$ (or simply \emph{$w$ in $\Fmc$}), the \emph{interpretation $C^{\Imc_{w}}$ of a concept $C$ in $w$}
%written $C^{\Imc_{w}}$,
is defined
%by taking:
as:
$
		(\neg D)^{\Imc_{w}} = \Delta_{w} \setminus D^{\Imc_{w}}, \quad (D \sqcap E)^{\Imc_{w}}  = D^{\Imc_{w}} \cap E^{\Imc_{w}},
		$
		$
		(\exists r.D)^{\Imc_{w}} = \{d \in \Delta_{w} \mid \exists  e \in D^{\Imc_{w}}{:} (d,e) \in r^{\Imc_{w}}\},
		$
		$
		 (\B_{i} D)^{\Imc_{w}} = \{ d \in \Delta_{w} \mid \llbracket D \rrbracket^{\Mmc}_{d} \in \Nmc_{i}(w) \},
		 $
%	\begin{align*}
%		(\neg D)^{\Imc_{w}} & = \Delta_{w} \setminus D^{\Imc_{w}}, \quad (D \sqcap E)^{\Imc_{w}}  = D^{\Imc_{w}} \cap E^{\Imc_{w}},\\ 
%%		(D \sqcap E)^{\Imc_{w}} & = D^{\Imc_{w}} \cap E^{\Imc_{w}}, \\
%		(\exists r.D)^{\Imc_{w}} & = \{d \in \Delta_{w} \mid \exists  e \in D^{\Imc_{w}}{:} (d,e) \in r^{\Imc_{w}}\},\\
%		 (\B_{i} D)^{\Imc_{w}} & = \{ d \in \Delta_{w} \mid \llbracket D \rrbracket^{\Mmc}_{d} \in \Nmc_{i}(w) \},
%	\end{align*}
where, for all %\nb{M: changed, to check (it was $d \in \Delta_{w}$ before)} 
$d \in \bigcup_{w \in \Wmc} \Delta_{w}$, the set
$\llbracket D \rrbracket^{\Mmc}_{d} = \{ v \in \Wmc \mid  d \in D^{\Imc_{v}} \}$
is called the \emph{truth set of $D$ with respect to \Mmc and $d$}. 
%\nb{O: wrt to \Mmc and $d$?}
%$\llbracket C \rrbracket^{\Mmc}_{d} = \{ v \in \Wmc \mid  d \in C^{\Imc_{v}} \}$
%is called the \emph{truth set of $C$ with respect to $d$}.
% (note that for constant domains ).
We say that a concept $C$ is \emph{satisfied in $\Mmc$} if there is $w$ in $\Fmc$ such that $C^{\Imc_{w}} \neq \eset$, and that $C$ is \emph{satisfiable} (over varying or constant neighbourhood models, respectively) if there is a (varying or constant domain, respectively) neighbourhood model in which it is satisfied.
%
The \emph{satisfaction of an $\MLALC{n}$ formula~$\p$ in $w$ of $\Mmc$}, written $\Mmc, w  \models \p$, is defined
%analogously to relational semantics, and
as follows:
%\begin{gather*}
%	\Mmc, w \models C\sqsubseteq D \quad \text{iff} \quad C^{\Imc_{w}} \subseteq D^{\Imc_{w}},\\
%%	\qquad
%	\Mmc, w  \models \neg \psi \quad \text{iff} \quad \Mmc, w  \not \models \psi, \\
%	\Mmc, w  \models \psi \land \chi \quad \text{iff} \quad \Mmc, w  \models \psi \text{ and } \Mmc, w  \models \chi,\\
%%	\qquad
%	\Mmc, w  \models \B_{i} \psi \quad \text{iff} \quad [ \psi ]^{\Mmc} \in \Nmc_{i}(w),
% \end{gather*} 
\begin{alignat*}{6}
	& \Mmc, w \models C\sqsubseteq D && \text{ iff } && C^{\Imc_{w}} \subseteq D^{\Imc_{w}}, 
	&& \ \
	\Mmc, w  \models C(a) && \text{ iff } && a^{\Imc_{w}} \in C^{\Imc_{w}}, \\
	& \Mmc, w \models \role(a,b) && \text{ iff } && (a^{\Imc_{w}},b^{\Imc_{w}}) \in \role^{\Imc_{w}},
	&& \ \
	\Mmc, w  \models \neg \psi && \text{ iff } && \Mmc, w  \not \models \psi, \\
	& \Mmc, w  \models \psi \land \chi && \text{ iff } && \Mmc, w  \models \psi \text{ and } \Mmc, w  \models \chi, 
	&& \ \
	\Mmc, w  \models \B_{i} \psi && \text{ iff } && \llbracket \psi \rrbracket^{\Mmc} \in \Nmc_{i}(w),
 \end{alignat*} 
%\\
%%
%\begin{tabular}{ccc}
%	$\Mmc, w \models C\sqsubseteq D$ & \text{iff} & $C^{\Imc_{w}} \subseteq D^{\Imc_{w}},$ \\
%	$\Mmc, w  \models \neg \psi$ & \text{iff} & $\Mmc, w  \not \models \psi,$ \\
%	$\Mmc, w  \models \psi \land \chi$ & \text{iff} & $\Mmc, w  \models \psi \text{ and } \Mmc, w  \models \chi,$ \\
%	$\Mmc, w  \models \B_{i} \psi$ & \text{iff} & $[ \psi ]^{\Mmc} \in \Nmc_{i}(w),$
% \end{tabular} 
% \\
%%\begin{align*}
%%%	{\color{red}{\Mmf, w  \models A(a) && \text{ \ iff \ } && a^{I} \in A^{\Imc_{w}}, \qquad
%%%	\Mmf, w \models \role(a,b) && \text{ \ iff \ } && (a^{I},b^{I}) \in \role^{\Imc_{w}},}} \\
%%	\Mmc, w  \models \neg \varphi&& \text{ \ iff \ } && \Mmc, w  \not \models \p, \qquad
%%	\Mmc, w  \models \varphi\land \psi && \text{ \ iff \ } && \Mmc, w  \models \varphi\text{ and } \Mmc, w  \models \psi, \\
%%%	\Mmf, w  \models \B_{i} \varphi&& \text{ \ iff \ } && \text{for all $v \in \reldomain$: if $w \relations_{i} v$, then } \Mmf, v  \models \p.
%%	\Mmc, w \models C\sqsubseteq D  && \text{ \ iff \ } && C^{\Imc_{w}} \subseteq D^{\Imc_{w}}, \qquad
%%	\Mmc, w  \models \B_{i} \varphi \text{ \quad iff \quad } [ \varphi]^{\Mmc} \in \Nmc_{i}(w), 
%% \end{align*} 
%
where
$\llbracket \psi \rrbracket^{\Mmc} = \{ v \in \Wmc \mid \Mmc, v \models \psi \}$ is the \emph{truth set of $\psi$}.
%i.e. the set the worlds $v$ that satisfy $\psi$.
%%$[ \psi ]^{\Mmc}$ denotes the set $\{ v \in \Wmc \mid \Mmc, v \models \psi \}$ of the worlds $v$ that satisfy $\psi$, called the \emph{truth set of $\psi$}.
As a consequence of the above definition, we obtain the following
condition for $\Diamond_{i}$ formulas:
$\Mmc, w \models \Diamond_{i} \psi$  iff  $\llbracket \neg \psi \rrbracket^{\Mmc} \notin \Nmc_{i}(w)$.
Given a
neighbourhood
frame $\Fmc = (\Wmc,  \{\Nmc_i \}_{i \in J})$
and a
neighbourhood
model $\Mmc = (\Fmc, \Imc)$,
we say that $\varphi$ is \emph{satisfied in $\Mmc$} if there is $w \in \Wmc$ such that
$\Mmc, w \models \varphi$,
and that $\p$ is \emph{satisfiable} (over varying or constant domain neighbourhood models, respectively) if it is satisfied in some (varying or constant domain, respectively) neighbourhood model.
%%% VALIDITY, LOGICAL IMPLICATION (not used)
Also, $\p$ is    \emph{valid in $\Mmc$}, $\Mmc \models \p$, if it is satisfied in all $w$ 
of $\Mmc$, and it is \emph{valid on $\Fmc$} if, for all $\Mmc$ based on $\Fmc$,
%and all $w \in W$, $\Mmf, w \models \p$,
$\p$ is valid in $\Mmc$,
writing $\Fmc \models \p$.
%%
%Moreover,
%%\nb{M: Used in ex only. Remove (with ex)?}
%$\p$ \emph{logically implies} a formula $\psi$,
%writing $\varphi\mdl \psi$,
%if $\Mmc, w \models \p$ implies $\Mmc, w \models \psi$,  for every $\Mmc$ and every $w$ in $\Mmc$.
%%
%%Recall
%%%\nb{M: Used? Check}
%%that the concept satisfiability problem can be reduced to the formula satisfiability 
%%problem, since $C$ is satisfiable iff $\lnot (C \sqs \bot)$ is satisfiable.
    

%Other semantical definitions can be easily adapted from the relational semantics case.

 

%%%%%%%%%%%%%%%%%%%%%%%%%%%%%%%
%\subsection{Frames and Satisfiability Problems}

%In the following, we use $\mathfrak F$ to stand either for an N- or R-frame, and $\mathfrak M$ for a N- or R-model.
%
%To define the $\MLALC{n}$ formula satisfiability problems studied in this paper, we consider the principles listed in
%following
%Table~\ref{tab:principles}
%(where $C,D$ and $\p, \psi$ are $\MLALC{n}$ concepts and formulas, respectively).
%In the table,
%Here,
%$S$ is either a
%(N- or R-)
%frame $\mathfrak F$, or a model $\mathfrak M$.
%For a principle $P$, if $S = \mathfrak{F}$ (respectively, $S = \mathfrak{M}$), we say that %\emph{$P$ holds on $\mathfrak{F}$} (respectively, \emph{in $\mathfrak{M}$}).
%
%\begin{table}
%\begin{center}
%\begin{tabular}{l l l}
%\hline
%(\emph{Congruence}) && $S \models C \equiv D$ implies $S \models \Box_{i} C \equiv \Box_{i} D$. \\
%&& $S \models \varphi\leftrightarrow \psi$ implies $S \models \Box_{i} \varphi\leftrightarrow \Box_{i} \psi$. \\
%\hline
%(\emph{Monotonicity}) && $S \models C \sqsubseteq D$ implies $S \models \Box_{i} C \sqsubseteq \Box_{i} D$. \\
%&& $S \models \varphi\to \psi$ implies $S \models \Box_{i} \varphi\to \Box_{i} \psi$. \\
%\hline
%(\emph{Agglomeration}) &\quad\quad& $S \models \Box_{i} C \sqcap \Box_{i} D \sqsubseteq \Box_{i} (C \sqcap D)$. \\
%&& $S \models \Box_{i} \varphi\land \Box_{i} \psi \to \Box_{i} (\phi \land \psi)$. \\
%\hline
%\textcolor{red}{(\emph{Agglomeration'})} & $C\sqcap D = E$ is valid implies $\Box_{i} C \sqcap \Box_{i} D \sqsubseteq \Box_{i} E$ is valid. \\
%& $\phi \land \psi \leftrightarrow \xi$ is valid implies $\Box_{i} \phi \land \Box_{i} \psi \to \Box_{i} \xi$ is valid. \\
%\hline
%(\emph{Necessitation}) && $S \models \top \sqsubseteq C$ implies $S \models \top \sqsubseteq \Box_{i} C$. \\
%&& $S \models \p$ implies $S \models \Box_{i} \p$. \\
%\hline
%\end{tabular}
%\end{center}
%\caption{Principles over frames and models.}
%\label{tab:principles}
%\end{table}
%\nb{T: Red text for necessitation. Is it ok? Or how should I write it? \\ M: I think it's ok}
%
%On the correspondence between the principles in
%Table~\ref{tab:principles} and conditions over frames and models,
%we have the following result (see e.g.~\cite{Pac}
%\nb{T: Changed. In Chellas there are no correspondence theorems. Ref to Pacuit. \\ M: Thanks}
%for the propositional case).

%\begin{restatable}{theorem}{PropValid}\label{prop:dlvalid}
%Given a neighbourhood frame $\Fmc$, we have that:
%
%	$(i)$ congruence holds on $\Fmc$;
%	$(ii)$ monotonicity holds on $\Fmc$ iff $\Fmc$ is supplemented;
%	$(iii)$ agglomeration holds on $\Fmc$ iff $\Fmc$ is closed under intersection;
%	$(iv)$ necessitation holds on $\Fmc$ iff $\Fmc$ contains the unit.
%Given an R-frame $\Fmf$, congruence, monotonicity, agglomeration, and necessitation hold on $\Fmf$.
%moreover, for every R-model $\Mmf$, they hold in $\Mmf$.
%\end{restatable}
%
%





%%%%%%%%%%%%%%%%%%%%%%%%%%%%%%%
\subsubsection{Relational Semantics}
%\nb{T: Is the extension of concept $\bot$ defined anywhere? \\ M: comes from abbreviation above}
A
\emph{relational frame} 
%\emph{$n$-relational frame} 
%shortened to \emph{\Rframe} when $n$ is clear from the context)
is a pair
$\Fmf = ( \reldomain, \{\relations_i\}_{i \in J})$,
with 
$\reldomain$ non-empty set and $\relations_i$ 
binary relation on $\reldomain$,
for $i \in J = \{ 1, \ldots, n \}$.
%called \emph{accessibility relation}.
%
An \emph{$\MLALC{n}$ (constant domain) relational model} 
based on a relational frame
%$n$\Rframe 
$\Fmf = ( W, \{ R_{i} \}_{i \in J})$
is a
pair
$\Mmf = ( \Fmf, I)$,
%triple
%$\Mmf = ( \Fmf, \Delta, I)$,
where
%$\Delta$ is a non-empty set, called the \emph{domain} of $\Mmf$,
$I$ is a function associating with every $w \in W$ an \ALC \emph{interpretation}
%(or \emph{model})
$I_w = (\Delta, \cdot^{I_{w}})$,
%where $\Delta$ is a non-empty set, called the \emph{constant domain} of $\Mmf$,
having non-empty \emph{constant domain} $\Delta$,
and where $\cdot^{I_{w}}$ is a function such that:
for all $A \in \NC$, $A^{I_{w}} \subseteq \Delta$;
for all $\role \in \NR$, $\role^{I_{w}} \subseteq \Delta {\times} \Delta$;
for all $a \in \NI$, $a^{I_{w}} \in \Delta$, and for all $u, v \in \reldomain$, $a^{I_u}=a^{I_{v}}$(denoted by $a^{I}$).
%
%\nb{O: replace $\B$ by $\B_i$ below \\ M: Done, thanks}
Given a relational model $M = (F, I)$ and a world $w \in W$ of $F$ (or simply $w$ in $F$), the 
\emph{interpretation of a concept $C$ in $w$}, written $C^{I_{w}}$, is defined by taking:
%
$
	(\neg C)^{I_{w}} = \Delta \setminus C^{I_{w}},$
	$
	(C \sqcap D)^{I_{w}}  = C^{I_{w}} \cap D^{I_{w}},
	$
	$
	(\exists \role.C)^{I_{w}} =
		\{d \in \Delta \mid \exists  e \in C^{I_{w}}{:}(d,e) \in \role^{I_{w}}\},
		$
		$
	(\B_{i} C)^{I_{w}} = 
		\{ d \in \Delta \mid \ \forall v \in W:
		w R_{i} v
\Rightarrow
	d \in C^{I_{v}} \}.
$
%\begin{align*}
%	(\neg C)^{I_{w}} & = \Delta \setminus C^{I_{w}}, \quad (C \sqcap D)^{I_{w}}  = C^{I_{w}} \cap D^{I_{w}},\\
%%	(C \sqcap D)^{I_{w}} & = C^{I_{w}} \cap D^{I_{w}}, \\
%	(\exists \role.C)^{I_{w}} & =
%%	\begin{aligned}[t]
%		\{d \in \Delta \mid \exists  e \in C^{I_{w}}{:}(d,e) \in \role^{I_{w}}\}, \\
%%	\end{aligned}\\
%	(\B_{i} C)^{I_{w}} & = 
%%	\begin{aligned}[t]
%		\{ d \in \Delta \mid \ \forall v \in W:
%%		\text{if} \
%		w R_{i} v,
%%		\\
%%	& \
%%	\text{then} \
%\Rightarrow
%	d \in C^{I_{v}} \}.
%%	\end{aligned}
% \end{align*}
%

A concept $C$ is \emph{satisfied in $\Mmf$} if there is $w$ in $\Fmf$ such that $C^{I_{w}} 
\neq \eset$, and that $C$ is \emph{satisfiable on relational models} if there is a relational model in which it is satisfied.
%
The \emph{satisfaction of a $\MLALC{}$ formula~$\p$ in $w$ of $\Mmf$}, written $\Mmf, w  \models \p$, is defined, for atoms, negation and conjunction, similarly to the previous case, and as follows for the $\Box_{i}$ case:
%\begin{alignat*}{3}
%	& \Mmf, w \models C\sqsubseteq D  && \text{ \ iff \ } && C^{I_{w}} \subseteq D^{I_{w}}, \\
%	& \Mmf, w  \models C(a) && \text{ \ iff \ } && a^{I} \in C^{I_{w}}, \\
%	& \Mmf, w \models \role(a,b) && \text{ \ iff \ } && (a^{I},b^{I}) \in \role^{I_{w}}, \\
%	& \Mmf, w  \models \neg \varphi && \text{ \ iff \ } && \Mmf, w  \not \models \p,
%	\end{alignat*}
%\begin{alignat*}{3}
%	& \Mmf, w  \models \varphi\land \psi && \text{ \ iff \ } && \Mmf, w  \models \varphi\text{ and } \Mmf, w  \models \psi, \\
$
%		&
		\Mmf, w  \models \B_{i} \varphi
%		&&
		\text{ \ iff \ }
%		&&
		\forall v \in \reldomain: w \relations_{i} v \Rightarrow \Mmf, v  \models \p.
		$
% \end{alignat*}
%
Given a relational frame $\Fmf = (\reldomain, \{\relations_i\}_{i \in J})$
and a relational model $\Mmf = (\Fmf, \Delta, I)$,
we say that $\varphi$ is \emph{satisfied in $\Mmf$} if there is $w \in \reldomain$ such that
$\Mmf, w \models \varphi$,
and that $\p$ is \emph{satisfiable on relational models} if it is satisfied in some relational model.
%
Also, $\p$ is said to be \emph{valid in $\Mmf$}, $\Mmf \models \p$, if it is satisfied in all $w$ 
of $\Mmf$, and it is \emph{valid on $\Fmf$} if, for all $\Mmf$ based on $\Fmf$,
%and all $w \in W$, $\Mmf, w \models \p$,
$\p$ is valid in $\Mmf$,
writing $\Fmf \models \p$.
%
%Moreover,
%%\nb{M: Used in ex only. Remove (with ex)?}
%$\p$ \emph{logically implies} a formula $\psi$,
%writing $\varphi\mdl \psi$,
%if $\Mmf, w \models \p$ implies $\Mmf, w \models \psi$,  for every $\Mmf$ and every $w\in W$ in $M$.
% in $\Mmf$.
%
%\subsubsection{Satisfiability Problems}
%\nb{O: it is strange to start talking about neighb semantics, go to relation, and then come back here. }
%\begin{itemize}
%	\item  all neighbourhood frames, for $\mathit{L} = \mathbf{E}$;
%	\item  supplemented neighbourhood frames, for $\mathit{L} = \mathbf{M}$;
%	\item  neighbourhood frames closed under intersection, for $\mathit{L} = \mathbf{C}$;
%	\item neighbourhood frames containing the unit, for $\mathit{L} = \mathbf{N}$;
%	\item neighbourhood frames satisfying corresponding combinations of properties above, for $\mathit{L} \in \{ \mathbf{MC}, \mathbf{MN}, \mathbf{CN},  \mathbf{MCN} \}$.
%\end{itemize}
%

%By the \emph{$\MLALC{n}$ formula satisfiability problem in a class of (respectively, N- or R-) frames $\Cmc$} we mean the problem of deciding whether an $\MLALC{n}$ formula is satisfied in a (respectively, N- or R-) model based on a frame in $\Cmc$.
%%
%The \emph{formula satisfiability problem for} $\EnALC{n}$, $\MnALC{n}$, and $\KnALC{n}$ is the $\MLALC{n}$ formula satisfiability problem in the class of N-frames, supplemented N-frames, and R-frames,
%respectively.

%%% EXAMPLE (not needed)
%For example, the formula satisfiability problem for ${\mathbf{C}}^{n}_{\ALC}$
%is the formula satisfiability problem in the class of neighbourhood frames closed under intersection. For every neighbourhood frame \Fmc in this class, it holds that 
%$\Fmc \models \Box_{i} C \sqcap \Box_{i} D \sqsubseteq \Box_{i} (C \sqcap D)$~\cite{DL19} .
%%while, for $\mathit{L} \in \{ \mathbf{MC}, \mathbf{MN}, \mathbf{CN} \}$, the above properties are combined in the obvious way. 

%The \emph{formula satisfiability problem for} $\EnALC{n}$, $\MnALC{n}$,
%{\color{blue} ${\mathbf{C}}^{n}_{\ALC}$, ${\mathbf{N}}^{n}_{\ALC}$,}
%\nb{M: todo remove red}
%{\color{red}{and $\KnALC{n}$}} is the $\MLALC{n}$ formula satisfiability problem in the class of neighbourhood frames, supplemented neighbourhood frames, {\color{blue}{neighbourhood frames closed under intersection, neighbourhood frames containing the unit,}} {\color{red}{and R-frames}},
%respectively.
















%%% FIGURE PANTHEON
%% Figure environment removed




 

%%%%%%%%%%%%%%%%%%%%%%%%%%%%%%%
\subsection{Frame Conditions and Formula Satisfiability}
%\subsubsection{Frames and Satisfiability Problems}







%Given $L \in \mathsf{Pantheon}$, 
%\todo{M: removed `suppl.', `clos. und. intersec.' and `contain. the unit' throughout the paper (very few and useless occurrences) -- do the same with principles? -- mention that this are just (more or less) standard naming conventions from the literature}
We consider the following conditions on neighbourhood frames $\Fmc = ( \Wmc, \{\Nmc_i \}_{i \in J})$. We say that  \emph{$\Fmc$ satisfies the}:
%\nb{M: todo}
%
\begin{alignat*}{3}
	\text{\emph{$\mathbf{E}$-condition}} && \text{ \ iff \ } & \text{ $\Nmc_{i}$ is a neighbourhood function; } \\
	\text{\emph{$\mathbf{M}$-condition}} && \text{ \ iff \ }& \text{ $\alpha\in \Nmc_{i}(w)$ and $\alpha\subseteq\beta$ implies $\beta\in \Nmc_{i}(w)$; } \\
	\text{\emph{$\mathbf{C}$-condition}} && \text{ \ iff \ } & \text{ $\alpha\in \Nmc_{i}(w)$ and $\beta\in \Nmc_{i}(w)$ implies $\alpha\cap\beta\in \Nmc_{i}(w)$; } \\
	\textnormal{\emph{$\mathbf{N}$-condition}}  && \text{ \ iff \ } & \text{ $\Wmc \in \Nmc_{i}(w)$; } \\
	\text{\emph{$\mathbf{T}$-condition}} && \text{ \ iff \ } & \text{ $\alpha \in \Nmc_{i}(w)$ implies $w \in \alpha$; } \\
	\text{\emph{$\mathbf{D}$-condition}} && \text{ \ iff \ } & \text{  $\alpha \in \Nmc_{i}(w)$ implies $\Wmc \setminus \alpha \not \in \Nmc_{i}(w)$; } \\
	\text{\emph{$\mathbf{P}$-condition}} && \text{ \ iff \ } & \text{ $\emptyset \not \in \Nmc_{i}(w)$; } \\
	\text{\emph{$\mathbf{Q}$-condition}} && \text{ \ iff \ } & \text{  $\Wmc \not \in \Nmc_{i}(w)$; }
\end{alignat*}
%\begin{description}
%	\item[\textnormal{\emph{$\mathbf{E}$-condition}}] iff $\Nmc_{i}$ is a neighbourhood function;
%	\item[\textnormal{\emph{$\mathbf{M}$-condition}}]
%%	(\emph{supplementation})
%	iff $\alpha\in \Nmc_{i}(w)$ and $\alpha\subseteq\beta$ implies $\beta\in \Nmc_{i}(w)$;
%	\item[\textnormal{\emph{$\mathbf{C}$-condition}}]
%%	(\emph{closure under intersection})
%	iff $\alpha\in \Nmc_{i}(w)$ and $\beta\in \Nmc_{i}(w)$ implies $\alpha\cap\beta\in \Nmc_{i}(w)$;
%	\item[\textnormal{\emph{$\mathbf{N}$-condition}}]
%%	(\emph{containment of unit})
%	iff $\Wmc \in \Nmc_{i}(w)$;
%	%\end{description}
%	%
%	%\begin{description}
%		\item[\textnormal{\emph{$\mathbf{T}$-condition}}] iff $\alpha \in \Nmc_{i}(w)$ implies $w \in \alpha$;
%			\item[\textnormal{\emph{$\mathbf{D}$-condition}}] iff $\alpha \in \Nmc_{i}(w)$ implies $\Wmc \setminus \alpha \not \in \Nmc_{i}(w)$;
%	\item[\textnormal{\emph{$\mathbf{P}$-condition}}] iff $\emptyset \not \in \Nmc_{i}(w)$;
%	\item[\textnormal{\emph{$\mathbf{Q}$-condition}}] iff $\Wmc \not \in \Nmc_{i}(w)$;
%\end{description}
for every $w\in \Wmc$, $\alpha,\beta\subseteq \Wmc$.
%
%{\color{blue}{
Combinations of conditions, such as the $\mathbf{EMCN}$-condition, are obtained by suitably joining the ones above.
Moreover, since the $\mathbf{E}$-condition is always satisfied by any neighbourhood frame, we often omit the letter $\mathbf{E}$ from this naming scheme, writing for instance `$\mathbf{MCN}$' in place of `$\mathbf{EMCN}$'.
%}}

%Finally,
%given a neighbourhood frame of the form
%$\Fmc = ( \Wmc, \{ \Nmc_i , \Nmc'_i \}_{i \in J})$,
%we say that it is an \emph{interaction frame} if it satisfies
%the following condition, for every $i \in J$ and $w \in \Wmc$:
%\begin{description}
%	\item[\textnormal{\emph{Interaction condition}:}] $\Nmc_{i}(w) \subseteq \Nmc'_{i}(w)$.
%\end{description}




%We say that a neighbourhood frame
%$\Fmc = ( \Wmc, \{\Nmc_i \}_{i \in J})$,
%with $J = \{1, \ldots, n\}$,
%is
%an \emph{$\EM^{n}$ frame} (or that it is \emph{supplemented}),
%a \emph{$\EC^{n}$ frame} (or that it is \emph{closed under intersection}),
%or an \emph{$\EN^{n}$ frame} (or that it \emph{contains the unit})
%if the neighbourhood functions $\Nmc_i$ satisfy, respectively, the following conditions, for every $i \in J$, $w\in \Wmc$, $\alpha,\beta\subseteq \Wmc$.
%
%\begin{description}
%	\item[\textnormal{\emph{$M$-condition} (\emph{supplementation}):}] $\alpha\in \Nmc_{i}(w)$ and $\alpha\subseteq\beta$ implies $\beta\in \Nmc_{i}(w)$.
%	\item[\textnormal{\emph{$C$-condition} (\emph{closure under intersection}):}] $\alpha\in \Nmc_{i}(w)$ and $\beta\in \Nmc_{i}(w)$ implies $\alpha\cap\beta\in \Nmc_{i}(w)$.
%	\item[\textnormal{\emph{$N$-condition} (\emph{containment of unit}):}] $\Wmc \in \Nmc_{i}(w)$.
%\end{description}
%
%In addition, we say that a neighbourhood frame
%%$\Fmc = ( \Wmc, \{\Nmc_i \}_{i \in J})$,
%%with $J = \{1, \ldots, n\}$,
%is
%an \emph{$\ED^{n}$ frame},
%an \emph{$\ET^{n}$ frame},
%an \emph{$\EP^{n}$ frame},
%or
%an \emph{$\EQ^{n}$ frame},
%if its neighbourhood functions satisfy, respectively, the following conditions, for every $i \in J$, $w\in \Wmc$, $\alpha\subseteq \Wmc$.
%
%%\todo[inline]{$D: \lnot (\Box \varphi \land \Box \lnot \varphi)$}
%
%\begin{description}
%	\item[\textnormal{\emph{$D$-condition}:}]  $\alpha \in \Nmc_{i}(w)$ implies $\Wmc \setminus \alpha \not \in \Nmc_{i}(w)$.
%	\item[\textnormal{\emph{$T$-condition}:}] $\alpha \in \Nmc_{i}(w)$ implies $w \in \alpha$.
%	\item[\textnormal{\emph{$P$-condition}:}] $\emptyset \not \in \Nmc_{i}(w)$.
%		\item[\textnormal{\emph{$Q$-condition}:}] $\Wmc \not \in \Nmc_{i}(w)$.
%\end{description}
%
%Finally,
%given a neighbourhood frame of the form
%$\Fmc = ( \Wmc, \{ \Nmc_i , \Nmc'_i \}_{i \in J})$,
%we say that it is an \emph{interaction frame} if it satisfies
%the following condition, for every $i \in J$ and $w \in \Wmc$.
%\begin{description}
%	\item[\textnormal{\emph{Interaction condition}:}] $\Nmc_{i}(w) \subseteq \Nmc'_{i}(w)$.
%\end{description}


%A frame is:
%\emph{supplemented} if,
%for all
%$i \in J$,
%$w\in \Wmc$, 
%$\alpha,\beta\subseteq \Wmc$, $\alpha\in \Nmc_{i}(w)$ and $\alpha\subseteq\beta$ implies $\beta\in \Nmc_{i}(w)$;
%%it is
%\emph{closed under intersection} if,
%for all
%$i \in J$,
%$w\in \Wmc$, $\alpha,\beta\subseteq \Wmc$, 
%$\alpha\in \Nmc_{i}(w)$ and $\beta\in \Nmc_{i}(w)$ implies $\alpha\cap\beta\in \Nmc_{i}(w)$;
%and
%%it
%\emph{contains the unit} if,
%for all
%$i \in J$,
%$w\in \Wmc, \Wmc \in \Nmc_{i}(w)$.











%In the following, let
%%\[
%$
%\mathsf{Cube} = \{ \E, \EM, \EC, \EN, \EMC, \EMN, \ECN, \EMCN
%\},
%$
%%\]
%and let $\mathsf{Pantheon}$ be
%%the union of
%%$\Log$ with
%the set (cf. Figure~\ref{fig:pantheon})
%\begin{align*}
%	\{
%	&
%	\mathbf{E},\\
%	&
%	\mathbf{EM},	 \mathbf{EC},
%	\mathbf{EN},
%	\mathbf{EP}, \mathbf{EQ},
%	\mathbf{ED}, \mathbf{ET}, \\
%	&
%	\mathbf{EMN},
%	\mathbf{EMC},
%	\mathbf{EMP},
%	\mathbf{{\color{red}{EMQ}}},
%	\mathbf{EMD},
%	\mathbf{EMT},\\
%	&
%	\mathbf{ECN},
%	\mathbf{ECP}, \mathbf{ECQ},
%	\mathbf{ECD}, \mathbf{ECT},\\
%	&
%	\mathbf{ENP}, 
%	\mathbf{END},
%	\mathbf{ENT}, \\
%	&	 
%	\mathbf{EPQ},
%	\mathbf{EPD}, \mathbf{EQD},
%	\mathbf{EQT},\\
%	&
%	\mathbf{EMNP},
%	\mathbf{EMND}, \mathbf{EMNT}, \\
%	&
%	\mathbf{EMCD}, \mathbf{EMCT},\\
%	&
%	\mathbf{ECND}, \mathbf{ECNT},
%	\mathbf{ECPQ},
%	\mathbf{ECQD}, \mathbf{ECQT}, \\
%	&
%	\mathbf{EMCN},
%	\mathbf{EPQD},\\
%	&
%	\mathbf{EMCND}, \mathbf{EMCNT}
%	\}
%\end{align*}


%\subsubsection{Relationships among %Neighbourhood Frame 
%	Conditions} 
%	\todo{M: todo add ignore $MQ$}

On the relationships among (combinations of) neighbourhood frame conditions, we make the following observations.
%\todo{O: we say above L belongs to pantheon, talk about L condition but here in L we omit E without giving explanation}

\begin{restatable}{theorem}{PropImplicationSystem}\label{prop:implicationsystem}
Given a neighbourhood frame
$\Fmc = ( \Wmc, \{\Nmc_i \}_{i \in J})$, the following statements hold, for $i \in J$.
\begin{enumerate}
%[label=\arabic*]
	\item If $\Nmc_i$ satisfies the $\mathbf{MQ}$-condition then, for every $w \in \Wmc$, $\Nmc_{i}(w) = \emptyset$. Hence, $\Nmc_i$ satisfies all but the $\mathbf{N}$-condition.
%	all conditions except for $\mathbf{N}$ are satisfied.
	\item\label{item:P-cond} $\Nmc_i$ satisfies the $\mathbf{P}$-condition, if $\Nmc_i$ satisfies one of the following:\\
		\begin{enumerate*}[label=(\roman*)]
%			\item $\mathbf{MQ}$-condition;
			\item $\mathbf{MD}$-condition;
			\item $\mathbf{ND}$-condition; or
			\item $\mathbf{T}$-condition.
		\end{enumerate*}
	\item\label{item:D-cond} $\Nmc_i$ satisfies the $\mathbf{D}$-condition, if $\Nmc_i$ satisfies one of the following:\\
		\begin{enumerate*}[label=(\roman*)]
%				\item $\mathbf{MQ}$-condition; %\nb{O:added}
			\item $\mathbf{CP}$-condition; or
			\item $\mathbf{T}$-condition.
		\end{enumerate*}	
	\item $\Nmc_i$ does not satisfy the $\mathbf{NQ}$-condition.
\end{enumerate}
%\begin{enumerate}
%	\item if $\Nmc_i$ satisfies the $\mathit{MQ}$-condition, then $\Nmc_i$ satisfies the $\mathit{P}$-condition and $\Nmc_{i}(w) = \emptyset$, for every $w \in \Wmc$;
%	\item if $\Nmc_i$ satisfies the $\mathit{MD}$-condition, then $\Nmc_i$ satisfies the $\mathit{P}$-condition;
%	\item if $\Nmc_i$ satisfies the $\mathit{CP}$-condition, then $\Nmc_i$ satisfies the $\mathit{D}$-condition;
%	\item $\Nmc_i$ does not satisfy the $\mathit{NQ}$-condition;
%	\item if $\Nmc_i$ satisfies the $\mathit{ND}$-condition, then $\Nmc_i$ satisfies the $\mathit{P}$-condition;
%		\item if $\Nmc_i$ satisfies the $\mathit{T}$-condition, then $\Nmc_i$ satisfies the $\mathit{P}$-condition;
%	\item if $\Nmc_i$ satisfies the $\mathit{T}$-condition, then $\Nmc_i$ satisfies the $\mathit{D}$-condition.
%\end{enumerate}
\end{restatable}
%














% Figure environment removed


%% Figure environment removed


















%{\color{blue}{ 
Based on these results, Figure~\ref{fig:pantheon} depicts the relations between combinations of frame conditions: nodes are (groups of equivalent) conditions (with the canonical representative underlined), and arrows represent logical implications.
Any combination containing the $\mathbf{NQ}$-condition has been omitted, as it leads to inconsistency (Theorem~\ref{prop:implicationsystem}, Point 4). Moreover, due to Theorem~\ref{prop:implicationsystem}, Point 1, any combination that includes the $\mathbf{MQ}$-condition is not considered, since
%$\Nmc_{i}(w) = \emptyset$ enforces that,
for any neighbourhood frame $\Fmc$ satisfying such condition and any
%$\MLnALC$ formula $\psi$, we have $\Fmc \models \Box \psi \leftrightarrow \mathsf{false}$,
$\MLnALC$ concept $C$, we have $\Fmc \models \Box_{i} C \equiv \bot$,
and similarly for formulas,
hence trivialising the modal operators.
Thus, we consider in the remainder the set $\Log$ of 39 non-equivalent combinations shown (as nodes or canonical representatives) in Figure~\ref{fig:pantheon}.
%}}










For $\Lvar \in \Log$,
we say that a neighbourhood frame
$\Fmc = ( \Wmc, \{\Nmc_i \}_{i \in J})$,
with $J = \{1, \ldots, n\}$, is an \emph{$L^{n}$ frame} iff its neighbourhood functions $\Nmc_i$, for $i \in J$, satisfy the \emph{$\Lvar$-condition},
obtained by  combining the conditions associated with letters in $\Lvar$.
%\nb{M: todo fix}
For a class of neighbourhood frames $\Cmc$, the \emph{satisfiability in $\MLALC{n}$ on} (\emph{varying} or \emph{constant domain}, resp.) \emph{neighbourhood models based on a frame in $\Cmc$} is the problem of deciding whether an $\MLALC{n}$ formula is satisfied in a (varying or constant domain, resp.) neighbourhood model based on a frame in $\Cmc$.
%
%Given $\mathit{L} \in \mathsf{Pantheon}$, 
Satisfiability in \emph{$\LnALC$ on} (\emph{varying} or \emph{constant domain}, respectively) \emph{neighbourhood models} is satisfiability in $\MLALC{n}$ on (varying or constant domain, resp.) neighbourhood models based on a frame in the class of $L^{n}$ frames.
%
Finally,  \emph{satisfiability  in $\KnALC{n}$ on}
(\emph{constant domain})
\emph{relational models} is satisfiability in $\MLALC{n}$  on
%constant domain
relational models based on any relational frame.


















%\subsubsection{Correspondence: Conditions and Principles} 

%In the following, we use $\mathfrak F$ to stand either for a neighbourhood or relational frame, and $\mathfrak M$ for a neighbourhood or relational model. \nb{O: move this note since this is not used in prop 2?}
%
%To define the $\MLALC{n}$ formula satisfiability problems studied in this paper,







% $\mathfrak M$.
%For a principle $P$, if $S = \mathfrak{F}$ (respectively, $S = \mathfrak{M}$), we say that \emph{$P$ holds in $\mathfrak{F}$} (respectively, \emph{in $\mathfrak{M}$}).
%
%\begin{comment}
%\begin{table*}
%\begin{center}
%\begin{tabular}{l l l}
%\hline
%\multirow{2}{*}{\emph{${E}$-principle} (\emph{congruence})} && $S \models C \equiv D$ implies $S \models \Box_{i} C \equiv \Box_{i} D$. \\
%&& $S \models \varphi\leftrightarrow \psi$ implies $S \models \Box_{i} \varphi\leftrightarrow \Box_{i} \psi$. \\
%\hline
%\multirow{2}{*}{\emph{${M}$-principle} (\emph{monotonicity})} && $S \models C \sqsubseteq D$ implies $S \models \Box_{i} C \sqsubseteq \Box_{i} D$. \\
%&& $S \models \varphi\to \psi$ implies $S \models \Box_{i} \varphi\to \Box_{i} \psi$. \\
%\hline
%\multirow{2}{*}{\emph{${C}$-principle} (\emph{agglomeration})} &\quad\quad& $S \models \Box_{i} C \sqcap \Box_{i} D \sqsubseteq \Box_{i} (C \sqcap D)$. \\
%&& $S \models \Box_{i} \varphi\land \Box_{i} \psi \to \Box_{i} (\varphi \land \psi)$. \\
%%\hline
%%\textcolor{red}{(\emph{Agglomeration'})} & $C\sqcap D = E$ is valid implies $\Box_{i} C \sqcap \Box_{i} D \sqsubseteq \Box_{i} E$ is valid. \\
%%& $\phi \land \psi \leftrightarrow \xi$ is valid implies $\Box_{i} \phi \land \Box_{i} \psi \to \Box_{i} \xi$ is valid. \\
%\hline
%\multirow{2}{*}{\emph{${N}$-principle} (\emph{necessitation})} && $S \models \top \sqsubseteq C$ implies $S \models \top \sqsubseteq \Box_{i} C$. \\
%&& $S \models \varphi$ implies $S \models \Box_{i} \p$. \\
%\hline
%\multirow{2}{*}{\emph{${P}$-principle}} && $S \models \top \sqsubseteq  \lnot \Box_{i} \bot$. \\
%&& $S \models \lnot \Box_{i} \mathsf{false}$. \\
%\hline
%\multirow{2}{*}{\emph{${Q}$-principle}} && $S \models \top \sqsubseteq  \lnot \Box_{i} \top$. \\
%&& $S \models  \lnot \Box_{i} \mathsf{true}$. \\
%\hline
%\multirow{2}{*}{\emph{${D}$-principle} \textcolor{blue}{(\emph{deontic})}}  && $S \models \Box_{i} C \sqsubseteq \Diamond_{i} C $ \\
%&& $S \models \Box_{i} \varphi \to \Diamond_{i} \varphi$. \\
%\hline
%\multirow{2}{*}{\emph{${T}$-principle}} && $S \models \Box_{i} C \sqsubseteq  C $ \\
%&& $S \models \Box_{i} \varphi \to \varphi$. \\
%\hline
%\end{tabular}
%\end{center}
%\caption{Principles over frames and models.}
%\label{tab:principles}
%\end{table*}
%\end{comment}

\begin{table*}
\begin{center}
\footnotesize
\begin{tabular}{l l l}
\toprule
\multirow{2}{*}{\emph{${\mathbf{E}}$-principle}} & $S \models C \equiv D$ implies $S \models \Box_{i} C \equiv \Box_{i} D$. \\
% (\emph{congruence})
 & $S \models \varphi\leftrightarrow \psi$ implies $S \models \Box_{i} \varphi\leftrightarrow \Box_{i} \psi$. \\
\midrule
\multirow{2}{*}{\emph{${\mathbf{M}}$-principle}} & $S \models C \sqsubseteq D$ implies $S \models \Box_{i} C \sqsubseteq \Box_{i} D$. \\
% (\emph{monotonicity})
 & $S \models \varphi\to \psi$ implies $S \models \Box_{i} \varphi\to \Box_{i} \psi$. \\
\midrule
\multirow{2}{*}{\emph{${\mathbf{C}}$-principle}} & $S \models \Box_{i} C \sqcap \Box_{i} D \sqsubseteq \Box_{i} (C \sqcap D)$. \\
% (\emph{agglomeration})
 & $S \models \Box_{i} \varphi\land \Box_{i} \psi \to \Box_{i} (\varphi \land \psi)$. \\
\midrule
\multirow{2}{*}{\emph{${\mathbf{N}}$-principle}} & $S \models \top \sqsubseteq C$ implies $S \models \top \sqsubseteq \Box_{i} C$. \\
% (\emph{necessitation})
 & $S \models \varphi$ implies $S \models \Box_{i} \p$. \\
\bottomrule
\end{tabular}
\quad
\begin{tabular}{l l l}
\toprule
\multirow{2}{*}{\emph{${\mathbf{T}}$-principle}} & $S \models \Box_{i} C \sqsubseteq  C $. \\
& $S \models \Box_{i} \varphi \to \varphi$. \\
\midrule
\multirow{2}{*}{\emph{${\mathbf{D}}$-principle}} & $S \models \Box_{i} C \sqsubseteq \Diamond_{i} C $. \\
%{(\emph{deontic})}
& $S \models \Box_{i} \varphi \to \Diamond_{i} \varphi$. \\
\midrule
\multirow{2}{*}{\emph{${\mathbf{P}}$-principle}} & $S \models \top \sqsubseteq  \lnot \Box_{i} \bot$. \\
& $S \models \lnot \Box_{i} \mathsf{false}$. \\
\midrule
\multirow{2}{*}{\emph{${\mathbf{Q}}$-principle}} & $S \models \top \sqsubseteq  \lnot \Box_{i} \top$. \\
& $S \models  \lnot \Box_{i} \mathsf{true}$. \\
\bottomrule
\end{tabular}
\end{center}
\caption{Principles over neighbourhood or relational frames and models $S$.}
\label{tab:principles}
\end{table*}


We now study the correspondence between %the neighbourhood  frame
conditions presented in Section~\ref{sec:sem} and
  the principles in
Table~\ref{tab:principles},
%(where $C,D$ and $\varphi, \psi$ are $\MLALC{n}$ concepts and formulas, respectively)
%and where, for $L \in \mathsf{Pantheon}$, 
where
$S$ is either a
%(N- or R-)
(neighbourhood or relational) frame %$\mathfrak F$, 
or a (neighbourhood or relational) model
and the $L$-principle is obtained by suitably combining the basic principles.
We say that the $L$-principle holds in $S$ if the corresponding
expressions  in Table~\ref{tab:principles} are satisfied.
On the correspondence between the principles in
Table~\ref{tab:principles} and conditions over frames and models,
we have the following results (see e.g.~\cite{Pac}
%\nb{T: Changed. In Chellas there are no correspondence theorems. Ref to Pacuit. \\ M: Thanks}
for the propositional case).


%\nb{T: Red text for necessitation. Is it ok? Or how should I write it? \\ M: I think it's ok}
%

%\todo{M: fixed varying domain case}


\begin{restatable}{proposition}{PropCorresp}\label{prop:corresp}
%The following statements hold.
%\begin{enumerate}
%	\item
	Given %$L \in \mathsf{Pantheon}$ and 
	a neighbourhood frame $\Fmc$, %we have that
the $\Lvar$-principle holds in $\Fmc$ iff $\Fmc$ satisfies the $\Lvar$-condition.
%	\item Given a relational frame $\Fmf$ and a relational model $\Mmf$ based on $\Fmf$, the $\mathit{EMCN}$-principle holds in $\Mmf$, and hence on $\Fmf$.
%%	\item Given a relational frame $\Fmf$, the $\mathit{EMCN}$-principle holds on $\Fmf$. Moreover, for every relational model $\Mmf$, the $\mathit{EMCN}$-principle holds in $\Mmf$.
%\end{enumerate}
\end{restatable}
%\begin{restatable}{theorem}{PropValid}\label{prop:dlvalid}
%Given a neighbourhood frame $\Fmc$, we have that:
%%
%	$(i)$ congruence holds on $\Fmc$;
%	$(ii)$ monotonicity holds on $\Fmc$ iff $\Fmc$ is supplemented;
%	$(iii)$ agglomeration holds on $\Fmc$ iff $\Fmc$ is closed under intersection;
%	$(iv)$ necessitation holds on $\Fmc$ iff $\Fmc$ contains the unit.
%Given a relational frame $\Fmf$, congruence, monotonicity, agglomeration, and necessitation hold on $\Fmf$; moreover, for every relational model $\Mmf$, they hold in $\Mmf$.
%\end{restatable}
%










\begin{restatable}{proposition}{PropValid}
\label{prop:valid}
%
%Given $L \in \mathsf{Pantheon}$, 
The following statements hold.
%
\begin{enumerate}
	\item For a (varying or constant domain) neighbourhood model $\Mmc$, we have that if $\Mmc$ satisfies the $\Lvar$-condition, then the $\Lvar$-principle holds in $\Mmc$.
	However, in general, the converse is not true.
	\item For a relational frame $\Fmf$ and a relational model $\Mmf$ based on $\Fmf$, the $\mathbf{EMCN}$-principle holds in $\Mmf$, hence in $\Fmf$.
%	Moreover, in $\Mmf$, and hence in $\Fmf$, we have that:
%	\begin{itemize}
%		\item  if the $\mathbf{T}$-principle holds, then the $\mathbf{D}$-principle holds;
%		\item the $\mathbf{D}$-principle holds iff the
%	$\mathbf{P}$-principle holds;
%		\item the $\mathbf{Q}$-principle does not hold.
%	\end{itemize}
%	\todo{AM: mention in proof}
%	}}
	Moreover, in $\Mmf$, hence in $\Fmf$,  the $\mathbf{D}$-principle holds iff the
	$\mathbf{P}$-principle holds, and the $\mathbf{Q}$-principle does not hold.
	%\todo{AM: mention in proof. T: proof added, removed claim about the T principle as I don't find it relevant (same behaviour as in the neighbourhood semantics)}
\end{enumerate}
%
\end{restatable}
%















%{\color{blue}{
%As a difference with neighbourhood frames, correspondence between 
%satisfaction of semantic conditions and validity of the corresponding principles does not generally holds for neighbourhood models.
%In particular, it is not always the case that 
%if an $\Lvar$-principle holds in a model $\Mmc$,
%then $\Mmc$ satisfies the $\Lvar$-condition.
%}}



%%%%%%%%%%%%%%%%%%%%%%%%%%%%%%%%%%%%%%%%%%%%%%%%%%%%%%%%%%%%%%%%%%%%%%
\endinput

%%% Local Variables:
%%% mode: latex
%%% TeX-master: "dl18"
%%% End:
