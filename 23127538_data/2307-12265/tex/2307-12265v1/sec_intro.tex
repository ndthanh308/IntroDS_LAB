\section{Introduction}


%%\nb{M: todo fix}
%At the propositional level,
%%(multi-modal)
%logics $\E^{n}$ and $\EM^{n}$ have both been used as a basis for weak deontic systems~\cite{AngEtAl,Che} (although $\EM^{n}$ suffers from several problems discussed in Section~\ref{sec:problem}), as well as to interpret praxeological operators, such as `agent $i$ has the ability to bring about $\p$'~\cite{Bro,Pac}.
%%
%Moreover, $\EM^{n}$ has been combined with $\ALC$, as a basis for further coalition logic 
%extensions of description logic languages~\cite{SeyJam09,SeyJam10}, and $\E^{n}$ modal operators have been applied 
%over $\ALC$ axioms to formalise reasoning about agents' intentions~\cite{ErdSey08} (however, without 
%establishing tight complexity results).
%%\nb{O: added}
%\\

%\todo[inline,caption={}]{
%Short-term todos
%\begin{itemize}
%	\item Individual names and assertions treated uniformly (add everywhere or remove from the paper?)
%	\item Other DLs based on combinations of systems from the classical cube ($\mathbf{MC}, \mathbf{MN}, \mathbf{CN}, \mathbf{MCN}$) + DLs based on combinations with other principles ($\mathbf{D}, \mathbf{T}, \mathbf{P}, \mathbf{Q}$)
%%	\item Tableaux for DL extensions of Elgesem's and Troquard logics of (coalitional) agency and ability
%	\item ``Blocks'' in tableaux rules
%	\item Satisfiability for fragments without modalised concepts on varying domains
%\end{itemize}
%}







%\todo[inline]{M: Introduction DL21}
%
%
%%\nb{M: Change! Add other applications}
%Several approaches to the formal study of normative, epistemic and action-based notions
%%such as obligations and permissions,
%are based on
%%a deontic interpretation of
%modal logic (ML) operators~\cite{Che,GabEtAl,Han}.
%%HilMcn.
%%~\cite{HilMcn}.
%%operators
%In the normative setting, for instance,
%the so-called \emph{standard deontic logic}
%(SDL)
%extends propositional logic with unary
%%deontic
%operators,
%%$\B$ (box) and $\D$ (diamond),
%intuitively interpreted as `it is obligatory' and `it is permitted'.
%%~\cite{Han}.
%First-order extensions
%%in a deontic setting
%have been considered as well~\cite{Han}.
%%Mak,Hin,HilMcn
%Research on autonomous systems~\cite{DenEtAl}, machine ethics~\cite{ArkEtAl}, 
%and normative multi-agent systems~\cite{PigTor}
%%BroEtAl,
%is drawing attention to %\nb{O: changed opening to drawing attention}
%%important challenges and application scenarios
%challenging application scenarios for deontic logics in computer science.
%Other
%motivations
%%\nb{M: `application' repeated 3 times, changed}
%%applications are for%\nb{O: changed}
%%provided by 
%come from knowledge management 
%in legal domains (e.g. legal ontologies and expert systems~\cite{Cas,Fra}),
%%Gor
%Semantic Web applications (e.g. legislative XML
%%BoeEtAl1
%and RuleML~\cite{BoeEtAl,LamHas}), as well as verification of normative systems, and modelling of the normative behaviour of organisations (e.g. company policies specifications or contracting~\cite{MeyEtAl}).
%%MeyWie,
%
%The semantics of MLs, and of SDL in particular, is traditionally based on \emph{relational frames}, consisting of a set of possible worlds endowed with a binary accessibility relation~\cite{Che,Han}.
%These structures, used to interpret modal
%operators (e.g. deontic, epistemic, dynamic, etc.), represent the connections between possible
%%alternative
%situations.
%%or state of affairs.
%%`necessarily' (\emph{alethic} modality),
%%`it is known/believed' (\emph{epistemic}/\emph{doxastic}), 
%%`always in the future' (\emph{temporal}), 
%%such as `it is obligatory'
%%(as well as to temporal, e.g. `always in the future', or epistemic/doxastic ones, e.g. `it is known/believed').
%%often represented by the
%%$\B$
%%operator.
%For instance, in SDL, a proposition is said to be obligatory in some possible world $w$, if it holds in all worlds related to $w$, interpreted as morally ideal alternatives to $w$.
%However, all the so-called \emph{normal} MLs, based on this semantics, face the problem of validating principles that, in several applications, can be hardly associated with an acceptable meaning.
%%
%%This is the case for instance of the axiom
%%$\textsf{K}$,
%%$\B (\p \to \psi) \to \B \p \to \B \psi$,
%%and of the inference rule
%%%of \emph{necessitation}
%%$\textsf{RN}$,
%%if
%%$\vdash \p$,
%%then
%%$\vdash \B \p$.
%%
%In SDL, these principles lead to several counter-intuitive conclusions,
%often
%presented in the form of \emph{deontic paradoxes}.
%%or puzzles.
%%(deontic contingency,
%%self-inconsistency and
%%universal obligatoriness,
%%% deontic explosion,
%%Good Samaritan paradox, contrary-to-duty paradox, Ross's paradox,
%%%paradox of commitment, paradox of the gentle murderer
%%etc.).
%%
%For instance, if it is obligatory to perform an action, and if this action always implies a 
%%unwanted
%negative
%consequence, then we are forced to conclude that also the latter is obligatory.
%%(see the \emph{Good Samaritan paradox}~\cite{Han}).
%%
%Problematic arguments like this one represent a strong limitation to the applicability of SDL 
%%in the formal analysis of
%to normative reasoning~\cite{Han}.
%%Cal,DenEtAl,HilMcn
%%As another example, given a deontic reading of
%%$\B$ (`it is obligatory'),
%%For instance,
%%in the case of Ross's paradox,
%%we have that, if it is obligatory to send an email
%%% deliver an order,
%%$\B \p$,
%%then it is also obligatory to send that email or to cancel it,
%%%deliver that order or to cancel it,
%%$\B (\p \lor \psi)$.
%%This
%%%counter-intuitive
%%conclusion
%%%known as \emph{Ross's paradox}~\cite{GabEtAl1},
%%is obtained by the application of $\textsf{RN}$ and $\textsf{K}$ to the tautology
%%$\p \to \p \lor \psi$.
%%
%%In an epistemic setting, these principles lead to the so-called \emph{logical omniscience} problem~\cite{Sim,Var2}, according to which an agent should know every logical validity, or everything implied by their background knowledge.
%
%%\nb{M: Ok}
%To overcome these problems, a different semantics, based on \emph{neighbourhood} 
%(or \emph{minimal}) 
%\emph{models}, has been proposed~\cite{Che}.
%%Sco,Seg,HugCre
%Instead of using a set of worlds endowed with an accessibility relation, this approach associates to each situation $w$ a family of sets of worlds.
%These sets
%%the \emph{neighbors} of $w$
%intuitively represent the propositions that are obligatory (or believed, brought about, etc.) in $w$.
%MLs based on this semantics can satisfy weaker principles, without validating those axioms and rules
%%such as
%%$\Ksf$
%%and
%%$\Rsf\Nsf$,
%that are common to all normal MLs.
%%~\cite{DalEtAl}.
%For this reason, they are called \emph{non-normal MLs}.
%%
%At the propositional level, non-normal MLs based on neighbourhood semantics have received considerable attention~\cite{Pac,DalEtAl},
%%Neg,GilMaf,HanEtAl,Orl
%%both from a proof- and a model-theoretic viewpoint.
%with results reducing validity in propositional non-normal MLs to validity in normal ones~\cite{KraWol,GasHer}.
%To increase the expressive power of these formalisms, first-order non-normal MLs based on neighbourhood semantics have been considered as well~\cite{Waa,CosPac,CalRot}.
%%,Cos,Cos1,Cal,Orl
%
%%\nb{M: Change! Mention literature!}
%%However,
%Not much has been done yet 
%%towards an 
%in applications of non-normal MLs to knowledge representation, in particular, to normative
%%or epistemic
%automated reasoning.
%%To the best of our knowledge,
%%for instance,
%Most of the modal description logics (MDLs) considered in the literature are based on
%%normal modal operators with
%the standard relational semantics~\cite{GabEtAl}.
%%HandbookDL,
%%Ros,DonEtAl,BaaEtAl1,
%Modal extensions of $\ALC$ with neighbourhood semantics have been introduced as a basis of coalition logic~\cite{SeyErd,SeyJam} and agent communication~\cite{ErdSey} languages
%%capable of representing
%%used in strategic reasoning over
%for reasoning over structured domains.
%However, in normative settings, these MDLs still share several problems of propositional normal MLs.
%Failing to
%%formally
%address this issue can lead to serious drawbacks
%%for
%%the application of
%%a logic
%%-based
%%approach
%to
%normative reasoning in knowledge-based systems.
%%\nb{O: changed by knowledge-based (AI system seems a bit vague)}
%%specification and verification of norms
%%normative reasoning
%%in trustworthy AI systems.
%%or to modelling of human-compliant epistemic reasoning in multi-agent systems~\cite{BelLom}.
%%In order to address these issues,
%In this paper we study non-normal MDLs, interpreted over neighbourhood models, satisfying only minimal requirements on the modal operators.
%With these formalisms, counter-intuitive inferences in normative scenarios can be blocked, while still retaining the expressive power needed in knowledge representation.
%%Thus, we introduce here some \emph{non-normal modal DLs} interpreted over
%%%suitably defined
%%neighbourhood models.
%%Using these formalisms, counterintuitive inferences in normative scenarios can be blocked, while still retaining the expressive power needed in knowledge representation.
%%
%%Using a reduction to normal modal DLs,
%%based on relational semantics,
%
%
%%Summary of sections: \ldots
%%\nb{M: Change! Content Section~\ref{sec:problem}}
%In Section~\ref{sec:prelim} we present MDLs, both recalling the standard relational semantics, and introducing the neighbourhood models used for non-normal MDLs.
%In Section~\ref{sec:problem} we model with MDLs a scenario involving normative notions, discussing deontic
%%in this setting the
%paradoxes due to relational semantics, and how they can be blocked using neighbourhood models.
%Then, in Section~\ref{sec:relation}, we study the complexity of the formula satisfiability problem 
%for non-normal MDLs.
%%which avoid the undesired inferences described in Section~\ref{sec:problem}.
%We prove $\NExpTime$-upper bounds for the complexity of the satisfiability problem, showing that reasoning
%%in non-normal modal DLs
%is not harder than in basic (normal) modal DLs with the relational semantics~\cite{GabEtAl}.
%Directions for future work are discussed in Section~\ref{sec:conc}.



%In multi-agent systems and in knowledge representation,
\emph{Modal logics} are powerful tools used to represent and reason about actions and abilities~\cite{Brown,Elg}, coalitions~\cite{Pau,Tro}, knowledge and beliefs~\cite{Ago,Bal,Var1,LismontMongin}, obligations and permissions~\cite{AngEtAl,Gob,Wright}, etc.
In combination with \emph{description logics}, they give rise to \emph{modal description logics}~\cite{WolZak98,GabEtAl03}, knowledge representation formalisms used for modal reasoning over an object domain and with a good trade-off between expressive power and decidability.

The standard \emph{relational semantics} for modal operators is given in terms of \emph{frames} consisting of a set of \emph{possible worlds} equipped with binary \emph{accessibility relations}.
The foundations of modal description logics, so far, have also mostly been studied with relational semantics.
However, all the modal systems interpreted
%under
with respect to
this semantics, known as \emph{normal}, validate principles that have been considered problematic or debatable for agency-based, coalitional, epistemic, or deontic applications, in that they lead to
%counterintuitive or
unacceptable conclusions,
%the unpleasant features discussed in the literature,
%The literature discusses,
e.g.,
%%for instance
%the problem of
\emph{logical omniscience} in epistemic settings~\cite{Var1}, as well as
%a number of so-called
\emph{agency} or \emph{deontic paradoxes} in the representation of agents' abilities~\cite{Elg} and obligations~\cite{Ross,Aqv,For}.




To overcome these problems,
a generalisation of relational semantics, known as \emph{neighbourhood semantics}, was introduced by Scott~\cite{Sco} and Montague~\cite{Mon}.
Since it avoids in general the problematic principles
%that are
validated by relational semantics, it has been used to interpret a number of \emph{non-normal} modal logics, first studied by C.I. Lewis~\cite{CIL}, Lemmon~\cite{Lem}, Kripke~\cite{Kripke},
%Scott~\cite{Sco}, Montague~\cite{Mon}, 
Segerberg~\cite{Seg}, and Chellas~\cite{Che}, among others.
A \emph{neighbourhood frame} consists of a set of worlds, each one associated with a ``neighbourhood'', i.e., a set of subsets of worlds.
%
Intuitively, a subset of worlds can be
thought of as
representing a fact
%\todo{Not all sets of worlds are propositions...}
in a model, namely, those worlds where that fact holds.
%the extension of a
%\nb{M: todo fix}
%%proposition
%fact
%%\todo{Not all sets of worlds are propositions...}
%in a model.
Hence, the idea is that every world is assigned to a collection of
facts,
%propositions,
those
%deemed as
%necessary,
that are
brought about,
known, obligatory, etc.,
in that world of the model.
%

%Non-normal modalities have been widely investigated as a way to extend propositional logic.
These are the neighbourhood semantics ingredients for \emph{propositional} non-normal modal logics.
A further line of research focuses on the behaviour of modal operators interpreted on neighbourhood frames in combination with \emph{first-order} logic.
%
In this direction, completeness results for first-order non-normal modal logics have been provided~\cite{Cos,CosPac}.
%
In addition, \emph{non-normal modal} \emph{description logics}, extending standard description logics, % such as $\ALC$,
with modal operators interpreted on neighbourhood frames, have been considered for knowledge representation applications~\cite{SeyErd09,DalEtAl19,DalEtAl22}, also in multi-agent coalitional settings~\cite{SeyJam09,SeyJam10}.
%\todo{Comment this? ``The focus, in this case, is on maintaining a balance between expressivity of languages and decidability of reasoning tasks (in particular, formula satisfiability), while also overcoming the limitations of relational models in the interpretation of modal operators thanks to neighbourhood semantics.''}






To illustrate the expressivity of non-normal modal description logic languages, as well some of the limitations of relational frames behind adoption of neighbourhood semantics, we provide an example based on a classic multi-agent purchase choreography scenario~\cite{MonEtAl10} (see the Appendix for a detailed version).
%
Our multi-agent
setting involves
a customer $\mathit{c}$ and a seller $\mathit{s}$, as well as
agency operators $\mathbb{D}_i$ and $\mathbb{C}_i$, for $i \in \{ c, s \}$, read as `agent $i$ does/makes' and `agent $i$ can do/make', respectively~\cite{Elg,GovernatoriRotolo}.
%
The formula
$
\label{eq:1eq}
\mathsf{Ord} \equiv \mathbb{D}_{c}\exists \mathsf{req}.(\mathsf{Prod} \sqcap \mathsf{InCatal} )
$
defines an order $\mathsf{Ord}$ as a request made by customer $c$ of an in-catalogue
%($\mathsf{InCatal}$)
product.

%
By stating
$
\label{eq:2eq}
\exists \mathsf{req}.(\mathsf{Prod} \sqcap \mathsf{InCatal} ) \sqsubseteq \mathsf{Confirm} \sqcup \lnot \mathsf{Confirm},
$
%\end{equation}
we can also enforce that any request of an in-catalogue product is either confirmed or not confirmed.
However, relational semantics validates the so-called \emph{$\mathbf{M}$-principle} (often called \emph{monotonicity}) as well, according to which $C \sqsubseteq D$ always entails $\mathbb{D}_{c} C \sqsubseteq \mathbb{D}_{c} D$, for any concepts $C, D$.
Thus,
from the $\mathbf{M}$-principle
and $\mathsf{Ord}$ definition,
%and the previous two formulas,
%from~(\ref{eq:1eq}),~(\ref{eq:2eq}) and the $\mathbf{M}$-principle above,
we obtain
$
\label{eq:monconc}
\mathsf{Ord} \sqsubseteq \mathbb{D}_{c} ( \mathsf{Confirm} \sqcup \lnot \mathsf{Confirm} ),
$
meaning that any order
%made by customer $c$
is made confirmed or not confirmed by $c$. This is an unwanted conclusion in our agency-based scenario, since customers' actions should be unrelated to order confirmation aspects.\footnote{Other approaches (out of the scope of this paper) to avoid such consequences would involve rejecting the principle of \emph{excluded middle}, as done e.g. in \emph{intuitionistic description logics}~\cite{Dep06,BozEtAl07,Sch15}.}


Moreover, the formula
$
\label{eq:agglprem}
\mathsf{SubmitOrd} \sqsubseteq \mathbb{C}_{s}\mathsf{Confirm} \sqcap \mathbb{C}_{s}\mathsf{PartConf} \sqcap \mathbb{C}_{s}\mathsf{Reject}
$
states that a submitted order can be confirmed, can be partially confirmed, and can be rejected by the seller $s$.
On relational frames,
$ \mathbb{C}_{s} C \sqcap \mathbb{C}_{s} D \sqsubseteq \mathbb{C}_{s}( C \sqcap D ) $
 is a valid formula, for any concepts $C, D$, known as the \emph{$\mathbf{C}$-principle} (or \emph{agglomeration}).
 Therefore, by %~(\ref{eq:agglprem}) and
 the $\mathbf{C}$-principle, under relational semantics we would be forced to conclude that
%\begin{equation}
$
\label{eq:agglconc}
\mathsf{SubmitOrd} \sqsubseteq \mathbb{C}_{s}(\mathsf{Confirm} \sqcap \mathsf{PartConf} \sqcap \mathsf{Reject}),
$
%\end{equation}
meaning that any submitted order is such that the seller $s$ has the ability to make it confirmed, partially confirmed, and rejected, all \emph{at once}, which is unreasonable.


Finally, consider the formula
%\begin{equation}
$
\label{eq:necprem}
\top \sqsubseteq \mathsf{Confirm} \sqcup \lnot \mathsf{Confirm},
$
i.e., the truism stating that anything is either confirmed or not confirmed.
%\end{equation}
By the so called \emph{$\mathbf{N}$-principle} (or \emph{necessitation}) of relational semantics, we have that if $\top \sqsubseteq C$ is valid on relational frames, then $\top \sqsubseteq \mathbb{D}_{c} C$ holds as well, for any concept $C$.
Thus, from
the $\mathbf{N}$-principle
it would follow
on relational semantics
that
$
\label{eq:necprem}
\top \sqsubseteq \mathbb{D}_{c} (\mathsf{Confirm} \sqcup \lnot \mathsf{Confirm}),
$
thereby forcing us to the consequence that every object is made by customer $c$ to be either confirmed or not confirmed, hence leading again to an unreasonable connection between customer's actions and confirmation of orders.




The $\mathbb{D}_{i}$ and $\mathbb{C}_{i}$ modalities are axiomatised similarly to~\protect\cite{Elg}, by means of additional principles as well:
$\mathbb{D}_{i}$ obeys the $\mathbf{C}$- (seen above) and \emph{$\mathbf{T}$-principle}
($
\mathbb{D}_{w} C  \sqsubseteq C
$),
stating a \emph{factivity of actions} principle, well-known also in epistemic logic;
and both satisfy the \emph{$\mathbf{Q}$-principle}
($
\label{eq:qprinc}
\top \sqsubseteq \lnot \mathbb{D}_{c} \top
$),
asserting a principle of \emph{impotence towards tautologies} that is unsatisfiable in relational frames, but admissible over neighbourhood ones, and the \emph{$\mathbf{E}$-principle} ($C \equiv D$ entails $\mathbb{D}_{i} C \equiv \mathbb{D}_{i} D$ and $\mathbb{C}_{i} C \equiv \mathbb{C}_{i} D$), valid both on relational and neighbourhood frames.








In this paper, which is an extension of~\cite{DalEtAl19,DalEtAl22},
%~\cite{DalEtAl19}
we investigate reasoning in a family of non-normal modal description logics, providing terminating, sound, and complete tableau algorithms for checking formula satisfiability on neighbourhood models based on \emph{varying domains} of objects.
Moreover,
%In addition,
we study the complexity of reasoning in a restricted fragment that disallows modalities on description logic concepts. Finally, for two modal description logics interpreted on \emph{constant domain} neighbourhood models, we adjust a reduction (known from the propositional case) to satisfiability with respect to standard relational semantics.

The paper is structured as follows.
%Section~\ref{sec:model} introduces, with an agency-based example, the expressivity of the language, and the differences between neighbourhood and relational semantics.
%After this,
Section~\ref{sec:prelim} provides the necessary definitions and the preliminary results on non-normal modal description logics.
In Section~\ref{sec:tableaux} we present the tableau algorithms for the family of logics here considered.
The case of fragments without modalised concepts is then studied in Section~\ref{sec:fragvardom}.
Section~\ref{sec:reasoncondom} contains the results for the constant domain case.
Finally, Section~\ref{sec:discuss} concludes the paper, discussing related work and possible future research directions.
%and Section~\ref{sec:conc}
















