\section{Tableaux for Formula Satisfiability}
%\subsection{Tableaux for $\LnALC$ Formula Satisfiability}
\label{sec:tableaux}

%\subsection{Tableaux for Non-normal Modal Description Logics}
% on Varying Domain Neighbourhood Models
%\section{Reasoning in Non-normal Modal Description Logics}


We provide terminating, sound, and complete tableau algorithms to check satisfiability of formulas in varying domain neighbourhood models. The notation partly adheres to that of~\cite{GabEtAl03},
while the model construction in the soundness proof is based on the strategy of~\cite{DalHyp}.
%\textcolor{red}{To cite: \cite{DL19} our previous work on non-normal DL, \cite{DalHyp} for the definition of the model in Theorem 2}
%\subsection{Tableaux for Non-normal Modal Description Logics}
%{\color{blue}{
%}}
%\todo{T: also $\lor$? M: yup, grazie}
%Adhering to the notation used in~\cite{SeyErd09},
%We require the following preliminary notions.
In this section, we use concepts and formulas in \emph{negation normal form} 
(\emph{NNF}) and, for this reason, we consider all the logical connectives $\sqcup, \lor, \forall, \Diamond$ as primitive, rather than defined. 
%Given an $\MLALC{n}$ formula $\p$, we denote by $\conneg(\p)$ and $\forneg(\p)$ the set of subconcepts and subformulas of $\p$, respectively.
For a concept or formula $\gamma$, we denote by $\dot{\lnot}\gamma$ the negation of $\gamma$ put in {NNF},
%defined as usual.
 defined as follows:
a concept is in \emph{NNF} if negation occurs in it only in front of concept names; a formula is in \emph{NNF} if all concepts in it are in NNF and negation occurs in the formula only in front of CIs or assertions of the form $r(a,b)$ (regarding
assertions of the form $A(a)$, we recall that a formula $\lnot \psi$, with $\psi = C(a)$, is equivalent to the assertion $D(a)$, with $D = \lnot C$).
Given an $\MLALC{n}$ formula $\p$, we assume without loss of generality that $\p$ is in NNF (using De Morgan laws) 
%and the fact that a formula $\lnot \psi$, with $\psi = C(a)$, is equivalent to the assertion $D(a)$, with $D = \lnot C$),
%}} 
%it contains CIs only of the form $\top \sqsubseteq C$, and every concept occurring in $\p$ is also in NNF.
and it contains CIs only of the form $\top \sqsubseteq C$,
since $C \sqsubseteq D$ is equivalent to $\top \sqsubseteq \lnot C \sqcup D$). 
%
We denote by $\con(\p)$ and $\for(\p)$ the set of subconcepts  and subformulas of $\p$, respectively, and then we set
$\conneg(\p) = \con(\p) \cup \{ \dot{\lnot}C \mid C \in \con(\p) \} \cup  \{ \top \}$ and $\forneg(\p) =  \for(\p) \cup \{ \dot{\lnot}\psi \mid \psi \in \for(\p) \}$.
The sets $\rol(\p)$ and $\ind(\p)$ are, respectively, the sets of role names and of  individual names  occurring in $\p$.
%\todo{M: added $\ind(\p)$ -- todo check and fix the rest}
%notion of a tableau for an $\MLALC{n}$ formula.
Let 
%$\fg(\p) = \forneg(\p) \cup \conneg(\p) \cup \rol(\p) \cup \{ \top \}$.
$\fg(\p) = \forneg(\p) \cup \conneg(\p) \cup \rol(\p) \cup \ind(\p)$ be the \emph{fragment induced by $\varphi$}.
%Note that, %as a consequence of 
%by our assumption on the form of CIs in $\p$, we have $\top\in\conneg(\p)$.\nb{M: to fix}
%%\todo{O: $\top$ is primitive? \\ M: No, it is defined, but no problem here (since it is in NNF by definition).}

%\nb{M: to check - merged T. defs. here (+ small changes)}
%{\color{blue}{

Moreover, let $\NV$ be a countable set of \emph{variables}, denoted by the letters $u, v$.
%The \emph{terms for $\p$} are either individual names in $\ind(\p)$ or variables in $\NV$, and they are denoted by the letters $x, y$.
The \emph{terms for $\p$}, denoted by the letters $x, y$, are either individual names in $\ind(\p)$ or variables in $\NV$.
We assume that the set of terms for $\p$ is
%\todo{O: add "strictly" since well-ordering is normally given by the relation $\leq$}
strictly well-ordered by the relation $<$.
In addition, let $\mathsf{N_{L}}$ be a countable set of \emph{labels}.
%well-ordered by the relation $\ll$.
%Old formulation
%Given an $\MLnALC$ formula $\p$, an \emph{$n$-labelled constraint for $\p$} takes the form $n: \psi$, or $n: C(x)$, or $n: r(x, y)$, where $n \in \mathsf{N_{L}}$, $\psi \in \forneg(\p)$,
%$x, y$ are {\color{blue}{terms for $\p$}},
%%$x, y \in \NV$,
%%$C \in \conneg(\p) \cup \{ \top \}$, 
%$C \in \conneg(\p)$, 
%and $r \in \rol(\p)$.
% New formulation (Tiz)
Given an $\MLnALC$ formula $\p$ and a label $n \in \mathsf{N_{L}}$, an \emph{$n$-labelled constraint for $\p$} takes the form $n: \psi$, or $n: C(x)$, or $n: r(x, y)$, where $\psi \in \forneg(\p)$,
$x, y$ are terms for $\p$,
$C \in \conneg(\p)$, 
and $r \in \rol(\p)$.
For every $n \in \mathsf{N_{L}}$, an \emph{$n$-labelled constraint system for $\p$} is a set $S_{n}$ of $n$-labelled constraints for $\p$.
%An \emph{$n$-labelled constraint system for $\p$} is a set $S_{n}$ of $n$-labelled constraints for $\p$.
A \emph{labelled constraint for $\p$} is an $n$-labelled constraint for $\p$, for some $n \in \mathsf{N_{L}}$, and similarly for a \emph{labelled constraint system for $\p$}.
%A \emph{completion set} $\T$ is a
%%set of labelled constraints for $\p$.
%(non-empty) union of labelled constraint systems for $\p$.
A
%\nb{M: changed - simplified (add `non-empty'?)}
\emph{completion set $\T$ for $\p$} is a non-empty
union
%set
of labelled constraint systems for $\p$,
and we set $\mathsf{L}_{\T} = \{ n \in \mathsf{N_{L}} \mid S_{n} \subseteq \T \}$.
%Finally, a set $\T = \bigcup^{m}_{n = 0} S_{n}$,  for some $m \in \mathsf{N_{L}}$, is called \emph{completion set}, where each $S_{n}$ is an $n$-labelled constraint system for $\p$.
%When no confusion can arise, we omit `$n$' and speak of `labelled constraint for $\p$' or of `labelled constraint system for $\p$'.

About terms, we adopt the following terminology.
%\nb{M: todo add blocking (cf. YB)}
A {{term}} $x$ \emph{occurs in $S_{n}$} if $S_{n}$ contains $n$-labelled constraints of the form $n: C(x)$ or $n: r(\tau,\tau')$, where $\tau = x$, or $\tau' = x$, and $n \in \mathsf{N_{L}}$.
In addition, a variable $u$ is said to be \emph{fresh for $S_{n}$} if $u$ does not occur in $S_{n}$.
%and
%{\color{red}{
%\todo{M: why? to discuss}
%$u > v$, for every $v$ that occurs in $S$.}}
(These notions can be used with respect to $\T$, whenever $S_{n} \subseteq \T$).
%{\color{red}{
%\todo{M: to remove}
%Without loss of generality, we assume that, whenever $x$ occurs in $S_{n}$, the $n$-labelled constraint $n: \top(x)$ is in $S_{n}$.
%}}
%{\color{red}{
%\todo{M: not used, remove?}
%Also, if $n : r(x, y) \in S_{n}$, we call $y$ an \emph{$r$-successor of $x$} with respect to $S_{n}$.
%}}
Finally, given variables $u, v$ in an $n$-labelled constraint system $S_{n}$, we say that $u$ is \emph{blocked by $v$ in $S_{n}$} if $u > v$ and $\{ C \mid n : C(u) \in S_{n} \} \subseteq \{ C \mid n : C(v) \in S_{n} \}$.



A completion set $\T$ contains a \emph{clash} if,
%{\color{blue}{
for some $m \in \mathsf{N_{L}}$,
%formula $\psi$,
concept $C$, role $r$, and terms $x, y$, one of the following holds:
%$\{m: \psi, m: \neg\psi\} \subseteq \T$,
%or
%$\{m: C(x), m: \lnot C(x)\} \subseteq \T$,
%for some $m \in \mathsf{N_{L}}$, and formula $\psi$ or concept $C$.
%$\mathsf{false} \in \T$;
%or
%$\bot(x)$;
%or
$\{m: (\top \sqsubseteq C), m: \lnot (\top \sqsubseteq C) \} \subseteq \T$;
%or
%$\{m: \psi, m: \neg\psi\} \subseteq \T$;
or
$\{m: A(x), m: \lnot A(x)\} \subseteq \T$;
or
$\{m: r(x,y), m: \lnot r(x,y)\} \subseteq \T$.
A completion set that does not contain a clash is \emph{clash-free}.
%}}
%Given a constraint system $S$ for $\p$, we say that $S$ contains a \emph{clash} if there exist a variable $x$ and a concept $C$ such that $\{ x: C, x: \lnot C \} \subseteq S$, or if $\{ \psi, \lnot \psi \} \subseteq S$, for some formula $\psi$.
%\todo{M: extend to $\mathit{L} \in \mathsf{Pantheon}$}
%

For every $\mathit{L} \in \Log$, we associate to $\Lvar$ the set of \emph{$\LnALC$-rules} from Figure~\ref{fig:rules} (bottom part) containing
$\mathsf{R}_{\land}$,
$\mathsf{R}_{\sqcap}$,
$\mathsf{R}_{\lor}$,
$\mathsf{R}_{\sqcup}$,
$\mathsf{R}_{\exists}$,
$\mathsf{R}_{\forall}$,
$\mathsf{R}_{\sqsubseteq}$,
$\mathsf{R}_{\not\sqsubseteq}$,
$\mathsf{R}_{\mathit{L}}$,
and $\mathsf{R}_{\mathit{L}\mathbf{X}}$, for every $\mathbf{X}\in\{\mathbf{N,T,P,Q,D}\}$ such that $\mathbf{X}\in\Lvar$. 
%
Given $\mathit{L} \in \Log$, a completion set $\T$ is $\LnALC$-\emph{complete} if no 
%\emph{$\LnALC$-rule} from Figure~\ref{fig:rules}
$\LnALC$-rule is applicable to $\T$,
%{\color{red}{
where $\gamma_{j}$ is either $\psi_{j} \in \forneg(\p)$ or $C_{j}(x_{j})$, with $C_{j} \in \conneg(\p)$, for $j = 1, \ldots, k$, and $\delta$ is either $\chi \in \forneg(\p)$ or $D(y)$, with $D \in \conneg(\p)$,
%}}
with respect to the \emph{application conditions} associated to each $\LnALC$-rule
from Figure~\ref{fig:rules} (top part).
%\nb{M: todo}
%\begin{itemize}
%
% Figure environment removed
%
%% T: Notion of completability is unnecessary
%{\color{blue}{Finally,  a completion set $\T$ is $\LnALC$-\emph{completable}
%if there exists $\T'\supseteq\T$ such that $\T'$ is clash-free and 
%$\LnALC$-complete,
%otherwise it is $\LnALC$-\emph{uncompletable}}}
%
The $\LnALC$-rules essentially state how to extend a completion set on the basis of the information contained in it.
Branching rules entail a \emph{non-deterministic choice} in the  completion set expansion.

%old sentence
%For each $\Lvar\in \Log$, we now define an algorithm based on $\LnALC$-rules for checking the
%$\LnALC$ formula satisfiability.
%We then prove that the algorithm terminates for every formula $\p$,
%and that it is sound and complete with respect to $\LnALC$ formula satisfiability.

For each $\Lvar\in \Log$,
we now present a
tableau-based non-deterministic decision procedure %$sat(\p)$
for the $\LnALC$ formula satisfiability problem on varying domain neighbourhood models,
%and
based on
%shown in
Algorithm~\ref{alg:tableau}
%which is
(simply referred to as
%the
\emph{$\LnALC$ tableau algorithm}).
%For each $\Lvar\in \Log$,
%we now present a
%tableau-based non-deterministic decision procedure %$sat(\p)$
%for the $\LnALC$ formula satisfiability problem on varying domain neighbourhood models,
%relying on Algorithm~\ref{alg:tableau}.
%based on
%%simply referred to as
%the
%\emph{$\LnALC$ tableau algorithm}
%%and
%shown
%in Algorithm~\ref{alg:tableau}.
%Observe that %Algorithm~\ref{alg:tableau} 
We have that a formula $\p$ is $\LnALC$ satisfiable if and only if there exists at least one execution of the
$\LnALC$ tableau algorithm
%for $\p$
that constructs an $\LnALC$-complete and clash-free completion set for $\p$.
%If an execution of the
%$\LnALC$ tableau algorithm for $\p$ constructs an $\LnALC$-complete and clash-free completion set for $\p$, then $\p$ is satisfiable.
%On the other hand, if all of its executions return a completion set for $\p$ that contain a clash, then $\p$ is unsatisfiable.
This non-deterministic algorithm
gives priority to non-generating $\LnALC$-rules,
i.e., those that do not introduce new variables or labels,
with respect to generating ones,
so to minimise the size of the completion set constructed by its application,
%and  it is sound and complete with respect to satisfiability in $\LnALC$.
%This algorithm 
and
terminates in exponential time 
for every formula $\p$.
Thus, we obtain the following.
%\nb{M: todo fix notation in Preliminaries}
%The $\LnALC$ tableau algorithm for $\p$ returns $\mathsf{satisfiable}$ if and only if %the procedure 
%%$exp(\T_{\p})$
%it constructs 
% an $\LnALC$-complete  and clash-free completion set for $\p$.
%This non-deterministic algorithm
%terminates in exponential time 
%for every formula $\p$,
%and  it is sound and complete with respect to satisfiability in $\LnALC$.
%Thus, we obtain the following.
%%\nb{M: todo fix notation in Preliminaries}

%\todo{T:All claims of soundness, completeness and termination disappeared. Is this on purpose? O: We say that in the text already}

\begin{theorem}
	\label{thm:upperbound}
	Satisfiability in $\LnALC$  on varying domain neighbourhood models is decidable in $\NExpTime$.
\end{theorem}

%%%%%%%%%%%%% 
%In the rest of this section,
%we prove that this non-deterministic algorithm
%terminates in exponential time 
%for every formula $\p$,
%and that it is sound and complete with respect to $\LnALC$ formula satisfiability.
%We start by establishing termination of the $\LnALC$ tableau algorithm.
%%%%%%%%%%%%
%For each $\Lvar\in \Log$,
%we now present
%a
%tableau-based
%{\color{blue}{non-deterministic}} algorithm {\color{blue}{$sat(\p)$}} based on $\LnALC$-rules for 
%{\color{blue}{deciding the $\LnALC$ formula satisfiability problem on varying domain neighbourhood models, shown  in Algorithm~\ref{alg:tableau}.}}
%We then prove that the algorithm terminates in exponential time 
%%\todo{O: added "in exponential time". T: thanks, added also ``non-deterministic''.}
%for every formula $\p$,
%and that it is sound and complete with respect to $\LnALC$ formula satisfiability.


%% OLD FORMULATION OF THE TABLEAU ALGORITHM FOR THE FORMULA SATISFIABILITY PROBLEM
%{\color{blue}{
%\begin{definition}[$\LnALC$ tableau algorithm for $\p$]
%Given an $\MLnALC$ formula $\p$, the \emph{$\LnALC$ tableau algorithm for $\p$} 
%starts with the initial completion set $\T_{\p} =  \{0 : \p\}$,
%and expands $\T_{\p}$ by means of the procedure $sat(\T_{\p})$ below.
%\todo{M: to discuss}
%
%\begin{algorithm}
%%  \KwIn{The initial completion set $\T_{\p} =  \{0 : \p\}$.
%%}
%%  \KwOut{``no'' if $\p$ is not $\LnALC$-satisfiable, 
%%  an $\LnALC$-complete completion set otherwise.}
%  \KwIn{a completion set $\T$.
%}
%%  \KwOut{``no'' if $\T$ is not $\LnALC$-satisfiable, 
%%  an $\LnALC$-complete completion set $\T' \supseteq \T$ otherwise.}
%  \KwOut{$\mathsf{satisfiable}$, if $\T$ is $\LnALC$-satisfiable;\newline$\mathsf{unsatisfiable}$, otherwise.}
%  \BlankLine
%  \uIf{$\T$ contains a clash}{return $\mathsf{unsatisfiable}$\;} 
%%  \uElseIf{a non-generating rule $\mathsf{R}$ is applicable to $\T$}{apply $\mathsf{R}$ to $\T$\;
%%   return $sat(\T)$\;}
%    \uElseIf{a rule $\mathsf{R} \in\{\mathsf{R}_{\land}, \mathsf{R}_{\lor}, \mathsf{R}_{\sqcap}, \mathsf{R}_{\sqcup}, \mathsf{R}_{\forall}, \mathsf{R}_{\sqsubseteq}, \mathsf{R}_{\mathit{L}\mathbf{T}} \}$ is applicable to $\T$}{apply $\mathsf{R}$ to $\T$\;
%    return $sat(\T)$\;}
%%    \uElseIf{a (modal or non-modal) generating rule $\mathsf{R}$ is applicable to $\T$}{apply $\mathsf{R}$ to $\T$\;
%%  return $sat(\T)$\;}
%      \uElseIf{a rule $\mathsf{R} \in\{\mathsf{R}_{\exists},  \mathsf{R}_{\mathit{L}}, \mathsf{R}_{\mathit{L}\mathbf{N}}, \mathsf{R}_{\mathit{L}\mathbf{P}}, \mathsf{R}_{\mathit{L}\mathbf{Q}}, \mathsf{R}_{\mathit{L}\mathbf{D}} \}$ is applicable to $\T$}{apply $\mathsf{R}$ to $\T$\;
%    return $sat(\T)$\;}
%  \Else{return $\mathsf{satisfiable}$\;}
%%\caption{$\LnALC$ tableau algorithm for $\p$}
%\caption{Decision procedure $sat(\T)$ for $\LnALC$ {\color{red}{formula (T:?)}} satisfiability problem on varying domain neighbourhood models}
%\label{alg:tableau}
%\end{algorithm}
%\end{definition}
%}}
%
%{\color{blue}{
%The algorithm gives priority to non-generating rules,
%that is those that do not introduce new variables or labels,
%with respect to generating ones,
%in order to minimize the size of the completion set constructed by its application.
%Observe that %Algorithm~\ref{alg:tableau} 
%the $\LnALC$ tableau algorithm for $\p$ returns $\mathsf{satisfiable}$ if and only if the procedure constructs 
% an $\LnALC$-complete  and clash-free completion set for $\p$.
%}}



%
%{\color{blue}{
%\begin{definition}[$\LnALC$ tableau algorithm for $\p$]
%Given an $\MLnALC$ formula $\p$, the \emph{$\LnALC$ tableau algorithm for $\p$} 
%%starts calling
%calls the sub-procedure $exp(\T_{\p})$ on the 
%initial completion set $\T_{\p} =  \{0 : \p\}$,
%%and expands $\T_{\p}$ by means of the procedure $sat(\T_{\p})$ below.
%%then
%and it returns $\mathsf{satisfiable}$ if
%$exp(\T_{\p})$ returns $\mathsf{completable}$,
%$\mathsf{unsatisfiable}$ otherwise.
%%and it returns $\mathsf{unsatisfiable}$ if
%%$exp(\T_{\p})$ returns $\mathsf{uncompletable}$.
%%\todo{M: to discuss}
%}}


\begin{algorithm}[t]
  \KwIn{the initial completion set $\T := \{0 : \p\}$ of an $\MLALC{n}$ formula $\p$ in NNF.}
%  \KwIn{an $\MLALC{n}$ formula $\p$ in NNF.}
  \KwOut{a completion set for $\p$, extending the initial one, that either contains a clash, or is complete and clash-free.}
%  \KwOut{$\mathsf{satisfiable}$, if $\p$ is $\LnALC$-satisfiable;\newline$\mathsf{unsatisfiable}$, otherwise.}
%  \BlankLine
%  $\T := \{0 : \p\}$\;
    \BlankLine
  \While{$\T$ is clash-free and not $\LnALC$-complete}{
  	\uIf{a rule $\mathsf{R} \in\{\mathsf{R}_{\land}, \mathsf{R}_{\lor}, \mathsf{R}_{\sqcap}, \mathsf{R}_{\sqcup}, \mathsf{R}_{\forall}, \mathsf{R}_{\sqsubseteq}, \mathsf{R}_{\mathit{L}\mathbf{T}} \}$ is applicable to $\T$}{apply $\mathsf{R}$ to $\T$\;}
      \uElseIf{a rule $\mathsf{R} \in\{\mathsf{R}_{\exists},  \mathsf{R}_{\mathit{L}}, \mathsf{R}_{\mathit{L}\mathbf{N}}, \mathsf{R}_{\mathit{L}\mathbf{P}}, \mathsf{R}_{\mathit{L}\mathbf{Q}}, \mathsf{R}_{\mathit{L}\mathbf{D}} \}$ is applicable to $\T$}{apply $\mathsf{R}$ to $\T$\;}
    }
%    \Return $\T$
%      \BlankLine
%  \uIf{$\T$ contains a clash}{\Return $\mathsf{unsatisfiable}$\;} 
%  \Else{\Return $\mathsf{satisfiable}$\;}
%\caption{$\LnALC$ tableau algorithm for $\p$}
\caption{
$\LnALC$ tableau algorithm
on varying domain neighbourhood models
for $\p$
%Tableau-based decision procedure for the $\LnALC$ formula satisfiability problem on varying domain neighbourhood models
 }
\label{alg:tableau}
\end{algorithm}









%In the rest of this section, we prove termination, soundness and completeness of the tableau algorithm given above. \todo{O: replace "above" with a reference to the algorithm?. T: thanks, fixed.}
%We start by showing that the $\LnALC$ tableau algorithm terminates.
%
%
%
%%% TERMINATION
%

%\newpage
%
%\subsection*{\textcolor{red}{Tentative Termination}}
%Let $\mathbf{T}$ be a completion set 
%constructed
%expanding the initial completion set $\mathbf{T}_{\p} =  \{0 : \p, 0 : \top(x) \}$ according to the 
%$\LnALC$ tableau algorithm for $\p$.
%%\textcolor{red}{We define $\mathsf{L}_{\mathbf{T}} = \{ n \in \mathsf{N_{L}} \mid S_{n} \subseteq \mathbf{T} \}$.}
%We prove the following two claims.




As an immediate consequence of the %above results 
correctness of the tableau we also obtain 
a (constructive) proof of the following kind of \emph{exponential model property}.
%A (constructive) proof of the following kind of \emph{exponential model property} is an immediate consequence of the above correctness result.
%\begin{corollary}
%For each $\mathit{L} \in \Log$,
%every $\LnALC$-satisfiable formula $\p$ has a model of at most exponential size in the length of $\p$.
%\end{corollary}
%\begin{corollary}
%For each $\mathit{L} \in \{\mathbf{E},\mathbf{M},\mathbf{N}\}$,
%every $\LnALC$-satisfiable formula $\p$ has a model 
%%of at most exponential size in the length of $\p$.
%with at most $p(|\fg(\p)|)$ worlds, each of them with a domain of at most 
%$2^{q(|\fg(\p)|)}$ elements, where $p$ and $q$ are polynomial functions.
%Moreover, for $\mathit{L} = \mathbf{C}$,
%every $\LnALC$-satisfiable formula $\p$ has a model 
%with at most $2^{p(|\fg(\p)|)}$ worlds, each of them with a domain of at most 
%$2^{q(|\fg(\p)|)}$ elements, where $p$ and $q$ are polynomial functions.
%\end{corollary}


%% Old corollary
%\begin{corollary}
%For $\mathit{L} \in \{\mathbf{E},\mathbf{M},\mathbf{N}\}$
%(respectively, $\mathit{L} = \mathbf{C}$),
%every $\LnALC$ satisfiable formula $\p$ has a model 
%%of at most exponential size in the length of $\p$.
%with at most $p(|\fg(\p)|)$ (respectively, at most $2^{p(|\fg(\p)|)}$) worlds, each of them having a domain with at most 
%$2^{q(|\fg(\p)|)}$ elements, where $p$ and $q$ are polynomial functions.
%\end{corollary}
%\begin{proof}
%By Theorem \ref{thm:completeness}, if $\p$ is $\LnALC$ satisfiable, then 
%there is a $\LnALC$-complete and clash-free completion set $\mathbf{T}$ for it.
%Then by Theorem \ref{thm:soundness},
%%basing on $\mathbf{T}$ we can define a
%there exists a model 
%$\Mmc = (\Wmc, \{ \Nmc_{i} \}_{i \in I}, \Imc)$
%for $\p$ where $\Wmc =  \mathsf{L}_{\mathbf{T}}$
%and for each $n\in\Wmc$, $\Delta_{n} = \{ x \in \mathsf{N_{V}} \mid x \ \text{occurs in} \ S_{n} \}$.
%By Theorem~\ref{thm:termination}, Claim~\ref{cla:termglobal}, it follows
%$|\Wmc| \leq |\fg(\p)|^2$ for $\mathit{L} \in \{\mathbf{E}, \mathbf{M}, \mathbf{N}\}$,
%and 
%$|\Wmc| \leq 2^{|\fg(\p)|} \cdot |\fg(\p)|$ for $\mathit{L} = \mathbf{C}$,
%finally by Theorem~\ref{thm:termination}, Claim~\ref{cla:termlocal}, 
%for each $n\in\Wmc$, $|\Delta_n|$ does not exceed $2^{q(|\fg(\p)|)}$,
%where $p$ and $q$ are polynomial functions.
%\end{proof}

%{\color{blue}{
\begin{restatable}{corollary}{Fmp}
Every $\LnALC$ satisfiable formula $\p$ has a model 
with at most $p(|\fg(\p)|)$ worlds,
if $\mathbf{C}\notin\Lvar$, 
and at most $2^{q(|\fg(\p)|)}$) worlds, 
if $\mathbf{C}\in\Lvar$,
each of them having a domain with at most 
$2^{r(|\fg(\p)|)}$ elements, with $p$, $q$, $r$ polynomial functions.
\end{restatable}
%
%}}

%where we set:
%		\begin{itemize}
%			\item for $\mathit{L} = \mathbf{E}$, $k = l = 1$;
%			\item for $\mathit{L} = \mathbf{M}$, $k = 1$, $l = 0$;
%			\item for $\mathit{L} = \mathbf{C}$, $k \geq 1$, $l = k$;
%			\item for $\mathit{L} = \mathbf{N}$, $k = l = 0$.
%		\end{itemize}


%\paragraph{Local expansion rules for $\LnALC$}
%
%\begin{itemize}
%	\item[$\mathsf{R}_{\land}$] If $\psi \land \chi \in S$ and $\{ \psi, \chi \} \not\subseteq S$, then set $S := S \cup \{ \psi, \chi \}$.
%	\item[$\mathsf{R}_{\lor}$] If $\psi \lor \chi \in S$ and $\{ \psi, \chi \} \cap S = \emptyset$, then set $S := S \cup \{ \vartheta \}$, where $\vartheta = \psi$ or $\vartheta = \chi$.
%	\item[$\mathsf{R}_{\sqcap}$] If $x : C \sqcap D \in S$ and $\{ x: C, x: D \} \not\subseteq S$, then set $S := S \cup \{ x: C, x: D \}$.
%	\item[$\mathsf{R}_{\sqcup}$] If $x: C \sqcup D \in S$ and $\{  x: C, x: D \} \cap S = \emptyset$, then set $S := S \cup \{ x : E \}$, where $E = C$ or $E = D$.
%	\item[$\mathsf{R}_{\exists}$] If $x : \exists r.C \in S$, $x$ is not blocked with respect to $S$, and there is no $r$-successor $y$ of $x$ with respect to $S$ such that $y : C \in S$, then
%	choose a fresh $y$ for $S$ and set $S := S \cup \{ (x,y) : r, y: C \}$.
%	\item[$\mathsf{R}_{\forall}$] If $x: \forall r.C$ and there is an $r$-successor $y$ of $x$ with respect to $S$ such that $y: C \not \in S$,
%	then set $S := S \cup \{ y : C \}$.
%	\item[$\mathsf{R}_{=}$] If $\top \sqsubseteq C \in S$ and $x: C \not \in S$, for a variable $x$ that occurs in $S$, then set $S := S \cup \{ x: C \}$.
%	\item[$\mathsf{R}_{\neq}$]  If $\lnot (\top \sqsubseteq C) \in S$ and there is no variable $x$ such that $x: \dot{\lnot}C \in S$, for a variable $x$ that occurs in $S$, then choose a fresh variable $x$ for $S$ and set $S := S \cup \{ x : \dot{\lnot}C \}$.
%\end{itemize}
%
%\paragraph{Global expansion rules for $\LnALC$}
%
%\begin{itemize}
%		\item[$\mathsf{R}_{\mathit{L}f}$]
%		If $\{ \Box_{i} \psi_{1}, \ldots, \Box_{i} \psi_{n}, \Diamond_{i} \chi \} \subseteq S$, and,
%			for every $S' \in \mathbf{T}$, % such that $S \neq S'$,
%			all of the following hold:
%				\begin{itemize}
%					\item[($0$)] $\{ \psi_{1}, \ldots, \psi_{n}, \chi \} \not \subseteq S'$; and
%					\item[($1$)] $\{ \lnot \psi_{1}, \lnot \chi \} \not \subseteq S'$; and 
%					\item[$\vdots$]
%					\item[($m$)]   $\{ \lnot \psi_{n}, \lnot \chi \} \not \subseteq S'$;
%				\end{itemize}			
%then choose a fresh $x$ for $S$, set new $S'$ such that:  
%				\begin{itemize}
%					\item[($0$)] $S' = \{ \psi_{1}, \ldots, \psi_{n}, \chi, x: \top \}$; or
%					\item[($1$)] $S' = \{ \lnot \psi_{1}, \lnot \chi, x: \top \}$; or
%					\item[$\vdots$]
%					\item[($m$)]  $S' = \{  \lnot \psi_{m}, \lnot \chi, x: \top \}$;
%				\end{itemize}
%and set $\mathbf{T} := \mathbf{T} \cup \{ S' \}$.
%
%
%
%	\item[$\mathsf{R}_{\mathit{L}c}$]
%If $\{ x:  \Box_{i} C_{1}, \ldots,  x: \Box_{i} C_{n}, y: \Diamond_{i} D \} \subseteq S$, and,
%			for every $S' \in \mathbf{T}$,
%			all of the following hold:
%				\begin{itemize}
%					\item[($0$)] $\{ x: C_{1}, \ldots, x: C_{n}, y: D \} \not \subseteq S'$; and
%					\item[($1$)] $\{ x: \lnot C_{1}, y: \lnot D \} \not \subseteq S'$; and 
%					\item[$\vdots$]
%					\item[($m$)]  $\{ x: \lnot C_{m}, y: \lnot D \} \not \subseteq S'$;
%				\end{itemize}			
%then choose a fresh $z$ for $S$, set new $S'$ such that:  
%				\begin{itemize}
%					\item[($0$)] $S' = \{ x: C_{1}, \ldots, x: C_{n}, y: D, z: \top \}$; or
%					\item[($1$)] $S' = \{ x: \lnot C_{1}, y: \lnot D, z: \top \}$; or
%					\item[$\vdots$]
%					\item[($m$)]  $S' = \{ x: \lnot C_{m}, y: \lnot D, z: \top \}$;
%				\end{itemize}	
%and set $\mathbf{T} := \mathbf{T} \cup \{ S' \}$.
%\end{itemize}
%
%where we set:
%		\begin{itemize}
%			\item for $\mathit{L} = \mathbf{E}$, $n = m = 1$;
%			\item for $\mathit{L} = \mathbf{M}$, $n = 1$, $m = 0$;
%			\item for $\mathit{L} = \mathbf{C}$, $n \geq 1$, $m = n$;
%			\item for $\mathit{L} = \mathbf{N}$, $n = m = 0$.
%		\end{itemize}
%
%
%
%
%\begin{itemize}
%%	\item[$\mathsf{R}_{\mathit{L}f}$]
%%		\begin{itemize}
%%		
%%			\item $\mathit{L} = \mathbf{E}$
%%			
%%			If $\{ \Box_{i} \psi, \Diamond_{i} \chi \} \subseteq S$ and both $\{ \psi, \chi \} \not \subseteq S'$ and $\{\lnot \psi, \lnot \chi \} \not \subseteq S'$, for every $S' \in \mathbf{T}$ such that $S \neq S'$,
%%			then choose a fresh $x$ for $S$, set new $S' = \{ \psi, \chi, x: \top \}$ and $S'' = \{ \lnot \psi, \lnot \chi, x : \top \}$, and set $\mathbf{T} := \mathbf{T} \cup \{ S^* \}$, where $S^* = S'$ or $S^* = S''$.
%%			
%%			\item $\mathit{L} = \mathbf{M}$
%%			
%%			If $\{ \Box_{i} \psi, \Diamond_{i} \chi \} \subseteq S$ and $\{ \psi, \chi \} \not \subseteq S'$, for every $S' \in \mathbf{T}$ such that $S \neq S'$,
%%			then choose a fresh $x$ for $S$, set new $S' = \{ \psi, \chi, x: \top \}$, and set $\mathbf{T} := \mathbf{T} \cup \{ S' \}$.
%%			
%%			\item $\mathit{L} = \mathbf{C}$
%%			
%%			For every $n \geq 1$, if $\{ \Box_{i} \psi_{1}, \ldots, \Box_{i} \psi_{n}, \Diamond_{i} \chi \} \subseteq S$, and,
%%			for every $S' \in \mathbf{T}$ such that $S \neq S'$,
%%			all of the following hold:
%%				\begin{itemize}
%%					\item $\{ \psi_{1}, \ldots, \psi_{n}, \chi \} \not \subseteq S'$; and
%%					\item $\{ \lnot \psi_{1}, \lnot \chi \} \not \subseteq S'$; and \ldots; and $\{ \lnot \psi_{n}, \lnot \chi \} \not \subseteq S'$;
%%				\end{itemize}			
%%then choose a fresh $x$ for $S$, set nen :  
%%				\begin{itemize}
%%					\item $S' = \{ \psi_{1}, \ldots, \psi_{n}, \chi, x: \top \}$;
%%					\item $S''_{1} = \{ \lnot \psi_{1}, \lnot \chi, x: \top \}$; \ldots; $S''_{n} = \{  \lnot \psi_{n}, \lnot \chi, x: \top \}$;
%%				\end{itemize}
%%and set $\mathbf{T} := \mathbf{T} \cup \{ S^* \}$, where $S^* = S'$ or $S^* = S''_{i}$, for some $i \in \{ 1, \ldots, n \}$.
%%
%%			\item $\mathit{L} = \mathbf{N}$
%%			
%%			If $\Box_{i} (\top \sqsubseteq \top) \not\in S$,
%%			then set $S := S \cup \{ \Box_{i} (\top \sqsubseteq \top) \}$.					
%%%			If $\Box_{i} (\top \sqsubseteq \top) \not\in S$,
%%%			then choose a fresh $x$ for $S$, set \nb{M: to discuss} new $S' = \{ \Box_{i} (\top \sqsubseteq \top), x: \top \}$, and set $\mathbf{T} := \mathbf{T} \cup \{ S' \}$.			
%%			
%%			\item $\mathit{L} = \mathbf{MC}$
%%			
%%			For every $n \geq 1$, if $\{ \Box_{i} \psi_{1}, \ldots, \Box_{i} \psi_{n}, \Diamond_{i} \chi \} \subseteq S$, and
%%			$\{ \psi_{1}, \ldots, \psi_{n}, \chi \} \not \subseteq S'$,
%%			for every $S' \in \mathbf{T}$ such that $S \neq S'$,
%%then choose a fresh $x$ for $S$, set new $S' = \{ \psi_{1}, \ldots, \psi_{n}, \chi, x: \top \}$, and set $\mathbf{T} := \mathbf{T} \cup \{ S' \}$.
%%			\item $\mathit{L} = \mathbf{MN}$
%%			
%%			Both $\mathsf{R}_{\mathit{L}f}$ rules for $\mathit{L} = \mathbf{M}$ and $\mathit{L} = \mathbf{N}$.
%%			
%%			\item $\mathit{L} = \mathbf{CN}$
%%			
%%			Both $\mathsf{R}_{\mathit{L}f}$ rules for $\mathit{L} = \mathbf{C}$ and $\mathit{L} = \mathbf{N}$.
%%			
%%		\end{itemize}
%%		
%%	\item[$\mathsf{R}_{\mathit{L}c}$]
%%	
%%		\begin{itemize}
%%			\item $\mathit{L} = \mathbf{E}$
%%			
%%			If $\{ x: \Box_{i} C, y : \Diamond_{i} D \} \subseteq S$ and both $\{ x' : C, y' : D \} \not \subseteq S'$ and $\{ x' : \lnot C, y' : \lnot D \} \not \subseteq S'$, for every $S' \in \mathbf{T}$ such that $S \neq S'$,
%%			then choose a fresh $z$ for $S$, set new $S' = \{ x : C, y : D, z: \top \}$ and $S'' = \{ x : \lnot C, y : \lnot D, z: \top \}$, and set $\mathbf{T} := \mathbf{T} \cup \{ S^* \}$, where $S^* = S'$ or $S^* = S''$.
%%			
%%			\item $\mathit{L} = \mathbf{M}$
%%			
%%			If $\{ x: \Box_{i} C, y : \Diamond_{i} D \} \subseteq S$ and $\{ x' : C, y' : D \} \not \subseteq S'$, for every $S' \in \mathbf{T}$ such that $S \neq S'$,
%%			then choose a fresh $z$ for $S$, set new $S' = \{ x : C, y : D, z: \top \}$, and set $\mathbf{T} := \mathbf{T} \cup \{ S' \}$.
%%			
%%			\item $\mathit{L} = \mathbf{C}$
%%			
%%			For every $n \geq 1$, if $\{ x:  \Box_{i} C_{1}, \ldots,  x: \Box_{i} C_{n}, y: \Diamond_{i} D \} \subseteq S$, and,
%%			for every $S' \in \mathbf{T}$ such that $S \neq S'$,
%%			all of the following hold:
%%				\begin{itemize}
%%					\item $\{ x: C_{1}, \ldots, x: C_{n}, y: D \} \not \subseteq S'$; and
%%					\item $\{ x: \lnot C_{1}, y: \lnot D \} \not \subseteq S'$; and \ldots; and $\{ x: \lnot C_{n}, y: \lnot D \} \not \subseteq S'$;
%%				\end{itemize}			
%%then choose a fresh $z$ for $S$, set nen :  
%%				\begin{itemize}
%%					\item $S' = \{ x: C_{1}, \ldots, x: C_{n}, y: D, z: \top \}$;
%%					\item $S'_{1} = \{ x: \lnot C_{1}, y: \lnot D, z: \top \}$; \ldots; $S'_{n} = \{ x: \lnot C_{n}, y: \lnot D, z: \top \}$;
%%				\end{itemize}
%%and set $\mathbf{T} := \mathbf{T} \cup \{ S^* \}$, where $S^* = S'$ or $S^* = S''_{i}$, for some $i \in \{ 1, \ldots, n \}$.
%%			
%%			\item $\mathit{L} = \mathbf{N}$
%%
%%			If $x: \Box_{i} \top \not\in S$,
%%			then set $S := S \cup \{ x: \Box_{i} \top, \}$.				
%%%			If $x: \Box_{i} \top \not\in S$,
%%%			then choose a fresh $z$ for $S$, set \nb{M: to discuss} new $S' = \{ x: \Box_{i} \top, z: \top \}$, and set $\mathbf{T} := \mathbf{T} \cup \{ S' \}$.			
%%			
%%			\item $\mathit{L} = \mathbf{MC}$
%%			
%%			For every $n \geq 1$, if $\{ x:  \Box_{i} C_{1}, \ldots,  x: \Box_{i} C_{n}, y: \Diamond_{i} D \} \subseteq S$, and $\{ x: C_{1}, \ldots, x: C_{n}, y: D \} \not \subseteq S'$,
%%			for every $S' \in \mathbf{T}$ such that $S \neq S'$,	
%%then choose a fresh $z$ for $S$, set new $S' = \{ x: C_{1}, \ldots, x: C_{n}, y: D, z: \top \}$,
%%and set $\mathbf{T} := \mathbf{T} \cup \{ S' \}$.
%%			
%%			\item $\mathit{L} = \mathbf{MN}$
%%			
%%						Both $\mathsf{R}_{\mathit{L}c}$ rules for $\mathit{L} = \mathbf{M}$ and $\mathit{L} = \mathbf{N}$.
%%			
%%			\item $\mathit{L} = \mathbf{CN}$
%%			
%%			Both $\mathsf{R}_{\mathit{L}c}$ rules for $\mathit{L} = \mathbf{C}$ and $\mathit{L} = \mathbf{N}$.
%%			
%%		\end{itemize}
%	\item[$\mathsf{R}_{\mathit{L}fc}$]
%		\begin{itemize}
%			\item $\mathit{L} = \mathbf{E}$ \ldots\nb{M: to add}
%			\item $\mathit{L} = \mathbf{M}$ \ldots
%			\item $\mathit{L} = \mathbf{C}$ \ldots
%			\item $\mathit{L} = \mathbf{N}$ \ldots
%			\item $\mathit{L} = \mathbf{MC}$ \ldots
%			\item $\mathit{L} = \mathbf{MN}$ \ldots
%			\item $\mathit{L} = \mathbf{CN}$ \ldots
%		\end{itemize}
%\end{itemize}
