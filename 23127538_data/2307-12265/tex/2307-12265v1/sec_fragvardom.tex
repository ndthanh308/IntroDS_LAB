%%%%%%%%%%%%%%%%%%%%%%%%%%%%%%%
\section{Fragments without Modalised Concepts}
%\subsection{Fragments without Modalised Concepts on Varying Domain}
\label{sec:fragvardom}
%\subsection{$\LnALCg$ on Varying Domain}
%%%%%%%%%%%%%%%%%%%%%%%%%%%%%%%
Here we study fragments of $\MLnALC$ without modalised concepts.
An \emph{$\MLnALCg$ formula} is defined similarly to the $\MLnALC$ case,
%but inductively built from $\ALC$ \emph{atoms} consisting of CIs (cf.~Section~\ref{sec:prelim}), or \emph{assertions} of the form $A(a)$ and $r(a,b)$, where $A \in \NC$, $r \in \NR$, and $a,b \in \NI$,
%and
by disallowing modalised concepts.
%
Given
$\mathit{L} \in \Log$,
%$\mathit{L} \in \{ \mathbf{E}, \mathbf{M}, \mathbf{C}, \mathbf{N} \}$,
 \emph{satisfiability in $\LnALCg$ on varying} %(respectively, \emph{constant})
  \emph{domain neighbourhood models} is $\MLnALCg$ satisfiability %problem 
  on varying %(respectively, constant) 
  domain neighbourhood models based on neighbourhood frames in the respective class for $\mathit{L}$.
%(cf. Section~\ref{sec:prelim}).
%
%(i.e., built from $\ALC$ concepts and CIs only).
An \emph{$\MLn$ formula}, instead, is defined analogously to $\MLnALCg$, except that we build it from the standard propositional (rather than $\ALC$) language over a countable set of \emph{propositional letters} $\mathsf{N_{P}}$, disjoint from \NC, \NR, and \NI.
%consider only propositional logic (instead of \ALC).
The semantics of
$\MLn$ formulas
is given in terms of \emph{propositional 
neighbourhood models} (or simply \emph{models}) $\Mmc^{\sf P} = (\Wmc, \{ \Nmc_{i} \}_{i \in J}, \Vmc)$,
where $(\Wmc, \{ \Nmc_{i} \}_{i \in J})$ is a neighbourhood frame,
with $J = \{ 1, \ldots, n \}$ in the following,
and $\Vmc: \NPr \rightarrow 2^{\Wmc}$ is a function 
mapping propositional letters
%in $\NPr$
to
sets of worlds
%subsets of the domain of worlds
(see~\cite{Che,Var2}).
%We say that a propositional 
%neighbourhood model is a $\EC^{n}$ or $\EN^{n}$ \emph{model} if it is based on a neighbourhood frame satisfying the corresponding conditions for $\EC^{n}$ and $\EN^{n}$ given in Section~\ref{sec:prelim}, respectively.
\emph{Satisfiability in $\ensuremath{\smash{\mathit{L}^{n}}}$} is satisfiability in $\MLn$ on propositional neighbourhood models based on neighbourhood frames in the respective class for $\mathit{L}$.
A propositional neighbourhood model based on a neighbourhood frame in the respective class for $\mathit{L}$ is called \emph{$\mathit{L}^{n}$ model}.

\newcommand{\setsymbols}{\ensuremath{\Sigma}\xspace}


We
prove
%establish
tight complexity results %$\ExpTime$ upper bounds 
for  %$\EALCg$  and $\MALCg$ 
$\LnALCg$ satisfiability on varying domain neighbourhood models, where $L\in \Log$,
%\todo{Patheon?},  
using the notion of 
a propositional abstraction of a formula 
(as in, e.g.,~\cite{Baader:2012:LOD:2287718.2287721}).
%Since $\ALC$ formula satisfiability is already $\ExpTime$-hard, 
%our upper bounds here are tight. 
%we have 
%a tight complexity result for the global cases.
%We show that
Here, one can separate the satisfiability test into two parts, 
one for the description logic dimension and 
one for the \neighborhood frame dimension.
%~ In this subsection, we also consider the lightweight DL called \EL, defined 
%~ as the fragment of \ALC which only allows conjunctions and existential quantification 
%~ in concept expressions. 
%
%Consider $\Lmc\in\{\ALC,\EL\}$. 
%\newcommand{\propmodel}{\ensuremath{M}\xspace}
%\newcommand{\propdomain}{\ensuremath{W}\xspace}
%\newcommand{\propneigh}{\ensuremath{N}\xspace}
%\newcommand{\propassign}{\ensuremath{I}\xspace}
For an $\MLnALCg$ formula $\varphi$, the 
\emph{propositional abstraction} $\prop{\varphi}$ is  
the result of replacing each \ALC atom $\pi$ in $\varphi$ by 
a propositional variable $p_{\pi} \in \NPr$.
%so that there is a one-to-one relationship between the \ALC atoms $\elaxiom$ occurring in $\varphi$ and the 
%propositional letters $p_{\elaxiom}$ used for the abstraction.  
%Let \NPr be a countably infinite set of propositional symbols disjoint from \NC, \NR, and \NI. \nb{O: added}
%We define
Define the set $\setsymbols_\varphi = \{p_{\elaxiom}\in\NPr\mid \elaxiom \text{ is an \ALC atom in }
\varphi \}$.
%The semantics of $\prop{\varphi}$ is given in terms of \emph{propositional neighbourhood models}
%$(\Wmc, \{ \Nmc_{i} \}_{i \in J}, \Vmc)$ for $L^{n}$, 
%where $(\Wmc, \{ \Nmc_{i} \}_{i \in J})$ is a neighbourhood frame, with $J = \{ 1, \ldots, n \}$,
%and $\Vmc: \NPr \rightarrow \Pmc(\Wmc)$ is a function 
%mapping propositional variables in $\NPr$ to
%sets of worlds
%%subsets of the domain of worlds
%(see~\cite{Che,Var2}). 
%
 %Given an $\MLnALCg$ formula $\varphi$, 
%and we say that
A (propositional neighbourhood) $L^{n}$ model 
$\propmodel = (\Wmc, \{ \Nmc_{i} \}_{i \in J}, \Vmc)$
%of $\prop{\varphi}$
%, defined as the set of variables 
%evaluated to true in the model, 
is \emph{$\setsymbols_\varphi$-consistent}
if,
for all $w\in \Wmc$,
the following \ALC formula is satisfiable:
$
\alcform = \bigwedge_{p_{\elaxiom}\in \formtp{\varphi}} {\elaxiom} \ \wedge \bigwedge_{p_{\elaxiom} \in
	\setsymbols_\varphi \setminus \formtp{\varphi}}
% \overline{\NPr(w)}}
\neg {\elaxiom},
$
where
$\formtp{\varphi} = \{p_{\elaxiom} \in \setsymbols_\varphi \mid w\in \Vmc(p_{\elaxiom})\}$.
%
%%%% OLD
%{\color{red}{
%Given an $\EALCg$ formula $\varphi$, we say that a propositional neighbourhood model 
%$\propmodel = (\Wmc, \{ \Nmc_{i} \}_{i \in J}, \Vmc)$
%of $\prop{\varphi}$
%%, defined as the set of variables 
%%evaluated to true in the model, 
%is \emph{$\consistent{\varphi}$}
%if, for all $w\in \Wmc$,
%the following \ALC formula is satisfiable $$\textstyle\bigwedge_{p_{\elaxiom}\in \NPr(w)} \ {\elaxiom} \ \wedge \
%\bigwedge_{p_{\elaxiom}\in \overline{\NPr(w)}}\ \neg {\elaxiom}$$
%where $\NPr(w) = \{p_{\elaxiom}\in \NPr(\varphi) \mid w\in \Vmc(p_{\elaxiom})\}$
%and $\overline{\NPr(w)}=\NPr(\varphi)\setminus\NPr(w)$. 
%}}
%We are now ready for proving Lemma~\ref{lem:propE}.
We  formalise the connection between %complexity of the satisfiability problem for 
$\LnALCg$ satisfiable formulas and their propositional abstractions %with consistent models
with the following lemma.
%, which is an adaptation of the 
%results obtained for other \ALC extensions~\cite{Baader:2012:LOD:2287718.2287721}.
%~ %We say that a model of $\prop{\varphi}$
%~ %is $\varphi$-consistent

%\nb{$\varphi$-consistent model}

\begin{restatable}{lemma}{LemmapropL}\label{lem:propL}
A
%$\MLnALCg$
formula $\varphi$ is
$\LnALCg$
satisfiable
on varying domain neighbourhood models
iff
%if, and only if, 
$\prop{\varphi}$ is satisfied in a $\setsymbols_\varphi$-consistent $L^{n}$ model.  
\end{restatable}
%




%\todo[inline]{M: satisfiability without modalised concepts in varying domains? for all systems?}


 
We now introduce definitions and notation used to prove our complexity result on fragments without modalised concepts.
Let $\setsymbols = \{p_{\elaxiom}\in\NPr\mid \elaxiom \text{ is an \ALC }$ $\text{atom in }\varphi 
 \}$,  for a fixed but arbitrary  $\MLnALCg$ formula $\varphi$,
and let $\phi$ be an $\MLn$ formula built from symbols in $\setsymbols$.
We
denote by  ${\sf sub}(\phi)$ the set 
of subformulas of $\phi$ closed under single negation.  
A \emph{valuation} for $\phi$ %and $\setsymbols$   
is a function $\nu: {\sf sub} (\phi)\rightarrow \{0,1\}$ that 
satisfies the   conditions:
(1) for all $\neg\psi\in {\sf sub} (\phi)$,
$\nu(\psi)=1$ iff $\nu(\neg\psi)=0$;
(2) for all $\psi_1\wedge \psi_2\in {\sf sub} (\phi)$,
$\nu(\psi_1\wedge \psi_2) = 1$ iff $\nu(\psi_1) = 1$
and $\nu(\psi_2) = 1$; 
and (3) $\nu(\phi)=1$. 
A valuation $\nu$ for $\phi$
is \emph{$\setsymbols$-consistent}
if
the following \ALC formula is satisfiable:
$
\textstyle\bigwedge_{\nu(p_\elaxiom)=1} \ {\elaxiom} \ \wedge \
\bigwedge_{\nu(p_\elaxiom)=0}\ \neg {\elaxiom},
$
where $p_\elaxiom\in\setsymbols$.  
%
Lemma~\ref{lem:proplemmaL} establishes that satisfiability of 
$\phi$ in a $\setsymbols$-consistent model is characterised  by the existence of a $\setsymbols$-consistent valuation satisfying suitable properties.
In the following, we use $\falseprop$ as an abbreviation for $p \land \neg p$, for a fixed but arbitrary   $p \in \mathsf{N_{P}}$. %\nb{O: moved closer to where used. T: perfect, thank you}


%\todo{on going}


 
 


\begin{restatable}{lemma}{Lemmapropvardi}\label{lem:proplemmaL}
%\nb{M: improved layout}
	%Let $\setsymbols$ be $\NPr(\varphi)$ for a fixed but arbitrary  $\MLnALCg$ formula $\varphi$
	%and let $\phi$ be a $\MLn$ formula built from symbols in $\setsymbols$.
	Given $\Lvar$ and  an $\MLn$ formula $\phi$ built from symbols in $\setsymbols$ (defined as above), 
	let:
%	$\boldsymbol{\kappa} =
%		| {\sf sub}({\phi}) |$, \text{if $\mathbf{C} \in \Lvar$};
%	$\boldsymbol{\kappa} = 1$, \text{if $\mathbf{C} \not \in \Lvar$}.
	\[
	\boldsymbol{\kappa} =
	\begin{cases}
		| {\sf sub}({\phi}) | , & \text{if $\mathbf{C} \in \Lvar$} \\
		%	\geq 1, & \text{if $\mathbf{C} \in \Lvar$} \\
		1, & \text{if $\mathbf{C} \not \in \Lvar$}
	\end{cases}.
	\]
	A formula $\phi$ is satisfied in a $\setsymbols$-consistent $\Lvar^{n}$ %\nb{check macro}
	model
	iff
	%if, and only if,
	there is   a $\setsymbols$-consistent valuation \valuation 
	for $\phi$ %and $\setsymbols$ 
	such that,
	for every $1 \leq k \leq \boldsymbol{\kappa}$,
	if $\B_i\psi_1, \dots, \B_i\psi_k, \B_i\chi\in{\sf sub}(\phi)$,
	$\valuation(\B_i\psi_j)=1$ for all $1\leq j \leq k$,
	%$\B_i\chi\in{\sf sub}(\prop{\varphi})$, 
	and $\valuation(\B_i\chi)=0$, then
\begin{enumerate}
\item
$
	(\bigwedge^{k}_{j=1}\psi_j\wedge\neg\chi) \vee \boldsymbol\vartheta
$
	is satisfied in a $\setsymbols$-consistent $\Lvar^{n}$ %\nb{check macro}
	model, 
	where:
$\boldsymbol\vartheta = \falseprop$, if $\mathbf{M}\in\Lvar$;
$\boldsymbol\vartheta = \bigvee^{k}_{j=1} (\neg\psi_j\wedge\chi)$, if $\mathbf{M}\not\in\Lvar$;
%	\[
%	\boldsymbol\vartheta =
%	\begin{cases}
%		\falseprop, & \text{if $\mathbf{M}\in\Lvar$} \\
%		\bigvee^{k}_{j=1} (\neg\psi_j\wedge\chi), & \text{if $\mathbf{M}\not\in\Lvar$}
%	\end{cases};
%	\]
	and 
\item
	for $\mathbf{X}\in\{\mathbf{N,T,P,Q,D}\}$,
	if $\mathbf{X}\in\Lvar$, then $\nu$ satisfies the condition $(\mathbf{X})$ below, for every $1 \leq k, h \leq \boldsymbol{\kappa}$:
	\begin{itemize}
		%\item if $\B_i\psi_1, \dots, \B_i\psi_k\in{\sf sub}(\prop{\varphi})$,
		%	$\valuation(\B_i\psi_j)=1$ for all $1\leq j \leq k$,
		% 	$\B_i\chi\in{\sf sub}(\prop{\varphi})$, and $\valuation(\B_i\chi)=0$, then
		%	%$(\bigwedge^{k}_{j=1}\psi_j\wedge\neg\chi) \vee \bigvee^{k}_{j=1} (\neg\psi_j\wedge\chi)$ 
		%	$(\bigwedge^{k}_{j=1}\psi_j\wedge\neg\chi) \vee \mathsf{D}$ 
		%	is satisfied in a $\varphi$-consistent $\Lvar$ %\nb{check macro}
		% 	model;
		%
		\item[($\mathbf{N}$)] if $\B_i\psi\in{\sf sub}(\phi)$ and
		$\valuation(\B_i\psi)=0$, then $\neg \psi$ 
		is satisfied in a $\setsymbols$-consistent $\Lvar^{n}$ model; 
		
		\item[($\mathbf{T}$)] if $\B_i\psi\in{\sf sub}(\phi)$ and 
		$\valuation(\B_i\psi)=1$   then
		$\valuation(\psi)=1$;
		
		\item[($\mathbf{P}$)] if $\B_i\psi_1, \dots, \B_i\psi_k\in{\sf sub}(\phi)$ and
		$\valuation(\B_i\psi_j)=1$ for all $1\leq j \leq k$, 
		then $\bigwedge^{k}_{j=1}\psi_j$
		is satisfied in a $\setsymbols$-consistent $\Lvar^{n}$ model; 
		
		\item[($\mathbf{Q}$)] if $\B_i\psi_1, \dots, \B_i\psi_k\in{\sf sub}(\phi)$ and
		$\valuation(\B_i\psi_j)=1$ for all $1\leq j \leq k$, 
		then $\bigvee^{k}_{j=1}\neg\psi_j$
		is satisfied in a $\setsymbols$-consistent $\Lvar^{n}$ model; 
		
		\item[($\mathbf{D}$)] if $\B_i\psi_1, \dots, \B_i\psi_k, \B_i\chi_1, \dots, \B_i\chi_h\in{\sf sub}(\phi)$,
		$\valuation(\B_i\psi_j)=1$ for all $1\leq j \leq k$, and
		$\valuation(\B_i\chi_\ell)=1$ for all $1\leq \ell \leq h$, 
		%		$\valuation(\B_i\psi_j)=\valuation(\B_i\chi_\ell)=1$ for all $1\leq j \leq k$,  $1\leq \ell \leq h$,
		then $(\bigwedge^{k}_{j=1}\psi_j \land \bigwedge^{h}_{\ell=1}\chi_\ell) \vee \boldsymbol\eta$
		is satisfied in a $\setsymbols$-consistent $\Lvar^{n}$ model,
		where:
%		$\boldsymbol\eta = \falseprop$, if $\mathbf{M}\in\Lvar$; 
%		$\boldsymbol\eta = \neg(\bigwedge^{k}_{j=1}\psi_j) \land \neg(\bigwedge^{h}_{\ell=1}\chi_\ell)$, if $\mathbf{M}\not\in\Lvar$; 
		\[
		\boldsymbol\eta =
		\begin{cases}
			\falseprop, & \text{if $\mathbf{M}\in\Lvar$} \\
			\neg(\bigwedge^{k}_{j=1}\psi_j) \land \neg(\bigwedge^{h}_{\ell=1}\chi_\ell), & \text{if $\mathbf{M}\not\in\Lvar$}
		\end{cases}.
		\]
	\end{itemize}
\end{enumerate}
	%
	%where
	%\begin{itemize}
	%\item
	%			$\mathsf{D} =
	%			\begin{cases}
		%				FALSE, & \text{if $\mathbf{M} \in L$} \\
		%				\bigvee^{k}_{j=1} (\neg\psi_j\wedge\chi), & \text{if $\mathbf{M} \not \in L$}
		%			\end{cases};
	%			$
	%			\item
	%			$\mathsf{k}
	%			\begin{cases}
		%				\geq 1, & \text{if $\mathbf{C} \in L$} \\
		%				= 1, & \text{if $\mathbf{C} \not \in L$}
		%			\end{cases};
	%			$
	%\end{itemize}
\end{restatable}










By using Lemmas~\ref{lem:propL}-\ref{lem:proplemmaL},
the following theorem provides a procedure that runs in exponential time to check $\LnALCg$ satisfiability on varying domains.
Since $\ALC$ formula satisfiability is already $\ExpTime$-hard, our upper bound is tight.








 
 
\begin{restatable}{theorem}{Satfragvardomexp}
\label{thm:satfragvardomexp}
	Satisfiability in $\LnALCg$   on varying domain neighbourhood models is \ExpTime-complete.
\end{restatable}
%




%{\color{red}{
%\todo{M: moved here, to fix}
%In conclusion, by Lemma~\ref{lem:propL}, one can decide $\LnALCg$ satisfiability of a formula $\varphi$ on varying domain neighbourhood models by deciding whether $\prop{\varphi}$ is satisfiable in a $\varphi$-consistent model, using the characterisation given in Lemma~\ref{lem:proplemmaL} above. 
%To establish complexity results,
%we use the fact that there are only polynomially %quadratically
% many   subformulas in $\prop{\varphi}$. 
%Satisfiability in
%$\ALC$ is $\ExpTime$-complete~\cite{GabEtAl03} and so, one can determine in exponential time
%whether a valuation is $\varphi$-consistent. For an $\ExpTime$ upper bound, one can
%deterministically compute all possible $\varphi$-consistent valuations for 
%$(\bigwedge^{k-1}_{j=1}\psi_j\wedge\neg\psi_k)$ (or $(\psi_1 \wedge \neg \psi_2)$) and
%decide satisfiability of $\prop{\varphi}$ by a $\varphi$-consistent model using a bottom-up %strategy (as in~\cite{Baader:2012:LOD:2287718.2287721}). Since satisfiability in $\ALC$ is %$\ExpTime$-hard, our upper bound is tight.



%}}
