%\subsection{Proofs for Section~\ref{sec:relation}}

\Theoremcomplealc*
%
\begin{proof}
This theorem is a consequence of the following claim, and 
the complexity of formula satisfiability in \KnALC{3n} constant domain relational models~\cite{KraWol,GasHer,GabEtAl03}.
\begin{claim} %{lemma}{Theoremclassicalred}\label{theor:classicalred}
%\nb{M: Change to lemma? Merge to next Th. as a claim?}
The $\EnALC{n}$ formula satisfiability problem on constant domain neighbourhood models can be reduced in polynomial time to the \KnALC{3n} formula satisfiability problem on constant domain relational models.
\end{claim}
%\Theoremclassicalred*
%
%\nb{M: Spostare def. $\Mmf$ fuori dal lemma?}
\begin{proof}[Proof of Claim]
Consider an \MLALC{n} formula $\p$
%satisfiable over N-models,
s.t.
$\Mmc, w \mdl \p$, for some constant domain neighbourhood model $\Mmc = (\Fmc, \Delta, \Int)$ with $\Fmc = (\Wmc, \{ \Nmc_{i} \}_{i \in J})$
%based on a neighborhood frame $\Fmc = (\Wmc, \Nmc)$ and having domain $\Delta$,
and some $w\in\Wmc$.
%\nb{O:changed}
We define a relational frame
$\Fmf = (W, \{ R_{i_{1}}, R_{i_{2}}, R_{i_{3}} \}_{i \in J})$
%$\Fmf = (W, \{ R_{i_{j}} \}_{j \in [1, 3]})$
%$\Fmf = (W, \relations_{i_1}, \relations_{i_2}, \relations_{i_3})$
and an $\MLALC{3n}$ relational model 
$\Mmf = (\Fmf, \Delta, I)$
%$\Mmf = (\Fmf, I)$ based on $\Fmf$ and having domain $\Delta$, s.t.:
such that:
%\nb{M: Aggiungere? Cambiare? Cf. \cite{GasHer}, p.15.}
%\nb{O: say that 0,1 are used to ensure that the sets are disjoint?}
	\begin{itemize}
		\item $W = \{ (w, 0) \mid w \in \Wmc \} \cup \{ (\alpha, 1) \mid \alpha \in \bigcup_{v \in \Wmc} \Nmc_{i}(v) \}$
%		\item $W = W \cup \bigcup_{w \in W} \Nmc_{i}(w)$;
%		{\color{red}{$W \cap \bigcup_{w \in W} \Nmc_{i}(w) = \eset$?}};
		\item $\relations_{i_1} = \{ ((w, 0), (\alpha, 1)) \mid \alpha \in \Nmc_{i}(w)\}$;
		\item $\relations_{i_2} = \{ ((\alpha, 1), (w, 0)) \mid w \in \alpha \}$
%		\nb{M: Aggiungere $w \in W$ per chiarezza, anche se superfluo?}
%		\item $\relations_{i_2} = \{ (U, w) \in W \times W \mid w \in W, U \in \Nmc_{i}(w) \colon w \in U \}$ \\
%		{\color{red}{$\relations_{i_2} = \{ (U, w) \in W \times W \mid U \in \bigcup_{v \in W} N(v), w \in W \colon w \in U \}$?}};
		\item $\relations_{i_3} = \{  ((\alpha, 1), (w, 0)) \mid w \not \in \alpha \}$ 
%		\item $\relations_{i_3} =  \{ (U, w) \in W \times W \mid w \in W, U \in \Nmc_{i}(w) \colon w \not \in U \}$; \\
%				{\color{red}{$\relations_{i_3} = \{ (U, w) \in W \times W \mid U \in \bigcup_{w \in W} \Nmc_{i}(w), w \in W \colon w \not \in U \}$?}};
		\item for every $(w, 0) \in W$, $I_{(w, 0)} = \Imc_{w}$; for every $(\alpha, 1) \in W$, $X^{I_{(\alpha, 1)}} = \eset$, for all $X \in \NC \cup \NR$, and $a^{I_{(\alpha, 1)}} = a^{\Int}$, for all $a \in \NI$.
%		\nb{M: Cambiare? Opzioni: \\ (a) tutto falso in $U$ $\to$ diverse def. semantiche; \\ (b) non definito in $U$ $\to$ interpret. $I$ funzione parziale? \\ (c) lasciare cosi}
	\end{itemize}
	%
The pairs $(w, 0), (\alpha, 1)$ are used to ensure that $W$ is the disjoint union of the sets of worlds $w$ and subsets $\alpha$ of $\Wmc$.

%	\begin{itemize}
%		\item $W = \Wmc \cup \{ (\alpha, 0) \mid \alpha \in \bigcup_{v \in W} N(v) \}$
%%		\item $W = W \cup \bigcup_{w \in W} \Nmc_{i}(w)$;
%%		{\color{red}{$W \cap \bigcup_{w \in W} \Nmc_{i}(w) = \eset$?}};
%		\item $\relations_{i_1} = \{ (w,  (\alpha, 0)) \mid w \in W, \alpha \in \Nmc_{i}(w)\}$;
%		\item $\relations_{i_2} = \{ ((\alpha, 0), w) \mid \alpha \in \bigcup_{v \in W} N(v), w \in \alpha \}$
%		\nb{Aggiungere $w \in W$ per chiarezza, anche se superfluo?}
%%		\item $\relations_{i_2} = \{ (U, w) \in W \times W \mid w \in W, U \in \Nmc_{i}(w) \colon w \in U \}$ \\
%%		{\color{red}{$\relations_{i_2} = \{ (U, w) \in W \times W \mid U \in \bigcup_{v \in W} N(v), w \in W \colon w \in U \}$?}};
%		\item $\relations_{i_3} = \{  ((\alpha, 0), w) \mid \alpha \in \bigcup_{v \in W} N(v), w \in W, w \not \in \alpha \}$ 
%%		\item $\relations_{i_3} =  \{ (U, w) \in W \times W \mid w \in W, U \in \Nmc_{i}(w) \colon w \not \in U \}$; \\
%%				{\color{red}{$\relations_{i_3} = \{ (U, w) \in W \times W \mid U \in \bigcup_{w \in W} \Nmc_{i}(w), w \in W \colon w \not \in U \}$?}};
%		\item for every $w \in \Wmc$, $I_{w} = \Imc_{w}$; for every $\alpha \in W \setminus \Wmc$, $X^{I(\alpha)} = \eset$, for all $X \in \NC \cup \NR$, and $a^{I(\alpha)} = a^{I}$, for all $a \in \NI$. \\
%%		\nb{M: Cambiare? Opzioni: \\ (a) tutto falso in $U$ $\to$ diverse def. semantiche; \\ (b) non definito in $U$ $\to$ interpret. $I$ funzione parziale? \\ (c) lasciare cosi}
%	\end{itemize}

Firstly we show, by induction on the structure of concepts $C$, that for all $d \in \Delta$ and all $w \in \Wmc$:
\[
d \in C^{\Int_w} \text{ iff } d \in (C\tr)^{I_{(w, 0)}}.
\]
For the base case $C = A \in \NC$, %it
the claim follows immediately from the definitions of $I$ and $\cdot\tr$. 
Assume the claim holds for $D$ and $E$. 
The inductive cases $C = \lnot D$ and $C = (D \sqcap E)$ are straightforward. 
We are left with the cases below.

$C = \exists r.D$.
We have that
$d \in (\exists r.D)^{\Imc_{w}}$
iff there is $d' \in D^{\Imc_{w}}$ such that $(d,d') \in r^{\Imc_{w}}$.
By i.h. and definition of $I$, this is equivalent to
$d' \in (D\tr)^{I_{(w, 0)}}$ and $(d,d') \in r^{I_{(w, 0)}}$,
which means that
$d \in (\exists r. (D\tr))^{I_{(w, 0)}}$. 
By definition of $\cdot\tr$, $d \in ((\exists r.D)\tr)^{I_{(w, 0)}}$.

$C = \B_{i} D$. 
By definition, we have that
$d \in (\B_{i} D)^{\Imc_{w}}$
iff 
$\llbracket D \rrbracket^{\Mmc}_{d} \in \Nmc_{i}(w)$.
Equivalently, iff there is
$\alpha \in \Nmc_{i}(w)$
s.t. for all $v \in \Wmc$,
$v \in \alpha \Leftrightarrow d \in D^{\Imc_{v}}$.
% iff $d \in \{ d' \in \Delta \mid \exists U \in \Nmc_{i}(w) \colon \forall v \in W : v \in U \Rightarrow d' \in D^{\Imc_{v}} \land v \not \in U \Rightarrow d' \not \in D^{\Imc_{v}} \}$.
By i.h. and definitions of $\relations_{i_2}$ and $\relations_{i_3}$, this means that there is $\alpha \in \Nmc_{i}(w)$ s.t.
$(i)$ for every $v \in \Wmc : (\alpha, 1) \relations_{i_2} (v, 0) \Rightarrow d \in (D\tr)^{I_{(v, 0)}}$
and
$(ii)$ for every $v \in \Wmc: (\alpha, 1) \relations_{i_3} (v, 0) \Rightarrow d \not \in (D\tr)^{I_{(v, 0)}}$.
%\nb{M: Correggere}
This holds iff there is
$\alpha \in \Nmc_{i}(w)$ s.t. $d \in (\B_{i_2} D\tr \sqcap \B_{i_3} \lnot D\tr)^{I_{(\alpha, 1)}}$. 
By definition of $\relations_{i_1}$, this means that there exists
$(\alpha, 1) \in W$ s.t. $(w, 0) \relations_{i_1} (\alpha, 1)$ and $d \in (\B_{i_2} D\tr \sqcap \B_{i_3} \lnot D\tr)^{I_{(\alpha, 1)}}$.
That is, $d \in \D_{i_1} (\B_{i_2} D\tr \sqcap \B_{i_3} \lnot D\tr)^{I_{(w, 0)}}$
iff, by definition of $\cdot\tr$,
$d \in ((\B_{i} D)\tr)^{I_{(w, 0)}}$.
\newline

We now show that for every $\MLALC{n}$ formula $\psi$ and every $w \in \Wmc$:
\[
	\Mmc, w \mdl \psi \text{ iff } \Mmf, (w, 0) \mdl \psi\tr
\]
For the case $\psi = C \sqs D$, it follows from the previous claim, while for $\psi = C(a)$ and $\psi = r(a,b)$, it is immediate from the definitions of $I$ and $\cdot\tr$, as well as from the claim above.
Assuming that the lemma holds for $\chi$ and $\zeta$, the inductive cases $\psi = \lnot \chi$ and $\psi = \chi \land \zeta$ are straightforward.
We prove the statement for modalised formulas.

$\psi = \B_{i} \chi$.
$\Mmc, w \mdl \B_{i} \chi$
iff, by definition,
$\llbracket \chi \rrbracket^{\Mmc}  \in \Nmc_{i}(w)$.
That is, iff there is $\alpha \in \Nmc_{i}(w)$ s.t. for all $v \in \Wmc : v \in \alpha \Leftrightarrow \Mmc, v \mdl \chi$.
By i.h. and definitions of $\relations_{i_2}, \relations_{i_3}$, this means that there is
$\alpha \in \Nmc_{i}(w)$ s.t.
$(i)$ for all $v \in \Wmc : (\alpha, 1) \relations_{i_2} (v, 0) \Rightarrow \Mmf, (v, 0) \mdl \chi\tr$
and
$(ii)$ for all $v \in \Wmc: (\alpha, 1) \relations_{i_3} (v, 0) \Rightarrow \Mmf, (v, 0) \not\mdl \chi\tr$,
iff there is $\alpha \in \Nmc_{i}(w)$ s.t.
$\Mmf, (\alpha, 1) \mdl \B_{i_2} \chi\tr \land \B_{i_3} \lnot \chi\tr$.
By definition of $\relations_{i_1}$, the previous step is equivalent to:
there is $(\alpha, 1) \in W$ s.t. $(w, 0) \relations_{1} (\alpha, 1)$ and $\Mmf, (\alpha, 1) \mdl \B_{i_2} \chi\tr \land \B_{i_3} \lnot \chi\tr$,
iff
$\Mmf, (w, 0) \mdl \D_{i_1} (\B_{i_2} \chi\tr \land \B_{i_3} \lnot \chi\tr)$.
By definition of $\cdot\tr$,
$\Mmf, (w, 0) \mdl (\B_{i} \chi)\tr$.


Thus, in particular, we obtain $\Mmf, (w, 0) \mdl \p\tr$.
\newline

Conversely, consider a $\MLALC{3n}$ formula $\p\tr$ s.t.
$\Mmf, w \mdl \p\tr$,
for some $\MLALC{3n}$ R-model
$\Mmf = (\Fmf, \Delta, I)$
based on
$\Fmf = (W, \{ \relations_{i_{j}} \}_{j \in [1, 3]})$,
and some
$w \in W$.
We define a $\MLnALC{n}$
neighbourhood model
$\Mmc = (\Fmc, \Delta, \Int)$
based on
$\Fmc = (\Wmc, \{ \Nmc_{i} \}_{i \in [1, n]})$
s.t.
$\Wmc = W$,
and for all $w \in W$:
\begin{itemize}
			\item $\alpha \in \Nmc_{i}(w)$ iff there is $v \in W$ s.t. $w \relations_{i_1} v$ and: $(i)$ for all $u \in W$, $v \relations_{i_2} u \Rightarrow u \in \alpha$, and $(ii)$ for all $u \in W$, $v \relations_{i_3} u \Rightarrow u \not \in \alpha$;
			\item $\Imc_{w} = I_{w}$.
\end{itemize} 

%%% OLD VERSION
%Conversely, consider a satisfiable \MLnALC{3} formula $\p\tr$  and let $\Mmf = (\Fmf, \Int)$ be a \MLnALC{3} relational model based on $\Fmc_r = (W, \relations_{i_1}, \relations_{i_2}, \relations_{i_3})$ and having domain $\Delta$, s.t. $\Mmf, w \mdl \p\tr$, for some $w \in W$. 
%Define a \MLALC{} neighborhood model model $\Mmc_n = (\Fmc_n, \Int)$ based on $\Fmc_n = (W, N)$ and having the same domain $\Delta$ s.t.:
%%Conversely, consider a satisfiable \MLnALC{3} formula $\p\tr$  and let $\Mmf = (\Fmf, I)$ be a \MLnALC{3} relational model based on $\Fmf = (W, \relations_{i_1}, \relations_{i_2}, \relations_{i_3})$ and having domain $\Delta$, s.t. $\Mmf, w \mdl \p\tr$, for some $w \in W$. Define a \MLALC{} model $\Mmc = (\Fmc, \Int)$ based on $\Fmc = (W, N)$ and having domain $\Delta$ s.t.:
%\begin{itemize}
%	\item $w \in W$ iff there is $v \in W$ s.t. $w \relations_{i_1} v$ or $v \relations_{i_2} w$ or $v \relations_{i_3} w$;
%	\item $N \colon W \to \Pmc(\Pmc(W))$ neighborhood function s.t., for all $w \in W$:
%		\begin{itemize}
%			\item $U \in \Nmc_{i}(w)$ iff there is $v \in W$ s.t. $w \relations_{i_1} v$ and for all $u \in W$, $v \relations_{i_2} u \Rightarrow u \in U$, and for all $t \in W$, $v \relations_{i_3} t \Rightarrow t \not \in U$.
%%			\item for all \MLALC{} concepts $C$, and for all $d \in \Delta$, $[C]^{\Mmc}_{d} \in \Nmc_{i}(w)$ iff there is $v \in W$ s.t. $w \relations_{i_1} v$ and for all $u \in  W$: $v \relations_{i_2} u \Rightarrow d \in C^{\Imc_{u}}$ and $v \relations_{i_3} u \Rightarrow d \not \in C^{\Imc_{u}}$;
%%			\item for all \MLALC{} formulas $\psi$, $[\psi]^{\Mmc} \in \Nmc_{i}(w)$ iff there is $v \in W$ s.t. $w \relations_{i_1} v$ and for all $u \in  W$: $v \relations_{i_2} u \Rightarrow \Mmc, u \mdl \psi$ and $v \relations_{i_3} u \Rightarrow \Mmc, u \not \mdl \psi$;
%		\end{itemize} 
%	\item for all $w \in W$, $\Imc_{w} = I_{w}$.
%\end{itemize}

Again, we show firstly, by induction on the structure of concepts $C$, that for all $d \in \Delta$ and all $w \in W$:
\[
d \in C^{\Imc_{w}} \text{ iff } d \in (C\tr)^{I_{w}}.
\]
For the base case $C = A \in \NC$, the claim follows from the definitions of $\Int$ and $\cdot\tr$. 
Assume the claim holds for $D$ and $E$. 
The inductive cases $C = \lnot D$ and $C = (D \sqcap E)$ are straightforward. 
We are left with the cases below.

$C = \exists r.D$.
We have that
$d \in (\exists r.D)^{\Imc_{w}}$
iff there is $d' \in D^{\Imc_{w}}$ such that $(d,d') \in r^{\Imc_{w}}$.
By i.h. and definition of $\Int$, this is equivalent to
$d' \in (D\tr)^{I_{w}}$ and $(d,d')\in r^{I_{w}}$, which means that $d \in (\exists r. (D\tr))^{I_{w}}$. 
By definition of $\cdot\tr$, $d \in ((\exists r.D)\tr)^{I_{w}}$.

$C = \B_{i} D$. 
We have  
$d \in (\B_{i} D)^{\Imc_{w}}$ iff 
$\llbracket D \rrbracket^{\Mmc}_{d} \in \Nmc_{i}(w)$.
By definition of $\Nmc_{i}$, this means that there is
$v \in W$ s.t. $w \relations_{i_1} v$
and:
$(i)$ for all $u \in W$: $v \relations_{i_2} u \Rightarrow d \in D^{\Imc_{u}}$
and
$(ii)$ for all $u \in W$: $v \relations_{i_3} u \Rightarrow d \not \in D^{\Imc_{u}}$.
By i.h. the previous step is equivalent to:
there is $v \in W$ s.t. $w \relations_{i_1} v$ and:
$(i)$ for all $u \in  W$: $v \relations_{i_2} u \Rightarrow d \in (D\tr)^{I(u)}$ and
$(ii)$ for all $u \in W$: $v \relations_{i_3} u \Rightarrow d \not \in (D\tr)^{I(u)}$.
Equivalently,
$d \in (\D_{i_1}(\B_{i_2} D\tr \sqcap \B_{i_3} \lnot D\tr))^{I_{w}}$
iff, by definition of $\cdot\tr$,
$d \in ((\B_{i} D)\tr)^{I_{w}}$.
\newline
%%% OLD VERSION
%We have  
%$d \in (\B D)^{\Imc_{w}}$ iff 
%$[ D ]^{\Mmc}_{d} \in \Nmc_{i}(w)$.
%By definition of $N$, this means that there is $v \in W$ s.t. $w \relations_{i_1} v$ and for all $u \in  W$: $v \relations_{i_2} u \Rightarrow d \in C^{\Imc_{u}}$ and $v \relations_{i_3} u \Rightarrow d \not \in C^{\Imc_{u}}$.
%By i.h. the previous step is equivalent to: there is $v \in W$ s.t. $w \relations_{i_1} v$ and for all $u \in  W$: $v \relations_{i_2} u \Rightarrow d \in (C\tr)^{I(u)}$ and $v \relations_{i_3} u \Rightarrow d \not \in (C\tr)^{I(u)}$, iff
%$d \in (\D_{i_1}(\B_{i_2} C\tr \sqcap \B_{i_3} \lnot C\tr))^{I_{w}}$.
%By definition of $\cdot\tr$, $d \in ((\B C)\tr)^{I_{w}}$.

We now prove, by induction on $\MLALC{n}$ formulas $\psi$, that for every $w \in W$:
\[
	\Mmc, w \mdl \psi \text{ iff } \Mmf, w \mdl \psi\tr
\]

For the case $\psi = C \sqs D$, it follows from the previous claim, while for $\psi = C(a)$ and $\psi = r(a,b)$, it is immediate from the definitions of $\Int$ and $\cdot\tr$, as well as from the claim above. Assuming that the lemma holds for $\chi$ and $\zeta$, the inductive cases $\psi = \lnot \chi$ and $\psi = \chi \land \zeta$ are straightforward. We prove the statement for modalised formulas.

$\psi = \B_{i} \chi$.
$\Mmc, w \mdl \Box_{i} \chi$ iff
$\llbracket \chi \rrbracket^{\Mmc}  \in \Nmc_{i}(w)$.
That is, there is $v \in W$ s.t. $w \relations_{i_1} v$ and $(i)$ for all $u \in  W$: $v \relations_{i_2} u \Rightarrow \Mmc, u \mdl \chi$ and $(ii)$ for all $u \in W$: $v \relations_{i_3} u \Rightarrow \Mmc, u \not \mdl \chi$.
By i.h., this is equivalent to: there is $v \in W$ s.t. $w \relations_{i_1} v$ and
$(i)$ for all $u \in  W$: $v \relations_{i_2} u \Rightarrow \Mmf, u \mdl \chi\tr$ and
$(ii)$ for all $u \in  W$: $v \relations_{i_3} u \Rightarrow \Mmf, u \not \mdl \chi\tr$.
The previous step means that: 
$\Mmf, w \mdl \D_{i_1} (\B_{i_2} \chi\tr \land \B_{i_3} \lnot \chi\tr)$
iff, by definition of $\cdot\tr$,
$\Mmf, w \mdl (\B_{i} \chi)\tr$.

%%% OLD VERSION
%$\psi = \B \chi$.
%$\Mmc, w \mdl \Box \chi$ iff
%$[ \chi ]^{\Mmc}  \in \Nmc_{i}(w)$.
%That is, there is $v \in W$ s.t. $w \relations_{i_1} v$ and for all $u \in  W$: $v \relations_{i_2} u \Rightarrow \Mmc, u \mdl \chi$ and $v \relations_{i_3} u \Rightarrow \Mmc, u \not \mdl \chi$.
%By i.h., this is equivalent to: there is $v \in W$ s.t. $w \relations_{i_1} v$ and for all $u \in  W$: $v \relations_{i_2} u \Rightarrow \Mmf, u \mdl \chi\tr$ and $v \relations_{i_3} u \Rightarrow \Mmf, u \not \mdl \chi\tr$, iff 
%$\Mmf, w \mdl \D_{i_1} (\B_{i_2} \chi\tr \land \B_{i_3} \lnot \chi\tr)$.
%By definition of $\cdot\tr$, $\Mmf, w \mdl (\B \chi)\tr$.


Therefore,  in particular, $\Mmc, w \mdl \p$.
%\qed
\end{proof}
\end{proof}





















\Theoremcomplmalc*
%
\begin{proof}
This theorem is a consequence of the following claim, and the complexity 
of formula satisfiability in $\KnALC{2n}$ constant domain relational models~\cite{KraWol,GasHer,GabEtAl03}. 
\begin{claim} %{lemma}{Theoremmonotonicred}\label{theor:monotonicred}
%\nb{M: Change to lemma? Merge to next Th. as a claim?}
%Satisfiability in $\MnALC{n}$ is reducible to satisfiability in $\KnALC{2n}$.
The $\MnALC{n}$ formula satisfiability problem on constant domain neighbourhood models can be reduced in polynomial time to the \KnALC{2n} formula satisfiability problem on constant domain relational models.
\end{claim}
 \begin{proof}[Proof of Claim]
The proof is analogous to that of Theorem~\ref{theor:complealc} (we show the cases for modalised concepts and formulas only).
Consider a $\MLALC{n}$ formula $\p$ satisfiable on supplemented neighbourhood frames, i.e., so that there is a neighbourhood model
$\Mmc = (\Fmc, \Delta, \Int)$
based on a supplemented neighbourhood frame
$\Fmc = (\Wmc, \{ \Nmc_{i} \}_{i \in J})$ and a $w$ in $\Mmc$ such that
$\Mmc, w \mdl \p$.
%Namely, there are: a \e{supplemented} neighborhood frame $\Fmc = (\Wmc, N)$, i.e., a frame s.t. for all $w \in \Wmc$ and all $\alpha, \beta \sbs \Wmc$, if $\alpha \in \Nmc_{i}(w)$ and $\alpha \sbs \beta$, then $\beta \in \Nmc_{i}(w)$; a neighborhood model $\Mmc = (\Fmc, \Int)$ based on $\Fmc$; and a $w \in M_{n}$ s.t. $\Mmc, w \mdl \p$.
We define a relational frame
$\Fmf = (W, \{ R_{i_{1}}, R_{i_{2}} \}_{i \in J})$,
and an $\MLALC{2n}$ relational model
$\Mmf = (\Fmf, \Delta, I)$ based on $\Fmf$, such that:
	\begin{itemize}
		\item $W = \{ (w, 0) \mid w \in \Wmc \} \cup \{ (\alpha, 1) \mid \alpha \in \bigcup_{v \in \Wmc} \Nmc_{i}(v) \}$
%		$W = \Wmc \uplus \bigcup_{w \in \Wmc} \Nmc_{i}(w)$, where $\uplus$ takes the disjoint union of (suitably indexed copies) of $\Wmc$ and $\bigcup_{w \in \Wmc} \Nmc_{i}(w)$;\nb{M:  Abuso di notazione con $\uplus$}
%		{\color{red}{$W \cap \bigcup_{w \in W} \Nmc_{i}(w) = \eset$?}};
		\item $\relations_{i_1} = \{ ((w, 0), (\alpha, 1)) \mid \alpha \in \Nmc_{i}(w)\}$;
		\item $\relations_{i_2} = \{ ((\alpha, 1), (w, 0)) \mid w \in \alpha \}$
%		\item $\relations_{i_1} = \{ (w, \alpha) \mid w \in \Wmc, \alpha \in \Nmc_{i}(w)\}$;
%		\item$\relations_{i_2} = \{ (\alpha, w) \mid \alpha \in \bigcup_{v \in \Wmc} N(v), w \in \Wmc, w \in \alpha \}$;
%		\nb{M: Togliere $w \in \Wmc$? (cf. sopra)}
%		\item $\relations_{i_2} = \{ (U, w) \in W \times W \mid w \in W, U \in \Nmc_{i}(w), w \in U \}$;
		\item for every $(w, 0) \in W$, $I_{(w, 0)} = \Imc_{w}$; for every $(\alpha, 1) \in W$, $X^{I_{(\alpha, 1)}} = \eset$, for all $X \in \NC \cup \NR$, and $a^{I_{(\alpha, 1)}} = a^{\Int}$, for all $a \in \NI$.
%		\item for all $w \in \Wmc$, $I_{w} = \Imc_{w}$; for all $\alpha \in W \setminus \Wmc$, $X^{I(\alpha)} = \eset$, for all $X \in \NC \cup \NR$, and $a^{I(\alpha)} = a^{\Int}$, for all $a \in \NI$.
%		\nb{M: Cambiare! Opzioni: \\ (a) falso in $U$ $\to$ diverse def. semantiche; \\ (b) non definito in $U$ $\to$ interpret. $I$ funzione parziale? \\ (c) lasciare cosi}
	\end{itemize}

%\nb{M: Check}
Firstly we show, by induction on the structure of concepts $C$, that for all $d \in \Delta$ and all $w \in \Wmc$:
\[
d \in C^{\Imc_{w}} \text{ iff } d \in (C\ttr)^{I(w,0)}.
\]
\noindent
$C = \B_{i} D$. 
We have  
$d \in (\B_{i} D)^{\Imc_{w}}$
iff 
%$d \in$
$\llbracket D \rrbracket^{\Mmc}_{d} \in \Nmc_{i}(w)$.
Since $\Fmc$ is supplemented, the previous step means that there is
$\alpha \in \Nmc_{i}(w) \colon \alpha \sbs \llbracket D \rrbracket^{\Mmc}_{d}$.
This is equivalent to:
there is $\alpha \in \Nmc_{i}(w)$
s.t. for every
$v \in \Wmc : v \in \alpha \Rightarrow d \in D^{\Imc_{v}}$.
% iff $d \in \{ d' \in \Delta \mid \exists U \in \Nmc_{i}(w) \colon \forall v \in W : v \in U \Rightarrow d' \in D^{\Imc_{v}} \land v \not \in U \Rightarrow d' \not \in D^{\Imc_{v}} \}$.
By i.h. and definition of $\relations_{i_2}$,
there is $\alpha \in \Nmc_{i}(w)$
s.t. for every
$v \in \Wmc : (\alpha, 1) \relations_{i_2} (v, 0) \Rightarrow d \in (D\ttr)^{I(v,0)}$.
Equivalently, there is
$\alpha \in \Nmc_{i}(w) \colon d \in (\B_{i_2} D\ttr)^{I(\alpha,1)}$.
%\nb{M: Correggere (cf. sopra)}
By definition of $\relations_{i_1}$, this means:
there exists
$(\alpha,1) \in W$
s.t.
$(w,0) \relations_{i_1} (\alpha,1)$ and $d \in (\B_{i_2} D\ttr)^{I(\alpha,1)}$, 
iff
$d \in (\D_{i_1} \B_{i_2} D\ttr )^{I(w,0)}$.
That is, by definition of $\cdot\tr$,
$d \in ((\B D_{i})\ttr)^{I(w,0)}$.
\newline

Then, we prove that for every $\MLALC{n}$ formula $\psi$ and every $w \in \Wmc$:
\[
	\Mmc, w \mdl \psi \text{ iff } \Mmf, (w,0) \mdl \psi\ttr
\]
\noindent
$\psi = \B_{i} \chi$.
$\Mmc, w \mdl \B_{i} \chi$ iff
$\llbracket \chi \rrbracket^{\Mmc}  \in \Nmc_{i}(w)$.
Since $\Fmc$ is supplemented, this is equivalent to: 
there is $\alpha \in \Nmc_{i}(w)$ s.t. $ \alpha \sbs \llbracket \chi \rrbracket^{\Mmc}$,
iff
there is $\alpha \in \Nmc_{i}(w)$ s.t. for all $v \in \Wmc : v \in \alpha \Rightarrow \Mmc, v \mdl \chi$.
By i.h. and definition of $\relations_{i_2}$, there is $\alpha \in \Nmc_{i}(w)$
s.t. for all $v \in \Wmc : (\alpha,1) \relations_{i_2} (v,0) \Rightarrow \Mmf, (v,0) \mdl (\chi\ttr)$.
This means that, for some $\alpha \in \Nmc_{i}(w)$,
$\Mmf, (\alpha,1) \mdl \B_{i_2} \chi\ttr$.
By definition of $\relations_{i_1}$, the previous step is equivalent to:
there is $(\alpha,1) \in W$ s.t.
$(w,0) \relations_{i_1} (\alpha,1)$
and
$\Mmf, (\alpha,1) \mdl \B_{i_2} \chi\ttr$.
That is,
$\Mmf, (w,0) \mdl \D_{i_1} \B_{i_2} \chi\ttr$.
By definition of $\cdot\ttr$,
$\Mmf, (w,0) \mdl (\B_{i} \chi)\ttr$.

Thus, in particular, we have $\Mmf, (w,0) \mdl \p\ttr$.
\newline

%\nb{M: Check}
Conversely, consider an $\MLALC{2n}$ formula $\p\ttr$ satisfiable on relational models:
$\Mmf, w \mdl \p\ttr$,
for some $\MLALC{2n}$ model
$\Mmf = (\Fmf, \Delta, I)$ based on $\Fmf = (W, \{ \relations_{i_{j}} \}_{j \in [1, 2]})$, and some $w \in W$. 
We define an $\MLALC{n}$ neighbourhood model
$\Mmc = (\Fmc, \Delta, \Int)$
based on
$\Fmc = (\Wmc, \{ \Nmc_{i} \}_{i \in [1, n]})$
s.t. $\Wmc = W$, and for all $w \in W$:
\begin{itemize}
%			\item $\Wmc = W$;
			\item $\alpha \in \Nmc_{i}(w)$ iff there is $v \in W$ s.t. $w \relations_{i_1} v$ and for all $u \in W$, $v \relations_{i_2} u \Rightarrow u \in \alpha$;
			\item $\Imc_{w} =I_{w}$.
\end{itemize} 

Firstly, notice that $\Fmc$ is supplemented: 
%by definition, we have that for all $w \in W$ and for all $\alpha \sbs W$, $\alpha \in \Nmc_{i}(w)$ iff there is $v \in W$ s.t. $w \relations_{i_1} v$ and for all $u \in W$, $v \relations_{i_2} u \Rightarrow u \in \alpha$. 
for all $w \in W$, if $\alpha \in \Nmc_{i}(w)$ and $\alpha \sbs \beta \sbs W$, then there is $v \in W$ s.t. $w \relations_{i_1} v$ and for all $u \in W$, $v \relations_{i_2} u \Rightarrow u \in \beta$, i.e., $\beta \in \Nmc_{i}(w)$.
We now show, by induction on the structure of concepts $C$, that for all $d \in \Delta$ and all $w \in W$:
\[
d \in C^{\Imc_{w}} \text{ iff } d \in (C\ttr)^{I_{w}}.
\]
\noindent
$C = \B_{i} D$. 
We have  
$d \in (\B_{i} D)^{\Imc_{w}}$ iff 
$\llbracket D \rrbracket^{\Mmc}_{d} \in \Nmc_{i}(w)$.
By definition of $\Nmc$, this means that there is
$v \in W$ s.t. $w \relations_{i_1} v$ and for all $u \in  W$: $v \relations_{i_2} u \Rightarrow d \in D^{\Imc_{u}}$.
By i.h. the previous step is equivalent to: there is $v \in W$ s.t. $w \relations_{i_1} v$ and for all $u \in  W$: $v \relations_{i_2} u \Rightarrow d \in (D\ttr)^{I_{u}}$, iff
$d \in (\D_{i_1}\B_{i_2} D\ttr)^{I_{w}}$.
By definition of $\cdot\ttr$, $d \in ((\B_{i} D)\ttr)^{I_{w}}$.
\newline

Finally, we prove, by induction on $\MLALC{n}$ formulas $\psi$, that for every $w \in W$:
\[
	\Mmc, w \mdl \psi \text{ iff } \Mmf, w \mdl \psi\ttr
\]
\noindent
$\psi = \B_{i} \chi$.
$\Mmc, w \mdl \B_{i} \chi$ iff
$\llbracket \chi \rrbracket^{\Mmc}  \in \Nmc_{i}(w)$.
That is, there is $v \in W$ s.t. $w \relations_{i_1} v$ and for all $u \in  W$: $v \relations_{i_2} u \Rightarrow \Mmc, u \mdl \chi$.
By i.h., this is equivalent to: there is $v \in W$ s.t. $w \relations_{i_1} v$ and for all $u \in  W$: $v \relations_{i_2} u \Rightarrow \Mmf, u \mdl \chi\ttr$,
iff 
$\Mmf, w \mdl \D_{i_1} \B_{i_2} \chi\ttr$.
By definition of $\cdot\ttr$,
$\Mmf, w \mdl (\B_{i} \chi)\ttr$.

Therefore, we have in particular that $\Mmc, w \mdl \p$, i.e., $\p$ is satisfiable on a supplemented neighbourhood model.
%\qed
\end{proof}
\end{proof}


