\subsection{Proofs for Section~\ref{sec:fragcondom}}



\LemmapropE*
%
\begin{proof}
 If formula $\varphi$ is $\LnALCg$ satisfiable on constant domain neighbourhood models then, clearly,
$\prop{\varphi}$ is satisfied in a $\varphi$-consistent $L^{n}$ model.  
We now argue about the converse direction. 
Suppose $\prop{\varphi}$ is satisfied in a $\varphi$-consistent $L^{n}$ model
$\propmodel = (\Wmc, \{ \Nmc_{i} \}_{i \in J}, \Vmc)$. 
We want to construct a constant domain neighbourhood model $\Mmc=(\Fmc,\Delta,\Imc)$, based on the $L^{n}$ frame $\Fmc = ( \Wmc, \{ \Nmc_{i} \}_{i \in J} )$, that satisfies 
$\varphi$.
%We define $\W$ as $\propdomain$ and \Nmc as $\propneigh$. 
%\todo{M: working here}
%The main point in this proof is the definition of \Imc and $\Delta$.
%
As $\propmodel$ is $\varphi$-consistent, for all $w\in \propdomain$,
\[
\alcform = \bigwedge_{p_{\elaxiom}\in \formtp{\varphi}} {\elaxiom} \ \wedge \bigwedge_{p_{\elaxiom} \in
\NPr(\varphi) \setminus \formtp{\varphi}}
% \overline{\NPr(w)}}
 \neg {\elaxiom},
\]
where
$\formtp{\varphi} = \{p_{\elaxiom} \in \NPr(\varphi) \mid w\in \Vmc(p_{\elaxiom})\}$,
%$$\formula(w) =  \bigwedge_{p_{\elaxiom}\in \NPr(w)} \hspace{-0.15cm} {\elaxiom} \ \wedge \hspace{-0.15cm} \bigwedge_{p_{\elaxiom}\in \overline{\NPr(w)}} \hspace{-0.15cm} \neg {\elaxiom}$$
is satisfied by an interpretation $\Jmc_{w}$.
It remains to argue that one can  define $\Imc$ 
where all $\Imc_{w}$ share the same domain $\Delta$ and 
the rigid individual name assumption holds (i.e., $a^{\Imc_{w}}=a^{\Imc_{v}}$, for all $w, v \in \W$ and all $a\in \NI$). 

By Lemma~\ref{lem:aux}, for each model $\Jmc_{w}$ of
$\alcform$
%$\formula(w)$
there is 
a quasimodel $\Qmc(w)$ for
$\alcform$.
%$\varphi$.
We denote by
$\Qmc_{\sf tp}(w)$ the set of concept types
  in  $\Qmc(w)$. 
We assume without loss of generality that
\begin{itemize}
\item[$(\ast)$] for all named types $t_a$, if $t_a\in\Qmc_{\sf tp}(w)$,
then $t=t_a\setminus\{a\}\in\Qmc_{\sf tp}(w)$.
\end{itemize}
We are now in position to define \Imc and $\Delta$. 
We define $\Delta$ as the set of functions $f: \W\rightarrow {\sf tp}(\varphi)$
such that (1) for all $w\in\W$, $f(w)\in \Qmc_{\sf tp}(w)$; and 
(2) $a\in f(w)$ iff $a\in f(u)$ for all $u,w\in \W$ and all $a\in \NI(\varphi)$.
By $(\ast)$ and the definition of $\Qmc_{\sf tp}(w)$,
we also have that (3) for all $w\in \W$ and all types $t\in\Qmc_{\sf tp}(w)$, 
there is $f\in\Delta$ such that $f(w)=t$. 
We denote by $f_a$ the unique element of $\Delta$ where $a$ occurs in it. 
For each $w\in \W$, we define the interpretation $\Imc_{w}$ as follows:
\begin{itemize}
\item for all $A\in\NC$, we have $f\in A^{\Imc_{w}}$ iff $A\in f(w)$;
\item for all $r\in\NR$, we have $(f,f')\in r^{\Imc_{w}}$ iff
$\{\neg D\mid \neg \exists r.D \in f(w)\}\subseteq f'(w)$.
\end{itemize}
Also, for all $w\in \W$ and all $a\in \NI(\varphi)$,
we require that $a^{\Imc_{w}}=f_a\in\Delta$ (for the other 
individual names the mapping is irrelevant for this proof, 
as long as it satisfies the rigid individual name assumption). 
By definition, $\Imc$ is a function mapping each $w\in\W$ to 
an interpretation $\Imc_{w}$ over the constant domain $\Delta$ and satisfying the %constraint 
rigid individual name assumption.

\todo{M: todo fix}
{\color{red}{
One can show that each $\Imc_{w}$ is a model of $\alcform$ using the fact that $\Qmc(w)$ is a quasimodel for $\alcform$. 
}}


{\color{blue}{
We first require the following result.

\todo{M: todo fix}
\begin{claim}\label{cl:localeq}
{\color{red}{
$\Jmc_{w} \models \alcform$ iff $\Imc_{w} \models \alcform$.
}}
\end{claim}
\begin{proof}
{\color{red}{\ldots}}
\end{proof}
}}

{\color{blue}{
Now, let $\Mmc=(\Fmc,\Delta,\Imc)$, with $\Fmc = ( \Wmc, \{ \Nmc_{i} \}_{i \in J} )$, $\Imc$, and $\Delta$ as defined above.
%By induction on the structure of subformulas $\psi$ of $\varphi$, we show
%in the following claim that,
%for every $w \in \Wmc$, we have
%$\propmodel, w \models \prop{\psi}$
%iff $\Mmc, w \models \psi$.
We now show the following claim.
 
\begin{claim}\label{cl:globaleq}
For every $\psi \in {\sf sub}(\varphi)$ and every $w \in \Wmc$, we have
$\propmodel, w \models \prop{\psi}$ iff
$\Mmc, w \models \psi$.
\end{claim}
\begin{proof}
The proof is by induction on the structure of subformulas $\psi$ of $\varphi$.
We first consider the base case.

\begin{itemize}
	\item $\psi = \pi$, where $\pi$ is an $\ALC$ atom in $\varphi$. Hence, $\prop{\psi} = p_\pi$, with $p_\pi$ propositional letter. By the semantics of propositional neighbourhood models,
$\propmodel, w \models p_{\pi}$ iff $w\in\Vmc(p_\pi)$. 
For every \ALC atom $\pi$ in $\varphi$, $w\in\Vmc(p_\pi)$ iff 
$\pi$ is a conjunct of $\hat{\varphi}_{\Vmc,w}$.
As $\propmodel$ is $\varphi$-consistent, we have that, for every $w\in \propdomain$,
the $\ALC$ formula
$\hat{\varphi}_{\Vmc,w}$
is satisfied by the $\ALC$ interpretation $\Jmc_{w} = (\Delta_{w}, \cdot^{\Jmc_{w}})$.
 Thus, $\pi$ is a conjunct of $\hat{\varphi}_{\Vmc,w}$ iff $\Jmc_w\models\pi$.
 \todo{AM: to fix, why?}
 {\color{red}{
 By
 Claim~\ref{cl:localeq},
% the semantics of $\MLnALCg$  neighbourhood models,
 }}
 the previous step is equivalent to
 $\Imc_w\models\pi$, i.e., $\Mmc, w \models \psi$.
 \end{itemize}
 For the inductive step, suppose that the claim holds for $\psi_1,\psi_2$. 
  We consider the following cases. 
 \begin{itemize}
 	\item $\psi=\neg\psi_1$: By the semantics of  propositional neighbourhood models,
 	$\propmodel, w \models \prop{\neg{\psi_1}}$ iff $\propmodel, w \not\models \prop{{\psi_1}}$. By the inductive hypothesis, Claim~\ref{cl:ind} holds for $\psi_1$.
 	By the contrapositive in each direction, $\propmodel, w \not\models \prop{{\psi_1}}$
 	iff $\Mmc, w \not\models \psi_1$. By the semantics of  $\MLnALCg$ neighbourhood models, $\Mmc, w \not\models \psi_1$ iff $\Mmc, w \models \neg\psi_1$.
\item $\psi=\psi_1\wedge\psi_2$: By the semantics of  propositional neighbourhood models,
$\propmodel, w \models \prop{{(\psi_1\wedge\psi_2)}}$ iff $\propmodel, w \models \prop{{\psi_1}}$ and $\propmodel, w \models \prop{{\psi_2}}$. By the inductive hypothesis, the claim holds for $\psi_1,\psi_2$.
So, $\propmodel, w \models \prop{{\psi_i}}$
iff $\Mmc, w \models \psi_i$, for $i\in \{1,2\}$. By the semantics of  $\MLnALCg$ neighbourhood models, $\Mmc, w \models \psi_1$ and $\Mmc, w \models \psi_2$ iff $\Mmc, w \models \psi_1\wedge \psi_2$.
\item $\psi=\B_{i} \psi_1$: By the semantics of  propositional neighbourhood models,
$\propmodel, w \models \prop{{(\B_{i} \psi_1)}}$ iff $\llbracket \prop{{\psi_1}}\rrbracket^{\propmodel} \in \Nmc_{i}(w)$ where
$\llbracket \prop{{\psi_1}} \rrbracket^{\propmodel} = \{ v \in \Wmc \mid \propmodel, v \models \prop{{\psi_1}} \}$. By the inductive hypothesis, the claim holds for $\psi_1$.
So, $\propmodel, v \models \prop{{\psi_1}}$
iff $\Mmc, v \models \psi_1$, for every $v\in\Wmc$. 
Thus, $\llbracket \prop{{\psi_1}} \rrbracket^{\propmodel}=\llbracket {{\psi_1}} \rrbracket^{\Mmc}$. By definition of $\propmodel$ and \Mmc, we have that $\Nmc_{i}(w)$
is the same in both $\propmodel$ and \Mmc, for every $w\in\Wmc$ and $i\in J$.
So $\llbracket \prop{{\psi_1}}\rrbracket^{\propmodel} \in \Nmc_{i}(w)$
iff $\llbracket {{\psi_1}} \rrbracket^{\Mmc} \in \Nmc_{i}(w)$.
By the semantics of  $\MLnALCg$ neighbourhood models, $\llbracket {{\psi_1}} \rrbracket^{\Mmc} \in \Nmc_{i}(w)$ iff $\Mmc, w \models \B_{i} \psi_1$.
 \end{itemize}
We have thus shown that for every subformula $\psi$ of $\varphi$ and every $w \in \Wmc$, we have
$\propmodel, w \models \prop{\psi}$ iff
$\Mmc, w \models \psi$.
\end{proof}
Since $\propmodel , v \models \prop{\varphi}$, for some $v \in \Wmc$, we conclude that $\varphi$ is $\LnALCg$ satisfiable. 
}}
\end{proof}
