\subsection{Fragments without Modalised Concepts on Constant Domain}
\label{sec:fragcondom}
%\subsection{Satisfiability on Constant Domain for $\LnALCg$}
%\subsection{Satisfiability in Fragments without Modalised Concepts}




%\paragraph{Satisfiability in $\EALCg$  and $\MALCg$}
%\nb{O: working here}


\newcommand{\clf}{\ensuremath{{\sf sub}_{\sf f}}\xspace}
\newcommand{\clc}{\ensuremath{{\sf sub}_{\sf c}}\xspace}
\newcommand{\individuals}[1]{\ensuremath{\NI(#1)}\xspace}
In this section,
%the proof of Lemma~\ref{lem:propE}
we use the classical
notion of a quasimodel~\cite{GabEtAl03}, defined here for the convenience of the reader. 
{\color{blue}{
Let $\varphi$ be an \emph{$\ALC$ formula}, that is, an $\MLnALC$ formula without any occurrence of modal operators.
We assume without loss of generality that assertions occurring in $\varphi$ are of the form $A(a)$, where $A \in \NC$, or $r(a,b)$.
}}
We denote by $\NI(\formula)$ the set of individual names occurring in $\formula$. 
 Denote by $\clf(\formula)$ the closure under single negation of the set
of all formulas occurring in~$\formula$.  Similarly, we denote by
$\clc(\formula)$ the closure under single negation of the set of all
concepts union the concepts $A_{a}, \exists r.A_{a}$, for any
$a\in\individuals{\formula}$ and $r$ a role occurring in $\formula$,
where $A_a$ is a fresh concept name.
%As usual, the basic elements of our quasimodels are types.
%
A \emph{concept type for $\formula$} is any subset~$t$ of
$\clc(\formula)\cup\individuals{\formula}$ such that:
\begin{itemize}%[label=$\mathbf{T\arabic*}$,leftmargin=*]
  \item\label{ct:neg} $\neg C\in t$ iff $C\not\in t$, for all
    $\neg C\in\clc(\formula)$;
  \item\label{ct:con} $C\sqcap D\in t$ iff $C,D\in t$, for all
    $C\sqcap D\in\clc(\formula)$;
%    and
 \item\label{ct:ind} $t$ contains at most one individual name in~$\individuals{\formula}$.
\end{itemize}

\noindent
Similarly, we define \emph{formula types} $t\subseteq\clf(\formula)$ for $\formula$ with the
  conditions:
\begin{itemize}%[label=$\mathbf{T\arabic*'}$,leftmargin=*]
  \item\label{ft:neg} $\neg\psi\in t$ iff $\psi\not\in t$, for all
    $\neg\psi\in\clf(\formula)$;
%    and
  \item\label{ft:fcon} $\psi\wedge\chi\in t$ iff $\psi,\chi\in t$, for all
    $\psi\wedge\chi\in\clf(\formula)$.
\end{itemize}
We   omit `for $\formula$'  when there is no risk of confusion. 
%A concept type describes one domain element at a single time point,
%while a formula type expresses constraints on all domain elements.
%
If $a\in t\ \cap\ \individuals{\formula}$, then $t$ describes a named element. 
We  write $t_a$ to indicate this and call it a \emph{named type}.
%
%
We say that a pair of concept types $(t,t')$ is \emph{$r$-compatible} if
%\begin{align*}
$\{\neg F\mid \neg \exists
r.F\in t\}\subseteq t'.$ %\nb{A: changed E to F \\O: Ok!}
%\end{align*}
A \emph{quasimodel} for~$\formula$ is a set $\Qmc$ of concept or formula types for~$\formula$ 
such that:
\begin{itemize}%[label=\textbf{Q\arabic*},leftmargin=*]
  \item\label{q:fseg} $\Qmc$ contains exactly one formula type~$t_\Qmc$;
  \item\label{q:ind} $\Qmc$ contains exactly one named type $t_a$ for
    each $a\in\individuals{\formula}$;
  \item\label{q:gci} for all $C\sqsubseteq D\in\clf(\formula)$, we have
    $C\sqsubseteq D\in t_\Qmc$ iff $C\in t$ implies $D\in t$
    for all concept types $t \in \Qmc$;
  \item\label{q:cass} for all $A(a)\in\clf(\formula)$, we have
    $A(a)\in t_\Qmc$ iff $A\in t_a$ %\nb{A: changed C to A \\ O: Ok!};
    \item \label{q:quasi}   $t \in \Qmc$ and $\exists r.D \in t$ implies there is $t' \in \Qmc$
    such that $D\in t'$ and $(t,t')$ is      $r$-compatible;
    \item \label{q:last} for all $r(a,b)\in\clf(\formula)$, we have
      $r(a,b)\in t_\Qmc$ iff   
     $(t_a,t_b)$ is $r$-compatible. 
\end{itemize}
 
 

As usual, every quasimodel for~$\formula$ describes an  
interpretation satisfying~$\formula$ and, conversely,   every such interpretation can be 
abstracted into a quasimodel for~$\formula$. We formalise the correspondence 
between models and quasimodels  with 
the following lemma. 

%\nb{TODO define quasimodel as in yellow but with ind names in concept types, define types}
\begin{lemma}\label{lem:aux}
There is a quasimodel for an \ALC formula   iff 
there is a model of it. 
\end{lemma}

%We are now ready for proving Lemma~\ref{lem:propE}.
We now formalise the connection between %complexity of 
the satisfiability problem for 
$\LnALCg$ formulas on constant domain neighbourhood models and their propositional abstractions %with consistent models
with the following lemma.
%, which is an adaptation of the 
%results obtained for other \ALC extensions~\cite{Baader:2012:LOD:2287718.2287721}.
%~ %We say that a model of $\prop{\varphi}$
%~ %is $\varphi$-consistent

%\nb{$\varphi$-consistent model}

\begin{restatable}{lemma}{LemmapropE}\label{lem:propE}
\nb{M: todo fix proof}
A
%$\MLnALCg$
formula $\varphi$ is
$\LnALCg$
satisfiable
on constant domain neighbourhood models
iff
$\prop{\varphi}$ is satisfied in a $\varphi$-consistent $L^{n}$ model.
%A formula $\varphi$ is $\EALCg$ satisfiable
%iff
%%if, and only if, 
%$\prop{\varphi}$ is satisfied in a $\varphi$-consistent $\E^{n}$ model.  
\end{restatable}
%


We can now use again to Lemma~\ref{lem:proplemmaL} to  characterise satisfiability of  $\prop{\varphi}$ in a $\varphi$-consistent model by the existence of a  $\varphi$-consistent valuation satisfying suitable corresponding properties.
%
%
%We
%%\nb{M: moved footnote here to save space}
%%\nb{M: todo fix notation}
%assume that the 
%primitive
%%propositional
%connectives used to build 
%%the
%propositional formulas are $\neg$ %, $\vee$, 
%and $\wedge$ ($\vee$ is expressed using 
%$\neg$ %, $\vee$, 
%and $\wedge$),
%%Moreover,
%and
%we
%denote by  ${\sf sub}(\prop{\varphi})$ the set 
%of subformulas of $\prop{\varphi}$ closed under single negation.  
%A \emph{valuation} for a propositional
%$\MLn$
%%modal logic
%formula $\prop{\varphi}$   
%is a function $\nu: {\sf sub} (\prop{\varphi})\rightarrow \{0,1\}$ that 
%satisfies the following conditions:
%%\footnote{Assuming that the 
%%primitive propositional connectives used to build 
%%the formulas are $\neg$ %, $\vee$, 
%%and $\wedge$ ($\vee$ is expressed using 
%%$\neg$ %, $\vee$, 
%%and $\wedge$).}:
%(1) for all $\neg\psi\in {\sf sub} (\prop{\varphi})$,
%$\nu(\psi)=1$ iff $\nu(\neg\psi)=0$;
%%(2) for all $\psi_1\vee \psi_2\in {\sf sub} (\prop{\varphi})$,
%%$\nu(\psi_1\vee \psi_2) = 1$ iff $\nu(\psi_1) = 1$
%%or $\nu(\psi_2) = 1$; 
%(2) for all $\psi_1\wedge \psi_2\in {\sf sub} (\prop{\varphi})$,
%$\nu(\psi_1\wedge \psi_2) = 1$ iff $\nu(\psi_1) = 1$
%and $\nu(\psi_2) = 1$; 
%and (3) $\nu(\prop{\varphi})=1$. 
%We say that a valuation for $\prop{\varphi}$ 
%is \emph{$\varphi$-consistent} if any %propositional
%neighbourhood model of the form $(\{w\}, \{ \propneigh_{i} \}_{i \in [1, n]}, \propassign)$ satisfying
%$w\in \propassign(p_\elaxiom)$ iff $\nu(p_\elaxiom)=1$, for all $p_\elaxiom\in \NPr(\varphi)$, 
% is $\varphi$-consistent.
%We now establish that satisfiability of 
%$\prop{\varphi}$ in a $\varphi$-consistent model is characterized 
%by the existence of a  $\varphi$-consistent valuation 
%satisfying the property described in Lemma~\ref{lem:propsat}.
%%
%\begin{restatable}{lemma}{Lemmapropsat}\label{lem:propsat}
%A formula $\prop{\varphi}$ is satisfied in a $\varphi$-consistent $\E^{n}$ model
%iff
%%if, and only if,
%there is   a $\varphi$-consistent valuation \valuation 
%for $\prop{\varphi}$ such that if $\B_i\psi_1$ and $\B_i\psi_2$ are in 
%${\sf sub}(\prop{\varphi})$, $\valuation(\B_i\psi_1)=1$, and 
%$\valuation(\B_i\psi_2)=0$, then $(\psi_1\wedge\neg\psi_2)\vee(\neg\psi_1\wedge\psi_2)$
%is satisfied in a $\varphi$-consistent $\E^{n}$ model. 
%\end{restatable}
%%
% \begin{proof}
% ($\Rightarrow$) Suppose that $\prop{\varphi}$ is satisfied in a world $w$ of a $\varphi$-consistent model 
% $\propmodel = (\propdomain, \{ \propneigh_{i} \}_{i \in [1,n]}, \propassign)$. That is, 
% $\propmodel, w\models \varphi$. We define a $\varphi$-consistent valuation for 
% $\prop{\varphi}$
% by setting $\nu(\psi)=1$ if $\propmodel, w\models \psi$ and $\nu(\psi) = 0$
% if  $\propmodel, w\not\models \psi$. 
% It is easy to check that $\nu$ is indeed a 
% $\varphi$-consistent valuation (given that $\propmodel$ is 
% $\varphi$-consistent). Assume that 
% $\B_i\psi_1$ and $\B_i\psi_2$ are in ${\sf sub}(\prop{\varphi})$, 
% $\nu(\B_i\psi_1)=1$ and $\nu(\B_i\psi_2)=0$. Then $\propmodel,w\models \B_i\psi_1$
% and $\propmodel,w\not\models \B_i\psi_2$. Thus, by definition, 
% $\propassign(\psi_1)\in \propneigh_i(w)$ and 
%  $\propassign(\psi_2)\not\in \propneigh_i(w)$.
%%Using the same argument of Lemma 3.1 in~\cite{Var2}, it follows that 
%So,
%   $\propassign(\psi_1)\neq \propassign(\psi_2)$.
%Then %it is easy to see that
% there 
%   is a world $u$ in the symmetrical difference of these sets 
%   such that $\propmodel,u\models (\psi_1\wedge\neg\psi_2)\vee(\neg\psi_1\wedge\psi_2)$. 
%   
% ($\Leftarrow$) Suppose there is a $\varphi$-consistent valuation $\nu$ for $\prop{\varphi}$ such that 
% if $\B_i\psi_1$ and $\B_i\psi_2$ are in 
%${\sf sub}(\prop{\varphi})$, $\valuation(\B_i\psi_1)=1$, and 
%$\valuation(\B_i\psi_2)=0$, then 
%there is a $\varphi$-consistent model \[\propmodel_{\psi_1,\psi_2}=(\propdomain_{\psi_1,\psi_2},
%\{ \propneigh_{{\psi_1,\psi_2}_{i}} \}_{i \in [1,n]},\propassign_{\psi_1,\psi_2})\]
% and a world 
%$w_{\psi_1,\psi_2}\in \propdomain_{\psi_1,\psi_2}$ such that 
%$\propmodel_{\psi_1,\psi_2}, w_{\psi_1,\psi_2}\models (\psi_1\wedge\neg\psi_2)\vee(\neg\psi_1\wedge\psi_2)$. 
%
%Let $\propmodel_1,\ldots,\propmodel_m$ be an enumeration of the models 
%$\propmodel_{\psi_1,\psi_2}$, as above.  That is, we take one model for each 
%pair $\psi_1,\psi_2$ where $\propmodel_j = (\propdomain_j, \{ \propneigh_{j_{i}} \}_{i \in [1,n]},\propassign_j)$, 
%and let $w_1,\ldots,w_m$ be an enumeration of the worlds $w_{\psi_1,\psi_2}$, 
%with $w_j\in \propdomain_j$. We assume w.l.o.g. that $\propdomain_j\cap \propdomain_k=\emptyset$ 
%for $j\neq k$. 
%
%In the following, we define a $\varphi$-consistent  model   $\propmodel = (\propdomain,\{ \propneigh_{i} \}_{i \in [1,n]}, \propassign)$ 
%for $\prop{\varphi}$. 
%Intuitively, we construct $\propmodel$ by taking the union of each 
%$\propmodel_j$ with the addition of a new world $w$ that 
%will satisfy $\prop{\varphi}$. 
%We define $\propdomain$ as $\bigcup_{1\leq j\leq n}\propdomain_j\cup \{w\}$, 
%where $w$ is fresh.
%%The tricky part of the proof is to define the assignment $\propassign$. 
%Before defining $\propneigh_{i}$ and $\propassign$, we define the function $J: {\sf sub}(\prop{\varphi})\rightarrow \Pmc(\Wmc)$
%with $J(\psi)=\bigcup_{0\leq j \leq m} \Vmc_j(\psi)$ for all $\psi\in {\sf sub}(\prop{\varphi})$, where %$I_i$ is as above for $1\leq i\leq n$, 
% %and
%$\Vmc_0: {\sf sub}(\varphi)\rightarrow  \Pmc(\{w\})$ is the function
%that assigns $\psi$ to $\{w\}$, if $\nu(\psi)=1$, 
%and to $\emptyset$, otherwise ($\Vmc_j$, for $1\leq j\leq m$, is as above).
%By construction, we have that $J(\neg \psi)=\propdomain\setminus J(\psi)$
%and $J(\psi_1\wedge \psi_2)=J(\psi_1)\cap J(\psi_2)$. 
%We define the assignment $\propassign$ as the function 
%$\propassign: \NPr(\varphi)\rightarrow \Pmc(\Wmc)$ satisfying 
% $\propassign(p_\elaxiom)=J(p_\elaxiom)$ for all $p_\elaxiom\in \NPr(\varphi)$. 
% 
%It remains to define $\propneigh_i$, for $1 \leq i \leq n$. 
%For $u\in \propdomain_j$ and $\B_i\psi \in {\sf sub}(\varphi)$, we   put $J(\psi) \subseteq \Wmc$ in $\propneigh_i(u)$ 
%precisely when $\propmodel_j,u \models \B_i \psi$.  
%%and $\alpha = J(\psi_\alpha)$
%%for some $\B_i\psi_\alpha \in {\sf sub}(\varphi)$.
%We claim that 
%if $\beta \in \Nmc_i(u)$ and $\beta = J(\psi)$ for some $\B_i\psi\in{\sf sub}(\prop{\varphi})$,
%then $\propmodel_j,u\models\B_i\psi$.
%Indeed, since $\beta = J(\psi)\in \Nmc_i(u)$, 
%we must have that $\propmodel_j,u\models \B_i\psi_{\beta}$ and  $\beta = J(\psi_{\beta})$ for 
%some $\B_i\psi_\beta \in{\sf sub}(\prop{\varphi})$.
%But since
%$J(\psi)=J(\psi_\beta)$, we also have $\propassign_j(\psi)=\propassign_j(\psi_\beta)$ 
%(recall that $\propdomain_j \cap \propdomain_k = \emptyset$ for  
%$k \neq j$), so $\propmodel_j,u\models \B_i\psi$ iff 
%$\propmodel_j,u \models \B_i \psi_\beta$.
%It follows that $\propmodel_j,u\models \B_i\psi$.
%
%Also, we put $\alpha \subseteq \Wmc$ in $\propneigh_i(w)$ (recall $w$ is the fresh 
%world introduced above in $\propdomain$) precisely 
%when $\nu(\B_i\psi_\alpha) = 1$ and $\alpha = J(\psi_\alpha)$ for some 
%$\B_i\psi_\alpha \in {\sf sub}(\prop{\varphi})$.
%We claim that if 
%$\beta \in \propneigh_i(w)$ and $\beta = J(\psi)$ for some $\B_i \psi \in{\sf sub}(\prop{\varphi})$ 
%then $\nu(\B_i \psi) = 1$.
%Indeed, since $\beta = J(\psi)\in \propneigh_i(u)$ 
%we must have that $\nu(\B_i\psi_\beta)=1$ and $\beta = J(\psi_\beta)$ for some 
%$\B_i\psi_\beta \in{\sf sub}(\prop{\varphi})$. 
%Suppose now that $\nu(\B_i\psi) = 0$. Then, by assumption, there exists a structure
%$\propmodel_j = (\propdomain_j, \{ \propneigh_{j_i} \}_{i \in [1, n]}, \propassign_j)$ and a world $w_j
%\in \propdomain_j$ such that $\propmodel_j,w_j \models (\psi_\beta \wedge \neg\psi)\vee(\neg\psi_\beta \wedge \psi)$. 
%It follows that $\propassign_j(\psi_\beta)\neq \propassign_j(\psi)$. 
%Consequently $J(\psi_\beta)\neq J(\psi)$, which is a contradiction.  
%%\nb{O :... talk about consistent}
%
%We now show by induction on the structure of formulas 
%that $\propassign$ and $J$ agree on ${\sf sub}(\prop{\varphi})$. 
%This holds by construction for atomic propositions. It is easy to deal 
%with propositional connectives, since we know that $J(\neg \psi)=\propdomain\setminus J(\neg \psi)$
%and  $J(\psi_1\wedge \psi_2)=J(\psi_1)\cap J(\psi_2)$ 
%and similarly for $\propassign$. Assume inductively that $\propassign(\psi) = J(\psi)$.
%Suppose first that $u\in J(\B_i\psi)$. Then, either $u=w$ and $\nu(\B_i\psi)=1$
%or $u\in \propdomain_j$ and $\propmodel_j,u\models\B_i\psi$. In either case 
%we have that $J(\psi)\in \propneigh_i(u)$. Since 
%$\propassign(\psi) = J(\psi)$, it follows that $\propmodel,u\models \B_i\psi$, 
%that is, $u\in \propassign(\B_i\psi)$. Suppose now that $u\in \propassign(\B_i\psi)$, 
%that is, $\propmodel,u\models \B_i\psi$, or, equivalently, $\propassign(\psi)\in \propneigh_i(u)$.
%Since $\propassign(\psi) = J(\psi)$ it follows that either $u=w$ and 
%$\nu(\B_i\psi)=1$ or $u\in \propdomain_j$ and $\propmodel_j,u\models \B_i\psi$. 
%In either case we have that $u\in J(\B_i\psi)$. 
%
%Since $\nu(\prop{\varphi})=1$, we have that $w\in J(\prop{\varphi})$, 
%and consequently $w\in \propassign(\prop{\varphi})$. That 
%is, $\propmodel,w\models\prop{\varphi}$. 
%The fact that $\propmodel$ is $\varphi$-consistent follows from 
%the fact that $\nu$, used to construct the assignment 
%related to $w$, is $\varphi$-consistent and 
%the models $\propmodel_1,\ldots,\propmodel_m$, used to define 
%the remaining worlds in $\Wmc$, are all $\varphi$-consistent. 
%\end{proof}
%
Finally, to check satisfiability of $\prop{\varphi}$ in a $\varphi$-consistent model, we \ldots
%use Lemma~\ref{lem:propsat} and the fact that
%there are only quadratically many formulas of the form $\psi_1\wedge\neg\psi_2$, 
%where $\psi_1$ and $\psi_2$ are subformulas of $\prop{\varphi}$. 
%We observe that satisfiability in \ALC is \ExpTime-complete~\cite{GabEtAl03}
%and so, one can determine in exponential 
%time whether a valuation is $\varphi$-consistent.
%For an $\ExpTime$ upper bound, one can deterministically compute 
%all possible $\varphi$-consistent valuations for $\psi_1\wedge\neg\psi_2$ 
%and decide satisfiability of $\prop{\varphi}$ by a $\varphi$-consistent
%model using a bottom-up strategy (as in~\cite{Var2}). 
%As satisfiability in \ALC 
%is \ExpTime-hard  our upper bound is tight.
This leads us to the following tight complexity result.

\begin{restatable}{theorem}{Theorempropsat}\label{thm:propsat}
The $\LnALCg$ formula satisfiability problem on constant domain neighbourhood models is $\ExpTime$-complete.
\end{restatable}

%To determine satisfiability of $\prop{\varphi}$ in a $\varphi$-consistent
%model, we use Lemma~\ref{lem:propsat} and the fact that
%there are only quadratically many formulas of the form $\psi_1\wedge\neg\psi_2$, 
%where $\psi_1$ and $\psi_2$ are subformulas of $\prop{\varphi}$. 
%We observe that satisfiability in \ALC is \ExpTime-complete~\cite{GabEtAl03}
%%~\cite{dlhandbook},
%%\nb{M: Changed ref. to shrink bib} 
%and so, one can determine in exponential 
%time whether a valuation is $\varphi$-consistent.
%For an $\ExpTime$ upper bound, one can deterministically compute 
%all possible $\varphi$-consistent valuations for $\psi_1\wedge\neg\psi_2$ 
%and decide satisfiability of $\prop{\varphi}$ by a $\varphi$-consistent
%model using a bottom-up strategy (as in~\cite{Var2}). 
%As satisfiability in \ALC 
%%and in the propositional modal logic \E{} 
%is \ExpTime-hard  our upper bound is tight.
%
%\begin{restatable}{theorem}{Theorempropsat}\label{thm:propsat}
%The $\EALCg$ formula satisfiability problem on constant domain neighbourhood models is $\ExpTime$-complete.
%\end{restatable}

%Regarding the proof for \MALCg, we first point out that 
%Lemma~\ref{lem:propE} can be easily adapted to \MALCg. 
%%the proof for the propositional case $\Mn^n$~\cite{Var2} can be similarly 
%%adapted to our combination with \ALC.
%%~ Lemma~\ref{lem:prop} can be easily adapted to \MALCg. 
%%~ We now formalize a variant of Lemma~\ref{lem:propsat} 
%%~ for \MALCg. 
%%We   establish a variant of Lemma~\ref{lem:propsat} 
%%tailored for $\Mn^{n}$ (see Proposition 3.8 in~\cite{Var2}).
%The proof for our $\ExpTime$ result for \MALCg 
%is analogous to the one given for \EALCg, except that here  we use 
%a variant of 
%%the following variant of
%Lemma~\ref{lem:propsat} 
%tailored for $\EM^{n}$
%%\nb{M: todo fix}
%(see Proposition 3.8 in~\cite{Var2}).
%%Lemma~\ref{lem:propsatm} and an adaptation of Lemma~\ref{lem:prop}.
%%\begin{restatable}{lemma}{Lemmapropsatm}\label{lem:propsatm}
%%A formula $\prop{\varphi}$ is $\Mn^{n}$ satisfiable by a $\varphi$-consistent model
%%if, and only if, there is   a $\varphi$-consistent valuation \valuation 
%%for $\prop{\varphi}$ such that if $\B_i\psi_1$ and $\B_i\psi_2$ are in 
%%${\sf sub}(\prop{\varphi})$, $\valuation(\B_i\psi_1)=1$, and 
%%$\valuation(\B_i\psi_2)=0$, then $\psi_1\wedge\neg\psi_2$
%%is $\Mn^{n}$ satisfiable by a $\varphi$-consistent model. 
%%\end{restatable}
%%the proof for the propositional case $\Mn^n$~\cite{Var2} can be similarly 
%%adapted to our combination with \ALC. 
%Thus, we obtain also the following result.
%%\nb{M: changed a bit. Do you prefer the previous sentence? Fine with me}
%%We are now ready for Theorem~\ref{thm:propsat}. 
%\begin{restatable}{theorem}{Theorempropsat}\label{thm:propsat}
%The $\MALCg$ formula satisfiability problem on constant domain neighbourhood models is $\ExpTime$-complete.
%%Satisfiability  in \MALCg %and \EEL{g} 
%% is \ExpTime-complete.
% % and 
%%\NP-complete, respectively. 
%\end{restatable}
%
%%~ Regarding \MALCg, the proof for the propositional case $\Mn^n$~\cite{Var2} can be similarly 
%%~ adapted to our combination with \ALC. 
%
%%~ \begin{restatable}{theorem}{Theorempropsat}\label{thm:propsat}
%%~ Satisfiability  in \MALCg %and \EEL{g} 
% %~ is \ExpTime-complete.
% %~ % and 
%%~ %\NP-complete, respectively. 
%%~ \end{restatable}
%
%%\paragraph{{\color{red}{M: $\ExpTime$-completeness of $\EALCg$ (and $\MALCg$, analogously)}}}
%%(Sketch, cf.~\cite{Baader:2012:LOD:2287718.2287721})
%%\nb{M: Check!}
%%
%%\medskip
%%\noindent
%%Upper bound:
%%
%%\medskip
%%1 -- Given a $\MLALC{}$ formula $\p$, compute its propositional abstraction $\prop{\p}$.
%%
%%\medskip
%%2 -- Define:
%%\[
%%S := \{ X_1, \ldots, X_k \} \subseteq \NPr(\{ p_1, \ldots, p_n \})
%%\]
%%\[
%%\p_{X_{i}} := (\bigwedge_{p_{j} \in X_{i}} \pi_{j} \land \bigwedge_{p_{j} \not \in X_{i}} \pi_{j} );
%%\qquad
%%\p_{X} := \bigwedge_{1 \leq i \leq k} \p_{X_{i}}
%%\]
%%\[
%%\prop{\p}^{S} := \prop{\p} \land \bigwedge_{1 \leq i \leq 3} \Box_{i} (\bigvee_{X \in S} (\bigwedge_{p \in X} p \land \bigwedge_{p \not \in X} p ))
%%\]
%%
%%\medskip
%%3 -- Prove:
%%
%%\begin{lemma}
%%A $\MLALC{3}^{g}$ formula $\p$ is satisfiable iff there is a set $S = \{ X_1, \ldots, X_k \} \subseteq \NPr(\{ p_1, \ldots, p_n \})$ such that the $\ML^{3}$ formula $\prop{\p}^{S}$ and the $\ALC$ formula $\p_{X}$ are satisfiable.
%%\end{lemma}
%%\begin{proof}
%%\end{proof}
%%
%%
%%\medskip
%%4 -- Define $S^{*}$ as the set of all subsets $X \subseteq \{p_1, \ldots, p_n \}$ such that the following $\ALC$ formula is satisfiable:
%%\[
%%\p_X := (\bigwedge_{p_{j} \in X} \pi_{j} \land \bigwedge_{p_{j} \not \in X} \pi_{j} )
%%\]
%%\[
%%\prop{\p}^{S^{*}} := \prop{\p} \land \bigwedge_{1 \leq i \leq 3} \Box_{i} (\bigvee_{X \in S^{*}} (\bigwedge_{p \in X} p \land \bigwedge_{p \not \in X} p ))
%%\]
%%
%%
%%\medskip
%%5 -- Prove:
%%
%%\begin{lemma}
%%Let $\p_1, \ldots, \p_k$ be $\ALC$ formulas over disjoint sets of individual, concept, and role names. Then $\p_1 \land \ldots \land \p_k$ is satisfiable iff, for each $i \in \{ 1, \ldots, k \}$, $\p_i$ is satisfiable.
%%\end{lemma}
%%\begin{proof}
%%\end{proof}
%%
%%\medskip
%%6 -- Prove:
%%\begin{lemma}
%%A $\MLALC{3}^g$ formula $\p$ is satisfiable iff the $\ML^{3}$ formula $\prop{\p}^{S^{*}}$ is satisfiable.
%%\end{lemma}
%%\begin{proof}
%%\end{proof}
%%
%%\medskip
%%7 -- Since computing $S^{*}$ and checking satisfiability of $\prop{\p}^{S^{*}}$ can be done in exponential time in the size of $\p$, using the translation $\cdot\tr$, we hae that the $\EALCg$ satisfiability problem is in $\ExpTime$.
%%
%%
%%\medskip
%%\noindent
%%Lower bound: from $\ExpTime$-hardness of $\ALC$.
%%
%%\begin{theorem}
%%The $\EALCg$ satisfiability problem is $\ExpTime$-complete.
%%\end{theorem}
%
%
%
%
%
%
%
%
%
%%\paragraph{Satisfiability in $\CALCg$  and $\NALCg$}
%%
%%In~\citeauthor{DL19}~\cite{DL19}, it is shown that 
%%$\EALCg$  and $\MALCg$ formula satisfiability problems on constant domain neighbourhood models are 
%%$\ExpTime$-complete.
%%
%We now show tight complexity results %$\ExpTime$ upper bounds 
%for $\CALCg$  and $\NALCg$,
%using again the notion of 
%a propositional abstraction of a formula 
%(as in, e.g.,~\cite{Baader:2012:LOD:2287718.2287721}).
%%Since $\ALC$ formula satisfiability is already $\ExpTime$-hard, 
%%our upper bounds here are tight. 
%%we have 
%%a tight complexity result for the global cases.
%%We show that
%Here, one can separate the satisfiability test into two parts, 
%one for the description logic dimension and 
%one for the 
%%dimension of the
%%\neighborhood
%%frame. 
%modal dimension.
%%~ In this subsection, we also consider the lightweight DL called \EL, defined 
%%~ as the fragment of \ALC which only allows conjunctions and existential quantification 
%%~ in concept expressions. 
%%
%%Consider $\Lmc\in\{\ALC,\EL\}$. 
%%\newcommand{\propmodel}{\ensuremath{M}\xspace}
%%\newcommand{\propdomain}{\ensuremath{W}\xspace}
%%\newcommand{\propneigh}{\ensuremath{N}\xspace}
%%\newcommand{\propassign}{\ensuremath{I}\xspace}
%The 
%\emph{propositional  abstraction} $\prop{\varphi}$ of an
%$\MLnALCg$
%%$\CALCg$ (respectively, $\NALCg$)
%formula $\varphi$ is  
%the result of replacing each $\ALC$
%CI
%%atom
%in $\varphi$ by 
%a propositional letter $p$, so that there is a $1:1$ relationship 
%between the $\ALC$
%CI
%%atoms
%$\elaxiom$ occurring in $\varphi$ and the 
%propositional letters $p_{\elaxiom}$ used for the abstraction.  
%%The semantics of $\prop{\varphi}$ is given in terms of \emph{propositional 
%%neighbourhood models} $ (\Wmc, \{ \Nmc_{i} \}_{i \in I}, \Vmc)$ for $\EC^{n}$ (respectively, $\EN^{n}$), 
%%where $(\Wmc, \{ \Nmc_{i} \}_{i \in I})$ is a neighbourhood frame 
%%and $\Vmc: \NPr \rightarrow 2^{\Wmc}$ is a function 
%%mapping propositional letters in $\NPr$ to
%%sets of worlds
%%%subsets of the domain of worlds
%%(see~\cite{Che,Var2}). 
%We set $\NPr(\varphi) = \{p_{\elaxiom}\in\NPr\mid \elaxiom \text{ is an \ALC
%CI
%%atom
%in }
%\varphi \}$.
%%
%Given an
%$\MLnALCg$
%%\nb{M: todo fix}
%%$\CALCg$ (respectively, $\NALCg$)
%formula $\varphi$, we say that a propositional neighbourhood model 
%$\propmodel = (\Wmc, \{ \Nmc_{i} \}_{i \in I}, \Vmc)$
%of $\prop{\varphi}$
%is \emph{$\consistent{\varphi}$}
%if, for all $w\in \Wmc$,
%the following \ALC formula is satisfiable $$\textstyle\bigwedge_{p_{\elaxiom}\in \NPr(w)} \ {\elaxiom} \ \wedge \
%\bigwedge_{p_{\elaxiom}\in \overline{\NPr(w)}}\ \neg {\elaxiom},$$
%where $\NPr(w) = \{p_{\elaxiom}\in \NPr(\varphi) \mid w\in \Vmc(p_{\elaxiom})\}$
%and $\overline{\NPr(w)}=\NPr(\varphi)\setminus\NPr(w)$. 
%We now formalise the connection between %complexity of the satisfiability problem for 
%$\MLnALCg$
%%$\CALCg$ and $\NALCg$
%formulas and their propositional abstractions %with consistent models
%with the following lemma, where $\mathit{L} \in \{ \EC, \EN \}$, obtained by adapting the proof of Lemma~\ref{lem:propE}
%%~\citeauthor{DL19}~\cite[Lemma~1]{DL19}.
%%, which is an adaptation of the 
%%results obtained for other \ALC extensions~\cite{Baader:2012:LOD:2287718.2287721}.
%%~ %We say that a model of $\prop{\varphi}$
%%~ %is $\varphi$-consistent
%
%%\nb{$\varphi$-consistent model}
%
%\begin{restatable}{lemma}{Lemmaprop}\label{lem:satfraglog}
%A formula $\varphi$ is $\LnALCg$ satisfiable
%on constant domain neighbourhood models
%iff
%%if, and only if, 
%$\prop{\varphi}$ is satisfied in a $\varphi$-consistent $\mathit{L}^{n}$ model.  
%\end{restatable}
%
%%\begin{restatable}{lemma}{Lemmaprop}\label{lem:general}
%%A formula $\prop{\varphi}$ is satisfied in a $\varphi$-consistent  
%%$\mathit{L}^{n}$
%%model
%%iff
%%if, and only if,
%%there is   a $\varphi$-consistent valuation \valuation 
%%for $\prop{\varphi}$ such that 
%%if $\B_i\psi_1,   \B_i\psi_2$ are in 
%%${\sf sub}(\prop{\varphi})$, $\valuation(\B_i\psi_1)=1$   and 
%%$\valuation(\B_i\psi_2)=0$, then either $(\psi_1\wedge\neg\psi_2)$ or 
%%$(\neg\psi_1\wedge\psi_2)$ 
%%is satisfied in a $\varphi$-consistent $\mathit{L}^{n}$
% %model. 
%%\begin{itemize}
%%\item if $\B_i\psi$ is in ${\sf sub}(\prop{\varphi})$ and
%%$\valuation(\B_i\psi_1)=0$ then $\neg \psi$ is 
%%$\EC^{n}$ satisfiable; and 
%%\item 
%%\end{itemize}
%%\end{restatable}
%
%We
%%\nb{M: moved footnote here to save space}
%%\nb{M: todo fix notation}
%assume that the 
%primitive
%%propositional
%connectives used to build 
%%the
%propositional formulas are $\neg$ %, $\vee$, 
%and $\wedge$ ($\vee$ is expressed using 
%$\neg$ %, $\vee$, 
%and $\wedge$),
%%Moreover,
%and
%we
%denote by  ${\sf sub}(\prop{\varphi})$ the set 
%of subformulas of $\prop{\varphi}$ closed under single negation.  
%A \emph{valuation} for a propositional
%$\MLn$
%%modal logic
%formula $\prop{\varphi}$   
%is a function $\nu: {\sf sub} (\prop{\varphi})\rightarrow \{0,1\}$ that 
%satisfies the following conditions:
%%\footnote{Assuming that the 
%%primitive propositional connectives used to build 
%%the formulas are $\neg$ %, $\vee$, 
%%and $\wedge$ ($\vee$ is expressed using 
%%$\neg$ %, $\vee$, 
%%and $\wedge$).}:
%(1) for all $\neg\psi\in {\sf sub} (\prop{\varphi})$,
%$\nu(\psi)=1$ iff $\nu(\neg\psi)=0$;
%%(2) for all $\psi_1\vee \psi_2\in {\sf sub} (\prop{\varphi})$,
%%$\nu(\psi_1\vee \psi_2) = 1$ iff $\nu(\psi_1) = 1$
%%or $\nu(\psi_2) = 1$; 
%(2) for all $\psi_1\wedge \psi_2\in {\sf sub} (\prop{\varphi})$,
%$\nu(\psi_1\wedge \psi_2) = 1$ iff $\nu(\psi_1) = 1$
%and $\nu(\psi_2) = 1$; 
%and (3) $\nu(\prop{\varphi})=1$. 
%We say that a valuation for $\prop{\varphi}$ 
%is \emph{$\varphi$-consistent} if any
%propositional
%neighbourhood
%model of the form $(\{w\}, \{ \propneigh_{i} \}_{i \in I}, \propassign)$ satisfying
%$w\in \propassign(p_\elaxiom)$ iff $\nu(p_\elaxiom)=1$, for all $p_\elaxiom\in \NPr(\varphi)$, 
% is $\varphi$-consistent.
%We now establish that satisfiability of 
%$\prop{\varphi}$ in a $\varphi$-consistent $\EC^{n}$ (respectively, $\EN^{n}$) model is characterized 
%by the existence of a  $\varphi$-consistent valuation 
%satisfying the property described in Lemma~\ref{lem:proplemma2} 
%(respectively, Lemma~\ref{lem:proplemmaN}).
%
%\begin{restatable}{lemma}{Lemmapropsec}\label{lem:proplemma2}
%A formula $\prop{\varphi}$ is satisfied in a $\varphi$-consistent $\EC^{n}$ %\nb{check macro}
%model
%iff
%%if, and only if,
%there is   a $\varphi$-consistent valuation \valuation 
%for $\prop{\varphi}$ such that 
%if $\B_i\psi_1, \dots, \B_i\psi_k$ are in 
%${\sf sub}(\prop{\varphi})$, $\valuation(\B_i\psi_j)=1$ for all $1\leq j < k$, and 
%$\valuation(\B_i\psi_k)=0$, then either $(\bigwedge^{k-1}_{j=1}\psi_j\wedge\neg\psi_k)$ or 
%$(\neg\psi_j\wedge\psi_k)$ for some $1\leq j < k$
%is satisfied in a $\varphi$-consistent $\EC^{n}$ %\nb{check macro}
% model. 
%%\begin{itemize}
%%\item if $\B_i\psi$ is in ${\sf sub}(\prop{\varphi})$ and
%%$\valuation(\B_i\psi_1)=0$ then $\neg \psi$ is 
%%$\EC^{n}$ satisfiable; and 
%%\item 
%%\end{itemize}
%\end{restatable}
%%
%\begin{proof}
%%\paragraph{Point 2}
%($\Rightarrow$) Suppose that $\prop{\varphi}$ is satisfied in a world $w$ of a $\varphi$-consistent $\EC^{n}$ model 
% $\propmodel = (\propdomain, \{ \propneigh_{i} \}_{i \in I}, \propassign)$. That is, 
% $\propmodel, w\models \prop{\varphi}$. We define a $\varphi$-consistent valuation for 
% $\prop{\varphi}$
% by setting $\nu(\psi)=1$ if $\propmodel, w\models \psi$ and $\nu(\psi) = 0$
% if  $\propmodel, w\not\models \psi$. 
% It is easy to check that $\nu$ is indeed a 
% $\varphi$-consistent valuation (given that $\propmodel$ is a  
% $\varphi$-consistent $\EC^{n}$ model). Assume  $\B_i\psi_1, \dots, \B_i\psi_k$ are in 
%${\sf sub}(\prop{\varphi})$, $\valuation(\B_i\psi_j)=1$ for all $1\leq j < k$, and 
%$\valuation(\B_i\psi_k)=0$.
%%
%Then $\propmodel,w\models \B_i\psi_j$ for all $1\leq j < k$,
% and $\propmodel,w\not\models \B_i\psi_k$.
%By definition, $(\B_i \psi_1 \wedge \ldots \wedge \B_i \psi_{k-1})\rightarrow \B_i (\psi_1\wedge \ldots \wedge \psi_{k-1})$ holds in $\EC^{n}$ models.
%So $\propmodel,w\models \B_i (\psi_1\wedge \ldots \wedge \psi_{k-1})$ 
% and $\propmodel,w\not\models \B_i\psi_k$.
%This means that $\valuation(\B_i(\bigwedge^{k-1}_{j=1}\psi_j))=1$
%while $\valuation(\B_i\psi_k)=0$.
%  By definition, 
% $\propassign(\bigwedge^{k-1}_{j=1}\psi_j)\in \propneigh_i(w)$   and 
%  $\propassign(\psi_k)\not\in \propneigh_i(w)$.
%%Using the same argument of Lemma 3.1 in~\cite{Var2}, it follows that 
%So,
%   $\propassign(\bigwedge^{k-1}_{j=1}\psi_j)\neq \propassign(\psi_k)$.
%Then, %it is easy to see that 
%there 
%   is a world $u$ in the symmetrical difference of these sets 
%   such that $\propmodel,u\models (\bigwedge^{k-1}_{j=1}\psi_j\wedge\neg\psi_k)\vee(\neg(\bigwedge^{k-1}_{j=1}\psi_j)\wedge\psi_k)$. 
%   
%%Assume  $\B_i\psi_1, \dots, \B_i\psi_k$ are in 
%%${\sf sub}(\prop{\varphi})$, $\valuation(\B_i\psi_j)=1$ for all $1\leq j < k$, and 
%%$\valuation(\B_i\psi_k)=0$.
%%By definition, $(\B_i \psi_1 \wedge \ldots \wedge \B_i \psi_{k-1})\rightarrow \B_i %(\psi_1\wedge \ldots \wedge \psi_{k-1})$ holds in $\EC^{n}$ models.
%%\nb{add this def somewhere} 
%%This means that $\valuation(\B_i(\bigwedge^{k-1}_{j=1}\psi_j))=1$.
%
%($\Leftarrow$) Suppose there is a $\varphi$-consistent valuation $\nu$ for $\prop{\varphi}$ such that 
%%there is   a $\varphi$-consistent valuation \valuation 
%%for $\prop{\varphi}$ such that 
%if $\B_i\psi_1, \dots, \B_i\psi_k$ are in 
%${\sf sub}(\prop{\varphi})$, $\valuation(\B_i\psi_j)=1$ for all $1\leq j < k$, and 
%$\valuation(\B_i\psi_k)=0$, then 
%there is a $\varphi$-consistent $\EC^{n}$ model $$\propmodel_{\bigwedge^{k-1}_{j=1}\psi_j,\psi_k}=(\propdomain_{\bigwedge^{k-1}_{j=1}\psi_j,\psi_k},\{ \propneigh_{{\bigwedge^{k-1}_{j=1}\psi_j,\psi_k}_{i}} \}_{i \in I},\propassign_{\bigwedge^{k-1}_{j=1}\psi_j,\psi_k})$$ and a world 
%$w_{\bigwedge^{k-1}_{j=1}\psi_j,\psi_k}\in \propdomain_{\bigwedge^{k-1}_{j=1}\psi_j,\psi_k}$ such that 
%$$\propmodel_{\bigwedge^{k-1}_{j=1}\psi_j,\psi_k}, w_{\bigwedge^{k-1}_{j=1}\psi_j,\psi_k}\models ((\bigwedge^{k-1}_{j=1}\psi_j)\wedge\neg\psi_k)\vee(\neg(\bigwedge^{k-1}_{j=1}\psi_j)\wedge\psi_k).$$ 
%
%%either $(\bigwedge^{k-1}_{j=1}\psi_j\wedge\neg\psi_k)$ or 
%%$(\neg\psi_j\wedge\psi_k)$ for some $1\leq j < k$
%%is satisfied in a $\varphi$-consistent $\EC^{n}$ %\nb{check macro}
%% model. 
%% if $\B_i\psi_1$ and $\B_i\psi_2$ are in 
%%${\sf sub}(\prop{\varphi})$, $\valuation(\B_i\psi_1)=1$, and 
%%$\valuation(\B_i\psi_2)=0$, then 
%%there is a model $\propmodel_{\psi_1,\psi_2}=(\propdomain_{\psi_1,\psi_2},
%%\{ \propneigh_{{\psi_1,\psi_2}_{i}} \}_{i \in I},\propassign_{\psi_1,\psi_2})$
%% and a world 
%%$w_{\psi_1,\psi_2}\in \propdomain_{\psi_1,\psi_2}$ such that 
%%$\propmodel_{\psi_1,\psi_2}, w_{\psi_1,\psi_2}\models (\psi_1\wedge\neg\psi_2)\vee(\neg\psi_1\wedge\psi_2)$. 
%Let $\propmodel_1,\ldots,\propmodel_m$ be an enumeration of the models 
%$\propmodel_{\bigwedge^{k-1}_{j=1}\psi_j,\psi_k}$, as above.  That is, we take one model $\propmodel_{\bigwedge^{k-1}_{l=1}\psi_l,\psi_k}$ for each 
%pair $j=\bigwedge^{k-1}_{l=1}\psi_l,\psi_k$ 
%where $\propmodel_j = (\propdomain_j, \{ \propneigh_{j_{i}} \}_{i \in I},\propassign_j)$, 
%and let $w_1,\ldots,w_m$ be an enumeration of the worlds
% $w_{\bigwedge^{k-1}_{l=1}\psi_l,\psi_k}$, 
%with $j=\bigwedge^{k-1}_{l=1}\psi_l,\psi_k$ and  $w_j\in \propdomain_j$. We assume without loss of generality that $\propdomain_j\cap \propdomain_k=\emptyset$ 
%for $j\neq k$. 
%
%In the following, we define a $\varphi$-consistent $\EC^{n}$ model   $\propmodel = (\propdomain,\{ \propneigh_{i} \}_{i \in I}, \propassign)$ 
%for $\prop{\varphi}$. 
%Intuitively, we construct $\propmodel$ by taking the union of each 
%$\propmodel_j$ with the addition of a new world $w$ that 
%will satisfy $\prop{\varphi}$. 
%We define $\propdomain$ as $\bigcup_{1\leq j\leq n}\propdomain_j\cup \{w\}$, 
%where $w$ is fresh.
%%The tricky part of the proof is to define the assignment $\propassign$. 
%Before defining $\propneigh_{i}$ and $\propassign$, we define the function $J: {\sf sub}(\prop{\varphi})\rightarrow 2^{\Wmc}$
%with $J(\psi)=\bigcup_{0\leq j \leq m} \Vmc_j(\psi)$ for all $\psi\in {\sf sub}(\prop{\varphi})$, where %$I_i$ is as above for $1\leq i\leq n$, 
% %and
%$\Vmc_0: {\sf sub}(\varphi)\rightarrow  2^{\{w\}}$ is the function
%that assigns $\psi$ to $\{w\}$, if $\nu(\psi)=1$, 
%and to $\emptyset$, otherwise ($\Vmc_j$, for $1\leq j\leq m$, is as above).
%By construction, we have that $J(\neg \psi)=\propdomain\setminus J(\psi)$
%and $J(\psi_1\wedge \psi_2)=J(\psi_1)\cap J(\psi_2)$. 
%We define the assignment $\propassign$ as the function 
%$\propassign: \NPr(\varphi)\rightarrow 2^{\Wmc}$ satisfying 
% $\propassign(p_\elaxiom)=J(p_\elaxiom)$ for all $p_\elaxiom\in \NPr(\varphi)$. 
% 
%It remains to define $\propneigh_i$, for
%$i \in I$.
%%$1 \leq i \leq n$. 
%For $u\in \propdomain_j$ we first put $\alpha \subseteq W$ in $\propneigh_i(u)$ 
%precisely when 
%%\begin{itemize}
%%\item 
%$\propmodel_j,u \models \B_i \psi_\alpha$ and $\alpha = J(\psi_\alpha)$
%for some $\B_i\psi_\alpha \in {\sf sub}(\varphi)$.
%Then, we close $\propneigh_i$ under intersection so that $\propmodel$ is a $\EC^{n}$ model. The next two claims establish that $\propneigh_i$ is as expected.
%%; and
%%\item 
%%\end{itemize}
%%We claim that 
%\begin{claim}
%If $\beta \in \Nmc_i(u)$ and $\beta = J(\psi)$ for some $\B_i\psi\in{\sf sub}(\prop{\varphi})$,
%then $\propmodel_j,u\models\B_i\psi$.
%\end{claim}
%Indeed, since $\beta = J(\psi)\in \Nmc_i(u)$, 
%we must have that $\propmodel_j,u\models \B_i\psi_{1,\beta}$, \ldots, 
%$\propmodel_j,u\models \B_i\psi_{m,\beta}$ and  $\beta = \bigcap^{m}_{l=1} J(\psi_{l,\beta})$ for 
%some $\B_i\psi_{1,\beta}, \ldots, \B_i\psi_{m,\beta} \in{\sf sub}(\prop{\varphi})$. %
%Since $\propneigh_i$ is closed under intersection,
%in fact, we have that $\propmodel_j,u \models \B_i (\bigwedge^m_{l=1}\psi_{l,\beta})$.
%%\nb{intersection}
%But since
%$J(\psi)=\bigcap^{m}_{i=1} J(\psi_{i,\beta})$, we also have $\propassign_j(\psi)=\bigcap^{m}_{l=1}\propassign_j(\psi_{l,\beta})$ 
%(recall that $\propdomain_j \cap \propdomain_k = \emptyset$ for  
%$k \neq j$), so $\propmodel_j,u\models \B_i\psi$ iff 
%$\propmodel_j,u \models \B_i (\bigwedge^m_{l=1}\psi_{l,\beta})$.
%It follows that $\propmodel_j,u\models \B_i\psi$.
%
%\medskip
%
%Regarding the fresh 
%world $w$ introduced above in $\propdomain$,
%we first put $\alpha \subseteq \Wmc$ in $\propneigh_i(w)$   precisely 
%when $\nu(\B_i\psi_\alpha) = 1$ and $\alpha = J(\psi_\alpha)$ for some 
%$\B_i\psi_\alpha \in {\sf sub}(\prop{\varphi})$.
%Then, we again close $\propneigh_i$ under intersection so that $\propmodel$ is a $\EC^{n}$ model.
%%We claim that 
%\begin{claim}
%If 
%$\beta \in \propneigh_i(w)$ and $\beta = J(\bigwedge^{k-1}_{l=1}\psi_l)$ for some $\B_i \psi_1, \ldots, \B_i \psi_{k-1} \in{\sf sub}(\prop{\varphi})$ 
%then  $\valuation(\B_i\psi_l)=1$ for all $1\leq l < k$.
%\end{claim}
%Indeed, since $\beta = J(\bigwedge^{k-1}_{l=1}\psi_l)\in \propneigh_i(w)$ 
%we must have that $\nu(\B_i\psi_{1,\beta})=1, \ldots, \nu(\B_i\psi_{m,\beta})=1$ and 
%$\beta = \bigcap^{m}_{i=1} J(\psi_{i,\beta})$ for some 
%$\B_i\psi_{1,\beta}, \ldots, \B_i\psi_{m,\beta} \in{\sf sub}(\prop{\varphi})$. %\nb{intersection}
%Suppose now that $\nu(\bigwedge^{k-1}_{l=1}\psi_l) = 0$. Then, by assumption, there exists a structure
%$\propmodel_j = (\propdomain_j, \{ \propneigh_{j_i} \}_{i \in I}, \propassign_j)$ and a world $w_j
%\in \propdomain_j$ such that $\propmodel_j,w_j \models (\bigwedge^{k-1}_{l=1}\psi_{l,\beta} \wedge \neg(\bigwedge^{k-1}_{l=1}\psi_l))\vee(\neg(\bigwedge^{k-1}_{l=1}\psi_{l,\beta}) \wedge (\bigwedge^{k-1}_{l=1}\psi_l))$. 
%It follows that $\propassign_j(\bigwedge^{k-1}_{l=1}\psi_{l,\beta})\neq \propassign_j(\bigwedge^{k-1}_{l=1}\psi_l)$. 
%Consequently $J(\bigwedge^{k-1}_{l=1}\psi_{l,\beta})\neq J(\bigwedge^{k-1}_{l=1}\psi_l)$, which is a contradiction.  
%%\nb{O :... talk about consistent}
%
%\medskip
%
%We now show by induction on the structure of formulas 
%that $\propassign$ and $J$ agree on ${\sf sub}(\prop{\varphi})$. 
%This holds by construction for atomic propositions. It is easy to deal 
%with propositional connectives, since we know that $J(\neg \psi)=\propdomain\setminus J(\neg \psi)$
%and  $J(\psi_1\wedge \psi_2)=J(\psi_1)\cap J(\psi_2)$ 
%and similarly for $\propassign$. Assume inductively that $\propassign(\psi) = J(\psi)$.
%Suppose first that $u\in J(\B_i\psi)$. Then, either $u=w$ and $\nu(\B_i\psi)=1$
%or $u\in \propdomain_j$ and $\propmodel_j,u\models\B_i\psi$. In either case 
%we have that $J(\psi)\in \propneigh_i(u)$. Since 
%$\propassign(\psi) = J(\psi)$, it follows that $\propmodel,u\models \B_i\psi$, 
%that is, $u\in \propassign(\B_i\psi)$. Suppose now that $u\in \propassign(\B_i\psi)$, 
%that is, $\propmodel,u\models \B_i\psi$, or, equivalently, $\propassign(\psi)\in \propneigh_i(u)$.
%Since $\propassign(\psi) = J(\psi)$ it follows that either $u=w$ and 
%$\nu(\B_i\psi)=1$ or $u\in \propdomain_j$ and $\propmodel_j,u\models \B_i\psi$. 
%In either case we have that $u\in J(\B_i\psi)$. 
%
%Since $\nu(\prop{\varphi})=1$, we have that $w\in J(\prop{\varphi})$, 
%and consequently $w\in \propassign(\prop{\varphi})$. That 
%is, $\propmodel,w\models\prop{\varphi}$. 
%The fact that $\propmodel$ is a $\EC^{n}$ model follows 
%from the definition of $\propneigh_i$.
%The fact that $\propmodel$ is a $\varphi$-consistent  model follows from 
%the fact that $\nu$, used to construct the assignment 
%related to $w$, is $\varphi$-consistent and 
%the models $\propmodel_1,\ldots,\propmodel_m$, used to define 
%the remaining worlds in $\Wmc$, are all $\varphi$-consistent %$\EC^{n}$ 
%models. 
%%Now the lemma follows from Lemma~\ref{lem:general}.
%%Take $\bigwedge^{k-1}_{j=1}\psi_j$ as $\psi'$
%\end{proof}
%
%
%
%\begin{restatable}{lemma}{LemmapropN}\label{lem:proplemmaN}
%A formula $\prop{\varphi}$ is satisfied in a $\varphi$-consistent $\EN^{n}$ model
%iff
%%if, and only if,
%there is   a $\varphi$-consistent valuation \valuation 
%for $\prop{\varphi}$ such that 
%\begin{enumerate}
%\item if $\B_i\psi$ is in ${\sf sub}(\prop{\varphi})$ and
%$\valuation(\B_i\psi)=0$, then $\neg \psi$ 
%is satisfied in a $\varphi$-consistent $\EN^{n}$ model; %and 
%\item if $\B_i\psi_1$ and $\B_i\psi_2$ are in 
%${\sf sub}(\prop{\varphi})$, $\valuation(\B_i\psi_1)=1$, and 
%$\valuation(\B_i\psi_2)=0$, then $(\psi_1\wedge\neg\psi_2)\vee(\neg\psi_1\wedge\psi_2)$
%is satisfied in a $\varphi$-consistent $\EN^{n}$ model. 
%\end{enumerate}
%\end{restatable}
%%
%\begin{proof}%[Sketch]
%We start with proving ($\Rightarrow$). 
%Suppose that $\prop{\varphi}$ is satisfied in a world $w$ of a $\varphi$-consistent $\EN^{n}$ model 
% $\propmodel = (\propdomain, \{ \propneigh_{i} \}_{i \in I}, \propassign)$. That is, 
% $\propmodel, w\models \prop{\varphi}$. We define a $\varphi$-consistent valuation for 
% $\prop{\varphi}$
% by setting $\nu(\psi)=1$ if $\propmodel, w\models \psi$ and $\nu(\psi) = 0$
% if  $\propmodel, w\not\models \psi$. 
% It is easy to check that $\nu$ is indeed a 
% $\varphi$-consistent valuation (given that $\propmodel$ is a  
% $\varphi$-consistent $\EN^{n}$ model). 
%
%\textit{Point 1.}
%%\paragraph{Point 1}
%By definition,  $\B_i {\sf true}$ holds in $\EN^{n}$ models. %\nb{add this def somewhere}
%Since there is a valuation $\valuation$ such that 
%$\valuation(\B_i\psi)=0$ we have that
%$\psi$ cannot be true in all valuations (otherwise
%$\psi\equiv {\sf true}$ would hold and  $\valuation$ 
%would violate %validity of 
%$\B_i {\sf true}$ in $\EN^{n}$ models).
%This means that 
%%We have that
% $\neg \psi$ is 
%$\EN^{n}$  satisfiable.
%
%\textit{Point 2.}
%%\paragraph{Point 2}
%Assume that 
% $\B_i\psi_1$ and $\B_i\psi_2$ are in ${\sf sub}(\prop{\varphi})$, 
% $\nu(\B_i\psi_1)=1$ and $\nu(\B_i\psi_2)=0$. Then $\propmodel,w\models \B_i\psi_1$
% and $\propmodel,w\not\models \B_i\psi_2$. Thus, by definition, 
% $\propassign(\psi_1)\in \propneigh_i(w)$ and 
%  $\propassign(\psi_2)\not\in \propneigh_i(w)$.
%%Using the same argument of Lemma 3.1 in~\cite{Var2}, it follows that 
%So,
%   $\propassign(\psi_1)\neq \propassign(\psi_2)$.
%Then, %it is easy to see that 
%there 
%   is a world $u$ in the symmetrical difference of these sets 
%   such that $\propmodel,u\models (\psi_1\wedge\neg\psi_2)\vee(\neg\psi_1\wedge\psi_2)$. 
%   
%%\medskip
%
%The proof of the converse ($\Leftarrow$)
%%for  the second bullet point 
%is as follows. %in Lemma~\ref{lem:general}.
%%It holds in $\EC^{n}$ models because $\B_i {\sf true}$
% Suppose there is a $\varphi$-consistent valuation $\nu$ for $\prop{\varphi}$ such that 
% Point~1
%%\textbf{Point 1}
%and
%Point~2
%%\textbf{Point 2}
%hold. 
%That is, 
%\begin{itemize}
%\item if $\B_i\psi$ is in ${\sf sub}(\prop{\varphi})$ and
%$\valuation(\B_i\psi)=0$ then 
%there is a $\varphi$-consistent $\EN^{n}$ model $\propmodel_{\psi}=(\propdomain_{\psi},
%\{ \propneigh_{{\psi}_{i}} \}_{i \in I},\propassign_{\psi})$
%%
% and a world 
%$w_{\psi}\in \propdomain_{\psi}$ such that 
%$\propmodel_{\psi}, w_{\psi}\models \neg\psi$; and 
%%$\neg \psi$ is 
%%$\EN^{n}$ satisfiable; and
%\item if $\B_i\psi_1$ and $\B_i\psi_2$ are in 
%${\sf sub}(\prop{\varphi})$, $\valuation(\B_i\psi_1)=1$, and 
%$\valuation(\B_i\psi_2)=0$, then 
%there is a $\varphi$-consistent $\EN^{n}$ model $\propmodel_{\psi_1,\psi_2}=(\propdomain_{\psi_1,\psi_2},
%\{ \propneigh_{{\psi_1,\psi_2}_{i}} \}_{i \in I},\propassign_{\psi_1,\psi_2})$
%%
% and a world 
%$w_{\psi_1,\psi_2}\in \propdomain_{\psi_1,\psi_2}$ such that 
%$\propmodel_{\psi_1,\psi_2}, w_{\psi_1,\psi_2}\models (\psi_1\wedge\neg\psi_2)\vee(\neg\psi_1\wedge\psi_2)$. 
%\end{itemize}
%Let $\propmodel_1,\ldots,\propmodel_m$ be an enumeration of $\EN^{n}$ models 
%$\propmodel_{\psi}$ and $\propmodel_{\psi_1,\psi_2}$, as above.  That is, we take one model $\propmodel_{\psi}$ and one model $\propmodel_{\psi_1,\psi_2}$ for each such
%subformula $j=\psi$ and pair of subformulas $j=\psi_1,\psi_2$ where $\propmodel_j = (\propdomain_j, \{ \propneigh_{j_{i}} \}_{i \in I},\propassign_j)$, 
%and let $w_1,\ldots,w_m$ be an enumeration of the worlds $w_{\psi}$ and $w_{\psi_1,\psi_2}$, 
%with $w_j\in \propdomain_j$. Assume without loss of generality that $\propdomain_j\cap \propdomain_k=\emptyset$ 
%for $j\neq k$. 
%%
%In the following, we define a $\varphi$-consistent $\EN^{n}$ model   $\propmodel = (\propdomain,\{ \propneigh_{i} \}_{i \in I}, \propassign)$ 
%for $\prop{\varphi}$. 
%
%Intuitively, we construct $\propmodel$ by taking the union of each 
%$\propmodel_j$ with the addition of a new world $w$ that 
%will satisfy $\prop{\varphi}$. 
%We define $\propdomain$ as $\bigcup_{1\leq j\leq n}\propdomain_j\cup \{w\}$, 
%where $w$ is fresh.
%%The tricky part of the proof is to define the assignment $\propassign$. 
%Before defining $\propneigh_{i}$ and $\propassign$, we define the function $J: {\sf sub}(\prop{\varphi})\rightarrow 2^{\Wmc}$
%with $J(\psi)=\bigcup_{0\leq j \leq m} \Vmc_j(\psi)$ for all $\psi\in {\sf sub}(\prop{\varphi})$, where %$I_i$ is as above for $1\leq i\leq n$, 
% %and
%$\Vmc_0: {\sf sub}(\prop{\varphi})\rightarrow  2^{\{w\}}$ is the function
%that assigns $\psi$ to $\{w\}$, if $\nu(\psi)=1$, 
%and to $\emptyset$, otherwise ($\Vmc_j$, for $1\leq j\leq m$, is as above).
%By construction, we have that $J(\neg \psi)=\propdomain\setminus J(\psi)$
%and $J(\psi_1\wedge \psi_2)=J(\psi_1)\cap J(\psi_2)$. 
%We define the assignment $\propassign$ as the function 
%$\propassign: \NPr(\varphi)\rightarrow 2^{\Wmc}$ satisfying 
% $\propassign(p_\elaxiom)=J(p_\elaxiom)$ for all $p_\elaxiom\in \NPr(\varphi)$. 
% 
%It remains to define $\propneigh_i$,
%for
%$i \in I$.
%%$1 \leq i \leq n$. 
%For $u\in \propdomain_j$ we put  $\alpha \subseteq \Wmc$ in $\propneigh_i(u)$ 
%precisely when $\alpha=\Wmc$, or, $\alpha = J(\psi_{\alpha})$ and
%$\propmodel_j,u \models \B_i \psi_{\alpha}$,  
%for some $\B_i\psi_{\alpha} \in {\sf sub}(\prop{\varphi})$.
%The next two claims establish that $\propneigh_i$ is as expected. 
%%We claim that 
%\begin{claim}
%If $\beta \in \Nmc_i(u)$ and $\beta = J(\psi)$ for some $\B_i\psi\in{\sf sub}(\prop{\varphi})$,
%then $\propmodel_j,u\models\B_i\psi$.
%\end{claim}
%\begin{proof}[Proof of Claim]
%Indeed, since $\beta = J(\psi)\in \Nmc_i(u)$, 
%we must have that either $\beta=\Wmc$ or $\propmodel_j,u\models \B_i\psi_{\beta}$ and  $\beta = J(\psi_{\beta})$ for 
%some $\B_i\psi_\beta \in{\sf sub}(\prop{\varphi})$.
%In the former case, as all $\propmodel_j$ models are $\EN^{n}$ models, 
%we have that $\propmodel_j,u\models \B_i\psi$.
%% (in this case, $\psi$
%%is true in all worlds in all $\EN^{n}$ models).
%In the latter,  since
%$J(\psi)=J(\psi_\beta)$, we also have $\propassign_j(\psi)=\propassign_j(\psi_\beta)$ 
%(recall that $\propdomain_j \cap \propdomain_k = \emptyset$ for  
%$k \neq j$), so $\propmodel_j,u\models \B_i\psi$ iff 
%$\propmodel_j,u \models \B_i \psi_\beta$.
%It follows that $\propmodel_j,u\models \B_i\psi$.
%\end{proof}
%
%Also, we put $\alpha \subseteq \Wmc$ in $\propneigh_i(w)$ (recall $w$ is the fresh 
%world introduced above in $\propdomain$) precisely 
%when $\alpha = \Wmc$ or $\nu(\B_i\psi_\alpha) = 1$ and $\alpha = J(\psi_\alpha)$ for some 
%$\B_i\psi_\alpha \in {\sf sub}(\prop{\varphi})$.
%
%\begin{claim}
%If 
%$\beta \in \propneigh_i(w)$ and $\beta = J(\psi)$ for some $\B_i \psi \in{\sf sub}(\prop{\varphi})$ 
%then $\nu(\B_i \psi) = 1$.
%\end{claim}
%\begin{proof}[Proof of Claim]
%Indeed, since $\beta = J(\psi)\in \propneigh_i(w)$ 
%we must have that either $\beta=\Wmc$
%or $\nu(\B_i\psi_\beta)=1$ and $\beta = J(\psi_\beta)$ for some 
%$\B_i\psi_\beta \in{\sf sub}(\prop{\varphi})$. 
%Suppose  that $\nu(\B_i\psi) = 0$ and $\beta\neq\Wmc$.
%Then, by assumption, there exists a $\varphi$-consistent $\EN^{n}$ model
%$\propmodel_j = (\propdomain_j, \{ \propneigh_{j_i} \}_{i \in I}, \propassign_j)$ and a world $w_j
%\in \propdomain_j$ such that $\propmodel_j,w_j \models (\psi_\beta \wedge \neg\psi)\vee(\neg\psi_\beta \wedge \psi)$. 
%It follows that $\propassign_j(\psi_\beta)\neq \propassign_j(\psi)$. 
%Consequently $J(\psi_\beta)\neq J(\psi)$, which is a contradiction.  
%Now, suppose   that $\nu(\B_i\psi) = 0$ and $\beta=\Wmc$.
%By assumption,
%%$\neg \psi$ is  satisfiable in a $\varphi$-consistent $\EN^{n}$ model.
%%This means that 
%there exists a $\varphi$-consistent $\EN^{n}$ model
%$\propmodel_j = (\propdomain_j, \{ \propneigh_{j_i} \}_{i \in I}, \propassign_j)$ and a world $w_j
%\in \propdomain_j$ such that $\propmodel_j,w_j \models \neg\psi$.
%It follows that $\propassign_j(\psi)\neq \Wmc$. Consequently $\beta= J(\psi)\neq \Wmc$, which is a contradiction.  Then,  $\nu(\B_i \psi) = 1$, as required.
%%\nb{changing}
%%, which means that
%%$\beta=\Wmc$.
%%\nb{O :... talk about consistent}
%\end{proof}
%
%We now show by induction on the structure of formulas 
%that $\propassign$ and $J$ agree on ${\sf sub}(\prop{\varphi})$. 
%This holds by construction for atomic propositions. It is easy to deal 
%with propositional connectives, since we know that $J(\neg \psi)=\propdomain\setminus J(\neg \psi)$
%and  $J(\psi_1\wedge \psi_2)=J(\psi_1)\cap J(\psi_2)$ 
%and similarly for $\propassign$. Assume inductively that $\propassign(\psi) = J(\psi)$.
%Suppose first that $u\in J(\B_i\psi)$. Then, either $u=w$ and $\nu(\B_i\psi)=1$
%or $u\in \propdomain_j$ and $\propmodel_j,u\models\B_i\psi$. In either case 
%we have that $J(\psi)\in \propneigh_i(u)$. Since 
%$\propassign(\psi) = J(\psi)$, it follows that $\propmodel,u\models \B_i\psi$, 
%that is, $u\in \propassign(\B_i\psi)$. Suppose now that $u\in \propassign(\B_i\psi)$, 
%that is, $\propmodel,u\models \B_i\psi$, or, equivalently, $\propassign(\psi)\in \propneigh_i(u)$.
%Since $\propassign(\psi) = J(\psi)$ it follows that either $u=w$ and 
%$\nu(\B_i\psi)=1$ or $u\in \propdomain_j$ and $\propmodel_j,u\models \B_i\psi$. 
%In either case we have that $u\in J(\B_i\psi)$. 
%
%Since $\nu(\prop{\varphi})=1$, we have that $w\in J(\prop{\varphi})$, 
%and consequently $w\in \propassign(\prop{\varphi})$. That 
%is, $\propmodel,w\models\prop{\varphi}$. 
%The fact that $\propmodel$ is $\varphi$-consistent follows from 
%the fact that $\nu$, used to construct the assignment 
%related to $w$, is $\varphi$-consistent and 
%the models $\propmodel_1,\ldots,\propmodel_m$, used to define 
%the remaining worlds in $\Wmc$, are all $\varphi$-consistent.
%The fact that  $\propmodel$ contains the unit is by construction, that is,
%we defined $\propmodel$
%%. That is, 
%so that for all 
%$i\in [1,n]$ and all $w\in\Wmc$, we have that $\Wmc\in\Nmc_i(w)$. 
%Thus, $\propmodel$ is a $\varphi$-consistent $\EN^{n}$ model that satisfies 
%$\prop{\varphi}$, as required. 
%\end{proof}
%
%
%
%To determine satisfiability of $\prop{\varphi}$ in a $\varphi$-consistent model, we use Lemma~\ref{lem:prop} and the characterizations above. 
%To establish complexity results, %for the upper bound of $\CALCg$, 
%we use the fact that there are only quadratically many   subformulas in $\prop{\varphi}$. 
%Satisfiability in
%\ALC is \ExpTime-complete  and so, one can determine in exponential time
%whether a valuation is $\varphi$-consistent. For an \ExpTime~upper bound, one can
%deterministically compute all possible $\varphi$-consistent valuations for 
%$(\bigwedge^{k-1}_{j=1}\psi_j\wedge\neg\psi_k)$ (or $(\psi_1 \wedge \neg \psi_2)$) and
%decide satisfiability of $\prop{\varphi}$ by a $\varphi$-consistent model using a bottom-up strategy (as in~\cite{Baader:2012:LOD:2287718.2287721}). Since satisfiability in \ALC is \ExpTime-hard, our upper bound is tight.
%
%\begin{theorem}
%The $\CALCg$ and $\NALCg$ formula satisfiability problems on constant domain neighbourhood models are \ExpTime-complete.
%%Satisfiability in  $\CALCg$  and $\NALCg$ is \ExpTime-complete.
%\end{theorem}
%
%\begin{restatable}{lemma}{LemmapropP}\label{lem:proplemmaP}
%	A formula $\prop{\varphi}$ is satisfied in a $\varphi$-consistent $\EP^{n}$ model
%	iff
%	%if, and only if,
%	there is   a $\varphi$-consistent valuation \valuation 
%	for $\prop{\varphi}$ such that 
%	\begin{enumerate}
%		\item if $\B_i\psi$ is in ${\sf sub}(\prop{\varphi})$ and
%		$\valuation(\B_i\psi)=1$, then $\psi$ 
%		is satisfied in a $\varphi$-consistent $\EP^{n}$ model; %and 
%		\item if $\B_i\psi_1$ and $\B_i\psi_2$ are in 
%		${\sf sub}(\prop{\varphi})$, $\valuation(\B_i\psi_1)=1$, and 
%		$\valuation(\B_i\psi_2)=0$, then $(\psi_1\wedge\neg\psi_2)\vee(\neg\psi_1\wedge\psi_2)$
%		is satisfied in a $\varphi$-consistent $\EP^{n}$ model. 
%	\end{enumerate}
%\end{restatable}
%
%\begin{restatable}{lemma}{LemmapropQ}\label{lem:proplemmaQ}
%	A formula $\prop{\varphi}$ is satisfied in a $\varphi$-consistent $\EQ^{n}$ model
%	iff
%	%if, and only if,
%	there is   a $\varphi$-consistent valuation \valuation 
%	for $\prop{\varphi}$ such that 
%	\begin{enumerate}
%		\item if $\B_i\psi$ is in ${\sf sub}(\prop{\varphi})$ and
%		$\valuation(\B_i\psi)=1$, then $\neg\psi$ 
%		is satisfied in a $\varphi$-consistent $\EQ^{n}$ model; %and 
%		\item if $\B_i\psi_1$ and $\B_i\psi_2$ are in 
%		${\sf sub}(\prop{\varphi})$, $\valuation(\B_i\psi_1)=1$, and 
%		$\valuation(\B_i\psi_2)=0$, then $(\psi_1\wedge\neg\psi_2)\vee(\neg\psi_1\wedge\psi_2)$
%		is satisfied in a $\varphi$-consistent $\EQ^{n}$ model. 
%	\end{enumerate}
%\end{restatable}
%
%\begin{restatable}{lemma}{LemmapropD}\label{lem:proplemmaD}
%	A formula $\prop{\varphi}$ is satisfied in a $\varphi$-consistent $\ED^{n}$ model
%	iff
%	%if, and only if,
%	there is   a $\varphi$-consistent valuation \valuation 
%	for $\prop{\varphi}$ such that 
%	\begin{enumerate}
%		\item if $\B_i\psi_1$ and $\B_i\psi_2$ are in ${\sf sub}(\prop{\varphi})$,
%		$\valuation(\B_i\psi_1)=1$ and $\valuation(\B_i\psi_2)=1$, then
%		$(\psi_1\wedge\psi_2)\vee(\neg\psi_1\wedge\neg\psi_2)$   
%		is satisfied in a $\varphi$-consistent $\ED^{n}$ model; %and 
%		\item if $\B_i\psi_1$ and $\B_i\psi_2$ are in 
%		${\sf sub}(\prop{\varphi})$, $\valuation(\B_i\psi_1)=1$, and 
%		$\valuation(\B_i\psi_2)=0$, then $(\psi_1\wedge\neg\psi_2)\vee(\neg\psi_1\wedge\psi_2)$
%		is satisfied in a $\varphi$-consistent $\ED^{n}$ model. 
%	\end{enumerate}
%\end{restatable}
%
%\begin{restatable}{lemma}{LemmapropT}\label{lem:proplemmaT}
%	A formula $\prop{\varphi}$ is satisfied in a $\varphi$-consistent $\ET^{n}$ model
%	iff
%	%if, and only if,
%	there is   a $\varphi$-consistent valuation \valuation 
%	for $\prop{\varphi}$ such that 
%	\begin{enumerate}
%		\item if $\B_i\psi$ is in ${\sf sub}(\prop{\varphi})$ and 
%		$\valuation(\B_i\psi)=1$   then
%		$\psi$   \todo{to check}
%		is satisfied in a $\varphi$-consistent $\ET^{n}$ model; %and 
%		\item if $\B_i\psi_1$ and $\B_i\psi_2$ are in 
%		${\sf sub}(\prop{\varphi})$, $\valuation(\B_i\psi_1)=1$, and 
%		$\valuation(\B_i\psi_2)=0$, then $(\psi_1\wedge\neg\psi_2)\vee(\neg\psi_1\wedge\psi_2)$
%		is satisfied in a $\varphi$-consistent $\ET^{n}$ model. 
%	\end{enumerate}
%\end{restatable}
%%\begin{restatable}{lemma}{LemmapropCMN}\label{lem:proplemmaCMN}
%%	A formula $\prop{\varphi}$ is satisfied in a $\varphi$-consistent $\EMCN^{n}$ model
%%	iff
%%	%if, and only if,
%%	there is   a $\varphi$-consistent valuation \valuation 
%%	for $\prop{\varphi}$ such that 
%%	%\begin{enumerate}
%%		%\item 
%%		if $\B_i\psi$ is in ${\sf sub}(\prop{\varphi})$ and
%%		$\valuation(\B_i\psi)=0$, then both $\bigwedge_{\valuation(\B_i\psi')=1}  \B_i\psi'\wedge\neg\psi$ and $\bigwedge_{\valuation(\B_i\psi')=1}  \psi'\wedge\neg\psi $
%%		are satisfied in a $\varphi$-consistent $\EMCN^{n}$ model. %; %and 
%%	%	\item if $\B_i\psi_1$ and $\B_i\psi_2$ are in 
%%%		${\sf sub}(\prop{\varphi})$, $\valuation(\B_i\psi_1)=1$, and %
%%%		$\valuation(\B_i\psi_2)=0$, then $(\psi_1\wedge\neg\psi_2)\vee(\neg\psi_1\wedge\psi_2)$
%%%		is satisfied in a $\varphi$-consistent $\EMCN^{n}$ model. 
%%	%\end{enumerate}
%%\end{restatable}
%%\begin{proof} 
%%($\Rightarrow$) Suppose that $\prop{\varphi}$ is satisfied in a world $w$ of a $\varphi$-consistent $\EMCN^{n}$ model 
%%$\propmodel = (\propdomain, \{ \propneigh_{i} \}_{i \in I}, \propassign)$. That is, 
%%$\propmodel, w\models \prop{\varphi}$. We define a $\varphi$-consistent valuation for 
%%$\prop{\varphi}$
%%by setting $\nu(\psi)=1$ if $\propmodel, w\models \psi$ and $\nu(\psi) = 0$
%%if  $\propmodel, w\not\models \psi$. 
%%It is easy to check that $\nu$ is indeed a 
%%$\varphi$-consistent valuation (given that $\propmodel$ is a  
%%$\varphi$-consistent $\EMCN^{n}$ model). \nb{Ana: to check}
%%Assume   $\B_i\psi$ is in ${\sf sub}(\prop{\varphi})$ and
%%$\valuation(\B_i\psi)=0$.
%%%
%%Then %$\propmodel,w\models \B_i\psi_j$ for all $1\leq j < k$,
%%%and 
%%$\propmodel,w\not\models \B_i\psi$.
%%By necessitation in $\EMCN^{n}$ models, $\propmodel,w\not\models \psi$.
%%So $\propmodel,w\models \neg\psi$.
%%By definition, $\propmodel,w\models \B_i\psi'$ when $\valuation(\B_i\psi')=1$.
%%This means that $\bigwedge_{\valuation(\B_i\psi')=1}  \B_i\psi'\wedge\neg\psi$ is  satisfied in a $\varphi$-consistent $\EMCN^{n}$ model.
%%%So $\propmodel,w\models \B_i (\psi_1\wedge \ldots \wedge \psi_{k-1})$ 
%%%and $\propmodel,w\not\models \B_i\psi_k$.
%%%This means that $\valuation(\B_i(\bigwedge^{k-1}_{j=1}\psi_j))=1$
%%%while $\valuation(\B_i\psi_k)=0$.
%%%
%%
%%We now argue about $\bigwedge_{\valuation(\B_i\psi')=1}  \psi'\wedge\neg\psi $. \nb{Ana: on going}
%%By definition, 
%%$\propassign(\bigwedge_{\valuation(\B_i\psi')=1}\psi')\in \propneigh_i(w)$   and 
%%$\propassign(\psi)\not\in \propneigh_i(w)$.
%%%Using the same argument of Lemma 3.1 in~\cite{Var2}, it follows that 
%%So,
%%$\propassign(\bigwedge^{k-1}_{j=1}\psi_j)\neq \propassign(\psi_k)$.
%%Then, %it is easy to see that 
%%there 
%%is a world $u$ in the symmetrical difference of these sets 
%%such that $\propmodel,u\models (\bigwedge_{\valuation(\B_i\psi')=1}\psi'\wedge\neg\psi)\vee(\neg(\bigwedge_{\valuation(\B_i\psi')=1}\psi')\wedge\psi)$.  
%%\end{proof}
%
%
%%\subsection{{\color{red}{Unified ``Vardi's lemma'' for all logics}}}
%
%
%
%
%
%\newpage

 
%%%%%%%%%%%%%%%%%%%%%%%%%%%%%%%%%%%%%%%%%%%%%%%%%%%%%%%%%%%%%%%%%%%%%%

%\section{Applications and Extensions}
%
%\subsection{Somebody Knows}