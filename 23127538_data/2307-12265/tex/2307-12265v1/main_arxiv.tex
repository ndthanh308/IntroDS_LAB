% This is samplepaper.tex, a sample chapter demonstrating the
% LLNCS macro package for Springer Computer Science proceedings;
% Version 2.21 of 2022/01/12
%
\documentclass[runningheads]{llncs}
%envcountsame
%nospthms



%%% PACKAGES

\usepackage[T1]{fontenc}
% T1 fonts will be used to generate the final print and online PDFs,
% so please use T1 fonts in your manuscript whenever possible.
% Other font encondings may result in incorrect characters.



%% Use the postscript times font!
%\usepackage{times}
%\usepackage{soul}


\usepackage{graphicx}
% Used for displaying a sample figure. If possible, figure files should
% be included in EPS format.
%
% If you use the hyperref package, please uncomment the following two lines
% to display URLs in blue roman font according to Springer's eBook style:
%\usepackage{color}
%\renewcommand\UrlFont{\color{blue}\rmfamily}

\usepackage{url}
%\usepackage[hidelinks]{hyperref}
\usepackage[small]{caption}


\let\proof\relax
\let\endproof\relax


\let\claim\relax


\usepackage{amsmath}
\usepackage{amsthm}
\usepackage{booktabs}
\urlstyle{same}
\usepackage{comment}
\usepackage{amsfonts}

\usepackage{multicol}

\usepackage{amsfonts}
\usepackage{amssymb}
\usepackage{latexsym}
\usepackage{stmaryrd}
\usepackage{wasysym}
\usepackage{mathrsfs}
\usepackage{graphicx}
\usepackage{rotating}
\usepackage{pifont}
\usepackage{booktabs}
\usepackage{array}
\newcolumntype{?}{!{\vrule width 1pt}}

\usepackage[numbers]{natbib}
%\usepackage{multibib}
%\bibliographystyle{named}

\usepackage{multirow}
\usepackage{tikz}
\usetikzlibrary{arrows}
\usepackage{color}
\usepackage{xcolor}
\usepackage[inline]{enumitem}
\usepackage{xspace}
\usepackage{thm-restate} 
\usepackage{bussproofs}
\def\ruleScoreFiller{\hrule height 1pt}

\usepackage{tcolorbox}

\usepackage[edges]{forest}
\usepackage{varwidth}
\usepackage{adjustbox}


%\usepackage[marginpar=3cm]{geometry}

\usepackage[textsize=footnotesize]{todonotes}
\setuptodonotes{backgroundcolor=white, textcolor=red}

% Per Tableau Algorithm
\usepackage[lined,linesnumbered,algoruled]{algorithm2e}

%\usepackage{ragged2e}
%\usepackage{setspace}
%\usepackage{fancyhdr}



\makeatletter
\newsavebox{\@brx}
\newcommand{\llangle}[1][]{\savebox{\@brx}{\(\m@th{#1\langle}\)}%
  \mathopen{\copy\@brx\kern-0.5\wd\@brx\usebox{\@brx}}}
\newcommand{\rrangle}[1][]{\savebox{\@brx}{\(\m@th{#1\rangle}\)}%
  \mathclose{\copy\@brx\kern-0.5\wd\@brx\usebox{\@brx}}}
\makeatother

%%% MACROS

%%% SIDE COMMENTS ENVIRONMENT

\newcommand{\nb}[1]{\textcolor{red}{$|$}\mbox{}\marginpar{\scriptsize\raggedright\textcolor{red}{#1}}}



%%% PARAGRAPH ENVIRONMENT

\newcommand{\mypar}[1]{\medskip\noindent\textbf{#1.}}


%%% USEFUL SHORTCUTS / NOTATION

\newcommand{\ext}[1]{\llangle#1\rrangle}

\newcommand{\e}{\emph}
\newcommand{\neighborhood}{neighbourhood\xspace} % neighborhood or neighbourhood?
\newcommand{\p}{\varphi}
\newcommand{\q}{\psi}
\newcommand{\s}{\sigma}
\newcommand{\reldomain}{\ensuremath{W}\xspace}
\newcommand{\relations}{\ensuremath{R}\xspace}
\newcommand{\role}{\ensuremath{r}\xspace}
\newcommand{\eset}{\emptyset}
\newcommand{\mdl}{\models}
\newcommand{\Int}{\ensuremath{\mathcal{I}}\xspace}

\newcommand{\dnot}{\ensuremath{\dot{\lnot}}\xspace}

\newcommand{\sbs}{\subseteq}
\newcommand{\sqs}{\sqsubseteq}

\newcommand{\lp}{^{\langle}}
\newcommand{\rp}{^{\rangle}}

\newcommand{\tr}{^{\dagger}}
%\newcommand{\trx}{^{\dagger_{x}}}
%\newcommand{\try}{^{\dagger_{y}}}
\newcommand{\ttr}{^{\ddagger}}
\newcommand{\startr}{\ensuremath{^{T}}\xspace}
%\newcommand{\startr}{\ensuremath{^{\star}}\xspace}

\newcommand{\assign}{\ensuremath{\mathfrak{a}}\xspace}


\newcommand{\valuation}{\ensuremath{\nu}\xspace}

\newcommand{\cstyle}{\mathsf}



\newcommand{\propmodel}{\ensuremath{\mathcal{M}^{\sf P}}\xspace}
\newcommand{\propdomain}{\ensuremath{\mathcal{W}}\xspace}
\newcommand{\propneigh}{\ensuremath{\mathcal{N}}\xspace}
%\newcommand{\propneigh}{\ensuremath{\{ \mathcal{N}_{i} \}_{i \in [1, n]}}\xspace}
\newcommand{\propassign}{\ensuremath{\mathcal{V}}\xspace}



\newcommand{\formula}{\ensuremath{\varphi}\xspace}


\newcommand{\alcform}{\ensuremath{\hat{\varphi}}_{\Vmc, w}\xspace}


\newcommand{\con}{\ensuremath{\mathsf{con}}\xspace}
\newcommand{\conneg}{\ensuremath{\mathsf{con}_{\dot{\lnot}}}\xspace}
\newcommand{\for}{\ensuremath{\mathsf{for}}\xspace}
\newcommand{\forneg}{\ensuremath{\mathsf{for}_{\dot{\lnot}}}\xspace}
\newcommand{\rol}{\ensuremath{\mathsf{rol}}\xspace}
\newcommand{\ind}{\ensuremath{\mathsf{ind}}\xspace}

\newcommand{\fg}{\ensuremath{\mathsf{Fg}}\xspace}


%\newcommand{   }{\ensuremath{    }\xspace}

\newcommand{\qs}{\ensuremath{\boldsymbol{q}}\xspace}
\newcommand{\runs}{\ensuremath{\mathfrak{R}}\xspace}
\newcommand{\tp}{\ensuremath{\mathsf{tp}}\xspace}


\newcommand{\formtp}[1]{\ensuremath{{\boldsymbol{f}}^{\Vmc, w}_{#1}}\xspace}



%%% LOGICAL OPERATORS

\newcommand{\U}{\ensuremath{\mathbin{\mathcal{U}}}}
\newcommand{\Until}{\ensuremath{\mathbin{\mathcal{U}}}}
\newcommand{\Since}{\ensuremath{\mathbin{\mathcal{S}}}}

%\newcommand{\Next}{\ensuremath{\ocircle}}
\newcommand{\Next}{{\ensuremath{\raisebox{0.25ex}{\text{\scriptsize{$\bigcirc$}}}}}}
\newcommand{\nxt}{{\ensuremath{\raisebox{0.25ex}{\text{\scriptsize{$\bigcirc$}}}}}}
\newcommand{\Of}{\ocircle_{F}}
\newcommand{\Op}{\ocircle_{P}}
\newcommand{\Wnext}{\CIRCLE}

\newcommand{\D}{\Diamond}
\newcommand{\Df}{\D_{F}}
\newcommand{\Dp}{\Diamond_{P}}

\newcommand{\B}{\Box}
\newcommand{\Bf}{\B_{F}}
\newcommand{\Bp}{\Box_{P}}

\newcommand{\Bast}{\boxast}





%% DL PROOFS
\newcommand{\prop}[1]{\ensuremath{#1_{\sf prop} }\xspace} 
\newcommand{\elaxiom}{\ensuremath{\pi}\xspace} 
\newcommand{\consistent}[1]{\ensuremath{#1\text{-consistent}}\xspace} 



%%% LANGUAGES

\newcommand{\DL}{\textit{DL}}
\newcommand{\TDL}{\textit{TDL}}
\newcommand{\LTL}{\textit{LTL}}

\newcommand{\DLite}{\textit{DL-Lite}}
\newcommand{\TDLite}{\textit{TDL-Lite}}

\newcommand{\ALC}{\ensuremath{\smash{\mathcal{ALC}}}\xspace}
\newcommand{\ALCQI}{\ensuremath{\mathcal{ALCQI}}}
\newcommand{\DLR}{\ensuremath{\mathcal{DLR}}}

\newcommand{\EL}{\ensuremath{\mathcal{EL}}\xspace}

%%% SIGNATURE

\newcommand{\NC}{\ensuremath{{\sf N_C}}\xspace}
\newcommand{\NI}{\ensuremath{{\sf N_I}}\xspace}
\newcommand{\NR}{\ensuremath{{\sf N_R}}\xspace}

\newcommand{\NPr}{\ensuremath{{\sf N_P}}\xspace}
\newcommand{\NAt}{\ensuremath{{\sf N_A}}\xspace}

\newcommand{\NV}{\ensuremath{{\sf N_{V}}}\xspace}




%%% MODAL LOGICS

%%%%%% FRAMES AND MODELS

\newcommand{\FrL}{\ensuremath{\smash{\mathsf{Fr}\mathbf{L}}}\xspace}


%%%%%% MODAL LANGUAGES


\newcommand{\ML}{\ensuremath{\smash{\mathcal{ML}}}\xspace}
\newcommand{\QML}{\ensuremath{\smash{\mathcal{QML}}}\xspace}
%\newcommand{\ML}{\ensuremath{\smash{\mathit{ML}}}\xspace}
%\newcommand{\QML}{\ensuremath{\smash{\mathit{QML}}}\xspace}

\newcommand{\MLn}{\ensuremath{\smash{\mathcal{ML}^{n}}}\xspace}
%\newcommand{\MLn}{\ensuremath{\smash{\mathit{ML}^{n}}}\xspace}

\newcommand{\MLALC}[1]{\ensuremath{\smash{\mathcal{ML}^{#1}_{\mathcal{ALC}}}}\xspace}
%\newcommand{\MLALC}[1]{\ensuremath{\smash{\mathit{ML}^{#1}_{\mathcal{ALC}}}}\xspace}

\newcommand{\MLtwoALC}{\ensuremath{\smash{\mathit{ML}^{2}_{\mathcal{ALC}}}}\xspace}
\newcommand{\MLthreeALC}{\ensuremath{\smash{\mathit{ML}^{3}_{\mathcal{ALC}}}}\xspace}





\newcommand{\MLnALC}{\ensuremath{\smash{\mathcal{ML}^{n}_{\mathcal{ALC}}}}\xspace}
%\newcommand{\MLnALC}{\ensuremath{\smash{\mathit{ML}^{n}_{\mathcal{ALC}}}}\xspace}
%\newcommand{\MLnALC}[1]{\ensuremath{\smash{\ML^{#1}_{\mathcal{ALC}}}}\xspace}

\newcommand{\MLnALCg}{\ensuremath{\smash{\mathcal{ALC}\textnormal{-}\mathcal{ML}^{n}}\xspace}}
%\newcommand{\MLnALCg}{\ensuremath{\smash{\mathcal{ALC}\textnormal{-}\mathit{ML}^{n}}\xspace}}
%\newcommand{\MLnALCg}{\ensuremath{\smash{\mathit{ML}^{n\mid{\sf g}}_{\mathcal{ALC}}}}\xspace}
%%\newcommand{\MLgALC}[1]{\ensuremath{\smash{\mathit{ML}^{n\mid{\sf g}}_{\mathcal{ALC}}}}\xspace}


%\newcommand{\MLALC}[1]{\ensuremath{\smash{\mathit{ML}^{#1}_{\mathcal{ALC}}}}\xspace}



%%% MODAL SYSTEMS

\newcommand{\Log}{\ensuremath{\smash{\mathsf{Pantheon}}}\xspace}
%\newcommand{\Log}{\ensuremath{\smash{\mathsf{Cube}}}\xspace}



%% Propositional non-normal modal logics
\newcommand{\thickness}{}

% classical non-normal modal logics
\newcommand{\logicnamestyle}[1]{\ensuremath{\smash{\mathbf{#1}}\xspace}}
\newcommand{\Estar}{\logicnamestyle{E}^*}
\newcommand{\EXstar}{\logicnamestyle{EX}^*}
\newcommand{\E}{\logicnamestyle{E}}
\newcommand{\EM}{\logicnamestyle{EM}}
\newcommand{\EC}{\logicnamestyle{EC}}
\newcommand{\EN}{\logicnamestyle{EN}}
\newcommand{\ET}{\logicnamestyle{ET}}
\newcommand{\ED}{\logicnamestyle{ED}}
\newcommand{\EP}{\logicnamestyle{EP}}
\newcommand{\EQ}{\logicnamestyle{EQ}}

\newcommand{\EMC}{\logicnamestyle{EMC}}
\newcommand{\EMN}{\logicnamestyle{EMN}}
\newcommand{\EMT}{\logicnamestyle{EMT}}
\newcommand{\EMD}{\logicnamestyle{EMD}}
\newcommand{\EMP}{\logicnamestyle{EMP}}
\newcommand{\EMQ}{\logicnamestyle{EMQ}}

\newcommand{\ECN}{\logicnamestyle{ECN}}
\newcommand{\ECT}{\logicnamestyle{ECT}}
\newcommand{\ECD}{\logicnamestyle{ECD}}
\newcommand{\ECP}{\logicnamestyle{ECP}}
\newcommand{\ECQ}{\logicnamestyle{ECQ}}

\newcommand{\ENT}{\logicnamestyle{ENT}}
\newcommand{\END}{\logicnamestyle{END}}
\newcommand{\ENP}{\logicnamestyle{ENP}}
\newcommand{\ENQ}{\logicnamestyle{ENQ}}

\newcommand{\ETD}{\logicnamestyle{ETD}}
\newcommand{\ETP}{\logicnamestyle{ETP}}
\newcommand{\ETQ}{\logicnamestyle{ETQ}}

\newcommand{\EDP}{\logicnamestyle{EDP}}
\newcommand{\EDQ}{\logicnamestyle{EDQ}}

\newcommand{\EPQ}{\logicnamestyle{EPQ}}

\newcommand{\EMCN}{\logicnamestyle{EMCN}}
\newcommand{\EMCT}{\logicnamestyle{EMCT}}
\newcommand{\EMCD}{\logicnamestyle{EMCD}}
\newcommand{\EMCP}{\logicnamestyle{EMCP}}
\newcommand{\EMCQ}{\logicnamestyle{EMCQ}}

\newcommand{\EMNT}{\logicnamestyle{EMNT}}
\newcommand{\EMND}{\logicnamestyle{EMND}}
\newcommand{\EMNP}{\logicnamestyle{EMNP}}
\newcommand{\EMNQ}{\logicnamestyle{EMNQ}}

\newcommand{\EMTD}{\logicnamestyle{EMTD}}
\newcommand{\EMTP}{\logicnamestyle{EMTP}}
\newcommand{\EMTQ}{\logicnamestyle{EMTQ}}

\newcommand{\EMDP}{\logicnamestyle{EMDP}}
\newcommand{\EMDQ}{\logicnamestyle{EMDQ}}

\newcommand{\ECNT}{\logicnamestyle{ECNT}}
\newcommand{\ECND}{\logicnamestyle{ECND}}
\newcommand{\ECNP}{\logicnamestyle{ECNP}}
\newcommand{\ECNQ}{\logicnamestyle{ECNQ}}

\newcommand{\ECTD}{\logicnamestyle{ECTD}}
\newcommand{\ECTP}{\logicnamestyle{ECTP}}
\newcommand{\ECTQ}{\logicnamestyle{ECTQ}}

\newcommand{\ECDQ}{\logicnamestyle{ECDQ}}
\newcommand{\ECPQ}{\logicnamestyle{ECPQ}}

\newcommand{\ENTD}{\logicnamestyle{ENTD}}
\newcommand{\ENTP}{\logicnamestyle{ENTP}}
\newcommand{\ENTQ}{\logicnamestyle{ENTQ}}

\newcommand{\ETDP}{\logicnamestyle{ETDP}}
\newcommand{\ETDQ}{\logicnamestyle{ETDQ}}
\newcommand{\ETPQ}{\logicnamestyle{ETDQ}}

\newcommand{\EDPQ}{\logicnamestyle{EDPQ}}

\newcommand{\EMCNT}{\logicnamestyle{EMCNT}}
\newcommand{\EMCND}{\logicnamestyle{EMCND}}
\newcommand{\EMCNP}{\logicnamestyle{EMCNP}}
\newcommand{\EMCNQ}{\logicnamestyle{EMCNQ}}

\newcommand{\EMCTD}{\logicnamestyle{EMCTD}}
\newcommand{\EMCTP}{\logicnamestyle{EMCTP}}
\newcommand{\EMCTQ}{\logicnamestyle{EMCTQ}}

\newcommand{\EMCDP}{\logicnamestyle{EMCDP}}
\newcommand{\EMCDQ}{\logicnamestyle{EMCDQ}}

\newcommand{\EMNTD}{\logicnamestyle{EMNTD}}
\newcommand{\EMNDP}{\logicnamestyle{EMNDP}}
\newcommand{\EMNTP}{\logicnamestyle{EMNTP}}
\newcommand{\EMNTQ}{\logicnamestyle{EMNTQ}}

\newcommand{\EMTDP}{\logicnamestyle{EMTDP}}

\newcommand{\ECTDP}{\logicnamestyle{ECTDP}}
\newcommand{\ECTDQ}{\logicnamestyle{ECTDQ}}
\newcommand{\ECTPQ}{\logicnamestyle{ECTPQ}}

\newcommand{\ECDPQ}{\logicnamestyle{ECDPQ}}

\newcommand{\ECNTD}{\logicnamestyle{ECNTD}}
\newcommand{\ECNTP}{\logicnamestyle{ECNTP}}
\newcommand{\ECNTQ}{\logicnamestyle{ECNTQ}}
\newcommand{\ECNDP}{\logicnamestyle{ECNDP}}
\newcommand{\ECNDQ}{\logicnamestyle{ECNDQ}}

\newcommand{\ENTDP}{\logicnamestyle{EMTDP}}

\newcommand{\ETDPQ}{\logicnamestyle{ETDPQ}}

\newcommand{\EMCNTD}{\logicnamestyle{EMCNTD}}
\newcommand{\EMCNTP}{\logicnamestyle{EMCNTP}}
\newcommand{\EMCNTQ}{\logicnamestyle{EMCNTQ}}

\newcommand{\EMCNDP}{\logicnamestyle{EMCNDP}}
\newcommand{\EMCNDQ}{\logicnamestyle{EMCNDQ}}

\newcommand{\EMCTDP}{\logicnamestyle{EMCTDP}}

\newcommand{\EMNTDP}{\logicnamestyle{EMNTDP}}

\newcommand{\ECNTDP}{\logicnamestyle{ECNTDP}}
\newcommand{\ECNTDQ}{\logicnamestyle{ECNTDQ}}

\newcommand{\ECTDPQ}{\logicnamestyle{ECTDPQ}}

\newcommand{\EMCNTDP}{\logicnamestyle{EMCTDP}}
\newcommand{\EMCNTDQ}{\logicnamestyle{EMCTDQ}}

\newcommand{\EMCNTDPQ}{\logicnamestyle{EMCTDPQ}}


\newcommand{\ETfour}{\logicnamestyle{ET4}}
\newcommand{\ETfive}{\logicnamestyle{ET5}}
\newcommand{\EfourXstar}{\logicnamestyle{E4X}^*}

\newcommand{\EDoneplus}{\logicnamestyle{ED}_1^+}
\newcommand{\EDtwoplus}{\logicnamestyle{ED}_2^+}
\newcommand{\ENDtwoplus}{\logicnamestyle{END}_2^+}
\newcommand{\ECDtwoplus}{\logicnamestyle{ECD}_2^+}
\newcommand{\ECNDtwoplus}{\logicnamestyle{ECND}_2^+}
\newcommand{\EMDtwoplus}{\logicnamestyle{EMD}_2^+}
\newcommand{\EMNDtwoplus}{\logicnamestyle{EMND}_2^+}
\newcommand{\EMCDtwoplus}{\logicnamestyle{EMCD}_2^+}
\newcommand{\EMCNDtwoplus}{\logicnamestyle{EMCND}_2^+}
\newcommand{\EDnplus}{\logicnamestyle{ED}_n^+}
\newcommand{\ENDnplus}{\logicnamestyle{END}_n^+}
\newcommand{\ECDnplus}{\logicnamestyle{ECD}_n^+}
\newcommand{\ECNDnplus}{\logicnamestyle{ECND}_n^+}
\newcommand{\EMDnplus}{\logicnamestyle{EMD}_n^+}
\newcommand{\EMNDnplus}{\logicnamestyle{EMND}_n^+}
\newcommand{\EMCDnplus}{\logicnamestyle{EMCD}_n^+}
\newcommand{\EMCNDnplus}{\logicnamestyle{EMCND}_n^+}
\newcommand{\EDthreeplus}{\logicnamestyle{ED}_3^+}

\newcommand{\EMstar}{\logicnamestyle{EM}^*}
\newcommand{\ECstar}{\logicnamestyle{EC}^*}
\newcommand{\ENstar}{\logicnamestyle{EN}^*}
\newcommand{\ETstar}{\logicnamestyle{ET}^*}
\newcommand{\EDstar}{\logicnamestyle{ED}^*}
\newcommand{\EPstar}{\logicnamestyle{EP}^*}
\newcommand{\Efourstar}{\logicnamestyle{E4}^*}
\newcommand{\EMfourstar}{\logicnamestyle{M4}^*}
%
\newcommand{\EMCstar}{\logicnamestyle{EMC}^*}
\newcommand{\EMNstar}{\logicnamestyle{EMN}^*}
\newcommand{\EDnplusstar}{\hEstar{D_n^+}}
\newcommand{\EMTstar}{\logicnamestyle{EMT}^*}
\newcommand{\EMDstar}{\logicnamestyle{EMD}^*}
\newcommand{\EMPstar}{\logicnamestyle{EMP}^*}
\newcommand{\EMDnplusstar}{\hMstar{D_n^+}}

\newcommand{\K}{\logicnamestyle{K}}
\newcommand{\KT}{\logicnamestyle{KT}}
\newcommand{\KD}{\logicnamestyle{KD}}
\newcommand{\KP}{\logicnamestyle{KP}}
\newcommand{\KDnplus}{\logicnamestyle{KD}_n^+}
\newcommand{\Kfour}{\logicnamestyle{K4}}

\newcommand{\Sone}{\logicnamestyle{S1}}
\newcommand{\Stwo}{\logicnamestyle{S2}}
\newcommand{\Sthree}{\logicnamestyle{S3}}
\newcommand{\Sfour}{\logicnamestyle{S4}}
\newcommand{\Sfive}{\logicnamestyle{S5}}
\newcommand{\Eone}{\logicnamestyle{E1}}
\newcommand{\Etwo}{\logicnamestyle{E2}}
\newcommand{\Ethree}{\logicnamestyle{E3}}

\newcommand{\CPL}{\logicnamestyle{CPL}}
\newcommand{\IPL}{\logicnamestyle{IPL}}

\newcommand{\Efour}{\logicnamestyle{E4}}
\newcommand{\Efive}{\logicnamestyle{E5}}
\newcommand{\EMfive}{\logicnamestyle{EM5}}

\newcommand{\M}{\logicnamestyle{EM}}
%\newcommand{\M}{\ensuremath{\mathcal{M}}\xspace}
\newcommand{\Mstar}{\logicnamestyle{EM}^*}
\newcommand{\MCstar}{\logicnamestyle{EMC}^*}
\newcommand{\MCNstar}{\logicnamestyle{EMCN}^*}
\newcommand{\MCNXstar}{\logicnamestyle{EMCNX}^*}



\newcommand{\C}{\logicnamestyle{EC}}
%\newcommand{\C}{\ensuremath{\smash{\mathbf{C}}}\xspace}
\newcommand{\N}{\logicnamestyle{EN}}

\newcommand{\Nn}{\ensuremath{\smash{\mathbf{EN}^{n}}}\xspace}




\newcommand{\EX}{\logicnamestyle{EX}}
\newcommand{\MX}{\logicnamestyle{MX}}
%\newcommand{\X}{\logicnamestyle{X}}
\newcommand{\X}{\logic}
\newcommand{\Lstar}{\logic^*}



%\newcommand{\hE}[1]{\logicnamestyle{E{#1}}}
%\newcommand{\hM}[1]{\logicnamestyle{M{#1}}}
\newcommand{\hE}[1]{\mathbf{E{#1}}}
\newcommand{\hM}[1]{\mathbf{M{#1}}}
\newcommand{\hEstar}[1]{\mathbf{E{#1}^*}}
\newcommand{\hMstar}[1]{\mathbf{M{#1}^*}}

%\newcommand{\E}{\ensuremath{\smash{\mathbf{E}}}\xspace}
%\newcommand{\EN}{\ensuremath{\smash{\mathbf{EN}}}\xspace}
%\newcommand{\EC}{\ensuremath{\smash{\mathbf{EC}}}\xspace}
%\newcommand{\ECN}{\ensuremath{\smash{\mathbf{ECN}}}\xspace}
%\newcommand{\EM}{\ensuremath{\smash{\mathbf{EM}}}\xspace}
%\newcommand{\EMN}{\ensuremath{\smash{\mathbf{EMN}}}\xspace}
%\newcommand{\EMC}{\ensuremath{\smash{\mathbf{EMC}}}\xspace}
%\newcommand{\EMCN}{\ensuremath{\smash{\mathbf{EMCN}}}\xspace}
%\newcommand{\K}{\ensuremath{\smash{\mathbf{K}}}\xspace}
%\newcommand{\M}{\ensuremath{\smash{\mathbf{M}}}\xspace}
%\newcommand{\proplogics}{\E(\axM,\axC,\axN)}

%% Axioms
\newcommand{\axM}{\textsf{M}}
\newcommand{\axC}{\textsf{C}}
\newcommand{\axN}{\textsf{N}}
\newcommand{\axK}{\textsf{K}}
\newcommand{\RE}{\textsf{RE}}
\newcommand{\RM}{\textsf{RM}}
\newcommand{\RN}{\textsf{RN}}


%%% Modal DLs

\newcommand{\LnALC}{\ensuremath{\mathit{L}^{n}_{\ALC}}\xspace}
%\newcommand{\LnALC}{\ensuremath{\mathbf{L}^{n}_{\ALC}}\xspace}

\newcommand{\EEL}[1]{\ensuremath{\smash{\mathbf{E}^{#1}_{\mathcal{EL}}}}\xspace}
\newcommand{\EALC}{\ensuremath{\smash{\mathbf{E}_{\mathcal{ALC}}}}\xspace}
\newcommand{\EnALC}[1]{\ensuremath{\smash{\mathbf{E}^{#1}_{\mathcal{ALC}}}}\xspace}



\newcommand{\MALC}{\ensuremath{\smash{\mathbf{EM}_{\mathcal{ALC}}}}\xspace}
\newcommand{\MnALC}[1]{\ensuremath{\smash{\mathbf{EM}^{#1}_{\mathcal{ALC}}}}\xspace}
%\newcommand{\MALC}{\ensuremath{\smash{\mathbf{M}_{\mathcal{ALC}}}}\xspace}
%\newcommand{\MnALC}[1]{\ensuremath{\smash{\mathbf{M}^{#1}_{\mathcal{ALC}}}}\xspace}

\newcommand{\KALC}{\ensuremath{\smash{\mathbf{K}_{\mathcal{ALC}}}}\xspace}

\newcommand{\KtwoALC}{\ensuremath{\smash{\mathbf{K}^{2}_{\mathcal{ALC}}}}\xspace}
\newcommand{\KthreeALC}{\ensuremath{\smash{\mathbf{K}^{3}_{\mathcal{ALC}}}}\xspace}
%\newcommand{\KnALC}{\ensuremath{\smash{\mathbf{K}^{n}_{\mathcal{ALC}}}}\xspace}
\newcommand{\KnALC}[1]{\ensuremath{\smash{\mathbf{K}^{#1}_{\mathcal{ALC}}}}\xspace}

\newcommand{\SfiveALCQI}{\ensuremath{\textbf{S5}_{\mathcal{ALCQI}}}}


\newcommand{\EALCg}{\ensuremath{\smash{{\mathcal{ALC}}\textnormal{-}{\mathbf{E}^{n}}}\xspace}}
\newcommand{\CALCg}{\ensuremath{\smash{{\mathcal{ALC}}\textnormal{-}{\mathbf{EC}^{n}}}\xspace}}
\newcommand{\NALCg}{\ensuremath{\smash{{\mathcal{ALC}}\textnormal{-}{\mathbf{EN}^{n}}}\xspace}}
\newcommand{\MALCg}{\ensuremath{\smash{{\mathcal{ALC}}\textnormal{-}{\mathbf{EM}^{n}}}\xspace}}
\newcommand{\LnALCg}{\ensuremath{\smash{{\mathcal{ALC}}\textnormal{-}{\mathit{L}^{n}}}\xspace}}
%\newcommand{\EALCg}{\ensuremath{\smash{{\mathcal{ALC}}\textnormal{-}{\mathbf{E}^{n}}}\xspace}}
%\newcommand{\CALCg}{\ensuremath{\smash{{\mathcal{ALC}}\textnormal{-}{\mathbf{C}^{n}}}\xspace}}
%\newcommand{\NALCg}{\ensuremath{\smash{{\mathcal{ALC}}\textnormal{-}{\mathbf{N}^{n}}}\xspace}}
%\newcommand{\MALCg}{\ensuremath{\smash{{\mathcal{ALC}}\textnormal{-}{\mathbf{M}^{n}}}\xspace}}
%\newcommand{\LnALCg}{\ensuremath{\smash{{\mathcal{ALC}}\textnormal{-}{\mathbf{L}^{n}}}\xspace}}
%%\newcommand{\EALCg}{\ensuremath{\smash{\mathbf{E}^{n\mid{\sf g}}_{\mathcal{ALC}}}}\xspace}
%%\newcommand{\CALCg}{\ensuremath{\smash{\mathbf{C}^{n\mid{\sf g}}_{\mathcal{ALC}}}}\xspace}
%%\newcommand{\NALCg}{\ensuremath{\smash{\mathbf{N}^{n\mid{\sf g}}_{\mathcal{ALC}}}}\xspace}
%%\newcommand{\MALCg}{\ensuremath{\smash{\mathbf{M}^{n\mid{\sf g}}_{\mathcal{ALC}}}}\xspace}
%%\newcommand{\LnALCg}{\ensuremath{\mathbf{L}^{n\mid{\sf g}}_{\ALC}}\xspace}

%%% TEMPORAL LOGICS

\newcommand{\QTL}{\mathcal{QTL}}
\newcommand{\QTLi}{{\ensuremath{\QTL^1}}}

\newcommand{\TALC}{\ensuremath{\smash{\textsl{T}_{\U}{\mathcal{ALC}}}}\xspace}



%%% COMPLEXITY CLASSES

\newcommand{\ACz}{{\ensuremath{\textsc{AC}^0}}}
\newcommand{\LogSpace}{\textsc{LogSpace}}
\newcommand{\NLogSpace}{\textsc{NLogSpace}}
\newcommand{\PTime}{\textsc{PTime}}
\newcommand{\NP}{\textsc{NP}}
\newcommand{\PSpace}{\textsc{PSpace}}
\newcommand{\ExpTime}{\textsc{ExpTime}}
\newcommand{\NExpTime}{\textsc{NExpTime}}
\newcommand{\ExpSpace}{\textsc{ExpSpace}}
\newcommand{\expsp}{\textsc{ExpSpace}}



%%% FRAKTUR LETTERS

\newcommand{\Amf}{\ensuremath{\mathfrak{A}}\xspace}
\newcommand{\Bmf}{\ensuremath{\mathfrak{B}}\xspace}
\newcommand{\Cmf}{\ensuremath{\mathfrak{C}}\xspace}
\newcommand{\Dmf}{\ensuremath{\mathfrak{D}}\xspace}
\newcommand{\Emf}{\ensuremath{\mathfrak{E}}\xspace}
%\newcommand{\Fmf}{\ensuremath{\mathfrak{F}}\xspace}
\newcommand{\Fmf}{\ensuremath{F}\xspace}
\newcommand{\Gmf}{\ensuremath{\mathfrak{G}}\xspace}
\newcommand{\Hmf}{\ensuremath{\mathfrak{H}}\xspace}
\newcommand{\Imf}{\ensuremath{\mathfrak{I}}\xspace}
\newcommand{\Jmf}{\ensuremath{\mathfrak{J}}\xspace}
\newcommand{\Kmf}{\ensuremath{\mathfrak{K}}\xspace}
\newcommand{\Lmf}{\ensuremath{\mathfrak{L}}\xspace}
%\newcommand{\Mmf}{\ensuremath{\mathfrak{M}}\xspace}
\newcommand{\Mmf}{\ensuremath{M}\xspace}
\newcommand{\Nmf}{\ensuremath{\mathfrak{N}}\xspace}
\newcommand{\Omf}{\ensuremath{\mathfrak{O}}\xspace}
\newcommand{\Pmf}{\ensuremath{\mathfrak{P}}\xspace}
\newcommand{\Qmf}{\ensuremath{\mathfrak{Q}}\xspace}
\newcommand{\Rmf}{\ensuremath{\mathfrak{R}}\xspace}
\newcommand{\Smf}{\ensuremath{\mathfrak{S}}\xspace}
\newcommand{\Tmf}{\ensuremath{\mathfrak{T}}\xspace}
\newcommand{\Umf}{\ensuremath{\mathfrak{U}}\xspace}
\newcommand{\Vmf}{\ensuremath{\mathfrak{V}}\xspace}
\newcommand{\Wmf}{\ensuremath{\mathfrak{W}}\xspace}
\newcommand{\Xmf}{\ensuremath{\mathfrak{X}}\xspace}
\newcommand{\Ymf}{\ensuremath{\mathfrak{Y}}\xspace}
\newcommand{\Zmf}{\ensuremath{\mathfrak{Z}}\xspace}



%%% MATHCAL LETTERS

\newcommand{\Amc}{\ensuremath{\mathcal{A}}\xspace}
\newcommand{\Bmc}{\ensuremath{\mathcal{B}}\xspace}
\newcommand{\Cmc}{\ensuremath{\mathcal{C}}\xspace}
\newcommand{\Dmc}{\ensuremath{\mathcal{D}}\xspace}
\newcommand{\Emc}{\ensuremath{\mathcal{E}}\xspace}
\newcommand{\Fmc}{\ensuremath{\mathcal{F}}\xspace}
\newcommand{\Gmc}{\ensuremath{\mathcal{G}}\xspace}
\newcommand{\Hmc}{\ensuremath{\mathcal{H}}\xspace}
\newcommand{\Imc}{\ensuremath{\mathcal{I}}\xspace}
\newcommand{\Jmc}{\ensuremath{\mathcal{J}}\xspace}
\newcommand{\Kmc}{\ensuremath{\mathcal{K}}\xspace}
\newcommand{\Lmc}{\ensuremath{\mathcal{L}}\xspace}
\newcommand{\Mmc}{\ensuremath{\mathcal{M}}\xspace}
\newcommand{\Nmc}{\ensuremath{\mathcal{N}}\xspace}
\newcommand{\Omc}{\ensuremath{\mathcal{O}}\xspace}

\newcommand{\Pmc}{\ensuremath{\mathcal{P}}\xspace}

\newcommand{\Qmc}{\ensuremath{\mathcal{Q}}\xspace}
\newcommand{\Rmc}{\ensuremath{\mathcal{R}}\xspace}
\newcommand{\Smc}{\ensuremath{\mathcal{S}}\xspace}
\newcommand{\Tmc}{\ensuremath{\mathcal{T}}\xspace}
\newcommand{\Umc}{\ensuremath{\mathcal{U}}\xspace}
\newcommand{\Vmc}{\ensuremath{\mathcal{V}}\xspace}
\newcommand{\Wmc}{\ensuremath{\mathcal{W}}\xspace}
\newcommand{\Xmc}{\ensuremath{\mathcal{X}}\xspace}
\newcommand{\Ymc}{\ensuremath{\mathcal{Y}}\xspace}
\newcommand{\Zmc}{\ensuremath{\mathcal{Z}}\xspace}



%%% BLACKBOARD LETTERS 

\newcommand{\Abl}{\ensuremath{\mathbb{A}}\xspace}
\newcommand{\Bbl}{\ensuremath{\mathbb{B}}\xspace}
\newcommand{\Cbl}{\ensuremath{\mathbb{C}}\xspace}
\newcommand{\Dbl}{\ensuremath{\mathbb{D}}\xspace}
\newcommand{\Ebl}{\ensuremath{\mathbb{E}}\xspace}
\newcommand{\Fbl}{\ensuremath{\mathbb{F}}\xspace}
\newcommand{\Gbl}{\ensuremath{\mathbb{G}}\xspace}
\newcommand{\Hbl}{\ensuremath{\mathbb{H}}\xspace}
\newcommand{\Ibl}{\ensuremath{\mathbb{I}}\xspace}
\newcommand{\Jbl}{\ensuremath{\mathbb{J}}\xspace}
\newcommand{\Kbl}{\ensuremath{\mathbb{K}}\xspace}
\newcommand{\Lbl}{\ensuremath{\mathbb{L}}\xspace}
\newcommand{\Mbl}{\ensuremath{\mathbb{M}}\xspace}
\newcommand{\Nbl}{\ensuremath{\mathbb{N}}\xspace}
\newcommand{\Obl}{\ensuremath{\mathbb{O}}\xspace}
\newcommand{\Pbl}{\ensuremath{\mathbb{P}}\xspace}
\newcommand{\Qbl}{\ensuremath{\mathbb{Q}}\xspace}
\newcommand{\Rbl}{\ensuremath{\mathbb{R}}\xspace}
\newcommand{\Sbl}{\ensuremath{\mathbb{S}}\xspace}
\newcommand{\Tbl}{\ensuremath{\mathbb{T}}\xspace}
\newcommand{\Ubl}{\ensuremath{\mathbb{U}}\xspace}
\newcommand{\Vbl}{\ensuremath{\mathbb{V}}\xspace}
\newcommand{\Wbl}{\ensuremath{\mathbb{W}}\xspace}
\newcommand{\Xbl}{\ensuremath{\mathbb{X}}\xspace}
\newcommand{\Ybl}{\ensuremath{\mathbb{Y}}\xspace}
\newcommand{\Zbl}{\ensuremath{\mathbb{Z}}\xspace}



%%% BOLDFACE LETTERS

\newcommand{\Abf}{\ensuremath{\mathbf{A}}\xspace}
\newcommand{\Bbf}{\ensuremath{\mathbf{B}}\xspace}
\newcommand{\Cbf}{\ensuremath{\mathbf{C}}\xspace}
\newcommand{\Dbf}{\ensuremath{\mathbf{D}}\xspace}
\newcommand{\Ebf}{\ensuremath{\mathbf{E}}\xspace}
\newcommand{\Fbf}{\ensuremath{\mathbf{F}}\xspace}
\newcommand{\Gbf}{\ensuremath{\mathbf{G}}\xspace}
\newcommand{\Hbf}{\ensuremath{\mathbf{H}}\xspace}
\newcommand{\Ibf}{\ensuremath{\mathbf{I}}\xspace}
\newcommand{\Jbf}{\ensuremath{\mathbf{J}}\xspace}
\newcommand{\Kbf}{\ensuremath{\mathbf{K}}\xspace}
\newcommand{\Lbf}{\ensuremath{\mathbf{L}}\xspace}
\newcommand{\Mbf}{\ensuremath{\mathbf{M}}\xspace}
\newcommand{\Nbf}{\ensuremath{\mathbf{N}}\xspace}
\newcommand{\Obf}{\ensuremath{\mathbf{O}}\xspace}
\newcommand{\Pbf}{\ensuremath{\mathbf{P}}\xspace}
\newcommand{\Qbf}{\ensuremath{\mathbf{Q}}\xspace}
\newcommand{\Rbf}{\ensuremath{\mathbf{R}}\xspace}
\newcommand{\Sbf}{\ensuremath{\mathbf{S}}\xspace}
\newcommand{\Tbf}{\ensuremath{\mathbf{T}}\xspace}
\newcommand{\Ubf}{\ensuremath{\mathbf{U}}\xspace}
\newcommand{\Vbf}{\ensuremath{\mathbf{V}}\xspace}
\newcommand{\Wbf}{\ensuremath{\mathbf{W}}\xspace}
\newcommand{\Xbf}{\ensuremath{\mathbf{X}}\xspace}
\newcommand{\Ybf}{\ensuremath{\mathbf{Y}}\xspace}
\newcommand{\Zbf}{\ensuremath{\mathbf{Z}}\xspace}


%%% SANS SERIF LETTERS

\newcommand{\Asf}{\ensuremath{\textsf{A}}\xspace}
\newcommand{\Bsf}{\ensuremath{\textsf{B}}\xspace}
\newcommand{\Csf}{\ensuremath{\textsf{C}}\xspace}
\newcommand{\Dsf}{\ensuremath{\textsf{D}}\xspace}
\newcommand{\Esf}{\ensuremath{\textsf{E}}\xspace}
\newcommand{\Fsf}{\ensuremath{\textsf{F}}\xspace}
\newcommand{\Gsf}{\ensuremath{\textsf{G}}\xspace}
\newcommand{\Hsf}{\ensuremath{\textsf{H}}\xspace}
\newcommand{\Isf}{\ensuremath{\textsf{I}}\xspace}
\newcommand{\Jsf}{\ensuremath{\textsf{J}}\xspace}
\newcommand{\Ksf}{\ensuremath{\textsf{K}}\xspace}
\newcommand{\Lsf}{\ensuremath{\textsf{L}}\xspace}
\newcommand{\Msf}{\ensuremath{\textsf{M}}\xspace}
\newcommand{\Nsf}{\ensuremath{\textsf{N}}\xspace}
\newcommand{\Osf}{\ensuremath{\textsf{O}}\xspace}
\newcommand{\Psf}{\ensuremath{\textsf{P}}\xspace}
\newcommand{\Qsf}{\ensuremath{\textsf{Q}}\xspace}
\newcommand{\Rsf}{\ensuremath{\textsf{R}}\xspace}
\newcommand{\Ssf}{\ensuremath{\textsf{S}}\xspace}
\newcommand{\Tsf}{\ensuremath{\textsf{T}}\xspace}
\newcommand{\Usf}{\ensuremath{\textsf{U}}\xspace}
\newcommand{\Vsf}{\ensuremath{\textsf{V}}\xspace}
\newcommand{\Wsf}{\ensuremath{\textsf{W}}\xspace}
\newcommand{\Xsf}{\ensuremath{\textsf{X}}\xspace}
\newcommand{\Ysf}{\ensuremath{\textsf{Y}}\xspace}
\newcommand{\Zsf}{\ensuremath{\textsf{Z}}\xspace}



% By TIZ

\newcommand{\V}{\ensuremath{\mathcal{V}}\xspace}
%\newcommand{\N}{\ensuremath{\mathcal{N}_{i}}\xspace}
\newcommand{\W}{\ensuremath{\mathcal{W}}\xspace}
\newcommand{\ltrset}{[}
\newcommand{\rtrset}{]}
\newcommand{\Nmodel}{N-model\xspace}
\newcommand{\Nframe}{N-frame\xspace}
\newcommand{\Rframe}{R-frame\xspace}
\newcommand{\Rmodel}{R-model\xspace}

\newcommand{\Lvar}{\mathit{L}}
%\newcommand{\falseprop}{{\color{red}{\mathsf{false}}}}
\newcommand{\falseprop}{\mathsf{ff}}
%\newcommand{\trueprop}{{\color{red}{\mathsf{true}}}}
\newcommand{\trueprop}{p \lor \lnot p}
\newcommand{\Uset}{\mathcal{U}}


%% macros for bussproofs
\newcommand{\ax}{\AxiomC}
\newcommand{\llab}{\LeftLabel}
\newcommand{\rlab}{\RightLabel}
\newcommand{\uinf}{\UnaryInfC}
\newcommand{\disp}{\DisplayProof}

%%% USELESS

%\newcommand{\A}{\ensuremath{\mathcal{A}}}
%\newcommand{\K}{\ensuremath{\mathcal{K}}}
%\newcommand{\I}{\ensuremath{\mathcal{I}}}
\newcommand{\T}{\mathcal{T}}
%
%\newcommand{\QT}{Q_\T}
%\newcommand{\QA}{Q_\A}
%
%\newcommand{\TuDLbn}{\ensuremath{\smash{\textsl{T}_{\mathcal{US}}\DLbn}}}%
%\newcommand{\TuDLcn}{\ensuremath{\smash{\textsl{T}_{\mathcal{US}}\DLcn}}}%
%
%\newcommand{\TdDLan}{\ensuremath{\smash{\textsl{T}_{F\!P}\DLan}}}%
%\newcommand{\TdDLbn}{\ensuremath{\smash{\textsl{T}_{F\!P}\DLbn}}}%
%\newcommand{\TdDLkn}{\ensuremath{\smash{\textsl{T}_{F\!P}\DLkn}}}%
%\newcommand{\TdDLcn}{\ensuremath{\smash{\textsl{T}_{F\!P}\DLcn}}}%
%
%\newcommand{\TdDLaHN}{\ensuremath{\smash{\textsl{T}_{\U\!\S}\rDLaHN}}}%
%\newcommand{\TdDLbHN}{\ensuremath{\smash{\textsl{T}_{\U\!\S}\rDLbHN}}}%
%\newcommand{\TdDLkHN}{\ensuremath{\smash{\textsl{T}_{\U\!\S}\rDLkHN}}}%
%\newcommand{\TdDLcHN}{\ensuremath{\smash{\textsl{T}_{\U\!\S}\rDLcHN}}}%
%
%\newcommand{\TdxDLan}{\ensuremath{\smash{\textsl{T}_{F\!P\!X}\DLan}}}
%\newcommand{\TdxDLbn}{\ensuremath{\smash{\textsl{T}_{F\!P\!X}\DLbn}}}%
%\newcommand{\TdxDLkn}{\ensuremath{\smash{\textsl{T}_{F\!P\!X}\DLkn}}}%
%\newcommand{\TdxDLcn}{\ensuremath{\smash{\textsl{T}_{F\!P\!X}\DLcn}}}%
%
%\newcommand{\zSVDLan}{\ensuremath{\smash{\textsl{T}_{U}\DLan}}}%
%\newcommand{\zSVDLb}{\ensuremath{\smash{\textsl{T}_{U}\DLbn}}}%
%\newcommand{\zSVDLk}{\ensuremath{\smash{\textsl{T}_{U}\DLkn}}}%
%\newcommand{\zSVDLc}{\ensuremath{\smash{\textsl{T}_{U}\DLcn}}}%
%
%\newcommand{\TxDLbn}{\ensuremath{\smash{\textsl{T}^{\,\ast}_{X}\DLbn}}}%
%\newcommand{\TdrDLbn}{\ensuremath{\smash{\textsl{T}^{\,\ast}_{F\!P}\DLbn}}}%
%\newcommand{\TurDLbn}{\ensuremath{\smash{\textsl{T}^{\,\ast}_{U}\DLbn}}}%
%
%\newcommand{\TdxrDLcn}{\ensuremath{\smash{\textsl{T}^{\,\ast}_{F\!P\!X}\DLcn}}}%
%\newcommand{\TdxrDLkn}{\ensuremath{\smash{\textsl{T}^{\,\ast}_{F\!P\!X}\DLkn}}}%





%%% ENVIRONMENTS

%% See https://www.overleaf.com/learn/latex/theorems_and_proofs
%% for a nice explanation of how to define new theorems, but keep
%% in mind that the amsthm package is already included in this
%% template and that you must *not* alter the styling.
%\newtheorem{example}{Example}
%\newtheorem{theorem}{Theorem}
%
%%\newtheorem{theorem}{Theorem}
%\newtheorem{definition}{Definition}
%%\newtheorem{example}{Example}
%
%%\theoremstyle{definition}
%%\newtheorem{definition}{Definition}
%
%%\theoremstyle{theorem}
%\newtheorem{lemma}[theorem]{Lemma}
%
%%\theoremstyle{theorem}
%%\newtheorem{theorem}[definition]{Theorem}
%
%%\theoremstyle{theorem}
%\newtheorem{corollary}[theorem]{Corollary}
%
%%\theoremstyle{remark}
%\newtheorem{claim}{Claim}[theorem]
%\newcounter{claimcount} 
%\setcounter{claimcount}{0}
%\spnewtheorem{claim}[claimcount]{Claim}
%\spnewcounter{claimcount} 
%\setcounter{claimcount}{0}
\spnewtheorem{claim}{Claim}{\itshape}{\rmfamily}
\usepackage{etoolbox}
\AtEndEnvironment{document}{\setcounter{claim}{0}}
%%
%
%\newtheorem{proposition}[theorem]{Proposition}

%\spnewtheorem{joke}{Joke}[subsection]{\bfseries}{\rmfamily}

%%% DOCUMENT

\begin{document}


%%% TITLE & AUTHORS


\title{Non-Normal Modal Description Logics \\ (Extended Version)}

%\titlerunning{Abbreviated paper title}
% If the paper title is too long for the running head, you can set
% an abbreviated paper title here



\author{%
Tiziano Dalmonte$^1$
%\orcidID{0000-1111-2222-3333}
\and
Andrea Mazzullo$^2$
%\orcidID{0000-1111-2222-3333}
\and
Ana Ozaki$^3$
%\orcidID{0000-1111-2222-3333}
\and
Nicolas Troquard$^1$
%\orcidID{0000-1111-2222-3333}
}

\authorrunning{T. Dalmonte et al.}
% First names are abbreviated in the running head.
% If there are more than two authors, 'et al.' is used.

\institute{
Free University of Bozen-Bolzano
\email{
\{name.surname\}@unibz.it}
\and
University of Trento
\email{
andrea.mazzullo@unitn.it}
\and
University of Oslo \& University of Bergen
\email{
anaoz@ifi.uio.no}
}


\maketitle

\begin{abstract}

Modal logics are widely used in multi-agent systems to reason about actions, abilities, norms, or epistemic states.
Combined with description logic languages, they
%their combinations with description logics are also
%are also a powerful tool to represent
%%structured
%knowledge over a domain of objects in such modal contexts.
are also a powerful tool to formalise modal aspects of ontology-based reasoning over an object domain.
However, the standard relational semantics for modalities is known to validate principles
%that can be
deemed problematic in agency, deontic, or epistemic applications.
To overcome these difficulties, weaker systems of so-called \emph{non-normal} modal logics, equipped with \emph{neighbourhood semantics} that generalise the relational one, have been investigated both at the propositional and at the description logic level.
We present here
%a systematic study of
a family of
%39
\emph{non-normal modal description logics}, obtained by extending $\ALC$-based languages with non-normal modal operators.
For formulas interpreted on neighbourhood models over varying domains, we provide a modular framework of terminating, correct, and complete tableau-based satisfiability checking algorithms in $\NExpTime$.
For a subset of these systems, we also consider a reduction to satisfiability on constant domain relational models.
Moreover, 
%with respect to neighbourhood models over either constant or varying domains, 
we investigate the satisfiability problem in fragments obtained by disallowing the application of modal operators to description logic concepts,
providing tight $\ExpTime$ complexity results.





%%% ARQNL22
%Non-normal modal logics, interpreted on neighbourhood models which generalise the usual relational semantics, have found application in several areas, such as epistemic, deontic, and coalitional reasoning. We present here preliminary results on reasoning in a family of modal description logics obtained by combining $\ALC$ with non-normal modal operators. First, we provide a framework of terminating, correct, and complete tableau algorithms to check satisfiability of formulas in such logics with the semantics based on varying domains. We then investigate the satisfiability problems in fragments of these languages obtained by restricting the application of modal operators to formulas only, and interpreted on models with constant domains, providing tight complexity results.


%%% DL19
%Non-normal modal logics based on neighbourhood semantics can be used to formalise normative,
%%praxeological,
%epistemic and coalitional reasoning in autonomous and multi-agent systems, since they do not validate
%principles
%%that are
%%known
%%to be problematic
%%to lead to problematic consequences
%known to be problematic
%in
%applications.
%These principles, satisfied by all modal logics interpreted over relational frames, also affect several modal description logics (MDLs)
%used in knowledge representation.
%%In this paper
%We study %satisfiability in
% \emph{non-normal MDLs},
%% \nb{O: modelling is the largest section, removed 'satisfiability in' M: Thanks}
%obtained by extending $\ALC$-based languages with non-normal modal operators.
%These logics increase the expressive power of their propositional counterparts, and allow for complex modelling of obligations, beliefs, abilities and strategies.
%On the computational side, standard reasoning tasks are not more difficult than in basic normal MDLs, with a $\NExpTime$ upper bound for satisfiability
%that can be lowered further
%%\nb{M: mention $\ExpTime$?}
%in fragments with modal operators only over axioms.


\end{abstract}

\section{Introduction}\label{sec:intro}
Federated learning is a popular approach for training machine learning models on decentralized data, where data privacy concerns or other constraints prevent centralized data aggregation~\citep{mcmahan2017communication,kairouz2019federated}. In federated learning, model updates are computed locally on each device (the \emph{client}) and then aggregated to train a global model at the center (the \emph{server}). This approach has gained traction due to its ability to leverage data from multiple sources while preserving privacy, security, and autonomy, and has the potential to make machine learning more participatory in a range of interesting problem domains~\citep{paml2020,jones2020nonrivalry,pentland2021building}.


\vspace{0.05in}
Federated learning is naturally most attractive when the participating clients have access to different data, leading to data heterogeneity~\citep{du2022flamby}. 
This heterogeneity can lead to significant fairness issues, where the performance of the global model can be biased towards the data distribution of some clients over others~\citep{dwork2012fairness,li2019fair,abay2020mitigating}. Heterogeneity can also hurt the generalization of the global model~\citep{quinonero2008dataset,mohri2019agnostic}. Specifically, if some clients have a disproportionate influence on the global model, the resulting model is neither fair nor will it generalize well to new clients. Such disparities are especially prevalent and detrimental in medical research, and have resulted in misdiagnosis and suboptimal treatment~\citep{graham2015disparities,albain2009racial,nana2021health}.

\vspace{0.05in}
To address these challenges, distributionally robust objectives (DRO) explicitly account for the heterogeneity across clients and seek to optimize performance under the worst-case data distribution across clients, rather than just the average performance~\citep{rahimian2019distributionally}. This approach can lead to more robust models that are less biased towards specific clients and more likely to generalize to new clients~\citep{mohri2019agnostic,duchi2023distributionally}. However, such robust objectives are significantly harder to optimize. Current algorithms have very slow convergence, potentially to the point of being impractical~\citep{ro2021communication}. This leads to the central question of our work:
\begin{center}
\parbox{0.9\textwidth}{\emph{Can we design federated optimization techniques for the DRO problem with convergence rates that match their average objective counterparts?}}
\vspace{-0.1in}
\end{center}



% Figure environment removed

\subsection{Our Contributions}
We summarize our contributions below.
\vspace{-0.1in}
\paragraph{Framework.} We present a general formulation for the cross-silo federated DRO problem:
\begin{equation}\label{eq:dro-def}
    \min_{\bx} \max_{\blambda \in \Lambda} \left\{ F(\bx, \blambda) := \sum_{i=1}^N\lambda_i\cdot f_{i}(\bx) - \psi(\blambda)\right\},
\end{equation}
where $f_i(\bx)$ is the loss suffered by client $i$. Instead of minimizing a simple average of the client losses, equation \eqref{eq:dro-def} incorporates weights using $\blambda \in \R^N$. The choice of $\blambda$ is made in a \emph{worst-case} manner, while being subject to the constraint set $\Lambda$ and regularized with $\psi(\blambda)$. As we will show, this formulation is a generalization of several specific fair objectives that have been proposed in the federated learning literature~\citep{mohri2019agnostic,li2019fair,li2020tilted,zhang2022proportional,pillutla2021federated}.

\vspace{-0.1in}
\paragraph{Algorithm.} The objective defined in equation \eqref{eq:dro-def} is a min-max problem and can be directly optimized using well-established algorithms such as gradient descent ascent (GDA). However, such approaches ignore the unique structure of our formulation, particularly the linearity of the interaction term between $\blambda$ and $\bx$. We leverage this to design an accelerated primal-dual (APD) algorithm~\citep{hamedani2021primal}. Additionally, we propose to use control variates (\`a~la {\sc Scaffold}) to correct the bias caused by local steps, making optimal use of local client computation \citep{karimireddy2020scaffold}. Our proposed method, {\algname}, combines these ideas to provide an efficient and practical algorithm, compatible with secure aggregation.

\vspace{-0.1in}
\paragraph{Convergence.} We provide strong convergence guarantees for {\algname} when ${f_i}$ are strongly convex. If $\psi(\blambda)$ is a generic convex function, we achieve an {accelerated} $O\left( {1}/{T^2} \right)$ rate of convergence. Furthermore, if $\psi$ is strongly convex, {\algname} converges linearly at a rate of $\exp\left(-O(T)\right)$. This represents the first federated approach for the DRO problem that achieves \emph{linear} convergence, let alone an \emph{accelerated} rate.
Finally, we extend our analysis to the stochastic setting, where we obtain an optimal rate of $O\left( {1}/{T} \right)$, and improve over the previous $O(1/\sqrt{T})$ rate. Thus, we show that the sample complexity as well as the communication complexity for the DRO problem matches that of the easier average objective.


\paragraph{Practical Evaluation.} We conducted comprehensive simulations and demonstrate accelerated convergence, robustness to data heterogeneity, and the ability to leverage local computations.

For deep learning models, we avail ourselves of a two-stage Train-Convexify-Train method~\citep{yu2022tct}. First, we train a deep learning model using conventional federated learning methods, such as FedAvg. Then, we apply {\algname} to fine tune a convex approximation. To evaluate our algorithms, we use several real-world datasets with various distributionally robust objectives, and we study the trade-off between the mean and tail accuracy of these methods.













%%%%%%%%%%%%%%%%%%%%%%%%%%%%%%%%%%%%%%%%%%%%%%%%%%%%%%%%%%%%%%%%%%%%%%
\section{Preliminaries}
\label{sec:prelim}

Here we introduce
%both non-normal and normal
modal description logics, first presenting their syntax, and then their semantics based on neighbourhood and relational models, respectively.
Finally, we introduce the family of frame conditions here considered. 

\subsection{Syntax}
%\paragraph{Syntax}

Let \NC, \NR and \NI be countably infinite and pairwise disjoint 
sets of \emph{concept}, \emph{role}, and \emph{individual names}, respectively.
%
An $\MLALC{n}$ \emph{concept} is an expression of the form
$
C ::= A \mid \lnot C \mid C \sqcap C \mid \exists \role.C \mid \B_{i} C,
$
where $A \in \NC$, $\role \in \NR$, and $\B_{i}$ such that
$i \in J = \{ 1, \ldots, n \}$.
%
%are
%modal operators.
%called \emph{boxes}.
%
An \emph{\MLALC{n} atom} is
%either
a \emph{concept inclusion} (\emph{CI}) of
the form $(C \sqsubseteq D)$, or an \emph{assertion} of the form $C(a)$ or
$\role(a,b)$, with $C, D$ \MLALC{n} concepts,
%$A \in \NC$,
$\role \in \NR$, and $a, b \in \NI$.
%
%
An \emph{\MLALC{n} formula}
has the form
%is an expression of the form
$
\varphi::= \pi \mid \neg \varphi\mid \varphi\land \varphi\mid \B_{i} \p,
$
where
$\pi$ is an \MLALC{n} atom
and
$i \in J$.
We use the following standard definitions for concepts:
%O: it is a bit strange to define abbrev here and then later in the paper say we use them as prim 
$\forall \role.C :=  \lnot \exists \role.\lnot C$;
$(C \sqcup D) :=  \lnot(\lnot C \sqcap \lnot D)$;
%$(C \Rightarrow D) := (\lnot C \sqcup D)$;
%$(C \Leftrightarrow D) := (C \Rightarrow D) \sqcap (D \Rightarrow C)$;
$\bot := A \sqcap \lnot A$,
$\top := A \sqcup \lnot A$
(for an arbitrarily fixed $A \in \NC$);
%$\top :=  \lnot \bot$;
%}}
$\D_{i} C := \lnot \B_{i} \lnot C$.
%($\D_{i}$ are called \emph{diamonds}).
%\nb{M: $\D$ primitivo?}
Concepts of the form $\B_{i} C$, $\D_{i} C$ are \emph{modalised concepts}.
Analogous conventions
%also
hold for formulas,
writing
$C \equiv D$ for $(C \sqsubseteq D) \land (D \sqsubseteq C)$
and setting
$\mathsf{false} := (\top \sqsubseteq \bot)$, $\mathsf{true} := (\bot \sqsubseteq \top)$.
%



%%%%%%%%%%%%%%%%%%%%%%%%%%%%%%%
\subsection{Semantics}\label{sec:sem}

%\nb{M: todo add brief intro}
%
%%%%%%%%%%%%%%%%%%%%%%%%%%%%%
%We
%now define 
%the neighbourhood and relational semantics.
We now define neighbourhood semantics, which (as already mentioned) can be seen as a generalisation of the relational semantics, introduced immediately after.

\subsubsection{Neighbourhood Semantics}
%\subsection{Neighbourhood Semantics for Propositional Modal Logic}
%\nb{M: Changed! neighbourhood frame $\to$ multimodal neighbourhood frame}
%\paragraph{Neighbourhood Semantics}
 A \emph{neighbourhood frame}, or simply \emph{frame},
%shortened to \emph{\Rframe} when $n$ is clear from the context)
is a pair
%\nb{M: Aggiunto $\Fmc$}
$\Fmc = ( \Wmc, \{\Nmc_i \}_{i \in J})$,
where 
$\Wmc$ is a non-empty set
of \emph{worlds}
and
%$\Nmc$ is a function associating to
$\Nmc_{i} \colon \W \rightarrow 2^{2^{\Wmc}}$ is   a \emph{neighbourhood function}, for each
\emph{agent}
%$1 \leq i \leq n$, $\Nmc_{i}$
$i \in J = \{1, \ldots, n\}$. %\nb{O: rephrased}
%$\N$ is a function $\W \longrightarrow 2^{2^{\Wmc}}$, called \emph{neighbourhood function}.
%
%
%\label{sec:semantics}
An \emph{\MLALC{n} varying domain neighbourhood model}, or simply \emph{model}, based on a neighbourhood frame $\Fmc$ is a pair
$\Mmc = (\Fmc, \Int)$,
where
$\Fmc = (\Wmc,  \{\Nmc_i \}_{i \in J})$ is a neighbourhood frame
%neighbourhood frame,
%$\Delta$ is a non-empty set called \emph{domain of $\Mmc$},
and $\Imc$ is a function associating with every $w \in \Wmc$ an \emph{$\ALC$ interpretation}
%(or \emph{model})
$\Imc_{w} = (\Delta_{w}, \cdot^{\Imc_{w}})$,
with non-empty \emph{domain} $\Delta_{w}$,
%so that $\Delta = \bigcup_{w \in \Wmc} \Delta_{w}$,
and where $\cdot^{\Imc_{w}}$ is a function such that:
for all $A \in \NC$, $A^{\Imc_{w}} \subseteq \Delta_{w}$;
for all $\role \in \NR$, $\role^{\Imc_{w}} \subseteq \Delta_{w} {\times} \Delta_{w}$;
for all $a \in \NI$, $a^{\Imc_{w}} \in \Delta_{w}$.
%%and for all $u, v \in \reldomain$, $a^{\Imc_{u}}= a^{\Imc_{v}}$(denoted by $a^{\Imc}$).
An \MLALC{n} \emph{constant domain neighbourhood model}
is defined in the same way, except that, for all $w,w'\in\Wmc$,
we have that $\Delta_{w}=\Delta_{w'}$
and, for all $u, v \in \Wmc$,
%\nb{O: \Wmc? M: thanks}
we require $a^{\Imc_{u}}= a^{\Imc_{v}}$ (denoted by $a^{\Imc}$), that is, individual names are \emph{rigid designators}.
%\todo{M: added}
We often write $\Mmc = (\Fmc, \Delta, \Imc)$ to denote a constant domain neighbourhood model $\Mmc = (\Fmc, \Imc)$ with domain $\Delta = \Delta_{w}$, for every $w \in \Wmc$.
%\nb{O: changed; M: thanks}
%
%A model $\M = \langle \Fmc, \Delta_{w}, \Imc \rangle$ is \emph{supplemented, closed under intersection, or contains the unit}
%\nb{M: Used? Check}
%if $\Fmc$ is supplemented, closed under intersection, or contains the unit, respectively.
%
%The latter condition means that the individual names are treated as \emph{rigid designators}.
%Moreover, since the domain $\Delta_{w}$ is shared by all worlds $w \in \Wmc$, we say that we accept the \emph{constant domain assumption}.
%
Given a model $\Mmc = (\Fmc, \Int)$ and a world $w \in \Wmc$ of $\Fmc$ (or simply \emph{$w$ in $\Fmc$}), the \emph{interpretation $C^{\Imc_{w}}$ of a concept $C$ in $w$}
%written $C^{\Imc_{w}}$,
is defined
%by taking:
as:
$
		(\neg D)^{\Imc_{w}} = \Delta_{w} \setminus D^{\Imc_{w}}, \quad (D \sqcap E)^{\Imc_{w}}  = D^{\Imc_{w}} \cap E^{\Imc_{w}},
		$
		$
		(\exists r.D)^{\Imc_{w}} = \{d \in \Delta_{w} \mid \exists  e \in D^{\Imc_{w}}{:} (d,e) \in r^{\Imc_{w}}\},
		$
		$
		 (\B_{i} D)^{\Imc_{w}} = \{ d \in \Delta_{w} \mid \llbracket D \rrbracket^{\Mmc}_{d} \in \Nmc_{i}(w) \},
		 $
%	\begin{align*}
%		(\neg D)^{\Imc_{w}} & = \Delta_{w} \setminus D^{\Imc_{w}}, \quad (D \sqcap E)^{\Imc_{w}}  = D^{\Imc_{w}} \cap E^{\Imc_{w}},\\ 
%%		(D \sqcap E)^{\Imc_{w}} & = D^{\Imc_{w}} \cap E^{\Imc_{w}}, \\
%		(\exists r.D)^{\Imc_{w}} & = \{d \in \Delta_{w} \mid \exists  e \in D^{\Imc_{w}}{:} (d,e) \in r^{\Imc_{w}}\},\\
%		 (\B_{i} D)^{\Imc_{w}} & = \{ d \in \Delta_{w} \mid \llbracket D \rrbracket^{\Mmc}_{d} \in \Nmc_{i}(w) \},
%	\end{align*}
where, for all %\nb{M: changed, to check (it was $d \in \Delta_{w}$ before)} 
$d \in \bigcup_{w \in \Wmc} \Delta_{w}$, the set
$\llbracket D \rrbracket^{\Mmc}_{d} = \{ v \in \Wmc \mid  d \in D^{\Imc_{v}} \}$
is called the \emph{truth set of $D$ with respect to \Mmc and $d$}. 
%\nb{O: wrt to \Mmc and $d$?}
%$\llbracket C \rrbracket^{\Mmc}_{d} = \{ v \in \Wmc \mid  d \in C^{\Imc_{v}} \}$
%is called the \emph{truth set of $C$ with respect to $d$}.
% (note that for constant domains ).
We say that a concept $C$ is \emph{satisfied in $\Mmc$} if there is $w$ in $\Fmc$ such that $C^{\Imc_{w}} \neq \eset$, and that $C$ is \emph{satisfiable} (over varying or constant neighbourhood models, respectively) if there is a (varying or constant domain, respectively) neighbourhood model in which it is satisfied.
%
The \emph{satisfaction of an $\MLALC{n}$ formula~$\p$ in $w$ of $\Mmc$}, written $\Mmc, w  \models \p$, is defined
%analogously to relational semantics, and
as follows:
%\begin{gather*}
%	\Mmc, w \models C\sqsubseteq D \quad \text{iff} \quad C^{\Imc_{w}} \subseteq D^{\Imc_{w}},\\
%%	\qquad
%	\Mmc, w  \models \neg \psi \quad \text{iff} \quad \Mmc, w  \not \models \psi, \\
%	\Mmc, w  \models \psi \land \chi \quad \text{iff} \quad \Mmc, w  \models \psi \text{ and } \Mmc, w  \models \chi,\\
%%	\qquad
%	\Mmc, w  \models \B_{i} \psi \quad \text{iff} \quad [ \psi ]^{\Mmc} \in \Nmc_{i}(w),
% \end{gather*} 
\begin{alignat*}{6}
	& \Mmc, w \models C\sqsubseteq D && \text{ iff } && C^{\Imc_{w}} \subseteq D^{\Imc_{w}}, 
	&& \ \
	\Mmc, w  \models C(a) && \text{ iff } && a^{\Imc_{w}} \in C^{\Imc_{w}}, \\
	& \Mmc, w \models \role(a,b) && \text{ iff } && (a^{\Imc_{w}},b^{\Imc_{w}}) \in \role^{\Imc_{w}},
	&& \ \
	\Mmc, w  \models \neg \psi && \text{ iff } && \Mmc, w  \not \models \psi, \\
	& \Mmc, w  \models \psi \land \chi && \text{ iff } && \Mmc, w  \models \psi \text{ and } \Mmc, w  \models \chi, 
	&& \ \
	\Mmc, w  \models \B_{i} \psi && \text{ iff } && \llbracket \psi \rrbracket^{\Mmc} \in \Nmc_{i}(w),
 \end{alignat*} 
%\\
%%
%\begin{tabular}{ccc}
%	$\Mmc, w \models C\sqsubseteq D$ & \text{iff} & $C^{\Imc_{w}} \subseteq D^{\Imc_{w}},$ \\
%	$\Mmc, w  \models \neg \psi$ & \text{iff} & $\Mmc, w  \not \models \psi,$ \\
%	$\Mmc, w  \models \psi \land \chi$ & \text{iff} & $\Mmc, w  \models \psi \text{ and } \Mmc, w  \models \chi,$ \\
%	$\Mmc, w  \models \B_{i} \psi$ & \text{iff} & $[ \psi ]^{\Mmc} \in \Nmc_{i}(w),$
% \end{tabular} 
% \\
%%\begin{align*}
%%%	{\color{red}{\Mmf, w  \models A(a) && \text{ \ iff \ } && a^{I} \in A^{\Imc_{w}}, \qquad
%%%	\Mmf, w \models \role(a,b) && \text{ \ iff \ } && (a^{I},b^{I}) \in \role^{\Imc_{w}},}} \\
%%	\Mmc, w  \models \neg \varphi&& \text{ \ iff \ } && \Mmc, w  \not \models \p, \qquad
%%	\Mmc, w  \models \varphi\land \psi && \text{ \ iff \ } && \Mmc, w  \models \varphi\text{ and } \Mmc, w  \models \psi, \\
%%%	\Mmf, w  \models \B_{i} \varphi&& \text{ \ iff \ } && \text{for all $v \in \reldomain$: if $w \relations_{i} v$, then } \Mmf, v  \models \p.
%%	\Mmc, w \models C\sqsubseteq D  && \text{ \ iff \ } && C^{\Imc_{w}} \subseteq D^{\Imc_{w}}, \qquad
%%	\Mmc, w  \models \B_{i} \varphi \text{ \quad iff \quad } [ \varphi]^{\Mmc} \in \Nmc_{i}(w), 
%% \end{align*} 
%
where
$\llbracket \psi \rrbracket^{\Mmc} = \{ v \in \Wmc \mid \Mmc, v \models \psi \}$ is the \emph{truth set of $\psi$}.
%i.e. the set the worlds $v$ that satisfy $\psi$.
%%$[ \psi ]^{\Mmc}$ denotes the set $\{ v \in \Wmc \mid \Mmc, v \models \psi \}$ of the worlds $v$ that satisfy $\psi$, called the \emph{truth set of $\psi$}.
As a consequence of the above definition, we obtain the following
condition for $\Diamond_{i}$ formulas:
$\Mmc, w \models \Diamond_{i} \psi$  iff  $\llbracket \neg \psi \rrbracket^{\Mmc} \notin \Nmc_{i}(w)$.
Given a
neighbourhood
frame $\Fmc = (\Wmc,  \{\Nmc_i \}_{i \in J})$
and a
neighbourhood
model $\Mmc = (\Fmc, \Imc)$,
we say that $\varphi$ is \emph{satisfied in $\Mmc$} if there is $w \in \Wmc$ such that
$\Mmc, w \models \varphi$,
and that $\p$ is \emph{satisfiable} (over varying or constant domain neighbourhood models, respectively) if it is satisfied in some (varying or constant domain, respectively) neighbourhood model.
%%% VALIDITY, LOGICAL IMPLICATION (not used)
Also, $\p$ is    \emph{valid in $\Mmc$}, $\Mmc \models \p$, if it is satisfied in all $w$ 
of $\Mmc$, and it is \emph{valid on $\Fmc$} if, for all $\Mmc$ based on $\Fmc$,
%and all $w \in W$, $\Mmf, w \models \p$,
$\p$ is valid in $\Mmc$,
writing $\Fmc \models \p$.
%%
%Moreover,
%%\nb{M: Used in ex only. Remove (with ex)?}
%$\p$ \emph{logically implies} a formula $\psi$,
%writing $\varphi\mdl \psi$,
%if $\Mmc, w \models \p$ implies $\Mmc, w \models \psi$,  for every $\Mmc$ and every $w$ in $\Mmc$.
%%
%%Recall
%%%\nb{M: Used? Check}
%%that the concept satisfiability problem can be reduced to the formula satisfiability 
%%problem, since $C$ is satisfiable iff $\lnot (C \sqs \bot)$ is satisfiable.
    

%Other semantical definitions can be easily adapted from the relational semantics case.

 

%%%%%%%%%%%%%%%%%%%%%%%%%%%%%%%
%\subsection{Frames and Satisfiability Problems}

%In the following, we use $\mathfrak F$ to stand either for an N- or R-frame, and $\mathfrak M$ for a N- or R-model.
%
%To define the $\MLALC{n}$ formula satisfiability problems studied in this paper, we consider the principles listed in
%following
%Table~\ref{tab:principles}
%(where $C,D$ and $\p, \psi$ are $\MLALC{n}$ concepts and formulas, respectively).
%In the table,
%Here,
%$S$ is either a
%(N- or R-)
%frame $\mathfrak F$, or a model $\mathfrak M$.
%For a principle $P$, if $S = \mathfrak{F}$ (respectively, $S = \mathfrak{M}$), we say that %\emph{$P$ holds on $\mathfrak{F}$} (respectively, \emph{in $\mathfrak{M}$}).
%
%\begin{table}
%\begin{center}
%\begin{tabular}{l l l}
%\hline
%(\emph{Congruence}) && $S \models C \equiv D$ implies $S \models \Box_{i} C \equiv \Box_{i} D$. \\
%&& $S \models \varphi\leftrightarrow \psi$ implies $S \models \Box_{i} \varphi\leftrightarrow \Box_{i} \psi$. \\
%\hline
%(\emph{Monotonicity}) && $S \models C \sqsubseteq D$ implies $S \models \Box_{i} C \sqsubseteq \Box_{i} D$. \\
%&& $S \models \varphi\to \psi$ implies $S \models \Box_{i} \varphi\to \Box_{i} \psi$. \\
%\hline
%(\emph{Agglomeration}) &\quad\quad& $S \models \Box_{i} C \sqcap \Box_{i} D \sqsubseteq \Box_{i} (C \sqcap D)$. \\
%&& $S \models \Box_{i} \varphi\land \Box_{i} \psi \to \Box_{i} (\phi \land \psi)$. \\
%\hline
%\textcolor{red}{(\emph{Agglomeration'})} & $C\sqcap D = E$ is valid implies $\Box_{i} C \sqcap \Box_{i} D \sqsubseteq \Box_{i} E$ is valid. \\
%& $\phi \land \psi \leftrightarrow \xi$ is valid implies $\Box_{i} \phi \land \Box_{i} \psi \to \Box_{i} \xi$ is valid. \\
%\hline
%(\emph{Necessitation}) && $S \models \top \sqsubseteq C$ implies $S \models \top \sqsubseteq \Box_{i} C$. \\
%&& $S \models \p$ implies $S \models \Box_{i} \p$. \\
%\hline
%\end{tabular}
%\end{center}
%\caption{Principles over frames and models.}
%\label{tab:principles}
%\end{table}
%\nb{T: Red text for necessitation. Is it ok? Or how should I write it? \\ M: I think it's ok}
%
%On the correspondence between the principles in
%Table~\ref{tab:principles} and conditions over frames and models,
%we have the following result (see e.g.~\cite{Pac}
%\nb{T: Changed. In Chellas there are no correspondence theorems. Ref to Pacuit. \\ M: Thanks}
%for the propositional case).

%\begin{restatable}{theorem}{PropValid}\label{prop:dlvalid}
%Given a neighbourhood frame $\Fmc$, we have that:
%
%	$(i)$ congruence holds on $\Fmc$;
%	$(ii)$ monotonicity holds on $\Fmc$ iff $\Fmc$ is supplemented;
%	$(iii)$ agglomeration holds on $\Fmc$ iff $\Fmc$ is closed under intersection;
%	$(iv)$ necessitation holds on $\Fmc$ iff $\Fmc$ contains the unit.
%Given an R-frame $\Fmf$, congruence, monotonicity, agglomeration, and necessitation hold on $\Fmf$.
%moreover, for every R-model $\Mmf$, they hold in $\Mmf$.
%\end{restatable}
%
%





%%%%%%%%%%%%%%%%%%%%%%%%%%%%%%%
\subsubsection{Relational Semantics}
%\nb{T: Is the extension of concept $\bot$ defined anywhere? \\ M: comes from abbreviation above}
A
\emph{relational frame} 
%\emph{$n$-relational frame} 
%shortened to \emph{\Rframe} when $n$ is clear from the context)
is a pair
$\Fmf = ( \reldomain, \{\relations_i\}_{i \in J})$,
with 
$\reldomain$ non-empty set and $\relations_i$ 
binary relation on $\reldomain$,
for $i \in J = \{ 1, \ldots, n \}$.
%called \emph{accessibility relation}.
%
An \emph{$\MLALC{n}$ (constant domain) relational model} 
based on a relational frame
%$n$\Rframe 
$\Fmf = ( W, \{ R_{i} \}_{i \in J})$
is a
pair
$\Mmf = ( \Fmf, I)$,
%triple
%$\Mmf = ( \Fmf, \Delta, I)$,
where
%$\Delta$ is a non-empty set, called the \emph{domain} of $\Mmf$,
$I$ is a function associating with every $w \in W$ an \ALC \emph{interpretation}
%(or \emph{model})
$I_w = (\Delta, \cdot^{I_{w}})$,
%where $\Delta$ is a non-empty set, called the \emph{constant domain} of $\Mmf$,
having non-empty \emph{constant domain} $\Delta$,
and where $\cdot^{I_{w}}$ is a function such that:
for all $A \in \NC$, $A^{I_{w}} \subseteq \Delta$;
for all $\role \in \NR$, $\role^{I_{w}} \subseteq \Delta {\times} \Delta$;
for all $a \in \NI$, $a^{I_{w}} \in \Delta$, and for all $u, v \in \reldomain$, $a^{I_u}=a^{I_{v}}$(denoted by $a^{I}$).
%
%\nb{O: replace $\B$ by $\B_i$ below \\ M: Done, thanks}
Given a relational model $M = (F, I)$ and a world $w \in W$ of $F$ (or simply $w$ in $F$), the 
\emph{interpretation of a concept $C$ in $w$}, written $C^{I_{w}}$, is defined by taking:
%
$
	(\neg C)^{I_{w}} = \Delta \setminus C^{I_{w}},$
	$
	(C \sqcap D)^{I_{w}}  = C^{I_{w}} \cap D^{I_{w}},
	$
	$
	(\exists \role.C)^{I_{w}} =
		\{d \in \Delta \mid \exists  e \in C^{I_{w}}{:}(d,e) \in \role^{I_{w}}\},
		$
		$
	(\B_{i} C)^{I_{w}} = 
		\{ d \in \Delta \mid \ \forall v \in W:
		w R_{i} v
\Rightarrow
	d \in C^{I_{v}} \}.
$
%\begin{align*}
%	(\neg C)^{I_{w}} & = \Delta \setminus C^{I_{w}}, \quad (C \sqcap D)^{I_{w}}  = C^{I_{w}} \cap D^{I_{w}},\\
%%	(C \sqcap D)^{I_{w}} & = C^{I_{w}} \cap D^{I_{w}}, \\
%	(\exists \role.C)^{I_{w}} & =
%%	\begin{aligned}[t]
%		\{d \in \Delta \mid \exists  e \in C^{I_{w}}{:}(d,e) \in \role^{I_{w}}\}, \\
%%	\end{aligned}\\
%	(\B_{i} C)^{I_{w}} & = 
%%	\begin{aligned}[t]
%		\{ d \in \Delta \mid \ \forall v \in W:
%%		\text{if} \
%		w R_{i} v,
%%		\\
%%	& \
%%	\text{then} \
%\Rightarrow
%	d \in C^{I_{v}} \}.
%%	\end{aligned}
% \end{align*}
%

A concept $C$ is \emph{satisfied in $\Mmf$} if there is $w$ in $\Fmf$ such that $C^{I_{w}} 
\neq \eset$, and that $C$ is \emph{satisfiable on relational models} if there is a relational model in which it is satisfied.
%
The \emph{satisfaction of a $\MLALC{}$ formula~$\p$ in $w$ of $\Mmf$}, written $\Mmf, w  \models \p$, is defined, for atoms, negation and conjunction, similarly to the previous case, and as follows for the $\Box_{i}$ case:
%\begin{alignat*}{3}
%	& \Mmf, w \models C\sqsubseteq D  && \text{ \ iff \ } && C^{I_{w}} \subseteq D^{I_{w}}, \\
%	& \Mmf, w  \models C(a) && \text{ \ iff \ } && a^{I} \in C^{I_{w}}, \\
%	& \Mmf, w \models \role(a,b) && \text{ \ iff \ } && (a^{I},b^{I}) \in \role^{I_{w}}, \\
%	& \Mmf, w  \models \neg \varphi && \text{ \ iff \ } && \Mmf, w  \not \models \p,
%	\end{alignat*}
%\begin{alignat*}{3}
%	& \Mmf, w  \models \varphi\land \psi && \text{ \ iff \ } && \Mmf, w  \models \varphi\text{ and } \Mmf, w  \models \psi, \\
$
%		&
		\Mmf, w  \models \B_{i} \varphi
%		&&
		\text{ \ iff \ }
%		&&
		\forall v \in \reldomain: w \relations_{i} v \Rightarrow \Mmf, v  \models \p.
		$
% \end{alignat*}
%
Given a relational frame $\Fmf = (\reldomain, \{\relations_i\}_{i \in J})$
and a relational model $\Mmf = (\Fmf, \Delta, I)$,
we say that $\varphi$ is \emph{satisfied in $\Mmf$} if there is $w \in \reldomain$ such that
$\Mmf, w \models \varphi$,
and that $\p$ is \emph{satisfiable on relational models} if it is satisfied in some relational model.
%
Also, $\p$ is said to be \emph{valid in $\Mmf$}, $\Mmf \models \p$, if it is satisfied in all $w$ 
of $\Mmf$, and it is \emph{valid on $\Fmf$} if, for all $\Mmf$ based on $\Fmf$,
%and all $w \in W$, $\Mmf, w \models \p$,
$\p$ is valid in $\Mmf$,
writing $\Fmf \models \p$.
%
%Moreover,
%%\nb{M: Used in ex only. Remove (with ex)?}
%$\p$ \emph{logically implies} a formula $\psi$,
%writing $\varphi\mdl \psi$,
%if $\Mmf, w \models \p$ implies $\Mmf, w \models \psi$,  for every $\Mmf$ and every $w\in W$ in $M$.
% in $\Mmf$.
%
%\subsubsection{Satisfiability Problems}
%\nb{O: it is strange to start talking about neighb semantics, go to relation, and then come back here. }
%\begin{itemize}
%	\item  all neighbourhood frames, for $\mathit{L} = \mathbf{E}$;
%	\item  supplemented neighbourhood frames, for $\mathit{L} = \mathbf{M}$;
%	\item  neighbourhood frames closed under intersection, for $\mathit{L} = \mathbf{C}$;
%	\item neighbourhood frames containing the unit, for $\mathit{L} = \mathbf{N}$;
%	\item neighbourhood frames satisfying corresponding combinations of properties above, for $\mathit{L} \in \{ \mathbf{MC}, \mathbf{MN}, \mathbf{CN},  \mathbf{MCN} \}$.
%\end{itemize}
%

%By the \emph{$\MLALC{n}$ formula satisfiability problem in a class of (respectively, N- or R-) frames $\Cmc$} we mean the problem of deciding whether an $\MLALC{n}$ formula is satisfied in a (respectively, N- or R-) model based on a frame in $\Cmc$.
%%
%The \emph{formula satisfiability problem for} $\EnALC{n}$, $\MnALC{n}$, and $\KnALC{n}$ is the $\MLALC{n}$ formula satisfiability problem in the class of N-frames, supplemented N-frames, and R-frames,
%respectively.

%%% EXAMPLE (not needed)
%For example, the formula satisfiability problem for ${\mathbf{C}}^{n}_{\ALC}$
%is the formula satisfiability problem in the class of neighbourhood frames closed under intersection. For every neighbourhood frame \Fmc in this class, it holds that 
%$\Fmc \models \Box_{i} C \sqcap \Box_{i} D \sqsubseteq \Box_{i} (C \sqcap D)$~\cite{DL19} .
%%while, for $\mathit{L} \in \{ \mathbf{MC}, \mathbf{MN}, \mathbf{CN} \}$, the above properties are combined in the obvious way. 

%The \emph{formula satisfiability problem for} $\EnALC{n}$, $\MnALC{n}$,
%{\color{blue} ${\mathbf{C}}^{n}_{\ALC}$, ${\mathbf{N}}^{n}_{\ALC}$,}
%\nb{M: todo remove red}
%{\color{red}{and $\KnALC{n}$}} is the $\MLALC{n}$ formula satisfiability problem in the class of neighbourhood frames, supplemented neighbourhood frames, {\color{blue}{neighbourhood frames closed under intersection, neighbourhood frames containing the unit,}} {\color{red}{and R-frames}},
%respectively.
















%%% FIGURE PANTHEON
%% Figure environment removed




 

%%%%%%%%%%%%%%%%%%%%%%%%%%%%%%%
\subsection{Frame Conditions and Formula Satisfiability}
%\subsubsection{Frames and Satisfiability Problems}







%Given $L \in \mathsf{Pantheon}$, 
%\todo{M: removed `suppl.', `clos. und. intersec.' and `contain. the unit' throughout the paper (very few and useless occurrences) -- do the same with principles? -- mention that this are just (more or less) standard naming conventions from the literature}
We consider the following conditions on neighbourhood frames $\Fmc = ( \Wmc, \{\Nmc_i \}_{i \in J})$. We say that  \emph{$\Fmc$ satisfies the}:
%\nb{M: todo}
%
\begin{alignat*}{3}
	\text{\emph{$\mathbf{E}$-condition}} && \text{ \ iff \ } & \text{ $\Nmc_{i}$ is a neighbourhood function; } \\
	\text{\emph{$\mathbf{M}$-condition}} && \text{ \ iff \ }& \text{ $\alpha\in \Nmc_{i}(w)$ and $\alpha\subseteq\beta$ implies $\beta\in \Nmc_{i}(w)$; } \\
	\text{\emph{$\mathbf{C}$-condition}} && \text{ \ iff \ } & \text{ $\alpha\in \Nmc_{i}(w)$ and $\beta\in \Nmc_{i}(w)$ implies $\alpha\cap\beta\in \Nmc_{i}(w)$; } \\
	\textnormal{\emph{$\mathbf{N}$-condition}}  && \text{ \ iff \ } & \text{ $\Wmc \in \Nmc_{i}(w)$; } \\
	\text{\emph{$\mathbf{T}$-condition}} && \text{ \ iff \ } & \text{ $\alpha \in \Nmc_{i}(w)$ implies $w \in \alpha$; } \\
	\text{\emph{$\mathbf{D}$-condition}} && \text{ \ iff \ } & \text{  $\alpha \in \Nmc_{i}(w)$ implies $\Wmc \setminus \alpha \not \in \Nmc_{i}(w)$; } \\
	\text{\emph{$\mathbf{P}$-condition}} && \text{ \ iff \ } & \text{ $\emptyset \not \in \Nmc_{i}(w)$; } \\
	\text{\emph{$\mathbf{Q}$-condition}} && \text{ \ iff \ } & \text{  $\Wmc \not \in \Nmc_{i}(w)$; }
\end{alignat*}
%\begin{description}
%	\item[\textnormal{\emph{$\mathbf{E}$-condition}}] iff $\Nmc_{i}$ is a neighbourhood function;
%	\item[\textnormal{\emph{$\mathbf{M}$-condition}}]
%%	(\emph{supplementation})
%	iff $\alpha\in \Nmc_{i}(w)$ and $\alpha\subseteq\beta$ implies $\beta\in \Nmc_{i}(w)$;
%	\item[\textnormal{\emph{$\mathbf{C}$-condition}}]
%%	(\emph{closure under intersection})
%	iff $\alpha\in \Nmc_{i}(w)$ and $\beta\in \Nmc_{i}(w)$ implies $\alpha\cap\beta\in \Nmc_{i}(w)$;
%	\item[\textnormal{\emph{$\mathbf{N}$-condition}}]
%%	(\emph{containment of unit})
%	iff $\Wmc \in \Nmc_{i}(w)$;
%	%\end{description}
%	%
%	%\begin{description}
%		\item[\textnormal{\emph{$\mathbf{T}$-condition}}] iff $\alpha \in \Nmc_{i}(w)$ implies $w \in \alpha$;
%			\item[\textnormal{\emph{$\mathbf{D}$-condition}}] iff $\alpha \in \Nmc_{i}(w)$ implies $\Wmc \setminus \alpha \not \in \Nmc_{i}(w)$;
%	\item[\textnormal{\emph{$\mathbf{P}$-condition}}] iff $\emptyset \not \in \Nmc_{i}(w)$;
%	\item[\textnormal{\emph{$\mathbf{Q}$-condition}}] iff $\Wmc \not \in \Nmc_{i}(w)$;
%\end{description}
for every $w\in \Wmc$, $\alpha,\beta\subseteq \Wmc$.
%
%{\color{blue}{
Combinations of conditions, such as the $\mathbf{EMCN}$-condition, are obtained by suitably joining the ones above.
Moreover, since the $\mathbf{E}$-condition is always satisfied by any neighbourhood frame, we often omit the letter $\mathbf{E}$ from this naming scheme, writing for instance `$\mathbf{MCN}$' in place of `$\mathbf{EMCN}$'.
%}}

%Finally,
%given a neighbourhood frame of the form
%$\Fmc = ( \Wmc, \{ \Nmc_i , \Nmc'_i \}_{i \in J})$,
%we say that it is an \emph{interaction frame} if it satisfies
%the following condition, for every $i \in J$ and $w \in \Wmc$:
%\begin{description}
%	\item[\textnormal{\emph{Interaction condition}:}] $\Nmc_{i}(w) \subseteq \Nmc'_{i}(w)$.
%\end{description}




%We say that a neighbourhood frame
%$\Fmc = ( \Wmc, \{\Nmc_i \}_{i \in J})$,
%with $J = \{1, \ldots, n\}$,
%is
%an \emph{$\EM^{n}$ frame} (or that it is \emph{supplemented}),
%a \emph{$\EC^{n}$ frame} (or that it is \emph{closed under intersection}),
%or an \emph{$\EN^{n}$ frame} (or that it \emph{contains the unit})
%if the neighbourhood functions $\Nmc_i$ satisfy, respectively, the following conditions, for every $i \in J$, $w\in \Wmc$, $\alpha,\beta\subseteq \Wmc$.
%
%\begin{description}
%	\item[\textnormal{\emph{$M$-condition} (\emph{supplementation}):}] $\alpha\in \Nmc_{i}(w)$ and $\alpha\subseteq\beta$ implies $\beta\in \Nmc_{i}(w)$.
%	\item[\textnormal{\emph{$C$-condition} (\emph{closure under intersection}):}] $\alpha\in \Nmc_{i}(w)$ and $\beta\in \Nmc_{i}(w)$ implies $\alpha\cap\beta\in \Nmc_{i}(w)$.
%	\item[\textnormal{\emph{$N$-condition} (\emph{containment of unit}):}] $\Wmc \in \Nmc_{i}(w)$.
%\end{description}
%
%In addition, we say that a neighbourhood frame
%%$\Fmc = ( \Wmc, \{\Nmc_i \}_{i \in J})$,
%%with $J = \{1, \ldots, n\}$,
%is
%an \emph{$\ED^{n}$ frame},
%an \emph{$\ET^{n}$ frame},
%an \emph{$\EP^{n}$ frame},
%or
%an \emph{$\EQ^{n}$ frame},
%if its neighbourhood functions satisfy, respectively, the following conditions, for every $i \in J$, $w\in \Wmc$, $\alpha\subseteq \Wmc$.
%
%%\todo[inline]{$D: \lnot (\Box \varphi \land \Box \lnot \varphi)$}
%
%\begin{description}
%	\item[\textnormal{\emph{$D$-condition}:}]  $\alpha \in \Nmc_{i}(w)$ implies $\Wmc \setminus \alpha \not \in \Nmc_{i}(w)$.
%	\item[\textnormal{\emph{$T$-condition}:}] $\alpha \in \Nmc_{i}(w)$ implies $w \in \alpha$.
%	\item[\textnormal{\emph{$P$-condition}:}] $\emptyset \not \in \Nmc_{i}(w)$.
%		\item[\textnormal{\emph{$Q$-condition}:}] $\Wmc \not \in \Nmc_{i}(w)$.
%\end{description}
%
%Finally,
%given a neighbourhood frame of the form
%$\Fmc = ( \Wmc, \{ \Nmc_i , \Nmc'_i \}_{i \in J})$,
%we say that it is an \emph{interaction frame} if it satisfies
%the following condition, for every $i \in J$ and $w \in \Wmc$.
%\begin{description}
%	\item[\textnormal{\emph{Interaction condition}:}] $\Nmc_{i}(w) \subseteq \Nmc'_{i}(w)$.
%\end{description}


%A frame is:
%\emph{supplemented} if,
%for all
%$i \in J$,
%$w\in \Wmc$, 
%$\alpha,\beta\subseteq \Wmc$, $\alpha\in \Nmc_{i}(w)$ and $\alpha\subseteq\beta$ implies $\beta\in \Nmc_{i}(w)$;
%%it is
%\emph{closed under intersection} if,
%for all
%$i \in J$,
%$w\in \Wmc$, $\alpha,\beta\subseteq \Wmc$, 
%$\alpha\in \Nmc_{i}(w)$ and $\beta\in \Nmc_{i}(w)$ implies $\alpha\cap\beta\in \Nmc_{i}(w)$;
%and
%%it
%\emph{contains the unit} if,
%for all
%$i \in J$,
%$w\in \Wmc, \Wmc \in \Nmc_{i}(w)$.











%In the following, let
%%\[
%$
%\mathsf{Cube} = \{ \E, \EM, \EC, \EN, \EMC, \EMN, \ECN, \EMCN
%\},
%$
%%\]
%and let $\mathsf{Pantheon}$ be
%%the union of
%%$\Log$ with
%the set (cf. Figure~\ref{fig:pantheon})
%\begin{align*}
%	\{
%	&
%	\mathbf{E},\\
%	&
%	\mathbf{EM},	 \mathbf{EC},
%	\mathbf{EN},
%	\mathbf{EP}, \mathbf{EQ},
%	\mathbf{ED}, \mathbf{ET}, \\
%	&
%	\mathbf{EMN},
%	\mathbf{EMC},
%	\mathbf{EMP},
%	\mathbf{{\color{red}{EMQ}}},
%	\mathbf{EMD},
%	\mathbf{EMT},\\
%	&
%	\mathbf{ECN},
%	\mathbf{ECP}, \mathbf{ECQ},
%	\mathbf{ECD}, \mathbf{ECT},\\
%	&
%	\mathbf{ENP}, 
%	\mathbf{END},
%	\mathbf{ENT}, \\
%	&	 
%	\mathbf{EPQ},
%	\mathbf{EPD}, \mathbf{EQD},
%	\mathbf{EQT},\\
%	&
%	\mathbf{EMNP},
%	\mathbf{EMND}, \mathbf{EMNT}, \\
%	&
%	\mathbf{EMCD}, \mathbf{EMCT},\\
%	&
%	\mathbf{ECND}, \mathbf{ECNT},
%	\mathbf{ECPQ},
%	\mathbf{ECQD}, \mathbf{ECQT}, \\
%	&
%	\mathbf{EMCN},
%	\mathbf{EPQD},\\
%	&
%	\mathbf{EMCND}, \mathbf{EMCNT}
%	\}
%\end{align*}


%\subsubsection{Relationships among %Neighbourhood Frame 
%	Conditions} 
%	\todo{M: todo add ignore $MQ$}

On the relationships among (combinations of) neighbourhood frame conditions, we make the following observations.
%\todo{O: we say above L belongs to pantheon, talk about L condition but here in L we omit E without giving explanation}

\begin{restatable}{theorem}{PropImplicationSystem}\label{prop:implicationsystem}
Given a neighbourhood frame
$\Fmc = ( \Wmc, \{\Nmc_i \}_{i \in J})$, the following statements hold, for $i \in J$.
\begin{enumerate}
%[label=\arabic*]
	\item If $\Nmc_i$ satisfies the $\mathbf{MQ}$-condition then, for every $w \in \Wmc$, $\Nmc_{i}(w) = \emptyset$. Hence, $\Nmc_i$ satisfies all but the $\mathbf{N}$-condition.
%	all conditions except for $\mathbf{N}$ are satisfied.
	\item\label{item:P-cond} $\Nmc_i$ satisfies the $\mathbf{P}$-condition, if $\Nmc_i$ satisfies one of the following:\\
		\begin{enumerate*}[label=(\roman*)]
%			\item $\mathbf{MQ}$-condition;
			\item $\mathbf{MD}$-condition;
			\item $\mathbf{ND}$-condition; or
			\item $\mathbf{T}$-condition.
		\end{enumerate*}
	\item\label{item:D-cond} $\Nmc_i$ satisfies the $\mathbf{D}$-condition, if $\Nmc_i$ satisfies one of the following:\\
		\begin{enumerate*}[label=(\roman*)]
%				\item $\mathbf{MQ}$-condition; %\nb{O:added}
			\item $\mathbf{CP}$-condition; or
			\item $\mathbf{T}$-condition.
		\end{enumerate*}	
	\item $\Nmc_i$ does not satisfy the $\mathbf{NQ}$-condition.
\end{enumerate}
%\begin{enumerate}
%	\item if $\Nmc_i$ satisfies the $\mathit{MQ}$-condition, then $\Nmc_i$ satisfies the $\mathit{P}$-condition and $\Nmc_{i}(w) = \emptyset$, for every $w \in \Wmc$;
%	\item if $\Nmc_i$ satisfies the $\mathit{MD}$-condition, then $\Nmc_i$ satisfies the $\mathit{P}$-condition;
%	\item if $\Nmc_i$ satisfies the $\mathit{CP}$-condition, then $\Nmc_i$ satisfies the $\mathit{D}$-condition;
%	\item $\Nmc_i$ does not satisfy the $\mathit{NQ}$-condition;
%	\item if $\Nmc_i$ satisfies the $\mathit{ND}$-condition, then $\Nmc_i$ satisfies the $\mathit{P}$-condition;
%		\item if $\Nmc_i$ satisfies the $\mathit{T}$-condition, then $\Nmc_i$ satisfies the $\mathit{P}$-condition;
%	\item if $\Nmc_i$ satisfies the $\mathit{T}$-condition, then $\Nmc_i$ satisfies the $\mathit{D}$-condition.
%\end{enumerate}
\end{restatable}
%














% Figure environment removed


%% Figure environment removed


















%{\color{blue}{ 
Based on these results, Figure~\ref{fig:pantheon} depicts the relations between combinations of frame conditions: nodes are (groups of equivalent) conditions (with the canonical representative underlined), and arrows represent logical implications.
Any combination containing the $\mathbf{NQ}$-condition has been omitted, as it leads to inconsistency (Theorem~\ref{prop:implicationsystem}, Point 4). Moreover, due to Theorem~\ref{prop:implicationsystem}, Point 1, any combination that includes the $\mathbf{MQ}$-condition is not considered, since
%$\Nmc_{i}(w) = \emptyset$ enforces that,
for any neighbourhood frame $\Fmc$ satisfying such condition and any
%$\MLnALC$ formula $\psi$, we have $\Fmc \models \Box \psi \leftrightarrow \mathsf{false}$,
$\MLnALC$ concept $C$, we have $\Fmc \models \Box_{i} C \equiv \bot$,
and similarly for formulas,
hence trivialising the modal operators.
Thus, we consider in the remainder the set $\Log$ of 39 non-equivalent combinations shown (as nodes or canonical representatives) in Figure~\ref{fig:pantheon}.
%}}










For $\Lvar \in \Log$,
we say that a neighbourhood frame
$\Fmc = ( \Wmc, \{\Nmc_i \}_{i \in J})$,
with $J = \{1, \ldots, n\}$, is an \emph{$L^{n}$ frame} iff its neighbourhood functions $\Nmc_i$, for $i \in J$, satisfy the \emph{$\Lvar$-condition},
obtained by  combining the conditions associated with letters in $\Lvar$.
%\nb{M: todo fix}
For a class of neighbourhood frames $\Cmc$, the \emph{satisfiability in $\MLALC{n}$ on} (\emph{varying} or \emph{constant domain}, resp.) \emph{neighbourhood models based on a frame in $\Cmc$} is the problem of deciding whether an $\MLALC{n}$ formula is satisfied in a (varying or constant domain, resp.) neighbourhood model based on a frame in $\Cmc$.
%
%Given $\mathit{L} \in \mathsf{Pantheon}$, 
Satisfiability in \emph{$\LnALC$ on} (\emph{varying} or \emph{constant domain}, respectively) \emph{neighbourhood models} is satisfiability in $\MLALC{n}$ on (varying or constant domain, resp.) neighbourhood models based on a frame in the class of $L^{n}$ frames.
%
Finally,  \emph{satisfiability  in $\KnALC{n}$ on}
(\emph{constant domain})
\emph{relational models} is satisfiability in $\MLALC{n}$  on
%constant domain
relational models based on any relational frame.


















%\subsubsection{Correspondence: Conditions and Principles} 

%In the following, we use $\mathfrak F$ to stand either for a neighbourhood or relational frame, and $\mathfrak M$ for a neighbourhood or relational model. \nb{O: move this note since this is not used in prop 2?}
%
%To define the $\MLALC{n}$ formula satisfiability problems studied in this paper,







% $\mathfrak M$.
%For a principle $P$, if $S = \mathfrak{F}$ (respectively, $S = \mathfrak{M}$), we say that \emph{$P$ holds in $\mathfrak{F}$} (respectively, \emph{in $\mathfrak{M}$}).
%
%\begin{comment}
%\begin{table*}
%\begin{center}
%\begin{tabular}{l l l}
%\hline
%\multirow{2}{*}{\emph{${E}$-principle} (\emph{congruence})} && $S \models C \equiv D$ implies $S \models \Box_{i} C \equiv \Box_{i} D$. \\
%&& $S \models \varphi\leftrightarrow \psi$ implies $S \models \Box_{i} \varphi\leftrightarrow \Box_{i} \psi$. \\
%\hline
%\multirow{2}{*}{\emph{${M}$-principle} (\emph{monotonicity})} && $S \models C \sqsubseteq D$ implies $S \models \Box_{i} C \sqsubseteq \Box_{i} D$. \\
%&& $S \models \varphi\to \psi$ implies $S \models \Box_{i} \varphi\to \Box_{i} \psi$. \\
%\hline
%\multirow{2}{*}{\emph{${C}$-principle} (\emph{agglomeration})} &\quad\quad& $S \models \Box_{i} C \sqcap \Box_{i} D \sqsubseteq \Box_{i} (C \sqcap D)$. \\
%&& $S \models \Box_{i} \varphi\land \Box_{i} \psi \to \Box_{i} (\varphi \land \psi)$. \\
%%\hline
%%\textcolor{red}{(\emph{Agglomeration'})} & $C\sqcap D = E$ is valid implies $\Box_{i} C \sqcap \Box_{i} D \sqsubseteq \Box_{i} E$ is valid. \\
%%& $\phi \land \psi \leftrightarrow \xi$ is valid implies $\Box_{i} \phi \land \Box_{i} \psi \to \Box_{i} \xi$ is valid. \\
%\hline
%\multirow{2}{*}{\emph{${N}$-principle} (\emph{necessitation})} && $S \models \top \sqsubseteq C$ implies $S \models \top \sqsubseteq \Box_{i} C$. \\
%&& $S \models \varphi$ implies $S \models \Box_{i} \p$. \\
%\hline
%\multirow{2}{*}{\emph{${P}$-principle}} && $S \models \top \sqsubseteq  \lnot \Box_{i} \bot$. \\
%&& $S \models \lnot \Box_{i} \mathsf{false}$. \\
%\hline
%\multirow{2}{*}{\emph{${Q}$-principle}} && $S \models \top \sqsubseteq  \lnot \Box_{i} \top$. \\
%&& $S \models  \lnot \Box_{i} \mathsf{true}$. \\
%\hline
%\multirow{2}{*}{\emph{${D}$-principle} \textcolor{blue}{(\emph{deontic})}}  && $S \models \Box_{i} C \sqsubseteq \Diamond_{i} C $ \\
%&& $S \models \Box_{i} \varphi \to \Diamond_{i} \varphi$. \\
%\hline
%\multirow{2}{*}{\emph{${T}$-principle}} && $S \models \Box_{i} C \sqsubseteq  C $ \\
%&& $S \models \Box_{i} \varphi \to \varphi$. \\
%\hline
%\end{tabular}
%\end{center}
%\caption{Principles over frames and models.}
%\label{tab:principles}
%\end{table*}
%\end{comment}

\begin{table*}
\begin{center}
\footnotesize
\begin{tabular}{l l l}
\toprule
\multirow{2}{*}{\emph{${\mathbf{E}}$-principle}} & $S \models C \equiv D$ implies $S \models \Box_{i} C \equiv \Box_{i} D$. \\
% (\emph{congruence})
 & $S \models \varphi\leftrightarrow \psi$ implies $S \models \Box_{i} \varphi\leftrightarrow \Box_{i} \psi$. \\
\midrule
\multirow{2}{*}{\emph{${\mathbf{M}}$-principle}} & $S \models C \sqsubseteq D$ implies $S \models \Box_{i} C \sqsubseteq \Box_{i} D$. \\
% (\emph{monotonicity})
 & $S \models \varphi\to \psi$ implies $S \models \Box_{i} \varphi\to \Box_{i} \psi$. \\
\midrule
\multirow{2}{*}{\emph{${\mathbf{C}}$-principle}} & $S \models \Box_{i} C \sqcap \Box_{i} D \sqsubseteq \Box_{i} (C \sqcap D)$. \\
% (\emph{agglomeration})
 & $S \models \Box_{i} \varphi\land \Box_{i} \psi \to \Box_{i} (\varphi \land \psi)$. \\
\midrule
\multirow{2}{*}{\emph{${\mathbf{N}}$-principle}} & $S \models \top \sqsubseteq C$ implies $S \models \top \sqsubseteq \Box_{i} C$. \\
% (\emph{necessitation})
 & $S \models \varphi$ implies $S \models \Box_{i} \p$. \\
\bottomrule
\end{tabular}
\quad
\begin{tabular}{l l l}
\toprule
\multirow{2}{*}{\emph{${\mathbf{T}}$-principle}} & $S \models \Box_{i} C \sqsubseteq  C $. \\
& $S \models \Box_{i} \varphi \to \varphi$. \\
\midrule
\multirow{2}{*}{\emph{${\mathbf{D}}$-principle}} & $S \models \Box_{i} C \sqsubseteq \Diamond_{i} C $. \\
%{(\emph{deontic})}
& $S \models \Box_{i} \varphi \to \Diamond_{i} \varphi$. \\
\midrule
\multirow{2}{*}{\emph{${\mathbf{P}}$-principle}} & $S \models \top \sqsubseteq  \lnot \Box_{i} \bot$. \\
& $S \models \lnot \Box_{i} \mathsf{false}$. \\
\midrule
\multirow{2}{*}{\emph{${\mathbf{Q}}$-principle}} & $S \models \top \sqsubseteq  \lnot \Box_{i} \top$. \\
& $S \models  \lnot \Box_{i} \mathsf{true}$. \\
\bottomrule
\end{tabular}
\end{center}
\caption{Principles over neighbourhood or relational frames and models $S$.}
\label{tab:principles}
\end{table*}


We now study the correspondence between %the neighbourhood  frame
conditions presented in Section~\ref{sec:sem} and
  the principles in
Table~\ref{tab:principles},
%(where $C,D$ and $\varphi, \psi$ are $\MLALC{n}$ concepts and formulas, respectively)
%and where, for $L \in \mathsf{Pantheon}$, 
where
$S$ is either a
%(N- or R-)
(neighbourhood or relational) frame %$\mathfrak F$, 
or a (neighbourhood or relational) model
and the $L$-principle is obtained by suitably combining the basic principles.
We say that the $L$-principle holds in $S$ if the corresponding
expressions  in Table~\ref{tab:principles} are satisfied.
On the correspondence between the principles in
Table~\ref{tab:principles} and conditions over frames and models,
we have the following results (see e.g.~\cite{Pac}
%\nb{T: Changed. In Chellas there are no correspondence theorems. Ref to Pacuit. \\ M: Thanks}
for the propositional case).


%\nb{T: Red text for necessitation. Is it ok? Or how should I write it? \\ M: I think it's ok}
%

%\todo{M: fixed varying domain case}


\begin{restatable}{proposition}{PropCorresp}\label{prop:corresp}
%The following statements hold.
%\begin{enumerate}
%	\item
	Given %$L \in \mathsf{Pantheon}$ and 
	a neighbourhood frame $\Fmc$, %we have that
the $\Lvar$-principle holds in $\Fmc$ iff $\Fmc$ satisfies the $\Lvar$-condition.
%	\item Given a relational frame $\Fmf$ and a relational model $\Mmf$ based on $\Fmf$, the $\mathit{EMCN}$-principle holds in $\Mmf$, and hence on $\Fmf$.
%%	\item Given a relational frame $\Fmf$, the $\mathit{EMCN}$-principle holds on $\Fmf$. Moreover, for every relational model $\Mmf$, the $\mathit{EMCN}$-principle holds in $\Mmf$.
%\end{enumerate}
\end{restatable}
%\begin{restatable}{theorem}{PropValid}\label{prop:dlvalid}
%Given a neighbourhood frame $\Fmc$, we have that:
%%
%	$(i)$ congruence holds on $\Fmc$;
%	$(ii)$ monotonicity holds on $\Fmc$ iff $\Fmc$ is supplemented;
%	$(iii)$ agglomeration holds on $\Fmc$ iff $\Fmc$ is closed under intersection;
%	$(iv)$ necessitation holds on $\Fmc$ iff $\Fmc$ contains the unit.
%Given a relational frame $\Fmf$, congruence, monotonicity, agglomeration, and necessitation hold on $\Fmf$; moreover, for every relational model $\Mmf$, they hold in $\Mmf$.
%\end{restatable}
%










\begin{restatable}{proposition}{PropValid}
\label{prop:valid}
%
%Given $L \in \mathsf{Pantheon}$, 
The following statements hold.
%
\begin{enumerate}
	\item For a (varying or constant domain) neighbourhood model $\Mmc$, we have that if $\Mmc$ satisfies the $\Lvar$-condition, then the $\Lvar$-principle holds in $\Mmc$.
	However, in general, the converse is not true.
	\item For a relational frame $\Fmf$ and a relational model $\Mmf$ based on $\Fmf$, the $\mathbf{EMCN}$-principle holds in $\Mmf$, hence in $\Fmf$.
%	Moreover, in $\Mmf$, and hence in $\Fmf$, we have that:
%	\begin{itemize}
%		\item  if the $\mathbf{T}$-principle holds, then the $\mathbf{D}$-principle holds;
%		\item the $\mathbf{D}$-principle holds iff the
%	$\mathbf{P}$-principle holds;
%		\item the $\mathbf{Q}$-principle does not hold.
%	\end{itemize}
%	\todo{AM: mention in proof}
%	}}
	Moreover, in $\Mmf$, hence in $\Fmf$,  the $\mathbf{D}$-principle holds iff the
	$\mathbf{P}$-principle holds, and the $\mathbf{Q}$-principle does not hold.
	%\todo{AM: mention in proof. T: proof added, removed claim about the T principle as I don't find it relevant (same behaviour as in the neighbourhood semantics)}
\end{enumerate}
%
\end{restatable}
%















%{\color{blue}{
%As a difference with neighbourhood frames, correspondence between 
%satisfaction of semantic conditions and validity of the corresponding principles does not generally holds for neighbourhood models.
%In particular, it is not always the case that 
%if an $\Lvar$-principle holds in a model $\Mmc$,
%then $\Mmc$ satisfies the $\Lvar$-condition.
%}}



%%%%%%%%%%%%%%%%%%%%%%%%%%%%%%%%%%%%%%%%%%%%%%%%%%%%%%%%%%%%%%%%%%%%%%
\endinput

%%% Local Variables:
%%% mode: latex
%%% TeX-master: "dl18"
%%% End:


%\section{Reasoning on Varying Domains}
\label{sec:reasonvardom}
%\section{Reasoning in Non-normal Modal Description Logics with Varying Domains}


\section{Tableaux for Formula Satisfiability}
%\subsection{Tableaux for $\LnALC$ Formula Satisfiability}
\label{sec:tableaux}

%\subsection{Tableaux for Non-normal Modal Description Logics}
% on Varying Domain Neighbourhood Models
%\section{Reasoning in Non-normal Modal Description Logics}


We provide terminating, sound, and complete tableau algorithms to check satisfiability of formulas in varying domain neighbourhood models. The notation partly adheres to that of~\cite{GabEtAl03},
while the model construction in the soundness proof is based on the strategy of~\cite{DalHyp}.
%\textcolor{red}{To cite: \cite{DL19} our previous work on non-normal DL, \cite{DalHyp} for the definition of the model in Theorem 2}
%\subsection{Tableaux for Non-normal Modal Description Logics}
%{\color{blue}{
%}}
%\todo{T: also $\lor$? M: yup, grazie}
%Adhering to the notation used in~\cite{SeyErd09},
%We require the following preliminary notions.
In this section, we use concepts and formulas in \emph{negation normal form} 
(\emph{NNF}) and, for this reason, we consider all the logical connectives $\sqcup, \lor, \forall, \Diamond$ as primitive, rather than defined. 
%Given an $\MLALC{n}$ formula $\p$, we denote by $\conneg(\p)$ and $\forneg(\p)$ the set of subconcepts and subformulas of $\p$, respectively.
For a concept or formula $\gamma$, we denote by $\dot{\lnot}\gamma$ the negation of $\gamma$ put in {NNF},
%defined as usual.
 defined as follows:
a concept is in \emph{NNF} if negation occurs in it only in front of concept names; a formula is in \emph{NNF} if all concepts in it are in NNF and negation occurs in the formula only in front of CIs or assertions of the form $r(a,b)$ (regarding
assertions of the form $A(a)$, we recall that a formula $\lnot \psi$, with $\psi = C(a)$, is equivalent to the assertion $D(a)$, with $D = \lnot C$).
Given an $\MLALC{n}$ formula $\p$, we assume without loss of generality that $\p$ is in NNF (using De Morgan laws) 
%and the fact that a formula $\lnot \psi$, with $\psi = C(a)$, is equivalent to the assertion $D(a)$, with $D = \lnot C$),
%}} 
%it contains CIs only of the form $\top \sqsubseteq C$, and every concept occurring in $\p$ is also in NNF.
and it contains CIs only of the form $\top \sqsubseteq C$,
since $C \sqsubseteq D$ is equivalent to $\top \sqsubseteq \lnot C \sqcup D$). 
%
We denote by $\con(\p)$ and $\for(\p)$ the set of subconcepts  and subformulas of $\p$, respectively, and then we set
$\conneg(\p) = \con(\p) \cup \{ \dot{\lnot}C \mid C \in \con(\p) \} \cup  \{ \top \}$ and $\forneg(\p) =  \for(\p) \cup \{ \dot{\lnot}\psi \mid \psi \in \for(\p) \}$.
The sets $\rol(\p)$ and $\ind(\p)$ are, respectively, the sets of role names and of  individual names  occurring in $\p$.
%\todo{M: added $\ind(\p)$ -- todo check and fix the rest}
%notion of a tableau for an $\MLALC{n}$ formula.
Let 
%$\fg(\p) = \forneg(\p) \cup \conneg(\p) \cup \rol(\p) \cup \{ \top \}$.
$\fg(\p) = \forneg(\p) \cup \conneg(\p) \cup \rol(\p) \cup \ind(\p)$ be the \emph{fragment induced by $\varphi$}.
%Note that, %as a consequence of 
%by our assumption on the form of CIs in $\p$, we have $\top\in\conneg(\p)$.\nb{M: to fix}
%%\todo{O: $\top$ is primitive? \\ M: No, it is defined, but no problem here (since it is in NNF by definition).}

%\nb{M: to check - merged T. defs. here (+ small changes)}
%{\color{blue}{

Moreover, let $\NV$ be a countable set of \emph{variables}, denoted by the letters $u, v$.
%The \emph{terms for $\p$} are either individual names in $\ind(\p)$ or variables in $\NV$, and they are denoted by the letters $x, y$.
The \emph{terms for $\p$}, denoted by the letters $x, y$, are either individual names in $\ind(\p)$ or variables in $\NV$.
We assume that the set of terms for $\p$ is
%\todo{O: add "strictly" since well-ordering is normally given by the relation $\leq$}
strictly well-ordered by the relation $<$.
In addition, let $\mathsf{N_{L}}$ be a countable set of \emph{labels}.
%well-ordered by the relation $\ll$.
%Old formulation
%Given an $\MLnALC$ formula $\p$, an \emph{$n$-labelled constraint for $\p$} takes the form $n: \psi$, or $n: C(x)$, or $n: r(x, y)$, where $n \in \mathsf{N_{L}}$, $\psi \in \forneg(\p)$,
%$x, y$ are {\color{blue}{terms for $\p$}},
%%$x, y \in \NV$,
%%$C \in \conneg(\p) \cup \{ \top \}$, 
%$C \in \conneg(\p)$, 
%and $r \in \rol(\p)$.
% New formulation (Tiz)
Given an $\MLnALC$ formula $\p$ and a label $n \in \mathsf{N_{L}}$, an \emph{$n$-labelled constraint for $\p$} takes the form $n: \psi$, or $n: C(x)$, or $n: r(x, y)$, where $\psi \in \forneg(\p)$,
$x, y$ are terms for $\p$,
$C \in \conneg(\p)$, 
and $r \in \rol(\p)$.
For every $n \in \mathsf{N_{L}}$, an \emph{$n$-labelled constraint system for $\p$} is a set $S_{n}$ of $n$-labelled constraints for $\p$.
%An \emph{$n$-labelled constraint system for $\p$} is a set $S_{n}$ of $n$-labelled constraints for $\p$.
A \emph{labelled constraint for $\p$} is an $n$-labelled constraint for $\p$, for some $n \in \mathsf{N_{L}}$, and similarly for a \emph{labelled constraint system for $\p$}.
%A \emph{completion set} $\T$ is a
%%set of labelled constraints for $\p$.
%(non-empty) union of labelled constraint systems for $\p$.
A
%\nb{M: changed - simplified (add `non-empty'?)}
\emph{completion set $\T$ for $\p$} is a non-empty
union
%set
of labelled constraint systems for $\p$,
and we set $\mathsf{L}_{\T} = \{ n \in \mathsf{N_{L}} \mid S_{n} \subseteq \T \}$.
%Finally, a set $\T = \bigcup^{m}_{n = 0} S_{n}$,  for some $m \in \mathsf{N_{L}}$, is called \emph{completion set}, where each $S_{n}$ is an $n$-labelled constraint system for $\p$.
%When no confusion can arise, we omit `$n$' and speak of `labelled constraint for $\p$' or of `labelled constraint system for $\p$'.

About terms, we adopt the following terminology.
%\nb{M: todo add blocking (cf. YB)}
A {{term}} $x$ \emph{occurs in $S_{n}$} if $S_{n}$ contains $n$-labelled constraints of the form $n: C(x)$ or $n: r(\tau,\tau')$, where $\tau = x$, or $\tau' = x$, and $n \in \mathsf{N_{L}}$.
In addition, a variable $u$ is said to be \emph{fresh for $S_{n}$} if $u$ does not occur in $S_{n}$.
%and
%{\color{red}{
%\todo{M: why? to discuss}
%$u > v$, for every $v$ that occurs in $S$.}}
(These notions can be used with respect to $\T$, whenever $S_{n} \subseteq \T$).
%{\color{red}{
%\todo{M: to remove}
%Without loss of generality, we assume that, whenever $x$ occurs in $S_{n}$, the $n$-labelled constraint $n: \top(x)$ is in $S_{n}$.
%}}
%{\color{red}{
%\todo{M: not used, remove?}
%Also, if $n : r(x, y) \in S_{n}$, we call $y$ an \emph{$r$-successor of $x$} with respect to $S_{n}$.
%}}
Finally, given variables $u, v$ in an $n$-labelled constraint system $S_{n}$, we say that $u$ is \emph{blocked by $v$ in $S_{n}$} if $u > v$ and $\{ C \mid n : C(u) \in S_{n} \} \subseteq \{ C \mid n : C(v) \in S_{n} \}$.



A completion set $\T$ contains a \emph{clash} if,
%{\color{blue}{
for some $m \in \mathsf{N_{L}}$,
%formula $\psi$,
concept $C$, role $r$, and terms $x, y$, one of the following holds:
%$\{m: \psi, m: \neg\psi\} \subseteq \T$,
%or
%$\{m: C(x), m: \lnot C(x)\} \subseteq \T$,
%for some $m \in \mathsf{N_{L}}$, and formula $\psi$ or concept $C$.
%$\mathsf{false} \in \T$;
%or
%$\bot(x)$;
%or
$\{m: (\top \sqsubseteq C), m: \lnot (\top \sqsubseteq C) \} \subseteq \T$;
%or
%$\{m: \psi, m: \neg\psi\} \subseteq \T$;
or
$\{m: A(x), m: \lnot A(x)\} \subseteq \T$;
or
$\{m: r(x,y), m: \lnot r(x,y)\} \subseteq \T$.
A completion set that does not contain a clash is \emph{clash-free}.
%}}
%Given a constraint system $S$ for $\p$, we say that $S$ contains a \emph{clash} if there exist a variable $x$ and a concept $C$ such that $\{ x: C, x: \lnot C \} \subseteq S$, or if $\{ \psi, \lnot \psi \} \subseteq S$, for some formula $\psi$.
%\todo{M: extend to $\mathit{L} \in \mathsf{Pantheon}$}
%

For every $\mathit{L} \in \Log$, we associate to $\Lvar$ the set of \emph{$\LnALC$-rules} from Figure~\ref{fig:rules} (bottom part) containing
$\mathsf{R}_{\land}$,
$\mathsf{R}_{\sqcap}$,
$\mathsf{R}_{\lor}$,
$\mathsf{R}_{\sqcup}$,
$\mathsf{R}_{\exists}$,
$\mathsf{R}_{\forall}$,
$\mathsf{R}_{\sqsubseteq}$,
$\mathsf{R}_{\not\sqsubseteq}$,
$\mathsf{R}_{\mathit{L}}$,
and $\mathsf{R}_{\mathit{L}\mathbf{X}}$, for every $\mathbf{X}\in\{\mathbf{N,T,P,Q,D}\}$ such that $\mathbf{X}\in\Lvar$. 
%
Given $\mathit{L} \in \Log$, a completion set $\T$ is $\LnALC$-\emph{complete} if no 
%\emph{$\LnALC$-rule} from Figure~\ref{fig:rules}
$\LnALC$-rule is applicable to $\T$,
%{\color{red}{
where $\gamma_{j}$ is either $\psi_{j} \in \forneg(\p)$ or $C_{j}(x_{j})$, with $C_{j} \in \conneg(\p)$, for $j = 1, \ldots, k$, and $\delta$ is either $\chi \in \forneg(\p)$ or $D(y)$, with $D \in \conneg(\p)$,
%}}
with respect to the \emph{application conditions} associated to each $\LnALC$-rule
from Figure~\ref{fig:rules} (top part).
%\nb{M: todo}
%\begin{itemize}
%
% Figure environment removed
%
%% T: Notion of completability is unnecessary
%{\color{blue}{Finally,  a completion set $\T$ is $\LnALC$-\emph{completable}
%if there exists $\T'\supseteq\T$ such that $\T'$ is clash-free and 
%$\LnALC$-complete,
%otherwise it is $\LnALC$-\emph{uncompletable}}}
%
The $\LnALC$-rules essentially state how to extend a completion set on the basis of the information contained in it.
Branching rules entail a \emph{non-deterministic choice} in the  completion set expansion.

%old sentence
%For each $\Lvar\in \Log$, we now define an algorithm based on $\LnALC$-rules for checking the
%$\LnALC$ formula satisfiability.
%We then prove that the algorithm terminates for every formula $\p$,
%and that it is sound and complete with respect to $\LnALC$ formula satisfiability.

For each $\Lvar\in \Log$,
we now present a
tableau-based non-deterministic decision procedure %$sat(\p)$
for the $\LnALC$ formula satisfiability problem on varying domain neighbourhood models,
%and
based on
%shown in
Algorithm~\ref{alg:tableau}
%which is
(simply referred to as
%the
\emph{$\LnALC$ tableau algorithm}).
%For each $\Lvar\in \Log$,
%we now present a
%tableau-based non-deterministic decision procedure %$sat(\p)$
%for the $\LnALC$ formula satisfiability problem on varying domain neighbourhood models,
%relying on Algorithm~\ref{alg:tableau}.
%based on
%%simply referred to as
%the
%\emph{$\LnALC$ tableau algorithm}
%%and
%shown
%in Algorithm~\ref{alg:tableau}.
%Observe that %Algorithm~\ref{alg:tableau} 
We have that a formula $\p$ is $\LnALC$ satisfiable if and only if there exists at least one execution of the
$\LnALC$ tableau algorithm
%for $\p$
that constructs an $\LnALC$-complete and clash-free completion set for $\p$.
%If an execution of the
%$\LnALC$ tableau algorithm for $\p$ constructs an $\LnALC$-complete and clash-free completion set for $\p$, then $\p$ is satisfiable.
%On the other hand, if all of its executions return a completion set for $\p$ that contain a clash, then $\p$ is unsatisfiable.
This non-deterministic algorithm
gives priority to non-generating $\LnALC$-rules,
i.e., those that do not introduce new variables or labels,
with respect to generating ones,
so to minimise the size of the completion set constructed by its application,
%and  it is sound and complete with respect to satisfiability in $\LnALC$.
%This algorithm 
and
terminates in exponential time 
for every formula $\p$.
Thus, we obtain the following.
%\nb{M: todo fix notation in Preliminaries}
%The $\LnALC$ tableau algorithm for $\p$ returns $\mathsf{satisfiable}$ if and only if %the procedure 
%%$exp(\T_{\p})$
%it constructs 
% an $\LnALC$-complete  and clash-free completion set for $\p$.
%This non-deterministic algorithm
%terminates in exponential time 
%for every formula $\p$,
%and  it is sound and complete with respect to satisfiability in $\LnALC$.
%Thus, we obtain the following.
%%\nb{M: todo fix notation in Preliminaries}

%\todo{T:All claims of soundness, completeness and termination disappeared. Is this on purpose? O: We say that in the text already}

\begin{theorem}
	\label{thm:upperbound}
	Satisfiability in $\LnALC$  on varying domain neighbourhood models is decidable in $\NExpTime$.
\end{theorem}

%%%%%%%%%%%%% 
%In the rest of this section,
%we prove that this non-deterministic algorithm
%terminates in exponential time 
%for every formula $\p$,
%and that it is sound and complete with respect to $\LnALC$ formula satisfiability.
%We start by establishing termination of the $\LnALC$ tableau algorithm.
%%%%%%%%%%%%
%For each $\Lvar\in \Log$,
%we now present
%a
%tableau-based
%{\color{blue}{non-deterministic}} algorithm {\color{blue}{$sat(\p)$}} based on $\LnALC$-rules for 
%{\color{blue}{deciding the $\LnALC$ formula satisfiability problem on varying domain neighbourhood models, shown  in Algorithm~\ref{alg:tableau}.}}
%We then prove that the algorithm terminates in exponential time 
%%\todo{O: added "in exponential time". T: thanks, added also ``non-deterministic''.}
%for every formula $\p$,
%and that it is sound and complete with respect to $\LnALC$ formula satisfiability.


%% OLD FORMULATION OF THE TABLEAU ALGORITHM FOR THE FORMULA SATISFIABILITY PROBLEM
%{\color{blue}{
%\begin{definition}[$\LnALC$ tableau algorithm for $\p$]
%Given an $\MLnALC$ formula $\p$, the \emph{$\LnALC$ tableau algorithm for $\p$} 
%starts with the initial completion set $\T_{\p} =  \{0 : \p\}$,
%and expands $\T_{\p}$ by means of the procedure $sat(\T_{\p})$ below.
%\todo{M: to discuss}
%
%\begin{algorithm}
%%  \KwIn{The initial completion set $\T_{\p} =  \{0 : \p\}$.
%%}
%%  \KwOut{``no'' if $\p$ is not $\LnALC$-satisfiable, 
%%  an $\LnALC$-complete completion set otherwise.}
%  \KwIn{a completion set $\T$.
%}
%%  \KwOut{``no'' if $\T$ is not $\LnALC$-satisfiable, 
%%  an $\LnALC$-complete completion set $\T' \supseteq \T$ otherwise.}
%  \KwOut{$\mathsf{satisfiable}$, if $\T$ is $\LnALC$-satisfiable;\newline$\mathsf{unsatisfiable}$, otherwise.}
%  \BlankLine
%  \uIf{$\T$ contains a clash}{return $\mathsf{unsatisfiable}$\;} 
%%  \uElseIf{a non-generating rule $\mathsf{R}$ is applicable to $\T$}{apply $\mathsf{R}$ to $\T$\;
%%   return $sat(\T)$\;}
%    \uElseIf{a rule $\mathsf{R} \in\{\mathsf{R}_{\land}, \mathsf{R}_{\lor}, \mathsf{R}_{\sqcap}, \mathsf{R}_{\sqcup}, \mathsf{R}_{\forall}, \mathsf{R}_{\sqsubseteq}, \mathsf{R}_{\mathit{L}\mathbf{T}} \}$ is applicable to $\T$}{apply $\mathsf{R}$ to $\T$\;
%    return $sat(\T)$\;}
%%    \uElseIf{a (modal or non-modal) generating rule $\mathsf{R}$ is applicable to $\T$}{apply $\mathsf{R}$ to $\T$\;
%%  return $sat(\T)$\;}
%      \uElseIf{a rule $\mathsf{R} \in\{\mathsf{R}_{\exists},  \mathsf{R}_{\mathit{L}}, \mathsf{R}_{\mathit{L}\mathbf{N}}, \mathsf{R}_{\mathit{L}\mathbf{P}}, \mathsf{R}_{\mathit{L}\mathbf{Q}}, \mathsf{R}_{\mathit{L}\mathbf{D}} \}$ is applicable to $\T$}{apply $\mathsf{R}$ to $\T$\;
%    return $sat(\T)$\;}
%  \Else{return $\mathsf{satisfiable}$\;}
%%\caption{$\LnALC$ tableau algorithm for $\p$}
%\caption{Decision procedure $sat(\T)$ for $\LnALC$ {\color{red}{formula (T:?)}} satisfiability problem on varying domain neighbourhood models}
%\label{alg:tableau}
%\end{algorithm}
%\end{definition}
%}}
%
%{\color{blue}{
%The algorithm gives priority to non-generating rules,
%that is those that do not introduce new variables or labels,
%with respect to generating ones,
%in order to minimize the size of the completion set constructed by its application.
%Observe that %Algorithm~\ref{alg:tableau} 
%the $\LnALC$ tableau algorithm for $\p$ returns $\mathsf{satisfiable}$ if and only if the procedure constructs 
% an $\LnALC$-complete  and clash-free completion set for $\p$.
%}}



%
%{\color{blue}{
%\begin{definition}[$\LnALC$ tableau algorithm for $\p$]
%Given an $\MLnALC$ formula $\p$, the \emph{$\LnALC$ tableau algorithm for $\p$} 
%%starts calling
%calls the sub-procedure $exp(\T_{\p})$ on the 
%initial completion set $\T_{\p} =  \{0 : \p\}$,
%%and expands $\T_{\p}$ by means of the procedure $sat(\T_{\p})$ below.
%%then
%and it returns $\mathsf{satisfiable}$ if
%$exp(\T_{\p})$ returns $\mathsf{completable}$,
%$\mathsf{unsatisfiable}$ otherwise.
%%and it returns $\mathsf{unsatisfiable}$ if
%%$exp(\T_{\p})$ returns $\mathsf{uncompletable}$.
%%\todo{M: to discuss}
%}}


\begin{algorithm}[t]
  \KwIn{the initial completion set $\T := \{0 : \p\}$ of an $\MLALC{n}$ formula $\p$ in NNF.}
%  \KwIn{an $\MLALC{n}$ formula $\p$ in NNF.}
  \KwOut{a completion set for $\p$, extending the initial one, that either contains a clash, or is complete and clash-free.}
%  \KwOut{$\mathsf{satisfiable}$, if $\p$ is $\LnALC$-satisfiable;\newline$\mathsf{unsatisfiable}$, otherwise.}
%  \BlankLine
%  $\T := \{0 : \p\}$\;
    \BlankLine
  \While{$\T$ is clash-free and not $\LnALC$-complete}{
  	\uIf{a rule $\mathsf{R} \in\{\mathsf{R}_{\land}, \mathsf{R}_{\lor}, \mathsf{R}_{\sqcap}, \mathsf{R}_{\sqcup}, \mathsf{R}_{\forall}, \mathsf{R}_{\sqsubseteq}, \mathsf{R}_{\mathit{L}\mathbf{T}} \}$ is applicable to $\T$}{apply $\mathsf{R}$ to $\T$\;}
      \uElseIf{a rule $\mathsf{R} \in\{\mathsf{R}_{\exists},  \mathsf{R}_{\mathit{L}}, \mathsf{R}_{\mathit{L}\mathbf{N}}, \mathsf{R}_{\mathit{L}\mathbf{P}}, \mathsf{R}_{\mathit{L}\mathbf{Q}}, \mathsf{R}_{\mathit{L}\mathbf{D}} \}$ is applicable to $\T$}{apply $\mathsf{R}$ to $\T$\;}
    }
%    \Return $\T$
%      \BlankLine
%  \uIf{$\T$ contains a clash}{\Return $\mathsf{unsatisfiable}$\;} 
%  \Else{\Return $\mathsf{satisfiable}$\;}
%\caption{$\LnALC$ tableau algorithm for $\p$}
\caption{
$\LnALC$ tableau algorithm
on varying domain neighbourhood models
for $\p$
%Tableau-based decision procedure for the $\LnALC$ formula satisfiability problem on varying domain neighbourhood models
 }
\label{alg:tableau}
\end{algorithm}









%In the rest of this section, we prove termination, soundness and completeness of the tableau algorithm given above. \todo{O: replace "above" with a reference to the algorithm?. T: thanks, fixed.}
%We start by showing that the $\LnALC$ tableau algorithm terminates.
%
%
%
%%% TERMINATION
%

%\newpage
%
%\subsection*{\textcolor{red}{Tentative Termination}}
%Let $\mathbf{T}$ be a completion set 
%constructed
%expanding the initial completion set $\mathbf{T}_{\p} =  \{0 : \p, 0 : \top(x) \}$ according to the 
%$\LnALC$ tableau algorithm for $\p$.
%%\textcolor{red}{We define $\mathsf{L}_{\mathbf{T}} = \{ n \in \mathsf{N_{L}} \mid S_{n} \subseteq \mathbf{T} \}$.}
%We prove the following two claims.




As an immediate consequence of the %above results 
correctness of the tableau we also obtain 
a (constructive) proof of the following kind of \emph{exponential model property}.
%A (constructive) proof of the following kind of \emph{exponential model property} is an immediate consequence of the above correctness result.
%\begin{corollary}
%For each $\mathit{L} \in \Log$,
%every $\LnALC$-satisfiable formula $\p$ has a model of at most exponential size in the length of $\p$.
%\end{corollary}
%\begin{corollary}
%For each $\mathit{L} \in \{\mathbf{E},\mathbf{M},\mathbf{N}\}$,
%every $\LnALC$-satisfiable formula $\p$ has a model 
%%of at most exponential size in the length of $\p$.
%with at most $p(|\fg(\p)|)$ worlds, each of them with a domain of at most 
%$2^{q(|\fg(\p)|)}$ elements, where $p$ and $q$ are polynomial functions.
%Moreover, for $\mathit{L} = \mathbf{C}$,
%every $\LnALC$-satisfiable formula $\p$ has a model 
%with at most $2^{p(|\fg(\p)|)}$ worlds, each of them with a domain of at most 
%$2^{q(|\fg(\p)|)}$ elements, where $p$ and $q$ are polynomial functions.
%\end{corollary}


%% Old corollary
%\begin{corollary}
%For $\mathit{L} \in \{\mathbf{E},\mathbf{M},\mathbf{N}\}$
%(respectively, $\mathit{L} = \mathbf{C}$),
%every $\LnALC$ satisfiable formula $\p$ has a model 
%%of at most exponential size in the length of $\p$.
%with at most $p(|\fg(\p)|)$ (respectively, at most $2^{p(|\fg(\p)|)}$) worlds, each of them having a domain with at most 
%$2^{q(|\fg(\p)|)}$ elements, where $p$ and $q$ are polynomial functions.
%\end{corollary}
%\begin{proof}
%By Theorem \ref{thm:completeness}, if $\p$ is $\LnALC$ satisfiable, then 
%there is a $\LnALC$-complete and clash-free completion set $\mathbf{T}$ for it.
%Then by Theorem \ref{thm:soundness},
%%basing on $\mathbf{T}$ we can define a
%there exists a model 
%$\Mmc = (\Wmc, \{ \Nmc_{i} \}_{i \in I}, \Imc)$
%for $\p$ where $\Wmc =  \mathsf{L}_{\mathbf{T}}$
%and for each $n\in\Wmc$, $\Delta_{n} = \{ x \in \mathsf{N_{V}} \mid x \ \text{occurs in} \ S_{n} \}$.
%By Theorem~\ref{thm:termination}, Claim~\ref{cla:termglobal}, it follows
%$|\Wmc| \leq |\fg(\p)|^2$ for $\mathit{L} \in \{\mathbf{E}, \mathbf{M}, \mathbf{N}\}$,
%and 
%$|\Wmc| \leq 2^{|\fg(\p)|} \cdot |\fg(\p)|$ for $\mathit{L} = \mathbf{C}$,
%finally by Theorem~\ref{thm:termination}, Claim~\ref{cla:termlocal}, 
%for each $n\in\Wmc$, $|\Delta_n|$ does not exceed $2^{q(|\fg(\p)|)}$,
%where $p$ and $q$ are polynomial functions.
%\end{proof}

%{\color{blue}{
\begin{restatable}{corollary}{Fmp}
Every $\LnALC$ satisfiable formula $\p$ has a model 
with at most $p(|\fg(\p)|)$ worlds,
if $\mathbf{C}\notin\Lvar$, 
and at most $2^{q(|\fg(\p)|)}$) worlds, 
if $\mathbf{C}\in\Lvar$,
each of them having a domain with at most 
$2^{r(|\fg(\p)|)}$ elements, with $p$, $q$, $r$ polynomial functions.
\end{restatable}
%
%}}

%where we set:
%		\begin{itemize}
%			\item for $\mathit{L} = \mathbf{E}$, $k = l = 1$;
%			\item for $\mathit{L} = \mathbf{M}$, $k = 1$, $l = 0$;
%			\item for $\mathit{L} = \mathbf{C}$, $k \geq 1$, $l = k$;
%			\item for $\mathit{L} = \mathbf{N}$, $k = l = 0$.
%		\end{itemize}


%\paragraph{Local expansion rules for $\LnALC$}
%
%\begin{itemize}
%	\item[$\mathsf{R}_{\land}$] If $\psi \land \chi \in S$ and $\{ \psi, \chi \} \not\subseteq S$, then set $S := S \cup \{ \psi, \chi \}$.
%	\item[$\mathsf{R}_{\lor}$] If $\psi \lor \chi \in S$ and $\{ \psi, \chi \} \cap S = \emptyset$, then set $S := S \cup \{ \vartheta \}$, where $\vartheta = \psi$ or $\vartheta = \chi$.
%	\item[$\mathsf{R}_{\sqcap}$] If $x : C \sqcap D \in S$ and $\{ x: C, x: D \} \not\subseteq S$, then set $S := S \cup \{ x: C, x: D \}$.
%	\item[$\mathsf{R}_{\sqcup}$] If $x: C \sqcup D \in S$ and $\{  x: C, x: D \} \cap S = \emptyset$, then set $S := S \cup \{ x : E \}$, where $E = C$ or $E = D$.
%	\item[$\mathsf{R}_{\exists}$] If $x : \exists r.C \in S$, $x$ is not blocked with respect to $S$, and there is no $r$-successor $y$ of $x$ with respect to $S$ such that $y : C \in S$, then
%	choose a fresh $y$ for $S$ and set $S := S \cup \{ (x,y) : r, y: C \}$.
%	\item[$\mathsf{R}_{\forall}$] If $x: \forall r.C$ and there is an $r$-successor $y$ of $x$ with respect to $S$ such that $y: C \not \in S$,
%	then set $S := S \cup \{ y : C \}$.
%	\item[$\mathsf{R}_{=}$] If $\top \sqsubseteq C \in S$ and $x: C \not \in S$, for a variable $x$ that occurs in $S$, then set $S := S \cup \{ x: C \}$.
%	\item[$\mathsf{R}_{\neq}$]  If $\lnot (\top \sqsubseteq C) \in S$ and there is no variable $x$ such that $x: \dot{\lnot}C \in S$, for a variable $x$ that occurs in $S$, then choose a fresh variable $x$ for $S$ and set $S := S \cup \{ x : \dot{\lnot}C \}$.
%\end{itemize}
%
%\paragraph{Global expansion rules for $\LnALC$}
%
%\begin{itemize}
%		\item[$\mathsf{R}_{\mathit{L}f}$]
%		If $\{ \Box_{i} \psi_{1}, \ldots, \Box_{i} \psi_{n}, \Diamond_{i} \chi \} \subseteq S$, and,
%			for every $S' \in \mathbf{T}$, % such that $S \neq S'$,
%			all of the following hold:
%				\begin{itemize}
%					\item[($0$)] $\{ \psi_{1}, \ldots, \psi_{n}, \chi \} \not \subseteq S'$; and
%					\item[($1$)] $\{ \lnot \psi_{1}, \lnot \chi \} \not \subseteq S'$; and 
%					\item[$\vdots$]
%					\item[($m$)]   $\{ \lnot \psi_{n}, \lnot \chi \} \not \subseteq S'$;
%				\end{itemize}			
%then choose a fresh $x$ for $S$, set new $S'$ such that:  
%				\begin{itemize}
%					\item[($0$)] $S' = \{ \psi_{1}, \ldots, \psi_{n}, \chi, x: \top \}$; or
%					\item[($1$)] $S' = \{ \lnot \psi_{1}, \lnot \chi, x: \top \}$; or
%					\item[$\vdots$]
%					\item[($m$)]  $S' = \{  \lnot \psi_{m}, \lnot \chi, x: \top \}$;
%				\end{itemize}
%and set $\mathbf{T} := \mathbf{T} \cup \{ S' \}$.
%
%
%
%	\item[$\mathsf{R}_{\mathit{L}c}$]
%If $\{ x:  \Box_{i} C_{1}, \ldots,  x: \Box_{i} C_{n}, y: \Diamond_{i} D \} \subseteq S$, and,
%			for every $S' \in \mathbf{T}$,
%			all of the following hold:
%				\begin{itemize}
%					\item[($0$)] $\{ x: C_{1}, \ldots, x: C_{n}, y: D \} \not \subseteq S'$; and
%					\item[($1$)] $\{ x: \lnot C_{1}, y: \lnot D \} \not \subseteq S'$; and 
%					\item[$\vdots$]
%					\item[($m$)]  $\{ x: \lnot C_{m}, y: \lnot D \} \not \subseteq S'$;
%				\end{itemize}			
%then choose a fresh $z$ for $S$, set new $S'$ such that:  
%				\begin{itemize}
%					\item[($0$)] $S' = \{ x: C_{1}, \ldots, x: C_{n}, y: D, z: \top \}$; or
%					\item[($1$)] $S' = \{ x: \lnot C_{1}, y: \lnot D, z: \top \}$; or
%					\item[$\vdots$]
%					\item[($m$)]  $S' = \{ x: \lnot C_{m}, y: \lnot D, z: \top \}$;
%				\end{itemize}	
%and set $\mathbf{T} := \mathbf{T} \cup \{ S' \}$.
%\end{itemize}
%
%where we set:
%		\begin{itemize}
%			\item for $\mathit{L} = \mathbf{E}$, $n = m = 1$;
%			\item for $\mathit{L} = \mathbf{M}$, $n = 1$, $m = 0$;
%			\item for $\mathit{L} = \mathbf{C}$, $n \geq 1$, $m = n$;
%			\item for $\mathit{L} = \mathbf{N}$, $n = m = 0$.
%		\end{itemize}
%
%
%
%
%\begin{itemize}
%%	\item[$\mathsf{R}_{\mathit{L}f}$]
%%		\begin{itemize}
%%		
%%			\item $\mathit{L} = \mathbf{E}$
%%			
%%			If $\{ \Box_{i} \psi, \Diamond_{i} \chi \} \subseteq S$ and both $\{ \psi, \chi \} \not \subseteq S'$ and $\{\lnot \psi, \lnot \chi \} \not \subseteq S'$, for every $S' \in \mathbf{T}$ such that $S \neq S'$,
%%			then choose a fresh $x$ for $S$, set new $S' = \{ \psi, \chi, x: \top \}$ and $S'' = \{ \lnot \psi, \lnot \chi, x : \top \}$, and set $\mathbf{T} := \mathbf{T} \cup \{ S^* \}$, where $S^* = S'$ or $S^* = S''$.
%%			
%%			\item $\mathit{L} = \mathbf{M}$
%%			
%%			If $\{ \Box_{i} \psi, \Diamond_{i} \chi \} \subseteq S$ and $\{ \psi, \chi \} \not \subseteq S'$, for every $S' \in \mathbf{T}$ such that $S \neq S'$,
%%			then choose a fresh $x$ for $S$, set new $S' = \{ \psi, \chi, x: \top \}$, and set $\mathbf{T} := \mathbf{T} \cup \{ S' \}$.
%%			
%%			\item $\mathit{L} = \mathbf{C}$
%%			
%%			For every $n \geq 1$, if $\{ \Box_{i} \psi_{1}, \ldots, \Box_{i} \psi_{n}, \Diamond_{i} \chi \} \subseteq S$, and,
%%			for every $S' \in \mathbf{T}$ such that $S \neq S'$,
%%			all of the following hold:
%%				\begin{itemize}
%%					\item $\{ \psi_{1}, \ldots, \psi_{n}, \chi \} \not \subseteq S'$; and
%%					\item $\{ \lnot \psi_{1}, \lnot \chi \} \not \subseteq S'$; and \ldots; and $\{ \lnot \psi_{n}, \lnot \chi \} \not \subseteq S'$;
%%				\end{itemize}			
%%then choose a fresh $x$ for $S$, set nen :  
%%				\begin{itemize}
%%					\item $S' = \{ \psi_{1}, \ldots, \psi_{n}, \chi, x: \top \}$;
%%					\item $S''_{1} = \{ \lnot \psi_{1}, \lnot \chi, x: \top \}$; \ldots; $S''_{n} = \{  \lnot \psi_{n}, \lnot \chi, x: \top \}$;
%%				\end{itemize}
%%and set $\mathbf{T} := \mathbf{T} \cup \{ S^* \}$, where $S^* = S'$ or $S^* = S''_{i}$, for some $i \in \{ 1, \ldots, n \}$.
%%
%%			\item $\mathit{L} = \mathbf{N}$
%%			
%%			If $\Box_{i} (\top \sqsubseteq \top) \not\in S$,
%%			then set $S := S \cup \{ \Box_{i} (\top \sqsubseteq \top) \}$.					
%%%			If $\Box_{i} (\top \sqsubseteq \top) \not\in S$,
%%%			then choose a fresh $x$ for $S$, set \nb{M: to discuss} new $S' = \{ \Box_{i} (\top \sqsubseteq \top), x: \top \}$, and set $\mathbf{T} := \mathbf{T} \cup \{ S' \}$.			
%%			
%%			\item $\mathit{L} = \mathbf{MC}$
%%			
%%			For every $n \geq 1$, if $\{ \Box_{i} \psi_{1}, \ldots, \Box_{i} \psi_{n}, \Diamond_{i} \chi \} \subseteq S$, and
%%			$\{ \psi_{1}, \ldots, \psi_{n}, \chi \} \not \subseteq S'$,
%%			for every $S' \in \mathbf{T}$ such that $S \neq S'$,
%%then choose a fresh $x$ for $S$, set new $S' = \{ \psi_{1}, \ldots, \psi_{n}, \chi, x: \top \}$, and set $\mathbf{T} := \mathbf{T} \cup \{ S' \}$.
%%			\item $\mathit{L} = \mathbf{MN}$
%%			
%%			Both $\mathsf{R}_{\mathit{L}f}$ rules for $\mathit{L} = \mathbf{M}$ and $\mathit{L} = \mathbf{N}$.
%%			
%%			\item $\mathit{L} = \mathbf{CN}$
%%			
%%			Both $\mathsf{R}_{\mathit{L}f}$ rules for $\mathit{L} = \mathbf{C}$ and $\mathit{L} = \mathbf{N}$.
%%			
%%		\end{itemize}
%%		
%%	\item[$\mathsf{R}_{\mathit{L}c}$]
%%	
%%		\begin{itemize}
%%			\item $\mathit{L} = \mathbf{E}$
%%			
%%			If $\{ x: \Box_{i} C, y : \Diamond_{i} D \} \subseteq S$ and both $\{ x' : C, y' : D \} \not \subseteq S'$ and $\{ x' : \lnot C, y' : \lnot D \} \not \subseteq S'$, for every $S' \in \mathbf{T}$ such that $S \neq S'$,
%%			then choose a fresh $z$ for $S$, set new $S' = \{ x : C, y : D, z: \top \}$ and $S'' = \{ x : \lnot C, y : \lnot D, z: \top \}$, and set $\mathbf{T} := \mathbf{T} \cup \{ S^* \}$, where $S^* = S'$ or $S^* = S''$.
%%			
%%			\item $\mathit{L} = \mathbf{M}$
%%			
%%			If $\{ x: \Box_{i} C, y : \Diamond_{i} D \} \subseteq S$ and $\{ x' : C, y' : D \} \not \subseteq S'$, for every $S' \in \mathbf{T}$ such that $S \neq S'$,
%%			then choose a fresh $z$ for $S$, set new $S' = \{ x : C, y : D, z: \top \}$, and set $\mathbf{T} := \mathbf{T} \cup \{ S' \}$.
%%			
%%			\item $\mathit{L} = \mathbf{C}$
%%			
%%			For every $n \geq 1$, if $\{ x:  \Box_{i} C_{1}, \ldots,  x: \Box_{i} C_{n}, y: \Diamond_{i} D \} \subseteq S$, and,
%%			for every $S' \in \mathbf{T}$ such that $S \neq S'$,
%%			all of the following hold:
%%				\begin{itemize}
%%					\item $\{ x: C_{1}, \ldots, x: C_{n}, y: D \} \not \subseteq S'$; and
%%					\item $\{ x: \lnot C_{1}, y: \lnot D \} \not \subseteq S'$; and \ldots; and $\{ x: \lnot C_{n}, y: \lnot D \} \not \subseteq S'$;
%%				\end{itemize}			
%%then choose a fresh $z$ for $S$, set nen :  
%%				\begin{itemize}
%%					\item $S' = \{ x: C_{1}, \ldots, x: C_{n}, y: D, z: \top \}$;
%%					\item $S'_{1} = \{ x: \lnot C_{1}, y: \lnot D, z: \top \}$; \ldots; $S'_{n} = \{ x: \lnot C_{n}, y: \lnot D, z: \top \}$;
%%				\end{itemize}
%%and set $\mathbf{T} := \mathbf{T} \cup \{ S^* \}$, where $S^* = S'$ or $S^* = S''_{i}$, for some $i \in \{ 1, \ldots, n \}$.
%%			
%%			\item $\mathit{L} = \mathbf{N}$
%%
%%			If $x: \Box_{i} \top \not\in S$,
%%			then set $S := S \cup \{ x: \Box_{i} \top, \}$.				
%%%			If $x: \Box_{i} \top \not\in S$,
%%%			then choose a fresh $z$ for $S$, set \nb{M: to discuss} new $S' = \{ x: \Box_{i} \top, z: \top \}$, and set $\mathbf{T} := \mathbf{T} \cup \{ S' \}$.			
%%			
%%			\item $\mathit{L} = \mathbf{MC}$
%%			
%%			For every $n \geq 1$, if $\{ x:  \Box_{i} C_{1}, \ldots,  x: \Box_{i} C_{n}, y: \Diamond_{i} D \} \subseteq S$, and $\{ x: C_{1}, \ldots, x: C_{n}, y: D \} \not \subseteq S'$,
%%			for every $S' \in \mathbf{T}$ such that $S \neq S'$,	
%%then choose a fresh $z$ for $S$, set new $S' = \{ x: C_{1}, \ldots, x: C_{n}, y: D, z: \top \}$,
%%and set $\mathbf{T} := \mathbf{T} \cup \{ S' \}$.
%%			
%%			\item $\mathit{L} = \mathbf{MN}$
%%			
%%						Both $\mathsf{R}_{\mathit{L}c}$ rules for $\mathit{L} = \mathbf{M}$ and $\mathit{L} = \mathbf{N}$.
%%			
%%			\item $\mathit{L} = \mathbf{CN}$
%%			
%%			Both $\mathsf{R}_{\mathit{L}c}$ rules for $\mathit{L} = \mathbf{C}$ and $\mathit{L} = \mathbf{N}$.
%%			
%%		\end{itemize}
%	\item[$\mathsf{R}_{\mathit{L}fc}$]
%		\begin{itemize}
%			\item $\mathit{L} = \mathbf{E}$ \ldots\nb{M: to add}
%			\item $\mathit{L} = \mathbf{M}$ \ldots
%			\item $\mathit{L} = \mathbf{C}$ \ldots
%			\item $\mathit{L} = \mathbf{N}$ \ldots
%			\item $\mathit{L} = \mathbf{MC}$ \ldots
%			\item $\mathit{L} = \mathbf{MN}$ \ldots
%			\item $\mathit{L} = \mathbf{CN}$ \ldots
%		\end{itemize}
%\end{itemize}



%%%%%%%%%%%%%%%%%%%%%%%%%%%%%%%
\section{Fragments without Modalised Concepts}
%\subsection{Fragments without Modalised Concepts on Varying Domain}
\label{sec:fragvardom}
%\subsection{$\LnALCg$ on Varying Domain}
%%%%%%%%%%%%%%%%%%%%%%%%%%%%%%%
Here we study fragments of $\MLnALC$ without modalised concepts.
An \emph{$\MLnALCg$ formula} is defined similarly to the $\MLnALC$ case,
%but inductively built from $\ALC$ \emph{atoms} consisting of CIs (cf.~Section~\ref{sec:prelim}), or \emph{assertions} of the form $A(a)$ and $r(a,b)$, where $A \in \NC$, $r \in \NR$, and $a,b \in \NI$,
%and
by disallowing modalised concepts.
%
Given
$\mathit{L} \in \Log$,
%$\mathit{L} \in \{ \mathbf{E}, \mathbf{M}, \mathbf{C}, \mathbf{N} \}$,
 \emph{satisfiability in $\LnALCg$ on varying} %(respectively, \emph{constant})
  \emph{domain neighbourhood models} is $\MLnALCg$ satisfiability %problem 
  on varying %(respectively, constant) 
  domain neighbourhood models based on neighbourhood frames in the respective class for $\mathit{L}$.
%(cf. Section~\ref{sec:prelim}).
%
%(i.e., built from $\ALC$ concepts and CIs only).
An \emph{$\MLn$ formula}, instead, is defined analogously to $\MLnALCg$, except that we build it from the standard propositional (rather than $\ALC$) language over a countable set of \emph{propositional letters} $\mathsf{N_{P}}$, disjoint from \NC, \NR, and \NI.
%consider only propositional logic (instead of \ALC).
The semantics of
$\MLn$ formulas
is given in terms of \emph{propositional 
neighbourhood models} (or simply \emph{models}) $\Mmc^{\sf P} = (\Wmc, \{ \Nmc_{i} \}_{i \in J}, \Vmc)$,
where $(\Wmc, \{ \Nmc_{i} \}_{i \in J})$ is a neighbourhood frame,
with $J = \{ 1, \ldots, n \}$ in the following,
and $\Vmc: \NPr \rightarrow 2^{\Wmc}$ is a function 
mapping propositional letters
%in $\NPr$
to
sets of worlds
%subsets of the domain of worlds
(see~\cite{Che,Var2}).
%We say that a propositional 
%neighbourhood model is a $\EC^{n}$ or $\EN^{n}$ \emph{model} if it is based on a neighbourhood frame satisfying the corresponding conditions for $\EC^{n}$ and $\EN^{n}$ given in Section~\ref{sec:prelim}, respectively.
\emph{Satisfiability in $\ensuremath{\smash{\mathit{L}^{n}}}$} is satisfiability in $\MLn$ on propositional neighbourhood models based on neighbourhood frames in the respective class for $\mathit{L}$.
A propositional neighbourhood model based on a neighbourhood frame in the respective class for $\mathit{L}$ is called \emph{$\mathit{L}^{n}$ model}.

\newcommand{\setsymbols}{\ensuremath{\Sigma}\xspace}


We
prove
%establish
tight complexity results %$\ExpTime$ upper bounds 
for  %$\EALCg$  and $\MALCg$ 
$\LnALCg$ satisfiability on varying domain neighbourhood models, where $L\in \Log$,
%\todo{Patheon?},  
using the notion of 
a propositional abstraction of a formula 
(as in, e.g.,~\cite{Baader:2012:LOD:2287718.2287721}).
%Since $\ALC$ formula satisfiability is already $\ExpTime$-hard, 
%our upper bounds here are tight. 
%we have 
%a tight complexity result for the global cases.
%We show that
Here, one can separate the satisfiability test into two parts, 
one for the description logic dimension and 
one for the \neighborhood frame dimension.
%~ In this subsection, we also consider the lightweight DL called \EL, defined 
%~ as the fragment of \ALC which only allows conjunctions and existential quantification 
%~ in concept expressions. 
%
%Consider $\Lmc\in\{\ALC,\EL\}$. 
%\newcommand{\propmodel}{\ensuremath{M}\xspace}
%\newcommand{\propdomain}{\ensuremath{W}\xspace}
%\newcommand{\propneigh}{\ensuremath{N}\xspace}
%\newcommand{\propassign}{\ensuremath{I}\xspace}
For an $\MLnALCg$ formula $\varphi$, the 
\emph{propositional abstraction} $\prop{\varphi}$ is  
the result of replacing each \ALC atom $\pi$ in $\varphi$ by 
a propositional variable $p_{\pi} \in \NPr$.
%so that there is a one-to-one relationship between the \ALC atoms $\elaxiom$ occurring in $\varphi$ and the 
%propositional letters $p_{\elaxiom}$ used for the abstraction.  
%Let \NPr be a countably infinite set of propositional symbols disjoint from \NC, \NR, and \NI. \nb{O: added}
%We define
Define the set $\setsymbols_\varphi = \{p_{\elaxiom}\in\NPr\mid \elaxiom \text{ is an \ALC atom in }
\varphi \}$.
%The semantics of $\prop{\varphi}$ is given in terms of \emph{propositional neighbourhood models}
%$(\Wmc, \{ \Nmc_{i} \}_{i \in J}, \Vmc)$ for $L^{n}$, 
%where $(\Wmc, \{ \Nmc_{i} \}_{i \in J})$ is a neighbourhood frame, with $J = \{ 1, \ldots, n \}$,
%and $\Vmc: \NPr \rightarrow \Pmc(\Wmc)$ is a function 
%mapping propositional variables in $\NPr$ to
%sets of worlds
%%subsets of the domain of worlds
%(see~\cite{Che,Var2}). 
%
 %Given an $\MLnALCg$ formula $\varphi$, 
%and we say that
A (propositional neighbourhood) $L^{n}$ model 
$\propmodel = (\Wmc, \{ \Nmc_{i} \}_{i \in J}, \Vmc)$
%of $\prop{\varphi}$
%, defined as the set of variables 
%evaluated to true in the model, 
is \emph{$\setsymbols_\varphi$-consistent}
if,
for all $w\in \Wmc$,
the following \ALC formula is satisfiable:
$
\alcform = \bigwedge_{p_{\elaxiom}\in \formtp{\varphi}} {\elaxiom} \ \wedge \bigwedge_{p_{\elaxiom} \in
	\setsymbols_\varphi \setminus \formtp{\varphi}}
% \overline{\NPr(w)}}
\neg {\elaxiom},
$
where
$\formtp{\varphi} = \{p_{\elaxiom} \in \setsymbols_\varphi \mid w\in \Vmc(p_{\elaxiom})\}$.
%
%%%% OLD
%{\color{red}{
%Given an $\EALCg$ formula $\varphi$, we say that a propositional neighbourhood model 
%$\propmodel = (\Wmc, \{ \Nmc_{i} \}_{i \in J}, \Vmc)$
%of $\prop{\varphi}$
%%, defined as the set of variables 
%%evaluated to true in the model, 
%is \emph{$\consistent{\varphi}$}
%if, for all $w\in \Wmc$,
%the following \ALC formula is satisfiable $$\textstyle\bigwedge_{p_{\elaxiom}\in \NPr(w)} \ {\elaxiom} \ \wedge \
%\bigwedge_{p_{\elaxiom}\in \overline{\NPr(w)}}\ \neg {\elaxiom}$$
%where $\NPr(w) = \{p_{\elaxiom}\in \NPr(\varphi) \mid w\in \Vmc(p_{\elaxiom})\}$
%and $\overline{\NPr(w)}=\NPr(\varphi)\setminus\NPr(w)$. 
%}}
%We are now ready for proving Lemma~\ref{lem:propE}.
We  formalise the connection between %complexity of the satisfiability problem for 
$\LnALCg$ satisfiable formulas and their propositional abstractions %with consistent models
with the following lemma.
%, which is an adaptation of the 
%results obtained for other \ALC extensions~\cite{Baader:2012:LOD:2287718.2287721}.
%~ %We say that a model of $\prop{\varphi}$
%~ %is $\varphi$-consistent

%\nb{$\varphi$-consistent model}

\begin{restatable}{lemma}{LemmapropL}\label{lem:propL}
A
%$\MLnALCg$
formula $\varphi$ is
$\LnALCg$
satisfiable
on varying domain neighbourhood models
iff
%if, and only if, 
$\prop{\varphi}$ is satisfied in a $\setsymbols_\varphi$-consistent $L^{n}$ model.  
\end{restatable}
%




%\todo[inline]{M: satisfiability without modalised concepts in varying domains? for all systems?}


 
We now introduce definitions and notation used to prove our complexity result on fragments without modalised concepts.
Let $\setsymbols = \{p_{\elaxiom}\in\NPr\mid \elaxiom \text{ is an \ALC }$ $\text{atom in }\varphi 
 \}$,  for a fixed but arbitrary  $\MLnALCg$ formula $\varphi$,
and let $\phi$ be an $\MLn$ formula built from symbols in $\setsymbols$.
We
denote by  ${\sf sub}(\phi)$ the set 
of subformulas of $\phi$ closed under single negation.  
A \emph{valuation} for $\phi$ %and $\setsymbols$   
is a function $\nu: {\sf sub} (\phi)\rightarrow \{0,1\}$ that 
satisfies the   conditions:
(1) for all $\neg\psi\in {\sf sub} (\phi)$,
$\nu(\psi)=1$ iff $\nu(\neg\psi)=0$;
(2) for all $\psi_1\wedge \psi_2\in {\sf sub} (\phi)$,
$\nu(\psi_1\wedge \psi_2) = 1$ iff $\nu(\psi_1) = 1$
and $\nu(\psi_2) = 1$; 
and (3) $\nu(\phi)=1$. 
A valuation $\nu$ for $\phi$
is \emph{$\setsymbols$-consistent}
if
the following \ALC formula is satisfiable:
$
\textstyle\bigwedge_{\nu(p_\elaxiom)=1} \ {\elaxiom} \ \wedge \
\bigwedge_{\nu(p_\elaxiom)=0}\ \neg {\elaxiom},
$
where $p_\elaxiom\in\setsymbols$.  
%
Lemma~\ref{lem:proplemmaL} establishes that satisfiability of 
$\phi$ in a $\setsymbols$-consistent model is characterised  by the existence of a $\setsymbols$-consistent valuation satisfying suitable properties.
In the following, we use $\falseprop$ as an abbreviation for $p \land \neg p$, for a fixed but arbitrary   $p \in \mathsf{N_{P}}$. %\nb{O: moved closer to where used. T: perfect, thank you}


%\todo{on going}


 
 


\begin{restatable}{lemma}{Lemmapropvardi}\label{lem:proplemmaL}
%\nb{M: improved layout}
	%Let $\setsymbols$ be $\NPr(\varphi)$ for a fixed but arbitrary  $\MLnALCg$ formula $\varphi$
	%and let $\phi$ be a $\MLn$ formula built from symbols in $\setsymbols$.
	Given $\Lvar$ and  an $\MLn$ formula $\phi$ built from symbols in $\setsymbols$ (defined as above), 
	let:
%	$\boldsymbol{\kappa} =
%		| {\sf sub}({\phi}) |$, \text{if $\mathbf{C} \in \Lvar$};
%	$\boldsymbol{\kappa} = 1$, \text{if $\mathbf{C} \not \in \Lvar$}.
	\[
	\boldsymbol{\kappa} =
	\begin{cases}
		| {\sf sub}({\phi}) | , & \text{if $\mathbf{C} \in \Lvar$} \\
		%	\geq 1, & \text{if $\mathbf{C} \in \Lvar$} \\
		1, & \text{if $\mathbf{C} \not \in \Lvar$}
	\end{cases}.
	\]
	A formula $\phi$ is satisfied in a $\setsymbols$-consistent $\Lvar^{n}$ %\nb{check macro}
	model
	iff
	%if, and only if,
	there is   a $\setsymbols$-consistent valuation \valuation 
	for $\phi$ %and $\setsymbols$ 
	such that,
	for every $1 \leq k \leq \boldsymbol{\kappa}$,
	if $\B_i\psi_1, \dots, \B_i\psi_k, \B_i\chi\in{\sf sub}(\phi)$,
	$\valuation(\B_i\psi_j)=1$ for all $1\leq j \leq k$,
	%$\B_i\chi\in{\sf sub}(\prop{\varphi})$, 
	and $\valuation(\B_i\chi)=0$, then
\begin{enumerate}
\item
$
	(\bigwedge^{k}_{j=1}\psi_j\wedge\neg\chi) \vee \boldsymbol\vartheta
$
	is satisfied in a $\setsymbols$-consistent $\Lvar^{n}$ %\nb{check macro}
	model, 
	where:
$\boldsymbol\vartheta = \falseprop$, if $\mathbf{M}\in\Lvar$;
$\boldsymbol\vartheta = \bigvee^{k}_{j=1} (\neg\psi_j\wedge\chi)$, if $\mathbf{M}\not\in\Lvar$;
%	\[
%	\boldsymbol\vartheta =
%	\begin{cases}
%		\falseprop, & \text{if $\mathbf{M}\in\Lvar$} \\
%		\bigvee^{k}_{j=1} (\neg\psi_j\wedge\chi), & \text{if $\mathbf{M}\not\in\Lvar$}
%	\end{cases};
%	\]
	and 
\item
	for $\mathbf{X}\in\{\mathbf{N,T,P,Q,D}\}$,
	if $\mathbf{X}\in\Lvar$, then $\nu$ satisfies the condition $(\mathbf{X})$ below, for every $1 \leq k, h \leq \boldsymbol{\kappa}$:
	\begin{itemize}
		%\item if $\B_i\psi_1, \dots, \B_i\psi_k\in{\sf sub}(\prop{\varphi})$,
		%	$\valuation(\B_i\psi_j)=1$ for all $1\leq j \leq k$,
		% 	$\B_i\chi\in{\sf sub}(\prop{\varphi})$, and $\valuation(\B_i\chi)=0$, then
		%	%$(\bigwedge^{k}_{j=1}\psi_j\wedge\neg\chi) \vee \bigvee^{k}_{j=1} (\neg\psi_j\wedge\chi)$ 
		%	$(\bigwedge^{k}_{j=1}\psi_j\wedge\neg\chi) \vee \mathsf{D}$ 
		%	is satisfied in a $\varphi$-consistent $\Lvar$ %\nb{check macro}
		% 	model;
		%
		\item[($\mathbf{N}$)] if $\B_i\psi\in{\sf sub}(\phi)$ and
		$\valuation(\B_i\psi)=0$, then $\neg \psi$ 
		is satisfied in a $\setsymbols$-consistent $\Lvar^{n}$ model; 
		
		\item[($\mathbf{T}$)] if $\B_i\psi\in{\sf sub}(\phi)$ and 
		$\valuation(\B_i\psi)=1$   then
		$\valuation(\psi)=1$;
		
		\item[($\mathbf{P}$)] if $\B_i\psi_1, \dots, \B_i\psi_k\in{\sf sub}(\phi)$ and
		$\valuation(\B_i\psi_j)=1$ for all $1\leq j \leq k$, 
		then $\bigwedge^{k}_{j=1}\psi_j$
		is satisfied in a $\setsymbols$-consistent $\Lvar^{n}$ model; 
		
		\item[($\mathbf{Q}$)] if $\B_i\psi_1, \dots, \B_i\psi_k\in{\sf sub}(\phi)$ and
		$\valuation(\B_i\psi_j)=1$ for all $1\leq j \leq k$, 
		then $\bigvee^{k}_{j=1}\neg\psi_j$
		is satisfied in a $\setsymbols$-consistent $\Lvar^{n}$ model; 
		
		\item[($\mathbf{D}$)] if $\B_i\psi_1, \dots, \B_i\psi_k, \B_i\chi_1, \dots, \B_i\chi_h\in{\sf sub}(\phi)$,
		$\valuation(\B_i\psi_j)=1$ for all $1\leq j \leq k$, and
		$\valuation(\B_i\chi_\ell)=1$ for all $1\leq \ell \leq h$, 
		%		$\valuation(\B_i\psi_j)=\valuation(\B_i\chi_\ell)=1$ for all $1\leq j \leq k$,  $1\leq \ell \leq h$,
		then $(\bigwedge^{k}_{j=1}\psi_j \land \bigwedge^{h}_{\ell=1}\chi_\ell) \vee \boldsymbol\eta$
		is satisfied in a $\setsymbols$-consistent $\Lvar^{n}$ model,
		where:
%		$\boldsymbol\eta = \falseprop$, if $\mathbf{M}\in\Lvar$; 
%		$\boldsymbol\eta = \neg(\bigwedge^{k}_{j=1}\psi_j) \land \neg(\bigwedge^{h}_{\ell=1}\chi_\ell)$, if $\mathbf{M}\not\in\Lvar$; 
		\[
		\boldsymbol\eta =
		\begin{cases}
			\falseprop, & \text{if $\mathbf{M}\in\Lvar$} \\
			\neg(\bigwedge^{k}_{j=1}\psi_j) \land \neg(\bigwedge^{h}_{\ell=1}\chi_\ell), & \text{if $\mathbf{M}\not\in\Lvar$}
		\end{cases}.
		\]
	\end{itemize}
\end{enumerate}
	%
	%where
	%\begin{itemize}
	%\item
	%			$\mathsf{D} =
	%			\begin{cases}
		%				FALSE, & \text{if $\mathbf{M} \in L$} \\
		%				\bigvee^{k}_{j=1} (\neg\psi_j\wedge\chi), & \text{if $\mathbf{M} \not \in L$}
		%			\end{cases};
	%			$
	%			\item
	%			$\mathsf{k}
	%			\begin{cases}
		%				\geq 1, & \text{if $\mathbf{C} \in L$} \\
		%				= 1, & \text{if $\mathbf{C} \not \in L$}
		%			\end{cases};
	%			$
	%\end{itemize}
\end{restatable}










By using Lemmas~\ref{lem:propL}-\ref{lem:proplemmaL},
the following theorem provides a procedure that runs in exponential time to check $\LnALCg$ satisfiability on varying domains.
Since $\ALC$ formula satisfiability is already $\ExpTime$-hard, our upper bound is tight.








 
 
\begin{restatable}{theorem}{Satfragvardomexp}
\label{thm:satfragvardomexp}
	Satisfiability in $\LnALCg$   on varying domain neighbourhood models is \ExpTime-complete.
\end{restatable}
%




%{\color{red}{
%\todo{M: moved here, to fix}
%In conclusion, by Lemma~\ref{lem:propL}, one can decide $\LnALCg$ satisfiability of a formula $\varphi$ on varying domain neighbourhood models by deciding whether $\prop{\varphi}$ is satisfiable in a $\varphi$-consistent model, using the characterisation given in Lemma~\ref{lem:proplemmaL} above. 
%To establish complexity results,
%we use the fact that there are only polynomially %quadratically
% many   subformulas in $\prop{\varphi}$. 
%Satisfiability in
%$\ALC$ is $\ExpTime$-complete~\cite{GabEtAl03} and so, one can determine in exponential time
%whether a valuation is $\varphi$-consistent. For an $\ExpTime$ upper bound, one can
%deterministically compute all possible $\varphi$-consistent valuations for 
%$(\bigwedge^{k-1}_{j=1}\psi_j\wedge\neg\psi_k)$ (or $(\psi_1 \wedge \neg \psi_2)$) and
%decide satisfiability of $\prop{\varphi}$ by a $\varphi$-consistent model using a bottom-up %strategy (as in~\cite{Baader:2012:LOD:2287718.2287721}). Since satisfiability in $\ALC$ is %$\ExpTime$-hard, our upper bound is tight.



%}}


\section{Reasoning on Constant Domain}
\label{sec:reasoncondom}
%\section{Reasoning in Non-normal Modal Description Logics with Constant Domain}



%%%%%%%%%%%%%%%%%%%%%%%%%%%%%%%%%%%%%%%%%%%%%%%%%%%%%%%%%%%%%%%%%%%%%%
%\subsection{Reductions to Satisfiability on Relational Models}
%\subsection{Reductions to $\K^{m}_{\ALC}$ Formula Satisfiability on Relational Models}
%\subsection{Reductions to Normal Modal Description Logics}
%
\label{sec:relation}

%\todo[inline]{M: add reduction of $\EMC^{n}_{\ALC}$ to $\K^{n}_{\ALC}$?}
	
We now study the complexity of the formula satisfiability problem in $\EnALC{n}$ and $\MnALC{n}$
on constant domain neighbourhood models.
%This result is then lowered %to $\ExpTime$ 
%for fragments, denoted by $\EALCg$  and $\MALCg$, in which the modal operators are applied \emph{globally}, i.e., over \ALC axioms only.
%%% OLD VERSION
%At the propositional level,
%%(multi-modal)
%logics $\Ebf^{n}$ and $\Mbf^{n}$ have both been used as a basis for weak deontic systems~\cite{AngEtAl,Che} (although $\Mbf^{n}$ suffers from several problems discussed in Section~\ref{sec:problem}), as well as to interpret praxeological operators, such as `agent $i$ has the ability to bring about $\p$'~\cite{Bro,Pac}.
%Moreover, $\Mbf^{n}$ has been combined with $\ALC$, as a basis for further coalition logic 
%extensions of description logic languages~\cite{SeyJam,SeyJam1}, and $\Ebf^{n}$ modal operators have been applied 
%over $\ALC$ axioms to formalise reasoning about agents' intentions~\cite{ErdSey} (however, without 
%establishing tight complexity results).
%%\nb{O: added}
%In this section we study the complexity of the formula satisfiability problem in $\EnALC{n}$ and $\MnALC{n}$. 
%This result is then lowered %to $\ExpTime$ 
%for fragments of these logics in which the modal operators are applied only over \ALC axioms.
%These fragments, denoted by $\EALCg$  and $\MALCg$,
%%\nb{M: Change notation?}
%are called \emph{global}.
%\nb{M: Change terminology/notation? It is fine with me as it is, but we don't have space for multimodal index $n$}
%\nb{M: mention papers~\cite{SeyErd, ErdSey}?}
%\paragraph{{\bf Satisfiability in $\EnALC{n}$ and $\MnALC{n}$.}}
We provide a $\NExpTime$ upper bound for satisfiability in $\EnALC{n}$ and $\MnALC{n}$ by using a reduction, lifted from the propositional case, to multi-modal $\KnALC{m}$.
The translation $\cdot\tr$ from \MLALC{n} to \MLALC{3n} is defined as~\cite{KraWol,GasHer}: 
	$A\tr = A$,
	$(\lnot C)\tr = \lnot C\tr$,
	$(C \sqcap D)\tr = C\tr \sqcap D\tr$,
	$(\exists \role.C)\tr = \exists \role.C\tr$;
$(C(a))\tr = C\tr(a)$,
$(r(a,b))\tr = r(a,b)$,
$(C \sqsubseteq D)\tr = C\tr \sqsubseteq D\tr$,
$(\lnot \psi)\tr  = \lnot \psi\tr$,
$(\psi \land \chi)\tr  = \psi\tr \land \chi\tr$;
$(\B_{i} \gamma)\tr = \D_{i_{1}} (\B_{i_{2}} \gamma\tr \circ \B_{i_{3}} \lnot \gamma\tr)$;
%
%\[
%%\begin{equation*}
%\begin{aligned}[c]
%	A\tr & = A \\
%	(\lnot C)\tr & = \lnot C\tr \\
%	(C \sqcap D)\tr & = C\tr \sqcap D\tr,\\
%	(\exists \role.C)\tr & = \exists \role.C\tr, \\
%\end{aligned}
%\qquad\qquad
%\begin{aligned}[c]
%(C(a))\tr & = C\tr(a) \\
%(r(a,b))\tr & = r(a,b), \\
%(C \sqsubseteq D)\tr & = C\tr \sqsubseteq D\tr, \\
%(\lnot \psi)\tr & = \lnot \psi\tr, \\
%(\psi \land \chi)\tr & = \psi\tr \land \chi\tr
%%			A\tr & = A, \\
%%			(\lnot C)\tr & = \lnot C\tr, \\
%%			(C \sqcap D)\tr & = C\tr \sqcap D\tr, \\
%%			(\exists \role.C)\tr & = \exists \role.C\tr, \\
%%			(C(a))\tr & = C\tr(a), \\
%%			(r(a,b))\tr & = r(a,b), \\
%%			(C \sqsubseteq D)\tr & = C\tr \sqsubseteq D\tr, \\
%%			(\lnot \gamma)\tr & = \lnot \gamma\tr, \\
%%			(\gamma \circ \delta)\tr & = \gamma\tr \circ \delta\tr, \\
%%			(\B_{i} \gamma)\tr & = \D_{i_{1}} (\B_{i_{2}} \gamma\tr \circ \B_{i_{3}} \lnot \gamma\tr)
%		\end{aligned}
%%\end{equation*}
%\]
%\[
%(\B_{i} \gamma)\tr = \D_{i_{1}} (\B_{i_{2}} \gamma\tr \circ \B_{i_{3}} \lnot \gamma\tr)
%\]
%\begin{align*}
%			A\tr & = A, \\
%%			(\lnot C)\tr & = \lnot C\tr, \\
%%			(C \sqcap D)\tr & = C\tr \sqcap D\tr, \\
%			(\exists \role.C)\tr & = \exists \role.C\tr, \\
%			(C(a))\tr & = C\tr(a), \\
%			(r(a,b))\tr & = r(a,b), \\
%			(C \sqsubseteq D)\tr & = C\tr \sqsubseteq D\tr, \\
%			(\lnot \gamma)\tr & = \lnot \gamma\tr, \\
%			(\gamma \circ \delta)\tr & = \gamma\tr \circ \delta\tr, \\
%			(\B_{i} \gamma)\tr & = \D_{i_{1}} (\B_{i_{2}} \gamma\tr \circ \B_{i_{3}} \lnot \gamma\tr)
%		\end{align*}
%%		\begin{align*}
%%			A\tr & = A, \\
%%			(\exists \role.C)\tr & = \exists \role.C\tr, \\
%%			(C \sqs D)\tr & = C\tr \sqs D\tr, \\
%%			(\vartheta)\tr & = \vartheta, \\
%%			(\lnot \gamma)\tr & = \lnot \gamma\tr, \\
%%			(\gamma \circ \delta)\tr & = \gamma\tr \circ \delta\tr, \\
%%			(\B_{i} \gamma)\tr & = \D_{i_{1}} (\B_{i_{2}} \gamma\tr \circ \B_{i_{3}} \lnot \gamma\tr)
%%		\end{align*}
where $A \in \NC$, $\role\in\NR$,
%$\vartheta$ is an assertion,
$\gamma$ is either an $\MLALC{n}$ concept or formula,
and $\circ \in \{ \sqcap, \land \}$ accordingly.
%
Using this translation, one can show that 
 satisfiability on neighbourhood models
%\nb{M: added}
is reducible to %the formula 
satisfiability on the
relational
%normal modal
models~\cite{KraWol,GasHer}.
%It follows from the reduction, given in Theorem~\ref{theor:classicalred}, 
%of the formula satisfiability problem for $\EALC$ to the same problem for $\KthreeALC$, which 
Since satisfiability in $\KnALC{3n}$ constant domain relational models is
$\NExpTime$-complete~\cite[Theorem 15.15]{GabEtAl03},
 we obtain the following complexity result.

\begin{restatable}{theorem}{Theoremcomplealc}\label{theor:complealc}
Satisfiability in $\EnALC{n}$  on constant domain neighbourhood models is decidable in $\NExpTime$.
\end{restatable}
%



%%% SKETCH
%\begin{proof}[Sketch]
%Let $\p$ be an \MLALC{n} formula s.t.
%$\Mmc, w \mdl \p$, for some \Nmodel $\Mmc = (\Fmc, \Delta, \Int)$
%and some $w$ in $\Fmc = (\Wmc, \{ \Nmc_{i} \}_{i \in [1, n]})$.
%We define an \Rframe
%$\Fmf = (W, \{ R_{i_{j}} \}_{i \in [1,n], j \in [1, 3]})$
%and an $\MLALC{3n}$ \Rmodel
%$\Mmf = (\Fmf, \Delta, I)$
%s.t.:
%	\begin{itemize}
%		\item $W = \{ (w, 0) \mid w \in \Wmc \} \cup \{ (\alpha, 1) \mid \alpha \in \bigcup_{v \in \Wmc} \Nmc_{i}(v) \}$
%		\item $\relations_{i_1} = \{ ((w, 0), (\alpha, 1)) \mid \alpha \in \Nmc_{i}(w)\}$;
%		\item $\relations_{i_2} = \{ ((\alpha, 1), (w, 0)) \mid w \in \alpha \}$
%		\item $\relations_{i_3} = \{  ((\alpha, 1), (w, 0)) \mid w \not \in \alpha \}$ 
%		\item for every $(w, 0) \in W$, $I(w, 0) = \Int(w)$; for every $(\alpha, 1) \in W$, $X^{I(\alpha, 1)} = \eset$, for all $X \in \NC \cup \NR$, and $a^{I(\alpha, 1)} = a^{\Int}$, for all $a \in \NI$.
%	\end{itemize}
%	%
%The pairs $(w, 0), (\alpha, 1)$ are used to ensure that $W$ is the disjoint union of the sets of worlds $w$ and subsets $\alpha$ of $\Wmc$.
%%
%By induction on concept and formulas occurring in $\p$,
%one can show that $\Mmf, (w, 0) \mdl \p\tr$.
%Conversely, given a $\MLALC{3n}$ formula $\p\tr$ s.t.
%$\Mmf, w \mdl \p\tr$,
%for some $\MLALC{3n}$ R-model
%$\Mmf = (\Fmf, \Delta, I)$
%based on
%$\Fmf = (W, \{ \relations_{i_{j}} \}_{i \in [1, n], j \in [1, 3]})$,
%and some
%$w \in W$,
%we define a $\MLnALC{n}$ N-model
%$\Mmc = (\Fmc, \Delta, \Int)$
%based on
%$\Fmc = (\Wmc, \{ \Nmc_{i} \}_{i \in [1, n]})$
%s.t.
%$\Wmc = W$,
%and for all $w \in W$:
%\begin{itemize}
%			\item $\alpha \in \Nmc_{i}(w)$ iff there is $v \in W$ s.t. $w \relations_{i_1} v$ and: $(i)$ for all $u \in W$, $v \relations_{i_2} u \Rightarrow u \in \alpha$, and $(ii)$ for all $u \in W$, $v \relations_{i_3} u \Rightarrow u \not \in \alpha$;
%			\item $\Int(w) = I(w)$.
%\end{itemize}
%Again, by induction, we obtain that $\Mmc, w \mdl \p$.
%\qed
%\end{proof}
 

%\subsection{The Monotonic Case}
The translation $\cdot\ttr$ from $\MLALC{n}$ to $\MLALC{2n}$ is defined as $\cdot\tr$ on all concepts and formulas, except for the modalised concepts or formulas $\gamma$:
%\begin{gather*}
	$(\B_{i} \gamma)\ttr = \D_{i_1} \B_{i_2} \gamma\ttr$.
%\end{gather*}
%\begin{definition}\label{def:monotonicmodel}
%\nb{M: Togliere, cfr. Sec. 3}
%A \e{supplemented $N$-frame}
%$\Fmc = (\Wmc, N)$
%is a $N$-frame such that for all
%$w \in \Wmc$ and all $\alpha, \beta \sbs \Wmc$,
%if
%$\alpha \in \Nmc(w)$ and $\alpha \sbs \beta$,
%then
%$\beta \in \Nmc(w)$.
%A \emph{supplemented $N$-model} is a $N$-model based on a supplemented $N$-frame.
%\end{definition}
%\begin{proof}(Sketch)
%\nb{M: Merge with next Th?}
%\end{proof}
%\noindent
We obtain an upper bound analogous to the one for $\EnALC{n}$ by a reduction %, shown in Theorem~\ref{theor:monotonicred}, 
of the formula satisfiability problem for $\MnALC{n}$
%\nb{O: typo here? \\ M: sure, thanks}
to the $\KnALC{2n}$ one~\cite{KraWol,GasHer,GabEtAl03}.
%which is $\NExpTime$-complete~\cite[Theorem 15.15]{GabEtAl}. 

\begin{restatable}{theorem}{Theoremcomplmalc}\label{theor:complmalc}
Satisfiability in $\MnALC{n}$   on constant domain neighbourhood models is decidable in $\NExpTime$.
\end{restatable}
%



%%% SKETCH
%\begin{proof}[Sketch]
%The proof is similar to the one of Theorem~\ref{theor:complealc}.
%%
%Given an \Nmodel based on a supplemented \Nframe satisfying an \MLALC{n} formula $\p$, we define an $\MLnALC{2n}$ \Rmodel satisfying $\p\ttr$ as above, by using relations $\relations_{i_1}$ and $\relations_{i_2}$ only.
%%
%To prove the inductive step for modalised formulas $\Box_{i} \psi$ occurring in $\p$, we use the fact that, in N-models $\Mmc$ based on supplemented N-frames $\Fmc = (\Wmc, \{ \Nmc_{i} \}_{i \in [1, n]})$,
%$\Mmc, w \mdl \B_{i} \psi$
%%$[ \p ]^{\Mmc}  \in \Nmc_{i}(w)$,
%is equivalent to:
%there is $\alpha \in \Nmc_{i}(w)$ s.t. $ \alpha \sbs [ \psi ]^{\Mmc}$.
%%
%Conversely, given a $\MLnALC{2n}$ \Rmodel
%$\Mmf = (\Fmf, \Delta, I)$
%based on
%$\Fmf = (W, \{ \relations_{i_{j}} \}_{i \in [1, n], j \in [1, 2]})$
%and satisfying $\p\ttr$, we define a \MLALC{n} \Nmodel
%$\Mmc = (\Fmc, \Delta, \Int)$
%based on
%$\Fmc = (\Wmc, \{ \Nmc_{i} \}_{i \in [1, n]})$
%s.t. $\Wmc = W$ and, for all $w \in W$: $\Int(w) =I(w)$;
%$\alpha \in \Nmc_{i}(w)$ iff there is $v \in W$ s.t. $w \relations_{i_1} v$ and for all $u \in W$, $v \relations_{i_2} u \Rightarrow u \in \alpha$.
%The \Nframe $\Fmc$ so defined is supplemented:
%for all $w \in W$, if $\alpha \in \Nmc_{i}(w)$ and $\alpha \sbs \beta \sbs W$, then there is $v \in W$ s.t. $w \relations_{i_1} v$ and for all $u \in W$, $v \relations_{i_2} u \Rightarrow u \in \beta$, i.e., $\beta \in \Nmc_{i}(w)$.
%Moreover, by induction, we have that $\Mmc$ satisfies $\p$.
%\qed
%\end{proof}


%%%%%%%%%%%%%%%%%%%%%%%%%%%%%%%%%%%%%%%%%%%%%%%%%%%%%%%%%%%%%%%%%%%%%%
\endinput



%\subsection{Fragments without Modalised Concepts on Constant Domain}
\label{sec:fragcondom}
%\subsection{Satisfiability on Constant Domain for $\LnALCg$}
%\subsection{Satisfiability in Fragments without Modalised Concepts}




%\paragraph{Satisfiability in $\EALCg$  and $\MALCg$}
%\nb{O: working here}


\newcommand{\clf}{\ensuremath{{\sf sub}_{\sf f}}\xspace}
\newcommand{\clc}{\ensuremath{{\sf sub}_{\sf c}}\xspace}
\newcommand{\individuals}[1]{\ensuremath{\NI(#1)}\xspace}
In this section,
%the proof of Lemma~\ref{lem:propE}
we use the classical
notion of a quasimodel~\cite{GabEtAl03}, defined here for the convenience of the reader. 
{\color{blue}{
Let $\varphi$ be an \emph{$\ALC$ formula}, that is, an $\MLnALC$ formula without any occurrence of modal operators.
We assume without loss of generality that assertions occurring in $\varphi$ are of the form $A(a)$, where $A \in \NC$, or $r(a,b)$.
}}
We denote by $\NI(\formula)$ the set of individual names occurring in $\formula$. 
 Denote by $\clf(\formula)$ the closure under single negation of the set
of all formulas occurring in~$\formula$.  Similarly, we denote by
$\clc(\formula)$ the closure under single negation of the set of all
concepts union the concepts $A_{a}, \exists r.A_{a}$, for any
$a\in\individuals{\formula}$ and $r$ a role occurring in $\formula$,
where $A_a$ is a fresh concept name.
%As usual, the basic elements of our quasimodels are types.
%
A \emph{concept type for $\formula$} is any subset~$t$ of
$\clc(\formula)\cup\individuals{\formula}$ such that:
\begin{itemize}%[label=$\mathbf{T\arabic*}$,leftmargin=*]
  \item\label{ct:neg} $\neg C\in t$ iff $C\not\in t$, for all
    $\neg C\in\clc(\formula)$;
  \item\label{ct:con} $C\sqcap D\in t$ iff $C,D\in t$, for all
    $C\sqcap D\in\clc(\formula)$;
%    and
 \item\label{ct:ind} $t$ contains at most one individual name in~$\individuals{\formula}$.
\end{itemize}

\noindent
Similarly, we define \emph{formula types} $t\subseteq\clf(\formula)$ for $\formula$ with the
  conditions:
\begin{itemize}%[label=$\mathbf{T\arabic*'}$,leftmargin=*]
  \item\label{ft:neg} $\neg\psi\in t$ iff $\psi\not\in t$, for all
    $\neg\psi\in\clf(\formula)$;
%    and
  \item\label{ft:fcon} $\psi\wedge\chi\in t$ iff $\psi,\chi\in t$, for all
    $\psi\wedge\chi\in\clf(\formula)$.
\end{itemize}
We   omit `for $\formula$'  when there is no risk of confusion. 
%A concept type describes one domain element at a single time point,
%while a formula type expresses constraints on all domain elements.
%
If $a\in t\ \cap\ \individuals{\formula}$, then $t$ describes a named element. 
We  write $t_a$ to indicate this and call it a \emph{named type}.
%
%
We say that a pair of concept types $(t,t')$ is \emph{$r$-compatible} if
%\begin{align*}
$\{\neg F\mid \neg \exists
r.F\in t\}\subseteq t'.$ %\nb{A: changed E to F \\O: Ok!}
%\end{align*}
A \emph{quasimodel} for~$\formula$ is a set $\Qmc$ of concept or formula types for~$\formula$ 
such that:
\begin{itemize}%[label=\textbf{Q\arabic*},leftmargin=*]
  \item\label{q:fseg} $\Qmc$ contains exactly one formula type~$t_\Qmc$;
  \item\label{q:ind} $\Qmc$ contains exactly one named type $t_a$ for
    each $a\in\individuals{\formula}$;
  \item\label{q:gci} for all $C\sqsubseteq D\in\clf(\formula)$, we have
    $C\sqsubseteq D\in t_\Qmc$ iff $C\in t$ implies $D\in t$
    for all concept types $t \in \Qmc$;
  \item\label{q:cass} for all $A(a)\in\clf(\formula)$, we have
    $A(a)\in t_\Qmc$ iff $A\in t_a$ %\nb{A: changed C to A \\ O: Ok!};
    \item \label{q:quasi}   $t \in \Qmc$ and $\exists r.D \in t$ implies there is $t' \in \Qmc$
    such that $D\in t'$ and $(t,t')$ is      $r$-compatible;
    \item \label{q:last} for all $r(a,b)\in\clf(\formula)$, we have
      $r(a,b)\in t_\Qmc$ iff   
     $(t_a,t_b)$ is $r$-compatible. 
\end{itemize}
 
 

As usual, every quasimodel for~$\formula$ describes an  
interpretation satisfying~$\formula$ and, conversely,   every such interpretation can be 
abstracted into a quasimodel for~$\formula$. We formalise the correspondence 
between models and quasimodels  with 
the following lemma. 

%\nb{TODO define quasimodel as in yellow but with ind names in concept types, define types}
\begin{lemma}\label{lem:aux}
There is a quasimodel for an \ALC formula   iff 
there is a model of it. 
\end{lemma}

%We are now ready for proving Lemma~\ref{lem:propE}.
We now formalise the connection between %complexity of 
the satisfiability problem for 
$\LnALCg$ formulas on constant domain neighbourhood models and their propositional abstractions %with consistent models
with the following lemma.
%, which is an adaptation of the 
%results obtained for other \ALC extensions~\cite{Baader:2012:LOD:2287718.2287721}.
%~ %We say that a model of $\prop{\varphi}$
%~ %is $\varphi$-consistent

%\nb{$\varphi$-consistent model}

\begin{restatable}{lemma}{LemmapropE}\label{lem:propE}
\nb{M: todo fix proof}
A
%$\MLnALCg$
formula $\varphi$ is
$\LnALCg$
satisfiable
on constant domain neighbourhood models
iff
$\prop{\varphi}$ is satisfied in a $\varphi$-consistent $L^{n}$ model.
%A formula $\varphi$ is $\EALCg$ satisfiable
%iff
%%if, and only if, 
%$\prop{\varphi}$ is satisfied in a $\varphi$-consistent $\E^{n}$ model.  
\end{restatable}
%


We can now use again to Lemma~\ref{lem:proplemmaL} to  characterise satisfiability of  $\prop{\varphi}$ in a $\varphi$-consistent model by the existence of a  $\varphi$-consistent valuation satisfying suitable corresponding properties.
%
%
%We
%%\nb{M: moved footnote here to save space}
%%\nb{M: todo fix notation}
%assume that the 
%primitive
%%propositional
%connectives used to build 
%%the
%propositional formulas are $\neg$ %, $\vee$, 
%and $\wedge$ ($\vee$ is expressed using 
%$\neg$ %, $\vee$, 
%and $\wedge$),
%%Moreover,
%and
%we
%denote by  ${\sf sub}(\prop{\varphi})$ the set 
%of subformulas of $\prop{\varphi}$ closed under single negation.  
%A \emph{valuation} for a propositional
%$\MLn$
%%modal logic
%formula $\prop{\varphi}$   
%is a function $\nu: {\sf sub} (\prop{\varphi})\rightarrow \{0,1\}$ that 
%satisfies the following conditions:
%%\footnote{Assuming that the 
%%primitive propositional connectives used to build 
%%the formulas are $\neg$ %, $\vee$, 
%%and $\wedge$ ($\vee$ is expressed using 
%%$\neg$ %, $\vee$, 
%%and $\wedge$).}:
%(1) for all $\neg\psi\in {\sf sub} (\prop{\varphi})$,
%$\nu(\psi)=1$ iff $\nu(\neg\psi)=0$;
%%(2) for all $\psi_1\vee \psi_2\in {\sf sub} (\prop{\varphi})$,
%%$\nu(\psi_1\vee \psi_2) = 1$ iff $\nu(\psi_1) = 1$
%%or $\nu(\psi_2) = 1$; 
%(2) for all $\psi_1\wedge \psi_2\in {\sf sub} (\prop{\varphi})$,
%$\nu(\psi_1\wedge \psi_2) = 1$ iff $\nu(\psi_1) = 1$
%and $\nu(\psi_2) = 1$; 
%and (3) $\nu(\prop{\varphi})=1$. 
%We say that a valuation for $\prop{\varphi}$ 
%is \emph{$\varphi$-consistent} if any %propositional
%neighbourhood model of the form $(\{w\}, \{ \propneigh_{i} \}_{i \in [1, n]}, \propassign)$ satisfying
%$w\in \propassign(p_\elaxiom)$ iff $\nu(p_\elaxiom)=1$, for all $p_\elaxiom\in \NPr(\varphi)$, 
% is $\varphi$-consistent.
%We now establish that satisfiability of 
%$\prop{\varphi}$ in a $\varphi$-consistent model is characterized 
%by the existence of a  $\varphi$-consistent valuation 
%satisfying the property described in Lemma~\ref{lem:propsat}.
%%
%\begin{restatable}{lemma}{Lemmapropsat}\label{lem:propsat}
%A formula $\prop{\varphi}$ is satisfied in a $\varphi$-consistent $\E^{n}$ model
%iff
%%if, and only if,
%there is   a $\varphi$-consistent valuation \valuation 
%for $\prop{\varphi}$ such that if $\B_i\psi_1$ and $\B_i\psi_2$ are in 
%${\sf sub}(\prop{\varphi})$, $\valuation(\B_i\psi_1)=1$, and 
%$\valuation(\B_i\psi_2)=0$, then $(\psi_1\wedge\neg\psi_2)\vee(\neg\psi_1\wedge\psi_2)$
%is satisfied in a $\varphi$-consistent $\E^{n}$ model. 
%\end{restatable}
%%
% \begin{proof}
% ($\Rightarrow$) Suppose that $\prop{\varphi}$ is satisfied in a world $w$ of a $\varphi$-consistent model 
% $\propmodel = (\propdomain, \{ \propneigh_{i} \}_{i \in [1,n]}, \propassign)$. That is, 
% $\propmodel, w\models \varphi$. We define a $\varphi$-consistent valuation for 
% $\prop{\varphi}$
% by setting $\nu(\psi)=1$ if $\propmodel, w\models \psi$ and $\nu(\psi) = 0$
% if  $\propmodel, w\not\models \psi$. 
% It is easy to check that $\nu$ is indeed a 
% $\varphi$-consistent valuation (given that $\propmodel$ is 
% $\varphi$-consistent). Assume that 
% $\B_i\psi_1$ and $\B_i\psi_2$ are in ${\sf sub}(\prop{\varphi})$, 
% $\nu(\B_i\psi_1)=1$ and $\nu(\B_i\psi_2)=0$. Then $\propmodel,w\models \B_i\psi_1$
% and $\propmodel,w\not\models \B_i\psi_2$. Thus, by definition, 
% $\propassign(\psi_1)\in \propneigh_i(w)$ and 
%  $\propassign(\psi_2)\not\in \propneigh_i(w)$.
%%Using the same argument of Lemma 3.1 in~\cite{Var2}, it follows that 
%So,
%   $\propassign(\psi_1)\neq \propassign(\psi_2)$.
%Then %it is easy to see that
% there 
%   is a world $u$ in the symmetrical difference of these sets 
%   such that $\propmodel,u\models (\psi_1\wedge\neg\psi_2)\vee(\neg\psi_1\wedge\psi_2)$. 
%   
% ($\Leftarrow$) Suppose there is a $\varphi$-consistent valuation $\nu$ for $\prop{\varphi}$ such that 
% if $\B_i\psi_1$ and $\B_i\psi_2$ are in 
%${\sf sub}(\prop{\varphi})$, $\valuation(\B_i\psi_1)=1$, and 
%$\valuation(\B_i\psi_2)=0$, then 
%there is a $\varphi$-consistent model \[\propmodel_{\psi_1,\psi_2}=(\propdomain_{\psi_1,\psi_2},
%\{ \propneigh_{{\psi_1,\psi_2}_{i}} \}_{i \in [1,n]},\propassign_{\psi_1,\psi_2})\]
% and a world 
%$w_{\psi_1,\psi_2}\in \propdomain_{\psi_1,\psi_2}$ such that 
%$\propmodel_{\psi_1,\psi_2}, w_{\psi_1,\psi_2}\models (\psi_1\wedge\neg\psi_2)\vee(\neg\psi_1\wedge\psi_2)$. 
%
%Let $\propmodel_1,\ldots,\propmodel_m$ be an enumeration of the models 
%$\propmodel_{\psi_1,\psi_2}$, as above.  That is, we take one model for each 
%pair $\psi_1,\psi_2$ where $\propmodel_j = (\propdomain_j, \{ \propneigh_{j_{i}} \}_{i \in [1,n]},\propassign_j)$, 
%and let $w_1,\ldots,w_m$ be an enumeration of the worlds $w_{\psi_1,\psi_2}$, 
%with $w_j\in \propdomain_j$. We assume w.l.o.g. that $\propdomain_j\cap \propdomain_k=\emptyset$ 
%for $j\neq k$. 
%
%In the following, we define a $\varphi$-consistent  model   $\propmodel = (\propdomain,\{ \propneigh_{i} \}_{i \in [1,n]}, \propassign)$ 
%for $\prop{\varphi}$. 
%Intuitively, we construct $\propmodel$ by taking the union of each 
%$\propmodel_j$ with the addition of a new world $w$ that 
%will satisfy $\prop{\varphi}$. 
%We define $\propdomain$ as $\bigcup_{1\leq j\leq n}\propdomain_j\cup \{w\}$, 
%where $w$ is fresh.
%%The tricky part of the proof is to define the assignment $\propassign$. 
%Before defining $\propneigh_{i}$ and $\propassign$, we define the function $J: {\sf sub}(\prop{\varphi})\rightarrow \Pmc(\Wmc)$
%with $J(\psi)=\bigcup_{0\leq j \leq m} \Vmc_j(\psi)$ for all $\psi\in {\sf sub}(\prop{\varphi})$, where %$I_i$ is as above for $1\leq i\leq n$, 
% %and
%$\Vmc_0: {\sf sub}(\varphi)\rightarrow  \Pmc(\{w\})$ is the function
%that assigns $\psi$ to $\{w\}$, if $\nu(\psi)=1$, 
%and to $\emptyset$, otherwise ($\Vmc_j$, for $1\leq j\leq m$, is as above).
%By construction, we have that $J(\neg \psi)=\propdomain\setminus J(\psi)$
%and $J(\psi_1\wedge \psi_2)=J(\psi_1)\cap J(\psi_2)$. 
%We define the assignment $\propassign$ as the function 
%$\propassign: \NPr(\varphi)\rightarrow \Pmc(\Wmc)$ satisfying 
% $\propassign(p_\elaxiom)=J(p_\elaxiom)$ for all $p_\elaxiom\in \NPr(\varphi)$. 
% 
%It remains to define $\propneigh_i$, for $1 \leq i \leq n$. 
%For $u\in \propdomain_j$ and $\B_i\psi \in {\sf sub}(\varphi)$, we   put $J(\psi) \subseteq \Wmc$ in $\propneigh_i(u)$ 
%precisely when $\propmodel_j,u \models \B_i \psi$.  
%%and $\alpha = J(\psi_\alpha)$
%%for some $\B_i\psi_\alpha \in {\sf sub}(\varphi)$.
%We claim that 
%if $\beta \in \Nmc_i(u)$ and $\beta = J(\psi)$ for some $\B_i\psi\in{\sf sub}(\prop{\varphi})$,
%then $\propmodel_j,u\models\B_i\psi$.
%Indeed, since $\beta = J(\psi)\in \Nmc_i(u)$, 
%we must have that $\propmodel_j,u\models \B_i\psi_{\beta}$ and  $\beta = J(\psi_{\beta})$ for 
%some $\B_i\psi_\beta \in{\sf sub}(\prop{\varphi})$.
%But since
%$J(\psi)=J(\psi_\beta)$, we also have $\propassign_j(\psi)=\propassign_j(\psi_\beta)$ 
%(recall that $\propdomain_j \cap \propdomain_k = \emptyset$ for  
%$k \neq j$), so $\propmodel_j,u\models \B_i\psi$ iff 
%$\propmodel_j,u \models \B_i \psi_\beta$.
%It follows that $\propmodel_j,u\models \B_i\psi$.
%
%Also, we put $\alpha \subseteq \Wmc$ in $\propneigh_i(w)$ (recall $w$ is the fresh 
%world introduced above in $\propdomain$) precisely 
%when $\nu(\B_i\psi_\alpha) = 1$ and $\alpha = J(\psi_\alpha)$ for some 
%$\B_i\psi_\alpha \in {\sf sub}(\prop{\varphi})$.
%We claim that if 
%$\beta \in \propneigh_i(w)$ and $\beta = J(\psi)$ for some $\B_i \psi \in{\sf sub}(\prop{\varphi})$ 
%then $\nu(\B_i \psi) = 1$.
%Indeed, since $\beta = J(\psi)\in \propneigh_i(u)$ 
%we must have that $\nu(\B_i\psi_\beta)=1$ and $\beta = J(\psi_\beta)$ for some 
%$\B_i\psi_\beta \in{\sf sub}(\prop{\varphi})$. 
%Suppose now that $\nu(\B_i\psi) = 0$. Then, by assumption, there exists a structure
%$\propmodel_j = (\propdomain_j, \{ \propneigh_{j_i} \}_{i \in [1, n]}, \propassign_j)$ and a world $w_j
%\in \propdomain_j$ such that $\propmodel_j,w_j \models (\psi_\beta \wedge \neg\psi)\vee(\neg\psi_\beta \wedge \psi)$. 
%It follows that $\propassign_j(\psi_\beta)\neq \propassign_j(\psi)$. 
%Consequently $J(\psi_\beta)\neq J(\psi)$, which is a contradiction.  
%%\nb{O :... talk about consistent}
%
%We now show by induction on the structure of formulas 
%that $\propassign$ and $J$ agree on ${\sf sub}(\prop{\varphi})$. 
%This holds by construction for atomic propositions. It is easy to deal 
%with propositional connectives, since we know that $J(\neg \psi)=\propdomain\setminus J(\neg \psi)$
%and  $J(\psi_1\wedge \psi_2)=J(\psi_1)\cap J(\psi_2)$ 
%and similarly for $\propassign$. Assume inductively that $\propassign(\psi) = J(\psi)$.
%Suppose first that $u\in J(\B_i\psi)$. Then, either $u=w$ and $\nu(\B_i\psi)=1$
%or $u\in \propdomain_j$ and $\propmodel_j,u\models\B_i\psi$. In either case 
%we have that $J(\psi)\in \propneigh_i(u)$. Since 
%$\propassign(\psi) = J(\psi)$, it follows that $\propmodel,u\models \B_i\psi$, 
%that is, $u\in \propassign(\B_i\psi)$. Suppose now that $u\in \propassign(\B_i\psi)$, 
%that is, $\propmodel,u\models \B_i\psi$, or, equivalently, $\propassign(\psi)\in \propneigh_i(u)$.
%Since $\propassign(\psi) = J(\psi)$ it follows that either $u=w$ and 
%$\nu(\B_i\psi)=1$ or $u\in \propdomain_j$ and $\propmodel_j,u\models \B_i\psi$. 
%In either case we have that $u\in J(\B_i\psi)$. 
%
%Since $\nu(\prop{\varphi})=1$, we have that $w\in J(\prop{\varphi})$, 
%and consequently $w\in \propassign(\prop{\varphi})$. That 
%is, $\propmodel,w\models\prop{\varphi}$. 
%The fact that $\propmodel$ is $\varphi$-consistent follows from 
%the fact that $\nu$, used to construct the assignment 
%related to $w$, is $\varphi$-consistent and 
%the models $\propmodel_1,\ldots,\propmodel_m$, used to define 
%the remaining worlds in $\Wmc$, are all $\varphi$-consistent. 
%\end{proof}
%
Finally, to check satisfiability of $\prop{\varphi}$ in a $\varphi$-consistent model, we \ldots
%use Lemma~\ref{lem:propsat} and the fact that
%there are only quadratically many formulas of the form $\psi_1\wedge\neg\psi_2$, 
%where $\psi_1$ and $\psi_2$ are subformulas of $\prop{\varphi}$. 
%We observe that satisfiability in \ALC is \ExpTime-complete~\cite{GabEtAl03}
%and so, one can determine in exponential 
%time whether a valuation is $\varphi$-consistent.
%For an $\ExpTime$ upper bound, one can deterministically compute 
%all possible $\varphi$-consistent valuations for $\psi_1\wedge\neg\psi_2$ 
%and decide satisfiability of $\prop{\varphi}$ by a $\varphi$-consistent
%model using a bottom-up strategy (as in~\cite{Var2}). 
%As satisfiability in \ALC 
%is \ExpTime-hard  our upper bound is tight.
This leads us to the following tight complexity result.

\begin{restatable}{theorem}{Theorempropsat}\label{thm:propsat}
The $\LnALCg$ formula satisfiability problem on constant domain neighbourhood models is $\ExpTime$-complete.
\end{restatable}

%To determine satisfiability of $\prop{\varphi}$ in a $\varphi$-consistent
%model, we use Lemma~\ref{lem:propsat} and the fact that
%there are only quadratically many formulas of the form $\psi_1\wedge\neg\psi_2$, 
%where $\psi_1$ and $\psi_2$ are subformulas of $\prop{\varphi}$. 
%We observe that satisfiability in \ALC is \ExpTime-complete~\cite{GabEtAl03}
%%~\cite{dlhandbook},
%%\nb{M: Changed ref. to shrink bib} 
%and so, one can determine in exponential 
%time whether a valuation is $\varphi$-consistent.
%For an $\ExpTime$ upper bound, one can deterministically compute 
%all possible $\varphi$-consistent valuations for $\psi_1\wedge\neg\psi_2$ 
%and decide satisfiability of $\prop{\varphi}$ by a $\varphi$-consistent
%model using a bottom-up strategy (as in~\cite{Var2}). 
%As satisfiability in \ALC 
%%and in the propositional modal logic \E{} 
%is \ExpTime-hard  our upper bound is tight.
%
%\begin{restatable}{theorem}{Theorempropsat}\label{thm:propsat}
%The $\EALCg$ formula satisfiability problem on constant domain neighbourhood models is $\ExpTime$-complete.
%\end{restatable}

%Regarding the proof for \MALCg, we first point out that 
%Lemma~\ref{lem:propE} can be easily adapted to \MALCg. 
%%the proof for the propositional case $\Mn^n$~\cite{Var2} can be similarly 
%%adapted to our combination with \ALC.
%%~ Lemma~\ref{lem:prop} can be easily adapted to \MALCg. 
%%~ We now formalize a variant of Lemma~\ref{lem:propsat} 
%%~ for \MALCg. 
%%We   establish a variant of Lemma~\ref{lem:propsat} 
%%tailored for $\Mn^{n}$ (see Proposition 3.8 in~\cite{Var2}).
%The proof for our $\ExpTime$ result for \MALCg 
%is analogous to the one given for \EALCg, except that here  we use 
%a variant of 
%%the following variant of
%Lemma~\ref{lem:propsat} 
%tailored for $\EM^{n}$
%%\nb{M: todo fix}
%(see Proposition 3.8 in~\cite{Var2}).
%%Lemma~\ref{lem:propsatm} and an adaptation of Lemma~\ref{lem:prop}.
%%\begin{restatable}{lemma}{Lemmapropsatm}\label{lem:propsatm}
%%A formula $\prop{\varphi}$ is $\Mn^{n}$ satisfiable by a $\varphi$-consistent model
%%if, and only if, there is   a $\varphi$-consistent valuation \valuation 
%%for $\prop{\varphi}$ such that if $\B_i\psi_1$ and $\B_i\psi_2$ are in 
%%${\sf sub}(\prop{\varphi})$, $\valuation(\B_i\psi_1)=1$, and 
%%$\valuation(\B_i\psi_2)=0$, then $\psi_1\wedge\neg\psi_2$
%%is $\Mn^{n}$ satisfiable by a $\varphi$-consistent model. 
%%\end{restatable}
%%the proof for the propositional case $\Mn^n$~\cite{Var2} can be similarly 
%%adapted to our combination with \ALC. 
%Thus, we obtain also the following result.
%%\nb{M: changed a bit. Do you prefer the previous sentence? Fine with me}
%%We are now ready for Theorem~\ref{thm:propsat}. 
%\begin{restatable}{theorem}{Theorempropsat}\label{thm:propsat}
%The $\MALCg$ formula satisfiability problem on constant domain neighbourhood models is $\ExpTime$-complete.
%%Satisfiability  in \MALCg %and \EEL{g} 
%% is \ExpTime-complete.
% % and 
%%\NP-complete, respectively. 
%\end{restatable}
%
%%~ Regarding \MALCg, the proof for the propositional case $\Mn^n$~\cite{Var2} can be similarly 
%%~ adapted to our combination with \ALC. 
%
%%~ \begin{restatable}{theorem}{Theorempropsat}\label{thm:propsat}
%%~ Satisfiability  in \MALCg %and \EEL{g} 
% %~ is \ExpTime-complete.
% %~ % and 
%%~ %\NP-complete, respectively. 
%%~ \end{restatable}
%
%%\paragraph{{\color{red}{M: $\ExpTime$-completeness of $\EALCg$ (and $\MALCg$, analogously)}}}
%%(Sketch, cf.~\cite{Baader:2012:LOD:2287718.2287721})
%%\nb{M: Check!}
%%
%%\medskip
%%\noindent
%%Upper bound:
%%
%%\medskip
%%1 -- Given a $\MLALC{}$ formula $\p$, compute its propositional abstraction $\prop{\p}$.
%%
%%\medskip
%%2 -- Define:
%%\[
%%S := \{ X_1, \ldots, X_k \} \subseteq \NPr(\{ p_1, \ldots, p_n \})
%%\]
%%\[
%%\p_{X_{i}} := (\bigwedge_{p_{j} \in X_{i}} \pi_{j} \land \bigwedge_{p_{j} \not \in X_{i}} \pi_{j} );
%%\qquad
%%\p_{X} := \bigwedge_{1 \leq i \leq k} \p_{X_{i}}
%%\]
%%\[
%%\prop{\p}^{S} := \prop{\p} \land \bigwedge_{1 \leq i \leq 3} \Box_{i} (\bigvee_{X \in S} (\bigwedge_{p \in X} p \land \bigwedge_{p \not \in X} p ))
%%\]
%%
%%\medskip
%%3 -- Prove:
%%
%%\begin{lemma}
%%A $\MLALC{3}^{g}$ formula $\p$ is satisfiable iff there is a set $S = \{ X_1, \ldots, X_k \} \subseteq \NPr(\{ p_1, \ldots, p_n \})$ such that the $\ML^{3}$ formula $\prop{\p}^{S}$ and the $\ALC$ formula $\p_{X}$ are satisfiable.
%%\end{lemma}
%%\begin{proof}
%%\end{proof}
%%
%%
%%\medskip
%%4 -- Define $S^{*}$ as the set of all subsets $X \subseteq \{p_1, \ldots, p_n \}$ such that the following $\ALC$ formula is satisfiable:
%%\[
%%\p_X := (\bigwedge_{p_{j} \in X} \pi_{j} \land \bigwedge_{p_{j} \not \in X} \pi_{j} )
%%\]
%%\[
%%\prop{\p}^{S^{*}} := \prop{\p} \land \bigwedge_{1 \leq i \leq 3} \Box_{i} (\bigvee_{X \in S^{*}} (\bigwedge_{p \in X} p \land \bigwedge_{p \not \in X} p ))
%%\]
%%
%%
%%\medskip
%%5 -- Prove:
%%
%%\begin{lemma}
%%Let $\p_1, \ldots, \p_k$ be $\ALC$ formulas over disjoint sets of individual, concept, and role names. Then $\p_1 \land \ldots \land \p_k$ is satisfiable iff, for each $i \in \{ 1, \ldots, k \}$, $\p_i$ is satisfiable.
%%\end{lemma}
%%\begin{proof}
%%\end{proof}
%%
%%\medskip
%%6 -- Prove:
%%\begin{lemma}
%%A $\MLALC{3}^g$ formula $\p$ is satisfiable iff the $\ML^{3}$ formula $\prop{\p}^{S^{*}}$ is satisfiable.
%%\end{lemma}
%%\begin{proof}
%%\end{proof}
%%
%%\medskip
%%7 -- Since computing $S^{*}$ and checking satisfiability of $\prop{\p}^{S^{*}}$ can be done in exponential time in the size of $\p$, using the translation $\cdot\tr$, we hae that the $\EALCg$ satisfiability problem is in $\ExpTime$.
%%
%%
%%\medskip
%%\noindent
%%Lower bound: from $\ExpTime$-hardness of $\ALC$.
%%
%%\begin{theorem}
%%The $\EALCg$ satisfiability problem is $\ExpTime$-complete.
%%\end{theorem}
%
%
%
%
%
%
%
%
%
%%\paragraph{Satisfiability in $\CALCg$  and $\NALCg$}
%%
%%In~\citeauthor{DL19}~\cite{DL19}, it is shown that 
%%$\EALCg$  and $\MALCg$ formula satisfiability problems on constant domain neighbourhood models are 
%%$\ExpTime$-complete.
%%
%We now show tight complexity results %$\ExpTime$ upper bounds 
%for $\CALCg$  and $\NALCg$,
%using again the notion of 
%a propositional abstraction of a formula 
%(as in, e.g.,~\cite{Baader:2012:LOD:2287718.2287721}).
%%Since $\ALC$ formula satisfiability is already $\ExpTime$-hard, 
%%our upper bounds here are tight. 
%%we have 
%%a tight complexity result for the global cases.
%%We show that
%Here, one can separate the satisfiability test into two parts, 
%one for the description logic dimension and 
%one for the 
%%dimension of the
%%\neighborhood
%%frame. 
%modal dimension.
%%~ In this subsection, we also consider the lightweight DL called \EL, defined 
%%~ as the fragment of \ALC which only allows conjunctions and existential quantification 
%%~ in concept expressions. 
%%
%%Consider $\Lmc\in\{\ALC,\EL\}$. 
%%\newcommand{\propmodel}{\ensuremath{M}\xspace}
%%\newcommand{\propdomain}{\ensuremath{W}\xspace}
%%\newcommand{\propneigh}{\ensuremath{N}\xspace}
%%\newcommand{\propassign}{\ensuremath{I}\xspace}
%The 
%\emph{propositional  abstraction} $\prop{\varphi}$ of an
%$\MLnALCg$
%%$\CALCg$ (respectively, $\NALCg$)
%formula $\varphi$ is  
%the result of replacing each $\ALC$
%CI
%%atom
%in $\varphi$ by 
%a propositional letter $p$, so that there is a $1:1$ relationship 
%between the $\ALC$
%CI
%%atoms
%$\elaxiom$ occurring in $\varphi$ and the 
%propositional letters $p_{\elaxiom}$ used for the abstraction.  
%%The semantics of $\prop{\varphi}$ is given in terms of \emph{propositional 
%%neighbourhood models} $ (\Wmc, \{ \Nmc_{i} \}_{i \in I}, \Vmc)$ for $\EC^{n}$ (respectively, $\EN^{n}$), 
%%where $(\Wmc, \{ \Nmc_{i} \}_{i \in I})$ is a neighbourhood frame 
%%and $\Vmc: \NPr \rightarrow 2^{\Wmc}$ is a function 
%%mapping propositional letters in $\NPr$ to
%%sets of worlds
%%%subsets of the domain of worlds
%%(see~\cite{Che,Var2}). 
%We set $\NPr(\varphi) = \{p_{\elaxiom}\in\NPr\mid \elaxiom \text{ is an \ALC
%CI
%%atom
%in }
%\varphi \}$.
%%
%Given an
%$\MLnALCg$
%%\nb{M: todo fix}
%%$\CALCg$ (respectively, $\NALCg$)
%formula $\varphi$, we say that a propositional neighbourhood model 
%$\propmodel = (\Wmc, \{ \Nmc_{i} \}_{i \in I}, \Vmc)$
%of $\prop{\varphi}$
%is \emph{$\consistent{\varphi}$}
%if, for all $w\in \Wmc$,
%the following \ALC formula is satisfiable $$\textstyle\bigwedge_{p_{\elaxiom}\in \NPr(w)} \ {\elaxiom} \ \wedge \
%\bigwedge_{p_{\elaxiom}\in \overline{\NPr(w)}}\ \neg {\elaxiom},$$
%where $\NPr(w) = \{p_{\elaxiom}\in \NPr(\varphi) \mid w\in \Vmc(p_{\elaxiom})\}$
%and $\overline{\NPr(w)}=\NPr(\varphi)\setminus\NPr(w)$. 
%We now formalise the connection between %complexity of the satisfiability problem for 
%$\MLnALCg$
%%$\CALCg$ and $\NALCg$
%formulas and their propositional abstractions %with consistent models
%with the following lemma, where $\mathit{L} \in \{ \EC, \EN \}$, obtained by adapting the proof of Lemma~\ref{lem:propE}
%%~\citeauthor{DL19}~\cite[Lemma~1]{DL19}.
%%, which is an adaptation of the 
%%results obtained for other \ALC extensions~\cite{Baader:2012:LOD:2287718.2287721}.
%%~ %We say that a model of $\prop{\varphi}$
%%~ %is $\varphi$-consistent
%
%%\nb{$\varphi$-consistent model}
%
%\begin{restatable}{lemma}{Lemmaprop}\label{lem:satfraglog}
%A formula $\varphi$ is $\LnALCg$ satisfiable
%on constant domain neighbourhood models
%iff
%%if, and only if, 
%$\prop{\varphi}$ is satisfied in a $\varphi$-consistent $\mathit{L}^{n}$ model.  
%\end{restatable}
%
%%\begin{restatable}{lemma}{Lemmaprop}\label{lem:general}
%%A formula $\prop{\varphi}$ is satisfied in a $\varphi$-consistent  
%%$\mathit{L}^{n}$
%%model
%%iff
%%if, and only if,
%%there is   a $\varphi$-consistent valuation \valuation 
%%for $\prop{\varphi}$ such that 
%%if $\B_i\psi_1,   \B_i\psi_2$ are in 
%%${\sf sub}(\prop{\varphi})$, $\valuation(\B_i\psi_1)=1$   and 
%%$\valuation(\B_i\psi_2)=0$, then either $(\psi_1\wedge\neg\psi_2)$ or 
%%$(\neg\psi_1\wedge\psi_2)$ 
%%is satisfied in a $\varphi$-consistent $\mathit{L}^{n}$
% %model. 
%%\begin{itemize}
%%\item if $\B_i\psi$ is in ${\sf sub}(\prop{\varphi})$ and
%%$\valuation(\B_i\psi_1)=0$ then $\neg \psi$ is 
%%$\EC^{n}$ satisfiable; and 
%%\item 
%%\end{itemize}
%%\end{restatable}
%
%We
%%\nb{M: moved footnote here to save space}
%%\nb{M: todo fix notation}
%assume that the 
%primitive
%%propositional
%connectives used to build 
%%the
%propositional formulas are $\neg$ %, $\vee$, 
%and $\wedge$ ($\vee$ is expressed using 
%$\neg$ %, $\vee$, 
%and $\wedge$),
%%Moreover,
%and
%we
%denote by  ${\sf sub}(\prop{\varphi})$ the set 
%of subformulas of $\prop{\varphi}$ closed under single negation.  
%A \emph{valuation} for a propositional
%$\MLn$
%%modal logic
%formula $\prop{\varphi}$   
%is a function $\nu: {\sf sub} (\prop{\varphi})\rightarrow \{0,1\}$ that 
%satisfies the following conditions:
%%\footnote{Assuming that the 
%%primitive propositional connectives used to build 
%%the formulas are $\neg$ %, $\vee$, 
%%and $\wedge$ ($\vee$ is expressed using 
%%$\neg$ %, $\vee$, 
%%and $\wedge$).}:
%(1) for all $\neg\psi\in {\sf sub} (\prop{\varphi})$,
%$\nu(\psi)=1$ iff $\nu(\neg\psi)=0$;
%%(2) for all $\psi_1\vee \psi_2\in {\sf sub} (\prop{\varphi})$,
%%$\nu(\psi_1\vee \psi_2) = 1$ iff $\nu(\psi_1) = 1$
%%or $\nu(\psi_2) = 1$; 
%(2) for all $\psi_1\wedge \psi_2\in {\sf sub} (\prop{\varphi})$,
%$\nu(\psi_1\wedge \psi_2) = 1$ iff $\nu(\psi_1) = 1$
%and $\nu(\psi_2) = 1$; 
%and (3) $\nu(\prop{\varphi})=1$. 
%We say that a valuation for $\prop{\varphi}$ 
%is \emph{$\varphi$-consistent} if any
%propositional
%neighbourhood
%model of the form $(\{w\}, \{ \propneigh_{i} \}_{i \in I}, \propassign)$ satisfying
%$w\in \propassign(p_\elaxiom)$ iff $\nu(p_\elaxiom)=1$, for all $p_\elaxiom\in \NPr(\varphi)$, 
% is $\varphi$-consistent.
%We now establish that satisfiability of 
%$\prop{\varphi}$ in a $\varphi$-consistent $\EC^{n}$ (respectively, $\EN^{n}$) model is characterized 
%by the existence of a  $\varphi$-consistent valuation 
%satisfying the property described in Lemma~\ref{lem:proplemma2} 
%(respectively, Lemma~\ref{lem:proplemmaN}).
%
%\begin{restatable}{lemma}{Lemmapropsec}\label{lem:proplemma2}
%A formula $\prop{\varphi}$ is satisfied in a $\varphi$-consistent $\EC^{n}$ %\nb{check macro}
%model
%iff
%%if, and only if,
%there is   a $\varphi$-consistent valuation \valuation 
%for $\prop{\varphi}$ such that 
%if $\B_i\psi_1, \dots, \B_i\psi_k$ are in 
%${\sf sub}(\prop{\varphi})$, $\valuation(\B_i\psi_j)=1$ for all $1\leq j < k$, and 
%$\valuation(\B_i\psi_k)=0$, then either $(\bigwedge^{k-1}_{j=1}\psi_j\wedge\neg\psi_k)$ or 
%$(\neg\psi_j\wedge\psi_k)$ for some $1\leq j < k$
%is satisfied in a $\varphi$-consistent $\EC^{n}$ %\nb{check macro}
% model. 
%%\begin{itemize}
%%\item if $\B_i\psi$ is in ${\sf sub}(\prop{\varphi})$ and
%%$\valuation(\B_i\psi_1)=0$ then $\neg \psi$ is 
%%$\EC^{n}$ satisfiable; and 
%%\item 
%%\end{itemize}
%\end{restatable}
%%
%\begin{proof}
%%\paragraph{Point 2}
%($\Rightarrow$) Suppose that $\prop{\varphi}$ is satisfied in a world $w$ of a $\varphi$-consistent $\EC^{n}$ model 
% $\propmodel = (\propdomain, \{ \propneigh_{i} \}_{i \in I}, \propassign)$. That is, 
% $\propmodel, w\models \prop{\varphi}$. We define a $\varphi$-consistent valuation for 
% $\prop{\varphi}$
% by setting $\nu(\psi)=1$ if $\propmodel, w\models \psi$ and $\nu(\psi) = 0$
% if  $\propmodel, w\not\models \psi$. 
% It is easy to check that $\nu$ is indeed a 
% $\varphi$-consistent valuation (given that $\propmodel$ is a  
% $\varphi$-consistent $\EC^{n}$ model). Assume  $\B_i\psi_1, \dots, \B_i\psi_k$ are in 
%${\sf sub}(\prop{\varphi})$, $\valuation(\B_i\psi_j)=1$ for all $1\leq j < k$, and 
%$\valuation(\B_i\psi_k)=0$.
%%
%Then $\propmodel,w\models \B_i\psi_j$ for all $1\leq j < k$,
% and $\propmodel,w\not\models \B_i\psi_k$.
%By definition, $(\B_i \psi_1 \wedge \ldots \wedge \B_i \psi_{k-1})\rightarrow \B_i (\psi_1\wedge \ldots \wedge \psi_{k-1})$ holds in $\EC^{n}$ models.
%So $\propmodel,w\models \B_i (\psi_1\wedge \ldots \wedge \psi_{k-1})$ 
% and $\propmodel,w\not\models \B_i\psi_k$.
%This means that $\valuation(\B_i(\bigwedge^{k-1}_{j=1}\psi_j))=1$
%while $\valuation(\B_i\psi_k)=0$.
%  By definition, 
% $\propassign(\bigwedge^{k-1}_{j=1}\psi_j)\in \propneigh_i(w)$   and 
%  $\propassign(\psi_k)\not\in \propneigh_i(w)$.
%%Using the same argument of Lemma 3.1 in~\cite{Var2}, it follows that 
%So,
%   $\propassign(\bigwedge^{k-1}_{j=1}\psi_j)\neq \propassign(\psi_k)$.
%Then, %it is easy to see that 
%there 
%   is a world $u$ in the symmetrical difference of these sets 
%   such that $\propmodel,u\models (\bigwedge^{k-1}_{j=1}\psi_j\wedge\neg\psi_k)\vee(\neg(\bigwedge^{k-1}_{j=1}\psi_j)\wedge\psi_k)$. 
%   
%%Assume  $\B_i\psi_1, \dots, \B_i\psi_k$ are in 
%%${\sf sub}(\prop{\varphi})$, $\valuation(\B_i\psi_j)=1$ for all $1\leq j < k$, and 
%%$\valuation(\B_i\psi_k)=0$.
%%By definition, $(\B_i \psi_1 \wedge \ldots \wedge \B_i \psi_{k-1})\rightarrow \B_i %(\psi_1\wedge \ldots \wedge \psi_{k-1})$ holds in $\EC^{n}$ models.
%%\nb{add this def somewhere} 
%%This means that $\valuation(\B_i(\bigwedge^{k-1}_{j=1}\psi_j))=1$.
%
%($\Leftarrow$) Suppose there is a $\varphi$-consistent valuation $\nu$ for $\prop{\varphi}$ such that 
%%there is   a $\varphi$-consistent valuation \valuation 
%%for $\prop{\varphi}$ such that 
%if $\B_i\psi_1, \dots, \B_i\psi_k$ are in 
%${\sf sub}(\prop{\varphi})$, $\valuation(\B_i\psi_j)=1$ for all $1\leq j < k$, and 
%$\valuation(\B_i\psi_k)=0$, then 
%there is a $\varphi$-consistent $\EC^{n}$ model $$\propmodel_{\bigwedge^{k-1}_{j=1}\psi_j,\psi_k}=(\propdomain_{\bigwedge^{k-1}_{j=1}\psi_j,\psi_k},\{ \propneigh_{{\bigwedge^{k-1}_{j=1}\psi_j,\psi_k}_{i}} \}_{i \in I},\propassign_{\bigwedge^{k-1}_{j=1}\psi_j,\psi_k})$$ and a world 
%$w_{\bigwedge^{k-1}_{j=1}\psi_j,\psi_k}\in \propdomain_{\bigwedge^{k-1}_{j=1}\psi_j,\psi_k}$ such that 
%$$\propmodel_{\bigwedge^{k-1}_{j=1}\psi_j,\psi_k}, w_{\bigwedge^{k-1}_{j=1}\psi_j,\psi_k}\models ((\bigwedge^{k-1}_{j=1}\psi_j)\wedge\neg\psi_k)\vee(\neg(\bigwedge^{k-1}_{j=1}\psi_j)\wedge\psi_k).$$ 
%
%%either $(\bigwedge^{k-1}_{j=1}\psi_j\wedge\neg\psi_k)$ or 
%%$(\neg\psi_j\wedge\psi_k)$ for some $1\leq j < k$
%%is satisfied in a $\varphi$-consistent $\EC^{n}$ %\nb{check macro}
%% model. 
%% if $\B_i\psi_1$ and $\B_i\psi_2$ are in 
%%${\sf sub}(\prop{\varphi})$, $\valuation(\B_i\psi_1)=1$, and 
%%$\valuation(\B_i\psi_2)=0$, then 
%%there is a model $\propmodel_{\psi_1,\psi_2}=(\propdomain_{\psi_1,\psi_2},
%%\{ \propneigh_{{\psi_1,\psi_2}_{i}} \}_{i \in I},\propassign_{\psi_1,\psi_2})$
%% and a world 
%%$w_{\psi_1,\psi_2}\in \propdomain_{\psi_1,\psi_2}$ such that 
%%$\propmodel_{\psi_1,\psi_2}, w_{\psi_1,\psi_2}\models (\psi_1\wedge\neg\psi_2)\vee(\neg\psi_1\wedge\psi_2)$. 
%Let $\propmodel_1,\ldots,\propmodel_m$ be an enumeration of the models 
%$\propmodel_{\bigwedge^{k-1}_{j=1}\psi_j,\psi_k}$, as above.  That is, we take one model $\propmodel_{\bigwedge^{k-1}_{l=1}\psi_l,\psi_k}$ for each 
%pair $j=\bigwedge^{k-1}_{l=1}\psi_l,\psi_k$ 
%where $\propmodel_j = (\propdomain_j, \{ \propneigh_{j_{i}} \}_{i \in I},\propassign_j)$, 
%and let $w_1,\ldots,w_m$ be an enumeration of the worlds
% $w_{\bigwedge^{k-1}_{l=1}\psi_l,\psi_k}$, 
%with $j=\bigwedge^{k-1}_{l=1}\psi_l,\psi_k$ and  $w_j\in \propdomain_j$. We assume without loss of generality that $\propdomain_j\cap \propdomain_k=\emptyset$ 
%for $j\neq k$. 
%
%In the following, we define a $\varphi$-consistent $\EC^{n}$ model   $\propmodel = (\propdomain,\{ \propneigh_{i} \}_{i \in I}, \propassign)$ 
%for $\prop{\varphi}$. 
%Intuitively, we construct $\propmodel$ by taking the union of each 
%$\propmodel_j$ with the addition of a new world $w$ that 
%will satisfy $\prop{\varphi}$. 
%We define $\propdomain$ as $\bigcup_{1\leq j\leq n}\propdomain_j\cup \{w\}$, 
%where $w$ is fresh.
%%The tricky part of the proof is to define the assignment $\propassign$. 
%Before defining $\propneigh_{i}$ and $\propassign$, we define the function $J: {\sf sub}(\prop{\varphi})\rightarrow 2^{\Wmc}$
%with $J(\psi)=\bigcup_{0\leq j \leq m} \Vmc_j(\psi)$ for all $\psi\in {\sf sub}(\prop{\varphi})$, where %$I_i$ is as above for $1\leq i\leq n$, 
% %and
%$\Vmc_0: {\sf sub}(\varphi)\rightarrow  2^{\{w\}}$ is the function
%that assigns $\psi$ to $\{w\}$, if $\nu(\psi)=1$, 
%and to $\emptyset$, otherwise ($\Vmc_j$, for $1\leq j\leq m$, is as above).
%By construction, we have that $J(\neg \psi)=\propdomain\setminus J(\psi)$
%and $J(\psi_1\wedge \psi_2)=J(\psi_1)\cap J(\psi_2)$. 
%We define the assignment $\propassign$ as the function 
%$\propassign: \NPr(\varphi)\rightarrow 2^{\Wmc}$ satisfying 
% $\propassign(p_\elaxiom)=J(p_\elaxiom)$ for all $p_\elaxiom\in \NPr(\varphi)$. 
% 
%It remains to define $\propneigh_i$, for
%$i \in I$.
%%$1 \leq i \leq n$. 
%For $u\in \propdomain_j$ we first put $\alpha \subseteq W$ in $\propneigh_i(u)$ 
%precisely when 
%%\begin{itemize}
%%\item 
%$\propmodel_j,u \models \B_i \psi_\alpha$ and $\alpha = J(\psi_\alpha)$
%for some $\B_i\psi_\alpha \in {\sf sub}(\varphi)$.
%Then, we close $\propneigh_i$ under intersection so that $\propmodel$ is a $\EC^{n}$ model. The next two claims establish that $\propneigh_i$ is as expected.
%%; and
%%\item 
%%\end{itemize}
%%We claim that 
%\begin{claim}
%If $\beta \in \Nmc_i(u)$ and $\beta = J(\psi)$ for some $\B_i\psi\in{\sf sub}(\prop{\varphi})$,
%then $\propmodel_j,u\models\B_i\psi$.
%\end{claim}
%Indeed, since $\beta = J(\psi)\in \Nmc_i(u)$, 
%we must have that $\propmodel_j,u\models \B_i\psi_{1,\beta}$, \ldots, 
%$\propmodel_j,u\models \B_i\psi_{m,\beta}$ and  $\beta = \bigcap^{m}_{l=1} J(\psi_{l,\beta})$ for 
%some $\B_i\psi_{1,\beta}, \ldots, \B_i\psi_{m,\beta} \in{\sf sub}(\prop{\varphi})$. %
%Since $\propneigh_i$ is closed under intersection,
%in fact, we have that $\propmodel_j,u \models \B_i (\bigwedge^m_{l=1}\psi_{l,\beta})$.
%%\nb{intersection}
%But since
%$J(\psi)=\bigcap^{m}_{i=1} J(\psi_{i,\beta})$, we also have $\propassign_j(\psi)=\bigcap^{m}_{l=1}\propassign_j(\psi_{l,\beta})$ 
%(recall that $\propdomain_j \cap \propdomain_k = \emptyset$ for  
%$k \neq j$), so $\propmodel_j,u\models \B_i\psi$ iff 
%$\propmodel_j,u \models \B_i (\bigwedge^m_{l=1}\psi_{l,\beta})$.
%It follows that $\propmodel_j,u\models \B_i\psi$.
%
%\medskip
%
%Regarding the fresh 
%world $w$ introduced above in $\propdomain$,
%we first put $\alpha \subseteq \Wmc$ in $\propneigh_i(w)$   precisely 
%when $\nu(\B_i\psi_\alpha) = 1$ and $\alpha = J(\psi_\alpha)$ for some 
%$\B_i\psi_\alpha \in {\sf sub}(\prop{\varphi})$.
%Then, we again close $\propneigh_i$ under intersection so that $\propmodel$ is a $\EC^{n}$ model.
%%We claim that 
%\begin{claim}
%If 
%$\beta \in \propneigh_i(w)$ and $\beta = J(\bigwedge^{k-1}_{l=1}\psi_l)$ for some $\B_i \psi_1, \ldots, \B_i \psi_{k-1} \in{\sf sub}(\prop{\varphi})$ 
%then  $\valuation(\B_i\psi_l)=1$ for all $1\leq l < k$.
%\end{claim}
%Indeed, since $\beta = J(\bigwedge^{k-1}_{l=1}\psi_l)\in \propneigh_i(w)$ 
%we must have that $\nu(\B_i\psi_{1,\beta})=1, \ldots, \nu(\B_i\psi_{m,\beta})=1$ and 
%$\beta = \bigcap^{m}_{i=1} J(\psi_{i,\beta})$ for some 
%$\B_i\psi_{1,\beta}, \ldots, \B_i\psi_{m,\beta} \in{\sf sub}(\prop{\varphi})$. %\nb{intersection}
%Suppose now that $\nu(\bigwedge^{k-1}_{l=1}\psi_l) = 0$. Then, by assumption, there exists a structure
%$\propmodel_j = (\propdomain_j, \{ \propneigh_{j_i} \}_{i \in I}, \propassign_j)$ and a world $w_j
%\in \propdomain_j$ such that $\propmodel_j,w_j \models (\bigwedge^{k-1}_{l=1}\psi_{l,\beta} \wedge \neg(\bigwedge^{k-1}_{l=1}\psi_l))\vee(\neg(\bigwedge^{k-1}_{l=1}\psi_{l,\beta}) \wedge (\bigwedge^{k-1}_{l=1}\psi_l))$. 
%It follows that $\propassign_j(\bigwedge^{k-1}_{l=1}\psi_{l,\beta})\neq \propassign_j(\bigwedge^{k-1}_{l=1}\psi_l)$. 
%Consequently $J(\bigwedge^{k-1}_{l=1}\psi_{l,\beta})\neq J(\bigwedge^{k-1}_{l=1}\psi_l)$, which is a contradiction.  
%%\nb{O :... talk about consistent}
%
%\medskip
%
%We now show by induction on the structure of formulas 
%that $\propassign$ and $J$ agree on ${\sf sub}(\prop{\varphi})$. 
%This holds by construction for atomic propositions. It is easy to deal 
%with propositional connectives, since we know that $J(\neg \psi)=\propdomain\setminus J(\neg \psi)$
%and  $J(\psi_1\wedge \psi_2)=J(\psi_1)\cap J(\psi_2)$ 
%and similarly for $\propassign$. Assume inductively that $\propassign(\psi) = J(\psi)$.
%Suppose first that $u\in J(\B_i\psi)$. Then, either $u=w$ and $\nu(\B_i\psi)=1$
%or $u\in \propdomain_j$ and $\propmodel_j,u\models\B_i\psi$. In either case 
%we have that $J(\psi)\in \propneigh_i(u)$. Since 
%$\propassign(\psi) = J(\psi)$, it follows that $\propmodel,u\models \B_i\psi$, 
%that is, $u\in \propassign(\B_i\psi)$. Suppose now that $u\in \propassign(\B_i\psi)$, 
%that is, $\propmodel,u\models \B_i\psi$, or, equivalently, $\propassign(\psi)\in \propneigh_i(u)$.
%Since $\propassign(\psi) = J(\psi)$ it follows that either $u=w$ and 
%$\nu(\B_i\psi)=1$ or $u\in \propdomain_j$ and $\propmodel_j,u\models \B_i\psi$. 
%In either case we have that $u\in J(\B_i\psi)$. 
%
%Since $\nu(\prop{\varphi})=1$, we have that $w\in J(\prop{\varphi})$, 
%and consequently $w\in \propassign(\prop{\varphi})$. That 
%is, $\propmodel,w\models\prop{\varphi}$. 
%The fact that $\propmodel$ is a $\EC^{n}$ model follows 
%from the definition of $\propneigh_i$.
%The fact that $\propmodel$ is a $\varphi$-consistent  model follows from 
%the fact that $\nu$, used to construct the assignment 
%related to $w$, is $\varphi$-consistent and 
%the models $\propmodel_1,\ldots,\propmodel_m$, used to define 
%the remaining worlds in $\Wmc$, are all $\varphi$-consistent %$\EC^{n}$ 
%models. 
%%Now the lemma follows from Lemma~\ref{lem:general}.
%%Take $\bigwedge^{k-1}_{j=1}\psi_j$ as $\psi'$
%\end{proof}
%
%
%
%\begin{restatable}{lemma}{LemmapropN}\label{lem:proplemmaN}
%A formula $\prop{\varphi}$ is satisfied in a $\varphi$-consistent $\EN^{n}$ model
%iff
%%if, and only if,
%there is   a $\varphi$-consistent valuation \valuation 
%for $\prop{\varphi}$ such that 
%\begin{enumerate}
%\item if $\B_i\psi$ is in ${\sf sub}(\prop{\varphi})$ and
%$\valuation(\B_i\psi)=0$, then $\neg \psi$ 
%is satisfied in a $\varphi$-consistent $\EN^{n}$ model; %and 
%\item if $\B_i\psi_1$ and $\B_i\psi_2$ are in 
%${\sf sub}(\prop{\varphi})$, $\valuation(\B_i\psi_1)=1$, and 
%$\valuation(\B_i\psi_2)=0$, then $(\psi_1\wedge\neg\psi_2)\vee(\neg\psi_1\wedge\psi_2)$
%is satisfied in a $\varphi$-consistent $\EN^{n}$ model. 
%\end{enumerate}
%\end{restatable}
%%
%\begin{proof}%[Sketch]
%We start with proving ($\Rightarrow$). 
%Suppose that $\prop{\varphi}$ is satisfied in a world $w$ of a $\varphi$-consistent $\EN^{n}$ model 
% $\propmodel = (\propdomain, \{ \propneigh_{i} \}_{i \in I}, \propassign)$. That is, 
% $\propmodel, w\models \prop{\varphi}$. We define a $\varphi$-consistent valuation for 
% $\prop{\varphi}$
% by setting $\nu(\psi)=1$ if $\propmodel, w\models \psi$ and $\nu(\psi) = 0$
% if  $\propmodel, w\not\models \psi$. 
% It is easy to check that $\nu$ is indeed a 
% $\varphi$-consistent valuation (given that $\propmodel$ is a  
% $\varphi$-consistent $\EN^{n}$ model). 
%
%\textit{Point 1.}
%%\paragraph{Point 1}
%By definition,  $\B_i {\sf true}$ holds in $\EN^{n}$ models. %\nb{add this def somewhere}
%Since there is a valuation $\valuation$ such that 
%$\valuation(\B_i\psi)=0$ we have that
%$\psi$ cannot be true in all valuations (otherwise
%$\psi\equiv {\sf true}$ would hold and  $\valuation$ 
%would violate %validity of 
%$\B_i {\sf true}$ in $\EN^{n}$ models).
%This means that 
%%We have that
% $\neg \psi$ is 
%$\EN^{n}$  satisfiable.
%
%\textit{Point 2.}
%%\paragraph{Point 2}
%Assume that 
% $\B_i\psi_1$ and $\B_i\psi_2$ are in ${\sf sub}(\prop{\varphi})$, 
% $\nu(\B_i\psi_1)=1$ and $\nu(\B_i\psi_2)=0$. Then $\propmodel,w\models \B_i\psi_1$
% and $\propmodel,w\not\models \B_i\psi_2$. Thus, by definition, 
% $\propassign(\psi_1)\in \propneigh_i(w)$ and 
%  $\propassign(\psi_2)\not\in \propneigh_i(w)$.
%%Using the same argument of Lemma 3.1 in~\cite{Var2}, it follows that 
%So,
%   $\propassign(\psi_1)\neq \propassign(\psi_2)$.
%Then, %it is easy to see that 
%there 
%   is a world $u$ in the symmetrical difference of these sets 
%   such that $\propmodel,u\models (\psi_1\wedge\neg\psi_2)\vee(\neg\psi_1\wedge\psi_2)$. 
%   
%%\medskip
%
%The proof of the converse ($\Leftarrow$)
%%for  the second bullet point 
%is as follows. %in Lemma~\ref{lem:general}.
%%It holds in $\EC^{n}$ models because $\B_i {\sf true}$
% Suppose there is a $\varphi$-consistent valuation $\nu$ for $\prop{\varphi}$ such that 
% Point~1
%%\textbf{Point 1}
%and
%Point~2
%%\textbf{Point 2}
%hold. 
%That is, 
%\begin{itemize}
%\item if $\B_i\psi$ is in ${\sf sub}(\prop{\varphi})$ and
%$\valuation(\B_i\psi)=0$ then 
%there is a $\varphi$-consistent $\EN^{n}$ model $\propmodel_{\psi}=(\propdomain_{\psi},
%\{ \propneigh_{{\psi}_{i}} \}_{i \in I},\propassign_{\psi})$
%%
% and a world 
%$w_{\psi}\in \propdomain_{\psi}$ such that 
%$\propmodel_{\psi}, w_{\psi}\models \neg\psi$; and 
%%$\neg \psi$ is 
%%$\EN^{n}$ satisfiable; and
%\item if $\B_i\psi_1$ and $\B_i\psi_2$ are in 
%${\sf sub}(\prop{\varphi})$, $\valuation(\B_i\psi_1)=1$, and 
%$\valuation(\B_i\psi_2)=0$, then 
%there is a $\varphi$-consistent $\EN^{n}$ model $\propmodel_{\psi_1,\psi_2}=(\propdomain_{\psi_1,\psi_2},
%\{ \propneigh_{{\psi_1,\psi_2}_{i}} \}_{i \in I},\propassign_{\psi_1,\psi_2})$
%%
% and a world 
%$w_{\psi_1,\psi_2}\in \propdomain_{\psi_1,\psi_2}$ such that 
%$\propmodel_{\psi_1,\psi_2}, w_{\psi_1,\psi_2}\models (\psi_1\wedge\neg\psi_2)\vee(\neg\psi_1\wedge\psi_2)$. 
%\end{itemize}
%Let $\propmodel_1,\ldots,\propmodel_m$ be an enumeration of $\EN^{n}$ models 
%$\propmodel_{\psi}$ and $\propmodel_{\psi_1,\psi_2}$, as above.  That is, we take one model $\propmodel_{\psi}$ and one model $\propmodel_{\psi_1,\psi_2}$ for each such
%subformula $j=\psi$ and pair of subformulas $j=\psi_1,\psi_2$ where $\propmodel_j = (\propdomain_j, \{ \propneigh_{j_{i}} \}_{i \in I},\propassign_j)$, 
%and let $w_1,\ldots,w_m$ be an enumeration of the worlds $w_{\psi}$ and $w_{\psi_1,\psi_2}$, 
%with $w_j\in \propdomain_j$. Assume without loss of generality that $\propdomain_j\cap \propdomain_k=\emptyset$ 
%for $j\neq k$. 
%%
%In the following, we define a $\varphi$-consistent $\EN^{n}$ model   $\propmodel = (\propdomain,\{ \propneigh_{i} \}_{i \in I}, \propassign)$ 
%for $\prop{\varphi}$. 
%
%Intuitively, we construct $\propmodel$ by taking the union of each 
%$\propmodel_j$ with the addition of a new world $w$ that 
%will satisfy $\prop{\varphi}$. 
%We define $\propdomain$ as $\bigcup_{1\leq j\leq n}\propdomain_j\cup \{w\}$, 
%where $w$ is fresh.
%%The tricky part of the proof is to define the assignment $\propassign$. 
%Before defining $\propneigh_{i}$ and $\propassign$, we define the function $J: {\sf sub}(\prop{\varphi})\rightarrow 2^{\Wmc}$
%with $J(\psi)=\bigcup_{0\leq j \leq m} \Vmc_j(\psi)$ for all $\psi\in {\sf sub}(\prop{\varphi})$, where %$I_i$ is as above for $1\leq i\leq n$, 
% %and
%$\Vmc_0: {\sf sub}(\prop{\varphi})\rightarrow  2^{\{w\}}$ is the function
%that assigns $\psi$ to $\{w\}$, if $\nu(\psi)=1$, 
%and to $\emptyset$, otherwise ($\Vmc_j$, for $1\leq j\leq m$, is as above).
%By construction, we have that $J(\neg \psi)=\propdomain\setminus J(\psi)$
%and $J(\psi_1\wedge \psi_2)=J(\psi_1)\cap J(\psi_2)$. 
%We define the assignment $\propassign$ as the function 
%$\propassign: \NPr(\varphi)\rightarrow 2^{\Wmc}$ satisfying 
% $\propassign(p_\elaxiom)=J(p_\elaxiom)$ for all $p_\elaxiom\in \NPr(\varphi)$. 
% 
%It remains to define $\propneigh_i$,
%for
%$i \in I$.
%%$1 \leq i \leq n$. 
%For $u\in \propdomain_j$ we put  $\alpha \subseteq \Wmc$ in $\propneigh_i(u)$ 
%precisely when $\alpha=\Wmc$, or, $\alpha = J(\psi_{\alpha})$ and
%$\propmodel_j,u \models \B_i \psi_{\alpha}$,  
%for some $\B_i\psi_{\alpha} \in {\sf sub}(\prop{\varphi})$.
%The next two claims establish that $\propneigh_i$ is as expected. 
%%We claim that 
%\begin{claim}
%If $\beta \in \Nmc_i(u)$ and $\beta = J(\psi)$ for some $\B_i\psi\in{\sf sub}(\prop{\varphi})$,
%then $\propmodel_j,u\models\B_i\psi$.
%\end{claim}
%\begin{proof}[Proof of Claim]
%Indeed, since $\beta = J(\psi)\in \Nmc_i(u)$, 
%we must have that either $\beta=\Wmc$ or $\propmodel_j,u\models \B_i\psi_{\beta}$ and  $\beta = J(\psi_{\beta})$ for 
%some $\B_i\psi_\beta \in{\sf sub}(\prop{\varphi})$.
%In the former case, as all $\propmodel_j$ models are $\EN^{n}$ models, 
%we have that $\propmodel_j,u\models \B_i\psi$.
%% (in this case, $\psi$
%%is true in all worlds in all $\EN^{n}$ models).
%In the latter,  since
%$J(\psi)=J(\psi_\beta)$, we also have $\propassign_j(\psi)=\propassign_j(\psi_\beta)$ 
%(recall that $\propdomain_j \cap \propdomain_k = \emptyset$ for  
%$k \neq j$), so $\propmodel_j,u\models \B_i\psi$ iff 
%$\propmodel_j,u \models \B_i \psi_\beta$.
%It follows that $\propmodel_j,u\models \B_i\psi$.
%\end{proof}
%
%Also, we put $\alpha \subseteq \Wmc$ in $\propneigh_i(w)$ (recall $w$ is the fresh 
%world introduced above in $\propdomain$) precisely 
%when $\alpha = \Wmc$ or $\nu(\B_i\psi_\alpha) = 1$ and $\alpha = J(\psi_\alpha)$ for some 
%$\B_i\psi_\alpha \in {\sf sub}(\prop{\varphi})$.
%
%\begin{claim}
%If 
%$\beta \in \propneigh_i(w)$ and $\beta = J(\psi)$ for some $\B_i \psi \in{\sf sub}(\prop{\varphi})$ 
%then $\nu(\B_i \psi) = 1$.
%\end{claim}
%\begin{proof}[Proof of Claim]
%Indeed, since $\beta = J(\psi)\in \propneigh_i(w)$ 
%we must have that either $\beta=\Wmc$
%or $\nu(\B_i\psi_\beta)=1$ and $\beta = J(\psi_\beta)$ for some 
%$\B_i\psi_\beta \in{\sf sub}(\prop{\varphi})$. 
%Suppose  that $\nu(\B_i\psi) = 0$ and $\beta\neq\Wmc$.
%Then, by assumption, there exists a $\varphi$-consistent $\EN^{n}$ model
%$\propmodel_j = (\propdomain_j, \{ \propneigh_{j_i} \}_{i \in I}, \propassign_j)$ and a world $w_j
%\in \propdomain_j$ such that $\propmodel_j,w_j \models (\psi_\beta \wedge \neg\psi)\vee(\neg\psi_\beta \wedge \psi)$. 
%It follows that $\propassign_j(\psi_\beta)\neq \propassign_j(\psi)$. 
%Consequently $J(\psi_\beta)\neq J(\psi)$, which is a contradiction.  
%Now, suppose   that $\nu(\B_i\psi) = 0$ and $\beta=\Wmc$.
%By assumption,
%%$\neg \psi$ is  satisfiable in a $\varphi$-consistent $\EN^{n}$ model.
%%This means that 
%there exists a $\varphi$-consistent $\EN^{n}$ model
%$\propmodel_j = (\propdomain_j, \{ \propneigh_{j_i} \}_{i \in I}, \propassign_j)$ and a world $w_j
%\in \propdomain_j$ such that $\propmodel_j,w_j \models \neg\psi$.
%It follows that $\propassign_j(\psi)\neq \Wmc$. Consequently $\beta= J(\psi)\neq \Wmc$, which is a contradiction.  Then,  $\nu(\B_i \psi) = 1$, as required.
%%\nb{changing}
%%, which means that
%%$\beta=\Wmc$.
%%\nb{O :... talk about consistent}
%\end{proof}
%
%We now show by induction on the structure of formulas 
%that $\propassign$ and $J$ agree on ${\sf sub}(\prop{\varphi})$. 
%This holds by construction for atomic propositions. It is easy to deal 
%with propositional connectives, since we know that $J(\neg \psi)=\propdomain\setminus J(\neg \psi)$
%and  $J(\psi_1\wedge \psi_2)=J(\psi_1)\cap J(\psi_2)$ 
%and similarly for $\propassign$. Assume inductively that $\propassign(\psi) = J(\psi)$.
%Suppose first that $u\in J(\B_i\psi)$. Then, either $u=w$ and $\nu(\B_i\psi)=1$
%or $u\in \propdomain_j$ and $\propmodel_j,u\models\B_i\psi$. In either case 
%we have that $J(\psi)\in \propneigh_i(u)$. Since 
%$\propassign(\psi) = J(\psi)$, it follows that $\propmodel,u\models \B_i\psi$, 
%that is, $u\in \propassign(\B_i\psi)$. Suppose now that $u\in \propassign(\B_i\psi)$, 
%that is, $\propmodel,u\models \B_i\psi$, or, equivalently, $\propassign(\psi)\in \propneigh_i(u)$.
%Since $\propassign(\psi) = J(\psi)$ it follows that either $u=w$ and 
%$\nu(\B_i\psi)=1$ or $u\in \propdomain_j$ and $\propmodel_j,u\models \B_i\psi$. 
%In either case we have that $u\in J(\B_i\psi)$. 
%
%Since $\nu(\prop{\varphi})=1$, we have that $w\in J(\prop{\varphi})$, 
%and consequently $w\in \propassign(\prop{\varphi})$. That 
%is, $\propmodel,w\models\prop{\varphi}$. 
%The fact that $\propmodel$ is $\varphi$-consistent follows from 
%the fact that $\nu$, used to construct the assignment 
%related to $w$, is $\varphi$-consistent and 
%the models $\propmodel_1,\ldots,\propmodel_m$, used to define 
%the remaining worlds in $\Wmc$, are all $\varphi$-consistent.
%The fact that  $\propmodel$ contains the unit is by construction, that is,
%we defined $\propmodel$
%%. That is, 
%so that for all 
%$i\in [1,n]$ and all $w\in\Wmc$, we have that $\Wmc\in\Nmc_i(w)$. 
%Thus, $\propmodel$ is a $\varphi$-consistent $\EN^{n}$ model that satisfies 
%$\prop{\varphi}$, as required. 
%\end{proof}
%
%
%
%To determine satisfiability of $\prop{\varphi}$ in a $\varphi$-consistent model, we use Lemma~\ref{lem:prop} and the characterizations above. 
%To establish complexity results, %for the upper bound of $\CALCg$, 
%we use the fact that there are only quadratically many   subformulas in $\prop{\varphi}$. 
%Satisfiability in
%\ALC is \ExpTime-complete  and so, one can determine in exponential time
%whether a valuation is $\varphi$-consistent. For an \ExpTime~upper bound, one can
%deterministically compute all possible $\varphi$-consistent valuations for 
%$(\bigwedge^{k-1}_{j=1}\psi_j\wedge\neg\psi_k)$ (or $(\psi_1 \wedge \neg \psi_2)$) and
%decide satisfiability of $\prop{\varphi}$ by a $\varphi$-consistent model using a bottom-up strategy (as in~\cite{Baader:2012:LOD:2287718.2287721}). Since satisfiability in \ALC is \ExpTime-hard, our upper bound is tight.
%
%\begin{theorem}
%The $\CALCg$ and $\NALCg$ formula satisfiability problems on constant domain neighbourhood models are \ExpTime-complete.
%%Satisfiability in  $\CALCg$  and $\NALCg$ is \ExpTime-complete.
%\end{theorem}
%
%\begin{restatable}{lemma}{LemmapropP}\label{lem:proplemmaP}
%	A formula $\prop{\varphi}$ is satisfied in a $\varphi$-consistent $\EP^{n}$ model
%	iff
%	%if, and only if,
%	there is   a $\varphi$-consistent valuation \valuation 
%	for $\prop{\varphi}$ such that 
%	\begin{enumerate}
%		\item if $\B_i\psi$ is in ${\sf sub}(\prop{\varphi})$ and
%		$\valuation(\B_i\psi)=1$, then $\psi$ 
%		is satisfied in a $\varphi$-consistent $\EP^{n}$ model; %and 
%		\item if $\B_i\psi_1$ and $\B_i\psi_2$ are in 
%		${\sf sub}(\prop{\varphi})$, $\valuation(\B_i\psi_1)=1$, and 
%		$\valuation(\B_i\psi_2)=0$, then $(\psi_1\wedge\neg\psi_2)\vee(\neg\psi_1\wedge\psi_2)$
%		is satisfied in a $\varphi$-consistent $\EP^{n}$ model. 
%	\end{enumerate}
%\end{restatable}
%
%\begin{restatable}{lemma}{LemmapropQ}\label{lem:proplemmaQ}
%	A formula $\prop{\varphi}$ is satisfied in a $\varphi$-consistent $\EQ^{n}$ model
%	iff
%	%if, and only if,
%	there is   a $\varphi$-consistent valuation \valuation 
%	for $\prop{\varphi}$ such that 
%	\begin{enumerate}
%		\item if $\B_i\psi$ is in ${\sf sub}(\prop{\varphi})$ and
%		$\valuation(\B_i\psi)=1$, then $\neg\psi$ 
%		is satisfied in a $\varphi$-consistent $\EQ^{n}$ model; %and 
%		\item if $\B_i\psi_1$ and $\B_i\psi_2$ are in 
%		${\sf sub}(\prop{\varphi})$, $\valuation(\B_i\psi_1)=1$, and 
%		$\valuation(\B_i\psi_2)=0$, then $(\psi_1\wedge\neg\psi_2)\vee(\neg\psi_1\wedge\psi_2)$
%		is satisfied in a $\varphi$-consistent $\EQ^{n}$ model. 
%	\end{enumerate}
%\end{restatable}
%
%\begin{restatable}{lemma}{LemmapropD}\label{lem:proplemmaD}
%	A formula $\prop{\varphi}$ is satisfied in a $\varphi$-consistent $\ED^{n}$ model
%	iff
%	%if, and only if,
%	there is   a $\varphi$-consistent valuation \valuation 
%	for $\prop{\varphi}$ such that 
%	\begin{enumerate}
%		\item if $\B_i\psi_1$ and $\B_i\psi_2$ are in ${\sf sub}(\prop{\varphi})$,
%		$\valuation(\B_i\psi_1)=1$ and $\valuation(\B_i\psi_2)=1$, then
%		$(\psi_1\wedge\psi_2)\vee(\neg\psi_1\wedge\neg\psi_2)$   
%		is satisfied in a $\varphi$-consistent $\ED^{n}$ model; %and 
%		\item if $\B_i\psi_1$ and $\B_i\psi_2$ are in 
%		${\sf sub}(\prop{\varphi})$, $\valuation(\B_i\psi_1)=1$, and 
%		$\valuation(\B_i\psi_2)=0$, then $(\psi_1\wedge\neg\psi_2)\vee(\neg\psi_1\wedge\psi_2)$
%		is satisfied in a $\varphi$-consistent $\ED^{n}$ model. 
%	\end{enumerate}
%\end{restatable}
%
%\begin{restatable}{lemma}{LemmapropT}\label{lem:proplemmaT}
%	A formula $\prop{\varphi}$ is satisfied in a $\varphi$-consistent $\ET^{n}$ model
%	iff
%	%if, and only if,
%	there is   a $\varphi$-consistent valuation \valuation 
%	for $\prop{\varphi}$ such that 
%	\begin{enumerate}
%		\item if $\B_i\psi$ is in ${\sf sub}(\prop{\varphi})$ and 
%		$\valuation(\B_i\psi)=1$   then
%		$\psi$   \todo{to check}
%		is satisfied in a $\varphi$-consistent $\ET^{n}$ model; %and 
%		\item if $\B_i\psi_1$ and $\B_i\psi_2$ are in 
%		${\sf sub}(\prop{\varphi})$, $\valuation(\B_i\psi_1)=1$, and 
%		$\valuation(\B_i\psi_2)=0$, then $(\psi_1\wedge\neg\psi_2)\vee(\neg\psi_1\wedge\psi_2)$
%		is satisfied in a $\varphi$-consistent $\ET^{n}$ model. 
%	\end{enumerate}
%\end{restatable}
%%\begin{restatable}{lemma}{LemmapropCMN}\label{lem:proplemmaCMN}
%%	A formula $\prop{\varphi}$ is satisfied in a $\varphi$-consistent $\EMCN^{n}$ model
%%	iff
%%	%if, and only if,
%%	there is   a $\varphi$-consistent valuation \valuation 
%%	for $\prop{\varphi}$ such that 
%%	%\begin{enumerate}
%%		%\item 
%%		if $\B_i\psi$ is in ${\sf sub}(\prop{\varphi})$ and
%%		$\valuation(\B_i\psi)=0$, then both $\bigwedge_{\valuation(\B_i\psi')=1}  \B_i\psi'\wedge\neg\psi$ and $\bigwedge_{\valuation(\B_i\psi')=1}  \psi'\wedge\neg\psi $
%%		are satisfied in a $\varphi$-consistent $\EMCN^{n}$ model. %; %and 
%%	%	\item if $\B_i\psi_1$ and $\B_i\psi_2$ are in 
%%%		${\sf sub}(\prop{\varphi})$, $\valuation(\B_i\psi_1)=1$, and %
%%%		$\valuation(\B_i\psi_2)=0$, then $(\psi_1\wedge\neg\psi_2)\vee(\neg\psi_1\wedge\psi_2)$
%%%		is satisfied in a $\varphi$-consistent $\EMCN^{n}$ model. 
%%	%\end{enumerate}
%%\end{restatable}
%%\begin{proof} 
%%($\Rightarrow$) Suppose that $\prop{\varphi}$ is satisfied in a world $w$ of a $\varphi$-consistent $\EMCN^{n}$ model 
%%$\propmodel = (\propdomain, \{ \propneigh_{i} \}_{i \in I}, \propassign)$. That is, 
%%$\propmodel, w\models \prop{\varphi}$. We define a $\varphi$-consistent valuation for 
%%$\prop{\varphi}$
%%by setting $\nu(\psi)=1$ if $\propmodel, w\models \psi$ and $\nu(\psi) = 0$
%%if  $\propmodel, w\not\models \psi$. 
%%It is easy to check that $\nu$ is indeed a 
%%$\varphi$-consistent valuation (given that $\propmodel$ is a  
%%$\varphi$-consistent $\EMCN^{n}$ model). \nb{Ana: to check}
%%Assume   $\B_i\psi$ is in ${\sf sub}(\prop{\varphi})$ and
%%$\valuation(\B_i\psi)=0$.
%%%
%%Then %$\propmodel,w\models \B_i\psi_j$ for all $1\leq j < k$,
%%%and 
%%$\propmodel,w\not\models \B_i\psi$.
%%By necessitation in $\EMCN^{n}$ models, $\propmodel,w\not\models \psi$.
%%So $\propmodel,w\models \neg\psi$.
%%By definition, $\propmodel,w\models \B_i\psi'$ when $\valuation(\B_i\psi')=1$.
%%This means that $\bigwedge_{\valuation(\B_i\psi')=1}  \B_i\psi'\wedge\neg\psi$ is  satisfied in a $\varphi$-consistent $\EMCN^{n}$ model.
%%%So $\propmodel,w\models \B_i (\psi_1\wedge \ldots \wedge \psi_{k-1})$ 
%%%and $\propmodel,w\not\models \B_i\psi_k$.
%%%This means that $\valuation(\B_i(\bigwedge^{k-1}_{j=1}\psi_j))=1$
%%%while $\valuation(\B_i\psi_k)=0$.
%%%
%%
%%We now argue about $\bigwedge_{\valuation(\B_i\psi')=1}  \psi'\wedge\neg\psi $. \nb{Ana: on going}
%%By definition, 
%%$\propassign(\bigwedge_{\valuation(\B_i\psi')=1}\psi')\in \propneigh_i(w)$   and 
%%$\propassign(\psi)\not\in \propneigh_i(w)$.
%%%Using the same argument of Lemma 3.1 in~\cite{Var2}, it follows that 
%%So,
%%$\propassign(\bigwedge^{k-1}_{j=1}\psi_j)\neq \propassign(\psi_k)$.
%%Then, %it is easy to see that 
%%there 
%%is a world $u$ in the symmetrical difference of these sets 
%%such that $\propmodel,u\models (\bigwedge_{\valuation(\B_i\psi')=1}\psi'\wedge\neg\psi)\vee(\neg(\bigwedge_{\valuation(\B_i\psi')=1}\psi')\wedge\psi)$.  
%%\end{proof}
%
%
%%\subsection{{\color{red}{Unified ``Vardi's lemma'' for all logics}}}
%
%
%
%
%
%\newpage

 
%%%%%%%%%%%%%%%%%%%%%%%%%%%%%%%%%%%%%%%%%%%%%%%%%%%%%%%%%%%%%%%%%%%%%%

%\section{Applications and Extensions}
%
%\subsection{Somebody Knows}



\section{Discussion}
\label{sec:discuss}

We investigated reasoning in non-normal modal description logics,
%After providing motivations and preliminaries for these logics, we have focused on the following two aspects.
focussing on:
$(i)$
%terminating, sound and complete
tableaux algorithms to check satisfiability of multi-modal description logics formulas in varying domain neighbourhood models based on classes of frames
for
%that characterise
39 different
non-normal
systems;
%relevant to agency, epistemic, and deontic scenarios;
%(based on conditions on neighbourhood frames that extend the \emph{classical cube}~\cite{LelPim19}, obtained by combinations of the $\mathbf{E}$-, $\mathbf{M}$-, $\mathbf{C}$-, and $\mathbf{N}$-conditions, with the $\mathbf{T}$-, $\mathbf{D}$-, $\mathbf{P}$-, and $\mathbf{Q}$-conditions, which are relevant to agency, epistemic, and deontic scenarios);
$(ii)$ complexity of satisfiability restricted to fragments with modal operators applied only over formulas,
%(thus without modalised concepts)
%and over neighbourhood models with varying domains;
and interpreted on varying domain models;
$(iii)$ preliminary reduction of formula satisfiability for two non-normal modal description logics to satisfiability in the standard relational semantics on a constant domain.
We now discuss possible future work.
%
%We now discuss future research directions in connection with relevant related work.
%

First, we intend to devise tableaux for formula satisfiability on neighbourhood models with constant domain, by solving the problem of newly introduced variables that do not occur in other previously expanded labelled constraints systems.
For instance, by applying the $\mathbf{M}^{n}_{\ALC}$-rules to the $n$-labelled constraint system $S_{n} = \{ n : \Diamond_{i} \exists r. A(x), \Box_{i} \lnot A(x) \}$, we get the $m$-labelled constraint system $S_{m} = \{ m : \exists r. A(x),  m : \lnot A(x), m : r(x,y), m : A(y) \}$.
The fresh variable $y$ in $S_{m}$ does not allow for the direct extraction of a constant domain model,
as no object in the domain of the world associated with $S_{n}$ would be capable of representing $y$ correctly.
%
An alternative approach involves \emph{quasimodels}~\cite{GabEtAl03}, to characterise satisfiability on constant domain
%neighbourhood
models in terms of structures representing ``abstractions'' of the actual models of a formula.
Objects across worlds can be represented by means of \emph{runs}, i.e., functions to guarantee
%that they do not violate
their modal properties and the constant domain assumption.
A similar strategy is presented in~\cite{SeyErd09,SeyJam09,SeyJam10},
where the definition of runs (which is not carried out in detail) involves the introduction of suitable world ``copies''.
%However, such a definition is not fully carried out.
We conjecture that a quasimodel-based approach with \emph{marked variables}, as
illustrated in~\cite{GabEtAl03}, can also be adopted to solve the constant domain model extraction issue.

%%% PREVIOUS VERSION
%First, we would like to adapt our tableau algorithms to check formula satisfiability on neighbourhood models with constant domain.
%This
%%Such an adaptation requires
%requires to address the
%%problem of the
%introduction of fresh variables
%%introduced in a certain labelled constraint system
%that do not occur
%%not occurring
%in other previously expanded labelled constraints systems.
%For instance, by applying the $\mathbf{M}^{n}_{\ALC}$-rules to the $n$-labelled constraint system $S_{n} = \{ n : \Diamond_{i} \exists r. A(x), \Box_{i} \lnot A(x) \}$, we get the $m$-labelled constraint system $S_{m} = \{ m : \exists r. A(x),  m : \lnot A(x), m : r(x,y), m : A(y) \}$.
%The
%%introduction of the
%fresh
%variable $y$ in $S_{m}$ does not allow us to directly extract a model with constant domain,
%%from a completion set with labelled constraint systems of this kind,
%since there would be no object in the domain of the world associated with $S_{n}$ capable of representing $y$ correctly.
%%the variable $y$ introduced in $S_{m}$.
%%
%A possible solution could involve suitably defined \emph{quasimodels}~\cite{GabEtAl03}, to equivalently characterise satisfiability on constant domain neighbourhood models in terms of structures representing ``abstractions'' of the actual models of a formula.
%The representation of objects across worlds would be via suitably defined functions, called \emph{runs}, to guarantee that they do not violate their modal properties and the constant domain assumption.
%%These notions could then be used in the soundness proof of the tableau algorithms, where one starts from from a complete and clash-free completion set to construct a quasimodel for a formula (in place of a concrete model), in turn implying its satisfiability.
%A similar approach is followed by~\cite{SeyErd09,SeyJam09,SeyJam10}
%with suitable ``copies'' of worlds introduced to address the problem of the definition of runs.
%%representing the behaviour of domain objects across worlds.
%In these works, however,
%such a definition is not carried out in detail.
%%it is not made explicit how such a definition should be carried out in detail.
%We conjecture that a quasimodel-based approach with \emph{marked variables}, as
%illustrated by~\cite{GabEtAl03},
%%~\shortcite{GabEtAl03},
%can be
%%fruitfully
%adopted to solve the constant domain model extraction issue.
%%from a complete and clash-free completion set for a formula.



Moreover, we aim at tight complexities for $\LnALC$ satisfiability, both in varying and in constant domain 
%neighbourhood
models.
%This problem requires in particular to develop proof strategies for the lower bounds.
While $\ALC$ formula satisfiability is $\ExpTime$-complete, it is unclear whether the upper bound for $\LnALC$ on varying or constant domain
neighbourhood
models can be improved to $\ExpTime$-membership, for any $\Lvar \in \Log$.
%Note that,
%It has to be noted that,
At the propositional level, the formula satisfiability problem for the systems based on the $L$-condition, with $\mathbf{C} \not \in L$, is $\NP$-complete, rising to $\PSpace$ if the $\mathbf{C}$-condition is included~\cite{Var2}.
%At the propositional level, the formula satisfiability problem for the systems based on combinations of the $\mathbf{E}$-, $\mathbf{M}$-, and $\mathbf{N}$-conditions is
%%known to be
%$\NP$-complete, with a rise to $\PSpace$-completeness for systems respecting the $\mathbf{C}$-condition~\cite{Var2}.
For normal modal description logics, instead, the (tight) $\NExpTime$-hardness results are based on complexity proofs of \emph{product logics} over relational product frames~\cite{GabEtAl03}, and cannot be immediately adapted to neighbourhood semantics, where an analogous notion of product is not yet well understood.
Nonetheless, we conjecture that the $\NExpTime$-hardness known for,
%some normal modal DLs with relational semantics,
e.g., $\mathbf{K}_\mathcal{ALC}$ on constant domain relational models, also holds in the neighbourhood case, at least in presence of the $\mathbf{C}$-condition.

%%% PREVIOUS VERSION
%Moreover, we aim at tight
%complexity results
%for $\LnALC$ formula satisfiability, both in varying and in constant domain neighbourhood models.
%%This problem requires in particular to develop proof strategies for the lower bounds.
%It is known that $\ALC$ formula satisfiability is $\ExpTime$-complete.
%However, we do not know whether the upper bound for $\LnALC$ formula satisfiability problem on varying or constant domain neighbourhood models can be improved to $\ExpTime$-membership, for any $\Lvar \in \Log$.
%%Note that,
%%It has to be noted that,
%At the propositional level, the formula satisfiability problem for the systems based on the $L$-condition, with $\mathbf{C} \not \in L$ , is $\NP$-complete, rising to $\PSpace$-completeness if the $\mathbf{C}$-condition is included~\cite{Var2}.
%%At the propositional level, the formula satisfiability problem for the systems based on combinations of the $\mathbf{E}$-, $\mathbf{M}$-, and $\mathbf{N}$-conditions is
%%%known to be
%%$\NP$-complete, with a rise to $\PSpace$-completeness for systems respecting the $\mathbf{C}$-condition~\cite{Var2}.



Finally, we plan to study: non-normal modal description logics in \emph{coalitional} and \emph{strategic} settings~\cite{Pau,Tro,SeyJam09}, with an interplay between abilities and powers of \emph{groups} of agents, rather than single ones;
%Further dimensions to explore concern:
additional description logics constructs (e.g. \emph{nominals}, \emph{inverse roles}, or \emph{number restrictions}~\cite{BaaEtAl17}); and \emph{interactions between modalities}, with axioms expressing e.g. that an agent \emph{can do} anything they \emph{actually do}, by means of formulas of the form $\mathbb{D}_{i}C \sqsubseteq \mathbb{C}_{i}C$ or $\mathbb{D}_{i}\varphi \to \mathbb{C}_{i}\varphi$.




%\todo[inline,caption={}]{
%M: todo add Long-term todos
%\begin{itemize}
%	\item Tableaux for interacting modalities + coalitions + other DL constructs (e.g. $\mathcal{ALCOIQ}$)
%	\item Tableaux for constant domains (marked variables strategy?)
%	\item (Un-)decidability with global roles?
%	\item Matching upper and lower bounds for satisfiability ($\ExpTime$-membership from Donini and Massacci $\ALC$ tableaux? $\NExpTime$-hardness from products?)
%\end{itemize}
%}






%\section{Conclusion}
%\label{sec:conc}
%
%
%We have investigated reasoning in non-normal modal description logics,
%%After providing motivations and preliminaries for these logics, we have focused on the following two aspects.
%focussing on the following three aspects.
%First, we have introduced terminating, sound and complete tableaux algorithms to check satisfiability of multi-modal description logics formulas in varying domain neighbourhood models based on classes of frames that characterise 39 different non-normal systems.
%We considered conditions on neighbourhood frames that extend the \emph{classical cube}~\cite{LelPim19}, obtained by combinations of the $\mathbf{E}$-, $\mathbf{M}$-, $\mathbf{C}$-, and $\mathbf{N}$-conditions, with the $\mathbf{T}$-, $\mathbf{D}$-, $\mathbf{P}$-, and $\mathbf{Q}$-conditions, which are relevant to agency, epistemic, and deontic scenarios.
%We have then studied the complexity of satisfiability restricted to fragments where modal operators can be applied to formulas only (thus without modalised concepts) and interpreted on neighbourhood models with  varying domains.
%Finally, we have moved first steps towards constant domains, providing a reduction of formula satisfiability for two non-normal modal description logics to satisfiability in the standard relational semantics on a constant domain.







%\newpage

\section*{Acknowledgements}

This research has been partially supported
by the Province of Bolzano and DFG through the project D2G2 (DFG grant n.~500249124).
Andrea Mazzullo acknowledges the support of the MUR PNRR project FAIR - Future AI Research (PE00000013) funded by the NextGenerationEU.
Ana Ozaki is supported by the Research Council of Norway, project number 316022.


\bibliographystyle{splncs04}
\bibliography{bib_nnmdl}


\newpage

\appendix


\section{Modelling Scenario
%with Non-Normal Modal Description Logics
}
\label{sec:model}

%To illustrate further the different principles validated by between relational and neighbourhood semantics,
%and to understand better the shortcomings of the former,
In the following,
%we illustrate with a running example
we present the modelling of an example scenario in the classic domain of multi-agent purchase choreography~\cite{MonEtAl10}.
The aim is twofold.
First, it displays some of the limitations of modalities defined on relational frames, motivating the adoption of neighbourhood semantics in modal extensions of description logics.
Second, it illustrates the expressivity of (non-normal) modal description logic languages, showing interactions between modalities and the constructs of the standard description logic $\ALC$.
%

Our multi-agent setting involves
a customer $\mathit{c}$,
a marketplace
$\mathit{m}$,
a seller $\mathit{s}$,
and
a warehouse $\mathit{w}$,
with
%formalised using a modal description logic language with the
agency operators $\mathbb{D}_i$ and $\mathbb{C}_i$, for $i \in \{ c, m, s, w \}$, read as `agent $i$ does/makes' and `agent $i$ can do/make', respectively~\cite{Elg,GovernatoriRotolo}. 
Concept names $\mathsf{Ord}$, $\mathsf{Prod}$, and $\mathsf{InCatal}$ are used to represent, respectively, orders, products, and the class of objects displayed as in-catalogue, while $\mathsf{req}$ is a role name for the request relation.
The formula
%\begin{equation}
$
\label{eq:1eq}
\mathsf{Ord} \equiv \mathbb{D}_{c}\exists \mathsf{req}.(\mathsf{Prod} \sqcap \mathsf{InCatal} )
$
%\end{equation}
defines an order $\mathsf{Ord}$ as a request made by customer $c$ of an in-catalogue
%($\mathsf{InCatal}$)
product.
%
Using the concept name $\mathsf{Confirm}$ to represent the class of objects that are confirmed, we also have that
%valid formula:
%\begin{equation}
$
\label{eq:2eq}
\exists \mathsf{req}.(\mathsf{Prod} \sqcap \mathsf{InCatal} ) \sqsubseteq \mathsf{Confirm} \sqcup \lnot \mathsf{Confirm},
$
%\end{equation}
meaning that any request of an in-catalogue product is either confirmed or not confirmed.
However, relational semantics validates the so-called \emph{$\mathbf{M}$-principle} (often called \emph{monotonicity}) as well, according to which $C \sqsubseteq D$ always entails $\mathbb{D}_{c} C \sqsubseteq \mathbb{D}_{c} D$, for any concepts $C, D$.
Thus,
from the $\mathbf{M}$-principle %and the previous two formulas,
%from~(\ref{eq:1eq}),~(\ref{eq:2eq}) and the $\mathbf{M}$-principle above,
we would
%immediately
obtain
%\begin{equation}
$
\label{eq:monconc}
\mathsf{Ord} \sqsubseteq \mathbb{D}_{c} ( \mathsf{Confirm} \sqcup \lnot \mathsf{Confirm} ),
$
%\end{equation}
meaning that any order
%made by customer $c$
is made confirmed or not confirmed by $c$. This is an unwanted conclusion in our agency-based scenario, since customers' actions should be unrelated to any aspect of order confirmation.


Moreover, assume that the concept name $\mathsf{SubmitOrd}$ stands for the class of submitted orders, and that $\mathsf{PartConf}$ and $\mathsf{Reject}$ are used in our knowledge base to represent, respectively, the partially confirmed and the rejected
entities.
%\todo{N: I removed the pairwise disjointness.}
%They could be set as pairwise disjoint with $\mathsf{Confirm}$ by means of the formulas $\mathsf{Confirm} \sqsubseteq \lnot \mathsf{PartConf}$, $\mathsf{PartConf} \sqsubseteq \lnot  \mathsf{Reject}$, $\mathsf{Confirm} \sqsubseteq \lnot \mathsf{Reject}$.
%$C \sqsubseteq \lnot D$, with $C, D \in \{  \mathsf{Confirm}, \mathsf{PartConf}, \mathsf{Reject} \}$ and $C \neq D$.
Now consider the formula
%\begin{equation}
$
\label{eq:agglprem}
\mathsf{SubmitOrd} \sqsubseteq \mathbb{C}_{s}\mathsf{Confirm} \sqcap \mathbb{C}_{s}\mathsf{PartConf} \sqcap \mathbb{C}_{s}\mathsf{Reject},
$
%\end{equation}
stating that a submitted order can be confirmed, can be partially confirmed, and can be rejected by the seller $s$.
On relational frames, however, we have that
$ \mathbb{C}_{s} C \sqcap \mathbb{C}_{s} D \sqsubseteq \mathbb{C}_{s}( C \sqcap D ) $
 is a valid formula, for any concepts $C, D$, known as the \emph{$\mathbf{C}$-principle} (or \emph{agglomeration}).
 Therefore, by %~(\ref{eq:agglprem}) and
 the $\mathbf{C}$-principle, under relational semantics we would be forced to conclude that
%\begin{equation}
$
\label{eq:agglconc}
\mathsf{SubmitOrd} \sqsubseteq \mathbb{C}_{s}(\mathsf{Confirm} \sqcap \mathsf{PartConf} \sqcap \mathsf{Reject}),
$
%\end{equation}
meaning that any submitted order is such that the seller $s$ has the ability to make it confirmed, partially confirmed, and rejected, all \emph{at once}, which is unreasonable.

%M-PRINCIPLE
%Concepts $\mathsf{Ord}$, $\mathsf{Prod}$ and $\mathsf{Reject}$ are used to represent, respectively, orders, products, and the class of things that are rejected, while $\mathsf{req}$ is a role for the request relation.
%The axiom
%$\top \sqsubseteq \mathsf{Ord} \sqcup \mathsf{Prod}$
%%and $\mathsf{Ord} \sqcap \mathsf{Prod} \sqsubseteq \bot$
%%partition the domain into disjoint classes of products and orders.
%states that the domain is covered by orders or products.
%Moreover, the formula
%\begin{equation}
%\label{eq:1eq}
%\mathsf{Ord} \equiv \mathbb{D}_{c}\exists \mathsf{req}.(\mathsf{Prod} \sqcap \mathsf{InCatal} )
%\end{equation}
%defines an order $\mathsf{Ord}$ as a request made by customer $c$ of an in-catalogue ($\mathsf{InCatal}$) product.
%Under relational semantics, (\ref{eq:1eq}) entails, for instance, that
%$
%\mathsf{Ord} \sqsubseteq \mathbb{D}_{c} ( \mathsf{Ord} \sqcup \mathsf{Prod} )
%$,
%%$
%%\mathsf{Ord} \sqsubseteq \mathbb{D}_{c} \forall \mathsf{req}.\mathsf{Prod}
%%$,
%meaning:
%any order is made by customer $c$ to be either an order or a product.
%%any order is made by customer $c$ to be such that it request only products.
%This unwanted conclusion comes from
%$\exists \mathsf{req}.(\mathsf{Prod} \sqcap \mathsf{InCatal} ) \sqsubseteq  \mathsf{Ord} \sqcup \mathsf{Prod}$
%%$\exists \mathsf{req}.(\mathsf{Prod} \sqcap \mathsf{InCatal} ) \sqsubseteq \forall \mathsf{req}.\mathsf{Prod}$,
%(due to the covering axiom), and the fact that, over relational frames, $C \sqsubseteq D$ always entails $\mathbb{D}_{c} C \sqsubseteq \mathbb{D}_{c} D$,  for any concepts $C, D$.



Consider now the formula
%\begin{equation}
$
\label{eq:necprem}
\top \sqsubseteq \mathsf{Confirm} \sqcup \lnot \mathsf{Confirm},
$
i.e., the truism stating that anything is either confirmed or not confirmed.
%\end{equation}
By the so called \emph{$\mathbf{N}$-principle} (or \emph{necessitation}) of relational semantics, we have that if $\top \sqsubseteq C$ is valid on relational frames, then $\top \sqsubseteq \mathbb{D}_{c} C$ holds as well, for any concept $C$.
Thus, from
the $\mathbf{N}$-principle of relational semantics
%~(\ref{eq:necprem})
%and the truism above,
it would follow
that
%\begin{equation}
$
\label{eq:necprem}
\top \sqsubseteq \mathbb{D}_{c} (\mathsf{Confirm} \sqcup \lnot \mathsf{Confirm}),
$
%\end{equation}
thereby forcing us to the consequence that every object is made by customer $c$ to be either confirmed or not confirmed, and hence leading again to an unreasonable connection between customer's actions and confirmation of orders.
%
In fact, since customer $c$ plays no role in confirmation actions, it is sensible to assume that, for any object of the domain, it is not the case that $c$ makes it confirmed or not confirmed. This can be achieved by the formula
%\begin{equation}
$
\label{eq:qprinc}
\top \sqsubseteq \lnot \mathbb{D}_{c} (\mathsf{Confirm} \sqcup \lnot \mathsf{Confirm}),
$
%\end{equation}
an instance of a principle sometimes known as the \emph{$\mathbf{Q}$-principle}, which is \emph{unsatisfiable} in relational frames, while admissible over neighbourhood ones.

Finally, we consider additional principles that can be adopted both in relational and neighbourhood semantics.
The formula
%\begin{equation}
$
\label{eq:tprinc}
\mathbb{D}_{w} \mathsf{Avail}  \sqsubseteq \mathsf{Avail}
$
%	\mathbb{D}_{c} \exists \mathsf{req}.(\mathsf{Prod} \sqcap \mathsf{InCatal} ) \sqsubseteq \exists \mathsf{req}.(\mathsf{Prod} \sqcap \mathsf{InCatal} )
%\end{equation}
states that
anything that is \emph{made} available by the warehouse $w$ is \emph{actually} available.
%anything that is \emph{made} by customer $c$ \emph{to be} a request of some in-catalogue products \emph{actually is} a request of some in-catalogue products.
This is an instance of the so-called \emph{$\mathbf{T}$-principle} (also \emph{factivity principle}), well-known in modal logic, particularly for its epistemic applications (if an agent \emph{knows} something, it has to be true).
Both in relational and neighbourhood semantics, the $\mathbf{T}$-principle entails the so-called \emph{$\mathbf{D}$-principle}.
This is instantiated by
%\begin{equation}
$
\label{eq:dprinc}
\mathbb{D}_{w} \mathsf{Avail}  \sqsubseteq \lnot \mathbb{D}_{w} \lnot \mathsf{Avail},
$
%\end{equation}
%\mathbb{D}_{w} \mathsf{Avail}  \sqsubseteq \hat{\mathbb{D}}_{cw} \mathsf{Avail}
%%	\mathbb{D}_{c} \exists \mathsf{req}.(\mathsf{Prod} \sqcap \mathsf{InCatal} ) \sqsubseteq \hat{\mathbb{D}}_{c}  \exists \mathsf{req}.(\mathsf{Prod} \sqcap \mathsf{InCatal} ),
%where $\hat{\mathbb{D}}_{c}$ abbreviates $\lnot \mathbb{D}_{c} \lnot$,
a formula asserting that
anything that is \emph{made available} by the warehouse $w$ is \emph{not made unavailable} by $w$.
%a request, made by customer $c$, of some in-catalogue products is not something that $c$ makes 
This principle, also well-known for its epistemic implications (anything that is \emph{known} by an agent is \emph{compatible} with their knowledge),
in relational semantics is \emph{equivalent} to a much lesser known principle, sometimes called \emph{$\mathbf{P}$-principle}.
An example of it is given by the formula
$
 \top \sqsubseteq \lnot \mathbb{D}_{w} (\mathsf{Avail} \sqcap \lnot \mathsf{Avail}),
$
which states that, for any object, it is not the case that the warehouse $w$ makes it both available \emph{and} unavailable.
Neighbourhood semantics, under which the $\mathbf{D}$- and the $\mathbf{P}$-principle are \emph{not equivalent}, allows for distinctions between modal constraints that are hence more fine-grained than in relational semantics.


%N-PRINCIPLE

%suppose that our knowledge base contains the formula
%\begin{equation}
%\label{eq:necprem}
%\mathsf{InvalidOrd} \sqsubseteq \mathbb{C}_{s} \mathsf{Cancel},
%\end{equation}
%asserting the seller $s$ has the power to cancel ($\mathsf{Cancel}$ any invalid order ($\mathsf{InvalidOrd}$).

%Under relational semantics, due to a principle known as \emph{necessitation} and here called \emph{$\mathbf{N}$-principle}, it follows that $\mathbb{D}_{w} \mathsf{Avail}$
%Clearly, the following formula is valid: $\top \sqsubseteq \mathsf{Avail} \sqcup \lnot \mathsf{Avail}$, that is, everything is either available or not available.

%The formula
%\begin{equation}
%%\label{eq:2eq}
%\top \sqsubseteq \forall \mathsf{req}.\mathsf{Prod}
%\end{equation}
%formalises that the range of the request relation $\mathsf{req}$ is a product $\mathsf{Prod}$.

%The formula $\top \sqsubseteq \forall \mathsf{req}.\mathsf{Prod}$ formalises that the range of the request relation $\mathsf{req}$ is a product $\mathsf{Prod}$
%
%\ldots
%
%The $\mathbb{C}_i$ operator expresses `agent $i$ can do/make'.
%

%The formula
%\begin{equation}
%\label{eq:2eq}
%\mathbb{C}_{w} \mathsf{Avail} \sqcap \mathbb{C}_{w} \lnot \mathsf{Avail} \sqsubseteq \forall .\mathsf{Prod}
%\end{equation}
%is valid on


%\todo[inline]{
%%$M, C, N$: counterexamples (valid in relational, not valid in neighbourhood)\\
%$Q$: always invalid in relational + example for validity on neighbourhood\\
%$P,D,T$: possibly hold in relational and in neighbourhood, examples for agency - $P, D$ equivalent in relational but not in neighbourhood semantics
%}

%To motivate further the modal description logic languages, we use description purchase choreography scenario, involving
%a customer $\mathit{c}$,
%a seller $\mathit{s}$,
%a warehouse $\mathit{w}$,
%and a marketplace
%$\mathit{m}$.
%The agency operator $\mathbb{D}_i$ is read as `agent $i$ does/makes', while $\mathbb{C}_i$ expresses `agent $i$ can do/make'.
%Valid principles: $\mathbb{D}_i C \sqsubseteq C$; $\mathbb{D}_i C \sqsubseteq \mathbb{C}_i C$
%

















%%%  FIGURE - SINGLE COLUMN VERSION
%% Figure environment removed


%%%  FIGURE - DOUBLE COLUMN VERSION
% Figure environment removed





























%The formulas discussed so far, to understand the shortcomings of relational frames and to motivate neighbourhood semantics, are part of
The full knowledge base describing the purchase choreography scenario
is reported in Figure~\ref{fig:purchasekb}.
%Remove $\mathbb{D}_{i}C \sqsubseteq \mathbb{C}_{i}C$ from caption? We are not able to treat it (no interacting modalities, cf. Discussion)
Notably, it displays a great range of interactions between modal and description logics constructs.
The modal operators $\mathbb{D}_{i}$ and $\mathbb{C}_{i}$ are axiomatized similarly to~\protect\cite{Elg}:
$\mathbb{D}_{i}$ obeys the $\mathbf{C}$- and $\mathbf{T}$-principles;
%$\mathbb{C}_{i}$
%obeys the $\mathbf{Q}$-principle,
%  {\color{red}{
%  and we have the axiom schema $\mathbb{D}_{i} C \sqsubseteq \mathbb{C}_{i} C$.
%  }}
  and both satisfy the $\mathbf{Q}$-principle, and the $\mathbf{E}$-principle:
  $C \equiv D$ entails $\mathbb{D}_{i} C \equiv \mathbb{D}_{i} D$ and $\mathbb{C}_{i} C \equiv \mathbb{C}_{i} D$.



On the left column, the first four axioms impose that the range of the request relation is a product, and that 
the classes of confirmed, partially confirmed, and rejected objects are all disjoint.
We have already discussed the fifth axiom, which is the definition of an order as a request, made by a customer, for some in-catalogue product.
The last four axioms on the left column define, respectively:
a submitted order as an order that is actually submitted by a customer;
an incomplete order as an order that is not submitted by the customer;
an invalid order as an order that cannot be submitted by the customer;
and a confirmed order as an order that is confirmed by the seller.

On the right column, the first formula states that an invalid order can be cancelled by the seller.
The second axiom asserts that, given a submitted order, the seller can check at the warehouse for availability of all of its requested products.
The third formula, as already discussed, requires that a submitted order can be confirmed, can be partially confirmed, and can be rejected by the seller.
The subsequent three axioms impose, respectively, the following constraints:
anything that is confirmed by the seller has to request only products made available by the warehouse;
anything that requests products that are made available, as well as products that are brought about to be unavailable by the warehouse, can be partially confirmed by the seller;
%\todo{I removed ``only'' (``can only be partially confirmed''): maybe we'd like $\lnot \mathbb{C}_s\mathsf{Conf} \sqsupseteq$ otherwise.}
finally, any request of products that are all unavailable at the warehouse can be rejected by the seller.
The third formula from the bottom of the right column states that the marketplace enforces that a confirmed order cannot be rejected by the seller.
%\todo{check for any possible inconsistency with Q-principle and $ConfirmOrd \sqsubseteq \lnot Reject$.}
The second last axiom defines a dispatched order as an order that the seller makes the warehouse dispatch.
Finally, the last formula imposes that the marketplace brings it about that any confirmed order is also dispatched.













%%%
%\todo[inline]{M: Introduction ARQNL22}
%
%
%Contexts involving
%%a certain degree of
%agency-based~\cite{Brown,Elg} and coalitional~\cite{Pau,Tro},
%epistemic and doxastic~\cite{Ago,Bal,Var1}, as well as deontic~\cite{AngEtAl,Gob,Wright}, reasoning capabilities populate the wide spectrum of settings where modal logics have found natural applications.
%In such scenarios, modal operators can be used to represent and reason about what agents, or groups of agents, respectively know, believe, have the capability, or have the permission, to bring about.
%
%The semantics of modal operators is usually given in terms of \emph{relational models}, based on frames consisting of a set of possible worlds equipped with suitable accessibility relations.
%However, all the modal systems interpreted by means of this kind of semantics, known as \emph{normal}, validate principles that have been considered problematic or debatable for the aforementioned applications, leading to counterintuitive or unacceptable conclusions.
%Among the unpleasant features discussed in the literature, one encounters for instance the problem of logical omniscience~\cite{Var1}, as well as a number of so-called paradoxes in the representation of agents' abilities~\cite{Elg} and obligations~\cite{Ross,Aqv,For}.
%
%To avoid the unwanted consequences of the relational semantics, several \emph{non-normal} modal logics have been proposed and studied, tracing back to the seminal works by C.I. Lewis~\cite{CIL}, Lemmon~\cite{Lem}, Kripke~\cite{Kripke}, Scott~\cite{Sco}, Montague~\cite{Mon}, Segerberg~\cite{Seg}, and Chellas~\cite{Che}.
%The semantics of such systems can be given in terms of \emph{neighbourhood models}, generalisations of the relational ones first introduced by Scott \cite{Sco} and Montague \cite{Mon}.
%In this setting, a frame consists of a set of worlds, each of which is associated with a set of subsets of worlds. Since a subset of worlds can be thought as a proposition (that is true in those worlds), this means that every world in a neighbourhood model is assigned to a set of propositions, those considered necessary with respect to that world.
%%
%This semantics both generalises the relational one, and avoids the drawbacks of the latter, since the modal principles validated on relational frames that are deemed as problematic for epistemic, coalitional or deontic applications do not hold in general on neighbourhood models.
%
%
%
%Non-normal modalities have been widely investigated as a way to extend propositional logic.
%A further line of research focuses on the behaviour of modal operators interpreted on neighbourhood frames in combination with first-order logic.
%%
%In this direction, a few works have provided completeness results for first-order non-normal modal logics~\cite{Cos,CosPac}.
%%
%In addition, non-normal modal extensions of \emph{description logics}, seen as fragments of first-order logic with a good trade-off between expressive power and computational complexity, have been considered for knowledge representation applications~\cite{SeyErd09, DL19}, also in multi-agent coalitional settings~\cite{SeyJam09,SeyJam10}.
%
%
%In this paper, we investigate
%satisfiability of non-normal modal extensions of description logics.
%In particular,
%%we study the logics associated with the class of
%%all neighbourhood frames,
%%supplemented neighbourhood frames, 
%%neighbourhood frames closed under intersection,
%%and neighbourhood frames containing the unit
%%(characterising, at the propositional level, the non-normal modal systems $\mathbf{E}$, $\mathbf{M}$, $\mathbf{C}$, and $\mathbf{N}$, respectively),
%we study the logics
%characterised by
%%associated with
%the class of all neighbourhood frames ($\mathbf{E}$),
%supplemented neighbourhood frames ($\mathbf{M}$), 
%neighbourhood frames closed under intersection ($\mathbf{C}$),
%and neighbourhood frames containing the unit ($\mathbf{N}$),
%and combine them with the prototypical $\ALC$ description logic. 
%We provide a framework of terminating, correct, and complete tableau algorithms to check 
%satisfiability in such logics interpreted
%in neighbourhood models
%with \emph{varying domains}
%(in this kind of semantics, the domains of the interpretations
%at each world
%%representing worlds
%can differ;
%cf.~Section~\ref{sec:prelim} for details).
%%regarding the semantics). 
%%Our proofs for the tableau algorithm include a characterization 
%%of neighbourhood models in terms of \emph{quasimodels}~\cite{?}, %cite yellow book
%%a classical technique used in modal logic to prove complexity results.  
%We then investigate the satisfiability problems in fragments of these languages obtained by restricting the application of modal operators to formulas only, and provide complexity upper bounds with \emph{constant domains}
%(in this case 
%%this is the more classical case in which
%the domains of the interpretations
%at every world
%%representing the worlds
%are the same).
%We leave satisfiability checking procedures for non-restricted languages interpreted on models with constant domain as open problems.
%%We leave the general case of satisfiability in these logics with constant domains as
%%open problems. 
%
%Full proof details are provided in an extended version of this paper~\cite{DalEtAl22arxiv}.
















































































%\section{Modelling}
%
%\subsection{Purchase Choreography Scenario}
%
%Abbreviations:
%$\mathit{c}$ := \emph{customer};
%$\mathit{s}$ := \emph{seller};
%$\mathit{w}$ := \emph{warehouse};
%$\mathit{m}$ := \emph{marketplace}
%\\
%Modalities: $\mathbb{D}_i C/\varphi$ - agent $i$ does/makes $C$/$\varphi$; $\mathbb{C}_i C/\varphi$ - agent $i$ can do/make $C$/$\varphi$ (alternative: $\mathbb{O}_i C/\varphi$ - agent $i$ oughts to do/make $C$/$\varphi$)
%\\
%Valid principles: $\mathbb{D}_i C \sqsubseteq C$; $\mathbb{D}_i C \sqsubseteq \mathbb{C}_i C$
%
%Cf. KB below:
%%
%\begin{align*}
%\top & \sqsubseteq \forall \mathsf{requests}.\mathsf{Product}
%\\
%\mathsf{Ord} & \equiv \mathbb{D}_{c}\exists \mathsf{req}.(\mathsf{Prod} \sqcap \mathsf{InCat} )
%\\
%\mathsf{IncomplOrd} & \equiv \mathsf{Ord} \sqcap \lnot \mathbb{D}_{c} \mathsf{Submit}
%\\
%%\mathsf{IncompleteOrder} & \sqsubseteq \lnot \mathsf{ExpiredOrder}
%%\\
%\mathsf{SubmitOrd} & \equiv
%\mathsf{Ord} \sqcap \mathbb{D}_{c} \mathsf{Submit}
%\\
%\mathsf{InvalidOrd} & \equiv  \mathsf{Order} \sqcap \lnot \mathbb{C}_{c} \mathsf{Submit}
%\\
%\mathsf{InvalidOrd} & \sqsubseteq \mathbb{C}_{s} \mathsf{Cancel}
%\\
%\mathsf{SubmitOrder} & \sqsubseteq \mathbb{C}_{s} \forall \mathsf{req}.
%(\mathbb{D}_{w} \mathsf{Avail} \sqcup \mathbb{D}_{w} \lnot \mathsf{Avail})
%\\
%\mathsf{SubmitOrd} & \sqsubseteq \mathbb{C}_{s}\mathsf{Confirm} \sqcap \mathbb{C}_{s}\mathsf{PartConf} \sqcap \mathbb{C}_{s}\mathsf{Reject}
%\\
%\mathbb{D}_{s} \mathsf{Confirm} & \sqsubseteq \forall \mathsf{req}.
%\mathbb{D}_{w}\mathsf{Avail}
%\\
%\mathbb{C}_{s} \mathsf{PartConf} & \sqsupseteq \exists \mathsf{req}.
%\mathbb{D}_{w}\mathsf{Avail} \ \sqcap
%\exists \mathsf{req}.
%\mathbb{D}_{w} \lnot \mathsf{Avail}
%\\
%\mathbb{C}_{s} \mathsf{Reject} & \sqsupseteq \forall \mathsf{req}.
%\mathbb{D}_{w}\lnot\mathsf{Avail}
%\\
%\mathsf{ConfirmOrd} & \equiv \mathsf{Ord} \sqcap \mathbb{D}_{s} \mathsf{Confirm}
%\\
%\mathbb{D}_{m}
%(
%\mathsf{ConfirmOrd} & \sqsubseteq
%\lnot \mathbb{C}_{s} \mathsf{Reject}
%)
%\\
%\mathsf{DispatchOrd} & \equiv \mathsf{Ord} \sqcap \mathbb{D}_{s} \mathbb{D}_{w} \mathsf{Dispatch}
%\\
%\mathbb{D}_{m}
%(
%\mathsf{ConfirmOrd} & \sqsubseteq
%\mathsf{DispatchOrd}
%)
%\end{align*}
%
%%% Figure environment removed
%%
%%\todo[inline,caption={}]{
%%M: todo complete/fix example
%%}
%
%
%%\begin{align*}
%%\mathsf{Order} & \sqsubseteq \mathbb{D}_{c}\exists \mathsf{requests}.\mathsf{Product} \sqcap \exists \mathsf{hasStatus}.\mathsf{OrderStatus}
%%\\
%%\mathsf{Product} & \sqsubseteq \exists \mathsf{hasStatus}.\mathsf{ProductStatus}
%%\\
%%\mathsf{OrderStatus} & \equiv \mathsf{Submitted}
%%%\sqcup \mathsf{Incomplete}
%%\sqcup \mathsf{Cancelled} \ \sqcup \\  
%%& \phantom{ \sqsubseteq \ \ } \mathsf{Confirmed} \sqcup \mathsf{PartiallyConfirmed} \sqcup \mathsf{Rejected}  \ \textit{(suitably disjoint)}
%%\\
%%\mathsf{ProductStatus} & \equiv \mathsf{Available} \sqcup \mathsf{Unavailable} \ (\textit{disjoint})
%%\\
%%\mathsf{SubmittedOrder} & \equiv \mathsf{Order} \sqcap \exists \mathsf{hasStatus}.\mathbb{D}_{c}\mathsf{Submitted}
%%\\
%%\mathsf{IncompleteOrder} & \equiv \mathsf{Order} \sqcap \forall \mathsf{hasStatus}.\lnot \mathbb{D}_{c}\mathsf{Submitted}
%%%\mathsf{IncompleteOrder} \equiv \mathsf{Order} \sqcap \exists \mathsf{hasStatus}.\mathsf{Incomplete}
%%\\
%%%\mathsf{CancelledOrder} & \equiv \mathsf{Order} \sqcap \exists \mathsf{hasStatus}.\mathsf{Cancelled},
%%%\
%%\mathbb{C}_{sl} ( \mathsf{IncompleteOrder} & \sqsubseteq  \mathsf{Order} \sqcap \exists \mathsf{hasStatus}. \mathbb{D}_{s} \mathsf{Cancelled})
%%%\mathbb{C}_{sl} ( \mathsf{Order} \sqcap \exists \mathsf{hasStatus}.\mathsf{Incomplete} & \sqsubseteq  \mathsf{Order} \sqcap \exists \mathsf{hasStatus}. \mathbb{D}_{sl} \mathsf{Cancelled})
%%\\
%%\mathsf{SubmittedOrder} & \sqsubseteq  \mathbb{C}_{s} \mathsf{ConfirmedOrder} \sqcup  \mathbb{C}_{s} \mathsf{PartiallyConfirmedOrder}  \sqcup \mathbb{C}_{s} \mathsf{RejectedOrder}  
%%\\
%%\mathsf{RejectedOrder} & \equiv \mathsf{SubmittedOrder} \sqcap \exists \mathsf{hasStatus}.\mathbb{D}_{s}\mathsf{Rejected}
%%\\
%%\mathsf{ConfirmedOrder} & \equiv \mathsf{SubmittedOrder} \sqcap \exists \mathsf{hasStatus}.\mathbb{D}_{s}\mathsf{Confirmed}
%%%\\
%%%\mathsf{ConfirmedOrder} & \sqsubseteq \lnot \mathsf{RejectedOrder}
%%\\
%%\mathbb{C}_{s} \mathsf{ConfirmedOrder} & \sqsubseteq \mathsf{Order} \sqcap \forall \mathsf{requests}.(\mathsf{Product} \sqcap  \exists \mathsf{hasStatus}.\mathbb{D}_{w}\mathsf{Available})
%%\\
%%\mathbb{C}_{s} \mathsf{PartiallyConfirmedOrder} & \sqsupseteq \mathsf{Order} \sqcap \exists \mathsf{requests}.(\mathsf{Product} \sqcap  \exists \mathsf{hasStatus}.\mathbb{D}_{w}\mathsf{Available}) \ \sqcap
%% \\
%%& \phantom{\sqsubseteq \ \ } 
%%\exists \mathsf{requests}.(\mathsf{Product}\sqcap  \exists \mathsf{hasStatus}.\mathbb{D}_{w}\mathsf{Unavailable})
%%\\
%%\mathbb{C}_{s} \mathsf{RejectedOrder} & \sqsupseteq \mathsf{Order} \sqcap \forall \mathsf{requests}.(\mathsf{Product} \sqcap  \exists \mathsf{hasStatus}.\mathbb{D}_{w}\mathsf{Unavailable})
%%\end{align*}
%
%
%%\subsection{Organisation Structure Scenario}
%%
%%\ldots
%
%
%
%
%
%
%
%
%%\subsection{Driving Licence Scenario}
%%
%%$\mathit{DMV}$ = Department of Motor Vehicles
%%
%%\noindent
%%$\cstyle{TruckDriver} \equiv \cstyle{Driver} \sqcap \exists \cstyle{hasLicence}.\mathbb{D}_{\mathit{DMV}} \cstyle{TruckDrivingLicence}$
%%
%%\noindent
%%$\cstyle{ExceptionalTruckDriver} \equiv \cstyle{TruckDriver} \sqcap \mathbb{D}_{\mathit{LocalAuthority}} \exists \cstyle{hasPermission.ExceptionalTransportationPermission}$
%%
%%\noindent
%%$\cstyle{Fine} \equiv \mathbb{D}_{\mathit{officer}} \cstyle{Issued} \sqcup \mathbb{D}_{\mathit{radar}} \cstyle{Issued}$
%%
%%\noindent
%%$\cstyle{QuestionableFine} \equiv \cstyle{Fine} \sqcap \mathbb{C}_{\mathit{judge}} \cstyle{Remitted}$
%%
%%\noindent
%%$\cstyle{SanctionedDriver} \equiv \cstyle{Driver} \sqcap \exists \cstyle{hasFine.Fine}$
%%
%%\noindent
%%$\cstyle{AcquittedDriver} \equiv \cstyle{SanctionedDriver} \sqcap \exists\cstyle{hasFine}.(\cstyle{Fine} \sqcap \mathbb{D}_{\mathit{judge}} \cstyle{Remitted})$
%%
%%\noindent
%%$\cstyle{ColludedDriver} \equiv \cstyle{SanctionedDriver} \sqcap \forall \cstyle{hasFine}.\mathbb{D}_{\mathit{judge}} \cstyle{Remitted}$
%%
%%\noindent
%%$\mathbb{C}_{\mathit{LocalAuthority}} (\cstyle{TaxiDriver} \sqsubseteq \exists \cstyle{hasPermission.CityCenterPermission})$
%%
%%\noindent
%%$\mathbb{C}_{\mathit{DMV}} \exists \cstyle{hasLicence.TruckDrivingLicence} \sqsubseteq \cstyle{ExpertDriver}$
%%
%%\noindent
%%$\mathbb{C}_{\mathit{DMV}} \exists \cstyle{hasLicence.CarLicence} \sqsubseteq \cstyle{Adult}$
%%
%%
%%\bigskip
%%
%%difference between
%%
%%$\mathbb{D}_{\mathit{DMV}} \exists \cstyle{hasLicence.TruckDrivingLicence}$
%%
%%and
%%
%%$\exists \cstyle{hasLicence}.\mathbb{D}_{\mathit{DMV}} \cstyle{TruckDrivingLicence}$?
%%
%%e.g. issue the licence vs. recognise the licence issued by another authority? 
%%
%%
%%\ldots
%
%
%
%
%
%
%
%
%
%
%
%
%
%
%
%
%
%
%
%
%
%
%
%
%
%
%
%
%
%
%
%
%
%
%%%% OLD EXAMPLE
%
%\begin{comment}
%
%% Figure environment removed
%
%\end{comment}


\section{Proofs for Section~\ref{sec:prelim}}


\PropImplicationSystem*
\begin{proof}
%\todo[inline]{M: todo sketch proof}
Point 1. Suppose that, for every $w\in \Wmc$, $\alpha,\beta\subseteq \Wmc$, we have: ($\mathbf{M}$-condition) $\alpha\in \Nmc_{i}(w)$ and $\alpha\subseteq\beta$ implies $\beta\in \Nmc_{i}(w)$; and ($\mathbf{Q}$-condition) $\Wmc \not \in \Nmc_{i}(w)$. This means that $\alpha \not \in \Nmc_{i}(w)$, for every $\alpha \subseteq \Wmc$, i.e., $\Nmc_{i}(w) = \emptyset$.
Thus, every condition, except for the $\mathbf{N}$-condition, is satisfied by $\Nmc_{i}$.

Point 2.
%Item $(i)$ follows from the fact that $\Nmc_{i}(w)=\emptyset$ 
%for every $w\in \Wmc$, as already stated in Point 1.
%Suppose~$(i)$ that, for every $w\in \Wmc$, $\alpha,\beta\subseteq \Wmc$, we have: ($\mathbf{M}$-condition) $\alpha\in \Nmc_{i}(w)$ and $\alpha\subseteq\beta$ implies $\beta\in \Nmc_{i}(w)$; and ($\mathbf{Q}$-condition) $\Wmc \not \in \Nmc_{i}(w)$. This means that $\alpha \not \in \Nmc_{i}(w)$, for every $\alpha \subseteq \Wmc$, hence in particular  ($\mathbf{P}$-condition) $\emptyset \not \in \Nmc_{i}(w)$.
%
Suppose~$(i)$ that, for every $w\in \Wmc$, $\alpha,\beta\subseteq \Wmc$, we have: ($\mathbf{M}$-condition) $\alpha\in \Nmc_{i}(w)$ and $\alpha\subseteq\beta$ implies $\beta\in \Nmc_{i}(w)$; and ($\mathbf{D}$-condition) $\alpha \in \Nmc_{i}(w)$ implies $\Wmc \setminus \alpha \not \in \Nmc_{i}(w)$.
%Since $\emptyset \subseteq \beta$, for every $\beta\subseteq \Wmc$, we have that
Towards a contradiction, suppose that $\emptyset \in \Nmc_{i}(w)$. Then, we have $\Wmc = \Wmc \setminus \emptyset \not \in \Nmc_{i}(w)$ as well. By contraposition, this implies in particular that $\emptyset \not \in \Nmc_{i}(w)$, a contradiction. Thus, ($\mathbf{P}$-condition) $\emptyset \not \in \Nmc_{i}(w)$

Moreover, suppose~$(ii)$ that, for every $\alpha\subseteq \Wmc$, we have: ($\mathbf{N}$-condition) $\Wmc \in \Nmc_{i}(w)$, i.e., $\Wmc \setminus \emptyset \in \Nmc_{i}(w)$; and ($\mathbf{D}$-condition) $\alpha \in \Nmc_{i}(w)$ implies $\Wmc \setminus \alpha \not \in \Nmc_{i}(w)$. By contraposition, we obtain that ($\mathbf{P}$-condition) $\emptyset \not \in  \Nmc_{i}(w)$.

Finally, suppose~$(iii)$ that, for every $w\in \Wmc$, $\alpha,\beta\subseteq \Wmc$, we have: ($\mathbf{T}$-condition) $\alpha\in \Nmc_{i}(w)$ implies $w \in \alpha$. Then, we have ($\mathbf{P}$-condition) $\emptyset \not \in \Nmc_{i}(w)$, for otherwise we would get a contradiction.

Point 3.
%By Point~1, if $(i)$ we have the $\mathbf{MQ}$-condition
%then the $\mathbf{D}$-condition is trivially satisfied. %\nb{O:added}
Suppose~$(i)$ that, for every $w\in \Wmc$, $\alpha,\beta\subseteq \Wmc$, we have: ($\mathbf{C}$-condition)
$\alpha\in \Nmc_{i}(w)$ and $\beta\in \Nmc_{i}(w)$ implies $\alpha\cap\beta\in \Nmc_{i}(w)$; and ($\mathbf{P}$-condition) $\emptyset \not \in  \Nmc_{i}(w)$. Given $\alpha \in \Nmc_{i}(w)$, suppose towards a contradiction that $\Wmc \setminus \alpha \in \Nmc_{i}(w)$. From this, we obtain that $\emptyset = \alpha \cap (\Wmc \setminus \alpha) \in \Nmc_{i}(w)$, a contradiction. Thus, we have that ($\mathbf{D}$-condition) $\alpha \in \Nmc_{i}(w)$ implies $\Wmc \setminus \alpha \not \in \Nmc_{i}(w)$.

Moreover, suppose~$(ii)$ that, for every $w\in \Wmc$, $\alpha,\beta\subseteq \Wmc$, we have: ($\mathbf{T}$-condition) $\alpha\in \Nmc_{i}(w)$ implies $w \in \alpha$. Consider $\alpha\in \Nmc_{i}(w)$ and suppose, towards a contradiction, that also $\Wmc \setminus \alpha \in \Nmc_{i}(w)$. We obtain that $w \in \alpha$ 
%\nb{O: fixed mistake, it was $w \in \Wmc$}
and $w \in \Wmc \setminus \alpha$ as well, a contradiction. Hence, we have that ($\mathbf{D}$-condition) $\alpha \in \Nmc_{i}(w)$ implies $\Wmc \setminus \alpha \not \in \Nmc_{i}(w)$.

Point 4. Straightforward, because otherwise we immediately have a contradiction.
% for every $w\in \Wmc$, $\alpha,\beta\subseteq \Wmc$,
%%
%\begin{description}
%	\item[\textnormal{\emph{$E$-condition}:}] $\Nmc_{i}$ is a neighbourhood function (always true);
%	\item[\textnormal{\emph{$M$-condition} (\emph{supplementation}):}] $\alpha\in \Nmc_{i}(w)$ and $\alpha\subseteq\beta$ implies $\beta\in \Nmc_{i}(w)$;
%	\item[\textnormal{\emph{$C$-condition} (\emph{closure under intersection}):}] $\alpha\in \Nmc_{i}(w)$ and $\beta\in \Nmc_{i}(w)$ implies $\alpha\cap\beta\in \Nmc_{i}(w)$;
%	\item[\textnormal{\emph{$N$-condition} (\emph{containment of unit}):}] $\Wmc \in \Nmc_{i}(w)$;
%	\item[\textnormal{\emph{$P$-condition}:}] $\emptyset \not \in \Nmc_{i}(w)$;
%		\item[\textnormal{\emph{$Q$-condition}:}] $\Wmc \not \in \Nmc_{i}(w)$;
%			\item[\textnormal{\emph{$D$-condition}:}]  $\alpha \in \Nmc_{i}(w)$ implies $\Wmc \setminus \alpha \not \in \Nmc_{i}(w)$;
%	\item[\textnormal{\emph{$T$-condition}:}] $\alpha \in \Nmc_{i}(w)$ implies $w \in \alpha$.
%\end{description}
\end{proof}

%\todo[inline]{M: add
%\\
%$T \to D$ \\ $T \to P$ \\ $N + Q \to \bot$ \\ $N + D \to P$ \\ $M + D \to P$ \\ $M + Q \to P$ \\ $M + Q \to \Nmc_{i}(w) = \emptyset$ \\ $C + P \to C +  D$ \\ \ldots}
%\begin{corollary}
%	The MQ-condition implies the P-condition.
%\end{corollary}
















\PropCorresp*
\begin{proof}
%\todo{M: todo fix proof}
Here we present a proof  only for $\MLALC{n}$ concept inclusions %\nb{O: concept inclusions?}
 and only for the basic principles $L \in \{ \mathbf{E, M, C, N, P, Q, D, T} \}$.
For $\MLALC{n}$ formulas, the proof is similar to the case of propositional non-normal modal logics (see e.g.~\cite{Pac}).
More complex principles  (e.g. $\mathbf{EMCN}$) %$L' \in \mathsf{Pantheon}$, 
%the corresponding results 
can be obtained by suitably combining the basic principles. %$L$.
%
%
%\emph{Point~1.}
%
%In the following, 
%Let $L \in \{ E, M, C, N, P, Q, D, T \}$ and 
Let $\Fmc = (\Wmc, \{ \Nmc_{i} \}_{i \in J})$ be a neighbourhood frame and let 
$L$ be as above.
%\nb{O: checking}
\begin{description}
	\item[\textnormal{$\mathit{L = \mathbf{E}}$.}]
%	[\textnormal{($\mathit{E}$-\emph{principle})}]
	If $C \equiv D$ is valid on $\Fmc$, then for all
	$\Mmc = (\Fmc, \Imc)$
%	$\Mmc = (\Fmc, \Delta, \Imc)$
	based on $\Fmc$, and all $w$ in $\Mmc$, we have 
$C^{\Int_w}  = D^{\Int_w}$.
Thus, for all $d \in \Delta_{w}$, $\llbracket C \rrbracket^{\Mmc}_{d} = \llbracket D \rrbracket^{\Mmc}_{d}$.
So for all $v\in\W$, $\llbracket C \rrbracket^{\Mmc}_{d} \in \Nmc_{i}(v)$ iff $\llbracket D \rrbracket^{\Mmc}_{d} \in \Nmc_{i}(v)$,
which implies $d\in (\B_{i} C)^{\Int_v}$ iff $d\in(\B_{i} D)^{\Int_v}$,
that is $(\B_{i} C)^{\Int_v} = (\B_{i} D)^{\Int_v}$.
Then, $\B_{i} C \equiv \B_{i} D$ is valid on $\Fmc$, for all $i\in J$.
%
	\item[\textnormal{$\mathit{L = \mathbf{M}}$.}]
%	($\mathit{M}$-\emph{principle})
From right to left,
assume that $\Fmc$ is supplemented and $C \sqsubseteq D$ is valid on $\Fmc$.
Then, for all
$\Mmc = (\Fmc, \Imc)$
%$\Mmc = (\Fmc, \Delta, \Imc)$
based on $\Fmc$, and all $w$ in $\Mmc$, we have 
$C^{\Int_w} \subseteq D^{\Int_w}$.
Thus, for all $d\in\Delta_{w}$, $\llbracket C \rrbracket^{\Mmc}_{d} \subseteq \llbracket D \rrbracket^{\Mmc}_{d}$.
By supplementation we have that, for all $v\in\W$, $\llbracket C \rrbracket^{\Mmc}_{d} \in \Nmc_{i}(v)$ implies $\llbracket D \rrbracket^{\Mmc}_{d} \in \Nmc_{i}(v)$.
So $d\in (\B_{i} C)^{\Int_v}$ implies $d\in(\B_{i} D)^{\Int_v}$,
that is $(\B_{i} C)^{\Int_v} \subseteq (\B_{i} D)^{\Int_v}$.
Then $\B_{i} C \sqsubseteq \B_{i} D$ is valid on $\Fmc$.
% 
For the left-to-right direction,
assume that $\Fmc$ is not supplemented. Then 
there are $w\in\Wmc$, $\alpha,\beta\subseteq\Wmc$ such that $\alpha\subseteq\beta$, $\alpha\in\Nmc_{i}(w)$ and $\beta\notin\Nmc_{i}(w)$.
We define over $\Fmc$ the model
$\Mmc= ( \Fmc, \Int )$,
%$\Mmc= ( \Fmc, \Delta, \Int )$,
where
$\Delta_{w} = \{d\}$, for every $w \in \Wmc$,
%$\Delta = \{d\}$,
and the interpretation of two concept names $A, B \in \NC$ is defined as follows:
$d\in A^{\Int_v}$ iff $v\in\alpha$, and
$d\in B^{\Int_v}$ iff $v\in\beta$
(and defined arbitrarily on all other symbols in $( \NC \cup \NR ) \setminus \{ A, B \}$).
%
As a consequence, we have that $\llbracket A \rrbracket_d^{\Mmc} = \alpha$ and $\llbracket B \rrbracket_d^{\Mmc} = \beta$,
which implies $\llbracket A \rrbracket_d^{\Mmc} = \llbracket A \rrbracket_d^{\Mmc} \cap \llbracket B \rrbracket_d^{\Mmc} = \llbracket A \sqcap B \rrbracket_d^{\Mmc}$.
%\nb{T: Do we need to show explicitly that $\llbracket A \rrbracket_d^{\Mmc} \cap \llbracket B \rrbracket_d^{\Mmc} = \llbracket A \sqcap B \rrbracket_d^{\Mmc}$? \\ M: i don't think so}
Thus $\llbracket A \sqcap B \rrbracket_d^{\Mmc}\in\Nmc_{i}(w)$ and $\llbracket B \rrbracket_d^{\Mmc}\notin \Nmc_{i}(w)$. 
By definition we have $d\in(\B_{i}(A\sqcap B))^{\Int_w}$ and $d\notin(\B_{i} B)^{\Int_w}$.
%
	\item[\textnormal{$\mathit{L = \mathbf{C}}$.}]
%	($\mathit{C}$-\emph{principle})
From right-to-left,
assume that $\Fmc$ is closed under intersection. 
Moreover, let
$\Mmc = (\Fmc, \Imc)$
%$\Mmc = (\Fmc, \Delta, \Imc)$
be a model based on $\Fmc$, with $w$ world of $\Mmc$, and $d \in \Delta$
such that $d\in(\B_{i} C \sqcap \B_{i} D)^{\Int_w}$.
Thus $d\in(\B_{i} C)^{\Int_w}$ and $d\in(\B_{i} D)^{\Int_w}$,
that is $[C]_d^{\Mmc}, [D]_d^{\Mmc} \in\Nmc_{i}(w)$.
By closure under intersection, $[C]_d^{\Mmc} \cap [D]_d^{\Mmc} =  [C\sqcap D]_d^{\Mmc}\in\Nmc_{i}(w)$.
Then $d\in(\B_{i} ( C \sqcap D))^{\Int_w}$.
%
For the left-to-right direction,
assume that $\Fmc$ is not closed under intersection.
Then, there are $w\in\Wmc$, $\alpha,\beta\subseteq\Wmc$ such that $\alpha, \beta\in\Nmc_{i}(w)$ and $\alpha\cap\beta\notin\Nmc_{i}(w)$.
We define over $\Fmc$ the model
$\Mmc = ( \Fmc, \Int )$,
%$\Mmc = ( \Fmc, \{d\}, \Int )$,
where
$\Delta_{w} = \{d\}$, for all $w \in \Wmc$,
%$\Delta = \{d\}$,
and the interpretation of two concept names $A, B \in \NC$ is defined as follows:
$d\in A^{\Int_v}$ iff $v\in\alpha$, and
$d\in B^{\Int_v}$ iff $v\in\beta$
(and defined arbitrarily on all other symbols in $( \NC \cup \NR ) \setminus \{ A, B \}$).
%
We have $\llbracket A \rrbracket_d^{\Mmc} = \alpha$ and $\llbracket B \rrbracket_d^{\Mmc} = \beta$, which implies
$d\in(\B_{i} A)^{\Int_w}$ and $d\in(\B_{i} B)^{\Int_w}$.
Moreover, $\llbracket A \sqcap B \rrbracket_d^{\Mmc} = \llbracket A \rrbracket_d^{\Mmc} \cap \llbracket B \rrbracket_d^{\Mmc} = \alpha\cap\beta \notin\Nmc_{i}(w)$.
Thus $d\notin(\B_{i}(A\sqcap B))^{\Int_w}$.
%
	\item[\textnormal{$\mathit{L = \mathbf{N}}$.}]
%($\mathit{N}$-\emph{principle})
From right-to-left,
assume that $\Fmc$ contains the unit and $\top \sqsubseteq C$ is valid on $\Fmc$.
Then for all
$\Mmc = (\Fmc, \Imc)$
%$\Mmc = (\Fmc, \Delta, \Imc)$
based on $\Fmc$, and all $w$ in $\Mmc$, we have 
$C^{\Int_w} = \Delta_{w}$. 
As a consequence, for all
$d\in\Delta_{w}$,
%$d\in\Delta$,
$\llbracket C \rrbracket^{\Mmc}_{d} = \W$.
By the property of containing the unit we have that, for all $v\in\W$, $\llbracket C \rrbracket^{\Mmc}_{d} \in \Nmc_{i}(v)$.
So $d\in (\B_{i} C)^{\Int_v}$ for all
$d\in\Delta_{v}$, 
%$d\in\Delta$, 
that is, $\top \sqs \B_{i} C$ is valid on $\Fmc$.
% 
For the left-to-right direction,
%other direction,
assume that $\Fmc$ does not contain the unit, i.e.,
there is $w\in\Wmc$ such that $\W\notin\Nmc_{i}(w)$.
Then, for all models
$\Mmc= ( \Fmc, \Int )$
%$\Mmc= ( \Fmc, \Delta, \Int )$
based on $\Fmc$,
all $w \in \Wmc$,
and all
$d\in\Delta_{w}$,
%$d\in\Delta$,
%it holds the following.
%By definition,
we have $d\in\top^{\Int_w}$.
Moreover, 
since $d\in (\B_{i}\top)^{\Int_w}$ iff $\llbracket \top \rrbracket_d^{\Mmc}\in\Nmc_{i}(w)$ iff $\W\in\Nmc_{i}(w)$,
we also have $d\notin (\B_{i}\top)^{\Int_w}$.
%
	\item[\textnormal{$\mathit{L = \mathbf{P}}$.}]
          %
          From right-to-left, assume that $\Fmc = ( \Wmc, \{\Nmc_i \}_{i \in J})$ satisfies the $\mathbf{P}$-condition. I.e., for all $w \in \Wmc$ and $i \in J$, $\emptyset \not\in \Nmc_i(w)$.
          Then, for all models
          $\Mmc= ( \Fmc, \Int )$
%       $\Mmc= ( \Fmc, \Delta, \Int )$
          based on $\Fmc$,
          all $w \in \Wmc$,
          and all
          $d\in\Delta_{w}$,
%          $d\in\Delta$,
          we have $\emptyset = \llbracket \bot \rrbracket^\Mmc_d \not\in \Nmc_i(w)$. So $d \not\in (\Box_i\bot)^{\Int_w}$, or equivalently $d \in (\lnot \Box_i\bot)^{\Int_w}$. Also,
          $d \in \Delta_{w} = \top^{\Int_w}$.
%          $d \in \Delta = \top^{\Int_w}$.
          So $\top^{\Int_w} \subseteq (\lnot \Box_i\bot)^{\Int_w}$. Then $\Mmc, w \models \top \sqsubseteq \lnot \Box_i \bot$. Hence $\top \sqsubseteq \lnot \Box_i \bot$ is valid on $\Fmc$.
          %
          For the
          left-to-right direction,
%          other direction,
          assume that $\Fmc$ does not satisfy the $P$-condition for some $i \in J$. This means that there is $w \in \Wmc$ such that $\emptyset \in \Nmc_i(w)$. So there exists a model
             $\Mmc= ( \Fmc, \Int )$
%          $\Mmc= ( \Fmc, \Delta, \Int )$
          based on $\Fmc$,
          a $w \in \Wmc$, and a
          $d\in\Delta_{w}$,
%          $d\in\Delta$,
          such that $\emptyset = \llbracket \bot \rrbracket^\Mmc_d \in \Nmc_i(w)$. So $d \in (\Box_i \bot)^{\Int_w}$, or equivalently $d \not\in (\lnot\Box_i \bot)^{\Int_w}$. But $d \in \Delta_{w} = \top^{\Int_w}$. So $\top^{\Int_w} \not\subseteq (\lnot\Box_i \bot)^{\Int_w}$. Then $\Mmc, w \not\models \top \sqsubseteq \lnot\Box_i\bot$. Hence $\top \sqsubseteq \lnot \Box_i \bot$ is not valid on $\Fmc$.

	\item[\textnormal{$\mathit{L = \mathbf{Q}}$.}]
          %
          From right to left, assume that $\Fmc = ( \Wmc, \{\Nmc_i \}_{i \in J})$ satisfies the $\mathbf{Q}$-condition. I.e., for all $w \in \Wmc$ and $i \in J$, $\Wmc \not\in \Nmc_i(w)$.
          Then, for all models
          $\Mmc= ( \Fmc, \Int )$
%          $\Mmc= ( \Fmc, \Delta, \Int )$
          based on $\Fmc$,
          all $w \in \Wmc$,
          and all
          $d\in\Delta_{w}$,
%          $d\in\Delta$,
          we have $\Wmc = \llbracket \top \rrbracket^\Mmc_d \not\in \Nmc_i(w)$. So $d \not\in (\Box_i\top)^{\Int_w}$, or equivalently $d \in (\lnot \Box_i\top)^{\Int_w}$. Also $d \in \Delta = \top^{\Int_w}$. So $\top^{\Int_w} \subseteq (\lnot \Box_i\top)^{\Int_w}$. Then $\Mmc, w \models \top \sqsubseteq \lnot \Box_i \top$. Hence $\top \sqsubseteq \lnot \Box_i \top$ is valid on $\Fmc$.
          %
          For the other direction, assume that $\Fmc$ does not satisfy the $\mathbf{Q}$-condition for some $i \in J$. This means that there is $w \in \Wmc$ such that $\Wmc \in \Nmc_i(w)$.
          So there exists a model
          $\Mmc= ( \Fmc, \Int )$
%          $\Mmc= ( \Fmc, \Delta, \Int )$
          based on $\Fmc$,
          a $w \in \Wmc$,
          and a
               $d\in\Delta_{w}$,
%          $d\in\Delta$,
          such that $\Wmc = \llbracket \top \rrbracket^\Mmc_d \in \Nmc_i(w)$. So $d \in (\Box_i \top)^{\Int_w}$, or equivalently $d \not\in (\lnot\Box_i \top)^{\Int_w}$. But $d \in \Delta = \top^{\Int_w}$. So $\top^{\Int_w} \not\subseteq (\lnot\Box_i \top)^{\Int_w}$. Then $\Mmc, w \not\models \top \sqsubseteq \lnot\Box_i\top$. Hence $\top \sqsubseteq \lnot \Box_i \top$ is not valid on $\Fmc$.
          
	\item[\textnormal{$\mathit{L = \mathbf{D}}$.}]
          %
          From right-to-left, assume that $\Fmc  = ( \Wmc, \{\Nmc_i \}_{i \in J})$ satisfies the $\mathbf{D}$-condition. Moreover, let
          $\Mmc= ( \Fmc, \Int )$
%          $\Mmc= ( \Fmc, \Delta, \Int )$
          based on $\Fmc$, with a world $w \in \Wmc$, and
          $d \in \Delta_{w}$,
%          $d \in \Delta$,
          a concept $C$, and $i \in J$, all arbitrarily chosen. Suppose that $d \in (\Box_i C)^{\Int_w}$. It means that $[C]^\Mmc_d \in \Nmc_i(w)$. By the $\mathbf{D}$-condition, $\Wmc \setminus [C]^\Mmc_d \not\in \Nmc_i(w)$. Equivalently, $[\lnot C]^\Mmc_d \not\in \Nmc_i(w)$. This means that $d \not\in (\Box_i\lnot C)^{\Int_w}$, or equivalently, that $d \in (\lnot\Box_i\lnot C)^{\Int_w}$. So $(\Box_iC)^{\Int_w} \subseteq (\lnot \Box_i\lnot C)^{\Int_w}$. Then $\Mmc,w \models \Box_i C \sqsubseteq \Diamond_i C$. Hence, $\Box_i C \sqsubseteq \Diamond_i C$ is valid on $\Fmc$.
          %
          For the left-to-right direction, assume that $\Fmc$ does not satisfy the $\mathbf{D}$-condition for $i \in J$. So there is a $w \in \Wmc$, such that, for some $\alpha \subseteq \Wmc$, we have $\alpha \in \Nmc_i(w)$ and $\Wmc \setminus \alpha \in \Nmc_i(w)$.
          We define
            $\Mmc= (\Fmc, \Int )$
%          $\Mmc= (\Fmc, \Delta, \Int )$
          based on $\Fmc$,
          where,
          for any $v \in \Wmc$,
          $\Delta_{v} = \{d\}$,
%          $\Delta = \{d\}$,
          and $A^{\Int_v} = \{d\}$ iff $v \in \alpha$ (and defined arbitrarily on all other symbols in $( \NC \cup \NR ) \setminus \{A\}$). We have $\llbracket A \rrbracket^\Mmc_d = \{v \in \Wmc \mid d \in A^{\Int_w}\} = \alpha$, and $\llbracket \lnot A \rrbracket^\Mmc_d = \Wmc \setminus \alpha$. So $d \in (\Box_i A)^{\Int_w}$ and $d \in (\Box_i \lnot A)^{\Int_w}$, and then $d \not \in (\lnot \Box_i \lnot A)^{\Int_w}$. So $(\Box_iA)^{\Int_w} \not \subseteq (\lnot \Box_i \lnot A)^{\Int_w}$. This means that, $\Mmc, w \not\models \Box_i A \sqsubseteq \Diamond_i A$. Hence, $\Box_i C \sqsubseteq \Diamond_i C$ is not valid on $\Fmc$.
          

	\item[\textnormal{$\mathit{L = \mathbf{T}}$.}]
          %
          From right-to-left, assume that $\Fmc = ( \Wmc, \{\Nmc_i \}_{i \in J})$ satisfies the $\mathbf{T}$-condition. Moreover, let
          $\Mmc= ( \Fmc, \Int )$
%          $\Mmc= ( \Fmc, \Delta, \Int )$
          based on $\Fmc$, with a world $w \in \Wmc$, and
          $d \in \Delta_{w}$,
%          $d \in \Delta$,
          a concept $C$, and $i \in J$, all arbitrarily chosen. Suppose that $d \in (\Box_i C)^{\Int_w}$. It means that $\llbracket C \rrbracket^\Mmc_d \in \Nmc_i(w)$. By the $\mathbf{T}$-condition, $w \in \llbracket C \rrbracket^\Mmc_d$. So $d \in C^{\Int_w}$. So $(\Box_i C)^{\Int_w} \subseteq C^{\Int_w}$. Then $\Mmc, w \models \Box_i C \sqsubseteq C$. Hence, $\Box_i C \sqsubseteq C$ is valid on $\Fmc$.
          %
          For the left-to-right direction, assume that $\Fmc$ does not satisfy the $\mathbf{T}$-condition for $i \in J$. So there is a $w \in \Wmc$, such that for some $\alpha \subseteq \Wmc$ we have $\alpha \in \Nmc_i(w)$ and $w \not\in \alpha$. 
          We define
          $\Mmc= (\Fmc, \Int )$
%          $\Mmc= (\Fmc, \Delta, \Int )$
          based on $\Fmc$, where
          $\Delta_{v} = \{d\}$, for any $v \in \Wmc$,
%          $\Delta = \{d\}$,
          and $A^{\Int_v} = \{d\}$ iff $v \in \alpha$ (and defined arbitrarily on all other symbols in $( \NC \cup \NR ) \setminus \{A\}$). We have $\llbracket A \rrbracket^\Mmc_d = \{v \in \Wmc \mid d \in A^{\Int_w}\} = \alpha$. So $d \in (\Box_i A)^{\Int_w}$. Since $w \not \in \alpha$, $w \in \Wmc \setminus \alpha$. That is, $w \in \Wmc \setminus \llbracket A \rrbracket^\Mmc_d$, or equivalently $w \in \llbracket \lnot A \rrbracket^\Mmc_d$. Then, $d \in (\lnot A)^{\Int_w}$, or equivalently $d \not \in (A)^{\Int_w}$. So, 
%          $(\lnot A)^{\Int_w} \not\subseteq A^{\Int_w}$, 
$(\Box_i A)^{\Int_w} \not\subseteq A^{\Int_w}$,
%\nb{O: fixed problem here} 
          meaning that  $\Mmc, w \not\models \Box_i A \sqsubseteq A$. Hence, $\Box_i C \sqsubseteq C$ is not valid on $\Fmc$, as required.
\qedhere
\end{description}
%\nb{T: Work in progress}
\end{proof}




















\PropValid*
\begin{proof}
\emph{Point~1.}
{{The $(\Rightarrow)$ direction follows from the proof of Proposition~\ref{prop:corresp}.
To see that the $(\Leftarrow)$ direction does not hold in general,
we provide the following counterexample
showing that the $\mathbf{T}$-principle holds in a model that does not satisfy the $\mathbf{T}$-condition.
Consider 
$\Mmc = ( \Wmc, \{\Nmc_i \}_{i \in J}, \Int)$,
where 
\begin{itemize}
\item $\Wmc = \{w, v\}$;
\item
$\Nmc_{i}(w) = \{\{v\},\Wmc\}$ and
$\Nmc_{i}(v) = \{\{w\},\Wmc\}$,
 for $i \in J$;
%\item $\Delta_{w} = \Delta_{v} = \{d\}$;
%\item $\Imc_{w} = \Imc_{v}$.
\item $\Imc_{w} = \Imc_{v}$, with $\Delta_{w} = \Delta_{v} = \{d\}$.
\end{itemize}
$\Mmc$ does not satisfy the $\mathbf{T}$-condition,
since $\{v\} \in \Nmc_{i}(w)$ but $w \notin \{v\}$.
We show that the $\mathbf{T}$-principle holds in $\Mmc$.

\begin{claim}
For all concepts $C$, % every concept $C$, 
$\llbracket C \rrbracket^{\Mmc}_{d} = \emptyset$ or
$\llbracket C \rrbracket^{\Mmc}_{d} = \Wmc$. % \{w,v\}$.
\end{claim}
\begin{proof}[Proof of Claim]
By induction on the construction of $C$.

For the base case $C = A \in \NC$,
it follows from the definition that either
$d \in A^{\Imc_{w}}$ and $d \in A^{\Imc_{v}}$,
hence 
$\llbracket A \rrbracket^{\Mmc}_{d} = \Wmc$, % \{w,v\}$,
or
$d \notin A^{\Imc_{w}}$ and $d \notin A^{\Imc_{v}}$,
hence 
$\llbracket A \rrbracket^{\Mmc}_{d} = \emptyset$.

We now show the inductive cases.
For $C = \lnot D, D \sqcap E$, the proof is immediate by applying the induction hypothesis.

For $C = \exists \role.D$, the proof follows by the application of the induction hypothesis
and the fact that $r^{\Imc_{w}} = r^{\Imc_{v}}$.

For $C = \B_{i} D$, by induction hypothesis
$\llbracket D \rrbracket^{\Mmc}_{d} = \emptyset$ or
$\llbracket D \rrbracket^{\Mmc}_{d} = \Wmc$. %\{w,v\}$.
In the first case,
$\llbracket D \rrbracket^{\Mmc}_{d} \notin \Nmc_{i}(w)$
and
$\llbracket D \rrbracket^{\Mmc}_{d} \notin \Nmc_{i}(v)$,
hence $w \notin \llbracket \B_{i} D \rrbracket^{\Mmc}_{d}$
and $v \notin \llbracket \B_{i} D \rrbracket^{\Mmc}_{d}$,
thus $ \llbracket \B_{i} D \rrbracket^{\Mmc}_{d} = \emptyset$.
In the second case,
$\llbracket D \rrbracket^{\Mmc}_{d} \in \Nmc_{i}(w)$
and
$\llbracket D \rrbracket^{\Mmc}_{d} \in \Nmc_{i}(v)$,
hence $w \in \llbracket \B_{i} D \rrbracket^{\Mmc}_{d}$
and $v \in \llbracket \B_{i} D \rrbracket^{\Mmc}_{d}$,
thus $ \llbracket \B_{i} D \rrbracket^{\Mmc}_{d} = \Wmc$. % \{w,v\}$.
\end{proof}

Now, given a concept $C$, suppose that $d \in (\B_{i} C)^{\Imc_{w}}$,
that is, $\llbracket C \rrbracket^{\Mmc}_{d} \in \Nmc_{i}(w)$.
By the claim, $\llbracket C \rrbracket^{\Mmc}_{d} = \emptyset$
or $\llbracket C \rrbracket^{\Mmc}_{d} = \Wmc$.
Since $\emptyset \notin \Nmc_{i}(w)$,
we have that $\llbracket C \rrbracket^{\Mmc}_{d} = \Wmc$,
and thus $w \in \llbracket C \rrbracket^{\Mmc}_{d}$.
This means that $d \in C^{\Imc_{w}}$.
By the same argument, we can show that 
$d \in (\B_{i} C)^{\Imc_{v}}$ implies
$d \in C^{\Imc_{v}}$.
Therefore $\Mmc \models \B_{i} C \sqsubseteq C$.
%for any concept $C$.
Similarly we can prove that $\Mmc \models \Box_{i} \varphi \to \varphi$,
for any formula $\varphi$.
}}



\emph{Point~2.}
Let $\Fmf = (W, \{ R_{i} \}_{i \in J})$ be a relational frame, and let $\Mmf = (F, \Delta, I)$ be a relational model based on $\Fmf$.

%\todo{M: add $\mathbf{E}$-principle}

($\mathbf{E}$-principle) Follows directly from the ($\mathbf{M}$-principle) case below.

($\mathbf{M}$-principle) Assume $C \sqsubseteq D$ valid in $\Mmf$. Then $C^{I_w} \subseteq D^{I_w}$, for all $w \in W$. Now, suppose that $d\in (\B_{i} C)^{I_{w}}$, for $d \in \Delta$ and $w \in W$.
For all $v \in W$, $w \relations_{i} v$ implies $d\in C^{I_{v}}$,
hence $d\in D^{I_{v}}$.
Therefore, $d\in (\B_{i} D)^{I_{w}}$.
%and so $\B_{i} C \sqs \B_{i} D$ is valid in $\Mmf$.
%

($\mathbf{C}$-principle) Assume $d\in(\B_{i} C \sqcap \B_{i} D)^{I_{w}}$, that is, $d\in(\B_{i} C)^{I_{w}}$ and $d\in(\B_{i} D)^{I_{w}}$.
Then, for all $v \in W$, $w\relations_{i} v$ implies $d\in C^{I_{v}}$ and $d\in D^{I_{v}}$,
that is $d\in (C\sqcap D)^{I_{v}}$.
Therefore, $d\in(\B_{i} ( C \sqcap D))^{I_{w}}$.
%

($\mathbf{N}$-principle) Assume $\top \sqsubseteq C$ valid in $\Mmf$.
Then, for all $w \in W$,  $C^{I_w} =\Delta$.
Thus, for all $d\in\Delta$ and all $v \in W$, $w \relations_{i} v$ implies $d\in C^{I_{v}}$.
In conclusion, for all $d \in \Delta$, $d\in (\B_{i} C)^{I_{w}}$. 
%Therefore $\top \sqsubseteq \B_{i} C$ is valid in $\Mmf$.

In conclusion, the
$\mathbf{E}$-,
$\mathbf{M}$-, $\mathbf{C}$-, and $\mathbf{N}$-principle
%monotonicity, agglomeration, and necessitation
hold in $M$, and hence in $F$.

{{
($\mathbf{D}$-principle and $\mathbf{P}$-principle are equivalent)
It is easy to see that both the $\mathbf{D}$- and the $\mathbf{P}$-principle
hold if and only if 
$R_{i}$ is serial, for all $i \in J$
(that is, for all $w \in W$, there is $v\in W$ such that $w R_i v$).


($\mathbf{Q}$-principle does not hold)
The $\mathbf{Q}$-principle is incompatible with the $\mathbf{N}$-principle, whcih holds in relational models.
Indeed, if both principles hold in the relational model $\Mmf$,
then $\Mmf\models\top\sqsubseteq \Box_i\top\sqcap\lnot\Box_i\top$,
against the fact that $\ALC$ domains are non-empty.
}}
%\todo{T: added short proofs for D, P, Q \\ M: thanks}
\end{proof}


%\section{Proofs for Section~\ref{sec:reasonvardom}}
\section{Proofs for Section~\ref{sec:tableaux}}




Here is an $\LnALC$ tableau algorithm example application.
%The following is example of application of the $\LnALC$ tableau algorithm.

%\subsection{Examples}
%\todo{T:added example}
\begin{example}
Consider the formula
%As an example of application of the $\LnALC$ tableau algorithm,
%we consider the formula
%$\p = $
$$\varphi = \lnot(\mathbb{D}_c\exists\mathsf{req}.(\mathsf{Prod} \sqcap \mathsf{InCatal}) \sqsubseteq
\mathbb{D}_c(\mathsf{Conf} \sqcup\lnot\mathsf{Conf})),$$
related to the discussion in Section~\ref{sec:model}.
We recall that the formula is unsatisfiable in models %satisfying 
validating the $\mathbf{M}$-condition,
and it is satisfiable otherwise.
Here we show that the algorithm provides different answers depending whether $\mathbf{M}\in\Lvar$.
%First, we rewrite $\p$ in NNF, replacing $\mathbb{D}_c$ with $\Box_c$ in order to fit the language of the tableau,
%thus obtaining
%$$\lnot(\top \sqsubseteq \Diamond_c\forall \mathsf{req}.
%(\lnot \mathsf{Prod} \sqcup \lnot \mathsf{InCatal}) \sqcup \Box_c(\mathsf{Conf} \sqcup \lnot \mathsf{Conf}))$$
First, we rewrite $\p$ in NNF, using $\widehat{\mathbb{D}}_c$ as the dual operator of $\mathbb{D}_c$,
thus obtaining
$$\lnot(\top \sqsubseteq \widehat{\mathbb{D}}_c\forall \mathsf{req}.
(\lnot \mathsf{Prod} \sqcup \lnot \mathsf{InCatal}) \sqcup \mathbb{D}_c(\mathsf{Conf} \sqcup \lnot \mathsf{Conf})).$$
We then consider the following applications of the tableau algorithm.
In the first case we assume $\mathbf{M}\in\Lvar$:


%\noindent
%$0 : \p$
%
%\noindent
%$0 : (\Box_c \exists\mathsf{req}.(\mathsf{Prod} \sqcap \mathsf{InCatal}) \sqcap
%\Diamond_c(\lnot\mathsf{Conf} \sqcap\mathsf{Conf}))(v)$ 
%\hfill ($\mathsf{R}_{\not\sqsubseteq}$)
%
%\noindent
%$1 : (\exists\mathsf{req}.(\mathsf{Prod} \sqcap \mathsf{InCatal}))(v)$
%\hfill ($\mathsf{R}_{\mathit{L}}$)
%
%\noindent
%$1 : (\lnot\mathsf{Conf} \sqcap\mathsf{Conf})(v)$
%\hfill ($\mathsf{R}_{\mathit{L}}$)
%
%\noindent
%$1 : (\lnot\mathsf{Conf})(v)$
%\hfill ($\mathsf{R}_{\sqcap}$)
%
%\noindent
%$1 : (\mathsf{Conf})(v)$
%\hfill ($\mathsf{R}_{\sqcap}$)

\medskip
\noindent
$0 : \p$

\noindent
$0 : \mathbb{D}_c \exists\mathsf{req}.(\mathsf{Prod} \sqcap \mathsf{InCatal}) \sqcap
\widehat{\mathbb{D}}_c(\lnot\mathsf{Conf} \sqcap\mathsf{Conf})(v)$ 
\hfill ($\mathsf{R}_{\not\sqsubseteq}$)

\noindent
$1 : \exists\mathsf{req}.(\mathsf{Prod} \sqcap \mathsf{InCatal})(v)$
\hfill ($\mathsf{R}_{\mathit{L}}$)

\noindent
$1 : \lnot\mathsf{Conf} \sqcap\mathsf{Conf}(v)$
\hfill ($\mathsf{R}_{\mathit{L}}$)

\noindent
$1 : \lnot\mathsf{Conf}(v)$
\hfill ($\mathsf{R}_{\sqcap}$)

\noindent
$1 : \mathsf{Conf}(v)$
\hfill ($\mathsf{R}_{\sqcap}$)

\medskip


\noindent
The completion set constructed by the application of the $\LnALC$ tableau algorithm contains a clash,
hence the algorithm returns $\mathsf{unsatisfiable}$ on input $\p$. 
Now, assume $\mathbf{M}\notin\Lvar$:

\medskip

\noindent
$0 : \p$

\noindent
$0 : \mathbb{D}_c \exists\mathsf{req}.(\mathsf{Prod} \sqcap \mathsf{InCatal}) \sqcap
\widehat{\mathbb{D}}_c(\lnot\mathsf{Conf} \sqcap\mathsf{Conf})(v)$ 
\hfill ($\mathsf{R}_{\not\sqsubseteq}$)

\noindent
$1 : \forall \mathsf{req}.(\lnot \mathsf{Prod} \sqcup \lnot \mathsf{InCatal})(v)$
\hfill ($\mathsf{R}_{\mathit{L}}$)

\noindent
$1 : \mathsf{Conf} \sqcup \lnot \mathsf{Conf}(v)$
\hfill ($\mathsf{R}_{\mathit{L}}$)

\noindent
$1 : \mathsf{Conf}(v)$
\hfill ($\mathsf{R}_{\sqcup}$)


\medskip

\noindent
The completion set 
%constructed by the application of the $\LnALC$ tableau algorithm
is clash-free and $\LnALC$-complete,
hence the algorithm returns $\mathsf{satisfiable}$ on input $\p$.
Note that the
latter
applications of $\mathsf{R}_{\mathit{L}}$
are only possible if $\mathbf{M}\notin\Lvar$.
%application of $\mathsf{R}_{\mathit{L}}$
%considered in the second example
%is only possible if $\mathbf{M}\notin\Lvar$.

\end{example}


%\nb{M: todo def. weight}
%{\color{blue}{
%\todo{T:Maybe the definition of weight can be moved to the appendix.}
In this appendix we prove termination, soundness, and completeness of the $\LnALC$ tableau algorithm.
We define the \emph{weight} $|C|$ of a concept $C$ in NNF as follows: $|A| = |\lnot A| = 0$; $|\exists r.D| = |\forall r.D| = |\Box_{i}D| = |\Diamond_{i}D| = |D| + 1$; $|D \sqcap E| = |D \sqcup E| = |D| + |E| + 1$. The \emph{weight $|\p|$} of a formula $\p$ in NNF is defined as:
$|C(a)| = |r(a,b)| = |\lnot r(a,b)| =
| (C \sqsubseteq D) | = | \lnot (C \sqsubseteq D) | = 0$; 
%$\Box_{i} \psi = | \psi | + 1$; 
$|\Box_{i} \psi| = |\Diamond_i\psi| = | \psi | + 1$;
$| \psi \land \chi | = | \psi \lor \chi | = | \psi | + | \chi | + 1$. %}}
Observe that, for a concept or formula $\gamma$, we have that 
$| \gamma | = | \dnot \gamma |$.


%\subsection{Proofs for Section~\ref{sec:tableaux}}


\begin{restatable}[Termination]{theorem}{Termination}
	\label{thm:termination}
	%Having started on the initial completion set $\T_{\p} =  \{0 : \p
	%0 : \top(x)
	%\}$, 
		The $\LnALC$ tableau algorithm for $\p$ terminates after at most $2^{p(|\fg(\p)|)}$ steps, where $p$ is a polynomial function.
%	\nb{O: strange here, talk about non-determinism}
%	The non-deterministic $\LnALC$ tableau algorithm with $\p$ as input terminates after at most $2^{p(|\fg(\p)|)}$ steps, where $p$ is a polynomial function.
%%	\nb{O: strange here, talk about non-determinism}
\end{restatable}
%




%%\vspace{3cm}
%\begin{theorem}[Termination]
%\label{thm:termination}
%The $\LnALC$ tableau algorithm terminates for every formula $\p$.
%\end{theorem}
%%\begin{proof}
%%The theorem follows as a consequence of the following two claims:
%%(i) For each label $n$, the algorithm generates finitely many $n$-labelled constraints.
%%``The algorithm behaves `locally' (i.e., with respect to a label $n$) as a standard terminating tableaux algorithm for ALC''.
%%(ii) The algorithm generates finitely many labels.
%%%$n$.
%%Then, the number of
%%%the
%%labels
%%%$n$
%%that can be generated by the algorithm is bounded by the number of possible combinations of modal subformulas (possibly converted into their NNF) and modalised concepts occurring in the initial formula $\p$.
%%\end{proof}
%\begin{proof}
%The theorem follows as a consequence of the following two claims:
%\begin{itemize}
%\item[(i)] For each label $n$, the algorithm generates finitely many $n$-labelled constraints.
%\item[(ii)] The algorithm generates finitely many labels $n$.
%\end{itemize}
%For (i), observe that for each $S_n$, the algorithm standardly builds an ALC-tableaux ... \textcolor{red}{size??}
%``The algorithm behaves `locally' (i.e., with respect to a label $n$) as a standard terminating tableaux algorithm for ALC''.
%%%$n$.
%%Then, the number of
%%%the
%%labels
%%%$n$
%%that can be 
%For (ii), observe that the number of labels $n$
%generated by the algorithm is bounded by the number of possible combinations of modal subformulas (possibly converted into their NNF) and modalised concepts occurring in the initial formula $\p$.
%\textcolor{red}{number of combinations??}
%\end{proof}





%%% SOUNDNESS




%\Termination*
\begin{proof}
We first require the following claims.
%The statement is a consequence of the following claims.
\begin{claim}
\label{cla:termlocal}
Let $\T$ be a completion set obtained by applying the $\LnALC$ tableau algorithm for $\p$.
For each $n \in \mathsf{L}_{\T}$,
%Let $S_{n}$ be a $n$-labelled constraint system for $\p$ in 
%$\T$.
the number of 
%constraints
$n$-labelled constraints
%of the form $n: \psi$, or $n: C(x)$, or $n: r(x, y)$ 
for $\p$
in $\T$ does not exceed $2^{q(|\fg(\p)|)}$, where $q$ is a polynomial function. 
\end{claim}
\begin{proof}[Proof of Claim]
%\nb{M: changed, to be checked}
We remark that, for each $S_n\subseteq\T$, the $\LnALC$ tableaux algorithm behaves exactly like a standard (non-modal) $\ALC$ tableaux algorithm (cf. e.g.~\cite[Theorem 15.4]{GabEtAl03}), 
%{{noting also that in our case we do not have to deal with individual names}}),
%\todo{T:Now we have individual names. Need to modify the proof?}
%\todo{T:Deleted sentence ``noting also that in our case we do not have to deal with individual names''}
{{except possibly for the additional rule $\mathsf{R}_{\mathit{L}\mathbf{T}}$ which introduces at most
$| \fg(\p) |$ $n$-labelled contraints}}.
\end{proof}

%% Old claim and proof, only for E, M , C, N
%\begin{claim}
%\label{cla:termglobal}
%Let $\mathbf{T}$ be a completion set obtained by applying the $\LnALC$ tableau algorithm for $\p$.
%For
%$\mathit{L} \in \{\mathbf{E}, \mathbf{M}, \mathbf{N}\}$,
%%when started on $\mathbf{T}_{\p}$,
%%the number of labels generated by the $\LnALC$ tableau algorithm does not exceed $|\fg(\p)|^2$.
%$|\mathsf{L}_{\mathbf{T}}| \leq |\fg(\p)|^2$.
%%%the cardinality of $\mathsf{L}_{\mathbf{T}}$
%%%does not exceed
%%%$|\fg(\p)|^2$.
%For
%$\mathit{L} = \mathbf{C}$,
%%when started on $\mathbf{T}_{\p}$,
%%the number of labels generated by the $\LnALC$ tableau algorithm does not exceed
%%$2^{|\fg(\p)|} \cdot |\fg(\p)|$.
%$|\mathsf{L}_{\mathbf{T}}| \leq 2^{|\fg(\p)|} \cdot |\fg(\p)|$.
%%%the cardinality of $\mathsf{L}_{\mathbf{T}}$
%%%does not exceed
%%%$2^{|\fg(\p)|} \cdot |\fg(\p)|$.
%\end{claim}
%\begin{proof}[Proof of Claim]
%%World
%Labels $n$ are generated in $\mathbf{T}$ by means
%of the application of the rule $\mathsf{R}_{\mathit{L}}$.
%For $\mathit{L} \in \{\mathbf{E}, \mathbf{M}, \mathbf{N}\}$,
%this rule is applied to two $n$-labelled contraints
%$n: \Box_i\gamma, n: \Diamond_{i}\delta$
%(for $\mathit{L} = \mathbf{N}$ possibly also to a single constraint
%$n: \Diamond_{i}\delta$),
%%whereas 
%while for $\mathit{L} = \mathbf{C}$
%it is applied to $k+1$ $n$-labelled contraints
%$n: \Box_i\gamma_1, ... n: \Box_i\gamma_{k}, n: \Diamond_{i}\delta$.
%By the application condition of $\mathsf{R}_{\mathit{L}}$,
%each such combination
%of
%%\textcolor{red}{modal
%constraints
%%}
%generates at most one label $m$.
%Therefore, the number of
%%world
%labels that can be generated in $\mathbf{T}$ is bounded by the number of possible
%such
%combinations,
%% of this kind,
%which is at most $|\fg(\p)|^2$, for $\mathit{L} \in \{\mathbf{E}, \mathbf{M}, \mathbf{N}\}$,
%and at most $2^{|\fg(\p)|} \cdot |\fg(\p)|$, for $\mathit{L} = \mathbf{C}$.
%%\nb{T: which symbol for multiplication? M: this one you used}
%\end{proof}

%% Claim and proof extended to all systems
\begin{claim}
\label{cla:termglobal}
{{Let $\T$ be a completion set obtained by applying the $\LnALC$ tableau algorithm for $\p$.
%If $\mathbf{C} \notin \mathit{L}$, then $|\mathsf{L}_{\T}| \leq c |\fg(\p)|^2$
%for some constant $c$.
%If $\mathbf{C} \in \mathit{L}$, then 
%$|\mathsf{L}_{\T}| \leq 2^{c |\fg(\p)|}$
%for some constant $c$.
Then $|\mathsf{L}_{\T}| \leq r( |\fg(\p)| )$
if $\mathbf{C} \notin \mathit{L}$, and 
$|\mathsf{L}_{\T}| \leq 2^{r'(|\fg(\p)|)}$
if $\mathbf{C} \in \mathit{L}$,
for some for some polynomial functions $r$ and $r'$.
}}
\end{claim}
\begin{proof}[Proof of Claim]
{{
Labels $n$ are generated in $\T$ by means
of the application of the rules $\mathsf{R}_{\mathit{L}}$,
$\mathsf{R}_{\mathit{L}\mathbf{N}}$,
$\mathsf{R}_{\mathit{L}\mathbf{P}}$,
$\mathsf{R}_{\mathit{L}\mathbf{Q}}$,
$\mathsf{R}_{\mathit{L}\mathbf{D}}$.
If $\mathbf{C}\notin\mathit{L}$,
these rules are applied to either one or two $n$-labelled contraints,
while if $\mathbf{C}\in\mathit{L}$,
they are applied to $k$, $k+1$ or $k + h$ $n$-labelled contraints.
By the application conditions of the rules,
each such combination
of constraints
generates at most one label $m$.
Therefore, the number of
labels that can be generated in $\T$ is bounded by the number of possible
such
combinations,
which is at most $2 \cdot |\fg(\p)|^2 + 3 \cdot |\fg(\p)|$,  if $\mathbf{C}\not\in\mathit{L}$,
and at most 
$2^{|\fg(\p)|} \cdot |\fg(\p)| +
|\fg(\p)| +
2^{|\fg(\p)| + 1} + 
2^{2|\fg(\p)|}$, 
if $\mathbf{C}\in\mathit{L}$.
}}
\end{proof}

{{
The theorem is then a consequence of the following observations.
Given a completion set $\T$ constructed by the $\LnALC$ tableau algorithm for $\p$,
we have by Claim~\ref{cla:termglobal} that
the number of applications of 
the rules
$\mathsf{R}_{\mathit{L}}$, $\mathsf{R}_{\mathit{L}\mathbf{N}}$, $\mathsf{R}_{\mathit{L}\mathbf{P}}$, $\mathsf{R}_{\mathit{L}\mathbf{Q}}$, and $\mathsf{R}_{\mathit{L}\mathbf{D}}$ 
is bounded by $| \mathsf{L}_{\T} |$, which is at most 
$r( |\fg(\p)| )$,
for
$\mathit{L}$ such that $\mathbf{C}\notin\Lvar$,
and at most
$2^{r'(|\fg(\p)|)}$, for $\Lvar$ such that
 $\mathbf{C} \in \mathit{L}$,
%for some for some polynomial functions $p$ and $q$.
where $r$ and $r'$ are polynomial functions.
%Moreover, since every application of the rules
%$\mathsf{R}_{\land}$ and $\mathsf{R}_{\lor}$ introduces a new formula to an $n$-labelled constraint, the total number of such rule applications is bounded by $| \mathsf{L}_{\T} | \cdot | \fg(\p) |$.
Moreover, for a given label $n$, 
the number of possible applications of the rules
$\mathsf{R}_{\land}$, $\mathsf{R}_{\lor}$ and $\mathsf{R}_{\mathit{L}\mathbf{T}}$ to constraints of the form $n: \psi$
is linearly bounded by $\fg(\p)$,
hence there are at most $| \mathsf{L}_{\T} | \cdot q'(| \fg(\p) |)$ such rule applications, where $q'$ is a polynominal function.
%Finally, by Claim~\ref{cla:termlocal}, the number of applications of rules
%$\mathsf{R}_{\sqcap}, \mathsf{R}_{\sqcup}, \mathsf{R}_{\forall}, \mathsf{R}_{\exists}, \mathsf{R}_{=}, \mathsf{R}_{\neq}$ per label $n$ is bounded by $2^{q(|\fg(\p)|)}$, where $q$ is a polynomial function, since these rules add a new constraint to an $n$-labelled constraint system.
%Thus, the overall number of such rule applications is bounded by $| \mathsf{L}_{\T} | \cdot 2^{q(|\fg(\p)|)}$.
Finally, by Claim~\ref{cla:termlocal}, 
for each label $n$, the number of applications of the rules
$\mathsf{R}_{\sqcap}, \mathsf{R}_{\sqcup}, \mathsf{R}_{\forall}, \mathsf{R}_{\exists}, \mathsf{R}_{\sqsubseteq}, \mathsf{R}_{\not\sqsubseteq}$  and $\mathsf{R}_{\mathit{L}\mathbf{T}}$ %per label $n$
 to constraints of the form $n: C(x)$ or $n: r(x, y)$ is bounded by $2^{q(|\fg(\p)|)}$, where $q$ is a polynomial function,
hence there are at most $| \mathsf{L}_{\T} | \cdot 2^{q(|\fg(\p)|)}$ such rule applications.
%Thus, 
It follows that the overall number of rule applications is bounded by 
%$| \mathsf{L}_{\T} | \cdot 2^{r(|\fg(\p)|)}$
$2^{p(|\fg(\p)|)}$
for some polynomial function $p$.
}}
\end{proof}





















We now proceed to prove that the $\LnALC$ tableau algorithm is sound. 
%\nb{O: maybe change the wording or even rewrite the theorems as one, now that the proofs were moved to the appendix}

%\newpage
%\begin{theorem}[Soundness]
%\label{thm:soundness}
%If, having started on the initial completion set $\T_{\p}$, the $\LnALC$ tableau algorithm constructs an $\LnALC$-complete and clash-free completion set for $\p$, then $\p$ is $\LnALC$ satisfiable.
%\end{theorem}
\begin{restatable}[Soundness]{theorem}{Soundness}
	\label{thm:soundness}
	If there exists an execution of the $\LnALC$ tableau algorithm for $\p$ that constructs a complete and clash-free completion set, then $\p$ is $\LnALC$ satisfiable.
%	If %, having started on the initial completion set $\T_{\p}$, %changed because this would make us look inside the alg, so simpler to say "the alg with blah as input"
%	the $\LnALC$ tableau algorithm with $\p$ as input returns $\mathsf{satisfiable}$, then $\p$ is $\LnALC$ satisfiable.
\end{restatable}
%






%\Soundness*
\begin{proof}
Suppose that  the $\LnALC$ tableau algorithm for $\p$
%returns $\mathsf{satisfiable}$.
%This means that it
constructs
 an $\LnALC$-complete and clash-free completion set $\T$ for $\p$.
% Given $\T$,
%let $\mathsf{L}_{\T} = \{ n \in \mathsf{N_{L}} \mid S_{n} \subseteq \T \}$.
We define, for $n \in \mathsf{L}_{\T}$, $\psi \in \forneg(\p)$, $C \in \conneg(\p)$, and $x$ occurring in $\T$,
%\begin{align*}
%	\lfloor \psi \rfloor & = \{ n \in \mathsf{L}_{\mathbf{T}} \mid n : \psi \in S_{n} \}, \\
%	\lfloor C \rfloor_{x} & = \{ n \in \mathsf{L}_{\mathbf{T}} \mid n : C(x) \in S_{n} \},
%\end{align*}
%and
%\begin{align*}
%	\lceil \psi \rceil & = \mathsf{L}_{\mathbf{T}} \setminus \{ n \in \mathsf{L}_{\mathbf{T}} \mid n : \dnot\psi \in S_{n}\}, \\
%	\lceil C \rceil_{x} & = \mathsf{L}_{\mathbf{T}} \setminus \{ n \in \mathsf{L}_{\mathbf{T}} \mid n : \dnot C(x) \in S_{n} \}.
%\end{align*}

%\begin{center}
\begin{align*}
	\lfloor C \rfloor_{x} & = \{ n \in \mathsf{L}_{\T} \mid n : C(x) \in S_{n} \}, \\
	\lceil C \rceil_{x} & = \mathsf{L}_{\T} \setminus \{ n \in \mathsf{L}_{\T} \mid n : \dnot C(x) \in S_{n} \}, \\
	\lfloor \psi \rfloor & = \{ n \in \mathsf{L}_{\T} \mid n : \psi \in S_{n} \}, \\
	\lceil \psi \rceil & = \mathsf{L}_{\T} \setminus \{ n \in \mathsf{L}_{\T} \mid n : \dnot\psi \in S_{n}\}. \\
\end{align*}
%\end{center}

\noindent
Moreover, define $\Gamma^{x}_{n} =  \{ \psi \mid n : \psi \in S_{n} \} \cup \{ C \mid  n : C(x) \in S_{n} \}$ and let $\gamma, \delta$ range over $\MLnALC$ formulas or concepts,
where: $\lfloor \gamma \rfloor_{x} = \lfloor \psi \rfloor$, if $\gamma = \psi$, and $\lfloor \gamma \rfloor_{x}  = \lfloor C \rfloor_{x} $, if $\gamma = C$; and similarly for $\lceil \gamma \rceil_{x}$.
%so that: if $\gamma = C$ and $\Box_{i}\gamma \in \Gamma^{x}_{n} $, then $\lceil \gamma \rceil = \lceil C \rceil_{x}$, $\rfloor \gamma \lfloor = \rfloor C \lfloor_{x}$; whereas, if $\gamma = \psi$ and $\Box_{i}\gamma \in \Gamma^{x}_{n} $< then $\lceil \gamma \rceil = \lceil \psi \rceil$, $\rfloor \gamma \lfloor = \rfloor \psi \lfloor$.
%
We set $\Mmc = (\Fmc, \Imc)$, with $\Fmc = (\Wmc, \{ \Nmc_{i} \}_{i \in J})$ and $\Imc_{n} = (\Delta_{n}, \cdot^{\Imc_{n}})$, for $n \in \Wmc$, defined as follows:
\begin{itemize}
	\item $\Wmc =  \mathsf{L}_{\T}$;
	\item for every $i \in J = \{1, \ldots, n\}$, we set $\Nmc_{i} \colon \W \rightarrow 2^{2^{\Wmc}}$ such that: %\todo{T: we are using $n$ in too many ways here}
%	\nb{M: to fix $\rfloor_{x}$}
%old&good
%		\begin{enumerate}[leftmargin=*, align=left]
%%		{{
%			\item[for $\mathit{L} = \mathbf{E}$:]
%			\[
%				\Nmc_{i}(n) = \big\{ \alpha \mid \textnormal{for some} \ \Box_{i}\gamma \in \Gamma^{x}_{n}  \colon \lfloor \gamma \rfloor_{x} \subseteq \alpha \subseteq \lceil \gamma \rceil_{x} \big\};
%			\]
%			\item[for $\mathit{L} = \mathbf{M}$:]
%			\[
%				\Nmc_{i}(n) = \big\{ \alpha \mid \textnormal{for some} \  \Box_{i}\gamma \in \Gamma^{x}_{n}  \colon \lfloor \gamma \rfloor_{x} \subseteq \alpha \big\};
%			\]
%			\item[for $\mathit{L} = \mathbf{C}$:]
%			\[
%				\Nmc_{i}(n) = \big\{ \alpha \mid \textnormal{for some} \ \Box_{i}\gamma_{1}  \in \Gamma^{{x}_{1}}_{n}, \ldots, \Box_{i}\gamma_{k} \in \Gamma^{{x}_{k}}_{n} \colon
%				\bigcap^{k}_{j = 1} \lfloor \gamma_{j} \rfloor_{{{x}_{j}}}
%				\subseteq \alpha \subseteq
%				\bigcap^{k}_{j = 1} \lceil \gamma_{j} \rceil_{{{x}_{j}}} \big\};
%			\]
%			\item[for $\mathit{L} = \mathbf{N}$:]
%%			\nb{M: changed, to check}
%				\[
%				\Nmc_{i}(n) = \big\{ \alpha \mid \textnormal{for some} \ \Box_{i}\gamma \in \Gamma^{x}_{n}  \colon \lfloor \gamma \rfloor_{x} \subseteq \alpha \subseteq \lceil \gamma \rceil_{x} \big\} \cup \Wmc;
%			\]
%		\end{enumerate}	
%		\begin{enumerate}[leftmargin=*, align=left]
%%		{{
%			\item[for $\mathit{L} = \mathbf{E}$:]
%			 \ \ $\Nmc_{i}(n) = \big\{ \alpha \mid \textnormal{for some} \ \Box_{i}\gamma \in \Gamma^{x}_{n}  \colon \lfloor \gamma \rfloor_{x} \subseteq \alpha \subseteq \lceil \gamma \rceil_{x} \big\}$;
%			\item[for $\mathit{L} = \mathbf{M}$:]
%			\ \ $\Nmc_{i}(n) = \big\{ \alpha \mid \textnormal{for some} \  \Box_{i}\gamma \in \Gamma^{x}_{n}  \colon \lfloor \gamma \rfloor_{x} \subseteq \alpha \big\}$;
%			\item[for $\mathit{L} = \mathbf{C}$:]
%			\[
%				\Nmc_{i}(n) = \big\{ \alpha \mid \textnormal{for some} \ \Box_{i}\gamma_{1}  \in \Gamma^{{x}_{1}}_{n}, \ldots, \Box_{i}\gamma_{k} \in \Gamma^{{x}_{k}}_{n} \colon
%				\bigcap^{k}_{j = 1} \lfloor \gamma_{j} \rfloor_{{{x}_{j}}}
%				\subseteq \alpha \subseteq
%				\bigcap^{k}_{j = 1} \lceil \gamma_{j} \rceil_{{{x}_{j}}} \big\};
%			\]
%			\item[for $\mathit{L} = \mathbf{N}$:]
%%			\nb{M: changed, to check}
%				\[
%				\Nmc_{i}(n) = \big\{ \alpha \mid \textnormal{for some} \ \Box_{i}\gamma \in \Gamma^{x}_{n}  \colon \lfloor \gamma \rfloor_{x} \subseteq \alpha \subseteq \lceil \gamma \rceil_{x} \big\} \cup \Wmc;
%			\]
%		\end{enumerate}

{{
		\begin{align*}
			\Nmc_{i}(n) =
			\big\{ \alpha \mid & \textnormal{\ for some \ }
%			1 \leq h \leq \mathsf{k} \textnormal{\ such that \ } \\
%			& \ \Box_{i}\gamma_{h}  \in \Gamma^{{x}_{h}}_{n}
%			\colon
%			\mathsf{LB}
%				\subseteq \alpha \subseteq
%				\mathsf{UB}
%				\big\}				
			\Box_{i}\gamma_{1}  \in \Gamma^{{x}_{1}}_{n}, \ldots, \Box_{i}\gamma_{\mathsf{k}} \in \Gamma^{{x}_{\mathsf{k}}}_{n}
			\colon \\
			& \mathsf{LB}(\overline{\gamma_{\mathsf{k}}})
				\subseteq \alpha \subseteq
				\mathsf{UB}(\overline{\gamma_{\mathsf{k}}})
				\big\}
				\cup \mathsf{S};
		\end{align*}
	where:
		\begin{itemize}
			\item $\mathsf{LB}(\overline{\gamma_{\mathsf{k}}}) = \bigcap^{\mathsf{k}}_{j = 1} \lfloor \gamma_{j} \rfloor_{{{x}_{j}}}$;
			\item
			$\mathsf{UB}(\overline{\gamma_{\mathsf{k}}}) =
			\begin{cases}
				\Wmc, & \text{if $\mathbf{M} \in L$} \\
				\bigcap^{\mathsf{k}}_{j = 1} \lceil \gamma_{j} \rceil_{{{x}_{j}}}, & \text{if $\mathbf{M} \not \in L$}
			\end{cases};
			$
%			\item $\mathsf{UB} = \Wmc$ if $\mathbf{M} \in L$, 
%			$\mathsf{UB} = \bigcap^{\mathsf{k}}_{j = 1} \lceil \gamma_{j} \rceil_{{{x}_{j}}}$ if $\mathbf{M} \not \in L$;
			\item
			$\mathsf{k}
			\begin{cases}
				\geq 1, & \text{if $\mathbf{C} \in L$} \\
				= 1, & \text{if $\mathbf{C} \not \in L$}
			\end{cases};
			$
			\item
			$\mathsf{S} =
			\begin{cases}
				\{ \Wmc \}, & \text{if $\mathbf{N} \in L$} \\
				\emptyset, & \text{if $\mathbf{N} \not \in L$}
			\end{cases};
			$
		\end{itemize}

}}
%%% OLD - NEIGHBOURHOOD FUNCTION DEFINITION
%\begin{itemize}
%	\item for $\mathit{L} = \mathbf{E}$:
%		\begin{align*}
%			\Nmc_{i}(n) = \big\{ \alpha \mid & \textnormal{\ for some} \ \Box_{i}\gamma \in \Gamma^{x}_{n} \colon \\
%			& \lfloor \gamma \rfloor_{x} \subseteq \alpha \subseteq \lceil \gamma \rceil_{x} \big\};
%		\end{align*}
%	\item for $\mathit{L} = \mathbf{M}$:
%		\[
%			\Nmc_{i}(n) = \big\{ \alpha \mid \textnormal{for some} \  \Box_{i}\gamma \in \Gamma^{x}_{n}  \colon \lfloor \gamma \rfloor_{x} \subseteq \alpha \big\};
%		\]
%	\item for $\mathit{L} = \mathbf{C}$:
%%		\[
%		\begin{align*}
%			\Nmc_{i}(n) =
%			\big\{ \alpha \mid & \textnormal{\ for some \ }\Box_{i}\gamma_{1}  \in \Gamma^{{x}_{1}}_{n}, \ldots, \Box_{i}\gamma_{k} \in \Gamma^{{x}_{k}}_{n} \colon \\
%%\phantom{\Nmc_{i}(n) = \{  \alpha \mid \, }
%			& \bigcap^{k}_{j = 1} \lfloor \gamma_{j} \rfloor_{{{x}_{j}}}
%				\subseteq \alpha \subseteq
%				\bigcap^{k}_{j = 1} \lceil \gamma_{j} \rceil_{{{x}_{j}}} \big\};
%		\end{align*}
%%		\]
%	\item for $\mathit{L} = \mathbf{N}$:
%		\begin{align*}
%			\Nmc_{i}(n) = \big\{ \alpha \mid & \textnormal{\ for some\ } \ \Box_{i}\gamma \in \Gamma^{x}_{n} \colon \\
%			& \lfloor \gamma \rfloor_{x} \subseteq \alpha \subseteq \lceil \gamma \rceil_{x} \big\} \cup \Wmc;
%		\end{align*}
%\end{itemize}

	\item
	{{
	$\Delta_{n} = \{ x \mid x \ \text{is a term occurring in} \ S_{n} \}$;
	}}
%	$\Delta_{n} = \{ x \in \mathsf{N_{V}} \mid x \ \text{occurs in} \ S_{n} \}$;
	\item $A^{\Imc_{n}} = \{ x \in \Delta_{n} \mid n : A(x) \in S_{n} \}$;
	\item
	{{
	$a^{\Imc_{n}} =
	\begin{cases}
		a, & \text{if $a$ occurs in $S_{n}$} \\
		\text{arbitrary}, & \text{otherwise}
	\end{cases}$
	;
%	\todo{M: todo discuss}
	}}
	\item $r^{\Imc_{n}} = \{ (x, y) \in \Delta_{n} \times \Delta_{n} \mid n : r(x, y) \in S_{n} \ \text{or} \ n : r(z, y) \in S_{n}, $
	for some $z$ blocking $x$ in $S_{n}\}$.
\end{itemize}

%First, we observe the following.
We require the following claims.

\begin{claim}
\label{cla:modelcond}
  For $\mathbf{X}\in\{\mathbf{M,C,N,T,P,Q,D}\}$,
  if $\mathbf{X}\in\Lvar$,
  then $\Mmc$ satisfies the $X$-condition.
\end{claim}
%\begin{proof}[Proof of Claim]
%
%\begin{itemize}
%\item For $\mathit{L}  = \mathbf{M}$, we have that $\Mmc = (\Fmc, \Int)$ is such that $\Fmc = (\Wmc,  \{\Nmc_i \}_{i \in I})$ is supplemented. Indeed,
%for all $n \in \Wmc$, $\alpha,\beta\subseteq \Wmc$, suppose that $\alpha\in \Nmc_{i}(n)$ and $\alpha \subseteq \beta$. By definition, this implies that: for some $\Box_{i} \gamma \in \Gamma^{x}_{n} $, $\lfloor \gamma \rfloor_{x} \subseteq \alpha \subseteq \beta$. Hence, $\beta \in \Nmc_{i}(n)$.
%	\item For $\mathit{L}  = \mathbf{C}$, we have that $\Mmc = (\Fmc, \Int)$ is such that $\Fmc = (\Wmc,  \{\Nmc_i \}_{i \in I})$ is closed under intersection. Indeed, for all $n \in \Wmc$, $\alpha,\beta\subseteq \Wmc$, suppose that $\alpha\in \Nmc_{i}(n)$ and $\beta\in \Nmc_{i}(n)$.
%	Now suppose that, for some
%				$\Box_{i}\gamma_{1}  \in \Gamma^{{x}_{1}}_{n}, \ldots, \Box_{i}\gamma_{k} \in \Gamma^{{x}_{k}}_{n} \colon
%				\bigcap^{k}_{j = 1} \lfloor \gamma_{j} \rfloor_{{x}_{j}}
%				\subseteq \alpha \subseteq
%				\bigcap^{k}_{j = 1} \lceil \gamma_{j} \rceil_{{x}_{j}}$
%	and, for some
%				$\Box_{i}\delta_{1} \in \Gamma^{y_{1}}_{n}, \ldots, \Box_{i}\delta_{h} \in \Gamma^{y_{h}}_{n}  \colon
%				 \bigcap^{h}_{j = 1} \lfloor \delta_{j} \rfloor_{y_{j}}
%				  \subseteq \beta \subseteq
%				  \bigcap^{h}_{j = 1} \lceil \delta_{j} \rceil_{y_{j}}$.
%%				 This implies that, for some $\Box_{i}\gamma_{1}  \in \Gamma^{{x}_{1}}_{n}, \ldots, \Box_{i}\gamma_{k} \in \Gamma^{{x}_{k}}_{n}$ and some $\Box_{i}\delta_{1} \in \Gamma^{y_{1}}_{n}, \ldots, \Box_{i}\delta_{h} \in \Gamma^{y_{h}}_{n} $, we have
%%				 \[
%%				 \bigcap^{k}_{j = 1} \lfloor \gamma_{j} \rfloor_{{x}_{j}} \cap \bigcap^{h}_{j = 1} \lfloor \delta_{j} \rfloor_{y_{j}}
%%				 \subseteq
%%				 \alpha \cap \beta
%%				 \subseteq
%%				 \bigcap^{k}_{j = 1} \lceil \gamma_{j} \rceil_{{x}_{j}} \cap \bigcap^{h}_{j = 1} \lceil \delta_{j} \rceil_{y_{j}}
%%				 \]
%%	This implies that $\alpha\cap\beta\in \Nmc_{i}(n)$.
%				 Then for some $\Box_{i}\gamma_{1}  \in \Gamma^{{x}_{1}}_{n}, \ldots, \Box_{i}\gamma_{k} \in \Gamma^{{x}_{k}}_{n}$ and some $\Box_{i}\delta_{1} \in \Gamma^{y_{1}}_{n}, \ldots, \Box_{i}\delta_{h} \in \Gamma^{y_{h}}_{n} $ the following holds, which in turn implies that
%				 $\alpha\cap\beta\in \Nmc_{i}(n)$:
%%				 \[
%%				 \bigcap^{k}_{j = 1} \lfloor \gamma_{j} \rfloor_{{x}_{j}} \cap \bigcap^{h}_{j = 1} \lfloor \delta_{j} \rfloor_{y_{j}}
%%				 \subseteq
%%				 \alpha \cap \beta
%%				 \subseteq
%%				 \bigcap^{k}_{j = 1} \lceil \gamma_{j} \rceil_{{x}_{j}} \cap \bigcap^{h}_{j = 1} \lceil \delta_{j} \rceil_{y_{j}}
%%				 \]				
%				 \begin{center}$\bigcap^{k}_{j = 1} \lfloor \gamma_{j} \rfloor_{{x}_{j}} \cap \bigcap^{h}_{j = 1} \lfloor \delta_{j} \rfloor_{y_{j}}
%				 \subseteq
%				 \alpha \cap \beta
%				 \subseteq
%				 \bigcap^{k}_{j = 1} \lceil \gamma_{j} \rceil_{{x}_{j}} \cap \bigcap^{h}_{j = 1} \lceil \delta_{j} \rceil_{y_{j}}$\end{center}
%				 
%	\item For $\mathit{L}  = \mathbf{N}$, we have that $\Mmc = (\Fmc, \Int)$, with $\Fmc = (\Wmc,  \{\Nmc_i \}_{i \in I})$, is such that $\Fmc$ contains the unit. Indeed, by construction, for all $n \in \Wmc$, $\Wmc \in \Nmc_{i}(n)$.
%\end{itemize}
%\end{proof}
\begin{proof}[Proof of Claim]

\quad

\begin{enumerate}[leftmargin=*, align=left]
	\item[$\mathbf{M}\in\Lvar$.] Suppose that $\alpha\in \Nmc_{i}(n)$ and $\alpha \subseteq \beta \subseteq \Wmc$. 
By definition, there are
$\Box_{i}\gamma_{1}  \in \Gamma^{{x}_{1}}_{n}, \ldots, \Box_{i}\gamma_{k} \in \Gamma^{{x}_{k}}_{n}$ 
%\todo{T:specify where}
such that $\mathsf{LB}(\overline{\gamma_{k}}) \subseteq \alpha$. 
Then $\mathsf{LB}(\overline{\gamma_{k}}) \subseteq \beta$, 
hence $\beta \in \Nmc_{i}(n)$.

	\item[$\mathbf{C}\in\Lvar$.] Suppose that $\alpha,\beta\in \Nmc_{i}(n)$.
Then there are
$\Box_{i}\gamma_{1}  \in \Gamma^{{x}_{1}}_{n}, \ldots, \Box_{i}\gamma_{k} \in \Gamma^{{x}_{k}}_{n}$ 
such that $\mathsf{LB}(\overline{\gamma_{k}}) \subseteq \alpha \subseteq \mathsf{UB}(\overline{\gamma_{k}})$,
and there are
$\Box_{i}\delta_{1}  \in \Gamma^{{x}_{1}}_{n}, \ldots, \Box_{i}\delta_{h} \in \Gamma^{{x}_{h}}_{n}$ 
such that $\mathsf{LB}(\overline{\delta_{h}}) \subseteq \beta \subseteq \mathsf{UB}(\overline{\delta_{h}})$.
Then 
$\mathsf{LB}(\overline{\gamma_{k}}) \cap \mathsf{LB}(\overline{\delta_{h}}) = \mathsf{LB}(\overline{\gamma_{k},\delta_h}) \subseteq \alpha\cap\beta
\subseteq \mathsf{UB}(\overline{\gamma_{k}}) \cap \mathsf{UB}(\overline{\delta_{h}}) = \mathsf{UB}(\overline{\gamma_{k},\delta_h})$,
which implies $\alpha\cap\beta\in \Nmc_{i}(n)$.
%				 \begin{center}$\bigcap^{k}_{j = 1} \lfloor \gamma_{j} \rfloor_{{x}_{j}} \cap \bigcap^{h}_{j = 1} \lfloor \delta_{j} \rfloor_{y_{j}}
%				 \subseteq
%				 \alpha \cap \beta
%				 \subseteq
%				 \bigcap^{k}_{j = 1} \lceil \gamma_{j} \rceil_{{x}_{j}} \cap \bigcap^{h}_{j = 1} \lceil \delta_{j} \rceil_{y_{j}}$\end{center}
				 
	\item[$\mathbf{N}\in\Lvar$.] By construction, $\Wmc \in \Nmc_{i}(n)$ for all $n \in \Wmc$.
	
	\item[$\mathbf{P}\in\Lvar$.] Suppose that $\alpha\in\Nmc_{i}(n)$. 
Then there are
$\Box_{i}\gamma_{1}  \in \Gamma^{{x}_{1}}_{n}, \ldots, \Box_{i}\gamma_{k} \in \Gamma^{{x}_{k}}_{n}$ 
such that $\mathsf{LB}(\overline{\gamma_{k}}) \subseteq \alpha \subseteq \mathsf{UB}(\overline{\gamma_{k}})$.
Since $\T$ is $\LnALC$-complete,
by the rule $\mathsf{R}_{\mathit{L}\mathbf{P}}$,
there is 
%$m\in\mathsf{L}_{\T}$ 
$m$
such that 
%$m\in \lfloor \gamma_{j} \rfloor_{{{x}_{j}}}$
$m: \gamma_{j} \in \T$ for all $1 \leq j \leq k$, that is $m\in\mathsf{LB}(\overline{\gamma_{k}})$.
Then $\alpha\not=\emptyset$.

	\item[$\mathbf{Q}\in\Lvar$.] Suppose that $\alpha\in\Nmc_{i}(n)$. 
Then there are
$\Box_{i}\gamma_{1}  \in \Gamma^{{x}_{1}}_{n}, \ldots, \Box_{i}\gamma_{k} \in \Gamma^{{x}_{k}}_{n}$ 
such that $\mathsf{LB}(\overline{\gamma_{k}}) \subseteq \alpha \subseteq \mathsf{UB}(\overline{\gamma_{k}})$.
Since $\T$ is $\LnALC$-complete,
by the rule $\mathsf{R}_{\mathit{L}\mathbf{Q}}$,
there is $m$ %$m\in\mathsf{L}_{\T}$ 
such that 
$m: \dot{\lnot}\gamma_{j} \in \T$ for some $1 \leq j \leq k$, that is $m\in\Wmc$ and $m\notin\lceil \gamma_{j} \rceil_{{{x}_{j}}}$,
hence $m\notin\mathsf{UB}(\overline{\gamma_{k}})$.
Then $\alpha\not=\Wmc$.

	\item[$\mathbf{D}\in\Lvar$.] Suppose that $\alpha,\beta\in \Nmc_{i}(n)$.
Then there are
$\Box_{i}\gamma_{1}  \in \Gamma^{{x}_{1}}_{n}, \ldots, \Box_{i}\gamma_{k} \in \Gamma^{{x}_{k}}_{n}$ 
such that $\mathsf{LB}(\overline{\gamma_{k}}) \subseteq \alpha \subseteq \mathsf{UB}(\overline{\gamma_{k}})$,
and there are
$\Box_{i}\delta_{1}  \in \Gamma^{{x}_{1}}_{n}, \ldots, \Box_{i}\delta_{h} \in \Gamma^{{x}_{h}}_{n}$ 
such that $\mathsf{LB}(\overline{\delta_{h}}) \subseteq \beta \subseteq \mathsf{UB}(\overline{\delta_{h}})$.
Since $\T$ is $\LnALC$-complete,
b the rule $\mathsf{R}_{\mathit{L}\mathbf{D}}$,
there is $m$ %$m\in\mathsf{L}_{\T}$ 
such that 
$m: \gamma_{j}, m: \delta_{\ell} \in \T$
for all $1\leq j\leq k$, $1\leq \ell \leq h$;
or
$m: \dot{\lnot}\gamma_{j}, m: \dot{\lnot}\delta_{\ell} \in \T$
for some $1\leq j\leq k$, $1\leq \ell \leq h$.
In the first case,
$m \in \mathsf{LB}(\overline{\gamma_{k}}) \cap \mathsf{LB}(\overline{\delta_{h}})$,
% = \mathsf{LB}(\overline{\gamma_{k},\delta_h})$; 
%which implies 
hence $m\in\alpha\cap\beta$.
In the second case, 
$m\in(\Wmc\setminus\mathsf{UB}(\overline{\gamma_{k}})) \cap (\Wmc\setminus\mathsf{UB}(\overline{\delta_{h}}))$,
hence 
$m\in(\Wmc\setminus\alpha)\cap(\Wmc\setminus\beta)$.
In either case $\beta\neq\Wmc\setminus\alpha$.

	\item[$\mathbf{T}\in\Lvar$.] Suppose that $\alpha\in\Nmc_{i}(n)$. 
Then there are
$\Box_{i}\gamma_{1}  \in \Gamma^{{x}_{1}}_{n}, \ldots, \Box_{i}\gamma_{k} \in \Gamma^{{x}_{k}}_{n}$ 
such that $\mathsf{LB}(\overline{\gamma_{k}}) \subseteq \alpha \subseteq \mathsf{UB}(\overline{\gamma_{k}})$.
Since $\T$ is $\LnALC$-complete,
by the rule $\mathsf{R}_{\mathit{L}\mathbf{T}}$,
$n: \gamma_{j} \in \T$ for all $1 \leq j \leq k$,
then $n\in \mathsf{LB}(\overline{\gamma_{k}})$,
thus $n\in\alpha$.\qedhere
\end{enumerate}
\end{proof}

%We then require the following claims.
%
\begin{claim}
\label{cla:conind}
For every $n \in \Wmc$, $C \in \conneg(\p)$, and $x \in \Delta_{n}$: if $n : C(x) \in S_{n}$, then $x \in C^{\Imc_{n}}$.
\end{claim}
\begin{proof}[Proof of Claim]
We show the claim by induction on the weight of $C$ (in NNF).
The base case of $C = A$ comes immediately from the definitions.
For the base case of $C = \lnot A$, suppose that $n : \lnot A(x) \in S_{n}$. Since $\T$ is clash-free, we have that $n : A(x) \not \in S_{n}$, and thus $x \not \in A^{\Imc_{n}}$ by definition of $A^{\Imc_{n}}$, meaning $x \in (\lnot A)^{\Imc_{n}}$.
The inductive cases of $C = D \sqcap E$ and $C = D \sqcup E$ come from the fact that $S_{n}$ is closed under $\mathsf{R}_{\sqcap}$ and $\mathsf{R}_{\sqcup}$, respectively, and straightforward applications of the inductive hypothesis.
We show the remaining cases (cf. also~\cite[Claim 15.2]{GabEtAl03}).

%\nb{M: added, to be checked}
\begin{enumerate}[leftmargin=*, align=left]
	\item[$C = \exists r.D$.]
Let $n : \exists r.D(x) \in S_{n}$, meaning that $\exists r.D \in \Gamma^{x}_{n}$. We distinguish two cases.
\begin{itemize}
\item[$(i)$] $x$ is not blocked by any variable in $S_{n}$. Since $S_{n}$ is closed under $\mathsf{R}_{\exists}$, there exists $y$ occurring in $S_{n}$ such that $n : r(x,y) \in S_{n}$ and $n : D(y) \in S_{n}$. Thus, by definition, $(x, y) \in r^{\Imc_{n}}$ and $n : D(y) \in S_{n}$. By inductive hypothesis, we obtain that $x \in (\exists r.D)^{\Imc_{n}}$.

\item[$(ii)$] $x$ is blocked by a variable in $S_{n}$, implying that there exists a $<$-minimal (since $<$ is a well-ordering) $y$ occurring in $S_{n}$ such that $y < x$ and $\{ E \mid n : E(x) \in S_{n} \} \subseteq \{ E \mid n : E(y) \in S_{n} \}$.
In turn, this implies that $y$ is not blocked by any other variable $z$ in $S_{n}$ (for otherwise $z$ would block $x$, with $z < y$, against the fact that $y$ is $<$-minimal).
By reasoning as in the case above, since $y$ is not blocked and $S_{n}$ is closed under $\mathsf{R}_{\exists}$, we have a variable $z$ occurring in $S_{n}$ such that $n : r(y,z) \in S_{n}$ and $n : D(z) \in S_{n}$.
Since $y$ blocks $x$, by definition we have that $(x, z) \in r^{\Imc_{n}}$, and by inductive hypothesis we get from $n : D(z)$ that $z \in D^{\Imc_{n}}$.
Thus, $x \in (\exists r.D)^{\Imc_{n}}$.
\end{itemize}

\item[$C = \forall r.D$.]
Let $n : \forall r.D(x) \in S_{n}$, meaning that $\forall r.D \in \Gamma^{x}_{n}$, and suppose that $(x, y) \in r^{\Imc_{n}}$. By definition, either $n : r(x,y) \in S_{n}$ or $n : r(z,y) \in S_{n}$, for some $z$ blocking $x$ in $S_{n}$.
In the former case, since $S_{n}$ is closed under $\mathsf{R}_{\forall}$, we get that $n : D(y) \in S_{n}$.
In the latter case, since $z$ blocks $x$ in $S_{n}$, we obtain $n : \forall r.D(z) \in S_{n}$; again, since $S_{n}$ is closed under $\mathsf{R}_{\forall}$, this implies that $n : D(y) \in S_{n}$.
Hence, in both cases, we have $n : D(y) \in S_{n}$.
By inductive hypothesis, this means that $y \in D^{\Imc_{n}}$.
Since $y$ was arbitrary, we conclude that $x \in (\forall r.D)^{\Imc_{n}}$.

%The inductive cases of $C = \exists r.D$ and $C = \forall r.D$ can be proved analogously to~\cite[Claim 15.2]{GabEtAl03}.\nb{M: todo add?}
%We show the %remaining
%modal cases.

%$C = \exists r.D$. \ldots\nb{M: todo add}
%
%$C = \forall r.D$. \ldots\nb{M: todo add}

\item[$C = \Box_{i} D$.]
Let $n : \Box_{i} D(x) \in S_{n}$, meaning that $\Box_{i} D \in \Gamma^{x}_{n}$.
		We have by inductive hypothesis that
	$\lfloor D \rfloor_{x} = \{ n \in \Wmc \mid n : D(x) \in S_{n} \} \subseteq \{ n \in \Wmc \mid x \in D^{\Imc_{n}} \} = \llbracket D \rrbracket^{\Mmc}_{x}$.
	By inductive hypothesis %(since $| D | = | \dnot D |$), 
	(since $| \dnot D | = | D |$), 
	we also have that
	$\{ n \in \Wmc \mid n : \dnot D(x) \in S_{n} \} \subseteq \{ n \in \Wmc \mid x \in (\dnot D)^{\Imc_{n}} \} = \llbracket \dnot D \rrbracket^{\Mmc}_{x} = \Wmc \setminus \llbracket D \rrbracket^{\Mmc}_{x}$.
	Hence, $\llbracket D \rrbracket^{\Mmc}_{x} \subseteq \Wmc \setminus \{ w \in \Wmc \mid n : \dnot D(x) \in S_{n} \} = \lceil D \rceil_{x}$. In conclusion, we have $\Box_{i} D \in \Gamma^{x}_{n} $ such that $\lfloor D \rfloor_{x} \subseteq \llbracket D \rrbracket^{\Mmc}_{x} \subseteq \lceil D \rceil_{x}$. Thus, by definition, $\llbracket D \rrbracket^{\Mmc}_{x} \in \Nmc_{i}(n)$, as required.
(If $\mathbf{M}\in\Lvar$, $\lfloor D \rfloor_{x} \subseteq \llbracket D \rrbracket^{\Mmc}_{x}$, and by definition this means $\llbracket D \rrbracket^{\Mmc}_{x} \in \Nmc_{i}(n)$, as required.)

	\item[$C = \Diamond_{i} D$.]
Let $n : \Diamond_{i}D(x) \in S_{n}$. 
We distinguish two cases.

\begin{itemize}
\item[$(i)$] There exists no $\Box_{i} \gamma \in \Gamma^{y}_{n}$.
Then if $\mathbf{N}\notin\Lvar$, $\Nmc_{i}(n) = \emptyset$, thus $\Wmc \setminus \llbracket D \rrbracket^{\Mmc}_{x} \not \in \Nmc_{i}(n)$, meaning that $x \in (\Diamond_{i}D)^{\Imc_{n}}$.
If instead $\mathbf{N}\in\Lvar$,
then $\Nmc_{i}(n) = \Wmc$.
Moreover, since $\T$ is $\LnALC$-complete, 
by the rule $\mathsf{R}_{\mathit{L}\mathbf{N}}$,
there is 
$m$
such that 
$m : D(x) \in S_{m}$. By inductive hypothesis, this implies $x \in D^{\Imc_{m}}$, that is, $\llbracket D \rrbracket^{\Mmc}_{x} \neq \emptyset$. Then we have $\Wmc \setminus \llbracket D \rrbracket^{\Mmc}_{x} \neq \Wmc$, and thus $\Wmc \setminus \llbracket D \rrbracket^{\Mmc}_{x} \not \in \Nmc_{i}(n)$. Hence, $x \in (\Diamond_{i}D)^{\Imc_{n}}$.


\item[$(ii)$] There exist $\Box_{i} \gamma_{1} \in \Gamma^{y_{1}}_{n}, \ldots, \Box_{i} \gamma_{k} \in \Gamma^{y_{k}}_{n}$.
		Since $\T$ is $\LnALC$-complete, there exists $m \in \Wmc$ such that:
$\gamma_{1}  \in \Gamma^{y_{1}}_{m}, \ldots, \gamma_{k} \in \Gamma^{y_{k}}_{m}$ and $D \in \Gamma^{x}_{m}$; or
$\dnot \gamma_{j} \in \Gamma^{y_{j}}_{m}$ and $\dnot D \in \Gamma^{x}_{m}$, for some $j\leq k$.
				By inductive hypothesis, the previous step implies that there exists $m \in \Wmc$ such that:
$\gamma_{1} \in \Gamma^{y_{1}}_{m}, \ldots, \gamma_{k} \in \Gamma^{y_{k}}_{m}$ and $x \in D^{\Imc_{m}}$; or
$\dnot \gamma_{j} \in \Gamma^{y_{j}}_{m}$ and $x \in \dnot D^{\Imc_{m}}$, for some $j\leq k$.
Thus
$\bigcap_{j = 1}^{k} \lfloor \gamma_{j} \rfloor_{y_{j}} \not\subseteq \Wmc \setminus \llbracket D \rrbracket^{\Mmc}_{x}$; or
$\Wmc \setminus \llbracket D \rrbracket^{\Mmc}_{x} \not\subseteq \bigcap_{j = 1}^{k}\lceil \gamma_{l} \rceil_{y_{l}}$.
Since this holds for every $\Box_{i} \gamma_{1} \in \Gamma^{y_{1}}_{n}, \ldots, \Box_{i} \gamma_{k} \in \Gamma^{y_{k}}_{n}$, we conclude that $\Wmc \setminus  \llbracket D \rrbracket^{\Mmc}_{x} \not \in \Nmc_{i}(n)$, i.e., $x \in (\Diamond_{i}D)^{\Imc_{n}}$, as required.
(If $\mathbf{M}\in\Lvar$, 
there exists $m \in \Wmc$ such that
$\gamma_{1}  \in \Gamma^{y_{1}}_{m}, \ldots, \gamma_{k} \in \Gamma^{y_{k}}_{m}$ and $D \in \Gamma^{x}_{m}$,
thus
$x \in D^{\Imc_{m}}$,
hence
$\bigcap_{j = 1}^{k} \lfloor \gamma_{j} \rfloor_{y_{j}} \not\subseteq \Wmc \setminus \llbracket D \rrbracket^{\Mmc}_{x}$,
therefore $\Wmc \setminus  \llbracket D \rrbracket^{\Mmc}_{x} \not \in \Nmc_{i}(n)$.)\qedhere
\end{itemize}
\end{enumerate}
\end{proof}



\begin{claim}
\label{cla:forind}
For every $n \in \Wmc$ and $\psi \in \conneg(\p)$: if $n : \psi \in S_{n}$, then $\Mmc, n \models \psi$.
\end{claim}
\begin{proof}[Proof of Claim]
We prove the claim by induction on the weight of $\p$ (in NNF).

%\todo{M: to discuss. T: What?}

\begin{enumerate}[leftmargin=*, align=left]
	\item[$\psi = C(a)$.] 
	{{Let $n : C(a) \in S_{n}$. By definition of $\Imc_{n}$ and Claim~\ref{cla:conind}, we have that $a^{\Imc_{n}} \in C^{\Imc_{n}}$, hence $\Mmc, n \models C(a)$. (For $\psi = \lnot C(a)$, recall that $\lnot C(a)$ is equivalent to $D(a)$ with $D = \lnot C$).}}



	\item[$\psi = r(a,b)$.]
{{Let $n: r(a,b) \in S_{n}$. By definition of $\Imc_{n}$, this implies $(a^{\Imc_{n}}, b^{\Imc_{n}}) \in r^{\Imc_{n}}$, hence $\Mmc, n \models r(a,b)$.}}

	\item[$\psi = \lnot r(a,b)$.]
{{ Let $n: \lnot r(a,b) \in S_{n}$. Since $\mathbf{T}$ is clash-free, we have that $n: r(a,b) \not \in S_{n}$. Thus, by definition
of $\Imc_{n}$
%$r^{\Imc_{n}}$,
we have $(a^{\Imc_{n}}, b^{\Imc_{n}}) \not \in r^{\Imc_{n}}$, meaning that $\Mmc, n \not \models r(a,b)$.
}}

	\item[$\psi = (\top \sqsubseteq C)$.] Let $n : \top \sqsubseteq C \in S_{n}$ and let $x \in \Delta_{n}$. Since $S_{n}$ is closed under $(\mathsf{R}_{\sqsubseteq})$ and $x$ occurs in $S_{n}$, we have that $n : C(x) \in S_{n}$. By Claim~\ref{cla:conind}, we have that $x \in C^{\Imc_{n}}$. Given that $x$ is arbitrary, we conclude that $\Mmc, n \models \top \sqsubseteq C$.

	\item[$\psi = \lnot (\top \sqsubseteq C)$.] Let $n : \lnot (\top \sqsubseteq C) \in S_{n}$. Since $S_{n}$ is closed under $(\mathsf{R}_{\not\sqsubseteq})$, there exists $x$ occurring in $S_{n}$ such that $n : \dnot C(x) \in S_{n}$. By Claim~\ref{cla:conind}, we obtain that $x \in (\dnot C)^{\Imc_{n}}$, for some $x \in \Delta_{w}$. Hence, $\Mmc, n \models \lnot (\top \sqsubseteq C)$.
\end{enumerate}

The inductive cases of 
$\psi = \chi \land \vartheta$
and
$\psi = \chi \lor \vartheta$ follow from the definitions and straighforward applications of the inductive hypothesis.
%
%The inductive cases of 
Moreover the inductive cases of 
$\psi = \Box_{i} \chi$
and 
$\psi = \Diamond_{i} \chi$ can be proved analogously to Claim~\ref{cla:conind}.
\end{proof}
%

Since,
%by~$(\mathbf{P0})$,
by definition,
we have 
$0 : \p \in S_{0} \subseteq \mathbf{T}$,
%there exists $w_{\p} \in \Wmc$ such that $\p \in \qs(w_{\p})$,
thanks to Claim~\ref{cla:forind} we obtain $\Mmc, 0 \models \p$.
{{Moreover,
by Claim~\ref{cla:modelcond}, $\Mmc$ is a
$\mathit{L}^n$ model.
Therefore $\p$ is $\LnALC$ satisfiable.}}
%for some $w_{\p} \in \Wmc$.
\end{proof}



















%%% COMPLETENESS

We finally show completeness of the $\LnALC$ tableau algorithm.

\begin{restatable}[Completeness]{theorem}{Completeness}
	\label{thm:completeness}
	If $\p$ is $\LnALC$ satisfiable, then there exists an execution of the $\LnALC$ tableau algorithm for $\p$ that constructs a complete and clash-free completion set.
%			If $\p$ is $\LnALC$ satisfiable, then
%			%, having started on the initial completion set $\T_{\p}$, 
%			the $\LnALC$ tableau algorithm
%			with $\p$ as input returns $\mathsf{satisfiable}$.
%			%}}
%	% constructs an $\LnALC$-complete and clash-free completion set for $\p$.
\end{restatable}
%






%\Completeness*
\begin{proof}
%We assume that $\mathbf{C}\in\Lvar$ and $\mathbf{M}\notin\Lvar$,
%the proof for the cases where $\mathbf{C}\notin\Lvar$ or $\mathbf{M}\in\Lvar$ can be obtained as a simplification
%of the present one.
In the proof we assume $\mathbf{C}\in\Lvar$,
for the case $\mathbf{C}\notin\Lvar$ consider $k = h = 1$.
Let $\Mmc = (\Fmc, \Imc)$ be an $\LnALC$-model satisfying $\p$, with $\Fmc = (\Wmc, \{ \Nmc \}_{i \in J})$, i.e.,
$\Mmc, w_{\p} \models \p$, for some $w_{\p} \in \Wmc$.
%
We require the following definitions and technical results.
%
%For every $d \in \Delta_{w}$, define $\tp^{\Imc_{w}}(d) = \{ C \in \conneg(\p) \mid d \in C^{\Imc_{w}} \}$,
%and let $T_{w} = \{ \tp^{\Imc_{w}}(d) \mid d \in \Delta_{w} \}$.
%Moreover, for every $t = \tp^{\Imc_{w}}(d)$, select a variable $x_{t} \in \NV$.
First, we let $\gamma, \delta$ (possibly indexed) range over $\MLnALC$ concepts and formulas, with $\llbracket \gamma \rrbracket^{\Mmc}_{d} = \llbracket \psi \rrbracket^{\Mmc}$, if $\gamma = \psi$, and $\llbracket \gamma \rrbracket^{\Mmc}_{d} = \llbracket C \rrbracket^{\Mmc}_{d}$, if $\gamma = C$.
%
Then, for $w \in \Wmc$ and $d \in \bigcup_{v \in \Wmc} \Delta_{v}$, define
$\Phi^{d}_{w} = \{ \psi \in \forneg(\p) \mid \Mmc, w \models \psi \} \cup \{ C \in \conneg(\p) \mid d \in C^{\Imc_{w}} \}$.
Observe that, if $C \in \Phi^{d}_{w}$, then $d \in \Delta_{w}$.
%%
%We now show that the following holds.
%\begin{claim}
%\label{cla:truth}
%For every $w \in \Wmc$ and every $d_{1}, \ldots, d_{k}, e \in \bigcup_{v \in \Wmc} \Delta_{v}$:
%				if $\Box_{i}\gamma_{1} \in \Phi^{d_{1}}_{w}, \ldots, \Box_{i} \gamma_{k} \in \Phi^{d_{k}}_{w}$ and $\Diamond_{i} \delta \in \Phi^{e}_{w}$, then there exists $v \in \Wmc$ such that:
%				\begin{enumerate}[label=$(\arabic*)$, start=0]
%					\item $\gamma_{1} \in \Phi^{d_{1}}_{v}, \ldots, \gamma_{k} \in \Phi^{d_{k}}_{v}$ and $\delta \in \Phi^{e}_{v}$; or
%					\item $\dnot \gamma_{1} \in \Phi^{d_{1}}_{v}$ and $\dnot \delta \in \Phi^{e}_{v}$; or
%					\item[] $\vdots$
%					\item[$(l)$] $\dnot \gamma_{l} \in \Phi^{d_{k}}_{v}$ and $\dnot \delta \in \Phi^{e}_{v}$;
%				\end{enumerate}
%where:
%for $\mathit{L} = \mathbf{E}$, $k = l = 1$;
%for $\mathit{L} = \mathbf{M}$, $k = 1$ and $l = 0$;
%for $\mathit{L} = \mathbf{C}$, $k \geq 1$ and $l = k$;
%for $\mathit{L} = \mathbf{N}$, $k = l = 1$ or $k = l = 0$.
%\end{claim}
%\begin{proof}
%
%We consider each $\mathit{L} \in \Log$.
%
%\begin{enumerate}[leftmargin=*, align=left]
%	\item[$\mathit{L} = \mathbf{E}$.] 
%		Assume $\Box_{i} \gamma \in \Phi^{d}_{w}, \Diamond_{i} \delta \in \Phi^{e}_{w}$, meaning that $\llbracket \gamma \rrbracket^{\Mmc}_{d} \in \Nmc_{i}(w)$ and $\Wmc \setminus \llbracket \delta \rrbracket^{\Mmc}_{e} \not \in \Nmc_{i}(w)$, i.e., $\llbracket \dnot \delta \rrbracket^{\Mmc}_{e} \not \in \Nmc_{i}(w)$. Towards a contradiction, suppose that, for every $v \in \Wmc$, the following holds:
%%		\begin{center}
%			($\gamma \not \in \Phi^{d}_{v}$ or $\delta \not \in \Phi^{e}_{v}$) and
%			($\dnot \gamma \not \in \Phi^{d}_{v}$ or $\dnot \delta \not \in \Phi^{e}_{v}$).
%%		\end{center}
%	Equivalently, for every $v \in \Wmc$:
%%		\begin{center}
%			($\gamma\in \Phi^{d}_{v}$ implies $\delta \not \in \Phi^{e}_{v}$) and
%			($\dnot \delta \in \Phi^{e}_{v}$ implies $\dnot \gamma \not \in \Phi^{d}_{v}$).
%%			($\dnot \gamma \in \Phi^{d}_{v}$ implies $\dnot \delta \not \in \Phi^{e}_{v}$).
%%		\end{center}
%	By definition, we have that $\gamma \in \Phi^{d}_{v}$ iff $\dnot \gamma \not \in \Phi^{d}_{v}$ and $\delta \not \in \Phi^{e}_{v}$ iff $\dnot \delta \in \Phi^{e}_{v}$. Thus, the previous step means:
%%		\begin{center}
%			($ \llbracket \gamma \rrbracket^{\Mmc}_{d} \subseteq \llbracket \dnot \delta \rrbracket^{\Mmc}_{e}$) and
%			($  \llbracket \dnot \delta \rrbracket^{\Mmc}_{e} \subseteq \llbracket \gamma \rrbracket^{\Mmc}_{d}$),
%			i.e.,
%%			($\llbracket \dnot \gamma \rrbracket^{\Mmc}_{d} \subseteq \llbracket \delta \rrbracket^{\Mmc}_{e}$).
%%		From this
%%%		since $W \setminus [ \dnot \psi ]^{\Mmc} = \llbracket \psi \rrbracket^{\Mmc}$,
%%		we have equivalently that
%			$\llbracket \gamma \rrbracket^{\Mmc}_{d} = \llbracket \dnot \delta \rrbracket^{\Mmc}_{e}$,
%		contradicting the assumption that $\llbracket \gamma \rrbracket^{\Mmc}_{d} \in \Nmc_{i}(w)$ and $\llbracket \dnot \delta \rrbracket^{\Mmc}_{e} \not \in \Nmc_{i}(w)$.
%	
%		
%			
%	\item[$\mathit{L} = \mathbf{M}$.] 
%	Assume $\Box_{i} \gamma \in \Phi^{d}_{w}, \Diamond_{i} \delta \in \Phi^{e}_{w}$, meaning that $\llbracket \gamma \rrbracket^{\Mmc}_{d} \in \Nmc_{i}(w)$ and $\Wmc \setminus \llbracket \delta \rrbracket^{\Mmc}_{e} \not \in \Nmc_{i}(w)$, i.e., $\llbracket \dnot \delta \rrbracket^{\Mmc}_{e} \not \in \Nmc_{i}(w)$. Towards a contradiction, suppose that, for every $v \in \Wmc$, the following holds:
%			$\gamma \not \in \Phi^{d}_{v}$ or $\delta \not \in \Phi^{e}_{v}$.
%	Equivalently, for every $v \in \Wmc$:
%			$\gamma\in \Phi^{d}_{v}$ implies $\delta \not \in \Phi^{e}_{v}$.
%%	By definition of $\qs(v)$, we have that $\gamma \in \Phi^{d}_{v}$ iff $\dnot \gamma \not \in \Phi^{d}_{v}$ and $\delta \not \in \Phi^{e}_{v}$ iff $\dnot \delta \in \Phi^{e}_{v}$.
%%	Thus,
%	By definition,
%	the previous step means
%			$ \llbracket \gamma \rrbracket^{\Mmc}_{d} \subseteq \llbracket \dnot \delta \rrbracket^{\Mmc}_{e}$.
%			Since $\Mmc$ is supplemented, we have that $\llbracket \dnot \delta \rrbracket^{\Mmc}_{e} \in \Nmc_{i}(w)$,
%			which is impossible.
%%		contrary to the assumption that $\llbracket \dnot \delta \rrbracket^{\Mmc}_{e} \not \in \Nmc_{i}(w)$.
%
%	
%	\item[$\mathit{L} = \mathbf{C}$.] 
%		Assume $\Box_{i} \gamma_{1} \in \Phi^{d_{1}}_{w}, \ldots, \Box_{i} \gamma_{k} \in \Phi^{d_{k}}_{w}, \Diamond_{i} \delta \in \Phi^{e}_{w}$, meaning that $\llbracket \gamma_{j} \rrbracket^{\Mmc}_{d_{j}} \in \Nmc_{i}(w)$, for $j = 1, \ldots, k$, and $\Wmc \setminus \llbracket \delta \rrbracket^{\Mmc}_{e} \not \in \Nmc_{i}(w)$, i.e., $\llbracket \dnot \delta \rrbracket^{\Mmc}_{e} \not \in \Nmc_{i}(w)$. Towards a contradiction, suppose that, for every $v \in \Wmc$, 
%%it is not the case that the following holds:
%%				\begin{itemize}
%%%				[label=$(\arabic*)$, start=0]
%%					\item $\gamma_{1} \in \Phi^{d_{1}}_{v}, \ldots, \gamma_{k} \in \Phi^{d_{k}}_{v}$ and $\delta \in \Phi^{e}_{v}$; or
%%					\item $\dnot \gamma_{1} \in \Phi^{d_{1}}_{v}$ and $\dnot \delta \in \Phi^{e}_{v}$; or
%%					\item[] $\vdots$
%%					\item $\dnot \gamma_{k} \in \Phi^{d_{k}}_{v}$ and $\dnot \delta \in \Phi^{e}_{v}$.
%%				\end{itemize}
%none of the following holds:
%%it is not the case that the following holds:
%$(0)$ $\gamma_{1} \in \Phi^{d_{1}}_{v}, \ldots, \gamma_{k} \in \Phi^{d_{k}}_{v}$ and $\delta \in \Phi^{e}_{v}$; 
%%or
%$(1)$ $\dnot \gamma_{1} \in \Phi^{d_{1}}_{v}$ and $\dnot \delta \in \Phi^{e}_{v}$; ...;
%%or
%$(k)$ $\dnot \gamma_{k} \in \Phi^{d_{k}}_{v}$ and $\dnot \delta \in \Phi^{e}_{v}$.
%Equivalently, for every $v \in \Wmc$,
%%				\begin{itemize}
%%%				[label=$(\arabic*)$, start=0]
%%					\item $\gamma_{1} \in \Phi^{d_{1}}_{v}, \ldots, \gamma_{k} \in \Phi^{d_{k}}_{v}$ implies $\delta \not \in \Phi^{e}_{v}$; and
%%					\item $\dnot \delta \in \Phi^{e}_{v}$ implies $\dnot \gamma_{1} \not \in \Phi^{d_{1}}_{v}$; and
%%					\item[] $\vdots$
%%					\item $\dnot \delta \in \Phi^{e}_{v}$ implies $\dnot \gamma_{k} \not \in \Phi^{d_{k}}_{v}$.
%%				\end{itemize}
%it holds that
%$(0)$ $\gamma_{1} \in \Phi^{d_{1}}_{v}, \ldots, \gamma_{k} \in \Phi^{d_{k}}_{v}$ implies $\delta \not \in \Phi^{e}_{v}$; and
%$(1)$ $\dnot \delta \in \Phi^{e}_{v}$ implies $\dnot \gamma_{1} \not \in \Phi^{d_{1}}_{v}$; ...
%and 
%$(k)$ $\dnot \delta \in \Phi^{e}_{v}$ implies $\dnot \gamma_{k} \not \in \Phi^{d_{k}}_{v}$.
%	By definition, from the previous step we obtain
%%					\begin{itemize}
%%%				[label=$(\arabic*)$, start=0]
%%					\item $\bigcap_{j = 1}^{k} \llbracket \gamma_{j} \rrbracket^{\Mmc}_{d_{j}} \subseteq \llbracket \dnot \delta \rrbracket^{\Mmc}_{e}$; and
%%					\item $\llbracket \dnot \delta \rrbracket^{\Mmc}_{e} \subseteq \llbracket \gamma_{1} \rrbracket^{\Mmc}_{d_{1}}$; and
%%					\item[] $\vdots$
%%					\item $\llbracket \dnot \delta \rrbracket^{\Mmc}_{e} \subseteq \llbracket \gamma_{k} \rrbracket^{\Mmc}_{d_{k}}$.
%%				\end{itemize}
%$(0)$ $\bigcap_{j = 1}^{k} \llbracket \gamma_{j} \rrbracket^{\Mmc}_{d_{j}} \subseteq \llbracket \dnot \delta \rrbracket^{\Mmc}_{e}$; and
%$(1)$ $\llbracket \dnot \delta \rrbracket^{\Mmc}_{e} \subseteq \llbracket \gamma_{1} \rrbracket^{\Mmc}_{d_{1}}$; ...
%and
%$(k)$ $\llbracket \dnot \delta \rrbracket^{\Mmc}_{e} \subseteq \llbracket \gamma_{k} \rrbracket^{\Mmc}_{d_{k}}$.
%				Hence $\bigcap_{j = 1}^{k} \llbracket \gamma_{j} \rrbracket^{\Mmc}_{d_{j}} =  \llbracket \dnot \delta \rrbracket^{\Mmc}_{e}$.
%			Since $\Mmc$ is closed under intersection, we obtain $\llbracket \dnot \delta \rrbracket^{\Mmc}_{e} \in \Nmc_{i}(w)$,
%			a contradiction.
%	
%	
%	
%\item[$\mathit{L} = \mathbf{N}$.] 
%	We distinguish two cases:
%%	\begin{itemize}
%%		\item Let $k = l = 0$. That is, there exists no $\Box_{i} \gamma \in \Phi^{d}_{w}$, while $\Diamond_{i} \delta \in \Phi^{e}_{w}$, meaning that $\Wmc \setminus \llbracket \delta \rrbracket^{\Mmc}_{e} \not \in \Nmc_{i}(w)$.
%%		Towards a contradiction, suppose that, for every $v \in \Wmc$, $\delta \not \in \Phi^{e}_{v}$.
%%			Since, by definition, we have $\delta \not \in \Phi^{e}_{v}$ iff $\dnot \delta \in \Phi^{e}_{v}$, the previous step means that $\Wmc \subseteq \llbracket \dnot \delta \rrbracket^{\Mmc}_{e}$, and hence $\llbracket \delta \rrbracket^{\Mmc}_{e} = \emptyset$. Thus, $\Wmc \not \in \Nmc_{i}(w)$, contradicting the fact that $\Mmc$ contains the unit.
%%
%%		\item Let $k = l = 1$. Hence, there exists $\Box_{i} \gamma \in \Phi^{e}_{w}$ and  $\Diamond_{i} \delta \in \Phi^{e}_{w}$. We then reason similarly to the case for $\mathit{L} = \mathbf{E}$.
%%	\end{itemize}
%$(i)$ Let $k = l = 0$. That is, there exists no $\Box_{i} \gamma \in \Phi^{d}_{w}$, while $\Diamond_{i} \delta \in \Phi^{e}_{w}$, meaning that $\Wmc \setminus \llbracket \delta \rrbracket^{\Mmc}_{e} \not \in \Nmc_{i}(w)$.
%		Towards a contradiction, suppose that, for every $v \in \Wmc$, $\delta \not \in \Phi^{e}_{v}$.
%			Since, by definition, we have $\delta \not \in \Phi^{e}_{v}$ iff $\dnot \delta \in \Phi^{e}_{v}$, the previous step means that $\Wmc \subseteq \llbracket \dnot \delta \rrbracket^{\Mmc}_{e}$, and hence $\llbracket \delta \rrbracket^{\Mmc}_{e} = \emptyset$. Thus, $\Wmc \not \in \Nmc_{i}(w)$, contradicting the fact that $\Mmc$ contains the unit.
%$(ii)$ Let $k = l = 1$. Hence, there exists $\Box_{i} \gamma \in \Phi^{e}_{w}$ and  $\Diamond_{i} \delta \in \Phi^{e}_{w}$. We then reason similarly to the case for $\mathit{L} = \mathbf{E}$.	\qedhere
%\end{enumerate}
%\end{proof}
%\bigskip
%\todo{T:merge text}
%
Moreover, given a completion set $\T$ for $\p$
and $S_{n} \subseteq \T$,
%let
%$\mathsf{L}_{\T} = \{ n \in \mathsf{N_{L}} \mid S_{n} \subseteq \T \}$.
%Moreover,
let $\Gamma^{x}_{n} = \{ \psi \mid n : \psi \in S_{n} \} \cup \{ C \mid n : C(x) \in S_{n} \}$.
%
We say that a completion set $\T$ for $\p$ is \emph{$\Mmc$-compatible} if
there exists a function $\pi$ from $\mathsf{L}_{\T}$ to $\Wmc$, and, for every $n \in \mathsf{L}_{\T}$, there exists a function $\pi_{n}$ from the set of 
%variables 
{{terms}}
occurring in $S_{n}$ to $\Delta_{\pi(n)}$, such that
$\gamma \in \Gamma^{x}_{n}$ implies $\gamma \in \Phi^{\pi_{n}(x)}_{\pi(n)}$.
%\pi(n) \in \llbracket \gamma \rrbracket^{\Mmc}_{\pi_{n}(x)}$.
%\nb{M: todo fix}
%\begin{itemize}
%	\item there exists a function $\pi \colon \mathsf{L}_{\T} \to \Wmc$
%%	from $N$ to $\Wmc$
%%	the set of labels of the labelled constraints in $\T$
%	such that $n : \psi \in S_{n}$ implies $\Mmc, \pi(n) \models \psi$, for every $\psi \in \for(\p)$;
%	\item for every $n \in \mathsf{L}_{\T}$, there exists a function $\pi_{n}$ from the set of variables occurring in $S_{n}$ to $\Delta_{\pi(n)}$ such that $n: C(x) \in S_{n}$ implies $\pi_{n}(x) \in C^{\Imc_{\pi(n)}}$.
%\end{itemize}
We  require the following claim.

\begin{claim}
\label{cla:compatible}
If a completion set $\T$ for $\p$ is $\Mmc$-compatible
{{and $\Mmc$ is an $\LnALC$-model}},
%\todo{T:added condition that $\Mmc$ is an $\LnALC$-model. \\ M: thanks}
then for every  $\LnALC$-rule $\mathsf{R}$ applicable to $\T$, there exists a completion set $\T'$ obtained from $\T$ by an application of $\mathsf{R}$ such that $\T'$ is $\Mmc$-compatible.
%If a completion set $\mathbf{T}$ for $\p$ is $\Mmc$-compatible and $\mathbf{T}'$ is obtained from $\mathbf{T}$ by an application of an $\LnALC$-rule $\mathsf{R}$, then $\mathbf{T}'$ is $\Mmc$-compatible.
\end{claim}
%%Old proof
%\begin{proof}
%{{Given an $\Mmc$-compatible completion set $\T$ for $\p$ and a label $n \in \mathsf{L}_{\T}$, let $\pi$ and $\pi_{n}$ be the functions provided by the definition of $\Mmc$-compatibility.
%We need to consider each $\LnALC$-rule $\mathsf{R}$.
% For $\mathsf{R} \in \{ \mathsf{R}_{\land}, \mathsf{R}_{\lor}, \mathsf{R}_{\sqcap}, \mathsf{R}_{\sqcup}, \mathsf{R}_{\forall}, \mathsf{R}_{\exists}, \mathsf{R}_{\sqsubseteq}, \mathsf{R}_{\not\sqsubseteq} \}$, we proceed similarly to~\cite[Claim 15.14]{GabEtAl03}.
%Here we consider %the case of $\mathsf{R}_{\mathit{L}}$:
%the modal rules.
%}}
%Suppose that $\mathsf{R}_{\mathit{L}}$ is applicable to $\T$.
%	Let $\Box_{i} \gamma_{1} \in \Gamma^{x_{1}}_{n}, \ldots, \Box_{i} \gamma_{k} \in \Gamma^{x_{k}}_{n}, \Diamond_{i} \delta \in \Gamma^{y}_{n}$.
%	 Since $\T$ is $\Mmc$-compatible,
%	 we have that $\Box_{i}\gamma_{1} \in \Phi^{\pi_{n}(x_{1})}_{\pi(n)}, \ldots, \Box_{i} \gamma_{k} \in \Phi^{\pi_{n}(x_{k})}_{\pi(n)}$ and $\Diamond_{i} \delta \in \Phi^{\pi_{n}(y)}_{\pi(n)}$.
%Thus, by Claim~\ref{cla:truth}, there exists $v \in \Wmc$ such that:
%%\begin{itemize}
%%\item $\gamma_{1} \in \Phi^{\pi_{n}(x_{1})}_{v}, \ldots, \gamma_{k} \in \Phi^{\pi_{n}(x_{k})}_{v}$ and $\delta \in \Phi^{\pi_{n}(y)}_{v}$; or
%%\item $\dnot \gamma_{j} \in \Phi^{\pi_{n}(x_{j})}_{v}$ and $\dnot \delta \in \Phi^{\pi_{n}(y)}_{v}$,
%%for some $j\leq l$;
%%\end{itemize}
%$\gamma_{1} \in \Phi^{\pi_{n}(x_{1})}_{v}, \ldots, \gamma_{k} \in \Phi^{\pi_{n}(x_{k})}_{v}$ and $\delta \in \Phi^{\pi_{n}(y)}_{v}$; or
%$\dnot \gamma_{j} \in \Phi^{\pi_{n}(x_{j})}_{v}$ and $\dnot \delta \in \Phi^{\pi_{n}(y)}_{v}$,
%for some $j\leq l$;
%%$(0)$ $\gamma_{1} \in \Phi^{\pi_{n}(x_{1})}_{v}, \ldots, \gamma_{k} \in \Phi^{\pi_{n}(x_{k})}_{v}$ and $\delta \in \Phi^{\pi_{n}(y)}_{v}$; or
%%$(1)$ $\dnot \gamma_{1} \in \Phi^{\pi_{n}(x_{1})}_{v}$ and $\dnot \delta \in \Phi^{\pi_{n}(y)}_{v}$; or
%%...
%%$(l)$ $\dnot \gamma_{k} \in \Phi^{\pi_{n}(x_{k})}_{v}$ and $\dnot \delta \in \Phi^{\pi_{n}(y)}_{v}$;
%where:
%for $\mathit{L} = \mathbf{E}$, $k = l = 1$;
%for $\mathit{L} = \mathbf{M}$, $k = 1$ and $l = 0$;
%for $\mathit{L} = \mathbf{C}$, $k \geq 1$ and $l = k$;
%for $\mathit{L} = \mathbf{N}$, $k = l = 1$ or $k = l = 0$.
%%
%By applying the rule $\mathsf{R}_{\mathit{L}}$ accordingly, one can obtain $\T'$ by adding
%$m : \gamma_1, \ldots, m : \gamma_k, m : \delta$, or $m : \dot{\lnot}\gamma_j, m : \dnot \delta $, for some $j\leq l$, to $\T$
%(recall that $m$ is fresh for $\T$ and $\gamma_{j}$ is either $\psi_{j} \in \forneg(\p)$ or $C_{j}(x_{j})$, with $C_{j} \in \conneg(\p)$, for $j = 1, \ldots, k$, and $\delta$ is either $\chi \in \forneg(\p)$ or $D(y)$, with $D \in \conneg(\p)$).
%%	 The application of $\mathsf{R}_{\mathit{L}}$ non-deterministically chooses to add to $\T$ either $\{ m : \gamma_1, \ldots, m: \gamma_k, m: \delta\}$, or $\{ m: \dot{\lnot}\gamma_j, o: \dot{\lnot}\delta\}$, for some $j\leq l$, where $m$ is the $\ll$-minimal label fresh for $\T$.
%%	 We set $\pi(m) = \ldots$
%By extending $\pi$ with $\pi(m) = v$, and $\pi_{m}$ with $\pi_{m}(x_{1}) = \pi_{n}(x_{1})$, \ldots, $\pi_{m}(x_{k}) = \pi_{n}(x_{k})$, $\pi_{m}(y) = \pi_{n}(y)$, we obtain that $\T'$ is $\Mmc$-compatible.
%\end{proof}
%
%To conclude, let $\T_{\p} = \{0 : \p
%%0 : \top(x)
%\}$ be the initial completion set for $\p$.
%%with $0$ being the label in $\mathbf{T}_{\p}$ and $x$ be the variable occurring in $\mathbf{T}_{\p}$.
%Define $\pi(0) = w_{\p}$ (where $\Mmc, w_{\p} \models \p$) and $\pi_{0}(x) = d$, for an arbitrary $d \in \Delta_{w_{\p}}$.
%Clearly, these functions ensure that $\T_{\p}$ is $\Mmc$-compatible.
%By Claim~\ref{cla:compatible}, we can apply the $\LnALC$-rules so that the obtained completion sets are $\Mmc$-compatible as well.
%From Theorem~\ref{thm:termination}, we have that the $\LnALC$ tableau algorithm eventually terminates, %returning an $\LnALC$-complete completion set for $\p$ that is clash-free by construction.
%providing an $\LnALC$-complete completion set for $\p$ that is clash-free by construction.
%\end{proof}
\begin{proof}
{{Given an $\Mmc$-compatible completion set $\T$ for $\p$ and a label $n \in \mathsf{L}_{\T}$, let $\pi$ and $\pi_{n}$ be the functions provided by the definition of $\Mmc$-compatibility.
We need to consider each $\LnALC$-rule $\mathsf{R}$.
 For $\mathsf{R} \in \{ \mathsf{R}_{\land}, \mathsf{R}_{\lor}, \mathsf{R}_{\sqcap}, \mathsf{R}_{\sqcup}, \mathsf{R}_{\forall}, \mathsf{R}_{\exists}, \mathsf{R}_{\sqsubseteq}, \mathsf{R}_{\not\sqsubseteq} \}$, we proceed similarly to~\cite[Claim 15.14]{GabEtAl03}.
Here we consider %the case of $\mathsf{R}_{\mathit{L}}$:
the modal rules.

%\todo{T: shouldn't we replace $\Nmc_{i}(w)$ with $\Nmc_{i}(\pi(n))$?}
\begin{enumerate}[leftmargin=*, align=left]
	\item[($\mathsf{R}_{\mathit{L}}$)]
Suppose that $\mathsf{R}_{\mathit{L}}$ is applicable to $\T$.
	Then there are $\Box_{i} \gamma_{1} \in \Gamma^{x_{1}}_{n}, \ldots, \Box_{i} \gamma_{k} \in \Gamma^{x_{k}}_{n}, \Diamond_{i} \delta \in \Gamma^{y}_{n}$.
	 Since $\T$ is $\Mmc$-compatible,
	 we have that $\Box_{i}\gamma_{1} \in \Phi^{\pi_{n}(x_{1})}_{\pi(n)}, \ldots, \Box_{i} \gamma_{k} \in \Phi^{\pi_{n}(x_{k})}_{\pi(n)}$ and $\Diamond_{i} \delta \in \Phi^{\pi_{n}(y)}_{\pi(n)}$,
	 meaning that $\llbracket \gamma_{j} \rrbracket^{\Mmc}_{d_{j}} \in \Nmc_{i}(\pi(n))$, for $j = 1, \ldots, k$, 
	 hence by the $C$-condition  $\bigcap_{j = 1}^{k} \llbracket \gamma_{j} \rrbracket^{\Mmc}_{d_{j}}  \in \Nmc_{i}(\pi(n))$,
	 and $\Wmc \setminus \llbracket \delta \rrbracket^{\Mmc}_{e} \not \in \Nmc_{i}(\pi(n))$, i.e., $\llbracket \dnot \delta \rrbracket^{\Mmc}_{e} \not \in \Nmc_{i}(\pi(n))$.
	 Then 
	 $\bigcap_{j = 1}^{k} \llbracket \gamma_{j} \rrbracket^{\Mmc}_{d_{j}} \neq \llbracket \dnot \delta \rrbracket^{\Mmc}_{e}$
	 (if $\mathbf{M}\in\Lvar$, 
	 $\bigcap_{j = 1}^{k} \llbracket \gamma_{j} \rrbracket^{\Mmc}_{d_{j}} \not\subseteq \llbracket \dnot \delta \rrbracket^{\Mmc}_{e}$).
It follows that there exists $v \in \Wmc$ such that
$\gamma_{1} \in \Phi^{\pi_{n}(x_{1})}_{v}, \ldots, \gamma_{k} \in \Phi^{\pi_{n}(x_{k})}_{v}$ and $\delta \in \Phi^{\pi_{n}(y)}_{v}$; or
$\dnot \gamma_{j} \in \Phi^{\pi_{n}(x_{j})}_{v}$ and $\dnot \delta \in \Phi^{\pi_{n}(y)}_{v}$,
for some $j\leq k$.
%To see this, suppose towards a contradiction that
%$\gamma_{1} \in \Phi^{d_{1}}_{v}, \ldots, \gamma_{k} \in \Phi^{d_{k}}_{v}$ implies $\delta \not \in \Phi^{e}_{v}$; and
%$\dnot \delta \in \Phi^{e}_{v}$ implies $\dnot \gamma_{j} \not \in \Phi^{d_{j}}_{v}$ for all $j\leq k$.
%This means that
%$\bigcap_{j = 1}^{k} \llbracket \gamma_{j} \rrbracket^{\Mmc}_{d_{j}} \subseteq \llbracket \dnot \delta \rrbracket^{\Mmc}_{e}$; and
%$\llbracket \dnot \delta \rrbracket^{\Mmc}_{e} \subseteq \llbracket \gamma_{j} \rrbracket^{\Mmc}_{d_{j}}$
%for all $j \leq k$, 
%that is
%$\llbracket \dnot \delta \rrbracket^{\Mmc}_{e} \subseteq \bigcap_{j = 1}^{k} \llbracket \gamma_{j} \rrbracket^{\Mmc}_{d_{j}}$;
%hence $\bigcap_{j = 1}^{k} \llbracket \gamma_{j} \rrbracket^{\Mmc}_{d_{j}} =  \llbracket \dnot \delta \rrbracket^{\Mmc}_{e}$.
%Since $\Mmc$ is closed under intersection, we obtain $\llbracket \dnot \delta \rrbracket^{\Mmc}_{e} \in \Nmc_{i}(\pi(n))$,
%a contradiction.
%
Then by applying the rule $\mathsf{R}_{\mathit{L}}$ %to $\T$ 
accordingly, 
%Then, depending on the case, we apply the rule $\mathsf{R}_{\mathit{L}}$ accordingly and 
we expand $\T$ to $\T'$ with
$m : \gamma_1, \ldots, m : \gamma_k, m : \delta$, or with $m : \dot{\lnot}\gamma_j, m : \dnot \delta $, for some $j\leq k$,
for some $m$ satisfying the application condition of $\mathsf{R}_{\mathit{L}}$.
Since $m$ is fresh, we can extend $\pi$ with $\pi(m) = v$, and $\pi_{m}$ with $\pi_{m}(x_{1}) = \pi_{n}(x_{1})$, \ldots, $\pi_{m}(x_{k}) = \pi_{n}(x_{k})$, $\pi_{m}(y) = \pi_{n}(y)$, thus obtaining that $\T'$ is $\Mmc$-compatible.

	\item[($\mathsf{R}_{\mathit{L}\mathbf{N}}$)]
	Suppose that $\mathsf{R}_{\mathit{L}\mathbf{N}}$ is applicable to $\T$.
	Then there is $\Diamond_{i} \delta \in \Gamma^{y}_{n}$.
	 Since $\T$ is $\Mmc$-compatible,
	 we have that $\Diamond_{i} \delta \in \Phi^{\pi_{n}(y)}_{\pi(n)}$,
%	 meaning that $\Wmc \setminus \llbracket \delta \rrbracket^{\Mmc}_{e} \not \in \Nmc_{i}(\pi(n))$, i.e., $\llbracket \dnot \delta \rrbracket^{\Mmc}_{e} \not \in \Nmc_{i}(\pi(n))$.
%	 It follows that there exists $v \in \Wmc$ such that $\delta \in \Phi^{\pi_{n}(y)}_{v}$. 
%	 To see this, suppose towards a contradiction that
%	 for every $v \in \Wmc$, $\delta \not \in \Phi^{e}_{v}$.
%	Since, by definition, we have $\delta \not \in \Phi^{e}_{v}$ iff $\dnot \delta \in \Phi^{e}_{v}$, the previous step means that $\Wmc \subseteq \llbracket \dnot \delta \rrbracket^{\Mmc}_{e}$, and hence $\llbracket \delta \rrbracket^{\Mmc}_{e} = \emptyset$. Thus, $\Wmc \not \in \Nmc_{i}(\pi(n))$, contradicting the fact that $\Mmc$ contains the unit.
	 meaning that $\Wmc \setminus \llbracket \delta \rrbracket^{\Mmc}_{e} \not \in \Nmc_{i}(\pi(n))$.
	 At the same time, by the $N$-condition, $\Wmc \in \Nmc_{i}(\pi(n))$,
	  hence $\llbracket \delta \rrbracket^{\Mmc}_{e} \not= \emptyset$,
	 that is there exists $v \in \Wmc$ such that $\delta \in \Phi^{\pi_{n}(y)}_{v}$. 
	Then we expand $\T$ with $m : \delta$, for some $m$ satisfying the application condition of $\mathsf{R}_{\mathit{L}\mathbf{N}}$.
Since $m$ is fresh, we can extend $\pi$ with $\pi(m) = v$, and $\pi_{m}$ with $\pi_{m}(y) = \pi_{n}(y)$, thus obtaining that $\T'$ is $\Mmc$-compatible.
	
	\item[($\mathsf{R}_{\mathit{L}\mathbf{P}}$)]
Suppose that $\mathsf{R}_{\mathit{L}\mathbf{P}}$ is applicable to $\T$.
	Then there are $\Box_{i} \gamma_{1} \in \Gamma^{x_{1}}_{n}, \ldots, \Box_{i} \gamma_{k} \in \Gamma^{x_{k}}_{n}$.
	 Since $\T$ is $\Mmc$-compatible,
	 we have that $\Box_{i}\gamma_{1} \in \Phi^{\pi_{n}(x_{1})}_{\pi(n)}, \ldots, \Box_{i} \gamma_{k} \in \Phi^{\pi_{n}(x_{k})}_{\pi(n)}$, 
	 meaning that $\llbracket \gamma_{j} \rrbracket^{\Mmc}_{d_{j}} \in \Nmc_{i}(\pi(n))$, for $j = 1, \ldots, k$,
	 hence $\bigcap_{j = 1}^{k} \llbracket \gamma_{j} \rrbracket^{\Mmc}_{d_{j}} \in \Nmc_{i}(\pi(n))$.
	 At the same time, by the $P$-condition, 
	 $\bigcap_{j = 1}^{k} \llbracket \gamma_{j} \rrbracket^{\Mmc}_{d_{j}} \neq \emptyset$,
	that is there exists $v \in \Wmc$ such that
	$\gamma_{1} \in \Phi^{\pi_{n}(x_{1})}_{v}, \ldots, \gamma_{k} \in \Phi^{\pi_{n}(x_{k})}_{v}$.
	Then we expand $\T$ with $m : \gamma_1, \ldots, m : \gamma_k$,
	for some $m$ satisfying the application condition of $\mathsf{R}_{\mathit{L}\mathbf{P}}$.
	Since $m$ is fresh, we can extend $\pi$ with $\pi(m) = v$, and $\pi_{m}$ with $\pi_{m}(x_{1}) = \pi_{n}(x_{1})$, \ldots, $\pi_{m}(x_{k}) = \pi_{n}(x_{k})$, thus obtaining that $\T'$ is $\Mmc$-compatible.

	\item[($\mathsf{R}_{\mathit{L}\mathbf{Q}}$)]
	Suppose that $\mathsf{R}_{\mathit{L}\mathbf{Q}}$ is applicable to $\T$.
	Then there are $\Box_{i} \gamma_{1} \in \Gamma^{x_{1}}_{n}, \ldots, \Box_{i} \gamma_{k} \in \Gamma^{x_{k}}_{n}$.
	 Since $\T$ is $\Mmc$-compatible,
	 we have that $\Box_{i}\gamma_{1} \in \Phi^{\pi_{n}(x_{1})}_{\pi(n)}, \ldots, \Box_{i} \gamma_{k} \in \Phi^{\pi_{n}(x_{k})}_{\pi(n)}$, 
	 meaning that $\llbracket \gamma_{j} \rrbracket^{\Mmc}_{d_{j}} \in \Nmc_{i}(\pi(n))$, for $j = 1, \ldots, k$,
	 hence $\bigcap_{j = 1}^{k} \llbracket \gamma_{j} \rrbracket^{\Mmc}_{d_{j}} \in \Nmc_{i}(\pi(n))$.
	 At the same time, by the $Q$-condition, 
	 $\bigcap_{j = 1}^{k} \llbracket \gamma_{j} \rrbracket^{\Mmc}_{d_{j}} \neq \Wmc$,
	that is there exists $v \in \Wmc$ such that
	$\gamma_{j} \notin \Phi^{\pi_{n}(x_{j})}_{v}$ for some $j\leq k$.
	Then by applying $\mathsf{R}_{\mathit{L}\mathbf{Q}}$ accordingly, we expand $\T$ with $m : \dot{\lnot}\gamma_j$,
	for some $m$ satisfying the application condition of $\mathsf{R}_{\mathit{L}\mathbf{Q}}$.
	Since $m$ is fresh, we can extend $\pi$ with $\pi(m) = v$, and 
%	\todo{AM: to discuss?}
	{{$\pi_{m}$ with $\pi_{m}(x_{j}) = \pi_{n}(x_{j})$,  thus obtaining that $\T'$ is $\Mmc$-compatible.}}

	\item[($\mathsf{R}_{\mathit{L}\mathbf{D}}$)]
	Suppose that $\mathsf{R}_{\mathit{L}\mathbf{D}}$ is applicable to $\T$.
	Then there are $\Box_{i} \gamma_{1} \in \Gamma^{x_{1}}_{n}, \ldots, \Box_{i} \gamma_{k} \in \Gamma^{x_{k}}_{n}, \Box_{i} \delta_{1} \in \Gamma^{y_{1}}_{n}, \ldots, \Box_{i} \delta_{h} \in \Gamma^{y_{h}}_{n}$.
	Since $\T$ is $\Mmc$-compatible,
	we have that $\Box_{i}\gamma_{1} \in \Phi^{\pi_{n}(x_{1})}_{\pi(n)}, \ldots, \Box_{i} \gamma_{k} \in \Phi^{\pi_{n}(x_{k})}_{\pi(n)}$,
	and $\Box_{i}\delta_{1} \in \Phi^{\pi_{n}(y_{1})}_{\pi(n)}, \ldots, \Box_{i} \delta_{h} \in \Phi^{\pi_{n}(y_{h})}_{\pi(n)}$, 
	meaning that $\llbracket \gamma_{j} \rrbracket^{\Mmc}_{d_{j}} \in \Nmc_{i}(\pi(n))$, for $j = 1, \ldots, k$,
	and $\llbracket \delta_{\ell} \rrbracket^{\Mmc}_{e_{\ell}} \in \Nmc_{i}(\pi(n))$, for $\ell = 1, \ldots, h$;
	hence $\bigcap_{j = 1}^{k} \llbracket \gamma_{j} \rrbracket^{\Mmc}_{d_{j}} \in \Nmc_{i}(\pi(n))$,
	and $\bigcap_{\ell = 1}^{h} \llbracket \delta_{\ell} \rrbracket^{\Mmc}_{e_{\ell}} \in \Nmc_{i}(\pi(n))$.
	By the $D$-condition, 
	$\bigcap_{\ell = 1}^{h} \llbracket \delta_{\ell} \rrbracket^{\Mmc}_{e_{\ell}} \not=\Wmc \setminus \bigcap_{j = 1}^{k} \llbracket \gamma_{j} \rrbracket^{\Mmc}_{d_{j}}$
	(if $\mathbf{M}\in\Lvar$, $\bigcap_{\ell = 1}^{h} \llbracket \delta_{\ell} \rrbracket^{\Mmc}_{e_{\ell}} \not \subseteq \Wmc \setminus \bigcap_{j = 1}^{k} \llbracket \gamma_{j} \rrbracket^{\Mmc}_{d_{j}}$).
	This means that there exists $v \in \Wmc$ such that
$\gamma_{1} \in \Phi^{\pi_{n}(x_{1})}_{v}, \ldots, \gamma_{k} \in \Phi^{\pi_{n}(x_{k})}_{v},
\delta_{1} \in \Phi^{\pi_{n}(y_{1})}_{v}, \ldots, \delta_{h} \in \Phi^{\pi_{n}(y_{h})}_{v}$; or
$\dnot \gamma_{j} \in \Phi^{\pi_{n}(x_{j})}_{v}$ and $\dnot \delta_{\ell} \in \Phi^{\pi_{n}(y_{\ell})}_{v}$,
for some $j\leq k$, $\ell \leq h$.
Then by applying the rule $\mathsf{R}_{\mathit{L}\mathbf{D}}$ %to $\T$ 
accordingly, 
we expand $\T$ to $\T'$ with
$m : \gamma_1, \ldots, m : \gamma_k, m : \delta_1, \ldots, m : \delta_h$, or with $m : \dot{\lnot}\gamma_j, m : \dnot \delta_{\ell} $, for some $j\leq k$, $\ell \leq h$,
and some $m$ satisfying the application condition of $\mathsf{R}_{\mathit{L}\mathbf{D}}$.
Since $m$ is fresh, we can extend $\pi$ with $\pi(m) = v$, 
and $\pi_{m}$ with $\pi_{m}(x_{j}) = \pi_{n}(x_{j})$, for $j \leq k$, 
$\pi_{m}(y_{\ell}) = \pi_{n}(y_{\ell})$, for $\ell \leq h$; thus obtaining that $\T'$ is $\Mmc$-compatible.


	\item[($\mathsf{R}_{\mathit{L}\mathbf{T}}$)]
	Suppose that $\mathsf{R}_{\mathit{L}\mathbf{T}}$ is applicable to $\T$.
	Then there is $\Box_{i} \gamma \in \Gamma^{x}_{n}$.
	 Since $\T$ is $\Mmc$-compatible,
	 we have that $\Box_{i}\gamma \in \Phi^{\pi_{n}(x)}_{\pi(n)}$, 
	 meaning that $\llbracket \gamma \rrbracket^{\Mmc}_{d} \in \Nmc_{i}(\pi(n))$.
	By the $T$-condition, 
	$\pi(n) \in \llbracket \gamma \rrbracket^{\Mmc}_{d}$,
	that is $\gamma \in \Phi^{\pi_{n}(x)}_{\pi(n)}$.
	Then the expansion $\T'$ of $\T$ with $n: \gamma$, obtained by the application of $\mathsf{R}_{\mathit{L}\mathbf{T}}$,
	 is $\Mmc$-compatible.\qedhere
\end{enumerate}
}}
By the argument of Theorem~\ref{thm:termination},
it can be seen that
after finitely many steps we obtain a complete completion set $\Tmc'$. Moreover, $\Tmc'$ is $\Mmc$-compatible, hence clearly clash-free.
\end{proof}


By Theorem~\ref{thm:termination}, we have that the $\LnALC$ tableau algorithm terminates after exponentially many steps in the size of the input formula. By Theorems~\ref{thm:soundness} and~\ref{thm:completeness}, the non-deterministic decision procedure based on the $\LnALC$ tableau algorithm is sound and complete with respect to satisfiability in varying domain neighbourhood models. 
%
Thus, we obtain the following result.
%\nb{M: todo fix notation in Preliminaries}

\begin{theorem}
	\label{thm:upperbound}
	Satisfiability in $\LnALC$  on varying domain neighbourhood models is decidable in $\NExpTime$.
\end{theorem}

To conclude, let $\T_{\p} = \{0 : \p
%0 : \top(x)
\}$ be the initial completion set for $\p$.
%with $0$ being the label in $\mathbf{T}_{\p}$ and $x$ be the variable occurring in $\mathbf{T}_{\p}$.
Define $\pi(0) = w_{\p}$ (where $\Mmc, w_{\p} \models \p$) and $\pi_{0}(x) = d$, for an arbitrary $d \in \Delta_{w_{\p}}$.
Clearly, these functions ensure that $\T_{\p}$ is $\Mmc$-compatible.
By Claim~\ref{cla:compatible}, we can apply the $\LnALC$-rules so that the obtained completion sets are $\Mmc$-compatible as well.
From Theorem~\ref{thm:termination}, we have that the $\LnALC$ tableau algorithm eventually terminates, %returning an $\LnALC$-complete completion set for $\p$ that is clash-free by construction.
providing an $\LnALC$-complete completion set for $\p$ that is clash-free by construction.
\end{proof}

























\Fmp*
\begin{proof}
By Theorem \ref{thm:completeness}, if $\p$ is $\LnALC$ satisfiable, then 
there is a $\LnALC$-complete and clash-free completion set $\T$ for it.
Then by Theorem \ref{thm:soundness},
%basing on $\mathbf{T}$ we can define a
there exists a model 
$\Mmc = (\Wmc, \{ \Nmc_{i} \}_{i \in J}, \Imc)$
for $\p$ where $\Wmc =  \mathsf{L}_{\T}$
and for each $n\in\Wmc$, $\Delta_{n} = \{ x \in \mathsf{N_{V}} \mid x \ \text{occurs in} \ S_{n} \}$.
By Theorem~\ref{thm:termination}, Claim~\ref{cla:termglobal}, it follows
$|\Wmc| \leq p(|\fg(\p)|)$,
if $\mathbf{C}\notin\Lvar$, 
and $|\Wmc| \leq 2^{q(|\fg(\p)|)}$, 
if $\mathbf{C}\in\Lvar$,
where $p$ and $q$ are polynomial functions.
Finally by Theorem~\ref{thm:termination}, Claim~\ref{cla:termlocal}, 
for each $n\in\Wmc$, $|\Delta_n|$ does not exceed $2^{r(|\fg(\p)|)}$,
where $r$ is a polynomial function.
\end{proof}









%\subsection{Proofs for Section~\ref{sec:fragvardom}}
\section{Proofs for Section~\ref{sec:fragvardom}}







\LemmapropL*
\begin{proof}
%{{
If an
$\MLnALCg$
formula $\varphi$ is $\LnALCg$ satisfiable
on varying domain neighbourhood models
then, clearly,
$\prop{\varphi}$ is satisfied in a $\setsymbols_\varphi$-consistent $L^{n}$ model.  
We now argue about the converse direction. 
Suppose that $\prop{\varphi}$ is satisfied in a $\setsymbols_\varphi$-consistent $L^{n}$ model
$\propmodel = (\Wmc, \{ \Nmc_{i} \}_{i \in J}, \Vmc)$. 
%We define $\W$ as $\propdomain$ and \Nmc as $\propneigh$. 
%The main point in this proof is the definition of $\Imc$.
%
%Given $w \in \Wmc$, let $\NPr(w) = \{p_{\elaxiom}\in \NPr(\varphi) \mid w\in \Vmc(p_{\elaxiom})\}$.
As $\propmodel$ is $\setsymbols_\varphi$-consistent, we have that, for every $w\in \propdomain$,
the $\ALC$ formula
$\alcform$
%\[
%\formula =  \bigwedge_{p_{\elaxiom}\in \formtp{\varphi}} {\elaxiom} \ \wedge \bigwedge_{p_{\elaxiom} \in
%\NPr(\varphi)\setminus\formtp{\varphi}}
%% \overline{\NPr(w)}}
% \neg {\elaxiom}
%\]
%and $\overline{\NPr(w)}=\NPr(\varphi)\setminus\NPr(w)$
is satisfied by an $\ALC$ interpretation, say $\Imc_{w} = (\Delta_{w}, \cdot^{\Imc_{w}})$.
We define the
varying domain neighbourhood model $\Mmc=(\Fmc,\Imc)$, where the $L^{n}$ frame $\Fmc = ( \W, \{ \Nmc_{i} \}_{i \in J} )$ is as above,
and where $\Imc$ is a function associating with each $w \in \Wmc$ the $\ALC$ interpretation $\Imc_{w}$.
By induction on the structure of subformulas $\psi$ of $\varphi$, it can be shown that,
for every $w \in \Wmc$, we have
$\propmodel, w \models \prop{\psi}$
iff
 $\Mmc, w \models \psi$. We show this in Claim~\ref{cl:ind}.
\begin{claim}\label{cl:ind}
For every subformula $\psi$ of $\varphi$ and every $w \in \Wmc$, we have
$\propmodel, w \models \prop{\psi}$ iff
$\Mmc, w \models \psi$.
\end{claim}
\begin{proof}
In the base case $\psi$ is an $\ALC$ atom $\pi$ in $\varphi$ and $\prop{\psi}$ is a propositional symbol $p_\pi$. By the semantics of propositional neighbourhood models,
$\propmodel, w \models \prop{\psi}$ iff $w\in\Vmc(p_\pi)$. 
For every \ALC atom $\pi$ in $\varphi$, $w\in\Vmc(p_\pi)$ iff 
$\pi$ is a conjunct of $\hat{\varphi}_{\Vmc,w}$.
As $\propmodel$ is $\setsymbols_\varphi$-consistent, we have that, for every $w\in \propdomain$,
the $\ALC$ formula
$\hat{\varphi}_{\Vmc,w}$
is satisfied by the $\ALC$ interpretation $\Imc_{w} = (\Delta_{w}, \cdot^{\Imc_{w}})$.
 Thus, $\pi$ is a conjunct of $\hat{\varphi}_{\Vmc,w}$ iff $\Imc_w\models\pi$.
 By the semantics of $\MLnALCg$  neighbourhood models,
 $\Imc_w\models\pi$ iff $\Mmc, w \models \psi$.
 Suppose that Claim~\ref{cl:ind} holds for $\psi_1,\psi_2$. 
 For the inductive step, we make the following case distinction on 
 the format of $\psi$. 
 \begin{itemize}
 	\item $\psi=\neg\psi_1$: By the semantics of  propositional neighbourhood models,
 	$\propmodel, w \models \prop{\neg{\psi_1}}$ iff $\propmodel, w \not\models \prop{{\psi_1}}$. By the inductive hypothesis, Claim~\ref{cl:ind} holds for $\psi_1$.
 	By the contrapositive in each direction, $\propmodel, w \not\models \prop{{\psi_1}}$
 	iff $\Mmc, w \not\models \psi_1$. By the semantics of  $\MLnALCg$ neighbourhood models, $\Mmc, w \not\models \psi_1$ iff $\Mmc, w \models \neg\psi_1$.
\item $\psi=\psi_1\wedge\psi_2$: By the semantics of  propositional neighbourhood models,
$\propmodel, w \models \prop{{(\psi_1\wedge\psi_2)}}$ iff $\propmodel, w \models \prop{{\psi_1}}$ and $\propmodel, w \models \prop{{\psi_2}}$. By the inductive hypothesis, Claim~\ref{cl:ind} holds for $\psi_1,\psi_2$.
So, $\propmodel, w \models \prop{{\psi_i}}$
iff $\Mmc, w \models \psi_i$, for $i\in \{1,2\}$. By the semantics of  $\MLnALCg$ neighbourhood models, $\Mmc, w \models \psi_1$ and $\Mmc, w \models \psi_2$ iff $\Mmc, w \models \psi_1\wedge \psi_2$.
\item $\psi=\B_{i} \psi_1$: By the semantics of  propositional neighbourhood models,
$\propmodel, w \models \prop{{(\B_{i} \psi_1)}}$ iff $\llbracket \prop{{\psi_1}}\rrbracket^{\propmodel} \in \Nmc_{i}(w)$ where
$\llbracket \prop{{\psi_1}} \rrbracket^{\propmodel} = \{ v \in \Wmc \mid \propmodel, v \models \prop{{\psi_1}} \}$. By the inductive hypothesis, Claim~\ref{cl:ind} holds for $\psi_1$.
So, $\propmodel, v \models \prop{{\psi_1}}$
iff $\Mmc, v \models \psi_1$, for every $v\in\Wmc$. 
Thus, $\llbracket \prop{{\psi_1}} \rrbracket^{\propmodel}=\llbracket {{\psi_1}} \rrbracket^{\Mmc}$. By definition of $\propmodel$ and \Mmc, we have that $\Nmc_{i}(w)$
is the same in both $\propmodel$ and \Mmc, for every $w\in\Wmc$ and $i\in J$.
So $\llbracket \prop{{\psi_1}}\rrbracket^{\propmodel} \in \Nmc_{i}(w)$
iff $\llbracket {{\psi_1}} \rrbracket^{\Mmc} \in \Nmc_{i}(w)$.
By the semantics of  $\MLnALCg$ neighbourhood models, $\llbracket {{\psi_1}} \rrbracket^{\Mmc} \in \Nmc_{i}(w)$ iff $\Mmc, w \models \B_{i} \psi_1$.
 \end{itemize}
We have thus shown that for every subformula $\psi$ of $\varphi$ and every $w \in \Wmc$, we have
$\propmodel, w \models \prop{\psi}$ iff
$\Mmc, w \models \psi$.
\end{proof}
Since $\propmodel , v \models \prop{\varphi}$, for some $v \in \Wmc$, we conclude that $\varphi$ is $\LnALCg$ satisfiable. 
%}
\end{proof}














\Lemmapropvardi*
\begin{proof}
{{In this proof, for any set $S\subseteq\{\mathbf{E,M,C,N,T,P,Q,D}\}$, we call $S$ model any neighbourhood model satisfying all conditions in $S$.}}
	We start with proving ($\Rightarrow$). 
	We consider the more complex case where $\mathbf{C}\in\Lvar$.
	For $\mathbf{C}\not\in\Lvar$ the proof simplifies by taking $k = 1$.
	Suppose that $\phi$ is satisfied in a world $w$ of a $\setsymbols$-consistent $\Lvar^{n}$ model 
	$\propmodel = (\propdomain, \{ \propneigh_{i} \}_{i \in J}, \propassign)$. That is, 
	$\propmodel, w\models \phi$. We define a $\setsymbols$-consistent valuation for 
	$\phi$
	by setting, for all $\psi \in {\sf sub}(\phi)$,
	$\nu(\psi)=1$ if $\propmodel, w\models \psi$ and $\nu(\psi) = 0$
	if  $\propmodel, w\not\models \psi$. 
	It is easy to check that $\nu$ is indeed a 
	$\setsymbols$-consistent valuation   (given that $\propmodel$ is a  
			$\setsymbols$-consistent $\Lvar^{n}$ model).   
	%
	Now assume that $\B_i\psi_1, \dots, \B_i\psi_k, \B_i\chi\in{\sf sub}(\phi)$,
	$\valuation(\B_i\psi_j)=1$ for all $1\leq j \leq k$,
	%$\B_i\chi\in{\sf sub}(\phi)$, 
	and $\valuation(\B_i\chi)=0$.
	Then $\propmodel, w\models \B_i\psi_1 \land ... \land \B_i\psi_n\land\neg\B_i\chi$.
	Since $\propmodel$ is a $\mathbf{EC}$ model, this means that
	$\propmodel\not\models \psi_1\land ... \land \psi_n \leftrightarrow \chi$, that is,
	there is a worlds $u$ such that 
	$\propmodel, u\models (\bigwedge^{k}_{j=1}\psi_j\wedge\neg\chi) \vee \bigvee^{k}_{j=1} (\neg\psi_j\wedge\chi)$.
	(If $\mathbf{M}\in\Lvar$, then 
	$\propmodel\not\models \psi_1\land ... \land \psi_n \to \chi$, that is,
	there $u$ such that 
	$\propmodel, u\models (\bigwedge^{k}_{j=1}\psi_j\wedge\neg\chi)$.)
	Since $\propmodel$ is $\setsymbols$-consistent this concludes the proof.
	Now we prove that $\nu$ satifies $(\mathbf{X})$ if $\mathbf{X}\in\Lvar$, for $\mathbf{X}\in\{\mathbf{N,T,P,Q,D}\}$.
	\begin{itemize}
		\item[($\mathbf{N}$)]
		%($N$) 
		If $\nu(\B_i\psi)=0$, then $\propmodel, w\not\models \B_i\psi$.
		Since $\propmodel$ is a $\mathbf{EN}$ model, this means that $\propmodel\not\models\psi$
		(otherwise $\propmodel\models\B_i\psi$),
		that is there is $u$ such that $\propmodel, u \models \neg\psi$.
		
		\item[($\mathbf{T}$)]
		%($T$)
		If $\nu(\B_i\psi)=1$, then $\propmodel, w\models \B_i\psi$, thus since $\propmodel$ is a $\mathbf{ET}$ model, $\propmodel\models\B_i\psi\to\psi$,
		hence $\propmodel, w \models \psi$, that is $\nu(\psi)=1$.
		
		\item[($\mathbf{P}$)]
		%($P$)
		If $\valuation(\B_i\psi_1) = ... = \valuation(\B_i\psi_k) = 1$, 
		then $\propmodel, w \models \B_i\psi_1\land ... \land \B_i\psi_k$.
		Since $\propmodel$ is a $\mathbf{ECP}$ model, $\propmodel, w \models \B_i(\psi_1\land ... \land \psi_k)$,
		and $\propmodel\models\neg\B_i\falseprop$.
		Then $\propmodel\not\models \psi_1\land ... \land \psi_k \leftrightarrow \falseprop$,
		thus there is $u$ such that $\propmodel, u \models \psi_1\land ... \land \psi_k$.
		
		\item[($\mathbf{Q}$)]
		%($Q$)
		If $\valuation(\B_i\psi_1) = ... = \valuation(\B_i\psi_k) = 1$, 
		then $\propmodel, w \models \B_i\psi_1\land ... \land \B_i\psi_k$.
		Since $\propmodel$ is a $\mathbf{ECQ}$ model, $\propmodel, w \models \B_i(\psi_1\land ... \land \psi_k)$,
		and $\propmodel\models\neg\B_i(\trueprop)$.
		Then $\propmodel\not\models \psi_1\land ... \land \psi_k \leftrightarrow \trueprop$,
		thus there is $u$ such that $\propmodel, u \models \neg\psi_1\lor ...\lor\neg\psi_k$.
		
		\item[($\mathbf{D}$)]
		%($D$)
		If $\valuation(\B_i\psi_j)= \valuation(\B_i\chi_\ell)=1$ for all $1\leq j \leq k$, $1\leq \ell \leq h$,
		then $\propmodel, w \models \bigwedge^{k}_{j=1}\B_i\psi_j \land \bigwedge^{h}_{\ell=1}\B_i\chi_\ell$.
		Since $\propmodel$ is a $\mathbf{ECD}$ model, 
		$\propmodel, w \models \B_i(\psi_i\land ... \land \psi_k) \land \B_i(\chi_1\land ... \land \chi_h)$, and
		$\propmodel\models\B_i\zeta\to\neg\B_i\neg\zeta$.
		Then $\propmodel\not\models \psi_1\land ... \land \psi_k \leftrightarrow \neg(\chi_1\land ... \land \chi_h)$,
		thus there is $u$ such that 
		$\propmodel, u \models (\psi_1\land ... \land \psi_k \land \chi_1\land ... \land \chi_h) \lor 
		(\neg(\psi_1\land ... \land \psi_k) \land \neg(\chi_1\land ... \land \chi_h))$.
		If $\mathbf{M}\in\Lvar$, then 
		$\propmodel\not\models \psi_1\land ... \land \psi_k \to \neg(\chi_1\land ... \land \chi_h)$,
		hence there is $u$ such that 
		$\propmodel, u \models \psi_1\land ... \land \psi_k \land \chi_1\land ... \land \chi_h$.
	\end{itemize}
	
	The proof of the converse ($\Leftarrow$) is as follows. 
	Suppose there is a $\setsymbols$-consistent valuation $\nu$ for $\phi$
	satisfying the conditions stated by the lemma.
	We construct a $\Lvar^{n}$ model $\propmodel$ and a world $w$ such that $\propmodel,w\models\phi$.
	By the condition, it follows that for all sets $\Psi$ of formulas
	$\B_i\psi_1, \dots, \B_i\psi_k$  in ${\sf sub}(\phi)$
	such that $\valuation(\B_i\psi_j)=1$ for all $1\leq j \leq k$,
	and all $\B_i\chi$ in ${\sf sub}(\phi)$ such that $\valuation(\B_i\chi)=0$, 
	there is a $\setsymbols$-consistent model \[\propmodel_{\Psi,\chi}=(\propdomain_{\Psi,\chi},
	\{ \propneigh_{{(\Psi,\chi)}_{i}} \}_{i \in J},\propassign_{\Psi,\chi})\]
	and a world 
	$w_{\Psi,\chi}\in \propdomain_{\Psi,\chi}$ such that 
	$\propmodel_{\Psi,\chi},w_{\Psi,\chi}\models(\bigwedge^{k}_{j=1}\psi_j\wedge\neg\chi) \vee \boldsymbol\vartheta$; moreover 
	if $\mathbf{X}\in\Lvar$, for $\mathbf{X}\in\{\mathbf{N,P,Q,D}\}$, the following hold: 
	\begin{itemize}
		\item[$(\mathbf{N})$]
		%$(N)$ 
		for all $\B_i\psi$ in ${\sf sub}(\phi)$ such that $\valuation(\B_i\psi)=0$, 
		there is a $\setsymbols$-consistent $\Lvar^{n}$ model 
		$\propmodel_{\psi}=(\propdomain_{\psi},  \{ \propneigh_{{\psi}_{i}} \}_{i \in J},\propassign_{\psi})$
		and a world 
		$w_{\psi}\in \propdomain_{\psi}$ such that 
		$\propmodel_{\psi}, w_{\psi} \models \neg\psi$;
		%\todo{should we talk about T? T: No, it is fine this way, because for T there is no additional model to consider.}
		\item[$(\mathbf{P})$]
		%$(P)$ 
		for all $\Psi = \{\B_i\psi_1, \dots, \B_i\psi_k\}\subseteq{\sf sub}(\phi)$
		such that $\valuation(\B_i\psi_j)=1$ for all $1\leq j \leq k$,
		there is a $\setsymbols$-consistent $\Lvar^{n}$ model 
		$\propmodel_{\Psi}=(\propdomain_{\Psi},  \{ \propneigh_{{\Psi}_{i}} \}_{i \in J},\propassign_{\Psi})$
		and a world 
		$w_{\Psi}\in \propdomain_{\Psi}$ such that 
		$\propmodel_{\Psi}, w_{\Psi} \models \psi_1\land...\land\psi_k$;
		
		\item[$(\mathbf{Q})$]
		% $(Q)$ 
		for all $\Psi = \{\B_i\psi_1, \dots, \B_i\psi_k\}\subseteq{\sf sub}(\phi)$
		such that $\valuation(\B_i\psi_j)=1$ for all $1\leq j \leq k$,
		there is a $\setsymbols$-consistent $\Lvar^{n}$ model 
		$\propmodel_{\Psi}=(\propdomain_{\Psi},  \{ \propneigh_{{\Psi}_{i}} \}_{i \in J},\propassign_{\Psi})$
		and a world 
		$w_{\Psi}\in \propdomain_{\Psi}$ such that 
		$\propmodel_{\Psi}, w_{\Psi} \models \neg\psi_1\lor ... \lor\neg\psi_k$;
		
		\item[$(\mathbf{D})$]
		%$(D)$ 
		for all $\Psi = \{\B_i\psi_1, \dots, \B_i\psi_k\}$, $\Lambda = \{\B_i\chi_1, \dots, \B_i\chi_h\}$,
		$\Psi,\Lambda\subseteq{\sf sub}(\phi)$
		such that 
		$\valuation(\B_i\psi_j)=\valuation(\B_i\chi_\ell)=1$ for all $1\leq j \leq k$,  $1\leq \ell \leq h$,
		there is a $\setsymbols$-consistent $\Lvar^{n}$ model 
		$\propmodel_{\Psi,\Lambda}=(\propdomain_{\Psi,\Lambda},  \{ \propneigh_{{(\Psi,\Lambda)}_{i}} \}_{i \in J},\propassign_{\Psi,\Lambda})$
		and a world 
		$w_{\Psi,\Lambda}\in \propdomain_{\Psi,\Lambda}$ such that 
		$\propmodel_{\Psi,\Lambda}, w_{\Psi,\Lambda} \models (\bigwedge^{k}_{j=1}\psi_j \land \bigwedge^{h}_{\ell=1}\chi_\ell) \vee \boldsymbol\eta$.
	\end{itemize}
	
	Let $\propmodel_1, ..., \propmodel_m$
	be an enumeration of all $\Lvar^{n}$ models listed above,
	where 
	$\propmodel_j = (\propdomain_j, \{ \propneigh_{j_{i}} \}_{i \in J},\propassign_j)$.
	That is, we take one model $\propmodel_{\Psi,\chi}$ 
	% for all $\Psi = \{\B_i\psi_1, \dots, \B_i\psi_k\}\subseteq{\sf sub}(\phi)$
	% such that $\valuation(\B_i\psi_j)=1$ for all $1\leq j \leq k$,
	% and all $\B_i\chi$ in ${\sf sub}(\phi)$ such that $\valuation(\B_i\chi)=0$;
	for each pair $(\Psi,\B_i\chi)$,
	where $\Psi = \{\B_i\psi_1, \dots, \B_i\psi_k\}\subseteq{\sf sub}(\phi)$,
	$\valuation(\B_i\psi_j)=1$ for all $1\leq j \leq k$,
	$\B_i\chi$ in ${\sf sub}(\phi)$, and
	$\valuation(\B_i\chi)=0$;
	and similarly 
	we take one model $\propmodel_\psi$, $\propmodel_\Psi$, or $\propmodel_{\Psi,\Lambda}$
	for all formulas or sets of formulas 
	%for all models 
	listed in items $(\mathbf{N})$, $(\mathbf{P})$, $(\mathbf{Q})$, $(\mathbf{D})$. %\todo{T?}
	Assume without loss of generality that 
	$\propdomain_j\cap \propdomain_\ell=\emptyset$ 
	for $j\neq \ell$. 
	%
	We define a $\setsymbols$-consistent $\Lvar^{n}$ model   
	$\propmodel = (\propdomain,\{ \propneigh_{i} \}_{i \in J}, \propassign)$ for $\phi$
	as follows.
	\begin{itemize}
		\item $\propdomain = \bigcup_{j = 1}^{m} \propdomain_j \cup \{w\}$, where $w$ is a new world.
		
		\item %Let $\ext{\cdot}$ be 
		Consider a function $\ext{\cdot}: {\sf sub}(\phi)\rightarrow \Pmc(\Wmc)$
		with $\ext{\psi}=\bigcup_{j = 1}^{m} \llbracket \psi \rrbracket^{\propmodel_j} \cup \ext{\psi}_0$ for all $\psi\in {\sf sub}(\phi)$, where %$I_i$ is as above for $1\leq i\leq n$, 
		%and
		$\ext{\cdot}_0: {\sf sub}(\varphi)\rightarrow  \Pmc(\{w\})$ is the function
		that assigns $\psi$ to $\{w\}$, if $\nu(\psi)=1$, 
		and to $\emptyset$, otherwise.
		% ($\Vmc_j$, for $1\leq j\leq m$, is as above).
		By construction, we have that $\ext{\neg \psi}=\propdomain\setminus \ext{\psi}$
		and $\ext{\psi_1\wedge \psi_2} =\ext{\psi_1}\cap \ext{\psi_2}$. 
		We define the assignment $\propassign$ as the function 
		$\propassign: \NPr(\varphi)\rightarrow \Pmc(\Wmc)$ satisfying 
		$\propassign(p_\elaxiom)=\ext{p_\elaxiom}$ for all $p_\elaxiom\in \NPr(\varphi)$. 
		
		\item It remains to define $\propneigh_i$, for $1 \leq i \leq n$.
		%We distinguish two cases. (i) 
		For $u\in \propdomain_j$,
		we define $\alpha\in\propneigh_i(u)$ if and only if 
		there is $\B_i\psi$ in ${\sf sub}(\phi)$ such that
		$\propmodel_j, u \models \B_i\psi$ and $\ext{\psi} = \alpha$;
		%(2) 
		and
		we define $\alpha\in\propneigh_i(w)$ if and only if 
		there is $\B_i\psi$ in ${\sf sub}(\phi)$ such that
		$\valuation(\B_i\psi)=1$ and $\ext{\psi} = \alpha$.
		% 
		Then if $\mathbf{C}\in\Lvar$,
		we close $\propneigh_i$ under intersection, 
		if $\mathbf{M}\in\Lvar$,
		we close $\propneigh_i$ under supersets,
		and if $\mathbf{N}\in\Lvar$,
		we extend $\propneigh_i(u)$ with $\propdomain$ for all
		$u\in\propdomain$,
		so that $\propmodel$ is a $\EC$, respectively a $\EM$,
		respectively a $\EN$, model.
	\end{itemize}
	
	We prove the following claim which ensures that
	$\propneigh_i$ is well-defined.
	
	\begin{claim}
		(i) For $u\in\propdomain_j$, if $\beta \in \Nmc_i(u)$ and 
		$\beta = \ext{\chi}$ for some 
		$\B_i\chi$ in ${\sf sub}(\phi)$,
		then $\propmodel_j,u\models\B_i\chi$.
		(ii) If $\beta \in \Nmc_i(w)$ and  $\beta = \ext{\chi}$ for some 
		$\B_i\chi$ in ${\sf sub}(\phi)$,
		then $\valuation(\B_i\chi) = 1$.
	\end{claim}
	\begin{proof}[Proof of Claim]
		We consider the case where $\mathbf{C},\mathbf{N}\in\Lvar$ and $\mathbf{M}\notin\Lvar$,
		for the other cases the proof can be easily adapted.
		
		(i) If $\beta \in \Nmc_i(u)$, then by definition 
		$\beta=\propdomain$, or
		$\beta = \bigcap_{\ell=1}^k \ext{\chi_\ell}$ for some $\B_i\chi_1, ..., \B_i\chi_k$ in ${\sf sub}(\phi)$ such that
		$\propmodel_j,u\models\bigwedge_{\ell=1}^k\B_i\chi_\ell$.
		%or $\beta=\propdomain$.
		If $\beta=\propdomain$, then $\ext{\chi}=\propdomain$,
		%thus $\llbracket \chi \rrbracket^{\propmodel_\ell} = \propdomain_\ell$ for all $\propmodel_\ell$,
		thus in particular 
		$\llbracket \chi \rrbracket^{\propmodel_j} = \propdomain_j$,
		and since $\propdomain_j\in\propneigh_{j_{i}}(u)$
		it holds
		$\propmodel_j, u \models \B_i\chi$.
		Otherwise
		$\ext{\chi} = \bigcap_{\ell=1}^k \ext{\chi_\ell}$,
		which implies 
		$\llbracket \chi \rrbracket^{\propmodel_j} =
		\bigcap_{\ell=1}^k\llbracket \chi_\ell \rrbracket^{\propmodel_j}$
		(because $\propdomain_j \cap \propdomain_k = \emptyset$ for  
		$k \neq j$).
		Since $\propmodel_j$ is a $\mathbf{EC}$ model,
		$\propmodel_j,u\models\B_i\bigwedge_{\ell=1}^k\chi_\ell$,
		then 
		%$\llbracket \bigwedge_{\ell=1}^k\chi_\ell \rrbracket^{\propmodel_j} =
		%\bigcap_{\ell=1}^k\llbracket \chi_\ell \rrbracket^{\propmodel_j} =
		%\llbracket \chi \rrbracket^{\propmodel_j}
		%\in\propneigh_{j_{i}}(u)$, 
		$\llbracket \bigwedge_{\ell=1}^k\chi_\ell \rrbracket^{\propmodel_j} 
		\in\propneigh_{j_{i}}(u)$,
		where
		$\llbracket \bigwedge_{\ell=1}^k\chi_\ell \rrbracket^{\propmodel_j} =
		\bigcap_{\ell=1}^k\llbracket \chi_\ell \rrbracket^{\propmodel_j} =
		\llbracket \chi \rrbracket^{\propmodel_j}$,
		therefore
		$\propmodel_j,u\models\B_i\chi$.
		
		(ii) If $\beta \in \Nmc_i(u)$, then by definition 
		$\beta=\propdomain$, or
		$\beta = \bigcap_{\ell=1}^k \ext{\chi_\ell}$ for some $\B_i\chi_1, ..., \B_i\chi_k$ in ${\sf sub}(\phi)$ such that
		$\valuation(\chi_\ell) = 1$ for all $1 \leq \ell \leq k$.
		If $\beta=\propdomain$, then $\ext{\chi} = \propdomain$,
		thus $\ext{\chi}_0 = \{w\}$, that is $\valuation(\chi) = 1$.
		By contradiction, suppose that 
		$\valuation(\B_i\chi) = 0$.
		Then by $(N)$, there are a $\setsymbols$-consistent $\Lvar^{n}$ model 
		$\propmodel_{\chi}$  and a world  $w_{\chi}$ such that 
		$\propmodel_{\chi}, w_{\chi} \not\models \chi$.
		One such model is enumerated among 
		$\propmodel_1, ..., \propmodel_m$, let it be $\propmodel_o$.
		Then $\llbracket \chi \rrbracket^{\propmodel_o} \not= \propdomain_o$, 
		thus $\ext{\chi} \not= \propdomain$, giving a contradiction.
		Therefore $\valuation(\B_i\chi) = 1$.
		If instead $\beta\not=\propdomain$,
		then $\ext{\chi} = \bigcap_{\ell=1}^k \ext{\chi_\ell}$.
		Suppose that $\valuation(\chi) = 0$.
		By the hypothesis of the lemma,
		there are a $\setsymbols$-consistent model $\propmodel_{\Lambda,\chi}$ and a world 
		$w_{\Lambda,\chi}$ such that 
		$\propmodel_{\Lambda,\chi},w_{\Lambda,\chi}\models(\bigwedge^{k}_{\ell=1}\chi_\ell\wedge\neg\chi) \vee \bigvee^{k}_{\ell=1} (\neg\chi_\ell\wedge\chi)$,
		where $\Lambda = \{\B_i\chi_1,...,\B_i\chi_k\}$.
		Then one such model is enumerated among 
		$\propmodel_1, ..., \propmodel_m$, let it be $\propmodel_o$.
		Thus 
		$\llbracket \chi \rrbracket^{\propmodel_o} \not=
		\llbracket \bigwedge_{\ell=1}^k\chi_\ell \rrbracket^{\propmodel_o}$, where
		$\llbracket \bigwedge_{\ell=1}^k\chi_\ell \rrbracket^{\propmodel_\ell} =
		\bigcap_{\ell=1}^k\llbracket \chi_\ell \rrbracket^{\propmodel_\ell}$.
		Since  $\propdomain_j \cap \propdomain_k = \emptyset$ for  
		$k \neq j$,  
		this implies $\ext{\chi} \not = \bigcap_{\ell=1}^k \ext{\chi_\ell}$,
		giving a contradiction.
		Therefore $\valuation(\B_i\chi) = 1$. 
	\end{proof}
	
	\begin{claim}
		For all $\psi\in{\sf sub}(\phi)$,
		%and all $u\in\propdomain$, $u \in \llbracket \psi \rrbracket^{\propmodel}$ if and only if $u\in \ext{\psi}$.
		$\llbracket \psi \rrbracket^{\propmodel} = \ext{\psi}$.
	\end{claim}
	\begin{proof}[Proof of Claim]
		By induction on the structure of formulas.
		For $p_\elaxiom\in \NPr(\varphi)$,
		$\llbracket p_\elaxiom \rrbracket^{\propmodel}=\ext{p_\elaxiom}$
		by definition of $\propassign$.
		For boolean connectives, the claim follows immediately from the inductive hypothesis and the fact
		that $\ext{\neg \chi} =\propdomain\setminus \ext{\chi}$ and $\ext{\chi_1\wedge \chi_2} = \ext{\chi_1}\cap \ext{\chi_2}$.
		%
		Suppose  that $u\in \ext{\B_i\chi}$.  
		Then, either $u=w$ and $\nu(\B_i\chi)=1$
		or $u\in \propdomain_j$ and $\propmodel_j,u\models\B_i\chi$. 
		By definition of $\propneigh_i$, in either case 
		we have that $\ext{\chi}\in \propneigh_i(u)$. By inductive hypothesis,
		$\llbracket \chi \rrbracket^{\propmodel} = \ext{\chi}$, it follows that $\propmodel,u\models \B_i\chi$, 
		that is, $u\in \llbracket \B_i\chi \rrbracket^{\propmodel}$. 
		%
		Suppose now that $u\in \llbracket \B_i\chi \rrbracket^{\propmodel}$, 
		that is, $\propmodel,u\models \B_i\chi$, or, equivalently,  $\llbracket \chi \rrbracket^{\propmodel}\in \propneigh_i(u)$.
		By inductive hypothesis,
		$\llbracket \chi \rrbracket^{\propmodel} = \ext{\chi}$, then
		by the previous claim, if $u=w$, then $\valuation(\B_i\chi)=1$,
		and if $u\in \propdomain_j$, then $\propmodel_j,u\models\B_i\chi$. 
		By definition of $\ext{\cdot}$, in either case we have that $u\in \ext{\B_i\psi}$. 
	\end{proof}
	
	
	\begin{claim}
		%  If $\mathbf{X}\in\Lvar$, for $\mathbf{X}\in\{\mathbf{M,C,N,T,P,Q,D}\}$,
		%  then $\propmodel$ is a $\mathbf{EX^*}$ model.
		For $\mathbf{X}\in\{\mathbf{M,C,N,T,P,Q,D}\}$,
		if $\mathbf{X}\in\Lvar$,
		then $\propmodel$ satisfies the $\mathbf{X}$-condition.
	\end{claim}
	\begin{proof}[Proof of Claim]
		For $\mathbf{X}\in\{\mathbf{M,C,N}\}$, 
		% that $\propmodel$ 
		% %is a $\mathbf{EX^*}$ model 
		% satisfies the $X$-condition
		% if $\mathbf{X}\in\Lvar$
		% is an immediate consequence of 
		the claim follows immediately from
		the definition of $\propneigh_i$.
		We consider the other cases,
		assuming %as before 
		that 
		% $\mathbf{C},\mathbf{N}\in\mathbf{L}$ and $\mathbf{M}\notin\mathbf{L}$
		% ($\mathbf{C}\notin\mathbf{L}$, $\mathbf{N}\notin\mathbf{L}$, or $\mathbf{M}\in\mathbf{L}$ 
		% the proof can be easily adapted).
		$\mathbf{C}\in\Lvar$ and $\mathbf{M}\notin\Lvar$
		(for $\mathbf{C}\notin\Lvar$ or $\mathbf{M}\in\Lvar$ the proof can be easily adapted).
		
		\begin{itemize}
			\item[$(\mathbf{T})$]  Suppose that $\alpha\in\propneigh_i(u)$.
			Then $\alpha = \propdomain$ (if $\mathbf{N}\in\Lvar$), 
			which implies $u\in\alpha$,
			or $\alpha = \bigcap_{\ell = 1}^k \ext{\psi_\ell}$ for $\B_i\psi_1, ..., \B_k\psi \in{\sf sub}(\phi)$.
			If $u\in\propdomain_j$, then $\propmodel_j, u \models \B_i\psi_1 \land ... \land \B_i\psi_k$. 
			Since $\propmodel_j$ is a $\mathbf{ET}$ model,
			$\propmodel_j\models \B_i\chi\to\chi$, thus $\propmodel_j, u \models \psi_1 \land ... \land \psi_k$,
			that is $u\in\llbracket \psi_\ell \rrbracket^{\propmodel_j}$ for all $1\leq \ell \leq k$.
			It follows $u\in \bigcap_{\ell = 1}^k \ext{\psi_\ell} = \alpha$. 
			If instead $u = w$, then $\valuation(\B_i\psi_\ell) = 1$  for all $1\leq \ell \leq k$.
			By the hypothesis of the proposition, $\valuation(\psi) = 1$ for all $1\leq \ell \leq k$, 
			thus $\ext{\psi_\ell}_0 = \{w\}$  for all $1\leq \ell \leq k$.
			therefore $u = w \in \bigcap_{\ell = 1}^k \ext{\psi_\ell} = \alpha$. 
			
			%[Proof without C]
			% If $u\in\propdomain_j$, then $\alpha = \ext{\psi}$ for a $\B_i\psi\in{\sf sub}(\prop{\varphi})$ such that
			% $\propmodel_j, u \models \B_i\psi$. Since $\propmodel_j$ is a $\mathbf{ET^*}$ model,
			% $\propmodel_j\models \B_i\psi\to\psi$, thus $\propmodel_j, u \models \psi$,
			% that is $u\in\llbracket \psi \rrbracket^{\propmodel_j}$.
			% It follows $u\in \ext{\psi} = \alpha$.
			% If instead $u = w$, then $\alpha = \ext{\psi}$ for a $\B_i\psi\in{\sf sub}(\prop{\varphi})$ such that
			% $\valuation(\B_i\psi) = 1$.
			% By the hypothesis of the proposition, $\valuation(\psi) = 1$, thus $\ext{\psi}_0 = \{w\}$,
			% therefore $u = w \in \ext{\psi} = \alpha$.
			
			\item[$(\mathbf{P})$]  Assume that $\emptyset\in\propneigh_i(u)$.
			Then $\emptyset = \bigcap_{\ell = 1}^k \ext{\psi_\ell}$ for $\B_i\psi_1, ..., \B_k\psi \in{\sf sub}(\phi)$.
			If $u\in\propdomain_j$, then $\propmodel_j, u \models \B_i\psi_1 \land ... \land \B_i\psi_k$,
			that is $\llbracket \psi_\ell \rrbracket^{\propmodel_j}\in\propneigh_{j_i}(u)$ for all $1\leq \ell \leq k$.
			By the $\mathbf{C}$-condition, $\bigcap_{\ell = 1}^k \llbracket \psi_\ell \rrbracket^{\propmodel_j} \in\propneigh_{j_i}(u)$,
			and by construction of $J$, 
			$\bigcap_{\ell = 1}^k \llbracket \psi_\ell \rrbracket^{\propmodel_j} = \emptyset$,
			contradicting the fact that $\propmodel_j$ is a $\mathbf{EP}$ model.
			If instead $u = w$, then $\valuation(\B_i\psi_\ell) = 1$  for all $1\leq \ell \leq k$.
			By item $(\mathbf{P})$ above, there are  $\propmodel_{\Psi}$ and  $w_{\Psi}$ such that 
			$\propmodel_{\Psi}, w_{\Psi} \models \psi_1\land...\land\psi_k$.
			One such model is enumerated among 
			$\propmodel_1, ..., \propmodel_m$, let it be $\propmodel_o$.
			Then $\llbracket \bigwedge_{\ell=1}^k\psi_\ell \rrbracket^{\propmodel_o} =
			\bigcap_{\ell = 1}^k \llbracket \psi_\ell \rrbracket^{\propmodel_o} \not=\emptyset$,
			thus $\bigcap_{\ell = 1}^k \ext{\psi_\ell}  \not=\emptyset$. 
			In either case $\emptyset\notin\propneigh_i(u)$.
			
			\item[$(\mathbf{Q})$]  Suppose that $\alpha\in\propneigh_i(u)$.
			Then $\alpha = \bigcap_{\ell = 1}^k \ext{\psi_\ell}$ for $\B_i\psi_1, ..., \B_k\psi \in{\sf sub}(\phi)$.
			If $u\in\propdomain_j$, then $\propmodel_j, u \models \B_i\psi_1 \land ... \land \B_i\psi_k$,
			that is $\llbracket \psi_\ell \rrbracket^{\propmodel_j}\in\propneigh_{j_i}(u)$ for all $1\leq \ell \leq k$.
			Since $\propmodel_j$ is a $\mathbf{EQ}$ model, $\llbracket \psi_\ell \rrbracket^{\propmodel_j} \not = \propdomain_j$,
			then $\bigcap_{\ell = 1}^k \ext{\psi_\ell}\not=\propdomain$.
			%
			If instead $u = w$, then $\valuation(\B_i\psi_\ell) = 1$  for all $1\leq \ell \leq k$.
			By item $(\mathbf{Q})$ above, there are  $\propmodel_{\Psi}$ and  $w_{\Psi}$ such that 
			$\propmodel_{\Psi}, w_{\Psi} \models \neg\psi_1\lor ... \lor\neg\psi_k$.
			One such model is enumerated among 
			$\propmodel_1, ..., \propmodel_m$, let it be $\propmodel_o$.
			Then $\llbracket \neg\psi_\ell \rrbracket^{\propmodel_o} \not=\emptyset$ for some $\B_i\psi_\ell$,
			that is $\llbracket \psi_\ell \rrbracket^{\propmodel_o} \not=\propdomain$,
			therefore $\bigcap_{\ell = 1}^k \llbracket \psi_\ell \rrbracket^{\propmodel} \not=\propdomain$.
			In either case $\alpha\not=\propdomain$, that is $\propdomain\notin\propneigh_i(u)$.
			
			\item[$(\mathbf{D})$]  Suppose that $\alpha,\beta\in\propneigh_i(u)$.
			Then $\alpha = \bigcap_{\ell = 1}^k \ext{\psi_\ell}$ for $\B_i\psi_1, ..., \B_i\psi_k \in{\sf sub}(\phi)$, and
			$\beta = \bigcap_{{\ell'} = 1}^k \ext{\chi_{\ell'}}$ for $\B_i\chi_1, ..., \B_i\chi_h \in{\sf sub}(\phi)$.
			If $u\in\propdomain_j$, then 
			$\propmodel_j, u \models \B_i\psi_1 \land ... \land \B_i\psi_k \land \B_i\chi_1 \land ... \land \B_i\chi_h$.
			Since $\propmodel_j$ is a $\mathbf{EC}$ model,
			$\propmodel_j, u\models \B_i \bigwedge_{\ell=1}^k\psi_\ell \land \B_i\bigwedge_{{\ell'}=1}^h\chi_{\ell'}$, and
			since $\propmodel_j$ is a $\mathbf{ED}$ model,
			$\propmodel_j\not\models \bigwedge_{\ell=1}^k\psi_\ell \leftrightarrow \neg\bigwedge_{{\ell'}=1}^h\chi_{\ell'}$,
			% That is, there is $v$ in $\propdomain_j$ such that 
			% $\propmodel_j, v \models (\bigwedge_{j=1}^k\psi_j \land \bigwedge_{{\ell'}=1}^h\chi_{\ell'}) \lor (\neg\bigwedge_{j=1}^k\psi_j \land \neg\bigwedge_{{\ell'}=1}^h\chi_{\ell'})$
			that is 
			$\bigcap_{\ell = 1}^k \llbracket \psi_\ell \rrbracket^{\propmodel_j} =
			\llbracket \bigwedge_{\ell=1}^k\psi_\ell \rrbracket^{\propmodel_j} \not=
			\llbracket \neg\bigwedge_{{\ell'}=1}^h\chi_{\ell'} \rrbracket^{\propmodel_j} =
			\propdomain_j \setminus \bigcap_{{\ell'}=1}^h \llbracket \chi_{\ell'} \rrbracket^{\propmodel_j}$.
			%
			If instead $u = w$, 
			then  $\valuation(\B_i\psi_\ell)=\valuation(\B_i\chi_{\ell'})=1$ for all $1\leq \ell \leq k$,  $1\leq \ell' \leq h$.
			By item $(\mathbf{D})$ above, there are
			$\propmodel_{\Psi,\Lambda}$ and $w_{\Psi,\Lambda}$ such that
			$\propmodel_{\Psi,\Lambda}, w_{\Psi,\Lambda} \models (\bigwedge^{k}_{\ell=1}\psi_\ell \land \bigwedge^{h}_{\ell'=1}\chi_{\ell'}) \vee (\neg(\bigwedge^{k}_{\ell=1}\psi_\ell) \land \neg(\bigwedge^{h}_{\ell'=1}\chi_{\ell'}))$.
			One such model is enumerated among 
			$\propmodel_1, ..., \propmodel_m$, let it be $\propmodel_o$.
			Then $\bigcap_{\ell = 1}^k \llbracket \psi_\ell \rrbracket^{\propmodel_o} =
			\llbracket \bigwedge_{\ell=1}^k\psi_\ell \rrbracket^{\propmodel_o} \not=
			\llbracket \neg\bigwedge_{{\ell'}=1}^h\chi_{\ell'} \rrbracket^{\propmodel_o} =
			\propdomain_o \setminus \bigcap_{{\ell'}=1}^h \llbracket \chi_{\ell'} \rrbracket^{\propmodel_o}$.
			%
			Thus in either case, 
			$\bigcap_{\ell = 1}^k \ext{\psi_\ell} \not=\propdomain \setminus \bigcap_{{\ell'} = 1}^k \ext{\chi_{\ell'}}$,
			that is, $\alpha\not=\propdomain\setminus\beta$.\qedhere
		\end{itemize} 
	\end{proof}
	
	
	\begin{claim}
		$\propmodel$ is $\setsymbols$-consistent.
	\end{claim}
	\begin{proof}[Proof of Claim]
		$\nu$, used to construct the assignment 
		related to $w$, is $\setsymbols$-consistent and 
		the models $\propmodel_1,\ldots,\propmodel_m$, used to define 
		the remaining worlds in $\Wmc$, are all $\setsymbols$-consistent. 
	\end{proof}
	
	Finally, since $\valuation(\phi)=1$, we have that $w\in \ext{\phi}$, 
	and consequently $\propmodel, w \models \phi$.
	Given that $\propmodel$ is a $\setsymbols$-consistent $\Lvar^{n}$ model, this concludes the proof.
\end{proof}





















\begin{algorithm}[t]
	\KwIn{$L$,  $\setsymbols$, and an $\MLn$ formula $\phi$ built from $\setsymbols$.}
	\KwOut{$\mathsf{satisfiable}$, if $\psi$ is  satisfiable in a $\setsymbols$-consistent $L^n$ model; $\mathsf{unsatisfiable}$, otherwise.}
	\BlankLine
	%	$r:=  \mathsf{unsatisfiable}$\;
	
	\For{each $\setsymbols$-consistent valuation $\valuation$ for $\phi$}{
		\uIf{$\mathsf{Check}(L,\setsymbols,\valuation,\phi)=1$}{
			\uIf{$L\cap\{ \mathbf{N},\mathbf{T},\mathbf{P},\mathbf{Q}\} \neq \emptyset$}{
				\uIf{$\mathsf{CheckNTPQ}(L,\setsymbols,\valuation,\phi)=1$}{
					\uIf{$\mathbf{D}\notin L$}{
						\Return $\mathsf{satisfiable}$\;
					}\uElseIf{$\mathsf{CheckD}(L,\setsymbols,\valuation,\phi){=}1$}
						{\Return $\mathsf{satisfiable}$\;}
				}
			}\uElseIf{$\mathbf{D}\in L$, $\mathsf{CheckD}(L,\setsymbols,\valuation,\phi){=}1$
		}{\Return $\mathsf{satisfiable}$\;} 
		}
		
	}
	%}
\BlankLine
\Return $\mathsf{unsatisfiable}$\;
%	\uIf{$\T$ contains a clash}{\Return $\mathsf{unsatisfiable}$\;} 
%	\Else{\Return $\mathsf{satisfiable}$\;}
%\caption{$\LnALC$ tableau algorithm for $\p$}
\caption{$\mathsf{Sat}$}
%		: Decision procedure %$sat(\p)$
%		for  the $\LnALCg$  formula satisfiability problem on varying domain neighbourhood models}
\label{alg:propSAT}
\end{algorithm}
















\begin{algorithm}[t]
	\KwIn{$L$, $\setsymbols$, a $\setsymbols$-consistent valuation $\valuation$, and an $\MLn$ formula $\phi$ built from $\setsymbols$.}
	\KwOut{$\mathsf{1}$, if $\valuation$ satisfies the conditions of Lemma~\ref{lem:proplemmaL}; $0$, otherwise.}
	\BlankLine
	%	$r:=  \mathsf{unsatisfiable}$\;
	\uIf{$\mathbf{C} \in \Lvar$}{
		$\boldsymbol{\kappa}:= | {\sf sub}({\phi}) |$\;
	}
	\uElse{$\boldsymbol{\kappa}:=1$}
	\BlankLine
	%\For{each $\setsymbols$-consistent valuation $\valuation$ for $\phi$}{
		\For{all $1\leq k\leq \boldsymbol{\kappa}$}{
			\For{ $\B_i\psi_1, \dots, \B_i\psi_k, \B_i\chi\in{\sf sub}(\phi)$,
				with $\valuation(\B_i\psi_j)=1$ for all $1\leq j \leq k$,
				%$\B_i\chi\in{\sf sub}(\prop{\varphi})$, 
				and $\valuation(\B_i\chi)=0$}{ 
				\uIf{$\mathbf{M}\in L$}{ 
					\uIf{$\mathsf{Sat}(L,\setsymbols,\bigwedge^{k}_{j=1}\psi_j\wedge\neg\chi)= \mathsf{unsatisfiable}$}{\Return $0$\;} 
				}
				\uElseIf{ $\mathsf{Sat}(L,\setsymbols,(\bigwedge^{k}_{j=1}\psi_j\wedge\neg\chi)\vee (\bigvee^{k}_{j=1} (\neg\psi_j\wedge\chi)))=\mathsf{unsatisfiable}$\;}
				{\Return $0$\;}
			}		
		}
		\Return $1$\;
 		\BlankLine
		%	\uIf{$\T$ contains a clash}{\Return $\mathsf{unsatisfiable}$\;} 
		%	\Else{\Return $\mathsf{satisfiable}$\;}
		%\caption{$\LnALC$ tableau algorithm for $\p$}
		\caption{$\mathsf{Check}$}
		%		: Decision procedure %$sat(\p)$
		%		for  the $\LnALCg$  formula satisfiability problem on varying domain neighbourhood models}
	\label{alg:prop1}
\end{algorithm}











\begin{algorithm}[t]
	\KwIn{$L$, $\setsymbols$, a $\setsymbols$-consistent valuation $\valuation$, and an $\MLn$ formula $\phi$ built from $\setsymbols$.}
	\KwOut{$\mathsf{1}$, if $\valuation$ satisfies the conditions of  Lemma~\ref{lem:proplemmaL}; $0$, otherwise.}
	\BlankLine
	%	$r:=  \mathsf{unsatisfiable}$\;
	\uIf{$\mathbf{C} \in \Lvar$}{
		$\boldsymbol{\kappa}:= | {\sf sub}({\phi}) |$\;
	}
	\uElse{$\boldsymbol{\kappa}:=1$}
	\BlankLine
	%\For{each $\setsymbols$-consistent valuation $\valuation$ for $\phi$}{
		 
		 	%\For{all $1\leq k,h\leq \boldsymbol{\kappa}$}{
		\For{all $1\leq k\leq \boldsymbol{\kappa}$}{
				
				\uIf{$\mathbf{N}\in L$}{	
					\For{ $\B_i\psi\in{\sf sub}(\phi)$  
						with $\valuation(\B_i\psi)=0$}{	
						\uIf{$\mathsf{Sat}(L,\setsymbols,\neg \psi)= 	\mathsf{unsatisfiable}$}{\Return $0$\;} 
					}
					\Return $1$\;	
				}
				
			\uIf{$\mathbf{T}\in L$}{
				\For{$\B_i\psi\in{\sf sub}(\phi)$ with  
					$\valuation(\B_i\psi)=1$}{
					\uIf{$\valuation(\psi)=0$}{
						\Return $0$\;}
				}
			}
			\uIf{$\mathbf{P}\in L$}{
				\For{$\B_i\psi_1, \dots, \B_i\psi_k\in{\sf sub}(\phi)$ with
					$\valuation(\B_i\psi_j)=1$ for all $1\leq j \leq k$}{
					\uIf{$\mathsf{Sat}(L,\setsymbols,\bigwedge^{k}_{j=1}\psi_j)= \mathsf{unsatisfiable}$}{\Return $0$\;} 
				}
			}
			\uIf{$\mathbf{Q}\in L$}{
				\For{$\B_i\psi_1, \dots, \B_i\psi_k\in{\sf sub}(\phi)$ with 
					$\valuation(\B_i\psi_j)=1$ for all $1\leq j \leq k$}{
					\uIf{$\mathsf{Sat}(L,\setsymbols,\bigvee^{k}_{j=1}\neg\psi_j)= \mathsf{unsatisfiable}$}{\Return $0$\;} 
				}
			}
		}
		 		%\item[($\mathbf{D}$)] if $\B_i\psi_1, \dots, \B_i\psi_k, \B_i\chi_1, \dots, \B_i\chi_h\in{\sf sub}(\phi)$,
			%$\valuation(\B_i\psi_j)=1$ for all $1\leq j \leq k$, and
			%$\valuation(\B_i\chi_\ell)=1$ for all $1\leq \ell \leq h$, 
			%		$\valuation(\B_i\psi_j)=\valuation(\B_i\chi_\ell)=1$ for all $1\leq j \leq k$,  $1\leq \ell \leq h$,
			%then $(\bigwedge^{k}_{j=1}\psi_j \land \bigwedge^{h}_{\ell=1}\chi_\ell) \vee \boldsymbol\eta$
			%is satisfied in a $\setsymbols$-consistent $\Lvar^{n}$ model,
			%where
			%\[
			%\boldsymbol\eta =
			%\begin{cases}
			% 	\falseprop, & \text{if $\mathbf{M}\in\Lvar$} \\
			%	\neg(\bigwedge^{k}_{j=1}\psi_j) \land \neg(\bigwedge^{h}_{\ell=1}\chi_\ell), & \text{if $\mathbf{M}\not\in\Lvar$}
			%\end{cases}.
			%\]
			\Return $1$\;
		
		
		\BlankLine
		%	\uIf{$\T$ contains a clash}{\Return $\mathsf{unsatisfiable}$\;} 
		%	\Else{\Return $\mathsf{satisfiable}$\;}
		%\caption{$\LnALC$ tableau algorithm for $\p$}
		\caption{$\mathsf{CheckNTPQ}$}
		%		: Decision procedure %$sat(\p)$
		%		for  the $\LnALCg$  formula satisfiability problem on varying domain neighbourhood models}
	\label{alg:prop}
\end{algorithm}









\begin{algorithm}[t]
	\KwIn{$L$, $\setsymbols$, a $\setsymbols$-consistent valuation $\valuation$, and an $\MLn$ formula $\phi$ built from $\setsymbols$.}
	\KwOut{$\mathsf{1}$, if $\valuation$ satisfies the conditions of Lemma~\ref{lem:proplemmaL}; $0$, otherwise.}
	\BlankLine
%	$r:=  \mathsf{unsatisfiable}$\;
	\uIf{$\mathbf{C} \in \Lvar$}{
$\boldsymbol{\kappa}:= | {\sf sub}({\phi}) |$\;
	}
 \uElse{$\boldsymbol{\kappa}:=1$}
 \BlankLine
	%\For{each $\setsymbols$-consistent valuation $\valuation$ for $\phi$}{

		\For{all $1\leq k,h\leq \boldsymbol{\kappa}$}{

		\For{$\B_i\psi_1, \dots, \B_i\psi_k, \B_i\chi_1, \dots, \B_i\chi_h\in{\sf sub}(\phi)$ with
			$\valuation(\B_i\psi_j)=1$, for all $1\leq j \leq k$, and
			$\valuation(\B_i\chi_\ell)=1$, for all $1\leq \ell \leq h$}{
			\uIf{$\mathbf{M}\in L$}{
				\uIf{$\mathsf{Sat}(L,\setsymbols,(\bigwedge^{k}_{j=1}\psi_j \land \bigwedge^{h}_{\ell=1}\chi_\ell))= \mathsf{unsatisfiable}$}{\Return $0$\;} 
			}
			\uElseIf  {$\mathsf{Sat}(L,\setsymbols,(\bigwedge^{k}_{j=1}\psi_j \land \bigwedge^{h}_{\ell=1}\chi_\ell)\vee (\neg(\bigwedge^{k}_{j=1}\psi_j) \land \neg(\bigwedge^{h}_{\ell=1}\chi_\ell)))= \mathsf{unsatisfiable}$}{\Return $0$\;}
			
		}
	}
	\Return $1$\;
	\BlankLine

	\caption{$\mathsf{CheckD}$}
%		: Decision procedure %$sat(\p)$
%		for  the $\LnALCg$  formula satisfiability problem on varying domain neighbourhood models}
	\label{alg:propD}
\end{algorithm}























\Satfragvardomexp*
\begin{proof}
Soundness and completeness of Algorithm~\ref{alg:propSAT} is given by Lemmas~\ref{lem:propL} and~\ref{lem:proplemmaL}. 
 We argue   that Algorithm~\ref{alg:prop} terminates in exponential time. 
 Since the \ALC satisfiability check is in exponential time, one can compute
 in exponential time (in the size of $\setsymbols$) all valuations $\valuation$ which are $\setsymbols$-consistent. The number of iterations in Line 1 of  Algorithm~\ref{alg:propSAT}
 is bounded by $2^{|\setsymbols|}$. It remains to argue that each iteration takes exponential time. Suppose $\prop{\varphi}$ is the original formula we want to check satisfiability.
 Since each iteration calls   the functions $\mathsf{Check}$, $\mathsf{CheckNTPQ}$, 
 $\mathsf{CheckD}$ and these functions can make recursive calls (to $\mathsf{Sat}$), we need to argue that (1) the number of recursive calls in exponentially bounded and (2) 
 the number of steps inside each function is also exponentially bounded.
 Regarding the latter, we argue that   the number of iterations of the ``for'' loops inside
 $\mathsf{Check}$, $\mathsf{CheckNTPQ}$, and $\mathsf{CheckD}$ is exponentially bounded by the number of subformulas of the formula given as input to each function (and each such formula has size linear in the size of $\prop{\varphi}$). If  the total number of recursive calls is exponentially bounded then Point (2) holds.
 So it remains to argue about Point (1). 
  Consider a computation tree where each node corresponds to a recursive call to  $\mathsf{Sat}$ and the parent relation in the tree is defined by the recursive calls.
 Since each recursive call reduces the number of nested epistemic 
 operators of the original formula $\prop{\varphi}$, any nested sequence  of   recursive calls
 is polynomial in the size of $\prop{\varphi}$. This means that the depth of such tree is polynomial in the size of $\prop{\varphi}$ and, since the number of children of each node is  exponentially bounded (see Point(2)), the total number of nodes of the tree is exponential in the size of $\prop{\varphi}$. We have thus shown that the number of recursive calls in exponentially bounded.
 As satisfiability in $\ALC$ is $\ExpTime$-hard, our upper bound is tight.
\end{proof}






\section{Proofs for Section~\ref{sec:reasoncondom}}






%\subsection{Proofs for Section~\ref{sec:relation}}

\Theoremcomplealc*
%
\begin{proof}
This theorem is a consequence of the following claim, and 
the complexity of formula satisfiability in \KnALC{3n} constant domain relational models~\cite{KraWol,GasHer,GabEtAl03}.
\begin{claim} %{lemma}{Theoremclassicalred}\label{theor:classicalred}
%\nb{M: Change to lemma? Merge to next Th. as a claim?}
The $\EnALC{n}$ formula satisfiability problem on constant domain neighbourhood models can be reduced in polynomial time to the \KnALC{3n} formula satisfiability problem on constant domain relational models.
\end{claim}
%\Theoremclassicalred*
%
%\nb{M: Spostare def. $\Mmf$ fuori dal lemma?}
\begin{proof}[Proof of Claim]
Consider an \MLALC{n} formula $\p$
%satisfiable over N-models,
s.t.
$\Mmc, w \mdl \p$, for some constant domain neighbourhood model $\Mmc = (\Fmc, \Delta, \Int)$ with $\Fmc = (\Wmc, \{ \Nmc_{i} \}_{i \in J})$
%based on a neighborhood frame $\Fmc = (\Wmc, \Nmc)$ and having domain $\Delta$,
and some $w\in\Wmc$.
%\nb{O:changed}
We define a relational frame
$\Fmf = (W, \{ R_{i_{1}}, R_{i_{2}}, R_{i_{3}} \}_{i \in J})$
%$\Fmf = (W, \{ R_{i_{j}} \}_{j \in [1, 3]})$
%$\Fmf = (W, \relations_{i_1}, \relations_{i_2}, \relations_{i_3})$
and an $\MLALC{3n}$ relational model 
$\Mmf = (\Fmf, \Delta, I)$
%$\Mmf = (\Fmf, I)$ based on $\Fmf$ and having domain $\Delta$, s.t.:
such that:
%\nb{M: Aggiungere? Cambiare? Cf. \cite{GasHer}, p.15.}
%\nb{O: say that 0,1 are used to ensure that the sets are disjoint?}
	\begin{itemize}
		\item $W = \{ (w, 0) \mid w \in \Wmc \} \cup \{ (\alpha, 1) \mid \alpha \in \bigcup_{v \in \Wmc} \Nmc_{i}(v) \}$
%		\item $W = W \cup \bigcup_{w \in W} \Nmc_{i}(w)$;
%		{\color{red}{$W \cap \bigcup_{w \in W} \Nmc_{i}(w) = \eset$?}};
		\item $\relations_{i_1} = \{ ((w, 0), (\alpha, 1)) \mid \alpha \in \Nmc_{i}(w)\}$;
		\item $\relations_{i_2} = \{ ((\alpha, 1), (w, 0)) \mid w \in \alpha \}$
%		\nb{M: Aggiungere $w \in W$ per chiarezza, anche se superfluo?}
%		\item $\relations_{i_2} = \{ (U, w) \in W \times W \mid w \in W, U \in \Nmc_{i}(w) \colon w \in U \}$ \\
%		{\color{red}{$\relations_{i_2} = \{ (U, w) \in W \times W \mid U \in \bigcup_{v \in W} N(v), w \in W \colon w \in U \}$?}};
		\item $\relations_{i_3} = \{  ((\alpha, 1), (w, 0)) \mid w \not \in \alpha \}$ 
%		\item $\relations_{i_3} =  \{ (U, w) \in W \times W \mid w \in W, U \in \Nmc_{i}(w) \colon w \not \in U \}$; \\
%				{\color{red}{$\relations_{i_3} = \{ (U, w) \in W \times W \mid U \in \bigcup_{w \in W} \Nmc_{i}(w), w \in W \colon w \not \in U \}$?}};
		\item for every $(w, 0) \in W$, $I_{(w, 0)} = \Imc_{w}$; for every $(\alpha, 1) \in W$, $X^{I_{(\alpha, 1)}} = \eset$, for all $X \in \NC \cup \NR$, and $a^{I_{(\alpha, 1)}} = a^{\Int}$, for all $a \in \NI$.
%		\nb{M: Cambiare? Opzioni: \\ (a) tutto falso in $U$ $\to$ diverse def. semantiche; \\ (b) non definito in $U$ $\to$ interpret. $I$ funzione parziale? \\ (c) lasciare cosi}
	\end{itemize}
	%
The pairs $(w, 0), (\alpha, 1)$ are used to ensure that $W$ is the disjoint union of the sets of worlds $w$ and subsets $\alpha$ of $\Wmc$.

%	\begin{itemize}
%		\item $W = \Wmc \cup \{ (\alpha, 0) \mid \alpha \in \bigcup_{v \in W} N(v) \}$
%%		\item $W = W \cup \bigcup_{w \in W} \Nmc_{i}(w)$;
%%		{\color{red}{$W \cap \bigcup_{w \in W} \Nmc_{i}(w) = \eset$?}};
%		\item $\relations_{i_1} = \{ (w,  (\alpha, 0)) \mid w \in W, \alpha \in \Nmc_{i}(w)\}$;
%		\item $\relations_{i_2} = \{ ((\alpha, 0), w) \mid \alpha \in \bigcup_{v \in W} N(v), w \in \alpha \}$
%		\nb{Aggiungere $w \in W$ per chiarezza, anche se superfluo?}
%%		\item $\relations_{i_2} = \{ (U, w) \in W \times W \mid w \in W, U \in \Nmc_{i}(w) \colon w \in U \}$ \\
%%		{\color{red}{$\relations_{i_2} = \{ (U, w) \in W \times W \mid U \in \bigcup_{v \in W} N(v), w \in W \colon w \in U \}$?}};
%		\item $\relations_{i_3} = \{  ((\alpha, 0), w) \mid \alpha \in \bigcup_{v \in W} N(v), w \in W, w \not \in \alpha \}$ 
%%		\item $\relations_{i_3} =  \{ (U, w) \in W \times W \mid w \in W, U \in \Nmc_{i}(w) \colon w \not \in U \}$; \\
%%				{\color{red}{$\relations_{i_3} = \{ (U, w) \in W \times W \mid U \in \bigcup_{w \in W} \Nmc_{i}(w), w \in W \colon w \not \in U \}$?}};
%		\item for every $w \in \Wmc$, $I_{w} = \Imc_{w}$; for every $\alpha \in W \setminus \Wmc$, $X^{I(\alpha)} = \eset$, for all $X \in \NC \cup \NR$, and $a^{I(\alpha)} = a^{I}$, for all $a \in \NI$. \\
%%		\nb{M: Cambiare? Opzioni: \\ (a) tutto falso in $U$ $\to$ diverse def. semantiche; \\ (b) non definito in $U$ $\to$ interpret. $I$ funzione parziale? \\ (c) lasciare cosi}
%	\end{itemize}

Firstly we show, by induction on the structure of concepts $C$, that for all $d \in \Delta$ and all $w \in \Wmc$:
\[
d \in C^{\Int_w} \text{ iff } d \in (C\tr)^{I_{(w, 0)}}.
\]
For the base case $C = A \in \NC$, %it
the claim follows immediately from the definitions of $I$ and $\cdot\tr$. 
Assume the claim holds for $D$ and $E$. 
The inductive cases $C = \lnot D$ and $C = (D \sqcap E)$ are straightforward. 
We are left with the cases below.

$C = \exists r.D$.
We have that
$d \in (\exists r.D)^{\Imc_{w}}$
iff there is $d' \in D^{\Imc_{w}}$ such that $(d,d') \in r^{\Imc_{w}}$.
By i.h. and definition of $I$, this is equivalent to
$d' \in (D\tr)^{I_{(w, 0)}}$ and $(d,d') \in r^{I_{(w, 0)}}$,
which means that
$d \in (\exists r. (D\tr))^{I_{(w, 0)}}$. 
By definition of $\cdot\tr$, $d \in ((\exists r.D)\tr)^{I_{(w, 0)}}$.

$C = \B_{i} D$. 
By definition, we have that
$d \in (\B_{i} D)^{\Imc_{w}}$
iff 
$\llbracket D \rrbracket^{\Mmc}_{d} \in \Nmc_{i}(w)$.
Equivalently, iff there is
$\alpha \in \Nmc_{i}(w)$
s.t. for all $v \in \Wmc$,
$v \in \alpha \Leftrightarrow d \in D^{\Imc_{v}}$.
% iff $d \in \{ d' \in \Delta \mid \exists U \in \Nmc_{i}(w) \colon \forall v \in W : v \in U \Rightarrow d' \in D^{\Imc_{v}} \land v \not \in U \Rightarrow d' \not \in D^{\Imc_{v}} \}$.
By i.h. and definitions of $\relations_{i_2}$ and $\relations_{i_3}$, this means that there is $\alpha \in \Nmc_{i}(w)$ s.t.
$(i)$ for every $v \in \Wmc : (\alpha, 1) \relations_{i_2} (v, 0) \Rightarrow d \in (D\tr)^{I_{(v, 0)}}$
and
$(ii)$ for every $v \in \Wmc: (\alpha, 1) \relations_{i_3} (v, 0) \Rightarrow d \not \in (D\tr)^{I_{(v, 0)}}$.
%\nb{M: Correggere}
This holds iff there is
$\alpha \in \Nmc_{i}(w)$ s.t. $d \in (\B_{i_2} D\tr \sqcap \B_{i_3} \lnot D\tr)^{I_{(\alpha, 1)}}$. 
By definition of $\relations_{i_1}$, this means that there exists
$(\alpha, 1) \in W$ s.t. $(w, 0) \relations_{i_1} (\alpha, 1)$ and $d \in (\B_{i_2} D\tr \sqcap \B_{i_3} \lnot D\tr)^{I_{(\alpha, 1)}}$.
That is, $d \in \D_{i_1} (\B_{i_2} D\tr \sqcap \B_{i_3} \lnot D\tr)^{I_{(w, 0)}}$
iff, by definition of $\cdot\tr$,
$d \in ((\B_{i} D)\tr)^{I_{(w, 0)}}$.
\newline

We now show that for every $\MLALC{n}$ formula $\psi$ and every $w \in \Wmc$:
\[
	\Mmc, w \mdl \psi \text{ iff } \Mmf, (w, 0) \mdl \psi\tr
\]
For the case $\psi = C \sqs D$, it follows from the previous claim, while for $\psi = C(a)$ and $\psi = r(a,b)$, it is immediate from the definitions of $I$ and $\cdot\tr$, as well as from the claim above.
Assuming that the lemma holds for $\chi$ and $\zeta$, the inductive cases $\psi = \lnot \chi$ and $\psi = \chi \land \zeta$ are straightforward.
We prove the statement for modalised formulas.

$\psi = \B_{i} \chi$.
$\Mmc, w \mdl \B_{i} \chi$
iff, by definition,
$\llbracket \chi \rrbracket^{\Mmc}  \in \Nmc_{i}(w)$.
That is, iff there is $\alpha \in \Nmc_{i}(w)$ s.t. for all $v \in \Wmc : v \in \alpha \Leftrightarrow \Mmc, v \mdl \chi$.
By i.h. and definitions of $\relations_{i_2}, \relations_{i_3}$, this means that there is
$\alpha \in \Nmc_{i}(w)$ s.t.
$(i)$ for all $v \in \Wmc : (\alpha, 1) \relations_{i_2} (v, 0) \Rightarrow \Mmf, (v, 0) \mdl \chi\tr$
and
$(ii)$ for all $v \in \Wmc: (\alpha, 1) \relations_{i_3} (v, 0) \Rightarrow \Mmf, (v, 0) \not\mdl \chi\tr$,
iff there is $\alpha \in \Nmc_{i}(w)$ s.t.
$\Mmf, (\alpha, 1) \mdl \B_{i_2} \chi\tr \land \B_{i_3} \lnot \chi\tr$.
By definition of $\relations_{i_1}$, the previous step is equivalent to:
there is $(\alpha, 1) \in W$ s.t. $(w, 0) \relations_{1} (\alpha, 1)$ and $\Mmf, (\alpha, 1) \mdl \B_{i_2} \chi\tr \land \B_{i_3} \lnot \chi\tr$,
iff
$\Mmf, (w, 0) \mdl \D_{i_1} (\B_{i_2} \chi\tr \land \B_{i_3} \lnot \chi\tr)$.
By definition of $\cdot\tr$,
$\Mmf, (w, 0) \mdl (\B_{i} \chi)\tr$.


Thus, in particular, we obtain $\Mmf, (w, 0) \mdl \p\tr$.
\newline

Conversely, consider a $\MLALC{3n}$ formula $\p\tr$ s.t.
$\Mmf, w \mdl \p\tr$,
for some $\MLALC{3n}$ R-model
$\Mmf = (\Fmf, \Delta, I)$
based on
$\Fmf = (W, \{ \relations_{i_{j}} \}_{j \in [1, 3]})$,
and some
$w \in W$.
We define a $\MLnALC{n}$
neighbourhood model
$\Mmc = (\Fmc, \Delta, \Int)$
based on
$\Fmc = (\Wmc, \{ \Nmc_{i} \}_{i \in [1, n]})$
s.t.
$\Wmc = W$,
and for all $w \in W$:
\begin{itemize}
			\item $\alpha \in \Nmc_{i}(w)$ iff there is $v \in W$ s.t. $w \relations_{i_1} v$ and: $(i)$ for all $u \in W$, $v \relations_{i_2} u \Rightarrow u \in \alpha$, and $(ii)$ for all $u \in W$, $v \relations_{i_3} u \Rightarrow u \not \in \alpha$;
			\item $\Imc_{w} = I_{w}$.
\end{itemize} 

%%% OLD VERSION
%Conversely, consider a satisfiable \MLnALC{3} formula $\p\tr$  and let $\Mmf = (\Fmf, \Int)$ be a \MLnALC{3} relational model based on $\Fmc_r = (W, \relations_{i_1}, \relations_{i_2}, \relations_{i_3})$ and having domain $\Delta$, s.t. $\Mmf, w \mdl \p\tr$, for some $w \in W$. 
%Define a \MLALC{} neighborhood model model $\Mmc_n = (\Fmc_n, \Int)$ based on $\Fmc_n = (W, N)$ and having the same domain $\Delta$ s.t.:
%%Conversely, consider a satisfiable \MLnALC{3} formula $\p\tr$  and let $\Mmf = (\Fmf, I)$ be a \MLnALC{3} relational model based on $\Fmf = (W, \relations_{i_1}, \relations_{i_2}, \relations_{i_3})$ and having domain $\Delta$, s.t. $\Mmf, w \mdl \p\tr$, for some $w \in W$. Define a \MLALC{} model $\Mmc = (\Fmc, \Int)$ based on $\Fmc = (W, N)$ and having domain $\Delta$ s.t.:
%\begin{itemize}
%	\item $w \in W$ iff there is $v \in W$ s.t. $w \relations_{i_1} v$ or $v \relations_{i_2} w$ or $v \relations_{i_3} w$;
%	\item $N \colon W \to \Pmc(\Pmc(W))$ neighborhood function s.t., for all $w \in W$:
%		\begin{itemize}
%			\item $U \in \Nmc_{i}(w)$ iff there is $v \in W$ s.t. $w \relations_{i_1} v$ and for all $u \in W$, $v \relations_{i_2} u \Rightarrow u \in U$, and for all $t \in W$, $v \relations_{i_3} t \Rightarrow t \not \in U$.
%%			\item for all \MLALC{} concepts $C$, and for all $d \in \Delta$, $[C]^{\Mmc}_{d} \in \Nmc_{i}(w)$ iff there is $v \in W$ s.t. $w \relations_{i_1} v$ and for all $u \in  W$: $v \relations_{i_2} u \Rightarrow d \in C^{\Imc_{u}}$ and $v \relations_{i_3} u \Rightarrow d \not \in C^{\Imc_{u}}$;
%%			\item for all \MLALC{} formulas $\psi$, $[\psi]^{\Mmc} \in \Nmc_{i}(w)$ iff there is $v \in W$ s.t. $w \relations_{i_1} v$ and for all $u \in  W$: $v \relations_{i_2} u \Rightarrow \Mmc, u \mdl \psi$ and $v \relations_{i_3} u \Rightarrow \Mmc, u \not \mdl \psi$;
%		\end{itemize} 
%	\item for all $w \in W$, $\Imc_{w} = I_{w}$.
%\end{itemize}

Again, we show firstly, by induction on the structure of concepts $C$, that for all $d \in \Delta$ and all $w \in W$:
\[
d \in C^{\Imc_{w}} \text{ iff } d \in (C\tr)^{I_{w}}.
\]
For the base case $C = A \in \NC$, the claim follows from the definitions of $\Int$ and $\cdot\tr$. 
Assume the claim holds for $D$ and $E$. 
The inductive cases $C = \lnot D$ and $C = (D \sqcap E)$ are straightforward. 
We are left with the cases below.

$C = \exists r.D$.
We have that
$d \in (\exists r.D)^{\Imc_{w}}$
iff there is $d' \in D^{\Imc_{w}}$ such that $(d,d') \in r^{\Imc_{w}}$.
By i.h. and definition of $\Int$, this is equivalent to
$d' \in (D\tr)^{I_{w}}$ and $(d,d')\in r^{I_{w}}$, which means that $d \in (\exists r. (D\tr))^{I_{w}}$. 
By definition of $\cdot\tr$, $d \in ((\exists r.D)\tr)^{I_{w}}$.

$C = \B_{i} D$. 
We have  
$d \in (\B_{i} D)^{\Imc_{w}}$ iff 
$\llbracket D \rrbracket^{\Mmc}_{d} \in \Nmc_{i}(w)$.
By definition of $\Nmc_{i}$, this means that there is
$v \in W$ s.t. $w \relations_{i_1} v$
and:
$(i)$ for all $u \in W$: $v \relations_{i_2} u \Rightarrow d \in D^{\Imc_{u}}$
and
$(ii)$ for all $u \in W$: $v \relations_{i_3} u \Rightarrow d \not \in D^{\Imc_{u}}$.
By i.h. the previous step is equivalent to:
there is $v \in W$ s.t. $w \relations_{i_1} v$ and:
$(i)$ for all $u \in  W$: $v \relations_{i_2} u \Rightarrow d \in (D\tr)^{I(u)}$ and
$(ii)$ for all $u \in W$: $v \relations_{i_3} u \Rightarrow d \not \in (D\tr)^{I(u)}$.
Equivalently,
$d \in (\D_{i_1}(\B_{i_2} D\tr \sqcap \B_{i_3} \lnot D\tr))^{I_{w}}$
iff, by definition of $\cdot\tr$,
$d \in ((\B_{i} D)\tr)^{I_{w}}$.
\newline
%%% OLD VERSION
%We have  
%$d \in (\B D)^{\Imc_{w}}$ iff 
%$[ D ]^{\Mmc}_{d} \in \Nmc_{i}(w)$.
%By definition of $N$, this means that there is $v \in W$ s.t. $w \relations_{i_1} v$ and for all $u \in  W$: $v \relations_{i_2} u \Rightarrow d \in C^{\Imc_{u}}$ and $v \relations_{i_3} u \Rightarrow d \not \in C^{\Imc_{u}}$.
%By i.h. the previous step is equivalent to: there is $v \in W$ s.t. $w \relations_{i_1} v$ and for all $u \in  W$: $v \relations_{i_2} u \Rightarrow d \in (C\tr)^{I(u)}$ and $v \relations_{i_3} u \Rightarrow d \not \in (C\tr)^{I(u)}$, iff
%$d \in (\D_{i_1}(\B_{i_2} C\tr \sqcap \B_{i_3} \lnot C\tr))^{I_{w}}$.
%By definition of $\cdot\tr$, $d \in ((\B C)\tr)^{I_{w}}$.

We now prove, by induction on $\MLALC{n}$ formulas $\psi$, that for every $w \in W$:
\[
	\Mmc, w \mdl \psi \text{ iff } \Mmf, w \mdl \psi\tr
\]

For the case $\psi = C \sqs D$, it follows from the previous claim, while for $\psi = C(a)$ and $\psi = r(a,b)$, it is immediate from the definitions of $\Int$ and $\cdot\tr$, as well as from the claim above. Assuming that the lemma holds for $\chi$ and $\zeta$, the inductive cases $\psi = \lnot \chi$ and $\psi = \chi \land \zeta$ are straightforward. We prove the statement for modalised formulas.

$\psi = \B_{i} \chi$.
$\Mmc, w \mdl \Box_{i} \chi$ iff
$\llbracket \chi \rrbracket^{\Mmc}  \in \Nmc_{i}(w)$.
That is, there is $v \in W$ s.t. $w \relations_{i_1} v$ and $(i)$ for all $u \in  W$: $v \relations_{i_2} u \Rightarrow \Mmc, u \mdl \chi$ and $(ii)$ for all $u \in W$: $v \relations_{i_3} u \Rightarrow \Mmc, u \not \mdl \chi$.
By i.h., this is equivalent to: there is $v \in W$ s.t. $w \relations_{i_1} v$ and
$(i)$ for all $u \in  W$: $v \relations_{i_2} u \Rightarrow \Mmf, u \mdl \chi\tr$ and
$(ii)$ for all $u \in  W$: $v \relations_{i_3} u \Rightarrow \Mmf, u \not \mdl \chi\tr$.
The previous step means that: 
$\Mmf, w \mdl \D_{i_1} (\B_{i_2} \chi\tr \land \B_{i_3} \lnot \chi\tr)$
iff, by definition of $\cdot\tr$,
$\Mmf, w \mdl (\B_{i} \chi)\tr$.

%%% OLD VERSION
%$\psi = \B \chi$.
%$\Mmc, w \mdl \Box \chi$ iff
%$[ \chi ]^{\Mmc}  \in \Nmc_{i}(w)$.
%That is, there is $v \in W$ s.t. $w \relations_{i_1} v$ and for all $u \in  W$: $v \relations_{i_2} u \Rightarrow \Mmc, u \mdl \chi$ and $v \relations_{i_3} u \Rightarrow \Mmc, u \not \mdl \chi$.
%By i.h., this is equivalent to: there is $v \in W$ s.t. $w \relations_{i_1} v$ and for all $u \in  W$: $v \relations_{i_2} u \Rightarrow \Mmf, u \mdl \chi\tr$ and $v \relations_{i_3} u \Rightarrow \Mmf, u \not \mdl \chi\tr$, iff 
%$\Mmf, w \mdl \D_{i_1} (\B_{i_2} \chi\tr \land \B_{i_3} \lnot \chi\tr)$.
%By definition of $\cdot\tr$, $\Mmf, w \mdl (\B \chi)\tr$.


Therefore,  in particular, $\Mmc, w \mdl \p$.
%\qed
\end{proof}
\end{proof}





















\Theoremcomplmalc*
%
\begin{proof}
This theorem is a consequence of the following claim, and the complexity 
of formula satisfiability in $\KnALC{2n}$ constant domain relational models~\cite{KraWol,GasHer,GabEtAl03}. 
\begin{claim} %{lemma}{Theoremmonotonicred}\label{theor:monotonicred}
%\nb{M: Change to lemma? Merge to next Th. as a claim?}
%Satisfiability in $\MnALC{n}$ is reducible to satisfiability in $\KnALC{2n}$.
The $\MnALC{n}$ formula satisfiability problem on constant domain neighbourhood models can be reduced in polynomial time to the \KnALC{2n} formula satisfiability problem on constant domain relational models.
\end{claim}
 \begin{proof}[Proof of Claim]
The proof is analogous to that of Theorem~\ref{theor:complealc} (we show the cases for modalised concepts and formulas only).
Consider a $\MLALC{n}$ formula $\p$ satisfiable on supplemented neighbourhood frames, i.e., so that there is a neighbourhood model
$\Mmc = (\Fmc, \Delta, \Int)$
based on a supplemented neighbourhood frame
$\Fmc = (\Wmc, \{ \Nmc_{i} \}_{i \in J})$ and a $w$ in $\Mmc$ such that
$\Mmc, w \mdl \p$.
%Namely, there are: a \e{supplemented} neighborhood frame $\Fmc = (\Wmc, N)$, i.e., a frame s.t. for all $w \in \Wmc$ and all $\alpha, \beta \sbs \Wmc$, if $\alpha \in \Nmc_{i}(w)$ and $\alpha \sbs \beta$, then $\beta \in \Nmc_{i}(w)$; a neighborhood model $\Mmc = (\Fmc, \Int)$ based on $\Fmc$; and a $w \in M_{n}$ s.t. $\Mmc, w \mdl \p$.
We define a relational frame
$\Fmf = (W, \{ R_{i_{1}}, R_{i_{2}} \}_{i \in J})$,
and an $\MLALC{2n}$ relational model
$\Mmf = (\Fmf, \Delta, I)$ based on $\Fmf$, such that:
	\begin{itemize}
		\item $W = \{ (w, 0) \mid w \in \Wmc \} \cup \{ (\alpha, 1) \mid \alpha \in \bigcup_{v \in \Wmc} \Nmc_{i}(v) \}$
%		$W = \Wmc \uplus \bigcup_{w \in \Wmc} \Nmc_{i}(w)$, where $\uplus$ takes the disjoint union of (suitably indexed copies) of $\Wmc$ and $\bigcup_{w \in \Wmc} \Nmc_{i}(w)$;\nb{M:  Abuso di notazione con $\uplus$}
%		{\color{red}{$W \cap \bigcup_{w \in W} \Nmc_{i}(w) = \eset$?}};
		\item $\relations_{i_1} = \{ ((w, 0), (\alpha, 1)) \mid \alpha \in \Nmc_{i}(w)\}$;
		\item $\relations_{i_2} = \{ ((\alpha, 1), (w, 0)) \mid w \in \alpha \}$
%		\item $\relations_{i_1} = \{ (w, \alpha) \mid w \in \Wmc, \alpha \in \Nmc_{i}(w)\}$;
%		\item$\relations_{i_2} = \{ (\alpha, w) \mid \alpha \in \bigcup_{v \in \Wmc} N(v), w \in \Wmc, w \in \alpha \}$;
%		\nb{M: Togliere $w \in \Wmc$? (cf. sopra)}
%		\item $\relations_{i_2} = \{ (U, w) \in W \times W \mid w \in W, U \in \Nmc_{i}(w), w \in U \}$;
		\item for every $(w, 0) \in W$, $I_{(w, 0)} = \Imc_{w}$; for every $(\alpha, 1) \in W$, $X^{I_{(\alpha, 1)}} = \eset$, for all $X \in \NC \cup \NR$, and $a^{I_{(\alpha, 1)}} = a^{\Int}$, for all $a \in \NI$.
%		\item for all $w \in \Wmc$, $I_{w} = \Imc_{w}$; for all $\alpha \in W \setminus \Wmc$, $X^{I(\alpha)} = \eset$, for all $X \in \NC \cup \NR$, and $a^{I(\alpha)} = a^{\Int}$, for all $a \in \NI$.
%		\nb{M: Cambiare! Opzioni: \\ (a) falso in $U$ $\to$ diverse def. semantiche; \\ (b) non definito in $U$ $\to$ interpret. $I$ funzione parziale? \\ (c) lasciare cosi}
	\end{itemize}

%\nb{M: Check}
Firstly we show, by induction on the structure of concepts $C$, that for all $d \in \Delta$ and all $w \in \Wmc$:
\[
d \in C^{\Imc_{w}} \text{ iff } d \in (C\ttr)^{I(w,0)}.
\]
\noindent
$C = \B_{i} D$. 
We have  
$d \in (\B_{i} D)^{\Imc_{w}}$
iff 
%$d \in$
$\llbracket D \rrbracket^{\Mmc}_{d} \in \Nmc_{i}(w)$.
Since $\Fmc$ is supplemented, the previous step means that there is
$\alpha \in \Nmc_{i}(w) \colon \alpha \sbs \llbracket D \rrbracket^{\Mmc}_{d}$.
This is equivalent to:
there is $\alpha \in \Nmc_{i}(w)$
s.t. for every
$v \in \Wmc : v \in \alpha \Rightarrow d \in D^{\Imc_{v}}$.
% iff $d \in \{ d' \in \Delta \mid \exists U \in \Nmc_{i}(w) \colon \forall v \in W : v \in U \Rightarrow d' \in D^{\Imc_{v}} \land v \not \in U \Rightarrow d' \not \in D^{\Imc_{v}} \}$.
By i.h. and definition of $\relations_{i_2}$,
there is $\alpha \in \Nmc_{i}(w)$
s.t. for every
$v \in \Wmc : (\alpha, 1) \relations_{i_2} (v, 0) \Rightarrow d \in (D\ttr)^{I(v,0)}$.
Equivalently, there is
$\alpha \in \Nmc_{i}(w) \colon d \in (\B_{i_2} D\ttr)^{I(\alpha,1)}$.
%\nb{M: Correggere (cf. sopra)}
By definition of $\relations_{i_1}$, this means:
there exists
$(\alpha,1) \in W$
s.t.
$(w,0) \relations_{i_1} (\alpha,1)$ and $d \in (\B_{i_2} D\ttr)^{I(\alpha,1)}$, 
iff
$d \in (\D_{i_1} \B_{i_2} D\ttr )^{I(w,0)}$.
That is, by definition of $\cdot\tr$,
$d \in ((\B D_{i})\ttr)^{I(w,0)}$.
\newline

Then, we prove that for every $\MLALC{n}$ formula $\psi$ and every $w \in \Wmc$:
\[
	\Mmc, w \mdl \psi \text{ iff } \Mmf, (w,0) \mdl \psi\ttr
\]
\noindent
$\psi = \B_{i} \chi$.
$\Mmc, w \mdl \B_{i} \chi$ iff
$\llbracket \chi \rrbracket^{\Mmc}  \in \Nmc_{i}(w)$.
Since $\Fmc$ is supplemented, this is equivalent to: 
there is $\alpha \in \Nmc_{i}(w)$ s.t. $ \alpha \sbs \llbracket \chi \rrbracket^{\Mmc}$,
iff
there is $\alpha \in \Nmc_{i}(w)$ s.t. for all $v \in \Wmc : v \in \alpha \Rightarrow \Mmc, v \mdl \chi$.
By i.h. and definition of $\relations_{i_2}$, there is $\alpha \in \Nmc_{i}(w)$
s.t. for all $v \in \Wmc : (\alpha,1) \relations_{i_2} (v,0) \Rightarrow \Mmf, (v,0) \mdl (\chi\ttr)$.
This means that, for some $\alpha \in \Nmc_{i}(w)$,
$\Mmf, (\alpha,1) \mdl \B_{i_2} \chi\ttr$.
By definition of $\relations_{i_1}$, the previous step is equivalent to:
there is $(\alpha,1) \in W$ s.t.
$(w,0) \relations_{i_1} (\alpha,1)$
and
$\Mmf, (\alpha,1) \mdl \B_{i_2} \chi\ttr$.
That is,
$\Mmf, (w,0) \mdl \D_{i_1} \B_{i_2} \chi\ttr$.
By definition of $\cdot\ttr$,
$\Mmf, (w,0) \mdl (\B_{i} \chi)\ttr$.

Thus, in particular, we have $\Mmf, (w,0) \mdl \p\ttr$.
\newline

%\nb{M: Check}
Conversely, consider an $\MLALC{2n}$ formula $\p\ttr$ satisfiable on relational models:
$\Mmf, w \mdl \p\ttr$,
for some $\MLALC{2n}$ model
$\Mmf = (\Fmf, \Delta, I)$ based on $\Fmf = (W, \{ \relations_{i_{j}} \}_{j \in [1, 2]})$, and some $w \in W$. 
We define an $\MLALC{n}$ neighbourhood model
$\Mmc = (\Fmc, \Delta, \Int)$
based on
$\Fmc = (\Wmc, \{ \Nmc_{i} \}_{i \in [1, n]})$
s.t. $\Wmc = W$, and for all $w \in W$:
\begin{itemize}
%			\item $\Wmc = W$;
			\item $\alpha \in \Nmc_{i}(w)$ iff there is $v \in W$ s.t. $w \relations_{i_1} v$ and for all $u \in W$, $v \relations_{i_2} u \Rightarrow u \in \alpha$;
			\item $\Imc_{w} =I_{w}$.
\end{itemize} 

Firstly, notice that $\Fmc$ is supplemented: 
%by definition, we have that for all $w \in W$ and for all $\alpha \sbs W$, $\alpha \in \Nmc_{i}(w)$ iff there is $v \in W$ s.t. $w \relations_{i_1} v$ and for all $u \in W$, $v \relations_{i_2} u \Rightarrow u \in \alpha$. 
for all $w \in W$, if $\alpha \in \Nmc_{i}(w)$ and $\alpha \sbs \beta \sbs W$, then there is $v \in W$ s.t. $w \relations_{i_1} v$ and for all $u \in W$, $v \relations_{i_2} u \Rightarrow u \in \beta$, i.e., $\beta \in \Nmc_{i}(w)$.
We now show, by induction on the structure of concepts $C$, that for all $d \in \Delta$ and all $w \in W$:
\[
d \in C^{\Imc_{w}} \text{ iff } d \in (C\ttr)^{I_{w}}.
\]
\noindent
$C = \B_{i} D$. 
We have  
$d \in (\B_{i} D)^{\Imc_{w}}$ iff 
$\llbracket D \rrbracket^{\Mmc}_{d} \in \Nmc_{i}(w)$.
By definition of $\Nmc$, this means that there is
$v \in W$ s.t. $w \relations_{i_1} v$ and for all $u \in  W$: $v \relations_{i_2} u \Rightarrow d \in D^{\Imc_{u}}$.
By i.h. the previous step is equivalent to: there is $v \in W$ s.t. $w \relations_{i_1} v$ and for all $u \in  W$: $v \relations_{i_2} u \Rightarrow d \in (D\ttr)^{I_{u}}$, iff
$d \in (\D_{i_1}\B_{i_2} D\ttr)^{I_{w}}$.
By definition of $\cdot\ttr$, $d \in ((\B_{i} D)\ttr)^{I_{w}}$.
\newline

Finally, we prove, by induction on $\MLALC{n}$ formulas $\psi$, that for every $w \in W$:
\[
	\Mmc, w \mdl \psi \text{ iff } \Mmf, w \mdl \psi\ttr
\]
\noindent
$\psi = \B_{i} \chi$.
$\Mmc, w \mdl \B_{i} \chi$ iff
$\llbracket \chi \rrbracket^{\Mmc}  \in \Nmc_{i}(w)$.
That is, there is $v \in W$ s.t. $w \relations_{i_1} v$ and for all $u \in  W$: $v \relations_{i_2} u \Rightarrow \Mmc, u \mdl \chi$.
By i.h., this is equivalent to: there is $v \in W$ s.t. $w \relations_{i_1} v$ and for all $u \in  W$: $v \relations_{i_2} u \Rightarrow \Mmf, u \mdl \chi\ttr$,
iff 
$\Mmf, w \mdl \D_{i_1} \B_{i_2} \chi\ttr$.
By definition of $\cdot\ttr$,
$\Mmf, w \mdl (\B_{i} \chi)\ttr$.

Therefore, we have in particular that $\Mmc, w \mdl \p$, i.e., $\p$ is satisfiable on a supplemented neighbourhood model.
%\qed
\end{proof}
\end{proof}





%\subsection{Proofs for Section~\ref{sec:fragcondom}}



\LemmapropE*
%
\begin{proof}
 If formula $\varphi$ is $\LnALCg$ satisfiable on constant domain neighbourhood models then, clearly,
$\prop{\varphi}$ is satisfied in a $\varphi$-consistent $L^{n}$ model.  
We now argue about the converse direction. 
Suppose $\prop{\varphi}$ is satisfied in a $\varphi$-consistent $L^{n}$ model
$\propmodel = (\Wmc, \{ \Nmc_{i} \}_{i \in J}, \Vmc)$. 
We want to construct a constant domain neighbourhood model $\Mmc=(\Fmc,\Delta,\Imc)$, based on the $L^{n}$ frame $\Fmc = ( \Wmc, \{ \Nmc_{i} \}_{i \in J} )$, that satisfies 
$\varphi$.
%We define $\W$ as $\propdomain$ and \Nmc as $\propneigh$. 
%\todo{M: working here}
%The main point in this proof is the definition of \Imc and $\Delta$.
%
As $\propmodel$ is $\varphi$-consistent, for all $w\in \propdomain$,
\[
\alcform = \bigwedge_{p_{\elaxiom}\in \formtp{\varphi}} {\elaxiom} \ \wedge \bigwedge_{p_{\elaxiom} \in
\NPr(\varphi) \setminus \formtp{\varphi}}
% \overline{\NPr(w)}}
 \neg {\elaxiom},
\]
where
$\formtp{\varphi} = \{p_{\elaxiom} \in \NPr(\varphi) \mid w\in \Vmc(p_{\elaxiom})\}$,
%$$\formula(w) =  \bigwedge_{p_{\elaxiom}\in \NPr(w)} \hspace{-0.15cm} {\elaxiom} \ \wedge \hspace{-0.15cm} \bigwedge_{p_{\elaxiom}\in \overline{\NPr(w)}} \hspace{-0.15cm} \neg {\elaxiom}$$
is satisfied by an interpretation $\Jmc_{w}$.
It remains to argue that one can  define $\Imc$ 
where all $\Imc_{w}$ share the same domain $\Delta$ and 
the rigid individual name assumption holds (i.e., $a^{\Imc_{w}}=a^{\Imc_{v}}$, for all $w, v \in \W$ and all $a\in \NI$). 

By Lemma~\ref{lem:aux}, for each model $\Jmc_{w}$ of
$\alcform$
%$\formula(w)$
there is 
a quasimodel $\Qmc(w)$ for
$\alcform$.
%$\varphi$.
We denote by
$\Qmc_{\sf tp}(w)$ the set of concept types
  in  $\Qmc(w)$. 
We assume without loss of generality that
\begin{itemize}
\item[$(\ast)$] for all named types $t_a$, if $t_a\in\Qmc_{\sf tp}(w)$,
then $t=t_a\setminus\{a\}\in\Qmc_{\sf tp}(w)$.
\end{itemize}
We are now in position to define \Imc and $\Delta$. 
We define $\Delta$ as the set of functions $f: \W\rightarrow {\sf tp}(\varphi)$
such that (1) for all $w\in\W$, $f(w)\in \Qmc_{\sf tp}(w)$; and 
(2) $a\in f(w)$ iff $a\in f(u)$ for all $u,w\in \W$ and all $a\in \NI(\varphi)$.
By $(\ast)$ and the definition of $\Qmc_{\sf tp}(w)$,
we also have that (3) for all $w\in \W$ and all types $t\in\Qmc_{\sf tp}(w)$, 
there is $f\in\Delta$ such that $f(w)=t$. 
We denote by $f_a$ the unique element of $\Delta$ where $a$ occurs in it. 
For each $w\in \W$, we define the interpretation $\Imc_{w}$ as follows:
\begin{itemize}
\item for all $A\in\NC$, we have $f\in A^{\Imc_{w}}$ iff $A\in f(w)$;
\item for all $r\in\NR$, we have $(f,f')\in r^{\Imc_{w}}$ iff
$\{\neg D\mid \neg \exists r.D \in f(w)\}\subseteq f'(w)$.
\end{itemize}
Also, for all $w\in \W$ and all $a\in \NI(\varphi)$,
we require that $a^{\Imc_{w}}=f_a\in\Delta$ (for the other 
individual names the mapping is irrelevant for this proof, 
as long as it satisfies the rigid individual name assumption). 
By definition, $\Imc$ is a function mapping each $w\in\W$ to 
an interpretation $\Imc_{w}$ over the constant domain $\Delta$ and satisfying the %constraint 
rigid individual name assumption.

\todo{M: todo fix}
{\color{red}{
One can show that each $\Imc_{w}$ is a model of $\alcform$ using the fact that $\Qmc(w)$ is a quasimodel for $\alcform$. 
}}


{\color{blue}{
We first require the following result.

\todo{M: todo fix}
\begin{claim}\label{cl:localeq}
{\color{red}{
$\Jmc_{w} \models \alcform$ iff $\Imc_{w} \models \alcform$.
}}
\end{claim}
\begin{proof}
{\color{red}{\ldots}}
\end{proof}
}}

{\color{blue}{
Now, let $\Mmc=(\Fmc,\Delta,\Imc)$, with $\Fmc = ( \Wmc, \{ \Nmc_{i} \}_{i \in J} )$, $\Imc$, and $\Delta$ as defined above.
%By induction on the structure of subformulas $\psi$ of $\varphi$, we show
%in the following claim that,
%for every $w \in \Wmc$, we have
%$\propmodel, w \models \prop{\psi}$
%iff $\Mmc, w \models \psi$.
We now show the following claim.
 
\begin{claim}\label{cl:globaleq}
For every $\psi \in {\sf sub}(\varphi)$ and every $w \in \Wmc$, we have
$\propmodel, w \models \prop{\psi}$ iff
$\Mmc, w \models \psi$.
\end{claim}
\begin{proof}
The proof is by induction on the structure of subformulas $\psi$ of $\varphi$.
We first consider the base case.

\begin{itemize}
	\item $\psi = \pi$, where $\pi$ is an $\ALC$ atom in $\varphi$. Hence, $\prop{\psi} = p_\pi$, with $p_\pi$ propositional letter. By the semantics of propositional neighbourhood models,
$\propmodel, w \models p_{\pi}$ iff $w\in\Vmc(p_\pi)$. 
For every \ALC atom $\pi$ in $\varphi$, $w\in\Vmc(p_\pi)$ iff 
$\pi$ is a conjunct of $\hat{\varphi}_{\Vmc,w}$.
As $\propmodel$ is $\varphi$-consistent, we have that, for every $w\in \propdomain$,
the $\ALC$ formula
$\hat{\varphi}_{\Vmc,w}$
is satisfied by the $\ALC$ interpretation $\Jmc_{w} = (\Delta_{w}, \cdot^{\Jmc_{w}})$.
 Thus, $\pi$ is a conjunct of $\hat{\varphi}_{\Vmc,w}$ iff $\Jmc_w\models\pi$.
 \todo{AM: to fix, why?}
 {\color{red}{
 By
 Claim~\ref{cl:localeq},
% the semantics of $\MLnALCg$  neighbourhood models,
 }}
 the previous step is equivalent to
 $\Imc_w\models\pi$, i.e., $\Mmc, w \models \psi$.
 \end{itemize}
 For the inductive step, suppose that the claim holds for $\psi_1,\psi_2$. 
  We consider the following cases. 
 \begin{itemize}
 	\item $\psi=\neg\psi_1$: By the semantics of  propositional neighbourhood models,
 	$\propmodel, w \models \prop{\neg{\psi_1}}$ iff $\propmodel, w \not\models \prop{{\psi_1}}$. By the inductive hypothesis, Claim~\ref{cl:ind} holds for $\psi_1$.
 	By the contrapositive in each direction, $\propmodel, w \not\models \prop{{\psi_1}}$
 	iff $\Mmc, w \not\models \psi_1$. By the semantics of  $\MLnALCg$ neighbourhood models, $\Mmc, w \not\models \psi_1$ iff $\Mmc, w \models \neg\psi_1$.
\item $\psi=\psi_1\wedge\psi_2$: By the semantics of  propositional neighbourhood models,
$\propmodel, w \models \prop{{(\psi_1\wedge\psi_2)}}$ iff $\propmodel, w \models \prop{{\psi_1}}$ and $\propmodel, w \models \prop{{\psi_2}}$. By the inductive hypothesis, the claim holds for $\psi_1,\psi_2$.
So, $\propmodel, w \models \prop{{\psi_i}}$
iff $\Mmc, w \models \psi_i$, for $i\in \{1,2\}$. By the semantics of  $\MLnALCg$ neighbourhood models, $\Mmc, w \models \psi_1$ and $\Mmc, w \models \psi_2$ iff $\Mmc, w \models \psi_1\wedge \psi_2$.
\item $\psi=\B_{i} \psi_1$: By the semantics of  propositional neighbourhood models,
$\propmodel, w \models \prop{{(\B_{i} \psi_1)}}$ iff $\llbracket \prop{{\psi_1}}\rrbracket^{\propmodel} \in \Nmc_{i}(w)$ where
$\llbracket \prop{{\psi_1}} \rrbracket^{\propmodel} = \{ v \in \Wmc \mid \propmodel, v \models \prop{{\psi_1}} \}$. By the inductive hypothesis, the claim holds for $\psi_1$.
So, $\propmodel, v \models \prop{{\psi_1}}$
iff $\Mmc, v \models \psi_1$, for every $v\in\Wmc$. 
Thus, $\llbracket \prop{{\psi_1}} \rrbracket^{\propmodel}=\llbracket {{\psi_1}} \rrbracket^{\Mmc}$. By definition of $\propmodel$ and \Mmc, we have that $\Nmc_{i}(w)$
is the same in both $\propmodel$ and \Mmc, for every $w\in\Wmc$ and $i\in J$.
So $\llbracket \prop{{\psi_1}}\rrbracket^{\propmodel} \in \Nmc_{i}(w)$
iff $\llbracket {{\psi_1}} \rrbracket^{\Mmc} \in \Nmc_{i}(w)$.
By the semantics of  $\MLnALCg$ neighbourhood models, $\llbracket {{\psi_1}} \rrbracket^{\Mmc} \in \Nmc_{i}(w)$ iff $\Mmc, w \models \B_{i} \psi_1$.
 \end{itemize}
We have thus shown that for every subformula $\psi$ of $\varphi$ and every $w \in \Wmc$, we have
$\propmodel, w \models \prop{\psi}$ iff
$\Mmc, w \models \psi$.
\end{proof}
Since $\propmodel , v \models \prop{\varphi}$, for some $v \in \Wmc$, we conclude that $\varphi$ is $\LnALCg$ satisfiable. 
}}
\end{proof}



















\end{document}