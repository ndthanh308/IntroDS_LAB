%\subsection{Proofs for Section~\ref{sec:fragvardom}}
\section{Proofs for Section~\ref{sec:fragvardom}}







\LemmapropL*
\begin{proof}
%{{
If an
$\MLnALCg$
formula $\varphi$ is $\LnALCg$ satisfiable
on varying domain neighbourhood models
then, clearly,
$\prop{\varphi}$ is satisfied in a $\setsymbols_\varphi$-consistent $L^{n}$ model.  
We now argue about the converse direction. 
Suppose that $\prop{\varphi}$ is satisfied in a $\setsymbols_\varphi$-consistent $L^{n}$ model
$\propmodel = (\Wmc, \{ \Nmc_{i} \}_{i \in J}, \Vmc)$. 
%We define $\W$ as $\propdomain$ and \Nmc as $\propneigh$. 
%The main point in this proof is the definition of $\Imc$.
%
%Given $w \in \Wmc$, let $\NPr(w) = \{p_{\elaxiom}\in \NPr(\varphi) \mid w\in \Vmc(p_{\elaxiom})\}$.
As $\propmodel$ is $\setsymbols_\varphi$-consistent, we have that, for every $w\in \propdomain$,
the $\ALC$ formula
$\alcform$
%\[
%\formula =  \bigwedge_{p_{\elaxiom}\in \formtp{\varphi}} {\elaxiom} \ \wedge \bigwedge_{p_{\elaxiom} \in
%\NPr(\varphi)\setminus\formtp{\varphi}}
%% \overline{\NPr(w)}}
% \neg {\elaxiom}
%\]
%and $\overline{\NPr(w)}=\NPr(\varphi)\setminus\NPr(w)$
is satisfied by an $\ALC$ interpretation, say $\Imc_{w} = (\Delta_{w}, \cdot^{\Imc_{w}})$.
We define the
varying domain neighbourhood model $\Mmc=(\Fmc,\Imc)$, where the $L^{n}$ frame $\Fmc = ( \W, \{ \Nmc_{i} \}_{i \in J} )$ is as above,
and where $\Imc$ is a function associating with each $w \in \Wmc$ the $\ALC$ interpretation $\Imc_{w}$.
By induction on the structure of subformulas $\psi$ of $\varphi$, it can be shown that,
for every $w \in \Wmc$, we have
$\propmodel, w \models \prop{\psi}$
iff
 $\Mmc, w \models \psi$. We show this in Claim~\ref{cl:ind}.
\begin{claim}\label{cl:ind}
For every subformula $\psi$ of $\varphi$ and every $w \in \Wmc$, we have
$\propmodel, w \models \prop{\psi}$ iff
$\Mmc, w \models \psi$.
\end{claim}
\begin{proof}
In the base case $\psi$ is an $\ALC$ atom $\pi$ in $\varphi$ and $\prop{\psi}$ is a propositional symbol $p_\pi$. By the semantics of propositional neighbourhood models,
$\propmodel, w \models \prop{\psi}$ iff $w\in\Vmc(p_\pi)$. 
For every \ALC atom $\pi$ in $\varphi$, $w\in\Vmc(p_\pi)$ iff 
$\pi$ is a conjunct of $\hat{\varphi}_{\Vmc,w}$.
As $\propmodel$ is $\setsymbols_\varphi$-consistent, we have that, for every $w\in \propdomain$,
the $\ALC$ formula
$\hat{\varphi}_{\Vmc,w}$
is satisfied by the $\ALC$ interpretation $\Imc_{w} = (\Delta_{w}, \cdot^{\Imc_{w}})$.
 Thus, $\pi$ is a conjunct of $\hat{\varphi}_{\Vmc,w}$ iff $\Imc_w\models\pi$.
 By the semantics of $\MLnALCg$  neighbourhood models,
 $\Imc_w\models\pi$ iff $\Mmc, w \models \psi$.
 Suppose that Claim~\ref{cl:ind} holds for $\psi_1,\psi_2$. 
 For the inductive step, we make the following case distinction on 
 the format of $\psi$. 
 \begin{itemize}
 	\item $\psi=\neg\psi_1$: By the semantics of  propositional neighbourhood models,
 	$\propmodel, w \models \prop{\neg{\psi_1}}$ iff $\propmodel, w \not\models \prop{{\psi_1}}$. By the inductive hypothesis, Claim~\ref{cl:ind} holds for $\psi_1$.
 	By the contrapositive in each direction, $\propmodel, w \not\models \prop{{\psi_1}}$
 	iff $\Mmc, w \not\models \psi_1$. By the semantics of  $\MLnALCg$ neighbourhood models, $\Mmc, w \not\models \psi_1$ iff $\Mmc, w \models \neg\psi_1$.
\item $\psi=\psi_1\wedge\psi_2$: By the semantics of  propositional neighbourhood models,
$\propmodel, w \models \prop{{(\psi_1\wedge\psi_2)}}$ iff $\propmodel, w \models \prop{{\psi_1}}$ and $\propmodel, w \models \prop{{\psi_2}}$. By the inductive hypothesis, Claim~\ref{cl:ind} holds for $\psi_1,\psi_2$.
So, $\propmodel, w \models \prop{{\psi_i}}$
iff $\Mmc, w \models \psi_i$, for $i\in \{1,2\}$. By the semantics of  $\MLnALCg$ neighbourhood models, $\Mmc, w \models \psi_1$ and $\Mmc, w \models \psi_2$ iff $\Mmc, w \models \psi_1\wedge \psi_2$.
\item $\psi=\B_{i} \psi_1$: By the semantics of  propositional neighbourhood models,
$\propmodel, w \models \prop{{(\B_{i} \psi_1)}}$ iff $\llbracket \prop{{\psi_1}}\rrbracket^{\propmodel} \in \Nmc_{i}(w)$ where
$\llbracket \prop{{\psi_1}} \rrbracket^{\propmodel} = \{ v \in \Wmc \mid \propmodel, v \models \prop{{\psi_1}} \}$. By the inductive hypothesis, Claim~\ref{cl:ind} holds for $\psi_1$.
So, $\propmodel, v \models \prop{{\psi_1}}$
iff $\Mmc, v \models \psi_1$, for every $v\in\Wmc$. 
Thus, $\llbracket \prop{{\psi_1}} \rrbracket^{\propmodel}=\llbracket {{\psi_1}} \rrbracket^{\Mmc}$. By definition of $\propmodel$ and \Mmc, we have that $\Nmc_{i}(w)$
is the same in both $\propmodel$ and \Mmc, for every $w\in\Wmc$ and $i\in J$.
So $\llbracket \prop{{\psi_1}}\rrbracket^{\propmodel} \in \Nmc_{i}(w)$
iff $\llbracket {{\psi_1}} \rrbracket^{\Mmc} \in \Nmc_{i}(w)$.
By the semantics of  $\MLnALCg$ neighbourhood models, $\llbracket {{\psi_1}} \rrbracket^{\Mmc} \in \Nmc_{i}(w)$ iff $\Mmc, w \models \B_{i} \psi_1$.
 \end{itemize}
We have thus shown that for every subformula $\psi$ of $\varphi$ and every $w \in \Wmc$, we have
$\propmodel, w \models \prop{\psi}$ iff
$\Mmc, w \models \psi$.
\end{proof}
Since $\propmodel , v \models \prop{\varphi}$, for some $v \in \Wmc$, we conclude that $\varphi$ is $\LnALCg$ satisfiable. 
%}
\end{proof}














\Lemmapropvardi*
\begin{proof}
{{In this proof, for any set $S\subseteq\{\mathbf{E,M,C,N,T,P,Q,D}\}$, we call $S$ model any neighbourhood model satisfying all conditions in $S$.}}
	We start with proving ($\Rightarrow$). 
	We consider the more complex case where $\mathbf{C}\in\Lvar$.
	For $\mathbf{C}\not\in\Lvar$ the proof simplifies by taking $k = 1$.
	Suppose that $\phi$ is satisfied in a world $w$ of a $\setsymbols$-consistent $\Lvar^{n}$ model 
	$\propmodel = (\propdomain, \{ \propneigh_{i} \}_{i \in J}, \propassign)$. That is, 
	$\propmodel, w\models \phi$. We define a $\setsymbols$-consistent valuation for 
	$\phi$
	by setting, for all $\psi \in {\sf sub}(\phi)$,
	$\nu(\psi)=1$ if $\propmodel, w\models \psi$ and $\nu(\psi) = 0$
	if  $\propmodel, w\not\models \psi$. 
	It is easy to check that $\nu$ is indeed a 
	$\setsymbols$-consistent valuation   (given that $\propmodel$ is a  
			$\setsymbols$-consistent $\Lvar^{n}$ model).   
	%
	Now assume that $\B_i\psi_1, \dots, \B_i\psi_k, \B_i\chi\in{\sf sub}(\phi)$,
	$\valuation(\B_i\psi_j)=1$ for all $1\leq j \leq k$,
	%$\B_i\chi\in{\sf sub}(\phi)$, 
	and $\valuation(\B_i\chi)=0$.
	Then $\propmodel, w\models \B_i\psi_1 \land ... \land \B_i\psi_n\land\neg\B_i\chi$.
	Since $\propmodel$ is a $\mathbf{EC}$ model, this means that
	$\propmodel\not\models \psi_1\land ... \land \psi_n \leftrightarrow \chi$, that is,
	there is a worlds $u$ such that 
	$\propmodel, u\models (\bigwedge^{k}_{j=1}\psi_j\wedge\neg\chi) \vee \bigvee^{k}_{j=1} (\neg\psi_j\wedge\chi)$.
	(If $\mathbf{M}\in\Lvar$, then 
	$\propmodel\not\models \psi_1\land ... \land \psi_n \to \chi$, that is,
	there $u$ such that 
	$\propmodel, u\models (\bigwedge^{k}_{j=1}\psi_j\wedge\neg\chi)$.)
	Since $\propmodel$ is $\setsymbols$-consistent this concludes the proof.
	Now we prove that $\nu$ satifies $(\mathbf{X})$ if $\mathbf{X}\in\Lvar$, for $\mathbf{X}\in\{\mathbf{N,T,P,Q,D}\}$.
	\begin{itemize}
		\item[($\mathbf{N}$)]
		%($N$) 
		If $\nu(\B_i\psi)=0$, then $\propmodel, w\not\models \B_i\psi$.
		Since $\propmodel$ is a $\mathbf{EN}$ model, this means that $\propmodel\not\models\psi$
		(otherwise $\propmodel\models\B_i\psi$),
		that is there is $u$ such that $\propmodel, u \models \neg\psi$.
		
		\item[($\mathbf{T}$)]
		%($T$)
		If $\nu(\B_i\psi)=1$, then $\propmodel, w\models \B_i\psi$, thus since $\propmodel$ is a $\mathbf{ET}$ model, $\propmodel\models\B_i\psi\to\psi$,
		hence $\propmodel, w \models \psi$, that is $\nu(\psi)=1$.
		
		\item[($\mathbf{P}$)]
		%($P$)
		If $\valuation(\B_i\psi_1) = ... = \valuation(\B_i\psi_k) = 1$, 
		then $\propmodel, w \models \B_i\psi_1\land ... \land \B_i\psi_k$.
		Since $\propmodel$ is a $\mathbf{ECP}$ model, $\propmodel, w \models \B_i(\psi_1\land ... \land \psi_k)$,
		and $\propmodel\models\neg\B_i\falseprop$.
		Then $\propmodel\not\models \psi_1\land ... \land \psi_k \leftrightarrow \falseprop$,
		thus there is $u$ such that $\propmodel, u \models \psi_1\land ... \land \psi_k$.
		
		\item[($\mathbf{Q}$)]
		%($Q$)
		If $\valuation(\B_i\psi_1) = ... = \valuation(\B_i\psi_k) = 1$, 
		then $\propmodel, w \models \B_i\psi_1\land ... \land \B_i\psi_k$.
		Since $\propmodel$ is a $\mathbf{ECQ}$ model, $\propmodel, w \models \B_i(\psi_1\land ... \land \psi_k)$,
		and $\propmodel\models\neg\B_i(\trueprop)$.
		Then $\propmodel\not\models \psi_1\land ... \land \psi_k \leftrightarrow \trueprop$,
		thus there is $u$ such that $\propmodel, u \models \neg\psi_1\lor ...\lor\neg\psi_k$.
		
		\item[($\mathbf{D}$)]
		%($D$)
		If $\valuation(\B_i\psi_j)= \valuation(\B_i\chi_\ell)=1$ for all $1\leq j \leq k$, $1\leq \ell \leq h$,
		then $\propmodel, w \models \bigwedge^{k}_{j=1}\B_i\psi_j \land \bigwedge^{h}_{\ell=1}\B_i\chi_\ell$.
		Since $\propmodel$ is a $\mathbf{ECD}$ model, 
		$\propmodel, w \models \B_i(\psi_i\land ... \land \psi_k) \land \B_i(\chi_1\land ... \land \chi_h)$, and
		$\propmodel\models\B_i\zeta\to\neg\B_i\neg\zeta$.
		Then $\propmodel\not\models \psi_1\land ... \land \psi_k \leftrightarrow \neg(\chi_1\land ... \land \chi_h)$,
		thus there is $u$ such that 
		$\propmodel, u \models (\psi_1\land ... \land \psi_k \land \chi_1\land ... \land \chi_h) \lor 
		(\neg(\psi_1\land ... \land \psi_k) \land \neg(\chi_1\land ... \land \chi_h))$.
		If $\mathbf{M}\in\Lvar$, then 
		$\propmodel\not\models \psi_1\land ... \land \psi_k \to \neg(\chi_1\land ... \land \chi_h)$,
		hence there is $u$ such that 
		$\propmodel, u \models \psi_1\land ... \land \psi_k \land \chi_1\land ... \land \chi_h$.
	\end{itemize}
	
	The proof of the converse ($\Leftarrow$) is as follows. 
	Suppose there is a $\setsymbols$-consistent valuation $\nu$ for $\phi$
	satisfying the conditions stated by the lemma.
	We construct a $\Lvar^{n}$ model $\propmodel$ and a world $w$ such that $\propmodel,w\models\phi$.
	By the condition, it follows that for all sets $\Psi$ of formulas
	$\B_i\psi_1, \dots, \B_i\psi_k$  in ${\sf sub}(\phi)$
	such that $\valuation(\B_i\psi_j)=1$ for all $1\leq j \leq k$,
	and all $\B_i\chi$ in ${\sf sub}(\phi)$ such that $\valuation(\B_i\chi)=0$, 
	there is a $\setsymbols$-consistent model \[\propmodel_{\Psi,\chi}=(\propdomain_{\Psi,\chi},
	\{ \propneigh_{{(\Psi,\chi)}_{i}} \}_{i \in J},\propassign_{\Psi,\chi})\]
	and a world 
	$w_{\Psi,\chi}\in \propdomain_{\Psi,\chi}$ such that 
	$\propmodel_{\Psi,\chi},w_{\Psi,\chi}\models(\bigwedge^{k}_{j=1}\psi_j\wedge\neg\chi) \vee \boldsymbol\vartheta$; moreover 
	if $\mathbf{X}\in\Lvar$, for $\mathbf{X}\in\{\mathbf{N,P,Q,D}\}$, the following hold: 
	\begin{itemize}
		\item[$(\mathbf{N})$]
		%$(N)$ 
		for all $\B_i\psi$ in ${\sf sub}(\phi)$ such that $\valuation(\B_i\psi)=0$, 
		there is a $\setsymbols$-consistent $\Lvar^{n}$ model 
		$\propmodel_{\psi}=(\propdomain_{\psi},  \{ \propneigh_{{\psi}_{i}} \}_{i \in J},\propassign_{\psi})$
		and a world 
		$w_{\psi}\in \propdomain_{\psi}$ such that 
		$\propmodel_{\psi}, w_{\psi} \models \neg\psi$;
		%\todo{should we talk about T? T: No, it is fine this way, because for T there is no additional model to consider.}
		\item[$(\mathbf{P})$]
		%$(P)$ 
		for all $\Psi = \{\B_i\psi_1, \dots, \B_i\psi_k\}\subseteq{\sf sub}(\phi)$
		such that $\valuation(\B_i\psi_j)=1$ for all $1\leq j \leq k$,
		there is a $\setsymbols$-consistent $\Lvar^{n}$ model 
		$\propmodel_{\Psi}=(\propdomain_{\Psi},  \{ \propneigh_{{\Psi}_{i}} \}_{i \in J},\propassign_{\Psi})$
		and a world 
		$w_{\Psi}\in \propdomain_{\Psi}$ such that 
		$\propmodel_{\Psi}, w_{\Psi} \models \psi_1\land...\land\psi_k$;
		
		\item[$(\mathbf{Q})$]
		% $(Q)$ 
		for all $\Psi = \{\B_i\psi_1, \dots, \B_i\psi_k\}\subseteq{\sf sub}(\phi)$
		such that $\valuation(\B_i\psi_j)=1$ for all $1\leq j \leq k$,
		there is a $\setsymbols$-consistent $\Lvar^{n}$ model 
		$\propmodel_{\Psi}=(\propdomain_{\Psi},  \{ \propneigh_{{\Psi}_{i}} \}_{i \in J},\propassign_{\Psi})$
		and a world 
		$w_{\Psi}\in \propdomain_{\Psi}$ such that 
		$\propmodel_{\Psi}, w_{\Psi} \models \neg\psi_1\lor ... \lor\neg\psi_k$;
		
		\item[$(\mathbf{D})$]
		%$(D)$ 
		for all $\Psi = \{\B_i\psi_1, \dots, \B_i\psi_k\}$, $\Lambda = \{\B_i\chi_1, \dots, \B_i\chi_h\}$,
		$\Psi,\Lambda\subseteq{\sf sub}(\phi)$
		such that 
		$\valuation(\B_i\psi_j)=\valuation(\B_i\chi_\ell)=1$ for all $1\leq j \leq k$,  $1\leq \ell \leq h$,
		there is a $\setsymbols$-consistent $\Lvar^{n}$ model 
		$\propmodel_{\Psi,\Lambda}=(\propdomain_{\Psi,\Lambda},  \{ \propneigh_{{(\Psi,\Lambda)}_{i}} \}_{i \in J},\propassign_{\Psi,\Lambda})$
		and a world 
		$w_{\Psi,\Lambda}\in \propdomain_{\Psi,\Lambda}$ such that 
		$\propmodel_{\Psi,\Lambda}, w_{\Psi,\Lambda} \models (\bigwedge^{k}_{j=1}\psi_j \land \bigwedge^{h}_{\ell=1}\chi_\ell) \vee \boldsymbol\eta$.
	\end{itemize}
	
	Let $\propmodel_1, ..., \propmodel_m$
	be an enumeration of all $\Lvar^{n}$ models listed above,
	where 
	$\propmodel_j = (\propdomain_j, \{ \propneigh_{j_{i}} \}_{i \in J},\propassign_j)$.
	That is, we take one model $\propmodel_{\Psi,\chi}$ 
	% for all $\Psi = \{\B_i\psi_1, \dots, \B_i\psi_k\}\subseteq{\sf sub}(\phi)$
	% such that $\valuation(\B_i\psi_j)=1$ for all $1\leq j \leq k$,
	% and all $\B_i\chi$ in ${\sf sub}(\phi)$ such that $\valuation(\B_i\chi)=0$;
	for each pair $(\Psi,\B_i\chi)$,
	where $\Psi = \{\B_i\psi_1, \dots, \B_i\psi_k\}\subseteq{\sf sub}(\phi)$,
	$\valuation(\B_i\psi_j)=1$ for all $1\leq j \leq k$,
	$\B_i\chi$ in ${\sf sub}(\phi)$, and
	$\valuation(\B_i\chi)=0$;
	and similarly 
	we take one model $\propmodel_\psi$, $\propmodel_\Psi$, or $\propmodel_{\Psi,\Lambda}$
	for all formulas or sets of formulas 
	%for all models 
	listed in items $(\mathbf{N})$, $(\mathbf{P})$, $(\mathbf{Q})$, $(\mathbf{D})$. %\todo{T?}
	Assume without loss of generality that 
	$\propdomain_j\cap \propdomain_\ell=\emptyset$ 
	for $j\neq \ell$. 
	%
	We define a $\setsymbols$-consistent $\Lvar^{n}$ model   
	$\propmodel = (\propdomain,\{ \propneigh_{i} \}_{i \in J}, \propassign)$ for $\phi$
	as follows.
	\begin{itemize}
		\item $\propdomain = \bigcup_{j = 1}^{m} \propdomain_j \cup \{w\}$, where $w$ is a new world.
		
		\item %Let $\ext{\cdot}$ be 
		Consider a function $\ext{\cdot}: {\sf sub}(\phi)\rightarrow \Pmc(\Wmc)$
		with $\ext{\psi}=\bigcup_{j = 1}^{m} \llbracket \psi \rrbracket^{\propmodel_j} \cup \ext{\psi}_0$ for all $\psi\in {\sf sub}(\phi)$, where %$I_i$ is as above for $1\leq i\leq n$, 
		%and
		$\ext{\cdot}_0: {\sf sub}(\varphi)\rightarrow  \Pmc(\{w\})$ is the function
		that assigns $\psi$ to $\{w\}$, if $\nu(\psi)=1$, 
		and to $\emptyset$, otherwise.
		% ($\Vmc_j$, for $1\leq j\leq m$, is as above).
		By construction, we have that $\ext{\neg \psi}=\propdomain\setminus \ext{\psi}$
		and $\ext{\psi_1\wedge \psi_2} =\ext{\psi_1}\cap \ext{\psi_2}$. 
		We define the assignment $\propassign$ as the function 
		$\propassign: \NPr(\varphi)\rightarrow \Pmc(\Wmc)$ satisfying 
		$\propassign(p_\elaxiom)=\ext{p_\elaxiom}$ for all $p_\elaxiom\in \NPr(\varphi)$. 
		
		\item It remains to define $\propneigh_i$, for $1 \leq i \leq n$.
		%We distinguish two cases. (i) 
		For $u\in \propdomain_j$,
		we define $\alpha\in\propneigh_i(u)$ if and only if 
		there is $\B_i\psi$ in ${\sf sub}(\phi)$ such that
		$\propmodel_j, u \models \B_i\psi$ and $\ext{\psi} = \alpha$;
		%(2) 
		and
		we define $\alpha\in\propneigh_i(w)$ if and only if 
		there is $\B_i\psi$ in ${\sf sub}(\phi)$ such that
		$\valuation(\B_i\psi)=1$ and $\ext{\psi} = \alpha$.
		% 
		Then if $\mathbf{C}\in\Lvar$,
		we close $\propneigh_i$ under intersection, 
		if $\mathbf{M}\in\Lvar$,
		we close $\propneigh_i$ under supersets,
		and if $\mathbf{N}\in\Lvar$,
		we extend $\propneigh_i(u)$ with $\propdomain$ for all
		$u\in\propdomain$,
		so that $\propmodel$ is a $\EC$, respectively a $\EM$,
		respectively a $\EN$, model.
	\end{itemize}
	
	We prove the following claim which ensures that
	$\propneigh_i$ is well-defined.
	
	\begin{claim}
		(i) For $u\in\propdomain_j$, if $\beta \in \Nmc_i(u)$ and 
		$\beta = \ext{\chi}$ for some 
		$\B_i\chi$ in ${\sf sub}(\phi)$,
		then $\propmodel_j,u\models\B_i\chi$.
		(ii) If $\beta \in \Nmc_i(w)$ and  $\beta = \ext{\chi}$ for some 
		$\B_i\chi$ in ${\sf sub}(\phi)$,
		then $\valuation(\B_i\chi) = 1$.
	\end{claim}
	\begin{proof}[Proof of Claim]
		We consider the case where $\mathbf{C},\mathbf{N}\in\Lvar$ and $\mathbf{M}\notin\Lvar$,
		for the other cases the proof can be easily adapted.
		
		(i) If $\beta \in \Nmc_i(u)$, then by definition 
		$\beta=\propdomain$, or
		$\beta = \bigcap_{\ell=1}^k \ext{\chi_\ell}$ for some $\B_i\chi_1, ..., \B_i\chi_k$ in ${\sf sub}(\phi)$ such that
		$\propmodel_j,u\models\bigwedge_{\ell=1}^k\B_i\chi_\ell$.
		%or $\beta=\propdomain$.
		If $\beta=\propdomain$, then $\ext{\chi}=\propdomain$,
		%thus $\llbracket \chi \rrbracket^{\propmodel_\ell} = \propdomain_\ell$ for all $\propmodel_\ell$,
		thus in particular 
		$\llbracket \chi \rrbracket^{\propmodel_j} = \propdomain_j$,
		and since $\propdomain_j\in\propneigh_{j_{i}}(u)$
		it holds
		$\propmodel_j, u \models \B_i\chi$.
		Otherwise
		$\ext{\chi} = \bigcap_{\ell=1}^k \ext{\chi_\ell}$,
		which implies 
		$\llbracket \chi \rrbracket^{\propmodel_j} =
		\bigcap_{\ell=1}^k\llbracket \chi_\ell \rrbracket^{\propmodel_j}$
		(because $\propdomain_j \cap \propdomain_k = \emptyset$ for  
		$k \neq j$).
		Since $\propmodel_j$ is a $\mathbf{EC}$ model,
		$\propmodel_j,u\models\B_i\bigwedge_{\ell=1}^k\chi_\ell$,
		then 
		%$\llbracket \bigwedge_{\ell=1}^k\chi_\ell \rrbracket^{\propmodel_j} =
		%\bigcap_{\ell=1}^k\llbracket \chi_\ell \rrbracket^{\propmodel_j} =
		%\llbracket \chi \rrbracket^{\propmodel_j}
		%\in\propneigh_{j_{i}}(u)$, 
		$\llbracket \bigwedge_{\ell=1}^k\chi_\ell \rrbracket^{\propmodel_j} 
		\in\propneigh_{j_{i}}(u)$,
		where
		$\llbracket \bigwedge_{\ell=1}^k\chi_\ell \rrbracket^{\propmodel_j} =
		\bigcap_{\ell=1}^k\llbracket \chi_\ell \rrbracket^{\propmodel_j} =
		\llbracket \chi \rrbracket^{\propmodel_j}$,
		therefore
		$\propmodel_j,u\models\B_i\chi$.
		
		(ii) If $\beta \in \Nmc_i(u)$, then by definition 
		$\beta=\propdomain$, or
		$\beta = \bigcap_{\ell=1}^k \ext{\chi_\ell}$ for some $\B_i\chi_1, ..., \B_i\chi_k$ in ${\sf sub}(\phi)$ such that
		$\valuation(\chi_\ell) = 1$ for all $1 \leq \ell \leq k$.
		If $\beta=\propdomain$, then $\ext{\chi} = \propdomain$,
		thus $\ext{\chi}_0 = \{w\}$, that is $\valuation(\chi) = 1$.
		By contradiction, suppose that 
		$\valuation(\B_i\chi) = 0$.
		Then by $(N)$, there are a $\setsymbols$-consistent $\Lvar^{n}$ model 
		$\propmodel_{\chi}$  and a world  $w_{\chi}$ such that 
		$\propmodel_{\chi}, w_{\chi} \not\models \chi$.
		One such model is enumerated among 
		$\propmodel_1, ..., \propmodel_m$, let it be $\propmodel_o$.
		Then $\llbracket \chi \rrbracket^{\propmodel_o} \not= \propdomain_o$, 
		thus $\ext{\chi} \not= \propdomain$, giving a contradiction.
		Therefore $\valuation(\B_i\chi) = 1$.
		If instead $\beta\not=\propdomain$,
		then $\ext{\chi} = \bigcap_{\ell=1}^k \ext{\chi_\ell}$.
		Suppose that $\valuation(\chi) = 0$.
		By the hypothesis of the lemma,
		there are a $\setsymbols$-consistent model $\propmodel_{\Lambda,\chi}$ and a world 
		$w_{\Lambda,\chi}$ such that 
		$\propmodel_{\Lambda,\chi},w_{\Lambda,\chi}\models(\bigwedge^{k}_{\ell=1}\chi_\ell\wedge\neg\chi) \vee \bigvee^{k}_{\ell=1} (\neg\chi_\ell\wedge\chi)$,
		where $\Lambda = \{\B_i\chi_1,...,\B_i\chi_k\}$.
		Then one such model is enumerated among 
		$\propmodel_1, ..., \propmodel_m$, let it be $\propmodel_o$.
		Thus 
		$\llbracket \chi \rrbracket^{\propmodel_o} \not=
		\llbracket \bigwedge_{\ell=1}^k\chi_\ell \rrbracket^{\propmodel_o}$, where
		$\llbracket \bigwedge_{\ell=1}^k\chi_\ell \rrbracket^{\propmodel_\ell} =
		\bigcap_{\ell=1}^k\llbracket \chi_\ell \rrbracket^{\propmodel_\ell}$.
		Since  $\propdomain_j \cap \propdomain_k = \emptyset$ for  
		$k \neq j$,  
		this implies $\ext{\chi} \not = \bigcap_{\ell=1}^k \ext{\chi_\ell}$,
		giving a contradiction.
		Therefore $\valuation(\B_i\chi) = 1$. 
	\end{proof}
	
	\begin{claim}
		For all $\psi\in{\sf sub}(\phi)$,
		%and all $u\in\propdomain$, $u \in \llbracket \psi \rrbracket^{\propmodel}$ if and only if $u\in \ext{\psi}$.
		$\llbracket \psi \rrbracket^{\propmodel} = \ext{\psi}$.
	\end{claim}
	\begin{proof}[Proof of Claim]
		By induction on the structure of formulas.
		For $p_\elaxiom\in \NPr(\varphi)$,
		$\llbracket p_\elaxiom \rrbracket^{\propmodel}=\ext{p_\elaxiom}$
		by definition of $\propassign$.
		For boolean connectives, the claim follows immediately from the inductive hypothesis and the fact
		that $\ext{\neg \chi} =\propdomain\setminus \ext{\chi}$ and $\ext{\chi_1\wedge \chi_2} = \ext{\chi_1}\cap \ext{\chi_2}$.
		%
		Suppose  that $u\in \ext{\B_i\chi}$.  
		Then, either $u=w$ and $\nu(\B_i\chi)=1$
		or $u\in \propdomain_j$ and $\propmodel_j,u\models\B_i\chi$. 
		By definition of $\propneigh_i$, in either case 
		we have that $\ext{\chi}\in \propneigh_i(u)$. By inductive hypothesis,
		$\llbracket \chi \rrbracket^{\propmodel} = \ext{\chi}$, it follows that $\propmodel,u\models \B_i\chi$, 
		that is, $u\in \llbracket \B_i\chi \rrbracket^{\propmodel}$. 
		%
		Suppose now that $u\in \llbracket \B_i\chi \rrbracket^{\propmodel}$, 
		that is, $\propmodel,u\models \B_i\chi$, or, equivalently,  $\llbracket \chi \rrbracket^{\propmodel}\in \propneigh_i(u)$.
		By inductive hypothesis,
		$\llbracket \chi \rrbracket^{\propmodel} = \ext{\chi}$, then
		by the previous claim, if $u=w$, then $\valuation(\B_i\chi)=1$,
		and if $u\in \propdomain_j$, then $\propmodel_j,u\models\B_i\chi$. 
		By definition of $\ext{\cdot}$, in either case we have that $u\in \ext{\B_i\psi}$. 
	\end{proof}
	
	
	\begin{claim}
		%  If $\mathbf{X}\in\Lvar$, for $\mathbf{X}\in\{\mathbf{M,C,N,T,P,Q,D}\}$,
		%  then $\propmodel$ is a $\mathbf{EX^*}$ model.
		For $\mathbf{X}\in\{\mathbf{M,C,N,T,P,Q,D}\}$,
		if $\mathbf{X}\in\Lvar$,
		then $\propmodel$ satisfies the $\mathbf{X}$-condition.
	\end{claim}
	\begin{proof}[Proof of Claim]
		For $\mathbf{X}\in\{\mathbf{M,C,N}\}$, 
		% that $\propmodel$ 
		% %is a $\mathbf{EX^*}$ model 
		% satisfies the $X$-condition
		% if $\mathbf{X}\in\Lvar$
		% is an immediate consequence of 
		the claim follows immediately from
		the definition of $\propneigh_i$.
		We consider the other cases,
		assuming %as before 
		that 
		% $\mathbf{C},\mathbf{N}\in\mathbf{L}$ and $\mathbf{M}\notin\mathbf{L}$
		% ($\mathbf{C}\notin\mathbf{L}$, $\mathbf{N}\notin\mathbf{L}$, or $\mathbf{M}\in\mathbf{L}$ 
		% the proof can be easily adapted).
		$\mathbf{C}\in\Lvar$ and $\mathbf{M}\notin\Lvar$
		(for $\mathbf{C}\notin\Lvar$ or $\mathbf{M}\in\Lvar$ the proof can be easily adapted).
		
		\begin{itemize}
			\item[$(\mathbf{T})$]  Suppose that $\alpha\in\propneigh_i(u)$.
			Then $\alpha = \propdomain$ (if $\mathbf{N}\in\Lvar$), 
			which implies $u\in\alpha$,
			or $\alpha = \bigcap_{\ell = 1}^k \ext{\psi_\ell}$ for $\B_i\psi_1, ..., \B_k\psi \in{\sf sub}(\phi)$.
			If $u\in\propdomain_j$, then $\propmodel_j, u \models \B_i\psi_1 \land ... \land \B_i\psi_k$. 
			Since $\propmodel_j$ is a $\mathbf{ET}$ model,
			$\propmodel_j\models \B_i\chi\to\chi$, thus $\propmodel_j, u \models \psi_1 \land ... \land \psi_k$,
			that is $u\in\llbracket \psi_\ell \rrbracket^{\propmodel_j}$ for all $1\leq \ell \leq k$.
			It follows $u\in \bigcap_{\ell = 1}^k \ext{\psi_\ell} = \alpha$. 
			If instead $u = w$, then $\valuation(\B_i\psi_\ell) = 1$  for all $1\leq \ell \leq k$.
			By the hypothesis of the proposition, $\valuation(\psi) = 1$ for all $1\leq \ell \leq k$, 
			thus $\ext{\psi_\ell}_0 = \{w\}$  for all $1\leq \ell \leq k$.
			therefore $u = w \in \bigcap_{\ell = 1}^k \ext{\psi_\ell} = \alpha$. 
			
			%[Proof without C]
			% If $u\in\propdomain_j$, then $\alpha = \ext{\psi}$ for a $\B_i\psi\in{\sf sub}(\prop{\varphi})$ such that
			% $\propmodel_j, u \models \B_i\psi$. Since $\propmodel_j$ is a $\mathbf{ET^*}$ model,
			% $\propmodel_j\models \B_i\psi\to\psi$, thus $\propmodel_j, u \models \psi$,
			% that is $u\in\llbracket \psi \rrbracket^{\propmodel_j}$.
			% It follows $u\in \ext{\psi} = \alpha$.
			% If instead $u = w$, then $\alpha = \ext{\psi}$ for a $\B_i\psi\in{\sf sub}(\prop{\varphi})$ such that
			% $\valuation(\B_i\psi) = 1$.
			% By the hypothesis of the proposition, $\valuation(\psi) = 1$, thus $\ext{\psi}_0 = \{w\}$,
			% therefore $u = w \in \ext{\psi} = \alpha$.
			
			\item[$(\mathbf{P})$]  Assume that $\emptyset\in\propneigh_i(u)$.
			Then $\emptyset = \bigcap_{\ell = 1}^k \ext{\psi_\ell}$ for $\B_i\psi_1, ..., \B_k\psi \in{\sf sub}(\phi)$.
			If $u\in\propdomain_j$, then $\propmodel_j, u \models \B_i\psi_1 \land ... \land \B_i\psi_k$,
			that is $\llbracket \psi_\ell \rrbracket^{\propmodel_j}\in\propneigh_{j_i}(u)$ for all $1\leq \ell \leq k$.
			By the $\mathbf{C}$-condition, $\bigcap_{\ell = 1}^k \llbracket \psi_\ell \rrbracket^{\propmodel_j} \in\propneigh_{j_i}(u)$,
			and by construction of $J$, 
			$\bigcap_{\ell = 1}^k \llbracket \psi_\ell \rrbracket^{\propmodel_j} = \emptyset$,
			contradicting the fact that $\propmodel_j$ is a $\mathbf{EP}$ model.
			If instead $u = w$, then $\valuation(\B_i\psi_\ell) = 1$  for all $1\leq \ell \leq k$.
			By item $(\mathbf{P})$ above, there are  $\propmodel_{\Psi}$ and  $w_{\Psi}$ such that 
			$\propmodel_{\Psi}, w_{\Psi} \models \psi_1\land...\land\psi_k$.
			One such model is enumerated among 
			$\propmodel_1, ..., \propmodel_m$, let it be $\propmodel_o$.
			Then $\llbracket \bigwedge_{\ell=1}^k\psi_\ell \rrbracket^{\propmodel_o} =
			\bigcap_{\ell = 1}^k \llbracket \psi_\ell \rrbracket^{\propmodel_o} \not=\emptyset$,
			thus $\bigcap_{\ell = 1}^k \ext{\psi_\ell}  \not=\emptyset$. 
			In either case $\emptyset\notin\propneigh_i(u)$.
			
			\item[$(\mathbf{Q})$]  Suppose that $\alpha\in\propneigh_i(u)$.
			Then $\alpha = \bigcap_{\ell = 1}^k \ext{\psi_\ell}$ for $\B_i\psi_1, ..., \B_k\psi \in{\sf sub}(\phi)$.
			If $u\in\propdomain_j$, then $\propmodel_j, u \models \B_i\psi_1 \land ... \land \B_i\psi_k$,
			that is $\llbracket \psi_\ell \rrbracket^{\propmodel_j}\in\propneigh_{j_i}(u)$ for all $1\leq \ell \leq k$.
			Since $\propmodel_j$ is a $\mathbf{EQ}$ model, $\llbracket \psi_\ell \rrbracket^{\propmodel_j} \not = \propdomain_j$,
			then $\bigcap_{\ell = 1}^k \ext{\psi_\ell}\not=\propdomain$.
			%
			If instead $u = w$, then $\valuation(\B_i\psi_\ell) = 1$  for all $1\leq \ell \leq k$.
			By item $(\mathbf{Q})$ above, there are  $\propmodel_{\Psi}$ and  $w_{\Psi}$ such that 
			$\propmodel_{\Psi}, w_{\Psi} \models \neg\psi_1\lor ... \lor\neg\psi_k$.
			One such model is enumerated among 
			$\propmodel_1, ..., \propmodel_m$, let it be $\propmodel_o$.
			Then $\llbracket \neg\psi_\ell \rrbracket^{\propmodel_o} \not=\emptyset$ for some $\B_i\psi_\ell$,
			that is $\llbracket \psi_\ell \rrbracket^{\propmodel_o} \not=\propdomain$,
			therefore $\bigcap_{\ell = 1}^k \llbracket \psi_\ell \rrbracket^{\propmodel} \not=\propdomain$.
			In either case $\alpha\not=\propdomain$, that is $\propdomain\notin\propneigh_i(u)$.
			
			\item[$(\mathbf{D})$]  Suppose that $\alpha,\beta\in\propneigh_i(u)$.
			Then $\alpha = \bigcap_{\ell = 1}^k \ext{\psi_\ell}$ for $\B_i\psi_1, ..., \B_i\psi_k \in{\sf sub}(\phi)$, and
			$\beta = \bigcap_{{\ell'} = 1}^k \ext{\chi_{\ell'}}$ for $\B_i\chi_1, ..., \B_i\chi_h \in{\sf sub}(\phi)$.
			If $u\in\propdomain_j$, then 
			$\propmodel_j, u \models \B_i\psi_1 \land ... \land \B_i\psi_k \land \B_i\chi_1 \land ... \land \B_i\chi_h$.
			Since $\propmodel_j$ is a $\mathbf{EC}$ model,
			$\propmodel_j, u\models \B_i \bigwedge_{\ell=1}^k\psi_\ell \land \B_i\bigwedge_{{\ell'}=1}^h\chi_{\ell'}$, and
			since $\propmodel_j$ is a $\mathbf{ED}$ model,
			$\propmodel_j\not\models \bigwedge_{\ell=1}^k\psi_\ell \leftrightarrow \neg\bigwedge_{{\ell'}=1}^h\chi_{\ell'}$,
			% That is, there is $v$ in $\propdomain_j$ such that 
			% $\propmodel_j, v \models (\bigwedge_{j=1}^k\psi_j \land \bigwedge_{{\ell'}=1}^h\chi_{\ell'}) \lor (\neg\bigwedge_{j=1}^k\psi_j \land \neg\bigwedge_{{\ell'}=1}^h\chi_{\ell'})$
			that is 
			$\bigcap_{\ell = 1}^k \llbracket \psi_\ell \rrbracket^{\propmodel_j} =
			\llbracket \bigwedge_{\ell=1}^k\psi_\ell \rrbracket^{\propmodel_j} \not=
			\llbracket \neg\bigwedge_{{\ell'}=1}^h\chi_{\ell'} \rrbracket^{\propmodel_j} =
			\propdomain_j \setminus \bigcap_{{\ell'}=1}^h \llbracket \chi_{\ell'} \rrbracket^{\propmodel_j}$.
			%
			If instead $u = w$, 
			then  $\valuation(\B_i\psi_\ell)=\valuation(\B_i\chi_{\ell'})=1$ for all $1\leq \ell \leq k$,  $1\leq \ell' \leq h$.
			By item $(\mathbf{D})$ above, there are
			$\propmodel_{\Psi,\Lambda}$ and $w_{\Psi,\Lambda}$ such that
			$\propmodel_{\Psi,\Lambda}, w_{\Psi,\Lambda} \models (\bigwedge^{k}_{\ell=1}\psi_\ell \land \bigwedge^{h}_{\ell'=1}\chi_{\ell'}) \vee (\neg(\bigwedge^{k}_{\ell=1}\psi_\ell) \land \neg(\bigwedge^{h}_{\ell'=1}\chi_{\ell'}))$.
			One such model is enumerated among 
			$\propmodel_1, ..., \propmodel_m$, let it be $\propmodel_o$.
			Then $\bigcap_{\ell = 1}^k \llbracket \psi_\ell \rrbracket^{\propmodel_o} =
			\llbracket \bigwedge_{\ell=1}^k\psi_\ell \rrbracket^{\propmodel_o} \not=
			\llbracket \neg\bigwedge_{{\ell'}=1}^h\chi_{\ell'} \rrbracket^{\propmodel_o} =
			\propdomain_o \setminus \bigcap_{{\ell'}=1}^h \llbracket \chi_{\ell'} \rrbracket^{\propmodel_o}$.
			%
			Thus in either case, 
			$\bigcap_{\ell = 1}^k \ext{\psi_\ell} \not=\propdomain \setminus \bigcap_{{\ell'} = 1}^k \ext{\chi_{\ell'}}$,
			that is, $\alpha\not=\propdomain\setminus\beta$.\qedhere
		\end{itemize} 
	\end{proof}
	
	
	\begin{claim}
		$\propmodel$ is $\setsymbols$-consistent.
	\end{claim}
	\begin{proof}[Proof of Claim]
		$\nu$, used to construct the assignment 
		related to $w$, is $\setsymbols$-consistent and 
		the models $\propmodel_1,\ldots,\propmodel_m$, used to define 
		the remaining worlds in $\Wmc$, are all $\setsymbols$-consistent. 
	\end{proof}
	
	Finally, since $\valuation(\phi)=1$, we have that $w\in \ext{\phi}$, 
	and consequently $\propmodel, w \models \phi$.
	Given that $\propmodel$ is a $\setsymbols$-consistent $\Lvar^{n}$ model, this concludes the proof.
\end{proof}





















\begin{algorithm}[t]
	\KwIn{$L$,  $\setsymbols$, and an $\MLn$ formula $\phi$ built from $\setsymbols$.}
	\KwOut{$\mathsf{satisfiable}$, if $\psi$ is  satisfiable in a $\setsymbols$-consistent $L^n$ model; $\mathsf{unsatisfiable}$, otherwise.}
	\BlankLine
	%	$r:=  \mathsf{unsatisfiable}$\;
	
	\For{each $\setsymbols$-consistent valuation $\valuation$ for $\phi$}{
		\uIf{$\mathsf{Check}(L,\setsymbols,\valuation,\phi)=1$}{
			\uIf{$L\cap\{ \mathbf{N},\mathbf{T},\mathbf{P},\mathbf{Q}\} \neq \emptyset$}{
				\uIf{$\mathsf{CheckNTPQ}(L,\setsymbols,\valuation,\phi)=1$}{
					\uIf{$\mathbf{D}\notin L$}{
						\Return $\mathsf{satisfiable}$\;
					}\uElseIf{$\mathsf{CheckD}(L,\setsymbols,\valuation,\phi){=}1$}
						{\Return $\mathsf{satisfiable}$\;}
				}
			}\uElseIf{$\mathbf{D}\in L$, $\mathsf{CheckD}(L,\setsymbols,\valuation,\phi){=}1$
		}{\Return $\mathsf{satisfiable}$\;} 
		}
		
	}
	%}
\BlankLine
\Return $\mathsf{unsatisfiable}$\;
%	\uIf{$\T$ contains a clash}{\Return $\mathsf{unsatisfiable}$\;} 
%	\Else{\Return $\mathsf{satisfiable}$\;}
%\caption{$\LnALC$ tableau algorithm for $\p$}
\caption{$\mathsf{Sat}$}
%		: Decision procedure %$sat(\p)$
%		for  the $\LnALCg$  formula satisfiability problem on varying domain neighbourhood models}
\label{alg:propSAT}
\end{algorithm}
















\begin{algorithm}[t]
	\KwIn{$L$, $\setsymbols$, a $\setsymbols$-consistent valuation $\valuation$, and an $\MLn$ formula $\phi$ built from $\setsymbols$.}
	\KwOut{$\mathsf{1}$, if $\valuation$ satisfies the conditions of Lemma~\ref{lem:proplemmaL}; $0$, otherwise.}
	\BlankLine
	%	$r:=  \mathsf{unsatisfiable}$\;
	\uIf{$\mathbf{C} \in \Lvar$}{
		$\boldsymbol{\kappa}:= | {\sf sub}({\phi}) |$\;
	}
	\uElse{$\boldsymbol{\kappa}:=1$}
	\BlankLine
	%\For{each $\setsymbols$-consistent valuation $\valuation$ for $\phi$}{
		\For{all $1\leq k\leq \boldsymbol{\kappa}$}{
			\For{ $\B_i\psi_1, \dots, \B_i\psi_k, \B_i\chi\in{\sf sub}(\phi)$,
				with $\valuation(\B_i\psi_j)=1$ for all $1\leq j \leq k$,
				%$\B_i\chi\in{\sf sub}(\prop{\varphi})$, 
				and $\valuation(\B_i\chi)=0$}{ 
				\uIf{$\mathbf{M}\in L$}{ 
					\uIf{$\mathsf{Sat}(L,\setsymbols,\bigwedge^{k}_{j=1}\psi_j\wedge\neg\chi)= \mathsf{unsatisfiable}$}{\Return $0$\;} 
				}
				\uElseIf{ $\mathsf{Sat}(L,\setsymbols,(\bigwedge^{k}_{j=1}\psi_j\wedge\neg\chi)\vee (\bigvee^{k}_{j=1} (\neg\psi_j\wedge\chi)))=\mathsf{unsatisfiable}$\;}
				{\Return $0$\;}
			}		
		}
		\Return $1$\;
 		\BlankLine
		%	\uIf{$\T$ contains a clash}{\Return $\mathsf{unsatisfiable}$\;} 
		%	\Else{\Return $\mathsf{satisfiable}$\;}
		%\caption{$\LnALC$ tableau algorithm for $\p$}
		\caption{$\mathsf{Check}$}
		%		: Decision procedure %$sat(\p)$
		%		for  the $\LnALCg$  formula satisfiability problem on varying domain neighbourhood models}
	\label{alg:prop1}
\end{algorithm}











\begin{algorithm}[t]
	\KwIn{$L$, $\setsymbols$, a $\setsymbols$-consistent valuation $\valuation$, and an $\MLn$ formula $\phi$ built from $\setsymbols$.}
	\KwOut{$\mathsf{1}$, if $\valuation$ satisfies the conditions of  Lemma~\ref{lem:proplemmaL}; $0$, otherwise.}
	\BlankLine
	%	$r:=  \mathsf{unsatisfiable}$\;
	\uIf{$\mathbf{C} \in \Lvar$}{
		$\boldsymbol{\kappa}:= | {\sf sub}({\phi}) |$\;
	}
	\uElse{$\boldsymbol{\kappa}:=1$}
	\BlankLine
	%\For{each $\setsymbols$-consistent valuation $\valuation$ for $\phi$}{
		 
		 	%\For{all $1\leq k,h\leq \boldsymbol{\kappa}$}{
		\For{all $1\leq k\leq \boldsymbol{\kappa}$}{
				
				\uIf{$\mathbf{N}\in L$}{	
					\For{ $\B_i\psi\in{\sf sub}(\phi)$  
						with $\valuation(\B_i\psi)=0$}{	
						\uIf{$\mathsf{Sat}(L,\setsymbols,\neg \psi)= 	\mathsf{unsatisfiable}$}{\Return $0$\;} 
					}
					\Return $1$\;	
				}
				
			\uIf{$\mathbf{T}\in L$}{
				\For{$\B_i\psi\in{\sf sub}(\phi)$ with  
					$\valuation(\B_i\psi)=1$}{
					\uIf{$\valuation(\psi)=0$}{
						\Return $0$\;}
				}
			}
			\uIf{$\mathbf{P}\in L$}{
				\For{$\B_i\psi_1, \dots, \B_i\psi_k\in{\sf sub}(\phi)$ with
					$\valuation(\B_i\psi_j)=1$ for all $1\leq j \leq k$}{
					\uIf{$\mathsf{Sat}(L,\setsymbols,\bigwedge^{k}_{j=1}\psi_j)= \mathsf{unsatisfiable}$}{\Return $0$\;} 
				}
			}
			\uIf{$\mathbf{Q}\in L$}{
				\For{$\B_i\psi_1, \dots, \B_i\psi_k\in{\sf sub}(\phi)$ with 
					$\valuation(\B_i\psi_j)=1$ for all $1\leq j \leq k$}{
					\uIf{$\mathsf{Sat}(L,\setsymbols,\bigvee^{k}_{j=1}\neg\psi_j)= \mathsf{unsatisfiable}$}{\Return $0$\;} 
				}
			}
		}
		 		%\item[($\mathbf{D}$)] if $\B_i\psi_1, \dots, \B_i\psi_k, \B_i\chi_1, \dots, \B_i\chi_h\in{\sf sub}(\phi)$,
			%$\valuation(\B_i\psi_j)=1$ for all $1\leq j \leq k$, and
			%$\valuation(\B_i\chi_\ell)=1$ for all $1\leq \ell \leq h$, 
			%		$\valuation(\B_i\psi_j)=\valuation(\B_i\chi_\ell)=1$ for all $1\leq j \leq k$,  $1\leq \ell \leq h$,
			%then $(\bigwedge^{k}_{j=1}\psi_j \land \bigwedge^{h}_{\ell=1}\chi_\ell) \vee \boldsymbol\eta$
			%is satisfied in a $\setsymbols$-consistent $\Lvar^{n}$ model,
			%where
			%\[
			%\boldsymbol\eta =
			%\begin{cases}
			% 	\falseprop, & \text{if $\mathbf{M}\in\Lvar$} \\
			%	\neg(\bigwedge^{k}_{j=1}\psi_j) \land \neg(\bigwedge^{h}_{\ell=1}\chi_\ell), & \text{if $\mathbf{M}\not\in\Lvar$}
			%\end{cases}.
			%\]
			\Return $1$\;
		
		
		\BlankLine
		%	\uIf{$\T$ contains a clash}{\Return $\mathsf{unsatisfiable}$\;} 
		%	\Else{\Return $\mathsf{satisfiable}$\;}
		%\caption{$\LnALC$ tableau algorithm for $\p$}
		\caption{$\mathsf{CheckNTPQ}$}
		%		: Decision procedure %$sat(\p)$
		%		for  the $\LnALCg$  formula satisfiability problem on varying domain neighbourhood models}
	\label{alg:prop}
\end{algorithm}









\begin{algorithm}[t]
	\KwIn{$L$, $\setsymbols$, a $\setsymbols$-consistent valuation $\valuation$, and an $\MLn$ formula $\phi$ built from $\setsymbols$.}
	\KwOut{$\mathsf{1}$, if $\valuation$ satisfies the conditions of Lemma~\ref{lem:proplemmaL}; $0$, otherwise.}
	\BlankLine
%	$r:=  \mathsf{unsatisfiable}$\;
	\uIf{$\mathbf{C} \in \Lvar$}{
$\boldsymbol{\kappa}:= | {\sf sub}({\phi}) |$\;
	}
 \uElse{$\boldsymbol{\kappa}:=1$}
 \BlankLine
	%\For{each $\setsymbols$-consistent valuation $\valuation$ for $\phi$}{

		\For{all $1\leq k,h\leq \boldsymbol{\kappa}$}{

		\For{$\B_i\psi_1, \dots, \B_i\psi_k, \B_i\chi_1, \dots, \B_i\chi_h\in{\sf sub}(\phi)$ with
			$\valuation(\B_i\psi_j)=1$, for all $1\leq j \leq k$, and
			$\valuation(\B_i\chi_\ell)=1$, for all $1\leq \ell \leq h$}{
			\uIf{$\mathbf{M}\in L$}{
				\uIf{$\mathsf{Sat}(L,\setsymbols,(\bigwedge^{k}_{j=1}\psi_j \land \bigwedge^{h}_{\ell=1}\chi_\ell))= \mathsf{unsatisfiable}$}{\Return $0$\;} 
			}
			\uElseIf  {$\mathsf{Sat}(L,\setsymbols,(\bigwedge^{k}_{j=1}\psi_j \land \bigwedge^{h}_{\ell=1}\chi_\ell)\vee (\neg(\bigwedge^{k}_{j=1}\psi_j) \land \neg(\bigwedge^{h}_{\ell=1}\chi_\ell)))= \mathsf{unsatisfiable}$}{\Return $0$\;}
			
		}
	}
	\Return $1$\;
	\BlankLine

	\caption{$\mathsf{CheckD}$}
%		: Decision procedure %$sat(\p)$
%		for  the $\LnALCg$  formula satisfiability problem on varying domain neighbourhood models}
	\label{alg:propD}
\end{algorithm}























\Satfragvardomexp*
\begin{proof}
Soundness and completeness of Algorithm~\ref{alg:propSAT} is given by Lemmas~\ref{lem:propL} and~\ref{lem:proplemmaL}. 
 We argue   that Algorithm~\ref{alg:prop} terminates in exponential time. 
 Since the \ALC satisfiability check is in exponential time, one can compute
 in exponential time (in the size of $\setsymbols$) all valuations $\valuation$ which are $\setsymbols$-consistent. The number of iterations in Line 1 of  Algorithm~\ref{alg:propSAT}
 is bounded by $2^{|\setsymbols|}$. It remains to argue that each iteration takes exponential time. Suppose $\prop{\varphi}$ is the original formula we want to check satisfiability.
 Since each iteration calls   the functions $\mathsf{Check}$, $\mathsf{CheckNTPQ}$, 
 $\mathsf{CheckD}$ and these functions can make recursive calls (to $\mathsf{Sat}$), we need to argue that (1) the number of recursive calls in exponentially bounded and (2) 
 the number of steps inside each function is also exponentially bounded.
 Regarding the latter, we argue that   the number of iterations of the ``for'' loops inside
 $\mathsf{Check}$, $\mathsf{CheckNTPQ}$, and $\mathsf{CheckD}$ is exponentially bounded by the number of subformulas of the formula given as input to each function (and each such formula has size linear in the size of $\prop{\varphi}$). If  the total number of recursive calls is exponentially bounded then Point (2) holds.
 So it remains to argue about Point (1). 
  Consider a computation tree where each node corresponds to a recursive call to  $\mathsf{Sat}$ and the parent relation in the tree is defined by the recursive calls.
 Since each recursive call reduces the number of nested epistemic 
 operators of the original formula $\prop{\varphi}$, any nested sequence  of   recursive calls
 is polynomial in the size of $\prop{\varphi}$. This means that the depth of such tree is polynomial in the size of $\prop{\varphi}$ and, since the number of children of each node is  exponentially bounded (see Point(2)), the total number of nodes of the tree is exponential in the size of $\prop{\varphi}$. We have thus shown that the number of recursive calls in exponentially bounded.
 As satisfiability in $\ALC$ is $\ExpTime$-hard, our upper bound is tight.
\end{proof}