\begin{abstract}

Modal logics are widely used in multi-agent systems to reason about actions, abilities, norms, or epistemic states.
Combined with description logic languages, they
%their combinations with description logics are also
%are also a powerful tool to represent
%%structured
%knowledge over a domain of objects in such modal contexts.
are also a powerful tool to formalise modal aspects of ontology-based reasoning over an object domain.
However, the standard relational semantics for modalities is known to validate principles
%that can be
deemed problematic in agency, deontic, or epistemic applications.
To overcome these difficulties, weaker systems of so-called \emph{non-normal} modal logics, equipped with \emph{neighbourhood semantics} that generalise the relational one, have been investigated both at the propositional and at the description logic level.
We present here
%a systematic study of
a family of
%39
\emph{non-normal modal description logics}, obtained by extending $\ALC$-based languages with non-normal modal operators.
For formulas interpreted on neighbourhood models over varying domains, we provide a modular framework of terminating, correct, and complete tableau-based satisfiability checking algorithms in $\NExpTime$.
For a subset of these systems, we also consider a reduction to satisfiability on constant domain relational models.
Moreover, 
%with respect to neighbourhood models over either constant or varying domains, 
we investigate the satisfiability problem in fragments obtained by disallowing the application of modal operators to description logic concepts,
providing tight $\ExpTime$ complexity results.





%%% ARQNL22
%Non-normal modal logics, interpreted on neighbourhood models which generalise the usual relational semantics, have found application in several areas, such as epistemic, deontic, and coalitional reasoning. We present here preliminary results on reasoning in a family of modal description logics obtained by combining $\ALC$ with non-normal modal operators. First, we provide a framework of terminating, correct, and complete tableau algorithms to check satisfiability of formulas in such logics with the semantics based on varying domains. We then investigate the satisfiability problems in fragments of these languages obtained by restricting the application of modal operators to formulas only, and interpreted on models with constant domains, providing tight complexity results.


%%% DL19
%Non-normal modal logics based on neighbourhood semantics can be used to formalise normative,
%%praxeological,
%epistemic and coalitional reasoning in autonomous and multi-agent systems, since they do not validate
%principles
%%that are
%%known
%%to be problematic
%%to lead to problematic consequences
%known to be problematic
%in
%applications.
%These principles, satisfied by all modal logics interpreted over relational frames, also affect several modal description logics (MDLs)
%used in knowledge representation.
%%In this paper
%We study %satisfiability in
% \emph{non-normal MDLs},
%% \nb{O: modelling is the largest section, removed 'satisfiability in' M: Thanks}
%obtained by extending $\ALC$-based languages with non-normal modal operators.
%These logics increase the expressive power of their propositional counterparts, and allow for complex modelling of obligations, beliefs, abilities and strategies.
%On the computational side, standard reasoning tasks are not more difficult than in basic normal MDLs, with a $\NExpTime$ upper bound for satisfiability
%that can be lowered further
%%\nb{M: mention $\ExpTime$?}
%in fragments with modal operators only over axioms.


\end{abstract}