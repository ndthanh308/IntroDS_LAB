%\documentclass[twocolumn,linenumbers]{aastex631}
%\documentclass[linenumbers]{aastex631}
\documentclass[twocolumn]{aastex631}
%% If you want to create your own macros, you can do so
%% using \newcommand. Your macros should appear before
%% the \begin{document} command.
%%
\usepackage{graphicx} 
\usepackage{threeparttable} 

\newcommand{\vdag}{(v)^\dagger}
\newcommand\aastex{AAS\TeX}
\newcommand\latex{La\TeX}

%\hypersetup{linkcolor=red,citecolor=blue,filecolor=cyan,urlcolor=magenta}

\begin{document}

\title{GRB 221009A: revealing a hidden afterglow during the prompt emission phase with  Fermi-GBM observations}
%\title{Fermi-GBM Discovery of GRB 221009A: Submerged Early Afterglow During the GRB Prompt Phase}


\author[0000-0001-6863-5369]{Hai-Ming Zhang}
\affil{School of Astronomy and Space Science, Nanjing University, Nanjing 210023, China, hmzhang@nju.edu.cn, ryliu@nju.edu.cn, xywang@nju.edu.cn}
\affil{Key laboratory of Modern Astronomy and Astrophysics (Nanjing University), Ministry of Education, Nanjing 210023, China}
\author[0000-0002-6036-985X]{Yi-Yun Huang}
\affil{School of Astronomy and Space Science, Nanjing University, Nanjing 210023, China, hmzhang@nju.edu.cn, ryliu@nju.edu.cn, xywang@nju.edu.cn}
\affil{Key laboratory of Modern Astronomy and Astrophysics (Nanjing University), Ministry of Education, Nanjing 210023, China}
\author[0000-0003-1576-0961]{Ruo-Yu Liu}
\affil{School of Astronomy and Space Science, Nanjing University, Nanjing 210023, China, hmzhang@nju.edu.cn, ryliu@nju.edu.cn, xywang@nju.edu.cn}
\affil{Key laboratory of Modern Astronomy and Astrophysics (Nanjing University), Ministry of Education, Nanjing 210023, China}
\author[0000-0002-5881-335X]{Xiang-Yu Wang}
\affil{School of Astronomy and Space Science, Nanjing University, Nanjing 210023, China, hmzhang@nju.edu.cn, ryliu@nju.edu.cn, xywang@nju.edu.cn}
\affil{Key laboratory of Modern Astronomy and Astrophysics (Nanjing University), Ministry of Education, Nanjing 210023, China}


%\correspondingauthor{Xiang-Yu Wang}
%\email{xywang@nju.edu.cn}

%% If you wish, you may supply running head information, although
%% this information may be modified by the editorial offices.
\shorttitle{GRB 221009A}
\shortauthors{ZHANG et al.}
%%

%% Mark off the abstract in the ``abstract'' environment. 
\begin{abstract}

Recently, LHAASO reported the detection of brightest-of-all-time GRB 221009A, revealing the early onset of a TeV afterglow.
However, there is no evidence of afterglow emission at such early time at other wavelengths. Here we report the discovery of a hidden afterglow component during the prompt emission phase with Fermi Gamma-Ray Burst Monitor (GBM) observations. We analyze the spectral evolution of the  X-ray/$\gamma$-ray emission of GRB 221009A measured by GBM  during the dips of two prompt emission pulses (i.e., intervals $T_{0}+[300-328]\rm~s$ and $T_{0}+[338-378]\rm~s$, where $T_0$ is the GBM trigger time). We find that the spectra at the dips transit from the Band function to a power-law function, indicating a transition from the prompt emission to the afterglow. After $\sim T_{0}+ 660 \rm~s$, the spectrum is well described by a power-law function and the afterglow becomes dominant. Remarkably, the underlying afterglow emission at the dips smoothly connect with the  afterglow after $\sim T_{0}+ 660 \rm~s$. The entire afterglow emission measured by GBM can be fitted by a power-law function   $F\sim t^{-0.95\pm0.05}$, where $t$ is the time since the first main pulse at $T^*=T_0+226~{\rm s}$, consistent with the TeV afterglow decay measured by LHAASO. The start time of this power-law decay  indicates that the afterglow peak  of GRB 221009A should be earlier than $T_{0}+300 \rm ~s$. We also test the possible presence of a jet break in the early afterglow light curve, finding that both the jet break model and single power-law decay model are consistent with the GBM data. The two models can not be distinguished with the GBM data alone because the  inferred jet break time is quite close to the end of GBM observations.

\end{abstract}

\keywords{Gamma-ray bursts (629) --- High energy astrophysics (739))}

\section{Introduction} 
\label{sec:intro}

%The emission of GRBs can be divided into two stages. One is prompt emission, whose light curve is very intense and is believed to be  resulted from internal shocks or other dissipation
%mechanisms that occur at small radii. The other is afterglow, whose light curve shows power-law decays in time  and is believed to originate from external shocks. In  some GRBs, the afterglow is so bright at early time that it
%can contribute detectable flux in the Fermi-GBM band
%pass (Giblin et al. 1999; Connaughton 2002). As seen
%in GRB 190114C, this afterglow flux can also overlap with the highly variable prompt emission.

On 9 October 2022, at 13:16:59.99 universal time (UT) (hereafter $T_0$), the Gamma-ray Burst Monitor (GBM) on the Fermi spacecraft triggered and located the burst GRB 221009A \citep{GBM2022GCN.32636....1V}. It is also detected by other gamma-ray detectors, such as the Fermi Large Area Telescope (Fermi/LAT; \citealt{LAT2022GCN.32658....1P}), Konus-Wind \citep{KW2022GCN.32668....1F,KW2023}, INTEGRAL \citep{INTEGRAL2023arXiv230316943R}, Swift-BAT \citep{swift2022GCN.32688....1K}, GECAM-C \citep{Gecam2023,Yang2023ApJ...947L..11Y} and LHAASO \citep{LHAASO_2023}.
The exceptionally large fluence of this event saturated almost all gamma-ray detectors during the main burst. The event fluence of GRB
221009A was measured to be  $\rm 0.2~erg~cm^{-2}$ in the energy range of 10 to
1000 keV \citep{KW2023,Gecam2023,GBM2023arXiv230314172L}, much higher than any previously detected GRB. 


LHAASO observed  GRB 221009A at the epochs covering both the prompt emission
phase and the early afterglow in the TeV band, revealing the onset of afterglow emission in the TeV band as early as $T_0+230 ~{\rm s}$ and a peak at $\sim T_0+244~{\rm s}$ \citep{LHAASO_2023}. However, there is no evidence of afterglow emission at such early time at other wavelengths. In the optical to X-rays, the observations start too late (thousands of seconds after the GBM trigger), so they missed the early afterglow. In the hard X-ray to Mev $\gamma$-ray energies, as measured by GBM, the prompt emission is so bright that the afterglow emission is likely to be hidden by the prompt emission. The GBM observations show that the last discernible pulse peaking around $T_0+575$ s is followed by a long, smooth decay period \citep{GBM2023arXiv230314172L}. \citet{GBM2023arXiv230314172L} suggest that this smooth decay is consistent with an afterglow origin and  the onset of external shock possibly starts at $\ga T_0+650$ s. This onset time is much later that that of the TeV afterglow measured by LHAASO. 

The apparent discrepancy could be due to that the afterglow onset time in the GBM energy range is much earlier, but it is hidden by the bright prompt emission. Indeed, in GRB 190114C, the afterglow emission is found to overlap with the highly variable prompt emission \citep{Ravasio2019A&A...626A..12R}. 
Motivated by this, we attempt to search for the afterglow emission at the dips of the GBM emission, where the prompt emission is at the lowest level.  We analyze the spectral evolution of the emission of GRB 221009A during  two intervals containing the dips  (i.e., intervals $T_{0}+[300-328] \rm ~s$ and $T_{0}+[348-378] \rm ~s$). We also analyze the spectral evolution of the emission after the last pulse peaking around $T_0+575$ s.  The data analysis and spectral results will be presented in \S 2. We find evidence of the transition from the prompt emission to afterglow at the two dips as well as during the period after the last pulse. Combining the afterglow data at the two dips with the late-time afterglow (after $T_0+660~{\rm s}$), we construct a light curve of the afterglow, which is presented in \S 3. In this section, we also discuss the modeling and implication of this afterglow component.  Finally, we give a summary in \S 4.

\section{Fermi-GBM data analysis} 
\label{sec:analysis}


Fermi-GBM is composed of twelve sodium iodide (NaI) detectors and two bismuth germanate (BGO) detectors \citep{GBM2009ApJ...702..791M}. The NaI detectors are sensitive to photons in the energy range from 8 keV to 900~keV, and the BGO detectors are sensitive in 200~keV-40~MeV. 
GRB 221009A was detected by Fermi-GBM beginning at the trigger time ($T_0$) and lasting until $T_0+1467$~s when it was occulted by the Earth.
In this work, we select two NaI detectors (NaI 4 and NaI
8) and one BGO detector (BGO 1) for preforming the data analysis, due to that they all have a smaller viewing angle, and the two NaI detectors stay within $60\degr$ until $T_0+1467$~s. 
We retrieved data files from the HEASARC online archive\footnote{\url{https://heasarc.gsfc.nasa.gov/FTP/fermi/data/gbm/daily/}} and the corresponding latest updated response matrix files (rsp2) from \citet{GBM2023arXiv230314172L}.
Considering that  the spectra analysis of NaI data show deviations between the model and the data below 20 keV\footnote{In this work, we find that the NaI data also show deviations between the model and the data below 40 keV at $>600~\rm s$.}  \citep{GBM2023arXiv230314172L} and  the presence of the Iodine K-edge at 33.17 keV \citep{2009ExA....24...47B}, we omit the NaI data below 40 keV in our analysis. Following  \citet{GBM2023arXiv230314172L}, we also omit the BGO data below 400 keV in our analysis, due to  that the low energy incident photons that are affected by LAT are not adequately modeled in the detector response.   \citet{10MeV2023arXiv230316223E} reported that a highly significant narrow emission feature  around 10 MeV was found in $T_0+[280-320]$~s, therefore, we ignore the BGO data greater than 8000 keV in order to avoid its possible contamination in our analysis. 

The GBM light curve, as shown in panels (a) of Figure \ref{lcsed1}, \ref{lcsed2} and \ref{lcsed3}, are derived from CSPEC type data\footnote{\url{https://fermi.gsfc.nasa.gov/ssc/data/access/gbm/}} of NaI 4, NaI 8 and BGO 1 in the energy range 40--8,000~keV. 
To model the background, we perform a standard polynomial fitting technique to the light curve over time intervals before and after emission episode. In this work, we select the same background time intervals as that shown in Figure 3 of \citet{GBM2023arXiv230314172L}, during which  a fourth order polynomial fit  matches  the orbital background estimate  the best.

We extract the spectra for this burst with the public Fermi-GBM Data Tools Python software package (GDT; \citep{GDT2022}). 
We exclude the bad time intervals (BTIs, i.e., $T_{0}+[219-277]~\rm s$ \citep{Liu2023ApJ...943L...2L,GBM2023arXiv230314172L} and $T_{0}+[508-514]~\rm s$ \citep{GBM2023arXiv230314172L}).
We start the analysis at $T_{0}+278~{\rm s}$ to avoid the impact of pile-up, {and take 10~s width as a bin until $\rm T_{0}+508~s$.}
To assess the spectral evolution after the BTI  (i.e. after $T_{0}+514~\rm s$), we
extracted a sequence of 50~s or 100~s width  bins until $T_0+1467$~s when it was occulted by the Earth. 


We pay special attention to three time intervals of the GBM observations during which the flux drops to a dip:
I)  $T_{0}+[278-328]~\rm s$,  II) $T_{0}+[328-378]~\rm s$  and III)  $T_0+[514-1467]~\rm s$  (see the left panels of Figures \ref{lcsed1}, \ref{lcsed2} and \ref{lcsed3}, where the count rate light curves around these time intervals are shown.). We divide each interval into several time slices for performing  spectral analysis, denoted with vertical dashed lines in the left panels of Figures \ref{lcsed1}, \ref{lcsed2} and \ref{lcsed3}. Throughout our spectral analyses, we test three spectral models, including a Band function (Band; \citet{band1993ApJ...413..281B}), a power-law (PL), and combinations of them (Band+PL).  The results of the spectral analysis of three selected time intervals are reported in  Table \ref{tab:specfit}. 


To determine the best spectral model for the GBM data of GRB 221009A, we employ Bayesian information criterion (BIC; \citet{BIC1978AnSta...6..461S}) to compare different models. The BIC is defined as $\rm {BIC} = \chi^2 + k {\rm ln} N$, where $\chi^2$ is the PGstat statistic, $k$ and $N$ are the number of free parameters of the model and the number of data points, respectively. 
In interval I ($T_{0}+[278-328]~\rm s$), as shown in Table \ref{tab:specfit}, we find that the spectra are best described by the Band function for slices of A and B { with high $\Delta_{\rm BIC}$ values; compared with PL function or Band+PL model}\footnote{As suggested by \citet{BIC2017JCAP...01..005N}, the strength of the evidence against the model with the higher BIC value can be summarized as follows. (1) If $0<\Delta_{\rm BIC}<2$, there is no evidence against the higher BIC model; (2) if $2<\Delta_{\rm BIC}<6$, positive evidence against the higher BIC model is given; (3) if $6<\Delta_{\rm BIC}<10$, strong evidence against the higher BIC model is given; (4) if $\Delta_{\rm BIC}>10$, a very strong evidence against the higher BIC model is given.}. For  slice C ($T_{0}+[300-308]~\rm s$), we find that the combinations of a Band function and a PL provides the best fit with a difference of BIC between the Band function and Band+PL function being $\Delta_{\rm BIC}=8.69$. Later on (i.e., during slices D and E), the spectra are best fitted by PL. This indicates that as the Band component decreases the PL component emerges. Since the prompt emission is usually characterized by a Band function while the afterglow is characterized by a PL, we may see a transition from the prompt emission to the underlying afterglow during this time interval. 


In interval II ($T_{0}+[328-378]~\rm s$),  we find that that the spectra are best described by the combinations of a Band function and a PL for slices of A and B. For slices C to E, the spectra are best fitted by a PL. This indicates that the prompt emission component drops and the underlying afterglow component emerges at the dip. 

In interval III ($T_0+[514-1467]~\rm s$), we find that  the spectra are best described by the Band function for slices of A and B. For slices C, we find that the combinations of a Band function and a PL provides the best fit. After this,  the spectra in all slices are best fitted by a PL.  Coincidently, the light curve after slice B shows a remarkably smooth decay in time.  Both the spectral and temporal characteristics point to a transition from the prompt emission to the afterglow. The smooth decay after the last pulse has been noticed in the work of \citet{GBM2023arXiv230314172L},  which also attributed the long, smooth decay period after the final pulse to the afterglow emission. The light curve observed by $INTEGRAL/{\rm IBIS}$ in 200--2600~keV also shows that the afterglow emission begins to dominate at $\sim T_0+630~\rm s$ \citep{INTEGRAL2023arXiv230316943R}. 

\section{Afterglow light curve}
\label{sec:lcfitting}

%\subsection{Single power-law model}

We extract the flux of the PL component in the energy range of 50--100 keV and 200--400 keV from the spectral analysis in all slices and plot them in Figure \ref{lc1} (see the black data points). Adopting the reference time of the afterglow light curve at $T^*=T_0+226~{\rm s}$ \citep{LHAASO_2023}, we find that these flux points form a remarkable power-law decay in time (see the left panel of Figure \ref{lc1}). The PL fit ($F\propto t^{\alpha_{pl}}$) results in slopes of $\alpha_{pl}=-0.94\pm0.02$ and $-0.95\pm0.05$ for 50--100 keV and 200-400 keV, respectively. This result strongly supports that the PL components in the spectral analysis originate from the afterglow emission. As the first data point of this PL afterglow is at $T_0+300 $~s, the peak of afterglow emission of GRB 221009A should be before $T_0+300 $ s.

This peak is much earlier than $t_{\rm peak}\ga T_0+650~{\rm s}$ obtained by \citet{GBM2023arXiv230314172L}.
To constrain the peak of the afterglow,  \citet{GBM2023arXiv230314172L} fit the light curve after the last bright pulse using a broken power-law (BPL) afterglow component (i.e., a rising PL component connected with a decay PL component) and two prompt emission pulses described by the model in \citet{Norris2005ApJ...627..324N}. 
%Our spectral analysis indicates that there is no evidence show that $t_{\rm peak}\ga T_0+600~{\rm s}$ in this time interval, explaining the discrepancy. 
Our spectral analysis indicates that there is no rising afterglow phase in this time interval, explaining the discrepancy. 

\subsection{Is there a jet break?}

Early TeV observations of GRB 221009A reveals a light curve steepening at $T_{\rm b}=T^*+ 670^{+230}_{-110}~{\rm s}$, which is interpreted as a jet break \citep{LHAASO_2023}. The combined Insight-HXMT and GECAM-C observations of the hard X-ray emission suggests a break in the light curve between $T^*+650\,{\rm s}$ and $T^*+1100\,{\rm s}$ (most likely at $T^*+\sim950$ s), consistent with the LHAASO result \citep{Gecam2023}. Motivated by this, we study whether a break in the afterglow light curve can be revealed from the GBM data. 

First, we compare a broken power-law model having two free slopes with a single power-law model in fitting the light curve of the afterglow emission in 50--100 keV. The best-fitting result gives a slope of $\alpha_1=-0.93\pm 0.02$ before the break and a slope of $\alpha_2=-1.82^{+0.42}_{-0.31}$ after the break and a break time of $t_b=T^*+1032.41^{+98.55}_{-129.66} ~{\rm s}$. Compared with the single power-law model, the broken power-law model has $\Delta \rm BIC=4.64$ ($\Delta \rm BIC$ is defined as $\rm BIC_2-BIC_1$, where $\rm BIC_1$ and $\rm BIC_2$ are the  $\rm BIC$ values for the power-law model and broken power-law model, respectively), indicating that there is no strong evidence against the broken power-law model. 

%This result also indicating that the broken power-law model can not be excluded.

Then, we consider a  scenario in which the slopes before and after the break are constrained by the jet break physics \citep{Rhoads1999ApJ...525..737R,Sari1999ApJ...519L..17S}. We consider two possible cases: I) one is that the jet break is due to the edge effect and the light curve steepens by a power of $\Delta \alpha =3/4$ (i.e., $\alpha_2=\alpha_1-3/4$) for a constant-density medium; II) the other is that the sideways expansion effect of the jet is significant so that the post-break decay is $t^{-p}$ with $p\simeq 2$ (here we set $\alpha_2=-2$).  In case I, the best-fitting result gives a slope of $\alpha_1=-0.92\pm 0.03$ before the break and  and a break time at $t_b=T^*+1004.75^{+109.29}_{-116.11}~{\rm s}$. Compared with the single power-law model, the broken power-law model has $\rm \Delta BIC=1.63$.
%, indicating that there is no evidence against the broken power-law model.
In case II, the best-fitting result shows that $\alpha_1=-0.93\pm 0.02$, $t_b=T*+1055.15^{+81.80}_{-83.42} {\rm s}$ and $\rm \Delta BIC=1.71$. We also apply the same analysis to the light curves of the afterglow emission in 200--400~keV. The slopes and break times are all consistent with the case for the 50--100~keV emission within the uncertainty. The fitting results are plotted in the right panel of Figure \ref{lc1} and summarized in  Table \ref{tab:LC}.
The $\rm BIC$ values indicate that the jet break model and the single power-law model fit the data almost equally well. This indicates that a jet break can neither be ruled out nor be favored using the GBM data alone. Nevertheless, the slopes and break time in the jet breakh model are consistent with those found in the LHAASO data within the $1\sigma$ uncertainty. 
The fact that the two models can not be distinguished is understandable since the inferred break time is quite close to the last few points of the GBM data.


We can constrain the lower limit of the jet break time through a chi-squared test (following the $\chi^2_{N-n}$ distribution with $N-n=17$ degrees of freedom, where $N$ is the number of data points and $n$ is number of free parameters of the broken power-law model), assuming that the jet break is present. To obtain a useful constraint, we need to fix the slope before the jet break, so we take $\alpha_1=-0.93$ following the above analysis. For the jet break  scenario in which the light curve steepens by a power of $\Delta \alpha =3/4$, we find that the lower limits (at the $95\%$ confidence level) of the jet break time are $671.2~{\rm s}$ for the afterglow in 50--100~keV and $405.7~{\rm s}$ for the afterglow in 200--400~keV, respectively. For the jet break scenario in which the  post-break decay is $t^{-2}$, we find that the lower limits (at the $95\%$ confidence level) of the jet break time are $776.7~{\rm s}$ for the afterglow in 50--100~keV and  $522.2~{\rm s}$ for the afterglow in 200--400~keV, respectively. These lower limits are not in contradiction with the jet break time at TeV energies measured by LHAASO within the uncertainty. 

\subsection{Interpretation of the afterglow light curve}

The hard X-ray afterglow emission measured by GBM should be produced by synchrotron emission of relativistic electrons accelerated in the forward shock. In this scenario, as the cooling frequency likely lies below the observed band, the expected photon index is $dN_{\gamma}/dE_\gamma\propto E_{\gamma}^{-(p+2)/2}$, where $p$ is the PL index of the electron energy distribution (i.e., $dN_e/dE_e\propto E_e^{-p}$). The observed photon index in 40--8,000 keV is about $-2.0$, implying that $p\simeq 2.0$.
In the synchrotron afterglow model, the expected decay slope is $F\propto t^{-(2-3p)/4}\simeq t^{-1.0}$ for a spherical shock \citep{Sari1998ApJ...497L..17S}. Both the measured pre-break slope  in  the  jet break model and the slope in the single PL decay model are  consistent with the expectation (see Table \ref{tab:LC}). 

\section{Summary}

%In some GRBs, the afterglow is so bright that it can lead to detectable flux in the Fermi-GBM band pass \citep{Giblin1999ApJ...524L..47G,Connaughton2002ApJ...567.1028C,Ravasio2019A&A...626A..12R}. 
Through the spectral analysis of the Fermi-GRB  data of GRB 221009A, we find clear evidence of afterglow emission during the dips of the  prompt emission pulses. The spectra show a transition from the Band function to a PL function during these periods. In the last discernible pulse (around $T_{0}+660$~s), we also find that the spectrum transits from the Band function to a PL function. During this period, the emerge of the afterglow emission is well consistent with light curve transition from a variable emission to the smooth decaying emission.
Remarkably, the afterglow flux at different intervals are smoothly connected in a single power-law, as shown in Figure \ref{lc1}. This suggests that the afterglow peak of GRB 221009A should be $t_{\rm peak}\la T_0+300 $~s. 

We also test if there is a jet break in the afterglow emission revealed from the GBM data, motivated by that a jet break is identified in the LHAASO TeV light curve. We find that  the jet break model can not be distinguished from the single power-law model using the  GBM data alone. This indicates that a jet break can neither be ruled out nor be favored using the GBM data alone. This is understandable since the inferred break time is quite close to the last GBM data.  Nevertheless, the slopes and break time in the jet break model are consistent with those found in the LHAASO data.  In future, combining the GBM data with other multi-wavelength data, one should be able to reach a conclusive result, which is, however, beyond the scope of the current work. 


\begin{acknowledgments}
We thank the Stephen Lesage for providing us the latest updated response matrix files and thank Jun Yang for helpful discussion in GBM data analysis.
The work is supported by the  National Key R$\&$D Program of China under grant No. 2022YFF0711404, the NSFC Grants No.12121003, No. 12203022 and No. U2031105, and China Manned Spaced Project (CMS-CSST-2021-B11).
\end{acknowledgments}



%\appendix

%% For this sample we use BibTeX plus aasjournals.bst to generate the
%% the bibliography. The sample631.bib file was populated from ADS. To
%% get the citations to show in the compiled file do the following:
%%
%% pdflatex sample631.tex
%% bibtext sample631
%% pdflatex sample631.tex
%% pdflatex sample631.tex

\bibliography{main}{}
\bibliographystyle{aasjournal}

%% This command is needed to show the entire author+affiliation list when
%% the collaboration and author truncation commands are used.  It has to
%% go at the end of the manuscript.
%\allauthors

%% Include this line if you are using the \added, \replaced, \deleted
%% commands to see a summary list of all changes at the end of the article.
%\listofchanges


% Figure environment removed

% Figure environment removed

% Figure environment removed


% Figure environment removed




\begin{table}[ht!]
\caption{ Spectral fitting results  in  three time intervals.}
\scalebox{0.73}{
\begin{threeparttable} 
%\begin{center}
    \begin{tabular}{lccccccccc}
        \hline\hline
        Time interval & $\alpha$ & $\beta$ &  $E_p$ & Index & $\rm Flux_{50-100}$ & $\rm Flux_{200-400}$  & PGstat/dof. & $\rm \Delta BIC$  \\
        after $T_{0}$ (s) &  & & keV & &$\rm erg~cm^{-2}~s^{-1}$ & $\rm erg~cm^{-2}~s^{-1}$ \\ \hline
&&&&Interval I\\
\hline
A($278-288$)\tnote{$\diamondsuit$} & $-1.37\pm0.01$ & $-2.37\pm0.06$ & $781.31\pm16.77$ & --   & $(7.58\pm0.29)\times10^{-6}$ & $(1.53\pm0.08)\times10^{-5}$ &632.80/267 & 7561.76 \\
B($288-298$)\tnote{$\diamondsuit$}  & $-1.53\pm0.02$ & $-2.37\pm0.13$ & $733.73\pm34.06$ & --  & $(3.61\pm0.20)\times10^{-6}$&$(6.02\pm0.52)\times10^{-6}$ &444.42/267 & 1273.63\\
($298-308$)\tnote{$\diamondsuit$}  &  $-1.77\pm0.21$ & $-2.11\pm0.25$ & $1172.97\pm316.22$ & --  & $(1.79\pm0.14)\times10^{-6}$ & $(2.37\pm0.33)\times10^{-6}$ &296.15/267 & 44.37\\
C($300-308$)\tnote{*}  & $-1.08\pm0.10$ & $-2.36\pm0.30$  & $584.16\pm69.68$ & $-1.93\pm0.02$   & $(1.44\pm0.14)\times10^{-6}$ & $(1.67\pm0.31)\times10^{-6}$ &265.39/265 & 8.69\\
D($308-318$)\tnote{$\dagger$}  & -- & -- & -- & $-1.94\pm0.02$  & $(1.24\pm0.12)\times10^{-6}$ & $(1.35\pm0.25)\times10^{-6}$ &330.24/269 & 0\\
E($318-328$)\tnote{$\dagger$}  & -- & -- & -- & $-2.01\pm0.02$  & $(1.07\pm0.11)\times10^{-6}$& $(1.05\pm0.22)\times10^{-6}$ &305.01/269 & 0\\
\hline
&&&&Interval II\\
\hline
A($328-338$)\tnote{*}  & $-0.91\pm0.03$  & $-4.01\pm0.84$ & $77.05^{+2.19}_{-2.15}$  & $-1.91\pm0.02$   & $(8.75\pm0.10)\times10^{-7}$ & $(9.95\pm2.12)\times10^{-7}$ &280.48/265 &12.25\\
B($338-348$)\tnote{*}  & $-0.89\pm0.03$  & $-4.00\pm0.66$ & $80.31\pm2.60$  & $-2.06^{+0.04}_{-0.02}$   & $(9.54\pm1.04)\times10^{-7}$ & $(9.21\pm2.04)\times10^{-7}$  &358.25/265 & 28.78\\
C($348-358$)\tnote{$\dagger$} & -- & -- & --  & $-2.15\pm0.02$  & $(1.05\pm0.11)\times10^{-6}$ &  $(8.49\pm1.96)\times10^{-7}$  &338.13/269 & 0\\
D($358-368$)\tnote{$\dagger$} & -- & -- & --  & $-2.15\pm0.03$   & $(8.89\pm1.00)\times10^{-7}$ & $(7.27\pm1.81)\times10^{-7}$  &323.54/269 & 0\\
E($368-378$)\tnote{$\dagger$} & -- & --  &-- & $-2.13\pm0.03$  & $(8.04\pm0.96)\times10^{-7}$ & $(6.71\pm1.74)\times10^{-7}$  &354.94/269 & 0\\
\hline
&&&&Interval III\\
\hline
A($514-564$)\tnote{$\diamondsuit$} & $-1.46\pm0.01$  & $-2.49^{+0.06}_{-0.07}$ & $  825.620^{+15.18}_{- 14.92}$  & --  & $(3.81\pm0.09)\times10^{-6}$ & $(6.98\pm0.25)\times10^{-6}$  &1275.51/267 & 7467.51\\
B($564-614$)\tnote{$\diamondsuit$} & $-1.60\pm0.01$  & $-2.18\pm0.03$ & $  187.99^{+3.86}_{- 3.75}$  & --  & $(1.86\pm0.06)\times10^{-6}$ & $(2.09\pm0.14)\times10^{-6}$&1058.44/267 & 1037.85\\
C($614-664$)\tnote{*} & $-1.18\pm0.02$  & $-4.11^{+2.10}_{-0.68}$ & $50.28^{+1.85}_{- 1.84}$  &$-1.97^{+0.03}_{-0.02}$  & $(4.37\pm0.31)\times10^{-7}$ & $(3.75\pm0.58)\times10^{-7}$&987.09/265 & 14.57\\
D($664-714$)\tnote{$\dagger$} & -- & -- &   --  & $-2.05\pm0.03$  & $(3.06\pm0.26)\times10^{-7}$ & $(2.71\pm0.50)\times10^{-7}$ &1061.35/269 & 0\\
E($714-764$)\tnote{$\dagger$} & -- & -- &   --  & $-2.04\pm0.03$  & $(2.59\pm0.24)\times10^{-7}$ & $(2.36\pm0.46)\times10^{-7}$ &1178.75/269 & 0\\
F($764-814$)\tnote{$\dagger$} &  -- & -- &   --  & $-2.01\pm0.03$  & $(2.27\pm0.23)\times10^{-7}$ & $(2.15\pm0.44)\times10^{-7}$ &1032.81/269 & 0\\
G($814-864$)\tnote{$\dagger$} & -- & -- &   --  & $-2.00\pm0.03$  & $(2.04\pm0.22)\times10^{-7}$ & $(1.99\pm0.42)\times10^{-7}$ &1067.46/269 & 0\\
H($864-914$)\tnote{$\dagger$} & -- & -- &   --  & $-1.98\pm0.04$  &$(1.83\pm0.20)\times10^{-7}$  & $(1.87\pm0.41)\times10^{-7}$ &908.40/269 & 0\\
I($914-964$)\tnote{$\dagger$} & -- & -- &   --  & $-1.99\pm0.04$  & $(1.69\pm0.20)\times10^{-7}$ & $(1.77\pm0.40)\times10^{-7}$ &798.30/269 & 0\\
J($964-1014$)\tnote{$\dagger$} & -- & -- &   --  & $-1.99\pm0.04$  & $(1.60\pm0.19)\times10^{-7}$ & $(1.68\pm0.39)\times10^{-7}$ &825.59/269 & 0\\
K($1014-1114$)\tnote{$\dagger$} & -- & -- &   --  & $-1.96\pm0.02$  & $(1.46\pm0.13)\times10^{-7}$  & $(1.54\pm0.26)\times10^{-7}$ &1330.05/269 & 0\\
L($1114-1214$)\tnote{$\dagger$} & -- & -- &   --  & $-1.96\pm0.02$  & $(1.35\pm0.12)\times10^{-7}$ & $(1.43\pm0.25)\times10^{-7}$ &1042.46/269 & 0\\
M($1214-1314$)\tnote{$\dagger$} & -- & -- &   --  & $-2.04\pm0.02$  & $(1.24\pm0.12)\times10^{-7}$& $(1.18\pm0.23)\times10^{-7}$ &979.62/269 & 0\\
N($1314-1467$)\tnote{$\dagger$} & -- & -- &   --  & $-2.09\pm0.02$  & $(1.02\pm0.09)\times10^{-7}$& $(8.99\pm1.63)\times10^{-8}$ & 922.16/269 & 0\\
\hline\hline
\end{tabular}
    %\end{center}    
    
\begin{tablenotes}   
        \footnotesize               
        \item[$\diamondsuit$]  The intervals can be well described by the Band function, the $\rm Flux_{50-100}$ and $\rm Flux_{200-400}$ are obtained from Band function, the $\rm \Delta_{BIC}$ of these intervals are defined as $\rm \Delta BIC=BIC_{PL}-BIC_{Band}$. The high $\rm \Delta BIC$ values are imply that very strong evidence against the PL model. 
        \item[*]  The intervals can be well described by the Band+PL function, the $\rm Flux_{50-100}$ and $\rm Flux_{200-400}$ are obtained from PL function, the $\rm \Delta_{BIC}$ of these intervals are defined as $\rm \Delta BIC=BIC_{Band}-BIC_{Band+PL}$. The high $\rm \Delta BIC$ values are imply that very strong evidence against the Band model. 
        \item[$\dagger$]  The intervals can be well described by the PL function, the $\rm Flux_{50-100}$ and $\rm Flux_{200-400}$ are obtained from PL function, the $\rm \Delta_{BIC}$ of these intervals are defined as $\rm \Delta BIC=BIC_{PL}-BIC_{PL}$. 		  
\end{tablenotes} 
\end{threeparttable} 
}
\label{tab:specfit}
\end{table}




\begin{table}[ht!]
\caption{Fitting results of the afterglow light curve in 50-100 keV and 200-400 keV band with  jet break models.}
\scalebox{0.75}{
\begin{threeparttable} 
%\begin{center}
    \begin{tabular}{lccc|cccccc}
        \hline\hline
      Energy band & \multicolumn{3}{c}{Power-law}  & \multicolumn{5}{c}{Broken Power-law} & \\ \hline
                  & $\alpha_{pl}$ &  $\chi^2/dof.$ &$\rm BIC_{1}$ & $\alpha_{1}$ & $\alpha_{2}$ & $t_{b}$(s) & $\chi^2/dof.$  & $\rm BIC_{2}$  & $\rm \Delta BIC$ \\
      \hline

     &  & & & $-0.93\pm0.02$ &  $-1.82^{+0.42}_{-0.31}$ & $1032.41^{+98.55}_{-129.66}$ & 10.86/16 & 22.84& 4.64  \\
  50-100 keV   & $-0.94\pm0.02$ & 12.21/18 & 18.20 & $-0.92\pm0.03$ & $\alpha_{2}=\alpha_{1}-3/4$ & $1004.75^{+109.29}_{-116.11}$ & 10.84/17 &19.83 & 1.63 \\
      &  & &  & $-0.93\pm0.02$ & $\alpha_{2}=-2.0$ & $1055.15^{+81.80}_{-83.42}$ & 10.92/17 &  19.91 & 1.71  \\
\hline
     &  & & & $-0.93\pm0.05$ &  $-1.95^{+1.00}_{-0.75}$ & $969.06^{+130.26}_{-188.80}$ &  2.18/16 &14.70 & 4.49  \\
  200-400 keV   & $-0.95\pm0.05$ & 4.22/18 &10.21 & $-0.92\pm0.05$ &  $\alpha_{2}=\alpha_{1}-3/4$ & $927.72^{+154.84}_{-149.15}$ & 2.83/17 &11.82 & 1.61  \\
      &  & & & $-0.93\pm0.05$ &  $\alpha_{2}=-2.0$ & $981.49^{+128.98}_{-121.48}$ & 2.73/17 &11.72 & 1.51  \\ 
\hline\hline
\end{tabular}
    %\end{center}    
    
\end{threeparttable} 
}
\label{tab:LC}
\end{table}

\end{document}

% End of file `sample631.tex'.
