% !TEX root = marangoz23humanoids.tex
The goal of our experimental setup is to provide a quantitative comparison of the performance of DawnIK with the state-of-the-art collision aware kinematics solver CollisionIK~\cite{rakita2021collisionik}. The evaluation is carried out with respect to trajectory tracking, collisions and singularities. 

\subsection{Experimental Setup}
\label{subsec:setup}
We carried out our experiments using the ROS-Noetic framework on a computer with core-i7 12700H processor, 32GB RAM and RTX3060 GPU. \textcolor{black}{The DawnIK solver runs at a frequency of 100 Hz for all the scenarios. }
% Figure environment removed

We consider the following four scenarios for the quantitative analysis of our DawnIK solver.
%In the fourth scenario, we do not compare against CollisionIK as it is cannot lite6be used for muti-arm systems. 
\begin{itemize}
	\item \textbf{Scenario 1} \figref{fig:scenario_1}: Single 6 DoF Lite6 arm~\cite{ufactory2023Lite6Collaborative}, controlled by the IK solver and tested for self-collisions.
	\item \textbf{Scenario 2} \figref{fig:scenario_2}: Dual-arm system where the Lite6 is controlled by the IK solver and an Xarm7 arm ~\cite{ufactory2023xArmCollaborative} is controlled by MoveIt
	\item \textbf{Scenario 3} \figref{fig:scenario_3}: HortiBot system, where the Lite6 arm is controlled by the IK solver and the two Xarm7 arms are controlled by MoveIt
	\item \textbf{Scenario 4}: Same dual arm system as in Scenario 2, but both the Lite6 and Xarm7 arms are controlled by two independent DawnIK solvers. 
	We do not compare with CollisionIK for Scenario 4 as it was not possible to adapt it for controlling multiple arms.
\end{itemize}
For the experiments, the controlled arms have to follow different given paths, as defined below, in each scenario.
Furthermore, the external arms execute a given fixed trajectory repeatedly in Scenarios 2 and 3. 
Each experiment is repeated 5 times.
%In this scenario, we test the Lite6 arm for self-collisions and collisions against the external Xarm7 arm. 
%Similarly, in scenario 3, the external arms repeat dual arm motions periodically while the lite 6 arm traces a path, while avoiding the dual arms. 
%In scenarios 2 and 3, we trace six different paths using the controlled arm- squares in X-Y and Y-Z planes, circles in X-Y and Y-Z planes and the figure 8 in X-Y and Y-Z planes. 
%The squares are of width 40cm and the circles are of radius 18 cm. The loops in figure 8 are circles of radius 10cm. 
%
%In scenario 3, we reuse the circle and figure 8 paths. 
%However, we omit the square paths as a major portion of the square paths intersect with the stationary arms. 
%Instead, we consider a stationary pose for the controlled arm, and also trace an object following path. 
%For the external dual arms, we simulate a trajectory where the left grasping arm moves to grasp a fruit while the right cutting arm moves to cut the fruit. The grasping arm then moves to drop the object in the front and then the two arms move to their initial poses. 

We use CollisionIK\cite{rakita2021collisionik} as our baseline, as it is the only other inverse kinematics solver capable of avoiding external collisions. 
% and hence could not be considered for the quantitative experiments. 
%Furthermore, for CollisionIK, we tried two different approaches. 
%Our first approach was to let multi-arm systems be considered as a single system for neural network based collision avoidance. 
%However, this approach was not successful due to the high number of actual collisions between the arms which the neural network could not approximate. 
We added the external arms as external collision objects and update their current poses to enable CollisionIK to consider them as dynamic external collision objects. 

The paths traced by the controlled arm are rigorously checked for collisions by using the fine mesh models of the arms. 
The trajectories traced out by the controlled arm and the external arms are recorded. 
Subsequently, each trajectory point is checked for collision using MoveIt's collision checking feature. 
Thus, we verify the effectiveness of the collision avoidance performed by both solvers.

\subsection{Results}
\begin{table*}[t] \centering
	\renewcommand{\arraystretch}{1.4}
	\begin{tabular}{|c|c|c|c|c|c|c|c|}\hline
		& x (mm)          & y (mm) & z (mm) & roll ($10^{-3}$rad) & pitch ($10^{-3}$rad) & yaw ($10^{-3}$rad)   & collisions\\ \hline
		S1-DawnIK      &\textbf{2.06 $\pm$ 5.60}               &5.93 $\pm$ 20.44  &\textbf{2.95 $\pm$ 9.59}  &0.27 $\pm$ 0.71  &0.11 $\pm$ 0.14  &\textbf{0.36 $\pm$ 1.03}   &0\\ \hline
		S1-CollisionIK &6.26 $\pm$ 9.37   &11.61 $\pm$ 22.88  &5.66 $\pm$ 12.07  &0.64 $\pm$ 1.15  &0.26 $\pm$ 0.46  & 0.92 $\pm$ 1.48  &0\\ \hline \hline
		S2-DawnIK      &20.06 $\pm$ 20.45                &18.01 $\pm$ 19.18  &12.43 $\pm$ 24.42  &0.92 $\pm$ 2.92  &0.58 $\pm$ 2.35  &\textbf{1.32 $\pm$ 1.78}   &0\\ \hline
		S2-CollisionIK &32.84 $\pm$ 39.50                &37.08 $\pm$ 48.40  &34.81 $\pm$ 55.50  &1.20 $\pm$ 1.73  &0.36 $\pm$ 0.68  &1.46 $\pm$ 1.78   &0\\ \hline \hline
		S3-DawnIK      &99.4 $\pm$ 134.65                &45.59 $\pm$ 57.34  &68.43 $\pm$ 79.22  &30.52 $\pm$ 55.43  &8.28 $\pm$ 17.76  &29.2 $\pm$ 58.01   &\textbf{0}\\ \hline
		S3-CollisionIK &27.48 $\pm$ 26.45                &39.46 $\pm$ 40.91  &31.10 $\pm$ 47.94  &6.208 $\pm$ 7.53  &0.981 $\pm$ 1.51  &4.98 $\pm$ 5.4   &38.13 \\ \hline
	\end{tabular}
	\captionsetup{width=0.97\linewidth, justification=justified}
	\caption{DawnIK versus CollisionIK mean end-effector errors with standard error, and mean number of collisions for the three different scenarios. S1, S2, and S3 indicate Scenarios 1, 2 and 3 respectively. The results demonstrate the ability of DawnIK to avoid collisions in all the three scenarios while being superior at trajectory tracking in Scenarios 1 and 2. Figures shown in bold indicate statistically significantly lower errors for p $<$ 0.05 using the Mann-Whitney U test. Note that in S3, no collisions occur using DawnIK solver whereas CollisionIK solutions lead to multiple collisions.}
	\label{tab:results}
\end{table*}
\subsubsection{Scenario 1}
% Figure environment removed
In the first scenario, we compare the performance of DawnIK with CollisionIK for path tracing, where self-collisions can occur, using the Lite6 arm. 
For this and the other experimental scenarios, we generated three different path shapes (squares, circles, and the figure eight) in the X-Y
and Y-Z planes to trace them with the controlled arm.
The squares are of width 40 cm and the circles are of radius 18cm. The loops for the eight are circles of radius 10cm. 
\textcolor{black}{As both DawnIK and CollisionIK are inverse kinematic solvers and not trajectory planners, the waypoints for these trajectories are precomputed at a sampling rate of 30 ms and fed to both the solvers.}

As can be seen from \tabref{tab:results}, both DawnIK and CollisionIK avoid collisions for all the experiments in Scenario 1.
However, as can be seen visually from \figref{fig:ee_error} and the values for the end-effector position and orientation errors in \tabref{tab:results}, DawnIK outperforms CollisionIK in terms of trajectory tracking. 
We calculated the mean trajectories over 5 iterations for the task of drawing the figure in X-Y and Y-Z plane and plot the curves in \figref{fig:eight_xy} and \figref{fig:eight_yz} respectively. 
When there is a probability for self-collision, CollisionIK deviates from the path noticeably whereas DawnIK deviates only to a minimum extent. 
This is due to the fact that DawnIK uses actual distances to compute the collision avoidance constraint, and CollisionIK approximates it using a neural network.

\subsubsection{Scenario 2}
% Figure environment removed
In the second scenario, the external Xarm7 arm is repeatedly performing a motion along the X axis, whereas the controlled arm traces a path.
We reuse the paths from Scenario 1. 
In Scenario 2, the external arm repeatedly interferes with the path of the controlled arm, thus causing both the IK solvers to deviate from the desired path.

Both the solvers are successful in avoiding collisions with themselves and with the external arm. 
However, as can been seen from \tabref{tab:results}, the error in end-effector position tracking is higher for CollisionIK as compared to that for DawnIK. 
\figref{fig:s2_eight_xy} also demonstrates that CollisionIK has the tendency to wander further away from the desired trajectory than to avoid the external collisions by moving within a local region. 
\subsubsection{Scenario 3}
% Figure environment removed
In the third scenario with the HortiBot system, we omit the square paths as a major portion of the square paths intersect with the stationary arms. 
Instead, we consider a stationary pose for the controlled arm, and also trace an object following path. 
For the external dual arms, we simulate a trajectory where the left arm moves to grasp a fruit while the right arm moves to cut the fruit. 
The grasping arm then moves to drop the object in the front and then the two arms move to their initial poses. 

The configuration of the three arm system and the trajectory of the external dual arms make it extremely challenging for the controlled arm to follow its path. 
With the conflicting constraints of trajectory following and collision avoidance, DawnIK successfully avoids collisions in all the trials. However, this comes at the cost of increased trajectory tracking error. CollisionIK, on the other hand, shows trajectory tracking error similar to that in Scenario 2. 
This leads to multiple states where collisions occur between the controlled arm and the dual arms. 
The offline collision checking with actual mesh models detected a large number of such collisions. 
This shows, that DawnIK is better suited to avoiding collisions with multiple arms. We also validated our method for scenario 3 in the real world using the HortiBot system as can be seen in the accompanying video\footnote{https://youtu.be/-k7XJkbAB6A}.  
\subsubsection{Scenario 4}
\begin{table*}[t] \centering
	\renewcommand{\arraystretch}{1.4}
	\begin{tabular}{|c|c|c|c|c|c|c|c|c|}\hline
		& x (mm)          & y (mm) & z (mm) & roll ($10^{-3}$rad) & pitch ($10^{-3}$rad) & yaw ($10^{-3}$rad)  & collisions\\ \hline
		S4-Xarm7      &17.15 $\pm$ 18.99               &21.21 $\pm$ 27.34  &24.67 $\pm$ 38.34  &6.98 $\pm$ 53.41  &0.99 $\pm$ 1.78  &6.64 $\pm$ 52.91  &0\\ \hline
		S4-Lite6 &23.08 $\pm$ 29.23   &29.30 $\pm$ 26.71  &28.56 $\pm$ 39.57  &1.27 $\pm$ 2.13  &0.44 $\pm$ 2.61  & 1.59 $\pm$ 1.65 &0\\ \hline 
	\end{tabular}
	\captionsetup{width=0.97\linewidth, justification=justified}
	\caption{Mean end-effector position and orientation errors, and mean number of collisions for Lite6 and Xarm7 for Scenario 4. As can been seen, even with the paths intersecting each other, no collisions occur for both Lite6 and Xarm7 arms. While in Scenario 2, the external Xarm7 was performing a trajectory beside the Lite6 arm, in this scenario, they are both occupying the same workspace. Inspite of that, the trajectory tracking errors are comparable showing the benefit of multi-arm collision-aware approach.}
	\label{tab:s4_results}
\end{table*}
% Figure environment removed

In Scenario 4, both the Lite 6 and the Xarm7 arms are independently controlled by DawnIK. Unlike Scenario 2, where the Xarm7 was performing a point-to-point motion beside the Lite 6 arm, in Scenario 4, it is always tracing the square path in the Y-Z plane. 
Additionally, the start positions are also quite close to each other.
The Lite 6 arm traces the same paths as in Scenario 2, except for the circle in the X-Y plane as it is infeasible due to the Xarm7's path. 

As can been seen in \tabref{tab:s4_results}, no collisions are reported even though the trajectories of both arms intersect multiple times. Both the arms are also able to track the end-effector positions with errors similar to Scenario 2. The 7 DoF Xarm7 has a better trajectory tracking performance as DawnIK implicitly makes use of the redundancy to find feasible solutions. 
\textcolor{black}{The trajectory tracking performance of xArm7 using DawnIK shows that while we do not explicitly exploit redundancy using recursive Jacobian implementations \cite{slotine1991general,zhao2021solving}, our method is still able to use the redundancy to avoid collisions while tracking the trajectory as close as possible.}
\figref{fig:s4_circle_yz} and \figref{fig:s4_eight_yz} show that with the decentralized approach, the two arms are able to track trajectories even in the same plane without any collisions. 
In \figref{fig:s4_square_yz}, the two arms have an extremely challenging task of following the same path while avoiding collisions with each other. However, no collision was detected and both arms are able to follow the square path where feasible. In all the four scenarios, the controlled arm does not get stuck in any singularity for both the solvers. 