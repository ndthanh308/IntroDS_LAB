% !TEX root = marangoz23humanoids.tex
With robots becoming ubiquitous, different arm based systems have to work in close proximity on the same task or related tasks concurrently. \textcolor{black}{These different systems may be controlled by software of different vendors and only their current joint positions and robot models are known, while no information regarding their task or their relative priorities is available.} 
Traditional approaches use workspace partitioning with virtual walls to avoid collisions between different arms or with humans \cite{takubo2002control}. 
However, this restricts the workspace and reduces the efficiency of the manipulation systems. 
At the other end of the spectrum, whole-body control approaches are used for humanoid manipulation planning~\cite{garcia2019integration, ferrari2017humanoid, sentis2006whole}.
These approaches are well suited for humanoids to maintain stability while simultaneously performing tasks
such as grasping, walking, and gaze stabilization
They are, however, sub-optimal for multi-arm systems that do not need to maintain coordination at all times. 
An example of this is the three-arm system HortiBot shown in \figref{fig:cover_fig}, which is added to the PATHoBot \cite{mccool21icra} for robotic harvesting of horticulture plants. 
The left and right arms are used for dual-arm manipulation for grasping and cutting respectively, whereas the observer or head arm, equipped with cameras is used for fruit mapping and tracking. 
Thus, the head arm needs to work independently during active fruit mapping, whereas during fruit harvesting it needs to provide the perception needed by the manipulation arms. 
In such scenarios, a collision-aware kinematics solver that can avoid collisions with the neighboring arms, static or moving, while following a specified end-effector path, is needed. 

% Figure environment removed 

In this paper, we present a novel non-linear optimization based approach called DawnIK to find an inverse kinematics solution for different Cartesian end-effector goals while avoiding self-collision and collision with other arms in the vicinity.
Our approach formulates different constraints including end-effector pose reaching as weighted cost functions with collision avoidance being the highest priority. 
The main contributions of our approach are as follows:
\begin{itemize}
	\item A decentralized approach to find collision-aware inverse kinematics where in addition to the controlled arm, externally controlled arms work in close proximity. 
	\item A method for fast and efficient collision distance calculation that scales well for multi-arm systems without using expensive perception algorithms.
	\item The capability to specify different end-effector constraints along with other constraints for singularity escape and kinodynamic limits.
\end{itemize}

Furthermore, we demonstrate through extensive experiments that our decentralized inverse kinematics solver can be applied to different multi-arm systems and outperforms an existing state-of-the-art inverse kinematic solver in terms of collision avoidance and trajectory tracking.
