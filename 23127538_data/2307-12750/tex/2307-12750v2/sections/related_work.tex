Robot arms are used to control the tool or sensor attached to the end-effector i.e. the last link. 
As the end-effector needs six degrees of freedom to reach a desired pose in the SE(3) space,  most robot arms have six joints, with some having an extra seventh joint to provide kinematic redundancy. 

For effective use of robot arms, fast and reliable computation of inverse kinematics, i.e. calculating the set of joint values  that can place the end-effector at an arbitrary position and/or orientation in the arm's workspace, is critical.
However, this is still a challenging task and depends on the robot arm design\cite{paul1979kinematic, pieper1969kinematics}. 
Analytical, numerical, learning\cite{malik2022deep, guo2019reinforcement}, and hybrid methods incorporating the former three, are the different categories of existing approaches\cite{aristidou2018inverse}. 
Although modern robot arms are designed such that closed form analytical solutions exist \cite{hawkins2013analytic, liu2017analytical} or can be constructed using code generation tools \cite{diankov2010automated}, they do not provide the flexibility for prioritization or relaxation of different constraints based on the task at hand. 

Typical numerical approaches use Jacobian methods \cite{buss2004introduction, buss2005selectively} to iteratively find the inverse kinematics solution for a desired pose around the current robot configuration. 
Inverse or pseudo-inverse Jacobian, or singular value decomposition (SVD) \cite{cao2023numerically} computations are expensive and especially prone to getting stuck in local minima. 
Recent approaches like TracIK\cite{beeson2015trac} and BioIK\cite{starke2020bio} have mitigated these problems to an extent using  random restarts with nonlinear optimization, and memetic algorithms that combine gradient-based optimization with genetic and particle swarm optimization, respectively. 
We tried to adapt BioIK\cite{starke2020bio} to include collision avoidance as a constraint.
However, with the constraint added, it suffered from a high failure rate in finding solutions.

As the above methods do not compute collision aware trajectories, recent research has focused on collision aware solvers.
Rakita\etal proposed RelaxedIK, a weighted-sum optimization method, that uses a neural network to approximate distances from collision states to find solutions that avoid self-collisions \cite{rakita2018relaxedik}. 
They also further extended it as CollisionIK to avoid collisions with environment obstacles using multi-objective constraint based optimization \cite{rakita2021collisionik}. 
Both RelaxedIK \cite{rakita2018relaxedik} and CollisionIK \cite{rakita2021collisionik} suffer from self-collisions due to the use of neural network approximation rather than actual distance computation. 
Also, in CollisionIK, only a maximum of three environment obstacles are considered and hence it is not suitable for avoiding another arm or arms moving in the vicinity. 

Approaches which exploit the null space of redundant manipulators for sequential solution of constraints are also deployed in different systems. 
Zhao\etal\cite{zhao2021solving} presented an approach for inverse kinematics of multiple redundant manipulators for collision avoidance using a combination of the inverse Jacobian and velocity obstacles. 
However their work is specific to seven DoF redundant manipulators.  

As our approach does not make any explicit use of redundancy, it can be applied in general to all manipulators and allows for the weighting of different goals to adapt to different tasks. 
More importantly, it provides a decentralized approach to collision-aware inverse kinematics computation for multiple arms for the first time.  
Additionally, our approach also does not depend on the expensive computation of inverse Jacobians for task priority resolution. 