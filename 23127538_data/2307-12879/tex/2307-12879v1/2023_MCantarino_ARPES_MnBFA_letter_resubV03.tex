%% LyX 2.3.6.1 created this file.  For more info, see http://www.lyx.org/.
%% Do not edit unless you really know what you are doing.
\documentclass[twocolumn,english]{revtex4-2}
\usepackage[T1]{fontenc}
\usepackage[utf8]{inputenc}
\setcounter{secnumdepth}{3}
\usepackage{color}
\usepackage{float}
\usepackage{bm}
\usepackage{amsmath}
\usepackage{physics}
\usepackage{verbatim}
\usepackage{babel}
\usepackage{amsbsy}
\usepackage{amstext}
\usepackage{amssymb}
\usepackage[pdftex]{graphicx}
\usepackage{pdfpages}
\usepackage[unicode=true,pdfusetitle,
 bookmarks=true,bookmarksnumbered=false,bookmarksopen=false,
 breaklinks=false,pdfborder={0 0 0},pdfborderstyle={},backref=false,colorlinks=true]
 {hyperref}
\hypersetup{
 urlcolor=blue, citecolor=blue}

\makeatletter
%%%%%%%%%%%%%%%%%%%%%%%%%%%%%% User specified LaTeX commands.
\usepackage{braket}

\AtBeginDocument{\let\LS@rot\@undefined}
\makeatother

\begin{document}
\title{Incoherent electronic band states in Mn substituted BaFe$_{2}$As$_{2}$}
\author{Marli R. Cantarino$^{1,2}$, Kevin R. Pakuszewski$^{3}$, Björn Salzmann$^{4}$, Pedro H. A. Moya$^{1}$, Wagner R. da Silva Neto$^{1}$, Gabriel S. Freitas$^{3}$, Pascoal G. Pagliuso$^{3,5}$, Claude Monney$^{4}$, Cris Adriano$^{3}$, and Fernando A. Garcia$^{1}$}
\affiliation{$^{1}$Instituto de Física, Universidade de São Paulo, 05508-090 São Paulo, SP, Brazil}
\affiliation{$^{2}$Brazilian Synchrotron Light Laboratory (LNLS), Brazilian Center for Research in Energy and Materials (CNPEM), Campinas-SP, 13083-970, Brazil}
\affiliation{$^{3}$Instituto de Física “Gleb Wataghin”, UNICAMP, 13083-859, Campinas-SP, Brazil}
\affiliation{$^{4}$Département de Physique, Université de Fribourg, CH-1700 Fribourg, Switzerland}
\affiliation{$^{5}$Los Alamos National Laboratory, Los Alamos, New Mexico 87545, USA}
\begin{abstract}
Chemical substitution is commonly used to explore new ground states in materials, yet the role of disorder is often overlooked. In Mn-substituted BaFe$_{2}$As$_{2}$ (MnBFA), superconductivity (SC) is absent, despite being observed for nominal hole-doped phases. Instead, a glassy magnetic phase emerges, associated with the $S=5/2$ Mn local spins. In this work, we present a comprehensive investigation of the electronic structure of MnBFA using angle-resolved photoemission spectroscopy (ARPES). We find that Mn causes electron pockets to shrink, disrupting the nesting condition in MnBFA. Notably, we propose that electronic disorder, along with magnetic scattering, primarily contributes to suppressing the itinerant magnetic order in MnBFA. This finding connects the MnBFA electronic band structure properties to the glassy magnetic behavior observed in these materials and suggests that SC is absent because of the collective magnetic impurity behavior that scatters the Fe-derived excitations. Moreover, we suggest that Mn tunes MnBFA to a phase in between the correlated metal in BaFe$_{2}$As$_{2}$ and the Hund insulator phase in BaMn$_{2}$As$_{2}$.
\end{abstract}
\maketitle
\emph{Introduction:} electronic correlated materials exhibit a rich phase diagram when subjected to partial chemical substitution of one of their constituting elements. Whereas a great deal of attention is devoted to superconducting (SC) phases driven by this strategy, non-SC phases also spark heated debate. One such example are the Ba(Fe$_{1-x}$Mn$_{x}$)$_{2}$As$_{2}$ materials which derive from the parent compound BaFe$_{2}$As$_{2}$ (BFA).

BFA is an Iron-based superconductor (IBS) material \citep{hosono_iron-based_2015}. This system undergoes nearly simultaneous phase transitions from a tetragonal to an orthorhombic phase and from a paramagnetic (PM) to a spin density wave (SDW) phase with a critical temperature ($T_{\text{SDW}}$) of about $140$ K \citep{rotter_spin-density-wave_2008}. A high-temperature SC phase can be driven in BFA by multiple partial chemical substitution strategies \citep{canfield_feas-based_2010,jiang_superconductivity_2009,sefat_superconductivity_2008,2009_Ni_Rh-Pd-BFA,Saha_2010_Pt-BFA,Li_2009_Ni-BFA,hosono_iron-based_2015}, including nominal hole doping \citep{rotter_superconductivity_2008,2009_Bukowski_Rb-BFA}, which invites an explanation for the absence of SC in Mn substituted BFA (MnBFA) \citep{kim_antiferromagnetic_2010,thaler_physical_2011,pandey_large_2011}.

Whereas it was proposed that SC is absent in MnBFA because the Mn-derived states remain localized and therefore charge doping is not caused by Mn substitution \citep{texier_mn_2012,suzuki_absence_2013}, the scattering of the Fe-derived SDW fluctuations by the Mn-derived Néel fluctuations is also believed to play the key role \citep{tucker_competition_2012,garcia_anisotropic_2019} in this phenomenology. The latter topic further invites an investigation into the relative relevance attributed to disorder or magnetic and impurity scattering \citep{fernandes_suppression_2013,gastiasoro_enhancement_2014,gastiasoro_unconventional_2016} caused by Mn.

Despite the absence of charge doping, changing electronic bands in MnBFA cannot be discarded since the hybridization between Fe and As states depends on Mn content \citep{de_figueiredo_orbital_2022}. Indeed, the electronic structure cannot be totally independent of the Mn content, since MnBFA is tuned to a Hund insulating state in BaMn$_{2}$As$_{2}$ \citep{yao_comparative_2011,antal_optical_2012,mcnally_hunds_2015}. Our motivation is thus to fill an important gap in this discussion: the detailed characterization of the electronic band structure of MnBFA samples, which is so far lacking despite previous experimental efforts \textcolor{blue}{\citep{suzuki_absence_2013}}. Employing an alternative In-flux method \citep{garitezi_synthesis_2013}, we grew high-quality MnBFA single crystals and performed angle-resolved photoemission spectroscopy (ARPES) experiments of Ba(Fe$_{1-x}$Mn$_{x}$)$_{2}$As$_{2}$ ($x=0.0$, $0.035$ and $0.085$, hereafter called BFA, Mn$3.5\%$ and Mn$8.5\%$ samples, respectively).

We find that Mn causes a sizable decrease of the electron pockets, contributing to the suppression of $T_{\text{SDW}}$. Our results, however, suggest that the suppression of $T_{\text{SDW}}$ is mainly an effect of disorder and magnetic scattering, both of which combine to preclude the formation of a SC ground state \citep{fernandes_suppression_2013,gastiasoro_enhancement_2014,gastiasoro_unconventional_2016}. Moreover, our findings support that the indirect exchange interaction between the Mn local moments is mediated by incoherent electronic states, explaining the glassy behavior of the Mn local moments \citep{inosov_possible_2013}. 

% Figure environment removed

\emph{Materials and methods:} Ba(Fe$_{1-x}$Mn$_{x}$)$_{2}$As$_{2}$ single crystals were grown using the In-flux method \citep{garitezi_synthesis_2013}. The final Mn content ($x$) was characterized by energy-dispersive x-ray spectroscopy (EDS) and by comparing the sample's $T_{\text{SDW}}$ to other $x$ vs. $T$ phase diagrams in literature \citep{thaler_physical_2011,texier_mn_2012} to benchmark the EDS determined values of $x.$ The ARPES experiments were performed at the Bloch beamline of the Max IV synchrotron in Lund, Sweden. The total energy resolution was set at about $8$ to $10$ meV for incident photon energies between $60$ and $81$ eV, and angular resolution of $0.1^{\circ}$. The samples were cleaved using Al posts inside the main preparation chamber (vacuum of $3\times10^{-10}$ mbar) and then transferred to the analyzer chamber (vacuum of $2\times10^{-11}$ mbar) for the experiments. More details in Supplemental material.

\emph{Results and discussion:} in Figs. \ref{fig:overview}$(a)$-$(c)$, we present a survey of the electronic band structures, as a function of Mn content, in the tetragonal PM state ($T=150$ K) of our samples. Measurements were taken along the high-symmetry directions and adopting linear beam polarizations, either linear horizontal (LH) or linear vertical (LV) as indicated in each panel. The crystal body-centered tetragonal geometry was adopted to label the Brillouin zone (BZ) high-symmetry points.

Band features are distinguishable for all samples, allowing a comprehensive characterization of the MnBFA electronic bands. In the simplest model, the IBS electronic bands derive from Fe 3$d$-states that are subjected to the effects of the As ligands, which break the Fe $3d$-states degeneracy and instill a strong orbital character to the electronic bands \citep{hosono_iron-based_2015,fernandes_iron_2022}. Based upon the selection rules for the ARPES intensity polarization dependence and guided by previous works \citep{fuglsang_jensen_angle-resolved_2011,brouet_impact_2012,yi_role_2017,pfau_detailed_2019,zhang_orbital_2011,thirupathaiah_orbital_2010}, the orbital character of the electronic bands were labeled.

The effects of Mn substitution on the electronic bands are examined in Figs. \ref{fig:overview}$(f)$-$(j)$. We focus on states close to $E_{F}$. To characterize how the hole pockets change as a function of Mn content, we track the hole bands with main $d_{xz/yz}$ and $d_{xy}$ orbital character close to $\Gamma$ and measured along the high-symmetry directions (Figs. \ref{fig:overview}$(a)$-$(c)$). Results are presented in Figs. \ref{fig:overview}$(f)$-$(h)$. Electronic states at the electron pockets around the X/Y points, however, have $C_{2v}$ point symmetry which is reflected in the idealized elliptical shape of the pockets. Therefore, we must also look at the bands in a direction perpendicular to the $\Gamma X$ direction. We consider a cut in our Fermi maps along the green dashed line shown in Fig. \ref{fig:overview}$(d)$, for the BFA case, representing the $YZ$ direction. The associated electron-like band is shown in Fig. \ref{fig:overview}$(e)$ and is called the ``shallow'' electron-like band as opposed to the ``deep'' electron-like band observed directly in Figs. \ref{fig:overview}$(a)$-$(c)$, as the blue points for $\Gamma X$ and LV polarization. The shallow (deep) electron-like band determines the minor (major) semi-axis of the electron pocket around $X/Y$ and has $d_{yz}$ ($d_{xy}$) main orbital character. In Figs. \ref{fig:overview}$(i)$ and $(j)$ we compile the deep and shallow electron-like bands as a function of Mn content.

Increasing hole pockets and shrinking electron pockets are the putative effects of the nominal hole doping caused by Mn. By inspection of Figs. \ref{fig:overview}$(f)$-$(j)$, the experimentally determined scenario is more involved. Bands forming the hole pockets and the deep electron-like band are barely affected (Figs \ref{fig:overview}$(f)$-$(h)$) whereas the intersection of the shallow electron-like band ($d_{yz}$ orbital character) with $E_{\text{F}}$ (which determines $\boldsymbol{k}_{\text{F}}$) is systematically decreasing. The lack of change in the hole pockets can be ascribed to the lack of effective charge doping \citep{texier_mn_2012,suzuki_absence_2013} but the distinct behavior of shallow and deep electron-like bands is completely unanticipated. We suggest that it is the effect of the changing Fe$3d_{yz/xz}$ and As$4p_{z}$ hybridization as a function of Mn content \citep{yao_comparative_2011,de_figueiredo_orbital_2022}.

% Figure environment removed

We thus resort to a quantitative analysis of the ARPES spectral function $A(\boldsymbol{k},E)$ to probe for other effects of Mn substitution. We fit momentum distribution curves (MDCs) to the expression for the one-particle $A(\boldsymbol{k},E)$ for a system of weakly correlated electrons \citep{sobota_angle-resolved_2021}. Our objective is to extract the electronic scattering rate $\Gamma(E)$ and the imaginary part of $\Sigma(E)$, both as a function of the binding energy $E$. We concentrate on extracting $\Gamma(E)$ and $\text{Im}\Sigma(E)$ from the MDCs analysis for the bands with $d_{yz}$ and $d_{xy}$ main orbital character in the measurements in direction $\Gamma X$ with LV polarization and $\Gamma M$ with LH polarization, respectively (Fig. \ref{fig:overview}$(a-c)$). The fitting results are presented, respectively, in Figs. \ref{fig:selfE}$(a)$ and $(b)$ (representative BFA data, see Supplemental material for the Mn$3.5\%$ and Mn$8.5\%$ related data). The fittings in Fig. \ref{fig:selfE}$(a)$ were obtained as in Refs. \citep{kurleto_about_2021,fink_linkage_2021}, whereas the fittings in Fig. \ref{fig:selfE}$(b)$ were obtained as explained in the Supplemental Material. The extracted values of $\Gamma(E)$ and the calculated $\text{Im}\Sigma(E)$ are shown in Figs. \ref{fig:selfE}$(c)$ and $(d)$ for all samples.

The broadening of the spectroscopic features, here measured by $\Gamma(E)$, contains intrinsic and extrinsic effects which also affect the determination of $\text{\text{Im}\ensuremath{\Sigma}(E)}$. A way around this problem is to focus on the rate of change of the quantities, which is less affected by the homogeneous broadening introduced by extrinsic effects. This is qualitatively captured by the lines drawn in Figs. \ref{fig:selfE}$(c)$ and $(d)$, which serve as guides to eyes and suggest a linear behavior close to $E_{F}$ for the probed quantities. Indeed, $\Gamma(E)$ and $\text{Im}\Sigma(E)$ do not follow the quadratic behavior expected for a Fermi liquid indicating the correlated nature of the metallic state in BFA and MnBFA, which is also observed for other substitutions \citep{kurleto_about_2021,fink_linkage_2021,fink_experimental_2017}. A linear increase is reminiscent of a Marginal Fermi Liquid \citep{sobota_angle-resolved_2021,varma_MFL_2020}, and an analysis based upon this picture is carried out in Supplemental material. Here we focus on model-independent conclusions.

Physically, being $\Gamma(E)$ proportional to the inverse of the quasiparticle lifetime, it is a measurement of the electronic states' coherence. If $\Gamma(E)$ increases at a faster pace as a function of $E$, one can conclude that electronic states' coherence is decreasing. This is precisely the observed effect caused by Mn substitution in all bands, as shown in Fig. \ref{fig:selfE}$(c)$. Similarly, $\text{Im}\Sigma(E)$ is a measurement of electronic correlations. The effects caused by Mn (\ref{fig:selfE}$(d)$) are orbital specific: $i)$ correlations increase monotonically for the bands with main $d_{yz}$ orbital character and $ii)$ correlations are not much affected in the case of bands with main $d_{xy}$ orbital character. Interestingly, the bands derived from the $d_{yz}$ orbitals are precisely the bands that form the shallow electron-pocket bands that are affected by Mn substitution. 

Thus, the effects caused by Mn substitution amount to: $i)$ shrinking electron pockets, $ii)$ an orbital-specific increase in electronic correlations, and $iii)$ the increase of the electronic bands' incoherence. To understand how relevant the electronic tuning caused by Mn is, we assess its effect on $T_{\text{SDW}}$ by comparing our results to those for Co substitutions (CoBFA). We assume the scenario wherein the SDW phase is stabilized by the nesting between hole and electron states \citep{fernandes_low-energy_2016,fernandes_iron_2022}. The Mn$8.5\%$ effect on the shallow electron-like band is comparable (albeit in the opposite direction) to that caused on the electron pockets by nearly the same amount of Co ($x_{\text{Co}}=0.08$) \citep{brouet_nesting_2009}. For $x_{\text{Co}}=0.08$, however, the SDW is already absent, because Co also causes the hole pockets to shrink, tuning CoBFA off the nesting condition. If one chooses a comparison between systems with the same $T_{\text{SDW}}$, our Mn$8.5\%$ sample ($T_{\text{SDW}}=66$ K) is closer to $x_{\text{Co}}=0.045$ ($T_{\text{SDW}}=65$ K). Again, one finds \citep{brouet_nesting_2009} that the main source of partial detuning off the nesting condition is the simultaneous change in electron and hole pockets. Therefore, whereas the electronic tuning caused by Mn may contribute to the suppression of the SDW phase, it cannot be the dominant effect. 

We suggest that this is the role of electronic disorder \citep{wadati_where_2010,levy_probing_2012,ideta_dependence_2013} in combination with the magnetic scattering between the SDW-type and Néel-type excitations \citep{tucker_competition_2012,fernandes_suppression_2013,wang_impact_2014,gastiasoro_enhancement_2014,garcia_anisotropic_2019}. The glassy magnetic behavior attributed to Mn spins \citep{inosov_possible_2013}, however, is likely only due to the band incoherence, since the interaction between the Mn local moments is mediated by incoherent states. A similar interplay between the itinerant magnetism and the local moments is also proposed by optical spectroscopy measurements of MnBFA \citep{kobayashi_carrier_2016}. It is intriguing that Mn, being close to Fe, behaves similarly to Zn which, in principle, is a much stronger impurity scatter, and not similar to Co \citep{levy_probing_2012,ideta_dependence_2013}. 

The increase in electronic correlations observed only for bands with $d_{yz}$ character, illustrates the orbital specificity of the chemical substitution effects in IBS materials \citep{de_figueiredo_orbital_2022,ye_extraordinary_2014}. It does not seem to fit in the Mott scenario \citep{de_medici_selective_2014,fanfarillo_electronic_2015} because of the absence of effective doping. In turn, it suggests that Mn tunes MnBFA from a correlated metal to a Hund insulating state, wherein the insulating ground state is determined mainly by a larger Hund's interaction resulting from the change in Fe$3d$As$4p$ hybridization \citep{yao_comparative_2011,mcnally_hunds_2015,stadler_differentiating_2021,de_figueiredo_orbital_2022}.

The upper and lower panels in Fig. \ref{fig:selfE}$(e)$ summarize our findings concerning the electronic structure of the PM state. The upper panel shows schematics of the BFA PM Fermi surface. Nested electron and hole states are connected by a $(\pi,\pi)$ vector drawn to scale. The lower panel shows the respective schematics for MnBFA. Mn causes a significant broadening of all electronic states and the shrinking and deformation of the electron pockets, resulting in the partial detuning of the nesting condition.

% Figure environment removed

We now turn to the reconstructed electronic structures, characterized at $T=20$ K, well below the magnetic transition. First, we discuss the nematic splitting $\Delta$, between bands with $d_{xz}$ and $d_{yz}$ orbital character, as a function of Mn. The second derivative of these electronic band maps along the $\Gamma X$ and $YZ$ directions are shown in Fig \ref{fig:ordered-state}$(a)$-$(b)$. The pink and green cuts are indicated in Fig \ref{fig:ordered-state}$(c)$, where we show a FS map for the BFA sample (obtained with LV polarization and along $\Gamma X$). We could not resolve the reconstructed ``petal-like'' $4$-fold symmetric shape recently reported \citep{watson_probing_2019}. From the $YZ$ cut, we can access $\Delta$ of the shallow electron-like band, reported as $\Delta=40$ meV for the BFA \citep{fuglsang_jensen_angle-resolved_2011}. This splitting, the consequence of breaking the degeneracy of the $d_{xz}/d_{yz}$ electron bands, manifests as well in the almost flat electron band close to the $X$ point, as shown in the lower panel of Fig \ref{fig:ordered-state}$(a)$. Indeed the entire band appears duplicated along the $\Gamma X$ direction. Using EDCs to fit the band positions, we superimpose the bands performing a rigid energy shift on the lower duplicated band to estimate $\Delta$ and Fig. \ref{fig:ordered-state}$(d)$ presents $\Delta$ as a function of Mn.

There is still debate about the nematic splitting size for these materials \citep{fedorov_energy_2019}, reported as $60$ meV for FeSe thin film \citep{zhang_distinctive_2016} and $70$ meV for BFA \citep{yi_symmetry-breaking_2011}, which is in good agreement with our findings. We can observe that for both bands there is no scaling between $\Delta$ and $T_{\text{SDW}}$: indeed, $\Delta$ decreases only about $20\%$ from its value for BFA whereas $T_{\text{SDW}}$ decreases by about $60\%$.\textcolor{blue}{{} }This splitting is currently understood as evidence of nematic ordering and consequence of the orthorhombic distortion. It is only for $x>0.1$ that Mn substitution suppresses the orthorhombic transition \citep{thaler_physical_2011,inosov_possible_2013,wang_impact_2014}. For $x<0.1$, $T_{\text{SDW }}$ and the orthorhombic distortion are intertwined, but the weak $\Delta$ dependence on Mn content for $x<0.1$ is not reflecting this phenomenology.

The presented ARPES results can be applied to an in-depth reexamination of previous Resonant Inelastic X-ray Scattering (RIXS) experiments of MnBFA samples \citep{garcia_anisotropic_2019}, which characterized the magnon dispersions along $\Gamma X$ and $\Gamma M$ as a function of Mn. In Fig. \ref{fig:ordered-state}$(g)$, we show the RIXS measured ($T=20$ K) magnon damping coefficients ($\gamma$) and the magnon bare frequencies $(\omega_{0}$) for BFA and Mn$8.0\%$ samples, as a function of the in-plane momentum, $||\boldsymbol{q}||$, along the main symmetry directions. As a function of Mn, $\omega_{0}$ remains unaffected, whereas $\gamma$ increases, with the excitations becoming overdamped ($\omega_{0}\lessapprox\gamma/2$) for almost all values of $||\boldsymbol{q}||$ in the case of the Mn$8.0\%$ sample. This abnormally large magnon damping is not observed in RIXS results for other IBS materials \citep{pelliciari_reciprocity_2019,pelliciari_local_2017,zhou_persistent_2013}.

The RIXS measured magnons are due to spin flips associated with the $d_{xy}$ orbitals \citep{kaneshita_spin_2011}. Our ARPES data show explicitly that the reconstructed electronic structure is related to an energy scale of the order of $60$ meV in the case of BFA. However, the relative change with Mn substitution in $\Delta$ is at most $20\%$ of this value, $\approx12$ meV, which is beyond the highest resolution RIXS experiments to date. In this sense, the positioning of the $d_{xy}$-derived bands in the reconstructed electron structure is also not affected by Mn content, which is compatible with the RIXS measured $\omega_{0}$ values being not affected by Mn. The magnon damping $\gamma$, however, is strongly increased by Mn, suggesting that the cooperative behavior between the Mn local moments and conduction electrons plays the key role in promoting the scattering of the Fe-derived excitations by the short-range Néel fluctuations \citep{fernandes_suppression_2013,gastiasoro_enhancement_2014,garcia_anisotropic_2019}. Therefore, the present observation by ARPES of increasing band incoherence, naturally connects with the RIXS results.

\emph{Conclusions and outlook:} our experiments and analysis show that Mn cause the electron pockets to shrink, despite the absence of effective charge doping. The dominant effect of Mn, however, is increasing the electronic bands' incoherence. We suggest that the latter and magnetic scattering \citep{fernandes_suppression_2013,gastiasoro_enhancement_2014,garcia_anisotropic_2019} are the control parameters for the evolution of $T_{\text{SDW}}$. We also show explicitly that Mn causes a sizable effect on the nematic splitting $\Delta$, which decreases with increasing Mn content. 

Our findings support that the indirect exchange between Mn local moments is mediated by incoherent electronic states, explaining the glassy behavior of Mn spins \citep{inosov_possible_2013}. Indeed, as previously observed \citep{thaler_physical_2011}, the suppression of $T_{\text{SDW}}$ as a function of Cr and Mn content in substituted BFA does not depend on the nature of the dopant, but the spin glass state of the Mn spins distinguishes the physics of Cr and Mn substitutions. The electronic band incoherence here observed should be a feature only of the MnBFA. We should also comment on a recent analysis of Mn and Cr substituted $1144$ IBS materials \citep{xu_superconductivity_2022,xu_superconductivity_2023}, which proposes that the amount of doped holes is not controlling the suppression of $T_{\text{C}}$ and $T_{\text{SDW}}$ for these substitutions.

Our findings also suggest that Mn substitution enhances electronic correlations in MnBFA, placing it in between the correlated metal and Hund insulating phases. Indeed, since charge doping is absent, the observed increase in correlations is likely a result of the changing orbital physics, favoring the Hund insulating scenario \citep{mcnally_hunds_2015,stadler_differentiating_2021}. In this regard, whereas our findings provide the mechanism for the glassy behavior of Mn spins, the specific proposition of a Griffths-like phase in MnBFA \citep{inosov_possible_2013} (or even in Mn substituted SrFe$_{2}$As$_{2}$ \citep{chen_miscibility_2021}) should be reexamined, since in general grounds it demands proximity to a Mott phase \citep{giovanni_anderson_2021,andrade_electronic_2009,tanaskovic_effective_2004}. 

Finally, given our results, the picture advanced in Refs. \citep{gastiasoro_enhancement_2014,gastiasoro_unconventional_2016} provide the most complete scenario to explain the absence of SC in MnBFA samples, since our work shows that disorder is an integral property of the electronic states in MnBFA.
\begin{acknowledgments}
We thank Eric C. Andrade for fruitful discussions. We acknowledge MAX IV Laboratory for time on Bloch Beamline under Proposal 20200293 and the support of Gerardina Carbone and Craig Polley. Research conducted at MAX IV, a Swedish national user facility, is supported by the Swedish Research council under contract 2018-07152, the Swedish Governmental Agency for Innovation Systems under contract 2018-04969, and Formas under contract 2019-02496. The Fundação de Amparo à Pesquisa do Estado de São Paulo financial support is acknowledged by M.R.C. (Grants No. 2019/05150-7 and No. 2020/13701-0), F.A.G. (Grant No. 2019/25665-1); K.R.P., P.G.P. and C.A. (Grant No. 2017/10581-1). P.G.P. and C.A. acknowledge financial support from CNPq: Grants No. 304496/2017-0, 310373/2019-0, and 311783/2021-0. Work at Los Alamos was supported by the U.S. Department of Energy, Office of Basic Energy Sciences, Division of Materials Science and Engineering: project ``Quantum Fluctuations in Narrow-Band Systems''.
\end{acknowledgments}

\bibliographystyle{apsrev4-1}
\bibliography{MnSubsBaFe2As2_electronic_magnetic}

\pagebreak
\widetext

\includepdf[pages={{},1-5}]{SupplementalMaterial_2023_MCantarino_ARPES_MnBFA_resub.pdf}


\end{document}
