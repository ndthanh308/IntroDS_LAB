
A $k$-Counter Net ($k$-CN) is a finite-state automaton equipped with $k$ integer counters that are not allowed to become negative, but do not have explicit zero tests. 
This language-recognition model can be thought of as labelled vector addition systems with states, some of which are accepting.
Certain decision problems for $k$-CNs become easier, or indeed decidable, when the dimension $k$ is small.
Yet, little is known about the effect that the dimension $k$ has on the class of languages recognised by $k$-CNs. 
Specifically, it would be useful if we could simplify algorithmic reasoning by reducing the dimension of a given CN.

To this end, we introduce the notion of dimension-primality for $k$-CN, whereby a $k$-CN is prime if it recognises a language that cannot be decomposed into a finite intersection of languages recognised by $d$-CNs, for some $d<k$. 
We show that primality is undecidable.
We also study two related notions: dimension-minimality (where we seek a single language-equivalent $d$-CN of lower dimension) and language regularity.
Additionally, we explore the trade-offs in expressiveness between dimension and non-determinism for CN.

% \shtodo{Suggestion for revised abstract:}
% A $k$-Counter Net ($k$-CN) is a finite-state automaton equipped with $k$ integer counters that are not allowed to become negative, but do not have explicit zero tests. They are language-recognition models, and can be thought of as labelled VASS with accepting states.
% Certain decision problems for $k$-CN become easier (or indeed, decidable) when the dimension $k$ is small, and yet little is known about the effect of the dimension $k$ on the class of languages recognised by $k$-CN. Specifically, it would be useful if we could simplify algorithmic reasoning by reducing the dimension of a given CN.

% To this end, we introduce the notion of dimension-primality for $k$-CN, whereby a $k$-CN is prime if its language cannot be recognised by an intersection of $d$-CNs, for $d<k$. We show that primality is undecidable, and study two related notions: dimension minimization, where we seek an equivalent $d$-CN of smaller dimension, and regularity (i.e., is the recognised language regular). We also explore the trade-offs in expressiveness between dimension and non-determinism for CN.

% \aytodo{Current Version}
% Vector Addition Systems with States (VASS) are finite state machines equipped with a set of integer counters that cannot decrease below zero and cannot be explicitly tested for zero. In this work we focus on labeled VASS, also known as Counter Nets, that serve as language recognizing models. Although there is an active research frontier exploring the boundaries between cabalities of VASSes of different dimensions, very little has been known about our ability to decompose a VASS into VASSes of lower dimension. In this work we introduce the notion of VASS Primality, signifying that the language of a labeled VASS cannot be expressed as an intersection of languages of VASSes of lower dimension, and show that VASS primality is undecidable. We prove that it is undecidable even when seeking a single equivalent VASS of lower dimension, rather than a finite set, a variant we call VASS dimension minimization. Conversely, we show that deciding deterministic VASS' regularity is decidable, and demonstrate various results exploring the trade-off between dimension and nondeterminism, showing how and when they can compensate for one another. 

% \hstodo{\henry{old suggestion:}
% A $k$-Counter Net ($k$-CN) is a finite state automata equipped with $k$ integer counters that must always take a non-negative value but cannot be tested for zero explicitly.
% This model can be seen as a Vector Addition Systems with States that has a label on every transition.
% Counter nets are a fundamental language recognition model that are well and widely studied, especially one-counter nets.
% We introduce the notion of dimension primality for counter nets which signifies whether the language of a counter net can be decomposed into an intersection of languages of lower dimension counter nets.
% We show that deciding whether a counter net is dimension prime is undecidable, in part by constructing a dimension prime two-counter net.
% Furthermore, given a counter net, it turns out that dimension minimisation is undecidable, that is to determine whether there exists single language equivalent counter net of lower dimension.
% Conversely, we show decidability of language regularity for deterministic counter nets, a problem that is known to be undecidable already for non-deterministic counter nets, already in dimension one.
% Additionally, we demonstrate various trade-offs in the expressivity of (non)-deterministic counter nets of various dimensions.
% }

% \aytodo{Before Implementing Shaull's Last Comments}
% Vector Addition Systems with States (VASS) are finite state machines equipped with a set of integer counters that cannot decrease below zero and cannot be explicitly tested for zero. In this work we focus on labeled VASS, also known as Counter Nets, that serve as language recognizing models, utilizing their strong semantics for recognizing more complex languages than other, weaker types of automata. 
% \shtodo{the sentence above is vague. What is it supposed to mean?}
% We introduce the notion of VASS Primality, signifying that the language of a labeled VASS cannot be expressed as an intersection of languages of VASSes of lower dimension, and show that deciding VASS primality is undecidable. We prove that it is undecidable even when restricting the set of VASS factors to be of size 1,
% \shtodo{confusing, since it can be taken to mean that there are a set of VASSes with 1 state.}
% a variant we call VASS dimension minimization. Conversely, we show that deciding deterministic VASS' regularity is decidable, and demonstrate various bite sized results
% \shtodo{I like the phrase ``bite sized results'', but not for the abstract/paper. Only for the presentation.}
% exploring the trade-off between dimension and nondeterminism, showing how and when they can compensate for one another. Our contribution stems from the fact that very little is known about our ability to decompose a VASS into VASSes of lower dimension, although there is an active research frontier exploring the boundaries between the cabalities of VASSes of different dimensions.
% \shtodo{The last sentence (or rather - it's contents) should be given earlier. Try to imaging yourself reading the abstract. The things that should go through your head are: ``Ok, VASS - I've heard of those. Hmm, simplifying/decomposing them does sound important! Oh, primality is indeed an interesting notion. Ah, you show it's undecidable. Oh, and this other problem is decidable, nice. And you also study something to do with nondeterminism v.s. dimensions. Got it.''}


% They are an alternative, and essentially equivalent, formalism of Petri Nets that is well suited for modeling concurrent systems in various fields.
% \shtodo{They are...fields is irrelevant for the abstract. Recall that the abstract is only meant to tell readers roughly what the paper is about. It's not a sales pitch.}
% VASSes of lower dimension typically yield more manageable decision problems, and while there is an active research frontier exploring the boundaries between the cabalities of VASSes of different dimensions, very little is known about our ability to decompose a VASS into VASSes of lower dimension. 
% \shtodo{Same thing -- we don't need to motivate the problem, just say what we're doing.}

% \shtodo{Overall: after the sentence introducing VASS, I suggest to add a sentence saying that we focus on Labelled-VASS, also called Counter Nets, and that these are language-recognizing machines. Then move on to saying what we study about them.\\
% After that, if it's not too long already, we can add a sentence explaining the importance of the contribution (which is roughly what you have above currently -- exploring bounds, bla bla).}