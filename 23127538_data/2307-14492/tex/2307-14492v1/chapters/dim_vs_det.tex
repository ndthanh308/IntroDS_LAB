\section{On Dimension vs. Nondeterminism}
\label{sec:dimension_vs_nondet}
\cref{thm:2CN_prime} shows that (unsurprisingly) increasing the dimension increases expressiveness, and that nondeterministic models are more expressive than deterministic ones. In particular, $k$-DCN can be decomposed by projection (\cref{prop:DCN_Projection}), and have decidable regularity (\cref{thm:DCN_regularity_decidable}).
It is therefore interesting to study the interplay between increasing the dimension and allowing nondeterminism. In this section we show that the two notions are incomparable, in a sense. 

We start by showing that nondeterminism can sometimes compensate for dimension.
Let $k\in \bbN$, and $\Sigma=\{a_1,\ldots,a_k,b_1,\ldots,b_k,c\}$. Consider the language 
$L_k=\{a_1^{n_1}a_2^{n_2}\cdots a_k^{n_k}b_ic^m\mid i\in [k]\wedge n_i\ge m\}$.
It is easy to construct a $k$-DCN for $L_k$, as depicted in \cref{fig:k_DCN_with_1CN_but_no_k-1_DCN} for $k=3$. By introducing nondeterminism, we can also construct a $1$-CN for $L_k$, as depicted in~\cref{fig:1_CN_for_k_DCN_with_1CN_but_no_k-1_DCN} (by guessing what which $b_i$ will be seen, and using the single counter in the $a_i^{n_i}$ part). However, we show that the dimension cannot be minimised while maintaining determinism.
% Figure environment removed
% Figure environment removed

\begin{proposition}
\label{prop:k_DCN_no_k-1_DCN}
$L_k$ is not recognisable by a $(k-1)$-DCN
\end{proposition}
\begin{proof}
Assume by way of contradiction that $\cD$ is $(k-1)$-DCN $\cD=\tup{\Sigma,Q,Q_0,\delta,F}$ such that $\lang(\cD)=L_k$.

Let $n>|Q|$ and for every $i\in [k]$ consider the word $w_i=a_1^na_2^n\cdots a_k^nb_ic^n\in L_k$. Since $\cD$ is deterministic and $n>Q$, all the accepting runs on the $w_i$ coincide up to the $b_i$ part, and have cycles in each $a_i^n$ segment, as well as in the $c^n$ segment (the latter may differ according to $i$). Let $M$ be the product of lengths of all these cycles.

Observe first that the cycles in all of the $a_i^n$ segments cannot decrease any counter. Indeed, otherwise by pumping the cycle we would have that there are no $\bbN$-runs on words starting with $a_1^n\cdots a_i^{n+tM}$, for some $t>0$, which is a contradiction since such words can be accepted with an appropriate suffix.

Thus, all $a_i$ cycles have non-negative effect for all counters. For each counter $i$, associate with $i$ the minimal segment index whose cycle strictly increases $i$. Since there are $k-1$ counters and $k$ segments this map is not surjective, i.e. there is a segment (w.l.o.g. the $a_k$ segment) such that every counter that is increased in the $a_k$ cycle is also increased in a previous segment. Then, however, there exist $s,t>0$ such that the word 
\[a_1^{n+sM}a_2^{n+sM}\cdots a_{k-1}^{s+sM}a_k^nb_kc^{n+tM}\] is accepted by $\cD$, which is a contradiction. 
\end{proof}

We now turn to show that conversely, dimension can sometimes compensate for nondeterminism. Consider the language $K_2=\{a^kb^\ell c^md^n\mid k\ge m\wedge \ell\ge n\}$. 
\begin{proposition}
    \label{prop:2DCN_no_1CN}
    $K_2$ can be recognised by a $2$-DCN, but not by a $1$-CN.
\end{proposition}
\begin{proof}
    Constructing a $2$-DCN for $K_2$ is easy -- the first counter is increased on $a$ and decreased on $c$, and the second is increased on $b$ and decreased on $d$.

    We now argue that there does not exist a $1$-CN recognising $K_2$.
    Assume by way of contradiction that there exists a $1$-CN $\cA$ recognising $K_2$.
    Suppose that $\cA$ comprises $n$ states and that $M$ is the absolute value of the largest transition effect.
    We analyse the behaviour of $\cA$ when accepting the word $w = a^mb^mc^md^m$ for $m > n\cdot M$.
    Since $m > n$, any accepting run on $w$ must take some cycle reading `$a$'s, some cycle reading `$b$'s, some cycle reading `$c$'s, and some cycle reading `$d$'s.

    First, it is never the case that a cycle reading `$c$'s or `$d$'s has non-negative effect, otherwise $\cA$ accepts a word $a^mb^mc^{m+x}d^m \notin K_2$ or $a^mb^mc^md^{m+x} \notin K_2$, for some $x > 0$.
    Now, among all accepting runs on $w$, it must be the case that either a positive cycle reading `$a$'s or a positive cycle reading `$b$'s was taken, otherwise we can remove the cycle and accept a word with too few `$a$'s or `$b$'s.
    
    Assume that there exists a positive cycle $\pi$ reading `$a$'s on some accepting run, and recall that there also exists a negative cycle $\nu$ reading `$d$'s.
    We can arbitrarily increase the counter value using $\pi$ to then take an additional iteration of $\nu$ later in the run.
    Hence, $a^{m+x}\,\#\,b^m\,\#\,c^m\,\#\,d^{m+y}$ is accepted by $\cA$ for some $y>0$ and some $x$ large enough, however $a^{m+x}\,\#\,b^m\,\#\,c^m\,\#\,d^{m+y} \notin K_2$.
    Symmetrically, the same argument holds if instead there does not exist a positive cycle reading `$a$'s but there does exist a positive cycle reading `$b$'s.
    Therefore, we are able to conclude that no such 1-CN $\cA$ recognising $K_2$ exists.
\end{proof}

% Recall that by~\cref{prop:DCN_Projection}, a $k$-DCN is composite by projection. While this clearly fails for $k$-CN, it raises the question of whether we can decide, given a $k$-CN, if its language is equal to the intersection of its projections. Unfortunately, we now show that this is generally undecidable.
% \begin{theorem}
% The following problem is undecidable: given a $2$-CN $\cA$, decide whether $\lang(\cA)=\lang(\cA|_1)\cap \lang(\cA|_2)$.
% \end{theorem}
% \begin{proof}
%     We will present a reduction from the language inclusion problem between two $1$-CNs, which is undecidable by~\cite{hofman2013decidability}: given two $1$-CNs $\cU,\cV$, is $\lang(\cU)\subseteq \lang(\cV)$?

%     Let $\cU=\tup{\Sigma,Q,Q_0,\delta,F}$ and $\cV=\tup{\Sigma,S,S_0,\mu,G}$ be $1$-CNs. We construct a $2$-CN $\cA$ as follows. Intuitively, $\cA$ starts by reading a fresh letter $\#$, after which it can either start simulating $\cV$ using its first counter, or take a transition with cost $(0,-1)$ to a copy of $\cU$, also simulating it on the first counter.
%     Clearly this initial $(0,-1)$ transition cannot be taken by $\cA$, and hence $\lang(\cA)=\#\cdot \lang(\cV)$.
    
%     However, we also immediately have that $\lang(\cA|_1)=\#\cdot (\lang(\cU)\cup \lang(\cV))$ and $\lang(\cA|_2)=\lang(\cV)$. Thus, it holds that $\lang(\cA)=\lang(\cA|_1)\cap \lang(\cA|_2)$ iff $\#\cdot \lang(\cV)=\#\cdot (\lang(\cV)\cap \lang(\cU)\cup \lang(\cV))$

%     % Figure environment removed
    
%     Let us construct $\Aa$; include a starting state $q$ branching to a modified copy $\Uu'$ of $\Uu$ via transition $s$ and a modified copy $\Vv'$ of $\Vv$ via transition $t$.
%     Importantly, the modifications apply the counter updates of the two 1-VASS only to the first counter of the 2-VASS $\Aa$.
%     More precisely, we define $\Uu' = (P, \Sigma, S', p_s, E)$ where $S' = \set{ (p, \sigma, (u, 0), p') : (p, \sigma, u, p') \in S}$ and $\Vv' = (Q, \Sigma, T', q_s, F)$ where $T' = \set{ (q, \sigma, (v, 0), q') : (q, \sigma, v, q') \in T}$.
%     Note that $\lang(\Uu') = \lang(\Uu)$ and $\lang(\Vv') = \lang(\Vv)$.
%     Consider the transitions $s = (q, \varepsilon, (0, -1), p_s)$ to $\Uu'$ and $t = (q, \varepsilon, (0, 0), q_s)$ to $\Vv'$.
%     See Figure~\ref{fig:extra-dimension-2-undecidable} for a schematic overview of $\Aa$.
%     Given the negative effect, the transition $s$ cannot be traversed from $q(0, 0)$ in $\Aa$, hence $\lang(\Aa) = \lang(\Vv') = \lang(\Vv)$.
%     However in the projected labelled 1-VASS $\Aa[1]$, the projection $(q, \varepsilon, 0, p_s)$ of the transition $s$ can be traversed from $q(0)$, therefore $\lang(\Aa[1]) = \lang(\Uu') \cup \lang(\Vv') = \lang(\Uu) \cup \lang(\Vv)$.
%     Thus $\lang(\Aa) \supseteq \lang(\Aa[1]) \iff \lang(\Vv) \supseteq \lang(\Uu) \cup \lang(\Vv) \iff \lang(\Uu) \subseteq \lang(\Vv)$.
% \end{proof}