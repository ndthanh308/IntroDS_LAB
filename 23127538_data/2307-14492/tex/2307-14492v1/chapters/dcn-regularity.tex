% %\section{Decidable Fragments in the Deterministic Setting}
% %\label{sec:decidable_DCN}
% We now turn our attention to decidable fragments of primality. A natural candidate for decidability is the deterministic fragment. 
% Recall that by \cref{prop:DCN_Projection}, a $k$-DCN is dimension minimal if and only if it is $(1,k)$-composite. 
% Thus, dimension-minimality ``captures'' primality, and is our focus therefore. 
% Following, we show that regularity, equivalent to being $(0,1)$-composite, is decidable for $k$-DCN, and being $(1,1)$-composite is decidable for $2$-DCN.
% \shtodo{make sure we actually show both}

\section{Regularity of $k$-DCN is Decidable}
\label{sec:decidable_DCN}
 We now turn our attention to decidable fragments of primality. A natural candidate for decidability is the deterministic fragment. 
Recall that by \cref{prop:DCN_Projection}, a $k$-DCN is dimension minimal if and only if it is $(1,k)$-composite. 
Thus, dimension-minimality ``captures'' primality, and is our focus. 
Following, we show that regularity, equivalent to being $(0,1)$-composite, is decidable for $k$-DCN.
%, and being $(1,1)$-composite is decidable for $2$-DCN.

%Our approach to proving decidability of regularity for $k$-DCN 
Our proof approach is to reduce the problem to regularity of Vector Addition Systems (VAS).
In our context, a VAS is a single-state $k$-DCN. In particular, we can then associate the letter corresponding to each transition with the transition itself (due to the single state and determinism). 
Regularity of VAS was shown to be decidable in~\cite{BlockeletS11}. 

\begin{remark}
\label{rmk:regularity_VASS}
Regularity of VAS with states (VASS) was shown to be decidable in~\cite{Demri13}. There, however, the model does not have accepting states. 
It is relatively straightforward to extend the techniques used there to our setting, but would require re-proving several results. Therefore, we take the approach through VAS.
\end{remark}

Throughout this section, we consider a $k$-DCN $\cD=\tup{\Sigma,Q,Q_0,\delta,F}$. 
Our construction proceeds as follows. 
First, we obtain from $\cD$ a $k$-DCN $\cC$ with the same structure, but such that its transitions are distinctly labelled. 
Next, we invoke a technique of Hopcroft and Pansiot~\cite{HopcroftP79}; we will construct a single-state $k+3$-DCN (namely a VAS) $\cU$ whose language ``approximates'' that of $\cC$, and in particular preserves regularity. 
Finally, we invoke the deciability of regularity for deterministic VAS~\cite{BlockeletS11}. 
The crux of the argument is to maintain regularity throughout, even though the language changes in each step.

In the following, a \emph{(resp. Non-) deterministic Finite Automaton} (resp. NFA, DFA) is a $0$-CN (resp. $0$-DCN).
\begin{lemma}
    \label{lem:DCN_to_relabelled}
    Let $\cD$ be as above, and define $\cC=\tup{\Gamma,Q,Q_0,\delta',F}$ where $\Gamma=\{\gamma_t\mid t\in \delta\}$ with $(p,\gamma_t,\vec{v},q)\in \delta'$ iff $t=(p,\sigma,\vec{v},q)\in \delta$. Then $\lang(\cD)$ is regular if and only if $\lang(\cC)$ is regular.
\end{lemma}
\begin{proof}
    For the first direction, assume $\lang(\cC)$ is regular, we show that $\lang(\cD)$ is regular.
    Let $\cA$ be a DFA such that $\lang(\cA) = \lang(\cC)$. 
    Now consider the process of reverting the labelling, to construct an NFA $\cA'$ with the same state space and same transition structure, but different labels, as follows.
    For a transition $(s, \gamma_t, t)$ in $\cA$, let $t = (p, \sigma, \vec{v}, q)$ be the transition in $\cD$ corresponding to $\gamma_t$, then we introduce in $\cA'$ the transition $(s, \sigma, t)$. Note that $\cA'$ may be nondeterministic.
    
    Now, since $\lang(\cA) = \lang(\cC)$, we claim that $\lang(\cA') = \lang(\cD)$. 
    Indeed, consider $w\in \lang(\cA')$ with $w=\sigma_1\cdots \sigma_n$, then there exist transitions $t_1,\ldots, t_n$ where each $t_i$ is labelled $\sigma_i$ such that $\gamma_{t_1}\cdots \gamma_{t_n}\in \lang(\cA)=\lang(\cC)$. Then, however, the sequence $t_1,\ldots, t_n$ represents an accepting run of $\cD$, which is labelled by $w=\sigma_1\cdots \sigma_n$, so $w\in \lang(\cD)$.

    Conversely, consider $w\in \lang(\cD)$ with $w=\sigma_1\cdots \sigma_n$, then the sequence of transitions in the accepting run of $\cD$ on $w$ is $t_1,\ldots, t_n$ such that $\gamma_{t_1}\cdots \gamma_{t_n}\in \lang(\cC)=\lang(\cA)$, so by the construction of $\cA'$ we have that $w=\sigma_1\cdots \sigma_n\in \lang(\cA')$. We conclude that if $\lang(\cC)$ is regular, then so is $\lang(\cD)$.

    For the second direction, assume $\lang(\cD)$ is regular, we show that $\lang(\cC)$ is regular. 
    Let $\cA$ be a DFA such that $\lang(\cA) = \lang(\cD)$. 
    We consider a product DFA $\cA\times \cD$ which operates as follows: its alphabet is $\Gamma$, and when reading letter $t=(p,\sigma,\vec{v},q)$ from state $(r,p)$ (where $r$ is a state of $\cA$ and $p\in Q$) it transitions to $(r',q)$ where $r'$ is where $\cA$ gets when reading $\sigma$ from $r$. Acceptance is determined by $\cA$. Intuitively, the product simulates $\cA$ while keeping track that the transitions being read actually compose a (possibly $\bbZ$) run of $\cD$. 

    We claim that $\lang(\cA\times \cD)=\lang(\cC)$. Indeed, $\cC$ accepts a sequence $\gamma_{t_1}\cdots \gamma_{t_n}$ iff $t_1\cdots t_n$ forms an accepting run in $\cD$, iff $t_1\cdots t_n$ forms a possibly $\bbZ$-run of $\cD$ and the letters on the transitions are $\sigma_1\cdots\sigma_n\in \lang(\cD)=\lang(\cA)$. 
    
    Note that this last ``iff'' relies on $\cD$ being deterministic, so that if we found any run of $\cD$ on an accepted word, then this run must be accepting.
\end{proof}

%Given a labelled DVASS $\Vv = (Q, T)$ over an alphabet $\Sigma$, we will construct the relabelled DVASS $r(\Vv) = (Q, T')$ over the alphabet $\Sigma' = \set{ \sigma_t : t \in T }$.
%For each transition $t = (p, a, \vec{x}, q) \in T$ in the original DVASS $\Vv$, a transition $(p, \sigma_t, \vec{x}, q) \in T'$ is present in the relabelled DVASS $r(\Vv)$.

% \begin{proposition}
%     Let $\Vv$ be a DVASS, then $\lang(\Vv)$ is regular if and only if $\lang(r(\Vv))$ is regular.
%     \label{pro:relabel-dvass}
% \end{proposition}
% \begin{proof}\henry{probably move to appendix}
%     \subparagraph*{$\lang(r(\Vv))$ is regular $\implies$ $\lang(\Vv)$ is regular:}
%     Assume that $\lang(r(\Vv))$ is regular, so there exists a DFA $\Dd$ such that $\lang(\Dd) = \lang(r(\Vv))$. 
%     Now consider the process of reverting the labelling, to construct a DFA $\Dd'$ with the same state space and same transition function structure, but different labels.
%     If a transition $(s, \sigma_t, t)$ is present in $\Dd$, then if $t = (p, a, \vec{x}, q)$ in the original DVASS $\Vv$, add a transition $(s, a, t)$ is present in $\Dd'$.
%     Now, since $\lang(\Dd) = \lang(r(\Vv))$, it must hold that $\lang(\Dd') = \lang(\Vv)$.
%     \shtodo{this ``must'' is not entirely trivial. Took me a while to convince myself of it. We should add a rigorous argument.}

%     \subparagraph*{$\lang(\Vv)$ is regular $\implies$ $\lang(r(\Vv)))$ is regular:}
%     Assume that $\lang(r(\Vv))$ is not regular, so there exists a word $x\,y\,z \in L(r(\Vv))$ such that $|y| \geq 1$ and there exists an $i \in \N$ such that  $x\,y^i\,z \notin L(r(\Vv))$.
%     Since $\Vv$ is deterministic so too is $r(\Vv)$, so there is exactly one path $\pi$ in $r(\Vv)$ that witnesses $x\,y\,z$ and there is exactly one path $\rho$ in $r(\Vv)$ that witnesses $x\,y^i\,z$.
%     The relabelled DVASS $r(\Vv)$ only differs from $\Vv$ by what labels are present on each transition.
%     Therefore, it must be true that the path $\pi$ witnesses the word $(x\,y\,z)'$ in $\Vv$ and the path $\rho$ witnesses the word $(x\,y^i\,z)'$ in $\Vv$, where for a given word $w = \sigma_{t_1}\sigma_{t_2}\cdots\sigma_{t_k} \in \lang(r(\Vv))$ the `un-relabelled' word $w' = a_1 a_2 \cdots a_k$ consists of letters $a_i$ labelling the transitions $t_i = (p, a_i, \vec{x}, q)$ in $\Vv$.
%     Since the accepting states have not been modified, and the counter updates are the same on each transition, then $x\,y\,z \in \lang(r(\Vv)) \implies (x\,y\,z)' = x'\,y'\,z'\in \lang(\Vv)$, and $(x\,y^i\,z)' \in \lang(r(\Vv)) \implies (x\,y^i\,z)'= x'\,{y'}^{\,i}\,z' \in \lang(\Vv)$.
%     Hence $\lang(\Vv)$ is not regular.
%     \aytodo{Idea to prove this direction: We construct a DFA that accepts $\lang(r(\Vv))$. We know that $\lang(\Vv))$ is regular, so let $D$ be a DFA that accepts it. Also let $V^-$ be $V$ without its weights. Claim - $A=V^- \times D$ is a DFA whose language is $\lang(r(\Vv)))$ (it reads transitions of $V$. for every letter - the component $V^-$ progresses wherever this transition takes it, and $D$ progresses according to the corresponding letter). Correctness proof: first direction - a word $w$ that represents a sequence of transitions that leads legally to an accepting state in $V$ can obviously also be accepted by $A$. Other direction - if a sequence of transitions doesn't lead legally to an accepting state in $V$ - then either it's not a legal sequence of transitions flow-control wise, or it is but some counter drops below 0. If it's the first case - the component $V^-$ catches it. If it's the latter - the component $D$ catches it. If $D$ doesn't catch such violation - then we have a word $u$ accepted by $D$ but not by $V$ (the only run of $V$ on $u$ ends due to a counter violation).}
% \end{proof}


%We continue by constructing a  that can simulate the original labelled DVASS whilst almost preserving the language.
We now employ a well-known technique of Hopcroft and Pansiot~\cite{HopcroftP79} that converts a $k$-VASS to a $k+3$-VAS by simulating the finite control states using three extra dimensions.
The first $k$ dimensions just mirror the $k$ dimensions in the original VASS and the last three dimensions operate the state simulation; one of the three extra dimensions explicitly maintains a value corresponding the the current state.
In the unlabelled (VASS) setting, the construction takes a transition $(p, \vec{x}, q)$ and spawns three transitions $\vec{v}_1$, $\vec{v}_2$, and $\vec{v}_3$ in the corresponding VAS.
Crucially, these vectors are chosen in such a way that if $(p, \vec{x}, q)$ is taken in the original VASS, then $\vec{v}_1, \vec{v}_2, \vec{v}_3$ can be taken in sequence.
Moreover, in the VAS, if $\vec{v}_1$ is taken, then the only transition that could follow is $\vec{v}_2$, and then the only transition that could follow is $\vec{v}_3$; ending with the last three dimensions taking values corresponding to the current state being $q$.

Following, in Lemma~\ref{lem:relabelled_to_VAS}, we show that one can replicate this construction for the DCN case starting with the distinctly-labelled $\cC$ and obtaining a single-state $k+1$-DCN $\cU$.
In particular, if a $a$-transition $(p, a, \vec{v}, q)$ is present in the $k$-DCN $\cC$, then labelled transitions $a_1,\vec{v_1}$, $a_2,\vec{v_2}$, and $a_3,\vec{v_3}$ are present in the constructed $k+3$-DCN $\cU$.

% For regularity testing, it is critical that $\cU$ we obtain has only one transition per label, so we first ensure that each letter $a$ labels at most one transition in the original VASS and then three distinct letters $a_1$, $a_2$, and $a_3$ label the transitions in the VAS.
% Note that one can use Proposition~\ref{pro:relabel-dvass} prior to Lemma~\ref{lem:labelled-dvas} to ensure that the DVAS obtained consists of one transition for each label.
% Since if there exists two $a$-transitions $(p, a, \vec{x}, q)$ and $(r, a, \vec{y}, s)$, then there would exist two $a_1$-transitions, two $a_2$-transitions, and two $a_3$-transitions in the labeled VAS making it not deterministic. \henry{maybe a comment here on syntactic vs semantic determinism being important?}

\begin{lemma}[Corollary of~{\cite[Lemma 2.1]{HopcroftP79}}]
    \label{lem:relabelled_to_VAS}
    Let $\cC$ be the distinctly-labelled $d$-DCN obtained as per \cref{lem:DCN_to_relabelled}, then there exists a single-state $(d+3)$-DCN $\cU$ over alphabet $\Upsilon=\{\gamma_1,\gamma_2,\gamma_3\mid \gamma\in \Gamma\}$ such that
    \begin{equation*}
    \lang(\cU) = 
    \left\{ 
    \begin{tabular}{l|c}
         $a_1 a_2 a_3 \, b_1 b_2 b_3  \,\cdots\, c_1,$ & \\
         $a_1 a_2 a_3 \, b_1 b_2 b_3  \,\cdots\, c_1 c_2,$  & $ a \, b \,\cdots\, c \in \lang(\cC) \wedge a, b, \ldots, c \in \Gamma$ \\
         $a_1 a_2 a_3 \, b_1 b_2 b_3  \,\cdots\, c_1 c_2 c_3$ &
    \end{tabular}       
    \right\}.
    \end{equation*}
\end{lemma}
\begin{proof}
    We give the construction of the VAS, this proof only differs from the proof given by Hopcroft and Pansoit~\cite{HopcroftP79} by the inclusion of transition labels.
    Assume that $\cC$ has $n$ states $Q = \{q_1, \ldots, q_n\}$.
    For each $i \in \set{1, \ldots, n}$, let $a_i = i$ and $b_i = (n+1)(n+1-i)$.
    For each transition $t = (q_i, \sigma, \vec{x}, q_j)$ in $\cC$, there are three transitions $t_1 = \sigma_1,(\vec{0}, -a_i, a_{n+1-i}-b_i, b_{n+1-i})$, $t_2 = \sigma_2,(\vec{0}, b_i, -a_{n+1-i})$, and $t_3 = \sigma_3,(\vec{x}, a_j - b_i, b_j, -a_i)$ in $\cU$ (we omit the states from the transitions, since there is only one state). 
    
    Observe that since each transition in $\cC$ has a distinct label, then each of the transitions in $\cU$ also has a distinct label, making $\cU$ a DCN.
    It remains to show that $\cU$ recognises the language given in the statement of this lemma.
    Suppose the word $\sigma_1\,\sigma_2\,\cdots\,\sigma_k \in \lang(\cC)$, then there is a path $\pi = (t_i)_{i=1}^k$ in $\cC$, so $t_i = (q_i, \sigma_i, \vec{x}_i, q_{i+1})$.
    Since $\Uu$ faithfully simulates $\cC$, the path $\pi' = (t_{i,1}\, t_{i,2}\, t_{i,3})_{i=1}^k$ in $\Uu$ witnesses the word $\sigma_{1,1} \,\,\! \sigma_{1,2} \,\,\!\sigma_{1,3} \; \sigma_{2,1} \,\,\! \sigma_{2,2} \,\,\! \sigma_{2,3} \,\cdots\, \sigma_{k,1} \,\,\! \sigma_{k,2} \,\,\! \sigma_{k,3}$. 
    In fact, the last three transitions need not all be executed; indeed $\sigma_{1,1} \,\,\! \sigma_{1,2} \,\,\! \sigma_{1,3} \,\cdots\, \sigma_{k,1} \in \lang(\Uu)$, $\sigma_{1,1} \sigma_{1,2} \sigma_{1,3} \,\cdots\, \sigma_{k,1} \,\,\! \sigma_{k,2} \in \lang(\Uu)$, and $\sigma_{1,1} \,\,\! \sigma_{1,2} \,\,\! \sigma_{1,3} \,\cdots\, \sigma_{k,1} \,\,\! \sigma_{k,2} \,\,\! \sigma_{k,3} \in \lang(\cU)$. 
\end{proof}
\begin{proposition}
\label{pro:triple-letter-regularity}
    Let $L \subseteq \{a, b, \ldots, c\}^*$ and $L' \subseteq \{ a_1, a_2, a_3, b_1, b_2, b_3, \ldots, c_1, c_2, c_3 \}^*$ such that 
    \begin{equation*}
    L' = 
    \left\{ 
    \begin{tabular}{l|c}
         $a_1 a_2 a_3 \, b_1 b_2 b_3  \,\cdots\, c_1,$ & \\
         $a_1 a_2 a_3 \, b_1 b_2 b_3  \,\cdots\, c_1 c_2,$  & $a \, b \,\cdots\, c \in L$ \\
         $a_1 a_2 a_3 \, b_1 b_2 b_3  \,\cdots\, c_1 c_2 c_3$ &
    \end{tabular}       
    \right\}.
    \end{equation*}
    Then $L$ is regular if and only if $L'$ is regular.    
    
\end{proposition}

\begin{proof}
    This is a basic exercise in automata. The direction $L$ is regular $\implies$ $L'$ is regular is proved by constructing a DFA for $L'$ that verifies letters are read in triplets (i.e., $a_1a_2a_3$, possibly stopping in the middle), and after every triplet simulates the corresponding transition in a DFA for $L$.

    The direction $L'$ is regular $\implies$ $L$ is regular is proved by constructing an NFA for $L$ that given letter $a$ simulates the transition of a DFA for $L'$ on $a_1a_2a_3$.
\end{proof}

\begin{theorem}[cf.~{\cite[Theorem 4.5]{BlockeletS11}}]
    \label{thm:dvas-regularity}
    DVAS regularity is decidable and is in EXPSPACE.
\end{theorem}

%\hstodo{Do we care for complexity? ... Theorem 4.5 in \cite{BlockeletS11} has complexity bounds ... we could try to get: regularity is \class{NL}/\class{PSPACE}-complete for fixed dimension in unary/binary encoded $d$-DVASS and is \class{EXPSPACE}-complete for non-fixed dimension $d$-DVASS?}

%\hstodo{{\bf How does this relate to `the regularity detection problem' in~\cite{Demri13}?}}

\begin{theorem}
    \label{thm:DCN_regularity_decidable}
    Regularity of $k$-DCN is decidable and is in EXPSAPCE.
\end{theorem}
\begin{proof}
    Given a $k$-CN $\cD$, construct a distinctly-labelled $k$-DCN $\cC$ as per \cref{lem:DCN_to_relabelled}.
    %Importantly, $\Ww$ recognises a regular language that is regular if and only if $\Vv$ recognises a regular language.
    Next, apply \cref{lem:relabelled_to_VAS} to obtain a single-state, distinctly-labelled $k+3$-DVAS $\cU$.
    %, we remark that this indeed a deterministic VAS because Proposition~\ref{pro:relabel-dvass} and the proof of Lemma~\ref{lem:labelled-dvas} ensure that each transition has a distinct label.
    Finally, treat $\cU$ as a VAS. 
    Indeed, regularity of VAS is with respect to the transition ``names'', which are unique in $\cU$ due to the distinct labels. 
    We can then decide the regularity of $\cU$ using \cite[Theorem 4.5]{BlockeletS11}. 

    We have that $\cU$ is regular iff $\lang(\cD)$ is regular by~\cref{lem:DCN_to_relabelled,lem:relabelled_to_VAS,pro:triple-letter-regularity}. Moreover, our construction can clearly be implemented in polynomial time, thus giving the same complexity bound.
    %Although $\Ww$ and $\Uu$ are not language equivalent, we use Proposition~\ref{pro:triple-letter-regularity} to maintain their regularity status.
    %Finally, Theorem~\ref{thm:dvas-regularity} tells us that DVAS regularity is decidable, so we can conclude that DVASS regularity is decidable.
\end{proof}