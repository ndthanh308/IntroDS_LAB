\section{Preliminaries}
\label{sec:preliminaries}
We denote by $\bbN$ the non-negative integers $\{0,1,\ldots\}$. We denote vectors in bold, e.g., $\vec{e}\in \bbZ^k$, and $\vec{e}[i]$ is the $i$-th coordinate. We write $[k]=\{1,\ldots,k\}$ for $k\ge 1$. We denote by $\Sigma^*$ the set of words over alphabet $\Sigma$, and by $|w|$ the length of a word $w\in \Sigma^*$.

A \emph{Counter Net of dimension $k$} ($k$-CN) $\cA$ is a quintuple $\cA=\tup{\Sigma, Q, Q_0, \delta, F}$ where  $\Sigma$ is a finite alphabet, $Q$ is a finite set of states, $Q_0$ is the set of initial states, $\delta \subseteq Q \times \Sigma \times \bbZ^k \times Q$ is a set of transitions, and $F \subseteq Q$ are the accepting states. 
A $k$-CN is \emph{deterministic}, denoted $k$-DCN, if $|Q_0| = 1$, and for every $p \in Q$ and $\sigma \in \Sigma$ there is at most one transition of the form $(p, \sigma, \vec{v}, q) \in \delta$. 
For a transition $(p, \sigma, \vec{v}, q)\in\delta$, we refer to $\vec{e}\in \bbZ^k$ as the \emph{effect} of the transition. 
% A \emph{Non-deterministic Finite Automaton} (NFA) is just a $0$-CN and a \emph{Deterministic Finite Automaton} (DFA) is just a $0$-DCN.
% We may refer to the finite automaton \emph{underlying} a $k$-CN, that is just the $0$-CN with the same alphabet, same set of states, same initial state, and same final states, the only difference is that there are no $k$-dimensional integer vector labels on the transitions.

An $\bbN$-configuration (resp. $\bbZ$-configuration) of a $k$-CN $\cA$ is a pair $(q,\vec{v})\in Q\times \bbN^k$ (resp. $(q,\vec{v})\in Q\times \bbZ^k$) representing the current state and values of the counters.
A transition $(p,\sigma,\vec{e},p')\in \delta$ is \emph{legal from $\bbN$-configuration $(q,\vec{v})$} if $\vec{v}+\vec{e}\in \bbN^k$, i.e., if all $k$ counters remain non-negative after the transition. A \emph{$\bbZ$-run} $\rho$ of $\cA$ on $w$ to be a sequence of $\bbZ$-configurations $\rho=(s_0,\vec{v_0}),(s_1,\vec{v_1}),\ldots (s_n,\vec{v_n})$ such that $(s_i,\sigma_i, v_{i+1}-v_i,s_{i+1})\in \delta$ for every $0 \leq i \leq n-1$.
An \emph{$\bbN$-run} is a $\bbZ$-run that visits only $\bbN$-configuration.
%$\rho$ of $\cA$ on a word $w=\sigma_1\cdots \sigma_n \in \Sigma^*$ is a sequence of $\bbN$-configurations $\rho=(s_0,\vec{v_0}),(s_1,\vec{v_1}),\ldots (s_n,\vec{v_n})$ such that $s_0\in Q_0$, $\vec{v_0}=\vec{0}$ and $(s_i,\sigma_i, \vec{v_{i+1}}-\vec{v_i},s_{i+1})\in \delta$ for every $0 \leq i \leq n-1$. 
Note that 
%since $\bbN$-configurations have non-negative vectors, 
all the transitions in an $\bbN$-run are legal. We omit $\bbN$ or $\bbZ$ from the run when it does not matter.
For a run $\rho=(s_0,\vec{v_0}),(s_1,\vec{v_1}),\ldots (s_n,\vec{v_n})$ of $\cA$, we denote $(s_0,\vec{v_0})\stackrel{\rho}{\to}(s_n,\vec{v_n})$. In addition we define the \emph{effect} of $\rho$ to be $\effect{\rho}=\vec{v_n}-\vec{v_0}$. 

%Note that every $\bbZ$-run is also an $\bbN$-run, but the converse is not true. %%%THE OPPOSITE!
An $\bbN$-run $\rho$ is \emph{accepting} if $s_0\in Q_0$, $\vec{v_0}=\vec{0}$ and $s_n\in F$. We say that $\cA$ \emph{accepts} $w$ if there is an accepting $\bbN$-run of $\cA$ on $w$. The \emph{language} of $\cA$ is then $\lang(\cA)=\{w\in \Sigma^* \mid \cA \text{ accepts } w\}$.   

An infix $\pi=(s_k,\vec{v_k}),(s_{k+1},\vec{v_{k+1}}),\ldots (s_{k+n},\vec{v_{k+n}})$ of a run $\rho$ is a \emph{cycle} if $s_k=s_{k+n}$ and is a \emph{simple cycle} if it is a cycle, and in addition it does not contain a cycle as a proper infix.  
%For a cycle $\pi=(s,\vec{v}),\ldots (s',\vec{v'})$ of a $1$-CN, 
We write that $\pi$ is $\positive$ (resp. $\nonnegative$ or $\negative$) if $\effect{\pi} > 0$ (resp. $\effect{\pi} \geq 0$ or $\effect{\pi} < 0$). 
%Likewise we say that $\pi$ is $\negative$ if $\effect{\pi} < 0$ and $\nonnegative$ if $\effect{\pi} \geq 0$. 

%We lift this colour scheme above to words and runs as follows. For a word $w=uv$ and either an $\bbN$-run or a $\bbZ$-run $\rho$, we write e.g., $\posCol{u}\nonnegCol{v}$ to denote that $\rho$ traverses a $\positive$ cycle when reading $u$, then a $\nonnegative$ cycle when reading $v$. Note that this does not preclude other cycles, e.g., there could also be negative cycles in the $u$ part, etc. That is, the coloring is not unique, but represents elements of the run.



%To simplify our analysis - we use these three colors to demonstrate properties of a run $\rho$ on a word $w$ by painting infixes of $w$. For example, if $w=uvx$, then we discuss a run $\rho$ on $\green{u}\violet{v}\red{x}$ to 

% We define $\rho$'s \emph{state projection} to be $\rho^Q=s_1,\ldots s_n$. 

% \hstodo{\henry{just dumped text from later section}
% We write vectors using boldface script, given a vector $\vec{v} \in \Z^n$, we use $\vec{v}[i] \in \Z$ to index the $i$-th component. 
% Additionally, we may write $\vec{v}[i..j]$ to denote the vector $(\vec{v}[i], \ldots, \vec{v}[j])$.
% In a similar fashion, given a deterministic or non-deterministic $d$-VASS $\Aa = (Q, \Sigma, T, q_s, F)$, we may write $\Aa[i]$ to denote the 1-VASS $(Q, \Sigma, T', q_s, F)$ where $T' = \set{ (p, \sigma, \vec{v}[i], q) : (p, \sigma, \vec{v}, q) \in T }$.
% We may also denote $\Aa[i..j]$ to denote the $(j-i+1)$-VASS $(Q, \Sigma, T', q_s, F)$ where $T' = \set{ (p, \sigma, \vec{v}[i..j], q) : (p, \sigma, \vec{v}, q) \in T }$.
% More generally, $\Aa[i]$ and $\Aa[i..j]$ may be referred to as the \emph{projections} of $\Aa$.
% }