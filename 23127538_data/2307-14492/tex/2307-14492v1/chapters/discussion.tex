\section{Discussion}
\label{sec:discussion}
Broadly, this work explores the expressive power given by the dimension of a $k$-CN. This is done by studying the \emph{dimension-minimality} problem, where we simply ask whether the dimension can be decreased, and by the more involved \emph{primality} problem, which also asks whether we can decrease the dimension, but allows a decomposition to multiple CNs. 
As we show, in general both primality and dimension-minimality are undecidable. Moreover, they remain undecidable even when we discard the degenerate dimension 0 case, which corresponds to finite memory, i.e., regular languages. On the other hand, this degenerate case is one where we can at least show decidability for DCNs.

We conclude with two open problems for future research.
\begin{itemize}
    \item \textbf{Example of a prime $k$-CN:} We demonstrate a prime $2$-CN. A natural question is whether we can find (and prove correctness of) an example of a prime $k$-CN for every $k\in \bbN$? And more generally, for every $d<k\in \bbN$ a $k$-CN that is $(d,\cdot)$-composite but not $(d-1,\cdot)$ composite?

    The difficulty involved in proving~\cref{thm:2CN_prime} seems to indicate that this might be highly nontrivial.
    \item \textbf{Decidability of dimension-minimality for $k$-DCN:} In~\cref{sec:decidable_DCN} we show that regularity is decidable for DCN, but this relies heavily on the finite-memory property of regular languages. Extending this to deciding whether a $k$-DCN has an equivalent $d$-DCN for $d<k$ seems technically challenging, even for $k=2$ and $d=1$.
\end{itemize}

