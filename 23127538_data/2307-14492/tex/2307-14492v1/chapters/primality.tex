\section{Primality and Compositeness}
\label{sec:primality}
We begin by presenting our main definition, followed by some basic properties.
\begin{definition}[Compositeness, Primality, and Dimension-Minimality]
\label{def:primality}
Consider a $k$-CN $\cA$, and let $d,n\in \bbN$. We say that $\cA$ is \emph{$(d,n)$-composite} if there exist $d$-CNs $\cB_1,\ldots,\cB_n$ such that $\lang(\cA)=\bigcap_{i=1}^n\lang(\cB_i)$. 
If $\cA$ is $(d,n)$-composite for some $d<k$ and $n\in \bbN$, we say $\cA$ is \emph{composite}. Otherwise,  $\cA$ is \emph{prime}.
If $\cA$ is not $(k-1,1)$-composite, we say that $\cA$ is \emph{dimension-minimal}.
\end{definition}
%Note that $(d,n)$-compositeness implies $(d',n')$-compositeness for every $d'\ge d$ and $n'\ge n$ (by taking redundant copies and redundant counters). 
\begin{remark}
    \label{rmk:compositeness_generalizes_reg}
    Note that the special case where $\cA$ is 
    %As mentioned in \cref{sec:introduction}, a $k$-CN $\cA$ is 
    $(0,n)$-composite (equivalently, $(0,1)$-composite) coincides with $\lang(\cA)$ being regular.
    %if and only if $L(\cA)$ is regular. 
    %Thus, compositeness generalizes regularity as well as dimension-minimality.
\end{remark}


Observe that in \cref{fig:composite_2CN_intro} we in fact show a composite $2$-DCN. We now show that every $k$-DCN is $(1,k)$-composite, by projecting to each of the counters separately. In particular, a $k$-DCN is prime only when $k=1$ and its language is not regular.
Formally, consider a $k$-DCN $\cD=\tup{\Sigma,Q,Q_0,\delta,F}$ and let $1\le i\le k$. We define the \emph{$i$-projection} to be the 1-DCN $\cD|_i=\tup{\Sigma,Q,Q_0,\delta|_i,F}$ where $\delta|_i=\{(q,\sigma,\vec{e}[i],q')\mid (q,\sigma,\vec{e}[i],q')\in \delta\}$. 

\begin{proposition}
\label{prop:DCN_Projection}
Every $k$-DCN $\cD$ is $(1,k)$-composite. Moreover, $\lang(\cD)=\bigcap_{i=1}^k\lang(\cA|_i)$.
\end{proposition}
\begin{proof}
Let $w\in \lang(\cD)$ and let $\rho$ be the accepting run of $\cD$ on $w$, then the projection of $\rho$ on counter $i$ induces an accepting run of $\cA|_i$ on $w$, thus $w\in \bigcap_{i=1}^k\lang(\cA|_i)$. Note that this direction does not use the determinism of $\cD$.

Conversely, let $w\in \bigcap_{i=1}^k\lang(\cA|_i)$, then each $\cA|_i$ has an accepting run $\rho_i$ on $w$. Since the structure of all the $\cA|_i$ is identical to that of $\cD$, all the runs $\rho_i$ have identical state sequences, and therefore are also a $\bbZ$-run of $\cD$ on $w$. Moreover, due to this being a single $\bbN$-run in each $\cA|_i$, it follows that all counter values remain non-negative in the corresponding run of $\cD$ on $w$. Hence, this is an accepting $\bbN$-run of $\cD$ on $w$, so $w\in \lang(\cD)$.
\end{proof}
\begin{remark}[Unambiguous CN are Composite]
\label{rmk:unambiguous}
The proof of \cref{prop:DCN_Projection} applies also to \emph{unambiguous} CN. Thus, the same result holds for them.
\end{remark}

Consider $k$-CNs $\cB_1,\ldots,\cB_n$. By taking their product, we can construct a $kn$-CN $\cA$ such that $\lang(\cA)=\bigcap_{i=1}^n \lang(\cB_i)$. In particular, if the $\cB_i$ are $1$-DCNs, then $\cA$ is an $n$-DCN. Combining this with \cref{prop:DCN_Projection} we have the following.
\begin{proposition}
\label{prop:DCN_dimension_minimal_iff_composite}
    Consider a $k$-DCN $\cD$, then $\cD$ is dimension-minimal if and only if it is not $(1,k-1)$-composite.
\end{proposition}
\begin{proof}
    If $\cD$ is $(1,k-1)$ composite, then there exist $1$-DCNs $\cB_1,\ldots,\cB_{k-1}$ such that $\lang(\cD)=\bigcap_{i=1}^{k-1} \lang(\cB_i)$. Define $\cB$ to be the product $k-1$-CN of $\cB_1,\ldots,\cB_{k-1}$, then $\lang(\cB)=\lang(\cD)$, so $\cD$ is not dimension-minimal.

    Conversely, if $\cD$ is not dimension-minimal, there exists w.l.o.g., a $k-1$-CN $\cB$ such that $\lang(\cB)=\lang(\cD)$. Then, we have $\lang(\cB)=\bigcap_{i=1}^{k-1} \lang(\cB|_i)$, so $\cD$ is $(1,k-1)$ composite.
\end{proof}

