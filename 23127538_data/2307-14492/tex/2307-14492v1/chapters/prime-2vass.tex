\section{A Prime 2-CN}
\label{sec:prime-2vass}
In this section we present our main technical contribution, namely an example of a prime $2$-CN. The technical difficulty arises from the need to prove that this example cannot be decomposed as an \emph{intersection} of \emph{nondeterministic} $1$-CNs. Since intersection has a universal flavour, and nondeterminism has an existential flavour, we have a sort of ``quantifier alternation'' which often introduces difficulties.

The importance of this example is threefold: first, it enables us to show in that primality is undecidable in \cref{sec:primality_undecidable}. Second, it sheds intuitive light on what makes a language prime. Lastly, we suspect the techniques developed here may be used in other settings for reasoning about counters in nondeterministic settings.

We start by presenting the prime $2$-CN, followed by an intuitive overview of the proof, before delving into the details.
\begin{example}
\label{xmp:prime2CN}
Consider the $2$-CN $\cP$ over alphabet $\Sigma=\{a,b,c,\#\}$ depicted in \cref{fig:prime-2vass}.
% Figure environment removed
Intuitively, $\cP$ starts by reading segments of the form $a^m\#$, where in each segment it nondeterministically chooses whether to increase the first or second counter by $m$. Then, it reads $b^{m_b}c^{m_c}$ and accepts if the value of the first and second counter were bigger than $m_b$ and $m_c$, respectively. Thus, $\cP$ accepts a word if its $a^m\#$ segments can be partitioned into two sets $I$ and $\overline{I}$ so that the sum of lengths of the segments in $I$ (resp. $\overline{I}$) is larger than the $b$ segment (resp. $c$ segment). 

For example, $a^{10}\#a^{20}\#a^{15}\#b^{15}c^{30}\in \lang(\cP)$, since segments 1 and 2 have length $30$, matching $c^{30}$ and segment $3$ matches $b^{15}$. However, $a^{10}\#a^{20}\#a^{15}\#b^{21}c^{21}\notin \lang(\cP)$, since in any partition of $\{10,20,15\}$, one set will have sum lower than $21$.

More precisely, we have the following:
\[
\lang(\cP)=\{a^{m_1}\#a^{m_2}\#\cdots\#a^{m_{t}}\#b^{m_b}c^{m_c} \mid \exists I\subseteq [t] \text{ s.t. }\sum_{i\in I}m_i\ge m_b \wedge \sum_{i\notin I}m_i\ge m_c\}
\]
\end{example}

\begin{theorem}
\label{thm:2CN_prime}
$\cP$ is prime. 
\end{theorem}
The high-level intuition behind \cref{thm:2CN_prime} is that any $1$-CN can either guess a subset of segments that covers $m_b$ or $m_c$, but not both, and in order to make sure the choices between two $1$-CNs form a partition, we need to fix the partition in advance. This is only possible if the number of segments is a-priori fixed, which is not (c.f., \cref{rmk:unbounded_compositeness}). This intuition, however, is far from a proof.

\subsection{Proof of \cref{thm:2CN_prime} -- Overview}
Assume by way of contradiction that $\cP$ is not a prime $2$-CN. Thus, there exist 1-CNs $\cV_1,\ldots \cV_k$ such that $\lang(\cP)=\bigcap_{1 \leq j \leq k}\lang(\cV_j)$.

Throughout the proof, we focus on words of the form $a^{m_1}\#a^{m_2}\#\cdots\#a^{m_{k+1}}\#b^{m_b}c^{m_c}$ for integers $\left\{m_i\right\}_{i=1}^{k+1},m_b,m_c \in \bbN$. 
We index the $a^{m_i}$ segments of these words, i.e., $a^{m_1}$ is the first segment, $a^{m_i}$ is segment $i$, etc. Note that we restrict our discussion to words with $k+1$ segments of $a$'s, where $k$ is the number of $\cV_j$ factors in the intersection. It is useful to think about each segment as ``paying'' for either $b$ or $c$. Then, a word is accepted if there is a way to choose for each segment whether it pays for $b$ or $c$, such that there is sufficient budget for both.

Let $i\in [k+1]$ and $j\in [k]$. We say that segment $i$ is \emph{bad} in $\cV_j$ if, intuitively, we can pump the length $m_i$ of segment $i$ simultaneously with both $m_b$ and $m_c$ to arbitrarily large lengths, such that the resulting words are accepted by $\cV_j$ (see \cref{def:good_bad} for the formal definition).
For example, consider the word $a^{10}\#a^{10}\#a^{10}\#b^{20}c^{10}\in \lang(\cP)$. If segment $2$ is bad for e.g., $\cV_1$, then there exist $x,y,z>0$ such that for every $t,t_b,t_c\in \bbN$ it holds that 
$
a^{10}\#a^{10+tx}\#a^{10}\#b^{20+t_b y}c^{10+t_c z}\in \lang(\cV_1).
$
Observe that such a segment is indeed ``bad'', since for large enough $t,t_b,t_c$, the resulting word is not in $\lang(\cP)$. Note, however, that the existence of a bad segment is not a contradiction, since the resulting pumped words might not be accepted by some other $\cV_{j'}$. 
%Indeed, because it can only pay for either $b$ or $c$, but both grow arbitrarily large while the remaining budget (without $m_i$) stays fixed.

In order to reach a contradiction, we need to show the existence of a segment $i$ that is bad for \emph{every} $\cV_j$. Moreover, we must also show that pumping the $a$'s in this segment, as well as $b$'s and $c$'s can be simultaneously done in all the $\cV_j$ together (i.e., $x,y,z$ are the same for all $j$). Then, pumping as above would reach a contradiction, since all the $\cV_j$ accept a word that is not in $\lang(\cP)$.

Our goal is therefore to establish a precise and usable definition of ``bad'' segments, and then find a word $w$ with $k+1$ segments where one of the segments is bad for all the $\cV_j$, and such that pumping can be done synchronously. 

\subsection{Pumping Arguments in 1-CN}
\label{sec:1vass-properties}
\shtodo{Maybe add a duplicate for colorblind/monochrome in the appendix, and a corresponding footnote here.}
In this section we establish some pumping results for 1-CN for the proof of \cref{thm:2CN_prime}. Throughout this section, we consider a 1-CN $\cV=\tup{\Sigma,Q,Q_0,\delta,F}$.

Our first lemma states the intuitive fact that without $\positive$ cycles, the counter value of a run is bounded.
\begin{lemma}\label{lem:maxcounternopositivecycles}
    Let $(q,n)$ be a configuration of $\cV$, let $W$ be the maximal positive weight in $\cV$, $\sigma \in \Sigma$, and $N\in\bbN$. If an 
    $\bbN$-run $\rho$ of $\cV$ on $\sigma^N$ from configuration $(q,n)$ does not traverse any $\positive$ cycle, then the maximal possible counter value anywhere along $\rho$ is $n+W|Q|$. 
\end{lemma}
\begin{proof}
    We prove the contra-positive: assume $\rho$ has counter value $n+W|Q|$, w.l.o.g. we can assume this occurs at the end. Since the maximal counter increase along a transition is $W$, it follows that there are at least $|Q|$ indices where the counter increases beyond any previous level. Thus, we can find indices $i_0,i_1,\ldots,i_{|Q|}$ such that the counter increases in the infix of $\rho$ between $i_j$ and $i_{j+1}$. Since these are $|Q|+1$ states, then there exists a state that is visited twice, which forms a $\positive$ cycle in $\rho$.
\end{proof}

The next lemma shows that long-enough runs must have $\nonnegative$ cycles.
\begin{lemma}\label{lem:largeenoughwordnonnegativecycle}
    %Consider a 1-CN $\cV=\tup{\Sigma,Q,Q_0,\delta,F}$. 
    Let $\sigma\in \Sigma$ and $(q,n)$ be an $\bbN$-configuration of $\cV$. Then there exists $N\in \bbN$ such that for all $N' \geq N$, every $\bbN$-run of $\cV$ on $\sigma^{N'}$ from $(q,n)$ traverses a $\nonnegative$ cycle. 
\end{lemma}
\begin{proof}
Let $W$ be the maximal positive transition effect in $\cV$, we show that $N=|Q|(n+W|Q|)$ satisfies the requirements.

    Assume by way of contradiction that $\cV$ can read $\sigma^{N}$ via an $\bbN$-run $\rho=(q_0,n_0=n)\stackrel{\rho}{\to} (q_N,n_N)$ that only traverses $\negative$ cycles. 

Since $\rho$ visits $N+1$ states, then by the Pigeonhole Principle, there exists a state $p \in Q$ that is visited $m \geq (N+1)/\abs{Q} > N/\abs{Q}$ many times in $\rho$.
Consider all the indices $0 \leq i_1 < i_2 < \ldots < i_m \leq N$ such that $p = q_{i_1} = \ldots = q_{i_N}$.
Each run segment $(q_{i_1}, n_{i_1}) \rightarrow (q_{i_2}, n_{i_2}), \ldots, (q_{i_{m-1}}, n_{i_{m-1}}) \rightarrow (q_{i_m}, n_{i_m})$ is a cycle in $\rho$, and therefore must have negative effect.
Thus $n_{i_1} > n_{i_2} > \ldots > n_{i_m} \geq 0$, so in particular $n_{i_1} \geq n_{i_m} + m \geq 0$ (as each cycle has effect at most $-1$.).

Moreover, $n_{i_1} < n + \abs{Q}\cdot W$ since the prefix $(q_0, n) \rightarrow (q_{i_1}, n_{i_1})$ cannot contain a non-negative cycle.
However, since $m >N/\abs{Q} = n + \abs{Q}\cdot W$ and $n_{i_1} \geq n_{i_m} + m \geq m \geq n + \abs{Q}\cdot W$, we get $n+|Q|\cdot W<n+ |Q|W$ which is a contradiction.
\end{proof}

Next, we show that runs with $\nonnegative$ and $\positive$ cycles have ``pumpable'' infixes.
\begin{lemma}\label{lem:cyclethensimplecycle}
    %Consider a 1-CN $\cV=\tup{\Sigma,Q,Q_0,\delta,F}$. 
    Let $\sigma\in \Sigma$ and consider a $\positive$ (resp. $\nonnegative$) cycle $\pi=(q_0,c_0)\stackrel{\sigma}{\to} (q_1,c_1) \stackrel{\sigma}{\to} \ldots (q_n=q_0,c_n)$ on $\sigma^n$ that induces an $\bbN$-run. Then, there is a sequence of (not necessarily contiguous) indices $0 \leq i_1 \leq \ldots \leq i_k \leq n$ such that $q_{i_1}\stackrel{\sigma}{\to} q_{i_2}\stackrel{\sigma}{\to}\cdots q_{i_k}$ is a simple $\positive$ (resp. $\nonnegative$) cycle with some effect $e>0$ (resp. $e \geq 0$). 
    
    In addition, this simple cycle is ``pumpable'' from the first occurrence of $q_{i_1}$ in $\pi$; namely, for all $m \in \bbN$ there is a run $\pi_m$ obtained from $\pi$ by traversing the cycle $m$ times so that
    %that coincides with $\pi$ only that it traverses the cycle $m$ times, thus 
    $\effect{\pi_m}=\effect{\pi}+em$. 
    %Formally, $\pi_m=(q_0,c_0)\stackrel{\sigma}{\to} \ldots (q_{i_1},c_{i_1}) \stackrel{\sigma}{\to} \ldots (q_{i_1},c_{i_1}+km)\stackrel{\sigma}{\to} \ldots (q_n,c_n+km)$ is an $\bbN$-run of $\cV$ on $\sigma^{n+mk}$ from $(q_0,c_0)$.
    % \gatodo{was:
    % In addition, this simple cycle is ``pumpable'' from the first occurrence of $q_{i_1}$ in $\pi$.
    % That is, $\pi_m=(q_0,c_0)\stackrel{\sigma}{\to} \ldots (q_{i_1},c_{i_1}) \rightarrow^{\sigma}\ldots (q_{i_1},c_{i_1}+em)\stackrel{\sigma}{\to} \ldots (q_n,c_n+em)$ is an $\bbN$-run of $\cV$ on $\sigma^{n+mk}$ from $(q_0,c_0)$ for every $m\in \bbN$ with effect $\effect{\rho}+ek$.

    % Please check things that I think are typos: at the end  $\rho \rightarrow \pi$, in the indices of the path $e \rightarrow k$, effect of the pumped path is $em$ (was $ek$)}
    
\end{lemma}
\begin{proof}
    We prove the $\nonnegative$ case, the $\positive$ case is proved mutatis-mutandis. 
    
    We define $\pi_m=(q_0,c_0)\stackrel{\sigma}{\to} \ldots (q_{i_1},c_{i_1}) \stackrel{\sigma}{\to} \ldots (q_{i_1},c_{i_1}+km)\stackrel{\sigma}{\to} \ldots (q_n,c_n+km)$. The proof is now by induction on the length of $\pi$. 
    
    The base of the induction is a cyclic $\bbN$-run of length $2$. In this case $\pi=(q_0,c_0)\stackrel{\sigma}{\to} (q_1=q_0,c_1)$ is itself a $\nonnegative$ simple cycle that is infinitely pumpable from $(q_0,c_0)$.

    We now assume correctness for length $n$, and discuss $\pi=(q_0,c_0)\stackrel{\sigma}{\to} (q_1,c_1) \stackrel{\sigma}{\to} \ldots (q_n=q_0,c_n)$ of length $n+1$. Let $0\leq j_1<j_2\leq n$ be indices such that $q_{j_1}=q_{j_2}$, for a maximal $j_1$. Note that the cycle $\tau=(q_{j_1},c_{j_1})\stackrel{\sigma}{\to} \ldots (q_{j_2},c_{j_2})$ must be simple. If $j_1=0$ and $j_2=n$, then $\pi$ itself is a simple $\nonnegative$ cycle, and the pumping argument is straightforward. Otherwise $\tau$ is nested. We now treat two cases separately:
    \begin{enumerate}
        \item $\tau$ is $\nonnegative$. In this case the induction hypothesis applies on $\tau$. We take the guaranteed constants $j_1 \leq i_1 \leq \ldots \leq i_k \leq j_2$, which apply to $\pi$ as well.
        \item $\tau$ is $\negative$. In this case we remove $\tau$ from $\pi$ to obtain $\pi'=(q_0,c_0)\stackrel{\sigma}{\to} \ldots (q_{j_1},c_{j_1}) \stackrel{\sigma}{\to} (q_{j_2+1},c'_{j_2+1}) \stackrel{\sigma}{\to} \ldots (q_n,c'_n)$, such that $c'_i \geq c_i$ for all $j_2+1 \leq i \leq n$. The induction hypothesis applies on $\pi'$, so let $i_1, \ldots, i_k$ be the guaranteed constants. Note that $i_1 \leq j_1$, since the cycle removed when obtaining $\pi'$ from $\pi$ is the last occurrence of a repetition of states in $\pi$. We therefore know that $q_{i_1}\stackrel{\sigma}{\to} q_{i_2}\stackrel{\sigma}{\to}\cdots q_{i_k}$ is a simple $\nonnegative$ cycle in $\pi'$ -- which applies to $\pi$ as well. In addition, it is infinitely pumpable from $\bbN$-configuration $(q_{i_1},c_{i_1})$ in $\pi'$ for $i_1 \leq j_1$. Indeed, since $\pi$ and $\pi'$ coincide up to and including $(q_{j_1},c_{j_1})$ between $\pi$ and $\pi'$ - this cycle is infinitely pumpable in $\pi$ as well.
    \end{enumerate}
\end{proof}

%We conclude the discussion about this lemma with several observations. Note that 

The simple cycle in \cref{lem:cyclethensimplecycle} has length $k<|Q|$. By pumping it $\frac{|Q|!}{k}$ times we obtain a pumpable cycle of length $|Q|!$. We thus conclude with the following. 
%times for all $n\in \bbN$, to obtain a run $\rho'$ on $\sigma^k+n|Q|!$ from a run $\rho$ on $\sigma^k$.  

\begin{corollary} \label{obs:nonnegativepumptofactorial}
    %Consider a $1$-CN $\cV=\tup{\Sigma,Q,Q_0,\delta,F}$ and 
    Let $\rho$ be an $\bbN$-run of $\cV$ on $\sigma^n$ that traverses a $\nonnegative$ cycle. For every $m\in \bbN$, we can construct an $\bbN$-run $\rho'$ of $\cV$ on $\sigma^{n+m|Q|!}$ such that $\effect{\rho'} \geq \effect{\rho}$ by pumping a $\nonnegative$ simple cycle in $\rho$.
\end{corollary}



% \gatodo{was:
% The simple cycle in \cref{lem:cyclethensimplecycle} has length $l<|Q|$. By pumping it  $\frac{|Q|!}{l}$ times we obtain a pumpable cycle of length $|Q|!$. We thus have the following. 
% %times for all $n\in \bbN$, to obtain a run $\rho'$ on $\sigma^k+n|Q|!$ from a run $\rho$ on $\sigma^k$.  

% \begin{corollary} \label{obs:nonnegativepumptofactorial}
%     %Consider a $1$-CN $\cV=\tup{\Sigma,Q,Q_0,\delta,F}$ and 
%     Let $\rho$ be an $\bbN$-run of $\cV$ on $\sigma^n$ that traverses a $\nonnegative$ cycle. For every $k\in \bbN$, we can construct an $\bbN$-run $\rho'$ of $\cV$ on $\sigma^{n+k|Q|!}$ such that $\effect{\rho'} \geq \effect{\rho}$ by pumping a $\nonnegative$ simple cycle in $\rho$.
% \end{corollary}

% changed:  $l \rightarrow k$ ($k$ is the length of the cycle), $k \rightarrow m$ ($m$ is the number of times the cycle is pumped)
% }



 
\subsection{Bad Segments}
\label{sec:bad_segments}
We lift the colour\footnote{The colours were chosen to be accessible for the colorblind. For a greyscale-friendly version, see \cref{apx:grey}} scheme of $\positive$ and $\nonnegative$ to words and runs as follows. For a word $w=uv$ and a run $\rho$, we write e.g., $\posCol{u}\nonnegCol{v}$ to denote that $\rho$ traverses a $\positive$ cycle when reading $u$, then a $\nonnegative$ cycle when reading $v$. Note that this does not preclude other cycles, e.g., there could also be negative cycles in the $u$ part, etc. That is, the coloring is not unique, but represents elements of the run.

Recall our assumption that $\lang(\cP)=\bigcap_{1 \leq j \leq k}\lang(\cV_i)$, and denote $\cV_j=\tup{\Sigma,Q^j,I^j,\delta^j,F^j}$ for all $j\in [k]$. Let $Q_{\max}=\max\{|Q_j|\}_{j=1}^k$ and denote  $\alpha=Q_{\max}!$. 
Further recall that we focus on words of the form $a^{m_1}\#a^{m_2}\#\cdots\#a^{m_{k+1}}\#b^{m_b}c^{m_c}$ for integers $\left\{m_i\right\}_{i=1}^{k+1},m_b,m_c \in \bbN$, and that we refer to the infix $a^{m_i}$ as Segment $i$, for $1 \leq i \leq k$.
We proceed to formally define good and bad segments.
% \gatodo{
% I added the recall on the language and what a segment is. I forgot what they were after reading Sec 4.2.. 

% Was: 

% Recall our assumption that $\lang(\cP)=\bigcap_{1 \leq j \leq k}\lang(\cV_i)$, and denote $\cV_j=\tup{\Sigma,Q^j,I^j,\delta^j,F^j}$ for all $j\in [k]$. Let $Q_{\max}=\max\{|Q_j|\}_{j=1}^k$ and denote  $\alpha=Q_{\max}!$. 
% Further recall that we focus on words of the form $a^{m_1}\#a^{m_2}\#\cdots\#a^{m_{k+1}}\#b^{m_b}c^{m_c}$ for integers $\left\{m_i\right\}_{i=1}^{k+1},m_b,m_c \in \bbN$, and that refer to the segment $a^{m_i}$ as Segment $i$, for $1 \leq i \leq k$.
% We now formally define good and bad segments.
% }
\begin{definition}[Good and Bad Segments]
\label{def:good_bad}
Segment $i$ is \emph{bad} in $\cV_j$ if there exist constants $\left\{m_i\right\}_{i=1}^{k+1},m_b,m_c \in \bbN$ such that all the following hold.
\begin{itemize}
    \item $\left\{m_i\right\}_{i=1}^{k+1},m_b,m_c$ are multiples of $\alpha$. 
    \item There is an accepting $\bbN$-run $\rho$ of $\cV_j$ on $w=a^{m_1}\#a^{m_2}\#\cdots\#a^{m_{k+1}}\#b^{m_b}c^{m_c}$ that adheres to one of the following three \emph{forms}:
    \begin{enumerate}
        \item $\nonnegCol{a^{m_1}}\#\nonnegCol{a^{m_2}}\#\cdots \nonnegCol{a^{m_{i-1}}}\#\posCol{a^{m_i}}\#a^{m_{i+1}}\#\cdots\#a^{m_{k+1}}\#b^{m_b}c^{m_c}$
        \item $\nonnegCol{a^{m_1}}\#\nonnegCol{a^{m_2}}\#\cdots \nonnegCol{a^{m_{i-1}}}\#\nonnegCol{a^{m_i}}\#\nonnegCol{a^{m_{i+1}}}\#\cdots\#\nonnegCol{a^{m_{k+1}}}\#\nonnegCol{b^{m_b}}\nonnegCol{c^{m_c}}$
        \item $\nonnegCol{a^{m_1}}\#\nonnegCol{a^{m_2}}\#\cdots \nonnegCol{a^{m_{i-1}}}\#\nonnegCol{a^{m_i}}\#\nonnegCol{a^{m_{i+1}}}\#\cdots\#\nonnegCol{a^{m_{k+1}}}\#\posCol{b^{m_b}}c^{m_c}$.
    \end{enumerate}
\end{itemize}

Segment $i$ is \emph{good} in $\cV_j$ if it is not bad.
%\shcomm{do we need ``good''?}\aycomm{Yes, it's used later on - look at the start of 6.4. Obviously we can always write "not bad" instead of good, but that seems pretty clunky to me.}
\end{definition}

\cref{lem:bad_segements_are_bad} below formalises the intuition that a bad segment can be pumped simultaneously with both the $b$ and $c$ segments, eventually generating a word that is not in $\lang(\cP)$, but is accepted by $\cV_j$.

% \gatodo{Suggestion, add a sentence that starts with:

% The following lemma formalizes the intuition that existence of a bad cycle implies that... 

% I'm not sure I understand what it implies. 

% My guess: 
% $V_j$ accepts an infinite sequence of words in which Segment $i$ is pumped together with the $b$'s and $c$'s. Specifically, note that $V_j$ accepts a word not in $L(\P)$.
% }

Intuitively, Forms 2 and 3 mean that all segments are bad. Indeed, Segment $i$ has a $\nonnegative$ cycle, so it can be pumped safely, and in Form 2 both $b$ and $c$ can be pumped using $\nonnegative$ cycles, whereas in Form 3 we can pump $b$ using a $\positive$ cycle, and use it to compensate for pumping $c$, even if the latter requires pumping a negative cycle.

Form 1 is the interesting case, where we use a $\positive$ cycle in Segment $i$ to compensate for pumping both $b$ and $c$. The requirement that all segments up to $i$ are $\nonnegative$ is at the heart of our proof, and is explained in \cref{sec:proof_of_2Prime}.
Formally, we have the following.

\begin{lemma}
    \label{lem:bad_segements_are_bad}
    Let $l$ be a bad segment in $\cV_j$, then there exist $x,y,z\in \bbN$ multiples of $\alpha$ such that for every $n\in \bbN$ the following word $w$ is accepted by $\cV_j$.
    \[w_n=a^{m_1}\#a^{m_2}\#\cdots\#a^{m_l+xn}\#a^{m_{l+1}}\#\cdots\#a^{m_{k+1}}\#b^{m_b+yn}c^{m_c+zn}\]
\end{lemma}
\begin{proof}
    We can choose $z=\alpha$, then take $y$ to be large enough so that Form 3 runs can compensate for negative cycles in $c^z$ using $\positive$ cycles in $b^y$, while not decreasing the counters in Form $2$ runs (we can find such $y$ that is a multiple of $\alpha$, since $\alpha$ is divisible by all lengths of simple cycles). 
    Finally, we choose $x$ so that Form 1 runs can compensate for $c^z$ and $b^y$ using $\positive$ cycles in $a^x$ in Segment $l$, again while not decreasing the counters in Forms $2$ and $3$.
\end{proof}



% \gatodo{Suggestion -- add the context of the following lemma. In addition, maybe "upgrade" it to be a "proposition" instead of lemma to make it stand out? 

% Context suggestion:
% Recall that our goal is to show that there is an $l \in [k+1]$ such that Segment $l$ is bad in all $V_j$, for $j \in [k]$. The following lemma is a key component. It shows that each $V_j$ has at most one segment that is good, and the proof intuitively follows since there are $k+1$ segments and $k$ components. 
% }

Recall that our goal is to show that there is a segment $l \in [k+1]$ that is bad in \emph{every} $V_j$, for $j \in [k]$. \cref{lem:two_segments_one_bad} below shows that each $V_j$ has at most one good segment. Therefore, there are at most $k$ good segments in total, leaving at least one bad segment as desired.
%and the proof intuitively follows since there are $k+1$ segments and $k$ components. 
%Our first result is that all but one segment are bad.

\begin{lemma}
\label{lem:two_segments_one_bad}
    Let $j\in [k]$ and $i_1< i_2\in [k+1]$. Then either Segment $i_1$ or $i_2$ is bad in $\cV_j$.
    %\gatodo{Add?: Thus, there is at most one segment that is good for $V_j$.}
\end{lemma}
\begin{proof}
% \shtodo{Why is the proof with $n_i$ and not $m_i$? We should be consistent with the definition to make it easier on the reader.}
% \aytodo{My logic was that the "real" $m_i$'s are what you called in the current version of the proof $n'_i$s. I think it's best to keep the $n_i$s as is and change the $n'_i$s to $m_i$s, but if you don't like it we can do differently}
% \shtodo{No, this is ok.}
    Since $j$ is fixed, denote $\cV_j=\tup{\Sigma,Q,Q_0,\delta,F}$. We inductively define constants $\left\{n_i\right\}_{i=1}^{k+1},n_b,n_c \in \bbN$ as follows. $n_1$ is a large-enough multiple of $\alpha$ so that \cref{lem:largeenoughwordnonnegativecycle} guarantees a $\nonnegative$ cycle in any accepting run of $\cV_j$ on $a^{n_1}$ from some $(q_0,0)$ with $q_0\in Q_0$. 
    Assume we have defined $n_1,\ldots n_{l-1}$, and consider the word $u=a^{n_1}\#a^{n_2}\#\cdots\#a^{n_{l-1}}\#$. Define $n=|u|W$ where $W$ is the maximal value in any transition of $\cV_j$.
    Since $u$ consists of $\frac{n}{W}$ letters, $n+1$ is greater than any counter value that can be reached by $\cV_j$ by reading $u$. We define $n_l$ to be a multiple of $\alpha$ large enough so that \cref{lem:largeenoughwordnonnegativecycle} guarantees a $\nonnegative$ cycle when reading $a^{n_l}$ from any configuration of the form $\{(q,n') \mid q\in Q,\ n\le n+1\}$. We set $n_b=n_c=\alpha$ (the choice of $n_b,n_c$ is slightly arbitrary). Finally, we set $w=a^{n_1}\#\cdots\#a^{n_{k+1}}\#b^{n_b}c^{n_c}$.

    Now, for every $x\in \bbN$, we obtain from $w$ a word $w_x$ by pumping $x\alpha$ $a$'s to segments $i_1,i_2$ and to the $b$ and $c$ segments. That is, let $n'_i=n_i+x\alpha$ for $i\in \{i_1,i_2\}$ and $n'_i=n_i$ for $i\notin \{i_1,i_2\}$, and let $n'_b=n_b+x \alpha$ and $n'_c=n_c+x \alpha$, then $w_x=a^{n'_1}\#\cdots\#a^{n'_{k+1}}\#b^{n'_b}c^{n'_c}$.
    % \[w_x= a^{n_1}\#\cdots\#a^{n_{i_1}+x\alpha}\#\cdots\#a^{n_{i_2}+x\alpha}\#\cdots\#a^{n_{k+1}}\#b^{n_b+x\alpha}c^{n_c+x\alpha}\]
    Observe that $w_x\in \lang(\cP)$. Indeed, since $n_{i_1} \geq n_b=\alpha$ and $n_{i_2} \geq n_c=\alpha$ we have that $n_{i_1}+x\alpha \geq n_b+x\alpha$ and $n_{i_2}+x\alpha \geq n_c+x\alpha$, so segments $i_1$ and $i_2$ can already pay for the $b$'s and $c$'s, respectively.
    %and any run of $\cP$ that accumulates on a different counter for segments $i_1,i_2$ is guaranteed to be accepting. 
    In particular, $w_x\in \lang(\cV_j)$ with some accepting $\bbN$-run $\rho_x$. 

    We choose a particular value of $x$, as follows. Consider $x$ and suppose some accepting $\bbN$-run $\rho_x$ as above does not traverse a $\positive$ cycle neither in segment $i_1$ nor in $i_2$. By \cref{lem:maxcounternopositivecycles}, the maximal possible counter value of $\rho_x$ after reading \[a^{n_1}\# \cdots\# a^{n_{i_1}+ x\alpha}\#\cdots\#a^{n_{i_2}+x\alpha}\#\cdots\#a^{n_{k+1}}\#\] is $M_b=(k+1+\sum_{z\in [k+1]\setminus\{i_1,i_2\}}n_z)W+2|Q|W$. Crucially, this value does not depend on $x$. Further, if there is no $\positive$ cycle in the segment of $b$'s as well, again the maximal counter value of $\rho$ up to the $c$ segment is bounded by $M_c=(k+2+\sum_{z\in [k+1]\setminus\{i_1,i_2\}}n_z)W+3|Q|W$, independent of $x$ and $M_b$. 
    
    By \cref{lem:largeenoughwordnonnegativecycle}, we can now choose $x$ large enough to satisfy that for every accepting $\bbN$-run $\rho_x$ on $w_x$:
    \begin{enumerate}
        \item If $\rho_x$ does not traverse any $\positive$ cycle in segments $i_1,i_2$, then $\rho_x$ has a $\nonnegative$ cycle reading $b^{(n_b+x\alpha)}$ from any configuration of the form $\{(q,M') \mid q\in Q,\ M'\le M_b\}$.
        \item If $\rho_x$ does not traverse any $\positive$ cycle in segment $i_1$ nor $i_2$, nor in the $b$ segment, $\rho_x$ has a $\nonnegative$ cycle reading $c^{(n_c+x\alpha)}$ from any configuration of the form $\{(q,M') | q\in Q,\ M'\le M_c\}$. 
    \end{enumerate}

    Having fixed $x$, we claim that one of $i_1,i_2$ is bad for the constants in $w_x$.     

    By construction, \cref{lem:largeenoughwordnonnegativecycle} guarantees that $\rho_x$ has a $\nonnegative$ cycles in segments $1,\ldots i_1-1$. If $\rho_x$ has a $\positive$ cycle in segment $i_1$, then $\rho_x$ is of Form 1: 
    \[\nonnegCol{a^{n_1}}\#\nonnegCol{a^{n_2}}\#\cdots\nonnegCol{a^{n_{i-1}}}\#\posCol{a^{n_{i_1}+x\alpha}}\#\cdots \#a^{n_{i_2}+x\alpha}\#\cdots\#a^{n_{k+1}}\#b^{n_b+x\alpha}c^{n_c+x\alpha}\]
    so $i_1$ is bad in $\cV_j$.
  
    If $\rho_x$ does not have a $\positive$ cycles in segment $i_1$, then again by construction \cref{lem:largeenoughwordnonnegativecycle} guarantees $\nonnegative$ cycles in segments $i_1,i_1+1,\ldots, i_2-1$. Indeed, for $i_1$ -- we are guaranteed a $\nonnegative$ cycle reading $a^{n_{i_1}}$, all the more so for $a^{n_{i_1}+x\alpha}$. 
    As for $i_1+1,\ldots i_2-1$ -- if $\rho_x$ does not have a $\positive$ cycle in segment $i_1$, then the maximal effect of segment $i_1$ is $W|Q|$. However, $n_{i_1+1}$ was constructed to guarantee a $\nonnegative$ cycle even in case the effect of segment $i_1$ is $Wn_{i_1} \geq W\alpha \geq W|Q|$. 
    
    If there is a $\positive$ cycle in segment $i_2$ - then $\rho_x$ is again of Form 1: 
    \[\nonnegCol{a^{n_1}}\#\nonnegCol{a^{n_2}}\#\cdots \nonnegCol{a^{n_{i_2-1}}}\#\posCol{a^{n_{i_2}+x\alpha}}\#a^{n_{i_2+1}}\#\cdots\#a^{n_{k+1}}\#b^{n_b+x\alpha}c^{m_c+x\alpha}\]
    and $i_2$ is bad in $\cV_j$. 
    
    Otherwise, using the same arguments as for segment $i_1$, we have that segments $i_2+1,\ldots,i_{k+1}$ contain $\nonnegative$ cycles. 
    In this case we are left with the $b$ and $c$ segments. The choice of $x$ guarantees a $\nonnegative$ cycle in the segment of $b$'s. If $\rho_x$ traverses a $\positive$ cycle in the $b$ segment, then $w_x$ is of Form 3: 
    \[\nonnegCol{a^{n_1}}\#\nonnegCol{a^{n_2}}\#\cdots \nonnegCol{a^{n_{k+1}}}\#\posCol{b^{n_b+x\alpha}}c^{n_c+x\alpha}\] 
    Finally, if there are no $\positive$ cycles in the $b$ segment, then the choice of $x$ again guarantees a $\nonnegative$ cycle in the $c$ segment, so $w_x$ is of Form 2: \[\nonnegCol{a^{n_1}}\#\nonnegCol{a^{n_2}}\#\cdots\nonnegCol{a^{n_{k+1}}}\#\nonnegCol{b^{n_b+x\alpha}}\nonnegCol{c^{n_c+x\alpha}}\]
    In the two latter cases both $i_1$ and $i_2$ are bad in $\cV_j$.
\end{proof}
\subsection{Proof of \cref{thm:2CN_prime}}
\label{sec:proof_of_2Prime}
We now have that each $\cV_j$ has at most one good segment (otherwise we would have two good segments $i_1$ and $i_2$, contradicting \cref{lem:two_segments_one_bad}). 
Therefore, all $1$-CNs $\cV_1,\ldots,\cV_k$ have at most $k$ good segments combined. Recall that our words have $k+1$ segments, and therefore there is at least one segment $l$ that is bad in every $\cV_j$. 
Note, however, that this segment may correspond to different constants in each $\cV_j$. That is, there exists constants $\{m_i^j,m_b^j,m_c^j\mid i\in [k+1],j\in [k]\}$ witnessing that segment $l$ is bad for each $\cV_j$. 
We group the $\cV_j$ according to the Form of their accepting runs $\rho_j$, as follows.
%$\{m_i^j\}_{i=1,j=1}^{i=k+1,j=k},\{m_b^j\}_{j=1}^k,\{m_c^j\}_{j=1}^k$ 
%such that for all $j\in [k]$ we have $a^{m^j_1}\#a^{m^j_2}\#\cdots\#a^{m^j_{k+1}}\#b^{m^j_b}c^{m^j_c} \in \lang(\cV_j)$ through a run $\rho_j$. Due to the conditions on the constants above, we are guaranteed that $\rho_j$ is in one of the following three forms:
%In particular, for every $j\in[k]$ the run $\rho_j$ in one of the Forms of \cref{def:good_bad}.
\begin{itemize} 
    \item Form 1 are $\nonnegCol{a^{m^j_1}}\#\nonnegCol{a^{m^j_2}}\#\cdots\#\posCol{a^{m^j_l}}\#a^{m^j_{l+1}}\#\cdots\#a^{m^j_{k+1}}\#b^{m^j_b}c^{m^j_c}$. 
    \item Form 2 are $\nonnegCol{a^{m^j_1}}\#\nonnegCol{a^{m^j_2}}\#\cdots\#\nonnegCol{a^{m^j_l}}\#\nonnegCol{a^{m^j_{l+1}}}\#\cdots\#\nonnegCol{a^{m^j_{k+1}}}\#\nonnegCol{b^{m^j_b}}\nonnegCol{c^{m^j_c}}$. 
    \item Form 3 are $\nonnegCol{a^{m^j_1}}\#\nonnegCol{a^{m^j_2}}\#\cdots\#\nonnegCol{a^{m^j_l}}\#\nonnegCol{a^{m^j_{l+1}}}\#\cdots\#\nonnegCol{a^{m^j_{k+1}}}\#\posCol{b^{m^j_b}}c^{m^j_c}$. 
\end{itemize}
We now find constants such that the resulting (single) word has $l$ as a bad segment in all $\cV_j$.
First, for $i\in [k+1]\setminus\{l\}$, define $m^{\max}_i=\max\{m^j_i\}_{1\leq j \leq k}$ (note that these are still multiples of $\alpha$). Similarly, define $m^{\max}_c=\max\{m^j_c\}_{1\leq j \leq k}$. It remains to fix constants $\newl$ and $\newb$, which we do in phases in the following. The resulting word is then
%We will now prove that each $\{\cV_j\}_{1\leq j\leq k}$ accepts the following word (for constants that will be explained shortly): \hfill 
\[
w=a^{m^{\max}_1}\#\cdots\#a^{{\newl}}\#a^{m^{\max}_{l+1}}\#\cdots\#a^{m^{\max}_{k+1}}\#b^{{\newb}}c^{m^{\max}_c}
\]
%We first define $m^{\max}_i=\mathsf{max}\{m^j_i\}_{1\leq j \leq k}$ for all $1\leq i \leq k+1$. 
Most steps in the flow of the analysis below are based on \cref{lem:cyclethensimplecycle,obs:nonnegativepumptofactorial}. 
We first (partially) handle Form 3. For such $\cV_j$, there is an accepting $\bbN$-run $\rho_j$ on \[\nonnegCol{a^{m^j_1}}\#\cdots\#\nonnegCol{a^{m^j_l}}\#\nonnegCol{a^{m^j_{l+1}}}\#\cdots\#\nonnegCol{a^{m^j_{k+1}}}\#\posCol{b^{m^j_b}}c^{m^j_c}\] 
By pumping $\nonnegative$ cycles as per \cref{obs:nonnegativepumptofactorial} in all segments by $l$ we obtain an accepting $\bbN$-run $\rho'_j$ on \[\nonnegCol{a^{m^{\max}_1}}\#\cdots\#\nonnegCol{a^{m^j_l}}\#\nonnegCol{a^{m^{\max}_{l+1}}}\#\cdots\#\nonnegCol{a^{m^{{\max}}_{k+1}}}\#\posCol{b^{m^j_b}}c^{m^j_c}\]
We now pump arbitrary cycles in the $c$ segment to construct a $\bbZ$-run $\rho''_j$ on: \[\nonnegCol{a^{m^{\max}_1}}\#\cdots\#\nonnegCol{a^{m^j_l}}\#\nonnegCol{a^{m^{\max}_{l+1}}}\#\cdots\#\nonnegCol{a^{m^{{\max}}_{k+1}}}\#\posCol{b^{m^j_b}}c^{m^{\max}_c}\] 
Next, we compensate for possible negative cycles in the $c$ segment by pumping a $\positive$ cycle in the $b$ segment. Thus, we construct an $\bbN$-run $\rho'''_j$ on \[\nonnegCol{a^{m^{\max}_1}}\#\cdots\#\nonnegCol{a^{m^j_l}}\#\nonnegCol{a^{m^{\max}_{l+1}}}\#\cdots\#\nonnegCol{a^{m^{{\max}}_{k+1}}}\#\posCol{b^{\newb}}c^{m^{\max}_c}\] where $\newb$ is chosen to be large enough such that $\rho'''_j$ is indeed an $\bbN$-run for all $1\leq j \leq k$.
Note that it remains to fix $\newl$.

We now turn to Form 1 with a similar process. 
We start with an accepting $\bbN$-run $\rho_j$ on \[\nonnegCol{a^{m^j_1}}\#\cdots\#\posCol{a^{m^j_l}}\#a^{m^j_{l+1}}\#\cdots\#a^{m^j_{k+1}}\#b^{m^j_b}c^{m^j_c}\]
Pump $\nonnegative$ cycles to obtain an accepting $\bbN$-run $\rho'_j$ on  \[\nonnegCol{a^{m^{\max}_1}}\#\cdots\#\posCol{a^{m^j_l}}\#a^{m^j_{l+1}}\#\cdots\#a^{m^j_{k+1}}\#b^{m^j_b}c^{m^j_c}\] 
obtain a $\bbZ$-run $\rho''_j$ by pumping arbitrary cycles in the remaining segments, including the $b$ segment:
\[\nonnegCol{a^{m^{\max}_1}}\#\cdots\#\posCol{a^{m^j_l}} \#a^{m^{\max}_{l+1}}\#\cdots\#a^{m^{\max}_{k+1}}\#b^{\newb}c^{m^{\max}_c}\] 
and compensate for negative cycles by taking $\newl$ large enough so that pumping $\positive$ cycles in segment $l$ yields an accepting $\bbN$-run $\rho'''_j$ on \[\nonnegCol{a^{m^{\max}_1}}\#\cdots\#\posCol{a^{\newl}}\#a^{m^{\max}_{l+1}}\#\cdots\#a^{m^{\max}_{k+1}}\#b^{\newb}c^{m^{\max}_c}\]
%, when $m^\text{spl}$ is chosen to be large enough such that every $\rho'''_j$ is indeed a legal run.

We now return to Form 3 and fix the $l$-th segment by pumping $\nonnegative$ cycles to construct an accepting $\bbN$-run on \[\nonnegCol{a^{m^{\max}_1}}\#\cdots\#\nonnegCol{a^{\newl}}\#\nonnegCol{a^{m^{\max}_{l+1}}}\#\cdots\#\nonnegCol{a^{m^{{\max}}_{k+1}}}\#\posCol{b^{\newb}}c^{m^{\max}_c}\]

We are left with Form 2, which are the easiest to handle. We simply pump $\nonnegative$ cycles in all segments to construct an accepting $\bbN$-run $\rho'_j$ on  \[\nonnegCol{a^{m^{\max}_1}}\#\cdots\#\nonnegCol{a^{\newl}}\#\nonnegCol{a^{m^{\max}_{l+1}}}\#\cdots\#\nonnegCol{a^{m^{\max}_{k+1}}}\#\nonnegCol{b^{\newb}}\nonnegCol{c^{m^{\max}_c}}\] 
%directly from $\rho_j$ on $\nonnegCol{a^{m^j_1}}\#\nonnegCol{a^{m^j_2}}\#\cdots\#\nonnegCol{a^{m^j_l}}\#\nonnegCol{a^{m^j_{l+1}}}\#\cdots\#\nonnegCol{a^{m^j_{k+1}}}\#\nonnegCol{b^{m^j_b}}\nonnegCol{c^{m^j_c}}$.

Note that the requirement for all segments leading up to $l$ to be $\nonnegative$ is crucial, otherwise we would not have been able to pump all the Forms simultaneously.

We now have that $w$ %$w=a^{m^{\max}_1}\#a^{m^{\max}_2}\#\cdots\#a^{m^\text{spl}}\#a^{m^{\max}_{l+1}}\#\cdots\#a^{m^{\max}_{k+1}}\#b^{m^\text{spb}}c^{m^{\max}_c}$ 
is accepted by every $\cV_j$, and segment $l$ is bad for all $\cV_j$ for it.
%We change the terminology to avoid clutter and refer to $w$ as  $w=a^{m_1}\#a^{m_2}\#\cdots\#a^{m_l}\#a^{m_{l+1}}\#\cdots\#a^{m_{k+1}}\#b^{m_b}c^{m_c}$. Not only does every $\cV_j$ accept $w$ - it does so through a run that either of the form $a^{m_1}\#a^{m_2}\#\cdots\#\posCol{a^{m_l}}\#a^{m_{l+1}}\#\cdots\#a^{m_{k+1}}\#b^{m_b}c^{m_c}$, or \hfill $w=a^{m_1}\#a^{m_2}\#\cdots\#\nonnegCol{a^{m_l}}\#a^{m_{l+1}}\#\cdots\#a^{m_{k+1}}\#\posCol{b^{m_b}}c^{m_c}$ or \hfill $w=a^{m_1}\#a^{m_2}\#\cdots\#\nonnegCol{a^{m_l}}\#a^{m_{l+1}}\#\cdots\#a^{m_{k+1}}\#\nonnegCol{b^{m_b}}\nonnegCol{c^{m_c}}$. 
By applying \cref{lem:bad_segements_are_bad} for each of the $\cV_j$ and taking global constants to be the products of the respective constants $x,y,z$ for each $\cV_j$, We can now find $X,Y,Z\in \bbN$ multiples of $\alpha$ such that for every $n\in \bbN$ the word %\[a^{m_1}\#a^{m_2}\#\cdots\#a^{m_l+xn}\#a^{m_{l+1}}\#\cdots\#a^{m_{k+1}}\#b^{m_b+yn}c^{m_c+zn}\] 
\[
w_n=a^{m^{\max}_1}\#\cdots\#a^{{\newl+Xn}}\#a^{m^{\max}_{l+1}}\#\cdots\#a^{m^{\max}_{k+1}}\#b^{{\newb+Yn}}c^{m^{\max}_c+Zn}
\]
is accepted by every $\cV_j$. 
We then choose $n$ large enough to satisfy $\sum_{i\in [k+1]\setminus\{l\}} m^{\max}_i < \min\{\newb+Yn,m^{\max}_c+Zn\}$, so that $w_n\notin \lang(\cP)$, since segment $l$ can only pay for $b$ or for $c$, and the remaining segments cannot pay for the remaining segment.

This contradicts the assumption that $\lang(\cP)=\bigcap_{j\in [k]}\lang(\cV_j)$, concluding the proof of \cref{thm:2CN_prime}.
\hfill\qed

\begin{remark}[Unbounded Compositeness]
\label{rmk:unbounded_compositeness}
The proof of~\cref{thm:2CN_prime} shows that if words with $k+1$ segments are allowed, then the language is not $(1,k)$ composite (and we use it to establish primality). 
By intersecting $\lang(\cP)$ with words that allow at most $k+1$ segments, we obtain a language that is not $(1,k)$ composite, but it is not hard to show that it is $(1,2^{k+1})$ composite. 
This shows that a $2$-CN can be composite, but require unboundedly many factors.
\end{remark}