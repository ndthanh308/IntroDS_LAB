\section{CN Primality is Undecidable}\label{sec:primality_undecidable}
% Recall that a $k$-CN $\cA$ is $(d,n)$-composite if there exist $d$-CNs $\cB_1,\ldots,\cB_n$ such that $\lang(\cA)=\bigcap_{i=1}^n\lang(\cB_i)$. If $\cA$ is not $(d,n)$-composite for any $d<k$ and $n\in \bbN$, we say $\cA$ is prime. 

% To refresh, given a $k$-CN $\cA$, we say that $\cA$ is \emph{$(d,n)$-composite} for some $d,n\in \bbN$ if there exist $d$-CNs $\cB_1,\ldots,\cB_n$ such that $\lang(\cA)=\bigcap_{i=1}^n\lang(\cB_i)$. If $\cA$ is not $(d,n)$-composite for any $d<k$ and $n\in \bbN$, we say $\cA$ is prime. 

In this section we consider the \emph{primality} and \emph{dimension minimality} decision problems: given a $k$-CN $\cA$, decide whether $\cA$ prime and whether $\cA$ is dimension-minimal, respectively.

We use our prime $2$-CN from~\cref{xmp:prime2CN} and the results of~\cref{sec:prime-2vass} to show that both problems are undecidable. 
Our proof is by reduction from the containment problem\footnote{Actually, the complement thereof.} for $1$-CN: given two $1$-CN $\cA,\cB$ over alphabet $\Sigma$, decide whether $\lang(\cA)\subseteq \lang(\cB)$. This problem was shown to be undecidable in~\cite{hofman2013decidability}.

We begin by describing the reduction (which works for both problems).

%Here we present a reduction from $1$-CN containment, which was shown to be undecidable in~\cite{hofman2013decidability}, to CN compositeness. From that we establish undecidability of CN compositeness, and undecidability of CN primality follows.

Consider an instance of 1-CN containment -- two 1-CNs $\cA$ and $\cB$ over alphabet $\Sigma$.
We construct a 2-CN $\cV$ as follows. Let $\Lambda$ be the alphabet of the 2-prime $\cP$ of \cref{xmp:prime2CN}, and let $\$\notin\Sigma\cup \Lambda$ be a fresh symbol.
Intuitively $\cC$ accepts words of the form $u\$v$ when either $u\in \lang(\cA)$ and $v$ is accepted by $\cP$ starting from the maximal counter $\cA$ ended with on $u$, or when $u\in \lang(\cB)$ and $v\in \Lambda^*$. 
Note that for $u \in \lang(\cB)$ and any word $v \in \Lambda^*$, we have $u\$ v \in \lang(\cC)$. 

Formally, we convert $\cA$ and $\cB$ to 2-CNs $\cA'$ and $\cB'$ by adding a counter and never modifying its value (i.e., transition $(p,\sigma,v,q)$ becomes $(p, \sigma,(v,0),q$)). 
We construct a 2-CN $\cC$ as follows (see~\cref{fig:primalityReduction}). We take $\cA', \cB'$ and $\cP$, and for every accepting state $q$ of $\cA'$ we introduce a transition $(q,\$,\vec{0},p_0)$ where $p_0$ is an initial state of $\cP$. We then add a new accepting state $q_{\top}$, and add the transitions $(q_{\top},\lambda,\vec{0},q_{\top})$ on every letter $\lambda\in \Lambda$ (i.e., $q_\top$ is an accepting sink for $\Lambda$) and $(s_,\$,\vec{0},q_{\top})$ from every accepting state $s$ of $\cB'$.
The initial states are those of $\cA'$ and of $\cB'$, and the accepting states are those of $\cP$ and $q_{\bot}$. 



% \gatodo{was:
% Consider an instance of 1-CN containment -- two 1-CNs $\cA$ and $\cB$ over alphabet $\Sigma$.
% We construct a 2-CN $\cV$ as follows. We convert $\cA$ and $\cB$ to 2-CNs $\cA'$ and $\cB'$ by adding a counter and never modifying its value (i.e., transition $(p,\sigma,v,q)$ becomes $(p, \sigma,(v,0),q$)). 

% Now, let $\Lambda$ be the alphabet of the 2-prime $\cP$ of \cref{xmp:prime2CN}, and let $\$\notin\Sigma\cup \Lambda$ be a fresh symbol. We construct a 2-CN $\cC$ as follows (see~\cref{fig:primalityReduction}). We take $\cA', \cB'$ and $\cP$, and for every accepting state $q$ of $\cA'$ we introduce a transition $(q,\$,\vec{0},p_0)$ where $p_0$ is an initial state of $\cP$. We then add a new accepting state $q_{\top}$, and add the transitions $(q_{\top},\sigma,\vec{0},q_{\top})$ on every letter $\sigma$ (i.e., $q_\top$ is an accepting sink) and $(s_,\$,\vec{0},q_{\top})$ from every accepting state $s$ of $\cB'$.

% The initial states are those of $\cA'$ and of $\cB'$, and the accepting states are those of $\cP$ and $q_{\bot}$. Intuitively $\cC$ accepts words of the form $u\$v$ when either $u\in \lang(\cA')$ and $v$ is accepted by $\cP$ starting from the maximal counter $\cA'$ ended with on $u$, or when $u\in \lang(\cB')$. 
% }
%as illustrated in \cref{fig:primalityReduction}. $\cA',\cB'$ are identical to $\cA,\cB$, except for an extra dimension that is not affected by any transition. $\cP$ denotes the prime VASS discussed in \cref{sec:prime-2vass}. We assume for the sake of convenience that $\cP$'s alphabet $\Sigma_\cP$, $\cA$'s alphabet $\Sigma_\cA$ and $\cB$'s alphabet $\Sigma_\cB$ satisfy $\Sigma_\cA \cap \Sigma_\cP = \emptyset$ and $\Sigma_\cB \cap \Sigma_\cP=\emptyset$, and that neither contain a \$.

% Figure environment removed


% \gatodo{Suggestion: instead of a lemma+corollary, make the corollary a theorem and its proof, the current lemma. Or some other constellation in which there is a theorem in this section.}

\begin{theorem}
\label{thm:primality and minimality are undecidable}
Primality and dimension minimality are undecidable, already for $2$-CN.
\end{theorem}
\begin{proof}
We prove the theorem by establishing the following:
\begin{enumerate}
    \item $\cC$ is not prime if and only if $\lang(\cA)\subseteq \lang(\cB)$.
    \item $\cC$ is not dimension-minimal if and only if $\lang(\cA)\subseteq \lang(\cB)$.
\end{enumerate}
Assume $\lang(\cA)\subseteq \lang(\cB)$, then intuitively the left component becomes redundant, making $\cC$ composite by projecting the unused second counter. 
Formally, we claim that $\lang(\cC)=\{u\$v\mid u\in \lang(\cB)\}$. Indeed, if $w\in \lang(\cC)$ then $w=u\$v$ with either $u\in \lang(\cA')=\lang(\cA)$ (and some condition on $v$) or $u\in \lang(\cB)$, but since $\lang(\cA)\subseteq \lang(\cB)$, this is equivalent to $u\in \lang(\cB)$. 
Since the second counter is not used in the right hand component, we can construct a $1$-CN equivalent to $\cC$ by projecting the right-hand component in \cref{fig:primalityReduction} on the first counter, and deleting the left-hand component entirely.
% \gatodo{I think that "projecting" is not the right word here, and I suggest "omitting" instead. There is another "projecting" below. 
% was: 
% Since the second counter is not used in the right hand component, we can construct a $1$-CN equivalent to $\cC$ by projecting the right-hand component in \cref{fig:primalityReduction} on the first counter. 
% }
It follows that in this case $\cC$ is not dimension minimal, and in particular is not prime.

For the converse, assume $\lang(\cA)\not\subseteq \lang(\cB)$, and let $u\in \lang(\cA) \setminus \lang(\cB)$. Denote $m=\max\{\effect{\rho} \mid \rho \text{ is an accepting run of } \cA \text{ on } u\}$. Then for a word $v$ we have that $u\$v\in \lang(\cC)$ if and only if $v$ is accepted in $\cP$ with initial counter $m$. 
Assume by way of contradiction that $\cC$ is not prime, then we can write $\lang(\cC)$ as an intersection of languages of $1$-CNs. Intuitively (but inaccurately) this is a contradiction since we then get that $\cP$ is not prime as well. 
More precisely, take $v=a^{m_1}\#a^{m_2}\#\cdots\#a^{m_{k+1}}\#b^{m_b}c^{m_c}$ for integers $\left\{m_i\right\}_{i=1}^{k+1},m_b,m_c \in \bbN$ and consider words of the form $u\$v$. We can carry out the analysis in \cref{sec:prime-2vass}, specifically~\cref{lem:two_segments_one_bad} mutatis-mutandis on $u\$v$, to show that the language of $\cP$ is not composite regardless of any fixed initial counter (in this case, $m$), i.e., the analogue of~\cref{thm:2CN_prime}.

We thus have that $\cC$ is prime, and in particular dimension-minimal, concluding the correctness of the reduction.

%, it is straightforward that $\cV$ is composite. Indeed, the left-hand component is redundant, and the right hand component is unused. Therefore $\cV$ has an equivalent $1$-CN $\cC$ obtained by dismissing the left-hand component and eliminating the second counter of the right-hand component. We briefly argue that $\lang(\cV)=\lang(\cC)$. If $w\in \lang(\cC)$, then $w\in \lang(\cV)$ through an identical run. Conversely, let $w\in\lang(\cV)$. If $w$ is accepted through the right-hand component - it is accepted by $\cC$, again, by an identical run. If $w$ is accepted through the left-hand component, we denote $w=u\$v$. Since $\lang(\cA)\subseteq \lang(\cB)$, we have that $u\in \lang(\cB)$, and $u\$v\in \lang(\cC)$. This completes the argument.

%We now prove the converse implication. If $\cV$ is composite, let $\cB_1,\ldots,\cB_n$ be $1$-CNs such that $\lang(\cV)=\bigcap_{i=1}^n\lang(\cB_i)$. Assume by way of contradiction that $\lang(\cA)\not\subseteq \lang(\cB)$. Let $u\in \lang(\cA) \setminus \lang(\cB)$, and let $m=\max\{\effect{\rho} \mid \rho \text{ is an accepting run of } \cB \text{ on } u\}$. We now focus on words of the form $u\$v$, 
%when $v$ is of the form $a^{m_1}\#a^{m_2}\#\cdots\#a^{m_{k+1}}\#b^{m_b}c^{m_c}$ for integers $\left\{m_i\right\}_{i=1}^{k+1},m_b,m_c \in \bbN$. Note that $u\$v\in \lang(\cV) \iff v\in \lang(\cP,m)$. \aytodo{Going with the mutatis mutandis approach, I don't see how to avoid the notion of being accepted by P from an initial counter m. If you prefer to avoid this mathematical notation and just write in with words then we can do that of course.} We now reach a contradiction based on the analysis of \cref{sec:bad_segments}. We use \cref{lem:two_segments_one_bad} \emph{mutatis-mutandis} and conclude that all segments of $a$'s except, perhaps, one, are bad in all $\left\{\cB_i\right\}_{1 \leq i \leq k}$. We then follow the exact reasoning of \cref{sec:proof_of_2Prime} and infer that a word $u\$v'$, that is obtained by pumping the $a$'s of the joint bad segment alongside the $b$'s and the $c$'s, is accepted by all $\left\{\cB_i\right\}_{1 \leq i \leq k}$ and therefore by their intersection, despite the fact that $u\$v'\notin \lang(\cV)$, hence the contradiction.
\end{proof}
