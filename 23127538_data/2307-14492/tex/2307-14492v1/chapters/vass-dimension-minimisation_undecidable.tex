\section{VASS Dimension Minimisation is Undecidable}
\label{sec:primality_undecidable}
\aytodo{Wait, I don't think we even need this. The primality undecidability proof also works for dimension minimization. The original reason we have both written is simply because we came up with this one much earlier. I left this entire section as is for the moment.}

The problem we discuss is the following: given a $k$-VASS $\cV$, is there a $k'$-VASS $\cV'$ for some $k'<k$ such that $\lang(\cV)=\lang(\cV')$. Note that by $\lang(\cV),\lang(\cV')$ we refer to words accepted starting from counter values of $\bar{0}$.

\subsection{1-VASS Regularity is Undecidable}

We prove undecidability using a reduction from a problem we call 1-VASS Regularity, which was shown to be undecidable in \cite{AlmagorY22} as part of the analysis in Section 4.1. In 1-VASS Regularity we ask - given a 1-VASS $\cV$ - is its langage regular?


% In addition to stating this result, we would like to establish the following:

% \begin{lemma}\label{lem:NonHomogenity}
% Let $\cV$ be a 1-VASS such that $\lang(\cV,0)$ is not regular. Then there are $k > 0$ and a word $w$ such that $w \notin \lang(\cV,0)$ but $w \in \lang(\cV,k)$. 
% \end{lemma}

% \begin{proof}
% If we pretend $\cV$ is an NFA $\cA$ (i.e., ignore the transition effects), then we know that $\lang(\cV,0) \neq \lang(\cA)$, since $\lang(\cV,0)$ is not regular. It is the case that $\lang(\cV,0) \subseteq \lang(\cA)$, so there must be a word $w\in \lang(\cA)$ such that $w \notin \lang(\cV,0)$. This can only happen if $w$ has an accepting path that cannot be taken due to a counter violation, and therefore $w$ can be accepted by $\cV$ with a large enough initial counter $k$.  
% \end{proof}

\subsection{The Reduction}

Given a 1-VASS $\cV$, we construct a 2-VASS $\cV_2$ by, intuitively, adding two new letters $a,b$ to the alphabet, and adding two self-loops to each state - the first reading an $a$ an gaining $+1$ on the second counter, and the second reading a $b$ and losing $-1$ on the second counter. Everything else remains the same. Counter effects of $\cV$ all take place in the first counter of $\cV_2$.

\subsection{Correctness}
Assume $\cV$'s language is regular. Then the language of $\cV_2$ is indeed the language of a 1-VASS that uses its states to capture the behaviour of $\cV$ and uses its counter to gain $+1$ for each $a$ and lose $-1$ for each $b$, while staying in the same state.

Now assume $\cV$'s language is not regular, and assume by way of contradiction that the language of $\cV_2$ can be captured by a 1-VASS $\cV_1$. Since $\lang(\cV_1)=\lang(\cV_2)$, we observe the following:

\begin{observation}\label{obs:padding}
$a^n w b^l\in \lang(\cV_1) \iff w\in \lang(\cV)$ for all $w$ not containing $a$'s or $b$'s and for all $n \geq l \in \bbN$.
\end{observation}

We continue by establishing the following:

\begin{lemma}\label{lem:onlyPositiveCycles}
Let $w\in \lang(\cV)$ and $N > |\cV_1|$. Then in every accepting path of $\cV_1$ on $a^Nwb^N$ - every cycle of $a$'s traveled has a strictly positive effect.
\end{lemma}
\begin{proof}
Correctness the lemma is established by simply observing that if a single non-positive cycle in a single accepting run exists - then by traveling it one less time we have that $a^{N-k}wb^N$ is accepted by $\cV_1$, although it shouldn't be.
\end{proof}

Our method from now on is simple (once you see it) - we construct an NFA $\cA$ and prove $\lang(\cA)=\lang(\cV)$. Since $\lang(\cV)$ is not regular - we then have our contradiction and we are done.

We construct $\cA$ based on the structure of $\cV_1$ by the following rules:
\begin{itemize}
    \item The states of $\cA$ are the states of $\cV_1$.
    \item The initial states of $\cA$ are every state $q$ that is reachable from $\cV_1$'s initial state by reading only $a$'s and traversing at least one cycle with a strictly positive effect.
    \item The accepting states of $\cA$ are the accepting states of $\cV_1$, and in addition every state in $\cV_1$ from which an accepting state of $\cV_1$ can be reached by reading only $b$'s.
    \item All transitions with $a$ or $b$ are removed.
    \item Everything else stays the same.
\end{itemize}

We claim that $\lang(\cA)=\lang(\cV)$. Proof by two-sided containment:
Let $w\in \lang(\cV)$. By \cref{obs:padding} - $a^Nwb^N$ is accepted by $\cV_1$ by some path $q_0 \xRightarrow{a^N} q_1 \xRightarrow{w} q_2 \xRightarrow{b^N} q_3$. By \cref{lem:onlyPositiveCycles} - the path $q_0 \xRightarrow{a^N} q_1$ contains a positive cycle. Thus, by definition of $\cA$ - $q_1$ is initial in $\cA$ and $q_2$ is accepting in $\cA$. Therefore $w\in \lang(\cA)$.

Now assume $w \in \lang(\cA)$, through some path $q_1 \xRightarrow{w} q_2$. By definition of $\cA$, there is a path (not necessarily a legal run) in $\cV_2$ as follows: $q_0 \xRightarrow{a^n} q_1 \xRightarrow{w} q_2 \xRightarrow{b^l} q_3$, such that $q_0 \xRightarrow{a^n} q_1$ contains a strictly positive cycle, $l \geq 0$, and $q_3$ is accepting. We pump the positive cycle of $a$ enough times to obtain an accepting run of $\cV_2$ on $a^Mwb^l$ such that $M$ is large enough to satisfy $M\geq l$ and also large enough so that the entire path can be traversed in terms of counter restrictions. We then have that $a^Mwb^l \in \lang(\cV_1)$ for $M \geq l$, and by $\cref{lem:onlyPositiveCycles}$ we get $w \in \lang(\cV)$.

This completes the proof.

