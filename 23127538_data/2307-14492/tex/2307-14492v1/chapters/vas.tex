\section{Labelled Vector Addition Systems}
\label{sec:vas}

\begin{claim}
    Given a labelled 1-VAS $\Vv$, there exists a labelled 1-DVAS $\Uu$ such that $\lang(\Uu) = \lang(\Vv)$.
    \label{clm:1vas-are-1dvas}
\end{claim}
\begin{proof}
    Given a 1-VAS $\Vv$, for each letter $\sigma \in \Sigma$ we will enroll the $\sigma$-vector with the greatest effect, hence $ \Uu \coloneqq \set{\sigma/\max\set{v : \sigma/(v) \in \Vv} : \sigma \in \Sigma}$.
    Note that, a VAS is deterministic if for each letter $\sigma$ there is at most one $\sigma$-vector.
    Lastly, the monotonicity property of VAS(S) implies $\lang(\Uu) = \lang(\Vv)$.
\end{proof}

\begin{equation}
    \Vv \coloneqq \set{ a/(1, 1), b(1, -1), b(-1, 1) }.
    \label{equ:2vas}
\end{equation}

\begin{claim}
    There does not exist a $k$-DVAS $\Uu$ that recognises the same language as $\Vv$ in Equation~\ref{equ:2vas}, for any $k \in \N$.
    \label{clm:2vas-is-non-deterministic}
\end{claim}
\begin{proof}
    First, observe that $\lang(\Vv) = a\,\set{a, b}^* \cup \set{\varepsilon}$.
    This is true because in order to read a `$b$', an `$a$' must have been read before.
    Furthermore, if an `$a$' has been read, then any number of `$b$'s can be read, this can be achieved by alternating which `$b$'-vector is used as the sum of their effects is zero.
    Now for sake of contradiction, assume there exists $k \in \N$ such that there exists a $k$-DVAS $\Uu$ such that $\lang(\Uu) = \lang(\Vv)$.
    Since $\Uu$ is deterministic, and the language $a\,\set{a, b}^*$ is defined over two letters `$a$' and `$b$', we know $\Uu = \set{ a/\vec{x}, b/\vec{y} }$.
    Since $a\,b^* \subseteq a\,\set{a, b}^*$, it must be true that $\vec{y} \geq \vec{0}$.
    This yields the contradiction as this implies $b \in \lang(\Uu)$, but $b \notin \lang(\Vv)$.
    Therefore, there does not exist a $k$-DVAS $\Uu$ such that $\lang(\Uu) = \lang(\Vv)$ for any $k \in \N$.
\end{proof}

\begin{claim}
    The 2-VAS $\Vv$ in Equation~\ref{equ:2vas} is dimension-prime.
    \label{clm:prime-2vas}
\end{claim}
\begin{proof}
    Suppose for sake of contradiction that there exists $n$ many 1-VAS $\Uu_1, \ldots, \Uu_n$ such that $\lang(\Uu_1) \cap \cdots \cap \lang(\Uu_n) = \lang(\Vv)$.
    For each $i \in \set{1, \ldots, n}$, we will consider a necessary feature that the 1-VAS $\Uu_i$ must have.
    Since $a\,b^* \subseteq \lang(\Vv)$, it must be true that there exists $v \geq 0$ such that $b/(v) \in \Uu_1$.
    Otherwise there must exist a large enough $k \in \N$ such that $a\,b^k \notin \lang(\Uu_i)$, this would mean that $a\,b^k \notin \lang(\Uu_1) \cap \cdots \cap \lang(\Uu_n)$.
    However, this does mean that $b \in \lang(\Uu_i)$.
    As this holds for each $i \in \set{1, \ldots, n}$, so $b \in \lang(\Uu_1) \cap \cdots \cap \lang(\Uu_n)$, so $\lang(\Uu_1) \cap \cdots \cap \lang(\Uu_n) \neq \lang(\Vv)$, creating a contradiction.
\end{proof}

\hstodo{I reckon the 4-VAS: $\set{ a/(1, 1, 0, 0), b(1, -1, 1, 1), b(-1, 1, 1, 1), c/(0, 0, 1, -1), c(0, 0, -1, 1)}$ is dimension-prime, but I'm not sure how to argue this. 
I've got some feeling that this can be generlised to `build' some regular languages this way (see the below photo).
Seems the number of states in the language equivalent DFA might (linearly) correspond with the dimension of the VAS? There's some room here for state-simulation for sure. 
I'm not sure how this lines up against the `transitional' $(d+3)-VAS$ that simulate $d$-VASS.
Nor am I sure what happens to the language if one applies this VASS to VAS transformation.}

{\centering% Figure removed}

\hstodo{The above image is just a photo of some of the examples I stumbled into when thinking about dimension prime VAS.
I'm also starting to think starting counter values are really important if you want to simulate states, I could be wrong ... but we may need to consider a variant where we start with some non-zero vector.
But just being sketchy about this for now, I don't want to put too much time into this if labeled VAS are a bit off topic.}