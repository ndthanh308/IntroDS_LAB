\section{1-VASS Properties}
\label{sec:1vass-properties}

We start by providing several helpful definition and analyzing several properties of 1-VASSes.

We would now like to establish the following three lemmas:

\begin{lemma}\label{lem:largeenoughwordnonnegativecycle}
    Let $\cA=\tup{Q,\Sigma,q_0,R,F}$ be a 1-VASS, let $(q,n)$ be a configuration of $\cA$ and let $\sigma \in \Sigma$. Then there exists $N\in \bbN$ such that for all $N' \geq N$, every run of $\cA$ on $\sigma^{N'}$ from configuration $(q,n)$ traverses a $\nonnegative$ cycle. 
\end{lemma}
\begin{proof}
    We will prove that $N=|Q|(n+W|Q|+2)$ satisfies the requirements, where $W$ is the maximal positive transition effect in $\cA$. 

    Assume by way of contradiction that $\cA$ can read $\sigma^{N}$ through a run $\rho=(q_1,n_1=n)\stackrel{\rho}{\to} (q_k,n_k)$ that only traverses $\negative$ cycles. We perform the following procedure on $\rho$ - we iterate over $\rho$ until we encounter the same state twice for the first time - which signifies a simple cycle. We then remove this cycle and correct counter values accordingly, to obtain a new run $\rho'$. Note that since all cycles in $\rho$ are $\negative$ - $\rho'$ is indeed a legal run of $\cA$, and $\effect{\rho'} > \effect{\rho}$. We perform an identical process on $\rho'$ to obtain a run $\rho''$, and so on, until we finish with a cycle-free run $\rho_f$. Note that $\rho_f$ satisfies $\effect{\rho_f} > \ldots > \effect{\rho'} > \effect{\rho}$. $\rho_f$ is cycle-free, therefore its length is at most $|Q|$. since $\rho_f$ starts from initial counter $n$, and taking into account that the maximal effect of $\cA$ is $W$ - the maximal counter value possible at the end of $\rho_f$ is $n+W|Q|$. Since $\rho$ is of length $N=|Q|(n+W|Q|+2)$, and taking into account that the maximal length of a simple cycle is $|Q|$, transitioning from $\rho$ to $\rho_f$ we have removed at least $n+W|Q|+1$ simple cycles. Therefore $W|Q| \geq \effect{\rho_f} \geq \effect{\rho}+n+W|Q|+1 \geq -n+(n+W|Q|+1)$ (the effect of $\rho$ has to be at least $-n$, otherwise it is not a legal run). After simplification we get $0 \geq 1$, which is a contradiction. 
\aytodo{I've done some minor changes. so the concluding argument is a bit different. Do you think there's anything not formal enough here?}
\shtodo{I'm pretty sure this can be simplified to a ``one-shot'' removal. I'll think.}
\hstodo{What about:

We will prove that $N = \abs{Q}(n + \abs{Q}\cdot\norm{\Aa})$ satisfies the requirements, where $W$ is the maximal positive transition effect in $\Aa$.

Assume for sake of contradiction that there exists a $\sigma^N$-run $\rho = ((q_i, n_i))_{i=0}^M$ where $q_0 = q$ and $n_0 = n$ that only traverses negative cycles.
Note that $M \geq N$.
By pigeonhole principle, there exists a state $p \in Q$ that is visited $m \geq M/\abs{Q} \geq N/\abs{Q}$ many times throughout the run $\rho$.
Consider all of the indices $0 \leq i_1 < i_2 < \ldots < i_m \leq N$ such that $p = q_{i_1} = \ldots = q_{i_N}$.
Each run segment $(q_{i_1}, n_{i_1}) \rightarrow (q_{i_2}, n_{i_2}), \ldots, (q_{i_{m-1}}, n_{i_{m-1}}) \rightarrow (q_{i_m}, n_{i_m})$ must have negative effect for it is a cycle.
Thus $n_{i_1} > n_{i_2} > \ldots > n_{i_m} \geq 0$, and so $n_{i_1} \geq n_{i_m} + m \geq 0$.
Moreover, $n_{i_1} < n + \abs{Q}\cdot\norm{\Aa}$ since the prefix $(q, n) \rightarrow (q_{i_1}, n_{i_1})$ cannot contain a non-negative cycle.
However, since $m \geq N/\abs{Q} = n + \abs{Q}\cdot\norm{\Aa}$ and $n_{i_1} \geq n_{i_m} + m \geq m \geq n + \abs{Q}\cdot\norm{A}$, which creates a contradiction.
}

\aytodo{I like it. The "cannot contain a non-negative cycle" part 3 lines from the end is based on Lemma 1.3 right?}
\end{proof}


\begin{lemma}\label{lem:cyclethensimplecycle}
    Let $\cA$ be a 1-VASS and $\pi=(q_0,c_0)\stackrel{\sigma}{\to} (q_1,c_1) \stackrel{\sigma}{\to} \ldots (q_n=q_0,c_n)$ be a cyclic run of $\cA$ such that $\pi$ is $\nonnegative$. Then there is a sequence of (not necessarily contiguous) indices $0 \leq i_1 \leq \ldots \leq i_k \leq n$ such that $q_{i_1}\stackrel{\sigma}{\to} q_{i_2}\stackrel{\sigma}{\to}\cdots q_{i_k}$ is a simple $\nonnegative$ cycle of $\cA$ with some effect $e \geq 0$. In addition, this simple cycle is infinitely pumpable from the former occurrence of $q_{i_1}$ in $\pi$. In other words, $\pi_m=(q_0,c_0)\stackrel{\sigma}{\to} \ldots (q_{i_1},c_{i_1}) \rightarrow^{\sigma}\ldots (q_{i_1},c_{i_1}+em)\stackrel{\sigma}{\to} \ldots (q_n,c_n+em)$ is a legal run of $\cA$ from $(q_0,c_0)$ for all $m$. 
\end{lemma}
\begin{proof}
    We prove by induction on the length of $\pi$. The base of the induction is a cyclic run of length 2. In this case $\pi=(q_0,c_0)\stackrel{\sigma}{\to} (q_1=q_0,c_1)$ is itself a $\nonnegative$ simple cycle that is infinitely pumpable from $(q_0,c_0)$.

    We now assume correctness for length n, and discuss $\pi=(q_0,c_0)\stackrel{\sigma}{\to} (q_1,c_1) \stackrel{\sigma}{\to} \ldots (q_n=q_0,c_n)$ of length $n+1$. Let $0\leq j_1<j_2\leq n$ be indices such that $q_{j_1}=q_{j_2}$, for a maximal $j_1$. Note that the cycle $\tau=(q_{j_1},c_{j_1})\stackrel{\sigma}{\to} \ldots (q_{j_2},c_{j_2})$ must be simple. If $j_1=0$ and $j_2=n$, then $\pi$ itself is a simple $\nonnegative$ cycle, and the pumping argument is straightforward. Otherwise $\tau$ is nested. We now treat two cases separately:
    \begin{enumerate}
        \item $\tau$ is $\nonnegative$. In this case the induction hypothesis applies on $\tau$. We take the guaranteed constants $j_1 \leq i_1 \leq \ldots \leq i_k \leq j_2$, which apply to $\pi$ just as well as they do to $\tau$.
        \item $\tau$ is $\negative$. In this case we remove $\tau$ from $\pi$ to obtain $\pi'=(q_0,c_0)\stackrel{\sigma}{\to} \ldots (q_{j_1},c_{j_1}) \stackrel{\sigma}{\to} (q_{j_2+1},c'_{j_2+1}) \stackrel{\sigma}{\to} \ldots (q_n,c'_n)$, such that $c'_i \geq c_i$ for all $j_2+1 \leq i \leq n$. The induction hypothesis applies on $\pi'$, and let $i_1\ \ldots i_k$ be the guaranteed constants. Note that $i_1 \leq j_1$, since the cycle removed transitioning from $\pi$ to $\pi'$ was the last occurrence of a repetition of states in $\pi$. We therefore know that $q_{i_1}\stackrel{\sigma}{\to} q_{i_2}\stackrel{\sigma}{\to}\cdots q_{i_k}$ is a simple $\nonnegative$ cycle in $\pi'$ - which applies to $\pi$ as well. In addition, it is infinitely pumpable from configuration $(q_{i_1},c_{i_1})$ in $\pi'$ for $i_1 \leq j_1$. Since all configurations are identical up to and including $(q_{j_1},c_{j_1})$ between $\pi$ and $\pi'$ - this cycle is infinitely pumpable in $\pi$ as well.
        \shtodo{This is nice! }
    \end{enumerate}

\end{proof}

We conclude the discussion about this lemma with several observations. Note that the existence of a $\nonnegative$ cycle in $\rho$ implies the existence of an infinitely pumpable $\nonnegative$ simple cycle of length $l$, which can be pumped $\frac{n|Q|!}{l}$ times for all $n\in \bbN$, to obtain a run $\rho'$ on $\sigma^k+n|Q|!$ from a run $\rho$ on $\sigma^k$.  

\begin{observation} \label{obs:nonnegativepumptofactorial}
    Let $\cA=\tup{Q,\Sigma,q_0,R,F}$ be a $1$-CN, and let $\rho$ be a run of $\cA$ on $w=\sigma^k$ that traverses a $\nonnegative$ cycle. Utilizing \cref{lem:cyclethensimplecycle}, we can construct a run $\rho'$ of $\cA$ on $\sigma^k+n|Q|!$ for any $n\in \bbN$ by pumping the guaranteed $\nonnegative$ simple cycle $c$ $\frac{n|Q|!}{|c|}$ times, such that $\effect{\rho'} \geq \effect{\rho}$.   
\end{observation}
\begin{observation} 
\label{obs:positiveornegativepumptofactorial}
    \cref{lem:cyclethensimplecycle}, as well as \cref{obs:nonnegativepumptofactorial}, apply to $\positive$ cycles, just as well as they do to $\nonnegative$ cycles \emph{mutatis-mutandis}. In addition, Note that the following simple claim also holds: let $\rho$ be a run of $\cA$ on $w=\sigma^k$ that traverses a cycle. Then we can construct a semi-run $\rho'$ of $\cA$ on $\sigma^k+n|Q|!$ for any $n\in \bbN$ by pumping any simple cycle. \aycomm{Do you think it's better to split it into two observations?}
\end{observation}
% \shtodo{This needs to be stated more formally (perhaps join the two observations to one).}
% \begin{observation} \label{obs:negativesemipumptofactorial}
%     Let $\cA=\tup{Q,\Sigma,q_0,R,F}$ be a 1-VASS, and let $\rho$ be a run of $\cA$ on $w=\sigma^k$ that traverses a cycle. Then there is a semi-run of $\cA$ on $\sigma^k+n|Q|!$ for all $n\in \bbN$. This is observed by simply pumping any simple cycle in $\rho$ (if $\rho$ itself is not simple - we take one of its innermost nested cycles arbitrarily). Since we only claim the existence of a semi-run, this innermost cycle's effect is irrelevant. 
% \end{observation}

The final claim in our preliminary analysis is the following:

\begin{lemma}\label{lem:maxcounternopositivecycles}
    Let $\cA=\tup{Q,\Sigma,q_0,R,F}$ be a 1-VASS, let $(q,n)$ be a configuration of $\cA$, let $W$ be the maximal positive weight in $\cA$, $\sigma \in \Sigma$, and $N\in\bbN$. If a run $\rho$ of $\cA$ on $\sigma^N$ from configuration $(q,n)$ does not traverse any $\positive$ cycle, then the maximal possible counter value at the end of $\rho$ is $n+W|Q|$. 
\end{lemma}

\begin{proof}
    If $\rho$ traverses cycles $\Pi=\{\pi_1,\ldots \pi_k\}$ - we remove a maximal cycle to obtain a new run $\rho'$ that traverses a set of cycles $\Pi'$. Since no cycle in $\rho$ is $\positive$, it is the case that $\effect{\rho'} \geq \effect{\rho}$. In addition, since we have removed a non-nested cycle, it is the case that $\Pi' \subseteq \Pi$, and therefore $\rho'$ does not traverse any $\positive$ cycle.  
    \shtodo{I think we need a more delicate argument. It could be the case that after removing a simple negative cycle, you're suddenly left with a positive cycle, and you can't ``repeat''. I think removing maximal cycles should do the trick.}
    \aytodo{Whoa, you're right, I completely overlooked that. Your idea works, I changed accordingly}
    We continue the process recursively until we are left with a cycle-less run $\rho_f$. Since it is cycle-free, the length of $\rho_f$ is at most $|Q|$, and therefore $\effect{\rho_f} \leq W|Q|$. Since $\effect{\rho_f} \geq \ldots \geq \effect{\rho'} \geq \effect{\rho}$, we have that $\effect{\rho} \leq W|Q|$ hence the counter value at the end of $\rho$ is no higher than $n+W|Q|$. 
\end{proof}