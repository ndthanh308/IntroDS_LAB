\documentclass{article}

% if you need to pass options to natbib, use, e.g.:
\PassOptionsToPackage{numbers, compress}{natbib}
% before loading neurips_2023


% ready for submission
% \usepackage[nonatbib]{neurips_2023}
% \usepackage{neurips_2023}
\usepackage[preprint]{neurips_2023}

% to compile a preprint version, e.g., for submission to arXiv, add add the
% [preprint] option:
%     \usepackage[preprint]{neurips_2023}


% to compile a camera-ready version, add the [final] option, e.g.:
%     \usepackage[final]{neurips_2023}


% to avoid loading the natbib package, add option nonatbib:
% \usepackage[nonatbib]{neurips_2023}
\usepackage[numbers,compress]{natbib}


\usepackage[utf8]{inputenc} % allow utf-8 input
\usepackage[T1]{fontenc}    % use 8-bit T1 fonts
\usepackage{hyperref}       % hyperlinks
\usepackage{url}            % simple URL typesetting
\usepackage{booktabs}       % professional-quality tables
\usepackage{amsfonts}       % blackboard math symbols
\usepackage{nicefrac}       % compact symbols for 1/2, etc.
\usepackage{microtype}      % microtypography
\usepackage{xcolor}         % colors

% add packages bellow
\usepackage{graphicx}
\usepackage{subfigure}
\usepackage{kotex}
\usepackage{duckuments}
% \usepackage{natbib}
% \usepackage{cite}
% For theorems and such
\usepackage{amsmath}
\usepackage{amssymb}
\usepackage{mathtools}
\usepackage{amsthm}
\usepackage[capitalize,noabbrev]{cleveref}
\usepackage[textsize=tiny]{todonotes}

% yh : figure related packages
\usepackage{wrapfig}
\usepackage{lipsum}
\usepackage[export]{adjustbox}

% yh : code related packages
\usepackage{listings}
\usepackage{xcolor}

\definecolor{codegreen}{rgb}{0,0.6,0}
\definecolor{codegray}{rgb}{0.5,0.5,0.5}
\definecolor{codepurple}{rgb}{0.58,0,0.82}
\definecolor{backcolour}{rgb}{0.95,0.95,0.92}

\lstdefinestyle{mystyle}{
    backgroundcolor=\color{backcolour},   
    commentstyle=\color{codegreen},
    keywordstyle=\color{magenta},
    numberstyle=\tiny\color{codegray},
    stringstyle=\color{codepurple},
    basicstyle=\ttfamily\footnotesize,
    breakatwhitespace=false,         
    breaklines=true,                 
    captionpos=b,                    
    keepspaces=true,                 
    numbers=left,                    
    numbersep=5pt,                  
    showspaces=false,                
    showstringspaces=false,
    showtabs=false,                  
    tabsize=2
}

\lstset{style=mystyle}

%%%%%%%%%%%%%%%%%%%%%%%%%%%%%%%%
% THEOREMS
%%%%%%%%%%%%%%%%%%%%%%%%%%%%%%%%
\theoremstyle{plain}
\newtheorem{theorem}{Theorem}[section]
\newtheorem{proposition}[theorem]{Proposition}
\newtheorem{lemma}[theorem]{Lemma}
\newtheorem{corollary}[theorem]{Corollary}
\theoremstyle{definition}
\newtheorem{definition}[theorem]{Definition}
\newtheorem{assumption}[theorem]{Assumption}
\theoremstyle{remark}
\newtheorem{remark}[theorem]{Remark}

\usepackage{algorithmic}% http://ctan.org/pkg/algorithms
\usepackage{algorithm}% http://ctan.org/pkg/algorithms
% \usepackage{algpseudocode}% http://ctan.org/pkg/algorithmicx


\newcommand\calF{\mathcal{F}}
\newcommand\calG{\mathcal{G}}
\newcommand\calM{\mathcal{M}}
\newcommand\calV{\mathcal{V}}
\newcommand\calU{\mathcal{U}}
\newcommand\calW{\mathcal{W}}
\newcommand\calP{\mathcal{P}}
\newcommand\calD{\mathbb{D}}
%%%%%%%%%%%%%%%%%
%% macros introduced by Luke 
\newcommand\mydef[1]{{\bf\em #1}}
%%%%%%%%%%%%%%%%%

\newcommand{\numviparams}{{| \lambda |}}
\newcommand{\scoreaccvars}[1]{s_1^{#1}, \ldots, s_{\numviparams}^{#1}}
\newcommand{\scoreaccvar}[2]{s_{#1}^{#2}}
\newcommand{\isdeterm}[1]{\text{Deterministic}({#1})}


\newcommand{\expect}[1]{\mathbb{E}\left[{#1}\right]}
\newcommand{\var}[1]{\mathbb{V}\left[ {#1} \right]}
\newcommand{\expectdist}[2]{\mathbb{E}_{#1}\left[ {#2} \right]}
\newcommand{\vardist}[2]{\mathbb{V}_{#1}\left[ {#2} \right]}
\newcommand{\cov}[2]{\mathbb{C}\text{ov}[{#1}][{#2}]}
\newcommand{\covv}[1]{\mathbb{C}\text{ov}[{#1}]}
\newcommand{\corr}[1]{\mathbb{C}\text{orr}[{#1}]}

\newcommand{\fix}[1]{\mathit{fix}\left({#1}\right)}
\newcommand{\sbr}[1]{\left\llbracket {#1} \right\rrbracket}
\newcommand{\ctxtype}[3]{{#1} \cong_\text{ctx} {#2} : {#3}}
\newcommand{\bigstep}[3]{{#1} \Downarrow_{#2} {#3}}


% PCF types
\newcommand{\bool}{\mathit{bool}}
\newcommand{\nat}{\mathit{nat}}

\newcommand{\ctx}[1]{\mathcal{C}\left[ {#1}\right] }
\newcommand{\pcft}[1]{\text{PCF}_{#1}}

\newcommand{\nfl}{\mathbb{N}_\bot}
\newcommand{\bfl}{\mathbb{B}_\bot}

% PCF constructs
\newcommand{\succc}[1]{\mathbf{succ}({#1})}
\newcommand{\succcn}[2]{\mathbf{succ}^{#1}({#2})}
\newcommand{\zero}{\mathbf{0}}
\newcommand{\zerotest}[1]{\mathbf{zero}\left({#1}\right)}
\newcommand{\pred}[1]{\mathbf{pred}\left( {#1} \right)}
\newcommand{\predn}[2]{\mathbf{pred}^{#1}\left( {#2} \right)}
\def\solvable{\#}

\newcommand{\true}{\mathbf{true}}
\newcommand{\false}{\mathbf{false}}
\newcommand{\pcffix}[1]{\mathbf{fix}\left({#1}\right)}
\newcommand{\pcffn}[3]{\mathbf{fn}~{#1}:{#2}\mathpunct{.}{#3}}
\newcommand{\pairtype}[2]{{#1} * {#2}}
\newcommand{\pairexp}[2]{\mathbf{pair}({#1}, {#2})}
\newcommand{\leftexp}[1]{\mathbf{left}({#1})}
\newcommand{\rightexp}[1]{\mathbf{right}({#1})}

\newcommand{\RationalPos}{\mathbb{Q}^{+}}

\newcommand{\meas}[1]{\mathbb{M}\left( {#1} \right) }
\newcommand{\integ}[1]{\sbr{#1}_I}

\newcommand{\notbigstep}[2]{{#1}~\cancel{\Downarrow}_{#2}}
\newcommand{\subtrace}[3]{{#1}^{{#2} \ldots {#3}}}
\newcommand{\supp}[1]{\textsf{supp}\left({#1}\right)}
\newcommand{\dom}[1]{\textsf{Dom}\left({#1}\right)}
\newcommand{\suppk}[2]{\textsf{Supp}^{#1}\left({#2}\right)}
\newcommand{\tracespace}{\bigcup_{n \in \mathbb{N}}[0, 1]^n}
\newcommand{\generictracespace}{\mathbb{T}}
\newcommand{\nnreals}{\mathbb{R}_{\geq 0}}
\newcommand{\posreals}{\mathbb{R}_{> 0}}
\newcommand{\reals}{\mathbb{R}}

\newcommand{\unrollkM}[2]{\textsf{unroll}_{#1}\left({#2}\right)}
\newcommand{\nphmcint}[5]{\Psi_\textsf{NP}\left({#1}, {#2}, {#3}, {#4}, {#5}\right)}

%SPCF constructs
\newcommand{\spcfvalues}{\Lambda^0_v}

\newcommand{\prevalueM}[1]{\textsf{value}^{-1}_{#1}(\spcfvalues{})}
\newcommand{\num}[1]{\underline{#1}}

% \theoremstyle{definition}
% \newtheorem{thm}{Theorem}
% \newtheorem{lem}{Lemma}
% \newtheorem{defn}{Definition}
% \newtheorem{conj}{Conjecture}
% \newtheorem{prop}{Proposition}

%\theoremstyle{definition}
%\newtheorem{defn}{Definition}[section]
%\newtheorem{example}[defn]{Example}
%
%
%\theoremstyle{plain}
%\newtheorem{thm}{Theorem}[section]
%\newtheorem{lem}[thm]{Lemma}
%\newtheorem{cor}[thm]{Corollary}
%\newtheorem{conj}[thm]{Conjecture}
%\newtheorem{prop}[thm]{Proposition}
%\newtheorem{remark}[thm]{Remark}

%% Proofs
%\let\oldproof\proof
%\renewcommand{\proof}{\color{blue}\oldproof}


\definecolor{codegreen}{rgb}{0,0.6,0}
\definecolor{codegray}{rgb}{0.5,0.5,0.5}
\definecolor{codepurple}{rgb}{0.58,0,0.82}
\definecolor{backcolour}{rgb}{0.95,0.95,0.92}

\lstdefinestyle{myStyle}{
    belowcaptionskip=1\baselineskip,
    breaklines=true,
    frame=none,
    basicstyle=\footnotesize\ttfamily,
    keywordstyle=\bfseries\color{green!40!black},
    commentstyle=\itshape\color{purple!40!black},
    identifierstyle=\color{blue},
    backgroundcolor=\color{gray!10!white},
    %backgroundcolor=\color{backcolour}, 
    numberstyle=\tiny\color{codegray},
    stringstyle=\color{codepurple},
    breakatwhitespace=false,                          
    keepspaces=true,                 
    numbers=left,       
    numbersep=5pt,                  
    showspaces=false,                
    showstringspaces=false,
    showtabs=false,                  
    tabsize=2,
}

% argmin/argmax
\DeclareMathOperator*{\argmax}{arg\,max}
\DeclareMathOperator*{\argmin}{arg\,min}

% Concatenation of lists
\newcommand\doubleplus{+\kern-1.3ex+\kern0.8ex}

% Program configurations
\newcommand{\tuple}[1]{\ensuremath{\langle #1 \rangle}}
% Rule based definitions
\newcommand{\Rule}[4][]{\ensuremath{\inferrule*[lab={\hypertarget{#2}{(\TirName{#2})}},#1]{#3}{#4}}}

% Calligraphic symbols
\newcommand{\calI}{{\mathcal I}} 
\newcommand{\calT}{{\mathcal T}}

%  Macro for new Y operator.
\newcommand{\yBounded}[3]{\mu^{#1}_{#2}\rvert_{#3}}

%%%%%%%%%%%%%%%%%
 
%%%%%%%%%%%%%%%%%

\newcommand{\expv}{\mathbb{E}}

\newcommand{\combTr}[2]{\left[\begin{matrix}
		#1\\
		#2
	\end{matrix} \right]}

\newcommand{\exType}[2]{\left\{\begin{matrix}
		#1\\
		#2
	\end{matrix} \right\}}
\newcommand{\myint}[1]{ [#1]}
\newcommand{\Uniform}{\ensuremath{\mathrm{Uniform}}}
\newcommand{\Normal}{\ensuremath{\mathrm{normal}}}
\DeclareMathOperator{\abs}{abs}
\DeclareMathOperator{\pdf}{pdf}

\newcommand{\intConf}[1]{\lceil#1\rceil}
\newcommand{\tr}{\boldsymbol{t}}

\newcommand{\sample}{\tt{sample}}
%\newcommand{\fix}{\texttt{fix}}
%\newcommand{\num}[1]{\underline{#1}}
\newcommand{\myif}{\texttt{if}}
\newcommand{\mylet}{\texttt{let} \, }
\newcommand{\myin}{\, \texttt{in} \,}
\newcommand{\mythen}{\, \texttt{then} \,}
\newcommand{\myelse}{\, \texttt{else} \,}
\newcommand{\score}{\tt{score}}
\newcommand{\tick}{\tt{tick}}

\newcommand{\term}{\tt{term}}
\newcommand{\pv}{\mathbf{v}}
\newcommand{\rv}{\mathbf{r}}

\newcommand{\interval}{\mathfrak{I}}

\newcommand{\typeReal}{\textbf{\textsf{R}}}

\newcommand{\symbolInt}{\myint{\cdot}}

\newcommand{\LambdaInterval}{\Lambda_{\interval}}
\newcommand{\LambdaSymbolic}{\Lambda_{\text{sym}}}

\newcommand{\toIntervalTerm}[1]{#1^{2\interval}}

%Others
\newcommand{\Sset}{\mathbb{S}}
\newcommand{\Iset}{\mathbb{I}}
\newcommand{\Rset}{\mathbb{R}}
\newcommand{\Nset}{\mathbb{N}}
\newcommand{\Zset}{\mathbb{Z}}

\newcommand{\Term}{\mathbb{T}}
\newcommand{\prob}{\mathbb{P}}
\newcommand{\expt}{\mathbb{E}}


\newcommand{\Leb}{\tt{Leb}}
\newcommand{\Red}{\tt{Red}}
\newcommand{\cost}{\text{cost}}

%\newcommand{\intervalab}[2]{\underline{[#1,#2]}}
\newcommand{\intervalab}{\underline{[a,b]}}
\newcommand{\interI}{\mathcal{I}}
\newcommand{\trans}{\mathcal{T}}

\newcommand{\iv}{\mathbb{I}}

% Programming language constructs
\newcommand{\lit}[1]{\underline{#1}}
\newcommand{\letIn}[1]{\mathsf{let}\,{#1}\,\mathsf{in}\,}
\newcommand{\fixLam}[2]{\mu {#1} {#2}.}
\newcommand{\ifElse}[3]{\mathsf{if} (#1 \le \num{0}) \, {#2} \,\mathsf{else}\, {#3}}

%%Basic notions
\newcommand{\pspace}{(\Omega,\mathcal{F},\probm)}
\newcommand{\probm}{\mathbb{P}}
\newcommand{\condexpv}[2]{{\expt}{\left[{#1} \mid {#2}\right]}}

\newcommand{\stdConf}[1]{(#1)}
%\newcommand{\intConf}[1]{\lceil#1\rceil}
%\newcommand{\intConf}[1]{(#1)}
%\newcommand{\symConf}[1]{\langle\!\langle  #1 \rangle\!\rangle}
%\newcommand\symPath[1]{(#1)}
\newcommand{\symPath}[1]{\langle\!\langle  #1 \rangle\!\rangle}
\newcommand\symConf[1]{(#1)}

\newcommand{\ifSimple}[3]{\mathsf{if}(#1, #2, #3)}
%\newcommand{\ifElse}[3]{\mathsf{if} (#1 \le 0) \, \allowbreak {#2} \, \allowbreak \mathsf{else}\, {#3}}
%\newcommand{\ifElse}[3]{\ifSimple{#1}{#2}{#3}}

%\newcommand{\trace}{\mathsf{s}}
%
%\newcommand\defn[1]{{\bf \em #1}}
\newcommand{\traces}{\mathbb{T}}
%
%\newcommand{\stdConf}[1]{(#1)}
%%\newcommand{\intConf}[1]{\lceil#1\rceil}
%\newcommand{\intConf}[1]{(#1)}
%%\newcommand{\symConf}[1]{\langle\!\langle  #1 \rangle\!\rangle}
%%\newcommand\symPath[1]{(#1)}
%\newcommand{\symPath}[1]{\langle\!\langle  #1 \rangle\!\rangle}
%\newcommand\symConf[1]{(#1)}

\newcommand{\valueSem}[1]{\mathsf{val}_{#1}} % value (semantics)
\newcommand{\weightSem}[1]{\mathsf{wt}_{#1}} % weight (semantics)
\newcommand{\measureSem}[1]{\llbracket #1 \rrbracket}
\newcommand{\posterior}{\mathsf{posterior}}


%%%%%%%%%
% 
%%%%%%%%
\newcommand{\loc}{\ell}
\newcommand{\locs}{\mathit{L}}
\newcommand{\blocs}{\mathit{L}_{\mathrm{b}}}

\newcommand{\iflocs}{\mathit{L}_{\mathrm{if}}}
\newcommand{\looplocs}{\mathit{L}_{\mathrm{while}}}

\newcommand{\alocs}{\mathit{L}_{\mathrm{a}}}
\newcommand{\wlocs}{\mathit{L}_{\mathrm{w}}}
\newcommand{\rlocs}{\mathit{L}_{\mathrm{r}}}
\newcommand{\Alocs}[1]{\mathit{L}_{\mathrm{A}}^{\mathsf{#1}}}
\newcommand{\Dlocs}{\mathit{L}_{\mathrm{nd}}}
\newcommand{\transitions}{{\rightarrow}}

%%% 
\newcommand{\plocs}{\mathit{L}_{\mathrm{p}}}
\newcommand{\tlocs}{\mathit{L}_{\mathrm{t}}}

\newcommand{\lin}{\loc_\mathrm{init}}
\newcommand{\lout}{\loc_\mathrm{out}}
\newcommand{\val}[1]{\mbox{\sl Val}_{#1}}

\newcommand{\pvars}{V_\mathrm{p}}
\newcommand{\rvars}{V_{\mathrm{r}}}
\newcommand{\pre}{\mathrm{pre}}

\newcommand{\sle}{\sqsubseteq}
\newcommand{\sge}{\sqsupseteq}

\newcommand{\lfp}{\mathrm{lfp}}
\newcommand{\gfp}{\mathrm{gfp}}

\newcommand{\rdvarjdis}{\mathcal D}
\newcommand{\sampset}{\textit{supp}}

\newcommand{\upd}{\mbox{\sl upd}}
\newcommand{\wet}{\mbox{\sl wt}}
\newcommand{\transset}{\mathfrak T}
\newcommand{\valin}{\pv_{\mathrm{init}}}
\newcommand{\ret}{\mbox{\sl ret}}

\newcommand{\win}{w_{\mathrm{init}}}

\newcommand{\sampdpd}{\overline{\Upsilon}}

\newcommand{\outmap}{\text{O}}
\newcommand{\sat}[1]{\langle #1 \rangle}
\newcommand{\monoid}{\mbox{\sl Monoid}}
\newcommand{\handelmanformat}{(\dagger)}

\newcommand{\trunc}{\mathcal{B}}

\newcommand{\ewt}{\mbox{\sl ewt}}
\newcommand{\statemap}{\text{St}}

\newcommand{\valrd}{{\mathbf{r}}}
\newcommand{\frmloc}{\ell^{\mathrm{src}}}
\newcommand{\toloc}{\ell^{\mathrm{dst}}}

\newcommand{\monomials}{\mathbf{M}}


% \title{Unsupervised Discovery of Semantic Latent \yh{Basis} in Diffusion Models}
% \title{Unsupervised Discovery of Semantic Latent Directions in Diffusion Models}
% \title{
%     \yh{Understanding the Semantic Structure of Diffusion Models using Riemannian Geometry}
% }
% \title{
%     Understanding the Geometry of the Latent Space of Diffusion Models
% }
\title{
    Understanding the Latent Space of Diffusion Models through the Lens of Riemannian Geometry
}

% The \author macro works with any number of authors. There are two commands
% used to separate the names and addresses of multiple authors: \And and \AND.
%
% Using \And between authors leaves it to LaTeX to determine where to break the
% lines. Using \AND forces a line break at that point. So, if LaTeX puts 3 of 4
% authors names on the first line, and the last on the second line, try using
% \AND instead of \And before the third author name.

\author{
Yong-Hyun Park$^{*1}$,~
Mingi Kwon$^{*2}$,~
Jaewoong Choi$^{3}$,~
Junghyo Jo$^{\dagger1}$,~
Youngjung Uh$^{\dagger2}$,
\vspace{+0.3em}
\\
\normalsize$^1$Seoul National University~~
$^2$Yonsei University~~
$^3$Korea Institute for Advanced Study\\
% \scriptsize {\texttt{\{enkeejunior1,jojunghyo\}@snu.ac.kr}}~~
% \scriptsize {\texttt{\{kwonmingi,yj.uh\}@yonsei.ac.kr}}~~
% \scriptsize {\texttt{chjw1475@kias.re.kr}}
}

% long version
% \author{
%     Yong-Hyun Park\\
%     % Department of Physics Education\\
%     Seoul National University\\
%     % Seoul, Korea \\
%     \texttt{enkeejunior1@snu.ac.kr} \\
%     \And
%     Mingi Kwon\thanks{Equal contributions}\\
%     % Department of Artificial Intelligence\\
%     Yonsei University\\
%     % Seoul, Korea \\
%     \texttt{kwonmingi@yonsei.ac.kr} \\
%     \And
%     Jaewoong Choi\\
%     Korea Institute for Advanced Study \\
%     \texttt{chjw1475@kias.re.kr} \\
%     \And
%     Junghyo Jo \\
%     % Department of Physics Education\\
%     Seoul National University\\
%     % Seoul, Korea \\
%     \texttt{jojunghyo@snu.ac.kr} \\
%     \And
%     Youngjung Uh\\
%     % Department of Artificial Intelligence\\
%     Yonsei University\\
%     % Seoul, Korea \\
%     \texttt{yj.uh@yonsei.ac.kr} \\
% }

% \newcommand{\citet}[1]{\citeauthor{#1} }


\begin{document}


\maketitle

% \blfootnote{$^*$Equal Contribution\hspace{3mm} $^dagger$ Corresponding authors}

\begin{abstract}
Despite the success of diffusion models (DMs), we still lack a thorough understanding of their latent space. 
To understand the latent space $\vx_t \in \mathcal{X}$, we analyze them from a geometrical perspective. 
Specifically, we utilize the pullback metric to find the local latent basis in $\mathcal{X}$ and their corresponding local tangent basis in $\mathcal{H}$, the intermediate feature maps of DMs. 
% {We validate latent basis by editing image through latent space traversal.}
% \modify{Discovered latent basis has unsupervised image editing capability, achieved through latent space traversal.}
{The discovered latent basis enables unsupervised image editing capability through latent space traversal.}
% The discovered latent basis allows for unsupervised image editing through latent space traversal."
% "The discovered latent basis provides unsupervised image editing capability through latent space traversal."
% "The discovered latent basis facilitates unsupervised image editing through latent space traversal
We investigate the discovered structure from two perspectives.
First, we examine how geometric structure evolves over {diffusion timesteps}. Through analysis, we show that {1) the model focuses on low-frequency components early in the generative process and attunes to high-frequency details later};
2) At early timesteps, different samples share similar tangent spaces;
and {3) The simpler datasets that DMs trained on, the more consistent the tangent space for each timestep.}
Second, we investigate how the geometric structure changes based on text conditioning in Stable Diffusion. The results show that 1) similar prompts yield {comparable} tangent spaces; and 2) the model depends less on text conditions in later timesteps.
To the best of our knowledge, this paper is the first to present image editing through $\vx$-space traversal and provide thorough analyses of the latent structure of DMs. 
% \jo{This paper presents the first proposal for image editing through x-space traversal and provides a thorough analysis of the geometry of DMs.}
%Our paper is the first to propose image editing through x-space traversal and analyze the geometry of DMs.

% mingi
% Despite the success of diffusion models (DMs), we still lack a thorough understanding of their latent space. 
% While image editing with Generative Adversarial Networks (GANs) builds upon latent space, DMs rely on editing the conditions such as text prompts.
% We present an unsupervised method to discover interpretable editing directions for the latent variables $\vx_t\in\mathcal{X}$ of DMs. 
% Our method adopts Riemannian geometry between \exspace{} and the intermediate feature maps \ehspace{} of the U-Nets to provide a deep understanding over the geometrical structure of \exspace{}. 
% The discovered local semantic basis and local tangent space by Riemannian geometry not only allow us to edit real images but also give us a rich understanding of DMs. 
% In this paper, we demonstrate that walking along the basis can successfully conduct real image editing in a semantically meaningful way in both unconditional diffusion models and Stable diffusion model. We also investigate what the DMs focus on within diffusion timesteps, samples, and datasets by analyzing discovered basis and tangent space. 

% \modify{they are globally consistent across different samples. Furthermore, editing in earlier timesteps edits coarse attributes, while ones in later timesteps focus on high-frequency details. We define the curvedness of a line segment between samples to show that \exspace{} is a curved manifold. Experiments on different baselines and datasets demonstrate the effectiveness of our method even on Stable Diffusion.} 
% Our source code will be publicly available for the future researchers.
\end{abstract}

\nnfootnote{$^*$Equal Contribution\hspace{3mm} $^\dagger$ Corresponding authors}
\vspace{-2.5em}

\section{Introduction}

Vision-based Bird's Eye View (BEV) representation\cite{lu2021graph,xie2023x, yang2023bevformer, bartoccioni2023lara, lin2022sparse4d} is an emerging perception formulation for autonomous driving. It transforms and maps the information from the image space to a unified 3D BEV space, which can be used for various perception tasks like 3D object detection and BEV map segmentation. Moreover, the unified BEV space can directly fuse other modalities like LiDAR without any cost, which is of great scalability.

As shown in \cref{fig:first}, the essence of BEV representation is the 2D to 3D mapping, which is a one-to-many ill-posed problem because a point in the image space corresponds to infinite collinear 3D points along the camera ray. To resolve this problem, we need to add an extra condition to make the 2D to 3D mapping a one-to-one well-posed problem. For the added extra condition, there are two kinds of methods, which are LSS\cite{philion2020lift} and OFT\cite{roddick2018orthographic}. LSS proposes to predict latent depth as the extra condition, which is implicitly estimated by end-to-end training. OFT directly maps the 2D information to 3D in the one-to-many fashion, while a network in BEV space is needed to implicitly select the dense mapped information in the vertical or height direction, which is also realized by end-to-end training. Both methods use extra depth or height conditions to resolve the mapping problem, but the extra condition is implicitly trained and used. In this way, the correctness of the mapping is not guaranteed, which might affect the performance of BEV representation.

% Figure environment removed

Motivated by the above observations, we propose to explicitly add and model extra conditions to realize better 2D to 3D mapping.
Similarly, some works\cite{park2021pseudo,li2022bevdepth} propose to directly learn depth as the extra condition with depth pre-training or LiDAR information. Different from using depth, we explicitly model the height condition in the mapping for the following reasons. First, we prove that height in the BEV space is equivalent to depth in the image space for the 2D to 3D mapping problem. Both ways can provide equivalent conditions to resolve the problem of mapping. In this way, we can realize well-defined one-to-one mapping between 2D and 3D. Second, the height information in the BEV space can be retrieved from the BEV annotations without any other data modalities like LiDAR, while depth condition needs extra pre-training or LiDAR. In this work, we use the height information from the object's 3D bounding box, which can be directly accessed from the ground truth. Third, the modeling in height can fit arbitrary camera rigs and types. For example, on NuScenes\cite{nuscenes2019}, the focal length of the backward camera is different from other cameras, resulting in different depth estimation patterns. In other words, different depth estimation network is needed for different cameras. While for the height condition, no matter which kind of camera configuration is used, it is processed with the same pattern in the BEV space. In this way, the height condition is more robust and flexible.

In this work, we propose a network that explicitly models height in the BEV space, which fulfills the condition needed for 2D to 3D mapping, termed as HeightFormer. Moreover, based on the height modeling, self-recursive height predictors are proposed to introduce the uncertainty of heights and segmentation maps which are used in the BEV query mask mechanism to produce high-precision detection results. In summary, the main contributions of our work are summarized as follows:
% \begin{itemize}
    % \item
    1) We give theoretical proof of the equivalence between height-based methods in BEV and depth-based methods in images, which is the basis of our work. The proof also demonstrates the feasibility of detection in the BEV space generated with predicted heights.
    % \item 
    2) {{\color{blue}}We propose to explicitly model heights in the BEV space without extra LiDAR supervision. A self-recursive predictor is proposed to model heights and a corresponding segmentation-based query mask is designed to handle positions whose heights cannot be defined.
    % \item 
    3) Experiments on NuScenes\cite{nuscenes2019} show that the proposed HeightFormer achieves the SOTA performance. Extensive quantitative and qualitative results show that it is feasible and effective to model heights in the BEV space and construct the BEV representation with predicted heights. The generalization analysis also shows that the proposed method can be applied to different methods, as a plugin and as compensation for depth modeling. }
% \end{itemize} 

\section{Related works}
\paragraph{Diffusion Models.}
Recent advances in DMs make great progress in the field of image synthesis and show state-of-the-art performance \cite{sohl2015deep, ho2020denoising, song2020denoising}. 
An important subject in the diffusion model is the introduction of gradient guidance, including classifier-free guidance, to control the generative process \cite{dhariwal2021diffusion,sehwag2022generating,avrahami2022blended,liu2021more,nichol2021glide,rombach2022high}. 
The work by \citet{song2020score} has facilitated the unification of DMs with score-based models using SDEs, enhancing our understanding of DMs as a reverse diffusion process.
However, the latent space is still largely unexplored, and our understanding is limited.

\paragraph{The study of latent space in GANs.} The study of latent spaces has gained significant attention in recent years. In the field of Generative Adversarial Networks (GANs), researchers have proposed various methods to manipulate the latent space to achieve the desired effect in the generated images \cite{ramesh2018spectral,patashnik2021styleclip,abdal2021styleflow, harkonen2020ganspace,shen2021closed,yuksel2021latentclr, pan2023drag}.
% \modify{For example, local latent space manipulation techniques \cite{ramesh2018spectral,patashnik2021styleclip,abdal2021styleflow} have been developed, as well as global manipulation techniques  \cite{harkonen2020ganspace,shen2021closed,yuksel2021latentclr}. 이젠 굳이 local global 나누지 말죠} 
More recently, several studies \cite{zhu2021low, choi2021not} have examined the geometrical properties of latent space in GANs and utilized these findings for image manipulations. These studies bring the advantage of better understanding the characteristics of the latent space and facilitating the analysis and utilization of GANs. In contrast, the latent space of DMs remains poorly understood, making it difficult to fully utilize their capabilities.

\paragraph{Image manipulation in DMs.}
Early works include \citet{choi2021ilvr} and \citet{meng2021sdedit} have attempted to manipulate the resulting images of DMs by replacing latent variables, allowing the generation of desired random images. 
However, due to the lack of semantics in the latent variables of DMs, current approaches have critical problems with semantic image editing.
Alternative approaches have explored the potential of using the feature space within the U-Net for semantic image manipulation. For example, \citet{kwon2022diffusion} have shown that the bottleneck of the U-Net, \ehspace{}, can be used as a semantic latent space. \modify{Specifically, they used CLIP \cite{radford2021learning} to identify directions within $\mathcal{H}$ that facilitate genuine image editing.} \citet{baranchuk2021label} and \citet{tumanyan2022plug} use the feature map of the U-Net for semantic segmentation and maintaining the structure of generated images. 
% Unlike previous works, our editing method directly traverses the latent variable along the latent basis.
Unlike previous works, our editing method \modify{finds the editing direction without supervision}, and directly traverses the latent variable along the latent basis.

% \yh{
% It is worth noting that, as a concurrent work, \citet{haas2023discovering} aims to find the semantic direction which could manipulate the generative process. However, they are only dealing with operations in a proxy semantic latent space, \ehspace{}. 
% % Moreover, they manipulate on synthetic latent variables and thus, it is  not a real image editing. 
% On the other hand, we manipulate the latent variable directly, but also find a semantic local subspace in \exspace{}, which gives us a useful insight for DM's mechanism.
% }

%%% ICML ver
% An important subject is the introduction of gradient guidance, including classifier-free guidance, to control the generative process \cite{dhariwal2021diffusion,sehwag2022generating,avrahami2022blended,liu2021more,nichol2021glide,rombach2022high}. \citet{choi2021ilvr} and \citet{meng2021sdedit} have attempted to manipulate the resulting images of DMs by replacing latent variables, allowing the generation of desired random images. However, due to the lack of semantics in the latent variables of DMs, current approaches have critical problems with semantic image editing.

%%% ICML ver
% Alternative approaches have explored the potential of using the feature space within the U-Net for semantic image manipulation. For example, \citet{baranchuk2021label} and \citet{tumanyan2022plug} use the feature map of the U-Net for semantic segmentation and maintaining the structure of generated images. \citet{kwon2022diffusion} have shown that the bottleneck of the U-Net, \ehspace{}, can be used as a semantic latent space.
% But the experimental observation lacks a theoretical understanding of the feature map of DMs.
% \yh{
% It is worth noting that, as a concurrent work, \citet{haas2023discovering} aims to find the semantic direction which could manipulate the generative process. However, they are only dealing with operations in a proxy semantic latent space, \ehspace{}. 
% % Moreover, they manipulate on synthetic latent variables and thus, it is  not a real image editing. 
% On the other hand, we manipulate the latent variable directly, but also find a semantic local subspace in \exspace{}, which gives us a useful insight for DM's mechanism.
% }


\paragraph{Riemannain Geometry.} Some studies have applied Riemannian geometry to analyze the latent spaces of deep generative models, such as Variational Autoencoders (VAEs) and GANs \cite{arvanitidis2017latent, shao2018riemannian, chen2018metrics, arvanitidis2020geometrically, lee2023explicit, pmlr-v162-lee22d, yonghyeon2021regularized}. \citet{shao2018riemannian} proposed a pullback metric on the latent space from image space Euclidean metric to analyze the latent space's geometry. This method has been widely used in VAEs and GANs because it only requires a differentiable map from latent space to image space.
% \modify{However, it has limitations such as a lack of evidence for applying the Euclidean metric in image space. 이부분 저희 이번 논문에서 해결을 안해준것 같은데..ㅎ}
However, no studies have investigated the geometry of latent space of DMs utilizing the pullback metric.
\section{Edge Guided GANs with Contrastive Learning}

\noindent \textbf{Framework Overview.}
Figure~\ref{fig:method} shows the overall structure of ECGAN for semantic image synthesis, which consists of a semantic and edge guided generator $G$ and a multi-modality discriminator $D$.
The generator $G$ consists of eight components:
(1) a parameter-sharing convolutional encoder $E$ is proposed to produce deep feature maps $F$; 
(2) an edge generator $G_e$ is adopted to generate edge maps $I'_e$ taking as input deep features from the encoder;
(3) an image generator $G_i$ is used  to produce intermediate images $I'$;
(4) an attention guided edge transfer module $G_t$ is designed to forward useful structure information from the edge generator to the image generator;
(5) the semantic preserving module $G_s$ is developed to selectively highlight class-dependent feature maps according to the input label for generating semantically consistent images $I''$;
(6) a label generator $G_l$ is employed to produce the label from $I''$;
(7) the similarity loss is proposed to calculate the intra-class and inter-class relationships.
(8) the contrastive learning module $G_c$ aims to 
model global semantic relations between training pixels, guiding pixel embeddings towards cross-image
category-discriminative representations that eventually improve the generation performance.

Meanwhile, to effectively train the network, we propose a multi-modality discriminator $D$ that distinguishes the outputs from both modalities, i.e., edge and image.

\subsection{Edge Guided Semantic Image Synthesis}
\noindent \textbf{Parameter-Sharing Encoder.}
The backbone encoder $E$ can employ any deep network architecture, e.g., the commonly used AlexNet \cite{krizhevsky2012imagenet},
VGG \cite{simonyan2015very}, and ResNet \cite{he2016deep}. 
We directly utilize the feature maps  from the last convolutional layer as deep feature representations, i.e., $F {=} E(S)$, where $E$ represents the encoder; $S{\in} \mathbb{R}^{N \times H \times W}$ is the input label, with $H$ and $W$ as width and height of the input semantic labels, and $N$ as the total number of semantic classes.
Optionally, one can always combine multiple intermediate feature maps to enhance the feature representation.
The encoder is shared by the edge generator and the image generator.
Then, the gradients from the two generators all contribute  to updating the parameters of the encoder.
This compact design can potentially enhance the deep representations as the encoder can simultaneously learn structure representations from the edge generation branch and appearance representations from the image generation branch.

% Figure environment removed

\noindent \textbf{Edge Guided Image Generation.}
As discussed, the lack of detailed structure or geometry guidance makes it extremely difficult for the generator to produce realistic local structures and details.
To overcome this limitation, we propose to adopt the edge as guidance.
A novel edge generator $G_e$ is designed to directly generate the edge maps from the input semantic labels.
This also facilitates the shared encoder to learn more local structures of the targeted images.
Meanwhile, the image generator $G_i$ aims to generate photo-realistic images from the input labels.
In this way, the encoder is boosted to learn the appearance information of the targeted images.

 

Previous works \cite{park2019semantic,liu2019learning,qi2018semi,chen2017photographic,wang2018high} directly use deep networks to generate the target image, which is challenging since the network needs to simultaneously learn appearance and structure information from the input labels.
In contrast, the proposed method  learns structure and appearance separately via the proposed edge generator and image generator. 
Moreover, the explicit guidance from the ground truth edge maps can also facilitate the training of the encoder.
The framework of both edge and image generators is illustrated in Figure~\ref{fig:edge_block}.
Given the feature maps from the last convolutional layer of the encoder, i.e., $F {\in} \mathbb{R}^{C \times H \times W}$, 
where $H$ and $W$ are the width and height of the features, and $C$ is the number of channels, the edge generator produces edge features and edge maps which are further utilized to guide the image generator to generate the intermediate image $I'$.
The edge generator $G_e$ contains $n$ convolution layers and correspondingly produces $n$ intermediate feature maps $F_e {=} \{ F_e^j\}_{j=1}^n$.
After that, another convolution layer with Tanh non-linear activation is utilized to generate the edge map $I'_e{\in} \mathbb{R}^{3 \times H \times W}$.
Meanwhile, the feature maps $F$ is also fed into the image generator $G_i$ to generate $n$ intermediate feature maps $F_i{=}\{F_i^j\}_{j=1}^n$.
Then another convolution operation with Tanh non-linear activation is adopted to produce the intermediate image $I'_i{\in} \mathbb{R}^{3 \times H \times W}$.
In addition, the intermediate edge feature maps $F_e$ and the edge map $I'_e$ are utilized to guide the generation of the image feature maps $F_i$ and the intermediate image $I'$ via the Attention Guided Edge Transfer as detailed below.

\noindent \textbf{Attention Guided Edge Transfer.}
We further propose a novel attention guided edge transfer module $G_t$ to explicitly employ the edge structure information to refine the intermediate image representations.
The architecture of the proposed transfer module $G_t$ is illustrated in Figure~\ref{fig:edge_block}.
To transfer useful structure information from edge feature maps $F_e {=} \{ F_e^j\}_{j=1}^n$ to the image feature maps $F_i{=}\{F_i^j\}_{j=1}^n$, the edge feature maps are firstly processed by a Sigmoid activation function to generate the corresponding attention maps $F_a {=}{\rm Sigmoid}(F_e) {=} \{ F_a^j\}_{j=1}^n$.
The attention aims to provide structural information (which cannot be provided by the input label map) within each semantic class.
Then, we multiply the generated attention maps with the corresponding image feature maps to obtain the refined maps, which incorporate local structures and details.
Finally, the edge refined features are element-wisely summed with the original image features to produce the final edge refined  features, which are further fed to the next convolution layer as $F_i^j {=}{\rm Sigmoid}(F_e^j) {\times} F_i^j {+} F_i^j  (j  {=}  1, \cdots, n)$.
In this way, the image feature maps also contain the local structure information provided by the edge feature maps.
Similarly, to directly employ the structure information from the generated edge map $I_e^{'}$ for image generation, we adopt the attention guided edge transfer module to refine the generated image directly with edge information as
\begin{equation}
\begin{aligned}
I' = {\rm Sigmoid}(I'_e) \times I'_i +  I'_i,
\end{aligned}\label{eqn:image}
\end{equation}
where $I'_a{=}{\rm Sigmoid}(I'_e)$ is the generated attention map. We also visualize the results in Figure~\ref{fig:diff2}.


% Figure environment removed


\subsection{Semantic Preserving Image Enhancement}

\noindent \textbf{Semantic Preserving Module}. Due to the spatial resolution loss caused by  convolution, normalization, and down-sampling layers, existing models \cite{wang2018high,park2019semantic,qi2018semi,chen2017photographic} cannot fully preserve the semantic information of the input labels as illustrated in Figure~\ref{fig:city_seg}.
For instance, the small ``pole'' is missing, and the large ``fence'' is incomplete.
To tackle this problem, we propose a novel semantic preserving module, which aims to select class-dependent feature maps and further enhance it through the guidance of the original semantic layout. 
An overview of the proposed semantic preserving module $G_s$ is shown in Figure \ref{fig:semantic_block}(left).
Specifically, the input of the module denoted as $\mathcal{F}$,
is the concatenation of the input label $S$, the generated intermediate edge map $I'_e$ and image $I'$, and the deep feature $F$ produced from the shared encoder $E$.
Then, we apply a convolution operation on $\mathcal{F}$ to produce a new feature map $\mathcal{F}_c$ with the number of channels equal to the number of semantic categories, where each channel corresponds to a specific semantic category (a similar conclusion can be found in \cite{fu2019dual}).
Next, we apply the averaging pooling operation on $\mathcal{F}_c$ to obtain the global information of each class, followed by a  Sigmoid activation function to derive scaling factors $\gamma'$ as in $\gamma' {=} {\rm Sigmoid} ({\rm AvgPool}(\mathcal{F}_c))$, where each value represents the importance of the corresponding class. 
Then, the scaling factor $\gamma'$ is adopted to reweight the feature map $\mathcal{F}_c$ and highlight corresponding class-dependent feature maps.
The reweighted feature map is further added with the original feature $\mathcal{F}_c$ to compensate for information loss due to multiplication, and 
produces $\mathcal{F}'_c {=} \mathcal{F}_c  {\times} \gamma' {+} \mathcal{F}_c$, where $\mathcal{F}'_c {\in} \mathbb{R}^{N \times H \times W}$.

After that, we perform another convolution operation on $\mathcal{F}'_c$ to obtain the feature map $\mathcal{F}' {\in} \mathbb{R}^{(C+N+3+3) \times H \times W}$ to enhance the representative capability of the feature. In addition, $\mathcal{F}'$ has the same size as the original input one $\mathcal{F}$, which makes the module flexible and can be plugged into other existing architectures without modifications of other parts to refine the output.
In Figure~\ref{fig:semantic_block}(right), we visualize three channels in~$\mathcal{F}'$ on Cityscapes, i.e., road, car, and vegetation.
We can easily observe that each channel learns well the class-level deep
representations.

Finally, the feature map $\mathcal{F}'$ is fed into a convolution layer followed by a Tanh non-linear activation layer to obtain the final result $I''$.
Our semantic preserving module enhances the representational power of the model by adaptively recalibrating semantic class-dependent feature maps, and shares similar spirits with style transfer \cite{huang2017arbitrary}, and SENet \cite{hu2018squeeze} and EncNet \cite{zhang2018context}. 
One intuitive example of the utility of the module is for the generation of small object classes: these classes are easily missed in the generation results due to spatial resolution loss, while our scaling factor can put an emphasis on small objects and help preserve them.

\noindent \textbf{Similarity Loss.} 
Preserving semantic information from isolated pixels is very challenging for deep networks. To explicitly enforce the network to capture the relationship between semantic categories, a new similarity loss is introduced. This loss forces the network to consider both intra-class and inter-class pixels for each pixel in the label. Specifically, a state-of-the-art pretrained model (i.e., SegFormer \cite{xie2021segformer}) is used to transfer the generated image $I''$ back to a label $S'' {\in} \mathbb{R}^{N \times H \times W}$, where $N$ is the total number of semantic classes, and $H$ and $W$ represent the width and height of the image, respectively. A conventional method uses the cross entropy loss between $S''$ and $S$ to address this problem. 
However, such a loss only considers the isolated pixel while ignoring the semantic correlation with other pixels.

To address this limitation, we construct a similarity map from $S{\in} \mathbb{R}^{N \times H \times W}$. 
Firstly, we reshape $S$ to $\hat{S}{\in} \mathbb{R}^{N {\times} M}$, where $M {=} H {×}W$. Next, we perform a matrix multiplication to obtain a similarity map $A{=}\hat{S}\hat{S}^\top {\in} \mathbb{R}^{M{\times}M}$. This similarity map encodes which pixels belong to the same category, meaning that if the j-\textit{th} pixel and the i-\textit{th} pixel belong to the same category, then the value of the j-\textit{th} row and the i-\textit{th} column in $A$ is 1; otherwise, it is 0.
Similarly, we can obtain a similarity map $A''$ from the label $S''$.
Finally, we calculate the binary cross entropy loss between the two similarity maps $\{a_m {\in}A, m{\in} [1, M^2]\}$ and $\{a''_m{\in}A'', m{\in}[1, M^2]\}$ as
\begin{equation}
	\begin{aligned}
\mathcal{L}_{sim}(S, S'') = - \frac{1}{M^2} \sum_{m=1}^{M^2} (a_m \log a''_m + (1-a_m)\log (1-a''_m)).	\end{aligned}\label{eq:similarityloss}
\end{equation}
This loss explicitly captures intra-class and inter-class semantic correlation, leading to better generation results.



% Figure environment removed


\subsection{Contrastive Learning for Semantic Image Synthesis}

\noindent\textbf{Pixel-Wise Contrastive Learning.}
Existing semantic image synthesis models use deep networks to map labeled pixels to a non-linear embedding space. However, these models often only take into account the ``local'' context of pixel samples within an individual input semantic layout, and fail to consider the ``global'' context of the entire dataset, which includes the semantic relationships between pixels across different input layouts. This oversight raises an important question: what should the ideal semantic image synthesis embedding space look like? Ideally, such a space should not only enable accurate categorization of individual pixel embeddings, but also exhibit a well-structured organization that promotes intra-class similarity and inter-class difference. 
That is, pixels from the same class should generate more similar image content than those from different classes in the embedding space.
Previous approaches to representation learning propose that incorporating the inherent structure of training data can enhance feature discriminativeness. 
Hence, we conjecture that despite the impressive performance of existing algorithms, there is potential to create a more well-structured pixel embedding space by integrating both the local and global context.

The objective of unsupervised representation learning is to train an encoder that maps each training semantic layout $S$ to a feature vector $v {=} B(S)$, where $B$ represents the backbone encoder network. The resulting vector $v$ should be an accurate representation of $S$. To accomplish this task, contrastive learning approaches use a training method that distinguishes a positive from multiple negatives, based on the similarity principle between samples. The InfoNCE \cite{van2018representation,gutmann2010noise} loss function, a popular choice for contrastive learning, can be expressed as
\begin{equation}
		\mathcal{L}_S =   -\log
		\frac {\exp(v  \cdot  v_+ /  \tau  )}{
			\exp (v  \cdot  v_+  / \tau) +  \sum _ {v_- \in N_ {S}} {\exp(v\cdot v_- / \tau) }},
\end{equation}
where $v_+$ represents an embedding of a positive for $S$, and $N_S$ includes embeddings of negatives. The symbol ``·'' refers to the inner (dot) product, and $\tau {>}0$ is a temperature hyper-parameter. It is worth noting that the embeddings used in the loss function are normalized using the $L_2$ method.

One limitation of this training objective design is that it only penalizes pixel-wise predictions independently, without considering the cross-relationship between pixels. 
To overcome this limitation, we take inspiration from \cite{wang2021exploring,khosla2020supervised} and propose a contrastive learning method that operates at the pixel level and is intended to regularize the embedding space while also investigating the global structures present in the training data (see Figure \ref{fig:method_contrastive}).
Specifically, our contrastive loss computation uses training semantic layout pixels as data samples. For a given pixel $i$ with its ground-truth semantic label $c$, the positive samples consist of other pixels that belong to the same class $c$, while the negative samples include pixels belonging to other classes $C{\setminus}{c}$. As a result, the proposed pixel-wise contrastive learning loss is defined as follows
\begin{equation}
		\mathcal{L}_i =   \frac {1}{|P_ {i}|}   \sum_{i_+ \in P_i}  -\log
		\frac {\exp(i  \cdot  i_+ /  \tau  )}{
			\exp (i  \cdot  i_+  / \tau) +  \sum _ {i_- \in N_ {i}} {\exp(i\cdot i_- / \tau) }}.
\label{eq:contrastive}
\end{equation}
For each pixel $i$, we use $P_i$ and $N_i$ to represent the pixel embedding collections of positive and negative samples, respectively. Importantly, the positive and negative samples and the anchor $i$ are not required to come from the same layout. The goal of this pixel-wise contrastive learning approach is to create an embedding space in which same-class pixels are pulled closer together, and different-class pixels are pushed further apart. The result of this process is that pixels with the same class generate image contents that are more similar, which can lead to superior generation performance.


\noindent \textbf{Multi-Scale Contrastive Learning.}
In this part, we extend the pixel-level loss function $\mathcal{L}_i$ in Eq. \eqref{eq:contrastive} to an
arbitrary scale loss function $\mathcal{L}_i^s$ for calculating the contrastive learning loss, where $s$ means the $s$-\textit{th} scale feature representation, and we have a total of $\mathcal{S}$ different scales. 
This strategy regularizes the feature space of different scales by pulling features of the same class closer and pulling features of different classes apart, leading to a more well-structured feature space.

The overview framework of the proposed multi-scale contrastive learning is shown in Figure \ref{fig:method_contrastive_multi}. 
First, the input layouts go through the backbone encoder network $B$ to obtain multi-scale representation.
Next, we use a weighted sum at different scales to constraint the multi-scale features
\begin{equation}
\begin{aligned}
& \mathcal{L}_{i}^{ms}  =  \sum_{s=1}^\mathcal{S} w_s \mathcal{L}_i^s = w_1 \mathcal{L}_i^1 + \cdots + w_s \mathcal{L}_i^s + \cdots + w_\mathcal{S} \mathcal{L}_i^\mathcal{S} = \\
&  w_1 \frac {1}{|P_{i}^1|}   \sum_{i_+^1 \in P_i^1}  {-}\log \frac {\exp(i^1  \cdot  i_+^1 /  \tau  )}{\exp (i^1  \cdot  i_+^1  / \tau) +  \sum _ {i_-^1 \in N_{i}^1} {\exp(i^1\cdot i_-^1 / \tau) }} \\
& + \cdots + \\
&  w_s \frac {1}{|P_{i}^s|}   \sum_{i_+^s \in P_i^s}  {-}\log \frac {\exp(i^s  \cdot  i_+^s /  \tau  )}{\exp (i^s  \cdot  i_+^s  / \tau) +  \sum _ {i_-^s \in N_{i}^s} {\exp(i^s\cdot i_-^s / \tau) }} \\
   & + \cdots + \\
& w_\mathcal{S} \frac {1}{|P_{i}^\mathcal{S}|}   \sum_{i_+^\mathcal{S} \in P_i^\mathcal{S}} {-}\log \frac {\exp(i^\mathcal{S}  \cdot  i_+^\mathcal{S} /  \tau  )}{
\exp (i^\mathcal{S}  \cdot  i_+^\mathcal{S}  / \tau) +  \sum _ {i_-^\mathcal{S} \in N_{i}^\mathcal{S}} {\exp(i^\mathcal{S}\cdot i_-^\mathcal{S} / \tau) }}.
\label{eq:contrastive_multi}
\end{aligned}
\end{equation}
To identify the semantic classes in each pixel of different scale feature maps, we use the original input layout downsampled to the spatial dimensions.
We select the feature pairs with the same semantic label and at the same scale as positive pairs. On the contrary, we choose the feature pairs with different semantic labels and within the same scale as negative pairs.
Specifically, for each pixel $i^s$, we use $P_i^s$ and $N_i^s$ to represent the pixel embedding collections of positive and negative samples at the $s$-\textit{th} scale feature representation, respectively. Noth that the positive and negative samples and the anchor $i^s$ are from different layouts but the same scale feature embedding space. The weights $[w_1, \cdots, w_s, \cdots, w_\mathcal{S}]$ control the contribution of each scale to the overall loss.
Note that the first scale loss $\mathcal{L}_i^1$ is the same as the pixel-wise contrastive learning $\mathcal{L}_i$ in Eq. \eqref{eq:contrastive}.

As shown in Figure \ref{fig:method_contrastive_multi}, we also need to push same-class features from different scales closer together and pull different-class features apart.
For instance, if we have two scales $s_p$ and $s_q$, we hope features of the same class to be close on scales $s_p$ and $s_q$ ($s_p{\neq}s_q$), and features of different classes to be far apart on both scales $s_p$ and $s_q$.
That is, local features should describe parts of objects/regions of their global structure of the object and vice versa.
Thus the cross-scale contrastive learning loss can be formulated as
\begin{equation}
\begin{aligned}
&\mathcal{L}_{i}^{cs} =  \sum_{s_p=1}^{s_p=S} \sum_{s_q=1}^{s_q=S} w_{s_p, s_q} \mathcal{L}_i^{s_p, s_q} = \\
& w_{1, 2} \mathcal{L}_i^{1, 2} + \cdots  + w_{1, s} \mathcal{L}_i^{1, s} + \cdots + w_{1, \mathcal{S}} \mathcal{L}_i^{1, \mathcal{S}} + \cdots + w_{s, \mathcal{S}} \mathcal{L}_i^{s, \mathcal{S}}.
\end{aligned}
\label{eq:contrastive_cross}
\end{equation}
We downsample the original input layout into layouts of different scales on the spatial dimension so that we can obtain the semantic labels at each scale. We select the feature pairs with the same semantic label but at different scales as positive samples. In contrast, we select feature pairs with different semantic labels and at different scales as negative samples.
By doing so, we can achieve a bidirectional local-global consistency for learning the encoder network.
The weights $[w_{1,2}, \cdots, w_{1,s}, \cdots, w_{1,\mathcal{S}}, \cdots, w_{s, \mathcal{S}}]$ control the contribution of each scale to the overall loss.

Eq. \eqref{eq:contrastive_multi} and \eqref{eq:contrastive_cross} can be added together to obtain our complete contrastive learning loss.

\noindent \textbf{Class-Specific Pixel Generation.}
To overcome the challenges posed by training data imbalance between different classes and size discrepancies between different semantic objects, we introduce a new approach that is specifically designed to generate small object classes and fine details. Our proposed method is a class-specific pixel generation approach that focuses on generating image content for each semantic class. Doing so can avoid the interference from large object classes during joint optimization, and each subgeneration branch can concentrate on a specific class generation, resulting in similar generation quality for different classes and yielding richer local image details.

An overview of the class-specific pixel generation method is provided in Figure~\ref{fig:method_contrastive}. 
After the proposed pixel-wise contrastive learning, we obtain a class-specific feature map for each pixel. 
Then, the feature map is fed into a decoder for the corresponding semantic class, which generates an output image $\hat{I}_{i}$. 
Since we have the proposed contrastive learning loss, we can use the parameter-shared decoder to generate all classes.
To better learn each class, we also utilize a pixel-wise $L_1$ reconstruction loss, which can be expressed as $\mathcal{L}_{L_1} {=} \sum_{i=1}^{N} \mathbb{E}_{I_i, \hat{I}_i} \lbrack \vert\vert I_i {-} \hat{I}_i \vert\vert_1 \rbrack.$
The final output $I_g$ from the pixel generation network can be obtained by performing an element-wise addition of all the class-specific outputs:
$I_g {=} I_{g_1} \oplus I_{g_2} \oplus \cdots \oplus I_{g_N}.$


\subsection{Model Training}
\noindent \textbf{Multi-Modality Discriminator.}
To facilitate the training of the proposed method for high-quality edge and image generation, a novel multi-modality discriminator is developed to simultaneously distinguish outputs from two modality spaces, i.e., edge and image. 
Since the edges and RGB images share the same structure, they can be learned using the multi-modality discriminator. In the preliminary experiment, we also tried to use two discriminators (i.e., an edge discriminator and an image discriminator), but no performance improvement was observed while increasing the model complexity. Thus, we use the proposed multi-modality discriminator.
The framework of the multi-modality discriminator is shown in Figure~\ref{fig:method}, which is capable of discriminating both real/fake images and edges. 
To discriminate real/fake edges, the discriminator loss considering the semantic label $S$ and the generated edge $I'_e$ (or the real edge $I_e$) is as
\begin{equation}
\begin{aligned}
\mathcal{L}_{\mathrm{CGAN}}(G_e, D) & = 
\mathbb{E}_{S, I_e} \left[ \log D(S, I_e) \right] \\
& +  \mathbb{E}_{S, I'_e} \left[\log (1 - D(S, I'_e)) \right],
\end{aligned}
\label{eqn:discriminator1}
\end{equation}
which guides the model to distinguish real edges from fake generated edges.
Further, to discriminate real/fake images, the discriminator loss regarding  semantic label $S$ and the generated images $I'$, $I''$ (or the real image $I$) is as Eq.~\eqref{eqn:discriminator2}, which guides the model to discriminate real/fake images,
\begin{equation}
	\begin{aligned}
\mathcal{L}_{\mathrm{CGAN}}(G_i, G_s, D)  & = (\lambda + 1) \mathbb{E}_{S, I} \left[ \log D(S, I) \right]  \\
& +  \mathbb{E}_{S, I'} \left[\log (1 - D(S, I')) \right] \\
& +  \lambda \mathbb{E}_{S, I''} \left[\log (1 - D(S, I'')) \right],
\end{aligned}
\label{eqn:discriminator2}
\end{equation}
where $\lambda$ controls the losses of the two generated images. 
The inclusion of $I'$ and $I''$ is a cascaded coarse-to-fine generation strategy \cite{tang2019multi}, i.e., $I'$ is the coarse result, while $I''$ is the refined result. 
The intuition is that $I''$ will be better generated based on $I'$, so we provide $I'$ to the discriminator to ensure that $I'$ is also realistic. 

% Figure environment removed



\noindent \textbf{Optimization Objective.}
Equipped with the multi-modality discriminator, we elaborate on the training objective for the proposed method as follows.
Five different losses, i.e., the multi-modality adversarial loss, the similarity loss, the contrastive learning loss, the discriminator feature matching loss $\mathcal{L}_{f}$, and the perceptual loss $\mathcal{L}_{p}$ are used to optimize the proposed ECGAN,
\begin{equation}
\begin{aligned}
\min_{G} \max_{D} \mathcal{L} & = \lambda_{c} \underbrace{(\mathcal{L}_{\mathrm{CGAN}}(G_e, D) 
+ \mathcal{L}_{\mathrm{CGAN}}(G_i, G_s, D))}_{\text{Multi-Modality Adversarial Loss}} \\
&  + \lambda_{s} \underbrace{\mathcal{L}_{sim}(S, S')  + \mathcal{L}_{sim}(S, S'')}_{\text{Similarity Loss}} \\
& +  \lambda_{l} \underbrace{\mathcal{L}_{i}^{ms} + \mathcal{L}_{i}^{cs} + \mathcal{L}_{L_1}}_{\text{Contrastive Learning Loss}}  \\
& + \lambda_{f}\underbrace{(\mathcal{L}_{f}(I_e, I'_e) {+} \mathcal{L}_{f}(I, I') {+} \lambda \mathcal{L}_{f}(I, I''))}_{\text{Discriminator  Feature Matching Loss}} \\
& + \lambda_{p} \underbrace{(\mathcal{L}_{p}(I_e, I'_e) {+} \mathcal{L}_{p}(I, I') {+} \lambda \mathcal{L}_{p}(I, I''))}_{\text{Perceptual Loss}},
\label{eq:loss} 
\end{aligned}
\end{equation}
where $\lambda_{c}$, $\lambda_{s}$, $\lambda_{l}$, $\lambda_{f}$, and $\lambda_{p}$ are the parameters of the corresponding loss that contributes to the total loss $\mathcal{L}$;
where $\mathcal{L}_{f}$ matches the discriminator intermediate features between the generated images/edges and the real images/edges; where $\mathcal{L}_{p}$ matches the VGG extracted features between the generated images/edges and the real images/edges.
By maximizing the discriminator loss, the generator is promoted to simultaneously  generate reasonable edge maps that can capture the local-aware structure information and generate realistic images semantically aligned with the input labels.




\subsection{Implementation Details}
\label{sm:Implementation}

For both the image generator $G_i$ and edge generator $G_e$, the kernel size and padding size of convolution layers are all $3 {\times} 3$ and 1 for preserving the feature map size.
We 	set $n{=}3$ for  generators $G_i$, $G_s$, and $G_t$.
The channel size of feature $F$ is set to $C{=}64$. 
For the semantic preserving module $G_s$, we adopt an adaptive average pooling operation.
Spectral normalization \cite{miyato2018spectral} is applied to all the layers in both the generator and discriminator.
Our method incorporates the use of the Canny edge detector \cite{canny1986computational} for the purpose of deriving edge maps essential to our training process. In the subsequent testing phase, our approach necessitates no supplemental data, maintaining the fairness of comparisons with other existing methods.

\begin{table*}[!t] \small
	\centering
	\caption{User study on Cityscapes, ADE20K, and COCO-Stuff. The numbers indicate the percentage of users who favor the results of the proposed ECGAN over the competing methods.}
%		\resizebox{1\linewidth}{!}{% 
	\begin{tabular}{lccc} \toprule
		AMT $\uparrow$                               & Cityscapes & ADE20K  &  COCO-Stuff \\ \midrule
		Our ECGAN vs. CRN~\cite{chen2017photographic}     & 88.8 {$\pm$ 3.4}      & 94.8 {$\pm$ 2.7} & 95.3 {$\pm$ 2.1}\\
		Our ECGAN vs. Pix2pixHD~\cite{wang2018high}        & 87.2 {$\pm$ 2.9}       & 93.6 {$\pm$ 3.1}  &  93.9 {$\pm$ 2.4} \\ 
		Our ECGAN vs. SIMS~\cite{qi2018semi}                      & 85.3 {$\pm$ 3.8}       & -     & - \\
		Our ECGAN vs. GauGAN~\cite{park2019semantic}    & 84.7 {$\pm$ 4.3}       & 88.4 {$\pm$ 3.7}  &  90.8 {$\pm$ 2.5} \\ 
		Our ECGAN vs. DAGAN~\cite{tang2020dual}             & 81.8 {$\pm$ 3.9}       & 86.2 {$\pm$ 3.6}  & -\\
		Our ECGAN vs. CC-FPSE~\cite{liu2019learning}         & 79.5 {$\pm$ 4.2}       & 85.1 {$\pm$ 3.9}  &  86.7 {$\pm$ 2.8} \\  
		Our ECGAN vs. LGGAN \cite{tang2020local}              & 78.4 {$\pm$ 4.7}       & 82.7 {$\pm$ 4.5} & - \\
		Our ECGAN vs. OASIS \cite{sushko2020you}             & 76.7 {$\pm$ 4.8}        & 80.6 {$\pm$ 4.5}  & 82.5 {$\pm$ 3.1}  \\	\bottomrule
	\end{tabular}
	\label{tab:atm1}
		\vspace{-0.2cm}
\end{table*}

\begin{table*}[!t] \small
	\centering
	\caption{User study on Cityscapes, ADE20K, and COCO-Stuff. The numbers indicate the percentage of users who favor the results of the proposed ECGAN++ over the proposed ECGAN.}
%		\resizebox{1\linewidth}{!}{% 
	\begin{tabular}{lccc} \toprule
		AMT $\uparrow$                               & Cityscapes & ADE20K  &  COCO-Stuff \\ \midrule
		Our ECGAN++ vs. Our ECGAN \cite{tang2023edge}           & 64.3 {$\pm$ 3.2}        & 67.5 {$\pm$ 3.8}  & 70.4 {$\pm$ 2.6}  \\	\bottomrule
	\end{tabular}
	\label{tab:atm2}
		\vspace{-0.2cm}
\end{table*}

\begin{table*}[!t] \small
	\centering
	\caption{Quantitative comparison of different methods on Cityscapes, ADE20K, and COCO-Stuff.
	}
	\resizebox{1\linewidth}{!}{% 
	\begin{tabular}{rlllllllll} \toprule
		\multirow{2}{*}{Method}  & \multicolumn{3}{c}{Cityscapes} & \multicolumn{3}{c}{ADE20K} & \multicolumn{3}{c}{COCO-Stuff} \\ \cmidrule(lr){2-4} \cmidrule(lr){5-7} \cmidrule(lr){8-10} 
		& mIoU $\uparrow$    & Acc $\uparrow$  & FID  $\downarrow$ & mIoU $\uparrow$    & Acc $\uparrow$  & FID  $\downarrow$  & mIoU $\uparrow$    & Acc $\uparrow$  & FID  $\downarrow$ \\ \midrule
		CRN~\cite{chen2017photographic}  & 52.4  & 77.1 & 104.7  & 22.4 & 68.8 & 73.3 & 23.7 & 40.4 & 70.4\\
		SIMS~\cite{qi2018semi}           & 47.2  & 75.5 & 49.7 & - & - & - & - & - & - \\
		Pix2pixHD~\cite{wang2018high}    & 58.3  & 81.4 & 95.0  & 20.3 & 69.2 & 81.8  & 14.6 & 45.8 & 111.5  \\ 
  		GauGAN~\cite{park2019semantic}   & 62.3  & 81.9 & 71.8  & 38.5 & 79.9 & 33.9 & 37.4 & 67.9 & 22.6\\
         DPGAN \cite{tang2021layout} & 65.2 & 82.6 & 53.0 & 39.2 & 80.4 & 31.7 & - & - & - \\
		DAGAN \cite{tang2020dual} & 66.1 & 82.6 & 60.3 &   40.5 &  81.6 & 31.9 & - & - & -\\
            SelectionGAN \cite{tang2019multi} & 83.8 & 82.4 & 65.2 & 40.1 & 81.2 & 33.1 & - & - & -\\
            SelectionGAN++ \cite{tang2022multi} & 64.5 & 82.7 & 63.4 & 41.7 & 81.5 & 32.2 & - & - & - \\
		LGGAN \cite{tang2020local} & 68.4 & 83.0 & 57.7 & 41.6 & 81.8 & 31.6 & - & - & -\\
            LGGAN++ \cite{tang2022local}  & 67.7 & 82.9 & 48.1 & 41.4 & 81.5 & 30.5 & - & - & - \\
		CC-FPSE~\cite{liu2019learning}  & 65.5 & 82.3 & 54.3 & 43.7 & 82.9 & 31.7 & 41.6 & 70.7 & 19.2 \\
            \hao{SCG \cite{wang2021image}} & \hao{66.9} & \hao{82.5} & \hao{49.5} & \hao{45.2} & \hao{83.8} & \hao{29.3} & \hao{42.0} & \hao{72.0} & \hao{18.1}\\
		OASIS \cite{sushko2020you} &   69.3 & - & 47.7 & 48.8 & - & 28.3 & 44.1 & - & 17.0  \\
		\hao{RESAIL \cite{shi2022retrieval}} & \hao{69.7} & \hao{83.2}
		& \hao{45.5} & \hao{49.3} & \hao{84.8} &\hao{30.2} & \hao{44.7} & \hao{73.1} & \hao{18.3} \\
		\hao{SAFM \cite{lv2022semantic}} &\hao{70.4} & \hao{83.1} &\hao{49.5} &\hao{50.1}&\textbf{\hao{86.6}}&\hao{32.8}& \hao{43.3} & \textbf{\hao{73.4}} & \hao{24.6} \\
            \hao{PITI \cite{wang2022pretraining}} & \hao{-} & \hao{-} & \hao{-} & \hao{-} & \hao{-} & \hao{-} & \hao{-} & \hao{-} & \hao{19.36}\\
            \hao{T2I-Adapter \cite{mou2023t2i}} & \hao{-} & \hao{-} & \hao{-} & \hao{-} & \hao{-} & \hao{-} & \hao{-} & \hao{-} & \hao{16.78}\\
            \hao{SDM \cite{wang2022semantic}} & \hao{-} & \hao{-} & \hao{\textbf{42.1}} & \hao{-} & \hao{-} & \hao{27.5} & \hao{-} & \hao{-} & \hao{15.9}\\
		ECGAN (Ours)                        & 72.2    & 83.1 & 44.5 & 50.6 & 83.1 & 25.8 & 46.3 & 70.5 & 15.7 \\
		ECGAN++ (Ours) & \textbf{73.3} (+1.1) & \textbf{83.9} (+0.8) & 42.2 (-2.3) & \textbf{52.7} (+2.1) & 85.9 (+2.8) & \textbf{24.7} (-1.1) & \textbf{47.9} (+1.6) & 72.3 (+1.8) & \textbf{14.9} (-0.8) \\
		\bottomrule
	\end{tabular}}
	\label{tab:sota}
	\vspace{-0.4cm}
\end{table*}

In our computation of the contrastive learning loss, we observe a direct correlation between the number of layouts used and the resultant performance, i.e., more layouts lead to enhanced performance. However, a plateau is reached when the count exceeds 8 layouts; additional layouts contribute only marginal improvements to performance, while significantly slowing down the overall training process. Thus, with the objective of striking a balance between performance efficiency and computational time, we elect to use 8 layouts as input for the calculation of contrastive learning loss.
We use features from four scales in Eq. \eqref{eq:contrastive_multi}, with feature map output strides of 1, 4, 8, and 16, to calculate the multi-scale contrastive learning loss. 
Meanwhile, we also downsample the input layout by 4, 8, and 16 times to obtain the label of the corresponding scale for calculating the multi-scale contrastive learning loss.
The weights $w_s$ in Eq. \eqref{eq:contrastive_multi} are set to  1, 0.7, 0.4, and 0.1 in a decreasing way for feature maps of strides 1, 4, 8, and 16, respectively.
Moreover, in order to balance the performance and efficiency, we adopt two cross-scale contrastive learning in Eq. \eqref{eq:contrastive_cross}, i.e., (s4, s8) and (s4, s16).
We set both weights in Eq. \eqref{eq:contrastive_cross} to 0.1.

Also, we follow the training procedures of GANs \cite{goodfellow2014generative} and alternatively train the generator $G$ and discriminator $D$, i.e., one gradient descent step on the discriminator and generator alternately. 
We use the Adam solver \cite{kingma2014adam} and set $\beta_1{=}0$, $\beta_2{=}0.999$.
$\lambda_{c}$,  $\lambda_{s}$, $\lambda_{l}$, $\lambda_{f}$, and $\lambda_{p}$ in Eq.~\eqref{eq:loss} is set to 1, 1, 1, 10, and 10, respectively.
All $\lambda$ in both Eq.~\eqref{eqn:discriminator2} and \eqref{eq:loss} are set to 2.
We conduct experiments on an NVIDIA DGX1 with 8 V100 GPUs. 



% Figure environment removed


% Figure environment removed

% Figure environment removed

\section{Findings and results}
In this section, we analyze the geometric structure of DMs with our method.
% In \sref{sec:local}, we demonstrate the potential of the local latent basis for real image editing.
\sref{sec:local} demonstrates that the latent basis found by our method can be used for image editing.
% In \sref{sec:local}, we validate our method by demonstrating the potential of the latent basis for real image editing.
In \sref{sec:evolution}, we investigate how the geometric structure of DMs evolves as the generative process progresses. 
% \textcircled{\raisebox{-0.9pt}{1}} The frequency domain of local latent basis changes from coarse to fine feature as the generative process progresses. % (\sref{sec:evolution-x})
% \textcircled{\raisebox{-0.9pt}{2}} The differences of local tangent spaces from various samples getting larger as the generative process progresses. % and show the possibility of parallel transport on a local basis. (\sref{sec:evolution-samples})
% \textcircled{\raisebox{-0.9pt}{3}} The differences of local tangent spaces from various timesteps depends on the complexity of the dataset.
%Lastly, in \sref{sec:text}, we discover that the geometric structure the geometric properties of the text-condition model change with a given text. % is provided.
Lastly, in \sref{sec:text}, we examine how the geometric properties of the text-condition model change with a given text. % is provided.



{The implementation details of our work are provided on \aref{appendix:implementation_detail}}. The source code for our experiments is included in the supplementary materials and will be publicly available upon publication.

%%% Move to Appendix
% \paragraph{Implementation details}
% We validated our method and provided analyses on several datasets, including ImageNet~\cite{deng2009imagenet}, LSUN-church/bedroom/cat/horse~\cite{yu2015lsun}, and CelebA-HQ~\cite{karras2018progressive} for DDPM~\cite{ho2020denoising}; and FFHQ~\cite{karras2019style}, Flowers~\cite{yu2015lsun} and AFHQ~\cite{choi2020stargan} for DDPM trained with P2 weighting~\cite{choi2022perception}. We also use Stable Diffusion (SD) version 2.1~\cite{rombach2022high} for the text-conditional diffusion model.
% For image editing, we use the official codes and pre-trained checkpoints for all baselines and keep the parameters \textit{frozen}. 
% The source code for our experiments is included in the supplementary materials and will be publicly available upon publication.

% \yh{
% Thorough experiments demonstrate the usefulness of our method in various aspects. Specifically, our approach identifies a latent basis in \exspace{}, which encompasses latent changes and exhibits a coarse-to-fine behavior, as detailed in the {\it section}. 
% Moreover, utilization of our method to Stable Diffusion allows us to investigate the effects of prompts on the local latent subspace. Our findings reveal that the impact of prompts is most pronounced at $0.7T$ and diminishes as the generative process progresses.
% }
% The latent basis in \exspace{} found by our method include latent changes and exhibit coarse-to-fine behavior {\it section}. 
% By applying our method to Stable Diffusion, we explore the impact of prompts on the local latent subspace and find that the magnitude of its impact are most significant at 0.7T and decrease as generative process progresses.
% By applying our method to Stable Diffusion, we  the latent subspace has positive correlation with the latent of \exspace{} is a spherically curved space {\it section}. Our method generalizes to stable diffusion {\it section}. Both the finding directions and the editing equation contribute to the nice properties of our method {\it section}.

%%% ICML ver
% Thorough experiments demonstrate the usefulness of our method in various aspects.
% The editing latent directions in \exspace{} found by our method include latent changes and exhibit coarse-to-fine behavior (\sref{sec:local}). \exspace{} is a spherically curved space (\sref{sec:slerp}). Our method generalizes to stable diffusion (\sref{sec:stable}). Both the finding directions and the editing equation contribute to the nice properties of our method (\sref{sec:ablation}). Our method outperforms the existing methods (\sref{sec:comparison}).

% We validate our method and provide analyzes in 
% Stable Diffusion version 2.1 ~\cite{rombach2022high},
% ImageNet, \cite{deng2009imagenet} LSUN-church, LSUN-bedroom, LSUN-cat, LSUN-horse \cite{yu2015lsun} CelebA-HQ \cite{karras2018progressive} for DDPM \cite{ho2020denoising},
% FFHQ, Flower \cite{yu2015lsun} DDPM trained with P2 weighting \cite{choi2022perception}.
% For image editing results, we only present results for LSUN-church, CelebA-HQ, and Stable Diffusion In the main paper. For more results from other datasets, please refer to the Appendix. 
% For the quantitative results, to ensure a fair comparison, we use unconditional DMs trained at the same resolution ($256^2$) and using the same scheduling (linear scheduling).
% % To minimise the impact of prompts, except when manipulating the latent variable $\vx{}_t$, we did not use any text conditions during the inversion and image generation process using DDIM. 
% For every experiments, we use the official codes and pre-trained checkpoints for all baselines and keep the parameters \textit{frozen}. Further implementation details are deferred to {\it section}.
% The source code for our experiments is included in the supplementary materials, and will be publicly available upon publication.
% We validated our method and provided analyses on several datasets, including ImageNet~\cite{deng2009imagenet}, LSUN-church/bedroom/cat/horse~\cite{yu2015lsun}, and CelebA-HQ~\cite{karras2018progressive} for DDPM~\cite{ho2020denoising}; and FFHQ~\cite{karras2019style}, Flowers~\cite{yu2015lsun} and AFHQ~\cite{choi2020stargan} for DDPM trained with P2 weighting~\cite{choi2022perception}. We also use Stable Diffusion (SD) version 2.1~\cite{rombach2022high} for the text-conditional diffusion model.
% For image editing, we use the official codes and pre-trained checkpoints for all baselines and keep the parameters \textit{frozen}. 
% \mingi{In the main paper, we present image editing results only for LSUN-church, CelebA-HQ, and SD.} For more results from other datasets, please see the Appendix.
% For the qualitative results, w
% We used unconditional DMs trained at the same resolution ($256^2$) and diffusion schedule (linear scheduling) to ensure a fair comparison. 
% For the latent variable $\vx_T$, we randomly sample from normal distribution $\mathcal{N}(0, \textbf{I})$. 
% The source code for our experiments is included in the supplementary materials and will be publicly available upon publication.



\subsection{Image editing with the latent basis}
% \subsection{\jo{latent-based editing with and without prompts}}
\label{sec:local}
% \yh{TODO : local figure 완성되고 나서 작성}

% In \sref{sec:sec3}, we introduce a method for determining local latent basis utilizing the pullback metric. 
In this subsection, we demonstrate the capability of our discovered latent basis for image editing. 
To extract the latent variables from real images for editing purposes, we use DDIM inversion.
In experiments with Stable Diffusion (SD), we do not use guidance, i.e., unconditional sampling, for both DDIM inversion and DDIM sampling processes.
This ensures that our editing results solely depend on the latent variable, not on other factors such as prompt conditions.
% This ensures that our editing result is solely the outcome of manipulating the latent variable.
% Consequently, we only utilize the text guide to extract the local latent structure and keep the generative process as unconditional sampling to minimize the influence of text.
% It is worth to note that we only use the text condition when we extract the local latent structure
% when given a text (refer to \fref{fig:text}). 

% \modify{add text conditional basis; cross attention이 특정 feature를 강조할텐데 그걸 pc를 구해도 다 텍스트에 관련된 basis만 나오는 것을 기대해야 하고, 우리가 관측한것도 이런 기대에 부합한다. text cond 피규어는 여러 local basis중에 하나 고른거고 나머지는 appendix봐라.}

Figures~\ref{fig:method} and \ref{fig:local_basis} illustrate the example results edited by the latent basis found by our method.
% demonstrate the discovery of diverse basis across different datasets which highlights the superiority of our approach \emph{without supervision} such as CLIP or a classifier. 
The latent basis clearly contains semantics such as age, gender, species, structure, and texture. 
% When modulating the strength of the basis, the resulting image undergoes continuous changes, as exemplified in \fref{fig:local_basis}. 
{Note that editing at timestep $T$ yields coarse changes such as age and species. On the other hand, editing at timestep $0.6T$ leads to fine changes, such as nose color and facial expression.} 
% Appendix provides more examples.
% In the subsequent section, we delve into a detailed analysis of the latent basis from these distinct timesteps.

% Significantly, editing at timestep $T$ results in coarse changes, whereas editing at timestep $0.6T$ produces finer changes, as exemplified by various examples. 
% In the following section, we will thoroughly explore and analyze the local semantic foundation derived from these distinct timesteps.

\fref{fig:local_basis_text_vk} demonstrates the example results edited by the various latent basis vectors.
Interestingly, using the text ``lion'' as a condition, 
the entire latent basis {captures lion-related attributes.} % aligns with the text.
% Moreover, \fref{fig:local_basis_text} shows that the latent basis can be aligned with the text not only for the type of object but also the action.
Furthermore, \fref{fig:local_basis_text} shows that the latent basis aligns with the text not only in terms of object types but also in relation to pose or action.
\modify{For a qualitative comparison with other state-of-the-art image editing methods, refer to 
\aref{appendix:comparisons}. For more examples of editing results, refer to \aref{appendix:additional_results}.}

% \fref{fig:local_basis_text_vk} demonstrates the impact on the basis of the tangent space when considering a given text. 
% One can expect the obtained basis from \ehspace{} becomes influenced by the text condition. 
% Consequently, anticipations align with our observations. 
% \modify{In the case of unconditional conditions}, the basis exhibits distinct characteristics. However, upon using the text ``lion" as a condition, noticeable lion-related modifications can be observed across all basis. This phenomenon enables editing within the semantic subspace that corresponds to the intended text, as depicted in \fref{fig:local_basis_text}. 
% We selected one of the bases, yet each basis consistently yielded results associated with the given text. It is intriguing that the capability for editing extends even to actions, such as running, even though we just added the basis only one time. Please refer to the appendix for other bases.
% In this subsection, we demonstrate our discovered latent basis through real image editing. 
% By adding the latent basis vector by utilizing the x-space guidance, we were able to apply semantic manipulations to the image.
% In Stable Diffusion, the latent basis manipulates the image in a way that is semantically aligned with the given prompt.

% \paragraph{Unconditioanl DMs}
% \yh{
% \fref{fig:local_basis} illustrates the example results edited by the directions found by our method \emph{without supervision} such as CLIP or a classifier. 
% The directions clearly contain semantics such as gender, age, ethnicity, facial expression, breed, and texture. 
% Interestingly, editing at timestep $T$ leads to coarse changes such as hair color, hair length, far breed. On the other hand, editing at the timestep $0.5T$ leads to fine changes such as make-up, hair texture, wrinkles, facial expression, and close breed. \aref{appendix:local} provides more examples.
% }\yh{
% \fref{fig:global_basis_main} demonstrates that the global directions in $\vx_t$ lead to the same semantic changes, such as rotation, age, furriness, or color in different samples. 
% It confirms that \exspace{} inherits the homogeneity of \ehspace{} via the pullback metric although \exspace{} is a metric-less space. \aref{appendix:global} provides more examples.
% }\yh{
% \paragraph{Stable Diffusion} 
% This section demonstrates that our method is generalized to Stable Diffusion \cite{rombach2022high}. Our method extracts latent directions in the learned latent space $\vz_t$ using the same procedure. \fref{fig:ldm} (a) shows the edited images along different directions on various timesteps. 
% The phenomena are similar to the image-based DMs: 
% editing at $t=T$ provides coarse changes, and editing at later timesteps provides more fine texture-ish changes such as cartoonization.
% }

% Furthermore, \fref{fig:ldm} (b) 는 prompt 가 주어지는 경우 local latent basis 를 활용한 결과를 보여주고 있다.  
% 예를 들어, jumping dog 

%%% ICML ver
% Furthermore, \fref{fig:ldm} (b) shows that a global semantic direction leads to the same \textit{zoom-out} effect on different samples. Contrary to the global directions in the image-based DMs, the global directions are found within a text prompt, i.e., each text prompt has its own global directions. \aref{appendix:additional_results} provides more examples where we find some odd cases indicating that the learned latent space may not follow the same assumptions of the image-based DMs or the text guidance somehow twists the manifold.

%%% ICML ver
% \fref{fig:global_btasis_main} demonstrates that the global directions in $\vx_t$ lead to the same semantic changes, such as roation, age, furriness, or color in different samples. It confirms that \exspace{} inherits the homogeneity of \ehspace{} via the pullback metric although \exspace{} is a metric-less space. \aref{appendix:global} provides more examples.



%\subsection{Evolution of Semantic Structure \yh{of DMs} over Timesteps.}
\subsection{{Evolution of latent structures during {generative} processes}}
\label{sec:evolution}


\begin{wrapfigure}{r}{6cm}
    \vspace{-1em}
% Figure removed
    \vspace{-1.5em}
    \caption{\textbf{{Power Spectral Density (PSD) of latent basis.}} {The PSD at $t = T$ (purple) exhibits a greater proportion of low-frequency signals, while the PSD at smaller $t$ (beige) reveals a larger proportion of high-frequency signals. The latent vectors $\vv_i$ are min-max normalized for visual clarity.}
    %\textbf{Power Spectral Density of latent basis}
    %The PSD at $t = T$ (purple line) shows a larger portion of
    %low-frequency signals, whereas the PSD at smaller $t$ (beige line) shows a larger portion of high-frequency signals. The latent vector $\vv_i$ are min-max normalized for visual purposes.
    }\label{figure:PSD}
    \vspace{-2em}
\end{wrapfigure}


In this subsection, we demonstrate how the latent structure evolves during the generative process and identify three trends. 
%\textcircled{\raisebox{-0.9pt}{1}} 
1) {The frequency domain of the latent basis changes from low to high frequency. It agrees with the previous observation that DMs generate samples in coarse-to-fine manner.} % The frequency domain of latent basis changes from coarse to fine feature as the generative process progresses. % (\sref{sec:evolution-x})
%\textcircled{\raisebox{-0.9pt}{2}} 
2) The difference between the tangent spaces of different samples increases over the generative process. It implies finding generally applicable editing direction in latent space becomes harder in later timesteps.
 % and show the possibility of parallel transport on a local basis. (\sref{sec:evolution-samples})
%\textcircled{\raisebox{-0.9pt}{3}} 
3) The differences of tangent spaces {between} timesteps depend on the complexity of the dataset.

% We examine the frequency domain of the local latent basis from different timesteps using the power spectral density (PSD), and find that DMs become more attuned to finer features as the process progresses. 
% We also investigate variations in the local tangent space using the geodesic metric, showing that simpler data leads to greater similarity in the local tangent subspace at different timesteps. This suggests that simpler data leads to increasingly similar features processed at each timestep.
% 이 관찰으로부터 서로 다른 timestep 유사한 signal 을 처리하지 않도록 timestep 을 배치하니 uniform timestep 에 비해 더 좋은 sampling fidelity 를 얻을 수 있었다. 




\paragraph{{Latent bases gradually evolve from low- to high-frequency structures.}}
%\paragraph{latent basis evolves from low frequencies to high frequencies.}
% \paragraph{The frequency domain of latent basis evolves from low-to-high.}
\label{sec:evolution-x}



% In \sref{sec:local}, we observed that coarse manipulations are performed at the beginning of the denoising step, and then finer manipulations are performed as the timestep progresses. 
% To quantitatively verify this observation, we plotted the power spectral density (PSD) of the discovered latent basis. 

\fref{figure:PSD} is the power spectral density (PSD) of the discovered {latent} basis over various timesteps. 
The early timesteps contain a larger portion of low frequency than the later timesteps and the later timesteps contain a larger portion of high frequency.

This suggests that the model focuses on low-frequency signals at the beginning of the generative process and then shifts its {focus} to high-frequency signals over time.
This result strengthens the common understanding
about the coarse-to-fine behavior of DMs over the generative process \cite{choi2022perception, daras2022multiresolution}.



% \yh{This result strengthens the common understanding that DMs start to create coarse features and gradually refine them over a generative process.}
% This suggests that the model focuses on low-frequency signals at the beginning of the generative process and then shifts its attention to high-frequency signals over time.
% long version
% \yh{
% In the forward diffusion process, high-frequency details are perturbed faster than low-frequency signals. Therefore, as the diffusion timestep approaches $t \approx T$, only the low-frequency signals remain in the image. Hence, it is reasonable for a model to reconstruct the original image by focusing on the remaining low-frequency signals.
% }
% short version
% \yh{
% Considering high-frequency perturbed faster in forward diffusion process, it is natural for model to focus on low-frequency signal when $t \approx T$.
% }
% The PSD plots for additional datasets are provided in \aref{appendix:PSD}. Note that each dataset demonstrates a consistent qualitative trend.

% \begin{wrapfigure}{R}{5cm}
% \vspace{-2em}
% % Figure removed
% \vspace{-2em}
% \caption{
% \textbf{Geodesic distance across tangent space of different samples in various diffusion timesteps}
% Each point represents the average geodesic distance between pairs of 15 samples. 
% The tangent spaces of different samples are only similar early in the generative process.
% }
% \label{homogenity-sample}
% % \vspace{-2em}
% \label{fig:homo}
% \end{wrapfigure}



\paragraph{{The discrepancy of tangent spaces from different samples increases along the generative process.}}
%\paragraph{The difference between the tangent spaces from various samples increases as the generative process progress.}
\label{sec:evolution-samples}

% In our previous findings, we investigated the evolution of the latent basis. The next step is to focus on the evolution of the basis's corresponding {\it ``representation''}, i.e. tangent space. % original 아래
% From now on, we are going to turn our attention to the evolution of the  corresponding {\it ``representation''}, i.e. tangent space.



% In our previous findings, we established that the signal frequency in DM undergoes changes depending on the timestep. In order to gain further insights, we will now examine the representation of the local basis. 

% It is important to emphasize that the local basis is extracted from each local tangent space. This allows us to utilize the geodesic metric to assess the similarity between the discovered local subspaces across various samples, considering the local tangent subspaces generated at different timesteps.
%As a starting point, we checked how similar the discovered local subspaces of the different samples are to each other. 
% To begin, we examined the similarity between the discovered local subspaces across different samples.
% As a starting point, we checked how similar the local tangent spaces $\mathcal{T}_{\mathbf{h}}$ of the different samples are to each other. 
% The distortion of different vector spaces can be measured by the Grassman metric. 

% \begin{wrapfigure}{r}{3.2cm}
% % \vspace{-3em}
% % Figure removed
% \vspace{-1em}
% \caption{Conceptual illustration of geodesic metric}\label{fig:geodesic_metric}
% \vspace{-1.25em}
% \label{fig:homo}
% \end{wrapfigure}


To investigate the geometry of the tangent basis, we employ a metric on the Grassmannian manifold. The Grassmannian manifold is a manifold where each point is a vector space, and the metric defined above represents the distortion across various vector spaces.
% By employing a metric on the Grassmannian manifold, one can determine the extent of distortion across various vector spaces.
We use \emph{geodesic metric}~\cite{choi2021not, ye2016schubert} to define the discrepancy between two subspaces $\{\mathcal{T}^{(1)}, \mathcal{T}^{(2)}\}$: 

% The distortion of different vector spaces can be found by the Grassman metric. We use \emph{geodesic metric}~\cite{choi2021not, ye2016schubert} to define the distortion between two local latent subspace centered at $\{\vh^{(1)}, \vh^{(2)}\}$: 

% % Figure environment removed

%%% Small Geodesic metric figure
% % Figure environment removed

\begin{equation}
D_{\text{geo}}(\mathcal{T}^{(1)}, \mathcal{T}^{(2)}) = \sqrt{\sum_k \theta_k^2},
\end{equation}


\begin{wrapfigure}{r}{6cm}
% Figure removed
    \vspace{-1em}
    \caption{
    \textbf{Geodesic distance across tangent space of different samples {at} various diffusion timesteps.}
    Each point represents the average geodesic distance between pairs of 15 samples.
    { It is notable that the similarity of tangent spaces among different samples diminishes as the generative process progresses.}
    %The tangent spaces of different samples are only similar early in the generative process.
    }
    \label{homogenity-sample}
\vspace{-3em}
\label{fig:homo}
\end{wrapfigure}


% Figure environment removed



where {$\theta_k$} denotes the $k$-th principle angle between $\mathcal{T}^{(1)}$ and $\mathcal{T}^{(2)}$. Intuitively, the concept of geodesic metric can be understood as an angle between two vector spaces.
%(\ifref{fig:geodesic_metric})
{Here, the comparison between two different spaces was conducted for 
$\{\mathcal{T}_{\vh_1}, \mathcal{T}_{\vh_2} \}$. 
Unlike the \exspace{}, the \ehspace{} assumes a Euclidean space which makes the computation of geodesic metric that requires an inner product between tangent spaces easier. 
The relationship between tangent space and latent subspace is covered in more detail in \aref{appendixsec:relationship_tangent_latent}.
}


% \begin{wrapfigure}{r}{6cm}
%     \vspace{-1em}
% % Figure removed
%     \vspace{-1.5em}
%     \caption{\textbf{{Power Spectral Density (PSD) of latent basis.}} {The PSD at $t = T$ (purple) exhibits a greater proportion of low-frequency signals, while the PSD at smaller $t$ (beige) reveals a larger proportion of high-frequency signals. The latent vectors $\vv_i$ are min-max normalized for visual clarity.}
%     %\textbf{Power Spectral Density of latent basis}
%     %The PSD at $t = T$ (purple line) shows a larger portion of
%     %low-frequency signals, whereas the PSD at smaller $t$ (beige line) shows a larger portion of high-frequency signals. The latent vector $\vv_i$ are min-max normalized for visual purposes.
%     }\label{figure:PSD}
%     \vspace{+2em}
% \end{wrapfigure}



\fref{fig:homo} demonstrates that the tangent space\uh{s} of the different samples are the most similar at $t=T$ and diverge as {timestep becomes} zero. 
% This aligns with the existing understanding that, after a certain timestep, the reverse diffusion trajectories of different samples are isolated from each other\cite{zhang2022gddim, wu2022uncovering}.


Moreover, the similarity across tangent space{s} allows us to effectively transfer the latent basis from one {sample} to another through parallel transport as shown in \fref{fig:parallel}.
% as showcased in \sref{sec:parallel}. 
% \fref{fig:parallel} demonstrates the edited results of parallel transporting the latent basis obtained from one sample to another sample.
% In $T$, where homogenity of tangent space exists, 
In $T$, {where the tangent spaces are homogeneous},
we consistently obtain semantically aligned editing results. 
{On the other hand, parallel transport at $t=0.6T$ does not lead to satisfactory editing because the tangent spaces are hardly homogeneous. Thus, we should examine the similarity of local subspaces to ensure consistent editing across samples.}
% However, when attempting parallel transport at $t=0.6T$, where homogeneity is lacking, it fails to produce satisfactory outcomes.
% This underscores the significance of considering the similarity of local subspaces.



% This is corroborated by the results depicted in \fref{fig:parallel}. 
% By transferring the local semantic direction from the first sample to subsequent samples, we consistently obtain semantically aligned editing results. However, it is important to note that when attempting to transfer the local basis at $t=0.6T$, where homogeneity is lacking, the parallel transport method fails to produce satisfactory outcomes. This underscores the significance of considering the similarity of local subspaces.

% \paragraph{\jo{Simpler datasets exhibit more consistent tangent spaces over timesteps.}}
\paragraph{DMs trained on simpler datasets exhibit more consistent tangent spaces over time.} 
% \paragraph{DMs trained on simple datasets have similar tangent spaces across different timesteps.} 
% \paragraph{Tangent basis evolves less if trained on simple dataset}
\label{sec:evolution-t}
% \modify{frequency는 다른데, 과연 그 representation도 많이 다를까? -> tangent space에서의 유사함은 representation의 유사함을 말한다. 이런 직관 설명 내용 추가.}
% In the previous section, we studied the evolution of the tangent space by comparing it across different samples. This time, we compare tangent space across different timesteps.
% In the previous section, we studied the evolution of the tangent space by comparing different samples. This time, we investigate how the tangent space change by comparing it across different timesteps.

% This suggests that the local latent basis that the model pays attention to at each point in time is similar. To help readers understand this, we illustrate in Figure 1b how much of the original feature captured by vector vi is lost when we transport it from one timestep to another, using parallel transport from t = T to 0.1T. As expected, when the local tangent space is similar, the transported vector maintains its existing signal, but it loses this pattern as we move further away from the original timestep. We found this relationship between local tangent space and local semantic subspace to hold true for all datasets. For a more detailed discussion, please refer to the appendix.

In \fref{fig:timestep} (a), we provide a distance matrix of the tangent spaces across different timesteps, measured by the geodesic metric. 
We observe that the tangent spaces are more similar to each other when a model is trained on CelebA-HQ, compared to ImageNet.
\uh{To verify this trend, we measure the geodesic distances between tangent spaces of different timesteps and plot the average distances of the same difference in timestep in \fref{fig:timestep} (b)}.
% To verify this trend, in \fref{fig:timestep} (c), we measured the average geodesic distance of the tangent space according to the difference of diffusion timesteps.
As expected, we find that DMs trained on datasets, that are generally considered simpler, have similar local tangent spaces over time.

% 눈으로 봤을때, 우리는 CelebA-HQ 와 같이 비교적 단순한 dataset 에 대해 학습한 경우, ImageNet 과 같이 복잡한 dataset 을 학습했을때와 비교했을때 시간 별 local tangent space 가 유사한 것을 확인할 수 있었다.
% In \fref{fig:timestep} (c), 이를 검증하기 위해 우리는 여러 dataset 에서 학습한 DM 의 timestep 거리별 local tangent space 의 평균 거리를 측정했다. 
% 기대한대로, 일반적으로 더 단순하다고 여겨지는 데이터셋에 학습한 DM 일수록 (e.g. CelebA-HQ > LSUN > ImageNet) 시간별 local tangent space 가 유사함을 확인할 수 있었다. 


% 
% parallel transport 부분은 appendix 로 넘기기
% 
% Notice that the similarity between tangent spaces implies \uh{consistency of latent basis across timesteps}. %a consistent latent basis at each timestep.
% In \fref{fig:timestep} (b), we parallel transport the latent vector $\vv_i$ to various tangent spaces and visualize the outcomes.
% As expected, when the tangent spaces are similar, the transported vector ${\vv'}_i$ retains the original signal. On the other hand, as we move to more distant timesteps, where the tangent space is farther apart, ${\vv'}_i$ deviates from the original signal.

% When the tangent space is similar, the transported vector ${\vv'}_i$ retains the original signal. However, when moved to a distant regime, ${\vv'}_i$ deviates from the original signal.
% The relationship between tangent space and latent subspace is covered in more detail in \aref{appendixsec:relationship_tangent_latent}.

% mingi
% In \fref{fig:timestep} (b), 이렇게 local tangent space 간 거리가 작은 경우, 우리는 
% In \fref{fig:timestep} (b), the similarity of local tangent space is indirectly visualized through the local basis $v_1$. \modify{b는 tangent space에서의 변화가 semantic subspace에서는 어떤지 보여주는 것이라는 설명 추가.} To aid visualization, we perform parallel transport of the local basis at T and 0.1T to the local tangent space at different time steps. These visualizations also confirm the similarity of tangent spaces between time steps. \modify{이러한 경향성은 모든 샘플들에 대해 나타나며, 이는 appendix봐라 라고 말하고, 이전에 넣어두었던 피규어 appendix에 넣기.}

% mingi
% Furthermore, we have observed a correlation between the difference in tangent space at various time steps and the complexity of the dataset. \fref{fig:timestep} (c) illustrates the average geodesic distance of the tangent space according to the disparity in time steps.  \modify{서로 다른 타임스텝에서 얼마나 급격하게 변화하는지 보여주고 있따~~ 추가} Both \fref{fig:timestep} (a) and (c) demonstrate that in the case of ImageNet, which is considered a highly complex dataset, the tangent spaces across time steps exhibit less similarity compared to CelebA-HQ, which is deemed a less complex dataset.

% Simple dataset 에서는 timestep 별 semantic structure 의 homogenity 를 만들어낸다.
% The simpler the data, the more similar the Local Tangent Basis per timestep.

% % Figure environment removed


% Figure environment removed

% 앞에서 우리는 DM 이 민감하게 반응하는 signal 의 frequency 가 timestep 에 따라 변화함을 확인했다. 그렇다면, 그 representation 은 어떨까?
% 이를 확인하기 위해, 이미지를 생성해나가는 과정의 각 timestep 에서 만들어지는 local tangent subspace 간 geodesic metric 을 측정했다. 

%%% ICML ver
% We further investigate the coarse-to-fine editing along the generative process from timestep $T$ to $0$. \fref{fig:PSD} (a) shows the example directions $\vv_i$ across different timesteps. At $T$, $\vv_i$ leads to coarse attribute changes in $\vx_0$ by blurry change in $\vx_T$. At $0.25T$, $\vv_i$ edits high-frequency details in both $\vx_0$ and $\vx_t$. \fref{fig:PSD} (b) shows the power spectral density (PSD) of $\vv_i$. We compute the PSD by taking $\vv_1, ..., \vv_{10}$ from 20 samples. The early timesteps contain a larger portion of low frequency than the later timesteps and the later timesteps contain a larger portion of high frequency. This phenomenon agrees with the tendency in the edited images. This results strengthens the common understanding of the timesteps~\cite{kwon2022diffusion,choi2022perception, daras2022multiresolution}.

% Figure environment removed

%\subsection{How the Text-Condition Controls the Geometry of DMs?}
\subsection{\uh{Effect of conditioning prompts on the latent structure}}
% \subsection{\jo{Text-conditioned control of geometry in latent spaces}}
\label{sec:text}
% In this subsection, we aim to find how prompts controls the generative process in geometrical perspective. 
% To demonstrate this, we analyze local semantic subspace and tangent space from randomly sample 50 captions from the MS-coco dataset \cite{lin2014microsoft}. Here we used $\vx{} \sim \mathcal{N}(0, \textbf{I})$ for latent variable.
% To explore this, we randomly sample 50 captions from the MS-coco dataset \cite{lin2014microsoft} and analyse the local semantic subspace and local tangent space extracted for $\vx{} \sim \mathcal{N}(0, \textbf{I})$. 

% 이를 살펴보기 위해, MS-coco dataset \cite{lin2014microsoft} 로부터 50개의 caption 을 random sample 한 뒤, $\vx{} \sim \mathcal{N}(0, \textbf{I})$ 에 대해 추출된 local semantic subspace 와 local tangent space 를 분석했다. 
% 앞에서 말했듯이, local semantic subspace 는 모델이 주목하는 signal 에 대응하고, local tangent space 는 그 signal 들에 대응하는 semantic represenation 에 대응한다. 

In this subsection, we aim to investigate how prompts influence the generative process from a geometrical perspective. 
We randomly sampled 50 captions from the MS-COCO dataset \cite{lin2014microsoft} and used them as text conditions.
% For the latent variable, we utilize $\vx{} \sim \mathcal{N}(0, \textbf{I})$. 
% Our objective is to explore the impact of text-conditioning on the geometry of DMs.



\paragraph{Similar text conditions induce similar tangent spaces.}
% \fref{fig:stable_text} depicts the relationship between text and the tangent space. 
In \fref{fig:stable_text} (a), we observe a negative correlation between the CLIP similarity of texts and the distance between tangent spaces. 
In other words, when provided with similar texts, the tangent spaces are more similar. 
 % model have similar tangent spaces.
% This implies why manipulating the local latent basis vector according to the given text will result in a corresponding edit (\ifref{fig:local_basis_text_vk}, \ifref{fig:local_basis_text})
The linear relationship between the text and the discrepancy of the tangent spaces is particularly strong in the early phase of the generative process as shown by $R^2$ score in \fref{fig:stable_text} (b).
% In \fref{fig:stable_text} (b),
% the $R^2$ score indicates that the linear relationship between the text and local tangent space is particularly strong during the early stages of the generative process. 

% In this subsection, we aim to find how prompts controls the generative process in geometrical perspective. 
% To demonstrate this, we analyze local semantic subspace and tangent space from randomly sample 50 captions from the MS-coco dataset \cite{lin2014microsoft}. Here we used $\vx{} \sim \mathcal{N}(0, \textbf{I})$ for latent variable.
% To explore this, we randomly sample 50 captions from the MS-coco dataset \cite{lin2014microsoft} and analyse the local semantic subspace and local tangent space extracted for $\vx{} \sim \mathcal{N}(0, \textbf{I})$. 

\paragraph{\uh{The generative process depends less on text conditions in later timesteps.}}
% \paragraph{\jo{Text-conditioning effects diminish over generative processes.}}
%\paragraph{The effect of the text becomes weaker, as the generative process progresses.}
\fref{fig:stable_text} (c) illustrates the \uh{distances between} local tangent spaces for given different prompts with respect to the timesteps.
% prompt 에 따른 local tangent space 의 차이가 generative process 가 진행됨에 따라 어떻게 달라지는지를 보여주고 있다.
%%% mingi
% \fref{fig:stable_text} (c) provides the geodesic distance in the tangent space between all texts and the geodesic distance in the semantic subspace between all texts. 
Notably, as the diffusion timestep approaches values below $0.7T$, the distances between the local tangent spaces start to decrease. 
It implies that the variation due to walking along the local tangent basis depends less on the text conditions, i.e., the text less influences the generative process, in later timesteps.
% This signifies that with smaller timesteps, the variation in the local tangent space becomes less dependent on the text, implying a reduced influence of the text. 
% \mingi{
% Furthermore, we also discover that the geodesic distance in the semantic subspace is maximized at $0.7T$. This observation highlights the disparity in the basis at $0.7T$, thus implying why performing edits at around $0.7T$ is well.
% }
It is a possible reason why the correlation between the similarity of prompts and the similarity of tangent spaces reduces over timesteps.
% This explains one of the reasons why the linearity between prompts and tangent space reduces over timesteps.
% However, as the generative process progresses, the strength of this relationship weakens. 
% So why does the relationship weaken as the generative process progresses?
% As one reason for this, we observe that the influence of text on the generative process decreases over time.
% \modify{t가 클때는 prompt가 다르면 tangent space의 변화가 있는데, t가 작아지면 prompt가 다름에도 tangent space가 비슷해진다는 이야기를 직접적으로 해주기. 결론 프롬프트의 영향이 t가 작아지면 줄어든다.}


% In this subsection, we aim to find how prompts controls the generative process in geometrical perspective. 
% To demonstrate this, we analyze local semantic subspace and tangent space from randomly sample 50 captions from the MS-coco dataset \cite{lin2014microsoft}. Here we used $\vx{} \sim \mathcal{N}(0, \textbf{I})$ for latent variable.
% To explore this, we randomly sample 50 captions from the MS-coco dataset \cite{lin2014microsoft} and analyse the local semantic subspace and local tangent space extracted for $\vx{} \sim \mathcal{N}(0, \textbf{I})$. 



% % Figure environment removed


% % Figure environment removed

% \begin{table}[t]
% \caption{Semantic path length for lerp, slerp, and geodesic paths in CelebA-HQ for DDPM++.}
% \label{tab:semantic_path_length}
% \vskip 0.15in
% \begin{center}
% \begin{small}
% % \begin{sc}
% \begin{tabular}{lc}
% \toprule
% Path & Semantic Path Length $(\mu \pm \sigma)$ \\
% \midrule
% lerp     & 10.29 $\pm$ 1.11 \\
% slerp    & 7.69 $\pm$ 0.87 \\
% geodesic & 5.98 $\pm$ 0.76 \\
% \bottomrule
% \end{tabular}
% % \end{sc}
% \end{small}
% \end{center}
% \vskip -0.1in
% \end{table}

% % Figure environment removed

% \subsection{Curved manifold of DMs} % \subsection{Indepth analysis on the latent space of DMs} 
% \label{sec:slerp}
% We present empirical grounds for the assumption in \sref{sec:method_local}: $\mathcal{X}$ is a curved manifold. Semantic path length between two points on a manifold is defined by the sum of the local warpage of the line segments which connects them along the manifold. We use \emph{geodesic metric}~\cite{choi2021not, ye2016schubert} to define the curvedness of a line segment $\{\vx^{(1)}, \vx^{(2)}\}$ as the angle between two tangent spaces centered at $\{\vh^{(1)}, \vh^{(2)}\}$: 
% \begin{equation}
% D_{\text{geo}}(\mathcal{T}_{\vh^{(1)}}, \mathcal{T}_{\vh^{(2)}}) = \sqrt{\sum_k \theta_k^2},
% \end{equation}
% where $\theta_k = \cos^{-1}(\sigma_k)$ denotes the $k$-th principle angle between $\mathcal{T}_{\vh^{(1)}}$ and $\mathcal{T}_{\vh^{(2)}}$.
% The angle is visualized in \fref{fig:slerp_lerp} (b). Then, the semantic path length becomes $\sum_l D_\text{geo}(\mathcal{T}_{\vh^{(l)}}, \mathcal{T}_{\vh^{(l+1)}})$, where $l$ denotes the segment index in the path. We set the number of segments to $30$.
% Then, the semantic path length increases as the path deviates further from the manifold.



% To verify the assumption, we compare the semantic path lengths of different paths, e.g., linear path, spherical path, and geodesic shooting path. 
% \fref{fig:slerp_lerp} (a) visualizes the manifold, linear path (lerp), and spherical path (slerp) and their corresponding path on $\mathcal{H}$ mapped by the function $f$. We computed the semantic path lengths for 50 randomly selected pairs of images. \tref{tab:semantic_path_length} shows that the semantic path length of slerp is smaller than lerp, indicating that the slerp path lies closer to the manifold than lerp, i.e., the manifold is curved. \fref{fig:slerp_lerp} (b) shows the distribution of the length of the segments along the path. Interestingly, the length of the lerp is high at the ends and shrinks to that of geodesics near the center. We suppose that the lerp path moves away from the original manifold and moves along another manifold.


% Our semantic path length resembles the perceptual path length (PPL, \citet{karras2019style}) regarding the summation along the interpolation path.
% PPL measures LPIPS \cite{zhang2018unreasonable} distance between resulting images along the path. Higher PPL between two latent variables indicates spikier interpolation of images accompanying artifacts. On the other hand, semantic path length measures how drastically the geometric structure changes between neighboring tangent spaces.





% \subsection{Ablation study}
% \subsection{Comparisons}
% \label{sec:ablation}
% We provide comparisons that include alternative approaches.
% First, we edit images by applying random directions instead of semantic latent directions. \fref{fig:ablation_random_dx_ours} shows that random directions seriously degrade the images. 
% This experiment validates the excellence of the latent directions found by our method.

% \paragraph{Random basis} 
% \paragraph{Zero-shot text-based editing} 

% % Figure environment removed

% % Figure environment removed

% \fref{fig:ablation_direct_dx_ours} demonstrates the necessity of normalization in \eref{eq:cpc}. While our full method produces plausible edited images even with extreme changes, removing the normalization leads to excessive saturation.


% % Figure environment removed

% \subsection{Comparison to other editing methods}
% \label{sec:comparison}
% As we introduce the first unsupervised editing in DMs, we compare our method with GANSpace \cite{harkonen2020ganspace} considering the mapping from $\mathcal{X}$ to $\mathcal{H}$ instead of $\mathcal{Z}$ to $\mathcal{W}$ in GANs. Accordingly, we find directions in $\mathcal{H}$ using PCA. \fref{fig:ganspace} shows their effects: they somewhat alter the attributes but accompany severe distortion or entanglement. On the contrary, our method finds the directions with the largest changes in $\mathcal{H}$ considering the geometrical structure leading to decent manipulation as shown in earlier results. \aref{appendixsec:comparison} describes more details for GANSpace.


% \vspace{-1em}
\section{Conclusion and Limitation}
In this work, we analyze the 2D to 3D mapping problem in constructing the BEV space and give proof of the equivalence between depth estimation in the image space and height estimation in the BEV space. Based on the proof, {{\color{blue}}
we propose HeightFormer which explicitly models heights in BEV without extra LiDAR supervision for car-side situations.} According to our experiments, the proposed self-recursive height predictor can model heights accurately, and the segmentation-based query mask effectively improves the performance of detection. We also show that height modeling is more robust to different camera rigs compared to depth modeling.

{{\color{blue}}
However, there is a limitation in this work. The ground truth heights are acquired by projecting bounding boxes, meaning only a few queries have heights defined. Even if we introduce LiDAR supervision, most grids have no LiDAR points falling in them. As a result, the heights of queries at these grids are still learned implicitly. To mitigate this issue, we filter these queries to avoid introducing irrelevant features in the sampling procedure.}




\bibliography{reference_papers}
\bibliographystyle{plainnat}

\appendix
\onecolumn

\renewcommand{\thetable}{A\arabic{table}}
\renewcommand{\thefigure}{A\arabic{figure}}
\setcounter{figure}{0}
\setcounter{table}{0}
\renewcommand{\theHtable}{A\arabic{table}}
\renewcommand{\theHfigure}{A\arabic{figure}}

\section{Societal Impacts \& Ethics Statements}
\label{appendix:social_impact}
Our research endeavors to unravel the geometric structures of the diffusion model and facilitate high-quality image editing within its framework. While our primary application resides within the creative realm, it is important to acknowledge that image manipulation techniques, such as the one proposed in our method, hold the potential for misuse, including the dissemination of misinformation or potential privacy implications. Therefore, the continuous advancement of technologies aimed at thwarting or identifying manipulations rooted in generative models remains of utmost significance.



\section{Implementation details}
\label{appendix:implementation_detail}

\paragraph{Models and datasets}
We validate our method and provide analyses on various models using the official code and pre-trained checkpoints. The available combinations of the models and the datasets are: 
DDPM~\cite{ho2020denoising} on ImageNet~\cite{deng2009imagenet}, LSUN-church/bedroom/cat/horse~\cite{yu2015lsun}, and CelebA-HQ~\cite{karras2018progressive}; and DDPM trained with {\it P2 weighting}~\cite{choi2022perception} on FFHQ~\cite{karras2019style}, Flowers~\cite{yu2015lsun} and AFHQ~\cite{choi2020stargan}. We also use Stable Diffusion (SD) version 2.1~\cite{rombach2022high} for the text-conditional diffusion model.
% including ImageNet~\cite{deng2009imagenet}, LSUN-church/bedroom/cat/horse~\cite{yu2015lsun}, and CelebA-HQ~\cite{karras2018progressive} for DDPM~\cite{ho2020denoising}; and FFHQ~\cite{karras2019style}, Flowers~\cite{yu2015lsun} and AFHQ~\cite{choi2020stargan} for DDPM trained with {\it P2 weighting}~\cite{choi2022perception}. We also use Stable Diffusion (SD) version 2.1~\cite{rombach2022high} for the text-conditional diffusion model.

For image editing, we use the official codes and pre-trained checkpoints for all baselines and keep the parameters \textit{frozen}. 
For analysis, we compare models with the same diffusion scheduling (linear schedule) and resolutions ($256^2$) to ensure a fair comparison, except Stable Diffusion.

\tref{tab:hyperparameter}1 summarizes various hyperparameter settings in our experiments. Specific details not covered in the main text are discussed in the following paragraphs.

\paragraph{Edit timestep ($t_{edit}$)}
For unconditional DMs, we show the editing results at $t_{edit} \in \{T, 0.8T, 0.6T\}$, while for Stable Diffusion, we show the editing results at $t_{edit} \in \{0.7T, 0.6T\}$. Note that our method allows manipulation at any timestep. 

\paragraph{Inversion step}
We conduct real image editing with DDIM inversion \cite{song2020denoising}. We set the number of steps to $100$ for obtaining the latent variable $\mathbf{x}_T$ and all experiments.
% real image editing 을 위해, DDIM inversion 을 통해 latent variable xT 를 얻었다. \cite{song2020denoising} 이때, 모든 실험에 대해 step 의 개수는 은 둘다 100 으로 고정했다. 

\paragraph{$\vx$-space guidance scale ($\gamma$)}
The value of $\gamma$ determines the magnitude of a single editing step by $\vx$-space guidance. Fortunately, through experimentation, we observed that the value of $\gamma$ does not have a significant impact on image quality unless it is excessively large.

% To obtain the latent code of a given image, we compute the latent code $\mathbf{x}_T$ using DDIM inversion. \cite{song2020denoising} The inversion step hyperparameter refers to the number of DDIM steps used to calculate the latent code. 

\paragraph{Low-rank approximation ($n$)}
We employ a low-rank approximation of the tangent space using $n=50$ for all settings.
% \yh{모든 setting 에서 $n=50$ 을 사용함. pca 를 통해 eigenvalue spectrum 을 그려주면 좋음.}

% In our work, we employ a low-dimensional approximation of the tangent space. Rather than fixing the dimensionality at $n$, we determined to dynamically choose $n$ based on the distribution of eigenvalues. More specifically, we approximated the tangent space with dimensions corresponding to eigenvalues with cumulative density below a given threshold. As such, Table 1 presents the threshold rather than the dimensionality $n$. It worth note that, despite being determined dynamically, the actual values of $n$ has stable for various images. For example, for $t = T, 0.75T, 0.5T, 0.25T$, the values of $n$ were approximately 25, 50, 75, and 100, respectively.

\paragraph{Quality boosting ($t_{boost}$)} While DDIM alone already generates high-quality images, \citet{karras2022elucidating} showed that including stochasticity in the process improves image quality and \citet{kwon2022diffusion} suggest similar technique: adding stochasticity at the end of the generative process. We employ this technique in our experiments on every experiment  after $t=0.2T$, except Stable Diffusion.

\paragraph{Computing resource}
For power-method approximation with $n=50$, it spends about 3-4 minutes on a single NVIDIA RTX 3090 (24GB). As $n$ specifies the number of bases, it can be as small as a user want to use for image editing. Reducing $n$ provides faster runtime, e.g., 10 seconds for $n=3$.
% One can use the power-method approximation with $n<50$ for saving time-consuming when only needing image editing.
% 우리는 NVIDIA 3090 24GB 1대로 모든 실험을 돌렸다. power-method approximation 을 통해 latent basis 를 구할때는, n 에 따라 다르지만, Stable Diffusion 을 포함한 모든 모델이 3~4분 내외로 걸린다. 만약, 이번 work 와 같이 분석이 목적이 아니라 editing 이 목적이면 더 작은 $n<50$ 을 사용할 수 있다. 이 경우에는 n 이 작을수록 걸리는 시간이 줄어든다. 

% \paragraph{Stable Diffusion}
% In order to mitigate the influence of classifier-free guidance, the strength of the guidance, denoted as $w$, was set to zero, utilizing only the text-conditional model. \cite{ho2022classifier} When generating the original Cyberpunk city images, we set the guidance strength as $w = 7.5$. The prompts utilized for the Cyberpunk city images were ``Cyberpunk city" and for the Van Gogh paintings, the prompt used was ``painting of Van Gogh." Through the process of DDIM inversion, latent codes $\mathbf{x}_T$, were generated given the appropriate prompts for each image, with the guidance strength also set to zero (i.e., $\text{guidance scale} = 1$ in the code).

\begin{table}[t]
\caption{Hyper-parameter settings.}
\label{tab:setting}
\begin{center}
\begin{small}
% \begin{sc}
\begin{tabular}{lcccccc}
\toprule
model & $t_{edit}$ & inversion step & $\gamma$ & $n$ & $t_{boost}$ \\
\midrule
Stable Diffusion    & $0.7T$ & 100 & 1     & 50 & $\times$  \\
                    & $0.6T$ & 100 & 2     & 50 & $\times$  \\
Unconditional DMs   & $T$    & 100 & 0.5   & 50 & $0.2T$    \\
                    & $0.8T$ & 100 & 1     & 50 & $0.2T$    \\
                    & $0.6T$ & 100 & 4     & 50 & $0.2T$    \\
\bottomrule
\end{tabular}
% \end{sc}
\end{small}
\end{center}
\vskip -0.1in
\end{table}
\label{tab:hyperparameter}
\section{Ablation study}

In this section, we validate our method with ablation study. 

\paragraph{Random $\vv$} 
To demonstrate the meaningfulness of the latent basis found by our method, we qualitatively compare its effect to na\"ive baseline: random directions.
% To demonstrate that the latent basis we obtained is meaningful, we provide experiments using random directions.
The first row in \fref{fig:random_v} shows that manipulating the images with a random vector `$\vv$' does not result in semantic editing but rather degrades images.
The second row shows the results of projecting the random `$\vv$' onto our obtained latent subspace. The projected results exhibit semantic manipulation such as pose changes without image distortion.
It indicates that the found latent subspace captures semantics in the latent space effectively.

% To verify the quality of the latent basis we obtained, we conducted a comparison with a random direction. First, as observed in \fref{fig:random_v}, random $\vv$ does not effectively allow for semantic editing. Furthermore, projecting the random $\vv$ onto the latent subspace reveals significant semantic manipulation. This implies that the latent subspace we discovered effectively captures the local semantic information of the latent space.

% Figure environment removed

% Figure environment removed


\paragraph{$\vx$-space guidance}
\label{appendixsec:ablation_x_guidance}
\fref{fig:x_space_guidance} demonstrates the the effectiveness of $\vx$-space guidance compared to a straightforward alternative: simple addition. First, $\vx$-space guidance produces higher quality images with similar meaning. Especially Stable Diffusion apparently benefits from $\vx$-space guidance regarding smoothness of the editing strength and artifacts. The difference is more significant at $t=0.6T$. Note that the meaning of the same directions may slightly differ between the two settings due to non-linearity of the U-Net.
% To investigate the effectiveness of $\vx$-space guidance, we qualitatively compare it to the na\"ive approach of direct addition. First, as shown in \fref{fig:x_space_guidance}, overall, $\vx$-space guidance allows qualitatively similar manipulations while enhancing their quality. Specifically, we observe that when using $\vx$-space guidance, artifacts are reduced to a greater extent in Stable Diffusion compared to the unconditional model, and the reduction is more significant at $t=0.6T$ compared to $t=T$. However, it is worth noting that the qualitative direction of the manipulations may slightly differ.

Currently, we do not have a deeper understanding of the underlying principles of $\vx$-space guidance. Exploring the reasons behind its ability to improve manipulation quality would be an interesting direction for future work.

% \clearpage


% Figure environment removed

\section{Comparative experiment to other state-of-the-art (SoTA) editing methods}
\label{appendix:comparisons}
We conduct qualitative comparisons with text-guided image editing methods. Our SoTA baseline methods include: (i) SDEdit \cite{meng2021sdedit}, (ii) Pix2Pix-zero \cite{parmar2023zero}, (iii) PnP \cite{tumanyan2022plug}, and (iv) Instruct Pix2Pix \cite{brooks2023instructpix2pix}. All comparisons were performed using the official code.
Please refer to \fref{fig:comparison} for the qualitative results.

We also compare the time complexity of each method. For a fair comparison, we only identify the first singular vector $\mathbf{v}_1$, i.e., $n=1$, and set the number of DDIM steps to 50. All experiments were conducted on an Nvidia RTX 3090. The runtime for each method is summarized in \tref{tab:comparison_time}.

The computation cost of our method remains comparable to other approaches, although the Jacobian approximation takes around 2.5 seconds for $n=1$. This is because we only need to identify the latent basis vector once at a specific timestep. Furthermore, our approach does not require additional preprocessing steps like generating 100 prompts with GPT and obtaining embedding vectors (as in Pix2Pix-zero), or storing feature vectors, queries, and key values (as in PnP). Our method also does not require finetuning (as in Instruct Pix2Pix). This leads to a significantly reduced total editing process time in comparison to other methods.

\begin{table}[t]
\caption{\modify{\textbf{Comparisons of the time complexity of state-of-the-art editing methods} 
% We used five attribution methods, including a random baseline, to compare the rankings. 
% A stronger correlation between the rankings signifies greater consistency in the results, irrespective of whether the model undergoes retraining or not.
}}
% \vskip 0.15in
\begin{center}
\begin{small}
\begin{tabular}{c|c|c}
\toprule
 Image Edit Method & Running time & Preprocessing \\
\midrule
Ours             &     11 sec    & N/A            \\
SDEdit           &      4 sec    & N/A            \\
Pix2Pix-zero     &     25 sec    & 4 min          \\
PnP              &     10 sec    & 40 sec         \\
Instruct Pix2Pix &     11 sec    & N/A            \\
\bottomrule
\end{tabular}
\end{small}
\end{center}
% \vskip -0.1in
\label{tab:comparison_time}
\end{table}


\section{More Discussions}

% \begin{wrapfigure}{!r}{6cm}
%     \vspace{-1.5em}
%     % Figure removed
%     \vspace{-1.5em}
%     \caption{\textbf{Parallel transport between similar tangent spaces creates similar latent directions.}
%     The horizontal axis represents the geodesic distance between tangent spaces from different timesteps, while the vertical axis represents the angle between the original latent direction and transported latent direction. 
%     Different colors represent various datasets. 
%     A positive relationship is observed between tangent space distance and the distortion induced by parallel transport.}
%     \label{fig:parallel_transport_x_theta}
%     \vspace{-2.5em}
% \end{wrapfigure} 

% Figure environment removed

\paragraph{Why do we measure the geodesic distance of the tangent spaces instead of the latent subspaces?}
\label{appendixsec:relationship_tangent_latent}
The geodesic distance on the Grassmannian manifold between two subspaces is defined as the $l_{2}$-norm of principal angles. 
To define angles between different vector spaces, an inner product needs to be defined. In our work, we define the inner product in $\mathcal{T}_\vx$ using the pullback metric. The issue is that the pullback metric is locally defined for each latent subspace $\mathcal{T}_\vx$ (\eref{eq:pullback}). Therefore, measuring angles between distant latent subspaces becomes challenging. On the other hand, \ehspace{} follows the assumption of the Euclidean metric. Consequently, even for distant tangent spaces, angles can be easily computed using the dot product.
In this regard, we measure the similarity between latent subspaces by exploiting the geodesic distance of their corresponding tangent spaces.
\yh{Furthermore, when compared to \exspace{}, \ehspace{} offers the advantage of being a semantic space, making it more suitable for measuring semantic similarity.}

% In our work, much of the analysis consists of measuring the geodesic distance between different tangent spaces, i.e., $D_{\text{geo}}(\mathcal{T}_\vh, \mathcal{T}_{\vh'})$. 
% This is because the geodesic distance measure the principle angle, based on the dot product. (See \aref{appendixsec:algorithm}). This makes sense in \ehspace{}, where the Euclidean metric is assumed, but not in \exspace{}.


% Figure environment removed


\paragraph{Similar tangent space implies similar latent subspace}
% \paragraph{What does it imply if the tangent space is similar?}
% In \fref{fig:parallel_transport_x_theta}, 서로 다른 timestep 에서 구한 tangent space 간 geodesic distance 와 그 둘 사이에서 parallel transport 한 latent direction 간 각도를 구한 것이다. 
% It is evident that as the geodesic distance decreases, the amount of distortion during parallel transport also reduces.
% \yh{이는 tangent space 가 유사하면 latent subspace 도 유사함을 의미한다.}
In \fref{fig:parallel_transport_x_theta}, we calculated the geodesic distance of tangent spaces obtained at different timesteps (or different samples at same the timestep) and the angle between the original latent direction and parallel transported direction between them.
It is evident that as the geodesic distance decreases, the amount of distortion during parallel transport also reduces.
% For parallel transport across different timesteps, the result is visually illustrated in \fref{fig:evolution-t-appendix}.

\modify{
Notice that the similarity between tangent spaces implies \uh{consistency of latent basis across timesteps}. %a consistent latent basis at each timestep.
In \fref{fig:evolution-t-appendix} (b), we parallel transport the latent vector $\vv_i$ to various tangent spaces and visualize the outcomes.
As expected, when the tangent spaces are similar, the transported vector ${\vv'}_i$ retains the original signal. On the other hand, as we move to more distant timesteps, where the tangent space is farther apart, ${\vv'}_i$ deviates from the original signal.
}



% Furthermore, \fref{fig:Homogenity_prompts_appendix}, we visualize latent direction given various prompts.
% 여기서도 마찬가지로, tangent space 간 거리가 가까워지는 0.6T 구간부터 latent direction 들이 유사해지는 양상을 확인할 수 있었다. 

% \label{appendixsec:relationship_tangent_latent}
% 두 tangent space 가 유사하다는건 그에 대응하는 latent space 에서 유사한 signal 을 찾을 수 있음을 의미합니다. 
% In \fref{fig:Homogenity_prompts_appendix}, 는 서로 다른 timestep 에서 tangent space $\mathcal{T}_\vh, \mathcal{T}_{\vh'}$ 를 뽑고, 하나의 tangnet space 에서 다른 tangent space 로 parallel transport 한 뒤, .
% Parallel transport 를 할 때 geodesic distance 가 작을수록 왜곡이 덜 일어나는 것은 자명하다. 하지만, \fref{fig:Homogenity_prompts_appendix} 는 latent vector 도 왜곡이 덜 일어남을 말해주고 있다. \fref{fig:evolution-t-appendix} 는 이 결과를 눈으로 보여주고 있다. 

% 또한 \fref{fig:Homogenity_prompts_appendix} 는 서로 다른 prompt 가 주어졌을때 latent vector 를 눈으로 보여주고 있다. tangent space 간 distance 가 클수록, latent vector 에 담긴 패턴이 많이 달라지는 것을 확인할 수 있다.

% % Figure environment removed

% \clearpage

\section{Algorithms}
\label{appendixsec:algorithm}

In this section, for reproducibility purposes, we provide the code for two important algorithms. The code is implemented using PyTorch \cite{paszke2017automatic}.

\paragraph{Jacobian subspace iteration}
The diffusion model has dimensions that are too large in both \exspace{} and \ehspace{}, making the computation of the Jacobian infeasible. To overcome this challenge, we attempt the Jacobian subspace iteration algorithm to approximate the singular value of the Jacobian, as proposed in \cite{haas2023discovering}. For details, please refer to \citet{haas2023discovering}.

\begin{lstlisting}[language=Python, caption=Python example, caption={\textbf{Jacobian subspace iteration}}, captionpos=b]
import torch # >= ver 2.0

def local_encoder_pullback(
        x, t, get_h, n=50, chunk_size=25, min_iter=10, max_iter=100, convergence_threshold=1e-4,
    ):
    '''
    Args
        - x : tensor ; latent variable
        - t : tensor ; diffusion timestep
        - get_h : function ; return h given x, t
        - n ; low-rank approximation dimension
        - chunk_size ; To avoid OOM error
        - min_iter (max_iter) ; minimum (maximum) number of iteration
        - convergence_threshold ; to check convergence of power-method
    '''
    # set number of chunk to avoid OOM
    num_chunk = n // chunk_size + 1

    # get dimensions of x space and h space
    h_shape = get_h(x, t).shape
    c_i, w_i, h_i = x.size(1), x.size(2), x.size(3)
    c_o, w_o, h_o = h_shape[1], h_shape[2], h_shape[3]

    # power-method
    a = torch.tensor(0., device=x.device, dtype=x.dtype)
    vT = torch.randn(c_i*w_i*h_i, n, device=x.device)
    vT, _ = torch.linalg.qr(vT)
    v = vT.T
    v = v.view(-1, c_i, w_i, h_i)

    for i in range(max_iter):
        v = v.to(device=x.device, dtype=x.dtype)
        v_prev = v.detach().cpu().clone()
        
        time_s = time.time()
        u = []
        v_buffer = list(v.chunk(num_chunk))
        for vi in v_buffer:
            g = lambda a : get_h(x + a*vi, t=t)
            ui = torch.func.jacfwd(
                g, argnums=0, has_aux=False, randomness='error'
            )(a)
            u.append(ui.detach().cpu().clone())
        u = torch.cat(u, dim=0)
        u = u.to(x.device, x.dtype)

        g = lambda x : torch.einsum(
            'b c w h, i c w h -> b', u, get_h(x, t=t)
        )
        v_ = torch.autograd.functional.jacobian(g, x)
        v_ = v_.view(-1, c_i*w_i*h_i)

        _, s, v = torch.linalg.svd(v_, full_matrices=False)
        v = v.view(-1, c_i, w_i, h_i)
        u = u.view(-1, c_o, w_o, h_o)
        
        convergence = torch.dist(v_prev, v.detach().cpu()).item()
        if torch.allclose(v_prev, v.detach().cpu(), atol=convergence_threshold) and (i > min_iter):
            break

    # reshape as a x space, h space vector
    u, s, vT = u.reshape(-1, c_o*w_o*h_o).T.detach(), s.sqrt().detach(), v.reshape(-1, c_i*w_i*h_i).detach()
    return u, s, vT
\end{lstlisting}

\paragraph{Geodesic metric}
For a detailed discussion on the geodesic metric, please refer to \citet{choi2021not} for more information.
% Geodesic metric 에 대한 구체적인 논의는 \citet{} 를 참고하라. 

\begin{lstlisting}[language=Python, caption={\textbf{Geodesic metric}}]
import torch

def geodesic_metric(U1, U2):
    _, S, _ = torch.linalg.svd(U1.T @ U2)
    th = torch.acos(S)
    return th.norm()
\end{lstlisting}

% power-method approximation
% \begin{algorithm}[ht!]
% \caption{Feature Direction}
% \label{alg:local_basis}
% \begin{algorithmic}[1]
% \REQUIRE {latent variable $\mathbf{x}$, timestep $t$, U-Net encoder $f : \mathcal{X} \times T \rightarrow \mathcal{H}$, Feature direction index $i$}
% \STATE {$J$             = Jacobian($f(\cdot, t)$)($\mathbf{x}$)}
% \STATE $U, S, V^{\tran}$ = SingularValueDecomposition($J$) 
% \STATE {$\mathbf{v}_i, \mathbf{u}_i$ = $V^{\tran}$[$i$, :], $U$[:, $i$]}
% \STATE {{\bfseries Return} $\mathbf{v}_i, \mathbf{u}_i$}
% \end{algorithmic}
% \end{algorithm}

%%%%%%%%%%%%%%%%%%%%%%%%%%%%%%%%%%%
% below is the additional figures %
%%%%%%%%%%%%%%%%%%%%%%%%%%%%%%%%%%%
\clearpage

\section{Additional results}
\label{appendix:additional_results}
\subsection{Latent basis}
\label{appendix:local}
\paragraph{Unconditional latent basis}
We provide more examples of image editing using the latent basis. Figure~\ref{fig:ffhq_vis}, \ref{fig:afhq_vis}, \ref{fig:flowers_vis}, \ref{fig:stable_7T} and \ref{fig:stable_6T} show that every latent basis produces different results and editing at timestep $T$ yields coarse changes while 0.6$T$ leads to fine changes. Stable Diffusion shows a similar trend; 0.7$T$ yields coarse changes while 0.6$T$ leads to fine changes. The results of $T$ in Stable Diffusion will be covered in the \sref{appen:more_discussion}. Please zoom in for the best view.

% Figure environment removed

% Figure environment removed

% Figure environment removed

% Figure environment removed

\clearpage

% Figure environment removed


\paragraph{latent basis with given prompt}
%As shown in \fref{fig:stable_various_vis}, using the prompt as a condition, the entire latent basis captures prompt-related attributes. 
As shown in \fref{fig:stable_various_vis}, when we condition a specific prompt, such as ``Zebra'' or ``Chimpanzee'', the entire latent basis corresponds to the prompt-related attributes.
Notably, Changes to ``zebra'', which are clear, all show similar results, but ``chimpanzee'' show different results. Nevertheless, it is clear that they are all related to ``chimpanzee''.

 

% Figure environment removed


% \subsection{}
\subsection{Image editing using latent basis vectors discovered with various prompts}

We provide additional examples of image editing using latent basis vectors discovered with various prompts. \fref{fig:stable_text_more}, \ref{fig:stable_text_more2} show image editing with various pictures and various prompts. 
%The prompt of ``[···cat··]'' is ``a cat dressed as a witch wearing a wizard hat in a haunted house''. 
For brevity, we denote the prompt "a cat dressed as a witch wearing a wizard hat in a haunted house" by "[···cat··]" in \fref{fig:stable_text_more}.

% Figure environment removed

\clearpage

% Figure environment removed

% \subsection{More discussion of using latent basis vectors discovered with various prompts}
\subsection{More discussion on the editing capability of the latent basis discovered with text conditions}
\label{appen:more_discussion}

In this subsection, we provide a discussion based on the failure cases of our approach. \fref{fig:stable_1T} shows the results of latent basis found at $t=T$ with Stable diffusion. Unlike unconditional models, the directions found at $t=T$ exhibit rapid and drastic unexpected changes. 
%However, it was observed that for landscape photos without a main object, the expected editing effects were demonstrated at any timesteps. 
However, landscape photos, which do not contain a main object, exhibited desired editing effects at any timesteps.
Moreover, in the case of landscapes, it is conjectured that the latent basis plays a significant role in representing patterns and textures. 
Analyzing the landscape images generated by Stable diffusion would be an interesting topic.

\fref{fig:more_limitations} presents examples of failure cases in our image editing using latent basis vectors discovered with various prompts.
(a) When using pose or action as a prompt, there are instances where the identity is not preserved.
(b) When the shape of the target subject differs significantly from the source image, the results are often unsatisfactory.
(c) There are cases where the preservation of the background is not achieved.
(d) It is challenging to make significant changes to the entire image.

Regarding the reasons for these failure cases, we emphasize two factors.
First, we manipulate in the \exspace{}. % ``$\vx_t$''. 
% We adopt a direct manipulation of $\vx_t$ to analyze the latent space of the diffusion model. 
The result in \fref{fig:more_limitations} (a) implies that \exspace{} is not a space where disentanglement for identity is achieved effectively.
On the other hand, in \ehspace{}, there are results indicating successful preservation of identity \cite{kwon2022diffusion, haas2023discovering}. 
Investigating the disentanglement capability of \exspace{} and any other distinguishing features it may have compared to \ehspace{} would be an interesting future research topic.

Secondly, we perform manipulation by adding and subtracting the "signal" that the model pays attention to in $\vx_t$.
% The directions we find represent the signal captured by the model's features. 
Here, The signal is captured from the current input $\vx_t$, which limits the deviation from the original form. Therefore, when there is a substantial difference in shape, such as transforming a giraffe into a tiger, the results may not be satisfactory. (\fref{fig:more_limitations} (b))
When we utilize text conditions, the latent basis aligns with the text information. This leads to not capturing background information, resulting in changes in the background when manipulated. It is also an interesting research topic to capture signals related to the background. (\fref{fig:more_limitations} (c))
% Since we utilize the top $n$ signals captured by the features, they often do not include background information. It is also an interesting research topic to capture signals related to the background. (\fref{fig:more_limitations} (c))
Since the model focuses on finer features as t approaches 0, if broad changes are desired, manipulation should be performed at $t=T$. However, manipulation at $t=T$ is unstable. Deep analysis of $\vx_t$ at $t=T$ in Stable diffusion is also an intriguing research topic. (\fref{fig:more_limitations} (d))

% Regarding the reasons for these failure cases, we emphasize two factors.
% First, the ``signal''. The directions we find represent the signal captured by the model's features. This signal is entirely determined by the given text, which can result in a loss of identity preservation. The signal is captured from the current input $\vx_t$, which limits the deviation from the original form. Therefore, when there is a substantial difference in shape, such as transforming a giraffe into a tiger, the results may not be satisfactory. Since we utilize the top $n$ signals captured by the features, they often do not include background information. It is also an interesting research topic to capture signals related to the background.

Despite these limitations, we have successfully achieved direct manipulation in the latent space $\vx_t$ at a single timestep, which, to our knowledge, is the first of its kind. Through this, we provide insights into the model and contribute to the understanding of the latent space, hopefully benefiting the diffusion community.

% Figure environment removed

% Figure environment removed

% % Figure environment removed

% % Figure environment removed

% % Figure environment removed

% % Figure environment removed

% % Figure environment removed

% % Figure environment removed

% % Figure environment removed

% % Figure environment removed

% % Figure environment removed

% % Figure environment removed

% % Figure environment removed


% \clearpage
% \subsection{Global feature direction}
% \label{appendix:global}

% % Figure environment removed

% % Figure environment removed

% % Figure environment removed




% \begin{ack}
% Use unnumbered first level headings for the acknowledgments. All acknowledgments
% go at the end of the paper before the list of references. Moreover, you are required to declare
% funding (financial activities supporting the submitted work) and competing interests (related financial activities outside the submitted work).
% More information about this disclosure can be found at: \url{https://neurips.cc/Conferences/2023/PaperInformation/FundingDisclosure}.


% Do {\bf not} include this section in the anonymized submission, only in the final paper. You can use the \texttt{ack} environment provided in the style file to autmoatically hide this section in the anonymized submission.
% \end{ack}



% \section{Supplementary Material}

% Authors may wish to optionally include extra information (complete proofs, additional experiments and plots) in the appendix. All such materials should be part of the supplemental material (submitted separately) and should NOT be included in the main submission.


% \section*{References}


% References follow the acknowledgments in the camera-ready paper. Use unnumbered first-level heading for
% the references. Any choice of citation style is acceptable as long as you are
% consistent. It is permissible to reduce the font size to \verb+small+ (9 point)
% when listing the references.
% Note that the Reference section does not count towards the page limit.
% \medskip


% {
% \small


% [1] Alexander, J.A.\ \& Mozer, M.C.\ (1995) Template-based algorithms for
% connectionist rule extraction. In G.\ Tesauro, D.S.\ Touretzky and T.K.\ Leen
% (eds.), {\it Advances in Neural Information Processing Systems 7},
% pp.\ 609--616. Cambridge, MA: MIT Press.


% [2] Bower, J.M.\ \& Beeman, D.\ (1995) {\it The Book of GENESIS: Exploring
%   Realistic Neural Models with the GEneral NEural SImulation System.}  New York:
% TELOS/Springer--Verlag.


% [3] Hasselmo, M.E., Schnell, E.\ \& Barkai, E.\ (1995) Dynamics of learning and
% recall at excitatory recurrent synapses and cholinergic modulation in rat
% hippocampal region CA3. {\it Journal of Neuroscience} {\bf 15}(7):5249-5262.
% }

% %%%%%%%%%%%%%%%%%%%%%%%%%%%%%%%%%%%%%%%%%%%%%%%%%%%%%%%%%%%%



\end{document}