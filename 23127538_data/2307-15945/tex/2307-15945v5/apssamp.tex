
\documentclass[%
 reprint,
%superscriptaddress,
%groupedaddress,
%unsortedaddress,
%runinaddress,
%frontmatterverbose, 
%preprint,
%preprintnumbers,
%nofootinbib,
%nobibnotes,
%bibnotes,
 amsmath,amssymb,
 aps,
%pra,
%prb,
%rmp,
%prstab,
%prstper,
%floatfix,
]{revtex4-2}


\usepackage{xcolor}

\usepackage{graphicx}% Include figure files
\usepackage{dcolumn}% Align table columns on decimal point
\usepackage{bm}% bold math
\usepackage{bigints}
\usepackage{bm}
\usepackage{hyperref}
\hypersetup{pdfborder=0 0 0} %to remove the red borders in content


\begin{document}

\preprint{APS/123-QED}

\title{Unconventional optical response in monolayer graphene \\ due to dominant intraband scattering}% Force line breaks with \\
%\thanks{A footnote to the article title}%

\author{Palash Saha}
\author{ Bala Murali Krishna Mariserla}%
 \email{bmkrishna@iitj.ac.in}
\affiliation{
 Ultrafast Physics Group, Department of Physics, Indian Institute of Technology Jodhpur, Rajasthan, India-342037 
}

\date{\today}
\begin{abstract}
Scattering dynamics influence the graphene's transport properties and inhibits the charge carrier deterministic behaviour. The intra/inter-band scattering mechanisms are vital for graphene’s optical conductivity response under specific considerations of doping. In this study, we systematically explored the impact of scattering on optical conductivity using a semi-classical multiband Boltzmann equation, incorporating both electron-electron and electron-phonon interactions collectively with a single phenomenological relaxation time constant. We found unconventional characteristics of linear optical response  with a significant deviation from the universal conductivity $(e^2/4\hbar)$ in doped monolayer graphene. This is explained through phenomenological relaxation rates under low doping regime with dominant intraband scattering. Such novel optical responses vanish at high temperatures or overdoping conditions due to strong Drude behaviour. With the aid of approximations around Dirac points we have developed analytical formalism for many body interactions and is in good agreement with the Kubo approaches.
\end{abstract}
\maketitle
\section{\label{sec:intro}Introduction}
%\textit{Introduction.\textbf{\textemdash}}
Unique and remarkable quantum properties of monolayer graphene with a honeycomb structure, have sparked significant interest in both linear \cite{katsnelson} and nonlinear optics \cite{Agrawal}. The Dirac cone symmetry is responsible for the ambipolar behaviour thus producing a surge of carrier concentration in suspended graphene \cite{ball1,ball2} and encapsulated graphene \cite{ball3}. This is in contrast to the semiconductors where a particular quasi-particle plays a major role for the transport properties \cite{cardona}. The charge carriers in graphene with zero rest mass resembles the relativistic entities and exhibit an effective velocity comparable to the speed of light\cite{mass1,mass2}. These distinctive characteristics of charge carriers in pristine graphene are at the helm of its high conductivity \cite{rev_con1,rev_con2}. In order to describe the electronic properties of different two dimensional materials, the tight-binding Hamiltonian is a first simplified non-interacting model and commonly used on monolayer, bilayer and twisted graphene \cite{TBM_BG1,TBM_BG2,TBM_BG3}. This basic model reveals the behavior of $\pi$-electrons in the graphene hexagonal lattice by considering the interaction between neighboring carbon atoms through electron hoping and overlapping of their atomic orbitals which successfully produces the electronic band structure including linear dispersion near Dirac points. To calculate the optical conductivity of graphene using the tight-binding Hamiltonian, one can employ the Kubo expression \cite{Kubo1,Kubo2,Kubo3} from the linear response theory \cite{mahan}. But the current-current correlation function in the Kubo expression requires summation over all the states in high-dimensional systems, which can become computationally expensive. 

Besides the Kubo approach, the Semiconductor Bloch equation (SBE) \cite{mahan,ferry} is an alternative and more effective to calculate the optical conductivity of semiconductors and two dimensional crystals. The SBE is derived from the time-dependent Schrödinger equation for periodic systems that provide great advantage in solving the out of equilibrium problems. Within the realm of this SBE, a cascade of  different studies has been focused to observe a spectrum of diverse optical phenomena. Out of which,  Cheng, Sipe and Vermuelen conducted extensive study on doped monolayer graphene to obtain linear and nonlinear optical conductivity with scattering \cite{CSV15} and without scattering \cite{CSV141} using approximations around Dirac Points. With the similar assumptions, they have calculated DC Current induced Second harmonic generation \cite{CSV142} at zero temperature limit. A quite similar research has been conducted by Hipolito, Taghizadeh, Alireza and Pedersen \cite{HTP1} to find the linear conductivity, third harmonic generation and optical Kerr effect for both monolayer and bilayer graphene. Identification of new kind of third order divergences during cross phase modulation and degenerate four wave mixing were carried out by Cheng, Sipe, Wu and Guo \cite{cheng_intraband} for centrosymmetric two dimensional materials. Naib and Sipe \cite{Naib} detected intraband and interband current with applying electric field through numerical simulation on undoped monolayer graphene at low temperatures. McGouran et al \cite{mcgouran} obtained dipole matrix elements through length gauge and found the linear and nonlinear Terahertz responses for undoped suspended bilayer graphene through numerical simulations at different low temperatures with scattering time of 80 fs. Study of carrier mobility, carrier density, band structure, linear and nonlinear Terahertz response of nitrogen doped graphene were reported \cite{nitrogen} using this technique. Cheng and Guo \cite{magnetic} studied the nonlinear magneto-optic effects in doped and gapped graphene and examined linear response, Third Harmonic Generation, Kerr effects, two photon absorption and four wave mixing under external magnetic field (\textit{B}=0.05 T). The previous studies were using nonlinear relaxation rates for studying higher order optical phenomenon and less focus on the influence of equal rates for different orders which should be exercised to understand anomalous linear optical responses.\\
\vspace{-0.05cm}
In this paper, the linear optical response for monolayer graphene is investigated using the SBE approach with the many particle approximations at different doping levels.Theoretically, the length gauge approach for the equation of motion with the density matrix has been used innumerable times to investigate various optical parameters.  The different scattering mechanisms are introduced in our calculations with the aid of phenomenological relaxation parameters to the modified SBE and the conductivity response is obtained from the first order density matrix which comprises of both intra- and inter-band components. Furthermore, the results are compared with experimental results to gain more insights about the Pauli blocking and conductivity response. Our formalism is in good agreement with the experimental data with slight deviations due to the  approximations near Dirac points, lower optical field energy and non-inclusion of thermal fluctuations. We have shown that the utilization of approximations around Dirac points bridge the results obtained from perturbative calculations and the Kubo expression under an external electric field.

%The paper is organized in the following way, Section-II describes the analytical methodology for optical conductivities. The obtained results are discussed in the section III and overall summary is presented in section IV.    
%\section{\label{sec:semi_approach}Semi-classical Approach}
% Figure environment removed
\section{Semi-classical Approach}
%\textit{Semi-classical Approach.\textbf{\textemdash}}
The SBE consists of perturbed Hamiltonian with radiation-interaction and scattering terms to study many-body interactions. For the graphene, unperturbed tight binding Hamiltonian ($\hat{H_0}$) deals with the nearest neighbour hopping with zero onsite energies. 
\begin{equation}
\hat{H}_0=\sum\limits_{n}^{}\sum\limits_{k}\varepsilon_{nk}c_{nk}^{\dag}c_{nk}
\end{equation}
The energy eigenvalues are denoted by $\varepsilon_{nk}$ and the coefficients $c^\dag$,$c$ are creation and annihilation operators with states \textit{n}=[1,2] in the momentum space (\textit{k}). Previously, the light matter interactions were treated with velocity gauge, but challenges arise for many body interactions because of the spurious divergences at lower frequencies due to its sensitivity and susceptibility to numerical inaccuracies \cite{Peres17}. These divergence issues can be easily tackled by using the length gauge through sum rules \cite{sipe_gahramani} and gauge invariant electric field, which emerged as a pivotal solution to the problem. However, it does possess an additional scalar dependence that disturb the diagonal property of the perturbed Hamiltonian in k-space. Because of that, there will be some overlap or coupling contributions which can be eliminated by the band isolation technique \cite{Aversa_Sipe}. 

The light-matter interacting term in the Hamiltonian $(\hat{H}_{eR})$ possesses interaction between electric field of the light  and  charge distribution of the matter. By considering electric dipole approximation along with long wavelength limit for uniform illumination, we have employed the length gauge approach with explicit dependence on external electric field \textbf{\textit{E}}(t) given by,
\begin{equation}  
\hat{H}_{eR}=-e\textbf{E}(t).\mathbf{\hat{r}}
\end{equation}
where, \textit{e} is the charge of the electron and ${\bm{\hat{r}}}$ being the spatial coordinates. The position operator in the length gauge approach is linked to the Berry connection \bm{$(\xi)$} when the bands are distinct under external electrical field.
{In the context of distinct bands, the Berry connection or Berry curvature may capture the geometric phase relationship between them when system undergoes adiabatic changes in parametric space.}
Close proximity of the Dirac cones (Fig.\ref{Energy_near_Dirac_points}) is no longer valid due to presence of external electric field radiation within the light-matter Hamiltonian \cite{CSV15} given by,
\begin{multline}
\hat{H}_{eR} = -e\textbf{E}(t).\sum_{n_1} \sum_{n_2} \sum_k c_{n_1k}^{\dag}\bm{\xi}_{n_1n_2k}c_{n_2k} -\\
ie\textbf{E}(t).\sum_{n} \sum_k \nabla_{k}c_{nk}^{\dag}c_{nk}
\label{light-matter}
\end{multline}

in which the last term possesses local band characteristics. The effect of electron-electron, electron scattering due to phonon, and impurities are dealt by scattering term in the Hamiltonian. In the equation of motion (Eq.(\ref{liou})), the rate of change due to scattering can be made proportional to the density of states with the inclusion of phenomenological relaxation parameter.
%for all orders.
\subsection{\label{sec:level2}Collisional Multiband Boltzmann Equation}

%\textit{Collisional Multiband Boltzmann Equation.\textbf{\textemdash}}
We have adopted Sipe-Aversa's SBE \cite{Aversa_Sipe} with length gauge perturbation theory to  develop the equation of motion and solved for the response function of monolayer graphene. In general, most of the calculations are always focused on interband scattering contribution while neglecting the intraband scattering. Using a modified SBE, we take intraband information directly into account with the aid of phenomenological relaxation parameters. To obtain expression for many body interactions, we followed semi-classical approach using the quantum Liouville equation,
\begin{equation}
    i\hbar\frac{\partial \hat{\rho}(t)}{\partial t}=[\hat{H},\hat{\rho}(t)]
\label{liou}
\end{equation}
which consists of the total Hamiltonian $(\hat{H})$ and the time dependent density operator denoted by,
\begin{equation}
\hat{\rho}(t)=\sum\limits_{n_1 n_2}^{}\rho_{n_1 n_2}(t) |n_1\rangle \langle n_2|
\label{density}
\end{equation}
where, $\hbar$ is the reduced Planck's constant.
Using Eq.(\ref{liou}) and (\ref{density}) the equation of motion in terms of density matrix is obtained,
\begin{multline}
i\hbar\frac{\partial {\rho_{n_1 n_2}}(t)}{\partial t}|n_1\rangle \langle n_2|= \hat{H}_0 |n_1\rangle\rho_{n_1 n_2}(t) \langle n_2|\\
-\rho_{n_1 n_2}(t) |n_1\rangle\langle n_2|\hat{H}_0 
-i\Gamma_{n_1n_2}\rho_{n_1n_2}(t)|n_1\rangle\langle n_2|\\
-e\textbf{E}(t).\langle n_1|[\mathbf{\hat{r}},\hat{\rho}(t)] |n_2\rangle|n_1\rangle\langle n_2|
\label{SBE}
\end{multline}
 where, $\mathit{\Gamma}$ is the phenomenological constant which contains the scattering parameters. In the presence of the scattering,  the time evolution of the density operator undergo various physical effects including damping or decoherence. We have used the band isolation identity \cite{Aversa_Sipe} to get a commutation relationship for the spatial coordinate $(\mathbf{\hat{r}})$ with a physical operator($\mathit{\hat{\Lambda}}$),
\begin{multline}
\int{dr\Psi_{n_1k_1}^{*}(r)[\mathbf{\hat{r}},\hat{\Lambda}]\Psi_{n_2k_2}(r)}=\\ \left[\bm{\xi}_{k_1k_2},\Lambda_{k_1k_2}\right]_{n_1,n_2}+i\nabla_k\Lambda_{n_1n_2k_1k_2}
\label{Blount}
\end{multline}
In general, the single-band Boltzmann equation can model the charge carrier transport in a single electronic band, which can be extended for the multiband.
To depict this different band characteristics, the above Eq.(\ref{Blount}) is extended for each \textit{k} values to perceive the required collisional multiband Boltzmann equation (CMBE),
%\begin{widetext}
\begin{multline}
i\hbar\frac{\partial {\rho_{n_1 n_2k}}(t)}{\partial t}=(\varepsilon_{n_1k}-\varepsilon_{n_2k})\rho_{n_1 n_2k}(t) \\
-ie\textbf{E}(t).\nabla_k\rho_{n_1n_2k}(t)- e\textbf{E}(t).\sum\limits_{n}\left(\bm{\xi}_{n_{1}nk}\rho_{nn_2k}(t) \right. \\
\left.- \rho_{n_1nk}(t)\bm{\xi}_{nn_2k}\right)-i\Gamma_{n_1n_2k}\rho_{n_1n_2k}(t)
\label{CMBE}
\end{multline}
The right hand side of this equation of motion is equivalent in form to the Quantum Boltzmann Equation (QBE). It consists of the kinetic energy terms which seem to combine with the dot product of scalar potential. Eliminating the reliance of this offset energy on position and maintaining system uniformity, the mentioned identity offers an alternative advantage. And, finally the extra driving term collectively accounts for all the collisions which is the identical term from QBE.
%\end{widetext}

\subsection{\label{sec:level3}Conductivity Expression}
%\textit{Conductivity Expression.\textbf{\textemdash}}
Composition of the \textit{j}$^{th}$ order CMBE is possible if the density matrix is considered to be sum of all orders,
 \begin{gather}
    \hbar \frac{\partial \rho_{n_1n_2k}^{\!(j)}(t)}{\partial t}=-i(\varepsilon_{n_1k}-\varepsilon_{n_2k})\rho^{\!(j)}_{n_1 n_2k}(t)-\Gamma_{n_1n_2k}^{\!(j)}\rho_{n_1n_2k}^{\!(j)}(t)\nonumber\\ +ie\textbf{E}(t).\left[\bm{\xi}_k,\rho_k^{\!(j-1)}(t)\right]_{n_1n_2}-e\textbf{E}(t).\nabla_k\rho_{n_1n_2k}^{\!(j-1)}(t)
\label{jSBE}
\end{gather}
The solution of the CMBE equation is in the following form,
\begin{equation}
    \rho_{n_1n_2k}^{\!(j)}=\int..\int\left[-e\frac{d\omega}{2\pi}E_\omega^ae^{-i\omega t}\right]^{j}\mathcal{S}_{n_1n_2k}^{\!(j)}(\omega_1,\omega_2,..\omega_j) 
\label{solution}
\end{equation}
where, $\mathcal{S(\omega)}$ is the  $j^{th}$ order frequency component. By making use of the perturbative expansion of the current density, a pathway opens for establishing a connection with the surface current density given by,
\begin{gather}
       e\sum\limits_{n_1n_2}\int\frac{d\textbf{k}}{4\pi^2}v^a_{n_1n_2k}\rho_{n_1n_2k}^{\!(j)}(t)=\nonumber\\
       \int..\int\left[\frac{d\omega}{2\pi}E_\omega^ae^{-i\omega t}\right]^{j}\sigma^{\!(j);abcd..}(\omega_1,\omega_2,..\omega_j)
       \label{current_density}
\end{gather}
After substitution of Eq.(\ref{solution}) in Eq.(\ref{current_density}), we get the the generalized formula for conductivity,
    \begin{equation}
    \sigma^{\!(j);abcd..}=-e^{j+1}\sum_{n_1n_2}\int\frac{d\textbf{k}}{4\pi^2}v^a_{n_1n_2k}\mathcal{S}_{n_1n_2k}^{\!(j);bcd..}(\omega_1,\omega_2,..\omega_j)
    \label{formula_conductivity}
    \end{equation}

% Figure environment removed
% Figure environment removed

Solving the above equation (see Appendix) produces the final expression which depends upon frequency and chemical potential ($\mathit{\mu}$) given by,
\begin{equation}
    \sigma_{intra}^{\!(1)xx}(\omega)=\frac{i\sigma_0}{\pi}\frac{4|\mu|}{\hbar\omega+i\Gamma^{\!(1)}_i}
\label{intra_con}
\end{equation}
and,
\begin{multline}
    \sigma_{inter}^{\!(1)xx}(\omega)=-\frac{\sigma_0}{\pi} \biggl[ -\pi+i\hspace{0.05cm} ln\left|\frac{\hbar\omega+2\mu+i\Gamma^{\!(1)}_e}{\hbar\omega-2\mu+i\Gamma^{\!(1)}_e} \right|\\
    +\frac{1}{2}ln\left|\frac{\hbar\omega+2\mu+\Gamma^{\!(1)}_e}{\hbar\omega-2\mu+\Gamma^{\!(1)}_e} \right| \biggl]
\label{inter_con}
\end{multline}
where, $\mathit{\Gamma_e}$ is the interband constant, $\mathit{\Gamma_i}$ is the intraband constant and $\sigma_0=(e^2/4\hbar)$ is the universal conductivity. In the case of graphene, the Dirac cone electronic dispersion leads to a unique Drude response and the expression of Drude conductivity for Dirac electrons is the intraband conductivity in Eq.(\ref{intra_con}), which coincides with the Boltzmann-Drude expression.
In connection to this, direct gap approximation is needed to be considered when the Berry connection is dealing with interband transition. It is well assumed that the momentum states in both bands ($|ck\rangle$ for CB and $|vk\rangle$ for VB) has to be aligned ($\Delta k=0$), thus restricting the oblique transitions. Not only, for Berry connection this approximation is true to any light matter coupling operator ($h^{'}$) whose matrix element is defined as,
\begin{equation}\left\langle ck|\hat{h^{'}}|vk\right\rangle=\int{dr\Psi_{ck}^{*}(r)\hat{h^{'}}\Psi_{vk}(r)} 
\end{equation}
whose Bloch functions are denoted by $\psi$.
\section{Results}
%\textit{Discussion.\textbf{\textemdash}}
In order to explore the linear optical conductivity of the graphene, we use the  expressions (Eq.(\ref{intra_con}) and (\ref{inter_con})) to examine the real and imaginary parts of the optical conductivity with scattering parameters as well as doping levels. The real component of the linear response term, without considering scattering or thermal effects, consists only of the interband contribution as a simple step function, showing optical shielding behavior consistent with  the reported literature \cite{CSV141}. In our case, with the inclusion of finite scattering, the band terms are expressed independently for transport mechanism. The scattering rates are taken from experimental findings , $\mathit{\Gamma_e}=4.14$ meV, $\mathit{\Gamma_i}=0.414$ meV \cite{Malard}, and $\mathit{\Gamma_e}=0.5$ meV, $\mathit{\Gamma_i}=65$ meV \cite{gu}, and  employed for a systematic investigation of multiband collisional events while tuning the chemical potential.
A vital observation from Fig.\ref{Drude} is showcased in terms of Drude response for different values of chemical potentials and scattering rates. During inter-band scattering dominance  a strong Drude response occurs as shown in Fig.\ref{Drude}(a), is a general case for large availability of free charge carriers. Whereas in the Fig.\ref{Drude}(b), it is strikingly evident that the intraband scattering emerges as a new phenomena which suppresses the strong free carrier response as well as increasing the bandwidth. Through the variation of the chemical potential, the intensity of Drude behavior is modulated, driven by the interplay between interband and intraband scattering mechanisms. Further we examine the real conductivity response in the broad infrared frequency range where Pauli Blocking (PB) window starts to appear for different scattering rates under the influence of doping. Fig.\ref{conductivity}(a) is demonstrated with respect to PB effect and found to be more effective with the reduction of interband scattering rate $(\mathit{\Gamma_e}=10 \mathit{\Gamma_i})$ where the slope of the PB curves has sharp cutoff. With higher scattering rates $(\mathit{\Gamma_e}=130 \mathit{\Gamma_i})$ the PB curves slope decreases because of radiation leakage or interband losses, similar  behavior is reported by Hipolito et al. \cite{HTP1}.\\
  When we reverse the scattering dominance process to intraband, we observed that the Drude behavior is suppressed and found  emergence of  new conductivity peaks at the end of the PB region which is illustrated in Fig.\ref{conductivity}(c). It is evident that, the conductivity goes higher than universal conductivity value owing to the prevailing influence of intraband scattering with $\mathit{\Gamma_e} = 0.5$ meV and $\mathit{\Gamma_i} = 65$ meV. This feature is consistent for different doping values but the peak amplitude decreases exponentially with increasing doping concentration. Aforementioned unique feature emerges beyond a specific doping threshold value due to the opening of bandgap, facilitating bound charge carriers, while below this value, the dominance shifts to the free carrier Drude response. The higher intraband scattering dominance in our case is closely matching  with the experimental results obtained by Basov et al.\cite{Henriksen} at specific external bias conditions. These experiments were conducted at 28V gate bias voltage (Fermi level at 0.18 eV) at 45K, in which the PB curve is bending in the conductivity plot from 0.223 eV to 0.447 eV. Whereas in our case, calculations have produced a sharp change in response at 0.367 eV (line \textbf{A}) at zero Kelvin which is lying in region where the slope reaches the maximum as depicted in Fig.\ref{conductivity}(c). These electrically neutral quasi-particles participate in band renormalization and compete with the PB mechanism to produce such unique optical responses during weak Coulomb screening.  
%\vspace{0.2cm}
The imaginary part of the conductivity for different scattering rates and doping levels are plotted in Fig.\ref{conductivity}(b,d). The absorption peaks are logarithmic divergences around PB in the conductivity curves.  These peak height increases with doping and amplitude gets quenched for a high scattering rates. For direct band gap of two dimensional materials, the energy levels of carriers are shifted due to the interactions between electrons and impurities and enhances the absorption rates at higher doping levels. With the variation of chemical potential one can effectively modulate the absorption to desired frequencies.  From Fig.\ref{conductivity}(d), the peak at 0.2 eV correspond to Mid-Infrared Range is smoothly shifted to  Near-Infrared Range with small increment of chemical potential. For a higher interband scattering $(\mathit{\Gamma_e}=130\mathit{\Gamma_i})$ the broadening of absorption peak occurs. The smearing in the real part and the broadening in the imaginary part primarily stems from the incorporation of thermal effects, which are inherently considered within our calculations.

\section{Conclusion}
We have conducted a systematic analytical study on linear optical properties of monolayer graphene under low doping conditions to determine the interplay between interband $\&$ intraband scattering mechanisms. A collisional multiband Boltzmann equation was constructed using the length gauge over the velocity gauge to account for many-body effects. 
The Drude behavior can be tuned through the phenomenological scattering parameters $({\mathit{\Gamma_e,\Gamma_i}})$ and a high-amplitude quenching effect is observed for interband scattering events leading to significant bandwidth broadening in imaginary conductivity. During the dominance of intraband scattering  under low doping regime, we observed the emergence of new peak which beats the universal conductivity for lower frequency range of PB window in the real conductivity spectrum. The leap in response by excitonic effects in monolayer graphene is primarily triggered by dominant intraband scattering rather than interband scattering. It is very crucial that the material parameters of graphene such as carrier density, temperature, and impurity concentration, will dictate the importance of different scattering processes for transport dynamics.
\vspace{-0.3cm}
\section*{Acknowledgments}
We greatfully acknowledge support from DST Science and Engineering Research Board (SERB), India under grant No. SERB:CRG/2022/008749 and IIT Jodhpur Seed grant No.I/SEED/BMK/20230017. 
\vspace{-0.25cm}
\appendix*
\section{Kubo Expressions}
\label{app:section1}
In the following, we present some major steps for calculation of the conductivity. 
As long as the external field or the applied filed perturbation is of the same order as the measured response, we are in the linear regime.  From the CMBE, the first order equation is deduced as follows,
\begin{multline}
\hbar \frac{\partial \rho_{n_1n_2k}^{\!(1)}(t)}{\partial t}=-i(\varepsilon_{n_1k}-\varepsilon_{n_2k})\rho^{\!(1)}_{n_1 n_2k}(t)-\Gamma_{n_1n_2}^{\!(1)}\rho_{n_1n_2k}^{\!(1)}(t)\\
+ie\textbf{E}(t).\left[\bm{\xi}_k,\rho_k^{\!(0)}\right]_{n_1n_2}-e\textbf{E}(t).\nabla_k\rho_k^{\!(0)}
\label{1st_CMBE}
\end{multline}
The solution in Eq.(\ref{solution}) is an initial guess where time derivative has only one time dependent variable,
\begin{equation}
\frac{\partial \rho_{n_1n_2k}^{\!(1)}(t)}{\partial t}=-i\omega_1\rho_{n_1n_2k}^{\!(1)}(t)
\label{derivative}
\end{equation}
If Eq.(\ref{derivative}) is substituted in the first order CMBE (Eq.(\ref{1st_CMBE})), the expression becomes,
\begin{multline}
-i\hbar\omega_1\rho_{n_1n_2k}^{\!(1)}(t) =
-i\hbar\omega_{n_1n_2k}\rho_{n_1n_2k}^{\!(1)}(t)-\Gamma_{n_1n_2}^{\!(1)}\rho_{n_1n_2k}^{\!(1)}(t)\\
+ie\textbf{E}(t).\left[\bm{\xi}_k,\rho_k^{\!(0)}\right]_{n_1n_2}-e\textbf{E}(t).\nabla_k\rho_k^{\!(0)}
\end{multline}
where, the two band frequency is actually the difference between any two bands ($\omega_{n_1n_2}=\omega_{n_1}-\omega_{n_2}$) and the phenomenological constant is actually dependent on the signs of $n_1$ and $n_2$.
Eventually, from the Fourier transform of electric field one can find the final form with the density matrix.
\begin{multline}
\rho_{n_1n_2k}^{\!(1)}(t)=(-e)\bigintsss{\frac{d\omega_1}{2\pi}E_{\omega_1}^ce^{-i\omega_1t}}\\
\times\frac{\left(\left[\xi^c_k,\rho_k^{\!(0)}\right]_{n_1n_2}+\frac{\partial}{\partial k_c}\rho_{n_1n_2k}^{\!(0)}\right)}{\hbar\omega_1-\hbar\omega_{n_1n_2k}+i\Gamma_{n_1n_2}^{\!(1)}}
\label{amplitude}
\end{multline}
From the above expression of linear amplitude one can separate out the interband (first term) and intraband (second term ) components.  Then inverse transform of Eq.(\ref{amplitude}) is substituted  in Eq.(\ref{formula_conductivity}) to get the first order conductivity expression which is a crucial step in matching the results of Kubo formalism .
%(Appendix \ref{app:section2}). 
Formulated within the proximity of Dirac nodes (see Supplemental Material \cite{supp}) first the intraband expression is established using the energy terms as follows:
$\varepsilon_{1k}$=$+\hbar v_Fk$=$\varepsilon$ and $\varepsilon_{2k}$=$-\hbar v_Fk$=$-\varepsilon$. Similarly, the velocity terms are given by: $v_{11k}^x$=$+v_{F}{k_x}/{k}$ and $v_{22k}^x$=$-v_{F}{k_x}/{k}$.
Eventually, the intraband linear conductivity expression can be written as,
\begin{equation}
\sigma_{intra}^{\!(1)xx}(\omega) =-e^2\int{\frac{d\textbf{k}}{4\pi^2}[v_{11k}^x\mathcal{S}_{11k}^{ \!(1);x}(\omega)}+
v_{22k}^x\mathcal{S}_{22k}^{ \!(1);x}(\omega)]    
\end{equation}
The intraband frequency coefficient is now placed in the above expression along with the approximations taken around the Dirac point,
%\begin{widetext}
\begin{multline}
\sigma_{\text{intra}}^{(1)xx}(\omega) = -e^2 \int \frac{d\textbf{k}}{4\pi^2} \biggl[ \frac{iv_{11k}^x}{\hbar\omega+i\Gamma_i^{(1)}} \left( \frac{\partial f(\varepsilon)}{\partial \varepsilon} \frac{\partial \varepsilon}{\partial k_x} \right) \\
+ \frac{iv_{22k}^x}{\hbar\omega+i\Gamma_i^{(1)}} \left(\frac{\partial f(-\varepsilon)}{\partial \varepsilon} \frac{\partial \varepsilon}{\partial k_x} \right)\biggr]
\label{B2}
\end{multline}
where, $\rho_{n_1n_2k}^{\!(0)}$=$f(+\varepsilon)$ or $f_{1k}$ is the Fermi-Dirac Distribution function. 
If the intraband velocity matrix elements are substituted, the expression can be rewritten as,
\begin{multline}
 \sigma_{intra}^{\!(1)xx}(\omega) =ie^2\int\frac{d\textbf{k}}{4\pi^2} \left(\frac{k_x}{k}\right)^2 \frac{v_F}{k}\\
 \times\frac{hv_Fk}{(\hbar\omega+i\Gamma_i^{\!(1)})} \left(\frac{\partial f(-\varepsilon)}{\partial\varepsilon}-\frac{\partial f(\varepsilon)}{\partial\varepsilon}\right)
\end{multline}
The transformation from momentum space to energy space is carried out to produce one half of the result obtained by Kubo expression,
\begin{equation}
\sigma_{intra}^{\!(1)xx}(\omega) =\frac{ie^2}{\pi\hbar(\hbar\omega+i\Gamma_i^{\!(1)})}\int\limits_0^{\infty}\varepsilon\left(\frac{\partial f(-\varepsilon)}{\partial\varepsilon}-\frac{\partial f(\varepsilon)}{\partial\varepsilon}\right)d\varepsilon
\label{Kubo_intra}
\end{equation}
Above conversion can also be carried out using an identity which is only applicable for two-dimensional isotropic materials given by, 
\begin{equation}
    \int{\frac{d\textbf{k}}{4\pi^2} \left(\frac{k_x}{k}\right)^2 \frac{v_F}{k}}=  \int{\frac{d\textbf{k}}{4\pi^2} \left(\frac{k_y}{k}\right)^2 \frac{v_F}{k}}=\frac{1}{\pi \hbar}\int\limits_0^{\infty}{d\varepsilon}
\end{equation}
The proposition of the cold semiconductor approximation is put forward with the aim of attaining outcomes that closely converge towards temperatures nearing absolute zero. Similarly, for the interband term the required velocity approximations are listed $v_{12k}^x$=$iv_{F}{k_y}/{k}$ and $v_{21k}^x$=$-iv_{F}{k_y}/{k}$ along with the Berry connection elements $\xi_{12k}^x$=${k_y}/{2k^2}$ and $\xi_{21k}^x$=${k_y}/{2k^2}$. From Eq.(\ref{formula_conductivity}) the interband linear conductivity expression is written as,
\begin{equation}
\sigma_{inter}^{\!(1)xx}(\omega)=-e^2\int{\frac{d\textbf{k}}{4\pi^2}[v_{12k}^x\mathcal{S}_{12k}^{ \!(1);x}(\omega)}+v_{21k}^x\mathcal{S}_{21k}^{ \!(1);x}(\omega)]
\end{equation}
The interband frequency coefficient values are now placed in the above expression to become,
\begin{multline}
\sigma_{inter}^{\!(1)xx}(\omega)=-e^2\int\frac{d\textbf{k}}{4\pi^2} \biggl[\frac{-v_{12k}^x\xi_{12k}^x\left(f_{1k}-f_{2k}\right)}{(\hbar\omega+i\Gamma_e^{\!(1)}-2\varepsilon)}\\
+\frac{v_{21k}^x\xi_{21k}^x\left(f_{1k}-f_{2k}\right)}{(\hbar\omega+i\Gamma_e^{\!(1)}+2\varepsilon))}\biggr]
\label{B8}
\end{multline}
If the interband velocity  matrix elements are substituted in the numerator with some simple steps the expression can be rewritten as,
 \begin{multline}
 \sigma_{inter}^{\!(1)xx}(\omega) =\frac{ie^2}{2}\int\frac{d\textbf{k}}{4\pi^2} \left(\frac{k_y}{k}\right)^2 \frac{v_F}{k}{\left[f(+\varepsilon)-f(-\varepsilon)\right] } \\ 
 \times\left[\frac{1}{(\hbar\omega+i\Gamma_e^{\!(1)}-2\varepsilon)}+\frac{1}{(\hbar\omega+i\Gamma_e^{\!(1)}+2\varepsilon)}\right]
 \end{multline}
Again, the transformation from momentum space to energy space is carried out to produce the other half of the result obtained by Kubo expression,
\begin{multline}
\sigma_{inter}^{\!(1)xx}(\omega)=\frac{ie^2}{\hbar\pi}\int\limits_{0}^{\infty}\frac{d\varepsilon}{2}\left[f(+\varepsilon)-f(-\varepsilon)\right]\\
\times\left[\frac{1}{(\hbar\omega+i\Gamma_e^{\!(1)}-2\varepsilon)}+\frac{1}{(\hbar\omega+i\Gamma_e^{\!(1)}+2\varepsilon)}\right]
\label{Kubo_inter}
\end{multline}
Equivalence between the solution obtained and the expression without accounting for any scattering phenomena \cite{CSV141} can be formally demonstrated through some easy mathematical steps. Another important criteria which needs to be considered during vanishing of the interband phenomenological constant towards the positive side otherwise the entire has to be taken in the reverse order which is not the target of our study. Furthermore, it should be noted that all the calculations hold true in the vicinity of absolute zero Kelvin. However, if necessary, appropriate adjustments to the chemical potential can be made to accommodate variations outside this temperature range. The indirect dependence of the scattering constant on temperature can be elucidated by expressing it in terms of curve fitting. Employing this approach, a more lucid comprehension of the thermal effects is attained, enabling the observation of temperature-related patterns and behaviors in the scattering constant.
%\end{widetext}
\nocite{*}
\bibliography{apssamp}% Produces the bibliography via BibTeX.


%%%%%%%%%% Merge with supplemental materials %%%%%%%%%%
\pagebreak
\widetext
\maketitle
\definecolor{dimmedtext}{RGB}{0,0,0}
\begin{center}
\textbf{ \large  Supplementary Material: Unconventional optical response in monolayer graphene
due to dominant intraband scattering }
\end{center}
%%%%%%%%%% Merge with supplemental materials %%%%%%%%%%
%%%%%%%%%% Prefix a "S" to all equations, figures, tables and reset the counter %%%%%%%%%%
\setcounter{equation}{0}
\setcounter{figure}{0}
\setcounter{table}{0}
\setcounter{page}{1}
\makeatletter
\renewcommand{\theequation}{S\arabic{equation}}
\renewcommand{\thefigure}{S\arabic{figure}}
\renewcommand{\bibnumfmt}[1]{[S#1]}
\renewcommand{\citenumfont}[1]{S#1}
%%%%%%%%%% Prefix a "S" to all equations, figures, tables and reset the counter %%%%%%%%%%


\section*{Energy Terms}
\label{app:section1}
{\textcolor{dimmedtext}{
The tight binding wavefunction consists of set of all 2$p_z$ orbitals for a particular state with two atomic orbitals in the unit cell, given by
\begin{equation} 
\phi_A(r)=\frac{1}{\sqrt{N}}\sum\limits_{R}e^{ik.R}\varphi_A(r-R)
\label{A1}
\end{equation}
The normalisation constant (N) is the total number unit cells present per unit area and these Wannier orbitals $(\varphi)$ are well localised around respective atomic sites (R) of Bravais lattice. To ensure the periodicity, another parameter ($\tau$) is included with the position which relates site of one sublattice to other sublattice. If it is absent for one site, it must be non zero for the other one.
The wavefunction corresponding to the electrons in real crystal can be written in terms of tight binding wavefunctions for two inequivalent sites A and B  denoted by $l$. The band indices are dropped to make the expressions visibly simpler,
\begin{equation}
    \Psi_k(r)=a_{k}\phi_A(r)+b_{k}\phi_B(r)
             =\sum\limits_{l=1}^2C_{lk}\phi_l(r)           
\end{equation}
Using Eq.(\ref{A1}), the wavefunction is written as,
\begin{equation}
\Psi_k(r)=\frac{1}{\sqrt{N}}e^{ik.r}\sum\limits_{R}e^{ik.(R-r)}\sum\limits_{l=1}^2\varphi_i(r-R)C_{lk} 
=\frac{1}{\sqrt{N}}e^{ik.r}u_k(r)
\label{periodic_term}
\end{equation}
This newly formed state with the periodic term is the Bloch function used as eigenvectors for the unperturbed Hamiltonian. The main approximation of the tight binding model is to present the respective Bloch functions with the aid of well localised atomic orbitals. Using the k-space transformation the unperturbed Hamiltonian is given by, 
\begin{equation}
\hat{H_0}=\sum\limits_{i}\varepsilon_Aa_i^{\dag}a_i+\sum\limits_{j}\varepsilon_Bb_j^{\dag}b_j+\sum\limits_{<i,j>}t_{ij}a_i^{\dag}b_j\\
+\sum\limits_{<i,j>}t_{ij}b_j^{\dag}a_i
     =\sum\limits_{k}c_k^{\dag}H_{ij}(k)c_k
     \label{A.4}
     \end{equation}
 The transfer integral matrix for monolayer graphene from Eq.(\ref{A.4}) with zero onsite energy is, 
    \begin{equation}{H_{ij}}=
     \begin{pmatrix}0 & \gamma_0s(k) \\\gamma_0s^{*}(k) & 0 \end{pmatrix}
     \label{A.5}
     \end{equation}
where, $c_{nk}^{\dag}$ and $c_{nk} $ operators are linear combinations of $a_k$ and $b_k$.
In the above equation, s(k) is the structure factor dependent upon positions in real space.
\begin{equation}
  s(k) =\sum\limits_me^{ik.\delta_m} = e^{ik_y a}+2e^{-i\frac{k_y a}{2}} \cos{\frac{\sqrt{3}}{2}k_xa}
\end{equation}
The measure of distance for particles in real space is denoted by $\delta_m $ and t is the hopping parameter equivalent to -$\gamma_0$, respectively. If the Dirac point K$(K')$ is given by $(\pm4\pi/{3\sqrt{3}a},0)$, the structure factor in the neighborhood of $K'$ can be written as,  
\begin{equation}
  s(K'+k) =e^{-ik_ya}+2e^{-i\frac{k_y a}{2}} \cos\left(\frac{\sqrt{3}}{2}k_xa-\frac{2\pi}{3}\right)
\end{equation}
To the lowest order in k, we achieve the final form,
\begin{equation}
  s(K'+k) =\frac{3a}{2}(k_x+ik_y)
\end{equation}
After performing diagonalization on Eq.(\ref{A.5}), we obtain the following eigenvalues,
\begin{equation}
    \varepsilon_{nK'+k}=\gamma_0|s(K'+k)|=\frac{3a\gamma_0}{2}|k|
\end{equation}
This is the case when there is negligence of overlap between orbitals on different sites. Based on the NN approximation the electron hopping is only allowed for one of the three neighbouring lattice sites. The energy is stated with the substitution of Fermi velocity $v_F=(3a\gamma_0/2\hbar) $, finally leading to a linear dispersion relation near the Dirac points.
\begin{equation}
    \varepsilon_{nK+k}=\pm \hbar v_F k
\end{equation}
}}
\section*{Velocity Terms}
{\textcolor{dimmedtext}{
As per requirement for any inter or intra-band parts we must start to mention the band indices (n) which is crucial for bifurcating the velocity components.  Due to the orthonormal nature of the wavefunction, it follows these following conditions,\\\\
Normalisation condition: 
\begin{equation}
\int{dr\Psi_{n_1k_1}^{*}(r)\Psi_{n_2k_2}(r)}=\delta_{n_1n_2}\delta(k_1-k_2)
\label{A11}
\end{equation}
Matrix element formula:
\begin{equation}
    \int{dr\Psi_{n_1k_1}^{*}(r)A\Psi_{n_2k_2}(r)}=A_{n_1n_2k_2}\delta(k_1-k_2)
    \label{A12}
    \end{equation}
By applying Eq.(\ref{A11}) the Hamiltonian matrix component is given by,
\begin{equation}
    H_{n_1n_2k_1k_2}=\int{dr\Psi_{n_1k_1}^{*}(r)H\Psi_{n_2k_2}(r)}=\varepsilon_{n_1k_1}\delta_{n_1n_2}\delta(k_1-k_2) 
 \end{equation} 
Now, let's simplify for the first half of the Blount expression.
\begin{equation}
    [\bm{\xi}_{k_2},H_{k_1k_2}]_{n_1n_2}=\left(\varepsilon_{n_2k_2}-\varepsilon_{n_1k_1}\right)
    \int{dr\Psi_{n_1k_1}^{*}(r)\bm{\xi}_{k_2}\Psi_{n_2k_2}(r)}
\end{equation}
Using Eq.(\ref{A12}), the commutation relation becomes,
\begin{equation}
[\xi_{k_2},H_{k_1k_2}]_{n_1n_2}=\left(\varepsilon_{n_2k_1}-\varepsilon_{n_1k_1}\right)\bm{\xi}_{n_1n_2k_1}\delta(k_1-k_2)
\label{A14}
\end{equation}
Consequently, we possess all the requisite expressions essential for constructing the velocity matrix element. We can start with the Heisenberg equation of motion followed by the use of Eq.(\ref{A12}) and Eq.(\ref{A14}) respectively,
\begin{flalign}
&\int dr \Psi_{n_1k_1}^*(r) v \Psi_{n_2k_2}(r) \nonumber \\
&= \int dr \Psi_{n_1k_1}^*(r) \frac{1}{i\hbar} [r, H] \Psi_{n_2k_2}(r) \nonumber \\
&= \frac{1}{i\hbar} \left( [\bm{\xi}_{k_2}, H_{k_2}]_{n_1,n_2} + i\nabla_k H_{n_1n_2k_2} \right) \nonumber \\
&= \frac{-i}{\hbar} \left( (\varepsilon_{n_2k_1} - \varepsilon_{n_1k_1}) \bm{\xi}_{n_1n_2k_1} + i\nabla_k \varepsilon_{n_1k_1} \delta_{n_1n_2} \right) \delta(k_1 - k_2) \nonumber \\
&= v_{n_1n_2k_1} \delta(k_1 - k_2)
\end{flalign}
The components consist of an intraband term  $(n_1=n_2)$ represented by \(\nabla_k\varepsilon_{n_1k_1}\delta_{n_1n_2}/{\hbar}\), and an interband term $(n_1\neq n_2)$ given by \(i\omega_{n_1n_2k_1}\bm{\xi}_{n_1n_2k_1}\), both of which are discussed collectively,
\begin{equation}
v_{n_1n_2k_1}=\frac{1}{\hbar}\nabla_k\varepsilon_{n_1k_1}\delta_{n_1n_2}+i\omega_{n_1n_2k_1}\bm{\xi}_{n_1n_2k_1}
\end{equation}
}}
\section*{Connection Terms}
{\textcolor{dimmedtext}{
After the evaluation of velocity matrix elements, the next exercise is solely focused to express Berry connections in a second quantised notation which can be further simplified to evaluate the approximation done near K points for the interband part. Specifically, in the context of condensed matter physics, the Berry's connection is related to the area of the unit cell through the expression for the Berry phase. If the subscript "uc" in area $A_{uc}$ signifies the unit cell of the crystal lattice, then the connection formula is given by,
\begin{equation}\bm{\xi}_{n_1n_2k}=i\int{\frac{dr}{A_{uc}}u_{n_1k}^{*}(r)\nabla_ku_{n_2k}(r)}
\end{equation}\\
As defined earlier in Eq.(\ref{periodic_term}), the periodic part of Bloch function is found to be, 
\begin{equation}
u_{nk}(r) = \sum_{R} e^{ik \cdot (R - r)} \left[ \varphi(r - R - \tau_A) a_{k}^n \right.
+ \left. \varphi(r - R - \tau_B) b_{k}^n \right]
\label{Periodic_second}
\end{equation}
Some general rules followed by the Wannier orbitals which will be used in the next steps are listed below,\\\\
Normalisation condition:
\begin{equation}
\int{\varphi^{*}(r-R-\tau_A)\varphi(r-R-\tau_A)dr}=1
\label{A.19}
\end{equation}
Orthogonal condition:
\begin{equation}\int{\varphi^{*}(r-R-\tau_A)\varphi(r-R-\tau_B)dr}=0\end{equation}
Space matrix element that links two Bloch functions situated at the zone-center of periodic lattice can be vague. The inherent nature of the problem highlights that the unit cell's dipole moment is influenced by the specific unit cell choice and is not detached from it.\\\\
Meaninglessness of space matrix element:
\begin{equation}\int{\varphi^{*}(r-R-\tau_A)(r-R-\tau_A)\varphi(r-R-\tau_A)dr}=0
\label{A.21}
\end{equation}
Let's begin with the gradient of $u_{n_2k}(r)$,
\begin{multline} 
\nabla_ku_{n_2k}(r)=\sum\limits_{R}i(R-r)e^{ik.(R-r)}\sum\limits_{R}e^{ik.(R-r)}
[\varphi(r-R-\tau_A)a_{k}^{n_2}+\varphi(r-R-\tau_B)b_{k}^{n_2}]\\+\sum\limits_{R}e^{ik.(R-r)}
\varphi(r-R-\tau_A)\nabla_ka_{k}^{n_2}
+\sum\limits_{R}e^{ik.(R-r)}\varphi(r-R-\tau_B)\nabla_kb_{k}^{n_2}
\label{A.22}
\end{multline}
The final expression will be a 2x2 matrix with the off diagonal elements getting null values due to the orthogonal relation.
So only, the diagonal elements [AA,BB] has to be disentangled. Placing Eq.(\ref{Periodic_second}) and (\ref{A.22}) in the connection formula, there should be two terms remaining for each of the diagonal elements, 
\begin{multline}
i\int{\frac{dr}{A_{uc}}u_{n_1k}^{*}(r)\nabla_ku_{n_2k}(r)}|_{AA_1}=
\frac{1}{A_{uc}}a_{k}^{n_1\dag}\sum\limits_{R'R}e^{ik.(R-R')}\\ \times
\int\varphi^{*}(r-R'-\tau_A)
(r-R)\varphi(r-R-\tau_A)dr a_k^{n_2}   
\end{multline}
%\end{widetext}
From Eq.(\ref{A.21}), the above equation can be written as,
\begin{equation}
    i\int{\frac{dr}{A_{uc}}u_{n_1k}^{*}(r)\nabla_ku_{n_2k}(r)}|_{AA_1}
    =\frac{1}{A_{uc}}a_{k}^{n_1\dag}\sum\limits_{R}\int{\varphi^{*}(r-R-\tau_A)\tau_A\varphi(r-R-\tau_A)dr}a_k^{n_2}
\end{equation}
Using Eq.(\ref{A.19}), it can be further simplified to,
    \begin{equation}
   i\int{\frac{dr}{A_{uc}}u_{n_1k}^{*}(r)\nabla_ku_{n_2k}(r)}|_{AA_1}
   =a_{k}^{n_1\dag}\tau_Aa_k^{n_2}
   \label{A.25}
    \end{equation}
and
\begin{equation}
    i\int{\frac{dr}{A_{uc}}u_{n_1k}^{*}(r)\nabla_ku_{n_2k}(r)}|_{AA_2}=a_{k}^{n_1\dag}i\nabla_ka_k^{n_2}
\end{equation}
If the exact exercise is done for the B sub-lattice we get,
\begin{equation} i\int{\frac{dr}{A_{uc}}u_{n_1k}^{*}(r)\nabla_ku_{n_2k}(r)}|_{BB_1}=a_{k}^{n_1\dag}\tau_Bb_k^{n_2}b_k^{n_2}\end{equation}
and
\begin{equation} i\int{\frac{dr}{A_{uc}}u_{n_1k}^{*}(r)\nabla_ku_{n_2k}(r)}|_{BB_2}=b_{k}^{n_1\dag}i\nabla_kb_k^{n_2}
\label{A.28}
\end{equation}
Coupling Eq.(\ref{A.25}-\ref{A.28}) in Eq.(\ref{Periodic_second}), we can easily obtain,
\begin{equation}  \bm{\xi}_{n_1n_2k}=a_{k}^{n_1\dag}(\tau_A+i\nabla_k)a_k^{n_2}+b_{k}^{n_1\dag}(\tau_B+i\nabla_k)b_k^{n_2}
\label{final_berry}
\end{equation}
The dipole interaction Hamiltonian which encompasses the necessary matrix element, for each sublattice is given by,
\begin{equation} 
\bm{\xi}_{n_1n_2k}=\begin{pmatrix}a_{k}^{n_1\dag} & b_{k}^{n_1\dag}\end{pmatrix}\begin{pmatrix}\tau_A+i\nabla_k & 0 \\0 & \tau_B+i\nabla_k \end{pmatrix}\begin{pmatrix}a_k^{n_2} \\b_k^{n_2} \end{pmatrix}
\end{equation}
It states the connection between two neighbouring wavefunctions which is applicable for both intraband and interband cases (Fig.\ref{berry_plot}), where the eigenvectors from Eq.(\ref{A.4}) are given by,
\begin{equation}\begin{pmatrix}a_k^{n_2} \\b_k^{n_2} \end{pmatrix}=\frac{1}{\sqrt2}\begin{pmatrix}\pm\frac{s(k)}{|s(k)|} \\1 \end{pmatrix}\end{equation}
% Figure environment removed
}}

%\begin{thebibliography}{11}

%\end{thebibliography}
\end{document}

