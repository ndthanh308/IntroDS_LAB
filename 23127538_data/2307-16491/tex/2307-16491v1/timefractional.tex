%% filename: amsart-template.tex
%% version: 1.1
%% date: 2014/07/24
%%
%% American Mathematical Society
%% Technical Support
%% Publications Technical Group
%% 201 Charles Street
%% Providence, RI 02904
%% USA
%% tel: (401) 455-4080
%%      (800) 321-4267 (USA and Canada only)
%% fax: (401) 331-3842
%% email: tech-support@ams.org
%% 
%% Copyright 2008-2010, 2014 American Mathematical Society.
%% 
%% This work may be distributed and/or modified under the
%% conditions of the LaTeX Project Public License, either version 1.3c
%% of this license or (at your option) any later version.
%% The latest version of this license is in
%%   http://www.latex-project.org/lppl.txt
%% and version 1.3c or later is part of all distributions of LaTeX
%% version 2005/12/01 or later.
%% 
%% This work has the LPPL maintenance status `maintained'.
%% 
%% The Current Maintainer of this work is the American Mathematical
%% Society.
%%
%% ====================================================================

%     AMS-LaTeX v.2 template for use with amsart
%
%     Remove any commented or uncommented macros you do not use.

\documentclass{amsart}
%\usepackage{graphicx}
%\usepackage{lipsum}

%\renewcommand{\baselinestretch}{1.05}

%\setlength{\textheight}{23.0cm}
%\setlength{\textwidth}{16.5cm}
%\setlength{\headsep}{0.5cm}% probably not wanted with amsart
%\setlength{\footskip}{1.0cm}
%\calclayout

\usepackage{latexsym,amssymb}
\usepackage{amsmath}
\usepackage{amsthm} 
\usepackage[mathscr]{eucal}
\usepackage{color}
\usepackage[abbrev]{amsrefs}
\usepackage[T1]{fontenc}


\usepackage{latexsym}
\usepackage{amssymb}
\usepackage{amsfonts}
\usepackage{enumerate}
\usepackage{mathrsfs}
\usepackage{color}
\usepackage{comment}

%\usepackage{showkeys}
%\usepackage{refcheck}

\newtheorem{Theorem}{\bf Theorem}[section]
\newtheorem{Lemma}{\bf Lemma}[section]
\newtheorem{Proposition}{\bf Proposition}[section]
\newtheorem{Corollary}{\bf Corollary}[section]
\newtheorem{Remark}{\bf Remark}[section]
\newtheorem{Example}{\bf Example}[section]
\newtheorem{Definition}{\bf Definition}[section]


\newenvironment{theorem}{\begin{Theorem}$\!\!\!$}{\end{Theorem}}
\newenvironment{lemma}{\begin{Lemma}$\!\!\!$}{\end{Lemma}}
\newenvironment{proposition}{\begin{Proposition}$\!\!\!$}{\end{Proposition}}
\newenvironment{corollary}{\begin{Corollary}$\!\!\!$}{\end{Corollary}}
\newenvironment{remark}{\begin{Remark}$\!\!\!$}{\end{Remark}}
\newenvironment{example}{\begin{Example}$\!\!\!$}{\end{Example}}
\newenvironment{definition}{\begin{Definition}$\!\!\!$}{\end{Definition}}

\numberwithin{equation}{section}



\newcommand{\usub}{\underline{u}}
\newcommand{\usup}{\overline{u}}
\newcommand{\vsub}{\underline{v}}
\newcommand{\vsup}{\overline{v}}
\newcommand{\wsub}{\underline{w}}
\newcommand{\wsup}{\overline{w}}
\newcommand{\N}{\mathbb{N}}
\newcommand{\x}{\bar{x}}
\newcommand{\2}{\mathfrak{2}}

\DeclareMathOperator{\supp}{\operatorname{supp}}
\DeclareMathOperator{\dist}{\operatorname{dist}}



\def\Xint#1{\mathchoice
{\XXint\displaystyle\textstyle{#1}}%
{\XXint\textstyle\scriptstyle{#1}}%
{\XXint\scriptstyle\scriptscriptstyle{#1}}%
{\XXint\scriptscriptstyle\scriptscriptstyle{#1}}%
\!\int}
\def\XXint#1#2#3{{\setbox0=\hbox{$#1{#2#3}{\int}$}
\vcenter{\hbox{$#2#3$}}\kern-.5\wd0}}
\def\ddashint{\Xint=}
\def\dashint{\Xint-}

\begin{document}
	
	\title[time-fractional semilinear heat equation]
	{On solvability of a time-fractional semilinear heat equation, and its quantitative approach to the classical counterpart}
	
	%    Remove any unused author tags.
	
	%    author one information
	\author{Kotaro Hisa}
	\address[Kotaro Hisa]{Mathematical Institute, Tohoku University, 6-3 Aoba, Aramaki, Aoba-ku, Sendai 980-8578, Japan.}
	%\curraddr{}
	\email{kotaro.hisa.d5@tohoku.ac.jp}
	%\thanks{}
	
	
	
	\author{Mizuki Kojima}%$^{\ast}$}
	\address[Mizuki Kojima]{Department of Mathematics, Tokyo Institute of Technology,
		2-12-1 Ookayama, Meguro-ku, Tokyo 152-8551, Japan. }
	%\curraddr{}
	\email{kojima.m.aq@m.titech.ac.jp}
	%\thanks{$^\ast$Corresponding author.}
	\thanks{The second author was supported by JST, the establishment of university fellowships towards the creation of science technology innovation, Grant Number JPMJFS2112.}
	
	
	\subjclass[2020]{Primary 35R11, Secondary 35K15}
	
	\keywords{Fractional differential equation, Semilinear heat equation, blow-up, global existence.}
	
	%\date{}
	
	%\dedicatory{}
	
	%%RUNNING HEAD
	%\pagestyle{myheadings}
	%\markboth{M. Kojima and K. Hisa}{nonnegative solutions of a time-fractional semilinear heat equation}
	
	\begin{abstract}
		We discuss the existence and nonexistence of nonnegative local and global-in-time solutions of  the time-fractional  problem 
		\[
		\partial_t^\alpha u -\Delta u = u^p,\quad  t>0,\,\,\, x\in{\bf R}^N, \qquad
		u(0) = \mu  \quad \mbox{in}\quad {\bf R}^N,
		\]
		where  $N\geq1$,  $0<\alpha<1$, $p>1$,  and
		$\mu$ is a nonnegative Radon measure on ${\bf R}^N$. 
		Here, $\partial_t^\alpha$ is the Caputo derivative of order $\alpha$. The corresponding usual equation $\partial_tu-\Delta u=u^p$ may not be globally or locally-in-time solvable, under certain critical situations. In contrast,  the solvability of the time-fractional equation is guaranteed, under such situations. In this paper, we deduce necessary and sufficient conditions on the initial data $\mu$ for the solvability of this equation. As application, we describe the collapse of the global and local-in-time solvability for the time-fractional equation as $\alpha \to1-0$.
		
	\end{abstract}
	
	\maketitle
	%%%%%%%%%%%%%%%%%%%%%%%%%%%%%%%%%%%%%
	%%%%%%%%%%%%%%%%%%%%%%%%%%%%%%%%%%%%%
	\section{Introduction}
	%%%%%%%%%%%%%%%%%%%%%%%%%%%%%%%%%%%%%
	%%%%%%%%%%%%%%%%%%%%%%%%%%%%%%%%%%%%%
	%%%%%%%%%%%%%%%%%%%%%%%%%%%%%%%%%%%%%
	\subsection{Introduction.}
	%%%%%%%%%%%%%%%%%%%%%%%%%%%%%%%%%%%%%
	We are interested in the existence and nonexistence of nonnegative solutions of the time-fractional semilinear parabolic equation
	\begin{equation}
	\label{TFF}
	\left\{
	\begin{split}
	\partial_t^\alpha u -\Delta u &= u^p\ &&\mbox{in}\ (0,T)\times{\bf R}^N,\\
	u(0) &= \mu\  &&\mbox{in}\ \ {\bf R}^N,\\
	\end{split}
	\right.
	\end{equation}
	where $N\geq1$, $T>0$, $p>1$, and
	$\mu$ is a nonnegative Radon measure on ${\bf R}^N$. 
	Here, $\partial_t^\alpha$ is the Caputo derivative of order $\alpha\in (0,1)$, that is 
	\[
	(\partial_t^\alpha f)(t) := \frac{1}{\Gamma(1-\alpha)} \frac{d}{dt} \int_0^t \frac{f(s)-f(0)}{(t-s)^{\alpha}}\,ds,
	\] 
	where $\Gamma$ is the usual Gamma function. This mathematical tool has been proposed to model the anomalous diffusion, which is different from the usual diffusion of materials based on the Brownian motion. For the mathematical treatment of time-fractional equations, see e.g. \cite{Podl99, GalWar20, LuchYama19}.
	
	Let us recall the case of $\alpha=1$, that is the Fujita-type equation
	\begin{equation}
	\label{Fjt}
	\partial_t  u -\Delta u = u^p,\quad  t>0,\,\,\, x\in{\bf R}^N, \qquad
	u(0) = \mu  \quad \mbox{in}\quad {\bf R}^N.
	\end{equation}
	The solvability of the Cauchy problem \eqref{Fjt} has been studied extensively by many mathematicians since the pioneering work due to Fujita \cite{Fujita66}. (See e.g., \cite{QuitSoup19}, which is a book including a good list of references for problem \eqref{Fjt}.) In what follows, set $p_F:=1+2/N$. This exponent works as the criteria for the global-in-time solvability of problem \eqref{Fjt}. Namely, the condition $1<p\le p_F$ implies the nonexistence of any global-in-time solutions, while $p>p_F$ guarantees the global-in-time solvability of the equation.
	Furthermore, under the doubly critical case, i.e., $p=p_F$, and $\mu\in L^1({\bf R}^N)$, it is known that problem \eqref{Fjt} may not have any local-in-time solutions for certain singular initial data. See \cite{BreCaz96, Miya21}. Note that \eqref{Fjt} is scale invariant in $L^1({\bf R}^N)$ if $p=p_F$.
	
	For any $z\in{\bf R}^N$ and $r>0$, let $B(z,r):=\{y\in{\bf R}^N; |x-y|<r\}$. About more detailed result for the solvability of \eqref{Fjt}, the first author of this paper and Ishige \cite{HisaIshige18} proved the following necessary conditions for the existence of nonnegative solutions:
	\begin{enumerate}[(i)]
		\item If problem \eqref{Fjt} possesses a nonnegative solution in $[0,T)\times{\bf R}^N$ for some $T>0$, then there exists $C_1=C_1(N,p)>0$ such that
		\[
		\sup_{z\in{\bf R}^N} \mu(B(z,\sigma)) \le C_1 \sigma^{N-\frac{2}{p-1}},\quad 0<\sigma<T^\frac{1}{2}.
		\]
		
		\item In the case of $p=p_F$, there exists $C_2=C_2(N)>0$ such that
		\[
		\sup_{z\in{\bf R}^N} \mu(B(z,\sigma)) \le C_2 \left[\log\left(e+\frac{T^\frac{1}{2}}{\sigma}\right)\right]^{-\frac{N}{2}},\quad 0<\sigma<T^\frac{1}{2}.
		\]
	\end{enumerate}
	See also \cite{BarasPierre85}. These necessary conditions are in fact the optimal estimates.
	
	We go back to problem \eqref{TFF}. This problem is known to have interesting aspects different from problem \eqref{Fjt}. Zhang and Sun \cite{ZhanSun15} showed that the problem \eqref{TFF} is not globally-in-time solvable if $1<p<p_F$, while the equation possesses a global-in-time solution if $p>p_F$.
	Interestingly, they also showed that \eqref{TFF} possesses a global-in-time solution for a certain initial value even if $p=p_F$. Moreover, \cite{GMS22, ZLS19} showed that the problem \eqref{TFF} is locally-in-time solvable for any $\mu \in L^1({\bf R}^N)$ in the case of $p=p_F$ (i.e., doubly critical situation), while the problem \eqref{Fjt} cannot be solved in this case. 
	Formally speaking, in the critical situation, the solvability of the time-fractional problem \eqref{TFF} is easier to obtain than the usual one \eqref{Fjt}. However, if the Caputo derivative is a naturally expanded concept of the usual one, then problems \eqref{TFF} and \eqref{Fjt} should be connected in some ways when $\alpha\to 1-0$.
	
	\begin{table}[h]\label{tab. comparison}
		\caption{Comparison of the solvability of the two equations with $p=p_F$.}
		\centering
		\begin{tabular}{ccc}
			\hline
			Equation & Global solution & Local solution for $\mu\in L^{1}({\bf R}^N)$ \rule[0mm]{0mm}{5mm}\\
			\hline \hline
			\eqref{Fjt} with $p=p_F$ &not exist& $\exists \mu\ge0$ admits no nonnegative sol. \rule[0mm]{0mm}{5mm}\\
			\eqref{TFF} with $p=p_F$ &exists& always exists\rule[0mm]{0mm}{5mm}\\
			\hline
		\end{tabular}
	\end{table}
	
	In the first part of this paper, we derive the analogous result of \cite{HisaIshige18} to the time-fractional Fujita equation \eqref{TFF}. In particular, optimal conditions for the solvability of \eqref{TFF} are presented.
	The second part of this paper is devoted to the application of the results. In particular, we focus on the behavior as $\alpha\to 1-0$ and formulate a collapse of the global and local-in-time solvability of problem \eqref{TFF} in critical situations.
	
	
	%%%%%%%%%%%%%%%%%%%%%%%%%%%%%%%%%%%%%
	\subsection{Notation and the definition of solutions.}
	%%%%%%%%%%%%%%%%%%%%%%%%%%%%%%%%%%%%%
	
	In order to state our main results, we introduce some notation and formulate the definition of solutions.
	For $t>0$ and $x\in{\bf R}^N$,  let $G$ be the fundamental solution of the linear heat equation in $(0,\infty)\times{\bf R}^N$, that is
	\[
	G(t,x) := \frac{1}{(4\pi t)^\frac{N}{2}} \exp\left(-\frac{|x|^2}{4t}\right).
	\]
	For $t>0$, $x\in{\bf R}^N$, and the Radon measure $\mu$ on ${\bf R}^N$, define
	\[
	e^{t\Delta} \mu(x) := \int_{{\bf R}^N} G(t, x-y) \,d\mu(y).
	\]
	We often identify a locally integrable function $\phi$ in ${\bf R}^N$ with the Radon measure $\phi\,dx$. Let us define
	\[
	P_{\alpha}(t):= \int_0^\infty h_\alpha (\theta) e^{t^\alpha \theta \Delta} \, d\theta, \quad S_{\alpha} (t) := \int_0^\infty \theta h_\alpha (\theta ) e^{t^\alpha\theta \Delta} \, d\theta
	\]
	for $t>0$ and $0<\alpha<1$. Here, $h_\alpha$ is a certain probability density function on $(0,\infty)$ which satisfies $h_\alpha>0$. In addition, we have
	\begin{equation}
	\label{eq:1.4}
	\int_0^\infty \theta^{\delta} h_\alpha(\theta) \, d\theta = \frac{\Gamma(1+\delta)}{\Gamma(1+\alpha\delta)}<\infty \quad \mbox{for} \quad \delta>-1
	\end{equation}
	and
	\begin{equation}
	\label{eq:1.4.5}
	\int_0^\infty h_\alpha (\theta ) e^{z\theta} \,d\theta = E_{\alpha,1} (z), \quad 
	\int_0^\infty \alpha \theta h_\alpha (\theta ) e^{z\theta} \,d\theta = E_{\alpha,\alpha} (z), 
	\quad \mbox{for} \quad z\in{\bf  C}.
	\end{equation}
	Here, $E_{\alpha,\beta}$ is the Mittag--Leffler functions, that is, for $\alpha,\beta \in{\bf  C}$ with
	${\rm Re}(\alpha)>0$ and $z\in{\bf  C}$,
	\[
	E_{\alpha,\beta}(z) := \sum_{k=0}^\infty \frac{z^k}{\Gamma(\alpha k+ \beta)}.
	\]
	%
	For $a,b>0$, let $B(a,b)$ be the Beta function, that is
	\[
	B(a,b):= \int_0^1(1-s)^{a-1} s^{b-1}\,ds.
	\]
	
	Next, we formulate the definition of solutions of problem \eqref{TFF}.
	\begin{definition}
		\label{Def:1.1}
		Let $u$ be a nonnegative measurable function in $(0,T]\times{\bf R}^N$, where $0<T<\infty$. We say that $u$ is a solution 
		of \eqref{TFF} in $[0,T]\times{\bf R}^N$ if u satisfies
		\[
		\infty> u(t,x) = [P_\alpha(t)\mu] (x) +\alpha\int_0^t (t-s)^{\alpha-1} [S_\alpha(t-s)u(s)^p](x) \,ds
		\]
		for almost all $t\in(0,T]$ and $x\in{\bf R}^N$.
		If $u$ satisfies the above equality with $=$ replaced by $\ge$, then $u$ is said to be a supersolution of \eqref{TFF} in $[0,T]\times{\bf R}^N$.
	\end{definition}
	%
	%%%%%%%%%%%%%%%%%%%%%%%%%%%%%%%%%%%%%
	\subsection{Main results.}
	%%%%%%%%%%%%%%%%%%%%%%%%%%%%%%%%%%%%%
	Now we are ready to state our main results of this paper. In the first theorem, we give necessary conditions for the existence of solutions of \eqref{TFF}.
	\begin{theorem}
		\label{Thm:1.1}
		Suppose that problem \eqref{TFF}  possesses a nonnegative solution in $[0,T]\times{\bf R}^N$, where $0<T<\infty$.
		Then there exists a constant $\gamma_1=\gamma_1(N,p,\alpha)>0$ satisfying $\limsup_{\alpha\to1-0} \gamma_1(\alpha)\in(0,\infty)$  such that $\mu$ satisfies
		\begin{equation}
		\label{Thm:1.1.1}
		\sup_{z\in{\bf R}^N} \mu(B(z,\sigma)) \le \gamma_1 \sigma^{N-\frac{2}{p-1}}
		\end{equation}
		for all $\sigma \in (0,T^{\alpha/2}]$.  In addition, if $p=p_F$, then there exists a constant $\gamma'_1=\gamma'_1(N,\alpha)>0$ satisfying $\limsup_{\alpha\to1-0} \gamma'_1(\alpha)\in(0,\infty)$  such that $\mu$ satisfies
		\begin{equation}
		\label{Thm:1.1.2}
		\sup_{z\in{\bf R}^N} \mu(B(z,\sigma)) \le \gamma'_1 \left(\int_{\sigma^{2/\alpha}/(16T)}^{1/4} t^{-\alpha} \, dt \right)^{-\frac{N}{2}}
		\end{equation}
		for all $\sigma \in (0,T^{\alpha/2}]$. 
	\end{theorem}
	
	\begin{remark}{\rm
			\begin{enumerate}[(i)]
				\item Taking the limit $\alpha\to1-0$ in \eqref{Thm:1.1.2} yields
				\begin{equation}
				\label{HI}
				\sup_{z\in{\bf R}^N} \mu(B(z,\sigma)) \le C\left(\log\frac{4T}{\sigma^2}\right)^{-\frac{N}{2}} \le C\left[\log\left(e+\frac{T}{\sigma^2}\right)\right]^{-\frac{N}{2}}
				\end{equation}
				for all $\sigma \in (0,T^{1/2}]$. The right-hand side of the above estimate coincides with its counterpart for \eqref{Fjt}. See \cite[Theorem 1.1 (2)]{HisaIshige18}. However, the estimate \eqref{Thm:1.1.2} suggests more than the analogy to \eqref{HI}. Indeed, $T=\infty$ can be permitted for suitable initial data, for \eqref{Thm:1.1.2}. In contrast, \eqref{HI} with $T=\infty$ implies $\mu \equiv 0$.
				
				\item Zhang and Sun \cite{ZhanSun15} showed that, if $1<p<p_F$ and $\mu \not \equiv0$ in ${\bf R}^N$, then problem \eqref{TFF} possesses no global-in-time solutions. We observe that Theorem~\ref{Thm:1.1} leads the same conclusion as \cite{ZhanSun15}.
			\end{enumerate}
		}
	\end{remark}
	
	Next, we give sufficient conditions for the solvability of problem \eqref{TFF}.
	We modify the arguments \cite{Soup17} and  \cite{ZhanSun15} and prove the following two theorems.
	
	\begin{theorem}
		\label{Thm:1.2}
		Let $1<p\le p_F$ and $T>0$. Then there exists $\gamma_2=\gamma_2(N,p,\alpha)>0$ such that, if $\mu$ is a nonnegative Radon measure in ${\bf R}^N$ satisfying
		\begin{equation}\label{Thm:1.2.1}
		\sup_{z\in{\bf R}^N} \mu(B(z,T^\frac{\alpha}{2})) \le \gamma_2  T^{\alpha(\frac{N}{2}-\frac{1}{p-1})},
		\end{equation}
		then problem \eqref{TFF} possesses a solution in $[0,T]\times{\bf R}^N$. In particular, 
		if $p\neq p_F$, then $\limsup_{\alpha\to1-0} \gamma_2(N,p,\alpha)\in (0,\infty)$. Moreover,
		if $p=p_F$, then we have
		\begin{equation*}
		\gamma_2(N, p_F, \alpha)\le C (1-\alpha)^{\frac{N}{2}}
		\end{equation*}
		for appropriate $C=C(N)>0$.
	\end{theorem}
	
	\begin{remark}
		\label{Rem:1.2}
		{\rm
			Let $p=p_F$.
			\begin{enumerate}[(i)]
				\item It follows from Theorem \ref{Thm:1.2} that if $\mu$ satisfies
				\[
				\mu({\bf R}^N)\le \gamma_2,
				\]
				then problem \eqref{TFF} possesses a global-in-time solution.
				
				\item Zhang and Sun \cite{ZhanSun15} showed that problem \eqref{TFF} possesses a global-in-time solution for a certain initial data.
				We observe that Theorem~\ref{Thm:1.2} leads the same conclusion as \cite{ZhanSun15}.
			\end{enumerate}
		}
	\end{remark}
	
	\begin{remark}
		{\rm
			Ghergu, Miyamoto, and Suzuki \cite{GMS22} proved the existence of nonnegative solutions for arbitrary nonnegative $L^1$ initial values in the case of $p=p_F$. 
			Every  measurable function $f \in L^1({\bf R}^N)$ satisfies
			\[
			\lim_{R\to+0} \int_{B(z,R)} |f(y)| \, dy =0
			\]
			for all $z\in{\bf R}^N$. This together with Theorem \ref{Thm:1.2} implies that for every nonnegative $f\in L^1({\bf R}^N)$, there exists  $T>0$ such that  problem \eqref{TFF} with $u(0)=f$  possesses a solution in $[0,T]\times{\bf R}^N$ in the case of $p=p_F$.
			Therefore, Theorem \ref{Thm:1.2} leads the same conclusion as \cite{GMS22}.
			See also \cite{OkaZhan23}.
		}
	\end{remark}
	
	\begin{theorem}
		\label{Thm:1.3}
		Suppose that $p>p_F$ and $T>0$. Let $1<r<\infty$  be such that
		\[
		\frac{2r}{p-1}<N.
		\]
		Then there exists $\gamma_3=\gamma_3(N,p, r, \alpha)>0$ such that, if $\mu$ is a nonnegative measurable function in ${\bf R}^N$ satisfying
		\begin{equation}
		\label{Thm:1.3.1}
		\sup_{z\in{\bf R}^N}\left(\int_{B(z,\sigma)} \mu(y)^r \, dy\right)^\frac{1}{r}\le \gamma_3 \sigma^{\frac{N}{r}-\frac{2}{p-1}}
		\end{equation}
		for all $\sigma \in (0,T^{\alpha/2}]$, then problem \eqref{TFF} possesses a solution in $[0,T]\times{\bf R}^N$. Moreover, $\limsup_{\alpha\to1-0}\gamma_3(N, p, r, \alpha)\in (0,\infty)$.
	\end{theorem}
	
	\begin{remark}
		%Optimality
		{\rm
			We verify the optimality of our main theorems. Indeed, let $p>p_F$ and $\mu(x)=\kappa |x|^{-2/(p-1)}$. It follows from \eqref{Thm:1.1.1} in Theorem \ref{Thm:1.1} that there exists a sufficiently large constant $C^*=C^*(N,\alpha,p)>0$ such that, if $\kappa>C^*$, then problem \eqref{TFF} possesses no local-in-time solutions. Conversely, by Theorem \ref{Thm:1.3}, if $\kappa<c^*$
			for some sufficiently small $c^*>0$, then problem \eqref{TFF} possesses a local-in-time solution. 
		}
	\end{remark}
	
	\begin{remark}
		{\rm
			We prove Theorem~\ref{Thm:1.2} and Theorem~\ref{Thm:1.3} by the contraction mapping theorem in the uniformly local Lebesgue and local Morrey space respectively. In fact, Oka and Zhanpeisov \cite{OkaZhan23} recently constructed the theory of solvability of \eqref{TFF} in the Bezov-Morrey framework, which is a more general concept than the local Morrey space. However, we aim to clearly exhibit the estimate of the existence time, rather than construct the functional analytical theory.
		}
	\end{remark}
	
	
	%%%%%%%%%%%%%%%%%%%%%%%%%%%%%%%%%%%%%
	\subsection{Behavior as $\alpha\to 1-0$}
	%%%%%%%%%%%%%%%%%%%%%%%%%%%%%%%%%%%%%
	
	It has already been mentioned that there are significant differences, in particular under the critical conditions, between the solvability of the usual Fujita equation \eqref{Fjt} and the time-fractional model \eqref{TFF}. In Section~\ref{section:Application}, we represent the collapse of the global and local-in-time solvability of problem \eqref{TFF} when $\alpha\to 1-0$.
	
	First, we focus on the global-in-time solvability of problem \eqref{TFF} as $\alpha\to 1-0$ in the case of $p=p_F$. Let $\mathcal{M}$ be a set of Radon measures on ${\bf R}^N$. Define
	\[
	\mathcal{G}_\alpha:= \{0\le \nu\in \mathcal{M}; \mbox{\eqref{TFF} with} \,\,u(0)=\nu \,\,\mbox{possesses a global-in-time solution.} \}.
	\]
	Here, $\nu\ge0$ means that $\nu\in \mathcal{M}$ is a nonnegative one.
	By Remark \ref{Rem:1.2} we see that $\mathcal{G}_\alpha\neq\phi$.
	We shall observe that $\mathcal{G}_{\alpha}$ \textit{shrinks to zero} as $\alpha\to 1-0$. More precisely, there exist $0<C_1<C_2$ such that
	\begin{equation*}
	C_1 (1-\alpha)^{\frac{N}{2}} \le \sup_{\nu\in\mathcal{G}_\alpha } \nu ({\bf R}^N) \le C_2(1-\alpha)^{\frac{N}{2}}
	\end{equation*}
	near $\alpha=1$. See Theorem~\ref{Thm:4.1}.
	
	Second, we discuss the local-in-time solvability of \eqref{TFF} in $L^{1}({\bf R}^N)$ in the case of $p=p_F$. Let
	\begin{equation}\label{eq. intro Miyamoto init data}
	f_{\epsilon}(x) :=|x|^{-N} \left|\log |x| \right|^{-\frac{N}{2}-1+\epsilon}\chi_{B(0,1/e)}
	\end{equation}
	with $0<\epsilon<N/2$. Miyamoto \cite{Miya21} showed that the $L^{1}({\bf R}^N)$ initial data $f_{\epsilon}$ admits no nonnegative local-in-time solutions of \eqref{Fjt} with $p=p_F$. Let $T_{\alpha}\left[\mu \right]$ be the lifespan of \eqref{TFF} with the initial data $\mu$. Then, we expect that $T_{\alpha}\left[ f_{\epsilon}\right]\to 0$ as $\alpha\to 1-0$. We shall justify that the intuition is true. Moreover, the optimal rate of the convergence $T_{\alpha}[f_{\epsilon}]\to 0$ shall be provided. See Theorem~\ref{Thm:4.2}.
	
	\vskip\baselineskip
	The rest of this paper is organized as follows.
	In Section~\ref{section:Necessary condition}, we prove Theorem~\ref{Thm:1.1}. The proof is essentially based on the method of ordinary differential inequalities introduced by \cite{LaiSie21} (see also \cite{FHIL23, HIT18}).
	In Section~\ref{section:Sufficient condition}, we prove Theorems \ref{Thm:1.2} and \ref{Thm:1.3}. When $\alpha=1$,  Theorems \ref{Thm:1.2} and \ref{Thm:1.3} coincide with \cite[Theorems 1.3 and 1.4]{HisaIshige18} for which supersolution method is used. However, our argument in the proof of Theorem \ref{Thm:1.3} is completely different from it. We introduce the uniformly local $L^q$ space and the local Morrey space $M^{q,\lambda}$ and apply the contraction mapping theorem to suitable functional spaces. 
	In Section~\ref{section. Collapse of the solvability}, we describe the collapse of the global and local-in-time solvability of problem \eqref{TFF} with $p=p_F$ as $\alpha \to 1-0$. 
	In Section~\ref{section:Application}, we apply our results to life span estimates for certain initial data.
	
	%%%%%%%%%%%%%%%%%%%%%%%%%%%%%%%%%%%%%
	%%%%%%%%%%%%%%%%%%%%%%%%%%%%%%%%%%%%%
	\section{Necessary conditions for the solvability.}\label{section:Necessary condition}
	%%%%%%%%%%%%%%%%%%%%%%%%%%%%%%%%%%%%%
	%%%%%%%%%%%%%%%%%%%%%%%%%%%%%%%%%%%%%
	In this section, we prove Theorem \ref{Thm:1.1}.
	We follow the argument \cite{FHIL23,HisaIshige18,LaiSie21} and obtain an inequality related to
	\[
	\int_{{\bf R}^N} G(t^\alpha,x) u(t,x) \,dx.
	\]
	
	\begin{proof}[Proof of Theorem \ref{Thm:1.1}]
		Suppose that $0<\rho\le (16^{-1}T)^{\alpha/2}$. Let $\overline{z} \in {\bf R}^N$ and $\overline{T}\in (T/4, T/2)$ be such that
		\begin{equation}
		\label{eq:3.1}
		\begin{split}
		\infty &> u(2\overline{T}, \overline{z} ) \\
		&\ge \alpha \int_0^{2\overline{T}} (2\overline{T}-s)^{\alpha-1}\int_0^\infty\theta h_\alpha(\theta)\\
		&\qquad\qquad\qquad\times\int_{{\bf R}^N} G((2\overline{T}-s)^\alpha \theta, \overline{z} -y) u(s,y)^p \, dyd\theta ds.
		\end{split}
		\end{equation}
		Set
		\begin{equation*}
		U(t):= \int_{{\bf R}^N} G(t^\alpha,x)u(t,x+\overline{z}) \,dx.
		\end{equation*}
		
		\noindent\textbf{Step 1.} First, we prove $U(t) <\infty $ for almost all $t\in (\rho^{2/\alpha},T/4]$.
		Applying the Jensen inequality to \eqref{eq:3.1} implies that
		\begin{equation}
		\label{eq:3.2}
		\begin{split}
		\infty &> u( 2\overline{T}, \overline{z}) \\
		& \ge C \alpha \overline{T}^{\alpha-1} \int_{\rho^{2/\alpha}}^{2\overline{T}} \int_{1}^{\infty}  h_\alpha(\theta)\int_{{\bf R}^N} G((2\overline{T}-s)^\alpha \theta,y) u(s,y+\overline{z})^p \, dyd\theta ds\\
		& \ge C \alpha \overline{T}^{\alpha-1} \int_{\rho^{2/\alpha}}^{\overline{T}} \int_{1}^{\infty} h_\alpha(\theta)\left(\int_{{\bf R}^N} G((2\overline{T}-s)^\alpha \theta,y) u(s,y+\overline{z}) \, dy\right)^p d\theta ds.
		\end{split}
		\end{equation}
		Since $s\le 2\overline{T}-s\le T$ for $0<s<\overline{T}(<T/2)$, we have
		\begin{equation}
		\label{eq:3.3}
		\begin{split}
		G((2\overline{T}-s)^\alpha \theta,y) 
		&= \frac{1}{(4\pi (2\overline{T}-s)^\alpha \theta)^{\frac{N}{2}} }\exp\left(-\frac{|y|^2}{4(2\overline{T}-s)^\alpha \theta}\right)\\
		&\ge \frac{1}{\theta^\frac{N}{2}}\left( \frac{s}{2\overline{T}-s}\right)^\frac{N\alpha}{2}\frac{1}{(4\pi s^\alpha )^{\frac{N}{2}} }\exp\left(-\frac{|y|^2}{4s^\alpha}\right)\\
		&\ge \frac{C\rho^N}{\theta^\frac{N}{2}T^\frac{N\alpha}{2}}G(s^\alpha,y)\\
		\end{split}
		\end{equation}
		for all $y\in{\bf R}^N$ and $s\in(\rho^{2/\alpha}, \overline{T})$.
		Combining \eqref{eq:3.2} and \eqref{eq:3.3}, we see that there exists a constant 
		$C_*=C_*(N,\alpha, p, \rho, T)>0$ such that
		\begin{equation*}
		\begin{split}
		\infty >  u(2\overline{T},\overline{z} )  &\ge C_* \int_{1}^{\infty} \theta^{-\frac{N}{2}}h_{\alpha}(\theta)d\theta \int_{\rho^{2/\alpha}}^{\overline{T}} U(s)^p \,ds\ge C_* \int_{\rho^{2/\alpha}}^{T/4} U(s)^p \,ds.
		\end{split}
		\end{equation*}
		This implies that $U(t)<\infty$ for almost all $t\in (\rho^{2/\alpha},T/4]$. Note that
		\[
		\infty>\int_{1}^{\infty} \theta^{-\frac{N}{2}} h_{\alpha}(\theta)d\theta>0.
		\]
		
		\noindent\textbf{Step 2.} For $\overline{z}\in{\bf R}^N$ and $t\in(\rho^{2/\alpha},T/4]$, set
		\begin{equation*}
		V(t):= t^\frac{N\alpha}{2}U(t), \qquad M:= \mu(B(\overline{z},\rho)).
		\end{equation*}
		%
		Second, we claim that
		\begin{equation}
		\label{eq:3.4}
		\begin{split}
		\infty>V(t) 
		&\ge C_1r_1(\alpha) M+	C_2 r_2(\alpha)T^{\alpha-1}\int_{\rho^{2/\alpha}}^t s^{-\frac{N\alpha}{2}(p-1)} V(s)^p \,ds
		\end{split}
		\end{equation}
		%
		for  almost all $t \in(\rho^{2/\alpha},T/4]$, where
		\[
		r_1(\alpha) = E_{\alpha,1} \left(-\frac{1}{2}\right)>0, \quad r_2(\alpha) =\alpha E_{\alpha,\alpha}  \left(-\frac{1}{2}\right)>0, 
		\]
		and $C_1, C_2$ are positive constants depending only on $N$ and $p$.
		By the definition of $E_{\alpha,\beta}(z)$, we see easily that
		\[
		\limsup_{\alpha\to1-0} r_i(\alpha) \in (0,\infty)
		\]
		for $i=1,2$. By Definition \ref{Def:1.1}, we have
		\begin{equation*}
		\begin{split}
		u(t,x+\overline{z}) 
		&= \int_0^\infty \int_{{\bf R}^N}h_\alpha(\theta) G(t^\alpha\theta, x+\overline{z}-y)\,d\mu(y)d\theta\\
		& +\alpha \int_0^t(t-s)^{\alpha-1} [S_\alpha (t-s) u(s)^p](x+\overline{z}) \, ds
		\end{split}
		\end{equation*}
		for almost all $t\in(\rho^{2/\alpha},T/4]$ and $x\in{\bf R}^N$.
		Multiplying $G(t^\alpha, x)$ by both side and integrating with respect to $x$, we obtain
		\begin{equation}
		\label{eq:2.5}
		\begin{split}
		\infty>U(t) &
		= \int_{{\bf R}^N} \left[\int_0^\infty h_\alpha (\theta) \left(\int_{{\bf R}^N} G(t^\alpha \theta, x+\overline{z}-y)G(t^\alpha,x)\,dx\right)\,d\theta\right]\,d\mu(y)\\
		&+\alpha \int_{{\bf R}^N} G(t^\alpha, x) \left(\int_0^t(t-s)^{\alpha-1}[S_\alpha(t-s)u(s)^p](x+z)\,ds\right)\,dx\\
		&=: I_1 + \alpha I_2
		\end{split}
		\end{equation}
		for almost all $t\in(\rho^{2/\alpha},T/4]$.
		For $I_1$, by the semigroup property of the heat kernel $G$, we deduce
		\begin{equation}
		\label{eq:2.6}
		\begin{split}
		I_1&= \int_{{\bf R}^N}\int_0^\infty h_\alpha (\theta) G(t^\alpha(1+\theta),\overline{z}-y)\,d\theta d\mu(y)\\
		&\ge C \left(\int_0^\infty h_\alpha (\theta) (1+\theta)^{-\frac{N}{2}}\, d\theta\right) t^{-\frac{N\alpha}{2}} \int_{B(z,\rho)} \exp\left(-\frac{|\overline{z}-y|^2}{4t^\alpha}\right)\,d\mu(y)\\
		&\ge C \left(\int_0^\infty h_\alpha (\theta) (1+\theta)^{-\frac{N}{2}}\,d\theta\right) t^{-\frac{N\alpha}{2}} M
		\end{split}
		\end{equation}
		for almost all $t\in(\rho^{2/\alpha},T/4]$.
		For $I_2$, similarly we use the semigroup property to get
		\begin{equation}
		\label{eq:2.7}
		\begin{split}
		I_2 = \int_0^t (t-s)^{\alpha-1} \int_{{\bf R}^N}\int_0^\infty \theta h_\alpha(\theta) G(t^\alpha+(t-s)^\alpha\theta, \overline{z}-y)u(s,y)^p\, d\theta dyds.
		\end{split}
		\end{equation}
		%
		Since $t>s$, we have
		\begin{equation*}
		\begin{split}
		&\exp\left(-\frac{|y-\overline{z}|^2}{4t^\alpha + 4(t-s)^\alpha \theta}\right) \ge \exp\left(-\frac{|y-\overline{z}|^2}{4s^\alpha}\right),\\
		&t^\alpha + (t-s)^\alpha \theta \le t^\alpha(1+\theta) = s^\alpha \left(\frac{t}{s}\right)^\alpha (1+\theta).
		\end{split}
		\end{equation*}
		This together with \eqref{eq:2.5}, \eqref{eq:2.6}, \eqref{eq:2.7}, and the Jensen inequality implies that
		\begin{equation}
		\label{eq:2.8}
		\begin{split}
		\infty>U(t) &\ge C_1 \left(\int_0^\infty h_\alpha (\theta) (1+\theta)^{-\frac{N}{2}} \, d\theta \right) t^{-\frac{N\alpha}{2}}M\\
		& + C_2\alpha  T^{\alpha-1} \left(\int_{0}^\infty \theta h_\alpha(\theta) (1+\theta)^{-\frac{N}{2}}\,d\theta\right) t^{-\frac{N\alpha}{2}} \int_{\rho^{2/\alpha}}^t s^{\frac{N\alpha}{2}} U(s)^p\,ds
		\end{split}
		\end{equation}
		for almost all $t\in(\rho^{2/\alpha},T/4]$.
		Since the function $\theta \mapsto e^{-\theta/2}(1+\theta)^{N/2}$ is bounded on $[0,\infty)$, we use \eqref{eq:1.4.5} to deduce 
		\begin{equation*}
		\begin{split}
		&\int_0^\infty h_\alpha (\theta) (1+\theta)^{-\frac{N}{2}} \, d\theta \ge \int_0^\infty h_\alpha (\theta) e^{-\frac{\theta}{2}} \, d\theta = E_{\alpha,1}\left(-\frac{1}{2}\right) = r_1(\alpha),\\
		&\alpha \int_{0}^\infty \theta h_\alpha(\theta) (1+\theta)^{-\frac{N}{2}}\,d\theta \ge \alpha \int_0^\infty \theta h_\alpha (\theta) e^{-\frac{\theta}{2}} \, d\theta = \alpha E_{\alpha,\alpha}\left(-\frac{1}{2}\right) = r_2(\alpha).\\
		\end{split}
		\end{equation*}
		Therefore \eqref{eq:2.8} yields \eqref{eq:3.4}.
		
		\noindent\textbf{Step 3.} We may then let $\zeta$ denote the unique local solution of the integral equation
		\begin{equation}
		\label{eq:3.5}
		\zeta(t) = C_1r_1(\alpha) M+	C_2 r_2(\alpha)T^{\alpha-1}\int_{\rho^{2/\alpha}}^t s^{-\frac{N\alpha}{2}(p-1)} \zeta(s)^p \,ds
		\end{equation}
		%
		for all $t \in(\rho^{2/\alpha},T/4]$. 
		Hence, $\zeta$ is the unique local solution of 
		\begin{equation*}
		\zeta'(t) = C_2 r_2(\alpha) T^{\alpha-1}t^{-\frac{N\alpha}{2}(p-1)} \zeta(t)^p, \qquad \zeta(\rho^\frac{2}{\alpha}) = C_1r_1(\alpha) M.
		\end{equation*}
		%
		By \eqref{eq:3.4}, applying the standard theory for ordinary differential equations to \eqref{eq:3.5},
		we see that the solution $\zeta$ exists in $[\rho^{2/\alpha}, T/4]$.
		This ordinary differential equation is easy to solve.
		It follows  that
		\[
		C_2r_2(\alpha) T^{\alpha-1} \int_{\rho^{2/\alpha}}^{T/4} t^{-\frac{N\alpha}{2}(p-1)} \, dt = \int_{\zeta(\rho^{2/\alpha})}^{\zeta(T/4)} \zeta^{-p}\,d\zeta 
		\le \int_{\zeta(\rho^{2/\alpha})}^{\infty} \zeta^{-p}\,d\zeta. 
		\]
		This implies that
		\[
		\mu(B(\overline{z},\rho))=M\le \frac{C}{r_1(\alpha) r_2(\alpha)^\frac{1}{p-1}} T^{-\frac{\alpha-1}{p-1}} \left(\int_{\rho^{2/\alpha}}^{T/4} t^{-\frac{N\alpha}{2}(p-1)} \, dt \right)^{-\frac{1}{p-1}}
		\]
		for all  $\rho\in (0, (T/4)^{\alpha/2}]$. 
		Setting $\sigma:= 4^\alpha \rho$ yields
		\begin{equation}
		\label{eq:3.6}
		\mu(B(\overline{z},4^{-\alpha}\sigma)) \le \frac{C}{r_1(\alpha) r_2(\alpha)^\frac{1}{p-1}} T^{-\frac{\alpha-1}{p-1}} \left(\int_{\sigma^{2/\alpha}/16}^{T/4} t^{-\frac{N\alpha}{2}(p-1)} \, dt \right)^{-\frac{1}{p-1}}
		\end{equation}
		for all $\sigma \in (0,T^{\alpha/2}]$. 
		Let $z\in{\bf R}^N$.
		Then we can take integer $m=m(N,\alpha)$ satisfying $\limsup_{\alpha\to1-0} m(\alpha)<\infty$,
		and $\{\overline{z}_i\}_{i=1}^m\subset {\bf R}^N$  satisfying \eqref{eq:3.6} such that 
		\begin{equation*}
		\begin{split}
		\mu(B(z,\sigma)) 
		&\le \sum_{i=1}^m \mu(B(\overline{z}_i,4^{-\alpha}\sigma)) \\
		&\le \frac{Cm}{r_1(\alpha) r_2(\alpha)^\frac{1}{p-1}} T^{-\frac{\alpha-1}{p-1}} \left(\int_{\sigma^{2/\alpha}/16}^{T/4} t^{-\frac{N\alpha}{2}(p-1)} \, dt \right)^{-\frac{1}{p-1}}
		\end{split}
		\end{equation*}
		for all $\sigma \in (0,T^{\alpha/2}]$. 
		Since $z\in{\bf R}^N$ is arbitrary, we obtain \eqref{Thm:1.1.2}.
		Furthermore, applying the above argument with $T = \sigma^{2/\alpha}$, we obtain \eqref{Thm:1.1.1}.
	\end{proof}
	
	
	%%%%%%%%%%%%%%%%%%%%%%%%%%%%%%%%%%%%%
	%%%%%%%%%%%%%%%%%%%%%%%%%%%%%%%%%%%%%
	\section{Sufficient conditions for the solvability.}\label{section:Sufficient condition}
	%%%%%%%%%%%%%%%%%%%%%%%%%%%%%%%%%%%%%
	%%%%%%%%%%%%%%%%%%%%%%%%%%%%%%%%%%%%%
	
	
	%%%%%%%%%%%%%%%%%%%%%%%%%%%%%%%%%%%%%
	\subsection{Preliminaries.}
	%%%%%%%%%%%%%%%%%%%%%%%%%%%%%%%%%%%%%
	Oka and Zhanpeisov \cite{OkaZhan23} already constructed the theory of solvability of \eqref{TFF} in the Bezov-Morrey space framework. However, in order to make this paper self-contained, and clearly exhibit the estimate of the existence time, we discuss the solvability of \eqref{TFF} in the Morrey framework here. In order to prove Theorem \ref{Thm:1.2},
	we introduce the uniformly local  $L^q$ space.
	For details, see e.g. \cite{MaeTera06, FujiIoku18, QuitSoup19}. In what follows, we denote by $\| \cdot | X\|$ a norm in a Banach space $X$.
	\begin{definition}
		Let $1\le p\le\infty$. 
		$L^p_{uloc}=L^p_{uloc}({\bf R}^N)$ is defined as the set of the  measurable function $f$ in ${\bf R}^N$ such that
		\[
		\|f|L^p_{uloc}\| := \sup_{z\in{\bf R}^N} \|f|L^p(B(z,1))\|<\infty.
		\]
	\end{definition}
	
	We note that $L^p_{uloc}({\bf R}^N)$ is the Banach space with norm $\|\cdot|L^p_{uloc}\|$.
	Furthermore, $M^1=M^1({\bf R}^N)$ is defined as the set of the  Radon measure $\mu$ on ${\bf R}^N$ such that
	\[
	\|\mu|M^1\| := \sup_{z\in{\bf R}^N} |\mu|(B(z,1)) <\infty,
	\]
	where $|\mu|$ is the total variation of $\mu$.
	The following proposition is key to the proof of Theorem \ref{Thm:1.2}.
	\begin{proposition}
		\label{LpLq} 
		Let $1\le q \le p \le\infty$. One has 
		\begin{equation*}
		\|e^{t\Delta} \mu|L^p_{uloc}\| \le C\left(1+t^{-\frac{N}{2}(\frac{1}{q}-\frac{1}{p})}\right) \|\mu|L^q_{uloc}\|
		\end{equation*}
		for $\mu\in L^{q}_{loc}({\bf R}^N)$ and $t>0$. In particular, if $q=1$, one has
		\begin{equation*}
		\|e^{t\Delta} \mu|L^p_{uloc}\| \le C\left(1+t^{-\frac{N}{2}(1-\frac{1}{q})}\right) \|\mu|M^1\|
		\end{equation*}
		for $\mu\in M^1({\bf R}^N)$ and $t>0$. 
	\end{proposition}
	\begin{proof}
		The first estimate is proved by \cite[Theorem 3.1]{MaeTera06}.
		We can prove the second one in the same way with $g\equiv1$ and $dy=d\mu(y)$.
	\end{proof}
	
	In order to prove Theorem \ref{Thm:1.3}, we introduce the local Morrey-type space. For details, see \cite{GM89, Kato92, KozoYama94}.
	\begin{definition}
		Let $1\le q< \infty$ and $0<\lambda \le N$. 
		The local Morrey space $M^{q,\lambda}=M^{q,\lambda}({\bf R}^N)$ is defined as the set of the measurable functions $f$ in ${\bf R}^N$ such that
		\[
		\|f|M^{q,\lambda}\| := \sup_{z\in{\bf R}^N} \sup_{R\in(0,1]} R^{\frac{1}{q}(\lambda-N)} \|f|L^q(B(z,R))\|<\infty.
		\]
	\end{definition}
	We note that $M^{q,\lambda}({\bf R}^N)$ is a Banach space with norm $\|\cdot|M^{q,\lambda}\|$.
	%\begin{definition}
	%Let $0<\lambda \le N$. 
	%The local measure space of the Morrey type $M^{\lambda}=M^{\lambda}({\bf R}^N)$ is defined as the set of the  Radon measures $\mu$ on ${\bf R}^N$ such that
	%\[
	%\|\mu\|_{M^{\lambda}} := \sup_{z\in{\bf R}^N} \sup_{R\in(0,1]} R^{\lambda-N}|\mu|(B(z,R))<\infty.
	%\]
	%\end{definition}
	%We note that $M^{\lambda}({\bf R}^N)$ is a Banach space with norm $\|\cdot\|_{M^{\lambda}}$.
	The following proposition is key to the proof of Theorem \ref{Thm:1.3}.
	%We can prove this proposition by a similar way to the proof of  \cite[Proposition 49.18]{QS} and \cite{Souplet}.
	\begin{proposition}
		\label{MpMq} 
		Let $1\le q \le p <\infty$ and $0< \lambda \le N$. Then there exists a constant $C>0$ such that
		\begin{equation}
		\label{MPMQ1}
		\|e^{t\Delta} \mu|M^{p,\lambda}\| \le C (1+t^{-\frac{\lambda}{2}(\frac{1}{q}-\frac{1}{p})}) \|\mu|M^{q,\lambda}\|
		\end{equation}
		for $\mu\in M^{q,\lambda}({\bf R}^N)$ and $t>0$. 
		%Furthermore, 
		%\begin{equation}
		%\label{MPMQ2}
		%\|e^{t\Delta} \mu\|_{M^{p,\lambda}} \le C (1+t^{-\frac{\lambda}{2}(1-\frac{1}{p})}) \|\mu\|_{M^{\lambda}}
		%\end{equation}
		%for $\mu\in M^{\lambda}({\bf R}^N)$ and $t>0$. 
	\end{proposition}
	
	In order to prove Proposition \ref{MpMq}, we recall the inhomogeneous Besov--Morrey space introduced by Kozono--Yamazaki \cite{KozoYama94}.
	Let $\zeta(t)$ be a $C^\infty$-function on $[0,\infty)$ such that $0\le \zeta(t)\le 1$, $\zeta(t)\equiv1$ for $t\in3/2$, and $\mbox{supp}\,\zeta \subset[0,5/3)$.
	For every $j\in{\bf Z}$, put $\varphi(\xi) := \zeta(2^{-j}|\xi|)-\zeta(2^{1-j}|\xi|)$ and $\varphi_{(0)}(\xi):=\zeta(|\xi|)$.
	Then we have $\varphi_{j}, \varphi_{(0)}\in C_{0}^\infty({\bf R}^N)$ for all $j\in{\bf Z}$, and 
	\[
	\varphi_{(0)}(\xi) + \sum_{j=1}^\infty \varphi_j(\xi) =1 \quad \mbox{for all} \quad \xi\in{\bf R}^N.
	\]
	
	\begin{definition}
		Let $1\le q \le p <\infty$, $1\le r \le \infty$, and $s\in{\bf R}$.
		The local Besov--Morrey space $N^s_{p,q,r}=N^s_{p,q,r}({\bf R}^N)$ is defined as the set of distributions  $f\in \mathcal{S}'$ such that
		\[
		\|f|{N^s_{p,q,r}}\| : = \|\mathcal{F}^{-1}\varphi_{(0)}(\xi)\mathcal{F}f|M^{q,\lambda}\| + \left\| \{2^{sj} \|\mathcal{F}^{-1} \varphi_j(\xi) \mathcal{F}f|M^{q,\lambda}\|\}_{j=1}^\infty|l^r\right\| <\infty,
		\]
		where $\lambda := qN/p \le N$ and $\mathcal{F}$ denotes the Fourier transform on ${\bf  R}^N$.
		%Here, $\|f|{N^s_{p,q,r}}\| $ denotes the norm of $f$ in $N^s_{p,q,r}({\bf R}^N)$.
	\end{definition}
	
	Next, we prepare two lemmas to prove Proposition \ref{MpMq}, which have been proved by Kozono--Yamazaki \cite[Theorem 2.5, Proposition 2.11, and Theorem 3.1]{KozoYama94}.
	
	\begin{lemma}
		Let $1\le q'\le p' <\infty$, $1\le r\le\infty$,  $s\in{\bf R}^N$, and $\theta\in(0,1)$.
		Then the following embeddings are continuous:
		\begin{equation}
		\label{em1}
		N^{s}_{p',q',r} \hookrightarrow N^{s-\frac{N}{p'}(1-\theta)}_{\frac{p'}{\theta}, \frac{q'}{\theta}, r},
		\end{equation}
		\begin{equation}
		\label{em2}
		N^{0}_{p',q',1} \hookrightarrow M^{q', \frac{q'N}{p'}} \hookrightarrow N^{0}_{p',q',\infty}.
		\end{equation}
	\end{lemma}
	
	\begin{lemma}
		Let $s<\sigma$ and $1\le q' \le p' <\infty$. Then there exists a constant $C>0$ such that
		\begin{equation}
		\label{smooth}
		\|e^{t\Delta} f|{N^\sigma_{p',q',1}}\| \le C (1+t^{\frac{s-\sigma}{2}}) \|f|{N^s_{p',q',\infty}}\|
		\end{equation}
		for all $f\in N^s_{p',q',\infty} $ and $t>0$.
	\end{lemma}
	
	\begin{proof}[Proof of Proposition \ref{MpMq}.]
		First, we prove \eqref{MPMQ1} in the case of $p=q$.
		Let $\mu \in M^{p,\lambda}$, $z\in {\bf R}^N$, and $R\in(0,1]$. 
		By the Jensen inequality, we have 
		\begin{equation*}
		\begin{split}
		\int_{B(z,R)} |e^{t\Delta} \mu (x)|^p\, dx
		& = \int_{B(z,R)}\left|\int_{{\bf R}^N} G(x-y,t) \mu(y)\,dy\right|^p\, dx\\
		& \le\int_{B(z,R)} \int_{{\bf R}^N} G(x-y,t) |\mu(y)|^p\,dydx\\
		& =   \int_{B(z,R)}\int_{{\bf R}^N} G(y,t) |\mu(x-y)|^p\,dydx\\
		&= \int_{{\bf R}^N} G(y,t) \int_{B(z,R)}|\mu(x-y)|^p\,dxdy\\
		&\le R^{N-\lambda} \|\mu|M^{p,\lambda}\|^p \int_{{\bf R}^N} G(y,t)\,dy\\
		&=  R^{N-\lambda} \|\mu|M^{p,\lambda}\|^p.
		\end{split}
		\end{equation*}
		Since $z\in{\bf R}^N$ and $R\in(0,1]$ are arbitrary, we have
		\[
		\|e^{t\Delta} \mu (x)|M^{p,\lambda}\| \le  \|\mu|M^{p,\lambda}\|.
		\]
		The proof \eqref{MPMQ1} in the case of $p=q$ is complete.
		
		
		
		Next, we prove \eqref{MPMQ1} the case of $p>q$.
		By \eqref{em2}  with 
		\[
		p'= \frac{pN}{\lambda} \quad \mbox{and} \quad q' = p,
		\]
		%($A\to B$ means "substituting $B$ into $A$"),
		we have 
		\begin{equation}
		\label{eq:em4}
		\begin{split}
		\|e^{t\Delta} \mu|M^{p,\lambda}\|
		& \le C \left\|e^{t\Delta} \mu |{N^0_{\frac{pN}{\lambda}, p,1}}\right\|\\
		\end{split}
		\end{equation}
		for all $t>0$. Note that
		\[
		N^{0}_{\frac{pN}{\lambda}, p,1} = 
		N^{N(1-\frac{q}{p})\frac{\lambda}{qN}-N(1-\frac{q}{p})\frac{\lambda}{qN}}_{\frac{qN}{\lambda}\frac{p}{q}, q \frac{p}{q},1}.
		\]
		By \eqref{em1} with
		\[
		\theta = \frac{q}{p}\in(0,1), \quad p'=\frac{qN}{\lambda}, \quad q'= q, \quad r=1,\quad s=N\left(1-\frac{q}{p}\right)\frac{\lambda}{qN},
		\]
		we have
		\begin{equation}
		\label{eq:em5}
		\left\|e^{t\Delta} \mu|{N^0_{\frac{pN}{\lambda}, p,1}}\right\| \le C \left\|e^{t\Delta} \mu|{N^{\lambda(\frac{1}{q}-\frac{1}{p})}_{\frac{qN}{\lambda}, q,1}}\right\|
		\end{equation}
		for all $t>0$.
		By  \eqref{em2} and \eqref{smooth} with 
		\[
		p'= \frac{qN}{\lambda}, \quad q'= q, \quad r=\infty, \quad s=0, \quad \sigma = \lambda\left(\frac{1}{q}-\frac{1}{p}\right)>0,
		\]
		\eqref{eq:em4}, and \eqref{eq:em5}, we have 
		\begin{equation*}
		%\label{eq:3.5}
		\begin{split}
		\|e^{t\Delta} \mu|M^{p,\lambda}\| \le C  \left\|e^{t\Delta} \mu|{N^{\lambda(\frac{1}{q}-\frac{1}{p})}_{\frac{qN}{\lambda}, q,1}}\right\|
		&\le C (1+t^{-\frac{\lambda}{2}(\frac{1}{q}-\frac{1}{p})})\left\| \mu|{N^{0}_{\frac{qN}{\lambda}, q,\infty}}\right\|\\
		&\le C (1+t^{-\frac{\lambda}{2}(\frac{1}{q}-\frac{1}{p})})\| \mu|M^{q,\lambda}\|\\
		\end{split}
		\end{equation*}
		for all $t>0$. We obtain \eqref{MPMQ1} and the proof is complete.
	\end{proof}
	%%%%%%%%%%%%%%%%%%%%%%%%%%%%%%%%%%%%%
	\subsection{In the case of $p\le p_F$.}
	%%%%%%%%%%%%%%%%%%%%%%%%%%%%%%%%%%%%%
	
	%\begin{proposition}
	%\label{MpMq} 
	%Let $1\le r \le q <\infty$. One has 
	%\begin{equation*}
	%\|e^{t\Delta} \mu\|_{M^q} \le C t^{-\frac{N}{2}(\frac{1}{r}-\frac{1}{q})} \|\mu\|_{M^r},
	%\end{equation*}
	%for $\mu\in M^{q}({\bf R}^N)$ and $t>0$.
	%\end{proposition}
	%\begin{proof}
	%
	%\end{proof}
	\begin{proof}[Proof of Theorem \ref{Thm:1.2}.]
		It suffices to consider the case of $T=1$. Indeed, for any solution $u$ of \eqref{TFF} in $[0,T]\times{\bf R}^N$, where $0<T<\infty$, we see that $u_\lambda (x,t) := \lambda^{2\alpha/(p-1)}u(\lambda^2 t , \lambda^\alpha x)$ with $\lambda:= \sqrt{T}$ is a solution of \eqref{TFF} in $[0,1]\times{\bf R}^N$.
		The proof is based on the contraction mapping theorem.
		Since $1<p\le p_F$, note ${N(p-1)/2}\le1$.
		Then \eqref{Thm:1.2.1} leads
		\[
		\sup_{z\in{\bf R}^N}\mu(B(z,1))\le \gamma_2,
		\]
		where $\gamma_2$ is the constant as in Theorem~\ref{Thm:1.2}.
		Set
		\[X :=L^\infty((0,1);L^{p}_{uloc}({\bf R}^N))
		\]
		and
		\[
		\|u|X\|:= \sup_{t\in(0,1)} t^\beta \|u(t)|L^p_{uloc}\|, \quad \beta := \frac{N\alpha}{2p}(p-1) \le \frac{\alpha}{p}.
		\]
		For $u\in X$, define
		\[
		\Phi[u](t):= P_\alpha(t)\mu +\alpha\int_0^t (t-s)^{\alpha-1} S_\alpha(t-s)|u(s)|^p \,ds.
		\]
		Denote $B_M := \{u\in X;\|u|X\|\le M\}$ for $M>0$, which will be chosen later, and
		\[
		d(u,v) := \|u-v|X\|,
		\]
		for $u,v\in B_M$.
		Observe that  $(B_M,d)$ is a metric space.
		We will show that $\Phi: B_M \to B_M$. Let $u\in B_M$ and $t\in (0,1)$. 
		Since 
		\[
		-\frac{N}{2}\left(1-\frac{1}{p}\right)>-1
		\]
		for $1<p\le p_F$, \eqref{eq:1.4} yields
		\[
		\int_0^\infty h_\alpha(\theta) (1+\theta^{-\frac{N}{2}(1-\frac{1}{p})} )\, d\theta= 1+\frac{\Gamma\left(1-\frac{N}{2}(1-\frac{1}{p})\right)}{\Gamma\left(1-\frac{N\alpha}{2}(1-\frac{1}{p})\right)} <\infty
		\]
		and
		\[
		\int_0^\infty h_\alpha(\theta) (\theta+\theta^{1-\frac{N}{2}(1-\frac{1}{p})} )\, d\theta=\frac{\Gamma\left(2\right)}{\Gamma\left(1+\alpha\right)}+\frac{\Gamma\left(2-\frac{N}{2}(1-\frac{1}{p})\right)}{\Gamma\left(1+\alpha-\frac{N\alpha}{2}(1-\frac{1}{p})\right)} <\infty.
		\]
		In particular, these constants are finite near $\alpha=1$. Observe that $\|f^p|L^1_{uloc}\| =\|f|L^p_{uloc}\|^p$ for all $f\in L^p_{uloc}({\bf R}^N)$.
		It follows from Proposition \ref{LpLq} and \eqref{Thm:1.2.1} that
		\begin{equation*}
		\begin{split}
		t^\beta \|P_\alpha (t) \mu |L^p_{uloc}\| 
		&\le t^\beta \int_0^\infty h_\alpha(\theta) \|e^{t^\alpha \theta \Delta}\mu|L^p_{uloc}\| \, d\theta\\
		&\le C \|\mu|M^1\|t^\beta\int_0^\infty h_\alpha(\theta )(1+(t^\alpha \theta)^{-\frac{N}{2}(1-\frac{1}{p})})\, d\theta \\
		&\le C\|\mu|M^1\|t^{\beta-\frac{N\alpha}{2}(1-\frac{1}{p})} \int_0^\infty h_\alpha(\theta )(1+\theta^{-\frac{N}{2}(1-\frac{1}{p})})\, d\theta \\
		&\le C_1 \gamma_2
		\end{split}
		\end{equation*}
		%
		for $0<t\le1$. Similarly,
		\begin{equation*}
		\begin{split}
		&t^\beta \left\| \alpha\int_0^t (t-s)^{\alpha-1} S_\alpha(t)|u(s)|^p\,ds |L^p_{uloc}\right\| \\
		&\le \alpha t^\beta \int_0^t (t-s)^{\alpha-1} \|S_\alpha(t)|u(s)|^p|L^p_{uloc}\|\,ds \\
		&\le C  t^\beta  \int_0^t (t-s)^{\alpha-1} \int_0^\infty \theta h_\alpha(\theta) (1+[(t-s)^\alpha \theta]^{-\frac{N}{2}(1-\frac{1}{p})})\||u(s)|^p|L^1_{uloc}\| \,d\theta ds\\
		&\le CM^pt^{\alpha(1-\frac{N}{2}(p-1))} B\left(\alpha-\frac{N\alpha}{2}\left(1-\frac{1}{p}\right), 1-\frac{N\alpha}{2}(p-1)\right)\\
		&\le C_2M^pB\left(\alpha-\frac{N\alpha}{2}\left(1-\frac{1}{p}\right), 1-\frac{N\alpha}{2}(p-1)\right)
		\end{split}
		\end{equation*}
		for $0<t\le1$.  
		Note that $C_1$ and $C_2$ are independent of $\alpha$. Set
		\begin{equation}\label{gamma_2}
		\gamma_2= c_* B\left(\alpha-\frac{N\alpha}{2}\left(1-\frac{1}{p}\right), 1-\frac{N\alpha}{2}(p-1)\right)^{-\frac{1}{p-1}}
		\end{equation}
		and
		\[
		M= 2C_1 c_*B\left(\alpha-\frac{N\alpha}{2}\left(1-\frac{1}{p}\right), 1-\frac{N\alpha}{2}(p-1)\right)^{-\frac{1}{p-1}},
		\]
		where $c_*>0$. 
		Taking sufficiently small $c_*>0$ if necessary, it follows from the above inequalities that
		\begin{equation*}
		%\label{sub:2}
		\begin{split}
		\|\Phi[u]|X\| 
		&= \sup_{t\in(0,1)} t^\beta\|\Phi[u](t)|L^p_{uloc}\| \\
		&\le (C_1c_*+ 2^pC_1^pC_2c_*^p)B\left(\alpha-\frac{N\alpha}{2}\left(1-\frac{1}{p}\right), 1-\frac{N\alpha}{2}(p-1)\right)^{-\frac{1}{p-1}} \\
		&\le 2C_1c_*B\left(\alpha-\frac{N\alpha}{2}\left(1-\frac{1}{p}\right), 1-\frac{N\alpha}{2}(p-1)\right)^{-\frac{1}{p-1}} = M.
		\end{split}
		\end{equation*}
		This implies that $\Phi: B_M \to B_M$. 
		
		Next, we will show that $\Phi: B_M \to B_M$ is a contraction mapping. 
		Let $u,v\in B_M$.
		Similarly to the above computations, we have
		\begin{equation}
		\label{sub:4}
		\begin{split}
		& t^\beta \|\Phi[u](t)-\Phi[v](t)|L^p_{uloc}\|\\
		&\le \alpha t^\beta \int_0^t (t-s)^{\alpha-1} \int_0^\infty \theta h_\alpha(\theta) \|e^{(t-s)^\alpha \theta \Delta}(|u(s)|^p-|v(s)|^p)|L^p_{uloc}\| \, d\theta ds\\
		&\le  C t^\beta \int_0^t (t-s)^{\alpha-1-\frac{N\alpha}{2}(1-\frac{1}{p})}  \||u(s)|^p-|v(s)|^p|L^1_{uloc}\| \, ds\\
		&\qquad\qquad\qquad\times \int_0^\infty  h_\alpha(\theta)(\theta+ \theta^{1-\frac{N}{2}(1-\frac{1}{p})}) \,d\theta\\
		&\le  C t^\beta \int_0^t (t-s)^{\alpha-1-\frac{N\alpha}{2}(1-\frac{1}{p})}  (\|u(s)|L^p_{uloc}\|^{p-1} + \|v(s)|L^p_{uloc}\|^{p-1})\\
		&\qquad\qquad\qquad\times \|u(s)-v(s)|L^p_{uloc}\| \, ds\\
		&\le  C t^\beta  M^{p-1}\int_0^t (t-s)^{\alpha-1-\frac{N\alpha}{2}(1-\frac{1}{p})} s^{-p\beta}\, ds \, d(u,v)\\
		&\le C_3M^{p-1}t^{\alpha(1-\frac{N}{2}(p-1))}B\left(\alpha-\frac{N\alpha}{2}\left(1-\frac{1}{p}\right), 1-\frac{N\alpha}{2}(p-1)\right)d(u,v)
		\end{split}
		\end{equation}
		for $0<t\le1$. Note that $C_3$ is independent of $\alpha$.
		Taking sufficiently small $c_*>0$ if necessary, \eqref{sub:4} yields 
		\[
		d(\Phi[u], \Phi[v]) \le 2^{p-1} C_1^{p-1}C_3 c_*^{p-1} d(u,v) \le \frac{1}{2} d(u,v)
		\] 
		for $u,v\in B_M$. Note that $c_*$ is independent of $\alpha$.
		This implies that $\Phi: B_M \to B_M$ is a contraction mapping. 
		Therefore, by the contraction mapping theorem, we know $\Phi$ has a unique fixed point $u\in B_M$. 
		Obviously, if $\mu$ is a nonnegative Radon measure on ${\bf R}^N$, we see that $u(x,t)\ge0$ for almost all $t\in(0,1]$ and $x\in{\bf R}^N$.
		This implies that $u$ is a nonnegative solution of \eqref{TFF} in $[0,1]\times{\bf R}^N$.
		
		Finally, let $p=p_F$. By the property of Beta function:
		\[
		B(x,y)=\frac{\Gamma(x)\Gamma(y)}{\Gamma(x+y)}
		\]
		and Gamma function:
		\[
		\Gamma(1+x)=x\Gamma(x),
		\]
		we can take $C>0$ independently of $\alpha$ such that
		\[
		B\left( \alpha -\frac{N\alpha}{2}\left(1-\frac{1}{p}\right), 1-\alpha \right)=C \Gamma(1-\alpha)=C(1-\alpha)^{-1}.
		\]
		By \eqref{gamma_2}, we complete the proof of Theorem \ref{Thm:1.2}.
	\end{proof}
	
	%%%%%%%%%%%%%%%%%%%%%%%%%%%%%%%%%%%%%
	\subsection{In the case of $p>p_F$.}
	%%%%%%%%%%%%%%%%%%%%%%%%%%%%%%%%%%%%%
	
	
	\begin{proof}[Proof of Theorem \ref{Thm:1.3}.]
		It suffices to consider the case of $T=1$.
		Then \eqref{Thm:1.3.1} leads 
		\begin{equation}
		\label{super:1}
		\|\mu|M^{r,\lambda}\|\le \gamma_3 
		\end{equation}
		where $\lambda := {2r/(p-1)}$. For simplicity of notation, set
		$|\cdot|_{r} := \|\cdot|M^{r,\lambda}\|$ for $r\ge 1$.
		Let $1<r<q<\infty$ be such that
		\[
		1<\frac{q}{p}<r<q \quad \mbox{and} \quad \lambda =\frac{2r}{p-1} <N.
		\]
		Note that $q>p$. 
		Set 
		\[
		Y:= L^\infty((0,1); M^{q,\lambda})
		\]
		and
		\[
		\|u|Y\| := \sup_{t\in(0,1)} t^\beta |u|_q, \quad \beta:= \frac{\alpha}{p-1}\left(1-\frac{r}{q}\right) <\frac{\alpha}{p}.
		\]
		For $u\in Y$, define
		\[
		\Phi[u](t):= P_\alpha(t)\mu +\alpha\int_0^t (t-s)^{\alpha-1} S_\alpha(t-s)|u(s)|^p \,ds.
		\]
		Denote $B_M := \{u\in Y;\|u|Y\|\le M\}$ for $M>0$, which will be chosen later, and
		\[
		d(u,v) := \|u-v|Y\|,
		\]
		for $u,v\in B_M$.
		Now $(B_M,d)$ is a complete metric space.
		We will show that $\Phi: B_M \to B_M$. 
		Since 
		\[
		-\frac{1}{p-1}\left(1-\frac{r}{q}\right)>-1, \quad 1-\frac{r}{q}>0,
		\]
		by \eqref{eq:1.4}, we have
		\[
		\int_0^\infty (1+\theta^{-\frac{1}{p-1}(1-\frac{r}{q})})h_{\alpha}(\theta)\, d\theta= 1+\frac{\Gamma\left(1-\frac{1}{p-1}(1-\frac{r}{q})\right)}{\Gamma\left(1-\frac{1}{p-1}(1-\frac{r}{q})\right)}<\infty
		\]
		and
		\[
		\int_0^\infty (\theta +\theta^{1-\frac{r}{q}})h_{\alpha}(\theta)\, d\theta = \frac{\Gamma(2)}{\Gamma(1+\alpha)}+\frac{\Gamma\left(2-\frac{r}{q}\right)}{\Gamma\left(1+\alpha-\frac{r}{q}\right)}<\infty.
		\]
		Note that these constants are finite near $\alpha=1$. Observe that $|f^p|_\frac{q}{p} = |f|_q^p$ for all $f\in M^{q,\lambda}({\bf R}^N)$. Let $u\in B_M$ and $t\in (0,1]$. 
		It follows from Proposition \ref{MpMq} and \eqref{super:1} that
		\begin{equation*}
		\begin{split}
		t^\beta |P_\alpha (t) \mu |_q
		&\le t^\beta \int_0^\infty h_\alpha(\theta) |e^{t^\alpha \theta \Delta}\mu |_q\, d\theta ds\\
		&\le C |\mu|_r\int_0^\infty h_\alpha(\theta )(1+\theta^{-\frac{1}{p-1}(1-\frac{r}{q})})\, d\theta \\
		&\le C_1\gamma_3
		\end{split}
		\end{equation*}
		for $0<t\le1$. Moreover,
		\begin{equation*}
		\begin{split}
		&t^\beta \left| \alpha\int_0^t (t-s)^{\alpha-1} S_\alpha(t-s)|u(s)|^p\,ds \right|_q \\
		&=  \alpha t^\beta \int_0^t (t-s)^{\alpha-1} \int_0^\infty \theta h_\alpha(\theta)|e^{(t-s)^\alpha \theta \Delta}|u(s)|^p|_q  \,d\theta ds\\
		&\le   C  t^\beta \int_0^t (t-s)^{(1-\frac{r}{q})\alpha-1} |u(s)|_q^p  \,ds \int_0^\infty  h_\alpha(\theta) (\theta +\theta^{1-\frac{r}{q}}) \,d\theta \\
		&\le CM^p t^\beta\int_0^t(t-s)^{(1-\frac{r}{q})\alpha-1}s^{-p\beta} \, ds\\
		&\le C_2M^pB\left(\alpha\left(1-\frac{r}{q}\right), 1-\frac{p\alpha}{p-1}\left(1-\frac{r}{q}\right) \right)\\
		\end{split}
		\end{equation*}
		for $0<t\le1$. 
		Note that $C_1$ and $C_2$ are independent of $\alpha$.
		Setting
		\begin{equation}
		\gamma_3 := c_* B\left(\alpha\left(1-\frac{r}{q}\right), 1-\frac{p\alpha}{p-1}\left(1-\frac{r}{q}\right) \right)^{-\frac{1}{p-1}}
		\end{equation}
		and
		\[
		M:= 2C_1c_*B\left(\alpha\left(1-\frac{r}{q}\right), 1-\frac{p\alpha}{p-1}\left(1-\frac{r}{q}\right) \right)^{-\frac{1}{p-1}}
		\]
		and taking sufficiently small $c_*>0$ if necessary, it follows from the above inequalities that
		\begin{equation*}
		\begin{split}
		%\label{super:2}
		\|\Phi[u]|Y\|
		&=\sup_{t\in(0,1)}t^\beta |\Phi[u](t)|_q \\
		&\le (C_1c_*+2^pC_1^pC_2c_*^p)B\left(\alpha\left(1-\frac{r}{q}\right), 1-\frac{p\alpha}{p-1}\left(1-\frac{r}{q}\right) \right)^{-\frac{1}{p-1}}\\
		& \le 2C_1c_*B\left(\alpha\left(1-\frac{r}{q}\right), 1-\frac{p\alpha}{p-1}\left(1-\frac{r}{q}\right) \right)^{-\frac{1}{p-1}}=M.
		\end{split}
		\end{equation*}
		This implies that $\Phi: B_M \to B_M$.
		
		
		Next, we will show that $\Phi: B_M \to B_M$ is a contraction mapping. 
		Let $u,v\in B_M$.
		Similarly to the above computations, we have
		\begin{equation}
		\label{super:3}
		\begin{split}
		& t^\beta |\Phi[u](t)-\Phi[v](t)|_q \\
		&\le \alpha t^\beta \int_0^t (t-s)^{\alpha-1} \int_0^\infty \theta h_\alpha(\theta) |e^{(t-s)^\alpha \theta \Delta}(|u(s)|^p-|v(s)|^p)|_q  \, d\theta ds\\
		&\le \alpha t^\beta \int_0^t (t-s)^{\alpha-1} \int_0^\infty \theta h_\alpha(\theta) 
		(1+[(t-s)^\alpha \theta]^{-\frac{r}{q}})|(|u(s)|^p-|v(s)|^p)|_\frac{q}{p}  \, d\theta ds\\
		&\le \alpha t^\beta\int_0^t (t-s)^{(1-\frac{r}{q})\alpha-1} |(|u(s)|^p-|v(s)|^p)|_\frac{q}{p}  \,  ds
		\int_0^\infty  h_\alpha(\theta) 
		(\theta+\theta^{1-\frac{r}{q}})\,d\theta \\
		&\le  C t^\beta \int_0^t (t-s)^{(1-\frac{r}{q})\alpha-1}  ||u(s)|^p-|v(s)|^p|_\frac{q}{p} \, ds\\
		&\le  C t^\beta \int_0^t (t-s)^{(1-\frac{r}{q})\alpha-1}  (|u(s)|_q^{p-1} + |v(s)|_q^{p-1})|u(s)-v(s)|_q  \, ds\\
		&\le  C  M^{p-1}t^\beta  \int_0^t (t-s)^{(1-\frac{r}{q})\alpha-1} s^{-p\beta}\, ds \, d(u,v)\\
		&\le C_3 M^{p-1} B\left(\alpha\left(1-\frac{r}{q}\right), 1-\frac{p\alpha}{p-1}\left(1-\frac{r}{q}\right) \right)d(u,v)
		\end{split}
		\end{equation}
		for $0<t\le1$. 
		Note that $C_3$ is independent of $\alpha$.
		Taking sufficiently small $c_*>0$ if necessary, \eqref{super:3} yields 
		\[
		d(\Phi[u], \Phi[v]) \le 2^{p-1}C_1^{p-1}C_3c_*^{p-1} d(u,v) \le \frac{1}{2} d(u,v)
		\] 
		for $u,v\in B_M$. 
		Note that $c_*$ is independent of $\alpha$.
		This implies that $\Phi: B_M \to B_M$ is a contraction mapping. 
		Therefore, by the contraction mapping theorem, we know $\Phi$ has a unique fixed point $u\in B_M$. 
		Obviously, if $\mu$ is a nonnegative measurable function in ${\bf R}^N$, we see that $u(x,t)\ge0$ for almost all $t\in(0,1]$ and $x\in{\bf R}^N$.
		This implies that $u$ is a nonnegative solution of \eqref{TFF} in $(0,1]\times{\bf R}^N$.
		Then Theorem \ref{Thm:1.3} follows.
	\end{proof}
	
	
	
	
	
	
	%\begin{proof}[Proof of Theorem \ref{Thm:1.2} in the case of $p= p_F$.]
	%It suffices to consider the case of $T=1$. Indeed, for any solution $u$ of \eqref{TFF} in $[0,T)\times{\bf R}^N$, where $0<T<\infty$, we see that $u_\lambda (x,t) := \lambda^{2\alpha/(p-1)}u(\lambda^2 t , \lambda^\alpha x)$ with $\lambda:= \sqrt{T}$ is a solution of \eqref{TFF} in $[0,1)\times{\bf R}^N$.
	%The proof is based on the contraction mapping theorem.
	%Since $p= p_F > 1+2\alpha /(\alpha N +2 -2\alpha)$, we have
	%\[
	%\frac{\alpha N(p-1)}{2(p\alpha-p+1)_+}>1,
	%\]
	%where $r_+ := \max\{r,0\}$ for $r\in{\bf R}$. Furthermore, since $p= p_F >(4-N+\sqrt{N^2+16})/4$, we have 
	%\[
	%\frac{N(p-1)}{2p(2-p)_+}>1.
	%\]
	%By these inequalities and 
	%\[
	%\frac{N(p-1)}{2p}<\frac{\alpha N(p-1)}{2(p\alpha-p+1)_+},
	%\]
	%we can take $q>p\ge p_F$ such that
	%\begin{equation}
	%\label{eq:4.1}
	%\frac{\alpha}{p-1}-\frac{1}{p}<\frac{N\alpha}{2q}<\frac{\alpha}{p-1}
	%\end{equation}
	%and 
	%\begin{equation}
	%\label{eq:4.2}
	%1-\frac{1}{q}<\frac{2}{N}.
	%\end{equation}
	%Let 
	%\[
	%\beta:= \frac{N\alpha}{2} \left(1-\frac{1}{q}\right) = \frac{\alpha}{p-1}-\frac{N\alpha}{2q}.
	%\]
	%By \eqref{eq:4.1} and \eqref{eq:4.2}, we see that
	%\begin{equation}
	%\label{eq:4.2.1}
	%0<p\beta<1, \quad \alpha = \frac{N\alpha}{2q}(p-1) +(p-1)\beta.
	%\end{equation}
	%Let $\mu\in M^{1}({\bf R}^N)$. By Proposition \ref{LpLq}, we have
	%\[
	%\|e^{t\Delta}\mu\|_{L^q_{uloc}} \le C (1+t^{-\frac{N}{2}(1-\frac{1}{q})}) \|\mu\|_{M^{1}}
	%\]
	%for $t>0$. This implies that 
	%\begin{equation}
	%\label{eq:4.3}
	%\begin{split}
	%\|P_{\alpha}(t)\mu\|_{L^q_{uloc}}
	%& \le \int_0^\infty h_{\alpha}(\theta) \|e^{t^\alpha \theta\Delta}\mu\|_{L^q_{uloc}}\,d\theta\\
	%& \le C  \|\mu\|_{M^{1}} \int_0^\infty h_\alpha(\theta)(1+(t^\alpha \theta)^{-\frac{N}{2}(1-\frac{1}{q})}) \, d\theta
	%\end{split}
	%\end{equation}
	%for $0<t\le1$.
	%Since it follows from \eqref{eq:4.2} that
	%\[
	%-\frac{N}{2} \left(1-\frac{1}{q}\right)>-1,
	%\]
	%\eqref{eq:1.4}  and \eqref{eq:4.3} yield
	%\[
	%\|P_{\alpha}(t)\mu\|_{L^q_{uloc}} \le  C (1+t^{-\beta}) \|\mu\|_{M^{1}} \le  C t^{-\beta}\|\mu\|_{M^{1}}.
	%\]
	%for $0<t\le1$.
	%Let $Y:=\{u\in L^\infty ((0,1); L^{q}_{uloc}({\bf R}^N)); \|u\|_Y <\infty\}$, where 
	%\[
	%\|u\|_Y := \sup_{t\in(0,1)} t^{\beta}\|u(t)\|_{L^q_{uloc}}.
	%\]
	%For $u\in Y$, define
	%\[
	%\Phi[u](t):= P_\alpha(t)\mu +\alpha\int_0^t (t-s)^{\alpha-1} S_\alpha(t-s)|u(s)|^p \,ds.
	%\]
	%Denote $B_M := \{u\in Y;\|u\|_Y\le M\}$ for $M>0$ and
	%\[
	%d(u,v) := \sup_{t\in(0,1)} t^{\beta}\|u(t)-v(t)\|_{L^q_{uloc}},
	%\]
	%for $u,v\in B_M$.
	%Now $(B_M,d)$ is a closed complete metric space.
	%We show that $\Phi: B_M \to B_M$. Let $u\in B_M$ and $t\in (0,1)$. We have
	%\begin{equation}
	%\label{eq:4.4}
	%\begin{split}
	%t^\beta \|\Phi[u](t)\|_{L^q_{uloc}}
	%&\le t^\beta \|P_{\alpha}(t)\mu\|_{L^q_{uloc}} + \alpha t^\beta \int_0^t (t-s)^{\alpha-1} \|S_\alpha (t-s)|u(s)|^p\|_{L^q_{uloc}}\, ds\\
	%&\le C \|\mu\|_{M^{1}} + \alpha t^\beta \int_0^t (t-s)^{\alpha-1} \int_0^\infty \theta h_\alpha (\theta) \|e^{(t-s)^\alpha\theta \Delta}|u(s)|^p\|_{L^q_{uloc}}\, d\theta ds.
	%\end{split}
	%\end{equation}
	%for $0<t< 1$.
	%By Proposition \ref{LpLq}, we have
	%\begin{equation}
	%\label{eq:4.5}
	%\begin{split}
	% \|e^{(t-s)^\alpha\theta \Delta}|u(s)|^p\|_{L^q_{uloc}}
	%&\le C  \left(1+[(t-s)^\alpha \theta]^{-\frac{N}{2}(\frac{p}{q}-\frac{1}{q})}\right) \||u(s)|^p\|_{L^\frac{q}{p}_{uloc}}\\
	%&\le C \left(1+[(t-s)^\alpha \theta]^{-\frac{N}{2}(\frac{p}{q}-\frac{1}{q})}\right) s^{-p\beta } (s^\beta \|u(s)\|_{L^q_{uloc}})^p\\
	%&\le C M^p \left(1+[(t-s)^\alpha \theta]^{-\frac{N}{2}(\frac{p}{q}-\frac{1}{q})}\right) s^{-p\beta }
	%\end{split}
	%\end{equation}
	%for $0<s<t<1$ and $\theta\in(0,\infty)$. Since $1<p=p_F<q$, we have
	%\[
	%1-\frac{N}{2q}(p-1)>-1.
	%\]
	%\eqref{eq:1.4} implies that
	%\[
	%\int_0^\infty \theta^{1-\frac{N}{2q}(p-1)} h_\alpha(\theta) \, d\theta <\infty.
	%\]
	%This together with \eqref{eq:4.2.1} and \eqref{eq:4.5} implies that
	%\begin{equation}
	%\label{eq:4.7}
	%\begin{split}
	%&\alpha t^\beta \int_0^t (t-s)^{\alpha-1} \int_0^\infty \theta h_\alpha (\theta) \|e^{(t-s)^\alpha\theta \Delta}|u(s)|^p\|_{L^q_{uloc}}\, d\theta ds\\
	%&\le Ct^\beta M^p \int_0^t (t-s)^{\alpha-1} (1+(t-s)^{-\frac{N\alpha}{2}(\frac{p}{q}-\frac{1}{q})})s^{-p\beta} \, ds\\
	%&\le Ct^\beta M^p \int_0^t (t-s)^{-\frac{N\alpha}{2}(\frac{p}{q}-\frac{1}{q})+\alpha-1} s^{-p\beta} \, ds\le CM^p
	%\end{split}
	%\end{equation}
	%for $0<t<1$. Taking sufficiently small $\|\mu\|_{M^{1}}$ and $M>0$ if necessary, it follows from
	%\eqref{eq:4.4} and \eqref{eq:4.7} that
	%\[
	%t^\beta \|\Phi[u](t)\|_{L^q_{uloc}} \le C \|\mu\|_{M^{1}}  +CM^p \le M
	%\]
	%for $0<t<1$. This implies $\Phi: B_M \to B_M$.
	%
	%Next we will prove $\Phi: B_M \to B_M$ is a contraction mapping. 
	%Let $u,v\in B_M$.
	%Similarly to the above computations, we have
	%\begin{equation}
	%\label{eq:4.8}
	%\begin{split}
	%&t^\beta \|\Phi[u](t)-\Phi[v](t)\|_{L^q_{uloc}}\\
	%&\le \alpha t^\beta \int_0^t (t-s)^{\alpha-1} \int_0^\infty \theta h_\alpha(\theta) \|e^{(t-s)^\alpha \theta \Delta}(|u(s)|^p-|v(s)|^p)\|_{L^q_{uloc}} \, d\theta ds\\
	%&\le  C t^\beta \int_0^t (t-s)^{-\frac{N\alpha}{2}(\frac{p}{q}-\frac{1}{q})+\alpha-1}  \||u(s)|^p-|v(s)|^p\|_{L^\frac{q}{p}_{uloc}} \, ds\\
	%&\le  C t^\beta \int_0^t (t-s)^{-\frac{N\alpha}{2}(\frac{p}{q}-\frac{1}{q})+\alpha-1}  (\|u(s)\|_{L^q_{uloc}}^{p-1} + \|v(s)\|_{L^q_{uloc}}^{p-1})\|u(s)-v(s)\|_{L^q_{uloc}} \, ds\\
	%&\le  C t^\beta  M^{p-1}\int_0^t (t-s)^{-\frac{N\alpha}{2}(\frac{p}{q}-\frac{1}{q})+\alpha-1} s^{-p\beta}\, ds \, d(u,v)\\
	%&\le CM^{p-1} d(u,v)
	%\end{split}
	%\end{equation}
	%for $0<t<1$. Taking sufficiently small $M>0$ if necessary, \eqref{eq:4.8} yields 
	%\[
	%d(\Phi[u], \Phi[v]) \le CM^{p-1} d(u,v) \le \frac{1}{2} d(u,v)
	%\]
	%for $u,v\in B_M$. This implies that $\Phi: B_M \to B_M$ is a contraction mapping. 
	%Therefore, by contraction mapping theorem, we know $\Phi$ has a unique fixed point $u\in B_M$. 
	%Obviously, if $\mu$ is a nonnegative Radon measure in ${\bf R}^N$, we see that $u(t,x)\ge0$ for almost all $t\in(0,1)$ and $x\in{\bf R}^N$.
	%Then Theorem \ref{Thm:1.2} in the case of $p= p_F$ follows.
	%\end{proof}
	
	
	%%%%%%%%%%%%%%%%%%%%%%%%%%%%%%%%%%%%%
	\section{Collapse of the solvability}\label{section. Collapse of the solvability}
	%%%%%%%%%%%%%%%%%%%%%%%%%%%%%%%%%%%%%
	
	%%%%%%%%%%%%%%%%%%%%%%%%%%%%%%%%%%%%%
	\subsection{Preliminaries.}
	%%%%%%%%%%%%%%%%%%%%%%%%%%%%%%%%%%%%%
	Let $A(x)$ and $B(x)$ be functions from a set where $x$ belongs to $[0,\infty)$. In what follows, we denote
	\[
	A(x) \simeq B(x) 
	\]
	if there exist $0<c<C$ in general depending on $\alpha$ such that 
	\[
	c B(x) \le A(x) \le CB(x)
	\]
	for all $x$ and
	\[
	\limsup_{\alpha\to1-0} c(\alpha)\in (0,\infty)\ \mbox{and}\ \limsup_{\alpha\to1-0} C(\alpha)\in (0,\infty).
	\]
	First, we define the life span of solutions of \eqref{TFF}.
	The following proposition has been proved in \cite{GMS22}.
	\begin{proposition}
		\label{Prop:min}
		If  problem \eqref{TFF} possesses a supersolution in $[0,T]\times{\bf R}^N$, then problem \eqref{TFF} possesses a solution $\overline{u}$ in $[0,T]\times{\bf R}^N$ in the sense of Definition {\rm \ref{Def:1.1}}.
		Furthermore, $\overline{u}$ sastisfies
		\begin{equation}
		\label{min}
		0\le \overline{u}(x,t) \le v(x,t) \quad \mbox{for a.a.} \quad t\in(0,T], \, x\in{\bf R}^N
		\end{equation}
		for all solutions $v$ of  \eqref{TFF} in $[0,T]\times{\bf R}^N$.
	\end{proposition}
	%
	We call the solution $\overline{u}$ satisfying \eqref{min} the minimal solution of \eqref{TFF}.
	Since the minimal solution is unique, we can define the life span of solutions of problem \eqref{TFF} as
	\[
	T_\alpha\left[ \mu \right]:=\sup\{T'>0 ; \mbox{the minimal solution of \eqref{TFF} exists in} \,\, [0,T']\times{\bf R}^N\}.
	\]
	By virtue of Proposition \ref{Prop:min}, we can regard the solutions which were obtained in Theorems \ref{Thm:1.2} and \ref{Thm:1.3} as the minimal solutions. 
	%%%%%%%%%%%%%%%%%%%%%%%%%%%%%%%%%%%%%
	\subsection{Collapse of the global solvability.}
	%%%%%%%%%%%%%%%%%%%%%%%%%%%%%%%%%%%%%
	The space $\mathcal{M}=\mathcal{M}({\bf R}^N)$ is defined as the set of the Radon measures $\mu$ on ${\bf R}^N$ such that their norm satisfy $\|\mu|\mathcal{M}\|:=|\mu|({\bf R}^N) <\infty$.
	%
	In the following, for a Radon measure $\mu$ on ${\bf R}^N$, $\mu\ge0$ will be taken to mean
	$\mu$ is a nonnegative one.
	For $r>0$, set
	\[
	\mathcal{B}^+(r):=\left\{0\le\mu\in\mathcal{M};\|\mu|\mathcal{M}\|\le r \right\}.
	\]
	%
	Define
	\[
	\mathcal{G}_\alpha:= \left\{0\le \nu\in \mathcal{M}; \mbox{\eqref{TFF} with} \,\,u(0)=\nu \,\,\mbox{possesses a global-in-time solution.} \right\}.
	\]
	In other words, $\mathcal{G}_{\alpha}= \left\{0\le \nu\in \mathcal{M}; T_{\alpha}[\nu]=\infty \right\}$. By Proposition~\ref{Prop:min}, we easily observe that $\mathcal{G}_{\alpha}$ is convex.
	
	\begin{theorem}
		\label{Thm:4.1}
		Let $p=p_F$. Then there exist positive constants $C_1 \le C_2$ independent of $\alpha$ such that
		\begin{equation}
		\label{Thm:4.1.1}
		\mathcal{B}^+\left( C_1(1-\alpha)^{\frac{N}{2}} \right)\subset \mathcal{G}_{\alpha}\subset \mathcal{B}^+\left( C_2(1-\alpha)^{\frac{N}{2}} \right)
		\end{equation}
		near $\alpha=1$. In particular,
		\begin{equation}
		\label{Thm:4.1.2}
		\sup_{\nu\in\mathcal{G}_\alpha } \|\nu|\mathcal{M}\| \simeq (1-\alpha)^\frac{N}{2}
		\end{equation}
		near $\alpha=1$ and 
		\begin{equation}
		\label{Thm:4.1.3}
		\lim_{\alpha\to1-0} \sup_{\nu\in\mathcal{G}_\alpha } \|\nu|\mathcal{M}\| =0.
		\end{equation}
	\end{theorem}
	
	
	
	\begin{proof} 
		The inclusion
		\[
		\mathcal{B}^+\left( C_1(1-\alpha)^{\frac{N}{2}} \right) \subset \mathcal{G}_{\alpha}
		\]
		is easily deduced from Theorem~\ref{Thm:1.2}. On the other hand, suppose that $\mu \in \mathcal{G}_\alpha$. Then, problem \eqref{TFF} possesses a global-in-time solution and
		\eqref{Thm:1.1.2} holds for all $T>1$.
		Letting $\sigma=T^{\alpha/4} (<T^{\alpha/2})$, we have
		\begin{equation*}
		\sup_{z\in{\bf R}^N} \mu(B(z,T^\frac{\alpha}{4})) \le \gamma'_1 \left(\int_{(16\sqrt{T})^{-1}}^{1/4} t^{-\alpha} \, dt \right)^{-\frac{N}{2}}.
		\end{equation*}
		%
		Since $\limsup_{\alpha\to1-0} \gamma'_1(\alpha) \in (0,\infty)$,
		we have
		\[
		\overline{\gamma} := \sup_{\alpha\in[1/2,1)} \gamma'_1(\alpha) \in (0,\infty).
		\]
		Now taking $T\to\infty$, then we obtain
		\begin{equation*}
		\mu({\bf R}^N) \le \gamma'_1\left(\int_{0}^{1/4} t^{-\alpha} \, dt \right)^{-\frac{N}{2}}= 4^{\alpha-1} \gamma'_1 (1-\alpha)^\frac{N}{2}\le \overline{\gamma}(1-\alpha)^\frac{N}{2}.
		\end{equation*}
		This implies that $\mu\in\mathcal{B}^+(C_2 (1-\alpha)^\frac{N}{2})
		$ with $C_2= \max\{\overline{\gamma},C_1\}$.
		Then  \eqref{Thm:4.1.1} follows. \eqref{Thm:4.1.2} and \eqref{Thm:4.1.3} are derived immediately from \eqref{Thm:4.1.1}.
	\end{proof}
	
	%%%%%%%%%%%%%%%%%%%%%%%%%%%%%%%%%%%%%%%%%%%%%%%%%%%%%
	\subsection{Collapse of the local solvability}
	%%%%%%%%%%%%%%%%%%%%%%%%%%%%%%%%%%%%%%%%%%%%%%%%%%%%%
	For $0<\epsilon<N/2$, let
	\[
	f_\epsilon(x) := |x|^{-N}\left|\log|x|\right|^{-\frac{N}{2}-1+\epsilon}\chi_{B(0,1/e)}.
	\]
	Note that $f_{\epsilon}\in L^1({\bf R}^N)$. Miyamoto \cite{Miya21} observes that \eqref{Fjt} with $\mu = f_{\epsilon}$ possesses no local-in-time solutions. In contrast, the local solvability of the time-fractional equation \eqref{TFF} for any $L^1({\bf R}^N)$ initial data is guaranteed. This difference between the unsolvability of problem \eqref{Fjt} and the solvability of problem \eqref{TFF} is connected by the life span estimate for the initial data $f_{\epsilon}$.
	
	In the following, we give lower estimates of the life span $T_\alpha[\mu]$.
	Let $\overline{T}$ be the existence time of the solution of problem\eqref{TFF} which was obtained in Theorems~\ref{Thm:1.2} or \ref{Thm:1.3}.
	Strictly speaking, although in the following arguments $\overline{T}$ is estimated, $T_\alpha[\mu]$ and $\overline{T}$ can be considered identical without loss of the generality since $\overline{T}\le T_\alpha$ holds.
	
	\begin{theorem}
		\label{Thm:4.2}
		Let $p=p_F$ and $0<\epsilon<N/2$. Then, for each $\kappa>0$, there exists $\alpha(\kappa,\epsilon)\in(0,1)$ such that for all $\alpha(\kappa,\epsilon)<\alpha<1$,
		\[
		\log \left( T_{\alpha}[\kappa f_{\epsilon}] \right)\simeq-\kappa^{\frac{2}{N-2\epsilon}}(1-\alpha)^{-\frac{N}{N-2\epsilon}}.
		\]
		In particular, for all $\kappa>0$,
		\[
		\lim_{\alpha \to 1-0} T_{\alpha}[\kappa f_{\epsilon}]=0.
		\]
	\end{theorem}
	
	\begin{proof}
		

		
		It follows that
		\begin{equation*}
		\sup_{z\in{\bf R}^N} \int_{B(z,\sigma)} f_{\epsilon}(x)\,dx = \int_{B(0,\sigma)} f_{\epsilon}(x)\,dx = C \left( \log \frac{1}{\sigma} \right)^{-\frac{N}{2}+\epsilon}
		\end{equation*}
		for all $\sigma\le 1/e$. 
		Firstly, let us deduce the upper estimate. 
		Define 
		\[
		\overline{T_\alpha} [\kappa f_\epsilon] := \min\left\{ T_{\alpha}[\kappa f_{\epsilon}], e^{-\frac{2}{\alpha}}\right\} \le  T_{\alpha}[\kappa f_{\epsilon}].
		\]
		We can assume that problem \eqref{TFF} with $\mu=\kappa f_\epsilon$ possesses a solution in $[0,\overline{T_\alpha}[\kappa f_\epsilon]]\times{\bf R}^N$.
		Since $(1-\alpha)^{\frac{1}{1-\alpha}}<1$, we can take 
		\[
		\sigma = \overline{T_\alpha} [\kappa f_\epsilon]^\frac{\alpha}{2}  (1-\alpha)^{\frac{\alpha}{2(1-\alpha)}} \le \overline{T_{\alpha}}[\kappa f_{\epsilon}]^\frac{\alpha}{2} \le e^{-1}
		\]
		in Theorem~\ref{Thm:1.1}. 
		Therefore, the necessary condition \eqref{Thm:1.1.2} provides
		\begin{equation*}
		\begin{split}
		\kappa \left( \log \frac{1}{\sigma} \right)^{-\frac{N}{2}+\epsilon}
		&\le C (1-\alpha)^{\frac{N}{2}} \left( \left( \frac{1}{4} \right)^{1-\alpha}-\left( \frac{1}{16} \right)^{1-\alpha}(1-\alpha) \right)^{-\frac{N}{2}}\\
		&\le C (1-\alpha)^{\frac{N}{2}} 
		\end{split}
		\end{equation*}
		near $\alpha =1$. Note that $C$ is independent of $\alpha$.
		This implies that
		\[
		\sigma \le \exp \left( -C \kappa^{\frac{2}{N-2\epsilon}} (1-\alpha)^{-\frac{N}{N-2\epsilon}} \right),
		\]
		that is 
		\begin{equation*}
		\begin{split}
		\overline{T_{\alpha}}[\kappa f_{\epsilon}]
		&\le (1-\alpha)^{-\frac{1}{1-\alpha}} \exp \left( -\frac{2}{\alpha}C \kappa^{\frac{2}{N-2\epsilon}} (1-\alpha)^{-\frac{N}{N-2\epsilon}} \right)\\
		&\le (1-\alpha)^{-\frac{1}{1-\alpha}} \exp \left( -C \kappa^{\frac{2}{N-2\epsilon}} (1-\alpha)^{-\frac{N}{N-2\epsilon}} \right)
		\end{split}
		\end{equation*}
		near $\alpha=1$. Here, we claim that
		\[
		(1-\alpha)^{-\frac{1}{1-\alpha}} \exp\left( -C \kappa^{\frac{2}{N-2\epsilon}} (1-\alpha)^{-\frac{N}{N-2\epsilon}} \right) \le \exp \left( -C' \kappa^{\frac{2}{N-2\epsilon}} (1-\alpha)^{-\frac{N}{N-2\epsilon}} \right)
		\]
		for appropriate $C'>0$. Indeed, it suffices to obtain that for any $K>0$ and $\beta>1$,
		\begin{equation}\label{eq. inequality}
		(1-\alpha)^{-\frac{1}{1-\alpha}} \exp \left( -K (1-\alpha)^{-\beta} \right)<1
		\end{equation}
		holds near $\alpha=1$, and we show that this inequality is true in Lemma~\ref{lem. inequality} below. 
		Therefore, it follows that
		\[
		\overline{T_\alpha} [\kappa f_\epsilon] = \min\left\{  T_{\alpha}[\kappa f_{\epsilon}],\ e^{-\frac{2}{\alpha}} \right\} \le \exp \left( -C \kappa^{\frac{2}{N-2\epsilon}} (1-\alpha)^{-\frac{N}{N-2\epsilon}} \right)
		\]
		near $\alpha =1$.
		Since the right-hand side of the above inequality tends to $0$ as $\alpha \to 1-0$,
		there exists a $\alpha(\kappa,\epsilon)\in (0,1)$ such that all the above calculations are justified and
		\[
		e^{-\frac{2}{\alpha}} \ge \exp \left( -C \kappa^{\frac{2}{N-2\epsilon}} (1-\alpha)^{-\frac{N}{N-2\epsilon}} \right)
		\]
		holds for $\alpha(\kappa,\epsilon ) <\alpha <1$.
		This implies that
		\[
		T_{\alpha}[\kappa f_{\epsilon}] \le \exp \left( -C \kappa^{\frac{2}{N-2\epsilon}} (1-\alpha)^{-\frac{N}{N-2\epsilon}} \right)
		\]
		for $\alpha(\kappa,\epsilon ) <\alpha <1$,
		and the upper estimate of $T_{\alpha}[\kappa f_{\epsilon}] $ has been obtained.
		
		Let us deduce the lower estimate. 
		Define
		\[
		\overline{T} := \exp \left( -C \kappa^\frac{2}{N-2\epsilon}(1-\alpha)^{-\frac{N}{N-2\epsilon}} \right)
		\]
		for appropriate $C>0$.
		It suffices to show that problem \eqref{TFF} possesses a solution in $[0,\overline{T}]\times{\bf R}^N$.
		Since $\lim_{\alpha\to1-0} \overline{T}(\alpha)=0$, we  may assume that $\overline{T}<e^{-1}$.
		We have
		\begin{equation*}
		\begin{split}
		\sup_{z\in{\bf R}^N} \int_{B(z,\overline{T}^{\alpha/2})}\kappa f_\epsilon(x)\,dx
		&= \int_{B(0,\overline{T}^{\alpha/2})}\kappa f_\epsilon(x)\,dx\\
		& = \kappa \left( \log \frac{1}{\overline{T}^{{\alpha/2}}} \right)^{-\frac{N}{2}+\epsilon}\\
		& = C (1-\alpha)^\frac{N}{2} \to 0\quad \mbox{as} \quad \alpha\to1-0.
		\end{split}
		\end{equation*}
		This implies that \eqref{Thm:1.2.1} holds near $\alpha=1$ and guarantees the solvability on $[0,\overline{T}]$.
		In particular, $\overline{T} \le T_{\alpha}[\kappa f_{\epsilon}]$ is verified
		and  the lower estimate of $T_{\alpha}[\kappa f_{\epsilon}] $ has been obtained.
	\end{proof}
	
	\begin{Lemma}\label{lem. inequality}
		The inequality \eqref{eq. inequality} is true. More precisely, for all $K>0$ and $\beta>1$,
		\[
		x^{-\frac{1}{x}} \exp\left( -K x^{-\beta} \right)<1
		\]
		for sufficiently small $x>0$.
	\end{Lemma}
	\begin{proof}
		The conclusion is equivalent to the following inequality
		\[
		x^{-1}\exp\left(-Kx^{1-\beta}\right)<1
		\]
		near $x=0$ and it is true since $ x^{-1}\exp\left(-Kx^{1-\beta}\right)\to 0$ as $x\to 0$.
	\end{proof}
	
	\begin{Remark}
		{\rm Actually, the lower estimate is true for all $0<\alpha<1$, while the upper estimate does not hold for any $0<\alpha<1$ in general. Indeed, for each $\alpha$, if $\kappa$ is sufficiently small, $T_{\alpha}[\kappa f_{\epsilon}]=\infty$ (See Theorem~\ref{Thm:4.1}). However, no matter how small $\kappa$ is, $T_{\alpha}[\kappa f_{\epsilon}]$ becomes finite if $\alpha$ approaches $1$ (collapse of the \textit{global} solvability), and $\alpha \to 1-0$ yields $T_{\alpha}[\kappa f_{\epsilon}]\to 0$ (collapse of the \textit{local} solvability).
		}
	\end{Remark}
	
	
	
	
	%%%%%%%%%%%%%%%%%%%%%%%%%%%%%%%%%%%%%
	\section{Applications.}\label{section:Application}
	%%%%%%%%%%%%%%%%%%%%%%%%%%%%%%%%%%%%%
	%%%%%%%%%%%%%%%%%%%%%%%%%%%%%%%%%%%%%
	
	\subsection{Preliminaries}
	
	We apply our results to obtain life span estimates for certain initial data.
	
	\subsection{Life span for the Dirac measure}
	%%%%%%%%%%%%%%%%%%%%%%%%%%%%%%%%%%%%%
	
	Let us consider the problem \eqref{TFF} with
	\begin{equation*}
	\mu=\kappa \delta_N
	\end{equation*}
	where $\kappa>0$ and $\delta_N$ is the $N$-dimensional Dirac measure concentrated at the origin.
	As a corollary of Theorems \ref{Thm:1.1} and \ref{Thm:1.2}, we have the following statement.
	\begin{corollary}
		\label{Cor:1.1}
		Assume $\mu = \kappa \delta_N$, where $\kappa>0$.
		Then one has the following:
		\begin{enumerate}[{\rm (i)}]
			\item If $p<p_F$, then the problem \eqref{TFF} possesses a local-in-time solution for all $\kappa>0$. Furthermore, we have
			\begin{equation*}
			T_{\alpha}[\kappa \delta_N] \simeq \kappa^{-\frac{2(p-1)}{\alpha (2-N(p-1))}}.
			\end{equation*}
			
			\item If $p=p_F$, there exists $\kappa(\alpha)>0$ such that:
			\begin{itemize}
				\item If $\kappa<\kappa(\alpha)$, then the problem \eqref{TFF} possesses a local-in-time solution;
				\item If $\kappa(\alpha)<\kappa$, then the problem \eqref{TFF} possesses no local-in-time solutions.
			\end{itemize}
			
			\item If $p>p_F$, then the problem \eqref{TFF} possesses no local-in-time solutions for all $\kappa>0$.
		\end{enumerate}
	\end{corollary}
	From Corollary \ref{Cor:1.1}, we see that $p=p_F$ is the threshold of the existence of nonnegative local-in-time solutions when the initial value is the Dirac measure. By using the corresponding condition in \cite{HisaIshige18} for problem \eqref{Fjt}, we observe that problem \eqref{Fjt} has no solutions for any $\kappa>0$ if $p=p_F$, while problem \eqref{TFF} can be solvable for small $\kappa>0$. By the similar proof to Theorem~\ref{Thm:4.1}, the following estimate is guaranteed:
	\begin{equation*}
	\kappa(\alpha)\simeq (1-\alpha)^{N/2}.
	\end{equation*}
	
	Let us take the sequence of functions which convergents to $\delta_N$:
	\begin{equation*}
	\psi_j(x):=j \chi_{B(0,r_j)},\ \ r_j:=\left( j \omega_N \right)^{-\frac{1}{N}}
	\end{equation*}
	where $j\in{\bf N}$ and $\omega_N$ is the volume of the $N$-dimensional unit ball. Let us obtain the life span estimate for $\mu=\kappa \psi_j$, and observe its behavior when $j\to \infty$.
	
	
	\begin{theorem}
		Let $p=p_F$. There exists $\kappa_1(\alpha)>0$ such that, if $\kappa_1(\alpha)<\kappa$, then,
		\begin{equation}\label{eq. Talpha for delta1}
		T_{\alpha}[\kappa\psi_j]\le C_{\kappa} j^{-\frac{2}{\alpha N}}
		\end{equation}
		for $C_{\kappa}>0$ depending on $\kappa>0$. We also have $\kappa_1(\alpha)\simeq (1-\alpha)^{N/2}$. Moreover, if $\kappa>0$ is sufficiently large, then
		\[
		C_{\kappa}= C \kappa^{-\frac{2}{\alpha N}}
		\]
		for appropriate $C>0$.
	\end{theorem}
	
	\begin{proof}
		It follows that
		\[
		\int_{B(0, \sigma)}  \psi_j(x)dx= \kappa j \omega_N \min\left\{ \sigma^{N}, r_j^N \right\}.
		\]
		Therefore, the necessary condition \eqref{Thm:1.1.2} requires that
		\[
		\kappa j\omega_N \min\left\{ \sigma^{N}, r_j^N \right\}\le \gamma_1'(\alpha) \left( \int^{1/4}_{\sigma^{2/\alpha}/(16T_{\alpha}[\kappa\psi_j])} t^{-\alpha}\,dt \right)^{-\frac{N}{2}}
		\]
		for all $\sigma\le T_{\alpha}[\kappa \psi_j]^{\alpha/2}$. We show that
		\[
		\kappa_1(\alpha)= 4^{\frac{N}{2}(1-\alpha)} \gamma_1'(\alpha)(1-\alpha)^{\frac{N}{2}}
		\]
		satisfies the desired argument. Indeed, suppose that $\kappa>\kappa_1(\alpha)$. If $r_j\ge T_{\alpha}[\kappa \psi_j]^{\alpha/2}$, we have nothing to prove. Hence, without the loss of generality, we assume that $r_j\le T_{\alpha}[\kappa \psi_j]^{\alpha/2}$. Then, we obtain
		\[
		\kappa \le \gamma_1'(\alpha)(1-\alpha)^{\frac{N}{2}} \left[ 4^{\alpha-1} - \left( \frac{r_j^{2/\alpha}}{16T_{\alpha}[\kappa\psi_j]} \right)^{1-\alpha} \right]^{-\frac{N}{2}}.
		\]
		Since $\kappa>\kappa_1(\alpha)$, this inequality gives that
		\[
		T_{\alpha}[\kappa\psi_j]\le \frac{16^{-1}r_j^{2/\alpha}}{\left[4^{\alpha-1} - \left(\gamma_1'(\alpha)(1-\alpha)^{\frac{N}{2}}\kappa^{-1}\right)^{\frac{2}{N}}\right]^{\frac{1}{1-\alpha}}}
		\]
		and \eqref{eq. Talpha for delta1} is proved. Next, we deduce the estimate for $C_{\kappa}$. We use \eqref{Thm:1.1.1} to get
		\[
		\kappa j \omega_N \min\left\{ \sigma^N, r_j^N \right\}\le \gamma_1(\alpha)
		\]
		for $\sigma\le T_{\alpha}[\kappa\psi_j]^{\alpha/2}$. Note that the condition $\gamma_1(\alpha)<\kappa$ provides that
		\[
		j\omega_N \min\left\{ \sigma^N, r_j^N \right\}< 1
		\]
		and therefore it must be satisfied that $\sigma<r_j$. Substituting $\sigma=T_{\alpha}[\kappa\psi_j]^{\alpha/2}$, we prove that
		\[
		T_{\alpha}[\kappa\psi_j] \le C \left( \kappa j \right)^{-\frac{2}{\alpha N}}
		\]
		for $\kappa>\gamma_1(\alpha)$ and the proof is complete.
	\end{proof}
	
	\begin{theorem}
		Let $p>p_F$.
		\begin{enumerate}[{\rm(i)}]
			\item For each $j\in{\bf N}$, there exists $\kappa_2(j)$ such that, \eqref{TFF} possesses a global-in-time solution if $\kappa<\kappa_2(j)$, while have no global-in-time solution if $\kappa>\kappa_2(j)$. Moreover,
			\begin{equation}\label{eq. kappa2n}
			\kappa_2(j) \simeq j^{\frac{2}{p-1}-1}.
			\end{equation}
			
			\item For appropriate $C>0$, we have
			\begin{equation}\label{eq. Talpha for delta2}
			\min \left\{  T_{\alpha}[\kappa\psi_j]^{\frac{\alpha}{2}}, r_j\right\}\le C \left( \kappa j \right)^{-\frac{p-1}{2}}.
			\end{equation}
			In particular, for all $\kappa>0$,
			\[
			\lim_{j\to \infty} T_{\alpha}[\kappa \psi_j] =0.
			\]
			Moreover, for sufficiently large $j\in{\bf N}$,
			\begin{equation*}\label{eq. Talpha for delta3}
			T_{\alpha}[\kappa\psi_j]\simeq \left( \kappa j \right)^{-\frac{p-1}{\alpha}}.
			\end{equation*}
		\end{enumerate}
	\end{theorem}
	
	\begin{proof}
		\rm{(i)} Let $p>p_F$. By Theorem~\ref{Thm:1.3}, the global solvability is guaranteed if
		\begin{equation*}
		\begin{aligned}
		\kappa j \omega_N^{\frac{1}{r}} \sigma^{\frac{N}{r}} &\le C \sigma^{\frac{N}{r}-\frac{2}{p-1}},\ \ &\mbox{for all}\ \ \sigma<r_j,\\
		\kappa j \omega_N^{\frac{1}{r}} r_j^{\frac{N}{r}} &\le C \sigma^{\frac{N}{r}-\frac{2}{p-1}},\ \ &\mbox{for all}\ \ \sigma\ge r_j,
		\end{aligned}
		\end{equation*}
		where $r>1$ satisfies
		\[
		\frac{2r}{p-1}<N.
		\]
		In particular, the estimate
		\begin{equation*}
		\kappa j \le C r_j^{-\frac{2}{p-1}}
		\end{equation*}
		deduces the global solvability. Therefore, for appropriate $C>0$, there exists a global-in-time solution if $\kappa = C j^{\frac{2}{p-1}-1}$. On the other hand, from Theorem~\ref{Thm:1.1}, if $T_{\alpha}[\kappa \psi_j]=\infty$, then
		\begin{equation*}
		\begin{aligned}
		\kappa j \omega_N \sigma^{N} &\le C \sigma^{N-\frac{2}{p-1}},\ \ &\mbox{for all}\ \ \sigma<r_j,\\
		\kappa j \omega_N r_j^{N} &\le C \sigma^{N-\frac{2}{p-1}},\ \ &\mbox{for all}\ \ \sigma\ge r_j.
		\end{aligned}
		\end{equation*}
		Therefore, $\kappa \le C j^{\frac{2}{p-1}-1}$ is imposed to obtain a global-in-time solution. Therefore, the estimate \eqref{eq. kappa2n} is guaranteed.
		
		\rm{(ii)} By the necessary condition \eqref{Thm:1.1.1}, for all $\sigma \le T_{\alpha}[\kappa \psi_j]^{\alpha/2}$,
		\[
		\kappa j \omega_N \min\left\{ \sigma^N, r_j^N \right\} \le \gamma_1(\alpha) \sigma^{N-\frac{2}{p-1}}.
		\]
		Substituting $\sigma= \min\left\{ T_{\alpha}[\kappa \psi_j]^{\alpha/2}, r_j \right\}$, the estimate \eqref{eq. Talpha for delta2} is obtained. Since $p>p_F$, for sufficiently large $j\in{\bf N}$, it follows that $r_j> C\left( \kappa j \right)^{-(p-1)/2}$. That is, $T_{\alpha}[\kappa \psi_j]^{\alpha/2}\le C \left( \kappa j \right)^{-(p-1)/2}$ holds for large $j\in{\bf N}$.
		
		On the other hand, let $j\in{\bf N}$ be sufficiently large so that $T_{\alpha}[\kappa \psi_j]^{\alpha/2}<r_j$. Let $\overline{T}<r_j$. By Theorem~\ref{Thm:1.3}, the condition
		\[
		\kappa j \omega_N^{\frac{1}{r}} \sigma^{\frac{N}{r}} \le C \sigma^{\frac{N}{r}-\frac{2}{p-1}}
		\]
		for all $\sigma<\overline{T}^{\alpha/2}$ guarantees that $\overline{T} \le T_{\alpha}[\kappa \psi_j]$. We easily see that we can take $\overline{T}= C \left( \kappa j \right)^{-(p-1)/2}$.
	\end{proof}
	
	
	%%%%%%%%%%%%%%%%%%%%%%%%%%%%%%%%%%%%%
	\subsection{Life span for decaying initial data.}
	%%%%%%%%%%%%%%%%%%%%%%%%%%%%%%%%%%%%%
	For problem \eqref{Fjt}, Lee-Ni \cite{LeeNi92} and the first author and Ishige \cite{HisaIshige18} obtained the sharp estimates of $T[\kappa\phi]$ as $\kappa\to+0$. Here, 
	\begin{equation*}
	\phi(x) := (1+|x|)^{-A}
	\end{equation*}
	and $A>0$. We can obtain the analogous estimates of the life span of solutions of problem \eqref{TFF} as an application of our main theorems. However, when $p=p_F$, one deduces the statement which will differ from the corresponding results.
	
	\begin{theorem}
		%p=p_F and A\ge N%
		Suppose that $p=p_F$ and $A \ge 2/(p-1)=N$.
		\begin{enumerate}[\rm(i)]
			\item Let $A>N$. Then, for each $\alpha\in (0,1)$, there exists $K(\alpha)>0$ such that,
			\begin{itemize}
				\item if $\kappa<K(\alpha)$, then $T_{\alpha}[\kappa\phi]=\infty$;
				\item if $\kappa>K(\alpha)$, then there exists $C>0$ such that 	for sufficiently small $\kappa>0$,
				\begin{equation}\label{eq. estimate of Tlambda1}
				T_{\alpha}[\kappa \phi]^{\frac{\alpha-1}{2}} \log T_{\alpha}[\kappa \phi] \le C \kappa^{-(p-1)}.
				\end{equation}
			\end{itemize}
			
			\item Let $A=N$. Then, there exists $C>0$ such that
			\begin{equation*}
			T_{\alpha}[\kappa \phi]^{\frac{\alpha-1}{2p}}\log T_{\alpha}[\kappa\phi]\le C\kappa^{-\frac{p-1}{p}} 
			\end{equation*}
			for sufficiently small $\kappa>0$.
		\end{enumerate}
	\end{theorem}
	
	
	\begin{proof}
		We have
		\begin{equation*}
		\begin{aligned}
		\int_{B(0,T^{\alpha/2})} (1+|x|)^{-A}\,dx &\le C \int_{0}^{T^{\alpha/2}} (1+r)^{-A+N-1}\,dr\\
		&\le \left\{
		\begin{aligned}
		&C \left(1-\left( 1+T^{\alpha/2} \right)^{N-A}\right)\ &&\mbox{if}\ A>N,\\
		&C \log\left(1+T^{\alpha/2}\right)\ &&\mbox{if}\ A=N.
		\end{aligned}
		\right.
		\end{aligned}
		\end{equation*}
		Therefore, for $A>N$, there exists a solution on $[0,T]$ if
		\begin{equation*}
		1-\frac{C\gamma_2(\alpha)}{\kappa}\le \left(1+T^{\frac{\alpha}{2}}\right)^{N-A}.
		\end{equation*}
		This inequality holds for all $T>0$ provided $\kappa \le C\gamma_2(\alpha)=C(1-\alpha)^{N/2}$. Hence,
		\begin{equation*}\label{eq. estimate of kappa1}
		T_{\alpha}[\kappa \phi]=\infty\ \ \mbox{if}\ \ \kappa\le C (1-\alpha)^{\frac{N}{2}}
		\end{equation*}
		for appropriate $C>0$. On the other hand, if $\kappa>C\gamma_2(\alpha)$,
		\begin{equation*}
		\left( 1- \frac{C\gamma_2(\alpha)}{\kappa} \right)^{-\frac{1}{A-N}}\le 1+T_{\alpha}[\kappa\phi]^{\frac{\alpha}{2}}
		\end{equation*}
		holds. In particular, $1\le T_{\alpha}[\kappa\phi]$ for sufficiently small $\kappa>0$. Let us deduce the estimate (\ref{eq. estimate of Tlambda1}). By Theorem~\ref{Thm:1.1}, we have
		\begin{equation*}
		C\kappa\le \kappa \int_{B(0,\sigma)}(1+|x|)^{-A}\,dx \le \gamma_1(\alpha) \left( \int_{\sigma^{2/\alpha}/(16T_{\alpha} [\lambda\phi])}^{1/4} t^{-\alpha}\,dt\right)^{-N/2}
		\end{equation*}
		for $\sigma\ge 1$ and $A>N$. For sufficiently small $\kappa>0$, let us take $1\le \sigma^{2/\alpha}=T_{\alpha}[\kappa\phi]^{1/2}<T_{\alpha}[\kappa \phi]$. Then,
		\begin{equation*}
		C\kappa \le \gamma_1(\alpha) \left( \int_{1/16T_{\alpha}[\kappa \phi]^{1/2}}^{1/4} t^{-\alpha}\,dt \right)^{-\frac{N}{2}}
		\end{equation*}
		holds. Therefore, we obtain
		\begin{equation*}
		T_{\alpha}[\kappa \phi]^{(\alpha-1)/2} \log T_{\alpha}[\kappa \phi]\le \int_{1/16T_{\alpha}[\kappa \phi]^{1/2}}^{1/4}t^{-\alpha}\,dt \le C\kappa^{-(p-1)}.
		\end{equation*}
		
		We can prove assertion (ii) by a similar way. Indeed, it follows that
		\[
		T_{\alpha}[\kappa\phi]^{\frac{\alpha-1}{2}} \log T_{\alpha}[\kappa \phi]\le C \kappa^{-\frac{2}{N}} \left( \log\left( e+ T_{\alpha}[\kappa \phi]^{\frac{\alpha}{4}} \right) \right)^{-\frac{2}{N}}.
		\]
		since
		\[
		C\log(e+\sigma) \le \int_{B(0,\sigma)}(1+|x|)^{-A}\,dx
		\]
		for $\sigma\ge 1$ and $A=N$.
	\end{proof}
	
	\begin{theorem}
		Sppose that $p<p_F$ or $A<2/(p-1)$. Then,
		\[
		T_{\alpha}[\kappa \phi] \simeq \left\{
		\begin{split}
		&\kappa^{-\left( \frac{\alpha}{p-1}-\frac{\alpha}{2}\min\left( A, N \right) \right)^{-1}}\ \ &&\mbox{if}\ A\neq N,\\
		&\left( \frac{\kappa^{-1}}{\log \kappa^{-1}} \right)^{\left( \frac{\alpha}{p-1}-\frac{\alpha N}{2} \right)^{-1}}\ \ &&\mbox{if}\ A=N.
		\end{split}
		\right.
		\]
	\end{theorem}
	
	\begin{proof}
		Firstly, we obtain the lower estimate. Let $p\ge p_F$ and $A<2/(p-1)$. It follows that $A<N$. Let us take $r>1$ which fulfills the assumption of Theorem~\ref{Thm:1.3}. In particular, it follows that $Ar<N$. Therefore, for all $x\in {\bf R}^N$ and $\sigma>0$,
		\begin{equation*}
		\left( \int_{B(0,\sigma)} \left( \kappa \phi(x) \right)^r \,dx \right)^{\frac{1}{r}}\le C \kappa \sigma^{\frac{N}{r}-A}
		\end{equation*}
		holds. Set
		\begin{equation*}
		\tilde{T}_{\kappa} := \delta \kappa^{-\left( \frac{\alpha}{p-1}-\frac{A\alpha}{2} \right)^{-1}}
		\end{equation*}
		for $0<\delta<1$. We have
		\begin{equation*}
		\kappa \sigma^{-A}= \sigma^{-\frac{2}{p-1}} \cdot \kappa \sigma^{\frac{2}{p-1}-A} \le \sigma^{-\frac{2}{p-1}}\cdot \kappa \tilde{T}_{\kappa}^{\frac{\alpha}{p-1}-\frac{A\alpha}{2}}= \delta^{\frac{\alpha}{p-1}-\frac{A\alpha}{2}} \sigma^{-\frac{2}{p-1}} 
		\end{equation*}
		for $0<\sigma <\tilde{T}_{\kappa}^{\alpha/2}$. We deduce the desired result by taking $\delta>0$ sufficiently small so that \eqref{Thm:1.3.1} with $T=\tilde{T}_{\kappa}$ is satisfied.
		
		Suppose that $p<p_F$. We see that
		\begin{equation*}
		\int_{B(0, \sigma)} \kappa \phi(x)\,dx \le \left\{
		\begin{split}
		& C\kappa\ &&\mbox{if}\ A>N,\\
		& C\kappa \log\left( 1+ \sigma \right)\ &&\mbox{if}\ A=N,\\
		& C\kappa \sigma^{\frac{N-A}{2}}\ &&\mbox{if}\ A<N.
		\end{split}
		\right.
		\end{equation*}
		Set
		\begin{equation*}
		\hat{T}_{\kappa}:= \left\{
		\begin{split}
		&\delta \kappa^{-\left( \frac{\alpha}{p-1}-\frac{\alpha}{2}\min(A,N) \right)^{-1}}\ &&\mbox{if}\ A\neq N,\\
		&\delta \left( \frac{\kappa^{-1}}{\log\left( \kappa^{-1} \right)} \right)^{\left( \frac{\alpha}{p-1}-\frac{N\alpha}{2} \right)^{-1}}\ &&\mbox{if}\ A=N.
		\end{split}
		\right.
		\end{equation*}
		We are able to take sufficiently small $\delta>0$ such that \eqref{Thm:1.2.1} is satisfied with $T=\hat{T}_{\kappa}$.
		
		Let us prove the upper estimate. It follows that for $\sigma>1$,
		\begin{equation}\label{lower est of phi}
		\int_{B(0,\sigma)} \kappa \phi(x)\,dx \ge \left\{
		\begin{split}
		&C \kappa\ &&\mbox{if}\ A>N,\\
		&C \kappa \log\left( e+\sigma \right)\ &&\mbox{if}\ A=N,\\
		&C \kappa \sigma^{N-A}\ &&\mbox{if}\ A<N.
		\end{split}
		\right.
		\end{equation}
		Therefore, by using \eqref{Thm:1.1.1} and \eqref{lower est of phi} with $\sigma=T_{\alpha}[\kappa\phi]^{\alpha/2}/2\ge 1$ for sufficiently small $\kappa>0$, we obtain the desired result.
	\end{proof}
	
	
	%%%%%%%%%%%%%%%%%%%%%%%%%%%%%%%%%%%%%
	\section*{Conclusion and Acknowledgments.}
	%%%%%%%%%%%%%%%%%%%%%%%%%%%%%%%%%%%%%
	In the present paper, we deduce the necessary and sufficient conditions for the solvability of \eqref{TFF}. As applications, we firstly observe the global-in-time solvability of \eqref{TFF} with $p=p_F$ fails when $\alpha\to1$. More precisely, this concept is represented as the shrinking of $\mathcal{G}_{\alpha}$. Moreover, we provide the life span estimate for certain initial data $f_{\epsilon}$ which admits no local-in-time solutions of \eqref{Fjt}. The collapse of the solvability of \eqref{TFF} for this initial data $f_{\epsilon}$ is represented as the convergence $T_{\alpha}\left[ f_{\epsilon} \right]\to0$.
	
	Finally, the authors of this paper would like to thank the referees for carefully reading the manuscript and relevant remarks. 
	
	
	%%%%%%%%%%%%%%%%%%%%%%%%%%%%%%%%%%%%%%
	%%%%%%%%%%%%    references    %%%%%%%%%%%%%%%%%%
	%%%%%%%%%%%%%%%%%%%%%%%%%%%%%%%%%%%%%%
	\bibliographystyle{plain}
	\bibliography{ref}
	
	
\end{document}

%-----------------------------------------------------------------------
% End of amsart-template.tex
%-----------------------------------------------------------------------
