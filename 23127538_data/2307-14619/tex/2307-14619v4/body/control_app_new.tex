%!TEX root = ../main.tex
\newcommand{\Delkinf}{\Delta_{\bK,\infty}}


	\newcommand{\Deluinf}{\Delta_{\bu,\infty}}
\newcommand{\Deluk}[1][k]{\Delta_{\bu,#1}}
\newcommand{\Delxk}[1][k]{\Delta_{\bx,#1}}
\newcommand{\diststau}[1][\tau]{\dist_{\cS,#1}}
\newcommand{\distsxtau}[1][\tau]{\dist_{\cS,\bx,#1}}
\newcommand{\distsutau}[1][\tau]{\dist_{\cS,\bu,#1}}
\newcommand{\distAtau}{\bar{\dist}_{\cA,\tau}}
\newcommand{\epsbar}{\bar{\epsilon}}
\newcommand{\Kric}{\matK^{\mathrm{ric}}}
\newcommand{\Pric}{\matP^{\mathrm{ric}}}
\newcommand{\ctrajbar}{\bar{\ctraj}}
\newcommand{\epsu}{\epsilon_{\bu}}
\newcommand{\Rem}{\mathrm{rem}}
\newcommand{\btilu}{\tilde{\bu}}
\newcommand{\trajoff}{\ctraj}
\newcommand{\Deltilxk}[1][k]{\tilde{\Delta}_{\bx,#1}}
\newcommand{\Delbarxk}[1][k]{\bar{\Delta}_{\bx,#1}}
\newcommand{\bhatAk}[1][k]{\hat{\bA}_{#1}}
\newcommand{\bhatBk}[1][k]{\hat{\bB}_{#1}}
\newcommand{\ctrajat}{\hat{\ctraj}}
	\newcommand{\ErrK}{\mathrm{Err}_{\mathbf{K}}}
\newcommand{\Erru}{\Err_{\mathbf{u}}}


\newcommand{\Cstabnum}[1]{C_{\mathrm{stab},#1}}

\newcommand{\Rnot}{R_0}
\newcommand{\rmax}{r_{\max}}
\newcommand{\distarnotx}{\dist_{\cA,\Rnot,\tau,\bx}}
\newcommand{\bxpr}{\bx'}
\newcommand{\bupr}{\bu'}
\newcommand{\Constu}{C_{\mathbf{u}}}
\newcommand{\Constuinf}{C_{\mathbf{u},\infty}}


\newcommand{\Constk}{C_{\mathbf{K}}}
\newcommand{\Constkx}{C_{\mathbf{K},\bxoff}}

\newcommand{\Constdelx}{C_{\bm{\Delta}}}
\newcommand{\Constxhat}{C_{\hat{\bx}}}
\newcommand{\constu}{c_{\bu}}
\newcommand{\constk}{c_{\bK}}
\newcommand{\constdelx}{c_{\bm{\Delta}}}


\newcommand{\Constxhatk}{C_{\bK,\hat{\bx}}}

\newcommand{\buoff}{\bu}
\newcommand{\bxoff}{\bx}
	\newcommand{\Term}{\mathrm{Term}}

\newcommand{\Aclhatk}[1][k]{\hat{\bA}_{\mathrm{cl},#1}}
\newcommand{\Phiclhat}[1]{\hat{\bm{\Phi}}_{\mathrm{cl},#1}}
\newcommand{\rem}{\mathrm{rem}}
\newcommand{\matKtil}{\tilde{\matK}}
\newcommand{\Path}{\mathscr{P}}
\newcommand{\ltwo}{\ell_2}
\newcommand{\ltwoop}{\ell_2,\op}
\newcommand{\DelKk}[1][k]{\Delta_{\bK,#1}}
\newcommand{\partx}{\partial x}
\newcommand{\partu}{\partial u}


\newcommand{\Ajac}{\mathbf{A}_{\mathrm{jac}}}
\newcommand{\Bjac}{\mathbf{B}_{\mathrm{jac}}}

\newcommand{\Bdyn}{B_{\mathrm{dyn}}}


\newcommand{\Lf}{L_{\mathrm{dyn}}}


\newcommand{\matX}{\mathbf{X}}
\newcommand{\matY}{\mathbf{Y}}
\newcommand{\matQ}{\mathbf{Q}}
\newcommand{\matLam}{\bm{\Lambda}}
\newcommand{\matPhi}{\bm{\Phi}}
\newcommand{\maxop}{\max,\op}


\newcommand{\matP}{\mathbf{P}}
\newcommand{\matK}{\mathbf{K}}
\newcommand{\matA}{\mathbf{A}}
\newcommand{\matB}{\mathbf{B}}
\newcommand{\matTheta}{\bm{\Theta}}
\newcommand{\Rstabtil}{\tilde{R}_{\mathrm{stab}}}
\newcommand{\betastab}{\beta_{\mathrm{stab}}}

\newcommand{\Aclkric}[1][k]{\matA_{\mathrm{cl},#1}^{\mathrm{ric}}}


\section{Stability in the Control System}\label{app:control_stability}
This section proves our various stability conditions. Precisely, we establish the following guarantee: 
\begin{proposition}\label{prop:ips_instant_app}  Let  $\cgamma,\cxi,\cbarbeta,\cbargamma,\lamiss$ be the constants defined in \Cref{asm:iss_body}. In terms of these,  define
\begin{align}
\alpha &=  \cbarbeta (4\cbargamma \min\{\cgamma,\cxi/4\cbargamma\} +  \cxi)\\
c_1 &:= 4\cbargamma\cbarbeta (2+\alpha\Lstab+2\Rdyn) \\
c_2 &:= \max\{1,c_1)^{-1}\min\{\cgamma,\cxi/2\cbargamma\}\\
c_3 &:= \frac{1}{\lamiss} \log(2e\cbarbeta)\\
c_4 &:= \cxi/2\\
c_5 &:= 2\cbarbeta
\end{align}
For actions $\seqa = (\sfk_k)_{1 \le k \le \tauc}$ where $\sfk_k(\bx) = \bbaru_k + \bbarK_k(\bx-\bbarx_k)$ are affine primitive controllers, define $\dmax(\seqa,\seqa') := \max_{1\le k \le \tauc}(\|\bbaru_{k}-\bbaru_{k}'\| + \|\bbarx_{k}-\bbarx_{k}'\| +\|\bbarK_{k}-\bbarK_{k}'\|)$, and let
 \begin{align}
 \distA(\seqa,\seqa') &:= c_1 \dmax(\seqa,\seqa') \cdot \I_{\infty}\{\dmax(\seqa,\seqa') \le c_2\} \label{eq:dA_app}
 \end{align}
 Then, if  $\tauc \ge c_3/\eta$,  the policy $\pist$ as defined in \Cref{def:Dexp_policies} satisfies $(\rips,\gamipsone,\gamipstwo,\distips)$-restricted IPS (\Cref{defn:ips_restricted}) with $\distA$ as above, and with
  \begin{align}
  \rips = c_4, \quad \gamipsone(u) = c_5 u \exp(-\eta(\tauc - \taum)/\Lstab), \quad\gamipstwo(u) = c_5 u.
  \end{align}
\end{proposition}
%One wrinkle in the exposition is that we are able to derive far sharper perturbation guarantees than are needed in our analysis. 

%However, as the guarantees are rather technically burdensome to derive, we endeavor to present the sharpest possible results so that we may save others from having to rederive these bounds in future applications. 
 \Cref{proof:of_prop_ips_instant} proves \Cref{prop:ips_instant_app}, based on a lemma whose proof is given in \Cref{proof:lem:iss_ips}. 

 In what follows, we justify our assumption of a stabilizing synthesis oracle, \Cref{asm:iss_body}. First, \Cref{sec:prop:master_stability_lem} shows that if the system dynamics are \emph{smooth}, than time-varying affine controllers whose gains stabilize the Jacobian linearization of the given system satisfy \Cref{defn:tiss}. This result is stated  in \Cref{sec:stab_of_trajectories}, along with the requisite assumptions, and proven in \Cref{sec:prop:master_stability_lem} , based on a lemma whose proof is given in \Cref{sec:lem:state_pert}. 


Finally, \Cref{sec:ric_synth} shows how a synthesis oracle can produce gains which stabilize the linearized dynamics can  obtained by solving the Riccati equation, assuming sufficient dynamical regularity. Finally, \Cref{sec:recursion_solutions} gives the solutions to various scalar recursions used in the proofs throughout.








\subsection{Proof of \Cref{prop:ips_instant_app}}\label{proof:of_prop_ips_instant}
We now translate the incremental stability guarantee about into the IPS guarantee needed by \Cref{prop:ips_instant_app}. The core technical ingredient is the following lemma, whose proof we defer to \Cref{proof:lem:iss_ips}.

\begin{lemma}[Trajectory-tracking via $\tiss$]\label{lem:iss_ips} Consider a given sequence $(\btilx_{t_h})$, and suppose that $\seqa_h = \sfk_{t_h:t_{h+1}-1}$ is local-$\tiss$ at $\btilx_{t_h}$ for each $1 \le h \le H$ (with parameters as in \Cref{defn:tiss}. Consider consistent trajectories $(\bx_{1:T+1},\bu_{1:T})$, $(\bx_{1:T+1}',\bu_{1:T})$ satisfying 
	\begin{align}
	\bu_{t} =  \sfk_{t}(\bx_{t}), \quad \bu_{t+1}' = \sfk_{t}'(\bx_{t}'), \quad \bx_{1}=\bx_1', \quad \max_{h}\|\btilx_{t_h}-\bx_{t_h}\| \le r\le \cxi/2
	\end{align}
	Further, define the sequence $(\tilde{\bx}_t)$ by setting, for each $h$, 
	\begin{align}
	\bhatx_{t_h} := \btilx_{t_h}, \quad \bhatx_{t_h+i} := f(\bhatx_{t_h+i},\sfk_t(\bhatx_{t_h+i-1})), \quad i \in \{1,2,\dots,\tauc-1\} \label{eq:bhat_dyn}
	\end{align}
	Then, the following guarantees hold
	\begin{itemize}
	\item[(a)] $\|\bx_{t_h+i} - \bhatx_{t_{h}+i}\| \le \betaiss(r,i)$ for $i \in \{0,1,2,\dots,\tauc-1\}$ and $h \in [H]$.
	\item[(b)]  Suppose that $\epsilon > 0$  satisfies
	\begin{align}
\gammaiss^{-1}(\betaiss(2\gammaiss(\epsilon),\tauc) \le \epsilon \le \min\{\cgamma,\gammaiss^{-1}(\cxi/4)\} \label{eq:eps_cond_general_two}
\end{align} 
and that one of the following hold
	\begin{align}
	&\max_{1 \le t \le T}\sup_{\|\delx\| \le \delR(\epsilon)}\|\sfk_t(\bx_t+\delx)-\sfk_t'(\bx_t+\delx)\| \le \epsilon, \quad  \alpha(\epsilon) := 2\betaiss(2\gammaiss(\epsilon),0), \quad \text{ or } \label{eq:alpha_iss_eq}\\
	&\max_{1 \le t \le T}\sup_{\|\delx\| \le \alpha(\epsilon)+\betaiss(r,0)}\|\sfk_t(\bhatx_t+\delx)-\sfk_t'(\bhatx_t+\delx)\| \le \epsilon, \label{eq:alphatil_iss_eq}
	\end{align}
	Then for all $h \in [H]$, $i \in \{0,1,\dots,\tauc\}$, and $t \in [T]$,
	\begin{align}
	\|\bu_t - \bu_t'\| \le \epsilon \le \alpha \quad \|\bx_{t_h+i} - \bx_{t_{h}+i}'\| \le \betaiss(2\gammaiss(\epsilon),i) + \gammaiss(\epsilon) \le \alpha \label{eq:bu_bx_pert}
	\end{align} 
	\end{itemize}
	\end{lemma}

\newcommand{\distAbar}{\bar{\dist}_{\cA}}
\newcommand{\Distalpha}{\mathsf{D}_{\alpha}}

As a consequence, we derive the following reduction from IPS and input-stability in the composite MDP to $\tiss$.
\begin{definition}[Instantiation of the composite MDP for general primitive controllers]\label{defn:composite_instant_general} In this section, we summarize the instantiation of the MDP in \Cref{app:end_to_end}:
\begin{itemize}
	\item States $\seqs_h = \pathc$ and $\dists,\disttvc,\distips$ are just as in \Cref{sec:analysis}. Moreover, $\pathm = \phimem \circ \pathc$. 
	\item The kernel $\Wsig(\cdot)$ is the same as \eqref{eq:Gaussian_kernel} in \Cref{app:end_to_end}, applying $\cN(0,\sigma^2 I)$ noise in the coordinates in $\pathm$. 
	\item Actions $\seqa_h$ are sequences of affine primitive controllers $\sfk_{1:\tauc}$.
	\item $\pist = (\pist_h)$ be the policy induced by the conditional distribution of $\seqa \mid \pathm$ as constructed in \Cref{def:Dexp_policies} in \Cref{app:end_to_end}.
\end{itemize}
\end{definition}
\begin{restatable}{lemma}{geninputstable}\label{lem:ips_and_input_stable_not_state_conditioned_general} Instantiate the composite MDP as in \Cref{defn:composite_instant_general}, with $\pist$ as in \Cref{def:Dexp_policies}.  Furthermore, suppose that under $(\ctraj_T,\seqa_{1:H}) \sim \Dexp$ with $\ctraj_T = (\bx_{1:T+1},\bu_T)$, the following both hold with probability one:
\begin{itemize} 
	\item Each action $\seqa_h$ satisfies our notion of incremental stability (\Cref{defn:tiss}) with moduli $\upgamma(\cdot),\upbeta(\cdot,\cdot)$, constants $\cgamma,\cxi$
	\item $\bx_{t} \in \cX_0$ for some set $\cX_0 \subset \R^{\dimx}$, and $\sfk_t \in \cK_0$ for some set of primitive controllers $\cK_0 \subset \cK$.\footnote{This can be directly generalized to a constraint on the composite states $\seqs_h$ and composite actions $\seqa_h$.}  
\end{itemize}
Finally, let $\epsilon_0 > 0$ satisfy \eqref{eq:eps_cond_general}, that is:
	\begin{align}
	\gammaiss^{-1}(\betaiss(2\gammaiss(\epsilon_0),\tauc) \le \epsilon_0 \le \min\{\cgamma,\gammaiss^{-1}(\cxi/4)\}, \label{eq:eps_cond_control}
	\end{align}
	For given $ \alpha > 0$, let $\Distalpha( \seqa,\seqa')$ be a function which, for all composite actions $\seqa = \sfk_{1:\tauc}$ satisfying $\sfk_i \in \cK_0$ all arbitrary composite actions $\seqa'=\sfk_{1:\tauc} \in \cK^{\tauc}$, satisfies  
	\begin{align}
	\Distalpha( \seqa,\seqa') \ge \sup_{\bx \in \cX_0}\sup_{\delx:\|\delx\| \le  \alpha} \max_{1 \le i \le \tauc} \|\sfk_i(\bx_i+\delx)-\sfk_i'(\bx_i+\delx)\|. 
	\end{align}
	and let 
	\begin{align}
	\distAbar( \seqa,\seqa;\alpha) :=  \uppsi(\Distalpha(\seqa,\seqa'))\cdot \cI_{\infty}\left\{\Distalpha( \seqa,\seqa') \le \epsilon_0\right\}, \quad  \uppsi(u) := 2\betaiss(2\gammaiss(u),0).
	\end{align}
	 Then, the following hold:
	\begin{itemize}
		\item[(a)] $\pist$ is input-stable with respect to $\dists,\disttvc$ as defined in \Cref{sec:analysis} 
		\begin{align}
		\distA( \seqa,\seqa') = \distAbar( \seqa,\seqa';\uppsi(\epsilon)), \quad
		\end{align}
		\item[(b)] For any $\rips \le \cxi/2$, $\pist$ is $(\rips,\gamipsone,\gamipstwo,\distips)$- restricted-IPS (\Cref{defn:ips_restricted}) with
		\begin{align}
		\distA( \seqa,\seqa') = \distAbar( \seqa,\seqa';\uppsi(\epsilon)+\upbeta(\rips,0)), \quad \gamipsone(r) = \betaiss(r,\tauc-\taum), \quad \gamipstwo(r) = \betaiss(r,0),
		\end{align}
		\end{itemize}
\end{restatable}
Note that the above lemma holds for general forms of incremental stability. Let us now instantiate it for the form of incremental stability of the form established in \Cref{prop:affine_inc_stable}. 
\begin{corollary}\label{cor:linear_scaling} Suppose that $\upgamma(\epsilon) = \cbargamma \cdot \epsilon$ and $\upbeta(\epsilon,k) = \cbargamma \upphi(k) \cdot \epsilon$. Then, as long as we take
\begin{align}
2\upphi(\tauc)\cbarbeta \le 1, \quad \epsilon_0 := \min\{\cgamma,\cxi/4\cbargamma\},
\end{align}
and setting $\uppsi(\epsilon) = \epsilon$, we have that
\begin{itemize}
		\item[(a)] $\pist$ is input-stable with $\distA( \seqa,\seqa') = \distAbar( \seqa,\seqa';4\cbargamma\cbarbeta \epsilon_0)$.
		\item[(b)] For any $\rips = \cxi$, $\pist$ is $(\rips,\gamipsone,\gamipstwo,\distips)$- restricted-IPS (\Cref{defn:ips_restricted}) with
		\begin{align}
		\distA( \seqa,\seqa') = 4\cbarbeta \cbargamma\distAbar( \seqa,\seqa';\cbarbeta (4\cbargamma\epsilon_0 +  r)), \quad \gamipsone(r) = \cbarbeta  \upphi(\tauc-\taum)r, \quad \gamipstwo(r) = \cbarbeta r
		\end{align}
	\end{itemize}
\end{corollary}
We are now ready to prove the main result of this appendix.

\begin{proof}[Proof of \Cref{prop:ips_instant_app}] Note that, by assumption, we are in the regime of \Cref{cor:linear_scaling}, with $\upphi(k) = e^{-\lamiss(k-1)}$ and $\epsilon_0 := \min\{\cgamma,\cxi/4\cbargamma\}$. We note that, under our assumption $\lamiss \le 1$,
\begin{align}
\upphi(k) = e^{\lamiss} e^{-\lamiss(k-1)} \le e\cdot e^{-\lamiss(k-1)}. \label{eq:phisimplif}
\end{align}
Hence, $2\upphi(\tauc)\cbarbeta \le 1$ for $\tauc \ge c_3= \log(2e\cbarbeta)/\lamiss$. 

Next, we develop $\Distalpha$. Express the primitive controllers $\sfk_i = (\bbaru_i,\bbarx_i,\bbarK_i)$ and  $\sfk_i' = (\bbaru_i',\bbarx_i',\bbarK_i')$. 
%\mscomment{TODO!!??? It suffices to consider $\seqa = (\sfk_k)$ with $\sfk_k = (\bbaru_k,\bbarx_k,\bbarK_k)$ for which $\bbarx_k \in \cX_0$ and $\|\bbarK_k\| \le \Ldyn$}
Recall
\begin{align}
\dmax(\seqa,\seqa') = \max_{1 \le i \le \tauc} \max\{\|\bbaru_i - \bbaru_i'\| + \|\bbarx_i-\bbarx_i'\| + \|\bbarK_i-\bbarK_i'\|\}.
\end{align}
By assumption, the expert distribution $\Dexp$ ensures that $\|\bx_t\| \le \Rdyn$ and that $\|\bbarK_t\| \le \Rstab$. Moreover, it also ensures $\|\bbarx_t\| \le \Rdyn$, since under the expert distribution, $\bbarx_t = \bx_t$. Thus, to find an upper upper bound on the distance $\Distalpha(\seqa,\seqa')$, it suffices to take $\cX_0 = \{\bx:\|\bx\| \le \Rdyn\}$ and bound the following quantity for all $\seqa = \sfk_{1:\tauc}$ and $\seqa' = \sfk_{1:\tauc}'$ for which $\sfk_i = (\bbaru_i,\bbarx_i,\bbarK_i)$ satisfies $\|\bbarx_i\| \le \Rdyn$ and $\|\bbarK_i\| \le \Rstab$: 
\begin{align}
	&\sup_{\bx:\|\bx\| \le \Rdyn}\sup_{\delx:\|\delx\| \le  \alpha} \max_{1 \le t \le \tauc} \|\sfk_t(\bx_t+\delx)-\sfk_t'(\bx_t+\delx)\|\\
	&= \sup_{\bx:\|\bx\| \le \Rdyn}\sup_{\delx:\|\delx\| \le  \alpha} \max_{1 \le t \le \tauc} \|\bbaru_t - \bbaru_t' +  (\bbarK_t-\bbarK_t')(\bx_t + \delx ) + \bbarK_t\bbarx_t-\bbarK_t'\bbarx_t'  \| \\
	&\le \max_{1 \le t \le \tauc} \|\bbaru_t - \bbaru_t'\| +  \|\bbarK_t-\bbarK_t)\|({\alpha}+\sup_{\bx:\|\bx\| \le \Rdyn}\|\bx\|) + \|\bbarK_t(\bbarx_t - \bbarx_t')\|) + \|(\bbarK_t-\bbarK_t)\bbarx_t'\| \\
	& \max_{1 \le t \le \tauc} \|\bbaru_t - \bbaru_t'\| +  \|\bbarK_t-\bbarK_t)\|({\alpha}+\Rdyn) + \Rstab\|\bbarx_t - \bbarx_t'\|) + \|\bbarK_t-\bbarK_t\|\|\bbarx_t'\| \tag{$\|\bbarK_t\| \le \Rstab$}\\
	&\le  \max_{1 \le t \le \tauc} \|\bbaru_t - \bbaru_t'\| +  \|\bbarK_t-\bbarK_t)\|({\alpha}+2\Rdyn + \|\bbarx_t-\bbarx_t'\|) + \Rstab\|\bbarx_t - \bbarx_t'\| \tag{$\|\bbarx_t\| \le \Rdyn$}\\
	&\le  \dmax(\seqa,\seqa')(1+\Rstab {\alpha}+2\Rdyn + \dmax(\seqa,\seqa'));
	\end{align}
	that is, we can take
	\begin{align}
	\Distalpha(\seqa,\seqa')= \dmax(\seqa,\seqa')(1+\Rstab {\alpha}+2\Rdyn + \dmax(\seqa,\seqa')).
	\end{align}
	Now, set $	 \alpha = 4\cbarbeta \cbargamma \epsilon_0 + \cbarbeta \rips = \cbarbeta (4\cbargamma \min\{\cgamma,\cxi/4\cbargamma\} +  \cxi)$. For $c_1 = 4\cbargamma\cbarbeta (2+\alpha\Rstab+2\Rdyn)$ and $c_2 = \max\{1,c_1\}^{-1}\min\{\cgamma,\cxi/4\cbargamma\}$. Then if $\dmax(\seqa,\seqa') \le  c_2$, then,
	\begin{align}
	\Distalpha( \seqa,\seqa') \le \dmax(\seqa,\seqa')(2+\Rstab {\alpha}+2\Rdyn) \le \min\{\cgamma,\cxi/4\cbargamma\}.
	\end{align}
	and, in particular, 
	\begin{align}
	 \distAbar( \seqa,\seqa' \mid \alpha) &\le 4\cbarbeta \cbargamma((2+\Rstab {\alpha}+2\Rdyn)\dmax(\seqa,\seqa')) = c_1\dmax(\seqa,\seqa')
	 \end{align}
	 Hence, unconditionally, 
	 \begin{align}
	 \distAbar( \seqa,\seqa' \mid \alpha) \le c_1 \dmax(\seqa,\seqa') \I_{\infty}\{\dmax(\seqa,\seqa') \le c_2\}
	 \end{align}
	 Thus, $\pi^\star$ satisfies $(\rips,\gamipsone,\gamipstwo,\distips)$- restricted-IPS (\Cref{defn:ips_restricted}) with $\rips = \cxi/2 = c_4$
		\begin{align}
		\distA( \seqa,\seqa') &= c_1 \dmax(\seqa,\seqa') \I_{\infty}\{(\dmax(\seqa,\seqa') \le c_2\}\\
		\gamipsone(r) &= \cbarbeta  \cdot\upphi(k),\quad
		\gamipstwo(r) = \cbarbeta r
		\end{align}
		Using \eqref{eq:phisimplif} and recalling $c_5 =e\cbarbeta $, we conclude  $(\rips,\gamipsone,\gamipstwo,\distips)$- restricted-IPS (\Cref{defn:ips_restricted}) with $\rips = \cxi/2 = c_4$ and
		\begin{align}
		\distA( \seqa,\seqa') &= c_1 \dmax(\seqa,\seqa') \I_{\infty}\{\dmax(\seqa,\seqa') \le c_2\}\\
		\gamipsone(r) &= c_5 e^{-\lamiss(\tauc-\taum)},\quad
		\gamipstwo(r) = c_5 r.
		\end{align}
		This concludes the proof.
\end{proof}


\subsubsection{Proof of \Cref{lem:ips_and_input_stable_not_state_conditioned_general}}
Let's prove \Cref{lem:ips_and_input_stable_not_state_conditioned_general}(a). Let $(\seqs_{1:H+1},\seqa_{1:H})$ be drawn from the distribution induces by $\pist$, and let $\seqa'_{1:H}$ be some other sequences of actions. The primitive controllers and states under the instantiation of the composite MDP for $\seqa_{1:H},\seqa_{1:H}$ $\seqs_{1:H+1}$ respectively be $\sfk_{1:T},\sfk_{1:T}'$ and $\bx_{1:T+1}$. Note that, by \Cref{lem:pistar_existence}(b), each $\bx_t$ has the same marginals as under the expert distribution $\Dexp$ and similarly so does $\seqa_h$, so by the assumption of the lemma, $\bx_t \in \cX_0$ and $\sfk_t \in \cK_0$ with probability one. Thus,
\begin{align}
\sup_{\bx \in \cX_0}\sup_{\delx:\|\delx\| \le \alpha} \max_{t_h \le t \le t_{h+1}-1} \|\sfk_t(\bx_t+\delx)-\sfk_i'(\bx_t+\delx)\| \le \Distalpha(\seqa,\seqa').
\end{align}
In particular if $\epsilon \le \epsilon_0$ and $\Distalpha(\seqa,\seqa') \le \epsilon$, then
\begin{align}
\sup_{\bx \in \cX_0}\sup_{\delx:\|\delx\| \le \alpha} \max_{t_h \le t \le t_{h+1}-1} \|\sfk_t(\bx_t+\delx)-\sfk_i'(\bx_t+\delx)\| \le \epsilon \le \epsilon_0,
\end{align} 
By \Cref{lem:iss_ips}, and the fact that $\upbeta(\epsilon,i)$ is non-increasing in $i$, we find  $\max_h\dists(\seqs_h,\seqs_h') = \max_{t}\{\|\bx_t- \bx_t'\|,\|\bu_t-\bu_t'\|\} \le \uppsi(\epsilon)$, as needed. 

%This is direct from  \eqref{eq:alpha_iss_eq},  in , and the fact that $\bx_t \in \cX_0$ and $\sfk_t \in \sfK_0$ with probability one. 

To prove \Cref{lem:ips_and_input_stable_not_state_conditioned_general}(b), let $(\seqs_{1:H+1},\stil_{1:H+1},\seqa_{1:H})$ be as in the definition of restricted IPS (\Cref{defn:ips_restricted}), let $\seqa_{1:H}'$ be an alternative sequence of composite actions, and unpack these into $(\bx_{1:T+1},\bu_{1:T})$, $(\btilx_{1:T+1},\btilu_{1:T})$, $\sfk_{1:T}$ and $\sfk_{1:T}$ as above. We let $\pathmtil = (\btilx_{t_h-\taum+1:t_h},\btilu_{t_h-\taum+1:t_h}) = \phimem \circ \stil_h$ denote the observation-chunk associated with $\stil_h$. It follows from \Cref{lem:pistar_existence}(d) and the construction in (\Cref{defn:ips_restricted}) that the distribution $(\pathmtil,\seqa_h)$ under this construction is absolutely continuous w.r.t. the distribution of $(\pathm,\seqa_h)$ under $\Dexp$. In particular, this implies that $\seqa_h= \sfk_{t_h:t_{h+1}-1}$ satisfies the incremental stability condition on $\btilx_{t_h}$, as well as the following property: let $\shat_{h+1} = F_h(\stil_h,\seqa_h)$, which concretely are states $(\bhatx_{t_h:t_{h+1}},\bhatu_{t_h:t_{h+1}-1})$ correponding to the dynamics induced by rolling out $\seqa_h=\sfk_{t_h:t_{h+1}-1}$ from $\bhatx_{t_h}$, depicted in \eqref{eq:bhat_dyn}. Then, absolute continuity of $(\pathmtil,\seqa_h)$ with respect to its analogues under $\Dexp$ implies that $\bhatx_{t_h:t_{h+1}}$ is absolutely continuous w.r.t. the distribution of $\bx_{t_h:t_{h+1}}$ under $\Dexp$. Hence, $\bhatx_{t} \in \cX_0$ for $t_h \le t \le t_{h+1}$. By a similarly argument, we also have  $\sfk_t \in \cK_0$ with probability one.Thus, we have
\begin{align}
\sup_{\bx\in \cX_0}\sup_{\|\delx\| \le \alpha}\max_{t_h \le t \le t_{h+1}-1} \|\sfk_t(\bx+\delx)-\sfk_t'(\bx+\delx)\| \le \Distalpha(\seqa_h,\seqa_h'), 
\end{align}
Hence, whenever $\Distalpha(\seqa_h,\seqa_h') \le \epsilon$ for  $\alpha = \uppsi(\epsilon) + \betaiss(r,0)$, then 
\begin{align}
\max_{1 \le t \le T}\sup_{\bx\in \cX_0}\sup_{\|\delx\| \le \uppsi(\epsilon) + \betaiss(r,0)}\|\sfk_t(\bx+\delx)-\sfk_t'(\bx+\delx)\| \le \epsilon \le \epsilon_0, \label{eq:forall_x_thing}
\end{align}
then we find (using $\bhatx_t \in \cX_0$)
\begin{align}
\max_{1 \le t \le T}\sup_{\|\delx\| \le \uppsi(\epsilon) + \betaiss(r,0)}\|\sfk_t(\bhatx_t+\delx)-\sfk_t'(\bhatx_t+\delx)\| \le \epsilon \le \epsilon_0
\end{align} 
Thus when \eqref{eq:forall_x_thing} is true for all $h$, \Cref{lem:iss_ips} again implies $\max_h\dists(\seqs_h,\seqs_h') = \max_{t}\{\|\bx_t- \bx_t'\|,\|\bu_t-\bu_t'\|\} \le \uppsi(\epsilon)$ (again, using $\upbeta(\cdot,i)$ being non-increasing in $i$). This concludes the proof.
 %one can show \mscomment{todo} that the distributions of $\bx_{t}$ and $\btilx_t$  are absolutely continuous with respect to the marginals of the state $\bbarx_t$, with $\bbarx_{1:T+1},\bbaru_{1:T} \sim \Dexp$. Hence, $\bx_t \in \cX_0$ with probability one. Thus, the input-state stability part of IPS follows from exactly the same argument as in \Cref{lem:ips_and_input_stable_not_state_conditioned_general}(a), and the bounds on $\gamipsone$ and $\gamipstwo$ follow directly from the bounds \Cref{lem:iss_ips}(a).
\qed



\subsection{Proof of \Cref{lem:iss_ips}}\label{proof:lem:iss_ips}
We begin with the following simplifying observation, which follows from considering the definition of local $\tiss$ with $\delu_t \equiv 0$ at time $t=0$:
	\begin{observation}\label{obs:simplify} $\betaiss(m,u) \ge u$ for any $u \in [0,\cxi)$.
	\end{observation}


	The inequality $\|\bx_{t_h+i} - \bhatx_{t_{h},i}'\| \le \betaiss(r,i)$ is an imediate consequence of local-$\tiss$ of $\seqa_h$ at $\bhatx_{t_h,0}$. Note further that this means that 
	\begin{align}
	\|\bx_{t_h+i} - \bhatx_{t_{h},i}'\| \le \betaiss(r,i) \le \betaiss(r,0) \le r \le \cxi/2. \label{eq:part_a_lem}
	\end{align}



	Let us prove $\|\bx_{t_h+i} - \bx_{t_{h}+i}'\| \le \betaiss(2\gammaiss(\epsilon),i) + \gammaiss(\epsilon)$. 
	Next, define $\delu_{t} = \sfk_t'(\bx_t') - \sfk_t(\bx_t')$ and $\delx_{t} = \bx_t'-\bx_t$. We begin by fixing a chunk $h$ and arguing along the lines of \citet[Proposition 3.1]{pfrommer2022tasil}.In what follows, we assume either \eqref{eq:alpha_iss_eq} or \eqref{eq:alphatil_iss_eq}, restated here for convenience:
	\begin{align}
	&\max_{1 \le t \le T}\sup_{\|\delx\| \le \delR(\epsilon)}\|\sfk_t(\bx_t+\delx)-\sfk_t'(\bx_t+\delx)\| \le \epsilon, \quad  \alpha(\epsilon) := 2\betaiss(2\gammaiss(\epsilon),0), \quad \text{ or } \label{eq:alpha_iss_eq_two}\\
	&\max_{1 \le t \le T}\sup_{\|\delx\| \le \alpha(\epsilon)+\betaiss(r,0)}\|\sfk_t(\bhatx_t+\delx)-\sfk_t'(\bhatx_t+\delx)\| \le \epsilon, \label{eq:alphatil_iss_eq_two}
	\end{align}


	\begin{claim}\label{claim:fixed_h} Fix $c_0 > 0$. Suppose that, at a given step $h$, $\|\delx_{t_h}\| \le c_0 \le \cxi/2$, and that $ \betaiss(c_0,0) + \gammaiss(\epsilon) \le \alpha$. Then, for all $0 \le i \le \tauc - 1$, $\|\delu_{t_h+i}\| \le \epsilon \le \alpha$ and
	\begin{align}
	\forall 0 \le i \le \tauc, \quad \|\delx_{t_h+i}\| \le \betaiss(c_0,i) + \gammaiss(\epsilon) \le \alpha
	\end{align} 
	\end{claim}
	\begin{proof} 
	We perform induction over $t \ge t_h$. Assume inductively that  $\|\delx_{t}\|  \le \betaiss(c_0,t - t_h) + \gammaiss(\epsilon) \le \alpha$ and $\max_{1 \le s \le t-1}\|\delu_s\| \le \epsilon$; note that this base case $t = t_h$ holds as $\betaiss(c_0,0) \le \alpha$ by \Cref{obs:simplify} and our assumption on $c_0$.  From  the inductive hypothesis and the condition \eqref{eq:alpha_iss_eq_two},
	\begin{align}
	\|\delu_{t}\| &= \|\sfk_{t}'(\bx_t') - \sfk_t(\bx_{t})\| \le \max_{t_{h}\le t\le t_{h+1}-1}\sup_{\|\delx\| \le \alpha}\|\sfk_t(\bx_t+\delx)-\sfk_t'(\bx_t+\delx)\| \le \epsilon.
	\end{align}
	Note that by \eqref{eq:part_a_lem}, \eqref{eq:alphatil_iss_eq_two} also suffices for the above to hold.  
	Hence, in either case $\max_{1 \le s \le t}\|\delu_s\| \le \Delta_h$.  As $\|\bx_{t_h}- \bx'_{t_h}\| \le \cxi/2$, the triangle inequality and \eqref{eq:part_a_lem} imply $\|\bx'_{t_h}- \bhatx_{t_h}\| \le \cxi$. This, and the fact that  $\epsilon \le \cgamma$, allows us to invoke our definition of incremental stabiity in \Cref{defn:tiss}, implying
	\begin{align}
	\|\delx_{t+1}\| \le \betaiss(c_0,t+1-t_h) + \gammaiss(\epsilon),
	\end{align}
	as needed.
	\end{proof}
	To conclude, we argue inductively on $h$ that we can take $c_0 = 2\gammaiss(\epsilon)$ in the above claim. First note that $2\gammaiss(\epsilon) \le \cxi/2$ by assumption. Thus, from \Cref{obs:simplify}, $\gammaiss(\epsilon) = \frac{1}{2}\cdot 2\gammaiss(\epsilon) \le \frac{1}{2}\betaiss(2\gammaiss(\epsilon),0)$. Hence, for $c_0 = 2\gammaiss(\epsilon)$ $\betaiss(c_0,0) + \gammaiss(\epsilon) \le \frac{3}{2}\betaiss(2\gammaiss(\epsilon),0) \le\alpha$.  Moreover, by assumption $\delx_{1} = 0$, the bound $\|\delx_{t_h}\| \le 2\gamma(\epsilon)$ holds trivially for step $h=1$. Assuming it holds for $h$, \Cref{claim:fixed_h} yields 
	\begin{align}
	\forall 0 \le i \le \tauc, \quad \|\delx_{t_h+i}\| \le \betaiss(c_0,i) + \gammaiss(\epsilon) \le \betaiss(c_0,0) + \gammaiss(\epsilon) \le \alpha,
	\end{align}
	where the final inequality follows from the computation above. Moreover, by taking $i = \tauc$,
	\begin{align}
	\|\delx_{t_{h+1}}\| = \|\delx_{t_h+\tauc}\| \le \betaiss(2\gamma(\epsilon),\tauc) + \gammaiss(\epsilon) \le 2\gammaiss(\epsilon),
	\end{align}
	where the last inequality is by the assumption of the lemma. \qed




\subsection{Synthesized Linear Controllers are Incrementally Stabilizing}\label{sec:stab_of_trajectories}

In this section, we give a sufficient condition for incremental stability of affine primitive controllers. Recal our notation of a length-$K$ \emph{control trajectory} is denoted $\ctraj = (x_{1:K+1},u_{1:K}) \in \Ctraj_K = (\R^{\dimx})^{K+1} \times (\R^{\dimu})^K$.  Given such a trajectory,  the \emph{Jacobian linearizations} are denoted 
\begin{align}
\bA_k(\ctraj) := \ddx \feta(\bx_k,\bu_k), \quad \bB_k(\ctraj) := \ddu \feta(\bx_k,\bu_k)
\end{align}
 for $k \in [K]$. Recalling our dynamics map $f(\cdot,\cdot)$, and step size $\eta > 0$,  we say that $\ctraj$ is \emph{feasible} if, for all $k \in [K]$, 
	\begin{align}
		\bx_{k+1} = f(\bx_k,\bu_k), \quad \text{where } f(\bx,\bu) = \bx + \eta \feta(\bx,\bu).
\end{align}
We now introduce a nother of \emph{regularity} on the dynamics, which essentially enforces boundedness and smoothness.  
	\begin{definition}[Trajectory Regularity]\label{defn:control_path_regular} A control trajectory $\ctraj = (\bx_{1:K+1},\bu_{1:K})$  is $(\Rdyn,\Lf,\Mf)$-regular if for all $k \in [K]$ and all $(\bx'_k,\bu'_k) \in \R^{\dimx} \times \R^{\dimu}$ such that $\|\bx'_k-\bx_k\| \vee \|\bu_k-\bu'_k\| \le \Rdyn$,\footnote{Here, $\| \nablatwo \feta(\bx'_t,\bu'_t)\|_{\op} $ denotes the operator-norm of a three-tensor.}
		\begin{align}
		\| \nabla \feta(\bx'_k,\bu'_k)\|_{\op} \le \Lf, \quad  \| \nablatwo \feta(\bx'_k,\bu'_k)\|_{\op} \le \Mf.
		\end{align}
	\end{definition}

	We also recall the definitions around Jacobian stabilization. We start with a definition of Jacobian stabilization for feedback gains, from which we then recover the definition of Jacobian stabilization for primitive controllers given in the body.
	\begin{definition}[Jacobian Stability]\label{defn:Jac_stab}
	Consider $\Rstab,\Lstab,\Bstab \ge 1$.  Consider sequence of gains $\matK_{1:K} \in (\R^{\dimu \times \dimu})^K$ and trajectory $\ctraj  = (\bx_{1:K+1},\bu_{1:K})\in \Ctraj_K$. We say that $(\ctraj,\bK_{1:K})$-is $(\Rstab,\Bstab,\Lstab)$-Jacobian Stable if $\max_{k}\|\matK_k\|_{\op} \le \Bstab$,  and if the closed-loop transition operators defined by
	\begin{align}
	\Phicl{k,j} := (\eye + \step \Aclk[k-1]) \cdot(\eye + \step\Aclk[k-2]) \cdot (\dots) \cdot (\eye + \step \Aclk[j])
	\end{align} 
	with $\Aclk[k] = \bA_k(\ctraj) + \bB_{k-1}(\ctraj) \bK_{k-1}$ satisfies the following inequality
	\begin{align}
	\|\Phicl{k,j}\|_{\op} \le \Bstab(1 - \frac{\eta}{\Lstab})^{k-j}.
	\end{align}
	\end{definition}



	The following proposition is proven in \Cref{sec:prop:master_stability_lem}, establishing incremental stability of affine gains. 
\newcommand{\cxione}{c_{\xi,1}}
\newcommand{\cxitwo}{c_{\xi,2}}

	\begin{proposition}[Incremental Stability of Affine Primitive Controller]\label{prop:affine_inc_stable} Suppose that $\ctrajbar = (\bbarx_{1:K+1},\bbaru_{1:K})$ is $(\Rdyn,\Ldyn,\Mdyn)$ regular, and suppose $(\ctrajbar,\bbarK_{1:K})$ is $(\Rstab,\Bstab,\Lstab)$ stable. Suppose that $\eta \le \Lstab/2$, that $\Rstab \ge 1$, define the constants
	\begin{align}
	 \cxione &= \frac{1}{4\Rstab\Bstab}\min\left\{1,\frac{1}{4\Lstab \Mdyn\Rstab\Bstab}\right\}\\
	 \cxitwo &= \min\left\{\frac{1}{96\Bstab\Mdyn\Rstab^2},\frac{\Rdyn}{32\Rstab}\right\}\\
	 \cxi &= \min\{\cxione,\cxitwo/2\}\\
	 \cgamma &= \min\left\{\frac{1}{48\Bstab\Mdyn\Rstab^2},\frac{\Rdyn}{16\Lstab\Rstab}\right\}\\
	 \cbarbeta &:=16\Bstab\\
	 \cbargamma &:= 8\Lstab\Bstab\Ldyn
	 \end{align}
	 and set
	 \begin{align}
	\upbeta(u,k) = \cbarbeta 
	\left(1 - \frac{\eta}{\Lstab}\right)^{k-1}\cdot u , \quad \upgamma(u) := \cbargamma \cdot u
	\end{align}
	Then, the controllers $\sfk_k(\bx) = \bbarK_k(\bx-\bbarx_k) + \bbaru_k$ are incrementally stabilizing in the sense of \Cref{defn:tiss} with moduli $\gammaiss(\cdot)$ and $\betaiss(\cdot,\cdot)$ and constants $ \cxi,\cgamma$ as above. 
	\end{proposition}
	 



\subsection{Proof of \Cref{prop:affine_inc_stable} (incremental stability of affine gains)}\label{sec:prop:master_stability_lem}
 We require the following lemma, proven in the section below.
 \begin{lemma}[Stability to State Perturbation]\label{lem:state_pert} Let $\ctrajbar   = (\bbarx _{1:K+1},\bbaru _{1:K}) \in \scrP_K$ be an $(\Rdyn,\Ldyn,\Mdyn)$-regular and feasible path, and let $\bK_{1:K}$ be gains such that $(\ctrajbar  ,\bK_{1:K})$ is $(\Rstab,\Bstab,\Lstab)$-stable. Assume that $\Rstab \ge 1$, $\Lstab \ge 2\eta$. Fix  another $\bxoff_1$ and define another trajectory $\trajoff$ via 
	\begin{align}\bu_k = \bbaru _k + \bK_k(\bxoff_k - \bbarx _k), \quad \bxoff_{k+1} = \bbarx _k + \eta \feta(\bxoff_k,\buoff_k)
	\end{align}
	Then,  if 
	\begin{align}\|\bxoff_1 - \bbarx _1\| \le \cxione := \frac{1}{4\Rstab\Bstab}\min\left\{1,\frac{1}{4\Lstab \Mdyn\Rstab\Bstab}\right\},
	\end{align}
	then 
	\begin{itemize}
		\item $\|\bxoff_{k+1} - \bbarx _{k+1} \| \le 2\Bstab\|\bxoff_1-\bbarx _1\|  \betastab^{k}$.
		\item $(\trajoff,\bK_{1:K})$ is $(\Rstab,2\Bstab,\Lstab)$-stable. 
		\item $\|\bB_k(\trajoff)\| \le \Ldyn$, and in addition, the trajectory $\trajoff$ is $(\Rdyn/2,\Ldyn,\Mdyn)$-regular. 
	\end{itemize} 
	\end{lemma}

	This lemma is proven in \Cref{sec:lem:state_pert} just below. Now, set $\betastab = (1-\eta/\Lstab)$, and define $\delx_k = \bx_k' - \bx_k$. Let $\bA_{t} = \frac{\partial}{\partial x} \feta(x,u)\big{|
}_{(x,u) = (\bx_k,\bu_k)}$
\begin{align}
\bx_{t+1}' = \bx_k + \eta \feta(\bx_k',\sfk_k(\bx_k') + \delu_k), \quad \bx_{t+1} = \bx_k + \eta \feta(\bx_k,\sfk_k(\bx_k))
\end{align}
This means that 
\begin{align}
\delx_{t+1} = \underbrace{(\eye + \eta (\bA_k + \bB_k\bbarK_k))}_{=\Aclk}\delx_k + \eta \bB_k \delu_k + \eta \Rem_k,
\end{align}
where $\Rem_k = \feta(\bx_k',\sfk_k(\bx_k') + \delu_k) - \feta(\bx_k,\sfk_k(\bx_k)) - (\bA_k + \bB_k\bbarK_k)\delx_k - \bB_k \delu_k$. Defining $\Phicl{k,j} := \Aclk[t-1]\Aclk[t-2]\dots\Aclk[s]$ and unfolding the recursion,
\begin{align}
\delx_{t+1} =  \eta \sum_{s=1}^t \Phicl{k+1,j+1}(\bB_s \delu_k + \Rem_k) + \Phicl{k+1,1}\delx_1
\end{align}
Define $\epsilon_k = \|\delx_k\|$ and $\epsu := \max_{1 \le t \le T}\|\delu_k\|$. Then, we have
\begin{align}
\epsilon_{t+1} &\le \eta \sum_{j=1}^k \|\Phicl{k+1,j+1}\| (\Ldyn\epsu + \|\Rem_k\|) + \|\Phicl{k+1,1}\|\epsilon_1 \\
&\overset{(i)}{\le} 2\Bstab\left(\eta \sum_{j=1}^k \betastab^{k-j} (\Ldyn\epsu + \|\Rem_k\|) + \betastab^{k}\epsilon_1\right)\\
&\overset{(ii)}{\le} 2\Bstab\left(\eta \sum_{j=1}^k \betastab^{k-j} (\Ldyn\epsu +  \Mdyn((1+2\Rstab^2)\epsilon_k^2 + 2\epsu^2)) + \betastab^t\epsilon_1\right) \\
&\overset{(iii)}{\le} 2\Bstab\left(\eta \sum_{j=1}^k \betastab^{k-j} (2\Ldyn\epsu +  \Mdyn(1+2\Rstab^2)\epsilon_k^2 ) + \betastab^t\epsilon_1\right) 
\label{eq:intermediate_stabilize_khing}
\end{align}
where we in (i) $\|\Phicl{k,j}\| \le 2\Bstab \betastab^{t-s}$, and $(ii)$ follows by \Cref{claim:rem}, stated and proven below, and the following inductive hypothesis 
\begin{align}
\max_{1 \le j \le k} \epsilon_k \le C' = \frac{\Rdyn}{4\Rstab}  \tag{Inductive Hypothesis}\label{eq:control_inductive},
\end{align}
and (ii) uses the assumption $\epsu \le \frac{\Ldyn}{2\Mdyn}$.  Setting $\Delta_1 = 2\Bstab\epsilon_1, \Delta_2 = 4\Bstab\Ldyn\epsu$ and $C = 2\Bstab\Mdyn(1+2\Rstab^2) \le 6\Bstab\Mdyn\Rstab^2$, \Cref{lem:third_recursion} implies
\begin{align}
\epsilon_{k} \le 4\Delta_1 \betastab^{k-1} + 2\Lstab\Delta_2 = \underbrace{8\Bstab\epsilon_1\betastab^{k-1}}_{\upbeta(\epsilon_1,k)} + \underbrace{8\Lstab\Bstab\Ldyn\epsu}_{\upgamma(\epsu)}
\end{align}
provided that that $\Delta_2\le\min\left\{\frac{1}{8CL}, \frac{C'}{4L}\right\}$ and $\Delta_1  \le \min\left\{\frac{1}{16CL},\frac{C'}{8}\right\}$ for $L = \Lstab$. Subsituting in relevant quantities and keeping the shorthand $L = \Lstab$,  it suffices that
\begin{align}
\min\left\{\frac{1}{16CL},\frac{C'}{8}\right\}& \ge \underbrace{\min\left\{\frac{1}{96\Bstab\Mdyn\Rstab^2},\frac{\Rdyn}{32\Rstab}\right\}}_{=\cxitwo/2} \ge \epsilon_1\\
\min\left\{\frac{1}{8CL}, \frac{C'}{4L}\right\}& \ge \underbrace{\min\left\{\frac{1}{48\Bstab\Mdyn\Rstab^2},\frac{\Rdyn}{16\Lstab\Rstab}\right\}}_{=\cgamma} \ge \epsu.
\end{align}
\qed


\begin{claim}\label{claim:rem} Suppose that $\epsu \le \frac{\Rdyn}{4}$ and $\epsilon_k \le \frac{\Rdyn}{4\Rstab}$. Then,  $\|\Rem_k\| \le \Mdyn((1+2\Rstab^2)\epsilon_k^2 + 2\epsu^2) $. 
\end{claim}
\begin{proof} Define $\bu_k = \sfk_k(\bx_k)$ and $\delu_k' = \sfk_k(\bx_k')  \delu_k - \bu_k$. We have that $\delu_k' = \delu_k + \sfk_k(\bx_k') - \sfk(\bx_k) = \delu_k + \bbarK_k (\bx'_k-\bx_k)$. We bound $\|\delu_k'\| \le \|\delu_k\| + \|\bbarK_k (\bx'_k-\bx_k)\| \le \|\delu_k\| + \Rstab\|\delx_k\|$,  where we recall $\|\bbarK_k\| \le \Rstab$ and $\delx_k = \bx'_k-\bx$. By assumption and definition, $\|\delu_k\| \le \epsu $ and by definition of $\epsilon_k$ we conclude that
\begin{align}
\|\delx_k\| \le \epsilon_k, \quad \|\delu_k'\| \le (1+\Rstab)\epsilon_k+\epsu  \label{eq:delutpr}
\end{align}
  Consider a curve $\bx_k(s) = \bx_k + s\delx_k$ and $\bu_k(s) = \delu_k' + \bu_k$. With these definition
\begin{align}
\Rem_k &= \feta(\bx_k(1),\bu_k(1)) - \feta(\bx_k(0),\bu_k(0)) - (\bA_k + \bB_k\bbarK_k)\delx_k - \bB_k \delu_k\\
&= \feta(\bx_k(1),\bu_k(1)) - \feta(\bx_k(0),\bu_k(0)) - \bA_k \bx_k + \bB_k\delu_k'\\
&= \underbrace{\frac{\partial}{\partial s} (\feta(\bx_k(s),\bu_k(s)))   -  \bA_k \delx_k + \bB_k \delu_k'}_{=0}\\
&\quad + \int_{0}^s (1-s)^2 \frac{\partial^2}{\partial s^2} (\feta(\bx_k(s),\bu_k(s)))^\top \big{|}_{s=0}(\delx_k,\delu_k')\rmd s
%&= \bA_k\bx_k + \bB_k\delu_k - (\bA_k + \bB_k\bbarK_k)\delx_k - \bB_k \delu_k + 
%
%\feta(\bx_k',\sfk_k(\bx_k') + \delu_k) - \feta(\bx_k,\sfk_k(\bx_k)) - (\bA_k + \bB_k\bbarK_k)\delx_k - \bB_k \delu_k
\end{align}
Thus, 
\begin{align}
\|\Rem_k\| &\le \|\frac{1}{2}\sup_{s \in [0,1]} \frac{\partial^2}{\partial s^2} (\feta(\bx_k(s),\bu_k(s)))\| \le \Mdyn \sup_{s}\|\dds (\bx_k(s),\bu_k(s)\|^2 \\
&\overset{(i)}{\le} \Mdyn(\|\delx_k\|^2 + \|\delu_k'\|^2)\\
&\le \Mdyn((1+2\Rstab^2)\epsilon_k^2 + 2\epsu^2) \tag{\eqref{eq:delutpr} and Am-GM},
\end{align}
To justify inequality  $(i)$, we observe that $\ctraj = (\bx_{1:K+1},\bu_{1:K})$ is $(\Rdyn/2,\Ldyn,\Mdyn)$ regular. Note tat $\sup_s\|\bx_k(s)-\bx_k\| = \|\delx_t\|$ and $\sup_s\|\bu_k(s)-\bu_k\| = \|\delu_t'\|$. Hence, by the definition of trajectory regularity (\Cref{defn:control_path_regular}), $(i)$ holds as long as we check that $\|\delx_k\| \vee \|\delu_k'\| \le \Rdyn/4$. As $\|\delx_k\| \vee \|\delu_k\| \le \max\{\Rstab\epsilon_k+\epsu,\epsilon_k\}$ and as we take $\Rstab \ge 1$, it suffices that $\epsu \le \frac{\Rdyn}{4}$ and $\epsilon_k \le \frac{\Rdyn}{4\Rstab}$, which is ensured by the claim.
\end{proof}

\subsection{Proof of \Cref{lem:state_pert} (state perturbation)}\label{sec:lem:state_pert}
Define $\Delbarxk = \bxoff_k - \bbarx _k$. Then
\begin{align}
\Delbarxk[k+1] &= \Delbarxk + \eta \left(\feta(\bxoff_k,\bbaru _k + \bbarK_k (\bxoff_k - \bbarx_k) - \feta(\bbarx _k,\bbaru _k)\right)\\
&= \Delbarxk + \eta (\bA_k(\ctrajbar) + \bB_k(\ctrajbar) \bK_k)\Delxk + \rem_k, \label{eq:recur}
\end{align}
where 
\begin{align}
\rem_k = \feta(\bxoff_k,\bbaru _k + \bK_k (\bxoff_k - \bbarx _k)) - \feta(\bbarx _k,\bbaru _k) - (\bA_k(\ctrajbar) + \bB_k(\ctrajbar) \bK_k)\Delbarxk.
\end{align}
\begin{claim}\label{claim:taylor_xhat} Take $\Rstab \ge 1$, and suppose $\|\Delbarxk\| \le \Rdyn/2\Rstab$. Then, 
\begin{align}
\|\bbarx _k - \bxoff_k\| \vee \|\bbaru _k - \buoff_k\| \le \Rdyn/2, \label{eq:close_Rstab_within}
\end{align}
and $\|\rem_k\| \le \Mdyn \Rstab^2 \|\Delbarxk\|^2$. 
\end{claim}
\begin{proof}[Proof]  Let $\buoff_k = \bbaru _k + \bK_k(\bxoff_k - \bbarx _k)$. The conditions of the claim imply $\|\buoff_k - \bbaru _k\| \vee\|\bxoff_k \vee \bbarx _k\| \le \Rdyn/2$.  From Taylor's theorem and the fact that $\ctrajbar $ is $(\Rdyn,\Ldyn,\Mdyn)$-regular imply that 
\begin{align}
\|\feta(\bxoff_k,\buoff_k) - \feta(\bbarx _k,\bbaru _k)\| &\le \frac{1}{2}\Mdyn(\|\bxoff_k- \bbarx _k\|^2 + \|\buoff_k - \bbaru _k\|)\\
&\le \frac{1}{2}(1+\Rstab^2)\Mdyn\|\bxoff_k- \bbarx _k\|^2\le \Rstab^2 \Mdyn \|\Delbarxk\|^2,
\end{align}
where again use $\Rstab \ge 1$ above. 
\end{proof}
Solving the recursion from \eqref{eq:recur}, we have 
	\begin{align}
	\Delbarxk[k+1] = \eta \sum_{j=1}^{k} \Phicl{k+1,j+1} \rem_{k}+ \Phicl{k+1,1}\Delbarxk[1].
	\end{align}
	Set $\betastab := (1-\frac \eta \Lstab)$, so that $M := \frac{\eta}{\betastab^{-1}-1} = \Lstab$.  By assumption,   $\|\Phicl{k,j}\| \le \Bstab \betastab^{k-j}$, so using \Cref{claim:taylor_xhat} implies that, if  $\max_{j \in [k]}\|\Delbarxk[j]\| \le \Rdyn/2\Rstab$ for all $j \in [k]$, 
	\begin{align}
	\|\Delbarxk[k+1]\| \le \eta \sum_{j=1}^{k} \Bstab\Mdyn\Rstab^2 \betastab^{k-j} \|\Delbarxk[j]\|^2 + \Bstab \betastab^{k} \|\Delbarxk[1]\|. 
	\end{align}
	Appling \Cref{lem:key_rec_one} with $\alpha = 0$, $C_1 = \Bstab\Mdyn\Rstab^2$, and $C_2 = \Bstab \ge 1$ and $M = \Lstab$ (noting $\betastab \ge 1/2$), it holds that for $\|\Delbarxk[1]\| = \epsilon_1 \le 1/4MC_1C_{3} = 1/4\Lstab \Mdyn\Rstab^2\Bstab^2$,
	\begin{align}
	\|\Delbarxk[k+1]\| \le 2\Bstab\|\Delbarxk[1]\|  (1 - \frac \eta \Lstab)^{k}. \label{eq:xhat_rec}
	\end{align}
	To ensure the inductive hypothesis that $\max_{j \in [k]} \|\Delbarxk[j]\| \le \Rdyn\Rstab$, it suffices to ensure that $2\Bstab\|\Delbarxk[1]\ \le \Rdyn/2\Rstab$, which is assumed by the lemma. Thus, we have shown that, if 
	\begin{align}
	\|\Delbarxk[1]\| \le \min\left\{\frac{\Rdyn}{2\Rstab\Bstab},\,\frac{1}{8\Lstab \Mdyn\Rstab^2\Bstab^2}\right\},
	\end{align}
	it holds that $\|\Delbarxk[k+1]\| \le 2\Bstab\|\Delbarxk[1]\|  (1 - \frac \eta \Lstab)^{k} \le R_0$ for all $k$.

	Next, we adress the stability of the gains for the perturbed trajectory $\trajoff$. Using $(\Rdyn,\Ldyn,\Mdyn)$-regularity of $\ctrajbar  $ and \eqref{eq:close_Rstab_within},
	\begin{align}
	&\left\|\bA_k(\trajoff) + \bB_k(\trajoff)\bK_k - \bA_k( \ctrajbar  ) + \bB_k( \ctrajbar  )\bK_k\right\|\\
	&= \left\|\begin{bmatrix} 
	\bA_k(\trajoff) - \bA_k(\ctrajbar  ) & \hat \bB_k(\trajoff) - \bB_k(\ctrajbar  )
		\end{bmatrix} \begin{bmatrix} \eye \\ \bK_k \end{bmatrix} \right\|\\
		&= \left\|(\nabla \feta(\xhat_k,\buoff_k) - \nabla \feta(\bbarx _k,\bbaru _k))\begin{bmatrix} \eye \\ \bK_k \end{bmatrix} \right\|\\
		&\le \Mdyn \left\|(\bxoff_k - \bbarx _k, \bK_k(\bxoff_k - \bbarx _k)\right\|\left\|\begin{bmatrix} \eye \\ \bK_k \end{bmatrix} \right\|\\
		&= \Mdyn\|\bxoff_k - \bbarx _k\|\left\|\begin{bmatrix} \eye \\ \bK_k \end{bmatrix} \right\|^2 \le \Mdyn\|\bxoff_k - \bbarx _k\|(1+\|\bK_k\|_{\op}^2)\\
		&= \Mdyn\|\bxoff_k - \bbarx _k\|\left\|\begin{bmatrix} \eye \\ \bK_k \end{bmatrix} \right\|^2 \le \Mdyn\|\bxoff_k - \bbarx _k\|(1+\|\bK_k\|_{\op}^2)\\
		&\le 2\Rstab^2\Mdyn\|\bxoff_k - \bbarx _k\|\\
		&\le 4\Bstab\Rstab^2\Mdyn\|\bxoff_1 - \bbarx _1\|\betastab^{k-1}, \quad \betastab = (1 - \frac \eta \Lstab).
	\end{align}
	Invoking \Cref{lem:mat_prod_pert} with $\betastab \ge 1/2$, 
	$\|\Phiclhat{k,j}\| \le 2\Bstab\betastab^{k-j}$ for all $j,k$ provided that $4\Bstab\Rstab^2\Mdyn\|\bxoff_1 - \bbarx _1\| \le 1/4\Bstab\Lstab$, which requires $\|\bxoff_1 - \bbarx _1\| \le 1/16\Bstab^2\Rstab^2\Lstab\Mdyn$. 

	The last part of the lemma uses $(\Rdyn,\Ldyn,\Mdyn)$-regularity of $\ctrajbar  $ and \eqref{eq:close_Rstab_within}.

\subsection{Ricatti synthesis of stabilizing gains.  }\label{sec:ric_synth}
In this section, we show that under a certain \emph{stabilizability} condition, it is always possible to synthesize primitive controllers satisfying Jacobian stability, \Cref{defn:Jac_stab},  with reasonable constants. We begin by defining our notion of stabilizability; we adopt the formulation based on Jacobian linearizations of non-linear systems 
 the discrete analogue of the senses proposed in 
which is consistent with \cite{pfrommer2023power,westenbroek2021stability}.
\begin{definition}[Stabilizability]\label{defn:stabilizable} A control trajectory $\ctraj = (\bx_{1:K+1},\bu_{1:K}) \in \scrP_{K}$ is $\Lfp$-Jacobian-Stabilizable if $\max_{k}\cV_{k}(\ctraj) \le \Lfp$, where for $k \in [K+1]$, $\cV_k(\ctraj)$ is defined by
\begin{align}
\cV_{k}(\ctraj) &:= \sup_{\xi:\|\xi \le 1}\left(\inf_{\tilde{\bu}_{1:s}} \|\tilde{\bx}_{K+1}\|^2 + \step \sum_{j=k}^{K} \|\tilde{\bx}_{j}\|^2 + \|\tilde{\bu}_{j}\|^2 \right)\\
&\text{s.t. } \tilde{\bx}_k = \xi, \quad \tilde{\bx}_{j+1} = \tilde{\bx}_j + \eta\left(\bA_j(\ctraj)\tilde{\bx}_j + \bB_j(\ctraj)\tilde{\bu}_j\right),
\end{align}
\end{definition}
Here, for simplicity, we use Euclidean-norm costs, though any Mahalanobis-norm cost induced by a positive definite matrix would suffice. We propose to synthesize gain matrices by performing a standard Ricatti update, normalized appropriately to take account of the step size $\eta > 0$ (see, e.g. Appendix F in \cite{pfrommer2023power}).
\begin{definition}[Ricatti update]\label{defn:ric_update} Given a path $\ctraj \in \Path_k$ with $\matA_k = \matA_k(\ctraj)$, $\matB_k = \matB_k(\ctraj)$ we define
\begin{align}
&\Pric_{K+1}(\ctraj) = \eye, \quad \Pric_{k}(\ctraj) = (\eye + \step\Aclk(\ctraj))^\top\Pric_{k+1}(\ctraj)(\eye + \step\Aclk(\ctraj)) + \step (\eye + \matK_k(\ctraj)\matK_k(\ctraj)^\top )\\
&\Kric_k(\ctraj) = (\eye + \step \matB_k^\top \Pric_{k+1}(\ctraj)\matB_k )^{-1}(\matB_k^\top \matP_{k+1}(\ctraj))(\eye + \eta \matA_k)\\
&\Aclkric(\ctraj) = \matA_k + \matB_k\matK_k(\ctraj).
\end{align}
\end{definition}
The main result of this section is that the parameters $(\Rstab,\Bstab,\Lstab)$ in \Cref{defn:Jac_stab} can be bounded in terms of $\Ldyn$ in \Cref{defn:control_path_regular}, and the bound $\Lfp$ defined above. 
\begin{proposition}[Instantiating the Lyapunov Lemma]\label{lem:instantiate_lyap} Let $\Ldyn,\Lfp \ge 1$, and let $\ctraj = (\bx_{1:K+1},\bu_{1:K})$ be $(\Rdyn,\Ldyn,\Mdyn)$-regular and $\Lfp$-Jacobian Stabilizable. Suppose further that $\step \le 1/5\Lf^2\Lfp$. Then, $(\ctraj,\Kric_{1:K})$-is $(\Rstab,\Bstab,\Lstab)$-Jacobian Stable, where 
\begin{align}
\Rstab = \frac{4}{3}\Lfp\Lf, \quad \Bstab = \sqrt{5}\Lf\Lfp, \quad \Lstab = 2\Lfp
\end{align}
\end{proposition}
\Cref{lem:instantiate_lyap} is proven in \Cref{sec:lem:instantiate_lyap} below. A consequence of the above proposition is that, given access to a smooth local model of dynamics, one can implement the synthesis oracle by computing linearizations around demonstrated trajectories, and solving the corresponding Ricatti equations as per the above discussions to synthesize the correct gains.

%\begin{corollary}\mscomment{explain synth oracle}
%\end{corollary}

\subsubsection{Proof of \Cref{lem:instantiate_lyap} (Ricatti synthesis of gains)}\label{sec:lem:instantiate_lyap}

	Throughout, we use the shorthand $\bA_k = \bA_k(\ctraj)$ and $\bB_k = \bB_k(\ctraj)$, recall that $\|\cdot\| $ denotes the operator norm when applied to matrices. We also recall our assumptions that $\Ldyn,\Lfp \ge 1$.
	We begin by translating our stabilizability assumption (\Cref{defn:stabilizable}) into the the $\bP$-matrices in \Cref{defn:ric_update}. The following statement recalls Lemma F.1 in \cite{pfrommer2023power}, an instantiation of well-known solutions to linear-quadratic dynamic programming (e.g. \cite{anderson2007optimal}).
	\begin{lemma}[Equivalence of stabilizability and Ricatti matrices]\label{lem:V_P_equiv} Consider a trajectory $(\bx_{1:K},\bu_{1:K})$, and define the parameter $\matTheta := (\Ajac(\bbarx _k,\bbaru _k),\Bjac(\bbarx _k,\bbaru _k))_{k \in [K]}$. Then, for all $k \in [K]$,
	\begin{align}
	\forall k \in [K], \quad \cV_{k}(\ctraj) = \|\matP_k(\matTheta)\|_{\op}
	\end{align}
	Hence, if $\ctraj$ is $\Lfp$-stabilizable, 
	\begin{align}
	\max_{k \in [K+1]}\|\matP_k(\matTheta)\|_{\op} \le \Lfp.
	\end{align}
	\end{lemma}

	\begin{lemma}[Lyapunov Lemma, Lemma F.10 in \cite{pfrommer2023power}]\label{lem:lyap_lem} Let $\matX_{1:K},\matY_{1:K}$ be matrices of appropriate dimension, and let $Q \succeq \eye$ and $\matY_k \succeq 0$. Define $\matLam_{1:K+1}$ as the solution of the recursion
	\begin{align}
	\matLam_{K+1} = \matQ, \quad \matLam_{k} = \matX_k^\top \matLam_{k+1} \matX_k + \step \matQ + \matY_k
	\end{align}
	Define the operator $\matPhi_{j+1,k} = \matX_j \cdot \matX_{j-1},\dots \cdot \matX_k$, with the convention $\matPhi_{k,k} = \eye$. Then, if $\max_{k}\|\eye - \matX_k\|_{\op} \le \kappa \step$ for some $\kappa \le 1/2\step$,
	\begin{align}
	\|\matPhi_{j,k}\|^2 \le \max\{1,2\kappa\}\max_{k \in [K+1]}\|\matLam_{k}\|(1 - \step \alpha)^{j-k}, \quad \alpha := \frac{1}{\max_{k \in [K+1]}\|\matLam_{1:K+1}\|}.
	\end{align}
	\end{lemma}

	\begin{claim}\label{claim:par:bounds_regular} If $\ctraj$ is $(0,\Lf,\infty)$-regular, then for all $k$, $\bA_k = \bA_k(\ctraj)$ and $\bB_k = \bB_k(\ctraj)$ satisfy $\max_{k\in [K]}\max\{\|\bA_k\|,\|\bB_k\|\} \le \Lf$. 
	\end{claim}
	\begin{proof} For any $k \in [K]$,
	\begin{align}
	\max\{\|\bA_k\|,\|\bB_k\|\} = \max\left\{\left\|\ddx f(\bbarx _k,\bbaru _k)\right\|,\left\|\ddu f(\bbarx _k,\bbaru _k)\right\|\right\} \le \left\|\nabla f(\bbarx _k,\bbaru _k)\right\| \le \Lf,
	\end{align}
	where the last inequality follows by regularity.
	\end{proof}
	\begin{claim}\label{claim:K_bound}Recall $\Kric_k(\ctraj) = (\eye + \step \matB_k^\top \Pric_{k+1}(\ctraj)\matB_k )^{-1}(\matB_k^\top \Pric_{k+1}(\ctraj))(\eye + \eta \matA_k)$. Then, if $\ctraj$ is $\Lfp$-stabilizable and $(0,\Lf,\infty)$-regular, and if $\eta \le 1/3\Lf$,
	\begin{align}
	\|\Kric_k(\ctraj)\| \le \frac{4}{3}\Lfp\Lf
	\end{align}
	\end{claim}
	\begin{proof} We bound
	\begin{align}
	\|\Kric_k(\ctraj)\| &\le \|\matB_k\|\|\Pric_{k+1}(\ctraj)\|(1+\eta\|\matA_k\|) \\
	&\le \Lf(1+\eta \Lf)\|\Pric_{k+1}(\ctraj)\| \tag{\Cref{claim:par:bounds_regular}}\\
	&\le \Lfp\Lf(1+\eta \Lf)\tag{\Cref{lem:V_P_equiv}, $\Lfp \ge 1$} \\
	&\le \frac{4}{3}\Lfp\Lf \tag{$\eta \le 1/3\Lf$.}
	\end{align} 
	\end{proof}
	 
	\begin{proof}[Proof of \Cref{lem:instantiate_lyap}] We want to show that $\Kric_{1:K}(\ctraj)$ is $(\Rstab,\Bstab,\Lstab)$-stabilizing.\Cref{claim:K_bound}  has already established that $\max_{k \in [K]}\|\Kric_k(\ctraj)\| \le \Rstab = \frac{4}{3}\Lfp\Lf$. 

	To prove the other conditions, we apply \Cref{lem:lyap_lem} with  $\bY_k = \bK_k(\matTheta)\bK_k(\matTheta)$, $\bQ = \eye$, and $\matX_k = \eye + \eta \Aclk(\matTheta)$. From \Cref{defn:ric_update}, let have that the term $\matLam_k$ in \Cref{lem:lyap_lem} is precise equal to $\matP_k(\matTheta)$. From \Cref{lem:V_P_equiv}, 
	\begin{align}
	\max_{k \in [K+1]}\|\matP_k(\matTheta)\|_{\op} = \max_{k \in [K+1]}\cV_k(\ctraj) \le \Lfp.
	\end{align}
	This implies that if $\max_{k}\|\matX_k - \eye\| \le \kappa \eta \le 1/2$, we have
	\begin{align}
	\|\Phicl{j,k}(\matTheta)\|^2 = 
	\|(\matX_j \cdot \matX_{j-1}\cdot \dots \matX_k)\| \le \max\{1,2\kappa\}\Lfp\left(1 - \frac{\step}{\Lfp}\right)^{j-k}.
	\end{align}
	It suffices to find an appropriate upper bound $\kappa$. We have
	\begin{align}
	\|\matX_k - \eye\| = \|\eta \Aclk(\matTheta)\| &\le \eta (\|\matA_k\| + \|\matB_k\|\|\matK_k(\matTheta)\|)\\
	&\le \eta \Lf(1 +\|\matK_k(\matTheta)\|)\\
	&\le \eta \Lf(1 + \frac{4}{3}\Lf\Lfp)\tag{\Cref{claim:K_bound}} \\
	&\le \frac{7}{3}\eta \Lf^2\Lfp \tag{$\Lfp,\Lf \ge 1$}
	\end{align}
	Setting $\kappa = \frac{7}{3}\Lf^2\Lfp.$, we have that as $\eta \le  \frac{1}{5\Lf^2\Lfp} \le \min\{\frac{3}{14 \Lf^2 \Lfp},\frac{1}{3\Lf}\}$ (recall $\Lf,\Lfp \ge 1$),
	we can bound 
	\begin{align}
	\max\{1,2\kappa\} \le \max\left\{1,\frac{14}{3}\Lf^2\Lfp\right\} \le \max\left\{1,5\Lf^2\Lfp\right\} = 5\Lf^2 \Lfp^2,
	\end{align}
	where again recall $\Lfp,\Lf \ge 1$.
	In sum, for $\eta \le  \frac{1}{5\Lf^2\Lfp}$, we have 
	\begin{align}
	\|\Phicl{j,k}\|^2 \le 5\Lf^2\Lfp^2\left(1 - \frac{\step}{\Lfp}\right)^{j-k}.
	\end{align}
	Hence,  using the elementary inequality $\sqrt{1 - a} \le (1 - a/2)$, 
	\begin{align}
	\|\Phicl{j,k}\| \le \sqrt{5}\Lf\Lfp\left(1 - \frac{\step}{\Lfp}\right)^{(j-k)/2} \le \sqrt{5}\Lf\Lfp\left(1 - \frac{\step}{2\Lfp}\right)^{j-k},
	\end{align}
	which shows that we can select $\Bstab = \sqrt{5}\Lf\Lfp$ and $\Lstab = 2\Lfp$.
	\end{proof}



\subsection{Solutions to recursions}\label{sec:recursion_solutions}
	\newcommand{\bPhi}{\bm{\Phi}}
This section contains the solutions to various recursions. 


	\begin{lemma}[First Key Recursion]\label{lem:key_rec_one} Let $C_1 > 0, C_2 \ge 1/2$, $\betastab \in (0,1)$, and suppose $\epsilon_1,\epsilon_2,\dots$ is a sequence satisfying $\epsilon_1 \le \bar \epsilon_1$, and 
	\begin{align}
	\epsilon_{k+1} \le C_2 \betastab^k \bar\epsilon_1 + C_1 \eta\sum_{j=1}^k \betastab^{k-j}\epsilon_j^2
	\end{align}
	Then, as long as $C_1 \le \beta(1-\beta)/2\eta$, it holds that $\epsilon_k \le 2C_2 \betastab^{k-1} \bar \epsilon_1$ for all $k$.
	\end{lemma}
	\begin{proof} Consider the sequence $\nu_k = 2C_2 \betastab^{k-1} \bar \epsilon_1 $. Since $C_2 \ge 1/2$, we have $\nu_1 \ge \bar \epsilon_1 \ge \epsilon_1$. Moreover, 
	\begin{align}
	C_2 \betastab^k \bar{\epsilon}_1 + C_1 \sum_{j=1}^k \betastab^{k-j}\nu_j &= C_2 \betastab^k \bar{\epsilon}_1 + 2C_1 C_2 \sum_{j=1}^k \betastab^{k+j-2} \bar \epsilon_1\\
	&= C_2 \betastab^k \bar{\epsilon}_1 \left(1 + \frac{2C_1}{\beta} \sum_{j=0}^{k-1} \betastab^{j}\right)\\
	&\le C_2 \betastab^k \bar{\epsilon}_1 \left(1 + \frac{2C_1\eta}{\beta(1-\beta)}\right)
	\end{align}
	Hence, for $C_1 \le \beta(1-\beta)/2\eta$, we have $C_2 \betastab^k \bar{\epsilon}_1 + C_1 \sum_{j=1}^k \betastab^{k-j}\nu_j \le 2C_2 \bar \epsilon_1 \betastab^k \le \nu_{k+1}$. This shows that the $(\nu_k)$ sequence dominates the $(\epsilon_k)$ sequence, as needed.
	\end{proof}
	\begin{lemma}[Second Key Recursion]\label{lem:key_rec_two} Let $c,\Delta,\eta > 0$, $\betastab \in (0,1)$ and let $\epsilon_1,\epsilon_2,\dots$ satisfy $\epsilon_1 \le c$ and 
	\begin{align}
	\epsilon_{k+1} \le c \betastab^k + c \eta  \Delta \betastab^{k-1}\sum_{j=1}^k \epsilon_j. 
	\end{align}
	Then, if $\Delta \le \frac{\beta(1-\beta)}{2c\eta}$, $\epsilon_{k+1} \le 2c\betastab^k$ for all $k$.
	\end{lemma}
	\begin{proof} Consider the sequence $\nu_k = 2c\betastab^{k-1}$. Since $\epsilon_1 \le c$, $\nu_1 \ge \epsilon_1$. Moreover, 
	\begin{align}
	c \betastab^k + c \eta  \Delta \betastab^{k-1}\sum_{j=1}^k \nu_j &\le c \betastab^k + 2c^2 \eta  \Delta \betastab^{k-1}  \sum_{j=1}^k \betastab^{j-1}\\
	&\le c \betastab^k + 2c^2 \eta  \Delta \betastab^{k-1}  \frac{1}{1-\beta}\\
	&\le c \betastab^k \left(1+2c\Delta   \frac{\eta}{\beta(1-\beta)}\right).
	\end{align}
	Hence, for $\Delta \le \frac{\beta(1-\beta)}{2c\eta}$, the above is at most $2c\betastab^k \le \nu_{k+1}$. This shows that the $(\nu_k)$ sequence dominates the $(\epsilon_k)$ sequence, as needed.
	\end{proof}


\begin{lemma}[Third Key Recursion]\label{lem:third_recursion}
Let $\eta > 0$, $\beta = (1 - \frac{\eta}{L})$, $L \ge 2\eta$, and let $C,C',\Delta_2,\Delta_1 > 0$. Suppose that $\epsilon_{1},\epsilon_2,\dots$ satisfies,
\begin{align}
\epsilon_{t+1} \le \eta \sum_{s=1}^t \beta^{t-s} (\Delta_2 +  C\epsilon_s^2)) + \beta^t\Delta_1
\end{align}
whenever $\max_{1 \le s \le t} \epsilon_t \le C'$. Suppose that 
\begin{align}
\Delta_2 \le \frac{1}{\max\{8CL, 4LC'\}} , \quad \Delta_1  \le \frac{1}{\max\{16 CL, 8C'\}}
 \end{align}
Then, for all $t$, 
\begin{align}
\epsilon_{t} \le 4\Delta_1 \beta^{t-1} + 2L\Delta_2.
\end{align}
\end{lemma}
\begin{proof} Consider $\epsbar_t = \alpha_1 \beta^{t-1} + \alpha_2$, with $\alpha_1 \ge \Delta_1$ and $\alpha_2 > 0$. As long as $\alpha_1 + \alpha_2 \le C'$, we have $\epsbar_t \le C'$ for all $t$. To show $\epsbar_t \ge \epsilon_t$, it suffices that $\epsbar_t \ge \eta \sum_{s=1}^t \beta^{t-s} (\Delta_2 +  C\epsbar_s^2)) + \beta^t\Delta_1$. To have this occur, we need
\begin{align}
&\eta \sum_{s=1}^t \beta^{t-s} (\Delta_2 +  C\epsbar_s^2)) + \beta^t \Delta_1\\
&\quad \le \eta \sum_{s=1}^t \beta^{t-s} (\Delta_2 +  2C\alpha_1^2\beta^{2(s-1)} + 2C\alpha_2^2) + \beta^t\Delta_1\\
&\quad \le \eta \sum_{s=1}^t \beta^{t-s} (\Delta_2 +  2C\alpha_1^2\beta^{2(s-1)} + 2C\alpha_2^2) + \beta^t\Delta_1\\
&\quad = (\Delta_2 + 2C\alpha_2^2)\cdot(\eta \sum_{s=1}^t \beta^{t-s}) +   2C\alpha_1^2 \cdot(\eta \sum_{s=1}^t \beta^{t-s}\beta^{2(s-2)}) + \beta^t\Delta_1\\
&\quad = (\Delta_2 + 2C\alpha_2^2)\cdot(\eta \sum_{s=1}^t \beta^{t-s}) +   \beta^{t-1} (2C\alpha_1^2 \cdot(\eta \sum_{s=1}^t \beta^{s-1})+\Delta_1)\\
&\quad \le  L(\Delta_2 + 2C\alpha_2^2) +   \beta^{t-1} (2C\alpha_1^2 L+\Delta_1),
\end{align}
where the last step upper bounds the geometric series with $\eta = (1-\eta/L)$. Assuming $\eta \le L/2$, the above is at most
\begin{align}
L(\Delta_2 + 2C\alpha_2^2) +   2\beta^{t} (2C\alpha_1^2 L+\Delta_1).
\end{align}
Matching terms, it is enough that 
\begin{align}
\alpha_2 \ge L(\Delta_2 + 2C\alpha_2^2), \quad \alpha_1 \ge 2(2C\alpha_1^2 L+\Delta_1), \quad \alpha_1 + \alpha_2 \le C'
\end{align}
Let's choose $\alpha_2 = 2L\Delta_2$ and $\alpha_1 = 4 \Delta_1$.  Then, it is enough that 
\begin{align}
L\Delta_2  \ge 8CL^2\Delta_2^2, \quad 2\Delta_1 \ge  32 C\Delta_1^2 L, \quad (2L\Delta_2 + 4\Delta_1) \le \frac{1}{C'}
\end{align}
For this, it suffices that $\Delta_2 \le \frac{1}{\max\{8CL, 4LC'\}}$ and $\Delta_1  \le \frac{1}{\max\{16 CL, 8C'\}}$. 
\end{proof}

	\begin{lemma}[Matrix Product Perturbation]\label{lem:mat_prod_pert} Define matrix products 
	\begin{align}\bPhi_{k,j} = \bX_{k-1} \cdot \bX_{k-2} \cdots \bX_j,\quad \bPhi_{k,j}' = \bX_{k-1}' \cdot \bX_{k-2}' \cdots \bX_j'.
	\end{align} 
	Let $\eta,\Delta,c > 0$ and $\betastab \in (0,1)$. If (a) $\bPhi_{k,j} \le \betastab^{k-j}$ for all $j \le k$, (b) $\|\bX_j - \bX_j'\| \le \eta \Delta \betastab^{j-1}$ for all $j \ge 1$  and (c) $\Delta \le \frac{\beta(1-\beta)}{2c\eta}$, then,  for all $j \le k$, $\|\bPhi_{k,j}'\| \le 2c\betastab^{k-j}$. 
	\end{lemma}
	\begin{proof} Without loss of generally, take $j = 1$. Then, letting $\bDelta_k = (\bX_k' - \bX_k)$,
	\begin{align}
	\bPhi_{k+1,1}' &= \bX_{k}' \cdot \bX_{k-2}' \cdots \bX_1'\\
	&= \bX_{k}' \cdot \bPhi_{k,1}'\\
	&= \bDelta_k\bPhi_{k,1}' + \bX_{k} \bPhi_{k,1}'\\
	&= \bDelta_k\bPhi_{k,1}' + \bX_{k} \bDelta_{k-1} \bPhi_{k-2,1}' + \bX_{k}\bX_{k-1}\bPhi_{k-2,1}'\\
	&= \bPhi_{k+1,k+1}\bDelta_k\bPhi_{k,1}' + \bPhi_{k+1,k} \bDelta_{k-1} \bPhi_{k-2,1}' + \bPhi_{k+1,k}\bPhi_{k-2,1}'\\
	&= \sum_{j=1}^{k}\bPhi_{k+1,j+1}\bDelta_{j}\bPhi_{j,1}' + \bPhi_{k+1,1}.
	\end{align}
	Thus, 
	\begin{align}
	\|\bPhi_{k+1,1}'\|_{\op} &\le c \eta \sum_{j=1}^{k}\betastab^{k-j}\|\bX_j-\bX_j'\|\|\bPhi_{j,1}'\| + c \betastab^k\\
	&\le c \eta \betastab^{k-1}\Delta\sum_{j=1}^{k}\|\bPhi_{j,1}'\| + c \betastab^k \tag{$\|\bX_j - \bX_j'\| \le \eta \Delta \betastab^{j-1}$}.
	\end{align}
	Define $\epsilon_j = \|\bPhi_{j,1}'\|$. Then, $\epsilon_1 = 1 \le c$, so \Cref{lem:key_rec_two} implies that for $\Delta \le \frac{(1-\beta)\beta}{2\eta}$, $ \|\bPhi_{k,1}'\| := \epsilon_k  \le 2c\betastab^k$ for all $k$.
	\end{proof}
\begin{comment}


\subsection{Proof of \Cref{prop:stabilizing}}

	 Introduce 
	 \begin{align}
	 \Delxk = \tilde{\bx}_{k} - \bbarx _{k},\quad \Deluk = \tilde{\bu}_k - \bbaru _k,
	 \end{align} and the shorthand $\bA_k= \bA_k(\ctraj)$ and $\bB_k= \bB_k(\ctraj)$. Note that $\tilde{\bu}_k - \bbaru _k$ does not include the feedback term. 
	 We expand
	\begin{align}
	\Delxk[k+1] &= \Delxk + \eta \left(f(\bbarx _k + \Delxk,\bbaru _k + \Deluk + \matKtil_k \Delxk) - f(\bbarx _k,\bbaru _k) \right)\\
	&= \Delxk + \eta \left(f(\bbarx _k + \Delxk,\bbaru _k + \Deluk + \matK_k \Delxk ) - f(\bbarx _k,\bbaru _k) \right) \\
	&\quad + \eta (\rem_{k,1})\\
	&= \Delxk + \eta \left(\underbrace{\ddx f(\bbarx _k,\bbaru _k)}_{=\bA_k} \Delxk + \underbrace{\ddu f(\bbarx _k,\bbaru _k)}_{=\bB_k} (\Deluk + \matK_k \Delxk)\right) \\
	&\qquad + \eta (\rem_{k,1} + \rem_{k,2})\\
	&= \Delxk + \eta \left(\Aclk\Delxk + \matB_k \Deluk \right) + \eta (\rem_{k,1} + \rem_{k,2}).
	\end{align}
	where, above
	\begin{align}
	\rem_{k,1} &= f(\bbarx _k + \Delxk,\bbaru _k + \Deluk + \matKtil_k \Delxk ) - f(\bbarx _k + \Delxk,\bbaru _k + \Deluk + \matK_k \Delxk )\\
	\rem_{k,2} &= f(\bbarx _k + \Delxk,\bbaru _k + \Deluk + \matK_k \Delxk ) -  f(\bbarx _k ,\bbaru _k)\\
	&\qquad- \ddx f(\bbarx _k,\bbaru _k) \Delxk + \ddu f(\bbarx _k,\bbaru _k) (\Deluk + \matK_k \Delxk) 
	\end{align}
	Solving the recursion, we have 
	\begin{align}
	\Delxk[k+1] = \sum_{j=1}^{k} \Phicl{k+1,j+1}(\bB_k \Deluk + \eta (\rem_{k,1}+\rem_{k,2})) + \Phicl{k+1,1}\Delxk[1].
	\end{align}
	We now bound the contribution of the remainder terms
	\begin{claim}\label{claim:remainder_claim}  Suppose that 
	\begin{align}
	\|\Delxk\|(1+\max\{\Rstab,\Rstabtil\}) + \|\Deluk\|  \le \Rdyn \label{eq:claim_remainder_claim}
	\end{align}
	Then, 
	\begin{align}
	 \|\rem_{k,1}\| + \|\rem_{k,2}\| \le \Mf\|\Deluk\|^2 + \Mf(1+\Rstab^2)\|\Delxk\|^2 +\Lf\|\matKtil_k-\matK_k\|\|\Delxk\|. 
	\end{align}
	\end{claim}
	Using $\|\matB_k\| \le \Lf$, we have
	\begin{align}
	&\|\Delxk[k+1]\| \\
	&\le \Lf\sum_{j=1}^{k} \step\|\Phicl{k+1,j+1}\|_{\op} \|\Deluk\|  +  \Mf\sum_{j=1}^{k}\step\|\Phicl{k+1,j+1}\|_{\op}\|\Deluk\|^2  \\
	&\quad + \sum_{j=1}^{k}\step\|\Phicl{k+1,j+1}\|_{\op}(\Mf(1+\Rstab^2)\|\Delxk\|^2 +\Lf\|\matKtil_k-\matK_k\|_{\op}\|\Delxk\|) +\|\Phicl{k+1,1}\|_{\op}\Delxk[1]\|.  \label{eq:Del_k_rec}
	\end{align}


	\begin{claim}[Key Recursion]\label{claim:main_recursion} Suppose \eqref{eq:claim_remainder_claim} holds for all $j \le k$, and that, in addition,
	\begin{align}
	& \max_{1 \le j \le k}\|\Delta_{\bx,j}\| \le \frac{1}{8\Lstab\Mf(1+\Rstab^2)}, \label{eq:recur_cond}\\
	&\|\Deluk[1:k]\|_{\ltwo} \le \frac{1}{\Mf},\quad \text{and} \quad \|\matKtil_{1:k}-\matK_{1:k}\|_{\ltwoop} \le \frac{1}{4\Lf\Lstab^{1/2}\Bstab} \label{eq:uKcond_thing}
	\end{align}
	Then,
	\begin{align}
	\|\Delxk[k+1]\| &\le  2\Bstab ((1+\Lf\Lstab^{1/2})\|\Deluk[1:k]\|_{\ltwo}    + \left(1-\frac{\step}{\Lstab}\right)^{k/2}\|\Delxk[1]\|) \label{eq:recur_conc}
	\end{align}
	\end{claim}
	We may now conclude. Recall our assumptions
	\begin{align}
	&\text{(a)}\quad(1+\Lf\Lstab^{1/2})\|\Deluk[1:K]\|_{\ltwo} +\|\Delxk[1]\|  \le \frac{1}{16\Bstab^2\Lstab\Mf(1+\Rstab^2)} \wedge \frac{1}{4\Bstab\Rdyn(1+\max\{\Rstab,\Rstabtil\})}\label{eq:u_cond}\\
	&\text{(b)}\quad\|\matKtil_{1:K}-\matK_{1:K}\|_{\ltwo} \le \frac{1}{4}(\Lf\Bstab\Lstab^{1/2})^{-1}.
	\end{align}
	Then, using $\Bstab,\Lstab \ge 1$, we see that \eqref{eq:uKcond_thing} is satisfied, and that check that \eqref{eq:recur_cond} and \eqref{eq:claim_remainder_claim} holds for $k=1$. Moreover, we can check that if \eqref{eq:recur_conc} holds us to some $k$, then \eqref{eq:recur_cond,eq:claim_remainder_claim} hold up to $k+1$.  Since \Cref{claim:main_recursion} in turn applies that \eqref{eq:recur_conc}, we conclude that, for all $k \in [K]$,
	\begin{align}
	\|\Delxk[k+1]\| &\le  2\Bstab ((1+\Lf\Lstab^{1/2})\|\Deluk[1:k]\|_{\ltwo}    + \left(1-\frac{\step}{\Lstab}\right)^{k/2}\|\Delxk[1]\|).
	\end{align}
	as needed.
	\qed


	\begin{proof}[Proof of \Cref{claim:remainder_claim}] Both follow from Taylor's theorem. We have
	\begin{align}
	&\|f(\bbarx _k + \Delxk,\bbaru _k + \Deluk + \matKtil_k \Delxk ) - f(\bbarx _k + \Delxk,\bbaru _k + \Deluk + \matK_k \Delxk )\|\\
	&= \|\int_{s=0}^1 \dds f(\bbarx _k + \Delxk,\bbaru _k + \Deluk + s(\matKtil_k-\matK_k)\Delxk +  \matK_k \Delxk ) \rmd s\|\\
	&= \max_{s \in [0,1]}\|\dds f(\bbarx _k + \Delxk,\bbaru _k + \Deluk + s(\matKtil_k-\matK_k)\Delxk +  \matK_k \Delxk )\|\\
	&= \max_{s \in [0,1]}\|\nabla_{(x,u)} f(\bbarx _k + \Delxk,\bbaru _k + \Deluk + s(\matKtil_k-\matK_k)\Delxk +  \matK_k \Delxk )\|\|\matKtil_k-\matK_k\|\\
	&\le \Lf\|\matKtil_k-\matK_k\|\|\Delxk\|. 
	\end{align}
	where the second-to-last inequality is by \mscomment{...}. For the second, define the curve $\bbarx_k(s) = \bbarx _k + s\Delxk$, $\bbaru_k(s) = \bbaru _k +  s(\Deluk + \matK_k \Delxk)$, and set $\bar{z}_k(s) = (\bar \bbarx _k(s),\bar \bbaru _k(s))$. Then,
	\begin{align}
	\|\rem_{k,2}\| &= \|f(\bar \bbarx _k(1),\bar \bbaru _k(1)) - f(\bar \bbarx _k(0),\bar \bbaru _k(0)) - \dds f(\bar \bbarx _k(s),\bar \bbaru _k(s)) \big{|}_{s= 0}\|\\
	&\le \frac{1}{2}\max_{s \in [0,1]} \|\frac{\rmd^2}{\rmd s^2} f(\bar \bbarx _k(s),\bar \bbaru _k(s))\|\\
	&\overset{(i)}{=} \frac{1}{2}\max_{s \in [0,1]} \left\|\nablatwo_{(x,u)} f(\bbarx_k(s),\bbaru_k(s))\left[\dds\bar{z}_k(s), \dds\bar{z}_k(s),:\right]\right\|\\
	&\overset{(ii)}{\le} \frac{1}{2}\Mf\|\dds \bar{z}_k(s)\|^2\\
	&= \frac{1}{2}\Mf\|(\Delxk, \Deluk + \matK_k\Delxk)\|^2\\
	&\le \Mf\|\Deluk\|^2 + \Mf(1+\|\matK_k\|^2)\|\Delxk\|^2\\
	&\le \Mf\|\Deluk\|^2 + \Mf(1+\Rstab^2)\|\Delxk\|^2
	\end{align}
	Summing these bounds concludes.
	\begin{align}
	&\|\rem_{k,1}\| + \|\rem_{k,2}\| \\
	&\quad \le \Mf\|\Deluk\|^2 + \Mf(1+\Rstab^2)\|\Delxk\|^2 +\Lf\|\matKtil_k-\matK_k\|\|\Delxk\|. 
	\end{align}
	\end{proof}

	\begin{proof}[Proof of \Cref{claim:main_recursion}] 
	Recall that, by assumption,
	\begin{align}
	\|\Phicl{k+1,j+1}\|_{\op} \le \Bstab\alpha^{k-j}, \quad \alpha = (1-\eta/\Lstab).
	\end{align}
	Thus, starting from \eqref{eq:Del_k_rec},
	\begin{align}
	\|\Delxk[k+1]\| &\le \Lf\sum_{j=1}^{k} \step\|\Phicl{k+1,j+1}\|_{\op} \|\Deluk[j]\|  +  \Mf\sum_{j=1}^{k}\step\|\Phicl{k+1,j+1}\|_{\op}\Deluk[j]\|^2  \\
	&\quad + \sum_{j=1}^{k}\step\|\Phicl{k+1,j+1}\|_{\op}(\Mf(1+\Rstab^2)\|\Delxk[j]\|^2 +\Lf\|\matKtil_j-\matK_j\|_{\op}\|\Delxk[j]\|) +\|\Phicl{k+1,1}\|\Delxk[1]\|\\
	&\le \Lf\Lstab\sum_{j=1}^{k} \step\alpha^{k-j} \|\Deluk[j]\|  +  \Bstab\Mf\sum_{j=1}^{k}\step\alpha^{k-j}\|\Deluk[j]\|^2  \\
	&\quad + \Bstab\sum_{j=1}^{k}\step\alpha^{k-j}(\Mf(1+\Rstab^2)\|\Delxk[j]\|^2 +\Lf\|\matKtil_j-\matK_j\|_{\op}\|\Delxk[j]\|)  + \Bstab\alpha^{k}\|\Delxk[1]\|\\
	%
	&\le \Lf\Bstab(\sum_{j=1}^{k} \step\alpha^{2(k-j)})^{1/2} (\sum_{j=1}^k\step \|\Deluk[j]\|^2)^{1/2}  +  \Bstab\Mf\sum_{j=1}^{k}\step\alpha^{k-j}\|\Deluk[j]\|^2  + \Bstab\alpha^{k}\|\Delxk[1]\|\\
	&\quad + \Bstab\sum_{j=1}^{k}\step\alpha^{k-j}(\Mf(1+\Rstab^2)\|\Delxk\|^2 +\Lf\|\matKtil_j-\matK_j\|_{\op}\|\Delxk\|) \\
	&\overset{(i)}{\le} \Bstab (\Lf\Lstab^{1/2}\|\Deluk[1:k]\|_{\ltwo}  +  \Mf\|\Deluk[1:k]\|_{\ltwo}^2  + \alpha^{k/2}\|\Delxk[1]\|)\\
	&\quad + \Bstab\sum_{j=1}^{k}\step\alpha^{k-j}(\Mf(1+\Rstab^2)\|\Delxk[j]\|^2 +\Lf\|\matKtil_j-\matK_j\|_{\op}\|\Delxk[j]\|) \\
	&\overset{(ii)}{\le} \Bstab ((1+\Lf\Lstab^{1/2})\|\Deluk[1:k]\|_{\ltwo}   + \alpha^{k/2}\|\Delxk[1]\|)\\
	&\quad + \Bstab\sum_{j=1}^{k}\step\alpha^{k-j}(\Mf(1+\Rstab^2)\|\Delxk[j]\|^2 +\Lf\|\matKtil_j-\matK_j\|_{\op}\|\Delxk[j]\|)  \label{eq:almost_last_recurse}
	\end{align}
	where in $(i)$ we upper bounded $\alpha^k \le \alpha^{k/2}$, $\alpha \le 1$, and $\sum_{j=1}^{k} \step\alpha^{2(k-j)} \le \sum_{j=1}^{k} \step\alpha^{k-j}\le \step \sum_{i \ge 0}\alpha^{i} = \Lstab$, and in $(ii)$, we used  the bound that $\|\Deluk[1:k]\|_{\ltwo} \le 1/\Mf$. Continuing,
	\begin{align}
	&\sum_{j=1}^{k}\step\alpha^{k-j}(\Mf(1+\Rstab^2)\|\Delxk[j]\|^2 +\Lf\|\matKtil_j-\matK_j\|_{\op}\|\Delxk\|) \\
	&\quad \le \Mf(1+\Rstab^2)\left(\sum_{j=1}^{k}\step\sqrt{\alpha}^{k-j}\right)\left(\max_{j \in [k]}\sqrt{\alpha}^{k-j} \|\Delxk[j]\|^2\right) \\
	&\qquad+\left(\max_{j \in [k]}\sqrt{\alpha}^{k-j}\|\Delxk[j]\|\right)  \Lf\left(\sum_{j=1}^{k}\step\sqrt{\alpha}^{k-j}\|\matKtil_j-\matK_j\|_{\op}\right)\\
	&\quad \le \Mf(1+\Rstab^2)\left(\sum_{j=1}^{k}\step\sqrt{\alpha}^{k-j}\right)\left(\max_{j \in [k]}\sqrt{\alpha}^{k-j} \|\Delxk[j]\|^2\right) \\
	&\qquad+\left(\max_{j \in [k]}\sqrt{\alpha}^{k-j}\|\Delxk[j]\|\right)\|\matKtil_{1:k}-\matK_{1:k}\|_{\ltwoop}  \Lf\left(\sum_{j=1}^{k}\step{\alpha}^{k-j}\right)^{1/2}\\
	&\quad\overset{(i)}{\le}  2\Lstab\Mf(1+\Rstab^2)\left(\max_{j \in [k]}\sqrt{\alpha}^{k-j} \|\Delxk[j]\|^2\right) + \Lf\Lstab^{1/2}\|\matKtil_{1:k}-\matK_{1:k}\|_{\ltwoop}\left(\max_{j \in [k]}\sqrt{\alpha}^{k-j}\|\Delxk[j]\|\right) \\
	&\quad\overset{(ii)}{\le}  \frac{1}{2\Bstab}\left(\max_{j \in [k]}\sqrt{\alpha}^{k-j} \|\Delxk[j]\|\right) \label{eq:x_term_bound}
	\end{align}
	where in $(i)$, we use $\left(\sum_{j=1}^{k}\step{\alpha}^{k-j}\right) \le \Lstab$ and $\sum_{j=1}^k\step \sqrt{\alpha}^{k-j} \le \sum_{i \ge 0}\step\sqrt{\alpha}^{i} \le 2\sum_{i \ge 0}\step\alpha^i \le \Lstab$; and in $(ii)$ we uses what is stipulated in \eqref{eq:recur_cond}:
	\begin{align}
	\|\matKtil_{1:k}-\matK_{1:k}\|_{\ltwoop} \le \frac{1}{4\Lf\Lstab^{1/2}\Bstab}, \quad \max_{1 \le j \le k}\|\Delta_{\bx,j}\| \le \frac{1}{8\Lstab\Mf(1+\Rstab^2)}.
	\end{align}
	Plugging in \eqref{eq:x_term_bound}  in \eqref{eq:almost_last_recurse} gives
	\begin{align}
	\|\Delxk[k+1]\| \le \Bstab ((1+\Lf\Lstab^{1/2})\|\Deluk[1:k]\|_{\ltwo}  + \sqrt{\alpha}^{k}\|\Delxk[1]\|) + \frac{1}{2}\max_{j \in [k]}\sqrt{\alpha}^{k-j} \|\Delxk[j]\|
	\end{align}
	By applying the same argument to $k' \le k$ (and using the the $\|\cdot\|_{\ltwo}$ and $\|\cdot\|_{\ltwoop}$ norm of a sequence is non-decreasing in length) gives
	\begin{align}
	\|\Delxk[k'+1]\| \le \Bstab ((\Lf\Lstab^{1/2}+1)\|\Deluk[1:k]\|_{\ltwo}  + \sqrt{\alpha}^{k'}\|\Delxk[1]\|) + \frac{1}{2}\max_{j \in [k']}\sqrt{\alpha}^{k'-j} \|\Delxk[j]\|
	\end{align}
	Multipling by $\sqrt{\alpha}^{k-k'} \le 1$ then gives 
	\begin{align}
	\sqrt{\alpha}^{k-k'}\|\Delxk[k'+1]\|& \le \sqrt{\alpha}^{k-k'}\Bstab ((\Lf\Lstab^{1/2}+1)\|\Deluk[1:k]\|_{\ltwo}  + \sqrt{\alpha}^{k}\|\Delxk[1]\|) + \frac{1}{2}\max_{j \in [k']}\sqrt{\alpha}^{k-j} \|\Delxk[j]\|\\
	& \le \Bstab ((\Lf\Lstab^{1/2}+1)\|\Deluk[1:k]\|_{\ltwo}  + \sqrt{\alpha}^{k}\|\Delxk[1]\|) + \frac{1}{2}\max_{j \in [k]}\sqrt{\alpha}^{k-j} \|\Delxk[j]\|
	\end{align}
	Thus, we have show that 
	\begin{align}
	\max_{k' \le k} \sqrt{\alpha}^{k-k'}\|\Delxk[k'+1]\| \le \Bstab ((\Lf\Lstab^{1/2}+1)\|\Deluk[1:k]\|_{\ltwo}  + \sqrt{\alpha}^{k}\|\Delxk[1]\|) + \frac{1}{2}\max_{j \in [k]}\sqrt{\alpha}^{k-j} \|\Delxk[j]\|. 
	\end{align}
	Rearranging gives
	\begin{align}
	&\Bstab ((\Lf\Lstab^{1/2}+1)\|\Deluk[1:k]\|_{\ltwo}  + \sqrt{\alpha}^{k}\|\Delxk[1]\|) \\
	&\ge \max_{k' \le k} \sqrt{\alpha}^{k-k'}\|\Delxk[k'+1]\| -  \frac{1}{2}\max_{j \in [k]}\sqrt{\alpha}^{k-j} \|\Delxk[j]\|\\
	&=\max_{j \in \{2,\dots,k+1\}} \sqrt{\alpha}^{k-j-1}\|\Delxk[j]\| -  \frac{1}{2}\max_{j \in [k]}\sqrt{\alpha}^{k-j} \|\Delxk[j]\|
	\end{align}
	As we also have $\Bstab \ge 1$, it also holds that $\Bstab (\Lf\Lstab^{1/2}\|\Deluk[1:k]\|_{\ltwo}  +  \Mf\|\Deluk[1:k]\|_{\ltwo}  + \sqrt{\alpha}^{k}\|\Delxk[1]\|) \ge \|\Delxk[1]\|$. Thus, in fact, we have
	\begin{align}
	&\Bstab ((\Lf\Lstab^{1/2}+1)\|\Deluk[1:k]\|_{\ltwo}  + \sqrt{\alpha}^{k}\|\Delxk[1]\|) \\
	&\ge \max_{j \in [k+1]} \sqrt{\alpha}^{k-j-1}\|\Delxk[j]\| -  \frac{1}{2}\max_{j \in [k]}\sqrt{\alpha}^{k-j} \|\Delxk[j]\|\\
	&\overset{(i)}{\ge} \max_{j \in [k+1]} \sqrt{\alpha}^{k-j-1}\|\Delxk[j]\| -  \frac{1}{2}\max_{j \in [k+1]}\sqrt{\alpha}^{k-j-1} \|\Delxk[j]\| \\
	&= \frac{1}{2}  \max_{j \in [k+1]}\sqrt{\alpha}^{k-j-1}\|\Delxk[j]\|\\
	&\ge \frac{1}{2}\|\Delxk[k+1]\|
	\end{align}
	where in $(i)$ we use $\alpha \le 1$. Multiplying both sides through by a factor of $2$ concludes.


	\end{proof}


\end{comment}
