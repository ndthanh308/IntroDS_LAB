%!TEX root = ../main.tex

\section{Proofs for  Generic Incrementally Stable Primitive Controllers}
\label{app:gen_controllers_proofs}

This section proves \Cref{thm:main_template_general,prop:TVC_main_general}, gneeralizing our guarantees to general primitive controllers. Note that, in this more general setting, we can no longer expect to bound the norm of the difference between two controllers evaluated at some point $\bx$ $\sfk_t(\bx)-\sfk_t(\bx')$ by differences in their parameter values. Instead, we opt for the more local notion of distance considered in \Cref{thm:main_template_general,prop:TVC_main_general}, via the localized distance $\dloc$ considered in \Cref{defn:dloc}. To this end, \Cref{app:gen:composite_mdp} begins  by generalizing the analysis of the composite MDP  to allow the distance $\distA$ take an additional state-argument (in order to capture the localization of the distance in $\dloc$). \Cref{app:ips_from_tis} then converts our assumption of incremental stability, \Cref{asm:tis}, into the IPS stability conditioned required in the composite MDP. Finally, we conclude of our intended results in \Cref{app:proof:general_controller}, following the same arguments as for affine primitive controllers in \Cref{app:end_to_end}. 



\subsection{Generalization of analysis in the composite MDP}\label{app:gen:composite_mdp}

Here, we consider a generalization of the analysis of the composite MDP where we allow $\distA$ to depend on state. Our analysis follows  \Cref{app:smoothcor_proof} and the proof of \Cref{thm:smooth_cor_decomp}. All notation here borrows from that section. Formally, we consider 
\begin{align}
\distAs(\cdot,\cdot \mid \cdot) : (\cA \times \cA) \times \cS \to \R_{\ge 0}.
\end{align}


We recall the direct decomposition in \Cref{defn:direct_decomp} of $\cS = \cZ \oplus \Scomp$, where we recall that $\cZ$ is the component that coincides with the `$\pathm$' component of the state in our instantiation. Further, recall that $\phimem$ is the projection onto the $\cZ$ component.
\begin{condition}[Measurability]
We require that $\distAs$ is measurable, and that, for all $\seqs$, the set $\{(\seqa',\seqa) : \distAs(\seqa',\seqa;\seqs) > \epsilon\}$ is open. and that $(\seqa',\seqa,\seqs)\mapsto \distAs(\seqa',\seqa;\seqs)$ is measurable. 
%Note that for this, it suffices that for all $\seqa$ and $\seqs$, the set $\{(\seqa',\seqa) : \distAs(\seqa',\seqa;\seqs) > \epsilon\}$ is open. 
We also assume that $\distAs(\seqa',\seqa;\seqs)$ only depends on $\seqs$ through $\phimem(\seqs)$. 
\end{condition}

We re-define a state-conditioned input stability as follows
\begin{definition}\label{defn:state_cond_stable} We say that a sequence $(\seqs_{1:H+1},\seqa_{1:H})$ is state-conditioned input-stable with respect to an auxilliary sequence $\tilde{\seqs}_{1:H+1}$ if 
\begin{align}
\dists(\seqs_{h+1}',\seqs_{h+1}) \vee \disttvc(\seqs_{h+1}',\seqs_{h+1}) \le  \max_{1 \le j \le h}\distAs\left(\seqa_{j}',\seqa_j \mid \tilde\seqs_j\right),~ \forall h \in [H]
\end{align}
\end{definition}

We now define a one-step error which is state dependent (allowing for $\distAs$). 
To simplify the exposition, we define  marginal gaps which ignore the now-state-dependent $\distAs$.
\newcommand{\drobs}[1][\epsilon]{\dist_{\mathrm{os},\cS,#1}}
\begin{definition}[Modified Imitation Gaps]\label{defn:mod_imit_gaps}
Define the state
\begin{align}
\drobs(\polhat_h(\seqs),\polst_h (\seqs)\mid \seqs') &:= \inf_{\coup_2}\Pr_{\coup_2}\left[\distAs(\seqahat_h,\seqast_h \mid \seqs') \le \epsilon \right],
\end{align}
where the infinum is over couplings $(\seqast_h, \hat \seqa_h) \sim \coup_2 \in \couple( \polhat_h(\seqs),\pol_h^\star(\seqs))$. Further define 
\begin{align}
\gapmargs := \inf_{\coup_1}\Pr_{\coup_1}\left[\max_{h \in [H]}\max\{\dists(\sstar_{h+1},\shat_{h+1}) > \epsilon\right], \quad \gapjoints := \max_{h \in [H]}\inf_{\coup_1}\Pr_{\coup_1}[\dists(\sstar_{h+1},\shat_{h+1}) > \epsilon]
\end{align}
where above $\coup_1$ ranges over the same couplings as in \Cref{defn:imit_gaps}.

\end{definition}


\paragraph{Guarantees under TVC of $\pihat$}. We now generalize \Cref{prop:IS_general_body} under the assumption that $\pihat$ is TVC. 

\begin{proposition}[Generalization of \Cref{prop:IS_general_body}]\label{prop:IS_general_body_state_cond}
Let $\polst$ be state-conditioned input-stable w.r.t. $(\dists,\distAs)$ and let $\polhat$ be $\gamma$-TVC. Then, for all $\epsilon > 0$, 
\begin{align}\gapjoints(\polhat \parallel \pist) \le  H\gamma(\epsilon) + \sum_{h=1}^H \Exp_{\sstar_h \sim \Psth}\drobs(\polhat_h(\sstar_h) \parallel \polst_h(\sstar_h)  \mid \sstar_h ).
\end{align}
\end{proposition}
\begin{proof}[Proof Sketch] The proof is nearly identical to the proof of \Cref{prop:IS_general_body} in \Cref{app:no_augmentation}. The only difference is that, when we measure the distance between $\astar_h \sim \pist_h(\sstar_h)$ and $\ainter_h \sim \pihat_h(\sstar_h)$, this distance is specified at $\sstar_h$. Hence, we replace $\distA(\ainter_h,\astar_h)$ with $\distA(\ainter_h,\astar_h \mid \sstar_h)$. This leads to use replacing $\drob(\polhat_h(\sstar_h) \parallel \polst(\sstar_h) ).$ with $\drobs(\polhat_h(\sstar_h) \parallel \polst_h(\sstar_h) \mid \sstar_h )$ in the final bound. 
\end{proof}



\paragraph{Guarantees with smoothing kernel.} Next, we turn to the generalization of \Cref{thm:smooth_cor,thm:smooth_cor_decomp} to allow for state-conditioned action distances. 
\begin{definition}\label{defn:sc_rips} Given non-decreasing maps $\gamipsone,\gamipstwo:\R_{\ge 0} \to  \R_{\ge 0}$ a  pseudometric $\distips:\cS \times \cS \to \R$ (possibly other than $\dists$ or $\disttvc$), and $\rips > 0$, we say a policy $\pi$ is \emph{$(\gamipsone,\gamipstwo,\distips,\rips)$-state-conditioned-restricted IPS} if it satisfies the conditions of \Cref{defn:ips_restricted}, with the only modification that for the constructed $\seqs_{1:H+1},\seqa_{1:H},\tilde{\seqs}_{1:H}$,  the condition that $\seqs_{1:H+1},\seqa_{1:H}$ is input-stable is replaced with ``state-conditioned input stable with respect to the sequence $\tilde \seqs_{1:H}$.'' More precisely, the condition is met if the following holds for any $r \in [0,\rips]$. Consider any sequence of kernels $\lawW_1,\dots,\lawW_H:\cS \to \laws(\cS)$ satisfying 
\begin{align}
\max_{h,\seqs \in \cS}\Pr_{\tilde \seqs\sim \lawW_h(\seqs)}[\distips(\tilde \seqs,\seqs) \le r] = 1, \quad \forall s, \quad \phimem \circ \lawW_h(\seqs_h) \ll \phimem \circ \Psth. 
\end{align}
 and define a process $\seqs_1 \sim \Dinit$, $\tilde\seqs_h \sim \lawW_h(\seqs_h),\seqa_h \sim \pi_h(\tilde \seqs_h)$, and $\seqs_{h+1} := F_h(\seqs_h,\seqa_h)$. Then, almost surely, 
 \begin{itemize} 
 	\item[(a)] the sequence $(\seqs_{1:H+1},\seqa_{1:H})$ is state-conditioned input stable  with respect to the sequence $\tilde \seqs_{1:H}  $
 	\item[(b)] $\max_{h \in [H]} \disttvc(F_h(\tilde\seqs_h,\seqa_h),\seqs_{h+1}) \le \gamipsone(r)$ and (c) $\max_{h \in [H]} \dists(F_h(\tilde\seqs_h,\seqa_h),\seqs_{h+1}) \le \gamipstwo(r)$.
 \end{itemize}
\end{definition}


\begin{theorem}\label{thm:state_cond_imit_general} Consider the setting of \Cref{thm:smooth_cor_decomp}, but with $\pist$ satisfies $(\gamipsone,\gamipstwo,\distips,\rips)$-state-conditioned-restricted IPS (\Cref{defn:sc_rips}) rather than (standard) restricted IPS (\Cref{defn:ips_restricted}). Again, let $\epsilon > 0$ and $r \in (0,\frac{1}{2}\rips]$, and efine 
\begin{align}
p_r := \sup_{\seqs}\Pr_{\seqs' \sim \Wsig(\seqs)}[\distips(\seqs',\seqs) >  r], \quad \epsilon' := \epsilon+\gamipstwo(2r)
\end{align} Then, for any policy $\pihat$,  both  $\gapjoints (\pihat \circ \Wsig \parallel \pistrep)$ and  $\gapmargs[\epsilon'] (\pihat \circ \Wsig \parallel \pist)$ are upper bounded by
\begin{align}
%\inf_{r > 0}  
H\left(2p_r +  3\gamma_{\sigma}(\max\{\epsilon,\gamipsone(2r)\})\right)  +  \sum_{h=1}^H\Exp_{\sstar_h \sim \Psth}\Exp_{\sstartil_h \sim \Wsig(\sstar_h) } \drobs( \pihat_{h}(\sstartil_h) \parallel \pidech(\sstartil_h) \mid \sstar_h)  . \label{eq:smooth_ub_general}
\end{align}
\end{theorem}
\begin{proof} The proof follows by modifying \Cref{thm:smooth_cor_general}, and hence \Cref{thm:smooth_cor_decomp} as a consequence. The key change is that we replace the event $\Bfsh = \left\{ \dista( \atelinter_h,\atel_h) > \epsilon \right\}$ \footnote{This is the special case of $\Bfsh = \left\{ \distavec( \atelinter_h,\atel_h) \not \preceq \epsvec \right\} $ with $\distavec$ being scalar valued (i.e. all coordinate identical).} in \Cref{defn:all_key_eents} with 
\begin{align}\Bfsh = \left\{ \dista( \atelinter_h,\atel_h \mid \ssq_h) > \epsilon \right\}, \label{eq:Bfsh}
\end{align}
and replace the event $\Qis$ in \Cref{defn:Qevents} with 
\begin{align}
\Qis :=\left\{\srep_{1:H+1},\arep_{1:H} \text{ is state-conditioned input-stable w.r.t. } \sreptil_{1:H}\right\}
\end{align}
We also define the following event
\begin{align}
\cQ_{\textsc{is},h}' :=\left\{\srep_{1:h+1},\arep_{1:h} \text{ is state-conditioned input-stable w.r.t. } \ssq_{1:h}\right\},
\end{align}
which considers input stability for $h \le H$ steps and shifts the reference sequence from $\sreptil_{1:h}$ to $\ssq_{1:h}$.  What changes as a result of these argument is as follows:
\begin{itemize}
    \item We check that \Cref{claim:stability_claim} goes through:
\begin{align}\Callbarh[h+1] \subset \Qall \cap \Callbarh \cap \Ballbarh.
\end{align}
    The first modification here is that, when $\srep_{1:H+1},\arep_{1:H}$ is input stable with respect to $\sreptil_{1:H}$,
    \begin{align}
    \dists(\shat_{h+1},\srep_{h+1}) \vee \disttvc(\shat_{h+1},\srep_{h+1}) \le  \max_{1 \le j \le h}\distAs\left(\ahat_{j},\arep_j \mid \sreptil_j\right),~ \forall h \in [H]
    \end{align}
    Since $\distAs(\seqa,\seqa' \mid \seqs)$ depends only on $\seqs$ through $\phimem(\seqs)$, then when
    \begin{align}
    \bigcap_{1\le j \le h}\Btelh[j] :=  \left\{ \arep_j = \atel_j, ~\phimem(\sreptil_j) = \phimem(\ssq_j) \right\}, \subset \Ballbarh,
    \end{align}
    holds, we have 
    \begin{align}
    \max_{1 \le j \le h}\distAs\left(\ahat_{j},\arep_j \mid \sreptil_j\right) = \max_{1 \le j \le h}\distAs\left(\ahat_{j},\atel_j \mid \ssq_j\right),~ \forall h \in [H]
    \end{align}
    Finally, on $\Ballbarh$, we have $\ahat = \atelinter$, so we get
    \begin{align}
    \max_{1 \le j \le h}\distAs\left(\ahat_{j},\arep_j \mid \sreptil_j\right) = \max_{1 \le j \le h}\distAs\left(\atelinter_{j},\atel_j \mid \ssq_j\right),~ \forall h \in [H],
    \end{align}
    which is at most $\epsilon$ under our second definition of $\Bfsh$.
    \item \Cref{lem:putting_couplings_together} goes unchanged
    \item We check that \Cref{lem:Qall_bound} goes through. This follows from the defintion of state-conditioned input-stable, using $\srep_{1:H+1},\arep_{1:H},\sreptil_{1:H}$ as  $\seqs_{1:H+1},\seqa_{1:H},\tilde{\seqs}_{1:H}$, so that when $\Pr_{\mu}[\Qis^c \cap \Qclose] = 0$.\footnote{Here, we replace $\pips$ in that lemma with failure probability $0$.}
    \item Recall that $\distAs(\seqa,\seqa';\seqs)$ depends only on $\seqs$ through $\phimem(\seqs)$. Hence, under the event 
    \begin{align}
    \bigcap_{1\le j \le h}\Btelh[j] :=  \left\{ \arep_j = \atel_j, ~\phimem(\sreptil_j) = \phimem(\ssq_j) \right\}, \subset \Ballbarh
    \end{align} 
    we have that  $\Qis$ implies $\cQ_{\textsc{is},h}'$. 
    \item Lemma \ref{lem:make_coupling} replaces $\drob( \pihat_{\sigma,h}(\ssq_h) \parallel \pireph(\ssq_h))$ with $\drobs( \pihat_{\sigma,h}(\ssq_h) \parallel \pireph(\ssq_h) \mid \ssq_h )$, i.e. the state-conditioned one-step error (that we condition on $\ssq_h$ comes from our re-definition of $\Bfsh$ in  \eqref{eq:Bfsh}. Thus, we get 
    $\gapjoints (\pihat \circ \Wsig \parallel \pistrep)$ and  $\gapmargs[\epsilon'] (\pihat \circ \Wsig \parallel \pist)$ are upper bounded by
	\begin{align}
%\inf_{r > 0}  
	H\left(2p_r +  3\gamma_{\sigma}(\max\{\epsilon,\gamipsone(2r)\})\right)  +  \sum_{h=1}^H\Exp_{\ssq_h} \drobs( \pihat_{\sigma,h}(\ssq_h) \parallel \pireph(\ssq_h) \mid \ssq_h)  . \label{eq:smooth_ub}
\end{align}
   \item Because $\ssq_h$ has marginal $\sstar_h \sim \Psth$, we can replace the terms $\Exp_{\ssq_h} \drobs( \pihat_{\sigma,h}(\ssq_h) \parallel \pireph(\ssq_h) \mid \ssq_h)$ with $\Exp_{\sstar_h \sim \Psth} \drobs( \pihat_{\sigma,h}(\sstar_h) \parallel \pireph(\sstar_h) \mid \sstar_h)$.
     \item Using the same data-processing argument as in the proof  as in \Cref{thm:smooth_cor_general}, we can bound
     \begin{align}
     \Exp_{\sstar_h \sim \Psth} \drobs( \pihat_{\sigma,h}(\sstar_h) \parallel \pireph(\sstar_h) \mid \sstar_h) \le \Exp_{\sstar_h \sim \Psth}\Exp_{\sstartil_h \sim \Wsig(\sstar_h)} \drobs( \pihat_{h}(\sstartil_h) \parallel \pidech(\sstartil_h) \mid \sstar_h).
     \end{align}
\end{itemize}
\end{proof}
\subsection{State-Conditioned Input-Stability and IPS in the Composite MDP via $\tiss$}\label{app:ips_from_tis}

\Cref{lem:ips_and_input_stable_not_state_conditioned_general} reduced IPS in the composite MDP to incremental stability in a form that applies primarily to affine primitive controllers.  In this section, we generalize  the lemma further to depend on a more localized distance reflecting the state-conditioned distance $\distAs(\cdot,\cdot;\cdot)$ in the composite MDP. 
Recall the local-distance between composite actions $\seqa = \sfk_{1:\tauc},\seqa'=\sfk'_{1:\tauc} \in \cA$ at state $\bx$ and scale $\alpha > 0$, defined in \Cref{defn:dloc} as
\begin{align}
\dloc(\seqa, \seqa' \mid \bx) := \max_{1 \le i \le \tauc} \sup_{\delx:\|\delx\| \le \alpha} \|\sfk_i(\bx_i+\delx)-\sfk_i'(\bx_i+\delx)\|, 
\end{align}
where above $\bx_{1} = \bx$, $\bx_{t+1}=f(\bx_t,\sfk_t(\bx))$ with $\seqa =\sfk_{1:\tauc}$.
\begin{lemma}\label{lem:ips_and_input_stable_state_conditioned} Instantiate the composite MDP as in \Cref{defn:composite_instant_general}, with $\pist$ as in \Cref{def:Dexp_policies}.  Furthermore, suppose that under $(\ctraj_T,\seqa_{1:H}) \sim \Dexp$ with $\ctraj_T = (\bx_{1:T+1},\bu_T)$, the following both hold with probability one:
\begin{itemize} 
    \item Each action $\seqa_h$ satisfies our notion of incremental stability (\Cref{defn:tiss}) with moduli $\upgamma(\cdot),\upbeta(\cdot,\cdot)$, constants $\cgamma,\cxi > 0$ (i.e. \Cref{asm:tis} holds)
\end{itemize}
Finally, let $\epsilon_0 > 0$ satisfy \eqref{eq:eps_cond_general}, that is:
    \begin{align}
    \gammaiss^{-1}(\betaiss(2\gammaiss(\epsilon_0),\tauc) \le \epsilon_0 \le \min\{\cgamma,\gammaiss^{-1}(\cxi/4)\}, \label{eq:eps_cond_control_app}
    \end{align}
    Further, given $\tilde\seqs \in \cS = \scrP_{\tauc}$ with last-step $\btilx_{\tauc}$,  consider the distance-like function
    \begin{align}
    \distAbar( \seqa,\seqa;\alpha,\tilde\seqs)  :=  \uppsi(\dloc(\seqa, \seqa' \mid \btilx_{\tauc}))\cdot \cI_{\infty}\left\{\dloc(\seqa, \seqa' \mid \btilx_{\tauc}) \le \epsilon_0\right\}, \quad  \uppsi(u) := 2\betaiss(2\gammaiss(u),0).
    \end{align}
     Then, the following hold:
    \begin{itemize}
        \item[(a)] $\pist$ is state-conditioned input-stable (\Cref{defn:state_cond_stable}) with respect to $\dists,\disttvc$ as defined in \Cref{sec:analysis} 
        \begin{align}
        \distA( \seqa,\seqa';\seqs) = \distAbar( \seqa,\seqa';\uppsi(\epsilon),\seqs), \quad
        \end{align}
        \item[(b)] For any $\rips \le \cxi/2$, $\pist$ is $(\rips,\gamipsone,\gamipstwo,\distips)$-state-conditioend restricted-IPS (\Cref{defn:sc_rips}) with
        \begin{align}
        \distA( \seqa,\seqa';\seqs) = \distAbar( \seqa,\seqa';\uppsi(\epsilon)+\upbeta(\rips,0),\seqs), \quad \gamipsone(r) = \betaiss(r,\tauc-\taum), \quad \gamipstwo(r) = \betaiss(r,0),
        \end{align}
        \end{itemize}
\end{lemma}


\begin{proof}[Proof of  \Cref{lem:ips_and_input_stable_state_conditioned}] The proof is nearly identical to that of  \Cref{lem:ips_and_input_stable_not_state_conditioned_general}, based on \Cref{lem:iss_ips}. The only difference is that, rather than using the worst-case bound $\bx_t \in \cX_0$, we condition on the relevant states. For part (a), we consider  $(\seqs_{1:H+1},\seqa_{1:H})$ be drawn from the distribution induces by $\pist$, and let $\seqa'_{1:H}$ be some other sequences of actions, and measure $\distAs(\seqa_h,\seqa_h;\seqs_h)$. Thus the relevant control-state to condition on is $\bx_{t_h}$ in the construction  of \Cref{lem:iss_ips}. For verifying (b), we instead condition on $\btilx_{t_h}$ because, as in \Cref{defn:sc_rips}, we measure the input-state stability condition for restricted-state-conditioned-IPS with sequence to the states $\tilde{\seqs}_{1:H}$.
\end{proof}


\subsection{Concluding the proof of \Cref{thm:main_template_general,prop:TVC_main_general}}\label{app:proof:general_controller}

\subsubsection{Proof of \Cref{prop:TVC_main_general}}
The result is a direct consequence of the following points. First, with our instantition of the composite MDP, we can bound  $\Imitmarg(\pihat) \le \gapmargs(\pihat \parallel \pist) \le \gapjoints(\pihat \parallel \pist)$ by the same argument in \Cref{lem:eq_loss_converstions}\footnote{Here, $\gapmargs,\gapjoints$ are defined in in \Cref{defn:mod_imit_gaps}. The only difference between these the standard gaps $\gapmarg,\gapjoint$ consider in \Cref{defn:mod_imit_gaps} is that they drop the closesness on composite actions, which is immaterial for $\Imitmarg(\pihat)$.}; by a similar argument, we have  $\Imitjoint(\pihat) \le \gapjoints(\pihat \parallel \pist)$ when $\Dexp$ has $\tau\le \taum$-bounded memory. The bound now follows from \Cref{prop:IS_general_body_state_cond},  the fact that \Cref{lem:ips_and_input_stable_state_conditioned} verifies the input-stability property (with $\epsilon,\tauc$ satisfies \eqref{eq:eps_cond_general}). \qed
\subsubsection{Proof of \Cref{thm:main_template_general}}

We begin with a lemma that upper bounds the imitation gaps by $\Delta_{\Iss,\sigma,h}(\pihat;\epsilon,\alpha(\epsilon) + 2\betaiss(2r,0))$ and other relevant terms. Essentially, the following lemma combines the general imitation guarantee in \Cref{thm:state_cond_imit_general} with the incremental stability analysis in \Cref{lem:iss_ips,lem:ips_and_input_stable_state_conditioned}.


\begin{lemma}\label{lem:smoothing_thing_annoying}
		Consider the instantiation of the composite MDP as in \Cref{defn:composite_instant_general}, let $r \le \cxi/4$, and recall $\alpha(\epsilon) = 2\betaiss(2\gammaiss(\epsilon),0)$. Further suppose that $\epsilon$ satisfy \Cref{eq:eps_cond_general}. Then, the the modified imitation gaps (whose definition we recall \Cref{defn:mod_imit_gaps}) 
		$\gapjoints[\alpha(\epsilon)] (\pihat \circ \Wsig \parallel \pistrep)$ and  $\gapmargs[\alpha(\epsilon)+\betaiss(2r,0)] (\pihat \circ \Wsig \parallel \pist)$ are bounded above by
\begin{align}  
H\left(4p_r +  3\gamma_{\sigma}\left(\max\{\alpha(\epsilon),\betaiss(2r,\tauc-\taum)\}\right)\right)  +  \sum_{h=1}^H\Delta_{\Iss,\sigma,h}(\pihat;\epsilon,\alpha(\epsilon) + 2\betaiss(2r,0)).
\end{align}
	\end{lemma}
Before proving the lemma, lets quickly show how it implies the desired theorem. We bound 
\begin{align}\Imitmarg[ 2\betaiss(2\gammaiss(\epsilon),0) + 2\betaiss(2r,0)] &\le \Imitmarg[ 2\betaiss(2\gammaiss(\epsilon),0) + \betaiss(2r,0)] \\
&= \Imitmarg[\alpha(\epsilon) + \betaiss(2r,0)]  (\pihat) \le \gapmargs[\alpha(\epsilon)+\betaiss(2r,0)] (\pihat \circ \Wsig \parallel \pist),
\end{align}
where the last inequality is due to as in the proof of \Cref{prop:TVC_main_general}, the intermediate inequality uses the definition of $\alpha(\epsilon)$, and the first inequality uses anti-monotonicity of $\Imitmarg$ in $\epsilon$. Moreover, as shown in the proof of \Cref{thm:main_template} in \eqref{eq:gamsig_for_gaussian} and \eqref{eq:p_r_bound}, we can take $ \gamsig(u) = \frac{u\sqrt{2\taum - 1}}{2\sigma}$ and for $p_r \le p$ when $r = \sigma \upomega_p$,  $\upomega_p := 2 \sqrt{5 \dimx + 2\log\left( \frac 1p \right)}$ and $\Wsig(\cdot)$ is the Gaussian Kernel in \eqref{eq:Gaussian_kernel}. Hence, we conclude that if $\sigma \le \cxi/4\upomega_p$, 
\begin{align}
\Imitmarg[\epsilon_1(p)] \le H\left(4p + \frac{3\sqrt{2\taum-1}}{2\sigma}\left(\max\left\{\epsilon_2,\betaiss(2\sigma\upomega_p,\tauc-\taum)\right\}\right)\right) + \sum_{h=1}^H\Delta_{\Iss,\sigma,h}(\pihat;\epsilon,\epsilon_1(p)),
\end{align}
where above $\epsilon_1(p) = 2\betaiss(2\gammaiss(\epsilon),0) + 2\betaiss(2\sigma \upomega_p,0)$ and $\epsilon_2 = 2\betaiss(2\gammaiss(\epsilon),0) $, as needed. Since $\gammaiss(\epsilon) \le 2 \sigma$, in we can choose $p = \frac{\gammaiss(\epsilon)}{\sigma 2}$ and upper bound $\upomega_p \le \upomega(\epsilon) := 2 \sqrt{5 \dimx + 2\log\left( \frac{2\sigma}{\gammaiss(\epsilon)} \right)}$. The bound now follows from this upper bound and the bound  $4p =  4\frac{2\cdot 2\gammaiss(\epsilon)}{\sigma 8} \le  4\frac{2\betaiss(2\gammaiss(\epsilon),0)}{\sigma 8} \le \frac{\epsilon_2}{2\sigma }$, the first inequality follows from \Cref{obs:simplify}.
\qed





	\newcommand{\sstarhhat}{\hat{\seqs}^\star_h}
	\newcommand{\couphat}{\hat{\coup}}

	\begin{proof}[Proof of \Cref{lem:smoothing_thing_annoying}]  Recall the replica and deconvolution kernels $\Qdech(\cdot),\Wreph(\cdot)$ defined in \Cref{defn:all_kernels}. We have that 
	\begin{align}
	\Exp_{\sstar_h \sim \Psth}\Exp_{\sstartil_h \sim \Wsig(\sstar_h) } \drobs[\alpha]( \pihat_{h}(\sstartil_h) \parallel \pidech(\sstartil_h) \mid \sstar_h) = \Exp_{\sstar_h \sim \Psth}\Exp_{\sstartil_h \sim \Wsig(\sstar_h) } \inf_{\coup}\Pr_{\coup}[\distAs(\seqa',\seqa \mid \sstar_h) > \alpha] \label{eq:the_above_thing}
	\end{align}
	where $\inf_{\coup}$ is over all couplings $\seqa_h \sim \pidech(\sstartil_h),\seqa_h' \sim \pihat(\sstartil_h)$. By the gluing lemma (\Cref{lem:couplinggluing}), each coupling in $\coup$ is equivalent to a coupling $\couphat$ over $(\sstar_h,\sstartil_h,\sstarhhat,\seqa,\seqa')$ where 
	\begin{itemize}
		\item $\sstar_h \sim \Psth$, $\sstartil_h \sim \Wsig(\sstar_h)$
		\item $\sstarhhat \mid \sstartil_h \sim \Qdech(\sstartil_h)$ for the deconvolution kernel defined in 
		\item $\seqa_h \sim \pist_h(\sstarhhat)$ and $\seqa_h' \sim \pihat(\sstartil_h)$
	\end{itemize}
	For couplings $\couphat$ of this form, and for $r > 0$, then, we can bound \eqref{eq:the_above_thing} via
	\begin{align}
	&\inf_{\couphat}  \Pr_{\couphat}[\distAs(\seqa_h',\seqa_h \mid \sstar_h) \le \alpha]\\
	&= \inf_{\couphat}\Exp_{\couphat}\I\{\distAs(\seqa_h',\seqa_h \mid \sstar_h) \le \alpha\}\\
	&=\inf_{\couphat}\Exp_{\couphat}\left[\I\left\{\sup_{\hat\seqs:\distips(\seqs,\sstarhhat) \le 2r}\distAs(\seqa_h',\seqa_h \mid \hat\seqs) > \alpha\right\} + \I\{\distips(\sstar_h,\sstarhhat) > 2r\} \right].
	\end{align}
	Because for any $\couphat$, $\sstarhhat \mid \sstartil_h \sim \Qdech(\sstartil_h)$, we see that the joint distribution $\sstar_h,\sstarhhat$ is independent of the coupling $\couphat$ and follows the replica distribution: $\sstarhhat \mid \sstar_h \sim \Wreph(\sstar_h)$. Consequently, by the Bayesian concentration lemma \Cref{lem:rep_conc}, the expected value of the term $\I\{\distips(\sstar_h,\sstarhhat) > 2r\}$ is at most $2p_r$. Hence, 
	\begin{align}
	\Exp_{\sstar_h \sim \Psth}\Exp_{\sstartil_h \sim \Wsig(\sstar_h) } \drobs[\alpha]( \pihat_{h}(\sstartil_h) \parallel \pidech(\sstartil_h) \mid \sstar_h)  \le 2p_r +  \inf_{\couphat}\Exp_{\couphat}\left[\I\left\{\sup_{\hat\seqs:\distips(\seqs,\sstarhhat) \le 2r}\distAs(\seqa_h',\seqa_h \mid \hat\seqs) > \alpha\right\}\right]
	\end{align}
	Again, marginalizing over $\sstar_h$ and using the form of the conditions, the right hand side of the above
	\begin{align}
	\Exp_{\sstar_h \sim \Psth}\Exp_{\sstartil_h \sim \Wsig(\sstar_h) } \drobs[\alpha]( \pihat_{h}(\sstartil_h) \parallel \pidech(\sstartil_h) \mid \sstar_h)  \le 2p_r +\inf_{\couphat}\Pr_{\couphat}\left[\sup_{\hat\seqs:\distips(\seqs,\sstarhhat) \le 2r}\distAs(\seqa_h',\seqa_h \mid \hat\seqs) > \alpha\right] \label{eq:second_to_last_eq_of_couphat}
	\end{align}
	
	Next, we  recall the function $\uppsi(u) := 2\betaiss(2\gammaiss(u),0)$, instantiate $\alpha = \uppsi(\epsilon)$, $\alpha' = \alpha + \betaiss(2r,0)$, $\rips = 2r$, and set  $\distAs(\seqa_h',\seqa_h \mid \seqs) $ to be 
    \begin{align}
     \distAs(\seqa_h',\seqa_h \mid \seqs) = \uppsi(\dloc(\seqa, \seqa' \mid \btilx_{\tauc}))\cdot \cI_{\infty}\left\{\dloc(\seqa, \seqa' \mid \btilx_{\tauc}) \le \epsilon_0\right\}, \quad  
     \end{align}
      Using that $\distips$ measures the Euclidean distance between the last control state of composite-state, we 
	\begin{align}
	\Pr_{\couphat}\left[\sup_{\seqs:\distips(\seqs,\sstarhhat) \le 2r}\distAs(\seqa_h',\seqa_h \mid \seqs) > \alpha\right] &= \Pr_{\couphat}\left[\sup_{\seqs:\distips(\hat\seqs,\sstarhhat) \le 2r} \dloc[\alpha'](\seqa_h, \seqa_h' \mid \hat\bx_{t_h}) > \epsilon\right]\\
	&= \Pr_{\couphat}\left[\sup_{\hat \bx_{t_h}:\| \bx_{t_h} - \hat \bx_{t_h}\| \le 2r} \dloc[\alpha'](\seqa_h, \seqa_h' \mid \hat\bx_{t_h}) > \epsilon\right]\\
	&= \Pr_{\couphat}\left[\sup_{\hat \bx_{t_h}:\| \bx_{t_h} - \hat \bx_{t_h}\| \le 2r} \max_{0 \le i < \tauc}\sup_{\delx:\|\delx\| \le \alpha'}\|(\sfk_{t_h+i}-\sfk_{t_h+i}')(\hat{\bx}_{t_h+i})\|  > \epsilon\right]\\
	&\le \Pr_{\couphat}\left[\sup_{\hat \bx_{t_h}:\| \bx_{t_h} - \hat \bx_{t_h}\| \le 2r} \max_{0 \le i < \tauc}\sup_{\delx:\|\delx\| \le \alpha''}\|(\sfk_{t_h+i}-\sfk_{t_h+i}')({\bx}_{t_h+i})\|  > \epsilon\right] \tag{the inequality just replaces $\alpha'$ with $\alpha''$}\\
	&= \Pr_{\couphat}\left[ \sup_{\hat \bx_{t_h}:\| \bx_{t_h} - \hat \bx_{t_h}\| \le 2r}\dloc[\alpha''](\seqa_h, \seqa_h' \mid \bx_{t_h})  > \epsilon\right], \label{eq:last_eq_of_couphat}
	\end{align}
	$\hat\bx_{t_h}$ is the first state in $\hat\seqs$, and $\bx_{t_h}$ the first state in $\sstarhhat$, $\hat\bx_{t_h:t_h+\tauc-1} = \rollout(\seqa_h;\hat\bx_{t_h})$,  $\hat\bx^\star_{t_h:t_h+\tauc-1} = \rollout(\seqa_h;\hat\bx_{t_h}^\star)$, and finally,
	\begin{align}
	\alpha'' := \underbrace{\alpha'}_{=\alpha + \betaiss(\rips,0)} + ~\Delta, \quad \Delta := \sup_{\hat \bx_{t_h}:\| \bx_{t_h} - \hat \bx_{t_h}\| \le 2r}\sup_{0 \le i < \tauc }\|\hat\bx_{t_h:t_h+\tauc-1} - \bx_{t_h:t_h+\tauc-1}\|. \label{eq:Delta_eq}
	\end{align}
	Now, we see that $\couphat$ ranges all couplinigs of the form
	\begin{itemize}
		\item $\sstarhhat \sim \Psth$ and $\sstartil_h \sim \Wsig(\sstarhhat)$ (by inverting the deconvolution)
		\item $\seqa_h' \sim \sstartil_h$ and $\seqa_h \sim \pist_h(\sstarhhat)$,
	\end{itemize}
	which we can see (under our instantiation of the composite MDP under \Cref{sec:analysis,app:end_to_end}) is equivalently to $\couphat$ ranging over all couplings in $\couphatsighbar$. Hence, by \Cref{asm:tis} (i.e. $\tiss$ of $\seqa_h$ at $\seqa_h$), we can bound $\Delta$ in \eqref{eq:Delta_eq} (using $r \le \cxi/4$) by $\Delta \le \betaiss(2r,0)$. Hence, we can bound $\alpha'' \le \alpha + 2\betaiss(2r,0)$, and thus we conclude (from \eqref{eq:last_eq_of_couphat} and \eqref{eq:second_to_last_eq_of_couphat}) that
	\begin{align}
	\Exp_{\sstar_h \sim \Psth}\Exp_{\sstartil_h \sim \Wsig(\sstar_h) } \drobs[\alpha]( \pihat_{h}(\sstartil_h) \parallel \pidech(\sstartil_h) \mid \sstar_h)  &\le 2p_r + \inf_{\couphat \in \couphatsighbar} \Pr_{\couphat}\left[\dloc[\alpha''](\seqa_h, \seqa_h' \mid \rollout(\seqa_h;\bx_{t_h})  > \epsilon\right] \\
	&= 2p_r + \Delta_{\Iss,\sigma,h}\left(\pihat;\epsilon,\alpha + 2\betaiss(2r,0)\right),
	\end{align}
	where the last equality is by definition of $ \Delta_{\Iss,\sigma,h}\left(\pihat;\epsilon,\alpha  + 2\betaiss(2r,0)\right)$.
	Consequently, from \Cref{thm:state_cond_imit_general}, for any policy $\pihat$,  both  $\gapjoints[\alpha ] (\pihat \circ \Wsig \parallel \pistrep)$ and  $\gapmargs[\alpha+\gamipstwo(2r)] (\pihat \circ \Wsig \parallel \pist)$ are upper bounded by
\begin{align}  
H\left(4p_r +  3\gamma_{\sigma}(\max\{\alpha,\gamipsone(2r)\})\right)  +  \sum_{h=1}^H\Delta_{\Iss,\sigma,h}\left(\pihat;\epsilon,\alpha+ 2\betaiss(2r,0)\right).
\end{align}
Subsituting in $\gamipsone(r) = \upbeta(r,\tauc-\taum), \quad \gamipstwo(r) = \upbeta(r,0)$,
we conclude that
\begin{align}
&\gapjoints[\alpha] (\pihat \circ \Wsig \parallel \pistrep) \vee \gapmargs[\alpha+\upbeta(2r,\tauc-\taum)] (\pihat \circ \Wsig \parallel \pist)\\
&\le H\left(4p_r +  3\gamma_{\sigma}(\max\{\alpha,\betaiss(2r,0)\})\right)  +  \sum_{h=1}^H\Delta_{\Iss,\sigma,h}\left(\pihat;\epsilon,\alpha + 2\betaiss(2r,0)\right).
\end{align}
Substituting in $\gamipsone(r) \le \beta(r,\tauc-\taum)$, $\gamipstwo(r) \le \upbeta(r,0)$  from \Cref{lem:ips_and_input_stable_state_conditioned}, as well as $\alpha = \uppsi(\epsilon) =2\betaiss(2\gammaiss(\epsilon),0) $ concludes. 
	\end{proof}
