%!TEX root = ../main.tex

\newcommand{\Paughproj}{\lawP^{\mathrm{proj}}_{\mathrm{aug},h}}
\newcommand{\seqz}{\mathsf{z}}
\newcommand{\zst}{\seqz^\star}
\newcommand{\zstil}{\tilde{\seqz}^\star}

\newcommand{\phiZ}{\phi_{\cZ}}
\newcommand{\phiV}{\phi_{\cV}}
\newcommand{\seqv}{\mathsf{v}}





\section{Imitation in the Composite MDP}\label{sec:imit_composite}
In this section, we prove our imitation guarantees in the composite MDP under the full generality of data augmentation.  The majority of this section is devoted to proving  a more general version of \Cref{thm:smooth_cor} that applies to vectorized notions of distance and helps tighten our bounds when instantiated in the control setting.  In Appendix \ref{app:generalizationsmooth}, we introduce some notation and state our most general result, \Cref{thm:smooth_cor_general}.  We then proceed to show that \Cref{thm:smooth_cor} follows from \Cref{thm:smooth_cor_general} and in Appendix \ref{app:smoothcor_general_proof}, we provide a detailed and rigorous proof of the main result.  In Appendix \ref{app:smoothcor_proof}, we show that the more general \Cref{thm:smooth_cor_general} impiles \Cref{thm:smooth_cor} from the text.

Throughout, we  also assume $\cS$ admits a direct decomposition. This is useful to capture the fact that we only apply smoothing on the $\pathm$ coordinates (memory chunk), not the full trajectory chunk $\pathc$.  
\begin{definition}[Direct Decomposition]\label{defn:direct_decomp} Let $\cS = \cZ \oplus \cV$ is a direct decomposition. We let $\phiZ$ and $\phiV$ denote projections onto the $\cZ$ and $\cV$ components, respectively.  We say that the $\cS = \cZ \oplus \cV$ is \emph{compatible} with the dynamics if  $F_h((\seqz,\seqv),\seqa) = F_h((\seqz,\seqv'),\seqa)$ for all $\seqv, \seqv' \in \cV$ and $\seqz \in \cZ$, and \emph{compatible} with policy $\pi$ if $\pi_h((\seqz,\seqv),\seqa) = \pi_h((\seqz,\seqv'),\seqa)$.; we define compatibility of a kernel $\lawW$ and of a pseudometric $\dist(\cdot,\cdot): \cS \times \cS \to \R_{\ge 0}$ with $\cS = \cZ \oplus \cV$ similarly.
\end{definition}
We emphasize that compatibility of dynamics with a direct decomposition does not make $\seqv$ irrelevant because $\dists$ still depends on $\seqv$.  For the purposes of the instantiation for control in the following appendix, we wish to control the imitation gaps on distances that do depend on $\seqv_h$, even though $\seqv_h$ does not figure directly into the dynamics.  Note that as defined, $\seqv_h$ does depend on the dynamics up until time $h-1$ and thus it is necessary to deal with this component in order to provide guarantees in $\dists$.

\subsection{A generalization of Theorem \ref{thm:smooth_cor}}\label{app:generalizationsmooth}
\newcommand{\epsvec}{\vec{\epsilon}}
\newcommand{\distsvec}{\vec{\dist}_{\cS}}
\newcommand{\distsi}[1][i]{{\dist}_{\cS,#1}}
\newcommand{\distsone}{\distsi[1]}

\newcommand{\distai}[1][i]{{\dist}_{\cA,#1}}
\newcommand{\distavec}{\vec{\dist}_{\cA}}
\newcommand{\gapjointvec}{\vec\Gamma_{\mathrm{joint},\epsvec}}
\newcommand{\gapmargvec}[1][\epsvec]{\vec\Gamma_{\mathrm{marg},#1}}
\newcommand{\drobvec}[1][\epsvec]{\vec{\dist}_{\mathrm{os},#1}}

We now state a generalization of \Cref{thm:smooth_cor}, which replaces a single distance by a vector of distances of dimension $K$; this will be useful for our instantiation of the composite MDP as a chunked control system in our final application (in particular, for deriving a bound on $\Imitfin$). It also showcases the most general structure accomodated by our proof technique. 

We begin by defining some notation:
\begin{itemize}
\item Let $K \in \N$ denote a dimension
\item Let $\epsvec \in \R_{\ge 0}^K$ denote a vector of tolerances
\item Let $\distsvec(\cdot,\cdot)$ denote a vector of pseudometrics $\distsi$ on $\cS$
\item Let $\distavec$ denote a vector of non-negative functions $\distai:\cA^2 \to \R_{\ge 0}$, not necessarily pseuometrics.
\item Let $\preceq$ denote vector wise inequality, and let the symbols $\wedge$ and $\vee$ be generalized to denote entrywise minima and maxima.  Similarly, addition of vectors is coordinate wise with scalars assumed to be broadcast appropriately.
\item We let $\distsi[1] = \disttvc$ denote the metric we consider for evaluating total variation distance. 
\end{itemize} 
We generalize We assume the following measure-theoretic regularity conditions, generalizing \Cref{ass:polishspaces} as follows.
\begin{assumption} \label{ass:polish_spaces_general}
    We assume that $\cS$ and $\cA$ are Polish spaces. This means they are metrizable, but we do not annotate their metrics because, e.g. the metric on $\cS$ may be other than $\dists$. We further assume that 
\begin{itemize}
\item $\distsi$ is a pseudometric and Borel measurable function from $\cS \times \cS \to \R_{\ge 0}$. 
\item For any $\epsilon \ge 0$, the set $\{(\seqa,\seqa') \in \cA \times \cA : \distai(\seqa,\seqa') > \epsilon\}$ is an open subset of $\cA\times \cA$; i.e. $\distai(\cdot,\cdot)$ is lower semicontinuous. In particular, this means $\distai$ is a Borel measurable function. Note that this implies that the 
\begin{align}\{(\seqa,\seqa') \in \cA \times \cA : \distavec(\seqa,\seqa') \not \preceq \epsvec\}.
\end{align}
is closed and thus measurable.
\end{itemize}
\end{assumption}
Note that the above assumption is the natural vectorized generalization of \Cref{ass:polishspaces}.  Next, we define vector versions of our imitation errors.
\begin{definition}[Imitation Errors, vector version]\label{defn:imit_gaps_vec} Given error parameter $\epsvec \in \R_{\ge 0}^K$, define 
\begin{itemize}
\item The \bfemph{vector joint-error} 
\begin{align}
\gapjointvec(\polhat \parallel \pist) := \inf_{\coup_1}\Pr_{\coup_1}\left[\exists h \in [H]: \distsvec(\shat_{h+1},\sstar_{h+1}) \vee \distavec(\seqast_h,\seqahat_h)   \not \preceq \epsvec\right],
\end{align} 
where the infimum is over trajectory couplings $((\shat_{1:H+1},\seqahat_{1:H}),(\sstar_{1:H+1},\seqa^\star_{1:H})) \sim \coup_1 \in \couple(\Dist_{\polhat},\Dist_{\polst})$ satisfying $\Pr_{\coup_1}[\shat_{1} = \sstar_1] = 1$.   
\item The \bfemph{vector marginal error} 
\begin{align}
\gapmargvec(\polhat \parallel \pist) := \max_{h \in [H]}\max\left\{\inf_{\coup_1}\Pr_{\coup_1}\left[\distsvec(\shat_{h+1},\sstar_{h+1})\not \preceq \epsvec\right],\, \inf_{\coup_1}\Pr_{\coup_1}\left[\distavec (\seqast_h,\seqahat_h)\not \preceq \epsvec\right]\right\}
\end{align} the same as the to joint-gap, with the ``$\max$'' outside the probability and infimum over couplings. 
\item The \bfemph{vector-wise one-step error}  
\begin{align}
\drobvec(\polhat_h(\seqs) \parallel \polst_h(\seqs)) := \inf_{\coup_2}\Pr_{\coup_2}\left[\distavec(\seqahat_h,\seqast_h) \not  \preceq \epsvec \right],
\end{align} where the infimum is over $(\seqast_h, \hat \seqa_h) \sim \coup_2 \in \couple( \bpolhat_h(\seqs),\bpol_h^\star(\seqs))$.
\end{itemize} 
\end{definition}

We now describe input stability. 
\begin{definition}[Input-Stability, vector version] \label{defn:fis_vector} A trajectory $(\seqs_{1:H+1},\seqa_{1:H})$ is \bfemph{input-stable} w.r.t. $(\distsvec,\distavec)$ if  all sequences $\seqs_1' = \seqs_1$ and $\seqs_{h+1}' = F_h(\seqs_h',\seqa_h')$ satisfy  
\begin{align}\distsi(\seqs_{h+1}',\seqs_{h+1}) \le  \max_{1 \le j \le h}\distai\left(\seqa_{j}',\seqa_j\right) ,\quad \forall h \in [H], i \in [K]
\end{align}
\end{definition} 


Finally, define input process stability. A slight technicality is that, in our instantiation, $\pist$ is taken to be a suitable regular condition probability of the joint distribution $\Dexp$ of expert trajectories. This means that $\pist$ can only really satisfy desired regularity conditions on  states visited with positive probabiliy by $\Dexp$. We address this subtlety by considering the following definition generalizing \Cref{defn:ips_body} in the body. We also restrict the kernels under consideration to those which produce distributions \emph{absolutely continuous} (\Cref{defn:abs_cont}) with respect to $\Psth$, and denoted with the $\ll$ comparator. More specifically, we only care about absolute continuity under the projections onto the $\cZ$ component of $\cS$. 
\begin{definition}[Input \& Process Stability, vector version]\label{defn:ips_vec}
Let $\pips \in (0,1)$, $\gamipsvec = (\gamipsi)_{1\le i \le K}$ be a collection non-decreasing maps $\gamipsi:\R_{\ge 0} \to  \R_{\ge 0}$, let   $\distips:\cS \times \cS \to \R$ be a pseudometric (possibly other than any of the $\distsi$), and $\rips > 0$.  We say a policy $\pist$ is \emph{$(\gamipsvec,\distips,\rips,\pips)$-(vectorwise-input-\&-process stable (vIPS)} if the following holds for any $r \in [0,\rips]$: 

Consider any sequence of kernels $\lawW_h:\cS \to \laws(\cS)$, $1\le h \le H$, satisfying 
\begin{align}
\forall h, \seqs \in \cS: \quad \Pr_{\tilde \seqs\sim \lawW_h(\seqs)}[\distips(\tilde \seqs,\seqs) \le r] = 1, \quad \phiZ \circ \lawW_h(\seqs) \ll \phiZ \circ \Psth. \label{eq:supp_contained}
\end{align}
Define a process $\seqs_1 \sim \Dinit$, $\tilde\seqs_h \sim \lawW_h(\seqs_h),\seqa_h \sim \pi_h(\tilde \seqs_h)$, and $\seqs_{h+1} := F_h(\seqs_h,\seqa_h)$. Then, with probability at least $1- \pips$,
\begin{itemize}
\item[(a)] the sequence $(\seqs_{1:H+1},\seqa_{1:H})$ is input-stable w.r.t $(\distsvec,\distavec)$ (as defined by \Cref{defn:fis_vector}).
\item[(b)]$\max_{h \in [H]} \distsi(F_h(\tilde\seqs_h,\seqa_h),\seqs_{h+1}) \le \gamipsi(r)$. 
\end{itemize}
\end{definition}
\newcommand{\epsvecmarg}{\epsvec_{\mathrm{marg}}}




We can now state our desired generalization. 



\begin{theorem}\label{thm:smooth_cor_general}   Suppose that there 
\begin{itemize}
\item[(a)]$\pist$ is $(\gamipsvec,\distips,\rips,\pips)$-vector IPS in the sense of \Cref{defn:ips_vec}.
\item[(b)] There is a direct decomposition of $\cS = \cZ \oplus \cV$, which associated projection maps $\phiZ$ and $\phiV$, and which is compatible with the dynamics, and policies $\pist$, $\pihat$, and smoothing kernel $\Wsig$, and $\distips$.
\item[(c)]  $\phiZ \circ \Wsig$ is $\gamma_{\sigma}$-TVC with respect to the pseudometric $\disttvc = \distsone$. 
\end{itemize} 
Let $\pihatsig$ be any policy which is $\gamhat$-TVC, also w.r.t. $\disttvc = \distsone$. Finally, let $\epsvec \in \R_{\ge 0}^K$, $r \in (0,\frac{1}{2}\rips]$, and define 
\begin{align}
p_r &:= \sup_{\seqs}\Pr_{\seqs' \sim \Wsig(\seqs)}[\distips(\seqs',\seqs) >  r], \quad \epsvecmarg := \epsvec + \gamipsvec(2r).
\end{align} Then, 
\begin{itemize}
\item For any policy $\pihat$,  both  $\gapjointvec (\pihatsig  \parallel \pistrep)$ and  $\gapmargvec[\epsvecmarg] (\pihatsig \parallel \pist)$ are upper bounded by%$\gapmarg[\epsilon + 2r](\pihat \circ \Wsig \parallel \pist)$ are both at most
\begin{align}
%\inf_{r > 0}  
\pips + H(2p_r + \gamhat(\epsvec_1) + (\gamhat + \gamtvcsig) \circ \gamipsone(2r))  + \sum_{h=1}^H\Exp_{\sstar_h \sim \Psth}\drobvec\,( \pihatsigh(\stel_h) \parallel \pistreph(\stel_h)) \label{eq:smooth_ub_app_one}
\end{align}
\item In the special case where $\pihatsig = \pihat \circ \Wsig$, we can take $\gamhat = \gamsig$, and obtain that $\gapjointvec(\pihatsig \parallel \pistrep)$ and $\gapmargvec[\epsvecmarg](\pihatsig \parallel \pist)$ are upper bounded by
\begin{align}
\pips + H\left(2p_r +  3\gamma_{\sigma}(\max\{\epsilon,\gamipsone(2r)\}\right)  + \textstyle \sum_{h=1}^H\Exp_{\sstar_h \sim \Psth}\Exp_{\sstartil_h \sim \Wsig(\sstar_h) } \drobvec( \pihat_{h}(\sstartil_h) \parallel \pidec(\sstartil_h)) . \label{eq:smooth_ub_app_two}
\end{align}
\end{itemize}
\end{theorem}
We note that \Cref{thm:smooth_cor} is  a special case of \Cref{thm:smooth_cor_general} and prove the former assuming the latter here at the end of the section.

\subsection{Proof of Theorem \ref{thm:smooth_cor_general} }\label{app:smoothcor_general_proof}



\subsubsection{Proof Overview and Coupling Construction}\label{sec:proof_construction}
We begin with an intuitive overview of the proof and partially construct the relevant intermediate trajectories used to define our coupling, after which we sketch the organization of the rest of Appendix \ref{app:smoothcor_general_proof}.

The proof proceeds by constucting a sophisticated coupling between the law of a trajectory evolving according to $\pihat$ and a trajectory evolving according to $\pistrep$ by introducing several intermediate sequences of composite states and composite actions.  

We partially specify this coupling below and formally construct it in Appendix \ref{app:proof_smooth_cor_general}.  Our construction is recursive and relies on the input and process stability as well as total variation continuity to show that if the trajectories generated by $\pistrep$ and $\pihat$ are close in $\drobvec[\epsvec]$ evaluated on states at step $h$, then they will remain close at step $h+1$.  There are a number of technical subtelties involved, especially those of a measure-theoretic nature, but much of the inuition can be gleaned from the following partial specification of the coupling $\coup$ over composite-state 
$(\shat_{1:H},\srep_{1:H},\stel_{1:H},\ssq_{1:H}) \subset \cS$, composite-actions  $(\arep_{1:H},\seqahat_{1:h},\atel_{1:H}) \subset \cK$ and interpolating composite-actions, $(\arepinter_{1:H},\atelinter_{1:H}) \subset \cA$. 

To define the construction, we define the probability kernels corresponding to the replica and deconvolution policies.  Note that these are slightly different from the definitions in the body due to the use of the direct decomposition; the intuition is the same, however.

\newcommand{\QdechZ}[1][h]{\lawW^{\star}_{\mathrm{dec},\cZ,h}}

\begin{definition}[Replica and Deconvolution Kernels]\label{defn:all_kernels} Let $\Paughproj$denote the joint distribution over $(\zst_h,\sstar_h,\zstil_h,\astar_h)$ under the generative process
\begin{align}
\sstar_h \sim \Psth, \quad \astar_h \sim \pist_h(\sstar_h), \quad 
\zst_h = \phiZ(\sstar_h), \quad \zstil_h \sim \phiZ\circ \Wsig(\sstar_h)
\end{align}
For $\seqz \in \cZ$, let $\QdechZ(\seqz)$ denote the distribution of $\zst_h$ conditioned on $\zstil_h = \seqz$, under $\Paughproj$. Given $\seqs = (\seqz,\seqv)$, define 
\begin{align}
&\Qdech(\seqs) = \QdechZ(\phiZ(\seqs)) \otimes \dirac_{\phiV(\seqs)}, \quad \\
&\Qreph(\seqs) = \Qdech \circ ( \Wsig(\phiZ(\seqs))\otimes \dirac_{\phiV(\seqs)}) =   (\QdechZ \circ \Wsig(\phiZ(\seqs)))\otimes \dirac_{\phiV(\seqs)}.
\end{align}
where we recall the dirac-delta $\dirac$. Equivalently, $\Qdech(\seqs)$ denotes the conditional sequence of $(\tilde \seqz,\seqv)$, where $\seqv = \phiV(\seqs)$, and $\tilde \seqz \sim \QdechZ(\seqs)$; $\Qreph$ can be expressed similarly. 
\end{definition}
We remark that $\Qdech$ and $\Qreph$ are both kernels and by \Cref{thm:durrett}, we may assume that the joint distribution over $(\sstar_h, \ssq_h)$ admits a regular conditional probability and thus these constructions are well-defined. 
\begin{remark}Note that the kernels $\Qdech$ and $\Qreph$ are  compatible with the decomposition $\cS = \cZ \oplus \cV$ by construction. Moreover, note that if $\seqs = (\seqz,\seqv)$, $\phiV \circ \Qdech(\seqs) = \phiV \circ \Qreph(\seqs)$ is the dirac-delta distribution supported on $\seqv$.
\end{remark}
\begin{lemma} Under our the assumption that $\pist$ and $\Wsig$ are compatible with the direct decomposition,  
\begin{align}
\pidech(\seqs) = \pist \circ \Qdech , \quad \pistreph(\seqs) = \pist \circ \Qreph 
\end{align}
\end{lemma}
\begin{proof} This follows imediately because $\pist$ and $\Wsig$ are  compatile with the direct decomposition, and by the definition of \Cref{defn:body_replica}.
\end{proof}


% Figure environment removed
\paragraph{A template for the coupling.} Our couplings are partially specified by the following generative process, and what remains unspecified are couplings between random variables at each each step $h$. In what follows, let $\cF_0$ denote the $\upsigma$-algebra generatived by $\shat_1 = \srep_1 = \stel_1 $. Let $\cF_h$ denote the sigma-algebra generated by  $(\shat_{1:h},\srep_{1:h},\stel_{1:h})$, $(\arep_{1:h},\sreptil_{1:h},\ssq_{1:h},\atel_{1:h},\seqahat_{1:h})$, and $(\arepinter_{1:h},\atelinter_{1:h})$.
\begin{itemize}
    \item The initial states are drawn as
    \begin{align}
    \shat_1 = \srep_1 = \stel_1 \sim \Dinit. 
    \end{align}
    \item The dynamics satisfy
    \begin{align}
    \shat_{h+1} = F_h(\shat_h,\seqahat_h), \quad \srep_{h+1} = F_h(\srep_h,\arep_h), \quad \stel_{h+1} = F_h(\ssq_h,\atel_h)
    \end{align}
    Note that determinism of the dynamics implies that $\stel_{h+1}$, $\srep_{h+1}$ and $\shat_{h+1}$ are $\cF_{h}$-measurable. 
    \item We generate
    \begin{align}
    &\sreptil_h \mid \cF_{h-1} \sim \Qreph(\srep_h), \quad \arep_h \mid \cF_{h-1},\sreptil_h \sim \pisth(\sreptil_h), \qquad \label{eq:trajevolve1} \\
    &\ssq_h \mid \cF_{h-1} \sim \Qreph(\stel_h), \quad \atel_h \mid \cF_{h-1},\ssq_h \sim \pisth(\ssq_h).\label{eq:trajevolve2}\\
    &\seqahat_h \mid \cF_{h-1} \sim \pihatsigh(\shat_h) \label{eq:trajevolve_ahat}
\end{align}
Importantly, we note that, marginalizing over $\ssq_h$ and $\sreptil_h$, respectively, $\atel_h \mid \cF_{h-1} \sim \pistreph(\stel)$ and $\arep_h \mid \cF_{h-1} \sim \pistreph(\srep_h)$.  
\item Lastly, we select interpolating actions via
\begin{align}
    &\arepinter_h \mid \cF_{h-1} \sim \pihatsigh(\srep_h), \qquad \atelinter_h \mid \cF_{h-1} \sim \pihatsigh(\stel_h)\label{eq:trajevolve3}
\end{align}
\end{itemize}
We will say $\coup$ is ``respects the construction'' as shorthand to mean that $\coup$ obeys the above equations.  The coupling is illustrated graphically in \Cref{fig:coupling_illustration}.  We now establish several key properties of the above constructions, separated into a subsection for the sake of clarity.


\paragraph{Organization of the remaining parts of Appendix \ref{app:smoothcor_general_proof}.}   In Appendix \ref{app:prop_of_deconv_replica}, we prove several prerequisite properties of the construction given above, including concentration of the smoothing kernel, and key properties of the replica distribution. Next, Appendix \ref{app:marg_imit_gap} shows that, due to these properties of the replica distribution, we can bound the marginal imitation gap by controlling the tracking of the teleporting sequence constructed above. Finally, in Appendix \ref{app:proof_smooth_cor_general} we formally construct the coupling and rigorously prove \Cref{thm:smooth_cor_general}.
\begin{comment}
\begin{observation} Let $\trajhat = (\shat_{1:H},\seqahat_{1:H})$, $\trajrep_{1:H} = (\sstar_{1:H},\seqast_{1:H})$, $\trajtel = (\stel_{1:H},\atel_{1:H})$.
\begin{itemize}
\item $\coup$ is an interpolating construction for $(\trajrep,\trajhat,\arepinter_{1:H})$ with respect to $(\pistrep,\pihat,(\cF_{h})_{h \ge 0})$.
\item $\coup$ is a teleporting construction for $(\trajtel,\trajrep,\ssq_{1:H})$ with respect to $(\pist,\Wsig,(\cF_h)_{h \ge 0})$. 
\end{itemize}
\end{observation}
\end{comment}




\newcommand{\Ctelh}[1][h]{\cC_{\mathrm{tel} ,#1}}
\newcommand{\Crephath}[1][h]{\cC_{ \hat{\seqs},#1}}
\newcommand{\Binterh}[1][h]{\cB_{ \mathrm{inter},#1}}
\newcommand{\Bhath}[1][h]{\cB_{\hat{\seqa},#1}}
\newcommand{\Btelh}[1][h]{\cB_{\mathrm{tel},#1}}
\newcommand{\Bfsh}[1][h]{\cB_{\mathrm{est},#1}}
\newcommand{\Callh}[1][h]{\cC_{\mathrm{all},#1}}
\newcommand{\Callbarh}[1][h]{\bar\cC_{\mathrm{all},#1}}
\subsubsection{Properties of smoothing, deconvolution, and replicas.}\label{app:prop_of_deconv_replica}

In this section, we establish several useful properties of smoothed and replica policies.  We begin by showing that smoothed policies are TVC.
\begin{lemma}\label{lem:pistrep_tvc}
The following hold
\begin{itemize}
    \item For any $h$, $\phiZ \circ \Qreph$ and $\pistreph$ are $\gamma_{\sigma}$ TVC.
    \item If $\pi$ is any policy compatible with the direct decomposition $\cS = \cZ \oplus \cV$ (in the sense of \Cref{defn:direct_decomp}), then $\pi\circ \Wsig$ is $\gamma_{\sigma}$-TVC.
\end{itemize}
\end{lemma}
\begin{proof} We observe that $\phiZ \circ \Qreph = \phiZ \circ \Qdech \circ \Wsig(\seqs)$. Moreover, we observe $\Qdech$ satisfies  $\phiZ \circ \Qdech(\seqs) =  \QdechZ \circ \phiZ$, so that $\phiZ \circ \Qreph = \QdechZ \circ \phiZ \circ \Wsig(\seqs)$. As $\phiZ \circ \Wsig$ is TVC, the first claim is a consequence of the data-processing inequality \Cref{cor:tv_two}. The second uses the fact that all listed objects involve composition of kernels with $\Wsig$.
\end{proof}
Next, we show that the replica construction preserves marginals. 
\begin{lemma}[Marginal-Preservation]\label{lem:replica_property} 
 There exists a coupling $\Pr$ of $\seqz_h \sim \phiZ \circ \Psth$, $\seqz_h' \sim \phiZ\circ\Wsig(\seqz_h,\cdot)$ (where ($\cdot$) denotes an irrelevant argument due to compatibility of $\Wsig$ with the direct decomposition), and $\tilde \seqz_h \sim \phiZ \circ \Qreph(\seqz_h,\cdot)$ (again, ($\cdot$) denotes an irrelevant argument) such that 
 \begin{align}
 (\seqz_h,\seqz_h') \overset{\mathrm{d}}{=}  (\tilde\seqz_h,\seqz_h').
 \end{align}
 In particular, for $\stel_h$ and $\ssq_h$ as in our construction, the marginal distributions of $\phiZ(\stel_h)$ and $\phiZ(\ssq_h)$ are the same, where $\stel_h \sim \Psth$ and  $\ssq_h \mid \stel_h \sim \Qreph(\stel_h)$.
\end{lemma}
\begin{proof}
    By \Cref{ass:polishspaces} and \Cref{thm:durrett}, we may assume that all joint distributions' conditional probabilities are regular conditional probabilities and thus almost surely equal to a kernel.  Moreover, since all kernels are compatible with the direct decomposition, it suffices to prove the special case of the trivial direct-decomposition where $\cZ = \cS$.  Fix a common measure $\pp$ over which $\stel_h, \ssq_h$, and $\mathsf{s}_h'$ are defined such that $\stel_h \sim \Psth$, $\mathsf{s}_h' \sim \Wsig(\stel_h)$, and $\ssq_h \sim \Wdeconvh(\mathsf{s}_h')$. Then for any measurable sets $A, B$, we have
    \begin{align}
        \pp(\stel_h \in A,\, \seqs_h' \in B) &= \pp(\seqs_h' \in B) \cdot \ee_{\seqs_h'}\left[\I[\seqs_h' \in B] \cdot \pp(\stel_h \in A | \seqs_h' ) \right] \\
        &= \pp(\seqs_h' \in B) \cdot \ee_{\seqs_h'}\left[\I[\seqs_h' \in B] \cdot \pp(\ssq_h \in A | \seqs_h' ) \right]\\
        &= \pp\left( \ssq_h \in A, \, \seqs_h' \in B \right),
    \end{align}
    where the first equality holds by the fact that we are working with regular conditional probabilities and Bayes' rule, the second equality holds by the definition of the deconvolution kernel above, and the last equality holds again by Bayes' rule and the tower rule for conditional expectations.

    To prove the second statement, we apply induction, again assuming that $\cZ = \cS$ as in the proof of the first statement.  Note that $\stel_1 \sim \Psth[1] = \Dinit$, and $\ssq_1 \sim \Qreph[1] \circ \Psth[1]$. Thus, from the first part of the lemma, $\phiZ (\stel_1) \sim \phiZ \circ \Psth[1]$. Now, suppose the induction holds up to step $h$. Then, $\ssq_h \sim \Psth$, as $\atel_h \sim \pist_h(\atel_h)$, then $\stel_{h+1} = F_{h}(\ssq_h,\atel_h) \sim \Psth[h+1]$. Again  $\ssq_{h+1} \sim \Qreph[h+1](\stel_{h+1})$, so that $\ssq_{h+1}$ has marginal $\Qreph[h+1]\circ \Psth[h+1] = \Psth[h+1]$, as needed.  
\end{proof}
We further show that $\Wreph$ can be defined to be absolutely continuous with respect to $\Psth$.
\begin{lemma}\label{lem:absolute_continuity}
    The kernel $\Wreph$ satisfies that $\phiZ \circ \Wreph \ll \phiZ \circ \Psth$ as laws, validating the second condition in \eqref{eq:supp_contained}.  It further holds that $\phiZ \circ \Wdeconvh \ll \phiZ \circ \Psth$.
\end{lemma}
\begin{proof}
    The first statement follows immediately from \Cref{lem:replica_property} because these distributions are the same.  The second statement follows immediately from the tower law of conditional expectation and the definition of $\Wdeconvh$.
\end{proof}

Lastly, we establish that the replica kernel inherits all concentration properties from the smoothing kernel.
\begin{lemma}[Replica Concentration]\label{lem:rep_conc} Recall that 
\begin{align}
p_r := \sup_{\seqs}\Pr_{\seqs' \sim \Wsig(\seqs) }[\distips(\seqs',\seqs) >  r].
\end{align} We then have
\begin{align}
\Pr_{\seqs_h \sim \Psth,\stil_h \sim \Qreph(\seqs_h)}[\distips(\stil_h,\seqs_h) > 2\rsmooth] \le 2p_r \label{eq:concentration_conv_two}
\end{align}
\end{lemma}
\begin{proof} %We fix a common measure $\Pr[\cdot]$ over $\seqs_h \sim \Psth,\stil_h \sim \Qreph(\seqs_h)$
Again, all terms -- $\Wsig,\Qreph,\Qdech$ and $\distips$ -- are compatible with the direct decomposition, it suffices to consider the case of the trivial direct decomposition under whcih $\cZ = \cS$.

Let $\Pr$ denote a distribution over $\seqs_h \sim \Psth$, $\seqs_h' \sim \Wsig(\seqs_h)$, and $\stil_h \sim \Qdech(\seqs_h')$. In this special case,  we see that $\stil_h \mid \seqs_h \sim \Qreph(\seqs_h)$\footnote{Notice that, for general $\cS = \cZ \oplus \cV$, this condition would become $\phiZ(\stil_h) \mid \phiZ(\seqs_h) \sim \phiZ \circ \Qreph(\phiZ(\seqs_h),\cdot)$, where the $\cdot$ argument is irrelevant.}. By a union bound,
\begin{align}\label{eq:dists_conv_bound_two}
\Pr_{\seqs_h \sim \Psth,\stil_h \sim \Qreph(\seqs_h)}[\distips(\seqs_h,\stil_h) > 2\rsmooth] &\le \Pr[\distips(\stil_h,\seqs'_h) > \rsmooth]  + \Pr[\distips(\seqs_h,\seqs'_h) > \rsmooth] \\
&= 2 \Pr[\distips(\seqs_h,\seqs'_h) > \rsmooth] \le 2p_r,
\end{align}
where the equality follows from the first statment of \Cref{lem:replica_property}.
\end{proof}
\begin{remark}Note that, in the previous lemma, it suffices that the following weaker condition holds: $\Pr_{\seqs \sim \Psth,\seqs' \sim \Wsig(\seqs)}[\distips(\seqs',\seqs) >  \rsmooth] \le p_r$, i.e. for concentration to hold only in distribution over $\seqs \sim \Psth$, instead of \emph{uniformly} over states.
\end{remark}
\newcommand{\Qtilreph}[1][h]{\tilde{\lawW}_{\repsymbol,#1}}


\subsubsection{Bounding the marginal imitation gaps in terms of the teleporting sequence error}\label{app:marg_imit_gap}
Before turning to the proof of \Cref{thm:smooth_cor_general}, we verify that closeness to the \emph{teleporting sequences} suffices to control error in marginal gap to $\pist$. The key property here is that the teleporting sequence, as shown in \Cref{lem:replica_property}, has the same marginal distribution over states as does $\pist$.

\begin{lemma}\label{lem:marg_imit_gap_tel} Let $\coup$ be any coupling obeying the construction of the couplings above. Then, 
\begin{align}
\gapmargvec(\pihatsig \parallel \pist) \le 
\Pr_{\coup}\left[\exists h \in [H]: \left\{\distsvec(\stel_{h+1},\shat_{h+1}) \not \preceq \epsvecmarg\right\} \cup \left\{ \distavec(\atel_h,\ahat_h) \not \preceq \epsvecmarg\right\}\right] 
\end{align}
\end{lemma}
\begin{proof}
 We begin with a (reverse) union bound.
\begin{align}
&\Pr_{\coup}\left[\exists h \in [H]: \left\{\distsvec(\stel_{h+1},\shat_{h+1}) \not \preceq \epsvecmarg\right\} \cup \left\{ \distavec(\atel_h,\ahat_h) \not \preceq \epsvecmarg\right\}\right] \\
&\ge\max_h\max\left\{\Pr_{\coup}\left[\distsvec(\stel_{h+1},\shat_{h+1}) \not \preceq \epsvecmarg\right],\, \Pr_{\coup}\left[\distavec(\atel_h,\ahat_h) \not \preceq \epsvecmarg\right]\right\}.
\end{align}
 By \Cref{lem:replica_property} implies that $\stel_h$ has the marginal distribution of $\sstar_h \sim \Psth$. Moreover, by construction, for each $h$, $\atel_h \mid \cF_h \sim \pistreph(\stel_h)$, Thus, for each $h$, $\stel_{h+1}$ and $\atel_h$ have the same \emph{marginals} as the marginals as $\sstar_{h+1}$ and $\astar_h$ under the distribution $\Dist_{\pist}$ induced by $\pist$. Hence, 
 \begin{align}
 \Pr_{\coup}\left[\distsvec(\stel_{h+1},\shat_{h+1}) \not \preceq \epsvecmarg\right] &\ge \inf_{\coup_1} \Pr\left[\distsvec(\sstar_{h+1},\shat_{h+1}) \not \preceq \epsvecmarg\right] \\
 \Pr_{\coup}\left[\distavec(\atel_{h},\ahat_{h}) \not \preceq \epsvecmarg\right] &\ge \inf_{\coup_1} \Pr\left[\distsvec(\astar_{h},\ahat_{h}) \not \preceq \epsvecmarg\right],
 \end{align}
 where the $\inf_{\coup_1}$ is, as  in \Cref{defn:imit_gaps,defn:imit_gaps_vec}, the infinum over couplings between $\Dist_{\pist}$ and $\Dist_{\pihat}$. Thus, 
 \begin{align}
&\Pr_{\coup}\left[\exists h \in [H]: \left\{\distsvec(\stel_{h+1},\shat_{h+1}) \not \preceq \epsvecmarg\right\} \cup \left\{ \distavec(\atel_h,\ahat_h) \not \preceq \epsvecmarg\right\}\right] \\
&\ge\max_h\max\left\{\inf \Pr_{\coup_1}\left[\distsvec(\sstar_{h+1},\shat_{h+1}) \not \preceq \epsvecmarg\right],\, \inf_{\coup}\Pr_{\coup_1}\left[\distavec(\astar_h,\ahat_h) \not \preceq \epsvecmarg\right]\right\}\\
&:= \gapmargvec(\pihatsig \parallel \pist).
\end{align}

\end{proof}

\subsubsection{Formal proof of Theorem \ref{thm:smooth_cor_general}}\label{app:proof_smooth_cor_general}
We now proceed to formally prove \Cref{thm:smooth_cor_general}

\paragraph{Key Events. } For the random variables defined above, we define three groups of events. 
\begin{itemize}
\item The \emph{coupling events}, denoted by $\cB$, which are controlled by carefully selecting a coupling.
\item The \emph{inductive events}, denoted by $\cC$, which we condition on when bounding the probability of the coupling events.
\item The \emph{stability events}, denoted by $\cQ$, which take advantage of the stability properties of the imitation policy. 
\end{itemize}
\newcommand{\Ballbarh}[1][h]{\bar\cB_{\mathrm{all},#1}}
\newcommand{\Ballh}[1][h]{\cB_{\mathrm{all},#1}}
\newcommand{\Qis}{\cQ_{\textsc{is}}}
\newcommand{\Qips}{\cQ_{\textsc{ips}}}
\newcommand{\Qclose}{\cQ_{\mathrm{close}}}
\newcommand{\Qall}{\cQ_{\mathrm{all}}}

\begin{definition}[Coupling Events]\label{defn:all_key_eents} Define the events
\begin{align}
    \Btelh &=  \left\{ \arep_h = \atel_h, ~\phiZ(\sreptil_h) = \phiZ(\ssq_h) \right\}\\
     \Bfsh &= \left\{ \distavec( \atelinter_h,\atel_h) \not \preceq \epsvec \right\} \\
    \Binterh &= \left\{ \atelinter_h = \arepinter_h  \right\} \\
    \Bhath &= \left\{  \arepinter_h = \seqahat_h \right\} \\
     \Ballh &= \Binterh \cap \Btelh \cap \Bfsh \cap \Bhath\\
    \Ballbarh &=  \bigcap_{j=1}^h \Ballh[h]
\end{align}
Notice that each of the events above are $\cF_{h}$-measurable. Moreover, note that on $\Ballbarh$, $\max_{1\le j \le h}\phiis(\seqahat_j,\arep_j) \le \epsilon$.
\end{definition}
\begin{definition}[Inductive Event]\label{def:inductive_event}

Define the events
\begin{align}
\Crephath &= \left\{  \distsvec(\srep_h, \shat_h) \preceq \epsvec \right\}, \\
\Ctelh &= \left\{  \distsvec(\srep_h, \stel_h) \preceq \gamipsvec(2r) \right\} \\
\Callh &:= \Crephath \cap \Ctelh\\
 \Callbarh &=  \bigcap_{j=1}^h \Callh[j]
\end{align}
Notice that all the above events are $\cF_{h-1}$-measurable, due to determinism of the dynamics. Note that also $\Pr_{\coup}[\Callbarh[1]] = 1$ for any $\coup$ that respects the construction (as $\srep_1 = \stel_1 = \shat_1$).
\end{definition} 
\begin{definition}[Stability Events] Define the events 
\begin{align} 
\Qclose &:= \left\{\forall h \in [H]: \distips(\srep_h,\sreptil_h) \le 2r \right\}\\
\Qis &:= \left\{(\srep_{1:H+1},\arep_{1:H}) \text{ is input-stable w.r.t. } (\distsvec,\distavec)\right\}\\
\Qips &:= \left\{\distsvec(F_h(\sreptil_{h},\arep_h),\srep_{h+1}) \le  \gamipsvec\circ \distips\left(\sreptil_h,\srep_{h}\right), \quad 1 \le j \le H\right\} \\
\Qall &:=  \Qips \cap \Qclose .
\end{align}
In words, $\Qclose$ the event on which $\srep_h$ and $\sreptil_h \sim \Qreph(\stel_h)$ are close, and $\Qis$and  $\Qips$ ensure consequencs of  (vector) input-stability and (vector) input process stability holds.
\end{definition}

\paragraph{Steps of the proof.}
First, we use stability to reduce the event $\Callbarh[h+1]$ to $\Callbarh \cap \Ballbarh$:
\begin{claim}[Stability Claim]\label{claim:stability_claim} By construction, 
\begin{align}\Callbarh[h+1] \subset \Qall \cap \Callbarh \cap \Ballbarh.
\end{align}
\end{claim} 
\begin{proof} It suffices to show that on $\Qall \cap \Callbarh \cap \Ballbarh$, $ \distsvec(\srep_{h+1},\shat_{h+1}) \preceq \epsvec$ and $\distsvec(\srep_{h+1},\stel_{h+1}) \preceq \gamipsvec(2r)$. By applying the event $\Qis$ to the sequence $\seqa'_h = \seqahat_h$ and $\seqs'_h = \shat_h$, we have that on $\Qall \subset \Qis$ that
\begin{align}
 \forall h \in [H], i \in [K], \quad \distsi(\srep_{h+1},\shat_{h+1}) \le  \max_{1 \le j \le h}\distai\left(\arep_j,\seqahat_{j}\right) \label{eq:Qis_consequence}
\end{align}


For the next point, note that the compatibility of the dynamics with the direct decomposition $\cS = \cZ \oplus \cV$ implies that there exists a dynamics map $F_h^{\cZ}$  for which 
\begin{align}
F_h(\seqs,\seqa) = F_h^{\cZ}(\phiZ(\seqs),\seqa).
\end{align}
Similarly, by applying $\Qips$ and $\Qclose$ and the event $\{\phiZ(\sreptil_h) = \phiZ(\ssq_h),\atel_h = \arep_h\}$ on $\Btelh$, it holds that on $\Qall \cap \Callbarh \cap \Ballbarh$ that, for all $h \in [H]$,
\begin{align}
\distsvec(\srep_{h+1},F_h(\sreptil_{h},\arep_h))  &= \distsvec(\srep_{h+1},F_h^\cZ(\phiZ(\sreptil_{h}),\arep_h)) \\
&= \distsvec(\srep_{h+1},F_h^\cZ(\phiZ(\ssq_{h}),\atel_h)) \tag{$\Btelh$}\\
&= \distsvec(\srep_{h+1},F_h(\ssq_{h},\atel_h)) \\
&= \distsvec(\srep_{h+1},\stel_{h+1})\\ 
&\le \gamipsvec\circ\distips\left(\stel_j,\ssq_{j}\right) \tag{$\Qips$}\\
&\le \gamipsvec\circ\distips\left(2r\right) \tag{$\Qclose$}.
\end{align}
\end{proof}
From \Cref{claim:stability_claim}, we decompose our error probability as follows:
\begin{lemma}[Key Error Decomposition] \label{lem:putting_couplings_together} Let $\coup$ respect the construction (in the sense of \Cref{sec:proof_construction}). Then, for any coupling $\coup$ which respects the construction,
\begin{align}
&\gapjointvec(\pihatsig \parallel \pirep) \vee \gapmargvec(\pihatsig \parallel \pist) \le \Pr_{\coup}[\Qall^c] + \sum_{h=1}^H\Pr_{\coup}[ \Ballbarh^c \cap \Callbarh \cap \Ballbarh[h-1]]\label{eq:Gamimit_decomp}
\end{align}
\end{lemma}
\begin{proof} In what follows, we use $\vec{v} \vee \vec{w}$ to denote the entrywise maximum of two vectors of the same dimension. Define the events $\cE_h := \Callbarh[h+1] \cap \Ballbarh$. Observe that the events are nested: $\cE_{h} \supset \cE_{h+1}$, and that on $\cE_H$, we have that for all $h \in [H]$
\begin{align}
\distsvec(\srep_{h+1},\shat_{h+1}) \vee \distavec(\arep_h,\ahat_h) &\preceq \epsvec \vee \distavec(\arep_h,\ahat_h) \tag{$\Crephath[h+1] \supset \Callbarh[h+1] \supset \cE_h$}\\
&\preceq \epsvec \tag{$\Ballbarh \supset \cE_h$}.
\end{align}
On $\Qall \cap \cE_H$, we have that
\begin{align}
\max_h \distsvec(\srep_h,\stel_h) \le \gamipsvec(2r), \quad \text{and}\quad \atel_h = \arep_h
\end{align}
Thus, by the triangle inequality and $\epsvecmarg = \epsvec + \gamipsvec(2r)$, on $\Qall \cap \cE_H$,
\begin{align}
\max_h \distsvec(\srep_h,\stel_h) \le \epsvecmarg, \quad \text{and}\quad  \distavec(\atel_h,\ahat_h)  =\distavec(\arep_h,\ahat_h)  \le \epsvec \le \epsvecmarg.
\end{align}
Thus, 
\begin{align}
&\Pr_{\coup}\left[\exists h \in [H]: \left\{\distsvec(\srep_{h+1},\shat_{h+1}) \vee \distavec(\arep_h,\ahat_h) \not \preceq \epsvec\right\} \cup \left\{\distsvec(\stel_{h+1},\shat_{h+1}) \vee \distavec(\atel_h,\ahat_h) \not \preceq \epsvecmarg\right\}\right]\\
&\le \Pr_{\coup}[(\Qall \cap \cE_H)^c] \label{eq:first_pr_coup_big}
\end{align}
In particular, this shows that
\begin{align}
\gapjointvec(\pihatsig \parallel \pirep)  \le \Pr_{\coup}[(\Qall \cap \cE_H)^c], 
\end{align}
and similarly, by \Cref{lem:marg_imit_gap_tel},
\begin{align}
\gapmargvec(\pihatsig \parallel \pist) \le \Pr_{\coup}[(\Qall \cap \cE_H)^c]
\end{align}
As $(\srep_{1:H+1},\arep_{1:H}) \sim \Dist_{\pistrep}$, \eqref{eq:first_pr_coup_big} shows that
\begin{align}\gapjointvec(\pihatsig \parallel \pirep) \vee \gapmargvec(\pihatsig \parallel \pirep)\le \Pr_{\coup}[(\Qall \cap \cE_H)^c].
\end{align} 
Let us conclude by bounding $\Pr_{\coup}[(\Qall \cap \cE_H)^c]$. Using the nesting structure $\cE_h = \bigcap_{j=1}^h \cE_j$, the peeling lemma, \Cref{lem:peeling_lem}, and a union bound, it holds that
\begin{align}
\Pr_{\coup}\left[(\Qall \cap \cE_H)^c\right] &\le \Pr_{\coup}[\Qall^c] + \Pr\left[ \exists h \in [H] \text{ s.t. } \left(\Qall \cap \cE_{h-1} \cap \cE_h^c\right) \text{ holds } \right ]\\
&\le \Pr_{\coup}[\Qall^c] + \sum_{h=1}^H\Pr_{\coup}\left[ \Qall \cap \cE_{h-1} \cap \cE_h^c \text{ holds } \right ]\\
&= \Pr_{\coup}[\Qall^c] + \sum_{h=1}^H\Pr_{\coup}\left[ \Qall \cap \Ballbarh[h-1] \cap \Callbarh[h]  \cap (\Ballbarh \cap \Callbarh[h+1])^c \text{ holds } \right ]\\
&= \Pr_{\coup}[\Qall^c] + \sum_{h=1}^H\Pr_{\coup}\left[ \Qall \cap \Ballbarh[h-1] \cap \Callbarh[h]  \cap \Ballbarh^c\right ]\\
&= \Pr_{\coup}[\Qall^c] + \sum_{h=1}^H\Pr_{\coup}\left[ \Qall \cap \Ballbarh[h-1] \cap \Callbarh[h]  \cap \Ballh^c\right ],
\end{align}
where the last step invokes \Cref{claim:stability_claim}.
\end{proof}
Next, we bound the contribution of $\Pr_{\coup}[\Qall^c]$ in \eqref{eq:Gamimit_decomp}, uniformly over all couplings.
\begin{lemma}\label{lem:Qall_bound} For all $\coup$ which respect the construnction, 
\begin{align}
\Pr_{\coup}[\Qall^c] \le \pips + 2Hp_r.
\end{align}
\end{lemma}
\begin{proof} $\Pr_{\coup}[\Qclose^c] = \Pr_{\coup}[\exists h: \distips(\stel_h,\ssq_h) > 2r] \le 2Hp_r$ by \Cref{lem:rep_conc} and a union bound. 


Let us now bound $\Pr_{\coup}[\Qclose \cap \Qips^c] \le \Pr_{\coup}[\Qips^c \mid \Qclose ]$. Define the kernels $\lawW_h(\seqs)$ to be equal to the kernel $\Wreph(\seqs)$ conditioned on the event $\seqs' \sim \Wreph(\seqs)$ satisfies $\distips(\seqs',\seqs) \le 2r$. Then, conditional on $\Qclose$, we see that the sequence $(\srep_{1:H+1},\sreptil_{1:H},\arep_{1:H})$ obeys the generative process
\begin{align}
\sreptil_{h} \mid \sreptil_{1:h-1},\srep_{1:h},\arep_{1:h-1} \sim \lawW_h(\seqs), \quad \arep_{h} \mid \sreptil_{1:h},\srep_{1:h},\arep_{1:h-1} \sim \pisth(\sreptil_h), \quad \srep_{h+1} = F_h(\srep_h,\arep_h).
\end{align} 
By construction, for each $h$, $\Pr_{\seqs' \sim \Wreph(\seqs)}[\distips(\seqs',\seqs) > 2r] = 0$. Thus, the definition of (vector) input process stability (\Cref{defn:ips_vec}) and assumption $r \le \frac{1}{2}\rips$ implies that $\Pr_{\coup}[\Qips^c \mid \Qclose ] \le \pips$.
\end{proof}
The remaining step of the proof is therefore to bound the second term in \eqref{eq:Gamimit_decomp}.
\begin{lemma}\label{lem:make_coupling}There exists a coupling $\coup$ which respects the construction and satisfies the following for any $h \in [H]$
\begin{align}
&\Pr_{\coup}[\Ballh^c \mid \cF_{h-1}] \\
&\le \gamhat \circ \disttvc(\srep_h,\shat_h) + (\gamhat + \gamtvcsig) \circ \disttvc(\srep_h,\stel_h)  + \drobvec\,( \pihatsigh(\stel_h) \parallel \pistreph(\stel_h)),~ \text{$\coup$-almost surely }
    %&\pp_{\coup}\left[ \Ballbarh^c \cap \Callbarh \cap \Ballbarh[h-1] \right] \leq \\
   % &\quad\delta + \gamtvc(\epsilon) + (\gamtvc + \gamsig)\circ\gamips(2r) + \Exp_{\stel_h \sim \coup}\drobvec[\epsilon](\pihat \parallel \pistrep \mid \stel_{h}, h)
    %2 \cdot p_r + 2 \cdot \gamtvc\left( \gamips(3r) \right) + \drob[\epsilon]\left( \pistrep, \pihat | \stel_{h+1}, h+1 \right) + \gamtvc(\epsilon).
\end{align}
Consequently, for all $h \in [H]$,
\begin{align}
&\Pr_{\coup}[ \Ballh^c \cap \Callbarh \cap \Ballbarh[h-1]] \\
&\le \gamhat(\epsvec_1) + (\gamhat + \gamtvcsig) \circ \gamipsone(2r) + 
\Exp_{\coup}[\drobvec\,( \pihatsigh(\stel_h) \parallel \pistreph(\stel_h))]
\end{align}
Moreover, $\seqs \mapsto \drobvec\,( \pihatsigh(\seqs) \parallel \pistreph(\seqs))$ is measurable. 
%as well as $\pp_{\coup}[\Callbarh[0]] = 1$.
\end{lemma}

\begin{proof}[Proof Sketch]
We begin by giving a high level overview of the construction, which is done recursively.  The key technical tool is \Cref{lem:couplinggluing} above, which allows us to transform any coupling $\coup$ between random variables $(X, Y)$ into a probability kernel $\coup(\cdot| X)$ mapping instances of $X$ to probability distributions on $Y$ such that $(X, Y) \sim \coup$ has the same law as $(X, Y \sim \coup(\cdot | X))$.  For each $h$, we then show that, assuming the coupling has kept the states and controls close together until time $h-1$, this will imply the following chain:
\begin{align}
    \underbrace{(\arep \leftrightarrow \atel)}_{\gamtvc \text{ and induction}} \to \underbrace{(\atel \leftrightarrow \atelinter)}_{\text{learning and sampling}} \to \underbrace{(\atelinter \leftrightarrow \arepinter)}_{\gamtvc \text{ and induction}} \to \underbrace{(\arepinter \leftrightarrow \seqahat)}_{\gamtvc \text{ and induction}}, \label{eq:mainproofoutline}
\end{align}
where the bidirectional arrows indicate individual couplings between the laws of the random variables that are constructed by the method outlined in text below and the single directional arrows denote the probability kernels described above. The full proof of the lemma is given in \Cref{sec:couplingconstruction}.
\end{proof}

\paragraph{Concluding the proof.}  Here, we finish the proof of \Cref{thm:smooth_cor_general}.  Recall that we wish to bound $\gapjointvec\,(\pihatsig \parallel \pirep) \vee \gapmargvec[\epsvecmarg](\pihatsig \parallel \pist)$. We begin by bounding $\gapjointvec\,(\pihatsig \parallel \pirep) \vee \gapmargvec[\epsvecmarg](\pihatsig \parallel \pistrep)$.  In light of \Cref{lem:putting_couplings_together}, it suffices to bound
 \begin{align}
 \Pr_{\coup}[\Qall^c] + \sum_{h=1}^H\Pr_{\coup}[ \Ballbarh^c \cap \Callbarh \cap \Ballbarh[h-1]], 
 \end{align}
 where $\coup$ is the coupling in \Cref{lem:make_coupling}.
Applying \Cref{lem:Qall_bound} and \Cref{lem:make_coupling},
\begin{align}
&\Pr_{\coup}[\Qall^c] + \sum_{h=1}^H\Pr_{\coup}[ \Ballbarh^c \cap \Callbarh \cap \Ballbarh[h-1]] \\
&\le \pips + 2Hp_r +   \sum_{h=1}^H\Pr_{\coup}[ \Ballbarh^c \cap \Callbarh \cap \Ballbarh[h-1]] \\
&\le   \pips + H(2p_r + \gamhat(\epsvec_1) + (\gamhat + \gamtvcsig) \circ \gamipsone(2r))  + \sum_{h=1}^H\Exp_{\stel_h \sim \coup}\drobvec\,( \pihatsigh(\stel_h) \parallel \pistreph(\stel_h))
\end{align}
To conclude, we note that for any $\coup$ which respects the construction, \Cref{lem:replica_property} ensures that $\stel_h$ as the marginal distribution of $\sstar_h \sim \pisth$. Thus, the above is at most
\begin{align}
\pips + H(2p_r + \gamhat(\epsvec_1) + (\gamhat + \gamtvcsig) \circ \gamipsone(2r))  + \sum_{h=1}^H\Exp_{\sstar_h \sim \Psth}\drobvec\,( \pihatsigh(\sstar_h) \parallel \pistreph(\sstar_h)) \label{eq:first_eq_i_showed}
\end{align}
which concludes the proof of \eqref{eq:smooth_ub_app_one} for $\gapjointvec(\pihat \parallel \pistrep)$. 

To prove \eqref{eq:smooth_ub_app_two} for $\gapjointvec(\pihat \parallel \pistrep)$, we consider the special case that $\pihatsig = \pihat \circ \Wsig$. By definition, $\pihatsigh =\pihat \circ \Wsig$. Thus, the  data-processing inequality for optimal transport (\Cref{cor:opt_trans}) 
\begin{align}\drobvec\,( \pihatsigh(\sstar_h) \parallel \pistreph(\sstar_h))  \le \Exp_{\seqs_h' \sim \Wsig(\sstar_h)}\drobvec\,( \pihat(\seqs_h') \parallel \pidech(\seqs_h')),
\end{align}
for all $\sstar_h$. Substituting this into \eqref{eq:first_eq_i_showed}, and setting $\gamhat = \gamsig$ (in view of \Cref{lem:pistrep_tvc}), finishes the argument.













\subsubsection{Proof of Lemma \ref{lem:make_coupling}}\label{sec:couplingconstruction}

Recall that \Cref{ass:polish_spaces_general} ensures all of the general measure-theoretic guarantees of Appendix \ref{app:prob_theory} hold true in our setting. Notably we need the gluing lemma (\Cref{lem:couplinggluing}) and the commuting of optimal transport metrics and conditional probabilities (\Cref{prop:MK_RCP}).

\paragraph{Proof strategy.} Our proof follows along similar lines as that of \Cref{prop:IS_general_body}, although with the added complication of including the smoothing.  We will inductively construct $\coup$.  A useful schematic for the construction at each step is the following diagram:
\begin{align}
    \underbrace{(\sreptil \leftrightarrow \ssq),(\arep \leftrightarrow \atel)}_{\Btelh} \to \underbrace{(\atel \leftrightarrow \atelinter)}_{\Bfsh} \to \underbrace{(\atelinter \leftrightarrow \arepinter)}_{\Binterh}\to \underbrace{(\arepinter \leftrightarrow \seqahat)}_{\Bhath}, \label{eq:mainproofoutline2}
\end{align}
where the events under each bidirectional arrow refer to the event such ensuring that there exists a coupling such that the objects are close.  We then will apply \Cref{lem:couplinggluing} to glue the individual couplings together.  We will then use \Cref{lem:peeling_lem} and a union bound to control the probability under our constructed coupling that any of the relevant events fail to hold, concluding the proof.

\newcommand{\coupfs}{\coup_{\mathrm{est}}}
\newcommand{\Ebarh}{\bar{\cE}_{h}}

\newcommand{\coupstel}{\coup_{\seqs,\mathrm{tel}}}  
\newcommand{\coupinterr}{\coup_{\mathrm{inter}}}  

\newcommand{\couptel}{\coup_{\mathrm{tel}}}  
\newcommand{\coupahat}{\coup_{\seqahat}}  
\paragraph{Recursive construction of $\coup$.} Let $h \ge 1$, and suppose that we have constructed the coupling $\coup^{(1:h-1)}$ for steps $1,\dots,h-1$ which respects the construction. Recall that $\cF_h$ denotes the sigma-algebra generated by  $(\shat_{1:h},\srep_{1:h},\stel_{1:h})$, $(\arep_{1:h},\sreptil_{1:h},\ssq_{1:h},\atel_{1:h},\seqahat_{1:h})$, and $(\arepinter_{1:h},\atelinter_{1:h})$. Notice that $\stel_{h+1},\srep_{h+1},\shat_{h+1}$ are determined by $\cF_h$ as well. Similarly, it can be seen from \Cref{defn:all_kernels} that $\phiV(\ssq_{h+1})$ and $\phiV(\sreptil_{h+1})$ are also determined by $\cF_{h}$ (since the replica kernel preserves the $\cV$-components). We summarize all these aforementioned variables in a random variable $Y_h$. Let $\cF_0$ denote the filtration generated by $\srep_1 = \stel_1 = \shat_1$. We let $Y_0 = (\srep_1,\stel_1,\shat_1)$. 

Correspondingly, let $Z_h$ denote the random variables $(\arep_{h},\phiZ(\sreptil_{h}),\phiZ(\ssq_{h}),\atel_{h},\seqahat_{h})$, and $(\arepinter_{h},\atelinter_{h})$ such that the joint law of these random variables respects the construction.  Our goal is then to specify, for each $h \in [H]$, a joint distribution of $(Y_{h-1},Z_{h})$.
Note that $Z_h,Y_{h-1}$ determines $Y_{h}$, and we call this induced law $\coup^{(h)}$.






We begin by specifying joint distributions conditional on $Y_{h-1}$ and subsets of $Z_h$, then glue them together by the gluing lemma. Below, we use use information-theoretic notation. 
\begin{itemize}
    \item By total variation continuity of $\phiZ \circ \Qreph$ (\Cref{lem:pistrep_tvc}),
    \begin{align}
    \TV(\pp_{\phiZ(\sreptil_{h}) \mid  Y_{h-1}},\pp_{\phiZ(\ssq_{h}) \mid  Y_{h-1}}) \le \gamtvcsig \circ \disttvc(\srep_h,\stel_h). 
    \end{align}
    Because $\arep_{h} \sim \pisth(\sreptil_{h+1})$ and $\atel_{h} \sim \pisth(\ssq_{h})$,  and $\pist$ is compatible with the decomposition $\cS = \cZ \oplus \cV$ (i.e. $\pisth(\seqs)$ is a function of $\phiZ(\seqs)$)
    \Cref{cor:tv_two} implies that (almost surely)
    \begin{align}
    \TV(\pp_{(\arep_h,\phiZ(\sreptil_{h}) \mid Y_{h-1}},\pp_{(\atel_h,\phiZ(\ssq_{h}) \mid  Y_{h-1}}) \le \gamtvcsig \circ \disttvc(\srep_h,\stel_h). 
    \end{align}
    Hence, \Cref{cor:first_TV} implies that there exists a coupling $\couptel^{(h)}$ over $Y_{h-1},(\phiZ(\sreptil_{h}),\arep_h),(\phiZ(\ssq_{h}),\atel_{h})$ respecting the construction such that $Y_h \sim \coup^{(h-1)}$ and such that (almost surely)
    \begin{align}\label{eq:couptel}
    \Exp_{\couptel^{(h)}}[\Btelh \mid Y_{h-1}] = \Pr_{\couptel^{(h)}}[(\phiZ(\sreptil_{h}),\arep_h) \ne (\phiZ(\ssq_{h}),\atel_{h})\mid Y_{h-1}] &\le \disttvc(\srep_h,\stel_h)].
    \end{align}
    \item In our construction, $\atel_h \mid Y_{h-1} \sim \pistreph(\stel_h)$, and $\atelinter_h \mid Y_{h-1} \sim \pihatsigh(\stel_h)$. 
    Thus, by definition of $\drobvec$, and the assumption $\I\{\distavec(\cdot,\cdot) \not\preceq \epsvec\}$ is lower semicontinuous, \Cref{prop:MK_RCP} implies that we may find a coupling $\coupfs^{(h)}$ of $(\atel_{h},\atelinter_{h},Y_{h-1})$ respecting the construction such that, almost surely,
    \begin{align}\label{eq:coupfs}
    \pp_{\coupfs^{(h)}}\left[\Bfsh^c \mid Y_{h-1} \right] &=  \pp_{\coupfs^{(h)}}\left[ \distavec(\atelinter_{h},\atel_{h}) \not \preceq \epsvec \mid Y_{h-1} \right] \\
    &= \drobvec\,( \pihatsigh(\stel_h) \parallel \pistreph(\stel_h))].
\end{align}
Moreover, that same proposition ensures measurability of $\seqs \to \drobvec\,( \pihatsigh(\seqs) \parallel \pistreph(\seqs))$.
\item Since $\atelinter_{h} \mid \cF_h \sim \pihatsigh(\stel_h)$ and $\arepinter_{h+1} \mid \cF_h \sim \pihatsigh(\srep_h)$, and since  $\pihatsigh(\cdot)$ is $\gamhat$-TVC by assumption,  
\begin{align}
\TV(\pp_{\atelinter_{h} \mid  Y_{h-1}},\pp_{ \arepinter_{h} \mid  Y_{h-1}}) \le \gamhat \circ \disttvc(\srep_h,\stel_h). 
\end{align}

\Cref{cor:first_TV}  implies that there is a coupling $\coupinterr^{(h)}$ between $(\atelinter_{h},\arepinter_{h},Y_{h-1})$ such that
\begin{align}\label{eq:coupinterr}
\pp_{\coupinterr^{(h)}}[\Binterh^c \mid Y_{h-1}] = \pp_{\coupinterr^{(h)}}\left[\atelinter_{h} \ne \arepinter_{h}  \mid Y_{h-1}\right] &\le  \gamhat \circ  \disttvc(\stel_h,\srep_h)
\end{align}
\item  Similarly, since $\arepinter_{h} \mid \cF_{h-1} \sim \pihat_h(\srep_h)$ and $\seqahat_{h+1} \mid \cF_{h-1}\sim \pihat_h(\shat_h)$, $\pihat_h(\cdot)$ is $\gamhat$-TVC,  \Cref{cor:first_TV} implies that there is a coupling $\coupahat^{(h)}$ between $(\arepinter_{h},\seqahat_{h},Y_{h-1})$ such that
\begin{align}\label{eq:coupahat}
\pp_{\coupahat^{(h)}}[\Bhath^c \mid Y_{h-1}] = \pp_{\coupahat^{(h)}}\left[\seqahat_{h} \ne \arepinter_{h} \mid Y_{h-1}  \right] \le \gamhat \circ \disttvc(\srep_h,\shat_h)
\end{align}
\end{itemize}

We can then apply the gluing lemma (\Cref{lem:couplinggluing}) to 
\begin{align}
X_{h,1} &= (\phiZ(\ssq_h),\atel_h,Y_{h-1}) \\ 
X_{h,2} &= (\phiZ(\sreptil_h),\arep_h,Y_{h-1}) \\
 X_{h,3} &= (\atel_h,\atelinter_h,Y_{h-1}) \\
  X_{h,4} &= (\atelinter_h,\arepinter_h,Y_{h-1}) \\
   X_{h,5} &= (\arepinter_h,\ahat_h,Y_{h-1})  
\end{align}
with 
\begin{align}
(X_{h,1},X_{h,2}) \sim \couptel^{(h)},\quad (X_{h,2},X_{h,3}) \sim  \coupfs^{(h)}, \quad (X_{h,3},X_{h,4})\sim \coupinterr^{(h)}, \quad (X_{h,4},X_{h,5})\sim\coupahat^{(h)}.
\end{align}
\Cref{lem:couplinggluing} guarantees the existence of a coupling $\mu^{(h)}$ consident with all sub-couplings $\couptel^{(h)}$, $\coupfs^{(h)},\coupinter^{(h)},\coupahat^{(h)}$. Then, $\coup^{(h)}$-almost surely (and using that $\cF_{h-1}$ is precisely the $\upsigma$-algebra generated by $Y_{h-1}$)
\begin{align}
&\Pr_{\coup^{(h)}}[\Ballh^c \mid \cF_{h-1}] \\
&\le \Pr_{\coup^{(h)}}[\Btelh^c \mid \cF_{h-1}] + \Pr_{\coup^{(h)}}[\Bfsh^c \cF_{h-1}] +  \Pr_{\coup^{(h)}}[\Binterh^c \cF_{h-1}]+\Pr_{\coup^{(h)}}[\Bhath^c \cF_{h-1}]\\
&\le \gamhat \circ \disttvc(\srep_h,\shat_h) + (\gamhat + \gamtvcsig) \circ \disttvc(\srep_h,\stel_h)  + \drobvec\,( \pihatsigh(\stel_h) \parallel \pistreph(\stel_h))\\
&= \gamhat \circ \disttvc(\srep_h,\shat_h) + (\gamhat + \gamtvcsig) \circ \disttvc(\srep_h,\stel_h)  + \drobvec\,( \pihatsigh(\stel_h) \parallel \pistreph(\stel_h))
\end{align}
This concludes the inductive construction.


For the second statement, notice that the events $\Callbarh \cap \Ballbarh[h-1]$ are $\cF_h$ measurable (thus determined by $\coup^{(h-1)}$) and, when they hold, $\distsvec(\srep_h,\stel_h) \preceq \gamipsvec(2r)$ and $\dists(\srep_h,\shat_h)  \preceq \epsvec$. For our purposes, we use $\disttvc = \distsi[1](\srep_h,\stel_h) \preceq \gamipsone(2r)$ and $\dists(\srep_h,\shat_h)  \preceq \epsvec_1$. Hence, 
\begin{align}
\max_{h\in [H]}\Pr_{\coup}[ \Ballh^c \cap \Callbarh \cap \Ballbarh[h-1]] &\le \gamhat(\epsvec_1) + (\gamhat + \gamtvcsig) \circ \gamipsone(2r) \\
&\quad+ \drobvec\,( \pihatsigh(\stel_h) \parallel \pistreph(\stel_h)).
\end{align}
The result follows.


\subsection{Proof of Theorem \ref{thm:smooth_cor}, and generalization to direct decompositions}\label{app:smoothcor_proof}

In this subsection, we consider the special case dealt with in \Cref{thm:smooth_cor}.  Note that there always exists a trivial direct decomposition that is compatible with all policies and dynamics simply by letting $\cV = \emptyset$ and $\cS = \cZ$.  We prove here the version of the result that involves a possibly nontrivial direct decomposition, as we will instantiate this in our control setting by letting $\cZ = \left\{ \pathm[h] \right\}$ and $\cS = \left\{ \pathc[h] \right\}$, i.e., projecting $\pathc[h]$ onto the last $\taum$ coordinates gives $\seqz_h$. We further consider a restriction of IPS to consider kernels absolutely continuous with respect to $\Psth$ in their $\cZ$ component. 
\begin{definition}[Restricted IPS]\label{defn:ips_restricted}
For a non-decreasing maps $\gamipsone,\gamipstwo:\R_{\ge 0} \to  \R_{\ge 0}$ a  pseudometric $\distips:\cS \times \cS \to \R$ (possibly other than $\dists$ or $\disttvc$), and $\rips > 0$, we say a policy $\pi$ is \emph{$(\gamipsone,\gamipstwo,\distips,\rips)$-restricted IPS} if the following holds for any $r \in [0,\rips]$. Consider any sequence of kernels $\lawW_1,\dots,\lawW_H:\cS \to \laws(\cS)$ satisfying 
\begin{align}
\max_{h,\seqs \in \cS}\Pr_{\tilde \seqs\sim \lawW_h(\seqs)}[\distips(\tilde \seqs,\seqs) \le r] = 1, \quad \forall s, \quad \phiZ \circ \lawW_h(\seqs_h) \ll \phiZ \circ \Psth. 
\end{align}
 and define a process $\seqs_1 \sim \Dinit$, $\tilde\seqs_h \sim \lawW_h(\seqs_h),\seqa_h \sim \pi_h(\tilde \seqs_h)$, and $\seqs_{h+1} := F_h(\seqs_h,\seqa_h)$. Then, almost surely, (a) the sequence $(\seqs_{1:H+1},\seqa_{1:H})$ is input-stable w.r.t $(\dists,\dista)$ (b) $\max_{h \in [H]} \disttvc(F_h(\tilde\seqs_h,\seqa_h),\seqs_{h+1}) \le \gamipsone(r)$ and (c) $\max_{h \in [H]} \dists(F_h(\tilde\seqs_h,\seqa_h),\seqs_{h+1}) \le \gamipstwo(r)$.
\end{definition}

Note that the above is a  slightly weaker condition than the one in \Cref{defn:ips_body} in the main text and consequently, the following theorem which uses it as an assumption implies \Cref{thm:smooth_cor} in the body.
\begin{theorem}\label{thm:smooth_cor_decomp} Suppose $\cS = \cZ \oplus \cV$ as in \Cref{defn:direct_decomp} and projections $\phiZ,\phiV$, which is compatible with the dynamics and with given policies $\pihat,\pist$, smoothing kernel $\Wsig$, and pseudometric $\distips$.
Suppose $\pist$ satisfies $(\gamipsone,\gamipstwo,\distips,\rips)$-restricted IPS (\Cref{defn:ips_restricted}) and $\phiZ \circ \Wsig$ is $\gamma_{\sigma}$-TVC. Let $\epsilon > 0$, $r \in (0,\frac{1}{2}\rips]$; define $p_r := \sup_{\seqs}\Pr_{\seqs' \sim \Wsig(\seqs)}[\distips(\seqs',\seqs) >  r]$ and $\epsilon' := \epsilon+\gamipstwo(2r)$. Then, for any policy $\pihat$,  both  $\gapjoint (\pihat \circ \Wsig \parallel \pistrep)$ and  $\gapmarg[\epsilon'] (\pihat \circ \Wsig \parallel \pist)$ are upper bounded by
\begin{align}
%\inf_{r > 0}  
H\left(2p_r +  3\gamma_{\sigma}(\max\{\epsilon,\gamipsone(2r)\})\right)  + \textstyle \sum_{h=1}^H\Exp_{\sstar_h \sim \Psth}\Exp_{\sstartil_h \sim \Wsig(\sstar_h) } \drob( \pihat_{h}(\sstartil_h) \parallel \pidec(\sstartil_h))  . \label{eq:smooth_ub}
\end{align}
\end{theorem}


Consider the special case $K = 2$ with $\distsi[1] = \disttvc$, $\distsi[2] = \dists$, $\distai[1] = \distai[2] = \dista$ and $\epsvec = (\epsilon, \epsilon)$.  In this case, applying \eqref{eq:smooth_ub_app_two}, we see that
\begin{align}
    &\gapjointvec(\pihatsig \parallel \pistrep) \vee \gapmargvec[\epsvecmarg](\pihatsig \parallel \pistrep) \\
    &\leq \pips + H\left(2p_r +  3\gamma_{\sigma}(\max\{\epsilon,\gamipsone(2r)\}\right)  + \textstyle \sum_{h=1}^H\Exp_{\sstar_h \sim \Psth}\Exp_{\sstartil_h \sim \Wsig(\sstar_h) } \drobvec( \pihat_{h}(\sstartil_h) \parallel \pidec(\sstartil_h))
\end{align}
We now observe that under this convention,
\begin{align}
    \gapjoint(\pihatsig \parallel \pistrep) &= \inf_{\coup_1} \pp_{\coup_1}[\max_{h \in [H]} \dists(\shat_{h+1}, \sstar_{h+1}) \vee \dista(\ahat_h, \astar_h) > \epsilon] \\
    &\leq \inf_{\coup_1} \pp_{\coup_1}\left[ \max_{h \in [H]} \left( \disttvc(\shat_{h+1}, \sstar_{h+1}), \dists(\shat_{h+1}, \sstar_{h+1}) \right) \vee \left( \dista(\ahat_h, \astar_h), \dista(\ahat_h, \astar_h) \right) \not \preceq \epsvec\right] \\
    &= \gapjointvec(\pihatsig \parallel \pistrep)
\end{align}
and similarly $\gapmarg[\epsilon'](\pihatsig \parallel \pist) \leq \gapmargvec[\epsvec + \gamips(2r)](\pihatsig \parallel \pist)$.  From the construction of $\distavec$, however, we see that $\left\{ \distavec(\seqa, \seqa') \not \preceq \epsvec \right\} = \left\{ \dista(\seqa, \seqa') > \epsilon \right\}$ for all $\seqa, \seqa'$ and thus for all $h \in [H]$,
\begin{align}
    \drobvec(\pihat_{h}(\sstartil_h) \parallel \pist_h(\sstartil_h)) &= \inf_{\coup_2} \pp_{\coup_2}\left[ \distavec(\seqahat_h, \seqast_h) \not \preceq \epsvec \right] \\
    &= \inf_{\coup_2} \pp_{\coup_2}\left[ \dista(\seqahat_h, \seqast_h) \geq \epsilon \right] \\
    &= \drob(\pihat_h(\sstartil_h) \parallel \pist_h(\sstartil_h)).
\end{align}
Plugging in to \eqref{eq:smooth_ub_app_two} concludes the proof.








