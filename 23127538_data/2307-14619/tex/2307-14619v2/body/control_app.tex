%!TEX root = ../neurips_main.tex
\newcommand{\Delkinf}{\Delta_{\bK,\infty}}
	\newcommand{\Deluinf}{\Delta_{\bu,\infty}}
\newcommand{\Deluk}[1][k]{\Delta_{\bu,#1}}
\newcommand{\Delxk}[1][k]{\Delta_{\bx,#1}}
\newcommand{\diststau}[1][\tau]{\dist_{\cS,#1}}
\newcommand{\distsxtau}[1][\tau]{\dist_{\cS,\bx,#1}}
\newcommand{\distsutau}[1][\tau]{\dist_{\cS,\bu,#1}}
\newcommand{\distAtau}{\bar{\dist}_{\cA,\tau}}

\newcommand{\Kric}{\matK^{\mathrm{ric}}}
\newcommand{\Pric}{\matP^{\mathrm{ric}}}
\newcommand{\ctrajbar}{\bar{\ctraj}}


\newcommand{\trajoff}{\ctraj}
\newcommand{\Deltilxk}[1][k]{\tilde{\Delta}_{\bx,#1}}
\newcommand{\Delbarxk}[1][k]{\bar{\Delta}_{\bx,#1}}
\newcommand{\bhatAk}[1][k]{\hat{\bA}_{#1}}
\newcommand{\bhatBk}[1][k]{\hat{\bB}_{#1}}
\newcommand{\ctrajat}{\hat{\ctraj}}
\newcommand{\Err}{\mathrm{Err}}
	\newcommand{\ErrK}{\mathrm{Err}_{\mathbf{K}}}
\newcommand{\Erru}{\Err_{\mathbf{u}}}


\newcommand{\Cstabnum}[1]{C_{\mathrm{stab},#1}}

\newcommand{\Rnot}{R_0}
\newcommand{\rmax}{r_{\max}}
\newcommand{\distarnotx}{\dist_{\cA,\Rnot,\tau,\bx}}
\newcommand{\bxpr}{\bx'}
\newcommand{\bupr}{\bu'}
\newcommand{\Constu}{C_{\mathbf{u}}}
\newcommand{\Constuinf}{C_{\mathbf{u},\infty}}


\newcommand{\Constk}{C_{\mathbf{K}}}
\newcommand{\Constkx}{C_{\mathbf{K},\bxoff}}

\newcommand{\Constdelx}{C_{\bm{\Delta}}}
\newcommand{\Constxhat}{C_{\hat{\bx}}}
\newcommand{\constu}{c_{\bu}}
\newcommand{\constk}{c_{\bK}}
\newcommand{\constdelx}{c_{\bm{\Delta}}}


\newcommand{\Constxhatk}{C_{\bK,\hat{\bx}}}

\newcommand{\buoff}{\bu}
\newcommand{\bxoff}{\bx}
	\newcommand{\Term}{\mathrm{Term}}

\newcommand{\Aclhatk}[1][k]{\hat{\bA}_{\mathrm{cl},#1}}
\newcommand{\Phiclhat}[1]{\hat{\bm{\Phi}}_{\mathrm{cl},#1}}
\newcommand{\rem}{\mathrm{rem}}
\newcommand{\matKtil}{\tilde{\matK}}
\newcommand{\Path}{\mathscr{P}}
\newcommand{\ltwo}{\ell_2}
\newcommand{\ltwoop}{\ell_2,\op}
\newcommand{\DelKk}[1][k]{\Delta_{\bK,#1}}
\newcommand{\partx}{\partial x}
\newcommand{\partu}{\partial u}

\newcommand{\dds}{\frac{\rmd}{\rmd s}}

\newcommand{\Ajac}{\mathbf{A}_{\mathrm{jac}}}
\newcommand{\Bjac}{\mathbf{B}_{\mathrm{jac}}}

\newcommand{\Bdyn}{B_{\mathrm{dyn}}}


\newcommand{\Lf}{L_{\mathrm{dyn}}}


\newcommand{\matX}{\mathbf{X}}
\newcommand{\matY}{\mathbf{Y}}
\newcommand{\matQ}{\mathbf{Q}}
\newcommand{\matLam}{\bm{\Lambda}}
\newcommand{\matPhi}{\bm{\Phi}}
\newcommand{\maxop}{\max,\op}


\newcommand{\matP}{\mathbf{P}}
\newcommand{\matK}{\mathbf{K}}
\newcommand{\matA}{\mathbf{A}}
\newcommand{\matB}{\mathbf{B}}
\newcommand{\matTheta}{\bm{\Theta}}
\newcommand{\Rstabtil}{\tilde{R}_{\mathrm{stab}}}
\newcommand{\betastab}{\beta_{\mathrm{stab}}}

\newcommand{\Aclkric}[1][k]{\matA_{\mathrm{cl},#1}^{\mathrm{ric}}}


\section{Stability in the Control System}\label{app:control_stability}
This section proves our various stability conditions. One wrinkle in the exposition is that we are able to derive far sharper perturbation guarantees than are needed in our analysis. However, as the guarantees are rather technically burdensome to derive, we endeavor to present the sharpest possible results so that we may save others from having to rederive these bounds in future applications. 

Importantly, this section also contains the definition of the constants $c_1,\dots,c_5 > 0$ present in \Cref{thm:main}, \Cref{prop:ips_instant}, and other main results (see \Cref{defn:IPS_consts}).

The section is organized as follows:
\begin{itemize}
	\item \Cref{sec:stabilitiy_prelims} recalls various preliminaries.
	\item  \Cref{sec:comp_prob_consts} provides the definition of numerous problem-dependent constants, all of which are polynomial in $(\Rdyn,\Ldyn,\Mdyn)$ and $(\Rstab,\Bstab,\Lstab)$ defined in \Cref{asm:Jacobian_Stable,asm:traj_regular}.
	\item \Cref{sec:IPS_guarantees} gives IPS guarantees in terms of the constants in the previous section. Specifically, it provides \Cref{defn:IPS_consts}, which instantiates the constants $c_1,\dots,c_5 > 0$ present in \Cref{thm:main}, \Cref{prop:ips_instant}, and other main results. We then state \Cref{cor:cor_stability_guarantee_all}, from which we derive \Cref{prop:ips_instant} used in the body. This corollary is derived from a sharper guarantee, \Cref{prop:main_stability_guarantee_all} (whose improvements over the corollary are detailed in \Cref{rem:stability_scaling}).
	\item The results in \Cref{sec:IPS_guarantees}  are derived from two building blocks in \Cref{sec:stab_of_trajectories}: \Cref{lem:state_pert} which bounds sensitive of regular trajectories to initial state, and \Cref{prop:master_stability_lem} which addresses perturbations of control inputs and gain.
	\item\Cref{prop:main_stability_guarantee_all} is derived from \Cref{prop:master_stability_lem}  in \Cref{sec:prop:main_stability_guarantee_all}. \Cref{lem:state_pert} and \Cref{prop:master_stability_lem} are proven in \Cref{sec:prop:master_stability_lem}  in \Cref{sec:prop:master_stability_lem}, respectively. 
	\item \Cref{sec:ric_synth} explains how to implement a synthesis oracle which produces Jacobian Stabilizing primitives controllers from trajectories which satisfy a natural stabilizability condition. 
	\item Finally, \Cref{sec:recursion_solutions} gives the solutions to various scalar recursions used in the proofs of \Cref{lem:state_pert} and \Cref{prop:master_stability_lem}.
	\end{itemize}

\subsection{Recalling preliminaries and assumptions.}\label{sec:stabilitiy_prelims} Recall the following definitions.  
\begin{itemize}
	\item A length-$K$ \emph{control trajectory} is denoted $\ctraj = (x_{1:K+1},u_{1:K}) \in \Ctraj_K = (\R^{\dimx})^{K+1} \times (\R^{\dimu})^K$. 
	\item Its \emph{Jacobian linearizations} are denoted $\bA_k(\ctraj) := \ddx \feta(\bx_k,\bu_k)$ and $\bB_k(\ctraj) := \ddu \feta(\bx_k,\bu_k)$ for $k \in [K]$. 
	\item Recalling our dynamics map $f(\cdot,\cdot)$, and step size $\eta > 0$,  we say $\ctraj$ is \emph{feasible} if, for all $k \in [K]$, 
	\begin{align}
	\bx_{k+1} = f(\bx_k,\bu_k), \quad \text{where } f(\bx,\bu) = \bx + \eta \feta(\bx,\bu).
	\end{align}

\end{itemize}
We regular the definition of regular trajectories from \Cref{sec:asms_ctrl_body}.
\begin{definition} A control path $\ctraj = (\bx_{1:K+1},\bu_{1:K})$  is $(\Rdyn,\Lf,\Mf)$-regular if for all $k \in [K]$ and all $(\bx'_k,\bu'_k) \in \R^{\dimx} \times \R^{\dimu}$ such that $\|\bx'_k-\bx_k\| \vee \|\bu_k-\bu'_k\| \le \Rdyn$,\footnote{Here, $\| \nablatwo \feta(\bx'_t,\bu'_t)\|_{\op} $ denotes the operator-norm of a three-tensor.}
	\begin{align}
	\| \nabla \feta(\bx'_k,\bu'_k)\|_{\op} \le \Lf, \quad  \| \nablatwo \feta(\bx'_k,\bu'_k)\|_{\op} \le \Mf.
	\end{align}
\end{definition}

We also recall the definitions around Jacobian stabilization. We start with a definition of Jacobian stabilization for feedback gains, from which we then recover the definition of Jacobian stabilization for primitive controllers given in the body.
\begin{definition}
Consider $\Rstab,\Lstab,\Bstab \ge 1$.  Consider sequence of gains $\matK_{1:K} \in (\R^{\dimu \times \dimu})^K$ and trajectory $\ctraj  = (\bx_{1:K+1},\bu_{1:K})\in \Ctraj_K$. We say that $(\ctraj,\bK_{1:K})$-is $(\Rstab,\Bstab,\Lstab)$-Jacobian Stable if $\max_{k}\|\matK_k\|_{\op} \le \Bstab$,  and if the closed-loop transition operators defined by
\begin{align}
\Phicl{k,j} := (\eye + \step \Aclk[k-1]) \cdot(\eye + \step\Aclk[k-2]) \cdot (\dots) \cdot (\eye + \step \Aclk[j])
\end{align} 
with $\Aclk[k] = \bA_k(\ctraj) + \bB_{k-1}(\ctraj) \bK_{k-1}$ satisfies the following inequality
\begin{align}
\|\Phicl{k,j}\|_{\op} \le \Bstab(1 - \frac{\eta}{\Lstab})^{k-j}.
\end{align}
\end{definition}
The definition of Jacobian stability of primitive controllers in \Cref{sec:asms_ctrl_body} may be recovered as follows.
\begin{definition}Consider $\Rstab,\Lstab,\Bstab \ge 1$. Consider a sequence of primitive controllers $\sfk_{1:K} \in \cK^K$, each expressed as $\sfk_{k}(\bx) = \bbaru_k = \bbarK_k (\bx_k-\bbarx_k)$ and $\ctraj =(\bx_{1:K+1},\bu_{1:K}) \in \Ctraj_K$. We say $(\ctraj,\sfk_{1:K})$ is Jacobian Stable if $\sfk_{1:K}$ is consistent with $\ctraj$, and if $(\ctraj,\bbarK_{1:K})$ is $\Rstab,\Lstab,\Bstab > 0$-Jacobian stable. 
\end{definition}
Note that in Jacobian stability (both with primitive controllers and with gain-matrices), we take all parameters to be no less than one. 


\subsubsection{Properties satisfied by $\pist$}
\newcommand{\Dpathm}{\cD_{\mathrm{exp},\pathm}}
Finally, we show that actions produced by $\pist$ in our control instantiation of the composite MDP satisy the assumptions in \Cref{asm:traj_regular,asm:Jacobian_Stable}.
\begin{lemma} Suppose \Cref{asm:traj_regular,asm:Jacobian_Stable}  hold. Let $\pist = (\pist_{h})_{1 \le h \le H}$ denote the policy constructed as a regular conditional probability from the conditionals of $\Dexp$. Furthermore, let $\Dpathm$ denote the distribution over $\pathm$ corresponding to $\ctraj_T \sim \Dexp$. Then, with probability one over $\pathm \sim \Dpathm$ and $\seqa_h \sim \pist_h(\pathm)$, expressed as $\pathm = (\bx_{t_h:t_h-\taum+1},\bu_{t_h-1:t_h - \taum + 1})$, and $\seqa_h = \sfk_{t_h:t_{h+1}-1}$.  Consider the unique feasible trajectory for which
\begin{align}
\pathc[h+1] = ( \bx_{t_h:t_{h+1}}', \bu_{t_h:t_{h+1}-1}'), \quad \bx_{t_h}' = \bx_{t_h}, \quad \bu_{t} = \sfk_{t}(\bx_{t}), \quad t_h \le t < t_{h+1}.
\end{align}
Then,
\begin{itemize}
	\item $\pathc[h+1]'$ is $(\Rdyn,\Ldyn,\Mdyn)$-regular
	\item $(\pathc[h+1]',\sfk_{t_h:t_{h+1}-1})$ is $(\Rstab,\Bstab,\Lstab)$-Jacobian stable.
\end{itemize} 
\end{lemma}
\begin{proof} Since $\pist$ in \Cref{def:Dexp_policies} is constructed as the regular conditional probabiliy of $\seqa_h \mid \pathm$ under $\Dexp$, $(\seqa_h,\pathm)$ is the above lemma have the same joint distribution as under $\Dexp$. Thus, the lemma follows from the assumptions \Cref{asm:traj_regular,asm:Jacobian_Stable} placed on $\Dexp$.
\end{proof}
The following is a direct consequence of the above lemma.
\begin{lemma}\label{lem:pist_good_actions} Consider the instantiation of the composite MDP for the control setting as in \Cref{sec:control_instant_body} and in \Cref{app:end_to_end}, with $\pist$ as in \Cref{def:Dexp_policies}, and  $\phiZ$  as in \Cref{defn:direct_decomp}. Suppose that $\lawW_1,\dots,\lawW_h: \cS \to \laws(\cS)$ satisfy\footnote{Recall the absolute-continuity comparator $\ll$ defined in \Cref{defn:abs_cont}.}
\begin{align}
\phiZ \circ \lawW_h(\seqs) \ll \phiZ \circ \Psth, \label{eq:Ws_cond_stab}
\end{align}
Consider a sequence of actions $\seqs_{1:H+1},\seqa_{1:H}$ generated via
\begin{align}
\seqa_h \sim \pist_h(\tilde \seqs_h), \quad  \tilde \seqs_h \sim  \lawW_h(\seqs_h), \quad \seqs_{h+1} = F_h(\seqs_h,\seqa_h), \quad \seqs_1 \sim \Dinit.
\end{align}
Let $\tilde \seqs_h = (\tilde \bx_{t_{h-1}:t_h},\tilde \bu_{t_{h-1}:t_h-1})$ and $\seqa_h = \sfk_{t_h:t_{h+1}-1}$. 
Then, with probability one, for each $h$, the  unique feasible trajectory for which
\begin{align}
\pathc[h+1] = ( \bx_{t_h:t_{h+1}}', \bu_{t_h:t_{h+1}-1}'), \quad \bx_{t_h}' = \bx_{t_h}, \quad \bu_{t} = \sfk_{t}(\bx_{t}), \quad t_h \le t < t_{h+1}.
\end{align}
satisfies
\begin{itemize}
	\item $\bx_{t_h}' = \tilde{\bx}_{t_h}$, and $\pathc[h+1]'$ is feasible and $(\Rdyn,\Ldyn,\Mdyn)$-regular
\item   $(\pathc[h+1]',\sfk_{t_h:t_{h+1}-1})$ is $(\Rstab,\Bstab,\Lstab)$-Jacobian Stable. 
\end{itemize} 
\end{lemma}


\begin{remark}[On the absolute continuity constraint in \eqref{eq:Ws_cond_stab}]\label{rem:eq:Ws_cond_stab} Recall that $\phiZ$ as defined in \Cref{defn:smoothing_instantiation} simply extracts the memory chunk $\pathm$ from the trajectory chunk $\pathc$. 
The condition $ \supp (\phiZ \circ \lawW_h(\seqs)) \subset \supp(\phiZ \circ \Psth)$ just means that the distribution of the memory chunk-components from $\lawW_h(\seqs)$is absolutely continuous with respect to the memory-chunks $\pathm$ under $\Dexp$.
\end{remark}

\newcommand{\maxtwo}{\max,2}

\subsubsection{Norm notation.} Lastly, given our parameter $\eta > 0$, we define two types of norms. First, for sequences of vectors $\bz_{1:K} \in (\R^d)^K$ and matrices $(\bX_{1:K}) \in \R^{d_1\times d_2})^K$, define 
\begin{align}
\|\bz_{1:K}\|_{\ltwo} = \left(\eta \sum_{k=1}^K \|\bz_k\|^2 \right), \quad \|\bX_{1:K}\|_{\ltwoop} = \left(\eta \sum_{k=1}^K \|\bX_k\|^2 \right),
\end{align}
where again the standard $\|\cdot\|$ notation denotes Euclidean norm for vectors and operator norm for matrices. We also define
\begin{align}
\|\bz_{1:K}\|_{\maxtwo} = \max_{1\le k \le K}\|\bz_k\|, \quad \|\bX_{1:K}\|_{\maxop} = \max_{1\le k \le K}\|\bX_k\|.
\end{align}


\subsection{Composite Problem Constants}\label{sec:comp_prob_consts}



We begin by writing down numerous problem constants, all of which are polynomial in the quantities $(\Rdyn,\Ldyn,\Mdyn)$ and $(\Rstab,\Bstab,\Lstab)$. First, we define the \bfemph{stability exponent}, 
\begin{align}
\betastab := (1-\frac{\eta}{\Lstab}) \in (0,1). \label{eq:betastab}
\end{align}



\begin{definition}\label{defn:stability_setup} Given the regularity parameters $\Rdyn,\Ldyn,\Mdyn$, stability parameters $\Rstab,\Bstab,\Lstab$, and the step size $\eta > 0$, we define the ``little-c'' constants
\begin{align}
	\constu = 12\Bstab\sqrt{\Lstab}\Ldyn, \quad \constk = 2\Bstab + 12 \Bstab \Lstab^{1/2}\Ldyn, \quad \constdelx = 6\Bstab
\end{align}
	as well as ``big-C'' constants
	\begin{align}
	%\Rstabtil &= \max\left\{1,\Rstab,\max_{k \in [K]}\|\bK_k'\|\right\}\\
	\Constu &:= \min\left\{\frac{\sqrt{\Lstab}\Ldyn}{\Mdyn},\frac{1}{256\Bstab^2 \Mdyn \Ldyn \Lstab^{3/2}}\right\}\\
	 \Constdelx&:= \frac{1}{4\cdot324\Bstab^2 \Mdyn \Lstab} \label{eq:xhattil_close}\\
	 \Constxhat &:= \min\left\{\frac{\Rdyn}{2\Rstab\Bstab},\frac{1}{16\Lstab \Mdyn\Rstab^2\Bstab^2}\right\} \\
	  \Constk &:= \min\left\{\frac{1}{24\sqrt{\Lstab}\Bstab\Ldyn},\frac{\Constxhat^{-1}\Ldyn}{8\cdot324\Bstab^2 \Mdyn \Lstab^{3/2}}\right\}\\
	 \Constxhatk &:= \frac{\Ldyn}{8\cdot324\Bstab^2 \Mdyn \Lstab^{3/2}}.
	  %, \frac{\Constxhat^{-1}\Rdyn}{12c_2\Rstabtil }\right\}\\
	  \end{align}
	  %Finally, define 
	  %\begin{align}
	 %\Constuinf &:= \frac{\Rdyn}{(3+12\constu\Rstabtil\Lstab)}
	 %\end{align}
\end{definition}
The ``little-c'' constants enter directly into our error bounds, where as the ``big-C'' constants function as constraints on errors, above which we lose guarantees. We define some additional ``big-C'' constants which take in a radius argument $R_0$.

\begin{definition}[Final Stability Constants]\label{defn:constants_for_stability_stuff} In term of the constants in \Cref{defn:constants_for_stability_stuff}, we define the following final stability constants, as functions of a parameter $R_0$:
\begin{align}
\Cstabnum{1}(R_0) &:= \min\left\{\Constu, \frac{\Constdelx}{4\constu}, \frac{\Rdyn}{R_0}\cdot\frac{1}{48\constu\constdelx}\right\} \\
\Cstabnum{2}(R_0) &:= \min\left\{\Constk,\frac{\betastab^{-\tauc/3}\Constdelx}{4\constu},\frac{\Rdyn}{R_0}\cdot\frac{\betastab^{-\tauc/3}}{48\constk\constdelx}\right\}\\
\Cstabnum{3}(R_0) &:= \frac{\Rdyn}{12R_0\constu \sqrt{\Lstab} +3}\\
\Cstabnum{4}(R_0)&:= \min\left\{\Constxhat, \frac{\Constxhatk}{\Constk},\frac{\Rdyn}{R_0}\cdot\frac{1}{12\constk}\right\}
\end{align}
\end{definition}





\begin{comment}
\begin{proposition}\label{prop:stabilizing} Consider a \emph{feasible}, $(\Rdyn,\Lf,\Mf)$-regular path $\ctraj = (x_{1:K+1},u_{1:K}) \in \Path_K$. Let $\Rstab \ge 0$ and $\Lstab,\Bstab \ge 1$, and let  $\matK_{1:K}$  $(\Rstab,\Bstab,\Lstab)$-stabilize $\ctraj$. Then, for any other sequences of inputs $\tilde{u}_{1:K}$, initial state $\tilde x_1$, and gains $\tilde \matK_{1:K}$ satisfying $\max_{k \in [K]} \|\tilde \matK_k\| \le \Rstabtil$, and the inequalities
\begin{align}
&\text{(a)}\quad(1+\Lf\Lstab^{1/2})\|u_{1:K} - \tilde{u}_{1:K}\|_{\ltwo} +\|x_1 - \tilde{x}_1\|  \le \frac{1}{16\Bstab^2\Lstab\Mf(1+\Rstab^2) \vee 4\Bstab\Rdyn(1+\max\{\Rstab,\Rstabtil\})}\\
&\text{(b)}\quad\|\matKtil_{1:K}-\matK_{1:K}\|_{\ltwoop} \le \frac{1}{4\Lf\Bstab\Lstab^{1/2}},
\end{align}
the path $\tilde \ctraj = (\tilde x_{1:K+1}, \tilde u_{1:K})$ given by the dynamics  $\tilde{x}_{t+1} = \tilde{x}_t + \eta f(\tilde{x}_t,\tilde{u}_t + \matKtil_k(\tilde{x}_t - x_t))$ with $\tilde{x}_1$ as above satisfies the following inequality for all $k \in [K]$:
\begin{align}
\|x_{k+1}-\tilde x_{k+1}\| &\le  (1+2\Bstab \Lf\Lstab^{1/2})\|\tilde{u}_{1:k}-u_{1:k}\|_{\ltwo}  + \Bstab\left(1-\frac{\step}{2\Lstab}\right)^{k}\|\tilde{x}_1 - x_1\|.
\end{align}
\mscomment{change in proof $\left(1-\frac{\step}{\Lstab}\right)^{k/2} \le \left(1-\frac{\step}{2\Lstab}\right)^{k}$}
\end{proposition}
\end{comment}



\subsection{ IPS Guarantees \& Proof of \Cref{prop:ips_instant}}\label{sec:IPS_guarantees}
Here we provide our main stability guarantees for the learned policy $\pist$ under \Cref{asm:Jacobian_Stable,asm:traj_regular}, from which we derive \Cref{prop:ips_instant}. This section adopts the notation from \Cref{sec:control_instant_body}.

We begin by introducing a functional form for our distances. 
\begin{definition}[Distances]\label{defn:diststau}
Let $\tauc$ be given, and let $0 \le \tau \le \tauc$. For $h > 1$ and chunk-states $\seqs_h = (\bx_{t_{h-1}:t_{h}},\bu_{t_{h-1}:t_{h}-1}) \in \scrP_{\tauf}$ and $\seqs_h' = (\bx_{t_{h-1}:t_{h}}',\bu_{t_{h-1}:t_{h}-1}')$, define
\begin{align}
\distsxtau(\seqs_h,\seqs_h')  &:= \max_{t\in [t_{h}-\tau:t_h]}\|\bx_t-\bx_t'\|\\
\distsutau(\seqs_h,\seqs_h')  &:= \max_{t \in [t_{h}-\tau:t_h-1]}\|\bu_t - \bu_t'\|\\
\diststau(\seqs_h,\seqs_h')  &:= \max\left\{\distsxtau(\seqs_h,\seqs_h'),\distsutau(\seqs_h,\seqs_h')\right\},
\end{align}
For $h = 1$, $\seqs_1 = \bbarx _1$ and $\seqs_1' = \bbarx _1'$, we define $\dists(\seqs_1,\seqs_1') = \|\bbarx _1 - \bbarx _1'\|$. Note therefore that 
\begin{align}\diststau[\tauc](\cdot,\cdot) = \dists(\cdot,\cdot), \quad \diststau[\taum-1](\cdot,\cdot) = \disttvc(\cdot,\cdot), \quad \diststau[0](\cdot,\cdot) = \distips(\cdot,\cdot) 
\end{align}
\end{definition}
Next, we introduce five problem-dependent constants $c_1,\dots,c_5$, all of which are stated in terms of the constants in \Cref{sec:comp_prob_consts}; one can readily check that these are all polynomial in the constants $(\Rdyn,\Ldyn,\Mdyn)$ and $(\Rstab,\Bstab,\Lstab)$ in \Cref{asm:Jacobian_Stable,asm:traj_regular}.
\begin{definition}[IPS Constants]\label{defn:IPS_consts} In terms of constants in \Cref{sec:comp_prob_consts}, we define the IPS constants as follows:
\begin{align}
c_1 &:= (6\max\{\Rstab(1+2\constu\sqrt{\Lstab}),\Bstab + \sqrt{\Lstab}\constk\}). \label{eq:c_one}\\
c_2 &:= \min\left\{\frac{\Cstabnum{1}(2\Rstab)}{4\Rstab\sqrt{\Lstab}},\frac{\Cstabnum{2}(2\Rstab)}{\sqrt{\Lstab}},  \frac{\Cstabnum{3}(2\Rstab)}{4\Rstab},\frac{1}{2\Rstab}\right\}. \label{eq:ctwo}
\end{align}
We further define  
\begin{align}
c_3 = 3\Lstab\log(2\constdelx), \quad c_4 = \min\{1,\Cstabnum{4}(2\Rstab)\}, \quad c_5 = 2(1+\Rstab)\Bstab. 
\end{align}
\end{definition}
In terms of the constants $c_1,c_2 > 0$ above, we introduce a family of distance-like functions on $\distAtau(\seqa,\seqa' \mid r):\cA \times \cA \to \R_{\ge 0} \cup \{\infty\}$, defined as follows.
\begin{definition}\label{defn:distAtau} Consider 
$\seqa = (\bbaru_{1:\tauc},\bbarx_{1:\tauc},\bbarK_{1:\tauc})$ and $\seqa' = (\bbaru_{1:\tauc}',\bbarx_{1:\tauc}',\bbarK_{1:\tauc}')$. \begin{align}
\distAtau(\seqa,\seqa' \mid r) &:= c_1\max_{1\le k \le \tauc}\left(\|\bbaru_{k}-\bbaru_{k}'\| + \|\bbarx_{k}-\bbarx_{k}'\| + re^{-\frac{\eta(\tauc-\tau)}{3\Lstab}}\|\bbarK_{k}-\bbarK_{k}'\| \right) \\
&\quad+\I_{0,\infty}\left\{\max_{1\le k \le \tauc}\left(\max\left\{\|\bbaru_{k}-\bbaru_{k}'\|,\|\bbarx_{k}-\bbarx_{k}'\|, \|\bbarK_{k}-\bbarK_{k}'\|\right\}\right) \le c_2\right\},
\end{align}
where $\I_{0,\infty}(\cE)$ is $0$ if clase $\cE$ is true and $\infty$ otherwise.
\end{definition}
In words, $\distAtau(\seqa,\seqa' \mid r) $ measures the maximal differences between $\bbaru_k-\bbaru_k'$, $\bbarx_k - \bbarx_k'$, and $\bbarK_k - \bbarK_k'$, subject to a constraint that each of these quantities is within some bound $c_2$. One the latter threshold is met, the dependence on $ \|\bbarK_{k}-\bbarK_{k}'\|$ is scaled down by $r$, and is also exponentially small in $\tauc - \tau$; this latter bit is not necessary for our results, but illustrates an interesting feature of our stability guarantees: \emph{ they are far less sensitive to errors in $\bbarK$ than to errors in $\bbaru$.}

In terms of $\distAtau(\seqa,\seqa' \mid r) $ defined above, we can now ensure the following stability guarantee.
\begin{corollary}\label{cor:cor_stability_guarantee_all} 
Suppose that $\tauc \ge c_3/\eta$  and $r \le c_4$, and consider any sequence of kernels $\{\lawW_h\}_{h=1}^h$, where $\lawW_h:\cS \to \laws(\cS)$, and\footnote{See \Cref{rem:eq:Ws_cond_stab} above for iterpretation of this condition below.}
\begin{align}
\max_{h,\seqs \in \cS}\Pr_{\tilde \seqs\sim \lawW_h(\seqs)}[\distips(\tilde \seqs,\seqs) \le r] = 1, \phiZ \circ \lawW_h(\seqs) \ll \phiZ \circ \Psth,
\end{align}
for $\phiZ$ is from the direct decomposition instantiated in \Cref{defn:smoothing_instantiation}, and where $\Psth$ denotes the law of $\pathc$ under $\Dexp$ as in \Cref{def:Dexp_policies}.


Define a process $\seqs_1 \sim \Dinit$, $\tilde\seqs_h \sim \lawW_h(\seqs_h),\seqa_h \sim \pi^\star_h(\tilde \seqs_h)$, and $\seqs_{h+1} := F_h(\seqs_h,\seqa_h)$. Then, almost surely, the following hold for all $0 \le \tau \le \tauc$:
\begin{itemize}
	\item For each $1 \le h \le H$, $\diststau(F_h(\tilde \seqs_h,\seqa_h),\seqs_h) \le c_5 re^{-\frac{\eta(\tauc-\tau)}{\Lstab}}$.
	\item For any sequence $(\seqa_{1:H}')$, the dynamics $\seqs_1' = \seqs_1$, $\seqs_{h+1} = F_h(\seqs_h',\seqa_h')$ satisfy 
	\begin{align}
	\max_{1\le h \le H+1}\diststau(\seqs_h,\seqs_h') \le \max_{1\le h \le H}\distAtau(\seqa_h,\seqa_h' \mid r).
	\end{align}
\end{itemize}
\end{corollary}
\iftoggle{workshop}{}{\begin{remark}[Additional Assumption]
	In the body, we stated \Cref{prop:ips_instant} without the additional assumption of absolute continuity of the kernels.  In the subsequent revision, we will update the proposition to include this condition, which we note is satisfied by both deconvolution and replica kernels.  Thus, there is no loss of generality in our final result.
\end{remark}}
\Cref{cor:cor_stability_guarantee_all} is derived in \Cref{sec:cor:cor_stability_guarantee_all} from an even more granular result stated just below. Before continuing, we explain how  \Cref{prop:ips_instant} follows. 
\begin{proof}[Proof of \Cref{prop:ips_instant}] This follows directly from the above corollary notice that $c_4$ is define to be at most $1$, so we always invoke the corollary with $r \le 1$, and thus $\distAtau(\seqa_h,\seqa_h' \mid r) \le \distAtau(\seqa_h,\seqa_h' \mid 1) \le \distA$. We remark that the guarantee only applies to kernels for which 
\end{proof}


\subsubsection{A more granular stability statement}\label{sssec:granular_stab}
Here, we state an even more granular stability guarantee. The notation is rather onerous, but captures another nice feature of our bound: that our stability depends not on the maximal errors over $\bbaru_k-\bbaru_k'$, $\bbarx_k-\bbarx_k'$, $\bbarx_k-\bbarx_k'$, but rather on $\ell_2$-errors. Again, not necessary for our guarantees, but it speaks to the sharpness of our perturbation bounds. See \Cref{rem:stability_scaling} at the end of the section for more discussion.






\newcommand{\distAu}{\dist_{\cA,\bu,\ltwo}}
\newcommand{\distAuinf}{\dist_{\cA,\bu,\infty}}
\newcommand{\distAxinf}{\dist_{\cA,\bx,\infty}}
\newcommand{\distAkinf}{\dist_{\cA,\bK,\infty}}

\newcommand{\distAx}{\dist_{\cA,\bx,\ltwo}}
\newcommand{\distAk}{\dist_{\cA,\bK,\ltwo}}
\newcommand{\radK}{\mathsf{rad}_{\bK}}
\begin{definition}[Action Differences (inputs and gains)]\label{defn:act_diffs} Consider $\seqa = (\bbaru_{1:\tauc},\bbarx_{1:\tauc},\bbarK_{1:\tauc})$ and $\seqa' = (\bbaru_{1:\tauc}',\bbarx_{1:\tauc}',\bbarK_{1:\tauc}')$. We define 
\begin{align}
\distAu(\seqa,\seqa') &= \max_{1\le k \le \tauc}\left(\eta\sum_{j=1}^k \betastab^{k-j}\|\bbaru_j - \bbaru_j'\|^2\right)^{1/2} \\
\distAx(\seqa,\seqa') &= \max_{1\le k \le \tauc}\left(\eta\sum_{j=1}^k \betastab^{k-j}\|\bbarx_j - \bbarx_j'\|^2\right)^{1/2} \\
\distAk(\seqa,\seqa') &= \max_{1\le k \le \tauc}\left(\eta\sum_{j=1}^k \betastab^{k-j}\|\bbarK_j - \bbarK_j'\|\right)^{1/2} .
\end{align}
We further define
\begin{align}
\distAuinf(\seqa,\seqa') &:=\max_{1\le k \le \tauc}\|\bbaru_k-\bbaru_k'\| = \|\bbaru_{1:\tauc}-\bbaru_{1:\tauc}'\|_{\maxtwo}\\
\distAxinf(\seqa,\seqa') &:=\max_{1\le k \le \tauc}\|\bbarx_k-\bbarx_k'\| = \|\bbarx_{1:\tauc}-\bbarx_{1:\tauc}'\|_{\maxtwo}\\
\distAkinf(\seqa,\seqa') &:=\max_{1\le k \le \tauc}\|\bbarK_k-\bbarK_k'\| = \|\bbarK_{1:\tauc}-\bbarK_{1:\tauc}'\|_{\maxop}.
\end{align}
and 
\begin{align}
\radK(\seqa) &:=\max_{1\le k \le \tauc}\|\bbarK_k\| = \|\bbarK_{1:\tauc}\|_{\maxop}.
\end{align}
\end{definition}
We note further that  as $\betastab \in (0,1)$ and $\eta\sum_{i \ge 0}\betastab^i = \Lstab$, we have
\begin{align}
\distAu(\seqa,\seqa') &\le \|\bbaru_{1:\tauc}-\bbaru_{1:\tauc}'\|_{\ltwo} \wedge \sqrt{\Lstab}\|\bbaru_{1:\tauc}-\bbaru_{1:\tauc}'\|_{\maxtwo}\\
&\le \|\bbaru_{1:\tauc}-\bbaru_{1:\tauc}'\|_{\ltwo} \wedge \sqrt{\Lstab}\distAuinf(\seqa,\seqa') \label{eq:ltwo_to_linf}
\end{align}
and analogously for $\distAx(\seqa,\seqa') $ and $\distAk(\seqa,\seqa') $. 
\newcommand{\Eclosenum}[1]{\cE_{\mathrm{close},\Rnot,#1}}

Next, recall the constants $\{\Cstabnum{i}(R_0)\}_{i=1}^4$  in \Cref{defn:constants_for_stability_stuff}, and $\constu,\constk$ in \Cref{defn:stability_setup}, all of which are polynomial in relevant problem parameters $(\Rdyn,\Ldyn,\Mdyn)$, $(\Rstab,\Bstab,\Lstab)$, and argument $R_0$. We now define a very general distance-like function between actions.
\newcommand{\distarnotu}{\dist_{\cA,R_0,\tau,\bu}}
\newcommand{\distarnot}{\dist_{\cA,R_0,\tau}}

\begin{definition}[Action Divergences]\label{defn:distarnottau}
Define, for $R_0 \ge 1$, the following
\begin{align}
\distarnot(\seqa_h,\seqa_h' \mid r) &:= 2((1+R_0)\distarnotx(\seqa_h,\seqa_h' \mid r) + \distarnotu(\seqa_h,\seqa_h' \mid r)),
\end{align}
where
\begin{align}
\distarnotu(\seqa_h,\seqa_h' \mid r) &:= \distAuinf(\seqa_h,\seqa_h') + R_0\distAxinf(\seqa_h,\seqa_h') + 2r\Bstab \betastab^{\tauc-\tau}\distAkinf(\seqa_h,\seqa_h'),
\end{align}
and where
\begin{align}
\distarnotx(\seqa,\seqa' \mid r) &= 2\constu(\distAu(\seqa,\seqa')+\Rnot \distAx(\seqa,\seqa')) + 2\constk r(\betastab)^{\frac{\tauc-\tau}{3}}\cdot\distAk(\seqa,\seqa') \\
&+ \I_{0,\infty}\{\bigcap_{i=1}^3 \Eclosenum{i}\} + \I_{0,\infty}\{\radK(\seqa)\vee \radK(\seqa') \le R_0\},
\end{align}
where $\I_{0,\infty}\{\cE\}$ denotes $0$ if clause $\cE$ is true, and $\infty$ otherwise, and where we define the clauses 
\begin{align}
\Eclosenum{1}(\seqa,\seqa') &= \{(\distAu(\seqa,\seqa')+\Rnot \distAx(\seqa,\seqa') \le \Cstabnum{1}(R_0)\}\\
\Eclosenum{2}(\seqa,\seqa') &= \{\distAk(\seqa,\seqa') \le \Cstabnum{2}(R_0)\}\\
\Eclosenum{3}(\seqa,\seqa') &= \{\distAuinf(\seqa,\seqa') + R_0 \distAxinf(\seqa,\seqa')\le \Cstabnum{3}(R_0)\}.
\end{align}
\end{definition}
Again, we see that asside from the $\I_{0,\infty}\{\cdot\}$ terms, our distances $\distarnotx(\seqa,\seqa' \mid r)$ depends only on $\ell_2$-guarantees.

We may now state our most general stability guarantee.
\begin{proposition}[Main Stability Guarantees]\label{prop:main_stability_guarantee_all} Suppose that 
\begin{align}\tauc \ge 3\Lstab\log(2\constdelx)/\eta.
\end{align} In addition, fix an $R_0 > 0$, and $r_{\max}$ such that $\rmax \le \Cstabnum{4}(R_0)$. 
Consider any sequence of kernels $\{\lawW_h\}_{h=1}^h$, where $\lawW_h:\cS \to \laws(\cS)$ and\footnote{Again, we refer to \Cref{rem:eq:Ws_cond_stab} for explanation of the second condition in the display \eqref{eq:Ws_cond_stab_two}}
\begin{align}
\max_{h,\seqs \in \cS}\Pr_{\tilde \seqs\sim \lawW_h(\seqs)}[\distips(\tilde \seqs,\seqs) \le r] = 1, \quad\phiZ \circ \lawW_h(\seqs) \ll \phiZ \circ \Psth,\label{eq:Ws_cond_stab_two}
\end{align}
 and define a process $\seqs_1 \sim \Dinit$, $\tilde\seqs_h \sim \lawW_h(\seqs_h),\seqa_h \sim \pi^\star_h(\tilde \seqs_h)$, and $\seqs_{h+1} := F_h(\seqs_h,\seqa_h)$. Then, almost surely, the following hold for all $0 \le \tau \le \tauc$:
\begin{itemize}
	\item For each $1 \le h \le H$, $\diststau(F_h(\tilde \seqs_h,\seqa_h),\seqs_h) \le 2(1+\Rstab)\Bstab r  \betastab^{(\tauc-\tau)}$.
	\item For any sequence $(\seqa_{1:H}')$, the dynamics $\seqs_1' = \seqs_1$, $\seqs_{h+1} = F_h(\seqs_h',\seqa_h')$ satisfy 
	\begin{align}
	\max_{1\le h \le H+1}\distsxtau(\seqs_h,\seqs_h') \le \max_{1\le h \le H}\distarnotx(\seqa_h,\seqa_h' \mid r).
	\end{align}
	and
	\begin{align}
	\max_{1\le h \le H+1}\diststau(\seqs_h,\seqs_h') \le \max_{1\le h \le H}\distarnot(\seqa_h,\seqa_h' \mid r).
	\end{align}
\end{itemize}
\end{proposition}
The above proposition is proven in 
\Cref{sec:prop:main_stability_guarantee_all}, wher it is derived from two key guarantees given in \Cref{sec:stab_of_trajectories} below. 

\begin{remark}[Remark on the Scaling]\label{rem:stability_scaling} We now justify the extreme granularity of the above result. We demonstrate that our guarantees satisfy the following favorable properties:
\begin{itemize}
	\item As in \Cref{cor:cor_stability_guarantee_all}, the dependence of $\bbarK_k - \bbarK_k'$ in the non $\I_{0,\infty}\{\cdot\}$ scales down with $r$ and with $r \cdot \betastab^{(\tauc - \tau)/3)}$, so that errors in $\bbarK_k$ become less relevant as $\tau \to \tauc$ and as $r \to 0$.
	\item If we restrict our attention only to errors in states, captured by $\distsxtau$, the non-$\I_{0,\infty}\{\cdot\}$ terms depend only on $\ell_2$-errors rather than maximal $\infty$-norm ones.
	\item In the special case where $\Rdyn = \infty$, i.e., the regularity properties in \Cref{asm:Jacobian_Stable} hold globally, then all terms $\Cstabnum{i}(R_0)$ defined in \Cref{defn:constants_for_stability_stuff} no longer need depend on $R_0$, as the terms in which $R_0$ appears have an $\Rdyn = \infty$ in the numerator, and each $\Cstabnum{i}(R_0)$ serves as an upper bound on a certain quantity of interest. Hence, we can drop the dependence on $R_0$ in all of these terms. $\Cstabnum{i}(R_0)$ (
	\item In particular, the term $\Cstabnum{3}(R_0)$ if equal to $\infty$ when $\Rdyn = \infty$. Thus, for $\Rdyn = \infty$, we can drop the indictor of $\Eclosenum{3}(\seqa,\seqa') := \{\distAuinf(\seqa,\seqa') + R_0 \distAxinf(\seqa,\seqa')\le \Cstabnum{3}(R_0)\}.$, and hence each $\distsxtau$ has depends only on $\ell_2$-type errors.
\end{itemize}
\end{remark}
The proof of
\Cref{prop:main_stability_guarantee_all} is given in \Cref{sec:prop:main_stability_guarantee_all}, derived from the results in the subsection directly below. Before we do this, we first demonstrate how \Cref{cor:cor_stability_guarantee_all} follows from \Cref{prop:main_stability_guarantee_all}. 
\subsubsection{Deriving  \Cref{cor:cor_stability_guarantee_all} from \Cref{prop:main_stability_guarantee_all}}\label{sec:cor:cor_stability_guarantee_all}
The proof is mostly notational bookkeeping. 


By assumption $\phiZ \circ \lawW_h(\seqs) \ll \phiZ \circ \Psth$ and \Cref{lem:pist_good_actions}, and the $\Rstab$-term in $(\Rstab,\Bstab,\Lstab)$-Jacobian stability, the action $\seqa_h$ with $\radK(\seqa) \le \Rstab$. Further, notice that the parameter $\betastab$ used throughout this section can be bounded by at most
\begin{align}
\betastab := (1 - \frac{\eta}{\Lstab}) \le \exp(-\eta/\Lstab).
\end{align}
Hence, \Cref{cor:cor_stability_guarantee_all} from 
\Cref{prop:main_stability_guarantee_all} as soon as we show that 
\begin{align}
\forall \seqa ~\text{ s.t.}~ \radK(\seqa) \le \Rstab, \quad \distarnot(\seqa_h,\seqa_h' \mid r)& \le 
\distAtau(\seqa,\seqa' \mid r).
\end{align}



Consider the action divergences in \Cref{defn:distarnottau}. Take $R_0 = 2\Rstab$, where $\Rstab \ge 1$ by assumption.  and upper bound $\distAu(\cdot,\cdot) \le \sqrt{\Lstab}\distAuinf(\cdot)$ (as in \eqref{eq:ltwo_to_linf}), and similarly for $\distAx(\cdot,\cdot)$ and $\distAk(\cdot,\cdot)$. .Then, 
\begin{align}
\distarnot(\seqa_h,\seqa_h' \mid r) &:= 2((1+R_0)\distarnotx(\seqa_h,\seqa_h' \mid r) + \distarnotu(\seqa_h,\seqa_h' \mid r))\\
&= \distAuinf(\seqa_h,\seqa_h') + R_0\distAxinf(\seqa_h,\seqa_h') + 2r\Bstab \betastab^{\tauc-\tau}\distAkinf(\seqa_h,\seqa_h')\\
&\quad + 2\constu(\distAu(\seqa,\seqa')+\Rnot \distAx(\seqa,\seqa')) + 2\constk r(\betastab)^{\frac{\tauc-\tau}{3}}\cdot\distAk(\seqa,\seqa') \\
&\qquad+ \I_{0,\infty}\{\bigcap_{i=1}^3 \Eclosenum{i}\} + \I_{0,\infty}\{\radK(\seqa)\vee \radK(\seqa') \le R_0\}\\
&\le 2\Rstab(1+2\constu\sqrt{\Lstab})(\distAuinf(\seqa_h,\seqa_h') + \distAxinf(\seqa_h,\seqa_h') \\
&\quad + (2\Bstab + 2\sqrt{\Lstab}\constk) r \exp^{-\frac{\eta(\tauc - \tau)}{3\Lstab}}\distAkinf(\seqa_h,\seqa_h') \\
&\qquad+ \I_{0,\infty}\{\bigcap_{i=1}^3 \Eclosenum{i}\} + \I_{0,\infty}\{\radK(\seqa)\vee \radK(\seqa') \le 2\Rstab\}\\
&\le \frac{c_1}{3}(\distAuinf(\seqa_h,\seqa_h') + \distAxinf(\seqa_h,\seqa_h') + r \exp^{-\frac{\eta(\tauc - \tau)}{3\Lstab}}\distAkinf(\seqa_h,\seqa_h'))\\
&\qquad+ \I_{0,\infty}\{\bigcap_{i=1}^3 \Eclosenum{i}\} + \I_{0,\infty}\{\radK(\seqa)\vee \radK(\seqa') \le 2\Rstab\},
\end{align}
where we recall from \eqref{eq:c_one}
\begin{align}
c_1 := 6\max\{\Rstab(1+2\constu\sqrt{\Lstab}),\Bstab + \sqrt{\Lstab}\constk\}. 
\end{align}
Let's now simplify the indictators. Restricting our attention to $\seqa$ with $\radK(\seqa) \le \Rstab$,  $\radK(\seqa') \le \Rstab + \distAkinf(\seqa_h,\seqa_h')$ by the triangle inequality. Thus, we can replace $\I_{0,\infty}\{\radK(\seqa)\vee \radK(\seqa') \le 2\Rstab\}$ with $\I_{0,\infty}\{ \distAkinf(\seqa_h,\seqa_h') \le \Rstab\}$. We now recall the definitions
\begin{align}
\Eclosenum{1}(\seqa,\seqa') &= \{(\distAu(\seqa,\seqa')+\Rnot \distAx(\seqa,\seqa') \le \Cstabnum{1}(R_0)\}\\
\Eclosenum{2}(\seqa,\seqa') &= \{\distAk(\seqa,\seqa') \le \Cstabnum{2}(R_0)\}\\
\Eclosenum{3}(\seqa,\seqa') &= \{\distAuinf(\seqa,\seqa') + R_0 \distAxinf(\seqa,\seqa')\le \Cstabnum{3}(R_0)\}.
\end{align}
Again, recall that we take $R_0 = 2\Rstab$. Again, invoke the upper bounds of the form $\distAu(\cdot,\cdot) \le \sqrt{\Lstab}\distAuinf(\cdot)$ (as in \eqref{eq:ltwo_to_linf}).Thus, $\bigcap_{i=1}^3 \Eclosenum{i}\cap \{\radK(\seqa)\vee \radK(\seqa') \le 2\Rstab\}$ holds as soon as 
\begin{align}
 \max\{\distAuinf(\seqa,\seqa'),\distAxinf(\seqa,\seqa')\distAkinf(\seqa,\seqa')\} \le c_2,
\end{align}
where we recall from
\begin{align}
c_2 := \min\left\{\frac{\Cstabnum{1}(2\Rstab)}{4\Rstab\sqrt{\Lstab}},\frac{\Cstabnum{2}(2\Rstab)}{\sqrt{\Lstab}},  \frac{\Cstabnum{3}(2\Rstab)}{4\Rstab},\frac{1}{2\Rstab}\right\}. 
\end{align}
In sum, for any $\seqa$ for which $\radK(\seqa) \le \Rstab$, we have
\begin{align}
\distarnot(\seqa_h,\seqa_h' \mid r) &\le \frac{c_1}{3}(\distAuinf(\seqa_h,\seqa_h') + \distAxinf(\seqa_h,\seqa_h') + r \exp^{-\frac{\eta(\tauc - \tau)}{3\Lstab}}\distAkinf(\seqa_h,\seqa_h'))\\
&\qquad+ \I_{0,\infty}\{\max\{\distAuinf(\seqa,\seqa'),\distAxinf(\seqa,\seqa')\distAkinf(\seqa,\seqa')\} \le c_2\}.
\end{align}
To conclude, we observe that, for  any nonnegative coefficients $a_1,a_2,a_3 > 0$ and sequences $v_{1,1:n},v_{2,1:n},v_{3,n} \ge 0$ in $\R^n$,
\begin{align}
 \sum_{i=1}^3a_i(\max_{j\in [n]}v_{i,j}) \le 3\max_{j\in [n]}\sum_{i=1}a_iv_{i,j}. 
\end{align}
Thus, if we express
$\seqa = (\bbaru_{1:\tauc},\bbarx_{1:\tauc},\bbarK_{1:\tauc})$ and $\seqa' = (\bbaru_{1:\tauc}',\bbarx_{1:\tauc}',\bbarK_{1:\tauc}')$, we can bound 
\begin{align}
\distarnot(\seqa_h,\seqa_h' \mid r)& \le 
\distAtau(\seqa,\seqa' \mid r) \\
&:= c_1\max_{1\le k \le \tauc}\left(\|\bbaru_{k}-\bbaru_{k}'\| + \|\bbarx_{k}-\bbarx_{k}'\| + re^{-\frac{\eta(\tauc-\tau)}{3\Lstab}}\|\bbarK_{k}-\bbarK_{k}'\| \right) \\
&\quad+\I_{0,\infty}\left\{\max_{1\le k \le \tauc}\left(\max\left\{\|\bbaru_{k}-\bbaru_{k}'\|,\|\bbarx_{k}-\bbarx_{k}'\|, \|\bbarK_{k}-\bbarK_{k}'\|\right\}\right) \le c_2\right\}.
\end{align}

\qed

\subsection{Stability guarantees for single control (sub-)trajectories.}\label{sec:stab_of_trajectories}
At the heart of the IPS guarantees in \Cref{sec:IPS_guarantees} above are two building blocks: one controller the perturbation of initial state around a regular (in the sense of \Cref{asm:traj_regular}) trajectory, and the second extending this guarantee to perturbations of control inputs and gains.
\begin{lemma}[Stability to State Perturbation]\label{lem:state_pert} Let $\ctrajbar   = (\bbarx _{1:K+1},\bbaru _{1:K}) \in \scrP_K$ be an $(\Rdyn,\Ldyn,\Mdyn)$-regular and feasible path, and let $\bK_{1:K}$ be gains such that $(\ctrajbar  ,\bK_{1:K})$ is $(\Rstab,\Bstab,\Lstab)$-stable. Assume, that $\Rstab \ge 1$, $\Lstab \ge 2\eta$. Fix  another $\bxoff_1$ and define another trajectory $\trajoff$ via 
\begin{align}\bu_k = \bbaru _k + \bK_k(\bxoff_k - \bbarx _k), \quad \bxoff_{k+1} = \bbarx _k + \eta \feta(\bxoff_k,\buoff_k)
\end{align}
Then,  if $\|\bxoff_1 - \bbarx _1\| \le \min\{(16\Lstab \Mdyn\Rstab^2\Bstab^2)^{-1},\frac{\Rdyn}{2\Rstab\Bstab}\}$, then 
\begin{itemize}
	\item $\|\bxoff_{k+1} - \bbarx _{k+1} \| \le 2\Bstab\|\bxoff_1-\bbarx _1\|  \betastab^{k}$.
	\item $(\trajoff,\bK_{1:K})$ is $(\Rstab,2\Bstab,\Lstab)$-stable. 
	\item $\|\bB_k(\trajoff)\| \le \Ldyn$.
\end{itemize} 
\end{lemma}
This lemma is proven in \Cref{sec:lem:state_pert}, and the following proposition in \Cref{sec:prop:master_stability_lem}. 
\begin{proposition}[Single Trajectory Stability Guarantee]\label{prop:master_stability_lem} Let $\ctrajbar   = (\bbarx _{1:K+1},\bbaru _{1:K}) \in \scrP_K$ be $(\Rdyn,\Ldyn,\Mdyn)$-regular and feasible, and let $\bK_{1:K}$ be such that $(\ctrajbar  ,\bK_{1:K})$ is $(\Rstab,\Bstab,\Lstab)$-stable. Assume $\Rstab \ge 1$, $\Lstab \ge 2\eta$,
and given another $\bxoff_1,\bxpr_1 \in \cX$, $\bbaru _{1:K}'$ and $\bK_{1:K}'$, define trajectoris $\ctraj = (\bx_{1:K+1},\bu_{1:K})$ and $\ctraj' = (\bx_{1:K+1}',\bu_{1:K}')$
\begin{align}
\bxoff_{k+1} &= \bxoff_k + \eta \feta(\bxoff_k,\buoff_k), \quad \buoff_k = \bbaru _k + \bK_k(\bxoff_k - \bbarx _k)\\
\bxpr_{k+1} &= \bxpr_k + \eta \feta(\bxpr_k,\bupr_k), \quad \bupr_k = \bbaru '_k + \bK_k'(\bxpr_k - \bbarx _k)
\end{align}
Let all constants be as defined in \Cref{defn:stability_setup}, and define (recalling the stability exponent $\betastab := (1-\frac{\eta}{\Lstab}) $) the terms
\begin{align}
	\Erru &:= \max_{k\in [K]} \left(\eta \sum_{j=1}^{k} \betastab^{k-j}\|\bbaru _j - \bbaru _j'\|^2\right)^{1/2}, \quad
	\ErrK := \max_{k\in [K]} \left(\eta \sum_{j=1}^{k} \betastab^{k-j}\|\bK_j - \bK_j'\|^2\right)^{1/2}
	\end{align}

 Then, the conclusions of \Cref{lem:state_pert} applies to the trajectory $\ctraj$, and moreover, for all $1\le k \le K$, 
\begin{align}
\|\bxoff_{k+1} - \bxpr_{k+1}\| \le \constu \Erru + \left(\constk\ErrK\|\bxoff_1 - \bbarx _1\| + \constdelx \|\bxoff_1 - \bxpr_1\|  \right)\betastab^{k/3},
\end{align}
provided that the following two conditions hold:
\begin{itemize} 
	\item The above error terms satisfy 
	\begin{align}
	\Erru &\le \Constu, \quad
	 \ErrK \le \Constk, \quad
	 \|\bxoff_1 - \bxpr_1\| \le \Constdelx, \quad \|\bxoff_1 - \bbarx _1\| \le \Constxhat, \quad \ErrK\|\bxoff_1 - \bbarx _1\| \le \Constxhatk 
	 \end{align}
	 \item In addition, if $\Rdyn < \infty$, $\Rstabtil:= \max\{1,\Rstab,\max_{1\le j \le K}\|\bK_k'\|\}$ and $\Deluinf:=\max_{j}\|\bbaru _j - \bbaru _j'\|$ satisfy
	 \begin{align}
	\Rdyn &\ge (4\Rstabtil\constu \sqrt{\Lstab} +1)\Deluinf + 4\Rstabtil \constk\|\bxoff_1 - \bbarx _1\|   + 4\Rstabtil\constdelx\|\bxoff_1 - \bxpr_1\|.
	\end{align}
\end{itemize}
\end{proposition}
The proofs of both this proposition and the lemma before it consist of translating the differences in trajectories into recursions satisfying certain functional forms. Taking norms, we obtain scalar recursions whose solutions are upper bounded in a series of technical lemmas detailed in \Cref{sec:recursion_solutions}. We believe these \Cref{prop:master_stability_lem,lem:state_pert} are useful more broadly in the study of perturbation of non-linear control systems. 

Notice that, for convenience, both the $\bx$ and $\bx'$ trajectories are stabilizing around the same $\bbarx$. This is for convenience, and simplifies the analysis. Indeed, difference generalizing to accomodate $\bx'$ stabilizing around $\bbarx'$ can be accomplished by a change of variables in the $\bbaru'$, which is precisely what is done in deriving \Cref{prop:main_stability_guarantee_all} in the section that follows. 



%(a) the sequence $(\seqs_{1:H+1},\seqa_{1:H})$ is input-stable w.r.t $(\dists,\dista)$ (b) $\max_{h \in [H]} \disttvc(F_h(\tilde\seqs_h,\seqa_h),\seqs_{h+1}) \le \gamipsone(r)$ and (c) $\max_{h \in [H]} \dists(F_h(\tilde\seqs_h,\seqa_h),\seqs_{h+1}) \le \gamipstwo(r)$. 

\subsection{Deriving \Cref{prop:main_stability_guarantee_all} from \Cref{prop:master_stability_lem}}\label{sec:prop:main_stability_guarantee_all}
The majority of this proof is (also) notational bookkeeping, whereby we convert two trajectories (in the abstract states/actions notation) into separate trajectories for each a sequence of $h=1,2,\dots,H = T/\tauc$, to each of which we apply \Cref{prop:master_stability_lem}.


\paragraph{Constructing the (perturbed) expert trajectory} We begin by unfolding the generative process for abstract-states $\seqs_1,\dots,\seqs_H$ in our proposition. Recall further that $\seqs_h = \pathc$ corresponds to the trajectory-chunk.

We let the (control) states and inputs for the corresponding sequence be denote as $(\bx_{1:T+1},\bu_{1:T})$ be generated as follows. Start with 
\begin{align}
\bx_1 \gets \seqs_1
\end{align}
drawn from the inital state distribution. Assume that we have constructed the states $\seqs_1,\dots,\seqs_{h-1}$; this meangs in particular that we have constructed $\bx_{1:t_h},\bu_{1:t_h-1}$, as well as the memory-chunks $\pathm[1],\dots,\pathm[j-1]$.  We extend the construction to step $h+1$ as follows:
\begin{itemize}
\item Define $\bx_{h,1} = \bx_{t_h}$
\item  Select a perturbation of the state  $\tilde \seqs_h = \pathctil = (\tilde{\bx}_{t_{h-1}:t_h},\tilde{\bu}_{t_{h-1}:t_h-1})$, with corresponding memory-chunk $\pathmtil = (\tilde{\bx}_{t_{h-\taum+1}:t_h},\tilde{\bu}_{t_{h}-\taum+1:t_h-1})$. As per the proposition, $\distips( \seqs_h,\tilde \seqs_h) \le r$. This means that $\|\bx_{t_h} - \tilde{\bx}_{t_h}\| \le r$.
\item Draw $\seqa_h = \sfk_{t_h:t_{h}:t_h+\taum-1} \sim \pist_h(\pathmtil)$.  We express
\begin{align}
\sfk_{t}(\bx) = \bbaru_t + \bbarK_t(\bx-\bbarx_t), \quad t_h \le t \le t_{h+1}-1,
\end{align}
and reindexed trajectory
\begin{align}
\sfk_{h,k} = \sfk_{t_h+k-1}.
\end{align}
Denote 
\begin{align}
\bbarx_{h,k} = \bbarx_{t_h+k-1}, \quad\bbaru_{h,k} = \bbaru_{t_h+k-1}, \quad \bbarK_{h,k} = \bbarK_{t_h+k-1}
\end{align}
and
\begin{align}
\ctrajbar_{[h+1]} = (\bbarx_{h,1:\tauc+1},\bbaru_{h,1:\tauc}).
\end{align}
\item  Moreover, because we assumg $\phiZ \circ \lawW_h \ll \phiZ \circ \Psth$, we inherbit the conclusions of \Cref{lem:pist_good_actions}. Hence, $\ctrajbar_{h+1}$ such be feasible, $(\Rdyn,\Ldyn,\Mdyn)$-regular, and $(\ctrajbar_h,\sfk_{h,1:\tauc})$ is be $(\Rstab,\Bstab,\Lstab)$-stable. In addition, \Cref{lem:pist_good_actions} ensures $\bbarx_{h,1} = \tilde \bx_{t_h}$.
%\begin{align}
%\bbaru _{h,k} = \sfk_{h,k}(\bbarx _{h,k}), \quad \bbarx _{h,1} = \tilde{\bx}_{t_h}, \quad \bbarx _{h,k+1} = f(\bbarx _{h,k},\bbaru _{h,k})
%\end{align}
%For $\sfk_{h,k}$ preserves this trajectory, this means that 
Consequently, we have that the composite action map $F_h$ satifies 
\begin{align}
 F_h(\tilde \seqs_h, \seqa_h) = \ctrajbar_{[h+1]} = (\bbarx_{h,1:\tauc+1},\bbaru_{h,1:\tauc}). \label{eq:next_til_state}
\end{align}
\item We execute $\seqa_h$ for $\tauc$ steps from our \emph{actual} state $\bx_t$ (not $\tilde{\bx}_t$), giving states and actions
\begin{align}
\bx_{t+1} = f(\bx_{t},\bu_{t}), \quad \bu_t =  \sfk_t(\bx_t), \quad \quad 1 \le t \le \tauc.
\end{align}
And define
\begin{align}
\bx_{h,k+1} = \bx_{t_h+k}, \quad \bu_{h,k} = \bu_{t_h+k-1}, \quad 1 \le k \le \tauc.
\end{align}
\item Finally, define the chunks the trajectories $\ctraj_{[h+1]} =(\bx_{h,1:\tauc+1},\bx_{h,1:\tauc}) $, which is equal to the next abstract-state
\begin{align}
\seqs_{h+1} = (\bx_{h,1:\tauc+1},\bu_{h,1:\tauc}) = (\bx_{t_h:t_{h+1}},\bu_{t_h:t_{h+1}-1})  \label{eq:next_real_state}
\end{align}
\end{itemize}

\paragraph{Construction of the imitation trajectory.} We now construct the imitation trajectory by setting $\bx_{1}' = \bx_1$, and 
\begin{itemize}
	\item For each $h$, select $\seqa_h' = (\sfk_{t_h:t_h+\tauc-1}') \in \cK^{\tauc}$. Define the re-indexed primitive controllers
	\begin{align}
	\sfk_{h,k}' = \sfk_{t_h+k-1}', 
	\end{align}
	and express 
	\begin{align}
	\sfk_{h,k}'(\bx) = \bbarK_{h,k}'(\bbarx - \bbarx_{k,h}') + \bbaru_{h,k}'.
	\end{align}
	\item Execute $\seqa_h'$ for $\tauc$ steps, giving states and actions
\begin{align}
\bx'_{t+1} = f(\bx'_{t},\bu'_{t}), \quad \bx_t =  \sfk_t'(\bx_t), \quad \quad 1 \le t \le \tauc.
\end{align}
And define
\begin{align}
\bxpr_{h,k+1} = \bxpr_{t_h+k}, \quad \bupr_{h,k} = \bupr_{t_h+k-1}, \quad 1 \le k \le \tauc.
\end{align}
\item  Finally, define the chunks
\begin{align}
\seqs_h' = (\bx_{t_h:t_{h+1}}',\bu_{t_h:t_{h+1}-1}') = (\bx_{h,1:\tauc+1}',\bu_{h,1:\tauc}').
\end{align}
\end{itemize}

\paragraph{Further Notation. }
Let's define the following errors analgous to \Cref{prop:master_stability_lem}.
\newcommand{\Erringuh}[1][h]{\Err_{\bbaru,#1}}
\newcommand{\ErrKh}[1][h]{\Err_{\bbarK,#1}}
\newcommand{\RstabtilH}{\tilde{R}_{\mathrm{stab},1:H}}
\newcommand{\Deluringinfh}[1][h]{\Delta_{\bbaru,\infty,#1}}


\begin{align}
\Erringuh^2 &=\max_{1\le k \le \tauc}\eta\sum_{j=1}^k \betastab^{k-j}\|\bbaru _{h,j} - \sfk_{h,j}'(\bbarx_{h,j})\|^2\\
\ErrKh^2 &= \max_{1\le k \le \tauc}\eta\sum_{j=1}^k \betastab^{k-j}\|\bbarK_{h,j} - \bbarK_{h,j}'\|^2\\
\Deluringinfh&:=\max_k\|\bbaru _{h,k} - \sfk_{h,j}'(\bbarx_{h,j})\|.
\end{align}
Importantly, in \Cref{prop:master_stability_lem}, it is assumeded that other the primed and unprimed sequence stabilize to the same $\bx_{h,k}$, whereas here, the primed sequence stabilized to $\bx_{h,k'}$. This is addressed by replacing the role of $\bu_{h,k}'$ with $\sfk_{h,k}'(\bbarx_{h,k})$.


\subsubsection{Interpreting the error terms.}
First, we unpack the above error terms.
\begin{lemma}\label{lem:ub_h_err_terms} Suppose $\max_h \distarnotx(\seqa_h,\seqa_h' \mid r)$ is finite. Then, 
\begin{align}
&\ErrKh = \distAk(\seqa_h,\seqa_h')\\ 
&\Erringuh \le \distAu(\seqa_h,\seqa_h') + R_0\distAx(\seqa_h,\seqa_h')\\
&\Deluringinfh =  \distAuinf(\seqa_h,\seqa_h')+ R_0\distAxinf(\seqa_h,\seqa_h')
\end{align}
\end{lemma}
\begin{proof}[Proof of \Cref{lem:ub_h_err_terms}]
 The equality of $\ErrKh$ follows from the reinxing $\bbarK_{h,k} = \bbarK_{t_h+k-1}$ and the definition of \Cref{defn:act_diffs}.Next, unpacking our notation of $\bbaru ,\bbaru '$, we compute
 \begin{align}
 \bbaru _{h,j} - \sfk_{h,j}(\bbarx _{h,j}) 
 &= \bbaru _{h,j} - \bbarK_{h,j}'(\bbarx '_{h,j}-\bbarx _{h,j})\\
 &= \bu_{t_{h}+k-1}' - \bu_{t_{h}+k-1}' - \bbarK_{t_{h}+k-1}'(\bx_{t_{h}+k-1} - \bx_{t_{h}+k-1}')
 \end{align}
 So that  as long as $\distarnotx(\seqa_h,\seqa_h' \mid r)$ for all $h$, then  $\|\bbarK_{h,j}\| \le \Rnot$. Thus 
 \begin{align}
 \|\bbaru _{h,j} - \bbaru _{h,j}' \| \le \|\bu_{t_{h}+k-1}' - \bu_{t_{h}+k-1}'\| + \Rnot\|\bx_{t_{h}+k-1} - \bx_{t_{h}+k-1}'\|
 \end{align}
 and thus by the triangle and moving the max outside the sum,
 \begin{align}
 \Erringuh &=\max_{1\le k \le \tauc} \left(\eta\sum_{j=1}^k \betastab^{k-j}\|\bbaru _{h,j} - \bbaru _{h,j}'\|^2\right)^{1/2}\\
 &\le\max_{1\le k \le \tauc} \left(\eta\sum_{j=1}^k \betastab^{k-j}\|\bu_{t_{h}+k-1}' - \bu_{t_{h}+k-1}'\|^2\right)^{1/2} \\
 &\quad+ R_0\max_{1\le k \le \tauc} \left(\eta\sum_{j=1}^k \betastab^{k-j}\|\bx_{t_{h}+k-1}' - \bx_{t_{h}+k-1}'\|^2\right)^{1/2}\\
 &\le \distAu(\seqa_h,\seqa_h') + R_0\distAx(\seqa_h,\seqa_h').
 \end{align}
  The inequality  $\Deluringinfh \le \distAuinf(\seqa_h,\seqa_h') + R_0\distAxinf(\seqa_h,\seqa_h')$ follows similarly.
\end{proof}
\subsubsection{An intermediate guarantee.}
Next, we establish an intermediate guarantee, from which \Cref{prop:main_stability_guarantee_all} is readily derived.
\begin{lemma}\label{lemma:master_ips_lem} Suppose $\max_h \distarnotx(\seqa_h,\seqa_h' \mid r)$ is finite, and further that $\tauc \ge 3\Lstab\log(2\constdelx)/\eta$.
Then,
\begin{itemize} 
	\item For all $k \in \{0,\dots,\tauc\}$ and $h \in [H]$,
 \begin{align}
 \|\bxoff_{h,k+1} - \bx_{h+1,k+1}'\|  \le  \max_{h'} \left(2\constu \Erringuh[h'] + 2r\constk\ErrKh[h']\betastab^{k/3}\right)
 \end{align},
 %\item In particular, for all  $h \in [H]$,
 % \begin{align}
 %\|\bxoff_{h,1} - \bx_{h,1}'\|  \le  \max_{h'}\left(
 %2\constu \Erringuh[h'] + 2r\constk\ErrKh[h']\betastab^{\tauc/3}\right).
 %\end{align}
 \item For all $h \in [H]$ and $1\le k \le \tauc$, 
 \begin{align}
 \|\bxoff_{h,k} - \bbarx _{h,k}\| \le 2\Bstab r  \betastab^{k-1}.
 \end{align}
\end{itemize}
\end{lemma}

\begin{proof}[Proof of \Cref{lemma:master_ips_lem}] 
First, an algebraic computation. Observe that $\log(1/\betastab) = \log(1/(1-\frac{\eta}{\Lstab})) = -\log(1-\frac{\eta}{\Lstab}) \ge \frac{\eta}{\Lstab}$. Hence, if $\tauc \ge 3\Lstab\log(2\constdelx)/\eta$,  we have $\tauc \ge 3\log(2\constdelx)/\log(1/\betastab)$, so that  
\begin{align}
\constdelx \betastab^{\tauc/3}\le 1/2. \label{eq:some_less_half_bound}
\end{align} 


We continue. Suppose $\max_h \distarnotx(\seqa_h,\seqa_h' \mid r)$ is finite. Then, from the definition of $\distarnotx$ in \Cref{defn:distarnottau}, the constants \Cref{defn:constants_for_stability_stuff}, and the inequalities in \Cref{lem:ub_h_err_terms} above, we can check that
\begin{align}
	&\max_h\Erringuh \le \max_h(\distAu(\seqa_h,\seqa_h') + R_0 \distAx(\seqa_h,\seqa_h') ) \le \Cstabnum{1}(R_0)\\
	&\max_h\ErrKh = \max_h\distAk(\seqa_h,\seqa_h')  \le \Cstabnum{2}(R_0)\\
	&\max_h\Deluringinfh = \max_h\distAuinf(\seqa_h,\seqa_h') \le \Cstabnum{3}(R_0)\\
	&r \le \Cstabnum{4}(R_0).
\end{align}


We begin with an induction on states $\|\bx_{h,1}-\bx_{h,1}'\|$ for $h \ge 1$. Recall the assumption that $\constdelx \betastab^{\tauc/3}\le 1/2$. We prove inductively that 
\begin{align}
\forall h \ge 1 , \|\bx_{h,1}'-\bx_{h,1}\| \le \max_{h'}\left(2\constu \Erringuh[h'] + 2r\constk\ErrKh[h'] \betastab^{\tauc/3})\right) \label{eq:my_induction_for_stability}
\end{align}
For the base case, we have $\bx_{1,1} = \bx_{1,1}'$. Now, suppose the result holds up to some $h \ge 1$. Using the definitions of various constants in \Cref{defn:constants_for_stability_stuff,defn:stability_setup}, and  $r \ge \|\bbarx _{h,1}-\bx_{h,1}\|$, as well as our inductive hypotheis, one can check that
\begin{align}
	&\Erringuh \le \Constu, \quad
	 \ErrKh \le \Constk\\
	  \quad
	 &\|\bx_{h,1}-\bx'_{h,1}\| \le \max_{h'}\left(2\constu \Erringuh[h'] + 2r\constk\ErrKh[h'] \betastab^{\tauc/3}\right) \le \Constdelx\\
	 &\|\bbarx _{h,1}-\bx_{h,1}\| \le \Constxhat, \quad \|\bbarx _{h,1}-\bx_{h,1}\| \ErrK \le \Constxhatk \\
	 &(4R_0\constu \sqrt{\Lstab} +1)\Deluinf + 4\Rnot \constk\|\bxoff_{h,1} - \bbarx _{h,1}\|   + 4\Rnot\constdelx\|\bxoff_{h,1} - \bxpr_{h,1}\|. \le \Rdyn.
	 \end{align}
	 Then, by \Cref{prop:master_stability_lem},
\begin{align}
\|\bxoff_{h+1,1} - \bx_{h+1,1}'\| &= \|\bxoff_{h,\tauc+1} - \bx_{h,\tauc+1}'\| \\
&\le \constu \Erringuh[h+1] + \left(\constk\ErrKh[h+1]\|\bxoff_1 - \bbarx _1\| + \constdelx \|\bxoff_{h,1} - \bx_{h,1}'\|  \right)\betastab^{\tauc/3}\\
&\le \constu \Erringuh[h+1] + \left(\constk\ErrKh[h+1] r + \constdelx \|\bxoff_{h,1} - \bx'_{h,1}\|  \right)\betastab^{\tauc/3} \tag{$\constdelx \betastab^{\tauc/3} \le \frac{1}{2}$, as established in \eqref{eq:some_less_half_bound}}\\
&\le \constu \Erringuh + r\constk\ErrKh\betastab^{\tauc/3} + \frac{1}{2}\max_{h'}\left(2\constu \Erringuh[h'] + 2r\constk\ErrKh[h'] \betastab^{\tauc/3})\right) \tag{inductive hypothesis}\\
&\le \max_{h'}\left(2\constu \Erringuh[h'] + 2r\constk\ErrKh[h'] \betastab^{\tauc/3})\right) 
\end{align}
  This establishes \eqref{eq:my_induction_for_stability}. A second invocation of \Cref{prop:master_stability_lem} gives
\begin{align}
&\|\bxoff_{h,k+1} - \bx_{h+1,k+1}'\| \\
&\le \constu \Erringuh[h+1] + \left(\constk\ErrKh[h+1]\|\bxoff_1 - \bbarx _1\| + \constdelx \|\bxoff_{h,1} - \bx'_{h,1}\|  \right)\betastab^{k/3}\\
&\le \constu \Erringuh[h+1] + \left(\constk\ErrKh[h+1] r + \constdelx \|\bxoff_{h,1} - \bx'_{h,1}\|  \right)\betastab^{k/3}\\
&\le \constu \Erringuh[h+1] + r\constk\ErrKh[h+1]\betastab^{k/3} + \frac{1}{2}\|\bxoff_{h,1} - \bx'_{h,1}\|\\
&\le \constu \Erringuh[h+1] + r\constk\ErrKh \betastab^{k/3} + \max_{h'}\left(\constu \Erringuh[h'] + r\constk\ErrKh[h'] \betastab^{\tauc/3})\right) \\
&\le \max_{h'}\left(2\constu \Erringuh[h'] + 2r\constk\ErrKh[h'] \betastab^{k/3})\right).
\end{align}
Moreover, as \Cref{prop:master_stability_lem} implies that the conclusions of \Cref{lem:state_pert} also hold, we further find that 
\begin{align}
 \|\bxoff_{h,k} - \bbarx _{h,k}\| \le  2\Bstab\|\bxoff_{h,k} - \bbarx _{h,k}\|   \betastab^{k-1} \le   2\Bstab r  \betastab^{k-1},
\end{align}
as needed.
\end{proof}
\subsubsection{Concluding the proof of \Cref{prop:main_stability_guarantee_all}.}
\begin{proof}[Completing the proof of \Cref{prop:main_stability_guarantee_all}.]
Let us start with the first item, bound $\distsxtau$. We may assume that $\distarnotx(\seqa_h,\seqa_h' \mid r)$ is finite for all $h$.

\paragraph{Controlling $\distsxtau(\seqs_h,\seqs_h')$.} Not that $\seqs_1 = \seqs_1'$. For any $2\le h \le H+1$,
\begin{align}{}
\distsxtau(\seqs_h,\seqs_h') &:=  \max_{t\in [t_{h}-\tau:t_h]}\|\bx_t-\bx_t'\|\\
&=\max_{\tauc - \tau \le k\le \tauc}\|\bxoff_{h-1,1+k} - \bx_{h-1,k+1}'\| \tag{our indexing scheme} \\
&\le  \max_{\tauc - \tau \le k\le \tauc} \max_{h'}\left(2\constu \Erringuh[h'] + 2r\constk\ErrKh[h']\betastab^{k/3}\right) \tag{\Cref{lemma:master_ips_lem}}\\
 &= \max_{h'} \left(2\constu \Erringuh[h'] + 2r\constk\ErrKh[h']\betastab^{(\tauc-\tau)/3}\right)\\
 &\le \max_{h'} \left(2\constu (\distAu(\seqa_{h'},\seqa_{h'}') + R_0\distAx(\seqa_{h'},\seqa_{h'}') ) + 2r\constk\distAk(\seqa_h,\seqa_h')\betastab^{(\tauc-\tau)/3}\right)\tag{\Cref{lem:ub_h_err_terms}}\\
 &\le \max_{h'}\distarnotx(\seqa_{h'},\seqa_{h'}' \mid r) \tag{\Cref{defn:distarnottau}} 
 \end{align}
 That is, 
 \begin{align}
 \distsxtau(\seqs_h,\seqs_h') \le \max_{h}\distarnotx(\seqa_{h},\seqa_{h}' \mid r) \label{eq:distarnottau_conclude}
 \end{align}

\paragraph{Bounding $\diststau$.} To bound $\diststau$, we also need to account for differents in inputs. We have
\begin{align}
&\bu_{h,k} - \bu_{h,k}' = \sfk_{h,k}(\bx_{h,k}) - \sfk_{h,k}'(\bx_{h,k}') \\
&= \bbaru_{h,k} - \bbaru_{h,k}' + \bbarK_t (\bx_{h,k} - \bbarx _{h,k}) - \bbarK_{h,k}' (\bx_{h,k}' -\bbarx_{h,k}')\\
&= \bbaru_{h,k} - \bbaru_{h,k}' + (\bbarK_{h,k} - \bbarK_{h,k}') (\bx_{h,k} - \bbarx_{h,k}) + \bbarK_{h,k}' (\bx_t - \bx_{h,k}'  -(\bbarx_{h,k} - \bbarx_{h,k}'))\\
\end{align}
Where $\distarnotx(\seqa_h,\seqa_h' \mid r)$ is finite for all $h$, then $\|\bbarK_{h,k}'\| = \|\bbarK_{t_h+k-1}\|\le R_0$. Thus, 
\begin{align}
&\|\bu_{h,k} - \bu_{h,k}'\| \\
&\le \|\bbaru_{h,k} - \bbaru_{h,k}'\| + R_0\|\bbarx_{h,k} - \bbarx_{h,k}'\| + R_0\|\bbarx_{h,k} - \bbarx_{h,k}'\| + \|\bbarK_{h,k} - \bbarK_{h,k}'\|\|\bx_{h,k} - \bbarx_{h,k}\|\\
&\le \distAuinf(\seqa_h,\seqa_h') + R_0\distAxinf(\seqa_h,\seqa_h') + R_0\|\bbarx_{h,k} - \bbarx_{h,k}'\| + \distAkinf(\seqa_h,\seqa') \|\bx_{h,k} - \bbarx_{h,k}\|\\
&\le \distAuinf(\seqa_h,\seqa_h') + R_0\distAxinf(\seqa_h,\seqa_h') + 2r\Bstab \betastab^{k-1}\distAkinf(\seqa_h,\seqa') + R_0\|\bbarx_{h,k} - \bbarx_{h,k}'\|  \label{eq:u_dist_inter}
\end{align}
Hence, 
\begin{align}
&\max_h\diststau(\seqs_h,\seqs_h') \\
&= \max_{h}\distsxtau(\seqa_h,\seqa_h') \vee \max_{h}\max_{\tauc-\tau\le k \le \tauc-1} \|\bu_{h,k+1} - \bu_{h,k+1}'\|\\
&= \max_{h}\distsxtau(\seqa_h,\seqa_h') + R_0\max_h \max_{\tauc-\tau\le k \le \tauc-1}\|\bbarx_{h,k+1} - \bbarx_{h,k+1}'\|\\
&\quad+ \max_h \max_{\tauc-\tau\le k \le \tauc-1}\left(\distAuinf(\seqa_h,\seqa_h') + R_0\distAxinf(\seqa_h,\seqa_h') + 2r\Bstab \betastab^{k}\distAkinf(\seqa_h,\seqa')\right) \\
&= (1+R_0)\max_h\distsxtau(\seqa_h,\seqa_h') \\
&\quad+ \max_h  \left(\underbrace{\distAuinf(\seqa_h,\seqa_h') + R_0\distAxinf(\seqa_h,\seqa_h') + 2r\Bstab \betastab^{\tauc-\tau}\distAkinf(\seqa_h,\seqa_h')}_{:= \max_h \distarnotu(\seqa_h,\seqa_h' \mid r)}\right)\\
&\le  (1+R_0)\max_h\distarnotx(\seqa_h,\seqa_h' \mid r) + \max_h \distarnotu(\seqa_h,\seqa_h' \mid r) \tag{ \eqref{eq:distarnottau_conclude}}\\
&\le  \max_h  2\left((1+R_0)\distarnotx(\seqa_h,\seqa_h' \mid r) + \distarnotu(\seqa_h,\seqa_h' \mid r)\right).
\end{align} 
\paragraph{Bounding $\diststau(\seqs_{h+1},F_h(\tilde\seqs_h,\seqa_h))$.}
 Next, by \eqref{eq:next_til_state} and \eqref{eq:next_real_state} that 
 \begin{align}
 \distsxtau(\seqs_{h+1},F_h(\tilde\seqs_h,\seqa_h)) &= \max_{\tauc - \tau \le k\le \tauc}\|\bxoff_{h,k+1} - \bbarx_{h,k+1}\| \tag{\eqref{eq:next_til_state} and \eqref{eq:next_real_state}}\\
 &= \max_{\tauc - \tau \le k\le \tauc}2\Bstab r  \betastab^{k} \tag{\Cref{lemma:master_ips_lem}}\\
 &=2\Bstab r  \betastab^{(\tauc-\tau)}.
 \end{align}
 Thus,
We have 
\begin{align}
\diststau(\seqs_{h+1},F_h(\tilde\seqs_h,\seqa_h)) &= \distsxtau(\seqs_{h+1},F_h(\tilde\seqs_h,\seqa_h)) \vee \max_{\tauc - \tau \le k\le \tauc-1}\|\bu_{h,k+1} - \bbaru_{h,k+1}\|\\
&= \distsxtau(\seqs_{h+1},F_h(\tilde\seqs_h,\seqa_h)) \vee \max_{\tauc - \tau \le k\le \tauc-1}\|\sfk_{h,k+1}(\bx_{h,k+1}) - \bbaru_{h,k+1})\|\\
&= \distsxtau(\seqs_{h+1},F_h(\tilde\seqs_h,\seqa_h)) \vee \max_{\tauc - \tau \le k\le \tauc-1}\|(\bbarK_{h,k+1}(\bx_{h,k+1}-\bbarx_{h,k+1})+\bbaru_{h,k+1})- \bbaru_{h,k+1})\|\\
&\le \distsxtau(\seqs_{h+1},F_h(\tilde\seqs_h,\seqa_h)) \vee \max_{\tauc - \tau \le k\le \tauc-1}\|\bbarK_{h,k+1}\|\|\bx_{h,k+1}-\bbarx_{h,k+1}\|\\
&\overset{(i)}{\le} \distsxtau(\seqs_{h+1},F_h(\tilde\seqs_h,\seqa_h)) \vee \max_{\tauc - \tau \le k\le \tauc-1}\Rstab\|\bx_{h,k+1}-\bbarx_{h,k+1}\|\\
&\le (1+\Rstab)\distsxtau(\seqs_{h+1},F_h(\tilde\seqs_h,\seqa_h))\\
&\le 2(1+\Rstab)\Bstab r  \betastab^{(\tauc-\tau)}
\end{align}
where in $(i)$, we used $\|\bbarK_{h,k+1}\| \le \Rstab$ because $(\ctrajbar_{[h+1]},\sfk_{h,1:\tauc})$ is $(\Rstab,\Bstab,\Lstab)$-stable, so that the gains are bounded in operator norm by $\Rstab$.
\end{proof}






\begin{comment}
\begin{align}
\gamma_0(\sfk, \sfkhat) &:= \I_{\infty}\{\|\bbarK_{1:K} - \bhatK_{1:K}\|_{\ltwoop} \le c_1\} + \I_{\infty}\{\|u_{1:K} - \hat u_{1:K}\|_{\ltwo} \le c_2\} + \I_{\infty}\{\max_{k} \|\bhatK_{k}\| \le c_3\}\\
&+ c_4(\|\{(\bar u_{i}^{\sfk} + \bbarK_k^{\sfk} \bar x_i^{\sfk}) - (\bar u_{i}^{\hat \sfk} + \bbarK_i^{\hat{\sfk}} \bar x_i^{\hat \sfk})\}_{1 \le i \le K}\|_{\ltwo}.
\end{align}
\begin{align}
\|x_{h+1,1} - \tilde{x}_{h+1,1}\| = \|x_{h,K+1}-\tilde x_{h,K+1}\| &\le  \gamma_0(\sfk, \sfkhat) + \Bstab\left(1-\frac{\step}{2\Lstab}\right)^{K}\|\tilde{x}_{h,1} - x_{h,1}\|.
\end{align}
\end{comment}

\subsection{Proof of \Cref{lem:state_pert} (state perturbation)}\label{sec:lem:state_pert}
Define $\Delbarxk = \bxoff_k - \bbarx _k$. Then
\begin{align}
\Delbarxk[k+1] &= \Delbarxk + \eta \left(\feta(\bxoff_k,\bbaru _k + \bbarK_k (\bxoff_k - \bbarx _k) - \feta(\bbarx _k,\bbaru _k)\right)\\
&= \Delbarxk + \eta (\bA_k + \bB_k \bK_k)\Delxk + \rem_k, \label{eq:recur}
\end{align}
where 
\begin{align}
\rem_k = \feta(\bxoff_k,\bbaru _k + \bK_k (\bxoff_k - \bbarx _k)) - \feta(\bbarx _k,\bbaru _k) - (\bA_k + \bB_k \bK_k)\Delbarxk.
\end{align}
\begin{claim}\label{claim:taylor_xhat} Take $\Rstab \ge 1$, and suppose $\|\Delbarxk\| \le \Rdyn/\Rstab$. Then, 
\begin{align}
\|\bbarx _k - \bxoff_k\| \vee \|\bbaru _k - \buoff_k\| \le \Rdyn, \label{eq:close_Rstab_within}
\end{align}
and $\|\rem_k\| \le \Mdyn \Rstab^2 \|\Delbarxk\|^2$. 
\end{claim}
\begin{proof}[Proof]  Let $\buoff_k = \bbaru _k + \bK_k(\bxoff_k - \bbarx _k)$. The conditions of the claim imply $\|\buoff_k - \bbaru _k\| \vee\|\bxoff_k \vee \bbarx _k\| \le \Rdyn$.  From Taylor's theorem and the fact that $\ctrajbar $ is $(\Rdyn,\Ldyn,\Mdyn)$-regular imply that 
\begin{align}
\|\feta(\bxoff_k,\buoff_k) - \feta(\bbarx _k,\bbaru _k)\| &\le \frac{1}{2}\Mdyn(\|\bxoff_k- \bbarx _k\|^2 + \|\buoff_k - \bbaru _k\|)\\
&\le \frac{1}{2}(1+\Rstab^2)\Mdyn\|\bxoff_k- \bbarx _k\|^2\le \Rstab^2 \Mdyn \|\Delbarxk\|^2,
\end{align}
where again use $\Rstab \ge 1$ above. 
\end{proof}
Solving the recursion from \eqref{eq:recur}, we have 
	\begin{align}
	\Delbarxk[k+1] = \eta \sum_{j=1}^{k} \Phicl{k+1,j+1} \rem_{k}+ \Phicl{k+1,1}\Delbarxk[1].
	\end{align}
	Set $\betastab := (1-\frac \eta \Lstab)$, so that $M := \frac{\eta}{\betastab^{-1}-1} = \Lstab$. Further, recall $R_0 \le \Rdyn/\Rstab$. By assumpion,   $\Phicl{k,j} \le \Bstab \betastab^{k-j}$, so using \Cref{claim:taylor_xhat} implies that, if  $\max_{j \in [k]}\|\Delbarxk[j]\| \le R_0 \le \Rdyn/\Rstab$ for all $j \in [k]$, 
	\begin{align}
	\|\Delbarxk[k+1]\| \le \eta \sum_{j=1}^{k} \Bstab\Mdyn\Rstab^2 \betastab^{k-j} \|\Delbarxk[j]\|^2 + \Bstab \betastab^{k} \|\Delbarxk[1]\|. 
	\end{align}
	Appling \Cref{lem:key_rec_one} with $\alpha = 0$, $C_1 = \Bstab\Mdyn\Rstab^2$, and $C_2 = \Bstab \ge 1$ and $M = \Lstab$ (noting $\betastab \ge 1/2$), it holds that for $\|\Delbarxk[1]\| = \epsilon_1 \le 1/4MC_1C_{3} = 1/4\Lstab \Mdyn\Rstab^2\Bstab^2$,
	\begin{align}
	\|\Delbarxk[k+1]\| \le 2\Bstab\|\Delbarxk[1]\|  (1 - \frac \eta \Lstab)^{k}. \label{eq:xhat_rec}
	\end{align}
	To ensure the inductive hypothesis that $\max_{j \in [k]} \|\Delbarxk[j]\| \le \Rdyn\Rstab$, it suffices to ensure that $2\Bstab\|\Delbarxk[1]\| \le R_0$, which is assumed by the lemma. Thus, we have shown that, if 
	\begin{align}\|\Delbarxk[1]\| \le \min\{1/2\Bstab R_0, 1/8\Lstab \Mdyn\Rstab^2\Bstab^2\},
	\end{align}
	it holds that $\|\Delbarxk[k+1]\| \le 2\Bstab\|\Delbarxk[1]\|  (1 - \frac \eta \Lstab)^{k} \le R_0$ for all $k$.

	Next, we adress the stability of the gains for the perturbed trajectory $\trajoff$. Using $(\Rdyn,\Ldyn,\Mdyn)$-regularity of $\ctrajbar  $ and \eqref{eq:close_Rstab_within},
	\begin{align}
	&\left\|\bA_k(\trajoff) + \bB_k(\trajoff)\bK_k - \bA_k( \ctrajbar  ) + \bB_k( \ctrajbar  )\bK_k\right\|\\
	&= \left\|\begin{bmatrix} 
	\bA_k(\trajoff) - \bA_k(\ctrajbar  ) & \hat \bB_k(\trajoff) - \bB_k(\ctrajbar  )
		\end{bmatrix} \begin{bmatrix} \eye \\ \bK_k \end{bmatrix} \right\|\\
		&= \left\|(\nabla \feta(\xhat_k,\buoff_k) - \nabla \feta(\bbarx _k,\bbaru _k))\begin{bmatrix} \eye \\ \bK_k \end{bmatrix} \right\|\\
		&\le \Mdyn \left\|(\bxoff_k - \bbarx _k, \bK_k(\bxoff_k - \bbarx _k)\right\|\left\|\begin{bmatrix} \eye \\ \bK_k \end{bmatrix} \right\|\\
		&= \Mdyn\|\bxoff_k - \bbarx _k\|\left\|\begin{bmatrix} \eye \\ \bK_k \end{bmatrix} \right\|^2 \le \Mdyn\|\bxoff_k - \bbarx _k\|(1+\|\bK_k\|_{\op}^2)\\
		&= \Mdyn\|\bxoff_k - \bbarx _k\|\left\|\begin{bmatrix} \eye \\ \bK_k \end{bmatrix} \right\|^2 \le \Mdyn\|\bxoff_k - \bbarx _k\|(1+\|\bK_k\|_{\op}^2)\\
		&\le 2\Rstab^2\Mdyn\|\bxoff_k - \bbarx _k\|\\
		&\le 4\Bstab\Rstab^2\Mdyn\|\bxoff_1 - \bbarx _1\|\betastab^{k-1}, \quad \betastab = (1 - \frac \eta \Lstab).
	\end{align}
	Invoking \Cref{lem:mat_prod_pert} with $\betastab \ge 1/2$, 
	$\|\Phiclhat{k,j}\| \le 2\Bstab\betastab^{k-j}$ for all $j,k$ provided that $4\Bstab\Rstab^2\Mdyn\|\bxoff_1 - \bbarx _1\| \le 1/4\Bstab\Lstab$, which requires $\|\bxoff_1 - \bbarx _1\| \le 1/16\Bstab^2\Rstab^2\Lstab\Mdyn$. 

	The last part of the lemma uses $(\Rdyn,\Ldyn,\Mdyn)$-regularity of $\ctrajbar  $ and \eqref{eq:close_Rstab_within}.

\subsection{Proof of \Cref{prop:master_stability_lem} (input and gain perturbation)}\label{sec:prop:master_stability_lem}
 Recall the trajectories $\bbarx _{k+1} = \bbarx _k + \eta \feta(\bbarx _k,\bbaru _k)$, and
\begin{align}
\bxoff_{k+1} &= \bxoff_k + \eta \feta(\bxoff_k,\buoff_k), \quad \buoff_k = \bbaru _k + \bK_k(\bxoff_k - \bbarx _k)\\
\bxpr_{k+1} &= \bxpr_k + \eta \feta(\bxpr_k,\bupr_k), \quad \bupr_k = \bbaru '_k + \bK_k'(\bxpr_k - \bbarx _k).
\end{align}
Further introduce the shorthand $\bhatAk= \bA_k(\ctrajat)$, $\bhatBk= \bB_k(\ctrajat)$, $\Aclhatk = \bhatAk + \bhatBk + \bK_k$, as well as
\begin{align}
\Delxk &= \bxpr_k - \bxoff_k, \quad \Deluk = \bbaru _k' - \bbaru _k, \quad \DelKk = \bK'_k - \bK\\
\Deltilxk &= \bxpr_k - \bbarx _k, \quad \Delbarxk = \bxoff_k - \bbarx _k,\\
\end{align}
Then,
 \begin{align}
	\Delxk[k+1] &= \Delxk + \eta \left(f(\bxoff_k + \Delxk,\bbaru _k + \Deluk + \matKtil_k \Deltilxk) - f(\bxoff_k,\bbaru _k + \bK_k\Delbarxk ) \right)\\
	%
	&= \Delxk + \eta \left(f(\bxoff_k + \Delxk,\bbaru _k + \Deluk + \matK_k \Deltilxk ) - f(\bxoff_k,\bbaru _k + \bK_k\Delbarxk) \right) \\
	&\quad + \eta (\rem_{k,1})\\
	&= \Delxk + \eta \left(\underbrace{\ddx f(\bxoff_k,\buoff_k)}_{=\bhatA_k} \Delxk + \underbrace{\ddu f(\bxoff_k,\buoff_k)}_{=\bhatB_k} (\Deluk + \matK_k \underbrace{\Deltilxk - \Delbarxk}_{\Delxk})\right) \\
	&\qquad + \eta (\rem_{k,1} + \rem_{k,2})\\
	&= \Delxk + \eta \left(\Aclhatk\Delxk + \bhatB_k \Deluk \right) + \eta (\rem_{k,1} + \rem_{k,2}).
	\end{align}
	where, above
	\begin{align}
	\rem_{k,1} &= \feta(\bxoff_k + \Delxk,\bbaru _k + \Deluk + \matK_k' \Deltilxk ) - \feta(\bxoff_k + \Delxk,\bbaru _k + \Deluk + \matK_k \Deltilxk )\\
	\rem_{k,2} &= \feta(\bxoff_k + \Delxk,\bbaru _k + \Deluk + \matK_k \Deltilxk ) -  f(\bxoff_k ,\bbaru _k + \matK_k \Delbarxk)\\
	&\qquad- \ddx \feta(\bxoff_k,\buoff_k) \Delxk + \ddu \feta(\bxoff_k,\buoff_k) (\Deluk + \matK_k (\Deltilxk - \Delbarxk)).
	\end{align}
	Solving the recursion,
	\begin{align}
	\Delxk[k+1] = \sum_{j=1}^{k} \Phiclhat{k+1,j+1}(\bhatB_j \Deluk[j] + \eta (\rem_{j,1}+\rem_{j,2})) + \Phiclhat{k+1,1}\Delxk[1] \label{eq:my_recur} %+ \Phicl{k+1,1}\Delxk[1].
	\end{align}
	Recall that \Cref{lem:state_pert} implies $(\bK_{1:K},\trajoff)$ is $(\Rstab,2\Bstab,\Lstab)$-stable. Thus, recalling $\betastab = (1 - \frac{\eta}{\Lstab}) \in [1/2,1)$, we have
	\begin{align}
	\|\Delxk[k+1]\| &\le \eta\sum_{j=1}^{k} \|\Phiclhat{k+1,j+1}\|(\|\bhatB_j\| \|\Deluk[j]\| +  \|\rem_{j,1}\|+\|\rem_{j,2}\|)) + \| \Phiclhat{k+1,1}\|\|\Delxk[1]\|\\
	&\le   \eta\sum_{j=1}^{k} 2\Bstab\betastab^{k-j}(\Ldyn \|\Deluk[j]\| +  \|\rem_{j,1}\|+\|\rem_{j,2}\|)) +2\Bstab \betastab^{k}\|\Delxk[1]\|.
	\label{eq:my_recur} %+ \Phicl{k+1,1}\Delxk[1].
	\end{align}
	Let us now bound each of these remainder terms. The following claim, as well as all subsequent claims, is proven at the end of the section. 

	\begin{claim}\label{lem:claim_rem_claim_hat} Suppose that it holds that for a given $k$, it holds that 
\begin{align}
\|\Delxk\| \le \constu \Erru + \constk  \ErrK\|\bxoff_1 - \bbarx _1\| +   \constdelx\|\bxoff_1 - \bxpr_1\|  \label{eq:xhat_xtil_recur_diff}
\end{align}
Then,
	\begin{align}
	\|\rem_{k,1}\| &\le \Ldyn \|\DelKk\|(\|\Delbarxk\| + \|\Delxk\|)\\
	 \|\rem_{k,2}\|&\le \frac{3}{2}\Mdyn \Rstab^2\|\Delxk\|^2 + \Mdyn\| \Deluk\|^2
	\end{align}
	\end{claim}
	We now proceed by strong induction on the condition in \eqref{eq:xhat_xtil_recur_diff}. Observe that if this condition holds for all $1 \le j \le k$, we have
	\begin{align}
	\|\Delxk[k+1]\| &\le   \eta\sum_{j=1}^{k} 2\Bstab\betastab^{k-j}(\Ldyn \|\Deluk[j]\| +  \|\rem_{j,1}\|+\|\rem_{j,2}\|)) +2\Bstab \betastab^{k}\|\Delxk[1]\|\\
	&\le \underbrace{\eta\sum_{j=1}^{k} 2\Bstab\betastab^{k-j}\left(\Ldyn \|\Deluk[j]\| +  \Mdyn\| \Deluk[j]\|^2\right)}_{=\Term_{1,k}} \\
	&\quad+ \eta\sum_{j=1}^{k} \betastab^{k-j}\left(\underbrace{3\Bstab\Mdyn}_{C_1}\| \Delxk[j]\|^2 + \underbrace{2\Bstab\Ldyn}_{C_2} \|\DelKk[j]\|\|\Delxk[j]\|\right)\\
	&\qquad + \underbrace{2\Bstab \betastab^{k}\|\Delxk[1]\| + \eta\sum_{j=1}^{k} 2\Ldyn\Bstab\betastab^{k-j}\|\DelKk[j]\|\|\Delbarxk[j]\|}_{\Term_{2,k}} \label{eq:last_display_hard_rec}
	\end{align}
	Define the terms 
	\begin{align}
	C_1 &:= 3\Bstab\Mdyn, \quad C_2 := 2\Bstab\Ldyn, \\
	\alpha &:= 2\Bstab\Erru \left(\Mdyn \Erru + \sqrt{\Lstab}\Ldyn\right)\\
	\bar \epsilon_1 &:= 2\Bstab\left(\|\Delxk[1]\| + 2 \Lstab^{1/2}\ErrK\|\Delbarxk[1]\| \right)
	\end{align}
	where above
	\begin{align}
	\Erru &:= \max_{k\in [K]} \left(\eta \sum_{j=1}^{k} \betastab^{k-j}\|\Deluk[j]\|^2\right)^{1/2}, \quad 
	\ErrK := \max_{k\in [K]} \left(\eta \sum_{j=1}^{k} \betastab^{k-j}\|\DelKk[j]\|^2\right)^{1/2}.
	\end{align} 
	We bound the two underlined terms in the above display.
	\begin{claim}\label{claim:termone} Recall $\Erru = \max_{k\in [K]} \left(\eta \sum_{j=1}^{k} \betastab^{k-j}\|\Deluk[j]\|^2\right)^{1/2}$. Then, for any $k$,
	\begin{align}
	\Term_{1,k} \le \alpha := 2\Bstab\Erru \left(\Mdyn \Erru + \sqrt{\Lstab}\Ldyn\right)
	\end{align}
	\end{claim}
	\begin{claim}\label{claim:termtwo} Assume $\betastab \in [1/2,1)$ and recall $\ErrK := \max_{k \in [K]}\left(\eta\sum_{j=1}^{k}\betastab^{j}\|\DelKk[j]\|^2\right)^{1/2}$. Then,
	\begin{align}
	\Term_2 \le \bar \epsilon_1 \betastab^{\frac{k}{2}}, \quad \bar \epsilon_1 := 2\Bstab\left(\|\Delxk[1]\| + 2 \Lstab^{1/2}\Ldyn\ErrK\|\Delbarxk[1]\| \right)
	\end{align}
	\end{claim}
	The previous two claims and \eqref{eq:last_display_hard_rec} show that as soon as \eqref{eq:xhat_xtil_recur_diff} holds for all indices $1 \le j \le k$,
	\begin{align}
	\|\Delxk[k+1]\|  \le \alpha + \bar{\epsilon}_1 \betastab^{k/2} + \eta \sum_{j=1}^k \betastab^{k-j}\left(C_1 \| \Delxk[j]\|^2 + C_2\|\DelKk[j]\|\|\Delxk[j]\|\right) \label{eq:DelxK_thing}
	\end{align}
	Set $\epsilon_j = \|\Delxk[j]\|$. 
	Note that $\bar{\epsilon}_1 \ge \epsilon_1$, $\betastab \in [1/2,1)$, we can apply \Cref{lem:key_rec_three} with $\delta_j \gets \|\DelKk[j]\|$ and $M \gets \frac{\eta}{1-\beta} = \Lstab$ to find that 
	\begin{align}
	\|\Delxk[k+1]\| &= \epsilon_{k+1} \le 3(\alpha + \bar{\epsilon}_1)\betastab^{k/3}\\
	\end{align}
	provided it holds that (we take$\Lstab \ge 1,\Bstab \ge 1$)
	\begin{align}
	  2\Bstab\Erru \left(\Mdyn \Erru + \sqrt{\Lstab}\Ldyn\right) &= \alpha \le \frac{1}{18 C_1 \Lstab} = \frac{1}{64\Bstab \Mdyn \Lstab}\\
	 2\Bstab\left(\|\Delxk[1]\| + 2 \Ldyn\Lstab^{1/2}\ErrK\|\Delbarxk[1]\| \right) &= \bar \epsilon_1 \le \frac{1}{108 C_1 \Lstab} = \frac{1}{324\Bstab \Mdyn \Lstab}\\
	 \ErrK &\le \frac{1}{12\sqrt{\Lstab}\max\{C_2,1\}} \le \frac{1}{24\sqrt{\Lstab}\Bstab\Ldyn}.
	 \end{align}
	 For these first two equation, it is enough that 

	 \begin{align}
	 \Erru &\le \min\left\{\frac{\sqrt{\Lstab}\Ldyn}{\Mdyn},\frac{1}{256\Bstab^2 \Mdyn \Ldyn \Lstab^{3/2}}\right\} \\
	 \ErrK &\le \frac{1}{24\sqrt{\Lstab}\Bstab\Ldyn}\\
	 \|\Delxk[1]\|& \le \frac{1}{4\cdot324\Bstab^2 \Mdyn \Lstab}\\
	 \ErrK\|\Delbarxk[1]\| &\le \frac{\Ldyn}{8\cdot324\Bstab^2 \Mdyn \Lstab^{3/2}}
	 \end{align}
	 for which $ \Erru \le \Constu$, $\|\Delxk[1]\| \le \Constdelx$,$\|\Delbarxk[1]\| \le \Constxhat$, $\ErrK \le \Constk$, $\ErrK\|\Delbarxk[1]\| \le \Constxhatk$.
	 Moreover, under the above condition on $\Erru$, we have
	 \begin{align}
	\|\Delxk[k+1]\| &\le 3(\alpha + \bar{\epsilon}_1)\betastab^{k/3}\\
	&\le 12\Bstab\sqrt{\Lstab}\Ldyn \Erru + 2\Bstab\left(\|\Delxk[1]\| + 2 \Lstab^{1/2}\Ldyn\ErrK\|\Delbarxk[1]\| \right)\betastab^{k/3}\\
	&\le 12\Bstab\sqrt{\Lstab}\Ldyn \Erru + 2\Bstab\left(\|\Delxk[1]\| + 2 \Lstab^{1/2}\Ldyn\ErrK\|\Delbarxk[1]\| \right)\betastab^{k/3}\\
	&\le \constu \Erru + \left(\constk\ErrK\|\Delbarxk[1]\| + \constdelx \|\Delxk[1]\|\right)\betastab^{k/3}.
	\end{align}
	This in turn shows that the inductive hypothesis \eqref{eq:xhat_xtil_recur_diff} holds, completing the induction. 
	\subsubsection{Deferred Claims}
	\begin{proof}[Proof of \Cref{lem:claim_rem_claim_hat}] 
	We argue in steps. Recall also $\Rstabtil$ be such that $\Rstabtil \ge \max_{k}\{\|\bK_k\|,\|\bK'_k\|,1\}$. 
	\paragraph{Ensuring within radius of regularity.} Our first step is to establish that the maximum of the following three terms is at most $\Rdyn$:
	\begin{align}
	&\|(\bxoff_k + \Delxk,\bbaru _k + \Deluk + \matK_k' \Deltilxk) - (\bbarx _k,\bbaru _k)\|  \\
	&\vee\|(\bxoff_k + \Delxk,\bbaru _k + \Deluk + \matK_k' \Deltilxk) - (\bbarx _k,\bbaru _k)\|  \\
	&\vee\|(\bxoff_k,\buoff_k) - (\bbarx _k,\bbaru _k)\| \le \Rdyn
	\end{align}
	First, we observe
	\begin{align}
	&\|(\bxoff_k + \Delxk,\bbaru _k + \Deluk + \matK_k' \Deltilxk) - (\bbarx _k,\bbaru _k)\| \\
	&\le \|(\bxoff_k + \Delxk,\bbaru _k + \Deluk + \matK_k' \Deltilxk) - (\bbarx _k,\bbaru _k)\| + \|\DelKk\|\|\Deltilxk\|\\
	&\le \|(\bxoff_k + \Delxk,\bbaru _k + \Deluk + \matK_k \Deltilxk) - (\bbarx _k,\bbaru _k)\| + \|\DelKk\|\|\Delxk\| + \|\DelKk\|\|\Delbarxk\|\\
	&\le \|(\bxoff_k + \Delxk,\bbaru _k + \Deluk + \matK_k \Delbarxk) -  (\bbarx _k,\bbaru _k)\| + \|\DelKk\|\|\Delxk\| + \|\DelKk\|\|\Delbarxk\| + \|\bK_k\|\|\Delxk\|\\
	&\le \|(\bxoff_k,\bbaru _k + \matK_k \Delbarxk) - (\bbarx _k,\bbaru _k)\| + (1+\|\DelKk\|)\|\Delxk\| + \|\DelKk\|\|\Delbarxk\| + \|\bK_k\|\|\Delxk\| + \|\Deluk\|\\
	&\le \|\Delbarxk\|(1+\|\bK\| ) + (1+\|\DelKk\|+\|\bK_k\|)\|\Delxk\| + \|\Delbarxk\| + \|\Deluk\|\\
	&\le (1+\|\bK_k\| + \|\DelKk\|)(\|\Delxk\| + \|\Delbarxk\|) + \|\Deluk\|\\
	%&\le (1+2\|\bK_k\| + \|\btilK_k\|)(\|\Delxk\| + \|\Delbarxk\|) + \|\Deluk\|\\
	&\le (2\Rstab + \max_{j}\|\bK_k - \bK_j'\|)(\|\Delxk\| + \|\Delbarxk\|) + \|\Deluk\|
	\end{align}
	Recall the notation
	\begin{align}
	\Delkinf:=\max_{j}\|\bK_j - \bK_j'\|, \quad \Deluinf:=\max_{j}\|\bbaru _j - \bbaru _j'\|.
	\end{align}
	Hence, it is enough that 
	\begin{align}
	(2\Rstab + \Delkinf)(\|\Delxk\| + \|\Delbarxk\|) + \Deluinf \le\Rdyn.
	\end{align}
	Thus, since $\|\Delxk\| \le \constu \Erru + (\constk  \ErrK-2\Bstab)\|\Delbarxk[1]\| +   \constdelx\|\Delxk[1]\|$ due to \eqref{eq:xhat_xtil_recur_diff} and $\|\Delbarxk\| \le 2\Bstab \|\Delbarxk[1]\|$ by \Cref{lem:state_pert}
	\begin{align} 
	\|\Delxk\| &\le \constu \Erru + \constk  \ErrK\|\Delbarxk[1]\| +   \constdelx\|\Delxk[1]\|\\
	\end{align}
	Hence, it is enough that 
	\begin{align}
	\Rdyn &\ge (2\Rstab + \Delkinf)(\constu \Erru + \constk  \ErrK\|\Delbarxk[1]\| +   \constdelx\|\Delxk[1]\|)) + \Deluinf,
	%&= (1+2\Rstab\Lstab\constu) \Deluinf +  \Delkinf (2\Rstab\constk\Lstab\|\Delbarxk[1]\| +\Lstab\constu\Deluinf)
	%( + \Delkinf)) + \constk \Lstab \Delkinf\|\Delbarxk[1]\| +   \constdelx\|\Delxk[1]\|)) + (2\Rstab + \Delkinf))\Deluinf
	\end{align}
	We can bound $2\Rstab + \Delkinf \le 4\Rstabtil$, and solving the geometric series, bound $\Erru \le \sqrt{\Lstab}\Deluinf$  and $\ErrK \le \sqrt{\Lstab}\Delkinf \le 2\sqrt{\Lstab}\Rstabtil$. Thus, it is enough that 
	\begin{align}
	\Rdyn &\ge (4\Rstabtil\constu \sqrt{\Lstab} +1)\Deluinf + 4\Rstabtil \constk\|\Delbarxk[1]\|   + 4\Rstabtil\constdelx\|\Delxk[1]\|.
	%&= (1+2\Rstab\Lstab\constu) \Deluinf +  \Delkinf (2\Rstab\constk\Lstab\|\Delbarxk[1]\| +\Lstab\constu\Deluinf)
	%( + \Delkinf)) + \constk \Lstab \Delkinf\|\Delbarxk[1]\| +   \constdelx\|\Delxk[1]\|)) + (2\Rstab + \Delkinf))\Deluinf
	\end{align}

	which is ensured by \Cref{prop:master_stability_lem}.



	 %holds under the assumption of the Proposition
	

	\paragraph{Controlling the first remainder.} Using that the relevant terms are within the radius of regularity,
	\begin{align}
	\|\rem_{k,1}\|&=\|\feta(\bxoff_k + \Delxk,\bbaru _k + \Deluk + \matK_k' \Deltilxk ) - \feta(\bxoff_k + \Delxk,\bbaru _k + \Deluk + \matK_k \Deltilxk )\| \\
	& \le \Ldyn \|(\matK_k' - \matK_k) \Deltilxk\| \\
	&\le \Ldyn \DelKk(\|\Delbarxk\| + \|\Delxk\|). 
	\end{align}
	\paragraph{Controlling the first remainder.} Using the definitions of $\bxoff_k = \bbarx _k + \Delbarxk$ $\buoff_t = \bbaru _k + \bK_k \Delbarxk$, and the fact that $(\bxoff_k,\buoff_k)$ is in the radius of regularity around $(\bbarx _k,\bbaru _k)$, a Taylor expansion implies
	\begin{align}
	\|\rem_{k,2}\| &= \left\|\begin{matrix}&\feta(\bxoff_k + \Delxk,\bbaru _k + \Deluk + \matK_k \Deltilxk ) -  f(\bxoff_k ,\bbaru _k + \matK_k \Delbarxk)\\
	& - \ddx \feta(\bxoff_k,\buoff_k) \Delxk + \ddu \feta(\bxoff_k,\buoff_k) (\Deluk + \matK_k (\Deltilxk - \Delbarxk))\end{matrix}\right\|\\
	&= \left\|\begin{matrix}&\feta(\bxoff_k + \Delxk,\bbaru _k + \Deluk + \matK_k \Deltilxk ) -  f(\bxoff_k ,\buoff_k)\\
	& - \ddx \feta(\bxoff_k,\buoff_k) \Delxk + \ddu \feta(\bxoff_k,\buoff_k) ( \bbaru _k + \Deluk + \matK_k \Deltilxk - \buoff_k)\end{matrix}\right\|\\
	&\le \frac{\Mdyn}{2}\left(\|\Delxk\|^2 + \| \bbaru _k + \Deluk + \matK_k \Deltilxk - \buoff_k\|\right)^2\\
	&= \frac{\Mdyn}{2}\left(\|\Delxk\|^2 + \| \Deluk + \matK_k (\Deltilxk - \Delbarxk)\|\right)^2\\
	&= \frac{\Mdyn}{2}\left(\|\Delxk\|^2 + \| \Deluk + \matK_k \Delxk\|\right)^2\\
	&= \frac{\Mdyn}{2}\left((1+2\|\matK_k\|^2)\|\Delxk\|^2 + 2\| \Deluk\|^2\right)^2\\
	&= \frac{3}{2}\Mdyn \Rstab^2\|\Delxk\|^2 + \Mdyn\| \Deluk\|^2
	\end{align}
	\end{proof}
	\begin{proof}[Proof of \Cref{claim:termone}] Recall $\Erru = \max_{k\in [K]} \left(\eta \sum_{j=1}^{k} \betastab^{k-j}\|\Deluk[j]\|^2\right)^{1/2}$ and $\betastab = 1 - \frac{\eta}{\Lstab}$. Then,
	\begin{align}
	\Term_{1,k} &= \eta\sum_{j=1}^{k} 2\Bstab\betastab^{k-j}\left(\Ldyn \|\Deluk[j]\| +  \Mdyn\| \Deluk[j]\|^2\right)\\
	&\le 2\Bstab\left(\Mdyn \Erru^2 + \Ldyn\cdot \eta\sum_{j=1}^{k} \betastab^{k-j} \|\Deluk[j]\| \right)\\
	&\le 2\Bstab\left(\Mdyn \Erru^2 + \Ldyn\cdot (\eta\sum_{j=1}^{k} \betastab^{k-j})^{1/2} (\eta\sum_{j=1}^{k} \betastab^{k-j}\|\Deluk[j]\|)^{1/2} \right)\\
	&\le 2\Bstab\left(\Mdyn \Erru^2 + \Ldyn\cdot (\eta\sum_{j=1}^{k} \betastab^{k-j})^{1/2} \Erru \right)\\
	&\le 2\Bstab\left(\Mdyn \Erru^2 + \Ldyn\cdot (\underbrace{\eta\frac{1}{\betastab^{-1}-1}}_{=\Lstab})^{1/2} \Erru \right)\\
	&= 2\Bstab\Erru \left(\Mdyn \Erru + \sqrt{\Lstab}\Ldyn\right)
	\end{align}
	\end{proof}
	
	\begin{proof}[Proof of \Cref{claim:termtwo}]
	\begin{align}
	\Ldyn^{-1}(\Term_{2,k} - 2\Bstab \betastab^{k}\|\Delxk[1]\|) &= \eta\sum_{j=1}^{k} 2\Bstab\betastab^{k-j}\|\DelKk[j]\|\|\Delbarxk[j]\|\\
	&= \eta\sum_{j=1}^{k} 2\Bstab\betastab^{k-j}\betastab^{j-1}\|\DelKk[j]\|\|\Delbarxk[1]\|\\
	&= 2\Bstab \|\Delbarxk[1]\| \cdot \eta\sum_{j=1}^{k} \betastab^{k-1}\|\DelKk[j]\|\\
	&= 2\Bstab \betastab^{\frac{k}{2}-1}\|\Delbarxk[1]\| \cdot \eta\sum_{j=1}^{k} \betastab^{(k-j)/2}\betastab^{j/2}\|\DelKk[j]\|\\
	&\le2\Bstab \betastab^{\frac{k}{2}-1}\|\Delbarxk[1]\| \le \left(\eta\sum_{j=1}^{k} \betastab^{j}\right)\left(\eta\sum_{j=1}^{k}\betastab^{j}\|\DelKk[j]\|^2\right)^{1/2}\\
	&\le2\Bstab \betastab^{\frac{k}{2}-1}\|\Delbarxk[1]\| \le \left(\eta\sum_{j=1}^{k} \betastab^{j}\right)^{1/2}\left(\eta\sum_{j=1}^{k}\betastab^{j}\|\DelKk[j]\|^2\right)^{1/2}\\
	&\le2\Bstab \Lstab^{1/2}\betastab^{\frac{k}{2}-1}\|\Delbarxk[1]\| \cdot \underbrace{\left(\eta\sum_{j=1}^{k}\betastab^{j}\|\DelKk[j]\|^2\right)^{1/2}}_{=\ErrK}\\
	&\le4\Bstab \Lstab^{1/2}\betastab^{\frac{k}{2}}\ErrK\|\Delbarxk[1]\| \tag{$\betastab \ge 1/2$}.
	\end{align}
	Thus,
	\begin{align}
	\Term_{2,k} &\le 4\Bstab \Ldyn\Lstab^{1/2}\betastab^{\frac{k}{2}}\ErrK\|\Delbarxk[1]\| + 2\Bstab \betastab^{k}\|\Delxk[1]\|\\
	&\le \betastab^{\frac{k}{2}}\left(4\Ldyn\Bstab \Lstab^{1/2}\ErrK\|\Delbarxk[1]\| + 2\Bstab\|\Delxk[1]\|\right)
	\end{align}
	\end{proof}
	





	%Let us show that $\|\bPhi_{k+1,1}'\|_{\op} \le 2c\sqrt{\beta}^k$ for $\Delta$ small enough. Indeed, this is true for $k = 1$. For $k > 1$, nduction implies
	%\begin{align}
	%\|\bPhi_{k+1,1}'\|_{\op} &\le  c \betastab^k + 2c \eta \Delta \sum_{j=1}^{k}\betastab^{k-j} \sqrt{\beta}^{(j-1)} \\
	%&=  c \betastab^k + 2c \eta \Delta\sum_{j=1}^{k} \sqrt{\beta}^{2k - 2j + (j-1)} \\
	%&= c \betastab^k + 2c \eta\Delta \sqrt{\beta}^k\sum_{j=1}^{k} \sqrt{\beta}^{k-1}\\
	%&= c \sqrt{\beta}^k( 1+  2\eta\Delta \sum_{j \ge 0}^{k} \sqrt{\beta}^{j})
	%\end{align}
	%Note that $\eta\sum_{j=1}^{k} \sqrt{\beta}^{k-1} \le 2\eta/(\betastab^{-1}-1) := 2M$. Thus, for $\Delta \le 1/4M$, the above is at most $2 c \sqrt{\beta}^k$, as needed.


\subsection{Ricatti synthesis of stabilizing gains.  }\label{sec:ric_synth}
In this section, we show that under a certain \emph{stabilizability} condition, it is always possible to synthesize primitive controllers satisfying \Cref{asm:Jacobian_Stable} with reasonable constants. We begin by defining our notion of stabilizability; we adopt the formulation based on Jacobian linearizations of non-linear systems 
 the discrete analogue of the senses proposed in 
which is consistent with \cite{pfrommer2023power,westenbroek2021stability}.
\begin{definition}[Stabilizability]\label{defn:stabilizable} A control trajectory $\ctraj = (\bx_{1:K+1},\bu_{1:K}) \in \scrP_{K}$ is $\Lfp$-Jacobian-Stabilizable if $\max_{k}\cV_{k}(\ctraj) \le \Lfp$, where for $k \in [K+1]$, $\cV_k(\ctraj)$ is defined by
\begin{align}
\cV_{k}(\ctraj) &:= \sup_{\xi:\|\xi \le 1}\left(\inf_{\tilde{\bu}_{1:s}} \|\tilde{\bx}_{K+1}\|^2 + \step \sum_{j=k}^{K} \|\tilde{\bx}_{j}\|^2 + \|\tilde{\bu}_{j}\|^2 \right)\\
&\text{s.t. } \tilde{\bx}_k = \xi, \quad \tilde{\bx}_{j+1} = \tilde{\bx}_j + \eta\left(\bA_j(\ctraj)\tilde{\bx}_j + \bB_j(\ctraj)\tilde{\bu}_j\right),
\end{align}
\end{definition}
Here, for simplicity, we use Euclidean-norm costs, though any Mahalanobis-norm cost induced by a positive definite matrix would suffice. We propose to synthesize gain matrices by performing a standard Ricatti update, normalized appropriately to take account of the step size $\eta > 0$ (see, e.g. Appendix F in \cite{pfrommer2023power}).
\begin{definition}[Ricatti update]\label{defn:ric_update} Given a path $\ctraj \in \Path_k$ with $\matA_k = \matA_k(\ctraj)$, $\matB_k = \matB_k(\ctraj)$ we define
\begin{align}
&\Pric_{K+1}(\ctraj) = \eye, \quad \Pric_{k}(\ctraj) = (\eye + \step\Aclk(\ctraj))^\top\Pric_{k+1}(\ctraj)(\eye + \step\Aclk(\ctraj)) + \step (\eye + \matK_k(\ctraj)\matK_k(\ctraj)^\top )\\
&\Kric_k(\ctraj) = (\eye + \step \matB_k^\top \Pric_{k+1}(\ctraj)\matB_k )^{-1}(\matB_k^\top \matP_{k+1}(\ctraj))(\eye + \eta \matA_k)\\
&\Aclkric(\ctraj) = \matA_k + \matB_k\matK_k(\ctraj).
\end{align}
\end{definition}
The main result of this section is that the parameters $(\Rstab,\Bstab,\Lstab)$ in \Cref{asm:Jacobian_Stable} can be bounded in terms of $\Ldyn$ in \Cref{asm:traj_regular}, and the bound $\Lfp$ defined above. 
\begin{proposition}[Instantiating the Lyapunov Lemma]\label{lem:instantiate_lyap} Let $\Ldyn,\Lfp \ge 1$, and let $\ctraj = (\bx_{1:K+1},\bu_{1:K})$ be $(\Rdyn,\Ldyn,\Mdyn)$-regular and $\Lfp$-Jacobian Stabilizable. Suppose further that $\step \le 1/5\Lf^2\Lfp$. Then, $(\ctraj,\Kric_{1:K})$-is $(\Rstab,\Bstab,\Lstab)$-Jacobian Stable, where 
\begin{align}
\Rstab = \frac{4}{3}\Lfp\Lf, \quad \Bstab = \sqrt{5}\Lf\Lfp, \quad \Lstab = 2\Lfp
\end{align}
\end{proposition}
\Cref{lem:instantiate_lyap} is proven in \Cref{sec:lem:instantiate_lyap} below. A consequence of the above proposition is that, given access to a smooth local model of dynamics, one can implement the synthesis oracle by computing linearizations around demonstrated trajectories, and solving the corresponding Ricatti equations as per the above discussions to synthesize the correct gains.

%\begin{corollary}\mscomment{explain synth oracle}
%\end{corollary}

\subsubsection{Proof of \Cref{lem:instantiate_lyap} (Ricatti synthesis of gains)}\label{sec:lem:instantiate_lyap}

	Throughout, we use the shorthand $\bA_k = \bA_k(\ctraj)$ and $\bB_k = \bB_k(\ctraj)$, recall that $\|\cdot\| $ denotes the operator norm when applied to matrices. We also recall our assumptions that $\Ldyn,\Lfp \ge 1$.
	We begin by translating our stabilizability assumption (\Cref{defn:stabilizable}) into the the $\bP$-matrices in \Cref{defn:ric_update}. The following statement recalls Lemma F.1 in \cite{pfrommer2023power}, an instantiation of well-known solutions to linear-quadratic dynamic programming (e.g. \cite{anderson2007optimal}).
	\begin{lemma}[Equivalence of stabilizability and Ricatti matrices]\label{lem:V_P_equiv} Consider a trajectory $(\bx_{1:K},\bu_{1:K})$, and define the parameter $\matTheta := (\Ajac(\bbarx _k,\bbaru _k),\Bjac(\bbarx _k,\bbaru _k))_{k \in [K]}$. Then, for all $k \in [K]$,
	\begin{align}
	\forall k \in [K], \quad \cV_{k}(\ctraj) = \|\matP_k(\matTheta)\|_{\op}
	\end{align}
	Hence, if $\ctraj$ is $\Lfp$-stabilizable, 
	\begin{align}
	\max_{k \in [K+1]}\|\matP_k(\matTheta)\|_{\op} \le \Lfp.
	\end{align}
	\end{lemma}

	\begin{lemma}[Lyapunov Lemma, Lemma F.10 in \cite{pfrommer2023power}]\label{lem:lyap_lem} Let $\matX_{1:K},\matY_{1:K}$ be matrices of appropriate dimension, and let $Q \succeq \eye$ and $\matY_k \succeq 0$. Define $\matLam_{1:K+1}$ as the solution of the recursion
	\begin{align}
	\matLam_{K+1} = \matQ, \quad \matLam_{k} = \matX_k^\top \matLam_{k+1} \matX_k + \step \matQ + \matY_k
	\end{align}
	Define the operator $\matPhi_{j+1,k} = \matX_j \cdot \matX_{j-1},\dots \cdot \matX_k$, with the convention $\matPhi_{k,k} = \eye$. Then, if $\max_{k}\|\eye - \matX_k\|_{\op} \le \kappa \step$ for some $\kappa \le 1/2\step$,
	\begin{align}
	\|\matPhi_{j,k}\|^2 \le \max\{1,2\kappa\}\max_{k \in [K+1]}\|\matLam_{k}\|(1 - \step \alpha)^{j-k}, \quad \alpha := \frac{1}{\max_{k \in [K+1]}\|\matLam_{1:K+1}\|}.
	\end{align}
	\end{lemma}

	\begin{claim}\label{claim:par:bounds_regular} If $\ctraj$ is $(0,\Lf,\infty)$-regular, then for all $k$, $\bA_k = \bA_k(\ctraj)$ and $\bB_k = \bB_k(\ctraj)$ satisfy $\max_{k\in [K]}\max\{\|\bA_k\|,\|\bB_k\|\} \le \Lf$. 
	\end{claim}
	\begin{proof} For any $k \in [K]$,
	\begin{align}
	\max\{\|\bA_k\|,\|\bB_k\|\} = \max\left\{\left\|\ddx f(\bbarx _k,\bbaru _k)\right\|,\left\|\ddu f(\bbarx _k,\bbaru _k)\right\|\right\} \le \left\|\nabla f(\bbarx _k,\bbaru _k)\right\| \le \Lf,
	\end{align}
	where the last inequality follows by regularity.
	\end{proof}
	\begin{claim}\label{claim:K_bound}Recall $\Kric_k(\ctraj) = (\eye + \step \matB_k^\top \Pric_{k+1}(\ctraj)\matB_k )^{-1}(\matB_k^\top \Pric_{k+1}(\ctraj))(\eye + \eta \matA_k)$. Then, if $\ctraj$ is $\Lfp$-stabilizable and $(0,\Lf,\infty)$-regular, and if $\eta \le 1/3\Lf$,
	\begin{align}
	\|\Kric_k(\ctraj)\| \le \frac{4}{3}\Lfp\Lf
	\end{align}
	\end{claim}
	\begin{proof} We bound
	\begin{align}
	\|\Kric_k(\ctraj)\| &\le \|\matB_k\|\|\Pric_{k+1}(\ctraj)\|(1+\eta\|\matA_k\|) \\
	&\le \Lf(1+\eta \Lf)\|\Pric_{k+1}(\ctraj)\| \tag{\Cref{claim:par:bounds_regular}}\\
	&\le \Lfp\Lf(1+\eta \Lf)\tag{\Cref{lem:V_P_equiv}, $\Lfp \ge 1$} \\
	&\le \frac{4}{3}\Lfp\Lf \tag{$\eta \le 1/3\Lf$.}
	\end{align} 
	\end{proof}
	 
	\begin{proof}[Proof of \Cref{lem:instantiate_lyap}] We want to show that $\Kric_{1:K}(\ctraj)$ is $(\Rstab,\Bstab,\Lstab)$-stabilizing.\Cref{claim:K_bound}  has already established that $\max_{k \in [K]}\|\Kric_k(\ctraj)\| \le \Rstab = \frac{4}{3}\Lfp\Lf$. 

	To prove the other conditions, we apply \Cref{lem:lyap_lem} with  $\bY_k = \bK_k(\matTheta)\bK_k(\matTheta)$, $\bQ = \eye$, and $\matX_k = \eye + \eta \Aclk(\matTheta)$. From \Cref{defn:ric_update}, let have that the term $\matLam_k$ in \Cref{lem:lyap_lem} is precise equal to $\matP_k(\matTheta)$. From \Cref{lem:V_P_equiv}, 
	\begin{align}
	\max_{k \in [K+1]}\|\matP_k(\matTheta)\|_{\op} = \max_{k \in [K+1]}\cV_k(\ctraj) \le \Lfp.
	\end{align}
	This implies that if $\max_{k}\|\matX_k - \eye\| \le \kappa \eta \le 1/2$, we have
	\begin{align}
	\|\Phicl{j,k}(\matTheta)\|^2 = 
	\|(\matX_j \cdot \matX_{j-1}\cdot \dots \matX_k)\| \le \max\{1,2\kappa\}\Lfp\left(1 - \frac{\step}{\Lfp}\right)^{j-k}.
	\end{align}
	It suffices to find an appropriate upper bound $\kappa$. We have
	\begin{align}
	\|\matX_k - \eye\| = \|\eta \Aclk(\matTheta)\| &\le \eta (\|\matA_k\| + \|\matB_k\|\|\matK_k(\matTheta)\|)\\
	&\le \eta \Lf(1 +\|\matK_k(\matTheta)\|)\\
	&\le \eta \Lf(1 + \frac{4}{3}\Lf\Lfp)\tag{\Cref{claim:K_bound}} \\
	&\le \frac{7}{3}\eta \Lf^2\Lfp \tag{$\Lfp,\Lf \ge 1$}
	\end{align}
	Setting $\kappa = \frac{7}{3}\Lf^2\Lfp.$, we have that as $\eta \le  \frac{1}{5\Lf^2\Lfp} \le \min\{\frac{3}{14 \Lf^2 \Lfp},\frac{1}{3\Lf}\}$ (recall $\Lf,\Lfp \ge 1$),
	we can bound 
	\begin{align}
	\max\{1,2\kappa\} \le \max\left\{1,\frac{14}{3}\Lf^2\Lfp\right\} \le \max\left\{1,5\Lf^2\Lfp\right\} = 5\Lf^2 \Lfp^2,
	\end{align}
	where again recall $\Lfp,\Lf \ge 1$.
	In sum, for $\eta \le  \frac{1}{5\Lf^2\Lfp}$, we have 
	\begin{align}
	\|\Phicl{j,k}\|^2 \le 5\Lf^2\Lfp^2\left(1 - \frac{\step}{\Lfp}\right)^{j-k}.
	\end{align}
	Hence,  using the elementary inequality $\sqrt{1 - a} \le (1 - a/2)$, 
	\begin{align}
	\|\Phicl{j,k}\| \le \sqrt{5}\Lf\Lfp\left(1 - \frac{\step}{\Lfp}\right)^{(j-k)/2} \le \sqrt{5}\Lf\Lfp\left(1 - \frac{\step}{2\Lfp}\right)^{j-k},
	\end{align}
	which shows that we can select $\Bstab = \sqrt{5}\Lf\Lfp$ and $\Lstab = 2\Lfp$.
	\end{proof}



\subsection{Solutions to recursions}\label{sec:recursion_solutions}
	\newcommand{\bPhi}{\bm{\Phi}}
This section contains the solutions to various recursions used in the proof of the two two results in  \Cref{sec:stab_of_trajectories}: \Cref{prop:master_stability_lem} (whose proof is given in \Cref{sec:prop:master_stability_lem} ) and  \Cref{lem:state_pert} (whose proof is given in \Cref{sec:lem:state_pert}).


	\begin{lemma}[First Key Recursion]\label{lem:key_rec_one} Let $C_1 > 0, C_2 \ge 1/2$, $\betastab \in (0,1)$, and suppose $\epsilon_1,\epsilon_2,\dots$ is a sequence satisfying $\epsilon_1 \le \bar \epsilon_1$, and 
	\begin{align}
	\epsilon_{k+1} \le C_2 \betastab^k \bar\epsilon_1 + C_1 \eta\sum_{j=1}^k \betastab^{k-j}\epsilon_j^2
	\end{align}
	Then, as long as $C_1 \le \beta(1-\beta)/2\eta$, it holds that $\epsilon_k \le 2C_2 \betastab^{k-1} \bar \epsilon_1$ for all $k$.
	\end{lemma}
	\begin{proof} Consider the sequence $\nu_k = 2C_2 \betastab^{k-1} \bar \epsilon_1 $. Since $C_2 \ge 1/2$, we have $\nu_1 \ge \bar \epsilon_1 \ge \epsilon_1$. Moreover, 
	\begin{align}
	C_2 \betastab^k \bar{\epsilon}_1 + C_1 \sum_{j=1}^k \betastab^{k-j}\nu_j &= C_2 \betastab^k \bar{\epsilon}_1 + 2C_1 C_2 \sum_{j=1}^k \betastab^{k+j-2} \bar \epsilon_1\\
	&= C_2 \betastab^k \bar{\epsilon}_1 \left(1 + \frac{2C_1}{\beta} \sum_{j=0}^{k-1} \betastab^{j}\right)\\
	&\le C_2 \betastab^k \bar{\epsilon}_1 \left(1 + \frac{2C_1\eta}{\beta(1-\beta)}\right)
	\end{align}
	Hence, for $C_1 \le \beta(1-\beta)/2\eta$, we have $C_2 \betastab^k \bar{\epsilon}_1 + C_1 \sum_{j=1}^k \betastab^{k-j}\nu_j \le 2C_2 \bar \epsilon_1 \betastab^k \le \nu_{k+1}$. This shows that the $(\nu_k)$ sequence dominates the $(\epsilon_k)$ sequence, as needed.
	\end{proof}
	\begin{lemma}[Second Key Recursion]\label{lem:key_rec_two} Let $c,\Delta,\eta > 0$, $\betastab \in (0,1)$ and let $\epsilon_1,\epsilon_2,\dots$ satisfy $\epsilon_1 \le c$ and 
	\begin{align}
	\epsilon_{k+1} \le c \betastab^k + c \eta  \Delta \betastab^{k-1}\sum_{j=1}^k \epsilon_j. 
	\end{align}
	Then, if $\Delta \le \frac{\beta(1-\beta)}{2c\eta}$, $\epsilon_{k+1} \le 2c\betastab^k$ for all $k$.
	\end{lemma}
	\begin{proof} Consider the sequence $\nu_k = 2c\betastab^{k-1}$. Since $\epsilon_1 \le c$, $\nu_1 \ge \epsilon_1$. Moreover, 
	\begin{align}
	c \betastab^k + c \eta  \Delta \betastab^{k-1}\sum_{j=1}^k \nu_j &\le c \betastab^k + 2c^2 \eta  \Delta \betastab^{k-1}  \sum_{j=1}^k \betastab^{j-1}\\
	&\le c \betastab^k + 2c^2 \eta  \Delta \betastab^{k-1}  \frac{1}{1-\beta}\\
	&\le c \betastab^k \left(1+2c\Delta   \frac{\eta}{\beta(1-\beta)}\right).
	\end{align}
	Hence, for $\Delta \le \frac{\beta(1-\beta)}{2c\eta}$, the above is at most $2c\betastab^k \le \nu_{k+1}$. This shows that the $(\nu_k)$ sequence dominates the $(\epsilon_k)$ sequence, as needed.
	\end{proof}

	\begin{lemma}[Third Key Recursion]\label{lem:key_rec_three} Let $C_1,C_2 > 0$, $\alpha \ge 0$, $\betastab \in (1/2,1)$, and let $\epsilon_1,\epsilon_2,\dots,$ and $\delta_1,\delta_2,\dots,$, and $\bar \epsilon_1 \ge \epsilon_1$  and be a sequence of real numbers satisfying
	\begin{align}
	\epsilon_{k+1} \le \alpha + \eta \sum_{j=1}^k \betastab^{k-j}(C_1 \epsilon_j^2 + C_2 \epsilon_j \delta_j) +  \betastab^{k/3} \bar \epsilon_1
	\end{align}
	Defin, $\Err_{\delta} := \max_k\eta \sum_{j=1}^k \betastab^{(k-j)} \delta_j^2$ and $M = \eta/(1-\beta)$. Then, as long as
	\begin{align}
	 \alpha \le \frac{1}{18 C_1 M}, \quad \bar \epsilon_1 \le \frac{1}{108 C_1 M}, \quad  \Err_{\delta} \le \frac{1}{12\sqrt{M}\max\{C_2,1\}},
	 \end{align}
	 the following holds for all $k \ge 0$:
	 \begin{align}
		\epsilon_{k+1} \le 3\alpha + 3\bar{\epsilon_1}\betastab^{k/3}.
		\end{align}
	\end{lemma}
	\begin{proof}[Proof of \Cref{lem:key_rec_three}] Consider a sequence 
	\begin{align}\nu_{k+1} = \alpha_{\star}  + c_{\star}\beta_{\star}^k \bar{\epsilon}_1, \quad \alpha_{\star} = 3\alpha, c_{\star} =3, \beta_{\star} = \betastab^{1/3}
	\end{align}
	 defined for $k \ge 0$, for some $\alpha_{\star} \ge \alpha$, $\beta_{\star} \in (\beta,1)$, and $c_{\star} \ge 1$. Then, $\nu_1 \ge \bar{\epsilon}_1, \ge \epsilon_1$. Let us define the term $B_k$ via
	 \begin{align}
	 B_k = \alpha + \eta \sum_{j=1}^k \betastab^{k-j}(C_1 \nu_j^2 + C_2 \nu_j \delta_j) +  \betastab^{k/3} \bar \epsilon_1.
	 \end{align}
	 It suffices to show $B_k \le \nu_{k+1}$ for all $k$. Introduce $\Term_{\nu,k} = \left(\eta \sum_{j=1}^k \betastab^{k-j} \nu_j^2\right)^{1/2}$ and $\Err_{\delta} = \max_{k}\left(\eta \sum_{j=1}^k \betastab^{k-j} \delta_j^2\right)^{1/2}$ Then, by Cauch-Schwartz,
	 \begin{align}
	 B_k &= \alpha + \eta \sum_{j=1}^k \betastab^{k-j}(C_1 \nu_j^2 + C_2 \nu_j \delta_j) +  \betastab^{k/3} \bar \epsilon_1\\
	 &\le \alpha + C_1 \Term_{\nu,k}^2 + C_2 \Term_{\nu,k} \Err_{\delta}+  \betastab^{k/3} \bar \epsilon_1.
	 \end{align}
	 We now bound
	 \begin{align}
	 \Term_{\nu,k}^2 &= \eta \sum_{j=1}^k \betastab^{k-j} \nu_j^2\\
	 &= \eta \sum_{j=1}^k \betastab^{k-j} (\alpha_{\star} + c_{\star}\bar{\epsilon}_1\beta_{\star}^{j-1})^2\\
	 &\le 2\eta \sum_{j=1}^k \betastab^{k-j} \alpha_{\star}^2 +  2\eta c_{\star}^2\bar \epsilon_1^2\sum_{j=1}^k \betastab^{k-j} \beta_{\star}^{2(j-1)}\\
	 &\le \frac{2\eta \alpha_{\star}^2}{1-\beta}  +  2\eta c_{\star}^2\bar \epsilon_1^2\sum_{j=1}^k \betastab^{k-j} \beta_{\star}^{2(j-1)}.
	 \end{align}
	Now, recalling $\beta_{\star} = \betastab^{1/3}$, we have
	\begin{align}
	\sum_{j=1}^k \betastab^{k-j} \beta_{\star}^{j-1} &= \sum_{j=1}^k \beta_{\star}^{3k- 3j} \beta_{\star}^{2(j-1)} = \sum_{j=1}^k \beta_{\star}^{3k- j - 2} \\
	&= \beta_{\star}^{2k-2}\sum_{j=1}^k \beta_{\star}^{k- j} = \beta_{\star}^{2k-2}\sum_{j \ge 0} \beta_{\star}^{j} \\
	&\le 3\beta_{\star}^{2k-2}\sum_{j \ge 0} \beta_{\star}^{3j} = 3\beta_{\star}^{2k} \betastab^{-2/3}\sum_{j \ge 0} \betastab^{j} \\
	&= \frac{3}{1-\beta}\beta_{\star}^{2k}\betastab^{-2/3} \le \frac{3}{\beta(1-\beta)}\beta_{\star}^{2k}.
	\end{align}
	Thus, adopting the shorthand $M = \eta/(1-\beta)$, and using the assumption $\betastab \ge 1/2$,
	\begin{align}
	\Term_{\nu,k}^2 &\le 2 \alpha_{\star}^2 M  +  12M c_{\star}^2\bar \epsilon_1^2\beta_{\star}^{2k}. 
	 \end{align}
	 Thus, 
	 \begin{align}
	 B_k &\le \alpha + C_1 \Term_{\nu,k}^2 + C_2 \Term_{\nu,k} \Err_{\delta}+  \betastab^{k} \bar \epsilon_1\\
	 &\le \alpha + 2 C_1  \alpha_{\star}^2 M  +  12C_1M c_{\star}^2\bar \epsilon_1^2\beta_{\star}^{2k} + \Err_{\delta}C_2\sqrt{2M}\alpha_{\star} + \Err_{\delta}C_2\sqrt{12M} c_{\star}\bar \epsilon_1\beta_{\star}^{k} +   \betastab^{k/3} \bar \epsilon_1\\
	 &= \alpha\left(1 + 2 C_1  \frac{\alpha_{\star}^2}{\alpha} M  + \frac{\alpha_{\star}}{\alpha} \Err_{\delta}C_2\sqrt{2M}\right)  + \beta_{\star}^k \bar \epsilon_1 \left(12C_1M c_{\star}^2\beta_{\star}^k\bar \epsilon_1 +  E_{\delta}\sqrt{12M} c_{\star} +\betastab^{k/3}\beta_{\star}^{-k}\right)\\
	 &\le \alpha\left(1 + 2 C_1  \frac{\alpha_{\star}^2}{\alpha} M  + \frac{\alpha_{\star}}{\alpha} \Err_{\delta}C_2\sqrt{2M}\right)  + \beta_{\star}^k \bar \epsilon_1 \left(12C_1M c_{\star}^2\bar \epsilon_1 +  E_{\delta}\sqrt{12M} c_{\star} \right)
	 \end{align}
	 where in the last line, we use $\beta_{\star} = \betastab^{1/3} \le 1$. Recalling $\alpha_{\star} = 3\alpha$ and $c = 3$, we have $B_{k} \le \alpha_{\star} + c_{\star} \bar{\epsilon}_1 \beta_{\star}^k = \nu_{k+1}$ as soon as
	 \begin{align}
	 1 &\ge 2 C_1  \frac{\alpha_{\star}^2}{\alpha} M  \vee \frac{\alpha_{\star}}{\alpha} \Err_{\delta}C_2\sqrt{2M} \vee 12C_1M c_{\star}^2\bar \epsilon_1 +  E_{\delta}\sqrt{12M} c_{\star} \\
	 &= 18 \alpha C_1  M   \vee 3\Err_{\delta}C_2\sqrt{2M} \vee 108C_1M \bar \epsilon_2 +  3E_{\delta}\sqrt{12M}\\
	 &= 18 \alpha C_1  M  \vee   108C_1M \bar \epsilon_1 \vee \Err_{\delta}(3C_2\sqrt{2M} \vee  3\sqrt{12M}\bar).
	 \end{align}
	 Thus, it suffices that 
	 \begin{align}
	 \alpha \le \frac{1}{18 C_1 M}, \quad \bar \epsilon_1 \le \frac{1}{108 C_1 M}, \quad  \Err_{\delta} \le \frac{1}{12\sqrt{M}\max\{C_2,1\}},
	 \end{align}
	  as needed.

	\end{proof}
	\begin{lemma}[Matrix Product Perturbation]\label{lem:mat_prod_pert} Define matrix products 
	\begin{align}\bPhi_{k,j} = \bX_{k-1} \cdot \bX_{k-2} \cdots \bX_j,\quad \bPhi_{k,j}' = \bX_{k-1}' \cdot \bX_{k-2}' \cdots \bX_j'.
	\end{align} 
	Let $\eta,\Delta,c > 0$ and $\betastab \in (0,1)$. If (a) $\bPhi_{k,j} \le \betastab^{k-j}$ for all $j \le k$, (b) $\|\bX_j - \bX_j'\| \le \eta \Delta \betastab^{j-1}$ for all $j \ge 1$  and (c) $\Delta \le \frac{\beta(1-\beta)}{2c\eta}$, then,  for all $j \le k$, $\|\bPhi_{k,j}'\| \le 2c\betastab^{k-j}$. 
	\end{lemma}
	\begin{proof} Without loss of generally, take $j = 1$. Then, letting $\bDelta_k = (\bX_k' - \bX_k)$,
	\begin{align}
	\bPhi_{k+1,1}' &= \bX_{k}' \cdot \bX_{k-2}' \cdots \bX_1'\\
	&= \bX_{k}' \cdot \bPhi_{k,1}'\\
	&= \bDelta_k\bPhi_{k,1}' + \bX_{k} \bPhi_{k,1}'\\
	&= \bDelta_k\bPhi_{k,1}' + \bX_{k} \bDelta_{k-1} \bPhi_{k-2,1}' + \bX_{k}\bX_{k-1}\bPhi_{k-2,1}'\\
	&= \bPhi_{k+1,k+1}\bDelta_k\bPhi_{k,1}' + \bPhi_{k+1,k} \bDelta_{k-1} \bPhi_{k-2,1}' + \bPhi_{k+1,k}\bPhi_{k-2,1}'\\
	&= \sum_{j=1}^{k}\bPhi_{k+1,j+1}\bDelta_{j}\bPhi_{j,1}' + \bPhi_{k+1,1}.
	\end{align}
	Thus, 
	\begin{align}
	\|\bPhi_{k+1,1}'\|_{\op} &\le c \eta \sum_{j=1}^{k}\betastab^{k-j}\|\bX_j-\bX_j'\|\|\bPhi_{j,1}'\| + c \betastab^k\\
	&\le c \eta \betastab^{k-1}\Delta\sum_{j=1}^{k}\|\bPhi_{j,1}'\| + c \betastab^k \tag{$\|\bX_j - \bX_j'\| \le \eta \Delta \betastab^{j-1}$}.
	\end{align}
	Define $\epsilon_j = \|\bPhi_{j,1}'\|$. Then, $\epsilon_1 = 1 \le c$, so \Cref{lem:key_rec_two} implies that for $\Delta \le \frac{(1-\beta)\beta}{2\eta}$, $ \|\bPhi_{k,1}'\| := \epsilon_k  \le 2c\betastab^k$ for all $k$.
	\end{proof}
\begin{comment}


\subsection{Proof of \Cref{prop:stabilizing}}

	 Introduce 
	 \begin{align}
	 \Delxk = \tilde{\bx}_{k} - \bbarx _{k},\quad \Deluk = \tilde{\bu}_k - \bbaru _k,
	 \end{align} and the shorthand $\bA_k= \bA_k(\ctraj)$ and $\bB_k= \bB_k(\ctraj)$. Note that $\tilde{\bu}_k - \bbaru _k$ does not include the feedback term. 
	 We expand
	\begin{align}
	\Delxk[k+1] &= \Delxk + \eta \left(f(\bbarx _k + \Delxk,\bbaru _k + \Deluk + \matKtil_k \Delxk) - f(\bbarx _k,\bbaru _k) \right)\\
	&= \Delxk + \eta \left(f(\bbarx _k + \Delxk,\bbaru _k + \Deluk + \matK_k \Delxk ) - f(\bbarx _k,\bbaru _k) \right) \\
	&\quad + \eta (\rem_{k,1})\\
	&= \Delxk + \eta \left(\underbrace{\ddx f(\bbarx _k,\bbaru _k)}_{=\bA_k} \Delxk + \underbrace{\ddu f(\bbarx _k,\bbaru _k)}_{=\bB_k} (\Deluk + \matK_k \Delxk)\right) \\
	&\qquad + \eta (\rem_{k,1} + \rem_{k,2})\\
	&= \Delxk + \eta \left(\Aclk\Delxk + \matB_k \Deluk \right) + \eta (\rem_{k,1} + \rem_{k,2}).
	\end{align}
	where, above
	\begin{align}
	\rem_{k,1} &= f(\bbarx _k + \Delxk,\bbaru _k + \Deluk + \matKtil_k \Delxk ) - f(\bbarx _k + \Delxk,\bbaru _k + \Deluk + \matK_k \Delxk )\\
	\rem_{k,2} &= f(\bbarx _k + \Delxk,\bbaru _k + \Deluk + \matK_k \Delxk ) -  f(\bbarx _k ,\bbaru _k)\\
	&\qquad- \ddx f(\bbarx _k,\bbaru _k) \Delxk + \ddu f(\bbarx _k,\bbaru _k) (\Deluk + \matK_k \Delxk) 
	\end{align}
	Solving the recursion, we have 
	\begin{align}
	\Delxk[k+1] = \sum_{j=1}^{k} \Phicl{k+1,j+1}(\bB_k \Deluk + \eta (\rem_{k,1}+\rem_{k,2})) + \Phicl{k+1,1}\Delxk[1].
	\end{align}
	We now bound the contribution of the remainder terms
	\begin{claim}\label{claim:remainder_claim}  Suppose that 
	\begin{align}
	\|\Delxk\|(1+\max\{\Rstab,\Rstabtil\}) + \|\Deluk\|  \le \Rdyn \label{eq:claim_remainder_claim}
	\end{align}
	Then, 
	\begin{align}
	 \|\rem_{k,1}\| + \|\rem_{k,2}\| \le \Mf\|\Deluk\|^2 + \Mf(1+\Rstab^2)\|\Delxk\|^2 +\Lf\|\matKtil_k-\matK_k\|\|\Delxk\|. 
	\end{align}
	\end{claim}
	Using $\|\matB_k\| \le \Lf$, we have
	\begin{align}
	&\|\Delxk[k+1]\| \\
	&\le \Lf\sum_{j=1}^{k} \step\|\Phicl{k+1,j+1}\|_{\op} \|\Deluk\|  +  \Mf\sum_{j=1}^{k}\step\|\Phicl{k+1,j+1}\|_{\op}\|\Deluk\|^2  \\
	&\quad + \sum_{j=1}^{k}\step\|\Phicl{k+1,j+1}\|_{\op}(\Mf(1+\Rstab^2)\|\Delxk\|^2 +\Lf\|\matKtil_k-\matK_k\|_{\op}\|\Delxk\|) +\|\Phicl{k+1,1}\|_{\op}\Delxk[1]\|.  \label{eq:Del_k_rec}
	\end{align}


	\begin{claim}[Key Recursion]\label{claim:main_recursion} Suppose \eqref{eq:claim_remainder_claim} holds for all $j \le k$, and that, in addition,
	\begin{align}
	& \max_{1 \le j \le k}\|\Delta_{\bx,j}\| \le \frac{1}{8\Lstab\Mf(1+\Rstab^2)}, \label{eq:recur_cond}\\
	&\|\Deluk[1:k]\|_{\ltwo} \le \frac{1}{\Mf},\quad \text{and} \quad \|\matKtil_{1:k}-\matK_{1:k}\|_{\ltwoop} \le \frac{1}{4\Lf\Lstab^{1/2}\Bstab} \label{eq:uKcond_thing}
	\end{align}
	Then,
	\begin{align}
	\|\Delxk[k+1]\| &\le  2\Bstab ((1+\Lf\Lstab^{1/2})\|\Deluk[1:k]\|_{\ltwo}    + \left(1-\frac{\step}{\Lstab}\right)^{k/2}\|\Delxk[1]\|) \label{eq:recur_conc}
	\end{align}
	\end{claim}
	We may now conclude. Recall our assumptions
	\begin{align}
	&\text{(a)}\quad(1+\Lf\Lstab^{1/2})\|\Deluk[1:K]\|_{\ltwo} +\|\Delxk[1]\|  \le \frac{1}{16\Bstab^2\Lstab\Mf(1+\Rstab^2)} \wedge \frac{1}{4\Bstab\Rdyn(1+\max\{\Rstab,\Rstabtil\})}\label{eq:u_cond}\\
	&\text{(b)}\quad\|\matKtil_{1:K}-\matK_{1:K}\|_{\ltwo} \le \frac{1}{4}(\Lf\Bstab\Lstab^{1/2})^{-1}.
	\end{align}
	Then, using $\Bstab,\Lstab \ge 1$, we see that \eqref{eq:uKcond_thing} is satisfied, and that check that \eqref{eq:recur_cond} and \eqref{eq:claim_remainder_claim} holds for $k=1$. Moreover, we can check that if \eqref{eq:recur_conc} holds us to some $k$, then \eqref{eq:recur_cond,eq:claim_remainder_claim} hold up to $k+1$.  Since \Cref{claim:main_recursion} in turn applies that \eqref{eq:recur_conc}, we conclude that, for all $k \in [K]$,
	\begin{align}
	\|\Delxk[k+1]\| &\le  2\Bstab ((1+\Lf\Lstab^{1/2})\|\Deluk[1:k]\|_{\ltwo}    + \left(1-\frac{\step}{\Lstab}\right)^{k/2}\|\Delxk[1]\|).
	\end{align}
	as needed.
	\qed


	\begin{proof}[Proof of \Cref{claim:remainder_claim}] Both follow from Taylor's theorem. We have
	\begin{align}
	&\|f(\bbarx _k + \Delxk,\bbaru _k + \Deluk + \matKtil_k \Delxk ) - f(\bbarx _k + \Delxk,\bbaru _k + \Deluk + \matK_k \Delxk )\|\\
	&= \|\int_{s=0}^1 \dds f(\bbarx _k + \Delxk,\bbaru _k + \Deluk + s(\matKtil_k-\matK_k)\Delxk +  \matK_k \Delxk ) \rmd s\|\\
	&= \max_{s \in [0,1]}\|\dds f(\bbarx _k + \Delxk,\bbaru _k + \Deluk + s(\matKtil_k-\matK_k)\Delxk +  \matK_k \Delxk )\|\\
	&= \max_{s \in [0,1]}\|\nabla_{(x,u)} f(\bbarx _k + \Delxk,\bbaru _k + \Deluk + s(\matKtil_k-\matK_k)\Delxk +  \matK_k \Delxk )\|\|\matKtil_k-\matK_k\|\\
	&\le \Lf\|\matKtil_k-\matK_k\|\|\Delxk\|. 
	\end{align}
	where the second-to-last inequality is by \mscomment{...}. For the second, define the curve $\bbarx_k(s) = \bbarx _k + s\Delxk$, $\bbaru_k(s) = \bbaru _k +  s(\Deluk + \matK_k \Delxk)$, and set $\bar{z}_k(s) = (\bar \bbarx _k(s),\bar \bbaru _k(s))$. Then,
	\begin{align}
	\|\rem_{k,2}\| &= \|f(\bar \bbarx _k(1),\bar \bbaru _k(1)) - f(\bar \bbarx _k(0),\bar \bbaru _k(0)) - \dds f(\bar \bbarx _k(s),\bar \bbaru _k(s)) \big{|}_{s= 0}\|\\
	&\le \frac{1}{2}\max_{s \in [0,1]} \|\frac{\rmd^2}{\rmd s^2} f(\bar \bbarx _k(s),\bar \bbaru _k(s))\|\\
	&\overset{(i)}{=} \frac{1}{2}\max_{s \in [0,1]} \left\|\nablatwo_{(x,u)} f(\bbarx_k(s),\bbaru_k(s))\left[\dds\bar{z}_k(s), \dds\bar{z}_k(s),:\right]\right\|\\
	&\overset{(ii)}{\le} \frac{1}{2}\Mf\|\dds \bar{z}_k(s)\|^2\\
	&= \frac{1}{2}\Mf\|(\Delxk, \Deluk + \matK_k\Delxk)\|^2\\
	&\le \Mf\|\Deluk\|^2 + \Mf(1+\|\matK_k\|^2)\|\Delxk\|^2\\
	&\le \Mf\|\Deluk\|^2 + \Mf(1+\Rstab^2)\|\Delxk\|^2
	\end{align}
	Summing these bounds concludes.
	\begin{align}
	&\|\rem_{k,1}\| + \|\rem_{k,2}\| \\
	&\quad \le \Mf\|\Deluk\|^2 + \Mf(1+\Rstab^2)\|\Delxk\|^2 +\Lf\|\matKtil_k-\matK_k\|\|\Delxk\|. 
	\end{align}
	\end{proof}

	\begin{proof}[Proof of \Cref{claim:main_recursion}] 
	Recall that, by assumption,
	\begin{align}
	\|\Phicl{k+1,j+1}\|_{\op} \le \Bstab\alpha^{k-j}, \quad \alpha = (1-\eta/\Lstab).
	\end{align}
	Thus, starting from \eqref{eq:Del_k_rec},
	\begin{align}
	\|\Delxk[k+1]\| &\le \Lf\sum_{j=1}^{k} \step\|\Phicl{k+1,j+1}\|_{\op} \|\Deluk[j]\|  +  \Mf\sum_{j=1}^{k}\step\|\Phicl{k+1,j+1}\|_{\op}\Deluk[j]\|^2  \\
	&\quad + \sum_{j=1}^{k}\step\|\Phicl{k+1,j+1}\|_{\op}(\Mf(1+\Rstab^2)\|\Delxk[j]\|^2 +\Lf\|\matKtil_j-\matK_j\|_{\op}\|\Delxk[j]\|) +\|\Phicl{k+1,1}\|\Delxk[1]\|\\
	&\le \Lf\Lstab\sum_{j=1}^{k} \step\alpha^{k-j} \|\Deluk[j]\|  +  \Bstab\Mf\sum_{j=1}^{k}\step\alpha^{k-j}\|\Deluk[j]\|^2  \\
	&\quad + \Bstab\sum_{j=1}^{k}\step\alpha^{k-j}(\Mf(1+\Rstab^2)\|\Delxk[j]\|^2 +\Lf\|\matKtil_j-\matK_j\|_{\op}\|\Delxk[j]\|)  + \Bstab\alpha^{k}\|\Delxk[1]\|\\
	%
	&\le \Lf\Bstab(\sum_{j=1}^{k} \step\alpha^{2(k-j)})^{1/2} (\sum_{j=1}^k\step \|\Deluk[j]\|^2)^{1/2}  +  \Bstab\Mf\sum_{j=1}^{k}\step\alpha^{k-j}\|\Deluk[j]\|^2  + \Bstab\alpha^{k}\|\Delxk[1]\|\\
	&\quad + \Bstab\sum_{j=1}^{k}\step\alpha^{k-j}(\Mf(1+\Rstab^2)\|\Delxk\|^2 +\Lf\|\matKtil_j-\matK_j\|_{\op}\|\Delxk\|) \\
	&\overset{(i)}{\le} \Bstab (\Lf\Lstab^{1/2}\|\Deluk[1:k]\|_{\ltwo}  +  \Mf\|\Deluk[1:k]\|_{\ltwo}^2  + \alpha^{k/2}\|\Delxk[1]\|)\\
	&\quad + \Bstab\sum_{j=1}^{k}\step\alpha^{k-j}(\Mf(1+\Rstab^2)\|\Delxk[j]\|^2 +\Lf\|\matKtil_j-\matK_j\|_{\op}\|\Delxk[j]\|) \\
	&\overset{(ii)}{\le} \Bstab ((1+\Lf\Lstab^{1/2})\|\Deluk[1:k]\|_{\ltwo}   + \alpha^{k/2}\|\Delxk[1]\|)\\
	&\quad + \Bstab\sum_{j=1}^{k}\step\alpha^{k-j}(\Mf(1+\Rstab^2)\|\Delxk[j]\|^2 +\Lf\|\matKtil_j-\matK_j\|_{\op}\|\Delxk[j]\|)  \label{eq:almost_last_recurse}
	\end{align}
	where in $(i)$ we upper bounded $\alpha^k \le \alpha^{k/2}$, $\alpha \le 1$, and $\sum_{j=1}^{k} \step\alpha^{2(k-j)} \le \sum_{j=1}^{k} \step\alpha^{k-j}\le \step \sum_{i \ge 0}\alpha^{i} = \Lstab$, and in $(ii)$, we used  the bound that $\|\Deluk[1:k]\|_{\ltwo} \le 1/\Mf$. Continuing,
	\begin{align}
	&\sum_{j=1}^{k}\step\alpha^{k-j}(\Mf(1+\Rstab^2)\|\Delxk[j]\|^2 +\Lf\|\matKtil_j-\matK_j\|_{\op}\|\Delxk\|) \\
	&\quad \le \Mf(1+\Rstab^2)\left(\sum_{j=1}^{k}\step\sqrt{\alpha}^{k-j}\right)\left(\max_{j \in [k]}\sqrt{\alpha}^{k-j} \|\Delxk[j]\|^2\right) \\
	&\qquad+\left(\max_{j \in [k]}\sqrt{\alpha}^{k-j}\|\Delxk[j]\|\right)  \Lf\left(\sum_{j=1}^{k}\step\sqrt{\alpha}^{k-j}\|\matKtil_j-\matK_j\|_{\op}\right)\\
	&\quad \le \Mf(1+\Rstab^2)\left(\sum_{j=1}^{k}\step\sqrt{\alpha}^{k-j}\right)\left(\max_{j \in [k]}\sqrt{\alpha}^{k-j} \|\Delxk[j]\|^2\right) \\
	&\qquad+\left(\max_{j \in [k]}\sqrt{\alpha}^{k-j}\|\Delxk[j]\|\right)\|\matKtil_{1:k}-\matK_{1:k}\|_{\ltwoop}  \Lf\left(\sum_{j=1}^{k}\step{\alpha}^{k-j}\right)^{1/2}\\
	&\quad\overset{(i)}{\le}  2\Lstab\Mf(1+\Rstab^2)\left(\max_{j \in [k]}\sqrt{\alpha}^{k-j} \|\Delxk[j]\|^2\right) + \Lf\Lstab^{1/2}\|\matKtil_{1:k}-\matK_{1:k}\|_{\ltwoop}\left(\max_{j \in [k]}\sqrt{\alpha}^{k-j}\|\Delxk[j]\|\right) \\
	&\quad\overset{(ii)}{\le}  \frac{1}{2\Bstab}\left(\max_{j \in [k]}\sqrt{\alpha}^{k-j} \|\Delxk[j]\|\right) \label{eq:x_term_bound}
	\end{align}
	where in $(i)$, we use $\left(\sum_{j=1}^{k}\step{\alpha}^{k-j}\right) \le \Lstab$ and $\sum_{j=1}^k\step \sqrt{\alpha}^{k-j} \le \sum_{i \ge 0}\step\sqrt{\alpha}^{i} \le 2\sum_{i \ge 0}\step\alpha^i \le \Lstab$; and in $(ii)$ we uses what is stipulated in \eqref{eq:recur_cond}:
	\begin{align}
	\|\matKtil_{1:k}-\matK_{1:k}\|_{\ltwoop} \le \frac{1}{4\Lf\Lstab^{1/2}\Bstab}, \quad \max_{1 \le j \le k}\|\Delta_{\bx,j}\| \le \frac{1}{8\Lstab\Mf(1+\Rstab^2)}.
	\end{align}
	Plugging in \eqref{eq:x_term_bound}  in \eqref{eq:almost_last_recurse} gives
	\begin{align}
	\|\Delxk[k+1]\| \le \Bstab ((1+\Lf\Lstab^{1/2})\|\Deluk[1:k]\|_{\ltwo}  + \sqrt{\alpha}^{k}\|\Delxk[1]\|) + \frac{1}{2}\max_{j \in [k]}\sqrt{\alpha}^{k-j} \|\Delxk[j]\|
	\end{align}
	By applying the same argument to $k' \le k$ (and using the the $\|\cdot\|_{\ltwo}$ and $\|\cdot\|_{\ltwoop}$ norm of a sequence is non-decreasing in length) gives
	\begin{align}
	\|\Delxk[k'+1]\| \le \Bstab ((\Lf\Lstab^{1/2}+1)\|\Deluk[1:k]\|_{\ltwo}  + \sqrt{\alpha}^{k'}\|\Delxk[1]\|) + \frac{1}{2}\max_{j \in [k']}\sqrt{\alpha}^{k'-j} \|\Delxk[j]\|
	\end{align}
	Multipling by $\sqrt{\alpha}^{k-k'} \le 1$ then gives 
	\begin{align}
	\sqrt{\alpha}^{k-k'}\|\Delxk[k'+1]\|& \le \sqrt{\alpha}^{k-k'}\Bstab ((\Lf\Lstab^{1/2}+1)\|\Deluk[1:k]\|_{\ltwo}  + \sqrt{\alpha}^{k}\|\Delxk[1]\|) + \frac{1}{2}\max_{j \in [k']}\sqrt{\alpha}^{k-j} \|\Delxk[j]\|\\
	& \le \Bstab ((\Lf\Lstab^{1/2}+1)\|\Deluk[1:k]\|_{\ltwo}  + \sqrt{\alpha}^{k}\|\Delxk[1]\|) + \frac{1}{2}\max_{j \in [k]}\sqrt{\alpha}^{k-j} \|\Delxk[j]\|
	\end{align}
	Thus, we have show that 
	\begin{align}
	\max_{k' \le k} \sqrt{\alpha}^{k-k'}\|\Delxk[k'+1]\| \le \Bstab ((\Lf\Lstab^{1/2}+1)\|\Deluk[1:k]\|_{\ltwo}  + \sqrt{\alpha}^{k}\|\Delxk[1]\|) + \frac{1}{2}\max_{j \in [k]}\sqrt{\alpha}^{k-j} \|\Delxk[j]\|. 
	\end{align}
	Rearranging gives
	\begin{align}
	&\Bstab ((\Lf\Lstab^{1/2}+1)\|\Deluk[1:k]\|_{\ltwo}  + \sqrt{\alpha}^{k}\|\Delxk[1]\|) \\
	&\ge \max_{k' \le k} \sqrt{\alpha}^{k-k'}\|\Delxk[k'+1]\| -  \frac{1}{2}\max_{j \in [k]}\sqrt{\alpha}^{k-j} \|\Delxk[j]\|\\
	&=\max_{j \in \{2,\dots,k+1\}} \sqrt{\alpha}^{k-j-1}\|\Delxk[j]\| -  \frac{1}{2}\max_{j \in [k]}\sqrt{\alpha}^{k-j} \|\Delxk[j]\|
	\end{align}
	As we also have $\Bstab \ge 1$, it also holds that $\Bstab (\Lf\Lstab^{1/2}\|\Deluk[1:k]\|_{\ltwo}  +  \Mf\|\Deluk[1:k]\|_{\ltwo}  + \sqrt{\alpha}^{k}\|\Delxk[1]\|) \ge \|\Delxk[1]\|$. Thus, in fact, we have
	\begin{align}
	&\Bstab ((\Lf\Lstab^{1/2}+1)\|\Deluk[1:k]\|_{\ltwo}  + \sqrt{\alpha}^{k}\|\Delxk[1]\|) \\
	&\ge \max_{j \in [k+1]} \sqrt{\alpha}^{k-j-1}\|\Delxk[j]\| -  \frac{1}{2}\max_{j \in [k]}\sqrt{\alpha}^{k-j} \|\Delxk[j]\|\\
	&\overset{(i)}{\ge} \max_{j \in [k+1]} \sqrt{\alpha}^{k-j-1}\|\Delxk[j]\| -  \frac{1}{2}\max_{j \in [k+1]}\sqrt{\alpha}^{k-j-1} \|\Delxk[j]\| \\
	&= \frac{1}{2}  \max_{j \in [k+1]}\sqrt{\alpha}^{k-j-1}\|\Delxk[j]\|\\
	&\ge \frac{1}{2}\|\Delxk[k+1]\|
	\end{align}
	where in $(i)$ we use $\alpha \le 1$. Multiplying both sides through by a factor of $2$ concludes.


	\end{proof}


\end{comment}
