%!TEX root = ../main.tex

\newcommand\hmmax{0}
\newcommand\bmmax{0}

\usepackage{boxedminipage}
\usepackage{multirow,nicefrac}
\usepackage{makecell,upgreek}
\usepackage{footnote}
\usepackage{longtable}
\usepackage[shortlabels]{enumitem}
\usepackage{tablefootnote}
\usepackage[T1]{fontenc}
\usepackage{verbatim} 
\usepackage[utf8]{inputenc}
\usepackage{subcaption}
\usepackage{booktabs}   
\usepackage{float}


\usepackage{xcolor}              
\usepackage{preamble/color-edits}
\addauthor{kz}{brown}
\addauthor{ms}{magenta}
\addauthor{ju}{cyan}
\addauthor{jp}{purple}


\newcommand{\dbtilde}[1]{\tilde{\raisebox{0pt}[0.85\height]{$\tilde{#1}$}}}
\newcommand{\newlytyped}[1]{{\color{green}#1}}
\newcommand{\remind}[1]{{\color{blue}#1}}



\usepackage{amsthm}

\iftoggle{arxiv}
{
  \newcommand{\nipspar}[1]{\paragraph{#1}}

}
{
  \newcommand{\nipspar}[1]{\textbf{#1}}

}
\iftoggle{arxiv}
{
  %\usepackage{subfigure}
  \usepackage{fullpage}
  \usepackage[margin=1in]{geometry}

  \usepackage[breaklinks=true]{hyperref}
 % \usepackage{graphicx}
  \usepackage{times}

  \renewcommand\theadalign{bc}
  \renewcommand\theadfont{\bfseries}
  \renewcommand\theadgape{\Gape[4pt]}
  \renewcommand\cellgape{\Gape[4pt]}
  \usepackage[noend]{algpseudocode}
  \usepackage{algorithm}
}
{
  \usepackage[breaklinks=true]{hyperref}

  \usepackage[noend]{algpseudocode}
  \usepackage{algorithm}
}



\usepackage{amsfonts,amssymb,mathrsfs}
\usepackage{mathtools}
\DeclareMathSymbol{\shortminus}{\mathbin}{AMSa}{"39}
\DeclareMathOperator*{\Expop}{\mathbb{E}}


\usepackage{prettyref}
\newcommand{\pref}[1]{\prettyref{#1}}

\usepackage[capitalise,nameinlink]{cleveref}
\Crefname{equation}{Eq.}{Eqs.}
\Crefname{assumption}{Assumption}{Assumptions}
\Crefname{condition}{Condition}{Conditions}
\Crefname{claim}{Claim}{Claims}

\usepackage{breakcites,bm}

\hypersetup{colorlinks,citecolor=blue,linkcolor = blue}
% better to load after ams math
\usepackage{latexsym}
\usepackage{relsize}
\usepackage{thm-restate}
\usepackage{appendix}

%\usepackage{thmtools,thm-restate}
%\usepackage[amsthm]{ntheorem}

\usepackage{xcolor}
\usepackage{dsfont}

\iftoggle{workshop}{
  \bibliographystyle{icml2023}
}{
\usepackage[numbers]{natbib}
\bibliographystyle{abbrvnat}
}

\newcommand{\N}{\mathbb{N}}
\newcommand{\R}{\mathbb{R}}
\newcommand{\Htwo}{\mathcal{H}_{2}}
\newcommand{\FigureOut}{{\color{blue} (\mathrm{FigureOut}) }}

%\newcommand{\defeq}{\stackrel{\mathrm{def}}{=}}
\newcommand{\defeq}{:=}


\numberwithin{equation}{section}
\newcommand\numberthis{\addtocounter{equation}{1}\tag{\theequation}}




\def\X{{\mathcal X}}
\def\H{{\mathcal H}}
\def\Y{{\mathcal Y}}
\newcommand{\eye}{\mathbf{I}}
% notation here
\newcommand{\ftil}{\tilde{f}}
\newcommand{\textnorm}[1]{\text{\normalfont #1}}

\def\hestinv{\tilde{\nabla}^{-2}}
\def\hest{\tilde{\nabla}^{2}}
\def\hessinv{\nabla^{-2}}
\def\hess{\nabla^2}
\def\grad{\nabla}
\newcommand{\nablatwo}{\nabla^{\,2}}
\def\ntil{\tilde{\nabla}}

\newcommand{\rmd}{\mathrm{d}}
\newcommand{\Diag}{\mathrm{Diag}}% notation



\newcommand{\Differential}{\mathsf{D}}



\newcommand{\re}{{\text{Re}}}
\newcommand{\mH}{\mathcal{H}}
\renewcommand{\sp}{\mathrm{span}}

\newcommand{\bzero}{\ensuremath{\mathbf 0}}
\newcommand{\y}{\ensuremath{\mathbf y}}
\newcommand{\z}{\ensuremath{\mathbf z}}
\newcommand{\h}{\ensuremath{\mathbf h}}
\newcommand{\K}{\ensuremath{\mathcal K}}
\newcommand{\M}{\ensuremath{\mathcal M}}
\def\bb{\mathbf{b}}
%\def\bc{\mathbf{c}}
\def\bB{\mathbf{B}}
\def\bC{\mathbf{C}}
%\def\be{\mathbf{e}}
\def\etil{\tilde{\mathbf{e}}}
%\def\bk{\mathbf{k}}
\def\bu{\mathbf{u}}
\def\x{\mathbf{x}}
\def\y{\mathbf{y}}
\def\w{\mathbf{w}}
\def\by{\mathbf{y}}
\def\bz{\mathbf{z}}
\def\bp{\mathbf{p}}
\def\bq{\mathbf{q}}
\def\br{\mathbf{r}}
\def\bu{\mathbf{u}}
\def\bv{\mathbf{v}}

\def\bw{\mathbf{w}}
%\def\ba{\mathbf{a}}
\def\bA{\mathbf{A}}
\def\bS{\mathbf{S}}
\def\bG{\mathbf{G}}
\def\bI{\mathbf{I}}
\def\bJ{\mathbf{J}}
\def\bP{\mathbf{P}}
\def\bQ{\mathbf{Q}}
\def\bV{\mathbf{V}}
\def\bone{\mathbf{1}}
\def\regret{\mbox{{Regret\ }}}
\def\newmethod{\mbox{{Observation Gradient\ }}}

\def\ytil{\tilde{{y}}}
\def\yhat{\hat{{y}}}
\def\what{\hat{{\bw}}}
\def\xhat{\hat{\mathbf{x}}}
\def\xbar{\bar{\mathbf{x}}}


\DeclareFontFamily{U}{mathx}{\hyphenchar\font45}
\DeclareFontShape{U}{mathx}{m}{n}{
      <5> <6> <7> <8> <9> <10>
      <10.95> <12> <14.4> <17.28> <20.74> <24.88>
      mathx10
      }{}
\DeclareSymbolFont{mathx}{U}{mathx}{m}{n}
\DeclareFontSubstitution{U}{mathx}{m}{n}
\DeclareMathAccent{\widecheck}{0}{mathx}{"71}
\DeclareMathAccent{\wideparen}{0}{mathx}{"75}

\newcommand{\ignore}[1]{}

\newcommand{\email}[1]{\texttt{#1}}

\newcommand{\BigOhSt}[1]{\BigOm^*\left({#1}\right)}
% Asymptotics
\DeclareMathOperator{\BigOm}{\mathcal{O}}
\newcommand{\BigOhPar}[2]{\BigOm_{#1}\left({#2}\right)}

\newcommand{\BigOh}[1]{\BigOm\left({#1}\right)}
\DeclareMathOperator{\BigOmtil}{\widetilde{\mathcal{O}}}
\newcommand{\BigOhTil}[1]{\BigOmtil\left({#1}\right)}
\DeclareMathOperator{\BigTm}{\Theta}
\newcommand{\BigTheta}[1]{\BigTm\left({#1}\right)}
\DeclareMathOperator{\BigWm}{\Omega}
\newcommand{\BigThetaTil}[1]{\widetilde{\BigTm}\left({#1}\right)}
\newcommand{\BigOmega}[1]{\BigWm\left({#1}\right)}
\newcommand{\BigOmegaSt}[1]{\BigWm^*\left({#1}\right)}
\newcommand{\BigOmegaTil}[1]{\widetilde{\BigWm}\left({#1}\right)}


\newcommand{\Alg}{\mathsf{alg}}
\newcommand{\iidsim}{\overset{\mathrm{i.i.d}}{\sim}}
\newcommand{\Regret}{\mathrm{Regret}}

\newcommand{\op}{\mathrm{op}}
\newcommand{\fro}{\mathrm{F}}
\newcommand{\algcomment}[1]{\textcolor{blue!70!black}{\footnotesize{\texttt{\textbf{\% #1}}}}}



	
	\theoremstyle{plain}
	%\theoremstyle{jmlrthm}
	% \newtheorem{theorem}{Theorem}[section]
	\newtheorem{theorem}{Theorem}
	\newtheorem{thminformal}{Informal Theorem}
	\newtheorem{oracle}{Oracle}[section]

	\newtheorem{lemma}{Lemma}[section]
	\newtheorem{claim}[lemma]{Claim}
%	\newtheorem{fact}[lemma]{Fact}
	\newtheorem{corollary}{Corollary}[section]
	\newtheorem{proposition}[lemma]{Proposition}

	\theoremstyle{definition}

	\newtheorem{definition}{Definition}[section]
    \newtheorem{desideratum}{Desideratum}
	\newtheorem{example}{Example}[section]
    \newtheorem{construction}{Construction}[section]


	\newtheorem{remark}{Remark}[section]
    \newtheorem{observation}[lemma]{Observation}

  \newtheorem{fact}{Fact}[section]	
  \newtheorem{assumption}{Assumption}[section]
  \newtheorem{condition}{Condition}[section]


\makeatletter
\newcommand{\neutralize}[1]{\expandafter\let\csname c@#1\endcsname\count@}
\makeatother


  \newenvironment{lemmod}[2]
  {\renewcommand{\thelemma}{\ref*{#1}#2}%
   \neutralize{lemma}\phantomsection
   \begin{lemma}}
  {\end{lemma}}


\newenvironment{thmmod}[2]
  {\renewcommand{\thetheorem}{\ref*{#1}#2}%
   \neutralize{theorem}\phantomsection
   \begin{theorem}}
  {\end{theorem}}


    \newenvironment{propmod}[2]
  {\renewcommand{\theproposition}{\ref*{#1}#2}%
   \neutralize{proposition}\phantomsection
   \begin{proposition}}
  {\end{proposition}}

     \newenvironment{defnmod}[2]
  {\renewcommand{\thedefinition}{\ref*{#1}#2}%
   \neutralize{definition}\phantomsection
   \begin{definition}}
  {\end{definition}}

   \newenvironment{asmmod}[2]
  {\renewcommand{\theassumption}{\ref*{#1}#2}%
   \neutralize{assumption}\phantomsection
   \begin{assumption}}
  {\end{assumption}}

  \newenvironment{cormod}[2]
  {\renewcommand{\thecorollary}{\ref*{#1}#2}%
   \neutralize{corollary}\phantomsection
   \begin{corollary}}
  {\end{corollary}}
  
   \newenvironment{condmod}[2]
  {\renewcommand{\thecondition}{\ref*{#1}#2}%
   \neutralize{condition}\phantomsection
   \begin{condition}}
  {\end{condition}}


\newtheorem*{theorem*}{Theorem}
\newtheorem*{lemma*}{Lemma}
\newtheorem*{corollary*}{Corollary}
\newtheorem*{proposition*}{Proposition}
\newtheorem*{claim*}{Claim}
\newtheorem*{fact*}{Fact}
\newtheorem*{observation*}{Observation}



\newtheorem*{definition*}{Definition}
\newtheorem*{remark*}{Remark}
\newtheorem*{example*}{Example}



\newtheoremstyle{named}{}{}{\itshape}{}{\bfseries}{}{.5em}{\Cref{#3} {\normalfont (informal)} }
{}
\theoremstyle{named}
\newtheorem*{informaltheorem}{Theorem}
\theoremstyle{plain}




\newenvironment{repthm}[2][]{%
  \def\theoremauxref{\cref{#2}}
  \begin{theoremaux}[#1]
}{%
  \end{theoremaux}
}


% bold serif
\DeclareMathAlphabet{\mathbfsf}{\encodingdefault}{\sfdefault}{bx}{n}

% operators
\DeclareMathOperator*{\argmin}{arg\,min}
\DeclareMathOperator*{\argmax}{arg\,max}
\DeclareMathOperator*{\conv}{conv}
\DeclareMathOperator*{\supp}{supp}
%\DeclareMathOperator*{\poly}{poly}
\let\Pr\relax
\DeclareMathOperator{\Pr}{\mathbb{P}}
\DeclareMathOperator*{\Prob}{\mathbb{P}}

% cases
\newcommand{\mycases}[4]{{
\left\{
\begin{array}{ll}
    {#1} & {\;\text{#2}} \\[1ex]
    {#3} & {\;\text{#4}}
\end{array}
\right. }}

\newcommand{\mythreecases}[6] {{
\left\{
\begin{array}{ll}
    {#1} & {\;\text{#2}} \\[1ex]
    {#3} & {\;\text{#4}} \\[1ex]
    {#5} & {\;\text{#6}}
\end{array}
\right. }}

% macros
\newcommand{\lr}[1]{\!\left(#1\right)\!}
\newcommand{\lrbig}[1]{\big(#1\big)}
\newcommand{\lrBig}[1]{\Big(#1\Big)}
\newcommand{\lrbra}[1]{\!\left[#1\right]\!}
\newcommand{\lrnorm}[1]{\left\|#1\right\|}
\newcommand{\lrset}[1]{\left\{#1\right\}}
\newcommand{\lrabs}[1]{\left|#1\right|}
%\newcommand{\set}[1]{\{#1\}}
\newcommand{\norm}[1]{\left|\left|#1\right|\right|}
\newcommand{\ceil}[1]{\lceil #1 \rceil}
\newcommand{\floor}[1]{\lfloor #1 \rfloor}

\renewcommand{\t}[1]{\smash{\tilde{#1}}}
\newcommand{\wt}[1]{\smash{\widetilde{#1}}}
\newcommand{\wh}[1]{\smash{\widehat{#1}}}
\newcommand{\OO}[1]{\O\lr{#1}}
\newcommand{\tTheta}{\wt{\Theta}}
\newcommand{\E}{\mathbb{E}}
\newcommand{\Exp}{\mathbb{E}}
\newcommand{\EE}[1]{\E\lrbra{#1}}
\newcommand{\var}{\mathrm{Var}}
\newcommand{\tsum}{\sum\nolimits}
\newcommand{\trace}{\mathrm{tr}}
\newcommand{\ind}[1]{\mathbb{I}\!\lrset{#1}}
\newcommand{\non}{\nonumber}
\newcommand{\poly}{\mathrm{poly}}
\newcommand{\I}{\mathbf{I}}
\newcommand{\rr}{\mathbb{R}}
\newcommand{\pp}{\mathbb{P}}
\newcommand{\ee}{\mathbb{E}}
\newcommand{\Xhat}{\widehat{X}}
\DeclareMathOperator{\tv}{TV}
\newcommand{\dkl}[2]{\mathrm{D}_{\mathrm{KL}}\left(#1 \parallel #2\right)}
\newcommand{\qtil}{\widetilde{q}}
\newcommand{\xtil}{\widetilde{\mathbf{x}}}
\newcommand{\abs}[1]{\left|#1\right|}
\newcommand{\inprod}[2]{\left\langle #1, #2 \right\rangle}




\renewcommand{\epsilon}{\varepsilon}

\usepackage{autonum}