%!TEX root = ../main.tex

\section{Complete Related Work}\label{app:related}

 \iftoggle{arxiv}
 {}
 {
%!TEX root = ../neurips_main.tex

\paragraph{Imitation Learning.} Over the past few years, there has been a significant surge of interest in utilizing machine learning techniques for the execution of exceedingly intricate manipulation and control tasks.  Imitation learning, whereby a policy is trained to mimic expert demonstrations, has emerged as a highly data efficient and effective method in this domain, with application to self-driving vehicles \citep{hussein2017imitation,bojarski2016end,bansal2018chauffeurnet}, visuomotor policies \citep{finn2017one,zhang2018deep}, and navigation tasks \citep{hussein2018deep}. A widely acknowledged challenge of imitation learning is distribution shift: since the training and test time distributions are induced by the expert and trained policies respectively, compounding errors in imitating the expert at test-time can lead the trained policy to explore out-of-distribution states \citep{ross2010efficient}. This distribution shift has been shown to result in the imitator making incorrect judgements regarding observation-action causality, often with catastrophic consequences \citep{de2019causal}. Prior work in this domain has predominantly attempted to mitigate this issue in the non-stochastic setting via online data augmentation strategies, sampling new trajectories to mitigate distribution shift \citep{ross2011reduction,ross2010efficient,laskey2017dart}. Among this class of methods, the DAgger algorithm in particular has seen widespread adoption \cite{ross2010efficient,sun2023mega,kelly2019hg}. These approaches have the drawback that sampling new trajectories or performing queries on the expert is often expensive or intractable. Due to these limitations, recent developments have focused on novel algorithms and theoretical guarantees for imitation learning in an offline, non-interactive environment \citep{chang2021mitigating,pfrommer2022tasil}. Our work is similarly focused on the offline setting, but is capable of handling stochastic, non-Markovian demonstrators. Unlike \citep{pfrommer2022tasil}, we do not require our expert demonstrations to be sampled from a stabilizing expert policy, instead utilizing a synthesis oracle to stabilize around the provided demonstrations. This is a significantly weaker requirement and enables the development of high-probability guarantees for human demonstrators, where sampling new trajectories and reasoning about the stability properties is not possible.

\paragraph{Denoising Diffusion Probabilistic Models.}  Denoising Diffusion Probabilistic Models (DDPMs) \citep{sohl2015deep,ho2020denoising} and their variant, Annealed Langevin Sampling \citep{song2019generative}, have seen enourmous empirical success in recent years, especially in state-of-the-art image generation \citep{ramesh2022hierarchical,nichol2021improved,song2020denoising}.  More relevant to this paper is their application to imitation learning, where they have seen success even without the proposed data augmentation in \citet{janner2022planning,chi2023diffusion,pearce2023imitating,hansen2023idql}.  DDPMs rely on learning the score function of the target distribution, which is generally accomplished through some kind of denoised estimation \citep{hyvarinen2005estimation,vincent2011connection,song2020sliced}.  On the theoretical end, annealed Langevin sampling has been studied with score estimators under a variety of assumptions including the manifold hypothesis and some form of dissapitivity \citep{raginsky2017non,block2020generative,block2020fast}, although these works have generally suffered from an exponential dependence on ambient dimension, which is unacceptable in our setting.  Of greatest relevance to the present paper are the concurrent works of \citet{chen2022sampling,lee2023convergence} that provide polynomial guarantees on the quality of sampling using a DDPM assuming that the score functions are close in an appropriate mean squared error sense.  We take advantage of these latter two works in order to provide concrete end-to-end bounds in our setting of interest.  To our knowledge, ours is the first work to consider the application of DDPMs to imitation learning under a rigorous theoretical framework, although we emphasize that this does not constitute a strong technical contribution as opposed to an instantiation of earlier work for the sake of completeness and concreteness.

\paragraph{Smoothing Augmentations.} Data augmentation with smoothing noise has become such common practice, its adoption is essentially folklore. While augmentation of actions which noise is common practice for exploration (see, e.g. \cite{laskey2017dart}), it is widely accepted that noising actions in the learned policy is not best practice, and thus it is more common to add noise to the \emph{states} at training time, preserving target actions as fixed \cite{ke2021grasping}. Our work gives an interpretation of this decision as enforcing that the learned policy obey the distributional continuity property we term TVC (\Cref{defn:tvc}), so that the policy selects similar actions on nearby states. Previous work has interpreted noise augmentation as providing robustness. Data augmentation has been demonstrated to provide more robustness in RL from pixels \citep{kostrikov2020image}, adaptive meta-learning \citep{ajay2022distributionally}, in more traditional supervised learning as well \citep{hendrycks2020jacob}.
}
\subsection{Comparison to prior notions of Stability.}

Prior work in guarantees for imitation learning focuses either on constraining the learned policy to be stable \cite{havens2021imitation,tu2022sample} or assume the expert policy is suitably stable \cite{pfrommer2022tasil}. 

The principal notion of stability used in these prior works is \emph{incremental-input-to-state} stability of the closed-loop system under a deterministic controller $\pi$: 

\begin{definition}[Incremental Input-to-State Stability]
    There exists class $\mathcal{K}$ function $\gamma$ and class $\mathcal{KL}$ function $\beta$ such that for any two initial conditions $\xi_1, \xi_2 \in \mathcal{X}$, the closed-loop dynamics under policy $\pi: \cal{X} \to \cal{U}$ given by $f_{\textrm{cl}}(x_t, \Delta_t) = f(x_t, \pi(x_t) + \Delta_t)$ satisfies:
    $$\|x_t(\xi_1; \{\Delta_s\}_{s=0}^t) - x_t(\xi_2; \{0\}_{s=0}^t)\| \leq \beta(\|\xi_1 - \xi_2\|) + \gamma\left(\max_{0 \leq s \leq t-1} \|\Delta_s\|\right),$$
    where $x_t(\xi; \{\Delta_s\}_{s=0}^{t-1})$ is the state at time $t$ under $f_{\textrm{cl}}$ with $x_0 = \xi$ and input perturbations $\{\Delta_s\}_{s=0}^{t-1}.$
\end{definition}
This notion of stability is quite restrictive, as the $\beta$-term necessitates that the dynamics converge irrespective of initial condition. Without time-varying dynamics this can only be achieved by a policy which stabilizes to an equilibrium point, as a policy which tracks a reference trajectory is unable to ``forget" the initial condition. Constraining learned policies such that they satisfy this notion of stability is also challenging. Tu et. al. \cite{tu2022sample} attempt to do so through regularization while Haven et. a. \cite{havens2021imitation} use matrix inequalities to satisfy this stability property under linear dynamics. Pfrommer et. at. \cite{pfrommer2022tasil} avoid this difficulty by relaxing the incremental stability to a local variant of stability:
\begin{definition}[$\eta$-Local Incremental Input-to-State Stability]\label{defn:related_local_stability}
    There exists class $\mathcal{K}$ function $\gamma$ such that for any $\xi \in \mathcal{X}$, the closed-loop dynamics under policy $\pi: \cal{X} \to \cal{U}$ given by $f_{\textrm{cl}}(x_t, \Delta_t) = f(x_t, \pi(x_t) + \Delta_t)$ satisfies:
    $$\|x_t(\xi; \{\Delta_s\}_{s=0}^t) - x_t(\xi; \{0\}_{s=0}^t)\| \leq \gamma\left(\max_{0 \leq s \leq t-1} \|\Delta_s\|\right),$$
    for all $\{\Delta_s\}_{s=0}^t$ where $\max_{0 \leq s \leq t} \|\Delta_s\| \leq \eta$.
\end{definition}

This weaker notion of incremental stability simply postulates the existence of a (local) input-perturbation to state-perturbation gain function $\gamma$. Since this stability property does not necessitate convergence across with different initial conditions and only under input perturbations of magnitude $\leq \eta$, this only necessitates that the expert policy can correct from small input perturbations.

% Figure environment removed

We further weaken this assumption, which we formalize in \Cref{asm:Jacobian_Stable} and abstract to the composite MDP through \Cref{defn:stability_setup}, by only requiring that a locally stabilizing controller can be synthesized per-demonstration. Through the introduction of a synthesis oracle which can generate locally stabilizing primitive controllers, we decouple the stability properties of the expert from the stabilizability of the underlying dynamical system. This allows for reasoning about generalization in the presence of bifurcations or conflicting demonstrations, which is precluded by \Cref{defn:related_local_stability} since an expert policy cannot simultaneously stabilize to multiple branches of a bifurcation.  For a concrete example, consider \Cref{fig:bifurcation}.  Indeed, continuity is the \emph{sine qua non} of stability and the example given demonstrates the necessity of augmentation to enforce the former.  In detail, the figure illustrates an example where an agent is navigating around an obstacle, providing a bifurcation.  Without augmentation, the demonstrator trajectories always navigate around the obstacle in the direction closer to their starting point, leading to a sharp discontinuity along a bisector of the obstacle.  On the other hand, the data augmentations allow for the policy to have some probability of navigating around the obstacle in the ``wrong'' direction, which leads to the notion of continuity we consider: total variation continuity.


% Figure environment removed
Because our notion of stability is applied in chunks, our theory is sufficiently flexible so as to allow for the learned policy to switch between expert demonstrations in a manner preserving the marginal distributions but not consistent with the joint distribution across the entire trajectory.  This flexibility is illustrated in \Cref{fig:figure_eight}, where we suppose that the demonstrator distribution consists both of trajectories traversing a figure ``8'' consistently in either a clockwise or counter-clockwise manner, with both orientations represented in the data set.  Due to the multi-modality at the critical point in the trajectory, there is ambiguity about which loop to traverse next; specifically, there may exist a policy that randomly select which loop to traverse each time the critical point is visited in such a way that the marginal distributions on states and actions is the same as that induced by the demonstrator.  Such a policy will, by definition, preserve the correct \emph{marginal} distributions across states and actions; at the same time, this policy has a different \emph{joint} distribution across all time steps from the demonstrator due to the possibility of traversing the same loop twice in a row.