
\begin{comment}
For simplicity, let $\Exp_y$ denote expectation under $\nu_1(\cdot \mid Y =y)\otimes \nu_2(\cdot \mid Y = y)$. Define the set of functions
\begin{align}
\scrF_{\phi} := \{(x_1,x_2) \mapsto f_1(x_1) + f_2(x_2) \le \phi(x_1,x_2): f_1,f_2 \text{ bounded and continuous}\}
\end{align}
By Kantorovich-Rubinstein Duality (henceforth, MK duality), 
\begin{align}
\psi(y) &:= \inf_{\coup \in \couple(\nu_1,\nu_2 \mid Y)}\Exp_{(X_1,X_2) \sim \coup}[\phi(X_1,X_2)] \\
&= \sup_{f \in \scrF_{\phi}}\Exp_{y}f(X_1,X_2).
\end{align}
where $\scrF_{\phi}$ denotes the set of continuous and bounded pairs $(f_1,f_2)$ where $f_i: \cX_i \to \R$ satisfying $\sum_{i=1}^2 f_i(X_i) \le \phi(X_1,X_2)$. 


\begin{claim}
\end{claim}


\begin{assumption} There exist a countable covering of $(\cY_j) \subset \cY$ such that, for each $\cY_j$, and all $\epsilon > 0$, there exists a compact set $\cU_{j,\epsilon} \subset \cX_1 \times \cX_2$ such that $\Exp_y[\phi(x_1,x_2)\I\{(x_1,x_2) \notin \cU_{j,\epsilon}\}] \le \epsilon$. 

 %a compact sets $\cK_{j,1,\epsilon}$ and $\cK_{j,2,\epsilon}$ for which, for all $y \in \cY_i$ and $i \in \{1,2\}$, $\nu_i(X_i \in \cK_{j,i,\epsilon} \mid Y = y) \ge 1 - \epsilon$.
\end{assumption}
\begin{proof} Fix a $j \in \N$, and $y \in \cY_j$. 
For any $y \in \cY_j$,
\begin{align}
\sup_{f \in \scrF_{\phi}}\Exp_{y}f(x_1,x_2) &= \sup_{f \in \scrF_{\phi}}\Exp_{y}f(x_1,x_2)\I\{(x_1,x_2) \in \cU_{j,1/m}\} + \sup_{f \in \scrF_{\phi}}\Exp_{y}f(x_1,x_2)\I\{(x_1,x_2) \in \cU_{j,1/m}\}^c\\
&\le \sup_{f \in \scrF_{\phi}}\Exp_{y}f(x_1,x_2)\I\{(x_1,x_2) \in \cU_{j,m}\} + \Exp_{y}\phi(x_1,x_2)\I\{(x_1,x_2) \in \cU_{j,1/m}\}\\
&\le \sup_{f \in \scrF_{\phi}}\Exp_{y}f(x_1,x_2)\I\{(x_1,x_2) \in \cU_{j,1/m}\}+  1/m. \label{eq:Fphi_something}
\end{align}
where we assume $\phi(\cdot,\cdot) \le B$.  By \mscomment{...}, for each $m$, there exists a countable subset $\scrF_{j,m} \subset \scrF_{\phi}$ consiting of functions such that, for each $y \in \cY_j$,
\begin{align}
\sup_{(x_1,x_2) \in \cU_{j,1/m}}\sup_{f \in \scrF_{\phi}}\inf_{f' \in \scrF_{j,m}}|f(x_1,x_2) - f'(x_1,x_2)| = 0.
\end{align}
Let $\scrF_{\phi,\N} = \bigcup_{j \ge 1,m \ge 1} \scrF_{j,m} \subset \scrF_{\phi}$. Then,
\begin{align}
\sup_{j,m \ge 1} \sup_{(x_1,x_2) \in \cU_{j,1/m}}\sup_{f \in \scrF_{\phi}}\inf_{f' \in \scrF_{\phi,\N}}|f(x_1,x_2) - f'(x_1,x_2)| = 0.
\end{align}
Hence, for each $j$ and each $y \in \scrY_{j}$, 
\begin{align}
\sup_{f \in \scrF_{\phi}}\Exp_{y}f(x_1,x_2)\I\{(x_1,x_2) \in \cU_{j,1/m}\}= \sup_{f \in \scrF_{\phi,\N}}\Exp_{y}f(x_1,x_2)\I\{(x_1,x_2) \in \cU_{j,1/m}\}
\end{align}
Hence, \eqref{eq:Fphi_something} implies that 
\begin{align}
\sup_{f \in \scrF_{\phi}}\Exp_{y}f(x_1,x_2) \le \sup_{f \in \scrF_{\phi,\N}}\Exp_{y}f(x_1,x_2)\I\{(x_1,x_2) \in \cU_{j,1/m}\}+  1/m. 
\end{align}
Then, for any $y \in \cY_j$

Therefore, for each $y \in \cY_j$,
\begin{align}
&\sup_{(f_1,f_2) \in \scrF_{\phi,[k]}}\inf_{(f_1',f_2') \in \scrF_{j,k,m}}\Exp_y|\sum_{i=1}^2(f(X_i) - f'(X_i))| \\
&\le \sup_{(f_1,f_2) \in \scrF_{\phi,[k]}}\inf_{(f_1',f_2') \in \scrF_{j,k,m}}\Exp_{y}\sum_{i=1}^2|f(X_i) - f'(X_i)| \I\{(X_1,X_2) \in \cK_{j,1,1/m}\times \cK_{j,2,1/m}\}\\
&\quad+ \sup_{(f_1,f_2) \in \scrF_{\phi,[k]}}\inf_{(f_1',f_2') \in \scrF_{j,k,m}}\Exp_{y}\sum_{i=1}^2|f(X_i) - f'(X_i)| \I\{(x_1,x_2) \in \cK_{j,1,1/m}\times \cK_{j,2,1/m}\}^c\\
&=\sup_{(f_1,f_2) \in \scrF_{\phi,[k]}}\inf_{(f_1',f_2') \in \scrF_{j,k,m}}\sum_{i=1}^2|f(X_i) - f'(X_i)| \Exp_{y}\I\{(X_1,X_2) \in \cK_{j,1,1/m}\times \cK_{j,2,1/m}\}^c\\
&\le 2k \Exp_{y}\I\{(X_1,X_2) \in \cK_{j,1,1/m}\times \cK_{j,2,1/m}\}^c\\
&\le 2k/m.
\end{align}
Therefore, defining the countable set $\cF_{k,\N} := \bigcup_{m \ge 1}\bigcup_{j \in \N}\cF_{j,k,m}$, it holds that for all $y \in \cY$,
\begin{align}
&\sup_{(f_1,f_2) \in \scrF_{\phi,[k]}}\inf_{(f_1',f_2') \in \scrF_{j,\N}}\Exp_y\sum_{i=1}^2|f(X_i) - f'(X_i)|  = 0.
\end{align}
\end{proof}


 and $\scrF_{\phi,\N}$ denotes any countable dense subset of $\scrF_{\phi}$, whose existence is guaranteed by the fact that $\cX_1 \times \cX_2$ is Polish. \mscomment{ADAM TODO justify: Stone Weirestress sigma compact}.

By the definition of a regular conditional probability measure and measurability of sums, the map $y \mapsto \sum_{i=1}^2\Exp_{X_i \sim \nu_1(\cdot \mid Y=y)}f_i(X_i)$ is measurable for any fixed $(f_1,f_2) \in \scrF_{\phi,\N}$ \mscomment{check}.  Thus, since countable suprema of measurable functions are measurable, $\psi: y \mapsto  \sup_{(f_1,f_2) \in \scrF_{\phi,\N}}\left(\sum_{i=1}^2\Exp_{X_i \sim \nu_1(\cdot \mid Y=y)}f_i(X_i)\right)$ is measuarable. 

For simplicity, enumerate the elements of $\scrF_{\phi,\N}$ as $(f_{1,j},f_{2,j})_{j \ge 1}$. Then, 
\begin{align}
\Exp_{Y \sim \mu_Y} \psi(Y) &=  \Exp_{Y \sim \mu_Y} \sup_{j \in \N}\left(\sum_{i=1}^2\Exp_{X_i \sim \nu_1(\cdot \mid Y=y)}f_{i,j}(X_i)\right)\\
&=  \Exp_{Y \sim \mu_Y} \sup_{\chi \text{measurable }:\cY \to \N}\left(\sum_{i=1}^2\Exp_{X_i \sim \nu_1(\cdot \mid Y=y)}f_{i,\chi(Y)}(X_i)\right)\\
&\overset{(1)}{=} \sup_{\chi \text{measurable }:\cY \to \N} \Exp_{Y \sim \mu_Y} \left(\sum_{i=1}^2\Exp_{X_i \sim \nu_1(\cdot \mid Y=y)}f_{i,\chi(Y)}(X_i)\right)\\
&= \sup_{\chi \text{measurable }:\cY \to \N} \sum_{i=1}^2\Exp_{(X_i,Y) \sim \nu_i}[f_{i,\chi(Y)}(X_i)]\\
&\overset{(2)}{=} \sup_{(g_1,g_2) \in \scrG_a} \sum_{i=1}^2\Exp_{(X_i,Y) \sim \nu_i}[g_i(X_i,Y)]
\end{align}
where, in $(2)$, we have defined the class of functions
\begin{align}
\scrG_{a} = \{(g_1(x_1,y),g_2(x_2,y)) = (f_{1,\chi(y)},f_{2,\chi(y)}: (f_{1,j},f_{2,j})\in \scrF_{\phi,\N}, \chi:\cY \to \text{measurable} \}.
\end{align}
Above, in $(1)$, we justify exchange of the integrable and supremum by using applying the monotone convergence theorem to the subsets of $\chi: \cY \to \{1,2,\dots,K\}$.\footnote{Specifically, define the functions $g_k = \sup_{\chi \text{measurable }:\cY \to [k]}\left(\sum_{i=1}^2\Exp_{X_i \sim \nu_1(\cdot \mid Y=y)}f_{i,\chi(Y)}(X_i)\right)$. Then, apply the monotone convergence theorem to $g_k - g_1$.  } We now claim that the following space of functions $\scrG_{b}$ is dense in $\scrG_a$: 
\begin{align}
\scrG_{b} = \{(g_1(x_1,y),g_2(x_2,y)):g_1(x_1,y) + g_2(x_2,y) \le \phi(x_1,x_2) , (g_1,g_2) \text{ bounded and continuous }\}
\end{align}
\begin{claim} For any $g_1(x_1,y),g_2(x_2,y) \in \scrG_a$ and $\epsilon > 0$, there exist a $g_1'(x_1,y),g_2'(x_2,y) \in \scrG_b$ such that 
\begin{align}
\left|\sum_{i=1}^2\Exp_{(X_i,Y) \sim \nu_i}[g_i(X_i,Y)] - \sum_{i=1}^2\Exp_{(X_i,Y) \sim \nu_i}[g_i'(X_i,Y)]\right| \le \epsilon
\end{align}
\end{claim}
\mscomment{ADAM TODO prove}
Let us first assume the claim is true. Then, we have shown
\begin{align}
\Exp_{Y \sim \mu_Y} \psi(Y) &= \sup_{(g_1,g_2) \in \scrG_b} \sum_{i=1}^2\Exp_{(X_i,Y) \sim \nu_i}[g_i(X_i,Y)]
\end{align}
Now, consider taking an expectation of $((X_1,Y_1),(X_2,Y_2))\sim \nu_1 \otimes \nu_2$ from the product measure. Note that the marginals of $Y_1$ and $Y_2$ are both equal to $\nu_Y$. We can write,
\begin{align}
\Exp_{Y \sim \mu_Y} \psi(Y) &= \sup_{(g_1,g_2) \in \scrG_b} \sum_{i=1}^2\Exp_{((X_1,Y_1),(X_2,Y_2)) \sim \nu_2\otimes \nu_2}[g_i(X_i,Y_1)] \\
&= \sup_{(g_1,g_2) \in \scrG_b} \Exp_{((X_1,Y_1),(X_2,Y_2)) \sim \nu_2\otimes \nu_2}[g_1(X_1,Y_1) + g_2(X_2,Y_1)]
\end{align} 
Let $\I_{\infty}(y_1 \ne y_2)$ denote the function which is equal to $\infty$ of $y_1 \ne y_2$, and $0$ otherwise. Observe that since the set $\{y_1 \ne y_2\}$ is open, $\I_{\infty}(y_1 \ne y_2)$ is lower-semicontinuous. 
Introduce another function class
\begin{align}
\scrG_{c} = \left\{(g_1(x_1,y_1),g_2(x_2,y_2)):g_1(x_2,y_1) + g_2(x_2,y_2) \le \phi(x_1,x_2) +\I_{\infty}(y_1 \ne y_2), (g_1,g_2) \text{ bounded and continuous }\right\}
\end{align}
We claim the following 
\begin{claim} The classes $\scrG_b$ and $\scrG_c$ are such that
\begin{align}
&\sup_{(g_1,g_2) \in \scrG_b} \Exp_{((X_1,Y_1),(X_2,Y_2)) \sim \nu_1\otimes \nu_2}[g_1(X_2,Y_1) + g_2(X_2,Y_1)] \\
&= \sup_{(g_1,g_2) \in \scrG_c} \Exp_{((X_1,Y_1),(X_2,Y_2)) \sim \nu_1\otimes \nu_2}[g_1(X_2,Y_1) + g_2(X_2,Y_2)],
\end{align}
where in the first expectation, $g_1$ and $g_2$ both us $Y_1$ as its argument, and in the second $g_2$ uses $Y_2$. 
\end{claim}
Assume this claim is also true, we then have established
\begin{align}
\Exp_{Y \sim \mu_Y} \psi(Y) 
&= \sup_{(g_1,g_2) \in \scrG_c} \Exp_{((X_1,Y_1),(X_2,Y_2)) \sim \nu_1\otimes \nu_2}[g_1(X_2,Y_1) + g_2(X_2,Y_1)]
\end{align}
By MK-duality once again (note that $\phi(X_1,X_2) +\I_{\infty}(Y_1 \ne Y_2)$ is lower-semicontinuous), the above is precisely equal to 
\begin{align}
\Exp_{Y \sim \mu_Y} \psi(Y) = 
\inf_{\coup \in \couple(\nu_1,\nu_2)}\Exp_{(X_1,Y_1,X_2,Y_2) \sim \coup}[\phi(X_1,X_2) +\I_{\infty}(Y_1 \ne Y_2)]
\end{align}
As $\Exp_{Y \sim \mu_Y} \psi(Y)$ is finite by assumption, so is $\inf_{\coup \in \couple(\nu_1,\nu_2)}\Exp_{(X_1,Y_1,X_2,Y_2) \sim \coup}[\phi(X_1,X_2) +\I_{\infty}(Y_1 \ne Y_2)]$. The only couplings for which the latter is finite are those for which $Y_1 = Y_2$ almost surely. Therefore, any feasible coupling of $(X_1,Y_1,X_2,Y_2)$ is equivalent to a coupling of $(X_1,X_2,Y) \in \couple_{\nu_Y}(\nu_1,\nu_2)$. This concludes.

\end{comment}
\begin{comment}
\subsection{Conditional Gluing}
While \Cref{lem:couplinggluing} shows that we can glue couplings together, it does not yet ensure that, if couple many conditional distributions simultaneously, we get out a valid probability distribution.



\begin{lemma}\label{lem:master_gluing} Let $(X_i)_{1\le i\le n+1}$ be random variables taking values in Polish spaces $(\cX_i)_{1 \le i \le n+1}$, and let $Y$ be a random variable taking values in a Polish space $(\cY,\dist_{\cY})$ with distribution $\mu_Y$. Let $\mu_i(X_i \mid Y)$ and $\phi_i : \cX_i \times \cX_{i+1} \to [0,1]$ be such that
\begin{itemize}
    \item For all but countably many $y \in \cY$,  and $i \in \{1,2,\dots,n+1\}$, $y \mapsto \mu_i(X_i \in \cdot \mid Y = y )$ is continuous at $y$ in total variation distance.
    \item $\phi_i : \cX_i \times \cX_{i+1} \to [0,1]$ is a lower-semicontinuous function.   
    \end{itemize} 
Then, there exists a probability measure $\nu$ over $Y,X_1,\dots,X_{i+1}$ such that $\nu(Y \in \cdot) = \mu_Y(Y \in \cdot)$, $\nu(X_i \in \cdot \mid Y) = \mu_i(X_i \in \cdot \mid Y)$ for each $1 \le i \le n+1$, and for all $1 \le i \le n$,
\begin{align}
\Exp[\phi_i(X_i,X_{i+1}) \mid Y =y] = \inf_{\coup \in \couple(\mu_i(\cdot \mid Y = y), \mu_{i+1(\cdot \mid Y = y)}))}[\phi_i(X_i,X_{i+1})]
\end{align}
\end{lemma}

\mscomment{modification for non-total variaation continuity}

\subsubsection{Proof of \Cref{lem:master_gluing}}
\begin{proof} Let $\cY_0$ denote the set of $\cY$ on which $y \mapsto \mu_i(\cY)$ is continuous for all $1\le i \le n+1$. By assumption, $\cY \setminus \cY_0$ is countable. Let us introduce the function
\begin{align}
\psi_i(y) = \inf_{\coup \in \couple(\mu_i(\cdot \mid Y = y), \mu_{i+1(\cdot \mid Y = y)}))}[\phi_i(X_i,X_{i+1})]. \label{eq:psi_def}
\end{align}
We begin by establishing its continuity.
\begin{claim} $y \mapsto \psi_i(y)$ is continous on $\cY_0$.
\end{claim}
\begin{proof} Fix some $y \in \cY_0$. By total variation continuity of $y \mapsto \mu_i(\cdot \mid Y = y)$ and  $y \mapsto \mu_{i+1}(\cdot \mid Y = y)$, there exists some $\delta$ such that, if $\dist_{\cY}(y,y') \le \delta$, then $\TV(\mu_i(\cdot \mid Y = y),\mu_i(\cdot \mid Y = y')) \le \epsilon$, and similarly for index $i+1$. By applying \Cref{lem:TV_perturbation}, it follows that for any coupling of $(X_i,X_{i+1})$ in $\couple(\mu_i(\cdot \mid Y = y), \mu_{i+1(\cdot \mid Y = y)}))$, there exists a coupling of $(X_i',X_{i+1}')$ in  $\couple(\mu_i(\cdot \mid Y = y'), \mu_{i+1(\cdot \mid Y = y')}))$ such that $\TV((X_i,X_{i+1}),(X_i',X_{i+1}')) \le 2\epsilon$. Thus, as the function $\phi_i$ is bounded in $[0,1]$, we conclude $\psi(y') \le \psi(y) + 2\epsilon$, as needed.
\end{proof}


\paragraph{Construction.} We now consider a sequence of measures $\nu_n$ of $(Y,X_1,\dots,X_{n+1})$. Since $\cY$ is Polish, there exists, for any $n$, a countable disjoint cover of $\cY$ by Borel sets $\cY_{n,j} \in \scrY_{n}$ of radius $1/n$. By ammending the cover if necessary, we may assume that for each $\cY_{n,j} $, either $\cY_{n,j} \subset \cY_0$ (the continuity select), or else $\cY_{n,j}$ is a singleton, and moreover, that for each of the countable $y \in \cY \setminus \cY_0$, $\{y\} \in \scrY_n$ (i.e. $\scrY_n$ contains singletons of all points of discontinuity of the measures $\mu_i(\cdot \mid Y)$).

For each $\cY_{n,k} \in \scrY_n$, pick a representative point $y_{n,j}$, and define the measure $\nu_{n}( (X_1,X_2) \in \cdot \mid Y = y)$ by 
\begin{enumerate}
    \item Assigning $y$ to its unique set $\cY_{n,j}$, and then to its representative point $y_{n,j}$. Denote this $y_{n}(y)$, its index $j_{n}(y)$ (i.e. $y_n(y) = y_{n,j_n(y)}$)
    \item Select couplings $\coup_{n,j,i}\in  \couple(\mu_{i}(\cdot \mid Y=y_{n,j}),\mu_{i+1}(\cdot \mid Y = y_{n,j}))$, for which $\Exp_{(X_i,X_{i+1}) \sim \coup_{n,j,i}}[\phi(X_i,X_{i+1})] \le \psi_i(y_{n,j}) +\frac{1}{n}$. Note that, necessarily, $\psi_i(y_{n,j}) \le \Exp_{(X_i,X_{i+1}) \sim \coup_{n,j,i}}[\phi(X_i,X_{i+1})]$. 
    \item By sequentially applying the gluing lemma (\Cref{lem:couplinggluing}), for every $(n,j)$, we can construct a joint coupling $\nu_{n,j}$ over $(X_1,\dots,X_{n+1})$ such that, for each $i$, $(X_i,X_j) \sim \coup_{n,j,i}$. Since the sets in $\scrY_n$ are countable and measurable, the the mapping $y \mapsto j_n(y)$ is measurable, so that $y \mapsto \nu_{n,j_n(y)}((X_1,\dots,X_{n+1})\in \cdot)$ defines a regular conditional probability. Thus, there exists a measure $\nu_n$ over $(Y,X_1,\dots,X_{n+1})$ with $Y \sim \mu_Y$, and $\nu_n(X_1,\dots,X_{n+1} \mid Y = y) \sim \nu_{n,j_n(y)}((X_1,\dots,X_{n+1})$. 

     %we can construct these couplings in such a way that there is a joint distribution $\nu_n$ over $(Y,X_1,X_2,\dots,X_{n+1})$ respecti
\end{enumerate}
By construction, $\nu_n$ satisfies
    \begin{align}
    \nu_n(X_i \in A\mid Y = y) &= \nu_n(X_i \in A\mid Y = y_{n}(y))\label{eq:nun_marg}\\
    \Exp_{(X_i,X_{i+1}) \sim \nu_{n}(\cdot\mid Y = y)}\phi_i(X_i,X_{i+1}) &\in \psi_i(y_n(y)) + [0,\frac{1}{n}] \label{eq:phi_marg}
    \end{align}
Note that our construction also ensures that 
\begin{align}
\forall y \in \cY \setminus \cY_0, n \ge 1,  \quad y_{n}(y) = y.
\end{align}
In particular, by continuity of $\mu_i(\cdot \mid Y = y)$ and $\psi_i(y)$ on all other $y \in \cY_0$,
\begin{align}
\lim_{n \to \infty} \nu_n(X_i \in A\mid Y = y_{n}(y)) = \mu_i(X_i \in A\mid Y = y), \quad \lim_{n \to \infty} \psi_i(y_n(y)) = \psi_i(y), \quad \label{eq:continuity_prop}
\end{align}
Thus, from \eqref{eq:nun_marg}, \eqref{eq:phi_marg}, and \eqref{eq:continuity_prop}, we immediate obtain
\begin{claim}\label{claim:limits} For all $y$ and each index $1\le i \le n+1$, $\lim_{n \to \infty}\nu_n(X_i \in A\mid Y = y) = \mu_i(X_i \in A\mid Y = y)$ and $\lim_{n \to \infty} \Exp_{(X_i,X_{i+1}) \sim \nu_{n}(\cdot\mid Y = y)}[\phi(X_i,X_{i+1})]= \psi(y)$. 
\end{claim}





\paragraph{Passing to a limit.} Next, we show that our measures $\nu_n$ satisfy an important property called \emph{tightness}, which we shall use to pass to a limiting measure. Specifically, we have
\begin{claim}\label{claim:tight} For every $\epsilon \ge 0$, there exists a compact subset $\cK \subset \cY \times \cX_1 \times \cX_2 \dots \times \cX_{n+1}$ such that, for all $n \ge n_0(\epsilon)$, $\sup_{n} \Pr_{\nu_n}[(Y,X_1,\dots,X_{n+1}) \notin \cK] \le \epsilon$.
\end{claim}
We prove the claim at the end of our demonstration. The property it establishes and Prokhorov's theorem \mscomment{cite} imply that there exists a subsequence $\nu_{n_k}$ such that $ \nu_{n_k}$ converges \emph{weakly} to a measure $\nu_{\star}$. This means that, for any lower-semicontinuous and bounded function $f(Y,X_1,\dots,X_n)$, we have
\begin{align}
\lim_{k\to \infty}\Exp_{\nu_{n_k}}[f(Y,X_1,\dots,X_n)] = \Exp_{\nu_{\star}}[f(Y,X_1,\dots,X_n)]
\end{align}
In particular, for any bounded continuous function of the form $f(Y,X_i)$,
\begin{align}
 \Exp_{\nu_{\star}}[f(Y,X_i)] &= \lim_{k\to \infty}\Exp_{\nu_{n_k}}[f(Y,X_i)]\\
 &= \lim_{k\to \infty}\Exp_{Y \sim \mu_Y}\Exp_{X_i \sim \nu_{n_k}(X_i \in \cdot \mid Y)}[f(Y,X_i)] \\
 &= \Exp_{Y \sim \mu_Y}\lim_{k \to \infty}\Exp_{X_i \sim \nu_{n_k}(X_i \in \cdot \mid Y)}[f(Y,X_i)] \tag{dominated convergence}\\
 &= \Exp_{Y \sim \mu_Y}\lim_{k \to \infty}\Exp_{X_i \sim \mu_{i}(\cdot \mid Y)}[f(Y,X_i)] \tag{\Cref{claim:limits}}
\end{align}
Since this equality holds for any bounded continuous function $f$, it follows that under $\nu_{\star}$, $Y \sim \mu_Y$ and $X_i \mid Y \sim \mu_i(\cdot \mid Y)$. \mscomment{show}


Next, consider the functions $h_i(y) = \Exp_{\nu_{\star}}[\phi_i(X_i,X_{i+1}) \mid Y = y]$. By the previous argument, the joint law of $X_i,X_{i+1} \mid Y $ is a coupling in $\couple(\mu_i(\cdot \mid Y),\mu_{i+1}(\cdot \mid Y))$. Hence, 
\begin{align}
h_i(y) \ge \psi_i(y) \text{ almost surely. } \label{eq:thing_as}
\end{align}
On the other hand, by Fubini's theorem and boundedness of $\phi_i$,
\begin{align}
\Exp_{\nu_{\star}}[h_i(Y)] &= \Exp_{\nu_{\star}}[\phi_i(X_i,X_{i+1})]\\ &= \lim_{k \to \infty}\Exp_{\nu_{n_k}}[\phi_i(X_i,X_{i+1})]  \tag{Fubini}\\
&= \lim_{k \to \infty}\Exp_{Y \sim \mu_Y}\Exp_{\nu_{n_k}(\cdot \mid Y)}[\phi_i(X_i,X_{i+1})]  \tag{weak convergence, $\phi_i$ is lower semicontinuous and bounded}\\
&= \Exp_{Y \sim \mu_Y}\lim_{k \to \infty}\Exp_{\nu_{n_k}(\cdot \mid Y)}[\phi_i(X_i,X_{i+1})]  \tag{dominated convergence}\\
&= \Exp_{Y \sim \mu_Y}[\psi_i(Y)]  \tag{\Cref{claim:limits}} \\
&= \Exp_{Y \sim \nu_{\star}}[\psi_i(Y)] \tag{marginal of $Y \sim \nu_{\star}$ is $\mu_Y$}
\end{align}
Hence, we have established that  $\Exp_{Y \sim \nu_{\star}}[h_i(Y)-\psi_i(Y) ] =0$. Together with \eqref{eq:thing_as}, it follows that 
\begin{align}
h_i(y) = \Exp_{\nu_{\star}}[\phi_i(X_i,X_{i+1}) \mid Y = y] = \psi_i(Y), \text{ almost surely}.
\end{align}
Recalling the definition $\psi_i(Y)$ from \eqref{eq:psi_def} concludes.
\begin{proof}[Proof of \Cref{claim:tight}] Let's fix a a compact set $\cK_Y$ such that $\Pr_{\mu_Y}[Y \notin \cK_Y] \le \frac{1}{3}$. Next, for each $i \in [n+1]$, fix a compact set $\cK_{X_i}$ such that $\Pr_{Y \sim \mu_Y}\Pr_{X_i \sim \mu_i(\cdot \mid Y}[X \notin X_i] \le \epsilon/3(n+1)$. 

These sets $\cK_Y,\cK_{X_i}$ exists because $\cY$ and $\cX_i$ are Polish spaces. Define 
\begin{align}\cK = \cK_Y \times \cK_{X_1} \times \cK_{X_2} \dots \times \cK_{X_{n+1}},
\end{align} 
as their product, which we note is compact.


Since $\cK_Y$ is compact, and the marginals $y \mapsto \mu_i(\cdot \mid Y)$ are continuous in TV on $\cY_0$, they are uniformly continuous on $\cY_0 \cap \cK_Y$. Thus, there exists some $n_0(\epsilon)$ sufficiently large such that for all $n \ge \nu_0(\epsilon)$, $\TV(\mu_i(\cdot \mid y = y),\mu_i(\cdot \mid y = y_{n}(y)) \le \epsilon/3(n+1)$ for all $y \in \cY_0 \cap \cK_Y$. This also holds trivially for $y \in (\cY \setminus \cY_0) \cap \cK_Y$, as then $y_n(y) = y$ for $y \in \cY \setminus \cY_0$.

Thus, since all $\nu_n$ have $Y \sim \mu_Y$,
\begin{align}
&\Pr_{\nu_n}[Y \notin \cK_Y ] = \Pr_{\mu_Y}[Y \notin \cK_Y ]  \le \frac{\epsilon}{3}\\
&\Pr_{\nu_n}[\{X_i \notin \cK_{X_i}\} \cap \{Y \in \cK_Y\}] = \Exp_{Y \sim \mu_Y}\I\{Y \in \cK_Y\}\Pr_{X_i \sim \mu_i(\cdot \mid Y = y_{n}(Y))}[X_i \notin \cK_{X_i}] \\
&= \Exp_{Y \sim \mu_Y}\I\{Y \in \cK_Y\}\Pr_{X_i \sim \mu_i(\cdot \mid Y )}[X_i \notin \cK_{X_i}]  + \Exp_{Y \sim \mu_Y}\I\{Y \in \cK_Y\}\TV(\mu_i(\cdot \mid y = y),\mu_i(\cdot \mid y = y_{n}(y))\\
&\le \frac{\epsilon}{n+1} + \frac{\epsilon}{n+1}
\end{align}
Hence, by a union bound, $\Pr_{\nu_n}[(Y,X_1,\dots,X_{n+1})\notin \cK] \le \epsilon$ for all $n \ge n_0(\epsilon)$, as needed.
\end{proof}

\end{proof}
\end{comment}