%\documentclass[aps,twocolumn,nofootinbib,citeautoscript,longbibliography,notitlepage,superscriptaddress,floatfix]{revtex4-1}%{revtex4-1}
%%\documentclass[prx,twocolumn,nofootinbib,citeautoscript,eqsecnum,longbibliography,notitlepage]{revtex4-1}
%\usepackage{graphicx}% Include figure files
%\usepackage{dcolumn}% Align table columns on decimal point
%\usepackage{bm}% bold math
%\usepackage{color}
%\usepackage{amsmath}
%\usepackage{tabularx}
%\usepackage{epstopdf}
%\usepackage{latexsym}
%\usepackage{amssymb}
%\usepackage{amsmath}
%\usepackage{colortbl}
%\usepackage{psfrag}
%\usepackage{bbm}
%\usepackage{bm}
%\usepackage{titlesec}
%\usepackage{dsfont}
%\usepackage{feynmp}
%\usepackage{slashed}
%\usepackage{multirow}
%\usepackage[tight]{subfigure}
%
%\usepackage[papersize={8.5in,11in}]{geometry}
%
%\usepackage{color}
%\definecolor{darkblue}{rgb}{0.,0.,0.4}
%\definecolor{darkred}{rgb}{0.5,0.,0.}
%\definecolor{BlueViolet}{RGB}{138,43,226}
%\definecolor{SkyBlue}{RGB}{30,144,255}
%\definecolor{DarkGreen}{RGB}{0,100,0}
%\usepackage[pdftex,colorlinks=true,linkcolor=darkblue,citecolor=blue,urlcolor=darkred]{hyperref}
%
%\usepackage{physics}
%
%\usepackage{cases}
%
%
%
%\geometry{top=2.5cm, left=2cm, right=2cm, bottom=2.5cm}
%
%
%\newcommand{\bk}{\mathbf{k}}
%\newcommand{\bp}{\mathbf{p}}
%\newcommand{\bq}{\mathbf{q}}
%\newcommand{\br}{\mathbf{r}}
%\newcommand{\bA}{\mathbf{A}}
%\newcommand{\be}{\mathbf{e}}
%
%\renewcommand{\epsilon}{\varepsilon}
%
%\def \nn{\nonumber \\}
%
%
%
%\newcommand{\intd}[2]{\int\negthinspace\negthickspace\frac{d^{#2}%
%#1}{(2\pi)^{#2}}}
%\newcommand{\intdef}[3]{\int_{#2}^{#3}\negthickspace\frac{d #1}{%
%2\pi}}
%
%\newcommand{\ve}{{\varepsilon}}
%%%%%%%%%%%%%%%%%%%%%
%
%
%%%%%%%%%%%%%%%%%%%%%%%%%%%%%%%%%%%%%%%%%%%%%%%%%%%%%%%%%%%
%%Added orders and notes by JW
%\allowdisplaybreaks[3] %һʽǡһҳŲ, ŵһҳĻ, һҳֿһƬ,
%                       %ͺܲ%һԱ֤ʽԿҳʾ
%                        %ֿѡ1234ԽҳǿԽ󣩡
%
%%\usepackage{CJK}
%
%%%%%%ɫ
%\newcommand{\red}[1]{\textcolor{red}{#1}}
%\newcommand{\blue}[1]{\textcolor{blue}{#1}}
%\newcommand{\green}[1]{\textcolor{green}{#1}}
%
%\makeatletter
%\newcommand{\rmnum}[1]{\romannumeral #1}
%\newcommand{\Rmnum}[1]{\expandafter\@slowromancap\romannumeral #1@}
%\makeatother
%
%%%%%%%%%%%%%%%%%%%%%%%%%%%%%%%%%%%%%%%%%%%%%%%%%%%%%%%%%%%%





%%Ϊ֮ǰİ汾
%%-----------------------------------------------------------------------------------------

%\documentclass[aps,showpacs,twocolumn,dvips,prb]{revtex4}
%\documentclass[aps,showpacs,twocolumn,dvips]{revtex4}
\documentclass[aps,twocolumn,superscriptaddress]{revtex4} %ȥdvipsֱpdf 뿴ɫ
\usepackage{feynmp}
\usepackage{amssymb}
\usepackage{amsmath}
\usepackage{epsfig}
\usepackage{graphicx}
\usepackage{subfigure}
\usepackage{mathrsfs}
\usepackage{hyperref}
\usepackage{cleveref}
\usepackage{cases}
\usepackage{bbm}
\usepackage{xcolor}
\usepackage{tabularx}
\usepackage{array}
\usepackage{multirow}
\usepackage{CJK}
\usepackage{lmodern}
\usepackage{textcomp}
\usepackage{makecell}
\usepackage{threeparttable}
\usepackage{ulem}

\newcommand{\red}[1]{\textcolor{red}{#1}}
\newcommand{\blue}[1]{\textcolor{blue}{#1}}
\newcommand{\green}[1]{\textcolor[rgb]{0.0745,0.641,0.227}{#1}}
\definecolor{light-gray}{gray}{0.65}
\newcommand{\gray}[1]{\textcolor{light-gray}{#1}}
\newcommand{\bm}[1]{\mbox{\boldmath{$#1$}}}
\allowdisplaybreaks[3]


%%---------------------------------------------------------------------------------------------


\begin{document}



%\begin{CJK*}{GBK}{song}


%%%%%%%%%%%%%%%%%%%%%%%%%%%%%%%%%%%%%%%%%%%%%%%%%%%%%%%%%%%%%%%%%%%%%%%%%%%%%%%%%%%%%%
%%%%%%%%%%%%%%%%%%%%%%%%%%%%%%%%%%%%%%%%%%%%%%%%%%%%%%%%%%%%%%%%%%%%%%%%%%%%%%%%%%%%%%



\title{Critical fates induced by interaction competition in the three-dimensional tilted Dirac semimetals}



\date{\today}

\author{Jing Wang}
\altaffiliation{Corresponding author: jing$\textunderscore$wang@tju.edu.cn}
\affiliation{Department of Physics, Tianjin University, Tianjin 300072, P.R. China}
\affiliation{Tianjin Key Laboratory of Low Dimensional Materials Physics and
Preparing Technology, Tianjin University, Tianjin 300072, P.R. China}

\author{Jie-Qiong Li}
\affiliation{Department of Physics, Tianjin University, Tianjin 300072, P.R. China}


\author{Wen-Hao Bian}
\affiliation{Department of Physics, Tianjin University, Tianjin 300072, P.R. China}


\author{Qiao-Chu Zhang}
\affiliation{Department of Physics, Tianjin University, Tianjin 300072, P.R. China}


\author{Xiao-Yue Ren}
\affiliation{Department of Physics, Tianjin University, Tianjin 300072, P.R. China}





\begin{abstract}

The interplay between Coulomb interaction, electron-phonon coupling, and phonon-phonon coupling has a significant impact on the low-energy behavior of three-dimensional type-I tilted Dirac semimetals. To investigate this phenomenon, we construct an effective theory, calculate one-loop
corrections contributed by all these interactions, and
establish the coupled energy-dependent flows of all associated interaction parameters
by adopting the renormalization group approach. Deciphering such coupled evolutions allows us to
determine a series of low-energy critical outcomes for these materials.
At first, we present the low-energy tendencies of all interaction parameters.
The tilting parameter exhibits distinct tendencies that
depend heavily upon the initial anisotropy of fermion velocities.  In comparison,
the latter is mainly dominated by its own initial value but less sensitive to the former.
With variance of these two quantities, parts of the interaction parameters are driven towards
the strong anisotropy in the low-energy, indicating the screened interaction in certain directions,
while others tend to move towards an approximate isotropy.  Additionally, we observe that the tendencies
of interaction parameters can be qualitatively clustered into three distinct
types of fixed points, accompanying
the potential instabilities around which certain interaction-driven phase transition is trigged.
Furthermore, approaching such fixed points leads to physical quantities, such as the density of states,
compressibility, and specific heat, exhibiting behavior that is significantly
different from their non-interacting counterparts and even deviates slightly
from Fermi-liquid behavior. Our investigation sheds light on the intricate
relationship between different types of interactions in these semimetals, and provide
useful insights into their fundamental properties.



\end{abstract}

%\pacs{\red{71.55.Jv, 71.10.-w}}

\maketitle

%%%%%%%%%%%%%%%%%%%%%%%%%%%%%%%%%%%%%%%%%%%%%%%%%%%%%%%%%%%%%%%%%%%%%%%%%%%%%%%%%%%%%
%%%%%%%%%%%%%%%%%%%%%%%%%%%%%%%%%%%%%%%%%%%%%%%%%%%%%%%%%%%%%%%%%%%%%%%%%%%%%%%%%%%%%


\section{Introduction}

In recent years, the study of Dirac semimetals (DSM) featuring the intermediate properties between
metals and insulators has become one of the most active fields in contemporary condensed
matter physics~\cite{Novoselov2005Nature,Castro2009RMP,Moore2010Nature,Hasan2010RMP,Qi2011RMP,
Vafek2014ARCMP,Wehling2014AP,Wang2012PRB,Young2012PRL,Steinberg2014PRL,Liu2014NM,Liu2014Science,
Xiong2015Science,Roy2009PRB,Roy2016,Roy-2014-2016,Savary2014PRB,Moon2014PRX,Montambaux}. Typically,
DSMs possess the Dirac cones and reduced Fermi surfaces that consist of several discrete
Dirac points with gapless low-energy excitations irrespective of the microscopic details,
and exhibit linear energy dispersions along two or three directions~\cite{Novoselov2005Nature,
Castro2009RMP,Moore2010Nature,Roy2009PRB,Wang2012PRB,Young2012PRL,Liu2014NM,Liu2014Science,
Xiong2015Science,Korshunov2014PRB,Hung2016PRB,Nandkishore2013PRB,Potirniche2014PRB,Nandkishore2017PRB,Roy2016,Roy-2014-2016}.
In particular, their unique properties are guaranteed and protected by certain kinds of symmetries including
time-reversal, space-reversal symmetry and etc.~\cite{Castro2009RMP,Vafek2014ARCMP}. However, these Dirac cones
can be stretched and thus tilted by breaking the
%fundamental Lorentz symmetry or so-called $t$-Lorentz symmetry~\cite{Jafari2019PRB-t}
certain fundamental symmetry (such as $t$-Lorentz symmetry)~\cite{Trescher2015PRB,Soluyanov2015Nature,Jafari2019PRB-t} or via an
additional force in certain direction~\cite{Mao2011ACS}. In this situation, the energy dispersions become anisotropic, with unequal
fermion velocities along different directions.  Henceforth, these materials are then known as the
tilted Dirac semimetals (tDSM)~\cite{Trescher2015PRB,Lee2018PRB,Lee2019PRB,Jafari2019PRB-t,Peres2010RMP,
Soluyanov2015Nature,Noh2017PRL,Fei2017PRB,Mao2011ACS}. Besides two-dimensional (2D) tilted Dirac cones were reported in organic compound
$\alpha-(\mathrm{BEDT-TTF})_{2}\mathrm{I}_{3}$ and certain mechanically deformed
graphene~\cite{Katayama2006JPSJ,Kobayashi2007JPSJ,Goerbig2008PRB},
three-dimensional (3D) tilted %Weyl cones were also realized later in $\mathrm{WTe}_{2}$~\cite{Soluyanov2015Nature}
cones were also realized later in $\mathrm{WTe}_{2}$~\cite{Soluyanov2015Nature}
and the Fulde-Ferrell ground state of a spin-orbit coupled fermionic
superfluid~\cite{Xu2015PRL} or a cold-atom optical lattice~\cite{Xu2016PRA}.
Conventionally, the tDSM can be clustered into two
distinct types based upon the tilted angles. Type-I tDSM still possesses
analogous Dirac cones as long as the tilted angle is insufficient to destroy
the point-like Fermi surface~\cite{Peres2010RMP,Jafari}. However, for Type-II tDSM,
such as $\mathrm{PdTe_2}$~\cite{Noh2017PRL,Fei2017PRB}
and $\mathrm{PtTe_2}$~\cite{Yan2017NC}, the Dirac point can be replaced by two straight lines
indicating the open Fermi surface once the tilted angle
is adequately large~\cite{Soluyanov2015Nature,Lee2018PRB,Jafari2019PRB-t}.


These tilted materials have recently garnered significant attention due to their unique
low-energy excitations and tilted Dirac cones~\cite{Shekhar2015NP,Parameswaran2014PRX,
Potter2014NC,Baum2015PRX,Arnold2016NC,Zhang2016NC,Lee2018PRB,Lee2019PRB,Fritz2017PRB,Fritz2019arXiv,
Jafari2018PRB,Alidoust2019arXiv,Yang2018PRB,Trescher2015PRB,Proskurin2015PRB,
Brien2016PRL,Zyuzin2016JETPL,Ferreiros2017PRB,Qiong2019NPJB}. Particularly, the effects
of Coulomb interaction and impurities on the low-energy properties of tDSM
were investigated by many groups~\cite{Lee2018PRB,Lee2019PRB,Fritz2017PRB,Fritz2019arXiv}.
However, previous studies on 3D tDSM inadequately took into account a number of physical ingredients, such as
phonons and electron-phonon interactions, and their interplay with Coulomb interaction.
These additional degrees of freedom may play a crucial role in determining the low-energy behavior of 3D tDSM,
and neglecting them could result in the partial or incomplete capture of important physical information that is
closely associated with such interactions. Therefore, to enhance our understanding of 3D tilted materials, it is essential
to carefully examine how the interplay between Coulomb interaction and electron-phonon
coupling affects the low-energy behavior of 3D tDSMs.


Without loss of generality, we within this work concentrate on the type-I 3D tDSM.
Compared to their 2D counterparts, they are more complicated but interesting. On one hand,
since the density of state of fermionic quasiparticles vanishes as approaching the Dirac point
of type-I tDSM~\cite{Peres2010RMP,Jafari,Fritz2019arXiv},
it is of necessity to take into account the effects of long-range
Coulomb interaction between low-energy fermionic excitations, which is marginal at the tree
level in the RG language. On the other side, the lattice vibrations in 3D materials are more
intricate and lead to the emergence of (acoustic or optical) phonons, which
exhibit different internal properties for ionic and covalent crystals
(for the sake of simplicity, this work is restricted to the latter in which the
coupling between phonon and Coulomb auxiliary field vanishes)~\cite{Ruhman2019PRX}.
Phonons can interact with each other via phonon-phonon
interactions and inevitably entangle with low-energy fermions, potentially
competing indirectly with Coulomb interaction. In this context, we expect
phonons and their related consequences to play an essential role in governing
the low-energy physics of 3D tDSM. In an effort for investigating the unusual behavior
of 3D tDSM in the low-energy regime, we are therefore forced to take into account all these items
on the same footing.


To this end, we employ the powerful renormalization group (RG)
approach~\cite{Wilson1975RMP,Polchinski9210046,Shankar1994RMP} to chalk out the central
information of 3D tDSM, which allows to unbiasedly treat all physical ingredients mentioned above.
Concretely, we construct the effective theory for type-I tDSM and derive the the coupled
energy-dependent evolutions of all interaction parameters after carrying out
the energy-shell RG analysis. To proceed, with the help of performing numerical analysis of such evolutions,
the low-energy fates of these interactions and their consequences on 3D tDSM are
addressed in details against the variations of both tilting parameter and fermion velocities.

At first, we examine how various interaction parameters behave in the low-energy regime under
the intimate competition between themselves. To be specific, we find that the tilting parameter is
insensitive to its own initial value but can be heavily influenced by the starting anisotropy of
fermion velocities. As to the fermion velocities, depending upon their initial anisotropy and tilting parameter,
their ratio can either increase, decrease or nearly remains unchanged in the low-energy regime.
Considering the dielectric constant, which characterizes the strength of
Coulomb interaction, it tends to exhibit strong anisotropy in the low-energy regime, indicating the
screened Coulomb interaction in certain direction. In comparison, we notice that both phonon velocities
and phonon-phonon interaction can either flow towards the approximate isotropy or exhibit the basic results
of the dielectric constant that flow towards some extreme anisotropy in the low-energy regime.
It is of also interest to point out that the coupling strength of electron-auxiliary bosonic interaction
and the electron-phonon interactions bear the similar behavior to that of the dielectric constant and
phonon velocities, respectively.


In addition, we obtain three kinds of fixed points by classifying the energy-dependent
tendencies of all interaction parameters and then investigate the leading instabilities around such
fixed points. Moreover, accompanying the instabilities, the critical tendencies of physical quantities,
including the density of states and compressibility as well as specific heat are carefully examined as
approaching these three distinct types of fixed points, which present different but interesting behavior
compared to their non-interacting counterparts.


The rest of this work is organized as follows. In Sec.~\ref{Sec_model}
we present the microscopic model and the low-energy effective theory.
The Sec.~\ref{Sec_RGEqs} is followed to bring out the RG transformations
and then derive the coupled energy-dependent RG equations of all interaction parameters in
our effective theory after taking into account all one-loop corrections.
Afterwards, upon evolving the RG flows we within Sec.~\ref{Sec_SM-tend-fates} address the
tendencies and fates of these interaction parameters in the low-energy regime.
In the forthcoming Sec.~\ref{Sec_FP-instab} and Sec.~\ref{Sec_critical_implications},
we present the dominant instabilities and behavior of physical implications as approaching
the fixed points, respectively. Finally, a brief summary is provided in Sec.~\ref{Sec_summary}.




\section{Microscopic model and effective theory}\label{Sec_model}


We concentrate on the 3D tilted Dirac semimetals,
which feature an additional term to tilt the energy bands.
Armed with the microscopic structure of 3D Dirac semimetals, the non-interacting
Hamiltonian density in the low-energy regime for the 3D tDSM can be formally
expressed as follows~~\cite{Goerbig2008PRB,Lee2018PRB,Fritz2019arXiv}
\begin{eqnarray}
\mathcal{H}_{0}=\zeta v_{z}k_{z}\sigma_{0}+\chi[v_{z}k_{z}\sigma_{z}
+v(k_{x}\sigma_{x}+k_{y}\sigma_{y})],\label{Eq H_0}
\end{eqnarray}
where $v$ and $v_z$ characterize the fermion velocities along the $oxy$ plane
and $z$ direction that is perpendicular to such plane. In addition, $\chi=\pm1$
stands for the chirality symmetry of Dirac point as well as $\sigma_i$ with
$i=1,2,3$ correspond to the Pauli matrices and $\sigma_0$ the identity
matrix.

Hereby, the dimensionless parameter $\zeta$ in Eq.~(\ref{Eq H_0}) serves as a tilting parameter which
enters into the energy dispersions as follows
\begin{eqnarray}
E_{\pm}(\mathbf{k})=\zeta v_zk_{z}\pm
\sqrt{(v_zk_{z})^2+(vk_x)^2+(vk_y)^2}\label{Eq dispersion}.
\end{eqnarray}
As a consequence, the presence of this very parameter can alter or reshape the
overall structure of Fermi surface by tilting the Dirac cones~\cite{Goerbig2008PRB,Lee2018PRB,Fritz2019arXiv}.
In practice, this compels us to compartmentalize two distinguished types of the tDSM~\cite{Lee2018PRB}.
The first is known as type-I tDSM and occurs when $|\zeta|<1$,
where the point-like Fermi surface remains robust and the renormalized Dirac cones are
preserved against tilted contributions. Instead, the other type dubbed the type-II tDSM,
arises when the tilted terms play a more significant role in formulating the Fermi surface at $|\zeta|>1$.
In this case, the point-like structure of the Fermi surface is sabotaged and replaced by two crossed nodal lines~\cite{Lee2018PRB,Lee2019PRB}.


% Figure environment removed

To proceed, we hereafter restrict our study to the 3D tpye-I tDSM, namely $|\zeta|<1$ in Eq.~(\ref{Eq dispersion})
as schematically shown in Fig.~\ref{fig1_schematic-tilted-dispersion}. Starting from the Hamiltonian
density Eq.~(\ref{Eq H_0}), we are left with the following noninteracting fermionic
action,
\begin{eqnarray}
S_{\psi}&=&\sum_{\chi,\alpha}\int_k\psi^{\dag}_{\chi\alpha}(\omega,\mathbf{k})
\{i\omega\sigma_{0}-\zeta v_{z}k_{z}\sigma_{0}
-\chi[v_{z}k_{z}\sigma_{z}\nonumber\\
&&+v(k_{x}\sigma_{x}+k_{y}\sigma_{y})]\}
\psi_{\chi\alpha}(\omega,\mathbf{k}),\label{Eq_S_psi}
\end{eqnarray}
with $\int_k=\int d\omega d^3\mathbf{k}/(2\pi)^4$. Herein the spinors $\psi^{\dag}_{\chi\alpha}(\omega,\mathbf{k})$
and $\psi_{\chi\alpha}(\omega,\mathbf{k})$ with spin degeneracy $\alpha=\pm1$
are employed to describe the excited fermionic quasiparticles from the Dirac points
in the first Brillouin zone~\cite{Goerbig2008PRB,Lee2018PRB,Fritz2019arXiv}. As we only consider the type-I tDSM,
the tilting parameter is compelled to $|\zeta|\in(0,1)$ with $\zeta\rightarrow0$
reducing to the normal (untilted) 3D Dirac systems.
It is worth highlighting that the tilted energy dispersion~(\ref{Eq dispersion})
signals that the tilted Dirac cones are symmetric under the sign change of tilting parameter.
Without loss of generality, we from now on restrict our study to the situation with a positive $\zeta$.
In addition, one can expects that the velocities are no longer isotropic
but become anisotropic for the tilted direction ($v_z$) and the other
two orientations ($v$) in the presence of the tilted terms.
To proceed, the free fermionic propagator can be readily derived from Eq.~(\ref{Eq_S_psi}).
\begin{eqnarray}
G^{-1}_{0}(k)
&=&(i\omega-\zeta v_{z}k_{z})\sigma_{0}
-\chi[v_{z}k_{z}\sigma_{z}
+v(k_{x}\sigma_{x}\nonumber\\
&&+k_{y}\sigma_{y})].\label{Eq_G_psi}
\end{eqnarray}


In addition to fermioic excitations~(\ref{Eq H_0}),
we for completeness bring out the contributions from phonons that characterize the potential
distortions of lattices. As the optical phonons are much more important than acoustic ones as approaching the potential
instability in the low-energy regime, we hereby only consider
the former~\cite{Ruhman2019PRX}.  In principle, there exits two different kinds
of modes corresponding to the transverse phonon and
longitudinal phonon. Accordingly, the involved phonon ingredients can be expressed
as follows~\cite{Ruhman2019PRX,Khmelnitskii1971SPSS,Strukov2012Book}
\begin{eqnarray}
S_{u}&=&\int d^{4}x\frac{1}{2}u_{j}(x)\left[(-\partial_{0}^{2})\delta_{ji}-C_{T}^{2}
(\nabla^{2}\delta_{ji}-\partial_{j}\partial_{i})\right.\nonumber\\
&&\left.-C_{L}^{2}\partial_{j}\partial_{i}\right]u_{i}(x)+V\int d^{4}x[u_{j}(x)u_{j}(x)]^2,\label{Eq_S_action-u}
\end{eqnarray}
where $u_{i}$ ($u_j$) represents
the phonon field with $i,j=x,y,z$, and $C_{T,L}$ specify the velocities of the transverse
and longitudinal phonons, as well as the coupling $V$ characterizes the self-interactions
among phonons themselves. In principle, the transverse phonon owns
a ``effective mass" which can played as a tuning parameter. To simplify, we
hereby assume it is small enough and neglect it in that the concerned regime
is adjacent to the potential phase transition where the phonons become gapless~\cite{Ruhman2019PRX}.
Under this circumstance, the free transverse and longitudinal
phonon propagators can be obtained in the momentum space,
\begin{eqnarray}
G_{\mathrm{u},ji}^{\mathrm{T}}(q_{0},\mathbf{q})
&=&\frac{\delta_{ji}-\widehat{q}_{j}\widehat{q}_{i}}
{q_{0}^{2}-w_{T}^2+C_{T}^{2}},\label{Eq_G_u-T}\\
G_{\mathrm{u},ji}^{\mathrm{L}}(q_{0},\mathbf{q})
&=&\frac{\widehat{q}_{j}\widehat{q}_{i}}
{q_{0}^{2}+C_{L}^{2}\mathbf{q}^{2}},\label{Eq_G_u-L}
\end{eqnarray}
with $\widehat{q}_{j}\widehat{q}_{i}\equiv q_{j}q_{i}/\mathbf{q}^2$ for $i,j=x,y,z$.
Besides their self-interactions in Eq.~(\ref{Eq_S_action-u}), the phonons are expected to couple with
the fermionic excitations, which can be constructed in the following~\cite{Ruhman2019PRX},
\begin{eqnarray}
S_{\mathrm{u}\psi}=\sum_j\lambda_{j}\int d^4x\psi^{\dag}\sigma_{j}\psi u_{j},\label{Eq_S_u-psi}
\end{eqnarray}
where the coupling $\lambda_{j}$ with $j=x,y,z$ are utilized to measure the
strength between fermion and phonon.

Further, it is of particular necessity to take into account the Coulomb interaction between
the low-energy fermionic excitations. For the sake of simplicity, such degrees of freedom
can be effectively established via introducing
an auxiliary bosonic field $\phi$ as~\cite{Lee2018PRB,Ruhman2019PRX,Moon2016PRB,Moon2016SR,Nandkishore2017PRB,Mandal2022PRB}
\begin{eqnarray}
S_{\mathrm{Coul}}
\!\!&=&\!\!\frac{1}{2}\int_{q}\phi(q_{0},\mathbf{q})D_{0}^{-1}(\mathbf{q})\phi(-q_{0},-\mathbf{q})\nonumber\\
&&+ig\!\!\int_{q}\!\phi(q_{0},\mathbf{q})
\psi^{\dag}(\omega+q_{0},\mathbf{k}+\mathbf{q})\sigma_{0}\psi(\omega,\mathbf{k}),\label{Eq_S_Coulomb}
\end{eqnarray}
where the free propagator for auxiliary bosonic field reads
\begin{eqnarray}
D_{0}(\mathbf{q})=\frac{4\pi}{\epsilon\mathbf{q}^2}.\label{Eq_G_phi}
\end{eqnarray}
Here the parameter $\epsilon$ serves as the dielectric constant and $g$ characterizes
the coupling strength between fermion and auxiliary bosonic field.


% Figure environment removed


On the basis of above presentations, we gather the physical elements including the
low-energy fermionic excitations~(\ref{Eq_S_psi}) and phonons~(\ref{Eq_S_action-u}) in conjunction with their
entanglements~(\ref{Eq_S_u-psi})-(\ref{Eq_S_Coulomb}), and eventually arrive at our effective theory as follows
\begin{eqnarray}
S_{\mathrm{eff}}=S_{\psi}+S_{\mathrm{u}}+S_{\mathrm{u}\psi}+S_{\mathrm{Coul}},\label{Eq_eff-action}
\end{eqnarray}
where the involved free propagators for fermion, phonon, and auxiliary bosonic field are presented in
Eq.~(\ref{Eq_G_psi}), Eqs.~(\ref{Eq_G_u-T})-(\ref{Eq_G_u-L}), and Eq.~(\ref{Eq_G_phi}), respectively.
Additionally, the associated tree level vertexes for the interactions
among low-energy fermion and phonons are collected in Fig.~\ref{fig2_interaction-vertex}.
With these in hand, we are suitable to construct all the one-loop diagrams contributing
to the interaction parameters, which are provided in
Appendix~\ref{appendix-one-loop-corrections} for details.
Afterwards, we adopt the effective action~(\ref{Eq_eff-action})
as our starting point to examine the critical fates of 3D tDSM in the low-energy regime
under the influence of Coulomb interaction, electron phonon interaction,
and phonon-phonon interaction.




\section{RG analysis and coupled evolutions}\label{Sec_RGEqs}


Given the type-I 3D tDSM are equipped with unique energy dispersions accompanied by
interesting tiled Dirac cones, it is hereby appropriate to adopt the energy-shell method
to perform the RG analysis~\cite{Wilson1975RMP,Polchinski9210046,Shankar1994RMP}.
This henceforth requires us to integrate an energy shell out one by one during the
process of RG transformations~\cite{Shankar1994RMP,Lee2018PRB,Huh2008PRB,She2010PRB,Wang2011PRB,Qiong2019NPJB}.
To this end, we introduce the useful transformations
and then exploit the Jacobian transformation
to parameterize the tiled energy dispersion~(\ref{Eq dispersion}) as
follows~\cite{Lee2018PRB,Lee2019PRB,Goerbig2008PRB,Qiong2019NPJB}
\begin{eqnarray}
v_zk_{z}&=&-\frac{\zeta}{1-\zeta^2}E+\frac{|E|}{1-\zeta^2}\cos\theta
\label{Eq kz},\\
vk_{x}&=&\frac{|E|}{\sqrt{1-\zeta^2}}\sin\theta\cos\varphi\label{Eq kx},\\
vk_{y}&=&\frac{|E|}{\sqrt{1-\zeta^2}}\sin\theta\sin\varphi\label{Eq ky},\\
\!\!\!\!\!\!\int \!\!dk_{x}dk_{y}dk_{z}\!\!
&=&\!\!\!\!\int\!\!dE\!\!\int^{\pi}_0\!\!\!\!d\theta \!\!\int^{2\pi}_0\!\!\!\!\!\!d\varphi
\frac{E^2\sin\theta(\eta_{E}-\zeta\cos\theta)}{v^2v_z(1-\zeta^2)^2,}\label{Eq E}
%\!\!\!\!\!\!\int \!\!dk_{x}dk_{y}dk_{z}\!\!
%&=&\!\!\!\!\int\!\!dE\!\!\int^{\pi}_0\!\!\!\!d\theta \!\!\int^{2\pi}_0\!\!\!\!\!\!d\varphi
%\frac{\blue{\eta_E}E^2\sin\theta(\eta_{E}-\zeta\cos\theta)}{v^2v_z(1-\zeta^2)^2,}\label{Eq E}
\end{eqnarray}
where $E$ with the $\eta_E$ collecting the signs of $E$ denotes the energy scale
as well as $\theta$ and $\varphi$ are two associated angles. Within the energy-shell framework,
the fast modes of degrees of freedom within the energy shell $\Lambda/b<E<\Lambda$ would
be at first integrated out, where $\Lambda$ is the energy scale and variable parameter $b$ can be
specified as $b=e^{-l}<1$ with $l>0$, and then the renormalized ``slow modes" are obtained, with which
the RG processes can be fulfilled via performing the
RG transformation rescalings~\cite{Wilson1975RMP,Polchinski9210046,Shankar1994RMP}.

% Figure environment removed


To proceed, we are going to obtain the RG rescaling transformations
that are helpful bridges connecting two successive RG steps. Following the
spirit of RG approach~\cite{Wilson1975RMP,Polchinski9210046,Shankar1994RMP},
the free frequency term in the non-interacting action~(\ref{Eq_S_psi}) can be regarded
as the initial fixed point that is invariant under the RG process.
In the collaborations with Eqs.~(\ref{Eq kz})-(\ref{Eq E}), it accordingly
gives rise to the following RG rescalings~\cite{Shankar1994RMP,Wang2011PRB,
Lee2018PRB,Lee2019PRB,Huh2008PRB,She2010PRB},
\begin{eqnarray}
&&\omega\longrightarrow\omega'e^{-l},\label{Eq_RG-rescaling-omega}\\
&&E\longrightarrow E'e^{-l},\\
&&\psi\longrightarrow\psi'e^{\frac{5}{2}l-\eta_\psi l},\\
&&\phi\longrightarrow\phi'e^{3l-\eta_\phi l},\\
&&u_T\longrightarrow u'_Te^{3l-\eta_{u_T}l}\\
&&u_L\longrightarrow u'_Le^{3l-\eta_{u_L}l},\label{Eq_RG-rescaling-u-L}
\end{eqnarray}
where $\eta_\psi$, $\eta_\phi$, $\eta_{u_T}$, and $\eta_{u_L}$
are the anomalous dimensions of related fields that capture the one-loop
corrections due to the all kinds of interactions in our effective
theory~(\ref{Eq_eff-action}). In order to determine them, we resort to
one-loop contributions to the free propagators
as shown in Fig.~\ref{fig3_propagators}. Performing the
straightforward calculations gives rise to
\begin{eqnarray}
\Sigma_{f}(i\omega,\mathbf{k})&=&-\{
[i\omega\mathcal{A}_{0}
-\zeta v_{z}k_{z}\mathcal{A}_{3}]\sigma_{0}
-\chi[\mathcal{A}_{2}v_{z}k_{z}\sigma_{z}\nonumber\\
&&+v\mathcal{A}_{1}(k_{x}\sigma_{x}
+k_{y}\sigma_{y})]\}l,\\
\Sigma_{b}(\mathbf{q})
&=&\frac{-[v^2(q_{x}^2+q_{y}^2)+v_{z}^2q_{z}^2]
g^2l}{4\pi^2v^2v_{z}(1-\zeta^2)},\\
\Sigma_{u_{T,L}}(q_0)
&=&\frac{-5\lambda_{j}\lambda_{i}\delta_{ji}q_{0}^2l}{2\pi^2v^2v_{z}},
\end{eqnarray}
for fermionic, auxiliary bosonic, and phonon propagators, respectively.
Here, the coefficients $\mathcal{A}_{0,1,2}$ are designated in
Eqs.~(\ref{Eq_A_1}), (\ref{Eq_A_2}), and (\ref{Rq_appendix-A0}) of Appendix~\ref{appendix-one-loop-corrections}.
As a result, such one-loop corrections leads to~\cite{Huh2008PRB,She2010PRB,Wang2011PRB}
\begin{eqnarray}
\eta_\psi=-\frac{\mathcal{A}_{0}}{2},\,\,\,\eta_\varphi=0,\,\,\,
\eta_{u^j_L}=\eta_{u^j_T}=-\frac{5\lambda^2_{j}}{8\pi^2v^2v_{z}},\label{Eq_eta}
\end{eqnarray}
with $j=x,y,z$ as aforementioned in Eq.~(\ref{Eq_S_u-psi}).
Subsequently, we need to perform the tedious calculations of one-loop corrections to the interaction
vertexes, which for convenience are presented in Appendix~\ref{appendix-one-loop-corrections}.
After combining the RG rescaling transformations~(\ref{Eq_RG-rescaling-omega})-(\ref{Eq_RG-rescaling-u-L})
with the anomalous dimensions~(\ref{Eq_eta}) and all one-loop corrections in
Appendix~\ref{appendix-one-loop-corrections}, the coupled RG equations of all
interaction parameters are derived as follows~~\cite{Shankar1994RMP,Lee2018PRB,Huh2008PRB,She2010PRB,Wang2011PRB}
\begin{eqnarray}
\frac{dv}{dl}&=&(-\mathcal{A}_{1}-2\eta_\psi)v,\label{Eq_RGEq-v}\\
\frac{dv_z}{dl}&=&(-\mathcal{A}_{2}-2\eta_\psi)v_z,\\
\frac{d\zeta}{dl}&=&(\mathcal{A}_{2}-\mathcal{A}_{3})\zeta,\\
\frac{d \epsilon}{dl}&=&\frac{g^2}{\pi v_{z}(1-\zeta^2)}\epsilon,\\
\frac{d \epsilon_z}{dl}&=&\frac{v_{z}g^2}{\pi  v^2(1-\zeta^2)}\epsilon_z,\\
\frac{d C_{T,L}}{dl}&=&-\eta_{u_x} C_{T,L},\\
\frac{d C^z_{T,L}}{dl}&=&-\eta_{u_z} C^z_{T,L},\\
%\frac{d C_{L}}{dl}&=&-\eta_{u_x} C_{L},\\
%\frac{d C^z_{L}}{dl}&=&-\eta_{u_z} C^z_{L},\\
\frac{dg}{dl}&=&(\mathcal{B}-2\eta_\psi )g,\\
\frac{d\lambda}{dl}&=&(\mathcal{D}-2\eta_\psi -\eta_{u_x})\lambda,\\
\frac{d\lambda_z}{dl}&=&(\mathcal{D}_z-2\eta_\psi -\eta_{u_z})\lambda_z,\\
\frac{dV^{xx}_{T,L}}{dl}&=&(-4\eta_{u_{x}} - \mathcal{F}^{xx}_{T,L})V^{xx}_{T,L},\\
\frac{dV^{zz}_{T,L}}{dl}&=&(-4\eta_{u_{z}} - \mathcal{F}^{zz}_{T,L})V^{zz}_{T,L},\\
\frac{dV^{xy}_{T,L}}{dl}&=&[-(2\eta_{u_x}+2\eta_{u_y}) - \mathcal{F}^{xy}_{T,L}]V^{xy}_{T,L},\\
\frac{dV^{xz}_{T,L}}{dl}&=&[-(2\eta_{u_x}+2\eta_{u_z}) - \mathcal{F}^{xz}_{T,L}]V^{xz}_{T,L},\label{Eq_RGEq-V_T-L}
%\frac{dV^{xx}_L}{dl}&=&(-4\eta_{u_{x}} - \mathcal{F}^{xx}_L)V^{xx}_L,\\
%\frac{dV^{zz}_L}{dl}&=&(-4\eta_{u_{z}} - \mathcal{F}^{zz}_L)V^{zz}_L,\\
%\frac{dV^{xy}_L}{dl}&=&[-(2\eta_{u_x}+2\eta_{u_y}) - \mathcal{F}^{xy}_L]V^{xy}_L,\\
%\frac{dV^{xz}_L}{dl}&=&[-(2\eta_{u_x}+2\eta_{u_z}) - \mathcal{F}^{xz}_L]V^{xz}_L,
\end{eqnarray}
where all the associated coefficients $\mathcal{A}_{1,2,3}$, $\mathcal{B}$,
$\mathcal{D}$, $\mathcal{D}_z$, and $\mathcal{F}_{T,L}$ are designated in Appendix~\ref{appendix-one-loop-corrections}.
It is hereby worth emphasizing that the directions $x$ and $y$ are isotropic as displayed in
Eq.~(\ref{Eq H_0}) and therefore we can neglect several flows of parameters including
$V^{yy}_{T,L}$ and $V^{yz}_{T,L}$ that are similar to their $xx$ and $xz$-component
counterparts to simplify our analysis.

% Figure environment removed



Such above energy-dependent coupled evolutions~(\ref{Eq_RGEq-v})-(\ref{Eq_RGEq-V_T-L})
encompass the low-energy information induced by the interplay among all interactions
in our effective theory. After decoding the physics from them, Fig.~\ref{fig4_lc-PT} schematically exhibits
the underlying properties ranging from the initial state to the lowest-energy limit.
We realize that the competitions among different sorts of interactions result in a
number of unique behaviors with decreasing the energy scale and even trigger some
instability that may drive a phase transition from a 3D tDSM to an $X$ phase
once the critical energy scale denoted by $l_c$ is approached.
The rest of our focus is put on the study of these low-energy consequences.
Specifically, we are going to investigate the tendencies of interaction parameters and their related
implications at $l<l_{c}$ in the upcoming section~\ref{Sec_SM-tend-fates},
and defer the potential instability
at $l\rightarrow l_c$ of Fig.~\ref{fig4_lc-PT} and critical behavior as $l$ approaches $l_c$ to
Sec.~\ref{Sec_FP-instab} and Sec.~\ref{Sec_critical_implications}, respectively.


% Figure environment removed



\section{Tendencies and fates of interaction parameters}\label{Sec_SM-tend-fates}



As aforementioned in Sec.~\ref{Sec_RGEqs}, the coupled RG evolutions encompass all
the low-energy properties of 3D tDSM under the influence of Coulomb interaction as well
as electron-phonon and phonon-phonon interactions. We within this
section are going to study the energy-dependent coupled RG flow
equations and attempt to extract the low-energy behavior of all pertinent parameters
under the intimate competitions of these interactions.


% Figure environment removed

%%% Figure environment removed


% Figure environment removed





\subsection{Fate of $\zeta/\zeta_{0}$ and two unstable scenarios}



Before proceeding further, it is of particular necessity to highlight the effective theory~(\ref{Eq_eff-action}) is
restricted to the type-I tDSM with $\zeta\in (0,1)$. In order to guarantee the RG equations are well-defined,
we accordingly commence with examining the low-energy behavior of tilting parameter $\zeta$. For convenience in future discussions,
we designate the system with $v_{0}<v_{z0}$, $v_{0}>v_{z0}$, $v_{0}=v_{z0}$ as
the Anisotropy-I, Anisotropy-II, and Isotropic cases, respectively.


As discussed in Sec.~\ref{Sec_model}, we hereby put our focus on the $\zeta>0$ case due to
the symmetric consideration. Carrying out the numerical analysis of RG equations~(\ref{Eq_RGEq-v})-(\ref{Eq_RGEq-V_T-L})
gives rise to the basic properties of $\zeta$ in Figs.~\ref{Fig_zeta-1}-\ref{Fig_zeta-2}.
At first, we fix the initial values of fermion velocities and study the evolution of $\zeta$ for
several representative initial values. One can find from Fig.~\ref{Fig_zeta-1}(a) $\zeta/\zeta_{0}$
decreases with lowering energy scales in Anisotropy-I case, and finally arrives at a finite value
that is always less than $1$ at certain critical value. In comparison, Fig.~\ref{Fig_zeta-1}(b) showcases
$\zeta/\zeta_{0}$ in Anisotropy-II initially decreases and then climbs up in the low energy regime,
but remains $\zeta/\zeta_{0}\leq1$ as well. Interestingly, we figure out that the $\zeta/\zeta_{0}$
is relatively less insensitive to the decrease of energy scale for a bigger initial tilting system.
The Isotropic case shows the similar basic results.
As a consequence, the type-I tDSM is robust against the
starting values of fermion velocities and hence the related equations are
valid.


Next, we consider the stability of tilting parameter under different values of fermion velocities.
Given the $\zeta$ is robust with variation of its own initial value for both Anisotropy-I and Anisotropy-II
shown in Fig.~\ref{Fig_zeta-1}, it is suitable to select a fixed $\zeta_0$ and examine the behavior with tuning the
fermion velocities. Fig.~\ref{Fig_zeta-2}(a) signals that in Anisotropy-I case, $\zeta/\zeta_{0}$ falls down and
gradually reaches a finite value within the type-I tDSM. Considering the Anisotropy-II case depicted in Fig.~\ref{Fig_zeta-2}(b),
$\zeta/\zeta_{0}$ bears similarities to that of Anisotropy-I at
$v_{z}/v_{0}<0.4$. However, it tends to grow at $v_{z}/v_{0}=0.4$ and is increased to $1.3$ at $v_{0}/v_{z0}=0.2$.
Even though $\zeta$ is still less than 1 with $\zeta/\zeta_{0}=1.3$, this implies that the $\zeta$ may
own a chance to be $\zeta>1$ and hence go beyond the type-I tDSM.

% Figure environment removed

In order to verify this, we provide Fig~\ref{Fig_zeta-2}(c) including more initial values for Anisotropy-II.
It exhibits that the $\zeta$ is indeed capable of exceeding 1 at $v_{z0}/v_0\approx0.12$, and in particular
the tilting parameter climbs up more rapidly and gains a bigger value while $v_{z0}/v_0$ is smaller than 0.12.
In this sense, we infer that the critical ratio of fermion velocities to trigger the transition from the Type-I tDSM to Type-II tDSM
is in the vicinity of $v_{z0}/v_0\approx0.12$. As schematically illustrated in
Fig.~\ref{Fig_phase}, there in principle exist two possible scenarios for such a transition. In one scenario,
the Type-I tDSM can be directly driven to an $X$ state
at $\emph{l}_{c}$ as illustrated in Fig.~\ref{Fig_phase}(a) under the competition among Coulomb interaction
and electron-phonon coupling as well as phonon-phonon interaction. In the other scenario, Fig.~\ref{Fig_phase}(b)
exhibits alternative version that the Type-I tDSM is first driven to Type-II tDSM at $l_*<l_c$ and then
enters into a $Y$ state.


To recapitulate, the tilting parameter $\zeta$ is robust with respect to its initial value, but relatively sensitive to
the ratio of fermion velocities. In particular, we identify that a critical value of the ratio, $v_{z0}/v_0\approx 0.12$, below which the Type-I tDSM becomes unstable and can potentially transition to the Type-II tDSM. However, our effective theory is confined to the Type-I tDSM and hence RG equations
are only well-defined within the Type-I tDSM, and therefore we from now on will
only consider the first scenario as displayed in Fig.~\ref{Fig_phase}(a) to
investigate the behavior of all other related parameters in the rest of this section
and judge the candidate phase for $X$ state and physical implications in the next two sections.









\subsection{Fates of fermion velocities}

Within the Type-I tDSM, we begin with studying the effects of interactions on fermion velocities.
By fixing their initial ratio, Fig.~\ref{Fig_vz-1} illustrates the basic tendencies
of the anisotropy of fermion velocities $v_{z}/v$. Starting from the Anisotropy-I, one can read from Fig.~\ref{Fig_vz-1}(a)
that $v_{z}/v$ gradually goes up as lowering the energy scale and the weaker tilting parameter
is preferable to support its increase. In sharp contrast, Fig.~\ref{Fig_vz-1}(b) presents that
the ratio $v_{z}/v$ that is away from the Anisotropy-II
receives a slight increase at first but falls to a certain value when the energy scale is lower enough.
Besides, both of its evolution and the final value are fairly insusceptible to the tilting parameter, which
is apparently distinct from its Anisotropy-I counterpart.
This implies that the qualitative behavior of anisotropy of fermion velocities largely relies on its beginning value compared to the strength of tilting parameter.

% Figure environment removed


It is therefore necessary to further verify its low-energy fate via tuning the initial condition of anisotropy. As shown in Fig.~\ref{Fig_vz-2} with varying the starting values of $v/v_z$, we
choose a representative tilting parameter $\zeta_{0}=0.5$ to show the energy-dependent
anisotropy of fermion velocities for both Anisotropy-I and Anisotropy-II.
Specifically, for the Anisotropy-I, the anisotropy of fermion velocities $v_{z}/v$
is heavily dependent upon the starting anisotropy. As displayed in Fig.~\ref{Fig_vz-2}(a),
$v_{z}/v$ is insusceptible to the energy scales and becomes relatively stable in the presence of
a weak starting anisotropy, but instead sensitive and rapidly grows with a strong initial anisotropy.
In comparison, Fig.~\ref{Fig_vz-2}(b) indicates that the $v_{z}/v$, departing from Anisotropy-II, can only receive slight negative corrections and even become saturated once the starting anisotropy is adequate weak.
Therefore, we figure that the fate of  $v_{z}/v$ primarily hinges upon the initial anisotropy,
which is sensitive to the value of $v_0/v_{z0}$ and $\zeta_0$ for starting from
Anisotropy-I and Anisotropy-II, respectively.





% Figure environment removed


% Figure environment removed


\subsection{Fates of $\epsilon_z/\epsilon$ and $g/g_0$}


Subsequently, we move to study the impacts of coupled interactions on the dielectric
constant that is an important quantity to measure the strength of Coulomb interactions.
Given the focus is put on a tilted 3D DSM,
we are more interested in examining the ratio of dielectric constant between different orientations.
By selecting a specific initial ratio of fermion velocities, Fig.~\ref{Fig_epsilon-1}
clearly displays that in the case of Anisotropy-I ($v_{0}/v_{z0}=0.5$), $\epsilon_{z}/\epsilon$
quickly climbs up, whereas in Anisotropy-II ($v_{z0}/v_{0}=0.5$), it rapidly falls down to zero.
Additionally, although these basic results are robust to the change of the initial value
of tilting parameter $\zeta_{0}$,
it is worth pointing out that a bigger $\zeta_0$ is much more helpful to enhance the critical value
of $\epsilon_{z}/\epsilon$ in Anisotropy-I and increase the critical energy scale that is inversely proportional to $l_c$ in Anisotropy-II, respectively.

In addition to checking the variation of tilting parameter, we also examine the stability of $\epsilon_{z}/\epsilon$
via tuning the initial anisotropy of fermion velocities. One can notice from Fig.~\ref{Fig_epsilon-2} that
$\epsilon_{z}/\epsilon$ exhibits a similar trend under the influence of
$v_0/v_{z0}$ as it does under the effects of $\zeta_0$ in Fig.~\ref{Fig_epsilon-1}. In other words,
as lowering the energy scale, it increases quickly and progressively vanishes while starting from
the Anisotropy-I and Anisotropy-II, respectively. Besides, a stronger anisotropy of fermion
velocities leads to a more quick increase or decrease.

% Figure environment removed

To wrap up, the dielectric constant evolves towards the strong anisotropy in the low-energy
due to the competition among various kinds of interactions. Depending on the departure from Anisotropy-I or
Anisotropy-II, it can either be driven to $\epsilon_z/\epsilon\gg1$
or $\epsilon_z/\epsilon\ll1$.  As the strength of Coulomb interaction is inversely
proportional to the dielectric constant, this indicates that the Coulomb interaction in the direction-$z$ or direction-$x,y$ would be
considerably screened in the low-energy regime. Before going further, it is worth emphasizing that the
basic behavior of $g/g_0$ that specifies the coupling strength between fermion and auxiliary bosonic field
is analogous to that of $\epsilon_z/\epsilon$ and hence not shown for brevity.

% Figure environment removed


% Figure environment removed



% Figure environment removed





\subsection{Fates of phonon velocities and $\lambda_z/\lambda$}


Next, we consider how the coupled interactions affect the phonon velocities.
For the sake of simplicity, the focus is primarily put on the velocities of transverse phonons as
their longitudinal counterparts show the similar behavior owing to the analogous structures of
RG equations addressed in Sec.~\ref{Sec_RGEqs}.

With decreasing the energy scale, Fig.~\ref{Fig_CTz-1} with fixed fermion velocities
presents the basic tendency of $C_{T}^{z}/C_{T}$ for the departure from Anisotropy-I.
One can notices from Fig.~\ref{Fig_CTz-1}(a) that $C_{T}^{z}/C_{T}$ slowly climbs up and goes towards the divergence as approaching a critical point. It is also worth pointing out that
the qualitative results are relatively independent of the initial
value of tilting parameter. Accordingly, we then choose a specific tilting parameter $\zeta_0=0.5$ to
investigate the impact of the variation of initial anisotropy of fermion velocities as
displayed in Fig.~\ref{Fig_CTz-1}(b). We notice that the $C_{T}^{z}/C_{T}$ is unambiguously prone to divergence at a critical energy scale that is increased by tuning down $v_0/v_{z0}$, as long as the initial ratio of fermion velocities
is below a critical value around 0.8. Otherwise, it is preferable to reach a finite
value that is slightly away from the isotropic case. This signals that the $C_{T}^{z}/C_{T}$ prefer
to go towards an extreme anisotropy, indicating the $z$ component of phonon velocity plays a dominant role,
in contrast to $x$ or $y$ component.


Compared to the case starting from Anisotropy-I, we are ware that the phonon velocities
exhibit much more interesting behavior as depicted in Fig.~\ref{Fig_CTz-2} for Anisotropy-II,
which heavily hinge upon both initial values of the fermion velocities and the tilting parameter.
Colloquially, it is unequivocal to learn from Fig.~\ref{Fig_CTz-2} that the low-energy fate
of $C_T^z/C_T$ is principally governed by the ferocious competition between $v_{z0}/v_0$ and $\zeta_0$
reflected in the coupled RG equations. To be concrete, once $v_{z0}/v_0$ is smaller than a
critical value ($\approx0.31$), one can clearly find from Fig.~\ref{Fig_CTz-2}(a) that it plays a leading role and drives $C_T^z/C_T$ to be an extreme anisotropy $C_T^z/C_T\rightarrow0$ at the lowest-energy limit, which is particularly robust against the value of $\zeta_0$. On the contrary, Fig.~\ref{Fig_CTz-2}(c) shows that
at a moderate initial value, the contribution of $v_{z0}/v_0$ plays a crucial role, leading to
another extreme anisotropy $C_T^z/C_T\rightarrow\infty$ upon reaching the critical energy scale.
Barring these, when $v_{z0}/v_0$ is around $v_{z0}/v_0\in[0.30,0.31]$ as presented
in Fig.~\ref{Fig_CTz-2}(b), we notice that the influence of $v_{z0}/v_0$ is subordinate to
the tilting parameter, indicating that $v_{z0}/v_0$ can either flow towards $C_T^z/C_T\rightarrow0$
or $C_T^z/C_T\rightarrow\infty$ by varying the value of $\zeta_0$.
It is also worth noting that $C_T^z/C_T$ in the context of a larger $v_{z0}/v_0>0.75$ only receives a slight
deviation from isotropy due to the concomitant effects of both $v_{z0}/v_0$ and $\zeta_0$.

To reiterate, the coupled RG equations in tandem with
the intimate competition between initial values of the fermion velocities and the tilting parameter
can drive the system towards an extreme anisotropy where $C_T^z/C_T$ approaches either zero or infinity,
or towards near-isotropy where $C_T^z/C_T$ is approximately one. In consequence, this signals that as the
energy scale decreases, phonon can exhibit distinct behaviors, such as playing a dominant role
in the $z$ direction or $xy$ plane as well as equivalent in all directions. In addition, we find that
the qualitative behavior of $\lambda_z/\lambda$ bears the similarity to that of $C_T^z/C_T$ and
hence not shown hereby for brevity.


% Figure environment removed



\subsection{Fates of phonon-phonon interactions}


At last, we move our attention to the low-energy tendencies of phonon-phonon interactions under the
influence of all interactions in our theory.

To fully understand the phonon-phonon interactions in this system,
we must analyze the behavior of various quantities arising
from all components of the phonon-phonon couplings~(\ref{Eq_S_action-u}).
$V_T^{xy}/V_T^{zz}$, $V_T^{xz}/V_T^{zz}$, $V_L^{xx}/V_L^{zz}$, $V_L^{xy}/V_L^{zz}$,
$V_L^{xz}/V_L^{zz}$, $V_T^{xx}/V_L^{xx}$, $V_T^{xy}/V_L^{xy}$, $V_T^{xz}/V_L^{xz}$, and $V_T^{zz}/V_L^{zz}$,
where the indexes $T$ and $L$ denote the transverse and longitudinal phonons, respectively.
It is obviously a challenging task to deal with them one by one. To simplify the our analysis, we can group them into three
categories: the transverse-phonon parts ($V_T^{xx}/V_T^{zz}$,
$V_T^{xy}/V_T^{zz}$, $V_T^{xz}/V_T^{zz}$), the longitudinal-phonon parts ($V_L^{xx}/V_L^{zz}$,
$V_L^{xy}/V_L^{zz}$, $V_L^{xz}/V_L^{zz}$ ), and the mixed parts ($V_T^{xy}/V_L^{xy}$,
$V_T^{xz}/V_L^{xz}$, $V_T^{zz}/V_L^{zz}$). Fortunately, the numerical analysis shows that
the members within the each category feature the analogous tendencies
as approaching the critical energy scales. Accordingly, we can adequately represent each category with a
single representative behavior.

Regarding the transverse-phonon parts, we choose to consider the behavior of $V_T^{xx}/V_T^{zz}$.
Starting from Anisotropy-I of fermion velocities, Fig.~\ref{Fig_VTxxVTzz-1}(a) manifestly
show that $V_T^{xx}/V_T^{zz}$ slowly decreases as $l$ increases, but then rapidly
goes towards zero nearby the critical energy. This indicates the $z$-component
phonon interaction becomes prevailingly dominant compared to that of $x,y$ components.
Compared to the initial value of tilting parameter $\zeta_0$, which only contributes minor corrections to
the basic tendencies, the critical energy scales can be considerably lowered with increasing the value of $v_0/v_{z0}$.
In this sense, $V_T^{xx}/V_T^{zz}$ cannot be driven to the extreme anisotropy $V_T^{xx}/V_T^{zz}\rightarrow0$ but eventually
flows to a slight anisotropy as shown in Fig.~\ref{Fig_VTxxVTzz-1}(b) at $v_0/v_{z0}=0.8$.
This suggests that the anisotropy of the fermion velocities play
a much more crucial role in pinning down the fates of
phonon interactions than the tilting parameter.

In comparison, beginning with the the Anisotropy-II of fermion velocities,
one can find from Fig.~\ref{Fig_VTxxVTzz-2} that $V_T^{xx}/V_T^{zz}$ bears similarities to
the behavior of phonon velocities discussed in the previous subsection.
If $v_{z0}/v_0$ is below a critical value ($\approx0.31$) or
takes a moderate initial value, it wins the competition with $\zeta_0$
and drives the system to an extreme anisotropy ($V_T^{xx}/V_T^{zz}\rightarrow\infty$ for the former
and $V_T^{xx}/V_T^{zz}\rightarrow0$ for the latter, respectively).
In addition, an adequate big value of $v_{z0}/v_0$ presents the anisotropy but makes
$V_T^{xx}/V_T^{zz}$ nearly isotropic as shown in Fig.~\ref{Fig_VTxxVTzz-2}(d).
Otherwise, Fig.~\ref{Fig_VTxxVTzz-2}(b) shows that the influence of $v_{z0}/v_0$ is subordinate to $\zeta_0$
around the critical value $v_{z0}/v_0\approx0.31$, which causes $V_T^{xx}/V_T^{zz}$ either
to an extreme anisotropy $V_T^{xx}/V_T^{zz}\rightarrow\infty$ or $V_T^{xx}/V_T^{zz}\rightarrow0$.

%% Figure environment removed


Since the the longitudinal-phonon parts share the similar fates with their transverse-phonon counterparts and henceforth not shown hereby for brevity. Subsequently, we concentrate on the mixed parts. As they possess the similar basic tendencies,
we choose to present the behavior of $V_T^{xx}/V_L^{xx}$ as shown in Fig.~\ref{Fig_VTxxVLxx-1}.
%and Fig.~\ref{Fig_VTxxVLxx-2}.



From Fig.~\ref{Fig_VTxxVLxx-1}(a) with fixing $v_0/v_{z0}=0.5$,
it can be inferred that in the scenario starting from the Anisotropy-I,
$V_T^{xx}/V_L^{xx}$ deviates a little from the isotropy as $l$ increases
and reaches a finite value at the critical energy scale. This indicates
transverse-phonon and longitudinal-phonon interactions are nearly contribute
equally and the basic results are similar upon varying the initial conditions.
In contrast, we notice that, when beginning with the Anisotropy-II, $V_T^{xx}/V_L^{xx}$
is heavily dependent upon the initial conditions as depicted Fig.~\ref{Fig_VTxxVLxx-1}(b)
for $v_0/v_{z0}=0.5$.
It decreases much more than its Anisotropy-I counterpart,
and in particular, it gains a big drop at some optional $\zeta_0\approx0.3$ or
$v_{z0}/v_0\approx0.6$, where it is driven to a strong anisotropy. Accordingly, once
we start from the Anisotropy-II, the longitudinal-phonon interactions play a more
significant role than transverse-phonon interactions. Additionally, we have also checked that
these results are relatively stable against the change of initial ratio of fermion velocities


To wrap up, we figure out that the initial anisotropy of the fermion velocities plays a
much more crucial role in determining the fates of phonon interactions. The transverse-phonon contributions
are subordinate to their longitudinal-phonon counterparts for the Anisotropy-II case, and
as to the Anisotropy-I situations, both of them are nearly equivalent.



\section{Potential instabilities}\label{Sec_FP-instab}



As presented in previous section, the interaction
parameters display a number of interesting behavior that stems from
the RG equations~(\ref{Eq_RGEq-v})-(\ref{Eq_RGEq-V_T-L}) ranging from
$l=0$ to $l\rightarrow l^{-}_c$ shown in Fig.~\ref{fig4_lc-PT} and
Fig.~\ref{Fig_phase}(a). The values of such parameters at $l_c$ construct the fixed point (FP)
in the phase space~\cite{Shankar1994RMP},
at which the potential instability from the 3D tDSM to an $X$ phase may be induced.
%\blue{Hereby, the $l_c$ ($T_c$) specifies the critical
%energy scale (critical temperature) and ``PT" is associated with the phase transition
%induced by the potential instability from the 3D tDSM to an $X$ phase. }
%\blue{Under this circumstance,
%we are in a suitable position within this section to investigate and judge which phase is
%the most preferable once certain instability is triggered among several kinds of
%potential candidates.}
After examining the behavior of parameters shown in Sec.~\ref{Sec_SM-tend-fates},
we find that the tendencies of FPs are heavily dependent upon the behavior of
$\zeta/\zeta_{0}$, $ v_{z}/v $, $\varepsilon_{z}/\varepsilon $, $ C_{T}^{Z}/C_{T} $,
and $V_{T}^{XX}/V_{T}^{ZZ}$, and can be qualitatively clustered into three distinct types of FPs
as presented in Table~\ref{Table_classify_FPs}. To simplify our analysis, we subsequently only focus on these three kinds
of FPs and endeavor to seek the most preferable instabilities around them.


\begin{table}
\caption{Collections of three distinct types of FPs on the basis of tendencies of interaction parameters.}
\vspace{0.3cm}
\centering{
\renewcommand\arraystretch{2}
\begin{tabular}{c|c|c|c|c|c}
\hline
\hline
 Types & $\zeta/\zeta_{0}$ & $ v_{z}/v $ & $\varepsilon_{z}/\varepsilon $ & $ C_{T}^{Z}/C_{T} $ & $V_{T}^{XX}/V_{T}^{ZZ}$ \\
\hline
 Type-I & $\red{\uparrow}$ & $\blue{\downarrow}$ & $\blue{\downarrow}$ & $\red{\uparrow}$ & $\blue{\downarrow}$ \\
\hline
 Type-II & $\red{\uparrow}$ & $\blue{\downarrow}$ & $\blue{\downarrow}$ & $\blue{\downarrow}$ & $\red{\uparrow}$ \\
\hline
 Type-III & $\blue{\downarrow}$ & $\red{\uparrow}$ & $\red{\uparrow}$ & $\red{\uparrow}$ & $\blue{\downarrow}$ \\
\hline
\hline
\end{tabular}}\label{Table_classify_FPs}
\end{table}




% Figure environment removed



Generally, the instability is an indicator of certain symmetry breaking with the development of
some fermionic bilinear~\cite{Maiti2010PRB,Halboth2000RPL,Halboth2000RPB,Nandkishore2012NP,
Cvetkovic2012PRB,Wang2020NPB}. In order to investigate the potential instability around the FPs,
we then introduce the following source terms to denote the potential phase $X$~\cite{Vafek2010PRB,Murray2014PRB,Roy2009.05055}
\begin{eqnarray}
S_{\mathrm{sou}}
&=&\int d\tau\int d^{2}\mathbf{x}
\Big\{\sum_{i=1}\Delta_{i}^{\mathrm{PH}}\psi^{\dag}\mathcal{M}_{i}^{\mathrm{PH}}\psi\nonumber\\
&&+\sum_{i=1}\left[\Delta_{i}^{\mathrm{PP}}\psi^\dagger\mathcal{M}_{i}^{\mathrm{PP}}\psi^{\ast}+h.c.\right]\Big\},\label{Eq_source-terms}
\end{eqnarray}
where $\mathcal{M}^{\mathrm{PH/PP}}_{i}$ respectively serve as the related matrices associated with
the fermionic bilinears in particle-hole and particle-particle channels, and
$\Delta^{\mathrm{PH/PP}}_{i}$ specify the strengths of the corresponding source terms.
As to our model, the primary candidates of instability and phase transition are exhibited in
Table.~\ref{Table_candidate-phases}~\cite{Roy2018RRX,Ruhman2019PRX}
To proceed, we add the source terms~(\ref{Eq_source-terms}) into the effective
action~(\ref{Eq_eff-action}) and calculate the
one-loop corrections to $\Delta^{\mathrm{PH/PP}}_{i}$, with which the related RG
equations for $\Delta^{\mathrm{PH/PP}}_{i}$ can be derived and compactly written as
\begin{eqnarray}
\frac{d\Delta^{\mathrm{PH/PP}}_{i}}{dl}=\mathcal{P}(\Delta^{\mathrm{PH/PP}}_{i},v,v_{z}...),
\end{eqnarray}
where the detailed expressions for $\mathcal{P}(\Delta^{\mathrm{PH/PP}}_{i},v,v_{z}...)$
are provided in Appendix~\ref{appendix-source-terms}.
Accordingly, we are now in a suitable position to evaluate the susceptibilities accompanied by the
source terms by resorting to the relationship~\cite{Vafek2010PRB,Cvetkovic2012PRB,Murray2014PRB,Zhai2021NPB}
\begin{eqnarray}
\delta\chi=\frac{\partial^2 f}{\partial\Delta(0)\partial\Delta^*(0)}
\end{eqnarray}
where $f$ denotes the free energy density. With all these in hand, we are capable of studying
the susceptibilities of all candidates phases to determine the
leading phase for phase $X$, which owns the strongest divergence of susceptibility~\cite{Cvetkovic2012PRB}.


\begin{table}
\caption{Potential candidates for instabilities triggered by the all interactions~\cite{Roy2018RRX,Ruhman2019PRX}.
Hereby, SC, AP and CDW are abbreviations for superconductivity, anisotropy parameter and charge density wave, respectively.
In addition, chiral $\textrm{SC}_1$ and $\textrm{SC}_2$ specify two distinct sorts of chiral superconducting states.}
\vspace{0.3cm}
\centering{
\renewcommand\arraystretch{2}
\begin{tabular}{c|c|c}
\hline
\hline
 Order parameters & Fermionic bilinears & Potential phases\\
\hline
 $\Delta_{0}^{PH}$ & $\mathcal{M}_{0}^{PH}=\sigma_{0}\tau_{0}$ & density\\
\hline
 $\Delta_{1}^{PH}$ & $\mathcal{M}_{1}^{PH}=\sigma_{0}\tau_{1}$ & x-current\\
\hline
 $\Delta_{2}^{PH}$ & $\mathcal{M}_{2}^{PH}=\sigma_{0}\tau_{2}$ & AP\\
\hline
 $\Delta_{3}^{PH}$ & $\mathcal{M}_{3}^{PH}=\sigma_{0}\tau_{3}$ & CDW\\
\hline
 $\Delta_{0}^{PP}$ & $\mathcal{M}_{0}^{PP}=\sigma_{2}\tau_{3}$ & s-wave SC\\
\hline
 $\Delta_{1}^{PP}$ & $\mathcal{M}_{1}^{PP}=\sigma_{2}\tau_{1}$ & chiral\ $\textrm{SC}_{1}$\\
\hline
 $\Delta_{2}^{PP}$ & $\mathcal{M}_{2}^{PP}=\sigma_{2}\tau_{0}$ & chiral\ $\textrm{SC}_{2}$\\
\hline
 $\Delta_{0(0,1,3)}^{PP}$ & $\mathcal{M}_{3i}^{PP}=\sigma_{(0,1,3)}\tau_{2}$ & triplet SC\\
\hline
\hline
\end{tabular}\label{Table_candidate-phases}
}
\end{table}



To proceed, we notice from Eqs.~(\ref{Eq_Delta_1})-(\ref{Eq_Delta_2}) characterizing the strengths of source terms
that certain candidates in Table~\ref{Table_candidate-phases} are degenerate to one-loop level. As a consequence,
it is convenient to divide them into four cases: Phase-A (density, $x$-current, AP or CDW), Phase-B
($s$-wave SC or tiplet $\mathrm{SC}_0$), Phase-C (chiral $\textrm{SC}_{1}$ or tiplet $\mathrm{SC}_1$)
and Phase-D (chiral $\textrm{SC}_{2}$ or tiplet $\mathrm{SC}_3$), respectively.
After performing the numerical analysis by combining the RG equations of both interaction parameters and $\Delta^{\mathrm{PH/PP}}_{i}$,
we obtain the energy-dependent susceptibilities of all four cases, as shown in
Fig.~\ref{Fig_four-chi_three-types} for three kinds FPs mentioned in Table-I.
Learning from Fig.~\ref{Fig_four-chi_three-types} for several representative initial conditions, the
basic results as approaching all three distinct kinds of FPs are obtained.
At first, it is clear that Phase-C is the leading instability once the system is driven to the type-II FP,
indicating that chiral $\textrm{SC}_{1}$ or tiplet $\mathrm{SC}_1$ becomes the dominant phase and henceforth the
best candidate for the phase $X$ displayed in Fig.~\ref{Fig_phase}(a).
In comparison, although Phase-C is not the overwhelming choice among other phases around the type-I and type-III FPs,
it is still the optimal sate for the phase $X$.  These results are
in qualitative agreement with the results in Ref.~\cite{Ruhman2019PRX}.



In this sense, our analysis suggests that there indeed exists some interaction-driven phase transition. Accompanying the
certain instability, there may result in critical physical implications, which we will deliver in the next section.





% Figure environment removed





\section{Critical implications around the instabilities}\label{Sec_critical_implications}

To proceed, we within this section examine the critical behavior of physical obversables including the
density of states (DOS), compressibility, and specific heat, as
approaching three kinds of potential instabilities that are classified in Table~\ref{Table_classify_FPs}
of Sec.~\ref{Sec_FP-instab}.

In principle, it is a very challenging task to derive the analytical expressions for
physical quantities directly from an interacting theory. Instead of delving into the exact
expressions, a suitable and operational strategy is to extract the physical implications
from the renormalized fermionic propagator~\cite{Mahan1990Book}.
Compared to their free counterparts in Eq.~(\ref{Eq_G_psi}), the fermion velocities and tilting parameter  are involved in the
coupled RG equations~(\ref{Eq_RGEq-v})-(\ref{Eq_RGEq-V_T-L}) and henceforth become energy-dependent,
which inherit the features of the interactions and effectively capture the basic tendencies as accessing the potential instabilities.
In the light of this strategy, we subsequently need to construct the
relationship between the physical implications and fermion velocities as well as tilting parameter.



\subsection{Density of states and compressibility}\label{Subsection_DOS-kappa}

At first, we consider the DOS and compressibility. After performing the analytical continuation,
the renormalized retarded fermion propagator is expressed as~\cite{Wang2012PRD}
%\begin{eqnarray}
%G^{\mathrm{ret}}(\omega,\mathbf{k})
%=\frac{(\omega - \zeta v_z k_z)\sigma_0
%+
%\chi[v_z k_z\sigma_z + v(k_x\sigma_x + k_y\sigma_y)]}
%{\omega^2+i\mathrm{sgn}(\omega-\zeta v_zk_z)\delta - 2\omega \zeta v_z k_z +\zeta^2 v_z^2 k_z^2
%- \chi^2\left[v_z^2k_z^2+v^2(k_x^2+k_y^2)\right]},
%\end{eqnarray}
\begin{eqnarray}
G^{\mathrm{ret}}(\omega,\mathbf{k})
&=&\{(\omega - \zeta v_z k_z)\sigma_0
+\chi[v_z k_z\sigma_z + v(k_x\sigma_x \nonumber\\
&&+ k_y\sigma_y)]\}/\{\omega^2+i\mathrm{sgn}(\omega-\zeta v_zk_z)\delta +\zeta^2 v_z^2 k_z^2\nonumber\\
&& - 2\omega \zeta v_z k_z-[v_z^2k_z^2+v^2(k_x^2+k_y^2)]\},
\end{eqnarray}
where $v,v_z$ and $\zeta$ are regarded to be energy-dependent. The DOS of fermion quasiparticles then
takes the form of
\begin{eqnarray}
\frac{\rho_{\mathrm{int}}(\omega)}{\Lambda^2_0}
&=&\int\frac{d^3\mathbf{k}}{(2\pi)^3}
\mathrm{Tr}\left\{-\frac{1}{\pi}
\mathrm{Im}\left[G^{\mathrm{ret}}(\omega,\mathbf{k})\right]\right\}.
\end{eqnarray}
Carrying out some calculations, we then obtain
\begin{widetext}
\begin{small}
\begin{numcases}{\frac{\rho_{\mathrm{int}}(\omega)}{\Lambda^2_0}=}
\int_{E,\theta,\omega}\left[\left|\omega - \frac{\zeta E(-\zeta+\cos\theta)}
{1-\zeta^{2}}\right|
\delta(E-\omega)+
\left|\omega + \frac{\zeta E(\zeta+\cos\theta)}
{\zeta^{2}-1}\right|
\delta\left(E-\frac{(1-\zeta^{2})\omega}
{1+\zeta^{2}+2\zeta \cos\theta}\right)\right],\!\!\!\!\!&
$\omega>0$,\label{Eq_rho_int_1}\\
~\nonumber\\
-\int_{E,\theta,\omega}
\left[\left|\omega + \frac{\zeta (\zeta-\cos\theta)E}
{1-\zeta^{2}}\right|
\delta\left(E-\frac{(\zeta^{2}-1)\omega}
{1+\zeta^{2}-2\zeta\cos\theta}\right)+\left|\omega - \frac{\zeta(\zeta+\cos\theta)E}
{1-\zeta^{2}}\right|
\delta\left(E+\omega\right)\right],\!\!\!\!\! \!\!\!\!&
$\omega<0$, \label{Eq_rho_int_2}
\end{numcases}
\end{small}
\end{widetext}
with $N$ the flavor of fermions and the $\int_{E,\theta,\omega}$ being designated as
\begin{eqnarray}
\int_{E,\theta,\omega}\equiv
\frac{N}{(2\pi)^2}
\int_{e^{-l_c}}^{1} dE
\int_0^\pi d\theta
\frac{E^2\sin\theta}
{v^{2} v_z (1-\zeta^{2})\omega}.
\end{eqnarray}
They can reduce to
\begin{eqnarray}
\frac{\rho_0(\omega)}{\Lambda^2_0}
=\frac{N\omega^2}{\pi^2v^2 v_z(1-\zeta^2)^2}.
\end{eqnarray}
at the noninteracting case.
%$0<\frac{(1-\zeta^2)\omega}{1+\zeta^2+2\zeta\cos\theta}<1$ for $\omega>0$, as well as
%$0<\frac{(\zeta^2-1)\omega}{1+\zeta^2-2\zeta\cos\theta}<1$ with $\omega<0$.

As to the compressibility, it originally is defined as $\kappa=\partial V/\partial F$~\cite{Schwabl2006Book},
where $V$ and $F$ are volume and compression force, respectively. Hereby, it is more convenient
to introduce the chemical potential $\mu$ and then calculate
by means of $\kappa=\partial n/\partial \mu$, where $n$ is the number of particles per area which
is directly associated with the DOS~\cite{Mahan1990Book,Sarma2007PRL}. After some calculations, we are left with
\begin{widetext}
\begin{small}
\begin{numcases}{\frac{\kappa_{\mathrm{int}}(\mu)}{\Lambda_0}= \!\!\!}
\!\!\!
\int_{E,\theta,\mu}
\left[\left|
\frac{\zeta(\zeta-\cos\theta)E}{1-\zeta^{2}}-2\mu\right|
\delta\left(E-\frac{2(1-
\zeta^{2})\mu}{1+\zeta^{2}-2\zeta \cos\theta}\right)+\left|2\mu+
\frac{\zeta (\zeta +\cos\theta)E}{1-\zeta^{2}}\right|
\delta\left(E-2\mu\right)\right],\!\!\!\!\!&
$\mu>0$,\label{Eq_kappa_int_1}\\
~\nonumber\\
\!\!\!
\int_{E,\theta,\mu}
\left[\left|
\frac{\zeta(\zeta-\cos\theta)E}{1-\zeta^{2}}-2\mu\right|
\delta\left(E+2\mu\right)+\left|2\mu+
\frac{\zeta (\zeta +\cos\theta)E}{1-\zeta^{2}}\right|
\delta\left(E+\frac{2(1-\zeta^{2})\mu}
{1+\zeta^{2}+2\zeta \cos\theta}\right)\right],\!\!\!\!\!\!\!\!\!\!\!\!\!\!\! &
$\mu<0$, \label{Eq_kappa_int_2}
\end{numcases}
\end{small}
\end{widetext}
where $\int_{E,\theta,\mu}$ is introduced as
\begin{eqnarray}
\int_{E,\theta,\mu}\equiv\frac{N}{8\pi^2}
\int_{e^{-l_c}}^{1} dE \int_0^\pi d\theta
\frac{E^2\sin\theta}{v^{2}v_z(1-\zeta^{2})\mu}.
\end{eqnarray}
Similarly, one can obtain its free limit expression via taking $l_c\rightarrow\infty$ and
$v,v_z,\zeta$ being the constants.


% Figure environment removed




With above information in hand, we can study the behaviors of DOS and $\kappa$ influenced by the interactions.
Combining Eqs.~(\ref{Eq_rho_int_1})-(\ref{Eq_rho_int_2}), (\ref{Eq_kappa_int_1})-(\ref{Eq_kappa_int_2}) and RG equations~(\ref{Eq_RGEq-v})-(\ref{Eq_RGEq-V_T-L}), their basic tendencies are obtained
as accessing three distinct kinds of instabilities.

We collect the behavior of DOS in Fig.~\ref{Fig_DOS-all-omega-0.5}. In the absence of interactions and
tilting parameter, the DOS exhibits a parabolic frequency dependence, $\rho_0(\omega)\propto\omega^2$,
and hence vanishes precisely at the Dirac point. In sharp contrast, as approaching three types of instabilities,
the interaction contributions coax the DOS to become a nonzero finite value at $\omega=0$, which is qualitatively
distinct from the free case, and the nonzero tilting parameter make it asymmetric
for $\omega>0$ and $\omega<0$. Away from $\omega=0$, the intimate competition between the interactions and
thermal fluctuations governs the behavior of DOS. We can find that it decreases at first and then climbs up at
$|\omega|\gtrsim\omega_c\approx0.25$. This indicates that below the critical frequency
$\omega_c$, the thermal fluctuations are subordinate to the interactions,
while above it, the thermal fluctuations dominate. In addition, although the DOS near all
three types of instabilities exhibit similar tendencies,
the type-II instability leads to a little more corrections compared to the other two types.



As to the compressibility dubbed $\kappa$,
Fig.~\ref{Fig_kappa-mu-0.5} shows the $\mu$-dependent evolutions of $\kappa$ in the vicinity of
three types of instabilities. In the free case, $\kappa_{\mathrm{free},\zeta=0}(\mu)$ displays a
nearly linear dependence on $\mu$ and become incompressible at $\mu=0$, which is consistent with the behavior
of free DOS. However, once the contributions from the interactions and tilting parameter are taken into account,
the behavior of $\kappa$ is qualitatively changed for $\mu\leq\mu_c\approx0.25$. It becomes much compressible and
reaches its maximus value at $\mu=0$. Then, the compressibility increases at $\mu>\mu_c$ and exhibits the
similar tendency in the free case for sufficiently large $\mu$, which can be ascribed to the
close interplay between interactions and thermal fluctuations. Besides, we notice that $\kappa$
gains more corrections nearby the type-II instability, and it appears that the tilting parameter
only bring minor corrections to the opposite signs of $\kappa$.









\subsection{Specific heat}

Next, we move to the specific heat of quasipartcicle.
The free energy density $f(T)$ can be written as
\begin{eqnarray}
f(T)
&=&-\frac{T}{V}\ln Z,
\end{eqnarray}
where the partition function is associated with~\cite{Kapusta1994Book}
\begin{eqnarray}
Z&=&
\prod_{n,k,\alpha}
\int \left[d(i\psi_{\alpha,n}^\dag)\right]
\left[d\psi_{\rho,n}\right]e^{S_0},
\end{eqnarray}
with $S_0$ denoting the fermionic part of our theory. After long but straightforward calculations,
we finally obtain
\begin{widetext}
\begin{eqnarray}
\frac{f(T)}{\Lambda_0^4}
&=&-\frac{T}{4\pi^2}
\int_{e^{-l_c}}^{1}dE
\int_0^\pi d\theta
\frac{E^2\sin\theta(1-\zeta \cos\theta)}
{v^{2} v_z (1-\zeta^{2})^2}\Bigg[\ln(1+e^{-\sqrt{x^+}})+\ln(1+e^{-\sqrt{y^+}})\Bigg]\nonumber\\
&&-
\frac{T}{4\pi^2}
\int_{e^{-l_c}}^{1}dE
\int_0^\pi d\theta
\frac{E^2\sin\theta(1+\zeta \cos\theta)}
{v^{2} v_z (1-\zeta^{2})^2}\Bigg[\ln(1+e^{-\sqrt{x^-}})+\ln(1+e^{-\sqrt{y^-}})\Bigg].
\end{eqnarray}
\end{widetext}
where the transformations $E\rightarrow E/\Lambda_0$ are used and
$T\rightarrow T/\Lambda_0$ and $x^{\pm},y^{\pm}$ are defined as
\begin{eqnarray}
x^+
&\equiv&
\frac{\mathcal{N}_{-}E^2}{(1-\zeta^2)^2T^2},\\
y^+
&\equiv&
\frac{(\zeta^2+1-2\zeta\cos\theta)^2E^2}{\mathcal{N}_{-}T^2},\\
x^-
&\equiv&\frac{\mathcal{N}_{+}E^2}{(1-\zeta^2)^2T^2},\\
y^-
&\equiv&
\frac{(\zeta^2+1+2\zeta\cos\theta)^2E^2}{\mathcal{N}_{+}T^2},
\end{eqnarray}
with $\mathcal{N}_{\mp}$ being denominated as
\begin{eqnarray}
\mathcal{N}_{\mp}
&=&\zeta^2(1+\zeta^2) \mp2\zeta(1+\zeta^2)
\cos\theta+(1+\zeta^2\cos(2\theta))\nonumber\\
&&+\sqrt{2\zeta^2\left(\zeta \mp\cos\theta\right)^2
\left[\zeta^2\mp
4\zeta\cos\theta+\zeta^2\cos(2\theta)+2\right]}.\nonumber
\end{eqnarray}

% Figure environment removed


In consequence, the specific heat can be derived as
\begin{eqnarray}
C_V(T)
=-T\frac{\partial^2 f(T)}{\partial T^2}.\label{Eq_Cv}
\end{eqnarray}
Again, the free limit $C^0_V$ can be obtained via taking $l_c\rightarrow\infty$ and
$v,v_z,\zeta$ being the specific constants.


After taking into account the RG equations of interaction parameters and the expression for
$C_V$~(\ref{Eq_Cv}), the numerical results for $C_V$ are shown in
Fig.~\ref{Fig_C_V-T} in the vicinity of three distinct kinds of instabilities.
At first, compared to the free case $C^0_V(T)\propto T^2$
in the absence of interactions and tilting parameter, we notice that the intimae
combination between the tilting parameter and interactions
yields to the close relationships and restrictions among the low-energy quasiparticles.
As a result, this is very harmful to the specific heat and cause the renormalized $C_V(T)$ to deviate a
little from the $T^2$ dependence, which bears witness to the signal of non-Fermi liquid
behavior~\cite{Mahan1990Book}. In addition, although $C_V$ is suppressed as approaching all three kinds of instabilities,
it can be unequivocally learn from Fig.~\ref{Fig_C_V-T} that the specific heat is less reduced for the type-II case
than that of the other two cases. This is qualitative in agreement with the basic tendencies of
the DOS and compressibility presented in Sec.~\ref{Subsection_DOS-kappa}.




To be brief, all these physical implications around the potential instabilities would be of particular help to
further study more interesting quantities of the related tilted materials.




\section{Summary}\label{Sec_summary}


In summary, we study the low-energy physics of type-I 3D tDSM under
the competitions among Coulomb interactions and electron-phonon coupling as well as phonon-phonon
interactions by adopting the powerful RG method~\cite{Wilson1975RMP,Polchinski9210046,Shankar1994RMP}
that treats all these ingredients on the same footing. By considering all one-loop corrections, we derived
the RG equations for all relevant parameters. With the aid of numerical analysis, we systematically studied
the low-energy behavior of these interactions and their effects on potential instabilities and physical properties.


To begin with, we examine the fate of the tilting parameter and observe two distinct scenarios,
depending on the initial anisotropy of fermion velocities. To ensure the self-consistency of our theory,
we only focus on the first scenario, as shown in Fig.~\ref{Fig_phase}(a).
Within such a scenario, we find that the anisotropy of fermion velocities is primarily
dependent on its own initial value compared to the strength of tilting parameter, and
can increase, decrease or remain nearly constant in the low-energy regime.
Regarding the ratio of dielectric constant $\epsilon_z/\epsilon$, it always flows towards either extreme anisotropy
$\epsilon_z/\epsilon\gg1$ or $\epsilon_z/\epsilon\ll1$ when starting form from Anisotropy-I or
Anisotropy-II, respectively, implying the screened Coulomb in the direction-$z$ or direction-$x,y$.
In addition, the fate of $g/g_0$ is analogous to that of $\epsilon_z/\epsilon$.
Compared to the $\epsilon_z/\epsilon$, we notice that both
the phonon velocities and phonon-phonon interactions
can either flow towards approximate isotropy or bear similarities to
the extreme anisotropy observed in $\epsilon_z/\epsilon$ in the low-energy regime.
Analogously, the anisotropy of electron-phonon interactions $\lambda_z/\lambda$ shares the
similar tendency of phonon velocities $C_T^z/C_T$.


Subsequently, we carefully investigate the tendencies of all interaction
parameters, which gives rise to three distinct types of FPs as displayed in Table~\ref{Table_classify_FPs}.
After introducing the source terms for the potential symmetry breakings and comparing their susceptibilities
accessing the FPs, we find that there exists some interaction-driven phase transition around the
FPs, with Phase-B or Phase-C being the preferred leading instability. Furthermore, the critical properties of physical quantities
including the density of states and compressibility as well as specific heat are briefly discussed as the system approaches these three
distinct types of FPs. In sharp contrast to their non-interacting counterparts, they exhibit very different behavior,
particularly around the Dirac point, and even deviate a little from the scope of Fermi-liquid behavior.
To recapitulate, we expect all these results would be of help to provide useful
clues for investigating the fascinating behavior of 3D tDSM and exploring other
related tilted materials in the future.



\section*{ACKNOWLEDGEMENTS}

We thank Y. H. Zhai, W. Liu and X. Z. Chu for useful discussions.
J.W. was partially supported by the National Natural Science Foundation of China
under Grant No. 11504360.



\appendix

\section{Collections of one-loop corrections}\label{appendix-one-loop-corrections}

After long calculations, we hereby provide the one-loop corrections to
the all vertex couplings in our effective
theory~(\ref{Eq_eff-action}) as displayed in Fig.~\ref{fig_appendix-1} due
to the competitions among the Coulomb interaction,
electron-phonon interaction, and phonon-phonon interaction, which can be formally expressed as follows,
\begin{eqnarray}
\delta g&=&\mathcal{B}gl,\label{Rq_appendix-1}\\
\delta\lambda&=&\mathcal{D}\lambda l,\\
\delta\lambda_z&=&\mathcal{D}_{z}\lambda_zl,\\
\delta V^{ij}_{T,L}&=&V^{ij}_{T,L}\mathcal{F}^{i,j}_{T,L}l,\,\,\mathrm{with}\,\,i,j=x,y,z,\label{Rq_appendix-2}
\end{eqnarray}
via neglecting the unimportant constant terms.
All related coefficients involving both in Eqs.~(\ref{Rq_appendix-1})-(\ref{Rq_appendix-2}) and
RG equations~(\ref{Eq_RGEq-v})-(\ref{Eq_RGEq-V_T-L}) are nominated in the following.

As to the coefficients $\mathcal{A}_{1}$ and $\mathcal{A}_{2}$, we have
\begin{eqnarray}
\mathcal{A}_{1}&\equiv&\int_{0}^{\pi}d\theta(\mathcal{A}_{11}-\mathcal{A}_{12}),\label{Eq_A_1}\\
\mathcal{A}_{2}&\equiv&
\int_{0}^{\pi}d\theta\big(\mathcal{A}_{21}+\mathcal{A}_{22}\big),\label{Eq_A_2}
\end{eqnarray}
with $\mathcal{A}_{11}$, $\mathcal{A}_{12}$, $\mathcal{A}_{21}$, and $\mathcal{A}_{22}$ being
\begin{widetext}
\begin{small}
\begin{eqnarray}
\mathcal{A}_{11}&\equiv&
\frac{-\eta_{\zeta}(1-\zeta^2)g^2\epsilon}{2\pi v^4v_{z}}
\frac{(1-\zeta^2)\sin^3\theta}
{\big[\epsilon\frac{\big(1-\zeta^2\big)\sin^2\theta}{v^2}
+\epsilon_{z}\big(\frac{\cos\theta-|\zeta|}{v_{z}}
\big)^2\big]^{2}}
+\frac{\eta_{\zeta}(1-\zeta^2)}{16\pi^2v^4v_{z}}
\frac{\lambda^2\mathcal{J}\big[\frac{(1-\zeta^2)\sin^2\theta}{v^2}
+\frac{2(\cos\theta-|\zeta|)^2}{v_{z}^2}\big]\sin\theta}{(1-|\zeta|\cos\theta)^{2}
\big[\big(1-|\zeta|\cos\theta+\mathcal{I}\big)^2
-\zeta^2\big(\cos\theta-|\zeta|\big)^2\big]^2}\nonumber\\
&&\times\Big\{
2(1-|\zeta|\cos\theta)
\big[(1-2\zeta^2)\big(\cos\theta-|\zeta|\big)^2
+\big(1-\zeta^2\big)\sin^2\theta\big]
\mathcal{I}+(1-\zeta^2)^2\big[\big(\cos\theta-|\zeta|\big)^2+\sin^2\theta\big]^2
\nonumber\\
&&+\!(1\!-\!\zeta^2)\big[\big(\cos\theta\!-\!|\zeta|\big)^2\!+\!\sin^2\theta\big]
\mathcal{I}^2\!
+\!2\zeta^2\big(\cos\theta\!-\!|\zeta|\big)^2
\big[3(1\!-\!|\zeta|\cos\theta)^2
\!+\!4(1\!-\!|\zeta|\cos\theta)\mathcal{I}\!+\!\mathcal{I}^2
\!-\!\zeta^2\big(\cos\theta\!-\!|\zeta|\big)^2\big]\nonumber\\
&&+\frac{(1-\zeta^2)[(\cos\theta-|\zeta|)^2
+(1-v)\sin^2\theta]}{\mathcal{I}}
\big\{2(1-|\zeta|\cos\theta)^{3}
+\big[5(1-|\zeta|\cos\theta)^2-\zeta^2\big(\cos\theta-|\zeta|\big)^2\big]\mathcal{I}\nonumber\\
&&+4(1-|\zeta|\cos\theta)\mathcal{I}^2+\mathcal{I}^{3}\big\}\Big\}
+\frac{\eta_{\zeta}(1-\zeta^2)^2}{16\pi^2v^4v_{z}}
\frac{\lambda\mathcal{J}\bigl[\frac{\lambda_{z}
(\cos\theta-|\zeta|)^2}{v_{z}}
+\frac{\lambda(1-\zeta^2)\sin^2\theta}{4v}\bigr]
\sin^3\theta}{(1-|\zeta|\cos\theta)^{2}
\big[\big(1-|\zeta|\cos\theta+\mathcal{I}\big)^2
-\zeta^2\big(\cos\theta-|\zeta|\big)^2\big]^2\mathcal{I}}
\nonumber\\
&&\times\Big\{2(1-|\zeta|\cos\theta)^{3}
+\big[5(1-|\zeta|\cos\theta)^2-\zeta^2\big(\cos\theta-|\zeta|\big)^2\big]\mathcal{I}
+4(1-|\zeta|\cos\theta)\mathcal{I}^2+\mathcal{I}^{3}\Big\},\\
\mathcal{A}_{12}&\equiv&
\frac{\eta_{\zeta}(1-\zeta^2)}{16\pi^2v^4v_{z}}
\frac{\frac{-2\mathcal{J}\lambda_{z}^2(\cos\theta-|\zeta|)^2}
{v_{z}^2}\sin\theta}{(1-|\zeta|\cos\theta)^{2}
\big[\big(1-|\zeta|\cos\theta
+\mathcal{I'}\big)^2-\zeta^2\big(\cos\theta-|\zeta|\big)^2\big]^2}
\Big\{2(1-|\zeta|\cos\theta)
\big[(1-2\zeta^2)\big(\cos\theta-|\zeta|\big)^2\nonumber\\
&&+\big(1-\zeta^2\big)\sin^2\theta\big]\mathcal{I'}
+(1-\zeta^2)^2\big[\big(\cos\theta-|\zeta|\big)^2
+\sin^2\theta\big]^2\!+\!(1\!-\!\zeta^2)[(\cos\theta\!-\!|\zeta|)^2\!+\!\sin^2\theta]\mathcal{I'}^2\nonumber\\
&&\!+\!2\zeta^2\big(\cos\theta\!-\!|\zeta|\big)^2
\big[3(1\!-\!|\zeta|\cos\theta)^2\!+\!4(1\!-\!|\zeta|\cos\theta)\mathcal{I'}
\!+\!\mathcal{I'}^2\!-\!\zeta^2\big(\cos\theta\!-\!|\zeta|\big)^2\big]
\!+\!\frac{(1-\zeta^2)\big[(\cos\theta-|\zeta|)^2
+(1-v)\sin^2\theta\big]}{\mathcal{I'}}\nonumber\\
&&\times\big\{2(1-|\zeta|\cos\theta)^{3}
+\big[\!5(1-|\zeta|\cos\theta)^2-\zeta^2\big(\cos\theta-|\zeta|\big)^2
\big]\mathcal{I'}
+4(1-|\zeta|\cos\theta)\mathcal{I'}^2+\mathcal{I'}^{3}\big\}\Big\}\nonumber\\
&&-\frac{\eta_{\zeta}(1-\zeta^2)^2}{16\pi^2v^4v_{z}}
\frac{2\lambda\mathcal{J}\big[
\frac{\lambda_{z}(\cos\theta\!-\!|\zeta|)^2}{v_{z}}
\!+\!\frac{\lambda(1-\zeta^2)\sin^2\theta}{4v}\big]
\sin^3\theta}{(1-|\zeta|\cos\theta)^{2}
\big[\big(1-|\zeta|\cos\theta
+\mathcal{I'}\big)^2-\zeta^2\big(\cos\theta-|\zeta|\big)^2\big]^2\mathcal{I'}}\nonumber\\
&&\times\Big\{2(1-|\zeta|\cos\theta)^{3}
+\big[\!5(1-|\zeta|\cos\theta)^2-\zeta^2\big(\cos\theta-|\zeta|\big)^2
\big]\mathcal{I'}
+4(1-|\zeta|\cos\theta)\mathcal{I'}^2+\mathcal{I'}^{3}\Big\}.\\
\mathcal{A}_{21}&\equiv&\frac{-\eta_{\zeta}(1-\zeta^2)\epsilon_{z}g^2}{\pi v^2v_{z}^3}\frac{\big(\cos\theta-|\zeta|\big)^2\sin\theta}
{\big[\epsilon\frac{\big(1-\zeta^2\big)\sin^2\theta}{v^2}
+\epsilon_{z}\big(\frac{\cos\theta-|\zeta|}{v_{z}}\big)^2\big]^{2}}
+\frac{\eta_{\zeta}(1-\zeta^2)}{16\pi^2v^2v_{z}^3}
\frac{2\lambda_{z}^2\mathcal{J}\frac{(1-\zeta^2)\sin^2\theta}{v^2}
\sin\theta}{(1-|\zeta|\cos\theta)^{2}
\big[\big(1-|\zeta|\cos\theta
+\mathcal{I}\big)^2-\zeta^2\big(\cos\theta-|\zeta|\big)^2
\big]^2}\nonumber\\
&&\times\Big\{(1-\zeta^2)^2\big[\big(\cos\theta-|\zeta|\big)^2+\sin^2\theta\big]^2
+2(1-|\zeta|\cos\theta)\big[(1-2\zeta^2)\big(\cos\theta-|\zeta|\big)^2
+\big(1-\zeta^2\big)\sin^2\theta\big]\mathcal{I}\nonumber\\
&&+(1-\zeta^2)\big[\big(\cos\theta-|\zeta|\big)^2
+\sin^2\theta\big]\mathcal{I}^2
+2\zeta^2(1-v_{z})\big(\cos\theta-|\zeta|\big)^2
\big[3(1-|\zeta|\cos\theta)^2+4(1-|\zeta|\cos\theta)
\mathcal{I}+\mathcal{I}^2-\zeta^2\big(\cos\theta-|\zeta|\big)^2\big]\nonumber\\
&&+\frac{\big(1-\zeta^2\big)\big[(1+2v_{z})\big(\cos\theta-|\zeta|\big)^2
+\sin^2\theta\big]}{\mathcal{I}}
\!\big\{2(1-|\zeta|\cos\theta)^{3}
+\!\big[5(1-|\zeta|\cos\theta)^2
\!\!-\!\!\zeta^2\big(\cos\theta\!-\!|\zeta|\big)^2\big]\mathcal{I}\!+\!\mathcal{I}^{3}
\!+4(1\!-\!|\zeta|\cos\theta)\mathcal{I}^2\big\}\!\Big\}\nonumber\\
&&-\frac{\eta_{\zeta}(1-\zeta^2)}{16\pi^2v^2v_{z}^3}
\frac{\mathcal{J}v_{z}(1-\zeta^2)\lambda
\lambda_{z}\sin^2\theta\big(\cos\theta-|\zeta|\big)^2\sin\theta}
{(1-|\zeta|\cos\theta)^{2}
\big[\big(1-|\zeta|\cos\theta
+\mathcal{I}\big)^2-\zeta^2\big(\cos\theta-|\zeta|\big)^2
\big]^2}\nonumber
\Big\{\zeta^2\big[3(1-|\zeta|\cos\theta)^2+4(1-|\zeta|\cos\theta)
\mathcal{I}+\mathcal{I}^2\nonumber\\
&&-\zeta^2\big(\cos\theta-|\zeta|\big)^2\big]
-\frac{1}{\mathcal{I}}\big\{2(1-|\zeta|\cos\theta)^{3}
+\big[5(1-|\zeta|\cos\theta)^2-\zeta^2\big(\cos\theta-|\zeta|\big)^2
\big]\mathcal{I}
+4(1-|\zeta|\cos\theta)\mathcal{I}^2
+\mathcal{I}^{3}\big\}\Big\},\\
\mathcal{A}_{22}&\equiv&\frac{\eta_{\zeta}(1-\zeta^2)}{16\pi^2v^2v_{z}^3}
\frac{2\mathcal{J}\big[-\frac{\lambda^2(1-\zeta^2)\sin^2\theta}{v^2}
+\frac{\lambda_{z}^2(\cos\theta-|\zeta|)^2}{v_{z}^2}\big]
\sin\theta}{(1-|\zeta|\cos\theta)^{2}
\big[\big(1-|\zeta|\cos\theta+\mathcal{I'}\big)^2
-\zeta^2\big(\cos\theta-|\zeta|\big)^2\big]^2}
\Big\{(1-\zeta^2)^2\big[\big(\cos\theta-|\zeta|\big)^2
+\sin^2\theta\big]^2\nonumber\\
&&+2(1-|\zeta|\cos\theta)
\big[(1-2\zeta^2)\big(\cos\theta-|\zeta|\big)^2
+\big(1-\zeta^2\big)\sin^2\theta\big]
\mathcal{I'}
+(1-\zeta^2)\big[\big(\cos\theta-|\zeta|\big)^2+\sin^2\theta\big]\mathcal{I'}^2+2\zeta^2(1-v_{z})\nonumber\\
&&\times\big(\cos\theta-|\zeta|\big)^2
\big[3(1-|\zeta|\cos\theta)^2+4(1-|\zeta|\cos\theta)
\mathcal{I'}
+\mathcal{I'}^2-\zeta^2\big(\cos\theta-|\zeta|\big)^2\big]\!
+\!\frac{(1-\zeta^2)\big[(1+2v_{z})\big(\cos\theta-|\zeta|\big)^2
+\sin^2\theta\big]}{\mathcal{I'}}\nonumber\\
&&\!\times\!\big\{2(1-|\zeta|\cos\theta)^{2}
+\big[5(1-|\zeta|\cos\theta)^2-\zeta^2\big(\cos\theta-|\zeta|\big)^2\big]\mathcal{I'}
+4(1-|\zeta|\cos\theta)\mathcal{I'}^2+\mathcal{I'}^{3}\big\}\Big\}+\frac{\eta_{\zeta}(1-\zeta^2)}{16\pi^2v^2v_{z}^3}\nonumber\\
&&\times
\frac{2\mathcal{J}(1-\zeta^2)\lambda\lambda_{z}v_{z}\sin^2\theta
\big(\cos\theta-|\zeta|\big)^2\sin\theta}
{(1-|\zeta|\cos\theta)^{2}
\big[\big(1-|\zeta|\cos\theta+\mathcal{I'}\big)^2
-\zeta^2\big(\cos\theta-|\zeta|\big)^2\big]^2}
\Big\{\zeta^2\big[3(1-|\zeta|\cos\theta)^2+4(1-|\zeta|\cos\theta)
\mathcal{I'}+\mathcal{I'}^2-\zeta^2\big(\cos\theta-|\zeta|\big)^2\big]\nonumber\\
&&-\frac{1}{\mathcal{I'}}
\big\{2(1-|\zeta|\cos\theta)^{3}
+\big[5(1-|\zeta|\cos\theta)^2-\zeta^2\big(\cos\theta-|\zeta|\big)^2\big]\mathcal{I'}
+4(1-|\zeta|\cos\theta)\mathcal{I'}^2+\mathcal{I'}^{3}\big\}\Big\},
\end{eqnarray}
and for the coefficients $\mathcal{A}_{0}$ and $\mathcal{A}_{3}$,
\begin{eqnarray}
\mathcal{A}_{0}
&=&-\frac{\eta_{\zeta}(1-\zeta^2)}{4\pi^2v^2v_{z}}
\int_{0}^{\pi}d\theta\sin\theta
\Big\{\frac{\big[(2\lambda^2+\lambda_{z}^2)(1-|\zeta|\cos\theta)
-\mathcal{J'}\big]}
{\big\{\big[(1-|\zeta|\cos\theta)+\mathcal{I}\big]^2
-\zeta^2\big(\cos\theta-|\zeta|\big)^2\big\}\mathcal{I}}
+\frac{\mathcal{J'}(1-|\zeta|\cos\theta)}
{\big\{\big[(1-|\zeta|\cos\theta)+\mathcal{I'}\big]^2
-\zeta^2\big(\cos\theta-|\zeta|\big)^2\big\}\mathcal{I'}}\Big\},\label{Rq_appendix-A0}\\
\mathcal{A}_{3}&\equiv&
\frac{\eta_{\zeta}(1-\zeta^2)}{4\pi^2v^2v_{z}}
\int_{0}^{\pi}d\theta\Bigl\{\frac{\sin\theta\big[(2\lambda^2+\lambda_{z}^2)
-\lambda^2\mathcal{J\frac{(1-\zeta^2)\sin^2\theta}{v^2}
+\lambda_{z}^2\frac{(\cos\theta-|\zeta|)^2}{v_{z}^2}}\big]}
{(1-|\zeta|\cos\theta)^{2}
\big[\big(1-|\zeta|\cos\theta+\mathcal{I}\big)^2
-\zeta^2(\cos\theta-|\zeta|)^2\big]^2}
\Bigl\{-\frac{1}{2}
\big\{(1-\zeta^2)^2\big[(\cos\theta-|\zeta|)^2
+\sin^2\theta\big]^2\nonumber\\
&&+2(1-|\zeta|\cos\theta)
\big[(1-2\zeta^2)\big(\cos\theta-|\zeta|\big)^2
+\big(1-\zeta^2\big)\sin^2\theta\big]\mathcal{I}
+(1-\zeta^2)\big[\big(\cos\theta-|\zeta|\big)^2
+\sin^2\theta\big]\mathcal{I}^2\big\}
+(1-\zeta^2)(\cos\theta-|\zeta|)^2\nonumber\\
&&\times\big[3(1-|\zeta|\cos\theta)^2
+4(1-|\zeta|\cos\theta)\mathcal{I}
+\mathcal{I}^2-\zeta^2(\cos\theta-|\zeta|)^2\big]+\frac{\big(1-\zeta^2\big)
\big[\sin^2\theta-(\cos\theta-|\zeta|)^2\big]}
{2\mathcal{I}}
\big\{2(1-|\zeta|\cos\theta)^{3}\nonumber\\
&&+\big[5(1-|\zeta|\cos\theta)^2-\zeta^2(\cos\theta-|\zeta|)^2\big]\mathcal{I}
+4(1-|\zeta|\cos\theta)\mathcal{I}^2+\mathcal{I}^{3}
\big\}\Bigr\}
+\frac{\sin\theta\mathcal{J'}}{(1-|\zeta|\cos\theta)^{2}
\big[\big(1-|\zeta|\cos\theta+\mathcal{I'}\big)^2
-\zeta^2(\cos\theta-|\zeta|)^2\big]^2}\nonumber\\
&&\times\Bigl\{-\frac{1}{2}
\big\{(1-\zeta^2)^2\big[(\cos\theta-|\zeta|)^2
+\sin^2\theta\big]^2+2(1-|\zeta|\cos\theta)
\big[(1-2\zeta^2)(\cos\theta-|\zeta|)^2
+\big(1-\zeta^2\big)\sin^2\theta\big]\mathcal{I'}\nonumber\\
&&+(1-\zeta^2)\big[(\cos\theta-|\zeta|)^2
+\sin^2\theta\big]\mathcal{I'}^2\big\}+(1-\zeta^2)(\cos\theta-|\zeta|)^2
\times\big[3(1-|\zeta|\cos\theta)^2+4(1-|\zeta|\cos\theta)\mathcal{I'}
+\mathcal{I'}^2-\zeta^2(\cos\theta-|\zeta|)^2\big]\nonumber\\
&&+\frac{\big(1-\zeta^2\big)
\big[\sin^2\theta-(\cos\theta-|\zeta|)^2\big]}
{2\mathcal{I'}}
\!\big\{2(1-|\zeta|\cos\theta)^{3}
+\big[5(1-\!|\zeta|\cos\theta)^2\!-\!\zeta^2(\cos\theta\!-\!|\zeta|)^2\big]\mathcal{I'}
+4(1\!-\!|\zeta|\cos\theta)\mathcal{I'}^2\!+\!\mathcal{I'}^{3}
\big\}\Bigr\}\!\Bigl\}.
\end{eqnarray}

% Figure environment removed


With respect to the other coefficients, we get
\begin{eqnarray}
\mathcal{B}&\equiv&
-\int_{0}^{\pi}d\theta
\frac{\sin\theta}{(1-|\zeta|\cos\theta)^2
\big[(1-|\zeta|\cos\theta+\mathcal{I})^2
-\zeta^2(\cos\theta-|\zeta|)^2\big]^2}
\frac{\eta_{\zeta}(1-\zeta^2)}{16\pi^2v^2v_{z}}\Big\{
2\big[(2\lambda^2+\lambda_{z}^2)
-\mathcal{J'}\big]\nonumber\\
&&\times\big\{\mathcal{I}^2(1-\zeta^2)
\big[(\cos\theta-|\zeta|)^2+\sin^2\theta\big]
+(1-\zeta^2)^2\big[(\cos\theta-|\zeta|)^2+\sin^2\theta\big]^2
+2\mathcal{I}(1-|\zeta|\cos\theta)
\big[(1-2\zeta^2)(\cos\theta-|\zeta|)^2\nonumber\\
&&+(1-\zeta^2)\sin^2\theta\big]\big\}
-\frac{1}{2\mathcal{I}}\big\{\mathcal{I}^{3}
+2\big(1-|\zeta|\cos\theta\big)^{3}
+4\mathcal{I}^2(1-|\zeta|\cos\theta)
+\mathcal{I}\big[5(1-|\zeta|\cos\theta)^2
-\zeta^2(\cos\theta-|\zeta|)^2\big]\big\}\nonumber\\
&&\times\big\{2\lambda^2(1-\zeta^2)\sin^2\theta+2\lambda_{z}^2
(\cos\theta-|\zeta|)^2-\frac{\lambda^2\mathcal{J}(1-\zeta^2)^2\sin^4\theta}{2v^2}
-\frac{2\lambda_{z}^2\mathcal{J}(\cos\theta-|\zeta|)^4}{v_{z}^2}\nonumber\\
&&-\frac{\frac{1}{2}\lambda^2(1-\zeta^2)^2\sin^4\theta
+4\lambda\lambda_{z}(1-\zeta^2)\sin^2\theta
(\cos\theta-|\zeta|)^2}{vv_{z}\big[\frac{(1-|\zeta|\cos\theta)^2}{v^2}
+\frac{(\cos\theta-|\zeta|)^2}{v_{z}^2}\big]}
+2\zeta^2(\cos\theta-|\zeta|)^2\big[2\lambda^2\!+\!\lambda_{z}^2
\!-\!\frac{\lambda^2(1-\zeta^2)\sin^2\theta}{v^2
\big[\frac{(1-\zeta^2)\sin^2\theta}{v^2}
\!+\frac{(\cos\theta-|\zeta|)^2}{v_{z}^2}\big]}\nonumber\\
&&-\frac{\lambda_{z}^2\mathcal{J}(\cos\theta-|\zeta|)^2}{v_{z}^2}\big]
\big\}
+\zeta^2\big[(2\lambda^2+\lambda_{z}^2)
-\mathcal{J'}](\cos\theta-|\zeta|)^2
\big[\mathcal{I}^2+4\mathcal{I}(1-|\zeta|\cos\theta)\nonumber\\
&&
+3(1-|\zeta|\cos\theta)^2-\zeta^2
(\cos\theta-|\zeta|)^2\big]\Big\}
-
\int_{0}^{\pi}d\theta\frac{\sin\theta}
{(1-|\zeta|\cos\theta)^{2}\big[
\big(\mathcal{I'}+1-|\zeta|\cos\theta\big)^2
-\zeta^2(\cos\theta-|\zeta|)^2\big]^2}\nonumber\\
&&\times\frac{\eta_{\zeta}(1-\zeta^2)}{16\pi^2v^2v_{z}}
\Big\{2\mathcal{J'}\big\{\mathcal{I'}^2(1-\zeta^2)
\big[(\cos\theta-|\zeta|)^2+\sin^2\theta\big]
+(1-\zeta^2)^2
\big[(\cos\theta-|\zeta|)^2+\sin^2\theta\big]^2\nonumber\\
&&+2\mathcal{I'}
(1-|\zeta|\cos\theta)
\big[(1-2\zeta^2)(\cos\theta-|\zeta|)^2
+(1-\zeta^2)\sin^2\theta\big]\big\}
-\frac{1}{2\mathcal{I'}}\big\{\mathcal{I'}^{3}
+2(1-|\zeta|\cos\theta)^{3}
+4\mathcal{I'}^2(1-|\zeta|\cos\theta)\nonumber\\
&&+\mathcal{I'}\big[5(1-|\zeta|\cos\theta)
-\zeta^2(\cos\theta-|\zeta|)^2\big]\big\}
\big\{\frac{\lambda^2\mathcal{J}(1-\zeta^2)^2\sin^4\theta}{2v^2}
+\frac{2\lambda_{z}^2\mathcal{J}(\cos\theta-|\zeta|)^4}{v_{z}^2}
\nonumber\\
&&+\frac{\frac{1}{2}\lambda^2(1-\zeta^2)^2\sin^4\theta
+4\lambda\lambda_{z}(1-\zeta^2)\sin^2\theta
(\cos\theta-|\zeta|)^2}{vv_{z}\big[\frac{(1-|\zeta|\cos\theta)^2}{v^2}
+\frac{(\cos\theta-|\zeta|)^2}{v_{z}^2}\big]}
+4\big\{
\frac{\lambda^2(1-\zeta^2)\sin^2\theta}{v^2
\big[\frac{(1-\zeta^2)\sin^2\theta}{v^2}
+\frac{(\cos\theta-|\zeta|)^2}{v_{z}^2}\big]}
+\frac{\lambda_{z}^2(\cos\theta-|\zeta|)^2}{v_{z}^2
\big[\frac{(1-\zeta^2)\sin^2\theta}{v^2}
+\frac{(\cos\theta-|\zeta|)^2}{v_{z}^2}\big]}\big\}\nonumber\\
&&\times\zeta^2(\cos\theta-|\zeta|)^2
\big\}
+\zeta^2\mathcal{J'}(\cos\theta-|\zeta|)^2
\big[\mathcal{I'}^2+4\mathcal{I'}(1-|\zeta|\cos\theta)
+3(1-|\zeta|\cos\theta)^2-\zeta^2(\cos\theta-|\zeta|)^2
\big]\Big\},\\
\mathcal{D}&\equiv&
\frac{-\lambda_{z}^2(1-\zeta^2)^2\eta_{\zeta}}{8\pi^2v^4v_{z}}
\int_{0}^{\pi}d\theta\frac{\sin^{3}\theta}{(1-|\zeta|\cos\theta)^{2}\mathcal{J}^{-1}
\big[(1-|\zeta|\cos\theta+\mathcal{I})^2
-\zeta^2(\cos\theta-|\zeta|)^2\big]^2}\nonumber\\
&&\times\Big\{\mathcal{I}^2
\big[(1-|\zeta|\cos\theta)^2-\zeta^2(\cos\theta-|\zeta|)^2\big]
+2\mathcal{I}(1-|\zeta|\cos\theta)
\big[(1-|\zeta|\cos\theta)^2-2\zeta^2(\cos\theta-|\zeta|)^2\big]
\nonumber\\
&&+\big[(1-|\zeta|\cos\theta)^2-\zeta^2(\cos\theta-|\zeta|)^2\big]^2\Big\}
-\frac{\lambda_{z}^2(1-\zeta^2)\eta_{\zeta}}{32\pi^2v^2v_{z}^3}
\int_{0}^{\pi}d\theta
\frac{\sin\theta(\cos\theta-|\zeta|)^{2}}{(1-|\zeta|\cos\theta)^{2}\mathcal{J}^{-1}
\big[(1-|\zeta|\cos\theta+\mathcal{I'})^2-\zeta^2(\cos\theta-|\zeta|)^2\big]^2}\nonumber\\
&&\times\Big\{\mathcal{I'}^2
\big[(1-|\zeta|\cos\theta)^2-\zeta^2(\cos\theta-|\zeta|)^2\big]
+2\mathcal{I'}(1-|\zeta|\cos\theta)
\big[(1-|\zeta|\cos\theta)^2-2\zeta^2(\cos\theta-|\zeta|)^2\big]\nonumber\\
&&+\big[(1-|\zeta|\cos\theta)^2-\zeta^2(\cos\theta-|\zeta|)^2\big]^2\Big\},\\
\mathcal{D}_{z}&\equiv&
\frac{(1-\zeta^2)\eta_{\zeta}}{8\pi^2v^2v_{z}}
\int_{0}^{\pi}d\theta
\frac{
\big\{\lambda_{z}^2\frac{(1-\zeta^2)\sin^{2}\theta}{v^2}
-\lambda^2\big[\frac{(1-\zeta^2)\sin^{2}\theta}{v^2}
+\frac{(\cos\theta-|\zeta|)^{2}}{v_{z}^2}\big]\big\}\sin\theta}
{(1-|\zeta|\cos\theta)^{2}\mathcal{J}^{-1}
\big[(1-|\zeta|\cos\theta+\mathcal{I})^2
-\zeta^2(\cos\theta-|\zeta|)^2\big]^2}\nonumber\\
&&\times\Big\{\mathcal{I}^2
\big[(1-|\zeta|\cos\theta)^2-\zeta^2(\cos\theta-|\zeta|)^2\big]
+2\mathcal{I}(1-|\zeta|\cos\theta)
\big[(1-|\zeta|\cos\theta)^2-2\zeta^2(\cos\theta-|\zeta|)^2\big]
+\big[(1-|\zeta|\cos\theta)^2\nonumber\\
&&-\zeta^2(\cos\theta-|\zeta|)^2\big]^2\Big\}
+\frac{(1-\zeta^2)\eta_{\zeta}}{32\pi^2v^2v_{z}}
\int_{0}^{\pi}d\theta
\frac{\sin\theta\big(\lambda_{z}^2\frac{(\cos\theta-|\zeta|)^{2}}{v_{z}^2}
-2\lambda^2\frac{(1-\zeta^2)
\sin^{2}\theta}{v^2}\big)}{(1-|\zeta|\cos\theta)^{2}\mathcal{J}^{-1}
\big[(1-|\zeta|\cos\theta+\mathcal{I'})^2
-\zeta^2(\cos\theta-|\zeta|)^2\big]^2}\nonumber\\
&&\times\Big\{\mathcal{I'}^2
\big[(1-|\zeta|\cos\theta)^2-\zeta^2(\cos\theta-|\zeta|)^2\big]
+2\mathcal{I'}(1-|\zeta|\cos\theta)
\big[(1-|\zeta|\cos\theta)^2-2\zeta^2(\cos\theta-|\zeta|)^2\big]
\nonumber\\
&&+\big[(1-|\zeta|\cos\theta)^2-\zeta^2(\cos\theta-|\zeta|)^2\big]^2
\Big\},\\
\mathcal{F}^{xx}_{T}&=&\frac{\eta_{\zeta}(1-\zeta^2)
}{8\pi^2v^2v_{z}}\int_{0}^{\pi}d\theta
\frac{\mathcal{J}^{2}\sin\theta(1-|\zeta|\cos\theta)}{\mathcal{I}^3}
\Big[\big(\frac{7}{4}V_{T}^{xx}
+\frac{(V_{T}^{xy})^2}{V_{T}^{xx}}
+8\frac{(V_{T}^{xz})^2}{V_{T}^{xx}}
+\frac{5}{2}V_{T}^{xy}
\big)\frac{(1-\zeta^2)^2\sin^{4}\theta}{4v^4}\nonumber\\
&&+\frac{\big[7(V_{T}^{xx})^2+4(V_{T}^{xy})^2\big](-|\zeta|+\cos\theta)^4}{2V_{T}^{xx}v_{z}^4}
+\big(\frac{7}{2}V_{T}^{xx}
+2\frac{(V_{T}^{xy})^2}{V_{T}^{xx}}
+\frac{5}{2}V_{T}^{xz}+2\frac{V_{T}^{xy}V_{T}^{xz}}{V_{T}^{xx}}\big)
\frac{(-|\zeta|+\cos\theta)^2
(1-\zeta^2)\sin^{2}\theta}{v^2v_{z}^2}
\Big],\\
\mathcal{F}^{zz}_{T}
&=&\frac{\eta_{\zeta}(1-\zeta^2)
}{8\pi^2v^2v_{z}}\int_{0}^{\pi}d\theta
\frac{\mathcal{J}^{2}\sin\theta(1-|\zeta|\cos\theta)}{\mathcal{I}^3}
\Big\{
+\frac{\big[14(V_{T}^{zz})^2+4(V_{T}^{xz})^2\big](1-\zeta^2)^2\sin^{4}\theta}{4V_{T}^{zz}v^4}
+\frac{4(V_{T}^{xz})^2(-|\zeta|+\cos\theta)^4}{V_{T}^{zz}v_{z}^4}\nonumber\\
&&+
\frac{\big[5(V_{T}^{xz})^{2}+4(V_{T}^{xz})^2\big](-|\zeta|+\cos\theta)^2
(1-\zeta^2)\sin^{2}\theta}{V_{T}^{zz}v^2v_{z}^2}
\Big\},\\
\mathcal{F}^{xy}_{T}
&=&\frac{\eta_{\zeta}(1-\zeta^2)
}{8\pi^2v^2v_{z}}\int_{0}^{\pi}d\theta
\frac{\mathcal{J}^{2}\sin\theta(1-|\zeta|\cos\theta)}{\mathcal{I}^3}
\Big[\big(\frac{(3V_{T}^{xx})^2}{2V_{T}^{xy}}
+V_{T}^{xy}+\frac{V_{T}^{xx}V_{T}^{xz}}{2V_{T}^{xy}}
+8\frac{(V_{T}^{xz})^2}{V_{T}^{xy}}
+3V_{T}^{xx}
\big)\frac{(1-\zeta^2)^2\sin^{4}\theta}{4v^4}\nonumber\\
&&+\frac{6V_{T}^{xx}(-|\zeta|+\cos\theta)^4}{v_{z}^4}
+\big(6V_{T}^{xx}+2V_{T}^{xz}
+\frac{3V_{T}^{xz}V_{T}^{xx}}{V_{T}^{xy}}\big)
\frac{(-|\zeta|+\cos\theta)^2
(1-\zeta^2)\sin^{2}\theta}{v^2v_{z}^2}
\Big],\\
\mathcal{F}^{xz}_{T}
&=&\frac{\eta_{\zeta}(1-\zeta^2)
}{8\pi^2v^2v_{z}}
\int_{0}^{\pi}d\theta
\frac{\mathcal{J}^{2}\sin\theta(1-|\zeta|\cos\theta)}{\mathcal{I}^3}
\Big[\big(2V_{T}^{xy}
+3V_{T}^{xx}
+16V_{T}^{zz}
\big)\frac{(1-\zeta^2)^2\sin^{4}\theta}{4v^4}\nonumber\\
&&+\frac{\big(3V_{T}^{xx}+2V_{T}^{xy}\big)(-|\zeta|+\cos\theta)^4}{v_{z}^4}
+\big(3V_{T}^{xx}
+2V_{T}^{xz}
+2V_{T}^{xy}
+\frac{2V_{T}^{xy}V_{T}^{zz}}{V_{T}^{xz}}
+\frac{2V_{T}^{xx}V_{T}^{zz}}{V_{T}^{xz}}\big)
\frac{(-|\zeta|+\cos\theta)^2
(1-\zeta^2)\sin^{2}\theta}{v^2v_{z}^2}
\Big],\\
\mathcal{F}^{xx}_{L}
&=&\frac{\eta_{\zeta}(1-\zeta^2)
}{8\pi^2v^2v_{z}}\int_{0}^{\pi}d\theta
\frac{\mathcal{J}^{2}\sin\theta(1-|\zeta|\cos\theta)}{\mathcal{I'}^3}
\big[\big(\frac{7}{4}V_{L}^{xx}
+\frac{(V_{L}^{xy})^2}{V_{L}^{xx}}
+\frac{5}{2}V_{L}^{xy}
\big)\frac{(1-\zeta^2)^2\sin^{4}\theta}{4v^4}
+\frac{2(V_{L}^{xz})^2(-|\zeta|+\cos\theta)^4}{V_{L}^{xx}v_{z}^4}\nonumber\\
&&+\big(\frac{5}{2}V_{L}^{xz}+2\frac{V_{L}^{xy}V_{L}^{xz}}{V_{L}^{xx}}\big)
\frac{(-|\zeta|+\cos\theta)^2
(1-\zeta^2)\sin^{2}\theta}{v^2v_{z}^2}
\Big],\\
\mathcal{F}^{zz}_{L}
&=&\frac{\eta_{\zeta}(1-\zeta^2)}{8\pi^2v^2v_{z}}\int_{0}^{\pi}d\theta
\frac{\mathcal{J}^{2}\sin\theta(1-|\zeta|\cos\theta)}{\mathcal{I'}^3}
\Big[\frac{4(V_{L}^{xz})^2(1-\zeta^2)^2\sin^{4}\theta}{4V_{L}^{zz}v^4}
+\frac{7V_{L}^{zz}(-|\zeta|+\cos\theta)^4}{2v_{z}^4}\nonumber\\
&&+\frac{5V_{L}^{xz}(-|\zeta|+\cos\theta)^2
(1-\zeta^2)\sin^{2}\theta}{v^2v_{z}^2}
\Big],\\
\mathcal{F}^{xy}_{L}&=&\frac{\eta_{\zeta}(1-\zeta^2)
}{8\pi^2v^2v_{z}}\int_{0}^{\pi}d\theta
\frac{\mathcal{J}^{2}\sin\theta(1-|\zeta|\cos\theta)}{\mathcal{I'}^3}
\Big[\big(\frac{(9V_{L}^{xx})^2}{4V_{T}^{xy}}
+V_{L}^{xy}+3V_{L}^{xx}
\big)\frac{(1-\zeta^2)^2\sin^{4}\theta}{4v^4}
+\frac{2(V_{T}^{xz})^2(-|\zeta|+\cos\theta)^4}{V_{L}^{xy}v_{z}^4}\nonumber\\
&&+\big(2V_{L}^{xz}+\frac{3V_{L}^{xx}V_{L}^{xz}}{V_{L}^{xy}}\big)
\frac{(-|\zeta|+\cos\theta)^2
(1-\zeta^2)\sin^{2}\theta}{v^2v_{z}^2}
\Big],\\
\mathcal{F}^{xz}_{L}&=&\frac{\eta_{\zeta}(1-\zeta^2)}{8\pi^2v^2v_{z}}
\int_{0}^{\pi}d\theta
\frac{\mathcal{J}^{2}\sin\theta(1-|\zeta|\cos\theta)}{\mathcal{I'}^3}
\Big[\frac{\big(4V_{L}^{xx}+2V_{L}^{xy}
\big)(1-\zeta^2)^2\sin^{4}\theta}{4v^4}
+3V_{L}^{zz}
\frac{(-|\zeta|+\cos\theta)^4}{v_{z}^4}\nonumber\\
&&+\big(2V_{L}^{xz}
+\frac{11V_{L}^{xx}V_{L}^{zz}}{4V_{L}^{xz}}
+\frac{3V_{L}^{xy}V_{L}^{zz}}{2V_{L}^{xz}}\big)
\frac{(-|\zeta|+\cos\theta)^2
(1-\zeta^2)\sin^{2}\theta}{v^2v_{z}^2}
\Big], %\\
%\mathcal{F}&\equiv&
%\frac{\eta_{\zeta}(1-\zeta^{2})}{4\pi^{2}v^{2}v_{z}}
%\int_{0^{\pi}}d\theta\frac{\sin\theta(1-|\zeta|\cos\theta)}{\mathcal{I}^{3}}
%\big\{8+\mathcal{J}^{2}\big[\frac{13(1-\zeta^{2}\sin^{4}\theta)}{2v^{4}}+6(\cos\theta-|\zeta|)^{4}
%+\frac{8(\cos\theta-|\zeta|)^{2}(1-\zeta^{2})\sin^{2}\theta}{v^{2}v_{z}^{2}}
%\big]\big\}\\
%\mathcal{F'}&\equiv&
%\frac{\eta_{\zeta}(1-\zeta^{2})}{4\pi^{2}v^{2}v_{z}}
%\int_{0^{\pi}}d\theta\frac{\sin\theta(1-|\zeta|\cos\theta)}{\mathcal{I}^{3}}
%\big\{2+\mathcal{J}^{2}\big[\frac{13(1-\zeta^{2}\sin^{4}\theta)}{2v^{4}}+6(\cos\theta-|\zeta|)^{4}
%+\frac{8(\cos\theta-|\zeta|)^{2}(1-\zeta^{2})\sin^{2}\theta}{v^{2}v_{z}^{2}}
%\big]\big\}
\end{eqnarray}
\end{small}
\end{widetext}
with $\mathcal{I}$, $\mathcal{I'}$, $\mathcal{J}$, and $\mathcal{J'}$ being defined as
\begin{small}
\begin{eqnarray}
\mathcal{I}&\equiv& \sqrt{C_{T}^{2}\frac{\left(1-\zeta^2\right)\sin^2\theta}
{v^2}+C_{T_{z}}^{2}\frac{\left(\cos\theta-|\zeta|\right)^2}
{v_{z}^2}},\\
\mathcal{I'}&\equiv& \sqrt{C_{L}^{2}\frac{\left(1-\zeta^2\right)\sin^2\theta}
{v^2}+C_{L_{z}}^{2}\frac{\left(\cos\theta-|\zeta|\right)^2}
{v_{z}^2}},\\
\mathcal{J}&\equiv& \big[\frac{(\cos\theta-|\zeta|)^2}{v_{z}^2}}{\frac{(1-\zeta^2)\sin^2\theta}{v^2}\big]^{-1},\\
\mathcal{J'}&\equiv& \frac{\lambda^2\frac{(1-\zeta^2)\sin^2\theta}{v^2}
+\lambda_{z}^2\frac{(\cos\theta-|\zeta|)^2}{v_{z}^2}}{\frac{(1-\zeta^2)\sin^2\theta}{v^2}
+\frac{(\cos\theta-|\zeta|)^2}{v_{z}^2}}.
\end{eqnarray}
\end{small}

% Figure environment removed



\section{Energy-dependent evolutions of source terms}\label{appendix-source-terms}

After performing the calculations of one-loop corrections to the strengths of source terms $\Delta^{\mathrm{PH/PP}}_{i}$
shown in Fig.~\ref{Fig_1L-source-terms}, we then derive the RG equations with the help of
RG rescalings~(\ref{Eq_RG-rescaling-omega})-(\ref{Eq_RG-rescaling-u-L}) as follows,
\begin{small}
\begin{eqnarray}
\frac{d\Delta^{\mathrm{PH}}_{0}}{dl}
\!\!&=&\!\!(1-2\eta_\psi)\Delta^{\mathrm{PH}}_{0}\label{Eq_Delta_1},\\
\frac{d\Delta^{\mathrm{PH}}_{1}}{dl}
\!\!&=&\!\!\left[1-2\eta_\psi+(\mathcal{R}_{1}^{+}-\mathcal{R}_{1}^{-})(-g^{2}-\lambda_{z}^{2})\right]\!\!\Delta^{\mathrm{PH}}_{1},\\
\frac{d\Delta^{\mathrm{PH}}_{2}}{dl}
&=&\left[1-2\eta_\psi+(\mathcal{R}_{2}^{+}-\mathcal{R}_{2}^{-})(-g^{2}-\lambda_{z}^{2})\right]\!\!\Delta^{\mathrm{PH}}_{2},\\
\frac{d\Delta^{\mathrm{PH}}_{3}}{dl}
\!\!&=&\!\!\left[1-2\eta_\psi\!+\!(\mathcal{R}_{3}^{+}\!-\!\mathcal{R}_{3}^{-})
(\lambda_{z}^{2}-g^{2}\!-2\lambda^{2})\right]\!\!\Delta^{\mathrm{PH}}_{3}.
\end{eqnarray}
\end{small}
for the particle-hole channel, and
\begin{small}
\begin{eqnarray}
\frac{d\Delta^{\mathrm{PP}}_{0}}{dl}
\!\!&=&\!\!\left[1-2\eta_\psi+(\mathcal{T}_{0}^{+}-\mathcal{T}_{0}^{-})(-g^{2}-\lambda_{z}^{2})\right]\!\!\Delta^{\mathrm{PP}}_{0},\\
\frac{d\Delta^{\mathrm{PP}}_{1}}{dl}
\!\!&=&\!\!\left[1-2\eta_\psi+(\mathcal{T}_{1}^{+}-\mathcal{T}_{1}^{-})(\lambda_{z}^{2}-g^{2}-2\lambda^{2})\right]\!\!\Delta^{\mathrm{PP}}_{1},\\
\frac{d\Delta^{\mathrm{PP}}_{2}}{dl}
\!\!&=&\!\!\left[1-2\eta_\psi+(\mathcal{T}_{2}^{+}-\mathcal{T}_{2}^{-})(-g^{2}-\lambda_{z}^{2})\right]\!\!\Delta^{\mathrm{PP}}_{2},\\
\frac{d\Delta^{\mathrm{PP}}_{30}}{dl}
\!\!&=&\!\!\left[1-2\eta_\psi+(\mathcal{T}_{0}^{+}-\mathcal{T}_{0}^{-})(-g^{2}-\lambda_{z}^{2})
\right]\Delta^{\mathrm{PP}}_{30},\\
\frac{d\Delta^{\mathrm{PP}}_{31}}{dl}
\!\!&=&\!\!\left[1-2\eta_\psi
+(\mathcal{T}_{1}^{+}-\mathcal{T}_{1}^{-})(\lambda_{z}^{2}-g^{2}-2\lambda^{2})\right]\!\!\Delta^{\mathrm{PP}}_{31},\\
\frac{d\Delta^{\mathrm{PP}}_{33}}{dl}
\!\!&=&\!\!\left[1-2\eta_\psi
+(\mathcal{T}_{2}^{+}-\mathcal{T}_{2}^{-})(-g^{2}-\lambda_{z}^{2})\right]\!\!\Delta^{\mathrm{PP}}_{33},\label{Eq_Delta_2}
\end{eqnarray}
\end{small}
for the particle-particle channel, where all the related coefficients are designated as
\begin{widetext}
\begin{small}
\begin{eqnarray}
\mathcal{R}_{1}^{+}&=&\int_{0}^{\pi}\frac
{(1-\zeta\cos{\theta})|\zeta^{2}-1|\sin{\theta}(\cos^{2}{\theta}-\zeta^{2}\cos^{2}{\theta}+\zeta^{2}-4\zeta\sqrt{1-\zeta^{2}}\cos{\theta}+1)}
{16\pi^{2}(\zeta^{2}-1)^{2}(1-2\zeta\sqrt{1-\zeta^{2}}\cos{\theta})^{\frac{3}{2}}}d\theta,\\
\mathcal{R}_{1}^{-}&=&\int_{0}^{\pi}\frac
{(1+\zeta\cos{\theta})|\zeta^{2}-1|\sin{\theta}(\cos^{2}{\theta}-\zeta^{2}\cos^{2}{\theta}+\zeta^{2}+4\zeta\sqrt{1-\zeta^{2}}\cos{\theta}+1)}
{16\pi^{2}(\zeta^{2}-1)^{2}(1+2\zeta\sqrt{1-\zeta^{2}}\cos{\theta})^{\frac{3}{2}}}d\theta,\\
\mathcal{R}_{2}^{+}&=&\int_{0}^{\pi}\frac
{(1-\zeta\cos{\theta})|\zeta^{2}-1|\sin{\theta}(\cos^{2}{\theta}-\zeta^{2}\cos^{2}{\theta}+\zeta^{2}-4\zeta\sqrt{1-\zeta^{2}}\cos{\theta}+1)}
{16\pi^{2}(\zeta^{2}-1)^{2}(1-2\zeta\sqrt{1-\zeta^{2}}\cos{\theta})^{\frac{3}{2}}}d\theta,\\
\mathcal{R}_{2}^{-}&=&\int_{0}^{\pi}\frac
{(1+\zeta\cos{\theta})|\zeta^{2}-1|\sin{\theta}(\cos^{2}{\theta}-\zeta^{2}\cos^{2}{\theta}+\zeta^{2}+4\zeta\sqrt{1-\zeta^{2}}\cos{\theta}+1)}
{16\pi^{2}(\zeta^{2}-1)^{2}(1+2\zeta\sqrt{1-\zeta^{2}}\cos{\theta})^{\frac{3}{2}}}d\theta,\\
\mathcal{R}_{3}^{+}&=&\int_{0}^{\pi}\frac
{(1-\zeta\cos{\theta})|\zeta^{2}-1|\sin{\theta}(\cos^{2}{\theta}-1)}
{16\pi^{2}(\zeta^{2}-1)(1-2\zeta\sqrt{1-\zeta^{2}}\cos{\theta})^{\frac{3}{2}}}d\theta,\\
\mathcal{R}_{3}^{-}&=&\int_{0}^{\pi}\frac
{(1+\zeta\cos{\theta})|\zeta^{2}-1|\sin{\theta}(\cos^{2}{\theta}-1)}
{16\pi^{2}(\zeta^{2}-1)(1+2\zeta\sqrt{1-\zeta^{2}}\cos{\theta})^{\frac{3}{2}}}d\theta,
\end{eqnarray}
and
\begin{eqnarray}
\mathcal{T}_{0}^{+}&=&\int_{0}^{\pi}\frac{\sin^{3}{\theta}(1-\zeta\cos{\theta})}
{16\pi^{2}\zeta(1-\zeta)^{2}(\frac{\cos{\theta}}{\sqrt{1-\zeta^{2}}}-\frac{\zeta}{1-\zeta^{2}})
\left[\sin^{2}{\theta}-(\zeta^{2}-1)(\cos^{2}{\theta}-\frac{2\zeta}{\sqrt{1-\zeta^{2}}}+\frac{\zeta^{2}}{1-\zeta^{2}})\right]}d\theta,\\
\mathcal{T}_{0}^{-}&=&\int_{0}^{\pi}\frac{\sin^{3}{\theta}(1+\zeta\cos{\theta})}
{16\pi^{2}\zeta(1-\zeta)^{2}(\frac{\cos{\theta}}{\sqrt{1-\zeta^{2}}}-\frac{\zeta}{1-\zeta^{2}})
\left[\sin^{2}{\theta}-(\zeta^{2}-1)(\cos^{2}{\theta}+\frac{2\zeta}{\sqrt{1-\zeta^{2}}}+\frac{\zeta^{2}}{1-\zeta^{2}})\right]}d\theta,\\
\mathcal{T}_{1}^{+}&=&\int_{0}^{\pi}\frac{\sin{\theta}(1-\zeta\cos{\theta})(\cos^{2}
{\theta}-\frac{2\zeta}{\sqrt{1-\zeta^{2}}}\cos{\theta}+\frac{\zeta^{2}}{1-\zeta^{2}})}
{16\pi^{2}\zeta(1-\zeta^{2})^{2}(\frac{\cos{\theta}}{\sqrt{1-\zeta^{2}}}-\frac{\zeta}{1-\zeta^{2}})
\left[\sin^{2}{\theta}-(\zeta^{2}-1)(\cos^{2}{\theta}-\frac{2\zeta}{\sqrt{1-\zeta^{2}}}+\frac{\zeta^{2}}{1-\zeta^{2}})\right]}d\theta,\\
\mathcal{T}_{1}^{-}&=&\int_{0}^{\pi}\frac{\sin{\theta}(1+\zeta\cos{\theta})(\cos^{2}{\theta}+\frac{2\zeta}
{\sqrt{1-\zeta^{2}}}\cos{\theta}+\frac{\zeta^{2}}{1-\zeta^{2}})}
{16\pi^{2}\zeta(1-\zeta^{2})^{2}(\frac{\cos{\theta}}{\sqrt{1-\zeta^{2}}}-\frac{\zeta}{1-\zeta^{2}})
\left[\sin^{2}{\theta}-(\zeta^{2}-1)(\cos^{2}{\theta}+\frac{2\zeta}{\sqrt{1-\zeta^{2}}}+\frac{\zeta^{2}}{1-\zeta^{2}})\right]}d\theta,\\
\mathcal{T}_{2}^{+}&=&\int_{0}^{\pi}\frac{-\sin^{3}{\theta}(1-\zeta\cos{\theta})}
{16\pi^{2}\zeta(1-\zeta^{2})^{2}(\frac{\cos{\theta}}{\sqrt{1-\zeta^{2}}}-\frac{\zeta}{1-\zeta^{2}})
\left[\sin^{2}{\theta}-(\zeta^{2}-1)(\cos^{2}{\theta}-\frac{2\zeta}{\sqrt{1-\zeta^{2}}}+\frac{\zeta^{2}}{1-\zeta^{2}})\right]}d\theta,\\
\mathcal{T}_{2}^{-}&=&\int_{0}^{\pi}\frac{-\sin^{3}{\theta}(1+\zeta\cos{\theta})}
{16\pi^{2}\zeta(1-\zeta^{2})^{2}(\frac{\cos{\theta}}{\sqrt{1-\zeta^{2}}}-\frac{\zeta}{1-\zeta^{2}})
\left[\sin^{2}{\theta}-(\zeta^{2}-1)(\cos^{2}{\theta}+\frac{2\zeta}{\sqrt{1-\zeta^{2}}}+\frac{\zeta^{2}}{1-\zeta^{2}})\right]}d\theta.
\end{eqnarray}
\end{small}

\end{widetext}





%%%%%%%%%%%%%%%%%%%%%%%%%%%%%%%%%%%%%%%%%%%%%%%%%%%%%%%%%%%%%%%%%%%%%%%%%%%%%%%%%%%%%
%%%%%%%%%%%%%%%%%%%%%%%%%%%%%%%%%%%%%%%%%%%%%%%%%%%%%%%%%%%%%%%%%%%%%%%%%%%%%%%%%%%%%

\begin{thebibliography}{10}


\bibitem{Novoselov2005Nature}
K. S. Novoselov, A. K. Geim, S. V. Morozov, D. Jiang, M. I. Katsnelson, I. V. Grigorieva,
S. V. Dubonos, and A. A. Firsov, \href{https://doi.org/10.1038/nature04233}{Nature {\bf 438}, 197 (2005).}

\bibitem{Castro2009RMP}
A. H. Castro Neto, F. Guinea, N. M. R. Peres, K. S. Novoselov, and A. K. Geim,
\href{https://link.aps.org/doi/10.1103/RevModPhys.81.109}{Rev. Mod. Phys. {\bf 81}, 109 (2009).}

\bibitem{Moore2010Nature}
J. E. Moore, \href{https://www.nature.com/articles/nature08916}{Nature {\bf 464}, 194 (2010).}


\bibitem{Hasan2010RMP}
M. Z. Hasan and C. L. Kane, \href{https://journals.aps.org/rmp/abstract/10.1103/RevModPhys.82.3045}
{Rev. Mod. Phys. {\bf 82}, 3045 (2010).}

\bibitem{Qi2011RMP}
X. L. Qi and S. C. Zhang, \href{ https://journals.aps.org/rmp/abstract/10.1103/RevModPhys.83.1057}
{Rev. Mod. Phys. {\bf 83}, 1057 (2011).}



\bibitem{Montambaux}
P. Dietl, F. Piechon, and G. Montambaux,
\href{https://journals.aps.org/prl/abstract/10.1103/PhysRevLett.100.236405}{Phys. Rev. Lett. {\bf 100}, 236405 (2008)};
G. Montambaux, F. Pi\'{e}chon, J. -N. Fuchs, and M. O. Goerbig,
\href{https://journals.aps.org/prb/abstract/10.1103/PhysRevB.80.153412}{Phys. Rev. B {\bf 80}, 153412 (2009)};
P. Delplace and G. Montambaux,
\href{https://journals.aps.org/prb/abstract/10.1103/PhysRevB.82.035438}{Phys. Rev. B {\bf 82}, 035438 (2010)};
L. -K. Lim, J. -N. Fuchs, and Gilles Montambaux,
\href{https://journals.aps.org/prl/abstract/10.1103/PhysRevLett.108.175303}{Phys. Rev. Lett. {\bf 108},
175303 (2012).}

\bibitem{Savary2014PRB}
L. Savary, J. Ruhman, J. W. F. Venderbos, L. Fu, and P. A. Lee,
\href{https://journals.aps.org/prb/abstract/10.1103/PhysRevB.96.214514}
{Phys. Rev. B {\bf 96}, 214514 (2017).}

\bibitem{Moon2014PRX}
L. Savary, E. -G. Moon and L. Balents,
\href{https://journals.aps.org/prx/abstract/10.1103/PhysRevX.4.041027}
{Phys. Rev. X {\bf 4}, 041027 (2014)};
H. Oh, S. Lee, Y. -B. Kim, and E. -G. Moon,
\href{https://journals.aps.org/prl/abstract/10.1103/PhysRevLett.122.167201}
{Phys. Rev. Lett. {\bf 122}, 167201 (2019).}

\bibitem{Steinberg2014PRL}
J. -A. Steinberg, S. -M. Young, S. Zaheer, C. -L. Kane, E. -J. Mele, and A. -M. Rappe,
\href{https://journals.aps.org/prl/abstract/10.1103/PhysRevLett.112.036403}
{Phys. Rev. Lett. {\bf 112}, 036403 (2014).}

\bibitem{Wang2012PRB}
Z. J. Wang, Y. Sun, X. Q. Chen, C. Franchini, G. Xu, H.-M. Weng, X. Dai, and Z. Fang,
\href{https://journals.aps.org/prb/abstract/10.1103/PhysRevB.85.195320}
{Phys. Rev. B {\bf 85}, 195320 (2012).}

\bibitem{Young2012PRL}
S. M. Young, S. Zaheer, J. C. Y. Teo, C. L. Kane, E. J. Mele and A. M. Rappe,
\href{https://journals.aps.org/prl/abstract/10.1103/PhysRevLett.108.140405}
{Phys. Rev. Lett. {\bf 108}, 140405 (2012).}


\bibitem{Liu2014NM}
Z. K. Liu, J. Jiang, B. Zhou, Z. J. Wang, Y. Zhang, H. M. Weng, D. Prabhakaran, S. K. Mo, H. Peng, P. Dudin, T. Kim, M. Hoesch, Z. Fang, X. Dai, Z. X. Shen, D. L. Feng, Z. Hussain, and Y. L. Chen,
\href{ https://www.nature.com/articles/nmat3990}
{Nat. Mater. {\bf 13}, 677 (2014).}

\bibitem{Liu2014Science}
Z. K. Liu, B. Zhou, Y. Zhang, Z. J. Wang, H. M. Weng, D. Prabhakaran, S. K. Mo, Z. X. Shen, Z. Fang, X. Dai, Z. Hussain, and Y. L. Chen,
\href{ https://science.sciencemag.org/content/343/6173/864}
{Science {\bf 343}, 864 (2014).}


\bibitem{Vafek2014ARCMP}
O. Vafek and A. Vishwanath, Annu.
\href{https://www.annualreviews.org/doi/abs/10.1146/annurev-conmatphys-031113-133841}
{Rev. Condens. Matter Phys. {\bf 5}, 83 (2014).}

\bibitem{Wehling2014AP}
T. O. Wehling, A. M. Black-Schaffer, and A. V. Balatsky, Adv.
\href{https://www.tandfonline.com/doi/abs/10.1080/00018732.2014.927109}
{Phys. {\bf 63}, 1 (2014).}

\bibitem{Xiong2015Science}
J. Xiong, S. K. Kushwaha, T. Liang, J. W. Krizan,
M. Hirschberger, W. Wang, R. J. Cava and N. P. Ong,
\href{https://science.sciencemag.org/content/350/6259/413}
{Science {\bf 350}, 413 (2015).}


\bibitem{Roy2009PRB}
R. Roy, \href{https://journals.aps.org/prb/abstract/10.1103/PhysRevB.79.195322}
{Phys. Rev. B {\bf 79}, 195322 (2009).}

\bibitem{Roy2016}
B. Roy, S. Das Sarma,
\href{https://journals.aps.org/prb/abstract/10.1103/PhysRevB.94.115137}
{Phys. Rev. B {\bf 94}, 115137 (2016)};
B. Roy, Y. Alavirad, and J. D. Sau,
\href{https://journals.aps.org/prl/abstract/10.1103/PhysRevLett.118.227002}
{Phys. Rev. Lett. {\bf 94}, 227002 (2017)};
B. Roy, R. -J. Slager, and V. Juricic,
\href{https://journals.aps.org/prx/abstract/10.1103/PhysRevX.8.031076}
{Phys. Rev. X {\bf 8}, 031076 (2018)} ;
B. Roy , V. Juricic, and S. Das Sarma,
\href{https://link.springer.com/content/pdf/10.1038/srep32446.pdf}
{Sci. Rep. {\bf 6}, 32446 (2016).}

\bibitem{Roy-2014-2016}
H. -H. Lai, B. Roy, and P. Goswami,
\href{https://arxiv.org/abs/1409.8675}
{arXiv: 1409.8675 (2014)};
P. Goswami, B. Roy, and S. Das Sarma,
\href{https://journals.aps.org/prb/abstract/10.1103/PhysRevB.95.085120}
{Phys. Rev. B {\bf 95}, 085120 (2017)};
A. L. Szabo, R. Moessner, and B. Roy,
\href{https://journals.aps.org/prb/abstract/10.1103/PhysRevB.103.165139}
{Phys. Rev. B {\bf 103}, 165139 (2021)};
B. Roy, S. A. Akbar Ghorashi, M. S. Foster, and A. H. Nevidomskyy,
\href{https://journals.aps.org/prb/abstract/10.1103/PhysRevB.99.054505}
{Phys. Rev. B {\bf 99}, 054505 (2019)}.

\bibitem{Korshunov2014PRB}
M. M. Korshunov, D. V. Efremov, A. A. Golubov and
O. V. Dolgov,
\href{https://journals.aps.org/prb/abstract/10.1103/PhysRevB.90.134517}
{Phys. Rev. B {\bf 90}, 134517 (2014).}

\bibitem{Hung2016PRB}
H. H. Hung, A. Barr, E. Prodan and G. A. Fiete,
\href{ https://journals.aps.org/prb/abstract/10.1103/PhysRevB.94.235132}
{Phys. Rev. B {\bf 94}, 235132 (2016).}

\bibitem{Nandkishore2013PRB}
R. Nandkishore, J. Maciejko, D. A. Huse, and S. L. Sondhi,
\href{ https://journals.aps.org/prb/abstract/10.1103/PhysRevB.87.174511}
{Phys. Rev. B {\bf 87}, 174511 (2013).}

\bibitem{Potirniche2014PRB}
I. D. Potirniche, J. Maciejko, R. Nandkishore, and S. L. Sondhi,
\href{ https://journals.aps.org/prb/abstract/10.1103/PhysRevB.90.094516}
{Phys. Rev. B {\bf 90}, 094516 (2014).}

\bibitem{Nandkishore2017PRB}
R. M. Nandkishore and S. A. Parameswaran,
\href{ https://journals.aps.org/prb/abstract/10.1103/PhysRevB.95.205106}
{Phys. Rev. B {\bf 95}, 205106 ( 2017).}

\bibitem{Soluyanov2015Nature}
A. A. Soluyanov, D. Gresch, Z. J. Wang, Q. S. Wu, M. Troyer, X. Dai,
and B. A. Bernevig,
\href{https://www.nature.com/articles/nature15768}
{Nature (London) {\bf 527}, 495 (2015)}.

\bibitem{Trescher2015PRB}
M. Trescher, B. Sbierski, P. W. Brouwer, and E. J. Bergholtz,
\href{https://journals.aps.org/prb/abstract/10.1103/PhysRevB.91.115135}
{Phys. Rev. B {\bf 91}, 115135 (2015)}.

\bibitem{Jafari2019PRB-t}
S. A. Jafari,
\href{https://journals.aps.org/prb/abstract/10.1103/PhysRevB.100.045144}
{Phys. Rev. B {\bf 100}, 045144 (2019)}.

\bibitem{Mao2011ACS}
Y. Mao, W. L. Wang, D. Wei, E. Kaxiras, and J. G. Sodroski,
\href{https://pubs.acs.org/doi/10.1021/nn103153x}
{ACS Nano {\bf 5}, 1395 (2011).}


\bibitem{Lee2018PRB}
Y. -W. Lee and Y. -L. Lee,
\href{https://journals.aps.org/prb/abstract/10.1103/PhysRevB.97.035141}
{Phys. Rev. B {\bf 97}, 035141 (2018)}.

\bibitem{Lee2019PRB}
Y. -L. Lee and Y. -W. Lee,
\href{https://journals.aps.org/prb/abstract/10.1103/PhysRevB.100.075156}
{Phys. Rev. B {\bf 100}, 075156 (2019)}.






\bibitem{Noh2017PRL}	
H. J. Noh, J. Jeong, E. J. Cho, K. Kim, B. I. Min, and B. G. Park,
\href{https://journals.aps.org/prl/abstract/10.1103/PhysRevLett.119.016401}
{Phys. Rev. Lett. {\bf 119}, 016401 (2017)}.

\bibitem{Fei2017PRB}
F. C. Fei, X. Y. Bo, R. Wang, B. Wu, J. Jiang, D. Z. Fu, M. Gao, H. Zheng, Y. L. Chen,
X. F. Wang, H. J. Bu, F. Q. Song, X. G. Wang, B. G. Wang, and G. H. Wang,
\href{https://journals.aps.org/prb/abstract/10.1103/PhysRevB.96.041201}
{Phys. Rev. B {\bf 96}, 041201(R) (2017)}.

\bibitem{Peres2010RMP}
N. M. R. Peres,
\href{https://journals.aps.org/rmp/abstract/10.1103/RevModPhys.82.2673}
{Rev. Mod. Phys. {\bf 82}, 2673 (2010)}.


\bibitem{Katayama2006JPSJ}
S. Katayama, A. Kobayashi, and Y. Suzumura,
\href{https://journals.jps.jp/doi/abs/10.1143/JPSJ.75.054705}
{J. Phys. Soc. Jpn. {\bf 75}, 054705 (2006)}.

\bibitem{Kobayashi2007JPSJ}
A. Kobayashi, S. Katayama, Y. Suzumura, and H. Fukuyama,
\href{https://journals.jps.jp/doi/abs/10.1143/JPSJ.76.034711}
{J. Phys. Soc. Jpn. {\bf 76}, 034711 (2007)}.

\bibitem{Goerbig2008PRB}
M. O. Goerbig, J. -N. Fuchs, G. Montambaux, and F. Pi\'{e}chon,
\href{https://journals.aps.org/prb/abstract/10.1103/PhysRevB.78.045415}
{Phys. Rev. B {\bf 78}, 045415 (2008)}.

\bibitem{Xu2015PRL}
Y. Xu, F. Zhang, and C. Zhang,
\href{https://journals.aps.org/prl/abstract/10.1103/PhysRevLett.115.265304}
{Phys. Rev. Lett. {\bf 115}, 265304 (2015)}.

\bibitem{Xu2016PRA}
Y. Xu and L. -M. Duan,
\href{https://journals.aps.org/pra/abstract/10.1103/PhysRevA.94.053619}
{Phys. Rev. A {\bf 94}, 053619 (2016)}.

\bibitem{Jafari}
S. A. Jafari,
\href{https://link.springer.com/content/pdf/10.1140/epjb/e2009-00128-1.pdf}
{Eur. Phys. J. B {\bf 68}, 537 (2009)};
Z. Jalali-Mola and S. A. Jafari,
\href{https://journals.aps.org/prb/abstract/10.1103/PhysRevB.100.075113}
{Phys. Rev. B {\bf 100}, 075113 (2019)}.

\bibitem{Yan2017NC}
M. Z. Yan, H. Q. Huang, K. N. Zhang, E. Wang, W. Yao, K. Deng, G. L. Wan,
H. Y. Zhang, M. Arita, H. T. Yang, Z. Sun, H. Yao, Y. Wu, S. S. Fan, W. H. Duan,
and S. Y. Zhou,
\href{https://www.nature.com/articles/s41467-017-00280-6}
{Nat. Commun {\bf 8}, 257 (2017)}.

\bibitem{Fritz2017PRB}
F. Detassis, L. Fritz, and S. Grubinskas,
\href{https://journals.aps.org/prb/abstract/10.1103/PhysRevB.96.195157}
{Phys. Rev. B {\bf 96}, 195157 (2017)}.

\bibitem{Qiong2019NPJB}
J. Q. Li, D. X. Zheng, and J. Wang,
\href{https://link.springer.com/article/10.1140/epjb/e2019-100373-9}
{Eur. Phys. J. B {\bf 92}, 274 (2019)}.

\bibitem{Shekhar2015NP}
C. Shekhar, A. K. Nayak, Y. Sun, M. Schmidt, M. Nicklas,
I. Leermakers, U. Zeitler, Y. Skourski, J. Wosnitza,
Z. Liu, Y. Chen, W. Schnelle, H. Borrmann, Y. Grin,
C. Felser, and B. Yan,
\href{https://www.nature.com/articles/nphys3372}
{Nature Physics {\bf 11}, 3372 (2015)}.

\bibitem{Parameswaran2014PRX}
S. A. Parameswaran, T. Grover, D. A. Abanin, D. A. Pesin,
and A. Vishwanath,
\href{https://journals.aps.org/prx/abstract/10.1103/PhysRevX.4.031035}
{Phys. Rev. X {\bf 4}, 031035 (2014)}.

\bibitem{Potter2014NC}
A. C. Potter, I. Kimchi, and A. Vishwanath,
\href{https://www.nature.com/articles/ncomms6161}
{Nature Communications {\bf 5}, 5161 (2014)}.

\bibitem{Baum2015PRX}
Y. Baum, E. Berg, S. A. Parameswaran, and A. Stern,
\href{https://journals.aps.org/prx/abstract/10.1103/PhysRevX.5.041046}
{Phys. Rev. X {\bf 5}, 041046 (2015)}.

\bibitem{Arnold2016NC}
F. Arnold, C. Shekhar, S. -C. Wu, Y. Sun, R. D. dos Reis, N. Kumar, M. Naumann,
M. O. Ajeesh, M. Schmidt, A. G. Grushin, J. H. Bardarson, M. Baenitz, D. Sokolov,
H. Borrmann, M. Nicklas, C. Felser, E. Hassinger, and B. Yan,
\href{https://www.nature.com/articles/ncomms11615}
{Nature Communications {\bf 7}, 11615 (2016)}.

\bibitem{Zhang2016NC}
C. -L. Zhang, S. -Y. Xu, I. Belopolski, Z. Yuan, Z. Lin, B. Tong, G. Bian, N. Alidoust,
C. -C. Lee, S. -M. Huang, T. R. Chang, G. Chang, C. -H. Hsu, H. -T. Jeng, M. Neupane,
D. S. Sanchez, H. Zheng, J. Wang, H. Lin, C. Zhang, H. -Z. Lu, S. -Q. Shen, T. Neupert,
M. Zahid Hasan, and S. Jia,
\href{https://doi.org/10.1038/ncomms10735}
{Nature Communications {\bf 7}, 10735 (2016)}.


\bibitem{Fritz2019arXiv}
T. S. Sikkenk and L. Fritz,
\href{https://journals.aps.org/prb/abstract/10.1103/PhysRevB.100.085121}
{Phys. Rev. B {\bf 100}, 085121 (2019)}.

\bibitem{Jafari2018PRB}
Z. Jalali-Mola and S. A. Jafari,
\href{https://journals.aps.org/prb/abstract/10.1103/PhysRevB.98.195415}
{Phys. Rev. B {\bf 98}, 195415 (2018)}.

\bibitem{Yang2018PRB}	
Z. -K. Yang, J. -R. Wang, and G. -Z. Liu,
\href{https://journals.aps.org/prb/abstract/10.1103/PhysRevB.98.195123}
{Phys. Rev. B {\bf 98}, 195123 (2018)}.

\bibitem{Proskurin2015PRB}
I. Proskurin, M. Ogata, and Y. Suzumura,
\href{https://journals.aps.org/prb/abstract/10.1103/PhysRevB.91.195413}
{Phys. Rev. B {\bf 91}, 195413 (2015)}.

\bibitem{Brien2016PRL}
T. E. O'Brien, M. Diez, and C. W. J. Beenakker,
\href{https://journals.aps.org/prl/abstract/10.1103/PhysRevLett.116.236401}
{Phys. Rev. Lett. {\bf 116}, 236401 (2016)};
Z. M. Yu, Y. Yao, and S. A. Yang,
\href{https://journals.aps.org/prl/abstract/10.1103/PhysRevLett.117.077202}
{Phys. Rev. Lett. {\bf 117}, 077202 (2016)};
M. Udagawa and E. J. Bergholtz,
\href{https://journals.aps.org/prl/abstract/10.1103/PhysRevLett.117.086401}
{Phys. Rev. Lett. {\bf 117}, 086401 (2016)};
S. Tchoumakov, M. Civelli, and M. O. Goerbig,
\href{https://journals.aps.org/prl/abstract/10.1103/PhysRevLett.117.086402}
{Phys. Rev. Lett. {\bf 117}, 086402 (2016)}.

\bibitem{Zyuzin2016JETPL}
A. A. Zyuzin and R. P. Tiwari,
\href{https://link.springer.com/article/10.1134/S002136401611014X}
{JETP Lett. {\bf 103}, 717 (2016)};
J. F. Steiner, A. V. Andreev, and D. A. Pesin,
\href{https://journals.aps.org/prl/abstract/10.1103/PhysRevLett.119.036601}
{Phys. Rev. Lett. {\bf 119}, 036601 (2017)}.

\bibitem{Ferreiros2017PRB}
Y. Ferreiros, A. A. Zyuzin, and J. H. Bardarson,
\href{https://journals.aps.org/prb/abstract/10.1103/PhysRevB.96.115202}
{Phys. Rev. B {\bf 96}, 115202 (2017)};
S. Saha and S. Tewari,
\href{https://link.springer.com/article/10.1140/epjb/e2017-80437-4}
{Eur. Phys. J. B {\bf 91}, 4 (2018)}.

\bibitem{Alidoust2019arXiv}
M. Alidoust and K. Halterman,
\href{https://journals.aps.org/prb/abstract/10.1103/PhysRevB.101.035120}
{Phys. Rev. B {\bf 101}, 035120 (2020)}.


\bibitem{Ruhman2019PRX}
V. Kozii, Z. Bi, and J. Ruhman,
\href{https://journals.aps.org/prx/abstract/10.1103/PhysRevX.9.031046}
{Phys. Rev. X {\bf 9}, 031046 (2019)}.

\bibitem{Wilson1975RMP}
K. G. Wilson,
\href{https://journals.aps.org/rmp/abstract/10.1103/RevModPhys.47.773}
{Rev. Mod. Phys. {\bf 47} 773 (1975)}.

\bibitem{Polchinski9210046}
J. Polchinski,
\href{https://arxiv.org/abs/hep-th/9210046}
{arXiv: hep-th/9210046 (1992)}.

\bibitem{Shankar1994RMP}
R. Shankar,
\href{https://journals.aps.org/rmp/abstract/10.1103/RevModPhys.66.129}
{Rev. Mod. Phys. {\bf 66}, 129 (1994)}.

\bibitem{Khmelnitskii1971SPSS}
D. E. Khmelnitskii and V. L. Shneerson,
Sov. Phys. Solid State {\bf 13}, 687 (1971).

\bibitem{Strukov2012Book}
B. A. Strukov and A. P. Levanyuk,
\emph{Ferroelectric Phenomena in Crystals: Physical Foundations}
(Springer Science, New York, 2012)

\bibitem{Moon2016SR}
G. -Y. Cho and E. -G. Moon,
\href{https://gfcfxdd05cbc5aacc4e82skonn0wbkqvnb659ffaah.eds.tju.edu.cn/articles/srep19198}
{Sci. Rep. {\bf 6}, 19198 (2016).}

\bibitem{Moon2016PRB}
Ye. Huh, E. -G. Moon, and Y. -B. Kim,
\href{https://journals.aps.org/prb/abstract/10.1103/PhysRevB.93.035138}
{Phys. Rev. B {\bf 93}, 035138 (2016).}

\bibitem{Mandal2022PRB}
I. Mandal and H. Freire,
\href{https://journals.aps.org/prb/abstract/10.1103/PhysRevB.103.195116}
{Phys. Rev. B {\bf 103}, 195116 (2022).}

\bibitem{Huh2008PRB}
Y. Huh, S. Sachdev,
\href{https://journals.aps.org/prb/abstract/10.1103/PhysRevB.78.064512}
{Phys. Rev. B {\bf 78}, 064512 (2008)};
E. -A. Kim, M. J. Lawler, P. Oreto, S. Sachdev, E. Fradkin,
and S. A. Kivelson,
\href{https://journals.aps.org/prb/abstract/10.1103/PhysRevB.77.184514}
{Phys. Rev. B {\bf 77}, 184514 (2008)}.

\bibitem{She2010PRB}
J. -H She, J. Zaanen, A. R. Bishop, and A. V. Balatsky,
\href{https://journals.aps.org/prb/abstract/10.1103/PhysRevB.82.165128}
{Phys. Rev. B {\bf 82}, 165128 (2010)}.

\bibitem{Wang2011PRB}	
J. Wang, G. -Z. Liu, and H. Kleinert,
\href{https://journals.aps.org/prb/abstract/10.1103/PhysRevB.83.214503}
{Phys. Rev. B {\bf 83}, 214503 (2011)}.

\bibitem{Dong2020PRB}
Y. M. Dong, Y. H. Zhai, D. X. Zheng, and J. Wang,
\href{https://journals.aps.org/prb/abstract/10.1103/PhysRevB.102.134204}
{Phys. Rev. B {\bf 102}, 134204 (2020)}.

\bibitem{Maiti2010PRB}
S. Maiti and A.V. Chubukov,
\href{https://journals.aps.org/prb/abstract/10.1103/PhysRevB.82.214515}
{Phys. Rev. B. {\bf 82}, 214515 (2010)}.

\bibitem{Halboth2000RPL}
C. J. Halboth and W. Metzner,
\href{https://journals.aps.org/prl/abstract/10.1103/PhysRevLett.85.5162}
{Phys. Rev. Lett. {\bf 85}, 5162 (2000)}.

\bibitem{Halboth2000RPB}
C. J. Halboth and W. Metzner,
\href{https://journals.aps.org/prb/abstract/10.1103/PhysRevB.61.7364}
{Phys. Rev. B {\bf 61}, 7364 (2000)}.

\bibitem{Nandkishore2012NP}
R. Nandkishore, L. S. Levitov, and A. V. Chubukov,
\href{https://www.nature.com/articles/nphys2208}
{Nat. Phys. {\bf 8}, 158 (2012)}.

\bibitem{Cvetkovic2012PRB}
V. Cvetkovi$\acute{c}$, R. E. Throckmorton, and O. Vafek,
\href{https://journals.aps.org/prb/abstract/10.1103/PhysRevB.86.075467}
{Phys. Rev. B {\bf 86}, 075467 (2012)}.

\bibitem{Wang2020NPB}
J. Wang, \href{https://doi.org/10.1016/j.nuclphysb.2020.115230}{
Nuclear Physics B {\bf 961}, 115230 (2020).}

\bibitem{Vafek2010PRB}
O. Vafek,
\href{https://journals.aps.org/prb/abstract/10.1103/PhysRevB.82.205106}
{Phys. Rev. B {\bf 82}, 205106 (2010)}.

\bibitem{Murray2014PRB}
J. M. Murray and O. Vafek,
\href{https://journals.aps.org/prb/abstract/10.1103/PhysRevB.89.201110}
{Phys. Rev. B {\bf 89}, 201110(R) (2014)}.

\bibitem{Zhai2021NPB}
Y. H. Zhai and J. Wang,
\href{https://www.sciencedirect.com/science/article/pii/S0550321321000687}
{Nucl. Phys. B {\bf 966}, 115371 (2021)}.

\bibitem{Roy2009.05055}
A. L. Szab\'{o} and B. Roy,
\href{https://arxiv.org/abs/2009.05055}
{arXiv: 2009.05055 (2020)}.

\bibitem{Roy2018RRX}
B. Roy and M. S. Foster,
\href{https://journals.aps.org/prx/abstract/10.1103/PhysRevX.8.011049}
{Phys. Rev. X {\bf 8}, 011049 (2018)}.

\bibitem{Mahan1990Book}
G. D. Mahan,
\href{https://books.google.com/books?hl=zh-CN&lr=&id=xzSgZ4-yyMEC&oi=fnd&pg=
PR9&dq=G.+Mahan,+Many-Particle+Physics,+2nd+edn.&ots=KwyYnd3pa0&sig=XATYtCSnUXKWDp6X125MqpSx3dA}
{\emph{Many-Particle Physics, 2nd edn.} (Plenum, New York, 2000)}.

\bibitem{Wang2012PRD}
J. Wang and G. -Z. Liu,
\href{http://dx.doi.org/10.1103/PhysRevD.85.105010}
{Phys. Rev. D {\bf 85}, 105010 (2012).}

\bibitem{Schwabl2006Book}
F. Schwabl, \emph{Statistical Mechanics} (Springer, Berlin, 2006), 2nd ed.

\bibitem{Sarma2007PRL}
E. -H. Hwang and B. Y. K. Hu, and S. -D. Sarma,
\href{https://journals.aps.org/prl/abstract/10.1103/PhysRevLett.99.226801}
{Phys. Rev. Lett. {\bf 99}, 226801 (2007).}


\bibitem{Kapusta1994Book}
J. I. Kapusta and C. Gale,
\href{https://books.google.com/books?hl=zh-CN&lr=&id=rll8dJ2iTpsC&oi=fnd&pg=PA1&dq=J.+I.+Kapusta+and+C.+Gale,
+Finite-Temperature+Field+Theory:+Principles+and+Applications&ots=yZHN6ltCeA&sig=TTbZJU2OmU4lKIymaGhNsnwPoK0}
{\emph{Finite-Temperature Field
Theory: Principles and Applications} (Cambridge, UK;
New York, 1994)}.




























\end{thebibliography}





%\end{CJK*}

\end{document}
