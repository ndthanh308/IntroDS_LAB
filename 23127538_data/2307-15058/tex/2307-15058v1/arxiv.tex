% This is samplepaper.tex, a sample chapter demonstrating the
% LLNCS macro package for Springer Computer Science proceedings;
% Version 2.20 of 2017/10/04
%
\documentclass[runningheads]{llncs}
%
\usepackage{comment}
\usepackage{graphicx}
\usepackage{tabularx}
\usepackage{hyperref}
\usepackage{color}
\usepackage{booktabs}
\usepackage{amsfonts, amsmath, amssymb}
\usepackage[thicklines]{cancel}
\usepackage{multirow, multicol, graphicx}
\usepackage[table,xcdraw]{xcolor}
\usepackage{threeparttable}
\usepackage{marvosym}
% Used for displaying a sample figure. If possible, figure files should
% be included in EPS format.
%
% If you use the hyperref package, please uncomment the following line
% to display URLs in blue roman font according to Springer's eBook style:
\renewcommand\UrlFont{\color{blue}\rmfamily}

\begin{document}
\newcommand{\wuzirui}[2]{\textcolor{red}{(#1)}\textcolor{blue}{#2}}
\newcommand{\yl}[1]{\textcolor{cyan}{\textbf{YL}: #1}}
\newcommand{\zknote}[1]{\textcolor{red}{\textbf{ZK}:(#1)}}
\newcommand{\ssec}[1]{\vspace{1mm}\noindent\textbf{#1.}}

\title{MARS: An Instance-aware, Modular and Realistic Simulator for Autonomous Driving}

% INITIAL SUBMISSION 
\def\CICAISubNumber{12}  % Insert your submission number here
%\begin{comment}
% \titlerunning{CICAI2023 submission ID \CICAISubNumber} 
% \authorrunning{CICAI2023 submission ID \CICAISubNumber} 
% \author{Anonymous CICAI submission}
% \institute{Paper ID \CICAISubNumber\vspace{-5mm}}
%\end{comment}
%******************

% CAMERA READY SUBMISSION
% \begin{comment}
\titlerunning{MARS}
% If the paper title is too long for the running head, you can set
% an abbreviated paper title here
%
\author{Zirui Wu\inst{1,2} \and
Tianyu Liu\inst{1,3} \and
Liyi Luo\inst{1,4}\and
Zhide Zhong\inst{1,5}\and
Jianteng Chen\inst{1,5}\and
Hongmin Xiao\inst{1,6}\and
Chao Hou\inst{1,7}\and
Haozhe Lou\inst{1,8}\and
Yuantao Chen\inst{1,9}\and\\
Runyi Yang\inst{1,10}\and
Yuxin Huang\inst{1,5}\and
Xiaoyu Ye\inst{1,5}\and\\
Zike Yan\inst{1}\and
Yongliang Shi\inst{1}\and
Yiyi Liao\inst{11}\and
Hao Zhao\inst{1}\thanks{Corresponding author. \email{zhaohao@air.tsinghua.edu.cn}.\\
Sponsored by Tsinghua-Toyota Joint Research Fund (20223930097) and Baidu Inc. through Apollo-AIR Joint Research Center.} %$^{\href{mailto:zhaohao@air.tsinghua.edu.cn}{\textrm{\Letter}}}$
}
%
\authorrunning{Z. Wu et al.}
\institute{$^1$ AIR, Tsinghua University; $^2$ System Thurst, HKUST(GZ); $^3$ HKUST; \\ $^4$ McGill University; $^5$ Beijing Institute of Technology; $^6$ National University of Singapore; $^7$ HKU; $^8$ University of Wisconsin Madison; $^9$ Xi’an University of Architecture and Technology; $^{10}$ Imperial College London; $^{11}$ Zhejiang University \vspace{-5mm}
 }
% First names are abbreviated in the running head.
% If there are more than two authors, 'et al.' is used.
%
% \institute{AiR, Tsinghua University  \and
 % System Thurst, The Hong Kong University of Science and Technology (Guangzhou) \and   The Hong Kong University of Science and Technology \and McGill University \and Beijing Institute of Technology \and The Hong Kong University \and Xi’an University of Architecture and Technology \and Imperial College; \and Zhejiang University
% \email{lncs@springer.com}\\
% \url{http://www.springer.com/gp/computer-science/lncs} \and
% ABC Institute, Rupert-Karls-University Heidelberg, Heidelberg, Germany\\
% \email{\{abc,lncs\}@uni-heidelberg.de}}
% \author{%
%   Zirui Wu$^{1,2}$,  Tianyu Liu$^{1,3}$, Liyi Luo$^{1,4}$, Zhide Zhong$^{1,5}$, Jianteng Chen$^{1,5}$, Hongmin Xiao$^{1,5}$,\\ \And Chao Hou$^{1,6}$, Haozhe Lou$^{1,7}$, Yuantao Chen$^{1,8}$, Runyi Yang$^{1,9}$, Yuxin Huang$^{1,5}$, Xiaoyu Ye$^{1,5}$,\\ \And Zike Yan$^{1}$,  Yongliang Shi$^{1}$, Yiyi Liao$^{10}$, Hao Zhao$^{1}$\thanks{Corresponding author (\href{mailto://zhaohao@air.tsinghua.edu.cn}{zhaohao@air.tsinghua.edu.cn}).}\\\And\\ 
%   $^1$ AiR, Tsinghua University; $^2$ System Thurst, HKUST(GZ); $^3$ HKUST; \\ $^4$ McGill University; $^5$ Beijing Institute of Technology; $^6$ The Hong Kong University; \\$^7$ University of Wisconsin-Madison; $^8$ Xi’an University of Architecture and Technology;\\ $^9$ Imperial College; $^{10}$ Zhejiang University
% }

% \end{comment}
%******************

\maketitle              % typeset the header of the contribution

\begin{abstract}
Nowadays, autonomous cars can drive smoothly in ordinary cases, and it is widely recognized that realistic sensor simulation will play a critical role in solving remaining corner cases by simulating them. To this end, we propose an autonomous driving simulator based upon neural radiance fields (NeRFs). Compared with existing works, ours has three notable features: (1) \textbf{Instance-aware}. Our simulator models the foreground instances and background environments separately with independent networks so that the static (e.g., size and appearance) and dynamic (e.g., trajectory) properties of instances can be controlled separately. (2) \textbf{Modular}. Our simulator allows flexible switching between different modern NeRF-related backbones, sampling strategies, input modalities, etc. We expect this modular design to boost academic progress and industrial deployment of NeRF-based autonomous driving simulation. (3) \textbf{Realistic}. Our simulator set new state-of-the-art photo-realism results given the best module selection. Our simulator will be \textbf{open-sourced} while most of our counterparts are not. Project page:  \url{https://open-air-sun.github.io/mars/}.
% This is evidenced by state-of-the-art performance on standard benchmarks KITTI and V-KITTI. For example, we can achieve 27.01 PSNR on V-KITTI using 25\% training data while the previous best-published result is 22.18. Our simulator will be \textbf{open-sourced} while most of our counterparts are not.

\keywords{Autonomous Driving Simulator  \and Neural Radiance Fields.}
\end{abstract}

% % Figure environment removed

\section{Introduction}
Autonomous driving~\cite{geiger_vision_2013,hu_planning-oriented_2023,zheng_steps_2023,li_lode_2023,10160470,jin_adapt_2023} is arguably the most important application of modern 3D scene understanding~\cite{chen_pq-transformer_2022,tian_vibus_2022} techniques. Nowadays, Robotaxis can run in big cities with up-to-date HD maps, handling everyday driving scenarios smoothly. However, once a corner case that lies out of the distribution of an autonomous driving algorithm happens on the road unexpectedly, the lives of passengers are put at risk. The dilemma is that while we need more training data about corner cases, collecting them in the real world usually means danger and high expenses. To this end, the community believes that photorealistic simulation~\cite{li_aads_2019,chen_geosim_2021,yang_unisim_2023,fu_panoptic_2022} is a technical path of great potential. If an algorithm can experience enormous corner cases in a simulator with a small sim-to-real gap, the performance bottleneck of current autonomous driving algorithms can be potentially addressed. 
% To this end, we present a NeRF-based simulator for autonomous driving that is \textbf{instance-aware, modular and realistic}.

Existing autonomous driving simulation methods have their own limitations. CARLA~\cite{dosovitskiy_carla_2017} is a widely used sensor simulator based upon traditional graphics engines, whose realism is restricted by asset modeling and rendering qualities. AADS~\cite{li_aads_2019} also exploits traditional graphics engines but demonstrates impressive photorealism using well-curated assets. On the other hand, GeoSim~\cite{chen_geosim_2021} introduces a data-driven scheme for realistic simulation by learning an image enhancement network. Flexible asset generation and rendering can be achieved through image composition with promisingly good geometry and realistic appearance.

In this paper, we take advantage of the realistic rendering ability of NeRFs for autonomous driving simulation. Training data captured from real-world environments guarantees a small sim-to-real gap. Several works also exploit NeRFs to model cars~\cite{niemeyer_giraffe_2021} and static backgrounds~\cite{fu_panoptic_2022} in outdoor environments. However, the inability to model complex dynamic scenes that are composed of both moving objects and static environments limits their practical use for real-world sensor simulation. Recently, Neural Scene Graph (NSG)~\cite{ost_neural_2021} decomposes dynamic scenes into learned scene graphs and learns latent representations for category-level objects. However, its multi-plane-based representation for background modeling cannot synthesize images under large viewpoint changes. 

% \yl{how about PNF and UniSim?} \wuzirui{adjust}{ \textbf{When compared with other methods in the literature as listed above}, 
To be specific, our central contribution is the very first \textbf{open-source} NeRF-based modular framework for photorealistic autonomous driving simulation. The proposed pipeline models foreground instances and background environments in a decomposed fashion. Different NeRF backbone architectures and sampling methods are incorporated in a unified manner with multi-modal inputs supported. The best module combination of the proposed framework achieves state-of-the-art rendering performance on public benchmarks with large margins, indicating photorealistic simulation results.


% Figure environment removed


% We demonstrate our proposed framework in Fig.~\ref{fig:main-figure}.
\vspace{-2mm}
\section{Method}
\vspace{-2mm}
\ssec{Overview}
As illustrated in~Fig.~\ref{fig:main-figure}, we aim to provide a modular framework for constructing compositional neural radiance fields, where realistic sensor simulation can be conducted for outdoor driving scenes.
% By driving scene (shown in Fig.~\ref{fig:main-figure} left-middle), 
A large unbounded outdoor environment with plenty of dynamic objects is taken into consideration.
% , which usually consist of RGB cameras, LiDARs, and IMU/GPS.

The input to the system consists of a set of RGB-images $\{\mathcal{I}_i\}^N$ (captured by vehicle-side or roadside sensors), sensor poses $\{\mathcal{T}_i\}^N$ (calculated using IMU/GPS signals), and object tracklets (including 3D bounding boxes $\{\mathcal{B}_{ij}\}^{N\times M}$, categories $\{\mathtt{type}_{ij}\}^{N\times M}$, and instance IDs $\{\mathtt{idx}_{ij}\}^{N\times M}$). $N$ is the number of input frames and $M$ is the number of tracked instances $\{\mathcal{O}_j\}^M$ across the whole sequence. An optional set of depth maps $\{\mathcal{D}_i\}^N$ and semantic segmentation masks $\{\mathcal{S}_i\}^N$ can also be adopted as extra supervision signals during training.
By constructing a compositional neural field, the proposed framework can simulate realistic sensor perception signals (including RGB images, depth maps, semantic segmentation masks, etc.) at given sensor poses. Instance editing on object trajectories and appearances is also supported.


\ssec{Pipeline} Our framework model each foreground instance and the background node compositionally.  As shown in Fig.~\ref{fig:main-figure}, when querying properties (RGB, depth, semantics, etc.) of a given ray $\mathbf{r}$, we first calculate its intersection with all visible objects' 3D bounding boxes to get the entering and leaving distances $[t_\mathtt{in}, t_\mathtt{out}]$. Afterward, both the background node (Fig.~\ref{fig:main-figure} left-top) and the foreground object nodes (Fig.~\ref{fig:main-figure} left-bottom) are queried, where each node samples a set of 3D points and uses its specific neural representation network to obtain point properties (RGB, density, semantics, etc.). Specifically, to query foreground nodes, we convert the ray origins and directions from world space into instance frames according to the object tracklets. Finally, all the ray samples from background and foreground nodes are composed and volume-rendered to produce pixel-wise rendering results (Fig.~\ref{fig:main-figure} right, Sec.~\ref{sec:rendering}).

We observe that the nature of background nodes (typically unbounded large-scale scenes) differs from the object-centric foreground nodes, while current works~\cite{kundu_panoptic_2022,ost_neural_2021} in sensor simulation use unified NeRF models. Our framework provides a flexible and open-sourced framework that supports different design choices of scene representations for background and foreground nodes and can easily incorporate new state-of-the-art methods of static scene reconstruction and object-centric reconstructions.

\subsection{Scene Representation}
\label{sec:scene-representation}

We decompose the scene into a large-scale unbounded NeRF (as the background node) and multiple object-centric NeRFs (as independent foreground nodes). Conventionally, a neural radiance field maps a given 3D point coordinate $\mathbf{x} = (x,y,z), \mathbf{x} \in \mathbb{R}^3$ and a 2D viewing direction $\mathbf{d}\in\mathbb{S}^2$ to its radiance $\mathbf{c}$ and volume density $\sigma$ shown in Eq.~\ref{eq:nerf-field}. Based upon this seminal representation, many variants have been proposed for different purposes, so we take a modular design.
\begin{equation}
\label{eq:nerf-field}
    f (\mathbf{x}, \mathbf{d}) = (\mathbf{c}, \sigma): [\mathbb{R}^3, \mathbb{S}^2] \to [\mathbb{R}^3, \mathbb{R}^+]
\end{equation}
The challenge of modeling unbounded background scene photo-realistically lies in accurately representing far regions, so we utilize the unbounded scene warping~\cite{barron_mip-nerf_2022} to contract the far region. For foreground nodes, we support both the code-conditioned representation $f(\mathbf{x}, \mathbf{d}, \mathbf{z})= (\textbf{c}, \sigma)$ ($\mathbf{z} \in \mathbb{R}^k$ denotes the instance-wise latent code) and the conventional ones, which will be explained as follows.

\subsubsection{Architectures.} In our modular framework, we support various NeRF backbones, which can be roughly categorized into two hyper-classes: MLP-based methods~\cite{mildenhall_nerf_2020,barron_mip-nerf_2021,barron_mip-nerf_2022}, or grid-based methods that store spatially-variant features in their hash grid voxel vertices~\cite{muller_instant_2022,tancik_nerfstudio_2023}. Although these architectures differ from each other in details, they follow the same high-level formulation of Eq.~\ref{eq:nerf-field} and are capsuled in modules under a unified interface in MARS.

% Through extensive experiments, we recommend parameterizing the background representation as 
While the MLP-based representations are simple in mathematical form, we give a formal exposition of grid-based methods.
The specific implementation of a multi-resolution feature grid $\{\mathcal{G}_\theta^{l}\}^L_{l=1}$ has layer-wise resolutions $\displaystyle R_l:=\lfloor R_\text{min}\cdot b^l\rfloor, b = \exp\left(\frac{\ln R_{\max{}} - \ln R_{\min{}}}{L-1}\right)$, where $R_{\min{}}, R_{\max{}}$ are the coarsest and the finest resolution~\cite{yu_monosdf_2022,muller_instant_2022}. The coordinates $\mathbf{x}$ are first scaled to each resolution before being processed by the ceiling and flooring operations to $\lceil \mathbf{x}\cdot R_l\rceil, \lfloor\mathbf{x}\cdot R_l\rfloor$ and hashed to obtain table indexes~\cite{muller_instant_2022}. The extracted feature vectors are then tri-linearly interpolated and decoded through a shallow MLP.
\begin{equation}
    (\mathbf{c}, \sigma) = f_\theta\left(\mathtt{interp}(\mathtt{hash\_and\_lookup}(\mathbf{x}, \{\mathcal{G}_\theta^{l}\}^L_{l=1})), \mathbf{d}\right).
\end{equation}

\ssec{Sampling}  We support various sampling strategies, including the recently proposed proposal network~\cite{barron_mip-nerf_2022}, which distills a density field from a radiance-free NeRF model to generate ray samples and other sampling schemes like coarse-to-fine sampling \cite{mildenhall_nerf_2020} or uniform sampling \cite{fridovich-keil_k-planes_2023} for flexibility.
% while we do not support multi-plane-based sampling method~\cite{ost_neural_2021} which only sample a few points on ray-plane intersections, which fail at synthesizing novel views and are not suitable for our simulation setting.


% % Figure environment removed

\ssec{Foreground Nodes}
% Our proposed approach disentangles foreground instances from the background to prevent ghost shadows in dynamic scenes. 
For rendering foreground instances, we first transform the projected rays into per-instance coordinate space and then infer the object-centric NeRFs in each instance-wise canonical space.
% Our modular simulator framework supports instance and category-wise models (Fig.~\ref{fig:main-figure} left-bottom) to represent foreground instances, while both of them are made instance-aware by transforming the rays from the world to the instance's canonical space.
The default setting of our framework uses code-conditioned models that exploit latent codes to encode instance features and \textbf{shared} category-level decoders to encode class-wise priors, allowing the modeling of many long tracklets with compact memory usage. Meanwhile, the conventional ones without code conditions are also supported in our framework.
% Specifically, we concatenate the positional encoded~\cite{mildenhall_nerf_2020} camera position and the shape latent codes to model the instance's shape with the shape networks. The output feature is then stacked with the positional encoded viewing directions $\textbf{d}$ and the texture latents, and passed the resulting input to the texture networks to obtain the instance's texture. 
We detailed our modified foreground representation (denoted as `Ours' in Sec.~\ref{sec:experiments}) in supplementary materials.
% Other design options are similar to the background node design choices.
% We use \textbf{category-level} model approaches that use latent codes to encode instance features by default for foreground node representations. To build our object library shown in Fig. \ref{fig:latents}, we train our model over all frames in the target sequence and store all the latent encodings.
% We follow CodeNeRF~\cite{jang_codenerf_2021} and use decoupled shape and texture latent codes, as shown in Fig. \ref{fig:latents}.



% Our decoupled shape and texture category-level model uses category-wise shape and texture MLPs, along with instance-wise shape and texture latents. Compared to category-level methods, this approach is more memory-efficient and enables flexible object manipulation within our simulator.


% In the \textbf{instance-wise} setting, each instance has its own set of model parameters. Our framework differs from previous methods such as PNF~\cite{kundu_panoptic_2022} and NSG~\cite{ost_neural_2021} that use standard NeRF~\cite{mildenhall_nerf_2020} methods to instantiate object instances. Instead, we support both multi-resolution hash tables-based methods~\cite{muller_instant_2022,tancik_nerfstudio_2023} and MLP-based NeRFs, providing more flexibility and options for object instantiation. While the instance-wise model is highly memory-consuming, it accurately represents dynamic scenes and is suitable for scenarios with small scale and short duration with high accuracy requirements.

% Figure environment removed

\subsection{Compositional Rendering}
\label{sec:rendering}
Figure~\ref{fig:compose} demonstrates the compositional rendering results. To render an image at a given camera pose $\mathcal{T}_i$, we cast a ray $\mathbf{r} = \mathbf{o} + t\mathbf{d}$ at each rendered pixel. For each ray $\mathbf{r}$, we first calculate the intersection interval $[t_\text{in}, t_\text{out}]$ with all visible foreground nodes $\mathcal{O}_{ij}$ (Fig.~\ref{fig:sampling}) and transform the samples $\{P_k^\text{obj-j}\}$ along the ray from world space into each foreground canonical space.
% with their bounding box $\{\mathcal{B}_{ij}\}$ before 
We also sample a set of 3D points along the ray ($\{P_k^\text{bg}\}$ as background samples. Samples in all nodes are first passed through their corresponding networks to obtain point-wise colors $\{\textbf{c}_k^{\text{bg, obj}}\}$, densities $\{\sigma_k^{\text{bg, obj}}\}$, and foreground semantic logits $\{\textbf{s}_k^{\text{bg}}\}$. 
% An optional sky model is also deployed to deal with appearances at an infinite distance (Sec.~\ref{sec:sky-and-removebg}). 
Considering that the semantic properties of foreground samples are actually their category label, we create a one-hot vector as:
\begin{equation}
    \textbf{s}_k^{\text{obj-j}}[l] = \left\{ \begin{matrix}
        &\sigma_k^{\text{obj-j}} &\qquad \text{if } l = \text{category of j's instance}\\
        &0 &\text{otherwise}
    \end{matrix} \right., \text{for}\ l\ \text{in category}.
\end{equation}
To aggregate the point-wise properties, we sort all the samples by their ray distance in world space and use the standard volume rendering process to render pixel-wise properties:
\begin{align}
    &\hat{\textbf{c}}(\textbf{r}) = \sum_{P_i}T_i\alpha_i \textbf{c}_i + (1 - \texttt{accum})\cdot \textbf{c}_\text{sky},\ \ T_i = \exp(-\sum_{k=1}^{i-1}\sigma_k\delta_k), \label{eq:render}\\
   & \hat{d}(\textbf{r})\! =\! \sum_{P_i}T_i\alpha_i t_i + (1 - \texttt{accum})\cdot \texttt{inf}, \ \hat{\textbf{s}}(\textbf{r})\! =\! \sum_{P_i}T_i\alpha_i \textbf{s}_i +(1 - \texttt{accum})\cdot \textbf{s}_\text{sky},
\end{align}
where $P_i\in\mathtt{sorted}(\{P_i^\text{bg, obj}\}), \alpha_i=1-\exp(-\sigma_i\delta_i), \delta_i=t_{i+1}-t_i$, $\texttt{accum} = \sum_{P_i} T_i \alpha_i$ , $\textbf{c}_\text{sky}$ is the rendered color from the Sky model (Sec.\ref{sec:sky-and-removebg}), $\texttt{inf}$ is the upper bound distance, and $\textbf{s}_\text{sky}$ is the one-hot semantic logits of the $\texttt{sky}$ category.

% Figure environment removed

\subsection{Towards Realistic Rendering}
\label{sec:sky-and-removebg}
\ssec{Sky Modeling}
% \label{sec:sky-model}
In our framework, we support the usage of a sky model to deal with appearances at infinite distance, where an MLP-based spherical environment map~\cite{rematas_urban_2022} is leveraged to model the infinitely far regions that never intersect opaque surfaces:
\begin{equation}
    f_\text{sky}(\mathbf{d}) = \textbf{c}_\text{sky}: \mathbb{S}^2 \to \mathbb{R}^3
\end{equation}

However, na\"ively blending the sky color $\textbf{c}_\text{sky}$ with background and foreground rendering (Eq.~\ref{eq:render}) leads to potential inconsistency. Therefore, we introduce a BCE semantic regularization to alleviate this issue:
\begin{equation}
    \mathcal{L}_\text{sky} = \texttt{BCE}(1 - \texttt{accum}, \mathcal{S}_\text{sky}).
\label{eq:sky_loss}
\end{equation}

\subsubsection{Resolving Conflict Samples.}
\label{sec:resolving conflict}
Due to the fact that our background and foreground sampling are done independently, there is a chance that background samples fall within the foreground bounding box (Fig.~\ref{fig:sampling} Background Truncated Samples). The compositional rendering may mistakenly classify foreground samples as background (referred to later as background-foreground ambiguity). As a result, after removing the foreground instance, artifacts will emerge in the background area (Fig.~\ref{fig:no_reg}). Ideally, with sufficient multi-view supervision signal, the system can automatically learn to distinguish between foreground and background during the training process. However, for a data-driven simulator, obtaining abundant and high-quality multi-view images is challenging for users as vehicles move fast on the road. The ambiguity is \textbf{NOT} observed in NSG~\cite{ost_neural_2021} as NSG only samples a few points on the ray-plane intersections, and is unlikely to have much background truncated samples.

% Figure environment removed

To address this issue, we devise a regularization term that minimizes the density sum of background truncated samples to minimize their influence during the rendering process as:
\begin{equation}
    \mathcal{L}_\texttt{accum} = \sum_{P_i^\text{(tr)}} \sigma_i,
\label{eq:acc_loss}
\end{equation}
where $\{P_i^\text{(tr)}\}$ denotes background truncated samples.

% This regularization term ensures that the weights of background samples are correctly distributed on the sampled rays and that the foreground and background parts are separated well.




\subsection{Optimization}
\label{sec:optimization}

To optimize our system, we minimize the following objective function:
\begin{equation}
    \mathcal{L} = \lambda_1\mathcal{L}_\text{color} + \lambda_2\mathcal{L}_\text{depth} + \lambda_3\mathcal{L}_\text{sem} + \lambda_4\mathcal{L}_\text{sky} + \lambda_5\mathcal{L}_\texttt{accum},
\end{equation}
where $\lambda_{1-5}$ are weighting parameters. $\mathcal{L}_\text{sky}$ and $\mathcal{L}_\texttt{accum}$ are explained in Eq.~\ref{eq:sky_loss} and~\ref{eq:acc_loss}.

% color loss
\noindent\textbf{Color Loss}: we adopt a standard MSE loss that minimizes the photo-metric errors as:
\begin{equation}
    \mathcal{L}_\text{color} = ||\textbf{c}(\textbf{r}) - \hat{\textbf{c}}(\textbf{r})||_2^2.
\end{equation}


% depth supervision @chao
\noindent\textbf{Depth Loss}:
We introduce a depth loss to address textureless regions and regions that are observed from sparse viewpoints. We have devised two strategies for supervising the geometry. Given depth data, we utilize a ray distribution loss derived from~\cite{deng_depth-supervised_2022}. On the other hand, if the depth data is not available, we utilize a mono-depth network and apply mono-depth loss following~\cite{yu_monosdf_2022}.
\begin{equation}
\mathcal{L}_{\text{depth}}=\left\{ \begin{matrix}
	&\mathcal{L}_{\text{sensor\_depth}} \qquad &\text{if depth data is available}\\
	&\mathcal{L}_{\text{mono\_depth}} &\text{if depth data is not available}\\
\end{matrix} \right. 
\end{equation}

% semantic supervision @zirui
\noindent\textbf{Semantic Losses}: we follow SemanticNeRF~\cite{zhi_-place_2021} and use a cross-entropy semantic loss $\mathcal{L}_\text{sem} = \texttt{CrossEntropy}(\textbf{s}(\textbf{r}), \mathcal{S}(\textbf{r}))$.

% % method-specific losses
% \noindent\textbf{Method-specific Losses}: we keep other method-specific loss terms that are necessary for the model to converge, e.g. the interlevel loss~\cite{barron_mip-nerf_2022} for the proposal networks and the total-variance loss for the grids~\cite{fridovich-keil_k-planes_2023}, denoted as $\mathcal{L}_\text{extra}$.



\section{Experiments}
\label{sec:experiments}
In this section, we provide extensive experimental results to demonstrate the proposed instance-aware, modular, and realistic simulator for autonomous driving.
We evaluate our method on scenes from the KITTI~\cite{geiger_vision_2013} dataset and the Virtual KITTI-2 (V-KITTI)~\cite{cabon_virtual_2020} dataset. In the following, we use \textbf{``our default setting''} to denote a grid-based NeRF with proposal sampler for the background node, and our modified category-level representation with coarse-to-fine sampler for foreground nodes.

\begin{table}[htbp]
  \vspace{-3mm}
  \centering
  \caption{Qunatitative results on image reconstruction task \& Comparisons on the settings with baseline methods. The dataset used for evaluation is KITTI.}
  \resizebox{0.8\textwidth}{!}{
    \begin{tabular}{cccccccc}
    \hline
          &  NeRF \cite{mildenhall_nerf_2020}  & NeRF+Time & NSG \cite{ost_neural_2021}  & PNF  \cite{kundu_panoptic_2022} & SUDS \cite{turki_suds_2023}  & Ours \\
    \hline
    PSNR $\uparrow$ &  23.34 & 24.18 & 26.66 & 27.48 & 28.31 & \textbf{29.06} \\
    SSIM $\uparrow$ &  0.662 & 0.677 & 0.806 & 0.870  & 0.876 & \textbf{0.885}  \\
    \hline
    Instance-aware & $\times$ & $\times$ & \checkmark & \checkmark & $\times$  & \checkmark\\
    Modular & $\times$ & $\times$ & $\times$ & $\times$ & $\times$  & \checkmark\\
    Open-sourced & \checkmark & - & \checkmark & $\times$ & \checkmark  & \checkmark\\
    \hline
    \end{tabular}%
    }
  \label{tab:recon-quant}%
  \vspace{-10mm}
\end{table}%

% Please add the following required packages to your document preamble:
% \usepackage[table,xcdraw]{xcolor}
% If you use beamer only pass "xcolor=table" option, i.e. \documentclass[xcolor=table]{beamer}
% \begin{table}[htbp]
%   \vspace{-3mm}
%   \centering
%   \caption{Qunatitative results on image reconstruction task}
%   \resizebox{0.75\textwidth}{!}{
% \begin{tabular}{cccccccc}
% \hline
%                      &  NeRF \cite{mildenhall_nerf_2020}  & NeRF+Time & NSG \cite{ost_neural_2021}  & PNF  \cite{kundu_panoptic_2022} & SUDS \cite{turki_suds_2023}  & Ours                 &                                                      \\ \hline
% PSNR                 & 23.34                & 24.18                & 26.66                & 27.48                & 28.31                & 29.06                & {\color[HTML]{FF0000} \ +0.76\ }  \
% SSIM                 & 0.662                & 0.677                & 0.806                & 0.87                 & 0.876                & 0.885                & {\color[HTML]{FF0000} \ +0.01\ } \\ \hline
% \multicolumn{1}{l}{} & \multicolumn{1}{l}{} & \multicolumn{1}{l}{} & \multicolumn{1}{l}{} & \multicolumn{1}{l}{} & \multicolumn{1}{l}{} & \multicolumn{1}{l}{} & \multicolumn{1}{l}{}                      
%     \end{tabular}%
%     }
%   \label{tab:recon-quant}%
%   \vspace{-10mm}
% \end{table}%


\begin{table}[ht]
\centering
\caption{Qunatitative results on novel view synthesis}
\resizebox{0.75\textwidth}{!}} &
  \multicolumn{3}{c}{\textbf{KITTI-50\%}} &
  \multicolumn{3}{c}{\textbf{KITTI-25\%}} \\ \cline{2-10} 
\multicolumn{1}{c}{} &
  PSNR $\uparrow$&
  SSIM $\uparrow$&
  LPIPS $\downarrow$&
  PSNR $\uparrow$&
  SSIM $\uparrow$&
  LPIPS $\downarrow$&
  PSNR $\uparrow$&
  SSIM $\uparrow$&
  LPIPS $\downarrow$\\ \hline
NeRF~\cite{mildenhall_nerf_2020} &
  18.56 &
  0.557 &
  0.554 &
  19.12 &
  0.587 &
  0.497 &
  18.61 &
  0.570 &
  0.510 \\
NeRF+Time &
  21.01 &
  0.612 &
  0.492 &
  21.34 &
  0.635 &
  0.448 &
  19.55 &
  0.586 &
  0.505 \\
NSG~\cite{ost_neural_2021} &
  21.53 &
  0.673 &
  0.254 &
  21.26 &
  0.659 &
  0.266 &
  20.00 &
  0.632 &
  0.281 \\
SUDS~\cite{turki_suds_2023} &
  22.77 &
  0.797 &
  0.171 &
  23.12 &
  \textbf{0.821} &
  \textbf{0.135} &
  20.76 &
  0.747 &
  0.198 \\
Ours &
  \textbf{24.23} &
  \textbf{0.845} &
  \textbf{0.160} &
  \textbf{24.00} &
  0.801 &
  0.164 &
  \textbf{23.23} &
  \textbf{0.756} &
  \textbf{0.177} \\
\rowcolor[HTML]{EFEFEF} 
\multicolumn{1}{c}{\cellcolor[HTML]{EFEFEF}} &
  {\color[HTML]{FF0000} +1.46} &
  {\color[HTML]{FF0000} +0.048} &
  {\color[HTML]{FF0000} -0.011} &
  {\color[HTML]{FF0000} +0.88} &
  {\color[HTML]{32CB00} -0.020} &
  {\color[HTML]{32CB00} +0.029} &
  {\color[HTML]{FF0000} +2.47} &
  {\color[HTML]{FF0000} +0.009} &
  {\color[HTML]{FF0000} -0.021} \\ \hline
\multicolumn{1}{c}{} &
  \multicolumn{3}{c}{\textbf{VKITTI-75\%}} &
  \multicolumn{3}{c}{\textbf{VKITTI-50\%}} &
  \multicolumn{3}{c}{\textbf{VKITTI-25\%}} \\ \cline{2-10} 
\multicolumn{1}{c}{} &
  PSNR $\uparrow$&
  SSIM $\uparrow$&
  LPIPS $\downarrow$&
  PSNR $\uparrow$&
  SSIM $\uparrow$&
  LPIPS $\downarrow$&
  PSNR $\uparrow$&
  SSIM $\uparrow$&
  LPIPS $\downarrow$\\ \hline
NeRF~\cite{mildenhall_nerf_2020} &
  18.67 &
  0.548 &
  0.634 &
  18.58 &
  0.544 &
  0.635 &
  18.17 &
  0.537 &
  0.644 \\
NeRF+Time &
  19.03 &
  0.574 &
  0.587 &
  18.90 &
  0.565 &
  0.610 &
  18.04 &
  0.545 &
  0.626 \\
NSG~\cite{ost_neural_2021} &
  23.41 &
  0.689 &
  0.317 &
  23.23 &
  0.679 &
  0.325 &
  21.29 &
  0.666 &
  0.317 \\
SUDS~\cite{turki_suds_2023} &
  23.87 &
  0.846 &
  0.150 &
  23.78 &
  0.851 &
  0.142 &
  22.18 &
  0.829 &
  0.160 \\
Ours &
  \textbf{29.79} &
  \textbf{0.917} &
  \textbf{0.088} &
  \textbf{29.63} &
  \textbf{0.916} &
  \textbf{0.087} &
  \textbf{27.01} &
  \textbf{0.887} &
  \textbf{0.104} \\
\rowcolor[HTML]{EFEFEF} 
\multicolumn{1}{c}{\cellcolor[HTML]{EFEFEF}} &
  {\color[HTML]{FF0000} +5.92} &
  {\color[HTML]{FF0000} +0.071} &
  {\color[HTML]{FF0000} -0.062} &
  {\color[HTML]{FF0000} +5.85} &
  {\color[HTML]{FF0000} +0.065} &
  {\color[HTML]{FF0000} -0.055} &
  {\color[HTML]{FF0000} +4.83} &
  {\color[HTML]{FF0000} +0.058} &
  {\color[HTML]{FF0000} -0.056} \\ \hline
\end{tabular}%
}
\vspace{-4mm}
\label{tab:nvs-quant}%
\end{table}


\subsection{Photorealistic Rendering} 
We validate the photorealistic rendering performance of our simulator by evaluating image reconstruction and novel view synthesis (NVS) following ~\cite{ost_neural_2021,turki_suds_2023}. 

\ssec{Baselines} We conduct qualitative and quantitative comparisons against other state-of-the-art methods: NeRF~\cite{mildenhall_nerf_2020}, NeRF with timestamp input (denoted as NeRF+Time), NSG~\cite{ost_neural_2021}, PNF~\cite{kundu_panoptic_2022}, and SUDS~\cite{turki_suds_2023}. Note that none of them simultaneously meet all three standards mentioned in Table.~\ref{tab:recon-quant}.

\ssec{Implementation Details} Our model is trained for 200,000 iterations with 4096 rays per batch, using RAdam as optimizers. The learning rate of the background node is assigned $1*10^{-3}$ decaying to $1*10^{-5}$, while that of $5*10^{-3}$ decaying to $1*10^{-5}$ in object nodes.

% Figure environment removed

% Figure environment removed

\ssec{Experiment Settings} The training and testing image sets in the image reconstruction setting are identical, while in the NVS task, we render the frames that are not included in the training data. Specifically, we hold out every 4$^\text{th}$ frames, every 2$^\text{nd}$ and 4$^\text{th}$ frames, and training with only one in every four frames, namely 25\%, 50\%, and 75\%. 

We follow the standard evaluation protocol in image synthesis and report Peak Signal-to-Noise Ratio (PSNR), Structural Similarity (SSIM), and Learned Perceptual Image Patch Similarity (LPIPS) ~\cite{zhang_unreasonable_2018} of our default setting for quantitative evaluations. Results are shown in Table~\ref{tab:recon-quant} for image reconstruction and Table~\ref{tab:nvs-quant} for NVS, which indicate that our method outperforms baseline methods in both settings. We can achieve 29.79 PSNR on V-KITTI using 75\% training data, while the best result previously published is 23.87.

\subsection{Instance-wise Editing} Our framework separately models background and foreground nodes, which allows us to edit the scene in an instance-aware manner. We qualitatively present our capability to remove instances, add new instances, and edit vehicle trajectories. In Fig.~\ref{fig:editing}, we show some editing examples of rotating and translating a vehicle, though more results can be found in our video clip.

% Figure environment removed

\subsection{The blessing of moduler design}

We use different combinations of background and foreground nodes, samplers, and supervision signals for evaluation, which is credited to our modular design. 

Note that some of the baseline methods in the literature actually correspond to an ablation entry in this table. For instance, PNF ~\cite{kundu_panoptic_2022} uses NeRF as background node representation and instance-wise NeRF as foreground node representation with semantic losses. NSG~\cite{ost_neural_2021} uses NeRF as background node representation and category-level NeRF as foreground representation, but with a multi-plane sampling strategy. Our default setting uses grid-based background node representation, and our proposed category-level method for foreground node representation.


% We follow prior works~\cite{turki_suds_2023,kundu_panoptic_2022,ost_neural_2021} and use the same images setup to train and evaluate for \textbf{Image Reconstruction} task. Experiments results are shown in Table~\ref{tab:recon-quant}. For \textbf{novel view synthesis} task, we showcase the ability of our framework to generate realistic rendering for frames that are not included in the training data. We evaluate the methods by employing different train/test splits, holding out every 4$^\text{th}$ frames, every 2$^\text{nd}$ and 4$^\text{th}$ frames, and training with only one in every four frames, namely 25\%, 50\%, and 75\%. We show quantitative results in Table~\ref{tab:nvs-quant}.


% \begin{table}[htbp]
%   \centering
%   \caption{Qunatitative results on image reconstruction task}
%     \begin{tabular}{cccccccc}
%     \hline
%           & SRN   & NeRF  & NeRF+Time & NSG   & PNF   & SUDS  & Ours \\
%     \hline
%     PSNR  & 18.83 & 23.34 & 24.18 & 26.66 & 27.48 & 28.31 & \textbf{29.06} \\
%     SSIM  & 0.590  & 0.662 & 0.677 & 0.806 & 0.870  & 0.876 & \textbf{0.885} \\
%     \hline
%     \end{tabular}%
%   \label{tab:recon-quant}%
% \end{table}%



% \ssec{Scene editing}


\subsection{Ablation Results}
\label{sec:ablation}
In this section, we analyze different experiment settings, verifying the necessity of our design. We reveal the impact of different design choices in background node representation, foreground node representation, etc. Specifically, we present all experiments with 50,000 iterations. Unlike prior works~\cite{turki_suds_2023,kundu_panoptic_2022,ost_neural_2021} that evaluate their method on a short sequence of 90 images, we use the full sequence from the dataset for all evaluation. Since they are not open-sourced and their exact evaluation sequences are not known, we hope our new benchmarking would standardize this important field. Quantitative evaluation can be found in Table~\ref{tab:ablation}.

For background and foreground nodes, we substitute our default model (ID 1 in Table~\ref{tab:ablation}) with MLP-based and grid-based model and list their metrics in row 2, 7-12. In the 3$^\text{rd}$-6$^\text{th}$ row, we show the effectiveness of our model components. For model and sampler, selected modules for background and foreground nodes are noted before and after the slash, respectively.


\begin{table}
\vspace{-5mm}
\centering
\caption{Quantitative evaluation for ablation studies}
\resizebox{0.85\textwidth}{!}{%
\begin{threeparttable}

\begin{tabular}{cccccccc|cccccc}
\hline
 &
  \multicolumn{7}{c|}{Settings} &
  \multicolumn{3}{c}{KITTI} &
  \multicolumn{3}{c}{V-KITTI} \\ \cline{2-14} 
\multirow{-2}{*}{ID} &
  Model &
  Sampler &
  Category &
  $\mathcal{L}_\text{sky}$ &
  \multicolumn{1}{c}{$\mathcal{L}_\text{depth}$} &
  $\mathcal{L}_\text{sem}$ &
  $\mathcal{L}_\texttt{accum}$ &
  PSNR $\uparrow$&
  SSIM $\uparrow$&
  LPIPS $\downarrow$&
  PSNR $\uparrow$&
  SSIM $\uparrow$&
  LPIPS $\downarrow$\\ \hline
\rowcolor[HTML]{EFEFEF} 
1\tnote{*} &
\multicolumn{1}{l}{\cellcolor[HTML]{EFEFEF}Grid / Ours} &
  \cellcolor[HTML]{EFEFEF}prop / c2f \tnote{$\dagger$} &
  \cellcolor[HTML]{EFEFEF} &
   &
   &
   &
   &
  \textbf{25.04} &
  \textbf{0.782} &
  \textbf{0.175} &
  \textbf{28.37} &
  \textbf{0.907} &
  \textbf{0.108} \\
2 &
  \textbf{MLP} / Ours &
  \textbf{c2f} / c2f &
   &
   &
  \multicolumn{1}{c}{} &
   &
   &
  20.14 &
  0.589 &
  0.476 &
  22.19 &
  0.664 &
  0.409 \\
\rowcolor[HTML]{EFEFEF} 
3 &
  Grid /   Ours &
  \cellcolor[HTML]{EFEFEF}prop / c2f &
  \cellcolor[HTML]{EFEFEF} &
   &
   &
   &
  × &
  21.35 &
  0.713 &
  0.242 &
  27.30 &
  0.881 &
  0.130 \\
4 &
  Grid /   Ours &
  prop / c2f &
   &
  × &
   &
  × &
   &
  23.68 &
  0.774 &
  0.181 &
  27.32 &
  0.881 &
  0.129 \\
\rowcolor[HTML]{EFEFEF} 
5 &
  Grid /   Ours &
  \cellcolor[HTML]{EFEFEF}prop / c2f &
  \cellcolor[HTML]{EFEFEF} &
   &
  \multicolumn{1}{c}{\cellcolor[HTML]{EFEFEF}×} &
   &
  \multicolumn{1}{l|}{\cellcolor[HTML]{EFEFEF}} &
  23.66 &
  0.769 &
  0.184 &
  27.30 &
  0.880 &
  0.128 \\
6 &
  Grid /   Ours &
  prop / c2f &
   &
  × &
  \multicolumn{1}{c}{} &
   &
  \multicolumn{1}{l|}{} &
  20.07 &
  0.723 &
  0.251 &
  27.42 &
  0.863 &
  0.148 \\
\rowcolor[HTML]{EFEFEF} 
7 &
  Grid / \textbf{MLP} &
  \cellcolor[HTML]{EFEFEF}prop / \textbf{c2f} &
  \cellcolor[HTML]{EFEFEF} &
   &
   &
   &
   &
  20.46 &
  0.709 &
  0.255 &
  26.46 &
  0.875 &
  0.132 \\
8 &
  Grid / \textbf{Grid} &
  prop / \textbf{prop} &
   &
   &
   &
   &
   &
  22.23 &
  0.741 &
  0.211 &
  25.22 &
  0.871 &
  0.134 \\
\rowcolor[HTML]{EFEFEF} 
9 &
  Grid / \textbf{MLP} &
  \cellcolor[HTML]{EFEFEF}prop / \textbf{c2f} &
  \cellcolor[HTML]{EFEFEF}× &
   &
   &
   &
   &
  20.98 &
  0.699 &
  0.257 &
  27.27 &
  0.881 &
  0.130 \\
10 &
  Grid / \textbf{Grid} &
  prop / \textbf{prop} &
  × &
   &
   &
   &
   &
  23.71 &
  0.763 &
  0.193 &
  26.65 &
  0.882 &
  0.125 \\ 
\rowcolor[HTML]{EFEFEF} 
11\tnote{*} &
  \textbf{MLP} / \textbf{MLP} &
  \cellcolor[HTML]{EFEFEF}\textbf{c2f} / \textbf{c2f} &
  &
  &
  &
  &
  &
  20.42 &
  0.592 &
  0.472 &
  21.77 &
  0.659 &
  0.410 \\
  \hline
\end{tabular}%
\begin{tablenotes}
    \footnotesize
    \item[$\dagger$] prop stands for proposal sampler, and c2f stands for coarse-to-fine sampler.
    \item[*] ID 1 is our default setting. ID 11 is similar to the setting of NSG \cite{ost_neural_2021} with coarse-to-fine sampler instead.
\end{tablenotes}    
\end{threeparttable}
}
\label{tab:ablation}
\vspace{-8mm}
\end{table}


% \ssec{Ablation on depth and semantic supervision}
% % Semantic segmentation and depth images are demonstrated in our default settings. We perform a ablation study of rendered segmentation and depth images on dynamic scenes from KITTI and Vitural KITTI2. Since depth and semantic loss help to restrict loss, we achieve a better PSNR and SSIM with these components.
% We show the effectiveness of provided segmentation masks and depth maps in the datasets in Table~\ref{tab:ablation}


% \subsubsection{Other ablation studies}
% We list other ablation studies in Table~\ref{tab:ablation}, e.g. the effectiveness of our proposed regularization losses. 

% \section{Limitations}


\section{Conclusion}
In this paper, we present a modular framework for photorealistic autonomous driving simulation based on NeRFs. Our \textbf{open-sourced} framework consists of a background node and multiple foreground nodes, enabling the modeling of complex dynamic scenes.
We demonstrate the effectiveness of our framework through extensive experiments. The proposed pipeline achieved state-of-the-art rendering performance on public benchmarks. We also support different design choices of scene representations and sampling strategies, offering flexibility and versatility in the simulation process.

\vspace{1mm}\textbf{Limitations.} Our method requires hours to train and is not capable of rendering in real-time. Besides, our method fails to consider the dynamic specular effects on glasses or other reflective materials that may cause artifacts in rendered images. Improving simulation efficiency and view-dependent effects will be our future work.

\clearpage
%
% ---- Bibliography ----
%
% BibTeX users should specify bibliography style 'splncs04'.
% References will then be sorted and formatted in the correct style.
%
\bibliographystyle{splncs04}
\bibliography{wuzirui_zotero}
\appendix
% \vspace{-2mm}
\section{Related Works}
% \vspace{-2mm}
\subsection{Neural Simulators}
Neural rendering has shown immense potential in representing autonomous driving scenarios. Some methods have been proposed to leverage this technology in simulators. 
Neural scene graph (NSG) \cite{ost_neural_2021} decomposes dynamic scenes by encoding object transformations into a learned scene graph, and learns latent representation to describe similar objects with the same implicit field. 
Panoptic Neural Fields (PNF) \cite{fu_panoptic_2022} went one step further to model instance-aware objects and jointly learn panoptic segmentation of the scene. 
Urban radiance field (URF) \cite{rematas_urban_2022} leverages predicted image segmentations to supervise densities on rays pointing at the sky. 
S-NeRF \cite{xie_s-nerf_2023} employs a camera transformation for object alignment and integrates reprojection and geometry confidence to enhance geometric consistency.
SUDS \cite{turki_suds_2023} factorizes the scene into three separate hash tables to efficiently encode static, dynamic, and far-field parts of the scene.
However, these methods failed to integrate the foreground and background well and leverage prior information about vehicles to create realistic scene representations in autonomous driving scenarios. Furthermore, the highly interconnected frameworks of these methods have limited their practical implementation in engineering projects. 

\subsection{Neural Scene Representations}
Neural Scene Representation has shown impressive results in image reconstruction and novel view synthesis. Neural Radiance Field(NeRF)\cite{mildenhall_nerf_2020} is the first to use implicit fields to represent scenes, various types of NeRFs have been proposed for acceleration. \cite{muller_instant_2022,tancik_nerfstudio_2023,chen_tensorf_2022} and better rendering quality \cite{barron_mip-nerf_2021,barron_mip-nerf_2022,wang_f2-nerf_2023}. NeRFStudio\cite{tancik_nerfstudio_2023} provides a unified framework for many of these approaches. However, most of them fail to reconstruct dynamic scenes with high quality, and there is no such framework available that offers a flexible module switch to seamlessly combine the distinct strengths of these approaches.


\section{More on the depth supervision}
We follow DS-NeRF \cite{deng_depth-supervised_2022} and utilize a ray distribution loss if the dense depth maps are available:
\begin{equation}
    \mathcal{L}_
    \text{sensor\_depth} = \sum_k\log h_k\exp\left(-\frac{(t_k-\mathcal{D}(\textbf{r}))}{2\hat{\sigma}_i^2}\right)\delta_k,
\end{equation}
where $\hat{\sigma}_i$ is the average reprojection error.

If the ground truth dense depth maps are not available, we use the monocular depth estimation network and employ a mono-depth loss \cite{yu_monosdf_2022} to regularize the geometry:
\begin{equation}
    \mathcal{L}_\text{mono\_depth} = \sum_\textbf{r}||w\hat{\textbf{d}}(\textbf{r}) + q - \mathcal{D}(\textbf{r})||_2^2,
\end{equation}
where $w$ and $q$ are the learnable scale and shift factors that are optimized through training for scale and shift-invariance.


\section{Details for foreground nodes}


% Figure environment removed

The overview of our improved foreground nodes is shown in Fig.\ref{fig:latents}. Our approach learns the shape and texture of each object instance separately by decoupling them into two parts of latent codes. We optimize these latent codes over the test sequence and store them in object libraries. The object library is depicted in Figure \ref{fig:latents}.

During rendering, we incorporate the priors that are queried from the object library into the NeRF model. The shape net is an 8-layer MLP, and the texture net is a 4-layer MLP following the vanilla NeRF\cite{mildenhall_nerf_2020}. We combine the latents and the model by concatenating the positional encoded \cite{mildenhall_nerf_2020} camera position and the shape latent codes to model the instance's shape with the shape networks. The output feature is then stacked with the positional encoded viewing directions $\textbf{d}$ and the texture latents, and passed the resulting input to the texture networks to obtain the instance's texture.  These improvements contribute to \textbf{state-of-the-art} performance.


% Figure environment removed


% Figure environment removed

\section{Synthetic traffic environment generation}

We present additional experiment results in Fig.~\ref{fig:decomposition} of decomposed rendering results on KITTI\cite{geiger_are_2012}, VKITTI\cite{cabon_virtual_2020}, and DAIR-V2X datasets \cite{yu_dair-v2x_2022}. Our method offers the benefits of background and foreground decomposition, which enables flexible editing of foreground nodes to generate novel synthetic traffic environments. Figure \ref{fig:edit} showcases the editing process, where the car can be deleted and added to a new position or orientation. These operations are equivalent to translating or rotating the target objects. These capabilities allow for the creation of diverse and customizable photorealistic traffic scenarios, which can be valuable for testing and evaluating autonomous driving systems and other traffic-related applications.

\end{document}
