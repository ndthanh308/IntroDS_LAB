% !TEX root =  ../main.tex

We have shown how to decide variants of \CRA minimization problems,
and have given complexity for the respective algorithms.
There are several ways in which these algorithms could be improved.
First, it would be worth reducing the gap between the lower and
the upper bounds on the length of the \LH.
Second, identifying a canonical invariant associated with the state-register
minimization problem would allow to derive a deterministic
algorithm for this problem.
Third, one could hope for better complexity if one only considers
the existence of equivalent CRA.
For instance, in~\cite{JeckerMP23} the authors give a \pspace algorithm
for the determinization problem (\ie $1$-register minimization problem)
in the case of a polynomially ambiguous automaton, via a quite different approach.


Another line of research consists in trying to use the techniques we developed
to solve the register minimization problem for other classes of \CRA,
for instance copyless \CRA
(which correspond to multi-sequential \WA).
Another ambitious goal is to consider register minimization in the context
of different semirings, but there all the linear algebra tools which are crucial
to solving these problems completely break down.
Similarly, it seems that register minimization for polynomial automata would be very difficult:
it was shown recently in~\cite{HrushovskiOPW23} that the strongest algebraic
invariant of a polynomial automaton is not computable.
One possibility may be to bound the ``degree'' of the invariants, where Z-affine sets
would correspond to algebraic sets of degree one.