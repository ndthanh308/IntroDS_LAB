% !TEX root = ../main.tex

\textbf{Weighted automata (\WA)}
 are a quantitative extension of finite state automata and have been studied since the sixties~\cite{Schutzenberger61b}. These automata define functions from words to a given semiring: each transition
has a weight in the semiring and the weight of an execution is the product of the weights of the
transitions therein; the non-determinism of the model is handled using
the sum of the semiring: the weight associated with a word is the sum of
the weights of the different executions over this word. Functions realized by weighted
automata are called rational series.
This fundamental model has been widely studied
during the last decades~\cite{HBWA}.
While some expressiveness results can be
obtained in a general framework (such as the equivalence with
rational expressions), the decidability status of important problems
heavily depends on the considered semiring. Amongst the classical problems of interest,
one can mention \emph{equivalence}, \emph{sequentiality} (resp. \emph{unambiguity}), which aims at determining whether there exists
an equivalent deterministic (resp. unambiguous) \WA, and \emph{minimization}, which aims at minimizing the number of states.

Weighted automata over a field (\eg the field of rationals $\mathbb{Q}$) enjoy many nice properties: the equivalence of weighted
automata is decidable and they can be minimized, and both can be done efficiently
(see \eg \cite[Theorem 4.10 and Corollary 4.17 (Chapter III)]{Sakarovitch09}). The sequentiality and
unambiguity
are also
decidable, as shown  recently in~\cite{BellS21,BS23}, with no complexity bounds however.
The most studied semirings which are not fields are the tropical semirings
and the semiring of languages, and in both cases equivalence is undecidable
(see~\cite[Section 3]{Daviaud20} and~\cite[Theorem 8.4]{Berstel79})
and no minimization algorithm is known. Regarding sequentiality, partial
decidability results have been obtained for these semirings
using the notion of twinning property~\cite{Choffrut77,Mohri97}.


\textbf{Cost register automata (\CRA)} have been introduced more recently by Alur \ea \cite{AlurDDRY13}.
A cost register automaton is a deterministic finite state automaton endowed with a finite number of
registers storing values from the semiring.
The registers are initialized by some values, then at each
transition the values are updated using the operations and constants of the semiring.
Several fragments of \CRA can be considered by restricting the operations allowed.
For instance, an easy observation
is that \WA are exactly \CRA with one state (however, one can observe that adding states does not extend expressiveness) and linear updates, \emph{i.e.}
updates of the form $X \coloneqq \sum_{i=1}^k X_i * c_i$
(intuitively, the new values of the registers only depend
linearly on the previous ones).
Thus, the model of linear \CRA is an alternative to \WA which
allows to trade non-determinism for registers.


\textbf{The register minimization problem.} As \CRA are finite state automata
extended with registers storing elements from the semiring, it is natural to aim
at minimizing the number of registers used.
For a given class $\mathcal{C}$ of \CRA, this problem asks,
given a \WA and a number $\rCRA$, whether there exists an
equivalent \CRA in $\mathcal{C}$ with at most $\rCRA$ registers.
From a practical point of view,
reducing the number of registers allows to reduce the memory usage, since a register can require unbounded memory.
From a theoretical point of view, this problem can be understood as a refinement of the classical
problem of minimization of WA.
Indeed, a \WA can be translated into a linear CRA with a single state, and as many registers as the number
of states of the \WA.
This problem has been studied in~\cite{DBLP:conf/icalp/AlurR13,DRT16,DaviaudJRV17} for
three different models of \CRA but in all these works,
the additive law of the semiring is not allowed (\emph{i.e.} updates of the form
$X \coloneqq Y+Z$ are forbidden).
It is worth noticing that~\cite{DRT16} encompasses the
case of \CRA over a field, with only updates of the form
$X \coloneqq Y*c$, with $c$ an element of the field.

While the minimal number of registers needed to realise a \WA (also known as the \emph{register complexity}) is upper bounded by the number of states of a minimal \WA,
it may be possible to build an equivalent \CRA with fewer registers, but more states.
Hence there is a \emph{tradeoff} between the number of states and the number of registers.
This leads to the following \emph{state-register minimization problem for \CRA} which asks,
for a class $\mathcal{C}$ of \CRA,
given a \WA and integers $\sCRA,\rCRA$
whether an equivalent \CRA in $\mathcal{C}$ with $\sCRA$ states and $\rCRA$ registers can be constructed.
In this framework, the classical minimization of \WA corresponds to minimizing the number of registers
while using only one state, for the class of linear \CRA.



\textbf{The linear hull.} As mentioned before, the case of fields
is well-behaved to obtain decidability results.
In their recent work~\cite{BellS21},
Bell and Smertnig introduced the notion of \emph{linear hull} of a \WA.
This notion is inspired by the algebraic theory needed to study polynomial automata but cast
into a linear setting.
A linear algebraic set (aka linear Zariski closed set) is a finite union of vector
subspaces: we later call them \emph{Z-linear sets}.
Given a Z-linear set $S = \bigcup_{i=1}^p V_i$, the dimension of $S$ is the maximum of the dimensions of the $V_i$'s. In this work, the size of the union, $p$, is called the
length of $S$.
Observe that such Z-linear sets were also used in~\cite{ColcombetP17}
for a category-theoretic approach to minimization of weighted automata over a field.
We say such a set is an invariant if it contains the initial vector and is
stable under the updates of the automaton.
Then the linear hull of a weighted automaton is the strongest Z-linear invariant.
In~\cite{BellS21}, Bell \& Smertnig show that computing the
linear hull of a minimal automaton allows to decide sequentiality and unambiguity.
In addition, in~\cite{BS23}, they show that
the linear hull can effectively be computed, without providing complexity bounds however.


\textbf{Contributions.} In this work, we deepen the analysis of the linear hull of a \WA
in order to solve the register and state-register minimization problems
for linear \CRA.
In addition, we also provide new algorithms
to compute the linear hull which come with complexity upper bounds, which can
be used to derive complexity results for minimization problems as well as for sequentiality
and unambiguity of \WA.
More precisely, our contributions are as follows:
\begin{itemize}
\item Firstly, we show that \emph{the register minimization problem} for the class
of linear \CRA over a field can be solved in \texptime.
To this end, given a rational series $f$, we show that the minimal number of registers
needed to realize $f$ using a linear \CRA is exactly the dimension of the linear hull
of a minimal \WA of $f$.
We then show that the linear hull of a \WA can be computed in
\texptime.
We show that this complexity drops down to \exptime for the particular case of commuting transition matrices
(which includes the case of a single letter alphabet), with a matching lower bound.

\item As a consequence of the computation of the linear hull of a \WA
and of results proved in~\cite{BellS21}, we obtain a \texptime upper bound for the problems of
\emph{sequentiality and unambiguity of weighted automata} over a field, closing a question
raised in~\cite{BS23}.

\item Secondly, we prove that the \emph{state-register minimization problem}
for linear \CRA can be solved in \nexptime.
More precisely, given a minimal \WA $A$,
we show a correspondence between Z-linear invariants of $A$
and linear \CRA equivalent to $A$.
This correspondence maps the length (resp. dimension) of the
invariant to the number of states (resp. registers) of the equivalent linear \CRA.
We then provide a (constructive) \nexptime algorithm that,
given a minimal \WA and two integers $\sCRA,\rCRA$,
guesses a well-behaved invariant allowing to exhibit a satisfying equivalent \CRA.


\item Last, we actually present these results in a more general setting, by considering \emph{affine}
\CRA, which are a
slight extension of linear \CRA allowing to use affine maps in the updates of the registers.
\end{itemize}


\textbf{Outline of the paper.}
\todo{PA: new}
We present the models of weighted automata and cost register automata in Section~\ref{sec:prelim}.
We then formally define the two problems we consider, \emph{i.e.} register and state-register minimization
problems, and state our main results in
Section~\ref{sec:pbres}. In Section~\ref{sec:characterization}, we introduce the
necessary topological notions to define
Z-linear/Z-affine set and invariants of weighted automata,
and detail our characterizations of the register and state-register
complexities of a rational series. Finally in Section~\ref{sec:algos},
we present our algorithms, as well as their consequences
in terms of decidability and complexity for the two problems we
consider.
Omitted proofs and more details for Sections~\ref{sec:characterization} and~\ref{sec:algos}
can be found in the appendix.
