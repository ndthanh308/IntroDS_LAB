% !TEX root =  ../main.tex

We present two original algorithms to solve the minimization problems
we consider.
It is worth observing the difference between the two characterizations we have obtained: while the register complexity can be computed from a canonical object
(the strongest Z-linear invariant of the \WA), the state-register complexity is based on the existence
of a particular Z-linear invariant.
This explains why we derive a non-deterministic
procedure for the latter, and a deterministic for the former.

\subsection{Algorithm for the state-register minimization problem}
\label{subsec:sr}

We provide here a \nexptime algorithm for the state-register minimization problem,
hence proving Theorem~\ref{thm:state-reg-min}.
The algorithm runs in \nptime in $\sCRA$, $\rCRA$, and the size of the automaton.
The fact that $\sCRA$ is given in binary explains the exponential discrepancy.


\textbf{Small representations of Z-affine sets}
Let $\mathcal R=(u,\mu,v)$ be a \WA of dimension $\dWA$ over an alphabet $\Sigma$.
Let $L=A_1\cup\cdots\cup A_\lInv$ be a Z-affine set of length $\lInv$ of $\mathbb K^\dWA$.

An $\mathcal R$-representation $R$ of $L$ is a set of $\lInv$ finite sets of words $S_1,\ldots, S_\lInv$
such that
$\affSpan{\set{u\mu(w)|\ w\in S_i}}=A_i$ for all $i\in \set{1,\cdots,\lInv}$.
The \emph{size} of $R$ is the sum of the lengths of all words appearing in $R$.
The following key lemma shows that all Z-affine invariants of $\mathcal R$ have small
$\mathcal R$-representations, up to considering stronger invariants.

\begin{lemma}
  \label{lem:rep}
  Let $\mathcal R$ be a \WA.
  Let $I$ be a Z-affine invariant of $\mathcal R$ of
  length $\lInv$ and dimension $\dInv$.
  There exists an $\mathcal R$-representation $R$ 
  of size $\leq \lInv^2 \dInv^2$ of
  a Z-affine invariant $J\subseteq I$, of dimension $\leq \dInv$ and length $\leq \lInv$.
\end{lemma}

This property allows to derive the non-deterministic algorithm.
First, minimization of a \WA over a field can be performed in polynomial time
(see \eg~\cite[Corollary 4.17]{Sakarovitch09}).
Then, let $\mathcal R$ be a minimal \WA and let $\dInv,\lInv$ be positive integers.
  From Lemma~\ref{lem:rep}, we know that a Z-affine invariant of dimension $\dInv$ and length $\lInv$
  can be represented in size $O(\dInv^2 \lInv^2)$
  (up to finding a stronger invariant with smaller dimension and length).
  The algorithm works thusly: first step is to guess an $\mathcal R$-representation $R$ of a Z-affine set.
  The second step is to check that $R$ represents an invariant,
  which can be done easily using basic linear algebra.
  From this one can compute an affine \CRA with $\rCRA$ registers and $\sCRA$ states.
  Moreover, if we require that $R$ is Z-linear, we obtain a linear \CRA.
  If $R$ is not an invariant, the computation rejects.
  Note that different accepting computations may give rise to different invariants and thus different CRAs.

\subsection{Algorithm for the computation of Z-affine invariants}
\label{subsec:computation}

We describe a deterministic procedure which, given a \WA $\mathcal R$
and an integer $c$, returns a Z-affine invariant $J$ which is stronger
that any Z-affine invariant $I$ of $\mathcal R$ of length at most $c$.
When $c$ is chosen large enough, this procedure returns the strongest
Z-affine invariant of $\mathcal R$.
A similar procedure works as
well for the computation of Z-linear invariants.

\begin{wrapfigure}{R}{0.45\textwidth}
\begin{minipage}{0.45\textwidth}
   \vspace{-3ex}
\begin{algorithm}[H]
  \caption{Computing a Z-affine invariant}
  \label{algo:compute}
  \begin{algorithmic}[1]
    \Require{A \WA $\mathcal R = (u,\mu,v)$ of dimension $\dWA$, an integer $c$}
    \Ensure{A Z-affine invariant $J$ of $\mathcal R$ stronger than
    $I_c(\calR)$}
    \State $J \coloneqq \{u\}$
    \While{$J$ is not an invariant of $\mathcal R$}
      \State Pick some component $A$ of $J$, and some matrix $M$ of $\mathcal{R}$ s.t.
      $A\cdot M \not\subseteq J$
      \State $J \coloneqq J \cup A\cdot M$
      \If{$\mathrm{length}(J) > c^d$}
        \State $J \coloneqq \textsc{reduce}(J)$
      \EndIf
    \EndWhile
    \State \Return $J$
  \end{algorithmic}
\end{algorithm}
\end{minipage}
\end{wrapfigure}
Intuitively, this procedure will build a Z-affine set $J$ as follows:
it starts with a set containing only the initial vector of $\mathcal R$,
and incrementally extends it until it forms an invariant.
During this process, it should ensure that $J$ is included in every
Z-affine invariant $I$ of $\mathcal R$ of length at most $c$.
This relies on the following easy observation: if such an invariant $I$ contains at least
$c+1$ points on the same affine line (\emph{i.e.} a 1-dimensional affine space, denoted $D$),
then $I$ must have a component that contains $D$.
Indeed, as $I$ has length at most $c$, one of its components contains two such points.
As this component is irreducible, it is an affine subspace, hence contains $D$.
This reasoning can be lifted to higher dimensions as follows.

Given a \WA $\mathcal R$, and $c\in \mathbb N$, we denote by $I_c(\calR)=\bigcap_{\mathrm{length}(I)\leq c} I$
the intersection of all Z-affine invariants of $\mathcal{R}$ with at most $c$ components.

\begin{lemma}
  \label{lem:subspaces}
  Let $\mathcal R$ be a \WA and let $c,k\in \mathbb N$.
  Let $A_1,\ldots,A_{c^{k}+1}\subseteq I_c(\calR)$ be affine spaces such that:
  for any $P\subseteq \intInterv{1}{c^{k}+1}$ with $|P|\geq c^{k-1}+1$, $\affSpan{\cup_{i\in P}A_i}$ has
  dimension $k$.
  Then $\affSpan{\cup_{i\in \intInterv{1}{c^{k}+1}}A_i}\subseteq I_c(\calR)$.
\end{lemma}

Using this lemma, we derive an effective procedure
to simplify a Z-affine set $J=A_1\cup\cdots\cup A_{c^{\dWA}+1}$ by ``merging''
two components.
We denote by $\textsc{reduce}(J)$
the resulting set.
\begin{claim}
  \label{claim:reduce}
  Let $\mathcal R=(u,\mu,v)$ be a \WA of dimension $\dWA$, let $c\in \mathbb N$.
  Let $A_1,\ldots,A_{c^{\dWA}+1}\subseteq I_c(\calR)$ be affine spaces.
  One can find $1\leq i<j\leq c^\dWA+1$ such that $\affSpan{A_i\cup A_j}\subseteq I_c(\calR)$, in time $O(c^{p(d)})$, for some fixed polynomial $p$.
\end{claim}

\begin{theorem}
  \label{thm:cpxAlgoDet}
  Algorithm~\ref{algo:compute} is correct and terminates in time $O(c^{p(d)})$.
\end{theorem}

\begin{proof}
  Let us first discuss termination.
  Because of line $5$-$7$, the length of $J$ is at most $c^d+1$.
  Moreover $J$ is an increasing Z-affine set, thus its value can be modified at most $(d+1)\cdot (c^d+1)$ times,
  thus from Claim~\ref{claim:reduce} the algorithm terminates in time $O(c^{p(d)})$.

  We now discuss correctness.
  We need to show that $J$ is stronger than $I_c(\calR)$.
  Initially, this holds.
  Moreover, if $A\subseteq I_c(\calR)$ is an affine set, then for any $M\in \mu(\Sigma$)$, A\cdot M\subseteq I_c(\calR)$, since $I_c(\calR)$ is invariant.
  Thus, line $4$ preserves the property that $J$ is stronger than $I_c(\calR)$.
  Using Claim~\ref{claim:reduce}, the \textsc{Reduce} subroutine also preserves this property,
  since it only merges components whose affine span is contained in $I_c(\calR)$.
\end{proof}

\subsection{Complexity of the register minimization problem}
\label{subsec:reg}

In order to compute the strongest Z-linear and Z-affine invariants of a \WA
using Algorithms~\ref{algo:compute}, it is sufficient to be able to bound their lengths.
The following result gives such bounds.

\begin{theorem}\label{thm:lengths}
Let $\mathcal{R} = (u,\mu,v)$ be a $\dWA$-dimensional \WA on a finite alphabet $\Sigma$.
We have the following upper bounds :
\begin{itemize}
  \item The lengths of $\linHull{\mathcal{R}}$ and $\affHull{\mathcal{R}}$ are
  at most doubly-exponential in $\dWA$.
  \item If $\genMono{\mu(\Sigma)}$ is commutative (\emph{e.g.} $\Sigma$ is unary), then the length of $\linHull{\mathcal{R}}$
  is at most exponential in $\dWA$.
\end{itemize}
  We also have the following lower bound (which also hold for \WA over a unary alphabet):
  \begin{itemize}
    \item For all $\dWA > 0$, there exist a $\dWA$-dimensional \WA having
    strongest Z-linear and Z-affine invariants with lengths exponential in $\dWA$.
  \end{itemize}
\end{theorem}


\begin{proof}[Proof sketch]
The first item is shown in~\cite{BS23}, where the authors sketch a proof of a double-exponential upper bound
on the length of the \LH of a \WA, using tools from algebraic geometry,
which holds for $\mathbb{Q}$ in particular and for any field $\mathbb{K}$
where there is a double-exponential bound on the maximal order of finite groups of invertible matrices
(see~\cite[Proposition 48 and Remark 41]{BS23}). Their proof can be adapted
to $\affHull{\mathcal{R}}$.
The proof of the second item relies on basic linear algebra
and on results and ideas from~\cite{BS23} for invertible matrices
(see~\cite[Lemma 13 and Theorem 10]{BS23}).
Last, the lower bound is shown using a family of \WA
$(\mathcal{R}_i)_{i\in \mathbb{N}}$ whose dimension is polynomial in $i$
and \LH has a length that is exponential in $i$.
It is defined, using permutation matrices
of dimension $p$, for some prime number $p$, which generate 
cyclic groups.
The family is obtained by
using block matrices composed of such permutation matrices.
All the details are given in Appendix~\ref{app:length}.
\end{proof}

Thanks to this theorem, using Algorithm~\ref{algo:compute} with a large enough $c$
(at most doubly-exponential in the dimension of the given \WA),
and thanks to Theorem~\ref{thm:cpxAlgoDet}, we can prove the following result:
\begin{theorem}
  \label{thm:cpxLH}
  The \LAH of a \WA is computable in 2-\exptime.
\end{theorem}

This allows us to prove Theorem~\ref{thm:reg-min}.
Indeed, given a \WA $\calR$,
we first compute an equivalent minimal \WA, which can be done in polynomial time
(see \eg~\cite[Corollary 4.17]{Sakarovitch09}).
  Then, using Algorithm~\ref{algo:compute}, we compute the 
  \LrespAH of $\mathcal{R}$.
  Corollary~\ref{cor:minRegLH} (\resp Corollary~\ref{cor:minRegLHAff})
ensures that its dimension is the register complexity
  of $f$ w.r.t.\ the class of linear (\resp affine) \CRA, and
  the effectiveness follows from Proposition~\ref{prop:linRepToCRA}
(\resp its affine version).


Moreover, thanks to Theorem~\ref{thm:cpxLH} and the results of~\cite{BellS21},
we also have:
\begin{theorem}
  The sequentiality and unambiguity of a rational series are in 2-\exptime.
\end{theorem}

Note that the complexities of the last two theorems drop down to \exptime
when we  have a simply exponential bound
on the length of the strongest invariant.
This is the case when one considers
unary alphabets or \WA with commuting transition matrices in the linear setting,
as stated in Theorem~\ref{thm:lengths}.
In these cases, the bound is sharp.
It is still not clear however whether it is possible to close the gap between
the bounds in the general case.

\begin{remark}\label{rmk:cpxOnField}
  It is also worth noting that, while the characterizations that we obtained
  are valid for any field, the complexities of the algorithms are given in terms of number of
  elementary operations over the considered field.
  Which means that they hold for fields where we can perform basic operations in
  polynomial time (such as $\mathbb{Q}$ or its finite extensions).
  Moreover, the general upper bounds on the lengths given by Theorem~\ref{thm:lengths}
  were proven only for fields verifying a specific property (which is verified by $\mathbb{Q}$).
  See the proof for more details.
\end{remark}

\subsection{State/register tradeoff}
Reducing the number of registers may increase the number of states and vice-versa.
The following theorem summarizes what we know on this tradeoff.

\begin{theorem}\label{thm:tradeoff}
Let $f$ be a rational series realized by some $d$-dimensional \WA $\calR$.
Consider some pair of integers $(n,k)$ optimal for $f$ w.r.t.\ the class of linear \CRA.
The inequalities
$1\leq n \leq \mathrm{length}(\linHull{\mathcal{R}}) = O(2^{2^d})$
and $\dim(\linHull{\mathcal{R}}) \leq k \leq d$ hold true.

(They are valid in the affine setting as well)
\end{theorem}

\begin{remark}\label{rk:length}
Building the \CRA from the strongest invariant is not always optimal.
There are
some cases where it is possible to reduce
the number of states of a \CRA exponentially, while keeping the minimal number of registers,
by choosing an invariant that is weaker than the strongest Z-linear/Z-affine invariant but shorter.
\end{remark}