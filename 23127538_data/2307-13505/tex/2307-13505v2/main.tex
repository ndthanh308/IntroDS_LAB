\documentclass[a4paper,UKenglish,cleveref, autoref, thm-restate]{lipics-v2021}
\pdfoutput=1
%This is a template for producing LIPIcs articles.
%See lipics-v2021-authors-guidelines.pdf for further information.
%for A4 paper format use option "a4paper", for US-letter use option "letterpaper"
%for british hyphenation rules use option "UKenglish", for american hyphenation rules use option "USenglish"
%for section-numbered lemmas etc., use "numberwithinsect"
%for enabling cleveref support, use "cleveref"
%for enabling autoref support, use "autoref"
%for anonymousing the authors (e.g. for double-blind review), add "anonymous"
%for enabling thm-restate support, use "thm-restate"
%for enabling a two-column layout for the author/affilation part (only applicable for > 6 authors), use "authorcolumns"
%for producing a PDF according the PDF/A standard, add "pdfa"

%\pdfoutput=1 %uncomment to ensure pdflatex processing (mandatatory e.g. to submit to arXiv)
%\hideLIPIcs  %uncomment to remove references to LIPIcs series (logo, DOI, ...), e.g. when preparing a pre-final version to be uploaded to arXiv or another public repository

%\graphicspath{{./graphics/}}%helpful if your graphic files are in another directory

%\usepackage[utf8]{inputenc}
%\usepackage[english]{babel}
\usepackage{amsfonts}
\usepackage{amsthm}
\usepackage{mathtools}
%\usepackage[shortlabels]{enumitem}
\usepackage{appendix}
\usepackage[disable]{todonotes}
\usepackage{lipsum}
\usepackage{wrapfig}

\usepackage{thm-restate}

\usepackage{pgf,tikz,pgflibraryarrows,pgffor,pgflibrarysnakes}
\tikzstyle{background}=[rectangle,fill=gray!10, inner sep=0.1cm, rounded corners=0mm]
\usepgflibrary{shapes}
\usetikzlibrary{snakes,automata}

\usepackage{float}
%\usepackage{newfloat}
%\DeclareFloatingEnvironment{algorithm}
%\usepackage[ruled]{algorithm}
\usepackage{algorithm}
\usepackage{algpseudocode}

\newcommand{\nl}[1]{\textbf{\color{blue}#1}}
\newcommand{\pa}[1]{\textbf{\color{green!50!black}#1}}
% New commands
\newcommand{\incr}{\,\mathrm{d}}
\newcommand{\set}[1]{\{#1\}}
\newcommand{\diff}[2]{\frac{\mathrm{d}{#1}}{\mathrm{d}{#2}}}
\newcommand{\pdiff}[2]{\frac{\partial{#1}}{\partial{#2}}}
\newcommand{\ndiff}[3][]{\frac{\mathrm{d}^{#1}{#2}}{\mathrm{d}{#3}^{#1}}}
\newcommand{\npdiff}[3][]{\frac{\partial^{#1}{#2}}{\partial{#3}^{#1}}}
\newcommand{\R}{\mathbb R}

% New environments
\newtheorem{remark}{Remark}


\bibliographystyle{plainurl}% the mandatory bibstyle

\title{\hbox{Minimizing Cost Register Automata over a Field}}

%\titlerunning{Dummy short title} %TODO optional, please use if title is longer than one line

\author{Yahia Idriss {Benalioua}}{Aix Marseille Univ, CNRS, LIS, Marseille, France}{yahia-idriss.benalioua@lis-lab.fr}{}{}%TODO mandatory, please use full name; only 1 author per \author macro; first two parameters are mandatory, other parameters can be empty. Please provide at least the name of the affiliation and the country. The full address is optional. Use additional curly braces to indicate the correct name splitting when the last name consists of multiple name parts.


\author{Nathan Lhote}{Aix Marseille Univ, CNRS, LIS, Marseille, France}{nathan.lhote@lis-lab.fr}{}{}


\author{Pierre-Alain Reynier}{Aix Marseille Univ, CNRS, LIS, Marseille, France}{pierre-alain.reynier@lis-lab.fr}{}{}



\authorrunning{Y.I. Benalioua, N. Lhote and P.-A. Reynier} %TODO mandatory. First: Use abbreviated first/middle names. Second (only in severe cases): Use first author plus 'et al.'

\Copyright{Yahia Idriss {Benalioua}, Nathan Lhote and Pierre-Alain Reynier} %TODO mandatory, please use full first names. LIPIcs license is "CC-BY";  http://creativecommons.org/licenses/by/3.0/

%\ccsdesc[100]{\textcolor{red}{Replace ccsdesc macro with valid one}} %TODO mandatory: Please choose ACM 2012 classifications from https://dl.acm.org/ccs/ccs_flat.cfm 

\ccsdesc{Theory of computation~Formal languages and automata theory~Automata extensions~Quantitative automata}


\keywords{Weighted automata, Cost Register automata, Zariski topology} %TODO mandatory; please add comma-separated list of keywords

\category{} %optional, e.g. invited paper

\relatedversion{This is the full version of a paper accepted at MFCS 2024} %optional,
%e.g. full version hosted on arXiv, HAL, or other respository/website
%\relatedversiondetails[linktext={opt. text shown instead of the URL}, cite=DBLP:books/mk/GrayR93]{Classification (e.g. Full Version, Extended Version, Previous Version}{URL to related version} %linktext and cite are optional

%\supplement{}%optional, e.g. related research data, source code, ... hosted on a repository like zenodo, figshare, GitHub, ...
%\supplementdetails[linktext={opt. text shown instead of the URL}, cite=DBLP:books/mk/GrayR93, subcategory={Description, Subcategory}, swhid={Software Heritage Identifier}]{General Classification (e.g. Software, Dataset, Model, ...)}{URL to related version} %linktext, cite, and subcategory are optional

%\funding{(Optional) general funding statement \dots}%optional, to capture a funding statement, which applies to all authors. Please enter author specific funding statements as fifth argument of the \author macro.


\funding{This work has been partly funded by the QuaSy project (ANR-23-CE48-0008).}

%\acknowledgements{I want to thank \dots}%optional

%\nolinenumbers %uncomment to disable line numbering

%\renewcommand{\baselinestretch}{.98}

%Editor-only macros:: begin (do not touch as author)%%%%%%%%%%%%%%%%%%%%%%%%%%%%%%%%%%
\EventEditors{Rastislav Kr\'{a}lovi\v{c} and Anton\'{i}n Ku\v{c}era}
\EventNoEds{2}
\EventLongTitle{49th International Symposium on Mathematical Foundations of Computer Science (MFCS 2024)}
\EventShortTitle{MFCS 2024}
\EventAcronym{MFCS}
\EventYear{2024}
\EventDate{August 26--30, 2024}
\EventLocation{Bratislava, Slovakia}
\EventLogo{}
\SeriesVolume{306}
\ArticleNo{16}
%%%%%%%%%%%%%%%%%%%%%%%%%%%%%%%%%%%%%%%%%%%%%%%%%%%%%%

\nolinenumbers
\hideLIPIcs

\begin{document}

\maketitle

%TODO mandatory: add short abstract of the document
\begin{abstract}
Weighted automata (\WA) are an extension of finite automata that define functions
 from words to values in a given semiring.
 An alternative deterministic model, called Cost Register Automata ($\CRA$),
 was introduced by Alur \ea It enriches deterministic finite automata with
 a finite number of registers, which store values, updated at each transition
 using the operations of the semiring.
It is known that \CRA with register updates defined by linear maps
 have the same expressiveness as \WA.
 Previous works have studied the register minimization problem:
 given a function computable by a \WA and an integer $k$,
 is it possible to realize it using a \CRA with at most $k$ registers?

 In this paper, we solve this problem for $\CRA$ over a field
 with linear register updates, using the notion of linear hull,
 an algebraic invariant of $\WA$
 introduced recently by Bell and Smertnig.
We then generalise the approach to solve a more
challenging problem, that consists in minimizing simultaneously the number
of states and that of registers.
 In addition, we also lift our results to the setting of \CRA with affine updates.
Last, while the linear hull was recently shown to be computable by Bell and Smertnig, no complexity bounds were given.
 To fill this gap, we provide two new algorithms to compute invariants of \WA. This allows us to
 show that the register (resp. state-register) minimization problem can be solved in 2-\exptime
 (resp. in \nexptime).
\end{abstract}



\section{Introduction}
\section{Introduction}
Current quantum hardware is unable to carry out universal quantum computations due to the buildup of errors that occur during the computation. 
The magnitude of the individual error is currently above the value that the Threshold Theorem requires in order to kick-start quantum error correction and fault-tolerant quantum computation~\cite[Section 10.6]{nielsen_chuang_2010}. 
Although the experimentally achieved fidelity rates are promising and the error bounds are inching closer to the required threshold, we will have to work for the foreseeable future with quantum hardware with errors that build-up during the computation.  This implies that we can only do a limited number of steps before the output of the computation has become completely uncorrelated with the intended one.

For fault-tolerant quantum computing, we repeat four steps: 
1) We apply a number of single and two-qubit quantum gates, in parallel whenever possible; 
2) We perform a syndrome measurement on a subset of the qubits; 
3) We perform fast classical computations to determine which errors have occurred and how to correct them; 
and, 4) We apply correction terms based on the classical computations.
We then repeat these four steps with a next sequence of gates. 
These four steps are essential to fault-tolerant quantum computing. 


The starting point of this work is to use the four steps outlined above, not to carry out error correction and fault-tolerant computation, but to enhance short, constant-depth, {\em uncorrected} quantum circuits that perform single qubit gates and {\em nearest-neighbor} two qubit gates. 
Since in the long run we will have to implement error-correction and fault-tolerant computation anyhow, and this is done by such a four-step process, why not make other use of this architecture? Moreover, on some of the quantum hardware platforms, these operations are already in place.
Embracing this idea we naturally arrive at the question: what is the computational power of \textit{low-depth} quantum-classical circuits organized as in the four steps outlined above? 
We thus investigate circuits that execute a small, ideally constant, number of stages, where at each stage we may apply, in parallel, single qubit gates and {\em nearest-neighbor} two qubit gates, followed by measurements, followed by low-depth classical computations of which the outcome can control quantum gates in later stages. 
It is not clear, at first, whether such circuits, especially with constant depth, can do anything remotely useful. 
But we will see that this is indeed the case: many quantum computations can be done by such circuits in constant depth. 
By parallelizing quantum computations in this way, we improve the overall computational capabilities of these circuits, as we do not incur errors on qubits that are idle, simply because qubits are not idle for a very long time. 
Furthermore, reducing the depth of quantum circuits, at the cost of increasing width, allows the circuit to be run faster even if errors occur.

The first usage of such a four-step layout, not to do error correction, but to perform computations, can be found in the paradigm of measurement-based quantum computing~\cite{gottesman1999demonstrating,raussendorf2001one,jozsa2006introduction,clark2007generalised}: 
A universal form of quantum computing where a quantum state is prepared and operations are performed by measuring qubits in different bases, depending on previous measurements and intermediate measurements.

\citeauthor{PhamSvore2013} were the first to formalize the four-step protocol for performing computations~\cite{PhamSvore2013}. They included specific hardware topologies by considering two-dimensional graphs for imposing constraints on qubit interactions. In their model, they develop circuits for particularly useful multi-qubit gates, including specifying costs in the width, number of qubits, depth, number of concurrent time steps, size, and total number of non-Identity operations.
As a result, they find an algorithm that factors integers in polylogarithmic depth.
\citeauthor{Browne:2011} showed that the main tool in the work by \citeauthor{PhamSvore2013}, the fan-out gate, can also be replaced by additional log-depth classical computations in the measurement-based quantum computing setting~\cite{Browne:2011}.

More recently, \citeauthor{Cirac:2021} introduced a scheme to implement unitary operations involving quantum circuits combined with Local Operations and Classical Communication ($\mathsf{LOCC}$) channels: $\mathsf{LOCC}$-assisted quantum circuits~\cite{Cirac:2021}. Similarly to the four-step scheme we just described, they allow for a short depth circuit to be run on the qubits, followed by one round of $\mathsf{LOCC}$, in which ancilla qubits are measured and local unitaries are applied based on the measurement outcomes. They show that in this model any 1D transitionally invariant matrix-product state (MPS) with fixed bond dimension is in the same phase of matter as the trivial state. Similar ideas can be found in~\cite{TVV_NonAbelianTopologicalOrder_2022, tantivasadakarn2021long}.

In this work, we introduce a new model, called \textit{Local Alternating Quantum-Classical Computations} ($\LAQCC$). In this model we alternate between running quantum circuits (constrained by locality), ending in the measurement of a subset of qubits, and fast classical computations based on the measurement results. The outcome of the classical computations are then used to control future quantum circuits. We allow for flexibility in this model, by giving different constraints to the power of both the quantum circuits and the classical circuits as well as the number of alternations between them. 
Most attention will be given to $\LAQCC$ containing quantum circuits of constant depth, classical circuits of logarithmic depth and at most a constant number of alternations between them. 
Any circuit constructed in this model is considered to be of constant depth. 
We restrict ourselves to logarithmic depth classical computations, as this is the first natural and non-trivial extension beyond constant-depth classical computations. 
Constant-depth classical computations do however also have an equivalent constant-depth quantum implementation.

The definition of $\LAQCC$ sharpens the original definition of \citeauthor{PhamSvore2013} by adding constraints to the intermediate classical computations. This allows us to bound the power of $\LAQCC$ from above. 

The main result of \citeauthor{Cirac:2021}, that 1D translational invariant MPS with fixed bond dimension can be prepared by $\mathsf{LOCC}$-assisted circuits, relies on local symmetries of the MPS. These symmetries allow them to prepare local states (on a constant number of qubits) and glue them together by doing one round of the appropriate entangling measurement and corrections, after which they run a round of local unitaries to get the desired result. This general scheme for preparing states that exhibit an MPS description with the appropriate local symmetries requires only geometrically local unitaries and one round of measurement and corrections an therefore is accessible in $\LAQCC$. Studying different local symmetries, known as Symmetry Protected Topological (SPT) phases of matter, to find measurement-based constant depth circuits for states is a broad ongoing field of research~\cite{TVV_NonAbelianTopologicalOrder_2022, tantivasadakarn2021long, smith2023deterministic}. 
All these schemes have a $\LAQCC$ implementation.

%$\LAQCC$-circuits also exist for general schemes of preparing local states, based on the local tensors, and gluing them together using one round of entangled measurement and corrections, based on the local symmetry. 
%The main result of \citeauthor{Cirac:2021}, that 1D translational invariant MPS with fixed bond dimension can be prepared by $\mathsf{LOCC}$-assisted circuits, relies heavily on local symmetries of the MPS and as a result also has an equivalent $\LAQCC$ implementation. 
%The corrections applied after the measurement round are local unitaries depending on the local symmetries of the MPS. 

 

%This general scheme of preparing local states, based on the local tensors, and gluing it together by doing one round of entangled measurement and corrections, based on the local symmetry, is accessible in $\LAQCC$.
Note however that \citeauthor{Cirac:2021} also suggest a circuit for the $W$-state.
This circuit uses sequentially and dependent measurement-based corrections of the ancilla qubits. 
These dependent measurements translate to sequential alternations between the quantum and classical circuits and therefore increase the total depth to linear depth, exceeding the constant-depth constraints imposed by $\LAQCC$-circuits. 

We study the power of the $\LAQCC$ model with respect to state preparation, showing that even with only constant quantum-depth and logarithmic classical depth it remains possible to prepare states with long-range entanglement.
Another surprising result is that it is unlikely that $\LAQCC$ circuits are classically simulatable. We show that any instantaneous quantum polynomial-time (IQP) circuit~\cite{Bremner2010,Shepherd2009} has an $\LAQCC$ implementation.
Classical simulation of IQP circuits implies the collapse of the polynomial hierarchy to the third level, which is not believed to be true~\cite{Bremner2017}. Therefore, we expect that $\LAQCC$ circuits are unlikely to be classically simulatable. We bound the power of $\LAQCC$ by showing that it is contained in $\QNC^1$, the class of polynomial-size, log-depth circuits.

Next, we also study the power that intermediate classical calculations can add to quantum computations, by considering a new model that alternates between polynomially many polynomial-depth quantum circuits and unbounded classical computations
We study this model by doing a complexity theoretical analysis, where we draw inspiration from the notions of complexity given by \citeauthor{RosenthalYuen:2022}, \citeauthor{MetgerYuen:2023}, and \citeauthor{Aaronson:2004}.
All three complexity notions are based on the notion of state preparation, instead of more traditional definition of complexity such as the decidability of a computational problem. 
The first two consider classes based on sequences of quantum states preparable by a polynomial-sized quantum circuit, where the circuits are uniformly generated by a computational class, for instance, the class $\mathsf{PSPACE}$, which results in the complexity class $\mathsf{StatePSPACE}$~\cite{RosenthalYuen:2022,MetgerYuen:2023}.
The third notion considers a relative complexity, where the complexity is measured between two given states, and is measured by the number of gates, from a given gate-set, required to transform one state in another state~\cite{Aaronson:2004}. 
For our definition of state preparation complexity, we drop the uniformity constraint from~\cite{RosenthalYuen:2022,MetgerYuen:2023} and define a class as $\mathsf{StateX}$, which refers to states preparable by circuits of type $\mathsf{X}$. 
As an example, if $\mathsf{X} = \QNC^0$, this results in the class $\mathsf{StateQNC^0}$, which is the set of states preparable from the $\ket{0}^n$ state by poly-size constant-depth circuits. 
This notion is similar to the relative complexity from~\cite{Aaronson:2004}, where one state is the  $\ket{0}^n$ state and instead of counting the number of gates we consider the set of states preparable by a fixed number of gates. Using this notion of complexity we show that any state preparable by an $\LAQCC^*$ circuit is also preparable by a $\mathsf{PostQPoly}$ circuit, the class of circuits of polynomial depth with an additional post-selection gate. 

All Clifford circuits have a constant-depth $\LAQCC$ implementation, implying that any stabilizer state can be implemented by a constant-depth $\LAQCC$ circuit, see Section~\ref{sec:clifford_circuits} for a proof of this statement. 
Efficient circuits for stabilizer states have been known already through measurement-based quantum computing. Therefore this paper focuses on the preparation of non-stabilizer states, and as a surprising result we find novel constant-depth protocols for four very natural classes of non-stabilizer states.
Despite the extensive research into these four classes of non-stabilizer states and the many applications of them, no efficient constant- or low-depth state preparation protocols are known yet. We specifically consider these four classes as they are all often used as initial states in other algorithms.

The first state is a uniform superposition over an arbitrary number of states. 
This state finds applications in many quantum algorithms, as they often start with a uniform superposition over multiple states. 
This superposition is often achieved by applying Hadamard gates to every qubit due to its simplicity to prepare. 
Yet, the analysis of many algorithms, such as Shor's algorithm~\cite{Shor:1997}, would benefit from a different initial superposition. 
The circuit to prepare the uniform superposition over an arbitrary number of states uses an exact version of Grover search as a subroutine, that turns a probabilistic circuit, with a known constant probability of success, into a deterministic circuit. 
We use the circuit for preparing a uniform superposition over an arbitrary number of states as a subroutine in the next two quantum state preparation protocols. 

The second state is the $W$-state, the uniform superposition over all computational basis states of Hamming-weight~$1$, a natural long-ranged entangled state that displays a fundamentally nonequivalent type of entanglement from the Greenberger–Horne–Zeilinger state~\cite{WState:2000}, for which $\LAQCC$-type constant-depth circuits were previously known~\cite{PhamSvore2013, Cirac:2021}. 
The $W$-state is often used as benchmark for new quantum hardware~\cite{Haffner2005,Neeley2010,GarciaPerez:2021}. 
A novel way to prepare the $W$-state therefore gives a new way to benchmark different quantum devices with each other. 
A circuit for preparing the $W$-state was given in~\cite{Cirac:2021}, but this implementation requires sequentially alternating measurements followed by local unitaries, which in the $\LAQCC$ model is not considered to be of constant depth. 
We improve this protocol by giving an $\LAQCC$ implementation of the $W$-state, based on a compress-uncompress method that links the one-hot and binary encoding of integers.

The third state considered is the Dicke state, a generalization of the $W$-state, a superposition over all computational basis states with Hamming-weight $k$~\cite{Dicke:1954}. 
Dicke states have relevance in various practical settings.
For instance, for quantum game theory~\cite{zdemir2007}, quantum storage~\cite{Bacon_Compress:2006,Plesch:2010}, quantum error correction~\cite{ouyang2014permutation}, quantum metrology~\cite{toth2012multipartite}, and quantum networking~\cite{prevedel2009experimental}. 
Dicke states have been used as a starting state for variational optimization algorithms, most notably Quantum Alternating Operator Ansatz (QAOA)~\cite{Hadfield2019}, to find solutions to problems such as Maximum k-vertex Cover~\cite{Brandhofer2022,cook2020quantum}.
The ground states of physical Hamiltonians describing one-dimensional chains tend to show a resemblance to Dicke states such as states resulting from the Bethe ansatz, making them an ideal starting state when investigating the ground state behavior of these Hamiltonians~\cite{TDL_BetheAnsatzDerivation:2010,B_ExcitedStateQuantumPhaseTransitions:2013,DickeTransitions:2021}. 
For instance, the algorithm by \citeauthor{van2021preparing}, who give an algorithm to prepare the Bethe ansatz eigenstates of the spin-1/2 XXZ spin chain, starts by first preparing a Dicke state~\cite{van2021preparing}. 
A Dicke-state preparation protocol based on the compress-uncompress methodology used in the $W$-state furthermore finds applications in entanglement distillation, where the entanglement of a large state is concentrated on only a few qubits. 
Efficient deterministic circuits for preparing Dicke states have been proposed by \citeauthor{bartschi2019deterministic}~\cite{bartschi2019deterministic, bartschi2022deterministic_short_depth}. 
They provide a quantum circuit of depth $\mathO(k \log(\frac{n}{k}))$, allowing arbitrary connectivity, to prepare a Dicke state, which they conjecture to be optimal when $k$ is constant. 
In this work, we provide a constant-depth $\LAQCC$ circuit below their conjectured bound already for constant $k$. 
However, this does not directly disprove their conjecture, as we allow for intermediate measurements and classical computations. 
More significantly, we even construct constant-depth $\LAQCC$ circuits for $k = \mathO(\sqrt{n})$ greatly improving their bound.
This construction extends the compress-uncompress method for the $W$-state combined with additional subroutines. 

We continue with a log-depth state preparation protocol for the Dicke-state for arbitrary $k$. 
This protocol implements an efficient transformation between the factoradic number representation and the combinatorial number representation of a positive integer. 
The combinatorial number representation relates directly to the Dicke state. 
The provided efficient transformation between number representation systems might be of independent interest. 

We conclude by modifying our protocol for preparing a Dicke-state to a protocol that prepares quantum many-body scar states in constant-depth. 
These states have low entanglement and longer coherence times than states with similar energy density.
These characteristics make many-body scar states interesting to analyze and relevant within physics.
Many-body scar states appear for instance in the AKLT model~\cite{AKLT:1987,MRBAR:2018,MRB:2018} and different spin models~\cite{SI:2019,MOBFR:2020}.
Known methods for preparing these states have polynomial-depth~\cite{Gustafson:2023}, whereas our circuit has constant depth. 

% We conclude by studying the power that intermediate classical calculations can add to quantum computations. 
% In this study, we define a new model that relaxes constant-depth quantum circuits to polynomial depth quantum circuits, log-depth classical calculations to unbounded classical computations and a constant number of alternations to a polynomial number of alternations. 
% We call this model $\LAQCC^*$. 
% We study this model by doing a complexity theoretical analysis, where we draw inspiration from the notions of complexity given by \citeauthor{RosenthalYuen:2022}, \citeauthor{MetgerYuen:2023}, and \citeauthor{Aaronson:2004}.
% All three complexity notions are based on the notion of state preparation, instead of more traditional definition of complexity such as the decidability of a computational problem. 
% The first two consider classes based on sequences of quantum states preparable by a polynomial-sized quantum circuit, where the circuits are uniformly generated by a computational class, for instance, the class $\mathsf{PSPACE}$, which results in the complexity class $\mathsf{StatePSPACE}$~\cite{RosenthalYuen:2022,MetgerYuen:2023}.
% The third notion considers a relative complexity, where the complexity is measured between two given states, and is measured by the number of gates, from a given gate-set, required to transform one state in another state~\cite{Aaronson:2004}. 
% For our definition of state preparation complexity, we drop the uniformity constraint from~\cite{RosenthalYuen:2022,MetgerYuen:2023} and define a class as $\mathsf{StateX}$, which refers to states preparable by circuits of type $\mathsf{X}$. 
% As an example, if $\mathsf{X} = \QNC^0$, this results in the class $\mathsf{StateQNC^0}$, which is the set of states preparable from the $\ket{0}^n$ state by poly-size constant-depth circuits. 
% This notion is similar to the relative complexity from~\cite{Aaronson:2004}, where one state is the  $\ket{0}^n$ state and instead of counting the number of gates we consider the set of states preparable by a fixed number of gates. Using this notion of complexity we show that any state preparable by an $\LAQCC^*$ circuit is also preparable by a $\mathsf{PostQPoly}$ circuit, the class of circuits of polynomial depth with an additional post-selection gate. 

\paragraph{Summary of results}
\begin{itemize}
    \item We give a new definition of a computational model that captures the power of the four step process: applying a constant number of layers of one- and two-qubit gates; performing a syndrome measurement; perform a fast classical computation determining corrections; apply corrections. We call this model \emph{Local Alternating Quantum Classical Computations}, or $\LAQCC$ for short. In this model we bound the allowed quantum operations, intermediate classical calculations, and number of rounds separately. In Section~\ref{sec:LAQCC_model} we define this model and give a list of operations based on results from literature contained in this computational model. In some of these operations we explicitly use that we allow for multiple, but at most constant, rounds  of corrections.
    \item  We show show that there exist $\LAQCC$ circuits that can not be weakly simulated in Section~\ref{sec:IQP_in_LAQCC}. We further show that for every $\LAQCC$ circuit there exists a $\QNC^1$ circuit simulating it perfectly, in Section~\ref{sec:LAQCC_in_QNC1}.
    \item We introduce a new type computational complexity for preparing states and show that the extension of $\LAQCC$ where we allow a polynomial number of rounds and unbounded classical computation, is contained in $\mathsf{PostQPoly}$, the class of polynomial circuits with post-selection, in Section~\ref{sec:Complexity results}.
    \item We show a protocol to prepare the uniform superposition state of size $q$ in $\LAQCC$ using $\mathO(\ceil{\log_2(q)}^2)$ qubits in Section~\ref{sec:superposition_modulo_q}. 
    \item We show a protocol to prepare the $W_n$ state in $\LAQCC$ using $\mathO(n\log(n))$ qubits in Section~\ref{sec:W_state_in_LAQCC}.
    \item We show two ways of preparing the Dicke-$(n,k)$ state. The first method is in $\LAQCC$, works up to $k = \mathO(\sqrt{n})$, uses $\mathO(n^2\log(n))$ qubits, and is found in Section~\ref{sec:dicke:small_k}. The second method is in $\LAQCC\text{-}\mathsf{LOG}$ (an extension of $\LAQCC$ allowing for logarithmic number of alterations instead of constant), works for any $k$, uses $\mathO(\text{poly}(n))$ qubits, and is found in Section~\ref{sec:Dicke_in_LAQCC_LOG}. 
    \item We extend on our $\LAQCC$ method of generating Dicke-$(n,k)$ states for $k = \mathO(\sqrt{n})$ and show a protocol to generate many-body scar states for a particular Hamiltonian in $\LAQCC$ (Section~\ref{sec:many_body_scar}). 
\end{itemize}
Summarized in a table, we provide the following state generation protocols:
\begin{table}[htb]
\centering
\begin{tabular}{l|l|l|l}
\textbf{State description} & \textbf{Width} & \textbf{Depth} & \textbf{Implementation}\\
\hline 
Uniform superposition mod $q$: $\frac{1}{\sqrt{q}} \sum_{i = 0}^{q-1}\ket{i}$ & $\mathO(\ceil{\log^2 q})$ & $\mathO(1)$ & Section~\ref{sec:superposition_modulo_q}\\

$W$-state: $\frac{1}{\sqrt{n}}\sum_{i = 0}^{n-1}\ket{e_i}$ & $\mathO(n \log n)$ & $\mathO(1)$ & Section~\ref{sec:W_state_in_LAQCC}\\

Dicke-$(n,k)$, $k = \mathO(\sqrt{n})$: $\binom{n}{k}^{-1/2}\sum_{x \in \{0,1\}^n: |x| = k} \ket{x}$ &  $\mathO(n^2\log n)$ & $\mathO(1)$ 
&Section~\ref{sec:dicke:small_k}\\

Dicke-$(n,k)$: $\binom{n}{k}^{-1/2}\sum_{x \in \{0,1\}^n: |x| = k} \ket{x}$ & $\mathO(\text{poly}(n))$ & $\mathO(\log n)$ &Section~\ref{sec:Dicke_in_LAQCC_LOG}\\

QMBS: $\ket{S_k} = \frac{1}{k! \sqrt{\mathcal N(n,k)}}(Q^\dagger)^k \ket{\Omega}$ &  $\mathO(n^2\log n)$ & $\mathO(1)$  &  Section~\ref{sec:many_body_scar}
\end{tabular}
\caption{Summary of state preparation protocols given in this paper.}
\label{tab:sate_prep}
\end{table}
In the entry for the quantum many-body scar state $Q$ denotes the raising operator and $\mathcal N(n,k)=\binom{n-k-1}{k}$. 
Section~\ref{sec:many_body_scar} will provide more details on the variables and the implementation. 

\paragraph{Organization of the paper}
\noindent We first introduce relevant preliminaries in Section~\ref{sec:preliminaries}. 
In Section~\ref{sec:LAQCC_model} we formally define the class of Local Alternating Quantum-Classical Computations ($\LAQCC$). We also show that any Clifford circuit can be implemented in constant depth $\LAQCC$ (a result based on a result from measurement-based quantum computing~\cite{jozsa2006introduction}). 
This result allows us to give many useful multi-qubit gates and routines in Section~\ref{sec:gates_created_in_LAQCC}. 
Beyond that we show that constant depth $\LAQCC$ circuits are contained in $\QNC^1$ and that any $\mathsf{IQP}$ circuit has an $\LAQCC$ implementation.
We conclude this section with an analysis of a more powerful instantiation of $\LAQCC$ and show an inclusion with respect to the class $\mathsf{PostQPoly}$, which is the class of circuits of polynomial depth with one additional post-selection gate. 
In Section~\ref{sec:state_prep_in_LAQCC} we give $\LAQCC$ circuit implementations for preparing the uniform superposition over an arbitrary number of states, the $W$-state and the Dicke state up to $k = \mathO(\sqrt{n})$. We furthermore give a log-depth circuit implementation for preparing the Dicke state for any $k$. We conclude by showing a $\LAQCC$ circuit for generating many body scar states of a particular type of Hamiltonian.




\section{Weighted Automata and Cost Register Automata}
\label{sec:prelim}
\section{Preliminaries}
In this section, we describe the necessary background for automated planning and the significance of the International Planning Competition. 

% \subsection{Ontology}
% A formal ontology is typically represented as a set of concepts, relations, and axioms. A concept represents a set of objects or entities that share common properties, while a relation represents a connection or association between two or more concepts. Axioms are statements that define the relationships between concepts and relations. It is a formal representation of knowledge that is designed to facilitate automated reasoning and information processing. It acts as a structured vocabulary that describes a domain and promotes interoperability, data integration, and communication between humans and machines. Formally, an ontology $O$ can be represented as a tuple $(C, R, A)$, where $C$ is the set of concepts, $R$ is the set of relations, and $A$ is the set of axioms. Each concept \textit{c} $\in$ $C$ can be represented as a set of attributes, denoted as $Att(c)$. Similarly, each relation \textit{r} $\in$ $R$ can be represented as a set of attributes, denoted as $Att(r)$.

% Ontology is a branch of philosophy that deals with the nature of existence and being. In the field of computer science, however, ontology refers to a formal representation of knowledge that is designed to facilitate automated reasoning and information processing. It is a structured vocabulary that describes a domain and promotes interoperability, data integration, and communication between humans and machines. Various tools and methodologies, including Protege and ontology editors, are available for ontology creation. Ontologies are increasingly important in artificial intelligence, knowledge engineering, and the semantic web, and researchers are exploring their potential in diverse domains and applications.

% Figure environment removed

\subsection{Automated Planning}

Automated planning, also known as AI planning, is the process of finding a sequence of actions that will transform an initial state of the world into a desired goal state \cite{ghallab2004automated}. It involves constructing a plan or a sequence of actions that will achieve a specified objective while respecting any constraints or limitations that may be present. Formally, automated planning can be defined as a tuple $(S, A, T, I, G)$, where:
\begin{itemize}
    \item $S$ is the set of possible states of the world
    \item $A$ is the set of possible actions that can be taken
    \item $T$ is the transition function that describes the effects of taking an action on the current state of the world
    \item $I$ is the initial state of the world
    \item $G$ is the desired goal state
\end{itemize}
Using this notation, the problem of automated planning can be framed as finding a sequence of actions $\prec a_1, a_2, ..., a_k\succ$ that will transform the initial state $I$ into the goal state $G$, while respecting any constraints or limitations on the actions. 
 % In automated planning, 
 A problem is defined in terms of a domain and a problem instance. The domain defines the possible actions that can be taken and the effects of each action, while the problem instance specifies the initial state of the world and the desired goal state. 
Various techniques can be used to solve the planning problem, such as search algorithms, constraint-based reasoning, and optimization methods. These techniques involve exploring the space of possible plans and selecting the one that satisfies the objective and any constraints. Figure \ref{fig:planning_bw} illustrates an automated planning scenario for the blocksworld domain, where an initial state can be transformed into a goal state by executing a sequence of actions.

% \noindent \textbf{Attributes modeled about a domain.}
%   %\noindent \textbf{Attributes modeled in a domain file}
%  \begin{enumerate}
%      \item \textbf{Requirements:} A list of requirements that the planner must satisfy in order to solve the domain. Requirements include durative actions, conditional effects, or negative preconditions. For example, in blocksworld domain with types involved, one of the requirements is \emph{typing}.
%     \item \textbf{Predicates:} Predicates are fundamental elements in the planning domain that define the properties of the world. They are used to describe the initial and goal states, as well as the preconditions and effects of actions. Predicates are usually defined as logical expressions over a set of variables, where each variable can take on a finite number of values. In the context of planning, predicates are typically used to represent facts about the world that can be true or false, such as the location of an object or the status of a machine. For example, in blocksworld domain, the predicate \verb|(on b1 b2)| could indicate that block 'b2' is on top of block 'b1'.
%      \item \textbf{Actions:} Actions are the basic units of change in the planning domain. They represent atomic operations that can be performed to transform the world from one state to another. Each action has a name, a set of parameters, preconditions that must be satisfied before the action can be executed, and effects that describe the changes that the action makes to the world. Actions can be used to model a wide variety of operations, ranging from simple movements or transformations to complex processes such as planning or decision-making. For example, in blocksworld domain, the action \verb|unstack b2 b1| can be used to unstack block 'b2' from block 'b1'. 
     
%      \item \textbf{Preconditions:} Preconditions are the conditions that must be true before an action can be executed. They are usually defined using predicates and can involve multiple variables. Preconditions can also be negative, which means that a certain condition must not be true for an action to be executed. In planning, preconditions ensure that actions are only executed when the necessary conditions have been met, such as ensuring that a machine is turned off before it is serviced. For example, in blocksworld domain, the action \verb|unstack b2 b1| has a precondition of \verb|(on b1 b2)|, meaning that for the action to be valid, the block 'b2' should be on top of block 'b1'.
     
%      \item \textbf{Effects:} Effects describe the changes that an action makes to the world. They are usually defined using predicates and can involve multiple variables. Effects can be positive, which means that a certain condition becomes true after the action is executed, or negative, which means that a certain condition becomes false after the action is executed. In the context of planning, effects are used to model the changes that result from executing an action, such as moving an object from one location to another or turning a machine on. For example, in blocksworld domain, when the action \verb|unstack b2 b1| is executed, one of its effect is \verb|(not (on b1 b2))|, indicating that block 'b2' is no longer on top of block 'b1'.
     
%      \item \textbf{Constants:} Constants are values that are fixed and do not change during the execution of the planning problem. They are used to represent objects or entities in the world that have a fixed value, such as the speed limit on a road. Constants can be used to simplify the planning problem by reducing the number of variables that need to be considered and by providing a fixed set of values that can be used in predicates and actions. For example, in blocksworld domain, the constant \emph{table} could represent the surface on which the blocks are initially placed.
     
%      \item \textbf{Types:} Types are used to classify objects or entities in the world based on their attributes or properties. They are used to define the domain of values that a variable can take on and can be used to constrain the values that are assigned to variables. In the context of planning, types are typically used to group related objects or entities together, such as cars or bicycles, and to specify the properties that are common to all members of a type, such as their color or size. For example, in blocksworld domain with types involved, one can represent the predicate as \verb|(on ?x - block ?y - block)| stating that the parameters in the predicate are of type \emph{block}.

%  \end{enumerate}


% ######### Shorter version for AI Planning preliminaries
% \subsection{Automated Planning}

% Automated planning, also known as AI planning, finds actions transforming an initial world state into a goal state \cite{ghallab2004automated}. It involves creating a plan, respecting constraints, defined as $(S, A, T, I, G)$ where $S$ is the world states set, $A$ is the actions set, $T$ is the state transition function, $I$ is the initial state, and $G$ is the goal state. The challenge is to find actions $\prec a_1, a_2, ..., a_k\succ$ converting $I$ to $G$ under constraints. 

% A problem has a domain (defining actions and effects) and an instance (specifying initial and goal states). Various techniques can be used to solve the planning problem, such as search algorithms, constraint-based reasoning, and optimization methods. These techniques involve exploring the space of possible plans and selecting the one that satisfies the objective and any constraints. Figure \ref{fig:planning_bw} illustrates an automated planning scenario for the blocksworld domain, where an initial state can be transformed into a goal state by executing a sequence of actions.

\noindent \textbf{Attributes modeled about a domain.}
 \begin{enumerate}
     \item \textbf{Requirements:} A list of requirements that the planner must satisfy to solve the given domain, e.g., \emph{typing} in blocksworld with types.
     \item \textbf{Predicates:} Define world properties, e.g., \verb|(on b1 b2)| in blocksworld.
     \item \textbf{Actions:} Units of change with preconditions and effects, e.g., \verb|unstack b2 b1| in blocksworld.
     \item \textbf{Preconditions:} Conditions for action execution, e.g., \verb|(on b1 b2)| for \\ \verb|unstack b2 b1|.
     \item \textbf{Effects:} Post-action world changes, e.g., \verb|(not (on b1 b2))| after \\ \verb|unstack b2 b1|.
     \item \textbf{Constants:} Fixed values, e.g., \emph{table} in blocksworld.
     \item \textbf{Types:} Classifications based on attributes, e.g., \\ \verb|(on ?x - block ?y - block)| in typed blocksworld.
 \end{enumerate}

\noindent \textbf{Attributes modeled about a problem instance from a domain.}
\begin{enumerate}
    \item \textbf{Name:} The name of the planning problem.
    \item \textbf{Domain:} The name of the planning domain that the problem belongs to.
    \item \textbf{Objects:} A list of objects that are present in the planning problem. Objects are typically defined in terms of their type and name. In the example shown in Figure \ref{fig:planning_bw}, objects are b1, b2, and b3.
    \item \textbf{Initial State:} A description of the initial state of the world, including the values of all relevant predicates. Figure \ref{fig:planning_bw} represents an example initial state.
    \item \textbf{Goal State:} A description of the desired goal state of the world, including the values of all relevant predicates. Figure \ref{fig:planning_bw} represents an example goal state.
\end{enumerate}

% \vspace{2cm}
\subsection{International Planning Competition (IPC)}

% IPC serves as a significant means of assessing and comparing various planning systems. By presenting new planners and benchmark problems each year, the competitions aim to stimulate the advancement of new planning methodologies and reflect current trends and challenges in the field. The competition comprises multiple tracks, each covering various planning problems such as classical, temporal, and probabilistic planning. These tracks include benchmark problems that evaluate the performance of planners concerning parameters such as plan quality, plan length, and run time. The results of these competitions provide insights into the current state-of-the-art in planning and help identify the strengths and weaknesses of different planning systems. IPC can serve as an excellent starting point for building a planning-related ontology as the benchmark problems used in these competitions can provide a comprehensive overview of the domain and the types of problems that planners need to solve. 

IPC is pivotal for evaluating and contrasting planning systems. Introducing new planners and benchmarks, it promotes innovative planning methodologies and reflects the field's evolving challenges. The competition has multiple tracks, such as classical and probabilistic planning, with benchmarks assessing plan quality, length, and run time. IPC results offer a glimpse into the latest in planning, highlighting system pros and cons. The benchmarks from IPC are ideal for crafting a planning-related ontology, encapsulating the domain's breadth and planners' challenges.



\section{Problems and Main Results}
\label{sec:pbres}
% !TEX root =  ../main.tex

\begin{definition}[Register minimization problem]
  \label{def:min-prob}
  Given a class $\mathcal{C}$ of \CRA, we ask:
  \begin{itemize}
    \item \textbf{Input:} a rational series $f$ realized by a given \WA,
    and an integer $\rCRA \in \mathbb{N}$
    \item \textbf{Question:} Does there exist a \CRA realizing $f$
    in the class $\mathcal{C}$ with at most $\rCRA$ registers?
  \end{itemize}
\end{definition}

We will show this problem is decidable for the classes of linear and affine \CRA:
\begin{restatable}{theorem}{thmRegMin}
\label{thm:reg-min}
The register minimization problem is decidable for the classes of linear
and affine \CRA
in \texptime.
Furthermore, the algorithm exhibits a solution when it exists.
\end{restatable}

For a rational series $f$, the minimal number of registers needed
to realize $f$ using \CRA in some class $\mathcal{C}$ 
is called its \emph{register complexity with respect to class $\mathcal{C}$}.
\todo{PA: new}
Dually, if one wants to minimize the number of states, then we know we can always build
a linear (hence affine) \CRA with a single state (Proposition~\ref{prop:linRepEquivCRA1Stt}).
A more ambitious goal is to try to reduce simultaneously the number of states 
and of registers, in some given class $\mathcal{C}$ of \CRA.
Observe that, in general, there is no \CRA with minimal numbers of both states 
and registers (see Example~\ref{ex:CRA}).
Given a rational series $f$, we say that a pair $(\sCRA,\rCRA)$ is \emph{optimal} if
$f$ can be realized by a \CRA
in class $\mathcal{C}$ with $\sCRA$ states and $\rCRA$ registers
and no \CRA of $\mathcal{C}$ realizing $f$ with at most $\sCRA$ states can have strictly
less than $\rCRA$ registers and vice-versa.

Formally, we call the \emph{state-register complexity with respect to class $\mathcal{C}$}
of a rational series $f$,
the set of optimal pairs of integers $(\sCRA,\rCRA)$.

This leads to the definition of a second minimization problem:
\begin{definition}[State-Register minimization problem]
  Given a class $\mathcal{C}$ of \CRA, we ask:
  \begin{itemize}
    \item \textbf{Input:} a rational series $f$ realized by a given \WA,
    and two integers $\sCRA, \rCRA \in \mathbb{N}$
    \item \textbf{Question:} Does there exist a \CRA realizing $f$
    in the class $\mathcal{C}$ with at most $\sCRA$ states and at most $\rCRA$ registers?
  \end{itemize}
\end{definition}

In the sequel, we solve this problem for linear and affine \CRA:
\begin{restatable}{theorem}{thmStateRegMin}
\label{thm:state-reg-min}
The state-register minimization problem is decidable for the classes of linear and affine \CRA
in \nexptime.
Furthermore, the algorithm exhibits a solution when it exists.
\end{restatable}

\begin{remark}
  The complexities we give are valid for fields where it is possible 
  to perform elementary operations efficiently (\eg $\mathbb{Q}$).
  See Remark~\ref{rmk:cpxOnField} for a more detailed discussion on the matter.
\end{remark}


\section{Characterizing the state-register complexity using invariants of WA}
\label{sec:characterization}
% !TEX root =  ../main.tex

\subsection{Zariski topologies and invariants of WA}
\label{subsec:Zariski}

Let $\mathbb{K}$ be a field and $n\in \mathbb{N}$.
The \emph{Zariski topology} on $\mathbb{K}^n$ is defined
as the topology whose closed sets are the sets of common roots of a finite collection of polynomials of $\mathbb{K}[X_1, \dots, X_n]$.
A linear version of this topology, called the \emph{linear Zariski topology},
was introduced by Bell and Smertnig in~\cite{BellS21}.
Its closed sets, which we will call \emph{Z-linear sets},
are finite unions of vector subspaces of $\mathbb{K}^n$.

A set $S \subseteq \mathbb{K}^n$ is called \emph{irreducible} if,
for all closed sets $C_1$ and $C_2$,
such that $S \subseteq C_1 \cup C_2$, we have either $S \subseteq C_1$ or $S \subseteq C_2$.
The Zariski topologies defined above are Noetherian topologies in which every closed set
can be written as a finite union of irreducible components.
We then define the \emph{dimension} of a Z-linear set as the maximum dimension of
its irreducible components and their number will be called its \emph{length}.

For a set $S \subseteq \mathbb{K}^n$, $\linClosure{S}$
will denote its closure in the linear Zariski topology.
In this topology, closed irreducible sets are vector subspaces of $\mathbb{K}^n$
and linear maps are continuous and closed maps (mapping closed sets to closed sets).
In particular, for all $S \subseteq \mathbb{K}^n$ and linear map
$f : \mathbb{K}^n \to \mathbb{K}^n$,$\linClosure{f(S)}= f(\linClosure{S})$.
Moreover, if $S \subseteq \mathbb{K}^n$ is irreducible and $f : \mathbb{K}^n \to \mathbb{K}^n$
is continuous, then $f(S)$ is irreducible.
These properties will be used implicitly in the following
(see~\cite[Lemma 3.5]{BellS21} for more details and references).

We will also define an affine version of this topology that enjoy the same properties
in Subsection~\ref{subsec:affine}.

\begin{definition}
  Let $\Sigma$ be a finite alphabet and let $\mathcal{R} = (u,\mu,v)$
  be a $\dWA$-dimensional \WA on $\Sigma$ over $\mathbb{K}$.
  A subset $I \subseteq \mathbb{K}^\dWA$ is called an \emph{invariant} of $\mathcal{R}$
  if $u \in I$ and, for all $w \in I$ and $a \in \Sigma$, $w \mu(a) \in I$.
  For two invariants $I_1$ and $I_2$, we say that $I_1$ is \emph{stronger} than $I_2$
  if $I_1 \subseteq I_2$.
  In particular, the strongest invariant of $\mathcal{R}$
  is its \emph{reachability set} $\lReachSet{\mathcal{R}} = u \mu(\Sigma^*)$.

  An invariant that is also a Z-linear set will be called a \emph{Z-linear} invariant.
  The strongest Z-linear invariant of $\mathcal{R}$ is the closure of
  $\lReachSet{\mathcal{R}}$ in the linear Zariski topology
  (which is well-defined since the topology is Noetherian).
\end{definition}

\begin{example}[Example~\ref{ex:WA} continued]
  \label{ex:linHullWA}
  The reachability set of the $\WA$ considered in Example~\ref{ex:WA}
  is $\lReachSet{\mathcal{R}} = \big\{ (2^{2n}, 0) \,\big|\, \allowbreak n \in \mathbb{N} \big\}
  \cup \left\{ (0, 2^{2n+1}) \,\middle|\, n \in \mathbb{N}\right\}$.
  Its \LH is then the union of the two coordinate axes of the plane
  $\linHull{\mathcal{R}} = \linSpan{1,0} \cup \linSpan{0,1}$.

  Indeed, the inclusion $\subseteq$ comes from the fact that $u = (1,0) \in \linSpan{1,0}$
  and $\linSpan{1,0} \cup \linSpan{0,1}$ is stable by multiplication by $\mu(a)$
  and the inclusion $\supseteq$ comes from the fact that, for the linear Zariski topology,
  $\left\{ (1,0) \right\}$ is dense in $\linSpan{1,0}$
  and $\left\{ (0,2) \right\}$ is dense in $\linSpan{0,1}$.
\end{example}

\begin{remark}\label{rk:lin_not_alg}
In the previous example, the strongest Z-linear invariant is actually the strongest
algebraic invariant (\emph{i.e.} closed in the Zariski topology).
Of course, this is not always the case.
\end{remark}

The Z-linear invariants of two \WA
realizing the same function do not necessarily coincide but,
since $\mathbb{K}$ is a field, it is well-known that for every rational series $f$,
there exists a (computable) minimal \WA realizing $f$
that is unique up to similarity in the following sense
(see~\cite[Proposition 4.10 (Chapter III)]{Sakarovitch09}):

\begin{definition}
  Let $\mathcal{R} = (u,\mu,v)$ and $\mathcal{R}' = (u',\mu',v')$
  be two $\dWA$-dimensional \WA over $\mathbb{K}$.

  $\mathcal{R}$ and $\mathcal{R}'$ are said to be \emph{similar}
  if there exists an invertible (change of basis) matrix $P \in \sqmatrSet{\mathbb{K}}{\dWA}$
  such that $u' = u P$, $\mu'(a) = P^{-1} \mu(a) P$ for all $a \in \Sigma$
  and $v' = P^{-1}v$.
\end{definition}

\begin{remark}
  \label{rmk:similarRep}
  The Z-linear invariants of two similar \WA $\mathcal{R}$ and $\mathcal{R}'$
  only differ by a change of basis.
  \ie there is a bijection between the Z-linear invariants
  of $\mathcal{R}$ and those of $\mathcal{R}'$ that, in particular, preserves the length and dimension.
\end{remark}

\subsection{Strongest invariants and  characterization}

The notion of \LH was introduced by Bell and Smertnig in~\cite{BellS21}, under the name ``linear hull''.
They showed, in~\cite{BS23}, that it is computable and can be used to decide
whether a \WA is equivalent to a deterministic (or an unambiguous) one.

\begin{theorem}[{\cite[Theorem 1.3]{BellS21}}]
  \label{thm:seqWA}
  A rational series $f$ can be realized by a sequential $\WA$
  iff the \LH of a minimal \WA realizing $f$ has dimension at most 1.
\end{theorem}

The following result 
generalizes this theorem by linking linear $\CRA$ to Z-linear invariants.
It constitutes the key characterization that will allow us to solve the minimization problems.
\begin{theorem}[Characterization]
  \label{thm:mainThm}
  Let $f$ be a rational series.
  Then $f$ can be realized by a linear $\CRA$ with $\sCRA$ states and $\rCRA$ registers
  iff there exists a minimal \WA realizing $f$ that has a Z-linear invariant
  of length at most $\lInv$ and dimension at most $\dInv$.
\end{theorem}

As we will see in Subsection~\ref{subsec:affine},
this theorem can also be extended to affine \CRA.

Observe that, thanks to Remark~\ref{rmk:similarRep}, the property of the above characterization
is actually valid for \emph{every} minimal \WA realizing $f$.
Moreover, since the dimension of the \LH is minimal, finding this dimension
allows to solve the register minimization problem for linear $\CRA$.
This is formalized in the following result, which generalizes
Theorem~\ref{thm:seqWA} thanks to Remark~\ref{rk:seq}.

\begin{corollary}
  \label{cor:minRegLH}
  The register complexity of a rational series $f$ w.r.t.\ the class of linear \CRA
  is the dimension of the \LH of any minimal \WA realizing $f$.
\end{corollary}

An immediate consequence of this result is that computing the strongest invariant
allows to decide the register minimization problem.

\begin{example}[Example~\ref{ex:WA} continued]
  As we have seen in Example~\ref{ex:linHullWA},
  $\linHull{\mathcal{R}}$ is 1-dimensional and has two irreducible components,
  thus $\llbracket \mathcal{R} \rrbracket$ can be realized by a
  $\CRA$ with two states and one register (depicted on the right of Figure~\ref{fig:CRA}).
\end{example}

\subsection{Invariants of minimal WA and correspondence with CRA}
\label{subsec:invWA_CRA}

\begin{proposition}
  \label{prop:dimLHMin}
  Let $\mathcal{R}$ be a \WA realizing a rational series $f$.
  If $\mathcal{R}$ has a Z-linear
  invariant of length $\lInv$ and dimension $\dInv$,
  then every minimal \WA realizing $f$ has a Z-linear
  invariant of length $\leq \lInv$ and dimension $\leq \dInv$.
\end{proposition}

Thanks to Remark~\ref{rmk:similarRep}, it suffices to show the existence of one
minimal \WA verifying the proposition, since they are all similar.
It is known (see Proposition~\ref{prop:caracMinRep}) that a minimal \WA can be obtained from a \WA by alternating
between two constructions which reduce the dimension to make it match
the one of the span of the left (resp. right) reachability set. The result then
follows from the next lemma, which states that both constructions
decrease the length and dimension of the invariants.
We prove it by considering an adequate change of basis, and
verifying that it preserves invariants.

\begin{lemma}
  \label{lem:leftRightMin}
  Let $\mathcal{R}$ be a \WA realizing a rational series $f$,
  let $S_{\mathcal{R}}$ be a Z-linear invariant of $\mathcal{R}$ of length $\lInv$ and dimension $\dInv$
  and let $r = \dim(\linSpan{\lReachSet{\mathcal{R}}})$.
  We can construct an $r$-dimensional \WA
  $\mathcal{R}'$ realizing $f$, with a Z-linear invariant $S_{\mathcal{R}'}$ of length $\leq \lInv$
  and dimension $\leq \dInv$.
  The same holds with $r = \dim(\linSpan{\rReachSet{\mathcal{R}}})$.
\end{lemma}

The next proposition allows to go from Z-linear invariants of \WA to \CRA.
This construction builds on the one of~\cite[Lemma 3.13]{BellS21}, in which they build an equivalent \WA from
the strongest Z-linear invariant of a \WA.
We show that an analogous construction is valid for any Z-linear invariant,
and that we can use states of \CRA to represent the different irreducible components of the invariant,
thus reducing the number of registers used to the dimension of the invariant.

\begin{proposition}
  \label{prop:linRepToCRA}
  Let $\mathcal{R}$ be a \WA.
  If $\mathcal{R}$ has a Z-linear invariant of length $\lInv$ and dimension $\dInv$,
  then there exists a linear \CRA $\mathcal{A}$, with $\sCRA$ states and $\rCRA$ registers,
  such that $\llbracket \mathcal{A} \rrbracket = \llbracket \mathcal{R} \rrbracket $.
\end{proposition}

The next proposition shows the converse direction, from \CRA to invariants
of \WA.
The construction is the classical one from \CRA to \WA.
The existence of the adequate invariant follows from the determinism of the \CRA which ensures
that in any reachable configuration, only coordinates associated with the reachable state of the \CRA can be non-zero.

\begin{proposition}
  \label{prop:CRATolinRep}
  Let $\mathcal{A}$ be a linear \CRA.
  If $\mathcal{A}$ has $\sCRA$ states and $\rCRA$ registers, then there exists a \WA $\mathcal{R}$,
  with a Z-linear invariant of length $\lInv$ and dimension $\dInv$,
  such that $\llbracket \mathcal{A} \rrbracket = \llbracket \mathcal{R} \rrbracket $.
\end{proposition}

Using the three previous propositions, we can finally prove the main characterization:
\begin{proof}[Proof of Theorem \ref{thm:mainThm}]
  Given a linear $\CRA$ with $\sCRA$ states and $\rCRA$ registers, we can construct,
  thanks to Proposition~\ref{prop:CRATolinRep},
  an equivalent \WA with a Z-linear invariant
  of length $\lInv$ and dimension $\dInv$.
  Then the desired minimal \WA exists thanks to Proposition~\ref{prop:dimLHMin}.

  Reciprocally, applying the construction of Proposition~\ref{prop:linRepToCRA}
  to any minimal \WA gives the desired linear $\CRA$.
\end{proof}

As we will discuss in the next subsection below,
the three propositions we used for this proof can also be adapted to yield the same result
for affine \CRA.

\subsection{Z-affine invariants and affine CRA} \label{subsec:affine}
All the results of Section~\ref{sec:characterization} can actually be extended to affine \CRA
using the \emph{affine Zariski topology} instead of the linear one.
It is a slight generalization of the linear Zariski topology where closed sets,
called \emph{Z-affine} sets, are finite unions of affine spaces instead of vector spaces,
with lengths and dimensions defined like in the linear case.
It is still a Noetherian topology coarser than the Zariski topology,
affine maps are continuous and closed maps in this topology
and, more broadly, it enjoys the same properties as the linear Zariski topology
we considered throughout this section.
For a set $S \subseteq \mathbb{K}^n$, we will denote by $\affClosure{S}$ it closure
in the affine Zariski topology and, similarly to the linear case,
for a \WA $\mathcal{R} = (u,\mu,v)$, we will call any invariant of $\mathcal{R}$ that is a Z-affine set
a \emph{Z-affine invariant} of $\mathcal{R}$.
Of course, the strongest Z-affine invariant of $\mathcal{R}$ is still the closure of its reachability set
\ie its ``affine hull'' $\affHull{\mathcal{R}}$ and Remark~\ref{rmk:similarRep} is still true
for Z-affine invariants.

We obtain the same characterization of Theorem~\ref{thm:mainThm} in the affine setting :
\begin{theorem}[Characterization]
  \label{thm:charAff}
  Let $f$ be a rational series.
  Then $f$ can be realized by an affine $\CRA$
  with $\sCRA$ states and $\rCRA$ registers iff there exists a minimal \WA realizing $f$ that has a Z-affine invariant
  of length at most $\lInv$ and dimension at most $\dInv$.
\end{theorem}

We can show that Propositions~\ref{prop:dimLHMin},~\ref{prop:linRepToCRA} and~\ref{prop:CRATolinRep}
are also true if we replace Z-linear invariants by Z-affine ones and linear \CRA by affine ones.
So, the proof of Theorem~\ref{thm:charAff} remains the same as Theorem~\ref{thm:mainThm}.
All the details can be found in Appendix~\ref{apx:proofChar}.

Of course, this theorem has the same consequences of its linear counterpart
and we obtain an affine version of Corollary~\ref{cor:minRegLH}
\begin{corollary}
  \label{cor:minRegLHAff}
  The register complexity of a rational series $f$ w.r.t.\ the class of affine \CRA
  is the dimension of the \AH of any minimal \WA realizing $f$.
\end{corollary}

Working in the affine Zariski topology instead of the linear one can decrease
the dimension of the strongest invariant by one, as shown in the following example.
\begin{example}
  \label{ex:sumPow2}
  On the alphabet $\Sigma = \left\{ a \right\}$,
  let $\mathcal{R} = (u, \mu, v)$,
  where $u = (1, 2)$, $\mu(a) = \begin{pmatrix}
                                  1 & 0 \\
                                  1 & 2
  \end{pmatrix}$ and $v = (1, 0)^t$,
  be a \WA (over $\mathbb{R}$) realizing the rational series $f$ defined by
  $f(a^n) = \sum_{i=0}^{n} 2^i = 2^{n+1}-1$.

  The reachability set of $\mathcal{R}$ is $\lReachSet{\mathcal{R}}
  = \big\{ \left(\sum_{i=0}^{n} 2^i , 2^{n+1}\right) \,\big|\, n \in \mathbb{N} \big\}$.

  For the linear Zariski topology, $\lReachSet{\mathcal{R}}$ is dense in $\mathbb{R}^2$.
  So the \LH $\linHull{\mathcal{R}} = \mathbb{R}^2$ is two-dimensional.
  However, note that, for all $(x,y) \in \lReachSet{\mathcal{R}}$, $y = x+1$.
  So, by an argument of density in the affine Zariski topology,
  the \AH $\affHull{\mathcal{R}}$ is the affine line $y=x+1$, which is one-dimensional.
\end{example}

Thus, in the case where the dimensions of the affine and linear hulls doesn't match,
using affine \CRA instead of linear \CRA can allow to save one register :

\begin{example}[Example~\ref{ex:sumPow2} continued]
  \label{ex:sumPow2CRA}
  The two $\CRA$ depicted on Figure~\ref{fig:sumPow2CRA} both realize the function of Example~\ref{ex:sumPow2}.
  On the left we have a linear $\CRA$ with two registers and, on the right,
  an affine $\CRA$ with only one register.
  The characterization theorems show that both have the minimal number of
  registers for their respective classes of $\CRA$.
\end{example}

%% fig affCRA


% Figure environment removed


\section{Algorithms and complexity for the minimization problems}
\label{sec:algos}
% !TEX root =  ../main.tex

We present two original algorithms to solve the minimization problems
we consider.
It is worth observing the difference between the two characterizations we have obtained: while the register complexity can be computed from a canonical object
(the strongest Z-linear invariant of the \WA), the state-register complexity is based on the existence
of a particular Z-linear invariant.
This explains why we derive a non-deterministic
procedure for the latter, and a deterministic for the former.

\subsection{Algorithm for the state-register minimization problem}
\label{subsec:sr}

We provide here a \nexptime algorithm for the state-register minimization problem,
hence proving Theorem~\ref{thm:state-reg-min}.
The algorithm runs in \nptime in $\sCRA$, $\rCRA$, and the size of the automaton.
The fact that $\sCRA$ is given in binary explains the exponential discrepancy.


\textbf{Small representations of Z-affine sets}
Let $\mathcal R=(u,\mu,v)$ be a \WA of dimension $\dWA$ over an alphabet $\Sigma$.
Let $L=A_1\cup\cdots\cup A_\lInv$ be a Z-affine set of length $\lInv$ of $\mathbb K^\dWA$.

An $\mathcal R$-representation $R$ of $L$ is a set of $\lInv$ finite sets of words $S_1,\ldots, S_\lInv$
such that
$\affSpan{\set{u\mu(w)|\ w\in S_i}}=A_i$ for all $i\in \set{1,\cdots,\lInv}$.
The \emph{size} of $R$ is the sum of the lengths of all words appearing in $R$.
The following key lemma shows that all Z-affine invariants of $\mathcal R$ have small
$\mathcal R$-representations, up to considering stronger invariants.

\begin{lemma}
  \label{lem:rep}
  Let $\mathcal R$ be a \WA.
  Let $I$ be a Z-affine invariant of $\mathcal R$ of
  length $\lInv$ and dimension $\dInv$.
  There exists an $\mathcal R$-representation $R$ 
  of size $\leq \lInv^2 \dInv^2$ of
  a Z-affine invariant $J\subseteq I$, of dimension $\leq \dInv$ and length $\leq \lInv$.
\end{lemma}

This property allows to derive the non-deterministic algorithm.
First, minimization of a \WA over a field can be performed in polynomial time
(see \eg~\cite[Corollary 4.17]{Sakarovitch09}).
Then, let $\mathcal R$ be a minimal \WA and let $\dInv,\lInv$ be positive integers.
  From Lemma~\ref{lem:rep}, we know that a Z-affine invariant of dimension $\dInv$ and length $\lInv$
  can be represented in size $O(\dInv^2 \lInv^2)$
  (up to finding a stronger invariant with smaller dimension and length).
  The algorithm works thusly: first step is to guess an $\mathcal R$-representation $R$ of a Z-affine set.
  The second step is to check that $R$ represents an invariant,
  which can be done easily using basic linear algebra.
  From this one can compute an affine \CRA with $\rCRA$ registers and $\sCRA$ states.
  Moreover, if we require that $R$ is Z-linear, we obtain a linear \CRA.
  If $R$ is not an invariant, the computation rejects.
  Note that different accepting computations may give rise to different invariants and thus different CRAs.

\subsection{Algorithm for the computation of Z-affine invariants}
\label{subsec:computation}

We describe a deterministic procedure which, given a \WA $\mathcal R$
and an integer $c$, returns a Z-affine invariant $J$ which is stronger
that any Z-affine invariant $I$ of $\mathcal R$ of length at most $c$.
When $c$ is chosen large enough, this procedure returns the strongest
Z-affine invariant of $\mathcal R$.
A similar procedure works as
well for the computation of Z-linear invariants.

\begin{wrapfigure}{R}{0.45\textwidth}
\begin{minipage}{0.45\textwidth}
   \vspace{-3ex}
\begin{algorithm}[H]
  \caption{Computing a Z-affine invariant}
  \label{algo:compute}
  \begin{algorithmic}[1]
    \Require{A \WA $\mathcal R = (u,\mu,v)$ of dimension $\dWA$, an integer $c$}
    \Ensure{A Z-affine invariant $J$ of $\mathcal R$ stronger than
    $I_c(\calR)$}
    \State $J \coloneqq \{u\}$
    \While{$J$ is not an invariant of $\mathcal R$}
      \State Pick some component $A$ of $J$, and some matrix $M$ of $\mathcal{R}$ s.t.
      $A\cdot M \not\subseteq J$
      \State $J \coloneqq J \cup A\cdot M$
      \If{$\mathrm{length}(J) > c^d$}
        \State $J \coloneqq \textsc{reduce}(J)$
      \EndIf
    \EndWhile
    \State \Return $J$
  \end{algorithmic}
\end{algorithm}
\end{minipage}
\end{wrapfigure}
Intuitively, this procedure will build a Z-affine set $J$ as follows:
it starts with a set containing only the initial vector of $\mathcal R$,
and incrementally extends it until it forms an invariant.
During this process, it should ensure that $J$ is included in every
Z-affine invariant $I$ of $\mathcal R$ of length at most $c$.
This relies on the following easy observation: if such an invariant $I$ contains at least
$c+1$ points on the same affine line (\emph{i.e.} a 1-dimensional affine space, denoted $D$),
then $I$ must have a component that contains $D$.
Indeed, as $I$ has length at most $c$, one of its components contains two such points.
As this component is irreducible, it is an affine subspace, hence contains $D$.
This reasoning can be lifted to higher dimensions as follows.

Given a \WA $\mathcal R$, and $c\in \mathbb N$, we denote by $I_c(\calR)=\bigcap_{\mathrm{length}(I)\leq c} I$
the intersection of all Z-affine invariants of $\mathcal{R}$ with at most $c$ components.

\begin{lemma}
  \label{lem:subspaces}
  Let $\mathcal R$ be a \WA and let $c,k\in \mathbb N$.
  Let $A_1,\ldots,A_{c^{k}+1}\subseteq I_c(\calR)$ be affine spaces such that:
  for any $P\subseteq \intInterv{1}{c^{k}+1}$ with $|P|\geq c^{k-1}+1$, $\affSpan{\cup_{i\in P}A_i}$ has
  dimension $k$.
  Then $\affSpan{\cup_{i\in \intInterv{1}{c^{k}+1}}A_i}\subseteq I_c(\calR)$.
\end{lemma}

Using this lemma, we derive an effective procedure
to simplify a Z-affine set $J=A_1\cup\cdots\cup A_{c^{\dWA}+1}$ by ``merging''
two components.
We denote by $\textsc{reduce}(J)$
the resulting set.
\begin{claim}
  \label{claim:reduce}
  Let $\mathcal R=(u,\mu,v)$ be a \WA of dimension $\dWA$, let $c\in \mathbb N$.
  Let $A_1,\ldots,A_{c^{\dWA}+1}\subseteq I_c(\calR)$ be affine spaces.
  One can find $1\leq i<j\leq c^\dWA+1$ such that $\affSpan{A_i\cup A_j}\subseteq I_c(\calR)$, in time $O(c^{p(d)})$, for some fixed polynomial $p$.
\end{claim}

\begin{theorem}
  \label{thm:cpxAlgoDet}
  Algorithm~\ref{algo:compute} is correct and terminates in time $O(c^{p(d)})$.
\end{theorem}

\begin{proof}
  Let us first discuss termination.
  Because of line $5$-$7$, the length of $J$ is at most $c^d+1$.
  Moreover $J$ is an increasing Z-affine set, thus its value can be modified at most $(d+1)\cdot (c^d+1)$ times,
  thus from Claim~\ref{claim:reduce} the algorithm terminates in time $O(c^{p(d)})$.

  We now discuss correctness.
  We need to show that $J$ is stronger than $I_c(\calR)$.
  Initially, this holds.
  Moreover, if $A\subseteq I_c(\calR)$ is an affine set, then for any $M\in \mu(\Sigma$)$, A\cdot M\subseteq I_c(\calR)$, since $I_c(\calR)$ is invariant.
  Thus, line $4$ preserves the property that $J$ is stronger than $I_c(\calR)$.
  Using Claim~\ref{claim:reduce}, the \textsc{Reduce} subroutine also preserves this property,
  since it only merges components whose affine span is contained in $I_c(\calR)$.
\end{proof}

\subsection{Complexity of the register minimization problem}
\label{subsec:reg}

In order to compute the strongest Z-linear and Z-affine invariants of a \WA
using Algorithms~\ref{algo:compute}, it is sufficient to be able to bound their lengths.
The following result gives such bounds.

\begin{theorem}\label{thm:lengths}
Let $\mathcal{R} = (u,\mu,v)$ be a $\dWA$-dimensional \WA on a finite alphabet $\Sigma$.
We have the following upper bounds :
\begin{itemize}
  \item The lengths of $\linHull{\mathcal{R}}$ and $\affHull{\mathcal{R}}$ are
  at most doubly-exponential in $\dWA$.
  \item If $\genMono{\mu(\Sigma)}$ is commutative (\emph{e.g.} $\Sigma$ is unary), then the length of $\linHull{\mathcal{R}}$
  is at most exponential in $\dWA$.
\end{itemize}
  We also have the following lower bound (which also hold for \WA over a unary alphabet):
  \begin{itemize}
    \item For all $\dWA > 0$, there exist a $\dWA$-dimensional \WA having
    strongest Z-linear and Z-affine invariants with lengths exponential in $\dWA$.
  \end{itemize}
\end{theorem}


\begin{proof}[Proof sketch]
The first item is shown in~\cite{BS23}, where the authors sketch a proof of a double-exponential upper bound
on the length of the \LH of a \WA, using tools from algebraic geometry,
which holds for $\mathbb{Q}$ in particular and for any field $\mathbb{K}$
where there is a double-exponential bound on the maximal order of finite groups of invertible matrices
(see~\cite[Proposition 48 and Remark 41]{BS23}). Their proof can be adapted
to $\affHull{\mathcal{R}}$.
The proof of the second item relies on basic linear algebra
and on results and ideas from~\cite{BS23} for invertible matrices
(see~\cite[Lemma 13 and Theorem 10]{BS23}).
Last, the lower bound is shown using a family of \WA
$(\mathcal{R}_i)_{i\in \mathbb{N}}$ whose dimension is polynomial in $i$
and \LH has a length that is exponential in $i$.
It is defined, using permutation matrices
of dimension $p$, for some prime number $p$, which generate 
cyclic groups.
The family is obtained by
using block matrices composed of such permutation matrices.
All the details are given in Appendix~\ref{app:length}.
\end{proof}

Thanks to this theorem, using Algorithm~\ref{algo:compute} with a large enough $c$
(at most doubly-exponential in the dimension of the given \WA),
and thanks to Theorem~\ref{thm:cpxAlgoDet}, we can prove the following result:
\begin{theorem}
  \label{thm:cpxLH}
  The \LAH of a \WA is computable in 2-\exptime.
\end{theorem}

This allows us to prove Theorem~\ref{thm:reg-min}.
Indeed, given a \WA $\calR$,
we first compute an equivalent minimal \WA, which can be done in polynomial time
(see \eg~\cite[Corollary 4.17]{Sakarovitch09}).
  Then, using Algorithm~\ref{algo:compute}, we compute the 
  \LrespAH of $\mathcal{R}$.
  Corollary~\ref{cor:minRegLH} (\resp Corollary~\ref{cor:minRegLHAff})
ensures that its dimension is the register complexity
  of $f$ w.r.t.\ the class of linear (\resp affine) \CRA, and
  the effectiveness follows from Proposition~\ref{prop:linRepToCRA}
(\resp its affine version).


Moreover, thanks to Theorem~\ref{thm:cpxLH} and the results of~\cite{BellS21},
we also have:
\begin{theorem}
  The sequentiality and unambiguity of a rational series are in 2-\exptime.
\end{theorem}

Note that the complexities of the last two theorems drop down to \exptime
when we  have a simply exponential bound
on the length of the strongest invariant.
This is the case when one considers
unary alphabets or \WA with commuting transition matrices in the linear setting,
as stated in Theorem~\ref{thm:lengths}.
In these cases, the bound is sharp.
It is still not clear however whether it is possible to close the gap between
the bounds in the general case.

\begin{remark}\label{rmk:cpxOnField}
  It is also worth noting that, while the characterizations that we obtained
  are valid for any field, the complexities of the algorithms are given in terms of number of
  elementary operations over the considered field.
  Which means that they hold for fields where we can perform basic operations in
  polynomial time (such as $\mathbb{Q}$ or its finite extensions).
  Moreover, the general upper bounds on the lengths given by Theorem~\ref{thm:lengths}
  were proven only for fields verifying a specific property (which is verified by $\mathbb{Q}$).
  See the proof for more details.
\end{remark}

\subsection{State/register tradeoff}
Reducing the number of registers may increase the number of states and vice-versa.
The following theorem summarizes what we know on this tradeoff.

\begin{theorem}\label{thm:tradeoff}
Let $f$ be a rational series realized by some $d$-dimensional \WA $\calR$.
Consider some pair of integers $(n,k)$ optimal for $f$ w.r.t.\ the class of linear \CRA.
The inequalities
$1\leq n \leq \mathrm{length}(\linHull{\mathcal{R}}) = O(2^{2^d})$
and $\dim(\linHull{\mathcal{R}}) \leq k \leq d$ hold true.

(They are valid in the affine setting as well)
\end{theorem}

\begin{remark}\label{rk:length}
Building the \CRA from the strongest invariant is not always optimal.
There are
some cases where it is possible to reduce
the number of states of a \CRA exponentially, while keeping the minimal number of registers,
by choosing an invariant that is weaker than the strongest Z-linear/Z-affine invariant but shorter.
\end{remark}

\section{Conclusion}
\section{Conclusion and Future Work}
In this work, I design corruption-robust algorithms for the Lipschitz contextual search problem. I present the \emph{agnostic checking} technique and demonstrate its effectiveness in designing corruption-robust algorithms. There are several open problems for future research. First, in the algorithm I propose for pricing loss, the schedule for agnostic checks is fixed upfront. Can the learner design an adaptive checking schedule for the pricing loss? Second, this work assumes the learner has knowledge of the Lipschitz constant $L$. Can the learner design efficient no-regret algorithms without knowledge of $L$? 

\bibliography{biblio}

\clearpage
\appendix

\begin{comment}
\section{System Architecture}
\label{appendix:architecture}
\system has a novel modularized system architecture with three key components: 
\emph{StreamManager}, 
\emph{TxnManager} and \emph{TxnScheduler}. 
These components are instantiated in each thread locally.
The execution outline of \system is presented in Algorithm~\ref{alg:algo}.
Transactional stream processing is continuous and potentially never ends (Line 1$\sim$8).
The dependency resolution and execution of state transactions are separated into two non-overlapping phases by punctuations~\cite{Tucker:2003:EPS:776752.776780} (Line 2 and 5), which guarantees that no subsequent input event will have a smaller timestamp. 
Effectively, a batch of state transactions is collected during the first phase, and processed during the second phase.

In the first phase (i.e., stream processing phase), 
the \emph{StreamManager} conducts preprocessing for every input event ($e$). Similar to some prior works~\cite{tstream}, state transactions may be issued but not immediately processed during preprocessing (Line 3).
The \emph{pre\_processing} and \emph{post\_processing} functions are exposed as APIs to users.
The \emph{TxnManager} handles dependency resolution (Line 4) among state transactions and insert decomposed operations to construct a \tpg. We discuss the detailed two-phase \tpg construction process in Section~\ref{subsec:construction}.

In the second phase  (i.e., transaction processing phase), 
the \emph{TxnManager} is first involved again to refine (Line 6) the constructed \tpg with further dependency resolution.
The \emph{TxnScheduler} 
schedules operations for concurrent execution based on the constructed \tpg according to the three dimensions of scheduling decisions (Line 7). 
In particular, a scheduling decision model $M$ is instantiated based on the constructed \tpg (Line 14).
\textbf{\circled{1}} Guided by $M$, execution threads adopt an exploration strategy (Section~\ref{subsec:explore}) to explore the constructed \tpg for operations available to be scheduled constrained by dependencies. 
\textbf{\circled{2}} 
During exploration, one or multiple operations may be treated as the 
% basic 
unit of scheduling (Section~\ref{subsec:granularity}). 
Subsequently, \textbf{\circled{3}} every thread executes operation(s) in the unit of scheduling with various abort handling mechanisms (Section~\ref{subsec:abort_handling}).
Only when state transactions are processed (i.e., committed or aborted) can the associated input events be postprocessed (Line 8) by the \emph{StreamManager} based on transaction processing results.
\end{comment}

\begin{comment}
\begin{algorithm}
\footnotesize
    \KwData{$e$ \tcp{Input event}}
    \KwData{$txn_{ts}$ \tcp{State transaction}}
    \KwData{$G$ \tcp{The currently constructed TPG}}
    \While{!finish processing of input streams}{
        \eIf(\tcp*[h]{Phase 1}){\text{$e$ is not a $punctuation$}}{
                $txn_{ts}$ $\gets$ PRE\_Processing($e$)\;
                \textbf{TPG\_Construction}($G$, $txn_{ts}$)\; 
          }(\tcp*[h]{Phase 2}){
                \textbf{TPG\_Refinement}($G$)\; 
                \textbf{TXN\_Scheduling}($G$)\; 
                POST\_Processing()\;
          }
    }
    
    \SetKwFunction{FMain}{TPG\_Construction}
    \SetKwProg{Fn}{Function}{:}{}
    \Fn{\FMain{$G$, $txn_{ts}$}}{
        $O_{1..k}$ $\gets$ \textbf{Partition} $txn_{ts}$\;
        \ForEach{\text{operation $O_{i}$ $\in$ $O_{1..k}$}}{
            \textbf{Identify} its \ld\;
            $G$ $\gets$ $G$ + $O_{i}$ \;
        }
    }
    \SetKwFunction{FMain}{TPG\_Refinement}
    \SetKwProg{Fn}{Function}{:}{}
    \Fn{\FMain{$G$}}{
        \ForEach{\text{vertex $e_{i}$ $\in$ $G$}}{
            \textbf{Identify} its \td, \pd\;
        }
    }
    
    \SetKwFunction{FMain}{TXN\_Scheduling}
    \SetKwProg{Fn}{Function}{:}{}
    \Fn{\FMain{$G$}}{
        $M$ $\gets$ Instantiated with $G$;\tcp{A decision model}
        \While{!finish scheduling of $G$
        }{
          \textbf{\circled{2}} $Scheduling Unit$ $\gets$ \textbf{\circled{1}} \emph{Explore}($G$, $M$)\; 
            \textbf{\circled{3}} \emph{Execute with Abort Handling} ($Scheduling Unit$)\; 
        }
    }
  \caption{Execution Outline of \system}
  \label{alg:algo}
\end{algorithm}
\end{comment}

\end{document}
