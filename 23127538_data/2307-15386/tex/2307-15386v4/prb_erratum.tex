\documentclass[prb,twocolumn,amsmath,amssymb,superscriptaddress]{revtex4-2}
\usepackage{epsfig,graphicx,graphics,float}
%\usepackage[bookmarks=true,colorlinks,linkcolor=blue,urlcolor=blue,citecolor=blue]{hyperref}
\usepackage[T1]{fontenc}
\usepackage[latin9]{inputenc}
\usepackage{appendix}
\usepackage{amscd}
\usepackage{bm}
\usepackage{psfrag} 
\usepackage{bbm} 
\usepackage{babel}
\usepackage{wasysym}
\usepackage{mathrsfs}
\usepackage{xcolor}
\usepackage{array}
\usepackage{subfigure}
\usepackage{verbatim} 
\usepackage{braket}
\usepackage{soul}
\usepackage[normalem]{ulem}


\newcommand{\pd}{ {\phantom{\dagger}} }
\newcommand{\be}{\begin{equation}}
\newcommand{\ee}{\end{equation}}
\newcommand{\bea}{\begin{eqnarray}}
\newcommand{\eea}{\end{eqnarray}}
\newcommand{\bmat}{\begin{pmatrix}}
\newcommand{\emat}{\end{pmatrix}}
\newcommand{\lb}{\left(}
\newcommand{\rb}{\right)}
\newcommand{\lsb}{\left[}
\newcommand{\rsb}{\right]}
\newcommand{\mc}{\mathcal}
\newcommand{\mb}{\mathbf} 
\newcommand{\mr}{\mathrm} 

\newcommand{\new}[1]{\textcolor{red}{#1}}
\newcommand{\tim}[1]{\textcolor{blue}{[{#1}]}}
\newcommand{\ale}[1]{\textcolor{violet}{[{#1}]}}
\newcommand{\buc}[1]{\textcolor{teal}{[{#1}]}}

\usepackage[final]{hyperref} % adds hyper links inside the generated pdf file
\hypersetup{
	colorlinks=true,       % false: boxed links; true: colored links
	linkcolor=blue,        % color of internal links
	citecolor=blue,        % color of links to bibliography
	filecolor=magenta,     % color of file links
	urlcolor=blue         
}

\setcounter{figure}{0}
\renewcommand{\thefigure}{E\arabic{figure}}

\setcounter{equation}{0}
\renewcommand{\theequation}{E\arabic{equation}}

\begin{document}
\title{Erratum: Nonreciprocal charge transport and subharmonic structure
 in voltage-biased Josephson diodes
 [Phys. Rev. B 109, 024504 (2024)]}

\author{A. Zazunov}
\affiliation{Institut f\"ur Theoretische Physik, Heinrich-Heine-Universit\"at, D-40225  D\"usseldorf, Germany}
\author{J. Rech}
\affiliation{Aix Marseille Univ., Universit\'e de Toulon, CNRS, CPT, Marseille, France}
\author{T. Jonckheere}
\affiliation{Aix Marseille Univ., Universit\'e de Toulon, CNRS, CPT, Marseille, France}
\author{B. Gr{\'e}maud}
\affiliation{Aix Marseille Univ., Universit\'e de Toulon, CNRS, CPT, Marseille, France}
\author{T. Martin}
\affiliation{Aix Marseille Univ., Universit\'e de Toulon, CNRS, CPT, Marseille, France}
\author{R. Egger}
\affiliation{Institut f\"ur Theoretische Physik, Heinrich-Heine-Universit\"at, D-40225  D\"usseldorf, Germany}


\maketitle

A mistake has been made in Sec.~IV of Ref.~\cite{PRB} when taking the 
zero-temperature limit of the general expression for the conductance (47) of a normal-superconducting (NS) junction. 
The correct form of Eq.~(48) in Ref.~\cite{PRB} must read
\begin{equation}\label{zeroT}
\frac{G(V)}{G_0}= \frac12 \left( \tilde I_{1}(eV)-\tilde I_{2}(-eV) \right),
\end{equation}
such that electron- and hole-type scattering channels ($s=1,2$) contribute to $G(V)$ with opposite signs of $V$. 
Since our results in Secs.~IV B-C are based on Eq.~(48), they have to be corrected accordingly. In particular, Eq.~(50) should be replaced by
\begin{equation}\label{GV}
    \frac{G(V)}{G_0}=1+\frac12\sum_{\alpha=\pm} e^{-2\tilde \gamma_\alpha(\alpha eV)} =1+|a_1(eV)|^2,
\end{equation}
where we took into account that $|a_1(eV)|=|b_2(-eV)|$, and hence Eq.~(51) should read
\begin{eqnarray}  \label{GT1}
  \frac{G(v,q\xi)}{G_0} &=& 1+   \Theta(1-|v-q\xi|) + \\ \nonumber &+&
\frac{\Theta(|v-q\xi|-1)}{\left(|v-q\xi|+
\sqrt{\left(v-q\xi\right)^2-1}\right)^2}.
\end{eqnarray}
This equation clearly reveals an asymmetry of $G(V)\ne G(-V)$ under voltage reversal for
$q\neq 0$, and hence one can find rectification in the NS case.   The corrected version of Fig.~3 in Ref.~\cite{PRB} is shown in Fig.~\ref{fig}.

% Figure environment removed


Unfortunately, we have missed this effect in our general expression (47) in Ref.~\cite{PRB} for $G(V)$. 
The subsequent analysis at zero temperature, due to the above mistake in Eq.~(48), 
led us to the erroneous conclusion that the symmetry relation $G(V) = G (-V)$ is always satisfied.  
Our corrected results agree with the results of Ref.~\cite{Fu}.

We thank M. Davydova, L. Fu and M. Geier for alerting us to the above point.

\bibliography{references}

\end{document}