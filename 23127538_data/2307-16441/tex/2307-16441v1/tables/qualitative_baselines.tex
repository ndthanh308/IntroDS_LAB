\begin{table}[!bh]
    \centering
    \def\arraystretch{1.0}
    \resizebox{\columnwidth}{!}{
    \setlength\tabcolsep{0pt}
    \footnotesize
    \renewcommand{\arraystretch}{0.0}
    \begin{tabular}{ccccc}
         $I_{\text{ref}}$ & \methodname & PT & SNP & SNP+  \\
         
        % Figure removed & 
        % Figure removed & 
        % Figure removed & 
        % Figure removed & 
        % Figure removed \\
        
                 % Figure removed & 
         % Figure removed & 
         % Figure removed & 
         % Figure removed & 
         % Figure removed \\
         
         % Figure removed & 
         % Figure removed & 
         % Figure removed & 
         % Figure removed & 
         % Figure removed \\
         
        % Figure removed & 
        % Figure removed & 
        % Figure removed & 
        % Figure removed & 
        % Figure removed \\
    \end{tabular}
    }
    \vspace{2mm}
    \captionof{figure}{Qualitative comparison of \methodname~with baselines. Given the {\color{blue} context} (blue) strokes, we generate 100 {\color{red} predicted} (red) stroke sequences and plot the one that better matches the {\color{green} ground truth} (green). %Our method produces the sequence closest to the ground truth, suggesting that the distribution of predicted stroke sequences better matches the ground truth. 
    \elia{First row, only \methodname~is able to produce a sequence whose stroke positioning is similar to the ones of the ground truth, while the baselines tend to unrealistically cluster the strokes in a tight area. Second row, successive strokes predicted by \methodname~have similar colors as in the dataset demonstrations, while the baselines unrealistically jump between the grass, the sky, and the trees}} %since no notion of color consistency is present.}}
    \label{fig:qualitative_baseline}
\end{table}
