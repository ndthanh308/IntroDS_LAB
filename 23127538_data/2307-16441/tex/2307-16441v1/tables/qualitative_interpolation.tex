\begin{table}[!t]
    \centering
    \def\arraystretch{1.0}
    \resizebox{\columnwidth}{!}{
    \setlength\tabcolsep{0pt}
    \footnotesize
    \renewcommand{\arraystretch}{0.0}
    \begin{tabular}{cccccc}
         $\alpha\!=\!0$ & $\alpha\!=\!0.2$ & $\alpha\!=\!0.4$ & $\alpha\!=\!0.6$ & $\alpha\!=\!0.8$ & $\alpha\!=\!1$ \\
        %% Figure removed &
       % Figure removed &
       % Figure removed &
       % Figure removed &
       % Figure removed &
       % Figure removed &
       % Figure removed \\
       
       %% Figure removed &
       % Figure removed &
       % Figure removed &
       % Figure removed &
       % Figure removed &
       % Figure removed &
       % Figure removed \\
       
        % Figure removed &
        % Figure removed &
        % Figure removed &
        % Figure removed &
        % Figure removed &
        % Figure removed \\
        
        %% Figure removed &
        % Figure removed &
        % Figure removed &
        % Figure removed &
        % Figure removed &
        % Figure removed &
        % Figure removed
    
    \end{tabular}
    }
    \vspace{2mm}
    \captionof{figure}{Latent code $z$ interpolation results (the reference image is omitted for better visualization). For each image, given the same {\color{blue} context} strokes, we sample two latent codes  $z_{\mathrm{start}}, z_{\mathrm{end}} \sim \mathcal{N}(0,1)$ and linearly interpolate, obtaining $z_i = (1 - \alpha) \cdot z_{\mathrm{start}} + \alpha \cdot z_{\mathrm{end}}$ (outlined in {\color{red} red}). The strokes smoothly transition changing position but focusing on the same object.}
    \label{fig:interpolation}
\end{table}