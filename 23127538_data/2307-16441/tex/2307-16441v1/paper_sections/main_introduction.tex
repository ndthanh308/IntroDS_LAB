% Figure environment removed

\section{Introduction}

One of the main objectives of image generation methods is enabling novel and more powerful ways in which humans can express their creativity. This objective inspired a lot of research and advancements in deep generative models, that are now able to produce outputs with photorealistic quality in several generation tasks. A recent trend in deep image generation is that of improving the way in which users can control  and interact with the generation process, thus providing tools to convey users intentions.
In this context, recent works allow users to generate or edit images with high quality by sketching (\cite{ghosh2019interactive,liu2021deflocnet}), modifying the semantic layouts (\cite{lee2020maskgan,ling2021editgan,park2019gaugan,zhu2020sean}), or providing a text prompt (\cite{bau2021paint,nichol2021glide,ramesh2021zero,xu2021predict}).
These methods, however, allow users to influence the final output only in an indirect manner, \textit{i.e.} through the sketched semantic layout or the input text.

Recently, several learning-based methods for painting generation have been proposed, commonly referred to as Neural Painting (NP) methods.
Differently from other generative approaches that operates in the pixel space, NP methods leverage a parameterized brushstroke representation which is more aligned to how humans  visualize and conceptualize an artwork (\cite{Kotovenko_2021_CVPR,liu2021painttransformer,zou2021stylized}).
The strokes-based vector representation offers several benefits compared to the pixel-based representation, such as the ability to modify or erase individual strokes. Additionally, separating the representation from the rendering process enables the strokes to be rendered at any desired output resolution.

Painting has historically been a powerful tool with which humans expressed their creativity. However, in this respect, current NP methods are inherently limited, as they are only designed to reconstruct and stylize a given target image, leaving no possibility for the user to influence the generation process.
Lacking the ability to integrate users' painting style, these methods are unsuitable in interactive scenario.
This work represents the first attempt to fill this gap in the literature \willi{and bring the next level of interaction to NP}.
Inspired by \textit{en plein air} painting, \textit{i.e.} the setting where a painter looks at an outdoor scene and tries to represent it on a canvas, we introduce the novel task of Interactive Neural Painting (see Fig.~\ref{fig:teaser}). Specifically, we propose an iterative and interactive process where, given a reference image the user would like to paint and an incomplete canvas, {a computational tool} {based on NP techniques} assists the user in drawing the painting. The tool provides multiple suggestions about the next strokes at each iteration, from which the user can choose to continue its artwork.
Such a tool speeds up the painting process but, differently from existing NP approaches, leaves the user a high degree of control on the final output, with the potential of making painting an artistic medium accessible not only to highly-skilled individuals. Our system can be integrated into digital drawing tools used by amateur and professional artists, such as Adobe Photoshop, GIMP, and Krita, as shown by our demo in the \emph{Supp. Mat.}.

We devise the first method for Interactive NP (INP). Our method, which we call \methodname, introduces a conditional transformer VAE architecture that generates stroke suggestions.
To ensure seamless interaction with the user, our method is specifically trained to produce stroke suggestions that closely match the dynamics of the painting process represented in a given dataset of painting demonstrations. 
\new{The dataset is built to reflect in a synthetic manner the main aspects valued by a human painter such as color consistency, local proximity and object-based painting. Additionally, artists typically begin by portraying a rough depiction of the reference image, and incrementally incorporate finer details during the painting process (\citet{zhao2020painting, Singh2021IntelliPaintTD}). We follow previous work (\cite{zou2021stylized, liu2021painttransformer}) and adopt a coarse-to-fine assumption in our dataset to reflect this behaviour, with rough strokes spatially covering the  canvas in the first stages of the painting an detailed localized strokes towards the final stages (see Sec.~\ref{sec:dataset} for more details).} 
To effectively learn the characteristics of the stroke dataset, we introduce a distribution matching loss that minimizes the discrepancies between the suggested strokes and the painting demonstrations. In addition, a two-stage VAE decoder is proposed that tightly integrates visual features into the stroke prediction process. 
Furthermore, we make our approach probabilistic by nature to capture the complex distribution of possible continuations given the current canvas state. In this way, \methodname~can produce multiple suggestions about what to paint next.
We demonstrate our method on two novel datasets which we specifically introduce for the INP task, built upon the \textit{ADE 20K Outdoor}~\cite{zhou2017ade20k} and \textit{Oxford-IIIT Pet}~\cite{parkhi2012cats} datasets. Our extensive evaluation shows that our model produces a wide set of suggestions that closely match the characteristics of the painting demonstrations. Quantitative comparison against state of the art NP methods, supported by results on a user study, demonstrates state-of-the-art performance of our method.


\noindent \textbf{Contributions.} To summarize, our main contributions are: 

\begin{itemize}
    \item The novel image generation task of INP,  which for the first time brings interactivity to neural painting.
    \item The first approach based on conditional transformer VAE to address this task, with specific architectural choices and training protocols.
    \item Two novel synthetic datasets and a set of evaluation metrics for training and evaluating INP models, to foster and assess the research in this new area.  
\end{itemize}