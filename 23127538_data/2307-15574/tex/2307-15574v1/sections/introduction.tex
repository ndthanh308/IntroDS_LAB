%------------------------------------------------------------------------------
\section{Introduction}
\label{sec:intro}
%------------------------------------------------------------------------------
Extended reality (XR), including augmented reality (AR) and virtual reality (VR), involves computationally intensive functionalities such as object detection, localization and mapping, and 3D graphics rendering.
Given the low-latency and high-throughput requirements of XR applications~\cite{morin2022toward, taleb2021extremely}, their associated high computing costs limit the XR experiences that can be supported on resource-constrained mobile devices.


To address this, there have been a number of efforts for distributed XR systems~\cite{lai2019furion, meng2020coterie, liu2020firefly, liu2018cutting, zhang2019rendering, liu2019edge, george2020openrtist, schneider2017augmented, cloudxr, azremoterendering, isar, azcustomvision}.
A common thread across them is to predetermine the functionalities that are offloaded to the server, based on assumptions about environmental factors: the client device capacities, network conditions, and workloads.
They provide support for offloading fixed functional components of perceptions or %application
rendering, and their benefits can be realized only in specific deployment contexts.

However, we argue that the distribution contexts will differ vastly in terms of the capabilities of the user devices and offload servers, the connectivity among them, and the XR workloads.
In such scenarios, current solutions will be limited in their ability to provide effective server assistance in various contexts.
Applications would need to rely on combinations of existing techniques to leverage the server in assisting with different functionalities.
It will require significant additional development and configuration efforts.



Currently, flexibility in XR workload distribution is missing due to a lack of adequate systems support.
There are previous offloading systems for flexible function migration~\cite{cuervo2010maui, chun2011clonecloud, kosta2012thinkair},
but it is hard to extend their benefits to XR due to their design limitation of function-level offloading (see \S\ref{sec:related} for more details).
Existing stream processing (SP) libraries can potentially enable flexible workload distribution by creating a pipeline at runtime.
While they are used in use cases similar to XR, \eg, multimedia streaming~\cite{gstreamer} and perception pipelines~\cite{beard2017raftlib, quigley2009ros}, the current SP frameworks lack some of the necessary features to adapt distributed stream processing (DSP) for use in XR.
As described in \S\ref{sec:streamprocessing}, a DSP system for XR should support efficient local communication for collocated pipeline components and blocking and non-blocking communication semantics to express the pipeline dependencies and synchronization.
Moreover, it should provide queue size management and multiple network protocol supports for data freshness requirements of distributed XR pipelines, which are not available in existing solutions.


Simply adding the missing features to existing SP libraries is not sufficient to enable the flexible distribution of XR pipelines.
Even if the SP libraries are extended with those DSP features, there are still issues about how to provide the features properly across the development and deployment phases.
Specifically, in existing SP libraries, a pipeline can be created at runtime, and it requires a user to connect pipeline components (compute kernels) via the developer-specified communication ports.
However, since the communication attributes among compute kernels are determined under the user's pipeline context, the user (not the developer of the kernel) should configure the communication attributes of the connection ports when creating a distributed pipeline.

In response, we present {\bf \sys} -- an open-source, flexibly configurable, and high-performance system for distributed XR.
To bring flexibility, we design \sys\ as a DSP system specialized for XR.
With \sys, XR pipelines can be flexibly created for various distribution scenarios at runtime by a user, without requiring any code modification in pipeline kernels.
We identify the key issues in using DSP systems for XR (\S\ref{sec:spissues}) and describe our design decisions which provide the necessary DSP features and address them (\S\ref{sec:designdecisions}).
\sys\ provides a framework to enable the flexible distribution of XR functionalities, streamlining the development and deployment phases.
The developers write their kernels without considering how and where each kernel runs in user pipelines.
The user can configure the communication attributes of the given components without any change.
This feature is realized by the \sys's kernel design with its port abstractions and interfaces (\S\ref{sec:flexrkernel}).
Once the developer writes a \sys\ compute kernel, 
it can
be flexibly deployed and executed in 
diverse distribution scenarios, per user configuration.

We demonstrate the effectiveness of \sys\ through experimental evaluation with three typical XR use cases and four distribution scenarios (\S\ref{sec:evaluation}).
Compared to the existing distributed XR systems~\cite{chen2018marvel, george2020openrtist, lai2019furion, liu2019edge, liu2018cutting, schneider2017augmented, zhang2019rendering}, \sys\ is shown to support all distribution scenarios by creating distributed pipelines with given kernels at runtime.
Our evaluation results show that the offloading effect of each scenario is different based on the workloads, offloading overheads, and device capacity, which support the importance of flexibility in XR workload distributions.
Overall, this paper makes the following contributions:

\begin{tightitemize}
  \item We describe the limitations of existing distributed XR systems with respect to the need for flexibility, and identify the required features for applying DSP to XR.
  \item We present \sys, a DSP system specialized for distributed XR, which addresses the design issues of DSP for XR and enables flexible distributions of XR pipelines.
    Our evaluation in different distribution scenarios demonstrates that \sys\ practically delivers on the promise of flexibility and performance.
  \item We fully open-source \sys, hoping that it would reduce the barriers for further research in the area of distributed XR\footnote{\url{https://github.com/gt-flexr/FleXR}}.
\end{tightitemize}
