%------------------------------------------------------------------------------
\section{Discussion}
\label{sec:discussion}
%------------------------------------------------------------------------------

FleXR offers flexibility in how an XR workload is distributed across a device and offload server(s). This opens up several opportunities for innovation and new research directions.

First, while with \sys\ the XR configuration can be tuned to the specifics of the deployment context to realize optimal workload distributions, currently, this requires manual effort from the system users.
To fully realize the potential of \sys, future research is needed on automated deployment and resource management.
New methods are needed to consider factors such as kernel costs, client and server capacity, network state, and offloading overheads, as well as to enable dynamic adaptation of the workload configurations.
Previous function offloading systems have used linear solvers and resource profilers for offloading decisions with static analysis~\cite{chun2011clonecloud, cuervo2010maui}, but this approach would be more complex in \sys\ as the kernel costs and offloading overheads are subject to change based on user and server situations.
%This would be an important direction for future research.

Second, by making it possible to integrate third-party components and application frameworks with \sys\ (\eg, the game engines), we make it possible to consider a future landscape of XR supported by distributed edge-cloud infrastructure, potentially with different performance, quality, or other properties, that can be combined in different ways to support complex future XR use cases. This new landscape opens up new challenges for distributed orchestration for XR, and also promotes the reuse of service functionality in different scenarios, thus enabling faster innovation.

Finally, while distributed service composition has been considered in other contexts~\cite{servicecomposition1,servicecomposition2}, several XR-specific aspects raise new challenges and opportunities that future work should address. These are related to performance/timeliness and quality tradeoffs that exist at the application level, sharing and reuse across users (\eg, as done for specific offload services in~\cite{ran2019sharear}), XR-specific transport protocols~\cite{braud2017future, abdallah2018delay}, new privacy concerns, \etc\ By creating and open sourcing the \sys\ infrastructure, we believe our work will facilitate such research directions.
