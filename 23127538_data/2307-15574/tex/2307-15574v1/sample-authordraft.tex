%\documentclass[sigconf,authordraft]{acmart}
% \documentclass[sigconf,anonymous]{acmart} % authordraft
\documentclass[sigconf]{acmart} % authordraft
\renewcommand\footnotetextcopyrightpermission[1]{} % copyright
\settopmatter{printfolios=true,printacmref=true}

\AtBeginDocument{%
  \providecommand\BibTeX{{%
    \normalfont B\kern-0.5em{\scshape i\kern-0.25em b}\kern-0.8em\TeX}}}

%\setcopyright{acmcopyright}
%\copyrightyear{2023}
%\acmYear{2023}
%\acmDOI{XXXXXXX.XXXXXXX}

%% These commands are for a PROCEEDINGS abstract or paper.
%\acmConference[MMSys'23]{Make sure to enter the correct
%  conference title from your rights confirmation emai}{June 7--10, 2023}{Vancouver, BC, Canada}
%%
%%  Uncomment \acmBooktitle if the title of the proceedings is different
%%  from ``Proceedings of ...''!
%%
%%\acmBooktitle{Woodstock '18: ACM Symposium on Neural Gaze Detection,
%%  June 03--05, 2018, Woodstock, NY}
%\acmPrice{15.00}
%\acmISBN{978-1-4503-XXXX-X/18/06}

% Additional packages
\usepackage[bottom]{footmisc}
\usepackage{romannum}
\usepackage{textcomp}
\usepackage{tabularx}
\usepackage{diagbox}
\usepackage{hhline}

\usepackage{bbding}
\usepackage{pifont}
\usepackage{wasysym}
\usepackage{soul}

\usepackage{tikz}
\usepackage{xcolor}
\newcommand*\circled[1]{\tikz[baseline=(char.base)]{
            \node[shape=circle,fill,inner sep=1pt] (char) {\textcolor{white}{#1}};}}

\usepackage{cleveref}
\crefformat{section}{\S#2#1#3}
\crefformat{subsection}{\S#2#1#3}
\crefformat{subsubsection}{\S#2#1#3}

% for source codes
\usepackage{listings}
\newcommand\YAMLcolonstyle{\color{red}}
\newcommand\YAMLkeystyle{\color{black}\small\ttfamily}
\newcommand\YAMLvaluestyle{\color{blue}}
\makeatletter

% here is a macro expanding to the name of the language
% (handy if you decide to change it further down the road)
\newcommand\language@yaml{yaml}

\expandafter\expandafter\expandafter\lstdefinelanguage
\expandafter{\language@yaml}
{
  keywords={true,false,null,y,n},
  keywordstyle=\color{darkgray}\bfseries,
  basicstyle={\YAMLkeystyle\footnotesize},                                 % assuming a key comes first
  sensitive=false,
  comment=[l]{\#},
  morecomment=[s]{/*}{*/},
  commentstyle=\color{purple},
  stringstyle=\YAMLvaluestyle,
  moredelim=[l][\color{orange}]{\&},
  moredelim=[l][\color{magenta}]{*},
  moredelim=**[il][\YAMLcolonstyle{:}\YAMLvaluestyle]{:},   % switch to value style at :
  morestring=[b]',
  morestring=[b]",
  literate =    {---}{{\ProcessThreeDashes}}3
                {>}{{\textcolor{red}\textgreater}}1
               {|}{{\textcolor{red}\textbar}}1
                {\ -\ }{{\mdseries\ -\ }}3,
}

% switch to key style at EOL
\lst@AddToHook{EveryLine}{\ifx\lst@language\language@yaml\YAMLkeystyle\fi}
\makeatother

\newcommand\ProcessThreeDashes{\llap{\color{cyan}\mdseries-{-}-}}

% C/C++
\usepackage{color}
\definecolor{dkgreen}{rgb}{0,0.6,0}
\definecolor{gray}{rgb}{0.5,0.5,0.5}
\definecolor{mauve}{rgb}{0.58,0,0.82}
\lstset{
  frame=tb,
  language=C++,
  commentstyle=\color{dkgreen},
  keywordstyle=\color{blue},
  numberstyle=\scriptsize\color{gray},
  stringstyle=\color{mauve},
  basicstyle={\footnotesize\ttfamily},
  breaklines=true,
  breakatwhitespace=false,
  showstringspaces=false,
  aboveskip=2mm,
  belowskip=2mm,
  columns=flexible,
  numbers=left,
  numbersep=5pt,
  xleftmargin=2em,framexleftmargin=1.5em,
  captionpos=b,
  tabsize=2,
  escapeinside={(*@}{@*)}
}

% for figures & subfigures
\usepackage{graphics}
\usepackage{caption}
\usepackage{subcaption}
\graphicspath{{./figures/}}

\begin{document}
\pagenumbering{gobble}

\newcommand{\sys}{FleXR}

\newcommand{\ada}[1]{\textcolor{magenta}{AG: #1}}
\newcommand{\ketan}[1]{\textcolor{blue}{KB: #1}}
\newcommand{\jin}[1]{\textcolor{brown}{JH: #1}}

\newcommand{\red}[1]{\textcolor{red}{#1}}
\newcommand{\green}[1]{\textcolor{green}{#1}}
\newcommand{\blue}[1]{\textcolor{blue}{#1}}

\newcommand{\tlide}{$\sim$}
\newcommand{\mul}{$\times$}
\newcommand{\eg}{\emph{e.g.}}
\newcommand{\ie}{\emph{i.e.}}
\newcommand{\etc}{\emph{etc.}}
\newcommand{\etal}{\emph{et al.}}

\newenvironment{tightitemize}%
 {\begin{list}{$\bullet$}{%
 		\setlength{\leftmargin}{10pt}
        \setlength{\itemsep}{0pt}%
        \setlength{\parsep}{0pt}%
        \setlength{\topsep}{0pt}%
        \setlength{\parskip}{0pt}%
        }%
 }%
{\end{list}}



\title{FleXR: A System Enabling Flexibly Distributed Extended Reality}

\author{Jin Heo}
\affiliation{%
  \institution{Georgia Institute of Technology}
  \city{Atlanta}
  \state{Georgia}
  \country{USA}}
\email{jheo33@gatech.edu}

\author{Ketan Bhardwaj}
\affiliation{%
  \institution{Georgia Institute of Technology}
  \city{Atlanta}
  \state{Georgia}
  \country{USA}}
\email{ketanbj@gatech.edu}

\author{Ada Gavrilovska}
\affiliation{%
  \institution{Georgia Institute of Technology}
  \city{Atlanta}
  \state{Georgia}
  \country{USA}}
\email{ada@cc.gatech.edu}

\renewcommand{\shortauthors}{Jin, et al.}

\begin{abstract}

The Fast Reciprocal Square Root Algorithm is a well-established approximation technique consisting of two stages: first, a coarse approximation is obtained by manipulating the bit pattern of the floating point argument using integer instructions, and second, the coarse result is refined through one or more steps, traditionally using Newtonian iteration but alternatively using improved expressions with carefully chosen numerical constants found by other authors. The algorithm was widely used before microprocessors carried built-in hardware support for computing reciprocal square roots. At the time of writing, however, there is in general no hardware acceleration for computing other fixed fractional powers. This paper generalises the algorithm to cater to all rational powers, and to support any polynomial degree(s) in the refinement step(s), and under the assumption of unlimited floating point precision provides a procedure which automatically constructs provably optimal constants in all of these cases. It is also shown that, under certain assumptions, the use of monic refinement polynomials yields results which are much better placed with respect to the cost/accuracy tradeoff than those obtained using general polynomials. Further extensions are also analysed, and several new best approximations are given.

\end{abstract}


\begin{CCSXML}
  <ccs2012>
  <concept>
  <concept_id>10010147.10010919</concept_id>
  <concept_desc>Computing methodologies~Distributed computing methodologies</concept_desc>
  <concept_significance>500</concept_significance>
  </concept>
  <concept>
  <concept_id>10010147.10010169</concept_id>
  <concept_desc>Computing methodologies~Parallel computing methodologies</concept_desc>
  <concept_significance>500</concept_significance>
  </concept>
  <concept>
  <concept_id>10010520.10010570</concept_id>
  <concept_desc>Computer systems organization~Real-time systems</concept_desc>
  <concept_significance>300</concept_significance>
  </concept>
  </ccs2012>
\end{CCSXML}

\ccsdesc[500]{Computing methodologies~Distributed computing methodologies}
\ccsdesc[500]{Computing methodologies~Parallel computing methodologies}
\ccsdesc[300]{Computer systems organization~Real-time systems}

\keywords{distributed stream processing, extended reality, edge computing, augmented reality, virtual reality}
\maketitle
% \pagestyle{plain}

\section{Introduction}
Current quantum hardware is unable to carry out universal quantum computations due to the buildup of errors that occur during the computation. 
The magnitude of the individual error is currently above the value that the Threshold Theorem requires in order to kick-start quantum error correction and fault-tolerant quantum computation~\cite[Section 10.6]{nielsen_chuang_2010}. 
Although the experimentally achieved fidelity rates are promising and the error bounds are inching closer to the required threshold, we will have to work for the foreseeable future with quantum hardware with errors that build-up during the computation.  This implies that we can only do a limited number of steps before the output of the computation has become completely uncorrelated with the intended one.

For fault-tolerant quantum computing, we repeat four steps: 
1) We apply a number of single and two-qubit quantum gates, in parallel whenever possible; 
2) We perform a syndrome measurement on a subset of the qubits; 
3) We perform fast classical computations to determine which errors have occurred and how to correct them; 
and, 4) We apply correction terms based on the classical computations.
We then repeat these four steps with a next sequence of gates. 
These four steps are essential to fault-tolerant quantum computing. 


The starting point of this work is to use the four steps outlined above, not to carry out error correction and fault-tolerant computation, but to enhance short, constant-depth, {\em uncorrected} quantum circuits that perform single qubit gates and {\em nearest-neighbor} two qubit gates. 
Since in the long run we will have to implement error-correction and fault-tolerant computation anyhow, and this is done by such a four-step process, why not make other use of this architecture? Moreover, on some of the quantum hardware platforms, these operations are already in place.
Embracing this idea we naturally arrive at the question: what is the computational power of \textit{low-depth} quantum-classical circuits organized as in the four steps outlined above? 
We thus investigate circuits that execute a small, ideally constant, number of stages, where at each stage we may apply, in parallel, single qubit gates and {\em nearest-neighbor} two qubit gates, followed by measurements, followed by low-depth classical computations of which the outcome can control quantum gates in later stages. 
It is not clear, at first, whether such circuits, especially with constant depth, can do anything remotely useful. 
But we will see that this is indeed the case: many quantum computations can be done by such circuits in constant depth. 
By parallelizing quantum computations in this way, we improve the overall computational capabilities of these circuits, as we do not incur errors on qubits that are idle, simply because qubits are not idle for a very long time. 
Furthermore, reducing the depth of quantum circuits, at the cost of increasing width, allows the circuit to be run faster even if errors occur.

The first usage of such a four-step layout, not to do error correction, but to perform computations, can be found in the paradigm of measurement-based quantum computing~\cite{gottesman1999demonstrating,raussendorf2001one,jozsa2006introduction,clark2007generalised}: 
A universal form of quantum computing where a quantum state is prepared and operations are performed by measuring qubits in different bases, depending on previous measurements and intermediate measurements.

\citeauthor{PhamSvore2013} were the first to formalize the four-step protocol for performing computations~\cite{PhamSvore2013}. They included specific hardware topologies by considering two-dimensional graphs for imposing constraints on qubit interactions. In their model, they develop circuits for particularly useful multi-qubit gates, including specifying costs in the width, number of qubits, depth, number of concurrent time steps, size, and total number of non-Identity operations.
As a result, they find an algorithm that factors integers in polylogarithmic depth.
\citeauthor{Browne:2011} showed that the main tool in the work by \citeauthor{PhamSvore2013}, the fan-out gate, can also be replaced by additional log-depth classical computations in the measurement-based quantum computing setting~\cite{Browne:2011}.

More recently, \citeauthor{Cirac:2021} introduced a scheme to implement unitary operations involving quantum circuits combined with Local Operations and Classical Communication ($\mathsf{LOCC}$) channels: $\mathsf{LOCC}$-assisted quantum circuits~\cite{Cirac:2021}. Similarly to the four-step scheme we just described, they allow for a short depth circuit to be run on the qubits, followed by one round of $\mathsf{LOCC}$, in which ancilla qubits are measured and local unitaries are applied based on the measurement outcomes. They show that in this model any 1D transitionally invariant matrix-product state (MPS) with fixed bond dimension is in the same phase of matter as the trivial state. Similar ideas can be found in~\cite{TVV_NonAbelianTopologicalOrder_2022, tantivasadakarn2021long}.

In this work, we introduce a new model, called \textit{Local Alternating Quantum-Classical Computations} ($\LAQCC$). In this model we alternate between running quantum circuits (constrained by locality), ending in the measurement of a subset of qubits, and fast classical computations based on the measurement results. The outcome of the classical computations are then used to control future quantum circuits. We allow for flexibility in this model, by giving different constraints to the power of both the quantum circuits and the classical circuits as well as the number of alternations between them. 
Most attention will be given to $\LAQCC$ containing quantum circuits of constant depth, classical circuits of logarithmic depth and at most a constant number of alternations between them. 
Any circuit constructed in this model is considered to be of constant depth. 
We restrict ourselves to logarithmic depth classical computations, as this is the first natural and non-trivial extension beyond constant-depth classical computations. 
Constant-depth classical computations do however also have an equivalent constant-depth quantum implementation.

The definition of $\LAQCC$ sharpens the original definition of \citeauthor{PhamSvore2013} by adding constraints to the intermediate classical computations. This allows us to bound the power of $\LAQCC$ from above. 

The main result of \citeauthor{Cirac:2021}, that 1D translational invariant MPS with fixed bond dimension can be prepared by $\mathsf{LOCC}$-assisted circuits, relies on local symmetries of the MPS. These symmetries allow them to prepare local states (on a constant number of qubits) and glue them together by doing one round of the appropriate entangling measurement and corrections, after which they run a round of local unitaries to get the desired result. This general scheme for preparing states that exhibit an MPS description with the appropriate local symmetries requires only geometrically local unitaries and one round of measurement and corrections an therefore is accessible in $\LAQCC$. Studying different local symmetries, known as Symmetry Protected Topological (SPT) phases of matter, to find measurement-based constant depth circuits for states is a broad ongoing field of research~\cite{TVV_NonAbelianTopologicalOrder_2022, tantivasadakarn2021long, smith2023deterministic}. 
All these schemes have a $\LAQCC$ implementation.

%$\LAQCC$-circuits also exist for general schemes of preparing local states, based on the local tensors, and gluing them together using one round of entangled measurement and corrections, based on the local symmetry. 
%The main result of \citeauthor{Cirac:2021}, that 1D translational invariant MPS with fixed bond dimension can be prepared by $\mathsf{LOCC}$-assisted circuits, relies heavily on local symmetries of the MPS and as a result also has an equivalent $\LAQCC$ implementation. 
%The corrections applied after the measurement round are local unitaries depending on the local symmetries of the MPS. 

 

%This general scheme of preparing local states, based on the local tensors, and gluing it together by doing one round of entangled measurement and corrections, based on the local symmetry, is accessible in $\LAQCC$.
Note however that \citeauthor{Cirac:2021} also suggest a circuit for the $W$-state.
This circuit uses sequentially and dependent measurement-based corrections of the ancilla qubits. 
These dependent measurements translate to sequential alternations between the quantum and classical circuits and therefore increase the total depth to linear depth, exceeding the constant-depth constraints imposed by $\LAQCC$-circuits. 

We study the power of the $\LAQCC$ model with respect to state preparation, showing that even with only constant quantum-depth and logarithmic classical depth it remains possible to prepare states with long-range entanglement.
Another surprising result is that it is unlikely that $\LAQCC$ circuits are classically simulatable. We show that any instantaneous quantum polynomial-time (IQP) circuit~\cite{Bremner2010,Shepherd2009} has an $\LAQCC$ implementation.
Classical simulation of IQP circuits implies the collapse of the polynomial hierarchy to the third level, which is not believed to be true~\cite{Bremner2017}. Therefore, we expect that $\LAQCC$ circuits are unlikely to be classically simulatable. We bound the power of $\LAQCC$ by showing that it is contained in $\QNC^1$, the class of polynomial-size, log-depth circuits.

Next, we also study the power that intermediate classical calculations can add to quantum computations, by considering a new model that alternates between polynomially many polynomial-depth quantum circuits and unbounded classical computations
We study this model by doing a complexity theoretical analysis, where we draw inspiration from the notions of complexity given by \citeauthor{RosenthalYuen:2022}, \citeauthor{MetgerYuen:2023}, and \citeauthor{Aaronson:2004}.
All three complexity notions are based on the notion of state preparation, instead of more traditional definition of complexity such as the decidability of a computational problem. 
The first two consider classes based on sequences of quantum states preparable by a polynomial-sized quantum circuit, where the circuits are uniformly generated by a computational class, for instance, the class $\mathsf{PSPACE}$, which results in the complexity class $\mathsf{StatePSPACE}$~\cite{RosenthalYuen:2022,MetgerYuen:2023}.
The third notion considers a relative complexity, where the complexity is measured between two given states, and is measured by the number of gates, from a given gate-set, required to transform one state in another state~\cite{Aaronson:2004}. 
For our definition of state preparation complexity, we drop the uniformity constraint from~\cite{RosenthalYuen:2022,MetgerYuen:2023} and define a class as $\mathsf{StateX}$, which refers to states preparable by circuits of type $\mathsf{X}$. 
As an example, if $\mathsf{X} = \QNC^0$, this results in the class $\mathsf{StateQNC^0}$, which is the set of states preparable from the $\ket{0}^n$ state by poly-size constant-depth circuits. 
This notion is similar to the relative complexity from~\cite{Aaronson:2004}, where one state is the  $\ket{0}^n$ state and instead of counting the number of gates we consider the set of states preparable by a fixed number of gates. Using this notion of complexity we show that any state preparable by an $\LAQCC^*$ circuit is also preparable by a $\mathsf{PostQPoly}$ circuit, the class of circuits of polynomial depth with an additional post-selection gate. 

All Clifford circuits have a constant-depth $\LAQCC$ implementation, implying that any stabilizer state can be implemented by a constant-depth $\LAQCC$ circuit, see Section~\ref{sec:clifford_circuits} for a proof of this statement. 
Efficient circuits for stabilizer states have been known already through measurement-based quantum computing. Therefore this paper focuses on the preparation of non-stabilizer states, and as a surprising result we find novel constant-depth protocols for four very natural classes of non-stabilizer states.
Despite the extensive research into these four classes of non-stabilizer states and the many applications of them, no efficient constant- or low-depth state preparation protocols are known yet. We specifically consider these four classes as they are all often used as initial states in other algorithms.

The first state is a uniform superposition over an arbitrary number of states. 
This state finds applications in many quantum algorithms, as they often start with a uniform superposition over multiple states. 
This superposition is often achieved by applying Hadamard gates to every qubit due to its simplicity to prepare. 
Yet, the analysis of many algorithms, such as Shor's algorithm~\cite{Shor:1997}, would benefit from a different initial superposition. 
The circuit to prepare the uniform superposition over an arbitrary number of states uses an exact version of Grover search as a subroutine, that turns a probabilistic circuit, with a known constant probability of success, into a deterministic circuit. 
We use the circuit for preparing a uniform superposition over an arbitrary number of states as a subroutine in the next two quantum state preparation protocols. 

The second state is the $W$-state, the uniform superposition over all computational basis states of Hamming-weight~$1$, a natural long-ranged entangled state that displays a fundamentally nonequivalent type of entanglement from the Greenberger–Horne–Zeilinger state~\cite{WState:2000}, for which $\LAQCC$-type constant-depth circuits were previously known~\cite{PhamSvore2013, Cirac:2021}. 
The $W$-state is often used as benchmark for new quantum hardware~\cite{Haffner2005,Neeley2010,GarciaPerez:2021}. 
A novel way to prepare the $W$-state therefore gives a new way to benchmark different quantum devices with each other. 
A circuit for preparing the $W$-state was given in~\cite{Cirac:2021}, but this implementation requires sequentially alternating measurements followed by local unitaries, which in the $\LAQCC$ model is not considered to be of constant depth. 
We improve this protocol by giving an $\LAQCC$ implementation of the $W$-state, based on a compress-uncompress method that links the one-hot and binary encoding of integers.

The third state considered is the Dicke state, a generalization of the $W$-state, a superposition over all computational basis states with Hamming-weight $k$~\cite{Dicke:1954}. 
Dicke states have relevance in various practical settings.
For instance, for quantum game theory~\cite{zdemir2007}, quantum storage~\cite{Bacon_Compress:2006,Plesch:2010}, quantum error correction~\cite{ouyang2014permutation}, quantum metrology~\cite{toth2012multipartite}, and quantum networking~\cite{prevedel2009experimental}. 
Dicke states have been used as a starting state for variational optimization algorithms, most notably Quantum Alternating Operator Ansatz (QAOA)~\cite{Hadfield2019}, to find solutions to problems such as Maximum k-vertex Cover~\cite{Brandhofer2022,cook2020quantum}.
The ground states of physical Hamiltonians describing one-dimensional chains tend to show a resemblance to Dicke states such as states resulting from the Bethe ansatz, making them an ideal starting state when investigating the ground state behavior of these Hamiltonians~\cite{TDL_BetheAnsatzDerivation:2010,B_ExcitedStateQuantumPhaseTransitions:2013,DickeTransitions:2021}. 
For instance, the algorithm by \citeauthor{van2021preparing}, who give an algorithm to prepare the Bethe ansatz eigenstates of the spin-1/2 XXZ spin chain, starts by first preparing a Dicke state~\cite{van2021preparing}. 
A Dicke-state preparation protocol based on the compress-uncompress methodology used in the $W$-state furthermore finds applications in entanglement distillation, where the entanglement of a large state is concentrated on only a few qubits. 
Efficient deterministic circuits for preparing Dicke states have been proposed by \citeauthor{bartschi2019deterministic}~\cite{bartschi2019deterministic, bartschi2022deterministic_short_depth}. 
They provide a quantum circuit of depth $\mathO(k \log(\frac{n}{k}))$, allowing arbitrary connectivity, to prepare a Dicke state, which they conjecture to be optimal when $k$ is constant. 
In this work, we provide a constant-depth $\LAQCC$ circuit below their conjectured bound already for constant $k$. 
However, this does not directly disprove their conjecture, as we allow for intermediate measurements and classical computations. 
More significantly, we even construct constant-depth $\LAQCC$ circuits for $k = \mathO(\sqrt{n})$ greatly improving their bound.
This construction extends the compress-uncompress method for the $W$-state combined with additional subroutines. 

We continue with a log-depth state preparation protocol for the Dicke-state for arbitrary $k$. 
This protocol implements an efficient transformation between the factoradic number representation and the combinatorial number representation of a positive integer. 
The combinatorial number representation relates directly to the Dicke state. 
The provided efficient transformation between number representation systems might be of independent interest. 

We conclude by modifying our protocol for preparing a Dicke-state to a protocol that prepares quantum many-body scar states in constant-depth. 
These states have low entanglement and longer coherence times than states with similar energy density.
These characteristics make many-body scar states interesting to analyze and relevant within physics.
Many-body scar states appear for instance in the AKLT model~\cite{AKLT:1987,MRBAR:2018,MRB:2018} and different spin models~\cite{SI:2019,MOBFR:2020}.
Known methods for preparing these states have polynomial-depth~\cite{Gustafson:2023}, whereas our circuit has constant depth. 

% We conclude by studying the power that intermediate classical calculations can add to quantum computations. 
% In this study, we define a new model that relaxes constant-depth quantum circuits to polynomial depth quantum circuits, log-depth classical calculations to unbounded classical computations and a constant number of alternations to a polynomial number of alternations. 
% We call this model $\LAQCC^*$. 
% We study this model by doing a complexity theoretical analysis, where we draw inspiration from the notions of complexity given by \citeauthor{RosenthalYuen:2022}, \citeauthor{MetgerYuen:2023}, and \citeauthor{Aaronson:2004}.
% All three complexity notions are based on the notion of state preparation, instead of more traditional definition of complexity such as the decidability of a computational problem. 
% The first two consider classes based on sequences of quantum states preparable by a polynomial-sized quantum circuit, where the circuits are uniformly generated by a computational class, for instance, the class $\mathsf{PSPACE}$, which results in the complexity class $\mathsf{StatePSPACE}$~\cite{RosenthalYuen:2022,MetgerYuen:2023}.
% The third notion considers a relative complexity, where the complexity is measured between two given states, and is measured by the number of gates, from a given gate-set, required to transform one state in another state~\cite{Aaronson:2004}. 
% For our definition of state preparation complexity, we drop the uniformity constraint from~\cite{RosenthalYuen:2022,MetgerYuen:2023} and define a class as $\mathsf{StateX}$, which refers to states preparable by circuits of type $\mathsf{X}$. 
% As an example, if $\mathsf{X} = \QNC^0$, this results in the class $\mathsf{StateQNC^0}$, which is the set of states preparable from the $\ket{0}^n$ state by poly-size constant-depth circuits. 
% This notion is similar to the relative complexity from~\cite{Aaronson:2004}, where one state is the  $\ket{0}^n$ state and instead of counting the number of gates we consider the set of states preparable by a fixed number of gates. Using this notion of complexity we show that any state preparable by an $\LAQCC^*$ circuit is also preparable by a $\mathsf{PostQPoly}$ circuit, the class of circuits of polynomial depth with an additional post-selection gate. 

\paragraph{Summary of results}
\begin{itemize}
    \item We give a new definition of a computational model that captures the power of the four step process: applying a constant number of layers of one- and two-qubit gates; performing a syndrome measurement; perform a fast classical computation determining corrections; apply corrections. We call this model \emph{Local Alternating Quantum Classical Computations}, or $\LAQCC$ for short. In this model we bound the allowed quantum operations, intermediate classical calculations, and number of rounds separately. In Section~\ref{sec:LAQCC_model} we define this model and give a list of operations based on results from literature contained in this computational model. In some of these operations we explicitly use that we allow for multiple, but at most constant, rounds  of corrections.
    \item  We show show that there exist $\LAQCC$ circuits that can not be weakly simulated in Section~\ref{sec:IQP_in_LAQCC}. We further show that for every $\LAQCC$ circuit there exists a $\QNC^1$ circuit simulating it perfectly, in Section~\ref{sec:LAQCC_in_QNC1}.
    \item We introduce a new type computational complexity for preparing states and show that the extension of $\LAQCC$ where we allow a polynomial number of rounds and unbounded classical computation, is contained in $\mathsf{PostQPoly}$, the class of polynomial circuits with post-selection, in Section~\ref{sec:Complexity results}.
    \item We show a protocol to prepare the uniform superposition state of size $q$ in $\LAQCC$ using $\mathO(\ceil{\log_2(q)}^2)$ qubits in Section~\ref{sec:superposition_modulo_q}. 
    \item We show a protocol to prepare the $W_n$ state in $\LAQCC$ using $\mathO(n\log(n))$ qubits in Section~\ref{sec:W_state_in_LAQCC}.
    \item We show two ways of preparing the Dicke-$(n,k)$ state. The first method is in $\LAQCC$, works up to $k = \mathO(\sqrt{n})$, uses $\mathO(n^2\log(n))$ qubits, and is found in Section~\ref{sec:dicke:small_k}. The second method is in $\LAQCC\text{-}\mathsf{LOG}$ (an extension of $\LAQCC$ allowing for logarithmic number of alterations instead of constant), works for any $k$, uses $\mathO(\text{poly}(n))$ qubits, and is found in Section~\ref{sec:Dicke_in_LAQCC_LOG}. 
    \item We extend on our $\LAQCC$ method of generating Dicke-$(n,k)$ states for $k = \mathO(\sqrt{n})$ and show a protocol to generate many-body scar states for a particular Hamiltonian in $\LAQCC$ (Section~\ref{sec:many_body_scar}). 
\end{itemize}
Summarized in a table, we provide the following state generation protocols:
\begin{table}[htb]
\centering
\begin{tabular}{l|l|l|l}
\textbf{State description} & \textbf{Width} & \textbf{Depth} & \textbf{Implementation}\\
\hline 
Uniform superposition mod $q$: $\frac{1}{\sqrt{q}} \sum_{i = 0}^{q-1}\ket{i}$ & $\mathO(\ceil{\log^2 q})$ & $\mathO(1)$ & Section~\ref{sec:superposition_modulo_q}\\

$W$-state: $\frac{1}{\sqrt{n}}\sum_{i = 0}^{n-1}\ket{e_i}$ & $\mathO(n \log n)$ & $\mathO(1)$ & Section~\ref{sec:W_state_in_LAQCC}\\

Dicke-$(n,k)$, $k = \mathO(\sqrt{n})$: $\binom{n}{k}^{-1/2}\sum_{x \in \{0,1\}^n: |x| = k} \ket{x}$ &  $\mathO(n^2\log n)$ & $\mathO(1)$ 
&Section~\ref{sec:dicke:small_k}\\

Dicke-$(n,k)$: $\binom{n}{k}^{-1/2}\sum_{x \in \{0,1\}^n: |x| = k} \ket{x}$ & $\mathO(\text{poly}(n))$ & $\mathO(\log n)$ &Section~\ref{sec:Dicke_in_LAQCC_LOG}\\

QMBS: $\ket{S_k} = \frac{1}{k! \sqrt{\mathcal N(n,k)}}(Q^\dagger)^k \ket{\Omega}$ &  $\mathO(n^2\log n)$ & $\mathO(1)$  &  Section~\ref{sec:many_body_scar}
\end{tabular}
\caption{Summary of state preparation protocols given in this paper.}
\label{tab:sate_prep}
\end{table}
In the entry for the quantum many-body scar state $Q$ denotes the raising operator and $\mathcal N(n,k)=\binom{n-k-1}{k}$. 
Section~\ref{sec:many_body_scar} will provide more details on the variables and the implementation. 

\paragraph{Organization of the paper}
\noindent We first introduce relevant preliminaries in Section~\ref{sec:preliminaries}. 
In Section~\ref{sec:LAQCC_model} we formally define the class of Local Alternating Quantum-Classical Computations ($\LAQCC$). We also show that any Clifford circuit can be implemented in constant depth $\LAQCC$ (a result based on a result from measurement-based quantum computing~\cite{jozsa2006introduction}). 
This result allows us to give many useful multi-qubit gates and routines in Section~\ref{sec:gates_created_in_LAQCC}. 
Beyond that we show that constant depth $\LAQCC$ circuits are contained in $\QNC^1$ and that any $\mathsf{IQP}$ circuit has an $\LAQCC$ implementation.
We conclude this section with an analysis of a more powerful instantiation of $\LAQCC$ and show an inclusion with respect to the class $\mathsf{PostQPoly}$, which is the class of circuits of polynomial depth with one additional post-selection gate. 
In Section~\ref{sec:state_prep_in_LAQCC} we give $\LAQCC$ circuit implementations for preparing the uniform superposition over an arbitrary number of states, the $W$-state and the Dicke state up to $k = \mathO(\sqrt{n})$. We furthermore give a log-depth circuit implementation for preparing the Dicke state for any $k$. We conclude by showing a $\LAQCC$ circuit for generating many body scar states of a particular type of Hamiltonian.


\section{Related Work}
%\subsection{Cost Volume based Deep Stereo Matching}
%Stereo matching is a typical problem that has been studied for decades and a well-known four-step pipeline \cite{scharstein2002taxonomy} has been established, where cost volume construction is an indispensable step. Current state-of-the-art stereo matching methods are all cost volume based methods and they can be categorized into two types. Typically, a cost volume is a 4D tensor of height, width, disparity, and features. The first category just uses a full correlation to generate a single-feature cost volume. Such methods are usually efficient but lose much information because of the decimation of feature channels. Many previous work, including Dispnet \cite{dispnet}, MADNet \cite{madnet}, IResNet \cite{iresnet} and AANet \cite{aanet}, belong to this category. The second category usually uses concatenation \cite{gcnet} or group-wise correlation \cite{gwcnet} to generate a multi-feature 4D cost volume. Such a method can achieve better performance while requiring higher computational complexity and memory consumption. Actually, a majority of the top-performing networks in public leaderboards belong to this category, such as GANet \cite{ganet}, CSPN \cite{cspn} and ACFNet \cite{acfnet}. These methods generally employ multiple 3D convolution layers to constantly regularize the 4D cost volume and then apply softmax over the disparity dimension to produce a discrete disparity probability distribution. The final predicted disparity is obtained by softly weighting indices according to their probability, which is also called soft argmin in GCNet \cite{gcnet}. However, soft argmin leaves the output susceptible to multi-modal disparity probability distributions. ACFNet \cite{acfnet} observes this problem and proposes to directly supervise the cost volume with unimodal ground truth distributions. In contrast, we define an uncertainty estimation to quantify the degree to which the cost volume tends to be multi-modal distribution, higher implies the higher possibility of estimation error.

\subsection{Multi-scale Cost Volume based Stereo Matching}
Cost volume construction is an indispensable step in the well-known four-step pipeline for stereo matching \cite{scharstein2002taxonomy, pamisurvey1, pamisurvey2}. Typically, current state-of-the-art stereo matching methods can be categorized into two types of cost volume-based methods, where the cost volume is a 4D tensor of height, width, disparity, and features. The first category usually uses the single-feature 3D cost volume generated by full correlation, which is efficient while losing much information due to the decimation of feature channels. Many real-time methods, such as Dispnet \cite{dispnet}, MADNet \cite{madnet, madnet_pami} and AANet \cite{aanet}, belongs to the category. Moreover, two-stage refinement \cite{mcvmfc} and pyramidal towers \cite{madnet} are commonly applied in the single-feature cost volume based network to construct multi-scale cost volume. The second category usually uses the multi-feature 4D cost volume generated by concatenation \cite{gcnet} or group-wise correlation \cite{gwcnet}, which can achieve better performance with higher computational complexity and memory consumption. Most top-performing networks, including GANet \cite{ganet}, CSPN \cite{cspn} and ACFNet \cite{acfnet} belong to this category. 
% In these methods, the 4D cost volume is constantly regularized by multiple 3D convolution layers and then a discrete disparity probability distribution can be produced by softmax. Next, the final predicted disparity can be obtained by softly weighting indices according to their probability \cite{gcnet}. However, such output is susceptible to multimodal disparity probability distributions and ACFNet \cite{acfnet} gives a solution by directly supervising the cost volume with unimodal ground truth distributions to alleviate this problem. 
Recently, to alleviate the high computational complexity and memory consumption when employing multi-feature 4D cost volumes, \cite{cvpmvsnet, cascade, uscnet} propose to use cascade cost volume representation in multi-view stereo. These methods usually first predict an initial disparity at the coarsest resolution of the image and then gradually refine the disparity by narrowing down the disparity search space. More closely related to our approach is Casstereo \cite{cascade}, which first extended such representation to stereo matching. It selected to uniform sample a pre-defined range to generate the next stage’s disparity search range. Instead, we employ pixel-level uncertainty estimation to adaptively adjust the next stage disparity searching range and generate pseudo-labels for subsequent domain adaptation. Our method also shares similarities with UCSNet \cite{uscnet}, which constructs uncertainty-aware cost volume in multi-view stereo while it doesn’t employ uncertainty estimation to generate pseudo-labels.

%\subsection{Multi-scale Cost Volume based Deep Stereo Matching} 
% \subsection{Multi-scale Cost Volume based Stereo Matching} 
%Multi-scale cost volume firstly was applied in the single-feature cost volume based network with the form of two-stage refinement \cite{mcvmfc} and pyramidal towers \cite{madnet}. Recently, cascade cost volume representation \cite{cvpmvsnet, cascade, uscnet} was proposed in multi-view stereo to alleviate the high computational complexity and memory consumption when employing multi-feature 4D cost volumes. These methods generally predict an initial disparity at the coarsest resolution of the image. Then, they will narrow down the disparity search space and gradually refine the disparity. More closely related to our approach is Casstereo \cite{cascade}, which first extended such representation to stereo matching. It selected to uniform sample a pre-defined range to generate the next stage’s disparity search range. Instead, we employ uncertainty estimation to adaptively adjust the next stage pixel-level disparity searching range and push the next stage's cost volume to be predominantly unimodal.

% The single-feature cost volume based network with the form of two-stage refinement \cite{mcvmfc} and pyramidal towers \cite{madnet} first employ multi-scale cost volume for stereo matching. Recently, to alleviate the high computational complexity and memory consumption when employing multi-feature 4D cost volumes, \cite{cvpmvsnet, cascade, uscnet} propose to use cascade cost volume representation in multi-view stereo, which generally predict an initial disparity at the coarsest resolution of the image. Then, the disparity search space is narrowed down and the disparity is gradually refined. More closely related to our approach is Casstereo \cite{cascade}, which first extended such representation to stereo matching. It selected to uniform sample a pre-defined range to generate the next stage’s disparity search range. Instead, we employ uncertainty estimation to adaptively adjust the next stage pixel-level disparity searching range and push the next stage's cost volume to be predominantly unimodal.

% Figure environment removed

\subsection{Robust Stereo Matching} 
There exist three categories of generalization definitions for robust stereo matching. 1) Cross-domain Generalization: the network’s ability to perform well on unseen scenes (cannot see the image pairs of the target domain in advance). Towards this end, Jia et al \cite{sungeneralizaiton} propose to incorporate scene geometry priors into an end-to-end network. Zhang et al \cite{dsmnet} introduce a domain normalization and a trainable non-local graph-based filter to construct a domain-invariant stereo matching network. 2) Adapt Generalization: the network’s ability to adapt pre-trained models to the new domain with unlabeled target data. Previous work usually pre-trains the models on synthetic data and then adapts it to new target domains with Graph Laplacian regularization \cite{zoom}, non-adversarial progressive color transfer \cite{adastereo}, and Knowledge Reverse Distillation \cite{aohnet}. More closely related to our approach are \cite{aohnet, unsuperviseddomainadaptation} in stereo matching and Monoresmatch \cite{monoresmatch} in monocular depth estimation, which also proposes to generate a pseudo-label for domain adaptation. However, these methods all select to employ classical stereo matching methods \cite{sgm} alongside with confidence estimators, e.g., left-right consistency check to generate pseudo-labels. That is all these methods need an independent method to generate corresponding pseudo-labels. Instead, the proposed method is an end-to-end network that can generate the predicted disparity map, corresponding uncertainty map and pseudo-labels jointly, which is a more simple, yet efficient way. 
% Instead, our proposed method can employ pixel-level and area-level uncertainty estimation to self-distill the predicted disparity maps of our pre-training model and generate sparse while reliable pseudo-labels to align the domain gap, which is a more simple, yet efficient way. 
3) Joint Generalization: the network’s ability to perform well on a variety of datasets with the same model parameters. MCV-MFC \cite{mcvmfc} introduces a two-stage finetuning scheme to achieve a good trade-off between generalization and fitting capability on multiple datasets. However, it doesn’t touch the inner difference between diverse datasets, e.g, the unbalanced disparity distribution. To further address this problem, we propose a cascade cost volume to adaptively the next stage disparity searching space, where the pixel-level uncertainty estimation is at the core.

% \subsection{Monocular Depth Estimation}
% Monocular depth estimation aims to estimate depth values from a single image, instead of stereo images or multiple frames in a video. This problem is ill-posed because of the ambiguity of object sizes. However, humans could estimate the depth from a single image with prior knowledge of the scenes. Recently, learning based methods were explored to learn depth values by supervised or unsupervised learning. Eigen et al. first employed Convolutional Neural Networks (CNN) to predict depth in a coarse-to-fine manner and further improved its performance by multi-task learning. Liu et al. presented deep convolutional neural fields model by combining deep model with continuous CRF. Li et al. [22] refined deep CNN outputs with a hierarchical CRF. Multi-scale continuous CRF was formulated into a deep sequential network by Xu et al. [45] to refine depth estimation. Unsupervised methods tried to train monocular depth estimation with stereo
% image pairs or image sequences and test on single images. Garg et al. [9] used novel image view synthesis loss to train a depth estimation network in an unsupervised way. Godard et al. [11] introduced left-right consistency regularization to improve the performance of view synthesis loss. Recently, some work also propose to use the stereo matching network as a proxy to learn depth from synthetic data or directly employ traditional stereo matching methods to distill proxies labels from the target domain, which proves the feasibility of distilling stereo matching networks to learn monocular depth estimation.



%------------------------------------------------------------------------------
\section{Design Challenges for \sys}
\label{sec:streamprocessing}
%------------------------------------------------------------------------------
There are intuitive advantages of building an XR system as an SP system.
An XR system processes inputs as data streams from device sensors such as cameras, inertial measurement units, \etc, and provides output as data streams such as field of view content in VR and graphic overlay in AR.
The SP design provides benefits of high throughput with pipeline parallelism and distributed computation with its modularity, but the application of SP in XR presents non-trivial issues.
In this section, we present the design space exploration we conducted to address those issues when building \sys.

\subsection{Issues with Stream Processing for XR}
\label{sec:spissues}
We use an example AR pipeline in Figure~\ref{fig:semantics}, to articulate the issues when applying SP to an XR system.
The application renderer overlays virtual objects on the camera frame based on the result from the object detector.
The objects can be manipulated using key inputs, and the AR scene with the overlaid objects is displayed to the end user.
Using this pipeline, we discuss the main issues below.

\noindent\textbf{\bf I1: Communication Cost. }
As shown in Figure \ref{fig:semantics}, there are a number of components in the pipeline across which data transmissions must occur.
Those data transfers need to be performed with least latency because high latency lowers the application's responsiveness and causes discrepancies between the real and virtual worlds.
However, the SP design increases the end-to-end latency as it requires data movement across the ports of the pipelined kernels~\cite{khare2019linearize, basanta2017patterns}.
Since XR functionalities process and produce large multimedia data, the overhead of the cross-kernel communications is significant and must be addressed in order to meet the latency constraints of XR.

% Figure environment removed

\noindent\textbf{\bf I2: Communication Semantics. }
In Figure~\ref{fig:semantics}, the downstream kernels, \ie, the renderer, has input dependencies with the upstream kernels.
Some of those dependencies are hard, \ie, an input must be received for the downstream kernel to execute, as is the case with the camera and renderer.
Other dependencies are soft, meaning an input is not required from the particular upstream kernel, as is the case with the keyboard or detector and renderer.
This property has implications on whether the execution of the upstream kernels should be synchronized with the downstream kernels in terms of their invocation frequencies.
In short, the type of dependency is specific to the functionalities that are connected, and this semantic information must be expressed by the application and adequately handled by the underlying system via  support for appropriate communication semantics.

\noindent\textbf{\bf I3: Data Recency. }
If the camera frames in Figure~\ref{fig:semantics} are delayed due to queuing delays when the frame data is transmitted, its freshness decreases.
As a result, the placement of the AR object would be off, thus lowering the quality of the AR experience, which is also established by prior work that stale data deteriorates the quality of XR experiences~\cite{li2020towards}.
Generally, as data is transmitted across kernels of different frequencies and execution times, it may result in queuing delays if the data is queued at any of the port buffers in a pipeline.
When the data contains  real-world contexts from sensors, \eg, camera frames, it is critical to ensure that it remains fresh with all pipeline components that process it. Thus, the SP system for XR must provide a way to manage data recency.


\subsection{Design Decisions for The Issues}
\label{sec:designdecisions}
\sys\ as a specialized DSP system for XR, incorporates solutions to address the issues raised in the previous section.

\noindent\textbf{\bf D1: Efficient Local Communication. }
The remote communication cost is unavoidable even with data compression, but the local communication should be efficient for low overheads.
We evaluated the suitability of several existing SP libraries in terms of their communication costs: RaftLib~\cite{beard2017raftlib}, GStreamer~\cite{gstreamer}, Python pipeline libraries~\cite{mpipe, pypeln}, and the robot operating system (ROS)~\cite{quigley2009ros}.

\begin{table}[]
  \caption{\label{tab:localcommunication} \small The local communication latencies between two kernels in milliseconds.}
  \resizebox{0.8\linewidth}{!}{
    \begin{tabular}{|l|c|c|c|c|}
      \hline
      \diagbox{Libraries}{Resolution} & 720p & 1080p & 1440p & 2160p \\ \hline
      ROS Pub/Sub\cite{quigley2009ros}    & 3.4           & 6.9            & 7.1            & 12.5           \\ \hline
      ROS Shm Pub/Sub \cite{rosshm}       & 2.2           & 4.3            & 5.9            & 10.2           \\ \hline
      Python Queue \cite{mpipe, pypeln}   & 14.3          & 24.1           & 30.4           & 52.1           \\ \hline
      Python Pipe \cite{mpipe, pypeln}    & 9.3           & 17.1           & 29.5           & 52.1           \\ \hline
      Python Shm \cite{pythonshm}         & 3.0           & 8.6            & 14.8           & 32.3           \\ \hline
      \textbf{GStreamer} \cite{gstreamer}          & 0.1           & 0.1            & 0.1            & 0.1            \\ \hline
      \textbf{RaftLib} \cite{beard2017raftlib}  & 0.1   & 0.1            & 0.1            & 0.1            \\ \hline
    \end{tabular}
  }
\end{table}

We measure the communication costs with two locally connected kernels and raw RGB frames of different resolutions.
Table~\ref{tab:localcommunication} shows the transmission latencies of the frames.
ROS and Python libraries provide process-level SP, where each kernel runs as a separate process and the processes communicate %with others
via interprocess communication (IPC) channels.
Based on our results, the local communication in process-level SP is hardly efficient for large multimedia data even with shared memory channels~\cite{pythonshm, rosshm}.
While the shared memory channel reduces the number of data copies, the data still needs to be copied between the shared memory and process memory regions.

GStreamer and RaftLib provide thread-level SP with zero-copy communication ports.
As kernel functions are threads in the same address space, local communication can be done without copy.
\emph{A DSP system for XR should leverage a  thread-level SP for efficient local communication of collocated kernels, and extend its communication with  support for remote communication.}

\noindent\textbf{\bf D2: Blocking and Non-blocking Semantics. }
%Distributed XR pipelines have kernels with different execution times and dependencies.
To handle the hard and soft dependencies and synchronize the kernel executions in a pipeline, providing
the proper communication semantics (blocking and non-blocking) for the local and
remote communication primitives (send and receive) is  essential~\cite{cypher1994semantics}.
The design of \sys\ handles this as a first-order concern when executing XR pipelines.

The send semantics of an output port is for synchronizing the kernel execution.
A blocking send blocks the execution of an upstream kernel function when the downstream kernel's queue is full,
and this backpressure leads to flow control and implicitly synchronizes the upstream to downstream kernels.
For non-blocking semantics, the upstream kernel continues when the downstream kernel cannot receive data on its input port. %take the message.
The output port requires both blocking and non-blocking send semantics in XR pipelines.
In the example AR pipeline of Figure~\ref{fig:semantics}, if only blocking semantics are supported, the camera kernel is synchronized to the longest path of object detection (blue line).
Even if the app renderer does not require the results from the object detector for every camera frame, the frame stream (green line) is blocked by the object detector.

The receive semantics of an input port is for kernel dependencies.
A blocking receive waits for the message from a port, and a non-blocking continues when there is no message. % from the port.
So, when the kernel is written, the primary inputs on which the kernel %functionality
depends (\eg, camera frame for a kernel performing frame processing) should be specified with blocking semantics.
For inputs generated by other sources (\eg, other sensors or user events), which can impact or steer the kernel processing but are only optionally used, the semantics should be with non-blocking to handle them.

When only a blocking receive is available, all input streams of a kernel are forced as mandatory.
The kernel execution and pipeline throughput are restricted by the lowest frequency input.
In Figure~\ref{fig:semantics}, the renderer is blocked until the key input arrives from the user (red line).
Even without the key input, the renderer execution is governed by the object detector, and the pipeline
throughput is limited by the path with the highest latency (blue line).

\emph{Supporting both blocking and non-blocking primitives for the input and output ports makes it possible
to correctly describe stream dependencies and synchronize kernel executions in XR pipelines.}

\noindent\textbf{\bf D3: Queuing Management and Network Protocols. }
Since poor data freshness causes discrepancies between the real and virtual worlds, it is crucial to manage data recency in XR pipelines.
% Recency management
This can be achieved by minimizing the queuing delays of a pipeline~\cite{little1961proof}.

For local communication, it is possible to bound the queuing delay by limiting the number of outstanding data entries %\st{messages}
in the port buffer.
For remote communication,  recency management becomes challenging because there is no way to control the queuing mechanisms of unknown
middleboxes across the backend network.
In this situation, recency management can be enabled by compromising communication reliability.
Reliable network protocols such as TCP~\cite{stevens1997tcp} guarantee in-order message delivery via retransmission and acknowledgment mechanisms.
However, in cases where recent data is prioritized (\eg, the object detection result on a live camera frame), the reliable protocols are inappropriate, and should be replaced with protocols favoring data timeliness over reliability even with data loss, \eg, RTP~\cite{schulzrinne2003rfc3550} and RTSP~\cite{schulzrinne1998real} over UDP.

Thus, \emph{the DSP system for XR should provide knobs for  data recency management via queue size management and support for multiple network protocols for local and remote kernel communications.}

%------------------------------------------------------------------------------
\section{\sys}
\label{sec:flexr}
%------------------------------------------------------------------------------

\subsection{Overview}
\label{sec:flexroverview}
To bring flexibility to XR workload distribution, we built \sys\ as a DSP system specialized for XR, taking the design benefits of modularity and task parallelism.
Driven by our design decisions, \sys\ is built on top of a thread-level SP library, RaftLib~\cite{beard2017raftlib}, and provides the benefit of efficient local communication for the collocated kernels ({\bf D1}).
For the communication semantics to enable kernel synchronization and dependencies of XR pipelines, we extend the semantics of the RaftLib port with support for non-blocking and for remote communication ({\bf D2}).
For recency management of local and remote communication, \sys\ allows setting the maximum number of messages in the local port buffer and specifying network protocols for remote ports at runtime ({\bf D3}).

Even with the necessary DSP features for XR, there are still issues about providing these features properly to the system stakeholders: developers writing kernels and users requesting distributed pipelines with given kernels.
\sys\ enables flexibility in configuring XR pipelines via its kernel abstraction with interfaces separating development- and deployment-time concerns.
While the developer implements an XR kernel function and knows its input dependencies, the user creates a distributed XR pipeline and configures the connectivity of kernels (local or remote), output-port semantics, and recency management mechanism of the pipeline context.
In addition, there can be a case where the user needs to connect an output of a kernel into multiple downstream kernels.

Our kernel abstraction provides the interfaces allowing the developer to register input and output ports and use the registered ports in the kernel function regardless of how they will be configured by a user.
The behavior of the registered ports becomes different based on the user-specified communication attributes at runtime.
The user can also branch dynamically an output port with different communication attributes and flexibly create distributed pipelines with various topologies without modifying the kernels. %codes.

% Figure environment removed

% Figure environment removed



Figure~\ref{fig:flexr_overview} shows the high-level design of \sys\ and how it operates.
\circled{1} The developers write kernels with our kernel abstractions for implementing their XR functionalities or incorporating existing functionality implementations by wrapping them in kernel functions.
\circled{2} With given kernels, the user requests a distributed pipeline as a YAML recipe describing pipelined kernels and their communication attributes.
\circled{3} The recipe parser parses the recipe and generates pipeline metadata of the kernels and connection information.
\circled{4} The local pipeline metadata is passed to the pipeline manager, and it creates a local pipeline by instantiating the kernels and configuring port connections.
\circled{5} The part of the recipe about the remote pipeline is sent to the request listener on the server.
\circled{6} The server's recipe parser parses the received recipe and generates pipeline metadata. If the pipeline works with external applications, it starts the applications.
\circled{7} Then, the server's pipeline manager also creates the server-side pipeline, and the remote ports of local and remote pipelines are connected.
\circled{8} The pipelines, distributed across the client and server, run with dataflow.


%%%%%%%%%%%%%%%%%%%%%%%%%%%%%%%%%%%%%%%%%%%%%%%%%%%%%%%%%%%%%%%%%%%%%%%%%%%%%%%%%%%%%%%%%%%%%%%%%%%%%%%%%%%%%%%%%%%%%%%
\subsection{Kernel Design}
\label{sec:flexrkernel}
In SP, the compute kernel is a pipeline component, which includes an execution function and communication ports.
To provide the necessary DSP features for flexible configuration of the communication attributes, we design a \sys\ kernel with two abstractions: the \sys\ port and port manager.
They abstract the different communication channels for local and remote operation of a \sys\ kernel, allowing the developer to write kernels without specifying the communication attributes and the user to configure the connection at runtime without modifying kernels.
A developer registers the input and output ports of the kernel with tags via the port manager interface, and the registered ports are instantiated and configured by the port manager based on the user recipe.



Figure~\ref{fig:kernel_design} shows our kernel design.
Each kernel has its ID, logger, frequency manager, execution function, and port manager.
The ID is used for the recipe parser and pipeline manager in Figure~\ref{fig:flexr_overview}.
The frequency manager adjusts the execution frequency when a kernel should run at a stable frequency, and the logger is for the developer to log the kernel events.
The execution function processes data from the input ports and sends out the result to the output port.


\noindent\textbf{Port Manager.}
When kernels are instantiated, the port manager of each kernel activates and dynamically branches the \sys\ ports based on the pipeline metadata.
In addition, it provides developers with  interfaces to use the \sys\ ports without considering how the ports will be configured by a user.

The port manager design is shown in Figure~\ref{fig:kernel_design}.
The manager has input and output port maps.
These port maps have the mapping information of the port tags registered by a developer and the \sys\ ports activated with the communication attributes by a user.
A kernel function can get inputs and send outputs via the port manager interfaces with the tag.
The branched port map contains the ports branched from the registered output port.
When a registered output port needs to be connected into multiple downstream ports with different communication attributes, the port manager activates the branched ports and keeps their mapping information to the registered port.
When a kernel function sends an output to the registered port, it is also sent through the branched ports by using this mapping information.

Listing~\ref{lst:kernelcode} shows the codes of an example kernel which a developer implements (\circled{1} in Figure~\ref{fig:flexr_overview}).
In Line~\ref{portS}-\ref{portE}, the developer registers the input and output ports with the tags.
The registered ports are used in Line~\ref{getS}-\ref{sendS} without specifying their connection types and branching states.
The port manager hides the complexities of using the dynamically instantiated ports from developers.


\begin{lstlisting}[caption={\small An example kernel with two input and one output ports registered by a developer.}, captionpos=b, label={lst:kernelcode}, float=h]
class ExampleKernel: public FleXRKernel {
  public:
    ExampleKernel() {
      portManager.registerInPortTag("in1", PortSemantics::BLOCKING);(*@\label{portS}@*)
      portManager.registerInPortTag("in2", PortSemantics::NONBLOCKING);(*@\label{portS2}@*)
      portManager.registerOutPortTag("out");(*@\label{portE}@*)
    }

    raft::kstatus run() {
      MsgType *in1=portManager.getInput<MsgType>("in1");(*@\label{getS}@*)
      MsgType *in2=portManager.getInput<MsgType>("in2");(*@\label{getS2}@*)
      MsgType *out=portManager.getOutputPlaceholder<MsgType>("out");(*@\label{getE}@*)

      /* Kernel Functionality ... */

      portManager.sendOutput("out", out);(*@\label{sendS}@*)
    }
}
\end{lstlisting}

\noindent\textbf{\sys\ Port } abstracts different local and remote communication ports and exposes a unified interface to the port manager.
When the pipeline manager creates a pipeline (\circled{4} and \circled{7} in Figure~\ref{fig:flexr_overview}), a kernel is instantiated and its ports are configured by the port manager with user-specified port connectivity, semantics, and recency management mechanism.
Since these communication attributes are determined by the contexts of the requested pipeline, the operation of a \sys\ port should differ based on the attributes given at runtime.
We design the \sys\ port abstraction as a state machine with the integrated interfaces.

The design of a \sys\ port is shown in Figure~\ref{fig:kernel_design}.
Each \sys\ port has the port semantics, connection state, and local and remote ports.
The port semantics is for specifying the communication semantics: blocking and non-blocking.
The connection state indicates whether it is local or remote, and the network protocol for the remote.
The local and remote ports are the actual communication channels internally used and interfaced by the \sys\ port abstraction.
Since the \sys\ port is an abstraction for different communication ports, it is extensible.
New network protocols and local channels can be seamlessly integrated into distributed XR pipelines.

Listing~\ref{lst:yamlrecipe} is part of an example pipeline recipe which a user provides (\circled{2} in Figure~\ref{fig:flexr_overview}).
The user creates a pipeline by specifying the kernels, their port attributes in Line 5, 7-8, 11-12, and 14-15 and connections in Line 17-22.
When the pipeline is created, the port manager activates the \sys\ port.
The activation instantiates an underlying channel corresponding to the specified attributes, and the channel is interfaced via the \sys\ port. % under the hood.
The \sys\ port provides uniform interfaces to the port manager while
behaving differently based on the underlying channel.


\begin{lstlisting}[caption={\small A part of the pipeline recipe for the example kernel in Listing~\ref{lst:kernelcode} and a connection.}, language=yaml, escapechar=|, captionpos=b, label={lst:yamlrecipe}, float=h]
- kernel   : ExampleKernel
  id       : example_kernel1
  input    :
    - port_name: in1
      connection_type: local
    - port_name: in2
      connection_type: remote
      remote_info: [RTP, 14802]
  output   :
    - port_name: out
      connection_type: local
      semantics: blocking
    - port_name: branched_out
      connection_type: remote
      remote_info    : [127.0.0.1, 14805, TCP]
      branched_from: out
- local_connections:
  - send_kernel: example_kernel1
    send_port_name: out
    recv_kernel: example_kernel2
    recv_port_name: input
    queue_size: 1
\end{lstlisting}


%%%%%%%%%%%%%%%%%%%%%%%%%%%%%%%%%%%%%%%%%%%%%%%%%%%%%%%%%%%%%%%%%%%%%%%%%%%%%%%%%%%%%%%%%%%%%%%%%%%%%%%%%%%%%%%%%%%%%%%
\subsection{Communication Semantics and Data Recency Management}
\label{sec:flexrcommunication}
To express the relationships among kernels and their dependencies and synchronization requirements, both blocking and non-blocking semantics are necessary for the local and remote communication primitives.
\sys\ supports the required semantics.
The local communication in \sys\ is based on the RaftLib port.
Since the send and receive primitives of the vanilla RaftLib port are only with blocking semantics, we extend them with non-blocking semantics by checking the queue buffer of the connected RaftLib ports.
A non-blocking send does not wait and continues when the queue connected to the downstream kernel is full.
A non-blocking receive continues without waiting when the queue to the upstream kernel is empty.
For remote communications, the socket and protocol implementations have interfaces with different semantics, and we map the underlying port interfaces to the \sys\ port.

The data recency management mechanism in \sys\ is to prevent  data from aging in the pipeline queues.
For local, \sys\ provides  recency management by limiting the number of messages in the queue buffer, which puts a bound on the maximum queuing delay~\cite{little1961proof}.
The recency management for remote communication is done by supporting  different network protocols, currently supporting TCP and RTP over UDP.
For TCP connection, the in-order and reliable delivery may lead to lower data timeliness due to its retransmission and acknowledgment mechanisms.
RTP over UDP has the advantage for data recency at the cost of data loss.
By supporting these different protocols and queue size management, the recency management is achieved for remote and local communications.

%%%%%%%%%%%%%%%%%%%%%%%%%%%%%%%%%%%%%%%%%%%%%%%%%%%%%%%%%%%%%%%%%%%%%%%%%%%%%%%%%%%%%%%%%%%%%%%%%%%%%%%%%%%%%%%%%%%%%%%
\subsection{Register-Activation Interface and Port-level Configuration}
\label{sec:flexrportlevel}
We embody the necessary DSP features for XR in the \sys\ kernel, but these features should be provided properly to the stakeholders for supporting the runtime flexibility in distributed XR pipelines.
The kernel developers know the input dependencies of their kernel functions, but it is unknown to them how their kernels are used in a pipeline which a user creates.
When requesting a pipeline, the user arranges the pipeline structure with the kernel communication attributes.
So, the user determines how the kernels operate within the pipeline.
To provide the features to the proper stakeholder, \sys\ has  register-activation interfaces of the port manager at a port granularity, which streamline the development and deployment phases but clearly separate the features provided to each phase.

Based on the information available to the development and deployment phases, we identify the proper stakeholder for each feature and make the interfaces expose it.
Table~\ref{tab:intfeat} summarizes the provided interfaces to each stakeholder.
The developers register ports and set the input-port dependencies as they know the kernel functionalities.
The connection type, branching outstream, output semantics, and recency management are specified by the user recipe because these attributes should be configured when the port manager activates \sys\ ports by the metadata of the user pipeline.

The register-activation interfaces are enabled by the \sys\ port and port manager, and the stakeholders use the \sys\ features through them.
When the developer registers the ports, the semantics of input ports are set via the port manager as shown in Line~\ref{portS}-\ref{portS2} of Listing~\ref{lst:kernelcode}.
The connection types, recency management, and output semantics are specified by the user recipe as shown in Listing~\ref{lst:yamlrecipe}.
The user can branch a single registered port with separate attributes in Line 13-16 of Listing~\ref{lst:yamlrecipe}.
When the pipeline manager instantiates the pipeline kernels, the user-specified attributes and branching are set for each port by the port manager.

\begin{table}[]
  \centering
  \caption{\label{tab:intfeat}\small \sys\ interface availability for the stakeholders to manipulate the features to resolve the DSP issues in \S\ref{sec:streamprocessing}.}
  \resizebox{0.8\linewidth}{!}{
    \tiny
    \begin{tabular}{|l|c|c|}
      \hline
      \diagbox{Feature}{Stakeholder}   & Developer   & User \\ \hline
      Port registration             & \ding{51}   &             \\ \hline
      Port activation               &             & \ding{51}   \\ \hline
      Output branching              &             & \ding{51}   \\ \hline
      Input semantics               & \ding{51}   &             \\ \hline
      Output semantics              &             & \ding{51}   \\ \hline
      Recency management            &             & \ding{51}   \\ \hline
    \end{tabular}
  }
\end{table}

%-------------------------------------------------------------------------------
\section{Implementation} \label{imple}
%-------------------------------------------------------------------------------

% Figure environment removed

\sys's implementation consists of (\romannumerber{1}) a fully-functional \sys switch prototype which implements the overall in-fabric logic; and (\romannumerber{2}) a set of software APIs exposed to applications. Our prototype is built upon a FPGA-assisted commodity switch while a P4 programmable switch implementation is also provided.

%\sys's implementation consists of (\romannumerber{1}) a FPGA-based fully-functional prototype which implements the self-defined switch logic and (\romannumerber{2}) a set of software APIs exposed to multicast applications. \sys's switch logic can also be implemented on the Tofino P4 switch, as described in $\S$\ref{dis}.

\parab{FPGA-based prototype.} We implement the group registration, data packet duplication, header modification, and feedback aggregation logics on an FPGA board. The board is equipped with a commodity FPGA chip~\cite{ultrascale} and four 100Gbps Ethernet interfaces. The FPGA resource utilization is shown in Table~\ref{tab:overhead}. We build our testbed with the FPGA board, a commodity Ethernet switch, and four servers, as illustrated in Fig.~\ref{fig:fpgaprototype}. Each server is equipped with a commodity RNIC. The FPGA board and four RNICs are connected to the commodity switch through 100Gbps Ethernet interfaces. 

The commodity switch is configured by Access Control List (ACL) to route the servers' multicast traffic to the FPGA board. The FPGA board identifies the multicast data (ACK\footnote{In Fig.~\ref{fig:fpgaprototype}, we use ACK to represent all types of feedback.}) packets through the specific packet header by \textit{Parser} and \textit{Arbiter}. The data (ACK) packets will be duplicated (aggregated) by \textit{Duplicator} (\textit{ACK Aggregator}). The resulting packets will be pushed in \textit{Queue System}, waiting for the \textit{Multiplexer} to schedule in case for queue competition. Finally, the duplicated data (aggregated ACK) packets are sent back to the commodity switch. During processing, the \textit{Multicast Forwarding Table} is accessed when needed. %\todo{describe Fig. 8} Note those resulting packets' destination IP would be unicast IPs, so the switch routes them as normal unicast packets.
%

\parab{P4-based implementation.}
\sys in-fabric logic can be implemented on the P4 switch as well with some special handling. For the one-to-many data forwarding, P4 switch duplicates packets at the Traffic Manager (TM). The extended table states in Fig.~\ref{fig:table} are stored in the egress pipeline, indexed by <GroupID, EgressPort>. Because the lookup key in P4 switch is at most 32bits, we can use the least significant 24bits of GroupIP plus the 8bits port number as the real index. 

For the many-to-one feedback aggregation, there are two challenges due to the limited capability of P4 switch. Commodity P4 switch switch contains many stages, each with minimal computation capability and independent memory. Firstly, a single stage cannot support the \textit{wrapped-around} PSN comparison. To handle it, we simplify the standard PSN comparison to match the stage's capability, resulting in a tighter PSN space reduced from $2^{23}$ to $2^{22}$. Secondly, a single stage cannot iterate the entire table entries and find the minimum PSN. To handle it, we leverage multiple stages, each responsible for partial entries. Thus, the maximum number of entries supported in each multicast group is limited by the total stage number.

Besides, the P4 switch lacks the computation capability to recalculate the Invariant Cyclic Redundancy Checksum (ICRC) for the modified (aggregated) data (ACK) packets. As a result, those packets would violate the ICRC validation and be discarded at the receiver. This is why we choose the FPGA-based prototype to evaluate \sys in this work. However, a recent work shows that some RNICs provide the ability to bypass ICRC validation~\cite{switchML}. 
%Although this can work, it's a compromised method having security risks. 
%So we select the more integrated FPGA-based prototype to evaluate in $\S$\ref{eva}. Other operators can choose their preferred implementation based on their requirements.

%\begin{algorithm}[t]
%	\caption{Update PSN record and find PSN minimum in P4}\label{alg:psncomp}
%	\begin{algorithmic}[1]
%		%\Function{Generation}{}
%		\State $ack.psn, ack.port\gets $ the PSN and port of ACK packet
%		\State $rec.psn, rec.port\gets$ the PSN and port recorded
%		\State $isTrigger\gets$ whether the packet is a trigger packet
%		%\State $last\_ack\_psn\gets$ last aggregated ACK's psn
%		%\State $min\_port\gets$ port with minimum $ack\_psn$ last time
%		\State min.psn = ack.psn;
%		\State \textcolor{purple}{// every stage compare the PSN}
%		\If{$ack.psn > rec.psn$ or $(ack.psn <= 24'b3fffff$ and $rec.psn >= 24'b600000)$}
%			\State min.psn = rec.psn;
%			\If{rec.port == ack.port}
%				\State rec.psn = ack.psn;
%			\EndIf
%		\EndIf
%		\State \textcolor{purple}{// last stage write the min PSN back}
%		\If{isTrigger} 
%			\State ack.psn = min.psn;
%			\State Forward ACK.
%		\EndIf
%	\end{algorithmic}
%\end{algorithm}

\parab{Software APIs.} We provide various communication libraries and middleboxes for \sys multicast support. Take the commonly-used OpenMPI as an example, we modify the OpenMPI (v4.1.1)~\cite{openmpi} and UCX (v2.3)~\cite{ucx} to adapt to \sys's design, as shown in Fig.~\ref{fig:fpgaprototype}. Specifically, we add a new implementation of $MPI\_Bcast$ and modify UCX for multicast QPs creation and data transmission. When the new $MPI\_Bcast$ is called, the MPI process calls the UCX to establish QPs for multicast. Multicast members exchange their QPs information, and the handshake starts, as described in Appendix \ref{apx:regis}. Once the multicast group is successfully established, the UCX finally calls the RDMA primitives defined in the well-known \textit{libibverbs}~\cite{libibverbs} to transmit data. The software modifications at the end-host are transparent to the upper-layer applications and don't require any RNIC or RDMA driver modification.
%\parab{Coalescence of unicast and multicast}
%When we design \sys, there is a question in our mind: \textit{which is better, maintaining unicast and multicast transports separately at end-host, or utilizing the in-network support to enabling them to match the same transport?} Because of the long-standing resource limit in RNICs and the emeging trend of shifting appropriate computation task to programmability network, we believe the latter is the correct selection.

\begin{table}[t]
	\small
    \centering
%	\begin{center}
%    \begin{tabular}{l|c|c|c}
    \begin{tabular}{|p{0.2\linewidth}|p{0.18\linewidth}|p{0.18\linewidth}|p{0.18\linewidth}|}
    \hline
    \textbf{Resource} & \hfil \textbf{LUT} & \hfil \textbf{Register} & \hfil \textbf{BRAM} \\
    \hline
   	\textbf{Usage} & \hfil 53169 & \hfil 15391 & \hfil 188 \\
    \hline
    \end{tabular}
%    \end{center}
    \caption{Resource usage of the \sys in-fabric logic.}
    \label{tab:overhead}
    \vspace{-0.25cm}
\end{table}

%\parab{Resource overhead}  Note that the size of multicast forwarding table is determined by the number of ports of the switch and doesn't scale up with the multicast group size. 2.7MB memory can support upto 1K multicast groups, which is satisfied in datacenter. We provide a detailed calculation of the maxmum group support in Appendix \ref{apx:cal}. 
\section{Evaluation} \label{sec:evaluation}

\begin{table*}[tbp]
\centering
\small
\begin{tabular}{cccccccccc}
\toprule
& \multicolumn{3}{c}{\msr} & \multicolumn{3}{c}{\negc} & \multicolumn{3}{c}{\wsj} \\
& Acc. & F1 & wF1 & Acc. & F1 & wF1 & Acc. & F1 & wF1 \\ \cmidrule(lr){2-4} \cmidrule(lr){5-7} \cmidrule(lr){8-10} 
\udel & 66.86 & 56.76 & 64.3 & \textbf{80.80} & 55.45 & 77.9 & 63.74 & 64.23 & 63.2 \\
\icsi & \underline{71.19} & 64.73 & 70.4 & 80.36 & 64.53 & \underline{78.6} & 64.62 & 64.15 & 63.4 \\
\cnts & 68.59 & 61.39 & 67.2 & 78.68 & 61.62 & 76.8 & 64.31 & 64.59 & 64.4 \\
\osu & 68.02 & 60.28 & 66.6 & 79.24 & 57.04 & 76.5 & 69.20 & 69.63 & 68.9 \\
\isg & 67.05 & 58.83 & 65.3 & 77.34 & 59.52 & 75.6 & 69.15 & 69.35 & 69.2 \\ \midrule
\bert & \textbf{71.68} & \underline{66.70} & \textbf{71.4} & 77.79 & \underline{72.87} & 77.7 & \underline{80.95} & \underline{80.93} & \underline{80.9} \\
\roberta & 70.91 & \textbf{67.53} & \underline{70.7} & \textbf{80.80} & \textbf{77.29} & \textbf{80.7} & \textbf{82.61} & \textbf{82.70} & \textbf{82.6} \\ \midrule
Average & 69.19 & 62.32 & 67.99 & 79.29 & 64.05 & 77.69 & 70.65 & 70.80 & 70.37 \\
\bottomrule
\end{tabular}
\caption{\label{tab:performance} Overall accuracy (Acc.), macro-averaged F1 (F1), and weighted-macro F1 (wF1) scores of the algorithms depicted in Section~\ref{sec:algorithm}. For instance, \msr-\udel refers to a C5.0 classifier trained on the \msr~corpus, using the feature set mentioned in \citet{greenbacker-mccoy-2009-udel}.}
%Its Acc., F1 and wF1 of this model are 66.86, 56.76, and 64.3, respectively.}
\end{table*}


In this section, we introduce the evaluation protocol and report the performance of the models.

\subsection{Implementation Details} \label{sec:implementation}

For \bert and \roberta, we used \textit{bert-base-cased} and \textit{roberta-base}, both from Hugging Face. For fine-tuning, we set the batch size to 16, the learning rate to 1e-3, the dropout rate to 0.5, and the size of the output layer to 256. We ran each model for 20 epochs and used the one that achieved the highest F1 score on the development set. The implementation details of the classic ML-based models can be found in Appendix~\ref{sec:appendixML}.

\subsection{Evaluation Protocol} \label{sec:protocol}

The main evaluation metric in the GREC-MSR shared tasks was accuracy. 
In addition to accuracy, we also report macro-F1 and weighted-macro F1. We argue that different metrics evaluate algorithms from different perspectives and provide us with different meaningful insights. 
For pragmatic tasks like REG, it makes sense to ask how well an algorithm performs on naturally distributed data which is often imbalanced. For these cases, reporting accuracy and weighted F1 are logical. 
Furthermore, analogous to other classification tasks, minority categories should not be overlooked. Take as an example the class \emph{description} in the \negc corpus, which occurs only 4\%. If a model fails to produce this class, the produced document might sound unnatural. Therefore, it is important to ensure that an algorithm is not over- or under-generating certain classes. Looking into accuracy and macro-F1 together provides insights into such cases.

\subsection{Performance of the Models}\label{subsec:overallacc}

The overall accuracy of the models, their macro F1, and their weighted-macro F1 are presented in Table \ref{tab:performance}. 
We also present the ranking of the models based on these scores in Appendix~\ref{sec:app_rank}. 


\paragraph{PLM-based Models.} The best-performing models across all corpora and metrics are PLM-based models.  In six out of nine rankings, \bert and \roberta are ranked as the top two models. The sole exception is \negc, where \bert is the second worst model. The benefit of using PLMs is the largest on the \wsj corpus. For example, \roberta improves the macro F1 score from 69.63 (i.e., the performance of the best ML-based model) to 82.70.


\paragraph{ML-based Models.} In contrast to the robust performance of the PLM models, the performance of the classic ML models is more corpus-dependent. In the case of \msr and \negc, \icsi is the best-performing model, while in the case of \wsj, it is at the bottom section of the rankings. Another interesting observation is the performance of the \udel models. In terms of accuracy, \udel has the highest performance in \negc, while it has the lowest performance in both \msr and \wsj. In terms of macro-F1 rankings, the \negc \udel model dropped from first to last place, whereas \bert improved from penultimate place to second place. In general, our ML models yielded lower scores than the original models used in the GREC study \citep{belz2009generating}. This could be attributed to a variety of factors, including differences in feature engineering and model parameters.

\paragraph{Comparing Different Metrics.} 

Upon comparing average scores across the three metrics, we observe that for \msr and \negc, PLMs are clear winners only when macro-F1 is the metric in question. However, for \wsj, PLMs are winners on all three metrics. This may be because the distribution of categories in \wsj is much more balanced than in the other two corpora.
\section{Discussion}
\label{sec: discussion}
\kmsdelete{In this work} We study \kmsreplace{Fairness-Aware PAC learning}{Fair-ERM} in the malicious noise model, and  in some cases allow 
the learner to maintain optimal overall accuracy despite the signal in Group $B$ being almost entirely washed out.
%when we allow learners to use the
%$\PQ$ randomized expansion of the hypothesis class $\mathcal{H}$
In particular we show that different fairness constraints have fundamentally different behavior in the presence of Malicious Noise, in terms of the amount of accuracy loss that a given level of Malicious Noise could cause a fairness-constrained learner to incur. 
The key to achieving our results, which are more optimistic than those in \cite{lampert}, is allowing for improper learners using the (P,Q)-randomized expansions of the given class $\mathcal{H}$.
%We \kmsreplace{present a picture of the}{prove upper and lower bounds on}
%accuracy loss for a range of fairness notions, given \kmsreplace{this simple randomization step.}{learning over $\PQ$.
%In general our results indicate Fair-ERM (given learning over $\PQ$) is more robust than claimed in \cite{lampert}.
The type of smoothness we create by using $\PQ$ seems to be a natural property that is likely shared by many natural hypothesis classes.

Fairness notions are motivated as a response to learned disparities when there is \kmsdelete{data corruption or} systemic error affecting \kmsdelete{the data for}
one group. 
Fairness notions are supposed to mitigate this by ruling out classifiers that have worse performance on a sub-group. 
This can peg both classifiers at a lower level of performance \kmsdelete{(e.g that the lower subgroup)} in order to \emph{motivate} \cite{hardt16} improving the data collection or labelling process to obtain more reliable performance. 
%So in \kmsreplace{some}{a} sense, sensitivity of the fairness notion to poor sub-group performance caused by malicious noise is the \textit{point} of fairness constraints! 
However, it also desirable that fairness constraints perform gracefully when subject to Malicious Noise because fairness constraints will be used in contexts where the data is unreliable and noisy and this might not be known to the learner.
This tension, exposed by our work, motivates 
%a revisiting of fairness notions from first principles approach and trying to axiomatize the 
%desired properties of a fairness intervention a la cryptography and privacy. \footnote{Work in multi-calibration \cite{multicalib} is a viable direction for this problem but it is unclear how 
%that and related notions behave with unreliable data. }
on going work studying the sensitivity level of fairness constraints. 
%If we we are to take a view, if a classifier is deployed 

\section{Conclusion and Future Work}
In this work, I design corruption-robust algorithms for the Lipschitz contextual search problem. I present the \emph{agnostic checking} technique and demonstrate its effectiveness in designing corruption-robust algorithms. There are several open problems for future research. First, in the algorithm I propose for pricing loss, the schedule for agnostic checks is fixed upfront. Can the learner design an adaptive checking schedule for the pricing loss? Second, this work assumes the learner has knowledge of the Lipschitz constant $L$. Can the learner design efficient no-regret algorithms without knowledge of $L$? 
\section*{Acknowledgment}
\label{sec:ack}
%------------------------------------------------------------------------------
We would like to thank the anonymous reviewers.
This work has been partially supported by NSF projects CCF-2217070 and CNS-1909769, the Applications Driving Architectures (ADA) Research
Center, a JUMP Center co-sponsored by SRC and DARPA, and by funding and equipment gifts from VMware and Intel.

\bibliographystyle{ACM-Reference-Format}
\bibliography{sample-base}

\end{document}
\endinput
%%
%% End of file `sample-authordraft.tex'.
