Constraint-based routing (CBR) is an active area of research. Several works in the literature have dealt with the problem of optimizing routes based on a number of link metrics such as available bandwidth, delay, jitter, packet loss, hop count or other types of costs. Based on the type and subset of metrics as well as the type of the optimization problem a number of works have been proposed. Goyal and Hjalmytsson \cite{qos-routing} tackle the bandwidth-bounded problem whereas \cite{qos-routing-inaccurate}, \cite{distributed-route} deal with the delay-bounded path search. Some of the examples in \cite{qos-multimedia} also deal with two-metric constraint optimizations like bandwidth-bounded/delay-bounded or bandwidth-optimized/delay-optimized. The Extended Bellman-Ford (EBF) and Extended Djikstra Shortest Path (EDSP) algorithms presented in \cite{qos-overview} are both tested on the bandwidth-bounded/cost-bounded problem. In the same work a number of constraint-based routing algorithms are presented along with their computational complexities. Ma and Steenkiste \cite{pathbwguarantees} propose a modified Bellman-Ford algorithm for a multi-constraint problem. The  Heuristic Multi-Constrained Optimal Path (HMCOP) algorithm proposed by Korkmaz and Krunz in \cite{multi-constrained} also attempts to find feasible paths that satisfy a set of constraints while maintaining high utilization of network resources. More recent works have also explored QoS-routing over SDN networks. \cite{qos-prioritazation-sdn} and \cite{realizing-qos-sdn} mainly focus on the optimization of bandwidth while OpenQoS \cite{openqs} and VSDN \cite{vsdn} use the hop-count and delay as QoS costraints. AmoebaNet \cite{AmoebaNet} also uses the bandwidth constraint along with a variant of Djikstra's algorithm for path computation. Finally, QoSS \cite{qoss} uses the usual QoS metrics along with a security level of nodes to compute optimal end-to-end paths. However, this work focuses more on maximizing the throughput of applications and does not perform flow admission.

To the best of our knowledge, in the literature, there does not exist a solution for the priority and admission of flows under minimal security constraints be that in a centralized (SDN) architecture or not. 