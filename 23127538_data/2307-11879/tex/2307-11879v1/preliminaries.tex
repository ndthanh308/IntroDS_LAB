\subsection{Software Defined Networking}\label{sec:sdn}
In traditional networks, the configuration, management, and data-forwarding interfaces are distributed / located at each of the data forwarding devices in the data-plane. The data-paths in the network are the result of the configuration on each of the forwarding devices; each of the devices has a local configuration and management interface. Thus, in order to re-configure the data-paths, several devices must be re-configured; as a consequence, while re-configuring each device the network may be in an inconsistent state, the process can be error-prone and slow. As an example, assume a data-plane in a traditional network as shown in Fig.~\ref{fig:sdn_ex}. Assume all flows from $h_1$ to $h_2$ follow the data-path depicted in solid arrows ($h_1\rightarrow r_1\rightarrow r_2\rightarrow h_2$); consider the link $(r_1,r_2)$ is highly loaded. In order to re-configure some of the traffic to use an alternative data-path, for example $h_1\rightarrow r_1\rightarrow r_3\rightarrow r_2\rightarrow h_2$ (depicted in dashed arrows in Fig.~\ref{fig:sdn_ex}), the forwarding devices $r_1,r_3$, and $r_2$ must be re-configured, independently.

% Figure environment removed

SDN overcomes these limitations by separating the control and data-plane layers \cite{sdn}. With a centralized SDN controller, SDN applications can automatically re-configure the SDN data-plane in a timely manner. Furthermore, the devices in the data-plane may have different configuration protocols and interfaces (called the southbound interface), while the SDN controller has a single communication protocol (northbound interface) with the applications; thus, simplifying communication with heterogeneous and vendor-agnostic data-planes. One of the most popular southbound protocols is the OpenFlow protocol \cite{openflow}, where the controller gets notifications of new traffic entering the data-plane (a forwarding device as a switch or a router) when no matching rules are found. Finally, SDN-enabled forwarding devices steer the incoming network packets based on so-called flow rules installed by the SDN applications (through the controller). It is important to note that our approach is generic to any centralized routing management paradigm. Nonetheless, as SDN is the only widespread technology that provides the desired capabilities, we focus on SDN as an enabler technology.
 

\subsection{Definitions and notations}

First, let us consider a transportation network, where each of the links of this network has an associated link security scheme, as an example:
\begin{enumerate}[label={\arabic* -- }]
    \setcounter{enumi}{-1}
    \item No security
    \item Link layer encryption, for example, with the Encryption Control Protocol~(ECP) \cite{rfc1968}
    \item Post Quantum Cryptography~(PQC) \cite{postquantumcrypto,psc_johana}
    \item Quantum Key Distribution~(QKD) \cite{seccomqkd}
\end{enumerate}

First, we formally define the notion of security schemes and link security schemes.

\begin{definition}\label{def:secscheme} (Security Scheme) The link security schemes is a set $X$; A security scheme is an element in $X$, where a total order $\preceq$ is defined for $X$, that is for all $\alpha,\beta, \gamma\in X$:
\begin{itemize}
    \item $\alpha\preceq \alpha$ ($\preceq$ is reflexive) 
    \item $\alpha\preceq \beta$ and $\beta\preceq\gamma\implies\alpha\preceq\gamma$ ($\preceq$ is transitive)
    \item $\alpha\preceq\beta$ and $\beta\preceq\alpha\implies \alpha=\beta$ ($\preceq$ is antisymmetric)
    \item $\alpha\preceq\beta$ or $\beta\preceq\alpha$ ($\preceq$ is total)
\end{itemize}
\end{definition}

The fact that the security schemes have a defined total order is important, as there is an intrinsic sense of comparison between them. Furthermore, any two schemes can be compared and one of them is always \emph{better} than the other. Thus, for  easiness, we use the integer values associated to the schemes as in the previously presented enumeration (and the associated order relation less or equal than, denoted $\leq$). Once having the notion of link security schemes, we can proceed to define a secure network.

\begin{definition}\label{def:secnet} (Network) A secure network $N$ is a directed and weighted graph $(V,E,s)$, where:
	\begin{itemize}
			\item $V$ is a set of nodes;
			\item $E\subseteq V\times V$ is a set of edges (ordered pairs of nodes);
			\item $s: E\to X\subseteq\mathds{Z}_{+}$ is a security level function, that maps an edge to a security scheme ($x\in X$).
	\end{itemize}
\end{definition}

This network intuitively represents a resource where links have an associated level of security (0=no security, 1=ECP, 2=PostQuantum, and 3=QKD). As an example, for the network presented in Fig.~\ref{fig:ex_topo}, the model representing it, is $(V,E,s)$, where:
\begin{itemize}
    \item $V=\{N_1,N_2,N_3,N_4\}$;
    \item $E=\{(N_1,N_2),(N_1,N_3),(N_1,N_4),(N_2,N_3),(N_2,N_4),(N_3,N_4),(N_2,N_1),(N_3,N_1),(N_4,N_1),(N_3,N_2),(N_4,N_2),(N_4,N_3)\}$;
    \item $s(e)=\begin{cases}
        		0, & \text{if } e \in \{(N_4,N_1),(N_2,N_4)\}\\
        		2, & \text{if } e \in \{(N_1,N_2),(N_4,N_3),(N_1,N_3)\}\\
        		3, & \text{if } e \in \{(N_1,N_4),(N_4,N_2),(N_3,N_4)\}\\       		
				1, & \text{otherwise}
				%1, & \text{if } e \in \{(N_2,N_1),(N_2,N_3),(N_3,N_2),(N_1,N_3)\}\\
        	\end{cases}\
    $;
\end{itemize} 

% Figure environment removed

Following the traditional notions in graphs, we formally consider a path in a graph as a sequence of edges, i.e., a path $p\in E^*$. As an example, the path $N_1\mapsto N_2\mapsto N_3\mapsto N_4$ is formally the sequence of edges $(N_1,N_2)(N_2,N_3)(N_3,N_4)$. As a path is a sequence, an empty path is denoted as $\epsilon$. Likewise, the concatenation of two paths $p$ and $q$ is denoted as $p\cdot q$. A \emph{simple} path, is a path that does not traverse twice any given node; for example, the path $(N_1,N_2)(N_2,N_3)(N_3,N_1)(N_1,N_4)$ is not a simple path as it traverses twice the node $N_1$. We note that, for convenience, we denote the source of an edge $e$ as $\mathbf{src}(e)$, and its destination as $\mathbf{dst}(e)$. Likewise, we denote the $i$-th edge of a path $p$ as $p_i$. We denote the first edge of a path with the index 1. Similarly, we denote the length of a path $p$ as $|p|$. It is important to note that a \emph{valid} path respects the constraint that $\forall i\in\{1,\ldots,|p|-1\} \; \mathbf{dst}(p_i)=\mathbf{src}(p_{i+1})$, i.e., the path is \emph{connected}. Finally, we denote a path $p$ starting at a node $a$ and finishing at a node $b$ (i.e., $\mathbf{src}(p_1)=a$ and $\mathbf{dst}(p_{|p|}=b$) as $p=a\to b$.

Having the definition of a network (with security levels), we focus our attention to the flows traversing such network.

\begin{definition}\label{def:sflow} (Flow) A network flow $f$ in a network $N=(V,E,s)$ is a tuple, $(o,d,h)$. $o,d\in V$, are source / origin ($o$) and destination ($d$) nodes for the flow, and $h\in\{0,1\}^k$ is the packet header (of $k$ bits) of the packets belonging to the flow, containing the common characteristics that distinguish it (from other flows). 
\end{definition}

For convenience, hereafter for a flow $f=(o',d',h')$ we use the following notations: $o(f)=o'$, $d(f)=d'$, and $h(f)=h'$. In our problem of interest, flows have associated minimal security requirements. For that reason, we define the following objects.

\begin{definition}\label{def:secreq} (Security requirement) A minimum security requirement for the header of the packets belonging to a flow $h(f)$ is a function $\mathcal{S}: \{0,1\}^k\to X\subseteq\mathds{Z}_{+}$.
\end{definition}

Having our flow and security definitions, we proceed to formally define the problem of interest.
