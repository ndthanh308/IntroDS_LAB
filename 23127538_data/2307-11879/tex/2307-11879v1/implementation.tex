We implemented the proposed solution as an SDN service using the ONOS \cite{onos} controller. ONOS is an Open Networking Foundation~(ONF) project and one of the two most widely used open-source SDN controllers. The complete solution is comprised of two software components i) The \verb|FARSec ONOS| service, written in Java which uses the ONOS Northbound API, and ii) the \verb|FARSec engine| service written in C++ which implements the algorithm described in Section~\ref{sec:farsec}. In general, the controller receives either new flow events or link change events (e.g., change in security level), and forwards the information to the \verb|FARSec ONOS| service. The service decides on the pertinent action (e.g., change the topology information) and forwards it to the \verb|FARSec engine| service. The latter responds with the expected mapping configuration, and the \verb|FARSec ONOS| service then pushes the rules via the controller. A graphic view of the overall process is shown in Figure~\ref{fig:farsec_arch}. A detailed description of each component is further presented.

% Figure environment removed

\subsection{FARSEC Engine}\label{sec:farsec-engine}
The \verb|FARSec engine| service takes three inputs which in the current version of our software are given as CSV files. The resources input (an example in Figure~\ref{fig:res_csv_ex}) contains a list of all the active links in the topology and their corresponding security level. The requests input (an example in Figure~\ref{fig:req_csv_ex}) contains the list of all the flows that need to be distributed in the network. Finally, the Service Level Agreement~(SLA) input (an example in Figure~\ref{fig:sla_csv_ex}) contains the security requirements that need to be respected when calculating the paths. The engine computes the solution of the FARSec instance, and outputs the mapping between the flows and the associated paths.

% Figure environment removed

% Figure environment removed

% Figure environment removed

\subsection{FARSEC ONOS Service}\label{sec:farsec-service}
The resources and requests CSV files that are consumed by the \verb|FARSec engine| service for the path computation are generated reactively by the \verb|FARSEC ONOS service|. FARSEC is comprised of three components. The resources file which contains the topology is updated whenever the service receives an event from the network. Examples of events are the appearance / disappearance of a device (Open vSwitch\cite{ovs}), link up / link down or a link whose security level has changed. The Topology Listener (object from ONOS API) component that is in charge of updating the resources file and feeding it to the engine. 

A separate component called the Flow Listener reacts on flow events. Every time an Open vSwitch device receives a packet for which it does not have a rule installed on its tables it sends an OpenFlow \verb|PacketIn| \cite{openflow} message. Upon reception of a \verb|PacketIn| message by the controller, the Flow Listener component of the \verb|FARSec ONOS| service gets the packet header and infers the source and destination switches from the respective source and destination IP addresses. Packets are generated by network hosts. Every host has an attachment point to a switch in the topology which is known to the controller so the latter can infer the source / destination switches. The service then updates the request CSV file with the new flow and passes it to the \verb|FARSec engine| service for path calculation. As a reminder, Figure~\ref{fig:farsec_arch} illustrates the whole architecture. 

\subsection{SLA Component}\label{sec:sla-component}
The SLA component is a CSV file that for the current FARSec version needs to be updated manually. The SLA CSV file is a list of security requirements. Each entry contains some IP header fields (see detailed description in Table~\ref{tab:sla_fields}) that can be matched against every incoming flow, and a corresponding minimal security level requirement for the matched flows. If a flow does not match any of the SLA entries it is given the default minimum security requirement of 0. 
\begin{table*}[!htb]
	\centering
	\begin{tabular}{|c|c|c|c|}
		\hline
		\textbf{Field} & \textbf{Description} 	& \textbf{Example Values} \\ \hline
		Communication Type & The IP protocol & UDP,TCP,ICMP  \\ \hline
		Source/Destination Address & The source/destination IP address & 0.0.0.0/0, 192.168.1.4/24  \\ \hline
		DSCP & The TOS field in the IP header & 0,1,8,40,46  \\ \hline
        Source/Destination Port & The source/destination port of the flow (can also be a range) &  80, 22, 5000-5010 \\ \hline
        MinSec & The minimum security requirement & 1,2,3,4 \\ \hline
	\end{tabular}
	\caption{SLA fields}
	\label{tab:sla_fields}
\end{table*}

Consider the SLA CSV file shown in Figure~\ref{fig:sla_csv_ex}. The semantic of this file is as follows. All UDP flows from any source, destination IP and any source port that have a destination port between 5000 and 5005 have a minimum security requirement of 2. Respectively, all TCP flows from the subnet 192.168.1.0/24 to subnet 192.168.2.0/24 that have 22 as the destination port have a security requirement of 4. In both examples, DSCP is not used so it is set to zero.

\subsection{Video Demonstrator}\label{sec:demo}
At the time of this writing, the source code is not publicly available but, it can be made available for research collaborations (the interested reader can contact the corresponding author). However, in order to better showcase the overall functionality a video can be found here: \href{http://www.ailab.airbus.com/TRUSTCOM23/Frontend/farsec-demo.mp4}{\url{http://www.ailab.airbus.com/TRUSTCOM23/Frontend/farsec-demo.mp4}}.
To better explain the functionality and to easily assess the expected results, we use the simple topology presented in Figure~\ref{fig:farsec_onos_topo}. Two UDP video flows are initiated from h1 (192.168.1.1, attached to switch S1) and are destined to h3 (192.168.3.1, attached to s3). Initially all links have a security tier of 4 and the flows have the minimum security requirement of 3. We then modify the link security tiers as well as the flows' security requirements and we observe how the FARSec application correctly performs rerouting and flow admission respecting the current security requirements of the traffic.

% Figure environment removed

%% Figure environment removed

