\subsection{Experimental evaluation}
In order to showcase the effectiveness of the proposed approach, semi-random instances were generated. Indeed, in order to ``test'' different topologies at once, a so-called double star (a rooted tree with three levels) topology has been used as a base. On the second level, nodes are interconnected to their immediate neighbors, forming a so-called bus topology. Finally, in the third level, one of the nodes has a fully connected component, forming a so-called mesh network. As an example, a randomly generated topology for 11 nodes corresponds to the one shown in Figure~\ref{fig:ex_test_topo}. The choice of the security levels is also done in a semi-random manner. The security level from the root to the first level were chosen as a random integer between 0 and 30; respectively, the security levels from the second level were chosen as a random integer between 0 and 10;  finally, the links at the third level were chosen as a random between 0 and two. The number of flows is set to be high, 64 times the sum of all outgoing edge weights. With these settings, we generate the secure network and the flow instances as inputs for the proposed solution.

% Figure environment removed

\paragraph{Experimental setup.} The experiments were performed in a commodity server with 48 AMD cores, and 128MB of RAM, running a Linux 20.04 LTS. Random instances were generated ranging from two to 50 nodes. The running time for solving the FARSec problem w.r.t. to the size of the network ($|V|$) is shown in Figure~\ref{fig:farsec_rt}. Similarly, the running time for solving the FARSec problem w.r.t. to the size of the flows ($|F|$) is shown in Figure~\ref{fig:farsec_rt_f}.  As can be seen, the running time of the FARSec algorithm is quite low (and generally depends on the dominating term, which seems to be the number of flows). For generating and visualizing the random instances, a web service has been created, at the time of this writing, it is accessible through the following URL: \href{http://www.ailab.airbus.com/TRUSTCOM23/Frontend/?size=S}{\url{http://www.ailab.airbus.com/TRUSTCOM23/Frontend/?size=S}}, where ``S'' must be replaced by a size, for example for an 11 node instance \href{http://www.ailab.airbus.com/TRUSTCOM23/Frontend/?size=11}{\url{http://www.ailab.airbus.com/TRUSTCOM23/Frontend/?size=11}}.

% Figure environment removed

% Figure environment removed