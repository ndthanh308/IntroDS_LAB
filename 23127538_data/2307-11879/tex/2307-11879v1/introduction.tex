%global motivation software and nets, also new schemes per link
Traditionally, computer networks used dedicated hardware appliances, that fulfilled specific network functions (e.g., a firewall to filter hazardous network packets, or a network router to properly forward network packets to the appropriate next hop, etc.). Currently, computer networks are shifting this device-based paradigm. Network functions are provided as software packages or virtualized appliances. Moreover, networks are becoming generally virtualized and being executed in interconnected computer clusters. This provides ease of creation, modification, and general maintenance tasks. When it comes to the management of network traffic, paradigms have also been shifting in recent years. Notably, the Software Defined Networking~(SDN) paradigm allows the control of traffic to be decoupled from the forwarding devices. This allows one to define the behavior of the network by any means of software technologies. The latter opens up a plethora of possibilities when it comes to different ways to steer network traffic.

%particular motivation -- what about all the existing but, not addressed security, also 
With the rise of quantum communications, real implementations, and particularly, the Quantum Key Distribution~(QKD), new ways of securing communications are possible. Particularly, QKD allows secure communication \cite{seccomqkd} without relying in computational hardness as means of guaranteeing security; rather, QKD harnesses the physical properties of quantum mechanics guaranteeing that the channel cannot be read without producing noticeable perturbations. Additionally, the rise of quantum computation has also motivated the creation of different security schemes, notably the Post-Quantum Cryptography~(PQC). PQC is considered to be an end-to-end encryption scheme \cite{postquantumcrypto,psc_johana}, however, PQC can be used in a per-link encapsulation basis, specially for transit networks, operators, or meta-operators. On the one hand, considering the novel encryption schemes and new software-centric networks it is easy to consider a network routing that steers flows through secure paths, depending the appropriate application requirements. On the other hand,  best-effort routing has long been the standard routing paradigm for the Internet. However, the continuous growth of the traffic has called for a new paradigm that takes Quality of Service~(QoS) or Quality of Experience~(QoE) into consideration.

%the problem we solve
In this paper, we propose an algorithm that searches for feasible paths using the minimum security level of a link as the constraint. The minimum security level is a policy-imposed rather than an QoS-imposed requirement; the routing decisions have to comply with some Service Level Agreement~(SLA). Our framework works under the software defined networking paradigm (for preliminary concepts on SDN please refer to Section~\ref{sec:sdn}), where a centralized controller has a complete view of the network topology and is responsible for calculating the paths and installing the forwarding rules in the network devices. Furthermore, we assume that the network has different levels of security on its links. Links can be insecure (unencrypted) or use an encryption scheme (e.g., IPSec, QKD, PQC, etc.) that will define their security level. Every incoming flow in the network has a minimum security requirement. The solution searches for a feasible path where every link in the path has a security level higher or equal to the flow's requirement. If a path cannot be found the flow is rejected. The proposed solution runs in polynomial time w.r.t. the number of nodes and the number of flows in the networks. Simulated instances confirm the efficiency of the proposed method, and furthermore, its pertinence by assigning only paths that respect the security constraints (see Section~\ref{sec:experiments}). Furthermore, we implement the solution by coupling the proposed algorithm with a custom SDN service developed as an ONOS\cite{onos} (one of the leading open-source SDN controllers) application. The service monitors a network topology comprised of OpenFlow\cite{openflow} switches and upon reception of events (e.g. network flows, link security modified etc.) it queries the FARSec engine to calculate the appropriate secure paths (based on their minimal security requirements and the topology security levels). It subsequently creates the the calculated data paths by installing the appropriate OpenFlow rules on the respective devices. We showcase the overall functionality with a video demonstrator (see Section~\ref{sec:demo}).    %I joined the results and the particular motivation as there was few of both

%we can leave the narrated table of contents cuz we have space so far

The remainder of this paper is structured as follows. Section~\ref{sec:prelim} presents the necessary background, notations and definitions. Section~\ref{sec:farsec} formally describes the problem of interest, proposes a non-naive solution, and assesses its computational complexity, while Section~\ref{sec:experiments} presents the experimental evaluation of the proposed solution and illustrates the obtained results. Section ~\ref{sec:imp} describes an open-source implementation of the proposed approach. Section~\ref{sec:relwork} presents the related work, and finally, Section~\ref{sec:conclusion} concludes the paper, and presents future avenues.
