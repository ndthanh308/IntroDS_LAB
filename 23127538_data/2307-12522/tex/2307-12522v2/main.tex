
\documentclass[acmsmall]{acmart}

%% Fonts used in the template cannot be substituted; margin 
%% adjustments are not allowed.
%%
\AtBeginDocument{%
  \providecommand\BibTeX{{%
    \normalfont B\kern-0.5em{\scshape i\kern-0.25em b}\kern-0.8em\TeX}}}

%% Rights management information.  This information is sent to you
%% when you complete the rights form.  These commands have SAMPLE
%% values in them; it is your responsibility as an author to replace
%% the commands and values with those provided to you when you
%% complete the rights form.
\setcopyright{acmcopyright}
\copyrightyear{2023}
\acmYear{2023}
\acmDOI{10.1145/1122445.1122456}


%%
%% These commands are for a JOURNAL article.
\acmJournal{TOSEM}
\acmVolume{X}
\acmNumber{Y}
\acmArticle{1}
\acmMonth{12}


\newcommand{\chen}[1]{\textcolor{red}{#1}}
\newcommand{\fix}[1]{\textcolor{black}{#1}}{\ignorespaces}
% \newcommand{\refix}[1]{{{#1}}}




%%
%% Submission ID.
%% Use this when submitting an article to a sponsored event. You'll
%% receive a unique submission ID from the organizers
%% of the event, and this ID should be used as the parameter to this command.
%%\acmSubmissionID{123-A56-BU3}


%%
%% end of the preamble, start of the body of the document source.
\usepackage{algorithm}
\usepackage{algpseudocode}
\renewcommand{\algorithmicrequire}{\textbf{Input:}}
\renewcommand{\algorithmicensure}{\textbf{Output:}}
\algnewcommand\algorithmicforeach{\textbf{for each}}
\algdef{S}[FOR]{ForEach}[1]{\algorithmicforeach\ #1\ \algorithmicdo}
\usepackage{amsmath}
\usepackage{subfigure} 
\usepackage{multirow}
\usepackage{booktabs}
\captionsetup{belowskip=-10pt}

% Comment out second line to disable.
\newcommand{\todo}[1]{}
\renewcommand{\todo}[1]{{\color{red} TODO: {#1}}}


\setcopyright{none}
\settopmatter{printacmref=false} % Removes citation information below abstract
\renewcommand\footnotetextcopyrightpermission[1]{} % removes footnote with conference information in first column


\begin{document}

%%
%% The "title" command has an optional parameter,
%% allowing the author to define a "short title" to be used in page headers.
\title{Automated Mapping of Adaptive App GUIs from Phones to TVs}

%%
%% The "author" command and its associated commands are used to define
%% the authors and their affiliations.
%% Of note is the shared affiliation of the first two authors, and the
%% "authornote" and "authornotemark" commands
%% used to denote shared contribution to the research.
\author{Han Hu}
\email{han.hu@monash.edu}
\affiliation{%
  \institution{Monash University}
  \streetaddress{Wellington Road}
  \city{Clayton}
  \state{Victoria}
  \country{Australia}
  \postcode{3800}
}

\author{Ruiqi Dong}
\email{rdong@swin.edu.au}
\affiliation{%
  \institution{Swinburne University of Technology}
  \streetaddress{John Street}
  \city{Hawthorn}
  \state{Victoria}
  \country{Australia}
  \postcode{3122}
}

\author{John Grundy}
\email{John.Grundy@monash.edu}
\affiliation{%
  \institution{Monash University}
  \streetaddress{Wellington Road}
  \city{Clayton}
  \state{Victoria}
  \country{Australia}
  \postcode{3800}
}

\author{Thai Minh Nguyen}
\email{mngu0072@student.monash.edu}
\affiliation{%
  \institution{Monash University}
  \streetaddress{Wellington Road}
  \city{Clayton}
  \state{Victoria}
  \country{Australia}
  \postcode{3800}
}

\author{Huaxiao Liu}
\email{liuhuaxiao@jlu.edu.cn}
\affiliation{%
  \institution{Jilin University}
  \streetaddress{No.2699, Qianjin Road}
  \city{Changchun}
  \state{Jilin}
  \country{PR China}
  \postcode{130015}
}

\author{Chunyang Chen}
\email{chunyang.chen@monash.edu}
\affiliation{%
  \institution{Monash University}
  \streetaddress{Wellington Road}
  \city{Clayton}
  \state{Victoria}
  \country{Australia}
  \postcode{3800}
}


%%
%% By default, the full list of authors will be used in the page
%% headers. Often, this list is too long, and will overlap
%% other information printed in the page headers. This command allows
%% the author to define a more concise list
%% of authors' names for this purpose.
\renewcommand{\shortauthors}{Hu, et al.}

%%
%% The abstract is a short summary of the work to be presented in the
%% article.
\begin{abstract}
  With the increasing interconnection of smart devices, users often desire to adopt the same app on quite different devices for identical tasks, such as watching the same movies on both their smartphones and TVs.
  However, the significant differences in screen size, aspect ratio, and interaction styles make it challenging to adapt Graphical User Interfaces (GUIs) across these devices. Although there are millions of apps available on Google Play, only a few thousand are designed to support smart TV displays. Existing techniques to map a mobile app GUI to a TV either adopt a responsive design, which struggles to bridge the substantial gap between phone and TV or use mirror apps for improved video display, which requires hardware support and extra engineering efforts. Instead of developing another app for supporting TVs, we propose a semi-automated approach to generate corresponding adaptive TV GUIs, given the phone GUIs as the input. Based on our empirical study of GUI pairs for TVs and phones in existing apps, we synthesize a list of rules for grouping and classifying phone GUIs, converting them to TV GUIs, and generating dynamic TV layouts and source code for the TV display. Our tool is not only beneficial to developers but also to GUI designers, who can further customize the generated GUIs for their TV app development. An evaluation and user study demonstrate the accuracy of our generated GUIs and the usefulness of our tool.  
\end{abstract}

%%
%% The code below is generated by the tool at http://dl.acm.org/ccs.cfm.
%% Please copy and paste the code instead of the example below.
%%
\begin{CCSXML}
<ccs2012>
   <concept>
       <concept_id>10011007.10011074.10011092</concept_id>
       <concept_desc>Software and its engineering~Software development techniques</concept_desc>
       <concept_significance>300</concept_significance>
       </concept>
   <concept>
       <concept_id>10003120.10003121.10003122</concept_id>
       <concept_desc>Human-centered computing~HCI design and evaluation methods</concept_desc>
       <concept_significance>300</concept_significance>
       </concept>
 </ccs2012>
\end{CCSXML}

\ccsdesc[300]{Software and its engineering~Software development techniques}
\ccsdesc[300]{Human-centered computing~HCI design and evaluation methods}


%%
%% Keywords. The author(s) should pick words that accurately describe
%% the work being presented. Separate the keywords with commas.
\keywords{graphic user interface, cross-screen, adaptive GUI}


%%
%% This command processes the author and affiliation and title
%% information and builds the first part of the formatted document.
\maketitle
\section{Introduction}
Current quantum hardware is unable to carry out universal quantum computations due to the buildup of errors that occur during the computation. 
The magnitude of the individual error is currently above the value that the Threshold Theorem requires in order to kick-start quantum error correction and fault-tolerant quantum computation~\cite[Section 10.6]{nielsen_chuang_2010}. 
Although the experimentally achieved fidelity rates are promising and the error bounds are inching closer to the required threshold, we will have to work for the foreseeable future with quantum hardware with errors that build-up during the computation.  This implies that we can only do a limited number of steps before the output of the computation has become completely uncorrelated with the intended one.

For fault-tolerant quantum computing, we repeat four steps: 
1) We apply a number of single and two-qubit quantum gates, in parallel whenever possible; 
2) We perform a syndrome measurement on a subset of the qubits; 
3) We perform fast classical computations to determine which errors have occurred and how to correct them; 
and, 4) We apply correction terms based on the classical computations.
We then repeat these four steps with a next sequence of gates. 
These four steps are essential to fault-tolerant quantum computing. 


The starting point of this work is to use the four steps outlined above, not to carry out error correction and fault-tolerant computation, but to enhance short, constant-depth, {\em uncorrected} quantum circuits that perform single qubit gates and {\em nearest-neighbor} two qubit gates. 
Since in the long run we will have to implement error-correction and fault-tolerant computation anyhow, and this is done by such a four-step process, why not make other use of this architecture? Moreover, on some of the quantum hardware platforms, these operations are already in place.
Embracing this idea we naturally arrive at the question: what is the computational power of \textit{low-depth} quantum-classical circuits organized as in the four steps outlined above? 
We thus investigate circuits that execute a small, ideally constant, number of stages, where at each stage we may apply, in parallel, single qubit gates and {\em nearest-neighbor} two qubit gates, followed by measurements, followed by low-depth classical computations of which the outcome can control quantum gates in later stages. 
It is not clear, at first, whether such circuits, especially with constant depth, can do anything remotely useful. 
But we will see that this is indeed the case: many quantum computations can be done by such circuits in constant depth. 
By parallelizing quantum computations in this way, we improve the overall computational capabilities of these circuits, as we do not incur errors on qubits that are idle, simply because qubits are not idle for a very long time. 
Furthermore, reducing the depth of quantum circuits, at the cost of increasing width, allows the circuit to be run faster even if errors occur.

The first usage of such a four-step layout, not to do error correction, but to perform computations, can be found in the paradigm of measurement-based quantum computing~\cite{gottesman1999demonstrating,raussendorf2001one,jozsa2006introduction,clark2007generalised}: 
A universal form of quantum computing where a quantum state is prepared and operations are performed by measuring qubits in different bases, depending on previous measurements and intermediate measurements.

\citeauthor{PhamSvore2013} were the first to formalize the four-step protocol for performing computations~\cite{PhamSvore2013}. They included specific hardware topologies by considering two-dimensional graphs for imposing constraints on qubit interactions. In their model, they develop circuits for particularly useful multi-qubit gates, including specifying costs in the width, number of qubits, depth, number of concurrent time steps, size, and total number of non-Identity operations.
As a result, they find an algorithm that factors integers in polylogarithmic depth.
\citeauthor{Browne:2011} showed that the main tool in the work by \citeauthor{PhamSvore2013}, the fan-out gate, can also be replaced by additional log-depth classical computations in the measurement-based quantum computing setting~\cite{Browne:2011}.

More recently, \citeauthor{Cirac:2021} introduced a scheme to implement unitary operations involving quantum circuits combined with Local Operations and Classical Communication ($\mathsf{LOCC}$) channels: $\mathsf{LOCC}$-assisted quantum circuits~\cite{Cirac:2021}. Similarly to the four-step scheme we just described, they allow for a short depth circuit to be run on the qubits, followed by one round of $\mathsf{LOCC}$, in which ancilla qubits are measured and local unitaries are applied based on the measurement outcomes. They show that in this model any 1D transitionally invariant matrix-product state (MPS) with fixed bond dimension is in the same phase of matter as the trivial state. Similar ideas can be found in~\cite{TVV_NonAbelianTopologicalOrder_2022, tantivasadakarn2021long}.

In this work, we introduce a new model, called \textit{Local Alternating Quantum-Classical Computations} ($\LAQCC$). In this model we alternate between running quantum circuits (constrained by locality), ending in the measurement of a subset of qubits, and fast classical computations based on the measurement results. The outcome of the classical computations are then used to control future quantum circuits. We allow for flexibility in this model, by giving different constraints to the power of both the quantum circuits and the classical circuits as well as the number of alternations between them. 
Most attention will be given to $\LAQCC$ containing quantum circuits of constant depth, classical circuits of logarithmic depth and at most a constant number of alternations between them. 
Any circuit constructed in this model is considered to be of constant depth. 
We restrict ourselves to logarithmic depth classical computations, as this is the first natural and non-trivial extension beyond constant-depth classical computations. 
Constant-depth classical computations do however also have an equivalent constant-depth quantum implementation.

The definition of $\LAQCC$ sharpens the original definition of \citeauthor{PhamSvore2013} by adding constraints to the intermediate classical computations. This allows us to bound the power of $\LAQCC$ from above. 

The main result of \citeauthor{Cirac:2021}, that 1D translational invariant MPS with fixed bond dimension can be prepared by $\mathsf{LOCC}$-assisted circuits, relies on local symmetries of the MPS. These symmetries allow them to prepare local states (on a constant number of qubits) and glue them together by doing one round of the appropriate entangling measurement and corrections, after which they run a round of local unitaries to get the desired result. This general scheme for preparing states that exhibit an MPS description with the appropriate local symmetries requires only geometrically local unitaries and one round of measurement and corrections an therefore is accessible in $\LAQCC$. Studying different local symmetries, known as Symmetry Protected Topological (SPT) phases of matter, to find measurement-based constant depth circuits for states is a broad ongoing field of research~\cite{TVV_NonAbelianTopologicalOrder_2022, tantivasadakarn2021long, smith2023deterministic}. 
All these schemes have a $\LAQCC$ implementation.

%$\LAQCC$-circuits also exist for general schemes of preparing local states, based on the local tensors, and gluing them together using one round of entangled measurement and corrections, based on the local symmetry. 
%The main result of \citeauthor{Cirac:2021}, that 1D translational invariant MPS with fixed bond dimension can be prepared by $\mathsf{LOCC}$-assisted circuits, relies heavily on local symmetries of the MPS and as a result also has an equivalent $\LAQCC$ implementation. 
%The corrections applied after the measurement round are local unitaries depending on the local symmetries of the MPS. 

 

%This general scheme of preparing local states, based on the local tensors, and gluing it together by doing one round of entangled measurement and corrections, based on the local symmetry, is accessible in $\LAQCC$.
Note however that \citeauthor{Cirac:2021} also suggest a circuit for the $W$-state.
This circuit uses sequentially and dependent measurement-based corrections of the ancilla qubits. 
These dependent measurements translate to sequential alternations between the quantum and classical circuits and therefore increase the total depth to linear depth, exceeding the constant-depth constraints imposed by $\LAQCC$-circuits. 

We study the power of the $\LAQCC$ model with respect to state preparation, showing that even with only constant quantum-depth and logarithmic classical depth it remains possible to prepare states with long-range entanglement.
Another surprising result is that it is unlikely that $\LAQCC$ circuits are classically simulatable. We show that any instantaneous quantum polynomial-time (IQP) circuit~\cite{Bremner2010,Shepherd2009} has an $\LAQCC$ implementation.
Classical simulation of IQP circuits implies the collapse of the polynomial hierarchy to the third level, which is not believed to be true~\cite{Bremner2017}. Therefore, we expect that $\LAQCC$ circuits are unlikely to be classically simulatable. We bound the power of $\LAQCC$ by showing that it is contained in $\QNC^1$, the class of polynomial-size, log-depth circuits.

Next, we also study the power that intermediate classical calculations can add to quantum computations, by considering a new model that alternates between polynomially many polynomial-depth quantum circuits and unbounded classical computations
We study this model by doing a complexity theoretical analysis, where we draw inspiration from the notions of complexity given by \citeauthor{RosenthalYuen:2022}, \citeauthor{MetgerYuen:2023}, and \citeauthor{Aaronson:2004}.
All three complexity notions are based on the notion of state preparation, instead of more traditional definition of complexity such as the decidability of a computational problem. 
The first two consider classes based on sequences of quantum states preparable by a polynomial-sized quantum circuit, where the circuits are uniformly generated by a computational class, for instance, the class $\mathsf{PSPACE}$, which results in the complexity class $\mathsf{StatePSPACE}$~\cite{RosenthalYuen:2022,MetgerYuen:2023}.
The third notion considers a relative complexity, where the complexity is measured between two given states, and is measured by the number of gates, from a given gate-set, required to transform one state in another state~\cite{Aaronson:2004}. 
For our definition of state preparation complexity, we drop the uniformity constraint from~\cite{RosenthalYuen:2022,MetgerYuen:2023} and define a class as $\mathsf{StateX}$, which refers to states preparable by circuits of type $\mathsf{X}$. 
As an example, if $\mathsf{X} = \QNC^0$, this results in the class $\mathsf{StateQNC^0}$, which is the set of states preparable from the $\ket{0}^n$ state by poly-size constant-depth circuits. 
This notion is similar to the relative complexity from~\cite{Aaronson:2004}, where one state is the  $\ket{0}^n$ state and instead of counting the number of gates we consider the set of states preparable by a fixed number of gates. Using this notion of complexity we show that any state preparable by an $\LAQCC^*$ circuit is also preparable by a $\mathsf{PostQPoly}$ circuit, the class of circuits of polynomial depth with an additional post-selection gate. 

All Clifford circuits have a constant-depth $\LAQCC$ implementation, implying that any stabilizer state can be implemented by a constant-depth $\LAQCC$ circuit, see Section~\ref{sec:clifford_circuits} for a proof of this statement. 
Efficient circuits for stabilizer states have been known already through measurement-based quantum computing. Therefore this paper focuses on the preparation of non-stabilizer states, and as a surprising result we find novel constant-depth protocols for four very natural classes of non-stabilizer states.
Despite the extensive research into these four classes of non-stabilizer states and the many applications of them, no efficient constant- or low-depth state preparation protocols are known yet. We specifically consider these four classes as they are all often used as initial states in other algorithms.

The first state is a uniform superposition over an arbitrary number of states. 
This state finds applications in many quantum algorithms, as they often start with a uniform superposition over multiple states. 
This superposition is often achieved by applying Hadamard gates to every qubit due to its simplicity to prepare. 
Yet, the analysis of many algorithms, such as Shor's algorithm~\cite{Shor:1997}, would benefit from a different initial superposition. 
The circuit to prepare the uniform superposition over an arbitrary number of states uses an exact version of Grover search as a subroutine, that turns a probabilistic circuit, with a known constant probability of success, into a deterministic circuit. 
We use the circuit for preparing a uniform superposition over an arbitrary number of states as a subroutine in the next two quantum state preparation protocols. 

The second state is the $W$-state, the uniform superposition over all computational basis states of Hamming-weight~$1$, a natural long-ranged entangled state that displays a fundamentally nonequivalent type of entanglement from the Greenberger–Horne–Zeilinger state~\cite{WState:2000}, for which $\LAQCC$-type constant-depth circuits were previously known~\cite{PhamSvore2013, Cirac:2021}. 
The $W$-state is often used as benchmark for new quantum hardware~\cite{Haffner2005,Neeley2010,GarciaPerez:2021}. 
A novel way to prepare the $W$-state therefore gives a new way to benchmark different quantum devices with each other. 
A circuit for preparing the $W$-state was given in~\cite{Cirac:2021}, but this implementation requires sequentially alternating measurements followed by local unitaries, which in the $\LAQCC$ model is not considered to be of constant depth. 
We improve this protocol by giving an $\LAQCC$ implementation of the $W$-state, based on a compress-uncompress method that links the one-hot and binary encoding of integers.

The third state considered is the Dicke state, a generalization of the $W$-state, a superposition over all computational basis states with Hamming-weight $k$~\cite{Dicke:1954}. 
Dicke states have relevance in various practical settings.
For instance, for quantum game theory~\cite{zdemir2007}, quantum storage~\cite{Bacon_Compress:2006,Plesch:2010}, quantum error correction~\cite{ouyang2014permutation}, quantum metrology~\cite{toth2012multipartite}, and quantum networking~\cite{prevedel2009experimental}. 
Dicke states have been used as a starting state for variational optimization algorithms, most notably Quantum Alternating Operator Ansatz (QAOA)~\cite{Hadfield2019}, to find solutions to problems such as Maximum k-vertex Cover~\cite{Brandhofer2022,cook2020quantum}.
The ground states of physical Hamiltonians describing one-dimensional chains tend to show a resemblance to Dicke states such as states resulting from the Bethe ansatz, making them an ideal starting state when investigating the ground state behavior of these Hamiltonians~\cite{TDL_BetheAnsatzDerivation:2010,B_ExcitedStateQuantumPhaseTransitions:2013,DickeTransitions:2021}. 
For instance, the algorithm by \citeauthor{van2021preparing}, who give an algorithm to prepare the Bethe ansatz eigenstates of the spin-1/2 XXZ spin chain, starts by first preparing a Dicke state~\cite{van2021preparing}. 
A Dicke-state preparation protocol based on the compress-uncompress methodology used in the $W$-state furthermore finds applications in entanglement distillation, where the entanglement of a large state is concentrated on only a few qubits. 
Efficient deterministic circuits for preparing Dicke states have been proposed by \citeauthor{bartschi2019deterministic}~\cite{bartschi2019deterministic, bartschi2022deterministic_short_depth}. 
They provide a quantum circuit of depth $\mathO(k \log(\frac{n}{k}))$, allowing arbitrary connectivity, to prepare a Dicke state, which they conjecture to be optimal when $k$ is constant. 
In this work, we provide a constant-depth $\LAQCC$ circuit below their conjectured bound already for constant $k$. 
However, this does not directly disprove their conjecture, as we allow for intermediate measurements and classical computations. 
More significantly, we even construct constant-depth $\LAQCC$ circuits for $k = \mathO(\sqrt{n})$ greatly improving their bound.
This construction extends the compress-uncompress method for the $W$-state combined with additional subroutines. 

We continue with a log-depth state preparation protocol for the Dicke-state for arbitrary $k$. 
This protocol implements an efficient transformation between the factoradic number representation and the combinatorial number representation of a positive integer. 
The combinatorial number representation relates directly to the Dicke state. 
The provided efficient transformation between number representation systems might be of independent interest. 

We conclude by modifying our protocol for preparing a Dicke-state to a protocol that prepares quantum many-body scar states in constant-depth. 
These states have low entanglement and longer coherence times than states with similar energy density.
These characteristics make many-body scar states interesting to analyze and relevant within physics.
Many-body scar states appear for instance in the AKLT model~\cite{AKLT:1987,MRBAR:2018,MRB:2018} and different spin models~\cite{SI:2019,MOBFR:2020}.
Known methods for preparing these states have polynomial-depth~\cite{Gustafson:2023}, whereas our circuit has constant depth. 

% We conclude by studying the power that intermediate classical calculations can add to quantum computations. 
% In this study, we define a new model that relaxes constant-depth quantum circuits to polynomial depth quantum circuits, log-depth classical calculations to unbounded classical computations and a constant number of alternations to a polynomial number of alternations. 
% We call this model $\LAQCC^*$. 
% We study this model by doing a complexity theoretical analysis, where we draw inspiration from the notions of complexity given by \citeauthor{RosenthalYuen:2022}, \citeauthor{MetgerYuen:2023}, and \citeauthor{Aaronson:2004}.
% All three complexity notions are based on the notion of state preparation, instead of more traditional definition of complexity such as the decidability of a computational problem. 
% The first two consider classes based on sequences of quantum states preparable by a polynomial-sized quantum circuit, where the circuits are uniformly generated by a computational class, for instance, the class $\mathsf{PSPACE}$, which results in the complexity class $\mathsf{StatePSPACE}$~\cite{RosenthalYuen:2022,MetgerYuen:2023}.
% The third notion considers a relative complexity, where the complexity is measured between two given states, and is measured by the number of gates, from a given gate-set, required to transform one state in another state~\cite{Aaronson:2004}. 
% For our definition of state preparation complexity, we drop the uniformity constraint from~\cite{RosenthalYuen:2022,MetgerYuen:2023} and define a class as $\mathsf{StateX}$, which refers to states preparable by circuits of type $\mathsf{X}$. 
% As an example, if $\mathsf{X} = \QNC^0$, this results in the class $\mathsf{StateQNC^0}$, which is the set of states preparable from the $\ket{0}^n$ state by poly-size constant-depth circuits. 
% This notion is similar to the relative complexity from~\cite{Aaronson:2004}, where one state is the  $\ket{0}^n$ state and instead of counting the number of gates we consider the set of states preparable by a fixed number of gates. Using this notion of complexity we show that any state preparable by an $\LAQCC^*$ circuit is also preparable by a $\mathsf{PostQPoly}$ circuit, the class of circuits of polynomial depth with an additional post-selection gate. 

\paragraph{Summary of results}
\begin{itemize}
    \item We give a new definition of a computational model that captures the power of the four step process: applying a constant number of layers of one- and two-qubit gates; performing a syndrome measurement; perform a fast classical computation determining corrections; apply corrections. We call this model \emph{Local Alternating Quantum Classical Computations}, or $\LAQCC$ for short. In this model we bound the allowed quantum operations, intermediate classical calculations, and number of rounds separately. In Section~\ref{sec:LAQCC_model} we define this model and give a list of operations based on results from literature contained in this computational model. In some of these operations we explicitly use that we allow for multiple, but at most constant, rounds  of corrections.
    \item  We show show that there exist $\LAQCC$ circuits that can not be weakly simulated in Section~\ref{sec:IQP_in_LAQCC}. We further show that for every $\LAQCC$ circuit there exists a $\QNC^1$ circuit simulating it perfectly, in Section~\ref{sec:LAQCC_in_QNC1}.
    \item We introduce a new type computational complexity for preparing states and show that the extension of $\LAQCC$ where we allow a polynomial number of rounds and unbounded classical computation, is contained in $\mathsf{PostQPoly}$, the class of polynomial circuits with post-selection, in Section~\ref{sec:Complexity results}.
    \item We show a protocol to prepare the uniform superposition state of size $q$ in $\LAQCC$ using $\mathO(\ceil{\log_2(q)}^2)$ qubits in Section~\ref{sec:superposition_modulo_q}. 
    \item We show a protocol to prepare the $W_n$ state in $\LAQCC$ using $\mathO(n\log(n))$ qubits in Section~\ref{sec:W_state_in_LAQCC}.
    \item We show two ways of preparing the Dicke-$(n,k)$ state. The first method is in $\LAQCC$, works up to $k = \mathO(\sqrt{n})$, uses $\mathO(n^2\log(n))$ qubits, and is found in Section~\ref{sec:dicke:small_k}. The second method is in $\LAQCC\text{-}\mathsf{LOG}$ (an extension of $\LAQCC$ allowing for logarithmic number of alterations instead of constant), works for any $k$, uses $\mathO(\text{poly}(n))$ qubits, and is found in Section~\ref{sec:Dicke_in_LAQCC_LOG}. 
    \item We extend on our $\LAQCC$ method of generating Dicke-$(n,k)$ states for $k = \mathO(\sqrt{n})$ and show a protocol to generate many-body scar states for a particular Hamiltonian in $\LAQCC$ (Section~\ref{sec:many_body_scar}). 
\end{itemize}
Summarized in a table, we provide the following state generation protocols:
\begin{table}[htb]
\centering
\begin{tabular}{l|l|l|l}
\textbf{State description} & \textbf{Width} & \textbf{Depth} & \textbf{Implementation}\\
\hline 
Uniform superposition mod $q$: $\frac{1}{\sqrt{q}} \sum_{i = 0}^{q-1}\ket{i}$ & $\mathO(\ceil{\log^2 q})$ & $\mathO(1)$ & Section~\ref{sec:superposition_modulo_q}\\

$W$-state: $\frac{1}{\sqrt{n}}\sum_{i = 0}^{n-1}\ket{e_i}$ & $\mathO(n \log n)$ & $\mathO(1)$ & Section~\ref{sec:W_state_in_LAQCC}\\

Dicke-$(n,k)$, $k = \mathO(\sqrt{n})$: $\binom{n}{k}^{-1/2}\sum_{x \in \{0,1\}^n: |x| = k} \ket{x}$ &  $\mathO(n^2\log n)$ & $\mathO(1)$ 
&Section~\ref{sec:dicke:small_k}\\

Dicke-$(n,k)$: $\binom{n}{k}^{-1/2}\sum_{x \in \{0,1\}^n: |x| = k} \ket{x}$ & $\mathO(\text{poly}(n))$ & $\mathO(\log n)$ &Section~\ref{sec:Dicke_in_LAQCC_LOG}\\

QMBS: $\ket{S_k} = \frac{1}{k! \sqrt{\mathcal N(n,k)}}(Q^\dagger)^k \ket{\Omega}$ &  $\mathO(n^2\log n)$ & $\mathO(1)$  &  Section~\ref{sec:many_body_scar}
\end{tabular}
\caption{Summary of state preparation protocols given in this paper.}
\label{tab:sate_prep}
\end{table}
In the entry for the quantum many-body scar state $Q$ denotes the raising operator and $\mathcal N(n,k)=\binom{n-k-1}{k}$. 
Section~\ref{sec:many_body_scar} will provide more details on the variables and the implementation. 

\paragraph{Organization of the paper}
\noindent We first introduce relevant preliminaries in Section~\ref{sec:preliminaries}. 
In Section~\ref{sec:LAQCC_model} we formally define the class of Local Alternating Quantum-Classical Computations ($\LAQCC$). We also show that any Clifford circuit can be implemented in constant depth $\LAQCC$ (a result based on a result from measurement-based quantum computing~\cite{jozsa2006introduction}). 
This result allows us to give many useful multi-qubit gates and routines in Section~\ref{sec:gates_created_in_LAQCC}. 
Beyond that we show that constant depth $\LAQCC$ circuits are contained in $\QNC^1$ and that any $\mathsf{IQP}$ circuit has an $\LAQCC$ implementation.
We conclude this section with an analysis of a more powerful instantiation of $\LAQCC$ and show an inclusion with respect to the class $\mathsf{PostQPoly}$, which is the class of circuits of polynomial depth with one additional post-selection gate. 
In Section~\ref{sec:state_prep_in_LAQCC} we give $\LAQCC$ circuit implementations for preparing the uniform superposition over an arbitrary number of states, the $W$-state and the Dicke state up to $k = \mathO(\sqrt{n})$. We furthermore give a log-depth circuit implementation for preparing the Dicke state for any $k$. We conclude by showing a $\LAQCC$ circuit for generating many body scar states of a particular type of Hamiltonian.



% \vspacebeforesection
\section{Background}
\label{sec:background}

In this section, we provide the necessary background information to ensure a comprehensive understanding of the attack described in this paper. We start with a description of the Distributed Hash Table (DHT) used by IPFS, followed by its content resolution mechanisms. We also detail techniques for network size estimation, necessary for our attack detection and mitigation mechanisms.

\vspacebeforesection
\subsection{IPFS DHT}
\label{sec:kad_dht}

We review the features of the Kademlia DHT~\cite{maymounkov2002kademlia} and its \texttt{libp2p} implementation~\cite{libp2p_github} that are the most relevant to our attack.
To participate in the DHT, each peer generates a public/private key pair and derives an identity $\peerid \in \{0,1\}^{256}$ as the hash of its public key.
Ideally, each peer generates a random key pair and, therefore, peer IDs are distributed uniformly and independently over the space $\{0,1\}^{256}$.
While honest nodes follow this rule, malicious nodes may generate and choose from an arbitrary number of key pairs.
Each peer maintains a routing table consisting of $m=256$ buckets.
The $i$-th bucket contains the addresses of up to $k=20$ peers whose peer IDs share a common prefix of exactly $i$ bits with the peer's own peer ID. 

%
A new participant node joins the IPFS network by contacting one of the hardcoded bootstrap nodes. This bootstrap node provides the new node with some initial peers allowing it to join the DHT. The new node uses this information to perform a walk through the DHT towards its own peer ID.
The walk allows to: \textit{(i)}~make sure that there is no other node in the network with the same ID; \textit{(ii)}~discover new peers and fill the newcomer's DHT routing table. At the same time, the newcomer establishes \bitswap~\cite{de2021accelerating} connections to a subset of encountered peers (usually around 300 of them). The core role of the \bitswap protocol is to enable bilateral content transfer and to play the role of a cache for recently-accessed content.

The main DHT operation $\Call{GetClosestPeers}{\key}$ returns the $k=20$ closest peers to $\key$. 
%
In Kademlia, the distance between two keys $x$ and $y$ in the key space is given by $x \oplus y \in \{0,...,2^{256}-1\}$, where $\oplus$ denotes the bitwise XOR operation on the keys; the resulting binary string is interpreted as an integer.
%
When a client wants to find the peers with IDs closest to $\key$, it sends a request to the $\alpha=3$ peers in its routing table whose peer IDs are closest to $\key$. Each of these peers returns the $k$ closest peers to $\key$ in its own routing table and the addresses of these peers. 
%
The client again sends a request to the $\alpha$ peers closest to $\key$, among peers in its routing table and those whose addresses it just received. This process repeats until the client does not find any more peers closer to $\key$.
Due to network churn and imperfect routing tables, we observed in our experiments that successive calls to $\Call{GetClosestPeers}{\key}$ do not always return the same set of $k=20$ peers (we provide more details in \Cref{sec:evaluation}, \Cref{fig:20closest}). This is an important limitation affecting our attack.

\vspacebeforesection
\subsection{Content Resolution in IPFS}
\label{sec:ipfs}

IPFS is a content-centric network.
It allows its participant to request files without specifying their location. 
%
Content is indexed by content IDs $\cid \in \{0,1\}^{256}$ that are derived from a hash of that content.
Both peer IDs and CIDs are used as keys in the DHT.
Each node can play the role of a \provider, \downloader, or \resolver. 
The process of content advertisement and resolution is illustrated in \Cref{fig:add_get_provider}.

%
When a \provider wishes to publish content with a given $\cid$ on IPFS, it creates a \emph{provider record} that contains $cid$ and the \provider's address.
During a $\Call{Provide}{\cid}$ operation, the \provider first uses $\Call{GetClosestPeers}{\cid}$ to locate the $k=20$ peers with their peer IDs closest to $\cid$, 
%
and then sends them a $\mathsf{PutProvider}$ message including the provider record (\Cref{fig:add_get_provider}(a)).
We call the peers that hold provider records for $\cid$ the \emph{resolvers} for $\cid$.

Each CID can have several \providers. In fact, by default, each IPFS client becomes a provider for each piece of content it downloads for a fixed amount of time (12h, 24h, or 48h depending on the client version or custom configuration). As a result, the system provides an auto-scaling feature with supply automatically rising with demand.

%
When a \downloader wishes to fetch a piece of content, it first sends a request to all its \bitswap peers. If none of them has the content, the \downloader uses the DHT-based resolution system. We stress that the \bitswap protocol plays the supporting role of a cache in the dissemination of popular files. However, the mechanism does not provide reliable content resolution, in particular for new or less popular content. %

When \bitswap unstructured search fails, the \downloader resolves $\cid$ using $\Call{FindProviders}{\cid}$. This operation uses a DHT walk identical to that of $\Call{GetClosestPeers}{\cid}$ to find $k$ \resolvers but also queries encountered nodes for a provider record for $\cid$ (\Cref{fig:add_get_provider}(b)). The process terminates when either 20 \providers have been found, or all \resolvers have been asked. Querying all encountered nodes (\ie, not only the designated \resolvers) is useful because some of the encountered nodes may have a provider record in their cache.
%

Upon receiving a provider record, the client connects to the address specified in the provider record to retrieve the actual content (\Cref{fig:add_get_provider}(c)).
Provider records are not authenticated, and therefore malicious \providers may respond with incorrect provider records (or may not respond at all). However, the integrity of the content is preserved because the hash of the retrieved content can be verified against its $\cid$.
%


%

\input{img/add_get_provider.tex}

\vspacebeforesection
\subsection{Network Size Estimator}
\label{sec:netsize}

The number of nodes in a decentralized system is generally unknown due to the avoidance of centralized membership management.
This number is nonetheless useful for optimizations, deciding on individual node configurations, or security mechanisms.
Various methods were proposed for the decentralized estimation of unstructured and structured networks~\cite{eli-sohl-dht-size-estimation,kostoulas2005decentralized, manku2003symphony}.
We use in this work a mechanism developed initially by Protocol Labs as part of a mechanism for decreasing the latency of publishing content in IPFS~\cite{network-size-estimation-notion,network-size-estimation-github-pr}.

%
%
%
%
%
%
%
%
%
%

Each node in the DHT refreshes its routing table periodically (every $10$ minutes in \texttt{libp2p}). 
For this, the node samples $m$ random keys (one for each bucket of its routing table)
%
and queries the DHT to obtain the $k=20$ closest peer IDs to each key.
Using these, the node then computes the average distance between each one of these keys $\key_j$ for $j=1,\dots,m$ and their $i$-th closest peer ID for $i=1,...,k$ (with $m=256$ and $k=20$).
\begin{equation}
    \label{equ:avg-dist}
    \overline{D}_i = \frac{1}{m} \sum_{j=1}^m \operatorname{dist}(\key_j, \peerid_{j}^{(i)})
\end{equation}
where $\peerid_{j}^{(i)}$ is the $i$-th closest peer ID to $\key_j$.
With $N$ peers in the DHT and peer IDs uniformly distributed in the hash space, the expected distance between a $\key$ and its $i$-th closest peer ID is $\frac{2^{256}i}{N+1}$. The node then runs a least square regression to compute the value of $N$ for which the expected distances best fit the empirical average distances, \ie,
\begin{equation}
    \label{equ:netsize-least-squares}
    \hat{N} = \arg\min_{N} \sum_{i=1}^k \left(\overline{D}_i - \frac{2^{256}i}{N+1}\right)^2.
\end{equation}
The resulting estimate $\hat{N}$ can be computed in closed form.
%

When a node starts running, it must perform DHT queries for a few random keys to initialize its network size estimate. 
Since a larger number of queries will result in higher accuracy, making more queries than what is needed to initialize one's routing table is recommended.
Thereafter, keeping the estimate up-to-date does not require any excess DHT queries beyond what is already used for refreshing the routing table as this is done frequently (every 10 minutes).

While the network size estimate has a stochastic variance resulting from the probability distribution of the honest peer IDs, it is hard for an attacker to bias the estimate significantly. Since the estimator uses the density of peer IDs around keys chosen uniformly at random, the adversary would require numerous Sybil nodes (on the order of the whole network size) to significantly affect the peer ID density around those keys.


\section{Empirical study of GUIs between phones and TVs}
\label{sec:empiricalStudy}
%Figure~\ref{fig:empirical} shows the whole process of our empirical study.
To better understand the current status and characteristics of GUI adaptation to smart TV, we conduct a large-scale empirical study to answer two questions: 1) How many phone apps support TV display? 2) How do app GUIs change between TVs and smartphones?

\subsection{RQ1 How many phone apps support TV displays?}
\label{sec:supportTV}

To answer RQ1, we analyze a large number of industrial Android apps from all 33 categories on Google Play~\cite{googleplay}.
We crawl the top 200 most popular apps in each category (at the time of Jun 2021).
Since some apps are not free, we obtain 5,580 apps that support Android phones by default.

According to the official guidelines of developing Android TV~\cite{AndroidTVDe}, it is compulsory for apps running on TV to declare a TV activity with an intent filter \textit{CATEGORY\_LEANBACK\_LAUNCHER} in the Android manifest file of Android projects.
In addition to the declaration, TV-enabled apps often have separate TV layouts called \textit{layout-television} and \textit{layout-tv}.
Therefore, we decompile Android APKs to check whether the specified intent filters or layout XML files exist.
We find only 298 of these apps supporting TV display, accounting for only 5.34\% of the total.
These 298 apps belong to 29 categories including \textit{Weather} (16.78\%), \textit{Education} (15.10\%), \textit{Tool} (11.07\%).
% Most apps that do not support TV will fail to install or run on the TV  if they are forced to do so owing to hardware and Android system version incompatibility.
 
Even if an app advertises that it supports a TV display, this does not imply that it is appropriately optimized for the TV.
We manually check 298 apps that claim to support TV displays.
We physically run the apps on a smart TV and critically evaluate how well various GUI pages adapt to the TV display. GUI pages exhibiting discrepancies such as substantial black margins at the screen's bilateral extents, discordant aspect ratio components, disordered layouts, and unoptimized navigation, among other factors -- which all may diminish the user experience on TV displays -- are regarded as inadequate instances of the TV adaptation.
We discover only 11 of 298 apps allow TV display on all GUI pages.
287 out of 298 apps modify a few representative GUI pages, such as the home or landing page, to accommodate TV displays.
These 287 apps have an average of 22.3 Android activities, but we only find support for TV display in an average of 4 Android activities.
%\textit{layout-land-television-v8.xml}, \textit{HomeActivity-TV.xml} and \textit{movie-television.xml}. 
Other GUIs of these apps also look poor on TV, with large black margins on both screen sides, and mismatched aspect ratio components, as the \emph{Direct mapping} in Figure~\ref{fig:currentExamples}.
% \todo{This seems a bit out of placed with the rest of the above - can you comment on number that seem to have custom TV apps - maybe these are the 'good' looking ones??}
Due to the significant difference between phone and TV displays, some development teams tend to develop separate apps for different platforms.

% Figure environment removed

\subsection{RQ2 How do app GUIs change between TVs and smartphones?}
\label{sec:rq2}
We collected custom-designed TV apps and their matching phone apps for a comparative study since there are relatively few apps that support both TV and phone.

\subsubsection{TV-phone App Pairs Collection}
\label{sec: appCollection}
% In Google Play, there is a specific category~\cite{googleTV} of apps supporting Android TV.
% Dangbei~\cite{dangbei} is one of the largest TV app stores.
We collected 249 TV apps from Google Play's TV category~\cite{googleTV} and 2,556 from Dangbei~\cite{dangbei}, which is one of the largest TV app stores. We eliminated any apps that have not been updated for over two years.

We then  matched the TV apps' corresponding smartphone apps on Google Play.
To begin, we search for phone apps with the same app name, developers, and category.
Second, if the corresponding phone version app cannot be found, we broaden our search to include apps with similar app names developed by the same developers.
Finally, for the TV apps that do not find the phone-version apps, four volunteers manually collect their matching phone apps.

%TV games are all deleted for no applicable GUI and match phone version game.
We match 1,405 TV-phone app pairs, with the three most common categories being \emph{Video (42\%)}, \emph{Education (23\%)} and \emph{Tool (21\%)}.
Video apps, such as \textit{Youtube}, \textit{IQIYI}, are most suitable for smart TVs due to the characteristics of TV itself, so video apps have become the most popular apps on TV.
% Educational apps, especially for children, are also ubiquitous on TVs, yet the GUIs of educational apps on phones and TVs are typically somewhat different.
% Some tools, like TV app store, remote control and projection control are also widely used in TV.
Because smart TVs are mostly used at home, there are a number of educational apps for children.
Tool apps, like TV app store, remote control and projection control are also ubiquitous on TVs.

% \textbf{Alignment between TV and Phone GUIs}
\subsubsection{TV-phone GUI Pair Collection}
\label{sec:guiCollection}
We use the DroitBot~\cite{li2017droidbot}, Fastbot~\cite{cai2020fastbot} and Uiautomator2~\cite{uiautomator2} to automatically explore apps and collect rendered screenshots and metadata of GUIs in apps.
%Four volunteers manually traverse rendered app GUIs in each app to collect screen information using custom tools we built. 
The metadata is a documentary object model (DOM) tree of current GUIs, which includes the hierarchy and properties (e.g., class, bounding box, layout) of UI components. 
We can infer the GUI hierarchy from the DOM Tree hierarchy in metadata.
After removing duplicates, we obtain 6,697 Android GUI data and 4,112 TV GUI data.
We notice that most TV apps simplify or restructure their GUIs to accommodate various usage scenarios and requirements of smart TVs during this process.

At TV-phone GUI pairing, we exploit the semantic similarity of Android activity names to pair the TV and phone GUIs automatically.
Then, we automatically compare UI components on GUI pages in order to match state-level GUI pairings in a lesser granularity.
Finally, we manually check the automatically discovered pairs and select the final valid TV-phone pairs.
We extract activity names from each GUI and encode them into numerical semantic vectors using a pre-trained BERT~\cite{devlin2019bert} model.
Then, we match the TV-phone GUI pairs by comparing their semantically close activity vectors.
For example, the GUI in activity \textit{homeActivity} and \textit{mainActivity} are matched by close semantic vectors.
However, one Android activity may have multiple Android fragments~\cite{fragment} and GUI states~\cite{machiry2013dynodroid, GUIstate} with different UI components and layouts in current industrial apps.
We further compare GUI components between phones and TVs to pair TV-phone GUIs at lower granularity.
In pairing, UI components are identified by their types and properties.
UI components between phones and TVs  with the same types and properties are considered paired GUI components.
For example, two \emph{TextViews} with the same texts, two \emph{ImageViews} with the same images, two \emph{Buttons} with the same texts are considered the paired components.
If more than half of the UI components in two GUIs are paired, they are considered a state-level TV-phone GUI pair.
% When faced with multiple interface activities, like \textit{tablayout}, we further trace the selected text in top/bottom tabs as current semantic strings to match the corresponding GUI.
Finally, we manually checked all discovered pairs and identified 589 TV-phone state-level GUI pairs with clear GUI correspondence between phone and TV components.


\subsubsection{GUI component grouping}
\label{sec:GUIgroup}
A series of UI components that are near in position and hierarchy of the GUI and tend to have the same functionality are referred to as GUI groups\footnote{Sometimes called GUI patterns}~\cite{guiBuild, neil2014mobile}.
Exploring these group changes based on UI groups rather than each individual component is more beneficial~\cite{neil2014mobile}.
So, we must first identify and categorize GUI groups that are common to both TV and phone before summarising the guidelines for GUI changes from phone to TV.

We perform an iterative-open coding process, which is widely used to generate categories in Software Engineering~\cite{seaman1999qualitative, hu2023first, hu2023look, chen2021my}. We do this on 120 randomly selected phone GUIs and 80 TV GUIs (approximately 2\% of collected phone and TV GUIs) to categorize their GUI groups.
Four volunteers with Android design experience undertook three steps in our open coding procedure.
At first, we are inspired by Google design~\cite{GoogleMaterial} and development guidelines~\cite{androidGuidelines}, every volunteer categorizes GUI groups in selected GUIs individually.
After the initial coding, let four volunteers have a discussion and merge conflicts.
They clarify scope boundaries among categories and misunderstandings in this step.
In the third step, they iterate to revise classifications and discuss with each other until a consensus is reached.
Finally, we determined 12 phone group types: \textit{Icon + Info}, \textit{Tool Bar},  \textit{Bottom Tab Layout}, \textit{Search}, \textit{Top Tab Layout}, \textit{Pic Side Info}, \textit{Pic + Info}, \textit{Side Nav}, \textit{Short Video Player}, \textit{Video/Music Player}, \textit{Big Pic} and \textit{List View} and 9 TV group types: \textit{Icon + Info}, \textit{Tool Bar}, \textit{Search}, \textit{Tab Layout}, \textit{Channel}, \textit{Grid Layout}, \textit{Pic + Info}, \textit{Video/Music Player} and \textit{List View}.
Figure~\ref{fig:phoneGroups} and \ref{fig:TVGroups} show examples of summarized phone and TV groups.
These phone and TV GUI group categories can be divided into two subcategories.
GUI groups in the first subcategories are widely used in both phone and TV, e.g. \textit{Tool Bar}, \textit{Search}, and \textit{Video/Music Player}.
GUI groups in the second subcategory only exist in phones or TVs respectively, e.g. such as \textit{Bottom Tab Layout} and \textit{Channel}.

To verify the accuracy of GUI grouping and classification, we first randomly sample 20 TV and smartphone apps in collected 1405 TV-phone app pairs in Section~\ref{sec: appCollection}, respectively.
The selected apps have covered all 6 TV app categories (Tool, Video, Music \& Audio, Education, Entertainment, and Weather) listed on Google Play~\cite{ggTV}.
Then, we manually calculate the distribution of each GUI group in selected TV and smartphone apps.
Finally, we find that these summarized GUI components groups have covered 93.69\% and 93.47\% of GUI groups on phones and TVs, respectively.
In our analysis, we observe the presence of uniquely shaped GUI components in both phone and TV interfaces. These components resist typical categorization due to their complex functional requirements in production and everyday use. They constitute 6.31\% and 6.53\% of phone and TV GUIs, respectively. These specific GUI components primarily appear within individual apps or among apps from the same developer. We categorize these components under an $Others$ group category for the purpose of our study. To address these edge cases in the GUI conversion process, we propose the development of default templates. Given the relatively minor percentage of GUI components in the $Others$ category and their limited appearance in specific applications, we assert that our categorization effectively represents the vast majority of commonly used phone and TV GUIs.

Table~\ref{tab:matchRules} shows the details of GUI group distributions.
Subcolumns \emph{Group} and \emph{Distribution} of columns \emph{Phone} and \emph{TV} denote the GUI group categories and distribution in the experiment.
TV's subcolumn \emph{Group} does not contain the \emph{Others} GUI group, which comprises 6.53\% of all TV GUIs in the experiment.
On phones, the most popular categories of components groups are \textit{Icon + Info} (13.31\%), \textit{Tool Bar} (11.41\%) and \textit{List View} (11.14\%), but on TV, categories \textit{Pic + Info} (19.13\%), \textit{Grid Layout} (13.37\%) and \textit{List View} (13.18\%) are most common categories.
The official guideline of TV GUI design~\cite{AndroidTVDe} suggests two principles for TV GUI design \emph{All TV GUIs should display in landscape mode} and
\emph{The core TV GUIs use card-like views instead of ListView or ViewPager to make better use of horizontal screen space and accommodate TV interaction}.
Standard current TV GUIs obey these two principles to use more grid layouts and card-like widgets.
Thus card-like categories \textit{Pic + Info} and \textit{Grid Layout} are more popular on TVs than on phones.



% TV UIs are relatively large, so \textit{Icon+info} has dropped from 13.31\% on phones to 6.85\% on TV.
%Most TV apps are audio-visual, so \textit{Video/Music Player} has increased from 3.50\%  to 10.56\% on TV.
% TV is not suitable for finger sliding and click, resulting in the decline of group \textit{Short Video Player}, \textit{Bottom Tab Layout} and \textit{Pic Side Info}.
% \textit{Bottom Tab Layout} has disappeared on TV too, because it in the TV is not convenient for users to click and switch.
% TV app always move phone's contents in \textit{Bottom Tab Layout} to \textit{Channel}.
% Pictures in TV either has no related information or the related information is under the picture. 
% It rarely happens that information is on both sides of the picture, so \textit{Pic Side Info} is rarely seen on TV.



% Figure environment removed
\setlength{\textfloatsep}{10pt}

\subsubsection{Component group alignment}
\label{sec:groupAlignment}
After conducting our GUI components grouping study, we notice that the design principles of TV and phone GUIs are vastly different, resulting in no obvious one-to-one alignments between most TV-phone GUI groups.
Furthermore, the contents of one phone GUI group may be dispersed throughout numerous groups in the TV GUI, and vice versa.
Therefore, we summarize heuristic rules from our collected TV-phone GUI pairs for automatic GUI mapping from phone to TV.

Firstly, we randomly divide the 589 TV-phone GUI pairs into experimental and validation sets in an 8:2 ratio.
In the experimental set, four volunteers follow the same three steps to perform open coding to analyze and extract conversion rules.
Table~\ref{tab:matchRules} shows the extracted GUI group match rules from phone to TV.
To accommodate how the TV and remote interact, each TV GUI group uses card-like views.
The converted TV GUI group recalculates the new size depending on the quantity and types of components in the existing GUI to fit the TV screen size. 
According to the GUI group study in Section~\ref{sec:GUIgroup}, we use \textit{Grid Layout} as our default template for mapping.
Component groups with the same meaning \textit{Icon + Info}, \textit{Tool Bar}, \textit{Search}, \textit{Top Tab Layout}, \textit{Video/Music Player} and \textit{List View} in phone and TV are transferred directly.
According to the characteristics and the official design guideline~\cite{AndroidTVDe} of TV, \textit{Pic Side Info}, \textit{Pic + Info} and \textit{Big Pic} are all converted to \textit{Pic+Info} in TV.
After exploration, current TV GUIs tend to replace components in the phone's \textit{Side Navigation} and \textit{Bottom Tab Layout} to \textit{Channel} in TV,
so we follow this trend.
\textit{Short Video Player} should use customized templates in TV, but there's no such TV app with this GUI feature at the moment.
As a result, we don't provide the corresponding TV group individually at the moment, instead relying on \textit{Video Player} to convert.

We use the validation set to verify these mapping rules.
Our first step involves manually identifying and extracting GUI components from respective groups within the phone GUI. 
Subsequently, we locate corresponding components in the matching TV GUI. 
Our final process entails verifying if these TV GUI components align with the anticipated TV groups and comply with the matching rules established in our experimental set. 
Given \emph{m} instances of \emph{Side Nav} groups in our validation set, and $n$ corresponding TV GUI groups classified as $Channel$ groups, we compute the mapping rule accuracy as the ratio $n/m$.
Note that if the phone GUI group is eliminated in the corresponding TV GUI, the case is considered invalid and will not be counted.
The \emph{Mapping Accuracy} in Table~\ref{tab:matchRules} demonstrates the correctness rate of each mapping rule.
Finally, we find that the correctness of rules 1, 2, 3, 4, 5 and 7 are 96\%, 99\%, 91\%, 99\%, 99\%, 99\% and 100\%, respectively, indicating that these direct mapping rules are accurate and universal.
Rules 8, 9, 10, 11, and 12 have an accuracy of 83\%, 95\%, and 87\%, 90\%, and 99\%, respectively, suggesting these change rules are also accurate and common.

\begin{table}[!htbp]
\setlength{\abovecaptionskip}{0pt}
\setlength{\belowcaptionskip}{0pt}
\caption{Component group matching between phone and TV. TV's subcolumn \emph{Group} does not contain the \emph{Others} GUI group, which comprises 6.53\% of all TV GUIs in the experiment.}
\scalebox{0.8}{
\begin{tabular}{|c|cc|cc|c|}
\toprule
\multirow{2}{*}{\textbf{Index}} & \multicolumn{2}{c|}{\textbf{Phone}}     & \multicolumn{2}{c|}{\textbf{TV}}  & \multirow{2}{*}{\textbf{Mapping Accuracy}}                     \\
~ & Group & Distribution & Group & Distribution & \\

\midrule
1 & Icon + Info  & 13.31\%       & Icon + Info  & 8.32\%  & 96\%                   \\
2 & Tool Bar  & 11.41\%          & Tool Bar    & 7.83\%   & 99\%                   \\
3 & List View      & 11.14\%     & List View          & 13.18\%  & 91\%             \\
4 & Top Tab Layout  & 8.88\%    & Top Tab Layout   & 7.68\%  & 99\%               \\
5 & Search    & 7.98\%          & Search       & 7.12\%  & 99\%                   \\
6 & Others & 6.31\% & Grid Layout (Default) & 13.37\% & 90\%  \\
7 & Video/Music Player & 3.50\% & Video/Music Player    & 7.56\%  & 100\%          \\

\midrule
8 & Pic Side Info  & 8.90\%     & \multirow{3}{*}{Pic + Info} & \multirow{3}{*}{19.13\%}  & 83\%   \\
9 & Pic + Info   & 8.67\% &             &         & 95\%             \\
10 & Big Pic     & 3.52\%         &        &      & 87\%                     \\

\midrule
11 & Bottom Tab Layout   & 10.31\% & \multirow{2}{*}{Channel}    & \multirow{2}{*}{9.28\%}   & 90\%   \\
12 & Side Nav  & 6.07\%  &             &        & 99\%             \\
\bottomrule
\end{tabular}%
}
\label{tab:matchRules}
\end{table}


\subsection{Summary and Implications}
\label{sec:implication}
Our empirical study shows that:
(1) Only 5.34\% of popular phone apps support TV displays. 
% IT companies are more inclined to develop a new TV app, which may lead to repeated efforts.
(2) In TV-phone GUI pairs, there is not much explicit one-to-one correspondence between phone and TV component groups.
(3) We summarize 12 and 9 categories of GUI components groups on phone and TV, covering 93.69\% and 93.47\% popular phone and TV GUIs, respectively. 
(4) We extract and evaluate 12 existing GUI group-mapping rules from phone to TV based on summarised GUI component groups.

% \todo{changed to semi-automated generation! :-) }
The lack of TV-display support for phone apps confirms the necessity of tool development for semi-automated GUI mapping between phone and TV.
That motivates our study and the empirical findings of component group mapping are the backbone of our proposed approach.

\definecolor{lightblue}{rgb}{0.63, 0.74, 0.78}
\definecolor{seagreen}{rgb}{0.18, 0.42, 0.41}
\definecolor{orange}{rgb}{0.85, 0.55, 0.13}
\definecolor{silver}{rgb}{0.69, 0.67, 0.66}
\definecolor{rust}{rgb}{0.72, 0.26, 0.06}

\colorlet{lightsilver}{silver!30!white}
\colorlet{darkorange}{orange!75!black}
\colorlet{darksilver}{silver!65!black}
\colorlet{darklightblue}{lightblue!65!black}
\colorlet{darkrust}{rust!85!black}

\tikzstyle{trainingstyle} = [rectangle, 
minimum width=2cm, 
minimum height=2cm,
text centered,
text width=1.5cm,
draw=black,
thick,
rounded corners=0.3cm, 
fill=lightsilver, label=(a)]


\tikzstyle{onnxstyle} = [rectangle, 
minimum width=2cm, 
minimum height=0.8cm, 
text centered,
draw=black, 
thick,
rotate=90,
rounded corners=0.3cm,
fill=lightsilver, label=right:(b)]

\tikzstyle{mtestyle} = [rectangle, 
minimum width=2.7cm, 
minimum height=0.8cm, 
text centered, 
text width=2.8cm,
thick,
draw=black, 
rotate=90,
thick,
rounded corners=0.3cm,
fill=lightsilver, label=right:(d)]

\tikzstyle{decodingstyle} = [rectangle, 
minimum width=2cm, 
minimum height=0.8cm, 
text centered,
draw=black,
rotate=0,
thick,
rounded corners=0.3cm,
fill=rust!30, label=(f)]

\tikzstyle{eucstyle} = [rectangle, 
minimum width=2cm, 
minimum height=1cm,
text centered,
text width=1.55cm,
draw=black,
thick,
rounded corners=0.3cm, 
fill=white, label={[xshift=0.15cm]right:(g)}]

\tikzstyle{border1style} = [rectangle, 
minimum width=2.5cm, 
minimum height=4.75cm,
text centered,
text depth=4cm,
draw=black,
rounded corners=0.3cm, 
thick,
fill=lightsilver,
fill opacity=1, label={[xshift=0.5cm]above:(e)}]

\tikzstyle{border2style} = [rectangle, 
minimum width=4.5cm, 
minimum height=4cm,
text centered,
text depth=3.5cm,
draw=black,
dashed,
very thick,
rounded corners=0.3cm, 
fill=seagreen,
fill opacity=0.1,
text opacity=1,
label=(c)]

\tikzstyle{hpcstyle} = [rectangle, 
minimum width=2cm, 
minimum height=2cm,
text centered,
text width=1.9cm,
draw=black,
thick,
rounded corners=0.3cm, 
fill=lightsilver, label=(h)]

\tikzstyle{arrow} = [very thick,->,>=latex]
\tikzstyle{darrow} = [very thick,<->,>=latex]
    
\begin{tikzpicture}[node distance=1cm]
\node (training) [trainingstyle] {Neural Network Training};

\node (onnx) [onnxstyle, below of=training, yshift=-1cm] {ONNX};

\node (border2) [border2style, right of=onnx, xshift=2.25cm,yshift=0cm] {RoseNNa};

\node (mte) [mtestyle, below of=border2, yshift=2.4cm, xshift=-0.3cm] {Model Topology \\ Encoding};

\node (border1) [border1style, right of=mte, xshift=1.05cm, yshift=-0.8cm] {Decoding};

\node (decoding) [decodingstyle, below of=border1,xshift=0cm,yshift=1.5cm] {C/Fortran};

\node (euc) [eucstyle, below of=decoding, yshift=-1.25cm] {User \\ Codebase};

\node (hpc) [hpcstyle, right of=border2, xshift=3cm, yshift=0cm] {HPC \\ Deployment};

\node (border2) [border2style, right of=onnx, xshift=2.25cm,yshift=0cm, fill=none] {RoseNNa};

\draw [arrow] (training) -- (onnx);
\draw [arrow] (onnx) -- (border2);
% \draw [arrow] (mte) -- (3.55,0);
\draw [darrow] (decoding) -- (euc);
\draw [arrow] (border2) -- (hpc);
\end{tikzpicture}

\section{Evaluation} \label{sec:evaluation}

\begin{table*}[tbp]
\centering
\small
\begin{tabular}{cccccccccc}
\toprule
& \multicolumn{3}{c}{\msr} & \multicolumn{3}{c}{\negc} & \multicolumn{3}{c}{\wsj} \\
& Acc. & F1 & wF1 & Acc. & F1 & wF1 & Acc. & F1 & wF1 \\ \cmidrule(lr){2-4} \cmidrule(lr){5-7} \cmidrule(lr){8-10} 
\udel & 66.86 & 56.76 & 64.3 & \textbf{80.80} & 55.45 & 77.9 & 63.74 & 64.23 & 63.2 \\
\icsi & \underline{71.19} & 64.73 & 70.4 & 80.36 & 64.53 & \underline{78.6} & 64.62 & 64.15 & 63.4 \\
\cnts & 68.59 & 61.39 & 67.2 & 78.68 & 61.62 & 76.8 & 64.31 & 64.59 & 64.4 \\
\osu & 68.02 & 60.28 & 66.6 & 79.24 & 57.04 & 76.5 & 69.20 & 69.63 & 68.9 \\
\isg & 67.05 & 58.83 & 65.3 & 77.34 & 59.52 & 75.6 & 69.15 & 69.35 & 69.2 \\ \midrule
\bert & \textbf{71.68} & \underline{66.70} & \textbf{71.4} & 77.79 & \underline{72.87} & 77.7 & \underline{80.95} & \underline{80.93} & \underline{80.9} \\
\roberta & 70.91 & \textbf{67.53} & \underline{70.7} & \textbf{80.80} & \textbf{77.29} & \textbf{80.7} & \textbf{82.61} & \textbf{82.70} & \textbf{82.6} \\ \midrule
Average & 69.19 & 62.32 & 67.99 & 79.29 & 64.05 & 77.69 & 70.65 & 70.80 & 70.37 \\
\bottomrule
\end{tabular}
\caption{\label{tab:performance} Overall accuracy (Acc.), macro-averaged F1 (F1), and weighted-macro F1 (wF1) scores of the algorithms depicted in Section~\ref{sec:algorithm}. For instance, \msr-\udel refers to a C5.0 classifier trained on the \msr~corpus, using the feature set mentioned in \citet{greenbacker-mccoy-2009-udel}.}
%Its Acc., F1 and wF1 of this model are 66.86, 56.76, and 64.3, respectively.}
\end{table*}


In this section, we introduce the evaluation protocol and report the performance of the models.

\subsection{Implementation Details} \label{sec:implementation}

For \bert and \roberta, we used \textit{bert-base-cased} and \textit{roberta-base}, both from Hugging Face. For fine-tuning, we set the batch size to 16, the learning rate to 1e-3, the dropout rate to 0.5, and the size of the output layer to 256. We ran each model for 20 epochs and used the one that achieved the highest F1 score on the development set. The implementation details of the classic ML-based models can be found in Appendix~\ref{sec:appendixML}.

\subsection{Evaluation Protocol} \label{sec:protocol}

The main evaluation metric in the GREC-MSR shared tasks was accuracy. 
In addition to accuracy, we also report macro-F1 and weighted-macro F1. We argue that different metrics evaluate algorithms from different perspectives and provide us with different meaningful insights. 
For pragmatic tasks like REG, it makes sense to ask how well an algorithm performs on naturally distributed data which is often imbalanced. For these cases, reporting accuracy and weighted F1 are logical. 
Furthermore, analogous to other classification tasks, minority categories should not be overlooked. Take as an example the class \emph{description} in the \negc corpus, which occurs only 4\%. If a model fails to produce this class, the produced document might sound unnatural. Therefore, it is important to ensure that an algorithm is not over- or under-generating certain classes. Looking into accuracy and macro-F1 together provides insights into such cases.

\subsection{Performance of the Models}\label{subsec:overallacc}

The overall accuracy of the models, their macro F1, and their weighted-macro F1 are presented in Table \ref{tab:performance}. 
We also present the ranking of the models based on these scores in Appendix~\ref{sec:app_rank}. 


\paragraph{PLM-based Models.} The best-performing models across all corpora and metrics are PLM-based models.  In six out of nine rankings, \bert and \roberta are ranked as the top two models. The sole exception is \negc, where \bert is the second worst model. The benefit of using PLMs is the largest on the \wsj corpus. For example, \roberta improves the macro F1 score from 69.63 (i.e., the performance of the best ML-based model) to 82.70.


\paragraph{ML-based Models.} In contrast to the robust performance of the PLM models, the performance of the classic ML models is more corpus-dependent. In the case of \msr and \negc, \icsi is the best-performing model, while in the case of \wsj, it is at the bottom section of the rankings. Another interesting observation is the performance of the \udel models. In terms of accuracy, \udel has the highest performance in \negc, while it has the lowest performance in both \msr and \wsj. In terms of macro-F1 rankings, the \negc \udel model dropped from first to last place, whereas \bert improved from penultimate place to second place. In general, our ML models yielded lower scores than the original models used in the GREC study \citep{belz2009generating}. This could be attributed to a variety of factors, including differences in feature engineering and model parameters.

\paragraph{Comparing Different Metrics.} 

Upon comparing average scores across the three metrics, we observe that for \msr and \negc, PLMs are clear winners only when macro-F1 is the metric in question. However, for \wsj, PLMs are winners on all three metrics. This may be because the distribution of categories in \wsj is much more balanced than in the other two corpora.

\section{Discussion}
\label{sec: discussion}
\kmsdelete{In this work} We study \kmsreplace{Fairness-Aware PAC learning}{Fair-ERM} in the malicious noise model, and  in some cases allow 
the learner to maintain optimal overall accuracy despite the signal in Group $B$ being almost entirely washed out.
%when we allow learners to use the
%$\PQ$ randomized expansion of the hypothesis class $\mathcal{H}$
In particular we show that different fairness constraints have fundamentally different behavior in the presence of Malicious Noise, in terms of the amount of accuracy loss that a given level of Malicious Noise could cause a fairness-constrained learner to incur. 
The key to achieving our results, which are more optimistic than those in \cite{lampert}, is allowing for improper learners using the (P,Q)-randomized expansions of the given class $\mathcal{H}$.
%We \kmsreplace{present a picture of the}{prove upper and lower bounds on}
%accuracy loss for a range of fairness notions, given \kmsreplace{this simple randomization step.}{learning over $\PQ$.
%In general our results indicate Fair-ERM (given learning over $\PQ$) is more robust than claimed in \cite{lampert}.
The type of smoothness we create by using $\PQ$ seems to be a natural property that is likely shared by many natural hypothesis classes.

Fairness notions are motivated as a response to learned disparities when there is \kmsdelete{data corruption or} systemic error affecting \kmsdelete{the data for}
one group. 
Fairness notions are supposed to mitigate this by ruling out classifiers that have worse performance on a sub-group. 
This can peg both classifiers at a lower level of performance \kmsdelete{(e.g that the lower subgroup)} in order to \emph{motivate} \cite{hardt16} improving the data collection or labelling process to obtain more reliable performance. 
%So in \kmsreplace{some}{a} sense, sensitivity of the fairness notion to poor sub-group performance caused by malicious noise is the \textit{point} of fairness constraints! 
However, it also desirable that fairness constraints perform gracefully when subject to Malicious Noise because fairness constraints will be used in contexts where the data is unreliable and noisy and this might not be known to the learner.
This tension, exposed by our work, motivates 
%a revisiting of fairness notions from first principles approach and trying to axiomatize the 
%desired properties of a fairness intervention a la cryptography and privacy. \footnote{Work in multi-calibration \cite{multicalib} is a viable direction for this problem but it is unclear how 
%that and related notions behave with unreliable data. }
on going work studying the sensitivity level of fairness constraints. 
%If we we are to take a view, if a classifier is deployed 


\section{Related work on REG in context}\label{sec:litreview}

Deciding about the form of a referring expression and determining its content are two different steps of the Referring Expression Generation (REG) task \citep{comreg2019}. In the current article, our focus is on the first step, namely determining the form of a referring expression. We will expand thiw work to the second task, namelz the content realisation of the REs in future work.
 

\subsection{Different RF categories} \label{subsec:refcateories}

As \citet{kibrik2016referential} put it, the \textit{basic} and binary referential choice is between the choice of a pronoun versus a variety of NPs. Studies addressing pronominalization, such as \citet{mccoy1999generating,poesio2004centering} and \citet{henschel2000pronominalization}, often focus on this binary distinction. More recent studies 
have looked at a wider range of referring expression (RE) types. For instance, the GREC shared tasks \citep{belz2009generating} exploited four RE types, namely pronoun, proper name, common noun and covert (empty) reference. \citet{kibrik2016referential} focused on a three-way distinction between the choice of a pronoun, proper name and common noun; while \citet{castro-ferreira-etal-2016-towards} classified REs into five categories of pronoun, proper name, common noun, demonstrative NP and empty reference.

\subsection{Different approaches to REG in context} \label{subsec:regapproach} 


Different methods are used to predict the referential choice in context. Rule-based approaches, such as \citet{passonneau1996using}, \citet{mccoy1999generating}, \citet{henschel2000pronominalization} and \citet{krahmer2002efficient}, employ different algorithms to predict RF choice-taking, for instance, centering rules or salience-based accounts into consideration. The GREC shared-task challenges, as one of the first systematic studies on the generation of REs in context, introduced new
feature-based Machine Learning (ML) solutions to this task (e.g. \citet{greenbacker-mccoy-2009-udel, hendrickx-etal-2008-cnts, bohnet-2008-g, favre-bohnet-2009-icsi}, among others). 
Following these shared tasks, \citet{kibrik2016referential} trained decision trees and regression models on the WSJ MoRA, a corpus of Wall Street Journal articles, using a large number of factors. In a more recent feature-based study, \citet{castro-ferreira-etal-2016-towards} trained Naive Bayes and Recurrent Neural Network (RNN) algorithms on the VaREG corpus, taking individual differences in the generation of REs into account. Over the past few years, deep learning approaches have been pre-dominantly used for an end-to-end generation of REs in context, predicting type and content of expressions altogether \citep{castro-ferreira-etal-2018-neuralreg,cao-cheung-2019-referring,cunha-etal-2020-referring,same-etal-2022-non}. \citet{chen-etal-2021-neural-referential} have used pre-trained language models for the choice of RF, but they only use the benchmark NLG dataset, namely WebNLG \citep{gardent-etal-2017-webnlg, castro-ferreira-etal-2018-enriching}, in their study and only use \bert. 


\section{Conclusion and Future Work}
In this work, I design corruption-robust algorithms for the Lipschitz contextual search problem. I present the \emph{agnostic checking} technique and demonstrate its effectiveness in designing corruption-robust algorithms. There are several open problems for future research. First, in the algorithm I propose for pricing loss, the schedule for agnostic checks is fixed upfront. Can the learner design an adaptive checking schedule for the pricing loss? Second, this work assumes the learner has knowledge of the Lipschitz constant $L$. Can the learner design efficient no-regret algorithms without knowledge of $L$? 



%%
%% The acknowledgments section is defined using the "acks" environment
%% (and NOT an unnumbered section). This ensures the proper
%% identification of the section in the article metadata, and the
%% consistent spelling of the heading.
% \begin{acks}
% To Robert, for the bagels and explaining CMYK and color spaces.
% \end{acks}

%%
%% The next two lines define the bibliography style to be used, and
%% the bibliography file.
\bibliographystyle{ACM-Reference-Format}
\bibliography{reference}

%%
%% If your work has an appendix, this is the place to put it.
\appendix


\end{document}
\endinput
%%
%% End of file `sample-authordraft.tex'.
