\section{Threats to validity}
Potential threats to validity in our user study for performance evaluation primarily stem from subjectivity or bias inherent in participants' abilities and backgrounds. 
Similarly, the experiment on GUI grouping accuracy could be influenced by individual interpretations of `logically related UI components.' We have taken measures to mitigate these threats: participants are all GUI development professionals with diverse skills and experiences. We've provided comprehensive examples of 'logically related' components pre-experiment to ensure a shared understanding. We refrain from revealing which results are ours or the baselines' during the user study. Moreover, results are objectively assessed by comparing the mIoU to the redesigned native TV GUI, with tool performance evaluated via a blend of user study and automated evaluation results.

In the usefulness evaluation, we primarily relied on `satisfaction score' and implementation time as the evaluative metrics for our approach and the baselines. This may not fully encapsulate all elements pertinent to GUI usability and aesthetics. 
However, while the `satisfaction score' is central to our usefulness evaluation, we employ other metrics in separate experiments designed to measure different aspects of our tool. Collectively, these various metrics corroborate the effectiveness of our proposed tool, contributing to a comprehensive assessment of its performance.
To broaden the scope of the tool's efficacy evaluation, we intend to introduce additional valid metrics in future work.
The diversity in development environments and backgrounds of developers across various regions and organizations poses a potential threat to our study's validity, as we may unintentionally miss certain perspectives. To address this, we actively sought participants from varied genders, nationalities, and affiliations. In future research phases, we plan to augment our participant base, especially targeting underrepresented sectors and regions, to strengthen the generalizability of our findings.

\begin{comment}
    
The existence of a small number of online service apps that leverage web technologies for cross-platform adaption poses an additional threat to the validity of user studies. Some cross-platform frameworks, such as Flutter~\cite{Flutter} and React JS~\cite{React}, claim support for Android TV adaptation. 
However, we discovered that there are no comparable apps for Android TV. 
Therefore, we did not collect any valid online service-based phone-TV app pairs.
There may be new GUI features in online service apps. 
Thus we will continue to collect online service apps in our future research and study their features to improve our algorithm.

The major internal validity of the conversion pipeline is that some GUI components are difficult to classify and hence challenging to convert appropriately according to the GUI category.
In the GUI grouping phase, there may be GUI pages that do not follow the Android GUI design principles, resulting in mistakes in GUI groups, which may cause errors in subsequent group classification and mapping. 
Therefore, pipeline errors may be introduced throughout these steps of our semi-automated TV GUI generation pipeline due to the possibility of an error during GUI recognition and grouping.
Errors in the GUI recognition and GUI grouping phrases may ultimately result in unsatisfactory overall performance.
Current data collection tools (UI Automator) cannot accurately obtain metadata of some third-party GUI and $WebView$ views, which further makes it difficult to classify.
To mitigate the validity, we use Grid Layout as the default template to map these unrecognized GUI groups, which can be seen in Table~\ref{tab:matchRules}.
Grid Layout templates can ensure that the information on mobile phones is not lost after mapping and basically meet the user experience.

The major threat to the external validity of the conversion pipeline is the fragmentation of Android devices. 
The screen, OS version, and UI styles of Android devices are markedly different and difficult to unify.
To mitigate this issue, we design language- and platform-free GUI DSL for code synthesis.
The DSL translator pre-installed on the relevant TV converts the DSL into the GUI source code based on the TV's properties.
Improper code translation may potentially result in a poor final GUI conversion. Consequently, we define the TV GUI libraries in advance and employ quality-checked GUI libraries to alleviate this potential threat.
\end{comment}

In our conversion pipeline, challenges arise from accurately classifying certain GUI components and potential deviations from Android design principles, leading to subsequent mapping errors. Additionally, current tools like UI Automator may not effectively capture metadata for specific third-party GUIs and $WebView$ views. To address these internal validity concerns, we default to a Grid Layout for unclassified GUI groups, ensuring information consistency, as described in Table~\ref{tab:matchRules}. Externally, Android's device fragmentation poses threats. To mitigate this, we utilize a universal GUI DSL for code synthesis, with a DSL translator tailoring the conversion to each TV's specifications, while leveraging predefined and quality-assured TV GUI libraries to enhance conversion reliability.


\begin{comment}
The potential generalization vulnerability of the UI grouping and conversion procedures based on heuristic rules is also a threat to validity.
Our method, while primarily tailored to a select set of GUI paradigms highlighted in the introduction, may have constraints in its universal application.
Notwithstanding these limitations, our expansive empirical studies across diverse GUI designs affirm the method's robustness in addressing a large segment of the app ecosystem. 
In Section~\ref{sec: appCollection}, we delineated our data collection process wherein we amassed a set of all available 2,805 TV apps from platforms Google Play and Dangbei. Subsequent empirical analyses were conducted on this dataset to substantiate that our methodologies and insights are applicable to prevailing GUI design paradigms. 
Our empirical studies in Section~\ref{sec:groupAlignment} also spanned a broad range of 1,405 TV-phone app pairs, yielding promising outcomes in our manually verified 589 real-world TV-phone GUI pairs, which suggests that while our approach might be specialized, it remains effective for a substantial segment of the app ecosystem. This efficacy, combined with our systematic pipeline for GUI conversion, provides a robust foundation for future endeavors.
we've evaluated our heuristics using real-world apps as detailed in Section~\ref{sec:groupEva} to demonstrate the high applicability of our rules in current mainstream GUI pages.
With Google's emphasis on card-like layouts for Android TV apps, our default grid layout-based template addresses rare UI types, and our implementation of OR constraints offers flexibility beyond prescriptive rules, ensuring adaptability.
Our systematic framework, while primarily designed for mobile-to-TV GUI conversions, acts as a cornerstone for prospective expansion into areas like mobile-to-smartwatch or augmented reality. As the GUI design paradigms continue to evolve, our infrastructure is well-positioned for integrating pioneering stipulations or essential modifications. Avenues for future research may delve into the scalability of our approach across varied platforms, underpinned by the foundational benchmarks we've established. Furthermore, we envisage utilizing deep learning to enhance UI grouping, gathering categorized components from diverse devices, and optimizing models for automated domain-specific GUI component synchronization.
\end{comment}

A potential limitation in the generalization of our UI grouping and conversion based on heuristic rules presents a threat to validity. Though our method is designed for specific GUI paradigms, empirical studies across varied GUI designs attest to its efficacy in a considerable portion of the app ecosystem. In Section~\ref{sec: appCollection}, we outline our data collection from Google Play and Dangbei, resulting in 2,805 TV apps. Our subsequent analysis, detailed in Section~\ref{sec:groupAlignment}, covered 1,405 TV-phone app pairs and yielded favorable results in our manual verification of 589 real-world GUI pairs. Despite its specificity, our method's effectiveness is evident in a significant app ecosystem segment. Google's Android TV apps, emphasizing card-like layouts, align with our grid layout template, ensuring adaptability with OR constraints. Looking ahead, our framework, optimized for mobile-to-TV conversions, serves as a robust foundation for extending into areas like mobile-to-smartwatch or augmented reality, as GUI paradigms evolve. Future research could focus on our method's scalability and potential integration of deep learning for enhanced UI grouping and component synchronization.
