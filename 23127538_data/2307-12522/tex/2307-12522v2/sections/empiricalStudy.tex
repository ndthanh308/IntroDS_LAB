\section{Empirical study of GUIs between phones and TVs}
\label{sec:empiricalStudy}
%Figure~\ref{fig:empirical} shows the whole process of our empirical study.
To better understand the current status and characteristics of GUI adaptation to smart TV, we conduct a large-scale empirical study to answer two questions: 1) How many phone apps support TV display? 2) How do app GUIs change between TVs and smartphones?

\subsection{RQ1 How many phone apps support TV displays?}
\label{sec:supportTV}

To answer RQ1, we analyze a large number of industrial Android apps from all 33 categories on Google Play~\cite{googleplay}.
We crawl the top 200 most popular apps in each category (at the time of Jun 2021).
Since some apps are not free, we obtain 5,580 apps that support Android phones by default.

According to the official guidelines of developing Android TV~\cite{AndroidTVDe}, it is compulsory for apps running on TV to declare a TV activity with an intent filter \textit{CATEGORY\_LEANBACK\_LAUNCHER} in the Android manifest file of Android projects.
In addition to the declaration, TV-enabled apps often have separate TV layouts called \textit{layout-television} and \textit{layout-tv}.
Therefore, we decompile Android APKs to check whether the specified intent filters or layout XML files exist.
We find only 298 of these apps supporting TV display, accounting for only 5.34\% of the total.
These 298 apps belong to 29 categories including \textit{Weather} (16.78\%), \textit{Education} (15.10\%), \textit{Tool} (11.07\%).
% Most apps that do not support TV will fail to install or run on the TV  if they are forced to do so owing to hardware and Android system version incompatibility.
 
Even if an app advertises that it supports a TV display, this does not imply that it is appropriately optimized for the TV.
We manually check 298 apps that claim to support TV displays.
We physically run the apps on a smart TV and critically evaluate how well various GUI pages adapt to the TV display. GUI pages exhibiting discrepancies such as substantial black margins at the screen's bilateral extents, discordant aspect ratio components, disordered layouts, and unoptimized navigation, among other factors -- which all may diminish the user experience on TV displays -- are regarded as inadequate instances of the TV adaptation.
We discover only 11 of 298 apps allow TV display on all GUI pages.
287 out of 298 apps modify a few representative GUI pages, such as the home or landing page, to accommodate TV displays.
These 287 apps have an average of 22.3 Android activities, but we only find support for TV display in an average of 4 Android activities.
%\textit{layout-land-television-v8.xml}, \textit{HomeActivity-TV.xml} and \textit{movie-television.xml}. 
Other GUIs of these apps also look poor on TV, with large black margins on both screen sides, and mismatched aspect ratio components, as the \emph{Direct mapping} in Figure~\ref{fig:currentExamples}.
% \todo{This seems a bit out of placed with the rest of the above - can you comment on number that seem to have custom TV apps - maybe these are the 'good' looking ones??}
Due to the significant difference between phone and TV displays, some development teams tend to develop separate apps for different platforms.

% Figure environment removed

\subsection{RQ2 How do app GUIs change between TVs and smartphones?}
\label{sec:rq2}
We collected custom-designed TV apps and their matching phone apps for a comparative study since there are relatively few apps that support both TV and phone.

\subsubsection{TV-phone App Pairs Collection}
\label{sec: appCollection}
% In Google Play, there is a specific category~\cite{googleTV} of apps supporting Android TV.
% Dangbei~\cite{dangbei} is one of the largest TV app stores.
We collected 249 TV apps from Google Play's TV category~\cite{googleTV} and 2,556 from Dangbei~\cite{dangbei}, which is one of the largest TV app stores. We eliminated any apps that have not been updated for over two years.

We then  matched the TV apps' corresponding smartphone apps on Google Play.
To begin, we search for phone apps with the same app name, developers, and category.
Second, if the corresponding phone version app cannot be found, we broaden our search to include apps with similar app names developed by the same developers.
Finally, for the TV apps that do not find the phone-version apps, four volunteers manually collect their matching phone apps.

%TV games are all deleted for no applicable GUI and match phone version game.
We match 1,405 TV-phone app pairs, with the three most common categories being \emph{Video (42\%)}, \emph{Education (23\%)} and \emph{Tool (21\%)}.
Video apps, such as \textit{Youtube}, \textit{IQIYI}, are most suitable for smart TVs due to the characteristics of TV itself, so video apps have become the most popular apps on TV.
% Educational apps, especially for children, are also ubiquitous on TVs, yet the GUIs of educational apps on phones and TVs are typically somewhat different.
% Some tools, like TV app store, remote control and projection control are also widely used in TV.
Because smart TVs are mostly used at home, there are a number of educational apps for children.
Tool apps, like TV app store, remote control and projection control are also ubiquitous on TVs.

% \textbf{Alignment between TV and Phone GUIs}
\subsubsection{TV-phone GUI Pair Collection}
\label{sec:guiCollection}
We use the DroitBot~\cite{li2017droidbot}, Fastbot~\cite{cai2020fastbot} and Uiautomator2~\cite{uiautomator2} to automatically explore apps and collect rendered screenshots and metadata of GUIs in apps.
%Four volunteers manually traverse rendered app GUIs in each app to collect screen information using custom tools we built. 
The metadata is a documentary object model (DOM) tree of current GUIs, which includes the hierarchy and properties (e.g., class, bounding box, layout) of UI components. 
We can infer the GUI hierarchy from the DOM Tree hierarchy in metadata.
After removing duplicates, we obtain 6,697 Android GUI data and 4,112 TV GUI data.
We notice that most TV apps simplify or restructure their GUIs to accommodate various usage scenarios and requirements of smart TVs during this process.

At TV-phone GUI pairing, we exploit the semantic similarity of Android activity names to pair the TV and phone GUIs automatically.
Then, we automatically compare UI components on GUI pages in order to match state-level GUI pairings in a lesser granularity.
Finally, we manually check the automatically discovered pairs and select the final valid TV-phone pairs.
We extract activity names from each GUI and encode them into numerical semantic vectors using a pre-trained BERT~\cite{devlin2019bert} model.
Then, we match the TV-phone GUI pairs by comparing their semantically close activity vectors.
For example, the GUI in activity \textit{homeActivity} and \textit{mainActivity} are matched by close semantic vectors.
However, one Android activity may have multiple Android fragments~\cite{fragment} and GUI states~\cite{machiry2013dynodroid, GUIstate} with different UI components and layouts in current industrial apps.
We further compare GUI components between phones and TVs to pair TV-phone GUIs at lower granularity.
In pairing, UI components are identified by their types and properties.
UI components between phones and TVs  with the same types and properties are considered paired GUI components.
For example, two \emph{TextViews} with the same texts, two \emph{ImageViews} with the same images, two \emph{Buttons} with the same texts are considered the paired components.
If more than half of the UI components in two GUIs are paired, they are considered a state-level TV-phone GUI pair.
% When faced with multiple interface activities, like \textit{tablayout}, we further trace the selected text in top/bottom tabs as current semantic strings to match the corresponding GUI.
Finally, we manually checked all discovered pairs and identified 589 TV-phone state-level GUI pairs with clear GUI correspondence between phone and TV components.


\subsubsection{GUI component grouping}
\label{sec:GUIgroup}
A series of UI components that are near in position and hierarchy of the GUI and tend to have the same functionality are referred to as GUI groups\footnote{Sometimes called GUI patterns}~\cite{guiBuild, neil2014mobile}.
Exploring these group changes based on UI groups rather than each individual component is more beneficial~\cite{neil2014mobile}.
So, we must first identify and categorize GUI groups that are common to both TV and phone before summarising the guidelines for GUI changes from phone to TV.

We perform an iterative-open coding process, which is widely used to generate categories in Software Engineering~\cite{seaman1999qualitative, hu2023first, hu2023look, chen2021my}. We do this on 120 randomly selected phone GUIs and 80 TV GUIs (approximately 2\% of collected phone and TV GUIs) to categorize their GUI groups.
Four volunteers with Android design experience undertook three steps in our open coding procedure.
At first, we are inspired by Google design~\cite{GoogleMaterial} and development guidelines~\cite{androidGuidelines}, every volunteer categorizes GUI groups in selected GUIs individually.
After the initial coding, let four volunteers have a discussion and merge conflicts.
They clarify scope boundaries among categories and misunderstandings in this step.
In the third step, they iterate to revise classifications and discuss with each other until a consensus is reached.
Finally, we determined 12 phone group types: \textit{Icon + Info}, \textit{Tool Bar},  \textit{Bottom Tab Layout}, \textit{Search}, \textit{Top Tab Layout}, \textit{Pic Side Info}, \textit{Pic + Info}, \textit{Side Nav}, \textit{Short Video Player}, \textit{Video/Music Player}, \textit{Big Pic} and \textit{List View} and 9 TV group types: \textit{Icon + Info}, \textit{Tool Bar}, \textit{Search}, \textit{Tab Layout}, \textit{Channel}, \textit{Grid Layout}, \textit{Pic + Info}, \textit{Video/Music Player} and \textit{List View}.
Figure~\ref{fig:phoneGroups} and \ref{fig:TVGroups} show examples of summarized phone and TV groups.
These phone and TV GUI group categories can be divided into two subcategories.
GUI groups in the first subcategories are widely used in both phone and TV, e.g. \textit{Tool Bar}, \textit{Search}, and \textit{Video/Music Player}.
GUI groups in the second subcategory only exist in phones or TVs respectively, e.g. such as \textit{Bottom Tab Layout} and \textit{Channel}.

To verify the accuracy of GUI grouping and classification, we first randomly sample 20 TV and smartphone apps in collected 1405 TV-phone app pairs in Section~\ref{sec: appCollection}, respectively.
The selected apps have covered all 6 TV app categories (Tool, Video, Music \& Audio, Education, Entertainment, and Weather) listed on Google Play~\cite{ggTV}.
Then, we manually calculate the distribution of each GUI group in selected TV and smartphone apps.
Finally, we find that these summarized GUI components groups have covered 93.69\% and 93.47\% of GUI groups on phones and TVs, respectively.
In our analysis, we observe the presence of uniquely shaped GUI components in both phone and TV interfaces. These components resist typical categorization due to their complex functional requirements in production and everyday use. They constitute 6.31\% and 6.53\% of phone and TV GUIs, respectively. These specific GUI components primarily appear within individual apps or among apps from the same developer. We categorize these components under an $Others$ group category for the purpose of our study. To address these edge cases in the GUI conversion process, we propose the development of default templates. Given the relatively minor percentage of GUI components in the $Others$ category and their limited appearance in specific applications, we assert that our categorization effectively represents the vast majority of commonly used phone and TV GUIs.

Table~\ref{tab:matchRules} shows the details of GUI group distributions.
Subcolumns \emph{Group} and \emph{Distribution} of columns \emph{Phone} and \emph{TV} denote the GUI group categories and distribution in the experiment.
TV's subcolumn \emph{Group} does not contain the \emph{Others} GUI group, which comprises 6.53\% of all TV GUIs in the experiment.
On phones, the most popular categories of components groups are \textit{Icon + Info} (13.31\%), \textit{Tool Bar} (11.41\%) and \textit{List View} (11.14\%), but on TV, categories \textit{Pic + Info} (19.13\%), \textit{Grid Layout} (13.37\%) and \textit{List View} (13.18\%) are most common categories.
The official guideline of TV GUI design~\cite{AndroidTVDe} suggests two principles for TV GUI design \emph{All TV GUIs should display in landscape mode} and
\emph{The core TV GUIs use card-like views instead of ListView or ViewPager to make better use of horizontal screen space and accommodate TV interaction}.
Standard current TV GUIs obey these two principles to use more grid layouts and card-like widgets.
Thus card-like categories \textit{Pic + Info} and \textit{Grid Layout} are more popular on TVs than on phones.



% TV UIs are relatively large, so \textit{Icon+info} has dropped from 13.31\% on phones to 6.85\% on TV.
%Most TV apps are audio-visual, so \textit{Video/Music Player} has increased from 3.50\%  to 10.56\% on TV.
% TV is not suitable for finger sliding and click, resulting in the decline of group \textit{Short Video Player}, \textit{Bottom Tab Layout} and \textit{Pic Side Info}.
% \textit{Bottom Tab Layout} has disappeared on TV too, because it in the TV is not convenient for users to click and switch.
% TV app always move phone's contents in \textit{Bottom Tab Layout} to \textit{Channel}.
% Pictures in TV either has no related information or the related information is under the picture. 
% It rarely happens that information is on both sides of the picture, so \textit{Pic Side Info} is rarely seen on TV.



% Figure environment removed
\setlength{\textfloatsep}{10pt}

\subsubsection{Component group alignment}
\label{sec:groupAlignment}
After conducting our GUI components grouping study, we notice that the design principles of TV and phone GUIs are vastly different, resulting in no obvious one-to-one alignments between most TV-phone GUI groups.
Furthermore, the contents of one phone GUI group may be dispersed throughout numerous groups in the TV GUI, and vice versa.
Therefore, we summarize heuristic rules from our collected TV-phone GUI pairs for automatic GUI mapping from phone to TV.

Firstly, we randomly divide the 589 TV-phone GUI pairs into experimental and validation sets in an 8:2 ratio.
In the experimental set, four volunteers follow the same three steps to perform open coding to analyze and extract conversion rules.
Table~\ref{tab:matchRules} shows the extracted GUI group match rules from phone to TV.
To accommodate how the TV and remote interact, each TV GUI group uses card-like views.
The converted TV GUI group recalculates the new size depending on the quantity and types of components in the existing GUI to fit the TV screen size. 
According to the GUI group study in Section~\ref{sec:GUIgroup}, we use \textit{Grid Layout} as our default template for mapping.
Component groups with the same meaning \textit{Icon + Info}, \textit{Tool Bar}, \textit{Search}, \textit{Top Tab Layout}, \textit{Video/Music Player} and \textit{List View} in phone and TV are transferred directly.
According to the characteristics and the official design guideline~\cite{AndroidTVDe} of TV, \textit{Pic Side Info}, \textit{Pic + Info} and \textit{Big Pic} are all converted to \textit{Pic+Info} in TV.
After exploration, current TV GUIs tend to replace components in the phone's \textit{Side Navigation} and \textit{Bottom Tab Layout} to \textit{Channel} in TV,
so we follow this trend.
\textit{Short Video Player} should use customized templates in TV, but there's no such TV app with this GUI feature at the moment.
As a result, we don't provide the corresponding TV group individually at the moment, instead relying on \textit{Video Player} to convert.

We use the validation set to verify these mapping rules.
Our first step involves manually identifying and extracting GUI components from respective groups within the phone GUI. 
Subsequently, we locate corresponding components in the matching TV GUI. 
Our final process entails verifying if these TV GUI components align with the anticipated TV groups and comply with the matching rules established in our experimental set. 
Given \emph{m} instances of \emph{Side Nav} groups in our validation set, and $n$ corresponding TV GUI groups classified as $Channel$ groups, we compute the mapping rule accuracy as the ratio $n/m$.
Note that if the phone GUI group is eliminated in the corresponding TV GUI, the case is considered invalid and will not be counted.
The \emph{Mapping Accuracy} in Table~\ref{tab:matchRules} demonstrates the correctness rate of each mapping rule.
Finally, we find that the correctness of rules 1, 2, 3, 4, 5 and 7 are 96\%, 99\%, 91\%, 99\%, 99\%, 99\% and 100\%, respectively, indicating that these direct mapping rules are accurate and universal.
Rules 8, 9, 10, 11, and 12 have an accuracy of 83\%, 95\%, and 87\%, 90\%, and 99\%, respectively, suggesting these change rules are also accurate and common.

\begin{table}[!htbp]
\setlength{\abovecaptionskip}{0pt}
\setlength{\belowcaptionskip}{0pt}
\caption{Component group matching between phone and TV. TV's subcolumn \emph{Group} does not contain the \emph{Others} GUI group, which comprises 6.53\% of all TV GUIs in the experiment.}
\scalebox{0.8}{
\begin{tabular}{|c|cc|cc|c|}
\toprule
\multirow{2}{*}{\textbf{Index}} & \multicolumn{2}{c|}{\textbf{Phone}}     & \multicolumn{2}{c|}{\textbf{TV}}  & \multirow{2}{*}{\textbf{Mapping Accuracy}}                     \\
~ & Group & Distribution & Group & Distribution & \\

\midrule
1 & Icon + Info  & 13.31\%       & Icon + Info  & 8.32\%  & 96\%                   \\
2 & Tool Bar  & 11.41\%          & Tool Bar    & 7.83\%   & 99\%                   \\
3 & List View      & 11.14\%     & List View          & 13.18\%  & 91\%             \\
4 & Top Tab Layout  & 8.88\%    & Top Tab Layout   & 7.68\%  & 99\%               \\
5 & Search    & 7.98\%          & Search       & 7.12\%  & 99\%                   \\
6 & Others & 6.31\% & Grid Layout (Default) & 13.37\% & 90\%  \\
7 & Video/Music Player & 3.50\% & Video/Music Player    & 7.56\%  & 100\%          \\

\midrule
8 & Pic Side Info  & 8.90\%     & \multirow{3}{*}{Pic + Info} & \multirow{3}{*}{19.13\%}  & 83\%   \\
9 & Pic + Info   & 8.67\% &             &         & 95\%             \\
10 & Big Pic     & 3.52\%         &        &      & 87\%                     \\

\midrule
11 & Bottom Tab Layout   & 10.31\% & \multirow{2}{*}{Channel}    & \multirow{2}{*}{9.28\%}   & 90\%   \\
12 & Side Nav  & 6.07\%  &             &        & 99\%             \\
\bottomrule
\end{tabular}%
}
\label{tab:matchRules}
\end{table}


\subsection{Summary and Implications}
\label{sec:implication}
Our empirical study shows that:
(1) Only 5.34\% of popular phone apps support TV displays. 
% IT companies are more inclined to develop a new TV app, which may lead to repeated efforts.
(2) In TV-phone GUI pairs, there is not much explicit one-to-one correspondence between phone and TV component groups.
(3) We summarize 12 and 9 categories of GUI components groups on phone and TV, covering 93.69\% and 93.47\% popular phone and TV GUIs, respectively. 
(4) We extract and evaluate 12 existing GUI group-mapping rules from phone to TV based on summarised GUI component groups.

% \todo{changed to semi-automated generation! :-) }
The lack of TV-display support for phone apps confirms the necessity of tool development for semi-automated GUI mapping between phone and TV.
That motivates our study and the empirical findings of component group mapping are the backbone of our proposed approach.