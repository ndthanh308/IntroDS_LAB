\section{Conclusion}
At the moment, adaptive technologies between phone and TV GUI are unable to fulfill the demands of app developers, and the cost of developing and maintaining new applications is prohibitive, therefore building an automated GUI conversion tool from phone to TV is challenging but worthwhile work for developers.
An automated approach for converting phone GUI to TV GUI is presented in this study.
Before proposing our approach, we carry out empirical studies to explore how many current apps support TV displays and how current apps convert phone GUI to TV GUI.
Our tool consists of four integral stages: GUI components grouping, template matching, layout optimization, and DSL for code synthesis.
Finally, we convert the generated GUI DSL to source code for rendering the final TV GUIs.
Our approach offers obvious benefits over existing mainstream technologies, according to an automated evaluation in 589 valid phone-TV GUI pairs and a user study from 20 Android professionals.
Besides, a pilot user study also illustrates the usefulness of our tool for app developers.

In the future, we will keep improving our algorithm for generating mapping GUIs from phone to TV.
With more and more apps developed for supporting TV, we will construct a large parallel corpus of TV-phone GUI pairs.
Based on that data, we will develop an end-to-end machine learning algorithm that will be more generalized than the current approach.
On the other hand, we will extend our tool to other platforms such as smartwatches, tablets, and vehicle screens.

\section*{Acknowledgements}

Grundy is supported by ARC Laureate Fellowship FL190100035.
