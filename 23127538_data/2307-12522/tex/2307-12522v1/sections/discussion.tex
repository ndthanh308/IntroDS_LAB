\section{Threats to validity}
\refix{Potential threats to validity in our user study for performance evaluation primarily stem from subjectivity or bias inherent in participants' abilities and backgrounds. Similarly, the experiment on GUI grouping accuracy could be influenced by individual interpretations of `logically related UI components.' We have taken measures to mitigate these threats: participants are all GUI development professionals with diverse skills and experiences. We've provided comprehensive examples of 'logically related' components pre-experiment to ensure a shared understanding. We refrain from revealing which results are ours or the baselines' during the user study. Moreover, results are objectively assessed by comparing the mIoU to the redesigned native TV GUI, with tool performance evaluated via a blend of user study and automated evaluation results.
In the usefulness evaluation, we primarily relied on `satisfaction score' and implementation time as the evaluative metrics for our approach and the baselines. This may not fully encapsulate all elements pertinent to GUI usability and aesthetics. 
However, while the `satisfaction score' is central to our usefulness evaluation, we employ other metrics in separate experiments designed to measure different aspects of our tool. Collectively, these various metrics corroborate the effectiveness of our proposed tool, contributing to a comprehensive assessment of its performance.
To broaden the scope of the tool's efficacy evaluation, we intend to introduce additional valid metrics in future work.}

The existence of a small number of online service apps that leverage web technologies for cross-platform adaption poses an additional threat to the validity of user studies. Some cross-platform frameworks, such as Flutter~\cite{Flutter} and React JS~\cite{React}, claim support for Android TV adaptation. 
However, we discovered that there are no comparable apps for Android TV. 
Therefore, we did not collect any valid online service-based phone-TV app pairs.
There may be new GUI features in online service apps. 
Thus we will continue to collect online service apps in our future research and study their features to improve our algorithm.

The major internal validity of the conversion pipeline is that some GUI components are difficult to classify and hence challenging to convert appropriately according to the GUI category.
In the GUI grouping phase, there may be GUI pages that do not follow the Android GUI design principles, resulting in mistakes in GUI groups, which may cause errors in subsequent group classification and mapping. 
Therefore, pipeline errors may be introduced throughout these steps of our semi-automated TV GUI generation pipeline due to the possibility of an error during GUI recognition and grouping.
Errors in the GUI recognition and GUI grouping phrases may ultimately result in unsatisfactory overall performance.
Current data collection tools (UI Automator) cannot accurately obtain metadata of some third-party GUI and $WebView$ views, which further makes it difficult to classify.
To mitigate the validity, we use Grid Layout as the default template to map these unrecognized GUI groups, which can be seen in Table~\ref{tab:matchRules}.
Grid Layout templates can ensure that the information on mobile phones is not lost after mapping and basically meet the user experience.

The major threat to the external validity of the conversion pipeline is the fragmentation of Android devices. 
The screen, OS version, and UI styles of Android devices are markedly different and difficult to unify.
To mitigate this issue, we design language- and platform-free GUI DSL for code synthesis.
The DSL translator pre-installed on the relevant TV converts the DSL into the GUI source code based on the TV's properties.
Improper code translation may potentially result in a poor final GUI conversion. Consequently, we define the TV GUI libraries in advance and employ quality-checked GUI libraries to alleviate this potential threat.


The potential generalization vulnerability of the UI grouping and conversion procedures based on heuristic rules is also a threat to validity.
GUI developers may not follow the prevalent UI layout guidelines when designing GUI pages, which may cause our grouping heuristics to become ineffective. 
To mitigate the threat and ensure that our approaches are applicable to all existing phone-TV GUI pair types, we evaluate our heuristics in external real-world apps in Section~\ref{sec:groupEva}.
Google advises that the Android TV app's UI be designed with a card-like layout, thus we implement a default grid layout-based template for dealing with the unexpected rare UI types. 
Additionally, we provide OR constraints for Android TV. These constraints are more flexible than rules and automatically calculate TV GUI layouts that correspond to the present GUI. 
The use of OR constraints broadens the generalizability of our method.
Our approach also proposes to convert GUI to a cross-platform GUI DSL that allows programmers to further customize the source code for various platforms and versions. 
In exceptional circumstances, the developer can rewrite a portion of the source code to conform to the current GUI layout specifications based on the generated DSL.
To improve the efficiency of UI grouping, we plan to collect the grouped UI components on mobile devices and TVs, train a deep learning model, and allow the deep learning model to automatically match GUI components of the same domain.