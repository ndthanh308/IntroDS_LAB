\section{Introduction}
Graphical User Interfaces (GUI) are ubiquitous in almost all modern desktop software, mobile applications (apps), and online websites.
They provide a visual bridge between a software application and end-users through which they can interact with each other.
A good visual design makes an application attractive and easy to use, which significantly affects the success of the application and the loyalty of its users~\cite{goldschmidt1994visual}.
With the pervasiveness and diversity of devices, especially mobile ones, users tend to use the same app across different platforms in different scenarios (e.g., watching YouTube on a phone on the bus while using it on a tablet or TV at home)~\cite{joorabchi2013real}.
\refix{When utilizing apps on TV and tablets, especially those within the media, education, and tools categories, the adaptation to a large-screen TV/tablet display substantially elevates the user experience.}
Therefore, to ensure usability,  software GUIs need to be adaptive to various screen sizes and densities, ranging from small 4.7-inch smartwatches to 75-inch smart TVs.


%Problem
Although one app may run well on different platforms with the same operating system (e.g., Android), the content may not display well without customization for different devices or screens.
For example, one app designed for the Android phone will only be displayed in the middle of the TV screen with a large dark space on the two sides due to the different aspect ratios as seen in Figure~\ref{fig:currentExamples} (\emph{Direct mapping}). 
Additionally, smart TVs and mobile phones interact differently. 
Mobile phones employ fingers to touch and swipe, while smart TVs require remote controls.
TVs need redesigned GUIs to accommodate different interactions as well as differences in size and layout.

Currently, there are three ways commonly used to deal with this issue.
First, many GUI development frameworks provide support of responsive and adaptive design~\cite{adaptiveUI} for developers to adapt the GUI to any possible screen size, such as Android's Material Design~\cite{laine2021responsive} and iOS\cite{bhuttoo2017responsive}.
The \emph{Desktop mode} in Figure~\ref{fig:currentExamples} shows an example of the responsive layout.
Since responsive and adaptive layout techniques simply adapt the GUI to the current device's screen size based on the size constraints and proportions of the original phone screen, the GUI has to clumsily enlarge some UI components, increase the distance between UI components, or alter the layout of UI components.
Although the GUI occupies the current screen, its layout is quite clumsy and significantly diminishes the user experience.
Responsive design is primarily focused on optimizing for small screens like smartphones and tablets. Therefore, it may not be optimized for larger screens like TVs, leading to issues when viewing mobile apps on TVs. Additionally, TVs often have different interaction methods than mobile devices, such as a remote control or voice commands, which may not be supported by the app's responsive design.

% To some extent, the adaptive responsive design may work well on different screens (e.g., phone to tablet) but cannot adapt to a large difference with different interactions (e.g., phone to TV). 

% Figure environment removed

% Desktop Mode is provided by phone manufacturers like Samsung and Huawei to optimize large screen GUI.
%     Direct mapping is running phone app on large screen smart TV.

Second, there are some screen mirroring apps (e.g., Chromecast~\cite{chromecast}), which are used to cast music, video, play games, and display photos on a big screen.
However, these are effective primarily for video projection rather than general app screens.
Phone manufacturers like Samsung, Apple, Huawei, and OnePlus have introduced the concept of Desktop Mode~\cite{huaweiDesk, Dex} for cross-screen GUI conversion via their phones.
Desktop Mode maps the current phone GUIs to larger external screens via an HDMI adaptor or WiFi by redesigning GUI layouts individually to improve the user experience with a larger screen.

%Limitation of existing approaches
Third, as revealed in our empirical study in Section~\ref{sec:supportTV}, developers often create a brand-new GUI for the corresponding TV application due to their significant differences.
While there are only a small number of thousands of apps supporting Android TV~\cite{AndroidTVApp}\footnote{Some apps are specifically developed for TV and cannot run on the Phone.},
this is a minuscule fraction compared with the millions of Android apps, which greatly limits users' app usage on the TV.
Developing both a mobile app and a standalone TV app from scratch is both time-consuming and labor-intensive.
Considering the user side, a comparable design could involve providing data or research on user preferences for consistency across different devices. This method reduces the need for them to learn new navigation and interaction patterns~\cite{li2017droidbot, lowdermilk2013user}.
Popular apps always share a similar design between phone apps and TV apps, for example, YouTube and Spotify. 
From the developer side, a consistent GUI can reduce development costs and time, as it allows developers to reuse code and design elements across different devices~\cite{lowdermilk2013user}.
Both the phone app and the corresponding TV app have similar functions and can share current designs and resources.
Reusing existing materials can result in significant savings on engineering expenses, and a similar user interface makes it easier for mobile app users to adapt to the TV app~\cite{mendoza2013mobile, lowdermilk2013user}.




%Our solution and Benefits
To overcome these limitations, we propose an approach for generating the appropriate GUI layout for Android TV based on the existing GUI design for phones.
Unlike responsive design, where a screen 'flows' from a phone design into a larger device, adaptive projection in our approach offers tailor-made solutions. 
Given the app GUI of a page on a mobile phone, our approach automatically generates the corresponding GUI code to make it adaptive to the TV screen.
The development team, including the visual designers and developers, will benefit from our approach.
Once they finish the phone app development, our approach can automatically generate the GUI design and TV implementation for any phone page.
Developers and designers can easily customize it to enhance their productivity without starting from scratch.

%empirical study
To build the tool, we first investigate how many apps currently support smart TV in 5,580 popular apps from Google Play\cite{googleplay}.
We find only 5.34\% of apps support running on smart TV.
Second, we perform a formative study on the characteristics of current TV-phone apps and GUIs.
We collect 1405 TV-phone app pairs from Google Play and Dangbei\cite{dangbei} but only collect 589 TV-phone GUI pairs with clear GUI correspondence in these app pairs.
We then summarize 12 and 9 categories of GUI components groups common on mobile and TV, respectively, and analyzed their corresponding conversion.
%summarisation of the approach 
Based on the empirical study results, we build our tool in the following steps: group isolated GUI components parsed from GUI metadata captured by UI Automator~\cite{uiautomator}, convert phone GUI groups into corresponding TV groups, optimize their layouts by OR constraints formulas~\cite{jiang2019orc}, and translate current TV GUI to our language - and platform free GUI domain specific language (DSL) for further rendering. 

%evaluation summerization
We choose two of the best-known and most commonly used technologies: desktop mode and direct mapping as the baselines for the evaluation.
We select mIoU, overall satisfaction, and structure rationality as the metrics to evaluate the converted TV GUIs by our approach and baselines.
We also recruited 20 participants with extensive experience in Android GUI development, design, and use to evaluate the converted TV GUIs. 
The experimental results show that TV GUIs converted by our approach achieve 21.05\%, 10.42\%, and 21.31\% improvement in mIoU, overall satisfaction, and structure rationality than the best current conversion techniques.
Besides, a pilot user study also provides the initial evidence of the usefulness of our tool for bootstrapping adaptive GUI from phone to TV.

%contribution list
Our contributions to this work are summarized as follows:
\begin{itemize}    
    \item As far as we know, this is the first study on automated GUI conversion from smartphone to TV GUI; 
    
    \item We propose a new approach to generate TV-hosted GUIs from Android mobile phone GUIs;
    
    \item We carry out an empirical study to understand the current status of GUI support in the phone app to TV display and how is the GUI mapping patterns between two platforms; and
    
    \item We demonstrate the effectiveness of our approach with extensive automated evaluation and manual checking. We also provide initial evidence of our tool's usefulness via a pilot user study.
\end{itemize}



% In this study, we are exploring a smart GUI transformation from phone to TV.
% We carry out a literature review of this direction including related works about cross-platform app analysis, GUI study, cross-platform app analysis, GUI migration across platforms and practical tools in industry.