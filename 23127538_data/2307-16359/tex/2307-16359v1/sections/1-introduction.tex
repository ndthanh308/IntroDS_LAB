Media bias influences society through its impact on the political voting system, because the enfranchised public learns about political parties and public policy through print, broadcast, and social media \cite{eberl_one_2017}. Media bias can be direct, e.g.\ favoritism towards a party \cite{alizadeh_effect_2015}, or indirect, e.g.\ filtering positive or negative information \cite{barzilai-nahon_gatekeeping_2009,dalessio_media_2000}, or arguing false equivalence \cite{boykoff_balance_2004}.  In the classic definition by Williams, media bias is ``volitional", ``influential" and ``threatening to widely held conventions" \cite{williams_unbiased_1975},  so its capacity to affect voters' opinions is substantial.

The dynamics of people's perceptions of media bias can be modeled through a network approach.  An individual agent's opinion about media bias is shaped directly by consuming media outputs and indirectly by observing the opinions of networked acquaintances about the same media outputs.  Many opinion dynamics models in the literature are deterministic, in the sense that every agent holds a single belief with complete certainty at an instant in time, although of course the belief changes from one instant to the next \cite{degroot_reaching_1974,vaz_martins_mass_2010,shi_evolution_2016,mcquade_social_2019}.  Such studies often focus on belief convergence or asymptotic learning between like-minded agents (``allies'') \cite{degroot_reaching_1974,hegselmann_opinion_2002} although there are exceptions that focus on antagonistic interactions between agents  (``opponents'') \cite{shi_evolution_2016}. In comparison, there is a lesser but growing focus in the media bias literature on probabilistic Bayesian learners, i.e.\ agents who hold a spectrum of uncertain beliefs at an instant in time \cite{low_discerning_2022,low_vacillating_2022,fang_opinion_2020,fang_social_2019}, similar to the real world.  Within the Bayesian framework, an agent's beliefs are represented by a probability distribution, which evolves deterministically in response to stochastic external stimuli (e.g.\ daily newspaper editorials) and peer pressure within the network, leading to drift and diffusion of beliefs.

Many opinion dynamics models treat agents as {\em persuadable}, that is, open to modifying their beliefs in response to their own observations as well as peer pressure from other agents in the network. In contrast, in this paper, we consider the role of partisans: {\em obdurate} agents who refuse to change their opinion, regardless of external inputs or peer pressure. Their obduracy may be subconscious and psychological or conscious and cynical (e.g.\ a deliberately disruptive strategy by politically motivated ``trolls'' on social media).  The idea of partisans was first proposed by Mobilia \& Mauro \cite{mobilia_does_2003}, who introduced the equivalent term ``zealots'' in a deterministic, non-Bayesian framework. Mobilia \& Mauro \cite{mobilia_does_2003} analyzed the inhomogeneous voter model, in which the network is a $d$-dimensional hypercubic lattice. They and other authors found that a single zealot can persuade the rest of the network to converge on the zealot's belief for $d < 3$, and that a relatively small number of zealots can hinder the formation of consensus or even a clear-cut majority opinion for arbitrary $d$ \cite{mobilia_role_2007,liggett_stochastic_nodate, belitsky_mixture_2001}.  Moreover, the long-term disposition of opinions within the network depends on the network graph structure (e.g.\ connectivity or scale invariance) and the opinions and locations of the partisans \cite{yildiz_binary_2013}. 

In this paper, we examine how partisans shape perceptions of media bias in a Bayesian framework for the first time.
The paper generalizes previous opinion dynamics models of media bias among Bayesian learners, in which all the agents are persuadable \cite{low_discerning_2022,low_vacillating_2022,fang_opinion_2020}. 
It also generalizes previous deterministic models involving partisans, such as the voter model \cite{mobilia_does_2003,yildiz_binary_2013}, to the new application of media bias.  The goal is to investigate the disruptive influence of partisans on perceptions of media bias as a function of their number within allies-only, opponents-only, and mixed networks.  The paper is structured as follows. Section \ref{sec:partisans} introduces the opinion update rule for the idealized model of a biased coin introduced in Ref.\ \cite{low_discerning_2022} and explains how partisans are implemented. Monte Carlo simulations are performed on complete networks with allies only, opponents only, and a mixture of allies and opponents in Sections \ref{sec:alliesonly}, \ref{sec:opponentonly}, and \ref{sec:mixed} respectively.  The goal of the simulations is to identify how partisans modify the phenomena observed in networks of Bayesian learners without partisans, such as the time-scale to reach asymptotic learning, the emergence of turbulent nonconvergence and intermittency, and the tendency of agents in opponents-only networks to reach wrong conclusions first \cite{low_discerning_2022,low_vacillating_2022,fang_opinion_2020}.  
A brief, preliminary study of how the above phenomena translate to partly connected networks is included in Sections \ref{subsec:wrong_conclusion_first} and \ref{sec:BA}, as a prelude to future work.
The results are interpreted briefly in terms of the social science theory of structural balance in Section \ref{sec:conclusion}. 