It is natural to ask whether the behavior observed in Sections \ref{sec:alliesonly} -- \ref{sec:mixed} is specific to complete networks. To what extent do the results depend on the network's connectivity?
In Section \ref{subsec:wrong_conclusion_first}, for example, we find that the tendency to reach the wrong conclusion first depends on the attachment parameter in Barab\'{a}si-Albert opponents-only networks.
Moreover, it is plausible intuitively that the influence of a partisan increases, as their connections to the rest of the network increase, and that their influence reaches a maximum in a complete network, where they are connected to every other agent.

In this section, we take a first pass at generalizing the results in Sections \ref{sec:alliesonly} -- \ref{sec:mixed} to partly connected networks, as foreshadowed in Section \ref{subsec:network}. We adopt Barab\'{a}si-Albert networks as a traditional test case, motivated by previous theoretical studies \cite{low_discerning_2022,low_vacillating_2022}, and the conditions in many real social settings \cite{barabasi_emergence_1999,tang_survey_2016,kumar_structure_2016,maniu_building_2011}, and defer the study of other network topologies to future work.
Exploring the behavior for all possible values of the attachment parameter $m$ is outside the scope of this paper. Instead, we focus on networks with small $m$, i.e.\ networks that are sparsely connected, to accentuate the distinction with the complete networks studied in Sections \ref{sec:alliesonly} -- \ref{sec:mixed}.
We leave the exploration of dueling partisans within Barab\'{a}si-Albert networks to a later paper.
In Section \ref{subsec:distance}, we test how the beliefs of persuadable agents in allies-only networks depend on the minimum distance to the partisans, and investigate the implications for achieving consensus. 
In Section \ref{subsec:frac_ba}, we test how $m$ affects the trend of dwell time versus partisan fraction. 
We explore briefly the behavior of mixed allegiances in Section \ref{subsec:mixed_ba}, and the optimal placement of partisans to achieve a sociopolitical goal in Section \ref{subsec:Optimizations}. The latter two topics are subtle and multifarious and will be investigated fully in future work.

\subsection{Distance to the partisans: dissolving consensus on the path to turbulent nonconvergence}
\label{subsec:distance}

In a complete, allies-only network, even one partisan is enough to prevent asymptotic learning; the global outcome is turbulent nonconvergence, as described in Section \ref{subsec:samethetap}. However, the complete connections between persuadable agents ensure, that the persuadable agents reach a consensus promptly, which is maintained even while the persuadable agents vacillate turbulently between believing in $\theta_0$ and $\theta_{\rm p}$. In a partly connected, allies-only network, it is still true that even one partisan is enough to prevent asymptotic learning; that is, the global outcome remains unchanged.
However, the persuadable agents never reach a consensus; they follow different paths while enacting turbulent nonconvergence, because some are adjacent to the partisan and others are not. We demonstrate this behavior for smaller ($n=3$) and larger ($n=100$) Barab\`{a}si-Albert networks in this section. 

Consider first an allies-only network with $n=3$, illustrated in Fig.\ \ref{fig:allies_link}.
Agent 1, shaded grey, is a partisan. 
Agents 2 and 3 are persuadable. 
Agent 2 is adjacent to the partisan. 
Agent 3 is not but it is connected indirectly to the partisan via agent 2, i.e.\ one step removed.
We run a simulation for randomized priors and coin tosses and $T = 10^4$, which is analogous to the simulation in Fig.\ \ref{fig:meanconverge} but with $n=3$ instead of $n=100$.
Fig.\ \ref{fig:allies_link_distr} shows a snapshot of the belief PDF at $t = 5 \times 10^3$.  
% Figure environment removed
We observe that the PDF is bimodal, as in Fig.\ \ref{fig:0.6truebias}, and that agent 2 is more confident in $\theta_{\rm p}$ than agent 3, because agent 2 is adjacent to the partisan, and agent 3 is not.
Further to the same point, Fig.\ \ref{fig:allies_link_partisan_belief} shows the evolution of $x_i(t, \theta = \theta_{\rm p})$ in the interval $5.0\times 10^3 \leq t \leq 5.5\times 10^3$. 
We find $ x_{2}(t, \theta = \theta_{\rm p}) > x_{3}(t, \theta = \theta_{\rm p})$ throughout the interval, i.e.\ persuadable agents who are closer to the partisan are more confident in $\theta_{\rm p}$ and less confident in $\theta_0$.
In a complete network, e.g.\ in Section \ref{subsec:samethetap}, we find $x_2(t,\theta_{\rm p}) = x_3(t,\theta_{\rm p})$ throughout the interval instead, i.e.\ consensus.

% Figure environment removed

Let us now repeat the test in the previous paragraph for a larger Barab\`{a}si-Albert network with $n=100$ and $m=3$, as depicted in Fig.\ \ref{fig:BA_allies_m3}, with agent 1 being the partisan. 
Fig.\ \ref{fig:nonconsensus} displays the evolution of the mean belief $\langle\theta\rangle$ of every agent in one particular simulation for $5 \times 10^3 \leq t \leq 5.1 \times 10^3$. 
Unlike in Fig.\ \ref{fig:meanconvergent}, where the mean beliefs of persuadable agents converge mutually within $t<15$ time steps, 
the mean beliefs of persuadable agents in Fig.\ \ref{fig:nonconsensus} never converge mutually. 
That is, at any instant $t$ in the plotted range, $\langle \theta \rangle$ ranges typically from 0.35 to 0.6 for the 99 persuadable agents. 
Overall, however, $\langle \theta \rangle$ for every agent follows a similar trajectory as a function of $t$, because all persuadable agents observe the same sequence of coin tosses.
\footnote{One can quantify the degree of consensus by calculating the Kullback-Leibler divergence between agents, a topic for future work. }
This behavior differs from allies-only complete networks with partisans, discussed in Section \ref{subsec:samethetap}, and allies-only Barab\'{a}si-Albert networks without partisans, discussed in Appendix B in Ref.\ \cite{low_discerning_2022}.
Persuadable agents disagree for two reasons: 
\begin{inparaenum}[(i)]
\item they feel the influence of the partisans differently, because some are adjacent to partisans and others are not; and
\item they feel the influence of other persuadable agents differently because they are connected to different numbers of persuadable agents.
\end{inparaenum}

In order to quantify further the cause of the breakdown in consensus, we define $d_{\rm p}$ to be the length of the shortest path from a selected persuadable agent to any partisan.
We use breadth first search (BFS) \cite{moore_shortest_1959} to find the shortest path, noting that BFS only works in graphs with positive edge weights (here, allies-only networks).
Let us also define $t_{\theta_0, i}$ to be the number of total (and not necessarily consecutive) timesteps the $i$-th agent satisfies $x_i(t, \theta = \theta_0) \geq 0.9$, i.e.\ the number of timesteps when the agent is very confident in $\theta_0$,  with the threshold 0.9 having been chosen arbitrarily.  
We run an ensemble of 100 simulations with randomized priors and coin tosses on a Barab\'{a}si-Albert allies-only network with $n =100$, $m =3 $ with one partisan for $T = 10^4$.

Fig.\ \ref{fig:allies_violin_samegraph} shows violin plots of $t_{\theta_0, i}$ for each agent, as a function of $d_{\rm p}$. 
The horizontal width is a smoothed version of the histogram that shows the number of persuadable agents with a certain $t_{\theta_0, i}$, which is reflected around the vertical axis to create the shape of the violin.
The white dot indicates the mean and the thick, vertical, black bar represents the interquartile range. 
The values of $d_{\rm p}$ in this particular network are 0, 1, 2, 3, where $d_{\rm p} = 0$ refers to the single partisan; i.e.\ there are no persuadable agents with $d_{\rm p} > 3$ in this network with $n=100$.
We find that $\max(d_{\rm p})$ is related to the choice of $m$; $\max(d_{\rm p})$ is higher in more sparsely connected networks, i.e.\ smaller $m$. 
Persuadable agents with larger $d_{\rm p}$ are more confident in $\theta_0$, as shown by the fact that the white dots and thick black bars in Fig.\ \ref{fig:allies_violin_samegraph} trend higher, as $d_{\rm p}$ increases.
We also find $\max(t_{\theta_0,i}) - \min(t_{\theta_0,i}) = 5890, 5766, 3475$ for $d_{\rm p} = 1,2,3$ respectively. 
Agents with the same $d_{\rm p}$ can still have different beliefs due to their different connectivity to other persuadable agents.
In summary, the dissolving consensus among persuadable agents in Barab\'{a}si-Albert networks is attributed to the difference in connectivity and $d_{\rm p}$ values, where $t_{\theta_0, i}$ is related to $d_{\rm p}$.

\subsection{Dwell time versus $f$ and $d_{\rm p}$}
\label{subsec:frac_ba}

In the context of turbulent nonconvergence, the persistence of the beliefs of a persuadable agent is captured via the dwell time $t_{\rm d}$ defined by Eq.\ \ref{eq:dwelltime} rather than the asymptotic learning time defined by Eq.\ \ref{eq:asymlearncondition}. In Fig.\ \ref{fig:dwell_vs_frac_partisan}, we find that $\langle t_{\rm d} \rangle$ increases with the partisan fraction $f$ in a complete network. Here, we check how the trend in  Fig.\ \ref{fig:dwell_vs_frac_partisan} changes as a function of the attachment parameter, when the network is partly connected.

We consider two Barab\`{a}si-Albert networks with $n=100$, a sparse one with $m = 3$, and one of medium density with $m=20$.  
The networks with $m=3$ and $m=20$ complement the complete networks with $m = n-1 = 99$ studied in Section \ref{subsec:dwelltime_and_frac}. 
% Why? m gives the degree of each newly added vertex to the network.  Graph-tool will do d = min(n, m) when choosing the degree for the new
% vertex, where n is the current graph size
Fig.\ \ref{fig:frac_p_m3} and Fig.\ \ref{fig:frac_p_m20} display $\langle t_{\rm d} \rangle$ versus $f$ for the $m=3$ and $m=20$ networks respectively. The curves are color-coded according to $d_{\rm p}$, e.g.\ the blue curve corresponds to evaluating $t_{\rm d}$ for the subpopulation of persuadable agents adjacent to a partisan ($d_{\rm p}=1$). The aim is to test how $t_{\rm d}$ depends on the distance to the nearest partisan, and what trade-off exists between $t_{\rm d}$ and $f$.
The cut-offs for $d_{\rm p} = 2,3,4$ occur, because the number of persuadable agents with $d_{\rm p} > 1$ drops to zero, when $f$ exceeds some threshold $f_{\rm max}(d_{\rm p})$; for example, we find $f_{\rm max}(d_{\rm p} = 3) = 0.46$ in Fig.\ \ref{fig:frac_p_m3}.

% Figure environment removed


Fig.\ \ref{fig:frac_BA} leads to four main conclusions. 
First, $\langle t_{\rm d} \rangle$ increases monotonically with $f$ for $d_{\rm p}=1$, with $d\langle t_{\rm d} \rangle / df$ increasing sharply for $f\gtrsim 0.6$, just like in Fig.\ \ref{fig:dwell_vs_frac_partisan}. The trend depends weakly on $m$ (compare the blue curves on the same axes in Fig.\ \ref{fig:frac_p_m3} and Fig.\ \ref{fig:frac_p_m20}). 
Second, $\langle t_{\rm d} \rangle$ decreases with $f$ at low $f \lesssim 0.2$ for $d_{\rm p} > 1$.
This is because the ``pull'' from partisans on persuadable agents with $d_{\rm p} >1 $ is weaker than the ``pull'' from the coin for low $f$, so agents dwell with belief close to $\theta_0$. 
As $f$ increases, while $d_{\rm p}$ is held fixed, the ``pull'' from the partisan increases, causing $t_{\rm d}$ at $\theta_0$ and hence $\langle t_{\rm d} \rangle$ to decrease.
Third, $\langle t_{\rm d} \rangle$ increases with $d_{\rm p}$ for $f \lesssim 0.2$ but decreases with $d_{\rm p}$ for $f \gtrsim 0.2$, as seen in Fig.\ \ref{fig:frac_p_m3}. 
This occurs because at low $f$, the ``pull'' from the partisans is weaker for agents with higher $d_{\rm p}$.
Their beliefs are dominated by the coin tosses, leading to longer $\langle t_{\rm d} \rangle$ at low $f$. 
For $f \gtrsim 0.2$, an increase in partisan population means that the ``pull'' from the partisans outweighs the ``pull'' from the coin tosses, leading to all agents dwelling at $\theta_{\rm p}$ longer.  However, persuadable agents closer to the partisans (lower $d_{\rm p}$) are more strongly influenced by the ``pull'' from the partisans, resulting in $\langle t_{\rm d} \rangle$ decreasing with $d_{\rm p}$. 
Fourth, the trends of $\langle t_{\rm d} \rangle$ versus $f$ for the $d_{\rm p}=1$ and $d_{\rm p} =2$ subpopulations resemble each other more closely for $m=20$ than for $m=3$ (compare Fig.\ \ref{fig:frac_p_m3} and Fig.\ \ref{fig:frac_p_m20}). 
This occurs because the contribution of each individual agent is weaker in a densely connected network, as the interaction between agents is averaged over all neighbors (through Eq.\ \ref{eq:xiprimed}). 
Hence the effect of $d_{\rm p}$ on $\langle t_{\rm d} \rangle$ is weaker on networks with greater $m$.  
One main difference between Fig.\ \ref{fig:dwell_vs_frac_partisan} and Fig.\ \ref{fig:frac_BA} is the maximum dwell time.
In complete networks, one obtains $t_{\rm d} \simeq 10^4$ for $f \gtrsim 0.7$, whereas in Barab\'{a}si-Albert networks with $m=3$ one obtains $t_{\rm d} \simeq 10^4$ for $f \geq 0.06$ and $d_{\rm p} = 1$. 
Agents with $d_{\rm p} = 1$ and $t_{\rm d} \simeq 10^4$ are adjacent to more than one partisan, and non-adjacent partisans also  influence their beliefs.

\subsection{Mixed allegiances}
\label{subsec:mixed_ba}

The question of mixed allegiances is subtle and multi-faceted even in complete networks, as demonstrated in Section \ref{sec:mixed}. 
A full investigation of mixed allegiances in partly connected networks, a more challenging problem, lies well outside the scope of this paper. 
Instead, as a foretaste of what can be done, we present here one representative example: a Barab\'{a}si-Albert network with $n=100$, $m=3$, a single partisan, and $A_{ij} =\pm 1$ with equal probability. The specific realization of this network studied here features 318 edges connecting allies and 270 edges connecting opponents.

When we simulate the above network for $T =10^4$, we observe that $\langle \theta \rangle$ evolves just like in a complete network, in a manner that resembles Fig.\ \ref{fig:big_mixed_mean} (the graph is omitted to avoid repetition). 
Specifically, 13 out of 99 persuadable agents reach asymptotic learning early in the simulation ($t < 10^3$), with asymptotic mean belief in the range of $0.5 \leq \langle\theta\rangle \leq 0.8 $. One persuadable agent correctly and stably infers $\theta_0$. 
The other 86 persuadable agents exhibit turbulent nonconvergence or intermittent behavior. 
In comparison, in the complete network investigated in Section \ref{subsec:larger_network}, 11 out of 99 persuadable agents reach asymptotic learning, and three persuadable agents correctly and stably infer $\theta_0$. The other 88 persuadable agents exhibit turbulent nonconvergence or intermittent behavior. 
In complete networks, we observe that only persuadable agents who are adjacent and allied to one or more partisans (i.e. $A_{i \rm{p}} = 1$) develop a peak at $\theta_{\rm p}$ in their belief PDF (see Section \ref{subsec:larger_network}). 
In contrast, in a Barab\'{a}si-Albert network, some persuadable agents who are connected but not adjacent to a partisan (e.g.\ $A_{ij} \neq 0$, $A_{j{\rm p}} \neq 0$, and $A_{i{\rm p}}=0$ for some $j \neq i$, ${\rm p}$) also develop a peak at $\theta_{\rm p}$.
Specifically, 29 agents who are connected but not adjacent to the partisan have $x(t = T, \theta = \theta_{\rm p}) \geq 0.001$. 
The 29 agents have $2 \leq d_{\rm p} \leq 4$, where $d_{\rm p}$ is defined as the shortest distance irrespective of the sign of the edges, cf. BFS in Section \ref{subsec:distance}. 
In general, the notion of distance depends on the sign of the edges; for example, it may be argued that two agents separated by three positive edges (i.e.\ three alliances) are ``closer'' than two agents separated by three negative edges (i.e.\ three adversarial links). A systematic study of this issue, with a generalized definition of $d_{\rm p}$, is deferred to future work.



\subsection{Manipulating collective opinion: optimizing the placement of partisans}
\label{subsec:Optimizations}

A challenging question with important social implications is how to optimize the placement of obdurate partisans to manipulate the beliefs of persuadable agents in the service of some social or political goal. 
Within the specific context of media bias, for example, one goal could be to camouflage the political bias of a media outlet, by convincing as many consumers as possible that the outlet is neutral ($\langle \theta \rangle \approx 0.5$, say) when in reality it is biased strongly ($\theta_0 \approx 0$ or $1$, say). 
The latter example involves persuading agents to believe something false. In other applications, both within and beyond the specific context of media bias, the goal may be to persuade agents to believe something true, e.g.\ the danger of drink driving, or the efficacy of a medical treatment. 
The obdurate partisans may be real humans or automated systems such as social network ``bots''. 
They may conduct their operations by espousing beliefs that agree with the target belief or, interestingly and counterintuitively, by espousing beliefs that disagree with the target belief, leveraging oppositional relationships in the network to shepherd persuadable agents towards the target belief. 
The long-term impact of partisanship and the optimal placement of obdurate partisans has been modeled previously in a deterministic framework (without belief PDFs) to investigate how to maximize political polarization \cite{yildiz_binary_2013,arendt_opinions_2015} or conformity with the partisans' belief \cite{abrahamsson_opinion_2019,klamser_zealotry_2017,masuda_evolution_2012}. 

In this section, we analyze briefly one specific, representative example of the above problem, as a foretaste of what can be investigated more broadly. The example involves a Barab\'{a}si-Albert network with $n=10$, $m=3$, and one partisan, depicted in Fig.\ \ref{fig:ba_10}. 
The degrees of the 10 vertices range from three to eight. The specific question asked is: where should the partisan be placed, to drive the beliefs of persuadable agents as far from $\theta_0$ as possible, i.e.\ to maximize $| \langle \theta \rangle - \theta_0 |$? 
To answer the question, we run 10 simulations with the same sequence of coin tosses for $T=10^4$. The simulations differ in what vertex the partisan occupies.
Let us define 
\begin{equation}\label{eq:bartheta0}
    \overline{t_{\theta_0}} = \frac{1}{(1-f) n} \sum_{i \neq {\rm p}} t_{\theta_0, i}, 
\end{equation}
where $f$ is the fraction of partisans in a network of size $n$. 
In Fig.\ \ref{fig:opti_scatter}, we observe that $\overline{t_{\theta_0}}$ is inversely related to the degree of the partisan. 
Furthermore, the number of persuadable agents with $t_{\theta_0}  = 0$ increases as the degree of the partisan increases, also as observed in Fig.\ \ref{fig:opti_scatter}. 
% DONE: which has shorter... - unclear what we are talking about here, what does "which refer to"? as written it refers to the degree of the partisan, but a degree can't have a shorter t_theta_0 - t_theta_0 is a property of agents, not degrees of vertices - please fix [sorry I can't suggest wording, I don't know what we mean]
The partisan's degree is equal to the number of persuadable agents with $d_{\rm p} = 1$ (adjacent to the partisan). Persuadable agents with $d_{\rm p} = 1$ have shorter $t_{\theta_0, i}$, shown in Fig. \ref{fig:allies_violin_samegraph}, resulting in a shorter $\overline{t_{\theta_0}}$.
Therefore, if we wish to mislead the persuadable agents, we should place the partisan at the vertex with maximum degree, i.e.\ the most connected vertex. 
This result makes sense intuitively. However, we caution that it is not expected to apply always in networks with mixed allegiances, which contain subnetworks with internal tensions and trade-offs, such as the unbalanced triad $G_2$ in Fig.\ \ref{fig:triad}. 
An interesting avenue for future work is to optimize both the placement and beliefs of (perhaps dueling) partisans to manipulate a numerical majority of agents to hold a target belief $\theta_{\rm t}$. 
This optimization task is related to but different from minimizing the average displacement $| \langle \theta \rangle - \theta_{\rm t} |$, and may be more relevant in electoral applications \cite{bravo-marquez_opinion_2012,druckman_impact_2005}.
% DONE: do we know a good, well-cited reference to opinion dynamics in electoral applications? no worries if not


% Figure environment removed

