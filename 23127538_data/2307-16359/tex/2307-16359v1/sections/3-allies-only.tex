
We prepare to study mixed networks containing allies and opponents by investigating first the impact of partisans in simpler networks containing allies only. A representative example of the evolution of an allies-only network is presented in Section \ref{subsec:samethetap}. It exemplifies the general result, that even a single partisan prevents the system from achieving equilibrium (asymptotic learning), because the coin and the partisan ``pull'' the persuadable agents in opposite directions. 
% DONE: add sentences introducing sec 3.3 and 3.4 too please
The distribution of dwell times in specific states, and the role played by the fraction of partisans in the network, are explored in Sections \ref{subsec:dwelltime_and_frac} and \ref{subsec:switching_between_belief} respectively. 
Section \ref{subsec:partisandifferentopinion} explores the opinion dynamics when there exist partisans with different beliefs.  The special cases $\theta_0=0$ and $\theta_0 = 1$ are discussed in \ref{sec:twoallies}. 

For the rest of the paper, except where stated otherwise, we take $\theta_0=0.6$, $\theta_{\rm p} = 0.3$, and $\mu = 0.25$. These values are arbitrary but representative.

\subsection{Disruption by a single partisan}
\label{subsec:samethetap}

Consider an allies-only network with $n=100$, containing 99 persuadable agents, and one partisan labelled `agent p'. We run a number of simulations with randomized priors and coin toss sequences. In this section, we focus on a representative simulation, whose output is plotted in Fig.\ \ref{fig:meanconverge}.
We find that even a single partisan disrupts the perceptions of the persuadable agents, by dragging them away from believing in the true coin bias. 
\ref{sec:coinvsprior} discusses how the priors and coin toss sequences influence the long-term behavior of the system. 
% DONE: check
In short, the priors are largely inconsequential, but a specific subsequence of coin tosses (say, five heads in a row) can have an appreciable impact on persuadable agents. 


% DONE: move this later to when we first talk about convergence - we need to set the stage for the reader first about what's happening qualitatively, before we quantify 

In a network containing only allies, persuadable agents' beliefs converge on a relatively short timescale of order 10 time steps, before vacillating between $\theta_0$ and $\theta_{\rm p}$ over the longer term.  Fig.\  \ref{fig:meanconvergent} and Fig.\ \ref{fig:meannonpartisan} display the evolution of the mean belief $\langle\theta\rangle$ in one particular simulation.  
The bold black dashed line represents the partisan, whose mean belief is constant.  
We run an ensemble of simulations with randomized prior and coin toss sequences, choosing different values for $\theta_0$ and $\theta_{\rm p}$ but excluding the special case of $\theta_0 = 0$ or 1 (i.e.\ the coin returns heads or tails only).
% DONE: I don't understand this last sentence - the previous sentence is all about an ensemble of simulations, now we're saying one simulation - very confusing
For brevity, we present the simulation described in Fig.\ \ref{fig:meanconverge} to demonstrate how a single partisan disrupts the belief of persuadable agents.
Similar behavior is observed throughout this ensemble for all different $\theta_0 \neq 0, 1, \theta_{\rm p}$.  

% Figure environment removed


Over the short term, the persuadable agents reach a consensus, in the sense that they agree among themselves, viz.\
$\text{max}_{\theta} |x_i(t,\theta) -x_j(t,\theta) | < \epsilon \text{ max}_{\theta} \left[ x_i(t, \theta), x_j(t, \theta) \right]$, 
for any pair of persuadable agents $i$ and $j$, where $\epsilon = 0.01$ is a user-selected tolerance.  
This is observed in Fig.\ \ref{fig:meanconvergent}, where $\langle \theta \rangle$ converges rapidly to approximately the same value for all 99 persuadable agents over $\sim 10$ time steps. 
Visually this is observed as a reduction in the spread of the 99 colored, zig-zag curves, as time passes. 
The zig-zag dynamics are a typical response to the coin: if it comes up heads, $\langle\theta\rangle$ increases, otherwise $\langle\theta\rangle$ decreases. For example, $\langle\theta\rangle$ increases monotonically for $4 \leq t \leq 8$, because the coin returns four heads in a row. 
At $t=14$, one has $\langle \theta \rangle \approx 0.58 \neq \theta_{\rm p}$ for all 99 persuadable agents; that is, the partisan is yet to exert their influence fully. 


Consensus is not the same as asymptotic learning. Over the long term, the persuadable agents vacillate as a group between believing in $\theta=\theta_0$ and $\theta=\theta_{\rm p}$, maintaining consensus among themselves. This is observed in Fig.\ \ref{fig:meannonpartisan} where $\langle \theta \rangle$ is plotted versus time for an arbitrary, representative, persuadable agent.  
Without any partisans, namely the situation shown by the blue curve in Fig.\ \ref{fig:meannonpartisan}, one finds $\theta \rightarrow \theta_0$ monotonically for large $t$, irrespective of the coin toss sequence, as in Fig.\ 2 in Ref.\ \cite{low_discerning_2022}. 
In contrast, the orange curve in Fig.\ 2b fluctuates in the range $0.32 \leq \langle\theta\rangle \leq 0.60$. The partisan disrupts the beliefs of the persuadable agents, causing them to vacillate indefinitely, even when there is only one partisan among 99 persuadable agents. 
The mean belief $\langle\theta\rangle$ spends $\approx 86 \%$ and $0.04 \%$ of the $8\times 10^3$ timesteps (after the initial transient $0\leq t \leq 2\times 10^3$) in Fig.\ \ref{fig:meanhist} near $\theta_0$ and $\theta_{\rm p}$ respectively, with $ | \langle\theta\rangle - \theta_0 | \leq 0.025 \theta_0$ and $ | \langle\theta\rangle - \theta_{\rm p} | \leq 0.025 \theta_{\rm p}$, and spends $\approx 14\%$ of the time near neither value. The dwell time in each state is quantified further in Section \ref{subsec:switching_between_belief}.

Once the persuadable agents gather enough information to move beyond their initial priors, at $t \gtrsim 2\times 10^3$, their belief PDF vanishes uniformly except at $\theta_0$ and $\theta_{\rm p}$, as shown in  Fig.\ \ref{fig:0.6truebias}.  
% The probability of $\theta_0$ and $\theta_{\rm p}$ (height of the two peaks) changes as indicated in Fig.\ \ref{fig:meannonpartisan} as mean for the distribution can be simplified as $\langle \theta \rangle =  {\rm Pr}(\theta_0) \cdot \theta_0 +  {\rm Pr}(\theta_{\rm p}) \cdot \theta_{\rm p}$. 
The heights of the peaks at $\theta_0$ and $\theta_{\rm p}$ fluctuate continuously. Fig.\ \ref{fig:meanhist} shows a histogram of $\langle \theta \rangle$ for $2 \times 10^3\leq t \leq 1\times 10^4$, with $\langle\theta\rangle$ sampled at each of the $8\times 10^3$ intervening time steps.
With only one partisan, we find $\langle\theta\rangle \geq 0.56$ during 90\% of the time. That is, the persuadable agents prefer $\theta_0$ without rejecting $\theta_{\rm p}$ completely. These preferences reverse, as the fraction of partisans increases, as discussed in Section \ref{subsec:switching_between_belief}. 
% As the internal signal is proportional to the difference in belief as shown in Eq.\ \eqref{eq:undatesecondhalf}, the internal interaction between persuadable allies is negligible after around 10 time steps.  Additionally, the persuadable agents observe the coin tosses equally, and they are each allied with the partisan, therefore their beliefs change simultaneously after they have converged.   
When the persuadable agents form a consensus around the true coin bias, with $x_j(t,\theta) \approx \delta(\theta - \theta_0)$ for most $j$, the displacement $\Delta x_i'$ in Eq.\ \eqref{eq:xiprimed} is dominated by the ``pull'' of the disagreeing partisan, who has $x_{\rm p}(t,\theta) = \delta(\theta - \theta_{\rm p})$ with $\theta_{\rm p} \neq \theta_0$. All the persuadable agents feel the partisan's ``pull'' simultaneously, as well as the influence from the coin toss described by Eqs.\ \eqref{eq:updatefirsthalf} and \eqref{eq:likelihood}. 
Hence the characteristic timescale to move away from $\langle \theta \rangle \approx \theta_0$ depends on $\mu^{-1}$ and the coin toss sequence. 
On the other hand, when the persuadable agents form a consensus around the partisan's belief, with $x_j(t,\theta)=\delta(\theta-\theta_{\rm p})$ for most $j$, the internal signals and hence the displacement $\Delta x_i'$ are small, and the coin toss dominates the update rule via Eq.\ \eqref{eq:updatefirsthalf} and \eqref{eq:likelihood}. The characteristic timescale to move away from $\langle\theta\rangle \approx \theta_{\rm p}$ depends only on the coin toss sequence.  
Hence, on balance, the persuadable agents feel a more durable ``pull'' when the consensus is formed around $\theta_0$ instead of $\theta_{\rm p}$. 
% DONE: [comment here on which happens faster, the partisan pull or coin toss sequence, if possible - numbers would help]

% DONE: Moblia para here and talk about why is it different 
The behavior of a network containing one partisan differs from Fig.\ \ref{fig:init_belief}, if the agents' beliefs are deterministic (single value per agent), or the update rule does not involve an external signal. 
For example, in the voter model analyzed by Mobilia et al.\ \cite{abrahamsson_opinion_2019,mobilia_does_2003,mobilia_voting_2005}, every agent's opinion converges to the partisan and subsequently remains unchanged. Once consensus is reached, the system enters a steady state, because the interaction between agents involves randomly adopting the belief of a neighbor, and all neighbors hold the same belief.
In Ref.\ \cite{abrahamsson_opinion_2019,mobilia_does_2003,mobilia_voting_2005}, the persuadable agents only change their belief in response to their neighbors, whereas the update rule in Section \ref{subsec:modelintro} responds to both the coin toss and the neighbors.  
When persuadable agents are exposed to two or more sources of contradictory information about $\theta$ (e.g.\ the coin tosses and the partisan), their beliefs do not settle to a steady state, as Fig.\ \ref{fig:meannonpartisan} illustrates. The disruption occurs even if there is only one partisan in a large network with $n \gg 1$ agents.

\subsection{Dwell times and the fraction of partisans in a network}
\label{subsec:dwelltime_and_frac}

In the presence of a partisan, persuadable agents never settle in their beliefs, and the asymptotic learning time defined by Eq.\ \eqref{eq:asymlearncondition} is unsuitable to quantify the long-term behavior of the system.  
Eq.\ \eqref{eq:sysasymlearncondition} holds temporarily, but in the long run we observe turbulent nonconvergence, as observed in Section 4.3 of Ref.\ \cite{low_discerning_2022} for different reasons. 
% Figure \ref{fig:asym_Vs_dwell} illustrate how the system can satisfy the condition for asymptotic learning at $t = 5346$ and $t = 6779$, but the belief of still changed as can be indicated as the number of consecutive step become 0 shortly after. 
To quantify the behavior of the persuadable agents, we define the dwell time $t_{\rm d}$ such that 
\begin{equation} \label{eq:dwelltime}
    \max_{\theta} |x_i(t', \theta) - x_i(t, \theta)| < \delta \max_{\theta} |x_i(t, \theta)|
\end{equation}
is true for $t < t' \leq t + t_{\rm d}$ and $\delta = 0.01$ is a user-selected tolerance.
Eq.\ \eqref{eq:dwelltime} defines blocks of time, where the system dwells in the same state to a good approximation, e.g.\ the system may spend $t_1 \leq t' \leq t_1+t_{\rm d1}$ with $\langle\theta\rangle \approx \theta_0$, then $t_2 \leq t' \leq t_2+t_{\rm d2}$ with $\langle\theta\rangle \approx \theta_{\rm p}$, and so on. The blocks of time do not overlap, nor are they necessarily contiguous; in the above examples, one may have $t_2 > t_1 + t_{\rm d1}$, if the two blocks are separated by an interval of turbulent nonconvergence.
Fig.\ \ref{fig:dwell_log_fit} shows a histogram of the 80,698 dwell times observed in an allies-only network with 99 persuadable agents, one partisan, and $T = 10^6$. For this network, dwelling rarely lasts for more than 50 time steps. 
The dwell time PDF $p(t_{\rm d})$ is exponential to a good approximation, with $p(t_{\rm d}) \propto \exp(-0.16 t_{\rm d})$ and hence $\langle t_{\rm d} \rangle \approx 5.4$\footnote{Note that $ \langle t_{\rm d} \rangle ^{-1}= (5.4)^{-1}$ does not equal 0.16 exactly, as expected for an exponential, because we cannot fit reliably to long dwell intervals ($t_{\rm d} >55$) due to small number statistics per bin.}.
We also find that 90\% of the dwell times satisfy $t_{\rm d} \leq 13$.
% We observe the exponential relation in the dwell time distribution with a coefficient of determination statistic \cite{javed_probability_nodate} $R^2 > 0.9$.  
Similar behavior is observed in networks with no partisans.
% DONE: [say exactly what we see there, what the experiment was, the fact that it's BA network, and also quote the best fit exponential in units - this will take 3-4 sentences - very important comparison]
For example, Fig.\ 5 in Ref.\ \cite{low_vacillating_2022} shows a simulation with $T = 10^6$ coin tosses, with the network generated using the Barab\'{a}si-Albert model with $n = 100$ and attachment parameter $m = 3$ \cite{hagberg_exploring_2008}. 
An exponential trend is also observed with best fit $p(t_{\rm d}) \propto \exp(-0.0069 t_d)$\footnote{
    Strictly speaking, Ref.\ \cite{low_vacillating_2022} distinguishes between two versions of the dwell time, termed turbulent ($t_{\rm t}$) and stable ($t_{\rm s}$), which are defined by Eq.\ \eqref{eq:dwelltime} with the conditions $t_{\rm d} < 100$ and $t_{\rm d} \geq 100$ respectively. 
    We cannot draw this distinction usefully in this paper, as we find ${\rm max}(t_{\rm d})< 100$ for most realizations of a complete network. 
    Nevertheless, it is interesting that all approaches lead to exponential PDFs for $p(t_{\rm d})$ (this paper), $p(t_{\rm t})$ (Ref.\ \cite{low_vacillating_2022}), and usually $p(t_{\rm s})$ (Ref.\ \cite{low_vacillating_2022}).
}.

% Figure environment removed

Fig.\ \ref{fig:dwell_vs_frac_partisan} shows how the dwell time depends on the fraction of partisans, $f$, with $0.01 \leq f \leq 0.99$ in steps of $0.01$.
When $f$ is low, the mean dwell time $\langle t_{\rm d} \rangle$ is short, and the persuadable agents frequently change their beliefs.  
As $f$ increases, $\langle t_{\rm d} \rangle$ and $\max(t_{\rm d})$ increase. 
More partisans produce a stronger internal signal between the persuadable agents and partisans, strengthening the ``pull'' towards $\theta_{\rm p}$ and away from $\theta_0$. 
For $f \gtrsim 0.6$, $\langle t_{\rm d} \rangle$ and $\max(t_{\rm d})$ rise sharply (note the logarithmic scale), while $\min(t_{\rm d}) = 1$ remains unchanged. 
Short dwell times occur early in the simulations, when the persuadable agents start to gather information about the coin sequence and partisans and are still influenced strongly by their initial priors. 
Long dwell times occur later in the simulations; for example, one obtains $\max(t_{\rm d}) \approx T = 10^4$, when the persuadable agents maintain their beliefs until the end of the simulation, once the initial transient dies out.
The flatter trend in $\langle t_{\rm d} \rangle$ for $f \gtrsim 0.8$ is caused by the ceiling $t_{\rm d} \leq T$.


The histogram in Fig.\ \ref{fig:dwell_log_fit} contains 80,698 dwell times covering a total of 435,080 out of $T=10^6$ time steps. Hence, for $f=10^{-2}$, the system spends 43.5\% of its time dwelling near $\theta_0$ or $\theta_{\rm p}$ and the rest of its time fluctuating in a state of turbulent nonconvergence similar to the one identified in Ref.\ \cite{low_vacillating_2022} (in the latter paper, the turbulence is caused by different dynamics unrelated to the presence of partisans). 
The fraction of time spent in turbulent nonconvergence depends on $f$. 
For example, at $f=0.5$, one finds 208 dwell time intervals covering a total of 2997 out of $T=10^4$ time steps, and the system spends 70\% of its time in a state of turbulent nonconvergence, whereas at $f=0.9$, one finds typically four dwell time intervals covering a total of 9952 out of $T=10^4$ time steps, and the system spends 0.4\% of its time in a state of turbulent nonconvergence. 
%  DONE: [finish by adding a sentence explaining these results physically if possible]
When $f$ is high, the ``pull'' from $fn$ partisans ${\rm p}_1,\dots,{\rm p}_{fn}$ on an arbitrary persuadable agent $i$ scales as $A_{i{\rm p}_1}+\dots+A_{i{\rm p}_{fn}}$ and is relatively large, which prevents the persuadable agents from changing their mind. 
Hence, we only observe turbulent nonconvergence early in the simulation, before persuadable agents dwell for a long time at $\theta_{\rm p}$. 

\subsection{Switching between beliefs}
\label{subsec:switching_between_belief}
% The difference belief for dwell time
What does $x_i(t,\theta)$ look like, when an agent dwells near some belief, and how does switching between beliefs correlate with $t_{\rm d}$ statistically? 
The picture is complicated, because $x_i(t,\theta)$ may peak at $\theta_0$, $\theta_{\rm p}$, or both $\theta_0$ and $\theta_{\rm p}$ at the same time, as shown in Fig.\ \ref{fig:0.6truebias}. Theoretically, to categorize the belief at each time step in a dwell interval, we should define an orthogonal set of belief templates and match $x_i(t,\theta)$ against all of them. 
However, this approach is challenging numerically, when $x_i(t,\theta)$ is continuously valued, and the number of belief templates is large. 
Instead, we assume that each dwell interval is characterized approximately by an average belief, defined by $\langle\theta\rangle$ at the final time step in the interval. We can then compare the belief and dwell statistics efficiently.

% we define the dwell time at the true bias $t_0$ when $\langle \theta \rangle \simeq \theta_0$, and dwell time at the partisan bias $t_{\rm p}$ when $\langle \theta \rangle \simeq \theta_{\rm p}$. 
% For any dwell time other than $t_0$ and $t_{\rm_p}$ the persuadable agents holds dueling belief of $\theta_0$ and $\theta_{\rm p}$,we denote this dwell time as $t_{\rm other}$.

% Figure environment removed


Fig.\ \ref{fig:dt_dti} displays histograms counting the number of dwell time steps (blue bars) and dwell intervals (orange bars) as functions of the mean belief $\langle\theta\rangle$ during the time steps and intervals respectively. The aim is to understand what beliefs are enduring or transitory at low, moderate, and high $f$. 
At low $f=0.1$ (Fig.\ \ref{fig:dt_dti_0.1}), the persuadable agents spend $\sim 10$ times longer at $\theta_0$ than at $\theta_{\rm p}$ (blue bars) but switch between $\theta_0$ and $\theta_{\rm p}$ frequently, as shown by the short dwell time $\langle t_{\rm d} \rangle = 2.65$ in Fig.\ \ref{fig:dwell_vs_frac_partisan}. 
The agents also dwell sometimes at $\langle\theta\rangle \neq \theta_0, \theta_{\rm p}$ but only rarely ($0.2\%$ of the time) and predominantly early in the simulation. 
At moderate $f=0.6$ (Fig.\ \ref{fig:dt_dti_0.6}), near the sharp low-to-high-$f$ transition observed in Fig.\ \ref{fig:dwell_vs_frac_partisan}, the persuadable agents dwell at $\langle\theta\rangle \approx 0.50$ and $0.55$, corresponding to the belief PDF peaking simultaneously at both $\theta_0$ and $\theta_{\rm p}$ with $x_i(t,\theta_{\rm p}) \gtrsim x_i(t,\theta_0)$ i.e.\ the persuadable agents are more confident in $\theta_{\rm p}$ while not completely discounting $\theta_0$. 
As $f$ increases, the persuadable agents grow their confidence in $\theta_{\rm p}$. 
At $f = 0.6$, 97\% of the dwell intervals have $x_i(t,\theta_{\rm p}) = 1$, while only 4\% have this property for $f = 0.1$. 
At high $f=0.9$ (Fig.\ \ref{fig:dt_dti_0.9}), agents never dwell at $\theta_0$ despite observing the coin continuously.  
The internal interaction from the partisans produces $x_i(t,\theta) = \delta(\theta-\theta_{\rm p})$ for all persuadable agents, i.e.\ the belief is zeroed out everywhere except at $\theta_{\rm p}$.  
This happens because Bayes's rule in Eqs.\ \eqref{eq:updatefirsthalf} and \eqref{eq:likelihood} is multiplicative.
Once the $i$-th persuadable agent achieves $x_i(t,\theta_0)=0$ at some $t$, they stop feeling the  ``pull'' from the coin at every subsequent $t' > t$, resulting in $x_i(t',\theta_0) = 0$ for all $t' > t$ and hence a long  dwell time $t_{\rm d} \lesssim T = 10^4$ at $\theta_{\rm p}$.  
As partisans become the supermajority ($f \gtrsim 0.8$) in the network, the persuadable agents are misled by the partisans and reject the true coin bias $\theta_0$ in favor of $\theta_{\rm p}$.


\subsection{Dueling partisans}
\label{subsec:partisandifferentopinion}


Now suppose that the network contains two groups of partisans ``p1'' and ``p2'' who disagree, with $\theta_{\rm p1}=0.3$, $\theta_{\rm p2}=0.9$, and $\theta_0 = 0.6$ (illustrative values chosen arbitrarily). We say that the partisans are ``dueling'' to convey that they disagree and compete in their influence, even though they may disagree accidentally or cordially rather than deliberately or aggressively.
The results resemble those described in Section \ref{subsec:samethetap}, except that the belief PDF becomes trimodal, with $x_i(t, \theta) \neq 0$ at $\theta = \theta_0, \theta_{\rm p1}, $ and $\theta_{\rm p2}$. 
Fig.\ \ref{fig:dueling_belief_turb} shows the time evolution of $x_i(t, \theta)$ at $\theta = \theta_0, \theta_{\rm p1}, $ and $\theta_{\rm p2} $ for the representative interval $ 4\times 10^3 \leq t \leq 4.5\times 10^3$ and the $i$-th arbitrary agent.
The red line shows $x_i(t,\theta_0)+x_i(t,\theta_{\rm p1}) + x_i(t,\theta_{\rm p2})$, which constantly equals one, indicating zero belief in $\theta \neq \theta_0, \theta_{\rm p1}, \theta_{\rm p2}$. 
The $i$-th agent favors $\theta_0$ without discounting $\theta_{\rm p1}$ and $\theta_{\rm p2}$ completely. During the $5\times 10^2$ plotted time steps, the orange curve for $x_i(t,\theta_0)$ never settles down, dropping eight times to $x_i(t,\theta_0) < 0.8$. It also drops twice to $x_i(t,\theta_0) < 0.6$, once each in favor of $\theta_{\rm p1}$ and $\theta_{\rm p2}$. 

Dueling partisans destabilize the beliefs of the persuadable agents more than a single partisan. Fig.\ \ref{fig:dueling_dt_f0.02} shows the dwell time histogram of two networks, both with $f = 0.02$,  where one network contains two partisans who agree (blue curve), and the other contains two partisans who disagree (orange curve).
The shorter times marked by the orange curve indicate that the persuadable agents change their belief more frequently in the company of dueling partisans.  
This makes sense intuitively; the persuadable agents are ``pulled'' in three directions rather than two. 
For example, when the persuadable agents dwell near $\theta_{\rm p1}$, where the internal signal between partisan p1 and the persuadable agents is negligible, the internal signal between the partisan p2 and the persuadable agents remains substantial.

What happens when uneven numbers of partisans disagree? Let $n_{\rm p1}$ and $n_{\rm p2}$ be the numbers of partisans with belief PDFs $\delta(\theta-\theta_{\rm p1})$ and $\delta(\theta-\theta_{\rm p2})$ respectively. 
We consider three situations, all with $f=0.4$ for the sake of illustration: 
(i) $n_{\rm p1}= 20 = n_{\rm p2}$, (ii) $n_{\rm p1} = 30, n_{\rm p2} = 10$, and (iii)  $n_{\rm p1} = 39, n_{\rm p2} = 1$. 
Fig.\ \ref{fig:uneven_belief} displays histograms counting the number of dwell intervals for cases (i), (ii), and (iii), presented as functions of $\langle\theta\rangle$ during the dwell interval like in Fig.\ \ref{fig:dt_dti}.
When the number of disagreeing partisans is equal, as in case (i), we observe roughly equal numbers of dwell intervals centered on $\theta_{\rm p1}$ (1774 times) and $\theta_{\rm p2}$ (1690 times) (blue bars in Fig.\ \ref{fig:uneven_belief}). 
In case (ii), we observe 2796 dwell intervals centered on $\theta_{\rm p1}$ and 1162 dwell intervals centered on $\theta_{\rm p2}$ (orange bars in Fig.\ \ref{fig:uneven_belief}), while in case (iii), we observe 2547 dwell intervals centered on $\theta_{\rm p1}$ and only 170 dwell intervals centered on $\theta_{\rm p2}$ (green bars in Fig.\ \ref{fig:uneven_belief}). 
The persuadable agents dwell more frequently at the belief held by most of the partisans. 

The number of dwell intervals at each belief is roughly proportional to the partisan population. For $ n_{\rm p1} / (n_{\rm p1}+n_{\rm p2}) = 0.75$, we find that 70\% of the dwell intervals happen at $\langle \theta \rangle \approx \theta_{\rm p1}$; for $ n_{\rm p1} / (n_{\rm p1}+n_{\rm p2}) = 0.98$, we find that 94\% of the dwell intervals happen at $\langle \theta \rangle \approx \theta_{\rm p1}$.  
Note that $\langle \theta \rangle$ does not exactly equal $\theta_{\rm p1}$ or $\theta_{\rm p2}$, because the belief PDF is bimodal, peaking at both $\theta_{\rm p1}$ and $\theta_{\rm p2}$.
The persuadable agents no longer believe in $\theta_0$ for the reason described at the end of Section \ref{subsec:switching_between_belief}. 

Fig.\ \ref{fig:uneven_intervals} shows the histograms of the number of dwell time steps as a function of the mean belief $\langle \theta \rangle$ for cases (i), (ii), and (iii), as well as a control network containing agreeing partisans for comparison.
We observe longer dwell times for $n_{\rm p1} \gg n_{\rm p2}$ (Fig.\ \ref{fig:uneven_intervals}) compared with $n_{\rm p1} = n_{\rm p2}$ or $n_{\rm p1} \gtrsim n_{\rm p2}$, due to the stronger ``pull'' from the dominant group, but $\langle t_{\rm d} \rangle$ is still shorter by a factor of $\approx 3$ than for networks in which all the partisans agree (red curve in Fig.\ \ref{fig:uneven_intervals}). 
Interestingly, even one disagreeing partisan is enough to shorten $\langle t_{\rm d} \rangle$ by a factor $\approx 3$; compare the green and red curves in Fig.\ \ref{fig:uneven_intervals}, for example. In contrast, there is not much difference between $n_{\rm p1}/n_{\rm p2} =39$ and $n_{\rm p1}/n_{\rm p2} = 1$, which yield $\langle t_{\rm d} \rangle = 3.4$ and $2.1$ respectively; see the green and blue curves in Fig.\ \ref{fig:uneven_intervals}. 

% Figure environment removed
