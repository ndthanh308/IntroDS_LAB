
In this appendix, we test how the initial priors $x_i(t=0,\theta)$ and the coin toss sequence affect the long-term evolution of $x_i(t,\theta)$ for persuadable agents.
% Same coin different prior

We start by simulating $10^4$ copies of the same network, which are identical, except that $x_i(t=0,\theta)$ for all the persuadable agents $i$ is drawn afresh in each copy from the truncated Gaussian distribution defined in Section \ref{subsec:automaton}.
All copies of the network witness the same coin toss sequence. Fig.\ \ref{fig:samecoin} plots the six independent pair-wise differences $\langle\theta\rangle_A - \langle\theta\rangle_B$ for four network copies (indexed by $A$ and $B$) and a single, arbitrary persuadable agent as functions of time.
The differences in mean belief evolve towards zero shortly after the simulation starts, with $| \langle \theta \rangle_A - \langle \theta \rangle_B | \leq 10^{-5}$ for $t \geq 5\times 10^2$.
This behavior is typical: the long-term belief PDF does not depend strongly on $x_i(t=0,\theta)$. It is also observed in networks without partisans \cite{acemoglu_opinion_2013}. 

Next we test the sensitivity to the coin toss sequence. In Fig.\ \ref{fig:sameprior}, we plot again the six pair-wise differences $\langle\theta\rangle_A - \langle\theta\rangle_B$ against time for a single, arbitrary, persuadable agent and four random, independent coin toss sequences, indexed by $A$ and $B$. 
The differences do not decay to zero, unlike in Fig.\ \ref{fig:samecoin}. 
Instead, they fluctuate steadily for all $0\leq t \leq 2\times 10^3$, with standard deviation $\approx 0.05$ throughout the interval. 
This behavior matches Section \ref{subsec:samethetap}: the partisan disrupts the system so that it never reaches equilibrium, and the coin tosses are an ongoing factor, competing with the pull of the partisan at every time step. 
% Figure environment removed

