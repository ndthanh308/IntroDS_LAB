We now turn our focus to more realistic networks containing mixed political allegiances, where some agents interact simultaneously with allies and opponents.
An exhaustive analysis of mixed networks lies outside the scope of this paper. 
A preliminary step towards this challenging problem without partisans was taken in Refs.\ \cite{low_discerning_2022} and \cite{low_vacillating_2022}, mainly (but not exclusively) in the context of Barab\'{a}si-Albert networks. 
Here we focus instead on the special (and simpler) case of complete networks to take advantage of the baseline studies in Sections \ref{sec:alliesonly} and \ref{sec:opponentonly}, trading off some richness in network topology in favor of simplifying and therefore clarifying the novel effects introduced by partisans. 
In Section \ref{subsec:unbalanced_triad}, we consider the dynamics of an unbalanced triad, which is the smallest nontrivial subunit of a mixed network and the driver of much (although not all) of the counterintuitive behavior reported in Refs.\ \cite{low_discerning_2022} and \cite{low_vacillating_2022}. 
In Section \ref{subsec:larger_network}, we consider a representative example with $n=100$, to gain a preliminary sense of how the behavior of larger mixed networks compares with allies- and opponents-only networks. 
The results in Section \ref{subsec:larger_network} point to some productive avenues for future work but are not intended to be exhaustive.




\subsection{Triads}
\label{subsec:unbalanced_triad}

Four unique triads with $n=3$ can be constructed. They are labeled $G_1, \dots, G_4$ in the top row of  Fig.\ \ref{fig:triad}.
$G_1$ and $G_4$ are allies-only and opponents-only networks, studied in Sections \ref{sec:alliesonly} and \ref{sec:opponentonly} respectively.
$G_3$ is nominally a mixed network but it behaves the same as an opponents-only network with $n=2$, when there are no partisans, as is clear visually. 
Interestingly, though, it exhibits counterintuitive behavior, when a partisan is introduced, as discussed below in this section. 
The unbalanced triad $G_2$ leads to important, counterintuitive, new behavior even without partisans, as demonstrated in Ref.\ \cite{low_discerning_2022}. 
It exhibits internal tension: agent 1 is attracted to the beliefs of its allies, agents 2 and 3, but the beliefs of agents 2 and 3 tend to diverge, because agents 2 and 3 are opponents. 
In other words, agent 1 is in the invidious position of striving to agree with two individuals, who are predisposed to disagree. In general, this leads to unsteady dynamics, such as turbulent nonconvergence \cite{low_discerning_2022}. 
In an unbalanced triad, the impact of a partisan depends sensitively on where the partisan is inserted; they may assume the role of agent 1 or agent 2 (equivalent to agent 3), as depicted in the bottom row of Fig.\ \ref{fig:triad}. 
We consider both scenarios in this section.
% We now focus on $G_2$ and $G_3$ with mixtures of allies and opponents.
% $G_2$ exhibits internal tension, as agent 1 is allied with two agents that oppose each other, while $G_3$ lacks internal tension, with allied agents 2 and 3 commonly opposing agent 1.  

% Figure environment removed

Let us begin with $G_{2{\rm p}1}$, which is the version of $G_2$ where the partisan (agent 1) is allied with persuadable opponents (Fig.\ \ref{fig:triad}, middle row, left graph). 
Fig.\ \ref{fig:G_2_prior} displays snapshots of $x_i(t=5 \times 10^3, \theta)$ for all three agents, while Fig.\ \ref{fig:G_2_mean} displays $\langle\theta\rangle$ versus time ($0\leq t \leq 1 \times 10^4$) for all three agents.
We observe turbulent nonconvergence for both persuadable agents, whereas only one agent (agent 1) experiences turbulent nonconvergence without a partisan in Ref.\ \cite{low_discerning_2022}. 
Agents 2 and 3 are both pulled by their alliances towards the partisan, while still observing the coin tosses. 
However, they strive to disagree with each other, so the peaks away from $\theta_{\rm p}$ in their bimodal belief PDFs occur at unequal values of $\theta$.
Dwelling still occurs, including for long intervals, e.g.\ $7374 \leq t \leq 9071$, which starts following a run of tails during $7369 \leq t \leq 7374$. 
Fig.\ \ref{fig:G_2} is a reminder that a hypothetical external observer should be cautious about inferring the truth of a specific belief by extrapolating from its popularity. In Fig.\ \ref{fig:G_2_prior}, for example, every agent believes in $\theta_{\rm p}$ to a greater or lesser extent, whereas some agents do not believe in $\theta_0$ at all, yet the partial consensus about $\theta_{\rm p} \neq \theta_0$ is misleading as a guide to the true bias.


% Figure environment removed

We now turn to $G_{2{\rm p}2}$, the version of $G_2$ where agent 1 is allied with the partisan (agent 2) and agent 3, who opposes the partisan (Fig.\ \ref{fig:triad}, middle row, right graph).
Note that switching agent 1 with agent 3, or selecting agent 3 as the partisan, leads to a network with the same connections and topology.
% The network when agent 3 is the partisan is identical to the network with agent 2 is the partisan, as they have the same connections and relations to both the partisan and the persuadable agents.  
Figs.\ \ref{fig:G_2_prior_fe} and \ref{fig:G_2_mean_fe} plot the same quantities as Figs.\ \ref{fig:G_2_prior} and \ref{fig:G_2_mean} respectively. 
They agree with the results in Ref.\ \cite{low_discerning_2022}, where agent 2 and 3 achieve asymptotic learning, %at $t_{\rm a} = 2355$, 
and agent 1 experiences turbulent nonconvergence, vacillating between the beliefs of agent 2 and 3.
% That is, selecting agent 2 as the partisan does not vary the behavior observed.
However, agent 3 reaches the right belief upon achieving asymptotic learning, unlike in Ref.\ \cite{low_discerning_2022}, because the partisan does not ``zero out'' agent 3's likelihood at $\theta_0$.

% Figure environment removed


Let us now consider $G_{3{\rm p}2}$, the version of $G_3$ where agent 2 (the partisan) opposes agent 1 and allies with agent 3 (Fig.\ \ref{fig:triad}, bottom row, right graph)\footnote{Network $G_{3{\rm p}1}$ (Fig.\ \ref{fig:triad}, bottom row, left graph), in which agent 1 (the partisan) is opposed to agents 2 and 3, who form a persuadable yet allied bloc, exhibits the same dynamics essentially as an opponents-only network with $n=2$ and one partisan. Both persuadable agents achieve asymptotic learning at $\theta_0$; see Section \ref{sec:opponentonly}.}. 
The results appear in Fig.\ \ref{fig:G_3} in the same format as Fig.\ \ref{fig:G_2}.
Agent 1, opposing both the partisan and agent 3, does not experience peer pressure as the beliefs do not overlap, as discussed in Section \ref{subsec:opponent_frac_partisan}.
Hence, agent 1 only responds to the coin tosses and achieves asymptotic learning at $t_{\rm a1} = 1119$. 
Interestingly, in Fig.\ \ref{fig:G_3_mean}, agent 3 agrees quickly with the partisan (with agreement reached by $t = 7$) and maintains that belief until $t=6100$, when a long interval of turbulent nonconvergence ensues, triggered by a sequence of six heads in a row. 
This is an example of the intermittency phenomenon observed without partisans in Ref.\ \cite{low_vacillating_2022}. 
Agent 3 realizes, via Eqs.\ \eqref{eq:updatefirsthalf} and \eqref{eq:likelihood}, that six heads in a row are unlikely to be consistent with $x_3(t=6100, \theta) \approx \delta(\theta-0.3)$, and hence starts to gain confidence in higher values of $\theta$, without discounting $\theta_{\rm p}=0.3$ completely.
However, agent 1 locks agent 3 out of $\theta = 0.6$, as discussed in Section 4.2 of Ref.\ \cite{low_discerning_2022}.
Hence agent 3 is driven towards the midpoint $\approx (\theta_{\rm p} + \theta_0)/2$, so that $x_2(t,\theta)$ becomes bimodal for $t > 6100$.
We remind the reader that turbulent nonconvergence does not happen in the $G_3$ triad without a partisan, as demonstrated in Ref.\ \cite{low_discerning_2022}. 

The intermittent behavior of agent 3 in Fig.\ \ref{fig:G_3} is only one possible behavior in the $G_{3{\rm p}2}$ triad. 
Agent 3 sometimes exhibits turbulent nonconvergence from the beginning of the simulation, for example, maintaining a bimodal belief PDF with one peak at $\theta_{\rm p}$ and one at $\theta \neq \theta_0$ disagreeing with agent 1. 
The statistics of the alternative forms of intermittent behavior are complicated, as shown in Ref.\ \cite{low_vacillating_2022} even without partisans, and will be studied fully in future work.



\subsection{Larger networks}
\label{subsec:larger_network}

% Figure environment removed

We now consider one representative example of a larger complete network with mixed allegiances, $n =100$, and one partisan. 
The sign of $A_{ij}$ is selected randomly with equal probability for all $i$ and $j$ and yields 2489 edges joining allies and 2461 edges joining opponents in the illustrative example analyzed here. 
Fig.\ \ref{fig:big_mixed_final} shows the belief PDF at $t = 10^4$ for each agent. 
Every PDF features a portion with $x_i(t,\theta) \neq 0$ for $| \theta - \theta_0 | \lesssim 0.1$. The PDFs are unequal for different agents and satisfy $x_i(t,\theta) \neq 0$ for multiple values of $\theta$ for some (but not all) agents. 
In addition, allies of the partisan develop a second peak at $\theta_{\rm p}$ (48 agents in this particular simulation). 
 
Fig.\ \ref{fig:big_mixed_mean} displays how $\langle\theta\rangle$ evolves for four selected agents, each displaying different but typical behavior. 
Agent 14 is allied to the partisan and exhibits turbulent nonconvergence as described in Section \ref{subsec:samethetap}. 
Agents 15, 16, and 30 oppose the partisan, satisfy $x_i(t,\theta_{\rm p})=0$ for $i=$ 15, 16, and 30, and exhibit intermittency as described in Ref.\ \cite{low_vacillating_2022}. 
Agent 15 always has a bimodal belief PDF, with the peaks moving in the range $0.5 \leq \theta \leq 0.7$ but never reaching $\theta_{\rm p} = 0.3$.
Agent 30 dwells for a long time (4238 time steps) at $\langle \theta\rangle \approx 0.6$, with PDF $\approx \delta(\theta - \theta_0)$, then suddenly transitions to turbulent nonconvergence at $t = 4239$, like agent 3 in Fig.\ \ref{fig:G_3_mean}. 
On the other hand, Agent 16 enters the long dwell interval $4165 \leq t \leq T$ with $x_{16}(t,\theta) \approx \delta(\theta-0.55)$.  
Given a longer simulation, it is possible that agent 16 will transition to turbulent nonconvergence at $t > 1\times 10^4$, like agent 30 and agent 3 in Fig.\ \ref{fig:G_3_mean}. 
A similar mixed network with $n=1000$ was also tested and returned similar results (not plotted for brevity). 


% One could characterize the above behaviors and examine each agent by drawing a more general link between the overall connectivity of each agent, as well as the relationship with the partisan.  
In future work, we will study large networks with mixed allegiances in systematic detail, with the aim of linking an agent's behavior to their connectivity in general and their relationship with the partisan in particular (which may be null on occasion, e.g.\ in a Barab\'{a}si-Albert network).




