In this appendix, for the sake of completeness, we discuss briefly the special case, where the coin returns heads or tails only. By way of illustration, we consider an allies-only network with $n = 100$ and one partisan. 
We focus on $\theta_0 = 0$, as the behavior for $\theta_0=1$ is analogous.

For $\theta_0=0$, the persuadable agents achieve asymptotic learning, achieving a bimodal final distribution 
with $x_i(t \geq t_{\rm a},\theta_0) \approx 0.995$ and  $x_i(t \geq t_{\rm a},\theta_{\rm p}) \approx 0.005$, as shown in Fig.\ \ref{fig:0truebias}.  
The persuadable agents still heed the partisan but weakly. 
Unlike the turbulent nonconvergence observed in Section \ref{subsec:samethetap}, all persuadable agents' beliefs asymptotically approach $\theta_0$ and $\theta_{\rm p}$, achieving asymptotic learning at $t_{\rm a} = 136$, as shown in Figs.\ \ref{fig:0truebias_mean}. 
This is because a coin with $\theta_0 = 0$ always returns tails.
From Eq.\ \eqref{eq:likelihood}, the likelihood is given by $P[S(t)| \theta] = 1-\theta $ for all $t$, which always equals one for $\theta = 0$ and is less than one for all other values of $\theta$. 
When the likelihood is multiplied by the prior according to Eq.\ \eqref{eq:undatesecondhalf}, one obtains $x_i(t+1, \theta) < x_i(t, \theta)$ for $\theta \neq 0$, and $x_i(t,\theta) \neq 0$ decreases iteratively and monotonically to zero. 
The second peak at $\theta_{\rm p}$ is obtained though Eq.\ \eqref{eq:xiprimed} from the internal interaction between allies.  
The persuadable agents always hold some belief in $\theta_{\rm p}$, as Eq.\ \eqref{eq:undatesecondhalf} is additive rather than multiplicative.

% Figure environment removed
It is interesting to ask how the persuadable agents respond, when they receive an unexpected external signal.
As a test, we run a simulation with $\theta_0=0$, returning $S(t) = {\rm tails}$ for $ 1\leq t \leq 500$, followed artificially by $S(t=501)= {\rm heads}$. 
The belief PDFs of the persuadable agents at $t = 500$ are identical to the PDF shown in Fig.\ \ref{fig:0truebias} but they change abruptly to $x_i(t=501,\theta) \approx \delta (\theta-\theta_{\rm p})$, after they observed a single heads. This readiness to agree with the partisan after a single, unexpected observation occurs for the following reason.
When persuadable agents observe $S(t=501) = {\rm heads}$, they infer $P[S(t=501)|\theta=0] = 0$ according to Eq.\ \eqref{eq:likelihood}. 
Therefore, we have $x_i(t = 500 + 1/2, \theta = 0) = 0$ via Eq.\ \eqref{eq:updatefirsthalf}, and $x_i(t=500+1/2,\theta_{\rm p}) = 1$ after renormalization.

