%% 
%% Copyright 2007-2020 Elsevier Ltd
%% 
%% This file is part of the 'Elsarticle Bundle'.
%% ---------------------------------------------
%% 
%% It may be distributed under the conditions of the LaTeX Project Public
%% License, either version 1.2 of this license or (at your option) any
%% later version.  The latest version of this license is in
%%    http://www.latex-project.org/lppl.txt
%% and version 1.2 or later is part of all distributions of LaTeX
%% version 1999/12/01 or later.
%% 
%% The list of all files belonging to the 'Elsarticle Bundle' is
%% given in the file `manifest.txt'.
%% 

%% Template article for Elsevier's document class `elsarticle'
%% with numbered style bibliographic references
%% SP 2008/03/01
%%
%% 
%%
%% $Id: elsarticle-template-num.tex 190 2020-11-23 11:12:32Z rishi $
%%
%%
% \documentclass[preprint,12pt]{elsarticle}
%% Use the option review to obtain double line spacing
%% \documentclass[authoryear,preprint,review,12pt]{elsarticle}
%% Use the options 1p,twocolumn; 3p; 3p,twocolumn; 5p; or 5p,twocolumn
%% for a journal layout:
% \documentclass[final,1p,times]{elsarticle}
%% \documentclass[final,1p,times,twocolumn]{elsarticle}
\documentclass[final,3p,times]{elsarticle}
%% \documentclass[final,3p,times,twocolumn]{elsarticle}
% \documentclass[final,5p,times]{elsarticle}
% \documentclass[final,5p,times,twocolumn]{elsarticle}

%% For including figures, graphicx.sty has been loaded in
%% elsarticle.cls. If you prefer to use the old commands
%% please give \usepackage{epsfig}

%% The amssymb package provides various useful mathematical symbols
\usepackage{amssymb}
%% The amsthm package provides extended theorem environments
\usepackage{amsthm}
\usepackage{amsmath}
\usepackage{libertine}
\usepackage{libertinust1math}
\usepackage[backref=page]{hyperref}
\usepackage{paralist}
\usepackage{graphicx}
\usepackage{algorithm}
\usepackage{algpseudocode}
\usepackage{xcolor}
\usepackage{subcaption}
\usepackage{makecell}
\usepackage{xspace}
% no indent at beginning of each para
\setlength{\parindent}{0pt}
\setlength{\parskip}{2.0ex plus0.5ex minus0.2ex} 

\graphicspath{{./figures/}}

\newcommand{\todo}[1]{{\color{red}~\textsf{[{\bf TODO}: #1]}}} %todos
\newcommand{\note}[1]{{\color{blue}~\textsf{[{\bf NOTE}: #1]}}} %note

% \newcommand{\theta_{\rm p}}{\theta_{\rm p}}

% define custom labels with references
\makeatletter
\newcommand\newtag[2]{#1\def\@currentlabel{#1}\label{#2}}
\makeatother

\algnewcommand{\Inputs}[1]{%
  \State \textbf{Inputs:}
  \Statex \hspace*{\algorithmicindent}\parbox[t]{.8\linewidth}{\raggedright #1}
}
\algnewcommand{\Initialize}[1]{%
  \State \textbf{Initialize:}
  \Statex \hspace*{\algorithmicindent}\parbox[t]{.8\linewidth}{\raggedright #1}
}

\newcommand{\etal}{\emph{et al.}\xspace}

%% The lineno packages adds line numbers. Start line numbering with
%% \begin{linenumbers}, end it with \end{linenumbers}. Or switch it on
%% for the whole article with \linenumbers.
%% \usepackage{lineno}

\journal{Physica A}

\begin{document}

\begin{frontmatter}

%% Title, authors and addresses

%% use the tnoteref command within \title for footnotes;
%% use the tnotetext command for theassociated footnote;
%% use the fnref command within \author or \address for footnotes;
%% use the fntext command for theassociated footnote;
%% use the corref command within \author for corresponding author footnotes;
%% use the cortext command for theassociated footnote;
%% use the ead command for the email address,
%% and the form \ead[url] for the home page:
%% \title{Title\tnoteref{label1}}
%% \tnotetext[label1]{}
%% \author{Name\corref{cor1}\fnref{label2}}
%% \ead{email address}
%% \ead[url]{home page}
%% \fntext[label2]{}
%% \cortext[cor1]{}
%% \affiliation{organization={},
%%             addressline={},
%%             city={},
%%             postcode={},
%%             state={},
%%             country={}}
%% \fntext[label3]{}

\title{Discerning media bias within a network of political allies and opponents: Disruption by partisans}

\author[inst1]{Yutong Bu\corref{cor1}}
\ead{buy1@student.unimelb.edu.au}
\cortext[cor1]{Corresponding author}


\affiliation[inst1]{organization={School of Physics},%Department and Organization
            addressline={University of Melbourne}, 
            city={Parkville},
            state={VIC 3010},
            country={Australia}}

\author[inst1,inst2]{Andrew Melatos}

\affiliation[inst2]{organization={Australian Research Council Centre of Excellence for Gravitational Wave Discovery (OzGrav)},%Department and Organization
            addressline={University of Melbourne}, 
            city={Parkville},
            state={VIC 3010},
            country={Australia}}

\begin{abstract}
%% Text of abstract
\begin{abstract}

This paper presents a low-cost network architecture for training large language models (LLMs) at hyperscale. We study the optimal parallelization strategy of LLMs and propose a novel datacenter network design tailored to LLM's unique communication pattern. We show that LLM training generates sparse communication patterns in the network and, therefore, does not require any-to-any full-bisection network to complete efficiently. As a result, our design eliminates the spine layer in traditional GPU clusters. We name this design a \textit{Rail-only} network and demonstrate that it achieves the same training performance while reducing the network cost by 38\% to 77\% and network power consumption by 37\% to 75\% compared to a conventional GPU datacenter. Our architecture also supports Mixture-of-Expert (MoE) models with all-to-all communication through forwarding, with only 8.2\% to 11.2\% completion time overhead for all-to-all traffic. We study the failure robustness of Rail-only networks and provide insights into the performance impact of different network and training parameters. \looseness=-1


\end{abstract}


\end{abstract}

% %%Graphical abstract
% \begin{graphicalabstract}
% \end{graphicalabstract}

% %%Research highlights
% \begin{highlights}
% \item Research highlight 1
% \item Research highlight 2
% \end{highlights}

% \begin{keyword}
% %% keywords here, in the form: keyword \sep keyword
% keyword one \sep keyword two
% %% PACS codes here, in the form: \PACS code \sep code
% \PACS 0000 \sep 1111
% %% MSC codes here, in the form: \MSC code \sep code
% %% or \MSC[2008] code \sep code (2000 is the default)
% \MSC 0000 \sep 1111
% \end{keyword}

\end{frontmatter}
 

%% main text
\section{Introduction}
\label{sec:intro}
\section{Introduction}
\label{sec:introduction}

The recent surge of Large Language Models (LLMs), such as GPT-3.5/4~\cite{bubeck_sparks_2023}, PaLM~\cite{chowdhery_palm_2022}, FLAN-T5~\cite{chung_scaling_2022}, and Alpaca~\cite{taori_stanford_2023}, has shown a promising trend of large pre-trained models to do a variety of tasks in a zero-shot setting (\ie without any new training data). Example tasks include question answering~\cite{omar2023chatgpt,robinson2023leveraging}, logic reasoning~\cite{wei_chain--thought_2023,zhou_least--most_2023}, machine translation~\cite{brants2007large,gulcehre2017integrating} \etc\ 
A number of experiments have revealed that, built on hundreds of billions of parameters, these LLMs have started to show the capability to understand the human common sense beneath the natural language and do proper reasoning and inference accordingly~\cite{bubeck_sparks_2023,nori_capabilities_2023}.

Among different applications, one particular question yet to be answered is how well LLMs can understand human mental health states through natural language.
Mental health problems represent a significant burden for individuals and societies worldwide.
A recent report suggested that more than 20\% of adults in the U.S. would experience at least one mental disorder in their lifetime~\cite{mental2022state} and 5.6\% of adults experienced a serious psychotic disorder that significantly impairs functioning~\cite{mental2023stats}. The global economy loses around \$1 trillion annually in productivity due to depression and anxiety alone~\cite{mentalcost2023}.

In the past decade, there has been a plethora of research in natural language processing (NLP) and computational social science on detecting mental health issues via online text data such as social media~(\eg \cite{guntuku_detecting_2017,eichstaedt2018facebook,coppersmith_clpsych_2015,de_choudhury_social_2013,de_choudhury_mental_2014}). However, most of these studies have focused on building domain-specific machine learning (ML) models (\ie one model for one particular task, such as stress detection~\cite{nijhawan2022stress,guntuku2019understanding}, depression prediction~\cite{eichstaedt2018facebook,tadesse2019detection,xu_leveraging_2019}, or suicide risk assessment~\cite{de_choudhury_discovering_2016,coppersmith2018natural}). Even for traditional pre-trained language models such as BERT, it needs to be finetuned for specific downstream tasks~\cite{devlin_bert_2019,liu_roberta_2019}.
Since natural language is a major component of mental health assessment and treatment~\cite{sharma2018mental,gkotsis2016language}, LLMs might be a potentially powerful tool to understand end-users' mental states based on the language users' wrote. These instruction-finetuned and general-purpose models can understand a variety of inputs and obviate the need to train multiple models for different tasks. Thus, we can envision using one LLM for a variety of mental-health-related tasks, such as multiple question-answering, reasoning, and inference.
Such a vision opens up a wide range of opportunities for UbiComp, Human-Computer Interaction (HCI), and mental health communities, such as online public health monitoring systems~\cite{patel2018psyheal,graham2019artificial}, intelligent assistants for mental counselors and supporters~\cite{sharma_towards_2021,sharma_humanai_2023}, mental-health-aware personal chatbots~\cite{abd2021perceptions,denecke2020mental}, to just name a few.
However, there is a lack of investigation into understanding, evaluating, and improving the capability of LLMs for mental health prediction tasks.

There are few very recent studies on the evaluation of LLMs (\eg ChatGPT) on mental-health-related tasks, most of which are in zero-shot settings with simple prompt engineering~\cite{yang_evaluations_2023,amin_will_2023,lamichhane_evaluation_2023}. Researchers have shown preliminary results that LLMs have some initial capability of predicting mental health disorders with natural language with some promising but still limited performance compared to state-of-the-art domain-specific NLP models~\cite{yang_evaluations_2023,lamichhane_evaluation_2023}.
This remaining gap is expected since existing general-purpose LLMs are not specifically trained on mental health tasks.
However, to achieve our vision of leveraging LLMs for mental health support and assistance, we need to answer the research question: \textbf{How to empower LLMs with more mental health domain knowledge and become an expert}?

We conducted a series of experiments with multiple LLMs, including Alpaca~\cite{noauthor_stanford_2023}, Alpaca-LoRA~\cite{hu_lora_2021}, and GPT-3.5~\cite{noauthor_introducing_2022}.
Considering the data availability, we focused on online social media data with high-quality human-generated mental health labels.
Our experiments contained three stages: (1) zero-shot prompting, where we experimented with various prompts related to mental health, (2) few-shot prompting, where we inserted examples into prompt inputs, and (3) instruction-finetuning, where we finetuned LLMs on multiple mental-health datasets with various tasks.

Our results indicate that zero-shot obtained promising but limited performance on multiple mental health prediction tasks across all models. GPT-3.5 had relatively better results since it has a larger scale. But their performance is still far from task-specific models. 
Meanwhile, providing a few shots in the prompt can improve the model performance to some extent ($\overline{\Delta}$ = 4.7\%), but the advantage is limited.
Finally and most importantly, we found that instruction-finetuning can significantly improve the model performance across multiple mental-health-related tasks at the same time. Our finetuned Alpaca, namely \textbf{Mental-Alpaca}, significantly outperforms the original GPT-3.5 ($\times$25 times of model size) by an average of 16.7\% on balance accuracy. 
Meanwhile, Mental-Alpaca can further perform on par with the task-specific state-of-the-art Mental-RoBERTa~\cite{ji_mentalbert_2021}. It is noteworthy that Mental-RoBERTa needs to be trained on each task individually, 
while our Mental-Alpaca can solve different tasks off the shelf. 
% We open-source our training code and model at [github link].
Our experiments present the first comprehensive evaluation of various techniques to enhance LLMs' capability in the mental health domain.

The contribution of our paper can be summarized as follows:
\begin{s_enumerate}
\item We present the first comprehensive evaluation of prompt engineering, few-shot, and finetuning techniques on multiple LLMs in the mental health domain.
\item With online social media data, our results reveal that finetuning on a variety of datasets can significantly improve LLM's capability on multiple mental-health-specific tasks simultaneously.
% We release our model \textbf{Mental-Alpaca} as the first open-source LLM targeted at mental health prediction tasks.
\item We provide a few technical guidelines for future researchers and developers on turning LLMs into experts in specific domains.
\end{s_enumerate}


\section{An idealized model of media bias: inferring the bias of a coin}
\label{sec:partisans}
% add paragraph
In this section, we extend the media bias model formulated by Low \& Melatos \cite{low_discerning_2022} to include obdurate partisans.  Section \ref{subsec:modelintro} summarizes how the evolution of perceptions about media bias maps onto an idealized model, in which a network of political allies and opponents infers the bias of a coin. A two-step update rule is presented, which updates the belief of each networked agent in response to independent observations of the coin and peer pressure from the network \cite{low_discerning_2022,low_vacillating_2022}.  
The implementation of partisans is described in Section \ref{subsec:intropartisans}, and the pseudocode for the resulting, iteratively updated automaton is presented in Section \ref{subsec:automaton}.  
The paper focuses on complete networks for technical reasons justified in Section \ref{subsec:network}. Complete networks are adequate to demonstrate the central points of the paper, but the study of other network topologies is an important issue which is deferred to future work (except for a brief discussion in Sections \ref{sec:opponentonly} and \ref{sec:BA}). 
The extended idealized model is compared with other models in the literature in Section \ref{subsec:relationtoothers}. 



\subsection{Updating beliefs in two steps: independent observations and peer pressure}
\label{subsec:modelintro}
In Ref.\ \cite{low_discerning_2022}, a network of $n$ agents attempts to learn the true bias $0 \leq \theta_0 \leq 1 $ of a coin though a sequence of $T$ coin tosses, where $\theta_0$ denotes the probability of heads for a single coin toss.  The coin toss is an analog for a piece of politically relevant information (e.g.\ editorials, articles), which the agents consume from a media outlet, and $\theta_0$ is an analog for the political bias of the media outlet (e.g.\ on a simplified, left-right spectrum).  The $i$-th agent harbors probabilistic beliefs about the political bias of the media outlet, or analogously the bias of the coin, which are described by the probability density function (PDF) $x_i(t,\theta)$ at time $t$. The beliefs can be multi-valued and hence uncertain in general; the agent may be equally confident about two distinct values of the coin's bias, for example, with $x_i(t,\theta_1)=0.5=x_i(t,\theta_2 \neq \theta_1)$. 
\footnote{Multimodal belief PDFs are realistic in various settings.  In the media bias context, for example, a reader of a newspaper's editorials may be unsure as to whether the newspaper leans right or left politically, if the editorials lean right on economic issues and left on social issues.  The reader may develop a bimodal belief PDF in this scenario (i.e., the newspaper actually leans right or left but not both, and the reader is unsure which option to prefer) instead of a unimodal belief PDF (i.e., the newspaper occupies the middle of the road politically, and the reader believes its bias lies between the right and left extremes).  Multimodality is more likely, when there are multiple ways to map the underlying bias variable to the observed output signal  (e.g., editorials), and readers (and maybe even the newspaper's editors themselves) do not know consciously which mapping applies in practice. The point is especially pertinent, when a reader overlays an oversimplified mental model (e.g., left-right dichotomy) on a more complicated bias structure (e.g., left-right economic-social matrix). }
Starting from initial PDFs $x_1(t=0,\theta), \dots, x_n(t=0,\theta)$, the beliefs of the $i$-th agent in the network are updated in two steps, every time a coin toss occurs. 
\begin{enumerate}
    \item The $i$-th agent observes the coin toss at time $t$ and invokes Bayes's rule to combine their prior beliefs at time $t$ with the result of the coin toss to generate an intermediate, provisional PDF $x_i'(t+1/2,\theta)$.
    \item The $i$-th agent is influenced by positive and negative peer pressure from their allies and opponents respectively and adjusts $x_i'(t+1/2,\theta)$ accordingly to obtain the fully updated PDF at the conclusion of that time step, $x_i(t+1,\theta)$.
\end{enumerate}
The outcome of each coin toss is a public external signal, which is simultaneous and unfiltered for all agents, i.e.\ all agents observe the outcome and accept it without demur.

The first half of the update rule above, based on an independent observation of the coin, follows immediately from Bayes's rule. We write
\begin{equation}\label{eq:updatefirsthalf}
    x'_i(t+1/2, \theta) = \frac{P[S(t) | \theta]}{P[S(t)]}x_i (t, \theta),
\end{equation}
where $S(t)$ is the outcome of the coin toss (heads or tails), $P[S(t)| \theta]$ is the likelihood function, and $P[S(t)] = \sum_{\theta}P[S(t)| \theta]x_i (t, \theta)$ is the normalizing constant. The likelihood function takes the form
\begin{equation}\label{eq:likelihood}
    P[S(t)| \theta] = \begin{cases} \theta, & \mbox{if } S(t)\mbox{ is heads} \\
    1-\theta,  & \mbox{if } S(t)\mbox{ is tails.} \end{cases}
\end{equation}

The second half of the update rule shapes the belief of each agent under the influence of political allegiances.  In much of the literature, agents are typically conditioned to copy, or at least move towards, the beliefs of their neighbors in the network; that is, they implicitly regard their neighbors as allies \cite{degroot_reaching_1974,hegselmann_opinion_2002,fang_social_2019,jadbabaie_non-bayesian_2012,deffuant_mixing_2000}.  In this paper, we generalize the network to include allies and opponents, such that antagonistic  interactions involving negative feedback between agents are taken into account.  The network is modelled by a graph, where nodes represent agents, and edges are tagged with the political allegiance between two agents.  A weighted graph with $n \times n$ adjacency matrix $A$ is used with entries 
\begin{equation}
    A_{ij} = 
    \begin{cases}
        + 1 , \quad & \text{agents $i$ and $j$ are allies} \\
        0 , \quad & \text{agents $i$ and $j$ are disconnected} \\
        - 1 , \quad & \text{agents $i$ and $j$ are opponents} \\
    \end{cases}
\end{equation}
We note that agents $i$ and $j$ can be connected (and therefore influence each other) indirectly through one or more third parties (e.g. $A_{ik}\neq0$ and $A_{kj} \neq 0$ for $k \neq i,j$), even if they are not connected directly ($A_{ij}=0$).  In this paper, we assume for simplicity that $A$ is symmetric (i.e.\ $A_{ij} = A_{ji}$),  and that allies and opponents exert equal and opposite peer pressure.

Under peer pressure from allies and opponents, the provisional belief PDF at $t+1/2$ of the $i$-th agent is updated according to
\begin{equation}\label{eq:undatesecondhalf}
    x_i(t+1, \theta) \propto \max\left[0, x_i '(t+1/2, \theta) + \mu \Delta x_i '(t + 1/2, \theta)\right]
\end{equation} 
to give the PDF at time $t+1$, concluding the two-step update rule. The proportionality constant is set by normalization.  The learning rate $0.0 < \mu \leq 0.5$ quantifies the susceptibility of an agent to their neighbors' beliefs and is covariant with the duration of each time step; halving $\mu$ is equivalent to doubling the time step without loss of generality.  The inequality $\mu \leq 0.5$ prevents the beliefs from overshooting and is borrowed from the Deffuant-Weisbuch model \cite{deffuant_mixing_2000}.  The displacement 
\begin{equation}\label{eq:xiprimed}
    \Delta x_i '(t + 1/2, \theta) = a_i^{-1} \sum_{i\neq j} A_{ij}[x_j '(t+1/2, \theta) - x_i '(t+1/2, \theta)]
\end{equation}
with $a_i = \sum_{j \neq i} |A_{ij} |$, equals the average difference in belief (at each $\theta$) between agent $i$ and all its neighbors. We only consider connected graphs, where every agent has at least one ally or opponent, i.e.\ $a_i \neq 0$ for all $i$. Equations \eqref{eq:undatesecondhalf} and \eqref{eq:xiprimed} drive an agent's belief towards its allies ($A_{ij} > 0$) and away from its opponents ($A_{ij} < 0$).  Without the maximization operation in equation \eqref{eq:undatesecondhalf}, the antagonistic interaction can produce $x_i(t+1, \theta) < 0$ via equation \eqref{eq:xiprimed}, which is invalid for a probability density; see Section 2.3 in Ref.\ \cite{low_discerning_2022} for more details.  All agents observe the coin toss and combine their prior belief with the result of the coin toss synchronously via equation \eqref{eq:updatefirsthalf} before agents are influenced by positive or negative peer pressure from the network via equation \eqref{eq:undatesecondhalf}, which completes the full time-step from $t$ to $t+1$. 

\subsection{Implementation of obdurate partisans}
\label{subsec:intropartisans}

A partisan is obdurate, in the sense that their opinion does not change with time in response to independent observations or peer pressure. 
In general, the PDF of a partisan can take any time-independent form, viz. $x_i(t,\theta)=x_i(\theta)$.  
In this paper, we  specialize mainly to the situation where partisans hold a single belief, with $ x_i(t,\theta) = \delta(\theta-\theta_{\rm p})$ for all $t$, except in Section \ref{subsec:partisandifferentopinion} where we treat dueling partisans.  
One key goal of the analysis is to test how the behavior of the network depends on $\theta_{\rm p}$. 
For example, does the impact of the partisan increase, as $|\theta_{\rm p} - \theta_0|$ increases?  
In general, when $x_i(\theta)$ is not simply a delta function, one can pose a more sophisticated question: can a partisan increase or decrease their impact on persuadable agents by adjusting the shape of $x_i(\theta)$? 
This question is relevant, when a partisan's beliefs are shaped by a deliberate strategy to manipulate the opinions of other agents in the network, instead of subconscious, psychological factors. Its study is postponed to future work.


In this paper, a partisan's PDF is implemented as a narrow Gaussian distribution with mean $\theta_{\rm p}$ and standard deviation $\sigma_{\rm p} = 10^{-3}$, truncated to the domain $0 \leq \theta \leq 1$.  Although obdurate partisans do not change their opinions, other agents still interact with them, learn from the partisans, and evolve their opinions accordingly. 


In this paper, for numerical convenience only, we apply Eqs.\ \eqref{eq:updatefirsthalf} and \eqref{eq:undatesecondhalf} to all agents synchronously, both persuadable and obdurate. We then reset the partisans' PDFs to their initial forms after applying Eq.\ \eqref{eq:updatefirsthalf} and do so again after applying Eq.\ \eqref{eq:undatesecondhalf}, to ensure that the partisans' beliefs are unchanged by observing the coin toss or interacting with other agents. In our implementation, the network graph is undirected, and the update steps in Eqs.\ \eqref{eq:updatefirsthalf} and \eqref{eq:undatesecondhalf} involve matrix multiplications (see \ref{sec:zoomzoom} for a more detailed discussion). Hence it is simpler to reset beliefs instead of checking for partisanship when applying Eqs.\ \eqref{eq:updatefirsthalf} and \eqref{eq:undatesecondhalf} via matrix multiplication.  Alternatively, one can employ a mask matrix to locate partisans in the network, and multiply by the mask matrix when evaluating Eqs.\ \eqref{eq:updatefirsthalf} and \eqref{eq:undatesecondhalf}, which costs additional space and runtime.  If we extend the model, such that $A$ is no longer symmetric, i.e.\ $A_{ij} \neq A_{ji}$, and use a directed graph to represent the network, partisans can be implemented as nodes with zero out-degree.
\footnote{That is, one sets $A_{i{\rm p}} = 0$ and $A_{{\rm p }i} \neq 0$ for all persuadable agents $i$ who are adjacent to the partisan.}



\subsection{Automaton}
\label{subsec:automaton}

We present a discrete-time automaton in Algorithm \ref{alg:algo} to implement the model in Sections \ref{subsec:modelintro} and \ref{subsec:intropartisans} and show clearly how the numerical simulation is set up and run.

\begin{algorithm}
    \caption{Probabilistic discrete-time automaton for the idealized coin bias application}\label{alg:algo}
    \begin{algorithmic}
        \Inputs {
            network topology $A_{ij}$\\
            \mbox{true coin bias $\theta_0$, partisan coin bias $\theta_{\rm p}$, learning rate $\mu$,
            maximum time-step $T$}
        }
        \Initialize {
            $x_p(t=0, \theta) \gets$ truncated Gaussian with mean $\theta_{\rm p}$, standard deviation $10^{-3}$\\  % 20 if theta in [theta_p - 0.025, theta_p + 0.025]
            \mbox{$x_{i\neq \rm p}(t=0, \theta) \gets$ truncated Gaussian with mean $\in [0, 1]$, standard deviation $\in [0.2, 0.8]$}
        }\\
        \State $t \gets 0$
        \Repeat { (for each time-step $t$)}
            \State Select $S(t) \sim$ Bernoulli $(\theta_0)$ \Comment{Simulate coin toss}
            \State Update beliefs of all agents in response to one coin toss at each time-step (Eq.\ \eqref{eq:updatefirsthalf})
            \State Reset beliefs of partisan agents to $x_p(t=0,\theta)$
            \State Blend beliefs of all agents with network neighbors (Eq.\ \eqref{eq:undatesecondhalf})            
            \State Reset beliefs of partisan agents to $x_p(t=0,\theta)$
            \State $t \gets t + 1$
        \Until{$t = T$}
        % all agents reach asymptotic learning (Eq.\ \eqref{eq:asymlearncondition}), or 
    \end{algorithmic}
\end{algorithm}


% DONE:  start this with a para which introduces the pseudocode and walks the reader through its general structure and key points - this isn't a repetition of the code, rather a guide to help the reader follow it 
The simulation is initialized with values of $\theta_0, \theta_{\rm p}, \mu, T$ and a network topology $A_{ij}$, which determines the number, connectivity, and political allegiances of the agents. The beliefs of persuadable agents are initialized as truncated Gaussian distributions, with randomly chosen mean in the range $[0.0, 1.0]$ and standard deviation in the range $[0.2,0.8]$.  
That is, one obtains $x_{i\neq {\rm p}} (t=0,\theta) \neq x_{j\neq {\rm p}} (t=0,\theta)$ for all $i \neq j$ in general.  We adopt the Gaussian instead of a uniform distribution, because the right-hand side of Eq.\ \eqref{eq:xiprimed} equals zero for uniform distributions, whereupon none of the agents would subsequently change their beliefs, which is not illuminating.
The beliefs of the partisans are initialized as discussed in Section \ref{subsec:intropartisans}.  For the purpose of numerical simulation, the continuous variable $\theta$ is discretized into 21 regularly-spaced values, $\theta \in \{0.00, 0.05, \ldots, 1.00\}$, following Ref.\ \cite{low_discerning_2022,low_vacillating_2022}\footnote{Interestingly, Ref.\ \cite{tee_quantized_2019} finds that the human brain quantizes probability into $\approx 16$ bins, based on observing the in-game decisions and profits of human gamblers. }.
This reflects the practical reality that human beliefs about media bias are coarse-grained.
% Figure environment removed

At each time step in Algorithm \ref{alg:algo}, a coin toss is simulated by a Bernoulli trial with success probability $\theta_0$.  

We are interested in the long-term behavior of the persuadable agents in the system.  For example, does the PDF of a particular persuadable agent tend to a constant function of $\theta$ or does it fluctuate indefinitely, as in the phenomenon of turbulent nonconvergence observed in Ref.\ \cite{low_vacillating_2022,mobilia_role_2007}?  If the belief of an agent remains unchanged within some tolerance (e.g. 1\%) for a user-selected amount of time $\tau_{\rm max}$, viz.\
\begin{equation} \label{eq:asymlearncondition}
    \max_{\theta} |x_i(t+\tau, \theta) - x_i(t, \theta)| < 0.01 \max_{\theta} |x_i(t, \theta)| \quad \text{for } 1 \leq \tau \leq \tau_{\rm max} , 
\end{equation}
then we say that the agent achieves asymptotic learning. If all the agents achieve asymptotic learning, viz.\
\begin{equation} \label{eq:sysasymlearncondition}
    \max_{\theta} |x_i(t+\tau, \theta) - x_i(t, \theta)| < 0.01 \max_{\theta} |x_i(t, \theta)| \quad \text{for } 1 \leq \tau \leq \tau_{\rm max} \text{ and all persuadable } i, 
\end{equation}
then we say that the system achieves asymptotic learning.  In this paper, we take $\tau_{\rm max} = 99$ typically, an arbitrary choice also made in Ref.\ \cite{low_discerning_2022}.  
If Eq.\ \eqref{eq:sysasymlearncondition} holds for $t \geq t_{\rm a}$, we call $t_{\rm a}$ the asymptotic learning time.  
The automaton iterates until the system achieves asymptotic learning, or the maximum run-time $T$ is reached.  
In this paper, we adopt the convention that the partisans are not part of the conditions \eqref{eq:asymlearncondition} and \eqref{eq:sysasymlearncondition} for asymptotic learning; they are obdurate, so they do not learn anything.

We aim to discover
\begin{inparaenum}[(i)]
    \item if partisans change $t_{\rm a}$, or even prevent the system from achieving asymptotic learning at all; and 
    \item if partisans mislead the persuadable agents to infer the bias of the coin incorrectly. 
\end{inparaenum}
Both questions (i) and (ii) have been asked in the context of deterministic models previously \cite{mobilia_does_2003,mobilia_role_2007,yildiz_binary_2013,mobilia_voting_2005,yildiz_discrete_2011,yildiz_opinion_2021,abrahamsson_opinion_2019,galam_role_2007,ghaderi_opinion_2014,klamser_zealotry_2017}, but they are raised here in the context of media bias and Bayesian learners for the first time. 
First, we consider allies-only networks in Section \ref{sec:alliesonly} and investigate how different numbers of partisans disrupt the beliefs of persuadable agents.  
We then investigate how partisans disrupt opponents-only networks in Section \ref{sec:opponentonly}.
Later, in Section \ref{sec:mixed}, we explore networks with both allies and opponents.
% and networks with both allies and opponents in Section \ref{sec:mixed}. 


% partisan setup
% initial belief of persuadables is gaussian
% asymptotic learning
% algorithm (listing) eg: file:///Users/buyutong/Downloads/4221-Article%20Text-7275-1-10-20190705.pdf


\subsection{Mathematical model of networks: complete versus Barab\'{a}si-Albert}
\label{subsec:network}
% Talk about what we used complete and why: 
% \begin{itemize}
%     \item all agents have the same connection to the partisan 
%     \item easy to quantify the impact of the partisan 
%     \item however, not a good representation of the society
% \end{itemize}
% But BA: 
% \begin{itemize}
%     \item more realisitic to the society (cite stuff, check nic's paper)
%     \item not all agaents will have the same connectivity to the partisan 
%     \item some agants are not connected directly, can see if there is an indirect impact
%     \item room for future work, show we have thought about this but not gonna do in this paper 
% \end{itemize}
We use mathematical graphs to model the sociopolitical connections between agents. In this paper we study complete graphs, which ensure that all agents (partisan and persuadable) are connected to each other.   This is not realistic in most actual social contexts, where everybody does not know everybody else.
%  DONE: explain clearly here the technical reason you explained to me before as to why incomplete graphs would cause us major difficulties - this is very important.
However, the central goal of this paper is to quantify how partisans disrupt opinion formation among persuadable agents.  
This is hard to test in a controlled fashion, when the network is partly connected.  
For example, some persuadable agents may connect mostly with partisans, while others may connect mostly with fellow persuadable agents, and the two groups exchange members unpredictably, when partly connected networks are generated randomly.  
In contrast, it is easier to perform controlled tests on a complete network, where everybody knows the same number of partisans or persuadable agents.  
In this paper, we investigate systematically how the connectivity of partisans affects opinion formation by adjusting the fraction of partisans in a complete network in Sections \ref{subsec:dwelltime_and_frac} and \ref{subsec:switching_between_belief}, secure in the knowledge that every persuadable agent knows every partisan in every random realization of the system. 
In future work, we intend to grapple with the challenge of partly connected networks and generalize the results to Barab\'{a}si-Albert networks, for instance, following Low \& Melatos \cite{low_discerning_2022,low_vacillating_2022}.
(Some preliminary simulations involving Barab\'{a}si-Albert networks are presented in Sections \ref{subsec:wrong_conclusion_first} and \ref{sec:BA}.)  
Barab\'{a}si-Albert networks are scale-free; their degree distribution follows a power law, which is a good approximation to many real-world networks \cite{barabasi_emergence_1999,tang_survey_2016,kumar_structure_2016,maniu_building_2011}.
In future work, for example, one can investigate the behavior of persuadable agents indirectly connected to a partisan via a chain of allies and/or opponents, and compare the influence of highly versus sparsely connected partisans.

In the studies by Mobilia \etal on zealotry \cite{mobilia_does_2003,mobilia_voting_2005,mobilia_role_2007,mobilia_nonlinear_2015}, the network is implemented on a $d$-dimensional cubic lattice of size $(2L + 1)^d$, where each agent (voter) is labelled by a vector $r$ with components $-L \leq r_i \leq L$ and $1 \leq i \leq d$. 
% DONE: insert a sentence saying whether or not the Mobilia lattice is complete (and why/why not) and whether it is scale-free (like BA) and why/why not
The lattice model used by Mobilia \etal is not a complete network as each node is only connected to its nearest neighbors.  Nor is it scale-free, as the degree distribution does not follow a power law; the degree of each node is constant and equals $2d$ for a $d$-dimensional cubic lattice.
% In Ref. \cite{vilela_three-state_2020}, Vilela et.\ al.\ extend the voter model by using Barab\'{a}si-Albert model 
The system is evolved by picking a random agent and checking whether or not they are partisans (zealots).  If they are partisans, nothing happens, as a partisan's opinion never changes. If not, the agent adopts the opinion of a random nearest neighbor.
This approach differs from the current paper, where every agent interrogates the PDF of every other agent to whom they are connected at every time step (as well as the public coin toss, of course).


\subsection{Relation to other models}
\label{subsec:relationtoothers}
The model introduced in previous sections is distinct from but related to models in the literature, particularly the DeGroot model \cite{degroot_reaching_1974} and the Deffuant-Weisbuch model \cite{deffuant_mixing_2000}.  
In the Deffuant-Weisbuch model, the adjacency matrix is modified according to  $A_{ij} \neq 0 $ for $|x_j(t) - x_i(t)| < \epsilon$, i.e.\ when the opinions of two agents differ sufficiently, they stop communicating with each other.  In contrast, the model in Section \ref{subsec:modelintro} holds $A_{ij}$ constant in time and therefore neglects the time-dependent ``echo chamber'' or ``silo'' effect captured by the Deffuant-Weisbuch model \cite{deffuant_mixing_2000}.
The belief displacement $\Delta x_i'$ in Eq.\ \eqref{eq:undatesecondhalf} takes inspiration from the DeGroot model, which averages the opinion of an agent's allies. 
The DeGroot and Deffuant-Weisbuch models are deterministic.  
A few models do exist that describe the opinion of an agent as a PDF like in this paper, including \cite{fang_opinion_2020,fang_social_2019,jadbabaie_non-bayesian_2012} for example.  
In particular, Fang \etal's \cite{fang_social_2019} model allows the external signal to be interpreted differently by agents, unlike in this paper, where the external signal is broadcast to and accepted by every agent.
Moreover, Fang \etal \cite{fang_social_2019} implemented a dynamic network, in which $A_{ij}$ depends on $x_i(t,\theta) - x_j(t,\theta)$, whereas $A_{ij}$ is static in this paper.
A more detailed comparison between the model in this paper and others is presented in Section 2.4 in Ref.\ \cite{low_discerning_2022}.  

% deterministic
Most existing work investigating zealots treats the opinions of persuadable agents deterministically, i.e.\ defined by a single number rather than a PDF.  Often, the opinions are also binary, e.g.\ voting for one party or the other. The voter model \cite{liggett_stochastic_nodate,castellano_statistical_2009}, for example, represents the opinion of agents by a discrete spin value ($+1$ or $-1$).  
Some papers about zealots test for the long-term emergence of a consensus, implying general agreement among all agents, whereas the notion of asymptotic learning in Eq.\ \eqref{eq:asymlearncondition} allows agents to believe steadily in a mixture of views, e.g. the $i$-th agent may hold two opinions $\theta_1$ and $\theta_2 \neq \theta_1$ with equal confidence for $\tau_{\rm max}$ consecutive time steps.

We are not aware of any model that considers antagonistic interactions between partisans and agents in the context of the media bias problem involving Bayesian learners. However, extensions of the DeGroot and Deffuant-Weibuch models certainly exist, that consider antagonistic interactions without partisans \cite{vaz_martins_mass_2010,shi_evolution_2016,chen_opinion_2019,he_discrete-time_2021}. The latter papers focus on the divergence of opinion, when antagonism is introduced. 

\section{Allies only}
\label{sec:alliesonly}

We prepare to study mixed networks containing allies and opponents by investigating first the impact of partisans in simpler networks containing allies only. A representative example of the evolution of an allies-only network is presented in Section \ref{subsec:samethetap}. It exemplifies the general result, that even a single partisan prevents the system from achieving equilibrium (asymptotic learning), because the coin and the partisan ``pull'' the persuadable agents in opposite directions. 
% DONE: add sentences introducing sec 3.3 and 3.4 too please
The distribution of dwell times in specific states, and the role played by the fraction of partisans in the network, are explored in Sections \ref{subsec:dwelltime_and_frac} and \ref{subsec:switching_between_belief} respectively. 
Section \ref{subsec:partisandifferentopinion} explores the opinion dynamics when there exist partisans with different beliefs.  The special cases $\theta_0=0$ and $\theta_0 = 1$ are discussed in \ref{sec:twoallies}. 

For the rest of the paper, except where stated otherwise, we take $\theta_0=0.6$, $\theta_{\rm p} = 0.3$, and $\mu = 0.25$. These values are arbitrary but representative.

\subsection{Disruption by a single partisan}
\label{subsec:samethetap}

Consider an allies-only network with $n=100$, containing 99 persuadable agents, and one partisan labelled `agent p'. We run a number of simulations with randomized priors and coin toss sequences. In this section, we focus on a representative simulation, whose output is plotted in Fig.\ \ref{fig:meanconverge}.
We find that even a single partisan disrupts the perceptions of the persuadable agents, by dragging them away from believing in the true coin bias. 
\ref{sec:coinvsprior} discusses how the priors and coin toss sequences influence the long-term behavior of the system. 
% DONE: check
In short, the priors are largely inconsequential, but a specific subsequence of coin tosses (say, five heads in a row) can have an appreciable impact on persuadable agents. 


% DONE: move this later to when we first talk about convergence - we need to set the stage for the reader first about what's happening qualitatively, before we quantify 

In a network containing only allies, persuadable agents' beliefs converge on a relatively short timescale of order 10 time steps, before vacillating between $\theta_0$ and $\theta_{\rm p}$ over the longer term.  Fig.\  \ref{fig:meanconvergent} and Fig.\ \ref{fig:meannonpartisan} display the evolution of the mean belief $\langle\theta\rangle$ in one particular simulation.  
The bold black dashed line represents the partisan, whose mean belief is constant.  
We run an ensemble of simulations with randomized prior and coin toss sequences, choosing different values for $\theta_0$ and $\theta_{\rm p}$ but excluding the special case of $\theta_0 = 0$ or 1 (i.e.\ the coin returns heads or tails only).
% DONE: I don't understand this last sentence - the previous sentence is all about an ensemble of simulations, now we're saying one simulation - very confusing
For brevity, we present the simulation described in Fig.\ \ref{fig:meanconverge} to demonstrate how a single partisan disrupts the belief of persuadable agents.
Similar behavior is observed throughout this ensemble for all different $\theta_0 \neq 0, 1, \theta_{\rm p}$.  

% Figure environment removed


Over the short term, the persuadable agents reach a consensus, in the sense that they agree among themselves, viz.\
$\text{max}_{\theta} |x_i(t,\theta) -x_j(t,\theta) | < \epsilon \text{ max}_{\theta} \left[ x_i(t, \theta), x_j(t, \theta) \right]$, 
for any pair of persuadable agents $i$ and $j$, where $\epsilon = 0.01$ is a user-selected tolerance.  
This is observed in Fig.\ \ref{fig:meanconvergent}, where $\langle \theta \rangle$ converges rapidly to approximately the same value for all 99 persuadable agents over $\sim 10$ time steps. 
Visually this is observed as a reduction in the spread of the 99 colored, zig-zag curves, as time passes. 
The zig-zag dynamics are a typical response to the coin: if it comes up heads, $\langle\theta\rangle$ increases, otherwise $\langle\theta\rangle$ decreases. For example, $\langle\theta\rangle$ increases monotonically for $4 \leq t \leq 8$, because the coin returns four heads in a row. 
At $t=14$, one has $\langle \theta \rangle \approx 0.58 \neq \theta_{\rm p}$ for all 99 persuadable agents; that is, the partisan is yet to exert their influence fully. 


Consensus is not the same as asymptotic learning. Over the long term, the persuadable agents vacillate as a group between believing in $\theta=\theta_0$ and $\theta=\theta_{\rm p}$, maintaining consensus among themselves. This is observed in Fig.\ \ref{fig:meannonpartisan} where $\langle \theta \rangle$ is plotted versus time for an arbitrary, representative, persuadable agent.  
Without any partisans, namely the situation shown by the blue curve in Fig.\ \ref{fig:meannonpartisan}, one finds $\theta \rightarrow \theta_0$ monotonically for large $t$, irrespective of the coin toss sequence, as in Fig.\ 2 in Ref.\ \cite{low_discerning_2022}. 
In contrast, the orange curve in Fig.\ 2b fluctuates in the range $0.32 \leq \langle\theta\rangle \leq 0.60$. The partisan disrupts the beliefs of the persuadable agents, causing them to vacillate indefinitely, even when there is only one partisan among 99 persuadable agents. 
The mean belief $\langle\theta\rangle$ spends $\approx 86 \%$ and $0.04 \%$ of the $8\times 10^3$ timesteps (after the initial transient $0\leq t \leq 2\times 10^3$) in Fig.\ \ref{fig:meanhist} near $\theta_0$ and $\theta_{\rm p}$ respectively, with $ | \langle\theta\rangle - \theta_0 | \leq 0.025 \theta_0$ and $ | \langle\theta\rangle - \theta_{\rm p} | \leq 0.025 \theta_{\rm p}$, and spends $\approx 14\%$ of the time near neither value. The dwell time in each state is quantified further in Section \ref{subsec:switching_between_belief}.

Once the persuadable agents gather enough information to move beyond their initial priors, at $t \gtrsim 2\times 10^3$, their belief PDF vanishes uniformly except at $\theta_0$ and $\theta_{\rm p}$, as shown in  Fig.\ \ref{fig:0.6truebias}.  
% The probability of $\theta_0$ and $\theta_{\rm p}$ (height of the two peaks) changes as indicated in Fig.\ \ref{fig:meannonpartisan} as mean for the distribution can be simplified as $\langle \theta \rangle =  {\rm Pr}(\theta_0) \cdot \theta_0 +  {\rm Pr}(\theta_{\rm p}) \cdot \theta_{\rm p}$. 
The heights of the peaks at $\theta_0$ and $\theta_{\rm p}$ fluctuate continuously. Fig.\ \ref{fig:meanhist} shows a histogram of $\langle \theta \rangle$ for $2 \times 10^3\leq t \leq 1\times 10^4$, with $\langle\theta\rangle$ sampled at each of the $8\times 10^3$ intervening time steps.
With only one partisan, we find $\langle\theta\rangle \geq 0.56$ during 90\% of the time. That is, the persuadable agents prefer $\theta_0$ without rejecting $\theta_{\rm p}$ completely. These preferences reverse, as the fraction of partisans increases, as discussed in Section \ref{subsec:switching_between_belief}. 
% As the internal signal is proportional to the difference in belief as shown in Eq.\ \eqref{eq:undatesecondhalf}, the internal interaction between persuadable allies is negligible after around 10 time steps.  Additionally, the persuadable agents observe the coin tosses equally, and they are each allied with the partisan, therefore their beliefs change simultaneously after they have converged.   
When the persuadable agents form a consensus around the true coin bias, with $x_j(t,\theta) \approx \delta(\theta - \theta_0)$ for most $j$, the displacement $\Delta x_i'$ in Eq.\ \eqref{eq:xiprimed} is dominated by the ``pull'' of the disagreeing partisan, who has $x_{\rm p}(t,\theta) = \delta(\theta - \theta_{\rm p})$ with $\theta_{\rm p} \neq \theta_0$. All the persuadable agents feel the partisan's ``pull'' simultaneously, as well as the influence from the coin toss described by Eqs.\ \eqref{eq:updatefirsthalf} and \eqref{eq:likelihood}. 
Hence the characteristic timescale to move away from $\langle \theta \rangle \approx \theta_0$ depends on $\mu^{-1}$ and the coin toss sequence. 
On the other hand, when the persuadable agents form a consensus around the partisan's belief, with $x_j(t,\theta)=\delta(\theta-\theta_{\rm p})$ for most $j$, the internal signals and hence the displacement $\Delta x_i'$ are small, and the coin toss dominates the update rule via Eq.\ \eqref{eq:updatefirsthalf} and \eqref{eq:likelihood}. The characteristic timescale to move away from $\langle\theta\rangle \approx \theta_{\rm p}$ depends only on the coin toss sequence.  
Hence, on balance, the persuadable agents feel a more durable ``pull'' when the consensus is formed around $\theta_0$ instead of $\theta_{\rm p}$. 
% DONE: [comment here on which happens faster, the partisan pull or coin toss sequence, if possible - numbers would help]

% DONE: Moblia para here and talk about why is it different 
The behavior of a network containing one partisan differs from Fig.\ \ref{fig:init_belief}, if the agents' beliefs are deterministic (single value per agent), or the update rule does not involve an external signal. 
For example, in the voter model analyzed by Mobilia et al.\ \cite{abrahamsson_opinion_2019,mobilia_does_2003,mobilia_voting_2005}, every agent's opinion converges to the partisan and subsequently remains unchanged. Once consensus is reached, the system enters a steady state, because the interaction between agents involves randomly adopting the belief of a neighbor, and all neighbors hold the same belief.
In Ref.\ \cite{abrahamsson_opinion_2019,mobilia_does_2003,mobilia_voting_2005}, the persuadable agents only change their belief in response to their neighbors, whereas the update rule in Section \ref{subsec:modelintro} responds to both the coin toss and the neighbors.  
When persuadable agents are exposed to two or more sources of contradictory information about $\theta$ (e.g.\ the coin tosses and the partisan), their beliefs do not settle to a steady state, as Fig.\ \ref{fig:meannonpartisan} illustrates. The disruption occurs even if there is only one partisan in a large network with $n \gg 1$ agents.

\subsection{Dwell times and the fraction of partisans in a network}
\label{subsec:dwelltime_and_frac}

In the presence of a partisan, persuadable agents never settle in their beliefs, and the asymptotic learning time defined by Eq.\ \eqref{eq:asymlearncondition} is unsuitable to quantify the long-term behavior of the system.  
Eq.\ \eqref{eq:sysasymlearncondition} holds temporarily, but in the long run we observe turbulent nonconvergence, as observed in Section 4.3 of Ref.\ \cite{low_discerning_2022} for different reasons. 
% Figure \ref{fig:asym_Vs_dwell} illustrate how the system can satisfy the condition for asymptotic learning at $t = 5346$ and $t = 6779$, but the belief of still changed as can be indicated as the number of consecutive step become 0 shortly after. 
To quantify the behavior of the persuadable agents, we define the dwell time $t_{\rm d}$ such that 
\begin{equation} \label{eq:dwelltime}
    \max_{\theta} |x_i(t', \theta) - x_i(t, \theta)| < \delta \max_{\theta} |x_i(t, \theta)|
\end{equation}
is true for $t < t' \leq t + t_{\rm d}$ and $\delta = 0.01$ is a user-selected tolerance.
Eq.\ \eqref{eq:dwelltime} defines blocks of time, where the system dwells in the same state to a good approximation, e.g.\ the system may spend $t_1 \leq t' \leq t_1+t_{\rm d1}$ with $\langle\theta\rangle \approx \theta_0$, then $t_2 \leq t' \leq t_2+t_{\rm d2}$ with $\langle\theta\rangle \approx \theta_{\rm p}$, and so on. The blocks of time do not overlap, nor are they necessarily contiguous; in the above examples, one may have $t_2 > t_1 + t_{\rm d1}$, if the two blocks are separated by an interval of turbulent nonconvergence.
Fig.\ \ref{fig:dwell_log_fit} shows a histogram of the 80,698 dwell times observed in an allies-only network with 99 persuadable agents, one partisan, and $T = 10^6$. For this network, dwelling rarely lasts for more than 50 time steps. 
The dwell time PDF $p(t_{\rm d})$ is exponential to a good approximation, with $p(t_{\rm d}) \propto \exp(-0.16 t_{\rm d})$ and hence $\langle t_{\rm d} \rangle \approx 5.4$\footnote{Note that $ \langle t_{\rm d} \rangle ^{-1}= (5.4)^{-1}$ does not equal 0.16 exactly, as expected for an exponential, because we cannot fit reliably to long dwell intervals ($t_{\rm d} >55$) due to small number statistics per bin.}.
We also find that 90\% of the dwell times satisfy $t_{\rm d} \leq 13$.
% We observe the exponential relation in the dwell time distribution with a coefficient of determination statistic \cite{javed_probability_nodate} $R^2 > 0.9$.  
Similar behavior is observed in networks with no partisans.
% DONE: [say exactly what we see there, what the experiment was, the fact that it's BA network, and also quote the best fit exponential in units - this will take 3-4 sentences - very important comparison]
For example, Fig.\ 5 in Ref.\ \cite{low_vacillating_2022} shows a simulation with $T = 10^6$ coin tosses, with the network generated using the Barab\'{a}si-Albert model with $n = 100$ and attachment parameter $m = 3$ \cite{hagberg_exploring_2008}. 
An exponential trend is also observed with best fit $p(t_{\rm d}) \propto \exp(-0.0069 t_d)$\footnote{
    Strictly speaking, Ref.\ \cite{low_vacillating_2022} distinguishes between two versions of the dwell time, termed turbulent ($t_{\rm t}$) and stable ($t_{\rm s}$), which are defined by Eq.\ \eqref{eq:dwelltime} with the conditions $t_{\rm d} < 100$ and $t_{\rm d} \geq 100$ respectively. 
    We cannot draw this distinction usefully in this paper, as we find ${\rm max}(t_{\rm d})< 100$ for most realizations of a complete network. 
    Nevertheless, it is interesting that all approaches lead to exponential PDFs for $p(t_{\rm d})$ (this paper), $p(t_{\rm t})$ (Ref.\ \cite{low_vacillating_2022}), and usually $p(t_{\rm s})$ (Ref.\ \cite{low_vacillating_2022}).
}.

% Figure environment removed

Fig.\ \ref{fig:dwell_vs_frac_partisan} shows how the dwell time depends on the fraction of partisans, $f$, with $0.01 \leq f \leq 0.99$ in steps of $0.01$.
When $f$ is low, the mean dwell time $\langle t_{\rm d} \rangle$ is short, and the persuadable agents frequently change their beliefs.  
As $f$ increases, $\langle t_{\rm d} \rangle$ and $\max(t_{\rm d})$ increase. 
More partisans produce a stronger internal signal between the persuadable agents and partisans, strengthening the ``pull'' towards $\theta_{\rm p}$ and away from $\theta_0$. 
For $f \gtrsim 0.6$, $\langle t_{\rm d} \rangle$ and $\max(t_{\rm d})$ rise sharply (note the logarithmic scale), while $\min(t_{\rm d}) = 1$ remains unchanged. 
Short dwell times occur early in the simulations, when the persuadable agents start to gather information about the coin sequence and partisans and are still influenced strongly by their initial priors. 
Long dwell times occur later in the simulations; for example, one obtains $\max(t_{\rm d}) \approx T = 10^4$, when the persuadable agents maintain their beliefs until the end of the simulation, once the initial transient dies out.
The flatter trend in $\langle t_{\rm d} \rangle$ for $f \gtrsim 0.8$ is caused by the ceiling $t_{\rm d} \leq T$.


The histogram in Fig.\ \ref{fig:dwell_log_fit} contains 80,698 dwell times covering a total of 435,080 out of $T=10^6$ time steps. Hence, for $f=10^{-2}$, the system spends 43.5\% of its time dwelling near $\theta_0$ or $\theta_{\rm p}$ and the rest of its time fluctuating in a state of turbulent nonconvergence similar to the one identified in Ref.\ \cite{low_vacillating_2022} (in the latter paper, the turbulence is caused by different dynamics unrelated to the presence of partisans). 
The fraction of time spent in turbulent nonconvergence depends on $f$. 
For example, at $f=0.5$, one finds 208 dwell time intervals covering a total of 2997 out of $T=10^4$ time steps, and the system spends 70\% of its time in a state of turbulent nonconvergence, whereas at $f=0.9$, one finds typically four dwell time intervals covering a total of 9952 out of $T=10^4$ time steps, and the system spends 0.4\% of its time in a state of turbulent nonconvergence. 
%  DONE: [finish by adding a sentence explaining these results physically if possible]
When $f$ is high, the ``pull'' from $fn$ partisans ${\rm p}_1,\dots,{\rm p}_{fn}$ on an arbitrary persuadable agent $i$ scales as $A_{i{\rm p}_1}+\dots+A_{i{\rm p}_{fn}}$ and is relatively large, which prevents the persuadable agents from changing their mind. 
Hence, we only observe turbulent nonconvergence early in the simulation, before persuadable agents dwell for a long time at $\theta_{\rm p}$. 

\subsection{Switching between beliefs}
\label{subsec:switching_between_belief}
% The difference belief for dwell time
What does $x_i(t,\theta)$ look like, when an agent dwells near some belief, and how does switching between beliefs correlate with $t_{\rm d}$ statistically? 
The picture is complicated, because $x_i(t,\theta)$ may peak at $\theta_0$, $\theta_{\rm p}$, or both $\theta_0$ and $\theta_{\rm p}$ at the same time, as shown in Fig.\ \ref{fig:0.6truebias}. Theoretically, to categorize the belief at each time step in a dwell interval, we should define an orthogonal set of belief templates and match $x_i(t,\theta)$ against all of them. 
However, this approach is challenging numerically, when $x_i(t,\theta)$ is continuously valued, and the number of belief templates is large. 
Instead, we assume that each dwell interval is characterized approximately by an average belief, defined by $\langle\theta\rangle$ at the final time step in the interval. We can then compare the belief and dwell statistics efficiently.

% we define the dwell time at the true bias $t_0$ when $\langle \theta \rangle \simeq \theta_0$, and dwell time at the partisan bias $t_{\rm p}$ when $\langle \theta \rangle \simeq \theta_{\rm p}$. 
% For any dwell time other than $t_0$ and $t_{\rm_p}$ the persuadable agents holds dueling belief of $\theta_0$ and $\theta_{\rm p}$,we denote this dwell time as $t_{\rm other}$.

% Figure environment removed


Fig.\ \ref{fig:dt_dti} displays histograms counting the number of dwell time steps (blue bars) and dwell intervals (orange bars) as functions of the mean belief $\langle\theta\rangle$ during the time steps and intervals respectively. The aim is to understand what beliefs are enduring or transitory at low, moderate, and high $f$. 
At low $f=0.1$ (Fig.\ \ref{fig:dt_dti_0.1}), the persuadable agents spend $\sim 10$ times longer at $\theta_0$ than at $\theta_{\rm p}$ (blue bars) but switch between $\theta_0$ and $\theta_{\rm p}$ frequently, as shown by the short dwell time $\langle t_{\rm d} \rangle = 2.65$ in Fig.\ \ref{fig:dwell_vs_frac_partisan}. 
The agents also dwell sometimes at $\langle\theta\rangle \neq \theta_0, \theta_{\rm p}$ but only rarely ($0.2\%$ of the time) and predominantly early in the simulation. 
At moderate $f=0.6$ (Fig.\ \ref{fig:dt_dti_0.6}), near the sharp low-to-high-$f$ transition observed in Fig.\ \ref{fig:dwell_vs_frac_partisan}, the persuadable agents dwell at $\langle\theta\rangle \approx 0.50$ and $0.55$, corresponding to the belief PDF peaking simultaneously at both $\theta_0$ and $\theta_{\rm p}$ with $x_i(t,\theta_{\rm p}) \gtrsim x_i(t,\theta_0)$ i.e.\ the persuadable agents are more confident in $\theta_{\rm p}$ while not completely discounting $\theta_0$. 
As $f$ increases, the persuadable agents grow their confidence in $\theta_{\rm p}$. 
At $f = 0.6$, 97\% of the dwell intervals have $x_i(t,\theta_{\rm p}) = 1$, while only 4\% have this property for $f = 0.1$. 
At high $f=0.9$ (Fig.\ \ref{fig:dt_dti_0.9}), agents never dwell at $\theta_0$ despite observing the coin continuously.  
The internal interaction from the partisans produces $x_i(t,\theta) = \delta(\theta-\theta_{\rm p})$ for all persuadable agents, i.e.\ the belief is zeroed out everywhere except at $\theta_{\rm p}$.  
This happens because Bayes's rule in Eqs.\ \eqref{eq:updatefirsthalf} and \eqref{eq:likelihood} is multiplicative.
Once the $i$-th persuadable agent achieves $x_i(t,\theta_0)=0$ at some $t$, they stop feeling the  ``pull'' from the coin at every subsequent $t' > t$, resulting in $x_i(t',\theta_0) = 0$ for all $t' > t$ and hence a long  dwell time $t_{\rm d} \lesssim T = 10^4$ at $\theta_{\rm p}$.  
As partisans become the supermajority ($f \gtrsim 0.8$) in the network, the persuadable agents are misled by the partisans and reject the true coin bias $\theta_0$ in favor of $\theta_{\rm p}$.


\subsection{Dueling partisans}
\label{subsec:partisandifferentopinion}


Now suppose that the network contains two groups of partisans ``p1'' and ``p2'' who disagree, with $\theta_{\rm p1}=0.3$, $\theta_{\rm p2}=0.9$, and $\theta_0 = 0.6$ (illustrative values chosen arbitrarily). We say that the partisans are ``dueling'' to convey that they disagree and compete in their influence, even though they may disagree accidentally or cordially rather than deliberately or aggressively.
The results resemble those described in Section \ref{subsec:samethetap}, except that the belief PDF becomes trimodal, with $x_i(t, \theta) \neq 0$ at $\theta = \theta_0, \theta_{\rm p1}, $ and $\theta_{\rm p2}$. 
Fig.\ \ref{fig:dueling_belief_turb} shows the time evolution of $x_i(t, \theta)$ at $\theta = \theta_0, \theta_{\rm p1}, $ and $\theta_{\rm p2} $ for the representative interval $ 4\times 10^3 \leq t \leq 4.5\times 10^3$ and the $i$-th arbitrary agent.
The red line shows $x_i(t,\theta_0)+x_i(t,\theta_{\rm p1}) + x_i(t,\theta_{\rm p2})$, which constantly equals one, indicating zero belief in $\theta \neq \theta_0, \theta_{\rm p1}, \theta_{\rm p2}$. 
The $i$-th agent favors $\theta_0$ without discounting $\theta_{\rm p1}$ and $\theta_{\rm p2}$ completely. During the $5\times 10^2$ plotted time steps, the orange curve for $x_i(t,\theta_0)$ never settles down, dropping eight times to $x_i(t,\theta_0) < 0.8$. It also drops twice to $x_i(t,\theta_0) < 0.6$, once each in favor of $\theta_{\rm p1}$ and $\theta_{\rm p2}$. 

Dueling partisans destabilize the beliefs of the persuadable agents more than a single partisan. Fig.\ \ref{fig:dueling_dt_f0.02} shows the dwell time histogram of two networks, both with $f = 0.02$,  where one network contains two partisans who agree (blue curve), and the other contains two partisans who disagree (orange curve).
The shorter times marked by the orange curve indicate that the persuadable agents change their belief more frequently in the company of dueling partisans.  
This makes sense intuitively; the persuadable agents are ``pulled'' in three directions rather than two. 
For example, when the persuadable agents dwell near $\theta_{\rm p1}$, where the internal signal between partisan p1 and the persuadable agents is negligible, the internal signal between the partisan p2 and the persuadable agents remains substantial.

What happens when uneven numbers of partisans disagree? Let $n_{\rm p1}$ and $n_{\rm p2}$ be the numbers of partisans with belief PDFs $\delta(\theta-\theta_{\rm p1})$ and $\delta(\theta-\theta_{\rm p2})$ respectively. 
We consider three situations, all with $f=0.4$ for the sake of illustration: 
(i) $n_{\rm p1}= 20 = n_{\rm p2}$, (ii) $n_{\rm p1} = 30, n_{\rm p2} = 10$, and (iii)  $n_{\rm p1} = 39, n_{\rm p2} = 1$. 
Fig.\ \ref{fig:uneven_belief} displays histograms counting the number of dwell intervals for cases (i), (ii), and (iii), presented as functions of $\langle\theta\rangle$ during the dwell interval like in Fig.\ \ref{fig:dt_dti}.
When the number of disagreeing partisans is equal, as in case (i), we observe roughly equal numbers of dwell intervals centered on $\theta_{\rm p1}$ (1774 times) and $\theta_{\rm p2}$ (1690 times) (blue bars in Fig.\ \ref{fig:uneven_belief}). 
In case (ii), we observe 2796 dwell intervals centered on $\theta_{\rm p1}$ and 1162 dwell intervals centered on $\theta_{\rm p2}$ (orange bars in Fig.\ \ref{fig:uneven_belief}), while in case (iii), we observe 2547 dwell intervals centered on $\theta_{\rm p1}$ and only 170 dwell intervals centered on $\theta_{\rm p2}$ (green bars in Fig.\ \ref{fig:uneven_belief}). 
The persuadable agents dwell more frequently at the belief held by most of the partisans. 

The number of dwell intervals at each belief is roughly proportional to the partisan population. For $ n_{\rm p1} / (n_{\rm p1}+n_{\rm p2}) = 0.75$, we find that 70\% of the dwell intervals happen at $\langle \theta \rangle \approx \theta_{\rm p1}$; for $ n_{\rm p1} / (n_{\rm p1}+n_{\rm p2}) = 0.98$, we find that 94\% of the dwell intervals happen at $\langle \theta \rangle \approx \theta_{\rm p1}$.  
Note that $\langle \theta \rangle$ does not exactly equal $\theta_{\rm p1}$ or $\theta_{\rm p2}$, because the belief PDF is bimodal, peaking at both $\theta_{\rm p1}$ and $\theta_{\rm p2}$.
The persuadable agents no longer believe in $\theta_0$ for the reason described at the end of Section \ref{subsec:switching_between_belief}. 

Fig.\ \ref{fig:uneven_intervals} shows the histograms of the number of dwell time steps as a function of the mean belief $\langle \theta \rangle$ for cases (i), (ii), and (iii), as well as a control network containing agreeing partisans for comparison.
We observe longer dwell times for $n_{\rm p1} \gg n_{\rm p2}$ (Fig.\ \ref{fig:uneven_intervals}) compared with $n_{\rm p1} = n_{\rm p2}$ or $n_{\rm p1} \gtrsim n_{\rm p2}$, due to the stronger ``pull'' from the dominant group, but $\langle t_{\rm d} \rangle$ is still shorter by a factor of $\approx 3$ than for networks in which all the partisans agree (red curve in Fig.\ \ref{fig:uneven_intervals}). 
Interestingly, even one disagreeing partisan is enough to shorten $\langle t_{\rm d} \rangle$ by a factor $\approx 3$; compare the green and red curves in Fig.\ \ref{fig:uneven_intervals}, for example. In contrast, there is not much difference between $n_{\rm p1}/n_{\rm p2} =39$ and $n_{\rm p1}/n_{\rm p2} = 1$, which yield $\langle t_{\rm d} \rangle = 3.4$ and $2.1$ respectively; see the green and blue curves in Fig.\ \ref{fig:uneven_intervals}. 

% Figure environment removed


\newpage


\section{Opponents only}
\label{sec:opponentonly}
We now change our focus to opponents-only networks to investigate the impact of partisans under antagonistic interactions. 
Section \ref{subsec:asym_learning} examines a network with only one partisan to build intuition through a baseline system. 
We find that a partisan does not prevent the system from achieving equilibrium; unlike in an allies-only network, opposing partisans do not ``pull'' the persuadable agents towards a unique belief $\theta_{\rm p} \neq \theta_0$, because everybody is in mutual opposition.  
The results concerning asymptotic learning resemble those without partisans in Ref.\ \cite{low_discerning_2022}.
Section \ref{subsec:wrong_conclusion_first} examines if the wrong conclusion is reached first in this simulation, as observed in Ref.\ \cite{low_discerning_2022}, for various network structures. 
The special case $\theta_{\rm p} = \theta_0$ is investigated, where indeed asymptotic learning does occur more slowly at $\theta_0$ than at other (wrong) values of $\theta$, just as in Ref.\ \cite{low_discerning_2022}. 
More generally, however, for $\theta_{\rm p} \neq \theta_0$, the connectivity of the network controls whether or not the wrong conclusion is reached first, unlike the system without partisans in Ref.\ \cite{low_discerning_2022}.
The asymptotic learning time $t_{\rm a}$ is computed as a function of the partisans fraction $f$ in Section \ref{subsec:opponent_frac_partisan}. 
% update this para with an extra sentence for sec 4.1/4.2/4.3 new structure - see yellow sticky notes on next page

\subsection{Asymptotic learning}
\label{subsec:asym_learning}
Opponents-only networks with and without partisans exhibit similar long-term outcomes: individual persuadable agents, and the system as a whole, achieve asymptotic learning as defined by Eqs.\ \eqref{eq:asymlearncondition} and \eqref{eq:sysasymlearncondition}, and as observed in Ref.\ \cite{low_discerning_2022}.

Consider a simulation like the one conducted in Section \ref{subsec:samethetap}, but with $A_{ij} = -1$ for all $i\neq j$.
We no longer observe turbulent nonconvergence as in Fig.\ \ref{fig:meannonpartisan}.
Fig.\ \ref{fig:mean_1_o} shows how $\langle\theta\rangle$ evolves in a particular simulation for $0 \leq t \leq 500$. 
Only the first 500 timesteps are shown in Fig.\ \ref{fig:mean_1_o} for brevity, but the simulation runs for $T=10^5$, and $\langle \theta \rangle$ tends to a constant for $t >221$ for every agent. 
That is, the system achieves asymptotic learning at $t_{\rm a} = 221$.


% Figure environment removed


The persuadable agents do not reach consensus, in contrast to Section \ref{subsec:samethetap}, because antagonistic interactions drive their beliefs apart.
% insert 3-4 sentences here saying exactly which beliefs they tend to (0.40-0.75 in steps of 0.05 sort of thing) and refer specifically to visual features of fig 6a - also introduce fig 6b but just focus on the number of persuadables at each theta, describe shape of distribution, and just say at end "The dependence of the histogram on $| \theta_0 - \theta_{\rm p}| $ is studied in Section 4.2."
The beliefs of persuadable agents cluster around the true bias $\theta_0$, as shown in Fig.\ \ref{fig:mean_1_o}, specifically in the range $0.40 \leq \theta \leq 0.75$ in steps of 0.05 due to the discretization discussed in Section \ref{subsec:automaton}. 
This behaviour is also observed in a Barab\'{a}si-Albert network, as shown in Fig.\ 10a in Ref.\ \cite{low_discerning_2022}. 
Fig.\ \ref{fig:diff_theta_p_o} presents a histogram of the number of persuadable agents with average belief $\langle\theta\rangle|_{t=t_{\rm a}}$. The distribution has the shape of a bell curve, peaking at $\langle\theta\rangle|_{t=t_{\rm a}} = \theta_0$. 
The dependence of the histogram on $|\theta_0 - \theta_{\rm p}| $ is studied in Section \ref{subsec:wrong_conclusion_first}.
% Although all agents are opponents, subset of agents ends up agreeing.
Multiple persuadable opponents can occupy the same $\langle\theta\rangle$ bin, because the internal peer pressure is zero if two opponents agree with each other according to Eq.\ \eqref{eq:xiprimed}, i.e.\ there is no belief repulsion between opponents who agree, allowing opponents to settle on the same belief. 
Eq.\ \eqref{eq:xiprimed} implies that such ``grudging agreement'' is unstable, when the network contains only two persuadable opponents, because a small disagreement grows with time via \eqref{eq:xiprimed}. 
However, when there are many persuadable opponents who hold adjacent beliefs, the situation stabilizes.  For example, agents with $x_i(t,\theta)=\delta(\theta-0.50)$ and $x_j(t,\theta)=\delta(\theta-0.60)$ exert opposing repulsive peer pressure on, and hence stabilize, multiple agents labeled $k$ with $x_k(t,\theta)=\delta(\theta-0.55)$.
This resembles the scenario, when two opponents begrudgingly agree because they are both forced into the same opinion by other opponents.  

Asymptotic learning in opponents-only networks containing more than one partisan occurs similarly to Fig.\ \ref{fig:mean_1_o}. It is investigated in Section \ref{subsec:opponent_frac_partisan} as part of a study of $t_{\rm a}$ versus $f$.


\subsection{Reaching the wrong conclusion first}
\label{subsec:wrong_conclusion_first}
A key observation in opponents-only Barab\'{a}si-Albert networks without partisans is that agents who reach the wrong conclusion do so before their opponents reach the right conclusion. 
This counterintuitive tendency is quantified in detail in Section 4.2 and 5.1 in Ref.\ \cite{low_discerning_2022}. 
Here we test if the tendency still holds for complete networks with a single partisan. In this section, unlike in the rest of Sections \ref{sec:alliesonly} -- \ref{sec:mixed}, we consider scale-free, Barab\'{a}si-Albert networks as well as complete networks, because it turns out that connectivity plays an important role in reaching the wrong conclusion first.

We start by investigating the special situation where the partisan believes in the true bias, with $\theta_{\rm p} = \theta_0 = 0.6$.
Table \ref{tab:right_and_wrong} compares the statistics of $ t_{\rm a}^{\rm right}$ and $ t_{\rm a}^{\rm wrong}$ for four different systems, where the superscripts ``right'' and ``wrong'' label $t_{\rm a}$ for the right and wrong conclusions respectively. 
When a partisan is present in a complete network with $\theta_{\rm p} = \theta_0$, the wrong conclusion does tend to be reached first, but the difference between $\langle t_{\rm a}^{\rm right} \rangle$ and $\langle t_{\rm a}^{\rm wrong} \rangle$ is smaller than in a Barab\'{a}si-Albert network, as evidenced by columns \ref{col:ab_no_partisan} and \ref{col:complete_06} in Table \ref{tab:right_and_wrong}. 
This is because, at the beginning of the simulation, the persuadable agents feel persistent repulsion from the partisan at $\theta = \theta_0$.
Persistent repulsion also affects how many agents converge on a particular final belief. 
Fig.\ \ref{fig:diff_theta_p_o} presents a histogram of the number of agents as a function of $\langle \theta \rangle|_{t=t_{\rm a}}$; as $x_i(t, \theta)$ peaks narrowly for all $i$ and $t \geq t_{\rm a}$, it is accurate to approximate agents' final beliefs by $\langle \theta\rangle|_{t=t_{\rm a}}$. 
For $\theta_{\rm p} =\theta_0 = 0.6$, i.e.\ the green bars in the histogram, $\langle\theta\rangle|_{t=t_{\rm a}} = 0.6$ is not the modal value, unlike for the control experiment with zero partisans (blue bars). Quantitatively, 11\% of the agents tend to $\langle\theta\rangle|_{t=t_{\rm a}} = 0.6$, compared to 14\% in the control experiment. This is the same trend identified in Ref.\ \cite{low_discerning_2022}, namely reaching the wrong conclusion first, except that $t_{\rm a}^{\rm right} - t_{\rm a}^{\rm wrong}$ is smaller. 

% This repulsion is also observed in the frequency at which agents asymptotically learn at $\theta_0$
% As all final distribution for agents are singularly peaked, we use $\langle \theta \rangle$ to represent the final belief. 
% Fig.\ \ref{fig:diff_theta_p_o} shows the distribution of beliefs for each persuadable agent.
% Most persuadable agents asymptotically learn on the belief close to $\theta_0$, 
% as shown by the blue and orange bar in Fig.\ \ref {fig:diff_theta_p_o}, 
% as the coin tosses generate a binomial distribution, centred at $\theta_0$, while the $\theta$ far away from $\theta_0$ are exponentially suppressed. 
% However, when $\theta_{\rm p} = \theta_0 = 0.6$, persuadable agents asymptotically learn on $\theta_0$ less often, compared to when $\theta_{\rm p} = 0.3$, due to the repulsion from the partisan in the early stage of the simulation. 
% After the early stage of the simulation where agents have greater confidence in their belief, the strong opinion of partisans no longer stands out, the persuadable agents no longer feel a strong repulsion at the true bias, hence reaching asymptotic learning at $\theta_0$ later. 

\begin{table}[h]
    \centering\begin{tabular}{|c||c|c||c|c||c|c||c|c|}
    \hline
     & 
    \multicolumn{2}{c||}{\makecell{\newtag{(a)}{col:ab_no_partisan} \\ No partisan \\ Barab\'{a}si-Albert}} &
    \multicolumn{2}{c||}{\makecell{\newtag{(b)}{col:complete_no_partisan} \\ No partisan \\ Complete}} &
    \multicolumn{2}{c||}{\makecell{\newtag{(c)}{col:complete_03} \\ $\theta_{\rm p} = 0.3 \neq \theta_0$ \\ Complete}} &
    \multicolumn{2}{c|}{\makecell{\newtag{(d)}{col:complete_06} \\ $\theta_{\rm p} = 0.6 = \theta_0$ \\ Complete}} \\
    \hline
    Property of $t_{\rm a}$
    &$t_{\rm a} ^{\rm right}$ & $t_{\rm a}^{\rm wrong}$ 
    & $t_{\rm a} ^{\rm right}$ & $t_{\rm a}^{\rm wrong}$
    & $t_{\rm a} ^{\rm right}$ & $t_{\rm a}^{\rm wrong}$
    & $t_{\rm a} ^{\rm right}$ & $t_{\rm a}^{\rm wrong}$\\
    \hline
    % Mean &784&226&66&64&68&65&72&64 \\ 
    % Standard deviation &563&243&26&30&30&33&34&33 \\
    First quartile &337&60&50&47&50&47&51&45 \\
    Median &715&92&60&57&60&58&63&56 \\
    Third quartile &1093&174&74&73&76&74&82&73 \\
    \hline
    \makecell{Total number of \\ asymptotic learning agents} &26110&33220&14181&85819&13919&85081&10926&88074 \\
    \hline
    \end{tabular}
    \caption{
        Summary statistics of $t_{\rm a}^{\rm right}$ and $t_{\rm a}^{\rm wrong}$, the asymptotic learning times for right and wrong beliefs respectively, accumulated over $10^3$ simulations, for $n=100$, $T=10^5$, and four different network structures.  
        The first, second, and third quartiles are listed for each network.}
    \label{tab:right_and_wrong}
\end{table}

Interestingly, the tendency to reach the wrong conclusion first does not occur in complete networks in the general case $\theta_{\rm p} \neq \theta_0$. 
Columns \ref{col:complete_no_partisan} and \ref{col:complete_03} in Table \ref{tab:right_and_wrong} show that one has $t_{\rm a}^{\rm right} \approx t_{\rm a}^{\rm wrong}$ within statistical fluctuations for complete networks with $\theta_{\rm p}=0.3 \neq \theta_0$ (column \ref{col:complete_03}) and without a partisan (column \ref{col:complete_no_partisan}). 
This differs from the behavior of Barab\'{a}si-Albert networks studied in Ref.\ \cite{low_discerning_2022}. 
The Barab\'{a}si-Albert attachment parameter $m$ is often small in applications ($m = 3$ in Ref.\ \cite{low_discerning_2022}), meaning that each Barab\'{a}si-Albert agent has only a few opponents, while each agent has 99 opponents in the complete network with $n=100$. 
Hence the influence of one particular opponent, and the ``lock-out effect'' identified in Section 4.2 in Ref.\ \cite{low_discerning_2022}, are less significant in a complete network than in a Barab\'{a}si-Albert network when calculating the sum in Eq.\ \eqref{eq:xiprimed}.


We test this effect by considering different values of $m$ for Barab\'{a}si-Albert networks with $n=100$. 
Fig.\ \ref{fig:bba_sweep} shows the difference in mean asymptotic learning time between the right and wrong beliefs for $1\leq m \leq 99$. 
We observe that for networks with fewer connections, the wrong conclusion is reached first.  
Quantitatively, we find $\langle t_{\rm a}^{\rm right} \rangle - \langle t_{\rm a}^{\rm wrong} \rangle \leq 0.01 \langle t_{\rm a}^{\rm wrong} \rangle $ for $m \geq 15$. 
This agrees with the findings of Low \& Melatos \cite{low_discerning_2022} on Barab\'{a}si-Albert networks with $m=3$.  
More generally, as $m$ increases, the connectivity within the network increases, and each agent is connected to more opponents, suppressing the influence of individual agents.
The light blue shading shows the range corresponding to two standard deviations.
The dispersion in $\langle t_{\rm a}^{\rm right} \rangle - \langle t_{\rm a}^{\rm wrong} \rangle$ decreases with $m$. 

% Talk to AM: why?
% The large fluctuation for low $m$ may be caused by the dependence on the initial conditions.
% The agent who is initially more confident in $\theta_0$ is more likely to infer the true bias according to Ref. \cite{low_discerning_2022}. 


\subsection{Fraction of partisans: effect on $t_{\rm a}$}
\label{subsec:opponent_frac_partisan}
We now increase the number of partisans in opponents-only networks and examine the behavior of persuadable agents. 
All networks achieve asymptotic learning. 
Turbulent nonconvergence is not observed, because partisans do not ``pull'' other agents (whom they oppose) towards their own partisan beliefs, even when the partisans agree among themselves.

Fig.\ \ref{fig:ta_vs_f} shows how the asymptotic learning time depends on the fraction of partisans $f$, all for $\theta_{\rm p} = 0.3$, with $0.01 \leq f \leq 0.99$ in steps of 0.01.
The persuadable agents achieve asymptotic learning more slowly as $f$ increases.
The reason is related to the ``lock-out'' mechanism identified in Section 4.2 of Ref.\ \cite{low_discerning_2022}. 
Consider opponents $i$ and $j$, whose beliefs do not overlap at some $\theta$. 
For $x'_i(t+1/2,\theta)=0$ and $x'_j(t+1/2,\theta) \geq 0$ without loss of generality, we have $A_{ij} [ x'_j(t+1/2,\theta) - x'_i(t+1/2,\theta) ] \leq 0$. 
Summing over all such opponents $j$ for fixed $i$, we obtain $x_i(t+1,\theta)=0$ from Eq.\ \eqref{eq:undatesecondhalf} at the value of $\theta$ being considered. 
Therefore agents cannot respond to the full likelihood of the coin, because part of the $\theta$ domain is zeroed out by opponents, i.e. $x_i(t,\theta') = 0$ implies $x'_i(t+1/2,\theta') = 0$ for locked-out $\theta'$ values, because Eq.\ \eqref{eq:updatefirsthalf} is multiplicative. 
For low $f$, there are more persuadable agents forming beliefs at various $\theta$ values, so more of the $\theta$ domain for every agent is locked out by opponents, causing every agent to ``see'' a narrow likelihood when observing the coin tosses.  
On the other hand, when networks contain more partisans (i.e.\ higher $f$), who all agree on $\theta_{\rm p}$, fewer $\theta$ values are occupied by persuadable agents, and all persuadable agents ``see'' more of the likelihood. 
As  $x_i(t,\theta)$ is normalized, agents who see a narrower likelihood achieve asymptotic learning faster than agents who see a wider likelihood. 
Therefore, $t_{\rm a}$ increases with $f$ in an opponents-only network.
% worth adding 1-2 sentences about the min/max range shaded in the figure and what it teaches us (specifics, not vague)
The light blue shading in Fig.\ \ref{fig:ta_vs_f} brackets the ensemble minimum and maximum, and the orange shading brackets the range out to two standard deviations.



% Figure environment removed

% \section{Allies and opponents}
% \label{sec:mixed}

% \section{Balanced and unbalanced network}
% \label{sec:Balanced}

\section{Mixed allegiances}
\label{sec:mixed}
We now turn our focus to more realistic networks containing mixed political allegiances, where some agents interact simultaneously with allies and opponents.
An exhaustive analysis of mixed networks lies outside the scope of this paper. 
A preliminary step towards this challenging problem without partisans was taken in Refs.\ \cite{low_discerning_2022} and \cite{low_vacillating_2022}, mainly (but not exclusively) in the context of Barab\'{a}si-Albert networks. 
Here we focus instead on the special (and simpler) case of complete networks to take advantage of the baseline studies in Sections \ref{sec:alliesonly} and \ref{sec:opponentonly}, trading off some richness in network topology in favor of simplifying and therefore clarifying the novel effects introduced by partisans. 
In Section \ref{subsec:unbalanced_triad}, we consider the dynamics of an unbalanced triad, which is the smallest nontrivial subunit of a mixed network and the driver of much (although not all) of the counterintuitive behavior reported in Refs.\ \cite{low_discerning_2022} and \cite{low_vacillating_2022}. 
In Section \ref{subsec:larger_network}, we consider a representative example with $n=100$, to gain a preliminary sense of how the behavior of larger mixed networks compares with allies- and opponents-only networks. 
The results in Section \ref{subsec:larger_network} point to some productive avenues for future work but are not intended to be exhaustive.




\subsection{Triads}
\label{subsec:unbalanced_triad}

Four unique triads with $n=3$ can be constructed. They are labeled $G_1, \dots, G_4$ in the top row of  Fig.\ \ref{fig:triad}.
$G_1$ and $G_4$ are allies-only and opponents-only networks, studied in Sections \ref{sec:alliesonly} and \ref{sec:opponentonly} respectively.
$G_3$ is nominally a mixed network but it behaves the same as an opponents-only network with $n=2$, when there are no partisans, as is clear visually. 
Interestingly, though, it exhibits counterintuitive behavior, when a partisan is introduced, as discussed below in this section. 
The unbalanced triad $G_2$ leads to important, counterintuitive, new behavior even without partisans, as demonstrated in Ref.\ \cite{low_discerning_2022}. 
It exhibits internal tension: agent 1 is attracted to the beliefs of its allies, agents 2 and 3, but the beliefs of agents 2 and 3 tend to diverge, because agents 2 and 3 are opponents. 
In other words, agent 1 is in the invidious position of striving to agree with two individuals, who are predisposed to disagree. In general, this leads to unsteady dynamics, such as turbulent nonconvergence \cite{low_discerning_2022}. 
In an unbalanced triad, the impact of a partisan depends sensitively on where the partisan is inserted; they may assume the role of agent 1 or agent 2 (equivalent to agent 3), as depicted in the bottom row of Fig.\ \ref{fig:triad}. 
We consider both scenarios in this section.
% We now focus on $G_2$ and $G_3$ with mixtures of allies and opponents.
% $G_2$ exhibits internal tension, as agent 1 is allied with two agents that oppose each other, while $G_3$ lacks internal tension, with allied agents 2 and 3 commonly opposing agent 1.  

% Figure environment removed

Let us begin with $G_{2{\rm p}1}$, which is the version of $G_2$ where the partisan (agent 1) is allied with persuadable opponents (Fig.\ \ref{fig:triad}, middle row, left graph). 
Fig.\ \ref{fig:G_2_prior} displays snapshots of $x_i(t=5 \times 10^3, \theta)$ for all three agents, while Fig.\ \ref{fig:G_2_mean} displays $\langle\theta\rangle$ versus time ($0\leq t \leq 1 \times 10^4$) for all three agents.
We observe turbulent nonconvergence for both persuadable agents, whereas only one agent (agent 1) experiences turbulent nonconvergence without a partisan in Ref.\ \cite{low_discerning_2022}. 
Agents 2 and 3 are both pulled by their alliances towards the partisan, while still observing the coin tosses. 
However, they strive to disagree with each other, so the peaks away from $\theta_{\rm p}$ in their bimodal belief PDFs occur at unequal values of $\theta$.
Dwelling still occurs, including for long intervals, e.g.\ $7374 \leq t \leq 9071$, which starts following a run of tails during $7369 \leq t \leq 7374$. 
Fig.\ \ref{fig:G_2} is a reminder that a hypothetical external observer should be cautious about inferring the truth of a specific belief by extrapolating from its popularity. In Fig.\ \ref{fig:G_2_prior}, for example, every agent believes in $\theta_{\rm p}$ to a greater or lesser extent, whereas some agents do not believe in $\theta_0$ at all, yet the partial consensus about $\theta_{\rm p} \neq \theta_0$ is misleading as a guide to the true bias.


% Figure environment removed

We now turn to $G_{2{\rm p}2}$, the version of $G_2$ where agent 1 is allied with the partisan (agent 2) and agent 3, who opposes the partisan (Fig.\ \ref{fig:triad}, middle row, right graph).
Note that switching agent 1 with agent 3, or selecting agent 3 as the partisan, leads to a network with the same connections and topology.
% The network when agent 3 is the partisan is identical to the network with agent 2 is the partisan, as they have the same connections and relations to both the partisan and the persuadable agents.  
Figs.\ \ref{fig:G_2_prior_fe} and \ref{fig:G_2_mean_fe} plot the same quantities as Figs.\ \ref{fig:G_2_prior} and \ref{fig:G_2_mean} respectively. 
They agree with the results in Ref.\ \cite{low_discerning_2022}, where agent 2 and 3 achieve asymptotic learning, %at $t_{\rm a} = 2355$, 
and agent 1 experiences turbulent nonconvergence, vacillating between the beliefs of agent 2 and 3.
% That is, selecting agent 2 as the partisan does not vary the behavior observed.
However, agent 3 reaches the right belief upon achieving asymptotic learning, unlike in Ref.\ \cite{low_discerning_2022}, because the partisan does not ``zero out'' agent 3's likelihood at $\theta_0$.

% Figure environment removed


Let us now consider $G_{3{\rm p}2}$, the version of $G_3$ where agent 2 (the partisan) opposes agent 1 and allies with agent 3 (Fig.\ \ref{fig:triad}, bottom row, right graph)\footnote{Network $G_{3{\rm p}1}$ (Fig.\ \ref{fig:triad}, bottom row, left graph), in which agent 1 (the partisan) is opposed to agents 2 and 3, who form a persuadable yet allied bloc, exhibits the same dynamics essentially as an opponents-only network with $n=2$ and one partisan. Both persuadable agents achieve asymptotic learning at $\theta_0$; see Section \ref{sec:opponentonly}.}. 
The results appear in Fig.\ \ref{fig:G_3} in the same format as Fig.\ \ref{fig:G_2}.
Agent 1, opposing both the partisan and agent 3, does not experience peer pressure as the beliefs do not overlap, as discussed in Section \ref{subsec:opponent_frac_partisan}.
Hence, agent 1 only responds to the coin tosses and achieves asymptotic learning at $t_{\rm a1} = 1119$. 
Interestingly, in Fig.\ \ref{fig:G_3_mean}, agent 3 agrees quickly with the partisan (with agreement reached by $t = 7$) and maintains that belief until $t=6100$, when a long interval of turbulent nonconvergence ensues, triggered by a sequence of six heads in a row. 
This is an example of the intermittency phenomenon observed without partisans in Ref.\ \cite{low_vacillating_2022}. 
Agent 3 realizes, via Eqs.\ \eqref{eq:updatefirsthalf} and \eqref{eq:likelihood}, that six heads in a row are unlikely to be consistent with $x_3(t=6100, \theta) \approx \delta(\theta-0.3)$, and hence starts to gain confidence in higher values of $\theta$, without discounting $\theta_{\rm p}=0.3$ completely.
However, agent 1 locks agent 3 out of $\theta = 0.6$, as discussed in Section 4.2 of Ref.\ \cite{low_discerning_2022}.
Hence agent 3 is driven towards the midpoint $\approx (\theta_{\rm p} + \theta_0)/2$, so that $x_2(t,\theta)$ becomes bimodal for $t > 6100$.
We remind the reader that turbulent nonconvergence does not happen in the $G_3$ triad without a partisan, as demonstrated in Ref.\ \cite{low_discerning_2022}. 

The intermittent behavior of agent 3 in Fig.\ \ref{fig:G_3} is only one possible behavior in the $G_{3{\rm p}2}$ triad. 
Agent 3 sometimes exhibits turbulent nonconvergence from the beginning of the simulation, for example, maintaining a bimodal belief PDF with one peak at $\theta_{\rm p}$ and one at $\theta \neq \theta_0$ disagreeing with agent 1. 
The statistics of the alternative forms of intermittent behavior are complicated, as shown in Ref.\ \cite{low_vacillating_2022} even without partisans, and will be studied fully in future work.



\subsection{Larger networks}
\label{subsec:larger_network}

% Figure environment removed

We now consider one representative example of a larger complete network with mixed allegiances, $n =100$, and one partisan. 
The sign of $A_{ij}$ is selected randomly with equal probability for all $i$ and $j$ and yields 2489 edges joining allies and 2461 edges joining opponents in the illustrative example analyzed here. 
Fig.\ \ref{fig:big_mixed_final} shows the belief PDF at $t = 10^4$ for each agent. 
Every PDF features a portion with $x_i(t,\theta) \neq 0$ for $| \theta - \theta_0 | \lesssim 0.1$. The PDFs are unequal for different agents and satisfy $x_i(t,\theta) \neq 0$ for multiple values of $\theta$ for some (but not all) agents. 
In addition, allies of the partisan develop a second peak at $\theta_{\rm p}$ (48 agents in this particular simulation). 
 
Fig.\ \ref{fig:big_mixed_mean} displays how $\langle\theta\rangle$ evolves for four selected agents, each displaying different but typical behavior. 
Agent 14 is allied to the partisan and exhibits turbulent nonconvergence as described in Section \ref{subsec:samethetap}. 
Agents 15, 16, and 30 oppose the partisan, satisfy $x_i(t,\theta_{\rm p})=0$ for $i=$ 15, 16, and 30, and exhibit intermittency as described in Ref.\ \cite{low_vacillating_2022}. 
Agent 15 always has a bimodal belief PDF, with the peaks moving in the range $0.5 \leq \theta \leq 0.7$ but never reaching $\theta_{\rm p} = 0.3$.
Agent 30 dwells for a long time (4238 time steps) at $\langle \theta\rangle \approx 0.6$, with PDF $\approx \delta(\theta - \theta_0)$, then suddenly transitions to turbulent nonconvergence at $t = 4239$, like agent 3 in Fig.\ \ref{fig:G_3_mean}. 
On the other hand, Agent 16 enters the long dwell interval $4165 \leq t \leq T$ with $x_{16}(t,\theta) \approx \delta(\theta-0.55)$.  
Given a longer simulation, it is possible that agent 16 will transition to turbulent nonconvergence at $t > 1\times 10^4$, like agent 30 and agent 3 in Fig.\ \ref{fig:G_3_mean}. 
A similar mixed network with $n=1000$ was also tested and returned similar results (not plotted for brevity). 


% One could characterize the above behaviors and examine each agent by drawing a more general link between the overall connectivity of each agent, as well as the relationship with the partisan.  
In future work, we will study large networks with mixed allegiances in systematic detail, with the aim of linking an agent's behavior to their connectivity in general and their relationship with the partisan in particular (which may be null on occasion, e.g.\ in a Barab\'{a}si-Albert network).






\section{Partly connected networks}
\label{sec:BA}
It is natural to ask whether the behavior observed in Sections \ref{sec:alliesonly} -- \ref{sec:mixed} is specific to complete networks. To what extent do the results depend on the network's connectivity?
In Section \ref{subsec:wrong_conclusion_first}, for example, we find that the tendency to reach the wrong conclusion first depends on the attachment parameter in Barab\'{a}si-Albert opponents-only networks.
Moreover, it is plausible intuitively that the influence of a partisan increases, as their connections to the rest of the network increase, and that their influence reaches a maximum in a complete network, where they are connected to every other agent.

In this section, we take a first pass at generalizing the results in Sections \ref{sec:alliesonly} -- \ref{sec:mixed} to partly connected networks, as foreshadowed in Section \ref{subsec:network}. We adopt Barab\'{a}si-Albert networks as a traditional test case, motivated by previous theoretical studies \cite{low_discerning_2022,low_vacillating_2022}, and the conditions in many real social settings \cite{barabasi_emergence_1999,tang_survey_2016,kumar_structure_2016,maniu_building_2011}, and defer the study of other network topologies to future work.
Exploring the behavior for all possible values of the attachment parameter $m$ is outside the scope of this paper. Instead, we focus on networks with small $m$, i.e.\ networks that are sparsely connected, to accentuate the distinction with the complete networks studied in Sections \ref{sec:alliesonly} -- \ref{sec:mixed}.
We leave the exploration of dueling partisans within Barab\'{a}si-Albert networks to a later paper.
In Section \ref{subsec:distance}, we test how the beliefs of persuadable agents in allies-only networks depend on the minimum distance to the partisans, and investigate the implications for achieving consensus. 
In Section \ref{subsec:frac_ba}, we test how $m$ affects the trend of dwell time versus partisan fraction. 
We explore briefly the behavior of mixed allegiances in Section \ref{subsec:mixed_ba}, and the optimal placement of partisans to achieve a sociopolitical goal in Section \ref{subsec:Optimizations}. The latter two topics are subtle and multifarious and will be investigated fully in future work.

\subsection{Distance to the partisans: dissolving consensus on the path to turbulent nonconvergence}
\label{subsec:distance}

In a complete, allies-only network, even one partisan is enough to prevent asymptotic learning; the global outcome is turbulent nonconvergence, as described in Section \ref{subsec:samethetap}. However, the complete connections between persuadable agents ensure, that the persuadable agents reach a consensus promptly, which is maintained even while the persuadable agents vacillate turbulently between believing in $\theta_0$ and $\theta_{\rm p}$. In a partly connected, allies-only network, it is still true that even one partisan is enough to prevent asymptotic learning; that is, the global outcome remains unchanged.
However, the persuadable agents never reach a consensus; they follow different paths while enacting turbulent nonconvergence, because some are adjacent to the partisan and others are not. We demonstrate this behavior for smaller ($n=3$) and larger ($n=100$) Barab\`{a}si-Albert networks in this section. 

Consider first an allies-only network with $n=3$, illustrated in Fig.\ \ref{fig:allies_link}.
Agent 1, shaded grey, is a partisan. 
Agents 2 and 3 are persuadable. 
Agent 2 is adjacent to the partisan. 
Agent 3 is not but it is connected indirectly to the partisan via agent 2, i.e.\ one step removed.
We run a simulation for randomized priors and coin tosses and $T = 10^4$, which is analogous to the simulation in Fig.\ \ref{fig:meanconverge} but with $n=3$ instead of $n=100$.
Fig.\ \ref{fig:allies_link_distr} shows a snapshot of the belief PDF at $t = 5 \times 10^3$.  
% Figure environment removed
We observe that the PDF is bimodal, as in Fig.\ \ref{fig:0.6truebias}, and that agent 2 is more confident in $\theta_{\rm p}$ than agent 3, because agent 2 is adjacent to the partisan, and agent 3 is not.
Further to the same point, Fig.\ \ref{fig:allies_link_partisan_belief} shows the evolution of $x_i(t, \theta = \theta_{\rm p})$ in the interval $5.0\times 10^3 \leq t \leq 5.5\times 10^3$. 
We find $ x_{2}(t, \theta = \theta_{\rm p}) > x_{3}(t, \theta = \theta_{\rm p})$ throughout the interval, i.e.\ persuadable agents who are closer to the partisan are more confident in $\theta_{\rm p}$ and less confident in $\theta_0$.
In a complete network, e.g.\ in Section \ref{subsec:samethetap}, we find $x_2(t,\theta_{\rm p}) = x_3(t,\theta_{\rm p})$ throughout the interval instead, i.e.\ consensus.

% Figure environment removed

Let us now repeat the test in the previous paragraph for a larger Barab\`{a}si-Albert network with $n=100$ and $m=3$, as depicted in Fig.\ \ref{fig:BA_allies_m3}, with agent 1 being the partisan. 
Fig.\ \ref{fig:nonconsensus} displays the evolution of the mean belief $\langle\theta\rangle$ of every agent in one particular simulation for $5 \times 10^3 \leq t \leq 5.1 \times 10^3$. 
Unlike in Fig.\ \ref{fig:meanconvergent}, where the mean beliefs of persuadable agents converge mutually within $t<15$ time steps, 
the mean beliefs of persuadable agents in Fig.\ \ref{fig:nonconsensus} never converge mutually. 
That is, at any instant $t$ in the plotted range, $\langle \theta \rangle$ ranges typically from 0.35 to 0.6 for the 99 persuadable agents. 
Overall, however, $\langle \theta \rangle$ for every agent follows a similar trajectory as a function of $t$, because all persuadable agents observe the same sequence of coin tosses.
\footnote{One can quantify the degree of consensus by calculating the Kullback-Leibler divergence between agents, a topic for future work. }
This behavior differs from allies-only complete networks with partisans, discussed in Section \ref{subsec:samethetap}, and allies-only Barab\'{a}si-Albert networks without partisans, discussed in Appendix B in Ref.\ \cite{low_discerning_2022}.
Persuadable agents disagree for two reasons: 
\begin{inparaenum}[(i)]
\item they feel the influence of the partisans differently, because some are adjacent to partisans and others are not; and
\item they feel the influence of other persuadable agents differently because they are connected to different numbers of persuadable agents.
\end{inparaenum}

In order to quantify further the cause of the breakdown in consensus, we define $d_{\rm p}$ to be the length of the shortest path from a selected persuadable agent to any partisan.
We use breadth first search (BFS) \cite{moore_shortest_1959} to find the shortest path, noting that BFS only works in graphs with positive edge weights (here, allies-only networks).
Let us also define $t_{\theta_0, i}$ to be the number of total (and not necessarily consecutive) timesteps the $i$-th agent satisfies $x_i(t, \theta = \theta_0) \geq 0.9$, i.e.\ the number of timesteps when the agent is very confident in $\theta_0$,  with the threshold 0.9 having been chosen arbitrarily.  
We run an ensemble of 100 simulations with randomized priors and coin tosses on a Barab\'{a}si-Albert allies-only network with $n =100$, $m =3 $ with one partisan for $T = 10^4$.

Fig.\ \ref{fig:allies_violin_samegraph} shows violin plots of $t_{\theta_0, i}$ for each agent, as a function of $d_{\rm p}$. 
The horizontal width is a smoothed version of the histogram that shows the number of persuadable agents with a certain $t_{\theta_0, i}$, which is reflected around the vertical axis to create the shape of the violin.
The white dot indicates the mean and the thick, vertical, black bar represents the interquartile range. 
The values of $d_{\rm p}$ in this particular network are 0, 1, 2, 3, where $d_{\rm p} = 0$ refers to the single partisan; i.e.\ there are no persuadable agents with $d_{\rm p} > 3$ in this network with $n=100$.
We find that $\max(d_{\rm p})$ is related to the choice of $m$; $\max(d_{\rm p})$ is higher in more sparsely connected networks, i.e.\ smaller $m$. 
Persuadable agents with larger $d_{\rm p}$ are more confident in $\theta_0$, as shown by the fact that the white dots and thick black bars in Fig.\ \ref{fig:allies_violin_samegraph} trend higher, as $d_{\rm p}$ increases.
We also find $\max(t_{\theta_0,i}) - \min(t_{\theta_0,i}) = 5890, 5766, 3475$ for $d_{\rm p} = 1,2,3$ respectively. 
Agents with the same $d_{\rm p}$ can still have different beliefs due to their different connectivity to other persuadable agents.
In summary, the dissolving consensus among persuadable agents in Barab\'{a}si-Albert networks is attributed to the difference in connectivity and $d_{\rm p}$ values, where $t_{\theta_0, i}$ is related to $d_{\rm p}$.

\subsection{Dwell time versus $f$ and $d_{\rm p}$}
\label{subsec:frac_ba}

In the context of turbulent nonconvergence, the persistence of the beliefs of a persuadable agent is captured via the dwell time $t_{\rm d}$ defined by Eq.\ \ref{eq:dwelltime} rather than the asymptotic learning time defined by Eq.\ \ref{eq:asymlearncondition}. In Fig.\ \ref{fig:dwell_vs_frac_partisan}, we find that $\langle t_{\rm d} \rangle$ increases with the partisan fraction $f$ in a complete network. Here, we check how the trend in  Fig.\ \ref{fig:dwell_vs_frac_partisan} changes as a function of the attachment parameter, when the network is partly connected.

We consider two Barab\`{a}si-Albert networks with $n=100$, a sparse one with $m = 3$, and one of medium density with $m=20$.  
The networks with $m=3$ and $m=20$ complement the complete networks with $m = n-1 = 99$ studied in Section \ref{subsec:dwelltime_and_frac}. 
% Why? m gives the degree of each newly added vertex to the network.  Graph-tool will do d = min(n, m) when choosing the degree for the new
% vertex, where n is the current graph size
Fig.\ \ref{fig:frac_p_m3} and Fig.\ \ref{fig:frac_p_m20} display $\langle t_{\rm d} \rangle$ versus $f$ for the $m=3$ and $m=20$ networks respectively. The curves are color-coded according to $d_{\rm p}$, e.g.\ the blue curve corresponds to evaluating $t_{\rm d}$ for the subpopulation of persuadable agents adjacent to a partisan ($d_{\rm p}=1$). The aim is to test how $t_{\rm d}$ depends on the distance to the nearest partisan, and what trade-off exists between $t_{\rm d}$ and $f$.
The cut-offs for $d_{\rm p} = 2,3,4$ occur, because the number of persuadable agents with $d_{\rm p} > 1$ drops to zero, when $f$ exceeds some threshold $f_{\rm max}(d_{\rm p})$; for example, we find $f_{\rm max}(d_{\rm p} = 3) = 0.46$ in Fig.\ \ref{fig:frac_p_m3}.

% Figure environment removed


Fig.\ \ref{fig:frac_BA} leads to four main conclusions. 
First, $\langle t_{\rm d} \rangle$ increases monotonically with $f$ for $d_{\rm p}=1$, with $d\langle t_{\rm d} \rangle / df$ increasing sharply for $f\gtrsim 0.6$, just like in Fig.\ \ref{fig:dwell_vs_frac_partisan}. The trend depends weakly on $m$ (compare the blue curves on the same axes in Fig.\ \ref{fig:frac_p_m3} and Fig.\ \ref{fig:frac_p_m20}). 
Second, $\langle t_{\rm d} \rangle$ decreases with $f$ at low $f \lesssim 0.2$ for $d_{\rm p} > 1$.
This is because the ``pull'' from partisans on persuadable agents with $d_{\rm p} >1 $ is weaker than the ``pull'' from the coin for low $f$, so agents dwell with belief close to $\theta_0$. 
As $f$ increases, while $d_{\rm p}$ is held fixed, the ``pull'' from the partisan increases, causing $t_{\rm d}$ at $\theta_0$ and hence $\langle t_{\rm d} \rangle$ to decrease.
Third, $\langle t_{\rm d} \rangle$ increases with $d_{\rm p}$ for $f \lesssim 0.2$ but decreases with $d_{\rm p}$ for $f \gtrsim 0.2$, as seen in Fig.\ \ref{fig:frac_p_m3}. 
This occurs because at low $f$, the ``pull'' from the partisans is weaker for agents with higher $d_{\rm p}$.
Their beliefs are dominated by the coin tosses, leading to longer $\langle t_{\rm d} \rangle$ at low $f$. 
For $f \gtrsim 0.2$, an increase in partisan population means that the ``pull'' from the partisans outweighs the ``pull'' from the coin tosses, leading to all agents dwelling at $\theta_{\rm p}$ longer.  However, persuadable agents closer to the partisans (lower $d_{\rm p}$) are more strongly influenced by the ``pull'' from the partisans, resulting in $\langle t_{\rm d} \rangle$ decreasing with $d_{\rm p}$. 
Fourth, the trends of $\langle t_{\rm d} \rangle$ versus $f$ for the $d_{\rm p}=1$ and $d_{\rm p} =2$ subpopulations resemble each other more closely for $m=20$ than for $m=3$ (compare Fig.\ \ref{fig:frac_p_m3} and Fig.\ \ref{fig:frac_p_m20}). 
This occurs because the contribution of each individual agent is weaker in a densely connected network, as the interaction between agents is averaged over all neighbors (through Eq.\ \ref{eq:xiprimed}). 
Hence the effect of $d_{\rm p}$ on $\langle t_{\rm d} \rangle$ is weaker on networks with greater $m$.  
One main difference between Fig.\ \ref{fig:dwell_vs_frac_partisan} and Fig.\ \ref{fig:frac_BA} is the maximum dwell time.
In complete networks, one obtains $t_{\rm d} \simeq 10^4$ for $f \gtrsim 0.7$, whereas in Barab\'{a}si-Albert networks with $m=3$ one obtains $t_{\rm d} \simeq 10^4$ for $f \geq 0.06$ and $d_{\rm p} = 1$. 
Agents with $d_{\rm p} = 1$ and $t_{\rm d} \simeq 10^4$ are adjacent to more than one partisan, and non-adjacent partisans also  influence their beliefs.

\subsection{Mixed allegiances}
\label{subsec:mixed_ba}

The question of mixed allegiances is subtle and multi-faceted even in complete networks, as demonstrated in Section \ref{sec:mixed}. 
A full investigation of mixed allegiances in partly connected networks, a more challenging problem, lies well outside the scope of this paper. 
Instead, as a foretaste of what can be done, we present here one representative example: a Barab\'{a}si-Albert network with $n=100$, $m=3$, a single partisan, and $A_{ij} =\pm 1$ with equal probability. The specific realization of this network studied here features 318 edges connecting allies and 270 edges connecting opponents.

When we simulate the above network for $T =10^4$, we observe that $\langle \theta \rangle$ evolves just like in a complete network, in a manner that resembles Fig.\ \ref{fig:big_mixed_mean} (the graph is omitted to avoid repetition). 
Specifically, 13 out of 99 persuadable agents reach asymptotic learning early in the simulation ($t < 10^3$), with asymptotic mean belief in the range of $0.5 \leq \langle\theta\rangle \leq 0.8 $. One persuadable agent correctly and stably infers $\theta_0$. 
The other 86 persuadable agents exhibit turbulent nonconvergence or intermittent behavior. 
In comparison, in the complete network investigated in Section \ref{subsec:larger_network}, 11 out of 99 persuadable agents reach asymptotic learning, and three persuadable agents correctly and stably infer $\theta_0$. The other 88 persuadable agents exhibit turbulent nonconvergence or intermittent behavior. 
In complete networks, we observe that only persuadable agents who are adjacent and allied to one or more partisans (i.e. $A_{i \rm{p}} = 1$) develop a peak at $\theta_{\rm p}$ in their belief PDF (see Section \ref{subsec:larger_network}). 
In contrast, in a Barab\'{a}si-Albert network, some persuadable agents who are connected but not adjacent to a partisan (e.g.\ $A_{ij} \neq 0$, $A_{j{\rm p}} \neq 0$, and $A_{i{\rm p}}=0$ for some $j \neq i$, ${\rm p}$) also develop a peak at $\theta_{\rm p}$.
Specifically, 29 agents who are connected but not adjacent to the partisan have $x(t = T, \theta = \theta_{\rm p}) \geq 0.001$. 
The 29 agents have $2 \leq d_{\rm p} \leq 4$, where $d_{\rm p}$ is defined as the shortest distance irrespective of the sign of the edges, cf. BFS in Section \ref{subsec:distance}. 
In general, the notion of distance depends on the sign of the edges; for example, it may be argued that two agents separated by three positive edges (i.e.\ three alliances) are ``closer'' than two agents separated by three negative edges (i.e.\ three adversarial links). A systematic study of this issue, with a generalized definition of $d_{\rm p}$, is deferred to future work.



\subsection{Manipulating collective opinion: optimizing the placement of partisans}
\label{subsec:Optimizations}

A challenging question with important social implications is how to optimize the placement of obdurate partisans to manipulate the beliefs of persuadable agents in the service of some social or political goal. 
Within the specific context of media bias, for example, one goal could be to camouflage the political bias of a media outlet, by convincing as many consumers as possible that the outlet is neutral ($\langle \theta \rangle \approx 0.5$, say) when in reality it is biased strongly ($\theta_0 \approx 0$ or $1$, say). 
The latter example involves persuading agents to believe something false. In other applications, both within and beyond the specific context of media bias, the goal may be to persuade agents to believe something true, e.g.\ the danger of drink driving, or the efficacy of a medical treatment. 
The obdurate partisans may be real humans or automated systems such as social network ``bots''. 
They may conduct their operations by espousing beliefs that agree with the target belief or, interestingly and counterintuitively, by espousing beliefs that disagree with the target belief, leveraging oppositional relationships in the network to shepherd persuadable agents towards the target belief. 
The long-term impact of partisanship and the optimal placement of obdurate partisans has been modeled previously in a deterministic framework (without belief PDFs) to investigate how to maximize political polarization \cite{yildiz_binary_2013,arendt_opinions_2015} or conformity with the partisans' belief \cite{abrahamsson_opinion_2019,klamser_zealotry_2017,masuda_evolution_2012}. 

In this section, we analyze briefly one specific, representative example of the above problem, as a foretaste of what can be investigated more broadly. The example involves a Barab\'{a}si-Albert network with $n=10$, $m=3$, and one partisan, depicted in Fig.\ \ref{fig:ba_10}. 
The degrees of the 10 vertices range from three to eight. The specific question asked is: where should the partisan be placed, to drive the beliefs of persuadable agents as far from $\theta_0$ as possible, i.e.\ to maximize $| \langle \theta \rangle - \theta_0 |$? 
To answer the question, we run 10 simulations with the same sequence of coin tosses for $T=10^4$. The simulations differ in what vertex the partisan occupies.
Let us define 
\begin{equation}\label{eq:bartheta0}
    \overline{t_{\theta_0}} = \frac{1}{(1-f) n} \sum_{i \neq {\rm p}} t_{\theta_0, i}, 
\end{equation}
where $f$ is the fraction of partisans in a network of size $n$. 
In Fig.\ \ref{fig:opti_scatter}, we observe that $\overline{t_{\theta_0}}$ is inversely related to the degree of the partisan. 
Furthermore, the number of persuadable agents with $t_{\theta_0}  = 0$ increases as the degree of the partisan increases, also as observed in Fig.\ \ref{fig:opti_scatter}. 
% DONE: which has shorter... - unclear what we are talking about here, what does "which refer to"? as written it refers to the degree of the partisan, but a degree can't have a shorter t_theta_0 - t_theta_0 is a property of agents, not degrees of vertices - please fix [sorry I can't suggest wording, I don't know what we mean]
The partisan's degree is equal to the number of persuadable agents with $d_{\rm p} = 1$ (adjacent to the partisan). Persuadable agents with $d_{\rm p} = 1$ have shorter $t_{\theta_0, i}$, shown in Fig. \ref{fig:allies_violin_samegraph}, resulting in a shorter $\overline{t_{\theta_0}}$.
Therefore, if we wish to mislead the persuadable agents, we should place the partisan at the vertex with maximum degree, i.e.\ the most connected vertex. 
This result makes sense intuitively. However, we caution that it is not expected to apply always in networks with mixed allegiances, which contain subnetworks with internal tensions and trade-offs, such as the unbalanced triad $G_2$ in Fig.\ \ref{fig:triad}. 
An interesting avenue for future work is to optimize both the placement and beliefs of (perhaps dueling) partisans to manipulate a numerical majority of agents to hold a target belief $\theta_{\rm t}$. 
This optimization task is related to but different from minimizing the average displacement $| \langle \theta \rangle - \theta_{\rm t} |$, and may be more relevant in electoral applications \cite{bravo-marquez_opinion_2012,druckman_impact_2005}.
% DONE: do we know a good, well-cited reference to opinion dynamics in electoral applications? no worries if not


% Figure environment removed




\section{Discussion and summary}
\label{sec:conclusion}
\section{Conclusion}
\label{sec:conclusion}
In this paper, we present the first comprehensive evaluation of multiple LLMs (Alpaca, Alpaca-LoRA, GPT-3.5) on mental health prediction tasks (binary and multi-class classification) via online text data. 
Our experiments cover zero-shot prompting, few-shot prompting, and instruction finetuning. The results reveal a number of interesting findings.
Our context enhancement strategy can robustly improve performance for all LLMs, and our mental health enhancement strategy can enhance models with large number of trainable parameters.
Meanwhile, few-shot prompting can also robustly improve model performance even by providing just one example per class.
Most importantly, our experiments show that instruction finetuning across multiple datasets can significantly boost model performance on various mental health prediction tasks at the same time. Our best finetuned model Mental-Alpaca performs on par with the state-of-the-art task-specific model Mental-RoBERTa.
We summarize our findings as a set of guidelines for future researchers, developers, and practitioners who want to empower LLMs with better knowledge of mental health for downstream tasks.



% \section*{ACKNOWLEDGMENTS}



\section*{Acknowledgements}
% must add Acknowledgements and feature Nicholas prominently - say what specifically we talked to him about - also did we start with his code and modify it? must acknowledge all contributions - also please see other group papers for official wording to acknowledge OzGrav (CE170100004)
We thank Nicholas Kah Yean Low for sharing the automaton code in Ref.\ \cite{low_discerning_2022}, helping us to understand the model in Ref.\ \cite{low_discerning_2022}, helping to interpret the different behavior of Barab\'{a}si-Albert and complete networks in this paper, and verifying some simulation results.
We thank Yi Shuen Christine Lee and Jarra Horstman for additional discussions.
We thank Liam Saliba for providing technical programming advice. 
We thank the anonymous referees for constructive feedback and for suggesting the investigation in Section \ref{sec:BA}.
AM acknowledges funding from the Australian Research Council Centre of Excellence for Gco (OzGrav) (CE170100004). 


%% The Appendices part is started with the command \appendix;
%% appendix sections are then done as normal sections
\appendix


\section{Simulation implementation}
\label{sec:zoomzoom}
The computer code implementing Algorithm 1 is written in \texttt{Python3} \cite{van_rossum_python_1995}.
Where possible, libraries with underlying \texttt{C++} implementations are used to speed up mathematical calculations.
Networks are generated and manipulated with the graph theory library \texttt{Graph-tool} \cite{peixoto_graph-tool_2014}, which has a \texttt{C++} backend.
The \texttt{NumPy} library \cite{harris_array_2020} is used for numerical calculations, with vectorization applied to various loops to take advantage of its fast linear algebra functionality.
At each timestep, $x_i(t,\theta)$ is stored as a matrix, where each agent occupies a row, and each column holds discretized $\theta$ values. 
The likelihood broadcast to all agents by element-wise matrix multiplication synchronously via Eq.\ \eqref{eq:updatefirsthalf}. 
The interaction between agents is symmetric ($A_{ij} = A_{ji}$). 
Eq.\ \eqref{eq:xiprimed} is implemented via element-wise matrix subtraction, which calculates all interactions synchronously. 
Eq.\ \eqref{eq:undatesecondhalf} applies a mask to the $\Delta x' (t+1/2, \theta)$ matrix to obtain $x_i(t+1,\theta)$. 
Batched simulations are run in parallel on 10 threads. 
A single complete simulation with $n=100$ and $T=10^4$ typically takes 0.7 seconds on a 2021 MacBook Pro (Apple M1 Max, 32GB RAM, 3.2 GHz clock speed, 10 cores).

\section{Special case of a deterministic coin with $\theta_0=0$ (or $\theta_0=1$)}
\label{sec:twoallies}
In this appendix, for the sake of completeness, we discuss briefly the special case, where the coin returns heads or tails only. By way of illustration, we consider an allies-only network with $n = 100$ and one partisan. 
We focus on $\theta_0 = 0$, as the behavior for $\theta_0=1$ is analogous.

For $\theta_0=0$, the persuadable agents achieve asymptotic learning, achieving a bimodal final distribution 
with $x_i(t \geq t_{\rm a},\theta_0) \approx 0.995$ and  $x_i(t \geq t_{\rm a},\theta_{\rm p}) \approx 0.005$, as shown in Fig.\ \ref{fig:0truebias}.  
The persuadable agents still heed the partisan but weakly. 
Unlike the turbulent nonconvergence observed in Section \ref{subsec:samethetap}, all persuadable agents' beliefs asymptotically approach $\theta_0$ and $\theta_{\rm p}$, achieving asymptotic learning at $t_{\rm a} = 136$, as shown in Figs.\ \ref{fig:0truebias_mean}. 
This is because a coin with $\theta_0 = 0$ always returns tails.
From Eq.\ \eqref{eq:likelihood}, the likelihood is given by $P[S(t)| \theta] = 1-\theta $ for all $t$, which always equals one for $\theta = 0$ and is less than one for all other values of $\theta$. 
When the likelihood is multiplied by the prior according to Eq.\ \eqref{eq:undatesecondhalf}, one obtains $x_i(t+1, \theta) < x_i(t, \theta)$ for $\theta \neq 0$, and $x_i(t,\theta) \neq 0$ decreases iteratively and monotonically to zero. 
The second peak at $\theta_{\rm p}$ is obtained though Eq.\ \eqref{eq:xiprimed} from the internal interaction between allies.  
The persuadable agents always hold some belief in $\theta_{\rm p}$, as Eq.\ \eqref{eq:undatesecondhalf} is additive rather than multiplicative.

% Figure environment removed
It is interesting to ask how the persuadable agents respond, when they receive an unexpected external signal.
As a test, we run a simulation with $\theta_0=0$, returning $S(t) = {\rm tails}$ for $ 1\leq t \leq 500$, followed artificially by $S(t=501)= {\rm heads}$. 
The belief PDFs of the persuadable agents at $t = 500$ are identical to the PDF shown in Fig.\ \ref{fig:0truebias} but they change abruptly to $x_i(t=501,\theta) \approx \delta (\theta-\theta_{\rm p})$, after they observed a single heads. This readiness to agree with the partisan after a single, unexpected observation occurs for the following reason.
When persuadable agents observe $S(t=501) = {\rm heads}$, they infer $P[S(t=501)|\theta=0] = 0$ according to Eq.\ \eqref{eq:likelihood}. 
Therefore, we have $x_i(t = 500 + 1/2, \theta = 0) = 0$ via Eq.\ \eqref{eq:updatefirsthalf}, and $x_i(t=500+1/2,\theta_{\rm p}) = 1$ after renormalization.



\section{Long-term influence of initial priors and coin toss sequence}
\label{sec:coinvsprior}

In this appendix, we test how the initial priors $x_i(t=0,\theta)$ and the coin toss sequence affect the long-term evolution of $x_i(t,\theta)$ for persuadable agents.
% Same coin different prior

We start by simulating $10^4$ copies of the same network, which are identical, except that $x_i(t=0,\theta)$ for all the persuadable agents $i$ is drawn afresh in each copy from the truncated Gaussian distribution defined in Section \ref{subsec:automaton}.
All copies of the network witness the same coin toss sequence. Fig.\ \ref{fig:samecoin} plots the six independent pair-wise differences $\langle\theta\rangle_A - \langle\theta\rangle_B$ for four network copies (indexed by $A$ and $B$) and a single, arbitrary persuadable agent as functions of time.
The differences in mean belief evolve towards zero shortly after the simulation starts, with $| \langle \theta \rangle_A - \langle \theta \rangle_B | \leq 10^{-5}$ for $t \geq 5\times 10^2$.
This behavior is typical: the long-term belief PDF does not depend strongly on $x_i(t=0,\theta)$. It is also observed in networks without partisans \cite{acemoglu_opinion_2013}. 

Next we test the sensitivity to the coin toss sequence. In Fig.\ \ref{fig:sameprior}, we plot again the six pair-wise differences $\langle\theta\rangle_A - \langle\theta\rangle_B$ against time for a single, arbitrary, persuadable agent and four random, independent coin toss sequences, indexed by $A$ and $B$. 
The differences do not decay to zero, unlike in Fig.\ \ref{fig:samecoin}. 
Instead, they fluctuate steadily for all $0\leq t \leq 2\times 10^3$, with standard deviation $\approx 0.05$ throughout the interval. 
This behavior matches Section \ref{subsec:samethetap}: the partisan disrupts the system so that it never reaches equilibrium, and the coin tosses are an ongoing factor, competing with the pull of the partisan at every time step. 
% Figure environment removed


% TODO: uncomment the above section! It's commented out because it loves to break compilation 🙃 


\newpage
%% If you have bibdatabase file and want bibtex to generate the
%% bibitems, please use
%
\bibliographystyle{elsarticle-num} 
\bibliography{paper1}

\end{document}

%%
%% End of file `elsarticle-template-num.tex'.
