\documentclass[11pt,a4paper]{scrartcl}

% Defines default style and includes several useful packages
\usepackage{ILD}

% Useful macros for writing ILDdp notes
\usepackage[symbol]{footmisc}
\usepackage{feynmf}
\usepackage{multirow}
\usepackage{textpos}
\usepackage{amsmath}
%\usepackage{lineno}
%\linenumbers

%============================================%
% Set up the title page
%============================================%

% Set the title of the note
\title{Measurement of the CPV Higgs mixing angle in ZZ-fusion at 1 TeV ILC}

% Set the ILD note number
% Numbering convention:
% TTTT = topic (phys, soft or tech)
% YYYY = year
% NNN = number
% Papers and topical papers:
%\ildphys{YYYY}{NNN} % Physics 
%\ildsoft{YYYY}{NNN} % Software
%\ildtech{YYYY}{NNN} % Technical
% ILD notes and conference proceedings:
%\ildpublic{TTTT}{YYYY}{NNN}% Public note 
%\ildinternal{TTTT}{YYYY}{NNN} % Internal note 
%\ildproc{TTTT}{YYYY}{NNN} % Proceedings
% Example:
%\ildphys{2023}{NNN}
%\ildpublic{phys}{2021}{001}
\ildproc{phys}{2023}{006}
% ILD numbering convention:
% https://confluence.desy.de/display/ILD/ILD+Publication+and+Speakers+Bureau

% Set the publication date
\date{\today}
%\date{\formatdate{18}{6}{2014}}

% Define the authors and their institutes, they will appear exactly in the order as they are added
% Footnotes can be added using the \thanks command
\addauthor{N. Vuka\u{s}inovi\'{c}}{\institute{1}}
\addauthor{I. Bo\v{z}ovi\'{c}-Jelisav\u{c}i\'{c}}{\institute{1}}
\addauthor{G. Ka\u{c}arevi\'{c}}{\institute{1}}


\addinstitute{1}{``VIN\u{C}A'' Institute of Nuclear Sciences - National Institute of the Republic of Serbia, University of Belgrade, Mike Petrovi\'{c}a Alasa 12-14, 11351 Belgrade, Serbia}



% Add "On behalf of ... (optional)"
%\onbehalfof{This work was carried out in the framework of the ILD concept group}

% Define an abstract for the note 
\abstract{
Although the studies of tensor structure of the Higgs boson interactions with vector bosons and fermions at CMS and ATLAS experiments have established that the J$^\mathrm{PC}$ quantum numbers of the Higgs boson should be 0$^\mathrm{++}$, small CP violation in the Higgs sector (up to 10\% contribution of the CP-odd state) cannot be excluded with the current experimental precision. We review possibilities to measure CP violating mixing angle $\mathrm{\Psi_{CP}}$ between scalar and pseudoscalar states, at a linear electron-positron collider, at center-of-mass energy of 1 TeV.

\vspace{8cm}
\centering{Talk presented at\\ the International Workshop on Future Linear Colliders (LCWS 2023), 15-19 May 2023.} 
\newline
\centering{C23-05-15.3.}\\
\vspace{2cm}
\textit{This work was carried out in the framework of the ILD concept group}

}


% Uncomment this line to remove the stamp with the ILD note number from the top right corner
% of the title page
%\notitlestamp

%============================================%
% Bibliography
%============================================%
% define the list of bibliography data files
%\addbibresource{./ref.bib}

%============================================%
% Search path for images
%============================================%
\graphicspath{ {./logos/}{./figures/} }

%============================================%
% Start of the actual document
%============================================%
\begin{document}

% generates the title page
\titlepage



\section{Introduction}
\label{sec:intro}
%% Figure environment removed

\section{Introduction}
Automatic 3D reconstruction of clothed humans using image inputs has gained increasing significance due to its potential applications in a wide array of AR/VR scenarios. High-fidelity reconstructions typically depend on sophisticated capture systems, which are developed with dense camera arrays~\cite{collet2015high,joo2015panoptic,joo2018total}, programmable light-stages~\cite{Vlasic2009, guo2019relightables}, and depth sensors~\cite{newcombe2011kinectfusion,DoubleFusion,BodyFusion,dou2016fusion4d,newcombe2015dynamicfusion}. However, stringent capture environments equipped with complex hardware pose significant challenges for consumer-level applications.


In this context, considerable research effort has been dedicated to developing methods that allow for more flexible capture configurations, such as utilizing a few RGB inputs. Among these works, learning implicit functions \cite{iccv2020PIFu, saito2020pifuhd, hong2021stereopifu} has proven effective in achieving highly detailed reconstructions by integrating the advancements of deep neural networks. These methods employ large multi-layer perceptrons (MLPs) to predict the occupancy probability or truncated signed distance function (TSDF) value of every queried 3D point based on its associated local feature, which is extracted from images. They can recover a continuous surface at arbitrary resolutions without topology restrictions.


However, in typical MLP-based implicit networks, the occupancy or TSDF value at each location is solved independently with planar image features, rendering them less capable of addressing challenging cases such as occlusions. Consequently, these methods suffer from generalization and robustness issues, particularly when tackling strong occlusions caused by large motion or multiple interacting humans. 
Some follow-up studies  \cite{zheng2021deepmulticap,zheng2021pamir,huang2020arch} utilize an extra geometric model, SMPL~\cite{Loper2015}, to improve robustness by introducing strong shape priors. 
Their success typically relies on the assumption of geometrical similarity \cite{huang2020arch} between the shape prior and target reconstruction, making them intractable for handling complex cases with loose clothes and sensitive to errors in SMPL model fitting.



%\ping{this paragraph sounds like `TSDF is better than MLP/SMPL, and we use TSDF to solve the problem'. But in Sec 3, we are telling a different story, saying `MLP needs a 3D convolutional encoder'. We need to make these two sections consistent.}\sicong{I think in this paragraph we claim that the TSDF}


%We opt for Trucated Signed Distance Funtion (TSDF) volumetric representations as they are naturally suitable for convolution operations, which have shown remarkable performance for learning hierarchical features on 2D visual perception tasks \cite{SunXLW19}. 
%Meanwhile, TSDF also describes the gradual geometry change around shape surface, which is not reflected by occupancy volume. 

We instead revisit the 3D volumetric representation and resort to 3D convolutional neural networks (CNNs) for feature learning, due to their impressive performance in feature learning and the ability to incorporate spatial context. However, volumetric methods and 3D convolution involve discretization, which might raise concerns regarding whether a discretized volume can preserve subtle geometric details as continuous representations learned in implicit functions. We investigate the relationship between volume resolution and quantization error on synthetic data by converting target mesh objects to TSDF volumes, as shown in Figure~\ref{fig:quantization_error}. We observe that the quantization errors are significantly reduced by increasing volume resolution and become nearly negligible when reaching a relatively high resolution (e.g., 512 or higher). In other words, achieving fine-detailed reconstruction is not supposed to be restricted by the use of volume representations as long as a proper volume resolution is utilized. Therefore, we present a method with high-resolution feature volumes, e.g., 256 and 512, while traditional volumetric methods \cite{varol18_bodynet,gilbert2018volumetric} are often limited to much lower resolutions, such as 32 or 128.



On the other hand, an increase in volume resolution may lead to a cubic growth of memory overhead \cite{8100085}. Reducing memory costs while guaranteeing the granularity of volumetric representations is necessary for pursuing high-quality reconstruction. Thus, we adopt a coarse-to-fine approach and cull away irrelevant voxels to build a sparse high-resolution feature volume. At the coarse level, the network computes an initial TSDF by applying a U-Net with sparse 3D CNN \cite{3DSemanticSegmentationWithSubmanifoldSparseConvNet} on the sparse feature volume, which is carved by a visual hull. Through our experiments, it turns out that more than 95\% of the volume grids are discarded by the visual hull culling, making the sparse 3D CNN efficient. At the fine level, the network focuses on a narrow band near the zero-level set of the initial TSDF and discretizes the narrow band with smaller voxels. By employing this narrow-band culling, we further shrink the sampling space, resulting in a relatively small range of grid numbers (usually 300K--500K in our experiments) even with a high volume resolution of 512. The remaining voxels in the narrow band are associated with features that fuse high-frequency information from the computed normal maps upon the low-frequency shape from the coarse level to compute the TSDF at high resolution. The final mesh is then extracted from the TSDF using the Marching-Cube algorithm ~\cite{Lorensen87marchingcubes}.
% Different from the u-net sturcture to preserve global topology context, we then apply a shallow 3dcnn to compute the final TSDF $D_{final}$ which contain more local geometry detail.




% \ping{this paragraph can be expanded. It is an important contribution and often ignored by other works. stress on the novel idea of regressing blending weights instead of colors}

In addition to geometry, high-quality mesh texture is also a crucial factor contributing to visual appearance. Directly computing a color field in 3D space, as in \cite{iccv2020PIFu}, struggles to capture high-frequency texture details, while the neural radiance field (NeRF) \cite{yu2020pixelnerf} or the DoubleField~\cite{shao2022doublefield} require expensive per-instance optimization and are often unstable for sparse input images. In contrast, we adopt an image-based rendering approach to compute a texture atlas map, which is efficient and widely supported in existing computer graphics tools. 
Specifically, we compute a blending weight at each 3D point on the mesh surface to determine its color as a weighted average of the colors at its image projections. The blending weights can be computed at a relatively coarse resolution, e.g., 512 volume resolution in our case, and leave texture details to the high-resolution images, such as 1K or 2K. Unlike previous methods that generate blurry texturing results under sparse input, our method generalizes well on both synthetic and real data with just a few input views. 
Figure~\ref{fig:teaser} shows two examples reconstructed by our method. Despite the challenging garment, pose, and occlusion, our method recovers faithful shape, normal, and texture on the right.

%with a wide variety of poses and clothing styles, and it is also adaptive to handle input image with arbitrary resolutions.
%\sicong{For this concern we claim that when the resolution of dicretized volume meets certain threshold (which is 256 in our experiment), the quantization error can be neglected.} 



In summary, the main contributions of this paper are as follows:
\begin{itemize}
\vspace{-0.1in}
  \item 
  We revisit the 3D volumetric representation and demonstrate that it can support clothed human reconstruction with equal or even better performance compared to implicit representation. 
  \item 
  We develop a memory and computation-efficient method for high-resolution volumetric reconstruction using sophisticated sparse 3D CNN, coarse-to-fine estimation, and voxel culling by visual hull and narrow bands. 
  \item 
  We introduce a novel method to compute a texture atlas map, which captures rich appearance details from high-resolution input images.
  \item 
  We achieve impressive results on standard benchmark datasets Twindom and MultiHuman, significantly reducing the point-2-surface (P2S) precision to approximately 0.2cm from just six input views, with more than $50\%$ error reduction compared to the state-of-the-art methods, including DoubleField~\cite{shao2022doublefield} and PIFuHD~\cite{saito2020pifuhd}.
\end{itemize}

In this study we present a method for the determination of the CP violating (CPV) mixing angle between scalar and pseudoscalar CP states. We assume that the SM-like Higgs boson 125 GeV mass eigenstate ($\mathrm{h_{125}}$) is a superposition of a scalar (CP-even) state H and a pseudoscalar (CP-odd) state A, via a mixing angle $\mathrm{\Psi_{CP}}$:

\begin{equation}
\label{hmix}
h_{125} = H \cdot cos\Psi_{CP} + A \cdot sin\Psi_{CP}
\end{equation}

A common framework for interpretation of CPV results from future Higgs $e^-e^+$ factories (both linear and circular), as well as from LHC and HL-LHC is given in the Snowmass CPV white paper \cite{rsnowm} for measurements based on angular observables and Effective Field Theory (EFT). The benchmark CPV parameter $f_{CP}$ quantifies contributions from CP-even and CP-odd amplitudes (here in example of a $H\rightarrow X$ decay) as:

\begin{equation}
\label{factorf}
f_{CP} = \frac{\Gamma_{H\rightarrow X}^{CP^{odd}}}{\Gamma_{H\rightarrow X}^{CP^{odd}} + \Gamma_{H\rightarrow X}^{CP^{even}}}
\end{equation}

The sensitivity target for factor $f_{CP}$ is set from theory assuming 68\% CL measurement of the mixed Higgs state with up to 10\% CP-odd contribution. The estimates of $f_{CP}$ in the Higgs-vector boson vertices (HVV) are obtained from EFT theory in the Higgsstrahlung process. This is the first study addressing the $f_{CP}$ ($\Psi_{CP}$) determination in HVV vertices in Vector Boson Fusion. CPV in HVV vertices occurs at loop level, differently from Higgs-fermion vertices where it is realized at the Born level. 
As can be seen in Fig. \ref{fig1} \cite{rsnowm}, the precision target for factor $f_{CP}$ in HVV vertices is set to $< 10^{-5}$, assuming up to 10\% contribution of the CP-odd state.

% Figure environment removed

The ILC (International Linear Collider) is a mature option for a linear $e^-e^+$ collider designed to operate from 250 GeV up to 500 GeV \cite{r2} in the center-of-mass, with projected integrated luminosities of 2 $\mathrm{ab^{-1}}$ and 4 $\mathrm{ab^{-1}}$, respectively. It offers possibility for upgrade to 1 TeV. Beam acceleration will be realized in superconducting accelerating cavities \cite{ilcacc}. The baseline design includes polarization scenario for $e^-$ and $e^+$ beams of 80\% and 30\%, respectively.
We have considered 1 $\mathrm{ab}^{-1}$ of data simulated either in the full ILD detector simulation (Mokka \cite{r3}) or fast ILD detector simulation (DELPHES \cite{r4}) with ILC-gen card, in ZZ-fusion at 1 TeV center-of-mass energy. Due to the fact that ZZ-fusion is a t-channel process $e^{\pm}$ final states are emmited at low polar angles. In the central tracker it will be found about 42\% of initially produced signal events. However, 1 TeV is found to be the optimal energy for this measurement in comparison to other center-of-mass energies, due to the most optimal interplay of centrality (which decreases with energy) and Higgs boson production cross-section (which increases with energy). As neutral current processes favor $e^-_{L}e^+_{R}$ and $e^-_{R}e^+_{L}$ polarisation of the initial state, measurements can be combined to reduce uncertainties. This is planned to be done, however at this instant we present results obtained with 100\% $e^-_{L}e^+_{R}$ beam polarization.

The ILD detector model \cite{rild} is assumed in this analysis. All detector subsystems are placed within a magnetic field of B = 3.5 T. ILD detector comprises: all-silicon vertex detector, gaseous (Time Projection Chamber) as a central tracker and compact Electromagnetic (ECAL) and Hadronic (HCAL) calorimeters. Apart from TPC, the tracking system comprises: two barrel components (the Silicon Inner Tracker SIT and the Silicon External Tracker SET), one End cap Tracker (ETD) and the Forward Tracker (FTD), covering between 140 mrad and 3000 mrad in polar angles. Excellent performances of the tracking system enable measurement of transverse momenta ($\mathrm{p_{T}}$) with an asymptotic resolution of $\sigma{ (1/p_{T})}$ $\sim$ 2 $\cdot$ 10$^{-5}$ GeV$^{-1}$ \cite{rild}. Particle reconstruction and identification relies on Particle Flow Algorithm (PFA) \cite{rpfa} using information from all detector subsystems. For example, PFA provides separation of jets that originate from Higgs boson and vector bosons ($\mathrm{W^{\pm}, Z^{0}}$) with 3-5\% jet energy resolution \cite{rild}. 

 

\section{CPV sensitive observable}
\label{sec:zzf}
This analysis exploits the CPV-sensitive angular observable ($\Delta\Phi$), defined as the angle between production $e^-$ and $e^+$ planes in the Higgs rest frame, as illustrated in Fig. \ref{fig3}. One should note that the angle $\Delta\Phi$ is not the only CPV sensitive angular observable, but, as is shown in \cite{rsara}, it carries the most information on the CP nature of a Higgs boson.

% Figure environment removed

$\Delta\Phi$ is calculated according to the following definition:

\begin{equation}
\label{deltafi}
\Delta\Phi =
\begin{cases}
      arccos(cos (\Delta\Phi)), \mathrm{sgn} (sin (\Delta\Phi)) \geq 0\\
      2 \pi - arccos(cos (\Delta\Phi)), \mathrm{sgn} (sin (\Delta\Phi)) \leq 0\\
        \end{cases}
\end{equation}

\noindent where 

\begin{equation}
\label{cosfi}
cos (\Delta\Phi) = \overrightarrow{n}_{1}\cdot\overrightarrow{n}_{2} \hspace{1cm} \mathrm{and} \hspace{1cm} \mathrm{sgn} (sin (\Delta\Phi)) = \frac{\overrightarrow{q}_{1}\cdot(\overrightarrow{n}_{1}\times\overrightarrow{n}_{2})}{|\overrightarrow{q}_{1}\cdot(\overrightarrow{n}_{1}\times\overrightarrow{n}_{2})|}
\end{equation}

\noindent Unit vectors $\overrightarrow{n}_{1}$ and $\overrightarrow{n}_{2}$ are defined by momenta of initial $\overrightarrow{q}_{{e}^{-(+)}_{i}}$ and final state $\overrightarrow{q}_{{e}^{-(+)}_{f}}$ electron (positron) and they are orthogonal to the corresponding production planes:

\begin{equation}
\label{n12}
\overrightarrow{n}_{1} = \frac{ q_{e^-_{i}}\times q_{e^-_{f}} }{ |q_{e^-_{i}}\times q_{e^-_{f}}| } \hspace{1cm} \mathrm{and} \hspace{1cm} \overrightarrow{n}_{2} = \frac{ q_{e^+_{i}}\times q_{e^+_{f}} }{ |q_{e^+_{i}}\times q_{e^+_{f}}|}
\end{equation}

\noindent In Eq. (\ref{n12}) $\overrightarrow{q}_{1}$ stands for momentum of $Z^{0}$ boson in the electron plane. Sign of the $sin (\Delta\Phi)$ from Eq. (\ref{cosfi}) actually shows if the positron plane rotates forward ($\mathrm{sgn}(sin (\Delta\Phi))$ = +1) or backward ($\mathrm{sgn}(sin (\Delta\Phi))$ = -1)  with respect to the electron plane following the right-handed rule with $Z^{0}$ boson in the electron plane emitted in the direction of a right-hand thumb. 


\section{Event samples}
\label{sec:evtsamples}

The exclusive $H\rightarrow b\bar{b}$ final state is considered, to avoid high cross-section radiative processes ( $e^+e^- \rightarrow e^+e^-\gamma$). In case of signal event samples, about 3,500 events are fully simulated with the ILD detector and in the presence of the beam induced backgrounds. This statistics correspond to $\sim$ 1/5 of expected number of events in 1 $\mathrm{ab^{-1}}$ in full physical range of polar angles. About 28,000 events are simulated with Delphes V3.4.2 fast simulation, with ILD detector model. The relevant background (Table \ref{fig1}) is also fully simulated.

\begin{table}[!h]
\centering
\caption{\label{table:1} List of signal and background processes at 1 TeV with corresponding cross section $\sigma$, expected number of events in 1 ab$^{-1}$ $N_{evt}$, number of simulated events $N_{sim}$ and number of selected events in 1 ab$^{-1}$ of data $N_{sel}$. }
%\vspace*{0.5 cm}
\begin{tabular}{ |l| l| l| p{3cm}| p{3 cm}|}
\hline
1 TeV & $\sigma (fb)$	& $N_{\mathrm{evt}} @ 1 \mathrm{ab^{-1}}$ & $N_{\mathrm{sim}}$ & $N_{\mathrm{sel}}@ 1 \mathrm{ab^{-1}}$   \\
\hline
Signal: $e^-e^+\rightarrow Hee; H\rightarrow b\bar{b}$ 	& 16 & 16016/8231$^{\mathrm{tracker}}$ & 27911 DELPHES, 3495 MC & 5658    \\ 
\hline
$e^-e^+\rightarrow q\bar{q}l^+l^- (l = \mu, \tau)^\dag$  & 255 & 255000 & 5886(1/43) & $/$      \\
\hline
$e^-e^+\rightarrow q\bar{q}$  & 9375 & 9375000 & 120343(1/78) & $/$   \\
\hline
$e^-e^+\rightarrow q\bar{q}l\nu^\dag$  & 4116 & 4116000 & 955058(1/4) & $/$    \\
\hline
\end{tabular}
\end{table}


\begin{footnotesize}
$\dag$ Currently there is only simulation for $l = \mu, \tau$. Electron state will be additionally added.
\end{footnotesize}

% Figure environment removed

% Figure environment removed

% Figure environment removed



By applying electron isolation in Delphes, events with exactly one isolated electron and one isolated positron are preselected. Rest of the particles are clustered in 2 jets by the Durham algorithm \cite{rdurham}. After preselection, about 1170 background events are left in 1 $\mathrm{ab^{-1}}$ of data (Fig. \ref{figstack} (left)). In general, $\Delta\Phi$ distribution for background is flat as it is CP-insensitive, though it is not obvious from Fig. \ref{figstack} (left) where limitted number of background events is selected with large scaling factors.
Further event selection requires: cut on di-jet invariant mass: 80 GeV $< m_{q\bar{q}} <$ 160 GeV, cut on Z boson masses: $m_{Z_{1}, Z_{2} } >$ 30 GeV, cut on transverse momentum of final state system electron and positron: $p_{T_{ee}} >$ 15 GeV, cut on missing transverse momentum: $p_{T_{miss}}<$ 150 GeV. After applying all these criteria background is fully suppressed while the total signal efficiency, including preselection, is about 68\%. $\Delta\Phi$ for signal after full event selection is given in the Fig. \ref{figstack} (right). Event selection doesn't bias $\Delta\Phi$ distribution and that is illustrated in the Fig. \ref{figfios} (left).


In order to quantify the impact of polar angle acceptance and detector reconstruction effects on the $\Delta\Phi$ observable, Whizard generator version 2.8.3 \cite{rwhizard} is used with the Higgs characterization model \cite{r8} within the UFO framework. As can be seen from Fig. \ref{figgencor} (left), acceptance limited by the tracker polar angles affects the $\Delta\Phi$ distribution with respect to the full physical range. On the other hand, comparing generated and reconstructed information in the tracker, it is obvious that detector effects are negligible as is illustrated in the Fig. \ref{figfios} (right). Thus reconstructed information has to be corrected for the acceptance effects. Corrected $\Delta\Phi$ distribution is illustrated in Fig. \ref{figgencor} (right). It reasonably reproduces distribution at the generator level in the full physical range, up to statistical fluctuations. 


\section{$\mathrm{\Psi_{CP}}$ determination}

Unlike $H\rightarrow \tau^+\tau^-$ decays where the shape of the angular observable is derived from the differential cross-section \cite{rjeans}, dependence of $\Delta\Phi$ on CP violating mixing angle $\mathrm{\Psi_{CP}}$ is not known. Due to that we applied a phenomenological approach in extraction of $\mathrm{\Psi_{CP}}$ from simulated (or experimental) data. We have assumed CP-odd admixture up to 17\%, corresponding to $\mathrm{\Psi_{CP}}$ values up to 0.2 rad. $\Delta\Phi$ distributions are changing for different assumptions on $\mathrm{\Psi_{CP}}$ value, in a way that the local minimum gets shifted accordingly (to the left or right for negative or positive $\mathrm{\Psi_{CP}}$ values, respectively). This is illustrated in Fig. \ref{figfipsi} (left and right)).

% Figure environment removed

We perform a fit around local minimum with the fit function to determine its position:

\begin{equation}
\label{fitfunc}
f(\Delta\Phi, \mathrm{\Psi_{CP}}) = A + B \cdot cos(a \cdot \Delta\Phi - b)     
\end{equation}

\noindent where $\mathrm{A, B}, a$ and $b$ are free parameters, and the ratio of coefficients $b$ and $a$ gives the minimum of the $\Delta\Phi$ distribution.
Example of such a fit performed on 10${^5}$ generated events for $\mathrm{\Psi_{CP}}$ = 0.1 is illustrated in Fig. \ref{figkoefkm} (left). Position of the minimum ($b/a$) over $\mathrm{\Psi_{CP}}$ true value is a linear function of the true value of $\mathrm{\Psi_{CP}}$. This is illustrated in Fig. \ref{figkoefkm} (right). 

% Figure environment removed
Thus the following equation holds:
\begin{equation}
\label{bia}
(b/a)/\mathrm{\Psi_{CP}} = k \cdot \mathrm{\Psi_{CP}} + m     
\end{equation}

\noindent Coefficients $k$ and $m$ can be determined from simulation assuming certain range of $\mathrm{\Psi_{CP}}$ values, while minimum can be measured from the (experimental) data as is illustrated in Fig. \ref{figkoefkm} (left). Thus $\mathrm{\Psi_{CP}}$ can be determined from Eq. (\ref{bia}) by solving this quadratic equation. Dependence of the extracted values $\mathrm{\Psi_{exp}}$ versus the true values $\mathrm{\Psi_{true}}$ is given in Fig. \ref{dissip}. For the $\mathrm{\Psi_{true}} \leq 0.1$ corresponding to up to 9\% admixture of the CP-odd state differences between $\mathrm{\Psi_{true}}$ and $\mathrm{\Psi_{exp}}$ central values is not larger than 5 mrad. 

% Figure environment removed

\section{$\mathrm{\Psi_{CP}}$ from reconstructed data - a pseudo-experiment} 

% Figure environment removed

% Figure environment removed

Using the parameters ($k$ and $m$) of the linear fit from simulation (Fig. \ref{figkoefkm} (right)), measurement of the local minimum from the reconstructed data (Fig. \ref{phireconstr}) gives the value of $\mathrm{\Psi_{CP}}$ = 0.9 mrad for the pure scalar state. In order to estimate statistical dissipation of the extracted $\mathrm{\Psi_{CP}}$, we run 2000 pseudo-experiments for the pure scalar state, each experiment with about 1 $\mathrm{ab^{-1}}$ of generated data. This is illustrated in Fig. \ref{pseudo}. Absolute statistical uncertainty ($\Delta\mathrm{\Psi_{CP}}$) of extracted $\mathrm{\Psi_{CP}}$ value is determined by RMS of the distribution from Fig. \ref{pseudo} reflecting the probabilistic nature of the analized sample in the context of statistical population. RMS is found to be 7 mrad. $\Delta\mathrm{\Psi_{CP}}$ value of 7 mrad corresponds to 68\% CL uncertainty to measure $\mathrm{\Psi_{CP}}$ = 0 in 1 ab$^{-1}$ of data. The latter translates to uncertainty of the CPV factor $f_{CP} \approx$ 4.9 $\cdot$ 10$^{-5}$, that is still slightly above the theoretical target. Further improvement is expected in combination of results obtained on samples with different beam polarizations, as mentioned in Sec.\ref{sec:intro}. Systematic uncertainty from the fit $\leq$ 1 mrad is significantly smaller compared to the statistical one.


\section{Conclusion}
In this paper we present the result of the first CPV mixing angle measurement based on angular observable, in HVV vertex at an $e^-e^+$ collider. Measurement is simulated in ZZ-fusion at 1 TeV ILC with the ILD detector. Local minimum of the angular observable is sensitive to the  $\mathrm{\Psi_{CP}}$  mixing angle and can be obtained from the phenomenological fit of the reconstructed (simulated or experimental) data. Individual measurement on the fully simulated data gives deviation of 0.9 mrad from the truth value for the pure scalar state. The method is stable for $\mathrm{\Psi_{CP}}$  variations up to 0.2 rad, corresponding to pseudoscalar admixture of $\sim$ 17\%. At 1 TeV ILC with 1 ab$^{-1}$ of $e^-_{L}e^+_{R}$ data, pure scalar should be measured with statistical uncertainty of 7 mrad at 68\% CL. Obtained statistical uncertainty corresponds to the precision of the CPV factor $f_{CP} \approx$ 4.9 $\cdot$ 10$^{-5}$. The obtained precision is not yet final, as ILC offers possibility to combine different beam polarizations. 



\section*{Acknowledgements}
We would like to acknowledge our colleagues from ILC IDT Working Group 3 for the technical support as well as for comments on the physics content, especially to Aleksander Filip \.{Z}arnecki for useful ideas and discussions leading to better understanding of the sensitive observable behavior.   
This research was funded by the Ministry of Education, Science and Technological Development of the Republic of Serbia and by the Science Fund of the Republic of Serbia through the Grant No. 7699827, IDEJE HIGHTONE-P.





% add references
%\printbibliography{}

\begin{thebibliography}{99}
%\begin{linenumbers}
%\linenumbers
\bibitem{rsnowm}
A. V. Gritsan et. al, \emph{Snowmass White Paper: Prospects of CP-violation measurements with the Higgs boson at future experiments}, \href{https://arxiv.org/pdf/2205.07715.pdf}{ arXiv:2205.07715v2 [hep-ex]} (2022).
 

\bibitem{r2}
T. Barklow et al., \emph{ILC Operating Scenarios}, ILC-NOTE-2015-068, \href{https://arxiv.org/pdf/1506.07830.pdf}{arXiv:1506.07830v1} [hep-ex] (2015).

\bibitem{ilcacc}
C. Adolphsen et al., \emph{The International Linear Collider Technical Design Report - Volume 3.II:
Accelerator Baseline Design}, \href{https://arxiv.org/ftp/arxiv/papers/1306/1306.6328.pdf}{arXiv:1306.6328} [physics.acc-ph] (2013).

\bibitem{r3}
P. Mora de Freitas and H. Videau, \emph{Detector Simulation with MOKKA/GEANT4: Present and Future, International Workshop on Linear Colliders}, JeJu Island, Korea, Technical Report No. LC-TOOL-2003-010, (2002).

\bibitem{r4}
J. de Favereau et al., \emph{DELPHES 3: a modular framework for fast simulation of a generic collider experiment}, Journal of High Energy Physics 2014, 57, \href{https://arxiv.org/abs/1307.6346}{arXiv:1307.6346} [hep-ex] (2014).

\bibitem{rild}
T. Behnke et al., \emph{The International Linear Collider Technical Design Report - Volume 4: Detectors},
\href{https://arxiv.org/abs/1306.6329}{arXiv:1306.6329} [physics.ins-det] (2013).


\bibitem{rpfa}
M. A. Thomson, \emph{Particle flow calorimetry and the pandora PFA algorithm}, Nucl. Instrum. Methods Phys. Res., Sect. A 611, 25 (2009). 

\bibitem{rsara}
S. Bolognesi et al., \emph{Spin and parity of a single-produced resonance at the LHC}, Phys. Rev. D 86 (2012) 095031, \href{https://arxiv.org/pdf/1208.4018.pdf}{arXiv:1208.4018} (2012).

\bibitem{rdurham}
S. Catani et al, \emph{New clustering algorithm for multi - jet cross-sections in $e^+e^-$ annihilation}, Phys. Lett. B 269 (1991) 432-438 (1991).

\bibitem{rwhizard}
W. Kilian, T. Ohl, and J. Reuter, \emph{WHIZARD: Simulating multi-particle processes at LHC and ILC}, Eur. Phys. J. C 71, 1742 (2011).

\bibitem{r8}
P. Artoisenet et al., \emph{A framework for Higgs characterization}, Journal of High Energy Physics 2013, 43, \href{https://arxiv.org/pdf/1306.6464.pdf}{arXiv:1306.6464} [hep-ph] (2013). 

\bibitem{rjeans}
D. Jeans and G. W. Wilson, \emph{Measuring the CP state of tau lepton pairs from Higgs decay at the ILC}, Phys. Rev. D 98 013007 (2018).

 
\end{thebibliography}


\end{document}