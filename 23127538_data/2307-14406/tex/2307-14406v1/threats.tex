\section{Threats to Validity}
\label{sec:threats}

Given the empirical nature of our study, we discuss several threats to the validity of this work according to the guidelines proposed by \cite{runeson2009guidelines}, and how these threats were partially mitigated in our study.

% Given the empirical nature of our study, potential threats can affect the study results. We classify and discuss these threats by following the recommendations suggested in \cite{Wohlin2000Experimentation}.

\textbf{Construct Validity} reflects on the extent of consistency between the operational measures of the study and the RQs. In this work, we depended on human activities, including data labelling and data extraction \& analysis, which would introduce personal bias. To reduce this threat, each step in the aforementioned human activities was conducted by two authors and a third author was involved to discuss and resolve the conflict in case of disagreement. Moreover, we also conducted a pilot data labelling to make sure that the two researchers achieved a consensus on what are code snippets in this study, which could also partially alleviate this threat. 

Another threat to the construct validity of this study is that we used mostly closed-ended questions in the industrial survey, which may affect the richness of the responses collected from the participants. However, as argued by \cite{reja2003openended}, open-ended questions have several disadvantages compared with closed-ended questions. For example, much long time to fill out the questionnaire might make participants do not participate in the survey at all. Participants may provide poor answers or even just skip when answering open-ended questions. Due to the above disadvantages, we chose to mainly use closed-ended questions in our survey. For some of the closed-ended questions, we also provided the ``Other'' field so that participants can fill in their own opinions if existing options do not cover their thoughts. Furthermore, to help participants better understand the open-ended questions in the survey, we provided two examples for each question. During the data analysis, we found that some participants only agreed with the provided examples without providing additional answers, which indicates that the provided examples may restrict participants from providing their own answers to the open-ended questions, thus affecting the richness of answers. Besides, another threat is that some of the responses from participants are written in Chinese, and translating the raw data from Chinese to English may lead to information lost or corruption. The two authors who extracted and analyzed the Chinese responses are native Chinese speakers, and the third author who is a native Chinese speaker as well was asked to check and refine the translation, which partially minimizes this threat.
% However, the provided examples come from the results of the pilot interview before the formal survey, that means the examples are provided by other participants. We argue that it could partially reduce the threat of the examples to the answers.

The last threat is concerning the size of our dataset. We collected 63 responses from our industrial survey, and we acknowledge that the small number of responses may threaten the validity of our findings. Therefore, we conjecture that we could obtain more convincing results by inviting more developers with code review experiences from more diverse communities to participate in the survey, which is also our next step.

\textbf{Internal Validity} considers the causal relationships between variables and results. A threat to internal validity in this study is that we used a programming language detection tool called Guesslang to assist us in labelling the code review comments containing code snippets. During code review, review comments might contain both literal contents and code snippets, which may affect the labelling accuracy of automatic tool (i.e., Guesslang), thus affecting the results of this study. To mitigate this threat, we only used Guesslang to help filter out review comments that most likely do not contain code snippets, in order to retrieve the review comments containing code snippets as many as possible. In addition, to make the threshold of filtering as accurate as possible, we first conducted a pilot labelling, and adjusted the threshold based on the results of the pilot labelling.

\textbf{External Validity} refers to the degree to which our study results and findings can be generalized in other cases (e.g., other projects in OpenStack and Qt communities, or projects in other communities). We selected active projects from the OpenStack and Qt communities since these two communities have made a serious investment in code review for many years and have been widely used in many studies related to code review. We argue that the selected communities and projects are representative and can increase the generalizability of our study results. 

In terms of the industrial survey, we invited developers from the OpenStack and Qt communities collected from our dataset, developers from well-known software companies in China, and developers from professional software development groups, which partially increases the external validity of the survey results. But we admitted that the findings of this study may not be generalized to all developers. In the future, we plan to invite more developers from various development groups (e.g., inner source development) to expand the scope of the industrial survey.
% We plan to invite developers in other domains in the future, which would make the results of this study more generalized.

\textbf{Reliability} refers to the replicability of a study for obtaining same or similar results. To improve the reliability, we made a research protocol with detailed procedure, which was discussed and confirmed by all the authors. Besides, all of the empirical steps in our study, including the data mining process, data labelling, and data extraction and analysis, were conducted and discussed by three authors. Furthermore, the dataset and analysis results of our study have been made publicly available online~\cite{replpack} in order to facilitate other researchers to replicate our study easily. We believe that these measures can partially alleviate this threat.