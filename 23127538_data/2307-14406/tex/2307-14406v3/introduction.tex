\section{Introduction}
\label{sec:introduction}

% Software peer review by a large community of developers and other stakeholders has proven to be a successful technique and one of the most important aspects of Open Source Software (OSS) development \citep{breu2010information, nurolahzade2009the, rigby2008open}. Nowadays, Modern Code Review (MCR) practices are becoming more lightweight, co-located, continuous, and asynchronous \citep{sadowski2018modern} due to the rise of agile and OSS development \citep{rigby2013convergent}. Bacchelli and Bird \citep{bacchelli2013expectations} referred MCR as a review that is informal, tool-based, and occurs regularly in practice. MCR not only provides assurance of software quality, but also contributes to design improvement, knowledge sharing, and code ownership \citep{nazir2020modern}.
Code review has proven to be one of the best practices in software development which is conducted by the developers other than the code author through a manual inspection of source code \citep{ackerman1989software}. Modern Code Review (MCR) is a type of code review that is informal, tool-based, and occurs regularly in practice \citep{bacchelli2013expectations}. Nowadays, MCR practices are becoming more lightweight, co-located, continuous, and asynchronous \citep{sadowski2018modern, badampudi2023modern}, due to the rise of agile and Open Source Software (OSS) development \citep{rigby2013convergent}. In a typical MCR process, reviewers and developers engage in asynchronous online discussions to exchange information with each other, ensuring that the proposed code changes are of sufficient quality and align with the direction of the project before they are accepted \citep{pascarella2018information}. MCR not only provides assurance of software quality \citep{davila2021systematic}, but also contributes to design improvement, knowledge sharing, and code ownership \citep{nazir2020modern}. 

% Modern Code Review is a process that requires a high degree of collaboration between reviewers and developers. During the MCR process, reviewers and developers engage in asynchronous discussions to exchange information with each other, ensuring that the proposed code changes are of sufficient quality and align with the direction of the project before they are accepted \citep{pascarella2018information}. Understanding the information in code review comments is a prerequisite of efficient code review process. When developers submit code changes, they are concerned about information related to whether they make mistakes in their new code and whether they follow their team's conventions \citep{ko2007information}, which can be obtained from review comments in code reviews. However, it is challenging for developers to get the necessary information in MCR. Although many tools (e.g., Gerrit, Veracode, Reshift, etc.) support the process of code review, developers and reviewers still have need of richer communication when in code reviews \citep{bacchelli2013expectations}. 
Understanding the information in code review comments is a prerequisite of efficient code review process. When developers submit code changes, they are concerned about information related to whether they make mistakes in their new code and whether they follow their team's coding conventions \citep{ko2007information}, which can be obtained from code review comments. However, it is challenging for developers to get the necessary information in MCR \citep{bacchelli2013expectations}. Although many tools (e.g., Gerrit, Veracode, Reshift) support the process of MCR, developers and reviewers still have the need of richer communications including a wide range of mechanisms in code review process \citep{bacchelli2013expectations}. \cite{sutherland2009can} investigated the code review practices of software product teams at Microsoft to understand the nature of the code review dialog and exchange of information, how the information was retained, and the nature of its later reuse. The results indicated that the retention and recovery of information in code reviews is not well supported in the current environment.

Code review comments are text-based, which may contain textual content like URL links and code snippets for developers to get useful information for a better understanding of the review comments. Previous studies have investigated the practice of link sharing and their purposes in code reviews, and have explored what types of information could be provided through link sharing \citep{wang2021understanding}. Similar to links, code snippets could also convey necessary information for developers in code reviews. However, the different nature of links and code snippets leads to their dedicated purposes and influences in the practice of code review. To the best of our knowledge, little study has investigated the use of code snippets in MCR, and it is still unknown about the purposes of providing code snippets and the practices and knowledge of using code snippets in code reviews.
% and it is still unknown about what types of information could be obtained by developers through providing code snippets in code reviews. 

To bridge this gap, we conducted this mixed-methods study to provide a comprehensive understanding of code snippets in code reviews. We mined code review discussions from four most active projects selected from the OpenStack\footnote{\url{https://www.openstack.org/}} community (Nova\footnote{\url{https://wiki.openstack.org/wiki/Nova}} and Neutron\footnote{\url{https://wiki.openstack.org/wiki/Neutron}}) and the Qt\footnote{\url{https://www.qt.io/}} community (Qt Base\footnote{\url{https://github.com/qt/qtbase}} and Qt Creator\footnote{\url{https://www.qt.io/product/development-tools}}) based on the number of closed code changes. The code review process of these four projects are all supported by Gerrit\footnote{\url{https://www.gerritcodereview.com}}, a Web-based code review platform built on the top of Git. 
% By using the RESTful API provided by Gerrit, we were able to collect all the closed code changes of the four selected projects updated in 2020 and 2021. 
We also used an online survey to explore code snippets in code reviews by following the guidelines provided by Kitchenham and Pfleeger \citep{kitchenham2008personal}. We sent the survey questionnaire to potential respondents from the popular developer groups (i.e., LinkedIn) and to practitioners from the four selected projects in OpenStack and Qt (i.e., Nova, Neutron, Qt Base, and Qt Creator). In total, we got 3,197 review comments containing code snippets by manually checking 69,604 code review discussions obtained from the mined data sources, and we received 63 valid responses from the survey questionnaire. A comprehensive quantitative and qualitative analysis were conducted to study the extent of using code snippets, the purposes of providing code snippets, and how developers treat code snippet suggestions in code reviews, in order to understand the practices and purposes of code snippets in MCR. Our results suggest that: (1) Code snippets are not prevalently used in code reviews, and most of the code snippets in review comments are provided by reviewers. (2) Reviewers make code snippet suggestions in code review with the aim of \textit{Suggestion} and \textit{Citation}, in which \textit{Suggestion} is the main purpose. (3) For code snippet suggestions, most developers would accept them. (4) The main reasons why developers not accept reviewers' code review suggestions are \textit{Difference in the opinions between developers and reviewers} and \textit{Reviewer's suggestion is flawed}. (5) Reviewers tend to provide code snippets in code reviews \textit{when code is more illustrate than words}. (6) For review comments containing code snippets, most developers hold positive attitudes. (7) Developers expect that code snippets provided by reviewers in code reviews can be \textit{Concise}, \textit{Correct}, and \textit{Executable}.


This paper extends our earlier work on studying code snippets in code reviews \citep{fu2022understanding} through the following additions:
\begin{enumerate}[itemindent=1.5em]
    \item We extended our dataset by including the code review data from two most active projects of the Qt community and collecting all the closed code changes updated in 2021, which are 127,182 code review comments in total.
    \item We explored the reasons why developers do not accept code snippet suggestions in code reviews (RQ4).
    \item We further investigated the scenarios when reviewers provide code snippets, developers' attitude towards code snippets, and the characteristics of code snippets that developers expect reviewers to provide in code reviews (RQ5, RQ6, and RQ7) through an industrial survey.
\end{enumerate}

The rest of this paper is structured as follows: Section \ref{sec:relatedWork} presents the related work. Section \ref{sec:methodology} describes the research design of this study. Section \ref{sec:results} provides the study results, which are further discussed in Section \ref{sec:discussion}. The potential threats to validity are clarified in Section \ref{sec:threats}, and Section \ref{sec:conclusions} concludes this work with future directions.
