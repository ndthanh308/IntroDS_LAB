\documentclass[12pt]{article}   	% use "amsart" instead of "article" for AMSLaTeX format
\usepackage{geometry}                		% See geometry.pdf to learn the layout options. There are lots.
\geometry{letterpaper}                   		
%\usepackage[parfill]{parskip}    		% Activate to begin paragraphs with an empty line rather than an indent
\usepackage{graphicx}				% Use pdf, png, jpg, or eps§ with pdflatex; use eps in DVI mode
\usepackage{caption}
\usepackage{subcaption}
\usepackage{amssymb}
\usepackage{amsthm}
\usepackage{csquotes}
\usepackage{amsmath}
\usepackage{mathtools}
\usepackage{float}
\usepackage{physics}
\MakeOuterQuote{"}
\newtheorem*{theorem*}{Theorem}
\newtheorem{definition}{Definition}
\newtheorem{theorem}{Theorem}[section]
\newtheorem{proposition}[theorem]{Proposition}
\newtheorem{lemma}[theorem]{Lemma}
\newtheorem{sublemma}[theorem]{Sub-Lemma}
\newtheorem{remark}[theorem]{Remark}
\newtheorem{corollary}[theorem]{Corollary}
\newtheorem{conjecture}[theorem]{Conjecture}
\newtheorem*{conjecture*}{Conjecture}
\title{Partial Regularity and Blowup for an Averaged Three-Dimensional Navier-Stokes Equation}
\author{Matei P. Coiculescu \thanks{Supported by an N.S.F. Graduate Research Fellowship}}
\begin{document}
\maketitle
\begin{abstract}
The Navier-Stokes Equations on the Euclidean space $\mathbb{R}^3$ may be written as
$$\partial_t u -\Delta u +\mathcal{E}(u,u) =0,$$
where the bilinear operator $\mathcal{E}$ formally allows dissipation of energy via the "cancellation identity" $\langle \mathcal{E}(u,u), u\rangle=0$. We find an averaged version of $\mathcal{E}$, call it $B$, satisfying the cancellation identity and such that 
$$\partial_t u +(-\Delta)^\alpha u +B(u,u)=0$$
admits a solution that blows up in finite time for all $\alpha \in (3/4,5/4)$. Moreover, we prove that the solution is smooth at blowup time away from a closed set of Hausdorff dimension at most $5-4\alpha$. Our result strongly suggests that obtaining a positive answer to the Navier-Stokes global regularity problem requires more than a refinement of partial regularity theory.
\end{abstract}
\tableofcontents
\section{Introduction}
The incompressible Navier-Stokes equations are a system of nonlinear partial differential equations that model the motion of an incompressible viscous fluid. In particular, if $u(x,t)$ is the velocity field of a fluid in $\mathbb{R}^3\times [0,T)$ dynamically changing in time $t$, with initial vector field $u_0$, external force $f$ and fluid pressure $p$, then the Navier-Stokes equations state that $u$ satisfies:
$$
\begin{gathered}
\partial_t u -\nu\Delta u+(u\cdot\nabla)u+\nabla p=f\\
\textrm{div } u=0\\
u(t=0)=u_0
\end{gathered}
$$
The parameter $\nu$ determines the viscosity of the fluid. Up to a rescaling of the solution $u$, we may assume that the viscosity $\nu=1$. We also consider the Navier-Stokes equations with "fractional dissipation". In particular let $\alpha\geq 0$ and consider the pseudodifferential operator $(-\Delta)^\alpha$ whose Fourier symbol is $\abs{\xi}^{2\alpha}$. Whenever $\alpha$ equals a positive integer $k$, the operator coincides, up to a sign, with the composition of the classical Laplacian $(-\Delta)$ with itself $k$ times. The unforced $\alpha$-dissipative Navier-Stokes system in $\mathbb{R}^3$ is the following system of pseduodifferential equations:
\begin{equation}
\label{FRACNS}
\begin{gathered}
\partial_t u+(-\Delta)^\alpha u+ (u\cdot \nabla)u +\nabla p = 0 \\
\textrm{div } u=0 \\
u(t=0)= u_0
\end{gathered}
\end{equation}
The $\alpha$-dissipative Navier-Stokes system is typically called hyperdissipative if $\alpha>1$ and hypodissipative if $\alpha<1$. The case when $\alpha=1$ coincides with the classical Navier-Stokes equations. For $\alpha\geq 5/4$ and for any smooth function $u_0$ which decays sufficiently fast at infinity, it is known that the system (\ref{FRACNS}) has a classical global-in-time solution. A very short proof of this fact when $\alpha >5/4$ can be found in the introduction of \cite{KP}. Global regularity is also known for an open set of initial data if $\alpha$ is slightly less than $5/4$, see \cite{CH}. Generally, when $\alpha<5/4$, the global existence of classical solutions is still an open question. As one of the Millennium problems, the question of global regularity is stated in \cite{F} as:
\begin{conjecture}
Let $u_0$ be a Schwartz vector field on $\mathbb{R}^3$. Is there a vector field $u(x,t)\in C^\infty(\mathbb{R}^3\times [0,\infty))$ and pressure function $p(x,t)\in C^\infty(\mathbb{R}^3\times [0,\infty))$ satisfying the Navier-Stokes equations with initial data $u_0$ and no external force?
\end{conjecture}

One approach towards proving global regularity is to show that the set where singularities of the vector field develop is small in some sense. In particular, one might want to find a bound on the Hausdorff dimension of such a set. Finding such bounds is within the purview of partial regularity theory, which was first initiated for elliptic partial differential equations. See, for example, the theorems in Almgren's monograph \cite{A}. 

We now provide a synopsis of the partial regularity theory for the Navier-Stokes equations. With characteristic prescience, Leray showed in \cite{L} that the set of singular times has $1/2$-dimensional Hausdorff measure equal to zero. More precisely, given a Leray-Hopf weak solution $u$ to the Navier-Stokes equations on $\mathbb{R}^3\times [0,\infty)$ satisfying the strong energy inequality, Leray showed the existence of a sequence of disjoint intervals of time $J_i$ with $J_0=(a,\infty)$, Lebesgue measure of $\mathcal{T}:= [0,\infty)-\cup_{i\geq0} J_i$ equal to zero, $u$ smooth on $\mathbb{R}^3\times J_i$ for all $i$ after modification on a Lebesgue null subset, such that for any $\epsilon>0$ there exists a covering $\{K_j\}_{j\in \mathbb{N}}$ of $\mathcal{T}$ by intervals satisfying
$$\sum_{j}(\textrm{length}(K_j))^{1/2}<\epsilon.$$
Scheffer wrote several papers, beginning with \cite{S}, bounding the Hausdorff dimension of the singular set in space at the the time of first blowup. We describe the theorem Scheffer obtains in \cite{S}. Taking $t_i$ to be the right endpoint of any interval $J_i$ as above, Scheffer proved that $u$ can be extended to a continuous function on 
$$(\mathbb{R}^3\times J_i)\cup ((\mathbb{R}^3-S)\times \{t_i\})$$
where $S$ is a closed set of finite $1$-dimensional Hausdorff measure.
The state-of-the-art partial regularity result was obtained by Caffarelli, Kohn, and Nirenberg in \cite{CKN} for a particular class of weak solutions called suitable weak solutions. Let $u$ be a suitable weak solution on any open subset of spacetime $\mathbb{R}^3\times\mathbb{R}$. If we define the singular set (in space-time) as
$$S= \{ (x,t) : u \not\in L^\infty \textrm{ in any neighborhood of } (x,t)\}$$
then the $1$-dimensional Hausdorff measure of $S$ is equal to zero. Away from the set $S$, the vector field $u$ is bounded, which  implies smoothness by the Serrin higher regularity theorem, see \cite{SE} for more details. The bound on the Hausdorff dimension obtained by Caffarelli, Kohn, and Nirenberg in 1982 has not yet been improved. 

Our present work is most closely related to the paper \cite{KP} of Katz and Pavlovic. The authors of \cite{KP} prove a partial regularity theorem like Scheffer's in \cite{S} using Fourier analysis. They show that if a smooth classical solution of the strictly hyperdissipative system (\ref{FRACNS}) on $\mathbb{R}^3\times[0,T)$ forms a singularity at finite time $T$, then 
$$\limsup_{t\to T} \abs{u(x,t)} <\infty$$
for all $x$ away from a set of Hausdorff dimension at most $1$. The authors of \cite{KP} employ a heuristic of "wavelets" by localizing in both physical and frequency space. We imagine space partitioned into cubes varying in scale and location, and we decompose our vector field into "wavelets" supported on these cubes. We obtain a partial regularity result by counting the number of cubes at each scale where "dampening" energy dissipation is dominated by the "concentrating" nonlinear term of the Navier-Stokes equation. The analysis is performed by understanding the dynamics of the "wavelet coefficient" of each cube. Essentially we, like the authors of \cite{KP}, shall follow this outline. 

The work of Colombo, De Lellis, and Massaccesi in \cite{CDM} provides the optimal extension of the Caffarelli-Kohn-Nirenberg to the hyperdissipative case, working with methods similar to those used in \cite{CKN}. However, we cannot see a way to transfer these techniques to the setting of our interest, so we chose to apply the Fourier-Analytic ideas of \cite{KP}. We also find it appropriate to also mention the work \cite{O}, whose author extends the result of \cite{KP} from classical solutions to Leray-Hopf weak solutions of the Navier-Stokes equations. As mentioned before, we can interpret the partial regularity theory for the Navier-Stokes equations as a step towards answering the global regularity question. For example, one could attempt to push the bound on the Hausdorff dimension of the singular set to zero, which would yield a globally regular vector field. See \cite{KP} for more details.

We now discuss the work done in the "negative" direction of the global regularity problem.
Besides numerical studies like \cite{H} that suggest the development of a finite-time singularity for the Navier-Stokes equations, the paper \cite{T} by Tao provides evidence that at the very least, attacking the problem of global regularity using abstract methods depending on basic $L^p$ bounds for the nonlinearity is a direction certain to fail. Let $\mathbb{P}$ denote the Leray projection on divergence-free vector fields, cf. Section 1.6 in \cite{TS}. Tao constructs a nonlinearity $C(u,v)$ that is some averaged version of the Euler bilinear operator
$$\mathcal{E}(u,v)=\frac{1}{2}\mathbb{P}\bigg((u\cdot \nabla) v+(v\cdot \nabla) u\bigg),$$
satisfies the cancellation identity and various "harmonic-type" estimates such as
$$\|C(u,v) \|_{p} \leq K\big(\|u\|_{q}\|\nabla v\|_{r}+\|v\|_{q}\|\nabla u\|_{r}\big)$$
where $1/p = 1/q+1/r$ and $1< p,q,r< \infty$, and whose associated pseudodifferential equation admits a finite-time singularity.

In our paper we show a result similar in spirit to Tao's: one that suggests that improving partial regularity results for the Navier-Stokes equations is not a path to answering the global regularity question, unless one uses more specific structural properties of the Navier-Stokes nonlinearity.

We denote the parabolic cylinder of radius $r$ at $(x,t)$ by:
$$\mathcal{P}_r(x,t):= \{ (y,s) : \abs{y-x}< r \textrm{ and } t-r^2< s < t\}$$
Here and in the sequel we denote the closed singular set at blowup time $T$ by
$$S_T:= \{ x\in\mathbb{R}^3 : \forall r>0, u\not\in  L^\infty_t C^\infty_x(\mathcal{P}_r(x,T))\}.$$
Our main theorem in this paper constrains the Hausdorff dimension of $S_T$ for blowup solutions of an averaged three-dimensional $\alpha$-dissipative Navier-Stokes Equation. In particular, we prove:
\begin{theorem}
\label{MAINMAIN}
Let $3/4<\alpha<5/4$ be arbitrary.
There exists a symmetric bilinear operator $B$, which is an averaged version of $\mathcal{E}$, obeys the cancellation identity
$$\langle B(u,u), u\rangle =0$$
for all $u\in H^{10}(\mathbb{R}^3)$ that are divergence-free in the distributional sense, and such that, for some Schwartz divergence-free initial vector field $u_0$, there is no global-in-time solution to
$$\begin{gathered} \partial_t u +(-\Delta)^\alpha u +B(u,u) +\nabla p =0\\
u(t=0)=u_0 \quad \quad \textrm{div } u=0.\end{gathered}
$$
Moreover, if $T$ is the time of first blowup, then the closed set $S_T$ has Hausdorff dimension at most $5-4\alpha$.
\end{theorem}

The main thrust of our paper is proving a general partial regularity result for a family of pseudodifferential equations of "Navier-Stokes type". Then proving Theorem \ref{MAINMAIN} amounts to finding an equation in the family admitting a finite-time singularity. The proof of our partial regularity theorem will be in Section 2, and we describe it in more detail here. First we define a certain class of bilinear operators.

\begin{definition}
\label{AMEN}
We call a symmetric bilinear operator $B$ $\textbf{amenable}$ if and only if:
\begin{enumerate}
\item $B$ is defined for Schwartz vector fields
\item $B$ satisfies the Cancellation Identity:
\begin{equation}\label{Cancel}
\langle \mathbb{P}B(u,u), u\rangle=0 \quad \textrm{ for divergence-free } u
\end{equation}
\item The linear maps $B^1_v(u):= \mathbb{P}B(u,v), B^2_v(u):= \mathbb{P}B(v,u)$ are pseudodifferential operators in some class $OPS^m_{1,1}$ (the definition of this class may be found in Section 5.1)
\item For all $\gamma> 0$ and for all $1\leq p_1, p_2, r\leq \infty$ satisfying
$$\frac{1}{p_1}+\frac{1}{p_2}+\frac{\gamma}{3} = \frac{1}{r}$$
we have:\begin{equation}
\label{amenineq}\| \mathbb{P}B(u,v) \|_{L^r} \leq K \| u\|_{L^{p_1}}(\|v\|_{L^{p_2}} +\| \nabla v\|_{L^{p_2}})\end{equation}
where $u\in L^{p_1}$, $v\in W^{1,p_2}$, and $K$ is some constant possibly depending on $\gamma$. (Note that the endpoint cases of integrability are included)
\item Scaling: There exists a number $\lambda>1$ such that for any $m\in \mathbb{Z}$:
$$B(u(\lambda^m y), u(\lambda^m y))(x) = \lambda^m B(u(y),u(y))(\lambda^m x)$$
\end{enumerate}
\end{definition}
Our main partial regularity theorem is:
\begin{theorem}
\label{PRMAINMAIN}
Let $3/4<\alpha<5/4$. Let $B$ be an amenable bilinear operator.
Suppose $u(t)$ is a smooth vector field on $\mathbb{R}^3\times [0,T) \to \mathbb{R}^3$ and suppose there exists a smooth function $p$ on $\mathbb{R}^3\times [0,T)$ such that:
$$
\begin{gathered}
\partial_t u +(-\Delta )^\alpha u+B(u,u)+\nabla p=0\\
\textrm{div } u=0\\
u(t=0)=u_0\end{gathered}
$$
If $T$ is the time of first blowup, then the closed set $S_T$ has Hausdorff dimension at most $5-4\alpha$.
\end{theorem}

Our theorem is a partial regularity result of wide generality. For instance the $\alpha$-dissipative Navier-Stokes equations were previously known to admit partial regularity whenever $\alpha>3/4$. The authors of \cite{TY} dealt with the hypodissipative case $3/4<\alpha <1$ and the work of \cite{CDM} dealt definitively with the hyperdissipative case. On the other hand, an immediate corollary of Theorem \ref{PRMAINMAIN} is a partial regularity result for the $\alpha$-dissipative Navier-Stokes equations whenever $\alpha>3/4$, which illustrates the versatility of our result. 

We believe it beneficial to the reader to now succinctly outline the rest of the paper. In Section 2 we prove Theorem \ref{PRMAINMAIN}. In particular, given the vector field solution $u$, we construct a set $M$ of Hausdorff dimension $5-4\alpha$ and prove that it contains the singular set $S_T$. In Section 3, we test the partial differential equation in the statement of Theorem \ref{PRMAINMAIN} with localizations of the solution in both frequency and physical space. We then prove estimates on the size of these wavelet coefficients that are used in the proofs of Section 2. In Section 4, we show that Tao's method is flexible enough to construct a blowup solution to the pseudodifferential equation associated to some amenable bilinear operator, which, together with Theorem \ref{PRMAINMAIN}, proves Theorem \ref{MAINMAIN}. In Section 5, we prove the lemmas that form our "technical toolbox" and that are used throughout the rest of the paper. 

We would like to thank our advisor Camillo de Lellis for suggesting the topic of this paper, for many helpful discussions, and for reviewing several drafts of our work. This paper would never have come to fruition without his support. We would also like to thank Vikram Giri for constructive conversations about mathematical fluid dynamics in general and about the paper \cite{KP} in particular.
\section{General Partial Regularity Results}
We refer the reader to Section 5.1 whenever we mention pseudodifferential operators and Section 5.2 whenever we mention the Paley-Littlewood partition of frequency space.
\subsection{Covering the Singular Set}
Let $\mathbb{P}$ denote the Leray projection. Let $\mathcal{F}$ denote the Fourier transform. We use a Littlewood-Paley partition of frequency space, which is described in Section 5.2. In particular, for $j\in \mathbb{Z}$ we have pseudodifferential operators $P_j$ with positive and smooth symbols $p_j(\xi)$ that are supported in $\frac{2}{3}2^j < \abs{\xi}<3\cdot 2^j$, which also satisfy $p_j(\xi) = p_0(2^{-j} \xi)$ and 
\begin{equation}\sum_{j\in \mathbb{Z}} p_j(\xi) =1.\end{equation}
We also define $\tilde{P}_j:= \sum_{k=-2}^{2} P_{j+k}$ to be the sum of all Littlewood-Paley projections whose symbols' supports intersect the support of $p_j(\xi)$. Likewise we may define $\tilde{p}_j(\xi)$ as $\sum_{k=-2}^{k=2} p_{j+k}(\xi)$.
We can consider $P_j f$ as a combination of wavelets supported on cubes of side-length $2^{-j}$. Keeping this heuristic in mind, we localize on cubes of side-length almost $2^{-j}$. That is, we fix an $\epsilon>0$ and localize within cubes of side-length $2^{-j(1-\epsilon)}$. Since the Hausdorff dimension estimate is a closed condition, an approximate localization like this will be acceptable. Our localization is achieved with the use of particularly chosen bump functions that we describe below.

Let $Q$ be a cube of side-length greater than $2^{-j(1-\epsilon)}$. Recall that $\lambda Q$ denotes the cube with the same center as $Q$ and with $\textrm{side-length}(\lambda Q) =\lambda\cdot \textrm{side-length}(Q)$. We define a bump function $1\geq \phi_{Q,j}\geq 0$ such that $\phi_{Q,j}=1$ on $Q$ and is zero outside of $(1+2^{-\epsilon j})Q$. We also require that for every multi-index $\alpha$, there is a constant $C_\alpha$ depending on $\alpha$ but independent of $Q$ such that
\begin{equation}
\abs{D^\alpha \phi_{Q,j}}\leq C_\alpha 2^{\abs{\alpha}j(1-\epsilon)}
\end{equation}
Following \cite{KP}, we call these bump functions of type $j$. In line with our previous discussion, we consider $\phi_{Q,j}P_j$ as a projection onto a localized wavelet, and, if $Q$ is a cube with side length exactly equal to $2^{-j(1-\epsilon)}$, we denote
$$\| \phi_{Q,j}P_j f\|_2:= f_Q$$
and say that $Q$ is at level $j$. We also define the first nuclear family, $\mathcal{N}^1(Q)$, of a cube $Q$ to be the union of five collections of cubes $A_{Q,i}$ where there are fewer than $2^{10}$ cubes in each $A_{Q,i}$ that cover $(1+2^{-\epsilon j})Q$. Moreover, we require that the cubes in $A_{Q,i}$ be at level $j-3+i$ and $i\in\{1,2,3,4,5\}$. We may recursively define $\mathcal{N}^l(Q)$ to be the union of the collections $\mathcal{N}^1(Q')$ for all $Q' \in \mathcal{N}^{l-1}(Q)$. With this definition, it is evident that
$$N(\mathcal{N}^l(Q))\leq 2^{13l},$$
where we have used the notation $N(A)$ to denote the cardinality of a set $A$.
As shorthand, we also write for every $l$:
$$f_{\mathcal{N}^l(Q)}^2 := \sum_{Q'\in\mathcal{N}^l(Q)}f_{Q'}^2.$$
Throughout Section 2 we use various properties satisfied by the bump functions $\phi_{Q,j}$ and the Littlewood-Paley projections $P_j$. Most of these properties can be classified as "commutator estimates" that allow us to move bump functions across Littlewood-Paley projections and vice versa with only the addition of a negligible error. For the benefit of exposition, we defer the technical proofs of these lemmas to Section 5; however, we shall use the results in Section 5 freely here.

We prove a general partial regularity theorem for systems of differential equations. These systems will generalize the Navier-Stokes equations in the sense that they arise from the Stokes equations with a nonlinearity of the form $B(u,u)$, where $B$ is an amenable operator in the sense of Definition \ref{AMEN}. Without loss of generality, we may assume in this section that the scaling term $\lambda$ in the fifth part of Definition \ref{AMEN} is equal to $2$. 

We denote the Leray Projection of $B$ by $\hat{B}:=\mathbb{P}B$ for the rest of this section. By the incompressibility condition on $u$ in the statement of the theorem, we can instead consider the following system of partial differential equations:
\begin{equation}
\label{PEQN}
\begin{gathered}
\partial_t u +(-\Delta)^\alpha u +\hat{B}(u,u)=0\\
u(t=0)=u_0.
\end{gathered}
\end{equation}
Then we prove:
\begin{theorem}
\label{PRTHM}
Let $3/4<\alpha < 5/4$.
Let $B$ be an amenable bilinear operator.
Suppose that $u(t)$ is a smooth vector field on $\mathbb{R}^3\times [0,T) \to \mathbb{R}^3$ that solves Equation (\ref{PEQN}).
If $T$ is the time of first blowup, then the closed set $S_T$ has Hausdorff dimension at most $5-4\alpha$.
\end{theorem}

The next subsection is devoted to proving Theorem \ref{PRTHM}. We remark that if we take
$$B(u,v) = \frac{1}{2}\bigg( (u\cdot \nabla)v +(v\cdot \nabla u)\bigg)$$
we have the usual $\alpha$-dissipative Navier-Stokes equations in Equation (\ref{PEQN}). Thus, our method here recovers a result similar to the previously mentioned works of \cite{S}, \cite{KP}, \cite{CDM}, and \cite{TY} on partial regularity.

We use that our solution $u$ is smooth up to time $T$, a fact we shall use frequently and usually without mention. Pairing Equation (\ref{PEQN}) with $u$ and using the cancellation identity in (\ref{Cancel}), we get:
\begin{equation}\label{NEEDIT}\langle\partial_t u, u \rangle + \langle(-\Delta)^\alpha u , u \rangle =0.\end{equation}
Integrating in time then gives us what is typically known as the energy dissipation law:
\begin{equation}
\label{Conserv}
\| u(t)\|_2^2 = \|u_0\|_2^2 -\int_0^t \|(-\Delta)^{\alpha/2} u\|_2^2 dt \leq \|u_0\|_2^2 \quad \forall t\in[0,T).\end{equation}
We shall examine the dynamics of the "wavelet coefficients" $u_Q:= \| \phi_{Q,j}P_j f\|_2$, where $\phi_{Q,j}$ is a bump function of type $j$ and $Q$ is a cube at level $j$. To this end, we pair the equation (\ref{PEQN}) with $P_j \phi_{Q,j}^2 P_j u$, and get, after simplification:
\begin{equation}
\label{COEFFODE}
\partial_t \bigg(\frac{1}{2} u_{Q}^2\bigg)= \quad \langle-\hat{B}(u,u),P_j \phi_{Q,j}^2 P_j u \rangle  -\langle(-\Delta)^{\alpha}u,P_j \phi_{Q,j}^2 P_j u \rangle. \end{equation}


For the rest of this section, we allow constants $K$ to depend on the $L^2$ norm of the initial data, the blowup time $T$, and fixed constants $\epsilon, \gamma$. 

We now describe the construction of a set $M$ that contains $S_T$, which is the singular set of Theorem \ref{PRTHM}. First we call a cube $Q$ at level $j$ \textbf{mildly bad} if and only if for some constant $K$ independent of $j$:
$$2^{3000j}\int_{T-2^{-3000j}}^T u_{\mathcal{N}^{1000/\epsilon}(Q)}^2dt+\int_0^T \int \sum_{k\geq j} 2^{2\alpha k}\abs{\phi_{Q,j}P_{k} u}^2 dxdt \geq K 2^{-(5-4\alpha)j-100\epsilon j-\gamma j}.$$
We shall consider the reverse inequality as the hypothesis of an $\epsilon$-regularity statement compatible with the scaling of the equations. In other words, 
$$(2^m)^{2\alpha -1}u((2^m)x, (2^m)^{2\alpha} t)$$
is a solution of Equation \ref{PEQN} for any $m\in \mathbb{Z}$, if $u(x,t)$ is a solution of the same equation. The terminology is justified because the "bad" cubes in \cite{KP} are also mildly bad.

We define the set $M_j$ to be the union of cubes $Q$ for all mildly bad cubes $Q$ at level $j$.
We first obtain a convenient cover of $M_j$:
\begin{proposition}\label{1cover1}
There is a covering $\mathcal{C}_j$ of $M_j$ by cubes at level $j$ such that for some constant $K$ independent of $j$:
$$N(\mathcal{C}_j ) \leq K 2^{(5-4\alpha)j+100\epsilon j+\gamma j}.$$
\end{proposition}
\begin{proof}
Let $\mathcal{A}$ be the collection of cubes $2^{3000/\epsilon}Q$ where $Q$ is mildly bad and at level $j$. By the Vitali Covering Lemma (Lemma \ref{Vitali}), there are disjoint cubes $2^{3000/\epsilon}Q'$ in $\mathcal{A}$ such that the collection of cubes $5\cdot 2^{3000/\epsilon}Q'$ covers $M_j$. However, any cube of sidelength $5\cdot 2^{3000/\epsilon}\cdot 2^{-j(1-\epsilon)}$ can be covered by $K$ cubes at level $j$ ($K$ depending on $\epsilon$ only), so we are led to define $\mathcal{C}_j$ to be the collection of the cubes at level $j$ that cover the cubes $5\cdot 2^{3000/\epsilon}Q'$. Now since the cubes $Q'$ are mildly bad with cubes $2^{3000/\epsilon}Q'$ disjoint, the (large) nuclear families $\mathcal{N}^{1000/\epsilon}(Q)$ are also disjoint. By the energy dissipation law and Lemma \ref{finiteband1} we may conclude:
$$N(\mathcal{C}_j)2^{-(5-4\alpha)j-100\epsilon j-\gamma j}\leq$$
$$\leq K \sum_{Q'} \bigg(\sum_{k\geq j}2^{2\alpha k} \int_0^T\int \abs{\phi_{Q',j}P_k u}^2 + 2^{3000j}\int_{T-2^{-3000j}}^T u^2_{\mathcal{N}^{1000/\epsilon}(Q')}dt\bigg) \leq$$
$$\leq K \bigg(  \sum_{k\geq j}2^{2\alpha k}\int_0^T \int \sum_{Q'}\abs{\phi_{Q',j}P_{k} u}^2 + 2^{3000j}\int_{T-2^{-3000j}}^T \sum_{Q'}u^2_{\mathcal{N}^{1000/\epsilon}(Q')}dt\bigg) \leq K,$$
which proves the lemma.
\end{proof}
\begin{corollary}\label{1HausBound}
The Hausdorff dimension of the set $M= \limsup_{j\to \infty} M_j$ is bounded by $5-4\alpha+100\epsilon + \gamma$.
\end{corollary}
\begin{proof}
This is an easy application of Lemma \ref{dimcompute} using Proposition \ref{1cover1}.
\end{proof}
Next we demonstrate that 
\begin{equation}\label{GOAL}x\not\in M \Longrightarrow x\not\in S_T\end{equation}
Then $M$ contains the set $S_T$. By Corollary \ref{1HausBound} and the arbitrariness of $\epsilon>0$ and $\gamma>0$, we shall have proven Theorem \ref{PRTHM}. Equation (\ref{GOAL}) is all that remains to be proven.

\subsection{Regularity Away from Mildly Bad Cubes}
We prove Equation (\ref{GOAL}).
Since $M=\limsup_{j\to \infty} M_j$, we have the equivalence
$$x\not\in M \Longleftrightarrow \exists j \textrm{ s.t. } \forall k>j \quad x\not\in M_k .$$
We denote $L_j$ the set of points $x$ such that $x\not\in M_k$ for all $k>j$. Since $L_{j'} \subset L_{j}$ whenever $j\geq j'$, it suffices to prove a regularity statement for $x\in L_{j}$ for any $j\geq j_0$ for some large $j_0$ in order to prove a regularity statement about $x$ in the complement of $M$.

Thus, by changing the constant $K$ in the definition of mildly bad cubes appropriately, we may assume that $x\in L_{j}$ for some $j\geq j_0$, where we are free to choose the integer $j_0$. We continue with this assumption for the rest of this subsection, in a manner that will be made explicit later, with the cost that our constants $K$ now depend on some fixed $j_0$ as well. 
\begin{proposition}
\label{1CritReg}
There exists a sufficiently large integer $j_0$ such that for any integer $j'> j_0$, for any integer $j\geq \max(j'/\delta, j' +1000/\epsilon)$, and for any cube $Q$ at level $j$ with $Q\subset L_{j'}$ we have
$$\limsup_{t\to T} u_Q(t) < 2^{-50j}.$$
\end{proposition}
\begin{proof}
We may well assume that $j_0\geq0$ from the outset.
We shall use the following notation from Section 3:
For a cube $Q$ at level $j$ we denote:
$$k<j \Rightarrow Q_k := 2^{(j-k)(1-\epsilon)}Q,\quad k\geq j\Rightarrow Q_k = (1+2^{-\epsilon k/2})Q.$$ 
Assume that the conclusion of the proposition is false. In other words, we assume that for any $j_0\in \mathbb{Z}$, there exists integers $j'\geq j_0$ and $j\geq \max(j'/\delta, j' +1000/\epsilon)$ and there exists a cube $Q$ at level $j$ with $Q\subset L_{j'}$ such that 
\begin{equation}\label{CONTRADICT}\limsup_{t\to T} u_Q(t) > 2^{-50j}.\end{equation}
Henceforth, we may assume that $j'\geq j_0$ and $j\geq \max(j'/\delta, j' +1000/\epsilon)$ are the smallest integers so that Equation (\ref{CONTRADICT}) holds.
Our first claim is that Equation (\ref{CONTRADICT}) implies that for any numbers $\lambda>0$ and $\delta>0$, we have:
\begin{equation}\label{BADDYTWOSHOES}u_Q(T-\delta) \leq \lambda \limsup_{t\to T} u_Q(t).\end{equation}
Indeed, by the smoothness of the velocity field before time $T$, we would otherwise have for some constants $\lambda, \delta>0$ and $K$ depending only on $\delta$
$$2^{-50j}< \limsup_{t\to T} u_Q(t)< \frac{1}{\lambda}u_Q(T-\delta)<\frac{K 2^{-1000j}}{\lambda},$$
which, choosing $j_0$ large enough, would lead to a contradiction. Henceforth, we let $t_n$ be a sequence of time converging to $T$ such that
$$\lim_{n\to \infty} u_Q(t_n) = \limsup_{t\to T} u_Q(t).$$
Without loss of generality we may truncate the sequence $t_n$ so that for all $n$:
\begin{equation}\label{GOODYTWOSHOES} T-2^{-3000j}< t_n\end{equation}

Using Corollary \ref{Integral2}, Equations (\ref{COEFFODE}), (\ref{CONTRADICT}), and (\ref{BADDYTWOSHOES}), our estimate on the dissipation term in Proposition \ref{dissipationestimate}, and Corollary \ref{dissipationestimate2}, we have
$$
\int_{t_n}^{T} \big(\langle-\hat{B}(u,u),P_j \phi_{Q,j}^2P_j u\rangle -K2^{2\alpha j}u_Q^2 +K 2^{(2\alpha-\epsilon)j}\sum_{Q'\in\mathcal{N}^1(Q)}u_{Q'}^2\big)dt \geq  2^{-100j}.
$$
Using Lemma \ref{almostneg}, we know that the $2^{(2\alpha-\epsilon)j}$ term is almost negligible in the sense that get for some $l<400/\epsilon$:
$$
\int_{t_n}^{T}\bigg( -K 2^{2\alpha j}u_{\mathcal{N}^l(Q)}^2 +K2^{(2\alpha-\epsilon)j}u_{\mathcal{N}^{l+1}(Q)}^2\bigg)dt \leq -K\int_{t_n}^T 2^{2\alpha j}u_{\mathcal{N}^l(Q)}^2 dt.
$$
for appropriate (and possibly different) constants $K$. The upshot of this is that
$$
\sum_{Q\in\mathcal{N}^l(Q)}\int_{t_n}^{T} \langle-\hat{B}(u,u),P_j \phi_{Q,j}^2P_j u\rangle dt \geq 2^{-100j}+K\int_{t_n}^T 2^{2\alpha j}u_{\mathcal{N}^l(Q)}^2 dt.$$
The number of cubes in $\mathcal{N}^l(Q)$ is bounded by a constant depending only on $\epsilon$, thus we also know there must at least be one single cube $Q'\in \mathcal{N}^l(Q)$ such that
\begin{equation}\label{1bad00}
\int_{t_n}^T \langle-\hat{B}(u,u),P_j \phi_{Q',j}^2P_j u\rangle dt \geq 2^{-100j}+K\int_{t_n}^T 2^{2\alpha j}u_{\mathcal{N}^l(Q)}^2 dt.\end{equation}
At this step, we use our estimates of the nonlinear term to reach a contradiction. We also note that by our assumptions, the cube $Q'$ is contained in no larger mildly bad cubes.
Recall that from Lemmas \ref{lowhigh}, \ref{highlow}, \ref{highhigh}, and \ref{local}, we have for any $\delta>0$:
$$\abs{ \langle-\hat{B}(u,u),P_j \phi_{Q',j}^2P_j u\rangle} \leq K2^{j(1+3\delta/2+\gamma)}u_{\mathcal{N}^1(Q')}u_{Q'}+K\sum_{k=\delta j}^{j-1000/\epsilon} 2^{\frac{3k}{2}+j +\gamma j}u_{Q'_k}u_{\mathcal{N}^1(Q')}u_{Q'}+$$
$$+ K \sum_{k>j+1000/\epsilon} 2^{\frac{149j}{100}+\gamma j +\frac{101k}{100}}u_{Q'}\|\phi_{Q'_j,j}P_k u\|_2^2 +K2^{5j/2+\gamma j}u_{Q'}u^2_{\mathcal{N}^{1000/\epsilon}(Q')}+K2^{-150j}.$$

Since $\alpha>3/4$ and due to the Sobolev embedding theorem, the time integral of all the terms above is finite and bounded by a constant $K$ that is uniform in $j$. Thus, the time integral of the above quantity is well-defined and we may manipulate it freely.
We shall use the definition of mildly bad cubes to appropriately bound the terms above and achieve a contradiction.
For the rest of the proof we denote $Q_1$ to be an arbitrary member of the collection $\mathcal{N}^1(Q')$. We first aim to bound the contribution from the "high-low" terms. We observe that since $\alpha>3/4$, we may choose $\delta>0$ small enough so that $2^{j(1+3\delta/2 +\gamma)}u_{Q_1}\leq K2^{2\alpha j}u_{\mathcal{N}^1(Q')}$. It follows that 
\begin{equation}\label{1bad1}\int_{t_n}^TK2^{j(1+3\delta/2+\gamma)}u_{\mathcal{N}^1(Q')}u_{Q'}dt\leq \int_{t_n}^TK2^{2\alpha j}u_{\mathcal{N}^1(Q')}^2dt.\end{equation}
Now let $Q_2$ be a cube at level $k$ that is in the collection of cubes $\{Q'_k\}$ such that $\delta j\leq k < j-1000/\epsilon$. The cube $Q_2$ is a rescaling of $Q'$ that contains $Q'$. Since $k\geq \delta j >j'$, $Q_2$ cannot be mildly bad. Likewise, the cube $Q_1$ cannot be mildly bad since $j> j'+1000/\epsilon$. 
We are going to estimate the following integral:
$$\int_{t_n}^T 2^{\frac{3k}{2}+j+\gamma j}u_{Q_1}u_{Q_2}dt.$$
Since $Q_1$ and $Q_2$ are both not mildly bad, we have:
$$\int_{t_n}^T 2^{3k}u_{Q_2}^2 dt \leq K 2^{(2\alpha-2)k-100\epsilon k -\gamma k }$$
$$\int_{t_n}^T 2^{2j+2\gamma j}u_{Q_1}^2 \leq K 2^{(4\alpha -3)j+\gamma j -100\epsilon j}(T-t_n)\leq K2^{-500j}.$$
Thus, by the Cauchy-Schwartz inequality and since $$1000/\epsilon<j-k\leq (1-\delta)j$$ we have, for a constant $K$ depending on $\epsilon$:
$$\int_{t_n}^T 2^{\frac{3k}{2}+j+\gamma j}u_{Q_1}u_{Q_2}dt \leq K 2^{-140j}.$$
Then, using the fact that $u_{Q'}\leq K$, summing over $k$ and $\mathcal{N}^1(Q')$, and since there are only around $j$ terms:
\begin{equation}\label{1bad2}
\sum_{k=\delta j}^{j-1000/\epsilon}\int_{t_n}^T 2^{\frac{3k}{2}+j+\gamma j}u_{\mathcal{N}^1(Q')}u_{Q_k'}u_{Q'}dt \leq K 2^{-140j}
\end{equation}
For the "local" frequencies, we again use $u_{Q'}\leq K$ and the definition of the mildly bad cubes to conclude:
$$
\int_{t_n}^T 2^{5j/2+\gamma j}u_{Q'}u^2_{\mathcal{N}^{1000/\epsilon}(Q')}dt \leq$$ \begin{equation} \label{1bad3}\leq \int_{t_n}^T2^{5j/2+\gamma j}u^2_{\mathcal{N}^{1000/\epsilon}(Q')}dt\leq K (T-t_n)2^{-(5-4\alpha)j+3j}\leq  K 2^{-140j}.   
\end{equation}
We now deal with the "high-high" frequency term. By the H\"{o}lder inequality, the Minkowski inequality for integrals, and since our sums are absolutely convergent, we have:
$$2^{\frac{149j}{100}+\gamma j} \int_{t_n}^T u_{Q'}\bigg( \sum_{k>j+1000/\epsilon}2^{\frac{101k}{100}}\|\phi_{Q'_j,j}P_k u\|_2^2\bigg)dt \leq$$ $$\leq 2^{\frac{149j}{100}+\gamma j}\sum_{k\geq j+1000/\epsilon} 2^{\frac{101k}{100}} \bigg(\int_{t_n}^T u_{Q'}^{5} dt\bigg)^{1/5}\bigg(\int_{t_n}^T\|\phi_{Q'_j,j}P_k u\|_2^{5/2} dt\bigg)^{4/5}.$$
Using the energy dissipation law, we perform some crude estimates and get:
$$2^{\frac{149j}{100}+\gamma j}\sum_{k\geq j+1000/\epsilon} 2^{\frac{101k}{100}} \bigg(\int_{t_n}^T u_{Q'}^{5} dt\bigg)^{1/5}\bigg(\int_{t_n}^T\|\phi_{Q'_j,j}P_k u\|_2^{5/2} dt\bigg)^{4/5}\leq $$
$$\leq K2^{\frac{149j}{100}+\gamma j}\sum_{k\geq j+1000/\epsilon} 2^{\frac{101k}{100}} \bigg(\int_{t_n}^T u_{Q'}^{2} dt\bigg)^{1/5}\bigg(\int_{t_n}^T\|\phi_{Q'_j,j}P_k u\|_2^{2} dt\bigg)^{4/5}.$$
Since $Q'$ is not mildly bad, we get:
$$2^{\frac{149j}{100}+\gamma j} \int_{t_n}^T u_{Q'}\bigg( \sum_{k>j+1000/\epsilon}2^{\frac{101k}{100}}\|\phi_{Q'_j,j}P_k u\|_2^2\bigg)dt \leq$$
$$\leq K2^{\frac{149j}{100}+\gamma j}\sum_{k\geq j+1000/\epsilon} 2^{\frac{101k}{100}} 2^{-600j} 2^{\frac{-8\alpha k}{5}} 2^{-\frac{-4(5-4\alpha)j}{5}}.$$
Since $\alpha>3/4$, the infinite sum is convergent and we certainly get:
\begin{equation} \label{1bad4} 2^{\frac{149j}{100}+\gamma j} \int_{t_n}^T u_{Q'}\bigg( \sum_{k>j+1000/\epsilon}2^{\frac{101k}{100}}\|\phi_{Q'_j,j}P_k u\|_2^2\bigg)dt \leq 2^{-140j}.\end{equation}
Using our assumptions (\ref{1bad00}), our estimates in (\ref{1bad1}), (\ref{1bad2}), (\ref{1bad3}), (\ref{1bad4}), and our estimation of the nonlinear term, we may conclude that
$$2^{-100j}+\int_{t_n}^T 2^{2\alpha j}u_{\mathcal{N}^l(Q)}^2 dt \leq K\int_{t_n}^T 2^{2\alpha j}u_{\mathcal{N}^l(Q)}^2 dt  + K2^{-140j}$$
which is a contradiction for $j>j_0$ and $j_0$ large enough (depending on $K$).
\end{proof}
A consequence of Proposition \ref{1CritReg} is:
\begin{corollary}\label{DOESIT}
Equation (\ref{GOAL}) is true.
\end{corollary}
\begin{proof}
Let $x\notin M$. Then there exists $j'\geq j_0$ and $j\geq \max(j'/\delta, j' +1000/\epsilon)$ depending on $x$ such that $x$ lies in a cube $Q$ at level $j$ satisfying $Q\subset L_{j'}$. By Proposition \ref{1CritReg} we know that 
$$\limsup_{t\to T} u_Q(t) < 2^{-50j}.$$
Thus, there exists a $\delta_1>0$ depending only on $j$, which depends only on $x$, such that
$$\sup_{t\in [T-\delta_1, T]} u_Q(t) < 3\cdot 2^{-50j}.$$
In particular, by Lemma \ref{finiteband1}, Lemma \ref{EMBED}, and since the number $50$ that appears above can be made arbitrarily large in the course of our proofs, we conclude there exists $r>0$ depending only on $x$ such that
$$x\in L^\infty_tC^\infty_x(\mathcal{P}_r(x,T)),$$
which proves that $x\not\in S_T$. What we have shown is equivalent to proving Equation (\ref{GOAL}).
\end{proof}
Finally, Corollary \ref{DOESIT} directly implies Theorem \ref{PRTHM}.

\section{Estimates on the Local Energy}
\subsection{Estimate for the Dissipation Term}
We first consider the dissipation term in Equation (\ref{COEFFODE}).
\begin{proposition}
\label{dissipationestimate}
Let $Q$ be a cube at level $j$. We have:
$$\langle(-\Delta)^\alpha u , P_j \phi_{Q,j}^2 P_j u \rangle \quad \geq -K2^{-100j}+K2^{2\alpha j}u_Q^2 -K 2^{(2\alpha-\epsilon)j}\sum_{Q'\in\mathcal{N}^1(Q')}u_{Q'}^2$$
\end{proposition}
\begin{proof}
Evidently we have:
 \begin{equation}
\label{32EQN}
\langle(-\Delta)^\alpha u , P_j \phi_{Q,j}^2 P_j u \rangle =\langle(-\Delta)^\alpha \phi_{Q,j}P_j u , \phi_{Q,j}P_j u \rangle+\langle[(-\Delta)^\alpha,\phi_jP_j] u , \phi_{Q,j} P_j u \rangle\end{equation}
It remains to estimate the two terms on the right-hand-side of Equation (\ref{32EQN}).
We begin by using Lemma \ref{commutator1} to state:
\begin{equation}\label{300EQN}\abs{\langle(-\Delta)^\alpha \phi_{Q,j}P_j u , \phi_{Q,j}P_j u \rangle -\langle (-\Delta)^\alpha \tilde{P}_j \phi_{Q,j} P_j u, \phi_{Q,j}P_j u \rangle } \leq K 2^{-400j}\end{equation}
and
\begin{equation}\label{3000EQN}\abs{\langle \phi_{Q,j}P_j u , \phi_{Q,j}P_j u \rangle -\langle  \tilde{P}_j \phi_{Q,j} P_j u, \phi_{Q,j}P_j u \rangle } \leq K 2^{-400j}.\end{equation}
Then, because of the localization in frequency space, we have
$$\langle (-\Delta)^\alpha \tilde{P}_j \phi_{Q,j} P_j u, \phi_{Q,j}P_j u \rangle =$$ \begin{equation}\label{400EQN}=\int \abs{\xi}^{2\alpha}\tilde{p}_j(\xi)\abs{\mathcal{F}(\phi_{Q,j}P_j u)(\xi)}^2d\xi \geq K 2^{2\alpha j}\langle \tilde{P}_j \phi_{Q,j}P_j u, \phi_{Q,j}P_j u\rangle.\end{equation}
Using Equation (\ref{300EQN}), Equation (\ref{3000EQN}), and Equation (\ref{400EQN}) we conclude that
\begin{equation}
\label{Step1}\langle(-\Delta)^\alpha \phi_{Q,j}P_j u , \phi_{Q,j}P_j u \rangle \quad \geq -K2^{-150j} + 2^{2\alpha j} u_Q^2.\end{equation}
Now we turn to the second term on the right-hand-side of Equation (\ref{32EQN}). First, by Lemma \ref{pseudoproduct}, we have $[(-\Delta)^\alpha,\phi_j P_j] \in OPS^{2\alpha-\epsilon}_{1,1-\epsilon}$. Thus, the symbol of this operator, which we denote by $q(x,\xi)$, satisfies
$$\abs{q(x,\xi)} \leq K((1+\abs{\xi}^2)^{1/2})^{2\alpha-\epsilon}.$$
Because of this bound and the localization of frequency, we get, proceeding as above:
$$\abs{\langle[(-\Delta)^\alpha,\phi_jP_j] u , \phi_{Q,j} P_j u \rangle}  \leq K 2^{(2\alpha-\epsilon)j}\abs{\langle u , \phi_{Q,j} P_j u \rangle}.$$
Letting $\tilde{Q}=(1+2^{-\epsilon j})Q$, we conclude using Lemma \ref{commutator1} (to commute Paley-Littlewood projections), Lemma \ref{bumpcommute} (to commute bump functions), the energy dissipation law, and the Cauchy-Schwartz inequality that:
$$\abs{\langle[(-\Delta)^\alpha,\phi_jP_j] u , \phi_{Q,j} P_j u \rangle} \leq  K 2^{(2\alpha-\epsilon)j}\abs{\langle u,\phi_{\tilde{Q},j}\tilde{P}_j \phi_{Q,j}P_j u \rangle} +K2^{-150j}\leq $$
$$\leq K 2^{(2\alpha-\epsilon)j}\|\phi_{\tilde{Q},j}\tilde{P}_j u\|_2\cdot u_{Q}+K2^{-150j} \leq K 2^{(2\alpha-\epsilon)j}\|\phi_{\tilde{Q},j}\tilde{P}_j u\|_2^2+K2^{-150j} .$$
Finally using Lemma \ref{collectbound}, we have
\begin{equation}
\label{Step2}
\abs{\langle[(-\Delta)^\alpha,\phi_jP_j] u , \phi_{Q,j} P_j u \rangle }\leq  K2^{(2\alpha-\epsilon)j}\sum_{Q'\in\mathcal{N}^1(Q')}u_{Q'}^2+K2^{-150j}.\end{equation}
Combining Equations (\ref{32EQN}), (\ref{Step1}), and (\ref{Step2}) yields the desired inequality.
\end{proof}
There is a similar inequality that holds for the whole nuclear family. 
\begin{corollary}
\label{dissipationestimate2}
Let $Q$ be a cube at level $j$. Let $l$ be arbitrary. Then:
$$\sum_{Q'\in\mathcal{N}^l(Q)}\langle(-\Delta)^\alpha u , P_j \phi_{Q',j}^2 P_j u \rangle \quad \geq -K2^{-100j}+K2^{2\alpha j}u_{\mathcal{N}^l(Q)}^2 -K 2^{(2\alpha-\epsilon)j}u_{\mathcal{N}^{l+1}(Q)}^2$$
\end{corollary}
\begin{proof}
The corollary is easily obtained by summing the inequality in Proposition \ref{dissipationestimate} over the cubes in $\mathcal{N}^l(Q)$. We just need to recall that $N(\mathcal{N}^l(Q))\leq 2^{13l}$ and that the exponent of the error term can be made arbitrarily negative.
\end{proof}
The term with $2^{(2\alpha-\epsilon)j}$ is also negligible in a certain sense that we make clear in the next lemma.
\begin{lemma}
\label{almostneg}
There exists $j_0\in \mathbb{Z}$ and $0<l<400/\epsilon$ such that for any $j\geq j_0$, for any arbitrary interval of time $J\subset [0,T]$, and for any cube $Q$ at level $j$, we have:
$$\int_J u^2_{\mathcal{N}^l(Q)} +K2^{-150j} \geq 2^{-\epsilon j} \int_J u^2_{\mathcal{N}^{l+1}(Q)}.$$
\end{lemma}
\begin{proof}
By the energy dissipation law we have $\int_J u^2_{\mathcal{N}^{l+1}(Q)} \leq K$ for a constant that is uniform in choice of $l$, $J$, $Q$ and $j$. In fact, $K$ depends only on the initial vector field and time $T$. Suppose the statement of the lemma is false. Then for any $j_0\in \mathbb{Z}$ and for all $l<400/\epsilon$ there exists a cube $Q$ at level $j\geq j_0$ such that:
$$\int_J u^2_{\mathcal{N}^l(Q)} +K2^{-150j} < 2^{-\epsilon j} \int_J u^2_{\mathcal{N}^{l+1}(Q)}<K2^{-\epsilon j}.$$
Thus, with a simple recursion argument we see that
$$\int_J u^2_{\mathcal{N}^2(Q)} +K2^{-150j} < K 2^{-399j}$$
which is a contradiction, if $j_0$ be large enough.
\end{proof}
\subsection{Estimate for the Nonlinear Term}
The next term we consider from Equation (\ref{COEFFODE}) is 
$$\langle-\hat{B}(u,u),P_j \phi_{Q,j}^2 P_j u \rangle \quad=\quad\langle-\phi_{Q,j}P_j \hat{B}(u,u),\phi_{Q,j}P_j u \rangle .$$
The following scheme for estimating multilinear functions is inspired by the paradifferential calculus of Bony, Coifman, and Meyer. It involves the following partition of frequency space, which is also the basis of \cite{KP}. We proceed similarly and have:
$$P_j\hat{B}(u,u) = H_{j,lh}+H_{j,hl}+H_{j,hh}+H_{j,loc}$$
where
$$\textrm{low-high frequncies:}\quad H_{j,lh} = \sum_{k<j-1000/\epsilon} P_j \hat{B}(P_k u, \tilde{P}_j u)$$
$$\textrm{high-low:}\quad H_{j,hl} = \sum_{k<j-1000/\epsilon} P_j \hat{B}(\tilde{P}_j u, P_k u)$$
$$\textrm{high-high:}\quad H_{j,hh} = \sum_{k>j+1000/\epsilon} P_j \hat{B}(\tilde{P}_k u, P_k u)+\sum_{k>j+1000/\epsilon} P_j \hat{B}(P_k u,\tilde{P}_k u)$$
$$\textrm{local:}\quad H_{j,loc} = \sum_{j-1000/\epsilon<k<j+1000/\epsilon} P_j \hat{B}(\tilde{P}_k u, P_k u)+\sum_{j-1000/\epsilon<k<j+1000/\epsilon} P_j \hat{B}(P_k u, \tilde{P}_k u).$$
For the rest of this subsection we fix a cube $Q$ at level $j$ and introduce the following notation:
$$k<j \Rightarrow Q_k := 2^{(j-k)(1-\epsilon)}Q,\quad k\geq j\Rightarrow Q_k = (1+2^{-\epsilon k/2})Q.$$ 
Also, we choose $\epsilon$ conveniently so that $1000/\epsilon$ is always a (possibly quite large) integer.
We first estimate the low-high frequency terms:
\begin{lemma}
\label{lowhigh}
There exists a constant $K>0$ such that for any arbitrary $\delta>0$ and $j\in\mathbb{Z}^+$ we have:
$$\abs{\langle-\phi_{Q,j}H_{j,lh},\phi_{Q,j}P_j u  \rangle } \leq$$ $$K2^{j(1+3\delta/2+\gamma)}u_{\mathcal{N}^1(Q)}u_Q+K\sum_{k=\delta j}^{j-1000/\epsilon} 2^{\frac{3k}{2}+j +\gamma j}u_{Q_k}u_{\mathcal{N}^1(Q)}u_Q+K2^{-150j}.$$
\end{lemma}
\begin{proof}
We first deal with the case when $\delta j \leq k <j-1000/\epsilon$ and consider a term in the sum $H_{j,lh}$:
$$\langle-\phi_{Q,j}P_j \hat{B}(P_k u, \tilde{P}_j u), \phi_{Q,j}P_j u \rangle .$$
Now by an application of Lemma \ref{bumpcommute}, we know that for any $f\in L^2$:
\begin{equation}\label{firstthing}\|\phi_{Q,j}P_jf-\phi_{Q,j}P_j\phi_{Q_k,k}f\|_2 \leq K2^{-200j}.\end{equation}
Since $B$ is an amenable bilinear operator, we have by assumption that $B^1_v(u)=\hat{B}(u,v)$ and $B^2_v(u)=\hat{B}(v,u)$ are pseudodifferential operators with symbols in some class $S^m_{1,1}$. Thus, by an application of Lemma \ref{quantbound}, we see:
\begin{equation} \label{secondthing} \|\phi_{Q_j,j}\hat{B}(P_ku,\tilde{P}_j u)-\phi_{Q_j,j}\hat{B}(\phi_{Q_k,k}P_k u,\tilde{P}_ju)\|_2\leq K2^{-200j}.\end{equation}
Using Equations (\ref{firstthing}) and (\ref{secondthing}) in tandem gets us:
$$\langle-\phi_{Q,j}P_j \hat{B}(P_k u, \tilde{P}_j u), \phi_{Q,j}P_j u \rangle  \quad =\quad \langle\phi_{Q,j}P_j\hat{B}(\phi_{Q_k,k}P_k u, \tilde{P}_j u),\phi_{Q,j}P_j u \rangle +R_{j,k}$$
where $R_{j,k}$ is an error term with absolute value less than, say, $K2^{-195j}$ for all $k$. In a similar fashion, we can commute bump functions to get:
$$\abs{\langle-\phi_{Q,j}P_j \hat{B}(P_k u,\tilde{P}_j u), \phi_{Q,j}P_j u \rangle } \leq$$
\begin{equation}
\label{est0} \abs{\langle\phi_{Q,j}P_j\hat{B}(\phi_{Q_k,k}P_k u, \phi_{Q_j,j}\tilde{P}_j u),\phi_{Q,j}P_j u \rangle }+K2^{-180j}.\end{equation}
Again since $\phi_{Q,j}P_j$ is a pseudodifferential operator with symbol in the class $S^0_{1,1-\epsilon}$, it is continuous $L^2\to L^2$ by Lemma \ref{L2CONT}, so:
$$\abs{\langle\phi_{Q,j}P_j\hat{B}(\phi_{Q_k,k}P_k u, \phi_{Q_j,j}\tilde{P}_j u),\phi_{Q,j}P_j u \rangle }\leq \|\hat{B}(\phi_{Q_k,k}P_k u, \phi_{Q_j,j}\tilde{P}_j u)\|_2 \cdot u_Q.$$
Since $B$ is an amenable bilinear operator, we use the property in (\ref{amenineq}) and get
$$\abs{\langle\phi_{Q,j}P_j\hat{B}(\phi_{Q_k,k}P_k u, \phi_{Q_j,j}\tilde{P}_j u),\phi_{Q,j}P_j u \rangle }\leq$$ \begin{equation}\label{est} u_Q\cdot \|\phi_{Q_k,k}P_k u\|_{\infty}(\|\nabla\phi_{Q_j,j}\tilde{P}_j u\|_{q(\gamma)}+\|\phi_{Q_j,j}\tilde{P}_j u\|_{q(\gamma)}),\end{equation}
where $q(\gamma):= 6/(3-2\gamma)$. For the rest of this proof we refer to $q(\gamma)$ by $q$ only. Now, we use Lemma \ref{ineq2} to get:
\begin{equation}\label{est1}\|\phi_{Q_k,k}P_k u\|_\infty \leq K 2^{3k/2}u_{Q_k} + K2^{-250k}\end{equation}
\begin{equation}\label{est2}\|\phi_{Q_j,j}\tilde{P}_j u\|_q \leq K2^{3j(1/2-1/q)}u_{\mathcal{N}^1(Q)}+K2^{-250j}=K2^{\gamma j}u_{\mathcal{N}^1(Q)}+K2^{-250j}.\end{equation}
To deal with the gradient term, we first see (where we use $\tilde{P}_j^2=\tilde{P}_j$ as well):
$$\nabla(\phi_{Q_j,j}\tilde{P}_j u ) =\phi_{Q_j,j}\nabla\tilde{P}_j u +(\nabla\phi_{Q_j,j})\tilde{P}_j u=$$ $$= \phi_{Q_j,j}\nabla\tilde{P}_j u +(\nabla \phi_{Q_j,j})\tilde{P}_j \phi_{Q,j}\tilde{P}_ju + (\nabla \phi_{Q_j,j})\tilde{P}_j (1-\phi_{Q,j})\tilde{P}_ju.$$
Since $\nabla\phi_{Q_j,j}$ and $(1-\phi_{Q,j})$ have disjoint supports, the last operator $(\nabla \phi_{Q_j,j})\tilde{P}_j (1-\phi_{Q_j,j})$ is smoothing and will have negligible $L^q$ norm. The middle operator $(\nabla \phi_{Q_j,j}) \tilde{P}_j $ has symbol in $S^{-\epsilon}_{1,1-\epsilon}$ by Lemma \ref{pseudoproduct} (where we treat $\nabla \phi_{Q_j,j}$ as any other smooth function with compact support). Lastly, we can use Lemma \ref{finiteband1} and the fact that $\tilde{P}_j \tilde{P}_j = \tilde{P}_j$ to conclude:
$$
\|\nabla(\phi_{Q,j}\tilde{P}_j u )\|_q \leq 2^j \|\phi_{Q,j}\tilde{P}_j u\|_q +2^{-\epsilon j}\|\phi_{Q,j}\tilde{P}_j u\|_q +K2^{-200j}$$
Thus, since $\epsilon>0$, we have:
\begin{equation}\label{est3}\|\nabla(\phi_{Q,j}\tilde{P}_j u )\|_q \leq 2^{j+\gamma j}u_{\mathcal{N}^1(Q)}+K2^{-200j}.\end{equation}
Combining our estimates in (\ref{est0}), (\ref{est}),(\ref{est1}), (\ref{est2}),(\ref{est3}), we get:
$$\abs{\langle-\phi_{Q,j}P_j \hat{B}(P_k u,\tilde{P}_j u), \phi_{Q,j}P_j u \rangle } \leq 2^{3k/2+j+\gamma j}u_{Q_k}u_{\mathcal{N}^1(Q)}u_Q +K2^{-150j}.$$
Summing over $\delta j\leq k<j-1000/\epsilon$ and recognizing that there are only around $j$ terms gets us the first term with the sum.
For the case when $k<\delta j$, we directly use the inequality satisfied by $\hat{B}$, our commutator estimates from Section 5, and the energy dissipation law to get:
$$\abs{\langle-\phi_{Q,j}P_j \hat{B}(P_k u,\tilde{P}_j u), \phi_{Q,j}P_j u \rangle } \leq $$
$$\leq2^j u_Q \|P_k u\|_\infty \|\phi_{Q,j}\tilde{P}_ju\|_q+K2^{-150j}\leq 2^{j+3\delta j/2+\gamma j}u_Q u_{\mathcal{N}^1(Q)}+K2^{-150j},$$
which finishes the proof.
\end{proof}
By symmetry of $\hat{B}$, the same result holds for the high-low terms:
\begin{lemma}
\label{highlow}
There exists a constant $K>0$ such that for any arbitrary $\delta>0$ and $j\in\mathbb{Z}$ we have:
$$\abs{\langle-\phi_{Q,j}H_{j,hl},\phi_{Q,j}P_j u  \rangle } \leq$$ $$ K2^{j(1+3\delta/2+\gamma)}u_{\mathcal{N}^1(Q)}u_Q+K\sum_{k=\delta j}^{j-1000/\epsilon} 2^{\frac{3k}{2}+j +\gamma j}u_{Q_k}u_{\mathcal{N}^1(Q)}u_Q+K2^{-150j}.$$
\end{lemma}
Our next effort is towards estimating the high-high frequency terms. 

\begin{lemma}
\label{highhigh}
For some constant $K$ independent of $j$ we have:
$$\abs{\langle \phi_{Q,j}H_{j, hh} , \phi_{Q,j}P_j u\rangle }\leq K \sum_{k>j+1000/\epsilon}  2^{\frac{149j}{100}+\gamma j +\frac{101k}{100}}u_Q\|\phi_{Q_j,j}P_k u\|_2^2 + K2^{-150j}.$$
\end{lemma}
\begin{proof}
As usual, we examine and wish to estimate a single term in the high-high sum where $k>j+1000/\epsilon$, which by symmetry we may write:
$$\langle \phi_{Q,j}P_j\hat{B}(\tilde{P}_k u, P_k u), \phi_{Q,j}P_j u\rangle.$$
As in the proof of Lemma \ref{lowhigh}, we can commute bump functions across our operators to get
$$\abs{\langle \phi_{Q,j}P_j\hat{B}(\tilde{P}_k u, P_k u), \phi_{Q,j}P_j u\rangle}\leq$$ \begin{equation}
\label{bound0} K\abs{\langle \phi_{Q,j}P_j\hat{B}(\phi_{Q_j,j}\tilde{P}_k u, \phi_{Q_j,j}P_k u), \phi_{Q,j}P_j u\rangle}+K2^{-200j}.\end{equation}
Since $B$ is amenable we can bound the trilinear expression:
$$
\abs{\langle \phi_{Q,j}P_j\hat{B}(\phi_{Q_j,j}\tilde{P}_k u, \phi_{Q_j,j}P_k u), \phi_{Q,j}P_j u\rangle}\leq$$ $$ (\|\phi_{Q_j,j}\tilde{P}_k u\|_{p_1}+\|\nabla\phi_{Q_j,j}\tilde{P}_k u\|_{p_1})\|\phi_{Q_j,j}P_k u\|_{p_2} \|\phi_{Q,j}P_j u\|_r 
$$
where
$$\frac{1}{p_1}+\frac{1}{p_2}+\frac{\gamma}{3}+\frac{1}{r}=1.$$
Henceforth, we can always have $0<\gamma<1/100$. Assuming this, we can pick $p_1,p_2,r$ so that
$$\frac{1}{p_1}=\frac{1}{2},\quad \frac{1}{p_2} = \frac{1}{2}-\frac{1}{300},\quad \frac{1}{r}= \frac{1}{300}-\frac{\gamma}{3}$$
By Lemma \ref{ineq2} we know that
\begin{equation}\label{bound1}
\|\phi_{Q,j}P_j u\|_r \leq 2^{\frac{149j}{100}+\gamma j}u_Q +K2^{-200j}\end{equation}
and
\begin{equation}\label{bound2}
\|\phi_{Q_j,j}P_k u\|_{p_2} \leq 2^{\frac{k}{100}}\|\phi_{Q_j,j}P_k\|_2+K2^{-200j} .\end{equation}
By the same reasoning as in the proof of Lemma \ref{lowhigh}, we get
\begin{equation}\label{bound3}
\|\nabla \phi_{Q_j,j}P_k u\|_2 \leq K2^k \sum_{l=-2}^{l=2} \|\phi_{Q_j,j}P_{k+l}u\|_2
\end{equation}
Combining our estimates (\ref{bound0}), (\ref{bound1}), (\ref{bound2}), (\ref{bound3}) yields:
$$\abs{\langle \phi_{Q,j}H_{j, hh} , \phi_{Q,j}P_j u\rangle }\leq K \sum_{k>j+1000/\epsilon} u_Q 2^{\frac{149j}{100}+\gamma j +\frac{101k}{100}}\|\phi_{Q_j,j}P_k u\|_2^2 + K2^{-150j}.$$
\end{proof}
Next we deal with the "local" frequencies around level $j$:
\begin{lemma}
\label{local}
For some constant $K$ independent of $j$ we have:
$$\abs{\langle \phi_{Q,j}H_{j, loc} , \phi_{Q,j}P_j u\rangle }\leq K2^{5j/2+\gamma j}u_Q u^2_{\mathcal{N}^{1000/\epsilon}(Q)}+K2^{-150j}.$$
\end{lemma}
\begin{proof}
We can explicitly write out the term we wish to bound by
$$\langle \phi_{Q,j}H_{j, loc} , \phi_{Q,j}P_j u\rangle = 2\sum_{l=-2}^{l=2}\sum_{j-1000/\epsilon<k}^{k<j+1000/\epsilon}\langle \phi_{Q,j}P_j\hat{B}(P_{k+l}u,P_ku),\phi_{Q,j}P_j u\rangle.$$
As in the proof of Lemma \ref{lowhigh}, we can use the triangle inequality and commute bump functions to get:
$$
\abs{\langle \phi_{Q,j}H_{j, loc} , \phi_{Q,j}P_j u\rangle} \leq$$ \begin{equation}
\label{bound01} K2^{-200j} + K\sum_{l=-2}^{l=2}\sum_{j-1000/\epsilon<k}^{k<j+1000/\epsilon}\abs{\langle \phi_{Q,j}P_j\hat{B}(\phi_{Q_j,j}P_{k+l}u,\phi_{Q_j,j}P_ku),\phi_{Q,j}P_j u\rangle}
\end{equation}
For appropriate choices of $k=k_0,l=l_0$ that maximize the expression in the sum in Equation (\ref{bound01}), and since our constants $K$ are allowed to depend on $\epsilon$, we have:
$$
K\sum_{l=-2}^{l=2}\sum_{j-1000/\epsilon<k<j+1000/\epsilon}\abs{\langle \phi_{Q,j}P_j\hat{B}(\phi_{Q_j,j}P_{k+l}u,\phi_{Q_j,j}P_ku),\phi_{Q,j}P_j u\rangle}\leq $$ \begin{equation}\label{bound05} \leq K\abs{\langle \phi_{Q,j}P_j\hat{B}(\phi_{Q_j,j}P_{k_0+l_0}u,\phi_{Q_j,j}P_{k_0}u),\phi_{Q,j}P_j u\rangle}
\end{equation}
We remark that by Lemma \ref{collectbound}:
\begin{equation}
\label{bound02}
\|\phi_{Q_j,j}P_k u\|_2 \leq u_{\mathcal{N}^{1000/\epsilon}(Q)}
\end{equation}
and by Lemma \ref{ineq2}:
\begin{equation}
\label{bound03}
\|\phi_{Q,j}P_j u\|_\infty \leq 2^{3j/2}u_Q +K2^{-200j}.
\end{equation}
As in the proof of Lemma \ref{lowhigh}, we have, where $q= 6/(3-2\gamma)$:
\begin{equation}
\label{bound04}
\|\nabla(\phi_{Q_j,j}P_{k_0+l_0} u )\|_q \leq 2^{j+\gamma j}u_{\mathcal{N}^{1000/\epsilon}(Q)}+K2^{-200j}.
\end{equation}
Combining our estimates in (\ref{bound01}), (\ref{bound05}), (\ref{bound02}), (\ref{bound03}), (\ref{bound04}), we get:
$$\abs{\langle \phi_{Q,j}H_{j, loc} , \phi_{Q,j}P_j u\rangle} \leq K2^{5j/2+\gamma j}u_Q u^2_{\mathcal{N}^{1000/\epsilon}(Q)}+K2^{-150j}.$$
\end{proof}

\section{Partial Regularity and Blowup}
\subsection{Flexibility of Blowup}
For the next two subsections, we assume the reader has a familiarity with the paper \cite{T} of Tao. We first give an overview of the construction in \cite{T}. Afterwards, we show the flexibility of his result. Indeed, Tao notes himself (in footnote 12 of \cite{T}), that his methods show blowup even when hyperdissipation approaches the critical value of $\alpha=5/4$. In this subsection we give some details showing why this is the case.

As before, we let $\langle, \rangle $ be the $L^2$ pairing for vector fields. Let $1<\lambda<2$. We denote by $H^{10}_{df}(\mathbb{R}^3)$ the space of vector fields on $\mathbb{R}^3$ that are divergence free in the distributional sense and whose first ten weak derivatives are square integrable.

 A basic local cascade operator is a bilinear operator $C: H_{df}^{10}(\mathbb{R}^3) \times H_{df}^{10}(\mathbb{R}^3)\to H_{df}^{10}(\mathbb{R}^3)^\ast$ defined via duality by:
$$\langle C(u,v), w \rangle \hspace{5px} =\sum_{n\in \mathbb{Z}}\lambda^{5n/2} \langle u, \psi_{1,n} \rangle \langle v, \psi_{2,n} \rangle  \langle w, \psi_{3,n} \rangle $$
where for $i\in \{1,2,3\}$,
$$\psi_{i,n}(x) = \lambda^{3n/2}\psi_i(\lambda^n x)$$
and $\psi_i$ is a Schwartz vector field whose Fourier transform is compactly supported within a small annulus around the unit sphere in $\mathbb{R}^3$. We also define a local cascade operator to be a finite linear combination of basic local cascade operators. Lastly, we call a basic local cascade operator a zero-momentum basic local cascade operator if its constituent Schwartz vector fields satisfy
$$\int_{\mathbb{R}^3} \psi_i dx=0.$$
Likewise, we may define a zero-momentum local cascade operator.

In \cite{T}, Tao proves the existence of a local cascade equation that admits a blowup solution. In particular, Tao proves the following:

\begin{theorem}[\cite{T}]
Let $1<\lambda<2$ be arbitrary. Then there exists a symmetric local cascade operator $C$ and a Schwartz divergence-free vector field $u_0$ such that the cancellation identity holds
\begin{equation}
\langle C(u,u),u \rangle =0 \textrm{ for all } u \in H^{10}_{df}(\mathbb{R}^3)\\
\end{equation}
and there does not exist any global mild solution $u: [0,\infty) \to H^{10}_{df}(\mathbb{R}^3)$ to the initial value problem
\begin{equation}
\begin{split}
\partial_t u -\Delta u +C(u,u)=0\\
u(\cdot,t=0) = u_0
\end{split}
\end{equation}
\end{theorem}

In this subsection we show that Tao's result is flexible in the sense that we find a local cascade operator with more restraints on the constituent Schwartz vector fields $\psi_i$ whose corresponding local cascade equation admits a blowup solution with hyperdissipation. In particular we prove:

\begin{theorem}
\label{BLOWUP}
Let $1<\lambda<2$ and $0<\alpha<5/4$ be arbitrary. Then there exists a zero-momentum symmetric local cascade operator $C$ and a Schwartz divergence-free vector field $u_0$ such that the cancellation identity holds
\begin{equation}
\langle C(u,u),u \rangle =0 \textrm{ for all } u \in H^{10}_{df}(\mathbb{R}^3)\\
\end{equation}
and there does not exist any global mild solution $u: [0,\infty) \to H^{10}_{df}(\mathbb{R}^3)$ to the initial value problem
\begin{equation}
\begin{split}
\partial_t u +(-\Delta)^\alpha u +C(u,u)=0\\
u(\cdot,t=0) = u_0
\end{split}
\end{equation}
\end{theorem}

First, we determine the structure our local cascade operator has to take in order to fulfill some basic requirements like the cancellation property. In particular, following Tao, we consider four balls $B_1, B_2, B_3, B_4$ in the annulus $\{ \xi : 1<\abs{\xi}\leq \frac{\lambda+1}{2}\}$, and we choose the balls so that the collection $\{ B_i, -B_i\}$ is mutually disjoint. We choose four zero-momentum divergence-free Schwartz vector fields $\psi_i$ such that $\mathcal{F}(\psi_i)$ has support in $B_i \cup -B_i$, and we normalize them so that $\| \psi_i\|_2 =1$. We have the $L^2$-rescaled functions $\psi_{i,n}:= \lambda^{3n/2}\psi_i(\lambda^n x)$ that also satisfy $\| \psi_{i,n}\|_2=1$. Let $S=\{(0,0,0),(1,0,0),(0,1,0),(0,0,1)\}$. We define the following local cascade operator (recall, a linear combination of basic local cascade operators):
$$C(u,v):= \sum_{n\in \mathbb{Z}} \sum_{\substack{\hspace{5px}i_1,i_2,i_3\in \{1,2,3,4\} \\(\mu_1,\mu_2,\mu_3)\in S} }a_{i_1,i_2,i_3,\mu_1,\mu_2,\mu_3}\lambda^{5n/2}\langle u,\psi_{i_1,n+\mu_1} \rangle \langle v,\psi_{i_2,n+\mu_2} \rangle \psi_{i_3,n+\mu_3}$$
To ensure that $C$ is a symmetric bilinear operator we require that
$$a_{i_1,i_2,i_3,\mu_1,\mu_2,\mu_3}=a_{i_2,i_1,i_3,\mu_2,\mu_1,\mu_3}.$$
The cancellation property is satisfied if we require
$$\sum_{\{a,b,c\}=\{1,2,3\}} a_{i_a,i_b,i_c,\mu_a,\mu_b,\mu_c}=0.$$
With these first conditions imposed on $C$, we are ready to study the basic dynamics of the pseudodifferential equation associated to $C$. 
Whenever applicable, we define the following Fourier projections that will act as "wavelets":
$$u_{i,n}(t)(x) := \mathcal{F}^{-1}(\chi_{\xi\in \lambda^n B_i \cup -B_i}(\xi)\cdot \mathcal{F}(u(t))(\xi))$$
and the following functions of time that behave like "wavelet coefficients":
$$X_{i,n}(t):= \langle u(t),\psi_{i,n} \rangle =\langle u_{i,n}(t),\psi_{i,n} \rangle .$$
Lastly, we define the following "local energies":
$$E_{i,n}(t):= \frac{1}{2}\|u_{i,n}(t)\|_2.$$
The behavior of the differential equation is determined by the dynamics of the functions $X_{i,n}(t)$, which is described in Lemma 4.1 of \cite{T}. Below, we only note the modifications in the case of fractional dissipation (as assuming zero-momentum $\psi$ has no effect on the lemma's conclusions).
\begin{lemma}
\label{devep}
Suppose $u$ is a global mild solution to the pseudodifferential equation associated to $C$ with initial data $\psi_{1,n_0}$ for some $n_0$ integer.
For any $i,n$ we have
$$\partial_t X_{i,n} = \sum_{i_1,i_2 \in \{1,2,3,4\}}\sum_{(\mu_1,\mu_2,\mu_3)\in S} a_{i_1,i_2,i,\mu_1,\mu_2,\mu_3}\lambda^{5(n-\mu_3)/2}X_{i_1,n-\mu_3+\mu_1}X_{i_2,n-\mu_3+\mu_2} + O(\lambda^{2\alpha n}E_{i,n}^2)$$
and
$$\frac{1}{2}X_{i,n}^2(t) \leq E_{i,n}(t) \leq \frac{1}{2}X_{i,n}^2(t) + O\bigg(\lambda^{2\alpha n} \int_0^t E_{i,n}(s)ds\bigg).$$
\end{lemma}
\begin{proof}
The vector field $u(t)$ is a mild solution of the pseudodifferential equation associated to $C$. Thus we write
\begin{equation}\label{mild}u(t) = e^{-t(-\Delta)^\alpha}u_0 +\int_0^t e^{(s-t)(-\Delta)^\alpha}C(u(s),u(s))ds.\end{equation}
As noted before, we have the "wavelet" decomposition of $u$:
$$u(t) = \sum_{n,i} u_{i,n}(t)$$
with
$$X_{i,n}(t) = \langle u(t),\psi_{i,n}\rangle = \langle u_{i,n}(t), \psi_{i,n}\rangle.$$
We project Equation (\ref{mild}) onto $\psi_{i,n}$ and get
$$
u_{i,n}(t) = e^{-t(-\Delta)^\alpha}X_{i,n}(0)\psi_{i,n}+  \psi_{i,n}\bigg(\sum_{(i_1,i_2,i,\mu_1,\mu_2,\mu_3)\in S}  a_{i_1,i_2,i,\mu_1,\mu_2,\mu_3}\lambda^{5(n-\mu_3)/2}$$
$$\int_0^t\langle u,\psi_{i_1,n+\mu_1-\mu_3} \rangle \langle u,\psi_{i_2,n+\mu_2-\mu_3} \rangle \bigg).
$$
If we differentiate with respect to the time variable $t$ and simplify we obtain
\begin{equation}\label{mildcoeff}
\partial_t u_{i,n} = -(-\Delta)^\alpha u_{i,n}+ \sum_{(i_1,i_2,i,\mu_1,\mu_2,\mu_3)\in S}  a_{i_1,i_2,i,\mu_1,\mu_2,\mu_3}\lambda^{5(n-\mu_3)/2} X_{i_1,n-\mu_3+\mu_1} X_{i_2,n-\mu_3+\mu_2}\psi_{i,n}
\end{equation}
Now we observe that (where $K$ depends only on $\psi_i$):
\begin{equation}\label{IBP}\abs{\langle -(-\Delta)^\alpha u_{i,n}, \psi_{i,n} \rangle} =\abs{ \langle u_{i,n} , -(-\Delta)^\alpha\psi_{i,n}\rangle} \leq K E_{i,n}^{1/2}(t)\lambda^{2\alpha n }.\end{equation}
Pairing (\ref{mildcoeff}) with $\psi_{i,n}$ and using (\ref{IBP}) gets us the first statement in the lemma.
The first inequality in the second statement follows from Cauchy-Schwartz. It remains to prove the second inequality. We recall the "local energy inequality" found by Tao, which is independent of the dissipation parameter $\alpha$. Equation (4.11) in \cite{T} is 
\begin{equation}\label{LEI}
\partial_t E_{i,n} \leq \sum_{i_1,i_2}\sum_{(\mu_1,\mu_2,\mu_3)\in S} a_{i_1,i_2,i,\mu_1,\mu_2,\mu_3} \lambda^{5(n-\mu_3)/2}X_{i_1,n-\mu_3+\mu_1}X_{i_2,n-\mu_3+\mu_1}X_{i,n}.
\end{equation}
It is now clear that if we multiply the evolution inequality for $X_{i,n}$ by $X_{i,n}$, subtract from (\ref{LEI}), and integrate in time we have the second desired inequality
$$E_{i,n}(t) \leq \frac{1}{2}X_{i,n}^2(t) +K\lambda^{2\alpha n} \int_0^t E_{i,n}(s)ds.$$
\end{proof}
The only significant change from the work in \cite{T} is the presence of $\alpha$ in the exponent of $\lambda$ in the dissipation term, which is in any case perturbative in the regime of Tao's blowup construction.

Tao's construction of a blowup solution to the pseudodifferential equation associated to $C$ continues with an intricate analysis of an infinite dimensional ODE system whose dynamics are contained within the dynamics of $C$. Tao proves that there exist a sequence of times $t_n$ converging to some finite $T$ so that for some $\epsilon_0>0$, at least $\lambda^{-\epsilon_0 n}$ of the energy is concentrated in the ball of radius $\lambda^{-n}$ around the spatial origin. 

 An essential part of the blowup construction is that the aforementioned concentration of energy occurs on timescales of order $t_{n+1}-t_n= \lambda^{(-\frac{5}{2}+O(\epsilon_0))n}$ which are signicantly smaller than the dissipation time scale which (for fractional dissipation of order $\alpha$) is $\lambda^{2\alpha n}$ by Lemma \ref{devep}. Obviously this can occur so long as $\alpha<5/4$. Since Tao never uses any assumption about the integral of the functions $\psi_{i,n}$, one can follow through Tao's construction and get blowup for a zero-momentum cascade operator for a pseudodifferential equation with hyperdissipation with $\alpha<5/4$. This is nothing less than Theorem \ref{BLOWUP} above.

\subsection{Additional Properties of Local Cascade Operators}
We have established Theorem \ref{BLOWUP}, where we constructed a zero-momentum, symmetric, local cascade operator $C(u,v)$ satisfying the cancellation identity for divergence-free vector fields whose associated $\alpha$-dissipative pseudodifferential equation has a solution blowing up in finite time from Schwartz initial data.  In Section 2, we proved a partial regularity result for $\alpha$-dissipative pseudodifferential equations with nonlinearities arising from amenable bilinear operators. It follows that if we desire to use Theorem \ref{PRTHM} and get a partial regularity result for the pseudodifferential equation from Theorem \ref{BLOWUP}, we simply have to verify that $C(u,v)$ is an amenable bilinear operator. 

That $C(u,v)$ is defined for Schwartz vector fields is obvious. That $C(u,v)$ satisfies the cancellation identity 
$$\langle\mathbb{P}C(u,u), u \rangle =0\quad \textrm{ for divergence-free } u$$
is equally as obvious. Indeed, since $C(u,u)$ is always divergence-free, we have $\mathbb{P}C(u,u)=C(u,u)$. It follows that
$$\langle \mathbb{P}C(u,u),u\rangle=\langle C(u,u),u\rangle=0 \textrm{ for all divergence-free } u.$$
The scaling property with the choice of $\lambda$ is also satisfied by $C$, and this is easy to check. Indeed, Tao also makes this observation. Thus, it remains to show that $C^1_v(u)=\mathbb{P}C(u,v)$ is a pseudodifferential operator in some class $S^m_{1,1}$ and the bilinear operator bound. We begin by showing the latter, but first we need an elementary lemma.

\begin{lemma}
\label{momentuse}
Let $\psi$ be a Schwartz divergence-free vector field that has zero momentum, i.e. $\int \psi =0$. Let $\Psi_i$ be the vector field given by Lemma \ref{DivF} so that $\textrm{div } \Psi_i = \psi_i$, the $i^{th}$ component of $\psi$. Let $\psi_{n}:= \lambda^{3n/2}\psi(\lambda^n x)$ be the $L^2$ rescaling of $\psi$ and likewise for all $\Psi_{i,n}$. Then for any vector field $v\in H^{10}_{df}$, we have:
$$\lambda^n \langle \psi_n, v\rangle =-\sum_i \int \Psi_{i,n}\cdot \nabla v_i $$
\end{lemma}
\begin{proof}
First we use Lemma \ref{DivF} and notice that:
$$\lambda^n \psi_{i,n}(x)= \lambda^{5n/2}\psi_i (\lambda^n x) = \lambda^{5n/2}(\textrm{div } \Psi_i)(\lambda^n x)=\lambda^{3n/2} \textrm{div }(\Psi_i(\lambda^n x))= \textrm{div } \Psi_{i,n}(x).$$
Then we use a simple integration by parts:
$$\lambda^n\langle \psi_n, v\rangle = \sum_i \int \lambda^{5n/2}\psi_{i}(\lambda^n x)v_i dx = \sum_i \int v_i \textrm{div } \Psi_{i,n}dx =-\sum_i \int \Psi_{i,n} \cdot \nabla v_i dx$$
\end{proof}
\begin{proposition}
Let $C(u,v)$ be a zero-momentum, symmetric local cascade operator. Then, for arbitrary $1>\gamma>0$ we have for any $1\leq p_1, p_2, r\leq \infty$ satisfying
$$\frac{1}{p_1}+\frac{1}{p_2}+\frac{\gamma}{3} = \frac{1}{r}.$$
that the following inequality holds:
$$\| C(u,v) \|_{L^r} \leq K \| u\|_{L^{p_1}}(\|v\|_{L^{p_2}} +\| \nabla v\|_{L^{p_2}})$$
where $K$ is some constant depending on $\gamma,\lambda$, and the vector fields $\psi_i$ used in the definition of $C(u,v)$.
\end{proposition}
\begin{proof}
It suffices to prove the proposition for any (zero-momentum, symmetric) basic local cascade operator:
$$C(u,v)=\sum_{n\in \mathbb{Z}}\lambda^{5n/2} \langle u, \psi_{1,n} \rangle \langle v, \psi_{2,n} \rangle  \psi_{3,n}(x).$$
Since the sum above is absolutely convergent, we can divide the operator $C$ into two parts:
$$C^+(u,v) := \sum_{n\geq 0}\lambda^{5n/2} \langle u, \psi_{1,n} \rangle \langle v, \psi_{2,n} \rangle  \psi_{3,n}(x)$$
and $C^-(u,v):= C(u,v) - C^+(u,v)$.
By a change of variables we observe that
\begin{equation}\label{57}\| \psi_{3,n} \|_{L^r} = \lambda^{3n/2} \| \psi_3(\lambda^n x)\|_{L^r} = \lambda^{3n/2-3n/r}\| \psi_3(u)\|_{L^r}=\lambda^{3n/2-3n/r}\| \psi_3\|_{L^r}.\end{equation}
By the absolute convergence of the sum, we may bound each component in turn. We begin with $C^-$, which is the more straightforward term. We observe that
$$\| C^-(u,v)\|_{L^r} =\| \sum_{n< 0}\lambda^{5n/2} \langle u, \psi_{1,n} \rangle \langle v, \psi_{2,n} \rangle  \psi_{3,n}(x)\|_{L^r}\leq$$
$$\leq \sum_{n< 0}\abs{\lambda^{5n/2} \langle u, \psi_{1,n} \rangle \langle v, \psi_{2,n} \rangle }\cdot \|\psi_{3,n}(x)\|_{L^r}\leq$$
$$\leq \|\psi_3\|_{L^r}\sum_{n<0} \lambda^{4n-3n/r}\abs{\langle u,\psi_{1,n} \rangle }\abs{\langle v,\psi_{2,n} \rangle }\leq$$
$$\leq \|\psi_3\|_{L^r}\sum_{n<0} \lambda^{4n-3n/r}\|u\|_{L^{p_1}}\|\psi_{1,n}\|_{L^{p_1'}}\|v\|_{L^{p_2}}\|\psi_{2,n}\|_{L^{p_2'}}$$
where the first inequality is the triangle inequality, the second is due to Equation (\ref{57}), and the third inequality is Holder's inequality, where $p'$ denotes the Holder conjugate exponent of a number $p$. Then, by the same reasoning that justifies Equation (\ref{57}), we can perform a change of variables in the integration and get
$$\|C^-(u,v)\|_{L^r} \leq \|\psi_3\|_{L^r}\|\psi_{1}\|_{L^{p_1'}}\|\psi_{2}\|_{L^{p_2'}}\sum_{n<0} \lambda^{7n-3n(1/r+1/p_1'+1/p_2')}\|u\|_{L^{p_1}}\|v\|_{L^{p_2}}.$$
Now by the hypothesis on our exponents, it follows that
$$\frac{1}{r}+\frac{1}{p_1'}+\frac{1}{p_2'}=\frac{1}{r}+2-\frac{1}{p_1}-\frac{1}{p_2}=2+\frac{\gamma}{3}.$$
Thus, we conclude, since $0<\gamma<1$ and $\lambda>1$:
$$\|C^-(u,v)\|_{L^r} \leq \|\psi_3\|_{L^r}\|\psi_{1}\|_{L^{p_1'}}\|\psi_{2}\|_{L^{p_2'}}\sum_{n<0}\lambda^{n(1-\gamma)}\|u\|_{L^{p_1}}\|v\|_{L^{p_2}}\leq K \|u\|_{L^{p_1}}\|v\|_{L^{p_2}}$$
where $K$ depends only on $\lambda, \gamma, \psi_i$. As with $C^-$ we can use Holder's inequality, a change of variable in the integration, and Lemma \ref{momentuse} to exchange $\lambda^n$ for a derivative to get:
$$\| C^+(u,v)\|_{L^r}\leq  \|\psi_3\|_{L^r}\|\psi_{1}\|_{L^{p_1'}}\sup_i \|\Psi_{2,i}\|_{L^{p_2'}}\sum_{n\geq 0} \lambda^{-\gamma n}\|u\|_{L^{p_1}}\|\nabla v\|_{L^{p_2}}$$
where $\textrm{div } \Psi_{2,i} = \psi_{2,i}$. This is where we use the zero-momentum condition.
Because $\gamma>0$ we conclude that
$$\| C^+(u,v)\|_{L^r}\leq  K\|u\|_{L^{p_1}}\|\nabla v\|_{L^{p_2}}$$
where $K$ depends only on $\lambda, \gamma, \psi_i$. Combining our estimates with one final application of the triangle inequality gets us
$$\| C(u,v)\|_{L^r}\leq  K\|u\|_{L^{p_1}}(\|v\|_{L^{p_2}}+\|\nabla v\|_{L^{p_2}})$$
where $K$ depends only on $\lambda, \gamma, \psi_i$. 
\end{proof}
To finish showing that $C(u,v)$ is an amenable bilinear operator, it remains to prove that $C^1_v(u)=\mathbb{P}C(u,v)$ is a pseudodifferential operator in some class $S^m_{1,1}$. That $C^2_v(u)$ is also in the same symbol class would follow immediately from the symmetry of $C$. 
\begin{proposition}
Let $C(u,v)$ be a symmetric local cascade operator, then $C^1_v(u)$ is a pseudodifferential operator with symbol in the class $S^{5/2}_{1,1}$.
\end{proposition}
\begin{proof}
It suffices to prove the proposition for basic local cascade operators of the form:
$$C(u,v)=\sum_{n\in \mathbb{Z}}\lambda^{5n/2} \langle u, \psi_{1,n} \rangle \langle v, \psi_{2,n} \rangle  \psi_{3,n}(x)$$
where we only deal with scalar functions. Using the fact that the Fourier transform is a unitary operator on $L^2$ and the absolute convergence of the sum, we write
$$C^1_v(u) = \iint \sum_{n\in \mathbb{Z}} \lambda^{5n/2} \psi_{3,n}(x) \langle v,  \psi_{2,n}\rangle\mathcal{F}(\psi_{2,n})(\xi) \mathcal{F}(u)(\xi) d\xi.$$
Expanding this and putting it into the form of a pseudodifferential operator we write (up to some dimensional constant):
$$C^1_v(u) = \iint \bigg(\sum_{n\in \mathbb{Z}} \lambda^{5n/2} \psi_{3}(\lambda^n x) \langle v,  \psi_{2,n}\rangle\mathcal{F}(\psi_{2})(\lambda^{-n}\xi)e^{-i x\cdot \xi} \bigg) \mathcal{F}(u)(\xi)e^{i x\cdot \xi} d\xi.$$
The infinite sum, however, has only one nonzero term for any given choice of $\xi$, because $\mathcal{F}(\psi_2)$ has nicely chosen compact support. In particular, the only nonzero term for any given $\xi$ is the $n^{th}$ term where $\abs{\xi} \sim \lambda^{n}$. It follows readily then that the symbol of the pseudodifferential operator $C^1_v$, the "infinite" sum in the expression above, is in fact in the symbol class $S^{5/2}_{1,1}$. 
\end{proof}
As a consequence of the two previous propositions as well as Theorems \ref{BLOWUP} and \ref{PRTHM}, we have:
\begin{theorem}
\label{MAIN}
Let $C(u,v)$ be the bilinear operator from Theorem \ref{BLOWUP}. Let $T$ be the blowup time to the associated pseudodifferential equation with $3/4<\alpha<5/4$.  Then the closed set $S_T$ has Hausdorff dimension at most $5-4\alpha$.
\end{theorem}
We recall here the following general theorem about local cascade operators proven by Tao in \cite{T}.
\begin{theorem}[\cite{T}]
Let $\lambda_0>1$ be an absolute constant sufficiently close to $1$. Then every local cascade operator (not necessarily zero-momentum) is an averaged Euler bilinear operator.
\end{theorem}
Thanks to Tao's general theorem about local cascade operators (Theorem 3.2 in \cite{T}) we have the following corollary of Theorem \ref{MAIN} by taking $\lambda$ near enough to $1$:
\begin{corollary}
\label{MAIN2}
Let $3/4<\alpha<5/4$ be arbitrary.
There exists a symmetric averaged Euler bilinear operator $B$ obeying the cancellation identity for $u\in H^{10}_{df}(\mathbb{R}^3)$ and a Schwartz divergence-free vector initial vector field so that there is no global-in-time solution to the associated $\alpha$-dissipative pseudodifferential equation Moreover if $T$ is the time of first blowup, then the closed set $S_T$ has Hausdorff dimension at most $5-4\alpha$.
\end{corollary}

\section{Technical Toolbox}
Throughout this subsection we shall work in the general Euclidean space $\mathbb{R}^n$. All our spaces will be of functions with domain $\mathbb{R}^n$, unless otherwise stated.
Our shorthand for the $L^p$ norm will be $\|\cdot \|_p$, and we denote the Fourier transform by $\mathcal{F}$. We denote the fractional (inhomogeneous) Sobolev space of order $s$ and integrability $p$ by:
$$W^{s,p}(\mathbb{R}^n):= \bigg\{f\in L^p(\mathbb{R}^n) : \mathcal{F}^{-1}\big((1+\abs{\xi}^2)^{\frac{s}{2}}\mathcal{F}(f)(\xi)\big)\in L^p(\mathbb{R}^n)\bigg\}.$$
Lastly, we denote $W^{s,2}(\mathbb{R}^n)$ by $H^s(\mathbb{R}^n)$ and use the shorthand $\|\cdot \|_s$ for the norm of $H^s(\mathbb{R}^n)$. Since we almost always deal with the spaces $H^s(\mathbb{R}^n)$, the use of this shorthand will always be clear in context.
The following embedding theorem is standard. A proof may be found in \cite{BCD}.
\begin{lemma}[Theorem 1.66 in \cite{BCD}]
\label{EMBED}
The space $H^s(\mathbb{R}^n)$ embeds continuously in the H\"{o}lder Space $C^{k,\alpha}(\mathbb{R}^n)$ provided that $s\geq n/2 +k+ \alpha$.
\end{lemma}
\subsection{Pseudodifferential Operators}
 A pseudodifferential operator is an operator $P$ on Schwartz functions $u$ with the property that:
$$P(x,D)u := (2\pi)^{-n/2}\int e^{i x\cdot \xi}p(x,\xi)\mathcal{F}(u)(\xi)d\xi.$$
where $p(x,\xi)$ is a smooth function on $\mathbb{R}^{2n}$. It is easily seen that such operators map the Schwartz space into itself. We can write $Pu=\mathcal{F}^{-1}(p\cdot\mathcal{F}(u))$ if the function $p(x,\xi)$ is a function of $\xi$ only. These operators are often called Fourier multipliers. Operators that are Fourier multipliers commute. In particular, if $P_1, P_2$ are two Fourier multipliers, then $P_1P_2= P_2 P_1$.

We call $p(x,\xi)$ the symbol of $P$, and we say that $p$ is in the symbol class $S^m_{\rho,\delta}$ where $m\in \mathbb{R}, 0\leq \rho,\delta\leq 1$ if and only if 
$$\abs{D^{\beta}_xD^\alpha_\xi p(x,\xi)} \leq C_{\alpha,\beta}((1+\abs{\xi}^2)^{1/2})^{m-\rho\abs{\alpha}+\delta\abs{\beta}}$$
for some constants $C_{\alpha,\beta}$. The associated operator $P$ is said to be in the class $OPS^m_{\rho,\delta}$. If $\delta<1$, pseudodifferential operators in the class $S^m_{\rho,\delta}$ can be defined instead on the dual of the Schwartz space, which we denote $\mathcal{S}'$. 

Let $\Lambda^s$ be the pseudodifferential operator with symbol $(1+\abs{\xi}^2)^{s/2}$, which is in the class $S^s_{1,0}$. We may recognize the Sobolev space $H^s(\mathbb{R}^n)$ as $\Lambda^{-s}L^2(\mathbb{R}^n)$.
We call a pseudodifferential operator, $P$, Sobolev-smoothing if and only if $P(H^s)\subset H^t$ for any real numbers $s$ and $t$. In other words, a Sobolev-smoothing operator $P$ maps any Sobolev space into any other Sobolev space. We also define the symbol class $S^{-\infty}_{\rho,\delta}$ to be
$$S^{-\infty}_{\rho,\delta} := \bigcap_{m\in \mathbb{R}} S^m_{\rho,\delta}$$
However, there is really only one class of symbols of infinitely negative order, a fact which we note below.
\begin{lemma}
Let $0\leq \rho,\delta \leq 1$ and let $p(x,\xi)\in S^{-\infty}_{\rho,\delta}$. Then $p(x,\xi)\in S^{-\infty}_{1,0}$.
\end{lemma}
\begin{proof}
Suppose $p(x,\xi)\in S^{-\infty}_{\rho,\delta}$. Let $\alpha,\beta$ and real number $N$ be arbitrary. By hypothesis we can choose $m<N-(1-\rho)\abs{\alpha}-\delta\abs{\beta}$ so that for all $(x,\xi)$:
$$\abs{D^\beta_x D^\alpha_\xi p(x,\xi)}\leq C_{\alpha,\beta} (1+\abs{\xi}^2)^{(m-\rho\abs{\alpha}+\delta\abs{\beta})/2}\leq C_{\alpha,\beta} (1+\abs{\xi}^2)^{(N-\abs{\alpha})/2}.$$
The arbitrariness of $N$ proves that $p(x,\xi)\in S^{-\infty}_{1,0}$.
\end{proof}
Henceforth we denote $S^{-\infty}_{1,0}$ by $S^{-\infty}$. We now recall three standard lemmas below. The proofs can be found in Chapter 0 of \cite{TA}.
\begin{lemma}
\label{pseudoproduct}
Let $P_j \in OPS^{m_j}_{\rho_j,\delta_j}$. Let $\rho = \min(\rho_1,\rho_2)$ and $\delta = \max(\delta_1,\delta_2)$. Suppose
$$0\leq \delta_2 <\rho_1\leq 1.$$
Then $$P_1P_2 =Q \in OPS^{m_1+m_2}_{\rho,\delta}.$$
\end{lemma}
\begin{lemma}
\label{pseudocont}
If $P \in OPS^m_{\rho,\delta}$, $\rho>0$, and $m<-n+\rho(n-1)$, then 
$$P : L^p(\mathbb{R}^n) \to L^p(\mathbb{R}^n),\quad \forall \hspace{10px}1\leq p\leq \infty$$
is continuous.
\end{lemma}
\begin{lemma}
\label{L2CONT}
If $P \in OPS^0_{\rho,\delta}$ and $0\leq \delta <\rho  \leq 1$, then 
$$P: L^2(\mathbb{R}^n)\to L^2(\mathbb{R}^n)$$ is continuous.
\end{lemma}
The following lemma, which deals with the "exotic" symbol classes $S^m_{1,1}$, is proven in the appendix of \cite{AT}. It provides an asymptotic expansion for operators in the exotic class.
\begin{lemma} 
\label{exotic1}
Let $0\leq \delta <1$.
Suppose $P\in OPS^m_{1,1}$ and $Q\in OPS^\mu_{1,\delta}$. Then $PQ \in OPS^{m+\mu}_{1,1}$ and we have $PQ=T_N+R_N$ where
$$T_N(x,\xi)= \sum_{\abs{\alpha}\leq N} \frac{i^{\abs{\alpha}}}{\alpha!} D^\alpha_\xi p(x,\xi)D^\alpha_x q(x,\xi)$$
and $R_N \in OPS^{m+\mu-N(1-\delta)}_{1,1}$.
\end{lemma}
Our next lemma gives us an inclusion of the operators in the $S^{-\infty}$ symbol class into the Sobolev-smoothing operators.
\begin{lemma}\label{SMOOTHIE}
If $P$ has its symbol in $S^{-\infty}$, then it is a Sobolev-smoothing operator.
\end{lemma}
\begin{proof}
Let $s,t$ be arbitrary real numbers, and suppose that $f\in H^t=\Lambda^{-t}L^2$, i.e. we may write $f= \Lambda^{-t}u$ for some function $u\in L^2$. We have to show that $Pf \in H^s=\Lambda^{-s}L^2$, but this amounts to showing that $\Lambda^{s}P\Lambda^{-t} u \in L^2$.

Since $P\in S^{-\infty}$, we have, for some $m$ small enough, that $P\in S^m_{1,0}$ with $m+s-t<-1$. Then, by Lemma \ref{pseudoproduct}, $\Lambda^{s}P\Lambda^{-t} \in S^{m+s-t}_{1,0}$, which is continuous from $L^2 \to L^2$ by Lemma \ref{pseudocont}. This proves that $P$ is Sobolev-smoothing.
\end{proof}
The three lemmas that follow are also about pseudodifferential operators. The first two say that certain products of operators are guaranteed to be Sobolev-smoothing.
\begin{lemma}
\label{smoothing}
Let $P$ be a pseudodifferential operator with symbol $p(\xi)$ in the class $S^m_{1,0}$. Let $\phi_1, \phi_2$ be two smooth functions with disjoint supports (and assume they are pseudodifferential operators in the class $S^0_{1,\delta}$ with $1>\delta\geq0$). Then, the composition $\phi_1P\phi_2$ is a smoothing pseudodifferential operator.
\end{lemma}
\begin{proof}
We first compute the symbol $p_2$ of $P_2=P\phi_2$. Since $1>\delta$, a standard result (see \cite{TA}) gets us the asymptotic expansion:
$$p_2(x,\xi) \sim \sum_{\alpha\geq 0} \frac{i^{\abs{\alpha}}}{\alpha !} D^\alpha_\xi p(\xi) D^{\alpha}_x \phi_2(x)$$
where the sum and asymptotic is interpreted in the sense that the sum over $\abs{\alpha}<N$ differs from $p_2(x,\xi)$ by an element of $S^{m-N(1-\delta)}_{1,\delta}$. Then, a similar asymptotic expansion can be found for the symbol of $Q= \phi_1 P_2=\phi_1 P \phi_2$, which is given by:
$$q(x,\xi) \sim \sum_{\alpha\geq 0} \frac{i^{\abs{\alpha}}}{\alpha !} D^\alpha_\xi \phi_1(x) D^{\alpha}_x p_2(x,\xi).$$
It follows that up to a smoothing operator we have:
$$q(x,\xi) \sim \phi_1(x) p_2(x,\xi).$$
Now, for any arbitrary $N$, we have that
$$q(x,\xi) \sim \phi_1(x) \sum_{N>\alpha\geq 0} \frac{i^{\abs{\alpha}}}{\alpha !} D^\alpha_\xi p(\xi) D^{\alpha}_x \phi_2(x)$$ up to an operator in $S^{m-N(1-\delta)}_{1,\delta}$. Since $\phi_1, \phi_2$ have disjoint supports, the above sum is zero for all $N$. We conclude that $q(x,\xi)\in S^{-\infty}_{1,\delta}$, so $Q=\phi_1 P\phi_2$ is an operator in $OPS^{-\infty}$, which, by Lemma \ref{SMOOTHIE}, means that $Q$ is Sobolev-smoothing.
\end{proof}
We separate the next lemma since it deals with the exotic class. The proof, however, is the same as for Lemma \ref{smoothing}.
\begin{lemma}
\label{exotic2}
Let $\phi_1,\phi_2$ be two smooth functions with disjoint supports in the class $S^{0}_{1,\delta}$, with $1>\delta\geq 0$. Let $P$ be a pseudodifferential operator in the class $OPS^m_{1,1}$. Then the composition $\phi_1 P\phi_2$ is a Sobolev-smoothing pseudodifferential operator.
\end{lemma}
\begin{proof}
We first deal with the composition $P_2=P\phi_2$, which by Lemma \ref{exotic1} lies in the class $OPS^m_{1,1}$. However, the same Lemma \ref{exotic1} also gives us the asymptotic expansion:
$$p_2(x,\xi)\sim \sum_{\abs{\alpha}\leq N} \frac{i^{\abs{\alpha}}}{\alpha!} D^\alpha_\xi p(x,\xi)D^\alpha_x \phi_2(x)$$
up to a term in $OPS^{m-N(1-\delta)}_{1,1}$. But then the symbol of $Q=\phi_1 P\phi_2$ is given by:
$$q(x,\xi)\sim \phi_1(x)\sum_{\abs{\alpha}\geq0} \frac{i^{\abs{\alpha}}}{\alpha!} D^\alpha_\xi p(x,\xi)D^\alpha_x \phi_2(x)$$
up to an operator in $S^{-\infty}$. Since $\phi_1, \phi_2$ have disjoint support, the above sum is zero, so it follows that $q\in S^{-\infty}$, as desired.
\end{proof}

The next lemma gives us a quantitative version of the smoothing property:
\begin{lemma}
\label{quantbound} Let $j\in\mathbb{Z}$ and let $P_j$ be a Paley-Littlewood projection operator. Let $\beta$ be an arbitrary real number, and assume that $\| f\|_{H^\beta}\leq 1$.
Let $Q$ be a pseudodifferential operator with symbol $q(x,\xi)$ in the class $S^m_{1,1}$. Let $\phi_1, \phi_2$ be two smooth functions with disjoint supports in the class $S^0_{1,\delta}$ where $1>\delta\geq0$. Then for any positive integer $N$, the composition $\phi_1Q\phi_2$ satisfies the following quantitative bound
$$\| P_j\phi_1 Q \phi_2 f\|_2 \leq K 2^{-jN}$$
with constant $K$ depending only on $N$ and $\beta$.
\end{lemma}
\begin{proof}
By Lemma \ref{exotic2}, we know that $\tilde{Q}:= \phi_1 Q \phi_2$ is a smoothing operator. In particular, we have for any real number $s$:
$$\| \tilde{Q} f\|_s \leq K \|f\|_\beta \leq K,$$
where the constant $K$ depends only on $s$ and $\beta$. By Lemma \ref{finiteband1}, we have
$$2^{sj} \|P_j \tilde{Q}f\|_2 \leq K \|\tilde{Q}f\|_s.$$
Together we get, for some other constant $K$ depending only on $s$ and $\beta$:
$$\|P_j \phi_1 Q\phi_2 f\|_2 \leq K 2^{-s j}.$$
Since $s$ was arbitrary, we achieve what we desired.
\end{proof}
\subsection{Paley-Littlewood Decomposition and Localization}
We use a Littlewood-Paley partition of frequency space. Namely, for $j\in \mathbb{Z}$ we have pseudodifferential operators $P_j$ with smooth symbols $p_j(\xi)$ that are supported in $\frac{2}{3}2^j < \abs{\xi}<3\cdot 2^j$, which also satisfy $p_j(\xi) = p_0(2^{-j} \xi)$ and 
\begin{equation}\sum_{j\in \mathbb{Z}} p_j(\xi) =1.\end{equation}
We also define $\tilde{P}_j:= \sum_{k=-2}^{2} P_{j+k}$ to be the sum of all Littlewood-Paley projections whose symbols' supports intersect the support of $p_j(\xi)$. Likewise we may define $\tilde{p}_j(\xi)$. We also note that
\begin{equation}
\label{proj1}
\tilde{P}_j P_j = P_j
\end{equation}
These Paley-Littlewood projections satisfy the following inequalities
\begin{lemma}
\label{PIneq}
For any $2\leq q \leq \infty$ there exists a constant $K$ independent of $j$ such that
$$\| P_j f\|_q \leq K 2^{nj\big(\frac{1}{2}-\frac{1}{q}\big)} \|P_j f\|_2$$
\end{lemma}
\begin{proof}
Let $\phi_j = \mathcal{F}^{-1}(\tilde{p}_j)$. Recalling the scaling law satisfied by the symbols $p_j$, we see that
\begin{equation}\label{scale1}\phi_j(x)=2^{nj}\phi_0(2^j x).\end{equation}
By definition, $\tilde{P}_jf = \phi_j \ast f$. By Young's convolution inequality (for which we need $q\geq 2$) and the projection property (\ref{proj1}) we have
$$\| P_j f\|_q = \|\tilde{P}_j P_jf\|_q \leq \| P_j f\|_2 \|\phi_j \|_{2q/(2+q)}$$
By the scaling property (\ref{scale1}) and a change of variables we get:
$$\|\phi_j\|_{2q/(2+q)} =2^{nj} \|\phi_0(2^jx)\|_{2q/(2+q)}= 2^{nj(1/2-1/q)}\|\phi_0\|_{2q/(2+q)}.$$
Combining the last two inequalities yields our result.
\end{proof}
The next two lemmas may together be characterized as representative of the "finite band property" relating derivatives with Littlewood-Paley projections using the localization in frequency space. For more details on the finite band property, see \cite{KR} and Chapter 2 in \cite{BCD}. The following result is just a restatement of a lemma from \cite{BCD} in our notation.
\begin{lemma}[Lemma 2.1 in \cite{BCD}]
\label{finiteband1}
For any $s\geq 0$ there exists a constant $K$ such that for any $j\in \mathbb{Z}$ and for all $1\leq p\leq \infty$ we have
$$\|P_j f\|_{W^{s,p}(\mathbb{R}^n)} \leq K 2^{js} \|P_j f\|_{L^p(\mathbb{R}^n)}$$
$$2^{js} \|P_j f\|_{L^p(\mathbb{R}^n)} \leq K \|P_j f\|_{W^{s,p}(\mathbb{R}^n)}.$$
\end{lemma}
We shall need the following $L^\infty$ estimates on the high-frequency cutoff of each bump function $\phi_Q$.
\begin{lemma}
\label{highfreq}
Let $\phi=\phi_{Q,j}$ be a bump function of type $j$. Define $$\phi_2:= \mathcal{F}^{-1}(\chi_{\abs{\xi}>\frac{1}{100}2^j}(\xi)\mathcal{F}\phi(\xi)).$$ We call $\phi_2$ the high frequency cutoff of $\phi$. Then, given any $N$, there is a constant $K$ depending only on $N$ and $\epsilon$ such that:
$$\max(\|\phi_2(x)\|_\infty,\|\mathcal{F}(\phi_2)(\xi)\|_\infty) \leq K 2^{-jN}.$$
\end{lemma}
\begin{proof}
By the Schwartz-Paley-Wiener Theorem, and since $\phi_2$ is smooth and compactly supported with Fourier transform $\mathcal{F}(\phi_2)$ supported away from the origin of frequency space, we may conclude that there is a constant $K$ depending only on $N$ such that
$$\| \mathcal{F}\phi_2(\xi)\|_\infty \leq K 2^{-jN}.$$
We prove a similar inequality for $\|\phi_2(x)\|_\infty$ by a rescaling argument. 
Let $N$ be arbitrary. Let $\psi$ be a smooth function with compact support such that $\textrm{supp}(\mathcal{F}(\psi))=\{\xi : \abs{\xi}\geq \frac{1}{100}\}$. We claim that if $\|D^\alpha \psi(x)\|_\infty\leq C_\alpha 2^{(-\epsilon\abs{\alpha}-n)j}$ for all $\alpha$, then $\| \psi(x)\|_\infty \leq K(N,\epsilon) 2^{(-N-n)j}$, where $K(N,\epsilon)$ depends only on $N$ and $\epsilon$ but not on $j$. Assuming the claim, let $\psi$ be a function satisfying the hypotheses of the claim and consider the rescaled function $$\tilde{\psi}(x):= 2^{nj}\psi(2^{j}x).$$ The Fourier transform of this function is
$$\mathcal{F}(\tilde{\psi})(\xi) = \mathcal{F}(\psi)(2^{-j}\xi).$$
The support of $\mathcal{F}(\tilde{\psi})$ is $\{\xi : \abs{\xi}\geq \frac{1}{100}2^{j}\}$, our claim yields
$$\|\tilde{\psi}(x)\|_\infty \leq 2^{-Nj},$$ and the following derivative bounds hold:
$$\abs{D^\alpha \tilde{\psi}(x)}\leq C_\alpha 2^{(\abs{\alpha}(1-\epsilon)))j}.$$
Proving the $L^\infty$ bound for $\tilde{\psi}$ assuming the above derivative bounds and high frequency support is equivalent to proving the $L^\infty$ bound claimed before for $\psi$. We recognize $\phi_2$ as one such function $\tilde{\psi}$, so we just have to prove the statement regarding $\psi$ to finish the proof of the lemma.

Consider instead $f=2^{nj}\psi$, and assume to the contrary that there exists an $N$ such that for any $K$, we have
$$\|f\|_\infty >K2^{-Nj}\quad\textrm{and}\quad \|D^\alpha f\|_\infty \leq C_\alpha 2^{-\epsilon\abs{\alpha}j} \quad \forall \alpha>0 $$
However, the uniform derivative bounds preclude the arbitrary size of $f$, which is a contradiction.
\end{proof}

In the statement of the following lemmas, the constant $100$ appears. The particular value of this constant is unimportant and may be replaced by any large number of our choosing.
The next lemma is a commutator estimate similar to Proposition 5.2 in \cite{KP}.
\begin{lemma}
\label{commutator1}
Let $2<q<\infty$ be arbitrary. 
Suppose $\| f\|_2 \leq K$. Suppose $\phi$ is a bump function of type $j$ and let $k\geq j$ be arbitrary. Then, for a constant $K$ depending only on $q$:
$$\|\phi P_k f-\tilde{P}_k\phi P_k f\|_2\leq K2^{-j(100\cdot\lfloor \frac{q}{q-2}\rfloor)}.$$
\end{lemma}
\begin{proof}
Given our bump function, we can decompose it into the sum $\phi = \phi_1+\phi_2$ such that $\mathcal{F}\phi_2(\xi)=\chi_{\abs{\xi}> \frac{1}{100}2^j}(\xi)\mathcal{F}\phi(\xi)$ for some $M$. By Lemma \ref{highfreq}, we know there is a constant $K$ depending only on $q$ such that:
$$\max\big(\|\phi_2(x)\|_\infty,\| \mathcal{F}\phi_2(\xi)\|_\infty\big) \leq K 2^{-j(100\cdot\lfloor \frac{q}{q-2}\rfloor)}.$$
On the other hand, we have:
\begin{equation}\label{YES}\tilde{P}_k\phi_1P_k f= \phi_1P_k f.\end{equation}
Indeed, up to a dimensional constant we can write
$$\tilde{P}_k\phi_1 P_k f= \iint e^{ix\cdot \xi} \tilde{p}_k(\xi)\mathcal{F}(\phi_1)(z)p_k(\xi-z)\mathcal{F}(f)(\xi-z)dzd\xi.$$
Now the integrand above is zero unless $\xi-z$ is in the support of $p_k$ and $\abs{z}<\frac{1}{100}2^j$, which together, and since $k\geq j$, imply that $\xi$ remains well inside the support of $\tilde{p}_k$. It follows that the $\tilde{p}_k$ term in the integrand can be discounted, so Equation (\ref{YES}) is true. Now observing that 
$$\phi P_k f- \tilde{P}_k \phi P_k f= \phi_1 P_k f-\tilde{P}_k \phi_1P_k f+\phi_2 P_k f-\tilde{P}_k\phi_2 P_k f$$ and that $\|f\|_2\leq K$, we conclude that
$$\|\phi P_k f- \tilde{P}_k \phi P_k f\|_2 = \|\phi_2 P_k f-\tilde{P}_k \phi_2 P_k f\|_2 \leq K\max(\|\phi_2(x)\|_\infty,\|\mathcal{F}(\phi_2)(\xi)\|_\infty)\leq$$ $$\leq K2^{-j(100\cdot\lfloor \frac{q}{q-2}\rfloor)}$$
\end{proof}
One case of the next commutator estimate is used implicitly in \cite{KP}, although neither a statement nor a proof is given there.
\begin{lemma}
\label{infcommute}
Let $2<q<\infty$ be arbitrary. 
Suppose $\| f\|_2 \leq K$. Suppose $\phi$ is a bump function of type $j$ and let $k\geq j$ be arbitrary. Then, for a constant depending only on $q$:
$$\|\phi P_k f-\tilde{P}_k\phi P_k f\|_\infty\leq K2^{-j(100\cdot\lfloor \frac{q}{q-2}\rfloor)}.$$
\end{lemma}
\begin{proof}
As before, we decompose $\phi=\phi_1+\phi_2$ where $\phi_2$ is the high frequency cutoff. By Lemma \ref{highfreq}, where we let $N=(100\cdot\lfloor \frac{q}{q-2}\rfloor)$, there is a constant $K$ depending only on $q$ such that:
$$\max\big(\|\phi_2(x)\|_\infty,\| \mathcal{F}\phi_2(\xi)\|_\infty\big) \leq K 2^{-j(100\cdot\lfloor \frac{q}{q-2}\rfloor)}.$$
As in the proof of Lemma \ref{commutator1}, we just have to estimate:
$$\|\phi_2 P_k f-\tilde{P}_k\phi_2 P_k f\|_\infty.$$
Now using the fact that $f\in L^2$, our $L^\infty$ bounds for $\phi_2$ and $\mathcal{F}(\phi_2)$, and Lemma \ref{PIneq}, we reach our desired conclusion.
\end{proof}
\begin{lemma}\label{GOODYTWO}
Let $\phi$ be a bump function of type $j$, let $k\geq j$ and $2<q<\infty$ be arbitrary. Suppose also that $\|f \|_2 \leq K$. Then:
$$\|\phi P_k f\|_\infty \leq K 2^{nk/2}\|\phi P_k f\|_2 +K 2^{-j(100\cdot\lfloor \frac{q}{q-2}\rfloor)}$$
\end{lemma}
\begin{proof}
This follows immediately from Lemma \ref{infcommute}, Lemma \ref{PIneq}, and the observation that
$$\phi P_k f= \phi P_k f -\tilde{P}_k\phi P_k f+ \tilde{P}_k\phi P_k f.$$
\end{proof}
The next lemma extends Lemma 5.3 in \cite{KP}.
\begin{lemma}
\label{ineq2}
Let $2\leq q \leq \infty$, $\|f\|_2\leq K$, and let $\phi$ be a bump function of type $j$. Let $k\geq j$ be arbitrary. Then:
$$\|\phi P_k f \|_q \leq K 2^{nk(1/2-1/q)}\|\phi P_k f\|_2 + K2^{-100j}$$
\end{lemma}
\begin{proof}
When $q=2$ the result is obvious. When $q=\infty$, the result is proven in Lemma \ref{GOODYTWO}, and the exponent in the error term can be made arbitrarily small. We prove the remaining cases by interpolation. First, it is evident from Lemma \ref{PIneq} that the function $\phi P_k f \in L^q$ for all $2\leq q \leq \infty$. In particular, since the $L^p$ norms are log-convex:
$$\|\phi P_k f\|_q \leq \|\phi P_k f\|_2^{2/q} \| \phi P_k f\|_\infty^{1-2/q}$$
For the remainder of the proof, we denote $A:= \|\phi P_k f\|_2$. Using Lemma \ref{GOODYTWO}, which is the case $q=\infty$ in the statement of our lemma, we have that:
$$\|\phi P_k f\|_q \leq A^{2/q}\bigg(2^{nk/2}A+ K2^{-100j\lfloor \frac{q}{q-2}\rfloor}\bigg)^{(q-2)/q}$$
Using the elementary inequality $(a+b)^p\leq 2^p(a^p+b^p)$ and the fact that $q>2$ gets us:
$$\|\phi P_k f\|_q \leq 2^{(q-2)/q} A^{2/q}\bigg(2^{\frac{nk}{2}\cdot\frac{q-2}{q}}A^{(q-2)/q}+ K2^{-100j}\bigg)$$
Using the fact that $\|f\|_2\leq K$ we conclude:
$$\|\phi P_k f\|_q \leq K 2^{nk(\frac{1}{2}-\frac{1}{q})}\|\phi P_k f\|_2 +K2^{-100j}$$
as desired.
\end{proof}
We can commute bump functions with Paley-Littlewood projections, as long as we add another bump function with slightly larger support.
\begin{lemma}
\label{bumpcommute}
Let $\phi_{Q,k}$ be a bump function at level $k$. Let $\beta<50$ and consider $f$ with $\| f\|_{H^\beta}\leq K$. Then for any $j$:
$$\|(1-\phi_{(1+2^{-\epsilon k/2})Q,k})P_j \phi_{Q,k} f\|_2 \leq K2^{-100k}$$
\end{lemma}
\begin{proof}
By Lemma \ref{smoothing}, we see that
$$(1-\phi_{(1+2^{-\epsilon k/2})Q,k})P_j\phi_{Q,k}$$
is a smoothing pseudodifferential operator. By Lemma \ref{quantbound}, we have the desired quantitative bound, where $N=100$, and $K$ depends on $N=100$, a fixed constant, and the given size of $f$. The same proof works to get arbitrary exponent $N$ instead.
\end{proof}
Note that in the previous lemma, the order of the bump functions in the composition was unimportant. The only essential fact was disjointness of support.
\begin{lemma}
\label{collectbound}
Let $\mathcal{A}$ be a collection of cubes all at level $j$. For any $f\in L^2$, we have:
$$\|\chi_E P_j f\|_2^2 \leq \sum_{Q\in\mathcal{A}} f_Q^2$$
\end{lemma}
\begin{proof}
We notice that
$$\int \chi_E \abs{P_j f}^2 \leq \int \bigg(\sum_{Q\in\mathcal{A}} \phi_{Q,j}^2\bigg)\abs{P_j f}^2 =\sum_{Q\in\mathcal{A}}f_Q^2.$$ 
\end{proof}
\subsection{Hausdorff Measure and Dimension}
We denote a ball of radius $r$ in $\mathbb{R}^n$ by $B_r$.
Let $E\subset \mathbb{R}^n$. We recall that the $d$-dimensional Hausdorff measure of $E$ is given by
$$H^d(E) := \sup_{\delta >0} \inf_{\mathcal{C}_\delta(E)} \sum_{B_r\in \mathcal{C}_\delta(E)}r^d$$
where we have taken an infimum over all coverings $C_\delta(E)$ of $E$ by balls of radius less than or equal to $\delta$.
We define the Hausdorff dimension of $E$ to be:
$$\mathcal{H}(E) = \inf \{ d : H^d(E)=0\}.$$
We need a way to compute the Hausdorff dimension that follows from a discretization by sets of scale $2^j$. In particular we have:
\begin{lemma}
\label{dimcompute}
Let $\mathcal{A}_j$ be a sequence of collections of balls in $\mathbb{R}^n$ so that each element of $\mathcal{A}_j$ has radius $2^{-j}$. Suppose that the number of balls in each $\mathcal{A}_j$, denoted $N(\mathcal{A}_j)$, is bounded by $N(\mathcal{A}_j)\leq C 2^{jd}$ where $C$ is independent of $j$. Let
$$E = \limsup_{j\to \infty}\mathcal{A}_j:= \cap_{j\in \mathbb{N}} \cup_{k>j} \cup_{B\in \mathcal{A}_k} B$$
be the set of points in infinitely many of the unions $\cup_{B\in\mathcal{A}_j} B$. Then $\mathcal{H}(E)\leq d$. \end{lemma}
\begin{proof}
By the definition of the Hausdorff dimension, it suffices to show that $H^s(E)=0$ for all $s>d$. To that end, pick $j$ large enough that $2^{-j} <\delta$. By construction of our set, $E$ can be covered by $\cup_{k>j} \cup_{B\in \mathcal{A}_k} B$. Evidently, we have
$$H^s(E)\leq \sum_{k>j} N(\mathcal{A}_k)(2^{-k})^s\leq C \sum_{k>j} 2^{kd}(2^{-k})^s $$
and the right-hand side above converges to zero as $j\to \infty$ so long as $d<s$.
\end{proof}
We also include in this subsection the standard Vitali covering lemma, which is often used in estimating the Hausdorff dimension of a set.
\begin{lemma}[Vitali]
\label{Vitali}
Let $\mathcal{A}$ be a collection of cubes. Then there is subcollection $\mathcal{A}'$ so that any two cubes in $\mathcal{A}'$ are pairwise disjoint and 
$$\bigcup_{Q\in \mathcal{A}} Q \subset \bigcup_{Q\in \mathcal{A}'} 5Q.$$
\end{lemma}
\subsection{Zero-Momentum Schwartz Vector Fields}
\begin{lemma}
\label{DivF}
A Schwartz function $\psi: \mathbb{R}^3 \to \mathbb{R}$ can be written as $\psi = \textrm{div } \Psi$ for some Schwartz vector field $\Psi$ if and only if $\int_{{\mathbb{R}^3}}\psi =0$.
\end{lemma}
\begin{proof}
One direction is relatively easy. Indeed suppose that $\psi = \textrm{div } \Psi$ for some Schwartz vector field and let $B_R$ be some ball centered at the origin of radius $R$. Then the divergence theorem gets us:
$$\int_{B_R} \psi = \int_{B_R} \textrm{div } \Psi = \int_{\partial B_R} \Psi \cdot \bf{n}$$
Since $\Psi$ is Schwartz and decays faster than any polynomial at infinity, the right hand side above tends to zero as $R\to \infty$ and the result is shown.\\\\
Now assume that $\int_{\mathbb{R}^3} \psi =0$; we desire to construct $\Psi$. Let $f(z)$ be any real-valued function in $C^\infty_c(\mathbb{R})$. Let 
$$\Gamma(x,y,z) = f(z)\cdot \int_{-\infty}^x \int_{-\infty}^y \int_{-\infty}^\infty \psi(r,s,t)dtdsdr$$
By our assumption on the integral of $\psi$, the function $\Gamma$ is a Schwartz function.
Now, 
$$\partial_x \Gamma = f(z)\int_{-\infty}^y \int_{-\infty}^\infty \psi(x,s,t)dtds$$
and
$$\partial_y \Gamma = f(z)\int_{-\infty}^x \int_{-\infty}^\infty \psi(r,y,t)dtdr.$$
So defining (also a Schwartz function by our assumption):
$$\Xi(x,y,z) = \int_{-\infty}^z (\psi - \partial_x \Gamma -\partial_y \Gamma)(x,y,t)dt$$
we observe that
$$\psi = \partial_x \Gamma + \partial_y \Gamma +\partial_z \Xi$$
so our desired vector field $\Psi = (\Gamma, \Gamma, \Xi)$.
\end{proof}

\subsection{The Continuity of the Energy Integral}
\begin{lemma}\label{Integral1}
Let $u$ be any solution of Equation (\ref{NEEDIT}), let $Q$ be any cube, and let $t_0<T$ be any time before blowup. Then the following equality holds:
$$\lim_{\epsilon\to 0} \int_{t_0}^{T-\epsilon} \frac{d}{dt} u_{Q}^2 dt = \int_{t_0}^T \frac{d}{dt}u_{Q}^2 dt $$
\end{lemma}
\begin{proof}
We recall equation (\ref{NEEDIT}):
$$\langle\partial_t u, u \rangle + \langle(-\Delta)^\alpha u , u \rangle =0$$
which we may rewrite as:
$$F(t):=\int_{\mathbb{R}^n} \partial_t u\cdot u\hspace{2px} dx = -\int_{\mathbb{R}^n} \abs{(-\Delta)^{\frac{\alpha}{2}}u}^2 dx.$$
By the energy dissipation law we may conclude that $F(t)\in L^1([t_0,T])$. Now if we denote $\chi_\epsilon(t)$ to be the indicator function of the interval of time $[t_0,T-\epsilon]$, we see by the energy dissipation law:
$$\abs{\chi_\epsilon(t)\frac{d}{dt}u_{Q}^2(t)}=\abs{2\chi_\epsilon(t)u_Q(t)\frac{d}{dt}u_Q(t)}\leq K\abs{F(t)}.$$
Thus, by the dominated convergence theorem, we achieve the desired result.
\end{proof}
As an immediate corollary, we have:
\begin{corollary}\label{Integral2}
Let $t_0<T$ be any time. Then 
$$ \int_{t_0}^{T} \frac{d}{dt} u_Q^2 dt = \limsup_{t\to T} u_Q^2(t) -u_Q^2(t_0).$$
\end{corollary}
\begin{proof}
Indeed, we may write
$$ \int_{t_0}^{T-\epsilon} \frac{d}{dt} u_Q^2 dt = u_Q^2(T-\epsilon) -u_Q^2(t_0),$$
take the $\limsup$ of the above expression as $\epsilon\to 0$ and use Lemma \ref{Integral1}.
\end{proof}
\begin{thebibliography}{10}

\bibitem{A}
Almgren, F.J.
\textit{Existence and Regularity almost everywhere of Solutions to Elliptic Variational Problems with Constraints}.
Memoirs of the AMS. 165. Providence, R.I., (1976).

\bibitem{AT}
Auscher, P., Taylor, M.E. 
\textit{Paradifferential Operators and Commutator Estimates}
Comm. PDE. 20. pp. 1743-1775. (1995).

\bibitem{BCD}
Bahouri, H., Chemin, J-Y, Danchin, R.
\textit{Fourier Analysis and Nonlinear Partial Differential Equations}.
Springer. (2011).

\bibitem{CKN}
Caffarelli, L., Kohn, R., Nirenberg, L. 
\textit{Partial Regularity of Suitable Weak Solutions to the Navier-Stokes Equations}.
Comm. Pure Appl. Math. 35. pp. 771-831. (1982).

\bibitem{CDM}
Colombo, M., De Lellis, C., Massaccesi, A.
\textit{The Generalized Caffarelli-Kohn-Nirenberg Theorem for the Hyperdissipative Navier-Stokes System}.
Comm. Pure Appl. Math. 73. pp. 609-663. (2020).

\bibitem{CH}
Colombo, M., Haffter, S.
\textit{Global Regularity for the Hyperdissipative Navier-Stokes equation below the critical order}.
J. Diff. Eqn. Vol. 275. pp. 815-836. (2021).

\bibitem{F}
Fefferman, C.L.
\textit{Existence and Uniqueness of the Navier-Stokes Equation}.
Clay Mathematics Institute. (2006).

\bibitem{H}
Hou, T.Y.
\textit{Potentially Singular Behavior of the 3D Navier-Stokes Equations}.
Found. Comput. Math. (2022).

\bibitem{KP}
Katz, N.H., Pavlovic, N. 
\textit{A cheap Caffarelli-Kohn-Nirenberg inequality for the Navier-Stokes equation with hyper-dissipation}.
Geom. Funct. Anal. 12. pp. 355-379. (2002).

\bibitem{KR}
Klainerman, S., Rodnianski, I.
\textit{A Geometric Approach to the Littlewood-Paley Theory}.
Geom. Funct. Anal. 16. pp. 126-163. (2006).

\bibitem{L}
Leray, J. 
\textit{Etude de diverses equations integrales non lineaires et de quelques problemes que pose l'hydrodynamique}.
J. Math. Pures Appl. (1933).

\bibitem{O}
Ozanski, W.S.
\textit{Partial Regularity of Leray-Hopf weak Solutions to the incompressible Navier-Stokes equations with hyperdissipation}.
Analysis and PDE. Vol 16. pp. 747-783. (2023).

\bibitem{S}
Scheffer, V. 
\textit{Partial Regularity of Solutions to the Navier-Stokes Equations}.
Pacific J. Math. Vol. 16. No. 2. (1976).

\bibitem{SE}
Serrin, J.
\textit{On the Interior Regularity of Weak Solutions of the Navier-Stokes equations}.
Arch. Ration. Mech. Anal. Vol. 9, pp. 187-195. (1962).

\bibitem{TY}
Tang, L., Yu, Y.
\textit{Partial Regularity of Suitable Weak Solutions to the Fractional Navier-Stokes Equations}.
Commun. Math. Phys. Vol. 334. pp. 1455-1482. (2015).

\bibitem{T}
Tao, T.
\textit{Finite time blowup for an averaged three-dimensional Navier-Stokes equation}.
J. Amer. Math. Soc. 29. pp. 601-674. (2016).

\bibitem{TA}
Taylor, M.E.
\textit{Pseudodifferential Operators and Nonlinear PDE}.
Birkhauser. (1991).

\bibitem{TS}
Tsai, T.P.
\textit{Lectures on the Navier-Stokes Equations}.
American Mathematical Society. (2018).


\end{thebibliography}
\end{document}