\documentclass[12pt]{article}   	% use "amsart" instead of "article" for AMSLaTeX format
\usepackage{geometry}                		% See geometry.pdf to learn the layout options. There are lots.
\geometry{letterpaper}                   		
%\usepackage[parfill]{parskip}    		% Activate to begin paragraphs with an empty line rather than an indent
\usepackage{graphicx}				% Use pdf, png, jpg, or eps§ with pdflatex; use eps in DVI mode
\usepackage{caption}
\usepackage{subcaption}
\usepackage{amssymb}
\usepackage{amsthm}
\usepackage{csquotes}
\usepackage{amsmath}
\usepackage{mathtools}
\usepackage{float}
\usepackage{physics}
\MakeOuterQuote{"}
\newtheorem*{theorem*}{Theorem}
\newtheorem{theorem}{Theorem}[section]
\newtheorem{definition}[theorem]{Definition}
\newtheorem{proposition}[theorem]{Proposition}
\newtheorem{lemma}[theorem]{Lemma}
\newtheorem{sublemma}[theorem]{Sub-Lemma}
\newtheorem{remark}[theorem]{Remark}
\newtheorem{corollary}[theorem]{Corollary}
\newtheorem{conjecture}[theorem]{Conjecture}
\newtheorem*{conjecture*}{Conjecture}
\title{Partial Regularity and Blowup for an Averaged Three-Dimensional Navier-Stokes Equation}
\author{Matei P. Coiculescu}
\begin{document}
\maketitle
\begin{abstract}
We prove two results that together strongly suggest that obtaining a positive answer to the Navier-Stokes global regularity question requires more than a refinement of partial regularity theory. First we prove that there exists a class of bilinear operators $\mathfrak{B}$, which contains the Euler bilinear operator $\mathcal{E}(u,v):=\frac{1}{2}\mathbb{P}(u\cdot\nabla v + v\cdot\nabla u)$, such that for any $B\in \mathfrak{B}$, $n\geq2$, $\alpha \in (1/2, (n+2)/4)$, and smooth solution $u$ of the pseudodifferential equation 
$\partial_t u +(-\Delta)^\alpha u +B(u,u)=0$ on $\mathbb{R}^n\times [0,T)$, we have that $u$ is also smooth at time $T$ away from a closed set of Hausdorff dimension at most $n+2-4\alpha$. Next we prove that, for the Euclidean space $\mathbb{R}^3$, there exists an operator $C(u,v)\in \mathfrak{B}$ that is an averaged version of $\mathcal{E}$, that formally allows the dissipation of energy by the "cancellation identity" $\langle C(u,u), u\rangle =0$, and whose corresponding pseudodifferential equation $\partial_t u +(-\Delta)^\alpha u +C(u,u)=0$ admits a solution that blows up in finite time for all $\alpha \in (1/2,5/4)$.
\end{abstract}

\section{Introduction}
The incompressible Navier-Stokes equations are a system of nonlinear partial differential equations that model the motion of an incompressible viscous fluid. In particular, if $u(x,t)$ is the velocity field of a fluid in $\mathbb{R}^3\times [0,T)$ dynamically changing in time $t$, with initial vector field $u_0$, external force $f$ and fluid pressure $p$, then the Navier-Stokes equations state that $u$ satisfies:
$$
\begin{gathered}
\partial_t u -\nu\Delta u+(u\cdot\nabla)u+\nabla p=f\\
\textrm{div } u=0\\
u(t=0)=u_0
\end{gathered}
$$
The parameter $\nu$ determines the viscosity of the fluid. Since its value has no influence on our work, we may assume that the viscosity $\nu=1$. We also consider the Navier-Stokes equations with "fractional dissipation". In particular let $\alpha\geq 0$ and consider the pseudodifferential operator $(-\Delta)^\alpha$ whose Fourier symbol is $\abs{\xi}^{2\alpha}$. Whenever $\alpha$ equals a positive integer $k$, the operator coincides, up to a sign, with the composition of the classical Laplacian $(-\Delta)$ with itself $k$ times. The unforced $\alpha$-dissipative Navier-Stokes system in $\mathbb{R}^3$ is the following system of pseudodifferential equations:
\begin{equation}
\label{FRACNS}
\begin{gathered}
\partial_t u+(-\Delta)^\alpha u+ (u\cdot \nabla)u +\nabla p = 0 \\
\textrm{div } u=0 \\
u(t=0)= u_0
\end{gathered}
\end{equation}
The $\alpha$-dissipative Navier-Stokes system is typically called hyperdissipative if $\alpha>1$ and hypodissipative if $\alpha<1$. The case when $\alpha=1$ coincides with the classical Navier-Stokes equations. For $\alpha\geq 5/4$ and for any smooth function $u_0$ which decays sufficiently fast at infinity, it is known that the system in Equation (\ref{FRACNS}) has a classical global-in-time solution. In the case of Equation (\ref{FRACNS}) on $\mathbb{R}^n$, classical global-in-time solutions are guaranteed whenever $\alpha >(n+2)/4$ and the initial data is smooth and has sufficient decay. We offer a proof of the latter fact in Section 5.6. Global regularity in $\mathbb{R}^3$ is also known for an open set of initial data if $\alpha$ is slightly less than $5/4$, see \cite{CH}. Generally, when $\alpha<5/4$, the global existence of classical solutions is still a major open question \cite{F}.

One approach towards proving global regularity is to show that the set where singularities of the vector field develop is small in some sense. In particular, one might want to find a bound on the Hausdorff dimension of such a set. Finding such bounds is within the scope of partial regularity theory. We now provide a short summary of the partial regularity theory for the classical Navier-Stokes equations. 

Leray showed in \cite{L} that the set of singular times has $1/2$-dimensional Hausdorff measure equal to zero. Scheffer wrote several papers, beginning with \cite{S}, bounding the Hausdorff dimension of the singular set in space at the the time of first blowup. The best known partial regularity result was obtained by Caffarelli, Kohn, and Nirenberg in \cite{CKN} for a particular class of weak solutions called suitable weak solutions. Let $u$ be a suitable weak solution on any open subset of spacetime $\mathbb{R}^3\times\mathbb{R}$. If we define the singular set (in space-time) as
$$S= \{ (x,t) : u \not\in L^\infty \textrm{ in any neighborhood of } (x,t)\}$$
then the $1$-dimensional Hausdorff measure of $S$ is equal to zero. Away from the set $S$, the vector field $u$ is bounded, which  implies that $u$ is smooth in the spatial variable by the Serrin higher regularity theorem, see \cite{SE} for more details. The bound on the Hausdorff dimension obtained by Caffarelli, Kohn, and Nirenberg in 1982 has not been improved. 

The work of Colombo, De Lellis, and Massaccesi in \cite{CDM} provides the optimal extension of the Caffarelli-Kohn-Nirenberg theory, dealing definitively with the hyperdissipative case. Tang and Yu in \cite{TY} and the erratum \cite{TY1} were the first to analyze the hypodissipative case $3/4<\alpha <1$. Kwon and Ozanski in \cite{KO} further sharpened the work done in \cite{TY}. Additionally, the only partial regularity result that we know of for non-stationary higher-dimensional Navier-Stokes equations is found in \cite{DG}, where Dong and Gu prove a result analogous to the one in \cite{CKN} in four dimensions. However, we cannot see a way to transfer these techniques to the setting of our interest. 

Our present work is inspired by the paper \cite{KP} of Katz and Pavlovic. The authors of \cite{KP} prove a partial regularity theorem like Scheffer's in \cite{S} using Fourier analysis and a heuristic of "wavelets" by localizing in both physical and frequency space in a way compatible with the Heisenberg Uncertainty Principle. One of the advantages of their approach when compared to the work of Scheffer or Caffarelli, Kohn, and Nirenberg is that the local energy inequality is unnecessary for the proof. Ozanski's work in \cite{O}, which clarifies and extends the Fourier analytic method of Katz and Pavlovic, gets a partial regularity result for Leray-Hopf weak solutions of the hyperdissipative Navier-Stokes equations as well as a bound on the box-counting dimension of the singular set.

The partial regularity portion of our paper, although inspired by \cite{KP}, differs in many significant ways from both \cite{KP} and \cite{O}. Our approach is to directly prove an $\epsilon$-regularity theorem on cubes using Fourier analytic methods. One innovation of ours is a carefully selected control on the kinetic energy in the hypothesis of the $\epsilon$-regularity theorem. Our approach avoids the "barrier" argument of both \cite{KP} and \cite{O}, which otherwise creates an obstruction to attacking the classical and hypodissipative Navier-Stokes equations by these methods. Moreover, by performing a highly technical analysis of pseudodifferential operators, we get a partial regularity result that holds for a large family of pseudodifferential equations in addition to the fractionally dissipative Navier-Stokes equations. In particular, this allows us to apply our partial regularity theorem to a pseudodifferential equation admitting a solution blowing up in finite time.

We now briefly discuss some work done in the "negative" direction of the global regularity problem.
Besides numerical studies like \cite{H} that suggest the development of a finite-time singularity for the Navier-Stokes equations, the paper \cite{T} by Tao provides evidence that at the very least, attacking the problem of global regularity using abstract methods depending on basic $L^p$ bounds for the nonlinearity is a direction certain to fail. Let $\mathbb{P}$ denote the Leray projection on divergence-free vector fields (see Section 1.6 in \cite{TS}). Tao constructs a nonlinearity $C(u,v)$ that is an averaged version of the Euler bilinear operator
$$\mathcal{E}(u,v)=\frac{1}{2}\mathbb{P}\bigg((u\cdot \nabla) v+(v\cdot \nabla) u\bigg),$$
satisfies the cancellation identity and various "harmonic-type" estimates such as
$$\|C(u,v) \|_{p} \leq K\big(\|u\|_{q}\|\nabla v\|_{r}+\|v\|_{q}\|\nabla u\|_{r}\big)$$
where $1/p = 1/q+1/r$ and $1< p,q,r< \infty$, and whose associated partial differential equation admits a finite-time singularity. Here and in the sequel, that an operator $C$ is an "averaged version of $\mathcal{E}$" means that:
$$\langle C(u,v) , w\rangle =$$ \begin{equation} \label{AVERAGE}= \int_\Omega \langle \mathcal{E}(m_{1,\omega}\textrm{Rot}_{R_{1,\omega}}\textrm{Dil}_{\lambda_1} u, m_{2,\omega}\textrm{Rot}_{R_{2,\omega}}\textrm{Dil}_{\lambda_2} v), m_{3,\omega}\textrm{Rot}_{R_{3,\omega}}\textrm{Dil}_{\lambda_3} w\rangle d\mu(\omega)\end{equation}
for all Schwartz vector fields $u,v,w$. Here, $\omega$ is a variable in the probability space $(\Omega, \mu)$, and the maps $R_{i,\omega}: \Omega \to SO(3), \lambda_{i,\omega}: \Omega \to [C^{-1}, C],$ and $m_{i,\omega}: \Omega \to \mathfrak{M}_0$ are measurable maps. We use the notation $\mathfrak{M}_0$ for the space of all Fourier multipliers $m$ of order $0$ with Schwartz symbols equipped with the topology (hence also the $\sigma$-algebra) generated by the Schwartz seminorms. The operators Dil and Rot are respectively the dilation and rotation operator in $\mathbb{R}^3$. See \cite{T} for more details.

Our main result suggests that improving partial regularity results for the Navier-Stokes equations is not a path to answering the global regularity question, unless one uses more specific structural properties of the Navier-Stokes nonlinearity. We denote the parabolic cylinder of radius $r$ at $(x,t)$ by:
$$\mathcal{P}_r(x,t):= \{ (y,s) : \abs{y-x}< r \textrm{ and } t-r^2< s < t\}$$
Here and in the sequel we denote the closed singular set at blowup time $T$ by
$$S_T:= \{ x\in\mathbb{R}^n : \forall r>0, u\not\in  L^\infty_t C^\infty_x(\mathcal{P}_r(x,T))\}.$$
Our main theorem provides an upper bound on the Hausdorff dimension of $S_T$ for blowup solutions of an averaged three-dimensional $\alpha$-dissipative Navier-Stokes Equation. In particular, we prove:
\begin{theorem}
\label{MAINMAIN}
Let $1/2<\alpha<5/4$ be arbitrary.
There exists a symmetric bilinear operator $C$ that is an averaged version of $\mathcal{E}$, that obeys the cancellation identity
$$\langle C(u,u), u\rangle =0$$
for all $u\in H^{10}(\mathbb{R}^3)$ that are divergence-free in the distributional sense, and such that, for some Schwartz divergence-free initial vector field $u_0$, there is no global-in-time solution to
$$\begin{gathered} \partial_t u +(-\Delta)^\alpha u +C(u,u) +\nabla p =0\\
u(t=0)=u_0 \quad \quad \textrm{div } u=0.\end{gathered}
$$
Moreover, if $T$ is the time of first blowup, then the closed set $S_T$ has Hausdorff dimension at most $5-4\alpha$.
\end{theorem}

The partial regularity portion of Theorem \ref{MAINMAIN} follows from a more general partial regularity theorem for a family of pseudodifferential equations of "Navier-Stokes type". Then, proving Theorem \ref{MAINMAIN} amounts to finding an equation in the family admitting a finite-time singularity. The proof of our partial regularity theorem will be in Section 2, and we now describe it in more detail. First we define a certain class $\mathfrak{B}$ of bilinear operators.

\begin{definition}
\label{AMEN}
We call a symmetric bilinear operator $B$ $\textbf{amenable}$ if and only if:
\begin{enumerate}
\item $B$ is defined for Schwartz vector fields
\item $B$ satisfies the Cancellation Identity:
\begin{equation}\label{Cancel}
\langle \mathbb{P}B(u,u), u\rangle=0 \quad \textrm{ for divergence-free } u
\end{equation}
\item The linear maps $B^1_v(u):= \mathbb{P}B(u,v), B^2_v(u):= \mathbb{P}B(v,u)$ are pseudodifferential operators in the class $OPS^m_{1,1}$ for some real number $m$ (the definition of this class may be found in Section 5.1)
\item For all $\gamma> 0$ and for all $1\leq p_1, p_2, r\leq \infty$ satisfying
$$\frac{1}{p_1}+\frac{1}{p_2}+\frac{\gamma}{3} = \frac{1}{r}$$
we have for some constant $K$ depending only on $\gamma$, $p_1$, $p_2$, and $r$: \begin{equation}
\label{amenineq}\| \mathbb{P}B(u,v) \|_{L^r} \leq K \| u\|_{L^{p_1}}(\|v\|_{L^{p_2}} +\| \nabla v\|_{L^{p_2}})\end{equation} for all $u\in L^{p_1}$ and $v\in W^{1,p_2}$ with support of diameter at most $2^{100}$ (note that the endpoint cases of integrability are included).
\item Scaling: There exists a number $\lambda>1$ such that for any $k\in \mathbb{Z}$:
$$B(u(\lambda^k y), u(\lambda^k y))(x) = \lambda^k B(u(y),u(y))(\lambda^k x)$$
\end{enumerate}
\end{definition}
The class $\mathfrak{B}$ is the collection of all amenable operators. Throughout the rest of our work, we may alternately say that $B\in \mathfrak{B}$ or that $B$ is amenable. Our partial regularity theorem is:
\begin{theorem}
\label{PRMAINMAIN}
Let $n\geq 2$ and $1/2<\alpha<(n+2)/4$, let $B$ be an amenable bilinear operator, and let $u_0$ be a Schwartz vector field.
Suppose $u(t)$ is a smooth vector field on $\mathbb{R}^n\times [0,T) \to \mathbb{R}^n$ and suppose there exists a smooth function $p$ on $\mathbb{R}^n\times [0,T)$ such that:
$$
\begin{gathered}
\partial_t u +(-\Delta )^\alpha u+B(u,u)+\nabla p=0\\
\textrm{div } u=0\\
u(t=0)=u_0\end{gathered}
$$
If $T$ is the time of first blowup, then the closed set $S_T$ has Hausdorff dimension at most $n+2-4\alpha$.
\end{theorem}

One corollary of Theorem \ref{PRMAINMAIN} is a partial regularity result for the $\alpha$-dissipative Navier-Stokes equations whenever $\alpha>1/2$ and for any dimension $n\geq 2$, which, since it is an improvement on what is already known, we state separately: 
\begin{corollary}
\label{CORAL}
Let $n\geq 2$ and $1/2<\alpha<(n+2)/4$. Let $u_0$ be a Schwartz vector field.
Suppose $u(t)$ is a smooth vector field on $\mathbb{R}^n\times [0,T) \to \mathbb{R}^n$ and suppose there exists a smooth function $p$ on $\mathbb{R}^n\times [0,T)$ such that:
$$
\begin{gathered}
\partial_t u +(-\Delta )^\alpha u+(u\cdot\nabla)u+\nabla p=0\\
\textrm{div } u=0\\
u(t=0)=u_0\end{gathered}
$$
If $T$ is the time of first blowup, then the closed set $S_T$ has Hausdorff dimension at most $n+2-4\alpha$.
\end{corollary}

Our Corollary \ref{CORAL} demonstrates a partial regularity result for the $\alpha$-dissipative Navier-Stokes equations for the farthest values of the parameter $\alpha$ expected by scaling. In particular, when $\alpha\geq (n+2)/4$, we know that solutions are globally regular, and when $\alpha \leq 1/2$, we see that the scaling of the equation predicts no useful partial regularity result. We should also mention that the authors of \cite{CDR} prove an ill-posedness result for weak (but not Leray-Hopf weak) solutions of Equation (\ref{FRACNS}) when $\alpha<1/2$, further evidence that it is a "natural" and non-dimensional endpoint for fractionally dissipative Navier-Stokes.

 We note that while previous results prove partial regularity of the solution in space-time, our result only concludes that the solution is "partially regular" in space at the blowup time. Thus, although our technique requires neither the local energy inequality nor an {\textit{a priori}} $L^3$ spacetime integrability to estimate the nonlinear term (and this allows us to reach $\alpha>1/2$ in any dimension), our theorem has a weaker conclusion and does not prove a space-time partial regularity result.

Our paper is organized as follows. In Section 2 we prove Theorem \ref{PRMAINMAIN}. In particular, given the vector field solution $u$, we construct a set $M$ of Hausdorff dimension $n+2-4\alpha$ and prove that it contains the singular set $S_T$. The energy estimates used in Section 2 are proven in Section 3, where we test the pseudodifferential equation against "wavelet" localizations of the solution (localizations in both frequency and physical space). In Section 4, we show that Tao's method is flexible enough to construct a blowup solution to the pseudodifferential equation associated to some amenable bilinear operator, which, together with Theorem \ref{PRMAINMAIN}, proves Theorem \ref{MAINMAIN}. In Section 5, we prove some technical lemmas that are used throughout the paper. 

We would like to thank our advisor, Professor Camillo De Lellis, for suggesting the topic of this paper, for many helpful discussions, and for reviewing several drafts of our work. We also thank him for his support and encouragement throughout the process of writing this paper. We thank Vikram Giri for helpful discussions about \cite{KP}. We would also like to acknowledge the support of the National Science Foundation in the form of an NSF Graduate Research Fellowship and under Grant No. DMS-1946175.


\section{General Partial Regularity Results}
We refer the reader to Section 5.1 whenever we mention pseudodifferential operators and to Section 5.2 whenever we mention the Littlewood-Paley partition of frequency space.
\subsection{Covering the Singular Set}
Let $\mathbb{P}$ denote the Leray projection. Let $\mathcal{F}$ denote the Fourier transform. We shall employ a Littlewood-Paley partition of frequency space. In particular, for $j\in \mathbb{Z}$ we have pseudodifferential operators $P_j$ with positive and smooth symbols $p_j(\xi)$ that are supported in $\frac{2}{3}2^j < \abs{\xi}<3\cdot 2^j$ and that also satisfy $p_j(\xi) = p_0(2^{-j} \xi)$ and 
\begin{equation}\sum_{j\in \mathbb{Z}} p_j(\xi) =1.\end{equation}
We also define $\tilde{P}_j:= \sum_{k=-2}^{2} P_{j+k}$ to be the sum of all Littlewood-Paley projections whose symbols' supports intersect the support of $p_j(\xi)$. Likewise we may define $\tilde{p}_j(\xi)$ as $\sum_{k=-2}^{k=2} p_{j+k}(\xi)$.
We can consider $P_j f$ as a combination of wavelets supported on cubes of sidelength $2^{-j}$. Keeping this heuristic in mind, we localize on cubes of sidelength almost equal to $2^{-j}$. That is, we fix an $\epsilon>0$ and localize within slightly larger cubes of sidelength $2^{-j(1-\epsilon)}$. Since the Hausdorff dimension estimate is a closed condition, an approximate localization like this will be acceptable. Our localization is achieved with the use of particularly chosen bump functions that we describe below. See also \cite{KP} and \cite{O}. 

Let $Q$ be a cube of sidelength greater than $2^{-j(1-\epsilon)}$. We denote $\lambda Q$ to be the cube with the same center as $Q$ and with $\textrm{sidelength}(\lambda Q) =\lambda\cdot \textrm{sidelength}(Q)$. We define a bump function $1\geq \phi_{Q,j}\geq 0$ such that $\phi_{Q,j}=1$ on $Q$ and is zero outside of $(1+2^{-\epsilon j})Q$. We also require that for every multi-index $\alpha$, there is a constant $C_\alpha$ depending on $\alpha$ but independent of $Q$ and $j$ such that
\begin{equation}
\label{DBOUNDS}
\abs{D^\alpha \phi_{Q,j}}\leq C_\alpha 2^{\abs{\alpha}j(1-\epsilon)}
\end{equation}
Following \cite{KP}, we call these bump functions of type $j$. In line with our previous discussion, we consider $\phi_{Q,j}P_j$ as a projection onto a localized wavelet, and, if $Q$ is a cube with side length exactly equal to $2^{-j(1-\epsilon)}$, we denote the "wavelet coefficient" by
$$\| \phi_{Q,j}P_j f\|_2:= f_Q$$
and say that $Q$ is at level $j$. We also define the first nuclear family, $\mathcal{N}^1(Q)$, of a cube $Q$ at level $j$ to be a union of five collections of cubes $A_{Q,i}$ where there are fewer than $2^{10}$ cubes in each $A_{Q,i}$ that cover $(1+2^{-\epsilon j})Q$. Moreover, we require that the cubes in $A_{Q,i}$ be at level $j-3+i$ and $i\in\{1,2,3,4,5\}$. We may recursively define $\mathcal{N}^l(Q)$ to be a union of the collections $\mathcal{N}^1(Q')$ for all $Q' \in \mathcal{N}^{l-1}(Q)$. With this definition, it is evident that
$$N(\mathcal{N}^l(Q))\leq 2^{13l},$$
where we have used the notation $N(A)$ to denote the cardinality of a set $A$. Note that the cardinality of the collection $\mathcal{N}^l(Q)$ is independent of the level $j$ of the cube $Q$.
As shorthand, we also write for every $l$:
$$f_{\mathcal{N}^l(Q)}^2 := \sum_{Q'\in\mathcal{N}^l(Q)}f_{Q'}^2.$$
We can also appropriately construct the nuclear family so that several useful properties hold, as we shall demonstrate in the next lemma. Henceforth, we may always assume that $0<\epsilon<1/2$.
\begin{lemma}
\label{NUKE}
We can construct nuclear families so that the following properties hold.
\begin{enumerate}
\item 
There exists a constant $K\geq 0$ depending only on $\epsilon$ so that for all $j\geq K$ and for all cubes $Q$ at level $j$, we can let each collection $A_{Q,1}$ and $A_{Q,2}$ in the nuclear family $\mathcal{N}^1(Q)$ consist of exactly one cube with the same center as $Q$.
\item
There exists a constant $K\geq 0$ depending only on $\epsilon$ so that for all $j\geq K$, for all cubes $Q$ at level $j$, for all $0<l<500/\epsilon$, and for every cube $Q'$ in $\mathcal{N}^l(Q)$, we have $Q'\cap Q \neq \emptyset$. 
\end{enumerate} 
\end{lemma}
\begin{proof}
We may henceforth assume that all the cubes in the nuclear family have sides parallel to the sides of $Q$. We remark that the sidelength of the cube $(1+2^{-\epsilon j})Q$ is 
$$(1+2^{-\epsilon j})2^{-j(1-\epsilon)} = 2^{-j(1-\epsilon)} + 2^{-j}.$$
On the other hand,
$$2^{-(j-1)(1-\epsilon)} \geq 2^{-j(1-\epsilon)} + 2^{-j}$$ is equivalent to 
$$2^{1-\epsilon} \geq 1+ 2^{-\epsilon j},$$
which holds, since $0<\epsilon <1/2$, if $j\geq 2/\epsilon$. Thus we can let each $A_{Q,1}$ and $A_{Q,2}$ contain a single cube with sides parallel with $Q$ and the same center as $Q$. Now we would like to guarantee that all the "finer" cubes in the nuclear family still intersect the cube $Q$. We begin by studying the cubes in the collection $A_{Q, 5}$ of the first nuclear family $\mathcal{N}^1(Q)$.

We may assume that all cubes in $A_{Q,5}, A_{Q,4},$ and $A_{Q,3}$ lie in the interior of the cube $\tilde{Q}:=(1+2^{-\epsilon j})Q$. Let $Q'$ be a cube in $A_{Q,5}$ that has sidelength $2^{-(j+2)(1-\epsilon)}$. The "worst" position of the cube $Q'$ would be if it lies in one of the corners of $\tilde{Q}$. In this case, $Q'$ would still intersect $Q$ if 
$$2^{-(1-\epsilon)(j+2)} > 2^{-j-1}$$
or, equivalently, if $\epsilon j> 1-2\epsilon$. The latter condition on $j$ would guarantee that all the cubes in $\mathcal{N}^1(Q)$ intersect $Q$, and we now iterate the argument for nuclear families $\mathcal{N}^l(Q)$.

Let $0<l< 500/\epsilon$. The cubes in $\mathcal{N}^l(Q)$ of smallest scale have sidelength $2^{-(j+2l)(1-\epsilon)}$. Such a cube is guaranteed to intersect the original cube $Q$ if 
$$2^{-(j+2l)} + \ldots + 2^{-(j+2)} +2^{-j} < 2^{-(j+2l)(1-\epsilon)}$$
or equivalently 
$$1 + \ldots + 2^{2l-2} + 2^{2l} < 2^{\epsilon(j+2l)}.$$
The inequality above holds if
$$j > \frac{\log_2(1+\ldots +2^{2l-2} + 2^{2l})}{\epsilon} -2l.$$
Since $l<500/\epsilon$, we conclude there exists a constant $K$ depending on $\epsilon$ such that for all $j\geq K$ for all cubes $Q$ at level $j$, for all $0<l<500/\epsilon$, and for every cube $Q'$ in $\mathcal{N}^l(Q)$, we have $Q'\cap Q \neq \emptyset$. 

\end{proof}

Throughout Section 2 we use various properties satisfied by the bump functions $\phi_{Q,j}$ and the Littlewood-Paley projections $P_j$. Most of these properties can be classified as "commutator estimates" that allow us to move bump functions across Littlewood-Paley projections and vice versa with only the addition of a negligible error. For the benefit of exposition, we defer the technical proofs of these lemmas to Section 5; however, we shall use the results in Section 5 freely here.

We shall now prove a general partial regularity theorem for systems of differential equations. These systems will generalize the Navier-Stokes equations in the sense that they arise from the Stokes equations with a nonlinearity of the form $B(u,u)$, where $B$ is an amenable operator in the sense of Definition \ref{AMEN}. Without loss of generality, we may assume in this section that the scaling term $\lambda$ in the fifth part of Definition \ref{AMEN} is equal to $2$. 

We denote the Leray Projection of $B$ by $\hat{B}:=\mathbb{P}B$ for the rest of this section. By the incompressibility condition on $u$ in the statement of the theorem, we can instead consider the following system of pseudodifferential equations:
\begin{equation}
\label{PEQN}
\begin{gathered}
\partial_t u +(-\Delta)^\alpha u +\hat{B}(u,u)=0\\
u(t=0)=u_0.
\end{gathered}
\end{equation}
Then we prove:
\begin{theorem}
\label{PRTHM}
Let $n\geq 2$ and $1/2<\alpha < (n+2)/4$, let $B$ be an amenable bilinear operator, and let $u_0$ be a Schwartz vector field.
Suppose that $u(t)$ is a smooth vector field on $\mathbb{R}^n\times [0,T) \to \mathbb{R}^n$ that solves Equation (\ref{PEQN}).
If $T$ is the time of first blowup, then the closed set $S_T$ has Hausdorff dimension at most $n+2-4\alpha$.
\end{theorem}

The next subsection is devoted to proving Theorem \ref{PRTHM}. We remark that if we take
$$B(u,v) = \frac{1}{2}\bigg( (u\cdot \nabla)v +(v\cdot \nabla u)\bigg)$$
then we have the usual $\alpha$-dissipative Navier-Stokes equations in Equation (\ref{PEQN}). Subsequently, we get Corollary \ref{CORAL}.

We use that our solution $u$ is smooth up to time $T$ and that the initial data $u_0$ is Schwartz, facts we shall use frequently and without reference. Pairing Equation (\ref{PEQN}) with $u$ and using the cancellation identity in Equation (\ref{Cancel}), we get:
\begin{equation}\label{NEEDIT}\langle\partial_t u, u \rangle + \langle(-\Delta)^\alpha u , u \rangle =0.\end{equation}
Integrating in time then gives us what is typically known as the energy dissipation law:
\begin{equation}
\label{Conserv}
\| u(t)\|_2^2 = \|u_0\|_2^2 -2\int_0^t \|(-\Delta)^{\alpha/2} u\|_2^2 dt \leq \|u_0\|_2^2 \quad \forall t\in[0,T).\end{equation}
We shall examine the dynamics of the "wavelet coefficients" $u_Q:= \| \phi_{Q,j}P_j f\|_2$, where $\phi_{Q,j}$ is a bump function of type $j$ and $Q$ is a cube at level $j$. To this end, we pair Equation (\ref{PEQN}) with $P_j \phi_{Q,j}^2 P_j u$, and get, after simplification:
\begin{equation}
\label{COEFFODE}
\partial_t \bigg(\frac{1}{2} u_{Q}^2\bigg)= \quad \langle-\hat{B}(u,u),P_j \phi_{Q,j}^2 P_j u \rangle  -\langle(-\Delta)^{\alpha}u,P_j \phi_{Q,j}^2 P_j u \rangle. \end{equation}
We note that the "wavelet coefficients" $u_Q$ are continuous in time, see Lemma \ref{Integral1} and Corollary \ref{Integral2}. In Section 3, we prove two main estimates for the terms on the right-hand-side of Equation (\ref{COEFFODE}). We estimate the "dissipation" term by
\begin{equation}\label{BIGEST1}
\langle(-\Delta)^\alpha u , P_j \phi_{Q,j}^2 P_j u \rangle \quad \geq -K2^{-200j}+K2^{2\alpha j}u_Q^2 -K 2^{(2\alpha-\epsilon)j}\sum_{Q'\in\mathcal{N}^1(Q')}u_{Q'}^2.
\end{equation}
We estimate the "nonlinear" term by
$$\abs{ \langle-\hat{B}(u,u),P_j \phi_{Q,j}^2P_j u\rangle} \leq K2^{j(1+\frac{n\delta}{2}+\frac{n\gamma}{3})}u_{\mathcal{N}^1(Q)}u_{Q}+$$ $$+K\sum_{k=\delta j}^{j-1000/\epsilon} 2^{\frac{nk}{2}+j +\frac{n\gamma j}{3}}u_{Q_k}u_{\mathcal{N}^1(Q)}u_{Q}+ K \sum_{k>j+1000/\epsilon} 2^{nj(\frac{1}{2}-\frac{2\gamma}{3}) +k(n\gamma +1)}u_{Q}\|\phi_{Q_j,j}P_k u\|_2^2 + $$ \begin{equation}\label{BIGEST2}K2^{\frac{(n+2)j}{2}+\frac{n\gamma j}{3}}u_{Q}u^2_{\mathcal{N}^{1000/\epsilon}(Q)}+K2^{-150j}.\end{equation}
We have used the following notation from Section 3. For a cube $Q$ at level $j$ we denote:
$$k<j \Rightarrow Q_k := 2^{(j-k)(1-\epsilon)}Q,\quad k\geq j\Rightarrow Q_k = (1+2^{-\epsilon k})Q.$$ 
The estimates hold for all cubes $Q$ at level $j\geq 0$, and we shall now clarify the role of the symbol $K$.

Throughout the rest of our paper, the symbol $K$ will denote a constant appropriately chosen so that every inequality where it appears is satisfied. The choice of $K$ will be clear at each instance of its use, and it will only depend on the $L^2$ norm of the initial data, the blowup time $T$, the dimension $n\geq 2$, and the fixed constants $\epsilon, \alpha$. Also, we introduce several other parameters that depend on $\alpha$ and the dimension $n\geq 2$: $\mu_0>3000$ that will appear in a forthcoming definition, $p_0>2$ that will be the exponent in an application of the H\"{o}lder Inequality, and the parameters $\gamma, \delta$, which will be used in Section 3 when we estimate the nonlinear term in Equation (\ref{COEFFODE}). Given $\alpha>1/2$ and $n\geq 2$, we shall carefully choose the parameters $\delta, \mu_0, p_0$, and $\gamma$ so that our argument closes.

We now describe the construction of a set $M$ that contains $S_T$, which is the singular set of Theorem \ref{PRTHM}. First we call a cube $Q$ at level $j$ \textbf{mildly bad} if and only if (for some constant $K$ that is independent of $j$ and that we shall later determine):
$$2^{\mu_0j}\int_{T-2^{-\mu_0j}}^T u_{\mathcal{N}^{1000/\epsilon}(Q)}^2dt+\int_0^T \int \sum_{k\geq j} 2^{2\alpha k}\abs{\phi_{Q_j,j}P_{k} u}^2 dxdt \geq K 2^{-(n+2-4\alpha)j-100\epsilon j-\gamma j}.$$
We shall consider the reverse inequality as the hypothesis of an $\epsilon$-regularity statement compatible with the scaling of the equations. In other words, 
$$(2^m)^{2\alpha -1}u((2^m)x, (2^m)^{2\alpha} t)$$
is a solution of Equation (\ref{PEQN}) for any $m\in \mathbb{Z}$, if $u(x,t)$ is a solution of the same equation. The terminology is justified because the "bad" cubes in \cite{KP} are also mildly bad (mildly bad cubes are, however, not necessarily bad). The main difference between our definition of mildly bad cubes and the definition of bad cubes used in \cite{KP} and \cite{O} is the carefully chosen control on the kinetic energy offered by the leftmost term. The addition of this term allows us to conclude an $\epsilon$-regularity-type argument for smoothness away from the mildly bad cubes without the use of a "barrier" argument like in \cite{KP} and \cite{O}.

We define the set $M_j$ to be the union of cubes $Q$ for all mildly bad cubes $Q$ at level $j$.
We first obtain a convenient cover of $M_j$:
\begin{proposition}\label{1cover1} There exists a constant $K$ depending only on $\epsilon$ so that 
for all $j\geq K$, there is a covering $\mathcal{C}_j$ of $M_j$ by cubes at level $j$ such that for some other constant $K$ independent of $j$:
$$N(\mathcal{C}_j ) \leq K 2^{(n+2-4\alpha)j+100\epsilon j+\gamma j}.$$
\end{proposition}
\begin{proof}
Let $\mathcal{A}$ be the collection of cubes $2^{3000/\epsilon}Q$ where $Q$ is mildly bad and at level $j$. By the Vitali Covering Lemma (Lemma \ref{Vitali}), there are disjoint cubes $2^{3000/\epsilon}Q'$ in $\mathcal{A}$ such that the collection of cubes $5\cdot 2^{3000/\epsilon}Q'$ covers $M_j$. However, any cube of sidelength $5\cdot 2^{3000/\epsilon}\cdot 2^{-j(1-\epsilon)}$ can be covered by $K$ cubes at level $j$ ($K$ depending on $\epsilon$ and $n\geq 2$ only), so we are led to define $\mathcal{C}_j$ to be the collection of the cubes at level $j$ that cover the cubes $5\cdot 2^{3000/\epsilon}Q'$. Now since the cubes $Q'$ are mildly bad with cubes $2^{3000/\epsilon}Q'$ disjoint, the unions of all the cubes in each nuclear family $\mathcal{N}^{1000/\epsilon}(Q')$ are also disjoint. Indeed, by Lemma \ref{NUKE} and by choosing $j$ sufficiently large, each such nuclear family is contained in the cube of sidelength $2^{-(j-2000/\epsilon)(1-\epsilon)}$ with the same center as $Q'$, which is strictly contained in the cube $2^{3000/\epsilon}Q'$. By the energy dissipation law and Lemma \ref{finiteband1} we may conclude:
$$N(\mathcal{C}_j)2^{-(n+2-4\alpha)j-100\epsilon j-\gamma j}\leq$$
$$\leq K \sum_{Q'} \bigg(\sum_{k\geq j}2^{2\alpha k} \int_0^T\int \abs{\phi_{Q'_j,j}P_k u}^2 + 2^{\mu_0j}\int_{T-2^{-\mu_0j}}^T u^2_{\mathcal{N}^{1000/\epsilon}(Q')}dt\bigg) \leq$$
$$\leq K \bigg(  \sum_{k\geq j}2^{2\alpha k}\int_0^T \int \sum_{Q'}\abs{\phi_{Q'_j,j}P_{k} u}^2 + 2^{\mu_0j}\int_{T-2^{-\mu_0j}}^T \sum_{Q'}u^2_{\mathcal{N}^{1000/\epsilon}(Q')}dt\bigg) \leq K,$$
which proves the lemma.
\end{proof}
\begin{corollary}\label{1HausBound}
The Hausdorff dimension of the set $M= \limsup_{j\to \infty} M_j$ is bounded by $n+2-4\alpha+100\epsilon + \gamma$.
\end{corollary}
\begin{proof}
This is an application of Lemma \ref{dimcompute} using Proposition \ref{1cover1}.
\end{proof}
Next we demonstrate that 
\begin{equation}\label{GOAL}x\not\in M \Longrightarrow x\not\in S_T\end{equation}
Then $M$ contains the set $S_T$. By Corollary \ref{1HausBound} and the arbitrary smallness of $\epsilon>0$ and $\gamma>0$, we shall have proven Theorem \ref{PRTHM}. Equation (\ref{GOAL}) is all that remains to be proven.

\subsection{Regularity Away from Mildly Bad Cubes}
We prove Equation (\ref{GOAL}).
Since $M=\limsup_{j\to \infty} M_j$, we have the equivalence
$$x\not\in M \Longleftrightarrow \exists j\geq 0 \textrm{ s.t. } \forall k>j \quad x\not\in M_k .$$
We denote $L_j$ the set of points $x$ such that $x\not\in M_k$ for all $k>j$. Since $L_{j'} \subset L_{j}$ whenever $j\geq j'$, it suffices to prove a regularity statement for $x\in L_{j}$ for any $j\geq j_0$ for some large $j_0$ in order to prove a regularity statement about $x$ in the complement of $M$.

Thus, by changing the constant $K$ in the definition of mildly bad cubes appropriately, we may assume that $x\in L_{j}$ for some $j\geq j_0$, where we are free to choose the integer $j_0$. We continue with this assumption for the rest of this subsection, in a manner that will be made explicit later.
\begin{proposition}
\label{1CritReg}
There exists a sufficiently large integer $j_0\geq 0$ such that for any integer $j'> j_0$, for any integer $j\geq \max(j'/\delta^2, j'+5000/\epsilon, (j'+400/\epsilon)/\delta)$, and for any cube $Q$ at level $j$ with $Q\subset L_{j'}$ we have
$$\limsup_{t\to T} u_Q(t) \leq 2^{-50j}.$$
\end{proposition}
\begin{proof}
Assume that the conclusion of the proposition is false. In other words, we assume that for any $j_0$, there exists integers $j'>j_0$ and $j\geq \max(j'/\delta^2, j'+5000/\epsilon, (j'+400/\epsilon)/\delta)$ and there exists a cube $Q$ at level $j$ with $Q\subset L_{j'}$ such that 
\begin{equation}\label{CONTRADICT}\limsup_{t\to T} u_Q(t) > 2^{-50j}.\end{equation}
Henceforth, we may assume that $j' > j_0$ and $j\geq \max(j'/\delta^2, j'+5000/\epsilon, (j'+400/\epsilon)/\delta)$ are the smallest integers so that Equation (\ref{CONTRADICT}) holds.
Our first claim is that Equation (\ref{CONTRADICT}) implies that for any numbers $\lambda>0$ and $\delta>0$ we can choose $j_0$ large enough so that:
\begin{equation}\label{BADDYTWOSHOES}u_Q(T-\delta) \leq \lambda \limsup_{t\to T} u_Q(t).\end{equation}
Indeed, by Lemma \ref{finiteband1} and the smoothness of the velocity field before time $T$, we would otherwise have for some constants $\lambda, \delta>0$ and $K$ depending only on $\delta$
$$2^{-50j}< \limsup_{t\to T} u_Q(t)< \frac{1}{\lambda}u_Q(T-\delta)<\frac{K 2^{-1000j}}{\lambda},$$
which, choosing $j_0$ large enough, would lead to a contradiction. Henceforth, we let $t_m$ be a sequence of times converging to $T$ such that
$$\lim_{m\to \infty} u_Q(t_m) = \limsup_{t\to T} u_Q(t).$$
Without loss of generality we may truncate the sequence $t_m$ so that for all $m$:
\begin{equation}\label{GOODYTWOSHOES} T-2^{-\mu_0j}< t_m\end{equation}

Using the continuity of the energy integral proven in Corollary \ref{Integral2}, Equations (\ref{COEFFODE}), (\ref{CONTRADICT}), and (\ref{BADDYTWOSHOES}), and our estimate on the dissipation term stated in Equation (\ref{BIGEST1}), we have, for $j_0$ large enough,
$$
\int_{t_m}^{T} \big(\langle-\hat{B}(u,u),P_j \phi_{Q,j}^2P_j u\rangle -K2^{2\alpha j}u_Q^2 +K 2^{(2\alpha-\epsilon)j}\sum_{Q'\in\mathcal{N}^l(Q)}u_{Q'}^2\big)dt\geq$$ $$ \geq  K2^{-100j} - K2^{-200j} \geq K2^{-100j}.
$$
Using Lemma \ref{almostneg}, we know that the $2^{(2\alpha-\epsilon)j}$ term is almost negligible in a manner elucidated by Proposition \ref{WHATWENEED}. The latter Proposition allows us to conclude that, for some $0<l<400/\epsilon$ and for a large enough choice of $j_0$, we have
$$
\sum_{Q'\in\mathcal{N}^l(Q)}\int_{t_m}^{T} \langle-\hat{B}(u,u),P_j \phi_{Q,j}^2P_j u\rangle dt \geq K2^{-100j}+K\int_{t_m}^T 2^{2\alpha j}u_{\mathcal{N}^l(Q)}^2 dt.$$
The number of cubes in $\mathcal{N}^l(Q)$ is bounded by a constant depending only on $\epsilon$, thus we also know there must at least be one single cube $Q'\in \mathcal{N}^l(Q)$ such that for a constant $K$ depending only on $\epsilon$:
\begin{equation}\label{1bad00}
\int_{t_m}^T \langle-\hat{B}(u,u),P_j \phi_{Q',j}^2P_j u\rangle dt \geq K2^{-100j}+K\int_{t_m}^T 2^{2\alpha j}u_{\mathcal{N}^l(Q)}^2 dt.\end{equation}
At this step, we use our estimates for the nonlinear term to reach a contradiction. We first note that by our contradiction assumption, we have $Q\subset L_{j'}$. In particular, for every $x\in Q$, which is a cube at level $j\geq \max(j'/\delta^2, j'+5000/\epsilon, (j'+400/\epsilon)/\delta)$, we have that $x\not\in M_k$ for all $k> j'$. Thus, the cube $Q$ and all larger cubes down to level $\delta^2 j$ or $j-5000/\epsilon$ (whichever is smaller) containing $Q$ are not mildly bad. We also claim that the cube $Q'$ is not mildly bad. 

Suppose to the contrary that $Q'$ is mildly bad. On the other hand since $0<l<400/\epsilon$ we can choose $j_0$ large enough, according to Lemma \ref{NUKE}, so that $Q'\cap Q\neq \emptyset$ is guaranteed. Let $x\in Q' \cap Q$, so that $x$ lies inside a mildly bad cube $Q'$ at a level greater than $j-400/\epsilon>j'$ and in addition $x\in L_{j'}$. This is a contradiction, therefore $Q'$ is not mildly bad. We next claim that the larger cubes $Q_k'$ containing $Q'$ are also not mildly bad whenever $\delta j\leq k < j-1000/\epsilon$. Indeed, this follows by the same argument as above since $Q' \subset Q_k$ and since $k-400/\epsilon>\delta j-400/\epsilon > j'$.

Recall that our estimate in Equation (\ref{BIGEST2}) states that for any $\delta>0$:
$$\abs{ \langle-\hat{B}(u,u),P_j \phi_{Q',j}^2P_j u\rangle} \leq K2^{j(1+\frac{n\delta}{2}+\frac{n\gamma}{3})}u_{\mathcal{N}^1(Q')}u_{Q'}+$$ $$+K\sum_{k=\delta j}^{j-1000/\epsilon} 2^{\frac{nk}{2}+j +\frac{n\gamma j}{3}}u_{Q'_k}u_{\mathcal{N}^1(Q')}u_{Q'}+ K \sum_{k>j+1000/\epsilon} 2^{nj(\frac{1}{2}-\frac{2\gamma}{3}) +k(n\gamma +1)}u_{Q'}\|\phi_{Q'_j,j}P_k u\|_2^2 +$$ $$+K2^{\frac{(n+2)j}{2}+\frac{n\gamma j}{3}}u_{Q'}u^2_{\mathcal{N}^{1000/\epsilon}(Q')}+K2^{-150j}.$$

We shall use the definition of mildly bad cubes to appropriately bound the terms above and achieve a contradiction.
For the rest of the proof we denote $Q_1$ to be an arbitrary member of the collection $\mathcal{N}^1(Q')$. Since $0<l<400/\epsilon$, we can argue as before, again using Lemma \ref{NUKE}, to say that $Q_1$ is not mildly bad. Let $Q_2$ be a cube at level $k$ that is in the collection of cubes $\{Q'_k\}$ such that $\delta j\leq k < j-1000/\epsilon$. The cube $Q_2$ is a rescaling of $Q'$ that contains $Q'$. We have previously shown that $Q_2$ cannot be mildly bad. 

Now we first aim to bound the contribution from the "high-low" terms. We observe that since $\alpha>1/2$, we may choose $\delta>0$ and $\gamma>0$ small enough so that $2^{j(1+\frac{n\delta}{2} +\frac{n\gamma}{3})}u_{Q_1}\leq K2^{2\alpha j}u_{\mathcal{N}^1(Q')}$. It follows that 
\begin{equation}\label{1bad1}\int_{t_m}^TK2^{j(1+\frac{n\delta}{2}+\frac{n\gamma}{3})}u_{\mathcal{N}^1(Q')}u_{Q'}dt\leq \int_{t_m}^TK2^{2\alpha j}u_{\mathcal{N}^1(Q')}^2dt.\end{equation}

We are going to estimate the following integral:
$$\int_{t_m}^T 2^{\frac{nk}{2}+j+\frac{n\gamma j}{3}}u_{Q_1}u_{Q_2}dt.$$
Since $Q_1$ and $Q_2$ are both not mildly bad, we have
$$\int_{t_m}^T 2^{nk}u_{Q_2}^2 dt \leq K 2^{(2\alpha-2)k-100\epsilon k -\gamma k }$$
using the bound on the "dissipation" and 
$$\int_{t_m}^T 2^{2j+\frac{2n\gamma j}{3}}u_{Q_1}^2 \leq K 2^{(4\alpha -n)j+\frac{\gamma(2n-3)j}{3} -100\epsilon j}(T-t_m)$$
using the bound on the "kinetic energy". Thus, by the Cauchy-Schwarz inequality we have:
$$\int_{t_m}^T 2^{\frac{nk}{2}+j+\frac{n\gamma j}{3}}u_{Q_1}u_{Q_2}dt \leq K2^{(1-\alpha+50\epsilon +\frac{\gamma}{2})(j-k)+j(3\alpha-\frac{n+2}{2}+\frac{\gamma(2n-6)}{6}-100\epsilon)}(T-t_m)^{1/2}.$$
We use that $1000/\epsilon<j-k\leq (1-\delta)j$ and let $\gamma, \epsilon$ sufficiently small and $\mu_0$ sufficiently large to conclude that for some constant $K$ depending only on $\epsilon$ and $n$:
$$\int_{t_m}^T 2^{\frac{nk}{2}+j+\frac{n\gamma j}{3}}u_{Q_1}u_{Q_2}dt \leq K 2^{-140j}.$$
Then, using the fact that $u_{Q'}\leq K$, summing over $k$ and $\mathcal{N}^1(Q')$, and since there are only around $j$ terms:
\begin{equation}\label{1bad2}
\sum_{k=\delta j}^{j-1000/\epsilon}\int_{t_m}^T 2^{\frac{nk}{2}+j+\frac{n\gamma j}{3}}u_{\mathcal{N}^1(Q')}u_{Q_k'}u_{Q'}dt \leq K j 2^{-140j}
\end{equation}
For the "local" frequencies, we use $u_{Q'}\leq K$, the definition of the mildly bad cubes, and a choice of $\mu_0$ sufficiently large depending on $n$ and $\alpha$ to conclude:
$$
\int_{t_m}^T 2^{\frac{(n+2)j}{2}+\frac{n\gamma j}{3}}u_{Q'}u^2_{\mathcal{N}^{1000/\epsilon}(Q')}dt \leq$$ \begin{equation} \label{1bad3}\leq K \int_{t_m}^T2^{\frac{(n+2)j}{2}+\frac{n\gamma j}{3}}u^2_{\mathcal{N}^{1000/\epsilon}(Q')}dt\leq K (T-t_m)2^{-(n+2-4\alpha)j+(\frac{n+2}{2}+\frac{n\gamma}{3})j}\leq  K 2^{-140j}.   
\end{equation}
We now deal with the "high-high" frequency term. By the H\"{o}lder inequality, the Minkowski inequality for integrals, and since our sums are absolutely convergent, we have:
$$2^{nj(\frac{1}{2}-\frac{2\gamma}{3})} \int_{t_m}^T u_{Q'}\bigg( \sum_{k>j+1000/\epsilon}2^{k(n\gamma+1)}\|\phi_{Q'_j,j}P_k u\|_2^2\bigg)dt \leq$$ $$\leq 2^{nj(\frac{1}{2}-\frac{2\gamma}{3})}\sum_{k\geq j+1000/\epsilon} 2^{k(\gamma n+1)} \bigg(\int_{t_m}^T u_{Q'}^{p_0} dt\bigg)^{\frac{1}{p_0}}\bigg(\int_{t_m}^T\|\phi_{Q'_j,j}P_k u\|_2^{\frac{2p_0}{p_0-1}} dt\bigg)^{\frac{p_0-1}{p_0}}.$$
Using the energy dissipation law and our assumption that $p_0>2$, we can perform some crude estimates (e.g. $u_Q^{p_0} \leq K u_Q^2$) to get:
$$2^{nj(\frac{1}{2}-\frac{2\gamma}{3})}\sum_{k\geq j+1000/\epsilon} 2^{k(\gamma n+1)} \bigg(\int_{t_m}^T u_{Q'}^{p_0} dt\bigg)^{\frac{1}{p_0}}\bigg(\int_{t_m}^T\|\phi_{Q'_j,j}P_k u\|_2^{\frac{2p_0}{p_0-1}} dt\bigg)^{\frac{p_0-1}{p_0}}\leq $$
$$\leq K2^{nj(\frac{1}{2}-\frac{2\gamma}{3})}\sum_{k\geq j+1000/\epsilon} 2^{k(\gamma n+1)} \bigg(\int_{t_m}^T u_{Q'}^{2} dt\bigg)^{\frac{1}{p_0}}\bigg(\int_{t_m}^T\|\phi_{Q'_j,j}P_k u\|_2^{2} dt\bigg)^{\frac{p_0-1}{p_0}}.$$
Since $Q'$ is not mildly bad, we get:
$$2^{nj(\frac{1}{2}-\frac{2\gamma}{3})} \int_{t_m}^T u_{Q'}\bigg( \sum_{k>j+1000/\epsilon}2^{k(\gamma n+1)}\|\phi_{Q'_j,j}P_k u\|_2^2\bigg)dt \leq$$
$$\leq K2^{nj(\frac{1}{2}-\frac{2\gamma}{3})}\sum_{k\geq j+1000/\epsilon} 2^{k(\gamma n+1)} 2^{\frac{-\mu_0j}{p_0}} 2^{\frac{(-2\alpha k)(p_0-1)}{p_0}} 2^{\frac{-(n+2-4\alpha)(p_0-1)j}{p_0}}.$$
By Lemma \ref{Algebra}, the infinite sum is convergent and well bounded. In fact we have:
\begin{equation} \label{1bad4} 2^{nj(\frac{1}{2}-\frac{2\gamma}{3})} \int_{t_m}^T u_{Q'}\bigg( \sum_{k>j+1000/\epsilon}2^{k(\gamma n+1)}\|\phi_{Q'_j,j}P_k u\|_2^2\bigg)dt \leq K2^{-140j}.\end{equation}
Using Equation (\ref{1bad00}), our estimates in Equations (\ref{1bad1}), (\ref{1bad2}), (\ref{1bad3}), (\ref{1bad4}), and our estimation of the nonlinear term, we may conclude that
$$2^{-100j}+K\int_{t_m}^T 2^{2\alpha j}u_{\mathcal{N}^l(Q)}^2 dt \leq \int_{t_m}^T 2^{2\alpha j}u_{\mathcal{N}^l(Q)}^2 dt  + Kj 2^{-140j}$$
which is a contradiction for $j>j_0$ and $j_0$ large enough.
\end{proof}
A consequence of Proposition \ref{1CritReg} is:
\begin{corollary}\label{DOESIT}
Equation (\ref{GOAL}) is true.
\end{corollary}
\begin{proof}
Let $x\notin M$. Then there exists $j'\geq j_0$ (where $j_0$ is the number found while proving Proposition \ref{1CritReg}) and $j\geq \max(j'/\delta^2, j'+5000/\epsilon, (j'+400/\epsilon)/\delta)$ depending on $x$ such that $x$ lies in a cube $Q$ at level $j$ satisfying $Q\subset L_{j'}$. By Proposition \ref{1CritReg} we know that 
$$\limsup_{t\to T} u_Q(t) < 2^{-50j}.$$
Thus, there exists a $\delta_1>0$ depending only on $j$, which depends only on $x$, such that
$$\sup_{t\in [T-\delta_1, T]} u_Q(t) < 3\cdot 2^{-50j}.$$
In particular, by Lemma \ref{finiteband1}, Lemma \ref{EMBED}, and since the number $50$ that appears above can be made arbitrarily large in the course of our proofs, we conclude there exists $r>0$ depending only on $x$ such that
$$x\in L^\infty_tC^\infty_x(\mathcal{P}_r(x,T)),$$
which proves that $x\not\in S_T$. What we have shown is equivalent to proving Equation (\ref{GOAL}).
\end{proof}
Finally, Corollary \ref{DOESIT} directly implies Theorem \ref{PRTHM}.

\section{Estimates for the Local Energy}
\subsection{Estimate for the Dissipation Term}
We first consider the dissipation term in Equation (\ref{COEFFODE}).
\begin{proposition}
\label{dissipationestimate}
Let $Q$ be a cube at level $j\geq 0$. We have:
$$\langle(-\Delta)^\alpha u , P_j \phi_{Q,j}^2 P_j u \rangle \quad \geq -K2^{-200j}+K2^{2\alpha j}u_Q^2 -K 2^{(2\alpha-\epsilon)j}\sum_{Q'\in\mathcal{N}^1(Q')}u_{Q'}^2$$
\end{proposition}
\begin{proof}
We have:
 \begin{equation}
\label{32EQN}
\langle(-\Delta)^\alpha u , P_j \phi_{Q,j}^2 P_j u \rangle =\langle(-\Delta)^\alpha \phi_{Q,j}P_j u , \phi_{Q,j}P_j u \rangle+\langle[(-\Delta)^\alpha,\phi_{Q,j}P_j] u , \phi_{Q,j} P_j u \rangle.\end{equation}
It remains to estimate the two terms on the right-hand-side of Equation (\ref{32EQN}).
We begin by using Lemma \ref{commutator1} to state:
\begin{equation}\label{300EQN}\abs{\langle(-\Delta)^\alpha \phi_{Q,j}P_j u , \phi_{Q,j}P_j u \rangle -\langle (-\Delta)^\alpha \tilde{P}_j \phi_{Q,j} P_j u, \phi_{Q,j}P_j u \rangle } \leq K 2^{-400j}\end{equation}
and
\begin{equation}\label{3000EQN}\abs{\langle \phi_{Q,j}P_j u , \phi_{Q,j}P_j u \rangle -\langle  \tilde{P}_j \phi_{Q,j} P_j u, \phi_{Q,j}P_j u \rangle } \leq K 2^{-400j}.\end{equation}
Then, because of the localization in frequency space, we have
$$\langle (-\Delta)^\alpha \tilde{P}_j \phi_{Q,j} P_j u, \phi_{Q,j}P_j u \rangle =$$ \begin{equation}\label{400EQN}=\int \abs{\xi}^{2\alpha}\tilde{p}_j(\xi)\abs{\mathcal{F}(\phi_{Q,j}P_j u)(\xi)}^2d\xi \geq K 2^{2\alpha j}\langle \tilde{P}_j \phi_{Q,j}P_j u, \phi_{Q,j}P_j u\rangle.\end{equation}
Using Equation (\ref{300EQN}), Equation (\ref{3000EQN}), and Equation (\ref{400EQN}) we conclude that
\begin{equation}
\label{Step1}\langle(-\Delta)^\alpha \phi_{Q,j}P_j u , \phi_{Q,j}P_j u \rangle \quad \geq -K2^{-200j} + 2^{2\alpha j} u_Q^2.\end{equation}
Now we turn to the second term on the right-hand-side of Equation (\ref{32EQN}). 

First, by Lemma \ref{pseudoproduct}, we have $[(-\Delta)^\alpha,\phi_j P_j] \in OPS^{2\alpha-\epsilon}_{1,1-\epsilon}$. Thus, the symbol of this operator, which we denote by $q(x,\xi)$, satisfies
$$\abs{q(x,\xi)} \leq K((1+\abs{\xi}^2)^{1/2})^{2\alpha-\epsilon}.$$
Because of this bound and the localization of frequency, we get, proceeding as above:
$$\abs{\langle[(-\Delta)^\alpha,\phi_jP_j] u , \phi_{Q,j} P_j u \rangle}  \leq K (2^{(2\alpha-\epsilon)j}+1)\abs{\langle u , \phi_{Q,j} P_j u \rangle}\leq K 2^{(2\alpha-\epsilon)j}\abs{\langle u , \phi_{Q,j} P_j u \rangle}.$$
Letting $\tilde{Q}=(1+2^{-\epsilon j})Q$, we conclude using Lemma \ref{commutator1} (to commute Paley-Littlewood projections), Lemma \ref{bumpcommute} (to commute bump functions), the energy dissipation law, and the Cauchy-Schwartz inequality that:
$$\abs{\langle[(-\Delta)^\alpha,\phi_jP_j] u , \phi_{Q,j} P_j u \rangle} \leq  K 2^{(2\alpha-\epsilon)j}\abs{\langle u,\phi_{\tilde{Q},j}\tilde{P}_j \phi_{Q,j}P_j u \rangle} +K2^{-150j}\leq $$
$$\leq K 2^{(2\alpha-\epsilon)j}\|\phi_{\tilde{Q},j}\tilde{P}_j u\|_2\cdot u_{Q}+K2^{-150j} \leq K 2^{(2\alpha-\epsilon)j}\|\phi_{\tilde{Q},j}\tilde{P}_j u\|_2^2+K2^{-150j} .$$
Finally using Lemma \ref{collectbound}, we have
\begin{equation}
\label{Step2}
\abs{\langle[(-\Delta)^\alpha,\phi_jP_j] u , \phi_{Q,j} P_j u \rangle }\leq  K2^{(2\alpha-\epsilon)j}\sum_{Q'\in\mathcal{N}^1(Q')}u_{Q'}^2+K2^{-150j}.\end{equation}
Combining Equations (\ref{32EQN}), (\ref{Step1}), and (\ref{Step2}) yields the desired inequality.
\end{proof}
There is a similar inequality that holds for nuclear families.
\begin{corollary}
\label{dissipationestimate2}
Let $Q$ be a cube at level $j$. Let $l$ be arbitrary. Then:
$$\sum_{Q'\in\mathcal{N}^l(Q)}\langle(-\Delta)^\alpha u , P_j \phi_{Q',j}^2 P_j u \rangle \quad \geq -K2^{-200j}+K2^{2\alpha j}u_{\mathcal{N}^l(Q)}^2 -K 2^{(2\alpha-\epsilon)j}u_{\mathcal{N}^{l+1}(Q)}^2$$
\end{corollary}
\begin{proof}
The corollary is obtained by summing the inequality in Proposition \ref{dissipationestimate} over the cubes in $\mathcal{N}^l(Q)$. We just need to recall that $N(\mathcal{N}^l(Q))\leq 2^{13l}$ and that the exponent of the error term can be made arbitrarily negative.
\end{proof}
The term with coefficient $2^{(2\alpha-\epsilon)j}$ is also negligible in a certain sense that we make clear in the following lemma.
\begin{lemma}
\label{almostneg}
For any $K_0>0$ there exists $j_0\in \mathbb{Z}$ and $0<l<400/\epsilon$ such that for any $j\geq j_0$, for any arbitrary interval of time $J\subset [0,T]$, and for any cube $Q$ at level $j$, we have:
$$\int_J u^2_{\mathcal{N}^l(Q)} +2^{-200j} \geq K_0 2^{-\epsilon j} \int_J u^2_{\mathcal{N}^{l+1}(Q)}.$$
\end{lemma}
\begin{proof}
By the energy dissipation law we have $\int_J u^2_{\mathcal{N}^{l+1}(Q)} \leq K$ for a constant that is uniform in choice of $l$, $J$, $Q$, and $j$. In fact, $K$ depends only on the initial vector field and time $T$. Suppose the statement of the lemma is false. Then there exists $K_0$ such that for any $j_0\in \mathbb{Z}$ and for all $l<400/\epsilon$ there exists a cube $Q$ at level $j\geq j_0$ such that:
$$\int_J u^2_{\mathcal{N}^l(Q)} +2^{-200j} < K_0 2^{-\epsilon j} \int_J u^2_{\mathcal{N}^{l+1}(Q)}<K_0 K2^{-\epsilon j}.$$
Thus, with a recursion argument we see that there exists a cube $Q$ at level $j\geq j_0$ such that
$$\int_J u^2_{\mathcal{N}^2(Q)} +2^{-200j} < K_0 K 2^{-399j}$$
which is a contradiction, if $j_0$ be large enough.
\end{proof}
Combining Corollary \ref{dissipationestimate2} and Lemma \ref{almostneg} yields:
\begin{proposition}
\label{WHATWENEED}
There exists a constant $K>0$, $j_0\in \mathbb{Z}$ and $0<l<400/\epsilon$ such that for any $j\geq j_0$, for any arbitrary interval of time $J\subset [0,T]$, and for any cube at level $j$, we have:
$$\sum_{Q'\in\mathcal{N}^l(Q)}\int_J \langle(-\Delta)^\alpha u , P_j \phi_{Q',j}^2 P_j u \rangle \quad \geq -K2^{-200j}+K2^{2\alpha j}\int_J u_{\mathcal{N}^l(Q)}^2 $$
\end{proposition}
\subsection{Estimate for the Nonlinear Term}
The next term we consider from Equation (\ref{COEFFODE}) is 
$$\langle-\hat{B}(u,u),P_j \phi_{Q,j}^2 P_j u \rangle \quad=\quad\langle-\phi_{Q,j}P_j \hat{B}(u,u),\phi_{Q,j}P_j u \rangle .$$
We shall use a scheme for estimating multilinear functions that is inspired by the Bony-Coifman-Meyer paradifferential calculus (for the details of which, see \cite{BONY}, \cite{COIF}, and \cite{TA}). It involves the following partition of frequency space, which is also used by the authors of \cite{KP} and \cite{O}. We may write
$$P_j\hat{B}(u,u) = H_{j,lh}+H_{j,hl}+H_{j,hh}+H_{j,loc}$$
where
$$\textrm{low-high frequncies:}\quad H_{j,lh} = \sum_{k<j-1000/\epsilon} P_j \hat{B}(P_k u, \tilde{P}_j u)$$
$$\textrm{high-low:}\quad H_{j,hl} = \sum_{k<j-1000/\epsilon} P_j \hat{B}(\tilde{P}_j u, P_k u)$$
$$\textrm{high-high:}\quad H_{j,hh} = \sum_{k>j+1000/\epsilon} P_j \hat{B}(\tilde{P}_k u, P_k u)+\sum_{k>j+1000/\epsilon} P_j \hat{B}(P_k u,\tilde{P}_k u)$$
$$\textrm{local:}\quad H_{j,loc} = \sum_{j-1000/\epsilon<k<j+1000/\epsilon} P_j \hat{B}(\tilde{P}_k u, P_k u)+\sum_{j-1000/\epsilon<k<j+1000/\epsilon} P_j \hat{B}(P_k u, \tilde{P}_k u).$$
For the rest of this subsection we fix a cube $Q$ at level $j$ and introduce the following notation:
$$k<j \Rightarrow Q_k := 2^{(j-k)(1-\epsilon)}Q,\quad k\geq j\Rightarrow Q_k = (1+2^{-\epsilon k})Q.$$ 
Also, we may choose $\epsilon$ conveniently so that $1000/\epsilon$ is always an integer.
We first estimate the low-high frequency terms:
\begin{lemma}
\label{lowhigh}
There exists a constant $K>0$ such that for any arbitrary $\delta>0$ and $j\in\mathbb{Z}^+$ we have:
$$\abs{\langle-\phi_{Q,j}H_{j,lh},\phi_{Q,j}P_j u  \rangle } \leq$$ $$K2^{j(1+\frac{n\delta}{2}+\frac{n\gamma}{3})}u_{\mathcal{N}^1(Q)}u_Q+K\sum_{k=\delta j}^{j-1000/\epsilon} 2^{\frac{nk}{2}+j +\frac{n\gamma j}{3}}u_{Q_k}u_{\mathcal{N}^1(Q)}u_Q+K2^{-150j}.$$
\end{lemma}
\begin{proof}
We first deal with the case when $\delta j \leq k <j-1000/\epsilon$ and consider a term in the sum $H_{j,lh}$:
$$\langle-\phi_{Q,j}P_j \hat{B}(P_k u, \tilde{P}_j u), \phi_{Q,j}P_j u \rangle .$$
Now by an application of Lemma \ref{commutator1}, we know that for any $f\in L^2$:
\begin{equation}\label{firstthing}\|\phi_{Q,j}P_jf-\tilde{P}_j\phi_{Q,j}P_jf\|_2 \leq K2^{-200j}.\end{equation}
Since $B$ is an amenable bilinear operator, we have by assumption that $B^1_v(u)=\hat{B}(u,v)$ and $B^2_v(u)=\hat{B}(v,u)$ are pseudodifferential operators with symbols in some class $S^m_{1,1}$. Thus by an application of Lemma \ref{quantbound}, also using Equation (\ref{firstthing}), we see:
\begin{equation} \label{secondthing} \|\phi_{Q_j,j}P_j\hat{B}(P_ku,\tilde{P}_j u)-\phi_{Q_j,j}P_j\hat{B}(\phi_{Q_k,k}P_k u,\tilde{P}_ju)\|_2\leq K2^{-200j}.\end{equation}
We may therefore write
$$\langle-\phi_{Q,j}P_j \hat{B}(P_k u, \tilde{P}_j u), \phi_{Q,j}P_j u \rangle  \quad =\quad \langle\phi_{Q,j}P_j\hat{B}(\phi_{Q_k,k}P_k u, \tilde{P}_j u),\phi_{Q,j}P_j u \rangle +R_{j,k}$$
where $R_{j,k}$ is an error term with absolute value less than $K2^{-195j}$ for all $k$. In a similar fashion, we can commute bump functions to get:
$$\abs{\langle-\phi_{Q,j}P_j \hat{B}(P_k u,\tilde{P}_j u), \phi_{Q,j}P_j u \rangle } \leq$$
\begin{equation}
\label{est0} \abs{\langle\phi_{Q,j}P_j\hat{B}(\phi_{Q_k,k}P_k u, \phi_{Q_j,j}\tilde{P}_j u),\phi_{Q,j}P_j u \rangle }+K2^{-180j}.\end{equation}
We therefore have the estimate:
$$\abs{\langle\phi_{Q,j}P_j\hat{B}(\phi_{Q_k,k}P_k u, \phi_{Q_j,j}\tilde{P}_j u),\phi_{Q,j}P_j u \rangle }\leq K \|\hat{B}(\phi_{Q_k,k}P_k u, \phi_{Q_j,j}\tilde{P}_j u)\|_2 \cdot u_Q.$$
Since $B$ is an amenable bilinear operator, we use the property in Equation (\ref{amenineq}) and get
$$\abs{\langle\phi_{Q,j}P_j\hat{B}(\phi_{Q_k,k}P_k u, \phi_{Q_j,j}\tilde{P}_j u),\phi_{Q,j}P_j u \rangle }\leq$$ \begin{equation}\label{est} u_Q\cdot \|\phi_{Q_k,k}P_k u\|_{\infty}(\|\nabla\phi_{Q_j,j}\tilde{P}_j u\|_{q(\gamma)}+\|\phi_{Q_j,j}\tilde{P}_j u\|_{q(\gamma)}),\end{equation}
where $q(\gamma):= 6/(3-2\gamma)$. For the rest of this proof we refer to $q(\gamma)$ by $q$ only. Now, we use Lemma \ref{ineq2} to get:
\begin{equation}\label{est1}\|\phi_{Q_k,k}P_k u\|_\infty \leq K 2^{nk/2}u_{Q_k} + K2^{-(250/\delta)k}\leq K 2^{nk/2}u_{Q_k} + K2^{-250j}\end{equation}
\begin{equation}\label{est2}\|\phi_{Q_j,j}\tilde{P}_j u\|_q \leq K2^{nj(1/2-1/q)}u_{\mathcal{N}^1(Q)}+K2^{-250j}=K2^{\frac{n\gamma j}{3}}u_{\mathcal{N}^1(Q)}+K2^{-250j}.\end{equation}
We recall that derivatives of a Sobolev-smoothing operator remain Sobolev smoothing, so we may use the "commutator" lemmas of Section 5 "within derivatives" as well. Now, to deal with the gradient term, we use $\tilde{P}_jP_j=P_j$ and Lemma \ref{commutator1} to commute Littlewood-Paley projections and get:
$$\nabla(\phi_{Q_j,j}\tilde{P}_j u ) = \nabla(P_j \phi_{Q_j,j}\tilde{P}_j u ) +R_j = P_j \nabla (\phi_{Q_j, j} \tilde{P}_j u) +R_j,$$ 
where $R_j$ is an error term of absolute value at most $K2^{-300j}$. By Lemma \ref{finiteband1}, we thus have
\begin{equation}\label{est3}\|\nabla(\phi_{Q,j}\tilde{P}_j u )\|_q \leq 2^{j+\frac{n\gamma j}{3}}u_{\mathcal{N}^1(Q)}+K2^{-200j}.\end{equation}


Combining our estimates in Equations (\ref{est0}), (\ref{est}), (\ref{est1}), (\ref{est2}), (\ref{est3}), we get:
$$\abs{\langle-\phi_{Q,j}P_j \hat{B}(P_k u,\tilde{P}_j u), \phi_{Q,j}P_j u \rangle } \leq 2^{nk/2+j+\frac{n\gamma j}{3}}u_{Q_k}u_{\mathcal{N}^1(Q)}u_Q +K2^{-150j}.$$
Summing over $\delta j\leq k<j-1000/\epsilon$ and recognizing that there are only around $j$ terms gets us the first term with the sum.
For the case when $k<\delta j$, we directly use the inequality satisfied by $\hat{B}$, our commutator estimates from Section 5, and the energy dissipation law to get:
$$\abs{\langle-\phi_{Q,j}P_j \hat{B}(P_k u,\tilde{P}_j u), \phi_{Q,j}P_j u \rangle } \leq $$
$$\leq2^j u_Q \|P_k u\|_\infty \|\phi_{Q_j,j}\tilde{P}_ju\|_q+K2^{-150j}\leq K2^{j+n\delta j/2+\frac{n\gamma j}{3}}u_Q u_{\mathcal{N}^1(Q)}+K2^{-150j},$$
which finishes the proof.
\end{proof}
By symmetry of $\hat{B}$, the same result holds for the high-low terms:
\begin{lemma}
\label{highlow}
There exists a constant $K>0$ such that for any arbitrary $\delta>0$ and $j\in\mathbb{Z}^+$ we have:
$$\abs{\langle-\phi_{Q,j}H_{j,hl},\phi_{Q,j}P_j u  \rangle } \leq$$ $$ K2^{j(1+\frac{n\delta}{2}+\frac{n\gamma}{3})}u_{\mathcal{N}^1(Q)}u_Q+K\sum_{k=\delta j}^{j-1000/\epsilon} 2^{\frac{nk}{2}+j +\frac{n\gamma j}{3}}u_{Q_k}u_{\mathcal{N}^1(Q)}u_Q+K2^{-150j}.$$
\end{lemma}
Our next effort is towards estimating the high-high frequency terms. 

\begin{lemma}
\label{highhigh}
For some constant $K$ and for all $j\in \mathbb{Z}^+$ we have
$$\abs{\langle \phi_{Q,j}H_{j, hh} , \phi_{Q,j}P_j u\rangle }\leq K \sum_{k>j+1000/\epsilon}  2^{nj(\frac{1}{2}-\frac{2\gamma}{3})+\gamma j +k(n\gamma +1)}u_Q\|\phi_{Q_j,j}P_k u\|_2^2 + K2^{-150j}.$$
\end{lemma}
\begin{proof}
As usual, we examine and wish to estimate a single term in the high-high sum where $k>j+1000/\epsilon$, which by symmetry we may write:
$$\langle \phi_{Q,j}P_j\hat{B}(\tilde{P}_k u, P_k u), \phi_{Q,j}P_j u\rangle.$$
As in the proof of Lemma \ref{lowhigh}, we can commute bump functions across our operators to get
$$\abs{\langle \phi_{Q,j}P_j\hat{B}(\tilde{P}_k u, P_k u), \phi_{Q,j}P_j u\rangle}\leq$$ \begin{equation}
\label{bound0} K\abs{\langle \phi_{Q,j}P_j\hat{B}(\phi_{Q_j,j}\tilde{P}_k u, \phi_{Q_j,j}P_k u), \phi_{Q,j}P_j u\rangle}+K2^{-200j}.\end{equation}
Since $B$ is amenable we can bound the trilinear expression:
$$
\abs{\langle \phi_{Q,j}P_j\hat{B}(\phi_{Q_j,j}\tilde{P}_k u, \phi_{Q_j,j}P_k u), \phi_{Q,j}P_j u\rangle}\leq$$ $$ (\|\phi_{Q_j,j}\tilde{P}_k u\|_{p_1}+\|\nabla\phi_{Q_j,j}\tilde{P}_k u\|_{p_1})\|\phi_{Q_j,j}P_k u\|_{p_2} \|\phi_{Q,j}P_j u\|_r 
$$
where
$$\frac{1}{p_1}+\frac{1}{p_2}+\frac{\gamma}{3}+\frac{1}{r}=1.$$
Henceforth, we can always assume that $0<\gamma<1/2$. With this assumption, we can pick $p_1,p_2,r$ so that
$$\frac{1}{p_1}=\frac{1}{2},\quad \frac{1}{p_2} = \frac{1}{2}-\gamma,\quad \frac{1}{r}= \frac{2\gamma}{3}$$
By Lemma \ref{ineq2} we know that
\begin{equation}\label{bound1}
\|\phi_{Q,j}P_j u\|_r \leq 2^{nj(\frac{1}{2}-\frac{2\gamma}{3})}u_Q +K2^{-200j}\end{equation}
and
\begin{equation}\label{bound2}
\|\phi_{Q_j,j}P_k u\|_{p_2} \leq 2^{nk\gamma}\|\phi_{Q_j,j}P_k\|_2+K2^{-200j} .\end{equation}
By the same reasoning as in the proof of Lemma \ref{lowhigh}, we get
\begin{equation}\label{bound3}
\|\nabla \phi_{Q_j,j}P_k u\|_2 \leq K2^k \sum_{l=-2}^{l=2} \|\phi_{Q_j,j}P_{k+l}u\|_2
\end{equation}
Combining our estimates from Equations (\ref{bound0}), (\ref{bound1}), (\ref{bound2}), (\ref{bound3}) yields:
$$\abs{\langle \phi_{Q,j}H_{j, hh} , \phi_{Q,j}P_j u\rangle }\leq K \sum_{k>j+1000/\epsilon} u_Q 2^{nj(\frac{1}{2}-\frac{2\gamma}{3})+k(n\gamma+1)}\|\phi_{Q_j,j}P_k u\|_2^2 + K2^{-150j}.$$
\end{proof}
Next we deal with the "local" frequencies around level $j$:
\begin{lemma}
\label{local}
For some constant $K$ we have for all $j\in \mathbb{Z}^+$:
$$\abs{\langle \phi_{Q,j}H_{j, loc} , \phi_{Q,j}P_j u\rangle }\leq K2^{\frac{(n+2)j}{2}+\frac{n\gamma j}{3}}u_Q u^2_{\mathcal{N}^{1000/\epsilon}(Q)}+K2^{-150j}.$$
\end{lemma}
\begin{proof}
We can explicitly write out the term we wish to bound by
$$\langle \phi_{Q,j}H_{j, loc} , \phi_{Q,j}P_j u\rangle = 2\sum_{l=-2}^{l=2}\sum_{j-1000/\epsilon<k}^{k<j+1000/\epsilon}\langle \phi_{Q,j}P_j\hat{B}(P_{k+l}u,P_ku),\phi_{Q,j}P_j u\rangle.$$
As in the proof of Lemma \ref{lowhigh}, we can use the triangle inequality and commute bump functions to get:
$$
\abs{\langle \phi_{Q,j}H_{j, loc} , \phi_{Q,j}P_j u\rangle} \leq$$ \begin{equation}
\label{bound01} K2^{-200j} + K\sum_{l=-2}^{l=2}\sum_{j-1000/\epsilon<k}^{k<j+1000/\epsilon}\abs{\langle \phi_{Q,j}P_j\hat{B}(\phi_{Q_j,j}P_{k+l}u,\phi_{Q_j,j}P_ku),\phi_{Q,j}P_j u\rangle}
\end{equation}
For appropriate choices of $k=k_0,l=l_0$ that maximize the expression in the sum in Equation (\ref{bound01}), and since our constants $K$ are allowed to depend on $\epsilon$, we have:
$$
K\sum_{l=-2}^{l=2}\sum_{j-1000/\epsilon<k<j+1000/\epsilon}\abs{\langle \phi_{Q,j}P_j\hat{B}(\phi_{Q_j,j}P_{k+l}u,\phi_{Q_j,j}P_ku),\phi_{Q,j}P_j u\rangle}\leq $$ \begin{equation}\label{bound05} \leq K\abs{\langle \phi_{Q,j}P_j\hat{B}(\phi_{Q_j,j}P_{k_0+l_0}u,\phi_{Q_j,j}P_{k_0}u),\phi_{Q,j}P_j u\rangle}
\end{equation}
We remark that by Lemma \ref{collectbound}:
\begin{equation}
\label{bound02}
\|\phi_{Q_j,j}P_k u\|_2 \leq u_{\mathcal{N}^{1000/\epsilon}(Q)}
\end{equation}
and by Lemma \ref{ineq2}:
\begin{equation}
\label{bound03}
\|\phi_{Q,j}P_j u\|_\infty \leq 2^{nj/2}u_Q +K2^{-200j}.
\end{equation}
As in the proof of Lemma \ref{lowhigh}, we have, where $q= 6/(3-2\gamma)$:
\begin{equation}
\label{bound04}
\|\nabla(\phi_{Q_j,j}P_{k_0+l_0} u )\|_q \leq 2^{j+\frac{n\gamma j}{3}}u_{\mathcal{N}^{1000/\epsilon}(Q)}+K2^{-200j}.
\end{equation}
Combining our estimates from Equations (\ref{bound01}), (\ref{bound05}), (\ref{bound02}), (\ref{bound03}), (\ref{bound04}), we get:
$$\abs{\langle \phi_{Q,j}H_{j, loc} , \phi_{Q,j}P_j u\rangle} \leq K2^{\frac{(n+2)j}{2}+\frac{n\gamma j}{3}}u_Q u^2_{\mathcal{N}^{1000/\epsilon}(Q)}+K2^{-150j}.$$
\end{proof}

\section{Partial Regularity and Blowup}
In this section we deal only with functions and vector fields defined on $\mathbb{R}^3$. For this reason, the parameter $n$ does not denote the dimension of space but rather an index of summation.
\subsection{Flexibility of Blowup}
We shall assume that the reader has a familiarity with the paper \cite{T} of Tao. We first give an overview of the construction in \cite{T}. Afterwards, we show the flexibility of the blowup result therein. Indeed, Tao notes himself (in footnote 12 of \cite{T}), that his methods show blowup even when hyperdissipation approaches the critical value of $\alpha=5/4$. In this subsection we give some details showing why this is the case while also modifying the technique to prove the existence of a blowup solution to the pseudodifferential equation associated to some amenable operator. 

As before, we let $\langle, \rangle $ be the $L^2$ pairing for vector fields. Let $1<\lambda<2$. We denote by $H^{10}_{df}(\mathbb{R}^3)$ the space of vector fields on $\mathbb{R}^3$ that are divergence free in the distributional sense and whose first ten weak derivatives are square integrable.

 A basic local cascade operator is a bilinear operator $C: H_{df}^{10}(\mathbb{R}^3) \times H_{df}^{10}(\mathbb{R}^3)\to H_{df}^{10}(\mathbb{R}^3)^\ast$ defined via duality by:
$$\langle C(u,v), w \rangle \hspace{5px} =\sum_{n\in \mathbb{Z}}\lambda^{5n/2} \langle u, \psi_{1,n} \rangle \langle v, \psi_{2,n} \rangle  \langle w, \psi_{3,n} \rangle $$
where for $i\in \{1,2,3\}$,
$$\psi_{i,n}(x) = \lambda^{3n/2}\psi_i(\lambda^n x)$$
and $\psi_i$ is a Schwartz vector field whose Fourier transform is compactly supported within a small spherical shell around the unit sphere in $\mathbb{R}^3$. We also define a local cascade operator to be a finite linear combination of basic local cascade operators. Lastly, we call a basic local cascade operator a zero-momentum basic local cascade operator if its constituent Schwartz vector fields satisfy
$$\int_{\mathbb{R}^3} \psi_i dx=0.$$
Likewise, we may define a zero-momentum local cascade operator.

In \cite{T}, Tao proves the existence of a local cascade equation that admits a solution blowing up in finite time. In particular, Tao proves the following:

\begin{theorem}[\cite{T}]
Let $1<\lambda<2$ be arbitrary. Then there exists a symmetric local cascade operator $C$ and a Schwartz divergence-free vector field $u_0$ such that the cancellation identity holds
\begin{equation}
\langle C(u,u),u \rangle =0 \textrm{ for all } u \in H^{10}_{df}(\mathbb{R}^3)\\
\end{equation}
and there does not exist any global mild solution $u: [0,\infty) \to H^{10}_{df}(\mathbb{R}^3)$ to the initial value problem
\begin{equation}
\begin{split}
\partial_t u -\Delta u +C(u,u)=0\\
u(\cdot,t=0) = u_0
\end{split}
\end{equation}
\end{theorem}

In this subsection we show that Tao's result is flexible in the sense that we find a local cascade operator with more restraints on the constituent Schwartz vector fields $\psi_i$ whose corresponding local cascade equation admits a blowup solution with fractional dissipation. In particular we prove:

\begin{theorem}
\label{BLOWUP}
Let $1<\lambda<2$ and $0<\alpha<5/4$ be arbitrary. Then there exists a zero-momentum symmetric local cascade operator $C$ and a Schwartz divergence-free vector field $u_0$ such that the cancellation identity holds
\begin{equation}
\langle C(u,u),u \rangle =0 \textrm{ for all } u \in H^{10}_{df}(\mathbb{R}^3)\\
\end{equation}
and there does not exist any global mild solution $u: [0,\infty) \to H^{10}_{df}(\mathbb{R}^3)$ to the initial value problem
\begin{equation}
\begin{split}
\partial_t u +(-\Delta)^\alpha u +C(u,u)=0\\
u(\cdot,t=0) = u_0
\end{split}
\end{equation}
\end{theorem}

First, we determine the structure our local cascade operator must have in order to fulfill some basic requirements like the cancellation property. In particular, following Tao, we consider four balls $B_1, B_2, B_3, B_4$ in the spherical shell $\{ \xi : 1<\abs{\xi}\leq \frac{\lambda+1}{2}\}$, and we choose the balls so that the unions $B_i\cup(-B_i)$ are mutually disjoint for $i\in \{1,2,3,4\}$. We choose four zero-momentum divergence-free Schwartz vector fields $\psi_i$ such that $\mathcal{F}(\psi_i)$ has support in $B_i \cup -B_i$, and we normalize them so that $\| \psi_i\|_2 =1$. We have the $L^2$-rescaled functions $\psi_{i,n}:= \lambda^{3n/2}\psi_i(\lambda^n x)$ that also satisfy $\| \psi_{i,n}\|_2=1$. Let $S=\{(0,0,0),(1,0,0),(0,1,0),(0,0,1)\}$. We define the following local cascade operator (in other words, a linear combination of basic local cascade operators):
$$C(u,v):= \sum_{n\in \mathbb{Z}} \sum_{\substack{\hspace{5px}i_1,i_2,i_3\in \{1,2,3,4\} \\(\mu_1,\mu_2,\mu_3)\in S} }a_{i_1,i_2,i_3,\mu_1,\mu_2,\mu_3}\lambda^{5n/2}\langle u,\psi_{i_1,n+\mu_1} \rangle \langle v,\psi_{i_2,n+\mu_2} \rangle \psi_{i_3,n+\mu_3}$$
To ensure that $C$ is a symmetric bilinear operator we require that
$$a_{i_1,i_2,i_3,\mu_1,\mu_2,\mu_3}=a_{i_2,i_1,i_3,\mu_2,\mu_1,\mu_3}.$$
The cancellation property is satisfied if we require
$$\sum_{\{a,b,c\}=\{1,2,3\}} a_{i_a,i_b,i_c,\mu_a,\mu_b,\mu_c}=0.$$
With these first conditions imposed on $C$, we are ready to study the basic dynamics of the pseudodifferential equation associated to $C$. 
Whenever applicable, we define the following Fourier projections that will act as "wavelets":
$$u_{i,n}(t)(x) := \mathcal{F}^{-1}(\chi_{\xi\in \lambda^n (B_i \cup -B_i)}(\xi)\cdot \mathcal{F}(u(t))(\xi))$$
and the following functions of time that behave like "wavelet coefficients":
$$X_{i,n}(t):= \langle u(t),\psi_{i,n} \rangle =\langle u_{i,n}(t),\psi_{i,n} \rangle .$$
Lastly, we define the following "local energies":
$$E_{i,n}(t):= \frac{1}{2}\|u_{i,n}(t)\|_2^2.$$
The behavior of the pseudodifferential equation is determined by the dynamics of the functions $X_{i,n}(t)$. This is described in Lemma 4.1 of \cite{T}. Below, we only note the modifications necessary in the case of fractional dissipation (assuming that $\psi$ is a zero-momentum vector field has no effect on the lemma's conclusions).
\begin{lemma}
\label{devep}
Suppose $u$ is a global mild solution to the pseudodifferential equation associated to $C$ with initial data $\psi_{1,n_0}$ for some $n_0$ integer.
For any $i,n$ we have
$$\partial_t X_{i,n} = \sum_{i_1,i_2 \in \{1,2,3,4\}}\sum_{(\mu_1,\mu_2,\mu_3)\in S} a_{i_1,i_2,i,\mu_1,\mu_2,\mu_3}\lambda^{5(n-\mu_3)/2}X_{i_1,n-\mu_3+\mu_1}X_{i_2,n-\mu_3+\mu_2} + O(\lambda^{2\alpha n}E_{i,n}^{1/2})$$
and
$$\frac{1}{2}X_{i,n}^2(t) \leq E_{i,n}(t) \leq \frac{1}{2}X_{i,n}^2(t) + O\bigg(\lambda^{2\alpha n} \int_0^t E_{i,n}(s)ds\bigg).$$
\end{lemma}
\begin{proof}
The vector field $u(t)$ is a mild solution of the pseudodifferential equation associated to $C$. Thus we write
\begin{equation}\label{mild}u(t) = e^{-t(-\Delta)^\alpha}u_0 +\int_0^t e^{(s-t)(-\Delta)^\alpha}C(u(s),u(s))ds.\end{equation}
As noted before, we have the "wavelet" decomposition of $u$:
$$u(t) = \sum_{n,i} u_{i,n}(t)$$
with
$$X_{i,n}(t) = \langle u(t),\psi_{i,n}\rangle = \langle u_{i,n}(t), \psi_{i,n}\rangle.$$
We pair Equation (\ref{mild}) with $\psi_{i,n}$ and get
$$
u_{i,n}(t) = e^{-t(-\Delta)^\alpha}X_{i,n}(0)\psi_{i,n}+  \psi_{i,n}\bigg(\sum_{i_1,i_2 \in \{1,2,3,4\}, (\mu_1,\mu_2,\mu_3)\in S}  a_{i_1,i_2,i,\mu_1,\mu_2,\mu_3}\lambda^{5(n-\mu_3)/2}$$
$$\int_0^t\langle u,\psi_{i_1,n+\mu_1-\mu_3} \rangle \langle u,\psi_{i_2,n+\mu_2-\mu_3} \rangle \bigg).
$$
If we differentiate with respect to the time variable $t$ and simplify we obtain
$$
\partial_t u_{i,n} = -(-\Delta)^\alpha u_{i,n}+$$ \begin{equation}\label{mildcoeff} \sum_{i_1,i_2\in \{1,2,3,4\}, (\mu_1,\mu_2,\mu_3)\in S}  a_{i_1,i_2,i,\mu_1,\mu_2,\mu_3}\lambda^{5(n-\mu_3)/2} X_{i_1,n-\mu_3+\mu_1} X_{i_2,n-\mu_3+\mu_2}\psi_{i,n}
\end{equation}
Now, by the localization in frequency space, we observe that (where $K$ depends only on $\psi_i$):
\begin{equation}\label{IBP}\abs{\langle -(-\Delta)^\alpha u_{i,n}, \psi_{i,n} \rangle} =\abs{ \langle u_{i,n} , -(-\Delta)^\alpha\psi_{i,n}\rangle} \leq K E_{i,n}^{1/2}(t)\lambda^{2\alpha n }.\end{equation}
Pairing Equation (\ref{mildcoeff}) with $\psi_{i,n}$ and using Equation (\ref{IBP}) gets us the first statement in the lemma.
The first inequality in the second statement follows from Cauchy-Schwarz. It remains to prove the second inequality. We recall the "local energy inequality" found by Tao, which is independent of the dissipation parameter $\alpha$. Equation (4.11) in \cite{T} is 
\begin{equation}\label{LEI}
\partial_t E_{i,n} \leq \sum_{i_1,i_2}\sum_{(\mu_1,\mu_2,\mu_3)\in S} a_{i_1,i_2,i,\mu_1,\mu_2,\mu_3} \lambda^{5(n-\mu_3)/2}X_{i_1,n-\mu_3+\mu_1}X_{i_2,n-\mu_3+\mu_1}X_{i,n}.
\end{equation}
Now, if we multiply the evolution inequality for $X_{i,n}$ by $X_{i,n}$, subtract this from Equation (\ref{LEI}), and integrate in time, we have the second desired inequality
$$E_{i,n}(t) \leq \frac{1}{2}X_{i,n}^2(t) +K\lambda^{2\alpha n} \int_0^t E_{i,n}(s)ds.$$
\end{proof}
The only significant change from the work in \cite{T} is the presence of $\alpha$ in the exponent of $\lambda$ in the dissipation term.

Tao's construction of a blowup solution to the pseudodifferential equation associated to $C$ continues with an intricate analysis of an infinite dimensional ODE system whose dynamics are contained within the dynamics of $C$. Tao proves that there exist a sequence of times $t_m$ converging to some finite $T$ so that for some $\epsilon_0>0$, at least $\lambda^{-\epsilon_0 n}$ of the energy is concentrated in the ball of radius $\lambda^{-n}$ around the spatial origin. 

 An essential part of the blowup construction is that the aforementioned concentration of energy occurs on timescales of order $t_{n+1}-t_m= \lambda^{(-\frac{5}{2}+O(\epsilon_0))n}$ which are signicantly smaller than the dissipation time scale which (for fractional dissipation of order $\alpha$) is $\lambda^{2\alpha n}$ by Lemma \ref{devep}. This can occur so long as $\alpha<5/4$. Since Tao never uses any assumption about the integral of the functions $\psi_{i,n}$, one can follow through Tao's construction and get a blowup solution to the pseudodifferential equation (with hyperdissipation of order $\alpha<5/4$) associated to some zero-momentum cascade operator. This is nothing less than Theorem \ref{BLOWUP} above.

\subsection{Additional Properties of Local Cascade Operators}
We have established Theorem \ref{BLOWUP}, where we constructed a zero-momentum, symmetric, local cascade operator $C(u,v)$ satisfying the cancellation identity for divergence-free vector fields whose associated $\alpha$-dissipative pseudodifferential equation has a solution blowing up in finite time from Schwartz initial data.  In Section 2, we proved a partial regularity result for $\alpha$-dissipative pseudodifferential equations with nonlinearities arising from amenable bilinear operators. It follows that if we desire to use Theorem \ref{PRTHM} and get a partial regularity result for the pseudodifferential equation from Theorem \ref{BLOWUP}, we have to verify that $C(u,v)$ is an amenable bilinear operator. 

That $C(u,v)$ is defined for Schwartz vector fields is clear. That $C(u,v)$ satisfies the cancellation identity 
$$\langle\mathbb{P}C(u,u), u \rangle =0\quad \textrm{ for divergence-free } u$$
is equally clear. Indeed, since $C(u,u)$ is always divergence-free, we have $\mathbb{P}C(u,u)=C(u,u)$. It follows that
$$\langle \mathbb{P}C(u,u),u\rangle=\langle C(u,u),u\rangle=0 \textrm{ for all divergence-free } u.$$
The scaling property, with the choice of scaling parameter $\lambda$, is clearly satisfied by $C$. Tao also makes this observation. Thus, it remains to show that $C^1_v(u)=\mathbb{P}C(u,v)$ is a pseudodifferential operator in some class $OPS^m_{1,1}$ and to show the bilinear operator bound. First we need to prove an auxiliary lemma.

\begin{lemma}
\label{momentuse}
Let $\psi$ be a Schwartz divergence-free vector field that has zero momentum, i.e. $\int \psi =0$. Let $\Psi_i$ be the vector field given by Lemma \ref{DivF} so that $\textrm{div } \Psi_i = \psi_i$, the $i^{th}$ component of $\psi$. Let $\psi_{n}:= \lambda^{3n/2}\psi(\lambda^n x)$ be the $L^2$ rescaling of $\psi$ and let $\Psi_{i,n}$ be the $L^2$ rescaling of $\Psi_i$. Then for any vector field $v\in H^{10}_{df}$, we have:
$$\lambda^n \langle \psi_n, v\rangle =-\sum_i \int \Psi_{i,n}\cdot \nabla v_i $$
\end{lemma}
\begin{proof}
Since $\psi$ has zero momentum, we can use Lemma \ref{DivF} and write
$$\lambda^n \psi_{i,n}(x)= \lambda^{5n/2}\psi_i (\lambda^n x) = \lambda^{5n/2}(\textrm{div } \Psi_i)(\lambda^n x)=\lambda^{3n/2} \textrm{div }(\Psi_i(\lambda^n x))= \textrm{div } \Psi_{i,n}(x).$$
Then we integrate by parts:
$$\lambda^n\langle \psi_n, v\rangle = \sum_i \int \lambda^{5n/2}\psi_{i}(\lambda^n x)v_i dx = \sum_i \int v_i \textrm{div } \Psi_{i,n}dx =-\sum_i \int \Psi_{i,n} \cdot \nabla v_i dx$$
\end{proof}
\begin{proposition}
\label{LPBOUNDS}
Let $C(u,v)$ be a zero-momentum, symmetric local cascade operator. Then, for arbitrary $0<\gamma<1$ and for any $1\leq p_1, p_2, r\leq \infty$ satisfying
$$\frac{1}{p_1}+\frac{1}{p_2}+\frac{\gamma}{3} = \frac{1}{r},$$
we have that the following inequality holds
$$\| C(u,v) \|_{L^r} \leq K \| u\|_{L^{p_1}}(\|v\|_{L^{p_2}} +\| \nabla v\|_{L^{p_2}})$$
for all $u\in L^{p_1}$ and $v\in W^{1,p_2}$.
The constant $K$ depends on $\gamma,\lambda$, and the vector fields $\psi_i$ used in the definition of $C(u,v)$.
\end{proposition}
\begin{proof}
It suffices to prove the proposition for any (zero-momentum, symmetric) basic local cascade operator:
$$C(u,v)=\sum_{n\in \mathbb{Z}}\lambda^{5n/2} \langle u, \psi_{1,n} \rangle \langle v, \psi_{2,n} \rangle  \psi_{3,n}(x).$$
Since the sum above is absolutely convergent, we can divide the operator $C$ into two parts:
$$C^+(u,v) := \sum_{n\geq 0}\lambda^{5n/2} \langle u, \psi_{1,n} \rangle \langle v, \psi_{2,n} \rangle  \psi_{3,n}(x)$$
and $C^-(u,v):= C(u,v) - C^+(u,v)$.
By a change of variables we observe that
\begin{equation}\label{57}\| \psi_{3,n} \|_{L^r} = \lambda^{3n/2} \| \psi_3(\lambda^n x)\|_{L^r} = \lambda^{3n/2-3n/r}\| \psi_3(u)\|_{L^r}=\lambda^{3n/2-3n/r}\| \psi_3\|_{L^r}.\end{equation}
By the absolute convergence of the sum, we may bound each component in turn. We begin with $C^-$, which is the more straightforward term. We observe that
$$\| C^-(u,v)\|_{L^r} =\| \sum_{n< 0}\lambda^{5n/2} \langle u, \psi_{1,n} \rangle \langle v, \psi_{2,n} \rangle  \psi_{3,n}(x)\|_{L^r}\leq$$
$$\leq \sum_{n< 0}\abs{\lambda^{5n/2} \langle u, \psi_{1,n} \rangle \langle v, \psi_{2,n} \rangle }\cdot \|\psi_{3,n}(x)\|_{L^r}\leq$$
$$\leq \|\psi_3\|_{L^r}\sum_{n<0} \lambda^{4n-3n/r}\abs{\langle u,\psi_{1,n} \rangle }\abs{\langle v,\psi_{2,n} \rangle }\leq$$
$$\leq \|\psi_3\|_{L^r}\sum_{n<0} \lambda^{4n-3n/r}\|u\|_{L^{p_1}}\|\psi_{1,n}\|_{L^{p_1'}}\|v\|_{L^{p_2}}\|\psi_{2,n}\|_{L^{p_2'}}$$
where the first inequality is the triangle inequality, the second is due to Equation (\ref{57}), and the third inequality is Holder's inequality, where $p'$ denotes the Holder conjugate exponent of a number $p$. Then, by the same reasoning that justifies Equation (\ref{57}), we can perform a change of variables in the integration and get
$$\|C^-(u,v)\|_{L^r} \leq \|\psi_3\|_{L^r}\|\psi_{1}\|_{L^{p_1'}}\|\psi_{2}\|_{L^{p_2'}}\sum_{n<0} \lambda^{7n-3n(1/r+1/p_1'+1/p_2')}\|u\|_{L^{p_1}}\|v\|_{L^{p_2}}.$$
Now by the hypothesis on our exponents, it follows that
$$\frac{1}{r}+\frac{1}{p_1'}+\frac{1}{p_2'}=\frac{1}{r}+2-\frac{1}{p_1}-\frac{1}{p_2}=2+\frac{\gamma}{3}.$$
Thus, we conclude, since $0<\gamma<1$ and $\lambda>1$:
$$\|C^-(u,v)\|_{L^r} \leq \|\psi_3\|_{L^r}\|\psi_{1}\|_{L^{p_1'}}\|\psi_{2}\|_{L^{p_2'}}\sum_{n<0}\lambda^{n(1-\gamma)}\|u\|_{L^{p_1}}\|v\|_{L^{p_2}}\leq K \|u\|_{L^{p_1}}\|v\|_{L^{p_2}}$$
where $K$ depends only on $\lambda, \gamma, \psi_i$. Now we proceed to estimate $C^+$. As with $C^-$ we can use Holder's inequality, a change of variables in the integration, and Lemma \ref{momentuse} (to exchange $\lambda^n$ for a derivative) to get:
$$\| C^+(u,v)\|_{L^r}\leq  \|\psi_3\|_{L^r}\|\psi_{1}\|_{L^{p_1'}}\sup_i \|\Psi_{2,i}\|_{L^{p_2'}}\sum_{n\geq 0} \lambda^{-\gamma n}\|u\|_{L^{p_1}}\|\nabla v\|_{L^{p_2}}$$
where $\textrm{div } \Psi_{2,i} = \psi_{2,i}$. This is where we use the zero-momentum condition.
Because $\gamma>0$ we conclude that
$$\| C^+(u,v)\|_{L^r}\leq  K\|u\|_{L^{p_1}}\|\nabla v\|_{L^{p_2}}$$
where $K$ depends only on $\lambda, \gamma, \psi_i$. Combining our estimates with one final application of the triangle inequality gets us
$$\| C(u,v)\|_{L^r}\leq  K\|u\|_{L^{p_1}}(\|v\|_{L^{p_2}}+\|\nabla v\|_{L^{p_2}})$$
where $K$ depends only on $\lambda, \gamma, \psi_i$. 
\end{proof}
We note that the conclusion of Proposition \ref{LPBOUNDS} is in fact stronger than what is necessary for $C$ to be amenable. 
To finish showing that $C(u,v)$ is an amenable bilinear operator, it remains to prove that $C^1_v(u)=\mathbb{P}C(u,v)$ is a pseudodifferential operator in some class $S^m_{1,1}$. That $C^2_v(u)$ is also in the same symbol class would follow immediately from the symmetry of $C$. 
\begin{proposition}
Let $C(u,v)$ be a symmetric local cascade operator, then $C^1_v(u)$ is a pseudodifferential operator with symbol in the class $S^{5/2}_{1,1}$.
\end{proposition}
\begin{proof}
It suffices to prove the proposition for basic local cascade operators of the form:
$$C(u,v)=\sum_{n\in \mathbb{Z}}\lambda^{5n/2} \langle u, \psi_{1,n} \rangle \langle v, \psi_{2,n} \rangle  \psi_{3,n}(x)$$
where we only deal with scalar functions. Using the fact that the Fourier transform is a unitary operator on $L^2$ and the absolute convergence of the sum, we write
$$C^1_v(u) = \iint \sum_{n\in \mathbb{Z}} \lambda^{5n/2} \psi_{3,n}(x) \langle v,  \psi_{2,n}\rangle\mathcal{F}(\psi_{2,n})(\xi) \mathcal{F}(u)(\xi) d\xi.$$
Expanding this and putting it into the form of a pseudodifferential operator we write (up to some dimensional constant):
$$C^1_v(u) = \iint \bigg(\sum_{n\in \mathbb{Z}} \lambda^{5n/2} \psi_{3}(\lambda^n x) \langle v,  \psi_{2,n}\rangle\mathcal{F}(\psi_{2})(\lambda^{-n}\xi)e^{-i x\cdot \xi} \bigg) \mathcal{F}(u)(\xi)e^{i x\cdot \xi} d\xi.$$
The infinite sum, however, has only one nonzero term for any given choice of $\xi$, because $\mathcal{F}(\psi_2)$ has nicely chosen compact support. In particular, the only nonzero term for any given $\xi$ is the $n^{th}$ term where $\abs{\xi} \sim \lambda^{n}$. It follows readily then that the symbol of the pseudodifferential operator $C^1_v$, the "infinite" sum in the expression above, is in fact in the symbol class $S^{5/2}_{1,1}$. 
\end{proof}
As a consequence of the two previous propositions as well as Theorems \ref{BLOWUP} and \ref{PRTHM}, we have:
\begin{theorem}
\label{MAIN}
Let $C(u,v)$ be the bilinear operator from Theorem \ref{BLOWUP}. Let $T$ be the blowup time to the associated pseudodifferential equation with $1/2<\alpha<5/4$.  Then the closed set $S_T$ has Hausdorff dimension at most $5-4\alpha$.
\end{theorem}
We recall here the following general theorem about local cascade operators proven by Tao in \cite{T}.
\begin{theorem}[Theorem 3.2 in \cite{T}]
Let $\lambda_0>1$ be an absolute constant sufficiently close to $1$. Then every local cascade operator (not necessarily zero-momentum) is an averaged (in the sense of Equation (\ref{AVERAGE})) Euler bilinear operator.
\end{theorem}
Due to the theorem above, we have the following corollary of Theorem \ref{MAIN}  (taking $\lambda$ near enough to $1$):
\begin{corollary}
\label{MAIN2}
Let $1/2<\alpha<5/4$ be arbitrary.
There exists a symmetric averaged Euler bilinear operator obeying the cancellation identity for $u\in H^{10}_{df}(\mathbb{R}^3)$ and a Schwartz divergence-free vector initial vector field so that there is no global-in-time solution to the associated $\alpha$-dissipative pseudodifferential equation Moreover if $T$ is the time of first blowup, then the closed set $S_T$ has Hausdorff dimension at most $5-4\alpha$.
\end{corollary}

\section{Technical Lemmas}
Throughout this subsection we shall work in the general Euclidean space $\mathbb{R}^n$. All our spaces will be of functions with domain $\mathbb{R}^n$, unless otherwise stated.
Our shorthand for the $L^p$ norm will be $\|\cdot \|_p$, and we denote the Fourier transform by $\mathcal{F}$. We denote the fractional (inhomogeneous) Sobolev space of order $s$ and integrability $p$ by:
$$W^{s,p}(\mathbb{R}^n):= \bigg\{f\in L^p(\mathbb{R}^n) : \mathcal{F}^{-1}\big((1+\abs{\xi}^2)^{\frac{s}{2}}\mathcal{F}(f)(\xi)\big)\in L^p(\mathbb{R}^n)\bigg\}.$$
Lastly, we denote $W^{s,2}(\mathbb{R}^n)$ by $H^s(\mathbb{R}^n)$ and use the shorthand $\|\cdot \|_s$ for the norm of $H^s(\mathbb{R}^n)$. Since we almost always deal with the spaces $H^s(\mathbb{R}^n)$, the use of this shorthand will always be clear in context.
The following embedding theorem is standard. A proof may be found in \cite{BCD}.
\begin{lemma}[Theorem 1.66 in \cite{BCD}]
\label{EMBED}
The space $H^s(\mathbb{R}^n)$ embeds continuously in the H\"{o}lder Space $C^{k,\alpha}(\mathbb{R}^n)$ provided that $s\geq n/2 +k+ \alpha$.
\end{lemma}
We shall also require the use of the Schwartz-Paley-Wiener theorem, which we state as the following lemma.
\begin{lemma}\label{SPW}
The vector space $C^\infty_c(\mathbb{R}^n)$ of smooth and compactly supported functions is algebraically and topologically isomorphic, via the Fourier transform, to the space of entire functions $F$ on $\mathbb{C}^n$ which satisfy the following growth bounds: there exists $B>0$ such that for every $M\in \mathbb{Z}^+$ there exists real constant $K_M$ such that for all $\xi\in \mathbb{C}^n$,
$$\abs{F(\xi)} \leq K_M (1+\abs{\xi})^{-M} e^{B\abs{\textrm{Im }(\xi)}}.$$
\end{lemma}
\subsection{Pseudodifferential Operators}
 A pseudodifferential operator is an operator $P$ on Schwartz functions $u\in \mathcal{S}$ with the property that:
$$P(x,D)u := (2\pi)^{-n/2}\int e^{i x\cdot \xi}p(x,\xi)\mathcal{F}(u)(\xi)d\xi.$$
where $p(x,\xi)$ is a smooth function on $\mathbb{R}^{2n}$. It is clear that such operators map the Schwartz space into itself. We can write $Pu=\mathcal{F}^{-1}(p\cdot\mathcal{F}(u))$ if the function $p(x,\xi)$ is a function of $\xi$ only. These operators are often called Fourier multipliers. Operators that are Fourier multipliers commute. In particular, if $P_1, P_2$ are two Fourier multipliers, then $P_1P_2= P_2 P_1$.

We call $p(x,\xi)$ the symbol of $P$, and we say that $p$ is in the symbol class $S^m_{\rho,\delta}$ where $m\in \mathbb{R}, 0\leq \rho,\delta\leq 1$ if and only if 
$$\abs{D^{\beta}_xD^\alpha_\xi p(x,\xi)} \leq C_{\alpha,\beta}((1+\abs{\xi}^2)^{1/2})^{m-\rho\abs{\alpha}+\delta\abs{\beta}}$$
for some constants $C_{\alpha,\beta}$. The associated operator $P$ is said to be in the class $OPS^m_{\rho,\delta}$. If $\delta<1$, pseudodifferential operators in the class $S^m_{\rho,\delta}$ can be defined instead on the dual of the Schwartz space, which we denote $\mathcal{S}'$. 

Let $\Lambda^s$ be the pseudodifferential operator with symbol $(1+\abs{\xi}^2)^{s/2}$, which is in the class $S^s_{1,0}$. We may recognize the Sobolev space $H^s(\mathbb{R}^n)$ as $\Lambda^{-s}L^2(\mathbb{R}^n)$.
We call a pseudodifferential operator, $P$, Sobolev-smoothing if and only if $P(H^s)\subset H^t$ for any real numbers $s$ and $t$. In other words, a Sobolev-smoothing operator $P$ continuously maps any Sobolev space into any other Sobolev space. We also define the symbol class $S^{-\infty}_{\rho,\delta}$ to be
$$S^{-\infty}_{\rho,\delta} := \bigcap_{m\in \mathbb{R}} S^m_{\rho,\delta}$$
However, there is really only one class of symbols of infinitely negative order, a fact which we note below.
\begin{lemma}
Let $0\leq \rho,\delta \leq 1$ and let $p(x,\xi)\in S^{-\infty}_{\rho,\delta}$. Then $p(x,\xi)\in S^{-\infty}_{1,0}$.
\end{lemma}
\begin{proof}
Suppose $p(x,\xi)\in S^{-\infty}_{\rho,\delta}$. Let $\alpha,\beta$ and real number $N$ be arbitrary. By hypothesis we can choose $m<N-(1-\rho)\abs{\alpha}-\delta\abs{\beta}$ so that for all $(x,\xi)$:
$$\abs{D^\beta_x D^\alpha_\xi p(x,\xi)}\leq C_{\alpha,\beta} (1+\abs{\xi}^2)^{(m-\rho\abs{\alpha}+\delta\abs{\beta})/2}\leq C_{\alpha,\beta} (1+\abs{\xi}^2)^{(N-\abs{\alpha})/2}.$$
The arbitrariness of $N$ proves that $p(x,\xi)\in S^{-\infty}_{1,0}$.
\end{proof}
Henceforth we denote $S^{-\infty}_{1,0}$ by $S^{-\infty}$. We now recall three standard lemmas below. The proofs may be found in Chapter 0 of \cite{TA}.
\begin{lemma}
\label{pseudoproduct}
Let $P_j \in OPS^{m_j}_{\rho_j,\delta_j}$. Let $\rho = \min(\rho_1,\rho_2)$ and $\delta = \max(\delta_1,\delta_2)$. Suppose
$$0\leq \delta_2 <\rho_1\leq 1.$$
Then $P_1P_2\in OPS^{m_1+m_2}_{\rho,\delta}$, and we have $P_1P_2=T_N+R_N$ where
$$T_N(x,\xi)= \sum_{\abs{\alpha}\leq N} \frac{i^{\abs{\alpha}}}{\alpha!} D^\alpha_\xi p_1(x,\xi)D^\alpha_x p_2(x,\xi)$$
and $R_N \in OPS^{m_1+m_2-N(\rho-\delta)}_{\rho,\delta}$.
\end{lemma}
\begin{lemma}
\label{pseudocont}
If $P \in OPS^m_{\rho,\delta}$, $\rho>0$, and $m<-n+\rho(n-1)$, then 
$$P : L^p(\mathbb{R}^n) \to L^p(\mathbb{R}^n),\quad \forall \hspace{10px}1\leq p\leq \infty$$
is continuous.
\end{lemma}
\begin{lemma}
\label{L2CONT}
If $P \in OPS^0_{\rho,\delta}$ and $0\leq \delta <\rho  \leq 1$, then 
$$P: L^2(\mathbb{R}^n)\to L^2(\mathbb{R}^n)$$ is continuous.
\end{lemma}
The following lemma, which deals with the "exotic" symbol classes $S^m_{1,1}$, is proven in the appendix of \cite{AT}. It provides an asymptotic expansion for operators in the exotic class.
\begin{lemma} 
\label{exotic1}
Let $0\leq \delta <1$.
Suppose $P\in OPS^m_{1,1}$ and $Q\in OPS^\mu_{1,\delta}$. Then $PQ \in OPS^{m+\mu}_{1,1}$ and we have $PQ=T_N+R_N$ where
$$T_N(x,\xi)= \sum_{\abs{\alpha}\leq N} \frac{i^{\abs{\alpha}}}{\alpha!} D^\alpha_\xi p(x,\xi)D^\alpha_x q(x,\xi)$$
and $R_N \in OPS^{m+\mu-N(1-\delta)}_{1,1}$.
\end{lemma}
Our next lemma gives us an inclusion of the operators with symbols in $S^{-\infty}$ into the Sobolev-smoothing operators.
\begin{lemma}\label{SMOOTHIE}
If $P$ has its symbol in $S^{-\infty}$, then it is a Sobolev-smoothing operator.
\end{lemma}
\begin{proof}
Let $s,t$ be arbitrary real numbers, and suppose that $f\in H^t=\Lambda^{-t}L^2$, i.e. we may write $f= \Lambda^{-t}u$ for some function $u\in L^2$. We have to show that $Pf \in H^s=\Lambda^{-s}L^2$, but this amounts to showing that $\Lambda^{s}P\Lambda^{-t} u \in L^2$.

Since $P\in S^{-\infty}$, we have, for some $m$ small enough, that $P\in S^m_{1,0}$ with $m+s-t<-1$. Then, by Lemma \ref{pseudoproduct}, $\Lambda^{s}P\Lambda^{-t} \in S^{m+s-t}_{1,0}$, which is continuous from $L^2 \to L^2$ by Lemma \ref{pseudocont}. This proves that $P$ is Sobolev-smoothing.
\end{proof}
The three lemmas that follow are also about pseudodifferential operators. The first two say that certain products of operators are guaranteed to be Sobolev-smoothing.
\begin{lemma}
\label{smoothing}
Let $P$ be a pseudodifferential operator with symbol $p(\xi)$ in the class $S^m_{1,0}$. Let $\phi_1, \phi_2$ be two smooth functions with disjoint supports (and assume they are pseudodifferential operators in the class $S^0_{1,\delta}$ with $1>\delta\geq0$). Then, the composition $\phi_1P\phi_2$ is a smoothing pseudodifferential operator.
\end{lemma}
\begin{proof}
We first compute the symbol $p_2$ of $P_2=P\phi_2$. Since $1>\delta$, a standard result (see \cite{TA}) gets us the asymptotic expansion:
$$p_2(x,\xi) \sim \sum_{\alpha\geq 0} \frac{i^{\abs{\alpha}}}{\alpha !} D^\alpha_\xi p(\xi) D^{\alpha}_x \phi_2(x)$$
where the sum and asymptotic expression is interpreted in the sense that the sum over $\abs{\alpha}<N$ differs from $p_2(x,\xi)$ by an element of $S^{m-N(1-\delta)}_{1,\delta}$. Then, a similar asymptotic expansion can be found for the symbol of $Q= \phi_1 P_2=\phi_1 P \phi_2$, which is given by:
$$q(x,\xi) \sim \sum_{\alpha\geq 0} \frac{i^{\abs{\alpha}}}{\alpha !} D^\alpha_\xi \phi_1(x) D^{\alpha}_x p_2(x,\xi).$$
It follows that up to a smoothing operator we have:
$$q(x,\xi) \sim \phi_1(x) p_2(x,\xi).$$
Now, for any arbitrary $N$, we have that
$$q(x,\xi) \sim \phi_1(x) \sum_{N>\alpha\geq 0} \frac{i^{\abs{\alpha}}}{\alpha !} D^\alpha_\xi p(\xi) D^{\alpha}_x \phi_2(x)$$ up to an operator in $S^{m-N(1-\delta)}_{1,\delta}$. Since $\phi_1, \phi_2$ have disjoint supports, the above sum is zero for all $N$. We conclude that $q(x,\xi)\in S^{-\infty}_{1,\delta}$, so $Q=\phi_1 P\phi_2$ is an operator in $OPS^{-\infty}$, which, by Lemma \ref{SMOOTHIE}, means that $Q$ is Sobolev-smoothing.
\end{proof}
We separate the next lemma from the previous one since it deals with the exotic class. The proof, however, is the same as for Lemma \ref{smoothing}.
\begin{lemma}
\label{exotic2}
Let $\phi_1,\phi_2$ be two smooth functions with disjoint supports in the class $S^{0}_{1,\delta}$, with $1>\delta\geq 0$. Let $P$ be a pseudodifferential operator in the class $OPS^m_{1,1}$. Then the composition $\phi_1 P\phi_2$ is a Sobolev-smoothing pseudodifferential operator.
\end{lemma}
\begin{proof}
We first deal with the composition $P_2=P\phi_2$, which by Lemma \ref{exotic1} lies in the class $OPS^m_{1,1}$. However, the same Lemma \ref{exotic1} also gives us the asymptotic expansion:
$$p_2(x,\xi)\sim \sum_{\abs{\alpha}\leq N} \frac{i^{\abs{\alpha}}}{\alpha!} D^\alpha_\xi p(x,\xi)D^\alpha_x \phi_2(x)$$
up to a term in $OPS^{m-N(1-\delta)}_{1,1}$. But then the symbol of $Q=\phi_1 P\phi_2$ is given by:
$$q(x,\xi)\sim \phi_1(x)\sum_{\abs{\alpha}\geq0} \frac{i^{\abs{\alpha}}}{\alpha!} D^\alpha_\xi p(x,\xi)D^\alpha_x \phi_2(x)$$
up to an operator in $S^{-\infty}$. Since $\phi_1, \phi_2$ have disjoint support, the above sum is zero, so it follows that $q\in S^{-\infty}$, as desired.
\end{proof}

The next lemma gives us a quantitative version of the smoothing property:
\begin{lemma}
\label{quantbound} Let $j\in\mathbb{Z}$ and let $P_j$ be a Littlewood-Paley projection operator. Let $\beta$ be an arbitrary real number, and assume that $\| f\|_{H^\beta}\leq 1$.
Let $Q$ be a pseudodifferential operator with symbol $q(x,\xi)$ in the class $S^m_{1,1}$. Let $\phi_1, \phi_2$ be two smooth functions with disjoint supports in the class $S^0_{1,\delta}$ where $1>\delta\geq0$. Then for any positive integer $N$, the composition $\phi_1Q\phi_2$ satisfies the following quantitative bound
$$\| P_j\phi_1 Q \phi_2 f\|_2 \leq K 2^{-jN}$$
with constant $K$ depending only on $N$ and $\beta$.
\end{lemma}
\begin{proof}
By Lemma \ref{exotic2}, we know that $\tilde{Q}:= \phi_1 Q \phi_2$ is a smoothing operator. In particular, we have for any real number $s$:
$$\| \tilde{Q} f\|_s \leq K \|f\|_\beta \leq K,$$
where the constant $K$ depends only on $s$ and $\beta$. By Lemma \ref{finiteband1}, we have
$$2^{sj} \|P_j \tilde{Q}f\|_2 \leq K \|\tilde{Q}f\|_s.$$
Together we get, for some other constant $K$ depending only on $s$ and $\beta$:
$$\|P_j \phi_1 Q\phi_2 f\|_2 \leq K 2^{-s j}.$$
Since $s$ was arbitrary, we achieve what we desired.
\end{proof}
\subsection{Littlewood-Paley Decomposition and Localization}
We use a Littlewood-Paley partition of frequency space. Namely, for $j\in \mathbb{Z}$ we have pseudodifferential operators $P_j$ with smooth symbols $p_j(\xi)$ that are supported in $\frac{2}{3}2^j < \abs{\xi}<3\cdot 2^j$, which also satisfy $p_j(\xi) = p_0(2^{-j} \xi)$ and 
\begin{equation}\sum_{j\in \mathbb{Z}} p_j(\xi) =1.\end{equation}
We also define $\tilde{P}_j:= \sum_{k=-2}^{2} P_{j+k}$ to be the sum of all Littlewood-Paley projections whose symbols' supports intersect the support of $p_j(\xi)$. Likewise we may define $\tilde{p}_j(\xi)$. We also note that
\begin{equation}
\label{proj1}
\tilde{P}_j P_j = P_j
\end{equation}
These Littlewood-Paley projections satisfy the following inequalities
\begin{lemma}
\label{PIneq}
For any $2\leq q \leq \infty$ there exists a constant $K$ independent of $j$ such that
$$\| P_j f\|_q \leq K 2^{nj\big(\frac{1}{2}-\frac{1}{q}\big)} \|P_j f\|_2$$
\end{lemma}
\begin{proof}
Let $\phi_j = \mathcal{F}^{-1}(\tilde{p}_j)$. Recalling the scaling law satisfied by the symbols $p_j$, we see that
\begin{equation}\label{scale1}\phi_j(x)=2^{nj}\phi_0(2^j x).\end{equation}
By definition, $\tilde{P}_jf = \phi_j \ast f$. By Young's convolution inequality (for which we need $q\geq 2$) and the projection property Equation (\ref{proj1}) we have
$$\| P_j f\|_q = \|\tilde{P}_j P_jf\|_q \leq \| P_j f\|_2 \|\phi_j \|_{2q/(2+q)}$$
By the scaling property in Equation (\ref{scale1}) and a change of variables we get:
$$\|\phi_j\|_{2q/(2+q)} =2^{nj} \|\phi_0(2^jx)\|_{2q/(2+q)}= 2^{nj(1/2-1/q)}\|\phi_0\|_{2q/(2+q)}.$$
Combining the last two inequalities yields our result.
\end{proof}
The next two lemmas may together be characterized as representative of the "finite band property" relating derivatives with Littlewood-Paley projections using the localization in frequency space. For more details on the finite band property, see \cite{KR} and Chapter 2 in \cite{BCD}. The following result is just a restatement of a lemma from \cite{BCD} in our notation.
\begin{lemma}[Lemma 2.1 in \cite{BCD}]
\label{finiteband1}
For any $s\geq 0$ there exists a constant $K$ such that for any $j\in \mathbb{Z}$ and for all $1\leq p\leq \infty$ we have
$$\|P_j f\|_{W^{s,p}(\mathbb{R}^n)} \leq K 2^{js} \|P_j f\|_{L^p(\mathbb{R}^n)}$$
$$2^{js} \|P_j f\|_{L^p(\mathbb{R}^n)} \leq K \|P_j f\|_{W^{s,p}(\mathbb{R}^n)}.$$
\end{lemma}
We shall need the following $L^\infty$ estimates on the high-frequency cutoff of each bump function $\phi_Q$.
\begin{lemma}
\label{highfreq}
Let $\phi=\phi_{Q,j}$ be a bump function of type $j$ (as in Section 2). Define $$\phi_2:= \mathcal{F}^{-1}(\chi_{\abs{\xi}>\frac{1}{100}2^j}(\xi)\mathcal{F}\phi(\xi)).$$ We call $\phi_2$ the high frequency cutoff of $\phi$. Then, given any $N$, there is a constant $K$ depending only on $N$ and $\epsilon$ such that:
$$\max(\|\phi_2(x)\|_\infty,\|\mathcal{F}(\phi_2)(\xi)\|_\infty) \leq K 2^{-jN}.$$
\end{lemma}
\begin{proof}
By the Schwartz-Paley-Wiener Theorem stated in Lemma \ref{SPW}, and since $\phi_2$ is smooth and compactly supported with Fourier transform $\mathcal{F}(\phi_2)$ supported away from the origin of frequency space, we may conclude that there is a constant $K$ depending only on $N$ such that
$$\| \mathcal{F}\phi_2(\xi)\|_\infty \leq K 2^{-jN}.$$
We prove a similar inequality for $\|\phi_2(x)\|_\infty$ by a rescaling argument. 
Let $N$ be arbitrary. Let $\psi$ be a smooth function with compact support such that $\textrm{supp}(\mathcal{F}(\psi))=\{\xi : \abs{\xi}\geq \frac{1}{100}\}$. We claim that if $\|D^\alpha \psi(x)\|_\infty\leq C_\alpha 2^{(-\epsilon\abs{\alpha}-n)j}$ for all $\alpha$, then $\| \psi(x)\|_\infty \leq K(N,\epsilon) 2^{(-N-n)j}$, where $K(N,\epsilon)$ depends only on $N$ and $\epsilon$ but not on $j$. Assuming the claim, let $\psi$ be a function satisfying the hypotheses of the claim and consider the rescaled function $$\tilde{\psi}(x):= 2^{nj}\psi(2^{j}x).$$ The Fourier transform of this function is
$$\mathcal{F}(\tilde{\psi})(\xi) = \mathcal{F}(\psi)(2^{-j}\xi).$$
The support of $\mathcal{F}(\tilde{\psi})$ is $\{\xi : \abs{\xi}\geq \frac{1}{100}2^{j}\}$, our claim yields
$$\|\tilde{\psi}(x)\|_\infty \leq 2^{-Nj},$$ and the following derivative bounds hold:
$$\abs{D^\alpha \tilde{\psi}(x)}\leq C_\alpha 2^{(\abs{\alpha}(1-\epsilon))j}.$$
Proving the $L^\infty$ bound for $\tilde{\psi}$ assuming the above derivative bounds and high frequency support is equivalent to proving the $L^\infty$ bound claimed before for $\psi$. We recognize $\phi_2$ as one such function $\tilde{\psi}$, so we just have to prove the statement regarding $\psi$ to finish the proof of the lemma.

Consider instead $f=2^{nj}\psi$, and assume to the contrary that there exists an $N$ such that for any $K$, we have
$$\|f\|_\infty >K2^{-Nj}\quad\textrm{and}\quad \|D^\alpha f\|_\infty \leq C_\alpha 2^{-\epsilon\abs{\alpha}j} \quad \forall \alpha>0 $$
However, the uniform derivative bounds preclude the arbitrary size of $f$, which is a contradiction.
\end{proof}

In the statement of the following lemmas, the constant $100$ appears. The particular value of this constant is unimportant and may be replaced by any large number of our choice. In addition, the hypothesis $k\geq j$ appears. This hypothesis can be weakened to $k\geq j-K$ for any given positive constant $K$ without any significant modification to the arguments used.
The next lemma is a commutator estimate similar to Proposition 5.2 in \cite{KP}.
\begin{lemma}
\label{commutator1}
Let $2<q<\infty$ be arbitrary. 
Suppose $\| f\|_2 \leq K$. Suppose $\phi$ is a bump function of type $j$ and let $k\geq j$ be arbitrary. Then, for a constant $K$ depending only on $q$:
$$\|\phi P_k f-\tilde{P}_k\phi P_k f\|_2\leq K2^{-j(100\cdot\lfloor \frac{q}{q-2}\rfloor)}.$$
\end{lemma}
\begin{proof}
Given our bump function, we can decompose it into the sum $\phi = \phi_1+\phi_2$ such that $\mathcal{F}\phi_2(\xi)=\chi_{\abs{\xi}> \frac{1}{100}2^j}(\xi)\mathcal{F}\phi(\xi)$ for some $M$. By Lemma \ref{highfreq}, we know there is a constant $K$ depending only on $q$ such that:
$$\max\big(\|\phi_2(x)\|_\infty,\| \mathcal{F}\phi_2(\xi)\|_\infty\big) \leq K 2^{-j(100\cdot\lfloor \frac{q}{q-2}\rfloor)}.$$
On the other hand, we have:
\begin{equation}\label{YES}\tilde{P}_k\phi_1P_k f= \phi_1P_k f.\end{equation}
Indeed, up to a dimensional constant we can write
$$\tilde{P}_k\phi_1 P_k f= \iint e^{ix\cdot \xi} \tilde{p}_k(\xi)\mathcal{F}(\phi_1)(z)p_k(\xi-z)\mathcal{F}(f)(\xi-z)dzd\xi.$$
Now if $\xi-z$ is in the support of $p_k$ and $\abs{z}<\frac{1}{100}2^j$, then, since $k\geq j$, we have that $\xi$ remains well inside the support of $\tilde{p}_k$. It follows that the $\tilde{p}_k$ term in the integrand can be discounted, so Equation (\ref{YES}) is true. Now observing that 
$$\phi P_k f- \tilde{P}_k \phi P_k f= \phi_1 P_k f-\tilde{P}_k \phi_1P_k f+\phi_2 P_k f-\tilde{P}_k\phi_2 P_k f$$ and that $\|f\|_2\leq K$, we conclude that
$$\|\phi P_k f- \tilde{P}_k \phi P_k f\|_2 = \|\phi_2 P_k f-\tilde{P}_k \phi_2 P_k f\|_2 \leq K\max(\|\phi_2(x)\|_\infty,\|\mathcal{F}(\phi_2)(\xi)\|_\infty)\leq$$ $$\leq K2^{-j(100\cdot\lfloor \frac{q}{q-2}\rfloor)}$$
\end{proof}
One case of the next commutator estimate is used implicitly in \cite{KP}, although neither a statement nor a proof is given there.
\begin{lemma}
\label{infcommute}
Let $2<q<\infty$ be arbitrary. 
Suppose $\| f\|_2 \leq K$. Suppose $\phi$ is a bump function of type $j$ and let $k\geq j$ be arbitrary. Then, for a constant depending only on $q$:
$$\|\phi P_k f-\tilde{P}_k\phi P_k f\|_\infty\leq K2^{-j(100\cdot\lfloor \frac{q}{q-2}\rfloor)}.$$
\end{lemma}
\begin{proof}
As before, we decompose $\phi=\phi_1+\phi_2$ where $\phi_2$ is the high frequency cutoff. By Lemma \ref{highfreq}, where we let $N=(100\cdot\lfloor \frac{q}{q-2}\rfloor)$, there is a constant $K$ depending only on $q$ such that:
$$\max\big(\|\phi_2(x)\|_\infty,\| \mathcal{F}\phi_2(\xi)\|_\infty\big) \leq K 2^{-j(100\cdot\lfloor \frac{q}{q-2}\rfloor)}.$$
As in the proof of Lemma \ref{commutator1}, we just have to estimate:
$$\|\phi_2 P_k f-\tilde{P}_k\phi_2 P_k f\|_\infty.$$
Now using the fact that $\|f\|_{L^2}\leq K$, our $L^\infty$ bounds for $\phi_2$ and $\mathcal{F}(\phi_2)$, and Lemma \ref{PIneq}, we reach our desired conclusion.
\end{proof}
\begin{lemma}\label{GOODYTWO}
Let $\phi$ be a bump function of type $j$, let $k\geq j$ and $2<q<\infty$ be arbitrary. Suppose also that $\|f \|_2 \leq K$. Then:
$$\|\phi P_k f\|_\infty \leq K 2^{nk/2}\|\phi P_k f\|_2 +K 2^{-j(100\cdot\lfloor \frac{q}{q-2}\rfloor)}$$
\end{lemma}
\begin{proof}
This follows immediately from Lemma \ref{infcommute}, Lemma \ref{PIneq}, and the observation that
$$\phi P_k f= \phi P_k f -\tilde{P}_k\phi P_k f+ \tilde{P}_k\phi P_k f.$$
\end{proof}
The next lemma extends Lemma 5.3 in \cite{KP}.
\begin{lemma}
\label{ineq2}
Let $2\leq q \leq \infty$, $\|f\|_2\leq K$, and let $\phi$ be a bump function of type $j$. Let $k\geq j$ be arbitrary. Then:
$$\|\phi P_k f \|_q \leq K 2^{nk(1/2-1/q)}\|\phi P_k f\|_2 + K2^{-100j}$$
\end{lemma}
\begin{proof}
When $q=2$ there is nothing to show. When $q=\infty$, the result is proven in Lemma \ref{GOODYTWO}, and the exponent in the error term can be made arbitrarily negative. We prove the remaining cases by interpolation. First, it is evident from Lemma \ref{PIneq} that the function $\phi P_k f \in L^q$ for all $2\leq q \leq \infty$. In particular, since the $L^p$ norms are log-convex:
$$\|\phi P_k f\|_q \leq \|\phi P_k f\|_2^{2/q} \| \phi P_k f\|_\infty^{1-2/q}$$
For the remainder of the proof, we denote $A:= \|\phi P_k f\|_2$. Using Lemma \ref{GOODYTWO}, which is the case $q=\infty$ in the statement of our lemma, we have that:
$$\|\phi P_k f\|_q \leq A^{2/q}\bigg(2^{nk/2}A+ K2^{-100j\lfloor \frac{q}{q-2}\rfloor}\bigg)^{(q-2)/q}$$
Using the elementary inequality $(a+b)^p\leq 2^p(a^p+b^p)$ for $p\geq 0$ and the fact that $q>2$ gets us:
$$\|\phi P_k f\|_q \leq 2^{(q-2)/q} A^{2/q}\bigg(2^{\frac{nk}{2}\cdot\frac{q-2}{q}}A^{(q-2)/q}+ K2^{-100j}\bigg)$$
Using the fact that $\|f\|_2\leq K$ we conclude:
$$\|\phi P_k f\|_q \leq K 2^{nk(\frac{1}{2}-\frac{1}{q})}\|\phi P_k f\|_2 +K2^{-100j}$$
as desired.
\end{proof}
We can commute bump functions with Littlewood-Paley projections, as long as we multiply with another bump function with slightly larger support.
\begin{lemma}
\label{bumpcommute}
Let $k\geq j$, let $\phi_{Q,j}$ be a bump function at level $j$. Let $\beta<100$ and consider $f$ with $\| f\|_{H^\beta}\leq K$. Then:
$$\|(1-\phi_{(1+2^{-\epsilon j})Q,j})P_k \phi_{Q,j} f\|_2 \leq K2^{-100j}$$
\end{lemma}
\begin{proof}
By Lemma \ref{smoothing}, we see that
$$(1-\phi_{(1+2^{-\epsilon j})Q,j})P_k\phi_{Q,j}$$
is a smoothing pseudodifferential operator. By Lemma \ref{commutator1} and since $\tilde{P}_k P_k = P_k$, we have
$$(1-\phi_{(1+2^{-\epsilon j})Q,j})P_k\phi_{Q,j} = P_k\phi_{Q,j} - \phi_{(1+2^{-\epsilon j})Q,j}P_k\phi_{Q,j}  =$$ $$= \tilde{P}_kP_k\phi_{Q,j} - \tilde{P}_k\phi_{(1+2^{-\epsilon j})Q,j}P_k\phi_{Q,j} +R_j = \tilde{P}_k((1-\phi_{(1+2^{-\epsilon j})Q,j})P_k\phi_{Q,j}) +R_j,$$
where $R_j$ is an error term satisfying $\abs{R_j}\leq K2^{-200j}$.
By Lemma \ref{quantbound}, we have a quantitative bound on the Sobolev smoothing operator that appears, thus proving our desired inequality. The same proof works to get an arbitrarily negative exponent in the error term instead.
\end{proof}
Note that in the previous lemma, the order of the bump functions in the composition was unimportant. The only essential fact was disjointness of support.
\begin{lemma}
\label{collectbound}
Let $\mathcal{A}$ be a collection of cubes at level $j$ covering a measurable set $E$. For any $f\in L^2$, we have:
$$\|\chi_E P_j f\|_2^2 \leq \sum_{Q\in\mathcal{A}} f_Q^2$$
\end{lemma}
\begin{proof}
We notice that
$$\int \chi_E \abs{P_j f}^2 \leq \int \bigg(\sum_{Q\in\mathcal{A}} \phi_{Q,j}^2\bigg)\abs{P_j f}^2 =\sum_{Q\in\mathcal{A}}f_Q^2.$$ 
\end{proof}
\subsection{Hausdorff Measure and Dimension}
We denote a ball of radius $r$ in $\mathbb{R}^n$ by $B_r$.
Let $E\subset \mathbb{R}^n$. We recall that, up to a dimensional constant multiple, the $d$-dimensional Hausdorff measure of $E$ is given by
$$H^d(E) := \sup_{\delta >0} \inf_{\mathcal{C}_\delta(E)} \sum_{B_r\in \mathcal{C}_\delta(E)}r^d$$
where we have taken an infimum over all coverings $C_\delta(E)$ of $E$ by balls of radius less than or equal to $\delta$.
We define the Hausdorff dimension of $E$ to be:
$$\mathcal{H}(E) = \inf \{ d : H^d(E)=0\}.$$
We need a way to compute the Hausdorff dimension that follows from a discretization by sets of scale $2^j$. In particular we have the following lemma from \cite{KP}:
\begin{lemma}
\label{dimcompute}
Let $\mathcal{A}_j$ be a sequence of collections of balls in $\mathbb{R}^n$ so that each element of $\mathcal{A}_j$ has radius $2^{-j}$. Suppose that the number of balls in each $\mathcal{A}_j$, denoted $N(\mathcal{A}_j)$, is bounded by $N(\mathcal{A}_j)\leq C 2^{jd}$ where $C$ is independent of $j$. Let
$$E = \limsup_{j\to \infty}\mathcal{A}_j:= \cap_{j\in \mathbb{N}} \cup_{k>j} \cup_{B\in \mathcal{A}_k} B$$
be the set of points in infinitely many of the unions $\cup_{B\in\mathcal{A}_j} B$. Then $\mathcal{H}(E)\leq d$. \end{lemma}
\begin{proof}
By the definition of the Hausdorff dimension, it suffices to show that $H^s(E)=0$ for all $s>d$. To that end, pick $j$ large enough that $2^{-j} <\delta$. By construction of our set, $E$ can be covered by $\cup_{k>j} \cup_{B\in \mathcal{A}_k} B$. Evidently, we have
$$H^s(E)\leq \sum_{k>j} N(\mathcal{A}_k)(2^{-k})^s\leq C \sum_{k>j} 2^{kd}(2^{-k})^s $$
and the right-hand side above converges to zero as $j\to \infty$ so long as $d<s$.
\end{proof}
We also include in this subsection the standard Vitali covering lemma, which is often used in estimating the Hausdorff dimension of a set.
\begin{lemma}[Vitali]
\label{Vitali}
Let $\mathcal{A}$ be a collection of cubes. Then there is subcollection $\mathcal{A}'$ so that any two cubes in $\mathcal{A}'$ are pairwise disjoint and 
$$\bigcup_{Q\in \mathcal{A}} Q \subset \bigcup_{Q\in \mathcal{A}'} 5Q.$$
\end{lemma}
\subsection{Zero-Momentum Schwartz Vector Fields}
\begin{lemma}
\label{DivF}
A Schwartz function $\psi: \mathbb{R}^3 \to \mathbb{R}$ can be written as $\psi = \textrm{div } \Psi$ for some Schwartz vector field $\Psi$ if and only if $\int_{{\mathbb{R}^3}}\psi =0$.
\end{lemma}
\begin{proof}
One direction is relatively straightforward. Indeed suppose that $\psi = \textrm{div } \Psi$ for some Schwartz vector field and let $B_R$ be some ball centered at the origin of radius $R$. Then the divergence theorem gets us:
$$\int_{B_R} \psi = \int_{B_R} \textrm{div } \Psi = \int_{\partial B_R} \Psi \cdot \bf{n}$$
Since $\Psi$ is Schwartz and decays faster than any polynomial at infinity, the right hand side above tends to zero as $R\to \infty$ and the result is shown.\\\\
Now assume that $\int_{\mathbb{R}^3} \psi =0$; we desire to construct $\Psi$. Let $f(z)$ be any real-valued function in $C^\infty_c(\mathbb{R})$. Let 
$$\Gamma(x,y,z) = f(z)\cdot \int_{-\infty}^x \int_{-\infty}^y \int_{-\infty}^\infty \psi(r,s,t)dtdsdr$$
By our assumption on the integral of $\psi$, the function $\Gamma$ is a Schwartz function.
Now, 
$$\partial_x \Gamma = f(z)\int_{-\infty}^y \int_{-\infty}^\infty \psi(x,s,t)dtds$$
and
$$\partial_y \Gamma = f(z)\int_{-\infty}^x \int_{-\infty}^\infty \psi(r,y,t)dtdr.$$
So defining (a function that is also a Schwartz function by our assumption on $\psi$):
$$\Xi(x,y,z) = \int_{-\infty}^z (\psi - \partial_x \Gamma -\partial_y \Gamma)(x,y,t)dt$$
we observe that
$$\psi = \partial_x \Gamma + \partial_y \Gamma +\partial_z \Xi$$
so our desired vector field $\Psi = (\Gamma, \Gamma, \Xi)$.
\end{proof}

\subsection{The Continuity of the Energy Integral}
\begin{lemma}\label{Integral1}
Let $u$ be any smooth solution of Equation (\ref{PEQN}) with Schwartz initial value $u_0$, let $Q$ be any cube, and let $t_0<T$ be any time before blowup. Then the following equality holds:
$$\lim_{\epsilon\to 0} \int_{t_0}^{T-\epsilon} \frac{d}{dt} u_{Q}^2 dt = \int_{t_0}^T \frac{d}{dt}u_{Q}^2 dt $$
\end{lemma}
\begin{proof}
The vector field $u(t)$ is a smooth (mild) solution of Equation (\ref{PEQN}),
$$\partial_t u +(-\Delta)^\alpha u +\hat{B}(u,u) =0,$$
with Schwartz initial data $u_0$. In particular, its $L^2$ norm is strongly continuous in time, so it remains to show that the localized norm $u_Q$ remains continuous in time. For this we proceed as in the proof of Lemma 2.7 in \cite{O}.

We observe that
$$\| \phi_{Q,j}P_ju(x,t) - \phi_{Q,j}P_ju(x,s) \|_{L^2(\mathbb{R}^n)}^2 \leq \int_{2Q} \abs{\int_{\mathbb{R}^n} p_j(x-\xi)(\mathcal{F}(u)(\xi,t)-\mathcal{F}(u)(\xi,s))d\xi}^2dx.$$
The integral inside the absolute value on the right-hand side above converges to zero because $u$ is a smooth mild solution whose $L^2$ norm is strongly continuous in time and because the Fourier transform is a unitary operator on $L^2$. Moreover, since $\|u\|_{L^2}\leq K$, the integral is dominated by:
$$\abs{\int_{\mathbb{R}^n} p_j(x-\xi)(\mathcal{F}(u)(\xi,t)-\mathcal{F}(u)(\xi,s))d\xi}\leq K \tilde{p}_j(x)$$
Thus, by the dominated convergence theorem, we conclude that the localization $u_Q$ remains strongly continuous in time. The desired limit then follows in a straightforward manner.
\end{proof}
As a corollary, we have:
\begin{corollary}\label{Integral2}
Let $t_0<T$ be any time. Then 
$$ \int_{t_0}^{T} \frac{d}{dt} u_Q^2 dt = \limsup_{t\to T} u_Q^2(t) -u_Q^2(t_0).$$
\end{corollary}
\begin{proof}
Indeed, we may write
$$ \int_{t_0}^{T-\epsilon} \frac{d}{dt} u_Q^2 dt = u_Q^2(T-\epsilon) -u_Q^2(t_0),$$
take the $\limsup$ of the above expression as $\epsilon\to 0$ and use Lemma \ref{Integral1}.
\end{proof}

\subsection{The Critical Exponent in Higher Dimensions}
We determine the critical exponent of hyperdissipation that guarantees global regularity for solutions of Equation (\ref{FRACNS}). A proof along the same lines for the hyperdissipative three-dimensional Navier-Stokes equations may be found in \cite{KP}.

\begin{proposition}
Let $\alpha>(n+2)/4$ and let $u(t):\mathbb{R}^n\times [0,\epsilon)\to \mathbb{R}^n$ be a smooth solution of Equation (\ref{FRACNS}) from Schwartz initial data. Then, $u(t)$ is smooth for all times $t\geq 0$. 
\end{proposition}
\begin{proof}
We let $\beta>0$ and test Equation (\ref{FRACNS}) against $(-\Delta)^\beta u$ to get:
$$\langle \partial_t u , (-\Delta)^\beta u \rangle +\langle (-\Delta)^\alpha u, (-\Delta)^\beta u\rangle +\langle (u\cdot\nabla)u, (-\Delta)^\beta u\rangle =0.$$
Using that $\textrm{div }u=0$, we may rewrite the equation as:
\begin{equation}\label{UH1} \frac{d}{dt}\big(\frac{1}{2}\|u\|_{\beta}^2\big)+\|u\|_{\alpha+\beta}^2 +\langle (-\Delta)^{\frac{\beta}{2}}\textrm{div}(u\otimes u), (-\Delta)^{\frac{\beta}{2}} u\rangle =0.\end{equation}
We estimate the trilinear term by:
\begin{equation}\label{UH2}\abs{\langle (-\Delta)^{\frac{\beta}{2}}\textrm{div}(u\otimes u), (-\Delta)^{\frac{\beta}{2}} u\rangle}\leq \| u\|_{W^{1,p}}\|u\|_{W^{\beta,q}}\|u\|_\beta\end{equation}
where $1/p+1/q=1/2$. Now, $\alpha=(n+2)/4$ is the minimum value of $\alpha$ so that both inequalities 
$$\frac{1}{2}-\frac{\alpha}{n}\leq \frac{1}{p}-\frac{1}{n}, \quad \textrm{and} \quad \frac{1}{2}-\frac{\alpha}{n}\leq \frac{1}{q}$$
hold for some choice $(p,q)$ with $1/p+1/q=1/2$ depending on $n\geq 2$. Since these inequalities hold, we may use the Sobolev Embedding Theorem and then Young's inequality to conclude that:
\begin{equation}\label{UH3} \| u\|_{W^{1,p}}\|u\|_{W^{\beta,q}}\|u\|_\beta\leq \|u\|_\alpha \|u\|_{\alpha+\beta}\|u\|_\beta\leq \frac{1}{2} \|u\|_{\alpha+\beta}^2 +\frac{1}{2}\|u\|_\alpha^2\|u\|_\beta^2.\end{equation}
Using Equations (\ref{UH1}), (\ref{UH2}), and (\ref{UH3}), we get
$$\frac{d}{dt}\big(\frac{1}{2}\|u\|_{\beta}^2\big) \leq \frac{1}{2}\|u\|_\alpha^2\|u\|_\beta^2$$
It follows that
$$\|u(t)\|_\beta^2 \leq \|u(0)\|_\beta^2 \exp\big(\int_0^t \|u(s)\|_\alpha^2ds\big).$$
However, by the energy dissipation law, we know
$$\int_0^t \|u(s)\|_\alpha^2 ds\leq \|u(0)\|_{L^2}$$
for all times $t$ whenever $u(t)$ is smooth. Since $\beta$ was arbitrary, $u_0$ was Schwartz, and $u(t)$ was smooth for small times $t\in [0,\epsilon)$, we get regularity for all times $t\geq 0$.
\end{proof}

\subsection{An Algebraic Lemma}

In the proof of our main partial regularity theorem, we required that certain parameters be appropriately chosen in terms of $\alpha$ and the dimension $n\geq2$ in order for our argument to close. We remark that such a choice cannot be made when $\alpha\leq 1/2$. The following lemma encapsulates an essential requirement of our parameters.
\begin{lemma}
\label{Algebra}
For any $n\geq 2$ and for any $\alpha \in (1/2, (n+2)/4)$, there exist $p_0> 2, \mu_0>3000,$ and $\gamma\in (0,1)$ such that:
\begin{equation}\label{ALG1}n(\frac{1}{2}-\frac{2\gamma}{3})-\frac{\mu_0}{p_0}-\frac{(n+2-4\alpha)(p_0-1)}{p_0}< -600\end{equation}
and
\begin{equation}\label{ALG2}(\gamma n+1)-\frac{2\alpha(p_0-1)}{p_0}<0.\end{equation}
\end{lemma}
\begin{proof}
Let $n\geq 2$ and $\alpha\in (1/2,(n+2)/4)$ be fixed. We first write:
$$\frac{p_0-1}{p_0}=1-\epsilon(p_0).$$
The positive function $\epsilon(p_0)\to 0$ as $p_0\to \infty$. We then use this expression to simplify the Inequality in Equation (\ref{ALG2}) as:
$$\gamma n +1-2\alpha+2\alpha \epsilon(p_0).$$
Now since $\alpha$ is strictly larger than $1/2$, we may choose $\gamma$ so that
$$0<\gamma < \frac{2\alpha-1}{3n}$$ and $p_0$ sufficiently large that
$$0< \epsilon(p_0)< \frac{2\alpha-1}{6\alpha}.$$
Having chosen $\gamma$ and $p_0$ in this fashion, we conclude that
$$(\gamma n+1)-\frac{2\alpha(p_0-1)}{p_0}=\gamma n +1-2\alpha+2\alpha \epsilon(p_0)\leq 1-2\alpha +\frac{2\alpha-1}{3}+\frac{2\alpha-1}{3}<\frac{1-2\alpha}{3}<0,$$
which shows that Equation (\ref{ALG2}) holds. Having thus chosen $\gamma$ and $p_0$, we simply choose a number $\mu_0$ sufficiently large that Equation (\ref{ALG1}) also holds, ending the proof.
\end{proof}
\begin{thebibliography}{10}

\bibitem{AT}
Auscher, P., Taylor, M.E. 
\textit{Paradifferential Operators and Commutator Estimates}
Comm. PDE. 20. pp. 1743-1775. (1995).

\bibitem{BCD}
Bahouri, H., Chemin, J-Y, Danchin, R.
\textit{Fourier Analysis and Nonlinear Partial Differential Equations}.
Springer. (2011).

\bibitem{BONY}
Bony, J.M.
\textit{Calcul Symbolique et equations non lineares}.
Ann. Sci. Ecole Norm. Sup. 14. pp. 209-246. (1981).

\bibitem{CKN}
Caffarelli, L., Kohn, R., Nirenberg, L. 
\textit{Partial Regularity of Suitable Weak Solutions to the Navier-Stokes Equations}.
Comm. Pure Appl. Math. 35. pp. 771-831. (1982).

\bibitem{COIF}
Coifman, R., Meyer, Y. \textit{Wavelets: Calderon-Zygmund and Multilinear Operators}.
Cambridge Studies in Advanced Mathematics. 48. (1997).

\bibitem{CDM}
Colombo, M., De Lellis, C., Massaccesi, A.
\textit{The Generalized Caffarelli-Kohn-Nirenberg Theorem for the Hyperdissipative Navier-Stokes System}.
Comm. Pure Appl. Math. 73. pp. 609-663. (2020).

\bibitem{CDR}
Colombo, M., De Lellis, C., De Rosa, L.
\textit{Ill-Posedness of Leray Solutions for the Hypodissipative Navier?Stokes Equations}
Comm. Math. Phys. 362. pp. 659-688. (2018).

\bibitem{CH}
Colombo, M., Haffter, S.
\textit{Global Regularity for the Hyperdissipative Navier-Stokes equation below the critical order}.
J. Diff. Eqn. Vol. 275. pp. 815-836. (2021).

\bibitem{DG}
Dong, H., Gu, X.
\textit{Partial Regularity of Solutions to the Four-Dimensional Navier-Stokes Equations}.
Dyn. Partial Differ. Equ. 11, no. 1, 53-69. (2014).

\bibitem{F}
Fefferman, C.L.
\textit{Existence and Uniqueness of the Navier-Stokes Equation}.
Clay Mathematics Institute. (2006).

\bibitem{H}
Hou, T.Y.
\textit{Potentially Singular Behavior of the 3D Navier-Stokes Equations}.
Found. Comput. Math. (2022).

\bibitem{KP}
Katz, N.H., Pavlovic, N. 
\textit{A cheap Caffarelli-Kohn-Nirenberg inequality for the Navier-Stokes equation with hyper-dissipation}.
Geom. Funct. Anal. 12. pp. 355-379. (2002).

\bibitem{KR}
Klainerman, S., Rodnianski, I.
\textit{A Geometric Approach to the Littlewood-Paley Theory}.
Geom. Funct. Anal. 16. pp. 126-163. (2006).

\bibitem{KO}
Kwon, H., Ozanski, W.S.
\textit{Local regularity of weak solutions of the hypodissipative Navier-Stokes equations}.
J. Funct. Anal. Vol 282. (2022).

\bibitem{L}
Leray, J. 
\textit{Etude de diverses equations integrales non lineaires et de quelques problemes que pose l'hydrodynamique}.
J. Math. Pures Appl. (1933).

\bibitem{O}
Ozanski, W.S.
\textit{Partial Regularity of Leray-Hopf weak Solutions to the incompressible Navier-Stokes equations with hyperdissipation}.
Analysis and PDE. Vol 16. pp. 747-783. (2023).

\bibitem{S}
Scheffer, V. 
\textit{Partial Regularity of Solutions to the Navier-Stokes Equations}.
Pacific J. Math. Vol. 16. No. 2. (1976).

\bibitem{SE}
Serrin, J.
\textit{On the Interior Regularity of Weak Solutions of the Navier-Stokes equations}.
Arch. Ration. Mech. Anal. Vol. 9, pp. 187-195. (1962).

\bibitem{TY}
Tang, L., Yu, Y.
\textit{Partial Regularity of Suitable Weak Solutions to the Fractional Navier-Stokes Equations}.
Commun. Math. Phys. Vol. 334. pp. 1455-1482. (2015).

\bibitem{TY1}
Tang, L., Yu, Y.
\textit{Erratum to: Partial Regularity of Suitable Weak Solutions to the Fractional Navier-Stokes Equations}.
Commun. Math. Phys. Vol. 335. pp. 1057-1063. (2015).

\bibitem{T}
Tao, T.
\textit{Finite time blowup for an averaged three-dimensional Navier-Stokes equation}.
J. Amer. Math. Soc. 29. pp. 601-674. (2016).

\bibitem{TA}
Taylor, M.E.
\textit{Pseudodifferential Operators and Nonlinear PDE}.
Birkhauser. (1991).

\bibitem{TS}
Tsai, T.P.
\textit{Lectures on the Navier-Stokes Equations}.
American Mathematical Society. (2018).


\end{thebibliography}
\end{document}