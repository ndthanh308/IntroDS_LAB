\documentclass[letterpaper, conference]{IEEEtran}
\usepackage[margin=54pt]{geometry}
\usepackage{afterpage}
\IEEEoverridecommandlockouts

\usepackage[cmex10]{amsmath}
\usepackage{cite}
\usepackage[table]{xcolor}
\usepackage{makecell}

\usepackage{outlines}
\usepackage{amssymb}
\usepackage{amsfonts}
\usepackage{pifont}
\newcommand{\cmark}{\ding{51}}
\newcommand{\xmark}{\ding{55}}
\usepackage{enumitem}
\usepackage{path}
\usepackage{verbatim}
\usepackage{algorithm}
\usepackage{algorithmic}
\usepackage{framed}
\usepackage{footnote}
\usepackage{color}
\usepackage{amsmath}
\usepackage{cases}
\usepackage{subfigure}
\usepackage{theorem}
\usepackage{xurl}

\usepackage{tabularx}
\usepackage{booktabs}
\usepackage[colorlinks=true, allcolors=black]{hyperref}
\usepackage{threeparttable}

\usepackage{graphicx}

\usepackage{pgfplots}
\usepackage{pgfplotstable}
\pgfplotsset{compat=newest}
\graphicspath{{graphics/}}

\newcommand{\uth}{$^{\mathrm{th}}$}

\newcommand{\R}{\ensuremath{\mathbb{R}}}
\newcommand{\N}{\ensuremath{\mathbb{N}}}
\newcommand{\spann}{\ensuremath{\operatorname{span}}}
\newcommand{\diag}{\ensuremath{\operatorname{diag}}}
\newcommand{\dist}{\ensuremath{\operatorname{dist}}}
\newcommand{\id}{\ensuremath{\operatorname{id}}}
\newcommand{\rank}{\ensuremath{\operatorname{rank}}}
\newcommand{\PP}{{\mathbf P}_1}

% AutoRef Command
\renewcommand{\figureautorefname}{Fig.}
% \renewcommand{\tableautorefname}{Table}
\renewcommand{\sectionautorefname}{Section}
% \renewcommand{\subsectionautorefname}{foobarbaz}
\newcommand{\algorithmautorefname}{Algorithm}
\newcommand{\subfigureautorefname}{Fig.}

\newtheorem{theorem}{Theorem}
{ \theorembodyfont{\normalfont} %\theorembodyfont{\rmfamily}
	\newtheorem{example}[theorem]{Example}
	\newtheorem{remark}[theorem]{Remark}
}
\newtheorem{assumption}[theorem]{Assumption}
\newtheorem{definition}[theorem]{Definition}
\newtheorem{lemma}[theorem]{Lemma}
\newtheorem{corollary}[theorem]{Corollary}
\newtheorem{proposition}[theorem]{Proposition}
\newtheorem{conjecture}[theorem]{Conjecture}
\def\QEDclosed{\mbox{\rule[0pt]{1.3ex}{1.3ex}}} % for a filled box
\def\QEDopen{{\setlength{\fboxsep}{0pt}\setlength{\fboxrule}{0.2pt}\fbox{\rule[0pt]{0pt}{1.3ex}\rule[0pt]{1.3ex}{0pt}}}}
\def\QED{\QEDopen}
\def\proof{\noindent\hspace{2em}{\itshape Proof: }}
\def\endproof{\hspace*{\fill}~\QED\par\endtrivlist\unskip}
\newcounter{enumctr}
\newenvironment{enum}{\begin{list}{(\roman{enumctr})}{\usecounter{enumctr}}}{\end{list}}

\DeclareFontFamily{U}{mathx}{\hyphenchar\font45}
\DeclareFontShape{U}{mathx}{m}{n}{<-> mathx10}{}
\DeclareSymbolFont{mathx}{U}{mathx}{m}{n}
\DeclareMathAccent{\widebar}{0}{mathx}{"73}

\newcommand{\MMcom}[1]{{\color{red}#1}}
\newcolumntype{C}{>{\centering\arraybackslash}X} % centered version of "X" type
\setlength{\extrarowheight}{1pt}

\begin{document}
    \newgeometry{top=72pt,bottom=54pt,left=54pt,right=54pt}
    
    \title{\LARGE Breathing Green: Maximising Health and Environmental Benefits for Active Transportation Users Leveraging Large Scale Air Quality Data}

    \author{Sen Yan, Shaoshu Zhu, Jaime B. Fernandez, Eric Arazo Sánchez, \\Yingqi Gu, Noel E. O’Connor, David O’Connor and Mingming Liu 
        \thanks{S. Yan, S. Zhu, J.B. Fernandez, E. Arazo, M. Liu and N. E. O'Connor are with the School of Electronic Engineering and SFI Insight Centre for Data Analytics at Dublin City University, Ireland. Y. Gu is with Alkermes plc, Dublin, Ireland. D. O'Connor is with the School of Chemical Sciences at Dublin City University. This project is now available on GitHub and can be accessed via {\tt \url{ https://github.com/SFIEssential/BreathGreen}}. Corresponding author's email: {\tt mingming.liu@dcu.ie}}
    }
    
    \maketitle

    \begin{abstract}
    \label{sec:abstract}
    


Low-rank adaptations (LoRA) are often employed to fine-tune large language models (LLMs) for new tasks. This paper investigates LoRA composability for cross-task generalization and introduces \lorahub, a strategic framework devised for the purposive assembly of LoRA modules trained on diverse given tasks, with the objective of achieving adaptable performance on unseen tasks. With just a few examples from a novel task, \lorahub enables the fluid combination of multiple LoRA modules, eradicating the need for human expertise. Notably, the composition requires neither additional model parameters nor gradients. Our empirical results, derived from the Big-Bench Hard (BBH) benchmark, suggest that \lorahub can effectively mimic the performance of in-context learning in few-shot scenarios, excluding the necessity of in-context examples alongside each inference input. A significant contribution of our research is the fostering of a community for LoRA, where users can share their trained LoRA modules, thereby facilitating their application to new tasks. We anticipate this resource will widen access to and spur advancements in general intelligence as well as LLMs in production.
Code will be available at \href{https://github.com/sail-sg/lorahub}{\texttt{github.com/sail-sg/lorahub}}.
    \end{abstract}

    \begin{IEEEkeywords}
        Air Pollution, Data Analysis, Micromobility, Recommendation System, Route Planning
    \end{IEEEkeywords}

    \section{Introduction}
    \label{sec: intro}
    \section{Introduction}

% Figure environment removed

Reinforcement Learning from Human Feedback (RLHF) has recently been used to great effect to align pretrained large language models (LLMs) to human preferences, optimizing for desirable qualities like harmlessness and helpfulness~\citep{bai2022training} and achieving state-of-the-art results across a variety of natural language tasks~\citep{openai2023gpt4}. %RLHF approaches fundamentally rely on collecting pairs of LLM outputs $(o_1, o_2)$ from a shared prompt $p$, with a human indicating which output in each pair is better on a specified attribute.
% A fundamental component of RLHF is a preference model derived from human labels, typically formatted as pairs of LLM outputs $(o_1, o_2)$ generated from a shared prompt $p$.

A standard RLHF procedure fine-tunes an initial unaligned LLM using an RL algorithm such as PPO~\citep{schulman2017proximal}, optimizing the LLM to align with human preferences. %\violet{not sure whether we need to provide this detail in the intro, especially this has nothing to do with our contribution.} % i feel like this context is useful later when e.g. explaining that context distillation is SFT
RLHF is thus critically dependent on a reward model derived from human-labeled preferences, typically \textit{pairwise preferences} on LLM outputs $(o_1, o_2)$ generated from a shared prompt $p$. % and labeled by humans. 

However, collecting human pairwise preference data, especially high-quality data, may be expensive and time consuming at scale. To address this problem, approaches have been proposed to obtain labels without human annotation, such as Reinforcement Learning from AI Feedback (RLAIF) and context distillation. 

\iffalse
raising the question of whether we can generate high-quality data for RLHF without using human labeling. %accurately-labeled preference pairs $(o_1, o_2)$
%, motivating model alignment approaches that aim to generate accurately-labeled preference pairs $(o_1, o_2)$ without human involvement. 
Two major categories of such approaches are . 
\fi

RLAIF approaches (e.g.,~\citet{bai2022constitutional}) simulate human pairwise preferences by scoring $o_1$ and $o_2$ with an LLM (Figure \ref{fig:rlcd_differences} center); the scoring LLM is often the same as the one used to generate the original pairs $(o_1, o_2)$. Of course, the resulting LLM pairwise preferences will be somewhat noisier compared to human labels. However, this problem is exacerbated by using the same prompt $p$ to generate both $o_1$ and $o_2$, causing $o_1$ and $o_2$ to often be of very similar quality and thus hard to differentiate (e.g., Table~\ref{tab:rlaif_bad_example}). Consequently, training signal can be overwhelmed by label noise, yielding lower-quality preference data. 

% While it avoids human labeling efforts, it has weakness. First, LLM preference labels will naturally be somewhat noisier compared to human labels. Furthermore, since the same prompt $p$ is used to generate both $o_1$ and $o_2$, their quality is often very similar and hard to differentiate (See Table~\ref{tab:rlaif_bad_example}). As a result, training signals can be overwhelmed by label noise, yielding lower-quality preference data. 

Meanwhile, context distillation methods (e.g., \citet{sun2023principle}) create more training signal by modifying the initial prompt $p$. 
%to create more significant training signal. 
The modified prompt $p_+$ typically contains additional context encouraging a \textit{directional attribute change} in the output $o_+$ (Figure \ref{fig:rlcd_differences} right). However, context distillation methods only generate a single output $o_+$ per prompt $p_+$, which is then used for supervised fine-tuning, losing the pairwise preferences which help RLHF-style approaches to 
%rather than using a RLHF-style preference model to 
derive signal from the contrast between outputs. 
Multiple works have observed that RL approaches using preference models for pairwise preferences can substantially improve over supervised fine-tuning by itself when aligning LLMs~\citep{ouyang2022training,dubois2023alpacafarm}. 

% conduct alignment by running supervised fine-tuning on model outputs $o_+$ generated from a modified prompt $p_+$. $p_+$ typically contains additional context encouraging desirable attributes (Figure \ref{fig:rlcd_differences} right), such as in \citet{sun2023principle}. However, multiple works have observed that RLHF-style approaches can substantially improve over supervised fine-tuning by itself when aligning LLMs~\citep{ouyang2022training,dubois2023alpacafarm}. 

Therefore, while both RLAIF and context distillation approaches have already been successfully applied in practice to align language models, we posit that it may be even more effective to combine the key advantages of both. That is, we will use RL with \textit{pairwise preferences}, while also using modified prompts to encourage \textit{directional attribute change} in outputs. %In particular, we will adapt the RLAIF data generation process with two different prompts rather than a single $p$, modifying both prompts similarly to context distillation. %\violet{this motivation is a little unexciting. I think we can more specifically discuss the potential benefits of our approach, like the benefits from RL: exploration/data generation; benefits from contrast. I don't think we get too much benefits from context distillation since we switched to the RL framework.} 

Concretely, we propose \oursfull{} (\ours{}). 
\ours{} generates preference data as follows. Rather than producing two i.i.d.\ model outputs $(o_1, o_2)$ from the same prompt $p$ as in RLAIF, \ours{} creates two variations of $p$: a \textit{positive prompt} $p_+$ similar to context distillation which encourages directional change toward a desired attribute, and a \textit{negative prompt} $p_-$ which encourages directional change \textit{against} it (Figure \ref{fig:rlcd_differences} left). We then generate model outputs $(o_+, o_-)$ respectively, and automatically label $o_+$ as preferred---that is, \ours{} automatically ``generates'' pairwise preference labels by construction. %, without further post hoc labeling.\violet{should make it clearer that our approach `generates' labels by construction} 
We then follow the standard RL pipeline of training a preference model followed by PPO. 

Compared to RLAIF-generated preference pairs $(o_1, o_2)$ from the same input prompt $p$, there is typically a clearer difference in the quality of $o_+$ and $o_-$ generated using \ours{}'s directional prompts $p_+$ and $p_-$, which may result in less label noise. %which may result in better training signal for the preference model. 
That is, intuitively, \ours{} exchanges having examples be \textit{closer to the classification boundary} for much more \textit{accurate labels} on average. Compared to standard context distillation methods, on top of leveraging pairwise preferences for RL training, \ours{} can derive signal not only from the positive prompt $p_+$ which improves output quality, but also from the negative prompt $p_-$ which degrades it. %\ours{} is not learning to imitate $o_+$, but to distill the \textit{contrast} between $o_+$ and $o_-$. 
Positive outputs $o_+$ don't need to be perfect; they only need to contrast with $o_-$ on the desired attribute while otherwise following a similar style.

% \todo{discuss our method and why intuitively it may be better.}

We evaluate the practical effectiveness of \ours{} through both human and automatic evaluations on three tasks, aiming to improve the ability of LLaMA-7B~\citep{touvron2023llama} to generate harmless outputs, helpful outputs, and high-quality story outlines. %\ours{} outperforms both RLAIF and context distillation baselines in pairwise comparisons on 
As shown in Sec. \ref{sec:experiments}, \ours{} substantially outperforms both RLAIF and context distillation baselines in pairwise comparisons when simulating preference data with LLaMA-7B, while still performing equal or better when simulating with LLaMA-30B. 
%On all three tasks, \ours{} substantially outperforms both RLAIF and context distillation baselines in pairwise comparisons---by a margin of at least 9\% and often more than 30\%---validating our method's efficacy. 
We will release all code at a later date, although in any case \ours{} is fairly easy to implement by modifying any reference RLAIF codebase. %We release all code at \todo{github link}.
    
    \section{Related Works} 
    \label{sec: review}
    Different approaches have been employed to address the challenge of reducing pollutant emissions, particularly in the context of route planning with an emission cap for green vehicles or automated vehicles. This topic has gained significant attention in recent research \cite{Islam2021, Abdullahi2021, Olgun2021, Cai2021}. On the other hand, studies have also explored the development of smart mobility devices aimed at minimising the impact of polluting devices on individuals. For instance, in \cite{Sweeney2019}, a cyber-physical control system was designed to regulate the interaction between cyclists and electric motors. This system effectively manages the ventilation rate of cyclists by intelligently employing electric motor assistance. Similarly, an algorithm was developed in \cite{Herrmann2018} to control the switching of hybrid vehicle motors between electric and combustion modes, aiming to protect cyclists from the harmful effects of exhaust-gas emissions. Other similar studies have also been conducted in \cite{Dong2019, Hong2021, Umezawa2022}.

Moreover, researchers have also explored methodologies for user-health-friendly path planning. In the study mentioned in \cite{Langbridge2022}, the authors propose the concept of personalised commute optimisation as a means to significantly reduce the risk of pollution exposure. They highlight the potential benefits of tailoring commute routes and modes of transportation to individual users, taking into account factors such as pollution levels, personal health conditions, and preferences. Additionally, the authors present a proof-of-concept system for optimising pollution inhalation during the commute. This system aims to provide users with personalised recommendations for travel routes and transportation options that minimize their exposure to pollutants, thereby promoting healthier and more sustainable commuting practices.

A recent study \cite{LuengoOroz2019} investigated road safety and exposure to air pollution along three routes from central Edinburgh to the science and engineering campus of the University of Edinburgh. The results revealed significant differences in ultrafine particle exposure across the three routes. Additionally, a study conducted in Celje, Slovenia, proposed an alternative route to reduce cyclist exposure to black carbon based on the analysis of air quality data \cite{Jereb2018}. In the study referenced \cite{Sweeney2017}, a cyber-physical system is proposed as a solution to mitigate the impact of urban pollution on cyclists. The system operates by indirectly controlling the breathing rate of cyclists when they traverse polluted areas. A notable study conducted in \cite{Samal2020} focused on optimising the routing plan for smart city users with a specific emphasis on reducing $PM_{10}$ levels. This was accomplished by implementing the Empirical Bayesian Kriging (EBK) model during the geospatial analysis phase. In a subsequent work \cite{Samal2021}, the authors extended their approach by incorporating convolutional neural networks to extract features and utilising long short-term memory models for capturing the sequential dependency of air quality. The results consistently demonstrated that although the predicted path might be longer than the shortest route, it effectively minimises the risk of pollution exposure. 

Furthermore, the authors devised an intelligent speed advisory system in \cite{Gu2018} that provides recommendations on a suitable speed for a group of cyclists, considering the varying fitness levels within the group or the different levels of electric assistance, particularly in cases where some or all cyclists utilize electric bikes. Similarly, the authors \cite{Wu2022} developed a random forest land-use model to map $PM_{2.5}$ concentrations by investigating the exposure among cyclists in three Asian cities. By employing spatial k-fold cross-validation, the authors found that the proposed model achieved an $R^2$ value exceeding 0.67. Notably, the study concluded that for cycling commuters, selecting alternative routes could result in a reduction of over 30\% in $PM_{2.5}$ exposure.

However, the studies mentioned earlier could benefit from the utilisation of real-world data and the consideration of multiple pollutants. In our work, we address these limitations by employing an optimisation algorithm to identify the route with the minimum air pollution levels. This is achieved by utilising real-world data that encompasses multiple pollutants, allowing users to customise the weights assigned to each pollutant based on their preferences and requirements.
    
    \section{Research Problem Statement}
    \label{sec: state}
    In this section, we describe the system model for our application. The main objective of our application is to recommend a green route to a cyclist that minimizes their exposure to a specified pollutant index while travelling from a specific starting point to a destination within a bounded geographical area. Specifically, we model the area of interest using a weighted directed graph $G = (V, E, W)$, where $V = \left\lbrace 1, \ldots, n \right\rbrace$ is the vertex set, and $E = \left\lbrace e_1, \ldots, e_m \right\rbrace$ is the set of edges with corresponding positive weights in $W = \left\lbrace w_1, \ldots, w_m \right\rbrace$. The nodes in the graph represent intersections or points of interest, while the edges indicate the road segments that connect these nodes. The weight $w_i$ associated with the edge $e_i$ represents the aggregated level of pollutant exposure that a cyclist may encounter while travelling along that edge, based on historical data measurements collected at that edge and the vulnerability of the cyclist to each pollutant being considered. To further illustrate this point, mathematically, let $K$ denote the total number of pollutants being considered, and let $p_{ij}$ be the average pollutant level for the $j$'th pollutant measured at the $i$'th edge, where $i = 1, \ldots, n$ and $j = 1, \ldots, K$, and $l_i$ be the length of the $i$'th edge, then $w_i$ can be defined as follows:
    
\begin{equation}
    w_i = l_i \sum_{j=1}^{K} \alpha_j p_{ij}
\end{equation}  

\noindent where $\alpha_1, \alpha_2, \ldots, \alpha_K$ are positive coefficients capturing the vulnerability of each pollutant to the cyclist, subject to the constraint $\sum_{j=1}^{K} \alpha_j = 1$ for normalization. It is worth noting that given the diversity of different pollutants when they are combined for impact assessment, a min-max scaling is employed for each type of pollutant, implying that $p_{ij} \in \left[0, 1\right]$ and $w_i \in \left[0, l_i\right]$ which can be seen as a parameter proportionally weighted based on the length of the $i$'th edge road segment $l_i$ and the vulnerability of the cyclist to the $K$ types of pollutants. This further implies that if the cyclist's vulnerability across different road segments is the same, the optimization model should favour the segment with the shortest length. 

Given this context, our optimization problem is to find a path $p = \langle v_1, v_2, \dots, v_h\rangle$ which consists of a sequence of vertices starting from the source vertex $v_1 \in V$ and terminating at the destination vertex $v_h \in V$ that minimizes the sum of the weights of its constituent edges. Mathematically, we have:

\begin{equation} \label{spp}
    \underset{p}{\operatorname{argmin}} \sum_{i=1}^{h-1} w(v_i, v_{i+1})
\end{equation}

\noindent where $w(u, v) \in W$ denotes the weight of the edge connecting vertices $u$ and $v$. 
    
    \section{System Architecture}
    \label{sec: arch}
    This section introduces the proposed system architecture designed to address the research questions outlined in \autoref{sec: state}. An overview of the system architecture is presented in \autoref{fig: overview}. The diagram illustrates the components of our system, which consists of a mobile application serving as the frontend to interact with users, a backend server responsible for data transfer and processing, a database for data storage, and an optimisation algorithm to generate optimised routes based on input road parameters. Our project is open-source on GitHub\footnote{\url{https://github.com/SFIEssential/BreathGreen}}, so any interested readers are encouraged to visit and check out.

% Figure environment removed

\subsection{Frontend}

    Our Android mobile application was developed using IntelliJ IDEA and subsequently tested on an Android Virtual Device featuring a $Google\,Play\,Intel\,Atom\,(x86)$ processor. The interactive map functionality within the application was implemented using the Google Maps API\footnote{\url{https://developers.google.com/maps/documentation/directions}}, which enables the display of origin and destination locations, as well as routes with the shortest distance or lowest pollutant index. 

\subsection{Backend}

    Based on a data-driven approach, for the purpose of real-time response, we utilised Flask \cite{grinberg2018flask}, a backend framework written in Python. Following the design of REST, the Flask server connects with frontend using JSON and calculates various metrics for self-optimisation. When starting the server, it loads the geographical information of Dublin from OpenStreetMap and appends the pollutant data collected from Google to generate a new digital map containing pollutant attributes on the top of roads. The routing algorithm that we propose calculates pollution level based on different setting acquired from users and output the entire route from start to the end with lowest degree of pollutants around. 

\subsection{Database \& Algorithm}

    The interaction between our database and the backend is bidirectional. Data received from the backend is stored in our database, and the database is also used to retrieve and return data to the backend as per its requirements.

    Similarly, the optimisation algorithm generates path recommendations based on the parameters received from the backend. The algorithm processes this information, calculate the optimised path and returns the results to the server. A more detailed explanation of the algorithm will be provided in the subsequent section.
    
    \section{Algorithm \& Experiments}
    \label{sec: algo_exp}
    \subsection{Algorithm}
    
    There are several well-known algorithms available to address \eqref{spp}, including Dijkstra's algorithm, Floyd-Warshall algorithm, and Bellman-Ford algorithm \cite{sapundzhi2018optimization}. These algorithms are designed to identify the shortest or most efficient path between two points in a weighted directed graph. For ease of reference, we shall now present a description of Dijkstra's Algorithm with reference to \cite{cormen2009introduction, Szczepanski2021} below. In \autoref{alg: dijkstra}, the vertex set is denoted by $Q$, and the array $dist$ contains the current distances from the starting point $s$ to other vertices, where $dist[v]$ represents the current distance from $s$ to vertex $u$. The array $prev$ contains pointers to the previous-hop nodes on the shortest path from $s$ to the given vertex, or equivalently, it is the next hop on the path from the given vertex to the source. The variable $alt$ denotes the length of the path from the root node to the neighbour node $v$ if it were to go through $u$. If this path is shorter than the current shortest path recorded for $v$, the current path would be replaced with this $alt$ path.
    
    \begin{algorithm}
        \caption{Dijkstra's Algorithm}
        \label{alg: dijkstra}
        \begin{algorithmic}
            \renewcommand{\algorithmicrequire}{\textbf{Input:}}
            \renewcommand{\algorithmicensure}{\textbf{Output:}}
            \REQUIRE $G=(V,E,W)$, a weighted directed graph, and $s, d \in V$, the starting point $s$ and destination point $d$
            \ENSURE  $p$, the shortest path for \eqref{spp}
            \FORALL{vertex $v$ in graph $G$}
                \STATE $dist[v] \gets infinite$
                \STATE $prev[v] \gets undefined$
                \STATE Add vertex $v$ to vertex set $Q$
            \ENDFOR
            \STATE $dist[s] \gets 0$
            \WHILE{$Q$ is not empty}
                \STATE $u \gets \underset{i}{\operatorname{argmin}}\{dist[i]\,|\,\forall i \in Q\} $
                \STATE Remove $u$ from $Q$
                \FORALL{neighbour $v$ of $u \in Q$}
                    \STATE $alt \gets dist[u] + w(u,v)$
                    \IF{$alt<dist[v]$}
                        \STATE $dist[v] \gets alt$
                        \STATE $prev[v] \gets u$
                    \ENDIF
                \ENDFOR
            \ENDWHILE
            \STATE Set $p$ as the shortest path to $d$ using $prev$ array
        \end{algorithmic}
    \end{algorithm}

\subsection{Dataset}
    
    This section describes the \texttt{AirView{\_}Dublin{\_}City} (\texttt{ADC}) dataset which is recently released and the preprocessing steps followed to address challenges faced during the data collection stage. Additionally, we provide insights and visualizations of the resulting dataset.
    
    % \begin{enumerate}
    %     \item Describe the air quality dataset([]). Done.
    %     \item Explain why we used the Road Segment table instead of the measurements table.  Done.
    %     \item Describe how this dataset is linked to OSM Dataset.  Done.
    %     \item Data preprocessing on the Road Segment table.  Done.
    %     \item Include some visualizations.  Done.
    % \end{enumerate}

    \textbf{Google Project Air View Data - Dublin City (May 2021 - August 2022):}
    This dataset was collected by Google and Dublin City Council as part of Project Air View Dublin. Google's first electric Street View car, equipped with Aclima’s mobile air sensing platform, drove through the roads of Dublin City measuring street-by-street air quality. The data collection process predominantly took place from Monday to Friday between 9:00 am and 5:00 pm from May 2021 to August 2022. Consequently, the dataset represents typical daytime weekday air quality. The sensors on the car measured pollution on each street and highway at 1-second intervals while driving with the flow of traffic at normal speeds. The pollutants included in the dataset are Carbon Monoxide (CO), Carbon Dioxide (CO2), Nitrogen Dioxide (NO2), NO (nitric oxide), Ozone (O3), and Particulate Matter PM2.5 (including size-resolved particle counts from 0.3 - 2.5 \textmu m). The following two versions of the dataset are publicly available\footnote{\url{https://data.smartdublin.ie/dataset/google-airview-data-dublin-city}}:
    
    \begin{outline}
        \1 \texttt{Airview{\_}Dublin{\_}City{\_}Measurements} (\texttt{ADC-M}) is the 1-second intervals data captured during the period.
        \1 \texttt{AirView{\_}Dublin{\_}City{\_}RoadData} (\texttt{ADC-R}) is the 1-second data points aggregated in approximately 50m road segments.
    \end{outline}

    In this work, the \texttt{ADC-R} was used because it groups the individual points/measurements from \texttt{ADC-M} into segments. To report the concentration of each pollutant per segment, first, they get the mean of the number of measurements on this road segment, then count the number of drive passes on this road segment. Finally, for each pollutant, they report the concentration (median of drive pass mean) in µg/m3. \texttt{ADC-R} also contains the OSM{\_}IDs segments to associate this dataset with the \texttt{OpenStreetMap} (\texttt{OSM}) dataset. The OSM{\_}IDs nodes were extracted and fed into the optimization algorithm. 

\subsection{Data Cleaning \& Prepossessing}

    In this work, the \texttt{ADC-R} dataset was used as mentioned previously. Even though this dataset is an aggregation of the raw data \texttt{ADC-M} and was already processed by Google, this work required the following pre-processing:
    
    \begin{outline}
        \1 Data interpolation for missing and negative values: the dataset includes NaNs and negative values, which may arise from sensor malfunctions or miscalibrations, potentially impacting the optimization algorithm. To address this issue, a linear interpolation method was applied to each pollutant column. Initially, the dataset was sorted by OSMID, and all negative values were replaced with NaNs. Subsequently, a linear interpolation was performed using the Python Pandas library, enabling the estimation of missing values and the smoothing of the dataset.

        \1 Cross-referencing of the \texttt{OSM} and \texttt{ADC-R} datasets: the optimisation algorithm utilizes the nodes and segments from the \texttt{OSM} dataset to determine the greener path. To establish a link between these two datasets, the OSM\_IDs are used as a reference. It is important to note that the vehicle used to collect the Google Air Quality dataset did not traverse all streets and roads in Dublin, resulting in measurements being available only for specific segments. To assign these measurements to the corresponding segments in the \texttt{OSM} dataset, the following process was conducted: 
        
            \2 Delimiting the area of interest: the focus was on the central region of Dublin. In order to cover the areas frequented by shared bikes, we define the area of interest based on the map of NOW dublinbikes\footnote{\url{https://www.dublinbikes.ie/en/mapping}} and all segments and nodes within the coordinates [-6.230852, 53.359967, -6.310015, 53.330091] (as shown in \autoref{fig: snapshot_dublin}) were extracted.
    
            \2 Matching process: by utilising the OSM\_ID column, a matching process was performed between the road dataset and the \texttt{OSM}. Segments existing in both datasets were selected and their corresponding pollutant measurements were merged. This resulted in the creation of a final dataset with the following format: [osm\_id, NO2\_ugm3, NO\_ugm3, CO2\_mgm3, CO\_mgm3, O3\_ugm3, PM25\_ugm3]. This final dataset serves as the input for the optimization algorithm.
            
        \1 Finally, in order to address the issue of missing data on particular segments, a clustering algorithm is employed to identify the nearest neighbours from those segments. In particular, the position of a road segment is represented by its edge points, encompassing two endpoints and, if present, all inflexion points. Subsequently, the BallTree algorithm is applied to select the $n$ closest neighbours for each target, which refers to the edge point in a road segment with missing data, within our dataset. The attributes of the target are then determined by calculating the mean value of its neighbours' attributes, thereby enabling the missing data to be filled in based on geographic relevance. The histogram depicted in \autoref{fig: neighbour hist} provides a visualisation of the distances between measurements and various segments. Considering the observed distribution of distances, the selection of 3, 5, or 10 neighbours is made.
        
    \end{outline}

    % Figure environment removed

    % Figure environment removed

    The data before and after cleaning and preprocessing is visually depicted on the map in \autoref{fig: data filling result}. The orange boundary corresponds to the area shown in \autoref{fig: snapshot_dublin}. The red and blue routes on the map represent the roads with and without missing values respectively. 

    % Figure environment removed

\subsection{Green Route Planning}

    Our experiments were designed to assess the impact of our algorithm on the selection of green routes in various real-world scenarios. To achieve this, we generated and compared the shortest routes with the greenest routes under different setups, as outlined below:
    
    \begin{enumerate}
         
    
    
        \item In the first setup, we considered all pollutants and assigned them equal weights. We conducted the experiments with a thousand randomly selected origin-destination pairs on the map, focusing primarily on walking and private micromobility modes such as private bicycles and e-bikes. Thus, we denote this setup as ``\textbf{Private Device}''.
        
        \item In the second setup, we again considered all pollutants and assigned them equal weights. However, this time we conducted the experiments with ten origin-destination pairs, selecting GPS coordinates from the map of NOW dublinbikes. This setup specifically targeted transportation systems with shared devices, such as shared bicycles and shared e-bikes, and we call this setup ``\textbf{Shared Device}''.

        \item In the third setup, we examined the individual influence of different pollutants. We conducted the experiments with one origin-destination pair selected on the map, varying the weights assigned to different pollutants. For example, we assigned a weight of 1 to one pollutant, such as CO, and a weight of 0 to the other pollutants. This allowed us to obtain green routes that focused solely on reducing the specific pollutant. This setup aimed to evaluate the performance of our optimization algorithm under different pollutant weightings. This setup is named ``\textbf{Single Pollutant}''.
        
    \end{enumerate}

    By conducting these experiments, we aimed to gain insights into the performance and effectiveness of our optimization algorithm in different real-world scenarios and pollutant weightings. The results and relevant discussions are included in \autoref{sec: res_dis}.

%   Useful links: 
% 1. https://data.smartdublin.ie/dataset/google-airview-data-dublin-city

    \section{Results \& Discussion}
    \label{sec: res_dis}
    In this section, we present a summary of our experimental results and provide visual representations. Furthermore, we offer detailed discussions and analysis of the obtained results.

\subsection{Private Device}

    We performed a series of experiments with a thousand different origin-destination pairs, and the summarised results are presented in \autoref{tab: private device results}. The findings clearly indicate that while the green path shows an average increase in trip distance of 15.15\% compared to the shortest route, it simultaneously leads to a reduction in pollutant intake by 17.87\%. \autoref{fig: p over d} visualised the relationship between the pollutant saving rate and distance increase rate, with yellow scatter points representing the data. The graph is partitioned into six regions based on the linear boundaries defined by the equations $y=0$, $y=0.25x$, $y=0.5x$, $y=x$, $y=2x$, $y=4x$ and $x=0$. Additionally, the percentage of scatter points within each region is presented on the graph. The analysis of the graph reveals that the majority of scatter points are located in the region bounded by $y=x$ and $y=2x$. This observation indicates that, in most cases, the rate of pollutant saving exceeds the rate of distance increase. Hence, our algorithm demonstrates a favourable trade-off between reducing pollutant inhalation and the increase in trip distance. 
    
    It is worth noting that in some cases, certain routes in our experiments coincide with both the shortest path and the green path, resulting in zero values for pollutant savings and additional distance.

    \begin{table}
        \caption{Private device results.}
        \label{tab: private device results}
        \begin{tabularx}{\linewidth}{@{\extracolsep{\fill}}c c c c c}
            \toprule
            & \makecell[c]{ShortPath \\ distance (m)} & \makecell[c]{GreenPath \\ distance (m)} & \makecell[c]{Pollutant \\ saving (\%)} & \makecell[c]{Extra \\ distance (\%)} \\
            \midrule
            mean    & 2870.31 & 3328.76 & 17.87 & 15.15 \\
            std     & 1267.07 & 1572.03 & 12.20 & 13.93  \\
            min     & 208.20  & 208.20  & 0     & 0     \\
            25\%    & 1927.26 & 2137.51	& 7.72  & 5.09  \\
            50\%    & 2820.44 & 3284.45 & 17.20 & 12.11 \\
            75\%    & 3711.74 & 4304.19 & 26.43 & 21.43 \\
            max     & 7303.63 & 9265.65 & 64.99 & 83.22 \\
            \bottomrule
        \end{tabularx}
    \end{table}

    % Figure environment removed

\subsection{Shared Device}

    Another series of experiments were conducted with ten different origin-destination pairs, and the results are summarised in \autoref{tab: shared device results}. Similar findings can be observed in this set of experiments compared to the previous ones. Our algorithm, when applied to shared micro-mobilities, also leads to reduced pollutant inhalation at the expense of a certain increase in trip distance.

    \begin{table}
        \caption{Shared device results.}
        \label{tab: shared device results}
        \begin{tabularx}{\linewidth}{@{\extracolsep{\fill}}c c c c c}
            \toprule
            & \makecell[c]{ShortPath \\ distance (m)} & \makecell[c]{GreenPath \\ distance (m)} & \makecell[c]{Pollutant \\ saving (\%)} & \makecell[c]{Extra \\ distance (\%)} \\
            \midrule
            mean    & 2460.99 & 2921.88 & 14.68 & 17.59 \\
            std     & 1206.03 & 1459.44 & 10.37 & 17.21 \\
            min     & 769.17  & 769.17  & 0     & 0     \\
            25\%    & 1480.41 & 1652.79	& 5.93  & 4.94  \\
            50\%    & 2413.75 & 2762.74 & 15.42 & 15.62 \\
            75\%    & 3464.61 & 4253.95 & 20.84 & 21.81 \\
            max     & 4264.24 & 5079.85 & 31.12 & 54.36\\
            \bottomrule
        \end{tabularx}
    \end{table}

\subsection{Single Pollutant}

    The paths determined by different pollutants are illustrated in \autoref{fig: single pollutant}. The red path represents the shortest route, and the yellow path corresponds to the route determined by NO, which results in a 52.07\% reduction in NO inhalation. Similarly, the green and blue paths, determined by CO and NO2 respectively, also lead to reductions in pollutant inhalation by 27.52\% and 50.57\%. It is evident from the figure that our algorithm is capable of recommending different routes for the same origin-destination pair based on varying pollutant weights.

    % Figure environment removed
    
    
    \section{Conclusions}
    \label{sec: con}
    Our research aimed to maximise health and environmental benefits for active transportation participants. We developed a system that recommends trip routes to minimise pollutant inhalation based on the analysis of air quality data in Dublin, Ireland, which was recently released by Google. Through various experiments, we evaluated the performance of our optimisation algorithm and found it to be reliable and practical in diverse real-world scenarios, catering to both private and shared micro-mobility users, such as e-bike riders and shared bike users. Our results lead to a remarkable reduction in pollutant intake, with a significant decrease of 17.87\% on average and a maximum of around 64.99\%. These findings underscore the efficacy and value of our approach in improving the environmental well-being of individuals. 

Furthermore, it is important to realise that our work could be improved by a more sophisticated and rigorous pollutant evaluation standard or metrics and the enhancement of completeness and comprehensiveness of the dataset. By addressing these areas of improvement, we can enhance the quality and reliability of our work, contributing to a more advanced and comprehensive understanding of the environmental impact of outdoor activities and the effectiveness of our system.

	
	

    \section*{Acknowledgement}
       \label{sec: ack}
            \section*{Acknowledgements}%

The authors express their gratitude and a fond thought to Hassan \ak, who
with Gabriella Pasi set out to define a fuzzy version of OSF logic. This
paper originates from their work on the definition of similarity-based
unification for OSF terms, extending the approach of
\cite{AitKaciPasi2020}.

    
    \bibliographystyle{ieeetr}
    \bibliography{references}
\end{document} 