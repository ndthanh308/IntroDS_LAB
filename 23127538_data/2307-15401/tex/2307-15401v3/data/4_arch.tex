This section introduces the proposed system architecture designed to address the research questions outlined in \autoref{sec: state}. An overview of the system architecture is presented in \autoref{fig: overview}. The diagram illustrates the components of our system, which consists of a mobile application serving as the frontend to interact with users, a backend server responsible for data transfer and processing, a database for data storage, and an optimisation algorithm to generate optimised routes based on input road parameters. Our project is open-source on GitHub\footnote{\url{https://github.com/SFIEssential/BreathGreen}}, so any interested readers are encouraged to visit and check out.

% Figure environment removed

\subsection{Frontend}

    Our Android mobile application was developed using IntelliJ IDEA and subsequently tested on an Android Virtual Device featuring a $Google\,Play\,Intel\,Atom\,(x86)$ processor. The interactive map functionality within the application was implemented using the Google Maps API\footnote{\url{https://developers.google.com/maps/documentation/directions}}, which enables the display of origin and destination locations, as well as routes with the shortest distance or lowest pollutant index. 

\subsection{Backend}

    Based on a data-driven approach, for the purpose of real-time response, we utilised Flask \cite{grinberg2018flask}, a backend framework written in Python. Following the design of REST, the Flask server connects with frontend using JSON and calculates various metrics for self-optimisation. When starting the server, it loads the geographical information of Dublin from OpenStreetMap and appends the pollutant data collected from Google to generate a new digital map containing pollutant attributes on the top of roads. The routing algorithm that we propose calculates pollution level based on different setting acquired from users and output the entire route from start to the end with lowest degree of pollutants around. 

\subsection{Database \& Algorithm}

    The interaction between our database and the backend is bidirectional. Data received from the backend is stored in our database, and the database is also used to retrieve and return data to the backend as per its requirements.

    Similarly, the optimisation algorithm generates path recommendations based on the parameters received from the backend. The algorithm processes this information, calculate the optimised path and returns the results to the server. A more detailed explanation of the algorithm will be provided in the subsequent section.