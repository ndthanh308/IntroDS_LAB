1. Negative values: 
https://www.indsci.com/en/blog/why-do-i-see-negative-readings-on-my-gas-monitor

2. Reviewer feedback:
The paper is a well-written description of a weighted-shortest-path algorithm that uses Open Street Map (OSM) street centerline maps and newly available Google Air Quality data to compare the pollution associated with traversing alternative paths through a metropolitan area. It identifies and compares the extent to which shortest pollution-exposure paths for common trips in Dublin are longer but less polluting than the comparable shortest paths. The results suggest that 15-18\% reductions in the air pollution absorbed by the breathing of a typical traveler are possible by choosing alternative paths that are longer, by similar percentages, than the shortest path.

The general idea is not new, but the use of publicly available datasets and a general-purpose, scalable application is new.  The paper uses recently available spatially-detailed pollution data that has been collected, aggregated, and made publicly available by Google.  The paper does a good job of explaining the relevant datasets, algorithms, APIs, and software used to implement a smartphone version of its minimum-pollution navigation tool.  The data, method, and results constitute reasonable choices and are well-explained.  The paper will be of interest to the audience and could be publishable with minor revisions.  My main concern about the paper is that it says too little about (a) the spatial variation in pollution levels within the study area, and (b) the sensitivity of the results to the level of aggregation. The next few paragraphs explain these issues.

Figure 4 maps the streets in the study area and colours street segments to show which were changed in the cleaning and pre-processing.  The differences are too hard to see given the map size and, moreover, the differences appear to show only the street segments not directly estimated by Google in ADC-R. This interpolation method appears reasonable, but more needs to be said about the temporal and spatial variation in pollution levels assigned to the OSM street segments by Google. There is no scale on the maps but the study area appears to be a roughly 8 x 4 km rectangle. The histograms in Fig. 3 are not adequately explained.  show the frequency of occurrence of street segments.  What is the total number of street segments that had pollution estimates, that were interpolated, and that remained unestimated?  What fraction of the interpolated segments had their value assigned using the 3, 5, or 10 closest neighbours (i.e., the splits shown in Fig. 3 a,b,c)?  Do the histograms show the distances to all road segments (that have pollution estimates) from those segments that needed interpolation where these segments had already been grouped into those whose interpolation was based on 3, 5, or 10 closest neighbours?  How can the mean of the 10-closest histogram in Fig. 3-c be around 4 km when the study area is only 8x4 km and has a high density of street segments?

Fig. 6 helps to show how much the shortest and safest paths differ for a sample origin/destination trip and a particular pollutant. That’s informative but it would be helpful to show the spatial variation in air pollution measurements across all the street segments by providing a table with descriptive statistics plus a thematic map of, for example, NO2 measurements.	To make room, combine the two maps in Fig. 4 into one map with colours showing three types of road segments: those with Google interpolations, those that you interpolated, and those that you left unestimated.

There is no discussion of temporal and spatial variation in pollution levels and too little mention of other assumptions regarding the pollution that a traveller will encounter.  The Google measurements average readings made by Google StreetView vans at one-second intervals as they traverse local streets from 9 am to 5 pm.  Pollution levels (from vehicles) on roadways are known to be quite sensitive to congestion levels and to disperse quickly as one moves away from streets (depending upon wind).  They are noticeably higher at intersections with traffic lights than along the middle of road segments that are experiencing free-flow travel.  The 9-5 window aggregates peak and off-peak periods and individual street segments are measured at very different times within the window depending on the path of the StreetView van.  Also, walkers and bikers don’t traverse a path at a constant speed and are likely to linger at intersections, especially if there are street lights or congestion.  Traversing paths that are separated from roadways by bike lanes, medians, etc. could encounter much less pollution than moving in or very near auto traffic lanes.

What assurance can the authors provide that the differences in estimated pollution are meaningful?	The paper is interesting enough to be publishable without extensive additional data, but the authors could do more to test the sensitivity of the results. The first step is to compare the pollution savings with some estimate of, say, the day-to-day; peak/off-peak; or one-block-away measurement differences.  Another possibility would be to adjust the ``least-polluting" algorithm to penalize certain turns at intersections and/or assume some variation in travel speed depending on the road segments traversed and the presence of street lights.