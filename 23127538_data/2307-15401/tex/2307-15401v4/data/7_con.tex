Our research aimed to maximise health and environmental benefits for active transportation participants. We developed a system that recommends trip routes to minimise pollutant inhalation based on the analysis of air quality data in Dublin, Ireland, which was recently released by Google. Through various experiments, we evaluated the performance of our optimisation algorithm and found it to be reliable and practical in diverse real-world scenarios, catering to both private and shared micro-mobility users, such as e-bike riders and shared bike users. Our results lead to a remarkable reduction in pollutant intake, with a significant decrease of 17.87\% on average and a maximum of around 64.99\%. These findings underscore the efficacy and value of our approach in improving the environmental well-being of individuals. 

Furthermore, the enhancement of our work can be achieved through the implementation of a more sophisticated and rigorous pollutant evaluation standard or metrics, as well as by augmenting the completeness and comprehensiveness of the dataset. Additional avenues for improvement include conducting comparisons of pollution savings against estimated variations, such as day-to-day differences, peak and off-peak measurements, or disparities between routes in close proximity. Another viable approach entails modifying the 'least-polluting' algorithm by introducing penalties for specific turns at intersections or incorporating diverse travel speeds based on the road segments traversed and the presence of street lights. Addressing these areas of refinement holds the potential to bolster the quality and reliability of our research, thereby contributing to a more advanced and comprehensive understanding of the environmental impact of outdoor activities and the effectiveness of our system.