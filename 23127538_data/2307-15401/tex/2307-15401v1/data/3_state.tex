In this section, we describe the system model for our application. The main objective of our application is to recommend a green route to a cyclist that minimizes their exposure to a specified pollutant index while travelling from a specific starting point to a destination within a bounded geographical area. Specifically, we model the area of interest using a weighted directed graph $G = (V, E, W)$, where $V = \left\lbrace 1, \ldots, n \right\rbrace$ is the vertex set, and $E = \left\lbrace e_1, \ldots, e_m \right\rbrace$ is the set of edges with corresponding positive weights in $W = \left\lbrace w_1, \ldots, w_m \right\rbrace$. The nodes in the graph represent intersections or points of interest, while the edges indicate the road segments that connect these nodes. The weight $w_i$ associated with the edge $e_i$ represents the aggregated level of pollutant exposure that a cyclist may encounter while travelling along that edge, based on historical data measurements collected at that edge and the vulnerability of the cyclist to each pollutant being considered. To further illustrate this point, mathematically, let $K$ denote the total number of pollutants being considered, and let $p_{ij}$ be the average pollutant level for the $j$'th pollutant measured at the $i$'th edge, where $i = 1, \ldots, n$ and $j = 1, \ldots, K$, and $l_i$ be the length of the $i$'th edge, then $w_i$ can be defined as follows:
    
\begin{equation}
    w_i = l_i \sum_{j=1}^{K} \alpha_j p_{ij}
\end{equation}  

\noindent where $\alpha_1, \alpha_2, \ldots, \alpha_K$ are positive coefficients capturing the vulnerability of each pollutant to the cyclist, subject to the constraint $\sum_{j=1}^{K} \alpha_j = 1$ for normalization. It is worth noting that given the diversity of different pollutants when they are combined for impact assessment, a min-max scaling is employed for each type of pollutant, implying that $p_{ij} \in \left[0, 1\right]$ and $w_i \in \left[0, l_i\right]$ which can be seen as a parameter proportionally weighted based on the length of the $i$'th edge road segment $l_i$ and the vulnerability of the cyclist to the $K$ types of pollutants. This further implies that if the cyclist's vulnerability across different road segments is the same, the optimization model should favour the segment with the shortest length. 

Given this context, our optimization problem is to find a path $p = \langle v_1, v_2, \dots, v_h\rangle$ which consists of a sequence of vertices starting from the source vertex $v_1 \in V$ and terminating at the destination vertex $v_h \in V$ that minimizes the sum of the weights of its constituent edges. Mathematically, we have:

\begin{equation} \label{spp}
    \underset{p}{\operatorname{argmin}} \sum_{i=1}^{h-1} w(v_i, v_{i+1})
\end{equation}

\noindent where $w(u, v) \in W$ denotes the weight of the edge connecting vertices $u$ and $v$. 