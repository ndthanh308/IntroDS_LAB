The detrimental effects of air pollution, specifically nitrogen oxides (NOx) and particulate matter (PM), on human health have emerged as a significant and escalating concern. These pollutants are predominantly emitted from diverse sources such as vehicle exhaust, industrial activities, and residential heating. The well-documented consequences on human health are substantial, resulting in a considerable number of excess deaths globally (7 million) and in Ireland (3300) annually. 

NOx refers to a group of gases that includes nitrogen oxide (NO) and nitrogen dioxide (NO2), both of which are significant contributors to air pollution. These gases have the potential to irritate the respiratory system and worsen pre-existing respiratory conditions, such as asthma, which is a prevalent health issue in Ireland (ranking fourth highest in the world). They can also increase the susceptibility to respiratory infections. Furthermore, long-term exposure to NOx has been associated with the development of cardiovascular diseases, including heart attacks and strokes. The elevated concentrations of NOx in urban areas, particularly in proximity to busy roads and industrial zones, pose a greater health risk to individuals, especially vulnerable groups such as the young, elderly, and those with preexisting health conditions.

Particulate matter (PM) refers to small particles suspended in the air, which are classified into two main categories based on their size: PM10 (particles with a diameter of 10 micrometres or less) and PM2.5 (particles with a diameter of 2.5 micrometres or less). These particles have the ability to penetrate deeply into the respiratory system when inhaled, leading to respiratory issues such as coughing, wheezing, and shortness of breath. Moreover, they can have detrimental effects on cardiovascular health, including an increased risk of heart attacks and irregular heartbeats. Additionally, exposure to PM has been associated with the development of lung cancer and adverse birth outcomes.

In Ireland, NOx and PM emissions originate from various sources. Road transportation, especially vehicles powered by diesel engines, constitutes a significant contributor to NOx emissions. The combustion of solid fuels, such as coal and peat, for residential heating purposes, is another prominent source of PM emissions. Industrial activities encompassing power generation and manufacturing processes also contribute significantly to the release of both NOx and PM pollutants into the atmosphere. It is crucial to address these diverse emission sources and implement effective measures to mitigate their impact on air quality and public health.

Mitigating exposure to air pollution can have significant benefits within Ireland, as the health concerns associated with it impose a cost of approximately €2.3 billion on the Irish State annually. Studies focusing on the environmental and public health implications in the transportation sector have underscored the urgent need to reduce greenhouse gas emissions and suspended particulate matter. Notably, research has been conducted to assess the levels of traffic-related pollutants on various school commuting routes in Canada \cite{Elford2019} and China \cite{An2022}, emphasising the importance of mitigating pollutant emissions, developing smart mobility devices that prioritise user health, and identifying more efficient routes with improved air quality. These efforts are crucial for safeguarding public health, especially for vulnerable road users.

In our research, we placed emphasis on identifying routes with minimal air pollution levels and offering corresponding recommendations to transportation users. The primary contributions of our work can be summarised as follows:

\begin{outline}
    \1 The analysis has been conducted on the air quality data from May 2021 to August 2022 in Dublin, Ireland.
    \1 A system has been developed to recommend trip routes to minimise pollutant inhalation for active transportation users.
    \1 Missing data has been filled in based on geographic relevance and used in experiments to examine the performance of our system in different situations.
\end{outline}

The structure of this paper is as follows. In \autoref{sec: review}, we provide a review of the relevant literature on the environmental and public health implications in the transportation field. The research question is explained in detail in \autoref{sec: state}. \autoref{sec: arch} describes the system architecture in our work, and \autoref{sec: algo_exp} describes and analyses our algorithm, the dataset and the experiments. The relevant discussions and analysis are included in \autoref{sec: res_dis}. Finally, we conclude our work in \autoref{sec: con} and discuss future plans and potential improvements.