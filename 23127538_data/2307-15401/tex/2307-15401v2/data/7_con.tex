Our research aimed to maximise health and environmental benefits for active transportation participants. We developed a system that recommends trip routes to minimise pollutant inhalation based on the analysis of air quality data in Dublin, Ireland, which was recently released by Google. Through various experiments, we evaluated the performance of our optimisation algorithm and found it to be reliable and practical in diverse real-world scenarios, catering to both private and shared micro-mobility users, such as e-bike riders and shared bike users. Our results lead to a remarkable reduction in pollutant intake, with a significant decrease of 17.87\% on average and a maximum of around 64.99\%. These findings underscore the efficacy and value of our approach in improving the environmental well-being of individuals. 

Furthermore, it is important to realise that our work could be improved by a more sophisticated and rigorous pollutant evaluation standard or metrics and the enhancement of completeness and comprehensiveness of the dataset. By addressing these areas of improvement, we can enhance the quality and reliability of our work, contributing to a more advanced and comprehensive understanding of the environmental impact of outdoor activities and the effectiveness of our system.
