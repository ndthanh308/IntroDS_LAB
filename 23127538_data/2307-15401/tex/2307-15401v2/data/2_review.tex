Different approaches have been employed to address the challenge of reducing pollutant emissions, particularly in the context of route planning with an emission cap for green vehicles or automated vehicles. This topic has gained significant attention in recent research \cite{Islam2021, Abdullahi2021, Olgun2021, Cai2021}. On the other hand, studies have also explored the development of smart mobility devices aimed at minimising the impact of polluting devices on individuals. For instance, in \cite{Sweeney2019}, a cyber-physical control system was designed to regulate the interaction between cyclists and electric motors. This system effectively manages the ventilation rate of cyclists by intelligently employing electric motor assistance. Similarly, an algorithm was developed in \cite{Herrmann2018} to control the switching of hybrid vehicle motors between electric and combustion modes, aiming to protect cyclists from the harmful effects of exhaust-gas emissions. Other similar studies have also been conducted in \cite{Dong2019, Hong2021, Umezawa2022}.

Moreover, researchers have also explored methodologies for user-health-friendly path planning. In the study mentioned in \cite{Langbridge2022}, the authors propose the concept of personalised commute optimisation as a means to significantly reduce the risk of pollution exposure. They highlight the potential benefits of tailoring commute routes and modes of transportation to individual users, taking into account factors such as pollution levels, personal health conditions, and preferences. Additionally, the authors present a proof-of-concept system for optimising pollution inhalation during the commute. This system aims to provide users with personalised recommendations for travel routes and transportation options that minimize their exposure to pollutants, thereby promoting healthier and more sustainable commuting practices.

A recent study \cite{LuengoOroz2019} investigated road safety and exposure to air pollution along three routes from central Edinburgh to the science and engineering campus of the University of Edinburgh. The results revealed significant differences in ultrafine particle exposure across the three routes. Additionally, a study conducted in Celje, Slovenia, proposed an alternative route to reduce cyclist exposure to black carbon based on the analysis of air quality data \cite{Jereb2018}. In the study referenced \cite{Sweeney2017}, a cyber-physical system is proposed as a solution to mitigate the impact of urban pollution on cyclists. The system operates by indirectly controlling the breathing rate of cyclists when they traverse polluted areas. A notable study conducted in \cite{Samal2020} focused on optimising the routing plan for smart city users with a specific emphasis on reducing $PM_{10}$ levels. This was accomplished by implementing the Empirical Bayesian Kriging (EBK) model during the geospatial analysis phase. In a subsequent work \cite{Samal2021}, the authors extended their approach by incorporating convolutional neural networks to extract features and utilising long short-term memory models for capturing the sequential dependency of air quality. The results consistently demonstrated that although the predicted path might be longer than the shortest route, it effectively minimises the risk of pollution exposure. 

Furthermore, the authors devised an intelligent speed advisory system in \cite{Gu2018} that provides recommendations on a suitable speed for a group of cyclists, considering the varying fitness levels within the group or the different levels of electric assistance, particularly in cases where some or all cyclists utilize electric bikes. Similarly, the authors \cite{Wu2022} developed a random forest land-use model to map $PM_{2.5}$ concentrations by investigating the exposure among cyclists in three Asian cities. By employing spatial k-fold cross-validation, the authors found that the proposed model achieved an $R^2$ value exceeding 0.67. Notably, the study concluded that for cycling commuters, selecting alternative routes could result in a reduction of over 30\% in $PM_{2.5}$ exposure.

However, the studies mentioned earlier could benefit from the utilisation of real-world data and the consideration of multiple pollutants. In our work, we address these limitations by employing an optimisation algorithm to identify the route with the minimum air pollution levels. This is achieved by utilising real-world data that encompasses multiple pollutants, allowing users to customise the weights assigned to each pollutant based on their preferences and requirements.