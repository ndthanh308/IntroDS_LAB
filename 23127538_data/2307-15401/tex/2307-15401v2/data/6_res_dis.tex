In this section, we present a summary of our experimental results and provide visual representations. Furthermore, we offer detailed discussions and analysis of the obtained results.

\subsection{Private Device}

    We performed a series of experiments with a thousand different origin-destination pairs, and the summarised results are presented in \autoref{tab: private device results}. The findings clearly indicate that while the green path shows an average increase in trip distance of 15.15\% compared to the shortest route, it simultaneously leads to a reduction in pollutant intake by 17.87\%. \autoref{fig: p over d} visualised the relationship between the pollutant saving rate and distance increase rate, with yellow scatter points representing the data. The graph is partitioned into six regions based on the linear boundaries defined by the equations $y=0$, $y=0.25x$, $y=0.5x$, $y=x$, $y=2x$, $y=4x$ and $x=0$. Additionally, the percentage of scatter points within each region is presented on the graph. The analysis of the graph reveals that the majority of scatter points are located in the region bounded by $y=x$ and $y=2x$. This observation indicates that, in most cases, the rate of pollutant saving exceeds the rate of distance increase. Hence, our algorithm demonstrates a favourable trade-off between reducing pollutant inhalation and the increase in trip distance. 
    
    It is worth noting that in some cases, certain routes in our experiments coincide with both the shortest path and the green path, resulting in zero values for pollutant savings and additional distance.

    \begin{table}
        \caption{Private device results.}
        \label{tab: private device results}
        \begin{tabularx}{\linewidth}{@{\extracolsep{\fill}}c c c c c}
            \toprule
            & \makecell[c]{ShortPath \\ distance (m)} & \makecell[c]{GreenPath \\ distance (m)} & \makecell[c]{Pollutant \\ saving (\%)} & \makecell[c]{Extra \\ distance (\%)} \\
            \midrule
            mean    & 2870.31 & 3328.76 & 17.87 & 15.15 \\
            std     & 1267.07 & 1572.03 & 12.20 & 13.93  \\
            min     & 208.20  & 208.20  & 0     & 0     \\
            25\%    & 1927.26 & 2137.51	& 7.72  & 5.09  \\
            50\%    & 2820.44 & 3284.45 & 17.20 & 12.11 \\
            75\%    & 3711.74 & 4304.19 & 26.43 & 21.43 \\
            max     & 7303.63 & 9265.65 & 64.99 & 83.22 \\
            \bottomrule
        \end{tabularx}
    \end{table}

    % Figure environment removed

\subsection{Shared Device}

    Another series of experiments were conducted with ten different origin-destination pairs, and the results are summarised in \autoref{tab: shared device results}. Similar findings can be observed in this set of experiments compared to the previous ones. Our algorithm, when applied to shared micro-mobilities, also leads to reduced pollutant inhalation at the expense of a certain increase in trip distance.

    \begin{table}
        \caption{Shared device results.}
        \label{tab: shared device results}
        \begin{tabularx}{\linewidth}{@{\extracolsep{\fill}}c c c c c}
            \toprule
            & \makecell[c]{ShortPath \\ distance (m)} & \makecell[c]{GreenPath \\ distance (m)} & \makecell[c]{Pollutant \\ saving (\%)} & \makecell[c]{Extra \\ distance (\%)} \\
            \midrule
            mean    & 2460.99 & 2921.88 & 14.68 & 17.59 \\
            std     & 1206.03 & 1459.44 & 10.37 & 17.21 \\
            min     & 769.17  & 769.17  & 0     & 0     \\
            25\%    & 1480.41 & 1652.79	& 5.93  & 4.94  \\
            50\%    & 2413.75 & 2762.74 & 15.42 & 15.62 \\
            75\%    & 3464.61 & 4253.95 & 20.84 & 21.81 \\
            max     & 4264.24 & 5079.85 & 31.12 & 54.36\\
            \bottomrule
        \end{tabularx}
    \end{table}

\subsection{Single Pollutant}

    The paths determined by different pollutants are illustrated in \autoref{fig: single pollutant}. The red path represents the shortest route, and the yellow path corresponds to the route determined by NO, which results in a 52.07\% reduction in NO inhalation. Similarly, the green and blue paths, determined by CO and NO2 respectively, also lead to reductions in pollutant inhalation by 27.52\% and 50.57\%. It is evident from the figure that our algorithm is capable of recommending different routes for the same origin-destination pair based on varying pollutant weights.

    % Figure environment removed