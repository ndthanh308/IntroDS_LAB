\section{Mesh editing}
\label{sec:mesh_editing}

We next describe step (4) of our pipeline (\secref{sec:pipeline}), namely, editing the topology of the mesh to simplify it in order to achieve good strips topology (objective \ref{obj:strips_topology}) and a good patch layout of both strips networks (objective \ref{obj:patch_layout}). This consists of aligning singularities and correcting winding strips (eg. the blue strips in \figref{fig:topological-defect}b and \figref{fig:routes}e). 
This step includes user interaction which we simplified to a small number of intuitive steps.
For our editing interface, we use the \emph{libigl} \cite{jacobson2018_libigl:-a-simple-c-geometry-processing-library} viewer.



\subsection{Topology Editing} 
\label{subsec:topology-editing}


% ----------------------------
% Topology editing
% ----------------------------

\subsubsection*{Dealing with topological defects} 
Our directional field, and consequent seamless parameterization and mesh, are optimized for smoothness and good alignment with directional constraints, which incurs well-placed singularities for the most part. However, the mixed-integer seamless integration can potentially cause \emph{misalignment} between two close-by singularities that should have been naturally paired. For example, in \figref{fig:topological-defect}b, the singularities are slightly misaligned along the route that goes around the right handle, while in \figref{fig:topological-defect}c they are aligned along all highlighted routes. Even such slight misalignment can cause long undesirable winding strips with self-folding geometry or very thin patches. We offer a paradigm for topological editing that can easily fix most defects with simple and minimal user control. 

% Figure environment removed

We note that there are paradigms for canonically partitioning quad meshes into singularity-free sub-meshes without altering the underlying topology of the mesh, such as motorcycle graphs
\cite{Eppstein-2008-Motorcycle_Graphs, Schertler_2018_Generalized_Motorcycle_Graphs_for_Imperfect_Quad-Dominant_Meshes}. However, such approaches would not work for our case since we are, in essence, partitioning strip networks instead of quad meshes. This means that we consider cuts along edges of the same color for the most part, thus few or no intersections are created where cuts can terminate. Therefore, we opt for a new topological editing workflow described below.


\paragraph{Atomic operations}
Editing a strip-decomposable mesh is a more constrained problem than editing typical quad meshes~\cite{tarini2010_practical-quad-mesh-simplification,daniels-ii2009_localized-quadrilateral-coarsening,peng2011_connectivity-editing-for-quadrilateral-meshes,takayama2013_sketch-based-generation-and-editing-of-quad-meshes}, as we have a strict requirement that $U$ and $V$ strips don't intermix, so that the strip decomposition remains unambiguous.  To this end, we devised two novel strip-based atomic operations, whose purpose is to rearrange the connectivity of adjacent strips and align vertices that lie on different strips: 

\begin{enumerate}
    \item \emph{Collapsing and splitting.} Given a strip $S_i$, described as a sequence of quadrilaterals, we can either subdivide it by splitting it into two adjacent topologically identical strips (\figref{fig:editing_operations}b) or collapse it so that its immediate neighbors are now adjacent (\figref{fig:editing_operations}c). 
    \item \emph{Rewiring.} Given a strip $S_j$, we rewire its quads by switching the edges of all quads, either one vertex forward or backward on the strip (\figref{fig:editing_operations}d). As a result, all transversal strips that intersect $S_j$ are reconfigured.
\end{enumerate}


% Figure environment removed 


\paragraph{Alignment of a pair of vertices}
Assume we want to align the vertices $A$ and $B$ that lie on the two sides of a strip $S_i$ (\figref{fig:editing_operations}a) along the $U$ (blue) direction. This can be achieved in two ways: either by collapsing strip $S_i$  (\figref{fig:editing_operations}c) or by rewiring a \emph{transversal} (green) strip along the $V$ direction that runs between $A$ and $B$, such as strip $S_j$ (\figref{fig:editing_operations}d). For vertices further apart, there might be an entire decision graph of possible choices. To make a reasonable choice for a sequence of operations for alignment, it is important to consider:


\begin{itemize}
    \item Collapsing removes mesh faces. If it is applied on many strips, or on a strip that covers a large part of the mesh, for example, a winding strip, it can lead to considerable geometric distortion. 
    \item Rewiring can misalign other pairs of vertices that were already aligned, because it changes the connectivity of all strips intersected by the transversal rewired strip.
\end{itemize}

The alignment of vertices can be achieved along either the $U$ or the $V$ direction using the presented atomic operations. For brevity, in the following, we consider alignment along the $U$ (blue) direction, which we call \emph{dominant}, and we call the $V$ (green) direction \emph{subdominant}.


\subsubsection*{User-controlled topology editing}
Determining what edits must be carried out is a difficult combinatorial problem that is hard to automate fully. In addition, editing decisions have a large impact on the final aesthetics of the produced object. Therefore, it makes sense to delegate some of the combinatorial choices to the user. On the other hand, reasoning about the desired topology of a mesh geometry is a challenge, as the connectivity of strips can be too complex for a novice user to grasp or change manually. As a practical compromise, we abstract full control over mesh connectivity away from the user, and instead have them follow a short pipeline where they choose pairs of vertices they wish to align. Our algorithm then automatically generates the sequence of atomic operations that realizes the chosen alignment. 
The accompanying video shows the full process of aligning singularities on various shapes. 


% Figure environment removed


\paragraph{Control pipeline}

Every alignment of a pair of vertices entails the following steps.

\begin{enumerate}[label=(op\arabic*)] 

\item 
\label{editop:choose_vertices}
The user selects two vertices $v_1$, $v_2$ to be aligned along the dominant direction (\figref{fig:control_pipeline}a). {Note that we allow $v_1=v_2$, for example, the vertex $v$ in \figref{fig:strips_editing}a,b can be aligned with itself around the handle.}
\item 
\label{editop:choose_route}
The algorithm enumerates all routes from $v_1$ to $v_ 2$, and the user selects one route along which to align (\figref{fig:control_pipeline}b). The default choice is the route that has the smallest non-zero length.
\item 
\label{editop:choose_ops}
The algorithm points out the ``best'' next step, and the user confirms its application (\figref{fig:control_pipeline}c). This is repeated until the two vertices are aligned along the selected route.
\end{enumerate}


\begin{wrapfigure}{r}{0.2\textwidth}
    \vspace{-15pt}
    % \hspace{-15pt}
    % Figure removed
    \vspace{-40pt}
\end{wrapfigure}
\paragraph{Choosing vertices to be aligned \ref{editop:choose_vertices}}
The singularity separatrices and winding strips are highlighted as guides (see \figref{fig:control_pipeline}a and inset) to help the user choose which vertices it makes sense to align.


\paragraph{Choosing routes between vertices \ref{editop:choose_route}}
Having chosen two vertices, $v_1$ and $v_2$, we automatically enumerate all relevant routes  between them, the set of which we denote by $R$.  Each route $R_i \in R$ consists of a sequence of $U_{R_i}$ (dominant) and a sequence of $V_{R_i}$ (subdominant) strips, which we collect as follows:
\begin{enumerate}
    \item \emph{Subdominant sequences $\{V_{R_i}\}$:}  We start from the vertex  ($v_1$ or $v_2$) that has the higher valence, 
    and trace every incident dominant edge until we intersect a subdominant edge directly connected to the other vertex (\figref{fig:routes} top). This defines the route $R_i$, and the crossed subdominant strips comprise its subdominant sequence $V_{R_i}$. If another singularity or a boundary is reached instead (\figref{fig:routes}c), the route is discarded as irrelevant. Therefore, the number of possible routes is bounded by the dominant-edge valence of the first vertex. In \figref{fig:routes}, we have two relevant routes $R_a$ and $R_b$.
 
    \item \emph{Dominant sequences $\{U_{R_i}\}$:} For each relevant route $R_i$ found in the previous step, we find the shortest among the paths of dominant (blue) strips between $v_1, v_2$ that intersects the sequence of subdominant strips $V_{R_i}$ (\figref{fig:routes}d,e). This is then the dominant strip sequence $U_{R_i}$. We define the \emph{route distance} $d_{R_i}$ as the length of $U_{R_i}$; it specifies the number of operations that must be performed for the two vertices to  align along $R_i$. In \figref{fig:routes} we have $d_{R_a} = 0$, i.e. the vertices are already aligned, and $d_{R_b} = 1$, i.e. the vertices are one strip apart, and therefore require one operation to be aligned.
\end{enumerate}

In \figref{fig:routes}b,e the depicted route $R_b$ is the smallest non-zero length route chosen by default.  

% Figure environment removed


\paragraph{Iterative next best step \ref{editop:choose_ops}}
Having chosen a route $R_i$ along which $v_1$ and $v_2$ should be aligned, the algorithm iteratively brings them towards alignment, one strip at a time, by  either (a) rewiring a subdominant strip $V_{R_i} \in V_R$, or (b) collapsing a dominant strip $U_{R_i} \in U_R$. To choose the strip to be modified, we consider the following  ``fitness'' criteria applied to all strips $S \in \{U_{R_i}, V_{R_i}\}$:
\begin{itemize}
    \item The total area $A$ of the strip 
    $ A(S) = \sum_{q \in S} A(q)$
    measured as the sum of the areas of its quadrilaterals $\{q\}$.
    \item The total deviation from perfect ``squareness'' $D= \sum_{q \in S} D(q)$. We measure the squareness of a quad $q$ as
    $$D(q) = \sum_{i=1}^4{\left|a_i - \frac{\pi}{2}\right|} + \sum_{i=1}^4{\left|l_i - \bar l\right|}$$

    where $a_i$ is the $i$-th inner angle of the quad, $l_i$ is the $i$-th edge length and $\bar l$ is the average edge length of the quad.
    
\end{itemize}
The fitness score for $S$ is then $1/(A(S) + D(S))$, and we choose the strip of highest fitness. As a result, we end up editing small and well-shaped strips, which helps avoid geometric artifacts.

In the example of  \figref{fig:routes}(b,e), $v_1$ and $v_2$ are one dominant strip apart along route $R_b$, thus one operation is required. All strips of $R_b$ are nicely shaped with almost square quads. Therefore, their area is the deciding factor for picking a strip.
Since the blue strip is winding (\figref{fig:routes}e), it has a large area sum, therefore, its fitness is very low. As a result, the green strip with the smallest area is selected for rewiring, resulting in the alignment of $v_1$ and $v_2$ along that route (\figref{fig:routes}f). 

While the algorithm iteratively offers the next atomic operation in this step, we allow the user to approve every such iteration before it is applied. This is useful because the proposed next step might still not be a preferable operation to perform (e.g. if all strips of a route are winding). Then the user can abort and attempt to align another pair of vertices, which can, for example, unwind these strips before returning to the current route. 

In both examples \figref{fig:routes}f and \figref{fig:strips_editing}b, the alignment of singularities removes the winding strips. In \figref{fig:unwinding_strips}, we show how aligning a non-singular pair of vertices can remove winding strips. Aligning any two vertices that lie on the same side of a winding strip in two consecutive turns unwinds the strip. 


% Figure environment removed


\paragraph{Locking alignments}
Rewiring operations may misalign already aligned pairs of vertices. To avoid this, the user may select vertex pairs whose alignment should remain locked. In that case, we lock the rewiring of any subdominant strips that intersect a dominant alignment path between the two locked vertices, by setting their fitness to $0$. Collapse operations are never blocked, as they can only bring vertices together but not apart.  For example, on \figref{fig:strips_editing}c, the alignment between vertices $A$ and $B$ has been locked, so none of the subdominant strips (in dark green) that intersect the path between $A$ and $B$ can be rewired. As a result, to align $v_1$ and $v_2$ along either one of the two available routes ($R_x$ or $R_y$), only the collapse of the highlighted dominant (blue) strip  $S$ is possible.  

% Figure environment removed


\subsubsection*{Post-processing}
Once all operations are completed, the algorithm selectively subdivides all elongated strips, smooths local deformations, and projects the resulting mesh to the original mesh boundary and surface, as follows.


\begin{wrapfigure}{r}{0.25\textwidth}
    \vspace{-15pt}
    %   \hspace{-15pt}
    % Figure removed
    \vspace{-30pt}
\end{wrapfigure}
 \paragraph{Selective Subdivision} We define the \emph{elongation} of a strip by the ratio of lengths of edges,
$r = max(l_U,l_V) / min(l_U,l_V)$, where $l_U$ and $l_V$ are the total lengths of the strip's $U$ and $V$ edges respectively. If $r > 2$, we subdivide the strip $\text{round}(r)$ times to reduce the elongated direction (see inset). This is done iteratively until no strip is elongated.

\paragraph{Smoothing}
We further reduce the geometric distortion by applying iterative implicit Laplacian smoothing on the vertices $v \in \VV_\QQQ$ by
$(I + dt L_\QQQ) v^{t+1} = v^t$
(\figref{fig:smoothing-boundaries_editing}). Each smoothing iteration consists of the following two steps:
\begin{enumerate}
    \item Smoothen the boundary of the quad mesh $\partial \MM_\QQQ$ as an independent curve and project it onto the original boundary $\partial \MM$.
    \item Smoothen $\MM_\QQQ$ keeping boundary vertices fixed and project it onto the original surface $\MM$.
\end{enumerate}
For the boundary smoothing we use $dt = 0.5$, and for the mesh smoothing, we use $dt = 0.001/\lambda$, where $\lambda$ is the smallest non-zero eigenvalue of the uniform quad-mesh Laplacian $L_\QQQ$.


% Figure environment removed





\subsection{Topological Partitioning}
\label{subsection:patitioning}

The topological partitioning of the resulting SDQ mesh into singularity-free strip sub-networks is useful for all sequential fabrication processes, as it allows to determine an ordering of the resulting strips up to orientation. For example this is useful for deciding the assembly sequence of a surface made of flat sheets, or a print sequence of a series of print paths for 3D printing. 

Once a good patch layout has been achieved with desired alignments of singularities, separating the strip network into good patches is trivial. As we have two transversal strip networks, the user must select along which network the cuts should be aligned. In the following we assume that partitioning happens along the $U$ direction, meaning that cuts run along the $U$ edges (except close to singularities with valence 2) and cut transversally the $V$ strips.   

To create such a partitioning, we make cuts along the $U$ separatrices of singularities (\figref{fig:singularities}). 
However, for singularities with valence 2 (\figref{fig:singularities} left), only cutting the separatrix in the $U$ direction does not separate the surface. In this case, we cut ``through'' the singularity along both the $U$ and $V$ directions.

% Figure environment removed

Having cut away all singularities, we also cut along short topological handles to obtain simply-connected networks where no singularity or separatrice lies in the interior of the patch. 
\figref{fig:full_editing_operations_pipeline_2} shows results of editing and topological partitioning on more complex shapes.

% Figure environment removed