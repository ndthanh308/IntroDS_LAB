\section{Overview}
\label{sec:Overview}

We aim to generate a fabrication-aware and editable SDQ mesh with edges aligned to user-defined directions. In this section, we describe in detail our objectives and pipeline. 

\subsection{Objectives}
We consider the following geometric objectives for the dual strip networks, commonly found in the fabrication scenarios mentioned in \secref{sec:intro}.

\begin{enumerate}

\item \emph{Alignment}. \label{obj:aligned}
The strip networks are aligned as well as possible with user-specified directions. 

\item \emph{Uniformity}. \label{obj:uniform} 
The strips have a width that is as uniform as possible, resulting in quads with as uniform as possible sizes.

\item \emph{Smoothness}. \label{obj:smooth}
The strips are as smooth as possible, meaning that they vary slowly in space without sharp turns or noisy boundaries.

\item \emph{Orthogonality}. \label{obj:orthogonal}
The two transversal strip networks are as orthogonal as possible to each other. This translates to an SDQ mesh where all regular vertices have four incident edges with angles close to $\frac{\pi}{2}$. 

\item \emph{Continuity}. \label{obj:continuous}
There are no interruptions or discontinuities of strips on the interior of the shape.  

\item \emph{Good strip topology}. \label{obj:strips_topology}
The strips have a good topology, meaning that they do not wind before they close or terminate on a boundary. Instead, they extend from boundary to boundary, or form closed loops without snake-like parts where they fold into themselves.

\item \emph{Good patch layout}. \label{obj:patch_layout}
Each strip network can be partitioned into a small number of patches of simply connected strips, i.e., it has a good patch layout.
\end{enumerate}


Objectives (\ref{obj:aligned})--(\ref{obj:strips_topology}) are optimized in the vector-field design stage (\secref{sec:Stripe_parametrizations}). In addition, objective (\ref{obj:continuous}) is further addressed in the subsequent seamless integration (\secref{subsec:integration-and-meshing}). Objectives (\ref{obj:strips_topology})--(\ref{obj:patch_layout}) are achieved by the topological operations that enable editing the mesh (\secref{sec:mesh_editing}).


\subsection{Pipeline}
\label{sec:pipeline}

Given a triangle mesh surface, possibly with open boundaries, our pipeline proceeds as follows.

\begin{enumerate}
\item	Collect user-specified directional constraints (\secref{par:alignment}).
\item	Compute two coupled integrable line fields that adhere to the objectives (\secref{subsec:guiding-fields}).
\item	Integrate the fields into two transversal parameterizations. Discretize those into two sets of strips, which, when overlaid, form an SDQ mesh (\secref{subsec:integration-and-meshing}).
\item	Carry out strip-based topological operations for eliminating winding strips and aligning singularities, with the help of user input, to create a good patch layout (\secref{sec:mesh_editing}).
\end{enumerate}
