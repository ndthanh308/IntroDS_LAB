\section{Related Work}
\label{sec:Related Work}

\subsection{Fabrication-aware strips}
Strip meshes have been studied extensively for their various applications in design and fabrication. 
\cite{Pottmann_2008_Freeform_Surfaces_from_Single_Curved_Panels} studied the refinement and convergence properties of strips and presented a method that partitions freeform surfaces into single curved strips. Spiraling, developable strips were used in \emph{Zippables} \cite{schuller2018_shape-representation-by-zippables} to zip a surface together from flat ribbons, while \cite{verhoeven2022_dev2pq:-planar-quadrilateral-strip-remeshing} extracted a quad dominant strip pattern of developable surfaces. \cite{Differentiable_Stripe_Patterns_Maestre_2023} use strips to represent bi-material distributions with high stiffness contrast and compute patterns that lead to an optimal approximation of high-level performance goals. 

The transversal strip network is introduced in \cite{Takezawa_2016_Fabrication_of_Freeform_Objects_by_Principal_Strips} for the fabrication of freeform objects from flat paper pieces using principal strips aligned with the principal curvature directions. 
However, this approach relies on tracing streamlines, which  are sensitive to local perturbations and can thus  only be applied to smooth surfaces with disc or annulus topology. In contrast, we apply our approach to more complex surfaces with a higher genus and more boundaries.

\cite{Oval_BRG_2021_Two-Colour_Topology_Finding_of_Quad-Mesh_Patterns} proposed an alternative way of generating a two-colored quad mesh topology, starting with fixed singularities and then combinatorially connecting them into patches that are subdivided to generate the final mesh.
This approach relies on pre-defined fixed singularities and thus does not enable control over the alignment of the resulting edges. In contrast, our method positions singularities dynamically during the vector-field optimization stage.

\cite{Schiftner_2010_Statics-Sensitive_Layout_of_Planar_Quadrilateral_Meshes} generate quad-dominant meshes by integrating two separable conjugate line field directions for structural purposes.
Similarly, in \cite{Zadravec_2010_Designing_Quad-dominant_Meshes_with_Planar_Faces}  quad-dominant meshes are created by integrating a pair of line fields aligned to principal curvature directions. In both cases, since the guiding line fields are separable and conjugate, the resulting quad meshes are strip-decomposable and quasi-planar. Their planarity is further improved with a perturbation optimization \cite{Liu_2006_Conical_meshes_and_developable_surfaces}. 
From our perspective these methods are the closest to our approach since we also rely on the integration of separable line fields. However, our work allows for more freedom and control of the directional constraints and location of singularities, and we consider a broader scope of application scenarios.

\subsection{Field-aligned parameterization and remeshing}
Our work uses a pair of line fields integrated into scalar functions from which the strips are extracted as their isolines, and are overlayed to create a field-guided SDQ mesh. We refer the reader to the surveys~\cite{goes2016_vector-field-processing-on-triangle-meshes,vaxman2016_directional-field-synthesis-design-and-processing} to get acquainted with field-aligned meshing methods. While our work also designs a quadrilateral mesh from directional-field-oriented parameterization (like eg.,~\cite{bommes2009_mixed-integer-quadrangulation,kalberer2007_quadcover---surface-parameterization-using}), it departs from such methods in that we create the quad mesh as two separable sets of \emph{strips} with embedded fabrication considerations. By that, our work is, in fact, more related to the concept of ``stripe patterns'' of ~\cite{knoppel2015_stripe-patterns-on-surfaces}, although those were not used for meshing or optimized for fabrication processes. 

Our design optimizes field integrability, which is essential for the fidelity of the parameterization and meshing. To this end, we aim to reduce curl, as considered in~\cite{diamanti2015_integrable-polyvector-fields,vekhter2019_weaving-geodesic-foliations}. However, our optimization follows an iterative smoothing and curl reduction process, adapted from methods such as ~\cite{sageman-furnas2019_chebyshev-nets-from-commuting,meekes2021_unconventional-patterns-on-surfaces,liu2020_practical-fabrication-of-discrete-chebyshev-nets}, and tailored to our specific requirements.



\subsection{Quadrilateral-mesh topology editing}
In the process of creating feasible, not entwined, and aesthetically pleasing strips, we introduce an interface for basic topology editing operations. These operations aim mainly to control the location and connectivity of the singularities of the mesh, which affect how the shape can be partitioned into a good patch layout. As such, these operations are similar in spirit to topology editing in quadrilateral meshes~\cite{peng2011_connectivity-editing-for-quadrilateral-meshes,takayama2013_sketch-based-generation-and-editing-of-quad-meshes,tarini2010_practical-quad-mesh-simplification,daniels-ii2009_localized-quadrilateral-coarsening}. Our operations, however, work on strips and not quadrilaterals, and as such have a more global effect, and guarantee that the strip-decomposability of the mesh is maintained. In~\cite{Noma2022FastEditing}, strip singularities are edited directly as scalar functions, but not as strip meshes. To the best of our knowledge, we are the first to perform topology editing on SDQ meshes.