\section{Limitations and Future Work}
\label{sec:Conclusion}

Our pipeline successfully produces quality SDQ meshes for various shapes in a semi-automatic manner that can be controlled and edited by the user. We next note some limitations and future directions.


\paragraph{Editing operations}
The topology editing operations are essential in promoting a good patch layout of the produced meshes. However, there is no guarantee that a ``good'' set of topological operations always exists for unwinding all strips and producing the desired alignments. For example, applying operations on winding strips can cause major distortion, and a mesh might even consist \emph{only} of winding strips. Another example of an undesirable solution is when all rewiring operations are blocked to maintain specific alignments, and only collapse operations on long winding strips are available. The worst-case scenario is when the only available operation is the collapse of a winding strip that covers the entire shape, which reduces the mesh to a single point, thus canceling any potential benefits of the operation. In particular, surfaces without open boundaries where strips can terminate are more likely to exhibit many winding strips.

A potential solution could be to enforce alignment already on the stage of seamless parameterization. However, this comes with the challenge of automatically deciding which singularities should be aligned, which is a difficult combinatorial problem that might result in large integration errors.

In addition, post-editing smoothing operations increase the alignment error and can create distortions close to the boundaries of shapes, especially at non-quad faces. A large number of editing operations that require a lot of smoothing afterward can lead to considerable irregularities on the boundaries. 

\paragraph{Mesh quality depends on user input}
Unanticipated inputs (e.g., shapes without open boundaries or directional constraints that produce highly noisy fields) may lead to infeasible, visually unappealing, or impractical results. Typically, this would mean many singularities, a bad patch layout, or very noisy strip networks. In addition, if the scale of the parametrization, which the user chooses, is too low, then the quads resolution might not be sufficient to capture the input surface's features. 

\paragraph{Fabrication-related properties} We are optimizing our strip networks for a selection of desirable fabrication-related properties that we have found most commonly at the state of the art. However, various other properties can be embedded in the optimization, according to the fabrication scenario that the SDQ mesh is produced for. For example, alignment of strips so that their boundaries are approximately geodesic curves is useful for weaving surfaces out of ribbons that bend more readily out-of-plane than in-plane \cite{vekhter2019_weaving-geodesic-foliations}.

\paragraph{Alignment presets} We propose three alignment presets, namely alignment to curvature, boundaries, or user-drawn directions. Another useful option would be alignment to principal stress directions or principal moments given a chosen load case, which can be instrumental for improving the structural properties of an object in various fabrication scenarios \cite{Schiftner_2010_Statics-Sensitive_Layout_of_Planar_Quadrilateral_Meshes, fang2020_reinforced-fdm:-multi-axis-filament-alignment}. 