\section{Results}
\label{sec:Results}
We present the results of our methodology for generating SDQ meshes on various inputs. As mentioned, we focus on relatively smooth geometries with open boundaries and without significant surface detail features. In addition, we apply SDQ meshes for the fabrication scenario of robotic non-planar 3D printing of shell surfaces. 

We use a set of metrics to quantify to what extent the requirements are met on the resulting meshes. Given a sufficiently high-resolution parametrization, these metrics are approximately invariant to the global scaling of the discretization, and are as follows:

\begin{itemize}
    \item \emph{Edge length uniformity error} ($\mathcal{L}$) is the standard deviation of the lengths of all non-boundary edges  divided by the average edge length. It evaluates the deviation from uniform length on the strip networks (objective \ref{obj:uniform}).
    
    \item \emph{Alignment error} ($\mathcal{A}$) with the user-specified directions  measures the difference between the input directional constraints and the orientation of the non-boundary edges of $\MM_\QQQ$ (objective \ref{obj:aligned}). To compare the orientation of an edge $e$ of $\MM_\QQQ$ with directional constraints on the faces of $\MM$, we take the projections of the start ($P_1$) and end ($P_2$) points of $e$ on the closest triangle faces $f_1$ and $f_2$ of $\MM$ that have directional constraints $a_1$, $a_2$ and confidence weights $\omega_1$, $\omega_2$. Note that unconstrained faces have $a, \omega = 0$, and we discard edges with both $f_1$ and $f_2$ unconstrained. After matching the signs of $a_1$ and $a_2$ we find the average constraint direction $a_e = (d_2 a_1 + d_1 a_2) / (d_1 + d_2)$, and the average confidence weight $\omega_e = (d_2 \omega_1 + d_1 \omega_2) / (d_1 + d_2)$, where $d_1$, $d_2$ are the distances of the projected points from the barycenters of $f_1$ and $f_2$. Then we set $\mathcal{A} = \frac{\sum \omega_e * \lvert \widehat{a_e} \cdot \widehat{(P_1 - P_2)} \lvert}{N_e}$, where $N_e$ is the number of edges projected on at least one constrained face. Note that since the directions $a$ constrain the gradient of the parametrization, perfect alignment means that the dominant edges of $\MM_\QQQ$ are orthogonal to the original constraints. 

    \item \emph{Orthogonality error} ($\mathcal{O}$) of the intersections of the strips around each vertex of $\MM_\QQQ$ measures how orthogonal the two strip networks are (objective \ref{obj:orthogonal}). $\mathcal{O} = \frac{\sum{\lvert \frac{\pi}{2} - \theta \lvert}}{N_c}$ for every non-boundary corner of $\MM_\QQQ$ with angle $\theta$, where $N_c$ is the total number of non-boundary corners.
    
\end{itemize}


In \figref{fig:energy_params}, we ablate the prioritizing of each energy term presented in \secref{subsec:energies}. 

The bar graphs show how the metrics $\mathcal{L}$,  $\mathcal{A}$ and $\mathcal{O}$ vary with the different optimization parameters. 

%%% --- Parameters testing figure --- %%%%
% Figure environment removed


We present a gallery of the resulting SDQ meshes and their patch layout along the $U$ direction for a series of input meshes using curvature-aligned directional constraints (\figref{fig:gallery}). The bar graphs indicate the evaluation of the metrics on the gallery shapes, and table \ref{table:performance} shows the details of the models and the algorithm's performance. The editing operations have the following effect on the evaluation measures; $\mathcal{L}$ and $\mathcal{O}$ remain constant or are reduced, while $\mathcal{A}$ increases. This means that the edge length uniformity and the orthogonality are improved, an effect caused by the iterative smoothing operations, while the alignment error increases, since the editing operations and smoothing slightly alter the orientations of the mesh edges.  

\paragraph{Alignment to principal curvature directions}
The alignment of the guiding fields to principal curvature directions is probably the most common scenario from the alignment options we present. However, in exceptional cases, it can produce undesirable results. This happens for meshes with large flat surfaces and undulating details, where the principal curvatures flip signs.
\begin{wrapfigure}{r}{0.5\textwidth}
    % \vspace{-15pt}
    % Figure removed
    \vspace{-25pt}
\end{wrapfigure}
The cloth surface (see inset) is one such example; our alignment with signed principal curvatures produces the result on the left. To correct this, we need to align instead with absolute curvature directions, where it is not the sign but the magnitude that defines the minimum and the maximum direction (inset, right).


% Figure environment removed
 
% Table
\begin{table}[h!]
\centering
\vspace{20pt}
\begin{tiny}
\setlength{\tabcolsep}{3pt} % Reduce column spacing
\begin{tabular}{c | c c c c c c c c c}
  \toprule
{model}  &  
\vtop{\hbox{\strut input tri mesh}\hbox{\strut  \#V , \#F}}  & 
\vtop{\hbox{\strut field}\hbox{\strut  iter.}}  & 
\vtop{\hbox{\strut field}\hbox{\strut  time } \hbox{\strut (s)}} & 
\#sing. &  
\vtop{\hbox{\strut quad mesh}\hbox{\strut  \#V , \#F}} & 
\vtop{\hbox{\strut \#patches}\hbox{\strut  (initial)}} & 
\#ops &
\vtop{\hbox{\strut \#patches}\hbox{\strut  (final)}} & 
\vtop{\hbox{\strut tracing}\hbox{\strut  8 paths/quad}\hbox{\strut  time (ms)}} \\
\midrule
%       V     F      iters          vf_time  sings     V      F      #N   #ops #N   #path tracing time
(a) & 1653 , 3136     & 20       &  3.34    & 0   & 1055 , 980      & 1  & 0  & -  & 28 \\ 
(b) & 4716 , 9186     & 18       &  8.42    & 1   & 2271 , 2147     & 3  & 0  & -  & 262 \\ 
(c) & 2887 , 5592     & 26       &  7.23    & 1   & 1659 , 1579     & 3  & 0  & -  & 134 \\ 
(d) & 2105 , 4092     & 25       &  4.99    & 3   & 1293 , 1228     & 4  & 0  & -  & 95 \\ 
(e) & 3260 , 6144     & 79       &  18.48   & 8   & 2156 , 2016     & 6  & 0  & -  & 22 \\ 
(f) & 2623 , 5014     & 30       &  7.03    & 4   & 3949 , 3804     & 4  & 0  & -  & 402 \\ 
(g) & 2495 , 4725     & 27       &  5.97    & 4   & 2752 , 2534     & 8  & 0  & -  & 123 \\ 
(h) & 3954 , 7606     & 15       &  5.34    & 1   & 886  , 793      & 3  & 0  & -  & 35 \\ 
(i) & 4144 , 8221     & 29       &  13.81   & 14  & 2568 , 2510     & 9  & 0  & -  & 130 \\ 
(j) & 1935 , 3791     & 22       &  3.85    & 2   & 1525 , 1479     & 3  & 0  & -  & 14 \\ 

(k) & 2072 , 3963     & 21       &  3.81    & 1   & 1214 , 1129     & 7  & 4  & 3  & 26 \\ 
(l) & 1753 , 3330     & 26       &  4.14    & 3   & 1783 , 1660     & 7  & 3  & 5  & 89 \\ 
(m) & 4759 , 9150     & 28       &  20.84   & 9   & 3470 , 3282     & 11 & 1  & 9  & 425 \\ 
(n) & 1653 , 3136     & 26       &  8.43    & 17  & 4931 , 4655     & 62 & 12 & 16 & 21 \\ 
(o) & 3714 , 7240     & 26       &  9.16    & 4   & 2688 , 2624     & 45 & 1  & 5  & 142 \\ 

% \bottomrule
\end{tabular}
\end{tiny}
\bigskip
\caption{ Details for all the presented results in \figref{fig:gallery}. The examples were run on a Windows 11 laptop with an AMD Ryzen 9 6900HS CPU, an NVIDIA GeForce RTX 3080 GPU and 32GB of RAM. The performance of the integration and meshing processes are not included as they are done with an external library \cite{vaxman2016_directional-field-synthesis-design-and-processing}.}
\label{table:performance}
\end{table}


In addition, we present two examples of applying our meshing
workflow to more typical test geometries within graphics using
alignment to principal curvature directions. These geometries have no open boundaries, therefore, they are not within the scope of geometries we investigate. In such geometries, the field optimization and meshing step lead to good 
results (see inset). However, it is usually hard or impossible to reach a good patch layout using editing operations. Due to the large number of singularities and lack of boundaries where strips can terminate, most strips are winding around the mesh, making the application of editing operations difficult as each edit alters a large part of the mesh with results that are hard to control. The presented examples display the strips layout directly after integration without any editing operations applied.  


% Figure environment removed


%%%%%%%%%%%%%%%%%
%%% 3D printing 
%%%%%%%%%%%%%%%%%

\subsection{Application of SDQ meshes for non-planar 3D printing}

We present a case study for applying the presented SDQ meshes to design non-planar print paths for robotic 3D printing of shell surfaces. 

\subsubsection{Characteristics of printing method}
\begin{wrapfigure}{r}{0.4\textwidth}
    \vspace{-35pt}
    % Figure removed
    \vspace{-35pt}
\end{wrapfigure}
We focus on printing \emph{shells} (inset, left), where only the surface of the object is printed, as opposed to printing volumes (inset, right), where multiple offsets and infill structures are created in the interior. While a closed, watertight input shape is required for printing a volume, shell printing is more commonly applied to shapes with open boundaries.

In particular, we investigate the 3D printing of \emph{standing} shells (inset, left), where only the first path of the print lies on external support, while all 
\begin{wrapfigure}{r}{0.4\textwidth}
    \vspace{-15pt}
    % Figure removed
    \vspace{-25pt}
\end{wrapfigure}
subsequent paths are supported by their preceding paths. This contrasts with the common ``horizontal'' printing of shells (inset, right), where the entire surface lies on existing support \cite{tam2017_additive-manufacturing-along-principal-stress, Bi2021}. This technique can save on material, time, and energy required for printing, as it requires considerably less external support. However, the printing of standing shells \cite{mitropoulou2020_print-paths-key-framing:-design-for-non-planar, zhong2020_ceramic-3d-printed-sweeping-surfaces} introduces an additional requirement for producing a feasible sequencing of the paths so that each path being printed is both accessible by the robotic arm, and fully supported by previously completed paths. 

\begin{wrapfigure}{r}{0.4\textwidth}
    \vspace{-15pt}
    % Figure removed
    \vspace{-25pt}
\end{wrapfigure}
In general, strip networks with branched connectivity (inset-left) do not have a feasible print sequence for the entire shape, meaning that they cannot be printed in one go as a standing shell.  
The partitioning process described in \secref{subsection:patitioning} decomposes the strip network into simply-connected patches (inset-right), where each has a feasible sequence. These are then printed separately and assembled afterwards. 


In addition to the topological cuts that separate the shape into simply-connected patches, in certain cases geometric cuts must also be introduced to split partitions into smaller pieces; This might be necessary for the following reasons, either the patches are too large to fit within the space of the fabrication setup, or they have an angle variation that exceeds what can be fabricated with the extruder. We document the number of geometric cuts for the presented prototypes in table \ref{table:prints_details}.

\subsubsection{Use of SDQ meshes}

\begin{wrapfigure}{r}{0.5\textwidth}
    \vspace{-5pt}
    % Figure removed
    \vspace{-30pt}
\end{wrapfigure}

An SDQ mesh consists of the overlay of two transversal networks of strips. We chose to align the paths along the $U$ (blue) direction and use the $V$ (green) direction to design rigidifying ribs on the printed surface.  

\begin{wrapfigure}{r}{0.5\textwidth}
    \vspace{-5pt}
    % Figure removed
    \vspace{-30pt}
\end{wrapfigure}

To trace the print paths along the dominant direction we subdivide each strip's subdominant edges $N$ times and connecting the subdivision points (see inset). On the boundaries of the mesh (shown in orange), we consider natural boundary conditions.

\subsubsection{Prototypes}

We present a series of fabricated prototypes that showcase the physical results that can be achieved with the presented methods. 
The TPMS (\figref{fig:prints}a) and half-sphere (\figref{fig:prints}b) models were printed with a UR10 robot, using a plastic extruder tool that prints filament of polyethylene terephthalate glycol (PETG) with diameter 2.75m, using a 2.5 mm aperture nozzle (\figref{fig:print_process}a). The Enepper (\figref{fig:prints}c) and Batwing (\figref{fig:prints}d) surfaces were printed with an IRB1600 ABB robot, using an MDPH2 pellet extruder \cite{Massive_Dimension_pellet_extruder} with a 3mm aperture nozzle, that prints recycled PIPG pellets reinforced with glass fiber (\figref{fig:print_process}b,c). Table \ref{table:prints_details} provides more information on the logistics of the printing process for each prototype.

The first path of each patch is printed on sacrificial support, and all subsequent paths are printed support-free. The surfaces are printed with an offset of size negligible to the surface's scale to increase the final print's stability (\figref{fig:print_process}a). 
The change of printing color is carried out manually without stopping the printing process, and its purpose is to differentiate the paths visually.


% Figure environment removed


% Figure environment removed


\begin{table}[h!]
\centering
\begin{tiny}
\setlength{\tabcolsep}{3pt} % Reduce column spacing
\begin{tabular}{c | c c c c c c}
\toprule
{model}  &  \vtop{\hbox{\strut dimensions}\hbox{\strut  (cm)}} & \#sings  &  \vtop{\hbox{\strut \#geometric}\hbox{\strut cuts}} &\#pieces  &  \vtop{\hbox{\strut total print}\hbox{\strut time (hrs)}}  &  \vtop{\hbox{\strut \% sacrificial}\hbox{\strut support}}  \\
\midrule
\figref{fig:prints}a   & 38 x 38 x 38  & 2  & 3  & 6   & 18.3   & 16\% \\
\figref{fig:prints}b   & 30 x 30 x 15  & 2  & 0  & 4   & 10.3   & 11\% \\
\figref{fig:prints}c   & 70 x 70 x 45  & 1  & 0  & 3   & 12.5   & 29\% \\
\figref{fig:prints}d   & 76 x 76 x 76  & 3  & 6  & 9   & 27.0   & 39\% \\
% \bottomrule
\end{tabular}
\end{tiny}
\caption{Details of printed prototypes presented in \figref{fig:prints}.}
\label{table:prints_details}
\end{table}


\begin{wrapfigure}{r}{0.35\textwidth}
    \vspace{-15pt}
    %   \hspace{-15pt}
    % Figure removed
    \vspace{-20pt}
\end{wrapfigure}
We create L-shaped connections between adjacent pieces to facilitate the assembly while allowing for tolerances. Then the pieces of smaller prototypes are connected with hot glue (inset, left), while larger prototypes are connected with screws (inset, right). 

%%%%%%%%%%%%%%%%%%%%%%%%%%%%%%%%%%%
%%% Comparison to state of the art
%%%%%%%%%%%%%%%%%%%%%%%%%%%%%%%%%%%

\subsection{Relation to existing methods}

In \emph{Stripe Patterns on Surfaces}, \cite{knoppel2015_stripe-patterns-on-surfaces} propose a method for synthesizing stripe patterns on triangulated surfaces, where stripe discontinuities are automatically inserted to achieve user-specified orientation and line spacing.
While the discontinuities improve the alignment and spacing between paths, they create interruptions that are problematic for various fabrication scenarios (\figref{fig:comparison_to_stripes} left, middle). In our method (\figref{fig:comparison_to_stripes} right), we sacrifice perfect alignment and uniform spacing to ensure a continuous parametrization so that no strip is interrupted in the interior of a patch. In addition, our discretization in an SDQ mesh enables the editing of the resulting strip networks. 

% Figure environment removed


In \emph{Fabrication of Freeform Objects by Principal Strips},
\cite{Takezawa_2016_Fabrication_of_Freeform_Objects_by_Principal_Strips} propose a method for designing orthogonal strips along the principal curvature directions of freeform surfaces, to be fabricated from flat paper sheets. In contrast to our approach, they trace the streamlines to achieve lines of curvature, which are sensitive to local bumps and need a density control method to ensure a uniform distribution over the surface. 
 In \figref{fig:comparison_to_principal_strips}, we compare visually their results to ours. In our method, we achieve a more uniform spacing of strips. In addition, complex models such as the beetle (\figref{fig:comparison_to_principal_strips}c) need to be pre-segmented manually into topological discs in \cite{Takezawa_2016_Fabrication_of_Freeform_Objects_by_Principal_Strips} before strips can be generated. In contrast, we can also parametrize such models as a whole (\figref{fig:comparison_to_principal_strips}c - right). 

% Figure environment removed


In \emph{Geometric Modeling with Conical Meshes and Developable Surfaces},
\cite{Liu_2006_Conical_meshes_and_developable_surfaces} propose a perturbation method for iteratively planarizing the faces of a quad mesh. The authors suggest that an ideal starting point for this optimization is a conjugate curve network, similar to what we produce when aligning our vector field to the principal curvature directions. Planarity is a very important property for various fabrication workflows. For example, strips that consist of planar quads are developable, meaning that they can be produced from flat sheets. When using alignment to principal curvature directions, the quads we produce can be expected to be nearly planar because we use conjugate curve networks to generate them. An additional optimization such as the one described in \cite{Liu_2006_Conical_meshes_and_developable_surfaces} can then planarize the faces further (\figref{fig:planarity}). Note that in this example, we ignore the non-quad faces that occur on the boundaries, displayed in white. 

% Figure environment removed






