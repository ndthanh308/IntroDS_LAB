\section{Introduction}
\label{sec:intro}

The representation of free-form surfaces using \emph{strips} has attracted significant attention for their various applications in architecture and manufacturing, such as the approximation of architectural envelopes using single-curved panels \cite{Pottmann_2008_Freeform_Surfaces_from_Single_Curved_Panels}, the fabrication of surfaces from flat sheets \cite{Takezawa_2016_Fabrication_of_Freeform_Objects_by_Principal_Strips}, the covering of surfaces with textiles \cite{Hilo_2017_prototype_website}, or the design and fabrication of corrugations/pleats \cite{Fornes_2017_Chrystalis_Amphitheater}. A free-form surface can be partitioned into strips using a parametrization along one direction discretized on a regular interval. We denote the collection of strips and their topology as a \emph{strip network}.

A one-dimensional parametrization, however, overlooks information pertaining to the orthogonal direction, which can be instrumental for various design and fabrication scenarios. Examples of such scenarios include woven structures, surfaces from sheets in crossing directions, warp-weft fabrication, custom rebar, cable-net structures, and the design of secondary transversal structural elements on structural strip patterns. In \emph{Minima-Maxima} \cite{Fornes_2017_Minima_MAxima} (\figref{fig:applications_of_SDQ_meshes}-left) a freeform surface is assembled from flat metal sheets forming two transversal strip networks that do not intermix. The \emph{Diamond chair} \cite{Bertoia_1952_Diamond_chair} (\figref{fig:applications_of_SDQ_meshes}-middle), which is fabricated using custom rebar, relies on the arrangement of metal rods along a crossing pattern where each transversal direction is realized with continuous rods. Finally, the \emph{NEST HiLo roof} \cite{Veenendaal2015_Hilo, VanMele2022_Hilo_cable_net, Hilo_2017_prototype_website} (\figref{fig:applications_of_SDQ_meshes}-right) is built with a cable-net and fabric formwork system, where two transversal strip networks are used both in the design for the assignment of force densities (bottom) and in the fabrication for the covering of the structure using fabric strips (top). More examples of applications of transversal strip networks are discussed in \cite{Oval_BRG_2021_Two-Colour_Topology_Finding_of_Quad-Mesh_Patterns}.


% Figure environment removed

In this work, we present a methodology for generating and editing transversal strip networks aligned with user-defined directions, while also enabling the incorporation of desirable properties for the fabrication of the strips. The type of geometries we address is relatively smooth geometries with open boundaries and without significant surface detail features.

The control over strip alignment has diverse applications. For instance, aligning strips with stress directions can enhance the mechanical properties of the resulting layout \cite{Schiftner_2010_Statics-Sensitive_Layout_of_Planar_Quadrilateral_Meshes}. Alignment with principal curvatures can yield a natural appearance, accentuate features, and facilitate the creation of nearly planar quads \cite{Liu_2006_Conical_meshes_and_developable_surfaces}. Moreover, alignment with user-defined curves enables a higher degree of design customization. Finally, maintaining alignment with boundaries ensures strips are consistently orthogonal or parallel to the boundary lines.


\cite{Pottmann_2008_Freeform_Surfaces_from_Single_Curved_Panels} studied strips as semi-discrete representations of freeform surfaces, using a two-dimensional surface parametrization with one continuous and one discrete parameter. We differ from this approach conceptually in that we discretize both directions. 
In particular, we consider the overlay of two transversal and topologically-coupled strip networks, such that each network discretizes its counterpart, producing what we call a \emph{strip-decomposable quad} (SDQ) mesh, as shown in \figref{fig:green_blue}. In this configuration, edges can be two-colored, i.e., we assign a color to every edge so that no two neighboring edges sharing a quad have the same color \cite{Oval_BRG_2021_Two-Colour_Topology_Finding_of_Quad-Mesh_Patterns}. Consequently, all vertices have even valences, and all faces are quadrilaterals, except for boundary faces. An SDQ mesh is therefore a quad mesh that can be unambiguously decomposed into two independent discrete strip networks that share the shame topology and discretization. 

% Figure environment removed

This fully discrete representation exhibits various advantages. Firstly, it is a compact representation that can be represented as a conventional quad mesh, a widely recognized standard across various modeling software. Additionally, as we show in this work, coupling two strip networks in this way enables editing and reconfiguring the strips using elementary topological operations, thereby providing greater flexibility and adaptability to the dual strip network.

Although SDQ meshes have numerous applications, there is a lack of techniques for their generation. Conventional quadrangulation methods, such as \cite{bommes2009_mixed-integer-quadrangulation,kalberer2007_quadcover---surface-parameterization-using}, are typically unsuitable, as they produce quad meshes with vertices of odd valence, making it impossible to two-color their edges unambiguously. A common workflow for creating SDQ meshes involves manually constructing low-resolution initial connectivity, either as a polygonal mesh \cite{2022_Tutorial_strip_morphologies_course, 2022_Tutorial_Digital_Fabrication_Workflow}, or as a skeletal graph \cite{Veenendaal2015_Hilo, 2023_Tutorial_with_stripper}, which is then subdivided until the desired resolution is reached. Furthermore, after generating an SDQ mesh, modifying the connectivity of its vertices, especially the combinatorics of the singularities, is challenging as local changes at the level of single quads can invalidate the strip decomposability. 

As a case study, we consider robotic non-planar 3D printing of freeform shell surfaces as a fabrication scenario that benefits from such a dual strip network, and apply our methodology for designing and fabricating non-planar print paths. 


\paragraph{Contributions}
We offer a pipeline for designing SDQ meshes while considering fabrication-related properties of strips. In particular, we contribute;
\begin{itemize}
    \item An algorithm to compute a directional field that considers fabrication objectives and integrates into two coupled transversal parameterizations corresponding to two transversal strip networks (\secref{sec:Stripe_parametrizations}).
    \item A set of simple and intuitive connectivity editing operations that enable extensive user control over the topology of the strips (\secref{sec:mesh_editing}).
\end{itemize}