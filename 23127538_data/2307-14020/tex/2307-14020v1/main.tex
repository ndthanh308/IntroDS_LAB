%% 
%% Copyright 2007-2020 Elsevier Ltd
%% 
%% This file is part of the 'Elsarticle Bundle'.
%% ---------------------------------------------
%% 
%% It may be distributed under the conditions of the LaTeX Project Public
%% License, either version 1.2 of this license or (at your option) any
%% later version.  The latest version of this license is in
%%    http://www.latex-project.org/lppl.txt
%% and version 1.2 or later is part of all distributions of LaTeX
%% version 1999/12/01 or later.
%% 
%% The list of all files belonging to the 'Elsarticle Bundle' is
%% given in the file `manifest.txt'.
%% 
%% Template article for Elsevier's document class `elsarticle'
%% with harvard style bibliographic references

\documentclass[preprint,12pt,authoryear]{elsarticle}

%% Use the option review to obtain double line spacing
%% \documentclass[authoryear,preprint,review,12pt]{elsarticle}

%% Use the options 1p,twocolumn; 3p; 3p,twocolumn; 5p; or 5p,twocolumn
%% for a journal layout:
%% \documentclass[final,1p,times,authoryear]{elsarticle}
%% \documentclass[final,1p,times,twocolumn,authoryear]{elsarticle}
%% \documentclass[final,3p,times,authoryear]{elsarticle}
%% \documentclass[final,3p,times,twocolumn,authoryear]{elsarticle}
%% \documentclass[final,5p,times,authoryear]{elsarticle}
%% \documentclass[final,5p,times,twocolumn,authoryear]{elsarticle}

%% For including figures, graphicx.sty has been loaded in
%% elsarticle.cls. If you prefer to use the old commands
%% please give \usepackage{epsfig}

%% The amssymb package provides various useful mathematical symbols
\usepackage{amssymb}
%% The amsthm package provides extended theorem environments
\usepackage{amsthm}
\usepackage{wrapfig}
\usepackage{hyperref}
\usepackage{natbib}
\usepackage{amsmath}
\usepackage{gensymb}
\usepackage{tikz}
\usepackage{pgfplots}
\usetikzlibrary{calc}

\usepackage[ruled]{algorithm2e} % For algorithms
\renewcommand{\algorithmcfname}{ALGORITHM}
\SetAlFnt{\small}
\SetAlCapFnt{\small}
\SetAlCapNameFnt{\small}
\SetAlCapHSkip{0pt}

\usepackage{array}

% !TEX root =  00-np3dp.txt

\usepackage{dsfont}
\usepackage{amsthm}
\usepackage{overpic}
\usepackage{contour}
\contourlength{1pt}
\usepackage{xspace}
\usepackage{sidecap}
\usepackage{rotating}
\usepackage[normalem]{ulem}
\usepackage{booktabs} % For formal tables
\usepackage{caption}
\usepackage{subcaption}
\usepackage{microtype}
\usepackage{booktabs}
\usepackage{enumitem}
\usepackage{amsthm}
\usepackage{wrapfig}
\usepackage{multirow}
\usepackage{tabularx}
\usepackage{nicefrac}
%\usepackage{algorithm}
%\usepackage{algorithmicx}
\usepackage{algpseudocode}
\usepackage{chngcntr}
\usepackage{apptools}
\usepackage[super]{nth}
\AtAppendix{\counterwithin{lemma}{section}}
\renewcommand{\algorithmicrequire}{\textbf{Input:}}
\renewcommand{\algorithmicensure}{\textbf{Output:}}

% olga: for strikethrough
\usepackage{ulem}

%\ifx\user\bram
\newtheorem{theorem}{Theorem}[section]
%\fi
\newtheorem{lemma}{Lemma}
\newtheorem{proposition}[theorem]{Proposition}
\newtheorem{corollary}[theorem]{Corollary}
%\newtheorem{defn}[theorem]{Definition}


\hyphenation{pa-ram-e-trize pa-ram-e-trized pa-ram-e-triz-ing
qua-ter-ni-on-ic}

\newcommand{\dash}{\discretionary{-}{}{-}\penalty1000\hskip0pt}
\newcommand{\HH}{\mathcal{H}}
\newcommand{\DD}{\mathcal{D}}
\newcommand{\LL}{\mathcal{L}}
\newcommand{\NN}{\mathbb{N}}
\newcommand{\RR}{\mathbb{R}}
\newcommand{\QQ}{\mathbb{Q}}
\newcommand{\QQQ}{\mathcal{Q}}
\newcommand{\CC}{\mathbb{C}}
\newcommand{\ZZ}{\mathbb{Z}}
\newcommand{\IM}{\mathop\textup{Im}\nolimits}
\newcommand{\RE}{\mathop\textup{Re}\nolimits}
%\newcommand{\Arg}{\textup{Arg}}
%\newcommand{\argmin}{\textup{argmin}}
\newcommand{\crr}{\textup{cr}}
\newcommand{\lcr}{\textup{lcr}}
\newcommand{\MobReg}{\textup{MR}}
\newcommand{\EucReg}{\textup{ER}}
\newcommand{\sign}{\mathop{\rm sign}}
\newcommand{\la}{\langle}
\newcommand{\ra}{\rangle}
\newcommand{\OO}{\mathcal{O}}
\newcommand{\MM}{\mathcal{M}}
\newcommand{\VV}{\mathcal{V}}
\newcommand{\XX}{\mathcal{X}}
\newcommand{\SSS}{\mathcal{S}}
\newcommand{\SSSstar}{\SSS^\ast}
\newcommand{\EE}{\mathcal{E}}
\newcommand{\BB}{\mathcal{B}}
\newcommand{\CCC}{\mathcal{C}}
\newcommand{\FF}{\mathcal{F}}
\newcommand{\TT}{\mathcal{T}}
\newcommand{\PP}{\mathcal{P}}
\newcommand{\kerM}{\text{ker}}
\newcommand{\imM}{\text{Image}}
\newcommand{\dVV}{\overline{\VV}}
\newcommand{\dEE}{\overline{\EE}}
\newcommand{\dTT}{\overline{\TT}}
\newcommand{\dMM}{\overline{\MM}}

\newcommand{\mobius}{M{\"o}\-bi\-us\xspace}
\newcommand{\mobiusreg}{M{\"o}bius\dash regular\xspace}

\newcommand{\num}{\renewcommand{\labelenumi}{(\roman{enumi})}
  \renewcommand{\theenumi}{(\roman{enumi})}}

\newcommand{\TODO}[1]{\colorbox{yellow}{TODO: $\ll$}#1\colorbox{yellow}{$\gg$}}

\newcommand{\mybullet}{$\bullet$\hspace{.6em}}

\newlength{\indentlaenge}
\setlength{\indentlaenge}{\parindent}
\newlength{\mylength}
\newlength{\mylengthzwei}
\setlength{\mylength}{\baselineskip}
\settoheight{\mylengthzwei}{A}
\addtolength{\mylength}{-1.5\mylengthzwei}

\newcommand{\AV}[1]{{\bf\textcolor{blue}{AV: #1}}}
\newcommand{\changed}[1]{{\bf\textcolor{magenta}{#1}}}

% Olga D's shortcuts
% https://tex.stackexchange.com/questions/58713/ref-should-use-enumerate-label-name
\usepackage{enumitem}
% https://tex.stackexchange.com/questions/253910/reference-to-enumerate-item-with-manually-set-label
\makeatletter
\newcommand{\myenumlabel}[2]{#2\def\@currentlabel{#2}\label{#1}}
\makeatother

\makeatletter
\newcommand{\specialcell}[1]{\ifmeasuring@#1\else\omit$\displaystyle#1$\ignorespaces\fi}
\makeatother
\newcommand{\pushright}[1]{\ifmeasuring@{#1}\else\omit\hfill$\displaystyle{#1}$\fi\ignorespaces}
\newcommand{\equ}[2]{{\begin{equation}#2\label{#1}\end{equation}}}
\newcommand{\equa}[2]{{\begin{eqnarray}#2\label{#1}\end{eqnarray}}}
\newcommand{\bdm}[1]{\begin{displaymath}{#1}\end{displaymath}}
% \newcommand{\itemz}[1]{{\begin{itemize}{#1}\end{itemize}}}
\newcommand{\mtrx}[1] {\begin{bmatrix}#1\end{bmatrix}}
% References ============================================================================
\renewcommand{\eqref}[1]{Eq.~(\ref{#1})}
\newcommand{\figref}[1]{Fig.~\ref{#1}}
\newcommand{\secref}[1]{Sec.~\ref{#1}}


\newcommand{\att}[1]{\textcolor{magenta}{{\slshape #1}}}
\newcommand{\setR}{\mathds{R}}
\newcommand{\setC}{\mathds{C}}
\renewcommand{\Re}[1]{\mathop{Re}\left(#1\right)}
\renewcommand{\Im}[1]{\mathop{Im}\left(#1\right)}

\def\cput(#1,#2)#3{\put(#1,#2){\hbox to 0pt{\hss{#3}\hss}}}
\def\lput(#1,#2)#3{\put(#1,#2){\hbox to 0pt{\hss{#3}}}}
\def\rput(#1,#2)#3{\put(#1,#2){\hbox to 0pt{{#3}\hss}}}

% Olga SH notation
\newcommand{\set}[1]{\mathcal{#1}}


%% The lineno packages adds line numbers. Start line numbering with
%% \begin{linenumbers}, end it with \end{linenumbers}. Or switch it on
%% for the whole article with \linenumbers.
%% \usepackage{lineno}

\journal{}

\begin{document}

\begin{frontmatter}

%% Title, authors and addresses

\title{Fabrication-Aware Strip-Decomposable Quadrilateral Meshes}

\author[1]{Ioanna Mitropoulou}
\author[2]{Amir Vaxman}
\author[3]{Olga Diamanti}
\author[1]{Benjamin Dillenburger}

\affiliation[1]{organization={ETH Zurich},
            country={Switzerland}}

\affiliation[2]{organization={University of Edinburgh},
            country={United Kingdom}}

\affiliation[3]{organization={TU Graz},
            country={Austria}}

\begin{abstract}
Strip-decomposable quadrilateral (SDQ) meshes, i.e., quad meshes that can be decomposed into two transversal strip networks, are vital in numerous fabrication processes; examples include woven structures, surfaces from sheets, custom rebar, or cable-net structures. However, their design is often challenging and includes tedious manual work, and there is a lack of methodologies for editing such meshes while preserving their strip decomposability. We present an interactive methodology to generate and edit SDQ meshes aligned to user-defined directions, while also incorporating desirable properties to the strips for fabrication. 
Our technique is based on the computation of two coupled transversal tangent direction fields, integrated into two overlapping networks of strips on the surface.
As a case study, we consider the fabrication scenario of robotic non-planar 3D printing of freefrom shell surfaces and apply the presented methodology to design and fabricate non-planar print paths.

\end{abstract}


%%Graphical abstract
\begin{graphicalabstract}
% Figure environment removed
\end{graphicalabstract}

%%Research highlights
\begin{highlights}
\item A methodology for generating strip-decomposable quad (SDQ) meshes by integrating a pair of transversal tangent 2-vector fields.
\item A set of editing operations for improving the strips topology and patch layout of an SDQ mesh. 
\item The application of SDQ meshes for robotic non-planar 3D printing of shells.
\end{highlights}


\begin{keyword}
strip-decomposable quad mesh \sep strip network \sep vector field \sep fabrication-aware \sep print path design \sep robotic 3D printing \sep non-planar 
\end{keyword}

\end{frontmatter}

%% \linenumbers

%% main text




% Figure environment removed



The understanding of 3D scene geometry is essential for many down-stream applications.  In robotics, it allows for accurate manipulation and motion planning considering the surrounding environment.  In the field of augmented reality, it allows for better mapping and rendering to bridge the virtual world to the real world.  With smartphones and robots that are equipped with high quality depth sensors, the task of 3D scene reconstruction is becoming feasible in these domains. 
%
These depth sensors allow for accurate reconstruction of the observed parts of the scene. However, to reconstruct the unseen parts, we must use prior information conditioned on the observed information. The missing information in the input image combined with the diversity in shapes, sizes, and depth distribution of the household objects presents a major challenge for scene reconstruction in-the-wild. 
%
In this paper, we study this problem in a general setting, where the goal is to reconstruct a complex scene with multiple novel objects, given only one RGB-D image of the scene.  
%



We present our method Rotate-Inpaint-Complete (\ours{}), which predicts both the 3D geometry and the texture of the unseen parts of the scene in the input image by leveraging the inpainting capabilities of large visual-language models.
%
Given an RGB-D image of a scene, first we generate novel views (RGB and depth images) by rotating and then projecting the input scene. Then we use a surface-aware masking method to select regions in the image to allow us to inpaint utilizing the powerful 2D inpainting capabilities of \dalle{}~\cite{ramesh2022hierarchical} for exposing the potential object geometry not visible in the input image. 
%
Finally, we optimize the depth images using the input depth values and occlusion boundaries and normals estimated from the inpainted images. These inpainted and completed novel RGB-D views provide the reconstructed scene geometry as a fused pointcloud with associated textures.
To mitigate the object hallucination and spatial inconsistency of predictions from \dalle{}, we integrate algorithmic features such as filtering inpainting outputs and enforcing consistency across viewpoints into our method that play a crucial role for generalizable, yet accurate and robust scene reconstruction.

%-------------------------------------------------------

We demonstrate our method on cluttered scenes with unseen household objects and categories. Through a series of rigorous quantitative experiments, we show that our approach outperforms baseline methods in settings where no training data is available.

% \subsection{Statement of Contributions}
In short, the contributions of this paper can be summarized as follows. \textit{i)} We present an integrated approach for scene completion of unseen objects under occlusion and clutter, by solving the problem through novel view inpainting and 2D to 3D scene lifting. \textit{ii)} We develop a method for selectively inpainting regions in the novel views of the input scene that enables synthesis of consistent 2D geometry. \textit{iii)} We train a 2D to 3D lifting method on the YCB-V~\cite{xiang2018posecnn} dataset and demonstrate the generalization capability on novel household objects and categories which is crucial for maintaining the generalization capability of our integrated scene reconstruction method.


\section{Related Work}
\label{sec:Related Work}

\subsection{Fabrication-aware strips}
Strip meshes have been studied extensively for their various applications in design and fabrication. 
\cite{Pottmann_2008_Freeform_Surfaces_from_Single_Curved_Panels} studied the refinement and convergence properties of strips and presented a method that partitions freeform surfaces into single curved strips. Spiraling, developable strips were used in \emph{Zippables} \cite{schuller2018_shape-representation-by-zippables} to zip a surface together from flat ribbons, while \cite{verhoeven2022_dev2pq:-planar-quadrilateral-strip-remeshing} extracted a quad dominant strip pattern of developable surfaces. \cite{Differentiable_Stripe_Patterns_Maestre_2023} use strips to represent bi-material distributions with high stiffness contrast and compute patterns that lead to an optimal approximation of high-level performance goals. 

The transversal strip network is introduced in \cite{Takezawa_2016_Fabrication_of_Freeform_Objects_by_Principal_Strips} for the fabrication of freeform objects from flat paper pieces using principal strips aligned with the principal curvature directions. 
However, this approach relies on tracing streamlines, which  are sensitive to local perturbations and can thus  only be applied to smooth surfaces with disc or annulus topology. In contrast, we apply our approach to more complex surfaces with a higher genus and more boundaries.

\cite{Oval_BRG_2021_Two-Colour_Topology_Finding_of_Quad-Mesh_Patterns} proposed an alternative way of generating a two-colored quad mesh topology, starting with fixed singularities and then combinatorially connecting them into patches that are subdivided to generate the final mesh.
This approach relies on pre-defined fixed singularities and thus does not enable control over the alignment of the resulting edges. In contrast, our method positions singularities dynamically during the vector-field optimization stage.

\cite{Schiftner_2010_Statics-Sensitive_Layout_of_Planar_Quadrilateral_Meshes} generate quad-dominant meshes by integrating two separable conjugate line field directions for structural purposes.
Similarly, in \cite{Zadravec_2010_Designing_Quad-dominant_Meshes_with_Planar_Faces}  quad-dominant meshes are created by integrating a pair of line fields aligned to principal curvature directions. In both cases, since the guiding line fields are separable and conjugate, the resulting quad meshes are strip-decomposable and quasi-planar. Their planarity is further improved with a perturbation optimization \cite{Liu_2006_Conical_meshes_and_developable_surfaces}. 
From our perspective these methods are the closest to our approach since we also rely on the integration of separable line fields. However, our work allows for more freedom and control of the directional constraints and location of singularities, and we consider a broader scope of application scenarios.

\subsection{Field-aligned parameterization and remeshing}
Our work uses a pair of line fields integrated into scalar functions from which the strips are extracted as their isolines, and are overlayed to create a field-guided SDQ mesh. We refer the reader to the surveys~\cite{goes2016_vector-field-processing-on-triangle-meshes,vaxman2016_directional-field-synthesis-design-and-processing} to get acquainted with field-aligned meshing methods. While our work also designs a quadrilateral mesh from directional-field-oriented parameterization (like eg.,~\cite{bommes2009_mixed-integer-quadrangulation,kalberer2007_quadcover---surface-parameterization-using}), it departs from such methods in that we create the quad mesh as two separable sets of \emph{strips} with embedded fabrication considerations. By that, our work is, in fact, more related to the concept of ``stripe patterns'' of ~\cite{knoppel2015_stripe-patterns-on-surfaces}, although those were not used for meshing or optimized for fabrication processes. 

Our design optimizes field integrability, which is essential for the fidelity of the parameterization and meshing. To this end, we aim to reduce curl, as considered in~\cite{diamanti2015_integrable-polyvector-fields,vekhter2019_weaving-geodesic-foliations}. However, our optimization follows an iterative smoothing and curl reduction process, adapted from methods such as ~\cite{sageman-furnas2019_chebyshev-nets-from-commuting,meekes2021_unconventional-patterns-on-surfaces,liu2020_practical-fabrication-of-discrete-chebyshev-nets}, and tailored to our specific requirements.



\subsection{Quadrilateral-mesh topology editing}
In the process of creating feasible, not entwined, and aesthetically pleasing strips, we introduce an interface for basic topology editing operations. These operations aim mainly to control the location and connectivity of the singularities of the mesh, which affect how the shape can be partitioned into a good patch layout. As such, these operations are similar in spirit to topology editing in quadrilateral meshes~\cite{peng2011_connectivity-editing-for-quadrilateral-meshes,takayama2013_sketch-based-generation-and-editing-of-quad-meshes,tarini2010_practical-quad-mesh-simplification,daniels-ii2009_localized-quadrilateral-coarsening}. Our operations, however, work on strips and not quadrilaterals, and as such have a more global effect, and guarantee that the strip-decomposability of the mesh is maintained. In~\cite{Noma2022FastEditing}, strip singularities are edited directly as scalar functions, but not as strip meshes. To the best of our knowledge, we are the first to perform topology editing on SDQ meshes.
\section{Overview}
\label{sec:Overview}

We aim to generate a fabrication-aware and editable SDQ mesh with edges aligned to user-defined directions. In this section, we describe in detail our objectives and pipeline. 

\subsection{Objectives}
We consider the following geometric objectives for the dual strip networks, commonly found in the fabrication scenarios mentioned in \secref{sec:intro}.

\begin{enumerate}

\item \emph{Alignment}. \label{obj:aligned}
The strip networks are aligned as well as possible with user-specified directions. 

\item \emph{Uniformity}. \label{obj:uniform} 
The strips have a width that is as uniform as possible, resulting in quads with as uniform as possible sizes.

\item \emph{Smoothness}. \label{obj:smooth}
The strips are as smooth as possible, meaning that they vary slowly in space without sharp turns or noisy boundaries.

\item \emph{Orthogonality}. \label{obj:orthogonal}
The two transversal strip networks are as orthogonal as possible to each other. This translates to an SDQ mesh where all regular vertices have four incident edges with angles close to $\frac{\pi}{2}$. 

\item \emph{Continuity}. \label{obj:continuous}
There are no interruptions or discontinuities of strips on the interior of the shape.  

\item \emph{Good strip topology}. \label{obj:strips_topology}
The strips have a good topology, meaning that they do not wind before they close or terminate on a boundary. Instead, they extend from boundary to boundary, or form closed loops without snake-like parts where they fold into themselves.

\item \emph{Good patch layout}. \label{obj:patch_layout}
Each strip network can be partitioned into a small number of patches of simply connected strips, i.e., it has a good patch layout.
\end{enumerate}


Objectives (\ref{obj:aligned})--(\ref{obj:strips_topology}) are optimized in the vector-field design stage (\secref{sec:Stripe_parametrizations}). In addition, objective (\ref{obj:continuous}) is further addressed in the subsequent seamless integration (\secref{subsec:integration-and-meshing}). Objectives (\ref{obj:strips_topology})--(\ref{obj:patch_layout}) are achieved by the topological operations that enable editing the mesh (\secref{sec:mesh_editing}).


\subsection{Pipeline}
\label{sec:pipeline}

Given a triangle mesh surface, possibly with open boundaries, our pipeline proceeds as follows.

\begin{enumerate}
\item	Collect user-specified directional constraints (\secref{par:alignment}).
\item	Compute two coupled integrable line fields that adhere to the objectives (\secref{subsec:guiding-fields}).
\item	Integrate the fields into two transversal parameterizations. Discretize those into two sets of strips, which, when overlaid, form an SDQ mesh (\secref{subsec:integration-and-meshing}).
\item	Carry out strip-based topological operations for eliminating winding strips and aligning singularities, with the help of user input, to create a good patch layout (\secref{sec:mesh_editing}).
\end{enumerate}

\section{Stripe parametrizations}
\label{sec:Stripe_parametrizations}


We describe steps (1)-(3) of the pipeline, where we first design a directional field and then integrate it into a parameterization suitable for our framework. 

\subsection{Transversal parametrizations}
\label{sec:stripe-parameterization}


We work with a triangular mesh $\MM = \left\{\VV,\EE,\FF\right\}$. 
We denote by $\mathcal{C}$ the set of \emph{corners} $(v,f)$, where $v \in \VV$ denotes a vertex and $\ f \in \FF$ a face adjacent to this vertex. We begin by constructing two transversal parameterizations $U$ (displayed throughout  in blue), and $V$ (displayed in green) on $\MM$ that represent the two strip networks. 


% Since printing a path can be done in both directions, 
Since strips are \emph{sign-invariant}, i.e., they can be traversed in both directions without any difference in the result, the parameterization must also be sign-invariant. To achieve that, we design a strip network as a branched $2$-function \cite{vaxman2016_directional-field-synthesis-design-and-processing}, which admits singularities of indices $\pm \frac{k}{2},\ k \in \ZZ$. Formally, a $2$-function $U:\mathcal{C} \rightarrow \RR$ is defined as the assignment of a scalar per corner, linearly interpolated inside each triangle. Furthermore, on every edge $e$ between vertices $i, j$ and adjacent to faces $f,g$, we have:
$$
U_{f,i}-U_{f,j} = s_e (U_{g,i}-U_{g,j}),\quad\text{for some } \, s_e = \pm 1.
$$
The specific choice of the sign $s_e$ is called the \emph{matching} on the edge $e$.
A regular patch $\mathcal{P} \subset \MM$ is a sub-mesh where a function $U$ can be ``combed'' into a single (1-)function without any sign ambiguity. Combing means re-signing all corner values on the patch so that they agree on all $e \in \mathcal{P}$ with $s_e=1$. Patches where this is not possible, are called \emph{singular}; they contain branching points, or \emph{singularities}, of the parameterization (\figref{fig:green_blue}, \ref{fig:vec_sings_to_mesh_sings}). 

The output of this stage is two $2$-functions $U$ and $V$, each representing one strip network. We design those functions by optimizing their gradients, which we denote with small case letters $u=\nabla U,v=\nabla V$. This is a common modus operandi to construct such \emph{seamless} parameterizations (eg.,~\cite{sageman-furnas2019_chebyshev-nets-from-commuting,meekes2021_unconventional-patterns-on-surfaces,verhoeven2022_dev2pq:-planar-quadrilateral-strip-remeshing}). $u$ and $v$ are the \emph{guiding fields} through which we can incorporate the objectives listed above. 


\subsection{Guiding fields}
\label{subsec:guiding-fields}

The gradient of a $2$-function is a $2$-vector field (\figref{fig:vec_sings_to_mesh_sings}), which is also sign-invariant and has  the same singularities. Furthermore, the field is \emph{face-based} and \emph{piecewise-constant}, with a single $2$-vector in each face $f \in \FF$, and it is curl-free; that means on edge $e$ shared by faces $f$, $g$ we have:
 \begin{equation}
\langle u_f, e \rangle = s_e \langle u_g,e \rangle,\ \ \langle v_f, e \rangle = s_e \langle v_g,e \rangle.
\label{eq:curl_conditions}
 \end{equation}

 
Our main variables are two $2$-fields $X, Y$, piecewise constant per face,  that represent the \emph{candidate gradient fields}  $u, v$ of the parameterizations.
The $2$-fields are expressed in the \emph{power} representation~\cite{knoppel2013_globally-optimal-direction} to achieve sign invariance: we represent $u_f$ and $v_f$ as complex numbers in an arbitrary local basis in each face $f$, and then define:
$$
X_f = (u_f)^2,\ Y_f = (v_f)^2.
$$


\subsubsection{Energies}
\label{subsec:energies}

 A useful energy term is a generic \emph{closeness} energy for power fields with the structure:
 $$
 E_C(K,R,M) = (K - R)^H M (K - R). %{\mathbf{1}^T M \mathbf{1}}.
 $$
 This energy measures how close the field $K$ is to an input field $R$ in a metric defined via the mass matrix $M$. $(.)^H$ denotes the conjugate transpose of $(.)$. With this, we set the following optimization energy terms.

\paragraph{\textbf{Alignment} - objective \ref{obj:aligned}} 
\label{par:alignment}
This energy term aims to keep the two 2-vector fields aligned with the input directional constraints.
Each constrained face $f_S$ in the set of constrained faces $\FF_S$ has a constraint tangent alignment $2$-vector given in the power representation $\alpha_f$, and a confidence weight $\omega_f$. We collect the confidence weights in a confidence matrix $W = \text{diag}(\omega_f)$, and also compute an orthogonal tangent constraint $2$-vector $\beta_f = -\alpha_f$ for the transversal direction $Y$. Then the alignment energy is written as closeness energy to the directional constraints. 

 \begin{equation}
 E_A(X, Y) = E_C(X_S, \alpha, W\cdot M_{\FF|S}) + E_C(Y_S, \beta, W\cdot M_{\FF|S}).
 \label{eq:soft-alignment-term}
 \end{equation}
$M_{\FF|S}$ is the diagonal mass matrix of triangle areas restricted to the alignment faces $\FF_S$, as this energy is only applied to the constrained faces in $\FF_S$. 

We employ three presets of input alignment constraints, each beneficial in a different context: %Throughout the  document we depict the $U$ direction with blue, and the $V$ direction with green color.
\begin{itemize}
    \item \emph{Curvature alignment.} We use principal curvature directions for alignment, with confidence value (Eq.~\ref{eq:soft-alignment-term}) for each face $\omega = \kappa_{\text{max}}-\kappa_{\text{min}}$, where $\kappa_{(\cdot)}$ are the principal curvatures.  We then choose as the  constrained faces $\FF_S$ those with the highest confidence values (by default: top $\% 10$). We set $\alpha$ and to $\beta$ be the square of the minimum and maximum principal curvature vectors respectively (\figref{fig:different_input_constraints}a).
    \item \emph{Boundary constraints.} We add each boundary face with one boundary edge $e$ to the constrained faces and set $\alpha$ and $\beta$ to be exactly orthogonal and parallel to $e$ respectively (\figref{fig:different_input_constraints}b), with confidence values $\omega = 1$ on all constrained faces.
    \item \emph{User-drawn directions.} The user draws a curve on the surface from which we extract constrained faces. We then set $\alpha$ to be orthogonal and to the curve and $\beta$ parallel (\figref{fig:different_input_constraints}c), with confidence values $\omega = 1$ on all constrained faces.
\end{itemize}


\paragraph{\textbf{Unit length} - objective \ref{obj:uniform}} This energy term aims to keep the 2-vector fields as close as possible to unit norm, and can be written as a closeness energy to 
$\hat{X}=\frac{X}{|X|}$ and $\hat{Y}=\frac{Y}{|Y|}$ (normalized $X$ and $Y$). 
    \begin{equation}
    E_U(X, Y) = E_C(X,\hat{X},M_\FF) + E_C(Y,\hat{Y},M_\FF)
    \end{equation}
where $M_{\FF}$ is the diagonal mass matrix of triangle areas. We note the use of the auxiliary variables $\hat{X}$ and $\hat{Y}$ that are ``snapshot'' (to a previous iteration in our algorithm), keeps this energy quadratic.


 \paragraph{\textbf{Smoothness} - objective \ref{obj:smooth}} 
 This energy term aims to keep the 2-vector fields as smooth as possible.
 For smoothness, we consider the $|\EE_I|\times|\FF|$ discrete \emph{covariant derivative} matrix $D$ for power $2$-fields (with the weights of ~\cite{brandt2018_modeling-in/i-symmetry-vector-fields-using}), which for every internal (non-boundary) edge $e \in \EE_I$ on adjacent faces $f$ and $g$ computes:
 $$
 (DX)_{|e} = \left(X_f\cdot (\overline{e}_f)^2 - X_g\cdot (\overline{e}_g)^2\right),
 $$
 where $\overline{e}_f$ (resp. $\overline{e}_g$) is the \emph{normalized} conjugate edge vector $e$ in the basis of $f$ (resp. $g$). We also need the $|\EE_I| \times |\EE_I|$ diagonal \emph{mass matrix} $M_D$, where $M_D(e,e) = \frac{3|e|^2}{A(f) + A(g)}$, and $A(\cdot)$ denotes the area operator. Then, the smoothness energy is:
 \begin{equation}
     E_S(X,Y) = X^H (D^H M_D D) X + Y^H (D^H M_D D) Y 
 \end{equation}


 \paragraph{\textbf{Orthogonality} - objective \ref{obj:orthogonal}} This energy is the only term that links $X$ with $Y$. It aims to keep the two 2-fields as orthogonal as possible.
 \begin{equation}
     E_O(X,Y) = (X+Y)^H M_\FF (X+Y) % {\mathbf{1}^T M_\FF \mathbf{1}}.
 \end{equation}
 When $X=-Y$ we get $u = \pm i\cdot v$, which implies that the vectors are perfectly orthogonal (and with equal magnitudes). 



 \paragraph{Normalization of energies}
 We normalize all energy terms to make them of comparable magnitudes with each other. The energies have the from $E = P^H \mathcal{A}P$. To normalize them, we divide by $\mathbf{1}^T \mathcal{A} \mathbf{1}$, where $\mathbf{1}$ is the vector of all $1$'s.

\subsubsection{Constraints}

 \paragraph{\textbf{Integrability}} 
 As described above, the guiding fields must be curl-free to represent the gradients of the parametrization functions. The curl reduction strategy aims to keep the curl of the fields $u$ and $v$ to zero during the optimization. Note that, unlike all previous energy terms, the curl of a field is not sign-invariant. Therefore we need to compute an explicit matching $s_e$ on every inner edge $e \in \EE_I$, as per \eqref{eq:curl_conditions}. 
 
 
\paragraph{Matching}
\label{par:matching}
To find the matching (i.e. the sign of each vector) we first compute $u_f = \sqrt{X_f}$ and $v_f = \sqrt{Y_f}$ in each face. Note that the solution to this has arbitrary signs. 
We then find the sign for one of the 2-fields, say $u$; namely for each edge $e$ between faces $f$, $g$, we find the sign $s_e \in \left\{-1,0,1\right\}$ per edge $e \in \EE_I$ so that it minimizes rotation effort between $\left\{u,-u\right\}_f$ and $\left\{u,-u\right\}_g$ (also known as the \emph{principal} matching~\cite{diamanti2014_designing-n-polyvector-fields-with}). We then update the signs of the other field to have the same matching. This results in two positively-oriented fields with the same topology by design.
 
\paragraph{Curl elimination} 
Equipped with the matching, we build the $|\EE_I|\times |\FF|$ \emph{curl matrix} $C_s$ which, on every edge $e$ between faces $f$ and $g$, computes, for the $u$ field:
 \begin{equation}
     (C_s\cdot u)_{|e} = \langle s_e u_g - u_f, e\rangle,
 \end{equation}
 and similarly for $v$. To eliminate curl, we solve a projection problem to the nearest curl-free field with the given matching $\mathbf{s}$, namely:
\begin{align*}
    (u,v)^* &= \text{argmin}|u^*-u|^2+|v^*-v|^2\ s.t.\\ C_s\begin{pmatrix}u^*\\v^*\end{pmatrix}&=0
\end{align*}
via the minimum $2$-norm projection on the null space of $C_s$ \cite{verhoeven2022_dev2pq:-planar-quadrilateral-strip-remeshing}.
 

\subsubsection{Optimization problem}
With these, our full optimization problem for the fields is:

\begin{align}
    (X, Y,u, v,s) = &\text{argmin}( \lambda_S E_S+ \lambda_OE_O + \lambda_U E_U+ \lambda_A E_A)        \label{equ:optimizationproblem}\\
    \text{subject to \quad} & C_s\cdot u_f = 0,\ C_s\cdot v_f = 0   ,\quad \forall f \in \FF
    \tag{\small curl-free} \\
    &X_f = u_f^2,\ Y_f = v_f^2, \quad \forall f \in \FF \tag{\small compatibility}
\end{align}



\subsection{Optimization}
\label{subsec:optimization}
\subsubsection*{Optimization algorithm}
The directional-field optimization problem in \eqref{equ:optimizationproblem} is nonlinear, due to the normalized compatibility and unit-length term, and discrete, because of the matching variable $s$. We use an alternating optimization scheme, summarized in Alg.~\ref{alg:full-optimization-directional-field}, where we alternate between reducing a quadratic energy $E^k$ with an implicit step, principal matching (in closed-form),  curl-elimination (another quadratic energy), and closed-form projection to the compatiblity constraints. To encourage convergence, we introduce a factor $\lambda_E$ that attenuates the objective terms in every iteration. 


\begin{algorithm}[h!]
\SetAlgoLined
 Initialize $k\leftarrow 0$, \\
 $\left(X^0,Y^0, E^0\right)\leftarrow \text{argmin} (\lambda_SE_S+\lambda_AE_A+\lambda_OE_O)$\hfill \emph{(variables initialization)}\\
$\forall f\in \FF,\ X_f^0\leftarrow \frac{X_f^0}{|X_f^0|},\ Y_f^0\leftarrow \frac{Y_f^0}{|Y_f^0|}\  $\hfill \emph{(normalization of initial solution)} \\
$\lambda_E^0 = 1$\\
 $\mu = \text{smallestEigenValue}(E^0,M_\FF),\ dt = \nicefrac{1}{\mu}$\\
 \Repeat{$ \|E^k-E^{k-1}\| < 10^{-4}$ \text{or} $k=100$}{
 	
    $(X^{k+1}, Y^{k+1}, E^{k+1}) \rightarrow \text{ImplicitStep}\left(X^k, Y^k,  E^k, M_\FF, dt\right)$\\

    $\hat{X}_f^{k+1}\leftarrow \frac{X_f^{k+1}}{|X_f^{k+1}|},\ \hat{Y}_f^{k+1}\leftarrow \frac{Y_f^{k+1}}{|Y_f^{k+1}|}$\hfill \emph{(re-normalization)} \\
    
    $(u^{k+1},v^{k+1},s)\leftarrow\text{PrincipalMatching}(X^{k+1},Y^{k+1})$\\

    $(u^{k+1},v^{k+1})\leftarrow\text{CurlElimination}(u^{k+1},v^{k+1},s)$ \\
    $X^{k+1}=(u^{k+1})^2$, $Y^{k+1}=(v^{k+1})^2$\hfill \emph{(compatibility)}\\
    $\lambda_E^{k+1} \leftarrow 0.8\lambda_E^k$\hfill \emph{(dampening)}\\
    $k \leftarrow k+1$\\
 }
 \caption{Directional-Field Optimization}
 \label{alg:full-optimization-directional-field}
\end{algorithm}


\paragraph{Implicit Step} The generic function $$y=\text{ImplicitStep}(x, A, M, dt)$$ performs an implicit step to minimize an objective of the type $E = x^TAx$ with the a $M$. It returns $y$ (and the energy) as the solution to the linear system:
$$
(M + dt\cdot A)y = Mx.
$$
Our time step $dt$ is inversely proportional to the scaling of the energy by the Fiedler value $\mu$, as explained in~\cite{sageman-furnas2019_chebyshev-nets-from-commuting}. 


\subsubsection*{Choice of parameters.}
\label{sec:directional_constraints}

We use the following energy weights: 
     smoothness $\lambda_S = 10.0$,
     orthogonality $\lambda_O = 2.0$,
     alignment  $\lambda_A = 0.1$,
     and unit $\lambda_U = 1.0$.
     % and previous $\lambda_P = 0.1$. \ioanna{verify}
$\lambda_E$ has an initial value of 1.0, which is multiplied by 0.8 after every iteration to encourage convergence to the constraints.


% Figure environment removed



\subsection{Integration and Meshing}
\label{subsec:integration-and-meshing}

Having optimized the two coupled fields $u$ and $v$, we integrate them into parameterization functions $U$ and $V$. 
To do that, we combine $u$ and $v$ in one 4-field so that the vectors in each face are arranged in the order $\left\{u,v,-u,-v\right\}$ in CCW fashion around the normal. The matching of the 4-field on every edge $\mathbf{s_e}$ equals $2 s_e$, i.e. it has values in $\left\{-2,0,2\right\}$ so that $u$ and $v$ never intermix in the 4-field. We can then use a standard frame-field integration and meshing algorithm to integrate the 4-field into the  functions $U$ and $V$. These functions result in two coupled transversal strip networks (Fig. \ref{fig:green_blue}) that are overlaid into a single quad mesh $\MM_\QQQ = \left\{\VV_\QQQ, \EE_\QQQ, \FF_\QQQ\right\}$ that is strip-decomposable by design. The singularities of the field are directly translated into singular vertices of $\MM_\QQQ$ (~\figref{fig:vec_sings_to_mesh_sings}). Integration and meshing are done using the library \texttt{Directional}~\cite{amir_vaxman_and_others_2021_5746726}.


Note that we use the same notation ($U$, $V$) for the parameterization functions and for resulting strips, as they essentially refer to the same geometry. $U$-strips (blue) are bounded by $U$-edges which are isolines of the integration of the $u$ vectors, i.e., always orthogonal to the $u$-field. Similarly, $V$-strips (green) are bounded by $V$-edges orthogonal to the $v$-field. 
$\MM_\QQQ$ has non-quad faces only on the boundaries, where the strips are cut by the boundary. 


% Figure environment removed



 \paragraph{Discussion} Our directional-field framework is inspired by  similar frameworks for integrable aligned parameterization. Specifically, there is considerable similarity with integrable $4$-fields (``frame fields'') since we create the combined field $\left\{u,v,-u,-v\right\}_f$ in each face. The overlay of our resulting parameterizations is technically a $4$-function (of the kind used for quad-meshing). However, existing algorithms for integrable frame fields also allow both directions to interchange in a way that permits $\pm \frac{1}{4}$ singularities, leading to a general quad mesh that does not necessarily separate into transversal strips without cuts or discontinuities (see \figref{fig:comparison-with-cross-field}). To the best of our knowledge, no existing algorithm specifically treats our case of SDQ meshes with directional fields.

 % Figure environment removed
\section{Mesh editing}
\label{sec:mesh_editing}

We next describe step (4) of our pipeline (\secref{sec:pipeline}), namely, editing the topology of the mesh to simplify it in order to achieve good strips topology (objective \ref{obj:strips_topology}) and a good patch layout of both strips networks (objective \ref{obj:patch_layout}). This consists of aligning singularities and correcting winding strips (eg. the blue strips in \figref{fig:topological-defect}b and \figref{fig:routes}e). 
This step includes user interaction which we simplified to a small number of intuitive steps.
For our editing interface, we use the \emph{libigl} \cite{jacobson2018_libigl:-a-simple-c-geometry-processing-library} viewer.



\subsection{Topology Editing} 
\label{subsec:topology-editing}


% ----------------------------
% Topology editing
% ----------------------------

\subsubsection*{Dealing with topological defects} 
Our directional field, and consequent seamless parameterization and mesh, are optimized for smoothness and good alignment with directional constraints, which incurs well-placed singularities for the most part. However, the mixed-integer seamless integration can potentially cause \emph{misalignment} between two close-by singularities that should have been naturally paired. For example, in \figref{fig:topological-defect}b, the singularities are slightly misaligned along the route that goes around the right handle, while in \figref{fig:topological-defect}c they are aligned along all highlighted routes. Even such slight misalignment can cause long undesirable winding strips with self-folding geometry or very thin patches. We offer a paradigm for topological editing that can easily fix most defects with simple and minimal user control. 

% Figure environment removed

We note that there are paradigms for canonically partitioning quad meshes into singularity-free sub-meshes without altering the underlying topology of the mesh, such as motorcycle graphs
\cite{Eppstein-2008-Motorcycle_Graphs, Schertler_2018_Generalized_Motorcycle_Graphs_for_Imperfect_Quad-Dominant_Meshes}. However, such approaches would not work for our case since we are, in essence, partitioning strip networks instead of quad meshes. This means that we consider cuts along edges of the same color for the most part, thus few or no intersections are created where cuts can terminate. Therefore, we opt for a new topological editing workflow described below.


\paragraph{Atomic operations}
Editing a strip-decomposable mesh is a more constrained problem than editing typical quad meshes~\cite{tarini2010_practical-quad-mesh-simplification,daniels-ii2009_localized-quadrilateral-coarsening,peng2011_connectivity-editing-for-quadrilateral-meshes,takayama2013_sketch-based-generation-and-editing-of-quad-meshes}, as we have a strict requirement that $U$ and $V$ strips don't intermix, so that the strip decomposition remains unambiguous.  To this end, we devised two novel strip-based atomic operations, whose purpose is to rearrange the connectivity of adjacent strips and align vertices that lie on different strips: 

\begin{enumerate}
    \item \emph{Collapsing and splitting.} Given a strip $S_i$, described as a sequence of quadrilaterals, we can either subdivide it by splitting it into two adjacent topologically identical strips (\figref{fig:editing_operations}b) or collapse it so that its immediate neighbors are now adjacent (\figref{fig:editing_operations}c). 
    \item \emph{Rewiring.} Given a strip $S_j$, we rewire its quads by switching the edges of all quads, either one vertex forward or backward on the strip (\figref{fig:editing_operations}d). As a result, all transversal strips that intersect $S_j$ are reconfigured.
\end{enumerate}


% Figure environment removed 


\paragraph{Alignment of a pair of vertices}
Assume we want to align the vertices $A$ and $B$ that lie on the two sides of a strip $S_i$ (\figref{fig:editing_operations}a) along the $U$ (blue) direction. This can be achieved in two ways: either by collapsing strip $S_i$  (\figref{fig:editing_operations}c) or by rewiring a \emph{transversal} (green) strip along the $V$ direction that runs between $A$ and $B$, such as strip $S_j$ (\figref{fig:editing_operations}d). For vertices further apart, there might be an entire decision graph of possible choices. To make a reasonable choice for a sequence of operations for alignment, it is important to consider:


\begin{itemize}
    \item Collapsing removes mesh faces. If it is applied on many strips, or on a strip that covers a large part of the mesh, for example, a winding strip, it can lead to considerable geometric distortion. 
    \item Rewiring can misalign other pairs of vertices that were already aligned, because it changes the connectivity of all strips intersected by the transversal rewired strip.
\end{itemize}

The alignment of vertices can be achieved along either the $U$ or the $V$ direction using the presented atomic operations. For brevity, in the following, we consider alignment along the $U$ (blue) direction, which we call \emph{dominant}, and we call the $V$ (green) direction \emph{subdominant}.


\subsubsection*{User-controlled topology editing}
Determining what edits must be carried out is a difficult combinatorial problem that is hard to automate fully. In addition, editing decisions have a large impact on the final aesthetics of the produced object. Therefore, it makes sense to delegate some of the combinatorial choices to the user. On the other hand, reasoning about the desired topology of a mesh geometry is a challenge, as the connectivity of strips can be too complex for a novice user to grasp or change manually. As a practical compromise, we abstract full control over mesh connectivity away from the user, and instead have them follow a short pipeline where they choose pairs of vertices they wish to align. Our algorithm then automatically generates the sequence of atomic operations that realizes the chosen alignment. 
The accompanying video shows the full process of aligning singularities on various shapes. 


% Figure environment removed


\paragraph{Control pipeline}

Every alignment of a pair of vertices entails the following steps.

\begin{enumerate}[label=(op\arabic*)] 

\item 
\label{editop:choose_vertices}
The user selects two vertices $v_1$, $v_2$ to be aligned along the dominant direction (\figref{fig:control_pipeline}a). {Note that we allow $v_1=v_2$, for example, the vertex $v$ in \figref{fig:strips_editing}a,b can be aligned with itself around the handle.}
\item 
\label{editop:choose_route}
The algorithm enumerates all routes from $v_1$ to $v_ 2$, and the user selects one route along which to align (\figref{fig:control_pipeline}b). The default choice is the route that has the smallest non-zero length.
\item 
\label{editop:choose_ops}
The algorithm points out the ``best'' next step, and the user confirms its application (\figref{fig:control_pipeline}c). This is repeated until the two vertices are aligned along the selected route.
\end{enumerate}


\begin{wrapfigure}{r}{0.2\textwidth}
    \vspace{-15pt}
    % \hspace{-15pt}
    % Figure removed
    \vspace{-40pt}
\end{wrapfigure}
\paragraph{Choosing vertices to be aligned \ref{editop:choose_vertices}}
The singularity separatrices and winding strips are highlighted as guides (see \figref{fig:control_pipeline}a and inset) to help the user choose which vertices it makes sense to align.


\paragraph{Choosing routes between vertices \ref{editop:choose_route}}
Having chosen two vertices, $v_1$ and $v_2$, we automatically enumerate all relevant routes  between them, the set of which we denote by $R$.  Each route $R_i \in R$ consists of a sequence of $U_{R_i}$ (dominant) and a sequence of $V_{R_i}$ (subdominant) strips, which we collect as follows:
\begin{enumerate}
    \item \emph{Subdominant sequences $\{V_{R_i}\}$:}  We start from the vertex  ($v_1$ or $v_2$) that has the higher valence, 
    and trace every incident dominant edge until we intersect a subdominant edge directly connected to the other vertex (\figref{fig:routes} top). This defines the route $R_i$, and the crossed subdominant strips comprise its subdominant sequence $V_{R_i}$. If another singularity or a boundary is reached instead (\figref{fig:routes}c), the route is discarded as irrelevant. Therefore, the number of possible routes is bounded by the dominant-edge valence of the first vertex. In \figref{fig:routes}, we have two relevant routes $R_a$ and $R_b$.
 
    \item \emph{Dominant sequences $\{U_{R_i}\}$:} For each relevant route $R_i$ found in the previous step, we find the shortest among the paths of dominant (blue) strips between $v_1, v_2$ that intersects the sequence of subdominant strips $V_{R_i}$ (\figref{fig:routes}d,e). This is then the dominant strip sequence $U_{R_i}$. We define the \emph{route distance} $d_{R_i}$ as the length of $U_{R_i}$; it specifies the number of operations that must be performed for the two vertices to  align along $R_i$. In \figref{fig:routes} we have $d_{R_a} = 0$, i.e. the vertices are already aligned, and $d_{R_b} = 1$, i.e. the vertices are one strip apart, and therefore require one operation to be aligned.
\end{enumerate}

In \figref{fig:routes}b,e the depicted route $R_b$ is the smallest non-zero length route chosen by default.  

% Figure environment removed


\paragraph{Iterative next best step \ref{editop:choose_ops}}
Having chosen a route $R_i$ along which $v_1$ and $v_2$ should be aligned, the algorithm iteratively brings them towards alignment, one strip at a time, by  either (a) rewiring a subdominant strip $V_{R_i} \in V_R$, or (b) collapsing a dominant strip $U_{R_i} \in U_R$. To choose the strip to be modified, we consider the following  ``fitness'' criteria applied to all strips $S \in \{U_{R_i}, V_{R_i}\}$:
\begin{itemize}
    \item The total area $A$ of the strip 
    $ A(S) = \sum_{q \in S} A(q)$
    measured as the sum of the areas of its quadrilaterals $\{q\}$.
    \item The total deviation from perfect ``squareness'' $D= \sum_{q \in S} D(q)$. We measure the squareness of a quad $q$ as
    $$D(q) = \sum_{i=1}^4{\left|a_i - \frac{\pi}{2}\right|} + \sum_{i=1}^4{\left|l_i - \bar l\right|}$$

    where $a_i$ is the $i$-th inner angle of the quad, $l_i$ is the $i$-th edge length and $\bar l$ is the average edge length of the quad.
    
\end{itemize}
The fitness score for $S$ is then $1/(A(S) + D(S))$, and we choose the strip of highest fitness. As a result, we end up editing small and well-shaped strips, which helps avoid geometric artifacts.

In the example of  \figref{fig:routes}(b,e), $v_1$ and $v_2$ are one dominant strip apart along route $R_b$, thus one operation is required. All strips of $R_b$ are nicely shaped with almost square quads. Therefore, their area is the deciding factor for picking a strip.
Since the blue strip is winding (\figref{fig:routes}e), it has a large area sum, therefore, its fitness is very low. As a result, the green strip with the smallest area is selected for rewiring, resulting in the alignment of $v_1$ and $v_2$ along that route (\figref{fig:routes}f). 

While the algorithm iteratively offers the next atomic operation in this step, we allow the user to approve every such iteration before it is applied. This is useful because the proposed next step might still not be a preferable operation to perform (e.g. if all strips of a route are winding). Then the user can abort and attempt to align another pair of vertices, which can, for example, unwind these strips before returning to the current route. 

In both examples \figref{fig:routes}f and \figref{fig:strips_editing}b, the alignment of singularities removes the winding strips. In \figref{fig:unwinding_strips}, we show how aligning a non-singular pair of vertices can remove winding strips. Aligning any two vertices that lie on the same side of a winding strip in two consecutive turns unwinds the strip. 


% Figure environment removed


\paragraph{Locking alignments}
Rewiring operations may misalign already aligned pairs of vertices. To avoid this, the user may select vertex pairs whose alignment should remain locked. In that case, we lock the rewiring of any subdominant strips that intersect a dominant alignment path between the two locked vertices, by setting their fitness to $0$. Collapse operations are never blocked, as they can only bring vertices together but not apart.  For example, on \figref{fig:strips_editing}c, the alignment between vertices $A$ and $B$ has been locked, so none of the subdominant strips (in dark green) that intersect the path between $A$ and $B$ can be rewired. As a result, to align $v_1$ and $v_2$ along either one of the two available routes ($R_x$ or $R_y$), only the collapse of the highlighted dominant (blue) strip  $S$ is possible.  

% Figure environment removed


\subsubsection*{Post-processing}
Once all operations are completed, the algorithm selectively subdivides all elongated strips, smooths local deformations, and projects the resulting mesh to the original mesh boundary and surface, as follows.


\begin{wrapfigure}{r}{0.25\textwidth}
    \vspace{-15pt}
    %   \hspace{-15pt}
    % Figure removed
    \vspace{-30pt}
\end{wrapfigure}
 \paragraph{Selective Subdivision} We define the \emph{elongation} of a strip by the ratio of lengths of edges,
$r = max(l_U,l_V) / min(l_U,l_V)$, where $l_U$ and $l_V$ are the total lengths of the strip's $U$ and $V$ edges respectively. If $r > 2$, we subdivide the strip $\text{round}(r)$ times to reduce the elongated direction (see inset). This is done iteratively until no strip is elongated.

\paragraph{Smoothing}
We further reduce the geometric distortion by applying iterative implicit Laplacian smoothing on the vertices $v \in \VV_\QQQ$ by
$(I + dt L_\QQQ) v^{t+1} = v^t$
(\figref{fig:smoothing-boundaries_editing}). Each smoothing iteration consists of the following two steps:
\begin{enumerate}
    \item Smoothen the boundary of the quad mesh $\partial \MM_\QQQ$ as an independent curve and project it onto the original boundary $\partial \MM$.
    \item Smoothen $\MM_\QQQ$ keeping boundary vertices fixed and project it onto the original surface $\MM$.
\end{enumerate}
For the boundary smoothing we use $dt = 0.5$, and for the mesh smoothing, we use $dt = 0.001/\lambda$, where $\lambda$ is the smallest non-zero eigenvalue of the uniform quad-mesh Laplacian $L_\QQQ$.


% Figure environment removed





\subsection{Topological Partitioning}
\label{subsection:patitioning}

The topological partitioning of the resulting SDQ mesh into singularity-free strip sub-networks is useful for all sequential fabrication processes, as it allows to determine an ordering of the resulting strips up to orientation. For example this is useful for deciding the assembly sequence of a surface made of flat sheets, or a print sequence of a series of print paths for 3D printing. 

Once a good patch layout has been achieved with desired alignments of singularities, separating the strip network into good patches is trivial. As we have two transversal strip networks, the user must select along which network the cuts should be aligned. In the following we assume that partitioning happens along the $U$ direction, meaning that cuts run along the $U$ edges (except close to singularities with valence 2) and cut transversally the $V$ strips.   

To create such a partitioning, we make cuts along the $U$ separatrices of singularities (\figref{fig:singularities}). 
However, for singularities with valence 2 (\figref{fig:singularities} left), only cutting the separatrix in the $U$ direction does not separate the surface. In this case, we cut ``through'' the singularity along both the $U$ and $V$ directions.

% Figure environment removed

Having cut away all singularities, we also cut along short topological handles to obtain simply-connected networks where no singularity or separatrice lies in the interior of the patch. 
\figref{fig:full_editing_operations_pipeline_2} shows results of editing and topological partitioning on more complex shapes.

% Figure environment removed
\section{Results}
\label{results_section}


To answer our research questions and confirm or reject our hypotheses, we conduct rigorous statistical and qualitative analyses. We measure appropriate reliance based on metrics defined in previous work: Relative AI reliance (RAIR) and relative self-reliance (RSR)~\cite{schemmer2023appropriate}. By doing so, we answer \textbf{RQ} \ref{rq1} and \textbf{RQ} \ref{rq2} in \Cref{section_ar}. We also calculate the participant's task accuracy before and after receiving AI and XAI advice to answer \textbf{RQ} \ref{rq3} (see \Cref{hai_performance}). This way, we can determine whether complementary team performance exists~\cite{hemmer2021human}. Additionally, based on the new metric, Deception of Reliance (DoR), we measure to what extent explanations deceive humans, answering \textbf{RQ} \ref{rq4} in \Cref{deception_section}. Lastly, we qualitatively analyze open-ended responses through rigorous inductive content analysis in \Cref{qual-res}. For all of our research questions, we look at two different types of explanations: natural language explanations that are focused on specific features present in the image and visual, example-based explanations showing the top three most similar example images from the training set. While our analyses look at both modalities, we do not intend to compare them directly. Therefore, we do not conclude one modality is better or worse than the other.

\subsection{Participant Statistics}
On average, the study takes $24$ minutes to complete.
In order to distinguish experts from non-experts, we perform K-means clustering ($k=2$) based on a principal component analysis with two components for four features from the bird species identification test (part A of \cref{experimentdesign}). These four features represent participants' scores in correctly identifying the family and species of the easy and the difficult bird images. By clustering the $136$ participants into the expert and non-expert group, we end up with $83$ experts and $53$ non-experts. With this clustering, the average bird identification test score (summing up all four scores in the identification test) for non-experts is $38.99\% (STD = 11.42\%)$ while the average test score for experts is $83.84\% (STD = 12.30\%)$\footnote{Participants performance on the bird identification test is shown in \cref{test-scores}, Appendix Section \ref{bird-test-details}.}. Of the $83$ experts, $42$ see example-based explanations, and $41$ see natural language explanations. Of the $53$ non-experts, $25$ see example-based explanations, and $28$ see natural language explanations. In terms of the fields that the $136$ participants represent, $45$ participants have an occupation primarily related to biology, conservation, and/or the environment. $26$ have an occupation primarily related to engineering and/or technology; $30$ are either researchers, students, or affiliated with education in some other way; $24$ have occupations in miscellaneous industries; and $11$ are retired. 


\subsection{Moderating Effects in Imperfect XAI Research Model}
\label{section_ar}

In order to test whether humans' level of expertise and the explanations' assertiveness moderate the relation of the correctness of explanations on humans' appropriate reliance, we conduct several moderation analyses utilizing the process macro model of \citet{hayes2017introduction}. 
An overview of the regression analyses is presented in \Cref{mod_anal}.
% on p. \pageref{mod_anal}.

\begin{table}[htbp!]

\caption{Moderation analyses of the correctness of natural language and example-based explanations on RAIR and RSR with the level of expertise and assertiveness as moderators. The coding of assertiveness used for the moderation analyses is provided. }

\begin{tabular}{P{2cm} P{1cm} P{1cm}}
\multicolumn{3}{c}{Coding of assertiveness} \\

\hline

assertiveness &  Z1 & Z2 \\ \hline \hline

\textit{neutral} & 0 & 0 \\  \hline
\textit{non-assertive} & 1 & 0 \\  \hline
\textit{assertive} & 0 & 1 \\  \hline
\\\\ \end{tabular}


\begin{threeparttable}


\begin{tabular}{m{1.5cm}R{0.9cm} R{0.9cm} R{0.002cm} R{0.9cm} R{0.9cm} R{0.002cm} R{0.9cm} R{0.9cm} R{0.002cm} R{0.9cm} R{0.9cm}} \hline
& \multicolumn{5}{c}{RAIR} && \multicolumn{5}{c}{RSR}\\
\cmidrule{2-6} \cmidrule{8-12}
& \multicolumn{2}{c}{Natural Language} &&  \multicolumn{2}{c}{Example-Based} & & \multicolumn{2}{c}{Natural Language} & & \multicolumn{2}{c}{Example-Based}  \\
\cmidrule{2-3} \cmidrule{5-6} \cmidrule{8-9} \cmidrule{11-12}
& coeff & p&  & coeff & p & & coeff & p&  & coeff & p \\
\hline \hline
const   & 1.26   & .00 & & .43 & .17 & & -17.16   & .98  & & -3.60  & .00    \\\hline
corr   & .57   & .29 & & 1.02 & .04 & & 13.25   & .98  & & -.4.36  & .74    \\\hline
exp   & 2.12  & .00 &&  -1.25 & .00 &  & 15.89   & .98 &  & 3.25  & .00    \\\hline
Z1   & -.46 & .24 &&  .26 & .47 &&  .14   & .26  & & -.09  & .83    \\\hline
Z2   & .00  & 1.00 &&  .00 & 1.00 & & -.31   & .58  & & .09  & .83    \\\hline 
exp x corr   & -1.00  & \best{.05} & & -1.04 & \best{.03} & & -13.33  & .98  & & -.73  & .57    \\\hline
Z1 x corr   & .46  & .44 &&  -.03 & .95 & & -.28   & .71  & & -.45  & .55    \\\hline
Z2 x corr   & .19  & .74 &&  -.16 & .77 & & .90   & .23  & & -.43  & .55   \\\hline
\end{tabular}
    \begin{tablenotes}
        \item[1] \textit{corr} --- \textit{correctness}; \textit{exp} --- \textit{level of expertise}
    \end{tablenotes}
    \end{threeparttable}

\label{mod_anal}

\end{table}


\subsubsection{Participants' level of expertise moderates the effect of the correctness of explanations on RAIR for natural-language explanations}
\label{mod_analysis_nle_rair}
As theoretically developed in Section \ref{theoretical_section}, we model the correctness of explanations as an independent variable. Accordingly, we model RAIR as the dependent variable. To account for the moderation effect of the level of expertise and assertiveness, we examine each variable as a moderator and report the interaction effects with the correctness of explanations. The results of this moderation analysis are shown in \Cref{mod_anal} (a detailed view is shown in \Cref{mod_anal_nle_rair} on p. \pageref{mod_anal_nle_rair} in the \Cref{mod_appendix}).

The moderation analysis shows that the interaction of the level of expertise with the correctness of explanations is significant (coeff $= -1.00$, p-value $= .05$). We observe a negative coefficient. Accordingly, the moderation effect on the relation of correctness on RAIR is higher for non-experts than for experts. \kmedit{In other words, non-experts change their initially incorrect decision to align with the correct AI advice more often than experts do when the natural language explanation is correct.} However, there is no significant effect in the interaction of assertiveness and the correctness of explanations. Thus, we conduct a regression analysis with the moderators as independent variables to evaluate for a direct effect of assertiveness as recommended by \citet{hayes2017introduction} and \citet{warner2012applied}. The results of the regression analysis show that there is no direct effect between assertiveness and RAIR (coeff $= .04$, p-value $= .77$). Thus, we confirm hypotheses \ref{hyp1} and \ref{hyp3} and reject hypothesis \ref{hyp5} for natural language explanations.


\subsubsection{Participants' level of expertise moderates the effect of the correctness of explanations on RAIR for example-based explanations}

Next, we present the moderation analysis of example-based explanations on RAIR. We set up the analysis for example-based explanations the same as the analyses for natural language explanations (see \Cref{mod_analysis_nle_rair}). As seen in \Cref{mod_anal}, there is a significant moderation effect of the level of expertise (coeff $= -1.04$, p-value $= .03$). The negative coefficient signals that this moderation is higher for non-experts than for experts. The correctness of the example-based explanations has a positive coefficient (coeff $= 1.02$, p-value $= .04$), and thus, correct explanations have a positive impact on RAIR. \psedit{Thus, if participants are provided with a correct explanation, they more often correctly follow the AI advice.} \kmedit{Overall, correct explanations result in humans changing their initially incorrect decisions to align with the correct AI advice more often, and this is especially prevalent among non-experts.}

Furthermore, the analysis reveals that assertiveness does not moderate the effect between correctness and RAIR. According to \citet{hayes2017introduction} and \citet{warner2012applied}, we drop the interaction term and conduct a regression analysis with assertiveness set as the independent variable. The result shows that there is no significant direct effect of assertiveness on RAIR (coeff $= -.04$, p-value $= .79$).
Hence, we confirm hypotheses \ref{hyp1} and \ref{hyp3} and reject hypothesis \ref{hyp5} for example-based explanations.

\subsubsection{Participant's level of expertise has a direct effect on RSR for natural language explanations}
\label{mod_analysis_nle_rsr}
In addition to analyzing whether the level of expertise and assertiveness moderate the effect of explanations' correctness on RAIR, we conduct the same analyses for the effect of explanations' correctness on RSR. For RSR, we look at all cases in which the AI prediction is giving incorrect advice (i.e., the prediction is wrong) and the initial human decision is correct \cite{schemmer2023appropriate}. 
% This next sentence seems to be out of place so I'm commenting it out for now.
% As outlined in \Cref{methodology}, correct explanations reflect the scenario in which the explanation is describing the AI predicted class.

% In this subsection, we report the moderation analysis of natural language explanations on RSR. The results can be seen . 
The moderation analysis in \Cref{mod_anal} for the natural language explanation shows that there is no significant effect of correctness on RSR moderated by level of expertise (coeff $= -13.33$, p-value $= .98$) and assertiveness (Z1 x corr.: coeff $= -.28$, p-value $= .71$; Z2 x corr.: coeff $= .90$, p-value $= .23$). 

Thus, we perform a regression analysis with the level of expertise and assertiveness as independent variables and drop the interaction terms. We observe that there is no significant effect of assertiveness on RSR (coeff $= .10$, p-value $= .58$). However, the level of expertise (coeff $= 3.18$, p-value $= .00$) has a significant effect on RSR. 
% \psedit{Thus, expert participants correctly follow the AI advice more often than non-expert participants.} 
\kmedit{With a positive coefficient, this means that experts dismiss incorrect AI advice more than non-experts when shown natural language explanations.}
\psdelete{The trend} \psedit{This can also be seen} in \cref{rsr-rair}\psedit{, which} tells us that experts have a higher RSR than non-experts. 
% This can also be seen in \Cref{rsr-rair}. 
Therefore, we reject hypothesis \ref{hyp2}, and additionally, hypothesis \ref{hyp4} as the level of expertise does not have a moderating role but has a direct effect on RSR for natural language explanations. On top of that, we reject hypothesis \ref{hyp6}.


\subsubsection{The correctness of explanations and participants' level of expertise have a direct effect on RSR for example-based explanations}

Lastly, we report the results of the moderation analysis for example-based explanations on RSR. The analysis is set up the same as it is in \Cref{mod_analysis_nle_rsr} but for the example-based explanations. 
% \Cref{mod_anal_ex_rsr} displays the results of this regression analysis.

We can see in \Cref{mod_anal} that the level of expertise does not significantly moderate the effect of correctness on RSR (coeff = -.73, p-value = .57).
Additionally, there is no significant moderation of assertiveness (Z1 x corr.: coeff $= -.45$, p-value $= .55$; Z2 x corr.: coeff $= -.43$, p-value $= .55$). Thus, we conduct a regression analysis without the interaction terms. The results of this analysis show no direct effect of assertiveness on RSR (coeff $= -.03$, p-value $= .86$). However, there is a significant direct effect of explanations' correctness on RSR (coeff $= -1.40$, p-value $= .00$) and a direct effect of level of expertise on RSR (coeff $= 3.05$, p-value $= 0.00$). \psedit{This means that experts more often correctly override the wrong AI advice and stick to their correct initial decision compared to non-experts for example-based explanations. Additionally, when the explanations are correct, participants more often correctly override wrong AI advice and stick to their correct initial decision.} Hence, experts have a higher RSR than non-experts, which can also be seen in \Cref{rsr-rair}. Moreover, as incorrect explanations have a higher impact on RSR, experts are able to identify false AI advice for incorrect explanations better. This also shows that experts are able to identify incorrect AI advice to a greater extent than non-experts; experts, in this case, rely more heavily on their own judgment.

Thus, we confirm hypothesis \ref{hyp2} and reject hypothesis \ref{hyp4} as the level of expertise does not take in a moderating role but has a direct effect on RSR for example-based explanations. On top of that, we reject hypothesis \ref{hyp6}.
\\
\\
Overall, the moderation analyses reveal that the level of expertise moderates the effect of the correctness of explanations on RAIR for both explanation modalities. Additionally, the analyses show that the level of expertise has a direct effect on RSR for both explanation modalities, and the correctness of explanations has a direct effect on RSR for example-based explanations.


\subsection{Human-AI Team Performance}
\label{hai_performance}
Hemmer et al. argue that interpretability is a key component of human-AI complementarity~\cite{hemmer2021human}. Several previous user studies have failed to show that incorporating XAI into AI systems can lead to CTP~\cite{fok2023search}. However, with a new dimension of XAI advice in \cref{reliance-model}, we can contribute to the current literature by investigating how the correctness of explanations affects CTP. By calculating the participants' performance before and after seeing the AI and XAI advice, we can determine whether CTP exists in the presence of imperfect XAI. 
As the analyses in \Cref{section_ar} reveal, the level of expertise impacts appropriate reliance in terms of RSR and RAIR. Thus, in comparing the human-AI team performance, we distinguish by participants' level of expertise. We use accuracy as the performance metric. \Cref{reliance-counts} presents the performance of AI and humans for each treatment.

% Figure environment removed

The AI's performance is always $50\%$ because the study was designed to show participants six birds that the model correctly classified and six that the model incorrectly classified. In \Cref{reliance-counts}, we see that when experts are paired with the AI, their performance improves by $8.74\%$ for the natural language modality and $9.53\%$ for the example-based modality. When experts are paired with AI, they perform $6.91\%$ better than the AI alone for natural language explanations and $5.36\%$ for example-based explanations.

While experts reach CTP, we do not see this for non-experts. However, we do see that the non-experts greatly improve their performance and nearly match the AI's performance when paired with the AI. Specifically, non-expert participants who see the natural language explanations improve their performance by $39.58\%$ (task accuracy of $45.83\%$), while non-expert participants who see the example-based explanations improve their performance by $34.67\%$ (task accuracy of $43.00\%$) when paired with the AI.

We can separate \cref{reliance-counts} into correct and incorrect explanations. When we only consider cases with correct explanations (\cref{hai-correct} in \cref{hai-appendix}), the non-experts' task accuracy is approximately the same as the AI alone: $48.81\%$ for natural language explanations and $49.33\%$ for example-based explanations. Experts reach CTP in both modalities. When only considering incorrect explanations (\cref{hai-incorrect} in \cref{hai-appendix}), we still see complementary team performance for the experts. However, the non-experts' task accuracy suffers more when shown incorrect explanations. Non-experts' task accuracy for natural language explanations is $42.86\%$ and $36.67\%$ for example-based explanations.

Additionally, we calculate two-sample t-tests to assess whether the trends in \cref{reliance-counts} are significant. The team performance of experts and AI is significantly higher than the team performance of non-experts and AI in both explanation modalities (natural language: p-value $= 0.00$, example-based: p-value $= 0.00$). Furthermore, we also compare the performance for correct and incorrect explanations. Here, we see the same results: experts achieve a significantly higher team performance than non-experts (correct explanations --- natural language: p-value = 0.00, example-based: p-value $= 0.01$; incorrect explanations --- natural language: p-value $= 0.00$, example-based: p-value $= 0.00$). 


\subsection{Deception caused by Imperfect XAI}
\label{deception_section}


In \Cref{rsr-rair}, we compare RAIR to RSR for both levels of expertise and the correctness of explanations. We show this comparison for example-based explanations (the graph on the left side of \Cref{rsr-rair}) and natural language explanations (the graph on the right side of \Cref{rsr-rair}). By measuring RAIR and RSR for incorrect and correct explanations separately, we can calculate the deception caused by imperfect XAI (refer to \cref{DAIR_aor_eq} on p. \pageref{DAIR_aor_eq}). We do not visualize assertiveness since we do not see any significant direct or moderation effects. 

%- for experts in nle: explanation is helping experts to determine why it is actually wrong
% - for example-based: incorrect ones are differing (potentially) but for correct ones its always the same thing so that's why people are switching and not based on the information content
% - we see the same thing for non-experts: the gap for RAIR is way higher 
% - being interpreted as cofnidence score (add this thought to the discussion section for this)

% when experts see nle - they are able to distinguish when the AI prediction is wrong for correct explanations because it is clear that the characteristics identified don't match 

% for ex -- RSR is dropping -- experts are being misled by the visuals for correct explanations, which is just reinforcing the AI advice  --- but showing visual explanations that are incorrect it is obvious to experts for when the AI is wrong
% Even non-experts can tell if it is three of the same birds or not 

% Assertiveness is playing out visually more than natural language explanations (add this point to the discussion section)


% Figure environment removed

The figure shows that experts have a higher RSR than non-experts for both incorrect and correct explanations across both explanation modalities, validating that experts rely more on their own initial decisions when AI advice is given. The most striking result that emerges from the data is that for example-based explanations, we observe that experts have a significantly higher RSR for incorrect explanations (RSR $= 0.57$) than correct explanations (RSR $= 0.29$), resulting in a negative $DoR_{RSR}$ of $-0.28$ (p-value $= 0.00$).
% A negative difference in deception means that correct example-based explanations are more deceptive than incorrect explanations when the AI advice is incorrect. 
As a result, experts are falsely relying on the AI advice when provided with correct example-based explanations\footnote{Note that correct example-based explanations are consistent in showing three images of the predicted class. Incorrect example-based explanations represent three images that do not correspond to the predicted class of the AI. Moreover, the examples shown are not consistent with the bird species displayed in 90\% of the \textit{correct advice, incorrect explanation} cases and in 40\% of the \textit{incorrect advice, incorrect explanation} cases in our study.}. This means that experts are prone to being misled by correct explanations when the AI advice is incorrect.
However, we do not see this trend for natural language explanations. Here, there is a positive $DoR_{RSR}$ of 0.09, which is not significant (p-value $= 0.41$).
For example-based explanations, the $DoR_{RAIR}$ is positive, meaning that experts rightly follow correct AI advice more often when provided with correct explanations than with incorrect explanations. The data shows a weak, significant positive deception of reliance for example-based explanations ($DoR_{RAIR} = 0.16$, p-value $= 0.10$). Similarly to the RSR cases, for the RAIR cases, the experts are provided with three consistent examples for correct explanations that represent the AI's correctly predicted bird species. The incorrectly provided explanations represent three images that can be inconsistent in the bird species. Thus, experts are deceived by such incorrect explanations even though the AI advice is correct.

Non-experts have, in both modalities, a similar $DoR_{RSR}$ indicating no significant difference in their RSR between correct and incorrect explanations. However, non-experts follow the correct AI advice for correct example-based explanations more often than for incorrect example-based explanations (significant with p-value $= 0.01$). For the latter, the three examples can show inconsistent bird specie(s) that are different from the ground truth of the shown image. Thus, the $DoR_{RAIR}$ for non-experts is at $0.26$. Interestingly, for natural language explanations, the incorrect explanations are not misleading as much ($DoR_{RAIR} = 0.03$, not significant, with a p-value $= 0.68$). This means that non-experts are not misled by incorrect explanations in natural language as much as by visual, example-based explanations. In general, non-experts have a higher RAIR than experts.

Overall, participants have a higher $DoR(RAIR, RSR)$  for example-based explanations (experts: $DoR(RAIR, RSR) = 0.32$; non-experts: $DoR(RAIR, RSR) = 0.26$) than for natural language explanations(experts: $DoR(RAIR, RSR) = 0.11$; non-experts: $DoR(RAIR, RSR) = 0.06$). This means that especially the correctness of example-based explanations has an impact on humans' decision-making behavior.

\subsection{Designing for Imperfect XAI} \label{qual-res}

At the end of the bird identification task, we ask participants: ``\textit{Under what circumstances would you prefer assertive (e.g.,
``definitely'', ``clearly'') versus non-assertive (e.g., ``might be'', ``appears to be'') versus neutral explanations
and why?}''. With imperfect XAI existing in human-AI collaborations, it is necessary not only to understand quantitatively how it impacts decision-makers but also qualitatively. Even though we do not observe the level of assertiveness to have a direct effect or a moderation effect on appropriate reliance, it is still valuable to analyze participants' preferences when it comes to the tone of explanations.

% Figure environment removed

Through inductive content analysis of participants' responses to this question, we derive two dimensions that researchers and designers should consider when developing and evaluating imperfect XAI in human-AI collaborations: \textit{AI Behaviors} and the \textit{Impact on Human-AI Teams}. Each dimension is made up of four themes that are derived from concepts that emerge in the responses, shown in \cref{qual-data}. We highlight those themes in bold. $25\%$ ($34$ participants) of the responses either do not provide reasoning for their opinion or do not answer the question such that it could be grouped into one of the eight themes we identify. We provide quotes from participants for each theme to structure the aggregated dimensions and shape our insights on designing for imperfect XAI. 


\subsubsection{Aggregated Dimension: AI Behaviors}

$54$ out of the $136$ participants answer the survey question with comments relating to the first aggregated dimension: AI Behavior. Participants rationalize that the AI's behavior determines when they prefer assertive, non-assertive, and neutral explanations. Within the AI Behavior dimension, four themes emerge from the participants' comments, such as the model's overall performance, whether the model's prediction is correct or not, the model's confidence in individual predictions, and the correctness/quality of the explanation for a given prediction. 

While only $6$ out of the $136$ participants make comments about the \textbf{model's performance}, it still provides interesting insight that should be considered. Instead of looking at the individual prediction level, these participants focus on the global performance of the model. One participant states that if developers find their model ``\textit{... to be 90\% accurate in your testing, use more definite language, but if it’s not there yet, consider making the tone more neutral and put more responsibility with the end user to interpret the field marks ...}''.

Looking at the individual prediction level, $8$ out of $136$ participants comment on the \textbf{correctness of the AI's prediction}. For example, one participant says that ``\textit{... a non-assertive response would be preferred since the AI selections were incorrect ...}'' while another participant says, ``\textit{I would prefer the assertive language to be accompanied by correct identifications}''.

Considering individual predictions on a more granular level, many participants ($31$ out of $136$) make comments related to \textbf{the confidence of the AI's prediction}. These participants comment on how this factor could be used to determine the tone of the explanation. Specifically, participants, ``\textit{... would prefer the level of assertiveness to depend on the level of confidence of the answer given by the AI}''. One participant expands upon that sentiment by specifying when non-assertive versus assertive tones should be used: ``\textit{I would prefer assertive sentences when the probability of the AI model is very high, while I would prefer non-assertive when the probability is very close to other classes of the model}''. 

$9$ out of $136$ participants make comments related to the \textbf{correctness and quality of the AI explanations}. One participant who sees example-based explanations comments on how some of the examples are incorrect and do not show the correct species. They use this specific situation to rationalize when they would prefer the tone of explanations to be assertive versus non-assertive:  
``\textit{I would prefer assertive explanations if the `similar' photos were actually of the correct species and if the explanation confirmed this. Otherwise, non-assertive explanations are more helpful}''. Another participant who makes a comment related to this theme agrees that assertive language should be used, ``\textit{when all of the reference pictures match up and there are no other similar-looking species}''. 
% Another participant shared a similar sentiment on using a non-assertive tone when the explanations are incorrect, ``\textit{I would prefer non-assertive explanations nearly every time because I didn't have confidence in the AI's explanations (and a few were incorrect)}''.
% When all of the reference pictures match up and there are no other similar looking species it should be assertive.

One participant who sees the natural language explanations comments on the quality and detail of the explanation being a factor to use when determining the tone of the explanation, ``\textit{If the bird description [natural language explanation] is very generic, i.e., brown wings, gray body, or yellow beak (traits that correspond to many birds), I'd rather the AI appear more cautious in its judgment and use non-assertive explanations. However, if the bird has some standout characteristic that the AI correctly identifies [through the natural language explanation], i.e., bright yellow body or red-tipped wings, etc., then assertive explanations seem more convincing}''.


\subsubsection{Aggregated Dimension: Impact on Human-AI Team} 

$48$ out of the $136$ participants answer this survey question with comments related to the human-AI team, such as the confidence and knowledge of the decision-maker, impact on the decision-maker, unambiguous input data, and characteristics of the input data. 

$10$ of $136$ prefer the level of assertiveness to align with their \textbf{own confidence level} and knowledge of the domain. For example, one participant says, ``\textit{I would prefer more assertive explanations when I don't feel very sure about my choice}''. Another participant adds that they would prefer assertive explanations if they ``\textit{... didn't know anything about the topic in question ...}''. However, one participant says, ``\textit{I would prefer neutral explanations if I'm unsure of the species and assertive if I'm confident in my identification}''.

% ``\textit{I would prefer assertive answers if I didn't know anything about the topic in question but would be satisfied with non-assertive if the answer was to back up any doubts of my own answer}''.

% ``\textit{In situations where I am not confident in my knowledge at all, I would prefer assertive explanations so that it would help me definitely understand the subject...I would like non-assertive and neutral explanations when I am somewhat confident, and I must make a decision based on my own knowledge mixed with some additional feedback}''.

$15$ out of the $136$ make comments related to the \textbf{impacts on the decision-maker}. Some comments consist of concerns related to being misled and over- or under-relying on the AI, while other comments motivate the benefits of having assertive explanations. For example, one participant states that ``\textit{Assertive words create more security while changing your opinion or trying to gain knowledge}''. Another participant who shares the same sentiment said, ``\textit{I would prefer assertive explanations because it would make me feel more secure about the answer}''.
However, some participants do not share the same sentiment about assertive explanations and pointed out the potential consequences of them: ``\textit{Assertive AI explanations were given for incorrect identifications, which would mislead users}''. Given that potential to be misled, another participant rationalizes they ``\textit{... would prefer non-assertive explanations because I [they] do not fully trust AI with bird ID just yet}''.

$17$ out of the $136$ participants rationalize that the assertiveness of explanations should be based on how \textbf{ambiguous the input} is for a given prediction. For example, one participant brings up the quality of the input image and the difficulty of the bird ID as a way to determine whether explanations should be assertive or not: ``\textit{When it comes to less distinctive IDs like most sparrows, or harder to ID circumstances like winter or females or juveniles, or situations with weird lighting or harder angles it makes sense to use non-assertive.}''. On a similar line of thought, another participant says, ``\textit{I would prefer assertive explanations for birds that have distinctive traits over similar bird species, and non-assertive or neutral explanations for birds that are similar with characteristics that are more difficult to tell apart}''.
% ``\textit{I would prefer assertive if the photo is clear and able to easily recognize most of the markings of the bird.  I would prefer non-assertive if the markings are not clear. The same applies to neutral explanations}''.

% ``Assertive makes most sense when used for very clear examples of distinctive birds, like the Magnolia Warbler or Cerulean Warbler I had in this set. When it comes to less distinctive ID’s like most sparrows, or harder to ID circumstances like winter or females or juveniles, or situations with weird lighting or harder angles it makes sense to use non-assertive. Many beginning and casual birders take ID’s from Merlin to be absolute so we need more birding tools with nuance.'' 

$6$ out of the $136$ participants commented on how the use of assertive explanations helped them realize various \textbf{characteristics of the birds}. For example, one participant values the assertive tone because it is ``\textit{... helpful for pointing out distinctive features that would help ID the bird.}''. They also think that ``\textit{... the non-assertive language was helpful for species that share similar characteristics (aka clear, non-streaked breast) with other species that share that characteristic}''.
\section{Limitations and Future Work}
\label{sec:Conclusion}

Our pipeline successfully produces quality SDQ meshes for various shapes in a semi-automatic manner that can be controlled and edited by the user. We next note some limitations and future directions.


\paragraph{Editing operations}
The topology editing operations are essential in promoting a good patch layout of the produced meshes. However, there is no guarantee that a ``good'' set of topological operations always exists for unwinding all strips and producing the desired alignments. For example, applying operations on winding strips can cause major distortion, and a mesh might even consist \emph{only} of winding strips. Another example of an undesirable solution is when all rewiring operations are blocked to maintain specific alignments, and only collapse operations on long winding strips are available. The worst-case scenario is when the only available operation is the collapse of a winding strip that covers the entire shape, which reduces the mesh to a single point, thus canceling any potential benefits of the operation. In particular, surfaces without open boundaries where strips can terminate are more likely to exhibit many winding strips.

A potential solution could be to enforce alignment already on the stage of seamless parameterization. However, this comes with the challenge of automatically deciding which singularities should be aligned, which is a difficult combinatorial problem that might result in large integration errors.

In addition, post-editing smoothing operations increase the alignment error and can create distortions close to the boundaries of shapes, especially at non-quad faces. A large number of editing operations that require a lot of smoothing afterward can lead to considerable irregularities on the boundaries. 

\paragraph{Mesh quality depends on user input}
Unanticipated inputs (e.g., shapes without open boundaries or directional constraints that produce highly noisy fields) may lead to infeasible, visually unappealing, or impractical results. Typically, this would mean many singularities, a bad patch layout, or very noisy strip networks. In addition, if the scale of the parametrization, which the user chooses, is too low, then the quads resolution might not be sufficient to capture the input surface's features. 

\paragraph{Fabrication-related properties} We are optimizing our strip networks for a selection of desirable fabrication-related properties that we have found most commonly at the state of the art. However, various other properties can be embedded in the optimization, according to the fabrication scenario that the SDQ mesh is produced for. For example, alignment of strips so that their boundaries are approximately geodesic curves is useful for weaving surfaces out of ribbons that bend more readily out-of-plane than in-plane \cite{vekhter2019_weaving-geodesic-foliations}.

\paragraph{Alignment presets} We propose three alignment presets, namely alignment to curvature, boundaries, or user-drawn directions. Another useful option would be alignment to principal stress directions or principal moments given a chosen load case, which can be instrumental for improving the structural properties of an object in various fabrication scenarios \cite{Schiftner_2010_Statics-Sensitive_Layout_of_Planar_Quadrilateral_Meshes, fang2020_reinforced-fdm:-multi-axis-filament-alignment}. 




%% The Appendices part is started with the command \appendix;
%% appendix sections are then done as normal sections
%% \appendix


 \bibliographystyle{elsarticle-harv} 
 \bibliography{bibliography}

%% else use the following coding to input the bibitems directly in the
%% TeX file.

\begin{thebibliography}{00}

%% \bibitem[Author(year)]{label}
%% Text of bibliographic item

% \bibitem[ ()]{}

\end{thebibliography}
\end{document}

\endinput
%%
%% End of file `elsarticle-template-harv.tex'.
