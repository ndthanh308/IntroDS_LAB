%!TEX root = ../main.tex

\section{Theoretical Results}
% One of two ways:
% give learning theory result first
% then analysis of minimal set
% then active learning
% OR
% 
% them move all proofs out -- gives us a sense of how long the draft is
% 
% don't so subsubsubsubsubsections 
% 
% We first provide a learning-theoretic result that captures the tradeoff between training set sample complexity, label space richness, and model dimension. %example in which we reduce the problem of predicting unobserved classes to that of binary classification. 
% Next, we characterize when it is possible to predict any class in the metric space using an optimally chosen subset of classes.
% Finally, we demonstrate an active learning strategy for selecting the optimal next class. All proofs are deferred to the Appendix.

\paragraph{Challenges and Opportunities}
The $\argmax$ of per-class model probabilities is a ubiquitous component of classification pipelines in machine learning. 
In order to predict unobserved classes using metric space information, $\loki$ replaces this standard component. 
As a simple but significant change to standard pipelines, $\loki$ opens up a new area for fundamental questions. 
There are three main flavors of theoretical questions that arise in this setting:
\begin{enumerate}
    \item How does the performance of $\loki$ change as a function of the number of samples? 
    \item What minimal sets of observed classes are required to predict any class in the metric space? 
    \item How can we acquire new classes that maximize the total number of classes we can predict? 
\end{enumerate}
Excitingly, we provide a comprehensive answer to these questions for the most common metric spaces used in practice.
First, we provide a general error bound in terms of the number of samples, observed classes, problem dimension, and the diameter of the metric space, that holds for any finite metric space. 
Second, we characterize the sets of observed classes that are required to enable prediction of any class, and show how this set differs for various types of metric spaces of interest. 
Finally, we provide an active learning algorithm for selecting additional classes to observe so as to maximize the number of classes that can be predicted and we characterize the types of metric spaces for which the locus can be computed efficiently. 
These results provide a strong theoretical grounding for $\loki$. %n exhaustive set of answers to the space of theoretical questions about $\loki$. 


%In this section, we address these questions for a variety of different metric spaces, including trees, phylogenetic trees, grid graphs, and the trivial case of the complete graph. 

%... We address all of them and comprehensively solve the entire theory space. 


%We provide a comprehensive theoretical analysis of $\loki$. 




%!TEX root = ../../main.tex

% \subsection{Learning Reduction to One Binary Classifier}
% What is the interplay between predicting unobserved classes based on metric space information and standard learning-theoretic notions like sample complexity? 
% Our first result illustrates this tradeoff for a simple setting.

% Suppose that we wish to perform $m$-class classification where the label space is endowed with a metric induced by the path graph. This problem is effectively ordinal classification---however, in contrast to standard approaches which train $m-1$ binary classifiers, we would like to do so using \textit{only a single binary classifier}. 

% \begin{lemma}[Reduction to one binary classifier]
% \label{lemma:learning}
% Let $G=(\set{V}, \set{E})$ with $\set{V} = \{V_i\}_{i=1}^m$ and $\set{E}=\{(V_i, V_{i+1})\}_{i=1}^{m-1}$ be a path graph. 
% Let $\set{Y}=\set{V}$ and let $\Lambda = \{V_1, V_m\}$ be the set of observed classes. 
% Under the following model: 
% \begin{align*}
%     \mathbb{P}(y=V_1 \ | \ x \text{ and } y \in \Lambda) =: \mathbb{P}_1 &= \frac{1}{1 + \exp\{-x^\top\beta\}}\\ 
%     &= \frac{d(V_\star, V_m)}{d(V_1, V_m)},
% \end{align*}
% \[ \mathbb{P}(y=V_m \ | \ x \text{ and } y \in \Lambda) =: \mathbb{P}_m = 1 - \mathbb{P}_1,\]
% where we have access to $\widehat{\mathbb{P}}_1$ and $\widehat{\mathbb{P}}_m = 1 - \widehat{\mathbb{P}}_1$, the sample complexity of estimating an intermediate unobserved class along the path, using $\loki$, is $O(d m^2)$. 
% \end{lemma}

% This means that the number of samples required to estimate $\mathbb{P}_1$ accurately enough to predict an unobserved class scales with $m^2$---quadratic in the number of intermediate unobserved nodes between $V_1$ and $V_m$. 

\subsection{Sample Complexity}
What is the interplay between predicting unobserved classes based on metric space information and standard learning-theoretic notions like the sample complexity needed to train a model? 
Our first result illustrates this tradeoff, and relates it to the squared distance loss. 
Suppose that we only observe $K \leq N$ of the classes at training time, and that we fit a $K$-class Gaussian mixture model. 
We use $\loki$ to adapt our pretrained classifier to enable classification of any of the $N$ classes. We have that (a formal statement and interpretation are in the Appendix),  
%\textit{regardless of whether or not they were observed at training time.}

% \begin{theorem}[\textbf{Informal} $\loki$ sample complexity]
% \label{theorem:learning}
%     Let $G=(\set{V}, \set{E})$ with $\set{V} = \{V_i\}_{i=1}^N$ be an arbitrary graph. 
%     Let $\set{Y}=\set{V}$ and let $\Lambda = \{V_j\}_{j \in [K]} \subseteq \set{Y}$ be the set of observed classes. 
%     Under the following logistic model: 
%     %\begin{align*}
%         $\mathbb{P}(\widehat{V} = V_i \ | \ x) =: \mathbb{P}_i = \frac{\exp\{-x^\top\beta_i\}}{\sum_{j\in [K]} \exp\{-x^\top\beta_j\}} \mathbf{1}\{ \widehat{V} \in \Lambda\}$, 
%     %\end{align*}
%     where we have access to estimates $\widehat{\mathbb{P}}_i$ for all $i \in [K]$, 
%     and assuming realizability for predicting some class $V_*$, 
%     %\begin{align*}
%     $    V_* \in m_{[\mathbb{P}_i]_{i \in [K]}} = \argmin_{V \in \set{Y}} \sum_{i \in [K]} \mathbb{P}_i d^2(V, V_i)$, 
%     %\end{align*}
%     then with high probability, the sample complexity of estimating the target $V_*$ with prediction $\widehat{V}$ is 
%     \[d^2(V_*, \widehat{V}) \leq O( {K^2 \sqrt{d}}\times\text{diam}(G) ^2)/({\alpha \sqrt{n}}) ,\]
%     %\begin{align*}
%     %    $d^2(V_*, \widehat{V}) \leq O\left( \frac{K^2 \sqrt{d}}{\alpha \sqrt{n}} \text{diam}(G) ^2\right),$
%     %\end{align*}
%     where $d$ is the dimensionality of the input, $\text{diam}(G)$ is the graph diameter, $n$ is the number of samples, and $\alpha$ is related to the sensitivity of the Fréchet variance to different node choices. 
% \end{theorem}
% %In other words, we achieve the same rate as standard supervised learning, but have to pay more depending on the interplay between the graph

\begin{theorem}[\textbf{Informal} $\loki$ sample complexity]
\label{theorem:learning}
    %
    Let $\mathcal{Y}$ be a set of classes represented by $d$ dimensional vectors under the Euclidean distance, and let $\Lambda \subseteq \mathcal{Y}$ be the set of $K$ observed classes. 
    %
    Assume that $n$ training examples are generated by an identity covariance Gaussian mixture model over classes $\Lambda$, and that test examples are generated over all classes $\mathcal{Y}$. 
    %
    Assume that we estimate a Gaussian mixture model on the training set and obtain probability estimates $\hat{\mathbb{P}}(y_i | x)$ for $i \in [K]$ for a sample $(x, y_*)$ from the test distribution. 
    %
    Then with high probability, under the following model, 
    \[
        \hat{y}_* \in m_\Lambda([\hat{\mathbb{P}}(y_i | x)]_{i \in [K]}) 
        = \argmin_{y \in \mathcal{Y}} \sum_{i \in [K]} \hat{\mathbb{P}}(y_i | x) d^2(y, y_i)
    \]
    the sample complexity of estimating target $y_*$ from the test distribution $\mathcal{D}_{\text{test}}$ with prediction $\hat{y}_*$ is:
    \[
        \E_{(x, y_*)\sim \mathcal{D}_{\text{test}}} [d^2(y_*, \hat{y}_*)] \leq O\left( \frac{d}{\alpha} \sqrt{\frac{\log K/\delta}{n}} \left( \frac{1}{ \left( R^{1 - \frac{2}{d}} - \frac{\log R}{R}\right) } + \sqrt{d} \right) \right)
    \]
    where $\alpha$ is related to the sensitivity of the Fréchet variance to different node choices. 

\end{theorem}
%!TEX root = ../../main.tex

\subsection{Optimal Label Subspaces}
\label{sec:optimal_subspaces}
Next we characterize the subset of distinct labels required to predict any label using our label model with respect to various types of metric spaces. 

We first consider label spaces whose metric is a tree graph---such metrics are, for example, related to performing hierarchical classification and weak supervision (where only partial labels are available). 
We consider a special type of tree called a phylogenetic tree, in which only the leaves can be designated as labels---phylogenetic trees are commonly used to relate the labels of image classification datasets. % such as ImageNet or CIFAR-100.
%
Afterwards we perform a similar analysis for grid graphs, which are important for label spaces that encode spatial information. 
Finally, we discuss the case in which no useful metric information is available, i.e., the complete graph.

Our goal in this section is to characterize the properties and size of $\{\lambda_i\}_{i=1}^K$ in each of these metric spaces such that we still have $\Pi({\bf\Lambda}) = \set{Y}$. 
%
We characterize `optimal' subsets of classes in each of the spaces under a certain notions of optimality.  
We provide several relevant definitions pertaining to this concept, starting with a notion of being able to predict any possible class using observed classes. 

\begin{definition}[Locus cover]
\label{def:loc_cover}
Given a set $\Lambda \subseteq \set{Y}$ for which we construct a tuple of its elements $\boldsymbol{\Lambda}$, if it holds that $\Pi(\boldsymbol{\Lambda}) = \set{Y}$, then $\Lambda$ is a locus cover. 
\end{definition}

Definition~\ref{def:loc_cover} captures the main idea of $\loki$---using some set of observed classes for which we can train classifiers, we would like to be able to predict additional unobserved classes using the geometry that relates the observed and unobserved classes. Namely, elements of $\Pi(\boldsymbol{\Lambda})$ are `reachable' using $\loki$. 
We refine this Definition to describe the trivial case that defaults to standard classification and the nontrivial case for which $\loki$ moves beyond standard classification. 

\begin{definition}[Trivial locus cover]
\label{def:trivial_loc_cover}
If $\Lambda = \set{Y}$, then $\Lambda$ is the trivial locus cover. 
\end{definition}

This Definition captures the notion of observing all of the classes in the label space. 
Here, all of the elements of $\set{Y}$ are trivially reachable using $\loki$. 
%Later, we show that the complete graph has only the trivial locus cover. 

\begin{definition}[Nontrivial locus cover]
\label{def:nontrivial_loc_cover}
A locus cover $\Lambda$ is nontrivial if $\Lambda \neq \set{Y}$. 
%$\Lambda$ is a nontrivial locus cover if $\Lambda$ is a locus cover with $\Lambda \neq \set{Y}$. 
\end{definition}

$\loki$ is more useful and interesting when faced with a nontrivial locus cover---under Definition~\ref{def:nontrivial_loc_cover}, we can use some subset of classes $\Lambda$ to predict any label in $\set{Y}$. 

\begin{definition}[Minimum locus cover]
\label{def:min_loc_cover}
Given a set $\Lambda \subseteq \set{Y}$, if $\Lambda$ is the smallest set that is still a locus cover, then it is a minimum locus cover. 
\end{definition}

In cases involving an extremely large number of classes, it is desirable to use $\loki$ on the smallest possible set of observed classes $\Lambda$ such that all labels in $\set{Y}$ can still be predicted. 
Definition~\ref{def:min_loc_cover} characterizes these situations---later, we obtain the minimum locus covers for all trees and grid graphs. 
It is worth noting that the minimum locus cover need not be unique for a fixed graph. 

\begin{definition}[Identifying locus cover]
\label{def:identifying_loc_cover}
Given a set $\Lambda \subseteq \set{Y}$, if $\Lambda$ is a locus cover where $\forall\, y \in \set{Y}, \, \exists\, \mathbf{w} \in \Delta^{|\Lambda|-1}$ such that $m_{\boldsymbol{\Lambda}}(\mathbf{w}) = \{ y \}$,
then $\Lambda$ is an identifying locus cover. 
\end{definition}

The Fréchet mean need not be unique---as an $\argmin$, it returns a set of minimizers. 
In certain metric spaces, the minimum locus cover can yield large sets of minimizers---this is undesirable, as it makes predicting a single class challenging. 
Definition~\ref{def:identifying_loc_cover} appeals to the idea of finding some set of classes for which the Fréchet mean \textit{always} returns a unique minimizer---this is desirable in practice, and in some cases, moreso than Definition~\ref{def:min_loc_cover}. 

\begin{definition}[Pairwise decomposable]
\label{def:pairwise_decomposable}
Given $\Lambda \subseteq \set{Y}$, $\Pi(\Lambda)$ is called pairwise decomposable when it holds that $\Pi(\Lambda) = \cup_{\lambda_1, \lambda_2 \in \Lambda} \Pi(\{\lambda_1, \lambda_2\}).$ 
\end{definition}

In many cases, the locus can be written in a more convenient form---the union of the locus of pairs of nodes. 
We refer to this definition as pairwise decomposability. 
Later, we shall see that pairwise decomposability is useful in computing the locus in polynomial time. 


\paragraph{Trees}
Many label spaces are endowed with a tree metric in practice: hierarchical classification, in which the label space includes both classes and superclasses, partial labeling problems in which internal nodes can represent the prediction of a set of classes, and the approximation of complex or intractable metrics using a minimum spanning tree. 
We show that for our purposes, trees have certain desirable properties that make them easy to use with $\loki$---namely that we can easily identify a locus cover that satisfies both Definition~\ref{def:min_loc_cover} and Definition~\ref{def:identifying_loc_cover}. 
Conveniently, we also show that any locus in any tree satisfies Definition~\ref{def:pairwise_decomposable}. 

We first note that the leaves of any tree yield the minimum locus cover. 
This is a convenient property---any label from any label space endowed with a tree metric can be predicted using $\loki$ using only classifiers trained using labels corresponding to the leaves of the metric space. 
This can be especially useful if the tree has long branches and few leaves. 
Additionally, for tree metric spaces, the minimum locus cover (Definition~\ref{def:min_loc_cover}) is also an identifying locus cover (Definition~\ref{def:identifying_loc_cover}). 
This follows from the construction of the weights in the proof of Theorem~\ref{thm:min_locus_trees} (shown in the Appendix) and the property that all paths in trees are unique. 
Finally, we note that any locus in any tree is pairwise decomposable---the proof of this is given in the Appendix (Lemma~\ref{lemma:tree_pairwise}). 
We will see later that this property yields an efficient algorithm for computing the locus. 

\paragraph{Phylogenetic Trees}
Image classification datasets often have a hierarchical tree structure, where only the leaves are actual classes, and internal nodes are designated as superclasses---examples include the ImageNet \cite{imagenet} and CIFAR-100 datasets \cite{cifar100}. 
Tree graphs in which only the leaf nodes are labeled are referred to as phylogenetic trees \cite{BILLERA2001733}. 
Often, these graphs are weighted, but unless otherwise mentioned, we assume that the graph is unweighted. 

For any arbitrary tree $T=(\set V, \set E)$, the set of labels induced by phylogenetic tree graph is $\set{Y} = \text{Leaves}(T)$. 
We provide a heuristic algorithm for obtaining locus covers for arbitrary phylogenetic trees in Algorithm~\ref{alg:phylo} (see Appendix). 
%The intuition of the algorithm design is that the endpoints of the longest paths, $\Gamma(y_i, y_j)$, are more likely to be a part of a small locus cover because shorter paths can leave lower subtrees unreachable. 
We prioritize adding endpoints of long paths to $\Lambda$, and continue adding nodes in this way until $\Pi(\Lambda)=\set{Y}$. 
Similarly to tree metric spaces, any phylogenetic tree metric space is pairwise decomposable.
We prove the correctness of Algorithm~\ref{alg:phylo} and pairwise decomposability of phylogenetic trees in the Appendix (Theorem~\ref{thm:algphylo} and Lemma~\ref{lemma:phylotree_pairwise}).
Later, we give algorithms for computing the set of nodes in an arbitrary locus in arbitrary graphs---if the locus is pairwise decomposable, the algorithm for doing so is efficient, and if not, it has time complexity exponential in $K$. 
Due to the pairwise decomposability of phylogenetic trees, this polynomial-time algorithm to compute $\Pi(\Lambda)$ applies. 
%As such, Algorithm~\ref{alg:phylo} has time complexity $O(K^4 D \max\{N|\mathcal{E}|, N^2 \log N\})$. 

\paragraph{Grid Graphs}
Classes often have a spatial relationship. 
For example, classification on maps or the discretization of a manifold both have spatial relationships---grid graphs are well suited to these types of spatial relationships. 
We obtain minimum locus covers for grid graphs satisfying Definition~\ref{def:min_loc_cover}, but we find that these are not generally identifying locus covers. 
On the other hand, we give an example of a simple identifying locus cover satisfying Definition~\ref{def:identifying_loc_cover}. 
Again, we find that grid graphs are in general pairwise decomposable and hence follow Definition~\ref{def:pairwise_decomposable}. 

We find that the pair of vertices on furthest opposite corners yields the minimum locus cover. 
While the set of vertices given by Theorem~\ref{thm:min_loc_grid} (found in the Appendix) satisfies Definition~\ref{def:min_loc_cover}, this set does not in general satisfy Definition~\ref{def:identifying_loc_cover}. 
This is because the path between any two vertices is not unique, so each minimum path of the same length between the pair of vertices can have an equivalent minimizer. 
On contrast, the following example set of vertices satisfies Definition~\ref{def:identifying_loc_cover} but it clearly does not satisfy Definition~\ref{def:min_loc_cover}.
\emph{Example}: Given a grid graph, the set of all corners is an identifying locus cover. 
On the other hand, the vertices given by Theorem~\ref{thm:min_loc_grid} can be useful for other purposes. 
Lemma~\ref{lemma:locus_grid_subspace} (provided in the Appendix) shows that subspaces of grid graphs can be formed by the loci of pairs of vertices in $\Lambda$. 
This in turn helps to show that loci in grid graphs are pairwise decomposable in general (see Lemma~\ref{lemma:grid_pairwise} in the Appendix). 


\paragraph{The Complete Graph}
The standard classification setting does not use relational information between classes. 
As before, we model this setting using the complete graph, and we show the expected result that in the absence of useful relational information, $\loki$ cannot help, and the problem once again becomes standard multiclass classification among observed classes. 
To do so, we show that there is no nontrivial locus cover for the complete graph (Theorem~\ref{thm:trivial_complete} in the Appendix). 

%Trivially, any locus on the complete graph is also pairwise decomposable. 
%It follows from Theorem~\ref{thm:trivial_complete} that 
%$$ \Pi(\Lambda) = \Lambda = \cup_{\lambda_i, \lambda_j \in \Lambda} \{\lambda_i, \lambda_j\} = \cup_{\lambda_i, \lambda_j \in \Lambda} \Pi(\{\lambda_i, \lambda_j\}). $$


%!TEX root = ../../main.tex

\subsection{Label Subspaces in Practice}
While it is desirable for the set of observed classes to form a minimum or identifying locus cover, it is often not possible to choose the initial set of observed classes a priori---these are often random. 
In this section, we describe the more realistic cases in which a random set of classes are observed and an active learning-based strategy to choose the next observed class. 
The aim of our active learning approach is, instead of randomly selecting the next observed class, to actively select the next class so as to maximize the total size of the locus---i.e., the number of possible classes that can be output using $\loki$. 
Before maximizing the locus via active learning, we must first address a much more basic question: can we even efficiently compute the locus? 

\paragraph{Computing the Locus}
We provide algorithms for obtaining the set of all classes in the locus, given a set of classes $\Lambda$. 
We show that when the locus is pairwise decomposable (Definition~\ref{def:pairwise_decomposable}), we can compute the locus efficiently using a polynomial time algorithm. 
When the locus is not pairwise decomposable, we provide a general algorithm that has time complexity exponential in $|\Lambda|$---we are not aware of a more efficient algorithm. 
We note that any locus for every type of graph that we consider in Section~\ref{sec:optimal_subspaces} is pairwise decomposable, so our polynomial time algorithm applies. 
Algorithms~\ref{alg:locus_pairwise}~and~\ref{alg:locus_general} along with their time complexity analyses can be found in the Appendix. 

% Algorithm~\ref{alg:locus_pairwise} assumes that the locus is pairwise decomposable. 
% It operates by iterating over all pairs of observed classes, $\lambda_i, \lambda_j \in \Lambda$. Since the locus is pairwise decomposable, we can reduce the problem to computing the individual loci between each pair $\lambda_i, \lambda_j$---each of which can be computed efficiently by sweeping over the choices of weights $\mathbf{w} = [w_1, w_2]$ and computing the Fréchet mean for each configuration. 
% The time complexity of Algorithm~\ref{alg:locus_pairwise} is $O(K^2 D \max\{N|\mathcal{E}|, N^2 \log N\})$ where $\mathcal{E}$ is the number of edges in the graph.
% Algorithm~\ref{alg:locus_pairwise} and its time complexity analysis are provided in the Appendix. 

% Computing the locus when pairwise decomposability does not hold appears to be more challenging, and has time complexity that scales with the size of $\Lambda$. 
% Algorithm~\ref{alg:locus_general} is a brute-force approach to computing the locus, which operates by sweeping over all possible choices of weights $\mathbf{w} \in [0, 1]^{K}$ and computing the Fréchet mean for each configuration.
% The time complexity of this algorithm is $O(D^K)$. 
% While it might be possible to improve this algorithm by exploiting the fact that $\mathbf{w}$ can be constrained to the simplex without loss of generality and by exploiting symmetries, we believe that this will not lead to the exponential speedup necessary to obtain an efficient algorithm for computing a non-pairwise decomposable locus. 
% Algorithm~\ref{alg:locus_general} and its time complexity analysis can be found in the Appendix. 


\paragraph{Large Locus via Active Next-Class Selection}
\label{sec:active_selection}
We now turn to actively selecting the next class to observe in order to maximize the size of the locus. 
For this analysis, we focus on the active learning setting when the class structure is a tree graph, as tree graphs are generic enough to apply to a wide variety of cases---including approximating other graphs using the minimum spanning tree. 
Assume the initial set of $K$ observed classes are sampled at random from some distribution. 
We would like to actively select the $K+1$st class such that $|\Pi(\Lambda)|$ with $\Lambda = \{\lambda\}_{i=1}^{K+1}$ is as large as possible. 

\begin{theorem}
\label{thm:active_largest}
Let $T = (\mathcal{Y}, \mathcal{E})$ be a tree graph and let $\Lambda \subseteq \mathcal{Y}$ with $K = |\Lambda|$. 
Let $T'$ % = (\Pi(\Lambda), \{(v_1, v_2): v_1, v_2 \in \Pi(\Lambda) \text{ and } (v_1, v_2) \in \mathcal{E}\})$ 
be the subgraph of the locus $\Pi(\Lambda)$. 
The vertex $v \in \mathcal{Y} \setminus \Lambda$ that maximizes $|\Pi(\Lambda \cup \{v\})|$ is the solution to the following optimization problem:
$\argmax_{y \in \mathcal{Y} \setminus \Pi(\Lambda)} d(y, b)$ s.t. $b \in \partial_{\text{in}} T'$ and $\Gamma(y, b) \setminus \{b\} \subseteq \mathcal{Y} \setminus \Pi(\Lambda)$. 
% \begin{argmaxi*}|s|
% {y \in \mathcal{Y} \setminus \Pi(\Lambda)}{d(y, b)}
% {}{}
% \addConstraint{b \in \partial_{\text{in}} T'}
% \addConstraint{\Gamma(y, b) \setminus \{b\} \subseteq \mathcal{Y} \setminus \Pi(\Lambda)}, 
% \end{argmaxi*}
where $\partial_{\text{in}} T'$ is the inner boundary of $T'$ (all vertices in $T'$ that share an edge with vertices not in $T'$). 
\end{theorem}

This procedure can be computed in polynomial time---solving the optimization problem in Theorem~\ref{thm:active_largest} simply requires searching over pairs of vertices. 
Hence we have provided an efficient active learning-based strategy to maximize the size of the locus for trees. 

% \paragraph{Trees}
% \begin{lemma}
% Let $y \in \Pi(\Lambda)$ and let $y' \in \Pi(\Lambda)$, Then $\Gamma (y, y') = \Gamma(y, z) \bigcup \Gamma(z, y')$, where $\Gamma(y, z) \subseteq \Pi(\Lambda)$ and $\Gamma(z, y') \setminus{z}$ has no vertices in $\Pi(\Lambda)$
% \end{lemma}
% \begin{proof}
% Proof by contradiction. Assume there is a node $\lambda \in \Gamma(z, y')$ also in $\Pi(\Lambda)$. Any pair of leaves $l_1$ and $l_2$ have unique path. We can find the shortest distance from $\lambda$ to the point $z \in \Pi(\Lambda)$. If the shortest geometric distance is 1, we can expend $\Gamma(y, z)$ to $\Gamma(y, \lambda)$, which contradict our initial Definition. Therefore, there should exist $v \notin \Pi(\Lambda)$ satisfied $dist(v, \lambda) = 1$ along the unique path between $z$ and $\lambda$. Since $\lambda \in \Pi(\Lambda)$, there should have a path between another point inside $\Pi(\Lambda)$ and $\lambda$. Then, it is a cycle. 
% \end{proof}

% TODO not sure we need this
% \paragraph{Grid graphs}
% \begin{lemma}
% TODO @Nick (define the algorithm clearly, Kaylee don't know how to define it formally): Add the $\max\min \lambda$ can give us the maximum span.
% \end{lemma}
% \begin{proof}
% From the lemma, we need the point with the longest distance toward any point inside $\Pi(\Lambda)$
% \end{proof}

