\PassOptionsToPackage{table}{xcolor}
\documentclass[letterpaper]{article} %DO NOT CHANGE THIS

\usepackage{times}  %Required
\usepackage{helvet}  %Required
\usepackage{courier}  %Required
\usepackage{url}  %Required
\usepackage{graphicx}  %Required
\usepackage{wrapfig}

\frenchspacing  %Required
\setlength{\pdfpagewidth}{8.5in}  %Required
\setlength{\pdfpageheight}{11in}  %Required

\usepackage[utf8]{inputenc} % allow utf-8 input
\usepackage[T1]{fontenc}    % use 8-bit T1 fonts
%\usepackage{lmodern}
% \usepackage{hyperref}       % hyperlinks
% \usepackage{url}            % simple URL typesetting
\usepackage{booktabs}       % professional-quality tables
\usepackage{amsfonts}       % blackboard math symbols
\usepackage{nicefrac}       % compact symbols for 1/2, etc.
\usepackage{microtype}      % microtypography
\usepackage{tikz}
\usetikzlibrary{shapes.misc, positioning}
\usepackage{thmtools,thm-restate}
% \usepackage{cite}
\usepackage{parskip}
\usepackage{authblk}
\usepackage{amsmath}  % Define \boldsymbol (in amsbsy too) and align
\usepackage{mathtools}
\usepackage{amsthm}
%\usepackage{unicode-math}
%%%%% NEW MATH DEFINITIONS %%%%%
\newtheorem{property}{Property}
\newtheorem{definition}{Definition}
\newtheorem{theorem}{Theorem}
\newtheorem{lemma}{Lemma}
\newtheorem{corollary}{Corollary}
\DeclarePairedDelimiter\abs{\lvert}{\rvert}
\DeclarePairedDelimiter\norm{\lVert}{\rVert}
\makeatletter
\let\oldabs\abs
\def\abs{\@ifstar{\oldabs}{\oldabs*}}
\let\oldnorm\norm
\def\norm{\@ifstar{\oldnorm}{\oldnorm*}}
\makeatother

% Mark sections of captions for referring to divisions of figures
\newcommand{\figleft}{{\em (Left) }}
\newcommand{\figcenter}{{\em (Center) }}
\newcommand{\figright}{{\em (Right)}}
\newcommand{\figtop}{{\em (Top) }}
\newcommand{\figbottom}{{\em (Bottom) }}
\newcommand{\captiona}{{\em (a) }}
\newcommand{\captionb}{{\em (b) }}
\newcommand{\captionc}{{\em (c) }}
\newcommand{\captiond}{{\em (d) }}

% Highlight a newly defined term
\newcommand{\newterm}[1]{{\bf #1}}


\def\figref#1{figure~\ref{#1}}
\def\Figref#1{Figure~\ref{#1}}
\def\twofigref#1#2{figures \ref{#1} and \ref{#2}}
\def\quadfigref#1#2#3#4{figures \ref{#1}, \ref{#2}, \ref{#3} and \ref{#4}}
\def\secref#1{section~\ref{#1}}
\def\Secref#1{Section~\ref{#1}}
\def\twosecrefs#1#2{sections \ref{#1} and \ref{#2}}
\def\secrefs#1#2#3{sections \ref{#1}, \ref{#2} and \ref{#3}}
\def\eqref#1{equation~\ref{#1}}
\def\Eqref#1{Equation~\ref{#1}}
% A raw reference to an equation---avoid using if possible
\def\plaineqref#1{\ref{#1}}
% Reference to a chapter, lower-case.
\def\chapref#1{chapter~\ref{#1}}
% Reference to an equation, upper case.
\def\Chapref#1{Chapter~\ref{#1}}
% Reference to a range of chapters
\def\rangechapref#1#2{chapters\ref{#1}--\ref{#2}}
% Reference to an algorithm, lower-case.
\def\algref#1{algorithm~\ref{#1}}
% Reference to an algorithm, upper case.
\def\Algref#1{Algorithm~\ref{#1}}
\def\twoalgref#1#2{algorithms \ref{#1} and \ref{#2}}
\def\Twoalgref#1#2{Algorithms \ref{#1} and \ref{#2}}
% Reference to a part, lower case
\def\partref#1{part~\ref{#1}}
% Reference to a part, upper case
\def\Partref#1{Part~\ref{#1}}
\def\twopartref#1#2{parts \ref{#1} and \ref{#2}}

% Random variables
\def\reta{{\textnormal{$\eta$}}}
\def\ra{{\textnormal{a}}}

% Random vectors
\def\rvepsilon{{\mathbf{\epsilon}}}
\def\rvtheta{{\mathbf{\theta}}}
\def\rva{{\mathbf{a}}}

% Elements of random vectors
\def\erva{{\textnormal{a}}}
\def\ervb{{\textnormal{b}}}

% Random matrices
\def\rmA{{\mathbf{A}}}
\def\rmB{{\mathbf{B}}}

% Elements of random matrices
\def\ermA{{\textnormal{A}}}
\def\ermB{{\textnormal{B}}}

\def\fvec{{\mathbf{f}}}
\def\bff{{\mathbf{f}}}
\def\bfg{{\mathbf{g}}}
% Vectors
\def\vzero{{\bm{0}}}
\def\vone{{\bm{1}}}
\def\vmu{{\bm{\mu}}}
\def\vtheta{{\bm{\theta}}}
\def\va{{\bm{a}}}
\def\vb{{\bm{b}}}
\def\vc{{\bm{c}}}
\def\vd{{\bm{d}}}
\def\ve{{\bm{e}}}
\def\vf{{\bm{f}}}
\def\vg{{\bm{g}}}
\def\vh{{\bm{h}}}
\def\vi{{\bm{i}}}
\def\vj{{\bm{j}}}
\def\vk{{\bm{k}}}
\def\vl{{\bm{l}}}
\def\vm{{\bm{m}}}
\def\vn{{\bm{n}}}
\def\vo{{\bm{o}}}
\def\vp{{\bm{p}}}
\def\vq{{\bm{q}}}
\def\vr{{\bm{r}}}
\def\vs{{\bm{s}}}
\def\vt{{\bm{t}}}
\def\vu{{\bm{u}}}
\def\vv{{\bm{v}}}
\def\vw{{\bm{w}}}
\def\vx{{\bm{x}}}
\def\vy{{\bm{y}}}
\def\vz{{\bm{z}}}

% Matrix
\def\mA{{\bm{A}}}

% Tensor
\DeclareMathAlphabet{\mathsfit}{\encodingdefault}{\sfdefault}{m}{sl}
\SetMathAlphabet{\mathsfit}{bold}{\encodingdefault}{\sfdefault}{bx}{n}
\newcommand{\tens}[1]{\bm{\mathsfit{#1}}}
\def\tA{{\tens{A}}}
\def\tB{{\tens{B}}}
\def\tC{{\tens{C}}}
\def\tD{{\tens{D}}}
\def\tE{{\tens{E}}}
\def\tF{{\tens{F}}}
\def\tG{{\tens{G}}}
\def\tH{{\tens{H}}}
\def\tI{{\tens{I}}}
\def\tJ{{\tens{J}}}
\def\tK{{\tens{K}}}
\def\tL{{\tens{L}}}
\def\tM{{\tens{M}}}
\def\tN{{\tens{N}}}
\def\tO{{\tens{O}}}
\def\tP{{\tens{P}}}
\def\tQ{{\tens{Q}}}
\def\tR{{\tens{R}}}
\def\tS{{\tens{S}}}
\def\tT{{\tens{T}}}
\def\tU{{\tens{U}}}
\def\tV{{\tens{V}}}
\def\tW{{\tens{W}}}
\def\tX{{\tens{X}}}
\def\tY{{\tens{Y}}}
\def\tZ{{\tens{Z}}}


% Graph
\def\gA{{\mathcal{A}}}
\def\gB{{\mathcal{B}}}
\def\gC{{\mathcal{C}}}
\def\dataset{{\mathcal{D}}}
\def\gE{{\mathcal{E}}}
\def\gF{{\mathcal{F}}}
\def\fourier{{\mathcal{F}}}
\def\gG{{\mathcal{G}}}
\def\gH{{\mathcal{H}}}
\def\gI{{\mathcal{I}}}
\def\gJ{{\mathcal{J}}}
\def\gK{{\mathcal{K}}}
\def\gL{{\mathcal{L}}}
\def\loss{{\mathcal{L}}}
\def\gM{{\mathcal{M}}}
\def\gN{{\mathcal{N}}}
\def\normal{{\mathcal{N}}}
\def\gaussian{{\mathcal{N}}}
\def\gO{{\mathcal{O}}}
\def\gP{{\mathcal{P}}}
\def\gQ{{\mathcal{Q}}}
\def\gR{{\mathcal{R}}}
\def\gS{{\mathcal{S}}}
\def\gT{{\mathcal{T}}}
\def\gU{{\mathcal{U}}}
\def\uniform{{\mathcal{U}}}
\def\gV{{\mathcal{V}}}
\def\gW{{\mathcal{W}}}
\def\gX{{\mathcal{X}}}
\def\gY{{\mathcal{Y}}}
\def\gZ{{\mathcal{Z}}}

\def\algebra{{\mathscr{A}}}
\def\borel{{\mathscr{B}}}
\def\manifold{{\mathscr{M}}}

% Sets
\def\sA{{\mathbb{A}}}
\def\sB{{\mathbb{B}}}
\def\complex{{\mathbb{C}}}
\def\sD{{\mathbb{D}}}
\def\expectation{{\mathbb{E}}}
\newcommand{\E}{\mathbb{E}}
\def\sF{{\mathbb{F}}}
\def\sG{{\mathbb{G}}}
\def\sH{{\mathbb{H}}}
\def\sI{{\mathbb{I}}}
\def\sJ{{\mathbb{J}}}
\def\sK{{\mathbb{K}}}
\def\sL{{\mathbb{L}}}
\def\sM{{\mathbb{M}}}
\def\natural{{\mathbb{N}}}
\def\sO{{\mathbb{O}}}
\def\sP{{\mathbb{P}}}
\def\rational{{\mathbb{Q}}}
\def\real{{\mathbb{R}}}
\newcommand{\R}{\mathbb{R}}
\def\sS{{\mathbb{S}}}
\def\sphere{{\mathbb{S}}}
\def\sT{{\mathbb{T}}}
\def\sU{{\mathbb{U}}}
\def\sV{{\mathbb{V}}}
\def\sW{{\mathbb{W}}}
\def\sX{{\mathbb{X}}}
\def\sY{{\mathbb{Y}}}
\def\integer{{\mathbb{Z}}}
\def\indicator{{\mathbbm{1}}}

% Entries of a matrix
\def\emLambda{{\Lambda}}
\def\emA{{A}}
\def\emB{{B}}
\def\emC{{C}}
\def\emD{{D}}
\def\emE{{E}}
\def\emF{{F}}
\def\emG{{G}}
\def\emH{{H}}
\def\emI{{I}}
\def\emJ{{J}}
\def\emK{{K}}
\def\emL{{L}}
\def\emM{{M}}
\def\emN{{N}}
\def\emO{{O}}
\def\emP{{P}}
\def\emQ{{Q}}
\def\emR{{R}}
\def\emS{{S}}
\def\emT{{T}}
\def\emU{{U}}
\def\emV{{V}}
\def\emW{{W}}
\def\emX{{X}}
\def\emY{{Y}}
\def\emZ{{Z}}
\def\emSigma{{\Sigma}}

% entries of a tensor
% Same font as tensor, without \bm wrapper
\newcommand{\etens}[1]{\mathsfit{#1}}
\def\etLambda{{\etens{\Lambda}}}
\def\etA{{\etens{A}}}
\def\etB{{\etens{B}}}
\def\etC{{\etens{C}}}
\def\etD{{\etens{D}}}
\def\etE{{\etens{E}}}
\def\etF{{\etens{F}}}
\def\etG{{\etens{G}}}
\def\etH{{\etens{H}}}
\def\etI{{\etens{I}}}
\def\etJ{{\etens{J}}}
\def\etK{{\etens{K}}}
\def\etL{{\etens{L}}}
\def\etM{{\etens{M}}}
\def\etN{{\etens{N}}}
\def\etO{{\etens{O}}}
\def\etP{{\etens{P}}}
\def\etQ{{\etens{Q}}}
\def\etR{{\etens{R}}}
\def\etS{{\etens{S}}}
\def\etT{{\etens{T}}}
\def\etU{{\etens{U}}}
\def\etV{{\etens{V}}}
\def\etW{{\etens{W}}}
\def\etX{{\etens{X}}}
\def\etY{{\etens{Y}}}
\def\etZ{{\etens{Z}}}

\def\ceil#1{\lceil #1 \rceil}
\def\floor#1{\lfloor #1 \rfloor}
\def\eps{{\epsilon}}

\newcommand{\pder}[1]{\frac{\partial}{\partial #1}}

\newcommand{\half}{\frac{1}{2}}
\newcommand{\limNinf}{\lim_{N \to \infty}}
\newcommand{\limTzero}{\lim_{\tau \to 0}}


\newcommand{\cmark}{\ding{51}}
\newcommand{\xmark}{\ding{55}}

\newcommand{\layer}{\mathcal{H}}
\newcommand{\defeq}{\triangleq}
%\newcommand{\defeq}{vcentcolon=}
\newcommand{\domain}{\Omega}
\newcommand{\grad}{\nabla}

\newcommand{\cin}{c_{\rm{in}}}
\newcommand{\cout}{c_{\rm{out}}}
\newcommand{\intdomain}{\int_{\domain}}
\newcommand{\network}{\gT}
\newcommand{\subnet}{\gK}
\newcommand{\map}{\gR} %\gR

\newcommand{\innerproduct}[2]{\langle #1, #2 \rangle}
\newcommand{\mcsum}[1][j]{\frac{1}{N}\sum_{#1=1}^N}

\newcommand{\inrspace}[1][c]{\gF_{#1}}

\DeclareMathOperator*{\argmax}{arg\,max}
\DeclareMathOperator*{\argmin}{arg\,min}

\let\ab\allowbreak

\usepackage{optidef}
%\usepackage{tablefootnote}
\usepackage{algorithm}
\usepackage{algpseudocode}

%%%%%%%%%%%%%%%%%%%%%%%%%%%%%%%%
% THEOREMS
%%%%%%%%%%%%%%%%%%%%%%%%%%%%%%%%
\theoremstyle{plain}
\newtheorem{theorem}{Theorem}[section]
\newtheorem{proposition}[theorem]{Proposition}
\newtheorem{lemma}[theorem]{Lemma}
\newtheorem{corollary}[theorem]{Corollary}
\theoremstyle{definition}
\newtheorem{definition}[theorem]{Definition}
\newtheorem{assumption}[theorem]{Assumption}
\theoremstyle{remark}
\newtheorem{remark}[theorem]{Remark}

\usepackage[table]{xcolor}


\newcommand{\set}[1]{\mathcal{#1}}
\usepackage[numbers,sort]{natbib}
\newcommand{\yrcite}[1]{\citeyearpar{#1}}

%%Allowing tables to go next to figures
%\usepackage{floatrow}
%% Table float box with bottom caption, box width adjusted to content
%\newfloatcommand{capbtabbox}{table}[][\FBwidth]

% Macros used in the paper
%\usepackage{macros}
\usepackage{basic}

%\newcommand{\set}[1]{\mathcal{#1}}
\newcommand\nonumberthis{\nonumber\refstepcounter{equation}}
\newcommand{\loki}{\textsc{Loki}}
\newcommand{\nick}[1]{\textcolor{purple}{[Nick: #1]}}


\title{Geometry-Aware Adaptation for Pretrained Models}

 \renewcommand\Authsep{\hspace{2em}}
 \renewcommand\Authand{\hspace{2em}}
 \renewcommand\Authands{\hspace{2em}}

 \renewcommand\Authfont{\normalfont}
 \renewcommand\Affilfont{\small}

 \author[$\dagger$]{Nicholas Roberts}
 \author[$\dagger$]{Xintong Li}
 \author[$\dagger$]{Dyah Adila}
 \author[$\dagger$]{Sonia Cromp}
 \author[$\dagger$]{\\Tzu-Heng Huang}
 \author[$\dagger$]{Jitian Zhao}
\author[$\dagger$]{Frederic Sala}

 \affil[$\dagger$]{University of Wisconsin-Madison}
 \affil[ ]{\footnotesize{\texttt{\{nick11roberts, fredsala\}@cs.wisc.edu}}}
 \affil[ ]{\footnotesize{\texttt{\{xli2224, adila, cromp, thuang273, jzhao326\}@wisc.edu}}}

\begin{document}

\maketitle

\begin{abstract}
    \begin{abstract}

The Fast Reciprocal Square Root Algorithm is a well-established approximation technique consisting of two stages: first, a coarse approximation is obtained by manipulating the bit pattern of the floating point argument using integer instructions, and second, the coarse result is refined through one or more steps, traditionally using Newtonian iteration but alternatively using improved expressions with carefully chosen numerical constants found by other authors. The algorithm was widely used before microprocessors carried built-in hardware support for computing reciprocal square roots. At the time of writing, however, there is in general no hardware acceleration for computing other fixed fractional powers. This paper generalises the algorithm to cater to all rational powers, and to support any polynomial degree(s) in the refinement step(s), and under the assumption of unlimited floating point precision provides a procedure which automatically constructs provably optimal constants in all of these cases. It is also shown that, under certain assumptions, the use of monic refinement polynomials yields results which are much better placed with respect to the cost/accuracy tradeoff than those obtained using general polynomials. Further extensions are also analysed, and several new best approximations are given.

\end{abstract}

\end{abstract} 


% Figure environment removed

\section{Introduction}
Automatic 3D reconstruction of clothed humans using image inputs has gained increasing significance due to its potential applications in a wide array of AR/VR scenarios. High-fidelity reconstructions typically depend on sophisticated capture systems, which are developed with dense camera arrays~\cite{collet2015high,joo2015panoptic,joo2018total}, programmable light-stages~\cite{Vlasic2009, guo2019relightables}, and depth sensors~\cite{newcombe2011kinectfusion,DoubleFusion,BodyFusion,dou2016fusion4d,newcombe2015dynamicfusion}. However, stringent capture environments equipped with complex hardware pose significant challenges for consumer-level applications.


In this context, considerable research effort has been dedicated to developing methods that allow for more flexible capture configurations, such as utilizing a few RGB inputs. Among these works, learning implicit functions \cite{iccv2020PIFu, saito2020pifuhd, hong2021stereopifu} has proven effective in achieving highly detailed reconstructions by integrating the advancements of deep neural networks. These methods employ large multi-layer perceptrons (MLPs) to predict the occupancy probability or truncated signed distance function (TSDF) value of every queried 3D point based on its associated local feature, which is extracted from images. They can recover a continuous surface at arbitrary resolutions without topology restrictions.


However, in typical MLP-based implicit networks, the occupancy or TSDF value at each location is solved independently with planar image features, rendering them less capable of addressing challenging cases such as occlusions. Consequently, these methods suffer from generalization and robustness issues, particularly when tackling strong occlusions caused by large motion or multiple interacting humans. 
Some follow-up studies  \cite{zheng2021deepmulticap,zheng2021pamir,huang2020arch} utilize an extra geometric model, SMPL~\cite{Loper2015}, to improve robustness by introducing strong shape priors. 
Their success typically relies on the assumption of geometrical similarity \cite{huang2020arch} between the shape prior and target reconstruction, making them intractable for handling complex cases with loose clothes and sensitive to errors in SMPL model fitting.



%\ping{this paragraph sounds like `TSDF is better than MLP/SMPL, and we use TSDF to solve the problem'. But in Sec 3, we are telling a different story, saying `MLP needs a 3D convolutional encoder'. We need to make these two sections consistent.}\sicong{I think in this paragraph we claim that the TSDF}


%We opt for Trucated Signed Distance Funtion (TSDF) volumetric representations as they are naturally suitable for convolution operations, which have shown remarkable performance for learning hierarchical features on 2D visual perception tasks \cite{SunXLW19}. 
%Meanwhile, TSDF also describes the gradual geometry change around shape surface, which is not reflected by occupancy volume. 

We instead revisit the 3D volumetric representation and resort to 3D convolutional neural networks (CNNs) for feature learning, due to their impressive performance in feature learning and the ability to incorporate spatial context. However, volumetric methods and 3D convolution involve discretization, which might raise concerns regarding whether a discretized volume can preserve subtle geometric details as continuous representations learned in implicit functions. We investigate the relationship between volume resolution and quantization error on synthetic data by converting target mesh objects to TSDF volumes, as shown in Figure~\ref{fig:quantization_error}. We observe that the quantization errors are significantly reduced by increasing volume resolution and become nearly negligible when reaching a relatively high resolution (e.g., 512 or higher). In other words, achieving fine-detailed reconstruction is not supposed to be restricted by the use of volume representations as long as a proper volume resolution is utilized. Therefore, we present a method with high-resolution feature volumes, e.g., 256 and 512, while traditional volumetric methods \cite{varol18_bodynet,gilbert2018volumetric} are often limited to much lower resolutions, such as 32 or 128.



On the other hand, an increase in volume resolution may lead to a cubic growth of memory overhead \cite{8100085}. Reducing memory costs while guaranteeing the granularity of volumetric representations is necessary for pursuing high-quality reconstruction. Thus, we adopt a coarse-to-fine approach and cull away irrelevant voxels to build a sparse high-resolution feature volume. At the coarse level, the network computes an initial TSDF by applying a U-Net with sparse 3D CNN \cite{3DSemanticSegmentationWithSubmanifoldSparseConvNet} on the sparse feature volume, which is carved by a visual hull. Through our experiments, it turns out that more than 95\% of the volume grids are discarded by the visual hull culling, making the sparse 3D CNN efficient. At the fine level, the network focuses on a narrow band near the zero-level set of the initial TSDF and discretizes the narrow band with smaller voxels. By employing this narrow-band culling, we further shrink the sampling space, resulting in a relatively small range of grid numbers (usually 300K--500K in our experiments) even with a high volume resolution of 512. The remaining voxels in the narrow band are associated with features that fuse high-frequency information from the computed normal maps upon the low-frequency shape from the coarse level to compute the TSDF at high resolution. The final mesh is then extracted from the TSDF using the Marching-Cube algorithm ~\cite{Lorensen87marchingcubes}.
% Different from the u-net sturcture to preserve global topology context, we then apply a shallow 3dcnn to compute the final TSDF $D_{final}$ which contain more local geometry detail.




% \ping{this paragraph can be expanded. It is an important contribution and often ignored by other works. stress on the novel idea of regressing blending weights instead of colors}

In addition to geometry, high-quality mesh texture is also a crucial factor contributing to visual appearance. Directly computing a color field in 3D space, as in \cite{iccv2020PIFu}, struggles to capture high-frequency texture details, while the neural radiance field (NeRF) \cite{yu2020pixelnerf} or the DoubleField~\cite{shao2022doublefield} require expensive per-instance optimization and are often unstable for sparse input images. In contrast, we adopt an image-based rendering approach to compute a texture atlas map, which is efficient and widely supported in existing computer graphics tools. 
Specifically, we compute a blending weight at each 3D point on the mesh surface to determine its color as a weighted average of the colors at its image projections. The blending weights can be computed at a relatively coarse resolution, e.g., 512 volume resolution in our case, and leave texture details to the high-resolution images, such as 1K or 2K. Unlike previous methods that generate blurry texturing results under sparse input, our method generalizes well on both synthetic and real data with just a few input views. 
Figure~\ref{fig:teaser} shows two examples reconstructed by our method. Despite the challenging garment, pose, and occlusion, our method recovers faithful shape, normal, and texture on the right.

%with a wide variety of poses and clothing styles, and it is also adaptive to handle input image with arbitrary resolutions.
%\sicong{For this concern we claim that when the resolution of dicretized volume meets certain threshold (which is 256 in our experiment), the quantization error can be neglected.} 



In summary, the main contributions of this paper are as follows:
\begin{itemize}
\vspace{-0.1in}
  \item 
  We revisit the 3D volumetric representation and demonstrate that it can support clothed human reconstruction with equal or even better performance compared to implicit representation. 
  \item 
  We develop a memory and computation-efficient method for high-resolution volumetric reconstruction using sophisticated sparse 3D CNN, coarse-to-fine estimation, and voxel culling by visual hull and narrow bands. 
  \item 
  We introduce a novel method to compute a texture atlas map, which captures rich appearance details from high-resolution input images.
  \item 
  We achieve impressive results on standard benchmark datasets Twindom and MultiHuman, significantly reducing the point-2-surface (P2S) precision to approximately 0.2cm from just six input views, with more than $50\%$ error reduction compared to the state-of-the-art methods, including DoubleField~\cite{shao2022doublefield} and PIFuHD~\cite{saito2020pifuhd}.
\end{itemize}
\vspacebeforesection
\section{Background}
\label{sec:background}

In this section, we provide the necessary background information to ensure a comprehensive understanding of the attack described in this paper. We start with a description of the Distributed Hash Table (DHT) used by IPFS, followed by its content resolution mechanisms. We also detail techniques for network size estimation, necessary for our attack detection and mitigation mechanisms.

\vspacebeforesection
\subsection{IPFS DHT}
\label{sec:kad_dht}

We review the features of the Kademlia DHT~\cite{maymounkov2002kademlia} and its \texttt{libp2p} implementation~\cite{libp2p_github} that are the most relevant to our attack.
To participate in the DHT, each peer generates a public/private key pair and derives an identity $\peerid \in \{0,1\}^{256}$ as the hash of its public key.
Ideally, each peer generates a random key pair and, therefore, peer IDs are distributed uniformly and independently over the space $\{0,1\}^{256}$.
While honest nodes follow this rule, malicious nodes may generate and choose from an arbitrary number of key pairs.
Each peer maintains a routing table consisting of $m=256$ buckets.
The $i$-th bucket contains the addresses of up to $k=20$ peers whose peer IDs share a common prefix of exactly $i$ bits with the peer's own peer ID. 

%
A new participant node joins the IPFS network by contacting one of the hardcoded bootstrap nodes. This bootstrap node provides the new node with some initial peers allowing it to join the DHT. The new node uses this information to perform a walk through the DHT towards its own peer ID.
The walk allows to: \textit{(i)}~make sure that there is no other node in the network with the same ID; \textit{(ii)}~discover new peers and fill the newcomer's DHT routing table. At the same time, the newcomer establishes \bitswap~\cite{de2021accelerating} connections to a subset of encountered peers (usually around 300 of them). The core role of the \bitswap protocol is to enable bilateral content transfer and to play the role of a cache for recently-accessed content.

The main DHT operation $\Call{GetClosestPeers}{\key}$ returns the $k=20$ closest peers to $\key$. 
%
In Kademlia, the distance between two keys $x$ and $y$ in the key space is given by $x \oplus y \in \{0,...,2^{256}-1\}$, where $\oplus$ denotes the bitwise XOR operation on the keys; the resulting binary string is interpreted as an integer.
%
When a client wants to find the peers with IDs closest to $\key$, it sends a request to the $\alpha=3$ peers in its routing table whose peer IDs are closest to $\key$. Each of these peers returns the $k$ closest peers to $\key$ in its own routing table and the addresses of these peers. 
%
The client again sends a request to the $\alpha$ peers closest to $\key$, among peers in its routing table and those whose addresses it just received. This process repeats until the client does not find any more peers closer to $\key$.
Due to network churn and imperfect routing tables, we observed in our experiments that successive calls to $\Call{GetClosestPeers}{\key}$ do not always return the same set of $k=20$ peers (we provide more details in \Cref{sec:evaluation}, \Cref{fig:20closest}). This is an important limitation affecting our attack.

\vspacebeforesection
\subsection{Content Resolution in IPFS}
\label{sec:ipfs}

IPFS is a content-centric network.
It allows its participant to request files without specifying their location. 
%
Content is indexed by content IDs $\cid \in \{0,1\}^{256}$ that are derived from a hash of that content.
Both peer IDs and CIDs are used as keys in the DHT.
Each node can play the role of a \provider, \downloader, or \resolver. 
The process of content advertisement and resolution is illustrated in \Cref{fig:add_get_provider}.

%
When a \provider wishes to publish content with a given $\cid$ on IPFS, it creates a \emph{provider record} that contains $cid$ and the \provider's address.
During a $\Call{Provide}{\cid}$ operation, the \provider first uses $\Call{GetClosestPeers}{\cid}$ to locate the $k=20$ peers with their peer IDs closest to $\cid$, 
%
and then sends them a $\mathsf{PutProvider}$ message including the provider record (\Cref{fig:add_get_provider}(a)).
We call the peers that hold provider records for $\cid$ the \emph{resolvers} for $\cid$.

Each CID can have several \providers. In fact, by default, each IPFS client becomes a provider for each piece of content it downloads for a fixed amount of time (12h, 24h, or 48h depending on the client version or custom configuration). As a result, the system provides an auto-scaling feature with supply automatically rising with demand.

%
When a \downloader wishes to fetch a piece of content, it first sends a request to all its \bitswap peers. If none of them has the content, the \downloader uses the DHT-based resolution system. We stress that the \bitswap protocol plays the supporting role of a cache in the dissemination of popular files. However, the mechanism does not provide reliable content resolution, in particular for new or less popular content. %

When \bitswap unstructured search fails, the \downloader resolves $\cid$ using $\Call{FindProviders}{\cid}$. This operation uses a DHT walk identical to that of $\Call{GetClosestPeers}{\cid}$ to find $k$ \resolvers but also queries encountered nodes for a provider record for $\cid$ (\Cref{fig:add_get_provider}(b)). The process terminates when either 20 \providers have been found, or all \resolvers have been asked. Querying all encountered nodes (\ie, not only the designated \resolvers) is useful because some of the encountered nodes may have a provider record in their cache.
%

Upon receiving a provider record, the client connects to the address specified in the provider record to retrieve the actual content (\Cref{fig:add_get_provider}(c)).
Provider records are not authenticated, and therefore malicious \providers may respond with incorrect provider records (or may not respond at all). However, the integrity of the content is preserved because the hash of the retrieved content can be verified against its $\cid$.
%


%

\input{img/add_get_provider.tex}

\vspacebeforesection
\subsection{Network Size Estimator}
\label{sec:netsize}

The number of nodes in a decentralized system is generally unknown due to the avoidance of centralized membership management.
This number is nonetheless useful for optimizations, deciding on individual node configurations, or security mechanisms.
Various methods were proposed for the decentralized estimation of unstructured and structured networks~\cite{eli-sohl-dht-size-estimation,kostoulas2005decentralized, manku2003symphony}.
We use in this work a mechanism developed initially by Protocol Labs as part of a mechanism for decreasing the latency of publishing content in IPFS~\cite{network-size-estimation-notion,network-size-estimation-github-pr}.

%
%
%
%
%
%
%
%
%
%

Each node in the DHT refreshes its routing table periodically (every $10$ minutes in \texttt{libp2p}). 
For this, the node samples $m$ random keys (one for each bucket of its routing table)
%
and queries the DHT to obtain the $k=20$ closest peer IDs to each key.
Using these, the node then computes the average distance between each one of these keys $\key_j$ for $j=1,\dots,m$ and their $i$-th closest peer ID for $i=1,...,k$ (with $m=256$ and $k=20$).
\begin{equation}
    \label{equ:avg-dist}
    \overline{D}_i = \frac{1}{m} \sum_{j=1}^m \operatorname{dist}(\key_j, \peerid_{j}^{(i)})
\end{equation}
where $\peerid_{j}^{(i)}$ is the $i$-th closest peer ID to $\key_j$.
With $N$ peers in the DHT and peer IDs uniformly distributed in the hash space, the expected distance between a $\key$ and its $i$-th closest peer ID is $\frac{2^{256}i}{N+1}$. The node then runs a least square regression to compute the value of $N$ for which the expected distances best fit the empirical average distances, \ie,
\begin{equation}
    \label{equ:netsize-least-squares}
    \hat{N} = \arg\min_{N} \sum_{i=1}^k \left(\overline{D}_i - \frac{2^{256}i}{N+1}\right)^2.
\end{equation}
The resulting estimate $\hat{N}$ can be computed in closed form.
%

When a node starts running, it must perform DHT queries for a few random keys to initialize its network size estimate. 
Since a larger number of queries will result in higher accuracy, making more queries than what is needed to initialize one's routing table is recommended.
Thereafter, keeping the estimate up-to-date does not require any excess DHT queries beyond what is already used for refreshing the routing table as this is done frequently (every 10 minutes).

While the network size estimate has a stochastic variance resulting from the probability distribution of the honest peer IDs, it is hard for an attacker to bias the estimate significantly. Since the estimator uses the density of peer IDs around keys chosen uniformly at random, the adversary would require numerous Sybil nodes (on the order of the whole network size) to significantly affect the peer ID density around those keys.

% !TeX root = ../MVFD_arxiv.tex


\section{The framework}\label{sec2}
In this section we  present a formal mathematical setup for the local regularity  for bivariate   stochastic processes  (also called scalar random fields, or simply random fields) and the data observed for such processes.

\subsection{Data}\label{sec:data}
Consider $N$ independent realizations, also called sheets, $X^{(1)},\ldots,X^{(j)},\ldots X^{(N)}$   of a stochastic process $X $ defined on a continuous domain $\cT\in\mathbb R^2$. For simplicity, we here focus on domains $\mathcal T$ in the plane, the extension to higher dimensions would not raise different challenges. For the purpose of describing our methodology, we distinguish three observational scenarios of the $N$ realizations. 
First, the ideal, infeasible situation where the sheets $X^{(j)}$ are \emph{completely observed}, that is without error over the entire domain $\cT$.  
The second case is the one where the $X^{(j)}$ are observed (measured) at some \emph{discrete points} in the domain $\cT$, \emph{without noise}. The domain points can be fixed to be the same for all the $X^{(i)}$'s (common design), or can be randomly drawn for each sheets separately (independent design). Finally, the most realistic scenario is the one where in the second case we admit that the realizations of $X$  are observed at discrete domain points \emph{with noise}. 



 To formally describe the second and third scenarios, let  $M_1, \dotsc,M_N$ be an independent sample of an integer-valued random variable $M$ with expectation $\EE[M]=\Mmu$. 
% which increases with $N$. 
In the independent design case, for each $1\leq j \leq N$, and given  $M_j$, let $\Tnm\in\cT$,  $1\leq m \leq  M_j$, be a random sample of a random vector $\TT\in\cT$. The $\Tnm$'s represent the observation points for the realization $\Xp{j}$. We assume that the realizations of $X$, $M$ and $\TT$ are mutually independent. 
In the common design case, $M\equiv \mathfrak m$ and the $\Tnm$'s are the same for all $j$.  Let $\mathcal T_{obs}^{(j)} $ denote the set of observation times $ \Tnm $, $1\leq m \leq M_j$, on the sheet $\Xp{j}$. With common design,  $\mathcal T_{obs}^{(j)} $ does not depend on $j$, while with independent design the expected cardinal of $\mathcal T_{obs}^{(j)} $ can be random with mean $\mathfrak m$.   The following presentation  includes both independent design and common design cases. \color{black}  Finally, the data  consist of  the pairs  $(\Ynm , \Tnm ) \in\mathbb R \times \cT $ where $\Ynm$ is defined as
\begin{equation}\label{model_eq}
	\Ynm = \Xn (\Tnm) + \enm, \quad\text{with}  \quad  \enm = \sigma(\Tnm,\Xn(\Tnm)) \unm, 
	\quad 1\leq i \leq N,  \; 1\leq m \leq M_j.
\end{equation}
Here, the $\unm  \in\mathbb R $ are independent copies of a centered variable $e$ with unit variance, and $\sigma^2(\cdot,\cdot)\geq 0$ is some unknown, bounded conditional variance  function which account for possibly heteroscedastic measurement errors. The case  $\sigma(t,x)\equiv 0$ corresponds to our second scenario, while in the third scenario we have positive conditional variance. 

For each $1\leq j \leq N$,  let $\widetilde X^{(j)}$ denote an observable approximation of $X^{(j)}$. If   the sheets $X^{(j)}$ were  completely observed, as in our infeasible first scenario, $\widetilde X^{(j)} = X^{(j)}$. 
When $X^{(j)}$ are observed  only at  some discrete points $\Tnm$,  arbitrary  $\widetilde X^{(j)}(\Tt)$ can be obtained by simple interpolation or defined equal to the value of $\widetilde X^{(j)}$ at the nearest neighbor of $\Tt$.  Finally, with noisy, discretely observed sheets,   $\widetilde X^{(j)}$  is a pilot nonparametric estimator  of $X^{(j)}$, such as kernel smoothing, splines \emph{etc}.  


Let us next introduce a general class of stochastic processes (random fields)  $X$ with  irregular realizations $X^{(j)}$, for which the regularity can vary over the domain $\mathcal T$. 


\subsection{A class of multivariate processes}
Let $\mathcal{T}$ be an open, bounded bi-dimensional rectangle with the closure included in $(0,\infty)^2$. In the following, $H_1,H_2 : \mathcal T \to (0,1)$ are two continuously differentiable functions such that 
\begin{equation}\label{low_thres}
\min_{i=1,2} \inf_{t\in\mathcal T} H_i(\Tt) >0.
\end{equation} 	
Let $\overline{H} = \max\{H_1 ,H_2\}$.
We also consider the vector-valued function $\mathbf{L}=(L_1^{(1)},L_2^{(1)},L_1^{(2)},L_2^{(2)}),$  where the components  are  non-negative, Lipschitz  continuous functions defined on $\mathcal{T}$ such that 
\begin{equation}\label{id_L}
	L_i^{(1)}(\Tt) +L_i^{(2)}(\Tt) >0,\qquad \forall \Tt\in\mathcal T, \; i=1,2.
\end{equation}
%We denote $\mathbf{L}=(L_1^{(1)},L_2^{(1)},L_1^{(2)},L_2^{(2)}),$ and assume that a constant $C_{\mathbf L}$ exists such that 
%\begin{equation}\label{up_thres}
%C_{\mathbf{L}}=\max_{i,j=1,2} \sup_{t\in\mathcal T} L_j^{(i)}(\Tt) <\infty.
%\end{equation} 	

Let $X$ be a real-valued, second order stochastic process defined on $(0,\infty)^2$. Let $(e_1,e_2)$ be the canonical basis of $\mathbb R ^2,$ and, for sufficiently small scalars $\Delta$, let   
$$
\theta_{\Tt}^{(i)}(\Delta)=\EE\left[\left\{X\left(\Tt-\frac{\Delta}{2}e_i\right)-X\left(\Tt+\frac{\Delta}{2}e_i\right)\right\}^2\right],\quad i=1,2.
$$ 


\begin{definition}\label{def}
Let $H_1$, $H_2$ satisfy \eqref{low_thres}.	The class $\mathcal {H}^{H_1,H_2}(\mathbf{L},\mathcal{T})$ is the set of stochastic processes $X$ satisfying the following condition:  constants $  \Delta_0, C,\beta>0$ exist such that 
	for any $\Tt\in \mathcal T$ and $ 0<\Delta\leq\Delta_0$,  
	\begin{equation}\label{as_repr}
		\left|\theta_{\Tt}^{(i)}(\Delta)-L_1^{(i)}(\Tt)\Delta^{2H_1(\Tt)} -L_2^{(i)}(\Tt)\Delta^{2H_2(\Tt)}\right|\leq C\Delta^{2 \overline{H}(\Tt)+\beta}, \quad i=1,2.
	\end{equation}
	Let  
	$$
	\mathcal{H}^{H_1,H_2} %=\mathcal{H}^{H_1,H_2}(\mathcal{T})
	=\bigcup_{\mathbf{L}}\mathcal {H}^{H_1,H_2}(\mathbf{L},\mathcal{T}) ,
	$$
	where the union is taken over the set of four-dimensional functions $\mathbf{L}$ with non negative positives Lipschitz  continuous components satisfying \eqref{id_L} 
	%and \eqref{up_thres}. 
	The functions $H_1,H_2$ define the local regularity of the process, while $\mathbf{L}$ represent  the local Hölder constants.
\end{definition}

Definition \ref{def}  is general, and extends the local regularity notion considered by \cite{GKP} for processes defined on a compact interval on the real line.  A main example we have in mind is the  multi-fractional Brownian  sheet (MfBs) with a time-deformation. MfBs  is a generalization of the standard fractional Brownian sheet, where the Hurst parameter is allowed to vary along the  domain. The definition of this general class of  processes and some of their properties are provided in Section \ref{BfMs}.




\subsection{Performance Indicators}\label{subsection:performance_indicators_definition}
We use three categories of indicators to analyze the performance of a scenario definition:

\textbf{Inefficiency Rate}: it is the difference in fuel spent between a traffic scenario in which the aircraft comply to DAA resolution advisories (RA), and the analogous scenario in which each vehicle follows its optimal path, without observation of traffic separation rules. In case of vertical deviations, a higher fuel rate is required for climb maneuvers, a lower fuel rate in descent maneuver, which increase the net total for the mission. The resulting value of this indicator is the average value over all traffic configurations in a scenario set. 

\textbf{Loss of Separation (LoS) Rate}: despite there being several ways of defining traffic separation, we examine just the simplest one, which is checking whether or not the vehicles are separated by at least a fixed \emph{minimum distance}. In order to include both DAIDALUS and ACAS sXu in the same tables, we use one separation distance from each one, respectively: 4,000 ft, which is the Horizontal Miss Distance, or HMD, used in DO-365B \cite{DO_365} to define the so-called \emph{Hazard Alert Zone} (HAZ), associated to the DAA Well-Clear (DWC) concept of separation; and 2,000 ft, used in DO-396 \cite{DO_396}, that defines the Loss of Well-Clear (LoWC) event in relation to large UAVs or manned aircraft. These indicators will denote the rate of scenarios, in a scenario set, where the distance between any aircraft pair fell below the afore mentioned threshold values. 

\textbf{Timeout Rate}: this indicates, in a scenario set, the rate of scenarios where any aircraft exceeded a maximum time without reaching its destination point. As pointed out in section \ref{section:scenario_definitions}, this phenomenon occurs because DAA Resolution Advisories (RAs) cause long chains of maneuvers that extend beyond the energy/fuel allowance of the vehicle, due to shortcomings in coordination. We use a time threshold of 1,000 seconds but, in practice, the timeout would be determined by the energy/fuel capacity of the vehicle.

\textbf{Scenario Computing Time}: the time needed to simulate a single scenario instance, in a single core of an Intel Xeon CPU, discounted the fact that multiple scenario instances can be run in parallel in a multi-core CPU. In our simulated scenarios, the DAA algorithm is called at least each 2 seconds, for each aircraft, but when the aircraft is in avoidance mode, that can happen more often. In the case of ACAS sXu, the requirement of receiving various messages to update a single track contribute to result in multiple calls per simulated second.


\subsection{Scenario Specifications and Labels}
In this study, a scenario specification is defined by features such as: the DAA algorithm used, the dimensionality (2-D or 3-D), if it uses extrinsic priorities or not, the target separation parameter, and possibly other features. The scenario labels used in this section encode these attributes:
\begin{itemize}
\item \texttt{dai\_ip\_2d\_4k}: DAIDALUS without extrinsic priorities, 2-D maneuvering and regular 4 kft Horizontal Miss Distance (HMD);
\item \texttt{dai\_ep\_2d\_4k}: similar to the above, with extrinsic priorities;
\item \texttt{sxu\_ip\_2d\_2k}: ACAS sXu with intrinsic priorities only, 2-D maneuvering and regular 2 kft LoWC threshold;
\item \texttt{sxu\_ep\_2d\_2k}: similar to the above, with extrinsic priorities;
\item \texttt{dai\_ep\_3d\_4k}: similar to \texttt{dai\_ep\_2d\_4k}, with 3-D maneuvering;
\item \texttt{dai\_ep\_2d\_2k}: similar to \texttt{dai\_ep\_2d\_4k}, with HMD reduced to 2 kft.
\end{itemize}

And there are other features and labels that will be mentioned below as needed.

\subsection{Performance Analysis}
The analysis of selected scenario specifications is summarized in table~\ref{table:performance_analysis}. The first notorious observation in this table is the effect of extrinsic priorities to decrease inefficiency. With regards to safety indicators, their effect is mixed, and we have to observe each case separately. In the case of DAIDALUS, priorities decreased the 2 kft LoS rate and, most drastically, the timeout rate, while increased the 4 kft LoS rate. In the case of ACAS sXu, priorities increased both LoS rate indicators, but decreased the timeout rate drastically. Based on our rule of thumb assessment, it can be said that DAIDALUS works better with extrinsic priorities, while ACAS sXu works better without them. We conjecture that the following reasons explain this fact: i) that DAIDALUS has more built-in symmetries than ACAS sXu; ii) that ACAS sXu has already built-in priority rules for multi-aircraft encounters, and extrinsic priorities may contradict with them.

\begin{table}[h]
    \caption{Summary of closed-loop performance indicators per scenario.}
    \label{table:performance_analysis}
    \begin{center}
    \begin{tabularx}{\columnwidth}{|p{0.20\columnwidth}|p{0.11\columnwidth}|p{0.09\columnwidth}|p{0.09\columnwidth}|p{0.1\columnwidth}|p{0.105\columnwidth}|}
        \hline
    Scenario spec. & Ineffici-ency rate & LoS rate 4~kft & LoS rate 2~kft & Timeout rate & Scenario comp. time (s)\\
    \hline
    \texttt{dai\_ip\_2d\_4k} & 9.71\% & 1.4E-2 & 6.5E-5 & 1.6E-2 & 6.5E-2\\
    \texttt{dai\_ep\_2d\_4k} & 4.83\% & 2.4E-2 & 4.1E-5 & 0 & 5.1E-2\\
    \texttt{sxu\_ip\_2d\_2k} & 20.7\% & 8.7E-1 & 4.1E-2 & 1.6E-3 & 7.8E+1\\
    \texttt{sxu\_ep\_2d\_2k} & 10.9\% & 9.0E-1 & 2.0E-1 & 1.8E-5 & 7.8E+1\\
    \texttt{dai\_ep\_3d\_4k} & 4.38\% & 8.9E-6 & 0 & 0 & 7.4E-2\\
    \texttt{dai\_ep\_2d\_2k} & 1.3\% & 9.0E-1 & 1.1E-1 & 0 & 3.9E-2\\
    \hline
    \end{tabularx}
    \end{center}
\end{table}

According to a line of reasoning, it would be expected, that, in the more efficient scenarios, the aircraft fly closer to each other and, therefore, there should be a higher probability of losing separation. But this is not the only principle at play, because, if the aircraft perform deviations with the least extra distance, while keeping separation, they stay less in the air and decrease the total number of conflicts. This becomes more understandable when we compare \texttt{dai\_ep\_2d\_4k} with \texttt{dai\_ep\_3d\_4k}, where the latter achieved a small advantage in efficiency, but a huge one in safety. \texttt{dai\_ep\_3d\_4k} is capable of shortening the total distances, but has a residual cost associated to vertical maneuvers, where the climb maneuvers spend fuel at higher rates. 

It cannot escape from observation that DAIDALUS performed much better than ACAS sXu in almost all indicators. In our opinion, it would be reasonable to expect that ACAS sXu would not excel in the LoS rates, especially that of 4 kft, because its first protection criterion is 2 kft, as a built-in feature. However, with smaller protection volumes, the deviations should be smaller and, by this reasoning, its expected inefficiency would be lower than that of DAIDALUS. But our results show otherwise when we compare the cases of DAIDALUS with those of ACAS sXu. The only case in which ACAS sXu obtained an advantage was for the LoS rates comparison between \texttt{sxu\_ip\_2d\_2k} and \texttt{dai\_ep\_2d\_2k}, which have the same separation target. In any case, the Los rate obtained for ACAS sXu meets the performance requirement of ASTM F3442 \cite{ASTM}, which uses the definition of LoWC Ratio (LR), which is the ratio between the LoS rate of 2 kft shown in table~\ref{table:performance_analysis}, with DAA active, and corresponding LoS rate with DAA inactive, the latter being 0.785 according  to our simulations. Thus, the resulting LR scores for ACAS sXu here are 0.052 and 0.253, respectively for the two ACAS sXu specs, which are well below the value of 0.4 from \cite{ASTM}, and consistent with the performance analysis of \cite{DO_396}. We conjecture that these scores would be lower in a future 3-D scenario spec of a ACAS sXu, by following the same improvement obtained with DAIDALUS. 

\subsection{Possible approximations to the closed-loop behavior}
Trying to alleviate the heavy computational load to simulate large numbers of different traffic configurations, in this multi-aircraft, closed-loop setup, we considered some approximated solutions, such as the use of Deep Neural Networks to emulate the ACAS Xu/sXu behavior, in the lines followed by \cite{Julian2018,Bak2022}. However, the existing solutions that we found available were developed for just one intruder aircraft, so they were not suitable for our study. Another possibility would be the exploitation of symmetry transformations \cite{Sibai2020}, however the history-dependent nature of the ACAS Xu/sXu algorithms, associated to the present closed-loop setup, make this possibility unpractical. Thus, we started exploring simpler ways of deducing closed-loop behavior without having to perform the full simulation of a scenario. So far, we tried to analyze correlations between measures of open-loop maneuvers and the closed loop performance. The features that we explored are:
\begin{itemize}
    \item \textbf{Distance flown until the end of the first deviation maneuver} ($\overline{M/D}$): we consider the total distance flown until a ``Clear-of-Conflict'' (CoC) event happens, that is, after one or more divergent maneuvers start in a scenario instance, we stop the scenario when the first divergent maneuver of any aircraft finishes and that individual aircraft is clear of conflict, the moment from which some decision must be made on how to continue the mission. We count the total number of maneuvers started, and divide it by the sum of the flown distances, across all scenario instances in an execution set associated to a scenario spec.
    \item \textbf{Average angle deviation maneuver} ($\overline{\alpha}$): using the same stopping rule of above, we account the angle difference between the heading angles of the aircraft at the beginning of the divergent maneuver and at the stopping moment. 
\end{itemize}
For each of these measures, we ran the 122,416 traffic configuration instances with the open-loop stopping rule. Here, all the scenario specifications are 2-dimensional, and we use abbreviated lables to achieve a better display in the graph legends. Namely, the scenario specifications in this subsection are defined as:
\begin{itemize}
    \item \texttt{D1}: DAIDALUS without extrinsic priorities and with deterministic sensor data;
    \item \texttt{D2}: DAIDALUS with extrinsic priorities and deterministic sensor data;
    \item \texttt{D3}: DAIDALUS with extrinsic priorities and Sensor Uncertainty Mitigation (SUM). This is a design feature \cite{Narkawicz2018} to mitigate uncertainty in sensor data, as for example, to determine the position of an intruder aircraft;
    \item \texttt{D4}: DAIDALUS with its Horizontal Miss Distance (HMD) set to 200 ft (the standard is 4,000 ft) and uncertain sensor data;
    \item \texttt{X1}: ACAS sXu without extrinsic priorities;
    \item \texttt{X2}: ACAS sXu with extrinsic priorities;
    \item \texttt{X3}: ACAS sXu with scenario downscaled to speed of 43 knots and cell radius of 1 km.
\end{itemize}
We used the values of $\overline{M/D}$ and $\overline{\alpha}$ obtained for each of the specs above as inputs to a linear regressor of inefficiency, as defined in section~\ref{subsection:performance_indicators_definition}, and generated a plot with the pairs of (true, predicted) values from this regression, as shown in fig.~\ref{fig:inefficiency_regression}. The trend line in the figure, which depicts the regressor, seems to represent a strong correlation, which is confirmed by the value of $R^2$. When considering each of the regression inputs separately, we obtain $R^2=0.73$ for $\overline{M/D}$ and $R^2=0.77$ for $\overline{\alpha}$, which show that they contribute with approximately equal predicting power. 
% Figure environment removed

We performed a similar analysis for the LoS indicators, but we found very little correlation, as $R^2=0.11$ for the 2~kft LoS indicator. Nevertheless, it can be concluded that these open-loop measurements are a good proxy for the closed-loop inefficiency, with the advantage that the predictor discounts the bias that may have been introduced by the closed-loop mission management system, which is not part of the DAA specification. A rough estimate for the computing time saved is of 68\% in the inefficiency case.      

\section{Experiments}
% \haizhou{Follow the same way of introduction as we did in Section2.}
% \noindent In this section, we will introduce datasets and experimental setups that we used. Then we evaluate our method, other self-supervised methods, and supervised methods under different distribution shifts (\ie, concept shifts and covariate shifts) under common settings (\ie, transductive, inductive settings). It has to note that we focus on node-level tasks (\eg, node classification) in this work. As for graph-level tasks, we leave it as our future work and some simple experiments can be found in Appendix~\ref{app:graph_classification}. 
In this section, we first introduce the experimental setup including datasets, training, and evaluation protocol in Section~\ref{sec:dataset}~and~\ref{sec:unsupervised}. 
% Next, we present our experimental setup and conduct extensive experiments to evaluate our method in Section~\ref{sec:unsupervised}. 
We then perform an ablation study to demonstrate the effectiveness of each proposed component in Section~\ref{sec:ablation}. 
Additionally, we analyze the impact of important hyper-parameters in Section~\ref{sec:sensitivity}. 
Subsequently, we integrate our method with various encoding models, showcasing the model-agnostic nature of our recipe in Section~\ref{sec:other_models}. 
Finally, we provide some qualitative results such as feature visualization in Section~\ref{sec:vis}.
It is important to note that we focus on node-level tasks (\eg, node classification) in this work. As for graph-level tasks, we leave it as our future work, while some simple experiments are also provided in Appendix~\ref{app:graph_classification}.

\subsection{Datasets}\label{sec:dataset}
There exist some benchmarks for evaluating graph out-of-distribution generalization~\cite{good,ji2022drugood,gds}. 
Among them, GOOD~\cite{good} is the most representative and comprehensive benchmark that curates more diverse graph datasets with diverse tasks, including single/multi-task graph classification, graph regression, and node classification involving more distribution shifts (\ie, concept shifts and covariate shifts). Hence in this work, we follow the evaluation protocol proposed in \cite{good}. Furthermore, we validate the effectiveness of our method in the datasets (\ie, Amazon-Photo, Elliptic) that are used in EERM~\cite{eerm}. The statistics and detailed introduction to these datasets can be found in Table~\ref{tab:dataset} and Appendix~\ref{app:datasets}.

\begin{table*}[htp]
\caption{The descriptions of datasets. ``Domain-Level'' means splitting by graphs, ``Time-Aware'' denotes splitting according to chronological order.``Word'' and ``Degree'' represent splitting according to word diversity and node degree respectively. ``Language'' means splitting by user language, suggesting the prediction should not be impacted by the language the user use. ``University'' denotes splitting according to the domain university, implying that the prediction of webpages should be based on word contents and link connections rather than university features. ``Color'' means that nodes are split according to node differences in covariate shift and color-label correlations in concept shift.}
\label{tab:dataset}
\centering
\begin{tabular}{cccccccc}
\toprule
Datasets     & Network Type        & \#Nodes & \#Edges & \#Attributes &\#Classes& Train/Val/Test Split     & Metric   \\
% Cora         & Artificial Transformation & 2,703   &         &              &         &                      & Accuracy \\
Amazon-Photo\footnotemark
             & Co-purchasing network      & 7,650   & 119,081   & 755          & 10      & Domain-Level         & Accuracy \\
Elliptic\footnotemark  
             & Bitcoin transactions       & 203,769 & 234,355   & 165          & 2       & Time-Aware           & F1-Score \\
GOOD-Cora    & Scientific publications    & 19,793  & 126,842   & 8,710         & 70      & Word/Degree          & Accuracy \\
% GOOD-Arxiv   & arXiv papers               & 169,343 & 2,315,598 & 128          & 40      & Time/Degree          & Accuracy \\
GOOD-Twitch  & Gamer network              & 34,120  & 892,346   & 128          & 2       & Language             & ROC-AUC  \\
GOOD-CBAS    & A BA-house graph           & 700     & 3,962     & 4             & 4       & Color                & Accuracy \\
GOOD-WebKB   & Webpage network            & 617     & 1,138     & 1,703         & 5       & University           & Accuracy \\
\bottomrule
\end{tabular}
\end{table*}
\footnotetext[5]{This dataset is adopted from~\cite{yang2016revisiting}. \cite{eerm} constructs ten graphs with different environment id’s for each graph.} 
\footnotetext[6]{The original is available on \hyperlink{https://www.kaggle.com/ellipticco/elliptic-data-set}{https://www.kaggle.com/ellipticco/elliptic-data-set}}

\subsection{Unsupervised Representation Learning}\label{sec:unsupervised}
\subsubsection{Transductive Setting}~\label{sec:trans}
% \noindent\textbf{Baselines.}\quad We conduct experiments with 12 baselines which consist of three categories: supervised methods and self-supervised generative methods, self-supervised contrastive methods. Specifically, we compare with three supervised baselines: empirical risk minimization~(ERM)~\cite{erm}, invariant risk minimization (IRM)~\cite{irm}, and a recent proposed graph OOD method dubbed EERM~\cite{eerm}. We also compare various unsupervised node-level representation learning methods: three self-supervised generative methods including GAE~\cite{gae}, VGAE~\cite{gae}, GraphMAE~\cite{gmae} and seven self-supervised contrastive methods: DGI~\cite{dgi}, MVGRL~\cite{mvgrl}, GRACE~\cite{grace}, RoSA~\cite{rosa}, BGRL~\cite{bgrl}, COSTA~\cite{costa}, SwAV~\cite{swav}. The descriptions of these methods can be found in Appendix~\ref{app:baselines}.
In this subsection, we focus on validating our proposed algorithm under the transductive setting, where the test nodes will participate in message passing~\cite{gilmer2017neural} during training following~\cite{good}. 

\noindent\textbf{Baselines.} We conduct experiments with 12 baselines from three categories: (i)~supervised methods, including empirical risk minimization~(\textbf{ERM})~\cite{erm}, invariant risk minimization (\textbf{IRM})~\cite{irm}, and a recent proposed graph OOD method \textbf{EERM}~\cite{eerm}; (ii)~self-supervised generative methods including Graph Autoencoder (\textbf{GAE})~\cite{gae}, Variational Graph Autoencoder (\textbf{VGAE})~\cite{gae}, Self-Supervised Masked Graph Autoencoders (\textbf{GraphMAE})~\cite{gmae}; (iii)~self-supervised contrastive methods including Deep Graph Infomax (\textbf{DGI})~\cite{dgi}, Contrastive Multi-View Representation Learning on Graphs (\textbf{MVGRL})~\cite{mvgrl}, Deep Graph Contrastive Representation Learning (\textbf{GRACE})~\cite{grace}, A Robust Self-Aligned Framework for Node-Node Graph Contrastive Learning (\textbf{RoSA})~\cite{rosa}, Bootstrapped Representation Learning on Graphs (\textbf{BGRL})~\cite{bgrl}, Covariance-Preserving Feature Augmentation for Graph Contrastive Learning (\textbf{COSTA})~\cite{costa}, Unsupervised Learning of Visual Features by Contrasting Cluster Assignments (\textbf{SwAV})~\cite{swav}. The detailed descriptions of these baselines can be found in Appendix~\ref{app:baselines}.

\noindent\textbf{Experimental setup.} We use the same graph encoder across different datasets for a fair comparison following~\cite{good}. We use grid search to find other hyper-parameters (\eg, learning rate, epochs) for different methods. For all experiments, we select the best checkpoints for ID and OOD tests according to results on ID and OOD validation sets following~\cite{good}, respectively. Experimental details and hyper-parameter selections are provided in Appendix~\ref{app:hyper}. For evaluating unsupervised methods, a linear classifier will be built on the frozen trained encoder after finishing pre-training. The reported results are the mean performance with standard deviation after 10 runs following~\cite{good}.

\noindent\textbf{Analysis.}\quad Based on the experimental results listed in Table~\ref{tab:trans_concept} and \ref{tab:trans_covariate}, we can draw the following conclusions: firstly, we find strong self-supervised methods (\eg, GRACE, BGRL, COSTA) are more robust to distribution shifts (concept shift in Table~\ref{tab:trans_concept} and covariate shift in Table~\ref{tab:trans_covariate}) compared to supervised methods. For instance, on GOOD-CBAS and GOOD-WebKB datasets, GRACE surpasses the best supervised method by large margins (over 6\% absolute improvement). Interestingly, we find the methods designed for OOD generalization (\ie, IRM) and graph OOD generalization (\ie, EERM) do not attain superior performance than the standard ERM on most of the datasets. For example, EERM shows superior OOD performance compared to ERM in only one experiment, and IRM outperforms ERM in four out of ten experiments across the conducted evaluations. This phenomenon is also observed in \cite{good,ahuja2020empirical,rosenfeld2021risks}, showcasing the challenge of achieving invariant prediction in non-Euclidean graph settings. 

Furthermore, our method surpasses other SOTA self-supervised methods on the OOD test set of all datasets by a considerable margin while achieving comparable performance in the in-distribution test set. For instance, on small datasets such as GOOD-CBAS and GOOD-WebKB, our method outperforms GRACE\footnote{MARIO is built up on GRACE according to our recipe. So, we make a comparison with GRACE here.} by over 2\% absolute accuracy on the OOD test set. On larger datasets such as GOOD-Cora and GOOD-Twitch, our method still outperforms other methods which shows its superiority. For instance, under covariate shift, MARIO surpasses other methods by over 7\% absolute accuracy on the GOOD-Twitch OOD test set. These statistics prove the effectiveness of our design.


\begin{table*}[htp]
\caption{Experimental results of all methods under concept shift. The bold font means the top-1 performance and the underline represents the second performance across the unsupervised methods. 'ID' represents in-distribution test performance and 'OOD' means out-of-distribution test performance. (OOM: out-of-memory on a GPU with 24GB memory)}
\label{tab:trans_concept}
\centering
\scalebox{0.95}{
\begin{tabular}{l|cc|cc|cc|cc|cc}
\toprule
\toprule
\multirow{3}{*}{concept shift} & \multicolumn{4}{c|}{GOOD-Cora}                   & \multicolumn{2}{c|}{GOOD-CBAS} & \multicolumn{2}{c|}{GOOD-Twitch} & \multicolumn{2}{c}{GOOD-WebKB} \\
                           & \multicolumn{2}{c}{word} & \multicolumn{2}{c|}{degree}& \multicolumn{2}{c|}{color}    & \multicolumn{2}{c|}{language}   & \multicolumn{2}{c}{university} \\
                           & ID         & OOD         & ID          & OOD          & ID            & OOD           & ID             & OOD            & ID            & OOD            \\
\midrule
ERM                        & 66.38±0.45 & 64.44±0.18  & 68.60±0.40  & 60.76±0.34   & 89.79±1.39    & 83.43±1.19    & 80.80±1.00     & 56.92±0.92     & 62.67±1.53    & 26.33±1.09     \\
IRM                        & 66.42±0.41 & 64.29±0.31  & 68.57±0.35  & 61.45±0.24   & 89.64±1.21    & 82.29±1.14    & 78.87±1.04     & 59.30±1.79     & 62.67±1.10    & 26.88±1.42     \\
EERM                       & 65.10±0.44 & 62.45±0.19  & 66.95±0.44  & 56.58±0.25   & 79.07±2.12    & 64.50±1.01    & OOM            & OOM            & 62.50±2.01    & 28.07±3.23      \\
\midrule
% Random-Init                & 37.53±1.74 & 32.12±1.24  & 37.82±1.71  & 27.74±1.14   &               &               &                &                & 60.33±2.21    & 27.07±1.70     \\
GAE                        & 60.65±0.89 & 58.00±0.55  & 62.59±1.11  & 53.44±0.80   & 75.28±1.36    & 68.07±2.05    & 81.25±0.81     & 51.51±1.05     & 62.17±3.34    & 25.78±1.85     \\
VGAE                       & 63.19±0.53 & 60.35±0.47  & 61.65±0.66  & 54.28±0.28   & 76.50±0.50    & 59.07±0.56    & 80.46±0.53     & 55.56±4.53     & 62.50±2.38    & 24.40±2.57     \\
GraphMAE                   & \underline{66.44±0.46} & \underline{64.87±0.30}  & 67.95±0.46  & 59.41±0.39   & 89.14±0.89    & 82.93±0.93    & 80.05±0.64     & 59.38±1.49     & 61.83±3.37    & 29.27±2.15     \\
DGI                        & 63.33±0.56 & 60.71±0.49  & 65.93±1.02  & 55.83±0.53   & 91.22±1.47    & 85.00±1.66    & 80.05±0.87     & 59.16±1.88     & 61.83±2.83    & 28.63±1.92      \\
MVGRL                      & OOM        & OOM         & OOM         & OOM          & 88.57±1.15    & 76.50±1.17    & OOM            & OOM            & 62.00±3.79    & 28.26±4.20     \\
GRACE                      & 65.61±0.61 & 63.92±0.44  & \textbf{68.59±0.35}  & 60.15±0.45   & 92.00±1.39    & 88.64±0.67    & \textbf{83.43±0.63}     & \underline{60.45±1.46}     & 64.00±3.43    & \underline{34.86±3.43}  \\
RoSA                       & 64.06±0.67 & 62.44±0.39  & 67.07±0.65  & 57.68±0.44   & 90.78±2.27    & 85.93±2.14    & 82.39±0.42     & 57.45±2.16     & 64.17±4.10    & 32.20±2.15     \\
BGRL                       & 65.18±0.43 & 63.43±0.45  & 66.83±0.80  & 59.63±0.38   & 92.36±1.16    & 87.14±1.60    & 82.52±0.60     & 55.48±1.48     & 63.67±2.33    & 31.47±3.43     \\
COSTA                      & 65.05±0.80 & 62.37±0.45  & 66.76±0.87  & 55.73±0.36   & \underline{93.50±2.62}    & \underline{89.29±3.11}    & 83.15±0.30 & 55.03±3.22     & 61.66±2.58    & 32.39±2.13 \\
% ArCL                       &            &             & 67.64±0.57  & 59.71±0.44   &               &               &                &                & 65.00±3.94    & 35.41±1.97 \\      
SwAV                       & 62.22±0.53 & 59.79±0.53  & 64.65±0.94  & 55.06±0.39   & 89.00±0.79    & 81.72±0.66    & \underline{83.32±0.15}     & 59.69±1.97     & \underline{65.17±3.76}    & 29.36±2.01    \\
\midrule
MARIO                       & \textbf{67.11±0.46} & \textbf{65.28±0.34}  & \underline{68.46±0.40}  & \textbf{61.30±0.28}   & \textbf{94.36±1.21}    & \textbf{91.28±1.10}    & 82.31±0.54     & \textbf{63.33±1.72}     & \textbf{65.67±2.81}    & \textbf{37.15±2.37}     \\
\bottomrule
\end{tabular}}
\end{table*}

\begin{table*}[htp]
\caption{Experimental results of all methods under covariate shift. The bold font means the top-1 performance and the underline represents the second performance across the unsupervised methods. 'ID' represents in-distribution test performance and 'OOD' means out-of-distribution test performance. (OOM: out-of-memory on a GPU with 24GB memory)}
\label{tab:trans_covariate}
\centering
\scalebox{0.95}{
\begin{tabular}{l|cc|cc|cc|cc|cc}
\toprule
\toprule
\multirow{3}{*}{covariate shift} & \multicolumn{4}{c|}{GOOD-Cora}                                   & \multicolumn{2}{c|}{GOOD-CBAS} & \multicolumn{2}{c|}{GOOD-Twitch} & \multicolumn{2}{c}{GOOD-WebKB} \\
                           & \multicolumn{2}{c}{word} & \multicolumn{2}{c|}{degree}& \multicolumn{2}{c|}{color}    & \multicolumn{2}{c|}{language}   & \multicolumn{2}{c}{university} \\
                           & ID         & OOD         & ID          & OOD          & ID            & OOD           & ID             & OOD            & ID            & OOD            \\
\midrule
ERM                        & 70.50±0.41 & 64.69±0.33  & 72.46±0.49  & 55.53±0.50   & 92.00±3.08    & 77.57±1.29    & 70.98±0.41     & 49.35±5.09     & 39.34±1.79    & 14.52±3.14   \\
IRM                        & 70.48±0.26 & 64.53±0.57  & 71.98±0.34  & 53.72±0.46   & 90.86±2.41    & 78.86±1.67    & 69.81±0.95     & 49.11±2.82     & 38.52±3.30    & 13.97±2.80     \\
EERM                       & OOM        & OOM         & OOM         & OOM          & 65.00±2.57    & 57.43±3.60    & OOM            & OOM            & 46.07±4.55    & 27.40±7.65     \\
\midrule
GAE                        & 56.63±0.79 & 48.93±0.93  & 66.30±0.88  & 34.01±0.87   & 73.00±2.16    & 60.86±3.01    & 67.24±1.23     & 47.65±2.49     & 45.08±6.32    & 28.02±6.29    \\
VGAE                       & 62.02±0.66 & 54.12±0.86  & 69.41±0.57  & 44.20±1.29   & 62.29±2.04    & 63.29±1.11    & 66.99±1.43     & \underline{50.48±4.58}     & 48.85±4.68    & 20.87±6.69     \\
GraphMAE                   & 68.14±0.43 & 64.00±0.33  & \textbf{73.36±0.56}  & 53.75±0.55   & 67.28±3.03    & 67.28±1.49    & 68.84±1.20     & 48.02±2.79     & 48.03±4.34    & 30.00±8.09     \\
DGI                        & 60.85±0.75 & 57.03±0.67  & 68.97±0.41  & 41.75±0.88   & 69.57±4.09    & 59.71±3.43    & 68.43±1.05     & 44.83±1.61     & 48.52±5.04    & 21.11±7.50     \\
MVGRL                      & OOM        & OOM         & OOM         & OOM          & 65.00±1.94    & 64.15±0.77    & OOM            & OOM           & \textbf{54.10±5.39}    & 16.59±6.51     \\
GRACE                      & \underline{68.77±0.33} & \underline{64.21±0.41}  & 72.69±0.34  & \underline{56.10±0.63}   & \underline{93.57±1.83}    & \underline{89.29±3.40}    & \underline{71.12±0.87} & 46.21±1.54 & 49.67±5.82    & 28.10±4.68    \\
RoSA                       & 68.19±0.56 & 62.48±0.61  & 71.04±0.62  & 52.72±0.79   & 84.71±4.14    &79.14±3.51     & 70.58±0.36     & 45.83±1.72     & 52.30±4.24    & \underline{34.24±7.92}     \\
BGRL                       & 67.23±0.43 & 61.33±0.36  & 72.11±0.39  & 49.15±0.73   & 89.00±2.56    & 79.86±3.29    & \textbf{71.43±0.53}     & 43.86±0.94     & 51.80±5.55    & 30.32±7.61    \\
COSTA                      & 65.28±0.60 & 60.33±0.53  & 70.65±0.62  & 54.03±0.28   & 92.29±1.59    & 82.71±2.74    & 69.29±1.37     & 49.07±2.13     & 50.49±3.01    & 29.84±4.75   \\
SwAV                       & 63.29±1.01 & 56.98±0.94  & 70.27±0.73  & 43.00±0.52   & 89.57±1.12    & 81.43±1.69    & 69.19±0.93     & 49.37±2.96     & 49.84±4.82    & 30.55±6.72   \\
\midrule
MARIO                       & \textbf{69.99±0.54} & \textbf{65.06±0.34}  & \underline{72.73±0.43}  & \textbf{57.73±0.45}  & \textbf{94.57±2.46}    & \textbf{91.00±2.48}     & 68.31±0.78 & \textbf{57.37±1.37}     & \underline{53.94±3.23}    & \textbf{35.24±4.98}   \\
\bottomrule
\end{tabular}}

\end{table*}

\subsubsection{Inductive Setting}
In this subsection, we conduct experiments under the inductive settings, where the test nodes are kept unseen during training. This setting is more suitable for domain generalization.
% But we think it is more convincing that conduct experiments under inductive settings which means test nodes are unseen during training. This setting is more appropriate for domain generalization.

\noindent\textbf{Baselines:} For GOOD-WebKB and GOOD-CBAS datasets, we adopt ERM, IRM, GraphMAE, and GRACE as our baselines. And for Amazon-Photo and Elliptic datasets, we select ERM, EERM, and GRACE as our baselines.

\noindent\textbf{Experimental setup:} For GOOD-WebKB and GOOD-CBAS datasets, we use the same model configuration in Section~\ref{sec:trans}.
% Besides, we add experiments on Amazon-Photo dataset~\cite{yang2016revisiting} and Elliptic~\cite{elliptic} dataset in this subsection. 
For Amazon-Photo dataset~\cite{yang2016revisiting} and Elliptic~\cite{elliptic} dataset, they consist of many snapshots (training data and testing data use different snapshots) which are naturally inductive. For Amazon-Photo dataset, we use 2-layer GCN~\cite{gcn} as the encoder and for elliptic dataset, we use 5-layer GraphSAGE~\cite{sage} as encoder following~\cite{eerm}.

% Figure environment removed

\noindent\textbf{Analysis:}
According to Figure~\ref{fig:amazon},\ref{fig:elliptic},\ref{fig:ind_con},\ref{fig:ind_cov}, we can draw following conclusions:
firstly, based on Figure~\ref{fig:amazon}, it is evident that our method outperforms other representative supervised and self-supervised methods on all test graphs (T1$\sim$T8). This superiority is reflected in the larger median value of our method compared to others. For instance, MARIO achieves over a 3\% absolute improvement compared to ERM in terms of the mean value of eight median values. Additionally, our method demonstrates higher stability across different random initializations, as indicated by the closer proximity of the first and third quartile values to the median value~(\eg, the difference of first and third quartile values of ERM, EERM, GRACE and MARIO are 4.2, 3.3, 6.7 and 1.0 on T8 respectively which indicates MARIO is much more stable than other methods). Furthermore, our method exhibits consistent performance across different graphs (\eg, The standard deviation of median values on T1$\sim$T8 for ERM, EERM, GRACE, and MARIO are 0.4, 1.1, 1.2, and 0.3, respectively.), indicating its robustness to environmental variations and its ability to extract invariant features: $g(G^e) \approx g(G^{e'})$ for all $e, e' \in \mathcal{E}^\text{train}$. In summary, our method showcases enhanced OOD generalization capabilities.
% $g(G^e)g(G^e^\prime)$ where $any e, e^\prime in \mathcal{E}^{train}$

Secondly, from the results presented in Figure~\ref{fig:elliptic}, we can observe that our method averagely harvests 10.9\% absolute improvement over GRACE and 12.5\% absolute improvement over EERM in terms of F1 scores on Elliptic dataset. This demonstrates the effectiveness of our method in handling distribution shifts and improving performance compared to existing approaches. It is worth noting that GRACE's performance worsens over time, indicating its inability to handle distribution shifts effectively. In contrast, our method consistently achieves better F1 scores, except for T9, which is caused by the dark market shutdown occurred after T7~\cite{elliptic}. The emergence of such an event introduces significant variations in data distributions, which subsequently results in performance degradation for all methods. Indeed, this event serves as an unpredictable external factor that introduces significant challenges for models trained on limited training data. The results indicate that the performance heavily depends on available training data. Nonetheless, our approach outperforms other methods even in such an extreme case. This highlights the effectiveness of our method in addressing distribution shifts and improving generalization performance.

Finally, based on the observations from Figure~\ref{fig:ind_con} and Figure~\ref{fig:ind_cov} MARIO demonstrates the best performances on both ID and OOD test sets for GOOD-WebKB and GOOD-CBAS datasets, under both concept shift and covariate shift. Notably, MARIO outperforms other methods by more than 3\% and 10\% absolute improvement on GOOD-WebKB and GOOD-CBAS, respectively, under covariate shift. We can draw similar conclusions as discussed in Section~\ref{sec:trans}. Even under the inductive setting, our method continues to demonstrate excellent OOD generalization capabilities and achieves comparable or even improved in-distribution test performance. These statistical results further validate the effectiveness of our method in handling distribution shifts and enhancing generalization performance.

Overall, the observations we have made provide strong evidence of the great capacity of our method for handling distribution shifts, validating its effectiveness and potential for real-world applications.



% Figure environment removed

% Figure environment removed


% Figure environment removed


\subsection{Ablation Studies}\label{sec:ablation}
\noindent Table~\ref{tab:aba} provides a detailed analysis of the effect of each component according to our proposed recipe for improving OOD generalization in graph contrastive learning. Let's examine the different variants of our method and their impact on performance.
Specifically, MARIO~(w/o ad) represents MARIO without  adversarial augmentation. MARIO~(w/o cmi) denotes we only maximize the mutual information between positive pairs without considering conditional mutual information. MARIO~(w/o cmi, ad) means a vanilla graph contrastive method that is similar to GRACE. 

From Table~\ref{tab:aba}, we can find MARIO~(w/o cmi) lags far behind MARIO on OOD test set which demonstrates appropriately minimizing the redundant information (\ie, conditional mutual information) is essential to improve OOD generalization of GCL methods. And adversarial augmentation can also boost OOD generalization because it can approximately serve as a supermum operator to learn more invariant features  discussed in Section~\ref{sec:aug}. Based on the analysis of these variants, it is evident that the proposed improvements on data augmentation and contrastive loss in the recipe are both effective in enhancing graph OOD generalization. Each component contributes to the overall performance improvement, and their combination leads to a stronger self-supervised graph learner in terms of graph OOD generalization. 

In short, the findings from Table~\ref{tab:aba} support the rationale behind your proposed recipe and provide empirical evidence of the effectiveness of each proposed component. By incorporating these enhancements, our method achieves superior performance in handling distribution shifts and improving graph OOD generalization in graph contrastive learning.
\begin{table*}[htp]
\caption{Ablation studies for MARIO by masking each component.}
\label{tab:aba}
\centering
\scalebox{0.9}{
\begin{tabular}{l|cc|cc|cc|cc|cc}
\toprule
\toprule
\multirow{3}{*}{concept shift} & \multicolumn{4}{c|}{GOOD-Cora}                       & \multicolumn{2}{c|}{GOOD-CBAS} & \multicolumn{2}{c|}{GOOD-Twitch} & \multicolumn{2}{c}{GOOD-WebKB} \\
                           & \multicolumn{2}{c}{word} & \multicolumn{2}{c|}{degree}& \multicolumn{2}{c|}{color}    & \multicolumn{2}{c|}{language}   & \multicolumn{2}{c}{university} \\
                           & ID         & OOD         & ID          & OOD          & ID            & OOD           & ID             & OOD            & ID            & OOD            \\
\midrule
MARIO                      & \textbf{67.11±0.46} & \textbf{65.28±0.34}  & \textbf{68.46±0.40}  & \textbf{61.30±0.28}      & \textbf{94.36±1.21}  & \textbf{91.28±1.10}    & 82.31±0.54     & \textbf{63.33±1.72}     & \textbf{65.67±2.81}    & \textbf{37.15±2.37}     \\
MARIO(w/o ad)              & 66.23±0.53 & 64.02±0.18  & 67.88±0.38  & 60.46±0.29   & 93.21±1.25    & 90.29±0.91    & 82.42±0.73     & 60.50±1.02     & 64.83±2.83    & 36.51±3.25    \\
MARIO(w/o cmi)             & 65.32±0.60 & 63.51±0.32  & 68.14±0.32  & 61.19±0.34   & 94.15±1.23    & 90.57±1.96    & \textbf{82.51±0.56}     & 61.41±2.63     & 64.50±4.35    & 35.78±2.53     \\
MARIO(w/o cmi, ad)         & 64.67±0.55 & 63.11±0.32  & 67.95±0.65  & 60.01±0.57   & 93.36±1.66    & 89.64±1.73    & 81.90±0.75     & 60.12±1.60     & 64.17±3.67    & 34.13±2.38     \\
\bottomrule
\end{tabular}}
\end{table*}
% & 65.32±0.60 & 63.51±0.32 exchange 64.67±0.55 & 63.11±0.32
% 68.14±0.32       id ood test: 60.95±0.43       ood ood test: 61.19±0.34


\subsection{Sensitivity Analysis}\label{sec:sensitivity}
\noindent In this subsection, we will analyze some important hyper-parameters of our method. We conduct sensitivity analysis on GOOD-WebKB dataset with concept shift, we chose two sensitive hyper-parameters (\ie, the coefficient $\gamma$ of condition mutual information in Equation~\ref{equ:cmi} and the number of prototypes $|C|$ in Equation~\ref{equ:pq}). The coefficient of CMI range in $[0.001, 0.01, 0.1, 0.5, 1]$ and the number of prototypes $|C|$ ranges in $[10, 50, 100, 200, 300]$. From Figure~\ref{fig:sensitivity}, we can observe that $\gamma$ reaches 0.1 and $|C|$ reaches 100 or 200 can achieve the best OOD test accuracy. Both higher and lower values of $\gamma$ result in suboptimal performance. This finding aligns with previous research such as DIB~\cite{dib}, indicating that an appropriate compression level is crucial for achieving optimal performance. Extremely high or low compression values are not ideal. 

Regarding the number of prototypes $|C|$, based on the results shown in Figure~\ref{fig:sensitivity}, it is found that setting $|C|=100$ leads to the best performance in terms of OOD test accuracy. This choice provides a moderate number of pseudo labels, which is beneficial for the learning process. 

Based on the sensitivity analysis, we determined that setting $\gamma=0.1$ and $|C|=100$ on most datasets. These hyperparameter values strike a balance between compression level and the number of prototypes, resulting in improved graph OOD generalization.
% Figure environment removed


\subsection{Integrated with Other Models}\label{sec:other_models}
% Figure environment removed

\begin{table}[htp]
\caption{Results of different learning approaches with different encoding models (\ie, GCN, GraphSAGE, GAT).}
\label{tab:others}
\centering
\scalebox{0.9}{
\begin{tabular}{cc|cc|cc}
\toprule
\toprule
\multirow{3}{*}{Model}& \multirow{3}{*}{Method} & \multicolumn{2}{c|}{GOOD-CBAS} & \multicolumn{2}{c}{GOOD-WebKB} \\
                & & \multicolumn{2}{c|}{color}    & \multicolumn{2}{c}{university} \\
                &   & ID          & OOD         & ID          & OOD            \\
\midrule
\multirow{3}{*}{GCN} 
&ERM               & 89.79±1.39 & 83.43±1.19  &  62.67±1.53 & 26.33±1.09         \\
&GRACE             & 92.00±1.39 & 88.64±0.67  &  64.00±3.43 & 34.86±3.43        \\
&MARIO             & 94.36±1.21 & 91.28±1.10  &  65.67±2.81 & 37.15±2.37        \\ \bottomrule
\multirow{3}{*}{SAGE} 
&ERM               & 95.07±1.51 & 75.14±1.19  & 73.67±2.08  & 46.33±3.42       \\
&GRACE             & 95.29±1.11 & 74.43±2.36  & 70.50±5.06  & 49.54±3.83        \\
&MARIO             & 96.00±1.07 & 76.29±3.01  & 71.00±3.82  & 51.74±4.63        \\ \bottomrule
\multirow{3}{*}{GAT} 
&ERM               & 78.64±3.63 & 72.93±2.64  & 61.33±3.71  & 28.99±2.63        \\
&GRACE             & 84.57±1.79 & 78.36±1.60  & 59.50±2.36  & 35.78±3.26        \\
&MARIO             & 84.93±1.95 & 80.43±1.89  & 62.17±4.78  & 38.17±3.10        \\
\bottomrule
\end{tabular}}
\end{table}



\noindent In the subsection, we demonstrate the model-agnostic nature of the recipe by integrating it with various graph neural network (GNN) models, including GCN, GraphSAGE, and GAT.

From Table~\ref{tab:others}, it can be observed that regardless of the specific GNN model used as the encoder, our method consistently achieves the best performance on the OOD test set. This indicates the effectiveness and robustness of our method across different GNN models.
By achieving superior performance across different GNN models, MARIO demonstrates its versatility and ability to improve the OOD generalization of various graph neural models. This highlights the broad applicability and effectiveness of our recipe in enhancing the performance of different GNN encoders.

Furthermore, we integrate our recipe with other GCL methods in Appendix~\ref{app:other_methods}. The results demonstrate our recipe can boost the OOD generalization ability of various GCL methods which means our recipe can serve as a plug-in for many current classical GCL methods.

% Figure environment removed

\subsection{Visualization}\label{sec:vis}
\subsubsection{Metric Score Curves}
We present metric score curves for ERM and MARIO, including training, ID validation, ID testing, OOD validation, and OOD testing accuracy, in Figure~\ref{fig:curve2}. Notably, MARIO demonstrates superior convergence with approximately 10\% absolute improvement on the OOD test set compared to ERM. Furthermore, MARIO effectively narrows the performance gap between in-distribution and out-of-distribution performance, showcasing its efficacy in enhancing OOD generalization for graph data. More metric score curves can be found in Appendix~\ref{app:curves}.


\subsubsection{Feature Visualization}
In order to assess the quality of learned embeddings, we adopt t-SNE~\cite{tsne} to visualize the node embedding on GOOD-Cora dataset (concept shift in word domain) using random-init of GCN, EERM, GRACE, and MARIO, where different classes have different colors in Figure~\ref{fig:vis}. For clarity, we select eight classes with the largest number of nodes to enhance the informativeness and interpretability of the visualization. We can observe that the 2D projection of node embeddings learned by MARIO has a better separation of clusters, which indicates the model can help learn representative features for downstream tasks. It has to note that we depict both ID nodes and OOD nodes in the same figure. 

Besides, we also separately visualize ID nodes and OOD nodes in the different figures in the Appendix~\ref{app:feature}. And we can find MARIO performs a clearer separation of clusters whether on ID nodes or OOD nodes compared to other methods.



\section{Conclusion and Future Work}
In this work, I design corruption-robust algorithms for the Lipschitz contextual search problem. I present the \emph{agnostic checking} technique and demonstrate its effectiveness in designing corruption-robust algorithms. There are several open problems for future research. First, in the algorithm I propose for pricing loss, the schedule for agnostic checks is fixed upfront. Can the learner design an adaptive checking schedule for the pricing loss? Second, this work assumes the learner has knowledge of the Lipschitz constant $L$. Can the learner design efficient no-regret algorithms without knowledge of $L$? 

\section*{Acknowledgements}
We are grateful for the support of the NSF under CCF2106707 (Program Synthesis for Weak Supervision) and the Wisconsin Alumni Research Foundation (WARF).

%  
%  \section{Learning Structures in the Weak Supervision Regime}
%  \label{sec:theory}
%  \section{Theoretical Analysis}
Assuming $X=\left\{x_1, x_2, \cdots, x_n\right\}, Y=\left\{y_1, y_2, \cdots, y_n\right\}, A=\left\{a_1, a_2, \cdots, a_n\right\}$ are node attribute set, node label set and node neighbor set respectively. Homophily is essentially a strong relevance between neighbors and labels, that is, neighbors can be determined when labels are known and labels can be determined when neighbors are given. Thus we give the following proposition:

\begin{proposition}\label{proposition:1}
Given a high homophily level graph $G$, conditional entropies $H(A|Y)$ and $H(Y|A)$ between its neighbor set $A$ and node label set $Y$ fulfill:

\begin{equation}
    0 \leq H(A|Y) \leq \epsilon \quad 0 \leq H(Y|A) \leq \epsilon,
\end{equation}

where $\epsilon\rightarrow0+$. 
\end{proposition}

Proposition.~\ref{proposition:1} implies the label of a node can be predicted or determined by the neighbor set of this node, which is similar to the assumption of GNNs. Therefore, the conditional entropies $H(A|Y)$ and $H(Y|A)$ tend to be 0. The proof of this proposition is provided in Appendix.~\ref{proof: proposition 1}. The prediction of GNN-based model $f$ can be represented as:

\begin{equation}
    \hat{Y} = f(X,A),
\end{equation}

which means model $f$ accepts $X$ and $A$ as inputs and outputs the predict label $\hat{Y}$. A well-trained GNN model is able to predict labels very certainly based on the node attributes and neighbors. Since entropy is a measurement of uncertainty, thus we hold:

\begin{proposition}\label{proposition:2}
For a well-trained GNN model $f^{*}$, the conditional entropy $H(Y|f^{*}(X,A))$ between the model output $f^{*}(X,A)$ and label $Y$ fulfill:

\begin{equation}
    0 \leq H(Y|f^{*}(X,A)) \leq \epsilon,
\end{equation}

where $\epsilon\rightarrow0+$. 
\end{proposition}

Proposition.~\ref{proposition:2} assuming a well-trained GNN model can perceive the distributions of attributes $X$ and neighbor $A$, thus learning label-related information and predicting labels accurately. The proof of Proposition.~\ref{proposition:2} is based on cross-entropy loss and is shown in Appendix.~\ref{proof: proposition 2}. Conversely, the target of Meta-attack is inducing the well-trained model $f^{*}$ to give false predictions by removing edges, i.e. perturbing the distribution of neighbor set $A$. After perturbing, $f^{*}$ predicts labels uncertainly, thus we have the following lemma:

\begin{lemma}\label{lemma:1}
    After the attack, the perturbed neighbor set $A^{'}$ meets inequality:
    
    \begin{equation}
        H(Y|X,A) < H(Y|X,A^{'})
    \end{equation}
\end{lemma}

Lemma.\ref{lemma:1} indicates labels can not be predicted or determined via the distributions of $X$ and $A^{'}$. Therefore, the well-trained model $f^{*}$ learns label irrelevance information from distributions and suffers a deterioration performance. The proof of Lemma.\ref{lemma:1} can be found in Appendix.~\ref{proof: lemma 1}.

To against the attack and perturbation, we need to minimize $H(Y|X,A^{'})$ thus the model can learn proper information from the distributions. However, for unsupervised models such as GCL, the label set $Y$ is unknown and results in the impossibility of calculation $H(Y|X,A^{'})$. 

Fortunately, we can utilize strong relevance between neighbor set $A$ and label set $Y$ shown in Proposition.~\ref{proposition:1} to minimize $H(Y|X,A^{'})$ indirectly. Specifically, we select neighbor nodes of each node by minimizing $tr(X^{T}LX)$ and thus minimize conditional entropy $H(X|A^{'})$. The meaning of minimizing $H(X|A^{'})$ is actually maximizing mutual information of $X, A^{'}, Y$:

\begin{lemma}\label{lemma:2}
    Minimizing $H(X|A^{'})$ is equal to maximize $I(X; A^{'}; Y)$.
\end{lemma}

Since minimizing $tr(X^{T}LX)$ is minimizing $H(X|A^{'})$, thus minimizing $tr(X^{T}LX)$ is maximizing mutual information $I(X; A^{'}; Y)$.

Based on Proposition.~\ref{proposition:1}, we further provide Lemma.~\ref{lemma:3}:

\begin{lemma}\label{lemma:3}
    Maximizing mutual information between label $Y$ and perturbed neighbor $A^{'}$: $I(Y; A^{'})$,  is equal to minimizing conditional entropy $H(Y|X,A^{'})$.
\end{lemma}

Lemma.~\ref{lemma:3} implies our target: minimizing $H(Y|X,A^{'})$ can be achieved by maximizing mutual information $I(Y; A^{'})$. Based on Lemma.~\ref{lemma:2} and Lemma.~\ref{lemma:3}, we further introduce Theorem.~\ref{theorem:1}:

\begin{theorem}\label{theorem:1}
    Minimizing conditional entropy $H(X|A^{'})$ is equal to minimizing conditional entropy $H(Y|X,A^{'})$.
\end{theorem}

Proof and more details about the above lemmas and theorem can be found in Appendix.~\ref{proof: lemma 2}, ~\ref{proof: lemma 3}, ~\ref{proof: theorem 1}.

Theorem.~\ref{theorem:1} shows the target of minimizing $tr(X^{T}LX)$ is actually minimizing conditional entropy $H(Y|X,A^{'})$, which is conversely to the attack. In other words, minimizing $tr(X^{T}LX)$ aims at obtaining a clean graph and thus defencing the attack.

%  
%  \section{Analysis}
%  \label{sec:analysis}
%  \subsection{Performance Indicators}\label{subsection:performance_indicators_definition}
We use three categories of indicators to analyze the performance of a scenario definition:

\textbf{Inefficiency Rate}: it is the difference in fuel spent between a traffic scenario in which the aircraft comply to DAA resolution advisories (RA), and the analogous scenario in which each vehicle follows its optimal path, without observation of traffic separation rules. In case of vertical deviations, a higher fuel rate is required for climb maneuvers, a lower fuel rate in descent maneuver, which increase the net total for the mission. The resulting value of this indicator is the average value over all traffic configurations in a scenario set. 

\textbf{Loss of Separation (LoS) Rate}: despite there being several ways of defining traffic separation, we examine just the simplest one, which is checking whether or not the vehicles are separated by at least a fixed \emph{minimum distance}. In order to include both DAIDALUS and ACAS sXu in the same tables, we use one separation distance from each one, respectively: 4,000 ft, which is the Horizontal Miss Distance, or HMD, used in DO-365B \cite{DO_365} to define the so-called \emph{Hazard Alert Zone} (HAZ), associated to the DAA Well-Clear (DWC) concept of separation; and 2,000 ft, used in DO-396 \cite{DO_396}, that defines the Loss of Well-Clear (LoWC) event in relation to large UAVs or manned aircraft. These indicators will denote the rate of scenarios, in a scenario set, where the distance between any aircraft pair fell below the afore mentioned threshold values. 

\textbf{Timeout Rate}: this indicates, in a scenario set, the rate of scenarios where any aircraft exceeded a maximum time without reaching its destination point. As pointed out in section \ref{section:scenario_definitions}, this phenomenon occurs because DAA Resolution Advisories (RAs) cause long chains of maneuvers that extend beyond the energy/fuel allowance of the vehicle, due to shortcomings in coordination. We use a time threshold of 1,000 seconds but, in practice, the timeout would be determined by the energy/fuel capacity of the vehicle.

\textbf{Scenario Computing Time}: the time needed to simulate a single scenario instance, in a single core of an Intel Xeon CPU, discounted the fact that multiple scenario instances can be run in parallel in a multi-core CPU. In our simulated scenarios, the DAA algorithm is called at least each 2 seconds, for each aircraft, but when the aircraft is in avoidance mode, that can happen more often. In the case of ACAS sXu, the requirement of receiving various messages to update a single track contribute to result in multiple calls per simulated second.


\subsection{Scenario Specifications and Labels}
In this study, a scenario specification is defined by features such as: the DAA algorithm used, the dimensionality (2-D or 3-D), if it uses extrinsic priorities or not, the target separation parameter, and possibly other features. The scenario labels used in this section encode these attributes:
\begin{itemize}
\item \texttt{dai\_ip\_2d\_4k}: DAIDALUS without extrinsic priorities, 2-D maneuvering and regular 4 kft Horizontal Miss Distance (HMD);
\item \texttt{dai\_ep\_2d\_4k}: similar to the above, with extrinsic priorities;
\item \texttt{sxu\_ip\_2d\_2k}: ACAS sXu with intrinsic priorities only, 2-D maneuvering and regular 2 kft LoWC threshold;
\item \texttt{sxu\_ep\_2d\_2k}: similar to the above, with extrinsic priorities;
\item \texttt{dai\_ep\_3d\_4k}: similar to \texttt{dai\_ep\_2d\_4k}, with 3-D maneuvering;
\item \texttt{dai\_ep\_2d\_2k}: similar to \texttt{dai\_ep\_2d\_4k}, with HMD reduced to 2 kft.
\end{itemize}

And there are other features and labels that will be mentioned below as needed.

\subsection{Performance Analysis}
The analysis of selected scenario specifications is summarized in table~\ref{table:performance_analysis}. The first notorious observation in this table is the effect of extrinsic priorities to decrease inefficiency. With regards to safety indicators, their effect is mixed, and we have to observe each case separately. In the case of DAIDALUS, priorities decreased the 2 kft LoS rate and, most drastically, the timeout rate, while increased the 4 kft LoS rate. In the case of ACAS sXu, priorities increased both LoS rate indicators, but decreased the timeout rate drastically. Based on our rule of thumb assessment, it can be said that DAIDALUS works better with extrinsic priorities, while ACAS sXu works better without them. We conjecture that the following reasons explain this fact: i) that DAIDALUS has more built-in symmetries than ACAS sXu; ii) that ACAS sXu has already built-in priority rules for multi-aircraft encounters, and extrinsic priorities may contradict with them.

\begin{table}[h]
    \caption{Summary of closed-loop performance indicators per scenario.}
    \label{table:performance_analysis}
    \begin{center}
    \begin{tabularx}{\columnwidth}{|p{0.20\columnwidth}|p{0.11\columnwidth}|p{0.09\columnwidth}|p{0.09\columnwidth}|p{0.1\columnwidth}|p{0.105\columnwidth}|}
        \hline
    Scenario spec. & Ineffici-ency rate & LoS rate 4~kft & LoS rate 2~kft & Timeout rate & Scenario comp. time (s)\\
    \hline
    \texttt{dai\_ip\_2d\_4k} & 9.71\% & 1.4E-2 & 6.5E-5 & 1.6E-2 & 6.5E-2\\
    \texttt{dai\_ep\_2d\_4k} & 4.83\% & 2.4E-2 & 4.1E-5 & 0 & 5.1E-2\\
    \texttt{sxu\_ip\_2d\_2k} & 20.7\% & 8.7E-1 & 4.1E-2 & 1.6E-3 & 7.8E+1\\
    \texttt{sxu\_ep\_2d\_2k} & 10.9\% & 9.0E-1 & 2.0E-1 & 1.8E-5 & 7.8E+1\\
    \texttt{dai\_ep\_3d\_4k} & 4.38\% & 8.9E-6 & 0 & 0 & 7.4E-2\\
    \texttt{dai\_ep\_2d\_2k} & 1.3\% & 9.0E-1 & 1.1E-1 & 0 & 3.9E-2\\
    \hline
    \end{tabularx}
    \end{center}
\end{table}

According to a line of reasoning, it would be expected, that, in the more efficient scenarios, the aircraft fly closer to each other and, therefore, there should be a higher probability of losing separation. But this is not the only principle at play, because, if the aircraft perform deviations with the least extra distance, while keeping separation, they stay less in the air and decrease the total number of conflicts. This becomes more understandable when we compare \texttt{dai\_ep\_2d\_4k} with \texttt{dai\_ep\_3d\_4k}, where the latter achieved a small advantage in efficiency, but a huge one in safety. \texttt{dai\_ep\_3d\_4k} is capable of shortening the total distances, but has a residual cost associated to vertical maneuvers, where the climb maneuvers spend fuel at higher rates. 

It cannot escape from observation that DAIDALUS performed much better than ACAS sXu in almost all indicators. In our opinion, it would be reasonable to expect that ACAS sXu would not excel in the LoS rates, especially that of 4 kft, because its first protection criterion is 2 kft, as a built-in feature. However, with smaller protection volumes, the deviations should be smaller and, by this reasoning, its expected inefficiency would be lower than that of DAIDALUS. But our results show otherwise when we compare the cases of DAIDALUS with those of ACAS sXu. The only case in which ACAS sXu obtained an advantage was for the LoS rates comparison between \texttt{sxu\_ip\_2d\_2k} and \texttt{dai\_ep\_2d\_2k}, which have the same separation target. In any case, the Los rate obtained for ACAS sXu meets the performance requirement of ASTM F3442 \cite{ASTM}, which uses the definition of LoWC Ratio (LR), which is the ratio between the LoS rate of 2 kft shown in table~\ref{table:performance_analysis}, with DAA active, and corresponding LoS rate with DAA inactive, the latter being 0.785 according  to our simulations. Thus, the resulting LR scores for ACAS sXu here are 0.052 and 0.253, respectively for the two ACAS sXu specs, which are well below the value of 0.4 from \cite{ASTM}, and consistent with the performance analysis of \cite{DO_396}. We conjecture that these scores would be lower in a future 3-D scenario spec of a ACAS sXu, by following the same improvement obtained with DAIDALUS. 

\subsection{Possible approximations to the closed-loop behavior}
Trying to alleviate the heavy computational load to simulate large numbers of different traffic configurations, in this multi-aircraft, closed-loop setup, we considered some approximated solutions, such as the use of Deep Neural Networks to emulate the ACAS Xu/sXu behavior, in the lines followed by \cite{Julian2018,Bak2022}. However, the existing solutions that we found available were developed for just one intruder aircraft, so they were not suitable for our study. Another possibility would be the exploitation of symmetry transformations \cite{Sibai2020}, however the history-dependent nature of the ACAS Xu/sXu algorithms, associated to the present closed-loop setup, make this possibility unpractical. Thus, we started exploring simpler ways of deducing closed-loop behavior without having to perform the full simulation of a scenario. So far, we tried to analyze correlations between measures of open-loop maneuvers and the closed loop performance. The features that we explored are:
\begin{itemize}
    \item \textbf{Distance flown until the end of the first deviation maneuver} ($\overline{M/D}$): we consider the total distance flown until a ``Clear-of-Conflict'' (CoC) event happens, that is, after one or more divergent maneuvers start in a scenario instance, we stop the scenario when the first divergent maneuver of any aircraft finishes and that individual aircraft is clear of conflict, the moment from which some decision must be made on how to continue the mission. We count the total number of maneuvers started, and divide it by the sum of the flown distances, across all scenario instances in an execution set associated to a scenario spec.
    \item \textbf{Average angle deviation maneuver} ($\overline{\alpha}$): using the same stopping rule of above, we account the angle difference between the heading angles of the aircraft at the beginning of the divergent maneuver and at the stopping moment. 
\end{itemize}
For each of these measures, we ran the 122,416 traffic configuration instances with the open-loop stopping rule. Here, all the scenario specifications are 2-dimensional, and we use abbreviated lables to achieve a better display in the graph legends. Namely, the scenario specifications in this subsection are defined as:
\begin{itemize}
    \item \texttt{D1}: DAIDALUS without extrinsic priorities and with deterministic sensor data;
    \item \texttt{D2}: DAIDALUS with extrinsic priorities and deterministic sensor data;
    \item \texttt{D3}: DAIDALUS with extrinsic priorities and Sensor Uncertainty Mitigation (SUM). This is a design feature \cite{Narkawicz2018} to mitigate uncertainty in sensor data, as for example, to determine the position of an intruder aircraft;
    \item \texttt{D4}: DAIDALUS with its Horizontal Miss Distance (HMD) set to 200 ft (the standard is 4,000 ft) and uncertain sensor data;
    \item \texttt{X1}: ACAS sXu without extrinsic priorities;
    \item \texttt{X2}: ACAS sXu with extrinsic priorities;
    \item \texttt{X3}: ACAS sXu with scenario downscaled to speed of 43 knots and cell radius of 1 km.
\end{itemize}
We used the values of $\overline{M/D}$ and $\overline{\alpha}$ obtained for each of the specs above as inputs to a linear regressor of inefficiency, as defined in section~\ref{subsection:performance_indicators_definition}, and generated a plot with the pairs of (true, predicted) values from this regression, as shown in fig.~\ref{fig:inefficiency_regression}. The trend line in the figure, which depicts the regressor, seems to represent a strong correlation, which is confirmed by the value of $R^2$. When considering each of the regression inputs separately, we obtain $R^2=0.73$ for $\overline{M/D}$ and $R^2=0.77$ for $\overline{\alpha}$, which show that they contribute with approximately equal predicting power. 
% Figure environment removed

We performed a similar analysis for the LoS indicators, but we found very little correlation, as $R^2=0.11$ for the 2~kft LoS indicator. Nevertheless, it can be concluded that these open-loop measurements are a good proxy for the closed-loop inefficiency, with the advantage that the predictor discounts the bias that may have been introduced by the closed-loop mission management system, which is not part of the DAA specification. A rough estimate for the computing time saved is of 68\% in the inefficiency case.      

%  
%  \section{Experimental Results}
%  \label{sec:exp}
%  \begin{table}[t]
	\centering
	\caption{Preallocation strategy results with $3$ machines per tool group and $10$ operations per lot}
	\label{tab:table}
	\figspace\scriptsize
	%	\resizebox{15.5cm}{!}{
		\begin{tabular}{|l%r
				cl||rr|rr|rr|rr|}
			%			\hline
			%			&                    &                      & %        &
			%			 \multicolumn{8}{c}{\textbf{M = 9}} \\
			\hline
			& \multicolumn{1}{@{\hspace{-3mm}}c@{\hspace{-3mm}}}{\textbf{9 Machines}}                   &                      & % &
			\multicolumn{2}{r|}{\textbf{70 Operations}}                 & \multicolumn{2}{r|}{\textbf{80 Operations}}                 & \multicolumn{2}{r|}{\textbf{90 Operations}}                 & \multicolumn{2}{r|}{\textbf{100 Operations}}                 \\
			& Size % \multicolumn{2}{c}{\textbf{Parameters}}            
			&        &
			Lot                         & Step                        & Lot                         & Step                        & Lot          & Step         & Lot          & Step         \\
			%			& size              % & setup % idx
			%			                  &         & 0                           & 1                           & 0                           & 1                           & 0            & 1            & 0            & 1            \\
			%			&                    & setup                &         &                             &                             &                             &                             &              &              &              &              \\
			\hline\hline
			\multirow{3}{*}{\textbf{Fixed}}    & \multirow{3}{*}{1} & % \multirow{3}{*}{0/1} &
			Makespan    & 483                         & 428                         & 489                         & 440                         & 486          & 531          & 592          & 553         \\
			&                    & %                     &
			Setup/Batch & 6/12                        & 2/12                        & 5/14                        & 0/13                        & 5/14         & 3/12         & 3/12         & 0/16         \\
			&                    & %                     &
			1\ts{st}/2\ts{nd} Stage & 2/1                         & TO/27                          & 6/2                        & TO/13                          & 11/13         & TO           & TO/78           & TO           \\
			\midrule
			\multirow{6}{*}{\textbf{Flexible}} & \multirow{3}{*}{2} & % \multirow{6}{*}{0}   &
			Makespan    & 483                         & 475                         & 592                         & 592                         & 592          & 539          & 745          & 698          \\
			&                    & %                     &
			Setup/Batch & 2/8                        & 0/9                        & 1/8                        & 1/8                        & 1/10         & 0/11          & 0/12          & 0/15          \\
			&                    & %                     &
			1\ts{st}/2\ts{nd} Stage & 5/1                         & TO                          & TO/114                          & TO/1                          & TO/130           & TO           & TO           & TO          \\
			\cline{2-11}
			%			& & & & & & & & & & &   \\
			& \multirow{3}{*}{3} & %                     &
			Makespan    & 559                         & --                          & 815                         & --                          & 1357 & -- & 1486 & -- \\ % \multicolumn{4}{c|}{\multirow{3}{*}{Assignment issue}}     \\
			&                    & %                     &
			Setup/Batch & 0/8                         & --                          & 0/8                        & --                          & 0/10 & -- & 10/18 & -- \\ %\multicolumn{4}{c|}{}                                      \\
			&                    & %                     &
			1\ts{st}/2\ts{nd} Stage & TO                       & --                          & TO/140                          & --                          & TO/79 & -- & TO & -- \\ %\multicolumn{4}{c|}{}                                      \\
			\midrule
			\multirow{6}{*}{\textbf{Setup}}    & \multirow{3}{*}{2} & % \multirow{6}{*}{1}   &
			Makespan    & 483                         & 475                         & 592                         & 592                         & 592          & 536          & 745          & 683          \\
			&                    & %                     &
			Setup/Batch & 2/8                        & 0/9                        & 1/8                        & 1/8                        & 1/10         & 0/12          & 0/13          & 0/16          \\
			&                    & %                     &
			1\ts{st}/2\ts{nd} Stage & 2/1                        & TO                          & TO/21                          & TO/25                          & TO/22           & TO           & TO/76           & TO           \\
			%			& & & & & & & & & & &   \\
			\cline{2-11}
			& \multirow{3}{*}{3} & %                     &
			Makespan    & \textbf{334}                         & --                          & \textbf{345}                         & --                          & \textbf{434}          & --           & \textbf{555}          & --           \\
			&                    & %                     &
			Setup/Batch & 0/8                         & --                          & 0/8                         & --                          & 0/11          & --           & 0/12          & --           \\
			&                    & %                     &
			1\ts{st}/2\ts{nd} Stage & TO/20                       & --                          & TO/123                          & --                          & TO           & --           & TO/73           & --           \\
			\hline
		\end{tabular}
		%	}
\end{table}
%
We constructed a scalable set of benchmark instances, focusing on sub-routes of
$10$ production operations for two product types from the SMT2020 simulation scenario~\cite{kopp2020smt2020}.
The $10$ operations in both sub-routes are processed by machines
belonging to three tool groups and do thus involve re-entrant flow,
as a lot visits the same tool group multiple times.
Moreover, the operations incorporate batching and specific setups, and machines undergo periodic maintenance operations.
In the following, we concentrate on instances with $9$ machines, i.e., $3$ per
tool group, and gradually increasing number of lots.
Further smaller- and larger-scale instances along with our implementation are
available online.\footref{foo:online}

We ran our experiments with \clingodl\ (version 1.4.0) on an Intel® Core™i7-8650U CPU Dell Latitude 5590 machine under Windows 10, imposing two time limits per run:
the first stage for makespan minimization is aborted at $450$ seconds, in which case the best schedule found so far % (if any) 
is taken as upper bound on the makespan for proceeding to minimize setup and batch violations with 
another $150$ seconds time limit.

Table~\ref{tab:table} reports the quality of best schedules obtained within the time limits for both optimization stages, split into `Makespan' and `Setup/Batch'
values, while two runtimes or `TO' for a timeout, respectively, are given in the
`1\ts{st}/2\ts{nd} Stage' rows, only listing a single `TO' entry in case both stages timed out.
The `Size' column provides the value taken for the constant \lstinline{sub_size},
limiting the number of machines in subgroups to which the operations are preallocated.
For the latter, the `Lot' columns include results with value \lstinline{0} for the constant \lstinline{lot_step}, where a common subgroup takes all operations for a lot, or for value \lstinline{1} in the `Step' columns, leading to their distribution among subgroups.

The `Size' value 1 necessarily leads to a fixed machine assignment, for which the
quality indicators clearly show that the `Step' strategy yields better schedules,
although it incurs more timeouts and thus fewer certain optima because operations on different lots increase the flexibility of execution sequences and thus search complexity.
While flexibility within subgroups by setting their `Size' to 2 or 3 in principle allows for improved schedules, we observe a deterioration due to sharply increasing instantiation size and search effort, as already observed in \cite{ali2023flexible}.
The setup strategy to differentiate operations and machines within subgroups,
activated by changing the constant \lstinline{by_setup},
aims to cut down the scheduling complexity in line with the optimization objectives by reducing the need for setup changes.
This leads to significantly improved schedules with `Size' 3, where the
`Lot' and `Step' preallocation strategies are indifferent and redundant results for the latter are omitted, up to a critical size reached with $100$~operations.

With our preliminary approach~\cite{ali2023flexible}, using a more naive and less feature-rich encoding of either fixed or fully flexible machine assignments, the
threshold at which problem size and combinatorics get prohibitive was reached at less than $50$ operations already.
Despite gearing up to double that size, our benchmark instances still represent small excerpts of the large-scale semiconductor fabs with more than $100$ tool groups and from $242$ to $543$ production operations per lot modeled by~\cite{kopp2020smt2020}.
%
The elevated complexity in comparison to basic settings like the traditional FJSP is mainly due to sophisticated setup and maintenance operations, requiring a detailed analysis of execution sequences on machines for SMSP.
We conjecture that similar scalability limits would also be encountered with MIP or CP encodings, yet the first-order modeling language of ASP with difference logic facilitates rapid prototyping and experimentation.
In fact, our performance evaluation aims to explore the feasibility of search and optimization, in order to come up with strategies for breaking down large SMSP instances into more manageable portions, e.g., focusing on some bottleneck tool groups or re-entrant flow of operations.

% This section will show the experimental results performed by applying the machine assignment strategies mentioned before, with several instances ranging from $30$ to $130$ steps and $6$ to $12$ machines. All experiments are run using an Intel\textsuperscript{\textregistered} Core\texttrademark{} i7-8650U CPU Dell Latitude 5590 machine under Windows 10. Our timeout limit is $600$ seconds, splitted to $450$ seconds for the makespan and $150$ seconds for the setup and batching. 

% We considered three tool groups for all generated instances in which batch processing, time/counter-based maintenance, and setup are considered. For generating the instances, we started with a small instance containing $30$ steps and $6$ machines where each tool group has $2$ machines and then we generate the next instance by adding one more lot, which has $10$ steps. We kept the tool group size till the fixed machine assignment strategy could not reach the optimum within the time limit. We created $3$ parameters \textit{size, idx} and \textit{setup} to activate a specific machine assignment strategy. The size determines the size of a sub-group in each tool group. The $idx$ defines the Job/Step-based indexing of all steps in the same tool group where all steps of the same lot will have the same index if the $idx = 0$ and Hence, they are assigned to the same sub-group/machine. If $idx = 1$, then each step in the tool group will have an identical index. The last parameter setup is to activate the setup strategy or not. If the $setup = 1$, then the setup strategy is applied; if $setup = 0$ then it's not applied.

% % To continue tomorrow isA :)
% Table \ref{tab:table01} shows the results of the instances with $2$ machines in each toll group. The first column refers to the strategy applied for the machine assignment. The second and third columns show the parameters for selecting a particular strategy. The assignment is fully flexible if the \textit{size} is greater than or equal to the number of machines in a tool group. Otherwise, the assignment is partially flexible. In the fourth column, we list our optimization criteria and the time limit for the makespan and setup/batching represented by 1st/2nd call. Each following two consecutive columns illustrate the results of an instance when the Job/Step-based indexing is selected. From the \ref{tab:table01}, we observed that the best-obtained results were achieved by the full flexible assignment in the first three instances and for the last instance, the setup strategy was the best. The fixed/setup strategies terminated within the time limit except for only one case.

% \begin{table}[h]
% 	\centering
% 	\caption{Comparison between the allocation strategies with 2 machines per tool group}
% 	\label{tab:table01}
% %	\resizebox{15.5cm}{!}{
% 		\begin{tabular}{|l%r
% 			cl||rr|rr|rr|rr|}
% 			\hline
% %			&                    &                      &         & \multicolumn{8}{c}{\textbf{M = 6}} \\
% %			\hline
% 			& \textbf{M = 6}                   & %                     &
% 			  & \multicolumn{2}{r|}{\textbf{Instance 01}}                 & \multicolumn{2}{r|}{\textbf{Instance 02}}                 & \multicolumn{2}{r|}{\textbf{Instance 03}}                 & \multicolumn{2}{r|}{\textbf{Instance 04}}                 \\
% 			& Size % \multicolumn{2}{c}{\textbf{Parameters}}            
% 			 &			         & Job                         & Step                        & Job                         & Step                        & Job          & Step         & Job          & Step         \\
% 			\hline
% %			& size               & setup %idx
% %			                  &         & 0                           & 1                           & 0                           & 1                           & 0            & 1            & 0            & 1            \\
% %			&                    & setup                &         &                              &                             &                             &                             &              &              &              &              \\
% 			\hline
% 			\multirow{3}{*}{\textbf{Fixed}}    & \multirow{3}{*}{1} & % \multirow{3}{*}{0/1} &
% 			 Makespan    & 409                         & 353                         & 409                         & 409                         & 525          & 424          & 525          & 493          \\
% 			&                    & %                     &
% 			 Setup/Batch & 5/6                         & 4/6                         & 4/8                         & 4/8                         & 4/9          & 1/9          & 3/11          & 2/10          \\
% 			&                    & %                     &
% 			 1\ts{st}/2\ts{nd}-Call & \textless{}1/\textless{}1 & \textless{}1/\textless{}1 & \textless{}1/\textless{}1 & \textless{}1/\textless{}1 & 31/1         & 137/6        & 37/11          & TO/53           \\
% 			\midrule
% 			\multirow{3}{*}{\textbf{Flexible}} & \multirow{3}{*}{2} & % \multirow{3}{*}{0}   &
% 			 Makespan   & \textbf{233}                         & --                          & \textbf{281}                         & --                          & \textbf{365}          & --           & 587          & --           \\
% 			&                    & %                     &
% 			 Setup/Batch & 0/5                         & --                          & 0/6                         & --                          & 0/8          & --           & 3/9          & --           \\
% 			&                    & %                     &
% 			 1\ts{st}/2\ts{nd}-Call & 7/0                         & --                          & TO/6                          & --                          & TO/83           & --           & TO           & --           \\
% 			\midrule
% 			\multirow{3}{*}{\textbf{Setup}}    & \multirow{3}{*}{2} & % \multirow{3}{*}{1}   &
% 			 Makespan  & 277                         & --                          & 321                         & --                          & 381          & --           & \textbf{419}          & --           \\
% 			&                    & %                     &
% 			 Setup/Batch & 0/4                         & --                          & 0/6                         & --                          & 0/8          & --           & 0/9          & --           \\
% 			&                    & %                     &
% 			 1\ts{st}/2\ts{nd}-Call & \textless{}1/\textless{}1 & --                          & 25/1                         & --                          & TO/12        & --           & TO/122           & -- \\
% 			 \hline
% 		\end{tabular}
% %	}
% \end{table}

% Table~\ref{tab:table02} summarizes the results of the subsequent $4$ instances where each tool group has $3$ machines. In this instances group, we can split the machines into sub-group by setting the \textit{size} parameter to $2$; in that case, we have two sub-groups in each tool group. The experiments demonstrated that the fixed strategy has the same or better performance than the flexible. In addition, the flexible strategy could not find a feasible solution for instances $7$ and $8$ when all machines were in the same group. On the other hand, the setup strategy performed better than the other two strategies when all machines were in one group, in addition to reaching the optimal value of the setup for all instances. 

% \begin{table}[h]
% 	\centering
% 	\caption{Comparison between the allocation strategies with 3 machines per tool group}
% 	\label{tab:table02}
% %	\resizebox{15.5cm}{!}{
% 		\begin{tabular}{|l%r
% 			cl||rr|rr|rr|rr|}
% %			\hline
% %			&                    &                      & %        &
% %			 \multicolumn{8}{c}{\textbf{M = 9}} \\
% 			\hline
% 			& \textbf{M = 9}                   &                      & % &
% 			 \multicolumn{2}{r|}{\textbf{Instance 05}}                 & \multicolumn{2}{r|}{\textbf{Instance 06}}                 & \multicolumn{2}{r|}{\textbf{Instance 07}}                 & \multicolumn{2}{r|}{\textbf{Instance 08}}                 \\
% 			& Size % \multicolumn{2}{c}{\textbf{Parameters}}            
% 			&        &
% 			 Job                         & Step                        & Job                         & Step                        & Job          & Step         & Job          & Step         \\
% %			& size              % & setup % idx
% %			                  &         & 0                           & 1                           & 0                           & 1                           & 0            & 1            & 0            & 1            \\
% %			&                    & setup                &         &                             &                             &                             &                             &              &              &              &              \\
% 			\hline\hline
% 			\multirow{3}{*}{\textbf{Fixed}}    & \multirow{3}{*}{1} & % \multirow{3}{*}{0/1} &
% 			 Makespan    & 525                         & 433                         & 525                         & 452                         & 525          & 521          & 643          & \textbf{559}          \\
% 			&                    & %                     &
% 			 Setup/Batch & 6/13                        & 1/13                        & 5/15                        & 0/14                        & 5/16         & 6/16         & 6/12         & 3/12         \\
% 			&                    & %                     &
% 			 1\ts{st}/2\ts{nd}-Call & 30/3                         & TO/153                          & 24/8                        & TO/63                          & 231/81         & TO           & TO           & TO           \\
% 			\midrule
% 			\multirow{6}{*}{\textbf{Flexible}} & \multirow{3}{*}{2} & % \multirow{6}{*}{0}   &
% 			 Makespan    & 525                         & 475                         & 650                         & 650                         & 650          & 595          & 745          & 742          \\
% 			&                    & %                     &
% 			 Setup/Batch & 2/11                        & 0/11                        & 1/12                        & 1/12                        & 6/13         & 4/14          & 3/17          & n/a          \\
% 			&                    & %                     &
% 			 1\ts{st}/2\ts{nd}-Call & 26/7                         & TO                          & TO/12                          & TO                          & TO           & TO           & TO           & TO           \\
% 			\cline{2-11}
% %			& & & & & & & & & & &   \\
% 			& \multirow{3}{*}{3} & %                     &
% 			 Makespan    & 744                         & --                          & 1206                         & --                          & 1698 & -- & n/a & -- \\ % \multicolumn{4}{c|}{\multirow{3}{*}{Assignment issue}}     \\
% 			&                    & %                     &
% 			 Setup/Batch & 2/12                         & --                          & n/a                        & --                          & 8/15 & -- & n/a & -- \\ %\multicolumn{4}{c|}{}                                      \\
% 			&                    & %                     &
% 			 1\ts{st}/2\ts{nd}-Call & TO                       & --                          & TO                          & --                          & TO & -- & TO & -- \\ %\multicolumn{4}{c|}{}                                      \\
% 			\midrule
% 			\multirow{6}{*}{\textbf{Setup}}    & \multirow{3}{*}{2} & % \multirow{6}{*}{1}   &
% 			 Makespan    & 525                         & 475                         & 650                         & 650                         & 643          & 553          & 745          & 642          \\
% 			&                    & %                     &
% 			 Setup/Batch & 2/11                        & 0/11                        & 1/12                        & 1/12                        & 1/14         & 0/13          & 1/14          & 1/16          \\
% 			&                    & %                     &
% 			 1\ts{st}/2\ts{nd}-Call & 44/2                        & TO                          & TO/4                          & TO/2                          & TO           & TO/7           & TO           & TO           \\
% %			& & & & & & & & & & &   \\
% 			\cline{2-11}
% 			& \multirow{3}{*}{3} & %                     &
% 			 Makespan    & \textbf{346}                         & --                          & \textbf{373}                         & --                          & \textbf{429}          & --           & 820          & --           \\
% 			&                    & %                     &
% 			 Setup/Batch & n/a                         & --                          & n/a                         & --                          & n/a          & --           & n/a          & --           \\
% 			&                    & %                     &
% 			 1\ts{st}/2\ts{nd}-Call & TO                       & --                          & TO                          & --                          & TO           & --           & TO           & --           \\
% 			\hline
% 		\end{tabular}
% %	}
% \end{table}

% Table~\ref{tab:table03} considers $4$ machines in each tool group and the flexible strategy obtained the best result for the first instance. However, it had the same feasibility issue when all machines were in the same group. For the rest instances, the setup strategy dominated when the machines were equally distributed into sub-groups. 

% From the conducted experiments, we can conclude that 
% \begin{itemize}
% 	\item The flexible assignment performed well on the small-scale.
% 	\item While increasing the scale, the setup strategy dominates in the most cases
% 	\item Assigning the steps of the same lot independently with the fixed assignment leads to better performance
% 	\item The Setup strategy has a significant impact in minimizing the setup objective through all instances
% 	\item The full flexible assignment has an assignment issue while increasing the number of machines
% \end{itemize}

% \begin{table}[h]
% 	\centering
% 	\caption{Comparison between the allocation strategies with 4 machines per tool group}
% 	\label{tab:table03}
% %	\resizebox{15.5cm}{!}{%
% 		\begin{tabular}{|l%r
% 			cl||rr|rr|rr|rr|}
% 			\hline
% %			&                    &                      &  &  \multicolumn{8}{c}{\textbf{M = 12}} 
% %			\\ \hline
% 			& \textbf{M = 12}                   & %                     & 
% 			 & \multicolumn{2}{r|}{\textbf{Instance 09}}                 & \multicolumn{2}{r|}{\textbf{Instance 10}}                 & \multicolumn{2}{r|}{\textbf{Instance 11}}                 & \multicolumn{2}{r|}{\textbf{Instance 12}}                 \\
% 			& Size % \multicolumn{2}{l}{\textbf{Parameters}}            
% 			 &			 &			 Job                    & Step                   & Job                    & Step                   & Job                    & Step                   & Job                    & Step                   \\
% %			& Size               & setup % idx
% %			                  &  & 0                      & 1                      & 0                      & 1                      & 0                      & 1                      & 0                      & 1                      \\
% %			&                    & setup                &  &  &                        &                        &                        &                        &                        &                        &                                               \\
% 			\hline\hline
% 			\multirow{3}{*}{\textbf{Fixed}}    & \multirow{3}{*}{1} & % \multirow{3}{*}{0/1} &
% 			 Makespan                 & 525                    & 453                    & 525                    & 452                    & 525                    & 493                    & 643                    & 561                    \\
% 			&                    & %                     &
% 			 Setup/Batch              & 7/19                   & 3/20                   & 7/20                  & n/a                   & 6/22                   & 4/20                   & 4/22                   & n/a                   \\
% 			&                    & %                     &
% 			 1\ts{st}/2\ts{nd}-Call              & 124/5                 & TO & 25/17                 & TO & 25/53                 & TO/142 & TO & TO \\
% 			\midrule
% 			\multirow{9}{*}{\textbf{Flexible}} & \multirow{3}{*}{2} & % \multirow{9}{*}{0}   &
% 			 Makespan                 & \textbf{373}                    & 503                    & 491                    & 778                    & 569                    & 569                    & 765                    & 1673                   \\
% 			&                    & %                     &
% 			 Setup/Batch              & n/a                    & 6/17                    & n/a                   & n/a                    & n/a                    & n/a                   & n/a                    & 12/24                  \\
% 			&                    & %                     &
% 			 1\ts{st}/2\ts{nd}-Call              & TO & TO & TO & TO & TO & TO & TO & TO \\
% 			\cline{2-11}
% %			& & & & & & & & & & &   \\
% 			& \multirow{3}{*}{3} & %                     &
% 			 Makespan                 & 709                    & 688                    & 800                    & 907                    & 876                    & 876                    & 905                    & 1643                   \\
% 			&                    & %                     &
% 			 Setup/Batch              & 5/17                    & n/a                   & 3/18                   & 5/19                   & n/a                   & n/a                   & n/a                  & 15/24                    \\
% 			&                    & %                     &
% 			 1st/2nd              & TO & TO & TO & TO & TO & TO & TO & TO \\
% 			\cline{2-11}
% %			& & & & & & & & & & &   \\
% 			& \multirow{3}{*}{4} & %                     &
% 			 Makespan                 & n/a & -- & n/a & -- & n/a & -- & n/a & -- \\ %\multicolumn{8}{c|}{\multirow{3}{*}{Assignment issue}}                                                                                                                                                 \\
% 			&                    & %                     &
% 			 Setup/Batch              & n/a & -- & n/a & -- & n/a & -- & n/a & -- \\ %\multicolumn{8}{c|}{}                                                                                                                                                                                  \\
% 			&                    & %                     &
% 			 1\ts{st}/2\ts{nd}-Call              & TO & -- & TO & -- & TO & -- & TO & -- \\ %\multicolumn{8}{c|}{}                                                                                                                                                                                  \\
% 			\midrule
% 			\multirow{9}{*}{\textbf{Setup}}    & \multirow{3}{*}{2} & % \multirow{9}{*}{1}   &
% 			 Makespan                 & 401                    & 396                    & 419                    & \textbf{416}                    & \textbf{419}                    & \textbf{419}                    & \textbf{457}                    & 471                    \\
% 			&                    & %                     &
% 			 Setup/Batch              & 0/15                   & 0/14                   & 0/16                   & 0/16                   & n/a                   & n/a                   & 0/21                    & n/a                    \\
% 			&                    & %                     &
% 			 1\ts{st}/2\ts{nd}-Call              & TO & TO/92 & TO & TO & TO & TO & TO & TO \\
% 			\cline{2-11}
% %			& & & & & & & & & & &   \\
% 			& \multirow{3}{*}{3} & %                     &
% 			 Makespan                 & 706                    & 642                    & 792                    & 753                    & 942                    & 942                    & 939                    & 894                    \\
% 			&                    & %                     &
% 			 Setup/Batch              & 1/14                    & n/a                    & 2/16                    & n/a                   & n/a                   & n/a                    & n/a                    & 1/22                    \\
% 			&                    & %                     &
% 			 1\ts{st}/2\ts{nd}-Call              & TO & TO & TO & TO & TO & TO & TO & TO \\
% 			\cline{2-11}
% %			& & & & & & & & & & &   \\
% 			& \multirow{3}{*}{4} & %                     &
% 			 Makespan                 & 679                    & -- & 1725                    & -- & n/a                    & -- & n/a                    & -- \\
% 			&                    & %                     &
% 			 Setup/Batch              & n/a                   & -- & n/a                    & -- & n/a                   & -- & n/a                   & -- \\
% 			&                    & %                     &
% 			 1st/2nd              & TO & -- & TO & -- & TO & -- & TO & -- \\
% 			\hline
% 		\end{tabular}%
% %	}
% \end{table}
%  %
%  %
%  \section{Conclusion}
%  \label{sec:conc}
%  \section{Conclusion}
\label{sec:conc}
In this paper, we propose a fast leakage and variability-aware thermal simulation method that also captures the temperature dependence of conductivity. We derive a closed-form of the Green's function considering all these effects using novel insights and algebraic techniques. Our approach provides fast and accurate solutions for both the steady-state and the transient thermal profile and has been validated with a wide variety of test cases. As device dimensions continue to shrink, process variation has become a serious problem. The methods proposed in this work can equip designers to tackle this problem and may spawn further research in this area.

%  
%  
\bibliography{main}
\bibliographystyle{plainnat}
\begin{comment}
\section{System Architecture}
\label{appendix:architecture}
\system has a novel modularized system architecture with three key components: 
\emph{StreamManager}, 
\emph{TxnManager} and \emph{TxnScheduler}. 
These components are instantiated in each thread locally.
The execution outline of \system is presented in Algorithm~\ref{alg:algo}.
Transactional stream processing is continuous and potentially never ends (Line 1$\sim$8).
The dependency resolution and execution of state transactions are separated into two non-overlapping phases by punctuations~\cite{Tucker:2003:EPS:776752.776780} (Line 2 and 5), which guarantees that no subsequent input event will have a smaller timestamp. 
Effectively, a batch of state transactions is collected during the first phase, and processed during the second phase.

In the first phase (i.e., stream processing phase), 
the \emph{StreamManager} conducts preprocessing for every input event ($e$). Similar to some prior works~\cite{tstream}, state transactions may be issued but not immediately processed during preprocessing (Line 3).
The \emph{pre\_processing} and \emph{post\_processing} functions are exposed as APIs to users.
The \emph{TxnManager} handles dependency resolution (Line 4) among state transactions and insert decomposed operations to construct a \tpg. We discuss the detailed two-phase \tpg construction process in Section~\ref{subsec:construction}.

In the second phase  (i.e., transaction processing phase), 
the \emph{TxnManager} is first involved again to refine (Line 6) the constructed \tpg with further dependency resolution.
The \emph{TxnScheduler} 
schedules operations for concurrent execution based on the constructed \tpg according to the three dimensions of scheduling decisions (Line 7). 
In particular, a scheduling decision model $M$ is instantiated based on the constructed \tpg (Line 14).
\textbf{\circled{1}} Guided by $M$, execution threads adopt an exploration strategy (Section~\ref{subsec:explore}) to explore the constructed \tpg for operations available to be scheduled constrained by dependencies. 
\textbf{\circled{2}} 
During exploration, one or multiple operations may be treated as the 
% basic 
unit of scheduling (Section~\ref{subsec:granularity}). 
Subsequently, \textbf{\circled{3}} every thread executes operation(s) in the unit of scheduling with various abort handling mechanisms (Section~\ref{subsec:abort_handling}).
Only when state transactions are processed (i.e., committed or aborted) can the associated input events be postprocessed (Line 8) by the \emph{StreamManager} based on transaction processing results.
\end{comment}

\begin{comment}
\begin{algorithm}
\footnotesize
    \KwData{$e$ \tcp{Input event}}
    \KwData{$txn_{ts}$ \tcp{State transaction}}
    \KwData{$G$ \tcp{The currently constructed TPG}}
    \While{!finish processing of input streams}{
        \eIf(\tcp*[h]{Phase 1}){\text{$e$ is not a $punctuation$}}{
                $txn_{ts}$ $\gets$ PRE\_Processing($e$)\;
                \textbf{TPG\_Construction}($G$, $txn_{ts}$)\; 
          }(\tcp*[h]{Phase 2}){
                \textbf{TPG\_Refinement}($G$)\; 
                \textbf{TXN\_Scheduling}($G$)\; 
                POST\_Processing()\;
          }
    }
    
    \SetKwFunction{FMain}{TPG\_Construction}
    \SetKwProg{Fn}{Function}{:}{}
    \Fn{\FMain{$G$, $txn_{ts}$}}{
        $O_{1..k}$ $\gets$ \textbf{Partition} $txn_{ts}$\;
        \ForEach{\text{operation $O_{i}$ $\in$ $O_{1..k}$}}{
            \textbf{Identify} its \ld\;
            $G$ $\gets$ $G$ + $O_{i}$ \;
        }
    }
    \SetKwFunction{FMain}{TPG\_Refinement}
    \SetKwProg{Fn}{Function}{:}{}
    \Fn{\FMain{$G$}}{
        \ForEach{\text{vertex $e_{i}$ $\in$ $G$}}{
            \textbf{Identify} its \td, \pd\;
        }
    }
    
    \SetKwFunction{FMain}{TXN\_Scheduling}
    \SetKwProg{Fn}{Function}{:}{}
    \Fn{\FMain{$G$}}{
        $M$ $\gets$ Instantiated with $G$;\tcp{A decision model}
        \While{!finish scheduling of $G$
        }{
          \textbf{\circled{2}} $Scheduling Unit$ $\gets$ \textbf{\circled{1}} \emph{Explore}($G$, $M$)\; 
            \textbf{\circled{3}} \emph{Execute with Abort Handling} ($Scheduling Unit$)\; 
        }
    }
  \caption{Execution Outline of \system}
  \label{alg:algo}
\end{algorithm}
\end{comment}
%  %\appendix
%  
%  First, we provide a glossary of terms and notation that we use throughout this paper for easy summary. Afterwards, we provide an extended discussion of related work. We give the proofs of our main results (the lemma and the two theorems). Next, we include a discussion on other aspects of generating information-theoretic lower bounds for the weak supervision setting. We also consider extending robust PCA-based techniques for structure learning \emph{without} the singleton separator set assumption. Finally, we give additional experimental details.
%  
%  \section{Glossary}
%  \label{sec:gloss}
%  \chapter*{Glossary of Sets and Constructions}
\label{chap:Glossary}
% next resets the equation numbers to start at 1 at the start of the chapter
\setcounter{equation}{0}
\renewcommand{\theequation}{\thechapter.\arabic{equation}}

We give a table that details all of the major sets and operators that are used in this thesis, for convenience and for reference.

%\begin{table}[h]
%\begin{tabular}{p{2.5cm}|p{8cm}|p{2cm}}
\begin{longtable}{p{2.5cm}|p{8.25cm}|p{1.75cm}}
\textbf{Name} & \textbf{Description} & \textbf{Thesis Ref.} \\
 \midrule
$m$-reducibility & Given two sets $A$ and $B$, $A$ is $m$-reducible to $B$, written $A \leq_m B$, if there exists some computable function $f: \omega \rightarrow \omega$ such that for all $x \in \omega$, $x \in A \iff f(x) \in B$ & \ref{def:mred}\\
\hline
Weihrauch Reducibility & Given two operators $f$ and $g$ on represented spaces, we say $f \leq_W g$, if there exist computable $H,K :\subseteq \omega^\omega \rightarrow \omega^\omega$ such that for any realizer $G \vdash g$, $F = K\langle id_{\omega^\omega}, GH \rangle$ is a realizer for $f$. & \ref{def:Weihrauch} \\
\hline
$WELL$ & The set of all indices $e$ such that $\varphi_e$ is the characteristic function of a well-founded tree $T \subseteq \omega^{<\omega}$. & \ref{def:WELL}\\
\hline
$ILL$ & The set of all indices $e$ such that $\varphi_e$ is the characteristic function of an ill-founded tree $T \subseteq \omega^{<\omega}$. & \ref{def:ILL}\\
\hline
$TILE$ & The set of all indices $e$ such that $\varphi_e$ is the characteristic function of an infinite Wang prototile set whose tilings are total in the plane. & \ref{def:TILE} \\
\hline
$WTILE$ & The set of all indices $e$ such that $\varphi_e$ is the characteristic function of an infinite Wang prototile set whose tilings are infinite, connected, but not necessarily total in the plane. & \ref{def:WTILE} \\
\hline
$SNT$ & The set of all indices $e$ such that $\varphi_e$ is the characteristic function of an infinite Wang prototile set whose connected tilings are all finite. & \ref{def:SNT}\\
\hline
$ATile$ & Set of all $e$ such that $\varphi_e$ is the characteristic function for a set of prototiles who planar tilings are all total and aperiodic. & \ref{def:ATile} \\
\hline
$PTile$ & Set of all $e$ such that $\varphi_e$ is the characteristic function for a set of prototiles who planar tilings are all total and periodic. & \ref{def:PTile} \\
\hline
$ATile_{FIN}$ & Set of all $e$ such that $\varphi_e$ is the characteristic function for a \emph{finite} set of prototiles who planar tilings are all total and aperiodic. & \ref{def:ATileFIN}\\
\hline
$PTile_{FIN}$ & Set of all $e$ such that $\varphi_e$ is the characteristic function for a \emph{finite} set of prototiles who planar tilings are all total and periodic. & \ref{def:PTileFIN}\\
\hline
AIT & The construction found in the proof of theorem \ref{thm:TILE-ILL} that creates an aperiodic prototile set given an ill-founded tree. & \ref{def:AITPIT}\\
\hline
PIT & The construction found in the proof of theorem \ref{thm:ILL-PTile} that creates an aperiodic prototile set given an ill-founded tree. & \ref{def:AITPIT}\\
\hline
$CT$ & The operator that takes some set of Wang prototiles as input and returns a total tiling of the plane. & \ref{def:ChooseTiling} \\
\hline
$CWPT$ & An operator that takes a set of Wang prototiles and returns a connected planar, but not necessarily total tiling. & \ref{ref:CWPT} \\
\hline
$CIPT$ & An operator that takes a prototile set $S$ that has total planar tilings, and returns an infinite `slice' of this tiling as a tiling of an infintie region of $\mathbb{Z}^2$. & \ref{def:CIPT} \\
\hline
$WIPT$ & An operator that takes a set of prototiles and return a tiling that has an infinite patch tiled within it, but we do not know where it is. & \ref{def:WIPT} \\
\hline
$DPW$ & The $DPW$ operator takes some set of prototiles and return a tiling that has an infinite connected patch within it. & \ref{def:DPW} \\
\hline
$C_{\omega^\omega}$ & Closed choice on Baire space - equivalent to finding a path through an ill-founded Baire space tree. & \ref{def:ClosedChoice} \\
\hline
$C_{2^\omega}$ & Closed choice on Cantor space - equivalent to Weak K\"onig's Lemma. & Sec \ref{sec:FurtherWeakTilingProblems} \\
\hline
$C_{\omega}$ & closed choice on the natural numbers - this takes a function $f: \omega \rightarrow \omega$ such that $range(f) \neq \omega$, and returns some point $n \notin range(f)$. & Sec. \ref{sec:FurtherWeakTilingProblems} \\
\end{longtable}
%\end{tabular}
%\end{table}

%  
%  \section{Extended Related Work}
%  \label{sec:related}
%  \section{Related Work}
\label{appsec: related work}
Bayesian causal discovery literature has primarily focused on inference in linear models with closed-form posteriors or marginalized parameters. Early works considered sampling directed acyclic graphs (DAGs) for discrete~\cite{cooper1992bayesian, madigan1995bayesian, heckerman2006bayesian} and Gaussian random variables~\cite{friedman2003being, tong2001active} using Markov chain Monte Carlo (MCMC) in the DAG space. However, these approaches exhibit slow mixing and convergence~\cite{eaton2012bayesian,grzegorczyk2008improving}, often requiring restrictions on number of parents~\cite{kuipers2017partition}. %Alternative exact dynamic programming methods are limited to small settings~\cite{koivisto2012advances}. 

Recent advances in variational inference~\cite{zhang2018advances} have facilitated graph inference in DAG space, with gradient-based methods employing the NOTEARS DAG penalty \cite{zheng2018dags}.\cite{annadani2021variational} samples DAGs from autoregressive adjacency matrix distributions, while \cite{lorch2021dibs} utilizes Stein variational approach \cite{liu2016stein} for DAGs and causal model parameters. \cite{cundy2021bcd} proposed a variational inference framework on node orderings using the gumbel-sinkhorn gradient estimator \cite{mena2018learning}. \cite{deleu2022bayesian,nishikawa2022bayesian} employ the GFlowNet framework \cite{bengio2021gflownet} for inferring the DAG posterior. Most methods, except\cite{lorch2021dibs} are restricted to linear models, while \cite{lorch2021dibs} has high computational costs and lacks DAG generation guarantees compared to our method.
% at least quadratic scaling complexity, both with respect to the number of nodes (due to the DAG penalty) as well as number of posterior samples. Our proposed approach instead has linear complexity with respect to number of posterior samples and does not require any additional DAG penalty.     

In contrast, \emph{quasi-Bayesian} methods, such as DAG bootstrap \cite{friedman2013data}, demonstrate competitive performance. DAG bootstrap resamples data and estimates a single DAG using PC \cite{spirtes2000causation}, GES \cite{chickering2002optimal}, or similar algorithms, weighting the obtained DAGs by their unnormalized posterior probabilities. Recent neural network-based works employ variational inference to learn DAG distributions and point estimates for nonlinear model parameters \cite{charpentier2022differentiable,geffner2022deep}.
%  
%  \section{Proofs}
%  \label{sec:app_proofs}
%  %!TEX root = ../main.tex

\newcommand{\Paughproj}{\lawP^{\mathrm{proj}}_{\mathrm{aug},h}}
\newcommand{\seqz}{\mathsf{z}}
\newcommand{\zst}{\seqz^\star}
\newcommand{\zstil}{\tilde{\seqz}^\star}

\newcommand{\phiZ}{\phi_{\cZ}}
\newcommand{\phiV}{\phi_{\cV}}
\newcommand{\seqv}{\mathsf{v}}





\section{Imitation in the Composite MDP}\label{sec:imit_composite}
In this section, we prove our imitation guarantees in the composite MDP under the full generality of data augmentation.  The majority of this section is devoted to proving  a more general version of \Cref{thm:smooth_cor} that applies to vectorized notions of distance and helps tighten our bounds when instantiated in the control setting.  In Appendix \ref{app:generalizationsmooth}, we introduce some notation and state our most general result, \Cref{thm:smooth_cor_general}.  We then proceed to show that \Cref{thm:smooth_cor} follows from \Cref{thm:smooth_cor_general} and in Appendix \ref{app:smoothcor_general_proof}, we provide a detailed and rigorous proof of the main result.  In Appendix \ref{app:smoothcor_proof}, we show that the more general \Cref{thm:smooth_cor_general} impiles \Cref{thm:smooth_cor} from the text.

Throughout, we  also assume $\cS$ admits a direct decomposition. This is useful to capture the fact that we only apply smoothing on the $\pathm$ coordinates (memory chunk), not the full trajectory chunk $\pathc$.  
\begin{definition}[Direct Decomposition]\label{defn:direct_decomp} Let $\cS = \cZ \oplus \cV$ is a direct decomposition. We let $\phiZ$ and $\phiV$ denote projections onto the $\cZ$ and $\cV$ components, respectively.  We say that the $\cS = \cZ \oplus \cV$ is \emph{compatible} with the dynamics if  $F_h((\seqz,\seqv),\seqa) = F_h((\seqz,\seqv'),\seqa)$ for all $\seqv, \seqv' \in \cV$ and $\seqz \in \cZ$, and \emph{compatible} with policy $\pi$ if $\pi_h((\seqz,\seqv),\seqa) = \pi_h((\seqz,\seqv'),\seqa)$.; we define compatibility of a kernel $\lawW$ and of a pseudometric $\dist(\cdot,\cdot): \cS \times \cS \to \R_{\ge 0}$ with $\cS = \cZ \oplus \cV$ similarly.
\end{definition}
We emphasize that compatibility of dynamics with a direct decomposition does not make $\seqv$ irrelevant because $\dists$ still depends on $\seqv$.  For the purposes of the instantiation for control in the following appendix, we wish to control the imitation gaps on distances that do depend on $\seqv_h$, even though $\seqv_h$ does not figure directly into the dynamics.  Note that as defined, $\seqv_h$ does depend on the dynamics up until time $h-1$ and thus it is necessary to deal with this component in order to provide guarantees in $\dists$.

\subsection{A generalization of Theorem \ref{thm:smooth_cor}}\label{app:generalizationsmooth}
\newcommand{\epsvec}{\vec{\epsilon}}
\newcommand{\distsvec}{\vec{\dist}_{\cS}}
\newcommand{\distsi}[1][i]{{\dist}_{\cS,#1}}
\newcommand{\distsone}{\distsi[1]}

\newcommand{\distai}[1][i]{{\dist}_{\cA,#1}}
\newcommand{\distavec}{\vec{\dist}_{\cA}}
\newcommand{\gapjointvec}{\vec\Gamma_{\mathrm{joint},\epsvec}}
\newcommand{\gapmargvec}[1][\epsvec]{\vec\Gamma_{\mathrm{marg},#1}}
\newcommand{\drobvec}[1][\epsvec]{\vec{\dist}_{\mathrm{os},#1}}

We now state a generalization of \Cref{thm:smooth_cor}, which replaces a single distance by a vector of distances of dimension $K$; this will be useful for our instantiation of the composite MDP as a chunked control system in our final application (in particular, for deriving a bound on $\Imitfin$). It also showcases the most general structure accomodated by our proof technique. 

We begin by defining some notation:
\begin{itemize}
\item Let $K \in \N$ denote a dimension
\item Let $\epsvec \in \R_{\ge 0}^K$ denote a vector of tolerances
\item Let $\distsvec(\cdot,\cdot)$ denote a vector of pseudometrics $\distsi$ on $\cS$
\item Let $\distavec$ denote a vector of non-negative functions $\distai:\cA^2 \to \R_{\ge 0}$, not necessarily pseuometrics.
\item Let $\preceq$ denote vector wise inequality, and let the symbols $\wedge$ and $\vee$ be generalized to denote entrywise minima and maxima.  Similarly, addition of vectors is coordinate wise with scalars assumed to be broadcast appropriately.
\item We let $\distsi[1] = \disttvc$ denote the metric we consider for evaluating total variation distance. 
\end{itemize} 
We generalize We assume the following measure-theoretic regularity conditions, generalizing \Cref{ass:polishspaces} as follows.
\begin{assumption} \label{ass:polish_spaces_general}
    We assume that $\cS$ and $\cA$ are Polish spaces. This means they are metrizable, but we do not annotate their metrics because, e.g. the metric on $\cS$ may be other than $\dists$. We further assume that 
\begin{itemize}
\item $\distsi$ is a pseudometric and Borel measurable function from $\cS \times \cS \to \R_{\ge 0}$. 
\item For any $\epsilon \ge 0$, the set $\{(\seqa,\seqa') \in \cA \times \cA : \distai(\seqa,\seqa') > \epsilon\}$ is an open subset of $\cA\times \cA$; i.e. $\distai(\cdot,\cdot)$ is lower semicontinuous. In particular, this means $\distai$ is a Borel measurable function. Note that this implies that the 
\begin{align}\{(\seqa,\seqa') \in \cA \times \cA : \distavec(\seqa,\seqa') \not \preceq \epsvec\}.
\end{align}
is closed and thus measurable.
\end{itemize}
\end{assumption}
Note that the above assumption is the natural vectorized generalization of \Cref{ass:polishspaces}.  Next, we define vector versions of our imitation errors.
\begin{definition}[Imitation Errors, vector version]\label{defn:imit_gaps_vec} Given error parameter $\epsvec \in \R_{\ge 0}^K$, define 
\begin{itemize}
\item The \bfemph{vector joint-error} 
\begin{align}
\gapjointvec(\polhat \parallel \pist) := \inf_{\coup_1}\Pr_{\coup_1}\left[\exists h \in [H]: \distsvec(\shat_{h+1},\sstar_{h+1}) \vee \distavec(\seqast_h,\seqahat_h)   \not \preceq \epsvec\right],
\end{align} 
where the infimum is over trajectory couplings $((\shat_{1:H+1},\seqahat_{1:H}),(\sstar_{1:H+1},\seqa^\star_{1:H})) \sim \coup_1 \in \couple(\Dist_{\polhat},\Dist_{\polst})$ satisfying $\Pr_{\coup_1}[\shat_{1} = \sstar_1] = 1$.   
\item The \bfemph{vector marginal error} 
\begin{align}
\gapmargvec(\polhat \parallel \pist) := \max_{h \in [H]}\max\left\{\inf_{\coup_1}\Pr_{\coup_1}\left[\distsvec(\shat_{h+1},\sstar_{h+1})\not \preceq \epsvec\right],\, \inf_{\coup_1}\Pr_{\coup_1}\left[\distavec (\seqast_h,\seqahat_h)\not \preceq \epsvec\right]\right\}
\end{align} the same as the to joint-gap, with the ``$\max$'' outside the probability and infimum over couplings. 
\item The \bfemph{vector-wise one-step error}  
\begin{align}
\drobvec(\polhat_h(\seqs) \parallel \polst_h(\seqs)) := \inf_{\coup_2}\Pr_{\coup_2}\left[\distavec(\seqahat_h,\seqast_h) \not  \preceq \epsvec \right],
\end{align} where the infimum is over $(\seqast_h, \hat \seqa_h) \sim \coup_2 \in \couple( \bpolhat_h(\seqs),\bpol_h^\star(\seqs))$.
\end{itemize} 
\end{definition}

We now describe input stability. 
\begin{definition}[Input-Stability, vector version] \label{defn:fis_vector} A trajectory $(\seqs_{1:H+1},\seqa_{1:H})$ is \bfemph{input-stable} w.r.t. $(\distsvec,\distavec)$ if  all sequences $\seqs_1' = \seqs_1$ and $\seqs_{h+1}' = F_h(\seqs_h',\seqa_h')$ satisfy  
\begin{align}\distsi(\seqs_{h+1}',\seqs_{h+1}) \le  \max_{1 \le j \le h}\distai\left(\seqa_{j}',\seqa_j\right) ,\quad \forall h \in [H], i \in [K]
\end{align}
\end{definition} 


Finally, define input process stability. A slight technicality is that, in our instantiation, $\pist$ is taken to be a suitable regular condition probability of the joint distribution $\Dexp$ of expert trajectories. This means that $\pist$ can only really satisfy desired regularity conditions on  states visited with positive probabiliy by $\Dexp$. We address this subtlety by considering the following definition generalizing \Cref{defn:ips_body} in the body. We also restrict the kernels under consideration to those which produce distributions \emph{absolutely continuous} (\Cref{defn:abs_cont}) with respect to $\Psth$, and denoted with the $\ll$ comparator. More specifically, we only care about absolute continuity under the projections onto the $\cZ$ component of $\cS$. 
\begin{definition}[Input \& Process Stability, vector version]\label{defn:ips_vec}
Let $\pips \in (0,1)$, $\gamipsvec = (\gamipsi)_{1\le i \le K}$ be a collection non-decreasing maps $\gamipsi:\R_{\ge 0} \to  \R_{\ge 0}$, let   $\distips:\cS \times \cS \to \R$ be a pseudometric (possibly other than any of the $\distsi$), and $\rips > 0$.  We say a policy $\pist$ is \emph{$(\gamipsvec,\distips,\rips,\pips)$-(vectorwise-input-\&-process stable (vIPS)} if the following holds for any $r \in [0,\rips]$: 

Consider any sequence of kernels $\lawW_h:\cS \to \laws(\cS)$, $1\le h \le H$, satisfying 
\begin{align}
\forall h, \seqs \in \cS: \quad \Pr_{\tilde \seqs\sim \lawW_h(\seqs)}[\distips(\tilde \seqs,\seqs) \le r] = 1, \quad \phiZ \circ \lawW_h(\seqs) \ll \phiZ \circ \Psth. \label{eq:supp_contained}
\end{align}
Define a process $\seqs_1 \sim \Dinit$, $\tilde\seqs_h \sim \lawW_h(\seqs_h),\seqa_h \sim \pi_h(\tilde \seqs_h)$, and $\seqs_{h+1} := F_h(\seqs_h,\seqa_h)$. Then, with probability at least $1- \pips$,
\begin{itemize}
\item[(a)] the sequence $(\seqs_{1:H+1},\seqa_{1:H})$ is input-stable w.r.t $(\distsvec,\distavec)$ (as defined by \Cref{defn:fis_vector}).
\item[(b)]$\max_{h \in [H]} \distsi(F_h(\tilde\seqs_h,\seqa_h),\seqs_{h+1}) \le \gamipsi(r)$. 
\end{itemize}
\end{definition}
\newcommand{\epsvecmarg}{\epsvec_{\mathrm{marg}}}




We can now state our desired generalization. 



\begin{theorem}\label{thm:smooth_cor_general}   Suppose that there 
\begin{itemize}
\item[(a)]$\pist$ is $(\gamipsvec,\distips,\rips,\pips)$-vector IPS in the sense of \Cref{defn:ips_vec}.
\item[(b)] There is a direct decomposition of $\cS = \cZ \oplus \cV$, which associated projection maps $\phiZ$ and $\phiV$, and which is compatible with the dynamics, and policies $\pist$, $\pihat$, and smoothing kernel $\Wsig$, and $\distips$.
\item[(c)]  $\phiZ \circ \Wsig$ is $\gamma_{\sigma}$-TVC with respect to the pseudometric $\disttvc = \distsone$. 
\end{itemize} 
Let $\pihatsig$ be any policy which is $\gamhat$-TVC, also w.r.t. $\disttvc = \distsone$. Finally, let $\epsvec \in \R_{\ge 0}^K$, $r \in (0,\frac{1}{2}\rips]$, and define 
\begin{align}
p_r &:= \sup_{\seqs}\Pr_{\seqs' \sim \Wsig(\seqs)}[\distips(\seqs',\seqs) >  r], \quad \epsvecmarg := \epsvec + \gamipsvec(2r).
\end{align} Then, 
\begin{itemize}
\item For any policy $\pihat$,  both  $\gapjointvec (\pihatsig  \parallel \pistrep)$ and  $\gapmargvec[\epsvecmarg] (\pihatsig \parallel \pist)$ are upper bounded by%$\gapmarg[\epsilon + 2r](\pihat \circ \Wsig \parallel \pist)$ are both at most
\begin{align}
%\inf_{r > 0}  
\pips + H(2p_r + \gamhat(\epsvec_1) + (\gamhat + \gamtvcsig) \circ \gamipsone(2r))  + \sum_{h=1}^H\Exp_{\sstar_h \sim \Psth}\drobvec\,( \pihatsigh(\stel_h) \parallel \pistreph(\stel_h)) \label{eq:smooth_ub_app_one}
\end{align}
\item In the special case where $\pihatsig = \pihat \circ \Wsig$, we can take $\gamhat = \gamsig$, and obtain that $\gapjointvec(\pihatsig \parallel \pistrep)$ and $\gapmargvec[\epsvecmarg](\pihatsig \parallel \pist)$ are upper bounded by
\begin{align}
\pips + H\left(2p_r +  3\gamma_{\sigma}(\max\{\epsilon,\gamipsone(2r)\}\right)  + \textstyle \sum_{h=1}^H\Exp_{\sstar_h \sim \Psth}\Exp_{\sstartil_h \sim \Wsig(\sstar_h) } \drobvec( \pihat_{h}(\sstartil_h) \parallel \pidec(\sstartil_h)) . \label{eq:smooth_ub_app_two}
\end{align}
\end{itemize}
\end{theorem}
We note that \Cref{thm:smooth_cor} is  a special case of \Cref{thm:smooth_cor_general} and prove the former assuming the latter here at the end of the section.

\subsection{Proof of Theorem \ref{thm:smooth_cor_general} }\label{app:smoothcor_general_proof}



\subsubsection{Proof Overview and Coupling Construction}\label{sec:proof_construction}
We begin with an intuitive overview of the proof and partially construct the relevant intermediate trajectories used to define our coupling, after which we sketch the organization of the rest of Appendix \ref{app:smoothcor_general_proof}.

The proof proceeds by constucting a sophisticated coupling between the law of a trajectory evolving according to $\pihat$ and a trajectory evolving according to $\pistrep$ by introducing several intermediate sequences of composite states and composite actions.  

We partially specify this coupling below and formally construct it in Appendix \ref{app:proof_smooth_cor_general}.  Our construction is recursive and relies on the input and process stability as well as total variation continuity to show that if the trajectories generated by $\pistrep$ and $\pihat$ are close in $\drobvec[\epsvec]$ evaluated on states at step $h$, then they will remain close at step $h+1$.  There are a number of technical subtelties involved, especially those of a measure-theoretic nature, but much of the inuition can be gleaned from the following partial specification of the coupling $\coup$ over composite-state 
$(\shat_{1:H},\srep_{1:H},\stel_{1:H},\ssq_{1:H}) \subset \cS$, composite-actions  $(\arep_{1:H},\seqahat_{1:h},\atel_{1:H}) \subset \cK$ and interpolating composite-actions, $(\arepinter_{1:H},\atelinter_{1:H}) \subset \cA$. 

To define the construction, we define the probability kernels corresponding to the replica and deconvolution policies.  Note that these are slightly different from the definitions in the body due to the use of the direct decomposition; the intuition is the same, however.

\newcommand{\QdechZ}[1][h]{\lawW^{\star}_{\mathrm{dec},\cZ,h}}

\begin{definition}[Replica and Deconvolution Kernels]\label{defn:all_kernels} Let $\Paughproj$denote the joint distribution over $(\zst_h,\sstar_h,\zstil_h,\astar_h)$ under the generative process
\begin{align}
\sstar_h \sim \Psth, \quad \astar_h \sim \pist_h(\sstar_h), \quad 
\zst_h = \phiZ(\sstar_h), \quad \zstil_h \sim \phiZ\circ \Wsig(\sstar_h)
\end{align}
For $\seqz \in \cZ$, let $\QdechZ(\seqz)$ denote the distribution of $\zst_h$ conditioned on $\zstil_h = \seqz$, under $\Paughproj$. Given $\seqs = (\seqz,\seqv)$, define 
\begin{align}
&\Qdech(\seqs) = \QdechZ(\phiZ(\seqs)) \otimes \dirac_{\phiV(\seqs)}, \quad \\
&\Qreph(\seqs) = \Qdech \circ ( \Wsig(\phiZ(\seqs))\otimes \dirac_{\phiV(\seqs)}) =   (\QdechZ \circ \Wsig(\phiZ(\seqs)))\otimes \dirac_{\phiV(\seqs)}.
\end{align}
where we recall the dirac-delta $\dirac$. Equivalently, $\Qdech(\seqs)$ denotes the conditional sequence of $(\tilde \seqz,\seqv)$, where $\seqv = \phiV(\seqs)$, and $\tilde \seqz \sim \QdechZ(\seqs)$; $\Qreph$ can be expressed similarly. 
\end{definition}
We remark that $\Qdech$ and $\Qreph$ are both kernels and by \Cref{thm:durrett}, we may assume that the joint distribution over $(\sstar_h, \ssq_h)$ admits a regular conditional probability and thus these constructions are well-defined. 
\begin{remark}Note that the kernels $\Qdech$ and $\Qreph$ are  compatible with the decomposition $\cS = \cZ \oplus \cV$ by construction. Moreover, note that if $\seqs = (\seqz,\seqv)$, $\phiV \circ \Qdech(\seqs) = \phiV \circ \Qreph(\seqs)$ is the dirac-delta distribution supported on $\seqv$.
\end{remark}
\begin{lemma} Under our the assumption that $\pist$ and $\Wsig$ are compatible with the direct decomposition,  
\begin{align}
\pidech(\seqs) = \pist \circ \Qdech , \quad \pistreph(\seqs) = \pist \circ \Qreph 
\end{align}
\end{lemma}
\begin{proof} This follows imediately because $\pist$ and $\Wsig$ are  compatile with the direct decomposition, and by the definition of \Cref{defn:body_replica}.
\end{proof}


% Figure environment removed
\paragraph{A template for the coupling.} Our couplings are partially specified by the following generative process, and what remains unspecified are couplings between random variables at each each step $h$. In what follows, let $\cF_0$ denote the $\upsigma$-algebra generatived by $\shat_1 = \srep_1 = \stel_1 $. Let $\cF_h$ denote the sigma-algebra generated by  $(\shat_{1:h},\srep_{1:h},\stel_{1:h})$, $(\arep_{1:h},\sreptil_{1:h},\ssq_{1:h},\atel_{1:h},\seqahat_{1:h})$, and $(\arepinter_{1:h},\atelinter_{1:h})$.
\begin{itemize}
    \item The initial states are drawn as
    \begin{align}
    \shat_1 = \srep_1 = \stel_1 \sim \Dinit. 
    \end{align}
    \item The dynamics satisfy
    \begin{align}
    \shat_{h+1} = F_h(\shat_h,\seqahat_h), \quad \srep_{h+1} = F_h(\srep_h,\arep_h), \quad \stel_{h+1} = F_h(\ssq_h,\atel_h)
    \end{align}
    Note that determinism of the dynamics implies that $\stel_{h+1}$, $\srep_{h+1}$ and $\shat_{h+1}$ are $\cF_{h}$-measurable. 
    \item We generate
    \begin{align}
    &\sreptil_h \mid \cF_{h-1} \sim \Qreph(\srep_h), \quad \arep_h \mid \cF_{h-1},\sreptil_h \sim \pisth(\sreptil_h), \qquad \label{eq:trajevolve1} \\
    &\ssq_h \mid \cF_{h-1} \sim \Qreph(\stel_h), \quad \atel_h \mid \cF_{h-1},\ssq_h \sim \pisth(\ssq_h).\label{eq:trajevolve2}\\
    &\seqahat_h \mid \cF_{h-1} \sim \pihatsigh(\shat_h) \label{eq:trajevolve_ahat}
\end{align}
Importantly, we note that, marginalizing over $\ssq_h$ and $\sreptil_h$, respectively, $\atel_h \mid \cF_{h-1} \sim \pistreph(\stel)$ and $\arep_h \mid \cF_{h-1} \sim \pistreph(\srep_h)$.  
\item Lastly, we select interpolating actions via
\begin{align}
    &\arepinter_h \mid \cF_{h-1} \sim \pihatsigh(\srep_h), \qquad \atelinter_h \mid \cF_{h-1} \sim \pihatsigh(\stel_h)\label{eq:trajevolve3}
\end{align}
\end{itemize}
We will say $\coup$ is ``respects the construction'' as shorthand to mean that $\coup$ obeys the above equations.  The coupling is illustrated graphically in \Cref{fig:coupling_illustration}.  We now establish several key properties of the above constructions, separated into a subsection for the sake of clarity.


\paragraph{Organization of the remaining parts of Appendix \ref{app:smoothcor_general_proof}.}   In Appendix \ref{app:prop_of_deconv_replica}, we prove several prerequisite properties of the construction given above, including concentration of the smoothing kernel, and key properties of the replica distribution. Next, Appendix \ref{app:marg_imit_gap} shows that, due to these properties of the replica distribution, we can bound the marginal imitation gap by controlling the tracking of the teleporting sequence constructed above. Finally, in Appendix \ref{app:proof_smooth_cor_general} we formally construct the coupling and rigorously prove \Cref{thm:smooth_cor_general}.
\begin{comment}
\begin{observation} Let $\trajhat = (\shat_{1:H},\seqahat_{1:H})$, $\trajrep_{1:H} = (\sstar_{1:H},\seqast_{1:H})$, $\trajtel = (\stel_{1:H},\atel_{1:H})$.
\begin{itemize}
\item $\coup$ is an interpolating construction for $(\trajrep,\trajhat,\arepinter_{1:H})$ with respect to $(\pistrep,\pihat,(\cF_{h})_{h \ge 0})$.
\item $\coup$ is a teleporting construction for $(\trajtel,\trajrep,\ssq_{1:H})$ with respect to $(\pist,\Wsig,(\cF_h)_{h \ge 0})$. 
\end{itemize}
\end{observation}
\end{comment}




\newcommand{\Ctelh}[1][h]{\cC_{\mathrm{tel} ,#1}}
\newcommand{\Crephath}[1][h]{\cC_{ \hat{\seqs},#1}}
\newcommand{\Binterh}[1][h]{\cB_{ \mathrm{inter},#1}}
\newcommand{\Bhath}[1][h]{\cB_{\hat{\seqa},#1}}
\newcommand{\Btelh}[1][h]{\cB_{\mathrm{tel},#1}}
\newcommand{\Bfsh}[1][h]{\cB_{\mathrm{est},#1}}
\newcommand{\Callh}[1][h]{\cC_{\mathrm{all},#1}}
\newcommand{\Callbarh}[1][h]{\bar\cC_{\mathrm{all},#1}}
\subsubsection{Properties of smoothing, deconvolution, and replicas.}\label{app:prop_of_deconv_replica}

In this section, we establish several useful properties of smoothed and replica policies.  We begin by showing that smoothed policies are TVC.
\begin{lemma}\label{lem:pistrep_tvc}
The following hold
\begin{itemize}
    \item For any $h$, $\phiZ \circ \Qreph$ and $\pistreph$ are $\gamma_{\sigma}$ TVC.
    \item If $\pi$ is any policy compatible with the direct decomposition $\cS = \cZ \oplus \cV$ (in the sense of \Cref{defn:direct_decomp}), then $\pi\circ \Wsig$ is $\gamma_{\sigma}$-TVC.
\end{itemize}
\end{lemma}
\begin{proof} We observe that $\phiZ \circ \Qreph = \phiZ \circ \Qdech \circ \Wsig(\seqs)$. Moreover, we observe $\Qdech$ satisfies  $\phiZ \circ \Qdech(\seqs) =  \QdechZ \circ \phiZ$, so that $\phiZ \circ \Qreph = \QdechZ \circ \phiZ \circ \Wsig(\seqs)$. As $\phiZ \circ \Wsig$ is TVC, the first claim is a consequence of the data-processing inequality \Cref{cor:tv_two}. The second uses the fact that all listed objects involve composition of kernels with $\Wsig$.
\end{proof}
Next, we show that the replica construction preserves marginals. 
\begin{lemma}[Marginal-Preservation]\label{lem:replica_property} 
 There exists a coupling $\Pr$ of $\seqz_h \sim \phiZ \circ \Psth$, $\seqz_h' \sim \phiZ\circ\Wsig(\seqz_h,\cdot)$ (where ($\cdot$) denotes an irrelevant argument due to compatibility of $\Wsig$ with the direct decomposition), and $\tilde \seqz_h \sim \phiZ \circ \Qreph(\seqz_h,\cdot)$ (again, ($\cdot$) denotes an irrelevant argument) such that 
 \begin{align}
 (\seqz_h,\seqz_h') \overset{\mathrm{d}}{=}  (\tilde\seqz_h,\seqz_h').
 \end{align}
 In particular, for $\stel_h$ and $\ssq_h$ as in our construction, the marginal distributions of $\phiZ(\stel_h)$ and $\phiZ(\ssq_h)$ are the same, where $\stel_h \sim \Psth$ and  $\ssq_h \mid \stel_h \sim \Qreph(\stel_h)$.
\end{lemma}
\begin{proof}
    By \Cref{ass:polishspaces} and \Cref{thm:durrett}, we may assume that all joint distributions' conditional probabilities are regular conditional probabilities and thus almost surely equal to a kernel.  Moreover, since all kernels are compatible with the direct decomposition, it suffices to prove the special case of the trivial direct-decomposition where $\cZ = \cS$.  Fix a common measure $\pp$ over which $\stel_h, \ssq_h$, and $\mathsf{s}_h'$ are defined such that $\stel_h \sim \Psth$, $\mathsf{s}_h' \sim \Wsig(\stel_h)$, and $\ssq_h \sim \Wdeconvh(\mathsf{s}_h')$. Then for any measurable sets $A, B$, we have
    \begin{align}
        \pp(\stel_h \in A,\, \seqs_h' \in B) &= \pp(\seqs_h' \in B) \cdot \ee_{\seqs_h'}\left[\I[\seqs_h' \in B] \cdot \pp(\stel_h \in A | \seqs_h' ) \right] \\
        &= \pp(\seqs_h' \in B) \cdot \ee_{\seqs_h'}\left[\I[\seqs_h' \in B] \cdot \pp(\ssq_h \in A | \seqs_h' ) \right]\\
        &= \pp\left( \ssq_h \in A, \, \seqs_h' \in B \right),
    \end{align}
    where the first equality holds by the fact that we are working with regular conditional probabilities and Bayes' rule, the second equality holds by the definition of the deconvolution kernel above, and the last equality holds again by Bayes' rule and the tower rule for conditional expectations.

    To prove the second statement, we apply induction, again assuming that $\cZ = \cS$ as in the proof of the first statement.  Note that $\stel_1 \sim \Psth[1] = \Dinit$, and $\ssq_1 \sim \Qreph[1] \circ \Psth[1]$. Thus, from the first part of the lemma, $\phiZ (\stel_1) \sim \phiZ \circ \Psth[1]$. Now, suppose the induction holds up to step $h$. Then, $\ssq_h \sim \Psth$, as $\atel_h \sim \pist_h(\atel_h)$, then $\stel_{h+1} = F_{h}(\ssq_h,\atel_h) \sim \Psth[h+1]$. Again  $\ssq_{h+1} \sim \Qreph[h+1](\stel_{h+1})$, so that $\ssq_{h+1}$ has marginal $\Qreph[h+1]\circ \Psth[h+1] = \Psth[h+1]$, as needed.  
\end{proof}
We further show that $\Wreph$ can be defined to be absolutely continuous with respect to $\Psth$.
\begin{lemma}\label{lem:absolute_continuity}
    The kernel $\Wreph$ satisfies that $\phiZ \circ \Wreph \ll \phiZ \circ \Psth$ as laws, validating the second condition in \eqref{eq:supp_contained}.  It further holds that $\phiZ \circ \Wdeconvh \ll \phiZ \circ \Psth$.
\end{lemma}
\begin{proof}
    The first statement follows immediately from \Cref{lem:replica_property} because these distributions are the same.  The second statement follows immediately from the tower law of conditional expectation and the definition of $\Wdeconvh$.
\end{proof}

Lastly, we establish that the replica kernel inherits all concentration properties from the smoothing kernel.
\begin{lemma}[Replica Concentration]\label{lem:rep_conc} Recall that 
\begin{align}
p_r := \sup_{\seqs}\Pr_{\seqs' \sim \Wsig(\seqs) }[\distips(\seqs',\seqs) >  r].
\end{align} We then have
\begin{align}
\Pr_{\seqs_h \sim \Psth,\stil_h \sim \Qreph(\seqs_h)}[\distips(\stil_h,\seqs_h) > 2\rsmooth] \le 2p_r \label{eq:concentration_conv_two}
\end{align}
\end{lemma}
\begin{proof} %We fix a common measure $\Pr[\cdot]$ over $\seqs_h \sim \Psth,\stil_h \sim \Qreph(\seqs_h)$
Again, all terms -- $\Wsig,\Qreph,\Qdech$ and $\distips$ -- are compatible with the direct decomposition, it suffices to consider the case of the trivial direct decomposition under whcih $\cZ = \cS$.

Let $\Pr$ denote a distribution over $\seqs_h \sim \Psth$, $\seqs_h' \sim \Wsig(\seqs_h)$, and $\stil_h \sim \Qdech(\seqs_h')$. In this special case,  we see that $\stil_h \mid \seqs_h \sim \Qreph(\seqs_h)$\footnote{Notice that, for general $\cS = \cZ \oplus \cV$, this condition would become $\phiZ(\stil_h) \mid \phiZ(\seqs_h) \sim \phiZ \circ \Qreph(\phiZ(\seqs_h),\cdot)$, where the $\cdot$ argument is irrelevant.}. By a union bound,
\begin{align}\label{eq:dists_conv_bound_two}
\Pr_{\seqs_h \sim \Psth,\stil_h \sim \Qreph(\seqs_h)}[\distips(\seqs_h,\stil_h) > 2\rsmooth] &\le \Pr[\distips(\stil_h,\seqs'_h) > \rsmooth]  + \Pr[\distips(\seqs_h,\seqs'_h) > \rsmooth] \\
&= 2 \Pr[\distips(\seqs_h,\seqs'_h) > \rsmooth] \le 2p_r,
\end{align}
where the equality follows from the first statment of \Cref{lem:replica_property}.
\end{proof}
\begin{remark}Note that, in the previous lemma, it suffices that the following weaker condition holds: $\Pr_{\seqs \sim \Psth,\seqs' \sim \Wsig(\seqs)}[\distips(\seqs',\seqs) >  \rsmooth] \le p_r$, i.e. for concentration to hold only in distribution over $\seqs \sim \Psth$, instead of \emph{uniformly} over states.
\end{remark}
\newcommand{\Qtilreph}[1][h]{\tilde{\lawW}_{\repsymbol,#1}}


\subsubsection{Bounding the marginal imitation gaps in terms of the teleporting sequence error}\label{app:marg_imit_gap}
Before turning to the proof of \Cref{thm:smooth_cor_general}, we verify that closeness to the \emph{teleporting sequences} suffices to control error in marginal gap to $\pist$. The key property here is that the teleporting sequence, as shown in \Cref{lem:replica_property}, has the same marginal distribution over states as does $\pist$.

\begin{lemma}\label{lem:marg_imit_gap_tel} Let $\coup$ be any coupling obeying the construction of the couplings above. Then, 
\begin{align}
\gapmargvec(\pihatsig \parallel \pist) \le 
\Pr_{\coup}\left[\exists h \in [H]: \left\{\distsvec(\stel_{h+1},\shat_{h+1}) \not \preceq \epsvecmarg\right\} \cup \left\{ \distavec(\atel_h,\ahat_h) \not \preceq \epsvecmarg\right\}\right] 
\end{align}
\end{lemma}
\begin{proof}
 We begin with a (reverse) union bound.
\begin{align}
&\Pr_{\coup}\left[\exists h \in [H]: \left\{\distsvec(\stel_{h+1},\shat_{h+1}) \not \preceq \epsvecmarg\right\} \cup \left\{ \distavec(\atel_h,\ahat_h) \not \preceq \epsvecmarg\right\}\right] \\
&\ge\max_h\max\left\{\Pr_{\coup}\left[\distsvec(\stel_{h+1},\shat_{h+1}) \not \preceq \epsvecmarg\right],\, \Pr_{\coup}\left[\distavec(\atel_h,\ahat_h) \not \preceq \epsvecmarg\right]\right\}.
\end{align}
 By \Cref{lem:replica_property} implies that $\stel_h$ has the marginal distribution of $\sstar_h \sim \Psth$. Moreover, by construction, for each $h$, $\atel_h \mid \cF_h \sim \pistreph(\stel_h)$, Thus, for each $h$, $\stel_{h+1}$ and $\atel_h$ have the same \emph{marginals} as the marginals as $\sstar_{h+1}$ and $\astar_h$ under the distribution $\Dist_{\pist}$ induced by $\pist$. Hence, 
 \begin{align}
 \Pr_{\coup}\left[\distsvec(\stel_{h+1},\shat_{h+1}) \not \preceq \epsvecmarg\right] &\ge \inf_{\coup_1} \Pr\left[\distsvec(\sstar_{h+1},\shat_{h+1}) \not \preceq \epsvecmarg\right] \\
 \Pr_{\coup}\left[\distavec(\atel_{h},\ahat_{h}) \not \preceq \epsvecmarg\right] &\ge \inf_{\coup_1} \Pr\left[\distsvec(\astar_{h},\ahat_{h}) \not \preceq \epsvecmarg\right],
 \end{align}
 where the $\inf_{\coup_1}$ is, as  in \Cref{defn:imit_gaps,defn:imit_gaps_vec}, the infinum over couplings between $\Dist_{\pist}$ and $\Dist_{\pihat}$. Thus, 
 \begin{align}
&\Pr_{\coup}\left[\exists h \in [H]: \left\{\distsvec(\stel_{h+1},\shat_{h+1}) \not \preceq \epsvecmarg\right\} \cup \left\{ \distavec(\atel_h,\ahat_h) \not \preceq \epsvecmarg\right\}\right] \\
&\ge\max_h\max\left\{\inf \Pr_{\coup_1}\left[\distsvec(\sstar_{h+1},\shat_{h+1}) \not \preceq \epsvecmarg\right],\, \inf_{\coup}\Pr_{\coup_1}\left[\distavec(\astar_h,\ahat_h) \not \preceq \epsvecmarg\right]\right\}\\
&:= \gapmargvec(\pihatsig \parallel \pist).
\end{align}

\end{proof}

\subsubsection{Formal proof of Theorem \ref{thm:smooth_cor_general}}\label{app:proof_smooth_cor_general}
We now proceed to formally prove \Cref{thm:smooth_cor_general}

\paragraph{Key Events. } For the random variables defined above, we define three groups of events. 
\begin{itemize}
\item The \emph{coupling events}, denoted by $\cB$, which are controlled by carefully selecting a coupling.
\item The \emph{inductive events}, denoted by $\cC$, which we condition on when bounding the probability of the coupling events.
\item The \emph{stability events}, denoted by $\cQ$, which take advantage of the stability properties of the imitation policy. 
\end{itemize}
\newcommand{\Ballbarh}[1][h]{\bar\cB_{\mathrm{all},#1}}
\newcommand{\Ballh}[1][h]{\cB_{\mathrm{all},#1}}
\newcommand{\Qis}{\cQ_{\textsc{is}}}
\newcommand{\Qips}{\cQ_{\textsc{ips}}}
\newcommand{\Qclose}{\cQ_{\mathrm{close}}}
\newcommand{\Qall}{\cQ_{\mathrm{all}}}

\begin{definition}[Coupling Events]\label{defn:all_key_eents} Define the events
\begin{align}
    \Btelh &=  \left\{ \arep_h = \atel_h, ~\phiZ(\sreptil_h) = \phiZ(\ssq_h) \right\}\\
     \Bfsh &= \left\{ \distavec( \atelinter_h,\atel_h) \not \preceq \epsvec \right\} \\
    \Binterh &= \left\{ \atelinter_h = \arepinter_h  \right\} \\
    \Bhath &= \left\{  \arepinter_h = \seqahat_h \right\} \\
     \Ballh &= \Binterh \cap \Btelh \cap \Bfsh \cap \Bhath\\
    \Ballbarh &=  \bigcap_{j=1}^h \Ballh[h]
\end{align}
Notice that each of the events above are $\cF_{h}$-measurable. Moreover, note that on $\Ballbarh$, $\max_{1\le j \le h}\phiis(\seqahat_j,\arep_j) \le \epsilon$.
\end{definition}
\begin{definition}[Inductive Event]\label{def:inductive_event}

Define the events
\begin{align}
\Crephath &= \left\{  \distsvec(\srep_h, \shat_h) \preceq \epsvec \right\}, \\
\Ctelh &= \left\{  \distsvec(\srep_h, \stel_h) \preceq \gamipsvec(2r) \right\} \\
\Callh &:= \Crephath \cap \Ctelh\\
 \Callbarh &=  \bigcap_{j=1}^h \Callh[j]
\end{align}
Notice that all the above events are $\cF_{h-1}$-measurable, due to determinism of the dynamics. Note that also $\Pr_{\coup}[\Callbarh[1]] = 1$ for any $\coup$ that respects the construction (as $\srep_1 = \stel_1 = \shat_1$).
\end{definition} 
\begin{definition}[Stability Events] Define the events 
\begin{align} 
\Qclose &:= \left\{\forall h \in [H]: \distips(\srep_h,\sreptil_h) \le 2r \right\}\\
\Qis &:= \left\{(\srep_{1:H+1},\arep_{1:H}) \text{ is input-stable w.r.t. } (\distsvec,\distavec)\right\}\\
\Qips &:= \left\{\distsvec(F_h(\sreptil_{h},\arep_h),\srep_{h+1}) \le  \gamipsvec\circ \distips\left(\sreptil_h,\srep_{h}\right), \quad 1 \le j \le H\right\} \\
\Qall &:=  \Qips \cap \Qclose .
\end{align}
In words, $\Qclose$ the event on which $\srep_h$ and $\sreptil_h \sim \Qreph(\stel_h)$ are close, and $\Qis$and  $\Qips$ ensure consequencs of  (vector) input-stability and (vector) input process stability holds.
\end{definition}

\paragraph{Steps of the proof.}
First, we use stability to reduce the event $\Callbarh[h+1]$ to $\Callbarh \cap \Ballbarh$:
\begin{claim}[Stability Claim]\label{claim:stability_claim} By construction, 
\begin{align}\Callbarh[h+1] \subset \Qall \cap \Callbarh \cap \Ballbarh.
\end{align}
\end{claim} 
\begin{proof} It suffices to show that on $\Qall \cap \Callbarh \cap \Ballbarh$, $ \distsvec(\srep_{h+1},\shat_{h+1}) \preceq \epsvec$ and $\distsvec(\srep_{h+1},\stel_{h+1}) \preceq \gamipsvec(2r)$. By applying the event $\Qis$ to the sequence $\seqa'_h = \seqahat_h$ and $\seqs'_h = \shat_h$, we have that on $\Qall \subset \Qis$ that
\begin{align}
 \forall h \in [H], i \in [K], \quad \distsi(\srep_{h+1},\shat_{h+1}) \le  \max_{1 \le j \le h}\distai\left(\arep_j,\seqahat_{j}\right) \label{eq:Qis_consequence}
\end{align}


For the next point, note that the compatibility of the dynamics with the direct decomposition $\cS = \cZ \oplus \cV$ implies that there exists a dynamics map $F_h^{\cZ}$  for which 
\begin{align}
F_h(\seqs,\seqa) = F_h^{\cZ}(\phiZ(\seqs),\seqa).
\end{align}
Similarly, by applying $\Qips$ and $\Qclose$ and the event $\{\phiZ(\sreptil_h) = \phiZ(\ssq_h),\atel_h = \arep_h\}$ on $\Btelh$, it holds that on $\Qall \cap \Callbarh \cap \Ballbarh$ that, for all $h \in [H]$,
\begin{align}
\distsvec(\srep_{h+1},F_h(\sreptil_{h},\arep_h))  &= \distsvec(\srep_{h+1},F_h^\cZ(\phiZ(\sreptil_{h}),\arep_h)) \\
&= \distsvec(\srep_{h+1},F_h^\cZ(\phiZ(\ssq_{h}),\atel_h)) \tag{$\Btelh$}\\
&= \distsvec(\srep_{h+1},F_h(\ssq_{h},\atel_h)) \\
&= \distsvec(\srep_{h+1},\stel_{h+1})\\ 
&\le \gamipsvec\circ\distips\left(\stel_j,\ssq_{j}\right) \tag{$\Qips$}\\
&\le \gamipsvec\circ\distips\left(2r\right) \tag{$\Qclose$}.
\end{align}
\end{proof}
From \Cref{claim:stability_claim}, we decompose our error probability as follows:
\begin{lemma}[Key Error Decomposition] \label{lem:putting_couplings_together} Let $\coup$ respect the construction (in the sense of \Cref{sec:proof_construction}). Then, for any coupling $\coup$ which respects the construction,
\begin{align}
&\gapjointvec(\pihatsig \parallel \pirep) \vee \gapmargvec(\pihatsig \parallel \pist) \le \Pr_{\coup}[\Qall^c] + \sum_{h=1}^H\Pr_{\coup}[ \Ballbarh^c \cap \Callbarh \cap \Ballbarh[h-1]]\label{eq:Gamimit_decomp}
\end{align}
\end{lemma}
\begin{proof} In what follows, we use $\vec{v} \vee \vec{w}$ to denote the entrywise maximum of two vectors of the same dimension. Define the events $\cE_h := \Callbarh[h+1] \cap \Ballbarh$. Observe that the events are nested: $\cE_{h} \supset \cE_{h+1}$, and that on $\cE_H$, we have that for all $h \in [H]$
\begin{align}
\distsvec(\srep_{h+1},\shat_{h+1}) \vee \distavec(\arep_h,\ahat_h) &\preceq \epsvec \vee \distavec(\arep_h,\ahat_h) \tag{$\Crephath[h+1] \supset \Callbarh[h+1] \supset \cE_h$}\\
&\preceq \epsvec \tag{$\Ballbarh \supset \cE_h$}.
\end{align}
On $\Qall \cap \cE_H$, we have that
\begin{align}
\max_h \distsvec(\srep_h,\stel_h) \le \gamipsvec(2r), \quad \text{and}\quad \atel_h = \arep_h
\end{align}
Thus, by the triangle inequality and $\epsvecmarg = \epsvec + \gamipsvec(2r)$, on $\Qall \cap \cE_H$,
\begin{align}
\max_h \distsvec(\srep_h,\stel_h) \le \epsvecmarg, \quad \text{and}\quad  \distavec(\atel_h,\ahat_h)  =\distavec(\arep_h,\ahat_h)  \le \epsvec \le \epsvecmarg.
\end{align}
Thus, 
\begin{align}
&\Pr_{\coup}\left[\exists h \in [H]: \left\{\distsvec(\srep_{h+1},\shat_{h+1}) \vee \distavec(\arep_h,\ahat_h) \not \preceq \epsvec\right\} \cup \left\{\distsvec(\stel_{h+1},\shat_{h+1}) \vee \distavec(\atel_h,\ahat_h) \not \preceq \epsvecmarg\right\}\right]\\
&\le \Pr_{\coup}[(\Qall \cap \cE_H)^c] \label{eq:first_pr_coup_big}
\end{align}
In particular, this shows that
\begin{align}
\gapjointvec(\pihatsig \parallel \pirep)  \le \Pr_{\coup}[(\Qall \cap \cE_H)^c], 
\end{align}
and similarly, by \Cref{lem:marg_imit_gap_tel},
\begin{align}
\gapmargvec(\pihatsig \parallel \pist) \le \Pr_{\coup}[(\Qall \cap \cE_H)^c]
\end{align}
As $(\srep_{1:H+1},\arep_{1:H}) \sim \Dist_{\pistrep}$, \eqref{eq:first_pr_coup_big} shows that
\begin{align}\gapjointvec(\pihatsig \parallel \pirep) \vee \gapmargvec(\pihatsig \parallel \pirep)\le \Pr_{\coup}[(\Qall \cap \cE_H)^c].
\end{align} 
Let us conclude by bounding $\Pr_{\coup}[(\Qall \cap \cE_H)^c]$. Using the nesting structure $\cE_h = \bigcap_{j=1}^h \cE_j$, the peeling lemma, \Cref{lem:peeling_lem}, and a union bound, it holds that
\begin{align}
\Pr_{\coup}\left[(\Qall \cap \cE_H)^c\right] &\le \Pr_{\coup}[\Qall^c] + \Pr\left[ \exists h \in [H] \text{ s.t. } \left(\Qall \cap \cE_{h-1} \cap \cE_h^c\right) \text{ holds } \right ]\\
&\le \Pr_{\coup}[\Qall^c] + \sum_{h=1}^H\Pr_{\coup}\left[ \Qall \cap \cE_{h-1} \cap \cE_h^c \text{ holds } \right ]\\
&= \Pr_{\coup}[\Qall^c] + \sum_{h=1}^H\Pr_{\coup}\left[ \Qall \cap \Ballbarh[h-1] \cap \Callbarh[h]  \cap (\Ballbarh \cap \Callbarh[h+1])^c \text{ holds } \right ]\\
&= \Pr_{\coup}[\Qall^c] + \sum_{h=1}^H\Pr_{\coup}\left[ \Qall \cap \Ballbarh[h-1] \cap \Callbarh[h]  \cap \Ballbarh^c\right ]\\
&= \Pr_{\coup}[\Qall^c] + \sum_{h=1}^H\Pr_{\coup}\left[ \Qall \cap \Ballbarh[h-1] \cap \Callbarh[h]  \cap \Ballh^c\right ],
\end{align}
where the last step invokes \Cref{claim:stability_claim}.
\end{proof}
Next, we bound the contribution of $\Pr_{\coup}[\Qall^c]$ in \eqref{eq:Gamimit_decomp}, uniformly over all couplings.
\begin{lemma}\label{lem:Qall_bound} For all $\coup$ which respect the construnction, 
\begin{align}
\Pr_{\coup}[\Qall^c] \le \pips + 2Hp_r.
\end{align}
\end{lemma}
\begin{proof} $\Pr_{\coup}[\Qclose^c] = \Pr_{\coup}[\exists h: \distips(\stel_h,\ssq_h) > 2r] \le 2Hp_r$ by \Cref{lem:rep_conc} and a union bound. 


Let us now bound $\Pr_{\coup}[\Qclose \cap \Qips^c] \le \Pr_{\coup}[\Qips^c \mid \Qclose ]$. Define the kernels $\lawW_h(\seqs)$ to be equal to the kernel $\Wreph(\seqs)$ conditioned on the event $\seqs' \sim \Wreph(\seqs)$ satisfies $\distips(\seqs',\seqs) \le 2r$. Then, conditional on $\Qclose$, we see that the sequence $(\srep_{1:H+1},\sreptil_{1:H},\arep_{1:H})$ obeys the generative process
\begin{align}
\sreptil_{h} \mid \sreptil_{1:h-1},\srep_{1:h},\arep_{1:h-1} \sim \lawW_h(\seqs), \quad \arep_{h} \mid \sreptil_{1:h},\srep_{1:h},\arep_{1:h-1} \sim \pisth(\sreptil_h), \quad \srep_{h+1} = F_h(\srep_h,\arep_h).
\end{align} 
By construction, for each $h$, $\Pr_{\seqs' \sim \Wreph(\seqs)}[\distips(\seqs',\seqs) > 2r] = 0$. Thus, the definition of (vector) input process stability (\Cref{defn:ips_vec}) and assumption $r \le \frac{1}{2}\rips$ implies that $\Pr_{\coup}[\Qips^c \mid \Qclose ] \le \pips$.
\end{proof}
The remaining step of the proof is therefore to bound the second term in \eqref{eq:Gamimit_decomp}.
\begin{lemma}\label{lem:make_coupling}There exists a coupling $\coup$ which respects the construction and satisfies the following for any $h \in [H]$
\begin{align}
&\Pr_{\coup}[\Ballh^c \mid \cF_{h-1}] \\
&\le \gamhat \circ \disttvc(\srep_h,\shat_h) + (\gamhat + \gamtvcsig) \circ \disttvc(\srep_h,\stel_h)  + \drobvec\,( \pihatsigh(\stel_h) \parallel \pistreph(\stel_h)),~ \text{$\coup$-almost surely }
    %&\pp_{\coup}\left[ \Ballbarh^c \cap \Callbarh \cap \Ballbarh[h-1] \right] \leq \\
   % &\quad\delta + \gamtvc(\epsilon) + (\gamtvc + \gamsig)\circ\gamips(2r) + \Exp_{\stel_h \sim \coup}\drobvec[\epsilon](\pihat \parallel \pistrep \mid \stel_{h}, h)
    %2 \cdot p_r + 2 \cdot \gamtvc\left( \gamips(3r) \right) + \drob[\epsilon]\left( \pistrep, \pihat | \stel_{h+1}, h+1 \right) + \gamtvc(\epsilon).
\end{align}
Consequently, for all $h \in [H]$,
\begin{align}
&\Pr_{\coup}[ \Ballh^c \cap \Callbarh \cap \Ballbarh[h-1]] \\
&\le \gamhat(\epsvec_1) + (\gamhat + \gamtvcsig) \circ \gamipsone(2r) + 
\Exp_{\coup}[\drobvec\,( \pihatsigh(\stel_h) \parallel \pistreph(\stel_h))]
\end{align}
Moreover, $\seqs \mapsto \drobvec\,( \pihatsigh(\seqs) \parallel \pistreph(\seqs))$ is measurable. 
%as well as $\pp_{\coup}[\Callbarh[0]] = 1$.
\end{lemma}

\begin{proof}[Proof Sketch]
We begin by giving a high level overview of the construction, which is done recursively.  The key technical tool is \Cref{lem:couplinggluing} above, which allows us to transform any coupling $\coup$ between random variables $(X, Y)$ into a probability kernel $\coup(\cdot| X)$ mapping instances of $X$ to probability distributions on $Y$ such that $(X, Y) \sim \coup$ has the same law as $(X, Y \sim \coup(\cdot | X))$.  For each $h$, we then show that, assuming the coupling has kept the states and controls close together until time $h-1$, this will imply the following chain:
\begin{align}
    \underbrace{(\arep \leftrightarrow \atel)}_{\gamtvc \text{ and induction}} \to \underbrace{(\atel \leftrightarrow \atelinter)}_{\text{learning and sampling}} \to \underbrace{(\atelinter \leftrightarrow \arepinter)}_{\gamtvc \text{ and induction}} \to \underbrace{(\arepinter \leftrightarrow \seqahat)}_{\gamtvc \text{ and induction}}, \label{eq:mainproofoutline}
\end{align}
where the bidirectional arrows indicate individual couplings between the laws of the random variables that are constructed by the method outlined in text below and the single directional arrows denote the probability kernels described above. The full proof of the lemma is given in \Cref{sec:couplingconstruction}.
\end{proof}

\paragraph{Concluding the proof.}  Here, we finish the proof of \Cref{thm:smooth_cor_general}.  Recall that we wish to bound $\gapjointvec\,(\pihatsig \parallel \pirep) \vee \gapmargvec[\epsvecmarg](\pihatsig \parallel \pist)$. We begin by bounding $\gapjointvec\,(\pihatsig \parallel \pirep) \vee \gapmargvec[\epsvecmarg](\pihatsig \parallel \pistrep)$.  In light of \Cref{lem:putting_couplings_together}, it suffices to bound
 \begin{align}
 \Pr_{\coup}[\Qall^c] + \sum_{h=1}^H\Pr_{\coup}[ \Ballbarh^c \cap \Callbarh \cap \Ballbarh[h-1]], 
 \end{align}
 where $\coup$ is the coupling in \Cref{lem:make_coupling}.
Applying \Cref{lem:Qall_bound} and \Cref{lem:make_coupling},
\begin{align}
&\Pr_{\coup}[\Qall^c] + \sum_{h=1}^H\Pr_{\coup}[ \Ballbarh^c \cap \Callbarh \cap \Ballbarh[h-1]] \\
&\le \pips + 2Hp_r +   \sum_{h=1}^H\Pr_{\coup}[ \Ballbarh^c \cap \Callbarh \cap \Ballbarh[h-1]] \\
&\le   \pips + H(2p_r + \gamhat(\epsvec_1) + (\gamhat + \gamtvcsig) \circ \gamipsone(2r))  + \sum_{h=1}^H\Exp_{\stel_h \sim \coup}\drobvec\,( \pihatsigh(\stel_h) \parallel \pistreph(\stel_h))
\end{align}
To conclude, we note that for any $\coup$ which respects the construction, \Cref{lem:replica_property} ensures that $\stel_h$ as the marginal distribution of $\sstar_h \sim \pisth$. Thus, the above is at most
\begin{align}
\pips + H(2p_r + \gamhat(\epsvec_1) + (\gamhat + \gamtvcsig) \circ \gamipsone(2r))  + \sum_{h=1}^H\Exp_{\sstar_h \sim \Psth}\drobvec\,( \pihatsigh(\sstar_h) \parallel \pistreph(\sstar_h)) \label{eq:first_eq_i_showed}
\end{align}
which concludes the proof of \eqref{eq:smooth_ub_app_one} for $\gapjointvec(\pihat \parallel \pistrep)$. 

To prove \eqref{eq:smooth_ub_app_two} for $\gapjointvec(\pihat \parallel \pistrep)$, we consider the special case that $\pihatsig = \pihat \circ \Wsig$. By definition, $\pihatsigh =\pihat \circ \Wsig$. Thus, the  data-processing inequality for optimal transport (\Cref{cor:opt_trans}) 
\begin{align}\drobvec\,( \pihatsigh(\sstar_h) \parallel \pistreph(\sstar_h))  \le \Exp_{\seqs_h' \sim \Wsig(\sstar_h)}\drobvec\,( \pihat(\seqs_h') \parallel \pidech(\seqs_h')),
\end{align}
for all $\sstar_h$. Substituting this into \eqref{eq:first_eq_i_showed}, and setting $\gamhat = \gamsig$ (in view of \Cref{lem:pistrep_tvc}), finishes the argument.













\subsubsection{Proof of Lemma \ref{lem:make_coupling}}\label{sec:couplingconstruction}

Recall that \Cref{ass:polish_spaces_general} ensures all of the general measure-theoretic guarantees of Appendix \ref{app:prob_theory} hold true in our setting. Notably we need the gluing lemma (\Cref{lem:couplinggluing}) and the commuting of optimal transport metrics and conditional probabilities (\Cref{prop:MK_RCP}).

\paragraph{Proof strategy.} Our proof follows along similar lines as that of \Cref{prop:IS_general_body}, although with the added complication of including the smoothing.  We will inductively construct $\coup$.  A useful schematic for the construction at each step is the following diagram:
\begin{align}
    \underbrace{(\sreptil \leftrightarrow \ssq),(\arep \leftrightarrow \atel)}_{\Btelh} \to \underbrace{(\atel \leftrightarrow \atelinter)}_{\Bfsh} \to \underbrace{(\atelinter \leftrightarrow \arepinter)}_{\Binterh}\to \underbrace{(\arepinter \leftrightarrow \seqahat)}_{\Bhath}, \label{eq:mainproofoutline2}
\end{align}
where the events under each bidirectional arrow refer to the event such ensuring that there exists a coupling such that the objects are close.  We then will apply \Cref{lem:couplinggluing} to glue the individual couplings together.  We will then use \Cref{lem:peeling_lem} and a union bound to control the probability under our constructed coupling that any of the relevant events fail to hold, concluding the proof.

\newcommand{\coupfs}{\coup_{\mathrm{est}}}
\newcommand{\Ebarh}{\bar{\cE}_{h}}

\newcommand{\coupstel}{\coup_{\seqs,\mathrm{tel}}}  
\newcommand{\coupinterr}{\coup_{\mathrm{inter}}}  

\newcommand{\couptel}{\coup_{\mathrm{tel}}}  
\newcommand{\coupahat}{\coup_{\seqahat}}  
\paragraph{Recursive construction of $\coup$.} Let $h \ge 1$, and suppose that we have constructed the coupling $\coup^{(1:h-1)}$ for steps $1,\dots,h-1$ which respects the construction. Recall that $\cF_h$ denotes the sigma-algebra generated by  $(\shat_{1:h},\srep_{1:h},\stel_{1:h})$, $(\arep_{1:h},\sreptil_{1:h},\ssq_{1:h},\atel_{1:h},\seqahat_{1:h})$, and $(\arepinter_{1:h},\atelinter_{1:h})$. Notice that $\stel_{h+1},\srep_{h+1},\shat_{h+1}$ are determined by $\cF_h$ as well. Similarly, it can be seen from \Cref{defn:all_kernels} that $\phiV(\ssq_{h+1})$ and $\phiV(\sreptil_{h+1})$ are also determined by $\cF_{h}$ (since the replica kernel preserves the $\cV$-components). We summarize all these aforementioned variables in a random variable $Y_h$. Let $\cF_0$ denote the filtration generated by $\srep_1 = \stel_1 = \shat_1$. We let $Y_0 = (\srep_1,\stel_1,\shat_1)$. 

Correspondingly, let $Z_h$ denote the random variables $(\arep_{h},\phiZ(\sreptil_{h}),\phiZ(\ssq_{h}),\atel_{h},\seqahat_{h})$, and $(\arepinter_{h},\atelinter_{h})$ such that the joint law of these random variables respects the construction.  Our goal is then to specify, for each $h \in [H]$, a joint distribution of $(Y_{h-1},Z_{h})$.
Note that $Z_h,Y_{h-1}$ determines $Y_{h}$, and we call this induced law $\coup^{(h)}$.






We begin by specifying joint distributions conditional on $Y_{h-1}$ and subsets of $Z_h$, then glue them together by the gluing lemma. Below, we use use information-theoretic notation. 
\begin{itemize}
    \item By total variation continuity of $\phiZ \circ \Qreph$ (\Cref{lem:pistrep_tvc}),
    \begin{align}
    \TV(\pp_{\phiZ(\sreptil_{h}) \mid  Y_{h-1}},\pp_{\phiZ(\ssq_{h}) \mid  Y_{h-1}}) \le \gamtvcsig \circ \disttvc(\srep_h,\stel_h). 
    \end{align}
    Because $\arep_{h} \sim \pisth(\sreptil_{h+1})$ and $\atel_{h} \sim \pisth(\ssq_{h})$,  and $\pist$ is compatible with the decomposition $\cS = \cZ \oplus \cV$ (i.e. $\pisth(\seqs)$ is a function of $\phiZ(\seqs)$)
    \Cref{cor:tv_two} implies that (almost surely)
    \begin{align}
    \TV(\pp_{(\arep_h,\phiZ(\sreptil_{h}) \mid Y_{h-1}},\pp_{(\atel_h,\phiZ(\ssq_{h}) \mid  Y_{h-1}}) \le \gamtvcsig \circ \disttvc(\srep_h,\stel_h). 
    \end{align}
    Hence, \Cref{cor:first_TV} implies that there exists a coupling $\couptel^{(h)}$ over $Y_{h-1},(\phiZ(\sreptil_{h}),\arep_h),(\phiZ(\ssq_{h}),\atel_{h})$ respecting the construction such that $Y_h \sim \coup^{(h-1)}$ and such that (almost surely)
    \begin{align}\label{eq:couptel}
    \Exp_{\couptel^{(h)}}[\Btelh \mid Y_{h-1}] = \Pr_{\couptel^{(h)}}[(\phiZ(\sreptil_{h}),\arep_h) \ne (\phiZ(\ssq_{h}),\atel_{h})\mid Y_{h-1}] &\le \disttvc(\srep_h,\stel_h)].
    \end{align}
    \item In our construction, $\atel_h \mid Y_{h-1} \sim \pistreph(\stel_h)$, and $\atelinter_h \mid Y_{h-1} \sim \pihatsigh(\stel_h)$. 
    Thus, by definition of $\drobvec$, and the assumption $\I\{\distavec(\cdot,\cdot) \not\preceq \epsvec\}$ is lower semicontinuous, \Cref{prop:MK_RCP} implies that we may find a coupling $\coupfs^{(h)}$ of $(\atel_{h},\atelinter_{h},Y_{h-1})$ respecting the construction such that, almost surely,
    \begin{align}\label{eq:coupfs}
    \pp_{\coupfs^{(h)}}\left[\Bfsh^c \mid Y_{h-1} \right] &=  \pp_{\coupfs^{(h)}}\left[ \distavec(\atelinter_{h},\atel_{h}) \not \preceq \epsvec \mid Y_{h-1} \right] \\
    &= \drobvec\,( \pihatsigh(\stel_h) \parallel \pistreph(\stel_h))].
\end{align}
Moreover, that same proposition ensures measurability of $\seqs \to \drobvec\,( \pihatsigh(\seqs) \parallel \pistreph(\seqs))$.
\item Since $\atelinter_{h} \mid \cF_h \sim \pihatsigh(\stel_h)$ and $\arepinter_{h+1} \mid \cF_h \sim \pihatsigh(\srep_h)$, and since  $\pihatsigh(\cdot)$ is $\gamhat$-TVC by assumption,  
\begin{align}
\TV(\pp_{\atelinter_{h} \mid  Y_{h-1}},\pp_{ \arepinter_{h} \mid  Y_{h-1}}) \le \gamhat \circ \disttvc(\srep_h,\stel_h). 
\end{align}

\Cref{cor:first_TV}  implies that there is a coupling $\coupinterr^{(h)}$ between $(\atelinter_{h},\arepinter_{h},Y_{h-1})$ such that
\begin{align}\label{eq:coupinterr}
\pp_{\coupinterr^{(h)}}[\Binterh^c \mid Y_{h-1}] = \pp_{\coupinterr^{(h)}}\left[\atelinter_{h} \ne \arepinter_{h}  \mid Y_{h-1}\right] &\le  \gamhat \circ  \disttvc(\stel_h,\srep_h)
\end{align}
\item  Similarly, since $\arepinter_{h} \mid \cF_{h-1} \sim \pihat_h(\srep_h)$ and $\seqahat_{h+1} \mid \cF_{h-1}\sim \pihat_h(\shat_h)$, $\pihat_h(\cdot)$ is $\gamhat$-TVC,  \Cref{cor:first_TV} implies that there is a coupling $\coupahat^{(h)}$ between $(\arepinter_{h},\seqahat_{h},Y_{h-1})$ such that
\begin{align}\label{eq:coupahat}
\pp_{\coupahat^{(h)}}[\Bhath^c \mid Y_{h-1}] = \pp_{\coupahat^{(h)}}\left[\seqahat_{h} \ne \arepinter_{h} \mid Y_{h-1}  \right] \le \gamhat \circ \disttvc(\srep_h,\shat_h)
\end{align}
\end{itemize}

We can then apply the gluing lemma (\Cref{lem:couplinggluing}) to 
\begin{align}
X_{h,1} &= (\phiZ(\ssq_h),\atel_h,Y_{h-1}) \\ 
X_{h,2} &= (\phiZ(\sreptil_h),\arep_h,Y_{h-1}) \\
 X_{h,3} &= (\atel_h,\atelinter_h,Y_{h-1}) \\
  X_{h,4} &= (\atelinter_h,\arepinter_h,Y_{h-1}) \\
   X_{h,5} &= (\arepinter_h,\ahat_h,Y_{h-1})  
\end{align}
with 
\begin{align}
(X_{h,1},X_{h,2}) \sim \couptel^{(h)},\quad (X_{h,2},X_{h,3}) \sim  \coupfs^{(h)}, \quad (X_{h,3},X_{h,4})\sim \coupinterr^{(h)}, \quad (X_{h,4},X_{h,5})\sim\coupahat^{(h)}.
\end{align}
\Cref{lem:couplinggluing} guarantees the existence of a coupling $\mu^{(h)}$ consident with all sub-couplings $\couptel^{(h)}$, $\coupfs^{(h)},\coupinter^{(h)},\coupahat^{(h)}$. Then, $\coup^{(h)}$-almost surely (and using that $\cF_{h-1}$ is precisely the $\upsigma$-algebra generated by $Y_{h-1}$)
\begin{align}
&\Pr_{\coup^{(h)}}[\Ballh^c \mid \cF_{h-1}] \\
&\le \Pr_{\coup^{(h)}}[\Btelh^c \mid \cF_{h-1}] + \Pr_{\coup^{(h)}}[\Bfsh^c \cF_{h-1}] +  \Pr_{\coup^{(h)}}[\Binterh^c \cF_{h-1}]+\Pr_{\coup^{(h)}}[\Bhath^c \cF_{h-1}]\\
&\le \gamhat \circ \disttvc(\srep_h,\shat_h) + (\gamhat + \gamtvcsig) \circ \disttvc(\srep_h,\stel_h)  + \drobvec\,( \pihatsigh(\stel_h) \parallel \pistreph(\stel_h))\\
&= \gamhat \circ \disttvc(\srep_h,\shat_h) + (\gamhat + \gamtvcsig) \circ \disttvc(\srep_h,\stel_h)  + \drobvec\,( \pihatsigh(\stel_h) \parallel \pistreph(\stel_h))
\end{align}
This concludes the inductive construction.


For the second statement, notice that the events $\Callbarh \cap \Ballbarh[h-1]$ are $\cF_h$ measurable (thus determined by $\coup^{(h-1)}$) and, when they hold, $\distsvec(\srep_h,\stel_h) \preceq \gamipsvec(2r)$ and $\dists(\srep_h,\shat_h)  \preceq \epsvec$. For our purposes, we use $\disttvc = \distsi[1](\srep_h,\stel_h) \preceq \gamipsone(2r)$ and $\dists(\srep_h,\shat_h)  \preceq \epsvec_1$. Hence, 
\begin{align}
\max_{h\in [H]}\Pr_{\coup}[ \Ballh^c \cap \Callbarh \cap \Ballbarh[h-1]] &\le \gamhat(\epsvec_1) + (\gamhat + \gamtvcsig) \circ \gamipsone(2r) \\
&\quad+ \drobvec\,( \pihatsigh(\stel_h) \parallel \pistreph(\stel_h)).
\end{align}
The result follows.


\subsection{Proof of Theorem \ref{thm:smooth_cor}, and generalization to direct decompositions}\label{app:smoothcor_proof}

In this subsection, we consider the special case dealt with in \Cref{thm:smooth_cor}.  Note that there always exists a trivial direct decomposition that is compatible with all policies and dynamics simply by letting $\cV = \emptyset$ and $\cS = \cZ$.  We prove here the version of the result that involves a possibly nontrivial direct decomposition, as we will instantiate this in our control setting by letting $\cZ = \left\{ \pathm[h] \right\}$ and $\cS = \left\{ \pathc[h] \right\}$, i.e., projecting $\pathc[h]$ onto the last $\taum$ coordinates gives $\seqz_h$. We further consider a restriction of IPS to consider kernels absolutely continuous with respect to $\Psth$ in their $\cZ$ component. 
\begin{definition}[Restricted IPS]\label{defn:ips_restricted}
For a non-decreasing maps $\gamipsone,\gamipstwo:\R_{\ge 0} \to  \R_{\ge 0}$ a  pseudometric $\distips:\cS \times \cS \to \R$ (possibly other than $\dists$ or $\disttvc$), and $\rips > 0$, we say a policy $\pi$ is \emph{$(\gamipsone,\gamipstwo,\distips,\rips)$-restricted IPS} if the following holds for any $r \in [0,\rips]$. Consider any sequence of kernels $\lawW_1,\dots,\lawW_H:\cS \to \laws(\cS)$ satisfying 
\begin{align}
\max_{h,\seqs \in \cS}\Pr_{\tilde \seqs\sim \lawW_h(\seqs)}[\distips(\tilde \seqs,\seqs) \le r] = 1, \quad \forall s, \quad \phiZ \circ \lawW_h(\seqs_h) \ll \phiZ \circ \Psth. 
\end{align}
 and define a process $\seqs_1 \sim \Dinit$, $\tilde\seqs_h \sim \lawW_h(\seqs_h),\seqa_h \sim \pi_h(\tilde \seqs_h)$, and $\seqs_{h+1} := F_h(\seqs_h,\seqa_h)$. Then, almost surely, (a) the sequence $(\seqs_{1:H+1},\seqa_{1:H})$ is input-stable w.r.t $(\dists,\dista)$ (b) $\max_{h \in [H]} \disttvc(F_h(\tilde\seqs_h,\seqa_h),\seqs_{h+1}) \le \gamipsone(r)$ and (c) $\max_{h \in [H]} \dists(F_h(\tilde\seqs_h,\seqa_h),\seqs_{h+1}) \le \gamipstwo(r)$.
\end{definition}

Note that the above is a  slightly weaker condition than the one in \Cref{defn:ips_body} in the main text and consequently, the following theorem which uses it as an assumption implies \Cref{thm:smooth_cor} in the body.
\begin{theorem}\label{thm:smooth_cor_decomp} Suppose $\cS = \cZ \oplus \cV$ as in \Cref{defn:direct_decomp} and projections $\phiZ,\phiV$, which is compatible with the dynamics and with given policies $\pihat,\pist$, smoothing kernel $\Wsig$, and pseudometric $\distips$.
Suppose $\pist$ satisfies $(\gamipsone,\gamipstwo,\distips,\rips)$-restricted IPS (\Cref{defn:ips_restricted}) and $\phiZ \circ \Wsig$ is $\gamma_{\sigma}$-TVC. Let $\epsilon > 0$, $r \in (0,\frac{1}{2}\rips]$; define $p_r := \sup_{\seqs}\Pr_{\seqs' \sim \Wsig(\seqs)}[\distips(\seqs',\seqs) >  r]$ and $\epsilon' := \epsilon+\gamipstwo(2r)$. Then, for any policy $\pihat$,  both  $\gapjoint (\pihat \circ \Wsig \parallel \pistrep)$ and  $\gapmarg[\epsilon'] (\pihat \circ \Wsig \parallel \pist)$ are upper bounded by
\begin{align}
%\inf_{r > 0}  
H\left(2p_r +  3\gamma_{\sigma}(\max\{\epsilon,\gamipsone(2r)\})\right)  + \textstyle \sum_{h=1}^H\Exp_{\sstar_h \sim \Psth}\Exp_{\sstartil_h \sim \Wsig(\sstar_h) } \drob( \pihat_{h}(\sstartil_h) \parallel \pidec(\sstartil_h))  . \label{eq:smooth_ub}
\end{align}
\end{theorem}


Consider the special case $K = 2$ with $\distsi[1] = \disttvc$, $\distsi[2] = \dists$, $\distai[1] = \distai[2] = \dista$ and $\epsvec = (\epsilon, \epsilon)$.  In this case, applying \eqref{eq:smooth_ub_app_two}, we see that
\begin{align}
    &\gapjointvec(\pihatsig \parallel \pistrep) \vee \gapmargvec[\epsvecmarg](\pihatsig \parallel \pistrep) \\
    &\leq \pips + H\left(2p_r +  3\gamma_{\sigma}(\max\{\epsilon,\gamipsone(2r)\}\right)  + \textstyle \sum_{h=1}^H\Exp_{\sstar_h \sim \Psth}\Exp_{\sstartil_h \sim \Wsig(\sstar_h) } \drobvec( \pihat_{h}(\sstartil_h) \parallel \pidec(\sstartil_h))
\end{align}
We now observe that under this convention,
\begin{align}
    \gapjoint(\pihatsig \parallel \pistrep) &= \inf_{\coup_1} \pp_{\coup_1}[\max_{h \in [H]} \dists(\shat_{h+1}, \sstar_{h+1}) \vee \dista(\ahat_h, \astar_h) > \epsilon] \\
    &\leq \inf_{\coup_1} \pp_{\coup_1}\left[ \max_{h \in [H]} \left( \disttvc(\shat_{h+1}, \sstar_{h+1}), \dists(\shat_{h+1}, \sstar_{h+1}) \right) \vee \left( \dista(\ahat_h, \astar_h), \dista(\ahat_h, \astar_h) \right) \not \preceq \epsvec\right] \\
    &= \gapjointvec(\pihatsig \parallel \pistrep)
\end{align}
and similarly $\gapmarg[\epsilon'](\pihatsig \parallel \pist) \leq \gapmargvec[\epsvec + \gamips(2r)](\pihatsig \parallel \pist)$.  From the construction of $\distavec$, however, we see that $\left\{ \distavec(\seqa, \seqa') \not \preceq \epsvec \right\} = \left\{ \dista(\seqa, \seqa') > \epsilon \right\}$ for all $\seqa, \seqa'$ and thus for all $h \in [H]$,
\begin{align}
    \drobvec(\pihat_{h}(\sstartil_h) \parallel \pist_h(\sstartil_h)) &= \inf_{\coup_2} \pp_{\coup_2}\left[ \distavec(\seqahat_h, \seqast_h) \not \preceq \epsvec \right] \\
    &= \inf_{\coup_2} \pp_{\coup_2}\left[ \dista(\seqahat_h, \seqast_h) \geq \epsilon \right] \\
    &= \drob(\pihat_h(\sstartil_h) \parallel \pist_h(\sstartil_h)).
\end{align}
Plugging in to \eqref{eq:smooth_ub_app_two} concludes the proof.









%  
%  \section{Extended Experimental Details}
%  \label{sec:extexp}
%  


With the refined error bound derived in Section~\ref{sec:key-lemmas-tec}, 
we can readily obtain more refined regret bounds for Algorithm~\ref{alg:main} to reflect the role of several problem-dependent quantities.  Most of the arguments in the analysis are similar to those in the previous work \citet{zhou2023sharp}. 
Detailed proofs are postponed to Appendix~\ref{sec:appfirst} and Appendix~\ref{app:var}.
% Our results show that 




%\subsection{Value-based and cost-based regret bounds}\label{sec:value-cost}



%We restate Theorem~\ref{thm:first} as below.
%
%
%\noindent\textbf{Theorem 2. (Restated)}\emph{
%For any $K \ge 1$, with probability $1-\delta$, the regret of Algorithm~\ref{alg:main} is 
%$$\widetilde{O}\left(\min\{\sqrt{SAH^2K v^*}+SAH^2,Kv^* \}\right)$$ where $v^*$ is the value of the optimal policy, i.e., $v^* =\max_{\pi} \mathbb{E}_{\pi,s_1\sim \mu}\left[\sum_{h=1}^H r_h\right]$.}

% The proof of Theorem~\ref{thm:first} is similar to that of Theorem~\ref{thm1}, except for that we provide refined analysis for some of the regret terms. See  Appendix~\ref{sec:appfirst} for more details.


%
\paragraph{Value-based regret bounds.} 
%
Thus far, we have not yet introduced the crucial quantity $v^{\star}$ in Theorem~\ref{thm:first}, 
which we define now. 
%
When the initial states are drawn from $\mu$, $v^{star}$ stands for the weighted optimal value: 
%
\begin{equation}
	v^{*} \coloneqq \mathbb{E}_{s\sim \mu}\big[ V_1^*(s) \big]. 
	\label{eq:defn-vstar-formal}
\end{equation}
%
Encouragingly, 
the value-dependent regret bound in Theorem~\ref{thm:first} is still minimax-optimal, 
as asserted by the following lower bound. 
%
\begin{theorem}\label{thm:lb1} 
Consider any $p\in [0,1]$ and $K\geq 1$. 
For any learning algorithm, there exists an MDP with $S$ states, $A$ actions and horizon $H$
	obeying $v^*\leq  Hp$ and 
	%
	\begin{equation}
		\mathbb{E}\big[\mathsf{Regret}(K)\big]  \gtrsim \min\big\{ \sqrt{SAH^3Kp},\, KHp \big\} .
	\end{equation}
	%
\end{theorem}
%
In fact, the construction of the hard instance (as required in Theorem~\ref{thm:lb1}) is quite simple. 
Design a new branch with $0$ reward and set the probability of reaching this branch to be $1-p$. 
Also, with probability $p$, we direct the learner to a hard instance with regret $\Omega(\min\{\sqrt{SAH^3Kp},KpH\})$ and optimal value $H$. This guarantees that the optimal value $v^* \leq Hp$ and that the expected regret is at least $\Omega(\min\{ \sqrt{SAH^3Kp},KHp   \}) \gtrsim \min\{ \sqrt{SAH^2Kv^*},Kv^*   \}$.  
See Appendix~\ref{app:lb} for more details.


%Recall that in Section~\ref{sec:pre}, we assume the reward satisfies Assumption~\ref{assum1}, which means $R_{h,s,a}\in \Delta ([0,H])$ and $\sum_{h=1}^H r_h \leq H$. 
%


%
\paragraph{Cost-based regret bounds.} 
%
Next, we turn to the cost-aware regret bound as in Corollary~\ref{thm:cost}. 
Note that all other results except for Corollary~\ref{thm:cost} are about rewards as opposed to cost. 
In order to facilitate discussion, let us first formally introduce the cost-based scenarios. 



Suppose that the reward distributions $\{R_{h,s,a}\}_{(s,a,h)}$ are replaced with the cost distributions $\{C_{h,s,a}\}_{(s,a,h)}$, 
where each distribution $C_{h,s,a}\in \Delta([0,H])$ has mean $c_h(s,a)$. 
In the $h$-th step of an episode, the learner pays an immediate cost $c_h\sim C_{h,s_h,a_h}$ instead of receiving an immediate reward $r_h$,  
and the objective of the learner is instead to minimize the total cost $\sum_{h=1}^H c_h$ (in an expected sense). 
The optimal cost quantity $c^*$ is then defined as 
%
\begin{equation}
	c^* \coloneqq \min_{\pi}\mathbb{E}_{\pi,s_1\sim \mu}\bigg[\sum_{h=1}^H c_h \bigg]. 
\label{eq:defn-cstar-formal}
\end{equation}
%
Similarly, we can re-define the $Q$-function and value function as follows:
%
\begin{align}
	 \mathtt{Q}_{h}^{\pi}(s,a) &\coloneqq \mathbb{E}_{\pi}\left[\sum_{h'=h}^H c_{h'} \,\Big|\, (s_h,a_h)=(s,a)\right] ,
	 && \forall (s,a,h)\in \mathcal{S}\times \mathcal{A}\times [H], \nonumber
	\\ 
	\mathtt{V}_h^{\pi}(s) &\coloneqq \mathbb{E}_{\pi}\left[\sum_{h'=h}^H c_{h'} \,\Big|\, s_h = s\right],
	&& \forall (s,h)\in \mathcal{S}\times \times [H], 
	\nonumber
\end{align}
%
where we use different fonts to differentiate them from the original Q-function and value function. 
The optimal cost function is then given by $\mathtt{Q}_h^*(s,a) = \min_{\pi}\mathtt{Q}_h^{\pi}(s,a)$ and $\mathtt{V}_h^*(s)=  \min_{\pi}\mathtt{V}_h^{\pi}(s)$. 
Given the definitions above, 
we overload the notation $\mathsf{Regret}(K)$ to denote the regret for the cost-based scenario as 
$$
	\mathsf{Regret}(K) \coloneqq \sum_{k=1}^K \Big( \mathtt{V}^{\pi^k}_{1}(s_1^k) -  \mathtt{V}_1^*(s_1^k) \Big).
$$
%
One can also simply regard the cost minimization problem as  reward maximization with negative rewards by choosing $r_h = -c_h$. 
This way allows us to apply Algorithm~\ref{alg:main} directly, except that \eqref{eq:updateq} is replaced by 
%
\begin{align}
Q_h(s,a) \,\leftarrow\, \max\left\{\min \left\{ \widehat{r}_h(s,a) + \widehat{P}_{s,a,h}V_{h+1}+b_h(s,a), \, 0 \right\} ,\,-H \right\}.
	\label{eq:updatecost}
\end{align}
%
%
%We restate Theorem~\ref{thm:cost} as below.
%
%\noindent \textbf{Theorem 3. (Restated)}
%\emph{
%For any $K \ge 1$, with probability $1-\delta$, the regret of Algorithm~\ref{alg:main} is} 
%$$\widetilde{O}\left(\min\{\sqrt{SAH^2K c^*}+SAH^2,K(H-c^*) \}\right).$$
%
%
Note that the proof of Corollary~\ref{thm:cost} closely resembles that of Theorem~\ref{thm:first}, 
which can be found in Appendix~\ref{app:cost}.
%
%, except that $\mathbb{E}[\sum_{k=1}^K\sum_{h=1}^H r_h^k] \leq Kv^*$, while $\mathbb{E}[\sum_{k=1}^K\sum_{h=1}^H c_h^k] \geq Kc^*$. Refer to Appendix~\ref{app:cost} for more details.



To confirm the tightness of  Corollary~\ref{thm:cost}, we develop the following matching lower bound, 
which basically employs the same hard instance as in the proof of Theorem~\ref{thm:lb1}. 
%
\begin{corollary}\label{corollary:costlb}
Consider any $p\in [0,\frac{1}{4}]$ and any $K\geq 1$.
For any algorithm, one can construct an MDP with $S$ states, $A$ actions and horizon $H$ 
 obeying $c^*= \Theta(Hp)$ and 
	%
	\[
		\mathbb{E}\big[\mathsf{Regret}(K)\big] \gtrsim \min\big\{ \sqrt{SAH^3Kp}+SAH^2, \,KH(1-p) \big\} .
	\]
	%
\end{corollary}





\paragraph{Variance-dependent regret bound.}
%\input{variance}

The final regret bound presented in Theorem~\ref{thm:var} depends on a sort of variance metrics. 
Towards this end, let us first make precise the variance metrics of interest:
%
\begin{itemize}
	\item[(i)] The first variance metric is defined as 
		%
		\begin{equation}
			\mathrm{var}_1 \coloneqq \max_{\pi}\mathbb{E}_{\pi}\Bigg[\sum_{h=1}^H \mathbb{V}\big(P_{s_h,a_h,h},V_{h+1}^*\big)+\sum_{h=1}^H \mathrm{Var}\big(R_h(s_h,a_h)\big) \Bigg],
			\label{eq:defn-var1}
		\end{equation}
		%
		where $\{(s_h,a_h)\}_{1\leq h\leq H}$ represents a sample trajectory under policy $\pi$. 
		This captures the maximal possible expected sum of variance with respect to the optimal value function $\{V_{h}^*\}_{h=1}^H$. 
		
	\item[(ii)] Another useful variance metric is defined as
		%
		\begin{equation}
			\mathrm{var}_2 \coloneqq \max_{\pi,s}\mathrm{Var}_{\pi}\bigg[\sum_{h=1}^H r_h \,\Big|\, s_1=s\bigg],
			\label{eq:defn-var2}
		\end{equation}
		%
		where $\{r_h\}_{1\leq h\leq H}$ denotes a sample sequence of immediate rewards under policy $\pi$. 
		This indicates the maximal possible variance of the accumulative reward. 
%
\end{itemize}
%
The interested reader is referred  to \citet{zhou2023sharp} for further discussion about these two metrics. 
Our final variance metric is then defined as
%
\begin{align}
	\mathrm{var} \coloneqq \min\big\{\mathrm{var}_1,\mathrm{var}_2 \big\} .
	\label{eq:defn-var-formal}
\end{align}
%


 %and restate Theorem~\ref{thm:var} as below.

%\noindent\textbf{Theorem 4. (Restated)}\emph{
% With probability $1-\delta$, the regret of Algorithm~\ref{alg:main} is at most $\tilde{O}(\min\{\sqrt{SAHK\mathrm{var}}+SAH^2,HK\})$.}


With the above metric $\mathrm{var}$ in mind, we can then revisit Theorem~\ref{thm:var}. 
When the transition model is fully deterministic, the regret bound in Theorem~\ref{thm:var} simplifies to $$\mathsf{Regret}(K)\leq \widetilde{O}\big(\min\big\{SAH^2,\,HK\big\}\big)$$ for any $K\geq 1$, which is roughly the cost of visiting each state-action pair. 
%
The full proof of Theorem~\ref{thm:var} is postponed to Appenndix~\ref{app:var}. 
%depend on tighter bounds for $T_2$, $T_4$, $T_5$ and $T_6$. Refer to Appenndix~\ref{app:var} for more details. 



To finish up, let us develop a matching lower bound to corroborate the tightness and optimality of Theorem~\ref{thm:var}. 
%
\begin{theorem}\label{thm:lb3}
Consider any $p\in [0,1]$ and any $K\geq 1$. For any algorithm, one can find an MDP with $S$ states, $A$ actions, and horizon $H$ satisfying $\max\{\frac{\mathrm{var}_1}{H^2},\frac{\mathrm{var}_2}{H^2}\}\leq p$ and 
	$$	
		\mathbb{E}\big[\mathsf{Regret}(K)\big]  \gtrsim \min\big\{\sqrt{SAH^3Kp}+SAH^2,\,KH\big\}.
	$$
\end{theorem}

The proof of Theorem~\ref{thm:lb3} resembles that of Theorem~\ref{thm:lb1}, except that we need to construct a hard instance when $K\leq SAH/p$. For this purpose, we construct a fully deterministic MDP  (i.e., all of its transitions are deterministic and all rewards are fixed), and show that the learner has to visit about half of the state-action-layer tuples in order to learn a near-optimal policy. 
The proof details are deferred to Appendix~\ref{app:lb}.








\end{document}