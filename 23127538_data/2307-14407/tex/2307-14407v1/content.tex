\title{Local multiplicity fluctuations in Pb$-$Pb collisions at $\sqrt{s_{\rm{NN}}}$ = 2.76 TeV with ALICE at the LHC}
\author[$ $]{Sheetal Sharma$^{*}$ and Ramni Gupta$^{\dagger}$ {(for the ALICE Collaboration)}}

\affil[$ *\dagger$]{Department of Physics, University of Jammu, Jammu, India}
%\affil[$*$]{\href{mailto:sheetal.sharma@cern.ch}{\texttt{sheetal.sharma@cern.ch}}, \href{mailto:ramni.gupta@cern.ch}{\texttt{ramni.gupta@cern.ch}}}
\affil[$ $]{$^{*}$sheetal.sharma@cern.ch}

\onehalfspacing
\maketitle

\date{}

%%%%%% Abstract %%%%%%
\begin{abstract}
Local multiplicity fluctuations are an useful tool to understand the dynamics of the particle production and the phase-space changes from quarks to hadrons in ultrarelativistic heavy-ion collisions. The study of scaling behavior of multiplicity fluctuations in geometrical configurations in multiparticle production can be performed using the factorial moments and recognized in terms of a phenomenon referred to as intermittency. 

%In this contribution, we present the analysis of the factorial moment performed on the multiplicity distributions of charged particles produced in Pb$-$Pb collisions at $\sqrt{s_{\rm{NN}}}$ = 2.76 TeV, recorded with the ALICE detector at the LHC. 
In this contribution, the analysis of the factorial moment is presented for the multiplicity distributions of charged particles produced in Pb$-$Pb collisions at $\sqrt{s_{\rm{NN}}}$ = 2.76 TeV, recorded with the ALICE detector at the LHC. The normalized factorial moments (NFM), $F_{q}$ of the spatial configurations of charged particles in two-dimensional angular ($\eta,\varphi$) phase space are calculated. For a system with dynamic fluctuations due to the characteristic critical behavior near the phase transition, $F_{q}$ exhibits power-law growth with increasing bin number or decreasing bin size which indicates self-similar fluctuations. Relating the $q^{\rm{th}}$ order NFM ($F_{q}$) to the second-order NFM ($F_{2}$), the value of the scaling exponent ($\nu$) is extracted,  which indicates the order of the phase transition within the framework of Ginzburg-Landau theory. The dependence of scaling exponent on the $p_{\rm{T}}$ bin width will be presented. The measurements are also compared with the corresponding results from the AMPT model and a Toy Monte Carlo (MC) simulation.

%This study analyzes the multiplicity distributions of charged particles produced in Pb--Pb collisions at $\sqrt{s_{\rm{NN}}}$ = 2.76 TeV as recorded by ALICE detector at the LHC. Normalized factorial moments (NFM) are determined and their scaling behaviour is studied. Scaling exponent is obtained by relating the $q^{\mathrm{th}}$ order NFM to the second-order NFM and its dependence on $p_{\rm{T}}$ bins and $p_{\rm{T}}$ bin width is presented.

%\lipsum[5] 
\end{abstract}

\newpage
\section{Introduction and Analysis Details}
Heavy-ion collisions offer a unique opportunity to investigate the creation and properties of quark--gluon plasma (QGP). As the QGP rapidly cools down, it undergoes a phase transition into hadronic matter that provides crucial information about the properties of the system. Spatial configurations of the charged particles produced in two dimensional ($\eta, \varphi$) phase space are proposed to be studied to understand the multiparticle production at LHC \cite{Hwa:2011bu}. The scaling behavior of the NFM of the multiplicity distributions is studied in the framework of the intermittency methodology \cite{Bialas:1985jb,Hwa:1992uq, Hwa:1992cn, Hwa:2011bu}. 


Event-by-event intermittency analysis is performed for charged particles produced in the midrapidity region and full azimuth in Pb--Pb collisions at $\sqrt{s_{\rm{NN}}}$ = 2.76 TeV recorded by the ALICE detector. The $F_{\rm{q}}$ coefficients are determined using the methodology described in Ref.~\cite{Gupta:2019zox} using Eq.~1 from Ref.~\cite{Sharma:2021cyp} 
%\begin{equation}
%F_{\rm{q}}(M)= \frac{\frac{1}{N} \displaystyle\sum_{e=1}^{N}\frac{1}{M}\sum_{i=1}^{M}f_{q}(n_{\rm{ie}})}{\left (\frac{1}{N} \displaystyle \sum_{e=1}^{N}\frac{1}{M}\sum_{i=1}^{M}f_{1}(n_{\rm{ie}}) \right )^q}
%\label{def}
%\end{equation}
for different number of bins M (partitioning in ($\eta, \varphi$) phase space). For the spatial fluctuations in the data that are scale independent, $F_{\rm{q}}(M)$ is proportional to $M^{\phi_{\rm{q}}}$ for \textit{q} $\geq 2$ and for $\phi_{\rm{q}} > 0$, is the intermittency index. This power law dependence is defined as intermittency. Independent of the observation of this behavior, a scaling exponent ($\nu$) is obtained from the dependence of $F_{\rm{q}}(M)$ (with \textit{q} $\geq 3$) on $F_{\rm{2}}(M)$. This scaling exponent is 1.304 in the framework of Ginzburg-Landau (GL) theory for the second order phase transition \cite{Hwa:1992cn}.

% Figure environment removed

\section{Results and Conclusion}
%Fig.\ref{mscalnnu}(left) shows the NFM for \texttt{q} = 2, 3, 4, and 5 as a function of the number of bins (M) for ALICE data and Toy MC. For all $q$, NFM shows a linear dependence on M in the case of ALICE data, which indicates the presence of scale-invariant spatial fluctuations. A good linear dependence of $F_{\rm{q}}$ on $F_{\rm{2}}$ is also observed and linear fit is performed in the region of higher M to obtain $\beta_\textrm{q}$. Dimensionless scaling exponent ($\nu$) is then the slope of $ln \beta_\textrm{q}$ vs ln (q-1). The plot of $\nu$ vs $p_{\rm{T}}$ bin width is studied as shown in Fig.\ref{mscalnnu}(right). The scaling exponent shows negligible dependence on $p_{\rm{T}}$ bin width within uncertainties.
For ALICE data, intermittency analysis is performed for the charged particles produced in $|\eta| \leq 0.8$ and full azimuth in different $p_{\rm{T}}$ bins. A Toy MC simulation with only statistical fluctuations is performed as baseline. The $M^{\rm{2}}$ dependence of $F_{\rm{q}}$ ($q$ = 2, 3, 4, 5) shows a significant deviation from baseline as presented in the left panel of Fig. \ref{mscalnnu}. 
The scaling exponent for different $p_{\rm{T}}$ bin widths is presented in the right panel of Fig. \ref{mscalnnu}. The data is also compared with AMPT model calculations, as well as GL theory predictions.
%In conclusion, intermittency analysis is performed for minimum bias events in Pb--Pb collisions at $\sqrt{s_{\rm{NN}}}$ = 2.76 TeV. 

 In summary, an intermittency signal, i.e., a linear behavior between ln~$F_{\rm{q}}$ and ln~$M^{\rm{2}}$ is observed at larger M values, indicating a scale-invariant pattern in the distribution of the particles. The scaling exponent shows no dependence on the $p_{\rm{T}}$ bin width within the uncertainties and is consistent with models that include critical fluctuations within the experimental uncertainties.


\begin{thebibliography}{50}
%\cite{Hwa:2011bu}
\bibitem{Hwa:2011bu}
R.~C.~Hwa and C.~B.~Yang,
%``Local Multiplicity Fluctuations as a Signature of Critical Hadronization at LHC,''
Phys. Rev. C \textbf{85} (2012), 044914
doi:10.1103/PhysRevC.85.044914
[arXiv:1111.6651 [nucl-th]].
%28 citations counted in INSPIRE as of 27 Jun 2023

%\cite{Bialas:1985jb}
\bibitem{Bialas:1985jb}
A.~Bialas and R.~B.~Peschanski,
%``Moments of Rapidity Distributions as a Measure of Short Range Fluctuations in High-Energy Collisions,''
Nucl. Phys. B \textbf{273} (1986), 703-718
doi:10.1016/0550-3213(86)90386-X
%1098 citations counted in INSPIRE as of 15 May 2023

%\cite{Hwa:1992uq}
\bibitem{Hwa:1992uq}
R.~C.~Hwa and M.~T.~Nazirov,
%``Intermittency in second order phase transition,''
Phys. Rev. Lett. \textbf{69} (1992), 741-744
doi:10.1103/PhysRevLett.69.741
%106 citations counted in INSPIRE as of 15 Mar 2023

%\cite{Hwa:1992cn}
\bibitem{Hwa:1992cn}
R.~C.~Hwa and J.~c.~Pan,
%``Intermittency in the Ginzburg-Landau theory,''
Phys. Lett. B \textbf{297} (1992), 35-38
doi:10.1016/0370-2693(92)91065-H
%24 citations counted in INSPIRE as of 15 Mar 2023

%\cite{Gupta:2019zox}
\bibitem{Gupta:2019zox}
R.~Gupta and S.~K.~Malik,
%``Intermittency study of charged particles generated in Pb-Pb collisions at $\sqrt{s_{\mathrm{NN}}}\text{= 2.76 TeV}$ using EPOS3,''
Adv. High Energy Phys. \textbf{2020} (2020), 5073042
[erratum: Adv. High Energy Phys. \textbf{2020} (2020), 7319894]
doi:10.1155/2020/5073042
[arXiv:1911.13111 [hep-ex]].
%4 citations counted in INSPIRE as of 20 Jun 2023

%\cite{Sharma:2021cyp}
\bibitem{Sharma:2021cyp}
S.~Sharma and R.~Gupta,
%``Intermittency Analysis of Toy Monte Carlo Events,''
SciPost Phys. Proc. \textbf{10} (2022), 024
doi:10.21468/SciPostPhysProc.10.024
[arXiv:2110.11901 [hep-ph]].
%0 citations counted in INSPIRE as of 15 Mar 2023
\end{thebibliography}


