\documentclass[10pt, English, table]{article}
\usepackage[a4paper,showframe = false, bindingoffset=0in, left=1in,right=1in,top=1in,bottom=1in, footskip=1 cm]{geometry}
\usepackage[utf8]{inputenc}
\usepackage{calc}
\usepackage{makecell}
\usepackage{textcomp}
\usepackage{gensymb}
\usepackage{enumitem}
\usepackage{caption}
\usepackage{subcaption}
\usepackage{boldline}
\usepackage{fancyhdr}
\usepackage{array}
\usepackage{etoolbox}
\usepackage{textcomp}
\usepackage{authblk}
\usepackage{booktabs}
\usepackage{boldline,multirow, hhline}
\usepackage{multicol}
\usepackage{lscape}
\usepackage[table, dvipsnames]{xcolor}
\usepackage[symbol]{footmisc}
\usepackage[numbers,sort&compress,super, square]{natbib}
%\usepackage{color, colortbl}
%\usepackage{picture,slashbox}
\usepackage[hyphens]{url}
\usepackage[bookmarks=false, pdfstartview={XYZ null null 1.00}]{hyperref}
\usepackage{caption} 
\usepackage{esvect}
\captionsetup[table]{skip=10pt}
\usepackage{amsmath}
\usepackage[mathscr]{euscript}
\usepackage{empheq}
\usepackage{graphicx}
\usepackage{fullpage}
\usepackage{titlesec}
\usepackage{wrapfig}
\usepackage{hyperref}
\usepackage{float}
\usepackage[section]{placeins}
\usepackage{enumitem}
\usepackage{tikz}
\usepackage[labelfont=bf]{caption}
\usepackage{amsmath,cases}
\usepackage[dvipsnames]{xcolor}
\usepackage[geometry]{ifsym}
\usepackage{cleveref}
\usepackage{fancyhdr}
\sloppy
\pagestyle{fancy}
\fancyhf{}
\renewcommand{\headrulewidth}{0pt} 
\renewcommand*{\Authands}{, }
\rfoot{APL Supplementary Material}
\lfoot{\textit{Kulkarni et al.} (2023)}
\cfoot{Page \thepage}
\renewcommand{\footrulewidth}{0.4pt}
\let\footnoterule\relax

\def\correspondingauthor{Corresponding author's email address: \href{varun14kul@gmail.com}{varun14kul@gmail.com}}
\newcommand*\samethanks[1][\value{footnote}]{\footnotemark[#1]}

\title{\vspace{-2.2cm} \textbf{\color{black} \Large SUPPLEMENTARY MATERIAL FOR}\\ \vspace{0.2 cm}  \large \textbf{\color{black} On interdependence of instabilities and average drop sizes in bag breakup}}
\author[1]{\large Varun Kulkarni $^*$}
\author[2]{\large Nikhil Shirdade}
\author[1]{\large Neil Rodrigues}
\author[1]{\large Vishnu Radhakrishna}
\author[1]{\\ \large Paul E. Sojka}
\affil[1]{\large School of Mechanical Engineering, \hspace{13cm} Purdue University, IN 47907, USA}
\affil[2]{\large Department of Mechanical Engineering, \hspace{13cm} The Pennsylvania State University, PA 16802, USA \hspace{13cm}$^*$ \correspondingauthor{}}


%\author{}
\date{}
\setlength\parindent{0pt}
\definecolor{Gray}{gray}{0.9}
\definecolor{light-gray}{gray}{0.95}
\definecolor{VeryLightGray}{rgb}{0.90, 0.90, 0.90}
\definecolor{pentane}{rgb}{0.0, 0.75, 1.0}
\definecolor{hexadecane}{rgb}{0.01, 0.75, 0.24}
\definecolor{so}{rgb}{1.0, 0.0, 0.25}

\newcommand{\tikzcircle}[2][black,fill=so]{\tikz[baseline=-0.5ex]\draw[#1,radius=#2] (0,0) circle ;}
\newcommand{\tikzcirclenofillso}[2][so,fill=white, line width=0.4mm]{\tikz[baseline=-0.5ex]\draw[#1,radius=#2] (0,0) circle ;}
\newcommand{\tikzcirclenofillpen}[2][pentane,fill=white, line width=0.4mm]{\tikz[baseline=-0.5ex]\draw[#1,radius=#2] (0,0) circle ;}
\newcommand{\tikzcirclenofillhex}[2][hexadecane,fill=white, line width=0.4mm]{\tikz[baseline=-0.5ex]\draw[#1,radius=#2] (0,0) circle ;}

\setlength{\belowrulesep}{0pt}
\setlength{\aboverulesep}{0pt}
\setlength\heavyrulewidth{0.05ex}
\setlength{\cmidrulewidth}{0.5pt}


\renewcommand{\thefootnote}{$\dagger$}
\setlength{\skip\footins}{1.0cm}
\renewcommand{\thetable}{S\arabic{table}}
\renewcommand{\thesection}{S\arabic{section}}
\newcommand{\ignore}[1]{}
\renewcommand{\thefootnote}{\arabic{footnote}}
\renewcommand{\theequation}{S\arabic{equation}}
\renewcommand{\thefigure}{S\arabic{figure}}
\makeatletter 
\makeatother

\titleformat*{\section}{\large\bfseries}

\captionsetup[figure]{font=small}

\hypersetup
{
	colorlinks = true,
	citecolor  = black,
	linkcolor  = black,
	urlcolor = black
}
\hypersetup{pdfstartview={XYZ null null 1.00}}

\setlist{leftmargin=3.2mm}
\setlength\extrarowheight{5pt}
\renewcommand{\contentsname}{\normalsize This pdf file contains 11 pages which includes:}

\makeatletter
\renewcommand*\l@figure{\@dottedtocline{1}{0em}{2.3em}}% Default: 1.5em/2.3em
\let\l@table\l@figure
\makeatother

\begin{document}
\maketitle
\tableofcontents
\thispagestyle{empty}

%\listoffigures
%\listoftables
\newpage

\section{Experimental determination of drop sizes, validation with existing experiments and theory}
The following discussion elaborates upon on the image processing procedure adopted to determine the rim and film drop sizes, their validation with existing experimental measurements and theoretical arguments to demonstrate their veracity. 
\subsection{Details of image processing to obtain drop sizes} {\label{Section3}}
Our experiments involve capturing videos using Vision Research Phantom v7 high speed digital camera at 4700 fps with an exposure time of 100 $\mu s$ and resolution of 800 $\times$ 600 pixels. A 105 mm focal length lens (Nikon AF Micro Nikkor) was attached to the camera and placed such that the videos were recorded from the front similar to our earlier study. \citep{Kulkarni2013, Kulkarni2014} The breakup and deformation of the drop was illuminated using a 1000 W Xenon arc lamp (Kratos model LH151N) as a light source. In this section we provide details of the video/image processing procedure adopted by us to determine rim, film drops sizes once the videos were captured. The steps listed below use the open source application \href{https://imagej.nih.gov/ij/}{\color{black}{NIH, \underline{ImageJ}}}.
% Figure environment removed 
\begin{enumerate}[label=\textup{(}\roman*\textup{)},font=\itshape,leftmargin=0.24in]
\item \textbf{Homogenizing background of raw image}: The recorded videos are first imported in \href{https://imagej.nih.gov/ij/}{\color{black}{NIH, \underline{ImageJ}}} to extract the requisite details. However, in a typically recorded video (8-bit grayscale) the background illumination is not even. To eliminate measurement errors which may be so introduced, we begin by appropriately homogenizing the background lighting by adjusting the brightness and contrast of the video. The background of one such image after this operation on the image sequence generated from the video is shown in Fig. \ref{FigS2}\textcolor{black}{(\textit{i})}. Next, we take a bare background image which does not contain any drops and divide it from the brightness/contrast adjusted image sequence previously obtained. The resulting image sequence has a white background and a clearly visible deformed drop or/and drop fragments in the foreground. 
\item \textbf{Filtering}: The white background image is then processed to determine the edges. The Sobel operator based filter frequently employed in edge detection algorithms is used to identify the boundaries of the drop fragment features in the image. A result of this filtering operation is shown in Fig. \ref{FigS2}\textcolor{black}{(\textit{ii})}. We assume that the gradient operation to identify the periphery of the drop yields nearly isotropic values around the drop.
\item \textbf{Binary}: Once a filtered image is obtained we invert its colors. The grayscale image so obtained is converted into a binary format by defining a gray-scale cutoff point, below which the fragments become black and above which they become white. Appropriate threshold values are determined based on most of the fragment drops being distinctly identified. The final result of this operation is shown in Fig. \ref{FigS2}\textcolor{black}{(\textit{iii})}.
\item \textbf{Outline}: In the last step, we segment the image obtained from the above procedure to demarcate its boundary (see Fig. \ref{FigS2}\textcolor{black}{(\textit{iv})}). The drops which are sufficiently in-focus all around their perimeter as identified in the previous steps are selected and their projected area $A$ is measured on the image. Note that threshold at half value on the gray scale was performed which locates a droplet boundary that is less dependent on defocus. Finally, we compute the diameter ($D$) of the fragments from their projected area, $A$ as $D = \sqrt{4A/ \pi}$. Since the area could potentially vary in small increments we get a finer resolution above $65\;\mu m$.
\end{enumerate}
The drop size measurements obtained after the above steps are repeated for 3 different trials for a given $We$ and liquid. From this data we are able to easily isolate the diameters of the rim drops from the film drops since they are small in number and large in size compared to the film drops. The minimum drop size we measured was $65\;\mu m$ which was computed from a measured area of $4225\;\mu m^2$. Finally, the average of these measurements for a particular $We$ and liquid is calculated and reported as $\langle D_{rim}\rangle$ and $\langle D_{film}\rangle$ in our paper which is then used for deriving their respective scaling with $We$. It is worth mentioning that the equivalent diameter of aspherical drops is considered as these eventually assume a spherical shape consistent with previous such measurements \citep{Zhao2011}.

\subsection{Comparison with previous experimental results}
The size of the fragmented drops is heavily dependent on the initial size of the drops and the chosen $We$. Larger drops produce bigger fragment/ drops which decrease in size with increasing $We$. Here we compare our measurements with selected literature \cite{Jackiw2022, Zhao2011, Gao2013, Radhakrishna2021, Guildenbecher2017} on the topic. We also emphasize that besides the initial drop size even the method used to fragment the initial drop affects the eventual drops sizes. For example, shock tube results may differ from those obtained by fragmentation due to impact with an air stream since the former subjects the drop to rapid deformation while the latter is more gradual. The criterion for the equivalence between the two methods can be found in the review paper by Guildenbecher et al.(2009)\citep{Guildenbecher2009}. 
Table \ref{TableR1} shows experimental data collected from various sources.
\begin{table}[htp!]
\centering
{
\begin{tabular}{ccccc}
\Xhline{1\arrayrulewidth}
\multirow{1}{*}{Authors Name} & \multirow{1}{*}{Film drop sizes}\qquad  & \multirow{1}{*}{Rim drop sizes} \qquad& \multirow{1}{*}{Initial drop size}\qquad  &\multirow{1}{*}{$We$} \\
\Xhline{1\arrayrulewidth}
Zhao et al.(2011) \citep{Zhao2011} \qquad & $\geq 120$\qquad & 800 $-$ 2400\qquad& 4100 $\pm$ 5900 & 9$-$23 \\
Gao et al. (2013)\citep{Gao2013} \qquad & $\geq 100$\qquad& n.a.\qquad & 2580 & 11 \\
Guildenbecher et al. (2017) \citep{Guildenbecher2017} \qquad & $\geq 65$\qquad & 100 $-$ 200\qquad& 2300 & 13 \\
Radhakrishna et al. (2021) \citep{Radhakrishna2021} \qquad & $\geq 75$\qquad & 300 $-$ 400\qquad& 3000 & 14 \\
Jackiw and Ashgriz (2022) \citep{Jackiw2022} \qquad & $\geq 85$\qquad& 300 $-$ 500\qquad& $1900 \pm 200$ & 7$-$15\\
Present Study \qquad & $\geq 65$\qquad & 350 $-$ 600\qquad& 2100$-$2600 & 12$-$20 \\
\Xhline{1\arrayrulewidth}
\end{tabular}
}
\caption{Measured film and rim drops (in $\mu m$) with associated initial drop diameter (in $\mu m$) at a given $We$ showing that our measurement of the film drops above $65 \mu m$ indeed fall in the range of previously reported measurements of this quantity. Data not available indicated as n.a.}
\label{TableR1}
\end{table}
The latest work by Jackiw and Ashgriz (2022) \citep{Jackiw2022} uses shadowgraphy while Gao et al.(2013)\citep{Gao2013} use digital holography validated by Phase Doppler Anemometry (PDA) measurements to measure drop sizes. Radhakrishna et al.(2021)\citep{Radhakrishna2021} and Guildenbecher et al.(2017)\citep{Guildenbecher2017} too use digital holography to measure fragment sizes with the latter being able to measure drops as small as $27\;\mu m$. All of them report fragmented drop sizes in excess of $65 \mu m$ and therefore our chosen resolution is sufficient.

\subsection{Theoretical proof that most fragment volume lies in drops $\geq 65\;\mu m$}
% Figure environment removed 
Fragmentation of liquids via formation of ligaments has been related to aggregation scenario which result in drop sizes which can be described by gamma distribution\citep{Villermaux2007}. The two parameter form of this distribution is given by,\citep{Jackiw2022}
\begin{equation}\label{twoGD}
\mathcal{F}\left(x = d/s\right) = \dfrac{x^{\alpha-1}e^{-x/\beta}}{\beta^{\alpha}\Gamma\left(\alpha\right)}
\end{equation}
In the above, $\alpha$ is the shape factor, $\beta$ is the scale factor and $s$ is a suitable constant. Choosing $s$ such that, $s = \langle d \rangle = D_{10}$ (number mean diameter) implies, $\langle x \rangle = 1$ and $\beta = \alpha^{-1}$ since from the definition\citep{Jackiw2022} of $\alpha$ and $\beta$ we can write, $\alpha\beta = \langle x\rangle$. This reduces eqn \ref{twoGD} to a single parameter distribution with a mean, $\langle d \rangle$ and written as,
\begin{equation}\label{oneGD}
\mathcal{P}\left(x = d/\langle d \rangle\right) = \dfrac{n^n}{\Gamma\left(n\right)}x^{n-1}e^{-nx}
\end{equation}
where, $\alpha$ has been replaced $n$ to facilitate comparison with the commonly used form.\citep{Villermaux2007, Zhao2011}

In our case both the rim and bag drops are produced by a ligament mediated mechanism (details of which are provided in the main manuscript) and therefore amenable to being fitted by eqn \ref{oneGD} as shown in Fig. \ref{FigS21}\textcolor{black}{(\textit{i})} and Fig. \ref{FigS21}\textcolor{black}{(\textit{ii})}. The combined distribution represented as a linear superposition of these two distributions is exponential as seen in Fig. \ref{FigS21}\textcolor{black}{(\textit{iii})} and is established experimentally and theoretically in earlier works \citep{Zhao2011, Villermaux2009, Villermaux2007} on bag breakup and fragmentation.

To prove that drops of sizes $< 65\;\mu m$ contain very less volume of the original drop, we consider the cumulative gamma distribution of the volume\citep{Zhao2011}  of all (rim \& film) drops denoted by the subscript, ``\textit{all}''. For our example case in Fig. \ref{FigS21}\textcolor{black}{(\textit{iii})} we obtain a fit expressed mathematically as, $\mathcal{P}\left(x_{all} = d_{all}/\langle d_{all} \rangle\right) = 2.5 e^{- d_{all}/\langle d_{all} \rangle}$ from which we may write the following cumulative distribution,
\begin{equation}\label{cpdf}
\mathcal{C}\left(x_{all} = d_{all}/\langle d_{all} \rangle\right) = \dfrac{\int\limits_0^{x_{all}} x_{all}^3 e^{-x_{all}} \,\mathrm{d}x_{all}}{\int\limits_0^{\infty} x_{all}^3 e^{-x_{all}} \,\mathrm{d}x_{all}} = 1 - e^{-x_{all}} \left(1 + x_{all} + \frac{x_{all}^2}{2!} + \frac{{x_{all}}^3}{3!}\right)
\end{equation}
In the above, $\mathcal{C}$ is the fraction of the total initial drop volume contained in drops of diameter less than $d_{all}$. For the test condition shown in Fig. \ref{FigS21}, $d_{all} = 65\;\mu m$ for $\langle d_{all} \rangle = D_{10} = 385\;\mu m$ we obtain using eqn \ref{cpdf}, $\mathcal{C}\left(x_{all} = 0.17\right) = 1.29\%$. This means only 1.29\% of the total volume is contained in drops of diameter $< 65\;\mu m$ for this condition. At other $We$ and for other liquids we see this percentage rise only to 3\% hence justifying our current measurement range in excess of $65\;\mu m$.

\section{Details of derivation of equation (5) in the manuscript: $\cfrac{{{{\overline{\lambda}}_{bag}^2}}}{{\overline{\lambda}\;}_{rim}^4\;{\overline{\lambda}}_{film}} \sim We^2$} {\label{SectionS1}}
When a vertically falling drop is impacted by cross stream of air it accelerates rapidly in radial and streamwise directions. Such accelerations are known to induce Rayleigh-Taylor instability since the heavier liquid (water) tries to accelerate into a lighter fluid (air). However, unlike the original case of superposed fluids investigated by Rayleigh\citep{Rayleigh1882} and Taylor\citep{Taylor1950} surface tension plays a significant role here. Accounting for it \citep{Bellman1954} also enables selection of the most amplified or maximum unstable wavelength \citep{Zhao2010,Zhao2011a, Kulkarni2013, Pfeiffer2020} and whose expression in dimensionless form (scaled by the initial drop diameter, $D_0$) can be written as,
\begin{equation}\label{eqn1}
\dfrac{\lambda}{D_0} \sim \sqrt{\dfrac{\sigma}{\rho_l \xi D_0^2}}
\end{equation}
where, $\lambda$ is the most amplified or maximum unstable wavelength [in m], $\rho_l$ is the density of the liquid drop [in kg$\cdot$m\textsuperscript{-3}], $\sigma_l$ is the surface tension of the liquid drop [in N$\cdot$m\textsuperscript{-1}] and $\xi$ acceleration experienced by the drop [in m$\cdot$s\textsuperscript{-2}]. 
To evaluate the corresponding expressions for eqn \ref{eqn1} for the rim and the bag we need to determine the appropriate acceleration, $\xi$ which is not known \textit{apriori}. This translates to evaluating the acceleration in the deforming liquid drop due to air impact. To do so we simply equate the dynamic pressure of the oncoming air stream, $\rho_a U_a^2$, where $\rho_a$ is the density of air [in kg$\cdot$m\textsuperscript{-3}] and $U_a$ is the velocity of the air stream [in m$\cdot$s\textsuperscript{-1}] to the stress induced in the drop manifesting as its inertia, $\rho_l U_l^2$ where, $U_l$ is the radial velocity of the expanding liquid drop [in m$\cdot$s\textsuperscript{-1}] which is also the velocity with which the drop gets squeezed in the streamwise direction. Therefore, we have \citep{Opfer2014},
\begin{equation}\label{eqn2}
U_l \approx \sqrt{\dfrac{\rho_a}{\rho_l}}U_a
\end{equation}
\subsection{Expression for ${\overline{\lambda}}_{rim}$, eqn (1) in the manuscript}{\label{rim}}
We can compute the acceleration in the rim by recognizing the equivalence of two expressions for inertia per unit area (in the radial, \textit{r} direction), the first being, $\rho_l h_{rim}\xi_{rim}$ and the second being (as stated above), $\rho_l U_l^2$. This simplifies to, $\xi_{rim} \sim U_l^2/ h_{rim}$ which can be substituted in the scaling expression, eqn \ref{eqn1} for most unstable wavelength for the rim giving rise to the following,
\begin{equation}\label{eqn3}
\dfrac{\lambda_{rim}}{D_0} \sim \sqrt{\frac{h_{rim}}{D_0}}\sqrt{\frac{\sigma}{\rho_l U_l^2 D_0}}
\end{equation}
Using eqn \ref{eqn2} and replacing $\lambda/ D_0$ by its more concise form, $\overline{\lambda}$ we obtain,
\begin{equation}\label{eqn4}
{\overline{\lambda}}_{rim} \sim \sqrt{\frac{h_{rim}}{D_0}}{We^{-1/2}}
\end{equation}
where, $We = \dfrac{\rho_a U_a^2 D_0}{\sigma}$ and eqn \ref{eqn4} being identical to eqn (1) in the manuscript.
\subsection{Expression for ${\overline{\lambda}}_{bag}$, eqn (2) in the manuscript}{\label{Section2}}
Following the same reasoning as Section \ref{rim} with the difference that we now compute inertia per unit area in the streamwise (\textit{z}) direction. To move forward, we choose the length scale, $D_{bag}$, which is the cross stream dimension of the disc-like deformed drop just before the Rayleigh-Taylor instability leading to the formation of the bag becomes apparent on its surface. The corresponding acceleration hence assumes the form, $\xi_{bag} \sim U_l^2/ D_{bag}$ which can be substituted in eqn \ref{eqn1}. Repeating the same algebra as before we arrive at the scaling relation for most unstable wavelength for the disc-like deformed drop as written below.
\begin{equation}\label{eqn5}
{\overline{\lambda}}_{bag} \sim \sqrt{\frac{D_{bag}}{D_0}}{We^{-1/2}}
\end{equation}
The above expression reproduces eqn (2) in the manuscript exactly. Note that $D_{bag}$ is entirely different from $D_{max}$ which is the maximum cross stream dimension and attained when the drop expands further radially.
\subsection{Expression for ${\overline{\lambda}}_{film}$, eqn (3) in the manuscript}{\label{film}}
Once the bag forms, the curved liquid film so formed expands rapidly, accelerating at the rate \citep{Villermaux2009}, $\xi_{film} \left(\sim U_l^2/D_0\right)$ in the radial direction (normal to the curved hemispherical bag as shown by blue arrows in Fig. 1 (\textit{a})-(\textit{iv})) . The thin central film is thus subjected to Rayleigh-Taylor instability too, much like the rim and disc-like deformed drop but with the crucial difference that waves develop on both interfaces of the film and their wavelength, $\lambda_{film} >> h_{film}$ where, $h_{film}$ is the thickness of the curved liquid film just before it ruptures and known to remain fairly constant from its inception \citep{Villermaux2009}. The expression for most unstable or amplified wavelength in such a case was first derived by Keller and Kolodner(1954) \citep{Keller1954, Bremond2005, Vledouts2016} and expressed in dimensionless form as,
\begin{equation}\label{eqn6}
\dfrac{\lambda_{film}}{D_0} \sim \dfrac{\sigma}{\rho_l \xi_{film}h_{film}}\left(\dfrac{1}{D_0}\right)
\end{equation}
Substituting for $\xi_{film}$ and using the definition of $We$ we may write the following,
\begin{equation}\label{eqn7}
\overline{\lambda}_{film}  \sim \left(\dfrac{h_{film}}{D_0}\right)^{-1} We^{-1}
\end{equation}
Finally, noting that mass (or volume) is conserved between the toroidal rim and the thin central film at all $We$ considered in our work and reasons for which can be found in discussion of Fig. 2 in the main manuscript and Section \ref{SectionS2} in this supplementary material we write in dimensionless form (scaled by $D_0$ and represented by overline), 
\begin{equation}\label{eqn8}
\overline{h}_{rim} \sim \sqrt{\overline{D}_{bag}\;\;\overline{h}_{film}}
\end{equation}
\subsection{Combining ${\overline{\lambda}}_{film}$, ${\overline{\lambda}}_{rim}$ and ${\overline{\lambda}}_{bag}$ to obtain eqn (5) in the manuscript}
\FloatBarrier
% Figure environment removed 
Rearranging equations \ref{eqn4}, \ref{eqn5} and \ref{eqn7} derived above such that $\overline{h}_{rim}$, $\overline{D}_{bag}$ and $\overline{h}_{film}$ in the above scaling are expressed in terms wavelengths, $\overline{\lambda}_{rim}$, $\overline{\lambda}_{bag}$ and $\overline{\lambda}_{film}$ and $We$ we write the following,
\begin{align}
\overline{h}_{rim}  & \sim {\overline{\lambda}}_{rim}^4 We                \label{eqn9}   \\
\overline{D}_{bag}  & \sim {\overline{\lambda}}_{bag}^2 We 			       \label{eqn10}   \\
\overline{h}_{film}   & \sim {\overline{\lambda}}_{film}^{-1} We^{-1}   \label{eqn11}  
\end{align}
Finally, substituting eqn \ref{eqn9}, \ref{eqn10} and \ref{eqn11} in eqn \ref{eqn8} we obtain the scaling relation given in eqn (5) of the manuscript (also plotted using open symbols in \ref{FigS0}), 
\begin{equation}\label{eqn12}
\frac{{{{\overline{\lambda}}_{bag}^2}}}{{\overline{\lambda}\;}_{rim}^4\;{\overline{\lambda}}_{film}} \sim We^2
\end{equation}

Next we provide details on how to experimentally measure $\lambda_{rim}$, $\lambda_{film}$ and $\lambda_{bag}$ below,
\begin{itemize}
\item \textbf{For bag} we measure the wavelength ($\lambda_{bag}$) the moment we see an apparent formation of thin central disk bounded by a toroidal rim after the disc-like cylindrical structure ceases to exist (see Fig. 1 (\textit{a})-(\textit{iii})) in the manuscript . This value is also almost equivalent to the inner diameter of the bag film that is later formed and may also be estimated reasonably from there.
\item \textbf{For rim}, the wavelength ($\lambda_{rim}$) is seen when waves form on the surface of the rim circumferentially (along the $\theta$ direction of Fig. 1 (\textit{a})-(\textit{iii}) in the manuscript). However, this may not be very easy to identify in its initial stages especially since we view it from the side. So, alternatively we wait for the instability to grow in amplitude and measure the wavelength when the bag bursts and only the intact rim remains as shown in Fig. 1 (\textit{a})-(\textit{vi}) in the main manuscript. This wavelength does not differ much from its value at its incipience.
\item \textbf{For curved thin film} of the bag we measure the wavelength ($\lambda_{film}$) along the curvature of the bag film in the frame before it bursts (see Fig. 1 (\textit{a})-(\textit{v})).
\end{itemize}
To conclude we also show a comparison between our scaling relation in \ref{eqn12} with existing data from literature (see Fig. \ref{FigS0}) to further validate our result. 

\section{$We$ independent conservation of mass between rim, central film and disc-like deformed drop} \label{SectionS2}
% Figure environment removed 
The development of two independent liquid segments namely, the rim and a thin central sheet are characteristic of thinning liquid masses \citep{Yarin1995, Wang2017} which is a consequence of a spatially varying flow profile and surface tension acting on the periphery of the deforming liquid. In our case, the thin central film begins to form once the pancake-like cylindrical deformed drop is influenced by the Rayleigh-Taylor instability \citep{Zhao2010, Zhao2011a} in the streamwise direction (see Fig. \ref{FigS1}(\textit{i}) at $tU_a/D_ 0 = 40$). This thin central film is bounded by a thicker rim both of which thin simultaneously as the drop moves further downstream in the streamwise direction. In this section we show that this thinning is independent of the $We$. 
To prove this we consider the rate of change of rim volume, $\forall_{rim}$ as given below,
\begin{equation}\label{eqn13}
\dfrac{\textrm{d}\forall_{rim}}{\textrm{d}t} = Q_{in}
\end{equation}
Due to equivalence of drop impact/collision phenomena on solids and in air (like our case) as stated in earlier works \citep{Opfer2014} we can write\citep{Yarin1995, Wang2017} the expression for the volume flow rate entering the rim ($Q_{in}$) as,
\begin{equation}\label{eqn14}
Q_{in} = r_{film} h_{film} (u_{film} - u_{rim})
\end{equation}
In the above, $u_{film} = u_{film}\left(r,t\right)$ is the velocity at any point in the thin central film, $u_{rim} = u_{rim}\left(r,t\right)$ is the velocity of the rim and, $h_{film} = h_{film}\left(r,t\right)$ is the thickness of the thin central film. Further, from prior works \citep{Yarin1995, Wang2017} the velocity of the expanding central thin liquid sheet at any time $t$ and radial location, $r_{film}$ is found to be, $u_{film} = h_{film}r_{film}/t$ (details of which can be found in the referenced papers). Substituting for $u_{film}$ in eqn \ref{eqn14} we obtain, $Q_{in} = h_{film}r_{film}^2/t$ assuming $u_{film} >> u_{rim}$. To make progress we take note of the fact that $\forall_{rim}+ \forall_{film} = \frac{\pi}{6}D_0^3$ at all times where, $\forall_{film}$ is the volume contained in the approximately cylindrical thin central film and equals $h_{film}r_{film}^2$ with $\frac{\pi}{6}D_0^3$ representing the original drop volume. We now can recast eqn \ref{eqn13} given this information as follows,
\begin{equation}\label{eqn15}
-\dfrac{\textrm{d}\forall_{film}}{\textrm{d}t} = \dfrac{\forall_{film}}{t}
\end{equation}
We can integrate eqn \ref{eqn15} from $\forall_{film} = 1$ to $\forall_{film}$ on the left hand side and $t = t_0$ to $t + t_0$ on the right hand side where, $t_0$ is the time when the deformed drop assumes a disc-like cylindrical shape with a cross stream length of $D_{bag}$ also referred in Section \ref{Section2} and shown in Fig. \ref{FigS1}\textcolor{black}{(\textit{i})}. On the other hand, $t$ is the measured time after $t_0$. The following expressions for $\forall_{rim}$, $\forall_{film}$ and $\forall_{rim}/\forall_{film}$ may therefore be deduced.
\begin{align}
\forall_{rim}  =   \dfrac{t}{t_0 + t}               									\label{eqn16}   \\
\forall_{film} =   \dfrac{t_0}{t_0 + t}            									\label{eqn17}   \\
\dfrac{\forall_{rim}}{\forall_{film}} =   \dfrac{t}{t_0}            \label{eqn18}   
\end{align}
Eqns \ref{eqn16}, \ref{eqn17} and \ref{eqn18} clearly demonstrate that $\forall_{rim}$, $\forall_{film}$ and the ratio $\forall_{rim}/\forall_{film}$ are independent of $We$. Our experiments show that the deformed drop discernibly separates into a film and rim at $t = 0.5t_0$ with $t_0 \approx 3$ to $5\;ms$ (also see Fig. 1 in the main manuscript). With these considerations, $\forall_{rim}/\forall_{film} \approx 0.5$ as confirmed by our experiments (see Fig. 2 (inset) in the main manuscript).

As a further demonstration of the validity of our result we calculate the scaling for $\overline{D}_{bag}$ from eqn \ref{eqn8} knowing that $\overline{h}_{film}$ and $\overline{h}_{rim}$ both vary as $We^{-1}$ (see Section \ref{Section3} and main manuscript). Doing so, we infer that $\overline{D}_{bag} \sim We^{-1}$ which is exactly confirmed by our experimental data in Fig. \ref{FigS1}\textcolor{black}{(\textit{ii})}. It is known that the Rayleigh-Taylor instability occurs in a time \citep{Lhuissier2012} which scales as, $\left(\sigma/\rho_l\xi^3\right)^{\frac{1}{4}}$. Choosing the time scale\citep{Kulkarni2014}, $D_0/U_a\sqrt{\rho_l/\rho_a}$ and acceleration, $\xi \sim U_l^2/D_0$ we obtain the dimensionless Rayleigh-Taylor instability time as, $\overline{\tau} \sim We^{-0.25}$ in addition to using the simplification, $\rho_a U_a^2 = \rho_l U_l^2$ as required. Moving further, we balance the inertia per unit area of the radially (in $r$ direction) expanding drop, $\rho_l \dot{r}^2$ with the capillary/surface tension pressure, $\sigma/D_0$ acting on it, which upon choosing the velocity scale as $U_l$ gives rise to the scaling, $\overline{\dot{r}} \sim We^{-0.5}$. Multiplying $\overline{\dot{r}}$ with $\overline{\tau}$ as obtained in the preceding expressions we get, $\overline{D}_{bag} = \overline{\dot{r}} \overline{\tau} \sim We^{-0.75}$ which is indeed very close to our experimental scaling exponent of $-0.8$ and the one demanded by our mass conservation $-1$.

To conclude we state that such $We$ independent mass conservation of rim and film has also been reported for drop impact on solids\citep{Wang2022} which our work extends to drop-on-air impacts.

\section{Rolling rim image details for Fig 3 and details of derivation leading to scaling for $\langle \overline{D}_{rim}\rangle$ and $\langle \overline{D}_{film}\rangle$} {\label{Section3}}
Curved liquid films are known to burst by initiation of a hole.\citep{Lhuissier2012,Poulain2018} For a drop undergoing bag breakup a similar curved liquid film presents itself and therefore it is natural to expect a congruent bursting process. In this section we show that the mechanism which governs bursting of bubbles can be extended to bursting of a bag, providing details on film drop production by such a process. 
% Figure environment removed 
In Fig. \ref{FigS3} \textcolor{black}{\textit{t} = 14, 15, 16 and 17 \textit{ms}} we show images of the intermediate steps once the hole is formed leading up to the bursting of the curved liquid film. A clear formation of rim is seen which develops once the hole forms in Fig. \ref{FigS3} \textcolor{black}{\textit{t} = 14 \textit{ms}}. As it recedes it is acted upon by surface tension giving rise to a rolling toroidal rim. Such rims, as discussed in the main manuscript are ubiquitous to drop impact phenomena, albeit here our situation is inverted, wherein we have a hole expanding in continuous liquid medium. The rolling toroidal rim is unstable and ejects drops periodically. It is interesting to note that such a rim is fed constantly by the curved liquid film and therefore never depletes in mass until it collides with the rim that bounds the bag. The foregoing argument means that it is reasonable to expect the diameter of rolling rim ($d_{roll}$), the ligaments ($d_{lig}$) and the film drops($D_{film}$) produced to be roughly the same. 

For our next step we refer the schematic used in \textcolor{black}{Fig. 3 (\textit{b})-(\textit{ii})} of the main manuscript to write the volume per unit width in rolling cylindrical rim (assumed to be planar as stated in the manuscript), $\frac{\pi}{4} d_{roll}^2$. This volume is continually fed by the planar sheet (per unit width) of thickness $h_{film}$ at a rate $U_{film}$ and in time, $\tau_{roll}$ corresponding to one time period between successive drop ejections (from Fig. \ref{FigS3} {\textcolor{black}{\textit{t} = 14 to 17 \textit{ms}}}. Therefore, we can write an expression similar to that developed by Lhuissier and Villermaux(2012)\citep{Lhuissier2012} as given below,
\begin{equation}\label{rolldia}
\langle D_{film} \rangle = d_{lig} = d_{roll} = \sqrt{\left(\dfrac{4}{\pi}\right) h_{film} U_{film} \tau_{roll}}
\end{equation}
Replacing, $\tau_{roll}$ by $h_{film}/U_{roll}$ since the film is stretched due to air flow and therefore needs to be separately calculated. Thicker films (with higher $h_{film}$) consequently have larger time between drop ejections. With the substitution for $\tau_{roll}$ eqn \ref{rolldia} now transforms to,
\begin{equation}\label{rolldia}
\langle D_{film} \rangle = \sqrt{\left(\dfrac{4}{\pi}\right) h_{film}^2 \dfrac{U_{film}}{U_{roll}}}
\end{equation}
In view of eqn \ref{rolldia}, to obtain closure we need to determine $U_{film}/U_{roll}$. To do so, we need to evaluate, $\dot{\mathcal{E}}_{film}$ which is the energy per unit time and per unit area assuming unit width of the moving film of depth $h_{film}$. This can be done by considering the force experienced by the moving film due to the dynamic air pressure acting initially on the drop which during the film expansion scales as, $(D_0^2 \rho_l U_{l}^2)$ and since it induces a velocity $U_{film}$ the energy per unit time carried by it is, $(D_0^2 \rho_l U_{l}^2)U_{film}$. The power (energy per unit time) so generated is delivered across a cross sectional area, $h_{film}\cdot1$ of the film assuming unit width which implies, $\dot{\mathcal{E}}_{film} \sim (D_0^2 \rho_l U_{l}^2)U_{film}/h_{film}$. The rolling rim absorbs this energy to maintain its diameter, $d_{roll}$ at all times (which depletes due to drop ejection) moving at $U_{roll}$ corresponding to an energy per unit time given by, $\sigma d_{roll} U_{roll}$. By considering the curved cylindrical surface area of the rolling rim per unit width, $d_{roll}\cdot1$ we can write the surface energy per unit time per unit area required to maintain the diameter of the rim at $d_{roll}$ as, $\dot{\mathcal{E}}_{roll} \sim \sigma d_{roll} U_{roll}/d_{roll}$. Equating $\dot{\mathcal{E}}_{film}$ to $\dot{\mathcal{E}}_{roll}$ results in,
\begin{equation}\label{UfilmUroll}
\frac{U_{film}}{U_{roll}} \sim \frac{\sigma h_{film}}{\rho_a U_a^2 D_0^2}  = We^{-1} \overline{h}_{film}
\end{equation}
Eqn \ref{UfilmUroll} provides necessary closure to solve for $\langle D_{film} \rangle$ in eqn \ref{rolldia}. We also mention in the passing that the substitution (as discussed previously), $\rho_l U_l^2 = \rho_a U_a^2$ is used in the above equation to write it in terms of $We$. Lastly, scaling for $\overline{h}_{rim}$ and $\overline{h}_{film}$ are identical being analogous phenomena (also see main manuscript) and can be derived from the force balance between liquid inertia, $\rho_l U_l^2$ and surface tension $\sigma/h_{film\;or\;rim}$\citep{Opfer2014, Lhuissier2012}. Such a calculation readily yields a scaling dependence of the form, $We^{-1}$ which when substituted in eqn \ref{UfilmUroll} leads to the following scaling alos shown as eqn (6) in the manuscript.
\begin{equation}\label{Dfilm}
\langle \overline{D}_{film} \rangle \sim We^{-2}
\end{equation}
The average rim drop size is a consequence of conservation of volume between one segment of the wavelength, $\lambda_{rim}$ of thickness, $h_{rim}$ and the expected spherical drop size, $\langle \overline{D}_{rim} \rangle$. The important thing to note here is that we consider the rim breakup to be driven by Rayleigh-Taylor instability since the number of drops generated within a wavelength of the rim, $\lambda_{rim}$ is nearly a constant (4-5). Hence, knowing that $\overline{h}_{rim}$ scales as $We^{-1}$ we obtain the following, which is equivalent to eqn (7) in the manuscript
\begin{equation}\label{Drim}
\langle \overline{D}_{rim} \rangle \sim We^{-1}
\end{equation}
\bibliographystyle{unsrtnat}
\bibliography{SI_APL_References}
\end{document}
