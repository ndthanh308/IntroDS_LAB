% ****** Start of file aipsamp.tex ******
%
%   This file is part of the AIP files in the AIP distribution for REVTeX 4.
%   Version 4.1 of REVTeX, October 2009
%
%   Copyright (c) 2009 American Institute of Physics.
%
%   See the AIP README file for restrictions and more information.
%
% TeX'ing this file requires that you have AMS-LaTeX 2.0 installed
% as well as the rest of the prerequisites for REVTeX 4.1
% 
% It also requires running BibTeX. The commands are as follows:
%
%  1)  latex  aipsamp
%  2)  bibtex aipsamp
%  3)  latex  aipsamp
%  4)  latex  aipsamp
%
% Use this file as a source of example code for your aip document.
% Use the file aiptemplate.tex as a template for your document.
\documentclass
[%
 aip, apl,
% jmp,
% bmf,
% sd,
% rsi,
 amsmath,amssymb,
%preprint,%
reprint,%
%author-year,%
%author-numerical,%
% Conference Proceedings
]{revtex4-2}
\usepackage{xcolor}
\usepackage{graphicx}% Include figure files
\usepackage{dcolumn}% Align table columns on decimal point
\usepackage{enumitem}
\usepackage{balance}
\usepackage{mathptmx,mathtools}
\usepackage[cmbraces,varbb,varvw]{newtxmath}
\usepackage{etoolbox}
\usepackage[scaled=1.0]{helvet}
\usepackage{titlesec}
\usepackage[colorlinks = true, linkcolor = blue, citecolor =blue, urlcolor = blue, linkbordercolor = white]{hyperref}
\usepackage{fancyhdr}
\pagestyle{fancy}
\fancyhead[L]{\normalfont\sffamily Kulkarni et al. (2023)}
\fancyhead[C]{\normalfont\sffamily APL Accepted Manuscript, Preprint}
\fancyhead[R]{\normalfont\sffamily \thepage}
\cfoot{}
\renewcommand{\headrulewidth}{0pt}
\titleformat{\section}
  {\normalfont\sffamily\large\bfseries\color{black}}
  {\thesection.}{1em}{}
\titleformat{\subsection}
  {\normalfont\sffamily\large\bfseries\color{black}}
  {\Alph{subsection}.}{1em}{}

\makeatletter
\def\@email#1#2
{%
 \endgroup
 \patchcmd{\titleblock@produce}
  {\frontmatter@RRAPformat}
  {\frontmatter@RRAPformat{\produce@RRAP{*#1\href{mailto:#2}{#2}}}\frontmatter@RRAPformat}
  {}{}
}%


\usepackage{filecontents}
\begin{filecontents}{APLRefs.bib}

@article{Villermaux2009,
  title={Single-drop fragmentation determines size distribution of raindrops},
  author={Villermaux, Emmanuel and Bossa, Benjamin},
  journal={Nature Physics},
  volume={5},
  number={9},
  pages={697--702},
  year={2009},
  publisher={Nature Publishing Group}
}

@article{Troitskaya2018,
  title={The “bag breakup” spume droplet generation mechanism at high winds. Part II: Contribution to momentum and enthalpy transfer},
  author={Troitskaya, Yu and Druzhinin, Oleg and Kozlov, D and Zilitinkevich, Sergei},
  journal={Journal of Physical Oceanography},
  volume={48},
  number={9},
  pages={2189--2207},
  year={2018},
  publisher={American Meteorological Society}
}

@article{Jalaal2012,
  title={Fragmentation of falling liquid droplets in bag breakup mode},
  author={Jalaal, M and Mehravaran, K},
  journal={International Journal of Multiphase Flow},
  volume={47},
  pages={115--132},
  year={2012},
  publisher={Elsevier}
}

@article{Scharfman2016,
  title={Visualization of sneeze ejecta: steps of fluid fragmentation leading to respiratory droplets},
  author={Scharfman, BE and Techet, AH and Bush, JWM and Bourouiba, L},
  journal={Experiments in Fluids},
  volume={57},
  pages={1--9},
  year={2016},
  publisher={Springer}
}

@article{Wei2020,
  title={Atomization of acoustically levitated droplet exposed to hot gases},
  author={Wei, Yanju and Yang, Yajing and Zhang, Jie and Deng, Shengcai and Liu, Shenghua and Law, Chung K and Saha, Abhishek},
  journal={Applied Physics Letters},
  volume={116},
  number={4},
  pages={044101},
  year={2020},
  publisher={AIP Publishing LLC}
}

@article{Guildenbecher2009,
  title={Secondary atomization},
  author={Guildenbecher, DR and L{\'o}pez-Rivera, C and Sojka, PE},
  journal={Experiments in Fluids},
  volume={46},
  number={3},
  pages={371--402},
  year={2009},
  publisher={Springer}
}

@article{Wang2022,
  title={Mass, momentum and energy partitioning in unsteady fragmentation},
  author={Wang, Y and Bourouiba, L},
  journal={Journal of Fluid Mechanics},
  volume={935},
  pages={A29},
  year={2022},
  publisher={Cambridge University Press}
}

@article{Kulkarni2014a,
  title={Bag breakup of low viscosity drops in the presence of a continuous air jet},
  author={Kulkarni, Varun and Sojka, Paul E},
  journal={Physics of Fluids},
  volume={26},
  number={7},
  pages={072103},
  year={2014},
  publisher={American Institute of Physics}
}

@Conference{Kulkarni2014b,
  title={Fragmentation dynamics in the droplet bag breakup regime},
  author={Kulkarni, Varun and Sojka, Paul},
  booktitle={Bulletin of the American Physical Society},
  pages={H12--001},
  publisher = {American Physical Society (APS)},
  year={2014}
}

@article{Krzeczkowski1980,
  title={Measurement of liquid droplet disintegration mechanisms},
  author={Krzeczkowski, Stefan A},
  journal={International Journal of Multiphase Flow},
  volume={6},
  number={3},
  pages={227--239},
  year={1980},
  publisher={Elsevier}
}

@article{Radhakrishna2021,
  title={Experimental characterization of secondary atomization at high Ohnesorge numbers},
  author={Radhakrishna, Vishnu and Shang, Weixiao and Yao, Longchao and Chen, Jun and Sojka, Paul E},
  journal={International Journal of Multiphase Flow},
  volume={38},
  pages={103591},
  year={2021},
  publisher={Elsevier}
}

@article{Jain2015,
  title={Secondary breakup of a drop at moderate Weber numbers},
  author={Jain, Mohit and Prakash, R Surya and Tomar, Gaurav and Ravikrishna, RV},
  journal={Proceedings of the Royal Society A: Mathematical, Physical and Engineering Sciences},
  volume={471},
  number={2177},
  pages={20140930},
  year={2015},
  publisher={The Royal Society Publishing}
}

@article{Chryssakis2008,
  title={A unified fuel spray breakup model for internal combustion engine applications},
  author={Chryssakis, Christos and Assanis, Dennis N},
  journal={Atomization and Sprays},
  volume={18},
  number={5},
  year={2008},
  publisher={Begel House Inc.}
}

@phdthesis{Kulkarni2013,
  title={An analytical and experimental study of secondary atomization of vibrational and bag breakup modes},
  author={Kulkarni, Varun},
  year={2013},
  school={Purdue University}
}

@article{Wang2014,
  title={Modeling of drop breakup in the bag breakup regime},
  author={Wang, Chao and Chang, Shinan and Wu, Hongwei and Xu, J},
  journal={Applied Physics Letters},
  volume={104},
  number={15},
  pages={154107},
  year={2014},
  publisher={American Institute of Physics}
}

@article{Joshi2022,
  title={Droplet deformation in secondary breakup: Transformation from a sphere to a disk-like structure},
  author={Joshi, Sumit and Anand, TNC},
  journal={International Journal of Multiphase Flow},
  volume={146},
  pages={103850},
  year={2022},
  publisher={Elsevier}
}

@article{Opfer2014,
  title={Droplet-air collision dynamics: Evolution of the film thickness},
  author={Opfer, L and Roisman, IV and Venzmer, J and Klostermann, M and Tropea, C},
  journal={Physical Review E},
  volume={89},
  number={1},
  pages={013023},
  year={2014},
  publisher={APS}
}

@article{Quan2006,
  title={Direct numerical study of a liquid droplet impulsively accelerated by gaseous flow},
  author={Quan, Shaoping and Schmidt, David P},
  journal={Physics of Fluids},
  volume={18},
  number={10},
  pages={102103},
  year={2006},
  publisher={American Institute of Physics}
}

@article{Jackiw2022,
  title={Prediction of the droplet size distribution in aerodynamic droplet breakup},
  author={Jackiw, Isaac M and Ashgriz, Nasser},
  journal={Journal of Fluid Mechanics},
  volume={940},
  year={2022},
  publisher={Cambridge University Press}
}

@article{Zhao2011b,
  title={Experimental study of drop size distribution in the bag breakup regime},
  author={Zhao, Hui and Liu, Hai Feng and Xu, Jian Liang and Li, Wei Feng},
  journal={Industrial \& Engineering Chemistry Research},
  volume={50},
  number={16},
  pages={9767--9773},
  year={2011},
  publisher={ACS Publications}
}

@article{Zhao2011a,
  title={Breakup characteristics of liquid drops in bag regime by a continuous and uniform air jet flow},
  author={Zhao, Hui and Liu, Hai-Feng and Cao, Xian-Kui and Li, Wei-Feng and Xu, Jian-Liang},
  journal={International Journal of Multiphase Flow},
  volume={37},
  number={5},
  pages={530--534},
  year={2011},
  publisher={Pergamon-Elsevier Science Ltd., Oxford, England}
}

@article{Taylor1950,
  title={The instability of liquid surfaces when accelerated in a direction perpendicular to their planes. I},
  author={Taylor, Geoffrey Ingram},
  journal={Proceedings of the Royal Society of London. Series A. Mathematical and Physical Sciences},
  volume={201},
  number={1065},
  pages={192--196},
  year={1950},
  publisher={The Royal Society London}
}

@article{Rayleigh1882,
  title={Investigation of the character of the equilibrium of an incompressible heavy fluid of variable density},
  author={Rayleigh, Rayleigh},
  journal={Proceedings of the London mathematical society},
  volume={1},
  number={1},
  pages={170--177},
  year={1882},
  publisher={Oxford Academic}
}

@article{Zhao2010,
  title={Morphological classification of low viscosity drop bag breakup in a continuous air jet stream},
  author={Zhao, Hui and Liu, Hai-Feng and Li, Wei-Feng and Xu, Jian-Liang},
  journal={Physics of Fluids},
  volume={22},
  number={11},
  pages={114103},
  year={2010},
  publisher={American Institute of Physics}
}

@article{Soni2020,
  title={Deformation and breakup of droplets in an oblique continuous air stream},
  author={Soni, Surendra Kumar and Kirar, Pavan Kumar and Kolhe, Pankaj and Sahu, Kirti Chandra},
  journal={International Journal of Multiphase Flow},
  volume={122},
  pages={103141},
  year={2020},
  publisher={Elsevier}
}

@inproceedings{Kulkarni2012a,
  title={Secondary atomization of Newtonian liquids in the bag breakup regime: Comparison of model predictions to experimental data},
  author={Kulkarni, V and Guildenbecher, DR and Sojka, PE},
  booktitle={ICLASS 2012, 12th International Conference on Liquid Atomization and Spray Systems, Heidelberg, Germany},
  year={2012}
}

@Conference{Kulkarni2012b,
  title={Bag Breakup of Viscous Drops},
  author={Kulkarni, Varun and Guildenbecher, Daniel and Firehammer, Stephanie and Sojka, Paul},
  booktitle={Bulletin of the American Physical Society},
  publisher = {American Physical Society (APS)},
  pages={L8--001},
  year={2012}
}

@Conference{Kulkarni2015,
  title={Fragment size distribution in viscous bag breakup of a drop},
  author={Kulkarni, Varun and Bulusu, Kartik V and Plesniak, Michael W and Sojka, Paul E},
  booktitle={Bulletin of the American Physical Society},
  publisher = {American Physical Society (APS)},
  pages={D32--007},
  year={2015}
}

@article{Theofanous2011,
  title={Aerobreakup of Newtonian and viscoelastic liquids},
  author={Theofanous, TG},
  journal={Annual Review of Fluid Mechanics},
  volume={43},
  pages={661--690},
  year={2011},
  publisher={Annual Reviews}
}

@article{Hsiang1995,
  title={Drop deformation and breakup due to shock wave and steady disturbances},
  author={Hsiang, L-P and Faeth, Gerard M},
  journal={International Journal of Multiphase Flow},
  volume={21},
  number={4},
  pages={545--560},
  year={1995},
  publisher={Elsevier}
}

@article{Sharma2022,
  title={Advances in droplet aerobreakup},
  author={Sharma, Shubham and Chandra, Navin Kumar and Basu, Saptarshi and Kumar, Aloke},
  journal={The European Physical Journal Special Topics},
  pages={1--15},
  year={2022},
  publisher={Springer}
}

@article{Gao2013,
  title={Quantitative, three-dimensional diagnostics of multiphase drop fragmentation via digital in-line holography},
  author={Gao, Jian and Guildenbecher, Daniel R and Reu, Phillip L and Kulkarni, Varun and Sojka, Paul E and Chen, Jun},
  journal={Optics Letters},
  volume={38},
  number={11},
  pages={1893--1895},
  year={2013},
  publisher={Optica Publishing Group}
}

@article{Yang2017,
  title={Transitions of deformation to bag breakup and bag to bag-stamen breakup for droplets subjected to a continuous gas flow},
  author={Yang, Wei and Jia, Ming and Che, Zhizhao and Sun, Kai and Wang, Tianyou},
  journal={International Journal of Heat and Mass Transfer},
  volume={111},
  pages={884--894},
  year={2017},
  publisher={Elsevier}
}


@article{Fang2022,
  title={Drop impact dynamics on solid surfaces},
  author={Fang, Wei and Zhang, Kaixuan and Jiang, Qi and Lv, Cunjing and Sun, Chao and Li, Qunyang and Song, Yanlin and Feng, Xi-Qiao},
  journal={Applied Physics Letters},
  volume={121},
  number={21},
  pages={210501},
  year={2022},
  publisher={AIP Publishing LLC}
}

@article{Savva2009,
  title={Viscous sheet retraction},
  author={Savva, Nikos and Bush, John WM},
  journal={Journal of Fluid Mechanics},
  volume={626},
  pages={211--240},
  year={2009},
  publisher={Cambridge University Press}
}

@article{Lhuissier2012,
  title={Bursting bubble aerosols},
  author={Lhuissier, Henri and Villermaux, Emmanuel},
  journal={Journal of Fluid Mechanics},
  volume={696},
  pages={5--44},
  year={2012},
  publisher={Cambridge University Press}
}

@article{Taylor1959,
  title={The dynamics of thin sheets of fluid. III. Disintegration of fluid sheets},
  author={Taylor, Geoffrey Ingram},
  journal={Proceedings of the Royal Society of London. Series A. Mathematical and Physical Sciences},
  volume={253},
  number={1274},
  pages={313--321},
  year={1959},
  publisher={The Royal Society London}
}

@article{Culick1960,
  title={Comments on a ruptured soap film},
  author={Culick, Fred EC},
  journal={Journal of applied physics},
  volume={31},
  number={6},
  pages={1128--1129},
  year={1960},
  publisher={American Institute of Physics}
}

@article{Pfeiffer2020,
  title={Merging of soap bubbles and why surfactant matters},
  author={Pfeiffer, Patricia and Zeng, Qingyun and Tan, Beng Hau and Ohl, Claus-Dieter},
  journal={Applied Physics Letters},
  volume={116},
  number={10},
  pages={103702},
  year={2020},
  publisher={AIP Publishing LLC}
}

@article{Keller1954,
  title={Instability of liquid surfaces and the formation of drops},
  author={Keller, Joseph B and Kolodner, Ignace},
  journal={Journal of Applied Physics},
  volume={25},
  number={7},
  pages={918--921},
  year={1954},
  publisher={American Institute of Physics}
}

@article{Vledouts2016,
  title={Explosive fragmentation of liquid shells},
  author={Vledouts, A and Quinard, J and Vandenberghe, N and Villermaux, E},
  journal={Journal of Fluid Mechanics},
  volume={788},
  pages={246--273},
  year={2016},
  publisher={Cambridge University Press}
}

@article{Mullagura2020,
  title={Bubble-Induced Rupture of Droplets on Hydrophobic and Lubricant-Impregnated Surfaces},
  author={Mullagura, Haritha N and Dash, Susmita},
  journal={Langmuir},
  volume={36},
  number={30},
  pages={8858--8864},
  year={2020},
  publisher={ACS Publications}
}

@article{Naidu2022,
  title={Impact dynamics of air-in-liquid compound droplets},
  author={Naidu, Deekshith P and Dash, Susmita},
  journal={Physics of Fluids},
  volume={34},
  number={7},
  pages={073604},
  year={2022},
  publisher={AIP Publishing LLC}
}

@article{Vadivukkarasan2020,
  title={Breakup morphology of expelled respiratory liquid: From the perspective of hydrodynamic instabilities},
  author={Vadivukkarasan, M and Dhivyaraja, K and Panchagnula, Mahesh V},
  journal={Physics of Fluids},
  volume={32},
  number={9},
  pages={094101},
  year={2020},
  publisher={AIP Publishing LLC}
}

@article{Mehrabian2013,
  title={Capillary breakup of a liquid torus},
  author={Mehrabian, Hadi and Feng, James J},
  journal={Journal of Fluid Mechanics},
  volume={717},
  pages={281--292},
  year={2013},
  publisher={Cambridge University Press}
}

@article{Pairam2009,
  title={Generation and stability of toroidal droplets in a viscous liquid},
  author={Pairam, E and Fern{\'a}ndez-Nieves, A},
  journal={Physical Review Letters},
  volume={102},
  number={23},
  pages={234501},
  year={2009},
  publisher={APS}
}

@article{Kulkarni2021,
  title={Coalescence and spreading of drops on liquid pools},
  author={Kulkarni, Varun and Lolla, Venkata Yashasvi and Tamvada, Suhas Rao and Shirdade, Nikhil and Anand, Sushant},
  journal={Journal of Colloid and Interface Science},
  volume={586},
  pages={257--268},
  year={2021},
  publisher={Elsevier}
}
\end{filecontents}

\makeatother

\begin{document}

\preprint{AIP/123-QED}
\thispagestyle{empty} 
% Figure environment removed
\clearpage

\title[\textit{\footnotesize Applied Physics Letters}]{\textcolor{blue}{On interdependence of instabilities and average drop sizes in bag breakup} \vspace{10pt}\\}
% Force line breaks with \\
\author{Varun Kulkarni}
\altaffiliation[Corresponding author, electronic mail: ]{\href{mailto:varun14kul@gmail.com}{varun14kul@gmail.com}}
\affiliation{School of Mechanical Engineering, Purdue University, West Lafayette, IN 47907, \textcolor{black}{USA}}
\author{Nikhil Shirdade}
\affiliation{Department of Mechanical Engineering, Pennsylvania State University, State College, PA 16802 ,\textcolor{black}{USA} \looseness=-1} 
\author{Neil Rodrigues}
\affiliation{School of Mechanical Engineering, Purdue University, West Lafayette, IN 47907, \textcolor{black}{USA}}
\author{Vishnu Radhakrishna}
\affiliation{School of Mechanical Engineering, Purdue University, West Lafayette, IN 47907, \textcolor{black}{USA}}
\author{Paul E. Sojka}
\affiliation{School of Mechanical Engineering, Purdue University, West Lafayette, IN 47907, \textcolor{black}{USA}}
\setcounter{page}{1}
\begin{abstract}
A drop exposed to cross flow of air experiences sudden accelerations which deform it rapidly ultimately proceeding to disintegrate it into smaller fragments. In this work, we examine the breakup of a drop as a bag film with a bounding rim resulting from acceleration induced Rayleigh-Taylor instabilities and characterized through the Weber number, \textit{We}, representative of the competition between the disruptive aerodynamic force imparting acceleration and the restorative surface tension force. Our analysis reveals a previously overlooked parabolic dependence ($\sim We^2$) of the combination of dimensionless instability wavelengths $({\bar{\lambda}}_{bag}^2/ {\bar{\lambda}}_{rim}^4 {\bar{\lambda}}_{film})$ developing on different segments of the deforming drop. Further, we extend these findings to deduce the dependence of the average dimensionless drop sizes for the rim, $\langle {\widebar{D}}_{rim} \rangle$ and bag film, $\langle {\widebar{D}}_{film} \rangle$ individually, on $We$ and see them to decrease linearly for the rim ($\sim We^{-1}$) and quadratically for the bag film ($\sim We^{-2}$). The reported work is expected to have far-reaching implications as it provides unique insights on destabilization and disintegration mechanisms based on theoretical scaling arguments involving the commonly encountered canonical geometries of a toroidal rim and a curved liquid film.
%\\ Area: Interdisciplinary Applied Physics, Surfaces and Interfaces.
\end{abstract}

\maketitle

\section*{\normalsize Introduction \vspace{-12pt}}
Atomization of a single drop impacted by a stream of air appears as the constituent element of diverse natural and industrial processes, examples of which can be seen in falling raindrops\cite{Villermaux2009}, sea aerosols\cite{Troitskaya2018}, sneeze ejecta \cite{Scharfman2016} and, liquid propellant combustion.\cite{Wei2020} Of notable importance in this process is the deformation that the drop undergoes en route to its complete fragmentation. Depending on the air velocity, the morphology exhibited by the deforming drop could vary, leading to different regimes of breakup. \cite{Guildenbecher2009} One of these is the topological transition of a drop into a near-hemispherical \textit{bag} bounded by a toroidal \textit{rim} termed as \textit{bag breakup} and is significant as it sets the lower limit for air velocity at which complete drop fragmentation is assured.\cite{Kulkarni2014a, Kulkarni2014b} To characterize features of this regime across drops of different initial diameter, $D_0$ (in \textit{m}) and surface tension, $\sigma$ (in \textit{N}$\cdot$\textit{m}\textsuperscript{-1}) the balance between the restoring capillary pressure, $\sigma /D_0$ and opposing aerodynamic pressure, $\rho_a U_a^2$ is used to form a dimensionless number known as the Weber number,\cite{Sharma2022, Kulkarni2014a, Guildenbecher2009}\textcolor{black}{$We \left(= \rho_a U_a^2 D_0 /\sigma \right)$}, where, $\rho_a$ is the air density (in \textit{kg}$\cdot$\textit{m}\textsuperscript{-3}) and $U_a$ is the air velocity (in \textit{m}$\cdot$\textit{s}\textsuperscript{-1}). In these terms, bag breakup in the presence of a continuous cross stream of air is found to occur between $12 \lessapprox We \lessapprox 20$. \cite{Kulkarni2014a, Krzeczkowski1980, Jain2015, Guildenbecher2009, Chryssakis2008}

The morphological evolution of a drop as it undergoes bag breakup (see Fig. \ref{Fig1}) from (\textit{i}) its initial undeformed spherical shape, can be briefly described by its following five main stages\cite{Kulkarni2014a, Kulkarni2013,Wang2014} (\textit{ii}) Initial flattening of the drop (\textit{iii}) Formation of thin central film and rim (\textit{iv}) Growth of thin central film into a bag (\textit{v}) Bursting of curved film into drops (\textit{vi}) Disintegration of the rim into drops. Whilst most studies have focused on initial deformation \citep{Wang2014, Joshi2022, Opfer2014, Quan2006} (\textit{ii}) and the final disintegration\citep{Jackiw2022, Villermaux2009, Zhao2011b} (\textit{v, vi}), details of the connection between stages (\textit{iii} and \textit{iv}), governed by hydrodynamic instabilities and their influence on fragment production namely, stages (\textit{v} and \textit{vi}) has remained unexplored. These instabilities destabilize the liquid drop-air interface \cite{Zhao2011a, Zhao2011b, Zhao2010} in accordance with the Rayleigh-Taylor instability, known to occur when a heavy fluid (liquid drop), accelerates into a lighter fluid (air) \cite{Taylor1950, Rayleigh1882} thereby controlling the ensuing breakup dynamics. In this letter, we investigate the details of such manifestations and in so doing unearth a yet unreported, intriguing correlation between the three main instability wavelengths ($\lambda_{rim}$, $\lambda_{bag}$ and $\lambda_{film}$) which develop as the drop deforms and extend those findings to examine the differences in $We$ scaling for the average rim and film drops, $\langle D_{film}\rangle$ and $\langle D_{rim}\rangle$ respectively.
% Figure environment removed

\section*{\normalsize \vspace{-20pt} Experimental Methods and Materials \vspace{-12pt}}
For our experiments, we use five test fluids which are four glycerine-water mixtures, 40\%, 50\%, 63\%, 70\% by weight and D.I. water. The fluids considered are incompressible with constant density, $\rho_l$ (in \textit{kg}$\cdot$\textit{m}\textsuperscript{-3}) and low dynamic viscosity, $\mu_l$ (in \textit{Pa}$\cdot$\textit{s}) which correspond to the dimensionless Ohnesorge number, $Oh = \mu_l/\sqrt{\rho_l \sigma D_0} < 0.1$ hence eliminating the role of viscosity in our findings and consistent with earlier studies.\cite{Jain2015,Soni2020,Zhao2011a} The apparatus and fluids used are similar to those used in our prior work \cite{Kulkarni2014a, Kulkarni2014b, Kulkarni2012a, Kulkarni2012b, Kulkarni2015} and consist of a converging nozzle producing a near flat air velocity profile. Drops ($D_0 \approx 2.1 - 2.6$ \textit{mm}) are released vertically from a syringe needle and are small enough to neglect gravitational effects. The air flow velocities are varied between 10 to 13 \textit{m}$\cdot$\textit{s}\textsuperscript{-1} which correspond to $12 \lessapprox We \lessapprox 20$ within the bag breakup regime. The experimental observations (see Fig.\ref{Fig1}\textcolor{blue}{(a), (\textit{i})-(\textit{vii})}) are recorded using videos taken at 4700 fps and a resolution of 800 $\times$ 600 pixels such that 1 pixel $\approx$ 65 $\mu m$. It is worth mentioning that drop breakup and deformation in our work is kindred with studies using shock waves\cite{Theofanous2011, Hsiang1995, Sharma2022} although bag breakup is observed for a slightly broader range of $10 \lessapprox We \lessapprox 24 \textrm{ to } 35$ due to differences in aerodynamic loading. 

To measure bag and rim drop sizes, 3 videos for a given $We$ and $Oh$ were analyzed using thresholding and post processing described in literature.\cite{Zhao2011b, Villermaux2009} Film drops were measured once the bag bursting is complete while the rim is still intact (see Fig. \ref{Fig1}\textcolor{blue}{(a), (\textit{v})}) and rim drops were measured upon complete disintegration of the rim (see Fig. \ref{Fig1}\textcolor{blue}{(a), (\textit{v})}). Node formation in the rim was not considered dominant due to low  \textit{Oh} and comparatively lower $We$ compared to previous works. \cite{Zhao2010, Jackiw2022} The measured bag and rim drop sizes are an arithmetic average of all rim drops ($\approx$ 50) and majority of the bag film drops ($\approx$ 1000) at a given $We$ for the five test fluids and above the spatial resolution of our imaging and as demonstrated in previous studies.\cite{Gao2013, Zhao2011b, Radhakrishna2021, Jackiw2022}A pixel resolution of $\approx$ 65 $\mu m$ was found to be adequate since majority of the drop fragments fall above 65 $\mu m$ and we are able to finely resolve drop sizes above this value (also see section S1 of supplementary material for additional details on the image processing procedure and validation).The total volume from the film drops ($\forall_{film}$) and the rim drops ($\forall_{rim}$) at each condition was combined to validate that nearly all the original drop volume ($\forall_{drop}$) is recovered. Lastly, in view of our experimental data, we assume that both the rim and curved film drops are monodispersed.
\vspace{-20pt}
\section*{\normalsize Drop deformation and separation of length scales \vspace{-10pt}}
Fig. \ref{Fig1}\textcolor{blue}{(a), (\textit{i})-(\textit{vii})} shows the evolution of the drop deformation as it moves downstream after it enters the air stream. It starts with the \textcolor{black}{flattening of the drop \citep{Yang2017, Jain2015} which makes it assume the form of a cylindrical disc of diameter,} \textcolor{black}{$D_{bag}$ (see sections S2 and S3 supplementary section for any details)} later transforming into a thin central circular sheet of thickness, $h_{film}$ bounded by a thicker rim, $h_{rim}$ as depicted in Fig. \ref{Fig1}\textcolor{blue}{(b)}. \textcolor{black}{Experimentally, $D_{bag}$ is measured and identified at the stage when the drop is a near cylindrical disc (see Fig. \ref{Fig1}\textcolor{blue}{(a) -(\textit{ii})}). It can also be demarcated approximately by the inner diameter of the bag once it forms initially. In the immediate moments after the disc structure deforms we see a distinct formation of a rim, at this stage $h_{rim}$ is identified as the thickness of the edge of the deformed drop (see Fig. \ref{Fig1}\textcolor{blue}{(a) -(\textit{iii})}). Note that $D_{bag}$ is entirely different from $D_{max}$ (see Fig. \ref{Fig1}\textcolor{blue}{(b)}) which is the maximum cross stream dimension and attained when the drop expands further radially. Lastly, $h_{film}$ is noted as the thickness of the central film when a distinct rim is formed and can also be approximated by the thickness of the bag when it bursts\citep{Opfer2014} (see Fig. \ref{Fig1}\textcolor{blue}{(a) -(\textit{iv})}).} Such non-uniform flattening to form a rim and a central thin film is characteristic of drop impact phenomena where the surface tension acting along the circumference collects liquid in a bounding rim leaving behind a thin central region. \cite{Fang2022} Analogous to such impact phenomena is the expansion of holes on liquid sheets \cite{Savva2009, Lhuissier2012, Culick1960, Taylor1959} which displays similar features albeit, contrastingly, the expanding rim is surrounded by a thin liquid sheet rather than enclosing it. 
\section*{\normalsize \centering \vspace{-20pt} Interdependence of instabilities}
\vspace{-10pt}
A direct consequence of the separation of length scales from $D_0$ to \textcolor{black}{$D_{bag}$}, $h_{rim}$ and $h_{film}$ is that each of these liquid segments is subjected to Rayleigh-Taylor instability individually, introduced by the same dynamic pressure, $\rho_a U_a^2$ of the oncoming air flow which drives the flattening of the drop with a velocity, $U_l$ developing an inertial stress in the drop of $\rho_l U_l^2$ such that\citep{Opfer2014}, $U_l = U_a \sqrt{\rho_a/\rho_l}$. This means that the accelerations experienced by each of the liquid segments can be deduced from their individual masses as determined by their respective length scales. We exploit this connection between the mass and acceleration in view of the fact that the wavelength of Rayleigh-Taylor instability strongly depends on the imposed acceleration (due to the dynamic air pressure) to establish a unique interdependence between the different wavelengths that develop on the drop, rim and, bag film as shown in Fig \ref{Fig1} \textcolor{blue}{(a)-(d)}.

To begin we notice that as the drop expands radially reaching \textcolor{black}{the cross stream dimension, $D_{bag}$ (while still in its cylindrical disc form)} the acceleration experienced by it in the streamwise direction ($\xi_{bag}$) from an initial zero streamwise (\textit{z}) velocity to the air jet velocity, $U_a$ starts to manifest itself in the form of Rayleigh-Taylor waves, in their \textit{first} form (corresponding to blue arrows in Fig. \ref{Fig1}\textcolor{blue}{(a), (\textit{ii})}). Characteristically, this is seen as a thick rim bounding a thin central film which later bulges into a bag signifying the amplitude of a wave of wavelength, $\lambda_{bag}$ (see Fig. \ref{Fig1}\textcolor{blue}{(a), (\textit{iii}) and (\textit{iv})}). The rim of thickness, $h_{rim}$ formed in the process is radially accelerated (shown by blue arrows in Fig. \ref{Fig1}\textcolor{blue}{(a), (\textit{iii})}) resulting in the emergence of waves of wavelength, $\lambda_{rim}$ (seen more prominently in Fig. \ref{Fig1}\textcolor{blue}{(a), (\textit{vi})} and sketch in Fig. \ref{Fig1}\textcolor{blue}{(d)}) on the periphery of the rim exhibiting the \textit{second} occurrence of Rayleigh-Taylor instability (also see section S2 of the supplementary material for more on experimental measurement details of $\lambda_{rim}$ and $\lambda_{bag}$).

By noting that surface tension enables mode selection in Rayleigh-Taylor instability,\cite{Pfeiffer2020, Zhao2011a,  Zhao2011b, Kulkarni2014a, Kulkarni2013} mathematically the most amplified wavelength in both the above scenarios scales as, $\sqrt{\sigma/\rho_l \xi}$ where $\xi$ (in \textit{m}$\cdot$\textit{s}\textsuperscript{-2}) is the acceleration imparted to the liquid surface. Individually, the accelerations in rim and bag are scale as, $\xi_{rim} \sim \textcolor{black}{U_l}^2/h_{rim}$ and $\xi_{bag} \sim \textcolor{black}{U_l}^2/\textcolor{black}{D_{bag}}$ respectively, using which we can write the following expressions for the most amplified wavelengths, made dimensionless using $D_0$ and represented by overbar $\overline{(\cdot)}$,
\begin{equation} \label{Eq1}
{\overline{\lambda}}_{rim} \sim \sqrt{\frac{h_{rim}}{D_0}}{We^{-1/2}}
\end{equation}
\begin{equation}\label{Eq2}
{\overline{\lambda}}_{bag} \sim \sqrt{\frac{D_{max}}{D_0}}{We^{-1/2}}
\end{equation}
% Figure environment removed
As the bag expands the near-hemispherical curved thin liquid film is accelerated radially (see Fig. \ref{Fig1}\textcolor{blue}{(a), (\textit{iv})}) leading to undulations of wavelength, $\lambda_{film}$ on its surface owing to Rayleigh-Taylor instability, now in its \textit{third} form and \textcolor{black}{experimentally measured just before the bag bursts}(see Fig. \ref{Fig1}\textcolor{blue}{(a), (\textit{v})}). However, unlike the previous occurrences of the instability, here it acts on both the interfaces (inside and outside) of the film with finite thickness, $h_{film}$ approximated as a flat interface since $h_{film} <<  R_{film}$ (see sketch in Fig. \ref{Fig1}\textcolor{blue}{(c)}) and $\lambda_{film} \gtrapprox h_{film}$. The existence of two interfaces stifles the development of the instability and the maximum dimensionless wavelength of instability\cite{Keller1954, Vledouts2016} for a film acceleration\citep{Villermaux2009} given by, $\xi_{film} \sim U_l^2/D_{0}$ now reads,
\begin{equation}\label{Eq3}
{\overline{\lambda}}_{film} \sim \left({\frac{h_{film}}{D_0}}\right)^{-1}{We^{-1}}
\end{equation} 
% Figure environment removed
% Figure environment removed
Since the drop has finite volume which is conserved at all times during its deformation, the distribution of volume in the rim, $\forall_{rim} (\sim h_{rim}^2 \textcolor{black}{D_{bag}}$) and the thin central disc-like sheet which later balloons into a bag, $\forall_{film} (\sim \textcolor{black}{D_{bag}}^2 h_{rim}$) equals the total drop volume, $\forall_{drop} (\sim d_0^3$). As confirmed by our experiments (see Fig. \ref{Fig2} inset), $\forall_{rim} \approx \alpha\forall_{film}$ \textcolor{black}{at all $We$} where, $\alpha \approx 0.5$ in our case and reported to be $\pm 0.2$ of this value elsewhere. Both $\forall_{rim}$ and $\forall_{film}$ \textcolor{black}{are independent\cite{Zhao2011b} of $We$ and recorded immediately when the deformed drop separates into a rim and a thin central film (see Fig. \ref{Fig1}\textcolor{blue}{(a), (\textit{iii})})}. Heuristically, once the thin central region of thickness $h_{film}$ and toroidal rim of thickness, $h_{rim}$ are formed, they thin uniformly such that irrespective of the We, the individual volumes are conserved. We provide further evidence for this in section S3 of supplementary material. Such $We$ independent volume conservation is not uncommon and has been reported recently for drop impacts on solids.\citep{Wang2022} Therefore, we can write, $\forall_{rim} \sim \forall_{film}$ which for $\textcolor{black}{D_{bag}} >> h_{rim}$ can be expanded to, $h_{rim}^2 \textcolor{black}{D_{bag}} \sim \textcolor{black}{D_{bag}^2} h_{film}$ giving rise to the expression,
\begin{equation}\label{Eq3a}
h_{rim} \sim \sqrt{h_{film}\textcolor{black}{D_{bag}}}
\end{equation} 
Using eqn \ref{Eq3a}, we eliminate the different length scales $h_{rim}$, \textcolor{black}{$D_{bag}$} and $h_{film}$ in lieu of the expressions for different $\lambda$ in eqns \ref{Eq1}-\ref{Eq3} to obtain,
\begin{equation}\label{Eq4}
\frac{{{{\overline{\lambda}}_{bag}^2}}}{{\overline{\lambda}\;}_{rim}^4\;{\overline{\lambda}}_{film}} \sim We^2
\end{equation}
Fig. \ref{Fig2} shows the plot for eqn \ref{Eq4} overlayed with the experimental data with an \textcolor{black}{$R^2 \approx 0.97$}. \textcolor{black}{The scaling relation obtained here should be of immense significance to secondary atomization studies and can be extended to higher $Oh$ \citep{Radhakrishna2021} with appropriate modifications accounting for viscous effects, broadening the range of $We$ for which our theory is valid. Similarity, for low viscosity but large drops, $\mathcal{O}(cm)$ our results should be directly applicable. Unfortunately, no experimental data is available for such drops yet but existing numerical work\citep{Jalaal2012} has interestingly reported bag breakup at $We$ as large as 106 for such scenarios which we find to compare well with our findings (see supplementary section S2 for any additional details)}.
\vspace{-18pt}
\section*{\normalsize \centering Mechanism of production of curved film drops}
\vspace{-12pt}
The waves (of wavelength $\lambda_{film}$) mentioned so far do not grow indefinitely but reach their denouement once the curved film disintegrates into smaller drops. As the instabilities grow on this film it experiences thickness modulations along its two interfaces puncturing it at its rightmost tip where it is stretched the most (see Fig. \ref{Fig3}\textcolor{blue}{(a), \textit{t} = 12 and 14 \textit{ms}}). The hole thus formed retracts collecting liquid mass in a ``rolling rim'' \citep{Culick1960, Taylor1959} as it moves along a curved path shown in Fig. \ref{Fig3}\textcolor{blue}{(a), (\textit{iii})} and is accelerated centripetally which results in the emergence of waves \cite{Lhuissier2012} whose crests grow forming equally spaced ligaments culminating with the periodic shedding of droplets (see Fig. \ref{Fig3}\textcolor{blue}{(a), \textit{t} = 17 \textit{ms}} and exploded view).

For predicting the average drop sizes (denoted by $\langle \cdot \rangle$) generated by this mechanism we assume that film drops of average diameter, $\langle{D}_{film}\rangle$ are produced from a ligament whose diameter, $d_{lig}$ equals that of the rolling rim, $d_{roll}$ which when expressed mathematically reads, $\langle{D}_{film}\rangle = d_{lig} = d_{roll}$. Hence, estimating $d_{roll}$ is sufficient to determine the film drop sizes. In order to do so, we assume that the mass accumulated in the rim (per unit width), $\frac{\pi}{4} d_{roll}^2$ in the time it takes to eject a single drop from the ligament, $\tau_{roll}$ is supplied by the flat liquid sheet (since $h_{film} << R_{film}$ ) which continually feeds this rim at a rate, $h_{film}U_{film}$ (see sketch in Fig. \ref{Fig3}\textcolor{blue}{(b), (\textit{i})-(\textit{iii})}). Equating the two expressions gives us, $\langle{D}_{film}\rangle = d_{roll} = \sqrt{(4/\pi) h_{film} U_{film} \tau_{roll}}$. Here, $\tau_{roll} = h_{film}/U_{roll}$ and $U_{roll}$ is the velocity with the rim rolls and $U_{film}$ is the velocity with which the mass enters the rolling rim which are different since the liquid sheet stretches due to the air flow while the hole retracts unlike bursting bubbles.\cite{Lhuissier2012} Therefore, we need to evaluate only, $h_{film}$ and $U_{film}/U_{roll}$ to obtain $\langle{D}_{film}\rangle$ in terms of $We$. 

To this end, we balance the energy (per unit time \textcolor{black}{per unit area assuming unit width}) being provided by the moving sheet \textcolor{black}{of thickness $h_{film}$} written as, $(D_0^2 \rho_l U_{l}^2) U_{film}/h_{film}$ with that required to maintain the surface energy of the rim (\textcolor{black}{per unit time per unit area assuming unit width}), $\sigma U_{roll}$, resulting in the relation, $\frac{U_{film}}{U_{roll}} \sim \frac{\sigma h_{film}}{\rho_a U_a^2 D_0^2} = We^{-1}{\overline{h}}_{film}$, after noting that $\rho_l U_l^2 = \rho_a U_a^2$, as stated previously \textcolor{black}{(for more explanation on the details we refer the reader to section S4 of supplementary material)}. The other unknown ${\overline{h}}_{film}$ can be inferred from the observation that the expansion of the hole driven by the surface tension is resisted by its inertia  \cite{Lhuissier2012, Savva2009} giving rise to, $h_{film} \sim \sigma/\rho_l U_l^2$ which in dimensionless terms is, ${\overline{h}}_{film} \sim We^{-1}$ eventually allowing us to write, 
\begin{equation}\label{Eq5}
\langle\overline{D}_{film}\rangle \sim We^{-2}  
\end{equation}
\section*{\normalsize \centering \vspace{-17pt} Mechanism of production of rim drops}
\vspace{-12pt}
Once the bag fragments completely as described above it leaves behind a destabilized free standing toroidal ring with distortions of wavelength, $\lambda_{rim}$ dictated by Rayleigh-Taylor instability which breaks up into drops owing to capillary (Rayleigh-Plateau) instability \citep{Villermaux2009, Jackiw2022} (see Fig. \ref{Fig3}\textcolor{blue}{(c), \textit{t} = 24, 28} and \textcolor{black}{32 \textit{ms}}). The volume in each wavelength of the straight (since $h_{rim} << R_{max}$) segment, $\lambda_{rim} h_{rim}^2$ comprises of $N (\approx \textrm{4 or 5 at all \textit{We}})$ drops of average diameter, $\langle D_{rim} \rangle$ (see Fig. \ref{Fig3}\textcolor{blue}{(d), (\textit{i}) and (\textit{ii})}). Further, we observe that inertia of the moving rim is opposed by the surface tension\citep{Opfer2014} such that, $h_{rim} \sim \sigma/\rho_l V_l^2$ which in its dimensionless form is, $\overline{h}_{rim} \sim We^{-1}$). Upon substituting for $h_{rim}$ and $\lambda_{rim}$ (from eqn \ref{Eq1}) in the scaling relation for volume conservation, $\langle D_{rim} \rangle \sim (\lambda_{rim} h_{rim}^2)^{1/3} $ we finally obtain,
\begin{equation}\label{Eq6}
\langle\overline{D}_{rim}\rangle \sim We^{-1}  
\end{equation}
The scaling relations derived in eqns \ref{Eq5}, \ref{Eq6} are plotted with the experimental data in Fig. \ref{Fig4} and a good match is seen. 
\vspace{-15pt}
\section*{\normalsize Summary and Conclusions \vspace{-10pt}}
\vspace{0pt}
In conclusion, we have shown that during bag breakup, wavelength of instabilities on the flattened drop, thin film of the bag and the rim are related to each other. We also establish how the air flow affects droplet sizes in the aftermath of these instabilities. Implications of our results go beyond bag breakup, potentially providing crucial insights and points of comparison with several investigations on curved liquid films \cite{Mullagura2020, Naidu2022, Lhuissier2012, Vadivukkarasan2020} and toroidal rims \cite{Mehrabian2013, Pairam2009} which lead to droplet generation. Besides these, an interesting equivalence between studies on drop impact onto a deep liquid pool \citep{Kulkarni2021} to bag breakup has been propounded where the liquid and gas phases are inverted\citep{Opfer2014}, here too our findings could be of relevance. Given these similarities, our work should benefit industrial applications and fundamental studies alike.
\vspace{-20pt}
\section*{\normalsize Supplementary Material}
\vspace{-10pt}
The supplementary material accompanying this manuscript contains (\textit{i}) additional details leading to the scaling relations, (\ref{Eq4}), (\ref{Eq5}) and (\ref{Eq6}), (\textit{ii}) further justification for $We$ independent conservation of volumes of bag and rim and, (\textit{iii}) image processing steps followed to obtain rim and bag drop sizes and validation of those measurements.
\vspace{-20pt}
\section*{\normalsize Acknowledgment}
\vspace{-10pt}
Financial support for this project through the US Army Research office (ARO) under the Multi University Research Initiative (MURI) award number W911NF-08-l-0171 is gratefully acknowledged.
\vspace{-20pt}
\section*{\normalsize Data Availability Statement}
\vspace{-10pt}
Data underlying the conclusions of the paper are available in the plots presented and can be provided upon request.
\vspace{0pt}
\section*{\normalsize Conflicts of Interest Statement}
\vspace{-10pt}
The authors have no conflicts of interests to disclose.
\vspace{-20pt}
\section*{\normalsize References}
\vspace{-20pt}
\bibliographystyle{apsrev4-2}
\bibliography{APLRefs}
\end{document}
%
% ****** End of file aipsamp.tex ******