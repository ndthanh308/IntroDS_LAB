\documentclass[a4paper,11pt,oneside]{article}

\usepackage{amsmath,amsthm,amsfonts,amssymb,amscd}
\usepackage{mathtools}
\usepackage{float}
\usepackage{subfig} % for subplots
\usepackage{enumerate}
\usepackage{booktabs}
\usepackage[noend]{algpseudocode}
\usepackage[normalem]{ulem}
\usepackage{graphicx,bm,xcolor}
\usepackage{algorithm}
\usepackage{algpseudocode}
%\usepackage{natbib}
\usepackage[hidelinks]{hyperref}
\usepackage{caption}
\usepackage[T1]{fontenc}
\usepackage{t1enc}
\usepackage[polish,german,english]{babel}
\usepackage{verbatim}
\usepackage{diagbox}
\usepackage{colortbl}
\usepackage{siunitx}

\usepackage{url}
\usepackage{authblk}

\usepackage{pgf,tikz,pgfplots}
\usepackage{mathrsfs}
\usetikzlibrary{arrows}
\usetikzlibrary{shapes.misc}
\usetikzlibrary{positioning}
\usetikzlibrary{decorations.pathreplacing} % curly braces

%https://tex.stackexchange.com/questions/123760/draw-crosses-in-tikz/124064
\tikzset{cross/.style={cross out, draw=black, minimum size=2*(#1-\pgflinewidth), inner sep=0pt, outer sep=0pt},
%default radius will be 1pt. 
cross/.default={1pt}}

\usepackage{tikz-cd}
\usepackage{svg}

\algnewcommand{\algorithmicgoto}{\textbf{go to}}
\algnewcommand{\Goto}[1]{\algorithmicgoto~\ref{#1}}

% handling todos
\usepackage[colorinlistoftodos,prependcaption,textsize=scriptsize,color=red!40!white]{todonotes}\usepackage{totcount}
\newcommand{\fasttodo}[1]{\todo[inline]{#1}}
\newcommand{\edit}[1]{\todo[inline,color=green!20!white]{#1}}
\newcommand{\warn}[1]{\todo[inline,color=orange!20!white]{#1}}
\newcommand{\question}[1]{\todo[inline,color=blue!20!white]{#1}}
\newcommand{\questiontw}[1]{\todo[inline,color=blue!20!white]{\underline{@T.W.:} #1}}
\newcommand{\anttw}[1]{\todo[inline,color=blue!20!white]{\underline{Antwort TW:} #1}}
\newcommand{\questiontr}[1]{\todo[inline,color=blue!20!white]{\underline{@T.R.:} #1}}
\newcommand{\todojr}[1]{\todo{\underline{@Julian:} #1}}
\newcommand{\antjr}[1]{\todo[inline,color=blue!20!white]{\underline{Antwort JR:} #1}}
\newcommand{\todoms}[1]{\todo{\underline{@Martyna:} #1}}

% define colors wrt color blindness (paper of Bang Wong)
% DO NOT CHANGE!
\definecolor{orange}{RGB}{230, 159, 0}
\definecolor{skyblue}{RGB}{86, 180, 233}
\definecolor{yellow}{RGB}{240, 228, 66}
\definecolor{blue}{RGB}{0, 114, 178}
\definecolor{vermillion}{RGB}{213, 94, 0}

\definecolor{brewer1}{HTML}{A6CEE3}
\definecolor{brewer2}{HTML}{1F78B4}
\definecolor{brewer3}{HTML}{B2DF8A}
\definecolor{brewer4}{HTML}{33A02C}

% comments for different authors
\newcommand{\msoszynska}[1]{{\color{orange} #1}}
\newcommand{\jroth}[1]{{\color{skyblue} #1}}
\newcommand{\twick}[1]{{\color{purple} #1}}
\newcommand{\trichter}[1]{{\color{blue} #1}}

% new changes to manuscript after Thomas Richter's comments
\newcommand\auth{\textcolor{brown}}
\newcommand\revrichter{\textcolor{magenta}}
%%%%%%%%%%%%%%%%%%%%%%%%%%%%%%%%%%%%%%%%%%%%%%%%%%%%%%%%%%%%%%%%%%%%%%%%%%%%%%%%

\usepackage[top=2cm, bottom=2cm, left=2cm, right=2cm]{geometry}
%\setlength{\parindent}{0mm}
%\setlength{\parskip}{2.0mm}
\renewcommand{\baselinestretch}{1.36}

%%%%%%%%%%%%%%%%%%%%%%%%%%%%%%%%%%%%%%%%%%%%%%%%%%%%%%%%%%%%%%%%%%%%%%%%%%%%%%%%


\theoremstyle{plain} % default
\newtheorem{theorem}{Theorem}[section]
\newtheorem{propositon}{Proposition}[section]
\newtheorem{lemma}[theorem]{Lemma}
\newtheorem{corollary}[theorem]{Corollary}
\newtheorem{remark}[theorem]{Remark}
\newtheorem{definition}[theorem]{Definition}
\newtheorem{example}[theorem]{Example}
\newtheorem{problem}[theorem]{Problem}
\newtheorem{form}[theorem]{Formulation}

\theoremstyle{definition} %
\newtheorem{assumption}{Assumption}
\newtheorem{claim}{Claim}
\newtheorem{algo}{Algorithm}
\newtheorem*{condition}{Condition}

\theoremstyle{remark} %
\newtheorem*{notation}{Notation}

\DeclareMathOperator*{\dG}{dG}
\DeclareMathOperator*{\cG}{cG}
\DeclareMathOperator*{\Ieff}{I_{\text{eff}}}

\newcommand\restrict[1]{\raisebox{-.5ex}{$|$}_{#1}}

\newcommand{\trialc}[1]{{{\textcolor{blue!60!black}{#1}}}}
\newcommand{\testc}[1]{{{\textcolor{green!40!black}{#1}}}}
\newcommand{\puc}[1]{{{\textcolor{brown!40!black}{#1}}}}

\definecolor{brewer1}{HTML}{A6CEE3}
\definecolor{brewer2}{HTML}{1F78B4}
\definecolor{brewer3}{HTML}{B2DF8A}
\definecolor{brewer4}{HTML}{33A02C}

%%%%%%%%%%%%%%%%%%%%%%%%%%%%%%%%%%%%%%%%%%%%%%%%%%%%%%%%%%%%%%%%%%%%%%%%%%%%%%%%
\begin{document}

%\listoftodos

\title{
A monolithic space-time temporal multirate finite element framework
for interface and volume coupled problems
}
\author[1,2]{Julian Roth}
\author[3]{Martyna Soszy\'nska}
\author[3]{Thomas Richter}
\author[1,2,4]{Thomas Wick}


\affil[1]{Leibniz Universit\"at Hannover, Institut f\"ur Angewandte
  Mathematik, Welfengarten 1, 30167 Hannover, Germany}
\affil[2]{Universit\'e Paris-Saclay, LMPS - Laboratoire de Mecanique Paris-Saclay,
91190 Gif-sur-Yvette, France}
\affil[3]{Otto-von-Guericke Universit\"at Magdeburg, Universit\"atsplatz 2, 39106 Magdeburg, Germany}
\affil[4]{Cluster of Excellence PhoenixD (Photonics, Optics, and
Engineering - Innovation Across Disciplines), Leibniz University Hannover, Germany}

\date{}
\maketitle

\setcounter{page}{1}

%%%%%%%%%%%%%%%%%%%%%%%%%%%%%%%%%%%%%%%%%%%%%%%%%%%%%%%%%%%%
%%                     ABSTRACT                           %%
%%%%%%%%%%%%%%%%%%%%%%%%%%%%%%%%%%%%%%%%%%%%%%%%%%%%%%%%%%%%
\begin{abstract}
In this work, we propose and computationally investigate a monolithic space-time 
multirate scheme for coupled problems. The novelty lies in the monolithic 
formulation of the multirate approach as this requires a careful design of the functional framework, corresponding discretization, and implementation.
Our method of choice is a tensor-product Galerkin space-time discretization. 
The developments are carried out for both prototype interface- and volume coupled problems such as coupled wave-heat-problems and a displacement equation 
coupled to Darcy flow in a poro-elastic medium. The latter is applied to 
the well-known Mandel's benchmark. Detailed computational investigations 
and convergence analyses give evidence that our monolithic multirate framework 
performs well.\\
\textbf{Keywords:} %multirate, temporal discretization, Galerkin space-time, 
%wave equation, heat equation, Mandel benchmark\\
Galerkin space-time ; multirate ; monolithic framework ; interface coupling ; volume coupling ; Mandel's benchmark
\textbf{MSC2020:} 65M50, 65M60, 74J05, 76S05
\end{abstract}

%%%%%%%%%%%%%%%%%%%%%%%%%%%%%%%%%%%%%%%%%%%%%%%%%%%%%%%%%%%%
%%                  INTRODUCTION                          %%
%%%%%%%%%%%%%%%%%%%%%%%%%%%%%%%%%%%%%%%%%%%%%%%%%%%%%%%%%%%%
\section{Introduction}
\label{sec_intro}

\noindent This work is devoted to space-time coupled problems with a
monolithic numerical solution using different temporal meshes, e.g. different time step sizes for each subproblem. Such schemes using different time meshes
for the subproblems are known as multirate approaches.
Multirate schemes are well-known specifically in ordinary differential equations \cite{Logg2003a,Logg2003b,SavHuVe07,SAVCENCO2009323}, porous media 
where Darcy flow and geomechanics couple
\cite{Dean2006,Almani2016,Bause2017,RyMaHeRoh15,ALMANI20192682,Almani2016_phd,RYBAK2014327}, phase-field fracture porous media \cite{JaWheWi21,AlLeeWheWi17,WheWiLee20},
Stokes-Darcy coupling \cite{ShZhLa13}, wave equation and Richardson equation \cite{SOCHALA20092122}, advection equations \cite{SCHLEGEL2009345}, and coupled flow and transport \cite{BrKoeBau22}. In \cite{Soszynska2021,Soszynska_phd}, 
the coupling of heat and wave and thermoelasticity was investigated, and therein
the time meshes are selected using a posteriori error control using
the dual-weighted residual method \cite{BeRa01}.

To the best of our knowledge, in all these studies, partitioned approaches, e.g. \cite{Dean2006,JaWheWi21,Almani2016} or one-way coupling techniques \cite{BrKoeBau22} were employed. This is reasonable since here multirate iterative 
coupling procedures allow using different programming codes for the subproblems 
and employing multirate schemes can yield very efficient numerical solution 
methods. On the other hand, 
monolithic methods are well-known for their robustness, and if preconditioners 
are available, they can also be more efficient than partitioned approaches. 
Such a monolithic scheme with a focus on the temporal scales 
(not yet multirate though)
for a challenging coupled problem, namely fluid-structure interaction, was proposed in \cite{MaKlWaGe15}. 

The governing model and discretization are based on space-time 
schemes \cite{LaStein19,TeBeLi92,TeBeMiJo92,BaGeRa10,SchVe08,RoThiKoeWi2023,DoeWie22} in which both the temporal and the spatial parts are discretized with 
Galerkin finite elements.
Specifically, in this work we are interested in tensor-product 
space-time finite element discretizations. Here, for efficiency reasons, the problems 
are solved on time-slabs (for the terminology time-slab; see \cite{HuHu88,JOHNSON1993117}), e.g. $(0, T_1), (T_1, T_2), ..., (T_{l-1}, T_l)$, where $T_l:= T$, rather than the entire time interval $(0,T)$.


The key objective of this work is to design and computationally investigate a 
space-time multirate framework in which the coupled system is treated in a 
variational-monolithic fashion. Therein, the coupling can be of interface 
type or achieved via volume coupling. An example for interface coupling 
is fluid-structure interaction \cite{BaKeTe13,Ri17_fsi} and an example 
for volume coupling are porous media equations \cite{Coussy2004} or phase-field fractured porous media \cite{Wick2020PFF}.
Our proposed concepts can deal with both situations. In order to develop and 
analyze our concepts, we however concentrate on prototype settings in the 
interface couplings such as a (linear) heat-wave equation system and 
a linearly coupled poro-elasticity problem, the so-called Biot equations 
\cite{Biot41a,Biot41b,Biot55,Biot7172,To92}.
One of the main tasks is to design functional frameworks and corresponding 
space-time discretizations for such multirate formulations. Here, the key 
is the correct prescription of coupling conditions within the function spaces
and their resulting discrete forms and the corresponding implementation. 
To the best of our knowledge such monolithic space-time multirate frameworks 
are novel. The systems are substantiated with various numerical tests 
for spatial 1d heat-wave equations, spatial 2d heat-wave equations, and the Biot 
system solved on the Mandel benchmark problem. The performances are evaluated 
in terms of goal functionals, their accuracy, and the cost complexity in terms 
of the required space-time degrees of freedom.

The outline of this paper is as follows. In Section \ref{sec_methodology}, 
the monolithic space-time methodology is introduced for abstract coupled problems. 
Next in Section \ref{sec_model_problems}, our two model problems are introduced, namely 
the interface-coupled heat-wave system, and secondly, the volume-coupled poro-elasticity 
problem. In Section \ref{sec_numerical_tests} three numerical tests are conducted.
Finally, in Section \ref{sec_conclusions}, our work is summarized.


%%%%%%%%%%%%%%%%%%%%%%%%%%%%%%%%%%%%%%%%%%%%%%%%%%%%%%%%%%%%
%%                  METHODOLOGY                           %%
%%%%%%%%%%%%%%%%%%%%%%%%%%%%%%%%%%%%%%%%%%%%%%%%%%%%%%%%%%%%
\section{Monolithic space-time multirate methodology}
\label{sec_methodology}
In this work, we consider the abstract interface coupled multiphysics problem
\begin{equation}\label{eq_abstract_interface_problem}
\begin{aligned}
    \mathcal{A}_1(u_1) &= f_1 \phantom{0}\qquad\quad\text{in } \Omega_1 \times I, \\
    \mathcal{A}_2(u_2) &= f_2 \phantom{0}\qquad\quad\text{in } \Omega_2  \times I, \\
    \mathcal{B}(u_1, u_2) &= 0 \phantom{f_1}\qquad\quad\text{in } \Gamma  \times I,
\end{aligned}
\end{equation}
with $u_1: \bar{\Omega}_1 \times \bar{I} \rightarrow \mathbb{R}^{d_1}$, $u_2: \bar{\Omega}_2 \times \bar{I} \rightarrow \mathbb{R}^{d_2}$,
and the abstract volume coupled multiphysics problem
\begin{equation}\label{eq_abstract_volume_problem}
\begin{aligned}
    \mathcal{A}_1(u_1, u_2) = f_1  \qquad\quad \text{in } \Omega \times I, \\
    \mathcal{A}_2(u_1, u_2) = f_2 \qquad\quad \text{in }  \Omega \times I,
\end{aligned}
\end{equation}
with $u_1: \bar{\Omega} \times \bar{I} \rightarrow \mathbb{R}^{d_1}$, $u_2: \bar{\Omega} \times \bar{I} \rightarrow \mathbb{R}^{d_2}$, and $d_1, d_2 \in \mathbb{N}$ depending on whether the solution fields are scalar-valued ($d_i = 1$) or vector-valued ($d_i > 1$).
Herein, $\mathcal{A}_1$ and $\mathcal{A}_2$ are possibly nonlinear spatio-temporal operators, $\mathcal{B}$ is the interface coupling operator, and $f_1, f_2$ are sufficiently regular right hand side functions. 
Additionally, these problems need to be completed by suitable initial and boundary conditions. The temporal domain is denoted by $I := (0,T)$ and the spatial domain $\Omega \subset \mathbb{R}^d$ with $d \in \{1,2,3\}$ decomposes for the interface coupled problem into the disjoint subdomains $\Omega_1$ and $\Omega_2$ with common interface $\Gamma := \bar{\Omega}_1 \cap \bar{\Omega}_2$.
\begin{remark}
 The methodology proposed in this paper can also be applied to multiphysics problems that contain both interface and volume coupling.
\end{remark}
Choosing a suitable continuous spatial function space $V(\Omega) = V^1(\Omega_1) \times V^2(\Omega_2)$ or $V(\Omega) = V^1(\Omega) \times V^2(\Omega)$, we define the space-time function space $X(I, V(\Omega))$ as
\begin{align*}
    X(I, V(\Omega)) := L^2(I, V(\Omega)) \cap H^1(I, V^\ast(\Omega))
\end{align*}
with the dual space of $V(\Omega)$ being denoted as $V^\ast(\Omega) = L(V(\Omega), \mathbb{R})$. Thus, for the abstract interface and volume coupled problems, we get a continuous spatio-temporal variational formulation:
    Find $U := \begin{pmatrix}
        u_1 \\ u_2
    \end{pmatrix} \in X(I, V(\Omega))$ such that
    \begin{align*}
        A(U)(\Phi) = F(\Phi) \qquad \forall \Phi := \begin{pmatrix}
        \Phi_1 \\ \Phi_2
        \end{pmatrix} \in L^2(I, V(\Omega)).
    \end{align*}
The weak space-time formulations of two model problems for interface and volume coupling will be described in Section \ref{sec_model_problems}.

In the following, we will use the notation
\begin{align*}
    (f,g) := (f,g)_{L^2(\Omega)} := \int_\Omega f \cdot g\ \mathrm{d}x, \qquad (\!(f,g)\!) := (f,g)_{L^2(I, L^2(\Omega))} := \int_I (f, g)\ \mathrm{d}t, \\
    \langle f,g \rangle := \langle f,g \rangle_{L^2(\Gamma)} := \int_\Gamma f \cdot g\ \mathrm{d}s, \qquad (\!\langle f,g\rangle\!) := (f,g)_{L^2(I, L^2(\Gamma))} := \int_I \langle f, g\rangle\ \mathrm{d}t.
\end{align*}
In this notation, $f \cdot g$ represents the Euclidean inner product if $f$ and $g$ are scalar- or vector-valued and represents the Frobenius inner product if $f$ and $g$ are matrices.

\subsection{Discretization in time and space}
\label{sec_discretization_space_time}
%\todo{INFO: Ich habe hier einige Formeln wegen der Lesbarkeit abgesetzt.} --> Gut.
Let the common coarse mesh 
\[
\mathcal{T}_k^{\text{coarse}} := \left\lbrace I_m^{\text{coarse}} := (t_{m-1}^{\text{coarse}}, t_m^{\text{coarse}}) \mid 1 \leq m \leq M^{\text{coarse}} \right\rbrace
\]
be a partitioning of time, i.e.  $\bar{I} = [0,T] = \bigcup_{m = 1}^{M^{\text{coarse}}} \bar{I}_m^{\text{coarse}}$, from which the 
temporal meshes $\mathcal{T}_k^1$ and $\mathcal{T}_k^2$ for the subproblems originate and are being created by local refinement. 
Consequently, 
\[
\mathcal{T}_k^1 := \left\lbrace I_m^1 := (t_{m-1}^1, t_m^1) \mid 1 \leq m \leq M^1 \right\rbrace
\]
and 
\[
\mathcal{T}_k^2 := \left\lbrace I_m^2 := (t_{m-1}^2, t_m^2) \mid 1 \leq m \leq M^2 \right\rbrace
\]
are also partitionings of time for the individual subproblems, i.e.  
\[
\bar{I} = [0,T] = \bigcup_{m = 1}^{M^1} \bar{I}_m^1 = \bigcup_{m = 1}^{M^2} \bar{I}_m^2, 
\]
with $M^i$ being the number of temporal elements for the solution variable $u^i$.
Additionally, we define by 
\begin{align*}
    \mathcal{T}_k^{\text{fine}} := \mathcal{T}_k^1 \cap \mathcal{T}_k^2 := \left\lbrace I_m^1 \cap I_m^2 \mid  I_m^1 \in \mathcal{T}_k^1,\ I_m^2 \in \mathcal{T}_k^2 \right\rbrace    
\end{align*}
the common fine temporal mesh.
These definitions can be visualized by the example in Figure \ref{fig:temporal_mesh_definitions}.
% Figure environment removed
\noindent Then as an intermediate step for discontinuous in time function spaces, we introduce the broken continuous level function space
{\small
\begin{align*}
    \tilde{X}((\mathcal{T}_k^1, \mathcal{T}_k^2), V(\Omega)) := \left\lbrace \begin{pmatrix}
    u_1 \\ 
    u_2 
    \end{pmatrix} \in L^2(I, L^2(\Omega)^{d_1+d_2})\middle| u_i\restrict{I_{m}^i} \in X(I_{m}^i, V^i(\Omega)),\  I_{m}^i \in \mathcal{T}_k^i,\ i \in \{1,2\} \right\rbrace \supset X(I, V(\Omega)),
\end{align*}
}
which allows for discontinuities of the solution between temporal elements of $\mathcal{T}_k^1$ and $\mathcal{T}_k^2$.
 By introducing jump terms between temporal elements, we can then define an abstract discontinuous spatio-temporal variational formulation:
    Find $U := \begin{pmatrix}
        u_1 \\ u_2
    \end{pmatrix} \in \tilde{X}((\mathcal{T}_k^1, \mathcal{T}_k^2), V(\Omega)) $ such that
    \begin{align*}
        \tilde{A}(U)(\Phi) = \tilde{F}(\Phi) \qquad \forall \Phi := \begin{pmatrix}
        \Phi_1 \\ \Phi_2
        \end{pmatrix} \in L^2(I, V(\Omega)).
    \end{align*}
We can thus define semi-discrete continuous ($\cG$) and discontinuous ($\dG$) in time function spaces by 
\begin{align*}
    X_k^{\cG(r)}((\mathcal{T}_k^1, \mathcal{T}_k^2), V(\Omega)) := \left\lbrace \begin{pmatrix}
    u_1 \\ 
    u_2 
    \end{pmatrix} \in C(\bar{I}, L^2(\Omega)^{d_1+d_2})\middle| u_i\restrict{I_{m}^i} \in P_r(I_{m}^i, V^i(\Omega)),\  I_{m}^i \in \mathcal{T}_k^i,\ i \in \{1,2\} \right\rbrace ,
\end{align*}
and
\begin{align*}
    X_k^{\dG(r)}((\mathcal{T}_k^1, \mathcal{T}_k^2), V(\Omega)) := \left\lbrace \begin{pmatrix}
    u_1 \\ 
    u_2 
    \end{pmatrix} \in L^2(\bar{I}, L^2(\Omega)^{d_1+d_2})\middle| u_i\restrict{I_{m}^i} \in P_r(I_{m}^i, V^i(\Omega)),\  I_{m}^i \in \mathcal{T}_k^i,\ i \in \{1,2\} \right\rbrace.
\end{align*}
Here, $P_r(I_m, Y)$ is the space of polynomials of order $r$, which map from the time interval $I_m$ into the space $Y$.
Finally, we replace the continuous spatial function space $V(\Omega) = V^1(\Omega) \times V^2(\Omega)$ by conforming finite element subspaces $V_h^1(\mathcal{T}_h^1) \subset V^1(\Omega)$ and $V_h^2(\mathcal{T}_h^2) \subset V^2(\Omega)$, where we define $V_h := V_h(\mathcal{T}_h) := V_h^1(\mathcal{T}_h^1) \times V_h^2(\mathcal{T}_h^2)$. Then the fully discrete function spaces are
\begin{align*}
    X_k^{\cG(r)}\left((\mathcal{T}_k^1, \mathcal{T}_k^2), V_h(\mathcal{T}_h)\right) \subset X_k^{\cG(r)}((\mathcal{T}_k^1, \mathcal{T}_k^2), V(\Omega))
\end{align*}
and 
\begin{align*}
    X_k^{\dG(r)}\left((\mathcal{T}_k^1, \mathcal{T}_k^2), V_h(\mathcal{T}_h)\right) \subset X_k^{\dG(r)}((\mathcal{T}_k^1, \mathcal{T}_k^2), V(\Omega)).
\end{align*}
Note that in this work we restrict ourselves to spatial meshes $\mathcal{T}_h^1$ and $\mathcal{T}_h^2$ that are constant in time, but with a few minor modifications, this framework can be extended to dynamical spatial meshes which change for each coarse temporal element $I_m \in \mathcal{T}_k^{\text{coarse}}$.

\subsection{Monolithic multirate time marching scheme}
\label{sec_multirate_tms}
Using suitable finite element trial and test spaces, many time marching schemes can be interpreted as variants of Galerkin time discretizations, e.g. the backward Euler method is equivalent to a $\dG(0)$ time 
discretization under certain assumptions for the quadrature formulae \cite{DeHaTr81,Joh88}.
Similarly, the Crank-Nicolson method can be interpreted as a variant of the $\cG(1)$ scheme.
Moreover, we notice that the Fractional-Step-$\theta$ scheme can be formulated as a Galerkin scheme \cite{MeidnerRichter2014}.
In this work, we will discuss how these time marching approaches can be used and extended to solve multirate in time formulations of multiphysics problems.
We will explain the methodology with an example of the backward Euler scheme: 
Find $U_{kh} \in X_k^{\dG(0)}\left((\mathcal{T}_k^1, \mathcal{T}_k^2), V_h(\mathcal{T}_h)\right)$ such that
\begin{align}\label{eq_decomposition_bilinear_form}
    \tilde{A}(U_{kh})(\Phi_{kh})  = \tilde{F}(\Phi_{kh}) \qquad \forall \Phi_{kh} \in X_k^{\dG(0)}\left((\mathcal{T}_k^1, \mathcal{T}_k^2), V_h(\mathcal{T}_h)\right).
\end{align}
Assuming that the PDE is linear, the bilinear form on the left side of this equation can be decomposed into
\begin{equation*}
\tilde{A}(U_{kh})(\Phi_{kh}) = A_1(U_{kh}^1)(\Phi_{kh}^1) + A_2(U_{kh}^2)(\Phi_{kh}^2) + B_1(U_{kh}^2)(\Phi_{kh}^1) + B_2(U_{kh}^1)(\Phi_{kh}^2),
\end{equation*}
where $A_1$ and $A_2$ are bilinear forms for the subproblems and $B_1$ and $B_2$ are bilinear forms containing the coupling terms.
% TW, Jul 10, 2023 in response to the B terms
Here, we emphasize that $B_1$ and $B_2$ can include both interface couplings or volume couplings. This should not be confused with the operator $\mathcal{B}$ that has been 
introduced for interface couplings in \eqref{eq_abstract_interface_problem} only, but does appear implicitly in the volume coupled problem \eqref{eq_abstract_volume_problem} as well (cf. Section \ref{sec_model_problems}).
From the definition of our time mesh structure, we have
$
\mathcal{T}_k^{\text{coarse}} \subset \mathcal{T}_k^1, \mathcal{T}_k^2 \subset \mathcal{T}_k^{\text{fine}}
$. We use this hierarchy to establish a method that borrows from both standard time marching methodology as well as space-time approach. Let us assume some fixed $I_{n-1}, I_{n}, I_{n + 1} \in \mathcal{T}_k^{\text{coarse}}$.
Since the temporal meshes $\mathcal{T}_k^1$ and $\mathcal{T}_k^2$ originate from the common coarse mesh through adaptive refinement, for each coarse element $I_n$ we can find elements $I_{n_1}^1, ..., I_{n_{N_n^1}}^1 \in \mathcal{T}_k^1$ and $I_{n_1}^2, ..., I_{n_{N_n^2}}^2 \in \mathcal{T}_k^2$ such that
\begin{equation}
\bar{I}_n = \bigcup_{N = 1}^{N_n^1} \bar{I}_{n_N}^1 = \bigcup_{N = 1}^{N_n^2} \bar{I}_{n_N}^2.
\label{time_partition}
\end{equation}
Our approach is based on simultaneously solving all of the equations given by the partition~(\ref{time_partition}) while using the last available solution from $I_{n-1}$ with respect to either $\mathcal{T}_k^1$ or $\mathcal{T}_k^2$ as the initial value. The time interpolation necessary to compute coupling conditions at temporal hanging nodes, i.e.  degrees of freedom that are contained in $\mathcal{T}_k^1$ and not in $\mathcal{T}_k^2$ or vice versa, is given by 
\begin{equation}
\bar{u}_i \Big{|}_{I_n} \coloneqq \frac{1}{|I_n|}\int_{I_n} u_i \ \mathrm{d}t\hspace*{0.5 cm} \text{for }i\in\{1,2\}.
\end{equation}
This approach does not require any modification of the scheme itself. The only difference between traditional time marching schemes and the scheme proposed here, is the order of solving the systems. In conventional schemes, all of the systems are solved sequentially. Here, systems corresponding to one time slab are solved simultaneously. To transition from a standard time marching scheme to this approach, for each time slab one has to build a matrix shown in Figure~\ref{multiblock}. Block $A_1$ represents all the unknowns in the first system without the coupling contributions. $A_2$ is a similar block with all of the unknowns for the second systems. In $B_1$ we have the coupling conditions from the first system and therefore the test functions align with $A_1$ and the trial functions correspond to $A_2$. In $B_2$ we have the coupling conditions of the second system. 


The method provides flexibility and allows for discretization of different terms using separate time methods. Although the size of the system significantly increases, one does not require any subcycling to resolve coupling conditions. Moreover, the resulting scheme is fully implicit which is highly desirable for stiff systems of equations \cite{aiken1985stiff, Hairer1996}. We solve the space-time linear system monolithically with the direct solver UMFPACK \cite{UMFPACK}, but the monolithic system could also be decoupled on the solver level and solved as a partitioned scheme with some iterative solver or some multigrid method; e.g., \cite{Saad2003}.

% Figure environment removed

\subsection{Temporal multirate tensor-product space-time FEM}
\label{sec_temporal_multirate_tp_st_fem}
Another alternative for solving the variational formulation of the multiphysics problem is using tensor-product space-time finite elements as in \cite{BrKoeBau22, RoThiKoeWi2023, FiRoWiChaFau2023}. 
The core idea is that a space-time FEM basis can be created by taking the tensor-product of the spatial and the temporal finite element basis.
Additionally, for efficiency reasons the variational problem does not need to be solved all-at-once on the space-time cylinder $\Omega \times I$, but will be solved forward in time on space-time slabs
\begin{align*}
    S_m := \Omega \times I_m, \qquad I_m \in \mathcal{T}_k^{\text{coarse}}.
\end{align*}
We then have one temporal element for one subproblem and $n \geq 1$ temporal elements for the other subproblem on a given slab $S_m$.
Using this tensor-product space-time FEM discretization, it is fairly straightforward to assemble terms of the variational formulation with trial and test functions from the same subproblem, i.e. the blocks $A_1$ and $A_2$ in Figure \ref{multiblock}.
However, the blocks $B_1$ and $B_2$ with coupling between the subproblems are more complicated, since they require the evaluation of temporal intervals with trial and test functions that belong to different temporal triangulations.
We discuss this problem by considering the interface coupling condition
$\mathcal{B}(u_1, u_2) = u_1 - u_2 = 0$ in $\Gamma  \times I$.
Using penalization to enforce the interface conditions though the weak formulation, we need to be able to assemble terms of the sort $\gamma(\!\langle u_2-u_1,\Phi_1\rangle\!)$ for the linear system. 
Note that the only difficulty lies in the assembly of $(\!\langle u_2,\Phi_1\rangle\!)$.
In the matrix, we thus need to be able to evaluate integrals like $(\!\langle \Phi_2,\Phi_1\rangle\!)$ with basis functions from both subproblems,
which using the tensor-product structure of $\Phi_1(x,t) = \Phi_h^1(x) \cdot \Phi_k^1(t)$ and $\Phi_2(x,t) = \Phi_h^2(x) \cdot \Phi_k^2(t)$ can be rewritten as 
\begin{align*}
    \langle\!\langle \Phi_2,\Phi_1\rangle\!\rangle = \left(\int_\Gamma \Phi_h^2(x) \cdot \Phi_h^1(x)\ \mathrm{d}x \right) \cdot \left(\int_I \Phi_k^2(t) \cdot \Phi_k^1(t)\ \mathrm{d}t \right).
\end{align*}
The spatial integral over the interface $\Gamma$ can be easily evaluated with most FEM packages. In particular, the evaluation of the interface integral can be found in step-46 of the deal.II \cite{dealii2019design, dealII94} tutorials.
Therefore, it only remains to be discussed how the temporal integral $\int_I \Phi_k^2(t) \cdot \Phi_k^1(t)\ \mathrm{d}t$ can be computed, unless the finite element library supports non-matching one dimensional meshes and finite element evaluations.
We will briefly explain the general approach, which does not require non-matching grid support for the temporal triangulations $\mathcal{T}_k^1 = \{ (a,b) \}$ and $\mathcal{T}_k^2 = \{ (a,\frac{a+b}{2}), (\frac{a+b}{2},b) \}$ and $\cG(1)$ basis functions.
% Figure environment removed
\noindent For this example, the matrix corresponding to the term $\langle\!\langle \Phi_2,\Phi_1\rangle\!\rangle$ is thus given by
\begin{align}\label{eq:tensor_product_coupling_matrix}
    \left(M_k R^T\right) \otimes M_h,
\end{align}
where $\otimes$ denotes the Kronecker product, i.e. for $A \in \mathbb{R}^{m \times n}$ and $B \in \mathbb{R}^{p \times q}$ we have
\begin{align*}
    A \otimes B :=  \begin{pmatrix}
        A_{11}B & \cdots & A_{1n}B \\
        \vdots & \ddots & \vdots \\
        A_{m1}B & \cdots & A_{mn}B
    \end{pmatrix} \in \mathbb{R}^{mp \times nq}.
\end{align*}
Furthermore, we have the spatial interface mass matrix
\begin{align*}
    M_h = \left\lbrace \int_\Gamma \Phi_h^{2,(j)}(x) \cdot \Phi_h^{1,(i)}(x)\ \mathrm{d}x 
 \right\rbrace_{1 \leq i \leq \# \text{DoFs}(\mathcal{T}_h^1),\ 1 \leq j \leq \# \text{DoFs}(\mathcal{T}_h^2)},
\end{align*}
the temporal mass matrix on the finer mesh
\begin{align*}
    M_k = \left\lbrace \int_I \Phi_k^{2,(j)}(t) \cdot \Phi_k^{2,(i)}(t)\ \mathrm{d}t 
 \right\rbrace_{1 \leq i,j \leq 3},
\end{align*}
and 
\begin{align*}
    R = \begin{pmatrix}
        1 & \frac{1}{2} & 0 \\
        0 & \frac{1}{2} & 1
    \end{pmatrix},
\end{align*}
which is the restriction matrix from the fine temporal basis $\Phi_k^{2}$ to the coarse temporal basis $\Phi_k^{1}$, i.e.
\begin{align*}
    \Phi_k^1 = R\Phi_k^2.
\end{align*}
This procedure can then also be extended to other temporal discretizations $\mathcal{T}_k^1$ and $\mathcal{T}_k^2$, as well as other types of $\cG(r)$ or $\dG(r)$ time discretizations, and is possibly already contained in most FEM libraries with multigrid capabilities. 

\begin{remark}
    Although the interface coupling matrix in (\ref{eq:tensor_product_coupling_matrix}) has been depicted as a tensor-product of a spatial and a temporal matrix, this is not actually performed in our implementation. Instead, the local space-time matrices are being assembled with the finer temporal basis and the restriction matrix is being applied prior to the distribution of the local contributions to the global matrix. This could possibly be applicable to a matrix-free space-time implementation as well and is also required for e.g. nonlinear problems when the system matrix cannot be formulated as the linear combination of tensor-products of temporal and spatial finite element matrices.
\end{remark}
\begin{remark}
    We also notice that a conceptionally similar approach has been proposed for ODEs 
    in \cite{Logg2003a,Logg2003b,SavHuVe07}
    and some single non-coupled PDEs \cite{JaLo08}. Therein, the temporal 
    basis is not obtained by restriction
    (as we do), but using the integration
    quadrature points.
\end{remark}

%%%%%%%%%%%%%%%%%%%%%%%%%%%%%%%%%%%%%%%%%%%%%%%%%%%%%%%%%%%%
%%               MODEL PROBLEMS                           %%
%%%%%%%%%%%%%%%%%%%%%%%%%%%%%%%%%%%%%%%%%%%%%%%%%%%%%%%%%%%%
\section{Model problems in strong forms and weak space-time forms}
\label{sec_model_problems}
To prepare the numerical tests considered in Section \ref{sec_numerical_tests}, we formulate two abstract model problems and derive their weak space-time formulations. 
First, we work with an interface coupled heat and wave equation as in \cite{Soszynska2021}.
Next, we derive the space-time weak form for poroelasticity similar to \cite{Girault2011, Bause2017}.

\subsection{Interface coupling: heat and wave equation}
\label{sec_model_problem_heatwave}
We couple the heat and wave equation across a common interface, which serves as a prototypical example for fluid-structure interaction since we couple a parabolic and a hyperbolic problem across an interface.
Consequently, we name the different subdomains `fluid' and `solid'.
\subsubsection{Strong form}
The wave equation is being solved in the solid domain $\Omega_s$ by finding displacement $u_s: \bar{\Omega}_s \times \bar{I} \rightarrow \mathbb{R}$ and velocity $v_s: \bar{\Omega}_s \times \bar{I} \rightarrow \mathbb{R}$ such that
\begin{align*}
    \partial_t v_s - \lambda \Delta_x u_s - \delta \Delta_x v_s  &= g_s \qquad\quad \text{in } \Omega_s \times I, \\
    \partial_t u_s &= v_s \qquad\quad \text{in } \Omega_s \times I, \\
    u_s = v_s &= 0 \qquad\quad \text{on } \Omega_s \times \{0\}, \\
    u_s = v_s &= 0 \qquad\quad \text{on } \Gamma_D^s \times I, \\
    \lambda \partial_{n_s} u_s + \delta \partial_{n_s} v_s &= 0 \qquad\quad \text{on } \Gamma_N^s \times I,
\end{align*}
where $\lambda, \delta \in \mathbb{R}^+$
and $g_s: \bar{\Omega}_s \times I \rightarrow \mathbb{R}$ is a sufficiently smooth right hand side.
For the (strong) damping term $\delta \Delta_x v_s$ and its mathematical influence, we refer the reader to \cite{GaSqu06}.
The heat equation is being solved in the fluid domain $\Omega_f$ by finding velocity $v_f: \bar{\Omega}_f \times \bar{I} \rightarrow \mathbb{R}$ such that
\begin{align*}
    \partial_t v_f - \nu \Delta_x v_f + \beta \cdot \nabla_x v_f &= g_f \qquad\quad \text{in } \Omega_f \times I,\\
    v_f &= 0 \qquad\quad \text{on } \Omega_f \times \{0\}, \\
    v_f &= 0 \qquad\quad \text{on } \Gamma_D^f \times I, \\
    \partial_{n_f} v_f &= 0 \qquad\quad \text{on } \Gamma_N^f \times I,
\end{align*}
with $\nu \in \mathbb{R}^+$, $\beta \in \mathbb{R}^d$ and $g_f: \bar{\Omega}_f \times I \rightarrow \mathbb{R}$ is a sufficiently smooth right hand side.
Similar to the mesh motion PDE in the Arbitrary Lagrangian Eulerian formulation of fluid-structure interaction, we harmonically extend the solid deformation $u_s$ to the fluid domain by finding the displacement $u_f: \bar{\Omega}_f \times \bar{I} \rightarrow \mathbb{R}$ such that
\begin{align*}
    -\Delta_x u_f &= 0 \qquad\quad \text{in } \Omega_f \times I, \\
    u_f &= 0 \qquad\quad \text{on } \Gamma_D^f \times I, \\
    \partial_{n_f} u_f &= 0 \qquad\quad \text{on } \Gamma_N^f \times I.
\end{align*}
Furthermore, the interface conditions are given by
\begin{align*}
    \lambda \partial_{n_s}u_s + \delta \partial_{n_s}v_s + \nu \partial_{n_f}v_f &= 0 \qquad\quad \text{on } \Gamma \times I,\\
    u_f&= u_s \qquad\quad \text{on } \Gamma \times I, \\
    v_f&= v_s \qquad\quad \text{on } \Gamma \times I.
\end{align*}
In the differential equations above, we assume homogeneous Dirichlet boundary conditions on $\Gamma_D^f$ resp. $\Gamma_D^s$ and homogeneous Neumann boundary conditions on $\Gamma_N^f := \partial \Omega \setminus (\Gamma_D^f \cup \Gamma)$ resp. $\Gamma_N^s := \partial \Omega \setminus (\Gamma_D^s \cup \Gamma)$.
Looking back at the notation in (\ref{eq_abstract_interface_problem}), we have $u_1 := U_f := (u_f, v_f)$, $d_1 = 2$ and otherwise replace the index $1$ by the letter 'f' for the fluid domain, i.e. $\Box_1 := \Box_f$.
Similarly, we have $u_2 := U_s := (u_s, v_s)$, $d_2 = 2$ and otherwise replace the index $2$ by the letter 's' for the solid domain, i.e. $\Box_2 := \Box_s$.
For spatio-temporal operators we have
\begin{align*}
    \mathcal{A}_1(u_1) &:= 
    \begin{pmatrix}
        \partial_t v_f - \nu \Delta_x v_f + \beta \cdot \nabla_x v_f \\
        -\Delta_x u_f
    \end{pmatrix}, \\
    \mathcal{A}_2(u_2) &:= 
    \begin{pmatrix}
        \partial_t v_s - \lambda \Delta_x u_s - \delta \Delta_x v_s \\
        \partial_t u_s - v_s
    \end{pmatrix}, 
\end{align*}
for the right hand sides we have
\begin{align*}
    f_1 &:= \begin{pmatrix}
        g_f \\ 0
    \end{pmatrix}, \\
    f_2 &:= \begin{pmatrix}
        g_s \\ 0
    \end{pmatrix},
\end{align*}
and for the Dirichlet and Neumann interface coupling operators we have
\begin{align*}
    \mathcal{B}_{D,u}(u_1, u_2) &:= u_f - u_s, \\
    \mathcal{B}_{D,v}(u_1, u_2) &:= v_f - v_s, \\
    \mathcal{B}_{N}(u_1, u_2) &:= \lambda \partial_{n_s}u_s + \delta \partial_{n_s}v_s + \nu \partial_{n_f}v_f.
\end{align*}
Herein, $n_f$ is the fluid normal vector, $n_s$ is the solid normal vector and $\partial_n g:= \nabla_x g \cdot n$ is the normal derivative of a function $g$.

The spatial function spaces \cite{Soszynska2021} are given by
\begin{align*}
   V_f(\Omega_f) &= \left(H^1_{0, \Gamma_D^f}(\Omega_f)\right)^2, \\
   V_s(\Omega_s) &= \left(H^1_{0, \Gamma_D^s}(\Omega_s)\right)^2, \\
   V(\Omega) &= V_f(\Omega_f) \times V_s(\Omega_s),
\end{align*}
i.e. they consist of one time weakly differentiable functions that vanish on the Dirichlet boundary $\Gamma_D^f$ resp. $\Gamma_D^s$.

\subsubsection{Weak space-time form}
Integrating the strong formulation over time and multiplying by a test function, we get the discontinuous in time weak formulation:

Find $U \in \tilde{X}\left((\mathcal{T}_k^f, \mathcal{T}_k^s), V(\Omega)\right)$ such that
\begin{align}\label{eq:variational_form_heat_wave}
    \tilde{A}(U)(\Phi)  = \tilde{F}(\Phi) \qquad \forall \Phi \in L^2(I, V(\Omega)).
\end{align}
The bilinear form and right hand side read
\begin{align*}
    \tilde{A}(U)(\Phi) &= \sum_{I_m \in \mathcal{T}_k^{\text{fine}}} \int_{I_m} (\partial_t v_f,\phi^{v_f})_{\Omega_f} + \nu ( \nabla_x v_f,\nabla_x \phi^{v_f})_{\Omega_f} + ( \beta \cdot \nabla_x v_f, \phi^{v_f})_{\Omega_f} + ( \nabla_x u_f,\nabla_x \phi^{u_f})_{\Omega_f} \ \mathrm{d}t \\
    &+ \sum_{I_m \in \mathcal{T}_k^{\text{fine}}} \int_{I_m} -\nu\langle\partial_{n_f} v_f,\phi^{v_f}\rangle_{\Gamma} -\langle\partial_{n_f} u_f,\phi^{u_f}\rangle_{\Gamma} + \frac{\gamma \nu}{h} \langle v_f-v_s, \phi^{v_f}\rangle_{\Gamma} + \frac{\gamma}{h} \langle u_f-u_s, \phi^{u_f}\rangle_{\Gamma} \ \mathrm{d}t \\
    &+ \sum_{m_f = 1}^{M^f-1}([v_f]_{m_f}, \phi_{m_f}^{v_f,+})_{\Omega_f} + (v_{f,0}^{+},\phi_0^{v_f,+})_{\Omega_f}\\
    &+ \sum_{I_m \in \mathcal{T}_k^{\text{fine}}} \int_{I_m} (\partial_t v_s,\phi^{v_s})_{\Omega_s}  + \lambda ( \nabla_x u_s,\nabla_x \phi^{v_s})_{\Omega_s} + \delta ( \nabla_x v_s,\nabla_x \phi^{v_s})_{\Omega_s} + (\partial_t u_s,\phi^{u_s})_{\Omega_s} - (v_s, \phi^{u_s})_{\Omega_s}  \ \mathrm{d}t \\
    &+ \sum_{I_m \in \mathcal{T}_k^{\text{fine}}} \int_{I_m} \nu \langle \partial_{n_f} v_f, \phi^{v_s}\rangle_{\Gamma} - \delta \langle \partial_{n_s} v_s, \phi^{v_s}\rangle_{\Gamma}  \ \mathrm{d}t \\
    &+ \sum_{m_s = 1}^{M^s-1}([v_s]_{m_s}, \phi_{m_s}^{v_s,+})_{\Omega_s} + ([u_s]_{m_s}, \phi_{m_s}^{u_s,+})_{\Omega_s} + (v_{s,0}^{+},\phi_0^{v_s,+})_{\Omega_s} + (u_{s,0}^{+},\phi_0^{u_s,+})_{\Omega_s}
\end{align*}
and 
\begin{align*}
    \tilde{F}(\Phi) &= (\!(g_f,\phi^{v_f})\!)_{\Omega_f \times I} + (v_f^0, \phi_0^{v_f,+})_{\Omega_f} + (\!(g_s,\phi^{v_s})\!)_{\Omega_s \times I} + (v_s^0, \phi_0^{v_s,+})_{\Omega_s} + (u_s^0, \phi_0^{u_s,+})_{\Omega_s},
\end{align*}
where
\begin{align*}
    U := (u_f, v_f, u_s, v_s), \quad \Phi := (\phi^{u_f}, \phi^{v_f}, \phi^{u_s}, \phi^{v_s}).
\end{align*}
Following the decomposition of the bilinear form from (\ref{eq_decomposition_bilinear_form}), we have
\begin{align*}
    A_1(U_f)(\Phi^f) &= \sum_{I_m \in \mathcal{T}_k^{\text{fine}}} \int_{I_m} (\partial_t v_f,\phi^{v_f})_{\Omega_f} + \nu ( \nabla_x v_f,\nabla_x \phi^{v_f})_{\Omega_f} + ( \beta \cdot \nabla_x v_f, \phi^{v_f})_{\Omega_f} + ( \nabla_x u_f,\nabla_x \phi^{u_f})_{\Omega_f} \ \mathrm{d}t  \\
     &\quad + \sum_{I_m \in \mathcal{T}_k^{\text{fine}}} \int_{I_m} -\nu\langle\partial_{n_f} v_f,\phi^{v_f}\rangle_{\Gamma} -\langle\partial_{n_f} u_f,\phi^{u_f}\rangle_{\Gamma} + \frac{\gamma \nu}{h} \langle v_f, \phi^{v_f}\rangle_{\Gamma} + \frac{\gamma}{h} \langle u_f, \phi^{u_f}\rangle_{\Gamma} \ \mathrm{d}t, \\
     &\quad + \sum_{m_f = 1}^{M^f-1}([v_f]_{m_f}, \phi_{m_f}^{v_f,+})_{\Omega_f} + (v_{f,0}^{+},\phi_0^{v_f,+})_{\Omega_f}\\
    A_2(U_s)(\Phi^s) &= \sum_{I_m \in \mathcal{T}_k^{\text{fine}}} \int_{I_m} (\partial_t v_s,\phi^{v_s})_{\Omega_s}  + \lambda ( \nabla_x u_s,\nabla_x \phi^{v_s})_{\Omega_s} + \delta ( \nabla_x v_s,\nabla_x \phi^{v_s})_{\Omega_s} + (\partial_t u_s,\phi^{u_s})_{\Omega_s} - (v_s, \phi^{u_s})_{\Omega_s}  \ \mathrm{d}t \\
    &\quad - \sum_{I_m \in \mathcal{T}_k^{\text{fine}}} \int_{I_m} \delta \langle \partial_{n_s} v_s, \phi^{v_s}\rangle_{\Gamma}  \ \mathrm{d}t \\
    &\quad + \sum_{m_s = 1}^{M^s-1}([v_s]_{m_s}, \phi_{m_s}^{v_s,+})_{\Omega_s} + ([u_s]_{m_s}, \phi_{m_s}^{u_s,+})_{\Omega_s} + (v_{s,0}^{+},\phi_0^{v_s,+})_{\Omega_s} + (u_{s,0}^{+},\phi_0^{u_s,+})_{\Omega_s}, \\
    B_1(U_s)(\Phi^f) &= \sum_{I_m \in \mathcal{T}_k^{\text{fine}}} \int_{I_m} -\frac{\gamma \nu}{h} \langle v_s, \phi^{v_f}\rangle_{\Gamma} - \frac{\gamma}{h} \langle u_s, \phi^{u_f}\rangle_{\Gamma} \ \mathrm{d}t,\\
    B_2(U_f)(\Phi^s) &= \sum_{I_m \in \mathcal{T}_k^{\text{fine}}} \int_{I_m} \nu \langle \partial_{n_f} v_f, \phi^{v_s}\rangle_{\Gamma}  \ \mathrm{d}t.
\end{align*}



%%%%%%%%%%%%%%%%%%%%%%%%%%%%%%%%%%%%%%%%%%%%%%%%%%%%%%%%%%%%%%%%%%%%5
\subsection{Volume coupling: poroelasticity}
\label{sec_model_problem_poroelasticity}

\subsubsection{Strong form}
The governing equations for poroelasticity read \cite{Coussy2004,LeSchref99}: Find pressure $p: \bar{\Omega} \times \bar{I} \rightarrow \mathbb{R}$ and displacement $u: \bar{\Omega} \times \bar{I} \rightarrow \mathbb{R}^d$ such that
\begin{align*}
    \partial_t(cp + \alpha(\nabla_x \cdot u)) - \frac{1}{\nu}\nabla_x \cdot (K(\nabla_x p - \rho g)) &= q \qquad\quad \text{in } \Omega \times I, \\
    -\nabla_x \cdot \sigma(u) + \alpha \nabla_x p &= f \qquad\quad \text{in } \Omega \times I,
\end{align*}
with the stress tensor
\begin{align*}
    \sigma(u) := \mu(\nabla_x u + (\nabla_x u)^T) + \lambda (\nabla_x \cdot u)I.
\end{align*}
A rigorous mathematical analysis of this problem from poroelasticity can be found in \cite{Showalter2000},
and this coupled system of equations is also known as Biot system \cite{Biot41a,Biot41b,Biot55,Biot7172,To92}.

Framing this in the notation in (\ref{eq_abstract_volume_problem}), we have $u_1 := u$, $d_1 := d$ and otherwise replace the index $1$ by the letter 'u' for the displacement u, i.e. $\Box_1 := \Box_u$.
In the same fashion, we have $u_2 := p$, $d_2 = 1$ and otherwise replace the index 2 by the letter 'p' for the pressure, i.e. $\Box_2 := \Box_p$. For the spatio-temporal operators we have 
\begin{align*}
    \mathcal{A}_1(u_1, u_2) &:= \partial_t(cp + \alpha(\nabla_x \cdot u)) - \frac{1}{\nu}\nabla_x \cdot (K(\nabla_x p - \rho g)),\\
    \mathcal{A}_2(u_1, u_2) &:= -\nabla_x \cdot \sigma(u) + \alpha \nabla_x p,
\end{align*}
and for the right hand sides we have
\begin{align*}
    f_1 &:= q, \\
    f_2 &:= f.
\end{align*}

Assuming homogeneous Dirichlet boundary conditions for the displacement on $\Gamma_D$ and inhomogeneous Neumann/traction boundary conditions 
\begin{align*}
    \sigma(u) \cdot n  &= t
\end{align*}
on $\Gamma_N = \partial\Omega\setminus \Gamma_D$, the spatial function spaces are given by
\begin{align*}
   V_u(\Omega) &= \left(H^1_{0, \Gamma_D}(\Omega)\right)^d, \\
   V_p(\Omega) &= H^1(\Omega), \\
   V(\Omega) &= V_u(\Omega) \times V_p(\Omega).
\end{align*}

\subsubsection{Weak space-time form}
We can now derive the space-time variational formulation for this problem. By integration by parts we get the variational formulation: Find $U := \{u, p\} \in X\left(I, V(\Omega)\right)$ such that
\begin{align*}
    c(\!(\partial_t p, \phi^p)\!) + \alpha (\!(\partial_t (\nabla_x \cdot u), \phi^p)\!) + \frac{K}{\nu}(\!(\nabla_x p, \nabla_x\phi^p)\!) &= (\!(q, \phi^p)\!) + \frac{K\rho}{\nu}(\!(g, \nabla_x\phi^p)\!), \\
    (\!(\mu(\nabla_x u + (\nabla_x u)^T) + \lambda (\nabla_x \cdot u)I, \nabla_x\phi^u)\!) - \alpha(\!(pI, \nabla_x\phi^u)\!) + \alpha (\!\langle pn, \phi^u\rangle\!)_{ \Gamma_{\text{top}} \times I} &= (\!(f, \phi^u)\!)  + (\!\langle t, \phi^u \rangle\!)_{ \Gamma_{\text{top}} \times I}\\
    \forall \Phi := \begin{pmatrix}
    \phi^u \\ 
    \phi^p 
    \end{pmatrix}  \in L^2\left(I, V(\Omega)\right).
\end{align*} % t := -\bar{t}e_y
Accounting for the discontinuities in time, we thus need to solve the problem: 

Find $U \in \tilde{X}\left((\mathcal{T}_k^u, \mathcal{T}_k^p), V(\Omega)\right)$ such that
\begin{align}\label{eq:variational_form_mandel}
    \tilde{A}(U)(\Phi)  = \tilde{F}(\Phi) \qquad \forall \Phi \in L^2(I, V(\Omega)).
\end{align}
The bilinear form and right hand side read
\begin{align*}
    \tilde{A}(U)(\Phi) &= \sum_{I_m \in \mathcal{T}_k^{\text{fine}}} \int_{I_m}  c(\partial_t p, \phi^p) + \alpha (\partial_t (\nabla_x \cdot u), \phi^p) + \frac{K}{\nu}(\nabla_x p, \nabla_x\phi^p) \ \mathrm{d}t \\
    &+ \sum_{I_m \in \mathcal{T}_k^{\text{fine}}} \int_{I_m} (\mu(\nabla_x u + (\nabla_x u)^T) + \lambda (\nabla_x \cdot u)I, \nabla_x\phi^u) - \alpha(pI, \nabla_x\phi^u) + \alpha\langle pn, \phi^u\rangle_{ \Gamma_{\text{top}}} \ \mathrm{d}t \\
    &+ \sum_{m_u = 1}^{M^u-1}\alpha([\nabla_x \cdot u]_{m_u}, \phi_{m_u}^{p,+}) + \alpha (\nabla_x \cdot u_{0}^{+},\phi_0^{p,+})
    + \sum_{m_p = 1}^{M^p-1}c([p]_{m_p}, \phi_{m_p}^{p,+}) + c (p_{0}^{+},\phi_0^{p,+})
\end{align*}
and 
\begin{align*}
    \tilde{F}(\Phi) &:= (\!(q, \phi^p)\!) + \frac{K\rho}{\nu}(\!(g, \nabla_x\phi^p)\!) + (\!(f, \phi^u)\!)  + (\!\langle t, \phi^u\rangle\!)_{ \Gamma_{\text{top}} \times I} + \alpha(\nabla_x \cdot u^0, \phi_0^{p,+}) + c(p^0, \phi_0^{p,+}),
\end{align*}
where
\begin{align*}
    U := (u, p), \quad \Phi := (\phi^u, \phi^p).
\end{align*}
Following the decomposition of the bilinear form from (\ref{eq_decomposition_bilinear_form}), we have
\begin{align*}
    A_1(u)(\phi^u) &= \sum_{I_m \in \mathcal{T}_k^{\text{fine}}} \int_{I_m} (\mu(\nabla_x u + (\nabla_x u)^T) + \lambda (\nabla_x \cdot u)I, \nabla_x\phi^u)\ \mathrm{d}t, \\
    A_2(p)(\phi^p) &= \sum_{I_m \in \mathcal{T}_k^{\text{fine}}} \int_{I_m}  c(\partial_t p, \phi^p) + \frac{K}{\nu}(\nabla_x p, \nabla_x\phi^p) \ \mathrm{d}t + \sum_{m_p = 1}^{M^p-1}c([p]_{m_p}, \phi_{m_p}^{p,+}) + c (p_{0}^{+},\phi_0^{p,+}), \\
    B_1(p)(\phi^u) &= \sum_{I_m \in \mathcal{T}_k^{\text{fine}}} \int_{I_m} - \alpha(pI, \nabla_x\phi^u) + \alpha\langle pn, \phi^u\rangle_{ \Gamma_{\text{top}}} \ \mathrm{d}t, \\
    B_2(u)(\phi^p) &= \sum_{I_m \in \mathcal{T}_k^{\text{fine}}} \int_{I_m} \alpha (\partial_t (\nabla_x \cdot u), \phi^p)\ \mathrm{d}t + \sum_{m_u = 1}^{M^u-1}\alpha([\nabla_x \cdot u]_{m_u}, \phi_{m_u}^{p,+}) + \alpha (\nabla_x \cdot u_{0}^{+},\phi_0^{p,+}).
\end{align*}
A derivation of a multirate time stepping scheme for this space-time formulation using Section \ref{sec_temporal_multirate_tp_st_fem} can be found in Appendix \ref{sec_timestepping_mandel}. 


%%%%%%%%%%%%%%%%%%%%%%%%%%%%%%%%%%%%%%%%%%%%%%%%%%%%%%%%%%%%
%%               NUMERICAL TESTS                          %%
%%%%%%%%%%%%%%%%%%%%%%%%%%%%%%%%%%%%%%%%%%%%%%%%%%%%%%%%%%%%
\section{Numerical tests}
\label{sec_numerical_tests}
For verification of our method, we perform numerical tests on three different multiphysics problems.
For the first two numerical tests, we perform computations for interface coupled heat and wave equations in 1+1D and 2+1D. For the 1+1D problem, we construct a manufactured solution and for the 2+1D problem, we use configuration 2.1 from \cite{Soszynska2021} with a transport term in the heat equation and a damping term in the wave equation.
As the third numerical test, we consider Mandel's problem 
\cite{Mandel1953,Cr63,Cheng88,AbChCuDeRo96,DuiMiWi22}, 
a poroelasticity benchmark problem in 2+1D, as an example of volume coupled problems.

The time marching codes have been implemented in FEniCS \cite{fenics2015} and the space-time FEM codes have been written in deal.II \cite{dealii2019design, dealII94}.

\subsection{1+1D heat and wave equation}
\label{sec_1d_heat_wave}
As the first numerical test, we consider a one-dimensional problem where we couple the heat and wave equation, cf. Section \ref{sec_model_problem_heatwave}. The time interval is $I:=(0,4)$, the spatial domain for the heat equation - the fluid domain - is $\Omega_f := (0,2)$ and the spatial domain for the wave equation - the solid domain - is $\Omega_s := (2,4)$. The interface is given by $\Gamma := \bar{\Omega}_f \cap \bar{\Omega}_s = \{2\}.$ 
To further simplify the model problem from Section \ref{sec_model_problem_heatwave}, we leave out the transport term in the heat equation and the damping term in the wave equation, i.e. $\beta = 0$ and $\delta = 0$. 
For our test we choose the parameters in the PDEs to be $\nu = 0.001$ and $\lambda = 1000$. The penalty parameter $\gamma$ for the Dirichlet interface conditions is $1000$.
The boundaries are defined as $\Gamma_D^f = \{0\}$, $\Gamma_N^s = \{4\}$ and the variational formulation is shown in (\ref{eq:variational_form_heat_wave}).
We prescribe the analytical solution as
\begin{align*}
    u_s(x,t) &= t^2 \cdot  \cos \left( \frac{\pi(x-2)}{2} \right)  &&\forall (x,t) \in \Omega_s \times I, \\
    u_f(x,t) &= t^2 \cdot \frac{x}{2}  &&\forall (x,t) \in \Omega_f \times I,
\end{align*}
and the velocities
\begin{align*}
    v_s(x,t) &= 2t \cdot  \cos \left( \frac{\pi(x-2)}{2} \right)  &&\forall (x,t) \in \Omega_s \times I, \\
    v_f(x,t) &= 2t \cdot \sin \left( \frac{\pi x}{4} \right)  &&\forall (x,t) \in \Omega_f \times I,
\end{align*}
which are shown in Figure \ref{fig:manufactured_solution_1d}.
% Figure environment removed
\noindent The right hand side functions are
\begin{align*}
    g_s(x,t) &=  2 \cos \left( \frac{\pi(x-2)}{2} \right) +  \frac{\pi^2t^2\lambda\cos(\frac{\pi(x-2)}{2})}{4} &&\forall (x,t) \in  \Omega_s \times I, \\
    g_f(x,t) &= 2 \sin \left( \frac{\pi x}{4} \right) + \frac{\pi^2t\nu\sin(\frac{\pi x}{4})}{8}  &&\forall (x,t) \in \Omega_f \times I.
\end{align*}

\subsubsection{Monolithic multirate time marching scheme results}
To compare the discrete solutions with the analytic ones, we define 
\begin{align*}
    J(U - U_{kh}) &= \|U - U_{kh}\|_{L^2(I, L^2(\Omega))}.
\end{align*}
We similarly define specific contributions as 
\begin{align*} 
    J_f(U - U_{kh}) &= \|U_f - U_{f,kh}\|_{L^2(I, L^2(\Omega_f))}, \\
    J_s(U - U_{kh}) &= \|U_s - U_{s, kh}\|_{L^2(I, L^2(\Omega_s))}.
\end{align*}
In the tables, we will analyze the errors denoted by 
\begin{equation*}
\eta_{kh} \coloneqq J(U - U_{kh}), \hspace*{0.5 cm}\eta_{kh}^f \coloneqq J_f(U - U_{kh}), \hspace*{0.5 cm}\eta_{kh}^s \coloneqq J_s(U - U_{kh}).
\end{equation*}



\begin{table}[H]
  \begin{center}
    \begin{tabular}{cccc|ccc}
        \toprule
        $|\mathcal{T}_k^{\text{coarse}}|$ & $|\mathcal{T}_k^f|$ & $|\mathcal{T}_k^s|$ & $|\mathcal{T}_k^f| : |\mathcal{T}_k^s|$ & $\eta_{kh}^f$ & $\eta^s_{kh}$ & $\eta_{kh}$ \\       
        \midrule
      25 & 25 & 25 & 1:1& $1.78 \cdot 10^{-2}$ & $3.73 \cdot 10^{-2}$ & $4.14 \cdot 10^{-2}$ \\
      50 & 50 & 50 & 1:1 & $8.78 \cdot 10^{-3}$  & $1.90 \cdot 10^{-2}$ &  $2.10 \cdot 10^{-2}$  \\
      100 & 100 & 100 & 1:1 & $4.46 \cdot 10^{-3}$& $9.77 \cdot 10^{-3}$ & $1.08 \cdot 10^{-2}$  \\
      200 & 200 & 200 & 1:1 & $2.57 \cdot 10^{-3}$& $5.16 \cdot 10^{-3}$ & $5.76 \cdot 10^{-3}$ \\
      400 & 400 & 400 & 1:1 & $1.93 \cdot 10^{-3}$  & $2.96 \cdot 10^{-3}$ &   $3.52 \cdot 10^{-3}$\\
      \bottomrule
    \end{tabular}
    \caption{Errors for fully uniform time-stepping.}
    \label{uniform_1D}
  \end{center}
\end{table}

\begin{table}[H]
  \begin{center}
    \begin{tabular}{cccc|ccc}
        \toprule
        $|\mathcal{T}_k^{\text{coarse}}|$ & $|\mathcal{T}_k^f|$ & $|\mathcal{T}_k^s|$ & $|\mathcal{T}_k^f| : |\mathcal{T}_k^s|$ & $\eta_{kh}^f$ & $\eta^s_{kh}$ & $\eta_{kh}$ \\       
        \midrule
      25 & 50 & 25 & 2:1& $2.40 \cdot 10^{-2}$ & $3.86 \cdot 10^{-2}$ & $4.54 \cdot 10^{-2}$ \\
      50 & 100 & 50 & 2:1 & $1.14 \cdot 10^{-2}$  & $1.98 \cdot 10^{-2}$ &  $2.29 \cdot 10^{-2}$  \\
      100 & 200 & 100 & 2:1 & $5.45 \cdot 10^{-3}$& $1.03 \cdot 10^{-2}$ & $1.17 \cdot 10^{-2}$  \\
      200 & 400 & 200 & 2:1 & $2.72 \cdot 10^{-3}$& $5.53 \cdot 10^{-3}$ & $6.16 \cdot 10^{-3}$ \\
      \bottomrule
    \end{tabular}
    \caption{Errors for one refinement in the fluid domain.}
    \label{refinement_fluid_1D}
  \end{center}
\end{table}

\begin{table}[H]
  \begin{center}
    \begin{tabular}{cccc|ccc}
        \toprule
        $|\mathcal{T}_k^{\text{coarse}}|$ & $|\mathcal{T}_k^f|$ & $|\mathcal{T}_k^s|$ & $|\mathcal{T}_k^f| : |\mathcal{T}_k^s|$ & $\eta_{kh}^f$ & $\eta^s_{kh}$ & $\eta_{kh}$ \\       
        \midrule
      25 & 25 & 50 & 1:2& $8.98 \cdot 10^{-3}$ & $2.10 \cdot 10^{-2}$ & $2.28 \cdot 10^{-2}$ \\
      50 & 50 & 100 & 1:2 & $5.06 \cdot 10^{-3}$  & $1.04 \cdot 10^{-2}$ &  $1.16 \cdot 10^{-2}$  \\
      100 & 100 & 200 & 1:2 & $3.33 \cdot 10^{-3}$& $5.21 \cdot 10^{-3}$ & $6.18 \cdot 10^{-3}$  \\
      200 & 200 & 400 & 1:2 & $2.65 \cdot 10^{-3}$& $2.80 \cdot 10^{-3}$ & $3.86 \cdot 10^{-3}$ \\
      \bottomrule
    \end{tabular}
    \caption{Errors for one refinement in the solid domain.}
    \label{refinement_solid_1D}
  \end{center}
\end{table}

In Tables~\ref{uniform_1D}, \ref{refinement_fluid_1D} and \ref{refinement_solid_1D} we show the results for the one-dimensional model problem. We consider three cases, fully uniform time-stepping, a set-up with one refinement in time in the fluid domain only, and a similar configuration where only the time-steps in the solid domain are refined. For all of the simulations, the space mesh remains unchanged and only the time meshes are refined. We use a rather coarse space mesh with 100 cells only. To illustrate how the error is decomposed into different parts, we show the values of $\eta_{kh}^f$, $\eta_{kh}^s$ as well as $\eta_{kh}$. In Table~\ref{uniform_1D} we collect the numbers for fully uniform time-stepping. We can clearly see linear convergence of the error expected for the backward Euler time-stepping scheme. Only on higher refinement levels does the convergence rate slightly deteriorate due to the space mesh's coarseness. We can notice that the contributions from the solid problem dominate. In Table~\ref{refinement_fluid_1D} we once refine the time-steps in the fluid domain while leaving the time-steps in the solid domain unchanged. Unsurprisingly, this change does not lead to a decrease in the overall error. In Table~\ref{refinement_solid_1D} we instead refine the solid time steps which leads to a desirable result in a reduction of the overall error. In both cases of partial refinement, the error contributions from the unrefined subproblems slightly increase.



\subsubsection{Tensor-product space-time FEM results}
For the space-time finite element discretization, we use a Galerkin discretization with $\dG(1)$ in time and linear finite elements in space. 
For this numerical test, we use a penalty parameter $\gamma = 1000$, an initial coarse temporal mesh with $|\mathcal{T}_k^{\text{coarse}}| = 4$ temporal elements and 10 spatial degrees of freedom each for fluid and solid. 
To test the performance of temporal multirate space-time FEM, we uniformly refine the fluid or solid temporal mesh up to 8 times. This means that e.g. for each 1 solid temporal element we have 256 fluid temporal elements. The initial mesh is being refined uniformly in space and time. To measure the convergence, we compute the error between the analytical solution $U$ and the finite element solution $U_{kh}$ using the quantity of interest
\begin{align*}
    J(U - U_{kh}) &= \|U - U_{kh}\|_{L^2(I, L^2(\Omega))}.
\end{align*}

Using a finer fluid temporal mesh, we get the convergence plot in Figure \ref{fig:convergence_space_time_1d_finer_fluid}. Therein, we observe that using a finer temporal mesh for the fluid in comparison to the solid does not improve the error between the finite element and the analytical solution.

% Figure environment removed

Repeating the same convergence tests again for a finer solid temporal mesh, the results are shown in Figure \ref{fig:convergence_space_time_1d_finer_solid}.
Here, we observe that using a finer temporal mesh for the solid in comparison to the fluid greatly reduces the error between the finite element and the analytical solution. 
In particular, comparing 1:1 and 1:4 temporal meshes, we observe that the error on the coarsest grid is four times lower when the temporal mesh of the solid is finer and we then get also optimal convergence rates.
Using even finer temporal meshes, e.g. 1:16, further reduces the error but going beyond that, i.e. 1:64 and 1:256, almost does not improve the finite element solution anymore. 

% Figure environment removed

In Figure \ref{fig:space_time_solution_1d}, we show an example of a space-time FEM solution on the coarsest spatial and temporal meshes. Here, the temporal mesh of the solid, shown in the upper half of the space-time domain, is four times as fine as for the fluid, which is plotted in the lower half of the space-time domain. Additionally, we visualize the error to the analytical solution and we observe especially in the velocity error that a discontinuous time discretization has been used.

% Figure environment removed

In Figure \ref{fig:space_time_interface_1d}, we now visualize the temporal evolution of the continuity of the displacement and the velocity at the interface of the space-time finite element solution from Figure \ref{fig:space_time_solution_1d}. 
Due to the different number of temporal elements for fluid and solid, we cannot expect a perfect match between the solutions at the interface. 
Nevertheless, the solid and the fluid displacement almost coincide. The solid and the fluid velocity are not too different either, but here we again observe the jumps in the solid solution between the temporal elements.

% Figure environment removed

\subsection{2+1D heat and wave equation}
\label{sec_2d_heat_wave}

As the second numerical test, we again consider a coupled heat and wave equation from Section \ref{sec_model_problem_heatwave} but now in two spatial dimensions. For this we use Configuration 2.1 from \cite{Soszynska2021}. The time interval is $I:=(0,1)$, the spatial domain for the heat equation - the fluid domain - is $\Omega_f := (0,4) \times (0,1)$ and the spatial domain for the wave equation - the solid domain - is $\Omega_s := (0,4) \times (-1,0)$. The interface is given by $\Gamma := \bar{\Omega}_f \cap \bar{\Omega}_s = (0,4) \times \{0\}.$
For our test we choose the parameters in the PDEs to be 
\begin{align*}
    \nu = 0.001, \quad \beta = \begin{pmatrix}
    2 \\ 0
\end{pmatrix}, \quad \lambda = 1000,\quad \delta = 0.1.   
\end{align*}
As right hand sides we choose
\begin{align*}
    g_s(x, t) &= 0, \\
    g_f(x, t) &= \begin{cases}
        e^{-\left((x_1-\frac{1}{2})^2+(x_2-\frac{1}{2})^2\right)}  & \text{if } 0 \leq t \leq 0.1,\\
        0 & \text{else}.
    \end{cases}
\end{align*}
The penalty parameter for the Dirichlet interface conditions is 1000.
The boundaries are defined as $\Gamma_D^f = (0,4) \times \{1\}$, $\Gamma_N^f = \{0,4\} \times (0,1)$, $\Gamma_D^s = \{0,4\} \times (-1,0)$, $\Gamma_N^s =(0,4) \times \{-1\}$ and the variational formulation is shown in (\ref{eq:variational_form_heat_wave}).

\subsubsection{Monolithic multirate time marching scheme results}
The simulations are performed on a coarse space mesh consisting of 60 and 15 elements in the horizontal and vertical directions, respectively. As a quantity of interest, we take the functional
\begin{equation}
    J(U_{kh}) = \nu \|\nabla_x v_{kh}\|^2_{L^2(I, L^2(\Omega_f))},
\label{goal_functional}
\end{equation}
Due to the lack of an analytical solution, the reference value is taken from computations on a fine temporal mesh given by $J(U) \coloneqq 4.3902649 \cdot 10^{-4}$. For this example, the error is defined as $\eta_{kh}^f \coloneqq J(U) - J(U_{kh})$.
\begin{table}[H]
  \begin{center}
    \begin{tabular}{cccc|c}
        \toprule
        $|\mathcal{T}_k^{\text{coarse}}|$ & $|\mathcal{T}_k^f|$ & $|\mathcal{T}_k^s|$ & $|\mathcal{T}_k^f| : |\mathcal{T}_k^s|$ & $\eta_{kh}^f$  \\       
        \midrule
      25 & 25 & 25 & 1:1& $1.43 \cdot 10^{-4}$  \\
      50 & 50 & 50 & 1:1 & $1.48 \cdot 10^{-4}$ \\
      100 & 100 & 100 & 1:1 & $-5.30 \cdot 10^{-6}$\\
      200 & 200 & 200 & 1:1 & $3.69 \cdot 10^{-5}$\\
      400 & 400 & 400 & 1:1 & $1.62 \cdot 10^{-5}$\\

      \bottomrule
    \end{tabular}
    \caption{Errors for fully uniform time-stepping.}
    \label{uniform_2D}
  \end{center}
\end{table}

\begin{table}[H]
  \begin{center}
    \begin{tabular}{cccc|c}
        \toprule
        $|\mathcal{T}_k^{\text{coarse}}|$ & $|\mathcal{T}_k^f|$ & $|\mathcal{T}_k^s|$ & $|\mathcal{T}_k^f| : |\mathcal{T}_k^s|$ & $\eta_{kh}^f$ \\       
        \midrule
      25 & 50 & 25 & 2:1& $1.28 \cdot 10^{-4}$ \\
      50 & 100 & 50 & 2:1 & $-1.57 \cdot 10^{-5}$  \\
      100 & 200 & 100 & 2:1 & $3.79 \cdot 10^{-4}$ \\
      200 & 400 & 200 & 2:1 & $2.28 \cdot 10^{-4}$\\

      \bottomrule
    \end{tabular}
    \caption{Errors for one refinement in the fluid domain.}
    \label{refinement_fluid_2D}
  \end{center}
\end{table}

\begin{table}[H]
  \begin{center}
    \begin{tabular}{cccc|c}
        \toprule
        $|\mathcal{T}_k^{\text{coarse}}|$ & $|\mathcal{T}_k^f|$ & $|\mathcal{T}_k^s|$ & $|\mathcal{T}_k^f| : |\mathcal{T}_k^s|$ & $\eta_{kh}^f$\\       
        \midrule
      25 & 25 & 50 & 1:2& $1.49\cdot 10^{-4}$ \\
      50 & 50 & 100 & 1:2 & $1.55 \cdot 10^{-4}$\\
      100 & 100 & 200 & 1:2 & $6.47 \cdot 10^{-6}$\\
      200 & 200 & 400 & 1:2 & $4.80\cdot 10^{-5}$\\

      \bottomrule
    \end{tabular}
    \caption{Errors for one refinement in the solid domain.}
    \label{refinement_solid_2D}
  \end{center}
\end{table}

In Tables~\ref{uniform_2D},~\ref{refinement_fluid_2D} and~\ref{refinement_solid_2D} we show the error values with respect to the goal functional. Table~\ref{uniform_2D} corresponds to simulations with fully uniform time-stepping. Table~\ref{refinement_fluid_2D} consists of results for time-stepping with partial refinement in the fluid domain, and Table~\ref{refinement_solid_2D} is an analogous case with refinement in the solid domain only. The coarsest time mesh has 25 time-steps. It is then refined up to 400 time-steps. The space mesh remains unchanged. At first, the uniform time-stepping does not exhibit a consistent convergence rate. However, on finer meshes we can observe the expected linear convergence rate. Similarly as in the one-dimensional case, partial refinement in the fluid domain significantly decreases the errors while refinement in the solid domain slightly increases them.


\subsubsection{Tensor-product space-time FEM results}
For the space-time finite element discretization, we use a Galerkin discretization with $\dG(1)$ in time and linear finite elements in space. 
For this numerical test, we use a penalty parameter $\gamma = 1000$, a spatial mesh with 80 spatial cells in the x-direction and 20 spatial cells in the y-direction, and an initial coarse temporal mesh with $|\mathcal{T}_k^{\text{coarse}}| = 50$ temporal elements.
To test the performance of temporal multirate space-time FEM, we uniformly refine the fluid or solid temporal mesh up to 4 times. This means that e.g. for each 1 solid temporal element we have 16 fluid temporal elements. The initial mesh is being refined uniformly only in time. To measure the convergence, we consider the quantity of interest defined in~(\ref{goal_functional})
and compare with the reference value $J(U) := 2.48587692 \cdot 10^{-4}$ from a simulation with $50000$ temporal elements.

Using a finer fluid temporal mesh, we get the convergence plot in Figure \ref{fig:convergence_space_time_2d_finer_fluid}. Therein, we observe that using a finer temporal mesh for the fluid in comparison to the solid greatly reduces the error between the finite element quantity of interest and its reference value. These improvements are greatest for 1:1, 2:1 and 4:1 temporal meshes. Even finer fluid temporal meshes have only marginal benefit, since although they initially reduce the error in the goal functional, on finer temporal meshes there are minor differences between the 4:1 and 16:1 temporal meshes.

% Figure environment removed

Performing the same convergence test for a finer solid temporal mesh, the results are shown in Figure \ref{fig:convergence_space_time_2d_finer_solid}. Here, we observe that using a finer temporal mesh for the solid in comparison to the fluid does not improve the error in the quantity of interest. This was to be expected since we only have a non-zero right hand side in the fluid domain and thus the solution behavior is mainly dictated by the heat equation. 
% Figure environment removed

%%%%%%%%%%%%%%%%%%%%%%%%%%%%%%%%%%%%%%%%%%%%%%%%%%%%%%%%%%%%%
\subsection{Mandel's problem}
\label{sec_mandel_problem}
Finally, as an example for volume coupled problems, we consider Mandel's problem \cite{Mandel1953, Gai2004, Guzman2012, Girault2011, Liu04, Wick2020PFF, AbChCuDeRo96, Cheng88}, which is a benchmark from poroelasticity from Section \ref{sec_model_problem_poroelasticity},
and more recently also for nonlinear poroelasticity \cite{DuiMiWi22}.
This problem has also been solved by decoupled multirate schemes in \cite{Dean2006, Almani2016, Bause2017, Borregales2019}.
Therein, one observes the so-called Mandel-Cryer effect \cite{Cr63} of 
a non-monotonic pressure evolution: first increasing pressure, followed by decreasing pressure. 
Let $\Omega := (\SI{0}{\meter}, \SI{100}{\meter}) \times (\SI{0}{\meter}, \SI{20}{\meter})$ with boundaries as shown in Figure \ref{fig:mandel_domain}.
% Figure environment removed
\noindent The initial and boundary conditions are given by
\begin{align*}
    p(0) &= p^0 = 0 &&\text{in } \Omega, \\
    u(0) &= u^0 = 0 &&\text{in } \Omega, \\
    \sigma(u) \cdot n &= -\bar{t}e_y &&\text{on } \Gamma_{\text{top}} \times I, \\
    p &= 0  &&\text{on } \Gamma_{\text{right}} \times I, \\
    u_y &= 0  &&\text{on } \Gamma_{\text{bottom}} \times I, \\
    u_x &= 0  &&\text{on } \Gamma_{\text{left}} \times I.
\end{align*}
The parameters for Mandel's problem are summarized in Table \ref{tab:params_mandel}.
\begin{table}[H]
    \centering
    \begin{tabular}{ |p{5cm}||p{3cm}|}
         \hline
         Parameter & Value \\
         \hline
            M & \SI{1.75e7}{\pascal} \\
            c & 1/M \\
            $\alpha$ & \SI{1}{\pascal\metre} \\
            $\nu$ & \SI{1e-3}{\metre\squared\per\second} \\
            K & \SI{1e-13}{\metre\squared} \\
            $\rho$ & \SI{1}{\kilogram\per\metre\cubed} \\
            $\bar{t}$ & \SI{1e7}{\pascal\metre} \\
             $g$ & $0$ \\
            $q$ & $0$ \\ 
            $f$ & $0$ \\
            $\mu$ & \SI{1e8}{} \\
            $\lambda$ & $\frac{2}{3} \times 10^8$ \\
         \hline
    \end{tabular}
    \caption{Parameters in Mandel's problem} 
    \label{tab:params_mandel}
\end{table}

\noindent We employ a space-time finite element discretization and a Galerkin discretization with $\dG(0)$ in time for this numerical test. In space, we use quadratic finite elements for the displacement $u$ and linear finite elements for the pressure $p$. 

For this numerical test, we use a fixed spatial mesh with 16 spatial cells in the x-direction and 16 spatial cells in the y-direction.
As a reference solution, we solve the Mandel with $\dG(0)$ elements and $500000$ temporal elements for displacement and pressure. The solution at the bottom boundary is shown in Figure \ref{fig:reference_solution_Mandel}. 

% Figure environment removed

To validate the efficiency of our proposed multirate methodology, we use the quantity of interest
\begin{align*}
    J(U_{kh}) &= \int_I \int_{\Gamma_{\text{bottom}}} p (x,t)\ \mathrm{d}(x,t),
\end{align*}
and compare with the reference value $J(U) := 8.718831\cdot10^{13}$ from the simulation with $500000$ temporal elements.

For the convergence studies, we employ an initial coarse temporal mesh with $|\mathcal{T}_k^{\text{coarse}}| = 1250$ temporal elements and then uniformly refine the displacement or pressure temporal mesh up to 4 times. This means that e.g. for each displacement temporal element we have 16 pressure temporal elements.

Using a finer displacement temporal mesh, we get the convergence plot in Figure \ref{fig:convergence_space_time_Mandel_finer_displacement}. 
Here, we see that using a finer temporal mesh for the displacement in comparison to the pressure does not improve the error in the quantity of interest and the error is identical for all temporal meshes in the first five significant digits. This is not surprising, since the constitutive equation for the displacement is a quasi-stationary problem and the quantity of interest only measures a pressure boundary integral over time. Therefore, one can expect only refinement of the temporal mesh of the pressure to lead to an error reduction.

% Figure environment removed

In Figure \ref{fig:convergence_space_time_mandel_finer_pressure}, we run the same convergence test for a finer temporal mesh for the pressure.
Now, it pays off using a finer temporal mesh for the pressure than for the displacement. Not only do we have linear convergence of the error under uniform refinement of both temporal meshes, but also for the refinement ratio of pressure to displacement. 
More concretely, the error roughly halves when going from 1:1 to 1:2, going from 1:2 to 1:4, etc. 
This shows that instead of using conventional time stepping schemes, which rely on the same temporal mesh for displacement and pressure, we can get the same error reduction by only increasing the number of space-time degrees of freedom for the pressure and using a fixed number of space-time degrees of freedom for the displacement.
This multirate scheme is particularly cost-effective when the displacement requires high-order spatial finite elements than the pressure, e.g. as in the case of Taylor-Hood elements. 

% Figure environment removed


%%%%%%%%%%%%%%%%%%%%%%%%%%%%%%%%%%%%%%%%%%%%%%%%%%%%%%%%%%%%
%%                   CONCLUSION                           %%
%%%%%%%%%%%%%%%%%%%%%%%%%%%%%%%%%%%%%%%%%%%%%%%%%%%%%%%%%%%%
\section{Conclusions and outlook}
\label{sec_conclusions}
In this work, we proposed a monolithic space-time multirate framework for 
interface-coupled and volume-coupled problems.
One first objective was a mathematical abstract formalism that allows to consider both interface-coupled and volume-coupled problems.
As such model problems 
a interface-coupled heat-wave system and a volume-coupled poro-elasticity problem 
were taken. The function spaces and tensor-product Galerkin finite element 
discretization were designed to treat the multirate idea in a monolithic fashion. 
Here, classical continuous finite elements were employed for the spatial 
discretization, while the temporal approximation was based on discontinuous finite elements. The concepts and numerical realization were discussed in great detail in Section \ref{sec_methodology}.
Three numerical tests were conducted in which the multirate schemes were computationally
analyzed in detail. Here, goal-functionals were adopted in order to study specific 
quantities of interest. Overall, the performances were excellent.
So far, the different time meshes were given a priori. A future extension is 
to determine the time meshes by an error estimator; see also \cite{Logg2003a} for early work in this direction
applied to ODEs. We have already used 
goal functionals as quantities of interest in the current paper. Consequently,
goal-oriented a posteriori error estimates will be a natural choice to
obtain such time meshes. 
Finally, our abstract formalism, corresponding algorithms, and implementation were designed in such a way that they allow 
for future extensions to nonlinear problems.


\appendix

\section{Derivation of a time-stepping scheme for the Mandel problem}
\label{sec_timestepping_mandel}

In the following, we will derive by hand the time-stepping scheme for the Mandel problem with one displacement temporal element and two pressure temporal elements. Other multirate time-stepping schemes can be derived in a similar fashion.
For this we take the variational formulation (\ref{eq:variational_form_mandel}) of the Mandel problem, remove the terms in the right hand side function which are zero for the Mandel problem and only consider a time step $I_m = (0, k)$. For the displacement, we have only the temporal element $(0,k)$ and for the pressure we have the temporal elements $(0, \frac{k}{2})$ and $(\frac{k}{2},k)$.
For $\dG(0)$ in time we then have the ansatz
\begin{align*}
    u(t) &= u_1 \cdot \underbrace{\chi_{(0,k)}(t)}_{=: \phi_k^{u, (1)}}, \\
    p(t) &= p_1 \cdot \underbrace{\chi_{(0,\frac{k}{2})}(t)}_{=: \phi_k^{p, (1)}} + p_2 \cdot \underbrace{\chi_{(\frac{k}{2},k)}(t)}_{=: \phi_k^{p, (2)}},
\end{align*}
where $\chi_{I_k}(\cdot)$ denotes the temporal indicator function of the temporal element $I_k$.
We introduce the spatial matrices (and vector)
\begin{align*}
    K^p &= \left\{ \frac{K}{\nu}(\nabla_x \phi_h^{p,(j)},\nabla_x \phi_h^{p,(i)}) \right\}_{i,j = 1}^{\# \text{DoFs}(\mathcal{T}_h^p)}, \\
    M^p &= \left\{ c(\phi_h^{p,(j)}, \phi_h^{p,(i)}) \right\}_{i,j = 1}^{\# \text{DoFs}(\mathcal{T}_h^p)}, \\
    \Sigma^u &= \left\{ (\sigma(\phi_h^{u,(j)}),\nabla_x\phi_h^{u,(i)}) \right\}_{i,j = 1}^{\# \text{DoFs}(\mathcal{T}_h^u)}, \\
    F^u &= \left\{ \langle -\bar{t}e_y, \phi_h^{u, (i)}\rangle_{ \Gamma_{\text{top}}} \right\}_{i = 1}^{\# \text{DoFs}(\mathcal{T}_h^u)}, \\
    B^{up} &= \left\{ -\alpha(\phi_h^{p,(j)}I,\nabla_x \phi_h^{u,(i)}) + \alpha\langle \phi_h^{p,(j)}n, \phi_h^{u,(i)}\rangle_{ \Gamma_{\text{top}}} \right\}_{i,j = 1}^{\# \text{DoFs}(\mathcal{T}_h^u), \# \text{DoFs}(\mathcal{T}_h^p)}, \\
    B^{pu} &= \left\{ \alpha(\nabla_x \cdot \phi_h^{u,(j)},\phi_h^{p,(i)})\right\}_{i,j = 1}^{\# \text{DoFs}(\mathcal{T}_h^p), \# \text{DoFs}(\mathcal{T}_h^u)}.
\end{align*}
Using these spatial matrices, the time stepping scheme for one displacement temporal element and two pressure temporal elements is given by
\begin{align*}
    &(\phi_k^{u, (1)}, \phi_k^{u, (1)})_{I_m} \Sigma^u u_1 + \left\{(\phi_k^{p, (j)}, \phi_k^{u, (1)})_{I_m}\right\}_{j=1}^2 \otimes B^{up} \begin{pmatrix}
        p_1 \\ p_2
    \end{pmatrix} =  (1, \phi_k^{u, (1)})_{I_m} F^u, \\
    &\left\{(\phi_k^{p, (j)}, \phi_k^{p, (i)})_{I_m}\right\}_{i,j=1}^2 \otimes K^{p} \begin{pmatrix}
        p_1 \\ p_2
    \end{pmatrix}  + 
    \begin{pmatrix}
        1 & 0 \\
        -1 & 1
    \end{pmatrix}
    \otimes M^p \begin{pmatrix}
        p_1 \\ p_2
    \end{pmatrix} \\
    &+ \begin{pmatrix}
        1 & 0
    \end{pmatrix}
    \otimes B^{pu} u_1
    = \begin{pmatrix}
        1 & 0
    \end{pmatrix}
    \otimes B^{pu} u_0
    + \begin{pmatrix}
        1 & 0
    \end{pmatrix}
    \otimes M^p p_0.
\end{align*}
Now we evaluate all temporal integrals from above
\begin{align*}
    (\phi_k^{u, (1)}, \phi_k^{u, (1)})_{I_m} &= k, \\
    (\phi_k^{p, (i)}, \phi_k^{p, (j)})_{I_m} &= \frac{k}{2}\delta_{ij}, \\
    (\phi_k^{p, (j)}, \phi_k^{u, (1)})_{I_m} &= \frac{k}{2},
\end{align*}
and arrive at the block linear system
\begin{align*}
    \begin{pmatrix}
        k\Sigma^u & \frac{k}{2}B^{up} & \frac{k}{2}B^{up} \\
        B^{pu} & \frac{k}{2}K^p + M^p & 0 \\
        0 & -M^p & \frac{k}{2}K^p + M^p
    \end{pmatrix}
    \begin{pmatrix}
        u_1 \\
        p_1 \\
        p_2
    \end{pmatrix}
    = 
    \begin{pmatrix}
        kF^u \\
        B^{pu}u_0 + M^p p_0 \\
        0
    \end{pmatrix}.
\end{align*}
In the space-time methodology from Section \ref{sec_temporal_multirate_tp_st_fem} the matrix 
has the same content, but the coupling blocks 
\begin{align*}
    \begin{pmatrix}
        B^{pu} \\ 0
    \end{pmatrix} \qquad \text{and} \qquad 
    \begin{pmatrix}
        \frac{k}{2}B^{up} & \frac{k}{2}B^{up}
    \end{pmatrix}
\end{align*}
are being computed by first assembling the temporal integrals with the finer temporal basis, i.e. we have
\begin{align*}
    \begin{pmatrix}
        B^{pu} & 0 \\ -B^{pu} & B^{pu}
    \end{pmatrix} \qquad \text{and} \qquad 
    \begin{pmatrix}
        \frac{k}{2}B^{up} & 0 \\ 0 & \frac{k}{2}B^{up}
    \end{pmatrix}.
\end{align*}
Using the restriction matrix
\begin{align*}
    R = \begin{pmatrix}
        1 & 1
    \end{pmatrix},
\end{align*}
we then have
\begin{align*}
    \begin{pmatrix}
        B^{pu} & 0 \\ -B^{pu} & B^{pu}
    \end{pmatrix} \begin{pmatrix}
        1 \\ 1
    \end{pmatrix} = \begin{pmatrix}
        B^{pu} \\ 0
    \end{pmatrix} \qquad \text{and} \qquad 
    \begin{pmatrix}
        1 & 1
    \end{pmatrix}
    \begin{pmatrix}
        \frac{k}{2}B^{up} & 0 \\ 0 & \frac{k}{2}B^{up}
    \end{pmatrix} = \begin{pmatrix}
        \frac{k}{2}B^{up} & \frac{k}{2}B^{up}
    \end{pmatrix}.
\end{align*}


%%%%%%%%%%%%%%%%%%%%%%%%%%%%%%%%%%%%%%%%%%%%%%%%%%%%%%%%%%%%
%%                  ACKNOWLEDGEMENTS                      %%
%%%%%%%%%%%%%%%%%%%%%%%%%%%%%%%%%%%%%%%%%%%%%%%%%%%%%%%%%%%%
\section*{Acknowledgements}
JR acknowledges the funding of the German Research Foundation (DFG) within the framework of the International Research Training Group on  Computational Mechanics Techniques in High Dimensions GRK 2657 under Grant Number 433082294. MS and TR acknowledge support by the Deutsche Forschungsgemeinschaft (DFG, German Research Foundation) - 314838170, GRK 2297 MathCoRe. TW
acknowledges support by the Deutsche Forschungsgemeinschaft 
(DFG) under Germany’s Excellence Strategy within the Cluster of Excellence PhoenixD (EXC 2122, Project ID 390833453).
In addition, we thank Hendrik Fischer for fruitful discussions and comments,
and Fleurianne Bertrand (TU Chemnitz) for some discussions on Mandel's problem.
%%%%%%%%%%%%%%%%%%%%%%%%%%%%%%%%%%%%%%%%%%%%%%%%%%%%%%%%%%%%



%%%%%%%%%%%%%%%%%%%
%	BIBLIOGRAPHY  %
%%%%%%%%%%%%%%%%%%%
%\bibliographystyle{abbrv}
%\bibliography{lit.bib}

\begin{thebibliography}{10}

\bibitem{AbChCuDeRo96}
Y.~N. Abousleiman, A.~Cheng, L.~Cui, E.~Detournay, and J.~C. Roegiers.
\newblock Mandel's problem revisited.
\newblock {\em Geotechnique}, 46:187--195, 1996.

\bibitem{aiken1985stiff}
R.~Aiken.
\newblock {\em Stiff Computation}.
\newblock Oxford University Press, 1985.

\bibitem{Almani2016_phd}
T.~Almani.
\newblock {\em Efficient Algorithms for Flow Models Coupled with Geomechanics
  for porous Media}.
\newblock PhD thesis, University of Texas at Austin, 2016.

\bibitem{Almani2016}
T.~Almani, K.~Kumar, A.~Dogru, G.~Singh, and M.~Wheeler.
\newblock Convergence analysis of multirate fixed-stress split iterative
  schemes for coupling flow with geomechanics.
\newblock {\em Computer Methods in Applied Mechanics and Engineering},
  311:180--207, 2016.

\bibitem{ALMANI20192682}
T.~Almani, K.~Kumar, G.~Singh, and M.~Wheeler.
\newblock Stability of multirate explicit coupling of geomechanics with flow in
  a poroelastic medium.
\newblock {\em Computers \& Mathematics with Applications}, 78(8):2682--2699,
  2019.

\bibitem{AlLeeWheWi17}
T.~Almani, S.~Lee, M.~F. Wheeler, and T.~Wick.
\newblock {Multirate Coupling for Flow and Geomechanics Applied to Hydraulic
  Fracturing Using an Adaptive Phase-Field Technique}.
\newblock {\em SPE Reservoir Simulation Conference}, Day 3 Wed, February 22,
  2017, 2017.

\bibitem{fenics2015}
M.~Alnæs, J.~Blechta, J.~Hake, A.~Johansson, B.~Kehlet, A.~Logg,
  C.~Richardson, J.~Ring, M.~E. Rognes, and G.~N. Wells.
\newblock The {FEniCS} {P}roject {V}ersion 1.5.
\newblock {\em Archive of Numerical Software}, Vol 3, 2015.

\bibitem{dealii2019design}
D.~Arndt, W.~Bangerth, D.~Davydov, T.~Heister, L.~Heltai, M.~Kronbichler,
  M.~Maier, J.-P. Pelteret, B.~Turcksin, and D.~Wells.
\newblock The {deal.II} finite element library: Design, features, and insights.
\newblock {\em Comput. Math. Appl.}, 81:407--422, 2021.

\bibitem{dealII94}
D.~Arndt, W.~Bangerth, M.~Feder, M.~Fehling, R.~Gassm{\"o}ller, T.~Heister,
  L.~Heltai, M.~Kronbichler, M.~Maier, P.~Munch, J.-P. Pelteret, S.~Sticko,
  B.~Turcksin, and D.~Wells.
\newblock The \texttt{deal.II} library, version 9.4.
\newblock {\em J. Numer. Math.}, 30(3):231--246, 2022.

\bibitem{BaGeRa10}
W.~Bangerth, M.~Geiger, and R.~Rannacher.
\newblock Adaptive {G}alerkin finite element methods for the wave equation.
\newblock {\em Comput. Methods Appl. Math.}, 10:3--48, 2010.

\bibitem{Bause2017}
M.~Bause, F.~Radu, and U.~Köcher.
\newblock Space–time finite element approximation of the biot poroelasticity
  system with iterative coupling.
\newblock {\em Comput. Methods Appl. Mech. Engrg.}, 320:745--768, 2017.

\bibitem{BaKeTe13}
Y.~Bazilevs, K.~Takizawa, and T.~Tezduyar.
\newblock {\em Computational Fluid-Structure Interaction: Methods and
  Applications}.
\newblock Wiley, 2013.

\bibitem{BeRa01}
R.~Becker and R.~Rannacher.
\newblock An optimal control approach to a posteriori error estimation in
  finite element methods.
\newblock {\em Acta Numerica, Cambridge University Press}, pages 1--102, 2001.

\bibitem{Biot41a}
M.~Biot.
\newblock Consolidation settlement under a rectangular load distribution.
\newblock {\em J. Appl. Phys.}, 12(5):426--430, 1941.

\bibitem{Biot41b}
M.~Biot.
\newblock General theory of three-dimensional consolidation.
\newblock {\em J. Appl. Phys.}, 12(2):155--164, 1941.

\bibitem{Biot55}
M.~Biot.
\newblock Theory of elasticity and consolidation for a porous anisotropic
  solid.
\newblock {\em J. Appl. Phys.}, 25:182--185, 1955.

\bibitem{Biot7172}
M.~Biot.
\newblock Theory of finite deformations of porous solids.
\newblock {\em Indiana Univ. Math. J.}, 21:597--620, 1971/72.

\bibitem{Borregales2019}
M.~Borregales, K.~Kumar, F.~A. Radu, C.~Rodrigo, and F.~J. Gaspar.
\newblock A partially parallel-in-time fixed-stress splitting method for
  biot’s consolidation model.
\newblock {\em Computers \& Mathematics with Applications}, 77(6):1466--1478,
  2019.
\newblock 7th International Conference on Advanced Computational Methods in
  Engineering (ACOMEN 2017).

\bibitem{BrKoeBau22}
M.~Bruchhäuser, U.~Köcher, and M.~Bause.
\newblock {On the Implementation of an Adaptive Multirate Framework for Coupled
  Transport and Flow}.
\newblock {\em J. Sci. Comput.}, 93(59):1--29, 2022.

\bibitem{Cheng88}
A.~H.-D. Cheng.
\newblock A direct boundary element method for plane strain poroelasticity.
\newblock {\em Int. J. Numer. Anal. Meth. Geomech.}, 12:551--572, 1988.

\bibitem{Coussy2004}
O.~Coussy.
\newblock {\em Poromechanics}.
\newblock Wiley, 2004.

\bibitem{Cr63}
C.~Cryer.
\newblock A comparison of the three-dimensional consolidation theories of biot
  and terzaghi.
\newblock {\em Q. J. Mech. Appl. Math.}, 16:401--412, 1663.

\bibitem{UMFPACK}
T.~A. Davis.
\newblock Algorithm 832: Umfpack v4.3---an unsymmetric-pattern multifrontal
  method.
\newblock {\em ACM Trans. Math. Softw.}, 30(2):196–199, jun 2004.

\bibitem{Dean2006}
R.~H. Dean, X.~Gai, C.~M. Stone, and S.~E. Minkoff.
\newblock {A Comparison of Techniques for Coupling Porous Flow and
  Geomechanics}.
\newblock {\em SPE Journal}, 11(01):132--140, 03 2006.

\bibitem{DeHaTr81}
M.~Delfour, W.~Hager, and F.~Trochu.
\newblock {Discontinuous Galerkin methods for ordinary differential equations}.
\newblock {\em Math. Comp.}, 36:455--473, 1981.

\bibitem{DoeWie22}
W.~D{\"o}rfler and C.~Wieners.
\newblock Space-time approximations for linear acoustic, elastic, and
  electro-magnetic wave equations.
\newblock Lectures Notes for the MFO seminar on wave phenomena, January 2022.

\bibitem{FiRoWiChaFau2023}
H.~Fischer, J.~Roth, T.~Wick, L.~Chamoin, and A.~Fau.
\newblock {MORe} {DWR}: Space-time goal-oriented error control for incremental
  {POD}-based {ROM}, 2023, \url{https://doi.org/10.48550/ARXIV.2304.01140}.

\bibitem{Gai2004}
X.~Gai.
\newblock {\em A Coupled Geomechanics and Reservoir Flow Model on Parallel
  Computers}.
\newblock Doctoral {Thesis}, University of Texas at Austin, 2004.

\bibitem{GaSqu06}
F.~Gazzola and M.~Squassina.
\newblock Global solutions and finite time blow up for damped semilinear wave
  equations.
\newblock {\em Ann. I. H. Poincar\'e}, 23:185--207, 2006.

\bibitem{Girault2011}
V.~Girault, G.~Pencheva, M.~Wheeler, and T.~Wildey.
\newblock Domain decomposition for {P}oroelasticity and {E}lasticity with {DG}
  jumps and mortars.
\newblock {\em Mathematical Models and Methods in Applied Sciences}, 21, 10
  2011.

\bibitem{Guzman2012}
H.~A.~F. Guzman.
\newblock {\em Domain Decomposition Methods in Geomechanics}.
\newblock Doctoral {Thesis}, University of Texas at Austin, 2012.

\bibitem{Hairer1996}
E.~Hairer and G.~Wanner.
\newblock {\em Solving Ordinary Differential Equations II. Stiff and
  Differential-Algebraic Problems}, volume~14 of {\em Series in Comput. Math.}
\newblock Springer Verlag, 01 1996.

\bibitem{HuHu88}
T.~J. Hughes and G.~M. Hulbert.
\newblock Space-time finite element methods for elastodynamics: Formulations
  and error estimates.
\newblock {\em Comput. Methods Appl. Mech. Engrg.}, 66(3):339 -- 363, 1988.

\bibitem{JaWheWi21}
M.~Jammoul, M.~F. Wheeler, and T.~Wick.
\newblock A phase-field multirate scheme with stabilized iterative coupling for
  pressure driven fracture propagation in porous media.
\newblock {\em Computers \& Mathematics with Applications}, 91:176--191, 2021.
\newblock Robust and Reliable Finite Element Methods in Poromechanics.

\bibitem{JaLo08}
J.~Jansson and A.~Logg.
\newblock Algorithms and data structures for multi-adaptive time-stepping.
\newblock {\em ACM Trans. Math. Softw.}, 35(3), oct 2008.

\bibitem{Joh88}
C.~Johnson.
\newblock Error estimates and adaptive time-step control for a class of
  one-step methods for stiff ordinary differential equations.
\newblock {\em SIAM J. Numer. Anal.}, 25(4):908--926, 1988.

\bibitem{JOHNSON1993117}
C.~Johnson.
\newblock Discontinuous galerkin finite element methods for second order
  hyperbolic problems.
\newblock {\em Comput. Methods Appl. Mech. Engrg.}, 107(1):117--129, 1993.

\bibitem{LaStein19}
U.~Langer and O.~Steinbach, editors.
\newblock {\em Space-time methods: {A}pplication to Partial Differential
  Equations}.
\newblock volume 25 of Radon Series on Computational and Applied Mathematics,
  Berlin. de Gruyter, 2019.

\bibitem{LeSchref99}
R.~W. Lewis and B.~Schrefler.
\newblock {\em The Finite Element Method in the Static and Dynamic Deformation
  and Consolidation of Porous Media, 2nd Edition}.
\newblock Wiley, 1999.

\bibitem{Liu04}
R.~Liu.
\newblock {\em Discontinuous Galerkin Finite Element Solution for
  Poromechanics}.
\newblock PhD thesis, The University of Texas at Austin, 2004.

\bibitem{Logg2003a}
A.~Logg.
\newblock Multi-adaptive galerkin methods for odes i.
\newblock {\em SIAM Journal on Scientific Computing}, 24(6):1879--1902, 2003.

\bibitem{Logg2003b}
A.~Logg.
\newblock Multi-adaptive galerkin methods for odes ii: implementation and
  applications.
\newblock {\em SIAM Journal on Scientific Computing}, 25(4):1119--1141, 2004.

\bibitem{Mandel1953}
J.~Mandel.
\newblock Consolidation des sols (Étude mathématique).
\newblock {\em Géotechnique}, 3(7):287--299, 1953.

\bibitem{MaKlWaGe15}
M.~Mayr, T.~Kl\"{o}ppel, W.~A. Wall, and M.~W. Gee.
\newblock A temporal consistent monolithic approach to fluid-structure
  interaction enabling single field predictors.
\newblock {\em SIAM Journal on Scientific Computing}, 37(1):B30--B59, 2015.

\bibitem{MeidnerRichter2014}
D.~Meidner and T.~Richter.
\newblock Goal-oriented error estimation for the fractional step theta scheme.
\newblock {\em Comput. Methods Appl. Math.}, 14(2):203--230, 2014.

\bibitem{Ri17_fsi}
T.~Richter.
\newblock {\em Fluid-structure interactions: models, analysis, and finite
  elements}.
\newblock Springer, 2017.

\bibitem{RoThiKoeWi2023}
J.~Roth, J.~P. Thiele, U.~K{\"o}cher, and T.~Wick.
\newblock Tensor-product space-time goal-oriented error control and adaptivity
  with partition-of-unity dual-weighted residuals for nonstationary flow
  problems.
\newblock {\em Computational Methods in Applied Mathematics}, 2023.

\bibitem{RYBAK2014327}
I.~Rybak and J.~Magiera.
\newblock A multiple-time-step technique for coupled free flow and porous
  medium systems.
\newblock {\em Journal of Computational Physics}, 272:327--342, 2014.

\bibitem{RyMaHeRoh15}
I.~Rybak, J.~Magiera, R.~Helmig, and C.~Rohde.
\newblock Multirate time integration for coupled saturated/unsaturated porous
  medium and free flow systems.
\newblock {\em Comput. Geosci.}, 19(2):299--309, 2015.

\bibitem{Saad2003}
Y.~Saad.
\newblock {\em Iterative methods for sparse linear systems}.
\newblock SIAM, 2003.

\bibitem{SAVCENCO2009323}
V.~Savcenco.
\newblock Construction of a multirate rodas method for stiff odes.
\newblock {\em Journal of Computational and Applied Mathematics},
  225(2):323--337, 2009.

\bibitem{SavHuVe07}
V.~Savcenco, W.~Hundsdorfer, and J.~G. Verwer.
\newblock A multirate time stepping strategy for stiff ordinary differential
  equations.
\newblock {\em BIT}, 47(1):137--155, 2007.

\bibitem{SCHLEGEL2009345}
M.~Schlegel, O.~Knoth, M.~Arnold, and R.~Wolke.
\newblock Multirate runge-kutta schemes for advection equations.
\newblock {\em Journal of Computational and Applied Mathematics},
  226(2):345--357, 2009.
\newblock Special Issue: Large scale scientific computations.

\bibitem{SchVe08}
M.~Schmich and B.~Vexler.
\newblock Adaptivity with dynamic meshes for space-time finite element
  discretizations of parabolic equations.
\newblock {\em SIAM J. Sci. Comput.}, 30(1):369 -- 393, 2008.

\bibitem{ShZhLa13}
L.~Shan, H.~Zheng, and W.~J. Layton.
\newblock {A decoupling method with different subdomain time steps for the
  nonstationary Stokes-Darcy model}.
\newblock {\em Numerical Methods for Partial Differential Equations},
  29(2):549--583, 2013.

\bibitem{Showalter2000}
R.~Showalter.
\newblock Diffusion in poro-elastic media.
\newblock {\em Journal of Mathematical Analysis and Applications},
  251(1):310--340, 2000.

\bibitem{SOCHALA20092122}
P.~Sochala, A.~Ern, and S.~Piperno.
\newblock {Mass conservative BDF-discontinuous Galerkin/explicit finite volume
  schemes for coupling subsurface and overland flows}.
\newblock {\em Comput. Methods Appl. Mech. Engrg.}, 198(27):2122--2136, 2009.

\bibitem{Soszynska_phd}
M.~Soszy{\'{n}}ska.
\newblock {\em Temporal Multiscale Simulations for Multiphysics Problems}.
\newblock PhD thesis, University of Magdeburg, 2023.

\bibitem{Soszynska2021}
M.~Soszy{\'{n}}ska and T.~Richter.
\newblock Adaptive time-step control for a monolithic multirate scheme coupling
  the heat and wave equation.
\newblock {\em BIT Numerical Mathematics}, 61(4):1367--1396, Dec 2021.

\bibitem{TeBeLi92}
T.~Tezduyar, M.~Behr, and J.~Liou.
\newblock A new strategy for finite element computations involving moving
  boundaries and interfaces - the deforming-spatial-domain/space-time
  procedure: I. the concept and the preliminary numerical tests.
\newblock {\em Comp. Methods Appl. Mech. Engrg.}, 94:339--351, 1992.

\bibitem{TeBeMiJo92}
T.~E. Tezduyar, M.~Behr, S.~Mittal, and A.~A. Johnson.
\newblock {\em Computation of Unsteady Incompressible Flows With the Finite
  Element Methods Space-Time Formulations, Iterative Strategies and Massively
  Parallel Implementations}, volume 143 of {\em New Methods in Transient
  Analysis, PVP-Vol. 246, AMD-Vol. 143}, pages 7--24.
\newblock ASME, New York, 1992.

\bibitem{To92}
I.~Tolstoy.
\newblock {\em Acoustic, elasticity, and thermodynamics of porous media,
  Twenty-one papers by M.A. Biot}.
\newblock Acoustical Society of America, New York, 1992.

\bibitem{DuiMiWi22}
C.~J. van Duijn, A.~Mikeli\'{c}, and T.~Wick.
\newblock Mandel's problem as a benchmark for two-dimensional nonlinear
  poroelasticity.
\newblock {\em Appl. Anal.}, 101(12):4267--4293, 2022.

\bibitem{WheWiLee20}
M.~F. Wheeler, T.~Wick, and S.~Lee.
\newblock {IPACS: Integrated Phase-Field Advanced Crack Propagation Simulator.
  An adaptive, parallel, physics-based-discretization phase-field framework for
  fracture propagation in porous media}.
\newblock {\em Comput. Methods Appl. Mech. Engrg.}, 367:113124, 2020.

\bibitem{Wick2020PFF}
T.~Wick.
\newblock {\em Multiphysics Phase-Field Fracture: Modeling, Adaptive
  Discretizations, and Solvers}.
\newblock De Gruyter, Berlin, Boston, 2020.

\end{thebibliography}



\end{document}



