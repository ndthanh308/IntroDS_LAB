\section{Introduction}
\IEEEPARstart{I}{n} recent years, advanced generative (vision) models have revolutionized various fields such as art creation, design, and human-computer interaction~\cite{nichol2021glide, ramesh2022hierarchical, rombach2022high, mj}. Despite their positive impact, these models have also given rise to new concerns, such as copyright infringement issues and content supervision. To address these concerns, model attribution, the process of identifying the source model of generated content, has gained increasing attention \cite{marra2019gans,yu2019attributing,xuan2019scalable,yang2022aaai,bui2022repmix, yang2023progressive}. It helps deter unauthorized copying and distribution, enabling content creators and rights holders to prove ownership and take legal action against infringements. Furthermore, model attribution allows regulators to identify and act against entities using generative models for harmful, illegal, or unethical purposes.
% This challenge becomes increasingly pressing as the number of newly-emerged generative models rapidly increases in the open world. 

% Figure environment removed

\section{Introduction}
Automatic 3D reconstruction of clothed humans using image inputs has gained increasing significance due to its potential applications in a wide array of AR/VR scenarios. High-fidelity reconstructions typically depend on sophisticated capture systems, which are developed with dense camera arrays~\cite{collet2015high,joo2015panoptic,joo2018total}, programmable light-stages~\cite{Vlasic2009, guo2019relightables}, and depth sensors~\cite{newcombe2011kinectfusion,DoubleFusion,BodyFusion,dou2016fusion4d,newcombe2015dynamicfusion}. However, stringent capture environments equipped with complex hardware pose significant challenges for consumer-level applications.


In this context, considerable research effort has been dedicated to developing methods that allow for more flexible capture configurations, such as utilizing a few RGB inputs. Among these works, learning implicit functions \cite{iccv2020PIFu, saito2020pifuhd, hong2021stereopifu} has proven effective in achieving highly detailed reconstructions by integrating the advancements of deep neural networks. These methods employ large multi-layer perceptrons (MLPs) to predict the occupancy probability or truncated signed distance function (TSDF) value of every queried 3D point based on its associated local feature, which is extracted from images. They can recover a continuous surface at arbitrary resolutions without topology restrictions.


However, in typical MLP-based implicit networks, the occupancy or TSDF value at each location is solved independently with planar image features, rendering them less capable of addressing challenging cases such as occlusions. Consequently, these methods suffer from generalization and robustness issues, particularly when tackling strong occlusions caused by large motion or multiple interacting humans. 
Some follow-up studies  \cite{zheng2021deepmulticap,zheng2021pamir,huang2020arch} utilize an extra geometric model, SMPL~\cite{Loper2015}, to improve robustness by introducing strong shape priors. 
Their success typically relies on the assumption of geometrical similarity \cite{huang2020arch} between the shape prior and target reconstruction, making them intractable for handling complex cases with loose clothes and sensitive to errors in SMPL model fitting.



%\ping{this paragraph sounds like `TSDF is better than MLP/SMPL, and we use TSDF to solve the problem'. But in Sec 3, we are telling a different story, saying `MLP needs a 3D convolutional encoder'. We need to make these two sections consistent.}\sicong{I think in this paragraph we claim that the TSDF}


%We opt for Trucated Signed Distance Funtion (TSDF) volumetric representations as they are naturally suitable for convolution operations, which have shown remarkable performance for learning hierarchical features on 2D visual perception tasks \cite{SunXLW19}. 
%Meanwhile, TSDF also describes the gradual geometry change around shape surface, which is not reflected by occupancy volume. 

We instead revisit the 3D volumetric representation and resort to 3D convolutional neural networks (CNNs) for feature learning, due to their impressive performance in feature learning and the ability to incorporate spatial context. However, volumetric methods and 3D convolution involve discretization, which might raise concerns regarding whether a discretized volume can preserve subtle geometric details as continuous representations learned in implicit functions. We investigate the relationship between volume resolution and quantization error on synthetic data by converting target mesh objects to TSDF volumes, as shown in Figure~\ref{fig:quantization_error}. We observe that the quantization errors are significantly reduced by increasing volume resolution and become nearly negligible when reaching a relatively high resolution (e.g., 512 or higher). In other words, achieving fine-detailed reconstruction is not supposed to be restricted by the use of volume representations as long as a proper volume resolution is utilized. Therefore, we present a method with high-resolution feature volumes, e.g., 256 and 512, while traditional volumetric methods \cite{varol18_bodynet,gilbert2018volumetric} are often limited to much lower resolutions, such as 32 or 128.



On the other hand, an increase in volume resolution may lead to a cubic growth of memory overhead \cite{8100085}. Reducing memory costs while guaranteeing the granularity of volumetric representations is necessary for pursuing high-quality reconstruction. Thus, we adopt a coarse-to-fine approach and cull away irrelevant voxels to build a sparse high-resolution feature volume. At the coarse level, the network computes an initial TSDF by applying a U-Net with sparse 3D CNN \cite{3DSemanticSegmentationWithSubmanifoldSparseConvNet} on the sparse feature volume, which is carved by a visual hull. Through our experiments, it turns out that more than 95\% of the volume grids are discarded by the visual hull culling, making the sparse 3D CNN efficient. At the fine level, the network focuses on a narrow band near the zero-level set of the initial TSDF and discretizes the narrow band with smaller voxels. By employing this narrow-band culling, we further shrink the sampling space, resulting in a relatively small range of grid numbers (usually 300K--500K in our experiments) even with a high volume resolution of 512. The remaining voxels in the narrow band are associated with features that fuse high-frequency information from the computed normal maps upon the low-frequency shape from the coarse level to compute the TSDF at high resolution. The final mesh is then extracted from the TSDF using the Marching-Cube algorithm ~\cite{Lorensen87marchingcubes}.
% Different from the u-net sturcture to preserve global topology context, we then apply a shallow 3dcnn to compute the final TSDF $D_{final}$ which contain more local geometry detail.




% \ping{this paragraph can be expanded. It is an important contribution and often ignored by other works. stress on the novel idea of regressing blending weights instead of colors}

In addition to geometry, high-quality mesh texture is also a crucial factor contributing to visual appearance. Directly computing a color field in 3D space, as in \cite{iccv2020PIFu}, struggles to capture high-frequency texture details, while the neural radiance field (NeRF) \cite{yu2020pixelnerf} or the DoubleField~\cite{shao2022doublefield} require expensive per-instance optimization and are often unstable for sparse input images. In contrast, we adopt an image-based rendering approach to compute a texture atlas map, which is efficient and widely supported in existing computer graphics tools. 
Specifically, we compute a blending weight at each 3D point on the mesh surface to determine its color as a weighted average of the colors at its image projections. The blending weights can be computed at a relatively coarse resolution, e.g., 512 volume resolution in our case, and leave texture details to the high-resolution images, such as 1K or 2K. Unlike previous methods that generate blurry texturing results under sparse input, our method generalizes well on both synthetic and real data with just a few input views. 
Figure~\ref{fig:teaser} shows two examples reconstructed by our method. Despite the challenging garment, pose, and occlusion, our method recovers faithful shape, normal, and texture on the right.

%with a wide variety of poses and clothing styles, and it is also adaptive to handle input image with arbitrary resolutions.
%\sicong{For this concern we claim that when the resolution of dicretized volume meets certain threshold (which is 256 in our experiment), the quantization error can be neglected.} 



In summary, the main contributions of this paper are as follows:
\begin{itemize}
\vspace{-0.1in}
  \item 
  We revisit the 3D volumetric representation and demonstrate that it can support clothed human reconstruction with equal or even better performance compared to implicit representation. 
  \item 
  We develop a memory and computation-efficient method for high-resolution volumetric reconstruction using sophisticated sparse 3D CNN, coarse-to-fine estimation, and voxel culling by visual hull and narrow bands. 
  \item 
  We introduce a novel method to compute a texture atlas map, which captures rich appearance details from high-resolution input images.
  \item 
  We achieve impressive results on standard benchmark datasets Twindom and MultiHuman, significantly reducing the point-2-surface (P2S) precision to approximately 0.2cm from just six input views, with more than $50\%$ error reduction compared to the state-of-the-art methods, including DoubleField~\cite{shao2022doublefield} and PIFuHD~\cite{saito2020pifuhd}.
\end{itemize}

Existing research seeks to identify the unique fingerprints on the images they generate, which can be leveraged to attribute a generated image to its source model. A commonly adopted model attribution paradigm frames the task as a multi-class classification problem~\cite{yu2019attributing, yang2022aaai, bui2022repmix, yang2023progressive}. In this setup, images generated by a limited and static set of models are used to train a classifier, where each image is labeled with a unique model ID. During testing, the test image comes from a seen model predefined in the training set, and the classifier would identify the model ID of the image based on the model fingerprint it has learned~\cite{yu2019attributing, yang2022aaai, bui2022repmix}. However, there arises a scenario where a testing image originates from an unseen model not present during training. To tackle this, some methodologies~\cite{yang2023progressive, abady2024siamese} adopt an open-set setup, assigning them an ``unknown" label. 

Despite these efforts, conventional methods, trained on a static set of seen models, struggle to adapt to the newly emerged unseen models dynamically. This limitation is exemplified in Figure~\ref{fig:intro}(a), which presents the t-SNE feature extraction results from an existing method~\cite{yang2023progressive}. We train this method on classical models such as GAN, VAE, and Flow models~\footnote{Include all the GAN, VAE, and Flow models in Table~\ref{tab:dataset}}, and test on both these and the emerging Diffusion models. The results reveal a significant generalization gap when the method encounters new types of generative models. This gap stems from the restricted scope of seen models that can be practically collected, whose fingerprint distributions may significantly differ from those of newly emerging models. As generative technology continues to evolve, the variety of unseen models expands, further challenging the capacity of existing methods to adapt dynamically. 

To bridge this gap, our goal is to develop a more generalized model fingerprint extractor capable of attributing unseen models efficiently in a zero-shot setting. Central to our approach is a model synthesis technique that creates plenty of synthetic models that mimic a broad range of fingerprint patterns in real-world generative models. To gain insights for replicating these patterns, our initial step is to explore why generative models display unique fingerprint patterns, which, although typically imperceptible in the spatial domain, manifest as distinct spectral patterns in high-frequency components. Through analyzing the architecture and parameters of generative models, we observe: 1) The type of basic network components such as upsampling, activation function, normalization, and parameters of convolution layers are key factors influencing the spectral patterns of the generated images. 2) Across the generative blocks within a large generative model, due to the upsampling layer that naturally attenuates earlier layers' high-frequency components, the last few generative blocks are more influential in determining the output's spectrum patterns.

We incorporate these observations into our model synthesis technique. Our approach is based on a shallow auto-encoder architecture, comprising fewer generative blocks compared to typical generative models. According to the discussion above, employing a synthesis architecture with a small number of generative blocks could ensure fidelity in mimicking fingerprints. We then increase the architecture diversity by varying the types of upsampling layers, activation functions, normalization layers, and the number and sequence of these layers. Additionally, we enhance the parameter diversity by altering training seeds. Consequently, this approach allows us to get 5760 synthetic models across 288 different architectures by minimizing the reconstruction loss. Owing to the simple architecture and training objective, training these synthetic models requires much less time than real generative models. Subsequently, we leverage these synthetic models to train the fingerprint extractor, to extract fingerprints from these models, and distinguish between each other. We combine classification and metric loss to enhance the discrimination of learned fingerprint embedding. 

% In this section, we introduce the model synthesis strategy that we use to create models for training the fingerprint feature extractor. Intuitively, a good synthesis strategy should generate models whose fingerprint patterns closely resemble those of real generative models, allowing these synthetic models to effectively simulate new models in an open environment. According to the discussion in Section~\ref{subsec:warm_up}, there are important factors to consider. First, the last few blocks tend to have a more profound influence on the distinguishable spectrum patterns of the output images. Thus, employing a synthesis architecture with a small number of generative blocks could ensure fidelity in mimicking fingerprints. Second, it is crucial to increase diversity by varying the types of upsampling layers, activation functions, normalization layers, and parameters. We explicitly incorporate these considerations into our method. Moreover, to further enhance diversity, we also consider variations in the number and sequence of layers.  

We establish two metrics to verify the effectiveness of our model synthesis strategy in terms of \textit{fidelity} and \textit{diversity}. The former assesses the authenticity of fingerprint pattern mimicking by comparing the spectrum distribution between synthetic and real models, particularly in high-frequency components. The latter evaluates the diversity of spectrum patterns of synthetic models by examining the extent to which the synthetic models occupy the spectrum representation space. The results indicate that the synthesis options we designed indeed enhance the distribution fidelity and diversity of spectrum patterns.

Experimental results demonstrate that our fingerprint extractor, despite being solely trained on synthetic models, exhibits strong zero-shot attribution capabilities on a broad scope generative models, including the classical GAN, VAE, Flow models, and emerging Diffusion models like Stable Diffusion and DalleE-3. Partial visualization results are shown in Figure~\ref{fig:intro}(b). In the model attribution scenarios, including model identification and verification on unseen models, our method significantly outperforms existing approaches, achieving accuracy improvements of over 40\% and 15\% respectively.

Our major contributions can be summarized as follows:

1) We propose to address a practical problem in applying model attribution in the real world, to expand the attribution target to unseen models in a zero-shot setting.

2) We propose to solve the zero-shot model attribution problem based on training on numerous synthetic models, which mimic the fingerprint patterns of real-world generative models. The synthesis strategy is motivated by observations on how generative model architecture building blocks and parameters influence fingerprint patterns and is validated through two designed metrics that examine synthetic models' fidelity and diversity.


3) Experimental results demonstrate that our fingerprint extractor, trained on synthetic models, exhibits strong generalization capabilities on a wide range of real-world generative models. Compared with existing methods, we improve the model identification and verification accuracy on unseen models by over 40\% and 15\% respectively.

% \inputfigure{generalization}
