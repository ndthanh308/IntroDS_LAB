% \section{Limitations and Social Impact}


\section{Conclusions and future work}

By analyzing generative network components in the frequency domain, we make the first attempt to identify sources for fingerprints of generative models. Utilizing the gained priors in a simulation strategy, we consequently improve the generality of fingerprint extraction in a range of model attribution scenarios. Our work opens the door to improving model fingerprint extraction based on synthetic data. For future works, it is promising to model the complexity and diversity of model fingerprints more accurately while balancing computational costs.

% . The value of and moving toward solving the model attribution in the open-world. 
% In addition, our work raises further questions for future research. How can we model the complexity of model fingerprints more accurately while balancing computational costs? Can model fingerprints provide additional insights for measuring the similarity and identifying the specific relationships between generative models, such as pruning and distillation? How to formulate unified features from different generative models for more generalized real/fake discrimination? Answering these questions supports harnessing positive aspects of generative technologies. 

% We make the first attempt to investigate the formation of model fingerprints of CNN-based generative models in the frequency domain. \\
% $\bullet$ We pioneer solving the known-unknown-unbalanced model attribution problem by pre-training on numerous low-cost synthetic data. \\ 
% $\bullet$ Extensive evaluations validate that the pre-trained fingerprint extractor shows superior transferability in verifying, identifying, and analyzing the relationship of real CNN-based generative models.

