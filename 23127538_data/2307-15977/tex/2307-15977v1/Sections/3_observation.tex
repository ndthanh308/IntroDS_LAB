\section{Frequential fingerprints of generative models}

% As indicated by the red circles, t
Model fingerprints, which serve as unique characteristics in generated images that differentiate different models, despite having been found to exist, have received little understanding regarding where they exist. The primary reason is they are visually imperceptible in the spatial domain. We begin to analyze them by observing the spectra of images generated by various generative models. See Fig.~\ref{fig:intro}, two distinct features could be observed on the spectra: 1) Grid-like patterns exhibiting varying strengths and periodicity (highlighted by yellow circles). 2) Frequency distribution across the entire spectrum, where the magnitudes of high and low-frequency components differ among the spectra. However, the underlying causes of these frequency patterns have received little attention thus far. In the following sections, we aim to provide empirical explanations for the formation of these patterns through an analysis of the generation process in the frequency domain.

