
\begin{abstract}
  % The abstract paragraph should be indented \nicefrac{1}{2}~inch (3~picas) on
  % both the left- and right-hand margins. Use 10~point type, with a vertical
  % spacing (leading) of 11~points.  The word \textbf{Abstract} must be centered,
  % bold, and in point size 12. Two line spaces precede the abstract. The abstract
  % must be limited to one paragraph.
% investigate the cause of model fingerprints.
It is verified in existing works that CNN-based generative models leave unique fingerprints on generated images. There is a lack of analysis about how they are formed in generative models. Interpreting network components in the frequency domain, we derive sources for frequency distribution and grid-like pattern discrepancies exhibited on the spectrum. These insights are leveraged to develop low-cost synthetic models, which generate images emulating the frequency patterns observed in real generative models. The resulting fingerprint extractor pre-trained on synthetic data shows superior transferability in verifying, identifying, and analyzing the relationship of real CNN-based generative models such as GAN, VAE, Flow, and diffusion.

% These observations are leveraged to build low-cost synthetic models with similar frequency patterns as real generative models. 
% We then analyze the fingerprint decay phenomenon in the feature propagation process. This leads to a 
% Our results pave the way to identify  
% \ie, how uniform frequency patterns are produced and transformed by network components. \yty{we find that }. 
% in model verification and open-set model identification scenarios. 
\end{abstract}