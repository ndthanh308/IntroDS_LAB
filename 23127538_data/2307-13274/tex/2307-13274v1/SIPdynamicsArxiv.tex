%preamble
%\documentclass[aps, prb, reprint, floatfix, superscriptaddress]{revtex4-2}
\documentclass[aps, prb, twocolumn, superscriptaddress, showpacs, floatfix]{revtex4-2}

\pdfminorversion=6
%%%%%%%%%%%%%%%%%%%%%%%%%%%%%%%%%%%%%%%%%%%%%
%packages
\usepackage{amsmath,amsbsy,amssymb,amsfonts}
\usepackage{microtype}

\usepackage{graphicx, color}
\usepackage{natbib}
%\usepackage{empheq}
\usepackage{hyperref}
\usepackage[capitalise]{cleveref}
\usepackage{subfigure}
\usepackage{float}

%\usepackage[normalem]{ulem}
\usepackage{soul,xcolor}
%\usepackage{marginnote}
%%%%%%%%%%%%%%%%%%%%%%%%%%%%%%%%%
%new commands
\newcommand{\Matr}[1]{\boldsymbol{\mathcal{\hat{#1}}}}
\newcommand{\crn}[1]{\hat{#1}^{\dagger}}%creation operator
\newcommand{\anh}[1]{\hat{#1}}%annihilation operator
\newcommand{\eps}{\epsilon}
\newcommand{\vc}[1]{\pmb{#1}}%bold vector notation including greeck letters
\newcommand{\angstrom}{\text{\normalfont\AA}}%Angstrom symbol in math mode
\newcommand{\vecop}[1]{\hat{\pmb{#1}}}%vector operator
\newcommand{\td}[1]{\tilde{#1}}
\newcommand{\tp}[1]{\textcolor{red}{#1}}
\newcommand{\tb}[1]{\textcolor{blue}{#1}}
\newcommand{\im}[0]{\mathrm{i}}
\newcommand{\ra}[1]{\textcolor{green}{#1}}
\newcommand{\no}[1]{\textcolor{red}{#1}}
\newcommand{\tx}[1]{\textcolor{red}{\textst{#1}}}
\newcommand{\isep}{\mathrel{{.}\,{.}}\nobreak}

\begin{document}
\setstcolor{red}
\graphicspath{{Reference_Figures/}}
%=========================================================
%title
\date{\today}

\title{Nonlinear Wavepacket Dynamics in Proximity to a Stationary Inflection Point}

\author{Serena Landers}
\affiliation{Wave Transport in Complex Systems Lab, Department of Physics, Wesleyan University, Middletown, CT-06459, USA}
\author{Arkady Kurnosov}
\email{akurnosov@wesleyan.edu}
\affiliation{Wave Transport in Complex Systems Lab, Department of Physics, Wesleyan University, Middletown, CT-06459, USA}
\author{William Tuxbury}
\affiliation{Wave Transport in Complex Systems Lab, Department of Physics, Wesleyan University, Middletown, CT-06459, USA}
\author{Ilya Vitebskiy}
\affiliation{Air Force Research Laboratory, Wright-Patterson AFB, OH, USA}
\author{Tsampikos Kottos}
\affiliation{Wave Transport in Complex Systems Lab, Department of Physics, Wesleyan University, Middletown, CT-06459, USA}

%=========================================================
%=========================================================
%abstract
\begin{abstract}
A stationary inflection point (SIP) in the Bloch dispersion relation of a periodic waveguide is an exceptional point degeneracy where three Bloch eigenmodes coalesce forming the so-called frozen mode with a divergent amplitude and vanishing group velocity of its propagating component. We have developed a theoretical framework to study the time evolution of wavepackets centered at an SIP. Analysis of the evolution of statistical moments distribution of linear pulses shows a strong deviation from the conventional ballistic wavepacket dynamics in dispersive media. The presence of nonlinear interactions dramatically changes the situation, resulting in a mostly ballistic propagation of nonlinear wavepackets with the speed and even the direction of propagation essentially dependent on the wavepacket amplitude. Such a behavior is unique to nonlinear wavepackets centered at an SIP.
\end{abstract}

\maketitle
%=========================================================
%=========================================================
The Bloch dispersion relation (BDR) of a periodic waveguide can develop exceptional points of degeneracy (EPD), where two or more Bloch eigenmodes coalesce. A well-known example is a regular band edge where two counter propagating Bloch modes collapse onto each other. Our investigation focuses on a stationary inflection point (SIP), where three Bloch eigenmodes (two evanescent and one propagating) coalesce (see \cite{Figotin2006,Figotin2011,Li2017,Tuxbury2022,Nada2021,Furman2023,HerreroParareda2022,Figotin2003} and references therein). In proximity to the SIP frequency, an incident wave can be completely converted into the frozen mode with diverging amplitude and vanishing group velocity of its propagating component \cite{Figotin2011,Li2017,Tuxbury2022,Figotin2003,Ballato2005}. The frozen mode regime is quite different from a common cavity resonance because its frequency is independent of the system dimensions and boundary conditions. The most remarkable features of the frozen mode regime include robustness with respect to structural imperfections and moderate losses \cite{Li2017,Tuxbury2022,Tuxbury2021,Gan2019}. The above properties make the frozen mode regime particularly attractive for the enhancement of various light-matter interactions, light amplification, and cavity-less lasing \cite{Ramezani2014,Yazdi2017,HerreroParareda2023}. 

The focus of this study is the unique dynamics of an SIP-centered wavepacket inside a periodic structure. Unlike the monochromatic frozen mode which involves non-Bloch Floquet eigenmodes \cite{Figotin2006,Figotin2011,Li2017,Tuxbury2022,Nada2021,Furman2023,HerreroParareda2022,Figotin2003}, the Gaussian wavepacket is a superposition of propagating Bloch modes with wavenumbers close to that of the SIP. Due to the SIP proximity, both the group velocity and its first derivative with respect to the Bloch wavenumber are infinitesimally small. As a consequence, both linear and nonlinear dynamics of an SIP-centered wavepacket demonstrate some interesting and unique features. Indeed, in the linear regime, the time evolution of the SIP-centered wavepacket does not involve ballistic propagation, which can be expected due to the zero group velocity at the SIP frequency. Remarkably though, the presence of nonlinearity changes the situation dramatically. We show that the SIP-centered nonlinear wavepackets can propagate ballistically with the speed and even the direction of propagation essentially dependent on the wavepacket amplitude. 
%=========================================================

%=========================================================
%%%%%%%%%%%%%%%%%%%%%%%%%%%%%%%%%%%%%%%%%%%%%%%%%%%
% Figure environment removed
%%%%%%%%%%%%%%%%%%%%%%%%%%%%%%%%%%%%%%%%%%%%%%%%%
\textit{Linear Dynamics}. For demonstration purposes we consider a minimal mathematical model which may support an SIP. It is provided by the temporal coupled mode theory equations
\begin{equation}\label{Eq:DynamicsX-spaceLinear}
i\frac{d\psi_{n}}{dt}  = - J\left(\psi_{n+1} + \psi_{n-1}\right) - J_{3}\left(\psi_{n+3} + \psi_{n-3}\right),
\end{equation} 
where $\psi_n(t)$ is the field amplitude at mode (site) $n = 1, \dots, N$. This model captures all of the features of SIP dynamics, and it can be also associated with a phenomenological description of a physical system. Some of the known examples of photonic setups which exhibit SIPs are shown in \cref{Fig:IntroFigure}. SIP-based systems can be also implemented in the acoustic metamaterial framework \cite{Chen2021}.

 In the case that \cref{Eq:DynamicsX-spaceLinear} describes a set of $N$ coupled resonators with (third-)nearest neighbor coupling constant $(J_3) J$, the variable $t$ indicates time. The same equation might also be used to describe the paraxial field propagation in multicore optical fibers. In this case, $t$ describes the paraxial propagation distance. Furthermore, without loss of generality, we have assumed periodic boundary conditions, i.e., $\psi_{n+N} = \psi_{n}$. \cref{Eq:DynamicsX-spaceLinear} may be decoupled in the Bloch mode representation (see \cref{Sec:ModelSetupAppendix} for details) with corresponding dispersion relation  
\begin{equation}\label{Eq:SIPspectrum}
\omega(q) = \varepsilon + 2J\cos q + 2J_{3}\cos3q,
\end{equation} 
that supports SIPs at $\bar{q} = \pm\pi/2$ for $J_{3}  = J/3$, see \cref{Fig:IntroFigure}~(d).

The dynamical equations should be supplemented with an initial condition; in the present study, we always assume preparation of a Gaussian packet in $q$-space, $\bar{\phi}(q)\propto\exp\left\{-(q - \bar{q})^{2}/2\sigma^{2}\right\}$,
where we assume the packet well confined to the first Brillouin zone, $\sigma\ll 2\pi$, and $\bar{q}$ is the reciprocal lattice vector associated with one of the SIPs. It can be shown that such initial condition implies a preparation of Gaussian packet of width $\sigma^{-1}$ in direct space. Assuming the initial wavepacket is centered at $n = 0$,  the time-dependence of the amplitude on the $n$-th site is given by
\begin{equation}\label{Eq:psinLinear}
\psi_{n}(t) \propto\int\limits_{-\pi}^{+\pi}dq e^{iqn}e^{-i\omega(q)t}\bar{\phi}(q),
\end{equation} 
where we have exploited the solution $\phi(q, t) = \exp\left\{i\omega(q) t\right\}\bar{\phi}(q)$ in $q$-space and the condition $N\gg 1$. This integral in \cref{Eq:psinLinear} can be evaluated analytically by employing a number of reasonable approximations. First, the fast convergence of the integral may be exploited by replacing the limits of integration: $\pm\pi\to\pm\infty$. Second, we can use a Taylor expansion of the dispersion relation $\omega(q)$, \cref{Eq:SIPspectrum}, in vicinity of $\bar{q}$, for which the cubic nature of the SIP gives
\begin{equation}\label{Eq:OmegaExpansionSIP}
\omega(q) \approx \omega_{0} + \frac{\alpha}{3}(q - \bar{q})^{3}, \, \alpha = \frac{1}{2}\frac{d^{3}\omega}{dq^{3}}\Big|_{q = \bar{q}}.
\end{equation} 
A virtue of this approximation transcends a mathematical simplification. Indeed, after utilizing it,  the validity of theoretical conclusions are independent of peculiarities present in the specific model  \cref{Eq:DynamicsX-spaceLinear},  as \cref{Eq:OmegaExpansionSIP} is applicable for any system featuring SIPs. Moreover, the dynamics are determined only by the parameters $\alpha$ and $\sigma$. For the present model,  the parameters we have introduced are given by  $\omega_{0} = \omega(\bar{q}) = \varepsilon \equiv 0$, $\alpha = 8J$, for $\bar{q} = -\pi/2$. 

Using the approximations we have introduced one can rewrite \cref{Eq:psinLinear} as 
\begin{multline}\label{Eq:psinstep2}
\psi_{n}(t) = \sqrt{2}\pi^{-3/4}\sigma^{-1/2}(\alpha t)^{-1/3}e^{-i\omega_{0}t}e^{i\bar{q}n}\times \\
\int\limits_{0}^{\infty}dx e^{-\epsilon x^{2}}\cos\left(\frac{x^{3}}{3} - zx\right),
\end{multline}
where $z = n(\alpha t)^{-1/3}$, $\epsilon = (1/2)(\alpha t)^{-2/3}\sigma^{-2}$. It can be shown (see \cref{Sec:IntegralAppendix}) that for $t\gg \alpha^{-1}(2\sigma^{2})^{3/2}$, the intensity $P_{n}(t) = |\psi_{n}(t)|^{2}$ on the $n$-th site takes the approximate form,
\begin{equation}\label{Eq:SIPsolution}
P_{n}(t) = 2\sqrt{\frac{\pi}{\sigma^{2}}}(\alpha t)^{-2/3}e^{-\frac{n}{\alpha\sigma^{2}t}}\mathrm{Ai}^{2}\left[-n(\alpha t)^{-1/3}\right],
\end{equation}
where $\mathrm{Ai}(-z)$ is Airy function \cite{AbramowitzStegun}. The numerically-evaluated intensity using \cref{Eq:DynamicsX-spaceLinear} as 
a function of position is reported in \cref{Fig:PvsX-MeanVSt} for $t = 0$ and $t_{2}>t_{1}\gg 0$ (solid lines), while the black dashed lines correspond to the analytical expression \cref{Eq:SIPsolution}.   

The solution we have derived corresponds to a forward propagation of the wavepacket as $P_{n}(t)$ decays quickly for $n<0$ due to the asymptotic behavior of the Airy function; it would be the opposite direction had we prepared the initial packet at the symmetric position in $q$-space, i.~e. at $+\pi/2$, where $\alpha<0$.



To characterize the wavepacket propagation, a good observable is the energy flow, $\mathcal{F}(t) = \sum_{n}n\dot{P}_{n}$, which is equivalent to a time-derivative of the first moment, $\langle n(t)\rangle$. Using \cref{Eq:SIPsolution} one can find the flow of the linear SIP dynamics to be   $\mathcal{F}_{\rm SIP} = {\sigma^{2}\alpha}/{2}$ (see \cref{Sec:FlowAppendix}). 

The same result can be obtained by observing an equality of the flow to the average group velocity, 
\begin{equation}\label{Eq:FlowGroupVel}
\mathcal{F}(t) = \langle v_{g}(t)\rangle = \int dq\left(\frac{\partial\omega}{\partial q}\right)|\phi(q, t)|^{2}.
\end{equation}
This equation remains a good approximation  in the presence of weak nonlinearity (see \cref{Sec:FlowVgProofAppendix}), and will be helpful for an explanation of a transition to ballistic transport and other nonlinear dynamical effects. 


 It is possible to consistently single out the anomalous transport features associated with the presence of the SIP in the framework  of the present model. 
 Assuming the long range coupling in \cref{Eq:DynamicsX-spaceLinear}  to be zero, $J_{3} = 0$, then $q=-\pi/2$ corresponds to an ordinary inflection point (OIP), $\omega^{\prime\prime}(q = -\pi/2)=0$, $\omega^{\prime}(q = -\pi/2)\neq 0$, such that the linear term of the $\omega(q)$-expansion dominates in its vicinity, as opposed to the cubic as was the case for an SIP.
 Therefore, \cref{Eq:psinLinear} is evaluated using the expansion $\omega(q) \approx \omega_{0} + v(q +\pi/2)$ instead of \cref{Eq:OmegaExpansionSIP}, where $v = \omega^{\prime}(q = -\pi/2) = 2J$ is the group velocity. 
Under the conditions of an OIP, integration of \cref{Eq:psinLinear} results in a direct space Gaussian packet of width $\sigma^{-1}$, propagating at constant velocity $v$. It can be shown that the flow associated with an OIP,  $\mathcal{F}_{\rm OIP} = \langle v_{g}\rangle =v$, does not depend on the initial wavepacket width, $\sigma$, which constitutes  ballistic propagation as opposed to SIP transport. 
Using explicit expressions for $v$ and $\alpha$ we can see $\mathcal{F}_{\rm SIP}/\mathcal{F}_{\rm OIP} = 2\sigma^{2}$. For instance, the value of $\sigma = 0.1$ used in our numerical simulations, implies a 50-fold reduction in the propagation speed of due to a deformation of the dispersion relation towards an SIP (see inset in \cref{Fig:PvsX-MeanVSt}). 
 
%%%%%%%%%%%%%%%%%%%%%%%%%%%%%%%%%%%%%%%%%%%%%%%%%%%
% Figure environment removed
%%%%%%%%%%%%%%%%%%%%%%%%%%%%%%%%%%%%%%%%%%%%%%%%%
%=========================================================

%=========================================================
%Onsite nonlinearity 
\textit{Nonlinear dynamics and ballistic crossover}. As we have established a theoretical framework for slow wave dynamics in linear systems that exhibit an SIP, we may explore how this framework is influenced by the presence of weak nonlinearity. By using the terminology ``weak,'' we imply that the nonlinearity could be treated perturbatively, i.~e. the linear eigenmode representation still provides a valid basis. This can be achieved by ensuring the nonlinear energy contribution is small as compared to the linear energy, measured from the ground state.

In general,  nonlinear effects that  impact  the dynamics in  periodic systems are expected to emerge. As nonlinearity provides a mechanism of wave mixing, the initial wavepacket in $q$-space does not remain constant. If more than one energy band exists, they may exchange energy. Here we investigate some of the possible nonlinear effects by introducing modifications to the model in \cref{Eq:DynamicsX-spaceLinear}.    

Probably, the most common type of nonlinearity is a uniform Kerr-type contribution to the onsite optical potential, which we introduce in the model  by adding the term $-\chi|\psi_{n}|^{2}\psi_{n}$ into rhs \cref{Eq:DynamicsX-spaceLinear}, where the nonlinear coefficient $\chi$ could be either positive (focusing Kerr) or negative (defocusing Kerr). The nonlinear effect of four-wave mixing causes a smearing and splitting of the wavepacket in $q$-space, introducing additional Bloch states to the wavepacket propagation which change the flow. 

In general, the nonlinear coefficient, $\chi$, by itself provides no information about nonlinear contributions. Technically, the dynamical equations could be rescaled to fix $\chi\equiv 1$, as only $(\chi/2)\sum_{n}|\psi_{n}|^{4}$ contributes to the total internal energy. In physical photonic networks the nonlinear contribution is governed by the incoming optical power $\mathcal{P} = \sum_n P_n(t = 0)$ rather than changes in the material properties. However,  for theoretical analysis it is convenient to  vary $\chi$ as the relevant parameter in different simulations, while keeping incident power constant.

%%%%%%%%%%%%%%%%%%%%%%%%%%%%%%%%%%%%%%%%%%%%%%%%%%%
% Figure environment removed
%%%%%%%%%%%%%%%%%%%%%%%%%%%%%%%%%%%%%%%%%%%%%%%%%%%%%
In \cref{Fig:Nonlinearity1} we see how the presence of nonlinearity affects the flow. The distinct crossover towards ballistic propagation between $\chi = 0.1$ and $\chi = 0.5$ is caused by spreading of the wavepacket in $q$-space. Indeed, while at $\chi = 0.1$ the wavepacket remains confined in vicinity of the SIP at $\bar{q} = -\pi/2$,  \cref{Fig:Nonlinearity1}~(a), at higher values of nonlinearity the states corresponding to sufficiently nonzero group velocities become populated (for example, see panel (b) for $\chi = 10$). This explanation is in agreement with \cref{Eq:FlowGroupVel}  and \cref{Fig:Nonlinearity1}~(c). 

In the linear system, the propagation speed in presence of an SIP depends on the wavepacket width: in the hypothetical case of a $q$-space delta-function initial condition, $\sigma\to 0$, the signal won't propagate  as the group velocity is identically zero, however,  broadening the wavepacket introduces proximal states whose group velocities are small but not entirely vanishing. The nonlinear effect causes a crossover to the ballistic transport regime as the Bloch-mode population becomes independent of the initial preparation.
%==============================================================================

%==============================================================================
%Signal deflection in roton dynamics
\textit{Control of propagation direction via roton dispersion induced by SIP management.} The dependence of the flow on the population numbers in $q$-space gives an idea how to control not only the signal speed, but also the direction via the manipulation of its incident power. Consider again the dispersion relation \cref{Eq:SIPspectrum}. When $J_{3} > J/3$, the inflection point $q = -\pi/2$ has a negative slope, as it is positioned between a local maximum (to its left) and a local minimum (to its right), creating the so-called roton dispersion relation \cite{Landau1941, Chen2021}.  Hence, in the linear system the initial preparation of a Gaussian packet centered at $q = -\pi/2$ will be followed by the energy propagating in the negative direction. However, as the nonlinearity exceeds some threshold value, the initial Gaussian in $q$-space splits and spreads, exciting states with predominantly  positive group velocity, so that $\langle v_{g}\rangle > 0$, turning the energy flow to the opposite direction. As one can see in \cref{Fig:Nonlinearity2} this effect takes place in a stationary regime, after the time required for the wavepacket to spread in $q$-space.
%%%%%%%%%%%%%%%%%%%%%%%%%%%%%%%%%%%%%%%%%%%%%%%%%%%
% Figure environment removed
%%%%%%%%%%%%%%%%%%%%%%%%%%%%%%%%%%%%%%%%%%%%%%%%%%%%%
%=========================================================

%=========================================================
%Effective dispersion relation
\textit{Nonlinear dispersion effect.} Another possible method of signal deflection is based on a modification of the dispersion relation by nonlinearity. In the one-channel model,  the onsite nonlinearity may cause a vertical shift of the dispersion relation but not deformation of the band. In systems with more complex unit cell structure, uniform onsite nonlinearities can alter relative onsite optical potentials between propagation channels, which deforms the effective dispersion relation. One can still demonstrate this phenomenon in the framework of a one-channel model by introducing nearest-neighbor nonlinear coupling, 
$J \to J(1 + \mu |\psi_{n}|^{2})$,
into \cref{Eq:DynamicsX-spaceLinear}. The Ablowitz-Ladik model modified by the presence of the third order interaction lacks an integral of motion associated with norm. This  implies some restrictions for physical applicability, though it still may be useful for its mathematical simplicity. 

In \cref{Fig:NonlinearCoupling}~(a) one can see that the flow (black dashed line) is positive from the beginning of the dynamical evolution, while the average group velocity (blue solid) is negative as one expects for the underlying roton linear system. Strictly speaking, \cref{Eq:FlowGroupVel} is not applicable as the nonlinearity cannot be treated perturbatively and the dispersion relation is not well-defined. However, \cref{Eq:SIPspectrum} may be conditionally restored for any time step if the coupling parameter $J$ is replaced by $J_{\rm eff}(t) = J[1 + \mu \delta(t)]$, where $\delta(t) \sim \langle|\psi_{n}(t)|^{2}\rangle$. Then, even though $J/(3J_{3}) < 1$ (a condition for a roton dispersion relation), the effective group velocity, $v_{g}^{{\rm eff}}(t) = \partial\omega^{\rm eff}(q, t)/\partial q$, in vicinity of the inflection point at $q = -\pi/2$ will remain positive whenever $\delta(t) > 1 - J/(3J_{3})$. One can see in \cref{Fig:NonlinearCoupling}~(a)  average values of the effective group velocity (red solid), $\langle v_{g}^{{\rm eff}}(t)\rangle$, which is in agreement with the flow.  

Apparently, this effect is only observable  in the short time range, as for $t\gg 0$ the value of $\langle|\psi_{n}|^{2}\rangle \propto 1/N$. Thus $\delta(t)$ becomes negligible and the dispersion relation converges to the roton profile. On the large time scale the direction of signal propagation is governed by the wavepacket distribution in $q$-space. The competition between these two processes may be clarified by \cref{Fig:NonlinearCoupling}~(b): at $t\sim 0$, the propagation is governed by the narrow Gaussian peak (solid blue line) probing the effective $\omega^{{\rm eff}}(q)$ (purple dash-dotted line), at $t\gg 0$ the band is restored toward the roton dispersion relation (black solid curve) while the wavepacket (red dots) probes the states outside the negative group velocity region. 

%%%%%%%%%%%%%%%%%%%%%%%%%%%%
% Figure environment removed
%%%%%%%%%%%%%%%%%%%%%%%%%%%%%
%=========================================================

%=========================================================
%=========================================================

%=========================================================
\textit{Conclusions.} In this study, we have developed a theoretical framework for linear and nonlinear dynamics of wavepackets centered at an SIP. In the linear regime, such pulses do not propagate ballistically, due to the zero group velocity at the SIP frequency. We have demonstrated that nonlinearity can result in ballistic propagation of SIP-centered pulses, with the speed and even direction of propagation essentially dependent on the pulse amplitude. This unique feature of SIP-supporting waveguides provides exciting opportunities for control and manipulation of electromagnetic and acoustic pulses.
%=========================================================
%=========================================================

\textit{Acknowledgments.} We acknowledge partial support from DEC (TK, SL),  NSF-EFMA 1641109 (WT),
Simons Foundation MPS-733698 (AK) and AFOSR LRIR 21RYCOR019 (IV).
%=========================================================

\bibliography{SlowLightBibliography}



%========================================================
%\newpage
\appendix
\onecolumngrid 
%=======================================================
%========================================================
\section{Model Setup\label{Sec:ModelSetupAppendix}}
A minimal model which may support a stationary inflection point (SIP) can be provided by the classical Hamiltonian
\begin{equation}\label{Eq:Hamiltonian1DAppendix}
H = \sum\limits_{n}\left[\varepsilon|\psi^{}_{n}|^{2} + J\left(\psi^{}_{n}\psi_{n+1}^{\ast} + c. c.\right) + J_{3} \left(\psi_{n}^{}\psi_{n+3}^{\ast}+ c. c.\right)\right],
\end{equation}
where $\psi^{}_{n}$, $i\psi^{\ast}_{n}$ are canonically conjugate dynamical variables. Assuming periodic boundary conditions, $\psi_{n+N} = \psi_{n}$,  one can rewrite the Hamiltonian as 
\begin{equation}\label{Eq:Hamiltonian1Dk-spaceAppendix}
H = \sum_{k}\omega(q_{k})|\phi_{k}|^{2}.
\end{equation}
The Bloch modes, $\phi^{}_{k}$ (and their canonically conjugate $i\phi^{\ast}_{k}$), are defined by the Fourier transform
\begin{equation}\label{Eq:FTAppendix}
\psi_{n} = \frac{1}{\sqrt{N}}\sum\limits_{k=-N/2}^{N/2-1}e^{-iq_{k}n}\phi_{k}, \, \phi_{k} = \frac{1}{\sqrt{N}}\sum\limits_{n}e^{iq_{k}n}\psi_{n},
\end{equation}
where $q_{k} = 2\pi k/N$, with a spectrum 
\begin{equation}\label{Eq:SIPspectrumAppendix}
\omega(q_{k}) = \varepsilon + 2J\cos q_{k} + 2J_{3}\cos3q_{k}.
\end{equation} 
One can see that for $J_{3}  = J/3$ the dispersion relation exhibits SIPs at $\bar{q} = \pm\pi/2$, as $\omega^{\prime}(\pm\pi/2) = \omega^{\prime\prime}(\pm\pi/2) = 0$, while $\omega^{(3)}(\pm\pi/2) = \mp 8J$. 

The direct space propagation of any wavepacket is governed by dynamical equations $\dot{\psi}_{n} = -\partial H/\partial(i\psi^{\ast}_{n})$, 
\begin{equation}\label{Eq:DynamicsX-spaceLinearAppendix}
i\frac{d\psi_{n}}{dt}  = - J\left(\psi_{n+1} + \psi_{n-1}\right) - J_{3}\left(\psi_{n+3} + \psi_{n-3}\right),
\end{equation} 
while the dynamics in reciprocal $q$-space are governed by the uncoupled equations
\begin{equation}\label{Eq:DynamicsK-spaceLinearAppendix}
i\frac{d\phi_{k}}{dt} = -\omega(q_{k})\phi_{k}.
\end{equation}

The dynamical equations should be supplied with an initial condition; in the present study we always assume preparation of a Gaussian packet in $q$-space, 
\begin{equation}\label{Eq:InitialCondK-space}
\phi_{k}(0) = \bar{\phi}_{k} = \left(\frac{4\pi}{N^{2}\sigma^{2}}\right)^{1/4}\exp\left\{-\frac{(q_{k} - \bar{q})^{2}}{2\sigma^{2}}\right\}.
\end{equation}
The packet is well confined to the first Brillouin zone, $\sigma\ll 2\pi$, and $\bar{q}$ is the reciprocal lattice vector associated with one of the SIPs. Such initial condition implies a preparation of a Gaussian packet of width $\sigma^{-1}$ in direct space. Assuming the initial wavepacket is centered at $n = 0$,  the time-dependence of the amplitude on the $n$-th site is given by
\begin{equation}\label{Eq:psinLinearAppendix}
\psi_{n}(t) = \frac{1}{\sqrt{N}}\sum\limits_{k}e^{iq_{k}n}\phi_{k}(t)\approx\frac{\sqrt{N}}{2\pi}\int\limits_{-\pi}^{+\pi}dq e^{iqn}e^{-i\omega(q)t}\bar{\phi}(q),
\end{equation} 
where we have exploited the solution $\phi_{k}(t) = \exp\left\{i\omega(q_{k}) t\right\}\bar{\phi}_{k}$ of \cref{Eq:DynamicsK-spaceLinearAppendix} and the condition $N\gg 1$ for continuous limit. 
%========================================================

%========================================================
\section{Integral Evaluation\label{Sec:IntegralAppendix}}
In this section we provide a detailed, though not rigorous, evaluation of integral which appears in \cref{Eq:psinstep2}, i.~e.
\begin{equation}\label{Eq:Izeps}
\mathcal{I}(z, \epsilon) = \frac{1}{\pi}\int\limits_{0}^{\infty}dx e^{-\epsilon x^{2}}\cos\left(\frac{x^{3}}{3} - zx\right).
\end{equation}
First, we notice that $\mathcal{I}(z, 0) = \mathrm{Ai}(-z)$, the Airy function, which is a solution of the differential equation \cite{AbramowitzStegun}
\begin{equation}\label{Eq:AiryEquation}
y^{\prime\prime} + zy = 0.
\end{equation} 
Noticing that 
\[
\frac{\partial^{2k}}{\partial z^{2k}}\cos\left(\frac{x^{3}}{3} - zx\right) = (-1)^{k}x^{2k}\cos\left(\frac{x^{3}}{3} - zx\right),
\]
and expanding $e^{-\epsilon x^{2}}$ into the Taylor series for any $\epsilon>0$, we get
\begin{multline}\label{Eq:IntegralSteps}
\mathcal{I}(z, \epsilon) = \frac{1}{\pi}\int\limits_{0}^{\infty}dx e^{-\epsilon x^{2}}\cos\left(\frac{x^{3}}{3} - zx\right) = \frac{1}{\pi}\int\limits_{0}^{+\infty}dx \sum\limits_{k = 0}^{\infty}\frac{(-1)^{k}\eps^{k}}{k!}x^{2k}\cos\left(\frac{x^{3}}{3} - zx\right) =\\
 \frac{1}{\pi}\int\limits_{0}^{+\infty}dx \sum\limits_{k = 0}^{\infty}\frac{\eps^{k}}{k!}\frac{\partial^{2k}}{\partial z^{2k}}\cos\left(\frac{x^{3}}{3} - zx\right) = \sum\limits_{k = 0}^{\infty}\frac{\eps^{k}}{k!}\frac{\partial^{2k}}{\partial z^{2k}}\mathcal{I}(z, 0) = \sum\limits_{k = 0}^{\infty}\frac{\eps^{k}}{k!}\frac{\partial^{2k}}{\partial z^{2k}}\mathrm{Ai}(-z).
\end{multline}
Introduce notations
\[
F(z) = \mathrm{Ai}(-z), \quad G(z) = \frac{d}{dz}\mathrm{Ai}(-z),
\]
then
\begin{equation}\label{Eq:AiryDer}
\begin{split}
&F^{(2)}(z) = -zF(z)\\
&F^{(4)}(z) = -2G(z) + z^{2}F(z) = z^{2}\left[-2z^{-2}G(z) + F(z)\right]\\
&F^{(6)}(z) = 4F + 6zG(z) - z^{3}F(z) = z^{3}\left[4z^{-3} + 6z^{-2}G(z) - F(z)\right]\\
&F^{(2k)}(z) = z^{k}\left[\hdots + (-1)^{k}F(z)\right].
\end{split}
\end{equation} 
 First, consider positive values of $z$. For $z > 1$ the strongest order of $z$ in the asymptotic approximation of the Airy function, $\mathrm{Ai}(-z)$,  is  
 \begin{equation}\label{Eq:AiryApprox1}
 \mathrm{Ai}(-z) \propto z^{-1/4}\sin\left(\frac{2}{3}z^{3/2} + \frac{\pi}{4}\right), \quad  \mathrm{Ai}^{\prime}(-z) \propto z^{1/4}\cos\left(\frac{2}{3}z^{3/2} + \frac{\pi}{4}\right).
 \end{equation}
Therefore, the $k$-th expression of \cref{Eq:AiryDer} is actually
\[
F^{(2k)}(z) = z^{k-1/4}\left[\mathcal{O}(z^{-3/2}) + (-1)^{k}z^{1/4}F(z)\right], \quad k \geqslant 2, \quad z^{1/4}F(z)\sim 1.
\]
For $z < 0$, Airy function is not periodic, but quickly decaying:
 \begin{equation}\label{Eq:AiryApprox2}
\mathrm{Ai}(-z) \propto |z|^{-1/4}\exp\left\{-\frac{2}{3}|z|^{3/2}\right\}, \quad  \mathrm{Ai}^{\prime}(-z) \propto |z|^{1/4}\exp\left\{-\frac{2}{3}|z|^{3/2}\right\},
 \end{equation}
 and 
 \[
F^{(2k)}(z) = |z|^{k-1/4}e^{-\frac{2}{3}|z|^{3/2}}\left[\mathcal{O}(|z|^{-3/2}) + |z|^{1/4}e^{+\frac{2}{3}|z|^{3/2}}F(z)\right], \quad |z|^{1/4}e^{+\frac{2}{3}|z|^{3/2}}F(z)\sim 1.
\]
Therefore we may approximate the derivatives as:
\[
\frac{\partial^{2k}}{\partial z^{2k}}\mathrm{Ai}(-z) \approx (-1)^{k}z^{k}\mathrm{Ai}(-z),
\]
and plugging it into \cref{Eq:IntegralSteps} we obtain
\begin{equation}\label{Eq:Solution}
\mathcal{I}(z, \epsilon)\approx \sum\limits_{k = 0}^{\infty}\frac{\eps^{k}}{k!}(-1)^{k}z^{k}\mathrm{Ai}(-z) = e^{-\epsilon z}\mathrm{Ai}(-z).
\end{equation}

We have to make one remark about this derivation: though integral \cref{Eq:Izeps} converges for any $\epsilon\geqslant 0$, the last step in \cref{Eq:IntegralSteps}, a change of integration and summation in their order, is not rigorous justified, as convergence is not guaranteed for any value of $\epsilon$. Actually, while the integral \cref{Eq:Izeps} converges  quicker for larger $\epsilon$, the sum converges better for $\epsilon < 1$. There is no contradiction here, it is a choice of the approximation domain. 
The parameters $\epsilon$, $z$ are not independent as they are introduced via physical variables
\[
\epsilon = (1/2)(\alpha t)^{-2/3}\sigma^{-2}, \quad z = n(\alpha t)^{-1/3},
\]
 in \cref{Eq:psinstep2} of the main text. To satisfy the initial condition, the integral {\cref{Eq:Izeps} should behave as $\mathcal{O}\left[t^{1/3}\right]$ for $t\to 0$. It is apparently not the case for \cref{Eq:Solution}. It only means that this approximation is not valid for $t\to 0$. Technically speaking, the time domain of guarantied applicability is
\[
\sigma^{-3}\ll \alpha t \ll n^{3},
\] 
quite a realistic range. Practically, one can see in comparison with the numerical simulations that the approximation qualitatively captures all the phenomena associated with SIP dynamics in almost the entire time domain.
%========================================================

%========================================================
\section{Flow in presence of SIP\label{Sec:FlowAppendix}}
In absence of losses we define flow as 
\begin{equation}
\mathcal{F}(t) \overset{\rm def}{=} \sum_{n}n\frac{d}{dt}\left|\psi(t, n)\right|^{2} = \frac{d}{dt}\sum_{n}n|\psi(t, n)|^{2}.
\end{equation}
Using the explicit expression for signal propagation in presence of the SIP at $q = -\pi/2$ (\cref{Eq:SIPsolution} of the main text) one can write 
\begin{multline}
\sum_{n}n|\psi(t, n)|^{2} 
=
  2\left(\frac{\pi}{\sigma^{2}}\right)^{1/2}\left(\alpha t\right)^{-2/3}\int\limits_{-\infty}^{+\infty} dx x e^{-\frac{x}{\sigma^{2}\alpha t}}\mathrm{Ai}^{2}\left[- x\left(\alpha t\right)^{-1/3}\right] 
  =\\
2\left(\frac{\pi}{\sigma^{2}}\right)^{1/2}\int\limits_{-\infty}^{+\infty} dy y e^{-\frac{y}{\sigma^{2}(\alpha t)^{2/3}}}\mathrm{Ai}^{2}\left(- y\right) 
= 
-2\left(\frac{\pi}{\sigma^{2}}\right)^{1/2}\frac{\partial}{\partial a}\left[\int\limits_{-\infty}^{+\infty} dy e^{-a y}\mathrm{Ai}^{2}\left(- y\right)\right]\Bigg|_{a = \frac{1}{\sigma^{2}(\alpha t)^{2/3}}} 
=\\
-2\left(\frac{\pi}{\sigma^{2}}\right)^{1/2}\frac{\partial}{\partial a}\left[\frac{e^{a^{3}/12}}{2\sqrt{\pi a}}\right]\Bigg|_{a = \frac{1}{\sigma^{2}(\alpha t)^{2/3}}} 
= 
\frac{\sigma^{2}\alpha t}{2}\exp\left\{\frac{1}{12\sigma^{6}(\alpha t)^{2}}\right\}\left[1 - \frac{1}{2\sigma^{6}(\alpha t)^{2}}\right] 
=\\
\frac{\sigma^{2}\alpha t}{2} + \mathcal{O}\left[\frac{1}{\sigma^{6}(\alpha t)^{2}}\right] \xrightarrow{t\gg \alpha(2\sigma)^{-3/2}} \frac{\sigma^{2}\alpha t}{2}.
\end{multline}
Therefore
\begin{equation}
\mathcal{F}(t) \xrightarrow{t\gg \alpha(2\sigma)^{-3/2}} \frac{\sigma^{2}\alpha}{2}.
\end{equation}

This result can also be obtained using equation $\mathcal{F}(t) = \langle v_{g}\rangle$:
\begin{multline}
\langle v_{g}\rangle 
= 
\int\limits_{-\pi}^{+\pi} dq \left(\frac{\partial\omega}{\partial q}\right)|\phi(q)|^{2}
\approx
  -\frac{2J}{\sqrt{\pi\sigma^{2}}}\int\limits_{-\infty}^{+\infty}dq \left[\sin q + \sin 3q\right]\exp\left\{-\frac{(q-\bar{q})^{2}}{\sigma^{2}}\right\}\\
 \overset{\bar{q} = -\pi/2}{\approx} 
 \frac{2J}{\sqrt{\pi\sigma^{2}}}\int\limits_{-\infty}^{+\infty}dp \left[-\frac{p^{2}}{2} + \frac{9p^{2}}{2}\right]\exp\left\{-\frac{p^{2}}{\sigma^{2}}\right\} 
 = 
\frac{\alpha}{\sqrt{2\pi(\sigma/\sqrt{2})^{2}}}\int\limits_{-\infty}^{+\infty}dp p^{2}\exp\left\{-\frac{p^{2}}{2(\sigma/\sqrt{2})^{2}}\right\} 
 =
 \frac{\alpha\sigma^{2}}{2}.
\end{multline}
%========================================================
\section{Flow and average velocity equality\label{Sec:FlowVgProofAppendix}}
The proof of \cref{Eq:FlowGroupVel} of the main text, $\mathcal{F}(t) = \langle v_{g}(q, t)\rangle$, is straightforward:
\begin{multline}
\mathcal{F}(t) 
=
 \frac{d}{dt}\int dx x|\psi(x, t)|^{2} 
 =
  \frac{d}{dt} \iint dq dp \int dx x \phi^{\ast}(q, t)\phi(p, t)e^{-i(q - p)x} 
=\\
 \frac{d}{dt} \iint dq dp \int dx x e^{i(p - q)x}e^{-i[\omega(p) - \omega(q)]t}C^{\ast}(q, t)C(p, t),
\end{multline}
where $C(q, t)$ is a slow function of time, in the linear system a constant. One can proceed further as
\begin{multline}
\mathcal{F}(t) = \frac{d}{dt} \iint dq dp e^{-i[\omega(p) - \omega(q)]t}C^{\ast}(q, t)C(p, t)(-i)\frac{\partial }{\partial p}\int dx e^{i(p - q)x} 
=\\
 \frac{d}{dt} \iint dq dp e^{-i[\omega(p) - \omega(q)]t}C^{\ast}(q, t)C(p, t)(-i)\frac{\partial }{\partial p}\delta(p - q) 
 =
 i\frac{d}{dt} \int dq e^{i\omega(q)t}C^{\ast}(q, t)\frac{\partial }{\partial q}\left[e^{-i\omega(q)t}C(q, t)\right], 
\end{multline}
where we use the equality 
\[
\int dx f(x)\frac{\partial }{\partial x}\delta(x - x_{0}) = -f^{\prime}(x_{0}). 
\]
Hence,
\begin{multline}
\mathcal{F}(t) = \frac{d}{dt} \int dq\left[\left(\frac{\partial\omega}{\partial q}\right)|C(q, t)|^{2} + iC^{\ast}(q, t)\frac{\partial C}{\partial q}\right] 
= \\
\int dq\left(\frac{\partial\omega}{\partial q}\right)|\phi(q, t)|^{2} + \int dq\left(\frac{\partial\omega}{\partial q}\right)t\frac{d}{dt}|\phi(q, t)|^{2} + \frac{i}{2}\frac{d}{dt}\int dq |\phi(q, t)|^{2}. 
\end{multline}
The last term is always equal to zero due to norm conservation,
\[
\frac{d}{dt}\int dq |\phi(q, t)|^{2} = 0.
\]
The second term is exactly equal  to zero in the linear system as $\dot{C_{0}}\equiv 0$. At $\chi\neq 0$ the second term  is still negligible. Indeed, the norm exchange between the Bloch modes slows down by approaching stationary regime, so  $t\frac{d}{dt}|\phi(q, t\to \infty )|^{2}\to 0$. The norm exchange rate, $\frac{d}{dt}|\phi(q)|^{2}$, is nonzero at $t\to 0$ only, which makes $t\frac{d}{dt}|\phi(q, t)|^{2}\Big|_{t\to 0}\to 0$ as well. Finally,
\[
\mathcal{F}(t) = \int dq\left(\frac{\partial\omega}{\partial q}\right)|\phi(q, t)|^{2} = \langle v_{g}\rangle
\]   
%========================================================
%\section{Unit cells with internal structure\label{Sec:UnitCellsAppendix}}
%In \cref{Sec:LinearModel} we have introduced \cref{Eq:Hamiltonian1D}, a minimal model which describes a system with SIP. However, implementing such systems in practice may require the use of periodic metamaterials composed of unit cells that possess internal structures.  
%========================================================
\end{document}