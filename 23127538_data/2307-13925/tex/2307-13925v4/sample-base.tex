 
\documentclass[sigconf]{acmart}

\AtBeginDocument{%
  \providecommand\BibTeX{{%
    \normalfont B\kern-0.5em{\scshape i\kern-0.25em b}\kern-0.8em\TeX}}}

\usepackage{algorithm}
\usepackage{algorithmic}

\usepackage{subcaption}
\usepackage{multirow}
\usepackage{indentfirst}

\usepackage{makecell}
\usepackage{bbding}
\usepackage{diagbox}
\usepackage{color}

\usepackage{balance}

\usepackage{wrapfig}
\newcommand{\red}[1]{\textcolor{red}{#1}}
\newcommand{\blue}[1]{\textcolor{blue}{#1}}
\newcommand{\revised}[1]{{\color{black} #1}}
\newcommand{\wjb}[1]{{\color{green} #1}}

\copyrightyear{2023}
\acmYear{2023}
\setcopyright{acmlicensed}\acmConference[MM '23]{Proceedings of the 31st ACM International Conference on Multimedia}{October 29-November 3, 2023}{Ottawa, ON, Canada}
\acmBooktitle{Proceedings of the 31st ACM International Conference on Multimedia (MM '23), October 29-November 3, 2023, Ottawa, ON, Canada}
\acmPrice{15.00}
\acmDOI{10.1145/3581783.3611876}
\acmISBN{979-8-4007-0108-5/23/10}




\begin{document}

\title{EasyNet: An Easy Network for 3D Industrial Anomaly Detection}
\author{Ruitao Chen}
\authornote{Equally contribute to this work}
 \affiliation{%
  \institution{Southern University of Science and Technology}
  \country{Shenzhen, China}}
 \email{chenrt2022@mail.sustech.edu.cn}
 
 \author{Guoyang Xie}
 \authornotemark[1]
 \affiliation{%
  \institution{Southern University of Science and Technology}
  \country{Shenzhen, China}}
      \affiliation{%
  \institution{University of Surrey}
  \country{Guildford GU2 7XH, UK}
  }
 \email{guoyang.xie@surrey.ac.uk}

 \author{Jiaqi Liu}
\authornotemark[1]
 \affiliation{%
  \institution{Southern University of Science and Technology}
  \country{Shenzhen, China}}
 \email{liujq32021@mail.sustech.edu.cn}
 
 
 \author{Jinbao Wang}
\authornote{Corresponding author}
 \affiliation{%
  \institution{Southern University of Science and Technology}
  \country{Shenzhen, China}}
 \email{linkingring@163.com}
 
 
  \author{Ziqi Luo}
 \affiliation{%
  \institution{Southern University of Science and Technology}
  \country{Shenzhen, China}}
 \email{luozq2022@mail.sustech.edu.cn}

 \author{Jinfan Wang}
 \affiliation{%
  \institution{Southern University of Science and Technology}
  \country{Shenzhen, China}}
  \affiliation{%
  \institution{Linkinsense}
  \country{Shenzhen, China}
  }
 \email{wangjf@sustech.edu.cn}
 
 \author{Feng Zheng}
 %\authornote{Corresponding author}
 \authornotemark[2]
 \affiliation{%
  \institution{CSE and RITAS, Southern University of Science and Technology}
  \country{Shenzhen, China}}
 \email{f.zheng@ieee.org}
 

\begin{CCSXML}
<ccs2012>
  <concept>
      <concept_id>10010147.10010178.10010224.10010225</concept_id>
      <concept_desc>Computing methodologies~Computer vision tasks</concept_desc>
      <concept_significance>500</concept_significance>
      </concept>
 </ccs2012>
\end{CCSXML}

\ccsdesc[500]{Computing methodologies~Computer vision tasks}

\renewcommand{\shortauthors}{Ruitao Chen et al.}

\begin{abstract}

3D anomaly detection is an emerging and vital computer vision task in industrial manufacturing (IM). Recently many advanced algorithms have been published, but most of them cannot meet the needs of IM. There are several disadvantages: i) difficult to deploy on production lines since their algorithms heavily rely on large pretrained models; ii) hugely increase storage overhead due to overuse of memory banks; iii) the inference speed cannot be achieved in real-time. To overcome these issues, we propose an easy and deployment-friendly network (called EasyNet) without using pretrained models and memory banks: firstly, we design a multi-scale multi-modality feature encoder-decoder to accurately reconstruct the segmentation maps of anomalous regions and encourage the interaction between RGB images and depth images; secondly, we adopt a multi-modality anomaly segmentation network to achieve a precise anomaly map; thirdly, we propose an attention-based information entropy fusion module for feature fusion during inference, making it suitable for real-time deployment. Extensive experiments show that EasyNet achieves an anomaly detection AUROC of 92.6\% without using pretrained models and memory banks. In addition, EasyNet is faster than existing methods, with a high frame rate of 94.55 FPS on a Tesla V100 GPU.
\end{abstract}


\keywords{3D anomaly detection, multi-modality fusion, unsupervised learning, industrial manufacturing}

\maketitle

\section{Introduction}\label{sec:introduction}

There is a strong need to propose a deployment-friendly 3D unsupervised anomaly detection (3D-AD) model to tap the gap, which brings 3D-AD's capabilities into the factory floor. Currently, most of anomaly detection methods~\cite{li2022towards, xie2023pushing, Xie2023IMIADII, liu2023deep} are based on 2D images. But in the quality inspection of industrial products, human inspectors utilize both color (RGB) characteristics and depth information to determine whether it is a defective product, where depth information is essential for anomaly detection. As shown in Figure~\ref{fig:3DAD-motivation}, for foam and peach, it is difficult to identify the anomalies from the RGB image alone. Though 3D-AD algorithms~\cite{Wang2023MultimodalIA, Rudolph2022AsymmetricSN, Bergmann2022AnomalyDI} are attracting interest from the academy, most of them are far from satisfactory for industrial manufacturing (IM). According to Figure~\ref{fig:easynet-motivation}, there are several issues: i) The cutting-edge 3D-AD methods steadily rely on the representational abilities of large pretrained models, leading to slow inference speed and huge storage overhead. ii) Feature embedding-based 3D-AD methods excessively use memory banks, leading to huge memory bank costs in real-world applications. Because of this, it is important and urgent to build an application-oriented 3D-AD model to meet the demands of IM. 
%3D anomaly detection, i.e., detecting abnormal points that deviate from normal pattern for 3D structures, is attracting interest from the academy.



% Figure environment removed

To avoid using large pretrained models and memory banks, we propose an easy but effective multi-modality anomaly detection and localization network, called \textbf{EasyNet}. Specifically, EasyNet consists of two parts, the Multi-modality Reconstruction Network (MRN) and the Multi-modality Segmentation Network (MSN). First, instead of using pretrained features directly, we generate synthesized anomalies on RGB images and depth images, reconstruct the original images with semantically reasonable free content, and obtain multi-scale features. At the same time, in order to simplify the anomaly detection process based on memory bank, we input the abnormal and reconstructed multi-scale features into a simple MSN to obtain an anomaly map. As shown in Figure \ref{fig:easynet-total-arch}, the entire architecture, including MRN and MSN, significantly encourages interaction between RGB and depth features.


To reduce the disturbance between RGB and depth images, we propose an attention-based information entropy fusion module. We find that some 3D-AD methods, like AST~\cite{rudolph2022asymmetric} and BTF~\cite{horwitz2022back} cannot fully utilize the advantage of multi-modality fusion, i.e., RGB-D performance is not competitive than RGB performance. The main reason is that there are no uniform abnormal patterns in RGB or depth images. For example, some anomalies can be detected by pure RGB images and depth information works as the noise and may degrade the overall anomaly detection performance. Hence, we propose a dynamic multi-modality fusion scheme to make use of RGB and depth features. The architecture of the fusion scheme is shown in Figure~\ref{fig:information_entropy_scheme}. Moreover, as shown in Table~\ref{tab:inference-accuracy}, our proposed fusion scheme is simple and much more computationally efficient than the aforementioned 3D-AD models~\cite{horwitz2022back, rudolph2022asymmetric, Wang2023MultimodalIA}. Our proposed attention-based information entropy fusion module is easy to train and apply, with outstanding performance and inference speed. As a result, EasyNet can achieve 92.6\% on MVTec 3D-AD and 86.9\% on Eyescandies in I-AUROC while running at 94.55 FPS, surpassing the previous best-published 3D-AD methods on accuracy and efficiency. 



Our contributions can be summarized as follows:
\begin{itemize}
    \item EasyNet is easy to implement and deploy for 3D unsupervised anomaly detection, i.e., eliminating the usage of pretrained models and memory banks, and achieves the fastest inference speed than the existing methods, with a high frame rate of 94.55 FPS on a Tesla V100 GPU.
    \item We propose an Attention-based Information Entropy Fusion Module to integrate the image features of the multi-modal characteristics well.
    \item We propose a Multi-modality Reconstruction Network(MRN) to accurately reconstruct the anomalous region and encourage the interaction of RGB and depth.
    \item We propose a Multi-modality Segmentation Network(MSN) to output the anomaly map precisely.
    \item EasyNet obtains the state-of-the-art result in Pure RGB. Note that EasyNet obtains the best anomaly detection I-AUROC of 92.6\% in RGBD.
\end{itemize} 


\section{Related Work}
Anomaly detection (AD) is a classical topic, which aims to distinguish normal samples and abnormal samples. Existing experimental settings usually only take normal samples as the training set, and evaluate the ability of the model to distinguish abnormal samples in the test set. The current unsupervised AD can be mainly divided into feature extraction-based methods and image reconstruction-based methods. The former is restricted by the pretrained model, while the latter is free from this limitation. Based on this idea, we design a reconstructive AD algorithm for RGB-D data, removing the restrictions on pretrained models and memory banks.

\subsection{2D Anomaly Detection}
Since the emergence of the MVTec AD dataset~\cite{bergmann2019mvtec}, research on AD in industrial 2D images has received more attention. Most existing research is based on this set for unsupervised AD tasks. 

There is more research on feature embedding-based methods than reconstruction-based methods. The most basic idea is to regard AD as a one-class classification problem and turn the AD problem into a problem of finding boundaries for classification. CutPaste~\cite{li2021cutpaste} and SimpleNet~\cite{liu2023simplenet} are representative methods. They make abnormal samples and change unsupervised AD datasets into supervised datasets. Teacher-student architecture is another useful approach. The teacher network distills knowledge to the student network by extracting features from normal samples. While the teacher network and the student network perform differently when producing abnormal samples and they detect anomalies through this characteristic~\cite{bergmann2020uninformed, Deng2022AnomalyDV}. Normalizing flow methods map samples into a Gaussian distribution, while abnormal samples deviate from this distribution~\cite{rudolph2021same, gudovskiy2022cflow}. Methods based on memory banks are simple but effective, whose ideas come from the k-nearest neighbors (KNN) algorithm. They store features of normal samples and calculate the distance between the features of test samples and the features of normal samples during testing to determine whether the samples are abnormal~\cite{defard2021padim, roth2022towards}.
As for reconstruction-based methods, most of them are similar in structure. They synthesize abnormal samples and restore abnormal samples to normal samples.  For example, DRAEM~\cite{zavrtanik2021draem} and NSA~\cite{schluter2022natural} synthesize abnormal samples in image level, while DSR~\cite{zavrtanik2022dsr} and UniAD~\cite{you2022unified} synthesize abnormal samples in feature level.

Generally, most of the 2D-AD methods use the pretrained model of natural images to extract RGB's features while they don't process depth information, so it is difficult to apply to 3D-AD directly. There is a certain gap between these two, and our method tries to get rid of this dependence so that 2D-AD can smoothly transition to 3D-AD.


\subsection{3D Anomaly Detection}
Different from 2D-AD, 3D-AD is a new research topic since the publication of MVTec 3D-AD~\cite{Bergmann2021TheM3}. As shown in Figure~\ref{fig:3DAD-motivation}, 3D-AD is a more challenging but also more promising research direction. The effective use of depth information can greatly improve detection accuracy in specific scenarios. On the other hand, how to integrate depth information and prevent it from interfering with RGB information is the current difficulty.

% Figure environment removed



Bergmann~\textit{et al.}~\cite{Bergmann2022AnomalyDI} introduce a point-cloud feature extraction network of the teacher-student model. During training, the features extracted by the student network and the teacher network are forced to be consistent. During the test, the differences between the features extracted by the teacher-student model are compared to locate anomalies. Horwitz~\textit{et al.}~\cite{horwitz2022back} combine hand-crafted 3D descriptors with the KNN framework, a classic AD approach. These two methods are efficient, but with poor performance. AST~\cite{Rudolph2022AsymmetricSN} gets a better result in MVTec 3D-AD. However, it only uses depth information to remove the background and still uses the 2D-AD method to detect anomalies and the depth information about items is ignored. Similar to BTF, but M3DM~\cite{Wang2023MultimodalIA} extracts features from point clouds and RGB images, respectively, and fuses them to make a decision, which has a better performance than treating RGB and depth as six-channel images as BTF. The visualization effect of M3DM is shown in the fourth row of Figure~\ref{fig:3DAD-motivation}. CPMF~\cite{cao2023complementary} also adopts the KNN paradigm, but the difference lies in the fact that the authors project the point cloud from different angles into 2D images and fuse the 2D image information obtained for detection.

In summary, existing 3D-AD models either suffer from poor performance or reliance on pretrained models and memory banks. In contrast, EasyNet is simple, effective, and without relying on pretrained models or memory banks. It achieves SOTA performance outperforming all previous methods without pre-training.


% Figure environment removed


\section{Approach}\label{sec:method}

\subsection{Problem Definition and Challenges}
\label{subsec:challenges}

Our 3D-AD setting is similar to M3DM~\cite{Wang2023MultimodalIA} and AST~\cite{rudolph2022asymmetric} and can be formally stated as follows. Given a set of training examples $\mathcal{T} = \left\{ t_{i}\right\}_{i=1}^{N}$, in which $\left\{ t_{1}, t_{2}, \cdots, t_{N}\right\}$ are the normal samples and each of them consists of paired images, RGB image $I_{rgb}$ and depth image $I_{depth}$. In addition, $\mathcal{T}_{n}$ belongs to a certain category, $c_{j}$, $c_{j} \in \mathcal{C}$, where $\mathcal{C}$ denotes the set of all categories. During testing, given a normal or abnormal sample from a target category $c_{j}$, the AD model should predict whether or not the test 3D object is anomalous and localize the anomaly region if the anomaly is detected.


The following are the main challenges. (1) Information on normal samples is limited, each category's training dataset only contains normal samples, i.e., no pixel-level annotations of $I_{rgb}$ and $I_{depth}$. (2) It is difficult to find an effective multi-modality fusion way for anomalies that may appear in RGB, depth, or both. Simply fusing these features may negatively impact overall AD performance. (3) Real-world applications have limited storage space, so it is impractical to build a model that uses large pretrained models and memory banks.

\subsection{EasyNet}
This section provides a complete description of EasyNet. As illustrated in Figure~\ref{fig:easynet-total-arch}, the proposed model comprises a multi-scale Multi-modality Reconstruction Network (MRN), a multi-scale Multi-modality Segmentation Network (MSN) and an attention-based information entropy fusion module, with the fusion network being exclusively applied during reasoning stages. The following sections elaborate on the design and functionality of each module.

\subsubsection{Multi-modality Reconstruction Network (MRN)}
The multi-modality reconstruction network establishes a task of image reconstruction. In this task, the network reconstructs the original image from an artificially corrupted image obtained from the simulator. The network is designed as an encoder-decoder structure to transform the local features of the input image into a mode that more closely resembles the normal sample distribution.

The framework of the simulator is depicted in Figure~\ref{fig:anomaly_generation}. We generate a foreground mask on the original depth image and apply a mask operation on the randomly generated Berlin noise figure. Our empirical evaluation reveals that only adding foreground noise exclusively assists the network in recognizing the noise on the foreground object rapidly. Then, the Berlin noise map undergoes binarization to produce positive and negative mask maps. Both random and original RGB images undergo weighting, along with the Berlin noise map and depth image. Finally, the resulting outputs include RGB and depth images with anomalies and masks.

% Figure environment removed


In the reconstruction task, we used the classic $L_{2}$ to reduce perceptual differences in RGB image reconstruction, we used the SSIM loss function in the RGB image reconstruction task. In our experiment, it is also found that spatial variation and multi-scale features have limited and even negative effects on depth images. Therefore, the final image reconstruction loss function should be:
\begin{equation}
\begin{aligned}
L_{rec}(I, I_r) & = L_{rec}^{RGB}(I, I_r) + L_{rec}^{depth}(I, I_r) \\
& = \lambda_1L_{SSIM}^{RGB}(I, I_r)+\lambda_2 l_2^{RGB}(I, I_r)+\lambda_3 l_2^{depth}(I, I_r),
\end{aligned}
\end{equation}
where $\lambda_1$, $\lambda_2$, $\lambda_3$ are loss balancing hyper-parameters, and all are set to 1 in our experimental setting.

\subsubsection{Multi-modality Segmentation Network (MSN)}
\label{subsubsec:multi_seg_net}


The MSN evaluates the normality of each time slot $(H, W)$. Similar to DRAEM~\cite{zavrtanik2021draem}, the training set samples are processed by the simulator and the discriminator performs mask identification by identifying the input of enhanced images and reconstructed images. The difference between DRAEM and EasyNet refers to the \textit{supplementary materials}. EasyNet extracts multi-layer features evaluated by discriminators through an MRN. MSN utilizes multi-layer reconstruction features and enhanced image features, which come from our assumption that some features that deviate from the normal distribution will be removed gradually with the deepening of the multi-modality reconstruction network. By comparing the difference of eigenvalues before and after removal, the locations of anomalies can be obtained.

When extracting reconstruction features and enhancing image features of multiple layers, we mainly adopt the first three layers of shallow networks of MRN and the last three layers of reconstructed features and carry out up-sampling operations to adapt for features of multiple layers. Moreover, we conduct ablation experiments. As shown in Section~\ref{subsubsec:ablation_study_number}, experiments show that when two-layer features are adopted, both the accuracy and computing cost of the network are optimized.
We use a two-layer multi-layer perceptron (MLP) to process multi-layer scale features extracted from RGB and depth images respectively. Finally, we use another two-layer MLP structure to combine the features of the two modes and perform positive and negative discriminations for each pixel in the image. As shown in Section~\ref{subsec:main_result}, the proposed straightforward strategy is successful in reaching its goal.

% Figure environment removed

\subsubsection{Attention-based Information Entropy Fusion Module}

As noted in Section~\ref{subsec:challenges}, an anomaly may occur solely in pure RGB or depth images, or both. The direct combination of both features may diminish the overall performance of AD and lead to an inverse outcome. So we generate multi-channel self-attention scores from input features in the input layer of MSN's multi-layer perception module, as shown in Figure~\ref{fig:information_entropy_scheme}, We then compare the information entropy of the channel that integrates RGB and depth features with that of the channel integrating only pure RGB features. 
We hypothesize that the greater the information entropy of the channel attention score, the richer the feature knowledge it contains. If fusion features enhance the information gain beyond RGB features, it could positively affect the performance of the results. The experimental results presented in Section~\ref{subsubsec:Information_Entropy} provide support for our theory. The mathematical representation of this process is shown in Formula~\ref{con:f_fusion}.


\begin{equation}
% \small
\footnotesize
\begin{aligned}
F_{fusion} = 
\begin{cases}
F_{RGB}+F_{depth},&f_{IE}(F_{RGB}+F_{depth})>f_{IE}(F_{RGB})+\alpha\\
F_{RGB},&f_{IE}(F_{RGB}+f_{depth}) \leq f_{IE}(F_{RGB})+\alpha
\end{cases}
\end{aligned}
\label{con:f_fusion}
\end{equation}
\revised{where $F_{fusion}$ represents the features after fusion, $F_{RGB}$ represents the features of RGB, $F_{depth}$ represents the features of depth, $f_{IE}( \cdot)$ represents the function of calculating information entropy, and $\alpha$ represents the threshold adjustment factor.

When calculating the loss between the predicted mask and the ground truth mask, we use the Focal Loss~\cite{Lin2017FocalLF} function (Formula~\ref{con:focal_loss}), which could well solve the problem of sample imbalance in the single-class classification of pixels.}

\begin{equation}
\begin{aligned}
L_{focal}(M, M_{out}) = -\alpha_t(1-p_t)^\gamma log(p_t),
\end{aligned}
\label{con:focal_loss}
\end{equation}
where $\alpha_t$ is a scaling factor related to class $t$, $\gamma$ is an adjustable parameter, $p_t$ corresponds to the predicted classification of pixel points, the abnormal category is 1, and the normal category is 0.

To sum up, EasyNet optimization objectives and tasks are reconstruction loss and classification loss. Finally, the overall loss of the network during training is as follows:
\begin{equation}
\begin{aligned}
L_{all}(I, I_r) = & L_{rec}^{RGB}(I, I_r) + L_{rec}^{depth}(I, I_r) + L_{focal}(M, M_{out})\\
= & \lambda_1L_{SSIM}^{RGB}(I, I_r)+\lambda_2 l_2^{RGB}(I, I_r)\\
 & +\lambda_3 l_2^{depth}(I, I_r)+ \lambda_4 L_{focal}(M, M_{out}),
\end{aligned}
\end{equation}
where $\lambda_1$, $\lambda_2$, $\lambda_3$, and $\lambda_4$ are loss balancing hyper-parameters. \revised{Easynet aims to meet the objectives of optimizing anomaly detection and reconstruction tasks while training, so we optimize the above objectives by assigning weights to different losses. All four $\lambda$ are set to 1 in our experimental setting.}


\subsubsection{Algorithms}
\revised{The EasyNet is implemented as Algorithm~\ref{algorithm_easynet}. when training, images $I_{rgb}$ and $I_{depth}$ are enhanced by the anomaly generator $\Phi_{ag}$ to produce augmented images $A_{rgb}$ and $A_{depth}$ respectively. The Multi-modality Reconstruction Network $\Phi_{rec}$ extracts multi-scale features ($F_{rgb}, F_{depth} $) and generate reconstructed images ($R_{rgb}, R_{depth}$) from these augmented images and origin images. The Multi-modality Segmentation Network $\Phi_{seg}$ generates an anomaly score maps $M$ and $M_{rgb}$ by fusion and pure RGB features. When inferring, the function $\Phi_{ai}$ generates corresponding self-attention information entropy scores from both RGB and RGB-D channels to combine RGB and depth features.}

\begin{algorithm}
	% \textsl{}\setstretch{1.8}
	\renewcommand{\algorithmicrequire}{\textbf{Input:}}
	\renewcommand{\algorithmicensure}{\textbf{Output:}}
	\caption{EasyNet pseudo-code}
	\label{algorithm_easynet}
	\begin{algorithmic}[1]
		\STATE \textbf{Input}: train dataloader $D_{train}$, test dataloader $D_{test}$, epochs
		\STATE \textbf{Output}: trained $\Phi_{rec}$ and $\Phi_{seg}$, $M$
        \STATE \textbf{Initialization ramdomly}:$\Phi_{rec}$ and $\Phi_{seg}$
        \STATE \textcolor{gray}{/*Training time*/}
        \FOR{$i = 0$ to epochs}
        \FOR{$I_{rgb}, I_{depth}, M_{gt} \leftarrow D_{train}$}
        % \tcc{\Phi_{ag}:Anomaly genaration}
        \STATE $A_{rgb}, A_{depth} = \Phi_{ag}(I_{rgb}, I_{depth})$
        % \Comment \Phi_{rec}:Reconstruction Network\\
        \STATE $F_{rgb}, F_{depth}, R_{rgb}, R_{depth} = \Phi_{rec}(A_{rgb}, A_{depth})$
        % \Comment{\Phi_{ai}:function of calculating channel self attention score}
        \STATE $F_{fusion} = Concat(F_{rgb}, F_{depth})$
        \STATE $M_{rgb} = \Phi_{seg}(F_{rgb})$
        \STATE $M = \Phi_{seg}(F_{fusion})$
        \STATE$L_{rgb} = \Phi_{loss}(R_{rgb}, I_{rgb}, M_{rgb}, M_{gt})$
        \STATE$L_{total} = \Phi_{loss}(R_{rgb}, R_{depth}, I_{rgb}, I_{depth}, M, M_{gt})$
        \STATE$L_{rgb}.backward, L_{total}.backward$
        \ENDFOR
        \ENDFOR
        \STATE \textcolor{gray}{/*Inference time*/}
        \FOR{$I_{rgb}, I_{depth}, M_{gt} \leftarrow D_{test}$}
        \STATE $A_{rgb}, A_{depth} = \Phi_{ag}(I_{rgb}, I_{depth})$
        \STATE $F_{rgb}, F_{depth}, R_{rgb}, R_{depth} = \Phi_{rec}(A_{rgb}, A_{depth})$
        \STATE $S_{rgb} = \Phi_{ai}(Feature_{rgb})$
        \STATE $S_{fusion} = \Phi_{ai}(F_{rgb}, F_{depth})$
        \IF{$S_{fusion}-S_{rgb} > \alpha$}
        \STATE $F_{fusion} = Concat(F_{rgb}, F_{depth})$
        \ELSE
        \STATE $F_{fusion} = F_{rgb}$
        \ENDIF
        \STATE $M = \Phi_{seg}(F_{fusion})$
        % \STATE \textbf{return} $M$
        \ENDFOR
	\end{algorithmic}  
\end{algorithm}




\begin{table*}[t]
    \centering
    \caption{I-AUROC score for anomaly detection of MVTec 3D-AD. The best is in red and the second best is in blue.}
    \vspace{-10pt}
    \resizebox{0.88\textwidth}{!}{
    \begin{tabular}{l l | c c c c c c c c c c | c | c | c }
    \toprule[0.5mm]
        ~ & \textbf{Method} & \textbf{Bagel} & \textbf{Cable Gland} & \textbf{Carrot} & \textbf{Cookie} & \textbf{Dowel} & \textbf{Foam} & \textbf{Peach} & \textbf{Potato} & \textbf{Rope} & \textbf{Tire} & \textbf{Mean} & \thead{\textbf{Memory} \\\textbf{Bank Usage}} & \thead{\textbf{pretrained}\\\textbf{Model Usage}}\\ \hline
        \multirow{11}{*}{\textbf{\thead{Pure\\Depth}}} & Depth GAN~\cite{Bergmann2021TheM3}  & 0.530  & 0.376  & 0.607  & 0.603  & 0.497  & 0.484  & 0.595  & 0.489  & 0.536  & 0.521  & 0.523 & & \\ 
        ~ & Depth AE~\cite{Bergmann2021TheM3} & 0.468  & \blue{0.731}  & 0.497  & 0.673  & 0.534  & 0.417  & 0.485  & 0.549  & 0.564  & 0.546  & 0.546  & & \\ 
        ~ & Depth VM~\cite{Bergmann2021TheM3}  & 0.510  & 0.542  & 0.469  & 0.576  & 0.609  & 0.699  & 0.450  & 0.419  & 0.668  & 0.520  & 0.546 & &  \\ 
        ~ & Voxel GAN~\cite{Bergmann2021TheM3} & 0.383  & 0.623  & 0.474  & 0.639  & 0.564  & 0.409  & 0.617  & 0.427  & 0.663  & 0.577  & 0.537 & &  \\ 
        ~ & Voxel AE~\cite{Bergmann2021TheM3}  & 0.693  & 0.425  & 0.515  & 0.790  & 0.494  & 0.558  & 0.537  & 0.484  & 0.639  & 0.583  & 0.571 & &  \\ 
        ~ & Voxel VM~\cite{Bergmann2021TheM3}  & 0.750  & \red{0.747}  & 0.613  & 0.738  & 0.823  & 0.693  & 0.679  & 0.652  & 0.609  & \blue{0.690}  & 0.699  & & \\ 
        ~ & 3D-ST~\cite{Bergmann2022AnomalyDI} & 0.862  & 0.484  & 0.832  & 0.894  & 0.848  & 0.663  & 0.763  & 0.687  & \red{0.958}  & 0.486  & 0.748 &  &\Checkmark   \\ 
        ~ & PatchCore+FPFH~\cite{Horwitz2022AnEI} & 0.825  & 0.551  & \blue{0.952}  & 0.797  & \blue{0.883}  & 0.582  & 0.758  & 0.889  & 0.929  & 0.653  & 0.782 &\Checkmark &    \\ 
        ~ & AST~\cite{rudolph2022asymmetric} & \blue{0.881}  & 0.576  & \red{0.965}  & \blue{0.957}  & 0.679  & \red{0.797}  & \red{0.990}  & \blue{0.915}  & \blue{0.956}  & 0.611  & \blue{0.833} &  &\Checkmark \\ 
        ~ & M3DM~\cite{Wang2023MultimodalIA} & \red{0.941}  & 0.651  & \red{0.965}  & \red{0.969}  & \red{0.905}  & \blue{0.760}  & \blue{0.880}  & \red{0.974}  & 0.926  & \red{0.765}  & \red{0.874} &\Checkmark &\Checkmark \\ 
        ~ & \textbf{EasyNet(ours)} & 0.735  & 0.678  & 0.747  & 0.864  & 0.719  & 0.716  & 0.713  & 0.725  & 0.885  & 0.687  & 0.747 &   &   \\ 
        \hline
        \multirow{8}{*}{\textbf{\thead{Pure\\RGB}}} & DifferNet~\cite{Rudolph2020SameSB} & 0.859  & 0.703  & 0.643  & 0.435  & 0.797  & 0.790  & 0.787  & 0.643  & 0.715  & 0.590  & 0.696 &  &\Checkmark \\ 
        ~ & PADiM~\cite{Defard2020PaDiMAP}  & \blue{0.975}  & 0.775  & 0.698  & 0.582  & 0.959  & 0.663  & 0.858  & 0.535  & 0.832  & 0.760  & 0.764  &\Checkmark  &\Checkmark \\ 
        ~ & PatchCore~\cite{Roth2021TowardsTR}  & 0.876  & 0.880  & 0.791  & 0.682  & 0.912  & 0.701  & 0.695  & 0.618  & 0.841  & 0.702  & 0.770 &\Checkmark &\Checkmark \\ 
        ~ & STEPM~\cite{Wang2021StudentTeacherFP}  & 0.930  & 0.847  & 0.890  & 0.575  & 0.947  & 0.766  & 0.710  & 0.598  & 0.965  & 0.701  & 0.793 &  &\Checkmark \\ 
        ~ & CS-Flow~\cite{Gudovskiy2021CFLOWADRU}  & 0.941  & 0.930  & 0.827  & 0.795  & \blue{0.990}  & 0.886  & 0.731  & 0.471  & \blue{0.986}  & 0.745  & 0.830  &  &\Checkmark \\ 
        ~ & AST~\cite{rudolph2022asymmetric} & 0.947  & 0.928  & 0.851  & \blue{0.825}  & 0.981  & \red{0.951}  & \blue{0.895}  & 0.613  & \red{0.992}  & \blue{0.821}  & \blue{0.880} &   &\Checkmark \\ 
        ~ & M3DM~\cite{Wang2023MultimodalIA} & 0.944  & 0.918  & 0.896  & 0.749  & 0.959  & 0.767  & \red{0.919}  & 0.648  & 0.938  & 0.767  & 0.850 &\Checkmark &\Checkmark \\ 
        ~ & SPADE~\cite{Cohen2020SubImageAD} & 0.771 & 0.793 & 0.760 & 0.531 & 0.848 & 0.683 & 0.646 & 0.460 & 0.879 & 0.502 & 0.687 &\Checkmark &\Checkmark \\ 
        ~ & FastFlow~\cite{Yu12021FastFlowUA} & 0.624 & 0.472 & 0.654 & 0.694 & 0.501 & 0.667 & 0.595 & 0.632 & 0.816 & 0.731 & 0.639 &  &\Checkmark \\ 
        ~ & RD4AD~\cite{Deng2022AnomalyDV} & \blue{0.975} & \blue{0.987} & \red{0.943} & 0.575 & \red{0.999} & 0.830 & 0.863 & 0.618 & 0.984 & \red{0.899} & 0.867 &  &\Checkmark \\ 
        ~ & STPM~\cite{Wang2021StudentTeacherFP} & 0.899 & 0.706 & 0.796 & 0.486 & 0.512 & 0.678 & 0.502 & \blue{0.666} & 0.962 & 0.581 & 0.679 &  &\Checkmark \\ 
        ~ & \textbf{EasyNet(ours)} & \red{0.982}  & \red{0.992}  & \blue{0.917}  & \red{0.953}  & 0.919  & \blue{0.923 } & 0.840  & \red{0.785}  & \blue{0.986}  & 0.742  & \red{0.904} &   &   \\ 
        \hline
        \multirow{11}{*}{\textbf{\thead{RGB+\\Depth}}} & Depth GAN~\cite{Bergmann2021TheM3}  & 0.538  & 0.372  & 0.580  & 0.603  & 0.430  & 0.534  & 0.642  & 0.601  & 0.443  & 0.577  & 0.532 &   &   \\ 
        ~ & Depth AE~\cite{Bergmann2021TheM3} & 0.648  & 0.502  & 0.650  & 0.488  & 0.805  & 0.522  & 0.712  & 0.529  & 0.540  & 0.552  & 0.595 &   &  \\ 
        ~ & Depth VM~\cite{Bergmann2021TheM3}  & 0.513  & 0.551  & 0.477  & 0.581  & 0.617  & 0.716  & 0.450  & 0.421  & 0.598  & 0.623  & 0.555 &  &  \\ 
        ~ & Voxel GAN~\cite{Bergmann2021TheM3} & 0.680  & 0.324  & 0.565  & 0.399  & 0.497  & 0.482  & 0.566  & 0.579  & 0.601  & 0.482  & 0.517 &  &  \\ 
        ~ & Voxel AE~\cite{Bergmann2021TheM3}  & 0.510  & 0.540  & 0.384  & 0.693  & 0.446  & 0.632  & 0.550  & 0.494  & 0.721  & 0.413  & 0.538 &  &  \\ 
        ~ & Voxel VM~\cite{Bergmann2021TheM3}  & 0.553  & 0.772  & 0.484  & 0.701  & 0.751  & 0.578  & 0.480  & 0.466  & 0.689  & 0.611  & 0.609 &  &  \\ 
        ~ & 3D-ST~\cite{Bergmann2022AnomalyDI} & 0.950  & 0.483  & \red{0.986}  & 0.921  & 0.905  & 0.632  & 0.945  & \red{0.988}  & 0.976  & 0.542  & 0.833 & & \Checkmark \\ 
        ~ & PatchCore+FPFH~\cite{Horwitz2022AnEI} & 0.918  & 0.748  & 0.967  & 0.883  & 0.932  & 0.582  & 0.896  & 0.912  & 0.921  & \red{0.886}  & 0.865 &\Checkmark & \\ 
        ~ & AST~\cite{rudolph2022asymmetric} & 0.983  & 0.873  & \blue{0.976}  & \blue{0.971}  & 0.932  & 0.885  & \red{0.974}  & \blue{0.981}  & \red{1.000}  & 0.797  & \blue{0.937} &  &\Checkmark \\ 
        ~ & M3DM~\cite{Wang2023MultimodalIA} & \red{0.994}  & \blue{0.909}  & 0.972  & \red{0.976}  & \red{0.960}  & \blue{0.942}  & \blue{0.973}  & 0.899  & 0.972  & \blue{0.850}  & \red{0.945} &\Checkmark &\Checkmark \\ 
        ~ & \textbf{EasyNet(ours)} & \blue{0.991}  & \red{0.998}  & 0.918  & 0.968  & \blue{0.945}  & \red{0.945}  & 0.905  & 0.807  & \blue{0.994}  & 0.793  & 0.926 & &  \\ 
         \bottomrule[0.5mm]
    \end{tabular}
    }
    \label{tab:mvtec-3d-benchmark}
\end{table*}



\section{Experiments}
\subsection{Experimental Details}

\subsubsection{Datasets}

\revised{We mainly used MVTec 3D-AD~\cite{Bergmann2021TheM3} and Eyescandies~\cite{bonfiglioli2022eyecandies} data sets in the experiment. MVTec 3D-AD dataset is the data set of the real scene, and Eyescandies is the data set of the virtual synthesis. More detailed introduction to these two datasets please refer to the \textit{supplementary materials}.}

\subsubsection{Evaluation Metrics}

Due to the unsupervised experimental setting, the common evaluation metrics we used include Area Under the Receiver Operating Characteristic Curve (AUROC) (I-AUROC and P-AUROC) and the Area Under the Precision-Recall curve (AUPR/AP), the explanation of I-AUROC and P-AUROC please refer to the \textit{supplementary materials}.



\begin{table*}[th]
    \centering
    \caption{I-AUROC score for anomaly detection of all categories of Eyescandies. The best is in red and the second best is in blue.}
        \vspace{-10pt}
    \resizebox{0.88\textwidth}{!}{
    \begin{tabular}{l l | c c c c c c c c c c | c | c | c }
    \toprule[0.5mm]
        ~ & \textbf{Method} & \textbf{\thead{Candy \\Cane}} & \textbf{\thead{Chocolate\\Cookie}} & \textbf{\thead{Chocolate\\Cookie}} & \textbf{Confetto} & \textbf{\thead{Gummy\\Bear}} & \textbf{\thead{Hazelnut\\Truffle}} & \textbf{\thead{Licorice\\Sandwich}} &\textbf{ Lollipop} & \textbf{\thead{Marsh-\\mallow}} & \textbf{\thead{Peppermint\\Candy}} & \textbf{Mean} & \textbf{\thead{Memory \\Bank usage}} & \textbf{\thead{pretrained\\Model Usage}}\\ \hline
        \multirow{5}{*}{\textbf{\thead{Pure\\Depth}}} & Raw~\cite{Horwitz2022AnEI} & \blue{0.654} & 0.510 & 0.563 & 0.451 & 0.433 & 0.454 & 0.472 & 0.515 & 0.626 & 0.366 & 0.504 &\Checkmark & \\ 
         ~ &HoG~\cite{Horwitz2022AnEI} & 0.653 & 0.510 & 0.470 & 0.723 & \blue{0.728} & \blue{0.520} & 0.717 & 0.667 & 0.699 & 0.742 & 0.643 &\Checkmark &\\ 
         ~ &SIFT~\cite{Horwitz2022AnEI} & 0.589 & 0.582 & 0.683 & \red{0.885} & 0.663 & 0.480 & \blue{0.778} & 0.702 & \blue{0.746} & \red{0.790} & 0.690 &\Checkmark &\\ 
         ~ &FPFH~\cite{Horwitz2022AnEI} & \red{0.670} & \blue{0.710} & \red{0.805} & \blue{0.806} & \red{0.748} & 0.515 & \red{0.794} & \red{0.757} & \red{0.765} & \blue{0.757} & \red{0.733} &\Checkmark &\\ 
        ~ & \textbf{EasyNet(ours)} & 0.629  & \red{0.716}  & \blue{0.768}  & 0.731  & 0.660  & \red{0.710}  & 0.712  & \blue{0.711}  & 0.688 & 0.731 & \blue{0.706} & &\\ 
        \hline
        \multirow{7}{*}{\textbf{\thead{Pure\\RGB}}}
         ~ & GANomaly~\cite{Akay2018GANomalySA} & 0.485  & 0.512  & 0.532  & 0.504  & 0.558  & 0.486  & 0.467  & 0.511  & 0.481  & 0.528  & 0.507 & & \\ 
        ~ & DFKDE~\cite{anomalib} & 0.539  & 0.577  & 0.482  & 0.548  & 0.541  & 0.492  & 0.524  & 0.602  & 0.658  & 0.591  & 0.555 &  &\Checkmark \\ 
         ~ & DFM~\cite{Ahuja2019ProbabilisticMO} & 0.532  & 0.776  & 0.624  & 0.675  & 0.681  & 0.596  & 0.685  & 0.618  & 0.964  & 0.770  & 0.692 &  &\Checkmark \\ 
         ~ & STEPM~\cite{Wang2021StudentTeacherFP}  & \blue{0.551}  & 0.654  & 0.576  & 0.784  & \blue{0.737}  & \blue{0.790}  & 0.778  & 0.620  & 0.840  & 0.749  & 0.708  &  &\Checkmark \\ 
         ~ & PaDiM~\cite{Defard2020PaDiMAP} & 0.531  & 0.816  & \blue{0.821}  & \blue{0.856}  & \red{0.826}  & 0.727  & \blue{0.784}  & 0.665  & \red{0.987}  & \blue{0.924}  & \blue{0.794} & \Checkmark  &\Checkmark\\ 
        ~ & AutoEncoder~\cite{Bonfiglioli2022TheED} &0.527  & \blue{0.848}  & 0.772  & 0.734  & 0.590  & 0.508  & 0.693  & \blue{0.760}  & 0.851  & 0.730  & 0.701  & & \\ 
        ~ & \textbf{EasyNet(ours)} & \red{0.723}  & \red{0.925}  & \red{0.849}  & \red{0.966}  & 0.705  & \red{0.815}  & \red{0.806}  & \red{0.851}  & \blue{0.975}  & \red{0.960}  & \red{0.858} & & \\ \hline
        \multirow{3}{*}{\textbf{\thead{RGB+\\Depth}}} & AutoenEoder~\cite{Bonfiglioli2022TheED} &0.529  & 0.861  & 0.739  & 0.752  & 0.594  & 0.498  & 0.679  & 0.651  & 0.838  & 0.750  & 0.689 & &\\
        ~ & PatchCore+FPFH~\cite{Horwitz2022AnEI} & \blue{0.606}  & \blue{0.904}  & \blue{0.792}  & \blue{0.939}  & \red{0.720}  & \blue{0.563}  & \blue{0.867}  & \blue{0.860}  & \red{0.992}  & \blue{0.842}  & \blue{0.809} &\Checkmark &\Checkmark\\
        ~ & \textbf{EasyNet(ours)} & \red{0.737}  &  \red{0.934}  &  \red{0.866}  &  \red{0.966} &  \blue{0.717}  &  \red{0.822}  &  \red{0.847}  &  \red{0.863} & \blue{0.977}  &  \red{0.960}  & \red{0.869}  & &  \\  
         \bottomrule[0.5mm]
    \end{tabular}
    }
    \label{tab:eyescandies-benchmark}
\end{table*}

\begin{table*}[th]
    \centering
    \caption{Ablation studies on an attention-based information entropy fusion module. The best is in red and the second best is in blue. }
        \vspace{-10pt}
    \resizebox{0.85\textwidth}{!}{
    \begin{tabular}{l|l | c c c c c c c c c c c}
    \toprule[0.5mm]
    \multirow{4}{*}{\textbf{MVTec 3D-AD}} & \textbf{Evaluation} & \textbf{Bagel} & \textbf{Cable gland} & \textbf{Carrot} & \textbf{Cookie} & \textbf{Dowel} & \textbf{Foam} & \textbf{Peach} & \textbf{Potato} & \textbf{Rope} & \textbf{Tire} & \textbf{Mean} \\ 
    \cline{2-13}
         & Gate Close & \blue{0.982}  & 0.992  & \blue{0.917}  & \blue{0.953}  & 0.919  & 0.923  & 0.840  & \blue{0.785}  & \blue{0.986}  & \blue{0.742}  & 0.904  \\ 
        & Gate Open & 0.974  & \blue{0.996}  & 0.914  & \red{0.968}  & \blue{0.941}  & \blue{0.938}  & \blue{0.882}  & 0.781  & 0.982  & \red{0.793}  & \blue{0.917}  \\ 
        &Gate Control & \red{0.991}  & \red{0.998}  & \red{0.918}  & \red{0.968}  & \red{0.945}  & \red{0.945}  & \red{0.905}  & \red{0.807}  & \red{0.994}  & \red{0.793}  & \red{0.926}  \\ 
        \Xhline{1.5pt}
   \multirow{4}{*}{\textbf{Eyescandies}} & \textbf{Evaluation} & \textbf{\thead{Candy \\Cane}} & \textbf{\thead{Chocolate\\Cookie}} & \textbf{\thead{Chocolate\\Cookie}} & \textbf{Confetto} & \textbf{\thead{Gummy\\Bear}} & \textbf{\thead{Hazelnut\\Truffle}} & \textbf{\thead{Licorice\\Sandwich}} &\textbf{ Lollipop} & \textbf{\thead{Marsh-\\mallow}} & \textbf{\thead{Peppermint\\Candy}} & \textbf{Mean} \\ 
   \cline{2-13}
         & Gate Close & \blue{0.723}  & \blue{0.925}  & \blue{0.849}  & \red{0.966 } & \blue{0.705}  & \blue{0.815}  & 0.806  & \blue{0.851}  & \blue{0.975}  & \red{0.960}  & \blue{0.857}  \\  
        & Gate Open & 0.722  & 0.919  & 0.827  & \blue{0.945}  & 0.685  & 0.813  & \blue{0.846}  & 0.850  & \blue{0.975}  & \blue{0.959}  & 0.854  \\ 
        & Gate Control & \red{0.737}  & \red{0.934} & \red{0.866}  & \red{0.966}  & \red{0.717}  & \red{0.822}  & \red{0.847}  & \red{0.863}  & \red{0.977}  & \red{0.960}  & \red{0.869}\\ 
        \bottomrule[0.5mm]
    \end{tabular}
    }   
    \label{tab:gate-abalation-study}
\end{table*}

\begin{table*}[th]
    \centering
    \caption{The ablation study for the number of fusion layers. The best is in red and the second best is in blue. }
        \vspace{-10pt}
    \resizebox{0.85\textwidth}{!}{
    \begin{tabular}{{l|l c c c c c c c c c c c}}
    \toprule[0.5mm]
        \textbf{RGB+Depth} & \textbf{Evaluation} & \textbf{Bagel} & \textbf{Cable gland} & \textbf{Carrot} & \textbf{Cookie} & \textbf{Dowel} & \textbf{Foam} & \textbf{Peach} & \textbf{Potato} & \textbf{Rope} & \textbf{Tire} & \textbf{Mean} \\ \hline
        
        \multirow{3}{*}{Image AUC} & 1 layer & \blue{0.967} & 0.981 & \blue{0.912} & \blue{0.890} & 0.901 & 0.908 & \blue{0.763} & \blue{0.717} & \red{1.000} & 0.731 & 0.877 \\ 
        ~& 2 layers & \red{0.974} & \blue{0.996} & \red{0.914} & \red{0.968} & \blue{0.941} & \red{0.938} & \red{0.882} & \red{0.781} & \blue{0.982} & \blue{0.793} & \red{0.917} \\
        ~& 3 layers & 0.964 & \red{1.000} & 0.884 & 0.889 & \red{0.967} & \blue{0.932} & 0.734 & 0.697 & \red{1.000} & \red{0.866} & \blue{0.893} \\ \hline

        \multirow{3}{*}{Image AP} & 1 layer & \blue{0.992} & 0.995 & \red{0.982} & \blue{0.969} & 0.976 & 0.975 & \blue{0.915} & 0.873 & \red{1.000} & 0.906 & 0.958 \\ 
        ~& 2 layers & \red{0.994} & \blue{0.999} & \blue{0.981} & \red{0.991} & \blue{0.984} & \red{0.984} & \red{0.966} & \red{0.930} & \blue{0.992} & \blue{0.927} & \red{0.975} \\ 
        ~& 3 layers & 0.991 & \red{1.000} & 0.975 & 0.966 & \red{0.992} & \blue{0.982} & 0.903 & \blue{0.916} & \red{1.000} & \red{0.963} & \blue{0.969} \\ \hline

        \multirow{3}{*}{Pixel AUC} & 1 layer & 0.875 & \blue{0.861} & \blue{0.963} & \blue{0.678} & 0.716 & \red{0.998} & \blue{0.969} & \blue{0.921} & 0.928 & \blue{0.861} & \blue{0.877} \\ 
        ~& 2 layers & \red{0.935} & \red{0.941} & \red{0.971} & \red{0.897} & \red{0.885} & \blue{0.997} & \red{0.992} & 0.888 & \blue{0.955} & 0.728 & \red{0.919} \\ 
        ~& 3 layers & \blue{0.904} & 0.717 & 0.836 & 0.651 & \blue{0.809} & \blue{0.997} & 0.914 & \red{0.942} & \red{0.986} & \red{0.970} & 0.873 \\ \hline

        \multirow{3}{*}{Pixel AP} & 1 layer & 0.020 & 0.025 & \blue{0.097} & \blue{0.015} & 0.026 & \red{0.803} & \blue{0.081} & \blue{0.032} & 0.133 & \red{0.163} & \blue{0.139} \\ 
        ~& 2 layers & \blue{0.039} & \red{0.062} & \red{0.188} & \red{0.025} & \blue{0.034} & 0.562 & \red{0.298} & \red{0.034} & \blue{0.144} & 0.031 & \red{0.142} \\ 
        ~& 3 layers & \red{0.041} & \blue{0.043} & 0.084 & 0.008 & \red{0.121} & \blue{0.648} & 0.017 & 0.014 & \red{0.249} & \blue{0.134} & 0.136 \\ 
        \bottomrule[0.5mm]
    \end{tabular}
    }
    \label{table:ablation_number_layer}
\end{table*}


\subsection{Experimental Results and Analysis}
\label{subsec:main_result}

\subsubsection{RGB+Depth on MVTec 3D-AD and Eyescandies}

Table~\ref{tab:mvtec-3d-benchmark} and Table~\ref{tab:eyescandies-benchmark} clearly demonstrate that EasyNet achieves the state-of-the-art performance on MVTec 3D-AD and Eyescandies without using a pretrained model and memory bank. Specifically, EasyNet outperforms AutoEncoder~\cite{Bonfiglioli2022TheED} and PatchCore+FPFH~\cite{Horwitz2022AnEI} in RGB+Depth setting of Eyescandies by a large margin, 20.7\% and 6.9\%, respectively. For MVTec 3D-AD, AST~\cite{rudolph2022asymmetric} and M3DM~\cite{Wang2023MultimodalIA} are the cutting-edge models in MVTec 3D-AD. However, both of them use large pretrained models or memory banks. In specific, M3DM uses two pretrained models (Point Transformer and Vision Transformer) to extract the features from depth images and RGB images. In addition, M3DM employs two large memory banks (average 6.098 GB) to store the features from depth images and RGB images. Due to strict storage limitations in practice, M3DM cannot be perfectly fit in real-world applications. Although PatchCore+FPFH~\cite{Horwitz2022AnEI} has a relatively small memory footprint (249.260 MB on average), the actual performance is not as good as Easynet. The memory bank size of M3DM and PatchCore+FPFH on the MVTec 3D-AD can be found in \textit{supplementary materials}. AST also adopts two EfficientNet-B5 as the feature extractors for depth images and RGB images, which violate the storage limitation in IM. Moreover, according to Table~\ref{tab:inference-accuracy}, massive usage of pretrained models will slow down the inference speed, which cannot meet the requirement of IM. Furthermore, the performance gap among EasyNet, AST and M3DM is very small, 2.1\% and 1.2\%. Hence, EasyNet is the best 3D-AD model to meet all the demands of IM.

\subsubsection{Pure RGB Performance}

In real-world applications, the limitation of the depth sensor is very large since the effective distance range of the depth sensor is 3 meters. In addition, most depth sensors are easily affected by the lighting condition. Hence, as for simulating the failure of the depth sensor, we conducted the experiment using only RGB images as the input. In the pure RGB branch of Table~\ref{tab:mvtec-3d-benchmark} and Table~\ref{tab:eyescandies-benchmark}, EasyNet achieves state-of-the-art performance in I-AUROC in the pure RGB track. In particular, EasyNet outperforms in pure RGB of Eyescandies by a large margin, 15.7\% to AutoEncoder~\cite{Bonfiglioli2022TheED} and 6.4\% to PaDiM~\cite{Defard2020PaDiMAP}. Moreover, EasyNet gets 5.97\% better I-AUROC score than M3DM and 2.65\% better I-AUROC score than AST. In total, The performance of EasyNet is robust even though the depth sensor is a failure. 

% Figure environment removed

\subsubsection{Attention-based Information Entropy Fusion Module}
\label{subsubsec:Information_Entropy}
Table~\ref{tab:gate-abalation-study} clearly illustrates the effectiveness of our proposed attention-based information entropy fusion module in EasyNet. \revised{The gate network is the key to control multi-feature fusion.} We conduct the ablation studies on three options, Gate Close, Gate Open and Gate Control, respectively. Gate Close means that EasyNet only utilizes RGB images as the input and ignores depth information. Gate Open denotes that EasyNet uses both the RGB images and the depth images and combines their features for evaluation. Gate Control means that EasyNet adopts an attention-based information entropy fusion module to select depth features during the inference case. In Gate Open, we discover that depth information may work as the noise and degrade the total performance if we select both RGB features and depth features during inference, so we design an attention-based information entropy fusion module to select the feature for fusion, which can enhance the performance of all classes in MVTec 3D-AD and Eyescandies. 

\subsubsection{Ablation study on the number of fusion layers}
\label{subsubsec:ablation_study_number}

As described in Section~\ref{subsubsec:multi_seg_net}, the MSN is utilized for segmenting anomalies by fusing enhanced multilayer image features and reconstructed ones. Table~\ref{table:ablation_number_layer} presents that EasyNet performs optimally when incorporating features from only two layers. Specifically, the performance metrics of I-AUROC and P-AUROC are respectively improved by 4.36\% and 4.57\% with two layers compared to one layer, and they are further increased by 2.62\% and 5.27\% with three layers compared to one layer. These results indicate that the use of two feature layers can achieve the best performance in both anomaly detection and localization. In contrast, using three feature layers leads to performance degradation, which verifies our hypothesis. The deepening of the multi-modality reconstruction network can gradually eliminate some abnormal feature parts, whereas using three feature layers introduces more feature removal of abnormal parts, thereby leading to poor discriminator performance.

\subsubsection{Accuracy VS Inference Speed} \label{sec:accuracy_vs_inference}

\begin{wraptable}{r}{0.23\textwidth}
    \centering
    \caption{Inference abilities.}
        \vspace{-10pt}
    \scalebox{0.8}{
    \begin{tabular}{l| c c}
    \toprule[0.3mm]
        \textbf{Method} & \textbf{I-AUROC} & \textbf{FPS} \\ \hline
        BTF~\cite{horwitz2022back} & 0.865 & 27.92 \\ 
        AST~\cite{rudolph2022asymmetric} & 0.937 & 41.94 \\ 
       M3DM~\cite{Wang2023MultimodalIA} & 0.945 & 0.10 \\ 
        \textbf{EasyNet(ours)} & \textbf{0.926} & \textbf{94.55} \\ 
        \bottomrule[0.3mm]
    \end{tabular}}
    \label{tab:inference-accuracy}
\end{wraptable}
As we previously described in Section~\ref{sec:introduction}, inference speed is one of the important factors to be considered in IM. Since the real production lines need to check each product in real time. Table~\ref{tab:inference-accuracy} shows EasyNet obtains the fastest speed among the cutting-edge anomaly detection methods without sacrificing performance. In particular, EasyNet gets 125\% FPS better than AST and 93900\% FPS better than M3DM. Regarding performance, the performance gap among EasyNet, AST and M3DM is very small, 2.1\% and 1.2\%. Therefore, EasyNet is the most deployment-friendly 3D-AD method for IM. 


\subsection{Visualization}

\revised{Figure~\ref{fig:result_of_mvtec_3d_ad} visualizes the performance of EasyNet on MVTec 3D-AD, demonstrating the effectiveness of the proposed method.
In Figure~\ref{fig:result_of_mvtec_3d_ad}, compared to existing 3D anomaly detection methods (AST~\cite{rudolph2022asymmetric}), EasyNet can significantly reduce false positive rates and achieve higher segmentation accuracy. Furthermore, the fusion method reduces false positives for anomalies such as \textit{cable gland}, compared to using only RGB images. Anomalies in \textit{peach} and \textit{potato} are also more clearly visible on depth images, indicating the importance of using depth image information in industrial AD. In addition, EasyNet is not affected by the domain gap between natural and industrial images and has a higher inference speed than existing methods. Note that we put the visualization results of EasyNet on Eyescandies in \textit{supplementary materials}.}

\section{Conclusions}
This paper addresses a promising and challenging task, i.e., deployment-friendly 3D-AD and proposes an easy but effective neural network (termed as EasyNet) to achieve competitive performance without using large pretrained models and memory banks. Specifically, as for getting rid of large pretrained models and memory banks, EasyNet employs MRN to implicitly detect and reconstruct the anomalies with semantically plausible anomaly-free content, while keeping the non-anomalous regions of the input image unchanged. Meanwhile, EasyNet proposes an MSN to produce an accurate anomaly segmentation map from the concatenated reconstructed RGB images and depth images and their original appearances. In the test phase, EasyNet adopts a self-attention information entropy score in the early fusion stage to select the informative depth features before fusing with RGB features. To this end, EasyNet achieves the fastest inference speed without sacrificing performance. 


\section{Acknowledgments}
This work was supported by the National Key R\&D Program of China (Grant NO. 2022YFF1202903), the National Natural Science Foundation of China (Grant NO. 62122035, 62206122), and the Key-Area Research and Development Program of Guangdong Province (2020B0101130003).

\bibliographystyle{ACM-Reference-Format}
\balance
\bibliography{sample-base}

\newpage
\beginsupplement
\begin{refsection}

\section{Pseudocode Example of Cumulative Disruption Algorithm} \label{sec:psuedocode}

For readers seeking a succinct code-like description of our cumulative disruption curve algorithm, we have included \cref{lst:psuedocode}.

\begin{lstlisting}[label=lst:psuedocode, language=Python, caption=Pseudocode for disruption algorithm]
disruption = []
for c in communities:
    remaining = 0
    original = 0
    removeCommunity(c)
    for user in users:
        if degree(user) > 0:
            remaining += degree(user)
            original += originalDegree(user)
    disruption += [1 - (remaining / original)]
\end{lstlisting}

Note that when calculating disruption on large networks, it is much more efficient to cache the size of the smallest community that each user participates in. We can then sort all users by the order in which they will be removed, and avoid computationally expensive references to a graph or adjacency matrix for each removal-step in the algorithm.

\section{Applications to Unipartite Networks} \label{sec:unipartite}

Our influence metric is intended for settings with clearly defined communities. For example, participation in subreddits, membership on a Mastodon server, or committing to a software code repository, all discretely identify users as members of those explicitly-bounded groups. However, network data is often presented in a unipartite configuration such as users following other users. If it is still desirable to delineate communities and measure their influence in these settings, then they can be converted into compatible bipartite networks using the following procedure:

\begin{enumerate}
    \item Apply a context-appropriate community detection algorithm to label each user as belonging to one community

    \item Create a vertex for each community

    \item Replace all user-user edges with user-to-community edges, where the edge weight is equal to the number of unipartite edges each user had to other nodes in that community

    \item Apply our influence metric to the resulting bipartite graph
\end{enumerate}

An example of this procedure is illustrated in \cref{fig:unipartite}, using a unipartite Watts-Strogatz small-world network (100 nodes, 5 neighbors, rewiring probability of 5\%), and label-propagation for community detection. The unipartite graph is shown in the top-left with community labels visualized with color. It is converted to a bipartite representation shown in the upper-right, and the effect of removing each community is illustrated in the bottom frame.

% Figure environment removed






\section{Calculating the Area Under the Disruption Curve} \label{sec:auc_explanation}

For \cref{fig:real_networks_auc,fig:toy_networks_auc,fig:assortativity_auc} we use the area under the disruption curve as a single-variable summary of how centralized a network is around its largest communities. To calculate the AUC, we use a trapezoidal approximation in logarithmic space.

We chose a trapezoidal approximation to calculate the area even with limited sample points from real-world networks. Integration is possible for purely analytic disruption curve simulations as in \cref{sec:analytic_simulations}, but this is not feasible for our non-Erd\H{o}s-R\'{e}nyi networks, so we use a trapezoidal approximation for all synthetic networks for consistency.

We measure the AUC in logarithmic space, because measuring in linear space would heavily weight the influence of the smallest communities that are removed last, and our primary interest is in examining the influence of the largest communities on the broader population. 

\section{Synthetic Network Topology Details} \label{sec:toy_examples}

We measure centralization on a variety of synthetic networks introduced in \cref{sec:disruption_toy}. In this section, we include further description and visualization of the synthetic networks used.

Bipartite Near-Star networks are analogous to a unipartite star network with duplicate edges, but in a bipartite setting. Starting with a unipartite star, replace each edge from the hub to a leaf with a two-path from the hub community to a new ``user" vertex, to the leaf community. Duplicate edges from the unipartite hub to leaves are converted into multiple users that share a community, and serve to break ties when pruning communities for disruption curves. This is illustrated in \cref{fig:star}.

% Figure environment removed

For our ``Powerlaw" networks we follow a bipartite configuration model. We first create vertices representing the desired number of communities and users. We then draw from a powerlaw distribution with an assigned $\gamma$ exponent, and assign the drawn degree to each community. Then, we create a corresponding number of edges, wiring each community to users drawn uniformly at random without replacement. This yields networks where communities follow a powerlaw degree distribution, while users follow a normal degree distribution.

Bipartite community-user networks can be visualized in a flat plane, as in \cref{fig:centralization-pl}, or as a multi-layer graph, as in \cref{fig:pl-toy}. A multi-layer representation may be beneficial for representing inter-community relationships that are not explained by shared users, such as Mastodon federation agreements, or shared moderator staff in two subverses. However, these multiplex relationships were deemed out-of-scope for our current work.

% Figure environment removed




\begin{comment}
  #data structure for the dispersion metric
  D = np.zeros(nm)

  #calculate dispersion
  cumu_sum=0
  for n in np.arange(0,nm):
    cumu_sum += n*pn[n]
    #calculate U_n
    if pn[n]==0:
      continue
    Pnpm = Pnm[n,:]/np.sum(Pnm[n,:])
    U=0
    for m in np.arange(0,mm):
      if(np.sum(Pnm[:,m])>0):
        Pnmp = Pnm[:,m]/np.sum(Pnm[:,m])
        prob = np.sum(Pnmp[n:-1])
        U+=Pnpm[m]*prob**(m-1)
    D[n] = n*pn[n]*(1-U)/(cumu_sum-n*pn[n]*U)
\end{comment}

\section{Mathematical Analysis of Disruption in Random Networks} \label{sec:analytic_simulations}

We here calculate the disruption curves for random bipartite networks parameterized by their joint-degree distribution. This approach therefore fixes the distribution $\lbrace g_m \rbrace$ of communities $m$ per user, the distribution $\lbrace p_n \rbrace$ of community size $n$, and the joint-distribution $P_{n,m}$ for the degree of the node and community involved in a random bipartite link. Beyond these constraints, the networks are fully random but allow us to explore the role of heterogeneous connectivity at the user and community level as well as the impact of correlations between both levels.

We wish to calculate the disruption $D(n)$ involved when removing communities of size $n'<n$ in these random networks. By definition of the bipartite network, we know that $np_n$ edges are removed when removing communities of size $n$. Once again, we define disruption as the fraction of \textit{remaining} edges disrupted by communities of size $n$ during the pruning process. It is thus given by the number of edges that belong to communities of size $n$ minus the fraction $u_n$ of those that are the sole edge of the corresponding users (since these users are removed in the pruning) divided by the number of edges belonging to communities of size equal or smaller than $n$ minus the $u_nnp_n$ users removed. We write:

\vspace{2em}
\begin{equation}
    D(n) = \frac{
            \eqnmarkbox[NavyBlue]{bigedges}{np_n}
            -
            \eqnmarkbox[OliveGreen]{prunededges}{u_nnp_n}
        }{
            \eqnmarkbox[WildStrawberry]{remainingedges}{\sum_{n'\leq n}n'p_{n'}}
            -
            \eqnmarkbox[OliveGreen]{prunededges2}{u_nnp_n}
        } \; .
    % Here's Laurent's original expression
    %D(n) = \frac{np_n-u_nnp_n}{-u_nnp_n + \sum_{n'\leq n}n'p_{n'}} \; .
\end{equation}
\annotate[yshift=1em]{above,left}{bigedges}{Edges to comms. of size n}
\annotate[yshift=1em]{above,right}{prunededges}{Edges to removed users}
%\annotate[yshift=-1em]{below,right}{prunededges2}{Edges for removed users}
\annotate[yshift=-0.5em]{below}{remainingedges}{Edges to comms. n or smaller}
\vspace{2em}

The quantity $u_n$ can also be defined as the probability that a random user of a community of size $n$ has no community smaller than $n$. It can therefore be calculated like so:

\vspace{1em}
\begin{equation}
    u_n = \mathlarger{\sum}_m 
        \eqnmarkbox[NavyBlue]{users_in_n_with_m}{\frac{P_{n,m}}{\sum_{m'}P_{n,m'}}}
        \left(
            \eqnmarkbox[OliveGreen]{users_with_m_larger_than_n}{\frac{\sum_{n'\geq n} P_{n',m}}{\sum_{n'}P_{n',m}}}
        \right)^{m-1} \; .
    %u_n = \mathlarger{\sum}_m \frac{P_{n,m}}{\sum_{m'}P_{n,m}} \left(\frac{\sum_{n'\geq n} P_{n',m}}{\sum_{n'}P_{n',m}}\right)^{m-1} \; .
    \label{eq:un}
\end{equation}
\annotate[yshift=1em]{above,right}{users_in_n_with_m}{Fraction of users in comm. \\ \sffamily \footnotesize size n that have m edges}
\annotate[yshift=-0.5em]{below,left}{users_with_m_larger_than_n}{Fraction of users with m edges\\ \sffamily \footnotesize in comms. larger than size n}
\vspace{2.5em}

In the previous equation, we sum over every possible type of node in a community of size $n$, which will have a number of \textit{other} communities $m-1$ proportional to $P_{n,m}$, and ask for all of these communities to be larger or equal to $n$, which will be proportional to the sum of $P_{n',m}$ over all $n'$ larger or equal to $n$. Normalizing the probabilities appropriately yields Eq. (\ref{eq:un}) as written.

Note that these equations assume that edges are unweighted, and that there are no duplicate edges, which is what we expect from an infinite random simple graph. In our real-world data sets there are often duplicate edges (for example, one user following several different users on a Mastodon instance), which we compress to weighted edges for convenience.

Despite this difference between the analytical expression and real socio-technical networks, the analysis of random infinite graphs can be useful to test how disruption is impacted by simple network statistics such as degree distributions or correlations in the joint community-user degree matrix $P_{n,m}$. 

In a simple experiment, we create a random Erd\H{o}s-R\'{e}nyi-like bipartite network and correlated equivalent networks with the same degree distributions and variable community-user degree matrices $P_{n,m}$. The random network has a simple $P^{\textrm{rand}}_{n,m} \propto np_n mg_m$ (normalized) which we can modify manually. To do so, we calculate the maximally correlated $P^{\textrm{max}}_{n,m}$ by assigning users with highest degrees $m_{\textrm{max}}$ to the largest communities available before doing the same to users with the next higher degree and so on all the way down. We can do the same to calculate $P^{\textrm{min}}_{n,m}$ by assigning users with the lowest degree to the largest communities and working our way up in the user degree distribution. We can then create arbitrary community-user degree matrix $P_{n,m}$ by interpolating between linearly with $(1-\rho) P^{\textrm{rand}}_{n,m} + \rho P^{\textrm{max}}_{n,m}$ or $(1-\rho) P^{\textrm{rand}}_{n,m} + \rho P^{\textrm{min}}_{n,m}$.

Our results are shown in \cref{fig:assortivity_random_networks}. We find that positive user-community degree correlations increase disruption and therefore \textit{centralizes} the resulting socio-technical network. Conversely, negative correlations decreases correlations and \textit{decentralizes} the network. That being said, the relative effect of correlations is relatively small as the networks are still otherwise completely random.

% Figure environment removed

\section{Further Analysis of Assortativity} \label{sec:supplemental_assortativity}

There are multiple interpretations of degree assortativity in a bipartite setting. The linear correlation between user degrees and community degrees measures whether high-degree users are likely to be connected to high-degree communities. In our network definitions edges represent activity, like follow relationships or participation in conversations, so this measures whether active users are likely to be connected to communities with lots of activity. However, a second metric of interest is whether large communities are likely to be connected to other large communities, or in other words, the  assortativity of a unipartite-projected community-community graph. This can also be broken into two sub-cases: assortativity of community size (do communities with many users share users with other high-population communities), and assortativity of degree (do communities with lots of activity share users with other high-activity communities).

These three notions of assortativity are not independent; we might expect that users with lots of activity are active in communities with high populations, and may act as bridges between multiple communities with high activity and high population. However, the three metrics are not guaranteed to correlate and should be measured separately.

While rewiring to promote user-community degree assortativity, we also plotted the changes in community-community degree assortativity, shown in \cref{fig:assortivity_user_vs_community}. Strikingly, the community assortativity \textit{decreases} as we rewire to promote user assortativity. This is because as we rewire edges to focus user connections on the largest communities we implicitly decrease the number of edges between communities. This also matches the changes in disruption in \cref{fig:assortativity_auc}: increasing assortativity may reconnect large and insular communities with the rest of the network, briefly increasing their influence, but continued assortativity rewiring also cuts bridges to and between smaller communities, yielding a sparse network that is far less centralized.

% Figure environment removed

To further explore the relationship between these types of assortativity, we also rewired networks in the reverse direction: for randomly selected pairs of edges, we rewired those edges to \textit{decrease} user to community activity assortativity. We have plotted the change in disruption curves (\cref{fig:disassortative_auc}) and correlation between assortativity metrics (\cref{fig:disassortivity_user_vs_community}). In most networks, decreasing activity assortativity lowers centralization, although the effect diminishes as the network topology more closely approximates a random network. The one exception is the Penumbra; this network has such sparse inter-community connections that any perturbation of edges increases the cross-community links and therefore \textit{increases} centralization.

% Figure environment removed

% Figure environment removed

\section{Cumulative Impact on Giant Component Size} \label{sec:giant_components}

Some readers may be interested in how removing large communities influences the giant component size on each network. This is closely related to the cumulative population size in the top sub-plots of \cref{fig:real_networks_size_comparison} and \cref{fig:toy_networks_size_comparison}. Intuition suggests that the size of the giant component will be inversely proportional to the number of cumulative communities removed; as more large communities are pruned, the giant component should shrink. This relationship holds so long as the remaining communities are interlinked, but falters once a ``bridge" community is removed and the giant component splinters. Therefore, sparsely connected networks where bridges are more prominent will have a chaotic giant component size, while more densely connected networks will present a smooth curve until most communities are pruned. This relationship is illustrated in \cref{fig:real_giant_component}. Most curves are smooth until the tail of the distribution, with two notable exceptions: Voat's giant component changes once the largest insular communities are removed (see \cref{fig:voat_render}), and the Penumbra's curve is much ``spikier" as a result of its highly sparse structure.

% Figure environment removed

Measuring the change in giant component size captures some of the same features as our disruption metric. In particular, removing large insular communities may not change the giant component size if the community is completely isolated from the giant component, so this captures some aspect of both the size and topological role of a community. However, the impact of a community is boolean: if it touches the giant component, then removing the community will shrink the giant component by the size of that community. There is no distinction between a minimally integrated and tightly integrated community. Measuring the impact of a community in terms of fraction of edges severed, rather than component vertex size, offers finer insight into the interplay between size distribution and network structure.



\section{Comparison to Network Bottlenecking} \label{sec:cheeger}

The Cheeger number \cite{cheeger} is a single-valued metric representing how large of a ``bottleneck" inhibits conductance across a graph. It is typically written as:

\vspace{2em}
\begin{equation}
    h(G) = \min \left\{
        \frac{
                \eqnmarkbox[NavyBlue]{cheeger_crossedges}{|\partial A|}
            }{
                \eqnmarkbox[OliveGreen]{cheeger_alledges}{|A|}
            }
        : \eqnmarkbox[WildStrawberry]{cheeger_subset}{A \subseteq V(G)}, 
        \eqnmarkbox[Plum]{cheeger_bounds}{0 < |A| \leq \frac{1}{2} |V(G)|}
    \right\} 
\end{equation}
\annotate[yshift=1.2em]{above}{cheeger_crossedges}{Edges crossing the boundary of A}
\annotate[yshift=-0.2em]{below}{cheeger_alledges}{All edges in+across A}
\annotate[yshift=0.8em]{above}{cheeger_subset}{A is a subset of vertices of G}
\annotate[yshift=-2em]{below,left}{cheeger_bounds}{A contains at most half of all vertices}
\vspace{2em}

Our measurement of how much a community influences a larger population, and the Cheeger measurement of whether a community is a ``bottleneck" bear some conceptual similarities. Therefore, we compare our metric to the Cheeger number in two ways. First, we create a ``local Cheeger number," following an identical equation $\frac{|\partial A|}{|A|}$, but where $A$ is defined as the set of communities we are pruning, rather than via a global search. Second, we estimate bounds on the global Cheeger value of the graph. Since evaluating the graph conductance of all possible subsets of vertices is an NP-hard problem \cite{kaibel2004expansion}, it is impractical to directly measure the Cheeger constant on most large graphs. Fortunately, the Cheeger inequality offers upper and lower bounds on the Cheeger number based on the second eigenvalue of the normalized Laplacian of the adjacency matrix of G as follows:

$$\lambda_2/2 \leq h(G) \leq \sqrt{2\lambda_2}$$

Since they are sparse, these bounds can be calculated even on large real-world datasets. 
Unfortunately, in our tests the bounds are quite wide (see \cref{fig:cheeger}), limiting the utility of this approximation. We have plotted a comparison of the ``local" Cheeger number, bounds of the global Cheeger number, and our disruption metric, for a variety of simulated networks.

% Figure environment removed

\printbibliography[heading=subbibliography]
\end{refsection}


\end{document}
\endinput
%%
%% End of file `sample-sigconf.tex'.
