\section{Related Work}
\label{sec:related}
Unlike key-value stores, database systems, or conventional SPEs, TSPEs such as Streaming-Ledger~\cite{Transactions2018}, \sstore~\cite{S-Store}, FlowDB~\cite{Affetti:2017:FIS:3093742.3093929}, \tstream~\cite{tstream}, and \system are based on a unique computational model, where each input tuple from data streams may involve multiple keys. Thus the processing of tuples can lead to potentially conflicting shared mutable state accesses. 
Such a unique system feature has been originally motivated by a list of stream applications~\cite{ACEP,Botan09} and is applied or encouraged to be applied in emerging use case scenarios~\cite{meehan2017data,Transactions2018,10.1007/978-3-030-19274-7_10}. 
Some of the novel design challenges and optimization opportunities of TSPEs have been discussed in previous works~\cite{S-Store,tstream}. 
The experimental results showed previously in Figure~\ref{fig:overview_comparison} also confirm that conventional SPEs can not efficiently handle the targeted applications of TSPEs.

Executing each state transaction one by one following the event sequence naturally leads to the correct schedule but seriously limits system concurrency~\cite{ACEP}. 
Recent works have proposed adopting partitioning and decomposition to optimize the performance of transaction processing, such as~\cite{Bernstein:1999:CCS:337919.337922,Recovery,Shasha:1995:TCA:211414.211427,BengChinTKDE16}. Similar ideas have also been adopted in TSPEs. 
For example, S-Store~\cite{S-Store} adopts state partitioning with extensions of guaranteeing state access ordering~\cite{S-Store-demo}, while TStream~\cite{tstream} adopts transaction decomposition to improve multicore scalability further. 
However, each existing system is designed with a non-adaptive scheduling strategy and favours a subset of workload characteristics.
\system deviates from existing solutions. 
It explores fine-grained workload characteristics of every batch of state transactions. It then makes the correct scheduling based on a decision model to morph the current scheduling strategy into a better-performing strategy.

Despite the large body of research on the scheduling problem in a general context~\cite{hall1997scheduling,T-storm,briskstream}, task scheduling for TSPEs presents subtle but unique requirements~\cite{10.1145/167088.167254,Robert2011,hou1994genetic,669967}, largely due to the integrated stream processing and transactional contexts~\cite{tstream}. For instance, 
the scheduling unit can be reconfigured by the system, e.g., by splitting state transactions into operations and regrouping by keys. It is thus difficult (if not impossible) to quantitatively model the objective function of scheduling plans in TSPEs. We hence propose to model the scheduling of TSPEs into a three-dimensional scheduling decision problem and guide it with a heuristic-based decision model. Furthermore, the scheduling overhead is now on the critical path, prohibiting any sophisticated optimization algorithms. 