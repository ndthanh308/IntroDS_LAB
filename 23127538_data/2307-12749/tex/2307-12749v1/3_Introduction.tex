\section{Introduction}
% The increasing demand for real-time stream processing in the context of data-intensive applications and the Internet of Things (IoT) has inspired the development of numerous Stream Processing Engines (SPEs), including Storm~\cite{storm}, Flink~\cite{flink}, and Spark-Streaming~\cite{spark}. However, supporting emerging stream applications involving shared mutable states — concurrently read and modified by multiple execution entities — poses significant challenges for mainstream SPEs, leading to issues with correctness~\cite{S-Store-demo} and efficiency~\cite{tstream}. To address these concerns, academia, and industry have turned their attention to TSPEs~\cite{tstream,Affetti:2017:FIS:3093742.3093929,S-Store,Transactions2018}, which offer built-in support for shared mutable states.
\myc{The growing demand for real-time stream processing in data-intensive applications and the Internet of Things (IoT) has prompted the development of numerous Stream Processing Engines (SPEs)}, including Storm~\cite{storm}, Flink~\cite{flink}, and Spark-Streaming~\cite{spark}. 
\myc{
% However, supporting growing stream applications involving shared mutable states (concurrently read and modified by multiple execution entities) creates significant obstacles for mainstream SPEs, resulting in issues with correctness~\cite{S-Store-demo} and efficiency~\cite{tstream}.
However, as stream applications grow and involve shared mutable states that are concurrently read and modified by multiple execution entities,
mainstream SPEs face significant challenges related to correctness~\cite{S-Store-demo} and efficiency~\cite{tstream}.
} 
To address these concerns, academia, and industry have turned their attention to TSPEs~\cite{tstream,Affetti:2017:FIS:3093742.3093929,S-Store,Transactions2018}, which offer built-in support for shared mutable states.

TSPEs employ transactional semantics to manage accesses to shared mutable states during continuous data stream processing. 
\myc{
% The current state-of-the-art TSPEs, despite substantial efforts in the field, primarily rely on static task scheduling strategies. None of the existing TSPEs, however, can maximize performance under various and dynamically changing workload characteristics, leaving a significant design space for scaling TSPEs on multicore processors largely unexplored.
Despite substantial efforts in the field, the current state-of-the-art TSPEs primarily rely on static task scheduling strategies, none of the existing TSPEs can maximize performance under different and dynamically changing workload characteristics, leaving a significant design space for scaling TSPEs on multicore processors largely unexplored. 
Additionally, due to complex control and data dependencies in workloads, existing TSPEs failed to fully leverage multicore parallelism.
}

\myc{
In this paper, we introduce \system, a novel TSPE designed to fill this gap. 
\system utilizes a Task Precedence Graph (\tpg)~\cite{hou1994genetic,669967,Robert2011,10.1145/167088.167254} that identifies the fine-grained dependencies among state access operations of a batch of state transactions. 
Based on \tpg, \system can schedule and execute concurrent state transactions dynamically.
To ensure both efficiency and correctness, \system further employs a three-stage execution paradigm, namely Planning, Scheduling, and Execution.
\begin{myenumerate}
\item \textit{Planning}: Based on a parallel two-phase \tpg construction process, \system efficiently tracks dependencies among transactions that may arrive out-of-order and involve special scenarios, i.e., access to non-deterministic states and windowed states.
\item \textit{Scheduling}: \system decomposes a scheduling strategy into three dimensions.
It dynamically adjusts scheduling decisions for \tpg on each dimension according to the input workloads and system state.
\item \textit{Execution}: \system correctly executes state transactions following scheduling decisions.
It is achieved by relying on the Stateful Task Precedence Graph (\stpg) to manage the lifecycles of transactions execution and the multi-versioning state table management to maintain the consistency and correctness of table entries.
\end{myenumerate}
}

\myc{
A preliminary version of this work was recently published in SIGMOD 2023~\cite{mao2023morphstream}, where we introduced the adaptive scheduling of concurrent state transactions in \system. 
}
Due to space constraints, many critical transaction execution details of \system were omitted. 
In this paper, we expand on those details and demonstrate the full breadth of \system's capabilities.
The advantages of \system extend beyond adaptive scheduling. 
\myc{
% In this paper, we go through \system's scalability in processing state transactions based on the employment of the three-stage execution paradigm.
In this paper, we further discuss the scalability of \system on processing state transactions based on the three-stage execution paradigm. 
Additionally, we discuss how \system expertly integrates support for windowing operations and handles non-deterministic state access, amplifying its performance and adaptability even further. 
}
% Moreover, \system supports innovative applications such as online event detection~\cite{sahin2019streaming} and stock exchange analysis~\cite{sse}. 
% These features not only corroborate \system's versatility but also underscore its potential to address the evolving challenges of modern stream processing.

We experimentally demonstrate the capacity of \system to \myc{achieve} substantial improvements in throughput and latency for handling real-world use cases, compared to existing TSPEs. 
% Its unique strength lies not only in the adaptive scheduling strategy, which dynamically adjusts to changing workload characteristics for enhanced efficiency, but also in its broader capabilities for varying streaming scenarios that demand window aggregation and non-deterministic state access.
Furthermore, we show how to use \system to create two innovative applications, online social event detection~\cite{sahin2019streaming} and stock exchange analysis~\cite{sse}, which demonstrate \system's versatility and broad applicability. We open source the code, data, and scripts at \url{https://github.com/intellistream/MorphStream}.
% Moreover, we also demonstrate how to implement two innovative applications online social event detection~\cite{sahin2019streaming} and stock exchange analysis~\cite{sse} on \system to confirm the versatility and broad applicability of \system.
% This paper presents an in-depth discussion of \system's system architecture and design choices, providing valuable insights for future research and application development in transactional stream processing. 

\myc{
We have organized the rest of the paper as follows: 
Section~\ref{sec:background} provides a comprehensive background on transactional stream processing and delves into the design challenges inherent in optimizing a TSPE. 
Section~\ref{sec:Design_overview} presents a detailed overview of the design and execution workflow of \system, offering insights into its core functionalities.
Section~\ref{sec:planning} discusses how \system tracks dependencies in workloads and constructs \tpg. 
Section~\ref{sec:scheduling} discusses the adaptive scheduling strategies in \system.
Section~\ref{sec:execution} offers a detailed examination of transaction execution based on \stpg.
% showcases novel applications supported by \system. 
Section~\ref{sec:implement} showcases the programming model and APIs provided by \system, as well as the underlying system architecture that enables the three-stage execution paradigm.
Section~\ref{sec:exp} evaluates \system's performance using various benchmarks and real-world workloads. 
Section~\ref{sec:related} discusses related research to \system.
Finally, Section~\ref{sec:conclusion} concludes the paper and outlines future research directions.
}
% \section{Introduction}
% The growing demand for real-time stream processing in data-intensive applications and the Internet of Things (IoT) has led to the development of various Stream Processing Engines (SPEs), such as Storm~\cite{storm}, Flink~\cite{flink}, and Spark-Streaming~\cite{spark}. However, many emerging stream applications~\cite{Botan12,10.1007/978-3-030-19274-7_10,ACEP,sl} involve shared mutable states, where application states may be concurrently read and modified by multiple execution entities (e.g., threads) during stream processing. These applications are difficult to be supported correctly~\cite{S-Store-demo} and/or efficiently~\cite{tstream} by mainstream SPEs. In response, Transactional Stream Processing Engines (TSPEs) have been proposed to offer built-in support for shared mutable states, attracting attention from both academia~\cite{tstream,Affetti:2017:FIS:3093742.3093929,S-Store} and industry~\cite{Transactions2018} recently.

% TSPEs commonly adopt transactional semantics during the processing of continuous data streams, where accesses to shared mutable states are modelled as state transactions. Multiple state transactions can be executed in parallel, and state consistency is guaranteed by various concurrent execution approaches~\cite{ACEP,S-Store,tstream}. Despite significant efforts, existing works still rely on static task scheduling strategies, such as state partitioning~\cite{S-Store} and transaction restructuring~\cite{tstream}, during concurrent state transaction execution. As a result, there is a large unexplored design space for scaling TSPEs on multicore processors, where various scheduling strategies can lead to significantly different trade-offs and performance behaviours under different workload characteristics. In particular, none of the existing TSPEs fully exploits multicore parallelism due to complex control and data dependencies in the workload.

% \myc{
% In this paper, we propose \system, a novel TSPE designed to address these challenges by scheduling and executing concurrent state transactions based on a Task Precedence Graph (TPG)~\cite{hou1994genetic,669967,Robert2011,10.1145/167088.167254}.
% Subequently, \system executes concurrent state transactions in three stages to achieve efficiency and correctness: 
% }

% \myc{
% 1) Planning: \system tracks dependencies among transactions that may arrive out-of-order and access non-deterministic states.
% It introduces a two-phase  \tpg construction process and virtual operations correspondingly. 
% }

% \myc{
% 2) Scheduling: 
% \system makes adaptive scheduling decisions based on  \tpg.
% One key aspect of \system is its ability to dynamically adjust along three scheduling dimensions: exploration strategy, granularity, and abort handling. 
% By fine-tuning these dimensions in response to the input streams and system state, \system can better allocate resources and optimize performance.
% }

% \myc{
% 3) Execution: 
% \system correctly executes state transactions following the scheduling decisions.
% The  \tpg is augmented with a finite state machine to form a Stateful Task Precedence Graph ( \stpg). 
% The  \stpg allows \system to correctly execute and abort transactions.
% }


% \myc{\system's dynamic scheduling is not its only strength. It stands out by seamlessly incorporating support for windowing operations, essential for various stream processing tasks that necessitate data processing within specific time or count-based windows. This unique feature enables efficient processing of windowed state transactions, thereby extending \system's usability to a diverse range of applications.}

% \myc{Moreover, \system rises to the challenge of non-deterministic state access, a problem that emerges due to variables like random numbers, user-defined functions, and timers. This robust handling of non-deterministic elements enhances \system's performance and adaptability, solidifying its position as a versatile tool in transactional stream processing.}

% Experimental evaluation underscores \system's capacity to deliver substantial enhancements in throughput and latency when managing real-world use cases, outpacing existing TSPEs. \system's ability to adaptively select an optimal scheduling strategy in the face of changing workload characteristics further highlights its efficiency and adaptability.

% \myc{
% Beyond conventional use cases, \system's unique capabilities empower it to handle emerging applications such as online event detection~\cite{sahin2019streaming} and shared index window join~\cite{shahvarani2019parallel}, both of which are often difficult for existing data processing systems to support efficiently. Online event detection facilitates real-time recognition and processing of significant events in continuous data streams. On the other hand, shared index window join expedites the process of merging multiple data streams based on shared attributes within a sliding window. These novel applications underline \system's versatility and highlight its potential to tackle the evolving challenges of modern stream processing.
% }

% In summary, we provide an in-depth discussion of \system's system architecture, detailing its design choices and the interactions between its components. The detailed documentation of \system's system architecture, including the support of stateful  \tpg, offers valuable insights for researchers and practitioners alike. By understanding the design principles, mechanisms, and trade-offs involved in \system, the community can build upon this work to develop new techniques, optimizations, and applications, pushing the boundaries of what is possible in the realm of transactional stream processing.

% The rest of the paper is organized as follows. Section 2 provides background information and reviews related work on TSPEs and scheduling strategies. Section 3 presents the architecture and design of \system, including the StreamManager, TxnManager, and TxnScheduler components. Section 4 discusses the adaptive scheduling strategies employed by \system and how they are integrated into the system. Section 5 showcases the novel applications supported by \system, such as online event detection and shared index window join. Section 6 evaluates the performance of \system through various benchmarks and real-world workloads. Finally, Section 7 concludes the paper and outlines future directions for research.