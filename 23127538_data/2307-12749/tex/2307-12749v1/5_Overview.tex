\section{Design of MorphStream}
\label{sec:Design_overview}
In this section, we delineate the design principles underpinning \system and its execution workflow.

\subsection{Design Principles}
The central principle of \system, designed to address the challenges outlined earlier, involves mapping the state transaction scheduling problem onto a \tpg scheduling problem. \system constructs a \tpg such that each vertex corresponds one-to-one with an operation. The dependencies among operations, as defined in Section~\ref{def:TD}, are then directly mapped onto edges connecting these vertices. Thus, processing all operations of a list of state transactions $L$, while respecting these dependencies, ensures a correct schedule. This concept is embodied in a \tpg, where, for example, operations $O_1$ $\sim$ $O_5$ in Figure~\ref{fig:dependencies_in_SL} and their dependencies naturally form a \tpg. Based on this principle, \system handles dependency identification and state transaction scheduling challenges via the following designs:

\paragraph{D1: Batched and Sorted Transaction Processing:} \system employs a two-phase \tpg construction process (Section~\ref{sec:planning}). The initial phase sorts the batched, out-of-order state transactions, while the second phase identifies dependencies based on the sorted transactions.

\paragraph{D2: Virtual Operation Implementation:} \system introduces \emph{virtual operation} (Section~\ref{sec:planning}) in the \tpg to handle non-deterministic state access. It tracks potential dependencies and helps anticipate state access needs. Furthermore, \system records states accessed by non-deterministic transactions after execution (Section~\ref{sec:execution}) to ensure accurate rollback when necessary.

\paragraph{D3: Generalized Structure for Window Operations:} \system addresses complexities introduced by window operations by generalizing their structure. By comparing overlapping windows (Section~\ref{sec:planning}), \system identifies dependencies among these operations. Furthermore, it retrieves records from the multi-versioning table within the window boundaries (Section~\ref{sec:execution}), ensuring accurate data processing within each window.

\subsection{Three-stage Execution Paradigm}
\label{subsec:overview}

% Figure environment removed

As depicted in Figure~\ref{fig:workflow}, \myc{\system executes concurrent state transactions} in three stages, each contributing to its efficient and accurate operation:

% \myc{TODO: We did not introduce two phases as shown in the Figure..}
% \myc{(i.e., Stream Processing Phase and Transaction Processing Phase)}

\textbf{\circled{1}} \textit{Planning:} \system constructs the \tpg by identifying fine-grained temporal, logical, and parametric dependencies within and among a batch of state transactions. This involves implementing the design strategies outlined above, including the two-phase \tpg construction process, the use of virtual operations, and the application of specific dependency tracking rules for window operations.

\textbf{\circled{2}} \textit{Scheduling:} operations are scheduled for execution based on the \tpg (Section~\ref{sec:scheduling}). A decision model (Section~\ref{subsec:model}) guides \system in making optimal scheduling decisions, considering varying workload characteristics.

\textbf{\circled{3}} \textit{Execution:} Threads execute operations concurrently based on the scheduling decisions, while ensuring state access correctness. \system employs a finite state machine for each operation to accurately capture its state access behavior during execution and aborting. It relies on the multi-versioning state table management to \myc{maintain the correctness and consistency of table entries}. \system supports window-based state access by querying a range of targeting state copies from the multi-versioning state table. Furthermore, it records state accesses for non-deterministic state transactions after execution to ensure accurate rollback if necessary.


% \system identifies the fine-grained temporal, logical, and parametric dependencies within and among a batch of state transactions to construct the corresponding \tpg in parallel. 
% \system applies the aforementioned designs to achieve dependency tracking and \tpg construction:
% 1) \system executes the two-step \tpg construction in the stream processing phase and the transaction processing phase correspondingly.
% % that ensures dependencies among out-of-order state transactions are accurately captured.
% 2) \system identifies dependencies among non-deterministic state transactions through inserting ``virtual operations'' during the two-phase \tpg construction.
% 3) \system applies the specific dependencies tracking rules to identify dependencies among window operations.

% Upon finishing transaction execution, final processing results as output stream are generated during the postprocess of input events based on the obtained shared mutable state access results.


