\usepackage{amsmath,amsfonts} %amssymb
\usepackage{algorithmic}
\usepackage{graphicx}
\usepackage{textcomp}
\usepackage[dvipsnames]{xcolor}
\usepackage{soul} % to allow \ul
\usepackage{amsfonts}
\usepackage{hyperref}
\hypersetup{colorlinks=false,%
            linkbordercolor=white,linkcolor=green,pdfborderstyle={/S/U/W 1}}

%%% CODE %%%
\usepackage[linesnumbered,ruled,vlined]{algorithm2e}
% \usepackage[dvipsnames]{xcolor}
% Definindo novas cores
 
% Configurando layout para mostrar codigos Java
\usepackage{listings}
\newcommand{\estiloJava}{
\lstset{
    language=Java,
    basicstyle=\ttfamily\scriptsize,
    keywordstyle=\color{jpurple}\bfseries,
    stringstyle=\color{red},
    commentstyle=\color{verde},
    morecomment=[s][\color{blue}]{/**}{*/},
    extendedchars=true,
    showspaces=false,
    showstringspaces=false,
    numbers=left,
    numberstyle=\tiny,
    breaklines=true,
    backgroundcolor=\color{cyan!10},
    breakautoindent=true,
    captionpos=b,
    xleftmargin=0pt,
    tabsize=2,
}}
\lstset{
    showstringspaces=false,
    basicstyle=\ttfamily,
    keywordstyle=\color{blue},
    commentstyle=\color[blue]{0.9},
    stringstyle=\color[RGB]{255,150,75},
    morekeywords={Event, EventBlotter}
}
 
\newcommand{\inlinecode}[2]{\colorbox{almond}{\lstinline[language=#1]$#2$}}

\usepackage{tikz}
\newcommand*\circled[1]{\tikz[baseline=(char.base)]{
            \node[shape=circle,fill,inner sep=1pt] (char) {\textcolor{white}{#1}};}}
\usepackage{makecell}

\renewcommand\theadalign{bc}
\renewcommand\theadfont{\bfseries}
\renewcommand\theadgape{\Gape[4pt]}
\renewcommand\cellgape{\Gape[4pt]}

\newcommand{\tony}[1]{{\textcolor{red}{#1}}}
\newcommand{\myc}[1]{{\textcolor{black}{#1}}}
\newcommand{\curry}[1]{{\textcolor{black}{#1}}}
\newcommand{\change}[1]{{\textcolor{black}{#1}}}
\newcommand{\system}{\textsl{MorphStream}\xspace}
\newcommand{\stpg}{S\text{-}TPG\xspace}
\newcommand{\tpg}{TPG\xspace}

\newcommand{\blk}{$BLK$\xspace}
\newcommand{\rdy}{$RDY$\xspace}
\newcommand{\exe}{$EXE$\xspace}
\newcommand{\abt}{$ABT$\xspace}

\newcommand{\td}{\texttt{TD}\xspace}
\newcommand{\pd}{\texttt{PD}\xspace}
\newcommand{\ld}{\texttt{LD}\xspace}

\newcommand{\tds}{\texttt{TD}s\xspace}
\newcommand{\pds}{\texttt{PD}s\xspace}
\newcommand{\lds}{\texttt{LD}s\xspace}

\newcommand{\tstream}{TStream\xspace}
\newcommand{\sstore}{S-Store\xspace}

\newcommand{\se}{\hyperref[symbol:se]{\underline{\textsf{s-explore}}}\xspace}
\newcommand{\nse}{\hyperref[symbol:nse]{\underline{\textsf{ns-explore}}}\xspace}
\newcommand{\fsu}{\hyperref[symbol:fsu]{\underline{\textsf{f-schedule}}}\xspace}
\newcommand{\csu}{\hyperref[symbol:csu]{\underline{\textsf{c-schedule}}}\xspace}
\newcommand{\ea}{\hyperref[symbol:ea]{\underline{\textsf{e-abort}}}\xspace}
\newcommand{\la}{\hyperref[symbol:la]{\underline{\textsf{l-abort}}}\xspace}

\usepackage{enumitem}
\newenvironment{myitemize}
{ \begin{itemize}[leftmargin=0.2in]	
		\vspace{-1ex}	
		\setlength{\itemsep}{0pt}
		\setlength{\parskip}{0pt}
		\setlength{\parsep}{0pt}    }
	{ 	 \end{itemize}                    }

\newenvironment{myenumerate}
{ \begin{enumerate}[leftmargin=0.2in]
		\vspace{-1ex}
		\setlength{\itemsep}{0pt}
		\setlength{\parskip}{0pt}
		\setlength{\parsep}{0pt}    }
	{ \end{enumerate}                  }
	
\usepackage{multirow}
\usepackage{subfig}

\usepackage{comment}

\usepackage{hyphenat}
\hyphenpenalty=10000
\tolerance=1000
\sloppy


\usepackage{url}
\def\UrlBreaks{\do\/\do-}

\newcommand{\compact}{\vspace{-5pt}}
\newcommand{\subcompact}{\vspace{-4pt}}

\usepackage[skip=3pt]{caption}
\setlength{\belowcaptionskip}{-4pt}

\usepackage{tcolorbox}
\usepackage{marginnote}
% \setlength\marginparsep{20pt}
\newcommand{\margi}[1]{
\marginnote{
% \vspace{+10pt}
% \begin{tcolorbox}[colframe=white,colback=white,boxrule=0pt,arc=0.0em,boxsep=-1mm,top=4pt,left=0pt,right=0pt]    
%    \textcolor{blue}{#1}
% \end{tcolorbox} 
}
}

\newcommand{\margii}[2]{
\marginnote{
% \vspace{#1}
% \begin{tcolorbox}[colframe=white,colback=white,boxrule=0pt,arc=0.0em,boxsep=-1mm,top=4pt,left=0pt,right=0pt]    
%    \textcolor{blue}{#2}
% \end{tcolorbox} 
}
}
\usepackage[misc,geometry]{ifsym}

\newcommand{\executionAlgo}{
\begin{algorithm}
\footnotesize
    \KwData{$e$ \tcp{Input event}}
    \KwData{$txn_{ts}$ \tcp{State transaction}}
    \KwData{$G$ \tcp{The currently constructed TPG}}
    \While{!finish processing of input streams}{
        \eIf(\tcp*[h]{Phase 1}){\text{$e$ is not a $punctuation$}}{
                $txn_{ts}$ $\gets$ PRE\_Processing($e$)\;
                \textbf{TPG\_Construction}($G$, $txn_{ts}$)\; 
          }(\tcp*[h]{Phase 2}){
                \textbf{TPG\_Refinement}($G$)\; 
                \textbf{TXN\_Scheduling}($G$)\; 
                POST\_Processing()\;
          }
    }
    
    \SetKwFunction{FMain}{TPG\_Construction}
    \SetKwProg{Fn}{Function}{:}{}
    \Fn{\FMain{$G$, $txn_{ts}$}}{
        $O_{1..k}$ $\gets$ \textbf{Partition} $txn_{ts}$\;
        \ForEach{\text{operation $O_{i}$ $\in$ $O_{1..k}$}}{
            \textbf{Identify} its \ld\;
            $G$ $\gets$ $G$ + $O_{i}$ \;
        }
    }
    \SetKwFunction{FMain}{TPG\_Refinement}
    \SetKwProg{Fn}{Function}{:}{}
    \Fn{\FMain{$G$}}{
        \ForEach{\text{vertex $e_{i}$ $\in$ $G$}}{
            \textbf{Identify} its \td, \pd\;
        }
    }
    
    \SetKwFunction{FMain}{TXN\_Scheduling}
    \SetKwProg{Fn}{Function}{:}{}
    \Fn{\FMain{$G$}}{
        $M$ $\gets$ Instantiated with $G$;\tcp{A decision model}
        \While{!finish scheduling of $G$
        }{
          \textbf{\circled{2}} $Scheduling Unit$ $\gets$ \textbf{\circled{1}} \emph{Explore}($G$, $M$)\; 
            \textbf{\circled{3}} \emph{Execute with Abort Handling} ($Scheduling Unit$)\; 
        }
    }
  \caption{\change{Execution Outline of \system}}
  \label{alg:algo}
\end{algorithm}
}
