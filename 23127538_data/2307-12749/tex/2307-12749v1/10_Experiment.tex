\section{Evaluation}
\label{sec:exp}
In this section, we conduct a comprehensive evaluation of \system, comparing it to alternative approaches. Note that these experiments are conducted under the assumption of no system failures during runtime. Providing efficient fault tolerance without compromising low latency and high throughput during normal operation poses a significant challenge for TSPEs, even in a single-node setting. This complexity arises from the intricate interplay of transactional and stream-oriented properties that TSPEs exhibit. The challenge could potentially magnify in a distributed environment. Enhancements for \system's fault tolerance are a subject of our separate, forthcoming work.
% In this section, we conduct a detailed experimental evaluation comparing \system to the alternative approaches. Note that, we assume there is no system failure at runtime. Guaranteeing fault tolerance without sacrificing low latency and high throughput during normal operation is still an open challenge for TSPEs~\cite{S-Store,tstream,AFFETTI202065}.
% This is a challenging problem even in a single node setting because of the non-trivial combination of both transaction and stream-oriented properties in TSPEs. We leave the fault tolerance enhancement of \system in a separate work.

In summary, we have made the following key observations.
\begin{itemize}
    \item Our experimental results show that \system outperforms conventional SPEs for TSP applications (Section~\ref{subsubsec:conventional}) by orders of magnitude. Because of the adaptive scheduling strategy, \system achieves up to 2.2x higher throughput and 69.1\% lower latency compared to the state-of-the-art TSPEs (Section~\ref{subsubsec:dynamic} and~\ref{subsubsec:mutiple}).
    \myc{Moreover, we also evaluate the performance of executing window non-deterministic queries on \system in Section~\ref{subsubsec:window_exp} and~\ref{subsubsec:nondeterministic_exp}.}
    \item In Section~\ref{subsec:overhead}, we show that \system spends more time on \tpg construction and exploration, but largely reduces the overhead of synchronization. A drawback of \system, however, is its high memory consumption (about 1.4x times higher) due to the more complex auxiliary data structures.
    \item We show that no one scheduling strategy can outperform others in all cases (Section~\ref{subsec:scheduling_decision}). Each scheduling decision has its own advantages and disadvantages under varying workload characteristics.
    \item In Section~\ref{subsec:modern_hardware}, we show that \system spends up to 2.3x fewer clock ticks and has a lower memory bound than \tstream and \sstore. Furthermore, \system has much better multicore scalability compared to prior schemes. 
    \item \myc{In Section~\ref{subsec:use_cases}, we demonstrate the practical usefulness of \system with two case studies: \textit{Online Social Event Detection} and \textit{Stock Exchange Analysis}. Specifically, \system can react to emerging events and output expected results in sub-second level when processing real-world workloads.}
\end{itemize}
 
\subcompact
\subsection{Evaluation Methodology}
We conduct all experiments on a dual-socket Intel Xeon Gold 6248R server with 384 GB DRAM. 
Each socket contains $24$ cores of 3.00GHz and 35.75MB of L3 cache. To isolate the impact of NUMA, we use one socket of the server in our experiments. We leave it as a future work to address the NUMA effect~\cite{briskstream}. We pin each thread on one core and assign 1 to 24 cores to evaluate the system scalability.
The OS kernel is \emph{Linux 4.15.0-118-generic}. 
We use \emph{JDK 1.8.0\_301}, set \emph{-Xmx} and \emph{-Xms} to be 300 GB. We use G1GC as the garbage collector across all the experiments and configure \system to not clear temporal objects such as the processed TPGs and multi-versions of states. 
We show the impact of clean-up and JVM GC in Section~\ref{subsec:memory-footprint}.
%\tony{The xmx and xms?}
%Based on GC logs, we find that no major GC occurs during the execution and minor GC contributes only xxx $\sim$ xxx\% to the total execution time across all the applications.
% The number of cores assigned to the system and the size of punctuation are system parameters that can be varied by users. 
% We vary both parameters in our experiments. 
\begin{comment}
Repeat with Number of Transactions ($T$)
\end{comment}
%We use the punctuation interval of 10240 as a default execution configuration, and vary it from 5120 to 81920 in our sensitivity studies. \tony{Discussion of the default settings of all other system parameters..}

% \subcompact
% \subsubsection{Benchmark Workloads}
We use three use cases: Streaming Ledger (SL), GrepSum (GS), Toll Processing (TP) from a benchmark proposed by our previous work~\cite{tstream} on evaluating the effectiveness of \system. For all these workloads, we follow the original application logic but tweak the configurations to bring more workload dependencies such that we can better expose the issues of existing TSPEs. In particular, 
% compared to the original~\cite{tstream}, 
we have configured ten times larger sizes of shared mutable states and generated more state transactions accessing overlapped states in our workload settings.
Additionally, we further present two additional use cases that process the real-world datasets: Online Social Event Detection~\cite{sahin2019streaming,olteanu2014crisislex} (OSED) and Stock Exchange Analysis~\cite{sse} (SEA) to illustrate the usefulness of \system in supporting complex real-world data processing scenarios.
%In particular, we have configured 10 times larger sizes of shared mutable states, and generate more state transactions accessing overlapped states in our workload settings compared to the original.


\textbf{\change{Tuning Workload Characteristics.}}
% The benchmark contains .
% We adopt the same implementation of SL in our system, while we make some modification to GS to make it more realistic. 
To better comprehend the system behaviour, we tune the following six workload characteristics. The default workload characteristics and varying ranges are summarized in Table~\ref{tab:default}. 
\emph{1). State Access Distribution ($\theta$):}
% State access distribution are varying for different workloads, where certain states may be more likely to be accessed than others.
Similar to~\cite{tstream}, we modelled the state access distribution as Zipfian skew, and tune the Zipfian factor to vary $\theta$.
\emph{2). Ratio of Aborting Transactions ($a$):} 
% The proportion of the number of transactions to be aborted. 
We tune $a$ by artificially adding transactions that violate the consistency property, such as the account balance can not become negative.
\emph{3). Transaction Length ($l$):}
We tune $l$ by varying the number of atomic state access operations in one transaction.
% Transaction length represents the number of atomic state access operations that must be decomposed for each transaction.
\emph{4). The complexity of a UDF ($C$):}
% \curry{Due to the \pd} (Definition~\ref{def:PD}), 
% Each operation may carry a user-defined function (i.e., the $f$ in Definition~\ref{def:PD}) with various computational complexity. 
We tune $C$ by adding a random delay in each user-defined function (i.e., the $f$ in Definition~\ref{def:PD}).
\emph{5). Number of State Access Per Operation ($r$):}
% an operation may need to read multiple states for resolving \pd. 
We vary the ratio of multiple state access operations to tune $r$.
\emph{6). Number of Transactions ($T$):} 
We tune $T$ by varying the punctuation interval.
% The number of transactions arriving during the punctuation interval.
%The number of state transactions to be processed together in a batch interval. 
%\tony{Double check, there is no mention about punctuation interval.}

% arrived during the punctuation interval.

% It is noteworthy that we have configured 10 times larger sizes of shared mutable states, and generate more state transactions accessing overlapped states in our workload settings compared to the original setting used by Zhang et al.~\cite{tstream}. Both changes are intended to bring more workload dependencies to better expose the issues of existing TSPEs such as TStream.
\textbf{Dynamic Workload Configurations.}
To evaluate the adaptability of \system, we follow the mechanism proposed by Ding et al.~\cite{dynamicWorkload} to generate dynamic workloads.
% We briefly describe how to generate dynamic workload through the following steps: 
% First, to further support diverse workloads, we cover a series of configurations, such as cyclic dependency, ratio of aborting transactions, computation complexity, number of state access per transaction, ratio of different type transactions, etc. 
% Second, 
% Due to a large number of tunable workload configurations and their wide range of adjustments, it is difficult to adjust all parameters in one dynamic workload. 
% Therefore, 
Specifically, we generate various phases of dynamic workloads by different trends, which determines the parameters we want to tune and how they change over time. For example, in a dynamic workload with an increasing tendency to abort transactions, we will increase the ratio of aborting transactions over time. 
% Finally, we can control the rate of workload transfer by setting the number of events per configuration.

%To comprehensively evaluate \system, we will also tweak both workloads with varying workload characteristics. 

%  we .
% to make it more realistic.
% The \emph{Sum} transactions
% are mainly processed by reading on selected states and performing summation among the obtained state values.
% Different from Zhang et al.~\cite{tstream}, 
% We modify transactions to apply multiple summations on different subsets of values to control the transaction length ($l$).
% In addition, we let each \emph{Sum} transaction write the summation result back to a specified state. For simplicity, we let it write to the first state it reads.
% Hence, the number of state access per operation ($r$) can be controlled by tuning the subset of states for summation.
% By default, we set $l=1$ and $r=10$ in GS.

\begin{table}[]
\centering
\caption{Workload default configuration}
\label{tab:default}
\resizebox{0.48\textwidth}{!}{
\begin{tabular}{|p{1.5cm}|p{1.8cm}|p{1.8cm}|p{1.8cm}|p{1.8cm}|p{2cm}|}
\hline
\textbf{Workload Char.} & \textbf{SL} & \textbf{GS} & \textbf{TP}  & \textbf{Tweaking ranges} \\ \hline
% $S$ & 2(tables) $\times$ 100k(records) & 1(table) $\times$ 100k(records) & - \\ \hline
$\theta$ & 0.20 & 0.20 & 0.20 & 0.0$\sim$1.0 \\ \hline
$a$ & 1\% & 1\% & 1\%  & 0$\sim$90\% \\ \hline
% $t$ & writeonly 25\% & writeonly 0\% & writeonly (0\%$\sim$100\%) \\ \hline
$l$ & 2 / 4 & 1 & 2  & 1$\sim$10 \\ \hline
$C$ & 10 us & 10 us & 10 us & 0$\sim$100 us \\ \hline
$r$ & 1 / 2  & 2 & 1  & 1$\sim$10 \\ \hline
$T$ & 10240  & 10240 & 40960  & 5120$\sim$81920 \\ \hline
\end{tabular}
}
\end{table} 

% \textbf{Experimental Setup.}


\subcompact
\subsection{Performance Evaluation}
\label{subsec:evaluation}
In this section, we conduct a series of experiments to confirm \system's superiority compared to the state-of-the-art.

\subsubsection{Comparing to Conventional SPEs}
\label{subsubsec:conventional}
In the first experiment, we show that TSPEs significantly outperform conventional SPEs when handling TSP applications. We use the default workload configuration shown in Table~\ref{tab:default} in this study. 
We have implemented SL on Flink-1.10.0.
Since the native Flink does not support shared mutable state accesses, we leveraged Redis-6.2.6 with a distributed lock, a common workaround, to store shared mutable states.
We deploy a standalone cluster with a single TaskManager configured with 24 slots and set the parallelism of SL to 24.
To avoid the OOM exception, we set the TaskManager heap size to 100GB.
When locking is disabled (denoted as w/o Locks), execution correctness is not guaranteed in Flink.
The detailed workload configuration is shown in Table~\ref{tab:default}.
As shown in Figure~\ref{fig:overview_comparison}, \system significantly outperforms the two state-of-the-art TSPEs, \tstream (1.6x) and \sstore (3.7x), and Flink (up to 117x).
It is noteworthy that Flink, a popular conventional SPE, achieves orders of magnitude lower throughput in this application. By disabling locks, its throughput increases but is still far lower than any of the TSPEs. In the following, we hence do not further compare \system with Flink.

% Figure environment removed

\subsubsection{Evaluation on Dynamic Workloads}
\label{subsubsec:dynamic}
In this experiment, we show that \system can always select a better-performing scheduling strategy under changing workloads, resulting in lower latency and higher throughput compared to state-of-the-art TSPEs. We use SL as the base application and divide the workloads into four phases. 
Figure \ref{fig:SL_Throughput} and Figure \ref{fig:SL_Latency} compare the throughput and latency of \system against two state-of-the-art TSPEs: \sstore~\cite{S-Store} and \tstream~\cite{tstream}. We mark each phase in the dynamic workload using the dotted grey box.
In the first phase, a large number of events consisting of \emph{Deposit} transactions arrive, and the state accesses distribution is scattered. 
As a result, there are lots of \lds and \tds but few \pds. At the same time, the vertex degree distribution is uniform as the state accesses are scattered. 
As guided by our decision model (Figure~\ref{fig:model}), 
\system selects the \se strategy to resolve a large number of dependencies and selects \csu
for scheduling since there are fewer \pds. 
\system achieves up to 1.27 times higher throughput compared to the second-best. 
In the second phase, we configure the workload with increasing key skewness over time. Hence,
dependencies are gradually contented among a small set of states, which facilitates the resolution of dependencies.
As expected, the performance of all approaches drops. 
\system gradually morphs from \se to \nse strategy to resolve dependencies in a more flexible manner, and constantly outperforms \sstore. 
In the third phase, we configure the workload with an increasing ratio of \emph{Transfer} transactions so that one of the two types of transactions in SL is called intensively in a short period of time. 
As the proportion of \emph{Transfer} transactions increases, there are more and more dependencies between scheduling units. Hence, \system gradually morphs from \csu to \fsu to reduce the dependency resolution overhead and result in a stable throughput.

There is no transaction abort in the first three phases, and the selection of aborting mechanism in \system does not matter. In the fourth phase, we increase the ratio of aborting transactions over time to evaluate the performance of the system under a dynamically changing ratio of aborting transactions. 
In the beginning, \system applies the \ea mechanism to eagerly abort when the operation fails and morphs to \la when aborts are frequent so that transaction aborts can be handled together to reduce context switching overhead.
The results show that \tstream's performance drops when transaction aborts appear. 
This is because of the rapidly increasing overhead of redoing the entire batch of transactions. 
In contrast, \system achieves relatively stable performance and is 2.2x to 3.4x higher than other schemes.
 
A further key takeaway from Figure \ref{fig:SL_Latency} is that \tstream and \sstore have significantly higher tail latency than \system. 
This is mainly because the scheduling strategies in \tstream and \sstore are limited for specific workload characteristics. When workload changes, such as increasing transaction aborts or key skewness, their efficiency drops significantly, resulting in higher processing latency. In contrast, \system dynamically morphs the scheduling strategy according to the change of workload characteristics to deal with different situations, thus achieving a constantly lower processing latency.

% Figure environment removed

% Figure environment removed
\subcompact
% \subsection{Performance Evaluation on TP}
% \label{subsec:TP}
%We divide the workloads of OB into the following phases.
%In the first phase shown in the Table ~\ref{tab:default}, at the beginning, users will bid on a large number of different items, which leads to a larger number of TD and a uniform vertex degree distribution. 
%In the second phase that follows, we assume that some very popular items appear, so a great quantity of users will bid on the same item. Therefore, we reflect this phenomenon by increasing the skewness of state access distribution in the configuration.
%In the third phase, as the auction progresses, invalid transactions such as the bid price is below the asking price or the quantities of the item are insufficient. So we configure the workload with an increasing ratio of abort transactions.
% \textbf{\system is more flexible to handle complex workload.}
% \noindent
\subsubsection{Evaluation of Multiple Scheduling Strategies}
\label{subsubsec:mutiple}
In TP, the road conditions in different regions can have varying characteristics, which can be divided into multiple groups. \tstream \cite{tstream} and \sstore \cite{S-Store}'s scheduling strategies may work well on one group of state transactions but not on another.
In contrast, \system's modular and flexible design allows it to employ multiple scheduling strategies concurrently.
For illustration, we configure the TP to contain two groups of state transactions simultaneously. 
In \emph{group 1}, the state access distribution of state transactions is skewed, and the ratio of aborting transactions is high. 
In \emph{group 2}, the state distribution of state transactions is uniform and transaction aborts occur rarely. 
\margii{-1pt}{R4O14}
\change{As guided by our decision model (Figure~\ref{fig:model})}, 
\system applies \nse, \csu, and \la for handling transactions of \emph{group 1}, and applies \se, \csu, and \ea for handling transactions of \emph{group 2}. We name such a combination of strategies a \emph{nested} configuration.

Figure~\ref{fig:TP_Throughput} shows the throughput comparison results. We can see that the throughput of the nested configuration is 40.9\% higher than \tstream and 117\% higher than \sstore. 
To further comprehend the advantage of the nested configuration, we compare it against two plain scheduling strategies: \nse, \csu, and \la (denoted as \emph{plain-1}) and \se, \csu, and \ea (denoted as \emph{plain-2}) for handling all transactions from both groups.
% which are either suitable for \emph{group 1} or \emph{group 2}, respectively. 
Unsurprisingly, the throughput of the nested configuration is 1.17$\times$ and 2.87$\times$ higher than that of each plain scheduling strategy. When the ratio of aborting transactions in \emph{group 1} is high, the \emph{plain-2} is bottlenecked by the frequent context switching overheads. 
%\tony{As the skewness of state access increases, the workloads become less balanced among threads, also hampering performance. -- what do you want to say here?} 
At the same time, as the skewness of state access increases in \emph{group 1}, the workloads become less balanced among threads, hampering the system performance when using \se in \emph{plain-2}.
As the state distribution of state transactions is uniform and the ratio of aborting transactions is low in \emph{group 2}, the \emph{plain-1} spends more time resolving dependencies and redoing the entire batch of transactions. 
The \emph{plain-1} performs better than \emph{plain-2} as there are fewer \pds and the computation complexity is low, but it is still lower than that of the \emph{nested} setting. 
% However, there are less \pd in Toll Processing and the computation complexity is low, so the drop in performance of \emph{plain-1} is not as obvious as that of \emph{plain-2}, but it is still lower than that of \emph{nested} setting. 
% \system's flexibility enables a "hybrid" type scheduling strategy for multiple state transaction groups, which get the best of the all scheduling strategies.

Figure~\ref{fig:TP_Latency} shows the comparison results of end-to-end processing latency. Thanks to the significantly improved performance, \system with nested configuration achieves very low processing latency. 
%-- explain why S-Store's latency keeps linearly increase?? and also the purple line. ---
Note that, \sstore spends more time on synchronization and inserting locks under a highly contended workload in \emph{group1} because dependent transactions are executed serially. Under a higher ratio of aborting transactions in \emph{group 1}, \emph{plain-2} spends lots of time achieving fine-grained state rollback because of the high context-switching and synchronization overhead of \se (Table~\ref{tab:decisions}), which is why \emph{plain-2} results in highest latency compared to others. 
% Figure environment removed

\subcompact
\subsubsection{Evaluation of Window-based Queries}
\label{subsubsec:window_exp}

\myc{
We implement an additional application \textit{GrepSum with window reads} as an example to illustrate how MorphStream supports windowing queries. 
Specifically, we modified the GrepSum to perform random state updates and periodical window readings. 
In particular, the application processes two types of state transactions: 
(1) it executes transactions with write-only operations on receiving input events with updating requests, 
and (2) it executes transactions with window-read and sum aggregation operations on receiving input events with
reading requests.
}

For this experiment, we retained the settings of GrepSum outlined in Table~\ref{tab:default}, with an abort ratio of 0 and a punctuation interval of 102400. The process involved periodic state access (one event with reading requests for every 100 input events), where 100 random states were accessed within a default window size of 1000 (which required reading states up to 1000 event-time old) for the GrepSum operation. We modified the window trigger periods and window sizes to simulate a variety of window query scenarios, and evaluated their impact on performance. The overall results are presented in Figure~\ref{figures:window_exp}.

We first adjust window sizes from 1k to 100k, with the performance outcomes depicted in Figure~\ref{fig:window_size_exp}. As anticipated, increasing the window size led to an escalation in state access overhead due to the need to read more state versions. This inflated overhead, in turn, decreased system throughput by up to 30\%.
We then vary the window trigger period from 100 to 10k events. The performance results, shown in Figure~\ref{fig:window_period_exp}, demonstrated that frequent window queries significantly impeded throughput, slowing it down by as much as 60\%.
These experiments underscore \system's optimization potential in window-based transactional stream processing. Opportunities lie in reducing redundant calculations in overlapping window operations. Exploring these improvements is part of our future work.

\subsubsection{Evaluation of Non-Deterministic Queries}
\label{subsubsec:nondeterministic_exp}

% Figure environment removed
% % Figure environment removed
\myc{
We implemented the \textit{GrepSum with non-deterministic queries} to demonstrate \system's support for non-deterministic state access. It involves processing state transactions that read specific states and compute summation results, which are then written back to a designated state. However, the state to be accessed can be deterministic or non-deterministic. Deterministic state access is based solely on the input event, while non-deterministic state access depends on additional factors such as user-defined functions, timers, or random values~\cite{Clonos}. 
}

\myc{
In this experiment, we focused on tuning the number of state transactions involving non-deterministic state access to investigate its impact on the performance of \system. The results, shown in Figure~\ref{fig:non_exp}, led to two key observations. First, the number of non-deterministic state accesses had no significant effect on the performance of \sstore. This is because \sstore executes dependent operations sequentially, resulting in minimal overhead for handling non-deterministic state access. Second, both \system and \tstream experienced notable performance degradation as the number of non-deterministic state accesses increased. This can be attributed to the higher \tpg planning overhead associated with a large number of non-deterministic state accesses, requiring the addition of virtual operations and tracking dependencies across all operation chains. The results of this experiment indicate a significant optimization space within \system to enable efficient tracking of dependencies for non-deterministic state access operations. For instance, one potential approach could involve predicting the accessed state and conducting pilot runs of non-deterministic state access operations. However, developing an efficient and low-overhead training model for this purpose poses challenges and remains a topic for future work.
% GS + (UDF + Random Number + Timers).
}

% For example, some roads in one area are congested, so the state access is more skewed. While in another area, the traffic condition is good and the state access distribution is less skewed.

% for multiple state transaction groups. 
% can still use one scheduling strategy to handle multiple groups of state transactions with different characteristics. In doing so, however, the scheduling strategy that works well on one group of state transactions may not work well on another group.

% For demonstration, 
% The two-layer setting is the scheduling strategy that uses \emph{OG\_NS\_Lazy} and \emph{OG\_DFS\_Eager} scheduling strategies for two groups of state transactions, respectively. 

% \curry{\system's modular and flexible design makes it possible to handle cases not find in the \tstream \cite{tstream} and S-Store \cite{S-Store} by employing different scheduling strategies for multiple state transaction groups. To demonstrate this advantage, we use the Toll Processing ~\cite{TP} as the base application and outline how \system can be improved to address this situation.   }

% \curry{ 
% Nonetheless, \tstream \cite{tstream} and S-Store \cite{S-Store} can still use one scheduling strategy to handle multiple groups of state transactions with different characteristics. In doing so, however, the scheduling strategy that works well on one group of state transactions may not work well on another group.
% For example, when the state access distributions of two state transaction groups are skewed in one and uniform in the other, neither \emph{structured exploration strategy} nor \emph{non-structured exploration strategy} can resolve dependencies well. To alleviate this concern, we leverage \system's modular and flexible design to create not one but multiple scheduling strategies, one for each group. This approach efficiently handles the different workload characteristics in multiple state transactions group.}

% \curry{To evaluate this situation, we configures the Toll Processing to contain two groups of state transactions. In \emph{group1}, the state access distribution of state transactions is skewed and the ratio of aborting transaction is high. On the contrary, in \emph{group2}, the state distribution of state transactions is uniform and transaction aborts occur rarely. }

% \curry{The result are shown in the Figure ~\ref{fig:TP_layer}, the throughput of the two-layer setting is 40.9\% better than \tstream and 117 ~\% better than S-Store. 

% To further outline the advantage of the two-layer approach, we compare the two-layer setting against two single-layer scheduling strategy: \emph{MorphStream(OG\_NS\_Lazy)} and \emph{MorphStream(OG\_DFS\_Eager)}, which are suitable for group1 and group2, respectively. Unsurprisingly, the throughput of the two-layer setting is 1.17$\times$ and 2.87$\times$ higher than that of each single-layer scheduling strategy. When the ratio of aborting transaction in \emph{group1} is high, the \emph{MorphStream(OG\_DFS\_Eager)} scheduling strategy is bottlenecked by the frequent context switching overheads. At the same time, as the skewness of state access increases, the workloads is not balanced among threads, also hampering performance. As the state distribution of state transactions is uniform and the ratio of aborting transactions is low in \emph{group2}, the \emph{MorphStream(OG\_NS\_Lazy)} scheduling strategy spend more time to resolve dependencies and redo the entire batch of transactions. However, there are less \pd in Toll Processing and the computation complexity is low, so the drop in performance of \emph{MorphStream(OG\_NS\_Lazy)} is not as obvious as that of \emph{MorphStream(OG\_DFS\_Eager)}, but it is still lower than that of two-layer setting.  \system's flexibility enables a "hybrid" type scheduling strategy for multiple state transaction groups, which get the best of the all scheduling strategies.}

% \subsubsection{
% \change{Evaluation of Stream Window Queries}
% }
\subcompact
% Figure environment removed


% Figure environment removed


% Figure environment removed

\subcompact
\subsection{Overhead}
\label{subsec:overhead}
\system achieves adaptive scheduling at the cost of more complex runtime operations such as data structures constructing and exploring available state access operations. 
These extra operations can negatively impact the system in the following two ways: 1) the complex construction and exploration process may increase the latency of transaction processing, and 2) the auxiliary data structure will increase the memory consumption of the application. 

\subcompact
\subsubsection{Latency overhead} 
Following a prior work~\cite{tstream}, we show the time breakdown in the following aspects. 
1) \emph{Useful Time} refers to the time spent on doing actual work including accessing shared mutable states and performing user-defined functions.
2) \emph{Sync Time} refers to the time spent on synchronization, including blocking time before lock insertion is permitted or blocking time due to synchronization barriers during mode switching.
3) \emph{Lock Time} refers to the time spent on inserting locks after it is permitted.
4) \emph{Construct Time} refers to the time spent on constructing the auxiliary data structures, e.g., \tpg in \system and operation chains in \tstream.
5) \emph{Explore Time} refers to the time spent on exploring available operations to process.
6) \emph{Abort Time} refers to the wasted computation time due to abort and redos.

Figure~\ref{fig:Breakdown} shows the time breakdown when the system runs the dynamic workload in Section~\ref{subsec:evaluation}. There are three key takeaways.
First, although \tstream and \system spend a significant portion of time during construction (\emph{Construct Time}), they successfully reduce synchronization (\emph{Sync Time}) and lock (\emph{Lock Time}) overhead compared to \sstore. This explains their better performance on multicore processors.
Second, \tstream has the highest abort time (\emph{Abort Time}) because \tstream has to redo the entire batch of transactions when a transaction abort happens. 
In contrast, \sstore spends little time in abort as it involves little redo of state transactions because dependent transactions are executed serially.
Third, we can see that \system still spends a significant fraction of time performing exploration (\emph{Explore Time}). This is mainly caused by excessive message-passing among threads. In the future, we plan to investigate more efficient exploration strategies such as prioritizing mechanisms~\cite{kwok1999static} in \system.

\subsubsection{Memory footprint} 
\label{subsec:memory-footprint}
We use the dynamic workload in Section~\ref{subsec:evaluation} to evaluate memory footprint. Figure~\ref{fig:memory_footprint} demonstrates the memory consumption of three TSPEs without GC involved. 
In the initialization stage, the memory consumption of all systems is almost the same. 
\system spends more time during initialization compared to \tstream as it needs to initialize more data structures to support adaptive scheduling. 
During runtime, \system and \tstream consume a similar amount of memory per batch of state transactions, and both consume much more than \sstore. This is because they construct auxiliary data structures for scheduling, and especially they may maintain multiple physical copies of each state at different timestamps during execution (Section~\ref{subsec:abort_handling}).
Note that, as we have configured \system to not clear temporal objects and the JVM size to be large enough (300GB), the total memory usage keeps increasing until execution is finished, during which no GC is triggered. 
We plan to incorporate stream compression~\cite{Cstream,CompressStreamDB} in \system to reduce such high memory footprints in future.

\subsubsection{Clean-up and GC overhead}
\label{subsec:GC}
Figure~\ref{fig:varying_jvm} shows the impact of clean-up under varying JVM sizes from 100GB to 300GB. We can see that enabling clear temporal objects brings down the performance of \system up to 12.8\%, and still outperforms TStream and S-Store. In Figure~\ref{fig:varying_jvm}(b), the memory usage fluctuates up and down when the JVM size is set to 100GB or 200 GB because the JVM periodically reclaims (GC) the temporary objects in the continued processing of data streams.
\subcompact
% Figure environment removed
% Figure environment removed
 
\subcompact
\subsection{Impact of Scheduling Decisions}
\label{subsec:scheduling_decision}
In this section, we evaluate the impact of varying scheduling decisions under different workload characteristics using GS due to its flexibility.
% We present the model decisions study to demonstrate how the lightweight model makes decisions under varying S-TPG properties.
% We vary the properties of the constructed S-TPG by varying workload characteristics introduced in Section~\ref{sec:model}.

\subsubsection{Impact of Exploration Strategies}
% We first study the impact of different exploration strategies (i.e., \nse vs. \se).
% As discussed in Section~\ref{subsec:model}, two key factors affect selecting better-performing exploration strategies, namely 1) the \emph{punctuation interval} and 2) \emph{workload skewness}.
We first study the effectiveness of different exploration strategies (i.e., \nse vs. \se) mainly affected by the \emph{punctuation interval} and the \emph{workload skewness}, as discussed in Section~\ref{subsec:model}.
% by tuning the \emph{punctuation interval} and workload \emph{skewness}.
% then use the \emph{batch size / keys} and \emph{skewness} to. 
Figure~\ref{fig:low_skewness} shows the effects of selecting different exploration strategies under varying \emph{punctuation interval} and low \emph{workload skewness}.
\nse works better when \emph{punctuation interval} is low, while \se works better when \emph{punctuation interval} is high. 
%This is because there is a linear proportional relationship between the \emph{punctuation interval} and the number of dependencies (\td/\pd) of the constructed \tpg.
This is due to the linear proportionality between the \emph{punctuation interval} and the number of dependencies (\tds/\pds) of the constructed \tpg.
When the \emph{punctuation interval} is low, \nse resolves the rare dependencies as soon as an operation has been successfully processed, leading to higher system concurrency. \se works better otherwise as the notification overhead of the \nse approach keeps increasing with more dependencies in the workloads. However, \se has a constant construction and synchronization overhead for dependencies resolution.
Figure~\ref{fig:state_access_skewness} shows the effects of selecting different exploration strategies under varying \emph{workload skewness} and high \emph{punctuation interval}. We can see that \se works better when the state accesses are uniformly distributed, i.e., the Zipf skew factor is 0. \nse works better when the state accesses are skewed. This is because a skewed workload leads to load unbalance among threads and intensifies the synchronization overhead when \se is applied as summarized in Table~\ref{tab:decisions}.
\subsubsection{Impact of Scheduling Granularities}
In this section, we study the effectiveness of different scheduling granularities (i.e., \fsu v.s. \csu), which are affected by the following key workload characteristics, namely \emph{cyclic/acyclic}, \emph{number of state access}, \emph{punctuation interval}, and the \emph{ratio of multi-accesses}.
First, 
Figure~\ref{fig:low_num_access} shows the results of different scheduling granularities under the workload with or without cyclic dependencies. \csu performs better when there is no cyclic dependency among the batched scheduling units since each thread can schedule a group of operations together as a scheduling unit to amortize the context switching overheads. 
However, our further experiments reveal that when there is a large number of state access, \fsu is always better than \csu, regardless of whether there are circular dependencies. This is mainly due to the fact that even without circular dependencies, a large number of state accesses will increase the number of \pd, causing a significant overhead on resolving the dependencies among operations of the same group.
Second,
Figure~\ref{fig:punctuation_interval} shows how varying \emph{punctuation interval} affects the selection of scheduling unit granularities when there are no cyclic dependencies. 
We set the number of state accesses to one to avoid the effect of \pd, so the punctuation interval only controls the number of \tds in the \tpg.
We can see that \csu achieves higher throughput at higher punctuation intervals. 
When the punctuation interval is high, the large number of \tds increases the context-switching overhead in \fsu, which is why the performance of \fsu decreases when the punctuation interval is large, such as 81920.
In contrast, \csu schedules the operations in a group resulting in lower context-switching overhead on resolving \td compared to the \fsu.
Third,
Figure~\ref{fig:state_access_number} shows that \fsu works better when the ratio of multiple state access is high, while \csu works better when the ratio is low. The ratio of multiple state access controls the number of \pds among operations, as we can see that the \pd affects the performance of \csu significantly. This is mainly because the execution concurrency drops when the number of \pds is high. 
 
  % Figure environment removed

\subsubsection{Impact of Abort Handling Mechanisms}
Finally, we study the impact of two abort handling mechanisms (i.e., \ea v.s. \la) mainly affected by the \emph{abort ratio} and the \emph{computation complexity} workload characteristics.
Figure~\ref{fig:high_abort_ratio} shows the comparison results of varying \emph{computation complexity} when the \emph{abort ratio} is high. 
A lower computation complexity leads to a low redo overhead, and \la handles frequent aborts together to reduce the context switching overheads. 
\ea is better otherwise, as it makes a minimum impact on the ongoing execution of other operations.
% shows that the \la works better when the abort ratio is high and the computation complexity is low. 
Figure~\ref{fig:abort_ratio} shows the results of different abort handling mechanisms under different abort ratios when the computation complexity is low. The results indicate that as the ratio of aborting transactions increases, \la works better. The key reason is that when the computation complexity is low, the context-switching overheads and the synchronization overhead among threads to achieve fine-grained state rollback and redo become the major bottlenecks.
\subcompact

\subcompact
\subsection{Impact of Modern Hardware}
\label{subsec:modern_hardware}
In this section, we compare \system with existing TSPEs on how they interact with modern multicore processors from the modern hardware architecture perspective. 
% And we then show how the evaluation results vary under increasing number of cores.

\textbf{Micro-architectural Analysis.}
We take SL as an example to show the breakdown of the execution time according to the Intel Manual. Figure~\ref{fig:uarch_topdown} compares the time breakdown of different TSPEs. We measure the hardware performance counters through Intel Vtune Profiler during the algorithm execution and compute the top-down metrics. 
We have three major observations.\margi{R5D8}First, the breakdown results reaffirm our previous analysis that \system spends up to 2.3x fewer clock ticks for transaction processing compared to \tstream and \sstore, because of its more efficient adaptive scheduling strategies. Second, all three TSPEs \change{are Memory Bounded}, i.e., a large proportion of CPU cycles are spent due to memory access instructions: \system (58.5\%), \tstream (63.3\%), and \sstore (80.9\%).
% \change{This means }.
The detailed profiling with Intel Vtune Profiler reveals that it is commonly due to the heavy usage of latches to resolve dependencies among transactions while accessing the shared-mutable state. Both \tstream and \sstore have a higher Memory Bound than \system due to the higher synchronization cost. Nevertheless, \change{Figure~\ref{fig:uarch_topdown} and\margii{-15pt}{R5D8}Figure~\ref{fig:Breakdown} jointly indicate that \system can adopt more efficient exploration strategies to further improve its performance.}
% reduce the synchronization overhead.

\begin{comment}
this is mainly because of the inevitable dependencies solving among transactions during accessing shared-mutable state. 
\tony{Nevertheless, both \tstream (63.3\%) and \sstore (80.9\%) show a higher Memory Bound than \system (58.5\%) due to the higher synchronization cost. -- are you sure?}
\end{comment}
% and failed to better utilize hardware resources. 
% Similar to our previous observation, 
% There are four further key takeaways.
% First,
% \emph{Abort Time} refers to the wasted computation time due to abort and redos.
% TStream has the highest abort time because TStream has to redo the entire batch of transactions when transaction abort happens.
% S-Store spends little time in abort as it involves little redo of state transactions because dependent transactions are executed serially.
% Second,
% both \system and TStream successfully reduce synchronization overhead (\emph{Sync Time}) compared to S-Store. This explains their better performance on multicore processors.
% Third,
% \emph{Construct Time} refers to the time spent on constructing data structure e.g., \stpg in \system and operation chains in TStream. There is no construction overhead in S-Store, as it does not dynamically decompose state transactions.
% In contrast, both TStream and \system spend a significant portion of time during construction, due to the costly transaction decomposition process. This also explains why \system is not linearly scalable.
% Our further investigation reveals that a construction bottleneck is at the sorting of operations at runtime. 
% % fine-grained dependency identification, mainly causing by the sorting of operations by timestamp (Section~\ref{subsec:construction}). 
% Further incorporating more efficient concurrent sorting algorithms~\cite{chhugani2008efficient} in \system is a valuable future work.
% Forth,
% \emph{Explore Time} refers to the time spent on exploring available operations to process.
% We can see that \system spends a significant fraction of time to perform exploration due to the message-passing among threads.
% More efficient exploration strategies such as prioritising mechanisms~\cite{kwok1999static} may be further incorporated in \system thanks to its modularized architecture. 
% % Figure environment removed

\textbf{Multicore Scalability.}
Figure~\ref{fig:multi_core} shows the scalability comparison among TSPEs, with two major observations. 
First, \system outperforms prior schemes with an increasing number of cores confirming the good scalability of \system. However, there is still a large room for further improving \system towards linearly scale-up, the reason being that it becomes memory bounded as Figure~\ref{fig:uarch_topdown} previously shown.
Second, when the number of cores is low, \system performs even worse than \sstore due to the large constant overhead of \tpg construction process. In a resource constraint setting, existing non-adaptive solutions may be more favoured.
% % Figure environment removed




\subcompact
\subcompact
\subsection{Case Study}
\label{subsec:use_cases}
\myc{
We further demonstrate the practical usefulness of \system with two case studies that need to maintain shared states with the requirements of high performance and correctness.
% : \textit{Online Social Event Detection} and \textit{Stock Exchange Analysis}.
}

\myc{
\subsubsection{Online Social Event Detection}
}
\myc{
Detecting unexpected data patterns on social media platforms is a pressing need known as online social event detection (OSED)~\cite{fedoryszak2019real,hasan2018survey}. This process entails continually processing streaming data while reading and updating three shared states: \textit{Word}, which represents meaningful concepts in alphabetic form; \textit{Tweet}, which denotes a standardized string of words; and \textit{Cluster}, which is a collection of tweets encoding similar event information.
}

% Figure environment removed

\myc{
Numerous ways have been proposed for OSED~\cite{hasan2018survey}. In this demonstration, we use a novel method called \textit{Hybrid Event Detection}~\cite{sahin2019streaming}, which consists of two stages:
\emph{(1) Burst Keyword Detection:}
This stage finds and labels words with a high frequency growth across windows as \textit{burst keywords}. 
\emph{(2) Tweet Clustering:}
Tweets containing \textit{burst keywords} are grouped together into tweet clusters based on cosine similarity. At the end of each window, clusters with significant growth rates are identified as output events.
}

\myc{
OSED exhibits several distinguishing features:
\emph{Firstly,} high data processing capacity is required to successfully capture trending topics from enormous streams of social media posts, along with low latency to respond to real-time event detection requests.
\emph{Secondly,} it requires highly concurrent accesses to shared states (i.e., \textit{Word}, \textit{Tweet}, and \textit{Cluster}) to track event evolution, making it difficult to assure state consistency in complex transactional dependencies.
\emph{Lastly,} the workload characteristics of the input social media post stream are highly dynamic, posing a challenge in allocating tasks effectively among executors.
}

\myc{
To address these challenges, \system structures the three types of shared state (\textit{Word}, \textit{Tweet}, and \textit{Cluster}) as shared mutable key-value pairs, as depicted in Figure~\ref{fig:Online_Event_Detection_Workflow}. Each pair contains multiple fields facilitating the necessary computations for event detection. To ensure state consistency, \system maps simultaneous state access operations as a single transaction. It effectively resolves transaction dependencies and adapts to the most suitable scheduling strategy under dynamic workload characteristics.
}

\myc{
The workflow of OSED on \system is illustrated in Figure~\ref{fig:Online_Event_Detection_Workflow}.
% represented as a DAG with six operators, each responsible for processing a specific type of state transaction through user-defined and system-provided APIs.
% The process begins with the \textit{Tweet Registrant}, which registers pre-processed tweets into the state, decomposes them into word tokens, and distributes them downstream. The \textit{Word Updater} then updates the frequency information of words into the state. The \textit{Trend Calculator} identifies burst keywords with significant increases in their TF-IDF values across windows and emits tweets containing these burst keywords for clustering. These actions are performed after updating the state of all words in the current window, ensuring that no burst keywords are overlooked, controlled by the \textit{punctuation} mechanism~\cite{mao2023morphstream}. Upon receiving filtered tweets, the \textit{Similarity Calculator} measures their similarities with existing clusters and determines the most suitable clusters. New clusters are initialized if no matching clusters are found. Once all tweets in the window are merged into clusters by the \textit{Cluster Updater}, the \textit{Event Selector} identifies clusters with high growth rates as events and propagates them as output. Additionally, clusters with zero updates for a long time are removed.
The \textit{Tweet Registrant} initiates the process by registering pre-processed tweets into the state, decomposing them into word tokens, and distributing them downstream. The \textit{Word Updater} then updates the words' frequencies to the state. The \textit{Trend Calculator} then identifies burst keywords with significant increases in TF-IDF values across windows and emits tweets with these burst keywords for clustering. These actions are carried out after the status of all words in the current window has been updated, ensuring that no burst keywords are overlooked, as controlled by the \textit{punctuation} mechanism~\cite{mao2023morphstream}. After receiving filtered tweets, the \textit{Similarity Calculator} compares them to existing clusters and determines the most suitable clusters. If no matching clusters are found, new clusters are created. Once all tweets in the window are merged into clusters by the \textit{Cluster Updater}, the \textit{Event Selector} identifies clusters with high growth rates as events and propagates them as output. Meanwhile, clusters with no updates over an extended period of time are eliminated.
}

% Figure environment removed

\myc{
We conducted our analysis using a dataset of real-world tweets~\cite{olteanu2014crisislex}, comprising English tweets from five crisis events that occurred in the United States between 2012 and 2013. This dataset consists of approximately 30,000 tweets, both event-related and non-event-related. For our application, we deployed four executors for each operator, with a total punctuation interval of 400 tweets. Each thread was allocated 100 tweets per batch to execute.
In Figure~\ref{fig:ed_performance}, we present the performance results of our online event detection implemented using \system, compared to the actual evolution of events over time. The popularity of events is measured by the number of new tweets merged into a specific event cluster within each time window. 
The results demonstrate that our online event detection, supported by \system, accurately detects the emergence of events and effectively captures changes in event popularity trends within seconds, as indicated by the time difference in event popularity summits. We also observed that \system achieved an overall throughput of up to 1.3k tweets per second for processing and detecting events. These findings provide compelling evidence of \system's ability to efficiently support complex real-time applications.
}



\subsubsection{Real-time Stock Exchange Analysis}
One common stock exchange analysis (SEA)~\cite{sse} task is to get the turnover rates of stocks by calculating the trade ratio of each stock for every period of time. 
% The input data are a stream of quotes and a traded stream that contains traded results of matched quotes.
The input data is a stream of quotes and a stream of trades containing the trade results of matching quotes.
The query joins the traded stream and the quotes stream ($S$) over the same stock id within the same period of time (i.e., window). 
% Such a stock analysis task can be implemented using the hash-based window join algorithm.
Specifically, given the unbounded nature of streaming data, windows are commonly employed to restrict the number of tuples involved in the computation. 
Subsequently, window join matches the traded and quote records for each stock to calculate its associated turnover rates.
% The overlapping windows provide opportunities for work-sharing, i.e., the result of one window evaluation can be used to help with
% the evaluation of the next window. 
% For example, a recent work~\cite{shahvarani2020parallel} successfully utilizes a shared window index to accelerate sliding window joins, reduce redundant memory access during tuple matching, and improve performance. This approach can be particularly useful in parallel and distributed systems, where the goal is to maximize resource utilization and minimize communication overhead.
The SEA can be implemented using the hash-based window join algorithm.
The algorithm maintains two hash tables, one for each input stream.
When it receives a tuple from Traded (or Quote) stream, it inserts the tuple into the hash table of Traded (or Quote) and immediately probes the hash table of the opposite stream Quote (or Traded). 

% Figure environment removed

% \myc{
% Indexing the contents of a sliding window necessitates a significant amount of concurrent updates, which must meet the critical requirements of high throughput and low latency to effectively support stream processing that relies on the index. 
% Intuitively, the arrival of each new tuple will trigger an update to the index structure. Shahvarani et al.~\cite{shahvarani2020parallel} propose an effective concurrency control mechanism to meet the demands of high-rate updates during indexed stream window join. Nevertheless, the proposed concurrency control mechanism is still based on locks, which are known to be costly.
% }

\myc{
With \system, we can intuitively map the hash table structure to the shared state and model insertion and probe requests to the hash table as state transactions. 
% Furthermore, we implement a lightweight data structure to capture the dynamic changes of indexes in streaming settings. 
An overview of the stock analysis workflow implemented by hash-based window join is shown in Figure~\ref{fig:Stock_Exchange_Workflow}. 
\system maintains two hash tables \(Index(Traded)\) and \(Index(Quotes)\) for streams \(Traded\) and \(Quotes\), respectively as shared state. 
The key $k$ of the state is the stock id and the value $<r, ...>$ contains all arrived tuples in the current window slide.
% The index state is implemented in the form of a hashmap $\texttt{<}key,address\texttt{>}$ that associates tuple keys with their in-memory storage addresses. 
% we address two key associated implementation challenges. 
% Firstly, we implement a lightweight data structure to capture the dynamic changes of indexes in streaming settings. 
% Secondly, we implement an efficient, non-blocking concurrency control scheme to ensure indexing correctness under high-intensive read and write operations. 
% The need to ensure high-throughput, low-latency stream processing further magnifies the challenges. We will demonstrate that by leveraging the scalable TSPE \system, we can efficiently maintain shared window indexes with transactional semantics and further optimize the performance of stream window operations on multicore processors.
}

\myc{
% \textbf{Workflow Overview.}
% For simplification, we based the shared window index optimization on \textit{Index-Based Window Join}~\cite{shahvarani2020parallel} with an equality predicate. 
% At runtime, indexes are dynamically updated by multiple worker threads as the window moves.
The join operation involves the following steps. 
When a new tuple \(r\) arrives from stream \(Traded\), \system searches for matching tuples in \(HashTable(Quotes)\) by efficient index lookup. Once matched tuples \(<s, ...>\) are identified, \system calculates turnover rates accordingly during a window slide and propagates the result $\texttt{<}r,s\texttt{>}$ as the join output. 
Subsequently, it deletes the expired tuple and inserts the new tuple \(r\) into \(HashTable(Traded)\), updating the multi-version state storage accordingly. 
All accesses to hash tables, such as insert and delete, are mapped to state access operations and are subsequently modelled as state transactions in \system. Only when all operations of a state transaction are performed successfully, the transaction is committed. Otherwise, the state is restored to the latest version before abort, and the transaction will be re-executed.
}

% % Figure environment removed

% Figure environment removed

% We sampled the turnover rates calculated from 100 continuous quotes/traded events in Figure~\ref{fig:stock_dataset}.


\myc{
Figure~\ref{fig:stock_performance} shows the performance results of the stock exchange analysis implemented based on \system.
% We use a real-world stock exchange dataset~\cite{sse} for this workload that contains tens of millions of quote and traded records. 
% We configured the batch interval as 40k and deployed the application with 4 executors and used the default scheduling strategy for evaluation.
For evaluation, we utilized a real-world stock exchange dataset~\cite{sse} containing tens of millions of quote and traded records. 
The application was deployed with 10 executors, and the batch interval was configured as 1k, where each executor will be evenly allocated with 1k records per batch for execution. 
% We set the default scheduling strategy for \system.
% We mainly showcase the expected accumulated matched result after a traded/quote event has been generated and the actual result output by \system.
The primary focus of the performance evaluation is on the expected accumulated matched result generated by traded/quote events and the actual result output by \system.
The performance results demonstrate that \system consistently outputs the actual results within the millisecond level.
We also measured that the throughput of the \system can process up to 70k events per second.
These findings confirm the efficiency of implementing real-time financial applications in \system, as it can achieve the ACID guarantees for transactions while maintaining high throughput and low latency.
}
 