\section{Conclusion}
\label{sec:conclusion}
In this work, we introduced \system, a TSPE designed to optimize parallelism and performance for stream applications managing shared mutable states. 
Through a unique three-stage execution paradigm, \system enables dynamic scheduling and parallel processing in TSPEs. 
The \emph{Planning Stage} effectively tracks dependencies using a two-phase \tpg construction process and virtual operations. 
The \emph{Scheduling Stage} dynamically adjusts exploration strategy, granularity, and abort handling, while \emph{Execution Stage} ensures correct transaction handling using a Stateful \tpg. 
Our experiment showcased the remarkable capabilities of \system, significantly outperforming existing TSPEs in various scenarios.
We plan to continue improving \system, with a focus on integrating fault tolerance mechanisms and addressing other challenges presented by evolving real-time stream applications.

% \section{Conclusion}
% \label{sec:conclusion}
% In this paper, we introduced \system, an innovative transactional stream processing engine designed to manage complex, large-scale real-time data stream processing tasks efficiently. \system achieves this by leveraging a flexible and adaptable API design coupled with an effective execution mechanism, which integrates transactional stream processing into a three-stage execution workflow.

% We showed how \system adapts to diverse application-specific requirements while maintaining high throughput and low latency. The transaction precedence graph (TPG) proposed in \system enables better scheduling decisions, reducing unnecessary aborts and improving system performance. The evaluation of \system confirmed its effectiveness and efficiency in processing transactional data streams, outperforming alternative approaches in various experimental scenarios.

% However, providing efficient fault tolerance remains a challenging aspect of TSPEs, even in a single-node setting due to the inherent complexity of combining transactional and stream-oriented properties. This complexity may potentially increase in a distributed environment. In the future, we aim to focus our work on enhancing \system's fault tolerance while maintaining its current efficiency in processing transactional data streams.

% Our work on \system demonstrates the potential of integrating transactions into stream processing. We believe this research will pave the way for more sophisticated and efficient real-time data stream processing systems, thereby supporting increasingly complex and demanding data-intensive applications in various domains.
% Transactional stream processing engines (TSPEs) have been increasingly gaining traction. 
% In this work, we show that the scheduling strategies of TSPEs can be decomposed into three dimensions of scheduling decisions, exhibiting trade-offs among execution concurrency, context switching overhead, and wasted computational efforts due to aborts.
% To this end, we propose a novel TSPE \system that is able to morph flexibly among scheduling strategies adapting to dynamically changing workload characteristics. 
% Guided by a lightweight decision model, \system can make the correct scheduling decision at runtime with minor overheads, which yields a multi-times performance improvement over the state-of-the-art.
 