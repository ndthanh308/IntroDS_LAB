\documentclass[a4paper, 10pt]{letter}

\address
{
}

\begin{document}

\begin{letter}

Dear Editor,\\[2ex]
we hereby submit our manuscript “\textit{Convergence of Digitized-Counterdiabatic QAOA: circuit depth versus free parameters}” for your consideration as a regular article in Physical Review B. \\[1ex]
We study the quantum approximate optimization algorithm (QAOA) and two of its variants, which exploit digitized counterdiabatic driving to improve the performance of the original algorithm. The recent interest in these improved methods is primarily due to their potential to generate shorter quantum circuits, offering a promising advantage in the noisy-intermediate scale quantum era, where longer computations suffer from accumulating gate errors.\\[1ex]
We contribute to this topic by applying counterdiabatic QAOA to the weighted and unweighted one-dimensional MaxCut problem. First of all, our results confirm that including counterdiabatic corrections can indeed be helpful to decrease the number of QAOA steps. Our remarkable finding is that, however, the total number of variational parameter required to achieve this faster convergence is independent of the QAOA variant for the particular model analyzed. Thus, the advantage of counterdiabatic QAOA comes at the cost of a more complex cost function landscape, as a result of the more complicated unitary operators employed in these methods. We believe that this tradeoff warrants further investigation, particularly considering the potential real-life applications of counterdiabatic QAOA in solving practical problems.\\[1ex]
We are confident that our manuscript can spark interest and drive further research in this direction, making it well suited to Physical Review B.\\[2ex]
Best Regards,\\
The authors.

\vspace{5mm}
Corresponding author: Mara Vizzuso

Mail: mara.vizzuso@unina.it

%\vspace{5mm}
%Suggested referees:
%\begin{itemize}
%    \item \dots
%    \item \dots
%    \item \dots
%\end{itemize}

\end{letter}

\end{document}
