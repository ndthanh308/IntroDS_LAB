\section{Related work}
In this section, we first discuss the existing literature of incentives in secure data deduplication, and then we discuss the schemes related to Blockchain-based data deduplication. 
Miao et al. \cite{miao2015payment} proposed one of the first works to have studied incentives in data deduplication. Their scheme encourages the cloud users to take part in data deduplication by offering incentives on storage fee. They show that the cloud users significantly saves the storage fee, and the CSP saves the storage costs when opted for dedup. However, they guarantee the fairness of storage fee by assuming a trusted party, which will compute the data deduplication rate. Another drawback is that their scheme fails to guarantee incentive compatibility for the CSP. 
 It is also not clear how the incentives are passed to cloud users who have already chosen deduplication when the new cloud users with the same data opt for deduplication. 
 Robatka and Mannan \cite{rabotka2016evaluation} have discussed various secure deduplication schemes and analyzed the incentive required for a CSP to opt those schemes. They show that CSPs do not have incentive or have tiny incentive to opt for deduplication when schemes like client-side encryption, DupLess \cite{bellare2013message}, probabilistic upload solutions \cite{harnik2010side} are used.  
 Liu et al. \cite{liu2015secure} abstractly discussed rewarding cloud users for adopting deduplication. However, they have not provided any concrete incentive model.
 Armknecht et al. \cite{armknecht2015transparent} proposed ClearBox, which encourages the CSPs to declare the deduplication rate to cloud users truthfully. They use an additional gateway to compute and attest the deduplication rate at the end of each round. However, the gateway can be compromised to report false dedup rate, and as it is a centralized entity, its availability cannot be guaranteed. Another drawback is that they have neither discussed a concrete incentive scheme nor discussed a payment mechanism between CSP and users.
 Youn et al. \cite{youn2015necessity} discussed disadvantages like privacy-disclosure risk for the first uploader of the data and the necessity of incentives to compensate for the first uploader disadvantages.
 Wang et al. \cite{wang2015modeling} constructed a defense scheme for the side-channel attacks in deduplication schemes. They modeled the attack by considering economic factors as a non-cooperative game between CSP and an attacker. 
 Liang et al. \cite{liang2019game} proposed a game-theoretic analysis of deduplication scheme. They constructed an incentive scheme based on a non-cooperative game between CSP and users. They also computed bounds on the incentives a CSP can offer to a cloud user for choosing dedup. Their extensive experimental analysis shows that their incentive mechanism satisfies the incentive rationality and incentive compatibility of both CSP and users. The main drawback in their scheme is that the CSP announces the storage fee based on dedup rate and its utility at specific intervals of time. A malicious CSP may report false deduplication rate and utility. According to \cite{liang2019game}, CSP adjusts the storage fee according to deduplication rate as the time passes, but these discounts at later times may lead to discrepancies in the storage fee paid by all the users who hold the same data and chosen deduplication.

Recently, some works\cite{li2018deduplication,wang2019blockchain,ming2022blockchain,huang2022blockchain} discussed deduplication using Blockchain network. The authors in \cite{li2018deduplication} use Blockchain to store tags computed on the files stored in the cloud.  After downloading data from the CSP, the cloud users can verify the tag of the downloaded data with the tag stored in Blockchain. Their objective is to provide integrity to the data stored in the cloud. Another Blockchain-based deduplication scheme is presented in \cite{wang2019blockchain}. They use a smart contract to facilitate payments between a CSP and a cloud user. However, the storage fee is computed by the CSP, and hence, a malicious CSP may report false storage fee. Another drawback is that they assume a fixed storage fee for all users irrespective of deduplication rate. They have also not discussed how their scheme realizes fair payments. Ming et al. \cite{ming2022blockchain} proposed a smart contract based deduplication scheme where a smart contract stores index of all the data stored at edge nodes of a cloud server. Whenever there is a request for data storage, the cloud searches the index stored at the smart contract and takes a decision to either store or to deduplicate the data. Huang et al. \cite{huang2022blockchain} also incorporated Blockchain to have efficient arbitration and also to distribute incentives. Similar to the proposed scheme, their scheme also achieves uniform payments by refunding the first uploader of the file whenever a new user opts for deduplication.
Recently, Li et al. \cite{li2022blockchain} and Song et al. \cite{song2023blockchain} demonstrated integrity auditing of deduplicated data using Blockchain. When compared to the above-discussed schemes, our proposed scheme not only provides confidentiality, integrity, incentives but also provides fair payments without a trusted party. 