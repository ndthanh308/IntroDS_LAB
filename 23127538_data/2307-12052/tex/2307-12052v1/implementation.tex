\section{Implementation}
We have written the contracts in solidity 0.8.0 using truffle framework. We have used a 2.50 GHz intel core i5 processor and a 4 GB RAM machine to run our experiments. We have deployed the proposed contract in a private Ethereum network (https://www.trufflesuite.com/ganache) which mimic public Ethereum network. However, our goal of this implementation is to deploy the contract in public Blockchain networks in real scenarios.
\subsection{Implementation of $\mathcal{B}_{DEDU}$}
We have tested the proposed smart contract $\mathcal{B}_{DEDU}$ multiple times, and each transaction's transactional cost and its equivalent financial cost is shown in Table \ref{tab:costs}. Observe that the contract deployment transaction consumes a large amount of gas; however, this is a one time cost for CSP.  Next, the create and request functionalities also consumes a large amount of gas due to the modification of contract storage variables. Storing data in a contract is an expensive operation in Ethereum. As the usrConf function executes main tasks like computing dedup rate, sending the storage fee to CSP, and sending refunds to users, the transaction to usrConf also consumes a large amount of gas. The gas consumption of usrConf varies and depends on the number of users opted for deduplication before the transaction initiator's call. We listed the gas consumption of usrConf function in Figure \ref{fig:graphusrConf}. Observe that the gas consumption increases with the increase in dedup rate, and it reaches more than the block gas limit.
\begin{table}
	\centering
	\begin{tabular}{ccc}
		\hline
		\textbf{Function} & \textbf{Caller} & \textbf{Gas cost}   \\ \hline
		Init (contract deployment) & CSP & 187467  \\
		Create & CSP & 143464 \\
		Request & User & 161168 \\
		Pay & User & 66558  \\
		CSPConf & CSP & 31457 \\
	%	CliConf & User & 191146 &0 &0.046\\
		Refund & User & 31779 \\
		Claim & CSP & 31549\\
		DeLink & User & 29318 \\
		\hline
	\end{tabular}
	\caption{Costs of interacting with $\mathcal{B}_{DEDU}$ contract.}
\label{tab:costs}
\end{table}
%\pgfplotsset{width=\textwidth,compat=1.3}
% Figure environment removed


\subsection{Experiment 1: Finding utility of the users and the CSP by varying $n_{d}^{c}(t)$ and $EF_{u}^{c}(t)$}
We have conducted experiments with the values shown in Table \ref{tab:parametersettings} adopted from \cite{gao2016game,liang2019game}. The utility of the users and the CSP are shown in Figure \ref{fig:graphsUserUtility} and Figure \ref{fig:graphsCSPUtility} respectively. We varied $EF_{u}^{c}(t)$ from 10\% to 50\% of $SF_{u}^{c}(t)$ in both the figures. We also varied the deduplication rate $n_{d}^{c}(t)$ as 10\%, 50\%, 90\%, 100\% of $N_{d}^{c}(t)$ . Observe that both the users and CSP obtains non-negative utilities when opted for dedup with $\mathcal{B}_{DEDU}$. This shows that our proposed scheme is individually rational.

 In Figure \ref{fig:graphsUserUtility}, the utilities of the users decreases as the $EF_{u}^{c}(t)$ increases. 
 %When the number of users is less and the dedup rate is also less then the user is not incentive compatible due to the new factor $EF_{u}^{c}(t)$. 
 However, for any constant $EF_{u}^{c}(t)$ value the average utility of users increases with increase in dedup rate. This result agrees with the general notion of increase in dedup rate increases the average utility of users. Figure \ref{fig:graphsUserUtility} also shows that the user is incentive-compatible that is $U_{u}^{1}(t)>U_{u}^{0}(t)$ 
 $\forall n_{d}^{c}(t) >1$. The results show that the proposed scheme is not incentive compatible for $n_{d}^{c}(t)=1$, due to the new parameter $EF_{u}^{c}(t)$. To make the user incentive compatible when $n_{d}^{c}(t)=1$, the $EF_{u}^{c}(t)$ value should be set to zero.
 
 Figure \ref{fig:graphsCSPUtility} shows that the CSP is not incentive compatible until $EF_{u}^{c}(t) \ge 36\%$ of $SF_{u}^{c}(t)$. This result is in line with \eqref{eq:minCF} when the values in Table \ref{tab:parametersettings} are substituted.
\begin{table}[!h]
	\centering
	\begin{tabular}{c|c}
		\hline
		\textbf{Parameter} & \textbf{Value in Ether}  \\ \hline \hline
		$P_{u}(t)$    &    2.165     \\
		$SF_{u}^{c}(t)$ & 0.165  \\
		$SC_{c}^{u}(t)$&  0.1  \\
		$AF_{c}(t)$ & 0.1\\
		\hline
	\end{tabular}
	\caption{Experiment Settings.}
	\label{tab:parametersettings}
\end{table}
\pgfplotsset{width=6cm,compat=1.3}
% Figure environment removed

% Figure environment removed
\subsection{Experiment 2: Testing $\mathcal{B}_{DEDU}$ and $\mathcal{B}_{I\text{-}DEDU}$ with public dataset}
In Experiment 1, we have considered only similar requests where all users have the same data file with the same size. Now, in this experiment, we have chosen a dataset which consists of information of Debian packages gathered from Debian popularity contest (https://popcon.debian.org/contrib/by\_inst). We took a snapshot of the number of packages, the number of installations of a single package, and size of each package on 7th May 2020. Each package represents a data file to be stored in the cloud, and the installations serve as the deduplication requests. They were a total of 403 packages (data files) and 270738 installations requests (users). The dataset is very diverse as it consists of data files having different sizes and each data file have a different number of installations. We uniformly distributed the requests among five CSPs and computed their utilities. As we discussed earlier, the inter-CSP deduplication is same as single CSP deduplication from the user point of view, and hence no change in the utility of users. But, the utility of the CSP changes as $U_{c}^{2}(t)=U_{c}^{1}(t)+AF_{in}^{c}(t)-AF_{out}^{c}(t)$. $AF^{c}(t)$ is the fee paid by a CSP $c_{i}$ to CSP $c_{j}$ for accessing the data stored at $c_{j}$. In Figure \ref{fig:interCSPUtilities}, we show the utilities of the above considered dataset by taking $EF_{u}^{c}(t)=40\%$ of $SF_{u}^{c}(t)$ and $n_{d}^{c}(t)=100\% \text{ of } N_{d}^{c}(t)$. Observe that the CSPs gain more utility when opted for inter-CSP deduplication. The gain is due to the increase in the dedup rate. Therefore, with our proposed incentive mechanism dedup with $\mathcal{B}_{I\text{-}DEDU}$ is more profitable than dedup with $\mathcal{B}_{DEDU}$ and dedup with $\mathcal{B}_{DEDU}$ is more profitable than no deduplication.
\pgfplotsset{width=0.8\textwidth,compat=1.3}
% Figure environment removed
\begin{table}
	\centering
	\begin{tabular}{p{1.2cm}|p{1.7cm}|p{1cm}|p{2cm}|p{2.2cm}|p{0.8cm}}
		\hline
		\textbf{Scheme} & \textbf{Category}& \textbf{Incen-tive} & \textbf{Feature of Incentives} & \textbf{Correctness of deduplication rate} & \textbf{Fair-ness} \\ \hline \hline
		\cite{miao2015payment} & server-controlled& \checkmark & IR & \checkmark (Trusted party) & x\\ \hline
		%\cite{liu2015secure} & client-controlled & x & - & x & x\\ \hline
		\cite{li2018deduplication} & Blockchain-controlled& x & x & x & x \\ \hline
		\cite{liang2019game} & server-controlled& \checkmark & IR, IC& \text{x} & x\\ \hline
		\cite{wang2019blockchain} & client-controlled & \checkmark & - & x & partial \\ \hline
		\cite{huang2022blockchain} & server-controlled & \checkmark & - & x  & \checkmark (Arbi-tration)  \\ \hline
		\cite{ming2022blockchain} & Blockchain-controlled & x & x & x & x \\ \hline
		Proposed scheme & Blockchain-controlled& \checkmark & IR, IC& \checkmark (Blockchain)& \checkmark\\
		\hline
	\end{tabular}
	\caption{Comparison with existing works}
	\label{tab:comparisons}
\end{table}
\section{Comparison with existing schemes}
Table \ref{tab:comparisons} shows the comparison of our work with \cite{miao2015payment}, \cite{li2018deduplication}, \cite{liang2019game}, \cite{wang2019blockchain}, \cite{huang2022blockchain} and \cite{ming2022blockchain} in terms of category, incentives, features of incentives, correctness in computation of deduplication rate and fairness. 
\cite{miao2015payment} provides correctness of dedup rate but uses the services of a trusted party known as deduplication rate manager. In contrast, our scheme does not rely on the trusted party and still obtain correctness. The incentive mechanism in \cite{miao2015payment} supports only IR-constraint, whereas our scheme supports both IR-constraint and IC-constraint. The dedup scheme proposed in \cite{li2018deduplication} focuses more on the integrity of the deduplicated data stored in the cloud, whereas our scheme focuses on incentives and fair payment mechanism. The incentive mechanism in \cite{liang2019game} supports both IR-constraint and IC-constraint but, an untrusted CSP computes the dedup rate. Also, the incentives in \cite{liang2019game} are time-variant, whereas our scheme supports uniform payments. %Comparing with \cite{liu2015secure}, \cite{liang2019game}, our scheme provides correctness on deduplication rate and fairness. 
Although \cite{wang2019blockchain} have considered fair payments, their scheme does not support fair payments in all the cases defined in Theorem \ref{th:fairpayments}. Huang et al. provides incentives, uniform payments as well as fairness through arbitration protocol. However, the arbitration protocol is not in-built into the protocol and requires multiple rounds of offline communication. Ming et al. \cite{ming2022blockchain} emphasize more on storing and retrieving deduplication information from smart contract but not on incentives and fairness.
