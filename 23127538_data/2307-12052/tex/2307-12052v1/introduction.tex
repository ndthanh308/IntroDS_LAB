\section{Introduction}
With the advent of cloud computing, outsourcing data to remote cloud storage servers  becomes a common practice \cite{6522585}. However, most of the data being uploaded is redundant \cite{meyer2012study} and thus wasting large storage spaces.  Storage systems use deduplication (dedup) techniques to eliminate redundancy. Dedup eliminates the need to upload and store redundant copies of data by verifying whether a data already exist in its storage before each upload. If the check is valid, then the data is not uploaded, and simply the corresponding user is added as owner to the existing file. The file link is sent to the requested user upon successful verification of Proof-of-Ownership. Dedup is broadly classified into two categories: client-controlled (C-DEDU) and server-controlled (S-DEDU). In C-DEDU, the user instead of sending the entire file to a cloud storage provider (CSP), he sends only a $tag$ computed on file to check for duplication \cite{youn2015necessity}. In S-DEDU, the cloud server checks for the duplication when it receives data from the user \cite{liang2019game}. Although there are many advantages, dedup introduces interesting challenges. First, to prevent a CSP from accessing sensitive information, it is a common practice to encrypt the data before uploading. If the data is encrypted using conventional encryption techniques, then it is difficult to apply deduplication techniques. Because two identical files that are encrypted with two different keys will generate two different cipher-texts which cannot be compared for similarity. Encrypted data dedup schemes are proposed based on convergent encryption \cite{douceur2002reclaiming,li2014hybrid,bellare2013message,keelveedhi2013dupless}, secret sharing \cite{li2015secure}, proof-of-ownerships \cite{yan2016deduplication}, keyword search \cite{wen2015bdo}, and password-authenticated key exchange \cite{li2015secure}. 

Second, it is clear that the greatest beneficiary of the dedup is the CSP as it saves storage cost. It is required to motivate cloud users to opt for dedup by offering incentives/discounts on storage fee. In literature, several schemes are proposed for secure deduplication, but only a few schemes \cite{miao2015payment, liu2015secure,youn2015necessity,rabotka2016evaluation, liang2019game,wang2019blockchain} discuss the incentives in deduplication. Miao et al. \cite{miao2015payment} presents an incentive scheme which reduces the storage fee of the cloud user when opted for deduplication. Liu et al. \cite{liu2015secure} states that the incentives are necessary for attracting cloud user towards deduplication. Youn et al. \cite{youn2015necessity} necessitate the need for incentives for the first data uploader. Rabotka et al. \cite{rabotka2016evaluation} discusses the role of incentives in various secure deduplication schemes. Liang et al. \cite{liang2019game} constructs a game-theoretic model to compute the bounds on incentives a CSP can offer to cloud user for opting dedup. Recently, Wang et al. \cite{wang2019blockchain} discusses the role of Blockchain in facilitating payments between CSP and cloud users. 

In most of the dedup schemes, the incentives are computed based on the dedup rate ($n_{d}^{c}(t)$). Dedup rate is defined as the number of cloud users holding a data $d$ and opted for deduplication at cloud $c$ at time $t$. Even though if the best incentive mechanism is available, either the cloud provider has to be trusted by the cloud user for fair computation of dedup rate or a trusted party has to be recruited for computing dedup rate correctly. Another problem in existing schemes is that both the CSP and the cloud user assume a trusted party like a Bank, to facilitates payment transfers between them. However, hiring a trusted party is costly and finding an ideal trusted party which will behave honestly at all times is difficult.

The recent progress in Blockchain technology allows a public Blockchain network to emulate the properties of a trusted party. The public Blockchain network is trusted for the immutability of data it possesses, the correctness of the code (smart contract) execution in its environment and its availability.

In this paper, we propose a new Blockchain-based secure cloud storage system where no party can influence the dedup rate, and also provides fair payments. We guarantee a fair dedup rate even if the CSP is untrusted and the cloud users are rational. We employ a convergent encryption (CE) scheme for providing data privacy and a proof-of-ownership (PoP) scheme for proving ownership of duplicated data. We assume that both CE and PoP schemes as black-boxes in our model, and we solely focus on designing a Blockchain-based secure cloud storage system with a new incentive mechanism.

We summarize the contributions of this paper as follows:
\begin{enumerate}
	\item The contributions in this paper are two-fold: First, we design a new incentive mechanism, and second, we design a new Blockchain-based dedup scheme by leveraging the immutability, trust, and correctness properties of a public Blockchain network.
	\item The proposed incentive mechanism motivates cloud users to choose dedup while ensuring profits for CSP. Experimental analysis shows that the proposed incentive mechanism is individually rational and incentive compatible for both users and CSP.
	\item  As most of the existing schemes focus on secure deduplication, we propose a dedup scheme which emphasizes correctness of dedup rate and fair payments between cloud user and CSPs. We design a smart contract ($\mathcal{B}_{DEDU}$) to realize the correctness and fairness of the proposed scheme.
	\item  We implement the proposed smart contracts using solidity and execute them on a private Ethereum network which emulates the public Ethereum network. We test the proposed smart contract for the publicly available dataset and present the transactional and financial costs of interacting with the smart contract.
\end{enumerate}
%Except \cite{wang2019blockchain}, all the schemes have a trusted party for computing storage fee based on deduplication rate and assumes a trusted party (like a bank) for facilitating payments between CSP and data holder. Finding an ideal party which will behave honestly all the time is difficult. 

%The following are the drawbacks of S-DEDU schemes:
%1) Typical S-DEDU schemes assume that all the data holders of same data are known initially so that they can share the storage fee and pay only their part of shared fee. But in real world, the data holders are not known upfront and their storage requests vary in time. 2) In some other schemes, when a request arrives at CSP, it computes the storage fee based on the deduplication rate and discount offered at that time. Although game theoretic analysis show that CSP cannot gain profits by reporting false deduplication rate, but in real world a malicious CSP may alter the deduplication rate so that the data holder pays more storage fee even though he opted for deduplication. 3) Many S-DEDU schemes presents analysis of different economic and incentive models, but no scheme presented about realizing payments between data holders and CSP.
%To substitute the drawbacks of S-DEDU schemes in this paper we present B-DEDU, a Blockchain based secure deduplication storage scheme which will solve the above drawbacks as follows:
%1) B-DEDU provides the flexibility of for data holders to send storage requests whenever they need, and their requests are processed by the smart contract almost instantaneously. 
%2) In B-DEDU, the deduplication rate is computed by a smart contract, so that no malicious entity can alter the deduplication rate and the data holders always pay the correct storage fee.
%3) In B-DEDU, smart contract act as a trusted mediator between data holder and CSP. Smart contract can hold real-world money, and this money is transferred to appropriate party by executing a set rules programmed into them.


