\section{Preliminaries}
\subsection{Convergent Encryption (CE)}
A convergent encryption scheme consists of four algorithms.
\begin{itemize}
	\item $K \leftarrow KeyGen_{CE}(d)$. The key generation algorithm takes the data file $d$ as input and outputs a convergent key $K$.
	\item $C \leftarrow Encrypt_{CE}(K,d)$. The symmetric encryption algorithm takes the convergent key $K$ and the file $d$ as input and generates a cipher-text $C$ as an output.
	\item $d \leftarrow Decrypt_{CE}(K,C)$. The decryption algorithm takes both the convergent key $K$ and cipher-text $C$ as input and outputs the original file $d$.
	\item $tag \leftarrow TG_{CE}(C)$. $TG$ is a tag generation algorithm which takes the cipher-text $C$ as input and outputs a hash value $tag$.
\end{itemize}
\subsection{Blockchain}
Bitcoin \cite{nakamoto2008bitcoin} is the first public Blockchain network proposed by Satoshi Nakamoto. Although its primary purpose is to transfer cryptocurrency known as bitcoin between peers without a trusted central authority, its fundamental concepts can be used as building blocks to construct many decentralized applications. In Blockchain network, the transactions are stored in a one-way append distributed ledger (blockchain) and a secure consensus protocol is executed by a set of decentralized peers known as miners to agree on a common global state of the ledger. The ledger holds the complete transaction history of the network. 
The blockchain is a sequence of blocks. Bitcoin uses the proof-of-work (PoW) consensus algorithm and most of the public Blockchain networks proposed later follows Bitcoin's PoW and commonly called as altcoins/Nakamoto-style ledgers.  


In PoW Blockchain networks, a block $b$ is of the form $b=\langle h,t,c \rangle$ where $h \in \{0,1\}^{s}, t \in \{0,1\}^{*}, c \in \mathbb{N}$ satisfying the conditions $(G(c,H(h,t))< D)$ and $(c \leq q)$
where $G(\cdot)$ and $H(\cdot)$ are cryptographic hash functions which outputs in length of $s$ bits, $D \in \mathbb{N}$ is known as block difficulty level set by the consensus algorithm and $q \in \mathbb{N}\footnote{implementation dependent}$. Let $b^{'} = \langle h^{'},t^{'},c^{'} \rangle$, be the right most block in the chain. The chain is extended to a longer chain by adding a new block $b = \langle h,t,c \rangle$ to $b^{'}$ such that it satisfies $h=G(c^{'},H(h^{'},t^{'}))$.
A consensus algorithm guarantees the security of the blockchain. PoW algorithm makes the miners compete to generate a new block periodically. The miners are rewarded for mining new blocks in the form of currency native to the Blockchain network (bitcoins in the case of Bitcoin). The PoW algorithm has several rules out of which the following two rules ensure the correctness of the execution of transactions. (1)While generating a new block, the miners verify all the transactions going to be added in a block, and (2) Miners check the validity of a new block generated by other miners before adding it to their local blockchain.  These verification steps make the Blockchain network trusted for correctness. The hardness in solving PoW puzzle makes the blockchain immutable, and a large number of participating miners ensures availability of Blockchain network. We use these properties to achieve fair privacy-preserving aggregation without a trusted third party between data owners and data buyer.
\subsection{Smart Contract}
A smart contract is a program stored in a blockchain and executed by the mining nodes of the network. The smart contract can hold many contractual clauses between mutually distrusted parties. Similar to transactions, the smart contract is also executed by miners and, its execution correctness is guaranteed by miners executing the consensus protocol. Assuming the underlying consensus algorithm of a Blockchain is secure, the smart contract can be thought of a program executed by a trusted global machine that will faithfully execute every instruction \cite{dong2017betrayal}.


Ethereum \cite{wood2014ethereum} is a major Blockchain network supporting smart contracts. A smart contract in Ethereum is a piece of code having a contract address, balance, storage and state. 
The changes in the smart contract storage change its state.
A $\mathcal{SC}$ is a collection of functions and data similar to a class in object-oriented programming. Whenever a transaction is sent to the contract's address with a function signature, then the corresponding function code is executed by an Ethereum virtual machine (in mining node).
Unlike Bitcoin, the Ethereum scripting language is Turing-complete which motivates the developers to write smart contracts for a wide variety of applications. However, to avoid developing complex smart contracts which may take long execution and verification times jeopardizing the security of the entire Blockchain network, Ethereum introduced the concept of gas. Every opcode in Ethereum scripting language costs a pre-defined gas. Whenever a transaction makes a function call, the total gas for the function execution is computed and converted into Ether which is the native cryptocurrency of Ethereum. This Ether is charged for the transaction initiator's Ethereum account and transferred to the miner who successfully mined a new block which includes the transaction.

