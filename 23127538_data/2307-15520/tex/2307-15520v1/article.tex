\documentclass[prb, preprint, superscriptaddress, showkeys]{revtex4-2}
\usepackage{amssymb, amsmath}
\usepackage{xr}
\usepackage{graphicx}
\usepackage{soul}
\usepackage{xcolor}

\newcommand{\ebc}[1]{\textcolor{red}{[EB: #1]}}
\newcommand{\eb}[2]{\st{#1}{\textcolor{red}{#2}}}
\newcommand{\mt}[1]{\textcolor{blue}{[MT: #1]}}
\newcommand{\mtm}[2]{\st{#1}{\textcolor{blue}{#2}}}
\newcommand{\plm}[2]{\st{#1}{\textcolor{green}{#2}}}
\newcommand{\ed}[2]{\st{#1}{\textcolor{brown}{#2}}}

\newcommand{\ens}{Laboratoire de Physique de l'Ecole normale sup\'erieure, ENS, Universit\'e PSL, CNRS, Sorbonne Universit\'e, Universit\'e de Paris, 24 rue Lhomond, 75005 Paris, France}
\newcommand{\trt}{Thales Research \& Technology, 91767 Palaiseau, France}

\externaldocument{supplementary-informations}

\begin{document}

\title{Raman Spectroscopy of Monolayer to Bulk $\bf{PtSe_2}$ Exfoliated Crystals}

\author{Marin Tharrault}
\affiliation{\ens}
\author{Eva Desgu\'e}
\affiliation{\trt}
\author{Dominique Carisetti}
\affiliation{\trt}
\author{Bernard Plaçais}
\affiliation{\ens}
\author{Christophe Voisin}
\affiliation{\ens}
\author{Pierre Legagneux}
\affiliation{\trt}
\author{Emmanuel Baudin}
\affiliation{\ens}

\keywords{2D materials, TMD, PtSe2, Raman spectroscopy, intrinsic properties, exfoliation, quality metrics}

\begin{abstract}
Raman spectroscopy is widely used to assess the quality of 2D materials thin films.
This report focuses on $\rm{PtSe_2}$, a noble transition metal dichalcogenide which has the remarkable property to transit from a semi-conductor to a semi-metal with increasing layer number. 
While polycrystalline $\rm{PtSe_2}$ can be grown with various crystalline qualities, getting insight into the monocrystalline intrinsic properties remains challenging. 
We report on the study of exfoliated 1 to 10 layers $\rm{PtSe_2}$ by Raman spectroscopy, featuring record linewidth. The clear Raman signatures allow layer-thickness identification and provides a reference metrics to assess crystal quality of grown films.
\end{abstract}

\maketitle

Transition Metal Dichalcogenides (TMDs) are promising materials for future electronic and optoelectronic devices, owing to their large optical absorption per layer and high electronic mobility \cite{Zhao2016SC_SM, Bonell2022MBE}. Furthermore, they feature strong van der Waals interlayer coupling, resulting in a tunable layer-dependent band structure. Among TMDs, thin films of Platinum Diselenide ($\rm{1T-PtSe_2}$) are semi-conductors and feature exceptional bandgap variations, with a transition to a semi-metal with increasing thickness \cite{Zhao2016SC_SM, Yu2018NtuNatCom, Villaos2019DFT, Kandemir2018RamanDFT}.
For this reason it can reach small bandgap values, permitting efficient photodetection in the infrared range \cite{Zhao2016SC_SM, Yu2018NtuNatCom, Ma2021ElectroExfoPhotodetect}. This makes $\rm{PtSe_2}$ a promising building block for optoelectronic devices operating in the telecom band, and several growth methods (CVD, TAC, MBE) are being developed to provide high-quality scalable films for industry \cite{Yu2018NtuNatCom, Wang2016NanoSheet, Jiang2019TAC, Yan2017MBE, Han2019CvdHorz, Hilse2020Selenization, Bonell2022MBE, Desgue2023MBE}.
These films are commonly characterized using a variety of methods, including electron diffraction, X-ray spectroscopy or diffraction, electron microscopy, atomic force microscopy, electronic transport measurements and optical spectroscopy \cite{Gatensby2014RamanTemXps, Bonell2022MBE}.


Among these methods, Raman spectroscopy presents several key advantages: fast, cheap and non-contact, it probes optical phonon transitions -- highly sensitive to defects -- and is therefore used as a primary characterization to identify the structure and assess film quality \cite{Neumann20152DRamanPeak, Banszerus2017RamanSubstrate, Mignuzzi2015RamanIrradMoS2, Pierce2018IrradGrMoS2Raman}.
Previous works on $\rm{PtSe_2}$ established the most salient Raman spectral features \cite{OBrien2016Raman, Szydowska2020RamanLPE, Gulo2020TempOpticalRaman, Lukas2021TAC, Yin2021RamanTemp, Yasuda2023RamanHelicity, Raczynski2023TempRaman}: they identified the optical vibrational phonon modes associated with the peaks, detailed the Raman peaks intensities and shifts evolution with the thickness, proposed quality metrics, and studied the polarization and temperature dependencies of the Raman peaks. 
However, due to the limited quality of the studied samples, the precise features and evolution of the Raman signature of few-layer intrinsic $\rm{PtSe_2}$ remained out of reach.

 
In this work, we report high-resolution Raman spectroscopy of record-quality exfoliated $\rm{PtSe_2}$ crystals of layer-controlled thickness.
We establish criteria for film quality, and demonstrate that the characteristic Raman signature of thin flakes enables the determination of layer count.
By providing reference Raman spectra of exfoliated single crystals, this work will guide the development of the continuously improving $\rm{PtSe_2}$ growth technology \cite{Jiang2019TAC, Bonell2022MBE, Desgue2023MBE}.
 

\section*{Samples Fabrication}

Chemical Vapour Transport grown $\rm{PtSe_2}$ crystals (HQ Graphene) are exfoliated on fused silica substrates, as shown in \textbf{figure \ref{fig:ramanSpectraPicsIdNLayers}a}. Au-assisted mechanical exfoliation provides flakes as thin as a monolayer thanks to the strong affinity of gold with Selenium atoms \cite{Desdai2016AuExfo, Huang2020AuExfo, Liu2020AuExfo}.

% Figure environment removed

Each flake's layer count is identified using absorption microspectroscopy and confirmed by Atomic Force Microscopy, as described in detail in reference \cite{articlePtSe2absorption}. 

\section*{Atomic Structure Signature} 

High-resolution Raman spectroscopy is performed using a $50\,\rm{\mu W}$ low-power $514\,\rm{nm}$ green laser source, together with a $50\,\rm{cm^{-1}}$ edgepass filter (additional details in experimental section).

Spectra of selected flakes are displayed in figure \ref{fig:ramanSpectraPicsIdNLayers}b and consist in 3 solitary peaks in the $150\,-\,260\,\rm{cm^{-1}}$ spectral range. The Raman signature of each flake is uniform over its whole surface and all thicknesses present similar total scattered Raman intensities.
The full set of spectra is included in the Supplementary Material (SM), figure 1.%\ref{fig:ramanSpectraFull}. 

The spectral shapes correspond to the 1T phase of $\rm{PtSe_2}$, identified with its 4 optical phonon modes: the Raman-active $E_g$ and $A_{1g}$ modes around $180\,\rm{cm^{-1}}$ and $210\,\rm{cm^{-1}}$ respectively, and the IR-active $A_{2u}$ and $E_u$ modes gathered as $LO$ (longitudinal optical) around $230\,\rm{cm^{-1}}$\cite{OBrien2016Raman, Kandemir2018RamanDFT, Fang2019AA_AB_DFT}, as depicted in \textbf{figure \ref{fig:ramanParameters}a}.

% Figure environment removed

Raman modes spectral weights and widths are assessed by a 4-components Lorentzian fit.
Each Lorenzian function $L(\tilde{\nu})$ is parameterized by its shift $\tilde{\nu}_0$, its integrated intensity $I$, and full width at half maximum (thereafter abbreviated as width) $\Gamma$, such that:

\begin{equation}
L(\tilde{\nu}) = \frac{I}{\pi}\frac{\Gamma/2}{(\tilde{\nu}-\tilde{\nu}_0)^2 + (\Gamma / 2)^2}
\end{equation}

Selected fit parameters are gathered in figures \ref{fig:ramanParameters}b-e. More extensive fit data can be found in figure SM2.%\ref{fig:ramanFullParameters}.


From optical absorption spectroscopy, almost all samples are identified as the most-stable AA stacking, as their optical absorption spectra match Density Functional 
Theory (DFT) predictions, as detailed in \cite{articlePtSe2absorption}. 
2L' samples are a noticeable exception, with a singular Raman signature (figure \ref{fig:ramanSpectraPicsIdNLayers}b) differing from the one of 2L. In fact, the comparison of absorption behavior with DFT predictions for 2L' suggests an AB stacking \cite{articlePtSe2absorption}, which has been theoretically identified as a stable phase for $\rm{1T-PtSe_2}$ \cite{Fang2019AA_AB_DFT, Kandemir2018RamanDFT, articlePtSe2absorption}, and is a common defect in grown films \cite{Desgue2023MBE, Xu2021STEMstackAB, Ryu2019STEMstackAB}.


\section*{Crystalline Quality}

In TMDs and graphene, Raman mode linewidth is very sensitive to the defect level
 \cite{Neumann20152DRamanPeak, Banszerus2017RamanSubstrate, Mignuzzi2015RamanIrradMoS2, Pierce2018IrradGrMoS2Raman}. $\rm{PtSe_2}$ is no exception, and the $E_g$ linewidth has been shown to increase with film defectiveness \cite{Lukas2021TAC, Szydowska2020RamanLPE, Jiang2019TAC}. 
For the 1L flakes studied in this work, a narrow $E_g$ peak is associated with the presence of sharp optical features in the absorption spectrum \cite{articlePtSe2absorption}, again indicating the relevance of this Raman spectral signature.

Recent works reported the growth of good quality $\rm{PtSe_2}$ films, using several methods: Thermally Assisted Conversion (TAC) \cite{Jiang2019TAC, Lukas2021TAC}, Chemical Vapor Deposition (CVD) \cite{Xu2019TransportDC, Gulo2020TempOpticalRaman} and Molecular Beam Epitaxy (MBE) \cite{Yan2017MBE, Desgue2023MBE}. Some research focused instead on Mechanichally Exfoliated (ME) $\rm{PtSe_2}$ crystals \cite{Szydowska2020RamanLPE, Das2021TransferedContacts3L, Bae2021PumpProbeExfo}. These works stand out for their narrow $E_g$ peaks (figure \ref{fig:ramanParameters}b), with linewidths values below $6\,\rm{cm^{-1}}$ -- reaching about $3.5\,\rm{cm^{-1}}$ for thick ME, TAC and MBE samples, about $4\,\rm{cm^{-1}}$ for ME and MBE 1L and 2L samples (however the film continuity of the MBE 1L is not established). 

In our work, the high quality of the $\rm{PtSe_2}$ flakes is assessed by the unprecedented narrow linewidth of this $E_g$ mode, from monolayer to bulk thicknesses. The $E_g$ linewidth reaches $4.2\,\rm{cm^{-1}}$ for $\rm{1L}$, $3.9\,\rm{cm^{-1}}$ for $\rm{2L}$ and around $2.5\,\rm{cm^{-1}}$ for thick samples ($\rm{7L}$ to bulk).
Among the $\rm{1L}$ exfoliated flakes studied, several feature large $E_g$ linewidths ($> 6\,\rm{cm^{-1}}$), as shown in figure SM2.%\ref{fig:ramanFullParameters}. 
We could not identify if such flakes were damaged during the exfoliation process, or if they originated from more defective areas of the original CVT crystal.
As reported by the aforementioned works, the $E_g$ peak features of strong increase of its linewidth with decreasing film thickness (figure \ref{fig:ramanParameters}b). This increase is commonly attributed to increasing defectiveness as the film thickness diminishes \cite{OBrien2016Raman}. In our study, it can rather be explained by the residual mechanical stresses produced during exfoliation, which for thinner flakes result in larger strains, and therefore larger linewidths.


Let us now turn our examination from the $E_g$ peak width to its absolute intensity, which appears as a good criterion for film quality. 
Indeed, for a given layer count, the thin exfoliated $\rm{PtSe_2}$ flakes feature a large sample-to-sample variability of their $E_g$ peak integrated intensities (figure \ref{fig:ramanParameters}f for the 1L case), while the $A_{1g}$ and $LO$ modes are unaffected.
We observe that higher $E_g$ peak intensity correlates with narrower linewidth (figure \ref{fig:ramanParameters}g, more thicknesses are studied in figure SM3,%\ref{fig:EgWidthInt}),
 therefore appearing as a robust metric for film quality.  


\section*{Number of Layers Identification}

It is tempting to identify the number of layers from Raman signatures, in particular in $\rm{PtSe_2}$ in which the strong thickness-dependence Raman spectra led several authors to propose metrics relying on the shift of the $E_g$ peak or the ratio of $A_{1g}$ to $E_g$ peak intensity. In this section, we show that these methods are imprecise, yielding inconsistent results, and we propose a new method for exfoliated $\rm{PtSe_2}$. 

The $E_g$ peak position exhibits a quasi-linear blueshift with the inverse number of layers $1\,/\,N_L$. The observed dependency in figure \ref{fig:ramanParameters}c is qualitatively compatible with previous studies, compiled as a grey dashed line and proposed as a metric by reference \cite{Szydowska2020RamanLPE}. However, the discrepancy between this metrics and our data is too large to allow layer count identification.


The $A_{1g}$ to $E_{g}$ peaks intensities is commonly considered as characteristic of the layer count \cite{Shi2019CvdAu, Szydowska2020RamanLPE, Qiu2021PumpProbe}.
Indeed the $A_{1g}$ mode involves out-of-plane motion of external $\rm{Se}$ atoms (figure \ref{fig:ramanParameters}a) and its intensity rises with the thickness, while longitudinal optical $LO$ modes disappear for high layer number \cite{OBrien2016Raman}. These trends are observed as well in our work (figure  \ref{fig:ramanSpectraPicsIdNLayers}b), but appear to be more complex for 1L and 2L. 
Nonetheless, the $A_{1g}$ to $E_g$ intensities ratio does not appear as a reliable signature of the layer count, as it differs significantly from one research to another (figure \ref{fig:ramanParameters}d, references therein). 
Moreover, as described above, we observe for low layer count important variations of the $E_g$ peak intensity resulting in $I_{A_{1g}}/I_{E_g}$ dispersion. 

We propose instead to identify the number of layers by using the $A_{1g}$ and $LO$ peaks which are more robust to the defect level. The $A_{1g}$ to $LO$ intensity ratio for each sample is represented in figure \ref{fig:ramanParameters}e, as a function of inverse number of layers $1/N_L$. One can see straight away  that the thickness can be evaluated from 3 to 10 layers using the linear law displayed in figure \ref{fig:ramanParameters}e:

\begin{equation}
    N_L = \frac{4.93}{I_{LO}\,/\,I_{A_{1g}} + 0.47}
\end{equation} 

Mono- and bi-layers are exceptions to this law, but can be easily identified. Both present similar $A_{1g}$ and $LO$ peaks intensities, and 1L has a remarkably low $A_{1g}$  to $E_g$ peaks intensities ratio, with $I_{A_{1g}}/I_{E_g} < 35\,\%$  with a singularly wide $LO$ peak, while 2L has relatively high ratio $I_{A_{1g}}/I_{E_g} > 45\,\%$, and a sharper $LO$ peak (figure SM2).%\ref{fig:ramanFullParameters}).


Beyond 10 layers, accurate determination of the layer count isn't critical anymore as AFM can provide a fair estimate.


\vspace{2ex}

\section*{Conclusion}

By using exfoliated CVT $\rm{PtSe_2}$ crystals with layer-defined thickness, this study provides reference Raman spectrographs of high-quality flakes, which are particularly important for benchmarking emerging high-quality growth methods.
We showed that crystalline quality can be assessed from the width and height of the $E_g$ mode peak, with $E_g$ linewidth narrowing down to $4.2\,\rm{cm^{-1}}$ for monolayer and $2.5\,\rm{cm^{-1}}$ for thicker exfoliated $\rm{PtSe_2}$. We observed that commonplace criteria for layer count identification, either based on $E_g$ peak shift or on $A_{1g}$ to $E_g$ peaks intensities ratio, leaves much to be desired for exfoliated high-quality $\rm{PtSe_2}$. This led us to propose a robust method in this latter case based on the $A_{1g}$ and $LO$ specific peaks pattern. 


\section*{Experimental Methods}

\subsection*{Samples}
A $60\,\rm{nm}$ gold film is deposited using evaporation on a substrate where the crystals were pre-exfoliated. The samples are then annealed at $150\rm{^\circ C}$ and the gold film is peeled using thermal release tape, thereby detaching few-layer $\rm{PtSe_2}$. The peeled gold film is then transferred to a target substrate, and the gold is removed using $\rm{KI_2}$ etching.

\subsection*{Raman spectrometry}
We use a Raman Qontor spectrometer from Renishaw with a $3000\,\rm{gr/mm}$ grating, a $\times 100$ microscope objective, a $514\,\rm{nm}$ laser source and a $50\,\rm{cm^{-1}}$ edgepass filter. The laser is operated at a power of $50\,\rm{\mu W}$, in order to suppress any thermal shift or broadening of the $\rm{PtSe_2}$ lines (the spectral shift of the main $E_g$ peak was evaluated to reach approximately $0.1\,\rm{cm^{-1}/m W}$). Each spectrum is integrated for $1\,\rm{min}$ and averaged 5 times. 
The spectrometer is calibrated using a $\mathrm{Si_{100}}$ crystal which possesses a Raman peak centered at $520.45\,\mathrm{{cm^{-1}}}$ \cite{Itoh2020RamanSiLine}.

%\bibliographystyle{MSP}
\bibliography{ref}


\end{document}
