\begin{filecontents}{article.aux}
\relax 
\providecommand\hyper@newdestlabel[2]{}
\providecommand\HyperFirstAtBeginDocument{\AtBeginDocument}
\HyperFirstAtBeginDocument{\ifx\hyper@anchor\@undefined
\global\let\oldnewlabel\newlabel
\gdef\newlabel#1#2{\newlabelxx{#1}#2}
\gdef\newlabelxx#1#2#3#4#5#6{\oldnewlabel{#1}{{#2}{#3}}}
\AtEndDocument{\ifx\hyper@anchor\@undefined
\let\newlabel\oldnewlabel
\fi}
\fi}
\global\let\hyper@last\relax 
\gdef\HyperFirstAtBeginDocument#1{#1}
\providecommand*\HyPL@Entry[1]{}
\citation{Zhao2016SC_SM,Bonell2022MBE}
\citation{Zhao2016SC_SM,Yu2018NtuNatCom,Villaos2019DFT,Kandemir2018RamanDFT}
\citation{Zhao2016SC_SM,Yu2018NtuNatCom,Ma2021ElectroExfoPhotodetect}
\citation{Yu2018NtuNatCom,Wang2016NanoSheet,Jiang2019TAC,Yan2017MBE,Han2019CvdHorz,Hilse2020Selenization,Bonell2022MBE,Desgue2023MBE}
\citation{Gatensby2014RamanTemXps,Bonell2022MBE}
\citation{Neumann20152DRamanPeak,Banszerus2017RamanSubstrate,Mignuzzi2015RamanIrradMoS2,Pierce2018IrradGrMoS2Raman,Stenger2017RamanHBn}
\citation{OBrien2016Raman,Szydowska2020RamanLPE,Gulo2020TempOpticalRaman,Lukas2021TAC,Yin2021RamanTemp,Yasuda2023RamanHelicity,Raczynski2023TempRaman}
\HyPL@Entry{0<</P(\376\377\0001)>>}
\newlabel{FirstPage}{{}{1}{}{section*.1}{}}
\@writefile{toc}{\contentsline {title}{Raman Spectroscopy of Monolayer to Bulk PtSe\textsubscript  {2} Exfoliated Crystals}{1}{section*.2}\protected@file@percent }
\citation{Jiang2019TAC,Bonell2022MBE,Desgue2023MBE}
\HyPL@Entry{1<</P(\376\377\0002)>>}
\@writefile{toc}{\contentsline {abstract}{Abstract}{2}{section*.1}\protected@file@percent }
\citation{HQGraphenePtSe2}
\citation{Desdai2016AuExfo,Huang2020AuExfo,Liu2020AuExfo}
\citation{articlePtSe2absorption}
\HyPL@Entry{2<</P(\376\377\0003)>>}
\@writefile{toc}{\contentsline {section}{\numberline {}Samples Fabrication}{3}{section*.3}\protected@file@percent }
\@writefile{lof}{\contentsline {figure}{\numberline {1}{\ignorespaces $\rm  {PtSe_2}$ exfoliated flakes and their Raman spectra. (a) Micrographs of $\rm  {PtSe_2}$ flakes, each region of interest is labeled with its layer count. (b) The associated Raman spectra displayed on the $150\,-\,260\,\rm  {cm^{-1}}$ range, and normalized to their maximum amplitude. }}{3}{figure.1}\protected@file@percent }
\newlabel{fig:ramanSpectraPicsIdNLayers}{{1}{3}{$\rm {PtSe_2}$ exfoliated flakes and their Raman spectra. (a) Micrographs of $\rm {PtSe_2}$ flakes, each region of interest is labeled with its layer count. (b) The associated Raman spectra displayed on the $150\,-\,260\,\rm {cm^{-1}}$ range, and normalized to their maximum amplitude}{figure.1}{}}
\citation{articlePtSe2absorption}
\citation{OBrien2016Raman,Kandemir2018RamanDFT,Fang2019AA_AB_DFT}
\citation{Szydowska2020RamanLPE}
\citation{Szydowska2020RamanLPE,Das2021TransferedContacts3L,Bae2021PumpProbeExfo,Jiang2019TAC,Lukas2021TAC,Xu2019TransportDC,Gulo2020TempOpticalRaman,Yan2017MBE,Desgue2023MBE}
\citation{articlePtSe2absorption}
\citation{Szydowska2020RamanLPE}
\citation{Szydowska2020RamanLPE,Das2021TransferedContacts3L,Bae2021PumpProbeExfo,Jiang2019TAC,Lukas2021TAC,Xu2019TransportDC,Gulo2020TempOpticalRaman,Yan2017MBE,Desgue2023MBE}
\citation{articlePtSe2absorption}
\citation{articlePtSe2absorption}
\citation{articlePtSe2absorption}
\citation{Fang2019AA_AB_DFT,Kandemir2018RamanDFT,articlePtSe2absorption}
\citation{Desgue2023MBE,Xu2021STEMstackAB,Ryu2019STEMstackAB}
\citation{Rokni2020WaterLayer}
\HyPL@Entry{3<</P(\376\377\0004)>>}
\@writefile{toc}{\contentsline {section}{\numberline {}Atomic structure signature}{4}{section*.4}\protected@file@percent }
\citation{Neumann20152DRamanPeak,Banszerus2017RamanSubstrate,Mignuzzi2015RamanIrradMoS2,Pierce2018IrradGrMoS2Raman}
\citation{Lukas2021TAC,Szydowska2020RamanLPE,Jiang2019TAC}
\citation{articlePtSe2absorption}
\HyPL@Entry{4<</P(\376\377\0005)>>}
\@writefile{lof}{\contentsline {figure}{\numberline {2}{\ignorespaces Evolution of Raman spectra characteristics with layer count and film quality. (a) Decomposition of the 4-components Lorentzian fit into three contributions : $E_g$, $A_{1g}$ and $LO$ modes, and schematic of atomic motions for $E_g$ and $A_{1g}$ modes. (b) $E_g$ peak width and (c) shift (the dashed line is a law derived from different studies in reference \cite  {Szydowska2020RamanLPE}). (d) Intensities ratio of $A_{1g}$ to $E_g$ modes and (e) $LO$ to $A_{1g}$ modes. Errorbars are included, displaying one standard deviation fitting uncertainty. The empty diamonds incorporate data not shown in figure \ref {fig:ramanSpectraPicsIdNLayers}, and the dashed line is a linear fit for $3$ to $10$ layers. For the sake of comparison, in (b) and (d) we reproduce digitalized Raman data from the literature \cite  {Szydowska2020RamanLPE, Das2021TransferedContacts3L, Bae2021PumpProbeExfo, Jiang2019TAC, Lukas2021TAC, Xu2019TransportDC, Gulo2020TempOpticalRaman, Yan2017MBE, Desgue2023MBE}, performed with green $532\,\rm  {nm}$ or $514\,\rm  {nm}$ light and where layer count is determined from film thickness considering an individual layer thickness of $0.5\,\rm  {nm}$ \cite  {articlePtSe2absorption}. 2L' is displayed in lighter blue to highlight the difference with other samples. }}{5}{figure.2}\protected@file@percent }
\newlabel{fig:ramanParameters}{{2}{5}{Evolution of Raman spectra characteristics with layer count and film quality. (a) Decomposition of the 4-components Lorentzian fit into three contributions : $E_g$, $A_{1g}$ and $LO$ modes, and schematic of atomic motions for $E_g$ and $A_{1g}$ modes. (b) $E_g$ peak width and (c) shift (the dashed line is a law derived from different studies in reference \cite {Szydowska2020RamanLPE}). (d) Intensities ratio of $A_{1g}$ to $E_g$ modes and (e) $LO$ to $A_{1g}$ modes. Errorbars are included, displaying one standard deviation fitting uncertainty. The empty diamonds incorporate data not shown in figure \ref {fig:ramanSpectraPicsIdNLayers}, and the dashed line is a linear fit for $3$ to $10$ layers. For the sake of comparison, in (b) and (d) we reproduce digitalized Raman data from the literature \cite {Szydowska2020RamanLPE, Das2021TransferedContacts3L, Bae2021PumpProbeExfo, Jiang2019TAC, Lukas2021TAC, Xu2019TransportDC, Gulo2020TempOpticalRaman, Yan2017MBE, Desgue2023MBE}, performed with green $532\,\rm {nm}$ or $514\,\rm {nm}$ light and where layer count is determined from film thickness considering an individual layer thickness of $0.5\,\rm {nm}$ \cite {articlePtSe2absorption}. 2L' is displayed in lighter blue to highlight the difference with other samples}{figure.2}{}}
\@writefile{toc}{\contentsline {section}{\numberline {}Crystalline Quality}{5}{section*.5}\protected@file@percent }
\citation{Jiang2019TAC,Lukas2021TAC}
\citation{Xu2019TransportDC,Gulo2020TempOpticalRaman}
\citation{Yan2017MBE,Desgue2023MBE}
\citation{Szydowska2020RamanLPE,Das2021TransferedContacts3L,Bae2021PumpProbeExfo}
\citation{OBrien2016Raman}
\citation{YuCardona2010SC,Miller2019FrohlichExciton}
\HyPL@Entry{5<</P(\376\377\0006)>>}
\@writefile{lof}{\contentsline {figure}{\numberline {3}{\ignorespaces $E_g$ peak variability of monolayer samples. (a) Raman spectra, and (b) their $E_g$ peak intensity versus width -- the trend is given by a linear fit (dashed line). In both figures, the color represents the measured $E_g$ linewidth. }}{6}{figure.3}\protected@file@percent }
\newlabel{fig:EgMode}{{3}{6}{$E_g$ peak variability of monolayer samples. (a) Raman spectra, and (b) their $E_g$ peak intensity versus width -- the trend is given by a linear fit (dashed line). In both figures, the color represents the measured $E_g$ linewidth}{figure.3}{}}
\citation{Szydowska2020RamanLPE}
\citation{Shi2019CvdAu,Szydowska2020RamanLPE,Qiu2021PumpProbe}
\citation{OBrien2016Raman}
\HyPL@Entry{6<</P(\376\377\0007)>>}
\@writefile{toc}{\contentsline {section}{\numberline {}Number of Layers Identification}{7}{section*.6}\protected@file@percent }
\citation{Itoh2020RamanSiLine}
\HyPL@Entry{7<</P(\376\377\0008)>>}
\@writefile{toc}{\contentsline {section}{\numberline {}Conclusion}{8}{section*.7}\protected@file@percent }
\@writefile{toc}{\contentsline {section}{\numberline {}Experimental Methods}{8}{section*.8}\protected@file@percent }
\@writefile{toc}{\contentsline {subsection}{\numberline {}Samples}{8}{section*.9}\protected@file@percent }
\bibdata{outputNotes,ref}
\bibcite{Zhao2016SC_SM}{{1}{2017}{{Zhao\ \emph  {et~al.}}}{{Zhao, Qiao, Yu, Yu, Xu, Lau, Zhou, Liu, Wang, Ji,\ and\ Chai}}}
\bibcite{Bonell2022MBE}{{2}{2021}{{Bonell\ \emph  {et~al.}}}{{Bonell, Marty, Vergnaud, Consonni, Okuno, Ouerghi, Boukari,\ and\ Jamet}}}
\bibcite{Yu2018NtuNatCom}{{3}{2018}{{Yu\ \emph  {et~al.}}}{{Yu, Yu, Wu, Singh, Zeng, Lin, Zhou, Lin, Suenaga, Liu,\ and\ Wang}}}
\bibcite{Villaos2019DFT}{{4}{2019}{{Villaos\ \emph  {et~al.}}}{{Villaos, Crisostomo, Huang, Huang, Padama, Albao, Lin,\ and\ Chuang}}}
\HyPL@Entry{8<</P(\376\377\0009)>>}
\@writefile{toc}{\contentsline {subsection}{\numberline {}Raman spectrometry}{9}{section*.10}\protected@file@percent }
\@writefile{toc}{\contentsline {section}{\numberline {}Acknowledgements}{9}{section*.11}\protected@file@percent }
\@writefile{toc}{\contentsline {section}{\numberline {}Data availability}{9}{section*.12}\protected@file@percent }
\@writefile{toc}{\contentsline {section}{\numberline {}References}{9}{section*.13}\protected@file@percent }
\bibcite{Kandemir2018RamanDFT}{{5}{2018}{{Kandemir\ \emph  {et~al.}}}{{Kandemir, Akbali, Kahraman, Badalov, Ozcan, Iyikanat,\ and\ Sahin}}}
\bibcite{Ma2021ElectroExfoPhotodetect}{{6}{2021}{{Ma\ \emph  {et~al.}}}{{Ma, Shao, Li, Dong, Hu, Zhou, Xu, Zhao, Fang, Li, Li, Wu, Zhao, Pennycook, Sow, Lee, Zhong, Lu, Ding, Wang, Li,\ and\ Lu}}}
\bibcite{Wang2016NanoSheet}{{7}{2016}{{Wang\ \emph  {et~al.}}}{{Wang, Li, Besenbacher,\ and\ Dong}}}
\bibcite{Jiang2019TAC}{{8}{2019}{{Jiang\ \emph  {et~al.}}}{{Jiang, Wang, Chen, Wu, Ba, Xuan, Sun, Gong, Bao, Shen, Lin, Meng, Wang,\ and\ Sun}}}
\bibcite{Yan2017MBE}{{9}{2017}{{Yan\ \emph  {et~al.}}}{{Yan, Wang, Zhou, Zhang, Zhang, Zhang, Yao, Lu, Yang, Wu, Yoshikawa, Miyamoto, Okuda, Wu, Yu, Duan,\ and\ Zhou}}}
\bibcite{Han2019CvdHorz}{{10}{2019}{{Han\ \emph  {et~al.}}}{{Han, Kim, Noh, Kim, Ji, Kwon, Yu, Ko, Okogbue, Oh, Chung, Jung, Lee,\ and\ Jung}}}
\bibcite{Hilse2020Selenization}{{11}{2020}{{Hilse\ \emph  {et~al.}}}{{Hilse, Wang,\ and\ Engel-Herbert}}}
\bibcite{Desgue2023MBE}{{12}{2023}{{Desgué\ \emph  {et~al.}}}{{Desgué, Legagneux \emph  {et~al.}}}}
\bibcite{Gatensby2014RamanTemXps}{{13}{2014}{{Gatensby\ \emph  {et~al.}}}{{Gatensby, McEvoy, Lee, Hallam, Berner, Rezvani, Winters, O’Brien,\ and\ Duesberg}}}
\bibcite{Neumann20152DRamanPeak}{{14}{2015}{{Neumann\ \emph  {et~al.}}}{{Neumann, Reichardt, Venezuela, Dr{\"o}geler, Banszerus, Schmitz, Watanabe, Taniguchi, Mauri, Beschoten, Rotkin,\ and\ Stampfer}}}
\HyPL@Entry{9<</P(\376\377\0001\0000)>>}
\bibcite{Banszerus2017RamanSubstrate}{{15}{2017}{{Banszerus\ \emph  {et~al.}}}{{Banszerus, Janssen, Otto, Epping, Taniguchi, Watanabe, Beschoten, Neumaier,\ and\ Stampfer}}}
\bibcite{Mignuzzi2015RamanIrradMoS2}{{16}{2015}{{Mignuzzi\ \emph  {et~al.}}}{{Mignuzzi, Pollard, Bonini, Brennan, Gilmore, Pimenta, Richards,\ and\ Roy}}}
\bibcite{Pierce2018IrradGrMoS2Raman}{{17}{2018}{{Maguire\ \emph  {et~al.}}}{{Maguire, Fox, Zhou, Wang, O'Brien, Jadwiszczak, Cullen, McManus, Bateman, McEvoy, Duesberg,\ and\ Zhang}}}
\bibcite{Stenger2017RamanHBn}{{18}{2017}{{Stenger\ \emph  {et~al.}}}{{Stenger, Schué, Boukhicha, Berini, Plaçais, Loiseau,\ and\ Barjon}}}
\bibcite{OBrien2016Raman}{{19}{2016}{{O’Brien\ \emph  {et~al.}}}{{O’Brien, McEvoy, Motta, Zheng, Berner, Kotakoski, Elibol, Pennycook, Meyer, Yim, Abid, Hallam, Donegan, Sanvito,\ and\ Duesberg}}}
\bibcite{Szydowska2020RamanLPE}{{20}{2020}{{Szydłowska\ \emph  {et~al.}}}{{Szydłowska, Hartwig, Tywoniuk, Hartman, Stimpel-Lindner, Sofer, McEvoy, Duesberg,\ and\ Backes}}}
\bibcite{Gulo2020TempOpticalRaman}{{21}{2020}{{Gulo\ \emph  {et~al.}}}{{Gulo, Yeh, Chang,\ and\ Liu}}}
\bibcite{Lukas2021TAC}{{22}{2021}{{Lukas\ \emph  {et~al.}}}{{Lukas, Hartwig, Prechtl, Capraro, Bolten, Meledin, Mayer, Neumaier, Kataria, Duesberg,\ and\ Lemme}}}
\bibcite{Yin2021RamanTemp}{{23}{2021}{{Yin\ \emph  {et~al.}}}{{Yin, Zhang, Tan, Chen, Chen, Li, Zhang, Zhang, Wang,\ and\ Li}}}
\bibcite{Yasuda2023RamanHelicity}{{24}{2023}{{Yasuda\ \emph  {et~al.}}}{{Yasuda, Kawada, Matsumoto, Kawaguchi,\ and\ Hayashi}}}
\bibcite{Raczynski2023TempRaman}{{25}{2023}{{Raczy{\'n}ski\ \emph  {et~al.}}}{{Raczy{\'n}ski, Nowak, Nowicki, El-Ahmar, Szybowicz,\ and\ Koczorowski}}}
\HyPL@Entry{10<</P(\376\377\0001\0001)>>}
\bibcite{HQGraphenePtSe2}{{26}{2021}{{HQG}}{{}}}
\bibcite{Desdai2016AuExfo}{{27}{2016}{{Desai\ \emph  {et~al.}}}{{Desai, Madhvapathy, Amani, Kiriya, Hettick, Tosun, Zhou, Dubey, Ager~III, Chrzan,\ and\ Javey}}}
\bibcite{Huang2020AuExfo}{{28}{2020}{{Huang\ \emph  {et~al.}}}{{Huang, Pan, Yang, Bao, Meng, Luo, Cai, Liu, Zhao, Zhou, Wu, Zhu, Huang, Liu, Liu, Cheng, Wu, Tian, Gu, Shi, Guo, Cheng, Hu, Zhao, Yang, Sutter, Sutter, Wang, Ji, Zhou,\ and\ Gao}}}
\bibcite{Liu2020AuExfo}{{29}{2020}{{Liu\ \emph  {et~al.}}}{{Liu, Wu, Bai, Chae, Li, Wang, Hone,\ and\ Zhu}}}
\bibcite{articlePtSe2absorption}{{30}{2023}{{Tharrault\ \emph  {et~al.}}}{{Tharrault, Ayari, Desgué, Arfaoui, Goff, Morfin, Palomo, Rosticher, Jaziri, Plaçais, Legagneux, Carosella, Voisin, Ferreira,\ and\ Baudin}}}
\bibcite{Fang2019AA_AB_DFT}{{31}{2019}{{Fang\ \emph  {et~al.}}}{{Fang, Liang, Feng,\ and\ Luo}}}
\bibcite{Das2021TransferedContacts3L}{{32}{2021}{{Das\ \emph  {et~al.}}}{{Das, Yang, Seo, Kim, Park, Kim, Seo, Kwak,\ and\ Chang}}}
\bibcite{Bae2021PumpProbeExfo}{{33}{2021}{{Bae\ \emph  {et~al.}}}{{Bae, Nah, Lee, Sajjad, Singh, Kang, Kim, Kim, Kim, Baik, Lee,\ and\ Sim}}}
\bibcite{Xu2019TransportDC}{{34}{2019}{{Xu\ \emph  {et~al.}}}{{Xu, Zhang, Liu, Zhang, Sun, Guo, Sheng, Wang, Luo, Wu, Wang, Hu, Xu, Sun, Zhou, Shi, Sun, Zhang,\ and\ Bao}}}
\HyPL@Entry{11<</P(\376\377\0001\0002)>>}
\bibcite{Xu2021STEMstackAB}{{35}{2021}{{Xu\ \emph  {et~al.}}}{{Xu, Liu, Song, Li, Li, Li, Wang, Bai,\ and\ Qi}}}
\bibcite{Ryu2019STEMstackAB}{{36}{2019}{{Ryu\ \emph  {et~al.}}}{{Ryu, Chen, Wen,\ and\ Warner}}}
\bibcite{Rokni2020WaterLayer}{{37}{2020}{{Rokni\ and\ Lu}}{{}}}
\bibcite{YuCardona2010SC}{{38}{2010}{{Yu\ and\ Cardona}}{{}}}
\bibcite{Miller2019FrohlichExciton}{{39}{2019}{{Miller\ \emph  {et~al.}}}{{Miller, Lindlau, Bommert, Neumann, Yamaguchi, Holleitner, H{\"o}gele,\ and\ Wurstbauer}}}
\bibcite{Shi2019CvdAu}{{40}{2019}{{Shi\ \emph  {et~al.}}}{{Shi, Huan, Hong, Xu, Yang, Zhang, Zou,\ and\ Zhang}}}
\bibcite{Qiu2021PumpProbe}{{41}{2020}{{Qiu\ \emph  {et~al.}}}{{Qiu, Liang, Guo, Fang, Li, Feng,\ and\ Luo}}}
\bibcite{Itoh2020RamanSiLine}{{42}{2020}{{Itoh\ and\ Shirono}}{{}}}
\bibstyle{apsrev4-2}
\citation{REVTEX42Control}
\citation{apsrev42Control}
\HyPL@Entry{12<</P(\376\377\0001\0003)>>}
\newlabel{LastBibItem}{{42}{13}{}{section*.13}{}}
\newlabel{LastPage}{{}{13}{}{}{}}
\gdef \@abspage@last{13}
\end{filecontents}

\begin{filecontents}{supplementary-material.aux}
\relax 
\providecommand\hyper@newdestlabel[2]{}
\providecommand\HyperFirstAtBeginDocument{\AtBeginDocument}
\HyperFirstAtBeginDocument{\ifx\hyper@anchor\@undefined
\global\let\oldnewlabel\newlabel
\gdef\newlabel#1#2{\newlabelxx{#1}#2}
\gdef\newlabelxx#1#2#3#4#5#6{\oldnewlabel{#1}{{#2}{#3}}}
\AtEndDocument{\ifx\hyper@anchor\@undefined
\let\newlabel\oldnewlabel
\fi}
\fi}
\global\let\hyper@last\relax 
\gdef\HyperFirstAtBeginDocument#1{#1}
\providecommand*\HyPL@Entry[1]{}
\bibdata{outputNotes}
\bibstyle{apsrev4-2}
\citation{REVTEX42Control}
\citation{apsrev42Control}
\HyPL@Entry{0<</P(\376\377\0001)>>}
\newlabel{FirstPage}{{}{1}{}{Doc-Start}{}}
\@writefile{toc}{\contentsline {title}{Raman Spectroscopy of Monolayer to Bulk PtSe\textsubscript  {2} Exfoliated Crystals - Supplementary Material}{1}{section*.1}\protected@file@percent }
\HyPL@Entry{1<</P(\376\377\0002)>>}
\@writefile{lof}{\contentsline {figure}{\numberline {1}{\ignorespaces Raw Raman spectra where the baseline has been substracted. Each spectrum originates from a different flake (for a total of 32 flakes). }}{2}{figure.1}\protected@file@percent }
\newlabel{fig:ramanSpectraFull}{{1}{2}{Raw Raman spectra where the baseline has been substracted. Each spectrum originates from a different flake (for a total of 32 flakes)}{figure.1}{}}
\HyPL@Entry{2<</P(\376\377\0003)>>}
\@writefile{lof}{\contentsline {figure}{\numberline {2}{\ignorespaces Raman spectra peaks parameters. Left panel: selected spectra displayed in main text fig. \ref {fig:ramanSpectraPicsIdNLayers} with the fit decomposition in $\rm  {E_{g}}$, $\rm  {A_{1g}}$ and $LO$ contributions in respectively red, green and blue. Right panel: The fit parameters extracted for all the measured spectra: peaks shifts, intensities, intensities ratios and full-width at half maximum (the $LO$ peak shift refers to the position of the LO maximum amplitude). The selected spectra parameters are displayed as diamonds, and others as circles. Only well-fitted parameters are represented on these plots. Each datapoint is displaying a Raman mode parameter originating from a different flake. The fit is performed by attributing a $1/\sqrt  {A}$ weight to each data point ($A$ being the amplitude), to account for photon shot noise. }}{3}{figure.2}\protected@file@percent }
\newlabel{fig:ramanFullParameters}{{2}{3}{Raman spectra peaks parameters. Left panel: selected spectra displayed in main text fig. \ref {fig:ramanSpectraPicsIdNLayers} with the fit decomposition in $\rm {E_{g}}$, $\rm {A_{1g}}$ and $LO$ contributions in respectively red, green and blue. Right panel: The fit parameters extracted for all the measured spectra: peaks shifts, intensities, intensities ratios and full-width at half maximum (the $LO$ peak shift refers to the position of the LO maximum amplitude). The selected spectra parameters are displayed as diamonds, and others as circles. Only well-fitted parameters are represented on these plots. Each datapoint is displaying a Raman mode parameter originating from a different flake. The fit is performed by attributing a $1/\sqrt {A}$ weight to each data point ($A$ being the amplitude), to account for photon shot noise}{figure.2}{}}
\HyPL@Entry{3<</P(\376\377\0004)>>}
\@writefile{lof}{\contentsline {figure}{\numberline {3}{\ignorespaces Raman spectroscopy mappings of 1L, 2L, 2L', 3L and 7L flakes. The $E_g$ mode shift, intensity and linewidth are extracted using the 4-components Lorentzian fit detailed in the main text. These quantities exhibit significant variations on the edges of the flakes, while they remain relatively uniform across the surface (within the fit standard deviation error, represented on top of each scale bar). For these measurements, a Raman Invia spectrometer (Renishaw) is used, with a $\sim 50\,\rm  {\mu W}$ $532\,\rm  {nm}$ laser, $30\,\rm  {s}$ integration time and a $1800\,\rm  {gr/mm}$ grating. }}{4}{figure.3}\protected@file@percent }
\newlabel{fig:ramanMaps}{{3}{4}{Raman spectroscopy mappings of 1L, 2L, 2L', 3L and 7L flakes. The $E_g$ mode shift, intensity and linewidth are extracted using the 4-components Lorentzian fit detailed in the main text. These quantities exhibit significant variations on the edges of the flakes, while they remain relatively uniform across the surface (within the fit standard deviation error, represented on top of each scale bar). For these measurements, a Raman Invia spectrometer (Renishaw) is used, with a $\sim 50\,\rm {\mu W}$ $532\,\rm {nm}$ laser, $30\,\rm {s}$ integration time and a $1800\,\rm {gr/mm}$ grating}{figure.3}{}}
\HyPL@Entry{4<</P(\376\377\0005)>>}
\@writefile{lof}{\contentsline {figure}{\numberline {4}{\ignorespaces Raman signatures of bilayer flakes, sorted as 2L (left) and 2L' (right). In dashed lines are represented the mean values of the 2L and 2L' spectra, highlighting that all bilayer samples can indeed be classified into these two categories.}}{5}{figure.4}\protected@file@percent }
\newlabel{fig:ramanBilayers}{{4}{5}{Raman signatures of bilayer flakes, sorted as 2L (left) and 2L' (right). In dashed lines are represented the mean values of the 2L and 2L' spectra, highlighting that all bilayer samples can indeed be classified into these two categories}{figure.4}{}}
\@writefile{lof}{\contentsline {figure}{\numberline {5}{\ignorespaces Raman spectroscopy of 1L, 2L and 3L flakes on crystalline sapphire and on fused silica substrates. (left) Scattered Raman intensity, normalized by the $E_g$ mode integrated intensity, and the (right) $E_g$ mode shift and linewidth. A total of 8 flakes deposited on sapphire and 11 flakes deposited on fused silica have been inspected, respectively in red and blue. Raman signatures, $E_g$ shifts and linewidths appear to be very similar. For these measurements, a Raman Invia spectrometer (Renishaw) is used, with a $\sim 50\,\rm  {\mu W}$ $532\,\rm  {nm}$ laser, $150\,\rm  {s}$ integration time and a $1800\,\rm  {gr/mm}$ grating.}}{5}{figure.5}\protected@file@percent }
\newlabel{fig:ramanSapphire}{{5}{5}{Raman spectroscopy of 1L, 2L and 3L flakes on crystalline sapphire and on fused silica substrates. (left) Scattered Raman intensity, normalized by the $E_g$ mode integrated intensity, and the (right) $E_g$ mode shift and linewidth. A total of 8 flakes deposited on sapphire and 11 flakes deposited on fused silica have been inspected, respectively in red and blue. Raman signatures, $E_g$ shifts and linewidths appear to be very similar. For these measurements, a Raman Invia spectrometer (Renishaw) is used, with a $\sim 50\,\rm {\mu W}$ $532\,\rm {nm}$ laser, $150\,\rm {s}$ integration time and a $1800\,\rm {gr/mm}$ grating}{figure.5}{}}
\HyPL@Entry{5<</P(\376\377\0006)>>}
\@writefile{lof}{\contentsline {figure}{\numberline {6}{\ignorespaces $E_g$ peak intensity correlation with its linewidth, for $\rm  {1L}$, $\rm  {2L}$, $\rm  {3L}$ -- and $\rm  {2L'}$ bilayer flakes conjectured to originate from AB staking. The dashed lines are fitted, indicating the global trend. Each datapoint is originating from a different flake. }}{6}{figure.6}\protected@file@percent }
\newlabel{fig:EgWidthInt}{{6}{6}{$E_g$ peak intensity correlation with its linewidth, for $\rm {1L}$, $\rm {2L}$, $\rm {3L}$ -- and $\rm {2L'}$ bilayer flakes conjectured to originate from AB staking. The dashed lines are fitted, indicating the global trend. Each datapoint is originating from a different flake}{figure.6}{}}
\newlabel{LastPage}{{}{6}{}{}{}}
\gdef \@abspage@last{6}
\end{filecontents}