%\documentclass[reprint,aps,pre,floatfix,onecolumn,fleqn,dvipdfmx]{revtex4-2}
\documentclass[reprint,aps,pre,floatfix,onecolumn,fleqn,superscriptaddress]{revtex4-2}
\usepackage{amsmath,amssymb}
\usepackage{graphicx}
\usepackage{xr}
\externaldocument{marui2}
\usepackage{natbib}

%図番号をS~に
\renewcommand{\thefigure}{S\arabic{figure}}
%式番号をS~に
\renewcommand{\theequation}{S\arabic{equation}}

%graphicsの読み込みを無効化
%\renewcommand{\includegraphics}[2][]{\fbox{figure here}}  


\begin{document}
\preprint{APS/123-QED}

\title{Supplemental Material: Erosion of synchronization and its prevention among noisy oscillators with simplicial interactions}
\author{Yuichiro Marui}
\affiliation{Department of Mathematical Informatics, Graduate School of Information Science and Technology}
\author{Hiroshi Kori}
\email[]{kori@k.u-tokyo.ac.jp}
\affiliation{Department of Mathematical Informatics, Graduate School of Information Science and Technology}
\affiliation{Department of Complexity Science and Engineering, The University of Tokyo}
\date{\today}

\maketitle


\section{Derivation of the stationary distribution $P_{\rm s}$}
Substituting $\Theta=0$ into Eq.~\eqref{eq: FPE}, we obtain the following Fokker-Planck equation:
\begin{equation}
  \label{eq: FPE reduced}
  \cfrac{\partial P(\theta, t)}{\partial t} = \cfrac{\partial}{\partial \theta} [\{K_1 R_1 \sin \theta + K_2 R_1^2 \sin 2 \theta\}P] + D \cfrac{\partial^2 P}{\partial \theta^2}.
\end{equation}
The stationary distribution $P_{\mathrm{s}}(\theta)$ is found as a solution to $\partial_t P = 0$. The general solution is
%\begin{align}
%  \cfrac{\partial}{\partial \theta} \{ (K_1 R_1 \sin \theta + K_2 R_1^2 \sin 2 \theta)P \} + D \cfrac{\partial^2 P}{\partial \theta^2} = 0.
%  \label{eq: stationary 1}
%\end{align}
%The solution to Eq. \eqref{eq: stationary 1} is
\begin{align}
  P_{\mathrm{s}}(\theta) = \exp \left(\cfrac{2 K_1 R \cos \theta + K_2 R^2 \cos 2 \theta}{2 D}\right) \left[ c_1 + c_2 \int_0^{\theta} \exp \left(-\cfrac{2 K_1 R \cos y + K_2 R^2 \cos 2 y}{2 D}\right) \mathrm{d}y \right],
\end{align}
where $c_1$ and $c_2$ are constants.
Because $P_{\mathrm{s}} (\theta) = P_{\mathrm{s}} (\theta + 2 \pi)$ for any $\theta$, $c_2$ vanishes.
From the normalization condition, $c_1$ is given as
%Therefore, we obtain
%\begin{align}
%  \label{eq: stationary distribution appendix}
%  P_{\mathrm{s}}(\theta) = c_1 \exp\left( \cfrac{2 K_1 R_1 \cos \theta + K_2 R_1^2 \cos 2 \theta}{2D} \right),
%\end{align}
%where $c_1$ is obtained by the normalizing constant:
\begin{align}
  c_1 = \cfrac{1}{\displaystyle\int_0^{2 \pi} \exp \left(\cfrac{2 K_1 R_1 \cos y + K_2 R_1^2 \cos 2y}{2 D}\right) \mathrm{d}y}.
  \label{constant 2}
\end{align}


%\section{Bifurcation analysis}
\section{Derivation of Eq. (\ref{amplitude eq})}
We expand the distribution $P(\theta, t)$ into the Fourier series:
\begin{align}
  \label{eq: Fourier series}
  P(\theta, t) = \cfrac{1}{2 \pi} + \cfrac{1}{2 \pi} \sum_{l \ne 0} P_l (t) e^{il \theta},
\end{align}
where $P_l= \bar P_{-l}$. Note that $P_{-1}=Z$ and $P_{1}=\bar Z$.
%The Fokker-Planck equation \eqref{eq: FPE}
Using $P_{\pm 1}$, Eq.~\eqref{eq: FPE reduced} can be rewritten as
\begin{align}
  \label{eq: FPE 2}
  \cfrac{\partial P}{\partial t} &= \cfrac{K}{2i} 
  \cfrac{\partial}{\partial \theta}\ [\{P_1 \exp(i \theta) - P_{-1} \exp(-i\theta)
%   \notag \\
   + P_1^2 \exp(2 i \theta) - P_{-1}^2 \exp(-2i\theta)
  \} P] + D \cfrac{\partial^2 P}{\partial \theta^2},
\end{align}
By further substituting Eq.~\eqref{eq: Fourier series} into Eq.~\eqref{eq: FPE 2}, we obtain
\begin{align}
  \label{eq: FPE Fourier l=1}
  \cfrac{\mathrm{d} P_{-1}}{\mathrm{d} t} &= - D P_{-1} + \cfrac{K_1}{2} (P_{-1} - P_{-2} P_{1})  %\notag \\
  + \cfrac{K_2}{2} (P_1 P_{-1}^2 - P_{-3} P_1^2),\\
  \label{eq: FPE Fourier l=2}
  \cfrac{\mathrm{d} P_{-2}}{\mathrm{d} t} &= - 4 D P_{-2} + K_1 (P_{-1}^2 - P_{-3} P_1) %\\
  + K_2 (P_{-1}^2 - P_{-4} P_1^2),\\
  \label{eq: FPE Fourier}
  \cfrac{\mathrm{d} P_l}{\mathrm{d} t} &= - l^2 D P_l + \cfrac{l K_1}{2} (P_{l - 1} P_1 - P_{l + 1} P_{-1}) %\notag \\
  + \cfrac{l K_2}{2}(P_{l - 2} P_1^2 - P_{l + 2} P_{-1}^2)\ \  \mbox{for $l \ne \pm 1, \pm 2$}.
\end{align}
Note that $P_1$ and $P_2$ obey the complex conjugate of the right hand side of Eq.~\eqref{eq: FPE Fourier l=1} and Eq.~\eqref{eq: FPE Fourier l=2}, respectively. 
%for $l = -1, -2$ we have complex conjugate as $l = 1, 2$, respectively.
%[$P_0=\frac{1}{2\pi}$を定義しておけば3番目の式だけで十分?]

We assume that the state $Z=0$ bifurcates at $K_1=K_{\rm c}$ and set
%Let $K_1$ a control parameter and $\mu$ be a bifurcation parameter:
\begin{align}
  \label{eq: bifurcation parameter}
  K_1 = K_c(1 + \mu),
\end{align}
where $\mu$ is the bifurcation parameter.
We define $\varepsilon$ by $\varepsilon = \sqrt{|\mu|}$ and introduce a scaled time $\tau$, by $\tau = \varepsilon^2 t$.
The time derivative then transformed as
\begin{align}
  \label{eq: time differentiation}
  \cfrac{\mathrm{d}}{\mathrm{d}t} \rightarrow \cfrac{\partial}{\partial t} + \varepsilon^2 \cfrac{\partial}{\partial \tau}.
\end{align}
We expand the Fourier series $P_l(t)$ into $P_{l, \nu}(t, \tau)$ as
\begin{align}
  \label{eq: expansion}
  P_l(t, \tau) = \varepsilon P_{l, 1} (t, \tau) + \varepsilon^2 P_{l, 2} (t, \tau) + \cdots.
\end{align}
Thus, order parameter $Z$ is also expanded as
\begin{align}
  \label{eq: order parameter expansion}
  Z(t, \tau) = \varepsilon Z_{1} (t, \tau) + \varepsilon^2 Z_{2} (t, \tau) + \cdots.
\end{align}
Substituting Eqs. (\ref{eq: bifurcation parameter}-\ref{eq: expansion}) into the Fourier series (\ref{eq: FPE Fourier l=2}-\ref{eq: FPE Fourier}),
we obtain
%yields the system of balance equations: 
\begin{align}
  \label{eq: P_l}
  \left( \cfrac{\partial}{\partial t} + l^2  D\right) P_{l, \nu} &= B_{l, \nu},\ \ (l \ne \pm 1)\\
  \label{eq: P_1}
  \cfrac{\partial}{\partial t} Z_{\nu} &= B_{- 1, \nu},\ \ 
\end{align}
where
%$B_{l, \nu}$ has the forms except for $l = 1, 2$: 
\begin{align}
  \label{eq: B_1_1}
  B_{-1,1} &= 0,\\
  \label{eq: B_1_2}
  B_{-1,2} &= -\cfrac{K_c}{2} P_{-2,1} P_{1,1},\\
  \label{eq: B_1_3}
  B_{-1,3} &= - \left(\cfrac{\partial}{\partial \tau} \mp 
  \cfrac{K_c}{2} \right) Z_1
  - \cfrac{K_c}{2}(P_{-2,2} P_{1,1}
%\notag\\ &
 + P_{-2,1} P_{1,2}) 
  + \cfrac{K_2}{2}(P_{1,1} Z_1^2 - P_{-3,1} P_{1,1}^2),\\
  \label{eq: B_2_2}
  B_{-2,2} &= K_c(Z_1^2 - P_{-3,1} P_{1,1}) + K_2  Z_1^2,
\end{align}
and, for $l \neq \pm 1, \pm 2$,
\begin{align}
  \label{eq: B_l_1}
  B_{l, 1} &= 0,\\
  \label{eq: B_l_2}
  B_{l, 2} &= \cfrac{l K_c}{2} (P_{l-1, 1} P_{1, 1} - P_{l+1, 1} P_{-1, 1}),\\
  \label{eq: B_l_3}
  B_{l, 3} &= - \cfrac{\partial}{\partial \tau} P_{l, 1} 
  + \cfrac{l K_c}{2} (P_{l-1, 1} P_{1, 2} + P_{l-1,2} P_{1,1}
%\notag\\ &
- P_{l+1,1} P_{-1,2} - P_{l+1,2} Z_1) 
%  \notag\\ &
+ \cfrac{l K_2}{2} (P_{l-2,1} P_{1,1}^2 - P_{l+2,1} Z_1^2).
\end{align}

We solve the system of Eqs. (\ref{eq: P_l},\ref{eq: P_1}).
Since $Z_1(t, \tau)$ is a function of $\tau$ because of Eqs. (\ref{eq: P_1}-\ref{eq: B_l_1}), 
we rewrite $Z_1$ as $Z_1(\tau)$.
Next, we obtain
\begin{align}
  \label{eq: P_{-3, 1}}
  P_{-3, 1} = 0
\end{align}
because of Eqs. (\ref{eq: P_1},\ref{eq: B_l_1}) for $l = 3$. By substituting Eq. \eqref{eq: P_{-3, 1}} into Eq. \eqref{eq: B_2_2}, $B_{-2, 2}$ reduces to
\begin{align}
  B_{-2, 2} = (K_c + K_2) Z_1^2.
\end{align}
From Eq. \eqref{eq: P_l}, we have the equation for $P_{-2, 2}$ as
\begin{align}
  \label{eq: P_{-2,2}}
  \left( \cfrac{\partial}{\partial t} + 4  D\right) P_{-2, 2} = (K_c + K_2) Z_1^2,
\end{align}
and 
\begin{align}
  P_{-2, 2} = \cfrac{(K_c + K_2)Z_1^2}{4 D}
\end{align}
is a solution to Eq. \eqref{eq: P_{-2,2}}.
In the same way as $P_{-3, 1}$, we obtain
%$P_{-2, 1}$ vanishes:
\begin{align}
  \label{eq: P_{-2, 1}}
  P_{-2, 1} = 0.
\end{align}

Substituting Eqs. (\ref{eq: P_{-3, 1}},\ref{eq: P_{-2,2}},\ref{eq: P_{-2, 1}})
into Eq.~\eqref{eq: B_l_3} for $l=-1$ yields 
\begin{align}
  \label{eq: B_{-1, 3}}
  B_{-1, 3} &= \left(\cfrac{\partial}{\partial \tau} \mp 
  \cfrac{K_c}{2} \right) Z_1
  - \cfrac{K_c(K_c + K_2)Z_1^2 P_{1,1}}{8 D}
  + \cfrac{K_2 P_{1,1} Z_1^2}{2}\\
  &= \left(\cfrac{\partial}{\partial \tau} \mp 
  \cfrac{K_c}{2} \right) Z_1
  - \cfrac{(K_c^2 + K_c K_2)Z_1 |Z_1|^2}{8 D}
  + \cfrac{K_2 Z_1 |Z_1|^2}{2}
\end{align}

Finally, from the solvability condition $B_{-1, 3} = 0$, we obtain the normalized equation in the first order:
\begin{align}
  \label{eq: Z_1}
  \cfrac{\partial Z_1}{\partial \tau} = \pm \cfrac{K_c}{2} Z_1 - \left( \cfrac{K_c^2 + K_c K_2}{8 D} - \cfrac{K_2}{2}\right) Z_1 |Z_1|^2.
\end{align}


\section{Time evolution of the order parameter $R$ }
In the main text, we only show the evolution equation of $R$ only for $K_1=0$.
Here, we derive this for $K_1\geq 0$.
We rewrite Eq.~\eqref{eq: model 1-1} as a gradient system:
\begin{equation}
 \dot \theta_m = -\cfrac{\partial}{\partial \theta_m} U(\theta_m,R) + \xi_m,
\end{equation}
where 
%The potential $U$ corresponding to Eq.~\eqref{eq: model 1-1} is given as
\begin{align}
  \label{eq: potential 2}
  U(\theta, R) = -\cfrac{1}{2} (K_2 R^2 \cos 2\theta + 2 K_1 R \cos \theta).
%  \dot{\theta} &= \cfrac{\partial}{\partial \theta} U(\theta, R) + \xi_m.
\end{align}
We show typical $U$ shapes as a function of $\theta$ in Fig. \ref{fig: potential}.
We only focus on the case in which the potential as a function of $\theta$ has two minima; i.e., $U$ is double-well.
Because 
\begin{align}
    \cfrac{\partial}{\partial \theta }U(\theta, R) &= K_2 R^2 \sin 2\theta + K_1 R \sin \theta \notag\\
    &= 2 K_2 R^2 \sin \theta \cos \theta + K_1 R \sin \theta \notag\\
    &= R \sin \theta (2 K_2 R \cos \theta + K_1),
    \label{eq: derivative condition}
\end{align}
the necessary and sufficient condition for the potential to be double-well is
\begin{align}
  \label{eq: necessary condition}
  \cfrac{K_1}{2 K_2 R} < 1.
\end{align}
%graphical view of potentials for two cases: single-well and double-well in Fig. \ref{fig: potential}.
% Figure environment removed
%In the analysis described
Below, we assume that Eq.~\eqref{eq: necessary condition} holds true.
The following quantities will be needed later.
The potential $U(\theta, R)$ has two minima:
\begin{align}
  U_{\min 1} &:= U(0, R) = - \cfrac{1}{2}(K_2 R^2 + 2 K_1 R),\\
  U_{\min 2} &:= U(\pi, R) = - \cfrac{1}{2}(K_2 R^2 - 2 K_1 R).
\end{align}
The maximum value $U_{\max}$ is
\begin{align}
  U_{\max} &:= U(\theta_{\max}, R) \notag\\
  &=-\cfrac{1}{2}\ (K_2 R^2 \cos 2 \theta_{\max} + 2 K_1 R \cos \theta_{\max}) \notag\\
  &= -\cfrac{1}{2} \left\{K_2 R^2 \left(\cfrac{K_1^2}{2 K_2^2 R^2} - 1\right) - 2 K_1 R \cfrac{K_1}{2 K_2 R} \right\} \notag\\
  &= -\cfrac{1}{2} \left(K_2 R^2 + \cfrac{K_1^2}{2 K_2 R}\right).
\end{align}
We define $\theta_{\max}$ as one of two maximum points of the potential $U(\theta)$ within the range $0<\theta<\pi$.
The potential barriers are
\begin{align}
  \Delta U_1 &:= U_{\max} - U_{\min 1} \notag \\
&= \cfrac{1}{2} \left(K_2 R^2 \cfrac{K_1^2}{2 K_2 R} + \cfrac{1}{2} (K_2 R^2 + 2 K_1 R)\right) \notag\\
  &= K_2 R^2 + K_1 R + \cfrac{K_1^2}{4 K_2},\\
  \Delta U_2  &:= U_{\max} - U_{\min 2} \notag \\
&= \cfrac{1}{2} \left(K_2 R^2 \cfrac{K_1^2}{2 K_2 R} + \cfrac{1}{2} (K_2 R^2 - 2 K_1 R)\right) \notag\\
  &= K_2 R^2 - K_1 R + \cfrac{K_1^2}{4 K_2}.
\end{align}
%Thirdly, we obtain the second derivatives of the potential at three extremal points:
The second derivatives of the potential at three extremal points are
\begin{align}
  \partial_{\theta}^2 U(\theta_{\max}, R) &= \cfrac{K_1^2}{2 K_2} - 2 K_2 R^2,\\
  \partial_{\theta}^2 U(\theta_{\min 1}, R) &= \partial_{\theta}^2 U(0) = 2 K_2 R^2 + K_1 R,\\
  \partial_{\theta}^2 U(\theta_{\min 2}, R) &= \partial_{\theta}^2 U(\pi, R) = 2 K_2 R^2 - K_1 R.
\end{align}

As in the case in which only the three-body interaction model exist, we assume that the phase distribution is approximately described by
\begin{align}
  P(\theta,t)=\eta(t) \delta(\theta) + (1 - \eta(t)) \delta(\theta - \pi).
  \tag{\ref{delta function}}
\end{align}
In this approximation, each oscillator takes the phase either $0$ or $\pi$.
By defining $H$ and $H^*$ as the states in which the phase of an oscillator having the phases $0$ and $\pi$, respectively, the transition process is schematically described as
\begin{align}
  \label{reaction}
  H \overset{k_+}{\underset{k_-}{\rightleftharpoons}} H^*,
\end{align}
where $k_\pm$ are the transition rates.
Similar to Eq.~\eqref{eq: Kramers rate general}, we obtain
\begin{align}
  k_+ (R) %&= k_+ = k_- \notag\\
  &= 2 \cfrac{\sqrt{|\partial_{\theta}^2 U(\theta_{\min 1}, R)\partial_{\theta}^2 U(\theta_{\max}, R)}|}{2\pi} \exp\left(- \cfrac{\Delta U_1}{D}\right), \label{k+}\\
  k_- (R) %&= k_+ = k_- \notag\\
  &= 2 \cfrac{\sqrt{|\partial_{\theta}^2 U(\theta_{\min 2}, R)\partial_{\theta}^2 U(\theta_{\max}, R)}|}{2\pi} \exp\left(- \cfrac{\Delta U_2}{D}\right). \label{k-}
\end{align}
Substituting the obtained expressions into Eqs.~\eqref{k+} and \eqref{k-}, we obtain
\begin{align}
  \label{eq: rate}
  k_+ (R) &= \cfrac{1}{\pi} \sqrt{(2 K_2 R^2 + K_1 R)\left|2 K_2 R^2 - \cfrac{K_1^2}{2 K_2} \right|} \exp \left(- \cfrac{1}{D} \left(K_2 R^2 + K_1 R + \cfrac{K_1^2}{4 K_2} \right) \right)\\
  k_- (R) &= \cfrac{1}{\pi} \sqrt{(2 K_2 R^2 - K_1 R)\left|2 K_2 R^2 - \cfrac{K_1^2}{2 K_2} \right|} \exp \left(- \cfrac{1}{D} \left(K_2 R^2 - K_1 R + \cfrac{K_1^2}{4 K_2} \right) \right).
\end{align}

We find that the time evolution of $R$ is
\begin{align}
  \dot{R} &= - k_+ (1 + R) + k_+ (1 - R)\\
  &= - \cfrac{1}{\pi} \sqrt{(2 K_2 R^2 + K_1 R)\left|2 K_2 R^2 - \cfrac{K_1^2}{2 K_2} \right|}
  \exp \left(- \cfrac{1}{D} \left(K_2 R^2 + K_1 R + \cfrac{K_1^2}{4 K_2} \right) \right) (1 + R) \notag\\
  &+ \cfrac{1}{\pi} \sqrt{(2 K_2 R^2 - K_1 R)\left|2 K_2 R^2 - \cfrac{K_1^2}{2 K_2} \right|}
\exp \left(- \cfrac{1}{D} \left(K_2 R^2 - K_1 R + \cfrac{K_1^2}{4 K_2} \right) \right) (1 - R) \notag\\
  &= - \cfrac{1}{\pi} \sqrt{(2 K_2 R^2 + K_1 R)\left(2 K_2 R^2 - \cfrac{K_1^2}{2 K_2} \right)}
  \exp \left(- \cfrac{1}{D} \left(K_2 R^2 + K_1 R + \cfrac{K_1^2}{4 K_2} \right) \right) (1 + R) \notag\\
  &+ \cfrac{1}{\pi} \sqrt{(2 K_2 R^2 - K_1 R)\left(2 K_2 R^2 - \cfrac{K_1^2}{2 K_2} \right)}
\exp \left(- \cfrac{1}{D} \left(K_2 R^2 - K_1 R + \cfrac{K_1^2}{4 K_2} \right) \right) (1 - R).
  \label{eq: R Kramers 2}
\end{align}
This equation is reduced to Eq.~\eqref{eq: time evolution of R} for $K_1=0$.
Figure \ref{fig: R Kramers} compares the dynamics of $R$ according to \eqref{eq: R Kramers 2} and that of the simulations of $N = 10^3$ oscillators following Eq. \eqref{eq: model 1}, which are in an excellent agreement. 


% Figure environment removed



\end{document}