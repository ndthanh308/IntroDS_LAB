\documentclass[aps, prb,twocolumn,showpacs,floatfix,superscriptaddress,nofootinbib]{revtex4-2}
\usepackage{graphics}
\usepackage{xcolor}
\usepackage{amsmath, amssymb,amsthm}
\usepackage{mathtools}
% \usepackage{braket}
\usepackage{physics}
% \usepackage{color}                                                     %%   change color
% \usepackage{bm}                                                        %%   bold math
% \usepackage{appendix}                                              %%   formating appendicies
\usepackage{bbold}
\usepackage[normalem]{ulem}
\usepackage[colorlinks=true,citecolor=blue,linkcolor=magenta]{hyperref}
\usepackage[caption=false]{subfig}


\newcommand{\diff}[1]{\mathrm{d}{#1}\,}
\renewcommand{\dfrac}[2]{\frac{\mathrm{d}{#1}}{\mathrm{d}{#2}}}
\newcommand{\pdfrac}[2]{\frac{\partial{#1}}{\partial{#2}}}
\newcommand{\ddfrac}[2]{\frac{\mathrm{d}^2{#1}}{\mathrm{d}{#2}^2}}
\newcommand*\mathinhead[2]{\texorpdfstring{$\boldsymbol{#1}$}{#2}}
\newcommand*\mathinheadshort[2]{\texorpdfstring{$#1$}{#2}}

\newcommand{\bs}[1]{{\boldsymbol{#1}}}
\newcommand{\bk}{\bs{k}}
\newcommand{\br}{\bs{r}}
\newcommand{\bp}{\bs{p}}
\newcommand{\bx}{\bs{x}}
\renewcommand{\Im}{\operatorname{Im}}
\renewcommand{\Re}{\operatorname{Re}}
% \newcommand{\Tr}{\operatorname{Tr}}
\newcommand{\Var}{\operatorname{Var}}

%my colors
\definecolor{MyBlue}{HTML}{1f77b4}
\definecolor{MyOrange}{HTML}{ff7f0e}
\definecolor{MyGreen}{HTML}{2ca02c}
\definecolor{MyRed}{HTML}{d62728}

\newcommand{\msout}[1]{\text{\sout{\ensuremath{#1}}}}
\newcommand{\old}[1]{\textcolor{blue}{\sout{#1}}}
\newcommand{\mold}[1]{\textcolot{blue}{\msout}}
\renewcommand{\dd}[1]{\textcolor{MyOrange}{DD: #1}}
\newcommand{\jj}[1]{\textcolor{MyGreen}{JJ: #1}}
\newcommand{\nc}[1]{\textcolor{MyRed}{NC: #1}}
\newcommand{\new}[1]{\textcolor{red}{#1}}
\newcommand{\com}[1]{\textcolor{orange}{#1}}

\begin{document}

% \title{Berezinskii diagrammatic technique without time-reversal symmetry, quantum boomerang effect}

\title{Berezinskii approach to disordered spin systems with asymmetric scattering
%without time-reversal symmetry 
and application to the quantum boomerang effect}


\author{Jakub Janarek}\email{jakub.janarek@uj.edu.pl}
\affiliation{Instytut Fizyki Teoretycznej,
Uniwersytet Jagiello\'nski,  \L{}ojasiewicza 11, PL-30-348 Krak\'ow, Poland}




\author{Nicolas Cherroret}\email{nicolas.cherroret@lkb.upmc.fr}
\affiliation{Laboratoire Kastler Brossel, Sorbonne Universit\'e, CNRS,
ENS-PSL Research University, Coll\`ege de France, 4 Place Jussieu, 75005
Paris, France}


\author{Dominique Delande}\email{dominique.delande@lkb.upmc.fr}
\affiliation{Laboratoire Kastler Brossel, Sorbonne Universit\'e, CNRS,
ENS-PSL Research University, Coll\`ege de France, 4 Place Jussieu, 75005
Paris, France}

%\date{\today}


\begin{abstract}
We extend the Berezinskii diagrammatic technique to one-dimensional disordered spin systems, in which time-reversal invariance is broken due to a spin-orbit coupling term inducing left-right asymmetric scattering.
We then use this formalism to theoretically describe the dynamics of the quantum boomerang effect, a recently discovered manifestation of Anderson localization. The theoretical results are confirmed by exact numerical simulations of wave-packet dynamics in a random potential.
\end{abstract}

\maketitle

\section{Introduction}

In the presence of a spatially disordered potential, quantum wave packets may experience, after an transient temporal spreading, a complete freezing of their density distribution due to the proliferation of destructive interference in the multiple scattering process. This phenomenon, which generically occurs in low dimension, is one of the most representative manifestations of Anderson localization \cite{Anderson1958}. As such, it has been primarily exploited in the experimental quest for the localization of cold atoms in random potentials \cite{Billy2008, Roati2008, Semeghini2015, Jendrzejewski2012a}. Recently, however, a variety of alternative signatures of Anderson localization has been identified. Those include the temporal freezing of the coherent backscattering effect in reciprocal space \cite{Cherroret2012, Ghosh2015, Cobus2016} or the universal growth of narrow peak structures in the density profile \cite{Hainaut2017, Hainaut2018} and momentum distribution \cite{Karpiuk2012, Ghosh2014, Loon2014}  of spreading wave packets %, respectively known as the mesoscopic echo \cite{Hainaut2017} and the coherent forward scattering \cite{Karpiuk2012, Ghosh2014, Loon2014} effects 
(see \cite{Cherroret2021} for a review).

Recently, yet another unexpected manifestation of Anderson localization, dubbed quantum boomerang effect (QBE), has been discovered \cite{Prat2019}. The QBE corresponds to a back-and-forth motion of the mean position of a quantum wave packet launched with a finite velocity in a given direction in a random potential. In one dimension, for instance, if the quantum particle is launched to the right, it will first move to the right over a distance of the order of the mean free path, then make a U-turn and eventually return to its starting point at long time. This phenomenon was also shown to exist in higher-dimensional random or pseudo-random systems \cite{Prat2019, Tessieri2021}, as well as in kicked-rotor models \cite{Tessieri2021}, where it was recently demonstrated experimentally \cite{Sajjad2021}.
While originally described in time-reversal-invariant (TRI) systems, recently the QBE was  also shown to exist in systems without time-reversal symmetry \cite{Janarek2022, Noronha22, Macri22}. In \cite{Janarek2022}, in addition, the QBE was characterized in the presence of a spin-orbit coupling mechanism inducing left-right asymmetric scattering between different spin states. This is also the scenario addressed in the present paper.
%The QBE, as a single-particle phenomenon, is influenced by inter-particle interactions leading to destruction of the quantum return to the origin \cite{Janarek2020a, Janarek2023}. 

At a theoretical level, describing the temporal dynamics of quantum wave packets in the presence of disorder is a challenging task \cite{Sanchez-Palencia2007, Skipetrov2008, Shapiro2012}. In one dimension, however, a very powerful analytical approach known as the Berezinskii diagrammatic technique has been developed \cite{Berezinskii1996}. Originally, this method was successfully used for calculating the  ac conductivity of electronic conductors in the localization regime or the long-time  density distribution of spreading wave packets \cite{Berezinskii1996, Gogolin1976, Wellens2016}, the predictions being exact in the limit of weak disorder. More recently, it also allowed to describe the QBE in TRI systems \cite{Prat2019} and, in the context of electron scattering, was extended to account for the presence of spin-orbit coupling \cite{Suleymanli2023}. 

In this paper, we extend the Berezinskii diagrammatic technique to TRI-broken spin-dependent systems in which a spin-orbit coupling term induces asymmetric scattering, as recently realized experimentally with cold atoms \cite{Lin2011, Hamner2015}. This formalism is developed in sections \ref{Sec:principles} and \ref{sec:diagrammatic_approach}. In Sec. \ref{sec:qbe} we then apply the method to the calculation of a specific observable, the mean position of a quantum-mechanical wave packet launched in a random potential with finite velocity. This provides a thorough theoretical description of the QBE in spin-orbit coupled systems with asymmetric scattering, complementing results obtained in the recent work \cite{Janarek2022}. 
Generally speaking, the formalism presented in this paper provides a practical analytical tool to characterize the dynamics of spinor wave packets in disordered systems with TRI-broken symmetry.


\section{Principles of the Berezinskii technique}
\label{Sec:principles}


We start by recalling the main ideas of the original Berezinskii diagrammatic technique used to compute the time-dependent transport properties of one-dimensional disordered systems.  The starting point is the single-particle Hamiltonian
\begin{equation}
\label{eq:generalH}
    H = H_0 + V(x),
\end{equation}
where $V(x)$ is a random (disorder) potential and $H_0$ is the disorder-free part of the Hamiltonian (e.g., $H_0 = p^2/2m$). We suppose that the random potential has a vanishing mean, $\overline{V(x)} = 0$, and follows a Gaussian statistics characterized by the two-point correlation function $\smash{\overline{V(x)V(x')} = \eta C(x'-x)}$, where $\eta$ is called the disorder strength. Symbol $\overline{(\ldots)}$ here denotes averaging over different disorder realizations. %The correlation $C(x'-x)$ has a finite length scale, denoted $\sigma_c$. 
The function $C(x'-x)$ quantifies the range of the spatial correlation of the disorder.
In the whole paper, we 
%assume for simplicity that the disorder has a Gaussian statistics and is 
restrict ourselves to a delta correlated potential, i.e., $C(x'-x) = \delta(x'-x)$.

In this paper, we aim at describing the time evolution of quantum-mechanical wave packets governed by an Hamiltonian of the type of Eq. (\ref{eq:generalH}). In the localization problem, this evolution is characterized by considering the disorder average of observables that depend quadratically on the wave function,  such as the density $n(x,t)$ or the mean position $\langle x(t)\rangle=\int dx\ x\ n(x,t)$ of the wave packet. These observables, by definition, can be expressed in terms of the disorder-averaged correlator $\overline{G^R(x,x',\epsilon)G^A(x'',x,\epsilon-\hbar\omega)}$ \cite{Akkermans2007book, Muller2011}, where $G^{R/A}(x,x',\epsilon) \equiv \bra{x} \left(\epsilon - H \pm i0^{+} \right)^{-1}\ket{x'}$ are the single-realization,  retarded and advanced Green's functions at energy $\epsilon$ associated with Hamiltonian (\ref{eq:generalH}). The energy difference $\hbar\omega$ introduced in the correlator allows us to capture the time dependence of observables after an inverse Fourier transform. The precise connection between $\langle x(t)\rangle$ and the Green's function correlator, for instance,  will be given in Sec. \ref{sec:diagrammatic_approach}. 

Both Green's functions $G^{R/A}$ that appear in the correlator may be computed in a perturbative fashion using the Born expansion \cite{Akkermans2007book}: 
\begin{equation}
\label{eq:born_expansion}
\begin{split}
    &G^{R/A}(x, x', \epsilon) = G^{R/A}_0(x, x', \epsilon)  \\
    &+\int\diff{x_1}G^{R/A}_0(x, x_1, \epsilon)V(x_1)G^{R/A}_0(x_1, x', \epsilon) \\ &+\int\diff{x_1}\diff{x_2}G^{R/A}_0(x, x_1, \epsilon) V(x_1)G^{R/A}_0(x_1, x_2, \epsilon) \\&\times V(x_2)G^{R/A}_0(x_2, x', \epsilon) + \ldots,
\end{split}
\end{equation}
where 
\begin{equation}\label{eq:greens_definition}
    G_0^{R/A}(x,x',\epsilon) \equiv \bra{x} \left(\epsilon - H_0 \pm i0^{+} \right)^{-1}\ket{x'}
\end{equation}
are the retarded and advanced Green's functions associated with the free part of the Hamiltonian. Physically, the expansion (\ref{eq:born_expansion}) describes a multiple scattering sequence involving scattering events on the random potential at points $x_1,\ x_2,\ \ldots$

% Figure environment removed

The average product $\overline{G^R(x,x',\epsilon)G^A(x'',x,\epsilon-\hbar\omega)}$ includes all possible correlations between two multiple scattering paths starting at the initial points $x'$ and $x''$, respectively, and both ending at the final point $x$. 
The starting point of the Berezinskii technique is to take advantage of the one-dimensional geometry, which enables us to order the  scattering events on a line:
\begin{equation}\label{eq:scattering_ordering}
    -\infty<x_1\leq \ldots \leq  x' \leq \ldots  \leq x \leq \ldots \leq x_i < \infty.
\end{equation}
Thanks to this ordering, each contribution to the product $G^R G^A$ may be represented by a diagram, like the one shown in Fig.~\ref{fig:contributing_diagram}(a) \cite{Berezinskii1996}, which combines a retarded (solid lines) and an advanced (dashed lines) multiple scattering sequence, respectively unfolded in the upper and lower parts of the diagram. The scattering events occur at the points $x_i$. In the example of Fig.~\ref{fig:contributing_diagram}(a), the upper sequence involves  8 scattering events (twice at point $x_3$), and the lower one  7 scatterings events.

The diagrams effectively contributing to the Green's function correlator $\overline{G^R G^A}$ do not have arbitrary shapes. Indeed, because of the assumed Gaussian statistics and the corresponding Wick's theorem, only diagrams whose scattering events can all be paired
%at a given scattering event 
appear. For instance, the diagram in Fig.~\ref{fig:contributing_diagram}(a) vanishes upon averaging because some scattering events cannot be paired.
%has a vanishing contribution to $\overline{G^R(x,x',\epsilon)G^A(x,x'',\epsilon-\hbar\omega)}$. 
Pairing scattering events at different points would  occur in the case of a weakly correlated disorder (for which the correlation length is smaller than the mean free path), a problem previously addressed in \cite{Rashba1976, Gogolin1976, Wellens2016}.

The second important approximation of the Berezinskii technique is to assume that among all possible diagrams contributing to the correlator, only those for which the phase factors induced by free-particle Green functions exactly compensate each other when $\omega\to0$ matter. This approximation, which holds true in the regime of ``weak disorder'' (see Sec. \ref{Sec:vertices}), amounts to imposing that there is exactly the same number of retarded and advanced Green's function in between any two successive scattering events $x_i$ and $x_{i+1}$. In turn, this 
%In order to find the relevant diagrams that do contribute to the correlator, we need to build them so that  scattering events are paired, and each diagram has to have zero total phase because contributions with nonzero phases are negligible under disorder averaging. Every diagram can be divided into intervals between consecutive scattering events, i.e. intervals between points $x_i$ and $x_{i+1}$. Each interval has a specific number of solid/dashed lines inside. This 
yields restrictions on the possible \emph{scattering vertices}, which are building blocks of the diagrams: aside the trivial constraint that the solid/dashed lines have to be continuous, all vertices have to be phaseless. %, with a similar condition holding for initial and final vertice.

With these conditions implemented, the Berezinskii diagrammatic technique provides a strategy to exactly sum all possible diagrams with nonzero contributions, as we now detail for the case of a spin-orbit Hamiltonian $H_0$. 
%Using this approach all diagrams can be systematically treated, going beyond Born approximation and may be used in situations exceeding weak disorder assumption. 


% \begin{itemize}
%     \item useful tool in 1D disordered systems, it is used to compute final density of particles (Berezinskii, Gogolin, Wellens), as well as the original QBE (Prat et al.); check for other works
%     \item goes beyond Born approximation used in weak disored limits, takes care of all possible contributions in the Born series 
%     \item can be used also for correlated disorders 
%     \item diagrammatic approach: show an example, like in work of Wellens/Tony/thesis
%     \item lines and vertices, approach to 'initial conditions'
%     \item phase cancellation rules: explain what they are (maybe without details?)
% \end{itemize}


\section{Diagrammatic approach without time-reversal symmetry}
\label{sec:diagrammatic_approach}

\subsection{Free Hamiltonian and Green's function}
In this work, we extend the standard Berezinskii technique to a one-dimensional spin system with spin-orbit coupling and Zeeman splitting breaking all anti-unitary time-reversal symmetries \cite{Lin2011, Hamner2015, Janarek2022}. The corresponding disorder-free Hamiltonian reads:
\begin{equation}\label{eq:h0}
    H_0 = \frac{\hbar^2 k^2}{2m} + \gamma \hbar k \sigma_z + \frac{\hbar \delta}{2}\sigma_z + \frac{\hbar \Omega}{2}\sigma_x,
\end{equation}
where  $\sigma_i$ are the usual Pauli matrices. The Hilbert space is spanned by two-dimensional complex-valued spinors. $\gamma$ is the strength of the spin-orbit coupling, $\Omega$ is the Rabi frequency and $\delta$ the detuning. 
Diagonalization of the Hamiltonian $H_0$ yields two energy bands denoted by $\pm$ with corresponding energies
\begin{equation}
    E_\pm(k) = \frac{\hbar^2 k^2}{2m} \pm \frac{\hbar}{2}\sqrt{\left(2\gamma k + \delta \right)^2 + \Omega^2}.
\end{equation}
Due to this band structure, for a given energy $\epsilon$ the Hamiltonian hosts either 2 or 4 possible eigenstates. From now on, we focus on the case where only two eigenstates are involved, which corresponds to a dynamics operating at energies belonging to the lower band only \cite{Janarek2022}. We denote by $k_\pm$ the momenta of these two states, and by $v_\pm = \frac{1}{\hbar}\left|dE_-(k_\pm)/dk\right|$ the associated velocities.
%At a given energy there may be up to 4 possible eigenstates of the system. Here, we focus on a case with only two states  at energy $\epsilon_0$ (i.e. in the lower band), with momenta $k_\pm$ and associated velocities $v_\pm = \frac{1}{\hbar}\left|\dfrac{E_-(k_\pm)}{k}\right|$.
As compared to the standard single-particle Hamiltonian $\tilde{H}_0 = p^2/2m$, it should be noted that the two involved momenta are not just of opposite sign, i.e., $k_- \neq - k_+$ (and, correspondingly, $v_- \neq v_+$). The left-right symmetry is therefore broken which, as will be seen below, constitutes the most significant difference as compared to the usual Berezinskii approach. In \cite{Suleymanli2023}, a much simpler situation was studied, where only the spin-orbit interaction is present (i.e. $\delta=\Omega=0$); in such a case, the dispersion relation is symmetric with respect to $k\to -k$, so that $v_-=v_+$, and the extension of the Berezinskii technique is rather easy. In contrast, the calculations presented in the present paper are more general and valid when (generalized) TRI is broken.
 
In the diagrammatic treatment of disorder scattering introduced in the previous section, a fundamental ingredient is the free Green's function (\ref{eq:greens_definition}), which we need to evaluate for the Hamiltonian (\ref{eq:h0}). To this aim, we use the definition
\begin{equation}
\label{eq:G0def}
    G_0^R(x,x',\epsilon)\equiv \int_{-\infty}^\infty\frac{dk}{2\pi}\frac{e^{ik(x-x')}}{\epsilon-E_-(k)+i0^+}.
\end{equation}
A careful calculation of the integral in momentum space provides us with 
%At a given energy there may be up to 4 possible eigenstates of the system. Here, we focus on a case with only two states  at energy $\epsilon_0$ (i.e. in the lower band), with momenta $k_\pm$ and associated velocities $v_\pm = \frac{1}{\hbar}\left|\dfrac{E_-(k_\pm)}{k}\right|$. Even though this situation resembles a standard single-particle Hamiltonian $\tilde{H}_0 = p^2/2m$, it is far more complicated because the momenta and velocities associated with the eigenstates are not opposite, i.e. $k_- \neq - k_+$ and $v_- \neq v_+$. 
%Such a case breaks the left-right symmetry. Depending on the direction of motion, the Green's function has different weight. This requires vast modifications to the standard Berezinskii approach. The disorder-free Green's functions are given by:
\begin{equation}
\label{eq:so_green_function}
    G_0^R(x, x', \epsilon) = 
    \begin{cases}
        -\frac{i}{\hbar v_+}e^{ik_+(x-x')}, \quad x - x' > 0 \\
        -\frac{i}{2\hbar}\left(\frac{1}{v_+} + \frac{1}{v_-}\right), \quad x = x' \\
        -\frac{i}{\hbar v_-}e^{ik_-(x-x')}, \quad x - x' < 0
    \end{cases}
\end{equation}
where, in particular, the diagonal value $ G_0^R(x, x, \epsilon$) is obtained by properly accounting for all the real and complex poles in the denominator of Eq. (\ref{eq:G0def}). Note that, strictly speaking, when $x-x'\ne0$ these expressions only hold at distances $|x-x'|$ larger than the de Broglie wavelength $2\pi/|k_\pm|$. 
This knowledge, however, is sufficient 
within the weak disorder limit (\ref{eq:weak_disorder}) where the Berezinskii approach operates. 
Because of translation invariance, the free Green's function $G_0^R(x,x',\epsilon) = G_0^R(x-x',\epsilon)$. Its Fourier transform is therefore diagonal in momentum space, with the diagonal value defined as $G_0^R(k, \epsilon) = \int\diff{r} e^{-ikr} G_0^R(r, \epsilon)$, where $r=x-x'$. Note that with this definition, Eq. (\ref{eq:so_green_function}) implies that $G^{R/A}(k,\epsilon)\neq G^{R/A}(-k,\epsilon)$,  contrary to TRI systems.
From Eq. (\ref{eq:so_green_function}), finally, the advanced Green's function follows from Hermitian conjugation, $G_0^A(x,x',\epsilon)=[G_0^R(x',x,\epsilon)]^*$. 
%\new{For the reason that the Green function is translationally invariant, i.e. $G_0^R(x,x',\epsilon) = G_0^R(x-x',\epsilon)$, in  momentum space is it diagonal. The Fourier transform reads: 
%\begin{equation}
 %   G_0^R(k, \epsilon) = \int\diff{r} e^{-ikr} G_0^R(r, \epsilon),
%\end{equation}
%where $r=x-x'$.
%}
%Notice, incidentally, that,  contrary to TRI systems, $G^{R/A}(k,\epsilon)\neq G^{R/A}(-k,\epsilon)$. 


In the following, we will also need the energy-shifted Green's function $G_0^A(x', x, \epsilon - \hbar \omega)$, where $\omega \ll \epsilon/\hbar$. To evaluate this object, we use the Taylor expansions $k_\pm(\epsilon-\hbar\omega)\approx k_\pm\mp\omega/v_\pm$, so that:
\begin{eqnarray}
\label{eq:gA_omega}
    G_0^A&&\!\!\!\!\!\!\!\!(x', x, \epsilon-\hbar\omega) =\nonumber \\
   && \begin{cases}
        \frac{i}{\hbar v_+}e^{-i(k_+-\omega/v_+)(x-x')}, \quad x - x' > 0 \\
        \frac{i}{2\hbar}\left(\frac{1}{v_+} + \frac{1}{v_-}\right), \quad x = x'\\
        \frac{i}{\hbar v_-}e^{-i(k_-+\omega/v_-)(x-x')}, \quad x - x' < 0
    \end{cases}
\end{eqnarray}
%With Eqs. (\ref{eq:so_green_function}) and (\ref{eq:gA_omega}) at hand, we have the building blocks needed for the diagrammatic treatment. 
%Note that although the potential $V(x)$ is the same for both spin components, it will generally couple the disorder-free eigenstates with the same energy but with different spin states and momenta. 


\subsection{Mean free times}
\label{Sec:mft}

Before constructing the diagrammatic approach based on the Hamiltonian $H_0+V(x)$, let us introduce 
%As was shown in \cite{Janarek2022}, this Hamiltonian belongs to the unitary symmetry class. Before proceeding with the calculation of the mean position (\ref{eq:com_freq2}), we introduce 
a few important scattering parameters that will be used in the following. The central one is the concept of scattering mean free time, which gives the average time scale between two consecutive scattering events. In the present case, however, two different mean free paths can be defined due to the left-right asymmetry. 
To find them, let us denote  by $|\pm\rangle=|k_\pm\rangle\otimes|\chi_\pm\rangle$ the two eigenstes of $H_0$, where $\ket{\chi_+}$ and $\ket{\chi_-}$ are the spin state components associated with the wave numbers $k_+$ and $k_-$, respectively.  This leads us to define $\tau_+$ and $\tau_-$, the scattering mean free times for the processes $|+\rangle\to |-\rangle$ and $|-\rangle\to |+\rangle$, respectively. At weak disorder they can be evaluated from the Fermi golden rule
\begin{equation}
\label{eq:taudef}
    \frac{1}{\tau_\pm}=\frac{2\pi}{\hbar} \,
    \overline{|\langle \mp|V|\pm\rangle|^2}\, \rho(E_-(k_\mp)),
\end{equation}
where $\rho(E_-(k_\mp))$ is the density of states evaluated at the energy $E_-(k_\mp)$ of the final state. Using that the disorder is uncorrelated, i.e., $\overline{V(x')V(x)}=\eta\delta(x'-x)$ (see Sec. \ref{Sec:principles}), we infer:
\begin{equation}
\label{eq:tau_pm}
    \tau_\pm = \frac{\hbar^2 v_\mp}{2\eta\kappa},
\end{equation}
where $\kappa \equiv \left|\bra{\chi_+}\ket{\chi_-}\right|^2$ is the overlap factor of the two spin states. In the following we will be also led to use the mean free time associated with the weighted sum of the two scattering processes:
\begin{equation}
\label{eq:taudef}
% \frac{1}{\tau}=\frac{1}{\tau_+}+\frac{1}{\tau_-},
\frac{1}{\tau}=\frac{1}{2}\left(\frac{1}{\tau_+}+\frac{1}{\tau_-}\right),
\end{equation}
which turns out to be the relevant time scale governing the boomerang effect, as will be shown in Sec. \ref{sec:qbe}. 
% \com{[NC: Jakub, you did not correct anything: is this definition of $\tau$ compatible with all subsequent formulas of the paper yes or no??]} 
Note that the validity of the Fermi golden rule used above is only guaranteed in the weak disorder limit described by Eq. (\ref{eq:weak_disorder}) below.

%\com{[NC: Let me make a few comments here. First, I rewrote this section to properly define the two mean free paths $\tau_\pm$. From them, $\tau$ is just presented as an effective mean free path that will turn out to be the relevant quantity appearing in the QBE. I think this addresses the main comment by Dominique. Second, I finally chose to remove any discussion on mean free paths, which I think is not mandatory  given that we are only interested in the time dependence of a global observable (the mean position). 
%In fact, to me it appears that the quantity $\ell=\tau_+ v_+=\tau_-v_-$ is only introduced for convenience in Eqs. 33-38 below (which I  blindly reproduced from the thesis) but disappears in the end of the calculation.. To be completely honest, I find this a bit strange and suspicious because: 1) why introducing artificially a physical parameter in the equations if it never appears in the end?? 2) As Dominique mentioned, in principle  the transport mean free path is the relevant quantity, and therefore the appearance of $\ell$ in the formulas is strange. We have unfortunately no time to address those points, but they leave me slightly uncomfortable...]}



%scattering mean free path $\ell\equiv \tau_+ v_+=\tau_- v_-$.
%$\tau$, which we evaluate from the Fermi golden rule
%\begin{equation}
%\label{eq:taudef}
%    \frac{\hbar}{\tau}=2\pi\kappa \int_{-\infty}^\infty\frac{dk'}{2\pi}\eta\, C(k')\delta(\epsilon-E_{-}(k'))
%\end{equation}
%at the operating energy $\epsilon$. 
%Here the quantity $\eta\, C(k')$ refers to the disorder power spectrum [i.e., the Fourier transform of the correlation function $\overline{V(x)V(x')}$, see Sec. \ref{Sec:principles}]. .
%For an uncorrelated potential, $C(k')=1$, so that Eq. (\ref{eq:taudef}) provides
%\begin{equation*}
%    \frac{1}{\tau}=\frac{1}{\tau_+}+\frac{1}{\tau_-}
%\end{equation*}
%where 
%\begin{equation}\label{eq:tau_pm}
%    \old{\tau_\pm = \frac{\hbar^2 v_\mp}{2\eta\kappa}.}
%    \new{\tau_\pm = \frac{\hbar^2 v_\mp}{\eta\kappa}.}
%\end{equation}
%\new{[NC: There is an inconsistency with the definition of $\tau$ given in Eq. (2.18) of the thesis, where there is an additional factor 2, please confirm that the present definition is OK]}.
%\dd{I agree that there is a problem here. In eq. (10), there are tow contributions, forward and backward scattering, and one must specify the inital state. No rocket science, simply one needs to be more rigorous.
%If one considers transport properties, one of the two contributions to the Fermi golden rule disappears, but this must be stated explictly.}
%\jj{If we keep $\tau=\tau_+\tau_-/(\tau_++\tau_-)$, then $\tau_\pm = \hbar^2v_\mp/\kappa\eta$, without $1/2$. This changes the factors of 2 later, but I guess this is the better way, we have used it in the PRB paper.}
%\dd{I only agree partially with Jakub. If we define $\tau_+$ and $\tau_-$ as the mean free times for the processes $+\to -$ and $-\to +$, there is no freedom for factors 2 and Eq. (2.15) is wrong. In fact, it is even at the level of the Fermi golden rule, eq.(\ref{eq:taudef}),
%that the problem lies. In this equation, there is no reference to the initial and final states. Then, there are contributions both from final states with $K_+$ and $k_-.$, i.e. both forward and backward scattering. If you want say $\tau_+$, you have to dress eq.(\ref{eq:taudef}) with a $|\chi_+\rangle$ state on the right and a $|\chi_-\rangle$ state on the left side, which makes $\kappa$ to appear. This is more or less what we did in the PRB paper. Without this step, the reader cannot understand. Even if we do not repeat the calculation of the PRB paper, we have to provide the reader with some explanations. Jakub, please add the necessary steps.% As a side remark, $\kappa$ has not been introduced at his point (it is introduced after eq. (\ref{eq:vertices_weights}), which makes the whole thing logically inconsistent. $\kappa$ must be introduced here. Also, the definition of $\tau$ looks a bit arbitrary at this point, but we can keep it as it is.}
%\jj{I know that there was a wrong factor of 2 in (2.15). I guess an intermediate step (say before Eq. (\ref{eq:tau_pm})) like this, will not resolve the problem:
%\begin{equation}
 %   \frac{\hbar}{\tau_\pm} = 2\pi\int \frac{\diff{k}}{2\pi}\left|\bra{\chi_\mp}V(k')\ket{\chi_\pm}\right|^2\delta(\epsilon-E_-(k')),
%\end{equation}
%where $|\bra{\chi_\mp}V(k')\ket{\chi_\pm}|^2 = \eta \kappa C(k')$, and $\kappa = \left|\bra{\chi_+}\ket{\chi_-}\right|^2$ Because it does not select the proper \emph{final state} branch of the density of states, where velocity comes in.}


%The two quantities $\tau_+$ and $\tau_-$ have the interpretations of the mean free times associated with scattering processes $+\to -$ (for $\tau_+)$ and $-\to+$ (for $\tau_-$). This shows that although the potential $V(x)$ is the same for both spin components, it   couples the disorder-free eigenstates with the same energy but with different spin states and momenta.

%In the following, we will also use the \new{scattering} mean free path $\ell=\tau_+ v_+=\tau_- v_-$. \new{Due to left-right asymmetry in the system, the transport mean free path $\ell_t$ is not equal to the scattering mean free path $\ell$. It can be calculated using classical approach, and is given by $\ell_t = \tau\sqrt{v_+ v_-}$ \cite{Janarek2022}.}
%\dd{We need to define here a TRANSPORT mean free path. It seems to me that definition given here differs from the one of $\ell_t$ in the pRB paper. Jakub, if I am correct, please fix this isssue.}
%\jj{In Berezinskii calculation we use the scattering mean free path, $\ell$, not the transport mean free path $\ell_t$. I agree that we should add a comment on that scattering and transport mean free paths are not equal. Is it enough? } 





\subsection{Vertices}
\label{Sec:vertices}
% Figure environment removed
At the core of the Berezinskii diagrammatic technique is the idea to transfer the propagating factors from the free Green's functions to the scattering events, called vertices. For example, assuming $x_i > x_j$, the free Green's function can be split as
\begin{equation}
\label{eq:greens_transfer_to_points}
    G_0^R(x_i, x_j, \epsilon) = \sqrt{-\frac{i}{\hbar v_+}}e^{ik_+ x_i}\sqrt{-\frac{i}{\hbar v_+}}e^{-ik_+ x_j},
\end{equation}
where we formally associate the weights and exponential factors to the vertices at points $x_i$ and $x_j$. The difference between TRI system and the TRI-broken case is that these factors depend on the direction of propagation. For example, in the TRI system the opposite case $x_j > x_i$ would result in just a change of sign of the phase factors in Eq.~(\ref{eq:greens_transfer_to_points}), whereas in the system with broken TRI also the velocities change. \\


\paragraph*{Initial vertices.}
We start by selecting the relevant initial vertices effectively contributing to the correlator $\overline{G^R(x,x',\epsilon)G^A(x'',x,\epsilon-\hbar\omega)}$. In general, scattering paths may start from any of the 4 vertices shown with their weights in Fig.~\ref{fig:init_vertices}. The vertices with advanced and retarded lines starting into opposite directions, i.e., vertices (c) and (d), carry exponential factors with phases $i(k_\pm x'' - k_\mp x')$. Upon integration over the starting points $x'$ and $x''$ [cf. Eq.~(\ref{eq:com_freq1}) in Sec.~\ref{sec:qbe}], they typically yield negligible contributions. Thus, we can restrict the analysis  to only two classes of initial vertices: (a) and (b). These classes, in turn, correspond to two different types of initial states for the dynamics, (a) with positive ($v_+$) and (b) with negative ($v_-$) initial velocity. 

A second simplification is based on the assumption that no scattering happens between the initial points $x'$ and $x''$ \cite{Prat2017a, Prat2019}. This invites us to introduce the Wigner variables $r = x' - x''$ and $\tilde{x} = (x' + x'')/2$. In the limit $\omega \to 0$, vertices (a) and (b) are thus approximated by their counterparts starting from a single point $\tilde{x}$. At the level of Green's functions, this simplification reads \cite{Wellens2016}:
\begin{equation}
\begin{split}\label{eq:initial_vertex_simplification}
    \overline{G^R(x,x',\epsilon)G^A(x'',x,\epsilon-\hbar\omega)} \approx \\
    e^{-ik_\epsilon r}\overline{G^R(x,\tilde{x},\epsilon)G^A(\tilde{x},x,\epsilon-\hbar\omega)}
\end{split}
\end{equation}
where $k_\epsilon$ is the wave number  satisfying the dispersion relation $\epsilon=E_-(k_\epsilon)$. \\

\paragraph*{Phaseless scattering vertices}

Our ultimate goal is to sum all significant contributions to the product of Green's functions $\overline{G^R G^A}$. This formidable task is, in general, out of reach except in the so-called weak-disorder limit
\begin{equation}
\label{eq:weak_disorder}
    k_\epsilon \ell\gg1,
\end{equation}
where $\ell=\tau_+v_+=\tau_- v_-$ is the scattering mean free path. Under this condition, only a limited set of scattering vertices that do not accumulate any phase and, as such, are not vanishingly small upon disorder averaging, should be considered when constructing correlation diagrams. 
The procedure to identify this set is detailed in Appendix \ref{Appendix:vertices} for clarity. It yields four families of vertices that are listed in Fig. \ref{fig:scattering_vertices}. One can easily check that the phase associated with each vertex is zero. For instance, the vertex a$_1$ originates from a factor of the type $ \eta G_0^R(x_i,x_{i-1})G_0^R(x_i,x_i) G_0^R(x_{i+1},x_i)$ in the disorder average of the Born expansion (\ref{eq:born_expansion}). With the help of the 
splitting procedure (\ref{eq:greens_transfer_to_points}) and of Eq. (\ref{eq:so_green_function}), this corresponds to a vertex weight
\begin{equation}
    \eta \sqrt{\frac{-i}{\hbar v_+}}e^{ik_+x_i}\times \frac{-i}{2\hbar}\Big(\frac{1}{v_+}+\frac{1}{v_-}\Big)\times \sqrt{\frac{-i}{\hbar v_+}}e^{-ik_+x_i},
\end{equation}
whose phase is indeed zero. In turn, the weights of all  phaseless scattering vertices are 
% Figure environment removed
\begin{align}\label{eq:vertices_weights}
    &\text{a\textsubscript{1/2}: } -\frac{\eta}{2\hbar^2 v_\pm}\left(\frac{1}{v_+} + \frac{1}{v_-}\right),\nonumber\\
    &\text{b\textsubscript{1/2}: } -\frac{\eta}{(\hbar v_\pm)^2}, \quad \text{b\textsubscript{3}: } -\frac{\eta}{(\hbar v_+)(\hbar v_-)} \nonumber\\
    &\text{c: }  - \frac{\eta\kappa}{(\hbar v_+)(\hbar v_-)} \\
   & \text{d\textsubscript{1/2}: } \frac{\eta}{(\hbar v_\pm)^2}, \quad
    \text{d\textsubscript{3/4}: } \frac{\eta}{(\hbar v_+)(\hbar v_-)}, \nonumber\\
   & \text{e: } \frac{\eta\kappa}{(\hbar v_+)(\hbar v_-)}\exp\left[i\omega x \left(\frac{1}{v_+} + \frac{1}{v_-}\right)\right] \nonumber\\
    &\text{f: } \frac{\eta\kappa}{(\hbar v_+)(\hbar v_-)}\exp\left[-i\omega x \left(\frac{1}{v_+} + \frac{1}{v_-}\right)\right]\nonumber
\end{align}
Notice that among all diagrams in Fig. \ref{fig:scattering_vertices}, the 
 vertices families c, e, and f involve a ``backscattering event'' in both the retarded and advanced parts. This implies that, in the spin system described by Hamiltonian~(\ref{eq:h0}), the associated weights include the spin-state overlap factor $\kappa = \left|\bra{\chi_+}\ket{\chi_-}\right|^2$. 
 %Denoting by $\ket{\chi_+}$ and $\ket{\chi_-}$ the spin components of eigenstates with wave numbers $k_+$ and $k_-$, respectively, this overlap factor is simply  $\kappa = \left|\bra{\chi_+}\ket{\chi_-}\right|^2$.

 

\subsection{Correlation diagrams}\label{sec:equations_for_diagrams}

Knowing all possible phaseless scattering vertices relevant to our problem, we now wish to write down the equations describing the  diagrams contributing to the correlator $\smash{\overline{G^R G^A}}$. An example of such a correlation diagram 
%constructed from phaseless vertices 
is shown in Fig.~\ref{fig:contributing_diagram}(b). Its generic structure can be divided into three left, right and central blocks denoted by $L$, $R$, and $Z$, as illustrated in Fig.~\ref{fig:contributing_diagram}(b). These different blocks are characterized by their total number of incoming and outgoing solid (retarded) and dashed (advanced) lines. 

We first consider the left blocks $L$. Because the scattering vertices change the number of lines by at most 2, these blocks  always have the same even number $2m'$ (with $m'$ an integer) of retarded and advanced lines attached. For instance, $L$ in Fig.~\ref{fig:contributing_diagram}(b) has $m' = 1$. With this property in mind, let us denote by $L_{m'}(\tilde{x})$ the sum of contributions from all $L$ blocks that have their right boundary at point $\tilde{x}$ with $2m'$ lines.
To calculate $L_{m'}(\tilde{x})$, we consider how it changes with an infinitesimal change of the boundary position, $\tilde{x}\to\tilde{x}+\delta x$, by counting all possibilities of adding new scattering vertices to $L_{m'}(\tilde{x})$. This  counting in detailed in Appendix \ref{Appendix:Lmblock} for clarity. Taking the limit $\delta x\to 0 $, it yields the following differential equation \cite{Janarek2021}:
\begin{align}
    \label{eq:l_m_full}
        &\dfrac{L_{m'}}{\tilde{x}} = -\frac{2m'\eta}{\hbar^2 v_+v_-}L_{m'}\left[1 + (m'-1)\kappa\right] +\frac{m'^2 \eta \kappa}{\hbar^2 v_+ v_-}\times\nonumber\\
        &\left[L_{m'+1}e^{i\omega \tilde{x}(\frac{1}{v_+} + \frac{1}{v_-})}\!+\!L_{m'-1}e^{-i\omega \tilde{x}(\frac{1}{v_+} + \frac{1}{v_-})}\right]\!.
\end{align}
This equation is solved by an Ansatz $L_{m'}(\tilde{x}) = \mathcal{L}_{m'}\exp\left[-im'\omega \tilde{x}({1}/{v_+} + {1}/{v_-})\right]$, which leads to an iterative equation for $\mathcal{L}_{m'}$:
\begin{equation}\label{eq:l_m_algebraic}
    s \mathcal{L}_{m'} + m'(\mathcal{L}_{m'+1} - \mathcal{L}_{m'-1} + 2\mathcal{L}_{m'}) = 0,
\end{equation}
where $s = 2  - {2}/{\kappa} + i\nu$ with $\nu=\omega (v_+ + v_-)\hbar^2/\kappa \eta$. The explicit solution of Eq.~(\ref{eq:l_m_algebraic}) is
\begin{equation}
\label{eq:Lmfinal}
\mathcal{L}_m(s) = - s\Gamma(m+1)\Psi(m+1,2;-s),
\end{equation}
 with $\Psi(a,b;z)$ the confluent hypergeometric function of the second kind. Note that in the usual case of spinless TRI systems, $\mathcal{L}_m$ satisfies a similar equation as Eq.~(\ref{eq:l_m_algebraic}), but with $s_\text{TRI} = 2i\omega v/\eta$, where $v$ is the velocity of the state at energy $\epsilon$ \cite{Berezinskii1996}. The main difference is that $s_\text{TRI}$ is fully imaginary, while in our case $s$ has a finite real part. 
%However, in the relevant regime $\omega\to 0$ the real part of $s$ can be neglected and does not influence the final result.

The treatment of the right block $R$ is fully analogous. Denoting by $R_m(x)$ the sum of all right-hand blocks which have their left boundary at point $x$ with $2m$ lines (with $m$ an integer), we find that $R_m(x) = L_m(-x)$ and, with a similar Ansatz, $\mathcal{R}_m = \mathcal{L}_m$.
%Analogously, we denote by $R_m(x)$  the sum of all right-hand blocks which have their left boundary at point $x$ with $2m$ lines. 

Let us finally consider the central block $Z$. As compared to $L$ and $R$, this block has one additional line which connects points $\tilde{x}$ and $x$, i.e., for left and right block having $2m'$ and $2m$ retarded and advanced lines attached, the central block $Z_{m',m}$ connecting  them has  $2m'+1$ lines at its left boundary and $2m+1$ lines at its right boundary. For instance, the diagram in Fig.~\ref{fig:contributing_diagram}(b) has $2m'+1=3$ and $2m+1=1$. 
% Figure environment removed
To derive a differential equation for $Z_{m',m}(\tilde{x},x)$, we have to make an assumption on the direction of the extra line. Its type depends on the sign of $x-\tilde{x}$ and introduces a kind of asymmetry because our problem differentiates left and right directions. Here, we assume $\tilde{x} - x < 0$, that is the additional line is going from left to right, like in Fig.~\ref{fig:contributing_diagram}(b).
%In the following to simplify notations, we leave out $m'$ and $\tilde{x}$ in $Z$: $Z_{m',m}(\tilde{x},x)\to Z_{\cdot,m}(,x)$ because we analyze only the right-side of the central block. 
The total derivative of $Z_{m',m}(\tilde{x},x)$ with respect to $x$ includes the contributions from scattering vertices but also it has to include the derivative of the final vertex. These final vertices are analyzed analogously to the initial vertices. Out of four possibilities only two are phaseless, and thus contribute to the final sum of diagrams. They correspond to vertices with lines incoming only from a single direction, i.e., both from left or both from right. The list of all phaseless initial and final vertices is summarized in Fig.~\ref{fig:initial_final_vertices}, together with their corresponding weights, denoted by $\Gamma_{\pm,.}$ and $\Gamma_{.,\pm}$ for initial and final vertices, respectively.

Computing the total derivative of the central block at the final point $x$,  assuming $\tilde{x} < x$, we find:
\begin{align}\label{eq:Z_m}
    &\dfrac{ Z_{m',m}(\tilde{x},x)}{x} =\nonumber \\
   & \pm \frac{i\omega}{v_\pm}Z_{m',m}(\tilde{x},x) - \frac{\eta}{\hbar^2 v_+ v_-} \left(2m^2\kappa + 2m + 1\right)Z_{m',m}(\tilde{x},x) \nonumber\\
    &+\frac{\eta \kappa}{\hbar^2 v_+v_-}\left[(m+1)^2 Z_{m',m+1}(\tilde{x},x)e^{i\omega x(\frac{1}{v_+} + \frac{1}{v_-})} \right. \nonumber\\
    &+\left.  m^2 Z_{m',m-1}(\tilde{x},x)e^{-i\omega x(\frac{1}{v_+} + \frac{1}{v_-})} \right].
\end{align}
The sign of the first term in the right-hand side depends on the final vertex type, i.e., $\Gamma_{\cdot, +}$ or $\Gamma_{\cdot, -}$.
It turns out, on the other hand, that this expression does not depend on the sign of $\tilde{x} - x$. Note that when $v_+ = v_-$ and $\kappa = 1$, Eqs.~(\ref{eq:l_m_full}) and (\ref{eq:Z_m}) reduce to the known spinless TRI case \cite{Berezinskii1996}. 
While we are not aware of any analytic solution for the differentio-recursive equation (\ref{eq:Z_m}), in general the direct knowledge of the full function $Z_{m',m}(\tilde{x},x)$ is not required for the computation of observables. An example of this will be given in the next section, when discussing the quantum boomerang effect.
%Unfortunately, we are not aware of any analytic solution for the recursion relation Eq.~(\ref{eq:Z_m}). However, it can be solved using a systematic quasi-analytic method to calculate the quantum boomerang effect in a system without time-reversal symmetry \cite{Janarek2022}.

We conclude this section by expressing the Green's function correlator in Eq. (\ref{eq:initial_vertex_simplification}) in terms of the blocks $L$, $R$ and $Z$ described above. For $\tilde{x}<x$, and if we suppose that the initial wave function only populates the state with initial velocity $v_+$ (this is the practical case that will be considered in Sec.  \ref{sec:qbe}), the correlator 
% \com{[NC: In the weights indicated in Fig. 4 there are $\hbar$ factors, which are absent in the equation below... Should we add these $\hbar^2$ factors in Eq. (22)??]}
\begin{align}
\label{eq:GRG1final}
\overline{G^R(x,\tilde{x},\epsilon)G^A(\tilde{x},x,\epsilon\!-\!\hbar\omega)}\!=\!\frac{1}{\hbar^2 v_+^2}\Gamma_{+,+}^{\tilde{x}<x}\!+\!\frac{1}{\hbar^2 v_+v_-}\Gamma_{+,-}^{\tilde{x}<x}
\end{align}
is the sum of two contributions corresponding to the two possible final vertices c and d in Fig. \ref{fig:initial_final_vertices}, with
\begin{align}
    &\Gamma_{+,+}^{\tilde{x}<x}=
    \sum_{m,m'=0}^\infty
    \mathcal{L}_{m'}(\tilde{x})Z_{m',m}(\tilde{x},x)\mathcal{R}_m(x)\\
    &\Gamma_{+,-}^{\tilde{x}<x}=
    \sum_{m,m'=0}^\infty
    \mathcal{L}_{m'}(\tilde{x})Z_{m',m}(\tilde{x},x)\mathcal{R}_{m+1}(x).
\end{align}
% \begin{itemize}
%     \item division of big diagramms into left/right central parts. Write explain how many outgoing/incoming lines etc
%     \item derive equations for R/L/Z
%     \item show (?) the differences and similarities to the original TRI Berezinskii
% \end{itemize}
In the opposite case $\tilde{x}>x$, finally, Eq. (\ref{eq:GRG1final}) still holds but with $\Gamma_{+,\pm}^{\tilde{x}<x}$ changed to
\begin{align}
    &\Gamma_{+,+}^{\tilde{x}>x}=
    \sum_{m,m'=0}^\infty
    \mathcal{L}_{m'+1}(x)Z_{m',m}(x,\tilde{x})\mathcal{R}_{m+1}(\tilde{x})\\
   & \Gamma_{+,-}^{\tilde{x}>x}=
    \sum_{m,m'=0}^\infty
    \mathcal{L}_{m'}(x)Z_{m',m}(x,\tilde{x})\mathcal{R}_{m+1}(\tilde{x}).
    \label{eq:gammapm}
\end{align}
Together with the solution of Eq. (\ref{eq:Z_m}), Eqs. (\ref{eq:GRG1final}--\ref{eq:gammapm}) constitute the final solution of the localization problem. In the next section, we apply this formalism to access the time evolution of a particular observable, the mean position of wave packets, featuring the quantum boomerang effect.



\section{Quantum boomerang effect without time-reversal symmetry}
\label{sec:qbe}

We now apply the above formalism to the theoretical  description of a concrete problem, the quantum boomerang effect (QBE). We recall that the QBE describes a back-and-forth motion of the mean position of a quantum particle  launched with nonzero initial velocity in a disordered potential. Here we describe this phenomenon based on the TRI-broken Hamiltonian $H_0+V$, with the free part $H_0$ defined by Eq. (\ref{eq:h0}).

\subsection{Mean position}

To describe the QBE within the Berezinskii technique, we consider for definiteness a wave packet initially launched in a disordered potential with the mean eigen wave number $k_+$ of the Hamiltonian (\ref{eq:h0}) in the corresponding spin state $|\chi_+\rangle$. We denote by $\epsilon_0=E_-(k_+)$ the associated energy. 
We thus write the initial wave function as
\begin{equation}
\label{eq:initialWF}
    \Psi_0(x)=\frac{1}{(\pi\sigma^2)^{1/4}}
    \exp\left(-\frac{x^2}{2\sigma^2}+i k_+x\right)
    |\chi_+\rangle,
\end{equation}
where $\sigma$ is the wave-packet width. 
%and $|\chi_+\rangle$ denotes the (eigen) spin state with corresponding wave number $k_+$. 
As explained in the previous sections, the ensuing dynamics of this state in the disorder gives rise to a coupling with the backward-propagating state of wave number $k_-$ and spin component $|\chi_-\rangle$. 

By definition, the disorder-average mean position is
\begin{equation}
        \langle x(t) \rangle \equiv \int \diff{x} x \overline{|\psi(x,t)|^2}.
\end{equation}
Using that $\psi(x,t) = \int\diff{x'}G^R(x,x',t)\Psi_0(x')$,
we can relate its Fourier transform $\langle x(\omega)\rangle=\int dt e^{i\omega t} \langle x(t)\rangle$ to the Green's function correlator as
% % The center of mass is thus expressed in terms of the product of retarded and advanced Green's functions:
% \begin{equation}\label{eq:com_with_greens}
% \begin{split}
%     \langle x(t)\rangle = \int\diff{x}\diff{x'}\diff{x''} x\, \overline{G^R(x, x', t)G^A(x'', x, -t)} \times \\
%     \times \Psi_0(x')\Psi_0(x'')^*.
% \end{split}
% \end{equation}
\begin{equation}
\begin{split}\label{eq:com_freq1}
   \langle x (\omega)\rangle  &= \frac{1}{2\pi\hbar} \int \diff{x}\diff{x'}\diff{x''}\diff{\epsilon} \Psi_0(x')\Psi^*_0(x'')\times \\ &x\, \overline{G^R(x, x', \epsilon)G^A(x'', x, \epsilon - \hbar \omega)},
\end{split}
\end{equation}
where we expressed the retarded and advanced Green's functions in the Fourier domain.
To simplify this expression, we make use of  Eq.~(\ref{eq:initial_vertex_simplification}), which leads to 
\begin{equation}
    \begin{split}
        \langle x(\omega)\rangle\! =\!\! \int\diff{x}\diff{\tilde{x}}\diff{\epsilon} x\, W(\tilde{x}, k_\epsilon) 
\overline{G^R(x,\tilde{x},\epsilon)G^A(\tilde{x},x,\epsilon\!-\!\hbar\omega)}\nonumber
    \end{split}
\end{equation}
with $W$ the Wigner distribution of the initial state:
\begin{equation}
    2\pi\hbar W(\tilde{x}, k_\epsilon)
    =\int dr e^{-i k_\epsilon r}
    \Psi_0(\tilde{x}+r/2)\Psi_0^*(\tilde{x}-r/2).
\end{equation}
For an initial wave function (\ref{eq:initialWF}) of spatial width $\sigma$ much smaller than the mean free path, we find $W(\tilde{x}, k_\epsilon)\approx \hbar^{-1}\delta(\tilde{x})\delta(k_\epsilon-k_+)$, such that, eventually,
%As argued in Sec.~\ref{sec:diagrammatic_approach}, not all diagrams contribute to the final average product of Green's functions. The first simplification used in our calculation is based on the choice of the initial vertices. Combining Eq.~(\ref{eq:initial_vertex_simplification}) and Eq.~(\ref{eq:com_freq1}) yields Wigner's functions for each of the two possible initial vertices. We remind that for the initial state with energy $\epsilon_0$, we have two possible momenta $k_+$ and $k_-$ with positive and negative initial velocities. For vertex (a) in Fig.~\ref{fig:initial_final_vertices}:
%\begin{equation}
 %   \begin{split}
  %      &\langle x(\omega)\rangle_{+} =\!\! \int\diff{x}\diff{\tilde{x}}\diff{\epsilon} \times\\ 
   %     &x\, W^{+}(\tilde{x}, k_+) \overline{G^R(x,\tilde{x},\epsilon)G^A(\tilde{x},x,\epsilon-\hbar\omega)}.
   % \end{split}
%\end{equation}
%Similarly, for vertex (b) Fig.~\ref{fig:initial_final_vertices}:
%\begin{equation}
%    \begin{split}
%        &\langle x(\omega)\rangle_{-} =\!\! \int\diff{x}\diff{\tilde{x}}\diff{\epsilon} \times\\ 
 %%       &x\, W^{-}(\tilde{x}, k_-) \%overline{G^R(x,\tilde{x},\epsilon)G^A(\tilde{x},x,\epsilon-\hbar\omega)}.
%    \end{split}
%\end{equation}
%Finally, to compute the expression for the center of mass, we introduce the spectral function $A(\epsilon,k) = \delta(\epsilon - \epsilon_k)$. In the weak disorder limit, we can use the free spectral function:
%\begin{equation}
%    A(\epsilon, k) = (\hbar v_+)^{-1}\delta(k-k_+) + (\hbar v_-)^{-1}\delta(k-k_-).
%\end{equation}
%It allows us to gather both initial states into one %integral:
%\begin{equation}
 %   W^\pm(\tilde{x},k_\pm) = \pm \hbar v_\pm \int_{\alpha}^{\pm\infty}\diff{k} W(\tilde{x},k)A(\epsilon,k),
%\end{equation}
%where $\alpha$ is an arbitrary constant such that $k_- < \alpha < k_+$, used to split contributions of $k_+$ and $k_-$. In the original study of the QBE \cite{Prat2019}, the calculation of CoM was started from $\langle x^2 \rangle$ and exploited Ehrenfest theorem, what was not favorable in our case. Despite this, we arrive with a final formula quite similar to the TRI case, e.g. for a positive velocity initial state we have:
\begin{equation}\label{eq:com_freq2}
    \langle x(\omega)\rangle \!=\! v_+\!\int_{-\infty}^\infty\!\!\!\!
    \diff{\Delta x}
    \Delta x\, \overline{G^R(x,\tilde{x},\epsilon_0)G^A(\tilde{x},x,\epsilon_0\!-\!\hbar\omega)},
\end{equation}
where for convenience we replaced the integral over $x$ by an integral over $\Delta x\equiv x-\tilde{x}$, using that the integrand depends only on $x-\tilde{x}$ due to statistical translational invariance. Equation (\ref{eq:com_freq2}) directly connects the average mean position to the Green's function correlator, which we now compute using the results of the previous section.



\subsection{Time evolution of the boomerang effect}

%We now come back to the computation of the mean position. 
Inserting the general Berezinskii result (\ref{eq:GRG1final}) into Eq. (\ref{eq:com_freq2}), we infer
\begin{equation}
    \langle x(\omega)\rangle=\langle x(\omega)\rangle_++\langle x(\omega)\rangle_-,
\end{equation}
where
\begin{equation}\label{eq:com_+}
    % \langle x(\omega)\rangle_{+} = \frac{\ell}{v_+}\sum_{m'} \left(\mathcal{L}_{m'} S^0_{m'} + \mathcal{L}_{m'+1}S^1_{m'}\right)
        \langle x(\omega)\rangle_{+} = \frac{2\ell}{v_+}\sum_{m'} \left(\mathcal{L}_{m'} S^0_{m'} + \mathcal{L}_{m'+1}S^1_{m'}\right)
\end{equation}
is the contribution of velocities $v_+$ [technically, of the final vertex c in Fig. \ref{fig:initial_final_vertices}], and
\begin{equation}\label{eq:com_-}
    % \langle x(\omega)\rangle_{-} = \frac{\ell}{v_-}\sum_{m'} \left(\mathcal{L}_{m'} S^2_{m'} + \mathcal{L}_{m'}S^3_{m'}\right)
        \langle x(\omega)\rangle_{-} = \frac{2\ell}{v_-}\sum_{m'} \left(\mathcal{L}_{m'} S^2_{m'} + \mathcal{L}_{m'}S^3_{m'}\right)
\end{equation}
is the contribution of velocities $v_-$ [final vertex d in Fig. \ref{fig:initial_final_vertices}]. Notice that we here introduced for convenience the mean free path $\ell=\tau_+v_+=\tau_-v_-$. In Eqs. (\ref{eq:com_+}) and (\ref{eq:com_-}), 
the two terms in the right-hand side are the contributions of $\tilde{x}<x$ and $\tilde{x}>x$, respectively, with the coefficients $\mathcal{L}_m$ defined by Eq. (\ref{eq:Lmfinal}). The quantities $S_m^i$, on the other hand, are given by spatial integrals of the block functions $\mathcal{R}_m$ and $Z_{m',m}$. For instance, we have
\begin{equation}
    \begin{split}
    S_{m'}^0=\frac{1}{2\ell}\sum_m &\int_0^\infty d\Delta x \Delta x e^{-im'\omega \tilde{x}(\frac{1}{v_+}+\frac{1}{v_-})}\\
    &\times Z_{m',m}(\tilde{x},x)
    e^{im\omega x(\frac{1}{v_+}+\frac{1}{v_-})}\mathcal{R}_m.
\end{split}
\end{equation}
%which can be re-expressed as
%\begin{equation}
 %   \begin{split}
  %  S_{m'}^0=\frac{1}{\ell}\sum_m \int_{-\infty}^x& d\tilde{x}(x-\tilde{x}) e^{-im'\nu \tilde{x}/2\ell}\\
   % &\times Z_{m',m}(\tilde{x},x)R_m(x),
%\end{split}
%\end{equation}
%after a change of variable and introduction of the %reduced frequency $\nu=\omega(v_++v_-)/(\kappa\eta)$.
To compute the coefficients $S_{m'}^0$, we perform a partial integration in the right-hand side and use Eq. (\ref{eq:Z_m}) to express the derivative $dZ_{m',m'}(\tilde{x},x)/d\tilde{x}$ in terms of $Z_{m',m}$. This provides us with the iterative equation
\begin{equation}
\label{eq:iterativeS0m}
\begin{split}
    % \ell Q_m^0&+i\nu\left(m+\frac{v_-}{v_++v_-}\right)S_m^0-\ell \eta\beta_m S_m^0\\
    % &+m^2 S_{m-1}^0+(m+1)^2 S_{m+1}^0=0
        2\ell Q_m^0&+i\nu\left(m+\frac{v_-}{v_++v_-}\right)S_m^0-2\ell \eta\beta_m S_m^0\\
    &+m^2 S_{m-1}^0+(m+1)^2 S_{m+1}^0=0
\end{split}
\end{equation}
where $\beta_m\equiv (2\kappa m^2+2m+1)/(\hbar^2v_+ v_- )$ and we remind that $\nu\equiv \omega(v_++v_-)\hbar^2/(\kappa\eta)$. The coefficient $Q_m^0$ is defined as 
\begin{equation}
    \begin{split}
    % Q_{m}^0=\frac{1}{\ell}\sum_m \int_0^\infty& d\Delta x e^{-im'\omega \tilde{x}(\frac{1}{v_+}+\frac{1}{v_-})}\\
    % &\times Z_{m',m}(\tilde{x},x)
    % e^{im\omega x(\frac{1}{v_+}+\frac{1}{v_-})}\mathcal{R}_m
        Q_{m}^0=\frac{1}{2\ell}\sum_m \int_0^\infty& d\Delta x e^{-im'\omega \tilde{x}(\frac{1}{v_+}+\frac{1}{v_-})}\\
    &\times Z_{m',m}(\tilde{x},x)
    e^{im\omega x(\frac{1}{v_+}+\frac{1}{v_-})}\mathcal{R}_m
\end{split}
\end{equation}
and is deduced from an iterative equation similar to Eq. (\ref{eq:iterativeS0m}):
\begin{equation}
\label{eq:iterativeQ0m}
\begin{split}
    \mathcal{L}_m&+i\nu\left(m+\frac{v_-}{v_++v_-}\right)Q_m^0-2\ell \eta\beta_m Q_m^0\\
    &+m^2 Q_{m-1}^0+(m+1)^2 Q_{m+1}^0=0.
\end{split}
\end{equation}
The coupled system of equations (\ref{eq:iterativeS0m}) and (\ref{eq:iterativeQ0m}) is closed, so that at a formal level it  can in principle be  solved to find the coefficients $S_m^0$ and, in turn, to compute the first sum in the right-hand side of Eq. (\ref{eq:com_+}). The calculation of the coefficients $S_m^1$, $S_m^2$ and $S_m^3$ that appear in Eqs. (\ref{eq:com_+}) and (\ref{eq:com_-}) follows the same lines. We provide the corresponding iterative equations they obey in Appendix \ref{Appendix:iterative} for the sake of completeness.
%Now, we can combine Eq.~(\ref{eq:com_freq2}) and the analysis of diagrams from Sec.~\ref{sec:equations_for_diagrams}. All diagrams can be divided into four families, depending on the initial and final vertices. We denote them by $\Gamma_{\pm,\pm}(x, \tilde{x})$. The connection between $\Gamma_{\pm,\pm}(x,\tilde{x})$ and parts of diagrams $L$, $Z$, and $R$ is not completely trivial. It depends on the sign of $x-\tilde{x}$. For example if we focus only on diagrams with positive initial and final velocities, for $\tilde{x} < x$:
%\begin{equation}
 %   \Gamma^{\tilde{x}<x}_{+,+} = \sum_{\substack{m=0, \\ m'=0}}^\infty L_{m'}(\tilde{x})Z_{m',m}(\tilde{x},x)R_m(x),
%\end{equation}
%while for $\tilde{x} > x$:
%\begin{equation}
 %   \Gamma^{\tilde{x}>x}_{+,+} = \sum_{\substack{m=0, \\ m'=0}}^\infty L_{m'+1}(\tilde{x})Z_{m',m}(\tilde{x},x)R_{m+1}(x).
%\end{equation}
%Using Eq.~(\ref{eq:com_freq2}) we can now write calculate contribution of diagrams with positive final and initial velocities to  the CoM:
%\begin{equation}
 %   \langle x(\omega)\rangle_{++}^{\tilde{x}<x} = \frac{v_+}{v_+^2}\int_0^\infty\diff{(x-\tilde{x})}(x-\tilde{x})\Gamma^{\tilde{x}<x}_{+,+},
%\end{equation}
%where the additional factor $v_+^{-2}$ comes from the weights of the initial and final vertices. Using definition of $\Gamma^{\tilde{x}<x}_{+,+}$ and equations for $L$, $Z$, and $R$, Eqs.(\ref{eq:l_m_algebraic}--\ref{eq:Z_m}) we can get the final result shown in \cite{Janarek2022}:
%\begin{equation}\label{eq:com_++}
%    \langle x(\omega)\rangle_{++} = \tau_+\sum_m \left(L_m S^0_m + L_{m+1}S^1_m\right),
%\end{equation}
%where $S^i$ are sum-integrals that can be computed in an iterative way \cite{Janarek2021}. The contribution from the family $\Gamma_{+,-}$ is gathered in very similar way:
%\begin{equation}\label{eq:com_+-}
 %   \langle x(\omega)\rangle_{+-} = \tau_-\sum_m \left(L_m S^2_m + L_{m}S^3_m\right),
%\end{equation}
%where $\tau_\pm$ are scattering mean free times between eigenstates at energy $\epsilon_0$: $\tau_+=\tau_{+\to-}$, $\tau_- = \tau_{-\to+}$, with $\tau_\pm = \hbar^2 v_\mp/\eta\kappa$ \cite{Janarek2022}.

% Figure environment removed

Instead of exactly computing all the $S_m^i$ coefficients, %appearing in Eqs. (\ref{eq:com_+}) and (\ref{eq:com_-}), 
however, the mean position can be conveniently evaluated from its short-time expansion using a Pad\'e approximant \cite{Prat2017a, Janarek2021}. The short-time expansion of $\langle x(t)\rangle_+$ and $\langle x(t)\rangle_-$ is systematically obtained by inserting the series $S_m^i(\nu)=\sum_n s^i_{m,n}/(i\nu)^n$ and $Q_m^i(\nu)=\sum_n q^i_{m,n}/(i\nu)^n$ in the iteratives relations (\ref{eq:iterativeS0m}), (\ref{eq:iterativeQ0m}) and (\ref{eq:iterativeS1m})--(\ref{eq:iterativeQ3m}), and computing the $s_{m,n}^i$ and $q_{m,n}^i$ coefficients at arbitrary order in $1/\nu$. This procedure eventually yields the following short-time expansion for the mean position:
\begin{equation}
\label{eq:com_quantum_short_time}
    \begin{split}
        &\frac{\langle x(t)\rangle}{v_+\tau} =  \frac{t}{\tau} - \frac{t^2}{2\tau^2} +  \frac{t^3}{6\tau^3} \\
        & - \frac{(1+\Delta(4+\Delta(8+\Delta(4+\Delta))))t^4}{24(1+\Delta)^4 \tau^4} + \mathcal{O}(t^5)
    \end{split}
\end{equation}
where $\Delta\equiv v_-/v_+$ and $\tau$ is defined by Eq. (\ref{eq:taudef}). In Appendix \ref{Appendix:short-time}, we also provide the corresponding expansions for the partial components $\langle x(t)\rangle_+$ and $\langle x(t)\rangle_-$ that respectively describe right and left-moving particles after the last scattering event. 
In Fig. \ref{fig:com_short_time} we show a comparison between an exact numerical calculation of $\langle x(t)\rangle_\pm$ based on a temporal wave-packet propagation with the disordered Schrödinger equation (details on the numerical simulations are given in the figure caption), and the short-time expansion up to order 11 obtained by solving the Berezinskii equations as explained above.  Numerical and theoretical results are in very good agreement without any fit parameter up to $t/\tau\approx 3$. This corresponds to a finite radius of convergence in time, which is also present in the  TRI version of the quantum boomerang effect \cite{Prat2019}. This radius seems to be dependent of the ratio of velocities $\Delta$. Most importantly, as indicated by Eq. (\ref{eq:com_quantum_short_time}), the expression of $\langle x(t)\rangle$ is no longer universal because it depends on the velocities' ratio starting from the 4th order. This  is a significant difference with the TRI quantum boomerang effect, which solely depends on the dimensionless time scale $t/\tau$ at all times. The TRI solution is fully recovered when $v_+ = v_-=v$.

It is also instructive to compare the exact, quantum-mechanical short-time expansion (\ref{eq:com_quantum_short_time}) with the classical prediction of the Boltzmann equation, which discards any interference in the multiple scattering process. At a classical level, the mean position is simply given by $\langle x(t)\rangle^\text{class.} = \tau v_+(1-e^{-t/\tau})$, which is essentially the same expression as in TRI systems (see the supplemental material of \cite{Janarek2022} for details on the classical calculation). This classical result has the short-time expansion
\begin{equation}
\begin{split}
    \frac{\langle x(t) \rangle^\text{class.}}{v_+\tau} = \frac{t}{\tau} - \frac{t^2}{2\tau} + \frac{t^3}{6\tau^3} - \frac{t^4}{24\tau^4} + \mathcal{O}(t^5)
\end{split}
\end{equation}
which starts to deviate from the quantum-mechanical prediction (\ref{eq:com_quantum_short_time}) starting from the 4th order. For completeness, in Appendix \ref{Appendix:short-time} we also provides the short-time expansions for the classical components $\langle x(t)\rangle_+^\text{class.}$ and $\langle x(t) \rangle_-^\text{class.}$.
% Figure environment removed

With the short-time expansion (\ref{eq:com_quantum_short_time}) at hand, we can infer the $\langle x(t)\rangle$ using a Pad\'e approximant of the full Taylor series \cite{Baker1975}. To this aim we use that, at long time, $\langle x(t)\rangle\propto 1/t^2$ (see below). With this knowledge, we compute the mean position at any time using
\begin{equation}
    \langle x(t)\rangle=v_+\tau\left(\frac{\tau}{t}\right)^2\underset{n\to\infty}{\lim} A_n(t)
\end{equation}
where $A_n(t)$ is a diagonal Pad\'e approximant \cite{Baker1975} whose coefficients are computed from the Taylor expansion at a desired order $n$ [e.g., Eq. (\ref{eq:com_quantum_short_time}) for $n=4$]. In Fig. \ref{fig:com_full_pade_with_class} we compare the exact numerical simulations for $\langle x(t)\rangle_+$, $\langle x(t)\rangle_-$ and $\langle x(t)\rangle$ to the corresponding  Pad\'e approximants contructed from the Berezinskii technique. The plots reveal the QBE: after a few mean free times, the mean position exhibits a maximum and eventually decays to zero. For all quantities, we find a very good agreement between the simulations and the Berezinskii approach up to long times.

Let us finally come back to the long-time behavior of $\langle x(t)\rangle$. The latter is most easily visualized in the inset of Fig. \ref{fig:com_full_pade_with_class}, which shows the mean position obtained from numerical simulations of the Schr\"odinger equation in log-log scale. We find that its long-time asymptotics is well approximated by a function scaling as $\alpha \log(\beta t/\tau)/(t/\tau)^2$, which is of the same form as in spinless TRI Hamiltonians \cite{Prat2017a}. In the present case of Hamiltonian (\ref{eq:h0}), however, a direct derivation of this asymptotic limit appears to be much more involved, and is left for future work.




\section{Conclusion}

In this paper, we have extended the Berezinskii diagrammatic technique describing the dynamics of Anderson localization in one dimension to TRI-broken disordered Hamiltonians including a spin-orbit coupling term that induces an asymmetry between right and left scattering processes. As an application of the formalism, we have computed the time evolution of the mean position of a wave-packet launched in a given direction, and recovered the quantum boomerang effect discussed in \cite{Janarek2022}. As an extension of this work, it would be interesting to extract analytical long-time, asymptotic expansions for the mean position in this system, or to characterize the dynamics of other observables such as the mean square width $\langle x^2(t)\rangle$ or the full density distribution of the wave packet.

\acknowledgments

We thank Tony Prat for useful discussions on the Berezinskii technique. N.C. acknowledges financial support from the Agence Nationale de la Recherche (grant ANR-19-CE30-0028-01 CONFOCAL). J.J. acknowledges the support of French Embassy in Poland through the \emph{Bourse du Gouvernement Fran\c{c}ais} program.

\appendix

\section{Identification of scattering vertices}
\label{Appendix:vertices}

In this appendix, we briefly explain the procedure used to identify the set of phaseless scattering vertices in Fig. \ref{fig:scattering_vertices}. To this aim, let us consider the diagram in Fig.~\ref{fig:contributing_diagram}(a) once more. The diagram can be split into spatial intervals lying between consecutive scattering events $x_i$ and $x_{i+1}$. Each interval contains a specific number of lines. There are in total 4 kinds of lines: retarded lines  and advanced lines (both in two possible directions).
The numbers of lines are denoted as $g_+$ and $g_-$ (for retarded lines), and $(g_+)'$ and $(g_-)'$ (for advanced lines), with the index $\pm$ indicating their direction. For example, the interval lying between the points $x'$ and $x_3$ in the diagram from Fig.~\ref{fig:contributing_diagram}(a) has $g_+ = 2,\ g_-=1,\ (g_+)'=2,$ and $(g_-)'=1$ lines. Each scattering event induces a definite change in the number of respective lines, which we denote by $\Delta g_\pm$  and $(\Delta g_\pm)'$. These changes determine the phases associated with scattering vertices. To find these phases, we first note that each incoming and outgoing retarded propagator line at point $x$ carries a phase that depends on the direction of the line and on its type:
\begin{itemize}
    \item every incoming (outgoing) \emph{positive} line, i.e. propagating to the right, carries a $k_+x$ ($-k_+x)$ phase,
    \item every incoming (outgoing) \emph{negative} line, i.e. propagating to the left, carries a $-k_-x$ ($k_-x)$ phase.
\end{itemize} 
% For the \emph{negative} (that is propagating to left) lines, the phases for incoming (outgoing) lines are $-ik_-x$ ($ik_-x$). 
For advanced lines, the phases have opposite signs. 

%The vertices with nonzero phases are negligible due to disorder averaging: different phases average to zero.
For vertices involving only one type of lines, e.g., only retarded Green's functions, the total phase $\phi$ of a scattering vertex is then calculated from the total change of the number of lines, i.e.: 
\begin{equation}
\phi=\pm (\Delta g_+ k_+ - \Delta g_- k_-)x.
\end{equation}
Hence, the phaselessness condition of the scattering vertex in the limit of $\omega\to 0$ is that the vertex does not change the total number of incoming and outgoing lines, that is, $\Delta g_\pm = 0$. This condition is very similar to the original TRI Berezinskii method, although in our system $k_+$ and $k_-$ do not cancel each other.
In the case of the mixed-line vertices involving both $G_0^R$ and $G_0^A$, the problem is slightly different: lines from $G^R_0$ and $G^A_0$ may cancel each other. The total phase of a vertex is
\begin{equation}
    \phi = [\left(\Delta g_+ - (\Delta g_+)'\right)k_+ - \left(\Delta g_- - (\Delta g_-)'\right)k_-]x.
\end{equation}
This phase is zero only if $\Delta g_\pm - (\Delta g_\pm)' = 0$. 

\section{Differential equation for $L_m$ blocks}
\label{Appendix:Lmblock}

% Figure environment removed
In this appendix, we provide details about the derivation of the differential equation (\ref{eq:l_m_full}) for the left blocks of correlation diagrams. 
To calculate $L_{m'}(\tilde{x})$, we consider how it changes with an infinitesimal change of the boundary position, say $\tilde{x}\rightarrow \tilde{x}+\delta x$ by adding all possible contributions from different scattering vertices. For this purpose, we number the lines on the boundary by assigning consecutive numbers to the outgoing and incoming lines, as presented in Fig.~\ref{spin:fig:lm_construction}. The figure also shows a schematic way of all adding new vertices to  $L_{m'}(\tilde{x})$, bearing in mind that the lines cannot create loops nor cross each other. We get the corresponding equation:
\begin{widetext}
\begin{equation}
\label{spin:eq:l_m_combinatorics}
    \begin{aligned}
        &L_{m'}(\tilde{x}+\delta x) = L_{m'}(\tilde{x}) +\frac{\eta\delta x}{2} L_{m'}(\tilde{x})\times(-2m')\left(\frac{1}{v_+}+\frac{1}{v_-}\right)^2
        \\[5pt]
        &+\eta\delta x L_{m'}(\tilde{x}) \biggl[- \frac{m'(m'-1)}{v_+^2} - \frac{m'(m'-1)}{v_-^2} -\frac{2m'^2}{v_+v_-} -\frac{2\kappa m'(m'-1)}{v_+v_-} + \frac{m'^2}{v_+^2}+\frac{m'^2}{v_-^2}+\frac{2m'^2}{v_+v_-}\biggr]\\[5pt]
        &+\frac{\eta\delta x \kappa}{v_+ v_-} \biggl[m'^2L_{m'+1}(\tilde{x})e^{i\omega \tilde{x}(\frac{1}{v_+}+\frac{1}{v_-})} + m'^2L_{m'-1}(\tilde{x})e^{-i\omega \tilde{x}(\frac{1}{v_+}+\frac{1}{v_-})}\biggr].
    \end{aligned}
\end{equation}
\end{widetext}
After taking the limit $\delta x \rightarrow 0$ and some simplifications, we finally obtain Eq. (\ref{eq:l_m_full}) of the main text.




\section{Iterative Berezinskii equations}
\label{Appendix:iterative}


In this appendix, we provide the coupled equations for for the $S_m^i$ coefficients ($i=1,2,3$) that appear in the expressions of the mean position, Eqs. (\ref{eq:com_+}) and (\ref{eq:com_-}) of the main text.

Using the same procedure as for $S_m^0$ and explained in the main text, we find the following coupled iterative equations for the $S_m^i, Q_m^i$ ($i=1,2,3$):
\begin{equation}
\label{eq:iterativeS1m}
\begin{split}
    % -\ell Q_m^1&+i\nu\left(m+\frac{v_+}{v_++v_-}\right)S_m^1-\ell \eta\beta_m S_m^1\\
    % &+m^2 S_{m-1}^1+(m+1)^2 S_{m+1}^1=0,
        -2\ell Q_m^1&+i\nu\left(m+\frac{v_+}{v_++v_-}\right)S_m^1-2\ell \eta\beta_m S_m^1\\
    &+m^2 S_{m-1}^1+(m+1)^2 S_{m+1}^1=0,
\end{split}
\end{equation}
\begin{equation}
\label{eq:iterativeQ1m}
\begin{split}
    % \mathcal{L}_{m+1}&+i\nu\left(m+\frac{v_+}{v_++v_-}\right)Q_m^1-\ell \eta\beta_m Q_m^1\\
    % &+m^2 Q_{m-1}^1+(m+1)^2 Q_{m+1}^1=0,
    \mathcal{L}_{m+1}&+i\nu\left(m+\frac{v_+}{v_++v_-}\right)Q_m^1-2\ell \eta\beta_m Q_m^1\\
    &+m^2 Q_{m-1}^1+(m+1)^2 Q_{m+1}^1=0,
\end{split}
\end{equation}
\begin{equation}
\label{eq:iterativeS2m}
\begin{split}
    % \ell Q_m^2&+i\nu\left(m+\frac{v_-}{v_++v_-}\right)S_m^2-\ell \eta\beta_m S_m^2\\
    % &+m^2 S_{m-1}^2+(m+1)^2 S_{m+1}^2=0,
    2\ell Q_m^2&+i\nu\left(m+\frac{v_-}{v_++v_-}\right)S_m^2-2\ell \eta\beta_m S_m^2\\
    &+m^2 S_{m-1}^2+(m+1)^2 S_{m+1}^2=0,
\end{split}
\end{equation}
\begin{equation}
\label{eq:iterativeQ2m}
\begin{split}
    % \mathcal{L}_{m+1}&+i\nu\left(m+\frac{v_-}{v_++v_-}\right)Q_m^2-\ell \eta\beta_m Q_m^2\\
    % &+m^2 Q_{m-1}^2+(m+1)^2 Q_{m+1}^2=0,
        \mathcal{L}_{m+1}&+i\nu\left(m+\frac{v_-}{v_++v_-}\right)Q_m^2-2\ell \eta\beta_m Q_m^2\\
    &+m^2 Q_{m-1}^2+(m+1)^2 Q_{m+1}^2=0,
\end{split}
\end{equation}
\begin{equation}
\label{eq:iterativeS3m}
\begin{split}
    % -\ell Q_m^3&+i\nu\left(m+\frac{v_+}{v_++v_-}\right)S_m^3-\ell \eta\beta_m S_m^3\\
    % &+m^2 S_{m-1}^3+(m+1)^2 S_{m+1}^3=0,
    -2\ell Q_m^3&+i\nu\left(m+\frac{v_+}{v_++v_-}\right)S_m^3-2\ell \eta\beta_m S_m^3\\
    &+m^2 S_{m-1}^3+(m+1)^2 S_{m+1}^3=0,
\end{split}
\end{equation}
\begin{equation}
\label{eq:iterativeQ3m}
\begin{split}
    % \mathcal{L}_{m+1}&+i\nu\left(m+\frac{v_+}{v_++v_-}\right)Q_m^3-\ell \eta\beta_m Q_m^2\\
    % &+m^2 Q_{m-1}^3+(m+1)^2 Q_{m+1}^3=0.
    \mathcal{L}_{m+1}&+i\nu\left(m+\frac{v_+}{v_++v_-}\right)Q_m^3-2\ell \eta\beta_m Q_m^2\\
    &+m^2 Q_{m-1}^3+(m+1)^2 Q_{m+1}^3=0.
\end{split}
\end{equation}

\section{Partial components $\langle x(t)\rangle_\pm$}
\label{Appendix:short-time}

We finally provide the short-time expansions for $\langle x(t)\rangle_+$ and $\langle x(t)\rangle_-$, and the exact expressions (valid at any time) of their classical counterparts $\langle x(t)\rangle^{\text{class.}}_+$ and $\langle x(t)\rangle^{\text{class.}}_-$:
\begin{align}
\label{eq:xplus_quantum}
        &\frac{\langle x(t)\rangle_+}{v_+\tau} =  \left[\frac{t}{\tau} - \frac{1}{1+\Delta}\frac{t^2}{\tau^2} +  \frac{3-\Delta}{6(1+\Delta)}\frac{t^3}{\tau^3} \right.  \\
        &\left. - \frac{\Delta(\Delta+1)(\Delta(\Delta^2+\Delta-3)-7)-2}{12(\Delta+1)^5}\frac{t^4}{\tau^4}\right] + \mathcal{O}(t^5),\nonumber
\end{align}
\begin{align}
\label{eq:xminus_quantum}
        &\frac{\langle x(t)\rangle_-}{v_+\tau} =  \left[\frac{1-\Delta}{2(1+\Delta)}\frac{t^2}{\tau^2} - \frac{1-\Delta}{3(1+\Delta)}\frac{t^3}{\tau^3} +  \right.  \\
        &\left. - \frac{\Delta(9-\Delta(\Delta(3\Delta(\Delta+3)+8)-8))+3}{24(\Delta+1)^5}\frac{t^4}{\tau^4}\right] + \mathcal{O}(t^5),\nonumber
\end{align}
\begin{align}
\label{eq:xplus_classical}
    \frac{\langle x(t)\rangle_{+}^{\text{class.}}}{\tau v_+} \!=&  \biggl[ \frac{2\Delta}{1 + \Delta}\bigl(1 - e^{-t/\tau}\bigr) + \frac{1 - \Delta}{1 + \Delta} \frac{t}{\tau}e^{-t/\tau} \biggr],
\end{align}
\begin{align}
\label{eq:xminus_classical}
    \frac{\langle x(t)\rangle_{-}^{\text{class.}}}{\tau v_+}\! =&  \frac{1- \Delta}{1+\Delta}\biggl(1 - e^{-t/\tau} - \frac{t}{\tau}e^{-t/\tau}\biggr).
\end{align}
\newline




\bibliography{biblio}



\end{document}


