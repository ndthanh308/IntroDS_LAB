\section{Three precision algorithms}
\label{sec:3p}

The three precision algorithms compute $\mf$ in double precision and
store the Jacobian $\mf'$ in a single precision matrix $\mj$. This means that
\begeq
\label{eq:jacerr}
\| \mf'(\vx_c) - \mj \| \le u_s \| \mf'(\vx_c) \|.
\endeq
Hence, using the terminology of \S~\ref{subsec:notation}
\begeq
\label{eq:errlist}
\epsilon_F = O(u_d) \mbox{ and } \epsilon_J = O(u_s).
\endeq

Our notation for interprecision transfers is to let 
$I_a^b$ be the transfer from precision $u_a$ to
$u_b$. If $u_a > u_b$, this promotion changes nothing
\[
I_a^b (x) = x
\]
if $x$ is in precision $u_a$. If $u_a < u_b$, then the interprecision
transfer rounds down, so
\[
\| I_a^b(x) - x \| \le u_b \| x \|.
\]
We will use these properties of interprecision transfer throughout
the remainder of the paper. We point out that when one rounds a
matrix or vector down to a lower precision, one must allocate memory
for the low precision object and that there is a cost to this.

We then round 
$\mj$ to half precision to obtain
\[
\mj_h = I_s^h (\mj)
\]
and factor $\mj_h$ in half precision to obtain
${\hat \ml} {\hat \mU}$.
We use the
half precision factorization as part of an iterative method to solve
\[
\mj \vs = - \mf(\vx_c).
\]
We terminate that iteration when 
\begeq
\label{eq:inexactJ}
\| \mj \vs + \mf(\vx_c) \| \le \eta_J \| \mf(\vx_c) \|.
\endeq

We summarize the three precision algorithm.
\begin{algo}
\label{alg:3p}
{$\mbox{\bf newton3p}(\mf, \vx, \tau_a, \tau_r, \eta_J)$}
\begin{algorithmic}
\STATE Evaluate ${\tilde \mf} = \mf(\vx) + \epsilon(\vx)$;  
\STATE $\tau \leftarrow \tau_r \| {\tilde \mf} \| + \tau_a$.
\WHILE{$\| {\tilde \mf} \| > \tau$}
\STATE Compute and store $F'(\vx)$ in single precision as $\mj$.
\STATE Store $\mj_h = I_s^h(\mj)$.
\STATE Find $\vs$ such that $\| \mj_h \vs + \mf(\vx) \| \le 
\eta_J \| \mf(\vx) \|$. 
\STATE $\vx \leftarrow \vx + \vs$
\STATE Evaluate ${\tilde \mf} = \mf(\vx) + \epsilon(\vx)$;
\ENDWHILE
\end{algorithmic}
\end{algo}
In Algorithm~\ref{alg:3p} we want to choose $\eta_J < 1$ small enough
so that the nonlinear iteration statistics are the same as those from
Newton's method itself.

Algorithm~\ref{alg:3p} looks like an inexact Newton iteration, but 
differs in that the condition on the step is \eqnok{inexactJ}
rather than the classical inexact Newton condition
\begeq
\label{eq:inexact}
\| \mf'(\vx_c) \vs + \mf(\vx_c) \| \le \eta \| \mf(\vx_c) \|.
\endeq
If we had \eqnok{inexact}, then we would get a local improvement 
estimate \cite{demboes,ctk:roots}
\begeq
\label{eq:inexloc}
\| \ve_+ \| = O(\| \ve_c \|^2 + \eta \|\ve_c\| + \epsilon_F).
\endeq
This will imply q-linear convergence of the nonlinear
iteration if $\eta$ is sufficiently small and the function
evaluation is exact ($\epsilon_F = 0$).

If we are able to show that we can chose $\eta_J$ so that \eqnok{inexact}
holds with $\eta = O(u_s)$, then, similar to the two precision case with
$\mj$ stored and factored in single precision, \eqnok{inexloc} will imply
\eqnok{errest2} and the nonlinear iteration
statistics will be the same as Newton's method with the Jacobian stored and
factored in double precision.

The use of an iterative method for the linear equation for the Newton
step means that the backward error in the 
factorization plays no role in the analysis of the nonlinear iteration.
However, that backward error
does affect the convergence of the linear iteration. We will describe
our two choices for the linear iteration in \S~\ref{sec:IR} but will
discuss the local improvement result for the
nonlinear iteration first.

\subsection{Local improvement of the nonlinear iteration}
\label{subsec:inexact}

We begin by showing that $\mj$ is nonsingular and estimating
$\| \mj^{-1} \|$. In the analysis we use the standard notation
\[
\kappa(\ma) = \| \ma \| \| \ma^{-1} \|
\]
for the condition number of a matrix $\ma$.

\begin{lemma}
\label{lem:mjok}

Assume that the standard assumptions \eqnok{close} hold and that 
\begeq
\label{eq:condok}
4 u_s \kappa(\mf'(\vx^*) ) < 1.
\endeq

Then $\mj$ is nonsingular and
\begeq
\label{eq:mjok}
\| \mj^{-1} \| \le
\frac{2 \| \mf'(\vx^*)^{-1} \|}{1 - 4 u_s \kappa(\mf'(\vx^*) )}.
\endeq
\end{lemma}

\begin{proof}

The standard assumptions and \eqnok{close} imply that
$\mf'(\vx_c)$ is nonsingular and
(see Lemma 4.3.1 from \cite{ctk:roots})
\begeq
\label{eq:twotimes}
\| \mf'(\vx_c) \| \le 2 \| \mf'(\vx^*) \|  \mbox{ and }
\| \mf'(\vx_c)^{-1} \| \le 2 \| \mf'(\vx^*)^{-1} \|.
\endeq
Hence, using \eqnok{twotimes},
\[
\| I - \mf'(\vx_c)^{-1} \mj \|
\le \| \mf'(\vx_c)^{-1} \| \| \mf'(\vx_c) - \mj \|
\le u_s \| \mf'(\vx_c)^{-1} \| \| \mf'(\vx_c) \|
\le 4 u_s \kappa(\mf'(\vx^*) ) < 1.
\]

So $\mf'(\vx_c)^{-1}$ is an approximate inverse of $\mj$. Therefore $\mj$
is nonsingular and
\[
\| \mj^{-1} \| \le 
\frac{\| \mf'(\vx_c)^{-1} \|}{1 - 4 u_s \kappa(\mf'(\vx^*) )}
\le
\frac{2 \| \mf'(\vx^*)^{-1} \|}{1 - 4 u_s \kappa(\mf'(\vx^*) )},
\]
proving the lemma.

\end{proof}

Assume the linear iterative method converges, which is not guaranteed, 
and that we terminate the linear iteration when 
\eqnok{inexactJ} holds. To prove the local improvement estimate
\eqnok{inexloc} we must connect
\eqnok{inexactJ} to the classic inexact Newton condition
\eqnok{inexact} for some $\eta < 1$. That will then imply
the estimate \eqnok{inexloc}.

We express the convergence estimates in terms of 
\begeq
\label{eq:pstar}
P^* = \frac{4 \| \kappa(\mf'(\vx^*)) \|}%
{1 - 4 u_s \kappa(\mf'(\vx^*) )}.
\endeq

\begin{lemma}
\label{lem:etaok}
Assume that the assumptions of Lemma~\ref{lem:mjok} and \eqnok{inexactJ} 
hold, that $u_s P^* < 1/2$, and that
\[
\eta_J < 1 - 2 u_s P^*.
\]
Then \eqnok{inexact} holds with
\begeq
\label{eq:etaok}
\eta \le \eta_J + (1 + \eta_J) u_s P^* < 1.
\endeq
\end{lemma}

\begin{proof}

Equation \eqnok{inexactJ} implies that
\[
\| \mj^{-1} \|^{-1} \| \vs \| \le \| \mj \vs \|
\le (1 + \eta_J) \| \mf(\vx_c) \|
\]
and hence, using Lemma~\ref{lem:mjok}
\begeq
\label{eq:stepest}
\| \vs \| \le \| \mj^{-1} \| (1 + \eta_J) \| \mf(\vx_c) \| 
\le \frac{2 \| \mf'(\vx^*)^{-1} \|}{1 - 4 u_s \kappa(\mf'(\vx^*) )}
(1 + \eta_J)  \| \mf(\vx_c) \|.
\endeq

We use \eqnok{inexactJ} again to obtain
\begeq
\label{eq:inexactok}
\begin{array}{ll}
\| \mf'(\vx_c) \vs + \mf(\vx_c) \| 
& \le 
\| \mj \vs + \mf(\vx_c) \| + 
\| (\mf'(\vx_c) - \mj) \vs \| \\
\\
& \le \eta_J \| \mf(\vx_c) \| + u_s \| \mf'(\vx_c) \| \| \vs \| \\
\\
& \le \eta_J \| \mf(\vx_c) \| + 2 u_s \| \mf'(\vx^*) \| \| \vs \|.
\end{array}
\endeq

Combining \eqnok{stepest} and \eqnok{inexactok} completes the proof.

\end{proof}

Now suppose we can obtain $\eta_J = O(u_s) = O(\sqrt{u_d})$, then 
\eqnok{errest2} holds and the local improvement estimate becomes
\begeq
\label{eq:errest3}
\| \ve_+ \| = O(\| \ve_c \|^2 + u_d )
\endeq
and the iteration statistics should be the same as Newton's method.
We will see exactly this in the results in \S~\ref{sec:results}.

The assumption that $u_s P^* < 1/2$ simply says that $\mf'(\vx^*)$ is
not horribly ill-conditioned. Ill-conditioning of $\mf'(\vx^*)$ does
not appear in the local improvement estimate directly, but does affect
the convergence of the linear iteration, as we will see in the next section.

\subsection{Iterative refinement}
\label{sec:IR}

Our first choice for an iterative method will be classic iterative
refinement \cite{wilkinson63} for solving a linear system
$\ma \vu = \vb$. Consistently with the application in this paper,
we will assume that the linear system is in single precision  and that
we factor the matrix in half precision. The reader should be aware that
one must store a half precision copy of $\ma$.
The basic algorithm is


\begin{algorithm}
\label{alg:ir}
{$\mbox{\bf IR}(\ma, \vb, \vu)$}
\begin{algorithmic}
\STATE $\vr = \vb - \ma \vu$
\STATE Store $\ma_h = I_s^h(\ma)$
\STATE Factor $\ma_h$ in half precision to
obtain computed factors ${\hat \ml}$ and ${\hat \mU}$.
\WHILE{$\| \vr \|$ too large}
\STATE $\vd = {\hat \mU}^{-1} {\hat \ml}^{-1} \vr$
\STATE $\vu \leftarrow \vu + \vd$
\STATE $\vr = \vb - \ma \vu$
\ENDWHILE
\end{algorithmic}
\end{algorithm}

In Algorithm~\ref{alg:ir} $\vu$ is the initial iterate on input
and the converged solution on output. Note that we are careful to use notation
to stress that we use the computed $LU$ factors in half precision.

One can express the iteration in closed form as
\[
\vu \leftarrow (\mi - {\hat \mU}^{-1} {\hat \ml}^{-1} \ma )\vu
+ {\hat \mU}^{-1} {\hat \ml}^{-1} \vb.
\]
Hence, Algorithm~\ref{alg:ir} is a linear stationary iterative method 
with iteration matrix
\[
\mm  = (\mi - {\hat \mU}^{-1} {\hat \ml}^{-1} \ma ).
\]
So, if the half precision factorization is a sufficiently
good approximation to $\ma$, then
$\| \mm \|$ will be small and iteration will converge rapidly, at least in
exact arithmetic. In the presence of rounding errors we would only expect a
local improvement result. 

Half precision can be very inaccurate and one must be prepared for the iteration
to converge slowly or even diverge. One can show that if the low precision
factorization is a reasonably good approximation to $\ma$, then one obtains
exactly the local improvement results one would like.
One such estimate is from \cite{CarsonHigham} using the $\ell^\infty$
norm on $R^N$. 
In the case here, where $u_h^2 = u_s$, one can show convergence if
\begeq
\label{eq:carson}
3 N u_h \mbox{cond}(\ma) < 1
\endeq
where $\mbox{cond}(\ma) = \| \, | \ma^{-1} | \, |\ma| \, \|_\infty$ and
$| \ma |$ is the matrix with entries $|a_{ij}|$. In that case the iteration
will reduce the linear residual by a factor $O(u_h)$ until
\begeq
\label{eq:backok}
\| \vb - \ma \vu \| = O( u_s \| \vb \| + \| \ma \|_\infty \| \vu \|_\infty).
\endeq

Now we interpret \eqnok{backok} in terms of the inexact Newton conditions
\eqnok{inexact} and \eqnok{inexactJ}. We have $\ma = \mf'(\vx_c)$,
$\vb = \mf(\vx_c)$ and the solution $\vu$ is the Newton step $\vs$. Since
$\| \vs \| = O(\| \mf(\vx_c) \|)$, the estimate \eqnok{backok} implies
\eqnok{inexactJ} with $\eta_J = O(u_s)$.

If the matrix $\ma$ is poorly conditioned, then \eqnok{carson} can fail
and then iterative refinement may fail to converge or fail to satisfy 
\eqnok{backok}. We will see this for the ill-conditioned example
in \S~\ref{sec:results}.

Even if $\| \mm \| > 1$, the condition number of
\[
{\hat \mU}^{-1} {\hat \ml}^{-1} \mj
\]
may be small enough to motivate using a Krylov method with
the low-precision factorization
as a preconditioner.
We use the GMRES-IR approach from \cite{CarsonHigham1,CarsonHigham}
with left preconditioning. This means that we solve 
\[
{\hat \mU}^{-1} {\hat \ml}^{-1} \ma \vd = 
{\hat \mU}^{-1} {\hat \ml}^{-1} \vr
\]
with GMRES to compute the defect $\vd$ in Algorithm~\ref{alg:ir}.
The preconditioner-vector product is computed with two triangular solves.
The results in \S~\ref{sec:results} show how this approach can improve
simple IR in one ill-conditioned case.


\subsection{Interprecision Transfers}
\label{sec:ipxf}

Finally, we must discuss some important details of interprecision
transfers and mixed precision operations.

For the two-precision implementation, when we solve
\begeq
\label{eq:fly}
{\hat \mj} \vs = - \mf(\vx_c)
\endeq
for the Newton step, we need to account for the interprecision
transfers. If we do nothing, then the triangular factors are in
precision $u_J$ and $\mf$ is in double precision. 
In that case each operation in the triangular solves will promote the low
precision matrix elements to double within the CPU registers.
This is called ``interprecision transfer on the fly''.

Interprecision transfer on the fly is $O(N^2)$ work
on interprecision transfers, but can be a noticeable cost for medium
to low dimensions even though the factorization cost is $O(N^3)$ work.
A way to avoid this cost is to round $\mf$ to
precision $u_J$ before the solve. One must take care if $\vx_c$ is near
the solution because rounding down, especially in half precision, could
result in an underflow to zero \cite{highamscaling}. The remedy for this
is to scale $\mf$ to a unit vector before rounding and then reverse
the scaling after the linear solve.
With this in mind one solves 
\begeq
\label{eq:nofly}
{\hat \mj} {\hat \vs}  = - I_d^J (\mf(\vx_c)/\| \mf(\vx_c) \|)
\endeq
entirely in the lower precision. This avoids interprecision transfers
during the triangular solves. The one promotes $\hat \vs$ and
reverses the scaling
\begeq
\label{eq:noflystep}
\vs = \| \mf(\vx_c) \| I_J^d {\hat \vs}
\endeq
to obtain a step $\vs$ in precision $u_J$. Then one would update the
solution via
\[
\vx_+ = \vx_c + \vs.
\]
This is exactly what we do in our Julia codes
\cite{ctk:siamfanl,ctk:sirev20}.
The reader should know that the steps $\vs$ computed with
\eqnok{fly} and \eqnok{nofly}-\eqnok{noflystep}
are different, but the performance of
the nonlinear iteration is unlikely to change.

For the linear iterative refinement iteration, the ideas are similar.
Interprecision transfers on the fly are implicit in our discussion
in \S~\ref{sec:IR} where we view iterative refinement as a 
stationary iterative method. Just as in the nonlinear case, one can
mitigate the interprecision transfer cost by replacing the step
\[
\vd = {\hat \mU}^{-1} {\hat \ml}^{-1} \vr
\]
from Algorithm~\ref{alg:ir} with 
\[
\vd = \| \vr \| I_j^s ({\hat \mU}^{-1} {\hat \ml}^{-1} I_s^h (\vr/|\ \vr \|) ).
\]
The iteration is no longer a stationary iterative method. Instead the
iteration is
\begeq
\label{eq:cheapir}
\vu \leftarrow \vu + 
\| \vb - \ma \vu \|
I_j^s \left({\hat \mU}^{-1} {\hat \ml}^{-1} I_s^h 
\left(\frac{\vb - \ma \vu}{\| \vb - \ma \vu \|} \right) \right).
\endeq
The fixed point map
is nonlinear and, because of the interprecision transfers, not even
continuous. However two approaches to interprecision transfer give 
the same results for all but the most ill-conditioned problems.

For IR-GMRES, however, using \eqnok{cheapir} will not suffice.
One must do the triangular solves in the higher
precision, single precision in the case of this paper, and hence assume
the interprecision transfer cost. One way to mitigate this cost is to
map the half precision factorization of $\mj_h$
to single precision before the solve. The 
cost of this is storage (one more copy of $\mj$), but the on-the-fly
interprecision cost is avoided. 

