%\documentclass[prd,twocolumn,groupaddress,amsmath,amssymb]{revtex4-2}
%\usepackage[dvipdfmx]{graphicx}% Include figure files
%\usepackage{dcolumn}% Eqnarray table columns on decimal point
%\usepackage{bm}% bold math 

%\usepackage[dvipdfmx]{color}% <-- Attention! [dvipdfmx] is important for figure.

%\usepackage{color}

\documentclass[reprint, amsmath, amssymb, aps, nofootinbib]{revtex4-2}
% \documentclass[preprint,aps,prb,longbibliography]{revtex4-2}
% \documentclass[aps,reprint,preprintnumbers,prl,draft]{revtex4-1}
% \usepackage[dvipdfmx]{graphicx}
% \usepackage[pdftex]{graphicx,xcolor}
\usepackage[pdftex]{graphicx}
% \usepackage{amsmath,amssymb,braket,bm}
\usepackage{bm}
%\usepackage{caption}
\usepackage{here}

\begin{document}
\title{Numerical analysis of a baryon and its dilatation modes in holographic QCD}
%\title{Dilatation modes in baryons in holographic QCD}
\author{Keiichiro~Hori}
\author{Hideo~Suganuma}
\affiliation{
% Division of Physics and Astronomy, 
% Graduate School of Science, 
Department of Physics, 
Kyoto University, \\ 
Kitashirakawaoiwake, Sakyo, Kyoto 606-8502, Japan}
\author{Hiroki~Kanda}
\affiliation{Yukawa Institute for Theoretical Physics (YITP), 
Kyoto University, 
Kitashirakawaoiwake, Sakyo, Kyoto 606-8502, Japan}
\date{\today} 
\begin{abstract}
We investigate a baryon and its dilatation modes 
in holographic QCD based on the Sakai-Sugimoto model, 
which is expressed as 
a 1+4 dimensional U($N_f$) gauge theory 
in the flavor space.
For spatially rotational symmetric systems, 
we apply a generalized version of the Witten Ansatz, 
and reduce 1+4 dimensional holographic QCD into a 1+2 dimensional Abelian Higgs theory in a curved space.
In the reduced theory, the holographic baryon is described  
as a two-dimensional topological object of an Abrikosov vortex.
We numerically calculate the baryon solution of holographic QCD 
using a fine and large lattice with 
spacing of 0.04~fm and size of 10~fm. 
Using the relation between the baryon size and 
the zero-point location of the Higgs field 
in the description with the Witten Ansatz,
we investigate a various-size baryon through this vortex description. 
As time-dependent size-oscillation modes 
(dilatation modes) of a baryon, we numerically 
obtain the lowest excitation energy of 577~MeV 
and deduce the dilatational excitation of a nucleon 
to be the Roper resonance N$^*$(1440). 

\end{abstract}

\maketitle

\section{Introduction}

Quantum chromodynamics (QCD) is established as 
the fundamental theory of strong interaction and 
characterized by ${\rm SU}(N_c)$ gauge symmetry and 
global ${\rm SU}(N_f)_L\times {\rm SU}(N_f)_R$ chiral symmetry. 
Owing to asymptotic freedom of QCD, 
high-energy hadron reactions can be analyzed using perturbative QCD.
In low energy regions, however, the QCD coupling becomes strong, 
and the perturbative method is no more applicable.
Therefore, for theoretical analyses of hadrons based on QCD, 
some nonperturbative methods are necessary such as lattice QCD.
%
Based on gauge/gravity duality \cite{M98} for D branes \cite{P95}
in superstring theory, holographic QCD is an interesting new tool 
to analyze the nonperturbative properties of QCD \cite{W98, KMMW04, SS05}.

%First, we give a brief historical overview of the string theory 
%and its application to hadron analyses.
Around 1970, the string theory was originally proposed by Nambu, Goto, and Polyakov for the description of hadrons \cite{Nambu,Goto,Polyakov81}. 
%
After establishment of QCD, the string theory was not used for the main research of hadrons. Instead, this framework was reformulated as the superstring theory in 1980s \cite{Green-Schwarz-Witten}
and has been studied as a plausible candidate of a grand unified theory including quantum gravity, and many studies have been constantly conducted to date. 
%In recent studies, the string theory provides a nonperturbative analytical
%method for QCD, which is called holographic QCD.

The superstring theory is formulated in 10 dimensional space-time 
%in a well-defined manner, 
and includes D branes, on which open string endpoints exist \cite{P95}. 
As a remarkable discovery by Polchinski, 
on the surface of $N$ overlapped D branes, 
U($N$) gauge symmetry emerges, 
and the $N$ D-branes system leads to U($N$) gauge theory \cite{P95}. 
In fact, on the $N$ D-branes, 
gauge fields $A_{ab}$ appear from the open string 
linking two D-branes labeled with $a$ and $b$ ($a,b =1,2, ..., N$).
%that is, the index $a$, $b$ correspond to 
%the degree of freedom of the D brane, 
%on which the open-string endpoint exists. 
%
On the other hand, around the D brane, 
a higher-dimensional supergravity theory is formed 
since the brane has mass, and multiple branes can be gravitational sources. 
%
Thus, there are two different theories relating to the D-brane,
and the gauge theory {\it on} the D-brane and 
the higher-dimensional gravity theory {\it around} the D-brane 
are conjectured to be equivalent \cite{M98}, 
which is called AdS/CFT correspondence or gauge/gravity duality. 
This remarkable equivalence between gauge and gravity theories 
was first proposed by Maldacena 
from a detailed analysis of four-dimensional ${\cal N}=4$ 
supersymmetric (SUSY) SU($N$) Yang-Mills theory 
and five-dimensional Anti-de Sitter (AdS) supergravity theory \cite{M98},
using large $N$ argument. 
This equivalence was first called as AdS/CFT correspondence 
and is generalized as gauge/gravity duality in more general concept. 
%
In this correspondence, 
%there is strong/weak coupling relation, and 
a strong-coupling gauge theory can be described with 
a weak-coupling gravity theory \cite{M98,W98,KMMW04,SS05}.
%(earlier studies for strong/weak duality \cite{MO77, GNO77})

With this correspondence, many researches have been performed, 
and one of the main category is to analyze strong-coupling QCD 
with a higher-dimensional classical gravity theory, 
which is called holographic QCD \cite{W98, Karch-Katz, EKSS, KMMW04, SS05}. 
In 1998, Witten formulated a non-SUSY version of holographic QCD 
for a four-dimensional pure-gluon Yang-Mills theory 
using $S^1$-compactified 
$N_c$ D4 branes, where periodic and antiperiodic boundary conditions are imposed on boson and fermion fields, respectively \cite{W98}.
In 2005, Sakai and Sugimoto proposed holographic QCD for 
four-dimensional full QCD, 
including massless chiral quarks, by adding the $N_f$ pair of 
${\rm D8}$ and $\overline{\rm D8}$ branes to $S^1$-compactified ${\rm D4}$ branes \cite{SS05}.
In fact, the D4/D8/$\overline{\rm D8}$ multi-D-brane system 
has SU($N_c$) gauge symmetry and 
${\rm SU}(N_f)_L\times {\rm SU}(N_f)_R$ chiral symmetry  
and leads to massless QCD in infrared scale.
Here, quarks and gluons appear from the massless modes 
of 4-8 and 4-4 strings, respectively.
%
In the large $N_c$ limit, 
$N_c$ D4 branes are dominant as a gravity source and 
are converted into a gravitational field 
via the AdS/CFT correspondence or the gauge/gravity duality. 
The 't~Hooft coupling $\lambda \equiv N_c g^2$ 
is a control parameter of the gauge side, 
and strong-coupling QCD with a large $\lambda$ 
corresponds to a weakly interacting gravitational theory. 
Then, this system is described by the D8 brane 
in the presence of a background gravitational field 
originating from the D4 brane. 
Nonperturbative properties of 
QCD can be analyzed with a classical gravity theory 

% As mentioned above, 
% strong-coupling gauge theories such as QCD 
% are considered to be mapped into weak-coupling gravity theories 
% due to the strong/weak coupling duality.

In the leading order of 1/$N_c$ expansion, 
the effective theory of the D8 brane in the D4-brane-induced 
background gravity is expressed with the Dirac-Born-Infeld (DBI) action and the Chern-Simons (CS) term. 
%
The leading order of $1/\lambda$ expansion is the DBI action, 
and the next leading of $1/\lambda$ is the CS term. 
%
Expanding the DBI action with $1/\lambda$, 
the leading order becomes 
a five-dimensional Yang-Mills theory 
in flavor space in a curved space, 
where a background gravity appears 
only in the fifth dimension. 
The CS term is a topological term 
responsible for anomalies in QCD. 
%
In holographic QCD, 
the five-dimensional holographic field is 
decomposed into four-dimensional meson fields,  
%of (pseudo)scalar and (axial)vector mesons, 
and this theory is successful in the mesons sector, 
that is, it describes low-lying meson masses, 
inter-meson couplings, and phenomenological laws 
in hadron physics \cite{SS05}. 
%
Regarding baryonic degrees of freedom, 
baryons do not appear explicitly in the holographic action. 
In fact, the baryon does not appear as a system component 
%and it is described with a soliton of mesons 
in holographic QCD. 

The absence of baryons in the effective action is 
a general result due to the large $N_c$ argument, 
because QCD is reduced into a weak-coupling theory 
of mesons and glueballs in the large $N_c$ limit \cite{W79}, 
where the baryon mass increases as ${\mathcal O}(N_c)$ and 
the baryon do not appear in the effective action. 
%
In the large $N_c$ argument, 
%including holographic QCD, 
the baryon is considered to appear as a soliton of mesons 
such as the Skyrmion in the Skyrme model \cite{S61,W79,ZB96}. 

Here, we briefly mention a historical overview of the Skyrme model.
The Skyrme model is a low-energy effective theory 
in hadron physics, first proposed by Skyrme in 1961 \cite{S61}. 
% 
After the quark model was proposed in 1964 \cite{GZ64}, 
main researches of strong interaction and hadrons were shifted to 
the quark theory, which was eventually developed into QCD. 
% 
In 1979, Witten revived the Skyrme soliton picture of baryons 
from a large $N_c$ viewpoint of QCD \cite{W79}, 
although the direct relation between the Skyrme model and 
QCD has not been clear. 
% 
In 2005, Sakai and Sugimoto showed 
a theoretical explicit connection between 
massless QCD and the Skyrme Lagrangian, 
which is derived as the pion sector in holographic QCD, 
using the gauge/gravity duality for a D-brane system \cite{SS05}. 
(Actually, the Sakai-Sugimoto model reduces into the Skyrme model, 
when massive mesons are dropped off, leaving only light pions.) 

%It provides interesting description of baryons 
%and low energy aspects of QCD, 
% However, QCD was established as the theory describing strong interaction and 
% This is a meson field theory, and 
% baryons appear as chiral topological solitons \cite{ZB96}. 
% 
% However, QCD was established as the theory describing strong interaction and 

Now, let us concentrate on baryons in holographic QCD. 
Also for holographic QCD, 
which is derived with large $N_c$ and is 
described with only meson fields, 
the baryon is described as a chiral soliton of mesons, 
that is, a topological object 
like a brane-induces Skyrmion \cite{NSK07, NSK09} or 
an instanton-like object \cite{W98b, HSSY07}. 
%
In fact, the baryon appears as 
an spatially extended object in holographic QCD.

To summarizes the above, 
holographic QCD is an analytical nonperturbative method for QCD 
and has direct connection with QCD, 
as a clearly strong advanced point. 
% 
In holographic QCD, 
while mesons appear in the action and can be directly treated, 
the baryon is described as an extended soliton of mesons.
Therefore, compared with the meson sector, 
the baryon sector is more difficult and 
has not been well studied in holographic QCD, 
in spite of several pioneering studies on the holographic baryon 
\cite{NSK07,HSSY07,HRYY07,HSS08,CI12,BS14,RSR14}. 
%which is a motivation of our study on baryons in holographic QCD. 
%
In addition, this baryonic soliton allows 
spatial dilatation modes as its excitation 
peculiar to the spatially-extended object, 
and thus we focus on this dilatation mode of 
the holographic baryon in the latter of our study. 

In this study, 
we investigate a single baryon and its dilatation modes 
in holographic QCD, adopting the Sakai-Sugimoto model 
formulated as a 1+4 dimensional U($N_f$) gauge theory 
in the flavor space.
%
For spatially rotational symmetric systems, 
we apply a generalized Witten Ansatz and reduce  
1+4 dimensional holographic QCD into 
a 1+2 dimensional Abelian Higgs theory, 
where the holographic baryon is expressed 
as an Abrikosov vortex.
We numerically calculate the baryon solution of holographic QCD 
using a fine and large lattice, 
keeping background gravity from the $N_c$ D4-brane. 

In addition, we investigate a various-size baryon 
and the dilatation mode (time-dependent size-oscillation) 
of a single baryon, 
using the relation between the baryon size and 
the zero-point location of the Higgs field 
in the reduced Abelian Higgs theory. 

This size oscillation is physically considered 
as a collective motion and 
is difficult to be described in the quark model.
%based on the single-particle picture. 
Instead, the size oscillation mode has been studied 
in the Skyrmion research, 
and its lowest excitation mode is identified 
as the Roper resonance N$^*$(1440) \cite{Roper,HS84,HH84,ZMK84,WE84,LZB84,BN84}. 

The Roper resonance is the first excited state 
of the nucleon N(940) with the positive parity and its energy being 1440 MeV. 
In the quark model, based on the single-particle picture, 
the first excited-state baryon is to have negative parity, 
and it contradicts the experimental data. 
% 
In lattice QCD, the numerical results with overlap fermion 
well reproduce the Roper resonance in terms of the excitation energy 
and the positive parity \cite{MCDDHLLZ,LHKLMWZ}. 
Here, lattice QCD is a powerful tool for the quantitative analysis of hadrons, 
but it is difficult to get state information of hadrons like the wave function 
due to path integral formalism, where all the states are integrated out. 
% 
As an alternative method, 
the Skyrme model seems to succeed to reproduce 
the mass and parity of N$^*$(1440) as the first excited state of N(940). 
In this chiral soliton picture, this first excited state 
is described as a dilatation or breathing mode of the ground-state soliton \cite{BN84,HS84,HH84,ZMK84,WE84,LZB84}. 

Therefore, we here investigate the dilatation mode of the baryon 
in holographic QCD and finally compare it with the Roper resonance N$^*$(1440) 
in terms of its mass and parity. 
%
The dilatation mode of baryon can be described also in holographic QCD. 
Note again that the Sakai-Sugimoto model reduces into the Skyrme model, 
when massive mesons except for pions are dropped off.
%
% focus on the dilatation mode of the baryon excitation 
% in holographic QCD for Roper resonance. 
%Beyond quark model! 
%
In fact, the holographic baryon appears as a soliton \cite{NSK07,HSSY07}, 
i.e., a spatially extended object, and therefore has dilatation mode. 
Note, however, that holographic dilatation 
is four-dimensional spatial oscillation including 
the extra spatial dimension rather than ordinary three-dimensional one. 

The organization of this paper is as follows.
In Section 2, we briefly review the Sakai-Sugimoto model as 
typical holographic QCD.
% WITTEN ANSATZ IN HQCD
In Section 3, we apply the Witten Ansatz in holographic QCD.
Owing to the Witten Ansatz, the 1+4 dimensional Yang-Mills theory 
reduces into a 1+2 dimensional Abelian Higgs theory. 
% VORTEX DESCRIPTION OF BARYONS
In Section 4, using the Witten Ansatz, 
we present the vortex description of baryons in holographic QCD, 
and we numerically obtain the ground-state solution of the holographic baryon
using a fine and large-volume lattice.
% SIZE-DEPENDENCE OF A HOLOGRAPHIC BARYON
In Section 5, we numerically analyze size dependence 
of the holographic baryon. 
In Section 6, we investigate time-dependent dilatational modes 
of a single baryon in holographic QCD. 
Section 7 is devoted for summary and conclusion. 



\section{Holographic QCD action in the Sakai-Sugimoto model}

In this section, as a starting point, 
we introduce the Sakai-Sugimoto model, 
one of the most successful holographic QCD \cite{SS05}. 
In the Sakai-Sugimoto model, 
four-dimensional massless QCD is constructed 
using the D4/D8/$\overline{\rm D8}$ multi-D-brane system, 
which comprises spatially $S^1$-compactified $N_c$ D4 branes 
attached with $N_f$ D8-$\overline{\rm D8}$ pairs.
Here, $N_c$ means the color number, and 
$N_f$ the light flavor number. 
This compactification breaks SUSY due to the (anti)periodic conditions for bosons(fermions), 
as was demonstrated for a D4 brane system by Witten \cite{W98}.
The compactification radius is $M_{\rm KK}^{-1}$, 
and this model parameter physically corresponds to 
a UV cutoff in holographic QCD.
%
This system is infrared equivalent to massless QCD, 
where chiral symmetry exists \cite{SS05}.
%
Using AdS/CFT correspondence (gauge/gravity duality), 
the $N_c$ D4 branes are transformed into a gravitational source, 
and the system becomes $N_f$ D8 branes in the D4 gravity background, 
which leads to the DBI and CS action at the leading order of $1/N_c$ 
within the probe approximation. 
In terms of $1/\lambda$ expansion, 
the DBI action includes its leading order, and the CS action is sub-leading.

From the multi-D-brane system which is infrared equivalent to massless QCD, 
the DBI action becomes 1+4 dimensional Yang-Mills theory on the flavor space of 
U($N_f$) $\simeq {\rm SU}(N_f) \times$ U(1) 
at the leading order of $1/\lambda$ expansion \cite{SS05}: 
\begin{eqnarray}
&&S_{\rm 5YM} = S_{\rm 5YM}^{{\rm SU}(N_f)} +S_{\rm 5YM}^{\rm U(1)}
\nonumber \\
&=& -\kappa \int d^4x dw \ \textrm{tr} 
\left[ \frac{1}{2}h(w)F_{\mu\nu} F^{\mu\nu} + k(w)F_{\mu w} F^{\mu w} \right]
\nonumber \\
&&-\frac{\kappa}{2} \int d^4x dw \ \left(\frac{1}{2}h(w) \hat F_{\mu\nu} \hat F^{\mu\nu} 
+ k(w) \hat F_{\mu w} \hat F^{\mu w} \right).~~~~~
\label{eq:5YM}
\end{eqnarray}
In this paper, we use $w$ for the extra fifth-coordinate in holographic QCD, 
$\hat{A}$ denotes U(1) gauge field and $\hat{F}$ U(1) field strength \cite{SH20}. 
% and use $\hat{A}$ and $\hat{F}$ for the U(1) sector \cite{SH20}. 
% and we also use $\hat{A}$ and $\hat{F}$ for U(1) gauge field and 
% %$\hat{F}$ for
% and U(1) field strength, respectively \cite{SH20}. 

For $M, N = t, x, y, z, w$, the field strengths are given by 
\begin{eqnarray}
F_{MN} &\equiv& \partial_M A_N-\partial_N A_M+i[A_M, A_N], 
\nonumber \\
\hat F_{MN} &\equiv& \partial_M \hat A_N-\partial_N \hat A_M, 
\end{eqnarray}
with the five-dimensional SU($N_f$) gauge field $A^M(x^\mu,w)$ 
and U(1) gauge field $\hat A^M(x^\mu,w)$, respectively. 
Throughout this paper, we take the $M_{\rm KK}=1$ unit together with the natural unit, 
and $\kappa$ is written as $\kappa = \frac{\lambda N_c}{216\pi^3}$ in this unit.
%
Note that, as a relic of $N_c$ D4-branes, there appear background gravity $k(w)$ and $h(w)$ 
depending on the extra fifth-coordinate $w$, 
\begin{eqnarray}
k(w) \equiv 1+w^2, \ \ h(w) \equiv k(w)^{-1/3}, 
\end{eqnarray}
in the $M_{\rm KK}=1$ unit. 
%
In Eq.~(\ref{eq:5YM}) at the leading order of $1/\lambda$ expansion, 
SU($N_f$) variables $A$ and U(1) variables $\hat A$ are completely separated, 
and hence we have divided $S_{\rm 5YM}$ into the SU($N_f$) sector $S_{\rm 5YM}^{{\rm SU}(N_f)}$ 
and the U(1) sector $S_{\rm 5YM}^{{\rm U}(1)}$. 

The $1/N_c$-leading holographic QCD 
also has the CS term \cite{SS05,HSSY07}
as the next leading order of $1/\lambda$.
The CS term is a topological term responsible 
for anomalies in QCD, and its explicit form is 
%  
\begin{eqnarray}
\label{eq:CS}
S_{\rm CS} &=& \frac{N_c}{24\pi^2} \int \omega_5({\cal A}) \nonumber \\
&=& \frac{N_c}{24\pi^2} \int {\rm tr} \left( {\cal A}{\cal F}^2 - \frac{i}{2}{\cal A}^3{\cal F} - \frac{1}{10}{\cal A}^5 \right), 
\end{eqnarray}
where ${\cal A} = A + \frac{1}{N_f}\hat{A}$ denotes the U($N_f$) gauge field \cite{SS05}.
% 
Note that SU($N_f$) variables $A$ and U(1) variables $\hat A$ are dynamically mixed 
in the CS term $S_{\rm CS}$ in Eq.~(\ref{eq:CS}).

In this paper, to analyze baryons in holographic QCD, 
we consider both Yang-Mills and CS parts for the case of $N_f=2$. 

\section{Witten Ansatz in holographic QCD}

Holographic QCD is formulated to be a 1+4 dimensional 
U($N_f$) non-Abelian gauge theory with a gravitational background $h(w)$ and $k(w)$, 
which would be fairly difficult to analyze. 
To avoid the difficulty and to proceed analytic calculations, 
most previous works \cite{HSSY07,HRYY07,HSS08}
were forced to take a flat background $h(w)=k(w)=1$ 
and to use the simple 't~Hooft instanton solution 
\cite{BPST, tH76} in the flat space, 
although $h(w)$ and $k(w)$ are the trace of D4-branes 
and are to be relevant ingredients. 
%Holographic QCD has several difficulties of 1+4 dimensional %space-time, non-Abelian gauge field and gravitational %background. 

To deal with holographic QCD for $N_f=2$ 
without reduction the gravitational backgrounds $h(w)$ and $k(w)$, 
we adopt the Witten Ansatz \cite{W77} in this paper. 
The Witten Ansatz is generally applicable for spatially-rotational symmetric system 
in the SU(2) Yang-Mills theory. 
%
Applying this to holographic QCD, the 1+4 dimensional non-Abelian theory 
transforms to a 1+2 dimensional Abelian Higgs theory. 
Accordingly, relevant topological objects are changed 
from instantons to vortices, as will be shown in Sect.~IV. 
In this section, we generalize the Witten Anstaz 
to be applicable to holographic QCD.

\subsection{Witten Ansatz for Euclidean Yang-Mills theory}

In this subsection, we briefly review the original Witten Ansatz \cite{W77}
applied for the Euclidean four-dimensional SU(2) Yang-Mills theory, 
which is formulated on three spatial coordinates $(x,y,z)$ and Euclidean time $t$.
%
For spatially-rotational symmetric systems, the Witten Ansatz can be applied as
\begin{eqnarray}
A_i^a(x,y,z,t) &=& \frac{\phi_2(r,t)+1}{r}\epsilon_{iak}\hat{x}_k + \frac{\phi_1(r,t)}{r} \hat{\delta}_{ia} \nonumber \\
& & + a_r(r,t) \hat{x}_i\hat{x}_a, \label{eqn:Original Witten Ansatz 1} \\
A_t^a(x,y,z,t) &=& a_t (r,t) \hat{x}^a 
\label{eqn:Original Witten Ansatz 2}
\end{eqnarray}
with $r \equiv (x_i x_i)^{1/2}$, $\hat x_i \equiv x_i/r$
and $\hat \delta_{ij} \equiv \delta_{ij}-\hat x_i \hat x_j$.

Using the Witten Ansatz, 
the four-dimensional SU(2) Yang-Mills theory is reduced into 
a two-dimensional Abelian Higgs theory as 
\begin{eqnarray}
S_{\rm YM}^{\rm SU(2)} &=& \int dt~d^3x\ \frac{1}{2} {\rm tr} F_{\mu\nu} F^{\mu\nu} \nonumber \\
&=& 4\pi \int^\infty_{-\infty} dt \int^\infty_0 dr \biggl[ |D_0\phi|^2 + |D_1\phi|^2 \nonumber \\
& & + \frac{1}{2r^2}(1-|\phi|^2)^2 + \frac{r^2}{2} f_{01}^2 \biggr],
\label{eq:YM->AH}
\end{eqnarray}
where the complex Higgs field $\phi(t,r)$, 
Abelian gauge field $a_\mu(t,r)$, its covariant derivative $D_\mu$, 
and field strength $f_{\mu\nu}$ in the Abelian Higgs theory are 
\begin{eqnarray}
\phi &\equiv& \phi_1+i\phi_2 \in {\bf C}, \
a_\mu \equiv (a_0,a_1), \cr 
D_\mu &\equiv& \partial_\mu - i a_\mu, \
f_{01} \equiv \partial_0 a_1-\partial_1 a_0. \ \
\end{eqnarray} 
%
Here, we have used $(0,1)=(t,r)$ for the index of 
two dimensional coordinates. 

\subsection{Generalized Witten Ansatz for 
SU(2)$_f$ sector in holographic QCD}

In this subsection, we generalize the Witten Ansatz 
to be applicable for holographic QCD.

The SU(2)$_f$ sector in holographic QCD is expressed as a 1+4 dimensional Yang-Mills theory with gravitational backgrounds $h(w)$ and $k(w)$. 
Holographic QCD already includes
four-dimensional Euclidean spatial coordinates ($x$, $y$, $z$, $w$) including the extra fifth-coordinate $w$, 
and instantons can be naturally introduced in holographic QCD without necessity of 
the Euclidean process or the Wick rotation. 

Describing the SU(2)$_f$ gauge field $A$ 
with the Pauli matrix $\tau^a$ as  
$A=A^a\frac{\tau^a}{2} \in {\rm su}(2)_f$ 
in holographic QCD, 
we take a generalized version of the Witten Ansatz 
for $(x,y,z)$-spatially rotational symmetric systems 
\cite{BS14,RSR14,SH20}, 
\begin{eqnarray}
A_0^a(t,x,y,z,w) &=& a_0 (t,r,w) \hat{x}^a, 
\label{eqn:Witten Ansatz 1} \\
A_i^a(t,x,y,z,w) &=& \frac{\phi_2(t,r,w)+1}{r}\epsilon_{iak}\hat{x}^k \cr
&+& \frac{\phi_1(t,r,w)}{r} \hat{\delta}_{ia} + a_r(t,r,w) \hat{x}_i\hat{x}_a,~~~ \label{eqn:Witten Ansatz 2} \\
A_w^a(t,x,y,z,w) &=& a_w (t,r,w) \hat{x}^a, 
\label{eqn:Witten Ansatz 3}
\end{eqnarray}
with $r \equiv (x_i x_i)^{1/2}$, $\hat x_i \equiv x_i/r$
and $\hat \delta_{ij} \equiv \delta_{ij}-\hat x_i \hat x_j$.
%
For 1+4 dimensional holographic QCD, we have extended 
the Witten Ansatz for $A^a_0$ component, 
considering the $(x,y,z)$-rotational symmetry. 
%
Note that this Ansatz is a general form 
when $(x,y,z)$-rotational symmetry is 
imposed on gauge fields. 

In the Witten Ansatz, 
the holographic field strength,  $F_{ij}$, $F_{0i}$, $F_{wi}$ and $F_{0w}$ 
are expressed as 
\begin{eqnarray}
\frac{1}{2}\epsilon_{ijk}F_{jk}^a &=& (\partial_1\phi_1 - a_1\phi_2)\frac{\hat{\delta}_{ai}}{r} + (1-\phi_1^2-\phi_2^2)\frac{\hat{x}_a\hat{x}_i}{r^2} \cr
 & & + (\partial_1\phi_1+a_1\phi_2)\frac{1}{r}\epsilon_{aik}\hat{x}_k, 
\\
F_{0i}^a &=& (\partial_0\phi_1+a_0\phi_2)\frac{\hat{\delta}_{ai}}{r} +  (\partial_0a_1-\partial_1a_0)\frac{\hat{x}_a\hat{x}_i}{r^2} \cr
& & - (\partial_0\phi_2-a_0\phi_1)\frac{1}{r}\epsilon_{aik}\hat{x}_k,
\\
F_{wi}^a &=& (\partial_2\phi_1+a_2\phi_2)\frac{\hat{\delta}_{ai}}{r} + (\partial_2a_1-\partial_1a_2)\frac{\hat{x}_a\hat{x}_i}{r^2} \cr & & - (\partial_2\phi_2-a_2\phi_1)\frac{1}{r}\epsilon_{aik}\hat{x}_k, \\
F_{0w}^a &=& (\partial_0a_2-\partial_2a_0)\hat{x}^a. 
\end{eqnarray}

With the Witten Ansatz, 
1+4 dimensional SU(2)$_f$ Yang-Mills sector of holographic QCD is reduced into 
a 1+2 dimensional Abelian Higgs theory on a curved space.
In fact, $S_{\rm 5YM}^{{\rm SU(2)}_f}$ is rewritten as 
\begin{eqnarray}
&&S_{\rm 5YM}^{{\rm SU(2)}_f} \cr
&=&-\kappa \int d^4x\ dw\ {\rm tr} \left[ 
\frac{1}{2} h(w) F_{\mu\nu} F^{\mu\nu} 
+ k(w)F_{\mu w} F^{\mu w} \right] \cr
&=& 4\pi\kappa \int^\infty_{-\infty} dt \int^\infty_0 dr \int^\infty_{-\infty} dw \biggl[ h(w) (|D_0\phi|^2 - |D_1\phi|^2) \cr
&-& k(w) |D_2\phi|^2 
- \frac{h(w)}{2r^2}(1-|\phi|^2)^2 \cr
&+& \frac{r^2}{2} \{ h(w) f_{01}^2 + k(w) f_{02}^2 - k(w) f_{12}^2 \} \biggr],
\label{eq:S5YM}
\end{eqnarray}
where the complex Higgs field $\phi(t,r,w) \in {\bf C}$, 
Abelian gauge field $a_\mu(t,r,w)$, its covariant derivative $D_\mu$, 
and field strength $f_{\mu\nu}$ in the Abelian Higgs theory are 
\begin{eqnarray}
\phi &\equiv& \phi_1+i\phi_2 \in {\bf C}, \
a_\mu \equiv (a_0,a_r,a_w), \cr 
D_\mu &\equiv& \partial_\mu - i a_\mu, \
f_{\mu\nu} \equiv \partial_\mu a_\nu-\partial_\nu a_\mu. \ \
\end{eqnarray}
%
Here, we have used $(0,1,2)=(t,r,w)$ for the index of 
1+2 dimensional coordinates. 
Note that the factor $k(w)$ appears in $D_2$ and $F_{12}$ 
associated with the index 2 ($w$), and otherwise the factor $h(w)$ appears. 

From this action, 
the static energy of the Yang-Mills part is obtained as
\begin{eqnarray}
&&E_{\rm 5YM}^{{\rm SU(2)}_f} [\phi(r,w), \vec a(r,w)]
\cr
&&= 4\pi\kappa \int^{\infty}_{0} dr \int^{\infty}_{-\infty} dw \biggl[ h(w) |D_1\phi|^2 + k(w) |D_2\phi|^2 \cr 
&& \qquad \qquad + \frac{h(w)}{2r^2}\{1-|\phi|^2\}^2 + \frac{r^2}{2} k(w) f_{12}^2 \biggr] \cr
&&= 4\pi \int^{\infty}_{0} dr r^2 \int^{\infty}_{-\infty} dw~{\cal E}^{{\rm SU(2)}_f}(r,w), 
\label{eqn:E_5YM}
\end{eqnarray}
with $\vec a=(a_1,a_2)=(a_r,a_w)$. 
This is similar with the 1+2 dimensional Ginzburg-Landau theory. 
% but there are differences of $(r,w)$ half plane, $r^2$ metric and 
% $h(w),k(w)$ gravity.
As the different point, however, 
there appears a nontrivail metric of $r^2$ \cite{W77}
in addition to holographic background gravity of 
$h(w)$ and $k(w)$ in the $(r,w)$ half plane. 
%
To include all the gravitational effects exactly, 
we construct the lattice formalism, as shown in Appendix~A, 
and perform the numerical calculation for holographic QCD. 

Finally in the subsection, 
we investigate 
the topological density $\rho_B$
in the Witten Ansatz. 
For the SU(2)$_f$ gauge configuration in the Witten Ansatz, 
the topological density $\rho_B$ in $(x,y,z,w)$-space is found to be \cite{SH20}
\begin{eqnarray}
\rho_B &\equiv& \frac{1}{16\pi^2} {\rm tr}\bigl( F_{MN} \tilde{F}_{MN} \bigr) 
= \frac{1}{32\pi^2} \epsilon_{MNPQ} {\rm tr}\bigl( F_{MN} F_{PQ} \bigr)
\nonumber \\ 
%&=&
%-\frac{1}{16\pi^2} %\epsilon_{ijk}F_{wi}^aF_{jk}^a
%\nonumber \\ 
&=&\frac{1}{8\pi^2r^2}
\{-i\epsilon_{ij}(D_i\phi)^*
D_j\phi+
\epsilon_{ij}
\partial_i a_j
(1-|\phi|^2)
\}
\nonumber \\ 
&=&\frac{1}{8\pi^2r^2}
\epsilon_{ij}
\partial_i
\{
\frac{1}{2i}(\phi^*\partial_j\phi-\phi\partial_j\phi^*)
+ a_j (1-|\phi|^2)
\}
\nonumber \\ 
&=&\frac{1}{8\pi^2r^2}\epsilon_{ij}\partial_i
\{a_j(1-|\phi|^2)+\partial_j \theta \cdot |\phi|^2\},
\label{eq:rho_B}
\end{eqnarray}
where $\theta\equiv {\rm arg}~\phi$ and 
Roman small letters 
$i, j$ take $(1,2)=(r,w)$.
The topological density $\rho_B$ is expressed as a total derivative. 
% 
Since we consider $(x,y,z)$-rotationally symmetric system, 
$\rho_B$ is independent of the spatial direction $(\hat x, \hat y, \hat z)$. 
In fact,  $\rho_B$ takes an SO(3) rotationally symmetric form of $\rho_B(r,w)$ 
in the second line of Eq.~(\ref{eq:rho_B}).


\subsection{U(1) sector in holographic QCD}

In this subsection, we consider 
the U(1) sector in holographic QCD. 
Hereafter, the capital-letter index denotes 
the Euclidean spatial index as $M=x, y, z, w$. 
%
Also for the U(1) gauge field $\hat A$, 
we respect the spatial SO(3) rotational symmetry \cite{BS14,RSR14}
as in the Witten Ansatz, and impose 
\begin{eqnarray}
\hat A_i(t,x,y,z,w) &=& \hat a_r(t,r,w) \hat{x}_i, 
%\hat A_0(t,x,y,z,w) &=& \hat a_0 (t,r,w), 
%\hat A_w(t,x,y,z,w) &=& \hat a_w (t,r,w), 
\end{eqnarray}
while $\hat A_0$ and $\hat A_w$ are treated to be arbitrary.
%
In this case, one finds $\hat F_{ij}=0$ and can take the $\hat a_r=0$ gauge, 
which simplifies $\hat A_i=0$.

%In the same way with the SU(2)$_f$ part, 
%we also use the Witten Ansatz for the U(1) gauge field $\hat{A}$ as
%\begin{align}
%    \hat{A}_i(t, x, y, z, w) = \hat{a}_r(t, r, w) \hat{x}_i .
%\end{align}
%This configuration leads to $\hat{F}_{ij}$. (ij=xyz). 

Then, the $1/\lambda$-leading term $S_{\rm 5YM}^{\rm U(1)}$ 
for the U(1) gauge field is written as \cite{SH20}
\begin{eqnarray}
S^{\rm U(1)}
&=& 
\frac{\kappa}{2}
\int d^4x dw \{ h(w) \hat F_{0i}^2+k(w) \hat F_{0w}^2- k(w) \hat F_{iw}^2\} \cr
&=& \int d^4x dw \biggl[ 
\frac{1}{2}\hat A_0 K \hat A_0-\frac{\kappa}{2} k(w) (\partial_i \hat A_w)^2
 \cr 
 & & + ({\rm time\hbox{-}derivative~terms})
\biggr],
% &=& 4\pi \int^{\infty}_{-\infty} dt \int^{\infty}_{0} dr \int^{\infty}_{-\infty} dw\ r^2\biggl[ 
% \frac{1}{2}\hat A_0(t,r,w) K \hat A_0(t,r,w) \cr
% & & + \frac{\kappa}{2} k(w) (\partial_r r^2 \hat A_w(t,r,w))^2
% \biggr],
\end{eqnarray}
using the SO(3)-symmetric non-negative hermite kernel 
\begin{eqnarray}
K &\equiv& -\kappa \{ h(w) \partial_i^2 + \partial_w k(w) \partial_w \}
\cr
&=&-\kappa \{ h(w) \frac{1}{r^2}\partial_r r^2 \partial_r 
+ \partial_w k(w) \partial_w \}.
\end{eqnarray}
In this calculation, 
we take $\hat{A}_i=0$ 
using the gauge and rotational symmetry. 

We consider the CS term $S_{\rm CS}$ 
as the next leading order of the $1/\lambda$ expansion. 
%
For the static SO(3)-rotationally symmetric configuration in the $A_0=0$ gauge, 
the CS term $S_{\rm CS}$ in Eq.~(\ref{eq:CS}) 
is transformed as \cite{HSSY07,BS14,RSR14,SH20}
\begin{eqnarray}
S_{\rm CS} &=& \frac{N_c}{24\pi^2} \epsilon_{MNPQ} \int d^4x dw \left[ \frac{3}{8}\hat{A}_0{\rm tr}(F_{MN}F_{PQ}) \right. \nonumber \\
& & \left. - \frac{3}{2}\hat{A}_M{\rm tr}(\partial_0 A_NF_{PQ}) + \frac{3}{4}\hat{F}_{MN}{\rm tr}(A_0F_{PQ}) \right. \nonumber \\
& & \left. + \frac{1}{16}\hat{A}_0\hat{F}_{MN}\hat{F}_{PQ} - \frac{1}{4}\hat{A}_M\hat{F}_{0N}\hat{F}_{PQ} \right] \nonumber \\
&=& \frac{N_c}{2}\int d^4x dw~ \rho_B \hat A_0, \label{eqn: S CS}
\end{eqnarray}
up to total derivative. 
%In this calculation, we have taken $\hat{A}_i=0$. 
% due to the gauge and rotational symmetry.
This is Coulomb-type interaction between 
the U(1) gauge potential $\hat A_0$ and 
the topological density 
$\rho_B \equiv \frac{1}{16\pi^2} {\rm tr}(F_{MN} \tilde{F}_{MN})$.

Then, the total U(1) action 
depending on the U(1) gauge field $\hat A$ is written as 
\begin{eqnarray}
&&S^{\rm U(1)} \equiv S^{\rm U(1)}+S_{\rm CS} \cr
&=& \int d^4x dw \biggl[ 
\frac{1}{2}\hat A_0 K \hat A_0 +\frac{N_c}{2}\rho_B \hat A_0 
-\frac{\kappa}{2} k(w) (\partial_i \hat A_w)^2
\biggr],~~~~~~
\label{eq:S^U(1)}
\end{eqnarray}
which leads to 
the field equations, 
\begin{eqnarray}
K \hat A_0 +\frac{N_c}{2}\rho_B=0, \ \ \partial_i^2 \hat A_w=0.
\label{eq:U(1)FE}
\end{eqnarray}
%
For the static configuration, 
the additional energy $E^{\rm U(1)}$ 
from U(1) sector $S^{\rm U(1)}$ 
is simply given by 
\begin{eqnarray}
&&E^{\rm U(1)}= -S^{\rm U(1)}/\int dt\cr
&=&\int d^3x dw \biggl[ 
-\frac{1}{2}\hat A_0 K \hat A_0 
-\frac{N_c}{2}\rho_B \hat A_0 
+\frac{\kappa}{2} k(w) (\partial_i \hat A_w)^2
\biggr].~~~~~~
\label{eq:E^U(1)org}
\end{eqnarray}

For the static case, $\hat A_w$ is dynamically isolated in the field equation (\ref{eq:U(1)FE}) 
and the last term is non-negative in the energy (\ref{eq:E^U(1)org}), 
and therefore we set $\hat A_w=0$, 
which satisfies the local energy-minimum condition. 
%
Then, one finds 
\begin{eqnarray}
E^{\rm U(1)}
=-\int d^3x dw \biggl[ 
\frac{1}{2}\hat A_0 K \hat A_0 
+\frac{N_c}{2}\rho_B \hat A_0 
\biggr].
\label{eq:E^U(1)}
\end{eqnarray}
%
For the SO(3) rotationally symmetric solution, 
we eventually obtain 
the static energy of the U(1) part: 
\begin{eqnarray}
&&E^{\rm U(1)}[\rho_B(r,w), \hat A_0(r,w)]
=4\pi \int_0^\infty dr r^2 \int_{-\infty}^{\infty}dw \cr 
&&\biggl[ 
\frac{1}{2}\hat A_0(r,w) 
K \hat A_0(r,w) 
+\frac{N_c}{2}\rho_B(r,w) 
\hat A_0(r,w) 
\biggr] \cr
&& = \int_0^\infty dr \int_{-\infty}^{\infty}dw \cr 
&& \biggl[ 
\frac{1}{2}\hat A_0(r,w) 
\tilde{K} \hat A_0(r,w) 
+2\pi N_c\tilde{\rho}_B(r,w) 
\hat A_0(r,w) 
\biggr] \cr 
&& = 4\pi \int_0^\infty dr r^2 \int_{-\infty}^{\infty}dw~{\cal E}^{\rm U(1)}(r,w), 
\label{eq:E^U(1)rw}
\end{eqnarray}
using $\tilde \rho_B(r,w) \equiv r^{2}\rho_B(r,w)$ and 
the hermite kernel $\tilde K$ in $(r,w)$-space, 
\begin{eqnarray}
\tilde K \equiv 4\pi r^2 K
=-4\pi \kappa \{h(w) \partial_r r^2 \partial_r + r^2 \partial_wk(w) \partial_w\}.~~~~ 
\label{eq:kernel_rw}
\end{eqnarray}
%
For the numerical calculation of the U(1) sector, we mainly use this energy functional. 
(For another expression of $E^{\rm U(1)}$ 
%without use of $\hat A^0$, 
after $\hat A^0$ path-integration, 
see Appendix~B.) 
 
Note again that, at the leading order of $1/\lambda$, 
the U(1) sector ($\hat A$) completely decouples 
with the SU(2)$_f$ sector ($\vec a$, $\phi$) 
because the leading term is only 
the Yang-Mills action (\ref{eq:5YM}).
%
However, at the next leading order of $1/\lambda$, 
the U(1) term affects the SU(2)$_f$ part through 
the CS term as Eq.~(\ref{eqn: S CS}). 

To summarize, the total energy $E$ comprises two parts,  
\begin{eqnarray}
    &&E[\phi(r,w), \vec a(r,w), \hat A_0(r,w)] \cr
    &=& E_{\rm 5YM}^{{\rm SU(2)}_f}[\phi, \vec a] 
    + E^{\rm U(1)}[\rho_B, \hat A_0], 
    \label{eqn:total energy}
\end{eqnarray} 
and the total energy density ${\cal E}(r,w)$ 
is written as 
%the sum of 
%${\cal E}_{\rm 5YM}^{{\rm SU(2)}_f}(r,w)$ 
%and ${\cal E}^{\rm U(1)}(r,w)$, 
\begin{eqnarray}
    {\cal E}(r,w)= {\cal E}^{{\rm SU(2)}_f}(r,w)
    + {\cal E}^{\rm U(1)}(r,w),
    \label{eqn:total energy density}
\end{eqnarray} 
and we have to deal with 
coupled field equations of the SU(2)$_f$ and U(1) sectors 
and will perform the numerical calculation 
in a consistent manner. 

\section{Vortex description of baryons}
\label{sec:vortex baryon}

In this section, 
we introduce vortex description of baryons  
in holographic QCD with 
applying the generalized Witten Ansatz. 
%
For a single baryon which is $(x,y,z)$-spatially rotational symmetric, 
applying the Witten Ansatz, we reduce the theory 
into a 1+2 dimensional Abelian Higgs theory in a curved space.
In the reduced theory, the holographic baryon is expressed as a two-dimensional topological object of an Abrikosov vortex.
%
We perform the numerical calculation 
of a $B=1$ solution of holographic QCD 
as the single ground-state baryon, 
using a fine and large lattice 
with spacing of 0.04 fm and size of 10 fm. 

%Our calculation is performed on a fine and large lattice
%with spacing of 0.04 fm and size of 10 fm. 

\subsection{
Holographic baryons in Witten Ansatz
%Topological baryons in
%Witten Ansatz in holographic QCD}
}

Large $N_c$ analyses of QCD indicate that 
explicit degree of freedoms are only mesons and glueballs, and 
baryons appear as solitons (topological objects) constructed with meson fields \cite{W79}. 
Also, holographic QCD based on large $N_c$ becomes an effective theory of mesons, 
and baryons appear as chiral solitons composed of meson fields 
in this framework \cite{SS05,NSK07}. 
Holographic QCD has four dimensional space $(x,y,z,w)$, and 
% The spatial dimension of holographic QCD is four, and 
instantons naturally appear as relevant topological objects
in the four-dimensional space. 
The topological objects are physically identified 
as baryons in holographic QCD \cite{HSSY07}
and called holographic baryons. 

Remarkably, the Witten Ansatz generally converts 
the topological description from a four-dimensional instanton into a two-dimensional vortex \cite{W77}. 
%
Accordingly, the vortex number is interpreted as the baryon number in holographic QCD with the Witten Ansatz
\cite{SH20}. 

In fact, the baryon number $B$ or the Pontryagin index is written by 
a contour integral in the $(r,w)$-plane 
\begin{eqnarray}
B &=& \int d^3x dw~ \rho_B \cr
&=&\frac{1}{2\pi}
\int_0^\infty dr \int_{-\infty}^\infty dw
\epsilon_{ij}\partial_i
\{a_j(1-|\phi|^2)+\partial_j \theta \cdot |\phi|^2\}
\nonumber \cr
&=& \oint_{r \ge 0} d{\bf s} \cdot \{{\bf a}(1-|\phi|^2)+\nabla \theta \cdot |\phi|^2\} \\
&=& \oint_{r \ge 0} d{\bf s} \cdot \nabla \theta, 
\label{eq:bnum}
\end{eqnarray}
where $\oint_{r \ge 0}$ denotes the contour integral around 
the whole half-plane of $(r,w)$ with $r\ge 0$. 
To keep the energy (\ref{eqn:E_5YM})  finite, we have imposed 
the following boundary conditions: 
% 
\begin{eqnarray}
    &|&\phi(r=0, ^{\forall} \! w)| =1, \\
    &|&\phi(^{\forall} \! r, w=\pm\infty)|= 
    |\phi(r=\infty, ^{\forall} \! w)| = 1, 
\end{eqnarray}
% 
at the edge of 
the $(r,w)$ half-plane.
%as the boundary condition 
%for the finite-energy solution. [See %Eq.~(\ref{eqn:E_5YM}).]
Thus, the baryon number $B$ is converted into the vortex number in this formalism \cite{SH20}.

\subsection{Abrikosov vortex solution for a baryon in holographic QCD}

In this subsection, we numerically calculate 
a ground-state baryon of holographic QCD 
through the Abrikosov vortex description 
in the 1+2 dimensional U(1) Abelian Higgs theory \cite{SH20}.

With imposing the global condition of $B=1$, 
we numerically minimize the total energy $E[\phi, \vec a, \hat A_0]$ in  Eq.~(\ref{eqn:total energy}),  
which is equivalent to solving the equation of motion (EOM) 
of holographic QCD for the single ground-state baryon.

Regarding the two parameters $M_{\rm KK}$ and $\kappa$ 
in holographic QCD, 
we take $M_{\rm KK} \simeq$ 948~MeV, and $\kappa=7.46 \times10^{-3}$ 
to reproduce $f_\pi \simeq$ 92.4~MeV and $m_\rho \simeq$ 776~MeV \cite{SS05,NSK07}.
In this study, we have used 
the Kaluza-Klein unit of $M_{\rm KK}=1$.

For the numerical calculation on the $(r, w)$ plane, 
we adopt a fine and large-size lattice 
with spacing of $0.2~M_{\rm KK}^{-1} \simeq 0.04~{\rm fm}$ and 
the extension of $0 \le r \le 250$ and $-125 \le w \le 125$, 
that is, the system size is 
$250 \times 250$ grid corresponding to 
$(50~M_{\rm KK}^{-1})^2 \simeq (10~{\rm fm})^2$
in the physical unit. 
%
(In this numerical calculation, there appears subtle cancellation, 
and use of a coarse lattice might lead to an inaccurate result.
Also, the lattice size should be increased 
until the volume dependence of physical quantities disappears.)

On this lattice, 
starting from the 't~Hooft solution \cite{BPST, tH76} 
as a $B=1$ topological configuration, 
we numerically perform 
minimization of the total energy 
$E[\phi, \vec a, \hat A_0]$ 
keeping the topological charge 
by an iterative improvement, that is, 
many-time iterative local replacements  
of the field variable of 
$\phi$, $\vec a$ and $\hat A_0$. 
(See Appendix~A for the detail.)
%
During the update, 
the Higgs field $\phi(x)$ always has a zero point to ensure $B=1$, 
and the Higgs zero-point generally moves so as to realize the ground state.
%
In this way, we eventually 
obtain the holographic fields 
$\phi(r,w)$, $\vec a(r,w)$ and $\hat A_0(r,w)$ 
for the ground-state baryon
as the true solution of EOM in holographic QCD. 

For the confirmation of numerical calculations, 
we also consider another different method, 
as shown in Appendix~B. 
In this alternative method, 
we integrate out the U(1) gauge field $\hat A_0$ 
and update only $\phi(r,w)$ and $\vec a(r,w)$ 
on the lattice based on Eq.~(\ref{eqn:U(1) energy}). 
%
We have confirmed that both methods give 
the same numerical results for holographic baryons.

To visualize gauge and Higgs fields composing the Abrikosov vortex, 
we take the Landau gauge for the U(1) gauge degrees of freedom,  $\partial_i a_i(r,w) = 0$.
Of course, main results including the total energy 
are gauge invariant and never depend on any gauge choices. 

For the single ground-state baryon, 
we show field configurations $\phi(r,w)$ and $\vec a(r,w)$ in Fig.~\ref{fig:phi a GS}, 
and also the U(1) gauge field $\hat A_0(r,w)$ in Fig.~\ref{fig:aU1 GS}. 
%
The Higgs field $\phi(r,w)$ indicates a clear topological structure 
characterizing the Abrikosov vortex, 
which is mapped into an instanton \cite{W77} via the Witten Ansatz 
and physically means a baryon in holographic QCD \cite{BS14,SH20}. 
%
In accordance with the field equation (\ref{eq:U(1)FE}), 
$\hat A_0$ is localized 
around the non-vanishing topological density $\rho_B$, 
which will be shown in Fig.~\ref{fig:ene TpC GS rwd}. 

% Figure environment removed

% Figure environment removed

Quantitatively, unlike the initial 't~Hooft solution, 
Fig.~\ref{fig:phi a GS} no longer indicates 
the symmetry between $r$ and $w$ 
for the ground-state baryon solution in holographic QCD. 
%
%a symmetry of the $r$ and $w$ direction of the ground-state baryon 
%configuration is broken in HQCD. 
%
For the ground-state baryon, 
both profiles of $\phi$ and $\vec a$ 
are found to be a little shrink in the $w$ direction,  
compared with four-dimensional spherical 't~Hooft solutions, which was also found 
in the previous numerical studies \cite{BS14,RSR14,SH20}.

% This ground-state solution provides a unique profile and a definite size. 

\subsection{Properties of 
the ground-state baryon 
in holographic QCD}

Now, we show the properties of 
the ground-state baryon 
in holographic QCD, which is numerically calculated by minimizing the total energy $E[\phi, \vec a, \hat A_0]$ in Eq.~(\ref{eqn:total energy}). 
In Fig.~\ref{fig:ene TpC GS rwd}, 
we show the topological density $\rho_B(r,w)$ 
and total energy density ${\cal E}(r,w)$ 
in the $(r,w)$-plane for the Abrikosov vortex solution in the 1+2 dimensional Abelian Higgs theory.
Both densities have a peak around $(r, w)=(0, 0)$ 
and are extended in both the $r$ and $w$ directions. 
%
We show in Fig.~\ref{fig:ene TpC GS rw} 
the densities multiplied by the integral measure factor $r^2$, 
i.e., $4\pi r^2 \rho_B(r,w)$ and $4\pi r^2 {\cal E}(r,w)$, 
for the ground-state baryon, 
since 
$\tilde \rho_B (r,w) \equiv r^2 \rho_B (r,w)$ 
is a primary variable in this numerical calculation, 
as shown in Eq.~(\ref{eq:kernel_rw}). 
The non-zero size of the baryon is 
due to the repulsive force from the CS term \cite{HSSY07}. 

% Figure environment removed

% Figure environment removed

In general, the integrated topological density is the baryon number $B$, 
and the baryon mass $M_B$ is given by integration of 
the total energy density ${\cal E}(r,w)$: 
% as the static energy of the ground-state baryon 
\begin{eqnarray}
B&=&\int d^3x dw~ \rho_B(r,w) \cr
&=&4\pi \int_0^\infty dr r^2 \int_{-\infty}^\infty dw~ \rho_B(r,w),
\\
M_B&=&\int d^3x dw~ {\cal E}(r,w) \cr
&=&
4\pi \int_0^\infty dr r^2 \int_{-\infty}^\infty dw~ {\cal E}(r,w).
\label{eqn:baryon mass}
\end{eqnarray}
By integration over the extra coordinate $w$, 
we obtain the ordinary densities in a three-dimensional space, 
\begin{eqnarray}
\rho_B(r) &\equiv& \int_{-\infty}^\infty dw~\rho_B(r,w), \\
{\cal E}(r) &\equiv& \int_{-\infty}^\infty dw~{\cal E}(r,w). 
\label{eqn:ground energy}
\end{eqnarray}
The mass $M_B$ and size for the ground-state baryon are estimated as
\begin{eqnarray}
M_B &=& 
E_{\rm 5YM}^{{\rm SU(2)}_f}+E^{\rm U(1)}  
=\int d^3x ~{\cal E}(r) \cr
&\simeq& 1.25~{M_{\rm KK}} \simeq 1.19~{\rm GeV}, 
\label{M_B}
\end{eqnarray} 
\begin{eqnarray}
\sqrt{\langle r^2 \rangle}_{\rho_B} 
&\equiv& \sqrt{\frac{\int d^3x~ \rho_B(r) r^2}{\int d^3x~ \rho_B(r)}}
\cr 
&\simeq& 2.58 {M_{\rm KK}}^{-1} 
 \simeq 0.54\ {\rm fm},
 \label{r^2_rho_B}
\end{eqnarray} 
\begin{eqnarray}
\sqrt{\langle r^2 \rangle}_{\cal E} 
&\equiv& \sqrt{\frac{\int d^3x~ {\cal E}(r) r^2}{\int d^3x~ {\cal E}(r)}} 
\cr
&\simeq& 2.93\ {M_{\rm KK}}^{-1} 
\simeq 0.61\ {\rm fm}.
\label{r^2_rho_E}
\end{eqnarray} 

Here, some cautions are commented. 
%The initial energy before the numerical process of minimizing iteration 
%is $1.4\ {\rm GeV}$, and it decreases along with solving EOM. 
% These minimizing iteration start from 't~Hooft solution. 
When the self-dual BPS-saturated 't~Hooft solution~\cite{BPST,tH76} 
is simply used, as was done in Ref.~\cite{HSSY07}, 
the holographic baryon has an overestimated mass 
$M_B^{\rm BPS} \simeq 1.35~M_{\rm KK} \simeq 1.28~{\rm GeV}$ 
and a smaller radius $\sqrt{\langle r^2\rangle_{\rho_B}^{\rm BPS}} 
%\simeq 1.8~M_{\rm KK}^{-1}\simeq 0.38~{\rm fm}$, 
\simeq 2.2~M_{\rm KK}^{-1}\simeq 0.46~{\rm fm}$, 
as shown in Appendix~C. 
%
(Because of such an overestimation, 
a small value of $M_{\rm KK}=500~{\rm MeV}$ was adopted 
to adjust baryon masses in Ref.\cite{HSSY07}, 
which significantly differs from $M_{\rm KK} \simeq 1~{\rm GeV}$ 
for the meson sector.)
%
Another caution is the numerical accuracy, 
and fine and large lattices are to be used for the numerical calculation. 
%
Owing to a relatively coarse and small-size lattice, 
the numerical results in the previous paper \cite{SH20} 
include about 20\% error for the baryon mass and size. 

Now, we investigate spatial distribution of 
the baryon-number and energy densities 
for the ground-state baryon in holographic QCD. 
%
Figure~\ref{fig:ene TpC GS r} shows
the baryon-number density $\rho_B(r)$ 
(i.e. topological density) 
and total energy density ${\cal E}(r)$, 
and their $r^2$-multiplied values, 
$4\pi r^2 \rho_B(r)$ and $4\pi r^2 {\cal E}(r)$. 
% 
One finds significant difference between the shapes of 
$\rho_B(r)$ and ${\cal E}(r)$ for the small $r$ region. 
% 
% Figure environment removed

Next, we investigate the energy contribution 
from the SU(2)$_f$ and U(1) parts, respectively. 
For the mass (total static energy)  
$M_B =E_{\rm 5YM}^{{\rm SU(2)}_f}+E^{\rm U(1)} 
$
of the ground-state holographic baryon, 
we obtain 
\begin{eqnarray}
E_{\rm 5YM}^{{\rm SU(2)}_f} &\simeq& 1.00 M_{\rm KK}
\simeq 0.95~{\rm GeV}, \\
E^{\rm U(1)} &\simeq& 0.25 M_{\rm KK}
\simeq 0.24~{\rm GeV}, 
\end{eqnarray}
and hence the SU(2)$_f$ contribution 
(leading order of $1/\lambda$ expansion) 
is found to be quantitatively dominant.

As spatial distribution, 
the SU(2)$_f$ and U(1) energy densities are expressed as 
\begin{eqnarray}
&&r^2{\cal E}^{{\rm SU(2)}_f} (r) = \kappa \int^{\infty}_{-\infty} dw \biggl[ h(w) |D_1\phi|^2 + k(w) |D_2\phi|^2 \cr 
&& \qquad \qquad + \frac{h(w)}{2r^2}\{1-|\phi|^2\}^2 + \frac{r^2}{2} k(w) f_{12}^2 \biggr], \\
\label{eqn:E_5YM (r)}
&&{\cal E}^{\rm U(1)}(r) = \int_{-\infty}^{\infty} dw \cr 
&& \biggl[ 
\frac{1}{2}\hat A_0(r,w) 
 K \hat A_0(r,w) 
+\frac{N_c}{2} \rho_B(r,w) 
\hat A_0(r,w) 
\biggr].
\label{eq:E^U(1) (r)}
\end{eqnarray}
%
Figure~\ref{fig:rho energy 4 types} 
shows 
SU(2)$_f$ energy density ${\cal E}^{{\rm SU(2)}_f}(r)$ and U(1) energy density ${\cal E}^{\rm U(1)}(r)$ of the ground-state baryon, together with the total energy density ${\cal E}(r)$ and baryon density $\rho_B(r)$. 
% Figure environment removed

The dominant contribution is the SU(2)$_f$ part, 
and the total value ${\cal E}(r)$ is approximated by 
the SU(2)$_f$ energy density ${\cal E}^{{\rm SU(2)}_f}(r)$, 
which seems to be consistent with $1/\lambda$ expansion.
In particular, ${\cal E}^{{\rm SU(2)}_f}(r)$ and ${\cal E}(r)$ has the same slope 
at the origin $r=0$ and show enhancement for small $r$ region, 
although it is masked by the integral measure $r^2$. 
%
The shape of ${\cal E}^{\rm U(1)}(r)$ seems to follow $\rho_B(r)$,  
reflecting the direct coupling beteen $\hat{A}_0$ and $\rho_B$ 
in the CS term (\ref{eqn: S CS}).  


Finally, we investigate self-duality breaking of the holographic baryon. 
%
The different shape between the topological and energy densities 
originates from the background gravity, $h(w)$ and $k(w)$, 
and presence of the CS term. 
%
In the case of $h(w)=k(w)=1$ without the CS term, 
the instanton solution has exact self-duality of 
$F_{MN} = \tilde F_{MN}$, 
leading to $\rho_B(r,w) \propto {\cal E}(r,w)$. 
Hence, the functional forms of $\rho_B(r)$ and ${\cal E}(r)$ 
are forced to be the same. 
In fact, between the shapes of $\rho_B(r)$ and ${\cal E}(r)$, 
the similarity indicates the self-dual tendency, 
and the different point indicates self-duality breaking. 

For the single baryon case $B=1$, 
we introduce the self-duality breaking parameter defined by 
\begin{eqnarray}
    \Delta_{\rm DB} &\equiv& 
    \frac{
    \int d^3x dw~{\rm tr}( F_{MN}F_{MN} - F_{MN}\tilde{F}_{MN}) }{ \int d^3x dw~ {\rm tr} (F_{MN}\tilde{F}_{MN})} \cr
    &=&
    \frac{1}{32\pi^2} \int d^3x dw~{\rm tr}
    ( F_{MN} - \tilde{F}_{MN})^2, 
    \label{eq:self-duality breaking parameter}
\end{eqnarray}
which is normalized by the topological quantity. 
This is non-negative and becomes zero only 
in the exact self-dual case.
The self-duality breaking of the ground-state baryon is 
found to be ${\Delta}_{\rm DB} \simeq 0.17$, 
which is non-zero but seems small. 
The small value might indicate that 
the true holographic configuration is close to be self-dual. 
Then, the ground-state baryon might be approximated 
by the self-dual 't~Hooft instanton in holographic QCD.

\section{Size-dependence of a holographic baryon}\label{sec:size-dependence}

As above mentioned, by using 
the Witten Ansatz, 1+4 dimensional 
holographic QCD is reduced into 
a 1+2 dimensional Abelian Higgs theory. 
%
Accordingly, the topological description 
of a baryon is changed from an instanton 
to a vortex. 
%
In the previous section, 
we have numerically obtained the ground-state solution in holographic QCD, 
where the baryon size is automatically determined by minimizing the total energy. 

Now, let us consider a holographic baryon with various size. 
%
Note that the size is originally one of the moduli of an instanton in a flat space, 
and different size baryons are to be 
degenerate in holographic QCD, 
if one sets $h(w)=k(w)=1$ and 
neglects the CS term. 
%
% Let us consider an excitation of the baryon.
%It is worth considering the energy of a baryon with different sizes 
%because the size is originally one of the moduli of an instanton in a flat space. 
%
In the real holographic QCD,
the size parameter is no more modulus, 
according to the background gravity and CS term. 
%
Nevertheless, the size might behave as a quasi-modulus in the holographic baryon, resulting in physical appearance of a soft vibrational mode as a low-lying excitation. 
%
Then, we investigate 
a baryon with various size 
and size dependence of the energy 
in holographic QCD in this section. 

First, we consider the ordinary Yang-Mills theory 
and examine the moduli relation between an instanton and a vortex 
in the Witten Ansatz, 
as was originally shown by Witten \cite{W77}. 

Next, we proceed to the holographic baryon, and consider how 
to obtain an arbitrary-size baryon as a solution of holographic QCD, 
and investigate the size dependence of 
the baryon mass. 

%In the first subsection, 
%we summarize relations between an instanton and 
%a vortex in the ordinary Yang-Mills theory, 
%which is important for our baryon size analysis.

%In the next subsection, in holographic QCD, we suggest a way to 
%get an arbitrary size baryon and investigate 
%the size dependence of the mass.

\subsection{Instanton-vortex correspondence in Yang-Mills theory}

In the four-dimensional Euclidean Yang-Mills theory, 
there exist topological solutions, instantons, 
where Euclidean time is necessary. 
A single instanton solution (BPST-'t~Hooft solution~\cite{BPST,tH76}) 
is written as 
\begin{eqnarray}
	A_\mu(x) = - \eta_{\mu\nu}^a\tau^a \frac{x^\nu}{(x-X)^2+R^2}, \label{eq:BPST 4dim}
\end{eqnarray}
where the instanton center locates at $x^\mu=X^\mu$, 
and $R$ denotes the instanton size. 
Together with color rotation, the location $X^\mu$ and the size $R$ are known as moduli, 
and they represent degrees of freedom for this topological solution, that is, their values do not affect the Yang-Mills action. 
Here, $\eta_{\mu\nu}^a$ denotes the 't~Hooft symbol \cite{tH76} defined by
\begin{eqnarray}
\eta_{\mu\nu}^a
%=\frac{1}{2i} {\rm tr}(\sigma_a \tau_\mu \tau^\dagger_\nu)=
=-\eta_{\nu\mu}^a=\Bigg\{\begin{array}{l}\epsilon_{a\mu\nu}~~~~{\rm for}~~\mu,\nu=1,2,3\\-\delta_{a\nu}~~~{\rm for}~~\mu=4 \\ ~~\delta_{a\mu}~~~{\rm for}~~\nu=4. 
\end{array} 
\end{eqnarray}
%
% The center $L^\mu$ and the size $a$ are moduli and do not affect the Yang-Mills action.
Taking its center $X^\mu=( \vec{0}, T)$, there is SO(3) rotational symmetry in $(x,y,z)$-space, and the Witten Ansatz is applicable. 
With the Witten Ansatz, 
the four-dimensional Yang-Mills theory is reduced into 
a two-dimensional Abelian Higgs theory, and 
this instanton can be described by a single vortex. 

To understand the relation 
between an instanton and a vortex, 
let us consider the 't~Hooft solution in the form of the Witten Ansatz. 
This represents a single instanton solution 
and is rewritten as 
\begin{eqnarray}
	A_i^a &=& \frac{2r}{r^2+(t-T)^2+R^2}\epsilon_{iaj} \hat{x}_j \cr 
	&& - \frac{2(t-T)}{r^2+(t-T)^2+R^2}(\hat{\delta}_{ia}+\hat{x}_i\hat{x}_a), \\
%	& & + \frac{-2(t-T)}{r^2+(t-T)^2+R^2} \hat{x}_j\hat{x}_a \\ 
	A_t^a &=& \frac{2r}{r^2+(t-T)^2+R^2} \hat{x}_a . 
\end{eqnarray}
%
By comparing the functional form 
with Eqs.~(\ref{eqn:Original Witten Ansatz 1}) and (\ref{eqn:Original Witten Ansatz 2}), 
SU(2) gauge fields can be converted into 
the fields of the reduced Abelian Higgs theory, 
and their forms are obtained as 
\begin{eqnarray}
&&    (\phi_1,\phi_2+1) =\frac{2r}{r^2+(t-T)^2+R^2}  
    \left( {-(t-T)}, {r} \right),      
%    \left( \frac{-2r(t-l)}{r^2+(t-l)^2+a^2}, -1 + \frac{2r^2}{r^2+(t-l)^2+a^2} \right) 
    \\ 
&& a_1  = \frac{-2(t-T)}{r^2+(t-T)^2+R^2},\ a_2 = \frac{2r}{r^2+(t-T)^2+R^2}.~~~~
\end{eqnarray}
The complex Higgs field 
$\phi=({\rm Re} \phi, 
\ {\rm Im} \phi) = (\phi_1, \phi_2)$ 
takes zero at $(r,t) = (R,T) \equiv \zeta$. 
The vortex number is counted as the zero-point number in the Higgs field $\phi$ 
in the $(r,t)$-plane. 
Now, there is one zero point at $\zeta$, and this configuration represents a single vortex. 

Thus, in the Witten Ansatz, 
the vortex corresponding to a single instanton 
has a zero point $\zeta$ of the Higgs field $\phi$. 
Remarkably, 
this Higgs-field zero point $\zeta$ relates to 
instanton parameters \cite{W77}, 
\begin{eqnarray}
    \zeta =(\zeta_{r}, \zeta_{t})= (R,\ T). \label{eqn:moduli relation}
\end{eqnarray}
% \begin{eqnarray}
%     l = {\rm Im} [z_0],\quad a = {\rm Re} [z_0], \label{eq:inst_vor_relation}
% \end{eqnarray}
In fact, the instanton size $R$ and Euclidean fourth-coordinate $T$ 
of the instanton center correspond to this zero point $\zeta$ in the Higgs field $\phi$. 
%
In this configuration, the topological density is written by 
\begin{eqnarray}
    \rho_B = \frac{6}{\pi^2} \frac{R^4}{[r^2+(t-T)^2+R^2]^4}.
\end{eqnarray}
The topological density localizes 
around $(r,t)=(0,T)$, 
and its extension is about the size parameter $R$, 
which reflects the instanton size. 

\subsection{Various-size baryon mass in holographic QCD}

In this subsection, 
using the above correspondence (\ref{eqn:moduli relation})
between the Higgs zero-point $\zeta$ and the instanton size $R$ in the Witten Ansatz, 
we try to control the baryon size by changing 
the location of the zero point $\zeta$ in the Higgs field $\phi$ in holographic QCD. 
%
Using this new viewpoint, 
we obtain various sizes of holographic baryons 
and investigate size dependence 
of the single baryon energy. 

To begin with, we reformulate the above argument 
for 1+4 dimensional holographic QCD.
We recall different points 
from the ordinary Yang-Mills theory. 
% 
First, holographic QCD already has four-dimensional Euclidean 
space of $(x,y,z,w)$, and the instanton can be naturally defined 
on this space including the fourth spatial coordinate $w$,  
without use of Euclidean process.
%Thus, we can apply the instanton form to this four-dimensional space 
%only by relabeling Euclidean-time coordinate $t$ 
%in the 't~Hooft solution 
%to the fourth spatial coordinate $w$. 
Second, there appear the gravitational factor, $h(w)$ and $k(w)$, 
and the CS term, which results in a repulsive interaction 
among the baryon density $\rho_B$ in holographic QCD. 
Owing to these effects, the 't~Hooft solution is no longer the exact solution but an approximate one in holographic QCD. 

Therefore, we use the 't~Hooft solution as 
a starting point for the $B=1$ configuration, 
and search for the true solution of holographic QCD 
by a numerical iterative method 
with keeping the topological property unchanged, 
similarly in Sect.~IV. 

% because we investigate a single baryon $B=1$ and their topological property is the same and the formalism too. 

%\subsection{Various size baryon mass in holographic QCD}

%Unlike the case of the Yang-Mills theory, 
% Since 1+4 dim holographic QCD already has a four-dimensional Euclidean space of $(x,y,z,w)$, 
% the instanton description is available without dealing with Euclidean time.
%
In 1+4 dimensional holographic QCD, taking the $A_0=0$ gauge, 
we use a holographic version of 
the 't~Hooft solution (\ref{eq:BPST 4dim}),  
as a starting point of the topological configuration. 
Here, we locate the instanton center at the four-dimensional 
spatial origin $(x,y,z,w)=(0,0,0,0)$ for symmetry of 
$(x,y,z)$-spatial rotation and $w$-reflection,
and then the initial configuration is set to be 
\begin{eqnarray}
	A_0 &=& 0, \\
	A_M &=& - \eta_{MN}\frac{x^N}{x^2+R^2} 
 \label{eqn:'t hooft soltion HQCD},
\end{eqnarray}
which can be rewritten as
\begin{eqnarray}
    % A_0^a &=& 0, \\
	A_i^a &=& \frac{2r}{r^2+w^2+R^2}\epsilon_{iaj} \hat{x}_j \cr 
	&& - \frac{2w}{r^2+w^2+R^2}(\hat{\delta}_{ia}+\hat{x}_i\hat{x}_a), \\
%	& & + \frac{-2w}{r^2+w^2+R^2} \hat{x}_j\hat{x}_a \\ 
	A_w^a &=& \frac{2r}{r^2+w^2+R^2} \hat{x}_a. 
\end{eqnarray}
%
%As above mentioned, 
For the $(x,y,z)$-rotational symmetric system, 
through the Witten Ansatz, 
the non-Abelian gauge fields are 
converted into 
the Higgs field $\phi$ 
and Abelian gauge field $\vec a$, 
%in an Abelian Higgs theory, 
and they are expressed as
%
%The corresponding Higgs $\phi$ and %gauge fields $a_\mu$ are written as
\begin{eqnarray}
&&    (\phi_1,\phi_2+1) =\frac{2r}{r^2+w^2+R^2} 
    \left( -w, r \right), \label{eqn:phi HQCD}\\ 
%&& a_0 = 0, \cr
&& a_1  = \frac{-2w}{r^2+w^2+R^2},\ a_2 = \frac{2r}{r^2+w^2+R^2}. 
\end{eqnarray}
%
In Eq.~(\ref{eqn:phi HQCD}), 
the zero point $\zeta$ of the Higgs field $\phi$ 
is found to locate at 
%\begin{eqnarray}
$    \zeta =(\zeta_{r}, \zeta_{w})= (R,\ 0),$ %\label{eqn:moduli relation HQCD}
%\end{eqnarray}
and this $\zeta_r$ determines the initial size of the holographic baryon.

From this initial configuration, similarly in Sect.~IV, 
we numerically search for the single baryon solution in holographic QCD 
by minimizing the total energy $E[\phi, \vec a, \hat A_0]$, 
with fixing the Higgs-zero location 
\begin{eqnarray}
\zeta=(\zeta_r,\zeta_w)=(R,0),
\label{eqn:moduli relation HQCD}
\end{eqnarray}
where the former $R$ corresponds to 
the instanton/baryon size. 
%
% **cut -> 
% between an instanton and a vortex, 
% This represents a single instanton solution 
% and is rewritten as 
%
% %
% % As the starting point, 
% % five dimensional version of
% % Eq.~(\ref{eq:BPST 4dim}) is useful 
% % for numerical calculation of 
% % the $B=1$ single baryon configuration, 
% %
% % The index $M$ takes spatial coordinates 1--4, including the fourth spatial direction $x^4 \equiv w$.
% %
% This configuration is an exact solution for the flat Yang-Mills theory, 
% but an approximated one in holographic QCD 
% because there are gravitational effects, i.e., non-flat metric $h(w)$ and $k(w)$, 
% and the CS term, which breaks the BPS saturation. 
%
% In the previous studies \cite{RSR14,SH20}, 
% a true solution is calculated numerically, and 
% its shape is distorted from a spherical one to the $w$ direction 
% compared with a spherical solution in Eq.~(\ref{eqn:'t hooft soltion HQCD}). 
% This ground-state solution provides a unique profile and a definite size. 
%
% ** -> cut 
%
% Now, we attempt to obtain different size baryons and investigate their mass. 
% Here we consider a rotational symmetric baryon, for which the Witten Ansatz is applicable to holographic QCD. 
% Using the Witten Ansatz, 1+4 dim SU($N_f$) Yang-Mills theory reduces to 1+2 dim Abelian Higgs theory. 
% In this reduction, rotationally symmetric instantons in Yang-Mills theory correspond to vortices 
% in Abelian Higgs theory. 
%There also exists the relation between instanton profiles and the zero point $\zeta$ of vortex $\phi$ field \cite{W77}. 
%
In fact, to consider various size baryons, 
we here use the correspondence between the instanton/baryon size 
and zero point $\zeta$ in the Higgs field $\phi$ composing 
the Abrikosov vortex \cite{W77}.
%
Note that the Higgs field must be zero 
at the center of the Abrikosov vortex 
to realize the finite energy, 
and therefore the Higgs field $\phi$ 
inevitably has a zero-point $\zeta$ 
in the presence of the vortex 
corresponding to $B=1$ in holographic QCD. 
%
Here, to change the zero-point location $\zeta_r=R$ 
corresponds to a baryon-size change, 
and its various changes lead to different-size baryons. 

In summary, to obtain various-size baryon solutions, 
we fix this zero-point location $\zeta=(\zeta_r,\zeta_w)=(R,0)$ of 
the Higgs field $\phi$ as a boundary condition 
and minimize the total energy $E$ numerically. 
%
When the total energy $E[\phi, \vec a, \hat A_0]$ 
is minimized against arbitrary local variation 
of $\phi$, $\vec a$ and $\hat A_0$,   
these fields satisfies the EOM of holographic QCD.
%even under the Higgs-zero constraint as a boundary condition.

For the numerical calculation on the $(r, w)$ plane, 
we use the same lattice used in Sect.~IV, that is, 
a fine and large-size lattice 
with spacing of $0.2 M_{\rm KK}^{-1} \simeq 0.04~{\rm fm}$ and 
the extension of $0 \le r \le 250$ and $-125 \le w \le 125$, 
of which physical size is  
$(50~M_{\rm KK}^{-1})^2 \simeq (10~{\rm fm})^2$. 

After an iterative improvement of holographic fields 
with the constraint of the Higgs zero-point location, 
we eventually obtain the Higgs and Abelian gauge fields, 
\begin{eqnarray}
\phi(r, w; R), \quad \vec a(r, w; R), \quad \hat{A}_0(r, w; R)
\end{eqnarray}
for the holographic baryon corresponding to the instanton size $R$.
Note again that the presence of a zero point of the Higgs field 
indicates a $B=1$ configuration, and 
the holographic fields obeying the local energy minimum condition 
also satisfy the EOM of holographic QCD. 

% and this change produces another  baryon.
% Thus, we can obtain a variety size of baryons in holographic QCD.
% The figure ~\ref{fig:various_size_conf} shows configurations with various size baryons
% % Figure environment removed
% Figure~\ref{fig:conf R=3.2} shows typical configurations of the Higgs field $\phi(r,w)$ for a baryon
% of size $R=3.2$. 
% Each configuration is obtained by solving the EOM under the constraint on the zero point $\zeta$ to be a fixed location, this is the exact solution of holographic QCD. 
% The upper and lower figures show the Higgs field $\phi = ({\rm Re}\phi, {\rm Im}\phi)$ with the zero point $\zeta=(R,0)$ at $R=1.4$ and $R=3.0$, respectively.
% It is noted that, for each constraint, our obtained configuration is a true numerical solution and satisfies the field equations of holographic QCD. 

% Figure environment removed
% 
Figure~\ref{fig:dilatation_potentail} shows 
the static energy $V(R)$ of the baryon with various size $R$. 
The potential minimum corresponds to the ground-state baryon. 
%the size $R$ dependence for baryon static energy $V(R)$. 
% The plots denote the static energy of the baryon with different given sizes. 
The numerical data are well fitted 
by a quadratic function, as shown in Fig.~\ref{fig:dilatation_potentail}. 
% since the data seem to be quadratic against the size $R$. 
% baryon size $a$ dependence of baryon energy $V$ and seems to be quadratic against size $R$. 
Note that all the symbols represent the solution of holographic QCD 
under the constraint of size fixing, 
which can be regarded as a boundary condition. 

We obtain the fitting quadratic function,
\begin{eqnarray}
&& V(R) \simeq A (R-R_0)^2 + M_0, \cr 
&& ~~~~~ \label{eqn:potential fitting data} \\
&&	A \simeq 0.063, \quad 
	R_0 \simeq 2.4, \quad
	M_0 \simeq 1.25, \notag{}
\end{eqnarray}
in the $M_{\rm Kk}=1$ unit.  
The value $M_0$ of potential minimum coincides with the previous calculation of the ground-state baryon mass $M_B \simeq 1.25~M_{\rm KK}$ in Eq.~(\ref{M_B}). 
%

% The size $R_0$ at potential minimum is also consistent with 
% the Higgs zero-point location for the ground-state holographic baryon 
% in Fig.~\ref{fig:phi a GS}. 

The size parameter $R=R_0$ which minimizes the static energy $V(R)$ of the baryon is found to be $R_0 \simeq 2.4$ 
in the $M_{\rm KK}=1$ unit. 
This result coincides with the ground-state result shown in Fig.~\ref{fig:phi a GS}, 
where the Higgs zero-point $\zeta_r$ locates at about $ R = R_0$. 
% 
% this result coincides with the ground-state result Eq.~(\ref{eqn:ground energy}). 
If there were no non-trivial gravity 
($h(w)$, $k(w)$) and no CS term, 
this size dependence would disappear and the potential $V(R)$ would become flat 
because the instanton size is originally a modulus, reflecting classical scale invariance of the Yang-Mills theory. 
In fact, this size dependence of holographic baryon mass originates from those gravitational effects and CS term. 

\subsection{Analysis of holographic baryon size}

In this subsection, we investigate  
actual size of holographic baryons 
obtained in the previous subsection 
in terms of the size parameter $R$, 
which corresponds to the instanton size. 

In our calculation, 
for each constraint of fixing the size $R$, 
we already obtain the topological density $\rho_B$ 
%and energy density ${\cal E}$,
\begin{eqnarray}
    \rho_B &=& \rho_B(r,w;R). 
\end{eqnarray}
%
Using this gauge-invariant local quantity, 
we investigate the size of the holographic baryon 
for each direction of $r$ and $w$. 
Here, we define 
$\rho_B$-weighted 
average of arbitrary $(x,y,z)$-rotational symmetric variable 
$O(r,w)$ as 
\begin{eqnarray}
    \langle O \rangle_{\rho_B(R)} &\equiv& 
    \frac{\int_0^\infty drr^2 \int_{-\infty}^\infty dw
    \rho_B(r,w;R) O(r,w)}
    {\int_0^\infty drr^2 \int_{-\infty}^\infty dw
    \rho_B(r,w;R)}. ~~~~ 
\end{eqnarray}

In the ordinary four-dimensional Euclidean Yang-Mills theory, 
the Pontryagin density $\rho$ of a single instanton 
with the size parameter $R$ and its center at the origin 
is given by 
\begin{eqnarray}
    \rho = \frac{1}{16\pi^2}{\textrm tr}\left( F_{\mu\nu}\tilde{F}_{\mu\nu} \right) = \frac{6}{\pi^2}\frac{R^4}{(x_\mu^2+R^2)^4},
\end{eqnarray}
and the mean square radius weighted with $\rho$ 
is evaluated as 
\begin{eqnarray}
\langle r^2 \rangle_\rho= \frac{3}{2}R^2, \quad
\langle t^2 \rangle_\rho=\frac{1}{2}R^2
\quad {\rm (YM~instanton)},
\end{eqnarray}
or equivalently
\begin{eqnarray}
\sqrt{\frac{2}{3}\langle r^2 \rangle_\rho}= 
\sqrt{2 \langle t^2 \rangle_\rho}=R
\quad {\rm (YM~instanton)},
\label{eq:instanton_size}
\end{eqnarray}
for $r\equiv (x^2+y^2+z^2)^{1/2}$ and the Euclidean time $t$.

% **** Under Const ***
% we define 
% $\rho_B$-weighted radius of the holographic baryon, 
% in a similar manner to Eqs.~(\ref{r^2_rho_B}) and (\ref{r^2_rho_E}) in Sect.~IV. 
% ***** Under Const ****

The holographic baryon is able to be dilatated simultaneously in both $r$ and $w$ direction, imposing the Higgs field zero-point constraints, 
and we obtain various size baryons. 
%
%
Based on the above relation, 
we consider size parameters 
of the holographic baryon 
in $r$ and $w$ directions, respectively. 
%
In a similar manner to Eq.~(\ref{r^2_rho_B}) in Sect.~IV, 
we define the radius weighted 
with the topological density $\rho_B$ as 
\begin{eqnarray}
    d_r(R) \equiv \sqrt{\frac{2}{3}\langle r^2 \rangle_{{\rho}_B(R)}}, \quad d_w(R) \equiv \sqrt{2\langle w^2 \rangle_{{\rho}_B(R)}}, ~~~~~~ 
\end{eqnarray}
in the direction $r$ and $w$, respectively. 
To compare with the instanton size parameter $R$, 
the factors, $2/3$ and $2$, have been introduced, considering Eq.~(\ref{eq:instanton_size}). 
In fact, the ordinary self-dual Yang-Mills instanton satisfies  
$d_r(R)=d_w(R)=R$. 

Figure~\ref{fig:shape} shows the $\rho_B$-weighted radius $d_r(R)$ and $d_w(R)$ as a function of $R$, 
weighted with the topological density. 

% Figure environment removed
The solid line denotes $d=R$ and 
is realized in the case of the ordinary four-dimensional YM theory ($h(w)=k(w)=1$) without the CS term. 
In other words, if the holographic baryon were described with 
the 't~Hooft solution, one would find $d_{r}(R)=d_{w}(R)=R$. 

%They would be due to the $h(w)$ or $k(w)$ 
%since the CS term acts r, w-direction equally
%
%dilation = approx r-direction oscillation in HQCD 

From Fig.~\ref{fig:shape}, 
$d_r$ and $d_w$ are found to be monotonically 
increasing along with $R$. 
This monotonical increase reflects 
that the change of Higgs zero-point gives 
the change of holographic baryon (instanton) size. 
Later, we investigate the dilatation mode 
by regarding the size $R$ as dynamical degree of freedom. 

Around the ground state, 
the value of $d_r(R)$ is larger than $d_w(R)$, 
i.e., $d_r(R) > d_w(R)$, 
and this means an oblate-shaped instanton 
for the holographic baryon \cite{RSR14}. 

The slope of $d_r(R)$ is larger than $d_w(R)$, 
and $d_w(R)$ is almost flat against $R$. 
Therefore, the size change of holographic baryons 
is approximately regarded as 
three-dimensional in $r$-direction 
rather than four-dimensional \cite{HSSY07}. 
(See Appendix C and D for 
four-dimensional size change 
using the BPS instanton.) 

These differences of the behavior for each direction 
would come from the nontrivial gravity fields, 
$h(w)$ and $k(w)$, 
because the CS term (\ref{eqn: S CS}) equally acts 
in $r$ and $w$-direction and 
does not break $(x,y,z,w)$ $O(4)$ symmetry. 
%
In fact, if $h(w)=k(w)=1$, $O(4)$ symmetry is exact and 
no $(r,w)$-asymmetry appears. 
Therefore, $h(w)$ and $k(w)$ are the very origin of 
$(r,w)$-asymmetry. 

The flatness of $d_w(R)$ against $R$ 
indicates that gravity fields $h(w)$ and $k(w)$ 
suppress the $w$-direction swelling. 
%
Approximately, one finds $d_r(R) \sim d_w(R)$, 
and then $d_r(R)$ seems to follow $d_w(R)$ 
and its soft $R$-dependence, 
which might imply that 
large deviation from the spherical shape 
is not favored energetically. 
%
%the spherical shape is favored energetically.
%
As the result, the slope of both parameters 
become small. 

% In the region larger than $R_0$, there appear an effect to shrink the size by the DBI action, which makes size zero, and $b_r$ becomes small. 
% In other words, in both regions, $b_r$ and $b_w$ will probably approach $R_0$. 
% $b_r$ and $b_w$ seem to have the same behavior. 
% The baryon will swell in the $r$ and $w$ direction in almost the same way, although slightly smaller slope than $b_r$ for both energy and topological density. 

% Although monotonical increasing, every slope is smaller than that in the spherical case.
% The gravity and Chern-Simons term have a non-trivial effect to these quantities. 

In the original Yang-Mills theory, 
the (anti)instanton appears as the (anti)self-dual solution.
In holographic QCD, owing to the presence of gravity ($h(w)$ and $k(w)$) and the CS term, 
the self-duality of the solution is explicitly broken. 
% For the single baryon case $B=1$, 
% we introduce the self-duality breaking parameter defined by 
% which is normalized by the topological quantity. 
% A small value means that a configuration is close to self-duality. 

Similarly for the ground-state baryon in Sec.~IV, 
we investigate self-duality breaking for 
various size holographic baryons. 
Figure~\ref{fig:duality_breaking} 
shows the self-duality breaking parameter 
$\Delta_{\rm DB}$ 
in Eq.~(\ref{eq:self-duality breaking parameter}) 
as a function of the size parameter $R$.
This quantity $\Delta_{\rm DB}$ is non-negative and 
becomes zero only in the exact self-dual case. 

% Figure environment removed

% **** change ****
% We found that the region where size parameters are closed to one 
% in the case of spherical shape in Fig.~\ref{fig:shape}, 
% and the minimum of duality breaking parameter is at $R\sim 2$. 
% Self-duality may be related to a four-dimensional spherical shape. 

One finds that the duality breaking parameter is minimized as 
$\Delta_{\rm DB} \simeq 0.17$ at $R\simeq 2.2$, which is close to  
the size $R_0 \simeq 2.4$ of the ground-state baryon. 
This value $\Delta_{\rm DB} \simeq 0.17$ seems to be small, 
and one might expect that the approximation of 
using the self-dual solution is not so bad. 

However, the duality breaking parameter $\Delta_{\rm DB}$ 
never becomes zero in holographic QCD, 
and this fact might have an important physical meaning 
for the baryon-baryon interaction as follows.

In the original four-dimensional Yang-Mills theory, 
the energy is classically bounded by BPS bound, 
and its minimum is achieved only if the configuration has self-duality.
However, holographic QCD has a non-trivial gravity and the CS term, 
and thus they distort the self-duality. 
%
If there were multi instantons satisfying BPS saturation, 
its action would be determined only by the topological charge, 
indicating the ``no interaction" between instantons.
Then, as an interesting possibility, 
the self-duality breaking in holographic QCD might be related to 
the baryon-baryon interaction or the nuclear force \cite{HSS09}. 

\section{Dilatation mode of a single baryon}
\label{sec:dilatation}

In the previous section, 
we investigated size dependence of the static energy $V(R)$ 
of a holographic baryon and showed that 
it seemed to be quadratic against the size $R$. 
Using this result, we numerically investigate 
time-dependent size oscillation modes, i.e., dilatation modes, 
of a single baryon in holographic QCD in this section. 

Since the instanton size $R$ is a key parameter 
to determine the baryon size in holographic QCD,
we describe the size oscillation of the holographic baryon 
by introducing time-dependence of size $R$, 
\begin{eqnarray}
    R \rightarrow R(t) = R_0 + \delta R(t), 
\end{eqnarray}
where $R_0$ denotes the size of the ground-state baryon 
% In this paper, we write $a$ as meaning $\delta a(t)$.
and the size $R(t)$ is expected to oscillate around this $R_0$. 

%To investigate the size oscillation, 
%we introduce a time dependence to size $R$, 
%where $R_0$ denotes the size, which provides %the lowest energy to the baryon mass.
%% In this paper, we write $a$ as meaning %$\delta a(t)$.
%We expect that size $R$ oscillates around this %$R_0$.

Note again that the size of an instanton is originally a moduli, 
and, if there were only the Yang-Mills term in flat space $h(w)=k(w)=1$, 
its value would not affect the energy of holographic QCD 
and the dilatation mode would appear as an exact zero mode. 
%
In reality, the gravity effect ($h(w)$, $k(w)$) 
and CS term break this moduli property 
of the size; however, we expect that 
the dilatation mode is inherited to be a soft mode
of the ground-state baryon  
and appears as a low-lying excitation 
in the single baryon spectrum in holographic QCD. 
%
As a characteristic property, 
the dilatation never changes the flavor 
and rotational properties and hence 
the dilatation excitation has the same quantum number 
as the ground state. 

This dilatation differs from an ordinary $(x,y,z)$-spatial size oscillation, 
because holographic QCD has an extra dimension $w$ 
and the holographic baryon extends also its direction. 
In fact, this extra-dimensional dilatation is peculiar to holographic QCD. 

We consider time-dependent variation of $R(t)$ 
around the ground-state size $R_0$, 
which physically means a dilataion of 
the holographic baryon. 
The Lagrangian of the size variable $R(t)$ is written as 
\begin{eqnarray}
    L[R] &=& \frac{1}{2} m_R \dot{R}^2 -V(R) \cr 
    &\simeq& \frac{1}{2} m_R \dot{R}^2 - \frac{1}{2} m_R \omega^2 (R-R_0)^2 
\end{eqnarray}
up to $\mathcal{O}((R-R_0)^2)$. 
We have already calculated the potential term $V(R)$ 
and have shown it to be almost quadratic
in the previous section. 

We calculate the dilatation mode of the holographic baryon  
as a collective coordinate motion of the size $R(t)$.
%
Note here that, to estimate the frequency $\omega$ of the dilatational mode,  
one only has to calculate the mass parameter $m_R$.
(Of course, $m_R$ is not equal to the baryon mass $M_B$.)
%
Here, we use adiabatic approximation 
that time-dependence of the holographic fields 
is only through the size $R(t)$. 
% 
%For the calculation of the frequency $\omega$, 
%the mass $m_R$ for the dilatation mode is needed. 
% 

%Therefore, we use an assumption or approximation 
%for the mass calculation.

\subsection{Numerical calculation}

In this subsection, 
we consider the baryon dilatation mode 
and investigate the excitation energy 
with keeping the gravity background, $h(w)$ and $k(w)$. 
%
We treat the size oscillation to be adiabatic 
and field motions are relatively faster than the size motion. 
%
Within this adiabatic treatment, 
the time dependence of holographic fields 
$\phi$ and $a_i$ is decided through only 
the baryon size $R(t)$.
Therefore, it is possible to write down field arguments symbolically as
\begin{eqnarray}
    \phi = \phi(r,w;R(t)), \quad
    a_i = a_i(r,w;R(t)) .
\end{eqnarray}
%There is no time-derivative of $\hat{A}_0$ involved %with calculation of kinetic term of size variable $R(t)$. 
%
Note that there is no contribution from $\hat{A}_0$ 
in calculating the kinetic term of size variable $R(t)$, because $\hat{A}_0$ has no time-derivative 
and never accompanies $\dot{R}(t)$. 

% ***** change : 
% The size motion is zero mode for instanton in the flat Yang-Mills theory, 
% and fields motions besides $R$ is higher adiabatic treatment on the size oscillation would be justified 
% and are expected to be higher excitations 
% even with the CS term and gravity. *****
% *******

Based on this adiabatic treatment, 
time-derivative is converted to $R$-derivative as 
\begin{eqnarray}
    % \frac{d}{dt} T[R(t)] &=& \dot{R} \frac{d}{dR} T[R]. \\
    % \frac{d}{dt} \phi(R(t)) &=& \dot{R} \frac{d}{dR} \phi(R(t)). 
    \frac{d}{dt} O[R(t)] = \dot{R} \frac{d}{dR} O[R]. 
\end{eqnarray}
In our framework, 
$O[R]$ is (numerically) calculable for arbitrary $R$, 
and the $R$-derivative is easily obtained numerically,  
\begin{eqnarray}
    \frac{d}{dR} O[R] \simeq \frac{O[R+\delta R] - O[R]}{\delta R}. 
\end{eqnarray}
We have already obtained holographic configurations $\phi(R(t))$ 
with various size $R$, 
and the kinetic term of $R(t)$ is expressed as
\begin{eqnarray}
    S_{\rm kin} & = & 4 \pi \kappa \int dt\ \int_0^\infty dr\ \int_{-\infty}^\infty dw \cr 
    & & \left[ h |\partial_0\phi|^2 + h\frac{r^2}{2}(\partial_0 a_1)^2  + k\frac{r^2}{2}(\partial_0 a_2)^2 \right] \cr
    & = & 4\pi\kappa \int dt\ \int_0^\infty dr\ 
    \int_{-\infty}^\infty dw \cr
    & & \left[ h \dot{R}^2 |\partial_R\phi|^2 + h\frac{r^2}{2}\dot{R}^2(\partial_R a_1)^2 + k\frac{r^2}{2}\dot{R}^2(\partial_R a_2)^2 \right] \cr
    & = & \int dt \ \frac{1}{2} m_R \dot{R}^2 , 
\end{eqnarray}
and mass parameter $m_R$ on the size variation 
is given by
\begin{eqnarray}
    m_R & \equiv & 8\pi\kappa \int^\infty_0 dr \int^\infty_{-\infty} dw \cr & & \left[ h|\partial_R\phi|^2 + h\frac{r^2}{2}(\partial_R a_1)^2 + k\frac{r^2}{2}(\partial_R a_2)^2 \right] . 
    \label{eqn:m_R}
\end{eqnarray}
% This assumption changes time derivatives of fields into functional derivatives of size $R(t)$. 
% Note that we choose this derivatives around the center of oscillation, $R_0$. 
% Since we already have various size configurations in the previous section and can calculate the size-derivatives, 
% the value of mass is calculable. 
Note that CS term also contains time derivative, 
but it is first order and 
no effect from the CS term for the kinetic term of dilatation mode. 
The numerical result is found to be 
\begin{eqnarray}
    m_R \simeq 0.34~M_{\rm KK} \simeq 322~{\rm MeV}. 
\end{eqnarray}
The important point of this numerical calculation is that gravitational factors, $h(w)$ and $k(w)$,  
are exactly included.

With this value of the mass parameter $m_R$ and the quadratic fitting of 
Eq.~(\ref{eqn:potential fitting data})
for the potential $V(R)$, we obtain 
the dilatational excitation energy 
 \begin{eqnarray}
    \omega = \sqrt{\frac{2A}{m_R}} \simeq 0.61 M_\textrm{KK} \simeq 577~\textrm{MeV} 
    \label{eq:dilatational excitation energy}
\end{eqnarray}
for the holographic baryon. 

In terms of $1/N_c$ expansion, 
the mass parameter 
$m_R \propto \kappa$
in Eq.(\ref{eq:dilatational excitation energy}) 
is $O(N_c)$, 
% 
and the potential $V(R)$ of 
the holographic baryon is $O(N_c)$, leading to  
%and thus 
$A = O(N_c)$. 
% 
Then, the dilatation excitation energy $\omega = \sqrt{2A/m_R}$ is $O(N_c^0)=O(1)$ quantity.

In Appendix~\ref{appendix:Rough analytical estimation for dilation modes}, 
we also consider a rough analytical estimation for the dilatation mode, 
when using 't~Hooft instanton, i.e., 
the solution in the case of $h(w)=k(w)=1$ without the CS term. 
This rough estimate gives a larger value of 771~MeV 
for the dilatation excitation energy, 
which seems consistent with 
0.816 $M_{\rm KK} \simeq$ 774 MeV of 
the excitation energy on instanton-size fluctuation  
in the previous research \cite{HSSY07} 
using 't~Hooft instanton. 

%********
%comparison with the previous research \cite{HSSY07}
%excitation energy = 774 MeV (Mkk 948 MeV)
%or 408 MeV (Mkk 500 MeV)
%value is consistent with our estimation 771 MeV in appendix D

\subsection{Discussion}

In the previous subsection, 
we numerically calculate the dilatation excitation energy, 
keeping the gravitational effect of $h(w)$ and $k(w)$. 
Note again that this background gravity is physically important 
because it is inherited from the $N_c$ D4 branes 
to express the original Yang-Mills theory. 
%
% This numerical result is slightly higher 
% than the rough analytical estimation in  Eq.~(\ref{eq:ana_result}), where the gravitational effect 
% has been removed cut as $h(w)=k(w)=1$.
%
% In fact, as the mass $m_R$ increases, 
% size $R$ may become difficult to oscillate 
% due to the gravity.
% Here, the gravitational effects from $h(w)$ and $k(w)$
% are qualitatively important and cannot be ignored.
% 
We have consistently performed a numerical calculation 
for both kinetic and potential terms 
to include the gravitational effect. 
We have eventually obtained 
the dilatation excitation energy of 577~MeV for 
the holographic baryon. 

This dilatation mode appears as an excited baryon with 
the same quantum number as the ground state 
since this rotationally symmetric dilatation 
never changes the quantum numbers on the flavor and rotation. 

Similar to the Skyrme soliton, 
for the description of definite spin/isospin states 
like N and $\Delta$, 
semi-classical quantization by adiabatic rotation 
\cite{HSSY07} is used for the holographic baryon. 
In this quantization process on the spin/isosipn, 
an $O(1/N_c)$ mass correction  
is added to the $O(N_c)$ baryon mass \cite{HSSY07}. 

In terms of $1/N_c$ expansion, 
the dilatation excitation energy is 
$O(N_c^0)$ as mentioned before, 
and thus the dilatation mode is more significant 
than the $O(1/N_c)$ rotational energy, 
which leads to N-$\Delta$ mass splitting. 
%
%among corrections to the holographic baryon. 
%
Furthermore, 
the correction from this rotational effect 
to the dilatation mode 
is higher order of $1/N_c$ expansion 
and then becomes negligible. 

Figure~\ref{fig:baryon splitting} 
shows schematic figure for the baryon mass splitting 
order by order in the $1/N_c$ expansion. 
At the leading order of $O(N_c)$, 
all the holographic baryons degenerate. 
Up tp $O(N_c^0)$, there appears 
the mass splitting of dilatation mode. 
Up to $O(N_c^{-1})$, 
the mass splitting from rotational effect 
is added and leads to N-$\Delta$ mass splitting. 
% The holographic baryon has 
%  and $O(N_c^{-1})$ splitting, 
% corresponding to dilatation and  effect, respectively. 
As the result, 
the order of low-lying baryon mass is to be 
${\rm N} < \Delta < {\rm N}^* < \Delta^*$ 
in term of the $1/N_c$ expansion. 

% the dilatation excitation energy 
% is expected to be almost unchanged. 

% Using the quantization by some adiabatic rotation of baryon, 
% N and $\Delta$ are described 
% with definite spin/isospin state, 
% 
% by adding the term $\frac{S(S+1)}{2I} \sim O(1/N_c)$ 
% to the baryon mass \cite{HSSY07}. 
% The correction from the rotational effect 
% for hedgehog baryons  
% is $S(S+1)/2I ~ O(1/Nc)$. 
% only proportional to $O(1/N_c)$ for the baryon mass 
% $M_B \sim O(N_c)$. 

% Figure environment removed

In holographic QCD, the dilatation excitation seems to appear as the first excitation of the ground-state baryon and has the same quantum number, e.g., positive parity. 
% of each baryon channel (each spin and flavor, i.e. N, $\Delta$, $\Lambda$ ...), 
% and has the same quantum number as the ground-state baryon, i.e., positive parity. 
In addition to the qualitative properties, 
the dilatation excitation energy is estimated as 577 MeV 
in holographic QCD. 
%
From these results on the same quantum number 
and the magnitude of the excitation energy, 
this dilatation mode of the nucleon N(940) 
would be identified as the Roper resonance N$^*$(1440),
which is positive parity and the first excitation of N(940). 
% These results indicate that 
% the Roper resonance N$^*$(1440) \cite{Roper} 
% is the candidate corresponding to 
% this dilatation mode of the nucleon N(940). 
%
For the $\Delta$(1232), 
$\Delta$(1600) might be identified to be 
the dilatational excitation mode. 

As an interesting possibility, 
the similar dilatation mode exists for every baryon 
as a universal phenomenon for a single baryon, 
because any extended stable soliton generally has such a dilatational mode. 
% as is argued in the Derrick theorem. 

The strange baryon mass is slightly larger, 
reflecting the non-zero strange quark mass; 
however, the SU(3)$_f$ flavor symmetry approximately holds. 
%
%the mass difference between u,d and strange quarks. 
%
%for the strange baryon, considering large $N_c$ compared to $N_f$ 
%
Here, we suppose that the mass excess of the strange quark 
is simply added to the holographic baryon, 
like the treatment of SU(3)$_f$ symmetry and its breaking 
in ordinary hadron physics \cite{GOR68}. 
Then, also for strange baryons, 
the dilatation mode would appear, 
and its excitation energy is expected to take 
a similar value as the above result of 577~MeV. 

Table \ref{tab:dilatational modes} presents the candidates of the dilational excitation in various channel of baryons, i.e., 
N, $\Delta$ and $\Lambda$, $\Sigma$, $\Sigma^*$ channel.  
\begin{table}[h]
 \centering
  % \begin{tabular}{lcccc}
  %  \hline
  %  baryon & excited baryon & excitation energy & theory \\
  %  \hline
  %  N(940) & N$^*$(1440) & 500 MeV & $\omega$ \\
  %    & N$^*$(1710) & 770 MeV & 2$\omega$ \\
  %  $\Delta$(1232) & $\Delta$(1600) & 370 MeV & $\omega$ \\
  %   & $\Delta$(1920) & 690 MeV & $2\omega$ \\
  %  $\Lambda$(1115) & $\Lambda$(1600) & 485 MeV & $\omega$ \\
  %  $\Sigma$(1190) & $\Sigma$(1660) & 470 MeV & $\omega$ \\
  %  $\Sigma^*$(1385) & $\Sigma^*$(1840) & 455 MeV & $\omega$ \\
  %  \hline
  % \end{tabular}
  \begin{tabular}{lcccc}
   \hline
   baryon & excited baryon & excitation energy & theory \\
   \hline
   N(940) & N$^*$(1440) & 500 MeV & $\omega$ \\
     & N$^*$(1710) & 770 MeV & 2$\omega$ \\
   $\Delta$(1232) & $\Delta$(1600) & 368 MeV & $\omega$ \\
    & $\Delta$(1920) & 688 MeV & $2\omega$ \\
   $\Lambda$(1116) & $\Lambda$(1600) & 484 MeV & $\omega$ \\
   $\Sigma$(1193) & $\Sigma$(1660) & 467 MeV & $\omega$ \\
   $\Sigma^*$(1385) & $\Sigma^*$(1780) & 395 MeV & $\omega$ \\
   \hline
  \end{tabular}
  \caption{
  Experimental candidates of the dilatational excitation mode in various baryon channel \cite{PDG}. 
  Each ground-state baryon has the excitation with the same quantum number. 
  Here, first-excited baryons are mainly listed, 
  such as N$^*$(1440) and $\Delta$(1600). 
  For each channel, the excitation energy seems 
  to take a consistent value with the theoretical one, 
  $\omega \simeq 577~{\rm MeV}$. Here, 
  N$^*(1710)$ and $\Delta(1920)$ are identified to the second excitation mode. 
  }
  \label{tab:dilatational modes}
\end{table}
%
%this result leads that every baryon with rotational symmetry 
%and the same quantum number which is equal to the ground state %have the same dilatation mode.

For each channel with the same quantum number, 
we find that the first excitation energy 
seems to be a consistent similar value 
to the dilatational one, 
$\omega \simeq 577~{\rm MeV}$, 
theoretically obtained above. 

% There are other baryons that are considered as the dilatation mode; 
% however, their spin and parity are not confirmed.
% Research and Confirm it experimentally!

% ******* Here and Summary ******* 
% How about dilatation modes in other channels? 
% Ξ(1320),Ξ*(1530),Ω(1673),
% Λc(2285), Σc(2455), Σc*(2520) 
% ~大雑把にはありそうだが、spin や parity が不明なので同定は困難

Also for multi-strange baryons 
in $\Xi$, $\Xi^*$ and $\Omega$ channel, 
we theoretically predict the dilatation modes 
with the excitation energy of about 577~MeV. 
%although it is difficult 
%to experimentally confirm 
%the quantum number of 
%those excited multi-strange baryon. 
%
However, 
the dilatational excited baryons for 
$\Xi(1320)$, $\Xi^*(1530)$, 
and $\Omega(1673)$ are not yet observed experimentally 
because spin-parity information 
is not yet confirmed for their excited baryons. 
In this respect, further experimental 
analyses are much desired for excited baryons in terms of 
the dilatational mode in each baryon channel. 

%In fact, experimentally, their quantum number, 
%i.e. spin-parity information, has not been %confirmed yet, 
%and its experimental further research 
%is desired. 

% These excitations may have the same quantum number and  
% reasonable excitation energy values as our calculation.
% If our expectation is true, every baryon has such 
% excitation in that energy region, and it seems not to fault.

% In this estimation, we consider the curved space effects for the calculation of both kinetic and potential terms.
% This curved space is unique and important for holographic QCD at the baryon's finite size with the CS term. 
% Therefore, we should consider this for a more strict estimation. 

% ***Go to Discussion for details ***
% dilatation appears as the same QN. 
% Since all the single baryons have finite sizes, 
% this motion may be seen universally 
% among the same quantum number ones. 

\section{Summary and Concluding Remarks}

We have investigated a baryon and its dilatation modes 
in holographic QCD based on the Sakai-Sugimoto model, 
which is constructed with $N_c$ D4 branes and $N_f$ D8/$\bar{\rm D8}$ branes in the superstring theory. 
This theory is expressed as 
a 1+4 dimensional U($N_f$) gauge theory 
in the flavor space.

We have adopted a generalized version of the Witten Ansatz for spatially rotational symmetric systems and have reduced 
1+4 dimensional holographic QCD into a 1+2 dimensional Abelian Higgs theory in a curved space. 
In this formulation, a four-dimensional instanton 
corresponding to a baryon is converted to 
a two-dimensional Abrikosov vortex. 
We have numerically calculated the baryon solution of 
holographic QCD using a fine and large-size lattice 
with spacing of 0.04~fm and size of 10~fm.

Using the relation between the baryon size and 
the zero-point location of the Higgs field in the Witten Ansatz,  
we have theoretically changed the size of holographic baryons  
and have investigated its properties, such as the energy and 
self-duality breaking, as function of the size parameter.
Here, each configuration is a solution of EOM of holographic QCD 
under the constraint of fixing Higgs zero-point. 

As time-dependent size-oscillation modes, 
we have investigated the dilatation modes of a baryon 
and have found that 
such a dilatational mode takes 
the excitation energy of 577~MeV. 
Since the dilatation does not change 
the quantum number including the parity, 
we have identified this dilatation mode for the nucleon N(940) 
as the Roper resonance N$^*$(1440).
%
We have conjectured that any baryons are expected 
to have such a dilatational excitation universally, 
and their excitation energy would be similar.

In this respect, further experimental analyses are much 
desired for excited baryons in terms of 
the dilatational mode in each baryon channel. 
In particular, the dilatational excited baryons for 
$\Xi(1320)$,  $\Xi^*(1530)$, and  $\Omega(1673)$ have not yet been observed  experimentally 
because spin-parity information is not yet confirmed for their excited baryons. 

% In this study, we have used adiabatic approximation in estimating of the kinetic term. However, further analysis is desired for  more quantitative evaluation of the dilatation property. 

As a caution, the calculated values presented in this paper 
are to be regarded as semi-quantitative estimates 
in an idealized case of large $N_c$ in the chiral limit, 
and they have some deviation from experimental values in the real world. 
In the following, we mention some quantitative limitations 
and cautions on the approach used in this study. 

This framework of holographic QCD has infrared equivalence 
with massless QCD and gives a useful analytical 
nonperturbative method to analyze QCD, 
which is a main reason to adopt holographic QCD in this study for baryons. 
On the quantitative accuracy, however, 
this framework includes some limitations. 
As an important caution, 
the present calculation is based on the 
$1/N_c$ and $1/\lambda$ expansion, 
and the starting holographic action is up to 
the $1/N_c$-leading and $1/\lambda$ sub-leading order. 
Therefore, for more accurate estimation, 
it is desirable to check the contribution from 
$1/N_c$ or $1/\lambda$ higher order terms. 
However, it is extremely difficult to extract 
the next order of $1/N_c$ and $1/\lambda$ expansions 
in holographic QCD. 

As a higher-order correction, 
there is an $O(1/N_c)$ rotational effect of 
the hedgehog configuration, 
which is used for semi-classical analysis 
of the Skyrmion investigation \cite{ANW83}. 
It is relatively $1/N_c^2$ smaller, 
compared with leading order $O(N_c)$ of the baryon mass, 
although it is necessary for the $N-\Delta$ splitting. 
%
Similarly, for the dilatational excitation, 
this correction appears as relatively $1/N_c^2$-smaller order, 
and therefore we have ignored this higher order in this paper. 
However, this higher-order correction might be desired 
to reproduce real experimental data. 

In addition, we have assumed spatially spherical shape 
of a baryon to apply the Witten Ansatz 
and adiabatic treatment for the dilatation dynamics, 
For more precise analysis of baryons, 
more sophisticated treatments might be desired. 
However, to go beyond the spherical symmetric solution, 
the Witten Ansatz is no more applicable, 
and then one has to deal with the four-dimensional analysis even for static baryons. 
To go beyond adiabatic approach 
is also a difficult problem widely appeared 
in theoretical physics, and 
one has to handle complicated local oscillation 
of all the holographic fields in the present case. 

We have used holographic QCD based on $N_f = 2$ 
in the chiral limit. 
Further extension including strangeness is 
an interesting subject in hadron physics, 
which can be done with $N_f=3$ holographic QCD \cite{MS17}. 
%
In the real world, u and d-quarks have 
small finite current mass of 2-5~MeV 
and s-quarks have current mass of about 93~MeV \cite{PDG}, 
and the non-zero quark mass explicitly breaks the chiral symmetry. 
%
In holographic QCD, however, it is difficult to introduce 
the finite quark mass or explicit chiral-symmetry breaking, 
and further theoretical development is required 
for quantitative argument of hyperons. 

We have mainly presented 
the excitation energy for the dilatational mode of baryons,
which appears in the same quantum numbers. 
%
%It is desired to show a theoretical way to 
%single out the dilatational mode 
%experimentally, 
%among many various excitation of baryons 
%
It is desired for theoretical progress 
to show how to distinguish 
the dilatational mode from other excitation 
experimentally. 
%in more direct manner.
%
To this end, we have to find out 
unique behavior for the dilatational mode, 
which is a future subject. 
The finding of such a peculiar quantity will 
lead to a better understanding of baryon spectra. 

\begin{acknowledgements}
H.S. is supported in part by the Grants-in-Aid for
Scientific Research [19K03869] from Japan Society 
for the Promotion of Science.
\end{acknowledgements}

\newpage

\appendix

\section{Lattice formalism of holographic QCD}

In Appendix~A, 
we show our lattice formalism for holographic QCD 
in the Witten Ansatz, 
by introducing a sizable finite lattice with a small spacing $a$. 

Let us begin with the static energy $E_{\rm 5YM}$ 
of the Yang-Mills term in the Witten Ansatz. 
For the explanation, we here repeat Eq.~(\ref{eqn:E_5YM}),  
\begin{eqnarray}
E_{\rm 5YM} &=& 4\pi\kappa \int^{\infty}_{0} dr \int^{\infty}_{-\infty} dw \biggl[ h(w) |D_1\phi|^2 + k(w) |D_2\phi|^2  \cr 
& & + \frac{h(w)}{2r^2}\{1-|\phi|^2\}^2 + \frac{r^2}{2} k(w) f_{12}^2 \biggr], 
\label{eq:E_5YM_appendix}
\end{eqnarray}
of which the field variables $\phi$ and $a_\mu$ 
depend on the two-dimensional spatial coordinate $(r,w)$.

%originating from the DBI action, 
%which is the leading order of the $1/N_c$ and $1/\lambda$ expansion. 
%
%In the Witten Ansatz, this theory reduces to 
%an Abelian Higgs theory with gravity,  
%and the static energy is expressed as 

For the U(1) gauge variable $a_\mu$ with $\mu=r,w$, 
we define the link variable 
\begin{eqnarray}
U_\mu(s) \equiv \exp\{ia~a_\mu(s+\frac{\hat{\mu}}{2})\} \in {\rm U(1)}
\end{eqnarray}
at the site $s=(s_r,s_w)$ on the two-dimensional lattice.  
Here, $\hat{\mu}$ denotes the $\mu$-directed vector 
with the length of $a$.

On the lattice with a small spacing $a$, 
one finds for the U(1) covariant derivative $D_\mu\phi$ as 
\begin{eqnarray}
& & -\{ \phi^\dagger(s)U_\mu(s)\phi(s+\hat{\mu}) + \phi^\dagger(s+\hat{\mu})U_\mu(s)^\dagger\phi(s) \} \cr
& & + |\phi(s)|^2 + |\phi(s+\hat{\mu})|^2 \cr
& = & - ( \phi-\frac{a}{2}\partial_\mu\phi )^\dagger (1+iaa_\mu-\frac{1}{2}a^2a_\mu^2)(\phi+\frac{a}{2}\partial_\mu\phi)\cr
& & - ( \phi+\frac{a}{2}\partial_\mu\phi )^\dagger (1-iaa_\mu-\frac{1}{2}l^2a_\mu^2)(\phi-\frac{a}{2}\partial_\mu\phi) \cr
& & + 2|\phi|^2 + \frac{a^2}{2}|\partial_\mu\phi|^2 + \mathcal{O}(a^3) \cr
& = & a^2\phi^\dagger(\stackrel{\leftarrow}{\partial_\mu}-ia_\mu)(\partial_\mu+ia_\mu)\phi + \mathcal{O}(a^3) \cr 
& = & a^2 |D_\mu\phi|^2 + \mathcal{O}(a^3),
\end{eqnarray}
where all omitted arguments are $s+\hat{\mu}/2$. 
%There are link variables 
%$U_\mu(s) = \exp(ia~a_\mu(s+\hat{\mu}/2))$, 
%where $s$ is site index and $a$ is a lattice spacing. 
%They represent lattice bonds with its two endpoints 
%and have both directions. 
%
The field strength $f_{12}$ is expressed with 
the U(1) plaquette variable,  
\begin{eqnarray}
    \square_{12}(s) &\equiv& U_1(s)U_2(s+\hat{1})U_1^*(s+\hat{2})U_2^*(s) \in {\rm U(1)}~~~~ \\
    f_{12}^2(s) &=& \frac{1}{a^2}[1 - {\rm Re} \{\square_{12}(s) \}].
\end{eqnarray}

%In holographic QCD, SU(2) and U(1) parts are decoupled 
%in the leading order $\mathcal{O}(N_c)$. 
%However, in the next leading order, there is the CS term, 
%and these parts are coupled. 

The additional U(1) energy $E^{\rm U(1)}$ 
in Eq.~(\ref{eq:E^U(1)rw}) is expressed by 
the coupling of the original U(1) gauge field $\hat A_0$ 
and the topological density $\rho_B$. 
%
Here, the temporal component $\hat A_0(r,w)$ 
is treated as a (spatially) site-variable 
on the $(r,w)$ lattice.
%
On the lattice, 
the baryon density is expressed as 
\begin{eqnarray}
\rho_B  & = & \frac{1}{8\pi^2r^2} [-i\epsilon_{ij}(D_i\phi)^*D_j\phi + \epsilon_{ij}\partial_ia_j(1-|\phi|^2) ] \cr
& = & \frac{1}{8\pi^2r^2} [ 2 {\rm Im} \{(D_1\phi)^*D_2\phi\} + f_{12}(1-|\phi|^2) ]~~~
\end{eqnarray}
with 
\begin{eqnarray}
    f_{12} (1-|\phi|^2) &=& \frac{1}{a^2} {\rm Im} \ \square_{12} \times (1-|\phi|^2)
\end{eqnarray}
and
\begin{eqnarray}
&& (D_1\phi)^*D_2\phi \cr
    &=& \frac{1}{4a^2} \{ U_1^*(s)\phi^*(s+\hat{1}) - U_1(s-\hat{1})\phi^*(s-\hat{1}) \} \cr
    & & \times \{ U_2(s)\phi(s+\hat{2}) - U_2^*(s-\hat{2})\phi(s-\hat{2}) \} \cr 
    &=& \frac{1}{4a^2} \left\{ \phi^*(s+\hat{1})U_1^*(s)U_2(s)\phi(s+\hat{2}) \right. \cr
    & & \left. - \phi^*(s-\hat{1})U_1(s-\hat{1})U_2(s)\phi(s+\hat{2}) \right. \cr
    & & \left. - \phi^*(s+\hat{1})U_1^*(s)U_2^*(s-\hat{2})\phi(s-\hat{2}) \right. \cr
    & & \left. + \phi^*(s-\hat{1})U_1(s-\hat{1})U_2^*(s-\hat{2})\phi(s-\hat{2}) \right\}. \cr
    & & 
\end{eqnarray}
To reduce the discretization error, 
we have used the above form for $(D_1\phi)^*D_2\phi$, 
and its lattice formalism is symbolically written as
% \begin{equation}
% 4a^2 {\rm Im}\left[ (D_1\phi)^*D_2\phi \right] =
% \vcenter{\hbox{% Figure removed}},
% \end{equation}
\begin{equation}
4a^2 (D_1\phi)^*D_2\phi = 
\vcenter{\hbox{% Figure removed}},
\end{equation}

\ 

\noindent
where the horizontal an vertical arrows represent 
$U_1$ and $U_2$, respectively, 
and the dots represent $\phi$. 
%
Thus, we define the topological density $\rho_B(r,w)$ 
and formulate the U(1) energy 
$E^{\rm U(1)}[\rho_B(r,w), \hat A_0(r,w)]$
in Eq.~(\ref{eq:E^U(1)rw}) on the $(r,w)$ lattice.

In this way, the total energy $E$ in holographic QCD 
is expressed as $E[\phi(s), \vec U(s), \hat A_0(s)]$, i.e., 
a function of $\phi(s)=\phi_1(s)+i\phi_2(s) \in {\bf C}$, $\vec U(s)\equiv (U_1(s), U_2(s))$ and $\hat A_0(s)$ at the spatial site $s=(s_r,s_w)$.
%
To find the solution of holographic QCD, 
we minimize the total energy $E$ 
by iterative improvement on 
$\phi(s)$, $\vec U(s)$ and $\hat A_0(s)$.
%
Looking at a specific site $s_0$, 
we consider only one variable $\phi(s_0)$,  
with fixing all other variables.
By taking variation of $\phi(s_0)$, 
we minimize the total energy $E$. 
Next, we consider only one link-variable $\vec U(s_0)$,  
with fixing all other variables, 
and take its variation to minimize $E$. 
Similarly, considering only one site-variable $\hat{A}_0(s_0)$,  
with fixing all other variables, 
we take its variation to minimize $E$. 
%
On each site on the lattice, we repeat the above process  
and update $\phi(s)$, $\vec U(s)$ and $\hat A_0(s)$.
% in order, which is called a sweep.
%
We iterate this sweep procedure many times 
so as to minimize the total energy $E$ in holographic QCD, 
and the solution is eventually obtained.


\section{Other expression of U(1)-part energy}

For the U(1) sector, we have mainly used 
Eq.~(\ref{eq:E^U(1)rw}) for 
the numerical calculation of $E^{\rm U(1)}$. 
%
However, there is another useful 
expression for the energy $E^{\rm U(1)}$ 
of the U(1) sector without $\hat A_0$,  
and we introduce this form in Appendix B.

By solving $\hat A_0$ 
using Eq.~(\ref{eq:U(1)FE}),
the energy (\ref{eq:E^U(1)}) becomes 
\begin{eqnarray}
E^{\rm U(1)} &=&\frac{N_c^2}{8}\int d^3 x dw~ \rho_B K^{-1} \rho_B
\nonumber \\
&=&\frac{N_c^2}{8}\int d^3 x dw \int d^3 x' dw' \cr
&&\rho_B(\vec x,w) K^{-1}(\vec x,w; \vec x',w') \rho_B(\vec x',w').
\end{eqnarray}
%apart from the leading energy term $E_{\rm 5YM}^{{\rm SU(2)}_f}$.

Since the kernel $K$ and topological density $\rho_B$ are 
SO(3) rotationally symmetric, the additional energy can be  expressed 
only with the $(r,w)$-coordinates:
\begin{eqnarray}
E^{\rm U(1)} &=& 2\pi^2 N_c^2 \int_0^\infty dr \int_{-\infty}^\infty dw 
\int_0^\infty dr' \int_{-\infty}^\infty dw'
\nonumber  \\ 
&& \tilde \rho_B(r,w) \tilde K^{-1}(r,w; r',w') \tilde \rho_B(r',w'),
\label{eqn:U(1) energy}
\end{eqnarray}
using $\tilde \rho_B(r,w) \equiv r^{2}\rho_B(r,w)$ and 
the hermite kernel $\tilde K \equiv 4\pi r^2 K$ in $(r,w)$-space. 
%\begin{eqnarray}
%\tilde K \equiv 4\pi r^2 K
%=-4\pi \kappa \{h(w) \partial_r r^2 %\partial_r + r^2 \partial_wk(w) \partial_w\}.~~~~ 
%\end{eqnarray}
% As same as SU(2) sector, U(1) part is described by applying Witten Ansatz,
% \begin{eqnarray}
%     \tilde{A}_0 = 0 \cr 
%     \tilde{A}_i = \tilde{a}_i \\
%     \tilde{A}_w = \tilde{a}_w \notag{} .
% \end{eqnarray}

On the lattice, the kernel $\tilde{K}$ in Eq.~(\ref{eq:kernel_rw})
is transformed into a differential form and is expressed as 
a matrix $\tilde{K}_L(r, w; r', w')$. 
Then, the kernel inverse $\tilde{K}_L^{-1}(r, w; r', w')$ is numerically obtained 
by taking the inverse matrix of the kernel $\tilde{K}_L (r, w; r', w')$. 
As a technical caution, the kernel $\tilde{K}_L (r, w; r', w')$ 
has a translational zero mode, reflecting the derivative form of $\tilde{K}$. 
%
Since this translational zero mode does not affect the total energy, 
the inverse of  $\tilde{K}$ has to be taken 
in the space except the spurious zero mode.
% 
In fact, using an orthogonal matrix $O$, the kernel $\tilde{K}_L$ is diagonalized as 
\begin{eqnarray}
    \tilde{K}_L = O \ {\rm diag}(0,\lambda_1,\lambda_2,\cdots) \ O^T
\end{eqnarray}
with non-zero eigenvalues $\lambda_n$. 
%
Then, we define its inverse $\tilde{K}^{-1}_L$ to be 
\begin{eqnarray}
    \tilde{K}_L^{-1} = O \ {\rm diag}(0,\lambda_1^{-1},\lambda_2^{-1},\cdots) \ O^T, 
\end{eqnarray}
which is equivalent to the appropriate removal of 
the spurious translational zero mode.
%
Using this kernel inverse $\tilde{K}_L^{-1}$ and the baryon density $\rho_B$, 
the energy $E^{\rm U(1)}$ of the CS term is calculated as 
$\rho_B \tilde{K}_L^{-1} \rho_B$ on the lattice. 

Thus, for the numerical calculation 
of $E_{\rm U(1)}$, there are two different methods:
one is to update the holographic fields 
$\phi(r,w)$, $\vec a(r,w)$ and $\hat A_0(r,w)$ 
on the lattice based on Eq.~(\ref{eq:E^U(1)rw});
the other is to update only 
$\phi(r,w)$ and $\vec a(r,w)$ using  
Eq.~(\ref{eqn:U(1) energy}).
%
We have confirmed that both methods give 
the same numerical results for holographic baryons.

\section{Holographic baryon using 
self-dual BPS instanton}

In Appendix~C, 
we investigate the holographic baryon  
when the self-dual BPS instanton 
in Eq.~(\ref{eqn:'t hooft soltion HQCD})
is used.
The self-dual BPS instanton is 
the 't~Hooft solution of the ordinary Yang-Mills theory,  
and hence, to be strict, this usage 
is justified in the case of 
flat space $h(w)=k(w)=1$ and ignoring the CS term. 
%
Substituting the BPS instanton with the size $R$ 
into the total energy $E$ in Eq.~(\ref{eqn:total energy})
including the CS term and the background gravity, 
$k(w)=1+w^2$ and $h(w)=k(w)^{-1/3}$, 
one obtains 
the static baryon energy $E(R)=V^{\rm BPS}(R)$ 
as the function of the instanton size $R$,  
and its minimum gives 
an approximate ground-state holographic baryon, 
which satisfies 
$M_B^{\rm BPS} \simeq 1.35~M_{\rm KK} \simeq 1.28~{\rm GeV}$ 
and  
$\sqrt{\langle r^2\rangle_{\rho_B}^{\rm BPS}} 
=\sqrt{\frac{3}{2}} R_0^{\rm BPS} \simeq 2.2~M_{\rm KK}^{-1}\simeq 0.46~{\rm fm}$, 
as shown in Fig.~\ref{fig:potential BPS}.
%
Thus, when the self-dual BPS instanton is used, 
the holographic ground-state baryon has 
larger mass and smaller size \cite{HSSY07} 
than the true solution numerically obtained in Sec.~\ref{sec:vortex baryon}.



For the approximate ground-state holographic baryon, 
the corresponding Higgs field $\phi$ and gauge field $\vec{a}$ are shown in Fig.~\ref{fig:phi a no iteration}. 
These fields give a topological density $\rho_B$, and 
$\hat{A}_0$ is obtained by solving EOM~(\ref{eq:U(1)FE}), 
as shown in Fig.~\ref{fig:aU1 no iteration}.
% Figure environment removed
%
% Figure environment removed
%
Figure~\ref{fig:e t no iteration} shows 
the $r^2$-multiplied topological and the energy densities.
% 
% Figure environment removed

The static baryon energy $V^{\rm BPS}(R)$ 
for the BPS configuration is shown in Fig.~\ref{fig:potential BPS}, and 
it is approximately fit with a quadratic function,  
\begin{align}
    &&    V^{\rm BPS}(R) \simeq A^{\rm BPS} ( R - R_0^{\rm BPS} )^2 + M^{\rm BPS}, \cr 
    && \label{eqn:BPS potential data} \\
    &&    A^{\rm BPS} \simeq 0.39,~ R_0^{\rm BPS} \simeq 1.8,~ M_0^{\rm BPS} \simeq 1.4, \notag{}  
    % \label{eqn:BPS potential data}
\end{align}
in the $M_{\rm KK}=1$ unit.

% Figure environment removed

Thus, when the self-dual BPS instanton is used, 
the holographic baryon has 
larger mass and smaller size \cite{HSSY07} 
than the true solution numerically obtained in Sec.~\ref{sec:vortex baryon}.

\section{Rough analytical estimation for dilation modes
using self-daul BPS instanton}
\label{appendix:Rough analytical estimation for dilation modes}

In Appendix~D, we consider 
a rough analytical estimation of the dilation mode
using the self-dual 't~Hooft BPS instanton,  
which is justified in the flat space $h(w)=k(w)=1$ 
and without the CS term. 

In this case, 
when the flat space approximation $h(w)=k(w)=1$ is used, 
the kinetic term $T$ of the size variable $R(t)$ is analytically expressed as 
\begin{eqnarray}
	T & \simeq & \kappa \int d^3x\ dw \textrm{tr}\left[ F_{0M}^2 \right] \notag{} \\
	& \simeq & \kappa \int d^3x\ dw \frac{24x^2}{(x^2+R_0^2)^4} R_0^2 \dot{R}^2 \notag{} \\
	& = & 48\pi^2\kappa \int dr \frac{r^5}{(r^2+R_0^2)^4} R_0^2 \dot{R}^2 \notag{} \\
	& = & 8\pi^2\kappa \ \dot{R}^2 = \frac12 m_R \dot{R}^2, 
\end{eqnarray}
where $\dot{R}$ means $\partial_t R$. 
Here, $h(w)=k(w)=1$ is used in the first line, 
and the instanton size $R_0$ appearing in the middle 
does not affect the result. 
%
In this way, the mass parameter $m_R$ 
on the size variation is estimated as 
\begin{eqnarray}
 m_R & = & 16\pi^2 \kappa \simeq 1.18. 
\end{eqnarray}
 
% cf. $m_R \simeq 0.34$
% -numerical 
% -correct treatment on 
% gravity effects (h,k) and CS term
% -adiabatic 
% $A^{BPS}$ = 0.39
% $A^{SOL}$ = 0.063
% **************

Form this mass parameter $m_R$ and 
the potential $V^{\rm BPS}(R)$ 
in Eq.~(\ref{eqn:BPS potential data}),  
one obtains a rough estimation of 
the excitation energy of the dilatation mode, 
\begin{eqnarray}
    \omega = \sqrt{\frac{2A^{\rm BPS}}{m_R}} = 0.81\ M_\textrm{KK} \simeq 771\ \textrm{MeV},
    \label{eq:ana_result}
\end{eqnarray}
which seems a larger value than the numerical result 
in Sect.~\ref{sec:dilatation}. 
This estimation can be analytically done 
but has no gravitational effect of $h(w)$ and $k(w)$ 
for both kinetic and potential terms,   
in addition to use of the self-dual BPS instanton. 
On these points, 
the numerical calculation presented 
in Sect.~\ref{sec:dilatation} has been developed. 

%Therefore, we need a numerical calculation for mass which %does not spoil these gravitational effects.
%In the next subsection, we introduce an assumption to %enable this calculation and do more exactly.



%In updating $\phi(s)$, 
%we construct energy part $E[\phi(s)]$ that depends on $\phi(s)$, 
%extracting from the total energy and taking variation
%$\delta\phi(s)$ 
%minimizing the energy $E[\phi(s)]$, $U_\mu(s)$ part in parallel. 
%Iterating this process many times at all sites and the solution 
%is eventually obtained.


\begin{thebibliography}{0} %for 1 digit

\bibitem{M98}
J.M. Maldacena, 
The large N limit of superconformal field theories and supergravity, 
Adv. Theor. Math. Phys. {\bf 2}, 231-252 (1998).

\bibitem{P95}
J. Polchinski, Dirichlet branes and Ramond-Ramond charges,  
Phys.~Rev.~Lett. {\bf 75}, 4724-4727 (1995); 
``String Theory'',  
(Cambridge Monographs on Mathematical Physics, 1998). 

\bibitem{W98}
E. Witten, 
Anti-de Sitter space and holography,  
Adv. Theor. Math. Phys. {\bf 2}, 253-291 (1998); 
Anti-de Sitter space, thermal phase transition, 
and confinement in gauge theories, 
Adv. Theor. Math. Phys. {\bf 2}, 505-532 (1998). 

\bibitem{KMMW04}
M. Kruczenski, D. Mateos, R. C. Myers, and D. J.Winters,
Towards a holographic dual of large-$N_c$ QCD, 
J. High Energy Phys. {\bf 05}, 041 (2004).

\bibitem{SS05}
T. Sakai and S. Sugimoto, 
Low energy hadron physics in holographic QCD, 
Prog. Theor. Phys. {\bf 113}, 843-882 (2005); 
More on a holographic dual of QCD, 
Prog. Theor. Phys. {\bf 114}, 1083-1118 (2005). 

\bibitem{Nambu}
Y. Nambu, Duality and hadrodynamics, Notes prepared for the Copenhagen High Energy Symposium (1970).

\bibitem{Goto}
T. Goto, Relativistic quantum mechanics of one-dimensional mechanical continuum and subsidiary condition of dual resonance model, 
Prog. Theor. Phys. {\bf 46}, 1560 (1971). 

\bibitem{Polyakov81}
A. M. Polyakov, Quantum geometry of bosoniv strings, 
Phys. Lett. {\bf B103}, 207 (1981).

\bibitem{Green-Schwarz-Witten}
M. B. Green, J. Schwarz, E. Witten, 
Superstring Theory, 
(Cambridge University Press, 1987) 
and its references.

% \bibitem{BEBGK}
% J. Babington, J. Erdmenger, N. Evans, Z. Guralnik, and I. Kirsch, 
% Chiral symmetry breaking and pions in non-supersymmetric gauge/gravity duals, 
% Phys. Rev. {\bf D69}, 066007 (2004).

% \bibitem{KMMW}
% M. Kruczenski, D. Mateos, R. Myers and D. Winters, 
% Toward a holographic dual of large-$N_C$ QCD, 
% , J. High Energy Phys. 05 (2004) 041.

%\bibitem{MO77}
%C. Montonen and D. Olive, 
%Magnetic monopoles as gauge particles?, 
%Phys. Lett. B72 117 (1977). 

%\bibitem{GNO77}
%P. Goddard, J. Nuyts, and D. Olive, 
%Gauge theories and magnetic charge, 
%Nucl. Phys. {\bf B125} 1 (1977). 

\bibitem{Karch-Katz}
A. Karch and E. Katz, 
Adding flavor to AdS/CFT, 
J. High Energy Phys. {\bf 06} (2002) 043.

\bibitem{EKSS}
J. Erlich, E. Katz, D. Son, and M. Stephanov, 
QCD and a Holographic Model of Hadrons, 
Phys. Rev. Lett. {\bf 95}, 261602 (2005). 

\bibitem{W79} 
E. Witten E, 
Baryons in the 1/N expansion,  
Nucl. Phys. {\bf B160}, 57-115 (1979). 

\bibitem{S61} 
T.H.R. Skyrme, A nonlinear field theory,  
Proc. Roy. Soc. {\bf A260}, 127-138 (1961);  
A unified field theory of mesons and baryons, 
Nucl. Phys. {\bf 31} 556-569 (1962). 

\bibitem{ZB96}
I. Zahed and G.E. Brown, 
The Skyrme model, 
Phys. Rept. {\bf 142}, 1 (1996). 

\bibitem{GZ64}
M. Gell-Mann, Phys. Lett. {\bf 8}, 214 (1964); 
G. Zweig, CERN-TH-401, CERN-TH-412 (1964).

\bibitem{NSK07}
K. Nawa, H. Suganuma and T. Kojo, 
Baryons in holographic QCD,  
%arXiv:hep-th/0612187,
Phys. Rev. {\bf D75}, 086003 (2007). 

\bibitem{NSK09}
K. Nawa, H. Suganuma and T. Kojo, 
Brane-induced Skyrmion on $S^3$: Baryonic matter in holographic QCD
Phys. Rev. {\bf D79}, 026005 (2009). 

\bibitem{W98b}
E. Witten, Baryons and branes in anti-de Sitter space, 
JHEP {\bf 07}, 006 (1998). 

\bibitem{HSSY07}
H. Hata, T. Sakai, S. Sugimoto and S. Yamato, 
Baryons from instantons in holographic QCD, 
Prog. Theor. Phys. {\bf 117}, 1157-1180 (2007). 

\bibitem{HSS08}
K. Hashimoto, T. Sakai and S. Sugimoto, 
Holographic baryons: Static properties and form factors 
from gauge/string duality,  
Prog. Theor. Phys. {\bf 120}, 1093-1137 (2008). 

\bibitem{HRYY07} 
D.K. Hong, M. Rho, H.U. Yee and P. Yi, 
Chiral dynamics of baryons from string theory, 
Phys. Rev. {\bf D76}, 061901 (2007); 
Dynamics of baryons from string theory and vector dominance,  
JHEP {\bf 09}, 063 (2007).

\bibitem{CI12}
A. Cherman and T. Ishii, 
Long-distance properties of baryons in the Sakai-Sugimoto model,  
Phys. Rev. {\bf D86}, 045011 (2012).

\bibitem{BS14} 
S. Bolognesi and P. Sutcliffe, The Sakai-Sugimoto soliton,  
JHEP {\bf 01}, 078 (2014).

\bibitem{RSR14} 
M. Rozali, J.B. Stang and M. van Raamsdonk,
Holographic baryons from oblate instantons,  
JHEP {\bf 02}, 044 (2014).

\bibitem{Roper}
L. D. Roper, 
Evidence for a $P_{11}$ Pion-Nucleon Resonance at 556 MeV, 
Phys. Rev. Lett. {\bf 12}, 340 (1964). 

\bibitem{HS84}
C. Hajduk and B. Schwesinger, 
The breathing mode of nucleons and ${\rm \delta}$-isobars in the Skyrme model, 
Phys. Lett. {\bf B140}, 172-174 (1984). 
\bibitem{HH84}
A. Hayashi and G. Holzwarth, 
Excited nucleon states in the Skyrme model, 
Phys. Lett. {\bf B140}, 175-180 (1984). 
\bibitem{ZMK84}
I. Zahed, U.-G. Meissner and U.B. Kaulfuss, 
Low-lying resonances in the Skyrme model using the semi-classical approximation, 
Nucl. Phys. {\bf A426} 524-541 (1984). 
\bibitem{WE84}
H. Walliser and G. Eckart, 
Baryon resonance as fluctuations of the skyrme soliton, 
Nucl. Phys. {\bf A429}, 514-526 (1984). 
\bibitem{LZB84}
K. F. Liu, J. S. Zhang, and G. R. E. Black, 
Time dependence of the Skyrme soliton, 
Phys. Rev. {\bf D30}, 2015 (1984). 
\bibitem{BN84}
J. Breit and C. R. Nappi, 
Phase Shifts of the Skyrmion Breathing Mode, 
Phys. Rev. Lett. 53, 889 (1984). 

\bibitem{MCDDHLLZ}
N. Mathur, Y. Chen, S.J. Dong, T. Draper, I. Horv´ath, F.X. Lee, K.F. Liu, and J.B. Zhang, 
Roper Resonance and $S_{11}$(1535) from Lattice QCD, 
Phys. Lett. {\bf B605}, 137 (2005). 
\bibitem{LHKLMWZ}
B. G. Lasscock, J. N. Hedditch, W. Kamleh, D. B. Leinweber, W. Melnitchouk, A. G. Williams, J. M. Zanotti, 
Even parity excitations of the nucleon in lattice QCD, 
Phys. Rev. {\bf D76}, 054510 (2007). 

\bibitem{BPST}
A.A. Belavin, A.M. Polyakov, A.S. Schwartz 
and Yu.S. Tyupkin, 
Pseudoparticle solutions of the Yang-Mills equations
Phys. Lett. {\bf B59}, 85-87 (1975).

\bibitem{tH76}
G.'t~Hooft, 
Computation of the quantum effects due to a four-dimensional pseudoparticle, 
Phys. Rev.{\bf D14}, 3432 (1976).

\bibitem{W77}
E. Witten, Some exact multi-instanton solutions of 
classical Yang-Mills theory, 
Phys. Rev. Lett. {\bf 38}, 121-124 (1977). 

\bibitem{SH20}
H. Suganuma and K. Hori, 
Topological objects in holographic QCD,
Phys. Scr. {\bf 95}, 074014 (2020). 

\bibitem{HSS09}
K.~Hashimoto, T.~Sakai and S.~Sugimoto, 
Nuclear Force from String Theory, 
Prog. Theor. Phys. {\bf 122}, 427-476 (2009). 

\bibitem{GOR68}
M. Gell-Mann, R. J. Oakes, and, B. Renner, 
Behavior of current divergences under SU(3) $\times$ SU(3), 
Phys. Rev. {\bf 175} 2195-2199 (1968). 

\bibitem{PDG}
R. L. Workman et al. (Particle Data Group), 
Prog. Theor. Exp. Phys. {\bf 2022}, 083C01 (2022). 

\bibitem{ANW83}
G. Adkins, C.R. Nappi and E. Witten, 
Static Properties of Nucleons in the Skyrme Model, 
Nucl. Phys. {\bf B228} 552-566 (1983). 

\bibitem{MS17}
K. Matsumoto and H. Suganuma, 
A Study of the H-dibaryon in Holographic QCD, 
JPS Conf. Proc. {\bf 13}, 020014 (2017); 
Holographic QCD for H-dibaryon (uuddss) 
EPJ Web Conf. {\bf 137}, 13018 (2017).

% \bibitem{CDG7779}
% C.G. Callan, R. Dashen and D.J. Gross, 
% A mechanism for quark confinement,  
% Phys. Lett. {\bf B66}, 375-381 (1979); 
% A theory of hadronic structure,  
% Phys. Rev. {\bf D19}, 1826-1855 (1979). 

\end{thebibliography}
\end{document}



% In holographic QCD, the action is defined in a five dimensional space and contains non-trivial gravity in a $w$ direction.
% Therefore, instantons corresponding to baryons change their profiles in conparison with four.
% However, the Yang-Mills term makes shrink a baryon size.

% One of the way to avoid this problem is adding the Chern-Simons term
% and there appears repulsive Coulomb force between instantons[hata, et al.].

% The solution which corresponds to the ground state is unique.
% Although
% In this section, we consider variational size baryons and study each energy.
% We try to change the baryon size and find that a constraint is necessary.
% This gives a potential as a functionof the baryon size.
% % We also show 
% % 前述したように今の手法では解とは異なるサイズの配位を作れない。
% % We take a constraint which gives a fixed size to a baryon.

% We consider that each baryon with different size baryons are rotationally symmetric
% and our method is use.
% Using the Witten Ansatz, 1+4 dim SU($N_f$) Yang-Mills theory reduces to 1+2 dim Abelian Higgs theory.
% In this reduction, rotationally symmetric instantons in Yang-Mills theory correspond to vorticesec in Abelian Higgs theory.
% Single instanton solution(BPHZ solution) is written as

% \begin{eqnarray}
% 	A_0 &= 0, \\
% 	A_M &= - \eta_{MN}\frac{x_N}{x^2+a^2}.
% \end{eqnarray}

% In this paper, we use the $A_0=0$ gauge. The index $M$ takes spatial coodinates $1\sim 4$, including the fourth spacial direction $w$.
% The variable $a$ is the instanton size and it locates at the origin.
% There are some relations between instanton profiles and zero point of vortex[],

% \begin{eqnarray}
% 	& (\textrm{instanton size}) & = \textrm{Re} [z_0] \\
% 	& (\textrm{instanton $w$-direction location}) & = \textrm{Im} [z_0] .
% \end{eqnarray}

% $z_0$ denotes a zero point of Higgs field $\phi$.
% これは4次元のQCDの解ではあるが、
% This configuration is no longer a solution in holographic QCD because there are gravity fields and Chern-Simons term.
% 本当の解はいくらか歪んだ形になる。(figureを見せる)

% In our dimensional reduction formulation, the insanton size is identical with real part of zero point location of fields $\phi$.
% % It is 尤もらしい to think the zero point as the size in this case.
% As discussed later, 零点の位置をバリオンのサイズとして扱うことの正当性を見る。
% In this paper, we assume that think that the zero point represents the size and call real part of zero point as size.
% aaaaaa

% We fix this zero point and calculate solutions.
% This constraint gives us a variety of size baryons.
% It is noted that our obtained configuration is solution
% and satisfies the field equations of holographic QCD on each constraint.
% % in fact にしても良い
% The figure ~\ref{fig:dilatation_potentail} shows configurations with different positions of zero points.







% The figure~\ref{fig:dilatation_potentail} shows 
% baryon size dependence of energy and seems to be quadratic against size $a$.
% All the points represent solutions in holographic QCD with fixed sizes.
% % and we performed the ground states on the constraint.
%  that there are two plots, for true $h,k$ case and $h= k= 1$ aprroximate case
% for the conparison we add plot for $h= k= 1$

% % Figure environment removed

% We fit this plot by quadratic function and the result is

% \begin{eqnarray}
% 	V(a) &=& A (a-B)^2 + C \\
% 	A &=& 0.049 \pm 0.00056,\\ 
% 	B &=& 1.9 \pm 0.0015,\\ 
% 	C &=& 1.05 \pm 6.1\times 10^{-5}.
% \end{eqnarray}

% % % Figure environment removed

% The value that minimizes the energy is 0.38 in $M_{\textrm KK}$ unit and this result coinsites our result we have shown.



% \section{Dilatation mode of baryons}

% In the previous section, we show the baryon mass depends on the size and its dependency is quadratic.
% The Baryon has a size potential and this suggests that there is a size oscillation mode of a baryon.

% In this section, we study the size oscillation mode in two different step.
% We start from analytical calculation and approximate the gravity in first subsection.
% Next, we bring a simple assumption but do not approximate it
% and do numerical calculation in second subsection.



% \subsection{Analytical calculation}

% We begin with the configuration of a single instanton in 1+4 dimension holographic QCD.
% This is a topological object in four dimensional Euclidean space coodinate $(x,y,z,w)$.

% To calculate a kinetice term for this mode, we bring a time dependence to the size:

% \begin{eqnarray}
%     a \rightarrow a(t) = a_I + \delta a(t).
% \end{eqnarray}

% We are considering variation around the instanton size $a_I$ which is a size without size fixing constraint.

% To carry out analytical calculation, we use an flat space approximation $h(w)=k(w)=1$.
% The kinetic energy of the dilatation mode Lagrangian is

% \begin{eqnarray}
% 	T & \sim \kappa \int d^3xdw \textrm{tr}\left[ F_{0M}^2 \right] \\
% 	& = \kappa \int d^3x dw \frac{24x^2}{(x^2+a_I^2)^4} a_I^2 \dot{a}^2 \\
% 	& = 48\pi^2\kappa \int dr \frac{r^5}{(r^2+a_I^2)^4} a_I^2 \dot{a}^2 \\
% 	& = 8\pi^2\kappa \ \dot{a}^2.
% \end{eqnarray}

% $\dot{a}$ means $\partial_t a$.
% This result is analytic, but has no gravitational effect.
% Therefore we take another estimation which do not spoil the gravity.


% \subsection{Numerical calculation}

% In previous subsection, we calculate the kinetic term and it does not include the gravity.
% Therefore, we do not ignore it by using a numerical calculation and an assumption.
% Assuming that the field variation is addiabatic, the configuration in the dilatation mode depends on time through only the size of the instanton.
% It is possible to write down field arguments simbolically

% \begin{eqnarray}
%     \phi = \phi(r,w;a(t))
% \end{eqnarray}

% % $a$ is input arbitalaly length in flat space, and unique number in holographic QCD.
% % So it is inappropriate to treat it as argument.
% As described before, we fix the zero point of fields and get arbitary size configuration of baryon.
% The kinetic term of size motion is difficult to calculate in straightforward way.
% So we take the above assumption to make it possible.
% Definite calculation is below.

% It is impossible to extract a kinetic information from the action without Adiabatic assuming.
% It makes time derivatives of fields into partial derivatives of size $a(t)$ and formulates a kinetic term.
% The important point is that gravitaional factors are not ignored and this is exact.
% %  in next-leading order.
% We prepare configurations with variable sizes by fixing zeros and do partial derivateves.

% % In our numerical calculation, we fix this zero in two dimensional space as getting lowest energy configurations.
% % This makes us to obtain variational sizes.\\

% \subsection{Dilatational mode excitation}

% Summarizing above calculation, we obtain the lagrangian for the instanton size up to

% Mass term is calculated analytically and numerically.

% Baryons may get excited by the finite excitation energy in the both ways, 
% and numerical one is slightly higher.
% The reason for this difference is the existing of gravitational field.
% Due to the gravity,$a$ gets harder to motion.

% This leads that every baryon with the rotational symmetry and the quantum number can have the same dilatation mode.
% Roper resonance is the candidate for corresponding this dilatational mode of nucleon.
% It is the first excitation and not fully unresolved problem.
% % Including the gravity, ~~

% In previous research[], the dilatational mode of baryons on its instanton size was calculated.
% But there are two problems in its calculation.
% First is that it does not consider the full gravitational effect.
% Its origin is the string theory and unique and important property for nonperturbative QCD analysis.
% So we take a numerical analysis and the result contains all the curved space effects up to next leading order.
% Second is that the model input parameter is not consistent with the meson sector.
% In this study, it is only $kappa$.

% %This result might be soft for a baryonic excitation.
% %The reason for this soft mode is that the size of instanton is originally 
% %the moduli and so the value of the size doesn't 
% %affects the mass of baryon if there is only Yang-Mills term in flat space.
% In this estimation, we take into account of the curved space effects for the calculation of the potential term.
% This curved space is unique and important for Holographic QCD at the point of the baryon's finite size with Chern-Simons term.
% % so we should consider this for a more strict estimation.

% \end{document}


In the previous research \cite{HSSY07}, 
Heta et al. investigated the holographic baryon 
and also obtained the lowest dilatational energy to be 
774 MeV, when the consistent parameter of 
$M_{\rm KK}\simeq 948{\rm MeV}$ with the meson sector is used. 
(This value seems consistent with that in Appendix D.)




******* HATA ******** Go to Intro-like 

In the previous research \cite{HSSY07}, 
the dilatational mode of baryons was calculated, 
but there was a problem in their calculation. 
They ignored the gravitational effect $h(w),k(w)$ for 
the instanton and used the 't~Hooft Ansatz, 
similar to our rough analytical estimate conducted, 
which is a solution for the flat four-dimensional YM theory.
However, this gravitational effect is inherited from 
the $N_c$ D4 branes to express the original Yang-Mills theory, 
and hence it is one of the key quantities 
for nonperturbative QCD analysis. 
As the result, their obtained masses were too large, 
and, to reproduce the baryon sector, 
they had to use quite different input parameters, 
which is inconsistent with the meson sector. 
% 
On the input parameter for the baryon sector, 
we have used the consistent values which is fitted by the meson spectrum.

**** HATA ****