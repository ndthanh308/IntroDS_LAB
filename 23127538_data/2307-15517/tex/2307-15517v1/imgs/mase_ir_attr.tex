\begin{table*}[]
    \centering
    \caption{\change{A model in MASE IR contains two constructs, operations and attributes. Each operation has a set of attributes for presenting both software and hardware parameters. MASE IR works with a set of actions and passes, which perform analysis and transformation. Actions are either a sequence of MASE passes or preprocess operations directly on {\tt torch.nn.Module}. }}
    \label{tab:mase_ir_attr}
\begin{subfigure}[b]{0.48\textwidth}
\resizebox{\textwidth}{!}{
\change{\begin{tabular}{l l p{6cm}}
\toprule
Categories & Types & Descriptions \\
\midrule
\multirow{6}{*}{Node/Ops} & \texttt{Conv2d} & Standard Pytorch operations \\
 & \texttt{Linear} & Standard Pytorch operations \\
 & \texttt{ReLU} & Standard Pytorch operations \\
 & \texttt{Softmax} & Standard Pytorch operations \\
 & \texttt{Layernorm} & Standard Pytorch operations \\
 & ... \\
 & \texttt{View} & Memory transformation \\
 & \texttt{Size} & Memory transformation \\
 & \texttt{Flatten} & Memory transformation \\
 & ... \\
\midrule
\multirow{3}{*}{Attributes} 
 & \texttt{args} & Input data shapes and types \\
 & \texttt{results} & Output data shapes and types \\
 & \texttt{interface} & Data interface with its predecessor and successor, e.g. handshake or no back pressure. \\
 & \texttt{tiling} & Tiling parameters for each data interface \\
 & \texttt{toolchain} & Synthesis flow for this node, e.g. vanilla MLIR HLS, optimised HLS or RTL kernels, user-defined kernels. \\
 & \texttt{param\_map} & Resource mapping of internal parameters. \\
 & \texttt{device\_id} & ID of hardware accelerator to map. \\
 & \texttt{rate} & Estimated input and output throughput of the node. \\
 & \texttt{area} & Estimated hardware resources of the node. \\
\bottomrule
\end{tabular}}}
\end{subfigure}
\begin{subfigure}[b]{0.48\textwidth}
\resizebox{\textwidth}{!}{
\change{\begin{tabular}{l l p{6cm}}
\toprule
Categories & Types & Descriptions \\
\midrule
\multirow{7}{*}{Actions} & \texttt{train} & Standard model training \\
 & \texttt{eval\_sw} & Standard software model evaluation for performance, accuracy and GPU power \\
 & \texttt{optimise\_sw} & Mixed-precision quantization (MPQ), Block-arithemtic quantization (BAQ) \\
 & \texttt{optimise\_hw} & Rate balancing, Resource partitioning \\
 & \texttt{emit\_hw} & Hardware design and test bench generation \\
 & \texttt{test\_hw} & Software hardware cosimulation \\
 & \texttt{eval\_hw} & Hardware evaluation for hardware resouces, performance and power \\
\midrule
\multirow{7}{*}{Passes} & \texttt{Init} & Traverse the IR and initialise attributes \\
 & \texttt{Quantize} & Perform quantization using the provided methods, e.g.  MPQ or BAQ \\
 & \texttt{DSE} & Explore hardware parameters and search for a good hardware mapping for given resources \\
 & \texttt{EmitVerilog} & Translate subgraphs in MASE IR to Verilog as the final hardware design \\
 & \texttt{EmitTB} & Generate Verilog test bench for the given hardware design \\
 & \texttt{Simulate} & Perform software and hardware co-simulation \\
 & \texttt{Synthesis} & Run back end synthesis tool for hardware implementation to bitstream \\
\bottomrule
\end{tabular}}}
\end{subfigure}
\end{table*}