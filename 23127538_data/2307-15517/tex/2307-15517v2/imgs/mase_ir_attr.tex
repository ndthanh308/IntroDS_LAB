\begin{table}[]
    \centering
    \caption{Key MASE passes used in this work. MASE contains 44 analysis and optimization passes. All these passes target different granularities varying from the model level to the bit level. These passes are general and type independent, which opens up opportunities for optimizing new data formats. Here we highlight \sw{software-specific} and \hw{hardware-specific} components.}
    \label{tab:mase_ir_attr}
% \resizebox{\textwidth}{!}{
{\footnotesize
\begin{tabular}{l p{6cm}}
\toprule
Names & Descriptions \\
\midrule
\sw{profile} & Profile variation of values for a given dataset, used to define the quantization search space. \\
\midrule
\sw{quantize} & Quantize a given model based on an input configuration, used to perform tensor-level mixed-precision quantization for a given data format. \\
\midrule
\hw{parallelize} & Exploit resource-constrained hardware parallelism based on a given hardware target, leading to a hardware design with high area efficiency. \\
\midrule
\texttt{evaluate} & Evaluate the hardware design based on a given expression of cost function, taking both model accuracy and area efficiency as arguments. \\
\midrule
\texttt{search} & Orchestrate existing search algorithms, such as random search and Tree-structured Parzen Estimator (TPE), to explore quantization search. \\
\midrule
\hw{emit} & Translate a co-design in {\tt MASE IR} into a dataflow hardware accelerator in SystemVerilog. \\
\bottomrule
\end{tabular}}
\end{table}