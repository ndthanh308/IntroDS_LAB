\documentclass[sigconf, nonacm]{acmart}
\usepackage{xcolor}
\usepackage{amsmath}
\usepackage{graphicx}
\usepackage{textcomp}
\usepackage{algpseudocode}
\usepackage{algorithm}
\usepackage{enumitem}
\usepackage{subcaption}
\usepackage{bm}
\usepackage{lipsum}
\usepackage{caption}
\usepackage{listings}
\usepackage{calc}
\usepackage{amsfonts}
\usepackage{tabu}
\usepackage{tikz}
\usepackage{makecell}
\usetikzlibrary{arrows.meta}
\usepackage{pgfplots}
\usepackage{multirow}
\usepackage{flushend}
\usepackage{booktabs}
\usepackage{hyperref}
\usepackage{cleveref}
\usepackage{soul}

\AtBeginDocument{%
  \providecommand\BibTeX{{%
    \normalfont B\kern-0.5em{\scshape i\kern-0.25em b}\kern-0.8em\TeX}}}

% \setcopyright{acmlicensed}
% \copyrightyear{2018}
% \acmYear{2018}
% \acmDOI{XXXXXXX.XXXXXXX}
%% These commands are for a PROCEEDINGS abstract or paper.
% \acmConference[Conference acronym 'XX]{Make sure to enter the correct
%   conference title from your rights confirmation emai}{June 03--05,
%   2018}{Woodstock, NY}
% \acmISBN{978-1-4503-XXXX-X/18/06}


\definecolor{codegreen}{rgb}{0,0.6,0}
\definecolor{codegray}{rgb}{0.5,0.5,0.5}
\definecolor{codepurple}{rgb}{0.58,0,0.82}
\definecolor{backcolour}{rgb}{0.95,0.95,0.92}
\definecolor{jcred}{HTML}{e31a1c}
\definecolor{jcgreen}{HTML}{33a02c}
\definecolor{jcblue}{HTML}{1f78b4}
\definecolor{jcorange}{HTML}{ff7f00}
\definecolor{jcpurple}{HTML}{6a3d9a}
\definecolor{jclightred}{HTML}{fb8072}
\definecolor{jclightgreen}{HTML}{b3de69}
\definecolor{jclightblue}{HTML}{80b1d3}
\definecolor{jclightorange}{HTML}{fdb462}
\definecolor{jclightpurple}{HTML}{bebada}
\definecolor{jcredl}{HTML}{fb8072}
\definecolor{jcgreenl}{HTML}{b3de69}
\definecolor{jcbluel}{HTML}{80b1d3}
\definecolor{jcorangel}{HTML}{fdb462}
\definecolor{jcpurplel}{HTML}{bebada}
\definecolor{jcbluem}{HTML}{488bb8}

\lstdefinestyle{mystyle}{
  frame=tblr,
  commentstyle=\color{codegreen},
  keywordstyle=\color{codepurple},
  basicstyle=\scriptsize\ttfamily,
  breakatwhitespace=false,         
  breaklines=true,                 
  captionpos=b,                    
  keepspaces=true,                 
  numbers=left,                    
  numbersep=5pt,                  
  showspaces=false,                
  showstringspaces=false,
  showtabs=false,                  
  tabsize=2,
  escapeinside={(*@}{@*)},
}
\lstset{style=mystyle}

\lstdefinelanguage{maseir}{%
  language     = python,
  morekeywords = {in, return, rate, MXInt},
}

\definecolor{jcblue}{HTML}{1f78b4}
\newcommand*\az[1]{\textcolor{jcorange}{\bf AZ: #1}}
\newcommand*\jc[1]{\textcolor{jcblue}{\bf JC: #1}}
\newcommand*\zy[1]{\textcolor{jcpurple}{\bf ZY: #1}}
\newcommand*\jcp[1]{{\em #1}}
\newcommand*\todo[1]{\textcolor{jcblue}{{\bf To polish: } #1}}
\newcommand*\best[1]{\textcolor{jcgreen}{\bf #1}}

\newcommand\capped[1]{\textcolor{jcblue}{\boldsymbol{#1}}}
\newcommand\varied[1]{\textcolor{jcred}{\boldsymbol{#1}}}

\usepackage{pifont}% http://ctan.org/pkg/pifont
\newcommand{\cmark}{\textcolor{jcgreen}{\ding{51}}}%
\newcommand{\xmark}{\textcolor{jcred}{\ding{55}}}%

\newcommand\gc[1]{\textcolor{blue}{{\bf [GC:} #1{\bf]}}}
\newcommand\gcc[2] {\textcolor{blue}{\st{#1} #2}}
\newcommand*\change[1]{\textcolor{blue}{#1}}

\newcommand*\sw[1]{\textcolor{jcblue}{\tt #1}}
\newcommand*\hw[1]{\textcolor{jcred}{\tt #1}}

\begin{filecontents}{boolq.dat}
index	int8	mixedint	mxint8	mixedmxint	mixedintb	mixedmxintb	float32
1	-0.890501348	-1.178442733	0.6261927737	-0.2099001618	5.51	4.72	66.39143731
2	0.5525231034	-0.484469631	0.5036282809	-0.04968024616	5.08	4.83	67.43119266
3	0.4220966537	-1.186871702	-0.5141495466	-0.6071335371	5.18	4.80	57.7370031
4	-0.1195867972	0.760648386	0.5344065058	0.4220966537	5.46	5.00	57.7370031
5	-1.016384119	-1.487060065	-0.339345051	-0.4719462979	5.24	4.99	66.0244648
\end{filecontents}

\begin{filecontents}{mnli.dat}
index	int8	mixedint	mxint8	mixedmxint	mixedintb	mixedmxintb	float32
1	-0.5794988021	-1.605709074	0.3860512573	-0.2735820078	5.61	4.70	82.02750891
2	-0.8061893075	-1.659251778	0.4638724611	-0.008658731392	5.39	4.89	82.71013754
3	-0.7747017242	0.5863023545	-0.5081972117	-0.6023347911	5.34	4.87	35.8125318
4	-0.4043317139	0.1531848482	0.01407269901	-0.3298178739	5.33	4.96	35.5170657
5	-0.5725366599	0.904330475	-0.6548613107	-0.6908263864	5.30	4.87	32.8171167
\end{filecontents}

\begin{filecontents}{qnli.dat}
index	int8	mixedint	mxint8	mixedmxint	mixedintb	mixedmxintb	float32
1	-0.6780773361	-1.536244707	0.2683068421	-0.1123228658	5.54	4.95	89.14515834
2	0.4597395982	-1.518088613	-0.5247461737	-0.6539547385	5.34	4.93	89.65769724
3	-0.633005089	0.5016870146	0.2053628363	-0.07019108085	5.31	4.77	51.3454146
4	0.0111219627	0.4736241708	0.412618444	-0.1348334738	5.40	4.99	51.14406
5	-0.5375408813	0.509499489	0.560170878	-0.04445767568	5.52	4.92	50.8877906
\end{filecontents}

\begin{filecontents}{qqp.dat}
index	int8	mixedint	mxint8	mixedmxint	mixedintb	mixedmxintb	float32
1	0.00812389768	-1.329623979	-0.399867118	-0.564798342	5.69	4.58	90.36111798
2	-0.2451255433	-1.374536312	0.4051017581	-0.1393853017	5.37	4.76	90.37843186
3	1.04252847	1.468181932	-0.4887191422	0.601930259	5.34	4.92	39.1911947
4	-0.9437001289	1.38856017	-0.4038336546	-0.4660008053	5.22	4.93	43.3341578
5	-0.4959792883	1.455710509	-0.7382362637	-0.7681299903	5.27	4.84	39.2431363
\end{filecontents}

\begin{filecontents}{rte.dat}
index	int8	mixedint	mxint8	mixedmxint	mixedintb	mixedmxintb	float32
1	-0.4768180312	-0.8355577989	0.6416816758	-0.1637295642	5.32	4.60	63.53790614
2	0.09966069979	-1.140572704	0.5963707899	-0.05386858671	5.44	4.97	68.59205776
3	-0.3319495935	0.5279415017	0.3428999056	-0.135953122	5.30	5.15	52.7075812
4	-0.3663588824	-0.1888320569	0.3811770152	-0.4934102272	5.26	4.88	55.234657
5	-0.7367674465	-0.4337149638	0.3043489712	-0.2866039791	5.36	4.79	55.234657
\end{filecontents}

\begin{filecontents}{sst2.dat}
index	int8	mixedint	mxint8	mixedmxint	mixedintb	mixedmxintb	float32
1	-0.1560876771	-1.470005663	-0.02221264038	-0.1560876771	5.33	5.18	91.97247706
2	-0.1895340956	-1.076900868	-0.2871645878	-0.3649552663	5.21	4.73	92.08715596
3	-1.422755446	-1.471146311	0.117240894	-0.6981085265	5.52	4.90	82.1100917
4	0.5987704325	0.2094658185	-0.19038311	-0.2878428069	5.45	4.87	51.7201835
5	-1.407234933	-1.379682994	-0.3223900216	-0.6075435566	5.38	4.93	76.4908257
\end{filecontents}
\begin{filecontents}{down_proj.dat}
block_id	layer	data
0	down_proj	0.00058
1	down_proj	0.001136
2	down_proj	0.37329
3	down_proj	0.003027
4	down_proj	0.006073
5	down_proj	0.008455
6	down_proj	0.011452
7	down_proj	0.01506
8	down_proj	0.017614
9	down_proj	0.020181
10	down_proj	0.02391
11	down_proj	0.02959
12	down_proj	0.033934
13	down_proj	0.036618
14	down_proj	0.042734
15	down_proj	0.053365
16	down_proj	0.068201
17	down_proj	0.083521
18	down_proj	0.099716
19	down_proj	0.120658
20	down_proj	0.147937
21	down_proj	0.176383
22	down_proj	0.190917
23	down_proj	0.201647
24	down_proj	0.219331
25	down_proj	0.231616
26	down_proj	0.247641
27	down_proj	0.264612
28	down_proj	0.303683
29	down_proj	0.361003
30	down_proj	1.248902
31	down_proj	4.422037
\end{filecontents}

\begin{filecontents}{gate_proj.dat}
block_id	layer	data
0	gate_proj	0.007369
1	gate_proj	0.01941
2	gate_proj	0.036743
3	gate_proj	0.046084
4	gate_proj	0.069922
5	gate_proj	0.099403
6	gate_proj	0.111216
7	gate_proj	0.1306
8	gate_proj	0.149792
9	gate_proj	0.155536
10	gate_proj	0.165331
11	gate_proj	0.183561
12	gate_proj	0.197641
13	gate_proj	0.192425
14	gate_proj	0.203987
15	gate_proj	0.219964
16	gate_proj	0.257455
17	gate_proj	0.274384
18	gate_proj	0.293374
19	gate_proj	0.320004
20	gate_proj	0.351224
21	gate_proj	0.375132
22	gate_proj	0.380222
23	gate_proj	0.398473
24	gate_proj	0.418204
25	gate_proj	0.45515
26	gate_proj	0.480616
27	gate_proj	0.530438
28	gate_proj	0.577889
29	gate_proj	0.653102
30	gate_proj	0.787752
31	gate_proj	1.092747
\end{filecontents}

\begin{filecontents}{k_proj.dat}
block_id	layer	data
0	k_proj	0.640564
1	k_proj	1.391764
2	k_proj	1.61566
3	k_proj	1.962584
4	k_proj	2.243196
5	k_proj	2.116435
6	k_proj	2.184222
7	k_proj	2.233603
8	k_proj	2.125899
9	k_proj	2.245574
10	k_proj	2.265057
11	k_proj	2.439854
12	k_proj	2.477168
13	k_proj	2.30886
14	k_proj	2.360348
15	k_proj	2.423574
16	k_proj	2.376156
17	k_proj	2.364948
18	k_proj	2.379515
19	k_proj	2.244871
20	k_proj	2.275567
21	k_proj	2.232513
22	k_proj	1.996517
23	k_proj	2.109262
24	k_proj	2.190735
25	k_proj	2.24573
26	k_proj	2.223619
27	k_proj	2.207722
28	k_proj	2.390896
29	k_proj	2.315675
30	k_proj	2.356001
31	k_proj	2.344723
\end{filecontents}

% this is a_hat
% \begin{filecontents}{mm_a.dat}
% block_id	layer	data
% 0	matmul_a	9.62E-06
% 1	matmul_a	2.34E-05
% 2	matmul_a	2.53E-05
% 3	matmul_a	0.000196
% 4	matmul_a	0.000189
% 5	matmul_a	0.000191
% 6	matmul_a	0.00018
% 7	matmul_a	0.000164
% 8	matmul_a	0.000146
% 9	matmul_a	0.000139
% 10	matmul_a	0.00012
% 11	matmul_a	0.000123
% 12	matmul_a	0.000116
% 13	matmul_a	0.000137
% 14	matmul_a	0.000134
% 15	matmul_a	0.000134
% 16	matmul_a	0.000146
% 17	matmul_a	0.000153
% 18	matmul_a	0.000173
% 19	matmul_a	0.000182
% 20	matmul_a	0.000183
% 21	matmul_a	0.000186
% 22	matmul_a	0.000194
% 23	matmul_a	0.000179
% 24	matmul_a	0.000204
% 25	matmul_a	0.000216
% 26	matmul_a	0.000218
% 27	matmul_a	0.000235
% 28	matmul_a	0.000202
% 29	matmul_a	0.00022
% 30	matmul_a	0.000214
% 31	matmul_a	9.54E-05
% \end{filecontents}

\begin{filecontents}{mm_a.dat}
block_id	layer	data
0	matmul_a	4.583201408
1	matmul_a	8.261958122
2	matmul_a	8.959178925
3	matmul_a	22.27571106
4	matmul_a	26.59131813
5	matmul_a	26.60893822
6	matmul_a	27.68237114
7	matmul_a	28.77270889
8	matmul_a	23.46584702
9	matmul_a	23.49845123
10	matmul_a	22.47806549
11	matmul_a	29.29229546
12	matmul_a	22.40054321
13	matmul_a	18.07605553
14	matmul_a	24.8982563
15	matmul_a	22.34719086
16	matmul_a	28.89003944
17	matmul_a	22.7196064
18	matmul_a	22.82881165
19	matmul_a	20.64634132
20	matmul_a	29.82372284
21	matmul_a	21.3174057
22	matmul_a	20.33133888
23	matmul_a	18.88310242
24	matmul_a	20.02614784
25	matmul_a	24.69710732
26	matmul_a	23.36770058
27	matmul_a	22.42767525
28	matmul_a	27.82991409
29	matmul_a	27.73001671
30	matmul_a	31.96291351
31	matmul_a	23.8020649
\end{filecontents}

\begin{filecontents}{mm_k.dat}
block_id	layer	data
0	matmul_k	0.640608
1	matmul_k	1.392699
2	matmul_k	1.615904
3	matmul_k	1.962592
4	matmul_k	2.243027
5	matmul_k	2.11656
6	matmul_k	2.185455
7	matmul_k	2.233595
8	matmul_k	2.124991
9	matmul_k	2.245548
10	matmul_k	2.263885
11	matmul_k	2.440051
12	matmul_k	2.477498
13	matmul_k	2.308379
14	matmul_k	2.36052
15	matmul_k	2.424104
16	matmul_k	2.37581
17	matmul_k	2.364925
18	matmul_k	2.380121
19	matmul_k	2.244859
20	matmul_k	2.27821
21	matmul_k	2.232826
22	matmul_k	1.996498
23	matmul_k	2.109643
24	matmul_k	2.19077
25	matmul_k	2.246051
26	matmul_k	2.223179
27	matmul_k	2.207945
28	matmul_k	2.390191
29	matmul_k	2.315938
30	matmul_k	2.356031
31	matmul_k	2.344867
\end{filecontents}


\begin{filecontents}{mm_q.dat}
block_id	layer	data
0	matmul_q	0.820417
1	matmul_q	0.909834
2	matmul_q	0.921817
3	matmul_q	1.380494
4	matmul_q	1.479777
5	matmul_q	1.449666
6	matmul_q	1.486864
7	matmul_q	1.55793
8	matmul_q	1.540984
9	matmul_q	1.577079
10	matmul_q	1.458236
11	matmul_q	1.47481
12	matmul_q	1.494928
13	matmul_q	1.501112
14	matmul_q	1.519643
15	matmul_q	1.494174
16	matmul_q	1.492791
17	matmul_q	1.513651
18	matmul_q	1.576856
19	matmul_q	1.515738
20	matmul_q	1.645746
21	matmul_q	1.528676
22	matmul_q	1.514499
23	matmul_q	1.476538
24	matmul_q	1.462168
25	matmul_q	1.603797
26	matmul_q	1.637066
27	matmul_q	1.64008
28	matmul_q	1.624713
29	matmul_q	1.649848
30	matmul_q	1.627374
31	matmul_q	1.345481
\end{filecontents}

\begin{filecontents}{o_proj.dat}
block_id	layer	data
0	o_proj	0.000849
1	o_proj	0.000558
2	o_proj	0.001073
3	o_proj	0.001042
4	o_proj	0.00255
5	o_proj	0.003543
6	o_proj	0.005321
7	o_proj	0.007146
8	o_proj	0.009754
9	o_proj	0.010718
10	o_proj	0.015698
11	o_proj	0.019184
12	o_proj	0.020214
13	o_proj	0.018718
14	o_proj	0.021011
15	o_proj	0.027764
16	o_proj	0.027841
17	o_proj	0.031504
18	o_proj	0.034132
19	o_proj	0.036384
20	o_proj	0.045454
21	o_proj	0.048235
22	o_proj	0.044412
23	o_proj	0.084039
24	o_proj	0.04761
25	o_proj	0.053319
26	o_proj	0.056479
27	o_proj	0.053287
28	o_proj	0.098619
29	o_proj	0.104148
30	o_proj	0.107262
31	o_proj	0.258544
\end{filecontents}

\begin{filecontents}{up_proj.dat}
block_id	layer	data
0	up_proj	0.005243
1	up_proj	0.01358
2	up_proj	0.023768
3	up_proj	0.032877
4	up_proj	0.044648
5	up_proj	0.057977
6	up_proj	0.071179
7	up_proj	0.08627
8	up_proj	0.095485
9	up_proj	0.107638
10	up_proj	0.118283
11	up_proj	0.135849
12	up_proj	0.143168
13	up_proj	0.151236
14	up_proj	0.162099
15	up_proj	0.181775
16	up_proj	0.207143
17	up_proj	0.222391
18	up_proj	0.240355
19	up_proj	0.255714
20	up_proj	0.284986
21	up_proj	0.302917
22	up_proj	0.314546
23	up_proj	0.331199
24	up_proj	0.346469
25	up_proj	0.376572
26	up_proj	0.407747
27	up_proj	0.463659
28	up_proj	0.542912
29	up_proj	0.676201
30	up_proj	0.863958
31	up_proj	0.996982
\end{filecontents}

\begin{filecontents}{v_proj.dat}
block_id	layer	data
0	v_proj	0.001518
1	v_proj	0.00716
2	v_proj	0.012575
3	v_proj	0.049147
4	v_proj	0.069648
5	v_proj	0.107492
6	v_proj	0.118758
7	v_proj	0.138635
8	v_proj	0.167575
9	v_proj	0.195837
10	v_proj	0.20727
11	v_proj	0.171863
12	v_proj	0.217414
13	v_proj	0.251737
14	v_proj	0.250857
15	v_proj	0.228967
16	v_proj	0.250477
17	v_proj	0.281871
18	v_proj	0.298244
19	v_proj	0.310965
20	v_proj	0.368823
21	v_proj	0.369557
22	v_proj	0.417844
23	v_proj	0.427004
24	v_proj	0.474482
25	v_proj	0.508455
26	v_proj	0.598923
27	v_proj	0.625629
28	v_proj	0.582517
29	v_proj	0.647496
30	v_proj	0.599704
31	v_proj	0.358502
\end{filecontents}


\begin{filecontents}{q_proj.dat}
block_id	layer	data
0	q_proj	0.820498
1	q_proj	0.908427
2	q_proj	0.921465
3	q_proj	1.379955
4	q_proj	1.479797
5	q_proj	1.448127
6	q_proj	1.486803
7	q_proj	1.556324
8	q_proj	1.540699
9	q_proj	1.577237
10	q_proj	1.45828
11	q_proj	1.474819
12	q_proj	1.494934
13	q_proj	1.50124
14	q_proj	1.519048
15	q_proj	1.49438
16	q_proj	1.492971
17	q_proj	1.513563
18	q_proj	1.576963
19	q_proj	1.515778
20	q_proj	1.646276
21	q_proj	1.528702
22	q_proj	1.514488
23	q_proj	1.476599
24	q_proj	1.46201
25	q_proj	1.603805
26	q_proj	1.636409
27	q_proj	1.640081
28	q_proj	1.624939
29	q_proj	1.6495
30	q_proj	1.626988
31	q_proj	1.345126
\end{filecontents}

\begin{filecontents}{mm_b.dat}
block_id	layer	data
0	matmul_b	0.00011662
1	matmul_b	0.000484521
2	matmul_b	0.000730962
3	matmul_b	0.000473927
4	matmul_b	0.001486813
5	matmul_b	0.002111672
6	matmul_b	0.003810423
7	matmul_b	0.005361772
8	matmul_b	0.008544927
9	matmul_b	0.009674242
10	matmul_b	0.013150034
11	matmul_b	0.015437667
12	matmul_b	0.016604085
13	matmul_b	0.016616968
14	matmul_b	0.016413284
15	matmul_b	0.018827401
16	matmul_b	0.016964803
17	matmul_b	0.018295461
18	matmul_b	0.017440423
19	matmul_b	0.016910914
20	matmul_b	0.022143988
21	matmul_b	0.019937353
22	matmul_b	0.018773805
23	matmul_b	0.027836557
24	matmul_b	0.019036479
25	matmul_b	0.0225095
26	matmul_b	0.022498665
27	matmul_b	0.016880989
28	matmul_b	0.0313977
29	matmul_b	0.029177757
30	matmul_b	0.033092171
31	matmul_b	0.044748198
\end{filecontents}

\begin{filecontents}{last.dat}
block_id	layer	data
0	last	0.001191478
1	last	0.003190377
2	last	0.006887271
3	last	0.387333661
4	last	0.396223694
5	last	0.424028605
6	last	0.437743604
7	last	0.451875061
8	last	0.468026608
9	last	0.486001253
10	last	0.511690974
11	last	0.540429235
12	last	0.577379823
13	last	0.607893944
14	last	0.647199512
15	last	0.710090756
16	last	0.787984908
17	last	0.90104425
18	last	1.067125559
19	last	1.281699061
20	last	1.564041376
21	last	1.921423674
22	last	2.303662777
23	last	2.856389999
24	last	3.330206394
25	last	3.881980658
26	last	4.374501228
27	last	4.91159153
28	last	5.606812
29	last	6.481289864
30	last	7.51671505
31	last	8.244023323
\end{filecontents}
\begin{filecontents}{alg_rs.dat}
t_rs	rs	t_tpe	tpe	t_nsga	nsga	t_qmc	qmc
25.00	2.26	23.00	2.16	25.00	2.24	21.00	2.26
73.00	2.29	67.00	2.26	71.00	2.27	62.00	2.26
510.00	2.30	500.00	2.39	513.00	2.33	456.00	2.34
650.00	2.33	640.00	2.39	654.00	2.33	580.00	2.34
3031.00	2.33	3022.00	2.40	3027.00	2.38	2664.00	2.36
\end{filecontents}
\begin{filecontents}{alg_tpe.dat}
t_rs	rs	t_tpe	tpe	t_nsga	nsga	t_qmc	qmc
25.00	2.26	23.00	2.16	25.00	2.24	21.00	2.26
49.00	2.26	45.00	2.21	48.00	2.26	41.00	2.26
73.00	2.29	67.00	2.26	71.00	2.27	62.00	2.26
143.00	2.29	134.00	2.27	141.00	2.29	123.00	2.34
165.00	2.29	156.00	2.36	165.00	2.29	144.00	2.34
441.00	2.29	432.00	2.39	441.00	2.33	393.00	2.34
2444.00	2.33	2436.00	2.40	2443.00	2.38	2163.00	2.36
3031.00	2.33	3022.00	2.40	3027.00	2.38	2664.00	2.36
\end{filecontents}
\begin{filecontents}{alg_nsga.dat}
t_rs	rs	t_tpe	tpe	t_nsga	nsga	t_qmc	qmc
25.00	2.26	23.00	2.16	25.00	2.24	21.00	2.26
49.00	2.26	45.00	2.21	48.00	2.26	41.00	2.26
73.00	2.29	67.00	2.26	71.00	2.27	62.00	2.26
96.00	2.29	89.00	2.26	94.00	2.29	82.00	2.27
210.00	2.29	200.00	2.36	213.00	2.33	185.00	2.34
1635.00	2.33	1625.00	2.39	1633.00	2.36	1453.00	2.36
2013.00	2.33	2004.00	2.39	2013.00	2.38	1787.00	2.36
3031.00	2.33	3022.00	2.40	3027.00	2.38	2664.00	2.36
\end{filecontents}

\begin{filecontents}{alg_qmc.dat}
t_rs	rs	t_tpe	tpe	t_nsga	nsga	t_qmc	qmc
25.00	2.26	23.00	2.16	25.00	2.24	21.00	2.26
96.00	2.29	89.00	2.26	94.00	2.29	82.00	2.27
120.00	2.29	111.00	2.26	117.00	2.29	103.00	2.29
143.00	2.29	134.00	2.27	141.00	2.29	123.00	2.34
1023.00	2.33	1013.00	2.39	1025.00	2.33	912.00	2.36
3031.00	2.33	3022.00	2.40	3027.00	2.38	2664.00	2.36
\end{filecontents}

\begin{filecontents}{table1.dat}
models	int8	mixedint	mxint8	mixedint_nodse	mixedintdse		bl8	bmf8	model
bertbase	1	1.715714413	0.8271765435	0.9902813912	1.113408789		0.6316797488	0.5781902687	1
bertlarge	1	1.436647779	0.7084716311	0.843524429	0.944902555		0.5940826834	0.5313534566	2
opt125m	1	1.64779531	0.8271765435	0.9902813912	1.086398751		0.6316797488	0.5781902687	3
opt350m	1	1.437503333	0.7075920764	0.8417999218	0.9610384717		0.59334475	0.5306934357	4
opt13b	1	1.353626601	0.6678824773	0.7929638194	0.9060581431		0.5600474687	0.5009120226	5
opt27b	1	1.314341047	0.6889586155	0.8133393229	0.9029569922		0.5663703493	0.5065672591	6
opt67b	1	1.39506101	0.6767590693	0.7964552285	0.8838265938		0.5436994036	0.4862901445	7
llama7b	1	1.566247103	0.7733622375	0.892585685	0.9658354531		0.523619803	0.4683307611	8
vicuna	1	1.581536477	0.7789400297	0.892585685	0.973119892		0.523619803	0.4683307611	9
alpaca	1	1.572862839	0.7789400297	0.892585685	0.991815272		0.523619803	0.4683307611	10
mean	1	1.501875218	0.7409169343	0.868055057	0.9634369624		0.563208909	0.5037396409	11
\end{filecontents}

\begin{filecontents}{table1a.dat}
models	bl8	bmf8	mxint8	model	int8	mixedint	mixedint_nodse	mixedintdse
bertbase	-0.4346008363	-1.637801829	-0.04715800113	1	-0.828904682	-1.551374356	-0.1347506861	-0.1347506861
bertlarge	-0.4697631246	-1.629700051	0.04715800113	2	0.0584639856	-1.568270971	-0.09279349772	-0.09198333224
opt125m	-0.6243852414	-1.641228139	0.04715800113	3	-0.1198495032	-1.468314831	-0.1198495032	-0.1198495032
opt350m	-0.5325250721	-1.603203935	0	4	-0.03678853311	-1.059881355	-0.03678853311	0.1111280363
opt13b	-1.482270295	-1.501523449	-0.4346008363	5	-1.420869612	-1.469459707	-0.5192240228	-0.688010253
opt27b	0.3758647143	-0.2559475494	0.2273981699	6	0.5987686274	0.2094613946	-0.1903877326	-0.06453286113
opt67b	-1.401352283	-1.444588105	0	7	-1.31092261	-1.379682528	-0.2331992416	-0.3223847087
llama7b	-1.34215215	-1.546225463	0.04715800113	8	-1.563444299	-1.415922746	-0.4177871272	-0.1604385166
vicuna	-1.460400148	-1.568308524	0.1284315811	9	-1.519953934	-1.441531754	-0.8650802744	-0.2694195375
alpaca	-1.270925604	-1.566961629	-0.2827353726	10	-1.52175377	-1.506059686	-0.2033321097	-0.2033321097
mean	-1.058266551	-1.567635599	0.02421876962	11	-1.133612349	-1.455129725	-0.1969081461	-0.2503781829
\end{filecontents}

\begin{filecontents}{gpua.dat}
mxint4	mxint6	mxint	index
-0.2414966733	0.2543546921	-0.1347506861	1
0.3625013239	0.2819646233	-0.09198333224	2
-0.2324370092	-0.1198495032	-0.1198495032	3
-0.03678853311	-0.03678853311	0.1111280363	4
-1.343514939	-0.4646981494	-0.688010253	5
-0.1903877326	-0.1903877326	-0.06453286113	6
-1.257937635	0.02905895008	-0.3223847087	7
-1.168111338	-0.4177871272	-0.1604385166	8
-1.368093409	-0.8650802744	-0.2694195375	9
-0.9739171822	-0.2033321097	-0.2033321097	10
-0.7466692213	-0.1565491513	-0.2503781829	11
\end{filecontents}

\begin{filecontents}{gpu.dat}
mxint4	mxint6	mixedmxint	index	mixedr	mx4r
442.7431615	345.8505787	388.8521763	1	1.12	1.28
115.7519194	92.16555347	103.2423768	2	1.12	1.26
442.7431615	345.8505787	379.4190621	3	1.10	1.28
115.3862587	91.70277071	104.6922058	4	1.14	1.26
28.58400631	22.63766518	25.86630105	5	1.14	1.26
3.883612767	3.151736876	3.499010523	6	1.11	1.23
5.182814504	4.207502704	4.669066948	7	1.11	1.23
4.814736842	4.096783515	4.432984787	8	1.08	1.18
4.814736842	4.096783515	4.466418854	9	1.09	1.18
4.814736842	4.096783515	4.552226778	10	1.11	1.18
26.45835484	21.47564984	23.89717302	11	1.11	1.23
\end{filecontents}
\begin{filecontents}{alg_base.dat}
t c
23.00	5.330814
44.00	5.351686
66.00	5.404999
87.00	5.42916
108.00	5.460105
457.00	5.461488
1208.00	5.479368
1315.00	5.482697
1442.00	5.483745
2646.00	5.491555
2693.00	5.505854
2836.00	5.505854
\end{filecontents}


\begin{filecontents}{alg_dse.dat}
t c
65.00	5.452857
130.00	5.473287
714.00	5.505811
844.00	5.528737
1442.00	5.562693
8702.00	5.562693
\end{filecontents}

\begin{document}


\title{A Dataflow Compiler for Efficient LLM Inference using Custom Microscaling Formats}

\author{Jianyi Cheng}
\affiliation{
\institution{University of Cambridge, UK}
\country{}
}
\email{jianyi.cheng@cl.cam.ac.uk}

\author{Cheng Zhang}
\affiliation{\institution{Imperial College London, UK}
\country{}
}
\email{cheng.zhang122@imperial.ac.uk}

\author{Zhewen Yu}
\affiliation{\institution{Imperial College London, UK}
\country{}
}
\email{zhewen.yu18@imperial.ac.uk}

\author{Christos-Savvas Bouganis}
\affiliation{\institution{Imperial College London, UK}
\country{}
}
\email{christos-savvas.bouganis@imperial.ac.uk}

\author{George A. Constantinides}
\affiliation{\institution{Imperial College London, UK}
\country{}
}
\email{g.constantinides@imperial.ac.uk}

\author{Yiren Zhao}
\affiliation{\institution{Imperial College London, UK}
\country{}
}
\email{a.zhao@imperial.ac.uk}

%% By default, the full list of authors will be used in the page
%% headers. Often, this list is too long, and will overlap
%% other information printed in the page headers. This command allows
%% the author to define a more concise list
%% of authors' names for this purpose.
\renewcommand{\shortauthors}{Cheng, et al.}

%%
%% The abstract is a short summary of the work to be presented in the
%% article.
\begin{abstract}
Model quantization represents both parameters (weights) and intermediate values (activations) in a more compact format, thereby directly reducing both computational and memory cost in hardware. The quantization of recent large language models (LLMs) faces challenges to achieve competitive memory density compared to other models such as convolutional neural networks, since values in LLMs require larger dynamic ranges.

Current hardware can expedite computation for LLMs using compact numerical formats such as low-bitwidth integers or floating-point numbers. Each has advantages: integer operations simplify circuit design, whereas floating-point calculations can enhance accuracy when a wider dynamic range is required. In this work, we seek an efficient data format that combines the best of both worlds: Microscaling (MX) formats. MX formats are efficient data formats that achieve both large dynamic ranges and high memory density.

In this paper, we propose a compiler named MASE for exploring mixed-precision MX formats on dataflow hardware accelerators for LLM inference. Our main contributions are twofold. First, we propose a novel orchestration abstraction to explore both software and hardware optimizations with new data formats. Second, MASE achieves LLM inference at an average precision of 4-bits, with minimal to no accuracy degradation. To our knowledge, MASE represents the first effort to harness fine-grain multi-precision MX formats in the design of LLM hardware accelerators. Over a range of LLMs and datasets, MASE achieves an average improvement of 24\% in $\Delta$ accuracy with an overhead of only 3\% in energy efficiency compared to designs using 8-bit fixed-point numbers.
\end{abstract}

% \received{20 February 2007}
% \received[revised]{12 March 2009}
% \received[accepted]{5 June 2009}

%%
%% This command processes the author and affiliation and title
%% information and builds the first part of the formatted document.
\maketitle

\title{Inclusive photon multiplicity at forward pseudorapidities in pp and p--Pb collisions \\ at $\sqrt{s_{\rm NN}}$ = 5.02~TeV with ALICE}
\author[1*]{Abhi Modak (for the ALICE Collaboration)}
\affil[1]{Bose Institute, Kolkata, India}
\affil[*]{Address correspondence to: abhi.modak@cern.ch}

\onehalfspacing
\maketitle

\date{}

%%%%%% Abstract %%%%%
\begin{abstract}

Global observables such as the pseudorapidity distributions of particle multiplicities in the final state are crucial to shed light into the physics processes involved in hadronic collisions. In proton--lead (p--Pb) collisions at Large Hadron Collider (LHC) energies, such measurements provide an important baseline to understand lead--lead (Pb--Pb) results by disentangling hot nuclear matter effects from the ones due to the cold nuclear matter. Multiplicity measurements can also put constraints on theoretical models describing the initial stages of the collision, e.g., to what degree the nucleon and the nuclei interact as dilute (partons) or dense (CGC-like) fields. The study of inclusive photon multiplicity aims to provide complementary measurements to those obtained with charged particles.

In these proceedings, the pseudorapidity distributions of inclusive photons at forward pseudorapidity (2.3~$<~\eta_{\rm \,lab}~<$~3.9) in pp and p--Pb collisions at $\sqrt{s_{\rm NN}}$ = 5.02 TeV are presented. The data samples were collected using the Photon Multiplicity Detector (PMD) of ALICE. The multiplicity dependence of photon production in p--Pb collisions is presented and a comparison with charged-particle distributions measured at mid-pseudorapidity is shown. The results are also compared with predictions from Monte Carlo event generators.

\end{abstract}

\section{Introduction}
One of the primary goals of heavy-ion collision experiments, such as ALICE, is to study and understand the properties of the deconfined state of nuclear matter, commonly known as the quark--gluon plasma (QGP). The first step in characterizing the produced QGP matter in these collisions is the measurement of pseudorapidity distributions of produced final-state particles. Such studies in pp and p--Pb collisions are also important as they provide the baselines for the interpretation of the measurements in heavy-ion collisions. This contribution reports the measurements of inclusive photon multiplicities for minimum bias pp, p--Pb collisions and for various multiplicity classes in p--Pb collisions at $\sqrt{s\rm_{NN}}$~=~5.02~TeV.

\section{Data analysis}

This analysis was performed using the ALICE~\cite{ALICE:Exp} data collected in 2013 during LHC Run~1 for p--Pb collisions and in 2015 during LHC Run~2 for pp collisions. The data from p--Pb collisions were recorded for two beam configurations: in one (denoted as p--Pb), the lead beam travelled towards positive $\eta_{\rm \,lab}$ and in the other configuration (denoted as Pb--p) it moved towards negative $\eta_{\rm \,lab}$. The pp data were analysed for inelastic events, whereas measurements in p--Pb collisions were performed for non-single diffractive interactions. Events with the reconstructed primary vertex position along the beam line, $|v_{z}|<10$ cm, from the nominal interaction point were considered. The multiplicity classes were determined by measuring the charged-particle multiplicity in the outer layer of the Silicon Pixel Detector~\cite{spd} at mid-pseudorapidity (denoted as CL1 estimator) and the energy deposited in the Pb-remnant side of the neutron calorimeter~\cite{zdc} at beam rapidity (denoted as ZNA estimator)~\cite{ALICE:ChPrpPbCent}. The raw distributions of photons were obtained by counting the number of reconstructed clusters (in the preshower plane of the Photon Multiplicity Detector (PMD)~\cite{pmd}) that satisfied the photon--hadron discrimination thresholds~\cite{ALICE:PMDpp,ALICE:PMDpPb}. The distributions were then corrected for various instrumental effects (detector inefficiency, limited acceptance, contaminations from hadron clusters and secondary particles produced in interactions with surrounding materials of the PMD) using a Bayesian unfolding method~\cite{BayesUnfold}. Systematic uncertainties from various sources (effect of upstream material in front of the PMD, hadron and secondary photon contaminations, event generator dependence, unfolding method) were estimated and then added in quadrature. The total systematic uncertainty was found to be around 9--10\%~\cite{ALICE:PMDpPb}.

% Figure environment removed

\section{Results and discussion}

Figure~\ref{dndeta_MB} presents the pseudorapidity distributions (d$N_{\rm \gamma}$/d$\eta_{\rm \,lab}$) of inclusive photons in pp, p--Pb, and Pb--p collisions at $\sqrt{s_{\rm NN}}$~=~5.02~TeV measured within 2.3~$<~\eta_{\rm \,lab}~<$~3.9 together with the measurements of charged-particle multiplicities (d$N_{\rm ch}$/d$\eta_{\rm \,lab}$) at mid-pseudorapidity~\cite{ALICE:ChPrppMB,ALICE:ChPrpPbMB,CMS:ChPrpPbMB}. The data from pp and Pb--p collisions are reflected around $\eta_{\rm \,lab}$~=~0 to extend the measurements in the region, $-3.9<\eta_{\rm \,lab}<-2.3$. The d$N_{\rm \gamma}$/d$\eta_{\rm \,lab}$ at forward pseudorapidity smoothly matches with the d$N_{\rm ch}$/d$\eta_{\rm \,lab}$ at mid-pseudorapidity indicating that the production mechanisms for charged and neutral pions are similar. The predictions from various MC models are also displayed in Fig.~\ref{dndeta_MB} and show similar values for photon (solid lines) and charged-particle (dashed lines) multiplicities at forward and backward pseudorapidities, while at mid-pseudorapidity the d$N_{\rm \gamma}$/d$\eta_{\rm \,lab}$ differs from the d$N_{\rm ch}$/d$\eta_{\rm \,lab}$. This difference is due to a mass effect in the transformation between $\mathrm{d}N/\mathrm{d}y$ and $\mathrm{d}N/\mathrm{d}\eta$ at $\eta \approx 0$. Both HIJING (v1.36)~\cite{hijing} and DPMJET (v3.0-5)~\cite{dpmjet} event generators fairly describe the measured d$N_{\rm ch}$/d$\eta_{\rm \,lab}$ in p--Pb collisions. The DPMJET slightly underpredicts the d$N_{\rm \gamma}$/d$\eta_{\rm \,lab}$ in the p-going side and reproduces the same within uncertainties in the Pb-going side. For pp collisions, both EPOS LHC~\cite{eposlhc} and PYTHIA 8 (v8.243) with the Monash 2013~tune~\cite{pythia8_monash} overestimate the photon and charged-particle multiplicity.

% Figure environment removed

Figure~\ref{dndeta_cent} shows the pseudorapidity distributions of both photons and charged particles measured in p--Pb collisions for three multiplicity classes (0--5\%, 20--40\% and 80--100\%) determined with the CL1 (top panel) and ZNA (bottom panel) estimators. The particle density in the highest multiplicity class (0--5\%) when considering the CL1 (ZNA) estimator reaches values thrice (twice) as large as those in minimum bias p--Pb collisions. A clear asymmetric shape is observed for d$N_{\rm ch}$/d$\eta_{\rm \,lab}$ in the highest multiplicity class (0--5\%) and the shape becomes symmetric, like in pp, in the lowest multiplicity class (80--100\%). HIJING describes the d$N_{\rm \gamma}$/d$\eta_{\rm \,lab}$ at forward pseudorapidity within the measurement uncertainties. For 80--100\% event class, HIJING overestimates (underestimates) the d$N_{\rm ch}$/d$\eta_{\rm \,lab}$ for the CL1 (ZNA) estimator.

\section{Conclusion}

The pseudorapidity distributions of inclusive photons were measured over a kinematic region of 2.3~$<~\eta_{\rm \,lab}~<$~3.9 for minimum bias pp, p--Pb, and Pb--p collisions and for different multiplicity classes in p--Pb collisions at $\sqrt{s_{\rm NN}}$~=~5.02~TeV. The d$N_{\rm \gamma}$/d$\eta_{\rm \,lab}$ at forward pseudorapidity was observed to follow the trend of similar measurements of charged particles at mid-pseudorapidity. The predictions from various MC describe the data within 15--20\%. These results will help to establish baselines for the interpretation of Pb--Pb collision data.

\printbibliography


%%
%% The next two lines define the bibliography style to be used, and
%% the bibliography file.
\bibliographystyle{ACM-Reference-Format}
\bibliography{ref}

\end{document}