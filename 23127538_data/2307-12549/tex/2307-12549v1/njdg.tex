\section{HC-NJDG Data Collection}
\label{sec:njdg}


High Court National Judicial Data Grid (HC-NJDG) was launched in July 2017 \cite{njdg_gamechg, njdg_hc}. We started collecting data from the portal on August 31, 2017. The last data used in this paper was collected on March 22, 2020. We have collected data on 229 randomly chosen dates, spanning a period of more than two and a half years.



To provide a glimpse of the data, some of the statistics, as collected on August 20, 2018, are provided in Table \ref{tab:njdg_stats}. The portal has more statistics available but we have chosen to present only the ones that are relevant for this paper. The data related to the number of pending cases in all the high courts in India is shown. The table also presents data on the number of monthly disposed and filed cases. 
% The cases are divided into two different kinds. The rows in the table show the division of cases based on the age of the cases, i.e., how old are they. The columns show the division based on the types of the cases, i.e., whether the cases are civil, criminal or writs. Hence, on this date, more than 3.3 million cases were pending in the high courts in India.


\begin{table}[h]
\centering
\footnotesize
    \begin{tabular}{ | l | r | r | r | r |}
    \hline
    Title & Civil & Criminal & Writs & Total \\ \hline
    Pending Cases & 1506780 & 769754 & 1114448 & 3390982 \\ \hline
    Cased Filed (monthly) & 27663 & 42404 & 32009 & 102063 \\ \hline
    Cased Disposed (monthly) & 28080 & 47368 & 36548 & 111996 \\ \hline
    \end{tabular}
    \caption{Some of the statistics available on the HC-NJDG portal\cite{njdg_hc}.}
    \label{tab:njdg_stats}
\end{table}


% \begin{table}[h]
% \centering
% \footnotesize
%     \begin{tabular}{ | l | r | r | r | r |}
%     \hline
%     Cases Pending & Civil & Criminal & Writs & Total \\ \hline
%     Over 10 years & 371332 & 138120 & 142642 & 652094 \\ \hline
%     Between 5-10 years & 370192 & 176909 & 257987 & 805088 \\ \hline
%     Between 2-5 years & 419455 & 214270 & 372992 & 1006717 \\ \hline
%     Less than 2 years & 345801 & 240455 & 340827 & 927083 \\ \hline
%     Total & 1506780 & 769754 & 1114448 & 3390982  \\ \hline
%     \end{tabular}
%     \caption{HC-NJDG data of pending cases in High Courts as on August 20, 2018.}
%     \label{tab:njdg_pending}
% \end{table}

% It also shows the aggregate of the cases that were listed for hearing on that particular date in all the high courts. The columns, like the previous table, represent the type of cases. In this table, we have also included another field, the number of judges in the high courts. Note that there was an error in the number of high court judges reported on that day. According to HC-NJDG portal the number of high court judges in India was 810, whereas according to the vacancy positions document (dated August 01, 2018) available at the Department of Justice website \cite{doj_vacancy}, the number of working strength of high courts was 659 and the approved strength was 1079.

% \begin{table}[h]
% \centering
% \footnotesize
%     \begin{tabular}{ | l | r | r | r | r |}
%     \hline
%     
%         & Civil & Criminal & Writs & Total \\ \hline
%     Cases Filed (last month)& 27663 & 42404 & 32009 & 102063 \\ \hline
%     Cases listed (today) & 11656 & 11793 & 13013 & 36462 \\ \hline
%     Cases Disposed (last month) & 28080 & 47368 & 36548 & 111996 \\ \hline
%     Total Judges & & & &810 \\ \hline
%     \end{tabular}
%     \caption{Number of cases filed (monthly) and disposed (monthly) and the number of cases listed (daily) and the number of judges in the high courts in India as on August 20, 2018.}
%     \label{tab:njdg_monitor}
% \end{table}

% \end{comment}


% \subsection{Scrutinizing the data so collected}

% After the data collection, we had to make sure that the data being used does not suffer from errors made during the download or during saving the files. After a careful inspection, we deleted some of the files which had errors. Table \ref{tab:hcnjdg_details} lists the amount of data that we can actually use from the data collected during the said period. 

Four high courts, namely, Allahabad High Court, Gauhati High Court, High Court of Jammu and Kashmir and High Court of Madhya Pradesh have joined HC-NJDG after we started collecting the data. Hence, they appear fewer number of times. However, majority of the high courts -- 20 to be precise -- had their presence on HC-NJDG when we started collecting the data, i.e., on August 31, 2017. 
% Hence, different high courts may have data on different dates.  

The $25^{th}$ High Court for the state of Telangana was formed on January 01, 2019. For consistency with the previous data, we have continuted to consider Telangana and Andhra High Court as one in our analysis. For this reason, only 24 high courts appear in our study. Note that the last day for data collection for this paper was March 22, 2020, just before the nation-wide lockdown was announced in India due to Covid-19 pandemic. We hypothesize that data pre-lockdown and during the lockdown would be very different and hence not comparable for our work.


% \subsection{Understanding the graphs}

We have plotted many graphs in the paper. In order to maintain coherence and simplicity, and to have a reach to wider audience, we have restricted ourselves to only two kinds of graphs, as explained below. 
% \begin{enumerate}
\paragraph{Temporal data graphs (Dates on horizontal axis)} The horizontal axis (also referred to as X-axis in the paper), consists of dates beginning August 31, 2017 to March 22, 2020 from left to right. The title of each graph is present on the top stating the name of the high court that plot corresponds to. If on some date data was not collected, then data is not shown against that date but the date is still present on the X-axis in the all cases. \fref{fig:hcpendency} is an example of this kind of graph.

\paragraph{Spatial data graphs (High Courts on horizontal axis)} In these graphs, the data from the high courts is plotted. The horizontal axis, or X-axis, in these graphs have 24 points, each representing one of the 24 high courts. Y-axis plots the value of the considered parameter. If a high court does not have a valid data for that parameter, its name still appears on the X-axis but have no value on the Y-axis. The title of the graph is present at the top. \fref{fig:sanctioned_working} is an example of this kind of graph.

% Choice of semilog Y-axis for some curves is made to accommodate wide range of values in one graph. \fref{fig:pend_date_c} is an example of such use. The numbers 1 and 1000 and almost 10 lakh (= 1 Million) are clearly visible in the same graph which would be difficult to show if Y-axis were linear. 

% \end{enumerate}



% \begin{itemize}
% \item Compare with HC NJDG \cite{njdg_hc} \cite{njdg_gamechg}. Can we show that the cases are simply being transfered? 
% \item Draw the curve of the data that you have collected. 
% \item Infer the current scheduling.
% \end{itemize}


% % Figure environment removed

% We will also point out that the implementation of NJDG is not free of glitches of now. A huge discrepancy that we had observed was on September 11, 2017. On that day the number of cases pending with all the High Courts was noted to be 4,990,031, far more than the usual number around 3,300,000. It is for this reason we see an unusual dip in the \fref{fig:eff_hc}. 

\begin{comment}
\subsection{Number of Judges in High Courts}

% Figure environment removed

% Figure environment removed



\fref{fig:hcjudges} shows the aggregate number of judges in the high courts of India for each date on which the data has been collected. Analysis of this data compared with the vacancy document available on the Department of Justice website \cite{doj_vacancy} suggests that the HC-NJDG data on the number of judges in the high courts is erroneous. In fact, according to HC-NJDG, the number of high court judges was way more than the approved strength of 1079 during November and December 2018. In order to cross verify the data, we also started downloading the vacancy document as per the Department of Justice website starting June 2018. This clearly shows that the number of judges reported by HC-NJDG portal is approximately 300 more than the actual number. Moreover, the number of judges is monotonically increasing, which is not the case actually. Hence, we use the data on the number of judges from the vacancy document \cite{doj_vacancy}.


In \fref{fig:jud_date_j}, we plot the spatial data. We have plotted the number of judges in each high court as available on August 06, 2018. Note that the Y-axis in this plot is on log scale. The data is plotted in the descending order of the number of judges in each high court from HC-NJDG data. The reasons for choosing this date is mainly because it is closest to August 01 which is the date when vacancy document was updated by the Department of Justice. It can be clearly seen that in some cases the number of high court judges reported by HC-NJDG is even higher than the approved strength provided in the vacancy document for that high court. To be precise, on August 06, 2018, the number of judges in High Court of Gauhati, Calcutta, Orissa, Jammu and Kashmir, Himachal Pradesh and Manipur was higher than the number of approved strength for the respective high courts.  
\end{comment}
% \subsection{Pending Cases in High Courts}
% 
% % Figure environment removed
% 
% \fref{fig:hcpendency} shows the number of pending cases in all the high courts of India. We have plotted the total number of pending cases in the high courts in India as obtained from the NJDG portal in our data set. It can be clearly seen that the data has few continuous clusters and few sudden jumps. While initial sudden jumps can be explained by the fact that few high courts have joined NJDG late and they may be taking time to converge to report stable number, the overall graph does not represent a healthy update culture. It also implies, in some cases that the updates have been erroneous. For example, on September 11, 2017, there is a completely isolated peak with pending number of cases around 5 million whereas the adjacent dates on both sides have less than 3.5 million. While it is true that the error was quickly ratified, there should be attempts towards not introducing such errors in the first place. In particular, after more than one year of inception, the graph should have started looking like a ``smooth continuous function", which as of now, it doesn't. For around six months between January 2018 to July 2018 the number of pending cases in high courts was reported to be more than 4.3 million which suddenly dropped down to 3.3 million in August 2018. Such sudden jumps need explanations to be reliably used for any practical significance. The reasons for such jumps must be investigated so that such errors do not get repeated on the portal.
% 
% % Figure environment removed
% 
% 
% \fref{fig:pend_date_c} shows the average of pending cases during the data collection period for all the high courts. Note that the Y-axis is in log scale and the data is sorted in descending order of the total number of pending cases at a high court. Hence, the first place is occupied by Allahabad High Court that has more than 700 thousand cases pending. The figure also implies that most of the high courts have huge number of pending cases except Sikkim High Court and newly established high courts for the states of Meghalaya and Tripura. It is also worth noting from this figure that Sikkim High Court has no cases that are pending for more than 10 years. Meghalaya and Tripura High Court have less than twenty cases pending for more than ten years. The rest of the high courts are having the ten plus years pending cases as a substantial percentage of their total pendency. 


% 
% % Figure environment removed

% \fref{fig:yearwise} shows the pendency in terms of the age of cases. From this data, we can see that the number of cases with age greater than five have mainly increased during last one year. However, the number of cases with 10+ years age, suddenly reduced from 9,95,031 on August 06, 2018 to 6,52,180 on August 13, 2018. A similar drop in number is seen for the cases pending for 5-10 years as well. The reasons for such an extraordinary decrease are unknown.

% A detailed study of pending cases in individual high courts is done in Section \ref{sec:pending_hc}.



% \subsection{Disposed, filed and listed cases}

% From the previous subsection, we understand that the average load on the judges of high courts is huge in terms of the number of pending cases. Now we see the rate of disposal, filing and listing of cases that can help us provide upper bounds on the time required to get rid of pendency. Unfortunately, NJDG as of now, updates data on monthly disposal and monthly filing of cases. The number of cases listed is provided on a daily basis. If the other two parameters could also be provided on a daily basis then our analysis and interpretations would be more accurate.

% \fref{fig:fdl_date} shows the data corresponding to the number of disposed, filed and listed cases as available on NJDG on August 06, 2018. The number of cases disposed and filed are provided monthly. The plot is ordered in descending values of the cases disposed. Also note that the Y-axis is on linear scale and still there is a very little difference between the number of cases listed daily and the number of cases disposed monthly. This may mean that the number of listed cases in a day may be reduced, which will be for the benefit of all the stakeholders. The judges will have more time to hear a case unlike now \cite{daksh_report}. The litigants will be better off because the chances of hearing their case will increase. The advocates will save time in appearing for the cases and hence get more time to prepare for cases. The court staff will have lesser files to move and manage.

% While we are not embracing the idea of using NJDG data of just one day, the above figure captures the gap between the three kinds of statistics that are closely related to piling up of pendency. The point here is that the causlists in the high courts should be prepared more scientifically. We back up our assertion by a more careful and rigorous analysis involving more data.



% The following equation governs the pendency of cases in ideal case when the updates take place every evening. 
% \begin{equation}
% P_c = P_m + f - d
% \end{equation}
% where, $P_c$ is the number of closing pending cases, $P_m$ is the number of cases pending in the morning, $f$ is the number of cases filed in a day and $d$ is the number of cases disposed in a day.
% 

% % Figure environment removed
% 
% It is shown in \fref{fig:fdl_hcall} that there is a huge gap between the number of cases listed and the number of cases disposed in a day. The number of cases disposed in a day is obtained by dividing the monthly statistics by 22 because we are assuming that there are 22 working days in a month. The goal should be to minimize the gap between the number of cases disposed and the number of cases listed on any given day. This would mean that most of the cases that are listed, should be disposed, rather than adjourned. This is the parameter that makes us claim that there is a room for more scientific preparation of causelists as there is a room for decreasing the gap between the number of cases listed and the number of cases disposed in a day. Similar is the case for individual high courts as presented in \fref{fig:fdl_hc}.
% 
% % Figure environment removed

% \fref{fig:fdl10_hcall} shows the number of disposed cases whose age was more than ten years. The graph is not expected to show any trend as the number of cases disposed depends on many circumstances. The only thing that can be reasonably concluded is that the number of disposal of more than ten year old cases was high during February 2018 to March 2018. It has decreased since then.
%However, one thing can be inferred is that the causelists preparations are indifferent to whether a case is ten years old or not. 

% % Figure environment removed





