\documentclass[10pt,conference,letterpaper]{IEEEtran}                    % onecolumn (standard format)
%\documentclass[smallcondensed]{svjour3}     % onecolumn (ditto)
% \documentclass[smallextended]{svjour3}       % onecolumn (second format)
%\documentclass[twocolumn]{svjour3}          % twocolumn
%
% \smartqed  % flush right qed marks, e.g. at end of proof
%
\usepackage{graphicx}
\usepackage{verbatim}
% \usepackage{amsmath}
\usepackage{multicol}
\usepackage{cases}
% \usepackage[final]{changes}
\usepackage{mathptmx}
\usepackage{setspace}
\usepackage{cite}
\usepackage{url}
\usepackage{hyperref}
\usepackage{tfrupee}
\usepackage{soul}
\usepackage{array,etoolbox}
\usepackage{spreadtab}
\usepackage{multirow}
\usepackage{subcaption}

\def\UrlBreaks{\do\/\do-}
\preto\tabular{\setcounter{magicrownumbers}{0}}
\newcounter{magicrownumbers}
\def\rownumber{}
% \normalem
\IEEEoverridecommandlockouts

% \usepackage{natbib}
%
% \usepackage{mathptmx}      % use Times fonts if available on your TeX system
%
% insert here the call for the packages your document requires
%\usepackage{latexsym}
% etc.
%
% please place your own definitions here and don't use \def but
% \newcommand{}{}
%
% Insert the name of "your journal" with
% \journalname{myjournal}
%


\newtheorem{observation}{Observation}

\def\hb{\hbox to 10.7 cm{}}
\newcommand{\fref}[1]{Fig.~\ref{#1}}
\newcommand{\sref}[1]{Section~\ref{#1}}
\newcommand{\cref}[1]{Chapter~\ref{#1}}
\newcommand{\coref}[1]{Corollary~\ref{#1}}
\newcommand{\tref}[1]{Theorem~\ref{#1}}
\newcommand{\lref}[1]{Lemma~\ref{#1}}
\newcommand{\junk}[1]{}

\begin{document}

\title{Estimating Time to Clear Pendency of Cases in High Courts in India using Linear Regression\thanks{Note: As of January 01, 2019, there are 25 high courts in India, however, for consistency with the earlier data, we have merged the High Court of Andhra Pradesh and the High Court of Telangana as Telangana and Andhra High Court.}}

% \thanks{{\bf Disclaimer: } The data is taken from HC-NJDG and may not reflect the actual status of the statistics in the Hon'ble High Courts. Due care has been taken in the analysis but some errors may still be there. Kindly notify the author in case of discovery of such errors. The reference to the Hon'ble High Courts should be construed as the reference being made for the department responsible for updates on HC-NJDG.}

% \author{\IEEEauthorblockN{Kshitiz Verma}
% \IEEEauthorblockA{LNMIIT Jaipur, India\\
% Email: vermasharp@gmail.com}
% }
% 
\author{\IEEEauthorblockN{Kshitiz Verma\IEEEauthorrefmark{1},
Anshu Musaddi\IEEEauthorrefmark{2},
Ansh Mittal\IEEEauthorrefmark{2},
Anshul Jain\IEEEauthorrefmark{2}}
\IEEEauthorblockA{kshitiz.verma@jklu.edu.in, \{17ucs185, 17ucs028, 17ucs029\}@lnmiit.ac.in}
\IEEEauthorblockA{\IEEEauthorrefmark{1} JK Lakshmipat University, Jaipur, India}
\IEEEauthorblockA{\IEEEauthorrefmark{2} The LNM Institute of Information Technology, Jaipur, India}
}
\thispagestyle{plain}
\pagestyle{plain}


\maketitle
\begin{abstract}

Indian Judiciary is suffering from burden of millions of cases that are lying pending in its courts at all the levels. The High Court National Judicial Data Grid (HC-NJDG) indexes all the cases pending in the high courts and publishes the data publicly. In this paper, we analyze the data that we have collected from the HC-NJDG portal on 229 randomly chosen days between August 31, 2017 to March 22, 2020, including these dates. Thus, the data analyzed in the paper spans a period of more than two and a half years. We show that:
\begin{itemize}
\item the pending cases in most of the high courts is increasing linearly with time.
\item the case load on judges in various high courts is very unevenly distributed, making judges of some high courts hundred times more loaded than others.
\item for some high courts it may take even a hundred years to clear the pendency cases if proper measures are not taken.
\end{itemize}
We also suggest some policy changes that may help clear the pendency within a fixed time of either five or fifteen years. Finally, we find that the rate of institution of cases in high courts can be easily handled by the current sanctioned strength. However, extra judges are needed only to clear earlier backlogs.
% \begin{enumerate}
% \item pending cases in most of the high courts are increasing linearly with time (Fig. \ref{fig:hcpendency}, \ref{fig:np_hc1}, \ref{fig:np_hc2_p2}).
% \item in last few months, the data on HC-NJDG is far more reliable compared to earlier trends (\fref{fig:np_hc1}).
% \item there is a huge difference in terms of average load of cases on judges of different high courts (\fref{fig:pend_date_r}).
% \item if all the high courts operate at their approved strength of judges, then for most of the high courts pendency can be nullified within 20 years from now (Fig. \ref{fig:years_combat_real}, \ref{fig:years_combat_national_avg_lowest}).
% \item the pending cases filed by women and senior citizens are disproportionately low, they together constitute less than 10\% of the total pending cases (\fref{fig:womsen_hc_date} - \ref{fig:wom_hc})
% \item a better scheduling process for preparing causelists in courts can help reducing the number of pending cases in the High Courts (\fref{fig:fdl_hc}).
% \item some statistics are not well defined (\fref{fig:po_hc3}).
% \end{enumerate}
 
% We motivate the problem of pendency in detail and emphasize the need of designing efficient algorithms for scheduling cases in High Courts in India so that the number of pending cases in the High Courts reduce at a satisfactory rate.

% \keywords{Indian Judiciary, Pending Cases, Optimal Scheduling, National Judicial Data Grid, e-Courts Project}
\end{abstract}

\section{Introduction}
Current quantum hardware is unable to carry out universal quantum computations due to the buildup of errors that occur during the computation. 
The magnitude of the individual error is currently above the value that the Threshold Theorem requires in order to kick-start quantum error correction and fault-tolerant quantum computation~\cite[Section 10.6]{nielsen_chuang_2010}. 
Although the experimentally achieved fidelity rates are promising and the error bounds are inching closer to the required threshold, we will have to work for the foreseeable future with quantum hardware with errors that build-up during the computation.  This implies that we can only do a limited number of steps before the output of the computation has become completely uncorrelated with the intended one.

For fault-tolerant quantum computing, we repeat four steps: 
1) We apply a number of single and two-qubit quantum gates, in parallel whenever possible; 
2) We perform a syndrome measurement on a subset of the qubits; 
3) We perform fast classical computations to determine which errors have occurred and how to correct them; 
and, 4) We apply correction terms based on the classical computations.
We then repeat these four steps with a next sequence of gates. 
These four steps are essential to fault-tolerant quantum computing. 


The starting point of this work is to use the four steps outlined above, not to carry out error correction and fault-tolerant computation, but to enhance short, constant-depth, {\em uncorrected} quantum circuits that perform single qubit gates and {\em nearest-neighbor} two qubit gates. 
Since in the long run we will have to implement error-correction and fault-tolerant computation anyhow, and this is done by such a four-step process, why not make other use of this architecture? Moreover, on some of the quantum hardware platforms, these operations are already in place.
Embracing this idea we naturally arrive at the question: what is the computational power of \textit{low-depth} quantum-classical circuits organized as in the four steps outlined above? 
We thus investigate circuits that execute a small, ideally constant, number of stages, where at each stage we may apply, in parallel, single qubit gates and {\em nearest-neighbor} two qubit gates, followed by measurements, followed by low-depth classical computations of which the outcome can control quantum gates in later stages. 
It is not clear, at first, whether such circuits, especially with constant depth, can do anything remotely useful. 
But we will see that this is indeed the case: many quantum computations can be done by such circuits in constant depth. 
By parallelizing quantum computations in this way, we improve the overall computational capabilities of these circuits, as we do not incur errors on qubits that are idle, simply because qubits are not idle for a very long time. 
Furthermore, reducing the depth of quantum circuits, at the cost of increasing width, allows the circuit to be run faster even if errors occur.

The first usage of such a four-step layout, not to do error correction, but to perform computations, can be found in the paradigm of measurement-based quantum computing~\cite{gottesman1999demonstrating,raussendorf2001one,jozsa2006introduction,clark2007generalised}: 
A universal form of quantum computing where a quantum state is prepared and operations are performed by measuring qubits in different bases, depending on previous measurements and intermediate measurements.

\citeauthor{PhamSvore2013} were the first to formalize the four-step protocol for performing computations~\cite{PhamSvore2013}. They included specific hardware topologies by considering two-dimensional graphs for imposing constraints on qubit interactions. In their model, they develop circuits for particularly useful multi-qubit gates, including specifying costs in the width, number of qubits, depth, number of concurrent time steps, size, and total number of non-Identity operations.
As a result, they find an algorithm that factors integers in polylogarithmic depth.
\citeauthor{Browne:2011} showed that the main tool in the work by \citeauthor{PhamSvore2013}, the fan-out gate, can also be replaced by additional log-depth classical computations in the measurement-based quantum computing setting~\cite{Browne:2011}.

More recently, \citeauthor{Cirac:2021} introduced a scheme to implement unitary operations involving quantum circuits combined with Local Operations and Classical Communication ($\mathsf{LOCC}$) channels: $\mathsf{LOCC}$-assisted quantum circuits~\cite{Cirac:2021}. Similarly to the four-step scheme we just described, they allow for a short depth circuit to be run on the qubits, followed by one round of $\mathsf{LOCC}$, in which ancilla qubits are measured and local unitaries are applied based on the measurement outcomes. They show that in this model any 1D transitionally invariant matrix-product state (MPS) with fixed bond dimension is in the same phase of matter as the trivial state. Similar ideas can be found in~\cite{TVV_NonAbelianTopologicalOrder_2022, tantivasadakarn2021long}.

In this work, we introduce a new model, called \textit{Local Alternating Quantum-Classical Computations} ($\LAQCC$). In this model we alternate between running quantum circuits (constrained by locality), ending in the measurement of a subset of qubits, and fast classical computations based on the measurement results. The outcome of the classical computations are then used to control future quantum circuits. We allow for flexibility in this model, by giving different constraints to the power of both the quantum circuits and the classical circuits as well as the number of alternations between them. 
Most attention will be given to $\LAQCC$ containing quantum circuits of constant depth, classical circuits of logarithmic depth and at most a constant number of alternations between them. 
Any circuit constructed in this model is considered to be of constant depth. 
We restrict ourselves to logarithmic depth classical computations, as this is the first natural and non-trivial extension beyond constant-depth classical computations. 
Constant-depth classical computations do however also have an equivalent constant-depth quantum implementation.

The definition of $\LAQCC$ sharpens the original definition of \citeauthor{PhamSvore2013} by adding constraints to the intermediate classical computations. This allows us to bound the power of $\LAQCC$ from above. 

The main result of \citeauthor{Cirac:2021}, that 1D translational invariant MPS with fixed bond dimension can be prepared by $\mathsf{LOCC}$-assisted circuits, relies on local symmetries of the MPS. These symmetries allow them to prepare local states (on a constant number of qubits) and glue them together by doing one round of the appropriate entangling measurement and corrections, after which they run a round of local unitaries to get the desired result. This general scheme for preparing states that exhibit an MPS description with the appropriate local symmetries requires only geometrically local unitaries and one round of measurement and corrections an therefore is accessible in $\LAQCC$. Studying different local symmetries, known as Symmetry Protected Topological (SPT) phases of matter, to find measurement-based constant depth circuits for states is a broad ongoing field of research~\cite{TVV_NonAbelianTopologicalOrder_2022, tantivasadakarn2021long, smith2023deterministic}. 
All these schemes have a $\LAQCC$ implementation.

%$\LAQCC$-circuits also exist for general schemes of preparing local states, based on the local tensors, and gluing them together using one round of entangled measurement and corrections, based on the local symmetry. 
%The main result of \citeauthor{Cirac:2021}, that 1D translational invariant MPS with fixed bond dimension can be prepared by $\mathsf{LOCC}$-assisted circuits, relies heavily on local symmetries of the MPS and as a result also has an equivalent $\LAQCC$ implementation. 
%The corrections applied after the measurement round are local unitaries depending on the local symmetries of the MPS. 

 

%This general scheme of preparing local states, based on the local tensors, and gluing it together by doing one round of entangled measurement and corrections, based on the local symmetry, is accessible in $\LAQCC$.
Note however that \citeauthor{Cirac:2021} also suggest a circuit for the $W$-state.
This circuit uses sequentially and dependent measurement-based corrections of the ancilla qubits. 
These dependent measurements translate to sequential alternations between the quantum and classical circuits and therefore increase the total depth to linear depth, exceeding the constant-depth constraints imposed by $\LAQCC$-circuits. 

We study the power of the $\LAQCC$ model with respect to state preparation, showing that even with only constant quantum-depth and logarithmic classical depth it remains possible to prepare states with long-range entanglement.
Another surprising result is that it is unlikely that $\LAQCC$ circuits are classically simulatable. We show that any instantaneous quantum polynomial-time (IQP) circuit~\cite{Bremner2010,Shepherd2009} has an $\LAQCC$ implementation.
Classical simulation of IQP circuits implies the collapse of the polynomial hierarchy to the third level, which is not believed to be true~\cite{Bremner2017}. Therefore, we expect that $\LAQCC$ circuits are unlikely to be classically simulatable. We bound the power of $\LAQCC$ by showing that it is contained in $\QNC^1$, the class of polynomial-size, log-depth circuits.

Next, we also study the power that intermediate classical calculations can add to quantum computations, by considering a new model that alternates between polynomially many polynomial-depth quantum circuits and unbounded classical computations
We study this model by doing a complexity theoretical analysis, where we draw inspiration from the notions of complexity given by \citeauthor{RosenthalYuen:2022}, \citeauthor{MetgerYuen:2023}, and \citeauthor{Aaronson:2004}.
All three complexity notions are based on the notion of state preparation, instead of more traditional definition of complexity such as the decidability of a computational problem. 
The first two consider classes based on sequences of quantum states preparable by a polynomial-sized quantum circuit, where the circuits are uniformly generated by a computational class, for instance, the class $\mathsf{PSPACE}$, which results in the complexity class $\mathsf{StatePSPACE}$~\cite{RosenthalYuen:2022,MetgerYuen:2023}.
The third notion considers a relative complexity, where the complexity is measured between two given states, and is measured by the number of gates, from a given gate-set, required to transform one state in another state~\cite{Aaronson:2004}. 
For our definition of state preparation complexity, we drop the uniformity constraint from~\cite{RosenthalYuen:2022,MetgerYuen:2023} and define a class as $\mathsf{StateX}$, which refers to states preparable by circuits of type $\mathsf{X}$. 
As an example, if $\mathsf{X} = \QNC^0$, this results in the class $\mathsf{StateQNC^0}$, which is the set of states preparable from the $\ket{0}^n$ state by poly-size constant-depth circuits. 
This notion is similar to the relative complexity from~\cite{Aaronson:2004}, where one state is the  $\ket{0}^n$ state and instead of counting the number of gates we consider the set of states preparable by a fixed number of gates. Using this notion of complexity we show that any state preparable by an $\LAQCC^*$ circuit is also preparable by a $\mathsf{PostQPoly}$ circuit, the class of circuits of polynomial depth with an additional post-selection gate. 

All Clifford circuits have a constant-depth $\LAQCC$ implementation, implying that any stabilizer state can be implemented by a constant-depth $\LAQCC$ circuit, see Section~\ref{sec:clifford_circuits} for a proof of this statement. 
Efficient circuits for stabilizer states have been known already through measurement-based quantum computing. Therefore this paper focuses on the preparation of non-stabilizer states, and as a surprising result we find novel constant-depth protocols for four very natural classes of non-stabilizer states.
Despite the extensive research into these four classes of non-stabilizer states and the many applications of them, no efficient constant- or low-depth state preparation protocols are known yet. We specifically consider these four classes as they are all often used as initial states in other algorithms.

The first state is a uniform superposition over an arbitrary number of states. 
This state finds applications in many quantum algorithms, as they often start with a uniform superposition over multiple states. 
This superposition is often achieved by applying Hadamard gates to every qubit due to its simplicity to prepare. 
Yet, the analysis of many algorithms, such as Shor's algorithm~\cite{Shor:1997}, would benefit from a different initial superposition. 
The circuit to prepare the uniform superposition over an arbitrary number of states uses an exact version of Grover search as a subroutine, that turns a probabilistic circuit, with a known constant probability of success, into a deterministic circuit. 
We use the circuit for preparing a uniform superposition over an arbitrary number of states as a subroutine in the next two quantum state preparation protocols. 

The second state is the $W$-state, the uniform superposition over all computational basis states of Hamming-weight~$1$, a natural long-ranged entangled state that displays a fundamentally nonequivalent type of entanglement from the Greenberger–Horne–Zeilinger state~\cite{WState:2000}, for which $\LAQCC$-type constant-depth circuits were previously known~\cite{PhamSvore2013, Cirac:2021}. 
The $W$-state is often used as benchmark for new quantum hardware~\cite{Haffner2005,Neeley2010,GarciaPerez:2021}. 
A novel way to prepare the $W$-state therefore gives a new way to benchmark different quantum devices with each other. 
A circuit for preparing the $W$-state was given in~\cite{Cirac:2021}, but this implementation requires sequentially alternating measurements followed by local unitaries, which in the $\LAQCC$ model is not considered to be of constant depth. 
We improve this protocol by giving an $\LAQCC$ implementation of the $W$-state, based on a compress-uncompress method that links the one-hot and binary encoding of integers.

The third state considered is the Dicke state, a generalization of the $W$-state, a superposition over all computational basis states with Hamming-weight $k$~\cite{Dicke:1954}. 
Dicke states have relevance in various practical settings.
For instance, for quantum game theory~\cite{zdemir2007}, quantum storage~\cite{Bacon_Compress:2006,Plesch:2010}, quantum error correction~\cite{ouyang2014permutation}, quantum metrology~\cite{toth2012multipartite}, and quantum networking~\cite{prevedel2009experimental}. 
Dicke states have been used as a starting state for variational optimization algorithms, most notably Quantum Alternating Operator Ansatz (QAOA)~\cite{Hadfield2019}, to find solutions to problems such as Maximum k-vertex Cover~\cite{Brandhofer2022,cook2020quantum}.
The ground states of physical Hamiltonians describing one-dimensional chains tend to show a resemblance to Dicke states such as states resulting from the Bethe ansatz, making them an ideal starting state when investigating the ground state behavior of these Hamiltonians~\cite{TDL_BetheAnsatzDerivation:2010,B_ExcitedStateQuantumPhaseTransitions:2013,DickeTransitions:2021}. 
For instance, the algorithm by \citeauthor{van2021preparing}, who give an algorithm to prepare the Bethe ansatz eigenstates of the spin-1/2 XXZ spin chain, starts by first preparing a Dicke state~\cite{van2021preparing}. 
A Dicke-state preparation protocol based on the compress-uncompress methodology used in the $W$-state furthermore finds applications in entanglement distillation, where the entanglement of a large state is concentrated on only a few qubits. 
Efficient deterministic circuits for preparing Dicke states have been proposed by \citeauthor{bartschi2019deterministic}~\cite{bartschi2019deterministic, bartschi2022deterministic_short_depth}. 
They provide a quantum circuit of depth $\mathO(k \log(\frac{n}{k}))$, allowing arbitrary connectivity, to prepare a Dicke state, which they conjecture to be optimal when $k$ is constant. 
In this work, we provide a constant-depth $\LAQCC$ circuit below their conjectured bound already for constant $k$. 
However, this does not directly disprove their conjecture, as we allow for intermediate measurements and classical computations. 
More significantly, we even construct constant-depth $\LAQCC$ circuits for $k = \mathO(\sqrt{n})$ greatly improving their bound.
This construction extends the compress-uncompress method for the $W$-state combined with additional subroutines. 

We continue with a log-depth state preparation protocol for the Dicke-state for arbitrary $k$. 
This protocol implements an efficient transformation between the factoradic number representation and the combinatorial number representation of a positive integer. 
The combinatorial number representation relates directly to the Dicke state. 
The provided efficient transformation between number representation systems might be of independent interest. 

We conclude by modifying our protocol for preparing a Dicke-state to a protocol that prepares quantum many-body scar states in constant-depth. 
These states have low entanglement and longer coherence times than states with similar energy density.
These characteristics make many-body scar states interesting to analyze and relevant within physics.
Many-body scar states appear for instance in the AKLT model~\cite{AKLT:1987,MRBAR:2018,MRB:2018} and different spin models~\cite{SI:2019,MOBFR:2020}.
Known methods for preparing these states have polynomial-depth~\cite{Gustafson:2023}, whereas our circuit has constant depth. 

% We conclude by studying the power that intermediate classical calculations can add to quantum computations. 
% In this study, we define a new model that relaxes constant-depth quantum circuits to polynomial depth quantum circuits, log-depth classical calculations to unbounded classical computations and a constant number of alternations to a polynomial number of alternations. 
% We call this model $\LAQCC^*$. 
% We study this model by doing a complexity theoretical analysis, where we draw inspiration from the notions of complexity given by \citeauthor{RosenthalYuen:2022}, \citeauthor{MetgerYuen:2023}, and \citeauthor{Aaronson:2004}.
% All three complexity notions are based on the notion of state preparation, instead of more traditional definition of complexity such as the decidability of a computational problem. 
% The first two consider classes based on sequences of quantum states preparable by a polynomial-sized quantum circuit, where the circuits are uniformly generated by a computational class, for instance, the class $\mathsf{PSPACE}$, which results in the complexity class $\mathsf{StatePSPACE}$~\cite{RosenthalYuen:2022,MetgerYuen:2023}.
% The third notion considers a relative complexity, where the complexity is measured between two given states, and is measured by the number of gates, from a given gate-set, required to transform one state in another state~\cite{Aaronson:2004}. 
% For our definition of state preparation complexity, we drop the uniformity constraint from~\cite{RosenthalYuen:2022,MetgerYuen:2023} and define a class as $\mathsf{StateX}$, which refers to states preparable by circuits of type $\mathsf{X}$. 
% As an example, if $\mathsf{X} = \QNC^0$, this results in the class $\mathsf{StateQNC^0}$, which is the set of states preparable from the $\ket{0}^n$ state by poly-size constant-depth circuits. 
% This notion is similar to the relative complexity from~\cite{Aaronson:2004}, where one state is the  $\ket{0}^n$ state and instead of counting the number of gates we consider the set of states preparable by a fixed number of gates. Using this notion of complexity we show that any state preparable by an $\LAQCC^*$ circuit is also preparable by a $\mathsf{PostQPoly}$ circuit, the class of circuits of polynomial depth with an additional post-selection gate. 

\paragraph{Summary of results}
\begin{itemize}
    \item We give a new definition of a computational model that captures the power of the four step process: applying a constant number of layers of one- and two-qubit gates; performing a syndrome measurement; perform a fast classical computation determining corrections; apply corrections. We call this model \emph{Local Alternating Quantum Classical Computations}, or $\LAQCC$ for short. In this model we bound the allowed quantum operations, intermediate classical calculations, and number of rounds separately. In Section~\ref{sec:LAQCC_model} we define this model and give a list of operations based on results from literature contained in this computational model. In some of these operations we explicitly use that we allow for multiple, but at most constant, rounds  of corrections.
    \item  We show show that there exist $\LAQCC$ circuits that can not be weakly simulated in Section~\ref{sec:IQP_in_LAQCC}. We further show that for every $\LAQCC$ circuit there exists a $\QNC^1$ circuit simulating it perfectly, in Section~\ref{sec:LAQCC_in_QNC1}.
    \item We introduce a new type computational complexity for preparing states and show that the extension of $\LAQCC$ where we allow a polynomial number of rounds and unbounded classical computation, is contained in $\mathsf{PostQPoly}$, the class of polynomial circuits with post-selection, in Section~\ref{sec:Complexity results}.
    \item We show a protocol to prepare the uniform superposition state of size $q$ in $\LAQCC$ using $\mathO(\ceil{\log_2(q)}^2)$ qubits in Section~\ref{sec:superposition_modulo_q}. 
    \item We show a protocol to prepare the $W_n$ state in $\LAQCC$ using $\mathO(n\log(n))$ qubits in Section~\ref{sec:W_state_in_LAQCC}.
    \item We show two ways of preparing the Dicke-$(n,k)$ state. The first method is in $\LAQCC$, works up to $k = \mathO(\sqrt{n})$, uses $\mathO(n^2\log(n))$ qubits, and is found in Section~\ref{sec:dicke:small_k}. The second method is in $\LAQCC\text{-}\mathsf{LOG}$ (an extension of $\LAQCC$ allowing for logarithmic number of alterations instead of constant), works for any $k$, uses $\mathO(\text{poly}(n))$ qubits, and is found in Section~\ref{sec:Dicke_in_LAQCC_LOG}. 
    \item We extend on our $\LAQCC$ method of generating Dicke-$(n,k)$ states for $k = \mathO(\sqrt{n})$ and show a protocol to generate many-body scar states for a particular Hamiltonian in $\LAQCC$ (Section~\ref{sec:many_body_scar}). 
\end{itemize}
Summarized in a table, we provide the following state generation protocols:
\begin{table}[htb]
\centering
\begin{tabular}{l|l|l|l}
\textbf{State description} & \textbf{Width} & \textbf{Depth} & \textbf{Implementation}\\
\hline 
Uniform superposition mod $q$: $\frac{1}{\sqrt{q}} \sum_{i = 0}^{q-1}\ket{i}$ & $\mathO(\ceil{\log^2 q})$ & $\mathO(1)$ & Section~\ref{sec:superposition_modulo_q}\\

$W$-state: $\frac{1}{\sqrt{n}}\sum_{i = 0}^{n-1}\ket{e_i}$ & $\mathO(n \log n)$ & $\mathO(1)$ & Section~\ref{sec:W_state_in_LAQCC}\\

Dicke-$(n,k)$, $k = \mathO(\sqrt{n})$: $\binom{n}{k}^{-1/2}\sum_{x \in \{0,1\}^n: |x| = k} \ket{x}$ &  $\mathO(n^2\log n)$ & $\mathO(1)$ 
&Section~\ref{sec:dicke:small_k}\\

Dicke-$(n,k)$: $\binom{n}{k}^{-1/2}\sum_{x \in \{0,1\}^n: |x| = k} \ket{x}$ & $\mathO(\text{poly}(n))$ & $\mathO(\log n)$ &Section~\ref{sec:Dicke_in_LAQCC_LOG}\\

QMBS: $\ket{S_k} = \frac{1}{k! \sqrt{\mathcal N(n,k)}}(Q^\dagger)^k \ket{\Omega}$ &  $\mathO(n^2\log n)$ & $\mathO(1)$  &  Section~\ref{sec:many_body_scar}
\end{tabular}
\caption{Summary of state preparation protocols given in this paper.}
\label{tab:sate_prep}
\end{table}
In the entry for the quantum many-body scar state $Q$ denotes the raising operator and $\mathcal N(n,k)=\binom{n-k-1}{k}$. 
Section~\ref{sec:many_body_scar} will provide more details on the variables and the implementation. 

\paragraph{Organization of the paper}
\noindent We first introduce relevant preliminaries in Section~\ref{sec:preliminaries}. 
In Section~\ref{sec:LAQCC_model} we formally define the class of Local Alternating Quantum-Classical Computations ($\LAQCC$). We also show that any Clifford circuit can be implemented in constant depth $\LAQCC$ (a result based on a result from measurement-based quantum computing~\cite{jozsa2006introduction}). 
This result allows us to give many useful multi-qubit gates and routines in Section~\ref{sec:gates_created_in_LAQCC}. 
Beyond that we show that constant depth $\LAQCC$ circuits are contained in $\QNC^1$ and that any $\mathsf{IQP}$ circuit has an $\LAQCC$ implementation.
We conclude this section with an analysis of a more powerful instantiation of $\LAQCC$ and show an inclusion with respect to the class $\mathsf{PostQPoly}$, which is the class of circuits of polynomial depth with one additional post-selection gate. 
In Section~\ref{sec:state_prep_in_LAQCC} we give $\LAQCC$ circuit implementations for preparing the uniform superposition over an arbitrary number of states, the $W$-state and the Dicke state up to $k = \mathO(\sqrt{n})$. We furthermore give a log-depth circuit implementation for preparing the Dicke state for any $k$. We conclude by showing a $\LAQCC$ circuit for generating many body scar states of a particular type of Hamiltonian.


\paragraph{Unlearning.}

The naive approach to machine unlearning is to retrain a model from scratch with each data deletion request. However, retraining is not feasible for companies with many large models or organizations with limited resources. Thus, the primary objective of machine unlearning is to provide efficient approximations to retraining. Early approaches in security and privacy attempt to achieve exact removal, where an unlearned model is identical to retraining, but are limited in model class~\citep{caoMakingSystemsForget2015, ginartMakingAIForget2019}. \citet{bourtouleMachineUnlearning2020} propose SISA, a flexible approach to exact unlearning that ``shards" a dataset, dividing it and training an ensemble of models where each can be retrained separately. More recent approaches propose approximate removal, requiring the unlearned model to be ``close" to the output of retraining. Some approximate removal methods focus on improving efficiency~\citep{wuDeltaGradRapidRetraining2020} and others try to preserve performance~\citep{wuPUMAPerformanceUnchanged2022}. While these methods apply to a large class of models, they have no formal guarantees on data removal. A second group of approximate approaches provide theoretical guarantees on the statistical indistinguishability of unlearned and retrained models. These noise-based methods leverage convex loss functions to guarantee unlearning with gradient updates~\citep{neelDescenttoDeleteGradientBasedMethods2020a} and Hessian methods~\citep{guoCertifiedDataRemoval2020, sekhariRememberWhatYou, izzoApproximateDataDeletion}. We augment this second set of approximate methods to simultaneously provide strong guarantees on data protection and preserve fairness performance while targeting a common class of models.

\paragraph{Fairness.}

There are a multitude of definitions for fairness in machine learning, such as individual fairness, multicalibration or multiaccuracy, and group fairness. Individual fairness~\citep{dwork2012fairness} posits that ``similar individuals should be treated similarly" by a model. 
On the other hand, recent work has focused on multicalibration and multiaccuracy~\citep{hebert2018multicalibration, kearns2018preventing, deng2023happymap}, where predictions are required to be calibrated across subpopulations. These subpopulation definitions can be highly expressive, containing many intersectional identities from protected groups. In this work, however, we focus on the most commonly studied form of fairness, group fairness, which seeks to balance certain statistical metrics across predefined subgroups. Group fairness literature has proposed various definitions of fairness, but the three most common definitions are Demographic Parity~\citep{zafar2017fairness, feldman2015certifying, zliobaite2015relation, calders2009building}, Equalized Odds, and Equality of Opportunity~\citep{hardt2016equality}. To achieve these definitions, there are generally three approaches to achieving group fairness: \emph{preprocessing} which attempts to correct dataset imbalance to ensure fairness~\citep{calmon2017optimized}, \emph{in-processing} which occurs during training by modifying traditional empirical risk minimization objectives to include fairness constraints~\citep{lowyStochasticOptimizationFramework2022, berk2017convex, agarwal2018reductions, martinez2020minimax}, and \emph{postprocessing} which modifies predictions to ensure fair treatment~\citep{alghamdi2022beyond, hardt2016equality}. 
In this work we focus on in-processing algorithms because they simply modify an objective to account for fairness rather than requiring an additional operation before or after each unlearning request which would also have to be made unlearnable.

\paragraph{Intersections.} Despite advancements in machine unlearning, the literature still lacks sufficient consideration of the downstream impacts of unlearning methods. While recent papers have explored the compatibility of the right to be forgotten with the right to explanation~\citep{krishna2023towards}, there is little work at the intersection of unlearning and fairness. In privacy literature, a thread of work has shown the incompatibility of group fairness with privacy~\citep{esipova2022disparate, bagdasaryan2019differential, cummings2019compatibility} but these incompatibilities arise due to privacy-specific methods, such as gradient clipping and differences in neighboring datasets. Fairness literature has studied the related problem of the influence of training data on fairness~\citep{wang2022understanding}, but does not provide any methods for unlearning. In unlearning literature, recent empirical studies have shown that unlearning can increase disparity~\citep{zhang2023forgotten}, other works have demonstrated the incompatibility of fairness and unlearning for the SISA algorithm \citep{kochno}, and one work~\citep{wang2023inductive} has provided a method to achieve removal and fairness but uses a sharding and retraining algorithm over fairness-corrected graph data for GNNs. In this paper, we propose the first efficient method which achieves fairness while being provably unlearnable without requiring retraining.
\section{HC-NJDG Data Collection}
\label{sec:njdg}


High Court National Judicial Data Grid (HC-NJDG) was launched in July 2017 \cite{njdg_gamechg, njdg_hc}. We started collecting data from the portal on August 31, 2017. The last data used in this paper was collected on March 22, 2020. We have collected data on 229 randomly chosen dates, spanning a period of more than two and a half years.



To provide a glimpse of the data, some of the statistics, as collected on August 20, 2018, are provided in Table \ref{tab:njdg_stats}. The portal has more statistics available but we have chosen to present only the ones that are relevant for this paper. The data related to the number of pending cases in all the high courts in India is shown. The table also presents data on the number of monthly disposed and filed cases. 
% The cases are divided into two different kinds. The rows in the table show the division of cases based on the age of the cases, i.e., how old are they. The columns show the division based on the types of the cases, i.e., whether the cases are civil, criminal or writs. Hence, on this date, more than 3.3 million cases were pending in the high courts in India.


\begin{table}[h]
\centering
\footnotesize
    \begin{tabular}{ | l | r | r | r | r |}
    \hline
    Title & Civil & Criminal & Writs & Total \\ \hline
    Pending Cases & 1506780 & 769754 & 1114448 & 3390982 \\ \hline
    Cased Filed (monthly) & 27663 & 42404 & 32009 & 102063 \\ \hline
    Cased Disposed (monthly) & 28080 & 47368 & 36548 & 111996 \\ \hline
    \end{tabular}
    \caption{Some of the statistics available on the HC-NJDG portal\cite{njdg_hc}.}
    \label{tab:njdg_stats}
\end{table}


% \begin{table}[h]
% \centering
% \footnotesize
%     \begin{tabular}{ | l | r | r | r | r |}
%     \hline
%     Cases Pending & Civil & Criminal & Writs & Total \\ \hline
%     Over 10 years & 371332 & 138120 & 142642 & 652094 \\ \hline
%     Between 5-10 years & 370192 & 176909 & 257987 & 805088 \\ \hline
%     Between 2-5 years & 419455 & 214270 & 372992 & 1006717 \\ \hline
%     Less than 2 years & 345801 & 240455 & 340827 & 927083 \\ \hline
%     Total & 1506780 & 769754 & 1114448 & 3390982  \\ \hline
%     \end{tabular}
%     \caption{HC-NJDG data of pending cases in High Courts as on August 20, 2018.}
%     \label{tab:njdg_pending}
% \end{table}

% It also shows the aggregate of the cases that were listed for hearing on that particular date in all the high courts. The columns, like the previous table, represent the type of cases. In this table, we have also included another field, the number of judges in the high courts. Note that there was an error in the number of high court judges reported on that day. According to HC-NJDG portal the number of high court judges in India was 810, whereas according to the vacancy positions document (dated August 01, 2018) available at the Department of Justice website \cite{doj_vacancy}, the number of working strength of high courts was 659 and the approved strength was 1079.

% \begin{table}[h]
% \centering
% \footnotesize
%     \begin{tabular}{ | l | r | r | r | r |}
%     \hline
%     
%         & Civil & Criminal & Writs & Total \\ \hline
%     Cases Filed (last month)& 27663 & 42404 & 32009 & 102063 \\ \hline
%     Cases listed (today) & 11656 & 11793 & 13013 & 36462 \\ \hline
%     Cases Disposed (last month) & 28080 & 47368 & 36548 & 111996 \\ \hline
%     Total Judges & & & &810 \\ \hline
%     \end{tabular}
%     \caption{Number of cases filed (monthly) and disposed (monthly) and the number of cases listed (daily) and the number of judges in the high courts in India as on August 20, 2018.}
%     \label{tab:njdg_monitor}
% \end{table}

% \end{comment}


% \subsection{Scrutinizing the data so collected}

% After the data collection, we had to make sure that the data being used does not suffer from errors made during the download or during saving the files. After a careful inspection, we deleted some of the files which had errors. Table \ref{tab:hcnjdg_details} lists the amount of data that we can actually use from the data collected during the said period. 

Four high courts, namely, Allahabad High Court, Gauhati High Court, High Court of Jammu and Kashmir and High Court of Madhya Pradesh have joined HC-NJDG after we started collecting the data. Hence, they appear fewer number of times. However, majority of the high courts -- 20 to be precise -- had their presence on HC-NJDG when we started collecting the data, i.e., on August 31, 2017. 
% Hence, different high courts may have data on different dates.  

The $25^{th}$ High Court for the state of Telangana was formed on January 01, 2019. For consistency with the previous data, we have continuted to consider Telangana and Andhra High Court as one in our analysis. For this reason, only 24 high courts appear in our study. Note that the last day for data collection for this paper was March 22, 2020, just before the nation-wide lockdown was announced in India due to Covid-19 pandemic. We hypothesize that data pre-lockdown and during the lockdown would be very different and hence not comparable for our work.


% \subsection{Understanding the graphs}

We have plotted many graphs in the paper. In order to maintain coherence and simplicity, and to have a reach to wider audience, we have restricted ourselves to only two kinds of graphs, as explained below. 
% \begin{enumerate}
\paragraph{Temporal data graphs (Dates on horizontal axis)} The horizontal axis (also referred to as X-axis in the paper), consists of dates beginning August 31, 2017 to March 22, 2020 from left to right. The title of each graph is present on the top stating the name of the high court that plot corresponds to. If on some date data was not collected, then data is not shown against that date but the date is still present on the X-axis in the all cases. \fref{fig:hcpendency} is an example of this kind of graph.

\paragraph{Spatial data graphs (High Courts on horizontal axis)} In these graphs, the data from the high courts is plotted. The horizontal axis, or X-axis, in these graphs have 24 points, each representing one of the 24 high courts. Y-axis plots the value of the considered parameter. If a high court does not have a valid data for that parameter, its name still appears on the X-axis but have no value on the Y-axis. The title of the graph is present at the top. \fref{fig:sanctioned_working} is an example of this kind of graph.

% Choice of semilog Y-axis for some curves is made to accommodate wide range of values in one graph. \fref{fig:pend_date_c} is an example of such use. The numbers 1 and 1000 and almost 10 lakh (= 1 Million) are clearly visible in the same graph which would be difficult to show if Y-axis were linear. 

% \end{enumerate}



% \begin{itemize}
% \item Compare with HC NJDG \cite{njdg_hc} \cite{njdg_gamechg}. Can we show that the cases are simply being transfered? 
% \item Draw the curve of the data that you have collected. 
% \item Infer the current scheduling.
% \end{itemize}


% % Figure environment removed

% We will also point out that the implementation of NJDG is not free of glitches of now. A huge discrepancy that we had observed was on September 11, 2017. On that day the number of cases pending with all the High Courts was noted to be 4,990,031, far more than the usual number around 3,300,000. It is for this reason we see an unusual dip in the \fref{fig:eff_hc}. 

\begin{comment}
\subsection{Number of Judges in High Courts}

% Figure environment removed

% Figure environment removed



\fref{fig:hcjudges} shows the aggregate number of judges in the high courts of India for each date on which the data has been collected. Analysis of this data compared with the vacancy document available on the Department of Justice website \cite{doj_vacancy} suggests that the HC-NJDG data on the number of judges in the high courts is erroneous. In fact, according to HC-NJDG, the number of high court judges was way more than the approved strength of 1079 during November and December 2018. In order to cross verify the data, we also started downloading the vacancy document as per the Department of Justice website starting June 2018. This clearly shows that the number of judges reported by HC-NJDG portal is approximately 300 more than the actual number. Moreover, the number of judges is monotonically increasing, which is not the case actually. Hence, we use the data on the number of judges from the vacancy document \cite{doj_vacancy}.


In \fref{fig:jud_date_j}, we plot the spatial data. We have plotted the number of judges in each high court as available on August 06, 2018. Note that the Y-axis in this plot is on log scale. The data is plotted in the descending order of the number of judges in each high court from HC-NJDG data. The reasons for choosing this date is mainly because it is closest to August 01 which is the date when vacancy document was updated by the Department of Justice. It can be clearly seen that in some cases the number of high court judges reported by HC-NJDG is even higher than the approved strength provided in the vacancy document for that high court. To be precise, on August 06, 2018, the number of judges in High Court of Gauhati, Calcutta, Orissa, Jammu and Kashmir, Himachal Pradesh and Manipur was higher than the number of approved strength for the respective high courts.  
\end{comment}
% \subsection{Pending Cases in High Courts}
% 
% % Figure environment removed
% 
% \fref{fig:hcpendency} shows the number of pending cases in all the high courts of India. We have plotted the total number of pending cases in the high courts in India as obtained from the NJDG portal in our data set. It can be clearly seen that the data has few continuous clusters and few sudden jumps. While initial sudden jumps can be explained by the fact that few high courts have joined NJDG late and they may be taking time to converge to report stable number, the overall graph does not represent a healthy update culture. It also implies, in some cases that the updates have been erroneous. For example, on September 11, 2017, there is a completely isolated peak with pending number of cases around 5 million whereas the adjacent dates on both sides have less than 3.5 million. While it is true that the error was quickly ratified, there should be attempts towards not introducing such errors in the first place. In particular, after more than one year of inception, the graph should have started looking like a ``smooth continuous function", which as of now, it doesn't. For around six months between January 2018 to July 2018 the number of pending cases in high courts was reported to be more than 4.3 million which suddenly dropped down to 3.3 million in August 2018. Such sudden jumps need explanations to be reliably used for any practical significance. The reasons for such jumps must be investigated so that such errors do not get repeated on the portal.
% 
% % Figure environment removed
% 
% 
% \fref{fig:pend_date_c} shows the average of pending cases during the data collection period for all the high courts. Note that the Y-axis is in log scale and the data is sorted in descending order of the total number of pending cases at a high court. Hence, the first place is occupied by Allahabad High Court that has more than 700 thousand cases pending. The figure also implies that most of the high courts have huge number of pending cases except Sikkim High Court and newly established high courts for the states of Meghalaya and Tripura. It is also worth noting from this figure that Sikkim High Court has no cases that are pending for more than 10 years. Meghalaya and Tripura High Court have less than twenty cases pending for more than ten years. The rest of the high courts are having the ten plus years pending cases as a substantial percentage of their total pendency. 


% 
% % Figure environment removed

% \fref{fig:yearwise} shows the pendency in terms of the age of cases. From this data, we can see that the number of cases with age greater than five have mainly increased during last one year. However, the number of cases with 10+ years age, suddenly reduced from 9,95,031 on August 06, 2018 to 6,52,180 on August 13, 2018. A similar drop in number is seen for the cases pending for 5-10 years as well. The reasons for such an extraordinary decrease are unknown.

% A detailed study of pending cases in individual high courts is done in Section \ref{sec:pending_hc}.



% \subsection{Disposed, filed and listed cases}

% From the previous subsection, we understand that the average load on the judges of high courts is huge in terms of the number of pending cases. Now we see the rate of disposal, filing and listing of cases that can help us provide upper bounds on the time required to get rid of pendency. Unfortunately, NJDG as of now, updates data on monthly disposal and monthly filing of cases. The number of cases listed is provided on a daily basis. If the other two parameters could also be provided on a daily basis then our analysis and interpretations would be more accurate.

% \fref{fig:fdl_date} shows the data corresponding to the number of disposed, filed and listed cases as available on NJDG on August 06, 2018. The number of cases disposed and filed are provided monthly. The plot is ordered in descending values of the cases disposed. Also note that the Y-axis is on linear scale and still there is a very little difference between the number of cases listed daily and the number of cases disposed monthly. This may mean that the number of listed cases in a day may be reduced, which will be for the benefit of all the stakeholders. The judges will have more time to hear a case unlike now \cite{daksh_report}. The litigants will be better off because the chances of hearing their case will increase. The advocates will save time in appearing for the cases and hence get more time to prepare for cases. The court staff will have lesser files to move and manage.

% While we are not embracing the idea of using NJDG data of just one day, the above figure captures the gap between the three kinds of statistics that are closely related to piling up of pendency. The point here is that the causlists in the high courts should be prepared more scientifically. We back up our assertion by a more careful and rigorous analysis involving more data.



% The following equation governs the pendency of cases in ideal case when the updates take place every evening. 
% \begin{equation}
% P_c = P_m + f - d
% \end{equation}
% where, $P_c$ is the number of closing pending cases, $P_m$ is the number of cases pending in the morning, $f$ is the number of cases filed in a day and $d$ is the number of cases disposed in a day.
% 

% % Figure environment removed
% 
% It is shown in \fref{fig:fdl_hcall} that there is a huge gap between the number of cases listed and the number of cases disposed in a day. The number of cases disposed in a day is obtained by dividing the monthly statistics by 22 because we are assuming that there are 22 working days in a month. The goal should be to minimize the gap between the number of cases disposed and the number of cases listed on any given day. This would mean that most of the cases that are listed, should be disposed, rather than adjourned. This is the parameter that makes us claim that there is a room for more scientific preparation of causelists as there is a room for decreasing the gap between the number of cases listed and the number of cases disposed in a day. Similar is the case for individual high courts as presented in \fref{fig:fdl_hc}.
% 
% % Figure environment removed

% \fref{fig:fdl10_hcall} shows the number of disposed cases whose age was more than ten years. The graph is not expected to show any trend as the number of cases disposed depends on many circumstances. The only thing that can be reasonably concluded is that the number of disposal of more than ten year old cases was high during February 2018 to March 2018. It has decreased since then.
%However, one thing can be inferred is that the causelists preparations are indifferent to whether a case is ten years old or not. 

% % Figure environment removed






\section{Pending Cases in High Courts}
\label{sec:pending_hc}

As discussed before, pendency in high courts is more than 10\% of the total pendency in India. Hence, concentrating on high courts capture the problem of pendency really well and offer much better quality data that can be studied to deduce meaningful conclusions.

% Figure environment removed


% \begin{comment}
% Figure environment removed


% % Figure environment removed

\fref{fig:hcpendency} shows the aggregate number of pending cases in all the high courts of India. We have plotted the total number of pending cases in the high courts in India as obtained from the HC-NJDG portal in our data set. The blue dots are the data collected from the HC-NJDG and the dashed red line is the best fit straight line to the data minimizing the mean squared error cost function. It can be clearly seen that the data has few continuous clusters and few sudden jumps. While initial sudden jumps can be explained by the fact that few high courts have joined NJDG late and they may be taking time to converge to report stable number, the overall graph does not represent a healthy update culture until around March 2019. However, since April 2019 the updates have been smooth and barring a few outliers, the updates have been consistent. This is already a good news. This means that commendable efforts have been made to make data on HC-NJDG more reliable. 

\fref{fig:sanctioned_working} compares the working strength of the high courts in comparison with the sanctioned strength. The average working strength has been computed for the period June 2018 to March 2020, i.e., 22 months. The data is collected from the vacancy document available on the website of the Department of Justice \cite{doj_vacancy}. We see that on an average, around 38\% seats of judges in high courts remain vacant.

% % Figure environment removed


% \fref{fig:pend_date_c} shows the average of pending cases during the data collection period for all the high courts. The first place is occupied by Allahabad High Court that has more than 700 thousand cases pending. The figure also implies that most of the high courts have huge number of pending cases except Sikkim High Court and newly established high courts for the states of Manipur, Meghalaya and Tripura. It is also worth noting from this figure that Sikkim High Court does not have cases that are pending for more than 10 years. Meghalaya and Tripura High Court have less than twenty cases pending for more than ten years. The rest of the high courts are having the ten plus years pending cases as a substantial percentage of their total pendency. 


% \end{comment}

\fref{fig:np_hc1} and \fref{fig:np_hc2_p2} show the pendency data for the individual high courts. We also plot a linear regression best fit line (the red dashed line) to estimate the trends in pendency rather than depending on just one day of data. The difference between the two figures is that \fref{fig:np_hc1} plots the data collected during the whole duration and \fref{fig:np_hc2_p2} shows the data since October 2019. Barring a few High Courts, \fref{fig:np_hc2_p2} is much better in terms of regular updates than \fref{fig:np_hc1}. In both the plots, the high courts appear in the lexicographic order of their names. 

In \fref{fig:np_hc1}, the graph of Allahabad High Court depicts a decent update culture. There is not much diversion from the best fit line (using linear regression) either. It can be seen that the number of pending cases in Allahabad High Court has been increasing linearly with time. 

Be it \fref{fig:np_hc1} or \fref{fig:np_hc2_p2} Bombay High Court has the poorest record of data update on HC-NJDG among all the high courts. In the whole data collection period, the data has been updated only once. Due to this reason nothing can be said about Bombay High Court reliably. A similar case is with Calcutta High Court. Even though the updates have been frequent, wrong data was uploaded on the portal. The total number of pending cases in Calcutta High Court is more than 250 thousand but the graph shows a different number. Hence, nothing can be said reliably about Calcutta High Court either. For Calcutta High Court, the subsequent analysis is done based on the data available from Calcutta High Court website \cite{calcutta_portal}. For Bombay High Court, the data is not available on its website either, so we had to resort to the Supreme Court annual reports \cite{SCAR}. These are the only two high courts whose data is not taken from HC-NJDG. 

In \fref{fig:np_hc1}, data from Chhattisgarh High Court follows a nice update trend and the best fit straight line looks representative of the increase. Hence, it can be deduced that the number of pending cases in Chhattisgarh High Court has been increasing linearly. The Delhi High Court also has a nice update culture almost throughout the data collection period and the pendency is increasing linearly for this high court too. The updates in Gujarat High Court were not frequent until August 2018. However, after that, the number of cases have been increasing linearly. We consider the best fit line drawn in the figures as representing the rate of increase from this graph. Himachal Pradesh High Court data has seen a surge in the number of cases from October 2019 to March 2020, the best fit line, is with positive slope and hence, the number of pending cases are increasing with time for Himachal Pradesh High Court as well. 

In \fref{fig:np_hc2_p2} represents much better updates of NJDG in the Common High Court for the UT of Jammu \& Kashmir and UT of Ladhakh as well as for Jharkhand High Court. Initial updates of the HC-NJDG data seem erroneous and hence we use \fref{fig:np_hc2_p2} for these high courts. The number of cases in the Common High Court for the UT of Jammu \& Kashmir and UT of Ladhakh follow a linear increase whereas the number of pending cases in Jharkhand High Court are decreasing linearly.

In \fref{fig:np_hc2_p2}, barring a few erroneous updates, Karnataka High Court and Kerala High Court have regularly updated data on HC-NJDG. Again, the best fit straight lines looks quite a good representative of the increase. The pendency for both these high courts is increasing too. \fref{fig:np_hc1} depicts that the champion of updating data on HC-NJDG is Madhya Pradesh High Court. Not even a single outlier. The best fit straight line almost coincides with the data. The number of pending cases at this high court is increasing as well. The next comes Madras High Court, which, doesn't seem to have a good update culture. However, it is good enough to be consistent and is increasing which is the current expected trend in most of the high courts. 

Manipur High Court seems to have reconciled data and hence, we do not consider the data for the whole data collection period but only for the last six months as presented in \fref{fig:np_hc2_p2}. The trend of increasing pendency, however little, can be seen for Manipur High Court as well. 

In \fref{fig:np_hc1}, Meghalaya High Court also seems to have an increasing rate of pending cases. The data points may look erroneous on the first look, however, there is a variation of just 250 cases on the whole scale. So such updates are realistically possible. 

Some kinds of reconciliation seems to have taken place for Orissa High Court as well. Hence, we take the trend from \fref{fig:np_hc2_p2}, which again shows an increasing trend in pendency. 

The updates for Patna High Court, Punjab and Haryana High Court and Rajasthan High Court look reasonable and the best fit seems to be representative of the trend that the pendency is increasing. We take the rate of increase from \fref{fig:np_hc1}. 

Sikkim High Court has very low number of pending cases. So taking last six months of trend may be more beneficial. We see that the pendency is decreasing. So we use \fref{fig:np_hc2_p2} for computing the best fit line, which shows a decrease in the number of pending cases. 

In \fref{fig:np_hc1}, for the hypothetical aggregate of Telangana and Andhra High Court the trend is again a linear increase in the pendency. 

Tripura High Court has done very well since October 2019. There is a close to perfect linear decrease in the number of pending cases \fref{fig:np_hc2_p2}. For Uttarakhand High Court, we consider the best fit line for computing the increase in pendency from \fref{fig:np_hc2_p2}. The figure for Uttarakhand High Court in \fref{fig:np_hc1} is quite unreliable.

Hence, from the above analysis, we can deduce that we have enough data for computing the rate of increase of pendency reliably for 21 high courts, as well as the hypothetical aggregate of Telangana and Andhra High Court. However, the updates for two high courts, viz., Bombay and Calcutta are too unreliable on HC-NJDG to be able to make any conclusions about their pendency statistics. Other sources have been used for inferring their data. 

% Figure environment removed

% \begin{table}[h]
% \centering
% \footnotesize
%     \begin{tabular}{ | l | c | c |}
%     \hline
%     
%     High Court  & Rate of Increase & Source data \\ \hline
%     Allahabad &  & \fref{fig:np_hc1} \\ \hline
%     Bombay &  & None \\ \hline
%     Calcutta &  & None \\ \hline    
%     Delhi &  & \fref{fig:np_hc1} \\ \hline
%     Gauhati &  & \fref{fig:np_hc1} \\ \hline
%     Gujarat &  & \fref{fig:np_hc1} \\ \hline    
%     Himachal &  & \fref{fig:np_hc1} \\ \hline
%     Jammu and Kashmir &  & \fref{fig:np_hc2_p2} \\ \hline
%     Jharkhand &  & \fref{fig:np_hc2_p2} \\ \hline
%     Karnataka &  & \fref{fig:np_hc1} \\ \hline
%     Kerala &  & \fref{fig:np_hc1} \\ \hline
%     Madhya Pradesh &  & \fref{fig:np_hc1} \\ \hline
%     Madras &  & \fref{fig:np_hc1} \\ \hline
%     Manipur &  & \fref{fig:np_hc2_p2} \\ \hline
%     Meghalaya &  & \fref{fig:np_hc1} \\ \hline
%     Orissa &  & \fref{fig:np_hc2_p2} \\ \hline
%     Patna &  & \fref{fig:np_hc1} \\ \hline
%     Punjab and Haryana &  & \fref{fig:np_hc1} \\ \hline    
%     Rajasthan &  & \fref{fig:np_hc1} \\ \hline
%     Sikkim &  & \fref{fig:np_hc2_p2} \\ \hline
%     Telangana and Andhra &  & \fref{fig:np_hc1} \\ \hline    
%     Tripura &  & \fref{fig:np_hc2_p2} \\ \hline
%     Uttarakhand &  & \fref{fig:np_hc2_p2} \\ \hline
%     \end{tabular}
%     \caption{Number of cases filed (monthly) and disposed (monthly) and the number of cases listed (daily) and the number of judges in the high courts in India as on August 20, 2018.}
%     \label{tab:njdg_monitor}
% \end{table}


\subsection{Ratio of Pendency to Judges}

The total pendency, in itself, does not provide any information until the number of judges in the respective high court is also taken into account. This subsection considers the ratio of \emph{pending cases/number of judges} as a parameter for each high court.




\fref{fig:pend_date_r} plots the ratio \emph{pending cases/number of judges} for each high court. The number of pending cases is calculated by taking the pendency on the last day of data collected. The number of judges, however, are taken from \fref{fig:sanctioned_working}. The results are plotted in the descending order of the ratio so calculated. This graph provides the distribution of workload on each high court and judges thereof. The blue dots show the ratio \emph{pending cases/working strength of judges} for the average working strength of each high court. For example, Rajasthan High Court has the maximum value of 19,374 pending cases for each judge whereas Sikkim High Court has the minimum ratio which is 78. Hence, statistically we can say that a judge in Rajasthan High Court has almost 250 times more load than a judge in Sikkim High Court. It can be seen that the situation is similar for most of the high courts. The mean of this ratio is 6908, i.e., the national average of the number of pending cases per judge. It signifies that on an average each sitting judge of the high courts in India needs to dispose 6908 cases to reduce pendency to zero, provided no more cases are filed. It also means that the judges in some of the high courts are insanely overburdened. In the interest of justice, urgent appointments are required so that the case load may be shared. Hence, the number of pending cases per judge is huge and the numbers are so high that it would not be unfair to state that they are simply beyond the capacity of the current number of working judges.
% Some input from technology is required to handle such huge numbers and ease the tasks of the judges.



% Figure environment removed



% % Figure environment removed

\section{Estimating Time Required to Combat Pendency}
\label{sec:combat}

Having discussed the trend of pending cases and the number of cases per judges in the high courts, we turn our attention to computing the time required to clear the pendency of the cases in high courts of India. Our attempt is the first -- to the best of our knowledge -- to be based on extremely rich statistical data to answer the question, ``How long will it take to reduce the pendency in the high courts to zero?".



\subsection{Rate of increase of pendency}
\fref{fig:hcpendency} presents increase in the number of total pending cases from August 31, 2017 to March 22, 2020. 
% Apart from plotting the data points obtained from HC-NJDG, we have also drawn a best fit straight line according to the mean squared error (MSE). In other words we have used linear regression to fit the data. If $P_i$ represents pendency according to HC-NJDG, and $\hat{P}_i$ represents pendency according to the best fit, $n$ is the total number of days of the available data, then, the best fit line minimizes,

% \begin{equation}
% MSE=\frac{1}{n}\sum_{i=1}^{n}(P_i-\hat{P}_i)^2
% \end{equation}






% 
% % Figure environment removed

We observe that the number of pending cases in the high courts in India is increasing at a rate of approximately 1135 cases per day. It is basically the slope of the best fit line in \fref{fig:hcpendency}. We have plotted a similar best fit line for all the high courts in \fref{fig:np_hc1} and \fref{fig:np_hc2_p2}. The slope for various high courts is taken as rate of increase of pendency. For most of the high courts it means that the pendency will never get over rather increase with time. 

\fref{fig:hcrate_all} shows the rate of increasing pendency for each high court as computed from the slopes of the best fit lines in \fref{fig:np_hc1} and \fref{fig:np_hc2_p2}. It is quite expected that Rajasthan High Court, whose ratio of pending cases to judges is very high, has the highest rate of increase of pendency. A similar observation may be made for several other high courts whose ratio of pending cases to judges is very high. 


% \subsection{Comparison of Pendency with Filed/Disposed Data}
\subsection{Towards Computing Time to Combat Pendency}



% \subsection{Towards Computing Time to Combat Pendency}

We use our analysis of NJDG data to find out answers to the following questions:
% Questions to be answered:
\begin{enumerate}
% \item What is the rate of disposal of cases in high courts? (\fref{fig:avg_disp})
\item What is the rate of disposal of cases per day per judge in high courts? (\fref{fig:avg_disp_judge})
% \item How long will it take to nullify pendency in the high courts if no new cases are filed? (\fref{fig:avg_time_filing_zero})
% \item How many more judges in high courts are required so as to make the rate of increase of pendency of that high court to zero? (\fref{fig:increase_judges})
\item If the number of judges in high courts increase linearly and reach their sanctioned strength in ten or twenty years from now, and the average disposal rate used for a judge is as provided in \fref{fig:avg_disp_judge}, then how many years are required to reduce the pendency of cases to zero? (\fref{fig:years_combat_real})
% \item If the number of judges in high courts is made equal to the respective approved strength of each high court, and the average disposal rate used for each judge as provided in \fref{fig:avg_disp_judge} but the minimum disposal rate used is the average, then how many years required to nullify the pendency? (\fref{fig:years_combat_national_avg_lowest})
\end{enumerate}



% Now we discuss the questions and the figures referred above in detail. 

% \fref{fig:avg_disp} plots the average daily disposal of cases for each high court. 

% Our analysis would have been more accurate had these statistics were provided on a daily basis. Apart from plotting the average for each high court, we also plot the average over all the high courts.  

% % Figure environment removed



Disposal related statistics are provided on NJDG portal on a monthly basis. Thus, we have divided the number by 30 to get the daily figure. In \fref{fig:avg_disp_judge}, we plot the number of cases disposed per judge per day for each high court. The national average is 5.93. This figure provides the average number of cases disposed by each high court judge in a day. We use these results to estimate the time required to nullify the pendency in different high courts in India.




\begin{comment}

\fref{fig:avg_time_filing_zero} provides an estimate of the years required to dispose all the cases if now new cases are filed. If no new cases are filed, then the courts will keep on disposing the cases at its current rate which will eventually lead to disposal of all the cases. We also plot the number of years required to dispose all the cases if no new cases are filed and the high courts are working at their respective approved strength. Obviously, in case of approved strength, the years required is lesser.


% Figure environment removed

\fref{fig:increase_judges} does not assume that no new cases are filed. It instead provides an insight on the number of judges required if the rate of increase of pendency is to be made zero, i.e., the pending number of cases should neither increase nor decrease. This provides a good sign for most of the high courts as the number of judges required to make the rate of increase equal to zero is less than the vacancy in that particular high court. Note that only High Court of Jammu and Kashmir and Madras High Court have the required number greater than the vacancy in these high courts. This means that even if the number of judges is made equal to the approved strength, the rate of increase of pendency will still be non zero. This further implies that in these two high courts, the pendency may never decrease.

% Figure environment removed
\end{comment}

% \subsection{Time Required to Combat Pendency}

% We assume a linear decrease in the number of pending cases, i.e., if at time $t=0$ pendency is $p_0$, and the rate of decreasing pendency is $\alpha$ then at time $t$ pendency $p_t$ is given by 
% \begin{equation} 
% p_t = p_0 -\alpha\cdot t
% \end{equation}
% 
% By putting $p_t=0$, and rearranging for $t$, we get, 
% \begin{equation} 
% t = \frac{p_0}{\alpha}
% \end{equation}

% Figure environment removed



We have enough information to compute the time required to nullify the pendency in high courts. We are assuming that the number of judges increase linearly every year. We define the following variables:
\begin{enumerate}
\item Assumed to be constant, disposal rate per judge per year, denoted as $d$, of a high court (extrapolated from \fref{fig:avg_disp_judge}),
\item Pendency $p_t$ at the start of any given year $t$ in a high court,
\item Working strength $w_t$ of a high court during any given year $t$,
\item Yearly rate of increase ($r_t$) of pendency for a high court when average working strength is $w_t$ (extrapolated from \fref{fig:hcrate_all}),
% \item Approved strength of a high court ($s$),
\end{enumerate}
Then the following holds:
\begin{equation}
p_t = p_{t-1} + r_{t-1}
\label{eq:1}
\end{equation}
% and,
\begin{equation}
r_t = r_{t-1} - d\cdot(w_t-w_{t-1})
\label{eq:2}
\end{equation}


where $p_0$ and $r_0$ are taken as the values of pendency on the last day of our data collection and from \fref{fig:hcrate_all} respectively. The later values are updated according to Eq. \ref{eq:1} and \ref{eq:2} to compute $t$ for which $p_t\leq 0$. 

\fref{fig:tikz} shows how the pendency may be decreasing. If we begin from $p_0$ at the rate $r_0$, then we reach at pendency $p_1$ at the end of the first year. Since the number of judges will increase at the end of the year, more number of cases will be disposed and the rate of increase of the cases will be lesser than the previous rate. This will continue until the rate of increase becomes zero and eventually becomes negative making pendency to hit 0 at some point in time. 

% Figure environment removed


\fref{fig:years_combat_real} shows the number of years required to nullify pendency in the high courts. It presents results assuming that the sanctioned strength of high courts are reached in ten years and twenty years respectively. We also assume that the rate of increase of the judges in both the cases is linear. We can see that there is a huge gap between the years taken to clear the pendency in the two cases. If we assume that the vacancy of judges in the high courts is to be filled in twenty years, then Himachal High Court and Madras High Court may take 150 and 113 years respectively. However, if we assume that the working strength of the high courts reach their sanctioned strength in ten years then the numbers for both the high courts mentioned above are 102 and 83 respectively, i.e., an improvement of 48 and 30 years respectively. On the other side of the spectrum we see Tripura High Court and Sikkim High Court that will take 2 years and 6 years respectively, irrespective of whether it takes ten or twenty years to fill the vacancy in these high courts. Thanks to the number of the pending cases, rate of decrease of pendency and the sufficient number of judges to handle that. Another extreme case is the Punjab and Haryana High Court. There is no plot against that high court in either case because the sanctioned strength, no matter whether reached in ten or twenty years, the rate of increase of pendency will still be positive rather than negative. We also see that the majority of the high courts will take more than twenty years if the sanctioned strength is reached in twenty years and more than 14 years if the sanctioned strength is reached in ten years. Hence, if only ten years are taken to fill the vacancy in high courts then substantially lesser number of years are required to clear the pendency. This is also reflected in the average number of years taken. For ten years to fill vacancy, on an average, it will take 25.3 years and for twenty years to fill vacancy, on an average it will take 35.35 years to clear the pendency. Hence, filling the vacancies in the high courts is a key to clearing the pendency. More details on the numbers used to plot \fref{fig:years_combat_real} is provided in Table \ref{tab:years} in the Appendix.


% Provided the above information, we need to find the value of rate of decreasing pendency $(\alpha)$ in terms of the known parameters. For most of the high courts the rate of increase is positive if only the working strength of the high courts is considered. Hence, to make the rate of increase negative, or to make the rate of decrease positive, we consider that each high court is working at its approved strength. $r$ is the rate of increase of cases when working strength of high courts is used. Thus, the rate of decrease of cases can be given by 
% \begin{equation} 
% \alpha = d\cdot(s-w) - r 
% \label{eq:alpha}
% \end{equation}
% 
% If the rate $\alpha$ computed in Eq. \ref{eq:alpha} is positive then the pendency will become zero sooner or later. However, if $\alpha\leq 0$, then the pendency will never decrease, until either the disposal rate per judge per day increase or the approved strength is increased.
% 
% 
% Putting all together, the following formula computes the number of working days required (denoted by $t$) to nullify the pendency in each high court:
% \begin{equation}
% t=\frac{p_0}{d\cdot(s-w)-r}
% \label{eq:days}
% \end{equation}
% 


% Since the above formula computes the number of working days, and each high court is supposed to function 210 days a year, the formula to compute the number of years (denoted by $y$) to nullify pendency is given by:
% \begin{equation}
% y=\frac{t}{210}%\cdot\frac{p_i}{e_i\cdot(s_i-w_i)-r_i}
% \label{eq:years}
% \end{equation}
% 
% \fref{fig:years_combat_real} use formula in Eq. \ref{eq:years} to compute the number of years required to nullify the pendency for each high court. Recall that the working strength of each high court is assumed to be its approved strength. Note that there is no point corresponding to the high courts of Jammu and Kashmir and Madras High Court because as noted in \fref{fig:increase_judges}, the number of required judges is more than the vacancy in these high courts. Thus, under the current constraints, the rate of decreasing of cases cannot be made positive. Thus, the pendency in these two high courts will still keep on increasing. The maximum is for Gauhati High Court and the minimum is for Sikkim High Court.  The average for the 22 high courts turn out to be a bit over 9 years. 


% Figure environment removed

\begin{wraptable}{r}{5.7cm}
    {
        \setlength{\tabcolsep}{0.015cm}
        \scriptsize
        \vspace{-4mm}
        \begin{tabular}{lllcclcc}
            \toprule
            \multirow{2}{*}{Method} & \multicolumn{4}{c}{Output} & \multicolumn{2}{c}{Input} & Success                                            \\\cmidrule(lr){2-5}\cmidrule(lr){6-7}
                                    & Generation                 & Rep.                      & Exec    & Pred & Pool    & Proprio & (\%)          \\
            \midrule
            BC-Z                    & FeedForward                & Delta                     & 1       & 10   & Avg     & \xmark  & 0.0           \\
                                    & FeedForward                & Delta                     & 4       & 10   & Avg     & \xmark  & 15.0          \\
                                    & FeedForward                & Delta                     & 8       & 10   & Avg     & \xmark  & 18.5          \\
            \midrule
            Ours
                                    & FeedForward                & Delta                     & 8       & 16   & Spatial & \cmark  & 29.0          \\
                                    & FeedForward                & Abs                       & 8       & 16   & Spatial & \cmark  & 35.5          \\
                                    & Diffusion                  & Delta                     & 8       & 16   & Spatial & \cmark  & 69.5          \\
                                    & Diffusion                  & Abs                       & 8       & 16   & Avg     & \cmark  & 76.5          \\
                                    & Diffusion                  & Abs                       & 8       & 16   & Spatial & \cmark  & \textbf{79.0} \\
            \midrule
        \end{tabular}
        \vspace{-3mm}
        \caption{
            \textbf{Policy Learning Ablations}.
            Action generation using diffusion models~\cite{ho2020denoising} robustly outperforms feed-forward models across other policy design decisions.
        }
        \label{tab:policy}
        \vspace{-6mm}
    }
\end{wraptable}

\section{Conclusion and Future Work}
In this work, I design corruption-robust algorithms for the Lipschitz contextual search problem. I present the \emph{agnostic checking} technique and demonstrate its effectiveness in designing corruption-robust algorithms. There are several open problems for future research. First, in the algorithm I propose for pricing loss, the schedule for agnostic checks is fixed upfront. Can the learner design an adaptive checking schedule for the pricing loss? Second, this work assumes the learner has knowledge of the Lipschitz constant $L$. Can the learner design efficient no-regret algorithms without knowledge of $L$? 

% \section*{Acknowledgements}
% The first author is supported by the Teachers Associateship for Research Excellence (TARE) fellowship (Sanction Number: TAR/2019/000437) provided by the Science and Engineering Research Board (SERB), Department of Science and Technology, India. 

% \begin{comment}
\bibliographystyle{IEEEtran}
\bibliography{/home/kshitizv/Dropbox/law/LVI/2017/legal_informatics}

% \input{biographies}
\begin{comment}
\section{System Architecture}
\label{appendix:architecture}
\system has a novel modularized system architecture with three key components: 
\emph{StreamManager}, 
\emph{TxnManager} and \emph{TxnScheduler}. 
These components are instantiated in each thread locally.
The execution outline of \system is presented in Algorithm~\ref{alg:algo}.
Transactional stream processing is continuous and potentially never ends (Line 1$\sim$8).
The dependency resolution and execution of state transactions are separated into two non-overlapping phases by punctuations~\cite{Tucker:2003:EPS:776752.776780} (Line 2 and 5), which guarantees that no subsequent input event will have a smaller timestamp. 
Effectively, a batch of state transactions is collected during the first phase, and processed during the second phase.

In the first phase (i.e., stream processing phase), 
the \emph{StreamManager} conducts preprocessing for every input event ($e$). Similar to some prior works~\cite{tstream}, state transactions may be issued but not immediately processed during preprocessing (Line 3).
The \emph{pre\_processing} and \emph{post\_processing} functions are exposed as APIs to users.
The \emph{TxnManager} handles dependency resolution (Line 4) among state transactions and insert decomposed operations to construct a \tpg. We discuss the detailed two-phase \tpg construction process in Section~\ref{subsec:construction}.

In the second phase  (i.e., transaction processing phase), 
the \emph{TxnManager} is first involved again to refine (Line 6) the constructed \tpg with further dependency resolution.
The \emph{TxnScheduler} 
schedules operations for concurrent execution based on the constructed \tpg according to the three dimensions of scheduling decisions (Line 7). 
In particular, a scheduling decision model $M$ is instantiated based on the constructed \tpg (Line 14).
\textbf{\circled{1}} Guided by $M$, execution threads adopt an exploration strategy (Section~\ref{subsec:explore}) to explore the constructed \tpg for operations available to be scheduled constrained by dependencies. 
\textbf{\circled{2}} 
During exploration, one or multiple operations may be treated as the 
% basic 
unit of scheduling (Section~\ref{subsec:granularity}). 
Subsequently, \textbf{\circled{3}} every thread executes operation(s) in the unit of scheduling with various abort handling mechanisms (Section~\ref{subsec:abort_handling}).
Only when state transactions are processed (i.e., committed or aborted) can the associated input events be postprocessed (Line 8) by the \emph{StreamManager} based on transaction processing results.
\end{comment}

\begin{comment}
\begin{algorithm}
\footnotesize
    \KwData{$e$ \tcp{Input event}}
    \KwData{$txn_{ts}$ \tcp{State transaction}}
    \KwData{$G$ \tcp{The currently constructed TPG}}
    \While{!finish processing of input streams}{
        \eIf(\tcp*[h]{Phase 1}){\text{$e$ is not a $punctuation$}}{
                $txn_{ts}$ $\gets$ PRE\_Processing($e$)\;
                \textbf{TPG\_Construction}($G$, $txn_{ts}$)\; 
          }(\tcp*[h]{Phase 2}){
                \textbf{TPG\_Refinement}($G$)\; 
                \textbf{TXN\_Scheduling}($G$)\; 
                POST\_Processing()\;
          }
    }
    
    \SetKwFunction{FMain}{TPG\_Construction}
    \SetKwProg{Fn}{Function}{:}{}
    \Fn{\FMain{$G$, $txn_{ts}$}}{
        $O_{1..k}$ $\gets$ \textbf{Partition} $txn_{ts}$\;
        \ForEach{\text{operation $O_{i}$ $\in$ $O_{1..k}$}}{
            \textbf{Identify} its \ld\;
            $G$ $\gets$ $G$ + $O_{i}$ \;
        }
    }
    \SetKwFunction{FMain}{TPG\_Refinement}
    \SetKwProg{Fn}{Function}{:}{}
    \Fn{\FMain{$G$}}{
        \ForEach{\text{vertex $e_{i}$ $\in$ $G$}}{
            \textbf{Identify} its \td, \pd\;
        }
    }
    
    \SetKwFunction{FMain}{TXN\_Scheduling}
    \SetKwProg{Fn}{Function}{:}{}
    \Fn{\FMain{$G$}}{
        $M$ $\gets$ Instantiated with $G$;\tcp{A decision model}
        \While{!finish scheduling of $G$
        }{
          \textbf{\circled{2}} $Scheduling Unit$ $\gets$ \textbf{\circled{1}} \emph{Explore}($G$, $M$)\; 
            \textbf{\circled{3}} \emph{Execute with Abort Handling} ($Scheduling Unit$)\; 
        }
    }
  \caption{Execution Outline of \system}
  \label{alg:algo}
\end{algorithm}
\end{comment}


\end{document}



