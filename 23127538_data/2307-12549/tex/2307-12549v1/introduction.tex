\section{Introduction}
\label{sec:intro}

\par Justice delayed is justice denied. Delays in a justice system don't just violate the fundamental, constitutional and human rights of a victim but they also have an adverse effects on the rights of the accused as well as those who are convicted. Recently, Supreme Court of India acquitted two persons, accused of a gang rape after 28 years of the incident \cite{twenty_eight}. In an another case, a litigant had to fight for more than four decades to retrieve possession\cite{unfortunate}. Such unfortunate examples are not exceptions in India. They contribute towards the disrepute of the judicial system, lower the faith of the people in judiciary and also impact the economic growth. The judicial delays may potentially translate to a loss of 0.48\% of the national Domestic Gross Product (GDP) of India\cite{daksh_report}. Hence, the government has all reasons to implement policies that help removing the pendency. We define \emph{pendency} and \emph{delay} as they are defined in The Report No. 245 of Law Commission of India\cite{lci_report245}:

\begin{enumerate}
\item \emph{Pendency: All cases instituted but not disposed of, regardless of when the case was instituted.}
\item \emph{Delay: A case that has been in the court/judicial system for longer than the normal time that it should take for a case of that type to be disposed of.}
\item Nullifying pendency: To make pendency equal to zero.
\end{enumerate}


Note that in our definition of pendency, even if a case was instituted just one day ago, it still counts as pendency. Hence, it is a very strict way of computing judicial backlogs. We will use this meaning of pendency throughout the paper. 

\par In the year 2009, it was accepted by the Prime Minister of India that the pendency in Indian courts is the maximum in the world \cite{max_india}. India does very poorly compared to the other major democracies of the world\cite{dushyant_doj}. As of October, 2020 more than 34 million cases were pending in all the levels in Indian courts \cite{NJDG}. A substantial percentage of them have been around for more than ten, twenty or even thirty years \cite{NJDG}\cite{94kcases}. Indian Judiciary has started digitization of courts, on the initiatives of Supreme Court of India, through e-Courts project to take help of the Information and Communications Technologies (ICT) in the judicial sector through its e-Committee \cite{sci_actionplan2005}\cite{sci_actionplan2014}. A great leap in providing free access to the judicial information was provided by the implementation of the National Judicial Data Grid (NJDG) \cite{NJDG}, an important outcome of the e-Courts projects, has data of more than 34 million cases pending in Indian courts at all the levels. The data is open and is available publicly.

Many attempts have been made to estimate the number of years required to clear the pendency of the cases. According to Justice V. V. Rao, a prediction made in 2010 \cite{justicerao320}, it would take 320 years to clear the pendency. A report of Delhi High Court stated that it would require 466 years to clear all the pendency in Delhi Courts \cite{delhi_466}. On the other hand, a commitment to clear all the pending cases in five years was made by the then Union Law Minister in 2011 \cite{thing_past} and later by the then CJI Justice H.L. Dattu in 2015\cite{fiveYearsJDattu}. Yet another commitment was made to make average disposal of cases as three years rather than then 15 years \cite{reforms_three}. Law Commission of India published a report discussing delays and pendency in July 2014\cite{lci_report245}. However, they primarily addressed the issue in lower judiciary. Moreover, the statistics and the situation has changed drastically in last six years as even though the number of judges have increased, the pendency has not decreased. National Judicial Data Grid for High Courts (HC-NJDG) \cite{njdg_hc}, provides data on pending cases in High Courts and was visioned to be a game changer\cite{njdg_gamechg}. 

% % Figure environment removed

% Figure environment removed

In this paper, we exclusively study data from HC-NJDG and report results related to High Courts only. As of October 2020, there were more than 5 million cases pending in the High Courts of India\cite{HC_5M}. We have chosen to study pendency in high courts only because they are generally well equipped with resources to implement the suggested measures effectively. Also, since the number of high courts is only 25 (the 25$^{th}$ started on January 1, 2019), analysis and results are easy to interpret. 

\par The launching of NJDG also caused a sudden jump in India's rank in the Ease of Doing Business Report by the World Bank. It led to an improvement of 30 ranks in the year 2017 compared to the year 2016 \cite{WB_lauds}. However, it is not yet time to celebrate its success. This effort can be called successful only after all the stakeholders (e.g. the judges, court staff, advocates and litigants) find it useful in reducing their burden and NJDG helps improving the efficiency of the whole judicial system. It is still very far from that stage. Note that the success of portals like NJDG depends immensely on individual high courts updating their data regularly on the portal. We have seen positive signs and are hopeful that NJDG is indeed making progress towards an improvement that can make difference. 

This study, to the best of our knowledge, is the first one to analyze NJDG data over such a long period of time. Most of the existing studies consider the data from only one day on NJDG. Hence, we differ from the other studies in this basic premises itself. We also also try to answer the question, "How reliable is single day analysis of NJDG?". We answer this as negative, i.e., the NJDG data collected on just one day may not be taken as reliable for any reasonable analysis. There have been instances when the data on NJDG was very erroneous and such days are not rare. For example, a recent article on the pendency statistics of Bombay High Court claimed that 4.64 lakh cases are pending \cite{bombay_hc464}. Our finding is that throughout the data collection period, the Bombay High Court has updated the number of pending cases only once. 



\subsection{Results in the paper}

Our work revolves around answering one central question. \emph{How long will it take to clear all the pending cases in the High Courts?} We summarize some of our results below:

\paragraph{Increasing pendency} Pending cases are increasing in most of the high courts rather than decreasing (Fig. \ref{fig:hcpendency}, \ref{fig:np_hc1}, \ref{fig:np_hc2_p2}). Hence, in the absence of clear policies and their implementation, pendency can never be cleared in most of the high courts.
% \paragraph{Need for regular updates} regular update of HC-NJDG is required for it to be useful. Data related to some high courts is not being updated regularly or is updated erroneously on the portal (\fref{fig:np_hc2}).
\paragraph{Load on judges in different high courts} There is a huge difference in terms of average load of cases on judges of different high courts (\fref{fig:pend_date_r}). For example, a judge of Rajasthan High Court has almost 250 times more load than a judge in Sikkim High Court. 
\paragraph{Years to clear the pendency} Assuming a linear increase in the number of cases as well as in the number of judges so that the high courts operate at their sanctioned strength of judges, then for most of the high courts pendency can be nullified (Fig. \ref{fig:years_combat_real}, \ref{fig:years_combat_national_avg_lowest}). However, the number of years required to do so may vary significantly. 
\paragraph{Proposed policies for clearing the pendency} \fref{fig:req_individual} and \fref{fig:increase_judges} may help the government of India to take informed decisions on the number of judges that need to be increased. We find that the current sanctioned strength of the high courts is adequate to take care of the newly instituted cases. They are, however, insufficient for clearing the previous backlog of cases. Hence, the government may enact legislation to create some temporary positions in high courts, just to clear the pendency, without changing the actual current sanctioned strength.





\subsection{Organization of the paper}
\par The rest of the paper is organized as follows: Section \ref{sec:rw} encompasses the scope of the work and the related studies. Section \ref{sec:njdg} discusses data collection and explains the graphs used in the paper. Section \ref{sec:pending_hc} elaborates on pending cases in the high courts. Section \ref{sec:combat} is home to the most important result of the paper in which we estimate the time required to nullify the pendency in the high courts. Section \ref{sec:policy} focuses on forming policies to ensure that the pendency decreases. Section \ref{sec:conclusion} concludes the paper.

% Section \ref{sec:proposed} presses the need for pitching in more computer science, machine learning in particular, to fight the problem of the inefficiency in current scheduling of cases in the courts.

