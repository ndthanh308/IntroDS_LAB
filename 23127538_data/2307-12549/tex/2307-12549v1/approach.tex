\section{Classification of Cases}
\label{sec:proposed}



The most difficult part of the problem is that it is not understood yet. In order to design an optimal scheduling we have to first define the notion of optimality in this case. However, we believe that a good classification of related cases can help long way in optimizing scheduling as discussed below.

With the huge amount of information available on NJDG, it is possible to classify related cases that can be heard by a judge on the same date. So instead of scheduling cases related to different laws, it makes more sense to schedule cases that are spatially closer in terms of which law is applicable to them. This will save the time of judges as even if the adjournments are to be made, some of the cases will be heard. The courts already maintain classification of cases. This information has to be integrated well with NJDG. Machine learning and natural language processing can play a big role in such classification. 

Like social and physical sciences, it may be expected here that the cases follow a Zipf distribution, i.e., there are few statutes and laws that are responsible for huge number of litigations and huge number of legislations are responsible only for a few litigations. If it indeed turns out to be true then it will aid in design of a better scheduling mechanism for cases. It is easier to do so for High Courts and the Supreme Court because the cases are already classified. 

While, as of now, we are not aware of the scheduling algorithms used by the courts, one thing is sure that there is a lot of scope for improvement in the scheduling of court cases. There is an urgent need to use better scheduling criteria for the High Courts and even better measures have to be taken for the district courts. This can save the time and effort of all the stakeholders involved and reduce the waste of resources on scheduling hearings that will never take place \cite{daksh_survey}. In this context, according to the study conducted in \cite{time-motion}, only 6 out of 50 listed cases in one of the courts were heard.



% Similarity of cases will have lesser context changes enabling the judges to be better prepared. The judges would be required to deal with the cases of a specific nature on any given day instead of a huge possibility of anything related to any legislation or statute. This will enable them to study only the laws and judgments related to the cases scheduled on the next day as they know what questions of law they can expect from the scheduled cases. 

% The courts have divided the types of cases but this is not sufficient. Even if 25 million cases are divided into 20-30 classes, it does not solve the problem. Mechanisms have to be designed to find the similarities such that solving one case should mean multiple of them are solved by citing the same law or the already decided cases, if available. Mechanisms have to be designed to find the sections of different acts and the articles of the Constitution of India to be applied in the case at hand. 

% Like social and physical sciences, we may expect that such a huge volume of pending cases follow a Zipf distribution, i.e., there must be very little number of Sections of the Acts that are responsible for most of the litigations and most of the acts have little number of litigations. In such a scenario, it is easier to find a related case that has already been decided by the Supreme Court or any High Court. The subordinate judiciary may just use those judgments to decide the case if such related cases can be found. The cases can be sorted based on the applicability of different laws and acts. Use of machine learning and natural language processing may help in this process.
