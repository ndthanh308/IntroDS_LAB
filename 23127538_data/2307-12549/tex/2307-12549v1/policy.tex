\section{Policy to Improve Access to Justice}
\label{sec:policy}
In this section, we comment on some of the required fundamental changes in various customs and enactment of laws to increase the number of judges in the high courts. In order to reduce pendency in the future, many policy level changes have been proposed by various studies. We focus on the number of judges required in the high courts. We advocate filling of the vacancy of judges in the high courts \cite{appointment_of_judges} \cite{judicial_manpower} and while filling the vacancies, age should be considered \cite{age_for_elevation}. Young judges should be elevated so that the retirement rate of the judges in a high court decrease. We:



% Figure environment removed




% \begin{comment}



% % Figure environment removed
% \end{comment}

% \subsection{Time Required to Combat Pendency}

% Provided the above information, we need to find the value of rate of decreasing pendency $(\alpha)$ in terms of the known parameters. For most of the high courts the rate of increase is positive if only the working strength of the high courts is considered. Hence, to make the rate of increase negative, or to make the rate of decrease positive, we consider that each high court is working at its approved strength. $r$ is the rate of increase of cases when working strength of high courts is used. Thus, the rate of decrease of cases can be given by 
% \begin{equation} 
% \alpha = d\cdot(s-w) - r 
% \label{eq:alpha}
% \end{equation}
% 
% If the rate $\alpha$ computed in Eq. \ref{eq:alpha} is positive then the pendency will become zero sooner or later. However, if $\alpha\leq 0$, then the pendency will never decrease, until either the disposal rate per judge per day increase or the approved strength is increased.
% 
% 
% Putting all together, the following formula computes the number of working days required (denoted by $t$) to nullify the pendency in each high court:
% \begin{equation}
% t=\frac{p_0}{d\cdot(s-w)-r}
% \label{eq:days}
% \end{equation}
% 


% Since the above formula computes the number of working days, and each high court is supposed to function 210 days a year, the formula to compute the number of years (denoted by $y$) to nullify pendency is given by:
% \begin{equation}
% y=\frac{t}{210}%\cdot\frac{p_i}{e_i\cdot(s_i-w_i)-r_i}
% \label{eq:years}
% \end{equation}
% 
% \fref{fig:years_combat_real} use formula in Eq. \ref{eq:years} to compute the number of years required to nullify the pendency for each high court. Recall that the working strength of each high court is assumed to be its approved strength. Note that there is no point corresponding to the high courts of Jammu and Kashmir and Madras High Court because as noted in \fref{fig:increase_judges}, the number of required judges is more than the vacancy in these high courts. Thus, under the current constraints, the rate of decreasing of cases cannot be made positive. Thus, the pendency in these two high courts will still keep on increasing. The maximum is for Gauhati High Court and the minimum is for Sikkim High Court.  The average for the 22 high courts turn out to be a bit over 9 years. 


% % Figure environment removed


\begin{enumerate}
\item argue the impact of increasing the efficiency of judges on pendency in high courts \fref{fig:years_combat_national_avg_lowest}. This may be increased by providing more staff and ICT infrastructure.
\item reason for the proposed sanctioned number of judges in high courts depending on the targets set to reduce pendency to zero.
\end{enumerate}

\fref{fig:years_combat_national_avg_lowest} presents the number of years to nullify pendency if the disposal rate of those high courts is increased to 5.93 for which it is lesser than that. In other words, we hypothetically increase the number of cases disposed per judge per day to 5.93, if it is lesser than that, unchanged otherwise. This figure represents the number of years corresponding to a very ambitious case in which we assume the minimum disposal of cases per high court judge per day. We see that there is a significant improvement in the average number of years required to nullify the pendency. The average has come down from 25.3 years to 20.61 years if it takes ten years to reach to the sanctioned strength and from 35.35 years to 29.48 years if it takes twenty years to reach the sanctioned strength. In this work, we have not tried to estimate the optimal number of cases a judge may dispose on an average in a day, without compromising on the quality of justice. However, if we are given such a number then we are in a position to deduce its impact on the pendency of cases in the high courts.

\fref{fig:req_individual} shows the number of judges required if the pendency in high courts is to be nullified in five or fifteen years. We also assume that the number of judges as well as the pendency increase linearly in high courts and that the proposed sanctioned strength reaches in five or fifteen years. To provide a comparison, we have also plotted the current working strength of judges in the high court. Thus, we insist that the only way to substantially reduce the pendency in the high courts is to increase the number of judges. All the other factors like introduction of technology, etc will have a role to play but the scarcity of the judges and supporting staff is the primary reason for pendency. The numbers are very high compared to the current working strength of the high courts and it is in sharp contrast with the fact that the number of judges have not changed much during the data collection period. The rate of appointment of judges is roughly canceled by the rate of retirement, leaving the average number of judges unchanged in the high courts. Elevation of younger judges may help solve this problem. Since the number of judges required is very high compared to the current working strength of judges, the whole purpose of this graph is to provide an estimate on how aggressively the judges should be elevated to the high courts. However, such large number of judges may not be required once the pendency is cleared. For completeness, we have also provided the details about the numbers in this figure in Table \ref{tab:time} in the Appendix.

\fref{fig:increase_judges} provides an insight on the number of judges required if the rate of increase of pendency is to be made zero, i.e., the pending number of cases should neither increase nor decrease. This provides a good sign for most of the high courts as the number of judges required to make the rate of increase equal to zero is less than the vacancy in that particular high court according to the current sanctioned strength. Note that only Punjab and Haryana High Court has the required number greater than the vacancy. This means that once the pendency is taken care of, the current sanctioned strength is capable of handling the volume of fresh cases that are instituted in the high courts. A similar finding is also reported in \cite{jud_history}.

From the above discussion, we can see that the the government will have to be a bit innovative to be able to aggressively clear the pendency. So without increasing the sanctioned strength of the permanent judges in the high courts, government may, through appropriate legislation, increase the number of judges in high courts by elevating judges purely for clearing the pendency. Once pendency is cleared, the current sanctioned strength of the high courts is sufficient to take care of the newly instituted cases.

A due analysis of the cost and the infrastructure required has to be done which is beyond the scope of the current paper.


% This means that even if the number of judges is made equal to the approved strength, the rate of increase of pendency will still be non zero. This further implies that in Punjab and Haryana High Court, the pendency may never decrease. Hence, it is necessary to increase the sanctioned number of judges through appropriate legislation. 










