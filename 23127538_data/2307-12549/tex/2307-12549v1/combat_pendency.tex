% Figure environment removed



% % Figure environment removed

\section{Estimating Time Required to Combat Pendency}
\label{sec:combat}

Having discussed the trend of pending cases and the number of cases per judges in the high courts, we turn our attention to computing the time required to clear the pendency of the cases in high courts of India. Our attempt is the first -- to the best of our knowledge -- to be based on extremely rich statistical data to answer the question, ``How long will it take to reduce the pendency in the high courts to zero?".



\subsection{Rate of increase of pendency}
\fref{fig:hcpendency} presents increase in the number of total pending cases from August 31, 2017 to March 22, 2020. 
% Apart from plotting the data points obtained from HC-NJDG, we have also drawn a best fit straight line according to the mean squared error (MSE). In other words we have used linear regression to fit the data. If $P_i$ represents pendency according to HC-NJDG, and $\hat{P}_i$ represents pendency according to the best fit, $n$ is the total number of days of the available data, then, the best fit line minimizes,

% \begin{equation}
% MSE=\frac{1}{n}\sum_{i=1}^{n}(P_i-\hat{P}_i)^2
% \end{equation}






% 
% % Figure environment removed

We observe that the number of pending cases in the high courts in India is increasing at a rate of approximately 1135 cases per day. It is basically the slope of the best fit line in \fref{fig:hcpendency}. We have plotted a similar best fit line for all the high courts in \fref{fig:np_hc1} and \fref{fig:np_hc2_p2}. The slope for various high courts is taken as rate of increase of pendency. For most of the high courts it means that the pendency will never get over rather increase with time. 

\fref{fig:hcrate_all} shows the rate of increasing pendency for each high court as computed from the slopes of the best fit lines in \fref{fig:np_hc1} and \fref{fig:np_hc2_p2}. It is quite expected that Rajasthan High Court, whose ratio of pending cases to judges is very high, has the highest rate of increase of pendency. A similar observation may be made for several other high courts whose ratio of pending cases to judges is very high. 


% \subsection{Comparison of Pendency with Filed/Disposed Data}
\subsection{Towards Computing Time to Combat Pendency}



% \subsection{Towards Computing Time to Combat Pendency}

We use our analysis of NJDG data to find out answers to the following questions:
% Questions to be answered:
\begin{enumerate}
% \item What is the rate of disposal of cases in high courts? (\fref{fig:avg_disp})
\item What is the rate of disposal of cases per day per judge in high courts? (\fref{fig:avg_disp_judge})
% \item How long will it take to nullify pendency in the high courts if no new cases are filed? (\fref{fig:avg_time_filing_zero})
% \item How many more judges in high courts are required so as to make the rate of increase of pendency of that high court to zero? (\fref{fig:increase_judges})
\item If the number of judges in high courts increase linearly and reach their sanctioned strength in ten or twenty years from now, and the average disposal rate used for a judge is as provided in \fref{fig:avg_disp_judge}, then how many years are required to reduce the pendency of cases to zero? (\fref{fig:years_combat_real})
% \item If the number of judges in high courts is made equal to the respective approved strength of each high court, and the average disposal rate used for each judge as provided in \fref{fig:avg_disp_judge} but the minimum disposal rate used is the average, then how many years required to nullify the pendency? (\fref{fig:years_combat_national_avg_lowest})
\end{enumerate}



% Now we discuss the questions and the figures referred above in detail. 

% \fref{fig:avg_disp} plots the average daily disposal of cases for each high court. 

% Our analysis would have been more accurate had these statistics were provided on a daily basis. Apart from plotting the average for each high court, we also plot the average over all the high courts.  

% % Figure environment removed



Disposal related statistics are provided on NJDG portal on a monthly basis. Thus, we have divided the number by 30 to get the daily figure. In \fref{fig:avg_disp_judge}, we plot the number of cases disposed per judge per day for each high court. The national average is 5.93. This figure provides the average number of cases disposed by each high court judge in a day. We use these results to estimate the time required to nullify the pendency in different high courts in India.




\begin{comment}

\fref{fig:avg_time_filing_zero} provides an estimate of the years required to dispose all the cases if now new cases are filed. If no new cases are filed, then the courts will keep on disposing the cases at its current rate which will eventually lead to disposal of all the cases. We also plot the number of years required to dispose all the cases if no new cases are filed and the high courts are working at their respective approved strength. Obviously, in case of approved strength, the years required is lesser.


% Figure environment removed

\fref{fig:increase_judges} does not assume that no new cases are filed. It instead provides an insight on the number of judges required if the rate of increase of pendency is to be made zero, i.e., the pending number of cases should neither increase nor decrease. This provides a good sign for most of the high courts as the number of judges required to make the rate of increase equal to zero is less than the vacancy in that particular high court. Note that only High Court of Jammu and Kashmir and Madras High Court have the required number greater than the vacancy in these high courts. This means that even if the number of judges is made equal to the approved strength, the rate of increase of pendency will still be non zero. This further implies that in these two high courts, the pendency may never decrease.

% Figure environment removed
\end{comment}

% \subsection{Time Required to Combat Pendency}

% We assume a linear decrease in the number of pending cases, i.e., if at time $t=0$ pendency is $p_0$, and the rate of decreasing pendency is $\alpha$ then at time $t$ pendency $p_t$ is given by 
% \begin{equation} 
% p_t = p_0 -\alpha\cdot t
% \end{equation}
% 
% By putting $p_t=0$, and rearranging for $t$, we get, 
% \begin{equation} 
% t = \frac{p_0}{\alpha}
% \end{equation}

% Figure environment removed



We have enough information to compute the time required to nullify the pendency in high courts. We are assuming that the number of judges increase linearly every year. We define the following variables:
\begin{enumerate}
\item Assumed to be constant, disposal rate per judge per year, denoted as $d$, of a high court (extrapolated from \fref{fig:avg_disp_judge}),
\item Pendency $p_t$ at the start of any given year $t$ in a high court,
\item Working strength $w_t$ of a high court during any given year $t$,
\item Yearly rate of increase ($r_t$) of pendency for a high court when average working strength is $w_t$ (extrapolated from \fref{fig:hcrate_all}),
% \item Approved strength of a high court ($s$),
\end{enumerate}
Then the following holds:
\begin{equation}
p_t = p_{t-1} + r_{t-1}
\label{eq:1}
\end{equation}
% and,
\begin{equation}
r_t = r_{t-1} - d\cdot(w_t-w_{t-1})
\label{eq:2}
\end{equation}


where $p_0$ and $r_0$ are taken as the values of pendency on the last day of our data collection and from \fref{fig:hcrate_all} respectively. The later values are updated according to Eq. \ref{eq:1} and \ref{eq:2} to compute $t$ for which $p_t\leq 0$. 

\fref{fig:tikz} shows how the pendency may be decreasing. If we begin from $p_0$ at the rate $r_0$, then we reach at pendency $p_1$ at the end of the first year. Since the number of judges will increase at the end of the year, more number of cases will be disposed and the rate of increase of the cases will be lesser than the previous rate. This will continue until the rate of increase becomes zero and eventually becomes negative making pendency to hit 0 at some point in time. 

% Figure environment removed


\fref{fig:years_combat_real} shows the number of years required to nullify pendency in the high courts. It presents results assuming that the sanctioned strength of high courts are reached in ten years and twenty years respectively. We also assume that the rate of increase of the judges in both the cases is linear. We can see that there is a huge gap between the years taken to clear the pendency in the two cases. If we assume that the vacancy of judges in the high courts is to be filled in twenty years, then Himachal High Court and Madras High Court may take 150 and 113 years respectively. However, if we assume that the working strength of the high courts reach their sanctioned strength in ten years then the numbers for both the high courts mentioned above are 102 and 83 respectively, i.e., an improvement of 48 and 30 years respectively. On the other side of the spectrum we see Tripura High Court and Sikkim High Court that will take 2 years and 6 years respectively, irrespective of whether it takes ten or twenty years to fill the vacancy in these high courts. Thanks to the number of the pending cases, rate of decrease of pendency and the sufficient number of judges to handle that. Another extreme case is the Punjab and Haryana High Court. There is no plot against that high court in either case because the sanctioned strength, no matter whether reached in ten or twenty years, the rate of increase of pendency will still be positive rather than negative. We also see that the majority of the high courts will take more than twenty years if the sanctioned strength is reached in twenty years and more than 14 years if the sanctioned strength is reached in ten years. Hence, if only ten years are taken to fill the vacancy in high courts then substantially lesser number of years are required to clear the pendency. This is also reflected in the average number of years taken. For ten years to fill vacancy, on an average, it will take 25.3 years and for twenty years to fill vacancy, on an average it will take 35.35 years to clear the pendency. Hence, filling the vacancies in the high courts is a key to clearing the pendency. More details on the numbers used to plot \fref{fig:years_combat_real} is provided in Table \ref{tab:years} in the Appendix.


% Provided the above information, we need to find the value of rate of decreasing pendency $(\alpha)$ in terms of the known parameters. For most of the high courts the rate of increase is positive if only the working strength of the high courts is considered. Hence, to make the rate of increase negative, or to make the rate of decrease positive, we consider that each high court is working at its approved strength. $r$ is the rate of increase of cases when working strength of high courts is used. Thus, the rate of decrease of cases can be given by 
% \begin{equation} 
% \alpha = d\cdot(s-w) - r 
% \label{eq:alpha}
% \end{equation}
% 
% If the rate $\alpha$ computed in Eq. \ref{eq:alpha} is positive then the pendency will become zero sooner or later. However, if $\alpha\leq 0$, then the pendency will never decrease, until either the disposal rate per judge per day increase or the approved strength is increased.
% 
% 
% Putting all together, the following formula computes the number of working days required (denoted by $t$) to nullify the pendency in each high court:
% \begin{equation}
% t=\frac{p_0}{d\cdot(s-w)-r}
% \label{eq:days}
% \end{equation}
% 


% Since the above formula computes the number of working days, and each high court is supposed to function 210 days a year, the formula to compute the number of years (denoted by $y$) to nullify pendency is given by:
% \begin{equation}
% y=\frac{t}{210}%\cdot\frac{p_i}{e_i\cdot(s_i-w_i)-r_i}
% \label{eq:years}
% \end{equation}
% 
% \fref{fig:years_combat_real} use formula in Eq. \ref{eq:years} to compute the number of years required to nullify the pendency for each high court. Recall that the working strength of each high court is assumed to be its approved strength. Note that there is no point corresponding to the high courts of Jammu and Kashmir and Madras High Court because as noted in \fref{fig:increase_judges}, the number of required judges is more than the vacancy in these high courts. Thus, under the current constraints, the rate of decreasing of cases cannot be made positive. Thus, the pendency in these two high courts will still keep on increasing. The maximum is for Gauhati High Court and the minimum is for Sikkim High Court.  The average for the 22 high courts turn out to be a bit over 9 years. 


% Figure environment removed
