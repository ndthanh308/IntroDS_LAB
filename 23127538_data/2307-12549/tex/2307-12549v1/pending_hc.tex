\section{Pending Cases in High Courts}
\label{sec:pending_hc}

As discussed before, pendency in high courts is more than 10\% of the total pendency in India. Hence, concentrating on high courts capture the problem of pendency really well and offer much better quality data that can be studied to deduce meaningful conclusions.

% Figure environment removed


% \begin{comment}
% Figure environment removed


% % Figure environment removed

\fref{fig:hcpendency} shows the aggregate number of pending cases in all the high courts of India. We have plotted the total number of pending cases in the high courts in India as obtained from the HC-NJDG portal in our data set. The blue dots are the data collected from the HC-NJDG and the dashed red line is the best fit straight line to the data minimizing the mean squared error cost function. It can be clearly seen that the data has few continuous clusters and few sudden jumps. While initial sudden jumps can be explained by the fact that few high courts have joined NJDG late and they may be taking time to converge to report stable number, the overall graph does not represent a healthy update culture until around March 2019. However, since April 2019 the updates have been smooth and barring a few outliers, the updates have been consistent. This is already a good news. This means that commendable efforts have been made to make data on HC-NJDG more reliable. 

\fref{fig:sanctioned_working} compares the working strength of the high courts in comparison with the sanctioned strength. The average working strength has been computed for the period June 2018 to March 2020, i.e., 22 months. The data is collected from the vacancy document available on the website of the Department of Justice \cite{doj_vacancy}. We see that on an average, around 38\% seats of judges in high courts remain vacant.

% % Figure environment removed


% \fref{fig:pend_date_c} shows the average of pending cases during the data collection period for all the high courts. The first place is occupied by Allahabad High Court that has more than 700 thousand cases pending. The figure also implies that most of the high courts have huge number of pending cases except Sikkim High Court and newly established high courts for the states of Manipur, Meghalaya and Tripura. It is also worth noting from this figure that Sikkim High Court does not have cases that are pending for more than 10 years. Meghalaya and Tripura High Court have less than twenty cases pending for more than ten years. The rest of the high courts are having the ten plus years pending cases as a substantial percentage of their total pendency. 


% \end{comment}

\fref{fig:np_hc1} and \fref{fig:np_hc2_p2} show the pendency data for the individual high courts. We also plot a linear regression best fit line (the red dashed line) to estimate the trends in pendency rather than depending on just one day of data. The difference between the two figures is that \fref{fig:np_hc1} plots the data collected during the whole duration and \fref{fig:np_hc2_p2} shows the data since October 2019. Barring a few High Courts, \fref{fig:np_hc2_p2} is much better in terms of regular updates than \fref{fig:np_hc1}. In both the plots, the high courts appear in the lexicographic order of their names. 

In \fref{fig:np_hc1}, the graph of Allahabad High Court depicts a decent update culture. There is not much diversion from the best fit line (using linear regression) either. It can be seen that the number of pending cases in Allahabad High Court has been increasing linearly with time. 

Be it \fref{fig:np_hc1} or \fref{fig:np_hc2_p2} Bombay High Court has the poorest record of data update on HC-NJDG among all the high courts. In the whole data collection period, the data has been updated only once. Due to this reason nothing can be said about Bombay High Court reliably. A similar case is with Calcutta High Court. Even though the updates have been frequent, wrong data was uploaded on the portal. The total number of pending cases in Calcutta High Court is more than 250 thousand but the graph shows a different number. Hence, nothing can be said reliably about Calcutta High Court either. For Calcutta High Court, the subsequent analysis is done based on the data available from Calcutta High Court website \cite{calcutta_portal}. For Bombay High Court, the data is not available on its website either, so we had to resort to the Supreme Court annual reports \cite{SCAR}. These are the only two high courts whose data is not taken from HC-NJDG. 

In \fref{fig:np_hc1}, data from Chhattisgarh High Court follows a nice update trend and the best fit straight line looks representative of the increase. Hence, it can be deduced that the number of pending cases in Chhattisgarh High Court has been increasing linearly. The Delhi High Court also has a nice update culture almost throughout the data collection period and the pendency is increasing linearly for this high court too. The updates in Gujarat High Court were not frequent until August 2018. However, after that, the number of cases have been increasing linearly. We consider the best fit line drawn in the figures as representing the rate of increase from this graph. Himachal Pradesh High Court data has seen a surge in the number of cases from October 2019 to March 2020, the best fit line, is with positive slope and hence, the number of pending cases are increasing with time for Himachal Pradesh High Court as well. 

In \fref{fig:np_hc2_p2} represents much better updates of NJDG in the Common High Court for the UT of Jammu \& Kashmir and UT of Ladhakh as well as for Jharkhand High Court. Initial updates of the HC-NJDG data seem erroneous and hence we use \fref{fig:np_hc2_p2} for these high courts. The number of cases in the Common High Court for the UT of Jammu \& Kashmir and UT of Ladhakh follow a linear increase whereas the number of pending cases in Jharkhand High Court are decreasing linearly.

In \fref{fig:np_hc2_p2}, barring a few erroneous updates, Karnataka High Court and Kerala High Court have regularly updated data on HC-NJDG. Again, the best fit straight lines looks quite a good representative of the increase. The pendency for both these high courts is increasing too. \fref{fig:np_hc1} depicts that the champion of updating data on HC-NJDG is Madhya Pradesh High Court. Not even a single outlier. The best fit straight line almost coincides with the data. The number of pending cases at this high court is increasing as well. The next comes Madras High Court, which, doesn't seem to have a good update culture. However, it is good enough to be consistent and is increasing which is the current expected trend in most of the high courts. 

Manipur High Court seems to have reconciled data and hence, we do not consider the data for the whole data collection period but only for the last six months as presented in \fref{fig:np_hc2_p2}. The trend of increasing pendency, however little, can be seen for Manipur High Court as well. 

In \fref{fig:np_hc1}, Meghalaya High Court also seems to have an increasing rate of pending cases. The data points may look erroneous on the first look, however, there is a variation of just 250 cases on the whole scale. So such updates are realistically possible. 

Some kinds of reconciliation seems to have taken place for Orissa High Court as well. Hence, we take the trend from \fref{fig:np_hc2_p2}, which again shows an increasing trend in pendency. 

The updates for Patna High Court, Punjab and Haryana High Court and Rajasthan High Court look reasonable and the best fit seems to be representative of the trend that the pendency is increasing. We take the rate of increase from \fref{fig:np_hc1}. 

Sikkim High Court has very low number of pending cases. So taking last six months of trend may be more beneficial. We see that the pendency is decreasing. So we use \fref{fig:np_hc2_p2} for computing the best fit line, which shows a decrease in the number of pending cases. 

In \fref{fig:np_hc1}, for the hypothetical aggregate of Telangana and Andhra High Court the trend is again a linear increase in the pendency. 

Tripura High Court has done very well since October 2019. There is a close to perfect linear decrease in the number of pending cases \fref{fig:np_hc2_p2}. For Uttarakhand High Court, we consider the best fit line for computing the increase in pendency from \fref{fig:np_hc2_p2}. The figure for Uttarakhand High Court in \fref{fig:np_hc1} is quite unreliable.

Hence, from the above analysis, we can deduce that we have enough data for computing the rate of increase of pendency reliably for 21 high courts, as well as the hypothetical aggregate of Telangana and Andhra High Court. However, the updates for two high courts, viz., Bombay and Calcutta are too unreliable on HC-NJDG to be able to make any conclusions about their pendency statistics. Other sources have been used for inferring their data. 

% Figure environment removed

% \begin{table}[h]
% \centering
% \footnotesize
%     \begin{tabular}{ | l | c | c |}
%     \hline
%     
%     High Court  & Rate of Increase & Source data \\ \hline
%     Allahabad &  & \fref{fig:np_hc1} \\ \hline
%     Bombay &  & None \\ \hline
%     Calcutta &  & None \\ \hline    
%     Delhi &  & \fref{fig:np_hc1} \\ \hline
%     Gauhati &  & \fref{fig:np_hc1} \\ \hline
%     Gujarat &  & \fref{fig:np_hc1} \\ \hline    
%     Himachal &  & \fref{fig:np_hc1} \\ \hline
%     Jammu and Kashmir &  & \fref{fig:np_hc2_p2} \\ \hline
%     Jharkhand &  & \fref{fig:np_hc2_p2} \\ \hline
%     Karnataka &  & \fref{fig:np_hc1} \\ \hline
%     Kerala &  & \fref{fig:np_hc1} \\ \hline
%     Madhya Pradesh &  & \fref{fig:np_hc1} \\ \hline
%     Madras &  & \fref{fig:np_hc1} \\ \hline
%     Manipur &  & \fref{fig:np_hc2_p2} \\ \hline
%     Meghalaya &  & \fref{fig:np_hc1} \\ \hline
%     Orissa &  & \fref{fig:np_hc2_p2} \\ \hline
%     Patna &  & \fref{fig:np_hc1} \\ \hline
%     Punjab and Haryana &  & \fref{fig:np_hc1} \\ \hline    
%     Rajasthan &  & \fref{fig:np_hc1} \\ \hline
%     Sikkim &  & \fref{fig:np_hc2_p2} \\ \hline
%     Telangana and Andhra &  & \fref{fig:np_hc1} \\ \hline    
%     Tripura &  & \fref{fig:np_hc2_p2} \\ \hline
%     Uttarakhand &  & \fref{fig:np_hc2_p2} \\ \hline
%     \end{tabular}
%     \caption{Number of cases filed (monthly) and disposed (monthly) and the number of cases listed (daily) and the number of judges in the high courts in India as on August 20, 2018.}
%     \label{tab:njdg_monitor}
% \end{table}


\subsection{Ratio of Pendency to Judges}

The total pendency, in itself, does not provide any information until the number of judges in the respective high court is also taken into account. This subsection considers the ratio of \emph{pending cases/number of judges} as a parameter for each high court.




\fref{fig:pend_date_r} plots the ratio \emph{pending cases/number of judges} for each high court. The number of pending cases is calculated by taking the pendency on the last day of data collected. The number of judges, however, are taken from \fref{fig:sanctioned_working}. The results are plotted in the descending order of the ratio so calculated. This graph provides the distribution of workload on each high court and judges thereof. The blue dots show the ratio \emph{pending cases/working strength of judges} for the average working strength of each high court. For example, Rajasthan High Court has the maximum value of 19,374 pending cases for each judge whereas Sikkim High Court has the minimum ratio which is 78. Hence, statistically we can say that a judge in Rajasthan High Court has almost 250 times more load than a judge in Sikkim High Court. It can be seen that the situation is similar for most of the high courts. The mean of this ratio is 6908, i.e., the national average of the number of pending cases per judge. It signifies that on an average each sitting judge of the high courts in India needs to dispose 6908 cases to reduce pendency to zero, provided no more cases are filed. It also means that the judges in some of the high courts are insanely overburdened. In the interest of justice, urgent appointments are required so that the case load may be shared. Hence, the number of pending cases per judge is huge and the numbers are so high that it would not be unfair to state that they are simply beyond the capacity of the current number of working judges.
% Some input from technology is required to handle such huge numbers and ease the tasks of the judges.


