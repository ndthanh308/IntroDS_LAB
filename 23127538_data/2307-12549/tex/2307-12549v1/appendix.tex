% ~\\
% ~\\
% ~\\

\section{Appendix}
We present \fref{fig:years_combat_real} and \fref{fig:req_individual} in the form of tables. 
\begin{table}[h!]
\centering
\footnotesize
% \small
    \begin{tabular}{ | l | c | c | c | c | c |}
    \hline
    & High Court & 10 Years & 20 Years \\ \hline
    1& Himachal & 102  & 150 \\ \hline
    2& Madras & 83 & 113 \\ \hline
    3& Uttarakhand & 64 & 81 \\ \hline
    4& Rajasthan & 39 & 59 \\ \hline
    5& Gauhati & 38 & 53 \\ \hline
    6& T and A & 27 & 39 \\ \hline
    7& J and K &24 &32 \\ \hline
    8& Bombay &23 &31 \\ \hline
    9& Calcutta &22 &27 \\ \hline
    10& Kerala &19 &26 \\ \hline
    11& Karnataka &19 &28 \\ \hline
    12& MP &18 &26 \\ \hline
    13& Chhattisgarh &14 & 21  \\ \hline
    14& Manipur & 14 &22  \\ \hline
    15& Allahabad &13 &18 \\ \hline
    16& Delhi &11 &17 \\ \hline
    17& Gujarat &10 &14 \\ \hline
    18& Orissa &9 &14 \\ \hline
    19& Patna &9 &13 \\ \hline
    20& Jharkhand &9 &11 \\ \hline
    21& Meghalaya &7 &10 \\ \hline
    22& Sikkim &6 &6 \\ \hline
    23& Tripura &2 &2 \\ \hline
    24& P and H & - & - \\ \hline
    \end{tabular}
    \caption{Used to create \fref{fig:years_combat_real}. The number of years required to clear the backlog if the sanctioned strength of the high courts is reached in ten and twenty years. Note that for Punjab and Haryana High Court, the backlog can never be cleared if the sanctioned strength is not increased. }
    \label{tab:years}
\end{table}


\begin{table}[t]
\centering
\footnotesize
% \small
    \begin{tabular}{ | l | c | c | c | c | c |}
    \hline
    & High Court & 5 Years & 15 Years & Sanctioned & Working \\ \hline
    1& P and H & 549 & 310 & 85 & 52 \\ \hline
    2& Allahabad & 277 & 161 & 160 & 102 \\ \hline
    3& Calcutta & 273 &111 & 72 & 39 \\ \hline
    4& Bombay & 197 & 119 & 94 & 69\\ \hline
    5& T and A & 169 & 96 & 61 & 27 \\ \hline
    6& Madras & 160 & 107 & 75 & 58\\ \hline
    7& Karnataka & 138 & 81 & 62 & 33 \\ \hline
    8& Rajasthan & 124 & 80 & 50 & 25 \\ \hline
    9& MP &106 & 65 & 53 & 33\\ \hline
    10& Kerala & 92 & 56 & 47 & 34 \\ \hline
    11& Delhi & 82 & 60 & 60 & 37 \\ \hline
    12& Gujarat & 64 & 52 & 52 & 28 \\ \hline
    13& Patna & 61 & 53 & 53 & 29 \\ \hline
    14& J and K & 51 & 26 & 17 & 9 \\ \hline
    15& Gauhati & 46 & 31 & 24 & 19 \\ \hline
    16& Chhattisgarh &35 & 23 & 22 &15  \\ \hline
    17& Orissa & 33 & 27 & 27 & 14\\ \hline
    18& Jharkhand & 32 & 25 & 25 & 19 \\ \hline
    19& Himachal & 29  & 20 & 13 & 9\\ \hline
    20& Uttarakhand & 25 & 15 & 11 & 9 \\ \hline
    21& Manipur &7 & 5 & 5 & 4  \\ \hline
    22& Meghalaya &4 & 4 & 4 & 2 \\ \hline
    23& Tripura &4 & 4 & 4 & 3 \\ \hline 
    24& Sikkim &3 &3 &3 & 3\\ \hline \hline
    25& Total & 2561 & 1534 & 1079 & 672 \\ \hline
    \end{tabular}
    \caption{Used to create \fref{fig:req_individual}. The number of judges required in each high court to clear the backlog in five or fifteen years. The third column is the current sanctioned strength of the high courts and the last column shows the average number of working judges in the high courts for 22 months starting June 2018 to March 2020.}
    \label{tab:time}
\end{table}
