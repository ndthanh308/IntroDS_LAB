\section{Scope and Related Work}
\label{sec:rw}
% In this section, we state the scope of our work. We also compare our work with major studies on pendency in courts in India. 

% \subsection{Scope of the work}
Before we proceed any further to discuss the technical findings of the paper, we first understand the scope and the limitations of our work. There have been many news reports, articles and studies on pendency in Indian courts \cite{arrears}  \cite{over_22lakh} \cite{jkpile} \cite{verma2018courts}. In this paper, instead of studying the pending cases in all the courts of India, we decided to limit ourselves to only the high courts. This limits the number of court complexes in our study to just 39 without decreasing the complexity of the problem as the high courts are more clogged with cases than subordinate courts. Moreover, the improvements suggested in this paper are relatively easier to implement in high courts than in lower courts because of better infrastructure and budget available. Thus, we would concretely know where things can be improved. Hence, throughout this paper, we have concentrated on the pendency in high courts rather than the subordinate courts.

In our study, we have not taken input from any real person. No judge, advocate, litigant or court staff was interviewed. This may have both positive and negative impact. Intervention of court staff and those who are involved in updating NJDG may have provided more insights to interpret our results. On the other side of it, their views might have biased our results. So we decided to leave it for future because we wanted our assessment to be purely technical and statistical based only on the observations made from the data that we have collected from NJDG. 


A study by Alok Prasanna Kumar has used number of the District and Magistrate courts, collected from the National Judicial Data Grid as of 18 March, 2016 \cite{comparative}. There are studies advocating to decrease the holidays available in judiciary\cite{funda_reason}. There are studies conducted by the Department of Justice as well\cite{dojeval}. While this study is very comprehensive, the results reported are different from ours and the parameters considered for evaluation are different as well. Various studies including \cite{vidhi1}, have conducted research on e-Court policies. The importance of data analysis of judicial data and the role of computer science is also suggested in \cite{judmess}. The Department of Justice also encourages research conducted on judicial reforms by means of funding \cite{doj_projects}. Another rich source of information on pending cases are the annual reports published by the Supreme Court of India \cite{SCAR}. 

The most relevant work that can be compared with our work is Law Commission of India report of July 2014 \cite{lci_report245}. The availability of data was a major concern for the authors of the report. They studied data on pendency at the end of years from 2002 to 2012. However, the primary focus of their analysis is the courts that are subordinate to the jurisdiction of high courts. Moreover, the pendency figures have more than doubled now compared to 2012. Hence an understanding of the rate of increase of cases is crucial in studying pendency. We make a clear distinction from the report by exclusively studing the high court data, collecting data over 229 days, and counting the contribution of each and every data point by using linear regression for the analysis.

Other relevant related work in this area is the Daksh report on the state of the Indian Judiciary \cite{daksh_report}. Their approach, however, is very different from ours. They have conducted a ground level research by surveying and obtaining the first hand experience of the litigants and other stake holders. Our work, on the other hand, relies completely on the data provided by the National Judicial Data Grid for High Courts (HC-NJDG). 

Hence, a lot of studies agree that the judicial throughput has to be increased. Either the number of vacations may be reduced or the number of judges may be increased. 
% Our results are however, very similar in some areas. For example, the Daksh report has also found that there is a lack of uniformity in the available data. There is no unanimous agreement on the number of judges/courts in the lower judiciary in the country. Our study claims that there are discrepancies even in the number of judges reported on HC-NJDG, let alone the subordinate judiciary. 

The issue has been of utmost importance to all the Chief Justices of India \cite{ranjan_plan}\cite{bobde_speedy}. Hence, a lot of research needs to be conducted in the area so as to help solve a crucial problem faced by Indian Judiciary. 


% \subsection{Existing literature in pendency in India}

