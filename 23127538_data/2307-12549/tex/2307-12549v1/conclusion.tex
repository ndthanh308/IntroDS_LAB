\section{Conclusion}
\label{sec:conclusion}


The problem of pending cases in India has taken an unimaginable form. In high courts alone, close to 4.5 million cases are pending of which around 20\% are pending for more than 10 years. We use the data collected from HC-NJDG portal for a period of more than two and a half years to study the trends in the pendency in the high courts. We realize that the pending cases are increasing for almost every high court. We use linear regression to capture the rate of increase of pendency in the high courts. We also make use of the data on disposed cases from the HC-NJDG portal to compute the number of cases disposed by each high court judge per day and use these statistics to estimate the number of years required to clear the pendency in the high courts. The number of pending cases is well beyond the capacity of the number of judges currently working in the high courts. Hence, the number of judges should be increased by taking necessary legislative measures.

The energy and efforts put in e-Courts project, NJDG in particular, must continue for few more years, if not decades, to see the real impact. Hence, more aid of ICT in judiciary must be sought to reduce the pendency of millions of cases and the use of artificial intelligence should be more than just welcome. 


% \begin{enumerate}
% \item Maintaining NJDG flawlessly is a daunting task. Multiple levels of checks are required to ensure that the data provided on it is free from errors. 
% \item Timely updates are an issue. Unless the updates on the portal are regular, it cannot be used for the envisaged purpose, which is to make it useful for reducing the pendency.
% \item The gap between the number of disposed cases and the number of listed cases is huge. A reduction in the number of listed cases may help all the stakeholders without compromising on the quality of justice. On the contrary, it may improve the quality of life for all the stakeholders.
% \item The number of cases filed by senior citizens and women are not at all proportional to their population. This is even more true when it comes to the criminal cases filed by women and senior citizens. Not many high courts are updating this field which may also result in the low numbers observed in the current data.
% \item There are few undefined fields on the HC-NJDG portal. It will be easier for readers to interpret the data if the portal is backed up by a documentation. 
% \item We have also estimated the number of years to elapse to nullify pendency provided that the working strength of each high court is same as its approved strength. As for the current working strength, pendency is only increasing.
% \end{enumerate}

% In the end, we hope that our work has served the role of bug reports for NJDG as well as helping in curbing pendency in high courts.

% In future, we would like to strengthen our results by studying lower courts on the same scale. Our goal is to publish similar results related to each state and Union Territory that has its presence on NJDG. In order to improve upon the state of the art, relatively new computer science areas in artificial intelligence like deep learning, natural language processing, etc have to be applied to better utilize the ICT infrastructure procured by the courts. The fundamental problem that the judiciary in India is currently suffering from is the problem of scalability. Deep learning algorithms have proven to be very useful in improving scalability where human like tasks need to be done. Its applications to reducing pendency might be one such area.



% We conclude this paper by mentioning that apart from the discomfort that the delay causes, improper scheduling has been leading to a huge loss of revenue to the litigants. In particular, in some cases the litigants travel hundreds of kilometers, if not thousands, to attend the hearing but the adjournments spoil their effort and spirit. A study states that around 0.48\% of the national GDP is wasted in travels for hearing that never take place\cite{daksh_report}. Thus, more scientific ways have to be designed to deal with this problem of scheduling of cases in the courts that will ease the pressure of the judges as well as the litigants and all the other stakeholders involved.
