% NWSguide.tex
% v1.0, released 14 Dec 2020
% Copyright 2020 Cambridge University Press
\documentclass{nws}

\usepackage{amsmath}
\usepackage{graphicx}

%Added packages and options
\usepackage{hyperref}
\hypersetup{colorlinks=true, linkcolor=blue, filecolor=magenta, urlcolor=blue, citecolor=blue, pdfauthor=Zachary P. Neal}
\newcommand{\doi}[1]{\href{http://dx.doi.org/#1}{https://doi.org/#1}}
\newcommand{\DOIprefix}{}
\bibliographystyle{nwslike_cust}  %Updated from OUP
\usepackage{lastpage}

\begin{document}
\lefttitle{Cambridge Authors}
\righttitle{Network Science}
\papertitle{Article}
\jnlPage{1}{\pageref{LastPage}}
\jnlDoiYr{2023}
\doival{10.1017/nws.xxxx.xx}
\papertitle{Original Article}
\title{Inferring social networks from observed groups}
%\author{~}
\author{Zachary P. Neal$^1$}
\affil{$^1$Michigan State University, USA (e-mail: {zpneal@msu.edu})}

\begin{abstract}
Collecting social network data directly from network members can be challenging. One alternative involves inferring a social network from individuals' memberships in observed groups, such as teams or clubs. Through a series of simulations, I explore when we can expect such inferences to be accurate. I find that an unobserved network can be inferred with high accuracy under a range of circumstances. In particular, I find that social networks inferred from observed groups are more accurate when (1) the unobserved network has a small world structure, (2) the groups are generated by a shuffling or agglomerative process, (3) a large number of groups are observed, and (4) the observed groups' compositions are tightly coupled to the unobserved network's structure. These findings offer guidance for researchers seeking to indirectly measure a social network of interest through observations of groups.
\end{abstract}
\begin{keywords}
backbone; bipartite; indirect measurement; inference; simulation; two-mode
\end{keywords}

\maketitle

\section{Introduction}
Collecting social network data directly from network members, for example through surveys or interviews, can be challenging due to the resource intensive nature of the data collection and the risks to data quality from missingness, reporting errors, and reactivity \citep{adams2020gathering,marsden2011survey}. These challenges have led social network researchers to seek alternate, indirect measurement methods. One common alternative involves inferring an unobserved social network from observed groups such as club membership, event participation, or simply spatial proximity \citep[e.g.,][]{breiger1974duality,newman2004coauthorship, mizruchi1996interlocks,andris2015rise,daniel2019}. However, little is known about the circumstances under which a social network inferred from observed groups accurately captures the unobserved social network of interest. That is, \textit{when can we accurately infer social networks from observed groups}?

To answer this question, I adopt a recent framework linking data and theory in network science \citep{peel2022statistical} to test the accuracy of inferences derived from observed groups under a range of circumstances. In a series of simulations, I experimentally vary the structure of the unobserved network, the generative process whereby groups emerge from the network, the extent to which group composition is coupled to the unobserved network, and the number of groups that are observed. I find that for many network structures and generative processes, the unobserved social network can be inferred with high accuracy from even a modest number of observations that exhibit only modest coupling. However, there are also some cases where the unobserved social network cannot be accurately inferred. Therefore, although inferring a social network from observed groups can offer a promising alternative to direct measurement, such inferences vary in accuracy and should be used with caution.

The remainder of the paper is organized in four sections. In the background section, I review the challenges with direct social network measurement and the potential of inferences from observed groups as a possible solution. In the methods section, I describe a simulation experiment designed to evaluate the accuracy of a network inferred from observed groups, as well as the generative models used to simulate these data and the backbone model used to draw these inferences. In the results section, I report the accuracy of networks inferred under experimentally-varied conditions, highlighting when such inferences are and are not accurate. Finally, in the discussion section, I identify opportunities for future research to improve the accuracy of these inferences, and offer recommendations for researchers wishing to infer social networks from observed groups.

\section{Background}
Like other types of data in the social sciences, when one is interested in a social network, it is common to collect information about social relationships directly from the network's members. Indeed, if we want to know who your friends are, for example, there is a strong intuitive appeal to simply asking you ``Who are your friends?'' Although there are many variations, direct collection of social network data typically takes place via a survey or interview, which includes one or more `name generator' questions like the one above \citep{adams2020gathering,marsden2011survey}.

Although the direct collection of social network data has a long history and the advantages of simplicity and directness, it can also present a number of challenges \citep{adams2020gathering,marsden2011survey}. First, direct data collection can be resource intensive. It requires a substantial time commitment from the respondent to provide the information, can require a substantial time commitment from the researcher when the data is collected in person, and can require substantial financial resources when participation incentives are offered. Second, self-reported network data is subject to reporting errors of omission (not reporting a tie that exists) and comission (reporting a tie that does not exist), which may be unintentional (recall error) or intentional \citep[misrepresentation;][]{wang2012measurement}. Third, data obtained via direct interaction with the respondent can be subject to reactivity bias. For example, the very act of asking a respondent to identify their friends may lead them to reflect on the nature of friendship, whether their number of friends is too high or too low, and other issues that may influence their response. Fourth, some participants may lack the capacity to provide the requested information, for example because they are too young \citep{neal2020systematic} or are not human \citep{krause2009animal}. Finally, because direct data collection requires network members' participation, low participation or response rates can introduce missingness against which many network analytic techniques are not robust \citep{kossinets2006effects}.

The severity of these challenges varies by context, and strategies exist for overcoming them, so they are not insurmountable obstacles. However, they have led network researchers to look for indirect methods of collecting social network data. Methods have been proposed for inferring social networks from observations of innovation adoption times \citep{gomez2012inferring} or interpersonal communication events \citep{de2010inferring}. However, one of the most widely used indirect methods involves inferring an unobserved social network from observed groups. This approach was formalized by \cite{breiger1974duality}, who illustrated how an interpersonal network could be derived from data about individuals' group affiliations, first using hypothetical data then empirical data on 18 womens' attendance at 14 social events. Briefly, if $\mathbf{A}$ is a binary affiliation matrix recording individuals' group affiliations such that $A_{ik} = 1$ if person $i$ is affiliated with group $k$, then $\mathbf{AA'}$ yields a weighted network adjacency matrix $\mathbf{N}$ where $N_{ij}$ records the number of groups with which both $i$ and $j$ are affiliated. Relying on the logic that if two people belong to many of the same groups or participate in many of the same events, then they likely interact, this approach proposes that (some suitably normalized version of) $\mathbf{N}$ can be viewed as a social network inferred from observed groups.

This approach has been used to indirectly to measure social networks in a wide range of contexts, and in some fields has become the \textit{de facto} standard approach. Unobserved networks of scientific collaboration are routinely inferred from researchers' observed paper authorships \citep[e.g.,][]{newman2004coauthorship}, unobserved networks of corporate executives are routinely inferred from their observed board memberships \citep[e.g.,][]{mizruchi1996interlocks}, unobserved networks of political alliance are routinely inferred from lawmakers observed memberships in voting blocs \citep[e.g.,][]{andris2015rise}, and unobserved social networks are routinely inferred from young childrens' play groups \citep[e.g.,][]{daniel2019}.

Despite its widespread use, this strategy for inferring social networks from observed groups has its own challenges. The mathematical transformation of person-by-group affiliation data into person-by-person network data, known as bipartite or two-mode projection, always yields networks with specific topological characteristics. For example, networks generated via projection are dense and have high levels of clustering, which occurs because ``a huge number of links [is] induced by the projection \citep[][p. 34]{latapy2008basic}. Backbone models offer one way to overcome this challenge by identifying and preserving only the statistically significant edges in a projection, thereby aiming to infer only real social connections from observed groups. Multiple such backbone models exist, however in this paper I focus on the stochastic degree sequence model \citep[SDSM;][]{neal2014backbone}, because it is the fastest and most flexible \citep{neal2021comparing}, and because there is preliminary evidence that it can accurately infer networks from observed groups \citep{neal2022inferring,gomes2022network}.

% Figure environment removed

But, what does it mean to accurately infer a network? \cite{peel2022statistical} recently offered a framework describing the relationships among unobserved networks, observed data, and inferred networks that offers an answer. Figure \ref{fig:framework} illustrates a portion of this framework, adapted to the present study; the letters identify factors that may impact accuracy. In a given setting, there exists an unobserved social network of interest. Although we cannot observe this network, we may still have some expectations about its likely structure (A), for example that because it is a social network it will exhibit small world properties. Through an unknown generative process (B), which may be tightly or weakly coupled to the network (C), groups emerge from this network. Although we do not know what the generative process is, we may still have some expectations about its likely mechanisms, for example, that old team members reshuffle to form new teams or that people agglomerate to form clubs. We can only observe the outcome of this generative process, by observing individuals' (black circles) membership in some number of groups (white squares, D). From these observed groups, we can attempt to infer a social network using a backbone extraction model such as SDSM. In this study, I am interested in understanding when the inferred network is similar to the unobserved network, and thus provides an accurate inference.

\section{Methods}
\subsection{Experimental design}
The accuracy of an inferred network may depend on the four labeled factors that appear in Figure \ref{fig:framework}. Table \ref{tab:design} summarizes a factorial experiment designed to evaluate how these factors influence the accuracy of network inference.

\begin{table}[]
\begin{tabular}{ll}
\hline
Factor & Levels \\
\hline
(A) Unobserved network & Random, Small World, Scale Free \\
(B) Generative Process & Shuffling, Agglomerative, Spatial \\
(C) Coupling & 0.6, 0.7, 0.8, 0.9, 1 \\
(D) Number & 50, 100, 500, 1000, 5000 \\
 & \\
Replications per condition & 50 \\
Outcome & Mean Matthews Correlation Coefficient \\
\hline
\end{tabular}
\vspace*{2mm}
\caption{Summary of factorial experimental design}
\label{tab:design}
\end{table}

First, accuracy may depend on the structure of the unobserved network (Factor A). In this experiment, I consider three network structures: random, small-world, and scale-free. I generate 50-node random networks using the Erd\H{o}s-R\'enyi model \citep{erdosrenyi}, where the probability of an edge is 0.08. I generate 50-node small-world networks using the Watts-Strogatz model \citep{watts1998collective}, where each node in a ring lattice is initially connected to its four nearest neighbors, then edges are re-wired with probability 0.05. Finally, I generate 50-node scale-free networks using the preferential attachment model \citep{barabasi1999emergence}, where two edges are added in each step. These settings all yield networks containing 50 nodes and about 100 edges, so network size and density is held constant.

Second, accuracy may depend on the generative process whereby groups form (Factor B). In this experiment, I consider three generative processes: shuffling, agglomerative, and spatial. A shuffling process views groups as emerging from a network when members of a prior group are shuffled with non-members to form a new group. An agglomerative process views groups as emerging from a network as an initial set of individuals attempts to recruit contacts to join the group. Finally, a spatial process views groups as emerging from a network when a group attempts to recruit members from a region of latent social space. I describe the operationalization of these generative processes into specific models in section \ref{sec:models}.

Third, accuracy may depend on the extent to which the composition of the observed groups is driven by or `coupled' with the unobserved network (Factor C). When the groups are tightly coupled, the memberships of the observed groups are driven strictly by the unobserved network's structure, and therefore are highly informative about the network. In contrast, when the generative process is loosely coupled, the memberships of the observed groups are driven only partly by the unobserved network's structure, and therefore are less informative about the network. In the experiment, I consider five levels of coupling, ranging from perfectly coupled (1) to nearly uncoupled (0.6). I describe the role of the coupling parameter in the generative models in section \ref{sec:models}

Finally, accuracy may depend on the number of groups that are observed (Factor D). Each observed group provides information upon which inferences can be based. Therefore, other things being equal, observing more groups should yield more accurate inferences. In the experiment, I consider five numbers of observed groups, ranging from 50 to 5000.

As Table \ref{tab:design} summarizes, these factors define a 3 structures $\times$ 3 generative processes $\times$ 5 levels of coupling $\times$ 5 numbers of observed groups factorial experiment, with 225 experimental conditions. Within each condition, I perform 50 replications and measure the accuracy of inferences in the condition using the mean Matthews correlation coefficient (MCC). The MCC is a special case of the Pearson correlation coefficient when both variables are binary, and ranges from 1 for perfect accuracy, to 0 when the inference is no better than random. It is more robust than other metrics for quantifying accuracy in binary classification tasks \citep[i.e., is an edge present or absent for this dyad;][]{chicco2020advantages}, however alternative measures of binary accuracy including Cohen's $\kappa$ and the jaccard coefficient exhibit identical patterns in the results.

The code necessary to replicate all results reported below is available at \url{https://osf.io/6vcxa/?view_only=4af927d678b0434fb4d4cef58d03b083}.
%\url{https://osf.io/6vcxa}.

\subsection{Group generative processes}
\label{sec:models}
Although much is known about how social relationships might form from groups, relatively little is known about how groups form from social networks. However, the basic process was sketched by \cite{feld1981focused} in his discussion of `foci,' which \cite{neal2023duality} formalized in three generative models: teams, clubs, and organizations. These models describe how a group might emerge from the relationships in a social network, and can be used to generate a two-mode network from a given one-mode network. Each model's operation is governed by a tuning parameter $p$ that controls how tightly coupled the groups are to the one-mode network's structure.

The teams model is an adaptation of a more complex, dynamic model of team formation \citep{guimera2005team}, and describes a shuffling process. In this model, cliques in a network are viewed as describing prior teams. A new team forms from one of these prior team. The new team contains the same number of members as the prior team from which it forms, however these positions are filled randomly by selecting prior team incumbents with probability $p$, and newcomers from elsewhere in the network with probability $1-p$. The tuning parameter $p$ controls how tightly coupled the composition of new teams is to the network. New teams are simply network cliques when $p = 1$, while new teams are random sets of network members when $p = 0.5$ because its members are chosen from prior members and newcomers with equal probability.

The clubs model draws on prior empirical findings about when an individual chooses to join a social group \citep{backstrom2006group,schaefer2022youth}, and describes an agglomerative process. In this model, cliques in the network are viewed as nascent clubs, that is, groups of people who might become a club if they can successfully recruit others. Recruitment occurs iteratively as existing members attempt to recruit their own contacts. Following prior empirical findings, in each iteration a given recruit chooses to join if the new club would have a minimum density of $p$. The tuning parameter $p$ controls how tightly coupled the composition of new clubs is coupled to the network. New clubs are simply network cliques when $p = 1$, while new clubs are increasingly decoupled from network cliques when $p < 1$ because new members may join clubs composed of even random sets of members.

The organizations model is a partial formalization of the `Blau space' model of organizational recruitment \citep{mcpherson1983ecology}, and describes a spatial process. In this model, the dimensions of a latent space into which the network can be embedded are viewed as describing socio-demographic characteristics of its members. A new organization forms by attempting to recruit members from a niche within this space. Members are successfully recruited from inside its niche with probability $p$, and from outside its niche with probability $1-p$. The tuning parameter $p$ controls how tightly coupled the composition of new organizations is to the network's structure. New organizations are composed of people who are directly or indirectly connected to one another, and thus proximate in latent space, when $p = 1$. In contrast, new organizations are composed of random sets of network members when $p = 0.5$ because the organization recruits members from inside and outside its niche with equal probability.

\subsection{Backbone extraction}
The literature contains many backbone extraction algorithms designed to identify and retain the most significant edges in a dense or weighted network, thereby yielding a sparse and unweighted `backbone' \citep[e.g.,][]{neal2022backbone,gomes2022network,hamann2016structure}. For the purposes of inferring a social network from observed groups, only those algorithms designed to extract the backbone from a two-mode projection are relevant. Of such algorithms, only two are accompanied by evidence that they can recover unobserved networks: the stochastic degree sequence model (SDSM) and fixed degree sequence model \citep[FDSM;][]{neal2021comparing,neal2022inferring,gomes2022network}. Both algorithms are candidates for inferring a network from observed groups, however I focus on SDSM because the FDSM is often too computationally intensive to be useful in practice.

The SDSM preserves an edge in a two-mode projection if its observed weight is statistically significantly larger than its expected weight under a null model. In the context of inferring social relationships from observed group memberships, the SDSM infers that $i$ and $j$ have a relationship if they are observed to be group co-members statistically significantly more often than if they each joined groups randomly. The specification of the null model and definition of `joined groups randomly' is what distinguishes SDSM from FDSM. Under SDSM, the null model describes an alternative world in which (a) \textit{on average} each person is a member of the same number of groups as their observed number of memberships, and (b) \textit{on average} each group's number of members is the same as its observed number of members, but (c) which people belong to which groups is otherwise random. The `on average' specification in these null model constraints allows SDSM to directly compute an edge's exact p-value, whereas FDSM omits this specification and requires computationally intensive simulations to estimate an edge's p-value.

\section{Results}
Figure \ref{fig:results} reports the results of all experimental conditions. The columns of panels correspond to the structure of the unobserved network, which could be either random (panels A, D, and E), small world (panels B, E, and H), or scale free (panels C, F, and I). The rows of panels correspond to the generative process whereby groups form, which could be either shuffling using the teams model (panels A-C), agglomerative using the clubs model (panels D-F), or spatial using the organizations model (panels G-I). Within each panel, the columns correspond to the level of coupling between the observed groups and unobserved network, while the rows correspond to the number of groups that are observed and from which inferences are drawn. Each square represents a single experimental condition and reports the mean MCC over 50 replications, with lighter shades of blue representing higher MCC values and thus conditions under which inferences are made with greater accuracy.

% Figure environment removed

There is substantial variation in the accuracy with which a social network can be inferred from observed groups. In many cases the unobserved social network can be inferred with perfect accuracy (MCC = 1), while in some cases the inference is no better than random (MCC = 0.05). Table \ref{tab:results} reports the standardized and unstandardized coefficients in a linear regression predicting MCC as a function of the four experimental factors. The lowest accuracy network structure (i.e. scale free) and generative process (i.e. spatial) are used as reference categories. The statistical significance of coefficients is not reported because the data were generated from an arbitrarily large number of numerical simulations (here, $N = 11250$), which guarantees that all coefficients are statistically significant by any conventional test. These coefficients, together with the patterns visible in Figure \ref{tab:results}, highlight which factors are associated with greater accuracy.

\begin{table}
\centering
\begin{tabular}{lll}
\hline
Factor & \textit{b} & $\beta$ \\
\hline
Intercept & -0.08 & -- \\
Structure: Random & 0.05 & 0.17 \\
Structure: Small World & 0.18 & 0.59 \\
Process: Shuffling & 0.41 & 1.37 \\
Process: Agglomerative & 0.43 & 1.44 \\
Number of groups & 5.7e-5 & 0.35 \\
Coupling & 0.25 & 0.12 \\
 & & \\
R$^2$ & \multicolumn{2}{c}{0.64} \\
\hline
\end{tabular}
\vspace*{2mm}
\caption{Regression predicting the accuracy of a social network inferred from observed groups, as a function of (a) structure of the true network, (b) group generative process, (c) number of observed groups, and (d) group coupling.}
\label{tab:results}
\end{table}

First, inferences are more accurate when they are based on more observed groups and on groups that are more tightly coupled to the unobserved network structure. This is an expected effect because inferences based on more, and higher-quality, information are more accurate in any context.

Second, inferences about unobserved networks with a small world structure are most accurate, followed by inferences about unobserved networks with a random structure, while inferences about unobserved networks with a scale free structure are least accurate. This effect may occur because small world networks, and to a lesser extent random networks, exhibit clustering and triadic closure, which two-mode projections can readily capture. In contrast, scale free networks exhibit little clustering or triadic closure, which is difficult for two-mode projections to capture.

Finally, inferences are more accurate when they are based on observed groups generated by a shuffling or agglomerative process than by a spatial process. This effect may occur because the shuffling process described by the teams model, and the agglomeration process described by the clubs model are both clique-based group formation processes, and two-mode projection is a clique-building mathematical transformation \citep{latapy2008basic}. In contrast, the latent space recruitment process described by the organizations model does rely on cliques.

The coefficient estimates in Table \ref{tab:results} mirror the patterns visible in Figure \ref{fig:results}. When a shuffling or agglomerative process generated the observed groups, social networks can be inferred with high accuracy by observing many tightly coupled groups (MCC = 0.65 - 1), and can still be inferred with modest accuracy by observing few weakly coupled groups (MCC = 0.17 - 0.72). In contrast, when the observed groups are generated from a scale free network by a spatial process, social networks cannot be accurately inferred by observing groups (MCC = 0.05 - 0.29).

\section{Discussion}
There are a number of practical and methodological challenges associated with collecting social network data directly from network members. These challenges have led network researchers to identify alternative, indirect data collection methods. Among the most widely used methods involves attempting to infer an unobserved social network from observed groups, such as club memberships or event participations. Although this approach is widely used, little is known about the circumstances under which a social network can accurately be inferred from such information.

In this paper, I conducted a simulation experiment to examine the impact of four factors on the accuracy of social networks inferred from observed groups: the structure of the unobserved network, the process responsible for generating the observed groups, the level of coupling between the groups and network, and the number of groups observed. The experimental results suggest that inferences are most accurate when drawn about an unobserved network with a small world structure, and when drawn from many observed groups, groups whose membership is tightly coupled to the network's structure, and groups generated by a shuffling or agglomerative process.

\subsection{Recommendations for inferring social networks}
These findings are abstract, but they can be used to inform practical recommendations for researchers wishing to infer social networks from observed groups.

First, and perhaps most obviously, researchers should \textit{use as many observed groups as possible when making inferences} because inferences drawn from more data are more accurate. Because the null model underlying SDSM already takes into account the fact that groups vary in size, it is not necessary to omit especially large or small groups.

Second, researchers should \textit{consider the extent to which observed groups are driven by the unobserved network} because inferences are more accurate when drawn from groups whose composition is tightly coupled to the network. Of course, this is often unknown, however some types of groups are likely to exhibit tighter coupling than others. For example, consider the case of inferring a social network among high school students. When a student decides whether to join a particular club (one type of observable group), this decision is likely heavily influenced by their network contacts \citep{schaefer2022youth}. In contrast, when a student decides whether to be a customer at a particular convenience store (another type of observable group), this decision may be less influenced by the network \citep{browning2017ecological}. Therefore, observations of club co-memberships may yield more accurate network inferences than observations of store co-patronages.

Third, researchers should \textit{consider the likely structure of the unobserved network} because inferences about small world networks are more accurate. Of course, the structure of the unobserved network is unknown. However, social networks often exhibit a small world structure, so in practice inferring a social network from observed groups will often be appropriate. However, if the unobserved social network is believed to have a scale free structure characterized by branching and intransitivity, for example a reporting hierarchy, inferences should be made with caution.

Finally, researchers should \textit{consider the generative process leading to the observed groups} because inferences drawn from groups that form via shuffling and agglomerative processes are more accurate. Of course, the generative process at play is unknown. However, as \cite{peel2022statistical} explain, researchers must still make informed assumptions about the generative processes that underlie their observed data. In this context, the type of groups observed may offer clues about the generative process. For example, small working groups that repeatedly form in a relatively closed setting such as teams may form through a process of old and new members shuffling as skills are matched to task demands. In contrast, larger social groups focused on shared interests such as clubs may grow over time as existing members recruit new members.

\subsection{Limitations and future directions}
These recommendations offer researchers preliminary guidance on when they might infer a social network from observed groups, and whether they should regard the inferred network as accurate. However, this measurement approach and these findings have two notable limitations. First, unlike social networks measured directly by asking network members a name generator question, where the type of relationship being measured is explicitly specified by the question (e.g., who are your \textit{friends}), the specific type of relationship that can be inferred from observed groups is unknown. Consider a social network inferred from observations of groups of legislators sponsoring bills, as is common in political network research. It is unclear if the relationships in such a network capture strategic political alliances, or friendship, or simply ideological similarity \citep{neal2022constructing}. This limitation likely cannot be overcome with better methods, and instead requires better theory. Therefore, in cases where a social network is inferred from observed groups, the researcher must offer a theory or rationale that sharing membership in a particular type of group is a valid indicator of a particular type of social relationship.

Second, although there exists some evidence that backbone models such as SDSM can recover a known empirical network \citep{neal2022inferring,gomes2022network}, the experiments conducted here rely on simulated not empirical data. Therefore, these findings require further verification in real-world data. While this is an important direction for future research, the data necessary for such a validation is rare. Specifically, it would require from a directly measured one-mode social network \textit{and} a set of groups independently observed \textit{in the same setting}.

These limitations notwithstanding, this study confirms that social networks can be inferred from observed groups, which is reassuring because the practice is already widespread. At the same time, it also highlights that such inferences are not always accurate, and identifies some of the factors that influence accuracy. In some cases these factors are easy to determine (e.g., how many groups did I observe), while in other cases they require careful consideration of the context (e.g., what generative process is responsible for the groups I observe). By knowing what factors are relevant, these findings and the associated recommendations provide researchers with practical guidance on using observed groups to infer social networks of interest that would be difficult to measure directly.

\section*{Data availability statement}
The code necessary to replicate these findings is available at \url{https://osf.io/6vcxa/?view_only=4af927d678b0434fb4d4cef58d03b083}.
%\url{https://osf.io/6vcxa}.

\section*{Funding statement}
This work was supported by the National Science Foundation (\#2016320 and \#2211744).

\section*{Competing interests statement}
None.

\begin{thebibliography}{}

\bibitem[{a}dams, 2020]{adams2020gathering}
{a}dams, j. (2020).
\newblock {\em Gathering social network data}.
\newblock Sage.

\bibitem[Andris et~al., 2015]{andris2015rise}
Andris, C., Lee, D., Hamilton, M.~J., Martino, M., Gunning, C.~E., \& Selden,
  J.~A. (2015).
\newblock The rise of partisanship and super-cooperators in the us house of
  representatives.
\newblock {\em PLOS One}, {\it 10}(4), e0123507.

\bibitem[Backstrom et~al., 2006]{backstrom2006group}
Backstrom, L., Huttenlocher, D., Kleinberg, J., \& Lan, X.
\newblock Group formation in large social networks: membership, growth, and
  evolution.
\newblock In {\em Proceedings of the 12th ACM SIGKDD international conference
  on Knowledge discovery and data mining} (2006), pp. 44--54.

\bibitem[Barab{\'a}si and Albert, 1999]{barabasi1999emergence}
Barab{\'a}si, A.-L. \& Albert, R. (1999).
\newblock Emergence of scaling in random networks.
\newblock {\em Science}, {\it 286}(5439), 509--512.

\bibitem[Breiger, 1974]{breiger1974duality}
Breiger, R.~L. (1974).
\newblock The duality of persons and groups.
\newblock {\em Social Forces}, {\it 53}(2), 181--190.

\bibitem[Browning et~al., 2017]{browning2017ecological}
Browning, C.~R., Calder, C.~A., Soller, B., Jackson, A.~L., \& Dirlam, J.
  (2017).
\newblock Ecological networks and neighborhood social organization.
\newblock {\em American Journal of Sociology}, {\it 122}(6), 1939--1988.

\bibitem[Chicco and Jurman, 2020]{chicco2020advantages}
Chicco, D. \& Jurman, G. (2020).
\newblock The advantages of the matthews correlation coefficient (mcc) over f1
  score and accuracy in binary classification evaluation.
\newblock {\em BMC genomics}, {\it 21}, 1--13.

\bibitem[Daniel et~al., 2019]{daniel2019}
Daniel, J.~R., Santos, A.~J., Fernandes, C., \& Vaughn, B.~E. (2019).
\newblock Network dynamics of affiliative ties in preschool peer groups.
\newblock {\em Social Networks}, {\it 57}, 63--69.

\bibitem[De~Choudhury et~al., 2010]{de2010inferring}
De~Choudhury, M., Mason, W.~A., Hofman, J.~M., \& Watts, D.~J.
\newblock Inferring relevant social networks from interpersonal communication.
\newblock In {\em Proceedings of the 19th international conference on World
  wide web} (2010), pp. 301--310.

\bibitem[Erd\H{o}s and R\'enyi, 1959]{erdosrenyi}
Erd\H{o}s, P. \& R\'enyi, A. (1959).
\newblock On random graphs.
\newblock {\em Publicationes Mathematicae}, {\it 6}, 290--297.

\bibitem[Feld, 1981]{feld1981focused}
Feld, S.~L. (1981).
\newblock The focused organization of social ties.
\newblock {\em American {J}ournal of {S}ociology}, {\it 86}(5), 1015--1035.

\bibitem[Ferreira et~al., 2022]{gomes2022network}
Ferreira, G., Henrique, C., Murai, F., Silva, A.~P., Trevisan, M., Vassio, L.,
  Drago, I., Mellia, M., \& Almeida, J.~M. (2022).
\newblock On network backbone extraction for modeling online collective
  behavior.
\newblock {\em PLOS One}, {\it 17}(9), e0274218.

\bibitem[Gomez-Rodriguez et~al., 2012]{gomez2012inferring}
Gomez-Rodriguez, M., Leskovec, J., \& Krause, A. (2012).
\newblock Inferring networks of diffusion and influence.
\newblock {\em ACM Transactions on Knowledge Discovery from Data (TKDD)}, {\it
  5}(4), 1--37.

\bibitem[Guimera et~al., 2005]{guimera2005team}
Guimera, R., Uzzi, B., Spiro, J., \& Amaral, L. A.~N. (2005).
\newblock Team assembly mechanisms determine collaboration network structure
  and team performance.
\newblock {\em Science}, {\it 308}(5722), 697--702.

\bibitem[Hamann et~al., 2016]{hamann2016structure}
Hamann, M., Lindner, G., Meyerhenke, H., Staudt, C.~L., \& Wagner, D. (2016).
\newblock Structure-preserving sparsification methods for social networks.
\newblock {\em Social Network Analysis and Mining}, {\it 6}, 1--22.

\bibitem[Kossinets, 2006]{kossinets2006effects}
Kossinets, G. (2006).
\newblock Effects of missing data in social networks.
\newblock {\em Social Networks}, {\it 28}(3), 247--268.

\bibitem[Krause et~al., 2009]{krause2009animal}
Krause, J., Lusseau, D., \& James, R. (2009).
\newblock Animal social networks: an introduction.
\newblock {\em Behavioral Ecology and Sociobiology}, {\it 63}, 967--973.

\bibitem[Latapy et~al., 2008]{latapy2008basic}
Latapy, M., Magnien, C., \& Del~Vecchio, N. (2008).
\newblock Basic notions for the analysis of large two-mode networks.
\newblock {\em Social Networks}, {\it 30}(1), 31--48.

\bibitem[Marsden, 2011]{marsden2011survey}
Marsden, P.~V.
\newblock Survey methods for network data.
\newblock In Scott, J. \& Carrington, P.~J., editors, {\em The SAGE handbook of
  social network analysis} (2011), pp. 370--388.

\bibitem[McPherson, 1983]{mcpherson1983ecology}
McPherson, M. (1983).
\newblock An ecology of affiliation.
\newblock {\em American Sociological Review}, {\it 48}, 519--532.

\bibitem[Mizruchi, 1996]{mizruchi1996interlocks}
Mizruchi, M.~S. (1996).
\newblock What do interlocks do? an analysis, critique, and assessment of
  research on interlocking directorates.
\newblock {\em Annual Review of Sociology}, {\it 22}(1), 271--298.

\bibitem[Neal, 2020]{neal2020systematic}
Neal, J.~W. (2020).
\newblock A systematic review of social network methods in high impact
  developmental psychology journals.
\newblock {\em Social Development}, {\it 29}(4), 923--944.

\bibitem[Neal et~al., 2022]{neal2022inferring}
Neal, J.~W., Neal, Z.~P., \& Durbin, C.~E. (2022).
\newblock Inferring signed networks from preschoolers' observed parallel and
  social play.
\newblock {\em Social Networks}, {\it 77}.

\bibitem[Neal, 2014]{neal2014backbone}
Neal, Z.~P. (2014).
\newblock The backbone of bipartite projections: Inferring relationships from
  co-authorship, co-sponsorship, co-attendance and other co-behaviors.
\newblock {\em Social Networks}, {\it 39}, 84--97.

\bibitem[Neal, 2022a]{neal2022backbone}
Neal, Z.~P. (2022)a.
\newblock backbone: An {R} package to extract network backbones.
\newblock {\em PLOS One}, {\it 17}a(5), e0269137.

\bibitem[Neal, 2022b]{neal2022constructing}
Neal, Z.~P. (2022)b.
\newblock Constructing legislative networks in r using incidentally and
  backbone.
\newblock {\em Connections}, {\it 42}b, 1--9.

\bibitem[Neal, 2023]{neal2023duality}
Neal, Z.~P. (2023).
\newblock The duality of networks and groups: Models to generate two-mode
  networks from one-mode networks.
\newblock {\em Network Science},, 1--14.

\bibitem[Neal et~al., 2021]{neal2021comparing}
Neal, Z.~P., Domagalski, R., \& Sagan, B. (2021).
\newblock Comparing alternatives to the fixed degree sequence model for
  extracting the backbone of bipartite projections.
\newblock {\em Scientific Reports}, {\it 11}, 23929.

\bibitem[Newman, 2004]{newman2004coauthorship}
Newman, M.~E. (2004).
\newblock Coauthorship networks and patterns of scientific collaboration.
\newblock {\em Proceedings of the National Academy of Sciences}, {\it
  101}(suppl 1), 5200--5205.

\bibitem[Peel et~al., 2022]{peel2022statistical}
Peel, L., Peixoto, T.~P., \& De~Domenico, M. (2022).
\newblock Statistical inference links data and theory in network science.
\newblock {\em Nature Communications}, {\it 13}(1), 6794.

\bibitem[Schaefer et~al., 2022]{schaefer2022youth}
Schaefer, D.~R., Khuu, T.~V., Rambaran, J.~A., Rivas-Drake, D., \&
  Uma{\~n}a-Taylor, A.~J. (2022).
\newblock How do youth choose activities? assessing the relative importance of
  the micro-selection mechanisms behind adolescent extracurricular activity
  participation.
\newblock {\em Social Networks},.

\bibitem[Wang et~al., 2012]{wang2012measurement}
Wang, D.~J., Shi, X., McFarland, D.~A., \& Leskovec, J. (2012).
\newblock Measurement error in network data: A re-classification.
\newblock {\em Social Networks}, {\it 34}(4), 396--409.

\bibitem[Watts and Strogatz, 1998]{watts1998collective}
Watts, D.~J. \& Strogatz, S.~H. (1998).
\newblock Collective dynamics of ‘small-world’networks.
\newblock {\em Nature}, {\it 393}(6684), 440--442.

\end{thebibliography}

\end{document}