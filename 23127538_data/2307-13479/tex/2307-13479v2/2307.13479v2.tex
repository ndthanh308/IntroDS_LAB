
%%%%%%%%%%%%%%%%%%%%%%%%%%%%%%%%%%%%%%%%%%%%%%%%%%%%%%%%%%      2023 Dec 16 revised  %%%%%%%%%%%%%%%%%%%%%%%%%%%%%%%%%%%%%%%%%%%%%%%%%%%%%%%%%%%%%%%%%%%%%%%%%%%%%%%%%%%%%%%%%%%%%%%%%%%%%%%%%%%%%%%

\documentclass[a4,12pt]{amsart}
%%%%%%%%%%%%%%%%%%%%%%%%%%%%%%%%%%%%%%%%%%%%%%%%%%%%%%%%
\oddsidemargin 0mm
\evensidemargin 0mm
\topmargin 0mm
\textwidth 160mm
\textheight 230mm
\tolerance=9999
%%%%%%%%%%%%%%%%%%%%%%%%%%%%%%%%%%%%%%%%%%%%%%%%%%%%%%%%
\usepackage{amssymb,amstext,amsmath,amscd,amsthm,amsfonts,enumerate,latexsym}
\usepackage{color}
\usepackage[dvipdfmx]{graphicx}
%\usepackage{showkeys}
\usepackage[all]{xy}
\usepackage{stmaryrd} %mapsfrom

\usepackage{bm}

\usepackage[dvipdfmx]{hyperref}

\usepackage{mathptmx}
%%%%%%%%%%%%%%%%%%%%%%%%%%%%%%%%%%%%%%%%%%%%%%%%%%%%%%%%%%%%%%%%%%%%%
\theoremstyle{plain}
\newtheorem{thm}{Theorem}[section]
\newtheorem{theorem}{Theorem}[section]
\newtheorem*{thm*}{Theorem}
\newtheorem*{cor*}{Corollary}
\newtheorem*{thma}{Theorem A}
\newtheorem*{thmb}{Theorem B}
\newtheorem*{mthm}{Main Theorem}
\newtheorem*{mcor}{Theorem \ref{maincor}}
\newtheorem{thms}[thm]{Theorems}
\newtheorem{prop}[thm]{Proposition}
\newtheorem{proposition}[theorem]{Proposition}
\newtheorem{lemma}[theorem]{Lemma}
\newtheorem{lem}[thm]{Lemma}
\newtheorem{cor}[thm]{Corollary}
\newtheorem{corollary}[thm]{Corollary}
\newtheorem{claim}{Claim}
%\setcounter{claim}{0}}
\newtheorem*{claim*}{Claim}

\theoremstyle{definition}
\newtheorem{defn}[thm]{Definition}
\newtheorem{definition}[thm]{Definition}
\newtheorem{ex}[thm]{Example}
\newtheorem{Example}[thm]{Example}
\newtheorem{remark}[thm]{Remark}
\newtheorem{fact}[thm]{Fact}
\newtheorem{conj}[thm]{Conjecture}
\newtheorem{conjecture}[thm]{Conjecture}
\newtheorem{ques}[thm]{Question}
\newtheorem*{quesa}{Question A}
\newtheorem*{quesb}{Question B}
\newtheorem{case}{Case}
\newtheorem{setup}[thm]{Setup}
\newtheorem{notation}[thm]{Notation}
\newtheorem{prob}[thm]{Problem}

\theoremstyle{remark}
\newtheorem{rem}[thm]{Remark}
\newtheorem*{pf}{{\sl Proof}}
\newtheorem*{tpf}{{\sl Proof of Theorem 1.1}}
\newtheorem*{cpf1}{{\sl Proof of Claim 1}}
\newtheorem*{cpf2}{{\sl Proof of Claim 2}}

\numberwithin{equation}{thm}

\newtheorem*{ac}{Acknowledgments}
\newtheorem*{sdc}{Statements and Declarations}

%%%%%%%%%%%%%%%%%%%%%%%%%%%%%%%%%%%%%%%%%%%%%%%%%%%%%%%%%%%%%%%%%%%
\def\soc{\operatorname{Soc}}
\def\xx{\text{{\boldmath$x$}}}
\def\L{\mathrm{U}}
\def\Z{\mathbb{Z}}
\def\D{\mathcal{D}}
\def\Min{\operatorname{Min}}
\def\Gdim{\operatorname{Gdim}}
\def\pd{\operatorname{pd}}
\def\GCD{\operatorname{GCD}}
\def\Ext{\operatorname{Ext}}
\def\X{\mathcal{X}}
\def\XX{\mathbb{X}}
\def\Im{\operatorname{Im}}
\def\sp{\operatorname{sp}}
\def\s{\operatorname{s}}
\def\Ker{\operatorname{Ker}}
\def\KK{\mathbb{K}}
\def\bbZ{\mathbb{Z}}
\def\bbN{\mathbb{N}}
\def\LL{\mathbb{U}}
\def\E{\operatorname{E}}

\def\Hom{\operatorname{Hom}}
%\def\lhom{\underline{\Hom}}
\def\RHom{\mathrm{{\bf R}Hom}}
\def\cext{\mathrm{\widehat{Ext}}\mathrm{}}
\def\Tor{\operatorname{Tor}}
\def\Max{\operatorname{Max}}
\def\Assh{\operatorname{Assh}}
\def\End{\mathrm{End}}
\def\rad{\mathrm{rad}}
\def\Soc{\mathrm{Soc}}
\def\Proj{\operatorname{Proj}}

\def\Mod{\mathrm{Mod}}
\def\mod{\mathrm{mod}}
\def\G{{\sf G}}
\def\T{{\sf T}}

\def\Coker{\mathrm{Coker}}
%\def\Ker{\mathrm{Ker}}
%\def\Im{\mathrm{Im}}
\def\st{{}^{\ast}}
\def\rank{\mathrm{rank}}
%\def\a{\mathfrak a}
\def\a{\mathrm a}
\def\b{\mathfrak b}
\def\c{\mathfrak c}
\def\e{\mathrm{e}}
\def\m{\mathfrak m}
\def\n{\mathfrak n}
\def\l{\mathfrak l}
\def\p{\mathfrak p}
\def\q{\mathfrak q}
\def\r{\mathfrak r}
\def\P{\mathfrak P}
\def\Q{\mathfrak Q}
\def\N{\Bbb N}
\def\C{\Bbb C}
\def\K{\mathrm{K}}
\def\H{\mathrm{H}}
\def\J{\mathrm{J}}
\def\Gr{\mathrm G}
\def\FC{\mathrm F}
\def\Var{\mathrm V}




\newcommand{\nCM}{\mathrm{nCM}}
\newcommand{\nSCM}{\mathrm{nSCM}}
\newcommand{\Aut}{\mathrm{Aut}}
\newcommand{\Att}{\mathrm{Att}}
\newcommand{\Ann}{\mathrm{Ann}}

\newcommand{\rma}{\mathrm{a}}
\newcommand{\rmb}{\mathrm{b}}
\newcommand{\rmc}{\mathrm{c}}
\newcommand{\rmd}{\mathrm{d}}
\newcommand{\rme}{\mathrm{e}}
\newcommand{\rmf}{\mathrm{f}}
\newcommand{\rmg}{\mathrm{g}}
\newcommand{\rmh}{\mathrm{h}}
\newcommand{\rmi}{\mathrm{i}}
\newcommand{\rmj}{\mathrm{j}}
\newcommand{\rmk}{\mathrm{k}}
\newcommand{\rml}{\mathrm{l}}
\newcommand{\rmm}{\mathrm{m}}
\newcommand{\rmn}{\mathrm{n}}
\newcommand{\rmo}{\mathrm{o}}
\newcommand{\rmp}{\mathrm{p}}
\newcommand{\rmq}{\mathrm{q}}
\newcommand{\rmr}{\mathrm{r}}
\newcommand{\rms}{\mathrm{s}}
\newcommand{\rmt}{\mathrm{t}}
\newcommand{\rmu}{\mathrm{u}}
\newcommand{\rmv}{\mathrm{v}}
\newcommand{\rmw}{\mathrm{w}}
\newcommand{\rmx}{\mathrm{x}}
\newcommand{\rmy}{\mathrm{y}}
\newcommand{\rmz}{\mathrm{z}}

\newcommand{\rmA}{\mathrm{A}}
\newcommand{\rmB}{\mathrm{B}}
\newcommand{\rmC}{\mathrm{C}}
\newcommand{\rmD}{\mathrm{D}}
\newcommand{\rmE}{\mathrm{E}}
\newcommand{\rmF}{\mathrm{F}}
\newcommand{\rmG}{\mathrm{G}}
\newcommand{\rmH}{\mathrm{H}}
\newcommand{\rmI}{\mathrm{I}}
\newcommand{\rmJ}{\mathrm{J}}
\newcommand{\rmK}{\mathrm{K}}
\newcommand{\rmL}{\mathrm{L}}
\newcommand{\rmM}{\mathrm{M}}
\newcommand{\rmN}{\mathrm{N}}
\newcommand{\rmO}{\mathrm{O}}
\newcommand{\rmP}{\mathrm{P}}
\newcommand{\rmQ}{\mathrm{Q}}
\newcommand{\rmR}{\mathrm{R}}
\newcommand{\rmS}{\mathrm{S}}
\newcommand{\rmT}{\mathrm{T}}
\newcommand{\rmU}{\mathrm{U}}
\newcommand{\rmV}{\mathrm{V}}
\newcommand{\rmW}{\mathrm{W}}
\newcommand{\rmX}{\mathrm{X}}
\newcommand{\rmY}{\mathrm{Y}}
\newcommand{\rmZ}{\mathrm{Z}}

\newcommand{\cala}{\mathcal{a}}
\newcommand{\calb}{\mathcal{b}}
\newcommand{\calc}{\mathcal{c}}
\newcommand{\cald}{\mathcal{d}}
\newcommand{\cale}{\mathcal{e}}
\newcommand{\calf}{\mathcal{f}}
\newcommand{\calg}{\mathcal{g}}
\newcommand{\calh}{\mathcal{h}}
\newcommand{\cali}{\mathcal{i}}
\newcommand{\calj}{\mathcal{j}}
\newcommand{\calk}{\mathcal{k}}
\newcommand{\call}{\mathcal{l}}
\newcommand{\calm}{\mathcal{m}}
\newcommand{\caln}{\mathcal{n}}
\newcommand{\calo}{\mathcal{o}}
\newcommand{\calp}{\mathcal{p}}
\newcommand{\calq}{\mathcal{q}}
\newcommand{\calr}{\mathcal{r}}
\newcommand{\cals}{\mathcal{s}}
\newcommand{\calt}{\mathcal{t}}
\newcommand{\calu}{\mathcal{u}}
\newcommand{\calv}{\mathcal{v}}
\newcommand{\calw}{\mathcal{w}}
\newcommand{\calx}{\mathcal{x}}
\newcommand{\caly}{\mathcal{y}}
\newcommand{\calz}{\mathcal{z}}

\newcommand{\calA}{\mathcal{A}}
\newcommand{\calB}{\mathcal{B}}
\newcommand{\calC}{\mathcal{C}}
\newcommand{\calD}{\mathcal{D}}
\newcommand{\calE}{\mathcal{E}}
\newcommand{\calF}{\mathcal{F}}
\newcommand{\calG}{\mathcal{G}}
\newcommand{\calH}{\mathcal{H}}
\newcommand{\calI}{\mathcal{I}}
\newcommand{\calJ}{\mathcal{J}}
\newcommand{\calK}{\mathcal{K}}
\newcommand{\calL}{\mathcal{L}}
\newcommand{\calM}{\mathcal{M}}
\newcommand{\calN}{\mathcal{N}}
\newcommand{\calO}{\mathcal{O}}
\newcommand{\calP}{\mathcal{P}}
\newcommand{\calQ}{\mathcal{Q}}
\newcommand{\calR}{\mathcal{R}}
\newcommand{\calS}{\mathcal{S}}
\newcommand{\calT}{\mathcal{T}}
\newcommand{\calU}{\mathcal{U}}
\newcommand{\calV}{\mathcal{V}}
\newcommand{\calW}{\mathcal{W}}
\newcommand{\calX}{\mathcal{X}}
\newcommand{\calY}{\mathcal{Y}}
\newcommand{\calZ}{\mathcal{Z}}

\newcommand{\fka}{\mathfrak{a}}
\newcommand{\fkb}{\mathfrak{b}}
\newcommand{\fkc}{\mathfrak{c}}
\newcommand{\fkd}{\mathfrak{d}}
\newcommand{\fke}{\mathfrak{e}}
\newcommand{\fkf}{\mathfrak{f}}
\newcommand{\fkg}{\mathfrak{g}}
\newcommand{\fkh}{\mathfrak{h}}
\newcommand{\fki}{\mathfrak{i}}
\newcommand{\fkj}{\mathfrak{j}}
\newcommand{\fkk}{\mathfrak{k}}
\newcommand{\fkl}{\mathfrak{l}}
\newcommand{\fkm}{\mathfrak{m}}
\newcommand{\fkn}{\mathfrak{n}}
\newcommand{\fko}{\mathfrak{o}}
\newcommand{\fkp}{\mathfrak{p}}
\newcommand{\fkq}{\mathfrak{q}}
\newcommand{\fkr}{\mathfrak{r}}
\newcommand{\fks}{\mathfrak{s}}
\newcommand{\fkt}{\mathfrak{t}}
\newcommand{\fku}{\mathfrak{u}}
\newcommand{\fkv}{\mathfrak{v}}
\newcommand{\fkw}{\mathfrak{w}}
\newcommand{\fkx}{\mathfrak{x}}
\newcommand{\fky}{\mathfrak{y}}
\newcommand{\fkz}{\mathfrak{z}}

\newcommand{\fkA}{\mathfrak{A}}
\newcommand{\fkB}{\mathfrak{B}}
\newcommand{\fkC}{\mathfrak{C}}
\newcommand{\fkD}{\mathfrak{D}}
\newcommand{\fkE}{\mathfrak{E}}
\newcommand{\fkF}{\mathfrak{F}}
\newcommand{\fkG}{\mathfrak{G}}
\newcommand{\fkH}{\mathfrak{H}}
\newcommand{\fkI}{\mathfrak{I}}
\newcommand{\fkJ}{\mathfrak{J}}
\newcommand{\fkK}{\mathfrak{K}}
\newcommand{\fkL}{\mathfrak{L}}
\newcommand{\fkM}{\mathfrak{M}}
\newcommand{\fkN}{\mathfrak{N}}
\newcommand{\fkO}{\mathfrak{O}}
\newcommand{\fkP}{\mathfrak{P}}
\newcommand{\fkQ}{\mathfrak{Q}}
\newcommand{\fkR}{\mathfrak{R}}
\newcommand{\fkS}{\mathfrak{S}}
\newcommand{\fkT}{\mathfrak{T}}
\newcommand{\fkU}{\mathfrak{U}}
\newcommand{\fkV}{\mathfrak{V}}
\newcommand{\fkW}{\mathfrak{W}}
\newcommand{\fkX}{\mathfrak{X}}
\newcommand{\fkY}{\mathfrak{Y}}
\newcommand{\fkZ}{\mathfrak{Z}}

\newcommand{\mapright}[1]{%
\smash{\mathop{%
\hbox to 1cm{\rightarrowfill}}\limits^{#1}}}

\newcommand{\mapleft}[1]{%
\smash{\mathop{%
\hbox to 1cm{\leftarrowfill}}\limits_{#1}}}

\newcommand{\mapdown}[1]{\Big\downarrow
\llap{$\vcenter{\hbox{$\scriptstyle#1\,$}}$ }}

\newcommand{\mapup}[1]{\Big\uparrow
\rlap{$\vcenter{\hbox{$\scriptstyle#1\,$}}$ }}

\def\depth{\operatorname{depth}}
\def\AGL{\operatorname{AGL}}
\def\Supp{\operatorname{Supp}}
%\def\Ann{\mathrm{Ann}}
\def\ch{\operatorname{ch}}
\def\ann{\operatorname{Ann}}
\def\GL{\operatorname{GL}}
\def\Ass{\operatorname{Ass}}
\def\DVR{\operatorname{DVR}}
%\def\Assh{\mathrm{Assh}}
%\def\id{\mathrm{id}}
\def\height{\mathrm{ht}}
%\def\grade{\mathrm{grade}}
%\def\codim{\mathrm{codim}}
\def\Spec{\operatorname{Spec}}
\def\tr{\operatorname{tr}}
%\def\Sing{\mathrm{Sing}}
\def\Syz{\mathrm{Syz}}
\def\hdeg{\operatorname{hdeg}}
\def\id{\operatorname{id}}
\def\reg{\operatorname{reg}}
\def\gr{\mbox{\rm gr}}
\def\red{\operatorname{red}}


\def\OCM{\operatorname{\rm \Omega \rm{CM}}}
\def\CM{\operatorname{\rm{CM}}}
\def\Ref{\operatorname{\rm{Ref}}}
\def\indCM{\operatorname{\rm{ind\hspace{0.1em} CM}}}
\def\indOCM{\operatorname{\rm{ind\hspace{0.1em}\Omega CM}}}
\def\ind{\operatorname{\rm{ind}}}


\def\A{{\mathcal A}}
\def\B{{\mathcal B}}
\def\R{{\mathcal R}}
\def\F{{\mathcal F}}
\def\Y{{\mathcal Y}}
\def\W{{\mathcal W}}
\def\M{{\mathcal M}}

\def\R{{\mathcalR}}

\def\yy{\text{\boldmath $y$}}
\def\rb{\overline{R}}
\def\ol{\overline}
\def\Deg{\mathrm{deg}}

\def\PP{\mathbb{P}}
\def\II{\mathbb{I}}

\newcommand{\dirsumleft}{<\! \! \! \!\oplus}
\newcommand{\dirsumright}{\oplus\! \! \! \!>}



\title[Remarks on almost Gorenstein rings]{Remarks on almost Gorenstein rings}

\author[Naoki Endo]{Naoki Endo}
\address{School of Political Science and Economics, Meiji University, 1-9-1 Eifuku, Suginami-ku, Tokyo 168-8555, Japan}
\email{endo@meiji.ac.jp}
\urladdr{https://www.isc.meiji.ac.jp/~endo/}

\author[Naoyuki Matsuoka]{Naoyuki Matsuoka}
\address{Department of Mathematics, School of Science and Technology, Meiji University, 1-1-1 Higashi-mita, Tama-ku, Kawasaki 214-8571, Japan}
\email{naomatsu@meiji.ac.jp}

\thanks{2020 {\em Mathematics Subject Classification.} 13H10, 13A15, 13A02.}
\thanks{{\em Key words and phrases.} Almost Gorenstein local ring, Almost Gorenstein graded ring, numerical semigroup}
\thanks{The first author was partially supported by JSPS Grant-in-Aid for Young Scientists 20K14299 and JSPS Grant-in-Aid for Scientific Research (C) 23K03058.}
%%%%%%%%%%%%%%%%%%%%%%%%
%%%%%%%%%%%%%%%%%%%%%%%%%%%%%%%%%%%%%%%%%%%%%%%%%%%%%%%%%%%%%

%%%%%%%%%%%%%%%%%%%%%%%%
%%%%%%%%%%%%%%%%%%%%%%%%%%%%%%%%%%%%%%%%%%%%%%%%%%%%%%%%%%%%%%%%%%%%%%%%%%%


\begin{document}


\maketitle

\setlength{\baselineskip} {15pt}


\begin{abstract}
This paper investigates the relation between the almost Gorenstein properties for graded rings and for local rings. Once $R$ is an almost Gorenstein graded ring, the localization $R_M$ of $R$ at the graded maximal ideal $M$ is almost Gorenstein as a local ring. The converse does not hold true in general (\cite[Theorems 2.7, 2.8]{GMTY2}, \cite[Example 8.8]{GTT}). However, it does for one-dimensional graded domains with mild conditions, which we clarify in the present paper. We explore the defining ideals of almost Gorenstein numerical semigroup rings as well.
\end{abstract}



%{\footnotesize \tableofcontents}

%%%%%%%%%%%%%%%%%%%%%%%%%%%%%%%%%%%%%%%%%%%%%%%%%%%%%%%%%%%%%%%%%%%%%%%%%%%%%%%%%%%%%%%%%%%%%%%%%%%%%%%%%%%%%%%%%%%%%%%%%%%%%%%%%%%%%%%%%%


\section{Introduction}

%\cite[Proposition 3.4]{GTT}より, 1次元CM局所環$R$に対して, $R$が\cite[Definition 3.3]{GTT}の意味でAGLなら, \cite[Definition 3.1]{GMP}の意味でAGLである。逆は, canonical idealの節減の存在性が言えれば正しい。よって, 解析的既約であったり, 剰余体が無限であれば正しく同値になる。しかし, 一般には反例がある()。
%また, \cite{GTT}から, AGGならAGLである。逆は反例がある。行列式環であれば正しい。

%やりたいことはCMの階層化。そのために, 良い性質を持つnon-Gor CM環を見つけたい。候補としてAG環の理論が提唱されている。イントロ AGLとAGGについてその意味や歴史的背景などを書く。定義から, AGGはAGLだけど逆は正しくない。反例を挙げる(2つ知っている。Rees環とvalue semigroup rings)。行列式環なら正しい(普通のやつと対称行列)。自然な問いとして, 1次元ならどうか?というものがありえる。

{\it An almost Gorenstein ring}, which we focus on in the present paper, is one of the attempts to generalize Gorenstein rings. The motivation for this generalization comes from the strong desire to stratify Cohen-Macaulay rings, finding new and interesting classes which naturally include that of Gorenstein rings. The theory of almost Gorenstein rings was introduced by Barucci and Fr{\"o}berg \cite{BF} in the case where the local rings are analytically unramified and of dimension one, e.g., numerical semigroup rings over a field. In 2013, their work inspired Goto, the second author of this paper, and Phuong \cite{GMP} to extend the notion of almost Gorenstein rings for arbitrary one-dimensional Cohen-Macaulay local rings. %in 2013. \textcolor{red}{(to extend the definition of almost Gorenstein rings for general one-dimensional Cohen-Macaulay local rings in 2013. など?)} 
More precisely, a one-dimensional Cohen-Macaulay local ring $R$ is called {\it almost Gorenstein} if $R$ admits a canonical ideal $I$ of $R$ such that $\rme_1(I) \le \rmr(R)$, where $\rme_1(I)$ denotes the first Hilbert coefficients of $R$ with respect to $I$ and $\rmr(R)$ is the Cohen-Macaulay type of $R$ (\cite[Definition 3.1]{GMP}). Two years later, Goto, Takahashi, and the first author of this paper \cite{GTT} defined the almost Gorenstein graded/local rings of arbitrary dimension. 
Let $(R, \m)$ be a Cohen-Macaulay local ring. Then $R$ is said to be an {\it almost Gorenstein local ring} if $R$ admits a canonical module $\rmK_R$ and there exists an exact sequence
$$
0 \to R \to \rmK_R \to C \to 0
$$ 
of $R$-modules such that $\mu_R(C) = \rme^0_\m(C)$ (\cite[Definition 3.3]{GTT}). Here, $\mu_R(-)$ (resp. $\rme^0_\m(-)$) denotes the number of elements in a minimal system of generators (resp. the multiplicity with respect to $\m$). When $\dim R=1$, if $R$ is an almost Gorenstein local ring in the sense of \cite{GTT}, then $R$ is almost Gorenstein in the sense of \cite{GMP}, and vice versa provided the field $R/\fkm$ is infinite (\cite[Proposition 3.4]{GTT}). 
However, the converse does not hold in general (\cite[Remark 3.5]{GTT}, see also \cite[Remark 2.10]{GMP}). 
Similarly as in local rings, a Cohen-Macaulay graded ring $R=\oplus_{n \ge 0}R_n$ with $k=R_0$ a local ring is called an {\it almost Gorenstein graded ring} if $R$ admits a graded canonical module $\rmK_R$ and there exists an exact sequence
$$
0 \to R \to \rmK_R(-a) \to C \to 0
$$ 
of graded $R$-modules with $\mu_R(C) = \rme^0_M(C)$ (\cite[Definition 8.1]{GTT}). Here, $M$ is the graded maximal ideal of $R$, $a=\rma(R)$ is the a-invariant of $R$, and $\rmK_R(-a)$ denotes the graded $R$-module whose underlying $R$-module is the same as that of $\rmK_R$ and whose grading is given by $[\rmK_R(-a)]_n = [\rmK_R]_{n-a}$ for all $n \in \Bbb Z$. 


Every Gorenstein local/graded ring is almost Gorenstein. %\textcolor{red}{(almost Gorenstein のみにするという手もある気がしました。高次元を含む場合に"almost Gorenstein ring"という言い回しがここまでで出現していないので。もちろんこのままで問題があるわけではありません。)}. 
The definitions assert that once $R$ is an almost Gorenstein local (resp. graded) ring, either $R$ is a Gorenstein ring, or even though $R$ is not a Gorenstein ring, the local (resp. graded) ring $R$ is embedded into the module $\rmK_R$ (resp. the graded module $\rmK_R(-a)$) and the difference $C$ behaves well.
Moreover, if $R$ is an almost Gorenstein graded ring, then the localization $R_M$ of $R$ at $M$ is an almost Gorenstein local ring, which readily follows from the definition. However, 
%\textcolor{red}{(On the other hand, などを入れる?接続詞がほしい気がしましたが,逆説は次の文章で使っているので。)}
the converse does not hold true in general (\cite[Theorems 2.7, 2.8]{GMTY2}, \cite[Example 8.8]{GTT}), even though it does for determinantal rings of generic, as well as symmetric, matrices over a field (\cite[Theorem 1.1]{CELW}, \cite[Theorem 1.1]{T}). 

In this paper we investigate the question of when the converse holds in one-dimensional rings. Throughout this paper, unless otherwise specified, let $R = \bigoplus_{n \ge 0}R_n$ be a one-dimensional Noetherian $\bbZ$-graded integral domain admitting a graded canonical module $\rmK_R$. We assume $k=R_0$ is a field, and $R_n \ne (0)$ and $R_{n+1} \ne (0)$ for some $n \ge 0$. 
%Assume $R$ admits a graded canonical module $\rmK_R$ of $R$. 
Let $M$ denote the graded maximal ideal of $R$.

With this notation this paper aims at proving the following result. 


\begin{thm}\label{main}
There exists a graded canonical ideal $J$ of $R$ containing a parameter ideal as a reduction,  
%\textcolor{red}{principal (あるいは parameter か, minimal reduction $(f)$ of $J$とするか)} minimal reduction of $J$, 
and the following conditions are equivalent. 
\begin{enumerate}[$(1)$]
\item $R$ is an almost Gorenstein graded ring. 
\item $R_{M}$ is an almost Gorenstein local ring in the sense of \cite[Definition 3.1]{GMP}.
\item $R_{M}$ is an almost Gorenstein local ring in the sense of \cite[Definition 3.3]{GTT}.
\end{enumerate}
\end{thm}

Theorem \ref{main} guarantees the existence of a (graded) canonical ideal admitting a parameter ideal as a reduction; hence the proof of \cite[Proposition 3.4]{GTT} shows that the conditions $(2)$ and $(3)$ are equivalent even though the field $R/M$ is finite. As we mentioned, by the definition of almost Gorenstein local/graded rings we only need to verify the implication $(2) \Rightarrow (1)$. What makes $(2)\Rightarrow (1)$ interesting and difficult is that the implication is not true in general.

%As for the equivalent conditions, we provide a proof of Theorem \ref{main} by showing $(1)\Rightarrow (3)\Rightarrow (2) \Rightarrow (1)$. 

  

%Our main contribution is to prove the implication 



%One should note that even if $A_N$ is an almost Gorenstein local ring, $A$ is not necessarily an almost Gorenstein graded ring, where $N$ denotes the unique graded maximal ideal of a Cohen-Macaulay graded ring $A$ (\cite[Theorems 2.7, 2.8]{GMTY2}, \cite[Example 8.8]{GTT}). In addition, the converse of $(3) \Rightarrow (2)$ does not hold in general (\cite[Remark 3.5]{GTT}, see also \cite[Remark 2.10]{GMP}). These facts are what makes $(2)\Rightarrow (1)$ interesting and difficult. 


Let us now explain how this paper is organized. We prove Theorem \ref{main} in Section 2 after preparing some auxiliaries. We also explore the explicit generators of defining ideals of almost Gorenstein numerical semigroup rings. Section 3 is devoted to providing examples illustrating Theorem \ref{main}. 


%%%%%%%%%%%%%%%%%%%%%%%%%%%%%%%%%%%%%%%%%%%%%%%%%%%%%%%%%%%%%%%%%%%%%%%%%%%%%%%%%%%%%%%%%%%%%%%%%%%%%%%%%%%%%%%%%%%%%%%%%%%%%%%%%%%%%%%%%%


\section{Proof of Theorem \ref{main}}

Let $S$ be the set of non-zero homogeneous elements in $R$. The localization $S^{-1}R = K[t, t^{-1}]$ of $R$ with respect to $S$ is a {\it simple} graded ring, i.e., every non-zero homogeneous element is invertible, where $t$ is a homogeneous element of degree $1$ (remember that $R_n \ne (0)$ and $R_{n+1} \ne (0)$) which is transcendental over $k$, and $K=[\, S^{-1}R\, ]_0$ is a field. 
%\textcolor{red}{($\deg t=1$というところに$R_n$と$R_{n+1}$が$0$でないという仮定が使われている気がしますが,その辺りは強調する必要はないでしょうか?上の状況設定を引き継いでいることを暗示するという意味もある気がします。)}
Let $\overline{R}$ be the integral closure of $R$ in its quotient field $\rmQ(R)$. 

We begin with the following, which has already appeared in \cite[Lemma 2.1]{E}. Because it plays an important role in our argument, we include a brief proof for the sake of completeness. 

\begin{lem}
The equality $\overline{R} = K[t]$ holds in $\rmQ(R)$. 
\end{lem}

\begin{proof}
As $R$ is an integral domain, we obtain that $\overline{R}$ is a graded ring and $\overline{R} \subseteq S^{-1}R=K[t, t^{-1}]$ (\cite[page 157]{ZS}). Since the field $k$ is Nagata, so is the finitely generated $k$-algebra $R$. Hence $\overline{R}$ is a module-finite extension of $R$. One can verify that $R_n = (0)$ for all $n<0$ and $R_0 = k$; hence $[\,\overline{R}\,]_n = (0)$ for all $n<0$, $L=[\,\overline{R}\,]_0$ is a field, and $k \subseteq L \subseteq K$. We set $N = \bigoplus_{n > 0}[\,\overline{R}\,]_n$. Since the local ring $\overline{R}_N$ of $\overline{R}$ at the maximal ideal $N$ is a DVR, the ideal $N$ is principal. Choose a homogeneous element $f \in \overline{R}$ of degree $q>0$ with $N = f\overline{R}$. Then 
$$
\overline{R} = L[N] = L[f] \subseteq S^{-1}R =K[t, t^{-1}]. 
$$ 
Besides, because $\overline{R}[f^{-1}] = L[f, f^{-1}]$ is a simple graded ring and $R \subseteq \overline{R}[f^{-1}]$, we have $S^{-1}R \subseteq \overline{R}[f^{-1}]=L[f, f^{-1}]$. Thus $K[t, t^{-1}] = L[f, f^{-1}]$, so that $K=L$ and $q = 1$. Consequently, $\overline{R} = L[f] = K[f] = K[t]$, as desired. 
\end{proof}




We are ready to prove Theorem \ref{main}.

\begin{proof}[Proof of Theorem \ref{main}]
Since $S^{-1}\rmK_R \cong S^{-1}R$ as a graded $S^{-1}R$-modules, we have an injective homomorphism $0 \to \rmK_R \overset{\varphi}{\to} S^{-1}R$ of graded $R$-modules. Choose $s \in S$ such that $s\cdot \varphi(\rmK_R) \subsetneq R$. Set $J = s\cdot \varphi(\rmK_R)$ and $q = -\deg s$. Then $\rmK_R(q) \cong J$ as a graded $R$-module.
By setting $a=\rma(R)$ and $\ell = -(q+a)$, we then have $J_{\ell} \ne (0)$ and $J_n =(0)$ for all $n< \ell$. We now choose a non-zero homogeneous element $f \in J_{\ell}$. Note that $\ell >0$ and $f \in \left[\ \!  \overline{R}\ \! \right]_{\ell}$. Therefore 
$$
J \overline{R} = t^{\ell}\overline{R} = f\overline{R}. %\text{\textcolor{red}{~~(本質的ではないですが,$J\overline{R} = t^{\ell}\overline{R}$ ではないのでしょうか?)}}
$$
This shows the equality $J^{r+1} = f J^r$ holds where $r = \mu_R(\overline{R})-1$. Here, we recall that $\mu_R(-)$ denotes the number of elements in a minimal system of generators. Hence, $J$ is a graded canonical ideal of $R$ which contains a parameter ideal $(f)$ of $R$ as a minimal reduction. 

As for the equivalent conditions, we only need to show the implication $(2) \Rightarrow (1)$. 
%We set $A=R_M$, $I=JA$, and $Q = fA$. Since $I^{r+1} = QI^r$, we see that $Q$ is a reduction of $I$. We denote by $\m$ the maximal ideal of $A$. By \cite[Proposition 3.4]{GTT} and the definitions of almost Gorenstein rings (\cite[Definitions 3.3, 8.1]{GTT}), 

$(2) \Rightarrow (1)$ 
We consider the exact sequence
$$
0 \to R \overset{\psi}{\to} J(\ell) \to C \to 0
$$
of graded $R$-modules with $\psi(1) = f$. Since $\m I = \m f$, we get $MJ=Mf$ and hence $MC = (0)$, i.e. $C$ is an Ulrich $R$-module. Therefore $R$ is an almost Gorenstein graded ring because $J(\ell) \cong \rmK_R(-a)$. This completes the proof.
\if0
Since $\m I =\m f$, we get $MJ = Mf$. We consider the exact sequence
$$
0 \to R \overset{\varphi}{\to} J(\ell) \to C \to 0
$$
of graded $R$-modules with $\varphi(1) = f$. Then $J(\ell) \cong \rmK_R(-a)$, and $MC=(0)$, i.e., $C$ is an Ulrich $R$-module. Hence $R$ is almost Gorenstein as a graded ring. This completes the proof. 

\textcolor{red}{(Then のあとに$J(\ell) \cong \rmK_R(-a)$があるのが少し気になったので,別バージョンを考えました。また,$f$など他の記号はそこまでの証明の記号を継続して用いているので,気をつけたほうがいいように感じたため,上で用いている$\varphi$は$\psi$に変更しています。順序を適当に変えただけで,何も変わっていません。)}
\fi
\end{proof}

%\textcolor{red}{当然検討された上でだと思うので,念のための確認ですが,ここで section を分ける,ということは考えなくてもいいでしょうか?分量だけを見ると Section 2 が軽くなりすぎる,という印象はあるかもしれませんが。また,Corollary 2.2 で区切る,という考えもあるかもしれません。このままにしておくのが一番筋はいいようには感じます。}

Let $\bbN$ be the set of non-negative integers. A {\it numerical semigroup} is a non-empty subset $H$ of $\bbN$ which is closed under addition, contains the zero element, and whose complement in $\bbN$ is finite. Every numerical semigroup $H$ admits a finite minimal system of generators, i.e., there exist positive integers $a_1, a_2, \ldots, a_{\ell} \in H~(\ell \ge 1)$ with $\gcd(a_1, a_2,\ldots , a_{\ell}) =1$ such that 
$H = \left<a_1, a_2, \ldots, a_\ell\right>=\left\{\sum_{i=1}^\ell c_ia_i ~\middle|~  c_i \in \Bbb N~\text{for~all}~1 \le i \le \ell \right\}$. For a field $k$, the ring $k[H] = k[t^{a_1}, t^{a_2}, \ldots, t^{a_\ell}]$ (or $k[[H]] =k[[t^{a_1}, t^{a_2}, \ldots, t^{a_\ell}]]$) is called the {\it numerical semigroup ring} of $H$ over $k$, where $t$ denotes an indeterminate over $k$. Note that the ring $k[H]$ satisfies the assumption of Theorem \ref{main}.  Let $M$ be the graded maximal ideal of $R$. 
Since every non-zero ideal in numerical semigroup rings admits its minimal reduction, the two definitions for almost Gorenstein local rings \cite[Definition 3.1]{GMP} and \cite[Definition 3.3]{GTT} are equivalent.
%the almost Gorenstein property for local rings are equivalent in both senses \cite[Definition 3.1]{GMP} and \cite[Definition 3.3]{GTT}. 
In addition, the local ring $k[H]_M$ is almost Gorenstein if and only if $k[[H]]$ is an almost Gorenstein local ring; equivalently, the semigroup $H$ is almost symmetric (\cite[Proposition 29]{BF}, \cite[Theorem 2.4]{N}). %Here, a numerical semigroup 

 
Hence we have the following, which recovers a result of Goto, Kien, Matsuoka, and Truong.

\begin{cor}[{\cite[Proposition 2.3]{GKMT}}]
A numerical semigroup ring $k[[H]]$ is an almost Gorenstein local ring if and only if $k[H]$ is an almost Gorenstein graded ring, or equivalently, $H$ is almost symmetric. 
\end{cor}

In the rest of this section, let $R=k[H]$ be the numerical semigroup ring over $k$ and 
$$
\rmc(H) = \min \{n \in \Bbb Z \mid m \in H~\text{for~all}~m \in \Bbb Z~\ \! \text{with~}\ \! m \ge n\}
$$
the conductor of $H$. We set $\rmf(H) = \rmc(H) -1$. Then $\rmf(H) = \max ~(\Bbb Z \setminus H)$, which is called the Frobenius number of $H$. 
Let 
$$
\mathrm{PF}(H) = \{n \in \Bbb Z \setminus H \mid n + a_i \in H~\text{for~all}~1 \le  i \le \ell\}
$$ 
be the set of pseudo-Frobenius numbers of $H$. The graded canonical module $\rmK_R$ has the form 
$$
\rmK_R = \sum_{c \in \mathrm{PF}(H)}Rt^{-c}
$$
whence $\rmf(H) = \rma(R)$ and $\# \mathrm{PF}(H) = \rmr(R)$ (\cite[Example (2.1.9), Definition (3.1.4)]{GW}). Here, $\rmr(R)$ denotes the Cohen-Macaulay type of $R$.  %\textcolor{red}{(type の記号はIntroの最初のほうで述べていますが,ここでもう一度述べますか?1.1の証明の中では$\mu_R(*)$は断りなしに使っています(これもIntroで述べてあるので問題ないと思います)。)}. 
We write $\mathrm{PF}(H) =\{c_1 < c_2 < \cdots < c_r\}$; hence $r=\rmr(R)$ and $c_r = \rma(R)$. For each $1 \le i \le r$, we set $b_i = -c_{r+1-i}$. Thus $\rmK_R = \sum_{i=1}^r Rt^{b_i}$. 

Let $S = k[x_1, x_2, . . . , x_{\ell}]$ be the weighted polynomial ring over the field $k$ with $x_i \in S_{a_i}$ for all $1 \le i \le \ell$. Consider the homomorphism $\varphi : S \to R$ of graded $k$-algebras defined by $\varphi (x_i) = t^{a_i}$ for all $1 \le i \le \ell$. 


Suppose $R$ is almost Gorenstein, but not a Gorenstein ring. Then $r \ge 2$. Since 
$R \subseteq t^{-b_1}\rmK_R = \sum_{i=1}^rRt^{b_i - b_1} \subseteq \overline{R}$, we have $M\rmK_R \subseteq Rt^{b_1}$, where $M=(t^h \mid 0<h \in H)$ is the graded maximal ideal of $R$. Hence, $c_r - c_i = c_{r-i}$ for all $1 \le i \le r-1$ 
 (see e.g., \cite[Proposition 2.3]{GKMT}, \cite[Theorem 3.11]{GMP}, \cite[Definition 8.1]{GTT}, and \cite[Theorem 2.4]{N}). We consider the graded $S$-linear map 
 \begin{equation*}
F=
\begin{matrix}
S\left(-b_1\right) \\
\oplus \\
S\left(-b_2\right) \\
\oplus \\
\vdots \\
\oplus \\
S\left(-b_r\right)
\end{matrix} \overset{\varepsilon}{\longrightarrow} \rmK_R \longrightarrow 0
\end{equation*}
defined by $\varepsilon(\bm{e}_i) = t^{b_i}$ for each $2 \le i \le \ell$ and $\varepsilon(\bm{e}_1) = -t^{b_1}$, where $\{\bm{e}_i\}_{1 \le i\le r}$ denotes the standard basis of $F$. Set $L = \Ker \varepsilon$. For each $2 \le i \le r$ and $1 \le j \le \ell$, we have $x_j t^{b_i} \in Rt^{b_1}$. Choose a homogeneous element $y_{ij} \in S$ of degree $a_j + b_i - b_1$ such that $x_j t^{b_i} - y_{ij} t^{b_1} = 0$; hence $x_j \bm{e}_i + y_{ij}\bm{e}_1 \in L$. Since $\{x_j \bm{e}_i + y_{ij}\bm{e}_1\}_{2 \le i \le r, \ 1 \le j \le \ell}$ forms a part of minimal basis of $L$, we have $q \ge (r-1)\ell$, where $q = \mu_S(L)$. Let $m = q-(r-1)\ell$. If $m > 0$, then we can choose homogeneous elements $z_1, z_2, \ldots, z_m \in S$ such that 
$\{x_j \bm{e}_i + y_{ij}\bm{e}_1, z_1\bm{e}_1, z_2\bm{e}_1, \ldots, z_m\bm{e}_1\}_{2 \le i \le r, \ 1 \le j \le \ell}$ is a minimal basis of $L$. Hence 
\begin{equation*}
G \overset{\Bbb M}{\longrightarrow}
F\overset{\varepsilon}{\longrightarrow} \rmK_R \longrightarrow 0
\end{equation*}
gives a minimal presentation of $\rmK_R$, where $G$ denotes a graded free $S$-module of rank $q$ and the $r \times q$ matrix $\Bbb M$ has the following form
$$
\Bbb M = 
\begin{bmatrix}
y_{21} ~y_{22}~ \cdots ~y_{2\ell} & y_{31}~ y_{32}~ \cdots ~y_{3\ell} & \cdots & y_{r1}~ y_{r2}~ \cdots ~y_{r\ell} & z_1 ~z_2 ~\cdots ~z_m \\
\ \  x_1 \ \  x_2\ \  \cdots \ \  x_{\ell} \ \  & 0 & 0 & 0  & 0 \\
0 & \ \ x_1 \ \ x_2 \ \ \cdots \ \ x_{\ell}\ \  & 0 & 0  & 0 \\
\vdots & \vdots & \ddots & \vdots & \vdots \\
0 & 0 & 0 & \ \ x_1 \ \ x_2 \ \ \cdots\ \  x_{\ell} \ \ & 0
\end{bmatrix}.
$$
%\textcolor{red}{(成分の間にスペースを入れ,上下がある程度揃うようにしました。スペーzスがない状況だと,積のように見えてしまうように感じます。GTTは,$x$のほうにも添字が2つあるおかげで下手な調整をせずに上下も揃っていて許容範囲だったような気がします。また,積であるという誤解を少しでも減らすよう,行列のサイズも明示するのはどうでしょう。$q$の代わりに$(r-1)\ell + m$ と書くことも考えましたが,さすがに煩わしい感じがしたのでやめました。)}
Note that $b_i - b_1 = c_{i-1}$ and $a_j + (b_i - b_1) \in H$ (remember that $c_r - c_i = c_{r-i}$). We write $a_j + (b_i - b_1) = d_1 a_1 + d_2a_2 + \cdots + d_{\ell}a_{\ell}$ for some $d_i \in \bbN$. Then, because $b_i - b_1 \not\in H$, we have $d_j =0$. As $y_{ij}$ has degree $a_j + b_i -b_1$, we may choose 
$$
y_{ij} = \prod_{1 \le k \le \ell, \ k \ne j}x_k^{d_k} \ \ \ \text{for all}\ \ \ 2 \le i \le r, \ 1 \le j \le \ell.
$$ 
With this notation we reach the following, where, for each $t \ge 1$, $\rmI_t(\Bbb X)$ denotes the ideal of $S$ generated by the $t \times t$ minors of a matrix $\Bbb X$.

%\textcolor{red}{(どこかで \cite[Theorem 7.8]{GTT} を参照しておくべきではないでしょうか?)}

\begin{thm}\label{2.3}
Suppose that $R=k[H]$ is almost Gorenstein, but not a Gorenstein ring. Then, for each $2 \le i \le r$, the difference $\deg y_{ij} - \deg x_j \ (= c_{i-1})$ is constant for every $1 \le j \le \ell$, and the defining ideal of $R$ has the following form
$$
\Ker \varphi = \sum_{i=2}^r
{\rmI_2}\begin{pmatrix}
y_{i1} & y_{i2} & \cdots & y_{i\ell} \\
x_1 & x_2 & \cdots & x_{\ell}
\end{pmatrix}
+ (z_1, z_2, \cdots, z_m).
$$
\end{thm}

\begin{rem}
For a higher dimensional {\it semi-Gorenstein} ring $A$, i.e., a special class of almost Gorenstein ring, the form of the defining ideals can be determined by the minimal free resolution of $A$ (\cite[Theorem 7.8]{GTT}). Our contribution in Theorem \ref{2.3} is that we succeeded in writing $y_{ij}$ concretely in case of numerical semigroup rings.
\end{rem}

\begin{rem} 
When the almost symmetric semigroup $H$ is minimally generated by four elements, 
Eto provided an explicit minimal system of generators of defining ideals of the semigroup rings $k[H]$ by using the notion of RF-matrices (\cite[Section 5]{Eto}). 
\end{rem}


%次の具体例のあたりをつけるのにGAPを使っている。

\begin{ex}
%Let $R = k[H]$ be the semigroup ring of a numerical semigroup $H$ over a field $k$. Then 
The semigroup ring $R=k[H]$ for a numerical semigroup $H$ described below is an almost Gorenstein graded ring, and its defining ideal $\Ker \varphi$ is given by the following form. 
\begin{enumerate}[$(1)$]
\item Let $H=\left<7,8,13,17,19\right>$. Then $\mathrm{PF}(H) = \{6,9,12,18\}$ and
\begin{eqnarray*}
\Ker \varphi \!\!&=&\!\! {\rmI_2}
\begin{pmatrix} 
x_3 & x_1^2 & x_5 & x_1x_2^2 & x_2x_4\\
x_1 & x_2 & x_3 & x_4 & x_5 
\end{pmatrix}
 \ + \ {\rmI_2} 
 \begin{pmatrix} 
 x_2^2 & x_4 & x_1^2x_2 & x_3^2 & x_1x_2x_3\\
x_1 & x_2 & x_3 & x_4 & x_5 
 \end{pmatrix} \\
 \!\!&+&\!\! {\rmI_2} 
 \begin{pmatrix} 
 x_5 & x_1x_3 & x_2x_4 & x_1^3x_2 & x_1^2x_4\\
x_1 & x_2 & x_3 & x_4 & x_5 
 \end{pmatrix}.
\end{eqnarray*}
\item Let $H=\left<11,13,14,16,31\right>$. Then $\mathrm{PF}(H) = \{15,17,19,34\}$ and
\begin{eqnarray*}
\Ker \varphi \!\!&=&\!\! {\rmI_2} 
\begin{pmatrix} 
x_2^2 & x_3^2 & x_2x_4 & x_5 & x_1^3x_2\\
x_1 & x_2 & x_3 & x_4 & x_5 
\end{pmatrix} 
 \ + \ {\rmI_2} 
\begin{pmatrix} 
x_3^2 & x_3x_4 & x_5 & x_1^3 & x_4^3\\
x_1 & x_2 & x_3 & x_4 & x_5 
\end{pmatrix} \\
 \!\!&+&\!\! {\rmI_2} 
\begin{pmatrix} 
x_3x_4 & x_4^2 & x_1^3 & x_1^2x_2 & x_1x_2^3\\
x_1 & x_2 & x_3 & x_4 & x_5 
\end{pmatrix} 
 \ + \ (x_1x_4-x_2x_3).
\end{eqnarray*}
\item Let $H=\left<13,15,16,18,19\right>$. Then $\mathrm{PF}(H) = \{17,20,23,40\}$ and
\begin{eqnarray*}
\Ker \varphi \!\!&=&\!\! {\rmI_2} 
\begin{pmatrix} 
x_2^2 & x_3^2 & x_2x_4 & x_3x_5 & x_4^2\\
x_1 & x_2 & x_3 & x_4 & x_5 
\end{pmatrix}
 \ + \ {\rmI_2}
\begin{pmatrix} 
x_2x_4 & x_3x_5 & x_4^2 & x_5^2 & x_1^3\\
x_1 & x_2 & x_3 & x_4 & x_5
\end{pmatrix} \\
 \!\!&+&\!\!  {\rmI_2} 
\begin{pmatrix} 
x_4^2 & x_5^2 & x_1^3 & x_1^2x_2 & x_1^2x_3\\
x_1 & x_2 & x_3 & x_4 & x_5
 \end{pmatrix}
\ + \ (x_1x_4-x_2x_3, x_1x_5-x_3^2, x_2x_5-x_3x_4).
\end{eqnarray*}
\end{enumerate}
\end{ex}




%※ 4元生成でtype3の場合は,おまけで出てくるのは1個のみで,それは$xw-yz$という形に限られます(これは衛藤先生の結果を読み,この形に書くことを考えれば得られる)。4元生成でtype2 の場合は,必ずおまけは2個です(これは米田先生の定理から読み解くとわかる)。5元生成の場合,type4でもおまけは1個とは限らない,ということになります。また,5元生成でtype5 の例もあるのでこのあたりを書き出すことも可能です。例えば$H=\left<14,15,17,19,20\right>$など。

The next provides an explicit minimal system of generators for defining ideals of $R=k[H]$ when $R$ has {\it minimal multiplicity}, i.e., the embedding dimension is equal to the multiplicity. 

\begin{cor}[cf. {\cite[Corollary 7.10]{GTT}}]
Suppose that $R=k[H]$ is almost Gorenstein, but not a Gorenstein ring. 
If $R$ has minimal multiplicity, 
the defining ideal of $R$ has the following form
$$
\Ker \varphi = \sum_{i=2}^r
{\rmI_2}\begin{pmatrix}
y_{i1} & y_{i2} & \cdots & y_{i\ell} \\
x_1 & x_2 & \cdots & x_{\ell}
\end{pmatrix}.
$$
\end{cor}

\begin{proof}
We maintain the notation as in this section. Since $R$ has minimal multiplicity, by \cite[Theorem 1]{Sally} $q = (\ell-2) \binom{\ell}{\ell-1} = (\ell-2)\ell=(r-1)\ell$, so that $m=0$.
\end{proof}


%%%%%%%%%%%%%%%%%%%%%%%%%%%%%%%%%%%%%%%%%%%%%%%%%%%%%%%%%%%%%%%%%%%%%%%%%%%%%%%%%%%%%%%%%%%%%%%%%%%%%%%%%%%%%%%%%%%%%%%%%%%%%%%%%%%%%%%%%%
\section{Examples of Theorem \ref{main}}

We close this paper by providing some examples. 
In this section, the almost Gorenstein property for local rings refers to the definition in the sense of \cite[Definition 3.3]{GTT}.
The first example indicates that Theorem \ref{main} does not hold unless $R$ is an integral domain. 

\begin{ex}[{\cite[Example 8.8]{GTT}}]\label{3.1}
Let $U=k[s, t]$ be the polynomial ring over a field $k$ and set $R = k[s, s^3t, s^3t^2, s^3t^3]$. We regard $U$ as a $\bbZ$-graded ring under the grading $k=U_0$ and $s, t \in U_1$. Let $M$ be the graded maximal ideal of $R$. Then the following assertions hold true. 
\begin{enumerate}[$(1)$]
\item $R$, $R/sR$ are not almost Gorenstein graded rings. 
\item $R_{M}$, $R_{M}/sR_{M}$ are almost Gorenstein local rings.
\end{enumerate}
\end{ex}

\begin{proof}
Let $S=k[X, Y, Z, W]$ be the polynomial ring over $k$. We consider $S$ as a $\bbZ$-graded ring with $k=S_0$, $X \in S_1$, $Y \in S_4$, $Z \in S_5$, and $W \in S_6$. Let $\varphi : S \to R$ be the graded $k$-algebra map such that $\varphi(X) = s$, $\varphi(Y) = s^3t$, $\varphi(Z) = s^3t^2$, and $\varphi(W) = s^3t^3$. By \cite[Example 8.8]{GTT}, $R$ is not almost Gorenstein graded ring with $\rma(R) = -2$, but the local ring $R_{M}$ is almost Gorenstein. 


%%%%%%%%%%%%%%%%%%%%%%%%%%%%%%%%%%%%%%%%
\if0
Then $\Ker \varphi = 
\rmI_2\left(\begin{smallmatrix}
X^3 & Y & Z \\
Y & Z & W
\end{smallmatrix}\right)
$, and the $S$-module $R$ has a graded minimal free resolution of the form
\begin{equation*}
0 \longrightarrow
\begin{matrix}
S\left(-13\right) \\
\oplus \\
S\left(-14\right) 
\end{matrix}
\overset{\ 
\left[\begin{smallmatrix}
X^3 & Y \\
Y & Z \\
Z & W
\end{smallmatrix}\right] \ 
}{\longrightarrow}
\begin{matrix}
S\left(-10\right) \\
\oplus \\
S\left(-9\right) \\
\oplus \\
S\left(-8\right) 
\end{matrix}
\overset{[\Delta_1 \ -\Delta_2 \ \Delta_3]}{\longrightarrow}
S \overset{\varphi}{\longrightarrow} R \longrightarrow 0
\end{equation*}
where $\Delta_1 = YW-Z^2$, $\Delta_2 = X^3W - YZ$, and $\Delta_3 = X^3Z-Y^2$. As $\rmK_S \cong S(-16)$, by taking $\rmK_S$-dual, we get the resolution of $\rmK_R$
 \begin{equation*}
0 \longrightarrow
S\left(-16\right) \\
\overset{\ 
\left[\begin{smallmatrix}
\ \  \Delta_1 \\
-\Delta_2 \\
\ \ \Delta_3
\end{smallmatrix}\ \right] \ 
}{\longrightarrow}
\begin{matrix}
S\left(-6\right) \\
\oplus \\
S\left(-7\right) \\
\oplus \\
S\left(-8\right) 
\end{matrix}
\overset{\ 
\left[\begin{smallmatrix}
X^3 & Y & Z\\
Y & Z & W \\
\end{smallmatrix}\right]
}{\longrightarrow}
\begin{matrix}
S\left(-3\right) \\
\oplus \\
S\left(-2\right) 
\end{matrix} \overset{\varepsilon}{\longrightarrow} \rmK_R \longrightarrow 0
\end{equation*}
as a graded $S$-module. Hence $\rma(R) = -2$ and $\rmr(R) = 2$. Note that $\dim_k [\rmK_R]_2 = 1$ and $[\rmK_R]_2 = k \eta$ where $\eta = \varepsilon \left(
\left(\begin{smallmatrix}
0 \\
1
\end{smallmatrix}\right)\right)$. We now assume $R$ is an almost Gorenstein graded ring, and seek a contradiction. Choose an exact sequence 
$$
0 \to R \overset{\psi}{\to} \rmK_R(2) \to C \to 0
$$
of graded $R$-modules with $C \ne (0)$ and $\mu_R(C) = \rme^0_M(C)$. Set $\xi = \psi(1)$. Then $0 \ne \xi \in [\rmK_R]_2$, so that $\xi = c \eta$ for some $0 \ne c \in k$. Thus $\Im \psi = R \eta$. Since $\eta = \varepsilon \left(
\left(\begin{smallmatrix}
0 \\
1
\end{smallmatrix}\right)\right)$, we have the isomorphisms
$$
C \cong \rmK_R/R\eta \cong \left[S/(X^3, Y, Z)\right](-3). 
$$
This makes a contradiction, because $1 = \mu_R(C) = \rme^0_M(C)$. Hence $R$ is not almost Gorenstein as a graded ring. By setting $\rho = \varepsilon \left(
\left(\begin{smallmatrix}
1 \\
0
\end{smallmatrix}\right)\right)$, we get
$$
\rmK_R/R\rho \cong  \left[S/(Y, Z, W)\right](-2).
$$
Consider the exact sequence
$$
0 \to R \to \rmK_R(3) \to D \to 0
$$
of graded $R$-modules. Then $\mu_R(D) = 1 = \rme^0_M(D)$, so the local ring $R_M$ is almost Gorenstein. 
\fi
%%%%%%%%%%%%%%%%%%%%%%%%%%%%%%%%%%%%%%%%


The exact sequence $0 \to R(-1) \overset{s}{\to} R \to R/sR \to 0$ of graded $R$-modules induces the isomorphism $\rmK_{(R/sR)} \cong (\rmK_R/s\rmK_R)(1)$. 
Note that $\rma(R/sR) = -1$. 
If $R/sR$ is an almost Gorenstein graded ring, we can choose an exact sequence 
$$
0 \to R/sR \overset{\Psi}{\to} (\rmK_R/s\rmK_R)(2) \to D \to 0
$$
of graded $R$-modules such that $MD=(0)$. Write $\Psi(1) = \overline{\xi}$ with $\xi \in [\rmK_R]_2$. We consider
$$
R \overset{\Phi}{\to} \rmK_R(2) \to C \to 0
$$
where $\Phi(1)=\xi$. Then $C/sC \cong D$. As $MD=(0)$, we get $\dim_RC \le 1$, Hence the map $\Phi$ is injective (\cite[Lemma 3.1]{GTT}), and $s$ is a non-zerodivisor on $C$ because $R/sR \otimes_R \Phi = \Psi$. Thus $\mu_R(C) = \rme^0_M(C)$. This makes a contradiction. As $X$ is superficial for $S/(Y, Z, W)$ with respect to the maximal ideal of $S$, by \cite[Theorem 3.7 (2)]{GTT}, we conclude that $R_M/sR_M$ is almost Gorenstein as a local ring. 
\end{proof}

\begin{rem}
We maintain the same notation as in Example \ref{3.1}. Let $T=k[Y, Z, W]$ be the polynomial ring over $k$. 
Note that $R/sR \cong T/(YW-Z^2, YZ, Y^2) = T/
\rmI_2
\left(
\begin{smallmatrix}
0 & Y & Z \\
Y & Z & W 
\end{smallmatrix}
\right) =V$. 
If we consider $T$ as a $\bbZ$-graded ring under the grading $k=T_0$, $Y \in T_4$, $Z \in T_5$, and $W \in T_6$, as we have shown in Example \ref{3.1} the ring $V$ is not almost Gorenstein as a graded ring. Whereas, if we consider $T$ as a $\bbZ$-graded ring with 
$k=T_0$ and $Y, Z, W \in T_1$, the $T$-module $V$ has a graded minimal free resolution of the form
\begin{equation*}
0 \longrightarrow
\begin{matrix}
T\left(-3\right) \\
\oplus \\
T\left(-3\right) 
\end{matrix}
\overset{\ 
\left[\begin{smallmatrix}
0 & Y \\
Y & Z \\
Z & W
\end{smallmatrix}\right] \ 
}{\longrightarrow}
\begin{matrix}
T\left(-2\right) \\
\oplus \\
T\left(-2\right) \\
\oplus \\
T\left(-2\right) 
\end{matrix}
\overset{[\Delta_1 \ -\Delta_2 \ \Delta_3]}{\longrightarrow}
T \overset{\varepsilon}{\longrightarrow} V \longrightarrow 0
\end{equation*}
where $\Delta_1 = YW-Z^2$, $\Delta_2 = -YZ$, and $\Delta_3 = -Y^2$.
Taking $\rmK_T$-dual, we get the resolution 
 \begin{equation*}
0 \longrightarrow
T\left(-3\right) \\
\overset{\ 
\left[\begin{smallmatrix}
\ \  \Delta_1 \\
-\Delta_2 \\
\ \ \Delta_3
\end{smallmatrix}\ \right] \ 
}{\longrightarrow}
\begin{matrix}
T\left(-1\right) \\
\oplus \\
T\left(-1\right) \\
\oplus \\
T\left(-1\right) 
\end{matrix}
\overset{\ 
\left[\begin{smallmatrix}
0 & Y & Z\\
Y & Z & W \\
\end{smallmatrix}\right]
}{\longrightarrow}
\begin{matrix}
T \\
\oplus \\
T
\end{matrix} \overset{\varepsilon}{\longrightarrow} \rmK_V \longrightarrow 0
\end{equation*}
of $\rmK_V$ as a graded $T$-module. We then consider the homomorphism
$$
V \overset{\Phi}{\to} \rmK_V \to C \to 0
$$
of graded $T$-modules defined by $\Phi(1) = \xi$, where $\xi = \varepsilon \left(
\left(\begin{smallmatrix}
1 \\
0
\end{smallmatrix}\right)\right)$. The isomorphisms $C \cong \rmK_V/V\xi \cong T/(Y, Z, W)$ guarantee that $\Phi$ is injective and $NC =(0)$ where $N=(Y, Z, W)T$. Thus $V$ is an almost Gorenstein graded ring. Hence the almost Gorenstein property for graded rings depends on the choice of its gradings. 
\end{rem}




\begin{ex}\label{3.3}
Let $S=k[X, Y, Z]$ be the polynomial ring over a field $k$. We consider $S$ as a $\bbZ$-graded ring under the grading $k=S_0$, $X \in S_3$, $Y \in S_1$, and $Z \in S_2$. Set $R = S/(Z^3- X^2, XY, YZ)$. Then $R$ is not an almost Gorenstein graded ring, but the local ring $R_M$ is almost Gorenstein, where $M$ denotes the graded maximal ideal of $R$.
\end{ex}

\begin{proof}
Let $I=(Z^3- X^2, XY, YZ)$. Note that $I=(X, Z) \cap (Z^3-X^2, Y) = 
\rmI_2
\left(
\begin{smallmatrix}
Z^2 & X & Y \\
X & Z & 0
\end{smallmatrix}
\right)$. Thus $R$ is a Cohen-Macaulay reduced ring with $\dim R=1$. 
Note that
\begin{equation*}
0 \longrightarrow
\begin{matrix}
S\left(-7\right) \\
\oplus \\
S\left(-6\right) 
\end{matrix}
\overset{\ 
\left[\begin{smallmatrix}
Z^2 & X \\
X & Z \\
Y & 0
\end{smallmatrix}\right] \ 
}{\longrightarrow}
\begin{matrix}
S\left(-3\right) \\
\oplus \\
S\left(-4\right) \\
\oplus \\
S\left(-6\right) 
\end{matrix}
\overset{[\Delta_1 \ -\Delta_2 \ \Delta_3]}{\longrightarrow}
S \overset{\varepsilon}{\longrightarrow} R \longrightarrow 0
\end{equation*}
gives a graded minimal free resolution of $R$, where $\Delta_1 = -YZ$, $\Delta_2 = XY$, and $\Delta_3 = Z^3-X^2$. Hence we get the resolution of $\rmK_R$ below
 \begin{equation*}
0 \longrightarrow
S\left(-6\right) \\
\overset{\ 
\left[\begin{smallmatrix}
\ \  \Delta_1 \\
-\Delta_2 \\
\ \ \Delta_3
\end{smallmatrix}\ \right] \ 
}{\longrightarrow}
\begin{matrix}
S\left(-3\right) \\
\oplus \\
S\left(-2\right) \\
\oplus \\
S 
\end{matrix}
\overset{\ 
\left[\begin{smallmatrix}
Z^2 & X & Y\\
X & Z & 0 \\
\end{smallmatrix}\right]
}{\longrightarrow}
\begin{matrix}
S\left(-1\right) \\
\oplus \\
S 
\end{matrix} \overset{\varepsilon}{\longrightarrow} \rmK_R \longrightarrow 0.
\end{equation*}
This shows $\rma(R) = 1$ and $[\rmK_R]_{-1} = k \xi$, where $\xi = \varepsilon \left(
\left(\begin{smallmatrix}
1 \\
0
\end{smallmatrix}\right)\right)$. Hence, for each homomorphism $\varphi : R \to \rmK_R(-1)$ of graded $S$-modules with $\varphi \ne 0$, we see that $\Im \varphi = R \xi$. Therefore
$$
\left(\rmK_R/R\xi \right)(-1) \cong S/(X, Z)
$$
which implies the map $\varphi$ is not injective; see \cite[Lemma 3.1 (2)]{GTT}. So $R$ is not almost Gorenstein as a graded ring. On the flip side, the elementary row operation
$$
\begin{pmatrix}
Z^2 & X & Y \\
X & Z & 0
\end{pmatrix} \  \ \longrightarrow \  \ 
\begin{pmatrix}
Z^2+X & X+Z & Y \\
X & Z & 0
\end{pmatrix}
$$
and the equality $(Z^2+X, X+Z, Y) = (X, Y, Z)$ in the local ring $S_N$ where $N=(X, Y, Z)S$ guarantee that $R_M$ is an almost Gorenstein local ring by \cite[Theorem 7.8]{GTT}.
\end{proof}

Example \ref{3.3} shows Theorem \ref{main} is no longer true even when $R$ is  reduced. As we show next, there is a counterexample of Theorem \ref{main} in case of homogenous reduced rings as well.



\begin{ex}
Let $S=k[X, Y, Z]$ be the polynomial ring over a field $k$. We consider $S$ as a $\bbZ$-graded ring under the grading $k=S_0$ and $X, Y, Z \in S_1$. Set $R=S/I$, where $I = (X, Y) \cap (Y, Z) \cap (Z, X) \cap (X, Y+Z)$. 
Then $R$ is not an almost Gorenstein graded ring, but the local ring $R_M$ is almost Gorenstein, where $M$ denotes the graded maximal ideal of $R$.
\end{ex}

\begin{proof}
Note that $I$ is an radical ideal of $R$ and $I=(XY, XZ, YZ(Y+Z)) = 
\rmI_2\left(\begin{smallmatrix}
Y+Z & 0 & Y \\
0 & X & YZ
\end{smallmatrix}\right)$. Then the homogeneous ring $R$ is Cohen-Macaulay,   reduced, and of dimension one. Set $\Delta_1 = XY$, $\Delta_2 = YZ(Y+Z)$, and $\Delta_3 = X(Y+Z)$. 
Since 
\begin{equation*}
0 \longrightarrow
\begin{matrix}
S\left(-3\right) \\
\oplus \\
S\left(-4\right) 
\end{matrix}
\overset{\ 
\left[\begin{smallmatrix}
Y+Z & 0 \\
0 & X \\
Y & YZ
\end{smallmatrix}\right] \ 
}{\longrightarrow}
\begin{matrix}
S\left(-2\right) \\
\oplus \\
S\left(-3\right) \\
\oplus \\
S\left(-2\right) 
\end{matrix}
\overset{[\Delta_1 \ -\Delta_2 \ \Delta_3]}{\longrightarrow}
S \overset{\varepsilon}{\longrightarrow} R \longrightarrow 0
\end{equation*}
forms a graded minimal free resolution of $R$, we get the resolution 
 \begin{equation*}
0 \longrightarrow
S\left(-3\right) \\
\overset{\ 
\left[\begin{smallmatrix}
\ \  \Delta_1 \\
-\Delta_2 \\
\ \ \Delta_3
\end{smallmatrix}\ \right] \ 
}{\longrightarrow}
\begin{matrix}
S\left(-1\right) \\
\oplus \\
S \\
\oplus \\
S \left(-1\right)
\end{matrix}
\overset{\ 
\left[\begin{smallmatrix}
Y+Z & 0 & Y\\
0 & X & YZ \\
\end{smallmatrix}\right]
}{\longrightarrow}
\begin{matrix}
S \\
\oplus \\
S \left(-1\right)
\end{matrix} \overset{\varepsilon}{\longrightarrow} \rmK_R \longrightarrow 0.
\end{equation*}
of $\rmK_R$ as a graded $S$-module. Thus $\rma(R)=1$ and $[\rmK_R]_{-1} = k \xi$, where $\xi = \varepsilon \left(
\left(\begin{smallmatrix}
0 \\
1
\end{smallmatrix}\right)\right)$. 
We have the elementary row operation
$$
\begin{pmatrix}
Y+Z & 0 & Y \\
0 & X & YZ
\end{pmatrix} \  \ \longrightarrow \  \ 
\begin{pmatrix}
Y+Z & X & Y+YZ \\
0 & X & YZ
\end{pmatrix}
$$
and the equality $(Y+Z, X, Y+YZ) = (X, Y, Z)$ in $S_N$ where $N=(X, Y, Z)S$. 
Similarly as in the proof of Example \ref{3.3}, we conclude that $R$ is not almost Gorenstein as a graded ring; while the local ring $R_M$ is almost Gorenstein. 
\end{proof}
%%%%%%%%%%%%%%%%%%%%%%%%%%%%%%%%%%%%%%%%%%%%%%%%%%%%%%%%%%%%%%%%%%%%%%%%%%%%%%%%%%%%%%%%%%%%%%%%%%%%%%%%%%%%%%%%%%%%%%%%%%%%%%%%%%%%%%%%%%%%%%%%






%%%%%%%%%%%%%%%%%%%%%%%%%%%%%%%%%%%%%%%%%%%%%%%%%%%%%%%%%%%%%%




\begin{thebibliography}{20}

\bibitem{BF}
{\sc V. Barucci and R. Fr\"{o}berg}, One-dimensional almost Gorenstein rings, {\em J. Algebra}, {\bf 188} (1997), no. 2, 418--442.

\bibitem{CELW}
{\sc E. Celikbas, N. Endo, J. Laxmi, and J. Weyman}, Almost Gorenstein determinantal rings of symmetric matrices, {\em Comm. Algebra}, {\bf 50} (2022), no.12, 5449--5458.

\bibitem{E}
{\sc N. Endo}, How many ideals whose quotient rings are Gorenstein exist$?$, arXiv:2305.19633.

\bibitem{Eto}
{\sc K. Eto}, Almost Gorenstein monomial curves in affine four space, {\em J. Algebra}, {\bf 488} (2017), 362--387. 

\bibitem{GKMT}
{\sc S. Goto, D. V. Kien, N. Matsuoka, and H. L. Truong}, Pseudo-Frobenius numbers versus defining ideals in numerical semigroup rings, {\em J. Algebra}, {\bf 508} (2018), 1--15.

\bibitem{GMP}
{\sc S. Goto, N. Matsuoka, and T. T. Phuong},  Almost Gorenstein rings, {\em J. Algebra}, {\bf 379} (2013), 355--381.

%\bibitem{GMTY1}
%{\sc S. Goto, N. Matsuoka, N. Taniguchi, and K.-i. Yoshida}, The almost Gorenstein Rees algebras of parameters, {\em J. Algebra}, {\bf 452} (2016), 263--278.

\bibitem{GMTY2}
{\sc S. Goto, N. Matsuoka, N. Taniguchi, and K.-i. Yoshida}, The almost Gorenstein Rees algebras over two-dimensional regular local rings, {\em J. Pure Appl. Algebra}, {\bf 220} (2016), 3425--3436.

%\bibitem{GMTY3}
%{\sc S. Goto, N. Matsuoka, N. Taniguchi, and K.-i. Yoshida}, On the almost Gorenstein property in Rees algebras of contracted ideals, {\em Kyoto J. Math.}, {\bf 59} (2019), no. 4, 769--785.


%\bibitem{GMTY4}
%{\sc S. Goto, N. Matsuoka, N. Taniguchi, and K.-i. Yoshida}, The almost Gorenstein Rees algebras of $p_g$-ideals, good ideals, and powers of the maximal ideals, {\em Michigan Math. J.}, {\bf 67} (2018), 159--174.

%\bibitem{GOTWY}
%{\sc S. Goto, K. Ozeki, R. Takahashi, K.-i. Watanabe, and K.-i. Yoshida}, Ulrich ideals and modules, {\em Math. Proc. Cambridge Philos. Soc.}, {\bf 156} (2014), no.1, 137--166.
%\bibitem{GOTWY2}
%{\sc S. Goto, K. Ozeki, R. Takahashi, K.-i. Watanabe, and K.-i. Yoshida}, Ulrich ideals and modules over two-dimensional rational singularities, {\em Nagoya Math. J.}, {\bf 221} (2016), no.1, 69--110.

%\bibitem{GRTT}
%{\sc S. Goto, M. Rahimi, N. Taniguchi, and H. L. Truong}, When are the Rees algebras of parameter ideals almost Gorenstein graded rings?, {\em Kyoto J. Math.}, {\bf 57} (2017), no.3, 655--666.

\bibitem{GTT}
{\sc S. Goto, R. Takahashi, and N. Taniguchi}, Almost Gorenstein rings -towards a theory of higher dimension, {\em J. Pure Appl. Algebra}, {\bf 219} (2015), 2666--2712.

%\bibitem{GTT2}
%{\sc S. Goto, R. Takahashi, and N. Taniguchi}, Ulrich ideals and almost Gorenstein rings, {\em Proc. Amer. Math. Soc.}, {\bf 144} (2016), 2811--2823.

\bibitem{GW}
{\sc S. Goto and K.-i. Watanabe}, On graded rings I, {\em J. Math. Soc. Japan}, {\bf 30} (1978), no. 2, 179--213.



%\bibitem{HK2}
%{\sc J. Herzog and E. Kunz}, Die Wertehalbgruppe eines lokalen Rings der Dimnension 1, {\em S.-Ber. Heidelberger Akad. Wiss. II. Abh.}, 1971.

%\bibitem{HK}
%{\sc J. Herzog and E. Kunz}, Der kanonische Modul eines-Cohen-Macaulay-Rings, {\em Lecture Notes in Mathematics}, {\bf 238}, Springer-Verlag, 1971.

\bibitem{N}
{\sc H. Nari}, Symmetries on almost symmetric numerical semigroups, {\em Semigroup Forum}, {\bf 86} (2013), no.1, 140--154.

%\bibitem{RG}
%{\sc J. C. Rosales and P. A. Garc\'{i}a-S\'{a}nchez}, Numerical semigroups.  Developments in Mathematics, 20. {\em Springer, New York}, 2009. 

\bibitem{Sally}
{\sc J. Sally}, Cohen-Macaulay local rings of maximal embedding dimension, {\em J. Algebra}, {\bf 56} (1979), 168--183.

\bibitem{T}
{\sc N. Taniguchi}, On the almost Gorenstein property of determinantal rings, {\em Comm. Algebra}, {\bf 46} (2018), no.3, 1165--1178.


\bibitem{ZS}
{\sc O. Zariski and P. Samuel}, Commutative Algebra Volume II, {\em Springer}, 1960.


\end{thebibliography}


\end{document}




