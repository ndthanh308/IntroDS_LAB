\documentclass[apl,
superscriptaddress,
% jmp,
% bmf,
% sd,
% rsi,
 amsmath,amssymb,
 longbibliography,
%preprint,%
 %reprint,
]{revtex4-2}
\usepackage{graphicx}% Include figure files
\usepackage{siunitx}
\DeclareSIUnit\counts{counts}

%\usepackage{lineno}

%\runningpagewiselinenumbers
%\linenumbers

\usepackage{bm}
\usepackage{dcolumn}
\usepackage[utf8]{inputenc}
\usepackage[T1]{fontenc}
\usepackage{mathptmx}
\usepackage{etoolbox}
\usepackage[section]{placeins}
\usepackage{hyperref}
\hypersetup{
    colorlinks=true,
    allcolors=blue
    }
\setlength{\parindent}{0pt}

\setcounter{figure}{0}
\renewcommand{\thefigure}{S\arabic{figure}}

\begin{document}


%%%%%%%%%%%%%%%%%%%%%%%%%%%%%%%%%%%%%%%%%%%%%%%%%%%%%%%%%%%%%%%%%%%%%%%%%%%%%%%%%%%%%%%%%%


\title{Supplementary Information for:\\Shallow Silicon Vacancy Centers with lifetime-limited optical linewidths in Diamond Nanostructures}
\author{Josh A. Zuber}
\thanks{These authors contributed equally.} 
\affiliation{Department of Physics, University of Basel, CH-4056 Basel, Switzerland}
\affiliation{Swiss Nanoscience Institute, University of Basel, CH-4056 Basel, Switzerland}
\author{Minghao Li}
\thanks{These authors contributed equally.} 
\affiliation{Department of Physics, University of Basel, CH-4056 Basel, Switzerland}
\author{Marcel.li Grimau Puigibert}
\affiliation{Department of Physics, University of Basel, CH-4056 Basel, Switzerland}
\author{Jodok Happacher}
\affiliation{Department of Physics, University of Basel, CH-4056 Basel, Switzerland}
\author{Patrick Reiser}
\affiliation{Department of Physics, University of Basel, CH-4056 Basel, Switzerland}
\author{Brendan J. Shields}
\affiliation{Department of Physics, University of Basel, CH-4056 Basel, Switzerland}
\author{Patrick Maletinsky}
\email{patrick.maletinsky@unibas.ch}
\affiliation{Department of Physics, University of Basel, CH-4056 Basel, Switzerland}
\affiliation{Swiss Nanoscience Institute, University of Basel, CH-4056 Basel, Switzerland}

\date{July 24, 2023}

\maketitle

%%%%%%%%%%%%%%%%%%%%%%%%%%%%%%%%%%%%%%%%%%%%%%%%%%%%%%%%%%%%%%%%%%%%%%%%%%%%%%%%%%%%%%%%%%

\section{\label{SISec:Yield}SiV$^-$ Yield Estimation}

We predicted our target implantation density of \qty{6e9} ions/cm$^2$ by using yield estimates from previous samples, where we observed a Si $\rightarrow$ SiV$^-$ conversion efficiency of $\sim0.03$ after implantation with $^{28}$Si ions at a fluence of \qty{1.5e11} ions/cm$^2$ and 80\,\unit{\kilo\electronvolt}, the first annealing procedure and parabolic reflector fabrication and of $\sim0.18$ after the second anneal. 

After the first annealing and after parabolic reflector (PR) fabrication, we could still find single SiV$^-$ centers in our PRs even at this high implantation fluence, which was not the case after the second anneal. Based on simple back-of-the envelope calculations, we estimated that a fluence of \qty{1e10} ions/cm$^2$ would yield single SiV$^-$ centers in PRs with diameters around 300\,\unit{\nano\meter} in combination with two annealing procedures. We then opted for the slightly lower fluence to facilitate subsequent characterization. The reason for the switch from $^{28}$Si to $^{29}$Si
was based on technicalities with the implantation provider. If a single stage implanter is used, it is possible that $^{28}$N$_2$ is implanted alongside $^{28}$Si, leading to unwanted contamination of the sample and an increased yield of NV centers. We do not expect an impact on optical coherence based on this change.

\section{\label{sec:SRIM}Stopping Range of Ions in Matter (SRIM) Simulation}

We estimate the depth of our emitters based on the Stopping Range of Ions in Matter (SRIM) Monte-Carlo simulations. The parameters for the simulation are as follows. For the ions, we select Si and choose a mass of 28.977\,amu, an energy of 80\,\unit{\kilo\electronvolt} and an angle of incidence of 7\,\unit{\degree}. For the diamond target, we select C with the density of diamond of 3.53\,\unit{\gram\per\cm\cubed}. The stopping range of the $^{29}$Si is estimated to be $54.3\pm13.9$\,\unit{\nano\meter}. We also note that due to our two high temperature annealing steps, we expect some graphitization of the surface to occur; this graphitization layer is removed by tri-acid cleaning, bringing the emitters in our samples even closer to the surface.
% Figure environment removed



\section{\label{Sec:RTCharac}Room Temperature PR field characterization}

The room temperature (RT) PR characterizations shown in the main text are performed on 220 PRs investigated in two array fields (confocal scans shown in Fig. \ref{SI_Fig:RT_scans}(a) and (c)). As the arrays are regularly fabricated, a series of systematic measurements can be easily performed on each PR. For every PR in the selected array, a ZPL spectrum measurement is conducted to filter out the empty PR that contain no detectable SiV$^-$. Then, for every non-empty PR, a saturation curve, $g^{(2)}(\tau)$ and RT excited state lifetime measurement are carried out successively. An exemplary RT ZPL spectrum under off-resonant excitation with a $515~$nm laser is shown in Fig. \ref{SI_Fig:RT_spect} (a) with the associated saturation curve shown in its inset. Fig. \ref{SI_Fig:RT_spect} (b) and (c) present the histograms of the ZPL linewidths and position for non-empty PRs. These results clearly show that the SiV$^-$ centers we created in PRs are highly homogeneous and mostly unstrained.
Fig. \ref{SI_Fig:RT_scans} (b) and (d) shows the number of SiV in the corresponding PR in the confocal scan in Fig. \ref{SI_Fig:RT_scans} (a) and (c) obtained by the systematic $g^{(2)}(\tau)$ measurements.

% Figure environment removed

% Figure environment removed

%%%%%%%%%%%%%%%%%%%%%%
\section{\label{SiSec:g2corr}$g^{(2)}(\tau)$ Background Correction}

For the background correction of second order correlation function $g^{2}(\tau)$ data, we adopt the method introduced by Brouri et al. \cite{brouriPhotonAntibunchingFluorescence2000}. 
The raw data $W(\tau)$ is obtained in the histogramming mode of the PicoHarp 300 by measuring the waiting time between two photon detectors in the Hanbury Brown and Twiss (HBT) setup. Since we operate in the low count rate regime, the histogram serves as a reliable approximation of the exact photon correlation obtained through full counting statistics in time tagger mode. Therefore, we normalize the raw data $W(\tau)$ by dividing it by its value at a large time delay ($\tau\gg\tau_c$) where $W(\tau)$ tends towards a constant value. 
\begin{equation}
    g^{(2)}_{norm}(\tau) = \frac{W(\tau)}{W(\tau\gg\tau_c)}
\end{equation}

This normalization method is validated by comparing the value of $W(\tau\gg\tau_c)$ with the coincidence number $C$ obtained from an ideal Poissonian photon source:

\begin{equation}\label{eq:g2norm}
   C = N_1N_2\Delta t T,
\end{equation}

where $N_1,N_2$ are the count rates on detectors 1 and 2 respectively, $\Delta t$ is the selected time bin width and $T$ is the total acquisition time. In our measurement, although we do not have access to exact $N_1,N_2$ during the acquisition, the current count rate on each detector is saved every five minutes. We found that in most of the cases, $C\sim W(\tau\gg\tau_c)$.

After normalizing the raw data, we further correct for the background by the signal-to-background ratio $\rho(P)=S(P)/(S(P)+B(P))$ for a given power $P$. This value is determined individually for each SiV$^-$ center by fitting the saturation curve and extracting the values of S(P) and B(P) from the fit at the laser intensity used to acquire the corresponding $W(\tau)$. The background corrected $g^{(2)}_{\text{corr}}(\tau)$ is then given by

\begin{equation}
   g^{(2)}_{\text{corr}}(\tau) = \frac{g^{(2)}_{\text{norm}}(\tau)-(1-\rho(P)^2)}{\rho(P)^2}.
\end{equation}

%We fit the normalized and background corrected $g^{(2)}_{\text{corr}}(\tau)$ according to a three-level rate equation model\,\cite{neuSinglePhotonEmission2011}

%\begin{equation}
 %   g^{(2)}(\tau) = 1 + a\cdot b \cdot \exp{(-c\cdot | \tau-\tau_1 |)} - a (1+b)\cdot \exp{(-d| \tau-\tau_2 |)}.
%\end{equation}

%%%%%%%%%%%%%%%%%%%%
\section{cr-PLE with 515\,\unit{\nano\meter} and 445\,\unit{\nano\meter} repump pulses}

% Figure environment removed

% Figure environment removed

\newpage

%%%%%%%%%%%%%%%%
\section{\label{SiSec:PLEStatAllData}PLE Statistics and Single Sweep discussion}

The statistics of PLE spectral quantities (linewidth, peak position and peak intensity) presented in the main text are extracted from so-called single sweeps along a long PLE measurement. In this section, we provide a detailed description of the actual laser sweeps in our PLE measurements and how we define the term "single sweep". The actual laser (forward-) sweep is achieved by sequentially adjusting the set point from the initial to the final detuning frequency (or vice-versa, respectively, for the backwards sweep), with each set point regulated by a PID loop via a wavemeter (HighFinesse WS-U). The dwell time for the laser at each set point is set to be $\sim0.2~$s allowing the PID loop sufficient time to reach the set point, while PL intensity is measured by an APD with an integration time of $20~$ms. The resulting back-and-forth actual laser sweeps possess a saw-tooth shape as shown in the lower panel of Fig. \ref{SI_Fig:sweep}. A typical number of set points we choose is 61 points for a one-way laser sweep across a detuning range from $-1~$GHz to $1~$GHz. The total time for one back-and-forth laser sweep is $\sim 26~$s. In order to strike a balance between a satisfactory signal-to-noise ratio in single sweep PLE spectra and ensuring the visibility of spectral diffusion dynamics, we define here two complete back-and-forth actual laser sweeps (or four one-way laser sweeps) to be one "single sweep" as illustrated in Fig. \ref{SI_Fig:sweep}, i.e. we average the data over these four actual laser sweeps to produce what we call a "single sweep". 
When analyzing the statistics of PLE spectral quantities of a single sweep, we adopt the overlapping sampling scheme outlined in the lower panel of Fig. \ref{SI_Fig:sweep}, where for a total number of actual laser one-way sweeps $N$, the number of samples for a single sweep would be $N-3$. However, when displaying a PLE trace as shown in upper panel of Fig. \ref{SI_Fig:sweep}, the successive four one-way sweeps compose the single sweep in the trace.


% Figure environment removed


% Figure environment removed
\vspace{-4cm}
% Figure environment removed
\vspace{-4cm}
% Figure environment removed

% Figure environment removed

%%%%%%%%%%%%%%%%%%
\section{\label{SiSec:PowerBroad}Power Broadening}

% Figure environment removed
 
%\bibliographystyle{abbrv} % We choose the "plain" reference style
\bibliography{SI/Manuscript1_SiVsInPillars} % Entries are in the refs.bib file

\end{document}

