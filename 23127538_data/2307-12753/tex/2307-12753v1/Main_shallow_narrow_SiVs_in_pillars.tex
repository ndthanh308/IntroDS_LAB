\pdfoutput=1
\documentclass[aps, pra,
superscriptaddress,
amsmath,amssymb,
longbibliography,
reprint
]{revtex4-2}
\usepackage{graphicx}

\usepackage{textcomp, gensymb}
%\setlength{\columnsep}{25pt}
%\usepackage{lineno}

%\runningpagewiselinenumbers
%\linenumbers


\usepackage{bm}
\usepackage{dcolumn}
\usepackage[utf8]{inputenc}
\usepackage[T1]{fontenc}
\usepackage{mathptmx}
\usepackage{etoolbox}
\usepackage{hyperref}
\hypersetup{
    colorlinks=true,
    allcolors=blue
    }

\setlength{\parindent}{0pt}
\def\bibsection{\section*{\refname}}

\begin{document}



%%%%%%%%%%%%%%%%%%%%%%%%%%%%%%%%%%%%%%%%%%%%%%%%%%%%%%%%%%%%%%%%%%%%%%%%%%%%%%%


\title{Shallow Silicon Vacancy Centers with lifetime-limited optical linewidths in Diamond Nanostructures}
\author{Josh A. Zuber}
\thanks{These authors contributed equally.} 
\affiliation{Department of Physics, University of Basel, CH-4056 Basel, Switzerland}
\affiliation{Swiss Nanoscience Institute, University of Basel, CH-4056 Basel, Switzerland}
\author{Minghao Li}
\thanks{These authors contributed equally.} 
\affiliation{Department of Physics, University of Basel, CH-4056 Basel, Switzerland}
\author{Marcel.li Grimau Puigibert}
\affiliation{Department of Physics, University of Basel, CH-4056 Basel, Switzerland}
\author{Jodok Happacher}
\affiliation{Department of Physics, University of Basel, CH-4056 Basel, Switzerland}
\author{Patrick Reiser}
\affiliation{Department of Physics, University of Basel, CH-4056 Basel, Switzerland}
\author{Brendan J. Shields}
\affiliation{Department of Physics, University of Basel, CH-4056 Basel, Switzerland}
\author{Patrick Maletinsky}
\email{patrick.maletinsky@unibas.ch}
\affiliation{Department of Physics, University of Basel, CH-4056 Basel, Switzerland}
\affiliation{Swiss Nanoscience Institute, University of Basel, CH-4056 Basel, Switzerland}

\date{July 24, 2023}

%%%%%%%%%%%%%%%%%%%%%%%%%%%%%%%%%%%%%%%%%%%%%%%%%%%%%%%%%%%%%%%%%%%%%%%%%%%%%%%



\begin{abstract}
The negatively charged silicon vacancy center (SiV$^-$) in diamond is a promising, yet underexplored candidate for single-spin quantum sensing at sub-kelvin temperatures and tesla-range magnetic fields.
A key ingredient for such applications is the ability to perform all-optical, coherent addressing of the electronic spin of near-surface SiV$^-$ centers.
We present a robust and scalable approach for creating individual, $\sim$50\,nm deep SiV$^-$ with lifetime-limited optical linewidths in diamond nanopillars through an easy-to-realize and persistent optical charge-stabilization scheme. The latter is based on single, prolonged 445\,nm laser illumination that enables continuous photoluminescence excitation spectroscopy, without the need for any further charge stabilization or repumping. Our results constitute a key step towards the use of near-surface, optically coherent SiV$^-$ for sensing under extreme conditions, and offer a powerful approach for stabilizing the charge-environment of diamond color centers for quantum technology applications.
\end{abstract}

\maketitle

%%%%%%%%%%%%%%%%%%%%%%%%%%%%%%%%%%%%%%%%%%%%%%%%%%%%%%%%%%%%%%%%%%%%%%%%%%%%%%%%%%

Diamond color centers represent the backbone for many research directions in quantum technologies, including sensing\,\cite{hedrichNanoscaleMechanicsAntiferromagnetic2021,bianNanoscaleElectricfieldImaging2021,neumannHighPrecisionNanoscaleTemperature2013,vindoletOpticalPropertiesSiV2022,liuSiliconVacancyNanodiamondsHigh2022}, quantum information processing\,\cite{hegdeEfficientQuantumGates2020}, and quantum communication\,\cite{bradleyTenQubitSolidStateSpin2019,englundDeterministicCouplingSingle2010,pompiliRealizationMultinodeQuantum2021}. 
In quantum sensing, the optically addressable electron spin of the nitrogen vacancy (NV) center has been successfully employed to sense various physical observables, including magnetic fields\,\cite{hedrichNanoscaleMechanicsAntiferromagnetic2021}, electric fields\,\cite{bianNanoscaleElectricfieldImaging2021}, and temperature\,\cite{neumannHighPrecisionNanoscaleTemperature2013}. In particular, scanning probe magnetometry based on single NV centers offers quantitative imaging with nanoscale resolution that enabled insights into physical systems that are inaccessible to classical approaches\,\cite{hedrichNanoscaleMechanicsAntiferromagnetic2021,thielProbingMagnetism2D2019}. 
However, the deployment of NV magnetometry in extreme conditions, such as mK temperatures and tesla-range magnetic fields is hampered by the near-surface NV's charge instability in cryogenic environments and limitations in coherent driving its electronic spin, which requires driving fields of tens of GHz in frequency. 
Yet, nanoscale sensing under such conditions would offer exciting opportunities to address interesting condensed matter systems, such as fractional quantum Hall effects\,\cite{bolotinObservationFractionalQuantum2009}, or unconventional superconductors\,\cite{ishidaSpintripletSuperconductivitySr2RuO41998}, by direct, nanoscale magnetic imaging.

The negatively charged silicon vacancy center (SiV$^-$) is an alternative diamond color center hosting an electronic spin that offers promising and advantageous properties for single-spin quantum sensing under extreme conditions. 
Compared to the NV center, the SiV$^-$ predominantly emits photons in the zero phonon line (ZPL) and thereby presents a more efficient spin-photon interface\,\cite{nguyenQuantumNetworkNodes2019}.
Moreover, the SiV$^-$ orbital and spin level structure generally allows for all-optical coherent driving of its ground-state spin\,\cite{beckerAllOpticalControlSiliconVacancy2018,beckerUltrafastAllopticalCoherent2016,pingaultAllOpticalFormationCoherent2014}.

% Figure environment removed

The inversion symmetry of SiV$^-$ leads to a vanishing electric dipole moment and renders its optical transition frequency insensitive to electric field fluctuations to first order\,\cite{rogersMultipleIntrinsicallyIdentical2014}. As a result, highly coherent photon emission with linewidths limited by the inverse excited state lifetime ($\sim1.7$\,ns) have been observed for SiV$^-$ centers far from the diamond surface\,\cite{schroderScalableFocusedIon2017}, or even in diamond nanocrystals\,\cite{hausslerPreparingSingleSiV2019}.
Together with the SiV$^-$ center's substantial electronic spin coherence times at mK temperatures\,\cite{sukachevSiliconVacancySpinQubit2017}, these properties open the exciting perspective to perform all-optical, coherent, nanoscale quantum sensing with SiV$^-$ spins.
Realizing this potential requires the ability to create SiV$^-$ centers with high optical coherence within a few tens of nanometers from the diamond surface in nanostructures suited for efficient sensing operation\,\cite{hedrichParabolicDiamondScanning2020}.
%that exhibit narrow, ideally lifetime-limited, optical linewidths.
However, this achievement has remained elusive so far, largely 
%no previous work has yet shown the realization of single, shallow and lifetime-limited SiV$^-$ in diamond nanostructures, a crucial prerequisite for single-spin quantum sensing. 
because shallow SiV$^-$ suffer from significant spectral instability induced by electric field noise originating from 
nearby diamond surfaces\,\cite{evansNarrowLinewidthHomogeneousOptical2016a,machielseQuantumInterferenceElectromechanically2019}, which is further exacerbated by diamond nanofabrication. 
Furthermore, resonant excitation of SiV$^-$, essential for all-optical sensing schemes, usually requires off-resonant charge-resetting laser pulses\,\cite{evansNarrowLinewidthHomogeneousOptical2016a,nicolasSubGHzLinewidthEnsembles2019,langLongOpticalCoherence2020,arjonamartinezPhotonicIndistinguishabilityTinVacancy2022}, that lead to additional spectral diffusion\,\cite{chuCoherentOpticalTransitions2014} and laser heating, both of which form further obstacles to the use of SiV$^-$ for quantum sensing. Here, we present a reproducible approach to address these challenges and to realize shallow ($\lesssim$ 50\,nm deep), single SiV$^-$ centers with high optical coherence in diamond nanopillars shaped into parabolic reflectors (PRs)\,\cite{hedrichParabolicDiamondScanning2020}. Our approach to SiV$^-$ creation and diamond nanofabrication produces a $\sim30\%$ yield in creating close to lifetime limited SiV$^-$ centers.\\

Importantly, we additionally introduce an easy-to-realize charge stabilization procedure that enables such narrow linewidths in close to $100\%$ of the shallow SiV$^-$ centers in our PRs. This charge stabilization consists of a single, prolonged exposure of SiV$^-$ to 445\,nm laser light, which has the striking effect of narrowing the transition linewidths for SiV$^-$ exhibiting initially broad lines. 445\,nm laser illumination furthermore enables continuous photoluminescence excitation (PLE) measurements without any need for optical charge repumping -- a PLE scheme that we refer to as charge-repump-free PLE (crf-PLE) and which we discuss further below. Most SiV$^-$ we investigated after this procedure show inhomogeneous linewdiths that fall within an approximate factor of two of the lifetime limit, with several instances showing near-lifetime limited single sweep linewidths. In Fig.\,\ref{Fig:1Overview}(e), we present the narrowest transition linewidth achieved with our approach, which shows a full width at half maximum (FWHM) Lorentzian linewidth of $\Delta\nu=100.4\pm6.9$\,MHz.

% Figure environment removed

%%%%%%%%%%%%%%%%%%%%%%%%%%%%%%%%%%%%%%%%%%%%%%%%%%%%%%%

\section*{\label{sec:SamplePrep}Results}


\textbf{Shallow and coherent SiV$^-$ in parabolic reflectors}

 Our diamond preparation and nanofabrication procedure is outlined in Fig.\,\ref{Fig:2Fab}(a): We begin with a commercially available electronic grade diamond (Element Six), sample A, which we implant (CuttingEdge Ions) with $^{29}$Si$^+$ ions at an angle of 7\degree, a dose of $6 \times 10^9$ ions/cm$^2$, and an implantation energy of 80\,keV. This energy yields a Stopping Range of Ions in Matter (SRIM) predicted emitter depth of $\sim$50\,nm [supplementary information (SI) Fig.\,S1]. In order to form SiV$^-$, we anneal the implanted diamond in a home-built high-vacuum oven at 400\,\degree C, 800\,\degree C and 1300\,\degree C for 4\,h, 11\,h and 2\,h respectively.
This corresponds to a slight modification of the procedure introduced by Evans et al.\,\cite{evansNarrowLinewidthHomogeneousOptical2016a}, where we increase the temperature of the last annealing step, as it has been shown that higher temperatures benefit the optical coherence of 
SiV$^-$\,\cite{langLongOpticalCoherence2020}. 
Successful SiV$^-$ creation is confirmed by observing its room-temperature (RT) ZPL around 738\,nm, under off-resonant optical excitation at a wavelength of $\lambda=515$\,nm.
Subsequently, in order to enhance the emitters' collection and excitation efficiencies, we nanofabricate parabolic reflectors (PRs) with diameters at the apex of $\sim 300$\,nm on the sample. 
For this, we use electron-beam lithography defined SiO$_x$ etch masks and a sequence of plasma etching steps (detailed elsewhere\,\cite{hedrichParabolicDiamondScanning2020}). 
PRs are arranged in arrays [Fig.\,\ref{Fig:2Fab}(b)] to facilitate both the fabrication procedure and systematic characterization of SiV$^-$. 
The fabricated PRs employ the same design otherwise used for diamond scanning tips in scanning NV magnetometry\,\cite{appelFabricationAllDiamond2016}, which will expedite future use of SiV$^-$ for scanning probe microscopy.

After diamond nanofabrication, we perform a second anneal identical to the first one, as Evans et al.\,\cite{evansNarrowLinewidthHomogeneousOptical2016a} have shown that the optical coherence of SiV$^-$ increases by removing sub-surface damage from the diamond lattice by annealing and subsequent acid cleaning. Additionally, we observe a significant increase of SiV$^-$ yield after the second annealing step, as many PRs do not show a SiV$^-$ ZPL immediately after fabrication at the implantation dose we employed. Thus the annealing steps before and after PR fabrication are a crucial ingredient for creating individual and coherent SiV$^-$ centers in nanostructures.\\

To characterize our PR arrays, we perform systematic measurements at RT using an automized, home-built confocal microscopy setup (see Methods). 
We measure optical spectra, off-resonant saturation curves [SI Fig.\,S2] and off-resonant second order correlation functions $g^{(2)}(\tau)$ for each pillar in the array. An exemplary data set of a background corrected $g^{(2)}(\tau)$ recorded from a PR is shown in Fig.\,\ref{Fig:2Fab}(c) with a fit revealing $g^{(2)}(0)=0.00\pm0.16$, indicating the presence of a single emitter in the PR. 
For background correction, we subtract from the raw autocorrelation data the uncorrelated background signal stemming from background photons, whose intensity we determined by recording photoluminescence saturation curves [see SI Fig.\,S2] -- a procedure proposed earlier by Brouri et al.\,\cite{brouriPhotonAntibunchingFluorescence2000} [SI Sec.~IV]. 
Subsequently, we estimate the number of emitters in a PR using the relationship $g^{(2)}(0)=1-\frac{1}{n}$, where $n$ is the number of emitters\,\cite{brouriPhotonAntibunchingFluorescence2000}, while in the absence of a SiV$^-$ ZPL, we assign $n=0$ to the PR. 
Using such data collected over 220~PRs, we produce a SiV$^-$ number distribution, which closely follows a Poisson distribution with a mean $\bar{n}=0.53$ emitters per pillar [Fig.\,\ref{Fig:2Fab}(d)].
A certain discrepancy between the data and the Poissonian fit can be assigned to uncertainties in the experimental determination of the background signal.\\ 

In the following, we present a detailed characterization of the optical properties of individual SiV$^-$ at cryogenic conditions.
We employ a closed-cycle cold-finger cryostat to cool the diamond sample to $\sim7$\,K, where we conduct photoluminescence excitation (PLE) experiments.
For this, we tune a narrow-linewidth diode laser near resonance with the C transition of the SiV$^-$ [Fig.\,\ref{Fig:1Overview}(d)] and collect phonon sideband (PSB) emission as a function of excitation laser frequency. 

In Fig.\,\ref{Fig:1Overview}(e), we present the PLE spectrum of the narrowest linewidth that we observed and that exhibits a Lorentzian FWHM of $\Delta\nu=100.4\pm6.9$\,MHz. 
These data were obtained by averaging PLE spectra of fourteen successive laser sweeps across the C transition, followed by a Lorentzian fit (for details, see next section).
To benchmark the linewidth, we measure the optical lifetime of this SiV$^-$ by pulsed laser excitation at a wavelength of 515\,nm, followed by time-tagged ZPL photon collection [Fig.\,\ref{Fig:2Fab}(e), right panel]. 
An exponential fit to the photon decay trace yields a radiative lifetime $\tau_r=1.69\pm0.04$\,ns that corresponds to a lifetime-limited optical linewidth of $\Delta\nu=(2\pi\tau_r)^{-1}=93.9\pm2.2$\,MHz. 
 
To our knowledge, this is the first record of a lifetime-limited linewidth reported for $\lesssim50$\,nm shallow SiV$^-$ 
%with an emitter-surface distance , 
embedded in a diamond nanostructure, as required for nanoscale quantum sensing.
In addition, we find resonant saturation powers of this SiV$^-$ of $P_{\rm sat}=23.0\pm3.1$\,nW and a saturation count rate of $9.7\pm0.7$\,kcounts/s [Fig.\,\ref{Fig:2Fab}(e), left panel], typical for our devices.\\

\textbf{Charge repump-free photoluminescence excitation of SiV$^-$ centers in diamond}\\
PLE experiments with solid state emitters often require regular application of optical charge-resetting pulses using off-resonant laser light\,\cite{evansNarrowLinewidthHomogeneousOptical2016a,nicolasSubGHzLinewidthEnsembles2019,langLongOpticalCoherence2020,yurgensSpectrallyStableNitrogenvacancy2022}. Such charge resetting pulses are usually applied at wavelengths between 510 to 532 nm, to undo de-ionization events the emitter can undergo under resonant excitation. 
Importantly, such repumping perturbs the charge environment of the emitter and thus induces inhomogeneous broadening\,\cite{orphal-kobinOpticallyCoherentNitrogenVacancy2023,evansNarrowLinewidthHomogeneousOptical2016a,arjonamartinezPhotonicIndistinguishabilityTinVacancy2022}, precluding the observation of lifetime-limited optical linewidths. 
We refer to this measurement scheme as charge-repumped PLE (cr-PLE). 
Recently, Görlitz et al.\ \cite{gorlitzCoherenceChargeStabilised2022} have shown that exposing SnV$^-$ centers in diamond to 445\,nm laser light enables crf-PLE, reduces spectral diffusion and increases the brightness of SnV$^-$. Their charge-state lifetime (i.e. effective measurement time in crf-PLE) is, however, limited to about an hour under resonant excitation. They suggest that the same approach could also be beneficial to other group-IV vacancies, such as SiV$^-$.\\


 % Figure environment removed
 
Our experiments on charge stabilization of SiV$^-$ with blue illumination revealed a similar, long-lasting effect. Prolonged (>$1$\,h) and high-intensity (>$5$\,mW) illumination of a PR with a 445\,nm laser led to persistently bright and stable PLE emission with narrow linewidths in crf-PLE, completely removing the need of charge resetting laser pulses, which is usually necessary for our samples.
An exemplary crf-PLE data set is depicted in Fig.\,\ref{fig:3PLE}(a), where the bottom (top) panel shows a sequence of 500 single crf-PLE sweeps (c.f.~SI Sec.~VI) and the corresponding average, respectively.
The data were continuously recorded over 14\,h, using exclusively near-resonant laser excitation.
The PLE resonance line retains its brightness and stability over the whole measurement duration and yields an averaged, inhomogeneously broadened linewidth of $\Delta\nu=211.5\pm0.5$\,MHz, within a factor of 2.15 of its lifetime limit of $98.9\pm0.9$\,MHz, which we evaluated by an independent excited state lifetime measurement at $7$\,K.\\

Having observed the positive impact of 445\,nm illumination on the optical properties of SiV$^-$ in resonant excitation, resulting in crf-PLE with improved linewidths and stability, we further address the reproducibility and effectiveness of this phenomenon. To do so, we investigated twelve PRs, nine of which contain single SiV$^-$, with the following measurement sequence: 
For each SiV$^-$, and before exposing them to anything other than 515\,nm laser light, we start by performing crf-PLE to assess the initial charge stability and linewidth.
During these measurements, we have observed three clearly distinct, roughly equally distributed SiV$^-$ populations, classified by their behaviour in crf-PLE: SiV$^-$ in population 1 exhibit continuous, stable and bright emission with narrow linewidths [Fig.\,\ref{fig:3PLE}(a)]; SiV$^-$ in population 2 present dimmer emission with large spectral diffusion and broader single sweep linewidths under continuous resonant excitation [Fig.\,\ref{fig:3PLE}(b) - top]; SiV$^-$ in population 3 show charge state instabilities (blinking), where it is not possible to perform continuous crf-PLE [Fig.\,\ref{fig:3PLE}(b) - bottom].
Secondly, we perform `traditional' cr-PLE using a 515\,nm charge repump laser to benchmark the PLE linewidth under this measurement scheme. 
Lastly, we continuously expose the SiV to 445\,nm excitation at $>5$\,mW for prolonged periods of time ($>1$\,h), and repeat the crf-PLE linewidth measurement.
From this measurement series, we found above all that SiV$^-$ initially in population 2 and 3 can be mapped to population 1 after prolonged $445$\,nm laser exposure, leading to drastically improved properties.

An exemplary result of the outcome of the above-mentioned measurement series performed on a single SiV$^-$ initially from population 3 is presented in Fig.\,\ref{fig:3PLE}(c).
The histograms show the probability of single sweep Lorentzian linewidths $\Delta\nu$ (top panel), center frequency spread $\nu_0-\bar{\nu}_0$ (middle panel) and excitation power normalized peak intensity $\text{I}_{\text{PLE}}$ (bottom panel) for cr-PLE and crf-PLE. The data clearly show how crf-PLE after 445\,nm laser exposure yields both a strongly reduced single sweep linewidth and center frequency spread, and a highly increased peak intensity compared to 515\,nm cr-PLE. The results we obtained in this way for the twelve investigated SiV$^-$ are summarized in Fig.\,\ref{fig:3PLE}(d)-(f). 
In 3(d), we show three representative data sets of PLE linewidths measured on single SiV$^-$ as a function of the number of single sweeps over which the data were averaged. We compare the result for 515 nm cr-PLE (green) and crf-PLE after illuminating their respective PRs with the 445 laser (blue) [SI Sec. VI for more data sets].
While after a few tens of sweeps, crf-PLE converges to a linewidth in the range of $\sim200$\,MHz, the averaged cr-PLE linewidths diverge as a function of the number of averaged sweeps. 
These data are testament to the absence of excess spectral diffusion, i.e.\ spectral wandering\,\cite{wolfowiczQuantumGuidelinesSolidstate2021} is completely eliminated when performing PLE without a charge repump laser, which enables long-time measurements without loss of optical coherence. 
Lastly, in order to further compare the two protocols, for each SiV$^-$ we extracted the center frequency standard deviation $\sigma(\nu_0)$ and the average power normalized peak intensity I$_{\text{PLE}}$ from all single sweep crf-PLE spectra, and plot them against the corresponding values for the same SiV$^-$ under 515\,nm cr-PLE (Fig.\,\ref{fig:3PLE}(e) and (f)).

We consistently find that in crf-PLE, SiV$^-$ exhibit higher peak PLE intensities, less background fluorescence, and display strikingly reduced linewidths and spectral diffusion.
However, it is noteworthy that for some SiV$^-$ initially in population 1, illumination with 445\,nm light slightly decreases their brightness and stability [SI Sec.~VI].
We confirmed that this beneficial effect of prolonged 445\,nm laser illumination is persistent and neither specific to a diamond sample nor the Si implantation dose.
Specifically, we confirmed the same behaviour as presented in Fig.\,\ref{fig:3PLE} for SiV$^-$ in a second sample B, which was prepared in the same way as the first sample, but with a $^{28}$Si implantation dose five times higher
(as a sole, slight difference between the two, we found that in the second sample, Population 3 made up for a smaller percentage than in the low-dose sample A).
For both samples, the beneficial effect of 445\,nm laser exposure persisted throughout the timescale of this study (several months) and was neither affected by continuous, high-power laser illumination (be it resonant or off-resonant), by long idle times in the dark, nor by thermal cycling of the samples.
These combined findings suggest that illumination of 445\,nm laser light is a generally applicable means for permanently stabilizing the charge-environment of shallow SiV$^-$ centers, such that coherent optical addressing can be performed without any further charge repumping.

Only in the case where such continuous crf-PLE measurements can be conducted on a given SiV$^-$ from the beginning (population 1), 445\,nm laser illumination might slightly deteriorate the SiV$^-$ optical properties and calls for cautious use of blue laser illumination.\\

%\vspace{2.5cm}

% Figure environment removed

\textbf{Addressing spectrally distinct individual SiV$^-$ in multi-SiV nanostructures}\\
The narrow and stable SiV$^-$ linewidths we demonstrated enable the addressing of individual emitters in nanostructures that contain multiple, spectrally distinct SiV$^-$.
Such a situation is illustrated in Fig.\,\ref{Fig:4MultipleSiV$^-$s}(a), where a crf-PLE measurement after prolonged 445\,nm laser illumination shows three PLE resonances, which we attribute to the C transitions of three separate SiV$^-$ hosted in a single PR on the high-density diamond sample.
The shift in their transition frequency likely results from local variations in strain or electric field in the surroundings of each SiV$^-$. A $g^{(2)}(\tau)$ measurement conducted under off-resonant optical excitation at $515$\,nm [Fig.\,\ref{Fig:4MultipleSiV$^-$s}(b)] reveals a value $g^{(2)}(0)=0.82\pm0.01$, which indicates that indeed more than one emitter is present in this particular PR. Since however, owing to our charge stabilization protocol, the resonances of the three SiV$^-$ remain spectrally distinct, one can individually address each SiV$^-$ despite their localization in a nanoscale volume.
We demonstrate this by resonantly driving each of the three SiV$^-$ and recording a corresponding $g^{(2)}(\tau)$ trace. Indeed, the three photon autocorrelation traces [Fig.\,\ref{Fig:4MultipleSiV$^-$s}(c)] all show values of $g^{(2)}(0)$ close to zero ($g^{(2)}(0)=0.00\pm0.03$, $0.00\pm0.03$, $0.00\pm0.03$, respectively), indicating that only one SiV$^-$ at a time is being optically excited in this case. 
For nanoscale quantum sensing with SiV$^-$, this result brings the interesting perspective of performing single-spin sensing in nanostructures containing small ensembles of spins, which would find immediate applications, for example, in covariance magnetometry\,\cite{rovnyNanoscaleCovarianceMagnetometry2022}.

\section*{\label{sec:Discussion}Discussion}
In this work, we demonstrated the robust and reproducible creation of single narrow-linewidth SiV$^-$ color centers located within a few tens of nanometers from the end facets of individual diamond nanopillars. 

Nearly all SiV$^-$ investigated here display inhomogenously broadened linewidths within a factor of two from the lifetime limit in crf-PLE over long timescales, and single sweep linewidths that, at times, approach their respective lifetime limit.
These results are enabled by a combination of a high temperature vacuum annealing step introduced after nanopillar fabrication, 
and the application of a novel, optical charge stabilization protocol based on extended, single-time exposure of SiV$^-$ to continuous-wave 445\,nm laser light.
The latter permanently and entirely removes the need for charge repumping in resonant excitation experiments and yields improvement in several key figures of merit of resonant optical excitation of SiV$^-$ centers. 
Specifically, the optical charge stabilization leads to enhanced brightness, reduced spectral diffusion, and charge state preservation for those SiV$^-$ which suffered from de-ionization under resonant excitation.
While the microscopic origins underlying the demonstrated optical charge stabilization scheme remain unexplained and depletion of the charge environment may play a role\,\cite{andersonElectricalOpticalControl2019}, we anticipate that our results will trigger significant further research in theory and experiment.

Our results constitute a major step towards the use of SiV$^-$ as nanoscale quantum sensors for applications under extreme conditions, such as single spin scanning magnetometry at sub-kelvin temperatures and tesla-range magnetic fields\,\cite{fuHighSensitivityMomentMagnetometry2020}. 
Furthermore, our easy-to-implement charge-stabilization scheme will find immediate applications in other quantum technology applications of SiV$^-$, including the development of quantum repeaters\,\cite{bayerQuantumRepeaterPlatform2022}, quantum networks\,\cite{sipahigilIntegratedDiamondNanophotonics2016b,nguyenQuantumNetworkNodes2019} or indistinguishable single photon sources\,\cite{sipahigilIndistinguishablePhotonsSeparated2014}. 
Lastly, it is conceivable that our approach for generating and stabilizing near-surface color centers with high optical coherence extends to other color centers in diamond or in other wide-bandgap hosts such as hBN\,\cite{grollControllingPhotoluminescenceSpectra2021} or SiC\,\cite{nagyHighfidelitySpinOptical2019}.\\

\section*{Methods}

\small\textbf{Optical Setups}

To characterize our diamond samples at room temperature (RT), we employ a home-built confocal microscopy setup. This setup is equipped with a continuous wave (cw) 515\,nm diode laser (Cobolt 06-MLD) for off-resonant excitation. At low temperature ($\sim7$\,K), the sample is housed in a variable-temperature closed-cycle cryostat (attocube attoDRY800) and we perform optical spectroscopy again with a home-built confocal microscopy setup. For this setup, we put to use cw 445\,nm and cw 515\,nm diode lasers (Cobolt 06-MLD) for off-resonant excitation as well as a narrow-linewidth (200\,kHz) tunable (720-739\,nm) diode laser (Sacher Lasertechnik LION) for resonant excitation. Frequency modulation and control of the LION is achieved through a PID loop of a HighFinesse WS-U wavelength meter. Additionally, we stabilize the intensity of the resonant laser through the PID loop of a home-built optical intensity stabilization (OIS) module (Physics Basel SP 999). In case of the resonant laser and if applicable to the experiment in question, we carve pulses with an acousto-optical modulator (AOM, G\&H R15210), whereas the off-resonant lasers can be electronically modulated out of the box. For both RT and LT, we focus laser light onto the sample with a 0.8 NA objective (Olympus LMPLFLN 100X) thermalized at RT, through which we also collect the signal. Photons are detected by silicon-based single-photon counting modules (SPCM, Excelitas AQRH-33-FC). We perform photon-correlation measurements using a time-correlated single-photon counting (TCSPC) system (PicoQuant PicoHarp 300) connected to the SPCMs. For optical lifetime measurements we use a pulsed laser (NKT Photonics SuperK Extreme EXW-12) tuned to 515\,nm and said TCSPC system. Optical spectra are recorded on the camera of a spectrograph (Princeton Instruments SP-2500).\\

\textbf{Data availability}\\
The data that support the findings of this study are available from the corresponding author upon reasonable request.

\section*{Acknowledgements}

We gratefully acknowledge Christoph Becher and Dennis Herrmann for fruitful discussions as well as Silvia Ruffieux for helping to fabricate PRs on sample A. 
We further acknowledge financial support through QuantERA project ``sensExtreme'' (Grant No.~$205573$), from the Swiss Nanoscience Institute, and through the Swiss NSF Project Grant No.~$188521$.

\section*{Author information}
These authors contributed equally: Josh A. Zuber, Minghao Li.\\

\textbf{Authors and Affiliations}\\
\textbf{Department of Physics, University of Basel, Klingelbergstrasse 82, Basel, CH-4056, Switzerland}\\
J. A. Zuber, M. Li, M. Grimau Puigibert, J. Happacher, P. Reiser, B. J. Shields \& P. Maletinsky\\

\textbf{Swiss Nanoscience Institute, Klingelbergstrasse 82, Basel, CH-4056, Switzerland}\\
J. A. Zuber \& P. Maletinsky\\

\textbf{Contributions}\\
All experiments, data analysis and sample preparation shown here were conducted by J.A.Z. and M.L. under the supervision of P.M.. M.G.P. started the project under P.M.'s supervision, built the first setup and fabricated preliminary samples. M.G.P., together with J.H., conducted preliminary studies on and fabricated Sample B. P.R. fabricated PRs on sample A with the help of J.A.Z. and M.L.. B.J.S. fabricated PRs on preliminary samples. J.A.Z., M.L. and P.M. wrote the manuscript. All authors discussed the data and commented on the manuscript.\\

\textbf{Competing interests}\\
The Authors declare no competing interests.\\

\textbf{Correspondence} and requests for materials should be addressed to P. Maletinsky

\newpage

\bibliographystyle{naturemag}
\bibliography{Main_shallow_narrow_SiVs_in_pillars.bib}

\end{document}
