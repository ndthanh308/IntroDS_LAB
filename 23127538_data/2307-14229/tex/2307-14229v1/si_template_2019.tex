
%%%%%%%%%%%%%%%%%%%%%%%%%%%%%%%%%%%%%%%%%%%%%%%%%%%%%%%%%%%%%%%%%%%%%%%%%
%
%  SUPPORTING INFORMATION TEMPLATE
%
%% ------------------------------------------------------------------------ %%
%
%
%Please use this template when formatting and submitting your Supporting Information.

 %\usepackage{graphicx}
 %\usepackage{url}
 \setcounter{figure}{0}
 \renewcommand{\thefigure}{S\arabic{figure}}
%
% You may need to use one of these options for graphicx depending on the driver program you are using. 
%
% [xdvi], [dvipdf], [dvipsone], [dviwindo], [emtex], [dviwin],
% [pctexps],  [pctexwin],  [pctexhp],  [pctex32], [truetex], [tcidvi],
% [oztex], [textures]


%\begin{document}

%% ------------------------------------------------------------------------ %%
%
%  TITLE
%
%% ------------------------------------------------------------------------ %%

%% Figure removed


\title{Supporting Information for "Rupture Dynamics of Cascading Earthquakes in a Multiscale Fracture Network"}


\authors{Kadek Hendrawan Palgunadi\affil{1}, Alice-Agnes Gabriel\affil{2,4}, Dmitry Garagash\affil{3}, Thomas Ulrich\affil{4}, Paul Martin Mai\affil{1}}


\affiliation{1}{Physical Science and Engineering, King Abdullah University of Science and Technology, Thuwal, Saudi Arabia}
\affiliation{2}{Institute of Geophysics and Planetary Physics, Scripps Institution of Oceanography, University of California, San Diego, CA, USA}
\affiliation{3}{Dalhousie University, Department Civil Resource Engineering, Halifax, Canada}
\affiliation{4}{Department of Earth and Environmental Sciences, Geophysics, Ludwig-Maximilians-Universit\"{a}t M\"{u}nchen, Munich, Germany}

%\begin{article}

%% ------------------------------------------------------------------------ %%
%
%  TEXT
%
%% ------------------------------------------------------------------------ %%

\noindent\rule{\textwidth}{1pt}

\section*{Introduction}

The supplementary material includes figures and videos that provide detailed representations of rupture processes in various scenarios described in the main paper, including variations in the orientation of $S{\!}H_\mathrm{max}$ ($\Psi$) and fluid injection scenarios. Figures present snapshots focusing on the physical processes involved in cascading rupture. Videos illustrate the space-time evolution of the rupture process from two different perspectives, in an ``exploded" view and in the original constellation of the fault network . Supplementary videos can be accessed at the following link: \url{https://bit.ly/FractureNetworkVideoSupps}.

\section*{Contents of this file}
\begin{enumerate}
    \item Figure \ref{fig:fracture_size}: Fracture size distribution.
    \item Figure \ref{fig:S1}: Rupture time of different orientation of $S{\!}H_\mathrm{max}$ ($\Psi$).
    \item Figure \ref{fig:S_main_fault_slip}: Slip only on the main fault.
    \item Figure \ref{fig:S2}: Slip of different orientation of $S{\!}H_\mathrm{max}$ ($\Psi$).
    \item Figure \ref{fig:S3}: Stereonet plot of slipped fractures overlain by relative prestress ratio ($\mathcal{R}$) for different $\Psi$ shown in lower hemisphere projection. 
    \item Figure \ref{fig:S3_1}: Map view of the slipped fractures and the main fault for $\Psi = 65^\circ$ (top panel) and $\Psi= 120^\circ$ (lower panel). 
    \item Figure \ref{fig:S3_2}: Depth slice of the slipped fractures every 0.5~km for $\Psi = 65^\circ$. 
    \item Figure \ref{fig:S5}: Snapshot focuses on the rupture front of the main fault without showing fractures for scenario $\Psi=120^\circ$.
    \item Figure \ref{fig:S_ruptureSpeed}: Supershear rupture speed.
    \item Figure \ref{fig:S6}: Exploded view of rupture time evolution and final slip of Scenario 2 for two examples.
    \item Figure \ref{fig:S_oneFamily}: Slip distribution if only considering one fracture family.
    \item Figure \ref{fig:S4}: Overview examples from 4 subsets of fracture-fracture interaction scenarios.
    \item Figure \ref{fig:S3_3}: Depth slice of the slipped fractures every 0.5~m for $\Psi = 120^\circ$. 
\end{enumerate}

%\section{Additional Text Material:}


\section*{Additional Supporting Information (Files uploaded separately)}
The separately uploaded files contain movies of different cases and scenarios explained in the main paper. The files comprise 26 movies showing the spatiotemporal evolution of slip rate.

\begin{enumerate}
    \item Case $\Psi = 40^\circ$:
    \begin{itemize}
        \item Movie S1a (\texttt{SR\_E40}): Slip rate (in [m/s]) presented in exploded view.
        \item Movie S1b (\texttt{SR\_N40}): Slip rate (in [m/s]) presented in original view.
    \end{itemize}
    
    \item Case $\Psi = 50^\circ$:
    \begin{itemize}
        \item Movie S2a (\texttt{SR\_E50}): Slip rate (in [m/s]) presented in exploded view.
        \item Movie S2b (\texttt{SR\_N50}): Slip rate (in [m/s]) presented in original view.
    \end{itemize}
    
    \item Case $\Psi = 60^\circ$:
    \begin{itemize}
        \item Movie S3a (\texttt{SR\_E60}): Slip rate (in [m/s]) presented in exploded view.
        \item Movie S3b (\texttt{SR\_N60}): Slip rate (in [m/s]) presented in original view.
    \end{itemize}
    
    \item Case $\Psi = 65^\circ$:
    \begin{itemize}
        \item Movie S4a (\texttt{SR\_E65}): Slip rate (in [m/s]) presented in exploded view.
        \item Movie S4b (\texttt{SR\_N65}): Slip rate (in [m/s]) presented in original view.
    \end{itemize}
    
    \item Case $\Psi = 70^\circ$:
    \begin{itemize}
        \item Movie S5a (\texttt{SR\_E70}): Slip rate (in [m/s]) presented in exploded view.
        \item Movie S5b (\texttt{SR\_N70}): Slip rate (in [m/s]) presented in original view.
    \end{itemize}
    
    \item Case $\Psi = 80^\circ$:
    \begin{itemize}
        \item Movie S6a (\texttt{SR\_E80}): Slip rate (in [m/s]) presented in exploded view.
        \item Movie S6b (\texttt{SR\_N80}): Slip rate (in [m/s]) presented in original view.
    \end{itemize}
    
    \item Case $\Psi = 90^\circ$:
    \begin{itemize}
        \item Movie S7a (\texttt{SR\_E90}): Slip rate (in [m/s]) presented in exploded view.
        \item Movie S7b (\texttt{SR\_N90}): Slip rate (in [m/s]) presented in original view.
    \end{itemize}
    
    \item Case $\Psi = 100^\circ$:
    \begin{itemize}
        \item Movie S8a (\texttt{SR\_E100}): Slip rate (in [m/s]) presented in exploded view.
        \item Movie S8b (\texttt{SR\_N100}): Slip rate (in [m/s]) presented in original view.
    \end{itemize}
    
    \item Case $\Psi = 110^\circ$:
    \begin{itemize}
        \item Movie S9a (\texttt{SR\_E110}): Slip rate (in [m/s]) presented in exploded view.
        \item Movie S9b (\texttt{SR\_N110}): Slip rate (in [m/s]) presented in original view.
    \end{itemize}
    
    \item Case $\Psi = 120^\circ$:
    \begin{itemize}
        \item Movie S10a (\texttt{SR\_E120}): Slip rate (in [m/s]) presented in exploded view.
        \item Movie S10b (\texttt{SR\_N120}): Slip rate (in [m/s]) presented in original view.
    \end{itemize}
    
    \item \textit{Case 4}, rupture nucleation on a fracture in damage zone at distance $1km$ from the main fault:
    \begin{itemize}
        \item Movie S11a (\texttt{SR\_FarSourceE65}): Slip rate (in [m/s]) presented in exploded view.
        \item Movie S11b (\texttt{SR\_FarSourceN65}): Slip rate (in [m/s]) presented in original view.
    \end{itemize} 
    
    \item \textit{Case 5}, similar to \textit{Case 4} with varying fluid pressure ratio ($\gamma$):
    \begin{itemize}
        \item Movie S12 (\texttt{SR\_Gamma05E65}): Slip rate (in [m/s]) presented in exploded view for $\gamma=0.5$.
        \item Movie S13 (\texttt{SR\_Gamma06E65}): Slip rate (in [m/s]) presented in exploded view for $\gamma=0.6$.
        \item Movie S14 (\texttt{SR\_Gamma07E65}): Slip rate (in [m/s]) presented in exploded view for $\gamma=0.7$.
        \item Movie S15 (\texttt{SR\_Gamma08E65}): Slip rate (in [m/s]) presented in exploded view for $\gamma=0.8$.
    \end{itemize}
\end{enumerate}

%\end{article}
\clearpage

\section*{Figures:} 

% Figure environment removed

\begin{sidewaysfigure}
    \centering
    % Figure removed
    \caption{Rupture times for different orientations of $S{\!}H_\mathrm{max}$ ($\Psi$).}
    \label{fig:S1}
\end{sidewaysfigure}

\begin{sidewaysfigure}
    \centering
    % Figure removed
    \caption{Final slip on the main fault for different orientations of $S{\!}H_\mathrm{max}$ ($\Psi$). The arrows in $\Psi = 50^\circ$ indicate a small amount of slip on the main fault. Please note the colorboar changes across all panels, reflecting smaller values of maximum slip, particularly for $\Psi < 70^\circ$, indicating less slip on the main fault previously not visible in the colorbar of the main manuscript.}
    \label{fig:S_main_fault_slip}
\end{sidewaysfigure}

\begin{sidewaysfigure}
    \centering
    % Figure removed
    \caption{Final fault slip for different orientations of $S{\!}H_\mathrm{max}$ ($\Psi$).}
    \label{fig:S2}
\end{sidewaysfigure}



% Figure environment removed

% Figure environment removed

% Figure environment removed


% Figure environment removed

%%%% reserved for rupture speed
% Figure environment removed

\begin{sidewaysfigure}
    \centering
    % Figure removed
    \caption{Exploded view of rupture time evolution and final slip of Scenario 2 for two examples: a) $\gamma = 0.6$ and b) $\gamma=0.7$.}
    \label{fig:S6}
\end{sidewaysfigure}


% Figure environment removed

% Figure environment removed

% Figure environment removed


%\section*{References}
%\bibliography{agusample.bib}

%\end{document}
