\documentclass{ledger}

\usepackage{mathrsfs}
\usepackage{mathtools}
\usepackage{centernot}
\usepackage{comment}

%AUTHOR: This is a bare-bones template into which you may put your paper to aid in formatting your submission to Ledger. Note that to work properly, you must have the files "ledger.cls", "ledgerbib.bst", and the folder "images" on hand. 

%AUTHOR: the preferred method to generate PDF output is to use 'pdflatex'
%To clean up after a successful build, try: 'latexmk -c main.tex'

%EDITOR: replace X's to set the data for the header and footer
\newcommand{\thefirstpagenum}[0]{X}
\newcommand{\thelastpagenum}[0]{X}
\newcommand{\theyear}[0]{20XX}
\newcommand{\thevol}[0]{X}
\newcommand{\thedoi}[0]{DOI 10.5195/LEDGER.\theyear.XXX}
\newcommand{\ledgerpages}[0]{\thefirstpagenum-\thelastpagenum}


%AUTHOR: please set these to generate correct PDF metadata
\hypersetup{pdfauthor={Vallarano Nicol`'o.; 
Squartini, Tiziano.; Tessone, Claudio J..}, pdftitle={Exploring the Bitcoin Mesoscale}}

%EDITOR: set the correct pageination during layout
%\setcounter{page}{\thefirstpagenum}


%AUTHOR: this can be used to highlight changed text, surround with \edit{} and
%uncomment either to determine color
%\newcommand{\edit}[1]{{\color{red} #1}}
\newcommand{\edit}[1]{#1}
	
\overfullrule=10pt

\title{Exploring the early Bitcoin Mesoscale}
\author{Nicol\`o Vallarano \thanks{Blockchain \& Distributed Ledger Technologies Group, Department of Informatics, University of Zurich, Switzerland \\ UZH Blockchain Center, University of Zurich, Switzerland},  Tiziano Squartini \thanks{IMT School for Advanced Studies, Piazza San Francesco 19, 55100 - Lucca (Italy) \\ Institute for Advanced Study (IAS), University of Amsterdam, Oude Turfmarkt
145, 1012 GC - Amsterdam (The Netherlands)}, Claudio J. Tessone\thanks{Blockchain \& Distributed Ledger Technologies Group,
Department of Informatics, University of Zurich, Switzerland \\ UZH Blockchain Center, University of Zurich, Switzerland}}

\pagestyle{pagemain}


%The Author should select the appropriate pretitle below:
\pretitle{
  \centering \selectfont LEDGER \LaTeX \ TEMPLATE \par
  %\centering \selectfont REVIEW ARTICLE \par
  %\centering \selectfont RESEARCH ARTICLE \par 
  \fontsize{24pt}{28pt}\selectfont} % Title is centered and at 24pt


\begin{document}

\maketitle

\thispagestyle{pagefirst}

\begin{abstract}
The open availability of the entire history of the Bitcoin transactions opens up the possibility to study this system at an unprecedented level of detail. This contribution is devoted to the analysis of the \emph{mesoscale} structural properties of the Bitcoin User Network (BUN), across its entire history (i.e. from 2009 to 2017). What emerges from our analysis is that the BUN is characterized by a \emph{core-periphery} structure a deeper analysis of which reveals a certain degree of \emph{bow-tieness} (i.e. the presence of a Strongly-Connected Component, an IN- and an OUT-component together with some tendrils attached to the IN-component). Interestingly, the evolution of the BUN structural organization experiences fluctuations that seem to be correlated with the presence of \emph{bubbles}, i.e. periods of price surge and decline observed throughout the entire Bitcoin history: our results, thus, further confirm the interplay between structural quantities and price movements observed in previous analyses.

%AUTHOR: keywords are OK to show for Review article, will be hidden and added to metadata for publication
\begin{keywords}
\item Bitcoin.
\item Networks.
\item Motifs.
\item Transactions.
\end{keywords}
\end{abstract}

\section{Introduction}
Introduced in 2008 by Satoshi Nakamoto with the release of a white paper \cite{nakamoto2008bitcoin}, Bitcoin is the first and most widely adopted \emph{cryptocurrency}. Loosely speaking, it consists of a decentralised peer-to-peer network to which users connect to exchange native tokens (i.e. the \emph{bitcoins}). After having been validated by the so-called \emph{miners} - according to the consensus rules that are part of the Bitcoin protocol \cite{halaburda2016beyond,glaser2017pervasive} - each transaction is included in a replicated database, i.e. the \emph{blockchain}. The cryptographical protocols Bitcoin rests upon aim at preventing the possibility for the same digital token to be spent more than once, in absence of a central, third party that guarantees the validity of the transactions themselves \cite{nakamoto2008bitcoin,antonopoulos2017mastering}: remarkably, the transaction-verification mechanism Bitcoin relies on allows its entire transaction history to be openly accessible - a feature that, in turn, allows it to be analyzable in the preferred representation.

The Bitcoin structural properties have been only recently started to be investigated: in \cite{javarone2018from}, the authors consider the Bitcoin user network at the \emph{macroscale}, in order to check for its small-worldness; in \cite{bovet2019evolving}, the authors explore the evolution of the \emph{local} properties of four different representations of Bitcoin (i.e. both the \emph{user} and the \emph{address} network, at both the \emph{daily} and \emph{weekly} time scale) with the aim of investigating their relationship with the price movements; in \cite{lin2020lightning}, the authors analyze the so-called Bitcoin Lightning Network (BLN), highlighting the increasingly centralized character of such a network structure.

With the present paper, we aim at contributing to this stream of research by studying the structure of the Bitcoin User Network (BUN) at the \emph{mesoscale}.


\section{Data}
\subsection{Data Subsection}
As said before, Bitcoin relies on a decentralised public ledger, the blockchain, that records all transactions of bitcoins among users. A transaction is nothing else than a set of input and output addresses: the output addresses that are said `unspent', i.e. not yet recorded on the ledger as input addresses, can be claimed, and therefore spent, only by the owner of the corresponding cryptographic key. This is the reason why one speaks of \emph{pseudonimity}: an observer of the blockchain can see all unspent addresses but cannot link them to the actual owners. Techniques exist, however, to infer the identity of users (hereby, this term will indicate `groups of addresses'): they rest upon the the so-called \emph{heuristics}, i.e. sets of rules taking advantage of the implementation of the Bitcoin protocol.

\begin{table}[!t]
\centering
\begin{tabular}{|l|l|l|l|}
\hline
Bubble & Start       & End         & Days \\ \hline
1      & 25 May 2012 & 18 Aug 2012 & 84   \\ \hline
2      & 3 Jan 2013  & 11 Apr 2013 & 98   \\ \hline
3      & 7 Oct 2013  & 23 Nov 2013 & 47   \\ \hline
4      & 31 Mar 2017 & 18 Dec 2017 & 155  \\ \hline
\end{tabular}
\caption{The four bitcoin bubbles as detected in \cite{wheatley2019bitcoin}.}
\label{tab:bubs}
\end{table}

\paragraph{Bitcoin Address Network (BAN).} The Bitcoin Address Network is the simplest network that can be constructed from the blockchain records: it is a directed, weighted graph whose nodes represent addresses. The direction and the weight of links are provided by the input-output relationships defining the transactions recorded on the blockchain. The only free parameter is represented by the temporal window chosen for the data aggregation. In this paper, we chose to aggregate these data at a daily time scale, i.e. the shortest scale that still guarantees that the resulting network is connected \cite{nakamoto2008bitcoin}.

\paragraph{Bitcoin User Network (BUN).} Since the same owner may control several addresses, one can derive a network of users whose nodes are \emph{clusters of addresses}. These clusters are derived by implementing different \textit{heuristics}: let us now provide a brief description of the ones that have been employed here and that have been have derived from the state-of-the-art literature \cite{androulaki2013evaluating,tasca2018evolution,harrigan2016unreasonable,ron2013quantitative}. 

The first heuristics is the so-called \emph{multi-input heuristics}: it is based on the assumption that two (or more) addresses that are part of the input of the same transaction are controlled by the same user. The key idea behind this heuristics is that the private keys of all addresses must be accessible to the creator of a transaction, in order to produce it. It is generally believed to be the safest one for clustering addresses.

The second heuristics is the so-called \emph{change-address identification heuristics} and is based upon the observation that transaction outputs must be fully spent upon re-utilisation; hence, the transaction creator usually controls also one of the output addresses. More specifically, we assume that if an output address is new and the amount transferred to it is lower than all the inputs, then it must belong to the input user.

Whenever the Bitcoin User Network is mentioned in the paper, we refer to the representation obtained by clustering the nodes of the BAN according to a combination of the two heuristics above. Naturally, users can employ different wallets that are not necessarily linked together by transactions: as a consequence, the user network we obtain should not be considered as a perfect representation of the actual network of users but, rather, an attempt enabling us to group addresses while minimising the presence of false positives. Once the addresses are grouped into `users', we build the network as follows: any two nodes $i$ and $j$ are connected via a directed edge from $i$ to $j$ if at least one transaction from one of the addresses defining $i$ to one of the addresses defining $j$ occurs at the considered time scale.\\

In the present work, when not otherwise stated, we consider the Bitcoin User Network at a weekly time scale. 


\paragraph{Price bubbles data}
The valuation of cryptocurrencies is an emerging field of studies, and there's no unanimous consensus on what the bitcoin real value should be nor on the actual definition to detect bitcoin bubbles. In this paper we rely on the bubbles identified in \cite{wheatley2019bitcoin}.
The method to identify bubbles developed in \cite{wheatley2019bitcoin} uses a generalized version of Metcalfe's law to determine bitcoin fundamental value, which is shown to be heavily exceeded, on at least four occasions, listed in table \ref{tab:bubs}.\\



\section{Results}
\paragraph{Connected Components}
% Figure environment removed
As a first inspection we observe the transaction networks connected components evolution over time.

Aside from the initial phase until fall 2010, we can detect the emergence of both the huge Weakly-Connected Component(WCC) and the Strongly-Connected Component(SCC) in two different points in time. Figure \ref{fig:cc_evolution} panel (c) shows the ratio between the largest connected component and the second largest one. As one can see both at the strong ad weak level the huge connected component emerge after 2012.

In panel (a) we observe that after the initial phase two clear different trends emerge at the strong and weak level: the huge weak connected component size is stable around the $80\%$ of the total number of nodes, while the huge strong connected component oscillates between the $10\%$ and the $40\%$ of the nodes; in the specific we observe a long plateau from 2014 to 2016 where the huge strong connected component size is around $40\%$ of the total nodes and then gets back to the stable value of $10\%$.

At last, panel (b) shows the number of connected components: it depicts a situation where the large majority of connected components is composed of a very small number of nodes.

It seems clear that while a large majority of nodes is connected, the paths drawn by transactions are usually one way, implying the reduced dimension of the strongly connected component. Then nodes out of the main cluster tend to be isolated, thus explaining the large number of connected components.
%As a first preliminary inspection we studied the number and size of the BUN connected components.
%Not only there is a large number of connected components (a large majority very small), but this number also increases over time, both for the weak and strong definitions.
%A giant component clearly emerges, couple of order of magnitude larger than the second connected component and comparable to the total size of the networks.
%The large number of connected components is justified by a huge amount of isolated nodes(our work hypothesis is they are users who move money among their various addresses).



\paragraph{Bitcoin Disassortativity}

% Figure environment removed

Let us now consider the assortativity of Bitcoin Transactions Networks. 
A network is said to be assortative when nodes with large degree tend to connect to each other, as opposed to disassortative networks, where nodes with large degree connects to nodes with low degree.
Following \cite{newman2003mixing}, on undirected networks one defines the assortativity coefficient $r_{und}$ as:
\begin{equation}\label{eq:newman_correlation}
r_{und}=\frac{\sum_{j,k}jk(e_{jk}-q_jq_k)}{\sigma^2_q}
\end{equation}
where the sum runs over the 'excess degrees' of a node: imagine to reach a vertex following a specific edge, the 'excess degree' of the vertex is the degree of the vertex minus that specific edge you followed.
 $q_k$ is the `excess degree' probability distribution, reading

\begin{equation}
q_k \propto p_{k+1}
\end{equation}
(with $p_{k+1}$ being the plain degree distribution), $\sigma^2_q$ is its standard deviation and $e_{jk}$ is the fraction of edges in the network connecting nodes of degree $j$ with nodes of degree $k$. Naturally, $\sum_je_{jk}=q_k$.

When considering directed networks \cite{noldus2015assortativity}, instead, four variants of the aforementioned Pearson coefficient can be calculated, i.e. the ones accounting for the correlation between out-degrees and out-degrees, out-degrees and in-degrees, in-degrees and out-degrees, in-degrees and in-degrees. For example, one of thee variants reads

\begin{equation}
r_{dir}^{out-in}=\frac{\sum_{j,k}jk(e_{jk}-q_j^{out}q_k^{in}) }{\sigma_{q^{out}}\sigma_{q^{in}}}
\end{equation}
where $e_{jk}$ now represents the percentage of edges starting from nodes whose out-degree is $j$ and ending on nodes whose in-degree is $k$. Naturally, it also holds true that $\sum_je_{jk}=q_k^{in}$.



Plotting the evolution of the aforementioned coefficients on our BUNs shows their weakly disassortative nature (see figure \ref{fig:degree_assortativity_r}). In particular, since $r_{dir}^{out-in}$ is `asymptotically' zero, one can conclude that $e_{jk}\simeq q_j^{out}q_k^{in}$ - and analogously for the other indices of direct assortativity (the years until 2011 can be considered as a `transient' period where the Bitcoin ecosystem was still of reduced dimensions, hence sensitive to even small structural changes).

\paragraph*{Bitcoin core-periphery structure.} Our first result concerns the Bitcoin \emph{core-periphery-ness}. In order to analyse it, we have run a recently-proposed method \cite{de2019detecting} based on the multivariate extension of the \emph{surprise} score function. 

% Figure environment removed
% Figure environment removed

% Figure environment removed


% >>>> NEW
The application of the multivariate surprise to the partition induced by the bow-tie structure reveals that it indeed induces a significant core-periphery structure (the surprise score function is steadily below the threshold of $5\%$), the core being the SCC and the periphery being composed by all the other nodes. Figure \ref{fig:core_pc} shows the evolution of the percentage of nodes composing the core  and the periphery of our BUNs. As expected from the results concerning the SCC, the periphery contains the vast majority of nodes throughout the entire Bitcoin history.

As fig. \ref{fig:core_pc} shows, there seem to be three different phases: the first one coincides with the biennium 2012-2014, during which the core portion of the BUN steadily rises until it reaches the 40\% of the network; afterwards, during the biennium 2014-2016, it remains quite constant; then, during the last two years covered by our dataset (i.e. 2016-2018), the core portion of the BUN shrinks and the percentage of nodes belonging to it goes back to the pre-2012 values. In order to gain insight into the correlations between the evolution of purely topological quantities and the Bitcoin price, let us plot the trend of the \emph{temporal z-score}, defined as

\begin{equation}\label{eq3}
z_X^{(t)}=\frac{X^{(t)}-\overline{X}}{s_X}
\end{equation}
for a generic quantity $X$, where the mean $\overline{X}=\sum_t\frac{X^{t}}{T}$ and the standard deviation $s_X=\sqrt{\overline{X^2}-\overline{X}^2}$ have been computed over a sample of values covering the six months before time $t$. 

% - quanto è lunga la rolling window? Esattamente sono 26 settimane, quindi 6 mesi e mezzo

As fig. \ref{fig:zscore_core} and \ref{fig:zscore_periphery} show, the calculation of the temporal z-score for the percentage of core nodes/periphery links reveals the presence of peaks in correspondence of the first three bubbles (identified by the shaded areas), thus indicating the existence of periods during which the price and the structural quantities of interest co-evolve. In particular, while the number of links within the core and the periphery rises significantly, with respect to the previous temporal interval, periods in-between the bubbles are, instead, characterized by a decrease of the statistical significance of the same quantities.

\paragraph{Bitcoin bow-tie structure.} 

The core-periphery structure of the BUN is characterized by an additional structure, known as \emph{bow-tie} structure. The definition of \emph{bow-tieness} rests upon the concept of \emph{reachability}: we say that $j$ is \textit{reachable} from $i$ if there exist a path from $i$ to $j$. A directed graph is said to be \textit{strongly connected} if any two nodes are mutually reachable. Mutual reachability is an equivalence relation on the vertices of a graph, the equivalence classes being the strongly connected components of the graph itself. The bow-tie decomposition of a graph, hence, consists of the following sets of nodes \cite{de2018reconstructing}:

\begin{itemize}
\item Strongly-connected Component:$\text{SCC}\equiv S$. \\
    Each node within the Strongly-Connected Component(SCC) can be reached by any other node within it. This means that a directed path exists connecting each node with each other node;
\item In-Component $\equiv\{i\in V\setminus S\:|\:\text{$S$ is reachable from $i$}\}$;
\item Out-Component $\equiv\{i\in V\setminus S\:|\:\text{$i$ is reachable from $S$}\}$;
\item Tubes $\equiv\{i\in V\setminus S\:\cup\:\text{IN}\cup\:\text{OUT}\:|\:\text{$i$ is reachable from IN and OUT is reachable from $i$}\}$;
\item In-Tendrils $\equiv\{i\in V\setminus S\:|\:\text{$i$ is reachable from IN and OUT is not reachable from $i$}\}$;
\item Out-Tendrils $\equiv\{i\in V\setminus S\:|\:\text{$i$ is not reachable from IN and OUT is reachable from $i$}\}$;
\item Others: $\equiv\{i\in V\setminus S\:\cup\:\text{IN}\:\cup\:\text{OUT}\:\cup\:\text{TUBES}\:\cup\:\text{IN-TENDRILS}\:\cup\:\text{OUT-TENDRILS}\}$.
\end{itemize}

% >>>>>> OLD
% Inspecting the bow-tie-ness of the BUN reveals that the core portion of the BUN is, actually, its SCC while the periphery gathers all other node subsets. Interestingly, while in the biennium 2
% 014-2016 the percentage of nodes constituting the SCC is larger than the percentage of nodes belonging to the other subsets, since 2016 this is no longer true: in fact, while both the SCC and the OUT-component shrink, the IN-component becomes the dominant portion of the network.\\
% Figure environment removed
% >>>>>> NEW
 Generally speaking, a large SCC, incorporating the vast majority of nodes, starts emerging in 2012, `stabilizes' around mid-2013 and persists until 2016. More specifically, during the biennium 2012-2013 the SCC steadily rises until it reaches $\simeq 30\%$ of the network size; afterwards, during the biennium 2014-2015, it remains quite constant; then, during the last two years covered by our data set (i.e. 2016-2018), it shrinks and the percentage of nodes belonging to it goes back to the pre-2012 values. While in the biennium 2014-2016 the percentage of nodes constituting the SCC is larger than the percentage of nodes belonging to the other subsets, since 2016 this is no longer true: in fact, while both the SCC and the OUT-component shrink, the IN-component becomes the dominant portion of the network.

Different results have been reported in \cite{maesa2019bow}: however, this may be due to the different data collection and data mining processes implemented there.



\paragraph{Dyadic motifs.} Let us now consider the structure of dyadic motifs, meaning the patterns involving two nodes.
Althought the number of dyadic motifs is very limited and of easy definition, these may be of interest to study the evolution of economic networks \cite{Squartini:2013, squartini2015stationarity}.

On a directed network $G(N, E)$, given two nodes $i,j \in E$ you can observe three disjoint possibilities:
\begin{itemize}
    \item a \textit{reciprocated dyad}: $(i,j), (j,i) \in E$, meaning a connection exists both from $i$ to $j$ and vice versa. We denote the total number of reciprocated dyads in the network as $L^{\leftrightarrow}$
    \item a \textit{non reciprocated dyad}: $(i,j)\in E$ or $(j,i)\in E$, meaning only one of the two possible links exists. We denote the total number of non reciprocated dyads as $L^{\rightarrow}$
    \item an \textit{empty dyad}$(i,j), (j,i) \notin E$, meaning no link exists between the two nodes.
    The total number of empty dyads is denoted by $L^{\centernot{\leftrightarrow}}$
\end{itemize}

\begin{table}[t!]
\centering
\begin{tabular}{c|c}
\hline
Dyadic motif $m$ & Abundance $N_m$ \\
\hline
$L^\nleftrightarrow$ & $\sum_{i\neq j}(1-a_{ij})(1-a_{ji})$\\
$L^\rightarrow$ & $\sum_{i\neq j}a_{ij}(1-a_{ji})$\\
$L^\leftrightarrow$ & $\sum_{i\neq j}a_{ij}a_{ji}$\\
\hline
\end{tabular}
\caption{Definition of the abundance of dyadic motifs.}
\label{tab:motifs_dyadic}
\end{table}


% >>>>> OLD 
% We want to explore deeper the 2-motifs structure of the networks dynamics over time, studying the evolution of reciprocated, non-reciprocated and empty dyads.
% In order to do so, we calculate the Directed Configuration Model of each BTC network snapshot we have, and then we use them to compute the time-series of each 2-motif z-score over the null model.
% >>>>>> NEW
Let us now study dyadic motifs by adopting a different approach with respect to the one employed so far. Instead of using the temporal $z$-score to spot `temporal' outliers, i.e. values that are statistically significant with respect to a time average, let us consider an index that points out quantities not compatible with a given null model. In order to do so, we will employ the Directed Binary Configuration Model as null model, from the Exponential Random Graph Model(ERGM) family\cite{Squartini:2013,garlaschelli2008maximum}. In this framework, a $z$-score of the kind

\begin{equation}\label{eq:zscore_sample}
z[X]=\frac{X(\mathbf{A}^*)-\langle X\rangle}{\sigma[X]}
\end{equation}
remains naturally defined, where $X(\mathbf{A}^*)$ is the empirical value of the quantity of interest (i.e. observed on the original network $\mathbf{A}^*$), $\langle X \rangle$ and $\sigma[X]$ are, respectively, its expectation value and its standard deviation, both computed on the ensemble induced by the DBCM. The interpretation of this $z$-score is the following one: values such that $z[X]>+3$ signal that the empirical value is significantly larger than expected while values such that $z[X]<-3$) signal that the empirical value is significantly smaller than expected. In both cases one may conclude that the empirical value $X(\mathbf{A}^*)$ is not compatible with the specific model and something else is required to fully account for it. On the other hand, if $-3\leq z[X]\leq+3$, there is no evidence of a significant deviation from the expected value and one may conclude that $X(\mathbf{A}^*)$ is completely explained by the constraints defining the model at hand.

% % Figure environment removed

This exercise gives us an idea of how much the weekly BTC transaction network differ from the randomized network derived from its degree distribution.
The results are shown in \ref{fig:dyads}. The first thing that stands out are the out-of-scale values of the reciprocated dyads z-score, which makes totale sense:the BUNs (and in general the networks obtained from BTC transactions) are very sparse network, were the large majority of nodes have very few connections; it makes sense that each bilateral connection would be very odd in comparison to a randomized network with the same number of nodes and links.

% >>>>> OLD
% The second thing we want to stress is the relationship between the bubbles growth periods(the red shades in the plot)  and the 2 motifs dynamics; the empty a non-reciprocated dyads have opposed and synced trends: empty dyads(non reciprocated dyads) increase(decrease) just before the bubble growth time to stabilise later on. Also we notice a huge bump just after Mt.Gox failure in 2014.
% >>>>> NEW
The interpretation of the behavior of empty and single dyads is analogous: by chance, a larger-than-observed number of non-reciprocated dyads are created, whence their over-representation within the DBCM ensemble and the negative $z$-score recovered by our analysis. In order to understand why this implies that the DBCM tends to create less-than-observed empty dyads, let us imagine to `destroy' a reciprocal dyad, by decoupling the two paired links: in order to create `more' single dyads, one of the two links mus be redirected towards a previously disconnected node; upon doing so, a reciprocal dyad disappears, as well as an empty dyad, while two single dyads are created.

% Figure environment removed

\paragraph{Centrality and centralization}

We computed four different centrality measures: degree centrality, closeness centrality, betweenness centrality and eigenvector centrality.

The normalized closeness centrality is the average length of the shortest path between a node and all the others. The more central a node is, the closer it is to all the other nodes.
Betweenness centrality denotes the number of times a node is on the shortest path connecting two other nodes. It was introduced to measure the control a node had on the communication layer between other nodes on the network
Eigenvector centrality measures the influence of a node over the network. Each node is assigned a relative score, based on the idea that being connected to high-scored nodes increases a node score.

We condensate the information given by centralities for each network datapoint by considering two different aggregate indices derived from each centrality distribution: the Gini index and the centralisation index.

%Following \cite{lin2020lightning}, we investigate Bitcoin (de)centralization and (in)equality by considering the Gini coefficient of our degree distributions. 

The Gini coefficient attempts to measure the unevenness of a distribution of a certain quantity\footnote{Usually, it is employed to measure the unevenness of the income distribution\cite{dixon1987bootstrapping}}: given a set of values $\{c_i\}_{i=1}^N$, the Gini index is defined as

\begin{equation}
G_c=\frac{\sum_{i=1}^N\sum_{j=1}^N|c_i-c_j|}{2N\sum_{i=1}^N{c_i}}
\end{equation}
and assumes values between $0$ and $1$; while a Gini index of $0$ indicates perfect evenness (e.g. everyone has exactly the same income), a Gini index of $1$ indicates perfect unevenness (e.g. a population whose entire income is concentrated in the hands of a single individual). 

Applying the Gini coefficient to the degrees of our BUNs aims at shedding light on the (un)evenness of the nodes degree centralization: while a value close to $0$ would depict an ecosystem where all actors have exactly the same number of interactions with each other, a value close to $1$ would indicate that there are nodes participating to the vast majority of transactions.

Centralization indices are global measures intended to measure the centrality of the entire network (instead of providing a rank of its nodes). In mathematical terms, the centralization reads

\begin{equation}
C_c=\frac{\sum_{i=1}^N(c^*-c_i)}{\max\left\{\sum_{i=1}^N(c^*-c_i)\right\}}
\end{equation}
where $c^*=\max\{c_i\}_{i=1}^N$ represents the empirical, maximum value of the chosen centrality measure (i.e. computed on the network under consideration) and the denominator is calculated over a benchmark graph, defined as the one providing the maximum attainable value of the quantity $\sum_{i=1}^N(c^*-c_i)$. 

The benchmark graph for which is nothing else than a star graph (with the same number of nodes of the network under inspection). For each centrality we should compute the centralization index by computing the star-graph centralization maximum value.

\begin{equation}
C_k=\frac{\sum_{i=1}^N(k^*-k_i)}{(N-1)(N-2)}
\end{equation}
and the degree centralization would reveal to us if (and, in case, `how much') Bitcoin has become similar to a star graph at a certain point during its history.

\begin{comment}
% Figure environment removed
\end{comment}
% Figure environment removed


From the results in figure \ref{fig:deg_centralisation} appears that, after a period of growth lasted until mid-2013, during which it reached values as large as $0.75$, the Gini coefficient has decreased and is now steadily around the value of $0.5$. Overall, we would like to stress that $0.5$ is not a small value: in fact, it describes an ecosystem where the $50\%$ of connections are incident to the $1\%$ of nodes. It is also interesting to notice the big leap down of the Gini coefficient in 2013: during that year, Mt. Gox (which managed $\simeq70\%$ of transactions at the time \cite{decker2014bitcoin}) started the down-ward spiral which eventually led to its bankruptcy in 2014: USD withdrawals halting, financial investigations and expensive lawsuits weakened the trading website ability to stay on the market. The final blow was the public discovery of a huge theft of around $750.000$ bitcoins, which went on undetected for years.

The huge decrease of the Gini coefficient may be, thus, related to the Mt. Gox `loss of prominence' in the Bitcoin ecosystem. On the other hand, bubble periods seem to have little correlation with the evolution of the Gini coefficient.\\

The evolution of the Gini coefficient may lead us to imagine that the Bitcoin ecosystem has become similar to a very centralized structure, pretty much similar to a star graph, at some point during its history. In order to answer this question, we have computed the so-called \textit{centralization index} at the weekly time scale (from \cite{lin2020lightning}). 



\begin{comment}

% Figure environment removed
\end{comment}


While during the initial phases of its life, Bitcoin was indeed quite similar to a star graph, figure \ref{fig:deg_centralisation} reveals that the degree-centralization has quickly stabilized around very small values.
Overall, we may, thus, conclude that Bitcoin is not evolving towards a star-like structure, where a single central node participates to all transactions.
However, the large value of the Gini coefficient let us suspect that there may be several hubs: hence, the unrealistic picture of a star-like structure may be replaced by the more realistic one depicting several `locally star-like' structures. 
The centers of these structures are `local hubs', i.e. vertices with a large number of connections, that are crossed by a large percentage of paths and that are connected among them. In order to justify the apparent contradiction, in the appendix we show a toy model which partially reproduces the results.


\paragraph{Small-world}

% Figure environment removed

In order to explore the small-word properties of Bitcoin Transaction Networks, we compute the Average Path Length (APL) $L$ and the global clustering coefficient $c$ evolution of the BUNs in our dataset\cite{albert2002statistical}. Because of computational constraints, the analysis is limited to the first half of the period under examination(until 2014). Please note that the measures are taken on the giant connected component.
The results are displayed in figure \ref{fig:small_world}: in the period under exam, both the measures seems to be stable. The clustering coefficient floats around very small values, in the order of $10^{-2}$, while the APL is stable around $10$ and $15$, with rare exceptions.

\begin{comment}
% Figure environment removed
\end{comment}
From figure \ref{fig:small_world} (a) it is clear that a linear relation between $L$ and $\log{N}$ does hold. This means that at least the close-connected behaviour of small-world network is present in Bitcoin Transaction Networks.
Keeping this in mind, we can't ignore that the BUNs we analysed have a very low clustering nature, meaning that it's hard to observe two neighbours nodes(meaning users) of a node connected among each others.

In order to put the clustering values in perspective we plot the BUNs coefficient versus the Erdos-Renyi randomized clustering coefficient.
In figure \ref{fig:small_world} (b) we can see that while the magnitude of the clustering is small both on the original and on the Erdos-Renyi randomization, we notice that constantly over time the original clustering is at least one order of magnitude larger than the Erdos-Renyi: this means that while there are not many triangles, Bitcoin Transactions Networks have more triangles than an equivalent random graph.

\section{Discussion}
%\textcolor{blue}{What an analysis based on network theory can add is the identification of periods otherwise not signalled. This is again the case for the period between the third and the fourth bubble where three spikes are visible.}\\

In this paper we analyse the impact of  structural properties of the Bitcoin transaction network on the generation and  crash of bubbles in the exchange with respect to fiat currencies. Specifically, we examine network features such as heterogeneity of the degree distributions and frequency of connectivity patterns (i.e., motifs). We find significant changes in these properties during the period of price bubbles. A more detailed analysis unveils that, during the first bubble, the frequency of motifs indicating the relationship hubs have with new, low-degree users changes significantly; this suggests that hubs take an even more important role in becoming liquidity providers. These results are confirmed in the second bubble: There, by analysing the heterogeneity of the in-degree, out-degree, and total degree distributions, we find that there is a significant widening (narrowing) of the out-degree (in-degree) distributions, whereas the total degree does not change its distribution significantly. 
 By performing additional analyses on the two largest hubs, we find that these structural changes - similar to what is observed during the first bubble - is likely to be caused by the \emph{centralising} role hubs take on as liquidity providers. 

Although we find that measures can explain well some price bubbles but not others, these results highlight that tracking properties of hubs in the transaction network is key for understanding the underlying mechanisms of a bubble. Moreover, at least in the first three Bitcoin bubbles, the behaviour of hubs significantly increased the systemic risk of the Bitcoin economy, eventually leading to systemic failure and sudden price crashes. 

These results also suggest that Bitcoin bubbles are difficult to forecast, but can be prevented, or at least alleviated, by introducing policies that aim at reducing the importance of large hubs in the network. In future work, we plan to extend our analysis by introducing new structural measures and by covering all the bubbles that happened to date.

These structural changes suggest the presence of two hubs that centralise the market by selling bitcoins to most of the traders that enter the market during the bubble, resulting in a significant increase of the systemic risk. Indeed, if only a few hubs account for most of the transactions in the network, if at any point in time one of them fails, the whole network may crash. This is exactly what happened on April 10th 2013, when Mt Gox, the major Bitcoin exchange, broke under the high trading volume, triggering the burst of the bubble.

%define the following sections to hide their Section Number (Notes Style)
\ledgernotes
\begin{comment}
\section*{Acknowledgements}

C.J.T. acknowledges financial support of the University of Zurich through the University Research Priority Program on Social Networks.

\section*{Author Contributions}

N.V. and F.M. performed the analysis. N.V., F.M., T.S. and C.J.T. designed the research. All authors wrote, reviewed and approved the manuscript.

\section*{Competing Interests}

The authors declare no competing financial interests.
\end{comment}




%AUTHOR: comment out if using thebibliography
%\theendnotes

%AUTHOR: please read ledgerbib.bst usage notes by opening it in a text editor. We have modified it to include the use of the @misc item type for the proper formatting of online sources.

\bibliographystyle{ledgerbib}
\bibliography{tempbib}

%AUTHOR: comment out, this is used to make sure the Creative Commons License
%image fits on page

\newpage 	
%define the following sections to have the Appendix Style

\appendix
\setcounter{section}{0}
\section{Surprise}
Originally proposed to detect communities \cite{aldecoa2013surprise,aldecoa2013exploring,nicolini2016modular}, the surprise reads

\begin{equation}\label{eq1}
\mathscr{S}=\sum_{l\geq l_\bullet^*}^{\min\{L, V_\bullet\}}\frac{\binom{V_\bullet}{l}\binom{V_\circ}{L-l}}{\binom{V}{L}}
\end{equation}
where $V$ is the volume of the network (coinciding with the total number of node pairs, i.e. $V=N(N-1)$ in the directed case), $V_\bullet$ is the total number of intracluster pairs (i.e. the number of node pairs \emph{within} the individuated communities), $L$ is the total number of links and $l^*$ is the observed number of intracluster links (i.e. \emph{within} the individuated communities). In other words, $\mathscr{S}$ is the p-value of an hypergeometric distribution, aimed at testing the statistical significance of a given partition. Such a distribution, in fact, describes the probability of observing $l$ successes in $L$ draws (without replacement) from a finite population of size $V$ that contains exactly $V_\bullet$ objects with the desired feature (in our case, being an intracluster pair), each draw being interpreted as either a `success' or a `failure': the lower this probability, the `better' the individuated partition (i.e. the more surprising its observation).

When carrying out a community detection exercise, links are understood as belonging to two different categories, i.e. the ones \emph{within} the clusters and the ones \emph{between} the clusters. The surprise-based formalism, however, can be extended to detect \emph{bimodular} structures, a terminology that is intended to describe either \emph{bipartite} or \emph{core-periphery} structures. In this case, three different `species' of links are needed (e.g. the core, the periphery and the core-periphery ones): for this reason, we need to consider the multinomial version of the surprise \cite{de2019detecting}, reading

\begin{equation}\label{eq2}
\mathscr{S}_\parallel=\sum_{i\geq l_\bullet^*}\sum_{j\geq l_\circ^*}\frac{\binom{V_\bullet}{i}\binom{V_\circ}{j}\binom{V-(V_\bullet+V_\circ)}{L-(i+j)}}{\binom{V}{L}}
\end{equation}
(the presence of three different binomial coefficients allows three different kinds of links to be accounted for). From a technical point of view, $\mathscr{S}_\parallel$ is the p-value of a multivariate hypergeometric distribution, describing the probability of $i+j$ successes in $L$ draws (without replacement), from a finite population of size $V$ that contains exactly $V_\bullet$ objects with a first specific feature and $V_\circ$ objects with a second specific feature, wherein each draw is either a `success' or a `failure': analogously to the univariate case, $i+j\in [l_\bullet^*+l_\circ^*,\min\{L,V_\bullet+V_\circ]$.

According to the interpretation proposed in \cite{de2019detecting}, revealing the core-periphery structure minimizing the surprise means individuating the partition that is least likely to be explained by invoking the Random Graph Model (RGM) with respect to the Stochastic Block Model (SBM). This, in turn, means that the BUN is indeed characterized by subgraphs with very different link densities, an evidence that cannot be explained by a model defined by just one global parameter (i.e. the one characterizing the RGM).
\section{Reciprocity}
The ratio of reciprocated dyads over the total number of links is called \emph{reciprocity}, and reads:

\begin{equation}
r=\frac{\sum_i\sum_{j(\neq i)}a_{ij}a_{ji}}{\sum_i\sum_{j(\neq i)}a_{ij}}\equiv\frac{L^{\leftrightarrow}}{L}
\end{equation}

 The evolution of the BUN reciprocity structure is plotted in fig. \ref{fig:rec}: the first observation concerns the absolute value of $r$ which amounts at few percents. This means that, overall, the reciprocity of the BUN is quite low (although the picture changes if we consider the value of the reciprocity within the SCC). Interestingly, however, the evolution of the reciprocity shows some peaks of activity in correspondence of the bubbles; the relevance of this kind of information is confirmed upon calculating the reciprocity temporal z-score defined in eq. \ref{eq3}: it rises significantly in correspondence of the bubbles and decreases during the inter-bubble periods.

% Figure environment removed

\section{Centralization formulas}

The benchmark graph for which is nothing else than a star graph (with the same number of nodes of the network under inspection): hence, here we report the mathematical formulas from  \cite{lin2020lightning} for the centralization indices:

\begin{equation}
C_k=\frac{\sum_{i=1}^N(k^*-k_i)}{(N-1)(N-2)}
\end{equation}
where $\{k_i\}_i$ is the degree centralization of each node.
\begin{equation}
    C_{c^C} = \frac{\sum_{i=1}^N (c^* - c^c_i)}{(N-1)(N-2)/(2n-3)}
\end{equation}
Where $\{c_i^c\}_i$ is the normalized closeness centrality of each node. 
\begin{equation}
    C_{b^C} = \frac{\sum_{i=1}^N (b^* - b^c_i)}{(N-1)^2(N-2)/2}
\end{equation}
Where $\{b_i^c\}_i$ is the normalized betweenness centrality of each node. 
 \cite{freeman1978centrality}.
\begin{equation}
    C_{e^C} = \frac{\sum_{i=1}^N (e^* - e^c_i)}{(\sqrt{N-1} - 1)(N-1)/(\sqrt{N-1}+N-1)}
\end{equation}
Where $\{b_i^c\}_i$ is the normalized eigenvector centrality of each node.

% Figure environment removed
\paragraph{Toy model}

A toy model can help understanding the two apparently contradictory results provided by the Gini coefficient and the degree-centralization. Imagine $N_h$ hubs connected between them and $N_l$ leaves connected to each of them; hence, total number of nodes is $N=N_h(N_l+1)$, the degree of each hub reads $k_h=(N_h-1)+N_l$, the degree of each leave reads $k_l=1$ and

\begin{eqnarray}
G_k&=&\frac{\sum_{i=1}^N\sum_{j=1}^N|k_i-k_j|}{2N\sum_{i=1}^N{k_i}}=\frac{2[(N_h-1)+N_l-1]N_h^2N_l}{2N\left[N_h(N_h-1)+2N_hN_l\right]}\nonumber\\
&=&\frac{[(N_h-1)+N_l-1]N_h^2N_l}{N_h(N_l+1)\left[N_h(N_h-1)+2N_hN_l\right]}\nonumber\\
&=&\frac{(N_h+N_l-2)N_l}{(N_l+1)\left[(N_h-1)+2N_l\right]}\simeq\frac{N_h+N_l}{N_h+2N_l};\nonumber\\
\end{eqnarray}
now, $G_k\simeq2/3$ if $N_l=N_h$, i.e. if each hub is linked to a number of `leaves' that matches the total number of hubs and $G_k\rightarrow1/2$ as $N_l\rightarrow+\infty$, i.e. if the number of leaves per hub becomes `very large'. In this setting, we have that

\begin{equation}
C_k=\frac{\sum_{i=1}^N(k^*-k_i)}{(N-1)(N-2)}=\frac{((N_h-1)+N_l-1)*N_hN_l}{(N_h(N_l+1)-1)(N_h(N_l+1)-2)}\simeq\frac{N_h+N_l}{N_hN_l}
\end{equation}
which amounts at $C_k\simeq 0.02$ if we set $N_h=N_l=100$. Hence, opportunely tuning the parameters of our model we can recover core-periphery structures for which a large Gini coefficient co-exists with a small degree-centralization.

\thispagestyle{pagelast}

%\theendnotes

\end{document}
