\documentclass[11pt,leqno]{article}
\usepackage[T1]{fontenc}
\usepackage{lipsum}
\usepackage{eso-pic}
%\AddToShipoutPictureBG{%
%\begin{tikzpicture}[remember picture, overlay]
%\coordinate (start-numbers) at ([yshift=-2cm]current page.north west);
%    \foreach \i in {1,...,65}
%        \node[font=\small, text width=12em, align=right]
%            at ([yshift=-4.3*\i mm]current page.north west) {\i};
%\end{tikzpicture}%
%}
%\lipsum
\usepackage[utf8]{inputenc}
\usepackage[english]{babel}
\usepackage{tikz}
	
%\usepackage{lineno}
%\linenumbers
%numbering the line
%\usepackage{cite}
\usepackage{graphicx}
\usetikzlibrary{calc}
\usetikzlibrary{intersections}
\usepackage{verbatim}
\usepackage{etoolbox}
\usepackage{tkz-euclide}
\usepackage{geometry}
\geometry{margin=1.3in}
\usepackage{mathtools}
\usepackage{color}
\usepackage[title]{appendix}
\usepackage[all]{xy}
\usepackage{tikz-cd}
\usepackage{blindtext}
\usepackage[T1]{fontenc}
\usepackage{amsthm}
\usepackage{amsfonts}
\usepackage{txfonts}
\usepackage{palatino,amssymb,epsfig}
\usepackage{latexsym,epsf,epic,amscd}
%\usetkzobj{all}
\usepackage{mathrsfs}
\usepackage{graphicx}
\usepackage{caption}
\usepackage{subcaption}
\usepackage{appendix}
\usepackage{amsmath}
\usepackage{bookmark}
\hypersetup{
colorlinks=true,
linkcolor=black,
citecolor=black,
anchorcolor=black,
urlcolor=black}
\allowdisplaybreaks[4]
\numberwithin{equation}{section}
\DeclareMathOperator{\Hessian}{Hess}
\DeclareMathOperator{\spn}{span}
\DeclareMathOperator{\interior}{int}
\DeclareMathOperator{\area}{Area}
\DeclareMathOperator{\arccoth}{arccoth}
\DeclareMathOperator{\arctanh}{arctanh}
\DeclareMathOperator{\arccot}{arccot}
\geometry{a4paper}
    \newcommand{\Addresses}{{
  \bigskip
  \footnotesize
  \noindent Te Ba, \href{batexu@hnu.edu.cn}{batexu@hnu.edu.cn}
  \newline\textit{School of Mathematics, Hunan University, Changsha 410082, P.R. China}\par\nopagebreak
  \medskip
  \noindent Guangming Hu, \href{18810692738@163.com}{18810692738@163.com}
  \newline\textit{ College of Science, Nanjing University of Posts and Telecommunications,
  Nanjing, 210003, P.R. China.}\par\nopagebreak
  \medskip
  \noindent Yu Sun, \href{yusun15185105160@163.com}{yusun15185105160@163.com}
  \newline\textit{School of Mathematics and Physics, Nanjing Institute of Technology, 211100, P.R. China.}\par\nopagebreak
}}
\title{Circle packings and total geodesic curvatures in hyperbolic background geometry}
\author{Te Ba,  Guangming Hu, Yu Sun}

\date{}
\newtheorem{theorem}{Theorem}[section]
\newtheorem{lemma}[theorem]{Lemma}
\newtheorem{proposition}[theorem]{Proposition}
\newtheorem{corollary}[theorem]{Corollary}
\theoremstyle{definition}
\newtheorem{definition}[theorem]{Definition}
\newtheorem{remark}[theorem]{Remark}
\begin{document}
\maketitle

\begin{abstract}
In this paper, we generalize the Andreev-Thurston circle packing theorem in the hyperbolic background geometry. Horocycles and hypercycles are also considered in the packing. We give an existence and rigidity result of the circle packing with conical singularities with respect to the total geodesic curvature on each vertex. As a consequence, we establish an equivalent condition for the convergence of the combinatorial geodesic curvature flow to the desired circle packing.


 \medskip
\noindent\textbf{Mathematics Subject Classification (2020)}: 52C25, 52C26, 53A70.
\end{abstract}

\section{Introduction}
\subsection{Background}
Let $(S,T)$ be a connected closed surface $S$ with triangulation $T$. Let $V$, $E$, $F$ be the sets of vertices, edges and triangles of $T$, respectively.  A circle packing metric on $(S,T)$ is a map $r:V\to\mathbb{R}_{+}$ such that the associated  polyhedral metric on $(S,T)$ is given by
 \[l(uv)=r(u)+r(v),\]
 where $u$, $v$ are the endpoints of edge $uv$. A circle packing metric is called Euclidean or hyperbolic if we calculate the geometry of triangles by Euclidean or hyperbolic trigonometric identities. A circle packing metric may have conical singularities at the center of the circles. The classical discrete Gaussian curvature is introduced to describe the singularity at the center of each circle, which is defined as the angle deficit at the vertex. The notion of circle patterns was proposed in the work of Thurston \cite{thurston} as a significant tool to study the hyperbolic structure on $3$-manifolds. Over the past decades, circle patterns have been bridging discrete conformal geometry \cite{Glick,BPS,GT,Gu1,Gu2}, combinatorics \cite{liu-zhou}, minimal surfaces \cite{bo-ho-sp,wyl} and others. Please refer to \cite{ste,bower} for more background. Below, we present Thurston's remarkable theorem regard to the existence and uniqueness of the hyperbolic  circle packing metric.

\begin{theorem}[Thurston-Andreev]\label{th-an}
Let $(S,T)$ be a closed triangulated surface. Let $V$, $F$ be the sets of vertices and faces of $T$. Let $F_I$ be the set of faces having at least one vertex in $I$ for the subset $I\subset V$. Then there exists a hyperbolic circle packing metric on $(S,T)$ with discrete Gaussian curvatures $x_i$ on $i\in V$ if and only if $(x_1,\cdots,x_{\vert V \vert})\in\Omega$, where
\[\Omega=\left\{(x_1,\cdots,x_{\vert V \vert})\in\mathbb{R}^{\vert V \vert}\ \vert\  x_i<2\pi,\sum\nolimits_{i\in I} x_i>2\pi\vert I\vert-\pi\vert F_I\vert\ \text{for each}\ I\subset V\right\}.\]
Moreover, the hyperbolic circle packing metric is unique if it exists.
\end{theorem}
There are many results related to Theorem \ref{th-an} and its proof. See, for example,  the works of Andreev \cite{andreev1}, Colin de Verdiere  \cite{colin}, Marden-Rodin \cite{ma-ro}, Bowers-Stephenson, \cite{bo-ste} Chow-Luo \cite{chow-Luo}, Bobenko-Springborn \cite{bo}, Xu \cite{xu}, Connelly-Gortler \cite{co-go} and Ge-Hua-Zhou \cite{ge1,ge2}.
\subsection{Main results}\label{mainresults}

In this paper, it is of interest to establish a hyperbolic circle packing metric on a non-compact surface. For simplicity of notations, we use one index to denote a vertex ($i\in V$), two indices to denote an edge ($ij$ is the arc on $S$ joining $i$, $j$) and three indices
to denote a face ($ijk$ is the region on $S$ bounded by $ij$, $jk$, $ik$). Let $\widetilde{S}$ be a non-compact surface obtained by removing a  set of finite points $V_P=\{1,\cdots,n\}\subset V$ and a set of finite disjoint open discs $\mathcal{B}=\{B_{1},\cdots,B_{m}\}$ from $S$ satisfying:
\begin{itemize}
\item[($I$)] For each $B_j\in\mathcal{B}$, there exists a unique vertex $i\in V\setminus V_P$ such that $i\in B_j$.
\item[($II$)] $i\cap\overline{B}_{j}=\emptyset$ for each $i\in V_P$, $B_j\in \mathcal{B}$.
\item[($III$)] $ij\cap\partial{B}_{j}$ contains a single point for each $ij\in E$ and $B_j\in\mathcal{B}$.

\end{itemize}

Owing to ($I$),  set $V_{\mathcal{B}}=\{j\in V\vert j\in B_j,B_j\in \mathcal{B}\}$. For each $j\in V_{\mathcal{B}}$,  $ij\cap\partial B_j$ contains a single point on $\partial B_i$, denoted by $\partial_{j}i$. Then we define the vertex set $V_{\partial}^{j}=\{\partial_{j}i\vert ij\in E\}$ for $j\in V_{\mathcal{B}}$ and $V_{\partial}=\cup_{j\in V_{\mathcal{B}}}V_{\partial}^{j}$. Set $$V_0=V\setminus (V_{\mathcal{B}}\cup V_P) \text{ and  }\widetilde{V}=V_{\partial}\cup V_0\cup V_P.$$

Suppose $i\in V_{\mathcal{B}}$ and $ijk\in F.$ Let $\partial_{i}jk$  be the arc on $\partial B_i$ from $\partial_{i}j$ to $\partial_{i}k$. Denote $E_{\partial}=\{\partial_{i}jk\vert i\in V_{\mathcal{B}},ijk\in F\}$ as the edge on $\partial \widetilde{S}$. Let $E_{0}$ be the edge set consisting of the following cases: If $ij\in E$ joins $i, j\in V_{0}$, the edge $ij$ holds; if $ij\in E$ joins $i\in V_{0}$ and  $j\in V_{\mathcal{B}}$,  we define a new edge joining $i$ and $\partial_{j}i$; if $ij\in E$ joins $i$, $j\in V_{\mathcal{B}}$,  we define a new edge joining $\partial_{i}j$ and $\partial_{j}i$.  Denote $E_{\infty}$ as the edge set consisting of the following cases: If $ij\in E$ joins $i, j\in V_{P}$, the new edge is obtained by removing two endpoints of $ij$ ;  if $ij\in E$ joins $i\in V_{0}$ and  $j\in V_{P}$,  we define a new edge by removing $j$; if $ij\in E$ joins $i\in V_{P}$ and $j\in V_{\mathcal{B}}$,  we define a new edge from $i$ to $\partial_{j}i$ which removes $i$. Then define
\[\widetilde{E}=E_{\partial}\cup E_{\infty}\cup E_0.\]
We define the face set $\widetilde{F}$ on $\widetilde{S}$ as follows:
\begin{itemize}
\item[($1$)] If $i,j,k\in V_0$, then the face in $\widetilde{F}$ is the original face $ijk$ in $F$.
\item[($2$)] If $i,j\in V_0$ and $k\in V_{\mathcal{B}}$, then the face in $\widetilde{F}$ is a quadrilateral obtained by removing $B_k$ from the face $ijk$ in $F$.
\item[($3$)] If $i\in V_0$ and $j,k\in V_{\mathcal{B}}$, then the face in $\widetilde{F}$ is a pentagon obtained by removing $B_j$, $B_k$ from the face $ijk$ in $F$.
\item[($4$)] If $i,j,k\in V_{\mathcal{B}}$, then the face in $\widetilde{F}$ is a hexagon obtained by removing $B_i$, $B_j$ and $B_k$  from the face $ijk$ in $F$.
\item[($5$)] If some vertices of $i,j,k$ belong to $V_{P}$ instead of $V_{0}$, then the face in  $\widetilde{F}$ is obtained by removing the corresponding vertices from the above faces.
\end{itemize}

 Next, $\widetilde{T}=(T,V_P,\mathcal{B})$ is called a \textbf{pseudo ideal triangulation} of $\widetilde{S}$ and $\widetilde{V}$, $\widetilde{E}$, $\widetilde{F}$ are the vertex, edge and face set of $\widetilde{T}$.  Denote $(\widetilde{S},\widetilde{T})$ as  $\widetilde{S}$ by removing $V_p$, $\mathcal{B}$ from a closed surface $S$ satisfying conditions ($I$)-($III$), along with a pseudo ideal triangulation $\widetilde{T}$.
If $V_P$, $\mathcal{B}$ are empty set, the pseudo ideal triangulation $\widetilde{T}$ is the same as the triangulation $T$ of $S$.


% Figure environment removed
In this paper, We introduce horocycles, hypercycles into the hyperbolic circle packing. See Figure \ref{fig1} for the geometric realization of circles, horocycles and hypercycles in the Poincar\'{e} disk model. The relationship between  radii and geodesic curvatures of circles, horocycles and hypercycles is shown in the Table \ref{relation}.

\begin{table}[htbp]
\centering
\footnotesize	
\begin{tabular}
{l|l|l|l}
 &  \small circle&  \small horocycle&  \small hypercycle  \\ \hline
 \small radius&  $0<r<+\infty$&  $r=+\infty$&  $0<r<+\infty$  \\ %\hline
 \small geodesic curvature&  $k=\coth r$&  $k=1$&  $k=\tanh r$  \\
 \small arc length&  $l=\theta\sinh r$& ----- &  $l=\theta\cosh r$  \\ \hline
\end{tabular}
\caption{\small The relationship of radii, geodesic curvatures and arc lengths}



\label{relation}
\end{table}
A \textbf{generalized hyperbolic circle packing metric} on $(\widetilde{S},\widetilde{T})$ is a geodesic curvature map $k:\widetilde{V}\to\mathbb{R}_{+}$ satisfying
\begin{itemize}
\item[($a$)] $k(\partial_{i}j)<1$ if $\partial_{i}j\in V_{\partial}^{i}$.
\item[($b$)] $k(\partial_{i}j)=k(\partial_{i}k)$ if $\partial_{i}j,\partial_{i}k\in V_{\partial}^{i}$.
\item[($c$)] $k(i)>1$ if $i\in V_0$.
\item[($d$)] $k(i)=1$ if $i\in V_P$.
\end{itemize}
Owing to ($b$), we can regard $k$ as a function defined on $V$. The induced polyhedral metric on $E_{\infty}\cup E_{0}$ is defined by $d:E_{\infty}\cup E_{0}\to\mathbb{R}_{+}$, where
\[d(ij)=\left\{
\begin{aligned}
&+\infty,&\ ij\in E_{\infty},\\
&\arccoth k(i)+\arccoth k(j),&ij\in E_0, i, j\in V_0,\\
&\arctanh k(i)+\arctanh k(j),&ij\in E_0, i, j\in V_{\partial},\\
&\arccoth k(i)+\arctanh k(j),&ij\in E_0, i\in V_0, j\in V_{\partial}.
\end{aligned}
\right.\]

Next, we provide a brief introduction to the geometric meaning of the generalized hyperbolic circle packing metric and its induced polyhedral metric.
\begin{itemize}

\item[($i$)] If $u,v\in V_0$, the geometric meaning of $d(uv)$ is the distance between the centers of two circles with curvature $k(u)$, $k(v)$.
\item[($ii$)] If $u,v\in V_{\partial}$, the geometric meaning of $d(uv)$ is the distance between of axis of two hypercycles with curvature $k(u)$, $k(v)$.
\item[($iii$)] If $u\in V_0$ and $v\in V_{\partial}$, the geometric meaning of $d(uv)$ is the distance between the center of the circle with curvature $k(u)$ and the axis of the hypercycle with curvature $k(v)$.
\item[($iv$)] If $u\in V_P$,  the geometric meaning of $d(uv)$ is the distance between the center of the circle with curvature $k(u)=1$ (a horocycle) to the center or axis of a circle, or a horocycle, or a hypercycle with  curvature $k(v)$, which is $+\infty$.
\end{itemize}


% Figure environment removed

If $\partial_{i}jk\in E_{\partial}$,  $\partial_{i}jk$ is an edge of one of the faces described in ($2$)-($4$), as marked in Figure \ref{fig2}. Notice that each face depicted in Figure \ref{fig2} has three edges belonging to $E_{\infty}\cup E_{0}$, meaning that their side lengths are determined by $d$. In order to determine the geometry of faces in $\widetilde{F}$, we define the angles at the endpoints of $\partial_{i}jk$ are all right angles, as shown in Figure \ref{fig2}. We assert that the geometry of each types of faces in $\widetilde{F}$ can be uniquely determined by $d$. Please refer to Lemma \ref{exist}  for proofs. Then the induced polyhedral metric on $d$, together with the right angles defined above, forms a polyhedral metric on $(\widetilde{S},\widetilde{T})$. For the sake of simplicity, we will refer to this polyhedral metric as the generalized hyperbolic circle packing metric on $(S,T)$. See Figure $\ref{fig3}$ for an example of a generalized hyperbolic circle packing metric.

% Figure environment removed

Note that each $ijk\in\widetilde{F}$ can be embedded into a configuration of three mutually tangent circles (with possibly horocycles or hypercycles), as shown in the Figure \ref{fig4}. Then for any generalized hyperbolic circle packing metric $k:V\to\mathbb{R}_{+}$ on $(\widetilde{S},\widetilde{T})$, there exists a hyperbolic circle packing (with possibly horocycles or hypercycles) on $\widetilde{S}$ induced by $k$. The total geodesic curvature of $k$ at $v\in V$ is defined as the total geodesic curvature of the circle, or horocycle, or hypercycle at $v\in V$ in the hyperbolic circle packing induced by $k$. It can be calculated by
\[L(v)=l(v)k(v),\]
where $l(v)$ is the length of the circle. The total geodesic curvature was first introduced in the work of Nie \cite{nie} as an important tool to study the existence and rigidity of the circle patterns in spherical background geometry. Motivated by \cite{nie}, the main goal of the paper is to provide the following existence and rigidity result for generalized hyperbolic circle packing metrics where the discrete Gaussian  curvature at each vertex is replaced by the total geodesic curvature of each vertex.

\begin{comment}
Note that each $ijk\in\widetilde{F}$ can be embedded into a configuration of three mutually tangent circles (with possibly horocycles or hypercycles), as shown in the Figure \ref{fig4}. Then there exists a hyperbolic circle packing (with possibly horocycles or hypercycles) on $\widetilde{S}$ induced by the generalized hyperbolic circle packing metric. The total geodesic curvature of the generalized circle packing metric at each vertex is defined as the total geodesic curvature of the circle, or horocycle, or hypercycle at each vertex in the hyperbolic circle packing induced by the generalized hyperbolic circle packing metric. The total geodesic curvature was first introduced in the work of Nie \cite{nie} as an important tool to study the existence and rigidity of the circle patterns in spherical background geometry. Motivated by \cite{nie}, the main goal of the paper is to provide the following existence and rigidity result for generalized hyperbolic circle packing metrics where the discrete Gaussian  curvature at each vertex is replaced by the total geodesic curvature of each vertex.
\end{comment}

% Figure environment removed


\begin{theorem}\label{mr}
Let $(S,T)$ be a connected closed surface with vertex, face set $V$, $F$. Let $F_I$ be the set of faces having at least one vertex in $I$ for subset $I\subset V$.
Then there exists $\widetilde{S}\subset S$ and the pseudo ideal triangulation $\widetilde{T}$ on $\widetilde{S}$ such that the generalized hyperbolic circle packing metric on $(\widetilde{S},\widetilde{T})$ having the total geodesic curvature $L_1,\cdots,L_{\vert V\vert}$ on each vertex if and only if $(L_1,\cdots,L_{\vert V\vert})\in\Omega$, where \[\Omega=\left\{(L_1,\cdots,L_{\vert V\vert})\in\mathbb{R}^{\vert V\vert}_{+}\left\vert\right.\sum\nolimits_{i=1}^{\vert V\vert} L_i<\pi\vert F_I\vert\ \ \text{for each}\ I\subset
 V\right\}.\]
Moreover, the generalized hyperbolic circle packing metric is unique if it exists.
\end{theorem}
\begin{remark}
The Euclidean version of Theorem \ref{mr} is reduced to Thurston-Andreev's Theorem in Euclidean background geometry. Because the total geodesic curvature at $v\in V$ can be calculated by
\[L(v)=l(v)k(v)=\Theta(v)r(v)\frac{1}{r(v)}=\Theta(v),\]
where $\Theta(v)$ is the the angle deficit at $v\in V$.
 %is equal to the angle deficit at the vertex in Euclidean background geometry.
\end{remark}
A more delicate problem is to search the generalized circle packing on the surface with prescribed total geodesic curvature on each disk. For this purpose, we follow the work of Chow-Luo \cite{chow-Luo} on combinatorial Ricci flows. There are many results on combinatorial Ricci flows, for example, see \cite{Feng, Gehua, ge2, GeJiang, Gu1, Gu2}.

Let $k_i$ be the geodesic curvature of the disk centered at $i$. We consider the flow
\begin{equation}\label{flow}
\frac{dk_i}{dt}=-k_i(L_i-\hat{L}_{i}),
\end{equation}
for $i=1,\cdots,\vert V\vert$ with an initial radius vector $k(0)\in\mathbb{R}^{\vert V\vert}_{+}$.
\begin{theorem}\label{mr2}
The solution $k(t)$ of the flow (\ref{flow}) exists for all time. The following two statements are equivalent:
\begin{itemize}
\item[($a$)] $k(t)$ converges as $t\to\infty$.
\item[($b$)] The prescribed total geodesic curvature $\{\hat{L}_i\}_{i\in V}$ satisfies \[\hat{L}_i>0,\quad\sum_{i\in I}\hat{L}_i<\pi\vert F_I\vert\]
for each subset $I\subset V$.

\end{itemize}
If one of the above statements holds, then the flow (\ref{flow}) converges exponentially fast to a unique generalized circle packing metric with the total geodesic curvature $\hat{L}_{i}$ at $i\in V$.
\end{theorem}
The paper is organized as follows: In Section 2, we study the structure of three mutually tangent circles (with possibly horocycle and hypercycle). In Section 3, applying the variational principle, we derive Theorem \ref{mr}.  In Section 4, we introduce some properties of flows (\ref{flow}) and prove Theorem \ref{mr2}.


\section{Three-circle configuration}
\subsection{Admissible space}
This section is based on some results from hyperbolic polygons. Hyperbolic cosine law indicates that a hyperbolic triangle can be determined by three side lengths of the triangle. The following three lemmas show that some special quadrilaterals, pentagons and hexagons can be determined by  three side lengths as well. Please refer to \cite[Chapter 2]{buser} for some hyperbolic trigonometric identities.
\begin{lemma}\label{4}
  For any $r_1,r_2,r_3>0$, there exists a unique hyperbolic quadrilateral such that the following statements hold:
  \begin{itemize}
      \item[($a$)] There exist two adjacent right angles.
      \item[($b$)] Except for the one with right angles at both ends, the other three sides have lengths $l_1$, $l_2$, $l_3$, where $l_i=r_j+r_k$ for $\{i,j,k\}=\{1,2,3\}$.
  \end{itemize}
\end{lemma}
\begin{proof}
Without loss of generality, we assume that the side facing the side with right angles at both ends has length $l_2$. Suppose $l_1\geq l_3$. It suffices to prove that there exists $x\in(0,l_1)$ such that there exists a hyperbolic quadrilateral with three right angles with side length $x,y,l_3$ and a hyperbolic right-angled triangle with side length $l_1-x,y,l_2$, as shown in the Figure \ref{exist4}. This is equivalent to show that there exists $x\in(0,l_1)$ such that
\[\sinh l_3=\sinh x\cosh y\ \text{and}\ \cosh l_2=\sinh(l_1-x)\cosh y,\]
which is equivalent to
\begin{equation}\label{2.1}\cosh y=\frac{\sinh l_3}{\sinh x}=\frac{\cosh l_2}{\cosh(l_1-x)}>1.\end{equation}
Set
\[f(x)=\frac{\sinh x}{\cosh(l_1-x)}.\]
Note that $f(x)$ is strictly increasing for $x\in(0,l_1)$. It is easy to see $f(x)\to 0$ when $x\to 0$ and $f(x)\to\sinh l_1$ when $x\to l_1$. We need to demonstrate there exists $x_0\in(0,l_1)$ such that
\begin{equation}\label{2.2}f(x_0)=\frac{\sinh l_3}{\cosh l_2}\end{equation}
and
\begin{equation}\label{2.3}l_3>x_0,\quad l_2>l_1-x_0.\end{equation}
Notice that (\ref{2.2}) can be proved by showing
\[\frac{\sinh l_3}{\cosh l_2}<\sinh l_1.\]
This is the direct result of $l_1\geq l_3$. Assume that $l_3\leq x_0$.  Then $l_2\leq l_1-x_0$, which follows from (\ref{2.1}). Then we obtain $l_2+l_3\leq l_1$, which contradict to $r_1>0$. Suppose $l_1<l_3$. We can prove the lemma by finding $x\in(0,l_3)$ that satisfies similar properties as above. We omit the proof here.

\end{proof}

\begin{lemma}\label{5}
  For any $r_1,r_2,r_3>0$, there exists a unique hyperbolic pentagon with four right angles such that the following statements hold:
  \begin{itemize}
      \item[($a$)] The two sides adjacent to the non-right angles have lengths $r_2+r_3$, $r_1+r_3$.
      \item[($b$)] The middle of the three sides with right angles at both ends has length $r_1+r_2$.
  \end{itemize}
\end{lemma}
\begin{proof}
It suffices to prove that there exists $x\in(0,l_3)$ such that there exist two hyperbolic quadrilaterals with three right angles with side length $x,y,l_1$ and $l_3-x,y,l_2$, as shown in the Figure \ref{exist5}. It is equivalent to show that there exists $x\in(0,l_3)$ such that
\[\sinh l_1=\sinh x\cosh y\ \text{and}\ \sinh l_2=\sinh(l_3-x)\cosh y,\]
It is equivalent to
\begin{equation}\label{2.4}\cosh y=\frac{\sinh l_1}{\sinh x}=\frac{\sinh l_2}{\sinh(l_3-x)}>1.\end{equation}
Set
\[g(x)=\frac{\sinh x}{\sinh(l_3-x)}.\]
 It is easy to see $g(x)$ is strictly increasing for $x\in(0,l_3)$. Furthermore, we know that $g(x)\to 0$ as $x\to 0$ and $g(x)\to+\infty$ as $x\to l_3$. Then for any $l_1,l_2>0$, there exists $x_0\in(0,l_3)$ satisfying
 \[\frac{\sinh x_0}{\sinh(l_3-x_0)}=\frac{\sinh l_1}{\sinh l_2}.\]
Observe that $x_0<l_1$, $l_3-x_0<l_2$. Assume that it is not true, then (\ref{2.4}) indicates that $x_0\geq l_1$, $l_3-x_0\geq l_2$. It follows that $l_3\geq l_1+l_2$, which leads to $r_3\leq 0$. Then we find $x_0\in(0,l_3)$ satisfying (\ref{2.4}), which completes the proof.
\end{proof}

% Figure environment removed


The following is the corollary of the classical result of the existence of hyperbolic right-angled hexagon.
\begin{lemma}\label{6}
 For any $r_1,r_2,r_3>0$, there exists a unique right angled hyperbolic hexagon whose three non-pairwise adjacent edge lengths are $r_1+r_2$, $r_2+r_3$, $r_3+r_1$.
\end{lemma}

If some $r_i=+\infty$ in the above lemmas, the polygon has some vertices at infinity. The existence and uniqueness still hold for the polygon satisfying the condition of the above lemmas.


\begin{lemma}\label{exist}
    For any $k_1,k_2,k_3>0$, there exist unique (up to isometry) mutually externally tangent hyperbolic circles (with possibly horocycles and hypercycles) and having $k_1$, $k_2$, $k_3$ as their geodesic curvatures.
\end{lemma}
\begin{proof}
Let us turn the problem into the proof of  the existence of polygons that satisfy the following conditions.
\begin{itemize}
\item [($a$)] Suppose $k_i>1$ for $i=1,2,3$. Set $r_i=\arccoth{k_i}$. There exists a unique triangle with side lengths $d_1$, $d_2$, $d_3$, where $d_i=r_j+r_k$.
Then we draw a circle of radius $r_k$ centered at the vertex of the triangle if the side facing the vertex has length $r_i+r_j$. These three circles are exactly mutually tangent to each other on three sides of the triangle. The curvature of the three circles is exactly $k_1$, $k_2$, $k_3$.
    \item [($b$)] Suppose $k_1,k_2>1$ and $k_3<1$. Set $r_i=\arccoth{k_i}$ for $i=1,2$ and $r_3=\arctanh{k_3}$. It suffices to prove the existence of  quadrilaterals described in Lemma \ref{4}. Because we can construct two circles with radius $r_2$, $r_3$ centered at two non-right angle vertices of the quadrilateral and construct a hypercycle of radius $r_1$, taking the side of the quadrilateral where all vertices are right angles as the axis, as shown in the Figure \ref{3a}. These three circles are mutually tangent to each other with curvatures $k_1$, $k_2$, $k_3$.
\item [($c$)] Suppose $k_1,k_3<1$ and $k_2>1$. Set $r_i=\arctanh{k_i}$ for $i=1,3$ and $r_2=\arccoth{k_2}$. Similar analysis as $(a)$, $(b)$ shows that it is equivalent to Lemma \ref{5}. See Figure \ref{3b} for an example.
    \item [($d$)] Suppose $k_i<1$ for $i=1,2,3$. Set $r_i=\arctanh{k_i}$ for $i=1,2,3$. This situation corresponds to Lemma \ref{6}. See Figure \ref{3c} for an example.
    \item [($e$)] Suppose $k_i=1$ for some $i\in\{1,2,3\}$. Then $r_i$ approaches to $+\infty$. Then it suffices to prove the existence of some ideal polygons. This can be seen as the corollary to the above results.
\end{itemize}
The uniqueness is derived from the fact that the isometry of the hyperbolic plane maps circles (resp. horocycles, hypercycles) to circles (resp. horocycles, hypercycles). Thus we complete the proof.
\end{proof}

% Figure environment removed
\subsection{Convex functionals}
Let $C_1$, $C_2$, $C_3$ be three mutually tangent circles (with possibly horocycles or hypercycles).
Let $l_i$ be the length of the arc between two points of tangency of $C_i$. Lemma \ref{exist} indicates $l_i (k_1,k_2,k_3)$ is well-defined on $\mathbb{R}^3_+$. The main result of this subsection is Lemma \ref{close}. We begin by introducing the following two preliminary lemmas.

\begin{lemma}\label{length}
Let $H$ be a horocycle and $G$ be a geodesic intersecting $H$ at two points. Suppose that the intersection angle of $H$ and $G$ is $\alpha$, which is not equal to $\frac{\pi}{2}$. Then the length of the horocycle segment between two intersection points is $2\tan\alpha$.
\end{lemma}
\begin{proof}
Without loss of generality, we assume the horocycle is $y=1$ on the upper half-plane model. Then the horocycle segment can be written as
\[
\left\{
\begin{aligned}
&x(t)=\tan\alpha(2t-1),\\
&y(t)=1.
\end{aligned}
\right.
\]
for $0\leq t\leq 1$. Denote $l$ by the length of the horocycle segment. By the formula of hyperbolic length of curves, we obtain
\[l=\int_0^1\frac{\sqrt{x'(t)^2+y'(t)^2}}{y(t)}dt=2\tan\alpha.\]
\end{proof}



\begin{lemma}\label{bigon}
    Let $\gamma_1$, $\gamma_2$ be two circle arcs intersect each other at the right angle. Let $k_1$ $l_1$, $k_2$ $l_2$ be their curvatures and lengths. If $k_2>1$, then the differential form \[\eta=l_1dk_1+l_2dk_2\]
    is closed.
\end{lemma}
\begin{proof}
We divide the proof into three cases.
\begin{itemize}
\item[($a$)] Suppose $k_1>1$, as depicted in Figure \ref{Bi}. Let $\theta_1$, $\theta_2$ be the angle of $\gamma_1$, $\gamma_2$. Set $r_i=\arccoth k_i$. We
 can find a  right triangle with acute angles $\theta_1/2$, $\theta_2/2$ by connecting two centers of $\gamma_1$, $\gamma_2$ and one of the intersection point of $\gamma_1$, $\gamma_2$. By hyperbolic trigonometric identities, we obtain
\[\cot\frac{\theta_1}{2}=\coth r_2\sinh r_1=\frac{k_2}{\sqrt{k_1^2-1}}.\]
 It follows that
\[\theta_1=2\arccot\frac{k_2}{\sqrt{k_1^2-1}}.\]
Then
\[\frac{\partial l_1}{\partial k_2}=\frac{\partial\theta_1}{\partial k_2}\sinh r_1=\frac{2}{1-k_1^2-k_2^2}=\frac{\partial l_2}{\partial k_1}.\]
\item[($b$)]

Suppose $k_1<1$, as depicted in Figure \ref{Bi}. Let $\theta_1$, $\theta_2$ be the angle of $\gamma_1$, $\gamma_2$. Set $r_1=\arctanh k_1$ and $r_2=\arccoth k_2$. Then we can find a quadrilateral with three right angles satisfying the edge angle relation. By hyperbolic trigonometric identities, we obtain
\[\coth\frac{\theta_1}{2}=\coth r_2\cosh r_1=\frac{k_2}{\sqrt{1-k_1^2}},\quad\cot\frac{\theta_2}{2}=\tanh r_1\sinh r_2=\frac{k_1}{\sqrt{k_2^2-1}}.\]
 It follows that
\[\theta_1=2\arccoth\frac{k_2}{\sqrt{1-k_1^2}},\quad\theta_2=2\arccot\frac{k_1}{\sqrt{k_2^2-1}}.\]
Then we derive
\[\frac{\partial l_1}{\partial k_2}=\frac{\partial\theta_1}{\partial k_2}\cosh r_1=\frac{2}{1-k_1^2-k_2^2}.\]
A similar deduction gives that
\[\frac{\partial l_2}{\partial k_1}=\frac{\partial\theta_2}{\partial k_1}\sinh r_2=\frac{2}{1-k_1^2-k_2^2}.\]



\item[($c$)]

Suppose $k_1=1$, as depicted in Figure \ref{Bi}. Set $r_2=\arccoth k_2$ and $\theta_2$  the angle of $\gamma_2$. We can find an ideal right triangle with an acute angle $\theta_2/2$ by connecting two centers of $\gamma_1$, $\gamma_2$ and one of the intersection points of $\gamma_1$, $\gamma_2$. By  hyperbolic trigonometric identities, we obtain
\begin{equation}\label{tri1}\sin\frac{\theta_2}{2}\cosh r_2=1.\end{equation}
Let $C_1$, $C_2$ be the center of $\gamma_1$, $\gamma_2$ and let $P$, $Q$ be the intersection points of $\gamma_1$, $\gamma_2$.  We connect the $P$, $Q$ by geodesic and let $D$ be the intersection point of $PQ$ and $C_1C_2$. It is easy to see that $\triangle PDC_2$ is a right triangle. By hyperbolic trigonometric identities, we obtain
\begin{equation}\label{tri2}\cosh r_2=\cot\angle DPC_2\cot\frac{\theta_2}{2}.\end{equation}
Combing (\ref{tri1}), (\ref{tri2}), we obtain
\[\tan\angle DPC_2=\frac{\cot\frac{\theta_2}{2}}{\cosh r_2}=\frac{\sqrt{\cosh^2 r_2-1}}{\cosh r_2}=\tanh r_2=\frac{1}{k_2}.\]
Lemma \ref{length} shows that
\[l_1=\frac{2}{k_2},\] which yields that
\[\frac{\partial l_1}{\partial k_2}=-\frac{2}{k^2_2}.\]
By $(a)$, $(b)$, we obtain
\[\lim_{k_1\to 1}\frac{\partial l_2}{\partial k_1}=-\frac{2}{k^2_2}.\]
By mean value theorem, it is easy to prove that $l_2(k_1,k_2)$ is $C^1$ smooth with respect to $k_1$ when $k_1=1$ and
\[\frac{\partial l_2}{\partial k_1}=-\frac{2}{k^2_2}\]
when $k_1=1$.
\end{itemize}
Combining the analysis above, we obtain
\[\frac{\partial l_1}{\partial k_2}=\frac{\partial l_2}{\partial k_1}.\]
Thus we complete the proof.
\end{proof}
% Figure environment removed
The  proof of the following lemma is similar to the work of Colin de Verdiere \cite{colin}.
\begin{lemma}\label{close}
The differential $1$-form $\omega=l_1dk_1+l_2dk_2+l_3dk_3$ is closed.
\end{lemma}
\begin{proof}
For any $k_1$, $k_2$, $k_3>0$, Lemma \ref{exist} shows that there exist three mutually tangent circles with curvatures $k_1$, $k_2$, $k_3$. Furthermore, we can draw a unique circle $C$ passing through three tangency points as shown in Figure \ref{Threecircle}. It is easy to prove that $C$ is a hyperbolic circle. Let $k$ be the curvature of this circle.
Then we obtain a continuous map $f:\mathbb{R}^{3}_{+}\to\mathbb{R}_{+}$ from the curvature of three mutually tangent circles to the curvature of $C$.
Let $l$ be the perimeter of $C$. Then we define
    \[\eta=l_1dk_1+l_2dk_2+l_3dk_3+ldk\]
on $\Lambda=\{(x,f(x))\vert x\in\mathbb{R}^{3}_{+}\}.$ Note that $\Lambda$ is diffeomorphism to $\mathbb{R}^{3}_{+}$.
It suffices to show that $\eta$ is closed on $\Lambda$. Let $C_i$ be the circle with curvature $k_i$. Set $l'_{i}$ as the length of arc of $C$ which lies in the $C_i$, as depicted in Figure \ref{Threecircle}. Note that
    \[\eta=\sum_{i=1}^{3}(l_idk_i+l'_{i}dk).\]
    It suffices to prove that
    \[\eta_i=l_idk_i+l'_{i}dk\]
    is closed, which is the direct result of Lemma \ref{bigon}.
\end{proof}
Let $L_{i}=l_ik_i$
be the total geodesic curvature of the arc between two points of tangency of $C_i$. It is easy to see $L_{i}$ is the smooth regarding $k_1$, $k_2$, $k_3$. Set $S_i=\ln k_i$. Then
\begin{equation}\label{2.7}\omega=l_1dk_1+l_2dk_2+l_3dk_3=L_1dS_1+L_2dS_2+L_3dS_3,\end{equation}
where $L_i=l_ik_i$ is the total geodesic curvature of the arc between two points of tangency of $C_i$.  Let us define the line integral function
\[F(x)=\int^{x}_{x_0}L_1dS_1+L_2dS_2+L_3dS_3\]
on $\mathbb{R}^{3}$. Lemma \ref{close} indicates that $F(x)$ is well-defined on $\mathbb{R}^{3}$.

\begin{remark}
Let $k_i=\coth r_i>1$. Then there exists a triangle formed by connecting the centers of three hyperbolic circles with radii $r_i$, $r_j$, $r_k$ and tangent to each other. Let $\theta_i$ be the inner angle opposite to side with length $r_j+r_k$. Then
\begin{equation}\label{remark}l_idk_i=\frac{-l_idr_i}{\sinh^2 r_i}=\frac{-\theta_idr_i}{\sinh r_i}=-\theta_idu_i,\end{equation}
where $u_i=\ln\tanh\frac{r_i}{2}$. Set
$\omega=\sum_{i=1}^{3}\theta_idu_i$
 and define $G(x)=\int^x\omega$. Colin de Verdiere \cite{colin} first introduced $G(x)=\int^x\omega$ to prove the uniqueness part of Theorem \ref{th-an} by variational principle. Then (\ref{2.7}) and (\ref{remark}) indicate $F(x)$ can be viewed as an extension of $G(x)$ with different geometric meaning.
\end{remark}

% Figure environment removed
The following lemma is directly obtained by the Gauss-Bonnet theorem.
\begin{lemma}\label{area}
Let $C_1$, $C_2$, $C_3$ be three mutually tangent circles. Let $\Omega$ be the region enclosed by three arcs between tangency points. Then
\[\area(\Omega)=\pi-L_1-L_2-L_3,\]
where $L_i$ is the total curvature of the arc between two points of tangency of $C_i$.
\end{lemma}

\begin{lemma}\label{convex}
The integral function $F(x)$ is strictly convex.
\end{lemma}
\begin{proof}
It is easy to see
\[\frac{\partial F(x)}{\partial x_i}=L_i.\]
 The Hessian matrix of $F(x)$ is
\[\mathrm{M}=\left[\frac{\partial L_{i}}{\partial S_j}\right]_{3\times 3}.\]
Recall that $C_1$, $C_2$, $C_3$ be three mutually tangent circles (with possibly horocycles or hypercycles) in the hyperbolic plane.
A key observation is that the Euclidean radius of $C_1$ decreases as $k_1$ increases and $k_2$, $k_3$ unchanged if $C_1$, $C_2$, $C_3$ are required to remain tangent to each other all the time.
Then $C_1$ will shrink to the tangency point of $C_2$, $C_3$ as $k_1$ increases. A direct corollary of above observation is that
\[\frac{\partial L_{i}}{\partial S_j}<0,\quad\frac{\partial\area(\Omega)}{\partial S_i}<0.\]
Lemma \ref{area} shows that
\[\area(\Omega)=\pi-L_1-L_2-L_3.\]
As a result, we have
\[\frac{\partial L_{i}}{\partial S_i}=-\frac{\partial(\area(\Omega)+L_{j}+L_{k})}{\partial S_i}>0.\]
Then
\[\left\vert\frac{\partial L_{i}}{\partial S_i}\right\vert-\left\vert\frac{\partial L_{j}}{\partial S_i}\right\vert-\left\vert\frac{\partial L_{k}}{\partial S_i}\right\vert=
\frac{\partial(L_{1}+L_{2}+L_{3})}{\partial S_i}=-\frac{\partial\area(\Omega)}{\partial S_i}>0.\]
Therefore, the Hessian matrix of $F(x)$ is a symmetric strictly diagonally dominant matrix with positive diagonal entries, which yields that $\mathrm{M}$ is positive definite.
\end{proof}

\subsection{Limit behavior}
\begin{lemma}\label{limit}
Let $C_1$, $C_2$, $C_3$ be three mutually tangent circles with curvature $k_1$, $k_2$, $k_3$. Let $L_i$ be the total curvature of the arc between two points of tangency of $C_i$. Let $0\leq c_i<+\infty$ and $\{r,s,t\}=\{1,2,3\}$. Then the following statements hold:

\begin{subequations}
\begin{align}
    \label{a}\lim_{k_r\to 0}L_r&=0,\\
\label{b}\lim_{(k_r,k_s,k_t)\to(+\infty,a,b)}L_r&=\pi,\\
\label{c}\lim_{(k_r,k_s,k_t)\to(+\infty,+\infty,c)}L_r+L_s&=\pi,\\
\label{d}\lim_{(k_r,k_s,k_t)\to(+\infty,+\infty,+\infty)}L_r+L_s+L_t&=\pi.        \end{align}\end{subequations}
%\end{align}
\end{lemma}
\begin{proof}
We divide the proof of (\ref{a}) into three cases.
\begin{itemize}
\item[($a$)] Suppose $k_{s}\to 0$, $k_{t}\to 0$. Denote the region enclosed by $C_r$, $C_s$, $C_t$ as $\Lambda_{r,s,t}$. Then $\Lambda_{r,s,t}$ approaches to an ideal triangle $T$. By Lemma \ref{area}, we have \[\area(T)-\area(\Lambda_{r,s,t})=L_r+L_s+L_t\to 0.\]
 Thus $L_r,L_s,L_t\to 0$.
    \item[($b$)] Suppose $k_{s}\to 0$, $k_t\to c (c\neq 0)$. Note that $L_r$, $L_s$ are strictly less than the corresponding total geodesic curvature in the case ($a$), see Figure \ref{infty} for an explanation. Thus $L_r,L_s\to 0$.
    \item[($c$)] Suppose $k_s,k_t\nrightarrow 0$. Then $\Lambda_{r,s,t}$ is bounded, which yields that $l_r$ is bounded. Thus $L_r\to 0$.
\end{itemize}
Next, we prove (\ref{b})-(\ref{d}). Note that the area of $\Lambda_{r,s,t}$ approaches to zero as one of $k_i\to+\infty$, where $i=r,s,t$. By Lemma \ref{area}, it can be derived that
\[L_r+L_s+L_t\to\pi\]
as one of $k_i\to+\infty$.
\begin{itemize}
  \item[($1$)] Suppose $(k_r,k_s,k_t)\to(+\infty,a,b)$. Then $l_s,l_t\to 0$, which implies $L_s,L_t\to 0$. Thus $L_r\to\pi$.  \item[($2 $)] Suppose $(k_r,k_s,k_t)\to(+\infty,+\infty,c)$. Then $l_t\to 0$, which implies $L_t\to 0$. Thus $L_r+L_s\to\pi$.  \item[($3$)] Suppose $(k_r,k_s,k_t)\to(+\infty,+\infty,+\infty)$. Let $\tau_{r,s,t}$ be the triangle formed by connecting the centers of $C_r$, $C_s$, $C_t$. Let $\theta_i$  be the inner angle of $\tau_{r,s,t}$ at the center of $C_i$, where $i=r,s,t$. Then \[L_i=\theta_i\cosh r_i\to\theta_i.\]
The area formula of the hyperbolic triangle demonstrates
\[\area(\tau_{r,s,t})=\pi-\theta_r-\theta_s-\theta_t\to 0,\]
which implies that
\[L_r+L_s+L_t\to\pi\]
where $L_r,L_s,L_t\nrightarrow 0$.\end{itemize}
\end{proof}

% Figure environment removed

\section{Variational principle}
The main target of this section is to prove Theorem \ref{mr} by the  variational principle. Recall that $T$ is the triangulation of $S$ and $V$ is the vertex set of $T$. We define
\begin{equation}\label{W}
W(x)=\sum_{ijk\in F} F(x_i,x_j,x_k)\end{equation}
on $\mathbb{R}^{\vert V\vert}$, where $x_p$ represents the value of the generalized circle packing metric at $p\in V$. Lemma \ref{convex} shows that $F(x_i,x_j,x_k)$ is strictly convex, which yields that $W(x)$ is strictly convex. Note that
\[\frac{\partial W(x)}{\partial x_i}=\frac{\partial}{\partial x_i}\sum_{pqr\in F} F(x_p,x_q,x_r)=\sum_{ijk\in F}L_i^{jk},\]
where $L_i^{jk}$ is the total curvature of the arc of $C_i$ between tangency points of $C_i$, $C_j$ and tangency points of $C_i$, $C_k$. Using substitution $K_i=\ln k_i$, the space of generalized circle packing metric on  $\widetilde{S}$ that satisfies the conditions (I)-(III) given in Section \ref{mainresults} can be characterized by $\mathbb{R}^{\vert V\vert}$.

The following property of convex functions is important in the proof of Theorem \ref{mr}. Please refer to \cite{Dai,Luo} for a proof.

\begin{lemma}\label{injective}
If $W:\mathbb{R}^n\to\mathbb{R}$ is a $C^2$-smooth strictly convex function, then its gradient $\nabla W:\mathbb{R}^n\to\mathbb{R}^n$ is a smooth embedding.
\end{lemma}
Combining Lemma \ref{injective}, we deduce that

\[\label{image map}\begin{aligned}
\nabla W:\quad\mathbb{R}^{\vert V\vert}\quad\quad&\to\quad\quad\mathbb{R}^{\vert V\vert}\\
(K_1,\cdots,K_{\vert V\vert})&\to(L_1,\cdots,L_{\vert V\vert})
\end{aligned}\]
realizes a smooth embedding from the space of generalized circle packing metrics to the total geodesic curvature on each disk.
Next we will give the image space of the mapping $\nabla W$.
\begin{lemma}\label{ho}
Let $W$ be defined as (\ref{W}). Then $\nabla W:\mathbb{R}^{\vert V\vert}\to\Omega$ is a homeomorphism, where
\[\Omega=\left\{(L_1,\cdots,L_{\vert V\vert})\in\mathbb{R}^{\vert V\vert}_{+}\ \vert\ \sum_{i\in I}L_i<\pi\vert F_I\vert\ \text{for any subset}\ I\subset V\right\}.\]
\end{lemma}
\begin{proof}

It suffices to prove that $\nabla W(\mathbb{R}^{\vert V \vert})=\Omega$. To this end, we need to analyze the boundary of $\nabla W(\mathbb{R}^{\vert V\vert})$ in $\mathbb{R}^{\vert V\vert}_{+}$. Take a sequence $K^{(m)}\in\mathbb{R}^{\vert V\vert}$ so that
\[\lim_{m\to\infty} K^{(m)}=a\in[-\infty,+\infty]^{\vert V\vert},\]
where $a(i)=0$ or $+\infty$ for some $i\in V$. We need to prove that $\nabla W(K^{(m)})$ converges to the boundary to $A$.

Let $I$ (resp. $I'$) be a subset of $V$ such that $a(i)=+\infty$ (resp. $a(i)=-\infty$) for $i\in I$ (resp. $i\in I'$). Let $F_I$ be the set of faces having at least one vertex in $I$. Suppose $ijk\in F_I$. Let us divide it into three situations.
\begin{itemize}
    \item[($a$)] There exists exactly one vertex, and say $i\in I$. By (\ref{b}), we have $L_{i}^{jk}\to\pi$ as $K^{(m)}\to a$.
    \item[($b$)] There exist exactly two vertices, and say $i,j\in I$. By (\ref{c}), we have $L_{i}^{jk}+L_{j}^{ik}\to\pi$ as $K^{(m)}\to a$.
    \item[($c$)] All vertices $i$, $j$, $k\in I$. By (\ref{d}), we have $L_{i}^{jk}+L_{j}^{ik}+L_{k}^{ij}\to\pi$ as $K^{(m)}\to a$.
\end{itemize}
Let us compute the sum of all total curvatures of disks in $I$. Let $F_{I_s}$ be the set of faces having exactly $s$ vertex in $I$. We group the total curvatures of disks in $I$ according to the faces of $F_I$ in which they lie. Then
\[\sum_{i\in I} L_i=\sum_{ijk\in F_I}L_{i}^{jk}\to\pi(\vert F_{I_1}\vert+\vert F_{I_2}\vert+\vert F_{I_3}\vert)=\pi\vert F_I\vert.\]
Suppose $v\in I'$. By (\ref{a}), we have
\[L_i=\sum_{ijk\in F}L_i^{jk}\to 0.\]
Lemma \ref{area} indicates that
\[L_{i}^{jk}+L_{j}^{ik}+L_{k}^{ij}<\pi\]
for each $ijk\in F$, which yields that
\[\sum_{i\in I} L_i<\pi\vert F_I\vert.\]
In addition, it is obvious that $L_i>0$. According to the arbitrariness of the choice of $I$, we derive that $\nabla W(K^{(m)})\in A$ and $\nabla W(K^{(m)})$ converges to the boundary of $A$ as $m\to+\infty.$ Hence, it can be inferred that $\nabla W(\mathbb{R}^{\vert V \vert})=A$. Thus we complete the proof.
\end{proof}

\begin{proof}[\textbf{Proof of Theorem \ref{mr}}]
Set $L=(L_1,\cdots,L_{\vert V\vert})\in\Omega$. Lemma \ref{ho} indicates that there exists \[K=(K_1,\cdots,K_{\vert V\vert})\in\mathbb{R}^{\vert V\vert}\]
such that $\nabla W(K)=L$. Set $k_i=e^{K_i}$. Then we define $k_L:V\to\mathbb{R}_{+}$ with $ k_{L}(i)=k_i$ satisfying $V_p^L=\{i\in V\vert k_L(i)=1\}$, $V_{0}^{L}=\{i\in V\vert k_L(i)>1\}$, $V_{B}^{L}=\{i\in V\vert k_L(i)<1\}$ satisfying the condition (I)-(III). Then $k_L$ is the generalized circle packing metric on $\widetilde{S}$ with the total geodesic curvature $L_1,\cdots,L_{\vert V\vert}$ on each vertex. This proves the 'if' part. The 'only if' is the direct result of Lemma \ref{ho}. Thus the theorem is proved.

\end{proof}


\section{Combinatorial curvature flow}
By the change of variables $K_i=\ln{k_i}$, we rewrite the flow (\ref{flow}) as
\begin{equation}\label{Flow}
\frac{dK_i}{dt}=-(L_i-\hat{L}_i).
\end{equation}
\begin{lemma}\label{lem4.1}
The flow (\ref{Flow}) exists a unique solution all the time.
\end{lemma}
\begin{proof}
Note that $L_i$ is a $C^1$ smooth function of $K_i$. By Cauchy-Lipschitz Theorem, the flow (\ref{Flow}) exists a unique solution $K(t)$ for some $t\in(0,\varepsilon)$. Now we prove that $\varepsilon=+\infty$. Assume on the contrary that $\varepsilon$ is finite. Then there exists a sequence $t_{m}\to \varepsilon$ such that $K\left(t_{m}\right)\to+\infty$\ \text{or}\ $-\infty$. Lemma \ref{ho} indicates that
\[\left\vert L_{i}\right\vert<c_i\pi\]
for some $c_i>0$. Thus we have
\[\left\vert\frac{dK_{i}}{dt}\right\vert
=\left\vert L_{i}-\hat{L}_{i}\right\vert
\leq\left\vert L_{i}\right\vert+\left\vert\hat{L}_{i}\right\vert
\leq c_i\pi+\left\vert\hat{L}_{i}\right\vert\leq N_i\]
for some $N_i>0$. This yields that
\[\left\vert K(t_{m})\right\vert\leq\left\vert K(0)\right\vert+N_i t_{m}.\]
Then we obtain
\[\lim_{m\to\infty}\left\vert K(t_{m})\right\vert\leq\left\vert K(0)\right\vert+N_i\varepsilon,\]
which yields a contradiction.
\end{proof}
Let us define
\[\Phi(K)=W(K)-\sum_{i=1}^{|V|}\hat{L}_iK_i.\]
Note that the flow (\ref{Flow}) is the negative gradient flow of $\Phi(K)$. It is obvious that $\Phi(K)$ is smooth and strictly convex. Set $\hat{L}=(\hat{L}_1,\cdots,\hat{L}_{\vert V\vert})$. It can be derived that $\Phi(K)$ exists a unique critical point if and only if there exists $\hat{K}\in\mathbb{R}^{\vert V\vert}$ such that $\nabla W(\hat{K})=\hat{L}$. It is equivalent to $\hat{L}\in\Omega$, which follows from Lemma \ref{ho}. Then we obtain the following lemma.
\begin{lemma}\label{lem4.2}
$\Phi(K)$ exists a unique critical point if and only if $\hat{L}\in\Omega$.
\end{lemma}
The following property of convex functions plays an important role in proving  Theorem \ref{mr2}, a proof of which could be found in \cite[Lemma 4.6]{Gexu}.
\begin{lemma}\label{lemma}
Suppose $f(x)$ is a $C^{1}$ smooth convex function on $\mathbb{R}^{n}$ with $\nabla f(x_{0})=0$ for some $x_{0}\in\mathbb{R}^{n}$. Suppose $f(x)$ is $C^{2}$ smooth and strictly convex in a neighborhood of $x_{0}$. Then the following statements hold:
\begin{itemize}
\item[($a$)] $\nabla f(x)\neq0$ for any $x\notin\mathbb{R}^{n}\setminus\{x_{0}\}$.
\item[$(b)$] Then $\lim_{\Vert x\Vert\to+\infty}f(x)=+\infty$.
\end{itemize}
\end{lemma}
Before we prove Theorem \ref{mr2}, we give a brief introduction to Lyapunov's Stability theory. Please refer to \cite{po,wal,khalil2015nonlinear} for more information. Let us consider the real autonomous systems
\begin{equation}\label{system}
\frac{dx}{dt}=f(x),
\end{equation}
where $f$ is continuous on the open set $D\subset\mathbb{R}^{n}$, which contains the origin. The following result is the classic Lyapunov's Stability Theorem.
\begin{theorem}\cite[Theorem 3.3]{khalil2015nonlinear}\label{lya}
Let $f(x)$ be a continuous function defined on an open set $D\subset\mathbb{R}^{n}$, which contains the origin, and $f(0)=0$. Let $V(x)$ be a continuously differentiable function defined over $D$ such that
\begin{itemize}
\item[($a$)] $V(0)=0$ and $V(x)>0$ for all $x\in D\setminus\{0\}$,
\item[($b$)] $\dot{V}(x)\leq0$ for all $x\in D$.
\item[($c$)] $\dot{V}(x)<0$ for all $x\in D\setminus\{0\}$,
\end{itemize}
Then $x(t)=0$ is an asymptotically stable equilibrium point of (\ref{system}). Suppose $D=\mathbb{R}^{n}$ and ($a$),($b$) hold. If
$V(x)\to+\infty$ as $\Vert x\Vert\to+\infty,$
then $x(t)=0$ is globally asymptotically stable.
\end{theorem}
\begin{proof}[\textbf{Proof of Theorem \ref{mr2}}]
Since the flow (\ref{flow}) is equivalent to the flow (\ref{Flow}), we can  prove this theorem for the flow (\ref{Flow}). Due to Lemma \ref{lem4.1}, denote $K(t)$ as the solution of the flow (\ref{Flow}).

First we prove $(a)\Rightarrow(b)$. Suppose $K(t)\to\hat{K}$ as $t\to\infty$. By the mean value theorem, there exists $\xi_n\in(n,n+1)$ such that
\[
\Phi(K(n+1))-\Phi(K(n))=\frac{d}{dt}\Phi(K(\xi_n))\to 0
\]
as $n\to+\infty$. Note that
\[\frac{d}{dt}\Phi(K(t))=\sum\nolimits_{i=1}^{\vert V\vert}\frac{\partial\Phi(K)}{\partial K_{i}}\frac{dK_{i}}{dt}=-\sum\nolimits_{i=1}^{\vert V\vert}\left(\frac{\partial\Phi(K)}{\partial K_{i}}\right)^2.\]
Consequently, we have
\[\left\vert\nabla\Phi(K(\xi_n))\right\vert\to 0\]
as $n\to+\infty$. Note that $K(\xi_n)\to \hat{K}.$ It can be concluded that $\nabla\Phi(\hat{K})=0$, which implies that $\nabla W(\hat{K})=\hat{L}$. Then Lemma \ref{ho} demonstrates $\hat{L}\in\ A$.

Next, we prove $(b)\Rightarrow(a)$. Assume that $\hat{L}\in A$. Lemma \ref{ho} shows that there exists a unique $\hat{K}\in\mathbb{R}^{\vert V\vert}$ such that $\nabla W(\hat{K})=\hat{L}$. It follows that $\nabla \Phi(\hat{K})=0$. Let us define
\[V(K)=\Phi(K)-\Phi(\hat{K}).\]
Combining to lemma \ref{lem4.2} and lemma \ref{lemma}, it is easy to verify that
\begin{itemize}
\item[($i$)] $V(K)\geq0$ and $V(K)>0$ for $K\neq\hat{K}$.
\item[($ii$)] $\frac{d}{dt}V(K(t))\leq 0$.
\item[($iii$)] $\frac{d}{dt}V(K(t))=0$ if and only if $K=\hat{K}$.
\item[($iv$)] $V(K)\to+\infty$ as $\Vert K\Vert\to+\infty$.
\end{itemize}


Then $K(t)=\hat{K}$ is the asymptotically stable equilibrium point of (\ref{Flow}). The convergence of $K(t)$ to $\hat{K}$ follows from the Theorem \ref{lya}.

The next step is to prove that $K(t)$ converges exponentially fast to $\hat{K}$. Define
\[C\left(K\right)=\sum\nolimits_{i=1}^{n}(L_i-\hat{L}_{i})^2.\]
Note that $W$ is strictly convex. Then there exists $\lambda>0$ such that
\[\begin{aligned}
\frac{dC\left(K(t)\right)}{dt}&=-2\left(L_i-\hat{L}_{i},\cdots,L_i-\hat{L}_{i}\right)^{\mathrm{T}}\mathrm{M}\left(L_i-\hat{L}_{i},\cdots,L_i-\hat{L}_{i}\right),\\
&\leq-2\lambda\sum\nolimits_{i=1}^{\vert V\vert}\left(L_i-\hat{L}_{i}\right)^2\\
&=-2\lambda C(K),
\end{aligned}\]
where $\mathrm{M}$ is the Hessian matrix of $W(K)$.
It means
\[\sum\nolimits_{i=1}^{n}(L_i-\hat{L}_{i})^2=C\left(K(t)\right)\leq C(K(0))e^{-2\lambda_0 t},\]
therefore, we have
\[\vert L_i-\hat{L}_{i}\vert\leq\sqrt{C(K(0))}e^{-\lambda_0 t}.\]
Then we obtain
\[
\vert K_{i}(t)-\hat{K}_{i}\vert
\leq\int_{t}^{\infty}\left\vert L_i-\hat{L}_{i}\right\vert dt
\leq\int_{t}^{\infty}C(K(0))e^{-\lambda_{0}t}dt\leq \frac{C(K(0))}{\lambda_0}e^{-\lambda_{0}t},
\]
for any $t>0$.
This gives the exponential convergence of the flow (\ref{Flow}).
\end{proof}




\section{Open questions}
This paper leaves behind several topics that merit investigation.
\begin{itemize}
\item[$1.$] Is it possible to obtain more topological properties of $\widetilde{S}$ in Theorem \ref{mr} based on  the total curvature at each vertex? For instance, given a prescribed total geodesic curvature, can we determine whether $\widetilde{S}$ is closed? If it is not closed, how do we recognize its boundary type?
\item[$2.$] Explore more rigidity results of circle patterns with respect to the total geodesic curvature. There are two generalized structures for circle packings, allowing for the intersection or separation of circles. These two types of generalized structures are determined by discrete Gaussian curvatures, which are established in the works of Thurston \cite{thurston}, Ge-Hua-Zhou \cite{ge1,ge2}, Xu \cite{xu}. We want to know if Theorem \ref{mr} can be extended to the above two generalized structures.
\item[$3.$]  Does there exist a constant discrete Gaussian curvature metric in $\Omega$ in Theorem \ref{mr}? If it exists, in what sense is it unique?
\item[$4.$] Let $(S,T)$ be a closed triangulated surface and let $(r_1,\cdots,r_{\vert v\vert})\in\mathbb{R}^N_+$ be a hyperbolic circle packing metric. Let $L_i$, $\Theta_i$ be the total geodesic curvature and angle deficit at $i\in V$. Set $A_i=\Theta_ir_i$. The Gauss-Bonnet theorem indicates that
   \[L_i-A_i-\Theta_i=0.\]
   Theorem \ref{th-an} indicates hyperbolic circle packing metrics are determined by angle deficit at each vertex. Theorem \ref{mr} shows that hyperbolic circle packing metrics are determined by total geodesic curvature at each vertex as well. We are interested in studying whether $A_i$ can be used to characterize the rigidity of hyperbolic circle packing metrics.
\item[$5.$] Since  the circle packing metric induces a special polyhedral surface,  whether  we can generalize total geodesic curvatures to the polyhedral surfaces or not. The relationship between the  discrete conformal structures on polyhedral surfaces and 3-dimensional hyperbolic geometry was first discovered by Bobenko-Pinkall-Springborn \cite{BPS}, in the case of vertex scaling, which is further studied by
 Zhang-Guo-Zeng-Luo-Yau-Gu \cite{ZGZ}. We want to know if there exists a relationship between the generalized circle packing induced by total geodesic curvatures and  $3$-dimensional
hyperbolic geometry.





\end{itemize}
\section{Acknowledgments}
Te Ba is supported by NSF of China (No. 11631010). Guangming Hu is supported by NSF of China (No. 12101275). The authors would like to thank Xin Nie, Xu Xu and Ze Zhou for helpful discussions.

\begin{thebibliography}{99}
\bibitem{andreev1} E. M. Andreev, \emph{Convex polyhedra in Lobachevsky spaces.} (Russian) Mat. Sb. (N.S.) 81 (123),
 1970, 445--478.
\bibitem{bo-ho-sp}  A.I. Bobenko, T. Hoffmann, B.A. Springborn, \emph{Minimal surfaces from circle patterns:
geometry from combinatorics}, Ann. of Math. 164 (2006), 231--264.
\bibitem{BPS} A. I. Bobenko, U. Pinkall and B. Springborn, \emph{ Discrete conformal maps and ideal hyperbolic polyhedra,} Geom. Topol. 19 (2015),  4, 2155--2215.
\bibitem{bo} A. I. Bobenko, B. A. Springborn, \emph{Variational principles for circle patterns and Koebe's theorem,} Trans. Amer. Math. Soc. 356 (2004), 659--689.
\bibitem{bower} P. L. Bowers, \emph{Combinatorics encoding geometry: the legacy of Bill Thurston in the story of one theorem. In the tradition of Thurston—geometry and topology}, 173--239, Springer, Cham, 2020.
\bibitem{bo-ste} P. L. Bowers, K. Stephenson, \emph{A branched Andreev-Thurston theorem for circle packings for
the sphere}, Proc. London Math. Soc. (3) 73 (1996), 185--215.
\bibitem{buser} P. Buser, \emph{Geometry and spectra of compact Riemann surfaces}, Progress in Mathematics
106, Springer Science and Business media, Boston, 2010.
\bibitem{chow-Luo} B. Chow, F. Luo, \emph{Combinatorial Ricci flows on surfaces}, J. Differential Geom. 63, (2003), 97--129.
\bibitem{colin} Y. Colin de Verdiere, \emph{Un principe variationnel pour les empilements de cercles}, Invent.
Math. 104, 1991, 655--669.
\bibitem{co-go} R. Connelly, S. J. Gortler, \emph{Packing disks by flipping and flowing}, Discrete Comput. Geom. 66 (2021), no. 4, 1262--1285.
\bibitem{Dai} J. Dai, X. D. Gu and F. Luo, \emph{Variational principles for discrete surfaces}, Advanced Lectures in Mathematics 4, Higher Education Press, Beijing, 2008.
\bibitem{Feng}
K. Feng, H. Ge, and B. Hua, \emph{Combinatorial Ricci flows and the hyperbolization of a class of compact 3-manifolds,} Geom, Topol., 26(3), 2022, 1349--1384.

\bibitem{Gehua} H. Ge, \emph{Combinatorial calabi flows on surfaces,} Trans.  Amer.
Math. Soc., 370(2), 2018, 1377--1391.


\bibitem{ge1} H. Ge, B. Hua, Z. Zhou, \emph{Circle patterns on surfaces of finite topological type}, Amer. J. Math. 143 (2021), 1397--1430.
\bibitem{ge2} H. Ge, B. Hua, Z. Zhou, \emph{Combinatorial Ricci flows for ideal circle patterns}, Adv. Math. 383 (2021), Paper No. 107698.

\bibitem{GeJiang}
H. Ge, W. Jiang, and L. Shen, \emph{On the deformation of ball packings,} Adv.  Math., 398(2022),  No. 108192.

\bibitem{Gexu}H. Ge and X. Xu, \emph{On a combinatorial curvature for surfaces with inversive distance circle packing
metrics,} J. Funct. Anal. 275 (2018), No. 3, 523--558.

\bibitem{Glick} D. Glickenstein, \emph{Discrete conformal variations and scalar curvature on piecewise flat two and three dimensional manifolds}, J. Differential Geom. 87 (2011), 201--238.
\bibitem{GT} D. Glickenstein, J. Thomas, \emph{Duality structures and discrete conformal variations of piecewise constant curvature surfaces,} Adv. Math. 320 (2017), 250--278.

\bibitem{Gu1} X. Gu, F. Luo, J. Sun and T. Wu. \emph{A discrete uniformization
theorem for polyhedral surfaces,} J. Differential Geom., 109(2), 2018, 223--256.

\bibitem{Gu2}X. Gu, R. Guo, F. Luo, J. Sun and T. Wu. \emph{A discrete uniformization theorem for polyhedral surfaces II,} J. Differential Geom., 109(3), 2018, 431--466.
\bibitem{khalil2015nonlinear} H. K. Khalil, \emph{Nonlinear control}, Pearson Education, Prentice-Hall, NJ, 2015.
\bibitem{wyl} W. Y. Lam,  \emph{Minimal surfaces from infinitesimal deformations of circle packings}, Adv. Math. 362 (2020), 106939, 24 pp.
\bibitem{liu-zhou}J. Liu, Z. Zhou, \emph{How many cages midscribe an egg}, Invent.Math. 203 (2016), 655--673.

\bibitem{Luo} F. Luo. \emph{Rigidity of polyhedral surfaces I,} J. Differential Geom., 96(2), 2014, 241--302.
\bibitem{ma-ro} A. Marden, B. Rodin, \emph{On Thurston’s formulation and proof of Andreev’s theorem,} LNM
1435, Springer-Verlag, Berlin, 1990.
\bibitem{nie} X. Nie, \emph{On circle patterns and spherical conical metrics}, preprint, \url{https://arxiv.org/abs/2301.09585}.
\bibitem{po} L.S. Pontryagin, \emph{Ordinary differential equations}, Addison-Wesley Publishing Company Inc., Reading, 1962.
\bibitem{ste} K. Stephenson, \emph{Introduction to circle packing. The theory of discrete analytic functions}, Cambridge University Press, Cambridge, 2005.
 \bibitem{thurston} W. P. Thurston, \emph{Geometry and topology of 3-manifolds,}
 Princeton lecture notes, 1976.
\bibitem{wal} W. Walter, \emph{Ordinary Differential Equations}, Springer-Verlag New York, 1998
\bibitem{xu} X. Xu, \emph{Rigidity of inversive distance circle packings revisited}, Adv. Math. 332 (2018), 476--509.

\bibitem{ZGZ} M. Zhang, R. Guo, W. Zeng, F. Luo, S.T. Yau, X. Gu, \emph{The unified discrete surface Ricci
flow,} Graphical Models, 76 (2014), 321--339.


\end{thebibliography}


\Addresses
\end{document}
