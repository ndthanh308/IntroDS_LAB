\documentclass[letterpaper,11pt]{article}

\usepackage[square,numbers]{natbib}

\usepackage[utf8]{inputenc}   
\usepackage[T1]{fontenc}  

\usepackage{pgf,tikz}
\usetikzlibrary{arrows}
\usepackage{multicol}
\usepackage{amsthm}
\usepackage{amsmath}
\usepackage{amssymb}
\usepackage{amsfonts}
\usepackage{stmaryrd}
\usepackage{mathabx}
%\usepackage{mn­sym­bol}
\usepackage{latexsym}
\usepackage{color}
\usepackage{graphics,graphicx,graphpap}
\usepackage{multirow}
\usepackage{rotating}
\usepackage[new]{old-arrows}
\usepackage[all]{xy}
\usepackage[letterpaper]{geometry}
\usepackage{subfigure}
\usepackage{hyperref}

\theoremstyle{plain}

\usepackage{setspace}
\onehalfspacing

\usepackage{tocbibind}

\DeclareMathAlphabet{\mathpzc}{OT1}{pzc}{m}{it}
\newtheorem{theorem}{Theorem}
\newtheorem{prop}[theorem]{Proposition}
\newtheorem{cor}[theorem]{Corollary}
\newtheorem{lem}[theorem]{Lemma}
\newtheorem{que}[theorem]{Question}
\newcommand{\aste}[1]{\stackrel{*}{#1}}
\newcommand{\hocolim}{\operatornamewithlimits{\mathrm{hocolim}}}
\newcommand{\colim}{\operatornamewithlimits{\mathrm{colim}}}

\title{The Forest Filtration of a Graph}
\author{Andr\'es Carnero Bravo}

\begin{document}
\maketitle
\begin{abstract}
    Given a graph $G$, we define a filtration of simplicial complexes associated to $G$, 
    $\mathcal{F}_0(G)\subseteq\mathcal{F}_1(G)\subseteq\cdots\subseteq\mathcal{F}_\infty(G)$ 
    where the first complex is the independence complex and the last the complex is formed by the 
    acyclic sets of vertices. We prove some properties of this filtration and we calculate 
    the homotopy type for various families of graphs.
\end{abstract}
\tableofcontents

\section{Introduction}
Given a simple graph $G$, the sets of edges that induce acyclic graphs form a simplicial complex, 
the independence complex of the graphic matroid associated to the graph. This complex is pure of dimension $n-k(G)-1$ and 
has the homotopy type of the wedge of $T_G(0,1)$ spheres, where $T_G(x,y)$ is the Tutte polynomial of the 
graph (see \citep{bjornermatroidshella}). If instead of edges we take vertices, we get a complex $\mathcal{F}_\infty(G)$ which does 
not have to be pure and can, for example, have the homotopy type of a wedge of spheres of  different dimension, as it does for the complex associated to the $k$-complete 
multipartite graph for $k\geq3$. This complex, defined in \citep{tesiszuffi} but not studied, 
is part of a filtration of complexes associated to the graph:
$$\mathcal{F}_0(G)\subseteq\mathcal{F}_1(G)\subseteq\mathcal{F}_2(G)\subseteq\cdots\subseteq\mathcal{F}_\infty(G),$$
where for each $d$ we take as simplices the set of vertices which induce acyclic graphs of maximum degree at most $d$. The case $d=0$ is 
the independence complex and the case $d=1$ is also called the $2$-independence complex \citep{salvetti2015}. To the best of our knowledge, 
the cases $2\leq d<\infty$ have not been studied before.

\section{Preliminaries}
We only consider simple graphs, so no loops or multiedges are allowed. Given a graph $G$, $V(G)$ will be the set of vertices 
and $E(G)$ the set of edges. For $S\subseteq V(G)$, $G[S]$ is the graph induced by the set $S$. For a vertex $v$, 
$N_G(v)=\{u\in V(G):\;uv\in E(G)\}$ is its open neighborhood, we omit the subindex $G$ if there is no risk of confusion. 
The degree of a vertex will be denoted $d_G(v)=|N_G(v)|$. The maximum and minimum degrees will be denoted $\Delta(G)$ and 
$\delta(G)$ respectively. We say $G$ is a forest if it does not have cycles. Given a graph $G$, its girth $g(G)$ is the 
length of the smallest cycle in $G$, if there is no cycle in $G$ we take $g(G)=\infty$.

Given two graphs $G$ and $H$, there are two graphs over the vertex set $V(G)\times V(H)$:
\begin{enumerate}
    \item The cartesian product $G\oblong H$, where $\{(u_1,v_1)(u_2,v_2)\}$ is an edge if $u_1=u_2$ and $v_1v_2\in E(H)$ or 
    $u_1u_2\in E(G)$ and $v_1=v_2$.
    \item The categorical product $G\times H$, where $\{(u_1,v_1)(u_2,v_2)\}$ is an edge if $u_1u_2\in E(G)$ and $v_1v_2\in E(H)$.
\end{enumerate}
For all the graph definitions not stated here we follow \citep{graphsanddigraphs}.

A simplicial complex $K$ on the vertex set $V$ is a family of subsets, called simplices, such that if $\sigma$ is in $K$, any 
subset $\tau$ is also in $K$. We say that a simplex $\delta$ has dimension $|\delta|-1$. The $q$-skeleton of a complex $K$, denoted 
$sk_qK$, is the subcomplex of all simplices with at most $q+1$ elements. 

Given a simplex $\sigma\in K$, its link is the simplicial complex 
$lk(\sigma)=\{\tau\subseteq V\colon\tau\cap\sigma=\emptyset \;\;\mbox{and}\;\;\tau\cup\sigma\in K\}$ and its \textit{star} is
$st(\sigma)=\{\tau\in K:\tau\cup\sigma\in K\}$. For a vertex we will write $lk(v)$ and $st(v)$ instead of $lk(\{v\})$ or 
$st(\{v\})$.

We will not distinguish between a complex and its geometric realization. All the homology and cohomology groups will be with 
integer coefficients.

\begin{theorem}[Whitehead's theorem, see {\citep[Corollary 4.33]{hatcher}}]\label{whiteheadhomologia}
If $X$ and $Y$ are simply connected CW-complexes and there is a continuous map $f\colon X\longrightarrow Y$ such 
that $f_*:H_n(X)\longrightarrow H_n(Y)$ is an isomorphism for each $n$, then $f$ is an homotopy equivalence.
\end{theorem}

Given a complex $X$ on $n$ vertices, its Alexander Dual is the complex 
$$X^*=\{\sigma\subseteq V(X)\colon\;V(X)-\sigma\notin X\}.$$

\begin{theorem}\label{dualidadalexander}(see \citep{bjorneralexander})
Let $X$ be a simplicial complex with $n$ vertex, then
$$\tilde{H}_i(X)\cong\tilde{H}^{n-i-3}(X^*)$$
\end{theorem}

A well know fact we will use is the following result.
\begin{theorem}\label{cwsimplcon}
If $X$ is an CW-complex simply connected such that $\tilde{H}_q(X)\cong\mathbb{Z}^a$ for some $q\geq2$ and the rest of the homology groups 
are trivial, then $X\simeq\displaystyle\bigvee_a\mathbb{S}^q$.
\end{theorem}

\begin{theorem}(see \citep{hatcher} example 4C.2)\label{gradconse}
If $X$ is an CW-complex simply connected such that $\tilde{H}_q(X)\cong\mathbb{Z}^a$, $\tilde{H}_{q+1}(X)\cong\mathbb{Z}^b$ for some 
$q\geq2$ and the rest of the homology groups are trivial, then $X\simeq\displaystyle\bigvee_a\mathbb{S}^q\vee\bigvee_b\mathbb{S}^{q+1}$.
\end{theorem}

Now we give the basic results of homotopy colimits we will need. 
Taking $\underline{n}=\{1,\dots,n\}$ and $\mathcal{P}_1(\underline{n})=\mathcal{P}(\underline{n})-\{\underline{n}\}$, 
a punctured $n$-cube $\mathcal{X}$ consists of:
\begin{itemize}
    \item a topological space $\mathcal{X}(S)$ for each $S$ in $\mathcal{P}_1(\underline{n})$, and
    \item a continuous function $f_{S\subseteq T}\colon\mathcal{X}(S)\longrightarrow\mathcal{X}(T)$ for each $S\subseteq T$,
\end{itemize}
such that $f_{S\subseteq S}=1_{\mathcal{X}(S)}$ and for any $R\subseteq S\subseteq T$ the following diagram comutes
\begin{equation*}
    \xymatrix{
    \mathcal{X}(R) \ar@{->}[r]^{f_{R\subseteq S}} \ar@{->}[dr]_{f_{R\subseteq T}}& \mathcal{X}(S) \ar@{->}[d]^{f_{S\subseteq T}}\\
     & \mathcal{X}(T).
    }
\end{equation*}  
A punctured $n$-cube of interest for a given topological space $X$ is the constant punctured cube $\mathcal{C}_X$, 
where $\mathcal{C}_X(S)=X$ for any set $S$ and all the functions are $1_X$.
The colimit of a punctured $n$-cube is the space
$$\colim(\mathcal{X})=\bigsqcup_{S\in\mathcal{P}_1(\underline{n})}\mathcal{X}(S)/\sim$$
where $\sim$ is the equivalence relation generated by $f_{S\subseteq T_1}(x_S)\sim f_{S\subseteq T_2}(x_S)$ for $T_1,T_2$ and $S\subseteq T_1,T_2$. From the definition
is clear that $\colim(\mathcal{C}_X)\cong X$ for any $X$.

For any $n\geq1$ and $S$ in $\mathcal{P}_1(\underline{n})$ we take:
$$\Delta(S)=\left\lbrace(t_1,t_2,\dots,t_n)\in \mathbb{R}^n\colon\;\sum_{i=1}^nt_i=1\mbox{ and }t_i=0\mbox{ for all }i\in S\right\rbrace$$
and $d_{S\subseteq T}\colon \Delta(T)\longrightarrow\Delta(S)$ the corresponding inclusion. Now, for a punctured 
$n$-cube $\mathcal{X}$, its homotopy colimit is 
$$\hocolim(\mathcal{X})=\bigsqcup_{S\in\mathcal{P}_1(\underline{n})}\mathcal{X}(S)\times\Delta(S)/\sim$$
where $(x_S,d_{S\subseteq T}(t))\sim(f_{S\subseteq T}(x_S),t)$. When $n=2$, we will specify the punctured $2$-cube as the diagram
\begin{equation*}
    \xymatrix{
    \mathcal{D}\colon & X\ar@{<-}[r]^{f} & Z \ar@{->}[r]^{g} & Y
    }
\end{equation*}
and its homotopy colimit is called the homotopy pushout. 

There is a recursive way to compute homotopy colimits of punctured $n$-cubes. Given a punctured $n$-cube $\mathcal{X}$ for $n\geq2$ and defining the punctured $(n-1)$-cubes
$\mathcal{X}_1(S)=\mathcal{X}(S)$ and $\mathcal{X}_2(S)=\mathcal{X}(S\cup\{n\})$,
we have that (Lemma 5.7.6 \citep{cubicalhomotopy})
$$\hocolim(\mathcal{X})\cong \hocolim\left(\mathcal{X}\left(\underline{n-1}\right)\longleftarrow \hocolim(\mathcal{X}_1)\longrightarrow \hocolim(\mathcal{X}_2)\right).$$

If for all  $S\subsetneq[n]$ the map 
$$\colim_{T\subsetneq S}X_T \longrightarrow X_S$$
is a cofibration, we say the punctured cube is cofibrant.  
If we have $X_1,\dots,X_n$ CW-complexes such that their intersections are subcomplexes, and we take the punctured cube given by the 
intersections and the inclusions between them, then the punctured cube is cofibrant and 
$\hocolim(\mathcal{X})\simeq \colim(\mathcal{X})$ (Proposition 5.8.25 \citep{cubicalhomotopy}).

\begin{lem}\label{homocolimpegado}
Let $X,Y,Z$ be spaces with maps $f\colon Z\longrightarrow X$ and $g\colon Z\longrightarrow Y$ such that both maps are null-homotopic. Then
$$\hocolim\left(\mathcal{S}\right)\simeq X\vee Y\vee \Sigma Z$$
where 
\begin{equation*}
    \xymatrix{
    \mathcal{S}\colon & Y \ar@{<-}[r]^{g} & Z \ar@{->}[r]^{f} & X
    }
\end{equation*}
\end{lem}

\begin{lem}\label{lemhomotopiagraf}
Let $X_1,\dots,X_k$ simplicial complexes such that the intersection of two or more is 
null-homotopic or empty, $X_i$ is connected for all $i$ and there is a graph $G$ of size $k$ and a 
bijection $\gamma\colon\{1,\dots,k\}\longrightarrow E(G)$ such that 
$\bigcap_{i\in S}\gamma(i)\neq\emptyset$ if an only if $\bigcap_{i\in S}X_i\neq\emptyset$ for all non-empty 
$S$ subset of $\{1,\dots,k\}$. Then 
$\displaystyle X=\bigcup_{i=1}^kX_i$ has the homotopy type of the nerve with the complexes 
$X_i$ attached to the corresponding point in the nerve.
\end{lem}
\begin{proof}
By induction on $k$. For $k=1,2$ the result is clear. Assume it is true for any $r\leq k$ and take 
$X_1,\dots,X_{k+1}$ simplicial complexes such that the intersection of two or more is 
null-homotopic or empty, $X_i$ is connected for all $i$ and there is a graph $G$ of size $k+1$ and a 
bijection $\gamma\colon\{1,\dots,k+1\}\longrightarrow E(G)$ such that 
$\bigcap_{i\in S}\gamma(i)\neq\emptyset$ if an only if $\bigcap_{i\in S}X_i\neq\emptyset$ for all non-empty 
$S$ subset of $\{1,\dots,k+1\}$. Now, 
take $\mathcal{N}$ the nerve complex of $X_1,\dots,X_{k+1}$. For any $i\in\{1,\dots,k+1\}$, $lk(i)$ is:
\begin{itemize}
\item[(a)] Empty if in the corresponding edge both vertices have degree $1$.
\item[(b)] Contractible if in the corresponding edge one vertex has degree $1$ and the other degree at 
least $2$.
\item[(c)] Homotopy equivalent to $\mathbb{S}^0$ if in the corresponding edge both vertices have degree at least 
$2$.
\end{itemize}
By the inductive formula for homotopy colimits of punctured cubes, the homotopy colimit of the intersection diagram associated to $X_1,\dots,X_{k+1}$ is homotopy equivalent to the homotopy pushout of the diagram
\begin{equation*}
\xymatrix{\mathcal{S}\colon\hocolim(\mathcal{S}_2)&\hocolim(\mathcal{S}_1) \ar@{->}[l] \ar@{->}[r]& X_{k+1}},
\end{equation*}
where $S_1$ is the homotopy colimit of the intersection diagram associated to $X_1\cap X_{k+1},\dots,X_k\cap X_{k+1}$, and 
$\mathcal{S}_2$ is the homotopy colimit of the intersection diagram associated to $X_1,\dots,X_k$. 
Now $\hocolim(\mathcal{S}_1)\simeq lk(k+1)$, so we have three possibilities:
\begin{itemize}
    \item[(a)] $\hocolim(\mathcal{S}_1)=\emptyset$, then $\hocolim(\mathcal{S})\simeq\hocolim(\mathcal{S}_2)\sqcup X_{k+1}$
    
    \item[(b)] $\hocolim(\mathcal{S}_1)\simeq*$, then $\hocolim(\mathcal{S})\simeq\hocolim(\mathcal{S}_2)\vee X_{k+1}$
    
    \item[(c)] $\hocolim(\mathcal{S}_1)\simeq\mathbb{S}^0$, then 
    $\hocolim(\mathcal{S})\simeq\hocolim(\mathcal{S}_2)\vee\mathbb{S}^1\vee X_{k+1}$
\end{itemize}
\end{proof}

\section{Definition and basic properties}
Let $G$ be a graph, we define its \textit{$d$-forest complex} as the complex
$$\mathcal{F}_d(G)=\{\sigma\subseteq V(G)\colon \:G[\sigma] \mbox{ is a forest with }\Delta(G[\sigma])\leq d\};$$
for $d=\infty$ we take 
$$\mathcal{F}_\infty(G)=\{\sigma\subseteq V(G)\colon \:G[\sigma] \mbox{  is a forest}\}.$$
For $d=0$, $\mathcal{F}_0(G)$ is the independence complex of $G$ and for $d=1$ is also called the $2$-independence complex of $G$ ---the $r$-independence complex of $G$ has as simplices sets $A \subseteq V(G)$ such that every connected component of $G[A]$ has at most $r$ vertices. Note 
that if $d+1=\min\{r\colon G\mbox{ is } K_{1,r}\mbox{-free}\}$, then $\mathcal{F}_l(G)=\mathcal{F}_d(G)$ for all $l\geq d$.

Given a graph $G$ let $t_d(G)=\max\{|V(T)|\colon T \mbox{ is an induced forest such that }\Delta(T)\leq d\}$, by definition 
$t_d(G)=\dim(\mathcal{F}_d(G))+1$, 
therefore knowing the homotopy type of $\mathcal{F}_d(G)$ or its homology groups gives us a lower bound for $t_d(G)$. 


\begin{theorem}
For any graph $G$ and all $d$, the pair $(\mathcal{F}_{d+1}(G),\mathcal{F}_d(G))$ is $d$-connected.
\end{theorem}
\begin{proof}
For any $d$, we have that $sk_i\mathcal{F}_d(G)=sk_i\mathcal{F}_{d+1}(G)$ for all $i\leq d$ because a forest of order $i+1$ has maximum 
degree at most $i$. Then all the cells in $\mathcal{F}_{d+1}(G)-\mathcal{F}_d(G)$ have dimension greater than $d$ and this implies 
the result (see \citep{hatcher} Corollary 4.12).
\end{proof}

By definition the following results are clear.

\begin{prop}
For $d\geq1$,
$$\mathcal{F}_d(K_n)\simeq\bigvee_{\frac{(n-1)(n-2)}{2}}\mathbb{S}^1.$$
\end{prop}

\begin{prop}
For $n\geq3$ and $d\geq2$,
$$\mathcal{F}_d(C_n)\cong\mathbb{S}^{n-2}.$$
\end{prop}

A subset of vertices $\sigma$ is an independent set if all of its subsets of cardinality $2$ are independent. This says that 
in order to be a simplex of the independence complex, a set of vertices only need have its $1$-skeleton 
contained in the complex. This type of complexes are called \emph{flag complexes}. Now, for 
$\mathcal{F}_1(G)$, its $1$-skeleton is the complete graph of the same order as $G$, therefore it is not a flag complex in general, because 
it is not contractible for all graphs. The following result tells us that it has an analogous property but for the $2$-skeleton, rather than the $1$-skeleton.

\begin{prop}
Let $\sigma$ be a subset of $V(G)$ such that all of its subsets of cardinality $3$ are simplices of $\mathcal{F}_1(G)$, then 
$\sigma$ is a simplex of $\mathcal{F}_1(G)$.
\end{prop}
\begin{proof}
If $|\sigma|\leq3$ the result is clear. Now let $\sigma=\{v_0,v_1,v_2,v_3\}$. Then, for $\tau=\{v_0,v_1,v_2\}$, we have that $G_\tau=G[\tau]$ is forest such that 
$\Delta(G_\tau)\leq1$. Now, $v_3$ at most can have one neighbor in $\tau$ and it must be a vertex 
of degree $0$ in $G_\tau$. Therefore $G_\sigma=G[\sigma]$ is a graph such that $\Delta(G_\sigma)\leq1$, which implies it is a forest 
and $\sigma$ is a simplex of $\mathcal{F}_1(G)$.

Assume the result is true for any subset of at most $k\geq 4$ vertices that has its $2$-skeleton in $\mathcal{F}_1(G)$. Let 
$\sigma=\{v_0,\dots,v_k\}$ a subset of $k+1$ vertices such that its $2$-skeleton is in $\mathcal{F}_1(G)$. By induction hypothesis, 
$\tau=\{v_0,\dots,v_{k-1}\}$ is a simplex of $\mathcal{F}_1(G)$, therefore, taking $G_\tau$ as before,
$$G_\tau\cong rK_1 \sqcup M_s$$
with $r,s\geq0$ and $r+2s=k+1$. By hypothesis, $v_k$ can not be adjacent to a vertex in $M_s$ and only can be adjacent to 
one vertex in $rK_1$. So $\sigma$ induces a graph with maximum degree at most $1$ and therefore $\sigma$ is a simplex of 
$\mathcal{F}_1(G)$.
\end{proof}
This can not be generalized for $\mathcal{F}_d(G)$ with $d\geq2$ as $\mathcal{F}_d(C_{d+3})$ shows.

If a simplicial complex $K$ is such that $\tilde{H}_q(K)\ncong0$, then $f_i(K)\geq f_i\left(\Delta^{q+1}\right)$ and $f_0(K)=q+2$ 
if and only if $K\cong\partial\left(\Delta^{q+1}\right)$.

\begin{prop}
Let $G$ be a graph of order $q+2$, with $q\geq1$, then:
\begin{enumerate}
    \item If $\tilde{H}_q(\mathcal{F}_q(G))\ncong0$, then $G\cong K_{_{1,q+1}}$ or $G\cong C_{q+2}$
    \item If $\tilde{H}_q(\mathcal{F}_\infty(G))\ncong0$, then $G\cong C_{q+2}$
\end{enumerate}
\end{prop}
\begin{proof}
For $d=q$ or $d=\infty$, we have that $\mathcal{F}_d(G)\cong\partial\left(\Delta^{q+1}\right)$ and  
for any proper subset of the vertices $S$, $\mathcal{F}_d(G[S])$ must be contractible. 
If $\Delta(G)=q+1$, then $G$ can not have cycles because $V(G)-\{v\}$ is a simplex for any vertex and 
$\mathcal{F}_\infty(G)\simeq*$. Take
$v$ a vertex such that $d_G(v)=q+1$, then $\mathcal{F}_q(G)=st(v)\cup\mathcal{F}_q(G-v)$ and, because 
$\tilde{H}_{q}(\mathcal{F}_q(G-v))\cong0$, using the Mayer-Vietoris sequence we have that 
$\tilde{H}_{q-1}(lk(v))\ncong0$. Therefore $lk(v)\cong\partial\left(\Delta^{q}\right)$. If $q=1$, then 
$lk(v)$ is two disjoint vertices from where it follows that $G\cong K_{1,2}$ or $G\cong C_3$. Assume $q\geq2$, 
then $N(v)$ must be an independent set and $G\cong K_{_{1,q+1}}$.

Assume $\Delta(G)\leq q$, then $G$ must have a cycle, otherwise $\mathcal{F}_d(G)\simeq*$ for $d=q$ or 
$d=\infty$. Let $C\leq G$ be an induced cycle. If $V(C)\subsetneq V(G)$, because any proper subset is a simplex, $V(C)$ 
will be a simplex, but this can not happen. Therefore $G\cong C_{q+2}$.
\end{proof}

\begin{prop}\label{proppuente}
If $e\in E(G)$ is bridge, then $\mathcal{F}_{\infty}(G)=\mathcal{F}_{\infty}(G-e)$.
\end{prop}

\begin{lem}
If $G=G_1\sqcup G_2$, then for all $d$,
$$\mathcal{F}_d(G)=\mathcal{F}_d(G_1)*\mathcal{F}_d(G_2).$$
\end{lem}

\begin{prop}
If $G=G_1\sqcup\cdots\sqcup G_k$, then for $d\geq0$, $$conn(\mathcal{F}_d(G))\geq2k-2+\sum_{i=1}^k conn(\mathcal{F}_d(G_i)).$$
\end{prop}
\begin{proof}
This follows from $\mathcal{F}_d(G)=\mathcal{F}_d(G_1)*\cdots*\mathcal{F}_d(G_k)$
\end{proof}

\begin{lem}
If $v$ is a vertex such that no cycle of $G$ contains it, then 
$\mathcal{F}_\infty(G)\simeq*$.
\end{lem}
\begin{proof}
Beacause $v$ does not belongs to a cycle, then $\mathcal{F}_\infty(G)=\{v\}*\mathcal{F}_\infty(G-v)$.
\end{proof}

\begin{cor}
If $\delta(G)\leq1$, then $\mathcal{F}_\infty(G)\simeq*$.
\end{cor}

\begin{cor}\label{corgrad2}
If $G$ has a vertex $v$ such that $N_G(v)=\{v_1,v_2\}$, then 
$\mathcal{F}_\infty(G)\simeq\Sigma lk_{_{\mathcal{F}_\infty(G)}}(v_i)$ for $i=1,2$.
\end{cor}
\begin{proof}
Because $N_G(v)=\{v_1,v_2\}$, then $d_{G-v_i}(v)=1$ and therefore 
$\mathcal{F}\infty(G-v_i)\simeq*$. Now $\mathcal{F}_\infty(G)\simeq\hocolim(\mathcal{S})$ with  
$\mathcal{S}\colon\mathcal{F}_\infty(G-v_i)\longhookleftarrow lk_{_{\mathcal{F}(G,\infty)}}(v_i)\longhookrightarrow st_{_{\mathcal{F}(G,\infty)}}(v_i)$, 
by Lemma \ref{homocolimpegado} we obtain the result.
\end{proof}

\begin{prop}
Let $G$ be a graph such that $\tilde{H}_q(\mathcal{F}_d(G))\ncong0$ for some $d$ and $q$, then $G$ has at least $q+2$ different
induced forests of $q+1$ vertices and maximun degree at most $d$.
\end{prop}

\begin{lem}\label{lemlinkvertrian}
Let $G$ be a graph that is the union of three graphs $G_1,G_2,G_0$ such that: 
\begin{itemize}
    \item $G_0\cong K_3$
    \item $V(G_0)=\{v,v_1,v_2\}$
    \item $V(G_1)\cap V(G_0)=\{v_1\}$, $V(G_2)\cap V(G_0)=\{v_2\}$ and $V(G_1)\cap V(G_2)=\emptyset$
\end{itemize}
Then, $lk_{\mathcal{F}_\infty(G)}(v)\simeq\hocolim(\mathcal{S})$ with $\mathcal{S}$ the diagram:
\begin{equation*}
    \xymatrix{\mathcal{F}_\infty(G_1)*\mathcal{F}_\infty(G_2-v_2)\ar@{<-^)}[r] & \mathcal{F}_\infty(G_1-v_1)*\mathcal{F}_\infty(G_2-v_2)\ar@{^(->}[r]& \mathcal{F}_\infty(G_1-v_1)*\mathcal{F}_\infty(G_2)}
\end{equation*}
\end{lem}
\begin{proof}
Because 
$$lk_{\mathcal{F}_\infty(G)}(v)=(\mathcal{F}_\infty(G_1)*\mathcal{F}_\infty(G_2-v_2))\cup(\mathcal{F}_\infty(G_1-v_1)*\mathcal{F}_\infty(G_2))$$
and
$$(\mathcal{F}_\infty(G_1)*\mathcal{F}_\infty(G_2-v_2))\cap(\mathcal{F}_\infty(G_1-v_1)*\mathcal{F}_\infty(G_2))=\mathcal{F}_\infty(G_1-v_1)*\mathcal{F}_\infty(G_2-v_2)$$
we have that 
$$lk_{\mathcal{F}_\infty(G)}(v)=\colim(\mathcal{S})\simeq\hocolim(\mathcal{S})$$
\end{proof}


\begin{prop}\label{conhomcuello}
Let $G$ be a graph with $n$ vertices and $g(G)<\infty$, then $\tilde{H}_i\left(\mathcal{F}_\infty(G)\right)\cong0$ for all 
$i<g(G)-2$.
\end{prop}
\begin{proof}
The Alexander Dual has dimension $n-g(G)-1$, thus $\tilde{H}^k\left(\mathcal{F}_\infty^*(G)\right)\cong0$ for all 
$k>n-g(G)-1$. By Theorem \ref{dualidadalexander}, $\tilde{H}_i\left(\mathcal{F}_\infty(G)\right)\cong0$ for all 
$i<g(G)-2$. 
\end{proof}

In the last proposition we saw that the girth gives us a lower bound for the homological connectivity of $\mathcal{F}_\infty(G)$, now we 
will see that this bound also works for the connectivity, first we show that $g(G)\geq4$ implies that $\mathcal{F}_\infty(G)$ is simply 
connected, by showing this for $\mathcal{F}_2(G)$.

\begin{prop}
Let $G$ be a graph with $g(G)\geq4$, then $\pi_1\left(\mathcal{F}_2(G)\right)\cong0$.
\end{prop}
\begin{proof}
We will prove it for connected graphs. We take $T$ a spanning tree of $G$ and take the finitely presented group $H_T$ 
with set of generators $E(G)\cup E(G^c)$ and with the following relations:
\begin{itemize}
    \item $uv=1$ for all the edges of $T$,
    \item $(uv)(vw)=uw$ if $\{u,v,w\}$ is a simplex of $\mathcal{F}_\infty(G)$.
\end{itemize}
we have that $H_T\cong\pi_1\left(\mathcal{F}_\infty(G)\right)$ (see \citep{rotmantop} Theorem 7.34).

Note that any triple of vertices $\{u,v,w\}$ spans a forest in $G$ because $g(G) \ge 4$, so the 2-skeleton of $\mathcal{F}_2(G)$ contains all possible triangles.

We will show that all generators $uv$ are trivial by induction on the distance $k=d_T(u,v)$. If $k=1$, this is clear by the first type of relation. Assume $uv$ is trivial if $d_T(u,v)\leq k$. Take $uv$ such that 
$d_T(u,v)=k+1$ and take $uw_1w_2\cdots w_kv$ the $uv$-path in $T$. Since $\{u,w_1,v\}$ is a simplex of $\mathcal{F}_2(G)$, the second relation implies $uv=(uw_1)(w_1v) = w_1v$ where we have $d_T(w_1,v)=k$.
\end{proof}

Because $\mathcal{F}_\infty(G)$ is always connected, using the last proposition, Proposition\ref{conhomcuello} and 
the Hurewicz Theorem we have the following result:
\begin{theorem}\label{coneccuello}
For any graph $G$, $conn\left(\mathcal{F}_\infty(G)\right)\geq g(G)-3$.
\end{theorem}


\section{Homotopy type calculations}
In this secction for various families of graphs we will calculate the homotopy type for the forest complexes, 
for every step in the filtration in most cases; and in some cases, like the cactus graphs, only for the last complex of the filtration.
\subsection{Some Graph families}
\subsubsection{Paths and cycles}
The homotopy type of all $r$-independence complexes of paths was calculated by Salvetti \citep{salvetti2018} using discrete
Morse theory. Here we give a different proof for $\mathcal{F}_1$ using homotopy pushouts, which also shows that 
the inclusion $\mathcal{F}_1(P_{4r+3})\longhookrightarrow\mathcal{F}_1(P_{4(r+1)})$ is a homotopy equivalence. This will allow us 
to calculate the homotopy type of $\mathcal{F}_1$ avoiding  discrete Morse theory, which was the tool used in \citep{singhhigher}.
\begin{prop}\citep{salvetti2018}
$$\mathcal{F}_1(P_n)\simeq\left\lbrace\begin{array}{cc}
    \mathbb{S}^{2r-1} &  \mbox{ if }n=4r\\
    \ast & \mbox{ if }n=4r+1\mbox{ or }n=4r+2\\
    \mathbb{S}^{2r+1} & \mbox{ if }n=4r+3
\end{array}
\right.$$
\end{prop}
\begin{proof}
For $r=0$, it is clear that $\mathcal{F}_1(P_1)\simeq*\simeq\mathcal{F}_1(P_2)$. For $P_3$, 
$\mathcal{F}_1(P_3)\cong K_3$. For $\mathcal{F}_1(P_4)$ 
$$lk(v_4)=\mathcal{F}_1(P_2)\cup\{v_3\}*\mathcal{F}_1(P_1)\simeq*$$
therefore the inclusion $i\colon \mathcal{F}_1(P_3)\longhookrightarrow\mathcal{F}_1(P_4)$ is a homotopy equivalence.

Next, we will prove that $\mathcal{F}_1(P_{4r+1})\simeq\mathcal{F}_1(P_{4r+2})\simeq*$  for all $r\geq1$.

Assume that it is true for any $1\leq r\leq k$. For $\mathcal{F}_1(P_{4(k+1)})$, by induction hypothesis,
$$lk(v_{4(k+1)})=\mathcal{F}_1(P_{4k+2})\cup\{v_{4k+3}\}*\mathcal{F}_1(P_{4k+1})\simeq*$$
therefore the inclusion $\mathcal{F}_1(P_{4k+3})\longhookrightarrow\mathcal{F}_1(P_{4(k+1)})$ is 
a homotopy equivalence.

Now, for $\mathcal{F}_1(P_{4(k+1)+1})$ we have
$$lk(v_{4(k+1)+1})=\mathcal{F}_1(P_{4k+3})\cup\{v_{4(k+1)}\}*\mathcal{F}_1(P_{4k+2}).$$
Setting $X=\mathcal{F}_1(P_{4k+2})$ and $Y=\{v_{4k+3}\}*\mathcal{F}_1(P_{4k+2})$, we have by induction hypothesis that 
$$X\cap Y=\mathcal{F}_1(P_{4k+2})\simeq*$$
therefore $\mathcal{F}_1(P_{4k+3})\longhookrightarrow lk(v_{4(k+1)+1})$ is a homotopy equivalence.
\begin{equation*}
    \xymatrix{
    \mathcal{F}_1(P_{4k+3}) \ar@{->}[r]_{\simeq} \ar@{->}[d] \ar@{->}@/^{5mm}/[rr]^{\simeq} & lk(v_{4(k+1)+1}) \ar@{->}[r] \ar@{->}[d] & \mathcal{F}_1(P_{4(k+1)}) \ar@{->}[d] \\
    st(v_{4(k+1)+1}) \ar@{->}[r] & st(v_{4(k+1)+1}) \ar@{->}[r]& \mathcal{F}_1(P_{4(k+1)+1})
    }
\end{equation*}

$$\mathcal{F}_1(P_{4(k+1)+1})\simeq st(v_{4(k+1)+1})\simeq*$$

For $\mathcal{F}_1(P_{4(k+1)+2})$:
$$lk(v_{4(k+1)+2})=\mathcal{F}_1(P_{4(k+1)})\cup\{v_{4(k+1)+1}\}*\mathcal{F}_1(P_{4k+3});$$
because $\mathcal{F}_1(P_{4k+3})\longhookrightarrow\mathcal{F}_1(P_{4(k+1)})$ is an homotopy equivalence, we have 
that $lk(v_{4(k+1)+2})\simeq*$ and therefore 
$$\mathcal{F}_1(P_{4(k+1)+2})\simeq\mathcal{F}_1(P_{4(k+1)+1})\simeq*.$$

We have that $\mathcal{F}_1(P_{4(k+1)})\simeq\mathcal{F}_1(P_{4k+3})$; now for this last complex:
$$lk(v_{4k+3})=\mathcal{F}_1(P_{4k+1})\cup\{v_{4k+2}\}*\mathcal{F}_1(P_{4k}),$$
where $\mathcal{F}_1(P_{4k+1})\simeq*$, therefore 
$$lk(v_{4k+3})\simeq\Sigma \mathcal{F}_1(P_{4k}).$$
Since $\mathcal{F}_1(P_{4k+2})\simeq*$, we have that $\mathcal{F}_1(P_{4k+3})\simeq\Sigma^2\mathcal{F}_1(P_{4k})$
and 
$$\mathcal{F}_1(P_{4(k+1)})\simeq\Sigma^2\mathcal{F}_1(P_{4k})\simeq\Sigma^2\mathbb{S}^{2k-1}\simeq\mathbb{S}^{2k+1}.$$
Doing the exact same argument we can see that $\mathcal{F}_1(P_{4(k+1)+3})\simeq\Sigma^2\mathcal{F}_1(P_{4(k+1)})$
and therefore 
$$\mathcal{F}_1(P_{4(k+1)+3})\simeq\Sigma^2\mathbb{S}^{2k+1}\simeq\mathbb{S}^{2k+3}.$$
\end{proof}

In the proof of the last proposition we saw that the inclusion $\mathcal{F}_1(P_{4k+3})\longhookrightarrow\mathcal{F}_1(P_{4(k+1)})$ 
obtained by erasing the last (or the first) vertex is an homotopy equivalence. We will use this fact in the following corollary.

\begin{cor}\citep{singhhigher}\label{corciclos1}
$$\mathcal{F}_1(C_n)\simeq\left\lbrace\begin{array}{cc}
    \displaystyle\bigvee_{3}\mathbb{S}^{2r-1} & \mbox{ if } n=4r \\
    \mathbb{S}^{2r-1} & \mbox{ if } n=4r+1 \\
    \mathbb{S}^{2r} & \mbox{ if } n=4r+2 \\
    \mathbb{S}^{2r+1} & \mbox{ if } n=4r+3
\end{array}\right.$$
\end{cor}
\begin{proof}
For $n=3,4$, the only possible simplices are a vertex or pair of vertices, any set with more vertices will have a $3$-path or a cycle. 
Therefore $\mathcal{F}_1(C_3)\cong K_3$ and $\mathcal{F}_1(C_4)\cong K_4$. For $n=5$, taking $v_1,v_2,v_3,v_4,v_5$ the vertices of 
the cycle with edges $v_iv_{i+1}$, the facets of $\mathcal{F}_1(C_5)$ are $\sigma_i=\{v_i,v_{i+2},v_{i+3}\}$. The edge $v_{i+2}v_{v_i+3}$ 
only is contained in $\sigma_i$, so we can collapse it for all $i$. Therfore $\mathcal{F}_1(C_5)\simeq\mathcal{F}_0(C_5)\cong\mathbb{S}^1$.

Assume $n\geq6$ and let $v_1,\dots,v_n$ be the vertices of the cycle. Then $lk(v_n)=K_1\cup K_2\cup K_3$ where
$$K_1=\mathcal{F}_1(C_n-v_n-v_2-v_{n-1})\cong C(\mathcal{F}_1(P_{n-4}))$$
$$K_2=\mathcal{F}_1(C_n-v_n-v_1-v_{n-2})\cong C(\mathcal{F}_1(P_{n-4}))$$
$$K_3=\mathcal{F}_1(C_n-v_n-v_1-v_{n-1})\cong \mathcal{F}_1(P_{n-3})$$
Now
$$K_1\cap K_2\cap K_3=K_1\cap K_2=\mathcal{F}_1(C_n-v_n-v_1-v_2-v_{n-1}-v_{n-2})\cong \mathcal{F}_1(P_{n-5})$$
$$K_1\cap K_3=\mathcal{F}_1(C_n-v_n-v_1-v_2-v_{n-1})\cong \mathcal{F}_1(P_{n-4})$$
$$K_2\cap K_3=\mathcal{F}_1(C_n-v_n-v_1-v_{n-1}-v_{n-2})\cong \mathcal{F}_1(P_{n-4})$$
$$K_1\cup K_2\simeq\Sigma\mathcal{F}_1(P_{n-5})$$

If $n=4r$, $K_1\cap K_2\cong\mathcal{F}_1(P_{4(r-2)+3})$, $K_3\simeq*$ and 
$K_1\cap K_3\cong\mathcal{F}_1(P_{4(r-1)})\cong K_2\cap K_3$. 
By the observation before the corollary, the inclusion 
$K_1\cap K_2\cap K_3\longhookrightarrow K_1\cap K_3$ is a homotopy equivalence. Therefore 
$(K_1\cup K_2)\cap K_3\simeq K_2\cap K_3$ and 
$$lk(v_n)\simeq\bigvee_{2}\mathbb{S}^{2r-2},$$
Since
$$\mathcal{F}_1(C_{4r}-v_n)\simeq\mathbb{S}^{2r-1},$$
we obtain the result.

If $n=4r+1$, $K_1\cap K_3\simeq K_2\cap K_3\cong\mathcal{F}_1(P_{4(r-1)+1})\simeq*$ and 
$K_2\cup K_3\simeq K_3$. Because $K_1\cap K_2\cap K_3=K_1\cap K_2$, we have that 
$$(K_2\cup K_3)\cap K_1\simeq K_1\cap K_3\simeq*$$
and 
$$K_1\cup K_2\cup K_3\simeq K_2\cup K_3\simeq K_3\cong\mathcal{F}_1(P_{4(r-1)+2})\simeq*.$$
Therefore $\mathcal{F}_1(C_{4r+1})\simeq\mathcal{F}_1(P_{4r})\simeq\mathbb{S}^{2r-1}$.

For $n=4r+2$ and $n=4r+3$, $\mathcal{F}_1(C_n-v_n)\simeq*$, therefore $\mathcal{F}_1(C_n)\simeq\Sigma lk(v_n)$.
If $n=4r+2$, $K_1\cap K_2\cong\mathcal{F}_1(P_{4(r-1)+1})\simeq*$ and 
$K_1\cap K_3,K_2\cap K_3\cong\mathcal{F}_1(P_{4(r-1)+2})\simeq*$. Then 
$K_1\cup K_2\simeq*$ and $(K_1\cup K_2)\cap K_3\simeq*$. From this we have that $lk(v_n)\simeq K_3$, therefore
$$\mathcal{F}_1(C_{4r+2})\simeq\Sigma\mathcal{F}_1(P_{4(r-1)+3})\simeq\mathbb{S}^{2r}.$$

If $n=4r+3$, $K_2\cap K_3\cong\mathcal{F}_1(P_{4(r-1)+3})$ and the inclusion $K_2\cap K_3\longhookrightarrow K_3$ is a homotopy 
equivalence, therefore $K_2\cup K_3\simeq*$. From this $lk(v_n)\simeq\Sigma(K_1\cap(K_2\cup K_3))$. Since 
$K_1\cap K_2\cap K_3=K_1\cap K_2$, we have that $K_1\cap(K_2\cup K_3)\simeq K_1\cap K_3$ and 
$$\mathcal{F}_1(C_{4r+3})\simeq\Sigma^2\mathcal{F}_1(P_{4(r-1)+3})\simeq\mathbb{S}^{2r+1}.$$
\end{proof}

\begin{prop}\label{propciclocuerda}
$$\mathcal{F}_{\infty}(C_n+e)\cong\mathbb{S}^{n-3}$$
\end{prop}
\begin{proof}
Assume the vertices of $G=C_n+e$ are labeled $v,w_1,\dots,w_r,u,w_{r+1},\dots,w_{r+k}$ with $e=vu$ (Figure \ref{c4c4}). 
Because $\mathcal{F}_\infty(G-v)\simeq*$, we have that $\mathcal{F}_\infty(G)\simeq\Sigma lk(v)$. Now, $lk(v)$ is formed by the subsets of 
$V(G-v)$ such that together with $v$ they do not induce a cycle, therefore the facets are 
$$\sigma_0=[w_1,\dots,w_r,w_{r+1},\dots,w_{r+k}]$$ 
and 
$$\sigma_{ij}=[w_1,\dots,\hat{w}_i,\dots,w_r,u,w_{r+1},\dots,,\hat{w}_{r+j},\dots,w_{r+k}]$$
for $1\leq i\leq r$, $1\leq j\leq k$.
If we call $K$ the complex form by $\sigma_0$ and its subsets, and $L$ the complex which facets are the simplices 
$\sigma_{ij}$, we get that $lk(v)=K\cup L$ and both of this complexes are contractible, therefore 
$lk(v)\simeq\Sigma K\cap L$.

Now, taking $X$ the complex with facets $[w_{r+1},\dots,,\hat{w}_{r+j},\dots,w_{r+k}]$  and 
$Y$ the complex with facets $[w_1,\dots,\hat{w}_i,\dots,w_r]$, we have that 
$K\cap L\cong X*Y$. Because $X\cong\mathbb{S}^{k-2}$ and 
$Y\cong\mathbb{S}^{r-2}$, we have that
$K\cap L\cong\mathbb{S}^{k-2}*\mathbb{S}^{r-2}\cong\mathbb{S}^{r+k-3}$ and, because $r+k=n-2$, 
$\mathcal{F}_\infty(G)\simeq\mathbb{S}^{n-3}$.
\end{proof}

% Figure environment removed


\subsubsection{Double stars}

Let $St_{r,s}$ be the \emph{double star} with $V(St_{r,s})=\{u_0,u_1,\dots,u_r,v_0,v_1,\dots,v_s\}$ and 
$E(St_{r,s})=\{u_iu_0\colon \;i>0\}\cup\{v_iv_0\colon \;i>0\}\cup\{u_0v_0\}$
\begin{prop}
$$\mathcal{F}_1(St_{r,s})\simeq\mathbb{S}^1$$
and for $2\leq d<\infty$
$$\mathcal{F}_d(St_{r,s})\simeq\bigvee_{{r-1\choose{d-1}}{s-1\choose{d-1}}}\mathbb{S}^{2d-1}$$
\end{prop}
\begin{proof}
For $\mathcal{F}_1(St_{r,s})$, the link of $u_0$ has as facets $\sigma_i=\{u_i,v_1,\dots,v_s\}$ for all $i$ and $\{v_0\}$, therefore
$$lk(u_0)\simeq\mathbb{S}^0.$$
Since $\mathcal{F}_1(St_{r,s}-u_0)\simeq*$, we have that $\mathcal{F}_1(St_{r,s})\simeq\Sigma lk(u_0)\simeq\mathbb{S}^1$.

For $d\geq2$, if $r\leq d-1$ or $s\leq d-1$, then $\mathcal{F}_d(St_{r,s})\simeq*$, because the set $\{u_1,\dots,u_r\}$ or the set 
$\{v_1,\dots,v_s\}$ would be contained in all facets.
Assume $r,s\geq d$. 
The facets of $\mathcal{F}_d(St_{r,s})$, besides 
$X=\{u_1,\dots,u_r,v_1\dots,v_s\}$, are of $3$ types:
\begin{enumerate}
    \item $\alpha_S=S\cup\{u_0,v_1,\dots,v_s\}$, where $S\subseteq\{u_1,\dots,u_r\}$ and $|S|=d$.
    \item $\beta_S=S\cup\{v_0,u_1,\dots,u_r\}$, where $S\subseteq\{v_1,\dots,v_s\}$ and $|S|=d$.
    \item $\sigma_{_{S_1,S_2}}=\{u_0,v_0\}\cup S_1\cup S_2$, where $S_1\subseteq\{u_1,\dots,u_r\}$, $S_2\subseteq\{v_1,\dots,v_s\}$ and 
    $|S_1|=|S_2|=d-1$.
\end{enumerate}
Take $\tau=\mathcal{P}(X)-\{\emptyset\}$, $\alpha$ the complex generated by $\{\alpha_S\}$, 
$\beta$ the complex generated by $\{\beta_S\}$ and $\sigma$ the complex generated by the $\{\sigma_{_{S_1,S_2}}\}$, 
$\mathcal{F}_d(St_{r,s})=\alpha\cup\beta\cup\sigma\cup\tau$.
Now, these four complexes are contractible and so are $\alpha\cap\sigma,\beta\cap\sigma,\alpha\cap\tau,\beta\cap\tau$. 
Also
$$\alpha\cap\beta\cap\sigma\cap\tau=\alpha\cap\sigma\cap\tau=\beta\cap\sigma\cap\tau=\alpha\cap\beta\cap\sigma=\sigma\cap\tau\cong 
sk_{d-2}\Delta^{r-1}*sk_{d-2}\Delta^{s-1}$$
and $\alpha\cap\beta\cap\tau=\alpha\cap\beta$. We compute the homotopy colimit of the punctured $4$-cube given by this union using the recursive formula given in the preliminaries. This what the formula gives applied to the top and bottom of the $4$-cube:

\begin{equation*}
\xymatrix{
\alpha\cap\beta\cap\sigma\cap\tau \ar@{->}[rr] \ar@{->}[dr] \ar@{->}[dd] & & \alpha\cap\beta\cap\sigma \ar@{->}[dd] \ar@{->}[rd] & & \\
 & \alpha\cap\sigma\cap\tau \ar@{->}[rr] \ar@{->}[dd] & & \sigma\cap\tau \ar@{->}[r] \ar@{->}[dd] & \alpha\cap\sigma \ar@{->}[dd] \\
\beta\cap\sigma\cap\tau \ar@{->}[rr] \ar@{->}[dr]^{\cong} & & \beta\cap\sigma \ar@{->}[dr]^{\simeq} &  & \\
  & \sigma\cap\tau \ar@{->}[rr] &  & \ast \ar@{->}[r] & \Sigma(\sigma\cap\tau)
}
\end{equation*}

\begin{equation*}
\xymatrix{
\alpha\cap\beta\cap\tau \ar@{->}[rr]^{\cong} \ar@{->}[dr] \ar@{->}[dd] & & \alpha\cap\beta \ar@{->}[dd] \ar@{->}[rd] & & \\
 & \alpha\cap\tau \ar@{->}[rr]^{\simeq} \ar@{->}[dd] & & \ast \ar@{->}[r] \ar@{->}[dd] & \alpha \ar@{->}[dd] \\
\beta\cap\tau \ar@{->}[rr] \ar@{->}[dr] & & \beta \ar@{->}[dr] &  & \\
  & \tau \ar@{->}[rr] &  & \ast \ar@{->}[r] & \ast
}
\end{equation*}
We find that the complex has the homotopy type of the following homotopy pushout:
$$\mathcal{S}\colon \ast\longleftarrow\Sigma(\sigma\cap\tau)\longrightarrow\tau$$
$$\hocolim(\mathcal{S})\simeq\Sigma^2(\sigma\cap\tau)\simeq\bigvee_{{r-1\choose{d-1}}{s-1\choose{d-1}}}\mathbb{S}^{2d-1}.$$
\end{proof}

\subsubsection{Cactus graphs}

For any graph $G$, we take the block graph $B(G)$ in which the vertices are the blocks of $G$ and the cut-vertices of $G$, where 
$vB$ is an edge if $v$ is a vertex of $B$. If $G$ is connected, then $B(G)$ is a tree.

A graph $G$ is a cactus graph if all of its blocks are isomorphic to a cycle or to $K_2$. We will say that 
a block is saturated if all of its vertices are cut vertices and $sb(G)$ is the number of saturated blocks. A 
vertex $v$ is saturated if it is shared by two or more saturated blocks, with $sv(G)$ the number of saturated vertices.

\begin{lem}\label{lemblocvertsatu}
Let $G$ be a cactus graph such that $sb(G)\geq1$, then there is a block $B$ such that either it does not have saturated vertices, or:
\begin{itemize}
    \item[(i)] it has only one saturated vertex $v$, and
    \item[(ii)] the connected component of $B(G)-v$ which contains $B$ does not have 
    any other saturated block.
\end{itemize}
\end{lem}
\begin{proof}
If there are no saturated vertices, there is nothing to prove. Assume $sv(G)\geq1$. If there is a saturated block 
without a saturated vertex, again there is nothing to prove. Assume all saturated blocks have at least one saturated vertex. 

Let $V_1$ be the set of all saturated blocks of $G$ and $V_2$ the set of all saturated vertices. In the subgraph  
$T=B(G)[V_1\cup V_2]$ all the leaves are blocks, because each saturated vertex is in at least two saturated blocks, therefore 
$d_T(v)\geq2$ for all the vertices of $V_2$. We take $L\subseteq V_1$ the set of all the leaves of $T$ and let $(B_1,B_2)$ be a pair 
in $L\times L$ such that 
$$d(B_1,B_2)=\max\{d(X,Y)\colon \;(X,Y)\in L\times L\}$$
Take $v_1$ the only saturated vertex in $B_1$ and $v_2$ the only saturated vertex in $B_2$. We claim that the only 
$B_1B_2$-path in $B(G)$ contains both $v_1$ and $v_2$. If not, then $B_1$ and $B_2$ are in different connected components of $T$ and, 
assuming $v_1$ is not in the $B_1B_2$-path, any leaf $B'$ in the same component of $B_1$ is such that 
$d(B',B_2)>d(B_1,B_2)$. Therefore $v_1$ and $v_2$ are in the only $B_1B_2$-path.

If in $B(G)-v_1$ there are saturated blocks in the same component than $B_1$, the distance between these and $B_2$ is larger that the 
distance between $B_1$ and $B_2$, which can not happen. Therefore $B_1$ and $v_1$ are as wanted. 
\end{proof}

\begin{lem}\label{homotdual}
Let $G$ be a cactus graph such that  all of its blocks are cycles and such that it does not have saturated blocks, then 
$$\mathcal{F}_\infty^*(G)\simeq\mathbb{S}^{b(G)-2}.$$
\end{lem}
\begin{proof}
Let $B_0,\dots,B_k$ be the blocks of $G$. If $k=0$, then $\mathcal{F}_\infty^*(G)=\emptyset=\mathbb{S}^{-1}$. Assume,
$k\geq1$. We take $X_i=V(G)-V(B_i)$ for all $i$, this are the facets of $\mathcal{F}_\infty^*(G)$ and we have that
$$\bigcap_{i=0}^kX_i=\emptyset$$
$$\bigcap_{i\in S}X_i\neq\emptyset, \;\; \forall S\subsetneq [k]$$
Then, its nerve is isomorphic to $\partial\Delta^k\cong\mathbb{S}^{k-1}$. Therefore,
$\mathcal{F}_\infty^*(G)\simeq\mathbb{S}^{b(G)-1}$.
\end{proof}

\begin{lem}\label{cactussimpcon}
Let $G$ be a cactus graph different from $K_3$, then $\mathcal{F}_\infty(G)$ is simply connected.
\end{lem}
\begin{proof}
If $G$ has only one block and $G$ is not $K_3$, $G$ must be a single vertex, $K_2$ or a cycle with at least $4$ vertices, thus 
$\mathcal{F}_\infty(G)$ is contractible or a sphere of dimension at least $2$. Assume $G$ has $k\geq2$ blocks.
For each block that is not isomorphic to $K_2$ we can erase one edge to we obtain $T$, a spanning tree of 
$G$ and $\mathcal{F}_\infty(G)$. Taking the free group $H_T$ with $E(G)\cup E(G^c)$ as generators and wtih the relations  
\begin{itemize}
    \item $uv=1$ for all the edges of $T$
    \item $(uv)(vw)=uw$ if $\{u,v,w\}$ is a simplex of $\mathcal{F}_\infty(G)$
\end{itemize}
we have that $H_T\cong\pi_1\left(\mathcal{F}_\infty(G)\right)$ (see \citep{rotmantop} Theorem 7.34). Take $uv\in E(G)\cup E(G^c)-E(T)$. 

If $u,v$ are in the same block, this block must be a cycle. If the cycle has $4$ or more vertices, there is a $uv$-path 
$uw_1w_2\cdots w_rv$  in $T$. Now, $\{u,w_1,v\},\{w_1.w_2,v\},\dots,\{w_{r-1},w_r,v\}$ are simplicies of $\mathcal{F}_\infty(G)$, then
$uv=w_1v=w_2v=\cdots=w_lv=1$. If the cycle is $uvw$, because there are $k\geq2$ blocks, one of the vertices must be a cut 
vertex:
\begin{itemize}
    \item If $u$ is a cut vertex, $u$ has a neighbor $x$ in another block such that $ux$ is in $T$. Then 
    $\{u,v,x\}$ is a simplex of $\mathcal{F}_\infty(G)$ and $uv=xv$. Now, $\{v,w,x\}$ and $\{u,w,x\}$ are simplices, thus 
    $xv=xw=uw=1$. The case in which $v$ is a cut vertex is analogous.
    \item If $w$ is a cut vertex, $w$ has a neighbor $x$ in another block such that $wx$ is in $T$. Then 
    $\{u,v,x\}$, $\{u,w,x\}$ and $\{v,w,x\}$ are simplices. Therefore $xv=vw=1=uw=ux$ and $uv=ux=1$.
\end{itemize}

If $u,v$ are in different blocks, then there are cut vertices $w_1,\dots,w_r$, with $r\geq1$, such that they are on the only $uv$-path in 
$T$ and $w_j$ it is not in the only $uw_i$-path for any $j>i$, and there are no more cut vertices in the path. Then 
$\{u,w_1,v\},\{w_1,w_2,v\},\dots,\{w_{r-1},w_r,v\}$ are simplices and 
$uv=w_1v=w_2v=\cdots=w_rv=1$.

Therefore $\pi_1(\mathcal{F}_\infty(G))\cong H_T\cong0$.
\end{proof}


\begin{cor}\label{cornoblocsat}
Let $G$ a cactus graph such all of its blocks are cycles and does not have saturated blocks, then 
$$\mathcal{F}_\infty(G)\simeq\mathbb{S}^{n-b(G)-1}.$$
\end{cor}
\begin{proof}
If $b(G)=1$, then $G$ is a cycle and $\mathcal{F}_\infty(G)\cong\mathbb{S}^{n-2}$. Assume $b(G)\geq2$, then, by Lemma \ref{cactussimpcon}, 
$\mathcal{F}_\infty(G)$ is simply connected and , by Lemma  \ref{homotdual}, $\mathcal{F}_\infty^*(G)\simeq\mathbb{S}^{b(G)-1}$. 
Therefore, by Theorem \ref{dualidadalexander}, $\mathcal{F}_\infty(G)$ is a simply connected complex such that its only nontrivial 
reduced homology group is in dimension $q=n-b(G)-1$, which is isomorphic to $\mathbb{Z}$. By Theorem \ref{cwsimplcon}, 
$\mathcal{F}_\infty(G)$ is homotopy equivalent to a sphere of the desired dimension.
\end{proof}

\begin{theorem}
If $G$ is a cactus graph then $\mathcal{F}_\infty(G)$ is either contractible or homotopy equivalent to a sphere of dimension at least 
$n-b(G)-1$.
\end{theorem}
\begin{proof}
If $\delta(G)=1$, then $\mathcal{F}_\infty(G)\simeq*$. Assume $\delta(G)=2$. If there is a cut vertex of degree $2$, then 
$\mathcal{F}_\infty(G)\simeq*$. Assume there is no cut vertex of degree $2$. If $G$ has a bridge $e$, then $G-e=G_1+G_2$ and, 
by Proposition \ref{proppuente}, 
$\mathcal{F}_\infty(G)=\mathcal{F}_\infty(G_1)*\mathcal{F}_\infty(G_2)$. If $G$ has more bridges, then we continue this 
process until we get that $\mathcal{F}_\infty(G)=\mathcal{F}_\infty(H_1)*\cdots*\mathcal{F}_\infty(H_{r+1})$, where $r$ 
is the number of bridges and each $H_i$ is a cactus graph such 
that every block is a cycle. So, if every $\mathcal{F}_\infty(H_i)$ has $n_i$ vertices, is not contractible and is homotopy equivalent to a 
sphere of dimension at least $n_i-b(H_i)-1$, $\mathcal{F}_\infty(G)$ will be homotopy equivalent to a sphere of dimension at least 
$n-b(G)+r-1>n-b(G)-1$. Therefore we only need to prove the result for cactus graphs which do not have blocks 
isomorphic to $K_2$. 

If $G$ does not have saturated blocks, by Corollary \ref{cornoblocsat}, 
$$\mathcal{F}_\infty(G)\simeq\mathbb{S}^{n-b(G)-1}.$$
So assume $sb(G)\geq1$, which implies that $b(G)\geq4$. 
Now, we prove the result by induction on $sv(G)$. If $sv(G)=0$, then take $B_0$ a saturated block of 
$G$ and $B_1,\dots,B_k$ the remaining blocks. Let $X_i=V(G)-V(B_i)$, then $X_0,X_1,\dots,X_k$ are the facets of 
$\mathcal{F}_\infty^*(G)$. Because $B_0$ is saturated, 
$$\bigcap_{i=1}^kX_i=\emptyset.$$
Let $S\subseteq [k]-\{0\}$ such that 
$$\sigma=\bigcap_{i\in S}X_i\neq\emptyset.$$
Then there is $0<j\leq k$ such that $j\notin S$ and $V(B_j)\cap\sigma\neq\emptyset$, with $B_j$ a non-saturated block or 
a saturated block (which can not share vertices with $B_0$).
Then there is a vertex $v$ in $V(B_j)$ such that $v$ is not vertex of the blocks with index in 
$S$ nor is a vertex of $B_0$, therefore $v\in X_0$, $v\in\sigma$ and 
$X_0\cap\sigma\neq\emptyset$. From this we get that the nerve is a cone with apex vertex $X_0$ and $\mathcal{F}_\infty^*(G)\simeq*$. 
Then, by Lemma \ref{cactussimpcon} and Theorem \ref{dualidadalexander}, 
$\mathcal{F}_\infty(G)$ is simply connected and all of its reduced homology groups are 
trivial. Therefore, by Theorem \ref{whiteheadhomologia}, $\mathcal{F}_\infty(G)$ is contractible.
This argument only used that there is an isolated saturated block, a saturated block which does not 
have saturated vertices; therefore we can assume that there is no isolated saturated block.

Assume the result is true for $sv(G)\leq k$ and let $G$ be a cactus graph with $sv(G)=k+1$ and with all of its blocks isomorphic to cycles. 
By Lemma \ref{lemblocvertsatu} there is $B_0$ a saturated block such that only one of its vertices is a saturated vertex, say $v$, and in 
the connected component of $B(G)-v$ which contains $B_0$ there are no more saturated blocks. We call $G_1$ the subgraph formed by the blocks 
in this connected component, and $G_2$ the subgraph induced by the remaining blocks. Then $G=G_1\cup G_2$ and $G_1\cap G_2\cong K_1$. Now 
$$lk_{\mathcal{F}_\infty(G)}(v)=lk_{\mathcal{F}_\infty(G_1)}(v)*lk_{\mathcal{F}_\infty(G_2)}(v)$$
We will show that $lk_{\mathcal{F}_\infty(G_1)}(v)\simeq*$. There are two possibilities: 
\begin{enumerate}
    \item $B_0\cong C_3$. Then $V(B_0)=\{v,v_1,v_2\}$ and $G_1=H_1\cup B_0\cup H_2$, with 
    $V(H_1)\cap V(B_0)=\{v_1\}$, $V(H_2)\cap V(B_0)=\{v_2\}$ and $V(H_1)\cap V(H_2)=\emptyset$.  Then, by Lemma \ref{lemlinkvertrian}, 
    $lk_{\mathcal{F}_\infty(G_1)}(v)\simeq\hocolim(\mathcal{S})$ with $\mathcal{S}$ the diagram:
    $$\mathcal{F}_\infty(H_1)*\mathcal{F}_\infty(H_2-v_2)\longhookleftarrow \mathcal{F}_\infty(H_1-v_1)*\mathcal{F}_\infty(H_2-v_2)\longhookrightarrow \mathcal{F}_\infty(H_1-v_1)*\mathcal{F}_\infty(H_2)$$
    By construction, $G_1$ does not have saturated blocks, then $\delta(H_1-v_1)=1$ or it has a cut vertex of degree $2$. 
    Therefore $\mathcal{F}_\infty(H_1-v_1)\simeq*$. Analogously, $\mathcal{F}_\infty(H_2-v_2)\simeq*$. 
    From this, we get that $\hocolim(\mathcal{S})\simeq*$.
    \item $B_0\cong C_n$ with $n\geq4$. Let $v_1,v_2$ be the neighbors of $v$ in $B_0$ and take $H$ be the graph obtained
    from $G_1$ by erasing $v$ and adding the edge $v_1v_2$. Then 
    $$lk_{\mathcal{F}_\infty(G_1)}(v)=\mathcal{F}_\infty(H)\simeq*,$$
    because $\mathcal{F}_\infty(H)$ has only one saturated block.
\end{enumerate}
Therefore $lk_{\mathcal{F}_\infty(G)}(v)\simeq*$ and $\mathcal{F}_\infty(G)\simeq \mathcal{F}_\infty(G-v)$. 
If there is a non-saturated block which contains $v$, then 
$\delta(G-v)=1$ or there is a cut vertex of degree $2$, and therefore $\mathcal{F}_\infty(G)\simeq*$. Assume that there is no non-saturated 
block with $v$ among its vertices. Now, in $G-v$, all the remaining edges of the blocks that contain $v$ are bridges, so we can remove 
them, let $H$ be the graph thus obtained. If $B_0,B_1,\dots,B_{l-1}$ are the blocks that contain $v$, 
with $n_0,n_1,\dots,n_{l-1}$ their respective orders, then  $H=H_1+\cdots+H_r$
where 
$$r=\displaystyle\sum_{i=0}^{l-1}n_i-1.$$
By inductive hypothesis, each $\mathcal{F}_\infty(H_i)$ is contractible or is homotopy equivalent to a sphere of dimension at least 
$|V(H_i)|-b(H_i)-1$. Then, $\mathcal{F}_\infty(H)$ is contractible or it has the homotopy type of a sphere of dimension at least
$$r-1+\sum_{i=1}^r|V(H_i)|-b(H_i)-1=n-1-(b(G)-l)-1=n-b(G)+l-2>n-b(G)-1.$$
\end{proof}

\subsection{Graph operations}
\subsubsection{Join of graphs}

Given two graphs $G$ and $H$ with disjoint vertex sets, we define their join as the graph $G*H$ with 
$V(G*H)=V(G)\cup V(H)$ and 
$$E(G*H)=E(G)\cup E(H)\cup\{uv\colon \;u\in V(G)\mbox{ and }v\in V(H)\}.$$
It is well-known that $\mathcal{F}_0(G*H)=\mathcal{F}_0(G)\sqcup\mathcal{F}_0(H).$
\begin{lem}\label{lemjoingraf}
Let $G$ and $H$ graphs with disjoint vertex sets with orders $n_1$ and $n_2$ respectively. Then:
\begin{enumerate}
    \item $\displaystyle\mathcal{F}_1(G*H)\simeq\mathcal{F}_1(G)\vee\mathcal{F}_1(H)\vee\bigvee_{n_1n_2-1}\mathbb{S}^1$
    \item If $\mathcal{F}_0(G)$ and $\mathcal{F}_0(H)$ are connected. Then, for all $d\geq2$
$$\mathcal{F}_d(G*H)\simeq\left(\bigvee_{n_2-1}\Sigma sk_{_{d-1}}\mathcal{F}_0(G)\right)\vee\left(\bigvee_{n_1-1}\Sigma sk_{_{d-1}}\mathcal{F}_0(H)\right)\vee\left(\bigvee_{(n_1-1)(n_2-1)}\mathbb{S}^2\right)\vee A\vee B$$
with $A=\mathcal{F}_d(G)\cup C(sk_{_{d-1}} \mathcal{F}_0(G))$ and $B=\mathcal{F}_d(H)\cup C(sk_{_{d-1}} \mathcal{F}_0(H))$
\end{enumerate}
\end{lem}
\begin{proof}
For $d=1$,
$$\mathcal{F}_1(G*H)=\mathcal{F}_1(G)\cup\mathcal{F}_1(H)\cup K_{n_1,n_2}.$$
Now $\mathcal{F}_1(G)\cap\mathcal{F}_1(H)\cap K_{n_1,n_2}=\mathcal{F}_1(G)\cap\mathcal{F}_1(H)=\emptyset$, therefore 
$\mathcal{F}_1(G*H)$ is homotopy equivalent to the homotopy pushout of 
$$X\longleftarrow sk_0\mathcal{F}_1(G) \longrightarrow\mathcal{F}_1(G),$$
where $X$ is the homotopy pushout of 
$$\mathcal{F}_1(H)\longleftarrow sk_0\mathcal{F}_1(H)\longrightarrow K_{n_1,n_2}.$$
Thus 
$$X\simeq\mathcal{F}_1(H)\vee\bigvee_{n_1(n_2-1)}\mathbb{S}^1.$$
From this the result follows.

For $d\geq2$,
$$\mathcal{F}_d(G*H)=\mathcal{F}_d(G)\cup\mathcal{F}_d(H)\cup K_1\cup K_2, $$
with $K_1=\bigcup_{u\in V(H)}\{u\}*sk_{_{d-1}}\mathcal{F}_0(G)$ and $K_2=\bigcup_{u\in V(G)}\{u\}*sk_{_{d-1}}\mathcal{F}_0(H)$. 
Now:
$$K_1\cong\bigvee_{n_2-1}\Sigma sk_{_{d-1}}\mathcal{F}_0(G),$$
$$K_2\cong\bigvee_{n_1-1}\Sigma sk_{_{d-1}}\mathcal{F}_0(H).$$
Taking $L_1=\mathcal{F}_d(G)$ and $L_2=\mathcal{F}_d(H)$, we have that
$$L_1\cap L_2=\emptyset,\;K_1\cap L_1=sk_{_{d-1}}\mathcal{F}_0(G),\;K_2\cap L_2=sk_{_{d-1}}\mathcal{F}_0(H),\;K_1\cap K_2\cong K_{n,m},$$
$$L_1\cap K_1\cap K_2=L_1\cap K_2\cong\bigvee_{n_1-1}\mathbb{S}^0,$$
$$L_2\cap K_2\cap K_1=L_2\cap K_1\cong\bigvee_{n_2-1}\mathbb{S}^0.$$
Taking $X=K_1\cup L_1$ and $Y=K_2\cup L_2$, we have that $\mathcal{F}_d(G*H)=X\cup Y$ and 
$X\cap Y=\left(L_1\cap K_2\right)\cup\left(L_2\cap K_1\right)\cup\left(K_1\cap K_2\right)=K_1\cap K_2$.
Therefore $\mathcal{F}(G*H,d)\simeq \hocolim(\mathcal{S})$ with 
$$\mathcal{S}\colon \;X\longhookleftarrow K_{n,m}\longhookrightarrow Y$$
Now, the inclusion $i\colon K_{n,m}\longhookrightarrow X$ is really the inclusion 
$K_{n,m}\longhookrightarrow K_1$, which is null-homotopic, and therefore $i$ is null-homotopic. In the same way we see 
that the inclusion in $Y$ is null-homotopic and that
$$\mathcal{F}_d(G*H)\simeq X\vee Y\vee\bigvee_{_{(n_1-1)(n_2-1)}}\mathbb{S}^2.$$
Now, $K_1\cap L_1=sk_{_{d-1}}\mathcal{F}_0(G)$ and its inclusion in $K_1$ is null-homotopic, therefore we can compute the homotopy type of $X$ by pasting these two homotopy pushout squares:
\begin{equation*}
    \xymatrix{
    sk_{_{d-1}}\mathcal{F}_0(G) \ar@{->}[r] \ar@{->}[d] & \ast \ar@{->}[r] \ar@{->}[d] & K_1 \ar@{->}[d] \\
    L_1 \ar@{->}[r] & L_1\cup C(sk_{_{d-1}}\mathcal{F}_0(G)) \ar@{->}[r] & K_1\vee (L_1\cup C(sk_{_{d-1}}\mathcal{F}_0(G))) \simeq X
    }
\end{equation*}
Now $L_1\cup C(sk_{_{d-1}}\mathcal{F}_0(G))=A$. With an similar argument for $Y$ we arrive 
at the result.
\end{proof}

With the last lemma we can construct graphs for which $\mathcal{F}_\infty(G)$ is not homotopy equivalent to a wedge of spheres.
Let $K$ be a triangulation of the projective plane and let $H$ be the complement graph of the $1$-skeleton of the baricentric 
subdivision, then $\mathcal{F}_0(G)\cong K$ and $G=P_4*H$ is a graph such that $\mathcal{F}_d(G)$ has torsion for all $d\geq3$.

\begin{lem}\label{lemcono}
Let $G$ be a graph and take $d\geq1$, then
$$\mathcal{F}_d(K_1*G)\simeq\mathcal{F}_d(G)\cup C(sk_{_{d-1}}\mathcal{F}_0(G))$$
\end{lem}
\begin{proof}
The link of the apex vertex is $sk_{_{d-1}}\mathcal{F}_0(G)$, thus the homotopy pushout square
\begin{equation*}
    \xymatrix{
    sk_{_{d-1}}\mathcal{F}_0(G) \ar@{->}[r] \ar@{->}[d] & \ast  \ar@{->}[d]\\
    \mathcal{F}_d(G) \ar@{->}[r] & \mathcal{F}_d(G)\cup C(sk_{_{d-1}}\mathcal{F}_0(G)) 
    }
\end{equation*}
computes $\mathcal{F}(K_1*G,d)$.
\end{proof}

\begin{theorem}
For the complete bipartite graph we have that $\mathcal{F}_0(K_{n,m})\simeq\mathbb{S}^0$, 
$$\mathcal{F}_1(K_{n,m})\simeq\bigvee_{nm-1}\mathbb{S}^1,$$
$$\mathcal{F}_d(K_{n,m})\simeq\bigvee_{(n-1)(m-1)}\mathbb{S}^2\vee\bigvee_{n{{m-1}\choose{d}}+m{{n-1}\choose{d}}}\mathbb{S}^d,$$
for $\infty>d\geq2$ and 
$$\mathcal{F}_\infty(K_{n,m})\simeq\bigvee_{(n-1)(m-1)}\mathbb{S}^2.$$
\end{theorem}
\begin{proof}
If $d=0$ is clear. The case $d=1$ is a particular case of Lemma \ref{lemjoingraf}. 
For $d\geq2$, by Lemma \ref{lemjoingraf} 
$$\mathcal{F}_d(K_{n,m})\simeq\left(\bigvee_{m-1}\Sigma sk_{_{d-1}}\mathcal{F}_0(K_n^c)\right)\vee\left(\bigvee_{n-1}\Sigma sk_{_{d-1}}\mathcal{F}_0(K_m^c)\right)\vee\left(\bigvee_{(n-1)(m-1)}\mathbb{S}^2\right)\vee A\vee B$$
with $A=\mathcal{F}_d(K_n^c)\cup C(sk_{_{d-1}} \mathcal{F}_0(K_n^c))$ and $B=\mathcal{F}_d(K_m^c)\cup C(sk_{_{d-1}} \mathcal{F}_0(K_m^c))$.

Now, for all $d,k,r$, 
$$\mathcal{F}_d(K_k^c)\cong\Delta^{k-1},\;sk_r\mathcal{F}_d(K_k^c)\simeq\bigvee_{{k-1}\choose{r+1}}\mathbb{S}^r;$$
therefore
$$A\simeq\bigvee_{{n-1}\choose{d}}\mathbb{S}^d;\;B\simeq\bigvee_{{m-1}\choose{d}}\mathbb{S}^d,$$
from which we obtain the result. 
\end{proof}


\begin{cor}
Let $G_1,G_2,\dots,G_k$ be vertex disjoint graphs. For $d\geq1$, if $\mathcal{F}_d(G_i)\simeq *$ for all $i$, then
$$\mathcal{F}_d(G_1*G_2*\dots*G_k)\simeq\bigvee_{\frac{(k-1)(k-2)}{2}}\mathbb{S}^1\vee\bigvee_{i<j}\mathcal{F}_d(G_i*G_j)$$
\end{cor}
\begin{proof}
Let $V_i$ be the vertex set of $G_i$ and take $G=G_1*G_2*\dots*G_k$. 
If we take vertices from more than two sets of the partition, we will always have a cycle, and therefore each facet of the complex 
is contained in $V_i\cup V_j$ for some $i\neq j$. Then, taking $X_{ij}=\mathcal{F}_d(G\left[V_i\cup V_j\right])$ for 
$i<j$, we have that $\displaystyle \mathcal{F}_d(G)=\bigcup_{i<j}X_{ij}$ and we can define a bijection 
$\gamma\colon \{ij\colon \;i<j\}\longrightarrow E(K_k)$ such that the hypothesis of Lemma \ref{lemhomotopiagraf} are 
achieved. 
\end{proof}

As an immediate consequence we have the homotopy type for the multipartite graphs
\begin{cor}\label{multi}
For $d\geq1$,
$$\mathcal{F}_d(K_{n_1,\dots,n_k})\simeq\bigvee_{\frac{(k-1)(k-2)}{2}}\mathbb{S}^1\vee\bigvee_{i<j}\mathcal{F}_d(K_{n_i,n_j}).$$
\end{cor}

\begin{theorem}\label{kozlovciclos}\citep{kozlovdire}
$$\mathcal{F}_0(C_n)\simeq\left\lbrace\begin{array}{cc}
    \mathbb{S}^{r-1}\vee\mathbb{S}^{r-1} & \mbox{ if }n=3r \\
    \mathbb{S}^{r-1} & \mbox{ if }n=3r+1\\
    \mathbb{S}^{r} &\mbox{ if }n=3r+2
\end{array}
\right.$$
\end{theorem}

\begin{prop}
Let $W_{n+1}$ be the wheel on $n+1$ vertices, then 
$$\mathcal{F}_d(W_{n+1})\simeq\left\lbrace\begin{array}{cc}
\mathbb{S}^{3r-2}\vee\mathbb{S}^{r}\vee\mathbb{S}^{r}&\mbox{if }n=3r\\
\mathbb{S}^{3r-1}\vee\mathbb{S}^{r}&\mbox{if }n=3r+1\\
\mathbb{S}^{3r}\vee\mathbb{S}^{r+1}&\mbox{if }n=3r+2
\end{array}\right.$$
for $d>\lfloor\frac{n}{2}\rfloor-1$ and
$$\mathcal{F}_1(W_{n+1})\simeq\left\lbrace\begin{array}{cc}
    \displaystyle\bigvee_{3}\mathbb{S}^{2r-1}\vee\bigvee_{n-1}\mathbb{S}^1 & \mbox{ if } n=4r\\
    \displaystyle\mathbb{S}^{2r-1}\vee\bigvee_{n-1}\mathbb{S}^1 & \mbox{ if } n=4r+1\\
    \displaystyle\mathbb{S}^{2r}\vee\bigvee_{n-1}\mathbb{S}^1 & \mbox{ if } n=4r+2\\
    \displaystyle\mathbb{S}^{2r+1}\vee\bigvee_{n-1}\mathbb{S}^1 & \mbox{ if } n=4r+3\\
\end{array}\right.$$
\end{prop}
\begin{proof}
Since $\alpha(C_n)=\lfloor\frac{n}{2}\rfloor$, for $d>\lfloor\frac{n}{2}\rfloor-1$ we have that 
$\mathcal{F}_0(C_n)=sk_{_{d-1}}\mathcal{F}_0(C_n)$. By Lemma \ref{lemcono},
$$\mathcal{F}_d(W_{n+1})\simeq \mathcal{F}_d(C_n)\cup C(\mathcal{F}_0(C_n)).$$
By Theorem \ref{kozlovciclos}, the inclusion of the intersection is null-homotopic, therefore
$$\mathcal{F}_d(W_{n+1})\simeq \mathcal{F}_\infty(C_n)\vee\Sigma\mathcal{F}_0(C_n)$$
For $d=1$, $sk_0\mathcal{F}_{0}(C_n,0)=\bigvee_{n-1}\mathbb{S}^0$, the rest of the proof is the same as before.
\end{proof}


\subsubsection{Graph products}

\begin{prop}
$$\mathcal{F}_\infty\left(P_2\oblong P_k\right)\simeq\left\lbrace\begin{array}{cc}
    \mathbb{S}^{4r-1} & \mbox{if } k=3r \\
    * & \mbox{if } k=3r+1 \\
    \mathbb{S}^{4r+2} & \mbox{if } k=3r+2.
\end{array}
\right.$$
\end{prop}
\begin{proof}
By Theorem \ref{coneccuello}, $\mathcal{F}_\infty\left(P_2\oblong P_k\right)$ is simply connected. We will show 
that it has at most one non-trivial reduced homology group. The Alexander dual of 
$\mathcal{F}_\infty\left(P_2\oblong P_k\right)$ has as maximal simplicies the complements of $X_i=\{(i,1),(i+1,1),(i,2),(i+1,2)\}$ 
for $1\leq i\leq k-1$. Taking $U_i=X_i^c$ and $U$ the cover formed by these $U_i$, we have that
$$\mathcal{N}(U)\simeq\mathcal{F}_0^*(P_k).$$
It is standard that \citep{kozlovdire}:
$$\mathcal{F}_0(P_k)\simeq\left\lbrace\begin{array}{cc}
\mathbb{S}^{r-1} & \mbox{ if } k=3r \\
* & \mbox{ if } k=3r+1 \\
\mathbb{S}^{r} & \mbox{ if } k=3r+2.
\end{array}
\right.$$
Thus, by Theorem \ref{dualidadalexander}, $\mathcal{N}(U)$ has non-trivial reduced cohomology groups if $k=3r$ or $k=3r+2$, 
in which case the groups are in dimensions  
are $2(r-1)$ and $2r-1$ respectively. Therefore $\mathcal{F}_\infty\left(P_2\oblong P_k\right)$ is contractible if 
$k=3r+1$ and 
$$\tilde{H}_{q}\left(\mathcal{F}_\infty\left(P_2\oblong P_{3r}\right)\right)\cong\left\lbrace\begin{array}{cc}
\mathbb{Z} & \mbox{ if } q=4r-1 \\
0 &  \mbox{ if } q\neq4r-1,
\end{array}
\right.$$
$$\tilde{H}_{q}\left(\mathcal{F}_\infty\left(P_2\oblong P_{3r+2}\right)\right)\cong\left\lbrace\begin{array}{cc}
\mathbb{Z} & \mbox{ if } q=4r+2 \\
0 &  \mbox{ if } q\neq4r+2.
\end{array}
\right.$$
By Theorem \ref{cwsimplcon}, in these cases the complex is homotopy equivalent to a sphere of the desired dimension.
\end{proof}

It is known \cite{homotopygoyal} that 

$$\mathcal{F}_0(K_n\times K_m)\simeq\bigvee_{(n-1)(m-1)}\mathbb{S}^1.$$
Now we will show what happens for $d\geq1$.
\begin{prop}
$$\mathcal{F}_1(K_n\times K_m)\simeq\bigvee_{\frac{(nm-4)(n-1)(m-1)}{4}}\mathbb{S}^2$$
\end{prop}
\begin{proof}
We take $V(K_r)=[r]-\{0\}$ for any $r$.
We proceed by induction on $n$. For $n=1$, the result is clear. 
For $n=2$ we will prove it by induction on $m$. For $m=1,2$ it is clear and for $m=3$, $K_2\times K_3\cong C_6$. 
Taking $v_i=(1,i)$ and $u_i=(2,i)$, we have that
$lk(v_m)=X\cup Y$, where 
$Y=\mathcal{F}_1(K_2\times K_m)-N[v_n]$ and $X$ is the complex with facets $\{u_i,v_i,u_m\}$ for $i\geq m-1$. Then $X\simeq*$, as it is a cone with apex $u_m$, and $X\cap Y\cong K_{i,m}\simeq*$. Therefore, 
$$lk(v_n)\simeq Y\cong\mathcal{F}_1(K_{1,m-1})\simeq\bigvee_{m-2}\mathbb{S}^1.$$
Taking $H=K_2\times K_m-v_m$, the link of $u_m$ in $\mathcal{F}_1(H)$ has as facets the simplex $\{u_1,\dots,u_{m-1}\}$ and 
the edges $\{u_i,v_i\}$ for $i\geq m-1$, therefore it is contractible and 
$$\mathcal{F}_1(H)\simeq\mathcal{F}_1(H-u_m)\cong\mathcal{F}_1(K_2\times K_{m-1})\simeq\bigvee_{\frac{(m-2)(m-3)}{2}}\mathbb{S}^2,$$
from which the result follows. 

Now assume the result is true for $K_r\times K_m$ for all $r\leq n-1$. Take 
$v_i=(n,i)$, $G_0=K_n\times K_m$, $G_i=G_{i-1}-v_i$ for $i\geq1$,
$X_{j,k}^i=|\{(j,k),(j,i),(n,k)\}|$ for $k\geq i+1$ and $j\leq n-1$, 
$X_{j,k}^i=|\{(j,k),(j,i)\}|$ for $k\leq i-1$ and $j\leq n-1$,
$$X^i=\bigcup_{k\neq i,\;j\leq n-1}X_{j,k}^i$$
and $Y^i=\mathcal{F}_1(G_{i-1}-N[v_i])$. Then, taking $L_{i}$ the link of $v_{i}$ in 
$\mathcal{F}_1(G_{i-1})$, we have that
$$L_i=X^i\cup Y^i.$$
Now, in $X^i$, the vertices $(j,k)$ with $j\leq n-1$ and $k\neq i$ are only in one facet and can be erased, therefore 
$X^i$ is homotopy equivalent to the subcomplex with maximal facets $\{(j,i),(n,k)\}$ with $k\geq i+1$ and $j\leq n-1$, which is isomorphic 
to $K_{n-1,m-i}$. Because $X^i\cap Y_i$ is isomorphic to this subacomplex, we have that 
$$L_i\simeq Y^i\cong \mathcal{F}_1(K_{n-i,m-1})\simeq\bigvee_{(m-1)(n-i)-1}\mathbb{S}^1$$
for $i\leq n-1$. Now, $L_n\simeq Y^n\simeq*$, therefore
$$\mathcal{F}_1(G_{n-1})\simeq\mathcal{F}_1(G_n)\cong\mathcal{F}_1(K_{n-1}\times K_{m})\simeq\bigvee_{\frac{((n-1)m-4)(n-2)(m-1)}{4}}\mathbb{S}^2.$$
From this we have that 
$$\mathcal{F}_1(G_0)\simeq\mathcal{F}_1(K_{n-1}\times K_{m})\vee\Sigma Y^1\vee\Sigma Y^2\vee\cdots\vee\Sigma Y^{n-1}.$$
Now $\Sigma Y^1\vee\Sigma Y^2\vee\cdots\vee\Sigma Y^{n-1}$ is homotopy equivalent to the wedge of
$$\sum_{i=1}^{m-1}i(n-1)-1=\frac{(n-1)m(m-1)}{2}-(m-1)$$
copies of the $2$-sphere. Since
$$\frac{((n-1)m-4)(n-2)(m-1)}{4}=\sum_{i=1}^{n-2}\frac{im(m-1)}{2}-(m-1),$$
we have that $\mathcal{F}_1(K_n\times K_m)$ is homotopy equivalent to the wedge of 
$$\sum_{i=1}^{n-1}\frac{im(m-1)}{2}-(m-1)=\frac{(nm-4)(n-1)(m-1)}{4}$$
$2$-spheres.
\end{proof}

% Figure environment removed

 \begin{lem}
 For $d\geq2$, $\mathcal{F}_{d+1}(K_2\times K_n)\simeq\mathcal{F}_d(K_2\times K_n)$
 \end{lem}
\begin{proof}
We know that $\mathcal{F}_d(K_2\times K_n)$ is simply connected for all $d\geq2$, because 
$\mathcal{F}_1(K_2\times K_n)$ is a wedge of $2$-spheres.
We will show that $H_q(\mathcal{F}_{d+1}(K_2\times K_n),\mathcal{F}_{d}(K_2\times K_n))\cong0$ for all $q$. 
We know that $H_q(\mathcal{F}_{d+1}(K_2\times K_n),\mathcal{F}_{d}(K_2\times K_n))\cong0$ for all $q\leq d$. 
For $q\geq d+3$, for any $q$-simplex $\sigma$ of $\mathcal{F}_{d+1}(K_2\times K_n)$, we can partition its 
vertices in two sets $V_1,V_2$ such that all the vertices in $V_i$ are of the form $(i,j)$ for some $j$. Next we show that
$|V_1|=0$ or $|V_2|=0$. If not, we can assume that
$$|V_1|\leq\left\lfloor\frac{d+3}{2}\right\rfloor\leq\left\lceil\frac{d+3}{2}\right\rceil\leq|V_2|$$
therefore $|V_2|\geq3$; there are several cases: 
\begin{itemize}
    \item If $|V_1|=1$, then $|V_2|\geq d+3$ and the vertex of $V_1$ has degree at least $d+2$, which can not 
    happen.
    \item If $|V_1|=2$, then $|V_2|\geq d+2$ and there will be at least two vertices of $V_2$ such their second 
    coordinates are different from those of the vertices of $V_1$; therefore there will be an induced $4$-cycle 
    in the vertices of $\sigma$, which can not happen. 
    \item If $|V_1|\geq3$, because $|V_2|\geq3$, there will be an induced $4$-cycle or an induced 
    $6$-cycle in the vertices of $\sigma$, which can not happen.
\end{itemize}
Therefore $|V_1|=0$ or $|V_2|=0$ and $\sigma$ is a simplex of $\mathcal{F}_{d}(K_2\times K_n)$. From this, 
we have that $H_q(\mathcal{F}_{d+1}(K_2\times K_n),\mathcal{F}_{d}(K_2\times K_n))\cong0$ for all $q\geq d+3$.

For $q=d+2$, the only $q$-simplices of ${F}_{d+1}(K_2\times K_n)$ which are not simplices of $\mathcal{F}_{d}(K_2\times K_n)$
are of the form $|V_1|=1$ and $|V_2|=d+2$ (or vice versa), where the only vertex of $V_1$ is adjacent to all but one vertex of 
$V_2$ (Figure \ref{simd2}). For $q=d+1$, the only $q$-simplices of ${F}_{d+1}(K_2\times K_n)$ which are not simplices 
of $\mathcal{F}_{d}(K_2\times K_n)$ are of the form $|V_1|=1$ and $|V_2|=d+1$ (or vice versa), where the only vertex of $V_1$ is adjacent 
to all the vertices of $V_2$ (Figure \ref{simd1}). From all this, we get that there are no relative $d+2$-cycles and 
that all of the relative $d+1$-cycles are images of some relative $d+2$-boundary. Therefore the remaining two 
relative homology groups are also trivial. 

From all this we have that the inclusion $\mathcal{F}_{d+1}(K_2\times K_n)\longhookrightarrow\mathcal{F}_{d}(K_2\times K_n)$
induces an isomorphism for all homology groups between simply connected complexes, therefore 
$\mathcal{F}_{d+1}(K_2\times K_n)\simeq\mathcal{F}_{d}(K_2\times K_n)$.
\end{proof}

\begin{prop}
For $d\geq2$,
$$\mathcal{F}_d(K_2\times K_n)\simeq\bigvee_{n\choose{3}}\mathbb{S}^4\vee\bigvee_{n-1\choose{3}}\mathbb{S}^3.$$
\end{prop}
\begin{proof}
We only have to prove it for $d=2$.
The result is clear for $n=1,2,3$. Assume $n\geq4$. Taking 
$\displaystyle k={n\choose{3}}$, let $X_1,\dots,X_k$ be the subcomplexes of $\mathcal{F}_2(K_2\times K_n)$ corresponding 
to all the induced $6$-cycles. Then $X_i\cong\mathbb{S}^4$. The other facets of $\mathcal{F}_2(K_2\times K_n)$, besides the ones 
in some $X_i$, are $\{1\}\times\underline{n}$ and $\{2\}\times\underline{n}$. Then 
$$\mathcal{F}_2(K_2\times K_n)=X_1\cup X_2\cup\cdots\cup X_k\cup Y_1\cup Y_2$$
where $Y_1=\mathcal{P}(\{1\}\times\underline{n})-\{\emptyset\}$ and 
$Y_2=\mathcal{P}(\{2\}\times\underline{n})-\{\emptyset\}$. Now we will calculate the homology of $\mathcal{F}_2(K_2\times K_n)$ 
using the Mayer-Vietoris spectral sequence. Taking $U=\{X_1,X_2,\dots,X_k,Y_1,Y_2\}$ and $\mathcal{U}=\mathcal{N}(U)$, 
the first page of the sequence is 
\begin{equation*}
    \xymatrix{
    \mathbb{Z}^k \ar@{<-}[r] & 0 \ar@{<-}[r] & 0 \ar@{<-}[r] & 0 \ar@{<-}[r] & 0\\
    0 \ar@{<-}[r] &  0 \ar@{<-}[r] & 0 \ar@{<-}[r] & 0 \ar@{<-}[r] & 0\\
    0 \ar@{<-}[r] &  0 \ar@{<-}[r] & 0 \ar@{<-}[r] & 0 \ar@{<-}[r] & 0\\
    0 \ar@{<-}[r] &  0 \ar@{<-}[r] & 0 \ar@{<-}[r] & 0 \ar@{<-}[r] & 0\\
    C_0(\mathcal{U}) \ar@{<-}[r] & C_1(\mathcal{U}) \ar@{<-}[r] & C_2(\mathcal{U}) \ar@{<-}[r] & C_3(\mathcal{U}) \ar@{<-}[r] & 0
    }
\end{equation*}
Because the nerve of $X_1,X_2,\dots,X_k$ is isomorphic to the nerve of $2$-simplices of $sk_2\Delta^{n-1}$, 
and $\mathcal{U}$ is isomorphic to the suspension of this nerve, we have that the second page is
\begin{equation*}
    \xymatrix{
    \mathbb{Z}^k \ar@{<-}[rrd] & 0 \ar@{<-}[rrd] & 0 \ar@{<-}[rrd] & 0  & 0\\
    0 \ar@{<-}[rrd] &  0 \ar@{<-}[rrd] & 0 \ar@{<-}[rrd] & 0  & 0\\
    0 \ar@{<-}[rrd] &  0 \ar@{<-}[rrd] & 0 \ar@{<-}[rrd] & 0  & 0\\
    0 \ar@{<-}[rrd] &  0 \ar@{<-}[rrd] & 0 \ar@{<-}[rrd] & 0 & 0\\
    \mathbb{Z}  & 0 & 0 & \mathbb{Z}^{r} & 0
    }
\end{equation*}
where $r={n-1\choose{3}}$. From this we have that $E_{p,q}^\infty=E_{p,q}^2$. Therefore 
$$\tilde{H}_q(\mathcal{F}_2(K_2\times K_n))\cong\left\lbrace
\begin{array}{cc}
    \mathbb{Z}^k & \mbox{ if } q=4 \\
    \mathbb{Z}^r & \mbox{ if } q=3 \\
    0 & \mbox{ if } q\neq4,3 \\
\end{array}\right.$$
Therefore, because  $\mathcal{F}_1(K_2\times K_n)$ is simply connected, $\mathcal{F}_2(K_2\times K_n)$ is a simply connected complex which 
satisfies the hypothesis of Theorem \ref{gradconse}, from which we see that is has the desired homotopy type.
\end{proof}

\begin{theorem}
For $d\geq2$,
$$\mathcal{F}_d(K_n\times K_m)\simeq\bigvee_a\mathbb{S}^4\vee\bigvee_{b+c}\mathbb{S}^3,$$
where $a={m\choose{2}}{n\choose{3}}+{n\choose{2}}{m\choose{3}}$, 
$b={m\choose{2}}{n-1\choose{3}}+{n\choose{2}}{m-1\choose{3}}$ and $c={n-1\choose{2}}{m-1\choose{2}}$.
\end{theorem}
\begin{proof}
In $\mathcal{F}_d(K_n\times K_m)$ the facets have their vertices contained in two rows or two columns, otherwise they will have 
a cycle. Then, taking the subgraphs
$$H_{i,j}=K_n\times K_m[\{(k,l)\colon\;l=i\mbox{ or }l=j\}],$$
$$G_{i,j}=K_n\times K_m[\{(k,l)\colon\;k=i\mbox{ or }k=j\}],$$
and the complexes $X_{i,j}=\mathcal{F}_d(H_{i,j})$, $Y_{i,j}=\mathcal{F}_d(G_{i,j})$, we have that
$$\mathcal{F}_d(K_n\times K_m)=\bigcup_{e\in E(K_m)}X_e\cup\bigcup_{e\in E(K_n)}Y_e$$
From the last Proposition we know that  
$$X_e\simeq\bigvee_{n\choose{3}}\mathbb{S}^4\vee\bigvee_{n-1\choose{3}}\mathbb{S}^3$$
$$Y_e\simeq\bigvee_{m\choose{3}}\mathbb{S}^4\vee\bigvee_{m-1\choose{3}}\mathbb{S}^3$$
Taking the Mayer-Vietoris spectral sequence, the first 
page looks like
\begin{equation*}
    \xymatrix{
    \mathbb{Z}^a \ar@{<-}[r] & 0 \ar@{<-}[r] & 0 \ar@{<-}[r] & 0 \ar@{<-}[r] & 0\\
    \mathbb{Z}^b \ar@{<-}[r] &  0 \ar@{<-}[r] & 0 \ar@{<-}[r] & 0 \ar@{<-}[r] & 0\\
    0 \ar@{<-}[r] &  0 \ar@{<-}[r] & 0 \ar@{<-}[r] & 0 \ar@{<-}[r] & 0\\
    0 \ar@{<-}[r] &  0 \ar@{<-}[r] & 0 \ar@{<-}[r] & 0 \ar@{<-}[r] & 0\\
    C_0(\mathcal{U}) \ar@{<-}[r] & C_2(\mathcal{U}) \ar@{<-}[r] & C_3(\mathcal{U}) \ar@{<-}[r] & C_4(\mathcal{U}) \ar@{<-}[r] & 0
    }
\end{equation*}
Where $\mathcal{U}$ is the nerve of the cover, 
$a={n\choose{2}}{m\choose{3}}+{m\choose{2}}{n\choose{3}}$ and $b={n\choose{2}}{m-1\choose{3}}+{m\choose{2}}{n-1\choose{3}}$. 
Now, $\mathcal{U}$ is isomorphic to the join of the nerve of the $X'$s with the nerve of the $Y'$s, which are 
homotopy equivalent to $K_m$ and $K_n$ respectively, therefore $\mathcal{U}\simeq\bigvee_{c}\mathbb{S}^3$ with 
$c={n-1\choose{2}}{m-1\choose{2}}$. From all this, we have that the second page of the sequence is
\begin{equation*}
    \xymatrix{
    \mathbb{Z}^a \ar@{<-}[rrd] & 0 \ar@{<-}[rrd] & 0 \ar@{<-}[rrd] & 0  & 0\\
    \mathbb{Z}^b \ar@{<-}[rrd] &  0 \ar@{<-}[rrd] & 0 \ar@{<-}[rrd] & 0  & 0\\
    0 \ar@{<-}[rrd] &  0 \ar@{<-}[rrd] & 0 \ar@{<-}[rrd] & 0  & 0\\
    0 \ar@{<-}[rrd] &  0 \ar@{<-}[rrd] & 0 \ar@{<-}[rrd] & 0 & 0\\
    \mathbb{Z}  & 0 & 0 & \mathbb{Z}^c & 0
    }
\end{equation*}
Therefore $E_{p,q}^\infty=E_{p,q}^2$ and
$$\tilde{H}_q(\mathcal{F}_d(K_n\times K_m))\cong\left\lbrace
\begin{array}{cc}
    \mathbb{Z}^a & \mbox{ if } q=4 \\
    \mathbb{Z}^{b+c} & \mbox{ if } q=3 \\
    0 & \mbox{ if } q\neq4,3 \\
\end{array}\right.$$
As in the proof of the last theorem, we have a simply connected complex which satisfies the hypothesis of Theorem \ref{gradconse}.
\end{proof}

In \citep{indcomplcartprod} it was shown that
$$\mathcal{F}_0(K_2\times K_m \times K_n)\simeq\bigvee_{\frac{(n-1)(m-1)(nm-2)}{2}}\mathbb{S}^3.$$
For other $d\geq1$, because $K_2\times K_2\cong K_2\sqcup K_2$ we have the following corollary
\begin{cor}
For $d\geq1$,
    $$\mathcal{F}_d(K_2\times K_2\times K_n)\simeq\left\lbrace\begin{array}{cc}
      \displaystyle\bigvee_{\frac{(n-2)^2(n-1)^2}{4}}\mathbb{S}^5  &  d=1\\
        \displaystyle\bigvee_{\binom{n}{3}^2}\mathbb{S}^9\vee\bigvee_{2\binom{n}{3}\binom{n-1}{3}}\mathbb{S}^8\vee\bigvee_{\binom{n-1}{3}^2}\mathbb{S}^7 & d\geq2
    \end{array}\right.$$
\end{cor}

\begin{que}
What is the homotopy type of $K_2\times K_m \times K_n$ for $d\geq1$?
\end{que}


\textbf{Acknowledgments.} The author wishes to thank Omar Antol\'in-Camarena for his comments and suggestions which improved this paper. 

\bibliographystyle{acm}
\bibliography{complejobosques}

\vspace{1cm}

\hspace{0cm}Andr\'es Carnero Bravo

\hspace{0cm}Instituto de Matem\'aticas, UNAM, Mexico City, Mexico

\hspace{0cm}\textit{E-mail address:} \href{mailto:acarnerobravo@gmail.com}{acarnerobravo@gmail.com}
\end{document}