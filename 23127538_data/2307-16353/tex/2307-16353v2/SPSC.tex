\documentclass[11pt]{article}
\usepackage[margin=1in]{geometry}
\usepackage{amsmath}
\usepackage{graphicx}
\usepackage{enumerate}
\usepackage{enumitem}
\usepackage{natbib}
\usepackage{url} % not crucial - just used below for the URL 

%\pdfminorversion=4
% NOTE: To produce blinded version, replace "0" with "1" below.

\usepackage{algorithmic} %For computer algorithm code in LATEX
\usepackage{algorithm} %For algorithm box around the algorithmic
\usepackage{setspace} %For setting spacing (e.g., double spacing, use \doublespacing)
% \usepackage{apacite}
\usepackage{rotating} %For rotating tables (provides sidewaysfigure and sidewaystable)

\usepackage{amsfonts, amssymb} %Either provides Real number and other font'related stuff.
\usepackage{amsthm}
% \usepackage{amsthm}

\usepackage{epstopdf} %for adding eps in pdfLatex	
\usepackage{hyperref} %For adding hyperlinks in documents
\usepackage{bm} %Bolds everything when uses \bm as a command
\usepackage{bbm}
% \usepackage{authblk} %Have multiple a??uthors with multiple affiliations
\usepackage{multirow} %Have multiple rows in a table
\usepackage{latexsym} %Latex-based math symbols. Generally load this in conjunction with amssymb
% \usepackage{apacite} % APA cite (Chan added)
\usepackage{rotating} %arbitrary rotate any object
\usepackage{titlesec} %centering section headings
\usepackage{caption} %captioning for tables
\usepackage{mathtools} %ceiling/floor commands
\usepackage{color} %Provides color for text.
% \usepackage{mathabx} % widecheck
\usepackage{diagbox}
\usepackage{pict2e}
\usepackage{etoolbox}
\usepackage[english]{babel}
\usepackage{empheq}

\usepackage{cases}
\usepackage{tikz}
\usetikzlibrary{arrows, chains, positioning, quotes, shapes.geometric, arrows.meta}

\usepackage{lato}



% Here are theorem/proposition/corollary envirnoments
% Theorem-styles
\newtheorem{theorem}{Theorem}[section]
\newtheorem{corollary}{Corollary}[section]
\newtheorem{lemma}{Lemma}[section]
\newtheorem{proposition}{Proposition}[section]
\newtheorem{conjecture}{Conjecture}[section]
\newtheorem{condition}{Condition}[section]

% Definitions
\theoremstyle{definition}
\newtheorem{definition}{Definition}[section]
\newtheorem{assumption}{Assumption}[section]
\newtheorem{assumptionNew}{Assumption}[section]
\newtheorem{example}{Example}[section]

% Remarks
\theoremstyle{remark}
\newtheorem{remark}{Remark}[section]

\theoremstyle{definition}
\newtheorem{REG}{\textit{Regularity Condition}}
\newtheorem{GREG}{\textit{General Regularity Condition}}

% shorthands
\definecolor{mycolor}{RGB}{0,200,200} 

% Operators
\newcommand{\indep}{\,\rotatebox[origin=c]{90}{$\models$}\,}
\newcommand{\nindep}{\not \hspace{-0.125cm} \rotatebox[origin=c]{90}{$\models$}\,}
\DeclareMathOperator*{\argmax}{arg\,max}
\DeclareMathOperator*{\argmin}{arg\,min}
\DeclareMathOperator*{\median}{median}
\newcommand{\EXP}{\text{E}}
\newcommand{\VAR}{\text{var}}
\newcommand{\AVER}{\mathbbm{P}}
\newcommand{\EMP}{\mathbbm{G}}
\newcommand{\cond}{\, \big| \,}
\newcommand{\Cond}{\, \Big| \,}
\newcommand{\COND}{\, \bigg| \,}
\newcommand{\con}{ ; }
\newcommand{\R}{\mathbbm{R}}
\newcommand{\ind}{\mathbbm{1}}
% \newcommand{\T}{^\intercal}
% \newcommand{\sT}{^{*\intercal}}
\newcommand{\T}{^\mathrm{\scriptscriptstyle T}}
\newcommand{\sT}{^{*\mathrm{\scriptscriptstyle T}}}
\newcommand{\D}{\mathcal{D}}
\newcommand{\Z}{\mathcal{Z}}
\newcommand{\VEC}[1]{\texttt{vec}_{#1}}
\newcommand{\MAT}[1]{\texttt{mat}_{#1}}
\newcommand{\cT}{\mathcal{T}}

% Variables
\newcommand{\bX}{{X}}
\newcommand{\bx}{{x}}
\newcommand{\bW}{{W}}
\newcommand{\bw}{{w}}
\newcommand{\bG}{{G}}
\newcommand{\bg}{{g}}
\newcommand{\ba}{{a}}
\newcommand{\bO}{{O}}
\newcommand{\bo}{{o}}
\newcommand{\Yo}{Y_1}
\newcommand{\Yz}{Y_0}
\newcommand{\bmu}{{\mu}}
\newcommand{\blambda}{{\lambda}}
\newcommand{\bgamma}{{\gamma}}
\newcommand{\bdelta}{{\delta}}
\newcommand{\btheta}{{\theta}}
\newcommand{\bbeta}{{\beta}}
\newcommand{\bZ}{{Z}}
\newcommand{\bepsilon}{{\epsilon}}
\newcommand{\bzero}{{0}}
\newcommand{\boldr}{{r}}
\newcommand{\bphi}{{\phi}}

\newcommand{\bsY}{{\mathcal{Y}}}
\newcommand{\bsL}{{\mathcal{L}}}

\newcommand{\SC}{\text{SC}}

% Other Notations
\newcommand{\ATT}{\text{ATT}}

% Super/Subscripts 
\newcommand{\potY}[2]{Y_{#1}^{(#2)}}
\newcommand{\potW}[2]{W_{#1}^{(#2)}}
\newcommand{\obsX}[1]{X_{#1}}
\newcommand{\obsY}[1]{Y_{#1}}
\newcommand{\obsW}[1]{W_{#1}}
\newcommand{\vobsW}[1]{\overrightarrow{W}_{#1}}
\newcommand{\bobsW}[1]{{W}_{#1}}
\newcommand{\bobsw}[1]{{w}_{#1}}
\newcommand{\LSS}{^{(-k)}}

% hyperlink
\newcommand{\HL}[1]{\hyperlink{(#1)}{(#1)}}
\newcommand{\HT}[1]{\hypertarget{(#1)}{(#1)}}

% Text shortcuts
\newcommand{\PI}{\text{PSC}}

\newcommand{\pre}{\text{pre}}
\newcommand{\post}{\text{post}}

\definecolor{light-gray}{gray}{0.7}
\newcommand{\CP}[1]{{\color{blue} Chan: #1}}
\newcommand{\CPgray}[1]{{\color{light-gray} Chan: #1}}
\newcommand{\CPadd}[1]{{\color{red}#1}}
\newcommand{\ETT}[1]{{\color{red} Eric: #1}}










\hypersetup{
colorlinks=true,
linkcolor=blue,
filecolor=blue,
urlcolor=blue,
citecolor=blue
}

\graphicspath{ {plot/} }% plot path

\definecolor{red1}{RGB}{255,64,64}
\definecolor{blue1}{RGB}{128,255,255}
\definecolor{green1}{RGB}{0,205,0}




%\newtheorem{axiom}{Axiom}
%\newtheorem{claim}[axiom]{Claim}
%\newtheorem{theorem}{Theorem}[section]
%\newtheorem{lemma}[theorem]{Lemma}
%\newtheorem{proposition}[theorem]{Proposition}
%\newtheorem{corollary}[theorem]{Corollary}


%\newtheorem*{theorem*}{\theoremnumber}
%\providecommand{\theoremnumber}{}
%\makeatletter
%\providecommand{\theoremnumber}{}
%\newenvironment{selftheorem}[2]
% {%
% \renewcommand{\theoremnumber}{Theorem #1}%
% \begin{theorem*}[#2]%
% \protected@edef\@currentlabel{#1}%
% }
% {%
% \end{theorem*}
% }
%\makeatother

%\theoremstyle{definition}
%\newtheorem{assumption}{Assumption}
%\newtheorem{definition}{Definition}
%\newtheorem{remark}{Remark}

%\newtheorem*{assumption*}{\assumptionnumber}
%\providecommand{\assumptionnumber}{}
%\makeatletter
%\providecommand{\assumptionnumber}{}
%\newenvironment{selfassumption}[2]
% {%
% \renewcommand{\assumptionnumber}{Assumption #1}%
% \begin{assumption*}[#2]%
% \protected@edef\@currentlabel{#1}%
% }
% {%
% \end{assumption*}
% }
%\makeatother






\definecolor{red1}{RGB}{255,204,204}
\definecolor{blue1}{RGB}{204,204,255}
\definecolor{light-gray}{gray}{0.7}


\doublespacing

\begin{document}



\setlength{\abovedisplayskip}{8pt}
\setlength{\belowdisplayskip}{8pt}
\setlength{\abovedisplayshortskip}{8pt}
\setlength{\belowdisplayshortskip}{8pt}

\title{\vspace*{-1cm}Single Proxy Synthetic Control}
 \author{
  Chan Park and Eric J. Tchetgen Tchetgen
  \\[0.2cm]
  {\small Department of Statistics and Data Science, The Wharton School, University of Pennsylvania}\\
    }
 \date{}
  \maketitle
\begin{abstract}
Synthetic control methods are widely used to estimate the treatment effect on a single treated unit in time-series settings. A common approach for estimating synthetic controls is to regress the treated unit's pre-treatment outcome on those of untreated units via ordinary least squares. However, this approach can perform poorly if the pre-treatment fit is not near perfect, whether the weights are normalized or not. In this paper, we introduce a single proxy synthetic control approach, which views the outcomes of untreated units as proxies of the treatment-free potential outcome of the treated unit, a perspective we leverage to construct a valid synthetic control. Under this framework, we establish alternative identification and estimation methodologies for synthetic controls and for the treatment effect on the treated unit. Notably, unlike a proximal synthetic control approach which requires two types of proxies for identification, ours relies on a single type of proxy, thus facilitating its practical relevance. Additionally, we adapt a conformal inference approach to perform inference about the treatment effect, obviating the need for a large number of post-treatment data. Lastly, our framework can accommodate time-varying covariates and nonlinear models. We demonstrate the proposed approach in a simulation study and a real-world application.

\end{abstract}
\noindent%
{\it Keywords:}  Average treatment effect on the treated, Conformal inference, Generalized method of moments, Prediction interval, Synthetic control

 

\documentclass[5p]{elsarticle}

%\usepackage[nodots]{numcompress}
\usepackage{amsmath,amsthm,amssymb,amsfonts}
\usepackage[colorlinks,urlcolor=blue,linkcolor=blue,citecolor=blue,anchorcolor=blue]{hyperref}
\usepackage{lineno}
\modulolinenumbers[5]
\biboptions{sort&compress}

\journal{Science Bulletin}

\begin{document}
	
	\begin{frontmatter}
		
		\title{Observation of $\pi$ solitons in oscillating waveguide arrays}
		%\tnotetext[mytitlenote]{Fully documented templates are available in the elsarticle package on \href{http://www.ctan.org/tex-archive/macros/latex/contrib/elsarticle}{CTAN}.}
		
		%%% Group authors per affiliation:
		\author[ad1,ad2]{Antonina A. Arkhipova\corref{equal}}
		\author[ad3]{Yiqi Zhang\corref{equal}}
		\cortext[equal]{These authors contributed equally to this work.}
		\author[ad1]{Yaroslav V. Kartashov\corref{cor}}
		\cortext[cor]{Corresponding author}
		\ead{yaroslav.kartashov@icfo.eu}
		\author[ad1,ad4]{Sergei A. Zhuravitskii}
		\author[ad1,ad4]{NNikolay N. Skryabin}
		\author[ad4]{Ivan V. Dyakonov}
		\author[ad1,ad4]{Alexander A. Kalinkin}
		\author[ad4]{Sergei P. Kulik}
		\author[ad1]{Victor O. Kompanets}
		\author[ad1]{Sergey V. Chekalin}
		\author[ad1,ad2]{Victor N. Zadkov}
		\address[ad1]{Institute of Spectroscopy, Russian Academy of Sciences, Troitsk, Moscow, 108840, Russia}
		\address[ad2]{Faculty of Physics, Higher School of Economics, Moscow, 105066, Russia}
		\address[ad3]{Key Laboratory for Physical Electronics and Devices of the Ministry of Education \& Shaanxi Key Lab of Information Photonic Technique, School of Electronic and Information Engineering, Xi'an Jiaotong University, Xi'an, 710049, China}
		\address[ad4]{Quantum Technology Centre, Faculty of Physics, M. V. Lomonosov Moscow State University, Moscow, 119991, Russia}
		
		
		\begin{abstract}
			Floquet systems with periodically varying in time parameters enable realization of unconventional topological phases that do not exist in static systems with constant parameters and that are frequently accompanied by appearance of novel types of the topological states. Among such Floquet systems are the Su-Schrieffer-Heeger lattices with periodically-modulated couplings that can support at their edges anomalous $\pi$ modes of topological origin despite the fact that the lattice spends only half of the evolution period in topologically nontrivial phase, while during other half-period it is topologically trivial. Here, using Su-Schrieffer-Heeger arrays composed from periodically oscillating waveguides inscribed in transparent nonlinear optical medium, we report experimental observation of photonic anomalous $\pi$ modes residing at the edge or in the corner of the one- or two-dimensional arrays, respectively, and demonstrate a new class of topological $\pi$ solitons bifurcating from such modes in the topological gap of the Floquet spectrum at high powers. $\pi$ solitons reported here are strongly oscillating nonlinear Floquet states exactly reproducing their profiles after each longitudinal period of the structure. They can be dynamically stable in both one- and two-dimensional oscillating waveguide arrays, the latter ones representing the first realization of the Floquet photonic higher-order topological insulator, while localization properties of such $\pi$ solitons are determined by their power.
		\end{abstract}
		
		\begin{keyword}
			Floquet topological insulators, $\pi$ states, edge solitons, SSH model
		\end{keyword}
		
	\end{frontmatter}
	
	%\linenumbers
	
	\section{Introduction}
	Photonic topological insulators~\cite{lu.np.8.821.2014, ozawa.rmp.91.015006.2019} are unique materials hosting localized topologically protected states at their edges by analogy with edge modes in electronic topological insulators, first predicted in solid-state physics~\cite{hasan.rmp.82.3045.2010, qi.rmp.83.1057.2011}. Various mechanisms of formation of the photonic topological edge states were discovered, most of which are associated with breakup of certain symmetries of the underlying system possessing specific degeneracies in the linear spectrum. The most representative feature of topological edge states is their remarkable robustness with respect to deformations of the structure, disorder, and their persistence for different geometries of the edge between topologically distinct materials. Their formation and robustness has been predicted and demonstrated for various photonic systems with broken time-reversal symmetry, for valley-Hall systems with broken inversion symmetry, and in higher-order topological insulators~\cite{haldane.prl.100.013904.2008, wang.nature.461.772.2009, hafezi.nphys.7.907.2011, lindner.nphys.7.490.2011, rechtsman.nature.496.196.2013, khanikaev.nmat.12.233.2013, hafezi.np.7.1001.2013, mittal.prl.113.087403.2014, wu.prl.114.223901.2015, gao.nc.7.11619.2016, klembt.nature.562.552.2018, noh.prl.120.063902.2018, yang.nature.565.622.2019, lustig.nature.609.931.2022, pyrialakos.nm.21.634.2022, xie.natrevphys.3.520.2021}. Particularly nontrivial situation is realized when topological phase is induced by periodic modulations of system parameters in the evolution variable~\cite{garanovich.pr.518.1.2012}, for example in the direction of light propagation. Characterization of such driven Floquet systems usually requires special topological invariants, as shown in~\cite{kitagawa.prb.82.235114.2010, rudner.prx.3.031005.2013, rudner.nrp.2.229.2020}. Among the most intriguing manifestations of topological effects in Floquet systems is the formation of unidirectional edge states as proposed in~\cite{lindner.nphys.7.490.2011} and observed at optical frequencies in~\cite{rechtsman.nature.496.196.2013}, observation of anomalous topological states~\cite{maczewsky.nc.8.13756.2017, mukherjee.nc.8.13918.2017, leykam.prl.117.013902.2016}, and of so-called anomalous $\pi$ modes associated with nonzero $\pi$ gap invariant and studied in~\cite{asboth.prb.90.125143.2014, dallago.pra.92.023624.2015, fruchart.prb.93.115429.2016, zhang.acs.4.2250.2017, petracek.pra.101.033805.2020}. Recent surge of interest to topological pumping in near-solitonic regime should be mentioned too~\cite{jurgensen.nature.596.63.2021, fu.prl.128.154101.2022, fu.prl.129.183901.2022}.
	
	$\pi$ modes are unique topological states that may appear in a quasi-energy spectrum of the Floquet system that due to modulation of its parameters spends half of the evolution period in ``instantaneous'' nontopological phase, while on the other half of the period it is topologically nontrivial.
	In tight-binding models describing Floquet systems, $\pi$ modes usually appear at the edges of the ``longitudinal'' Brillouin zone with quasi-energies equal to $\pm \pi/T$ (where $T$ is period of the Floquet system), in contrast to conventional ``zero-energy'' edge states in static topological systems, hence the notion of $\pi$ modes that we also use in this work for convenience. So far, the photonic $\pi$ modes have been observed only in linear regime in one-dimensional modulated Su-Schrieffer-Heeger (SSH) arrays in a microwave range~\cite{cheng.prl.122.173901.2019} and at optical frequencies in non-Hermitian or plasmonic SSH arrays~\cite{wu.prr.3.023211.2021, song.lpr.15.2000584.2021, sidorenko.prr.4.033184.2022} with high refractive index contrast, where, however, considerable losses limit propagation distances to hundreds of micrometers. $\pi$ modes may have applications in the design of systems supporting high-quality cavity modes~\cite{ota.optica.6.786.2019, xie.lpr.14.1900425.2020}, for realization of low-threshold lasers~\cite{zhang.light.9.109.2020, kim.nc.11.5758.2020, zhong.apl.6.040802.2021, ota.nano.9.547.2020}, in strongly correlated electron–photon systems~\cite{bloch.nature.606.41.2022}, and other areas. They have been also encountered beyond the realm of optics, e.g., in acoustics~\cite{cheng.prl.129.254301.2022, zhu.nc.13.11.2022}. Nevertheless, to date the photonic $\pi$ modes remain unobserved in higher-dimensional conservative systems and their nonlinear analogs were never reported experimentally. At the same time, photonic Floquet systems offer unique testbed for the exploration of nonlinear effects in specifically designed low-loss topological guiding structures, where observation of so far elusive class of $\pi$ solitons is possible.
	
	It should also be mentioned that nonlinearity is playing an increasingly important role in all-optical control of topological systems, see recent reviews~\cite{smirnova.apr.7.021306.2020, ozawa.rmp.91.015006.2019}. In particular, nonlinearity may stimulate modulational instability of the nonlinear edge states~\cite{lumer.pra.94.021801.2016, leykam.prl.117.143901.2016, kartashov.optica.3.1228.2016}, it leads to rich bistability effects for edge states in pumped dissipative resonator structures~\cite{kartashov.prl.119.253904.2017, mandal.acs.10.147.2023, pernet.nphys.18.678.2022}, it may cause power-controlled topological transitions~\cite{maczewsky.science.370.701.2020}, and enables the formation of topological solitons both in the bulk of the insulator ~\cite{lumer.prl.111.243905.2013, mukherjee.science.368.856.2020} and at its edges~\cite{leykam.prl.117.143901.2016, ablowitz.pra.96.043868.2017, gulevich.sr.7.1780.2017, li.prb.97.081103.2018, smirnova.lpr.13.1900223.2019, zhang.prl.123.254103.2019, ivanov.acs.7.735.2020, zhong.ap.3.056001.2021, smirnova.prr.3.043027.2021, mukherjee.prx.11.041057.2021, xia.light.9.147.2020, guo.ol.45.6466.2020, kartashov.prl.128.093901.2022}. The important property of such solitons is that they remain localized due to nonlinearity, and at the same time they inherit topological protection from linear edge modes, from which they usually bifurcate. Corner solitons in higher-order topological insulators have been reported too~\cite{zangeneh.prl.123.053902.2019, kirsch.np.17.995.2021, hu.light.10.164.2021}. Very recently it was theoretically predicted~\cite{zhong.pra.107.L021502.2023} that Floquet topological systems may support a new class of topological solitons, qualitatively different from previously observed unidirectional states \cite{mukherjee.prx.11.041057.2021} -- namely, $\pi$ soliton -- that represents dynamically oscillating nonlinear Floquet state with a quasi-energy in the topological bandgap that exactly reproduces its intensity distribution after each longitudinal period of the structure. Even strongly localized $\pi$ solitons are practically free from radiative losses that usually restrict propagation distances for unidirectional edge solitons in Floquet waveguiding systems~\cite{ivanov.acs.7.735.2020, mukherjee.prx.11.041057.2021}.
	
	In this work, we report on the experimental observation of $\pi$ solitons in one- and two-dimensional Floquet waveguide arrays, where nontrivial topological properties arise due to $z$-periodic oscillations of waveguide centers in each unit cell of the structure. The arrays considered here are inscribed in a transparent nonlinear dielectric medium (fused silica) using the technique of direct femtosecond laser writing~\cite{tan.ap.3.024002.2021, li.ap.4.024002.2022, lin.us.2021.9783514.2021} and represent SSH-like structures, which, however, are not static, but spend half of the $z$-period in ``instantaneous'' topological phase, while during other half of the period they are ``instantaneously'' non-topological, as defined by periodically varying intra- and inter-cell coupling strengths. In two dimensions such arrays represent the realization of the photonic Floquet higher-order insulator. Floquet spectrum of such arrays is characterized by the presence of in-gap topological $\pi$ modes, from which robust $\pi$ solitons can bifurcate in the nonlinear regime. We observe such solitons using single-site excitations, study their periodic evolution with distance, and dependence of their localization properties on the amplitude of the waveguide oscillations and power.
		
	% Figure environment removed
	
	\section{Results and discussions}
	We consider paraxial propagation of a light beam along the $z$ axis of the medium with focusing cubic nonlinearity and shallow transverse modulation of the refractive index that can be described by the nonlinear Schr\"odinger-like equation for the dimensionless light field amplitude $\psi$:
	\begin{equation}\label{eq1}
		i \frac{\partial \psi}{\partial z} = -\frac{1}{2} \left( \frac{\partial^2}{\partial x^2} + \frac{\partial^2}{\partial y^2} \right) \psi
		-\mathcal{R}(x,y,z) \psi - |\psi|^{2} \psi.
	\end{equation}
	Here $x,y$ are the scaled transverse coordinates, $z$ is the propagation distance that plays in Eq.~(\ref{eq1}) the same role as time in the Schrödinger equation describing a quantum particle in a potential, and the function ${\mathcal R}(x,y,z)$ describes array of periodically oscillating waveguides.
	For details of normalization of Eq.~(\ref{eq1}) see Section S1 in the Supplementary materials .
	
	\subsection{1D $\pi$ solitons}
	
	First of all, for observation of 1D $\pi$ solitons we consider the SSH-like arrays of oscillating waveguides containing 7 unit cells. Refractive index distribution in such arrays can be described by the following function:
	\[
	{\mathcal R} = p \sum_{m} [e^{-(x^2_{1m}/a_x^2+y^2/a_y^2)} +  e^{-(x^2_{2m}/a_x^2+y^2/a_y^2)}],
	\]
	where $x_{1m} = x_m + d/2 + r\cos(\omega z)$ and $x_{2m} = x_m - d/2 - r\cos(\omega z)$ are the $x$-coordinates of the waveguide centers in each unit cell containing two waveguides, $\omega=2\pi/Z$ is the spatial frequency of oscillations of the waveguide centers, $Z$ is the longitudinal period of the array, $x_m=x-2md$, $m$ is the integer index of the cell, $r$ is the amplitude of the waveguide oscillations, which was varied from $1$ to $11\,\mu\textrm{m}$, $d=30\,\mu\textrm{m}$ is the spacing between waveguides at $r=0$ (i.e. unit cell size is $2d$), $a_x=2.5~\mu \textrm{m}$ and $a_y=7.5~\mu \textrm{m}$ are the widths of waveguides that are elliptical due to writing process, and $p$ is the array depth proportional to the refractive index contrast $\delta n$ in the structure (see Section S1 in the Supplementary materials ). Schematic illustration of such array is presented in Fig.~\ref{fig1}a. As one can see, the separation $d-2r\cos(\omega z)$ between two waveguides in the unit cell (intracell separation) of this structure varies dynamically, leading to periodic transformation between ``instantaneously'' topological (inter-cell coupling exceeds intracell one) and non-topological (inter-cell coupling is weaker than intra-cell one) SSH configurations. Microphotographs of such 1D fs-laser written arrays in fused silica at different distances within the sample are presented in Fig.~\ref{fig1}b. The array period $Z=33\,\textrm{mm}$ was selected such that our samples contained three full $z$-periods of this Floquet structure (see Section S2 in the Supplementary materials ).
	
	Nontrivial topological properties in this system arise due to longitudinal variations of the structure (oscillations of the waveguides). Its modes are the Floquet states $\psi=u(x,y,z) e^{ibz}$, where $b$ is a quasi-propagation constant [for first Brillouin zone $b\in [-\omega/2,+\omega/2)$], and $u(x,y,z)=u(x,y,z+Z)$ is the $Z$-periodic complex field that satisfy the equation
	\begin{equation}\label{eq2}
		bu = \frac{1}{2} \left( \frac{\partial^2}{\partial x^2} + \frac{\partial^2}{\partial y^2} \right) u + \mathcal{R}u + i \frac{\partial u}{\partial z} +|u|^2 u.
	\end{equation}
	Neglecting nonlinear term in Eq.~(\ref{eq2}) we calculate linear spectrum of 1D array using the method proposed in~\cite{leykam.prl.117.013902.2016} (see Section S5 in the Supplementary materials ). The transformation of linear spectrum with increase of the amplitude $r$ of waveguide oscillations is shown in Fig.~\ref{fig1}c. Quasi-propagation constant $b$ is defined modulo $\omega$ and in Fig.~\ref{fig1}c we show the spectrum within three longitudinal Brillouin zones. Gray lines correspond to the delocalized bulk modes, while the red lines correspond to the linear topological $\pi$ modes~\cite{asboth.prb.90.125143.2014, dallago.pra.92.023624.2015, fruchart.prb.93.115429.2016, zhang.acs.4.2250.2017, petracek.pra.101.033805.2020}. Notice that they emerge around the points, where Floquet replicas of the bands spectrally overlap, and longitudinal modulation hybridizes states at the band edges lifting their degeneracy and opening the topological gap. Because our structure is symmetric, $\pi$ modes appear near both edges of the array. Quasi-propagation constants of $\pi$ modes are located in the forbidden gap in the Floquet spectrum that guarantees absence of coupling with bulk states. Their localization near a given edge increases with increase of the gap width, cf Fig.~\ref{fig1}d and  e. Such modes show strong shape transformations within longitudinal period, but exactly reproduce their shape after each period $Z$. Remarkably, Fig.~\ref{fig1}e clearly shows that global intensity maximum of the $\pi$ mode is not always located in the edge waveguide. For instance, at $z=Z/2$, where the array is in instantaneous nontopological phase, the intensity maximum switches into next to edge waveguide, while at $z=Z$, exactly after one oscillation period, where structure returns into instantaneous topological configuration, the light also switches back to the edge waveguide. Already for $r = 7\,\mu\textrm{m}$ the $\pi$ mode contracts practically to single waveguide in some points within evolution period that enables its efficient excitation in the experiment [this determined our choice of the initial ``phase'' of the waveguide oscillations in Fig.~\ref{fig1}a; linear spectrum clearly does not depend on this phase]. Very similar results and Floquet spectrum were obtained also for arrays with odd number of waveguides (where one of the unit cells is incomplete), because even in this case due to waveguide oscillations both edges periodically pass through stages when truncation becomes topological or nontopological and therefore support $\pi$ modes.
	
	The appearance of topological $\pi$ modes in the spectrum of this Floquet system is associated with nonzero value of the $\pi$ gap invariant $w_{\pi}$ (see Section S4 in the Supplementary materials  for details of its calculation and literature~\cite{fruchart.prb.93.115429.2016,cheng.prl.122.173901.2019}). One can observe the formation of $\pi$ modes in the spectrum of truncated array when $w_{\pi}=1$ (for sufficiently large oscillation amplitudes $r$), while these modes are absent when $w_{\pi}=0$ (e.g., at $r\to 0$).
	
	Inspecting spectrum in Fig.~\ref{fig1}c one can see that while longitudinal modulation with frequency $\omega$ creates Floquet replicas of bands and localized $\pi$ modes, by itself it does not induce parametric resonances between localized and bulk modes, which nevertheless can occur, if additional weak modulation of optical potential at frequencies $2\omega$, $3\omega$, $\cdots$ is added that keeps $Z$-periodicity of array.
	
	The $\pi$ solitons are the topological nonlinear Floquet states bifurcating from the linear $\pi$ modes. To find their profiles we iteratively solve Eq.~(\ref{eq2}) with the last nonlinear term included (see Section S6 in the Supplementary materials ), by varying soliton power $U=\iint |\psi|^2dxdy$ and calculating for each $U$ corresponding $Z$-periodic soliton profile $u(x,y,z)$, quasi-propagation constant $b$, and averaged amplitude $A=Z^{-1}\int_z^{z+Z} \textrm{max}|\psi|dz$. The $\pi$ solitons bifurcate from linear $\pi$ modes, as evident from the representative $b(U)$ dependence in Fig.~\ref{fig2}a, where quasi-propagation constant of linear mode is shown by the dashed line. Their amplitude $A$ increases with the power (Fig.~\ref{fig2}b). Importantly, nonlinearity changes the location of $b$ inside the topological gap, gradually shifting it towards the bulk band (gray region). This is accompanied by changes in soliton localization in the $(x,y)$ plane (it may first increase and then decrease depending on the value of $r$), especially when $b$ shifts into the band, where coupling with the bulk modes occurs. Periodic transformation of soliton intensity distribution with $z$ is illustrated in Fig.~\ref{fig2}c. It should be stressed that for our parameters even for the amplitude of oscillations $r\sim9~\mu \textrm{m}$ the $\pi$ solitons obtained here are robust objects that practically do not radiate and survive over hundreds of $Z$ periods that is beneficial in comparison with the previously observed unidirectional edge solitons~\cite{mukherjee.prx.11.041057.2021}.
	
	We tested stability of such states by adding broadband small noise (typically, $5 \%$ in amplitude) into input field distributions and propagating such perturbed $\pi$ solitons over distances $\sim 500Z$ that allows to detect the presence of even very weak instabilities. Such stability analysis has shown that for our parameters and for amplitudes of oscillations $r>5~\mu\textrm{m}$ 1D solitons belonging to forbidden gap are stable, while they become unstable, when they shift into band. Notice that stabilization of such states that have powers well below power of Townes soliton in uniform cubic medium is consistent with arguments of Ref. \cite{sakaguchi.pre.89.032920.2014}.
	
	% Figure environment removed
	
	% Figure environment removed
	
	To demonstrate 1D $\pi$ solitons experimentally we inscribed in a fused silica sample the series of SSH-like arrays with the different amplitudes $r$ of waveguide oscillations ranging from $1$ to $11~\mu\textrm{m}$, with a step in $r$ of $2~\mu\textrm{m}$ using fs-laser writing technique (see Fig.~\ref{fig1}b with exemplary microphotographs of the array and Section S2 in the Supplementary materials  for the details of inscription). While full sample length contains three $Z$-periods of the structure, to demonstrate dynamics in the internal points of the last period, we additionally inscribed arrays with fractional lengths $2.25Z$, $2.50Z$, $2.75Z$ with the same parameters (see Section S2 in the Supplementary materials  for the details).
	
	% Figure environment removed
	
	In experiments we excited the waveguide at the left edge using the fs-laser pulses of variable energy $E$ (for correspondence between pulse energy $E$ and input peak power in the waveguide see Section S3 in the Supplementary materials ). Output intensity cross-sections at $y=0$ (red lines) and 2D distributions (blue insets) are compared in Fig.~\ref{fig3} with the results of theoretical simulations of the single-site excitation with different input powers $U$ in the frames of Eq.~(\ref{eq1}) (black-red insets). In all cases theoretical results well agree with the experimental observations. In Fig.~\ref{fig3}a and top right image in Fig.~\ref{fig3}c we show how output beam localization progressively increases in linear regime (low pulse energies $E=20\,\textrm{nJ}$) with increase of the amplitude $r$  of the waveguide oscillations. Efficient excitation of well-localized linear $\pi$ modes is evident for $r\ge 7\,\mu\textrm{m}$, while for $r=5\,\mu\textrm{m}$ the excitation efficiency is lower and some fraction of power penetrates into the bulk of the array. First rows in Fig.~\ref{fig3}b and c illustrate that $\pi$ mode undergoes strong oscillations on one $Z$ period, main intensity maximum switches into second waveguide at $z=2.50Z$ (consistently with the dynamics of exact state in Fig.~\ref{fig1}e), but it returns to the edge one at $z=3.00Z$ illustrating periodic evolution.
	
	
	By increasing pulse energy we observe the formation of the 1D $\pi$ solitons. As mentioned above, they are in-gap topological states bifurcating from $\pi$ modes under the action of nonlinearity. Nonlinearity leads to soliton reshaping (in particular, for $r=7-9\,\mu\textrm{m}$ it slightly broadens with increase of $U$), but when its quasi-propagation constant shifts into the band, strong radiation into the bulk occurs. This is most clearly visible for $r=7\,\mu\textrm{m}$ (Fig.~\ref{fig3}b), where solitons were observed well-localized near the edge for the pulse energies $E<350\,\textrm{nJ}$, but radiating around $E\sim430\,\textrm{nJ}$ (notice that at this energy the level of radiation becomes visible only after three $Z$ periods). Further increase of the pulse energy leads to stronger radiation. At $r=9\,\mu\textrm{m}$ the range of the pulse energies, where formation of robust $\pi$ solitons is observed substantially increases (Fig.~\ref{fig3}c). Well localized $\pi$ solitons performing $Z$-periodic oscillations are observed for the pulse energies $E<900\,\textrm{nJ}$ (rows 1-3) and only at $E\sim1000\,\textrm{nJ}$ small radiation due to the coupling with the bulk modes appears (row 4). Simulations over much larger distances $(z>100Z)$ confirm robustness of such dynamically excited nonlinear Floquet states with in-gap quasi-propagation constants. It should be stressed that excitation of the waveguide in the bulk of the above arrays does not yield localization for considered pulse energies.
	
	\subsection{2D $\pi$ solitons}
	
	For observation of 2D $\pi$ solitons we utilize 2D generalization of the SSH array with oscillating waveguides. Unit cell of such an array (quadrimer) contains four waveguides, whose centers oscillate with period $Z$ along the diagonals of the unit cell. We consider sufficiently large structure containing $5\times5$ unit cells. Refractive index distribution in this Floquet structure is described by the function
	\[
	\begin{split}
		{\mathcal R}  =p \sum_{m,n} [& e^{-(x^2_{1m}/a_x^2+y^2_{1n}/a_y^2)} + e^{-(x^2_{2m}/a_x^2+y^2_{1n}/a_y^2)} + \\
		&e^{-(x^2_{1m}/a_x^2+y^2_{2n}/a_y^2)} + e^{-(x^2_{2m}/a_x^2+y^2_{2n}/a_y^2)}],
	\end{split}
	\]
	where  $x_{1m,2m} = x_m \pm d/2 \pm r\cos(\omega z)$ and ${y_{1n,2n} = y_n} \pm d/2 \pm r \cos(\omega z)$ are the coordinates of centers of four waveguides in the unit cell with $x_m=x-2md$ and $y_n=y-2nd$, $m,n$ are the integers. In 2D case the oscillation period was taken as $Z=49.5\,\textrm{mm}$, so that sample contained two full longitudinal array periods. Spacing between waveguides at $r=0 ~\mu\textrm{m}$ was set to $d=32\,\mu\textrm{m}$, and to achieve more uniform coupling between elliptic waveguides, their longer axes were oriented along the diagonal of the array (see schematics in Fig.~\ref{fig4}a and microphotographs of inscribed structure at different distances in Fig.~\ref{fig4}b). As one can see, such structure realizes the photonic Floquet higher-order insulator periodically switching between ``instantaneous'' topological and non-topological phases.
	
	% Figure environment removed
	
	Dependence of quasi-propagation constants $b$ of the Floquet eigenmodes of the 2D array, obtained from linear version of Eq.~(\ref{eq2}), on amplitude of waveguide oscillations $r$ shown in Fig.~\ref{fig4}c reveals the formation of 2D $\pi$ modes (red lines) that reside in the corners of the structure, but in comparison with 1D case they appear in sufficiently narrow range of oscillation amplitudes. This is a consequence of substantially more complex spectrum of static 2D SSH structures \cite{hu.light.10.164.2021} featuring four bands in topological phase (in contrast to only two bands in 1D SSH arrays), that in our case experience folding due to longitudinal array modulation, resulting in a very complex Floquet spectrum. For instance, quasi-propagation constants of 2D $\pi$ modes may overlap with the band, as it also happens with eigenvalues of usual corner modes in static higher-order insulators~\cite{hu.light.10.164.2021}. To obtain such spectrum, where topological gap can be opened by longitudinal array modulation, we had to select not too small depth of potential $p=5$ to ensure that the width of the bulk bands is comparable with the width of the longitudinal Brillouin zone and no too strong band folding occurs. Example of the $\pi$ mode performing periodic oscillations (only one corner is shown) is depicted in Fig.~\ref{fig4}d. The properties of 2D $\pi$ solitons, whose family was obtained from nonlinear Eq.~(\ref{eq2}) using iterative method, are summarized in Fig.~\ref{fig5}. As in the 1D case, quasi-propagation constant of 2D solitons crosses the gap with increase of power $U$ and enters into the band (Fig.~\ref{fig5}a). For the selected amplitude $r=7\,\mu\textrm{m}$ the soliton exists practically in the entire gap, because $b$ of linear $\pi$ mode from which it bifurcates is located near the lower gap edge (we have checked that this linear $\pi$ mode indeed falls into forbidden gap of bulk system by calculating quasienergy spectrum of periodic, i.e. infinite in the transverse direction, Floquet array). The average amplitude $A$ increases with $U$ (Fig.~\ref{fig5}b). The intensity distributions at different distances illustrating periodic $\pi$ soliton evolution in $z$ are presented in Fig.~\ref{fig5}c. Despite the fact that this state is 2D and oscillates strongly, the collapse is suppressed and one observes very robust propagation for all powers, when soliton resides in the gap. This conclusion was also supported by the results of propagation of weakly perturbed 2D $\pi$ solitons over large distances. For $r=7\,\mu\textrm{m}$ all such states in the gap were found stable.
	
	% Figure environment removed
	
	To observe 2D $\pi$ solitons we inscribed the series of 2D oscillating arrays with various amplitudes of waveguide oscillations up to $r=9~\mu\textrm{m}$. Excitations in the right corner of the array were used, but it should be stressed that excitations of other corners yields nearly identical results due to high symmetry and uniformity of the array. At low pulse energies $E=10\,\textrm{nJ}$ one observes strong diffraction into the bulk at $r=3\,\mu\textrm{m}$ (Fig.~\ref{fig6}a), while efficient excitation of linear $\pi$ modes takes place at amplitudes $r\ge 5\,\mu\textrm{m}$ (Fig.~\ref{fig6}b and c). To the best of our knowledge, this constitutes the first observation of 2D $\pi$ modes in photonics. Increasing pulse energy at low $r\sim 3\,\mu\textrm{m}$ results first in concentration of light in the bulk of the sample and then its gradual displacement toward the corner (Fig.~\ref{fig6}a). By  contrast, for $r\ge 5\,\mu\textrm{m}$ one observes the formation of $\pi$ solitons, whose range of existence in terms of input power grows with increase of $r$. Thus, at $r=5\,\mu\textrm{m}$ the well-localized solitons form at pulse energies $E<300~\textrm{nJ}$, while at $E\sim 400~\textrm{nJ}$ strong radiation into the bulk occurs (Fig.~\ref{fig6}b) due to nonlinearity-induced shift into the allowed band. At $r=7\,\mu\textrm{m}$ one observes the formation of the $\pi$ solitons even at $E\sim 600~\textrm{nJ}$ (Fig.~\ref{fig6}c) with tendency for slight increase of secondary intensity maxima in soliton profile at highest power levels that is observed also in exact soliton solution of Eq.~(\ref{eq2}). Excitations in other corners of the array (e.g., top one) yield similar results confirming the $\pi$ soliton formation, while excitations in the bulk strongly diffract at these pulse energies.
	

	
	\section{Conclusion}
	
	We presented experimental observation of a new type of $\pi$ solitons in nonlinear Floquet system, where nontrivial topology arises from periodic modulation of the underlying photonic structure in evolution variable (along the light propagation path). Such solitons exist both in 1D and 2D geometries and they show exceptionally robust evolution due to practically absent radiative losses at considered periods and amplitudes of oscillations. The results obtained here may be used in the design of a class of topological Floquet lasers based on $\pi$ modes, for the control and enhancement of parametric processes, such as generation of new harmonics assisted by topology of the Floquet system, and for design of new types of on-chip all-optically controlled topological devices.
	
 
	\section*{Conflict of interest}
	The authors declare that they have no conflict of interest.
		
	\section*{Acknowledgments}
	This research is funded by the research project FFUU-2021-0003 of the Institute of Spectroscopy of the Russian Academy of Sciences and partially by the RSF grant 21-12-00096. Y. Z. acknowledges funding by the National Natural Science Foundation of China (Grant Nos.: 12074308). S. Z. acknowledges support by the Foundation for the Advancement of Theoretical Physics and Mathematics “BASIS” (Grant No.: 22-2-2-26-1).
	
 
	

	\section*{Author contributions}
	Yiqi Zhang and Yaroslav V. Kartashov formulated the problem. Sergei A. Zhuravitskii, Nikolay N. Skryabin, Ivan V. Dyakonov, and Alexander A. Kalinkin fabricated the samples. Antonina A. Arkhipova, Victor O. Kompanets, and Sergey V. Chekalin performed experiments. Yiqi Zhang performed numerical modeling. Yaroslav V. Kartashov, Sergei P. Kulik, and Victor N. Zadkov supervised the work. All co-authors took part in discussion of results and writing of manuscript.
	
	%	\bibliography{my_library}
	\begin{thebibliography}{78}
		\providecommand{\natexlab}[1]{#1}
		\providecommand{\url}[1]{\texttt{#1}}
		\providecommand{\href}[2]{#2}
		\providecommand{\path}[1]{#1}
		\providecommand{\eprint}[1]{\href{http://arxiv.org/abs/#1}{\path{#1}}}
		\providecommand{\DOIprefix}{doi:}
		\providecommand{\ArXivprefix}{arXiv:}
		\providecommand{\URLprefix}{URL: }
		\providecommand{\Pubmedprefix}{pmid:}
		\providecommand{\doi}[1]{\href{http://dx.doi.org/#1}{\path{#1}}}
		\providecommand{\Pubmed}[1]{\href{pmid:#1}{\path{#1}}}
		\providecommand{\BIBand}{and}
		\providecommand{\bibinfo}[2]{#2}
		\ifx\xfnm\undefined \def\xfnm[#1]{\unskip,\space#1}\fi
		%Type = Article
		\bibitem[{Lu et~al.(2014)Lu, Joannopoulos and
			Solja{\v{c}}i{\'c}}]{lu.np.8.821.2014}
		\bibinfo{author}{Lu\xfnm[ L.]}, \bibinfo{author}{Joannopoulos\xfnm[ J.D.]},
		\bibinfo{author}{Solja{\v{c}}i{\'c}\xfnm[ M.]}.
		\newblock \bibinfo{title}{Topological photonics}.
		\newblock \bibinfo{journal}{Nat Photonics}
		\bibinfo{year}{2014};\bibinfo{volume}{8}(\bibinfo{number}{11}):\bibinfo{pages}{821--829}.
		
		%Type = Article
		\bibitem[{Ozawa et~al.(2019)Ozawa, Price, Amo, Goldman, Hafezi, Lu
			et~al.}]{ozawa.rmp.91.015006.2019}
		\bibinfo{author}{Ozawa\xfnm[ T.]}, \bibinfo{author}{Price\xfnm[ H.M.]},
		\bibinfo{author}{Amo\xfnm[ A.]}, \bibinfo{author}{Goldman\xfnm[ N.]},
		\bibinfo{author}{Hafezi\xfnm[ M.]}, \bibinfo{author}{Lu\xfnm[ L.]}, et~al.
		\newblock \bibinfo{title}{Topological photonics}.
		\newblock \bibinfo{journal}{Rev Mod Phys}
		\bibinfo{year}{2019};\bibinfo{volume}{91}:\bibinfo{pages}{015006}.
		
		%Type = Article
		\bibitem[{Hasan and Kane(2010)}]{hasan.rmp.82.3045.2010}
		\bibinfo{author}{Hasan\xfnm[ M.Z.]}, \bibinfo{author}{Kane\xfnm[ C.L.]}.
		\newblock \bibinfo{title}{Colloquium: Topological insulators}.
		\newblock \bibinfo{journal}{Rev Mod Phys}
		\bibinfo{year}{2010};\bibinfo{volume}{82}:\bibinfo{pages}{3045--3067}.
		
		%Type = Article
		\bibitem[{Qi and Zhang(2011)}]{qi.rmp.83.1057.2011}
		\bibinfo{author}{Qi\xfnm[ X.L.]}, \bibinfo{author}{Zhang\xfnm[ S.C.]}.
		\newblock \bibinfo{title}{Topological insulators and superconductors}.
		\newblock \bibinfo{journal}{Rev Mod Phys}
		\bibinfo{year}{2011};\bibinfo{volume}{83}:\bibinfo{pages}{1057--1110}.
		
		%Type = Article
		\bibitem[{Haldane and Raghu(2008)}]{haldane.prl.100.013904.2008}
		\bibinfo{author}{Haldane\xfnm[ F.D.M.]}, \bibinfo{author}{Raghu\xfnm[ S.]}.
		\newblock \bibinfo{title}{Possible realization of directional optical
			waveguides in photonic crystals with broken time-reversal symmetry}.
		\newblock \bibinfo{journal}{Phys Rev Lett}
		\bibinfo{year}{2008};\bibinfo{volume}{100}:\bibinfo{pages}{013904}.
		
		%Type = Article
		\bibitem[{Wang et~al.(2009)Wang, Chong, Joannopoulos and
			Solja{\v{c}}i{\'c}}]{wang.nature.461.772.2009}
		\bibinfo{author}{Wang\xfnm[ Z.]}, \bibinfo{author}{Chong\xfnm[ Y.]},
		\bibinfo{author}{Joannopoulos\xfnm[ J.D.]},
		\bibinfo{author}{Solja{\v{c}}i{\'c}\xfnm[ M.]}.
		\newblock \bibinfo{title}{Observation of unidirectional backscattering-immune
			topological electromagnetic states}.
		\newblock \bibinfo{journal}{Nature}
		\bibinfo{year}{2009};\bibinfo{volume}{461}:\bibinfo{pages}{772--775}.
		
		%Type = Article
		\bibitem[{Hafezi et~al.(2011)Hafezi, Demler, Lukin and
			Taylor}]{hafezi.nphys.7.907.2011}
		\bibinfo{author}{Hafezi\xfnm[ M.]}, \bibinfo{author}{Demler\xfnm[ E.A.]},
		\bibinfo{author}{Lukin\xfnm[ M.D.]}, \bibinfo{author}{Taylor\xfnm[ J.M.]}.
		\newblock \bibinfo{title}{Robust optical delay lines with topological
			protection}.
		\newblock \bibinfo{journal}{Nat Phys}
		\bibinfo{year}{2011};\bibinfo{volume}{7}(\bibinfo{number}{1}):\bibinfo{pages}{907--912}.
		
		%Type = Article
		\bibitem[{Lindner et~al.(2011)Lindner, Refael and
			Galitski}]{lindner.nphys.7.490.2011}
		\bibinfo{author}{Lindner\xfnm[ N.H.]}, \bibinfo{author}{Refael\xfnm[ G.]},
		\bibinfo{author}{Galitski\xfnm[ V.]}.
		\newblock \bibinfo{title}{Floquet topological insulator in semiconductor
			quantum wells}.
		\newblock \bibinfo{journal}{Nat Phys}
		\bibinfo{year}{2011};\bibinfo{volume}{7}(\bibinfo{number}{1}):\bibinfo{pages}{490--495}.
		
		%Type = Article
		\bibitem[{Rechtsman et~al.(2013)Rechtsman, Zeuner, Plotnik, Lumer, Podolsky,
			Dreisow et~al.}]{rechtsman.nature.496.196.2013}
		\bibinfo{author}{Rechtsman\xfnm[ M.C.]}, \bibinfo{author}{Zeuner\xfnm[ J.M.]},
		\bibinfo{author}{Plotnik\xfnm[ Y.]}, \bibinfo{author}{Lumer\xfnm[ Y.]},
		\bibinfo{author}{Podolsky\xfnm[ D.]}, \bibinfo{author}{Dreisow\xfnm[ F.]},
		et~al.
		\newblock \bibinfo{title}{Photonic {F}loquet topological insulators}.
		\newblock \bibinfo{journal}{Nature}
		\bibinfo{year}{2013};\bibinfo{volume}{496}:\bibinfo{pages}{196--200}.
		
		%Type = Article
		\bibitem[{Khanikaev et~al.(2013)Khanikaev, Hossein~Mousavi, Tse, Kargarian,
			MacDonald and Shvets}]{khanikaev.nmat.12.233.2013}
		\bibinfo{author}{Khanikaev\xfnm[ A.B.]}, \bibinfo{author}{Hossein~Mousavi\xfnm[
			S.]}, \bibinfo{author}{Tse\xfnm[ W.K.]}, \bibinfo{author}{Kargarian\xfnm[
			M.]}, \bibinfo{author}{MacDonald\xfnm[ A.H.]}, \bibinfo{author}{Shvets\xfnm[
			G.]}.
		\newblock \bibinfo{title}{Photonic topological insulators}.
		\newblock \bibinfo{journal}{Nat Mater}
		\bibinfo{year}{2013};\bibinfo{volume}{12}(\bibinfo{number}{1}):\bibinfo{pages}{233–239}.
		
		%Type = Article
		\bibitem[{Hafezi et~al.(2013)Hafezi, Mittal, Fan, Migdall and
			Taylor}]{hafezi.np.7.1001.2013}
		\bibinfo{author}{Hafezi\xfnm[ M.]}, \bibinfo{author}{Mittal\xfnm[ S.]},
		\bibinfo{author}{Fan\xfnm[ J.]}, \bibinfo{author}{Migdall\xfnm[ A.]},
		\bibinfo{author}{Taylor\xfnm[ J.M.]}.
		\newblock \bibinfo{title}{Imaging topological edge states in silicon
			photonics}.
		\newblock \bibinfo{journal}{Nat Photonics}
		\bibinfo{year}{2013};\bibinfo{volume}{7}(\bibinfo{number}{12}):\bibinfo{pages}{1001--1005}.
		
		%Type = Article
		\bibitem[{Mittal et~al.(2014)Mittal, Fan, Faez, Migdall, Taylor and
			Hafezi}]{mittal.prl.113.087403.2014}
		\bibinfo{author}{Mittal\xfnm[ S.]}, \bibinfo{author}{Fan\xfnm[ J.]},
		\bibinfo{author}{Faez\xfnm[ S.]}, \bibinfo{author}{Migdall\xfnm[ A.]},
		\bibinfo{author}{Taylor\xfnm[ J.M.]}, \bibinfo{author}{Hafezi\xfnm[ M.]}.
		\newblock \bibinfo{title}{Topologically robust transport of photons in a
			synthetic gauge field}.
		\newblock \bibinfo{journal}{Phys Rev Lett}
		\bibinfo{year}{2014};\bibinfo{volume}{113}:\bibinfo{pages}{087403}.
		
		%Type = Article
		\bibitem[{Wu and Hu(2015)}]{wu.prl.114.223901.2015}
		\bibinfo{author}{Wu\xfnm[ L.H.]}, \bibinfo{author}{Hu\xfnm[ X.]}.
		\newblock \bibinfo{title}{Scheme for achieving a topological photonic crystal
			by using dielectric material}.
		\newblock \bibinfo{journal}{Phys Rev Lett}
		\bibinfo{year}{2015};\bibinfo{volume}{114}:\bibinfo{pages}{223901}.
		
		%Type = Article
		\bibitem[{Gao et~al.(2016)Gao, Gao, Shi, Yang, Lin, Xu
			et~al.}]{gao.nc.7.11619.2016}
		\bibinfo{author}{Gao\xfnm[ F.]}, \bibinfo{author}{Gao\xfnm[ Z.]},
		\bibinfo{author}{Shi\xfnm[ X.]}, \bibinfo{author}{Yang\xfnm[ Z.]},
		\bibinfo{author}{Lin\xfnm[ X.]}, \bibinfo{author}{Xu\xfnm[ H.]}, et~al.
		\newblock \bibinfo{title}{Probing topological protection using a designer
			surface plasmon structure}.
		\newblock \bibinfo{journal}{Nat Commun}
		\bibinfo{year}{2016};\bibinfo{volume}{7}(\bibinfo{number}{1}):\bibinfo{pages}{11619}.
		
		%Type = Article
		\bibitem[{Klembt et~al.(2018)Klembt, Harder, Egorov, Winkler, Ge, Bandres
			et~al.}]{klembt.nature.562.552.2018}
		\bibinfo{author}{Klembt\xfnm[ S.]}, \bibinfo{author}{Harder\xfnm[ T.H.]},
		\bibinfo{author}{Egorov\xfnm[ O.A.]}, \bibinfo{author}{Winkler\xfnm[ K.]},
		\bibinfo{author}{Ge\xfnm[ R.]}, \bibinfo{author}{Bandres\xfnm[ M.A.]}, et~al.
		\newblock \bibinfo{title}{Exciton-polariton topological insulator}.
		\newblock \bibinfo{journal}{Nature}
		\bibinfo{year}{2018};\bibinfo{volume}{562}(\bibinfo{number}{7728}):\bibinfo{pages}{552--556}.
		
		%Type = Article
		\bibitem[{Noh et~al.(2018)Noh, Huang, Chen and
			Rechtsman}]{noh.prl.120.063902.2018}
		\bibinfo{author}{Noh\xfnm[ J.]}, \bibinfo{author}{Huang\xfnm[ S.]},
		\bibinfo{author}{Chen\xfnm[ K.P.]}, \bibinfo{author}{Rechtsman\xfnm[ M.C.]}.
		\newblock \bibinfo{title}{Observation of photonic topological valley {H}all
			edge states}.
		\newblock \bibinfo{journal}{Phys Rev Lett}
		\bibinfo{year}{2018};\bibinfo{volume}{120}:\bibinfo{pages}{063902}.
		
		%Type = Article
		\bibitem[{Yang et~al.(2019)Yang, Gao, Xue, Zhang, He, Yang
			et~al.}]{yang.nature.565.622.2019}
		\bibinfo{author}{Yang\xfnm[ Y.]}, \bibinfo{author}{Gao\xfnm[ Z.]},
		\bibinfo{author}{Xue\xfnm[ H.]}, \bibinfo{author}{Zhang\xfnm[ L.]},
		\bibinfo{author}{He\xfnm[ M.]}, \bibinfo{author}{Yang\xfnm[ Z.]}, et~al.
		\newblock \bibinfo{title}{Realization of a three-dimensional photonic
			topological insulator}.
		\newblock \bibinfo{journal}{Nature}
		\bibinfo{year}{2019};\bibinfo{volume}{565}(\bibinfo{number}{7741}):\bibinfo{pages}{622--626}.
		
		%Type = Article
		\bibitem[{Lustig et~al.(2022)Lustig, Maczewsky, Beck, Biesenthal, Heinrich,
			Yang et~al.}]{lustig.nature.609.931.2022}
		\bibinfo{author}{Lustig\xfnm[ E.]}, \bibinfo{author}{Maczewsky\xfnm[ L.J.]},
		\bibinfo{author}{Beck\xfnm[ J.]}, \bibinfo{author}{Biesenthal\xfnm[ T.]},
		\bibinfo{author}{Heinrich\xfnm[ M.]}, \bibinfo{author}{Yang\xfnm[ Z.]},
		et~al.
		\newblock \bibinfo{title}{Photonic topological insulator induced by a
			dislocation in three dimensions}.
		\newblock \bibinfo{journal}{Nature}
		\bibinfo{year}{2022};\bibinfo{volume}{609}(\bibinfo{number}{1}):\bibinfo{pages}{931–935}.
		
		%Type = Article
		\bibitem[{Pyrialakos et~al.(2022)Pyrialakos, Beck, Heinrich, Maczewsky,
			Kantartzis, Khajavikhan et~al.}]{pyrialakos.nm.21.634.2022}
		\bibinfo{author}{Pyrialakos\xfnm[ G.G.]}, \bibinfo{author}{Beck\xfnm[ J.]},
		\bibinfo{author}{Heinrich\xfnm[ M.]}, \bibinfo{author}{Maczewsky\xfnm[
			L.J.]}, \bibinfo{author}{Kantartzis\xfnm[ N.V.]},
		\bibinfo{author}{Khajavikhan\xfnm[ M.]}, et~al.
		\newblock \bibinfo{title}{Bimorphic {Floquet} topological insulators}.
		\newblock \bibinfo{journal}{Nat Mater}
		\bibinfo{year}{2022};\bibinfo{volume}{21}(\bibinfo{number}{6}):\bibinfo{pages}{634--639}.
		
		%Type = Article
		\bibitem[{Xie et~al.(2021)Xie, Wang, Zhang, Zhan, Jiang, Lu
			et~al.}]{xie.natrevphys.3.520.2021}
		\bibinfo{author}{Xie\xfnm[ B.]}, \bibinfo{author}{Wang\xfnm[ H.X.]},
		\bibinfo{author}{Zhang\xfnm[ X.]}, \bibinfo{author}{Zhan\xfnm[ P.]},
		\bibinfo{author}{Jiang\xfnm[ J.H.]}, \bibinfo{author}{Lu\xfnm[ M.]}, et~al.
		\newblock \bibinfo{title}{Higher-order band topology}.
		\newblock \bibinfo{journal}{Nat Rev Phys}
		\bibinfo{year}{2021};\bibinfo{volume}{3}(\bibinfo{number}{1}):\bibinfo{pages}{520–532}.
		
		%Type = Article
		\bibitem[{Garanovich et~al.(2012)Garanovich, Longhi, Sukhorukov and
			Kivshar}]{garanovich.pr.518.1.2012}
		\bibinfo{author}{Garanovich\xfnm[ I.L.]}, \bibinfo{author}{Longhi\xfnm[ S.]},
		\bibinfo{author}{Sukhorukov\xfnm[ A.A.]}, \bibinfo{author}{Kivshar\xfnm[
			Y.S.]}.
		\newblock \bibinfo{title}{Light propagation and localization in modulated
			photonic lattices and waveguides}.
		\newblock \bibinfo{journal}{Phys Rep}
		\bibinfo{year}{2012};\bibinfo{volume}{518}:\bibinfo{pages}{1--79}.
		
		%Type = Article
		\bibitem[{Kitagawa et~al.(2010)Kitagawa, Berg, Rudner and
			Demler}]{kitagawa.prb.82.235114.2010}
		\bibinfo{author}{Kitagawa\xfnm[ T.]}, \bibinfo{author}{Berg\xfnm[ E.]},
		\bibinfo{author}{Rudner\xfnm[ M.]}, \bibinfo{author}{Demler\xfnm[ E.]}.
		\newblock \bibinfo{title}{Topological characterization of periodically driven
			quantum systems}.
		\newblock \bibinfo{journal}{Phys Rev B}
		\bibinfo{year}{2010};\bibinfo{volume}{82}:\bibinfo{pages}{235114}.
		
		%Type = Article
		\bibitem[{Rudner et~al.(2013)Rudner, Lindner, Berg and
			Levin}]{rudner.prx.3.031005.2013}
		\bibinfo{author}{Rudner\xfnm[ M.S.]}, \bibinfo{author}{Lindner\xfnm[ N.H.]},
		\bibinfo{author}{Berg\xfnm[ E.]}, \bibinfo{author}{Levin\xfnm[ M.]}.
		\newblock \bibinfo{title}{Anomalous edge states and the bulk-edge
			correspondence for periodically driven two-dimensional systems}.
		\newblock \bibinfo{journal}{Phys Rev X}
		\bibinfo{year}{2013};\bibinfo{volume}{3}:\bibinfo{pages}{031005}.
		
		%Type = Article
		\bibitem[{Rudner and Lindner(2020)}]{rudner.nrp.2.229.2020}
		\bibinfo{author}{Rudner\xfnm[ M.S.]}, \bibinfo{author}{Lindner\xfnm[ N.H.]}.
		\newblock \bibinfo{title}{Band structure engineering and non-equilibrium
			dynamics in {F}loquet topological insulators}.
		\newblock \bibinfo{journal}{Nat Rev Phys}
		\bibinfo{year}{2020};\bibinfo{volume}{2}(\bibinfo{number}{5}):\bibinfo{pages}{229--244}.
		
		%Type = Article
		\bibitem[{Maczewsky et~al.(2017)Maczewsky, Zeuner, Nolte and
			Szameit}]{maczewsky.nc.8.13756.2017}
		\bibinfo{author}{Maczewsky\xfnm[ L.J.]}, \bibinfo{author}{Zeuner\xfnm[ J.M.]},
		\bibinfo{author}{Nolte\xfnm[ S.]}, \bibinfo{author}{Szameit\xfnm[ A.]}.
		\newblock \bibinfo{title}{Observation of photonic anomalous {F}loquet
			topological insulators}.
		\newblock \bibinfo{journal}{Nat Commun}
		\bibinfo{year}{2017};\bibinfo{volume}{8}:\bibinfo{pages}{13756}.
		
		%Type = Article
		\bibitem[{Mukherjee et~al.(2017)Mukherjee, Spracklen, Valiente, Andersson,
			\"Ohberg, Goldman et~al.}]{mukherjee.nc.8.13918.2017}
		\bibinfo{author}{Mukherjee\xfnm[ S.]}, \bibinfo{author}{Spracklen\xfnm[ A.]},
		\bibinfo{author}{Valiente\xfnm[ M.]}, \bibinfo{author}{Andersson\xfnm[ E.]},
		\bibinfo{author}{\"Ohberg\xfnm[ P.]}, \bibinfo{author}{Goldman\xfnm[ N.]},
		et~al.
		\newblock \bibinfo{title}{Experimental observation of anomalous topological
			edge modes in a slowly driven photonic lattice}.
		\newblock \bibinfo{journal}{Nat Commun}
		\bibinfo{year}{2017};\bibinfo{volume}{8}:\bibinfo{pages}{13918}.
		
		%Type = Article
		\bibitem[{Leykam et~al.(2016)Leykam, Rechtsman and
			Chong}]{leykam.prl.117.013902.2016}
		\bibinfo{author}{Leykam\xfnm[ D.]}, \bibinfo{author}{Rechtsman\xfnm[ M.C.]},
		\bibinfo{author}{Chong\xfnm[ Y.D.]}.
		\newblock \bibinfo{title}{Anomalous topological phases and unpaired {Dirac}
			cones in photonic {F}loquet topological insulators}.
		\newblock \bibinfo{journal}{Phys Rev Lett}
		\bibinfo{year}{2016};\bibinfo{volume}{117}:\bibinfo{pages}{013902}.
		
		%Type = Article
		\bibitem[{Asb\'oth et~al.(2014)Asb\'oth, Tarasinski and
			Delplace}]{asboth.prb.90.125143.2014}
		\bibinfo{author}{Asb\'oth\xfnm[ J.K.]}, \bibinfo{author}{Tarasinski\xfnm[ B.]},
		\bibinfo{author}{Delplace\xfnm[ P.]}.
		\newblock \bibinfo{title}{Chiral symmetry and bulk-boundary correspondence in
			periodically driven one-dimensional systems}.
		\newblock \bibinfo{journal}{Phys Rev B}
		\bibinfo{year}{2014};\bibinfo{volume}{90}:\bibinfo{pages}{125143}.
		
		%Type = Article
		\bibitem[{Dal~Lago et~al.(2015)Dal~Lago, Atala and
			Foa~Torres}]{dallago.pra.92.023624.2015}
		\bibinfo{author}{Dal~Lago\xfnm[ V.]}, \bibinfo{author}{Atala\xfnm[ M.]},
		\bibinfo{author}{Foa~Torres\xfnm[ L.E.F.]}.
		\newblock \bibinfo{title}{Floquet topological transitions in a driven
			one-dimensional topological insulator}.
		\newblock \bibinfo{journal}{Phys Rev A}
		\bibinfo{year}{2015};\bibinfo{volume}{92}:\bibinfo{pages}{023624}.
		
		%Type = Article
		\bibitem[{Fruchart(2016)}]{fruchart.prb.93.115429.2016}
		\bibinfo{author}{Fruchart\xfnm[ M.]}.
		\newblock \bibinfo{title}{Complex classes of periodically driven topological
			lattice systems}.
		\newblock \bibinfo{journal}{Phys Rev B}
		\bibinfo{year}{2016};\bibinfo{volume}{93}:\bibinfo{pages}{115429}.
		
		%Type = Article
		\bibitem[{Zhang et~al.(2017)Zhang, Kartashov, Li, Zhang, Zhang, Beli{\'c}
			et~al.}]{zhang.acs.4.2250.2017}
		\bibinfo{author}{Zhang\xfnm[ Y.Q.]}, \bibinfo{author}{Kartashov\xfnm[ Y.V.]},
		\bibinfo{author}{Li\xfnm[ F.]}, \bibinfo{author}{Zhang\xfnm[ Z.Y.]},
		\bibinfo{author}{Zhang\xfnm[ Y.P.]}, \bibinfo{author}{Beli{\'c}\xfnm[ M.R.]},
		et~al.
		\newblock \bibinfo{title}{Edge states in dynamical superlattices}.
		\newblock \bibinfo{journal}{ACS Photon}
		\bibinfo{year}{2017};\bibinfo{volume}{4}(\bibinfo{number}{9}):\bibinfo{pages}{2250--2256}.
		
		%Type = Article
		\bibitem[{Petr\'a\ifmmode~\check{c}\else \v{c}\fi{}ek and
			Kuzmiak(2020)}]{petracek.pra.101.033805.2020}
		\bibinfo{author}{Petr\'a\ifmmode~\check{c}\else \v{c}\fi{}ek\xfnm[ J.]},
		\bibinfo{author}{Kuzmiak\xfnm[ V.]}.
		\newblock \bibinfo{title}{Dynamics and transport properties of {Floquet}
			topological edge modes in coupled photonic waveguides}.
		\newblock \bibinfo{journal}{Phys Rev A}
		\bibinfo{year}{2020};\bibinfo{volume}{101}:\bibinfo{pages}{033805}.
		
		%Type = Article
		\bibitem[{J\"urgensen et~al.(2021)J\"urgensen, Mukherjee and
			Rechtsman}]{jurgensen.nature.596.63.2021}
		\bibinfo{author}{J\"urgensen\xfnm[ M.]}, \bibinfo{author}{Mukherjee\xfnm[ S.]},
		\bibinfo{author}{Rechtsman\xfnm[ M.C.]}.
		\newblock \bibinfo{title}{Quantized nonlinear {Thouless} pumping}.
		\newblock \bibinfo{journal}{Nature}
		\bibinfo{year}{2021};\bibinfo{volume}{596}(\bibinfo{number}{7870}):\bibinfo{pages}{63--67}.
		
		%Type = Article
		\bibitem[{Fu et~al.(2022{\natexlab{a}})Fu, Wang, Kartashov, Konotop and
			Ye}]{fu.prl.128.154101.2022}
		\bibinfo{author}{Fu\xfnm[ Q.]}, \bibinfo{author}{Wang\xfnm[ P.]},
		\bibinfo{author}{Kartashov\xfnm[ Y.V.]}, \bibinfo{author}{Konotop\xfnm[
			V.V.]}, \bibinfo{author}{Ye\xfnm[ F.]}.
		\newblock \bibinfo{title}{Nonlinear {Thouless} pumping: Solitons and transport
			breakdown}.
		\newblock \bibinfo{journal}{Phys Rev Lett}
		\bibinfo{year}{2022}{\natexlab{a}};\bibinfo{volume}{128}:\bibinfo{pages}{154101}.
		
		%Type = Article
		\bibitem[{Fu et~al.(2022{\natexlab{b}})Fu, Wang, Kartashov, Konotop and
			Ye}]{fu.prl.129.183901.2022}
		\bibinfo{author}{Fu\xfnm[ Q.]}, \bibinfo{author}{Wang\xfnm[ P.]},
		\bibinfo{author}{Kartashov\xfnm[ Y.V.]}, \bibinfo{author}{Konotop\xfnm[
			V.V.]}, \bibinfo{author}{Ye\xfnm[ F.]}.
		\newblock \bibinfo{title}{Two-dimensional nonlinear {Thouless} pumping of
			matter waves}.
		\newblock \bibinfo{journal}{Phys Rev Lett}
		\bibinfo{year}{2022}{\natexlab{b}};\bibinfo{volume}{129}:\bibinfo{pages}{183901}.
		
		%Type = Article
		\bibitem[{Cheng et~al.(2019)Cheng, Pan, Wang, Zhang, Yu, Gover
			et~al.}]{cheng.prl.122.173901.2019}
		\bibinfo{author}{Cheng\xfnm[ Q.]}, \bibinfo{author}{Pan\xfnm[ Y.]},
		\bibinfo{author}{Wang\xfnm[ H.]}, \bibinfo{author}{Zhang\xfnm[ C.]},
		\bibinfo{author}{Yu\xfnm[ D.]}, \bibinfo{author}{Gover\xfnm[ A.]}, et~al.
		\newblock \bibinfo{title}{Observation of anomalous $\ensuremath{\pi}$ modes in
			photonic {F}loquet engineering}.
		\newblock \bibinfo{journal}{Phys Rev Lett}
		\bibinfo{year}{2019};\bibinfo{volume}{122}:\bibinfo{pages}{173901}.
		
		%Type = Article
		\bibitem[{Wu et~al.(2021)Wu, Song, Gao, Chen, Zhu and
			Li}]{wu.prr.3.023211.2021}
		\bibinfo{author}{Wu\xfnm[ S.]}, \bibinfo{author}{Song\xfnm[ W.]},
		\bibinfo{author}{Gao\xfnm[ S.]}, \bibinfo{author}{Chen\xfnm[ Y.]},
		\bibinfo{author}{Zhu\xfnm[ S.]}, \bibinfo{author}{Li\xfnm[ T.]}.
		\newblock \bibinfo{title}{Floquet $\pi$ mode engineering in {non-Hermitia}n
			waveguide lattices}.
		\newblock \bibinfo{journal}{Phys Rev Research}
		\bibinfo{year}{2021};\bibinfo{volume}{3}:\bibinfo{pages}{023211}.
		
		%Type = Article
		\bibitem[{Song et~al.(2021)Song, Chen, Li, Gao, Wu, Chen
			et~al.}]{song.lpr.15.2000584.2021}
		\bibinfo{author}{Song\xfnm[ W.]}, \bibinfo{author}{Chen\xfnm[ Y.]},
		\bibinfo{author}{Li\xfnm[ H.]}, \bibinfo{author}{Gao\xfnm[ S.]},
		\bibinfo{author}{Wu\xfnm[ S.]}, \bibinfo{author}{Chen\xfnm[ C.]}, et~al.
		\newblock \bibinfo{title}{Gauge-induced {Floquet} topological states in
			photonic waveguides}.
		\newblock \bibinfo{journal}{Laser Photon Rev}
		\bibinfo{year}{2021};\bibinfo{volume}{15}(\bibinfo{number}{8}):\bibinfo{pages}{2000584}.
		
		%Type = Article
		\bibitem[{Sidorenko et~al.(2022)Sidorenko, Fedorova~(Cherpakova), Abouelela,
			Kroha and Linden}]{sidorenko.prr.4.033184.2022}
		\bibinfo{author}{Sidorenko\xfnm[ A.]},
		\bibinfo{author}{Fedorova~(Cherpakova)\xfnm[ Z.]},
		\bibinfo{author}{Abouelela\xfnm[ A.]}, \bibinfo{author}{Kroha\xfnm[ J.]},
		\bibinfo{author}{Linden\xfnm[ S.]}.
		\newblock \bibinfo{title}{Real- and {Fourier}-space observation of the
			anomalous $\ensuremath{\pi}$ mode in {Floquet} engineered plasmonic waveguide
			arrays}.
		\newblock \bibinfo{journal}{Phys Rev Res}
		\bibinfo{year}{2022};\bibinfo{volume}{4}:\bibinfo{pages}{033184}.
		
		%Type = Article
		\bibitem[{Ota et~al.(2019)Ota, Liu, Katsumi, Watanabe, Wakabayashi, Arakawa
			et~al.}]{ota.optica.6.786.2019}
		\bibinfo{author}{Ota\xfnm[ Y.]}, \bibinfo{author}{Liu\xfnm[ F.]},
		\bibinfo{author}{Katsumi\xfnm[ R.]}, \bibinfo{author}{Watanabe\xfnm[ K.]},
		\bibinfo{author}{Wakabayashi\xfnm[ K.]}, \bibinfo{author}{Arakawa\xfnm[ Y.]},
		et~al.
		\newblock \bibinfo{title}{Photonic crystal nanocavity based on a topological
			corner state}.
		\newblock \bibinfo{journal}{Optica}
		\bibinfo{year}{2019};\bibinfo{volume}{6}(\bibinfo{number}{6}):\bibinfo{pages}{786--789}.
		
		%Type = Article
		\bibitem[{Xie et~al.(2020)Xie, Zhang, He, Wu, Dang, Peng
			et~al.}]{xie.lpr.14.1900425.2020}
		\bibinfo{author}{Xie\xfnm[ X.]}, \bibinfo{author}{Zhang\xfnm[ W.]},
		\bibinfo{author}{He\xfnm[ X.]}, \bibinfo{author}{Wu\xfnm[ S.]},
		\bibinfo{author}{Dang\xfnm[ J.]}, \bibinfo{author}{Peng\xfnm[ K.]}, et~al.
		\newblock \bibinfo{title}{Cavity quantum electrodynamics with second-order
			topological corner state}.
		\newblock \bibinfo{journal}{Laser Photon Rev}
		\bibinfo{year}{2020};\bibinfo{volume}{14}(\bibinfo{number}{8}):\bibinfo{pages}{1900425}.
		
		%Type = Article
		\bibitem[{Zhang et~al.(2020)Zhang, Xie, Hao, Dang, Xiao, Shi
			et~al.}]{zhang.light.9.109.2020}
		\bibinfo{author}{Zhang\xfnm[ W.]}, \bibinfo{author}{Xie\xfnm[ X.]},
		\bibinfo{author}{Hao\xfnm[ H.]}, \bibinfo{author}{Dang\xfnm[ J.]},
		\bibinfo{author}{Xiao\xfnm[ S.]}, \bibinfo{author}{Shi\xfnm[ S.]}, et~al.
		\newblock \bibinfo{title}{Low-threshold topological nanolasers based on the
			second-order corner state}.
		\newblock \bibinfo{journal}{Light Sci Appl}
		\bibinfo{year}{2020};\bibinfo{volume}{9}(\bibinfo{number}{1}):\bibinfo{pages}{109}.
		
		%Type = Article
		\bibitem[{Kim et~al.(2020)Kim, Hwang, Smirnova, Jeong, Kivshar and
			Park}]{kim.nc.11.5758.2020}
		\bibinfo{author}{Kim\xfnm[ H.R.]}, \bibinfo{author}{Hwang\xfnm[ M.S.]},
		\bibinfo{author}{Smirnova\xfnm[ D.]}, \bibinfo{author}{Jeong\xfnm[ K.Y.]},
		\bibinfo{author}{Kivshar\xfnm[ Y.]}, \bibinfo{author}{Park\xfnm[ H.G.]}.
		\newblock \bibinfo{title}{Multipolar lasing modes from topological corner
			states}.
		\newblock \bibinfo{journal}{Nat Commun}
		\bibinfo{year}{2020};\bibinfo{volume}{11}(\bibinfo{number}{1}):\bibinfo{pages}{5758}.
		
		%Type = Article
		\bibitem[{Zhong et~al.(2021{\natexlab{a}})Zhong, Kartashov, Szameit, Li, Liu
			and Zhang}]{zhong.apl.6.040802.2021}
		\bibinfo{author}{Zhong\xfnm[ H.]}, \bibinfo{author}{Kartashov\xfnm[ Y.V.]},
		\bibinfo{author}{Szameit\xfnm[ A.]}, \bibinfo{author}{Li\xfnm[ Y.]},
		\bibinfo{author}{Liu\xfnm[ C.]}, \bibinfo{author}{Zhang\xfnm[ Y.]}.
		\newblock \bibinfo{title}{Theory of topological corner state laser in {Kagome}
			waveguide arrays}.
		\newblock \bibinfo{journal}{APL Photon}
		\bibinfo{year}{2021}{\natexlab{a}};\bibinfo{volume}{6}(\bibinfo{number}{4}):\bibinfo{pages}{040802}.
		
		%Type = Article
		\bibitem[{Ota et~al.(2020)Ota, Takata, Ozawa, Amo, Jia, Kante
			et~al.}]{ota.nano.9.547.2020}
		\bibinfo{author}{Ota\xfnm[ Y.]}, \bibinfo{author}{Takata\xfnm[ K.]},
		\bibinfo{author}{Ozawa\xfnm[ T.]}, \bibinfo{author}{Amo\xfnm[ A.]},
		\bibinfo{author}{Jia\xfnm[ Z.]}, \bibinfo{author}{Kante\xfnm[ B.]}, et~al.
		\newblock \bibinfo{title}{Active topological photonics}.
		\newblock \bibinfo{journal}{Nanophoton}
		\bibinfo{year}{2020};\bibinfo{volume}{9}(\bibinfo{number}{3}):\bibinfo{pages}{547--567}.
		
		%Type = Article
		\bibitem[{Bloch et~al.(2022)Bloch, Cavalleri, Galitski, Hafezi and
			Rubio}]{bloch.nature.606.41.2022}
		\bibinfo{author}{Bloch\xfnm[ J.]}, \bibinfo{author}{Cavalleri\xfnm[ A.]},
		\bibinfo{author}{Galitski\xfnm[ V.]}, \bibinfo{author}{Hafezi\xfnm[ M.]},
		\bibinfo{author}{Rubio\xfnm[ A.]}.
		\newblock \bibinfo{title}{Strongly correlated electron-photon systems}.
		\newblock \bibinfo{journal}{Nature}
		\bibinfo{year}{2022};\bibinfo{volume}{606}(\bibinfo{number}{7912}):\bibinfo{pages}{41--48}.
		
		%Type = Article
		\bibitem[{Cheng et~al.(2022)Cheng, Bomantara, Xue, Zhu, Gong and
			Zhang}]{cheng.prl.129.254301.2022}
		\bibinfo{author}{Cheng\xfnm[ Z.]}, \bibinfo{author}{Bomantara\xfnm[ R.W.]},
		\bibinfo{author}{Xue\xfnm[ H.]}, \bibinfo{author}{Zhu\xfnm[ W.]},
		\bibinfo{author}{Gong\xfnm[ J.]}, \bibinfo{author}{Zhang\xfnm[ B.]}.
		\newblock \bibinfo{title}{Observation of $\ensuremath{\pi}/2$ modes in an
			acoustic {Floquet} system}.
		\newblock \bibinfo{journal}{Phys Rev Lett}
		\bibinfo{year}{2022};\bibinfo{volume}{129}:\bibinfo{pages}{254301}.
		
		%Type = Article
		\bibitem[{Zhu et~al.(2022)Zhu, Xue, Gong, Chong and Zhang}]{zhu.nc.13.11.2022}
		\bibinfo{author}{Zhu\xfnm[ W.]}, \bibinfo{author}{Xue\xfnm[ H.]},
		\bibinfo{author}{Gong\xfnm[ J.]}, \bibinfo{author}{Chong\xfnm[ Y.]},
		\bibinfo{author}{Zhang\xfnm[ B.]}.
		\newblock \bibinfo{title}{Time-periodic corner states from {Floquet}
			higher-order topology}.
		\newblock \bibinfo{journal}{Nat Commun}
		\bibinfo{year}{2022};\bibinfo{volume}{13}(\bibinfo{number}{1}):\bibinfo{pages}{11}.
		
		%Type = Article
		\bibitem[{Smirnova et~al.(2020)Smirnova, Leykam, Chong and
			Kivshar}]{smirnova.apr.7.021306.2020}
		\bibinfo{author}{Smirnova\xfnm[ D.]}, \bibinfo{author}{Leykam\xfnm[ D.]},
		\bibinfo{author}{Chong\xfnm[ Y.]}, \bibinfo{author}{Kivshar\xfnm[ Y.]}.
		\newblock \bibinfo{title}{Nonlinear topological photonics}.
		\newblock \bibinfo{journal}{Appl Phys Rev}
		\bibinfo{year}{2020};\bibinfo{volume}{7}(\bibinfo{number}{2}):\bibinfo{pages}{021306}.
		
		%Type = Article
		\bibitem[{Lumer et~al.(2016)Lumer, Rechtsman, Plotnik and
			Segev}]{lumer.pra.94.021801.2016}
		\bibinfo{author}{Lumer\xfnm[ Y.]}, \bibinfo{author}{Rechtsman\xfnm[ M.C.]},
		\bibinfo{author}{Plotnik\xfnm[ Y.]}, \bibinfo{author}{Segev\xfnm[ M.]}.
		\newblock \bibinfo{title}{Instability of bosonic topological edge states in the
			presence of interactions}.
		\newblock \bibinfo{journal}{Phys Rev A}
		\bibinfo{year}{2016};\bibinfo{volume}{94}:\bibinfo{pages}{021801}.
		
		%Type = Article
		\bibitem[{Leykam and Chong(2016)}]{leykam.prl.117.143901.2016}
		\bibinfo{author}{Leykam\xfnm[ D.]}, \bibinfo{author}{Chong\xfnm[ Y.D.]}.
		\newblock \bibinfo{title}{Edge solitons in nonlinear-photonic topological
			insulators}.
		\newblock \bibinfo{journal}{Phys Rev Lett}
		\bibinfo{year}{2016};\bibinfo{volume}{117}:\bibinfo{pages}{143901}.
		
		%Type = Article
		\bibitem[{Kartashov and Skryabin(2016)}]{kartashov.optica.3.1228.2016}
		\bibinfo{author}{Kartashov\xfnm[ Y.V.]}, \bibinfo{author}{Skryabin\xfnm[
			D.V.]}.
		\newblock \bibinfo{title}{Modulational instability and solitary waves in
			polariton topological insulators}.
		\newblock \bibinfo{journal}{Optica}
		\bibinfo{year}{2016};\bibinfo{volume}{3}(\bibinfo{number}{11}):\bibinfo{pages}{1228--1236}.
		
		%Type = Article
		\bibitem[{Kartashov and Skryabin(2017)}]{kartashov.prl.119.253904.2017}
		\bibinfo{author}{Kartashov\xfnm[ Y.V.]}, \bibinfo{author}{Skryabin\xfnm[
			D.V.]}.
		\newblock \bibinfo{title}{Bistable topological insulator with
			exciton-polaritons}.
		\newblock \bibinfo{journal}{Phys Rev Lett}
		\bibinfo{year}{2017};\bibinfo{volume}{119}:\bibinfo{pages}{253904}.
		
		%Type = Article
		\bibitem[{Mandal et~al.(2023)Mandal, Liu and Zhang}]{mandal.acs.10.147.2023}
		\bibinfo{author}{Mandal\xfnm[ S.]}, \bibinfo{author}{Liu\xfnm[ G.G.]},
		\bibinfo{author}{Zhang\xfnm[ B.]}.
		\newblock \bibinfo{title}{Topology with memory in nonlinear driven-dissipative
			photonic lattices}.
		\newblock \bibinfo{journal}{ACS Photon}
		\bibinfo{year}{2023};\bibinfo{volume}{10}(\bibinfo{number}{1}):\bibinfo{pages}{147--154}.
		
		%Type = Article
		\bibitem[{Pernet et~al.(2022)Pernet, St-Jean, Solnyshkov, Malpuech, Zambon,
			Fontaine et~al.}]{pernet.nphys.18.678.2022}
		\bibinfo{author}{Pernet\xfnm[ N.]}, \bibinfo{author}{St-Jean\xfnm[ P.]},
		\bibinfo{author}{Solnyshkov\xfnm[ D.D.]}, \bibinfo{author}{Malpuech\xfnm[
			G.]}, \bibinfo{author}{Zambon\xfnm[ N.C.]}, \bibinfo{author}{Fontaine\xfnm[
			Q.]}, et~al.
		\newblock \bibinfo{title}{Gap solitons in a one-dimensional driven-dissipative
			topological lattice}.
		\newblock \bibinfo{journal}{Nat Phys}
		\bibinfo{year}{2022};\bibinfo{volume}{18}(\bibinfo{number}{1}):\bibinfo{pages}{678--684}.
		
		%Type = Article
		\bibitem[{Maczewsky et~al.(2020)Maczewsky, Heinrich, Kremer, Ivanov, Ehrhardt,
			Martinez et~al.}]{maczewsky.science.370.701.2020}
		\bibinfo{author}{Maczewsky\xfnm[ L.J.]}, \bibinfo{author}{Heinrich\xfnm[ M.]},
		\bibinfo{author}{Kremer\xfnm[ M.]}, \bibinfo{author}{Ivanov\xfnm[ S.K.]},
		\bibinfo{author}{Ehrhardt\xfnm[ M.]}, \bibinfo{author}{Martinez\xfnm[ F.]},
		et~al.
		\newblock \bibinfo{title}{Nonlinearity-induced photonic topological insulator}.
		\newblock \bibinfo{journal}{Science}
		\bibinfo{year}{2020};\bibinfo{volume}{370}(\bibinfo{number}{6517}):\bibinfo{pages}{701--704}.
		
		%Type = Article
		\bibitem[{Lumer et~al.(2013)Lumer, Plotnik, Rechtsman and
			Segev}]{lumer.prl.111.243905.2013}
		\bibinfo{author}{Lumer\xfnm[ Y.]}, \bibinfo{author}{Plotnik\xfnm[ Y.]},
		\bibinfo{author}{Rechtsman\xfnm[ M.C.]}, \bibinfo{author}{Segev\xfnm[ M.]}.
		\newblock \bibinfo{title}{Self-localized states in photonic topological
			insulators}.
		\newblock \bibinfo{journal}{Phys Rev Lett}
		\bibinfo{year}{2013};\bibinfo{volume}{111}:\bibinfo{pages}{243905}.
		
		%Type = Article
		\bibitem[{Mukherjee and Rechtsman(2020)}]{mukherjee.science.368.856.2020}
		\bibinfo{author}{Mukherjee\xfnm[ S.]}, \bibinfo{author}{Rechtsman\xfnm[ M.C.]}.
		\newblock \bibinfo{title}{Observation of {F}loquet solitons in a topological
			bandgap}.
		\newblock \bibinfo{journal}{Science}
		\bibinfo{year}{2020};\bibinfo{volume}{368}(\bibinfo{number}{6493}):\bibinfo{pages}{856--859}.
		
		%Type = Article
		\bibitem[{Ablowitz and Cole(2017)}]{ablowitz.pra.96.043868.2017}
		\bibinfo{author}{Ablowitz\xfnm[ M.J.]}, \bibinfo{author}{Cole\xfnm[ J.T.]}.
		\newblock \bibinfo{title}{Tight-binding methods for general longitudinally
			driven photonic lattices: Edge states and solitons}.
		\newblock \bibinfo{journal}{Phys Rev A}
		\bibinfo{year}{2017};\bibinfo{volume}{96}:\bibinfo{pages}{043868}.
		
		%Type = Article
		\bibitem[{Gulevich et~al.(2017)Gulevich, Yudin, Skryabin, Iorsh and
			Shelykh}]{gulevich.sr.7.1780.2017}
		\bibinfo{author}{Gulevich\xfnm[ D.R.]}, \bibinfo{author}{Yudin\xfnm[ D.]},
		\bibinfo{author}{Skryabin\xfnm[ D.V.]}, \bibinfo{author}{Iorsh\xfnm[ I.V.]},
		\bibinfo{author}{Shelykh\xfnm[ I.A.]}.
		\newblock \bibinfo{title}{Exploring nonlinear topological states of matter with
			exciton-polaritons: Edge solitons in kagome lattice}.
		\newblock \bibinfo{journal}{Sci Rep}
		\bibinfo{year}{2017};\bibinfo{volume}{7}(\bibinfo{number}{1}):\bibinfo{pages}{1780}.
		
		%Type = Article
		\bibitem[{Li et~al.(2018)Li, Ye, Chen, Kartashov, Ferrando, Torner
			et~al.}]{li.prb.97.081103.2018}
		\bibinfo{author}{Li\xfnm[ C.]}, \bibinfo{author}{Ye\xfnm[ F.]},
		\bibinfo{author}{Chen\xfnm[ X.]}, \bibinfo{author}{Kartashov\xfnm[ Y.V.]},
		\bibinfo{author}{Ferrando\xfnm[ A.]}, \bibinfo{author}{Torner\xfnm[ L.]},
		et~al.
		\newblock \bibinfo{title}{Lieb polariton topological insulators}.
		\newblock \bibinfo{journal}{Phys Rev B}
		\bibinfo{year}{2018};\bibinfo{volume}{97}:\bibinfo{pages}{081103}.
		
		%Type = Article
		\bibitem[{Smirnova et~al.(2019)Smirnova, Smirnov, Leykam and
			Kivshar}]{smirnova.lpr.13.1900223.2019}
		\bibinfo{author}{Smirnova\xfnm[ D.A.]}, \bibinfo{author}{Smirnov\xfnm[ L.A.]},
		\bibinfo{author}{Leykam\xfnm[ D.]}, \bibinfo{author}{Kivshar\xfnm[ Y.S.]}.
		\newblock \bibinfo{title}{Topological edge states and gap solitons in the
			nonlinear {D}irac model}.
		\newblock \bibinfo{journal}{Laser Photon Rev}
		\bibinfo{year}{2019};\bibinfo{volume}{13}(\bibinfo{number}{12}):\bibinfo{pages}{1900223}.
		
		%Type = Article
		\bibitem[{Zhang et~al.(2019)Zhang, Chen, Kartashov, Konotop and
			Ye}]{zhang.prl.123.254103.2019}
		\bibinfo{author}{Zhang\xfnm[ W.]}, \bibinfo{author}{Chen\xfnm[ X.]},
		\bibinfo{author}{Kartashov\xfnm[ Y.V.]}, \bibinfo{author}{Konotop\xfnm[
			V.V.]}, \bibinfo{author}{Ye\xfnm[ F.]}.
		\newblock \bibinfo{title}{Coupling of edge states and topological {B}ragg
			solitons}.
		\newblock \bibinfo{journal}{Phys Rev Lett}
		\bibinfo{year}{2019};\bibinfo{volume}{123}:\bibinfo{pages}{254103}.
		
		%Type = Article
		\bibitem[{Ivanov et~al.(2020)Ivanov, Kartashov, Szameit, Torner and
			Konotop}]{ivanov.acs.7.735.2020}
		\bibinfo{author}{Ivanov\xfnm[ S.K.]}, \bibinfo{author}{Kartashov\xfnm[ Y.V.]},
		\bibinfo{author}{Szameit\xfnm[ A.]}, \bibinfo{author}{Torner\xfnm[ L.]},
		\bibinfo{author}{Konotop\xfnm[ V.V.]}.
		\newblock \bibinfo{title}{Vector topological edge solitons in {F}loquet
			insulators}.
		\newblock \bibinfo{journal}{ACS Photon}
		\bibinfo{year}{2020};\bibinfo{volume}{7}(\bibinfo{number}{3}):\bibinfo{pages}{735--745}.
		
		%Type = Article
		\bibitem[{Zhong et~al.(2021{\natexlab{b}})Zhong, Xia, Zhang, Li, Song, Liu
			et~al.}]{zhong.ap.3.056001.2021}
		\bibinfo{author}{Zhong\xfnm[ H.]}, \bibinfo{author}{Xia\xfnm[ S.]},
		\bibinfo{author}{Zhang\xfnm[ Y.]}, \bibinfo{author}{Li\xfnm[ Y.]},
		\bibinfo{author}{Song\xfnm[ D.]}, \bibinfo{author}{Liu\xfnm[ C.]}, et~al.
		\newblock \bibinfo{title}{{Nonlinear topological valley {Hall} edge states
				arising from type-{II} {Dirac} cones}}.
		\newblock \bibinfo{journal}{Adv Photon}
		\bibinfo{year}{2021}{\natexlab{b}};\bibinfo{volume}{3}(\bibinfo{number}{5}):\bibinfo{pages}{056001}.
		
		%Type = Article
		\bibitem[{Smirnova et~al.(2021)Smirnova, Smirnov, Smolina, Angelakis and
			Leykam}]{smirnova.prr.3.043027.2021}
		\bibinfo{author}{Smirnova\xfnm[ D.A.]}, \bibinfo{author}{Smirnov\xfnm[ L.A.]},
		\bibinfo{author}{Smolina\xfnm[ E.O.]}, \bibinfo{author}{Angelakis\xfnm[
			D.G.]}, \bibinfo{author}{Leykam\xfnm[ D.]}.
		\newblock \bibinfo{title}{Gradient catastrophe of nonlinear photonic
			valley-{Hall} edge pulses}.
		\newblock \bibinfo{journal}{Phys Rev Research}
		\bibinfo{year}{2021};\bibinfo{volume}{3}:\bibinfo{pages}{043027}.
		
		%Type = Article
		\bibitem[{Mukherjee and Rechtsman(2021)}]{mukherjee.prx.11.041057.2021}
		\bibinfo{author}{Mukherjee\xfnm[ S.]}, \bibinfo{author}{Rechtsman\xfnm[ M.C.]}.
		\newblock \bibinfo{title}{Observation of unidirectional solitonlike edge states
			in nonlinear {Floquet} topological insulators}.
		\newblock \bibinfo{journal}{Phys Rev X}
		\bibinfo{year}{2021};\bibinfo{volume}{11}:\bibinfo{pages}{041057}.
		
		%Type = Article
		\bibitem[{Xia et~al.(2020)Xia, Juki\'c, Wang, Smirnova, Smirnov, Tang
			et~al.}]{xia.light.9.147.2020}
		\bibinfo{author}{Xia\xfnm[ S.]}, \bibinfo{author}{Juki\'c\xfnm[ D.]},
		\bibinfo{author}{Wang\xfnm[ N.]}, \bibinfo{author}{Smirnova\xfnm[ D.]},
		\bibinfo{author}{Smirnov\xfnm[ L.]}, \bibinfo{author}{Tang\xfnm[ L.]}, et~al.
		\newblock \bibinfo{title}{Nontrivial coupling of light into a defect: the
			interplay of nonlinearity and topology}.
		\newblock \bibinfo{journal}{Light Sci Appl}
		\bibinfo{year}{2020};\bibinfo{volume}{9}(\bibinfo{number}{1}):\bibinfo{pages}{147}.
		
		%Type = Article
		\bibitem[{Guo et~al.(2020)Guo, Xia, Wang, Song, Chen and
			Yang}]{guo.ol.45.6466.2020}
		\bibinfo{author}{Guo\xfnm[ M.]}, \bibinfo{author}{Xia\xfnm[ S.]},
		\bibinfo{author}{Wang\xfnm[ N.]}, \bibinfo{author}{Song\xfnm[ D.]},
		\bibinfo{author}{Chen\xfnm[ Z.]}, \bibinfo{author}{Yang\xfnm[ J.]}.
		\newblock \bibinfo{title}{Weakly nonlinear topological gap solitons in
			{Su-Schrieffer-Heeger} photonic lattices}.
		\newblock \bibinfo{journal}{Opt Lett}
		\bibinfo{year}{2020};\bibinfo{volume}{45}(\bibinfo{number}{23}):\bibinfo{pages}{6466--6469}.
		
		%Type = Article
		\bibitem[{Kartashov et~al.(2022)Kartashov, Arkhipova, Zhuravitskii, Skryabin,
			Dyakonov, Kalinkin et~al.}]{kartashov.prl.128.093901.2022}
		\bibinfo{author}{Kartashov\xfnm[ Y.V.]}, \bibinfo{author}{Arkhipova\xfnm[
			A.A.]}, \bibinfo{author}{Zhuravitskii\xfnm[ S.A.]},
		\bibinfo{author}{Skryabin\xfnm[ N.N.]}, \bibinfo{author}{Dyakonov\xfnm[
			I.V.]}, \bibinfo{author}{Kalinkin\xfnm[ A.A.]}, et~al.
		\newblock \bibinfo{title}{Observation of edge solitons in topological trimer
			arrays}.
		\newblock \bibinfo{journal}{Phys Rev Lett}
		\bibinfo{year}{2022};\bibinfo{volume}{128}:\bibinfo{pages}{093901}.
		
		%Type = Article
		\bibitem[{Zangeneh-Nejad and Fleury(2019)}]{zangeneh.prl.123.053902.2019}
		\bibinfo{author}{Zangeneh-Nejad\xfnm[ F.]}, \bibinfo{author}{Fleury\xfnm[ R.]}.
		\newblock \bibinfo{title}{Nonlinear second-order topological insulators}.
		\newblock \bibinfo{journal}{Phys Rev Lett}
		\bibinfo{year}{2019};\bibinfo{volume}{123}:\bibinfo{pages}{053902}.
		
		%Type = Article
		\bibitem[{Kirsch et~al.(2021)Kirsch, Zhang, Kremer, Maczewsky, Ivanov,
			Kartashov et~al.}]{kirsch.np.17.995.2021}
		\bibinfo{author}{Kirsch\xfnm[ M.S.]}, \bibinfo{author}{Zhang\xfnm[ Y.]},
		\bibinfo{author}{Kremer\xfnm[ M.]}, \bibinfo{author}{Maczewsky\xfnm[ L.J.]},
		\bibinfo{author}{Ivanov\xfnm[ S.K.]}, \bibinfo{author}{Kartashov\xfnm[
			Y.V.]}, et~al.
		\newblock \bibinfo{title}{Nonlinear second-order photonic topological
			insulators}.
		\newblock \bibinfo{journal}{Nat Phys}
		\bibinfo{year}{2021};\bibinfo{volume}{17}(\bibinfo{number}{9}):\bibinfo{pages}{995--1000}.
		
		%Type = Article
		\bibitem[{Hu et~al.(2021)Hu, Bongiovanni, Juki\'c, Jajti\'c, Xia, Song
			et~al.}]{hu.light.10.164.2021}
		\bibinfo{author}{Hu\xfnm[ Z.]}, \bibinfo{author}{Bongiovanni\xfnm[ D.]},
		\bibinfo{author}{Juki\'c\xfnm[ D.]}, \bibinfo{author}{Jajti\'c\xfnm[ E.]},
		\bibinfo{author}{Xia\xfnm[ S.]}, \bibinfo{author}{Song\xfnm[ D.]}, et~al.
		\newblock \bibinfo{title}{Nonlinear control of photonic higher-order
			topological bound states in the continuum}.
		\newblock \bibinfo{journal}{Light Sci Appl}
		\bibinfo{year}{2021};\bibinfo{volume}{10}(\bibinfo{number}{1}):\bibinfo{pages}{164}.
		
		%Type = Article
		\bibitem[{Zhong et~al.(2023)Zhong, Kartashov, Li and
			Zhang}]{zhong.pra.107.L021502.2023}
		\bibinfo{author}{Zhong\xfnm[ H.]}, \bibinfo{author}{Kartashov\xfnm[ Y.V.]},
		\bibinfo{author}{Li\xfnm[ Y.]}, \bibinfo{author}{Zhang\xfnm[ Y.]}.
		\newblock \bibinfo{title}{$\ensuremath{\pi}$-mode solitons in photonic
			{Floquet} lattices}.
		\newblock \bibinfo{journal}{Phys Rev A}
		\bibinfo{year}{2023};\bibinfo{volume}{107}:\bibinfo{pages}{L021502}.
		
		%Type = Article
		\bibitem[{Tan et~al.(2021)Tan, Wang, Xu and Qiu}]{tan.ap.3.024002.2021}
		\bibinfo{author}{Tan\xfnm[ D.]}, \bibinfo{author}{Wang\xfnm[ Z.]},
		\bibinfo{author}{Xu\xfnm[ B.]}, \bibinfo{author}{Qiu\xfnm[ J.]}.
		\newblock \bibinfo{title}{Photonic circuits written by femtosecond laser in
			glass: improved fabrication and recent progress in photonic devices}.
		\newblock \bibinfo{journal}{Adv Photon}
		\bibinfo{year}{2021};\bibinfo{volume}{3}(\bibinfo{number}{11}):\bibinfo{pages}{024002}.
		
		%Type = Article
		\bibitem[{Li et~al.(2022)Li, Kong and Chen}]{li.ap.4.024002.2022}
		\bibinfo{author}{Li\xfnm[ L.]}, \bibinfo{author}{Kong\xfnm[ W.]},
		\bibinfo{author}{Chen\xfnm[ F.]}.
		\newblock \bibinfo{title}{Femtosecond laser-inscribed optical waveguides in
			dielectric crystals: a concise review and recent advances}.
		\newblock \bibinfo{journal}{Adv Photon}
		\bibinfo{year}{2022};\bibinfo{volume}{4}(\bibinfo{number}{11}):\bibinfo{pages}{024002}.
		
		%Type = Article
		\bibitem[{Lin and Hong(2021)}]{lin.us.2021.9783514.2021}
		\bibinfo{author}{Lin\xfnm[ Z.]}, \bibinfo{author}{Hong\xfnm[ M.]}.
		\newblock \bibinfo{title}{Femtosecond laser precision engineering: From micron,
			submicron, to nanoscale}.
		\newblock \bibinfo{journal}{Ultrafast Sci}
		\bibinfo{year}{2021};\bibinfo{volume}{2021}:\bibinfo{pages}{9783514}.
		
		%Type = Article
		\bibitem[{Sakaguchi et~al.(2014)Sakaguchi, Li and
			Malomed}]{sakaguchi.pre.89.032920.2014}
		\bibinfo{author}{Sakaguchi\xfnm[ H.]}, \bibinfo{author}{Li\xfnm[ B.]},
		\bibinfo{author}{Malomed\xfnm[ B.A.]}.
		\newblock \bibinfo{title}{Creation of two-dimensional composite solitons in
			spin-orbit-coupled self-attractive {Bose-Einstein} condensates in free
			space}.
		\newblock \bibinfo{journal}{Phys Rev E}
		\bibinfo{year}{2014};\bibinfo{volume}{89}:\bibinfo{pages}{032920}.
		
	\end{thebibliography}
	
	
	
\end{document}


\newpage

\newpage

\appendix

%\renewcommand{\thesection}{S\arabic{section}}
%\renewcommand{\thesection}{S\arabic{section}}

\section*{Supplementary Material}

This document provides details of ``Single Proxy Synthetic Control.''	In Section \ref{sec:supp:Detail}, we provide details of the paper. In Section \ref{sec:supp:nonparametric full}, we provide details on the nonparametric single proxy synthetic control framework. Lastly, we provide the proofs of the results in Section \ref{sec:supp:proof}. 

\section{Details of the Paper} \label{sec:supp:Detail}

\subsection{Inconsistency of the Ordinary Least Squares Estimator}	\label{sec:supp:OLS}

Following \citet{FermanPinto2021}, we provide details on why synthetic controls obtained from the ordinary least squares (OLS) may be inconsistent. For simplicity, we consider an unconstrained case, in which equation \eqref{eq-OLS} of the main paper reduces to:
\begin{align}
\tag{\ref{eq-OLS}}
\widehat{\bgamma}_{\text{OLS}}
=
\argmin_{\bgamma}
Q(\bgamma)
\ , &&
Q(\bgamma)
= 
\frac{1}{T_0}
\sum_{t=1}^{T_0}
\big(
Y_t - \bW_{\D t}\T \bgamma
\big)^2
\ .
\end{align}
For a fixed $\bgamma=(\gamma_{\D_1},\ldots,\gamma_{\D_d}) \T$, the probability limit of $Q(\bgamma)$ as $T_0 \rightarrow \infty$  is given as follows: 
\begin{align}
\nonumber
\lim_{T_0 \rightarrow \infty}
Q(\bgamma)
&=
\lim_{T_0 \rightarrow \infty}
\frac{1}{T_0}
\sum_{t=1}^{T_0} 
\big(
Y_t - 
\bW_{\D t}\T \bgamma
\big)^2
\\
\nonumber
& 
=
\lim_{T_0 \rightarrow \infty}
\frac{1}{T_0}
\sum_{t=1}^{T_0} \Big\{
\bW_{\D t}\T (\bgamma^\dagger - \bgamma)
+ 
e_{0t} 
-
\sum_{i \in \D} \gamma_i^{\dagger} e_{it} 
\Big\}^2
\\
\nonumber
& 
=
\lim_{T_0 \rightarrow \infty}
\frac{1}{T_0}
\sum_{t=1}^{T_0} 
\bigg\{
\sum_{i \in \D}
(\gamma_i^\dagger - \gamma_i) \bmu_i\T 
\blambda_t
+ 
e_{0t}
-
\sum_{i=1}^{N} \gamma_i e_{it} 		
\bigg\} ^2
\\
\label{eq-problimit}
&
=
\sum_{i \in \D} (\gamma_i^\dagger - \gamma_i)^2 \bmu_i\T \Lambda	\bmu_i
+
\bigg( 1+\sum_{i \in \D} \gamma_i^2 \bigg) \sigma_e^2
\ .
\end{align}
where the second and third lines hold from \eqref{eq-SC Equation} and \eqref{eq-IFEM} of the main paper, respectively, which are restated below:
\begin{align}	\tag{\ref{eq-IFEM}}
Y_{t}
&
=
\text{\makebox[1.25cm]{$ \tau_{t}^* A_t + $}}
\bmu_{0} \T
\blambda_t
+
e_{0t}
\ ,
&&
\EXP \big( e_{0t} \cond \blambda_t ) = 0
\nonumber
\\
W_{it}
&
=
\text{\makebox[1.25cm]{}}
\bmu_{i} \T
\blambda_t
+
e_{it}
\ , 
&&
\EXP \big( e_{it} \cond \blambda_t ) = 0
\ , 
&&
i=1,\ldots,N \ ,
&&
t=1,\ldots,T \ .
\nonumber
\end{align}
and
\begin{align}		\tag{\ref{eq-SC Equation}}
\potY{t}{0}
=
\bW_{\D t} \T \bgamma^\dagger
+ 
e_{0t} 
-
\sum_{i \in \D} \gamma_i^{\dagger} e_{it} 
\ , 
\quad \quad
t=1,\ldots, T
\ .
\end{align}
The last line holds under the following additional assumptions on $\blambda_t$ and $e_{it}$ as $T_0 \rightarrow \infty$:
\begin{align*}
& 
\frac{1}{T_0} \sum_{t=1}^{T_0} \blambda_t = o_P(1) \ ,
&&
\frac{1}{T_0} \sum_{t=1}^{T_0} \blambda_t \blambda_t \T = \Lambda + o_P(1)
\\
&
\frac{1}{T_0} \sum_{t=1}^{T_0} {e}_{it} = o_P(1) \ ,
&&
\frac{1}{T_0} \sum_{t=1}^{T_0} {e}_{it} {e}_{jt} = \ind(i=j) \sigma_e^2 
\ , 
&&
\frac{1}{T_0} \sum_{t=1}^{T_0} e_{it} \blambda_t = o_P(1) 
 \ .
\end{align*}
where $\Lambda$ is positive semidefinite. Clearly, $\gamma_i^\dagger$ is not the minimizer of \eqref{eq-problimit} unless $\sigma_e^2 = 0$, i.e., a noiseless setting. Therefore, the OLS weights defined in \eqref{eq-OLS} converge to the minimizer of $Q(\bgamma)$ as $T_0 \rightarrow \infty$, which is different from the true synthetic control weights $\bgamma^\dagger$ satisfying $ \mu_{0} = \sum_{i \in \mathcal{D}} \gamma_i^{\dagger} \mu_i$. This implies that the OLS estimator is inconsistent for $\bgamma^\dagger$ unless $\sigma_e^2=0$.

\subsection{A Synthetic Control Estimator based on Regularized Generalized Method of Moments} \label{sec:supp:Regularized GMM}

We consider the following $\ell_2$-regularized generalized method of moments (GMM) by including ridge regularization in GMM:
\begin{align}						\label{eq-GMM-L2}
\big(
\widehat{\bgamma}_{\lambda}
,
\widehat{\bbeta}_{\lambda}
\big)
&
=
\argmin_{(\bgamma,\bbeta)}
\Big[
\big\{ \widehat{\Psi}( \bgamma, \bbeta) \big\} \T 
\widehat{\Omega}
\big\{ \widehat{\Psi}( \bgamma, \bbeta) \big\} 
+ 
\lambda \big\| \bgamma \big\|_2^2
\Big] \ .
\end{align}
We remark that other forms of regularization, such as lasso regularization \citep{Lasso1996}, are possible.
However, there are several advantages of using ridge regularization. 
First, for given $\lambda$, we can obtain a closed-form solution of the synthetic control weights to \eqref{eq-GMM-L2} as follows:
\begin{align*}
&
\widehat{\bgamma}_{\lambda}
=
\Big( \widehat{\bG}_{YW}\T \widehat{\Omega} \widehat{\bG}_{YW } + \lambda \widehat{\Omega} \Big)^{-1}
\Big( \widehat{\bG}_{YW }\T \widehat{\Omega} \widehat{\bG}_{YY } \Big) \ , 
\\
&
\widehat{\bG}_{YW} = 
\frac{1}{T_0}
\sum_{t =1}^{T_0} \bg_t(Y_t) \bW_{\D t}\T
\ , \
\widehat{\bG}_{YY } = 
\frac{1}{T_0}
\sum_{t=1}^{T_0} \bg_t(Y_t) Y_t \ .
\end{align*}
Second, we can establish the asymptotic normality of the estimator in \eqref{eq-GMM-L2} when $\lambda$ is chosen at a certain rate:
\begin{theorem}	\label{thm:Reg GMM}
Suppose the conditions of Theorem \ref{thm:AN} of the main paper are satisfied. Additionally, suppose the regularization parameter has a rate of $\lambda = o(T^{-1/2})$. Then, $T \rightarrow \infty$, we have
\begin{align*}
\sqrt{ T }
\Bigg\{
\begin{pmatrix}
\widehat{\bgamma}_{\lambda}
\\
\widehat{\bbeta}_{\lambda}
\end{pmatrix}
-
\begin{pmatrix}
\bgamma^* 
\\
\bbeta^*
\end{pmatrix}
\Bigg\}
\text{ converges in distribution to }
N \Big( 0, \Sigma_1^* \Sigma_2^* \Sigma_1\sT \Big) \ .
\end{align*}
Here, $\Sigma_1^*
=
\big( G\sT \Omega^* G^* \big)^{-1} G\sT \Omega^* $ and $
\Sigma_2^* 
=
\lim_{T \rightarrow \infty} \VAR \big\{ \sqrt{T} \cdot \widehat{\Psi}(\bgamma^*, \bbeta^*) \big\}$ where
\begin{align*}
G^* 
=
\lim_{T \rightarrow \infty} \frac{1}{T} \sum_{t=1}^{T}
\EXP
\bigg\{
\frac{\partial \Psi( \bO_t \con \bgamma,\bbeta) }{ \partial (\bgamma,\bbeta)\T }
\bigg\}
\bigg|_{ \bgamma=\bgamma^* , \bbeta=\bbeta^* }
\ , \
\Omega^* = \lim_{T \rightarrow \infty} \widehat{\Omega} \ .
\end{align*}
\end{theorem}
We remark that similar results are established in \citet{FuKnight2000, Fu2003, Caner2009}. In addition, we can establish a similar result when covariates are included. 







Following  Theorem \ref{thm:AN}, we may use $\widehat{\Sigma}_1
=
\big( \widehat{G}\T \widehat{\Omega} \widehat{G} \big)^{-1} \widehat{G}\T \widehat{\Omega}$ as an estimator of $\Sigma_1^*$.  Alternatively, to incorporate ridge regularization, one can use $ \widehat{\Sigma}_1
=
\big( \widehat{G}_{\lambda} \T \widehat{\Omega} \widehat{G}_{\lambda} \big)^{-1} \widehat{G}_{\lambda}\T \widehat{\Omega}$ where $\widehat{G}_{\lambda} = T^{-1} \sum_{t=1}^{T} \partial \Psi_{\lambda} ( \bO_t \con \bgamma, \bbeta) / \partial (\bgamma, \bbeta)\T \big|_{\bgamma = \widehat{\bgamma}, \bbeta = \widehat{\bbeta}} $ where
\begin{align*}
\Psi_\lambda (\bO_t \con \bgamma, \bbeta) 
= 
\begin{bmatrix}
(1-A_t)
\bg_t(Y_t) \big( Y_t - \bW_{\D t} \T \bgamma \big) 
\\
A_t \frac{\partial \tau(t \con \bbeta)}{\partial \bbeta \T} \big\{ Y_t - \bW_{\D t} \T \bgamma - \tau(t \con \bbeta) \big\}
\\
\sqrt{\lambda} \cdot \bgamma
\end{bmatrix}
\in \R^{p+b+d}
\ .
\end{align*}
The matrix $\Sigma_2^*$ can be estimated by a heteroskedasticity and autocorrelation consistent estimator with or without incorporating ridge regularization. Or, one can use block bootstrap methods to construct a variance estimator; see Section \ref{sec:supp:BB} for details. Lastly, we choose $\lambda$ based on leave-one-out cross-validation; see Algorithm \ref{alg:LOOCV} below.
\begin{algorithm}[!htb]
\begin{algorithmic}[1]
\REQUIRE Length of the pre-treatment periods $T_0$
\FOR{$t=1,\ldots,T_0$}

\STATE Let $\widehat{\bG}_{YW,(-t)}$ and $\widehat{\bG}_{YY,(-t)}$ be
\begin{align*}
\widehat{\bG}_{YW,(-t)} = 
\frac{1}{T_0-1}
\sum_{s=1, s\neq t}^{T_0} \bg_t(Y_s) \bW_{\D s}\T
\ , \
\widehat{\bG}_{YY,(-t)} = 
\frac{1}{T_0-1}
\sum_{s=1, s\neq t}^{T_0} \bg_t(Y_s) Y_s
\end{align*}


\STATE Let $\widehat{\bgamma}_{(-t),\lambda}
=
\big\{ \widehat{\bG}_{YW,(-t)}\T \widehat{\Omega} \widehat{\bG}_{YW,(-t)} + \lambda \widehat{\Omega} \big\}^{-1}
\big\{ \widehat{\bG}_{YW,(-t)}\T \widehat{\Omega} \widehat{\bG}_{YY,(-t)} \big\}$

\STATE Calculate the leave-one-out residual $
\widehat{e}_{t,\lambda} = Y_t - \bW_{\D t} \T \widehat{\bgamma}_{(-t),\lambda}$

\ENDFOR

\STATE Obtain the mean on the leave-one-out residuals $
\overline{e}_{t,\lambda} = T_0^{-1} \sum_{t=1}^{T_0} \widehat{e}_{t,\lambda}$

\RETURN Obtain the optimal $\lambda$ that minimizes the absolute value of $\overline{e}_{t,\lambda}$:
\begin{align*}
\lambda_{\text{opt}} = \argmin_{\lambda} \big| \overline{e}_{t,\lambda} \big|
\end{align*}
\end{algorithmic}
\caption{Leave-one-out Cross-validation for Choosing the Regularization Parameter $\lambda$}
\label{alg:LOOCV}
\end{algorithm}

\subsection{A Heteroskedasticity and Autocorrelation Consistent Covariance Matrix Estimator}		\label{sec:supp:HAC}

We provide details of a heteroskedasticity and autocorrelation consistent (HAC) covariance matrix estimator, which are obtained by following approaches of \citet{NW1987} and \citet{Andrews1991}. Let $\big( \widehat{\bgamma}, \widehat{\bbeta} \big) $ and $\Psi (\bO_t \con \bgamma , \bbeta)$ be the GMM estimators used in Theorem \ref{thm:AN} and the corresponding estimating function, respectively. Then, for a given bandwidth $\omega >0$ and a kernel function $\mathcal{K}(z)$, a heteroskedasticity and autocorrelation consistent estimator of $\Sigma_2^* =
\lim_{T \rightarrow \infty} \VAR \big\{ \sqrt{T} \cdot \widehat{\Psi}(\bgamma^*, \bbeta^*) \big\}$ is given as 
\begin{align*}
\widehat{\Sigma}_2
=
\frac{1}{T}
\sum_{t=1}^{T}
\left[
\begin{array}{l}
\big\{
\Psi( \bO_t \con \widehat{\bgamma} , \widehat{\bbeta} )
\big\}
\big\{
\Psi( \bO_t \con \widehat{\bgamma} , \widehat{\bbeta} )
\big\}\T
\\
+
\sum_{s=1}^{T} 
\mathcal{K} \big( s / \omega \big)
\big\{
\Psi( \bO_t \con \widehat{\bgamma} , \widehat{\bbeta} )
\big\}
\big\{
\Psi( \bO_{t+s} \con \widehat{\bgamma} , \widehat{\bbeta} )
\big\}\T
\\
+
\sum_{s=1}^{T} 
\mathcal{K} \big( s / \omega \big)
\big\{
\Psi( \bO_{t+s} \con \widehat{\bgamma} , \widehat{\bbeta} )
\big\}
\big\{
\Psi( \bO_{s} \con \widehat{\bgamma} , \widehat{\bbeta} )
\big\}\T
\end{array}
\right] \ .
\end{align*} 

Popular choices for the kernel function are Bartlett and quadratic spectral functions, which are defined as follows:
\begin{itemize}[itemsep=0cm,leftmargin=0.4cm]
\item Bartlett kernel: $\mathcal{K}(z) = \big\{ 1- |z| \big\} \ind\big\{ |z| \leq 1 \big\}$
\item Quadratic spectral kernel: $\mathcal{K}(z) = \big\{ {25}/{(12 \pi^2 z^2)} \big\} \cdot \big\{ \sin (6\pi z/5) / (6\pi z/5) - \cos (6\pi z/5) \big\} $
\end{itemize}

For these two kernel functions, the bandwidth parameter $\omega$ can be chosen based on the approximation to the first-order autoregressive model; see Algorithm \ref{alg:bandwidth} for details. We use the quadratic spectral kernel function for the simulation studies and the data analysis of the main paper.

\begin{algorithm}[!htb]
\begin{algorithmic}[1]
\STATE Let $\big( \widehat{\bgamma}, \widehat{\bbeta} \big) $ and $\Psi_{\post} (\bO_t \con \bgamma , \bbeta) \in \R^b$ be the GMM estimators used in Theorem \ref{thm:AN} and the corresponding estimating function related to $\bbeta$, respectively. 

\FOR{$s=1,\ldots,b$}

\STATE Fit AR(1) model for the $s$th component of the time series $\big\{ \Psi_{\post}(\bO_t \con \widehat{\bgamma}, \widehat{\bbeta} ) \big\}_{t=1,\ldots,T}$. 
\STATE Let $\widehat{\rho}_s$ and $\widehat{\sigma}_s^2$ be the estimated coefficient of the autoregressive coefficient and the estimated variance of the error from the AR(1) model above, respectively.
\ENDFOR

\STATE For Barlett and quadratic spectral kernel functions, we choose the bandwidth as
\begin{align*}
&
\omega_{\text{Bartlett}} 
=
1.1447 \big\{ \alpha_1 \cdot T \big\}^{1/3}
, 
&&
\hspace*{-0.25cm}
\alpha_1
=
\bigg\{ \sum_{s=1}^{b} \frac{ \widehat{\sigma}_s^4 }{ (1-\widehat{\rho}_s)^4 } \bigg\}^{-1}
\bigg\{ \sum_{s=1}^{b} \frac{ 4 \widehat{\rho}_s^2 \widehat{\sigma}_s^4 }{ (1-\widehat{\rho}_s)^6 (1+\widehat{\rho}_s)^2} \bigg\} 		
\\
&
\omega_{\text{QS}} 
=
1.3221 \big\{ \alpha_2 \cdot T \big\}^{1/5} 		
,
&&
\hspace*{-0.25cm}
\alpha_2
=
\bigg\{ \sum_{s=1}^{b} \frac{ \widehat{\sigma}_s^4 }{ (1-\widehat{\rho}_s)^4 } \bigg\}^{-1}
\bigg\{ \sum_{s=1}^{b} \frac{ 4 \widehat{\rho}_s^2 \widehat{\sigma}_s^4 }{ (1-\widehat{\rho}_s)^8} \bigg\} \ .
\end{align*}

\RETURN Bandwidth parameters $\omega_{\text{Bartlett}}$ and $\omega_{\text{QS}}$.
\end{algorithmic}
\caption{Choice of Bandwidth Parameters for Bartlett and Quadratic Spectral Kernel Functions}
\label{alg:bandwidth}
\end{algorithm}



\subsection{Block Bootstrap} \label{sec:supp:BB}

In this section, we provide a moving block bootstrap method \citep{Kunsch1989,Liu1992} adapted to our setting. Algorithm \ref{alg:MBB} provides details of the block bootstrap implementation. We remark that other block bootstrap methods can be adopted with minor modifications; see \citet{Lahiri1999} for examples of block bootstrap methods.

\begin{algorithm}[!htb]
\begin{algorithmic}[1]
\REQUIRE Length of the block $L < T_0$, Number of bootstrap repetitions $B$

\STATE Let the pre- and post-treatment blocks be
\begin{align*}
&
B_{\pre,1} = \big\{ \bO_1,\ldots,\bO_L \big\} , 
&& \ldots \ , 
&& B_{\pre,T_0-L+1} = \big\{ \bO_{T_0-L+1},\ldots,\bO_{T_0} \big\} 	
\\
&
B_{\post,1} = \big\{ \bO_{T_0+1},\ldots, \bO_{T_0+L} \big\} , 
&& \ldots \ , 
&& B_{\post,T_0-L+1} = \big\{ \bO_{T-L+1},\ldots, \bO_{T} \big\} 	
\end{align*} 	

\FOR{$b=1,\ldots,B$}

\STATE Randomly sample $K_{\pre} = \lceil T_0/L \rceil$ pre-treatment blocks and $K_{\post} = \lceil T_1/L \rceil$ post-treatment blocks with replacement, respectively; we denote these blocks as $\big\{ B_{\pre,1}^{(b)} ,\ldots, B_{\pre,K_{\pre}}^{(b)} \big\}$ and $\big\{ B_{\post,1}^{(b)} ,\ldots, B_{\post,K_{\post}}^{(b)} \big\}$

\STATE Choose the first $T_0$ and $T_1$ observations from the resampled blocks, i.e.,
\begin{align*}
&
\big\{ \bO_1^{(b)},\ldots,\bO_{T_0} ^{(b)} \big\}
=
\text{first $T_0$ observations of } \big\{ B_{\pre,1}^{(b)} ,\ldots, B_{\pre,K_{\pre}}^{(b)} \big\}
\\
&
\big\{ \bO_{T_0+1}^{(b)},\ldots,\bO_{T} ^{(b)} \big\}
=
\text{first $T_1$ observations of } \big\{ B_{\post,1}^{(b)} ,\ldots, B_{\post,K_{\post}}^{(b)} \big\}
\end{align*}

\STATE Calculate $\widehat{\bbeta}^{(b)}$ from the GMM in Section \ref{sec:Estimation} using $\big\{ \bO_1^{(b)},\ldots,\bO_{T_0} ^{(b)} , \bO_{T_0+1}^{(b)},\ldots,\bO_{T} ^{(b)} \big\}$. 	

\ENDFOR

\RETURN Report the variance of the bootstrap estimates $\big\{ \widehat{\bbeta}^{(1)}, \ldots, \widehat{\bbeta}^{(B)} \big\}$
\end{algorithmic}
\caption{Moving Block Bootstrap in Single Proxy Synthetic Control Framework}
\label{alg:MBB}
\end{algorithm}


The choice of block length $L$ is critical to the performance of block bootstrap methods. The optimal choice of $L$ for minimizing mean square error is known to be $O(T^{1/3})$. In the simulation studies in Section \ref{sec:Sim}, we use the bandwidth of the Bartlett kernel function $\omega_{\text{Bartlett}}$ in Algorithm \ref{alg:bandwidth} of which the rate is $O(T^{1/3})$. As discussed, this choice seems reasonable based on the simulation results reported in Section \ref{sec:supp:Simulation}. 
 

\subsection{An Example of Time-varying $g_t$ Functions}	\label{sec:supp:time g function}

Suppose that $\potY{t}{0}$ in the pre-treatment period appears to be nonstationary based on statistical procedures, such as Box-Pierce and Ljung-Box tests \citep{BoxPierce1970, LjungBox1978}. Under this case, one may use time periods in estimation of the synthetic control, which may improve performance. In order to implement this, one can use a time-varying coefficient function $\bg_t$ in the pre-treatment estimating function $\Psi_\pre$. For example, one can specify $\bg_t$ as follows:
\begin{align}   \label{eq-example gt}
\bg_t(y)
=
\begin{bmatrix}
\bphi_{\potY{t}{0}} (y)
\\
\bphi_{\mathcal{T}} (t)
\end{bmatrix}
\in \R^{p_y + p_t}
\end{align}
where $\bphi_{\potY{t}{0}}: \text{supp} (\{ \potY{t}{0}\,| \, t=1,\ldots,T_0 \}) \rightarrow \R^{p_y}$ and $\bphi_{\mathcal{T}}: [0,T] \rightarrow \R^{p_t}$ are collections of $p_y$ and $p_t$ bases functions associated with $\potY{t}{0}$ and $t$, respectively. Using a time-varying $\bg_t$, one can obtain a synthetic control estimator and an ATT estimator by following the proposed approach in the main paper. 
The aforementioned adjustment is also applicable to the conformal inference discussed in Section \ref{sec:Conformal}. In particular, under null hypotheses $H_0: \xi_t^* = \xi_{0t}$ for $t\in \{T_0+1,\ldots,T\}$, the pre-treatment estimating function $\Psi_{\pre}$ is given as follows:
\begin{align*}
\Psi_{\pre} (\bO_t \con \bgamma, \xi_{0,T_0+1},\ldots,\xi_{0,T})
=
\left\{
\begin{array}{lll}
\bg_t (Y_t)
\big( Y_t - \bW_{\D t} \T \bgamma \big)
&
\quad
&
t = 1,\ldots,T_0
\\
\bg_t ( Y_t - \xi_{0t})
\big( Y_t - \xi_{0t} - \bW_{\D t} \T \bgamma \big)
&
\quad
&
t = T_0+1,\ldots,T
\end{array}
\right. 
\ ,
\end{align*}
where $g_t$ is specified as \eqref{eq-example gt}. The rest of the procedure remains the same as in Section \ref{sec:Conformal}. In Section \ref{sec:supp:Simulation PI}, we demonstrate that incorporating a time-varying $g_t$ improves the performance of the conformal approach in Section \ref{sec:Conformal} in the presence of nonstationarity. 



% Under this specification of $\bg_t$, the synthetic control weights are estimated from the modified GMM over the pre-treatment periods as follows:
% \begin{align*}
% &
% \widehat{\bgamma}
% =
% \argmin_{\bgamma}
% \big\{ 
% \widehat{\Psi}_{\pre} ( \bgamma ) 
% \big\} \T
% \widehat{\Omega}_{\pre}
% \big\{ 
% \widehat{\Psi}_{\pre} ( \bgamma ) 
% \big\}
% \ , \
% \Psi_{\pre} (\bO_t \con \bgamma)
% =
% \bg_t(Y_t , t)
% \big( Y_t - \bW_{\D t} \T \bgamma \big)
% \ , \
% t = 1,\ldots,T_0 \ .
% \end{align*}
% The predicted value of the treatment-free post-treatment potential outcome of the treated unit remains the same form as $\widehat{Y}_t^{(0)} = \bW_{\D t}\T \widehat{\bgamma}$ for $t=T_0+1,\ldots,T$. In the construction of the synthetic control, we remark that the post-treatment periods are not used. More specifically, although the pre-treatment periods $t=1,\ldots,T_0$ are used to construct the synthetic control weights by serving as an argument of $\bg_t$, the post-treatment periods $t=T_0+1,\ldots,T$ are not used in the prediction. 

% An alternative strategy is to use the time period as an exogenous covariate by considering the following pre-treatment estimating equation:
% \begin{align*}
% \Psi_{\pre} (\bO_t \con \bgamma, \bdelta)
% =
% \bg_t(Y_t)
% \big\{ Y_t - \bW_{\D t} \T \bgamma - \mu(t \con \bdelta) \big\} \ ,
% \end{align*}
% where $\mu(t \con \bdelta)$ is a user-specified function parametrized by $\bdelta$, which is employed to address nonstationary behavior of $Y_t$ and $\bW_{\D t}$. For instance, one can choose $\mu(t \con \bdelta) = \delta_0 + \delta_1 t$. Technically speaking, this requires an additional assumption that $\potY{t}{0} = \EXP \big\{ \bW_{\D t} \T \bgamma - \mu(t \con \bdelta) \cond \potY{t}{0} \big\}$ almost surely for all $t=1,\ldots,T$. Under this framework, the predicted value of the treatment-free post-treatment potential outcome of the treated unit has a form of $\widehat{Y}_t^{(0)} = \bW_{\D t}\T \widehat{\bgamma} + \mu(t \con \widehat{\bdelta})$ for $t=T_0+1,\ldots,T$. Unfortunately, choosing a good candidate for $\mu$ can be problematic in practice because, even though $\mu$ may be a good choice over the pre-treatment periods, it may not be a suitable choice for the post-treatment periods. In other words, the extrapolation of $\mu$ over the post-treatment periods could lead to a significant amount of bias, especially when $t$ is very far from $T_0$. Moreover, in numerous real-world datasets, a well-constructed synthetic control of donors $\bW_{\D t} \T \widehat{\bgamma}$ might already eliminate the nonstationary time trends of $\potY{t}{0}$. In such cases, introducing $\mu$ could result in non-diminishing bias. Hence, we suggest setting $\mu(t \con \bdelta)=0$ for all $t$.




\subsection{Selection of a Donor Pool} \label{sec:supp:Donors}

We provide a procedure for how to choose a donor pool when many donor candidates are available. In Algorithm \ref{alg:Donor Pool}, we present details of the procedure. The key idea of the procedure is to select donors that seem to satisfy Assumption \ref{assumption:valid proxy}, which is $\bW_{i t} \nindep \potY{t}{0}$ for all $i \in \D$ and $t=1,\ldots,T_0$. In other words, donor candidates that appear to be independent of $\potY{t}{0}$ are discarded. The approach is akin to the widely-used backward selection technique employed in linear regression models. 

% We provide additional reasons why we consider the regression model in line 4 of Algorithm \ref{alg:Donor Pool}. It is well-known that there may be spurious relationships among time series, especially in the presence of nonstationarity. Therefore, regressing a donor $W_{it}$ on $\potY{t}{0}$ without using other donors may exhibit a statistically significant relationship even if they are statistically independent. Therefore, using this marginal regression model may not be useful for selecting donors. To address this issue, we propose to include the other donors $\big\{W_{st} \cond s \in \D \setminus \{i\} \big\}$ as additional regressors. By doing so, we may account for spurious relationships and detect associations between $W_{it}$ and $\potY{t}{0}$ better.

\begin{algorithm}[!htb]
\begin{algorithmic}[1]
\REQUIRE Number of donor candidates $N$, Threshold p-value level $\alpha$ (e.g., $\alpha=0.05$)

\STATE Initiate with $\D = \{ 1,\ldots,N \}$
\WHILE{Until \texttt{break} in line 11}
\FOR{$i \in \D$}
\STATE Using pre-treatment periods $t=1,\ldots, T_0$, fit a linear regression model of $W_{it}$ on $(\potY{t}{0}, \{ W_{st} \}_{s \in \D \setminus \{i\} })$, i.e., $
W_{it} = b_{iY} \potY{t}{0} + \sum_{s \neq t} b_{i s} W_{st} + e_{it}$
\STATE Let $p_i$ be the p-value of testing $H_0: b_{iY}=0$ based on the ordinary least squares estimator
\ENDFOR 	
\STATE Find a donor associated with the largest p-value, i.e., $m = \argmax_{i \in \D} p_i$
\IF{$p_m > \alpha$}
\STATE Drop $m$ from the donor pool, i.e., $\D \leftarrow \D \setminus \{m \}$
\ELSE
\STATE \texttt{Break} the while loop
\ENDIF
\ENDWHILE

\RETURN A subset of donors $\mathcal{D} \subseteq \{1,\ldots,N \}$		

\end{algorithmic}
\caption{Choice of a Donor Pool}
\label{alg:Donor Pool}
\end{algorithm}


We provide some detailed explanations for each step. Initially, we include all $W_{it}$ for $i=1,\ldots,N$ in the donor pool, denoted by $\D$ (line 1). Next, we conduct regression models, treating each donor as a dependent variable, while considering $\potY{t}{0}$ and the other donors in $\D$ as independent variables (line 4). From each regression model, we calculate the p-value of the coefficient associated with $\potY{t}{0}$ (line 5). If the largest p-value exceeds the predetermined threshold level $\alpha$, we discard the donor associated with the largest p-value from $\D$ (line 9). This iterative process continues until all p-values are below the threshold level $\alpha$ (line 11), resulting in the selection of the remaining donors as the final donor pool (line 14).

We provide further justification for employing the regression model in line 4 of Algorithm \ref{alg:Donor Pool}. It is well-recognized that spurious relationships among time series (especially in the presence of nonstationarity) can lead to misleading results. In the context of synthetic control, regressing a donor $W_{it}$ solely on $\potY{t}{0}$ without adjusting other donors may exhibit a statistically significant relationship even if they are statistically independent. Thus, relying solely on this marginal regression model may not be sufficient for selecting appropriate donors. In order to address this concern, we propose including the other donors $\big\{W_{st} \cond s \in \D \setminus {i} \big\}$ as additional regressors. By doing so, we can account for potential spurious relationships and better detect genuine associations between $W_{it}$ and $\potY{t}{0}$. This refined approach improves the reliability and accuracy of the donor pool selection process.
 

\subsection{Extension: Covariate Adjustment}		\label{sec:Cov}

In practice, a rich collection of measured exogenous covariates may be available. One may want to incorporate these covariates in the synthetic control analysis because using these covariates may improve efficiency. In this Section, we provide details on the SPSC framework by incorporating measured covariates. Specifically, we denote $q$-dimensional measured exogenous covariates for unit $i = 0,\ldots,N$ at time $t=1,\ldots,T$ as $\bX_{it} \in \R^{q}$; we remind the readers that $i=0$ is the treated unit and $i=1,\ldots,N$ are the untreated units. Let $\bX_{\D t} = ( \bX_{\D_1 t}\T,\ldots,\bX_{\D_d t}\T)\T \in \R^{dq}$ be the collection of all measured covariates of donors $\D$ at time $t$. To account for covariates, we modify Assumptions \ref{assumption:valid proxy} and \ref{assumption:SC} as follows:
\begin{assumption}[Proxy in the Presence of Covariates] \label{assumption:valid proxy Cov}
There exists a set of control units $\D = \{ \D_1,\ldots,\D_d \} \subseteq \{1,\ldots,N \}$ satisfying
\begin{align*}
&
\bW_{i t} \nindep \potY{t}{0} \cond (\bX_{0t}, \bX_{\D t})
\ , \quad \quad
i \in \D
\ , \quad \quad
t = 1,\ldots,T_0 \ .
\end{align*}
\end{assumption}

\begin{assumption}[Existence of Synthetic Control in the Presence of Covariates]
\label{assumption:SC Cov}
For all $t=1,\ldots,T$, there exist $\bgamma^* = (\gamma_{\D_1}^*,\ldots,\gamma_{\D_d}^*)\T$, $\bdelta_{0}^{*} \in \R^q$, and $\bdelta_\D^* = ( \bdelta_{\D_1}\sT,\ldots,\bdelta_{\D_d}\sT)\T \in \R^{dq}$ that satisfy $\EXP \big[
\big\{
\potY{t}{0} 
-
\bX_{0 t} \T \bdelta_{0}^*
\big\}
-
\big\{
\bW_{\D t}\T \bgamma^*
-
\bX_{\D t} \T \bdelta_{\D}^*
\big\}
\cond
\potY{t}{0}, \bX_{0t} , 
\bX_{\D t}
\big]
=
0$.
\end{assumption}
Similar to the result established under the absence of covariates, we establish the following identification results under Assumptions \ref{assumption:valid proxy Cov} and \ref{assumption:SC Cov} when covariates are available:
\begin{theorem} \label{thm:identification cov}
Under Assumptions \ref{assumption:consistency}, \ref{assumption:noitf}, \ref{assumption:valid proxy Cov}, and \ref{assumption:SC Cov}, the synthetic control weights $\bgamma^*$ satisfy $\EXP \big \{ (Y_t - \bX_{0t}\T \bdelta_0^*) - ( \bW_{it}\T \bgamma^* - \bW_{\D t}\T \bdelta_\D^* ) \cond Y_t, \bX_{0t} , 
\bX_{\D t} \big\} = 0 $ for $t=1,\ldots,T_0$. Moreover, we have $
\EXP \big\{ \potY{t}{0} - \bX_{0t}\T \bdelta_0^* \big\} 
= \EXP \big( \bW_{\D t}\T \bgamma^* - \bX_{\D t}\T \bdelta_\D^* \big)$ for any $t=1,\ldots,T$. 
Lastly, the ATT is identified as $\tau_t^*
=
\EXP
\big\{
\big(
Y_t
-
\bX_{0 t} \T \bdelta_{0}^*
\big)
-
\big(
\bW_{\D t}\T \bgamma^*
-
\bX_{\D t} \T \bdelta_{\D}^*
\big) 
\big\}$ for $t = T_0+1,\ldots,T$. 
\end{theorem}
Leveraging the result of the Theorem, estimation and inference of the ATT with covariate adjustment can be established, which is a straightforward extension of Section \ref{sec:Estimation}. First, we define the following estimating function:
\begin{align}			\label{eq-Moment-Cov}
&
\Psi_{\text{Cov}}( \bO_t \con \bgamma, \bbeta, \bdelta_0, \bdelta_\D)
\\
&
=
\begin{bmatrix}
(1-A_t)
\bg_t(Y_t, \bX_{0t}, \bX_{\D t})
\big\{ 
\big(
Y_t
-
\bX_{0 t} \T \bdelta_{0}
\big)
-
\big(
\bW_{\D t}\T \bgamma
-
\bX_{\D t} \T \bdelta_{\D}
\big)
\big\}
\\
A_t 
\frac{\partial \tau(t \con \bbeta) }{\partial \bbeta } 
\big\{
\big(
Y_t
-
\bX_{0 t} \T \bdelta_{0}
\big)
-
\big(
\bW_{\D t}\T \bgamma
-
\bX_{\D t} \T \bdelta_{\D}
\big) - \tau (t \con \bbeta)
\big\}
\end{bmatrix}
\in \R^{p_x+b} \ ,
\nonumber
\end{align}
where $\bO_t = (Y_t,\bW_{\D t}, \bX_{0t}, \bX_{\D t}, A_t)$ is the collection of the observed data at time $t$, $\bg_t(\cdot)$ is a $p_x$-dimensional user-specified function of $(Y_t,\bX_{0t},\bX_{\D t})$ (which can be time-varying) with $p_x \geq \text{dim}(\bW_{\D t}, \bX_{0t} , \bX_{\D t}) = d+q+dq$, and $\tau(t \con \bbeta)$ is a user-specified treatment effect function. We assume that the treatment effect function is chosen so that the associated error process is weakly dependent:
\begin{assumption}[Weakly Dependent Error in the Presence of Covariates] \label{assumption:weakdep Cov}
Let $\epsilon_t$ be $\epsilon_t = \big(
Y_t
-
\bX_{0 t} \T \bdelta_{0}^*
\big)
-
\big(
\bW_{\D t}\T \bgamma^*
-
\bX_{\D t} \T \bdelta_\D^*
\big) - \tau (t \con \bbeta^*) $. Then, the error process $\big\{ \epsilon_t \cond t = 1,\ldots,T \big\}$ satisfies Assumption \ref{assumption:weakdep}, i.e., $\text{corr}(\epsilon_{t}, \epsilon_{t+t'})$ converges to 0 as $t' \rightarrow \pm \infty$.
\end{assumption}	
We then establish the asymptotic normality of the GMM estimators $( \widehat{\bgamma}, \widehat{\bbeta}, \widehat{\bdelta}_0, \widehat{\bdelta}_\D ) $; see the formal statement below:
\begin{theorem}	\label{thm:AN Cov}
Suppose Assumptions \ref{assumption:consistency}, \ref{assumption:noitf}, \ref{assumption:valid proxy Cov}, \ref{assumption:SC Cov}, and \ref{assumption:weakdep Cov} hold, $(\bgamma^*,\bbeta^*,\bdelta_0^*,\bdelta_{\D}^*)$ are unique, and Regularity Conditions in Section \ref{sec:supp:AN} hold. Let $( \widehat{\bgamma}, \widehat{\bbeta}, \widehat{\bdelta}_0, \widehat{\bdelta}_\D ) $ be the GMM estimators where the estimating function \eqref{eq-Moment-Cov} is used, i.e., 
\begin{align*}
\big(
\widehat{\bgamma}
,
\widehat{\bbeta}
, 
\widehat{\bdelta}_0
,
\widehat{\bdelta}_\D 
\big)
=
\argmin_{(\bgamma,\bbeta)}
\big\{ \widehat{\Psi}_{\text{Cov}}( \bgamma, \bbeta, \delta_0, \delta_{\D}) \big\} \T
\widehat{\Omega}_{\text{Cov}}
\big\{ \widehat{\Psi}_{\text{Cov}}( \bgamma, \bbeta, \delta_0, \delta_{\D}) \big\} \ ,
\end{align*}
where $	\widehat{\Psi}_{\text{Cov}}(\bgamma, \bbeta, \delta_0, \delta_{\D})
= T^{-1}
\sum_{t=1}^{T} \Psi_{\text{Cov}} (\bO_t \con \bgamma, \bbeta, \delta_0, \delta_{\D})$ is the empirical mean of the estimating function and $\widehat{\Omega}_{\text{Cov}} \in \R^{(p+b) \times (p+b)}$ a user-specified symmetric positive definite block-diagonal matrix as $\widehat{\Omega}_{\text{Cov}} = \text{diag}( \widehat{\Omega}_{\text{Cov},\pre} , \widehat{\Omega}_{\text{Cov},\post} )$. Then, as $T \rightarrow \infty$, we have
\begin{align*}
\sqrt{ T }
\left\{
\begin{pmatrix}
\widehat{\bgamma}
\\
\widehat{\bbeta}
\\
\widehat{\bdelta}_0
\\
\widehat{\bdelta}_\D 
\end{pmatrix}
-
\begin{pmatrix}
\bgamma^*
\\
\bbeta^* 
\\
\bdelta_0^*
\\
\bdelta_\D^*
\end{pmatrix}
\right\}
\text{ converges in distribution to }
N \big( 0, \Sigma_{\text{Cov},1}^* \Sigma_{\text{Cov},2}^* \Sigma_{\text{Cov},1}\sT \big) \ ,
\end{align*}
where
\begin{align*}
&
\Sigma_{\text{Cov},1}^*
=
\big( G_{\text{Cov}}\sT \Omega_{\text{Cov}}^* G_{\text{Cov}}^* \big)^{-1} G_{\text{Cov}}\sT \Omega_{\text{Cov}}^* 
\quad , \quad 
\Sigma_{\text{Cov},2}^* 
=
\lim_{T \rightarrow \infty} \VAR \big\{ \sqrt{T} \cdot \widehat{\Psi}_{\text{Cov}}(\bgamma^*, \bbeta^*,\delta_0^*,\delta_{\D}^*) \big\}
\\
&
G_{\text{Cov}}^* 
=
\lim_{T \rightarrow \infty} \frac{1}{T} \sum_{t=1}^{T}
\EXP
\bigg\{
\frac{\partial \Psi_{\text{Cov}}( \bO_t \con \bgamma,\bbeta,\delta_0,\delta_{\D}) }{ \partial (\bgamma,\bbeta,\delta_0,\delta_{\D})\T }
\bigg\}
\bigg|_{ \bgamma=\bgamma^* , \bbeta=\bbeta^*, \delta_0 = \delta_0^*,\delta_{\D}=\delta_{\D}^* }
\quad , \quad
\Omega_{\text{Cov}}^* = \lim_{T \rightarrow \infty} \widehat{\Omega}_{\text{Cov}} \ .
\end{align*}
\end{theorem}
Estimators of $\Sigma_{\text{Cov},1}^*$ and $\Sigma_{\text{Cov},2}^*$ can be similarly defined as in Section \ref{sec:Estimation}, thus we omit the details here.





\subsection{Additional Simulation Studies} \label{sec:supp:Simulation}

We restate the data generating process of the simulation studies in Section \ref{sec:Sim}. The length of pre- and post-treatment periods were given by $T_0 = T_1 \in \{ 50, 100, 250, 1000 \}$ and the number of donors were given by $d \in \{2,5,9\}$. For each value of $T_0$, $T_1$, and $d$, we generated all errors for $t=1,\ldots,T$ based on the AR(2):
\begin{align*}
&
\epsilon_{x_i,t} = 0.2 \epsilon_{x_i,t-1} + 0.1 \epsilon_{x_i,t-2} + \eta_{x_i,t}
\ , \ i=0,\ldots,d \ ,
&&
\epsilon_{y,t} = 0.2 \epsilon_{y,t-1} + 0.1 \epsilon_{y,t-2} + \eta_{y,t}
\\
&
\epsilon_{w_i,t} = 0.2 \epsilon_{w_i,t-1} + 0.1 \epsilon_{w_i,t-2} + \eta_{w_i,t}
\ , \ i=1,\ldots,d \ ,
&&
\epsilon_{\tau,t} = 0.2 \epsilon_{\tau,t-1} + 0.1 \epsilon_{\tau,t-2} + \eta_{\tau,t}
\end{align*}
where $\eta$ were generated from a standard normal distribution. The errors $\epsilon_{t}$ at $t=-1,0$ were initialized to equal zero. The exogenous covariates $\bX_t = \{ X_{0t}, X_{\D_1 t} , \ldots,X_{\D_d t} \}$ were generated as 
\begin{align*}
 X_{it} = 0.2 X_{i,t-1} + 0.1 X_{i,t-2} + \epsilon_{x_i,t} \ , \ i=0,\ldots,d \ .
\end{align*}
The treatment-free potential outcomes at $t=1,\ldots,T$ were generated as 
\begin{align*}
 \potY{t}{0} = 0.2 \potY{t-1}{0} + 0.1 \potY{t-2}{0} + t/T_0 + \delta X_{0t} + \epsilon_{y,t} \ .
\end{align*}
We considered two cases for $\delta \in \{0,1\}$, which encodes whether the covariates are predictive of $\potY{t}{0}$, in which case $\delta=1$, or not predictive, in which case $\delta=0$, respectively. The potential outcomes at $t=1,\ldots,T$ were under treatment were generated as 
\begin{align*}
 \potY{t}{1} = \potY{t}{0} + 3 A_t + \epsilon_{\tau,t} \ , \ 
 t=1,\ldots,T \ .
\end{align*}
Therefore, the ATT is $\tau_t^* = 3$ for all $t=T_0+1,\ldots,T$, and $\xi_t^* = \potY{t}{1} - \potY{t}{0} = 3 \ind(t \geq T_0) + \epsilon_{\tau,t}$. Lastly, we generated $\bW_{\D t} \in \R^{d}$ so that it satisfies Assumptions \ref{assumption:valid proxy} and \ref{assumption:SC} if $\delta=0$, and the corresponding assumptions made for incorporating covariates if $\delta=1$. Specifically, we considered the following data generating process for $\bW_{\D t}$ according to the number of donors $d \in \{2,5,9\}$:
\begin{itemize}[leftmargin=0cm, itemsep=0cm]
\item[($d=2$)] 
\begin{align*}
\bW_{\D t}
=
\begin{bmatrix}
2 & -1 \\
-1 & 2 
\end{bmatrix}
\begin{bmatrix}
1 \\ \potY{t}{0}
\end{bmatrix}
+
\delta
\bX_{\D t}
+ 
\bepsilon_{\D t}
\end{align*} 
The true synthetic control weights are $\bgamma^* = (1/3,2/3)$.


\item[($d=5$)] 
\begin{align*}
\bW_{\D t}
=
\begin{bmatrix}
2 & -1 & 0 & 0 & 0 
\\
0 & 2 & -1 & 0 & 0
\\
0 & 0 & 2 & -1 & 0
\\
0 & 0 & 0 & 2 & -1
\\
-1 & 0 & 0 & 0 & 2
\end{bmatrix}
\begin{bmatrix}
1 \\ \potY{t}{0} \\ 0.5 \big\{ \potY{t}{0} \big\}^2 \\ \ind \big\{ \potY{t}{0} > 3 \big\} \\ \ind \big\{ \potY{t}{0} < 0 \big\}
\end{bmatrix}
+
\delta 
\bX_{\D t}
+ 
\bepsilon_{\D t}
\end{align*} 
The true synthetic control weights are $\bgamma^* \simeq ( 0.26 , 0.52 , 0.03 , 0.06 , 0.13)$. 


\item[($d=9$)] 
\begin{align*}
\bW_{\D t}
=
\begin{bmatrix}
2 & -1 & 0 & 0 & 0 & 0 & 0 & 0 & 0
\\
0 & 2 & -1 & 0 & 0 & 0 & 0 & 0 & 0
\\
0 & 0 & 2 & -1 & 0 & 0 & 0 & 0 & 0
\\
0 & 0 & 0 & 2 & -1 & 0 & 0 & 0 & 0
\\
0 & 0 & 0 & 0 & 2 & -1 & 0 & 0 & 0
\\
0 & 0 & 0 & 0 & 0 & 2 & -1 & 0 & 0
\\
0 & 0 & 0 & 0 & 0 & 0 & 2 & -1 & 0
\\
0 & 0 & 0 & 0 & 0 & 0 & 0 & 2 & -1
\\
-1 & 0 & 0 & 0 & 0 & 0 & 0 & 0 & 2
\end{bmatrix}
\begin{bmatrix}
1 \\ \potY{t}{0} \\ 0.5 \big\{ \potY{t}{0} \big\}^2 \\ 
\ind \big\{ \potY{t}{0} > 3 \big\} \\ \ind \big\{ \potY{t}{0} < 0 \big\}
\\
\ind \big\{ \potY{t}{0} \in [0,1) \big\} \\ \ind \big\{ \potY{t}{0} \in [1,2) \big\}
\\
\exp \big[ 0.4 \big\{ \potY{t}{0} - 1.5 \big\} \big]
\\
\exp \big[ -0.4 \big\{ \potY{t}{0} - 1.5 \big\} \big]
\end{bmatrix}
+
\delta 
\bX_{\D t}
+ 
\bepsilon_{\D t}
\end{align*} 
The true synthetic control weights are $\bgamma^* \simeq (0.25,0.501,0.002,0.004,0.008,0.016,0.031,0.063,0.125)$.
\end{itemize} 

We set $\bg_t(\potY{t}{0})$ as time-invariant cubic B-spline bases functions with dimensions equal to twice the number of donors, i.e., $\bg_t(y) = \mathfrak{b}_{2d}( \cdot )$ where $\mathfrak{b}_{k}( \cdot )$ the $k$-dimensional cubic B-spline bases function. The knots of the spline functions were chosen based on the empirical quantiles of the pre-treatment outcomes. 



In Table \ref{tab:supp:Table00}, we first present numerical summaries of the simulation studies considered in the main paper. Each table is written in the following format:
\begin{itemize}[itemsep=0cm,leftmargin=0cm]
\item Bias row shows the empirical bias of 500 estimates;
\item ASE row shows the asymptotic standard error obtained from the sandwich variance estimator;
\item BSE row shows the bootstrap standard error obtained from the approach in Section \ref{sec:supp:BB} of the Supplementary Material;
\item ESE row shows the standard deviation of 500 estimates;
\item MSE row shows the mean squared error of 500 estimates;
\item Cover (ASE) and Cover (BSE) show the empirical coverage rates of 95\% confidence intervals based on the asymptotic and bootstrap standard errors, respectively;
\item Bias, standard errors, and mean squared error are scaled by factors of 10, 10, and 100, respectively, for readability.	
\end{itemize}
We remark that the results in Table \ref{tab:supp:Table00} are similar to those in Table \ref{tab:Sim:Constant d9d0}.

% Figure environment removed

\begin{table}[!htp]
\renewcommand{\arraystretch}{1.05} \centering
\footnotesize
\setlength{\tabcolsep}{3pt} 
\hspace*{-0.25cm}
\begin{tabular}{|ccc|cccc|cccc|cccc|}
\hline
\multicolumn{3}{|c|}{Estimator} & \multicolumn{4}{c|}{OLS} & \multicolumn{4}{c|}{SPSC} & \multicolumn{4}{c|}{SPSC-Ridge} \\ \hline
\multicolumn{1}{|c|}{$\ \delta \ $} & \multicolumn{1}{c|}{$\ d \ $} & $T_0$ & \multicolumn{1}{c|}{50} & \multicolumn{1}{c|}{100} & \multicolumn{1}{c|}{250} & 1000 & \multicolumn{1}{c|}{50} & \multicolumn{1}{c|}{100} & \multicolumn{1}{c|}{250} & 1000 & \multicolumn{1}{c|}{50} & \multicolumn{1}{c|}{100} & \multicolumn{1}{c|}{250} & 1000 \\ \hline

\multicolumn{1}{|c|}{\multirow{21}{*}{0}} & \multicolumn{1}{c|}{\multirow{7}{*}{2}} & Bias ($\times$10) & \multicolumn{1}{c|}{$5.042$} & \multicolumn{1}{c|}{$4.987$} & \multicolumn{1}{c|}{$4.931$} & \multicolumn{1}{c|}{$4.918$} & \multicolumn{1}{c|}{$-0.204$} & \multicolumn{1}{c|}{$-0.107$} & \multicolumn{1}{c|}{$-0.165$} & \multicolumn{1}{c|}{$-0.003$} & \multicolumn{1}{c|}{$0.325$} & \multicolumn{1}{c|}{$0.170$} & \multicolumn{1}{c|}{$-0.032$} & \multicolumn{1}{c|}{$0.047$} \\ \cline{3-15}
\multicolumn{1}{|c|}{} & \multicolumn{1}{c|}{} & ASE ($\times$10) & \multicolumn{1}{c|}{$2.366$} & \multicolumn{1}{c|}{$1.749$} & \multicolumn{1}{c|}{$1.161$} & \multicolumn{1}{c|}{$0.605$} & \multicolumn{1}{c|}{$2.930$} & \multicolumn{1}{c|}{$2.116$} & \multicolumn{1}{c|}{$1.383$} & \multicolumn{1}{c|}{$0.706$} & \multicolumn{1}{c|}{$2.841$} & \multicolumn{1}{c|}{$2.086$} & \multicolumn{1}{c|}{$1.374$} & \multicolumn{1}{c|}{$0.705$} \\ \cline{3-15}
\multicolumn{1}{|c|}{} & \multicolumn{1}{c|}{} & BSE ($\times$10) & \multicolumn{1}{c|}{$2.620$} & \multicolumn{1}{c|}{$1.952$} & \multicolumn{1}{c|}{$1.332$} & \multicolumn{1}{c|}{$0.742$} & \multicolumn{1}{c|}{$3.385$} & \multicolumn{1}{c|}{$2.371$} & \multicolumn{1}{c|}{$1.611$} & \multicolumn{1}{c|}{$0.860$} & \multicolumn{1}{c|}{$3.127$} & \multicolumn{1}{c|}{$2.332$} & \multicolumn{1}{c|}{$1.577$} & \multicolumn{1}{c|}{$0.858$} \\ \cline{3-15}
\multicolumn{1}{|c|}{} & \multicolumn{1}{c|}{} & ESE ($\times$10) & \multicolumn{1}{c|}{$2.874$} & \multicolumn{1}{c|}{$1.979$} & \multicolumn{1}{c|}{$1.264$} & \multicolumn{1}{c|}{$0.605$} & \multicolumn{1}{c|}{$3.331$} & \multicolumn{1}{c|}{$2.430$} & \multicolumn{1}{c|}{$1.486$} & \multicolumn{1}{c|}{$0.741$} & \multicolumn{1}{c|}{$3.270$} & \multicolumn{1}{c|}{$2.403$} & \multicolumn{1}{c|}{$1.470$} & \multicolumn{1}{c|}{$0.742$} \\ \cline{3-15}
\multicolumn{1}{|c|}{} & \multicolumn{1}{c|}{} & MSE ($\times$100) & \multicolumn{1}{c|}{$33.665$} & \multicolumn{1}{c|}{$28.777$} & \multicolumn{1}{c|}{$25.914$} & \multicolumn{1}{c|}{$24.551$} & \multicolumn{1}{c|}{$11.113$} & \multicolumn{1}{c|}{$5.906$} & \multicolumn{1}{c|}{$2.231$} & \multicolumn{1}{c|}{$0.549$} & \multicolumn{1}{c|}{$10.779$} & \multicolumn{1}{c|}{$5.791$} & \multicolumn{1}{c|}{$2.157$} & \multicolumn{1}{c|}{$0.551$} \\ \cline{3-15}
\multicolumn{1}{|c|}{} & \multicolumn{1}{c|}{} & Cover (ASE) & \multicolumn{1}{c|}{$0.458$} & \multicolumn{1}{c|}{$0.230$} & \multicolumn{1}{c|}{$0.034$} & \multicolumn{1}{c|}{$0.000$} & \multicolumn{1}{c|}{$0.916$} & \multicolumn{1}{c|}{$0.904$} & \multicolumn{1}{c|}{$0.928$} & \multicolumn{1}{c|}{$0.940$} & \multicolumn{1}{c|}{$0.908$} & \multicolumn{1}{c|}{$0.902$} & \multicolumn{1}{c|}{$0.932$} & \multicolumn{1}{c|}{$0.940$} \\ \cline{3-15}
\multicolumn{1}{|c|}{} & \multicolumn{1}{c|}{} & Cover (BSE) & \multicolumn{1}{c|}{$0.512$} & \multicolumn{1}{c|}{$0.326$} & \multicolumn{1}{c|}{$0.048$} & \multicolumn{1}{c|}{$0.000$} & \multicolumn{1}{c|}{$0.920$} & \multicolumn{1}{c|}{$0.924$} & \multicolumn{1}{c|}{$0.958$} & \multicolumn{1}{c|}{$0.968$} & \multicolumn{1}{c|}{$0.916$} & \multicolumn{1}{c|}{$0.916$} & \multicolumn{1}{c|}{$0.954$} & \multicolumn{1}{c|}{$0.968$} \\ \cline{2-15}
\multicolumn{1}{|c|}{} & \multicolumn{1}{c|}{\multirow{7}{*}{5}} & Bias ($\times$10) & \multicolumn{1}{c|}{$1.871$} & \multicolumn{1}{c|}{$1.786$} & \multicolumn{1}{c|}{$2.050$} & \multicolumn{1}{c|}{$1.934$} & \multicolumn{1}{c|}{$0.004$} & \multicolumn{1}{c|}{$0.110$} & \multicolumn{1}{c|}{$0.376$} & \multicolumn{1}{c|}{$0.091$} & \multicolumn{1}{c|}{$-0.222$} & \multicolumn{1}{c|}{$-0.088$} & \multicolumn{1}{c|}{$0.221$} & \multicolumn{1}{c|}{$0.161$} \\ \cline{3-15}
\multicolumn{1}{|c|}{} & \multicolumn{1}{c|}{} & ASE ($\times$10) & \multicolumn{1}{c|}{$2.354$} & \multicolumn{1}{c|}{$1.721$} & \multicolumn{1}{c|}{$1.140$} & \multicolumn{1}{c|}{$0.586$} & \multicolumn{1}{c|}{$4.611$} & \multicolumn{1}{c|}{$2.896$} & \multicolumn{1}{c|}{$1.714$} & \multicolumn{1}{c|}{$0.772$} & \multicolumn{1}{c|}{$2.958$} & \multicolumn{1}{c|}{$2.081$} & \multicolumn{1}{c|}{$1.356$} & \multicolumn{1}{c|}{$0.673$} \\ \cline{3-15}
\multicolumn{1}{|c|}{} & \multicolumn{1}{c|}{} & BSE ($\times$10) & \multicolumn{1}{c|}{$2.855$} & \multicolumn{1}{c|}{$2.035$} & \multicolumn{1}{c|}{$1.361$} & \multicolumn{1}{c|}{$0.719$} & \multicolumn{1}{c|}{$5.633$} & \multicolumn{1}{c|}{$3.440$} & \multicolumn{1}{c|}{$2.223$} & \multicolumn{1}{c|}{$1.049$} & \multicolumn{1}{c|}{$3.589$} & \multicolumn{1}{c|}{$2.523$} & \multicolumn{1}{c|}{$1.702$} & \multicolumn{1}{c|}{$0.870$} \\ \cline{3-15}
\multicolumn{1}{|c|}{} & \multicolumn{1}{c|}{} & ESE ($\times$10) & \multicolumn{1}{c|}{$2.834$} & \multicolumn{1}{c|}{$2.021$} & \multicolumn{1}{c|}{$1.258$} & \multicolumn{1}{c|}{$0.629$} & \multicolumn{1}{c|}{$4.333$} & \multicolumn{1}{c|}{$2.955$} & \multicolumn{1}{c|}{$1.657$} & \multicolumn{1}{c|}{$0.759$} & \multicolumn{1}{c|}{$3.411$} & \multicolumn{1}{c|}{$2.236$} & \multicolumn{1}{c|}{$1.464$} & \multicolumn{1}{c|}{$0.693$} \\ \cline{3-15}
\multicolumn{1}{|c|}{} & \multicolumn{1}{c|}{} & MSE ($\times$100) & \multicolumn{1}{c|}{$11.518$} & \multicolumn{1}{c|}{$7.265$} & \multicolumn{1}{c|}{$5.780$} & \multicolumn{1}{c|}{$4.137$} & \multicolumn{1}{c|}{$18.740$} & \multicolumn{1}{c|}{$8.729$} & \multicolumn{1}{c|}{$2.882$} & \multicolumn{1}{c|}{$0.583$} & \multicolumn{1}{c|}{$11.659$} & \multicolumn{1}{c|}{$4.999$} & \multicolumn{1}{c|}{$2.189$} & \multicolumn{1}{c|}{$0.505$} \\ \cline{3-15}
\multicolumn{1}{|c|}{} & \multicolumn{1}{c|}{} & Cover (ASE) & \multicolumn{1}{c|}{$0.808$} & \multicolumn{1}{c|}{$0.782$} & \multicolumn{1}{c|}{$0.550$} & \multicolumn{1}{c|}{$0.106$} & \multicolumn{1}{c|}{$0.952$} & \multicolumn{1}{c|}{$0.960$} & \multicolumn{1}{c|}{$0.956$} & \multicolumn{1}{c|}{$0.958$} & \multicolumn{1}{c|}{$0.898$} & \multicolumn{1}{c|}{$0.922$} & \multicolumn{1}{c|}{$0.930$} & \multicolumn{1}{c|}{$0.942$} \\ \cline{3-15}
\multicolumn{1}{|c|}{} & \multicolumn{1}{c|}{} & Cover (BSE) & \multicolumn{1}{c|}{$0.868$} & \multicolumn{1}{c|}{$0.850$} & \multicolumn{1}{c|}{$0.686$} & \multicolumn{1}{c|}{$0.200$} & \multicolumn{1}{c|}{$0.962$} & \multicolumn{1}{c|}{$0.964$} & \multicolumn{1}{c|}{$0.984$} & \multicolumn{1}{c|}{$0.984$} & \multicolumn{1}{c|}{$0.928$} & \multicolumn{1}{c|}{$0.956$} & \multicolumn{1}{c|}{$0.974$} & \multicolumn{1}{c|}{$0.978$} \\ \cline{2-15}
\multicolumn{1}{|c|}{} & \multicolumn{1}{c|}{\multirow{7}{*}{9}} & Bias ($\times$10) & \multicolumn{1}{c|}{$1.305$} & \multicolumn{1}{c|}{$1.524$} & \multicolumn{1}{c|}{$1.479$} & \multicolumn{1}{c|}{$1.401$} & \multicolumn{1}{c|}{$-0.206$} & \multicolumn{1}{c|}{$0.055$} & \multicolumn{1}{c|}{$-0.037$} & \multicolumn{1}{c|}{$-0.098$} & \multicolumn{1}{c|}{$-0.175$} & \multicolumn{1}{c|}{$-0.035$} & \multicolumn{1}{c|}{$-0.046$} & \multicolumn{1}{c|}{$-0.128$} \\ \cline{3-15}
\multicolumn{1}{|c|}{} & \multicolumn{1}{c|}{} & ASE ($\times$10) & \multicolumn{1}{c|}{$2.356$} & \multicolumn{1}{c|}{$1.674$} & \multicolumn{1}{c|}{$1.098$} & \multicolumn{1}{c|}{$0.568$} & \multicolumn{1}{c|}{$4.461$} & \multicolumn{1}{c|}{$3.008$} & \multicolumn{1}{c|}{$1.927$} & \multicolumn{1}{c|}{$0.930$} & \multicolumn{1}{c|}{$3.245$} & \multicolumn{1}{c|}{$2.358$} & \multicolumn{1}{c|}{$1.487$} & \multicolumn{1}{c|}{$0.747$} \\ \cline{3-15}
\multicolumn{1}{|c|}{} & \multicolumn{1}{c|}{} & BSE ($\times$10) & \multicolumn{1}{c|}{$2.999$} & \multicolumn{1}{c|}{$1.990$} & \multicolumn{1}{c|}{$1.303$} & \multicolumn{1}{c|}{$0.700$} & \multicolumn{1}{c|}{$5.549$} & \multicolumn{1}{c|}{$3.433$} & \multicolumn{1}{c|}{$2.127$} & \multicolumn{1}{c|}{$1.107$} & \multicolumn{1}{c|}{$4.049$} & \multicolumn{1}{c|}{$2.777$} & \multicolumn{1}{c|}{$1.799$} & \multicolumn{1}{c|}{$0.938$} \\ \cline{3-15}
\multicolumn{1}{|c|}{} & \multicolumn{1}{c|}{} & ESE ($\times$10) & \multicolumn{1}{c|}{$2.956$} & \multicolumn{1}{c|}{$1.934$} & \multicolumn{1}{c|}{$1.149$} & \multicolumn{1}{c|}{$0.603$} & \multicolumn{1}{c|}{$4.182$} & \multicolumn{1}{c|}{$2.434$} & \multicolumn{1}{c|}{$1.614$} & \multicolumn{1}{c|}{$0.800$} & \multicolumn{1}{c|}{$3.578$} & \multicolumn{1}{c|}{$2.261$} & \multicolumn{1}{c|}{$1.455$} & \multicolumn{1}{c|}{$0.721$} \\ \cline{3-15}
\multicolumn{1}{|c|}{} & \multicolumn{1}{c|}{} & MSE ($\times$100) & \multicolumn{1}{c|}{$10.422$} & \multicolumn{1}{c|}{$6.054$} & \multicolumn{1}{c|}{$3.507$} & \multicolumn{1}{c|}{$2.325$} & \multicolumn{1}{c|}{$17.500$} & \multicolumn{1}{c|}{$5.916$} & \multicolumn{1}{c|}{$2.600$} & \multicolumn{1}{c|}{$0.648$} & \multicolumn{1}{c|}{$12.810$} & \multicolumn{1}{c|}{$5.104$} & \multicolumn{1}{c|}{$2.116$} & \multicolumn{1}{c|}{$0.536$} \\ \cline{3-15}
\multicolumn{1}{|c|}{} & \multicolumn{1}{c|}{} & Cover (ASE) & \multicolumn{1}{c|}{$0.854$} & \multicolumn{1}{c|}{$0.788$} & \multicolumn{1}{c|}{$0.710$} & \multicolumn{1}{c|}{$0.306$} & \multicolumn{1}{c|}{$0.956$} & \multicolumn{1}{c|}{$0.962$} & \multicolumn{1}{c|}{$0.980$} & \multicolumn{1}{c|}{$0.970$} & \multicolumn{1}{c|}{$0.922$} & \multicolumn{1}{c|}{$0.940$} & \multicolumn{1}{c|}{$0.958$} & \multicolumn{1}{c|}{$0.946$} \\ \cline{3-15}
\multicolumn{1}{|c|}{} & \multicolumn{1}{c|}{} & Cover (BSE) & \multicolumn{1}{c|}{$0.916$} & \multicolumn{1}{c|}{$0.842$} & \multicolumn{1}{c|}{$0.802$} & \multicolumn{1}{c|}{$0.480$} & \multicolumn{1}{c|}{$0.978$} & \multicolumn{1}{c|}{$0.980$} & \multicolumn{1}{c|}{$0.984$} & \multicolumn{1}{c|}{$0.982$} & \multicolumn{1}{c|}{$0.950$} & \multicolumn{1}{c|}{$0.958$} & \multicolumn{1}{c|}{$0.986$} & \multicolumn{1}{c|}{$0.986$} \\ \hline \hline
\multicolumn{1}{|c|}{\multirow{21}{*}{1}} & \multicolumn{1}{c|}{\multirow{7}{*}{2}} & Bias ($\times$10) & \multicolumn{1}{c|}{$4.175$} & \multicolumn{1}{c|}{$3.921$} & \multicolumn{1}{c|}{$3.928$} & \multicolumn{1}{c|}{$3.941$} & \multicolumn{1}{c|}{$-0.212$} & \multicolumn{1}{c|}{$-0.260$} & \multicolumn{1}{c|}{$-0.123$} & \multicolumn{1}{c|}{$-0.001$} & \multicolumn{1}{c|}{$0.197$} & \multicolumn{1}{c|}{$-0.086$} & \multicolumn{1}{c|}{$-0.044$} & \multicolumn{1}{c|}{$0.028$} \\ \cline{3-15}
\multicolumn{1}{|c|}{} & \multicolumn{1}{c|}{} & ASE ($\times$10) & \multicolumn{1}{c|}{$2.329$} & \multicolumn{1}{c|}{$1.674$} & \multicolumn{1}{c|}{$1.091$} & \multicolumn{1}{c|}{$0.566$} & \multicolumn{1}{c|}{$2.820$} & \multicolumn{1}{c|}{$1.995$} & \multicolumn{1}{c|}{$1.289$} & \multicolumn{1}{c|}{$0.663$} & \multicolumn{1}{c|}{$2.754$} & \multicolumn{1}{c|}{$1.973$} & \multicolumn{1}{c|}{$1.284$} & \multicolumn{1}{c|}{$0.662$} \\ \cline{3-15}
\multicolumn{1}{|c|}{} & \multicolumn{1}{c|}{} & BSE ($\times$10) & \multicolumn{1}{c|}{$2.485$} & \multicolumn{1}{c|}{$1.845$} & \multicolumn{1}{c|}{$1.275$} & \multicolumn{1}{c|}{$0.697$} & \multicolumn{1}{c|}{$4.281$} & \multicolumn{1}{c|}{$3.087$} & \multicolumn{1}{c|}{$2.089$} & \multicolumn{1}{c|}{$1.117$} & \multicolumn{1}{c|}{$4.178$} & \multicolumn{1}{c|}{$3.118$} & \multicolumn{1}{c|}{$2.063$} & \multicolumn{1}{c|}{$1.103$} \\ \cline{3-15}
\multicolumn{1}{|c|}{} & \multicolumn{1}{c|}{} & ESE ($\times$10) & \multicolumn{1}{c|}{$2.780$} & \multicolumn{1}{c|}{$1.854$} & \multicolumn{1}{c|}{$1.223$} & \multicolumn{1}{c|}{$0.581$} & \multicolumn{1}{c|}{$3.313$} & \multicolumn{1}{c|}{$2.125$} & \multicolumn{1}{c|}{$1.453$} & \multicolumn{1}{c|}{$0.677$} & \multicolumn{1}{c|}{$3.356$} & \multicolumn{1}{c|}{$2.120$} & \multicolumn{1}{c|}{$1.450$} & \multicolumn{1}{c|}{$0.673$} \\ \cline{3-15}
\multicolumn{1}{|c|}{} & \multicolumn{1}{c|}{} & MSE ($\times$100) & \multicolumn{1}{c|}{$25.141$} & \multicolumn{1}{c|}{$18.810$} & \multicolumn{1}{c|}{$16.921$} & \multicolumn{1}{c|}{$15.870$} & \multicolumn{1}{c|}{$10.998$} & \multicolumn{1}{c|}{$4.575$} & \multicolumn{1}{c|}{$2.122$} & \multicolumn{1}{c|}{$0.457$} & \multicolumn{1}{c|}{$11.278$} & \multicolumn{1}{c|}{$4.493$} & \multicolumn{1}{c|}{$2.100$} & \multicolumn{1}{c|}{$0.453$} \\ \cline{3-15}
\multicolumn{1}{|c|}{} & \multicolumn{1}{c|}{} & Cover (ASE) & \multicolumn{1}{c|}{$0.556$} & \multicolumn{1}{c|}{$0.358$} & \multicolumn{1}{c|}{$0.094$} & \multicolumn{1}{c|}{$0.000$} & \multicolumn{1}{c|}{$0.896$} & \multicolumn{1}{c|}{$0.946$} & \multicolumn{1}{c|}{$0.922$} & \multicolumn{1}{c|}{$0.940$} & \multicolumn{1}{c|}{$0.894$} & \multicolumn{1}{c|}{$0.938$} & \multicolumn{1}{c|}{$0.920$} & \multicolumn{1}{c|}{$0.942$} \\ \cline{3-15}
\multicolumn{1}{|c|}{} & \multicolumn{1}{c|}{} & Cover (BSE) & \multicolumn{1}{c|}{$0.574$} & \multicolumn{1}{c|}{$0.440$} & \multicolumn{1}{c|}{$0.142$} & \multicolumn{1}{c|}{$0.000$} & \multicolumn{1}{c|}{$0.970$} & \multicolumn{1}{c|}{$0.988$} & \multicolumn{1}{c|}{$0.990$} & \multicolumn{1}{c|}{$0.998$} & \multicolumn{1}{c|}{$0.964$} & \multicolumn{1}{c|}{$0.984$} & \multicolumn{1}{c|}{$0.980$} & \multicolumn{1}{c|}{$0.996$} \\ \cline{2-15}
\multicolumn{1}{|c|}{} & \multicolumn{1}{c|}{\multirow{7}{*}{5}} & Bias ($\times$10) & \multicolumn{1}{c|}{$1.646$} & \multicolumn{1}{c|}{$1.686$} & \multicolumn{1}{c|}{$1.620$} & \multicolumn{1}{c|}{$1.630$} & \multicolumn{1}{c|}{$-0.160$} & \multicolumn{1}{c|}{$-0.002$} & \multicolumn{1}{c|}{$-0.012$} & \multicolumn{1}{c|}{$0.006$} & \multicolumn{1}{c|}{$0.135$} & \multicolumn{1}{c|}{$0.096$} & \multicolumn{1}{c|}{$0.040$} & \multicolumn{1}{c|}{$0.032$} \\ \cline{3-15}
\multicolumn{1}{|c|}{} & \multicolumn{1}{c|}{} & ASE ($\times$10) & \multicolumn{1}{c|}{$2.387$} & \multicolumn{1}{c|}{$1.667$} & \multicolumn{1}{c|}{$1.078$} & \multicolumn{1}{c|}{$0.551$} & \multicolumn{1}{c|}{$3.696$} & \multicolumn{1}{c|}{$2.262$} & \multicolumn{1}{c|}{$1.312$} & \multicolumn{1}{c|}{$0.628$} & \multicolumn{1}{c|}{$2.791$} & \multicolumn{1}{c|}{$1.915$} & \multicolumn{1}{c|}{$1.219$} & \multicolumn{1}{c|}{$0.608$} \\ \cline{3-15}
\multicolumn{1}{|c|}{} & \multicolumn{1}{c|}{} & BSE ($\times$10) & \multicolumn{1}{c|}{$2.854$} & \multicolumn{1}{c|}{$1.935$} & \multicolumn{1}{c|}{$1.270$} & \multicolumn{1}{c|}{$0.678$} & \multicolumn{1}{c|}{$4.907$} & \multicolumn{1}{c|}{$3.543$} & \multicolumn{1}{c|}{$2.232$} & \multicolumn{1}{c|}{$1.102$} & \multicolumn{1}{c|}{$4.768$} & \multicolumn{1}{c|}{$3.347$} & \multicolumn{1}{c|}{$2.121$} & \multicolumn{1}{c|}{$1.070$} \\ \cline{3-15}
\multicolumn{1}{|c|}{} & \multicolumn{1}{c|}{} & ESE ($\times$10) & \multicolumn{1}{c|}{$2.919$} & \multicolumn{1}{c|}{$2.010$} & \multicolumn{1}{c|}{$1.151$} & \multicolumn{1}{c|}{$0.590$} & \multicolumn{1}{c|}{$4.111$} & \multicolumn{1}{c|}{$2.335$} & \multicolumn{1}{c|}{$1.326$} & \multicolumn{1}{c|}{$0.669$} & \multicolumn{1}{c|}{$3.775$} & \multicolumn{1}{c|}{$2.221$} & \multicolumn{1}{c|}{$1.324$} & \multicolumn{1}{c|}{$0.650$} \\ \cline{3-15}
\multicolumn{1}{|c|}{} & \multicolumn{1}{c|}{} & MSE ($\times$100) & \multicolumn{1}{c|}{$11.212$} & \multicolumn{1}{c|}{$6.875$} & \multicolumn{1}{c|}{$3.947$} & \multicolumn{1}{c|}{$3.005$} & \multicolumn{1}{c|}{$16.895$} & \multicolumn{1}{c|}{$5.441$} & \multicolumn{1}{c|}{$1.754$} & \multicolumn{1}{c|}{$0.446$} & \multicolumn{1}{c|}{$14.244$} & \multicolumn{1}{c|}{$4.930$} & \multicolumn{1}{c|}{$1.752$} & \multicolumn{1}{c|}{$0.423$} \\ \cline{3-15}
\multicolumn{1}{|c|}{} & \multicolumn{1}{c|}{} & Cover (ASE) & \multicolumn{1}{c|}{$0.822$} & \multicolumn{1}{c|}{$0.778$} & \multicolumn{1}{c|}{$0.660$} & \multicolumn{1}{c|}{$0.180$} & \multicolumn{1}{c|}{$0.946$} & \multicolumn{1}{c|}{$0.946$} & \multicolumn{1}{c|}{$0.940$} & \multicolumn{1}{c|}{$0.932$} & \multicolumn{1}{c|}{$0.884$} & \multicolumn{1}{c|}{$0.914$} & \multicolumn{1}{c|}{$0.938$} & \multicolumn{1}{c|}{$0.928$} \\ \cline{3-15}
\multicolumn{1}{|c|}{} & \multicolumn{1}{c|}{} & Cover (BSE) & \multicolumn{1}{c|}{$0.872$} & \multicolumn{1}{c|}{$0.820$} & \multicolumn{1}{c|}{$0.756$} & \multicolumn{1}{c|}{$0.316$} & \multicolumn{1}{c|}{$0.956$} & \multicolumn{1}{c|}{$0.978$} & \multicolumn{1}{c|}{$0.996$} & \multicolumn{1}{c|}{$0.996$} & \multicolumn{1}{c|}{$0.964$} & \multicolumn{1}{c|}{$0.982$} & \multicolumn{1}{c|}{$0.990$} & \multicolumn{1}{c|}{$0.992$} \\ \cline{2-15}
\multicolumn{1}{|c|}{} & \multicolumn{1}{c|}{\multirow{7}{*}{9}} & Bias ($\times$10) & \multicolumn{1}{c|}{$1.358$} & \multicolumn{1}{c|}{$1.188$} & \multicolumn{1}{c|}{$1.236$} & \multicolumn{1}{c|}{$1.277$} & \multicolumn{1}{c|}{$-0.164$} & \multicolumn{1}{c|}{$-0.329$} & \multicolumn{1}{c|}{$-0.231$} & \multicolumn{1}{c|}{$-0.092$} & \multicolumn{1}{c|}{$0.058$} & \multicolumn{1}{c|}{$-0.300$} & \multicolumn{1}{c|}{$-0.348$} & \multicolumn{1}{c|}{$-0.245$} \\ \cline{3-15}
\multicolumn{1}{|c|}{} & \multicolumn{1}{c|}{} & ASE ($\times$10) & \multicolumn{1}{c|}{$2.503$} & \multicolumn{1}{c|}{$1.720$} & \multicolumn{1}{c|}{$1.096$} & \multicolumn{1}{c|}{$0.577$} & \multicolumn{1}{c|}{$3.955$} & \multicolumn{1}{c|}{$2.684$} & \multicolumn{1}{c|}{$1.572$} & \multicolumn{1}{c|}{$0.714$} & \multicolumn{1}{c|}{$3.202$} & \multicolumn{1}{c|}{$2.167$} & \multicolumn{1}{c|}{$1.320$} & \multicolumn{1}{c|}{$0.623$} \\ \cline{3-15}
\multicolumn{1}{|c|}{} & \multicolumn{1}{c|}{} & BSE ($\times$10) & \multicolumn{1}{c|}{$3.033$} & \multicolumn{1}{c|}{$2.028$} & \multicolumn{1}{c|}{$1.293$} & \multicolumn{1}{c|}{$0.713$} & \multicolumn{1}{c|}{$5.646$} & \multicolumn{1}{c|}{$3.904$} & \multicolumn{1}{c|}{$2.478$} & \multicolumn{1}{c|}{$1.309$} & \multicolumn{1}{c|}{$5.437$} & \multicolumn{1}{c|}{$3.907$} & \multicolumn{1}{c|}{$2.439$} & \multicolumn{1}{c|}{$1.254$} \\ \cline{3-15}
\multicolumn{1}{|c|}{} & \multicolumn{1}{c|}{} & ESE ($\times$10) & \multicolumn{1}{c|}{$3.137$} & \multicolumn{1}{c|}{$1.990$} & \multicolumn{1}{c|}{$1.259$} & \multicolumn{1}{c|}{$0.613$} & \multicolumn{1}{c|}{$4.005$} & \multicolumn{1}{c|}{$2.573$} & \multicolumn{1}{c|}{$1.502$} & \multicolumn{1}{c|}{$0.714$} & \multicolumn{1}{c|}{$3.747$} & \multicolumn{1}{c|}{$2.304$} & \multicolumn{1}{c|}{$1.379$} & \multicolumn{1}{c|}{$0.675$} \\ \cline{3-15}
\multicolumn{1}{|c|}{} & \multicolumn{1}{c|}{} & MSE ($\times$100) & \multicolumn{1}{c|}{$11.665$} & \multicolumn{1}{c|}{$5.363$} & \multicolumn{1}{c|}{$3.109$} & \multicolumn{1}{c|}{$2.005$} & \multicolumn{1}{c|}{$16.036$} & \multicolumn{1}{c|}{$6.713$} & \multicolumn{1}{c|}{$2.304$} & \multicolumn{1}{c|}{$0.517$} & \multicolumn{1}{c|}{$14.018$} & \multicolumn{1}{c|}{$5.386$} & \multicolumn{1}{c|}{$2.019$} & \multicolumn{1}{c|}{$0.515$} \\ \cline{3-15}
\multicolumn{1}{|c|}{} & \multicolumn{1}{c|}{} & Cover (ASE) & \multicolumn{1}{c|}{$0.842$} & \multicolumn{1}{c|}{$0.832$} & \multicolumn{1}{c|}{$0.742$} & \multicolumn{1}{c|}{$0.396$} & \multicolumn{1}{c|}{$0.954$} & \multicolumn{1}{c|}{$0.970$} & \multicolumn{1}{c|}{$0.946$} & \multicolumn{1}{c|}{$0.956$} & \multicolumn{1}{c|}{$0.930$} & \multicolumn{1}{c|}{$0.940$} & \multicolumn{1}{c|}{$0.926$} & \multicolumn{1}{c|}{$0.908$} \\ \cline{3-15}
\multicolumn{1}{|c|}{} & \multicolumn{1}{c|}{} & Cover (BSE) & \multicolumn{1}{c|}{$0.874$} & \multicolumn{1}{c|}{$0.866$} & \multicolumn{1}{c|}{$0.830$} & \multicolumn{1}{c|}{$0.564$} & \multicolumn{1}{c|}{$0.974$} & \multicolumn{1}{c|}{$0.982$} & \multicolumn{1}{c|}{$0.990$} & \multicolumn{1}{c|}{$0.996$} & \multicolumn{1}{c|}{$0.970$} & \multicolumn{1}{c|}{$0.990$} & \multicolumn{1}{c|}{$0.996$} & \multicolumn{1}{c|}{$0.988$} \\ \hline

\end{tabular}
\caption{Summary Statistics of the Estimation Results Under the Constant ATT $\tau_t^*=3$ for $t=T_0+1,\ldots,T$.} 
\label{tab:supp:Table00}
\end{table}

\newpage



Next, we consider an additional case where the ATT is linear as $\potY{t}{1} = \potY{t}{0} + \beta_0^* \ind (T_0< t) + \beta_1^* \ind (t-T_0)_+/T_0 + \epsilon_{\tau,t} $ where $\beta_0^*=3$ and $\beta_1^*=3$, and $(a)_{+} = \max(a,0)$. As in the main paper, we first present plots for the empirical distributions of the estimators. The plots have the same format as Figure \ref{fig:Sim:Constant} of the main paper, i.e.,
\begin{itemize}[itemsep=0cm,leftmargin=0.4cm]
\item The left, center, and right columns are associated with the number of donors $(d=2,5,9)$;
\item The top and bottom plots are associated with whether covariates are excluded $(\delta=0)$ or not $(\delta=1)$;
\item The vertical solid segments represent the range of the central 95\% of 500 estimates obtained by each estimation method;
\item The dots represent the empirical mean of 500 estimates obtained by each estimation method;
\item The light gray, gray, and black colors show the estimator types and the shape of the dots show the length of the pre-treatment period, respectively;
\item The red horizontal line shows the zero bias.
\end{itemize}
Figures \ref{fig:Sim:Linear1} and \ref{fig:Sim:Linear2} visually summarize the result. We remark that the OLS estimator is biased, especially for the intercept $\beta_0^*=3$.

% Figure environment removed	



\newpage

Next, we present the numerical summaries in Table \ref{tab:supp:Table0} and Table \ref{tab:supp:Table1}. We find that the SPSC estimator with ridge regularization performs the best, agreeing with the findings in the main paper.
\begin{table}[!htp]
\renewcommand{\arraystretch}{1.05} \centering
\footnotesize
\setlength{\tabcolsep}{3pt} 
\hspace*{-0.25in}
\begin{tabular}{|ccc|cccc|cccc|cccc|}
\hline
\multicolumn{3}{|c|}{Estimator} & \multicolumn{4}{c|}{OLS} & \multicolumn{4}{c|}{SPSC} & \multicolumn{4}{c|}{SPSC-Ridge} \\ \hline
\multicolumn{1}{|c|}{$\ \delta \ $} & \multicolumn{1}{c|}{$\ d \ $} & $T_0$ & \multicolumn{1}{c|}{50} & \multicolumn{1}{c|}{100} & \multicolumn{1}{c|}{250} & 1000 & \multicolumn{1}{c|}{50} & \multicolumn{1}{c|}{100} & \multicolumn{1}{c|}{250} & 1000 & \multicolumn{1}{c|}{50} & \multicolumn{1}{c|}{100} & \multicolumn{1}{c|}{250} & 1000 \\ \hline

\multicolumn{1}{|c|}{\multirow{21}{*}{0}} & \multicolumn{1}{c|}{\multirow{7}{*}{2}} & Bias ($\times$10) & \multicolumn{1}{c|}{$4.230$} & \multicolumn{1}{c|}{$3.675$} & \multicolumn{1}{c|}{$3.992$} & \multicolumn{1}{c|}{$3.955$} & \multicolumn{1}{c|}{$0.065$} & \multicolumn{1}{c|}{$-0.416$} & \multicolumn{1}{c|}{$-0.008$} & \multicolumn{1}{c|}{$0.006$} & \multicolumn{1}{c|}{$0.405$} & \multicolumn{1}{c|}{$-0.231$} & \multicolumn{1}{c|}{$0.067$} & \multicolumn{1}{c|}{$0.036$} \\ \cline{3-15}
\multicolumn{1}{|c|}{} & \multicolumn{1}{c|}{} & ASE ($\times$10) & \multicolumn{1}{c|}{$3.733$} & \multicolumn{1}{c|}{$2.809$} & \multicolumn{1}{c|}{$1.888$} & \multicolumn{1}{c|}{$1.008$} & \multicolumn{1}{c|}{$4.289$} & \multicolumn{1}{c|}{$3.176$} & \multicolumn{1}{c|}{$2.120$} & \multicolumn{1}{c|}{$1.123$} & \multicolumn{1}{c|}{$4.221$} & \multicolumn{1}{c|}{$3.153$} & \multicolumn{1}{c|}{$2.114$} & \multicolumn{1}{c|}{$1.122$} \\ \cline{3-15}
\multicolumn{1}{|c|}{} & \multicolumn{1}{c|}{} & BSE ($\times$10) & \multicolumn{1}{c|}{$6.131$} & \multicolumn{1}{c|}{$5.021$} & \multicolumn{1}{c|}{$3.677$} & \multicolumn{1}{c|}{$2.373$} & \multicolumn{1}{c|}{$6.537$} & \multicolumn{1}{c|}{$5.148$} & \multicolumn{1}{c|}{$3.751$} & \multicolumn{1}{c|}{$2.330$} & \multicolumn{1}{c|}{$6.423$} & \multicolumn{1}{c|}{$5.013$} & \multicolumn{1}{c|}{$3.735$} & \multicolumn{1}{c|}{$2.342$} \\ \cline{3-15}
\multicolumn{1}{|c|}{} & \multicolumn{1}{c|}{} & ESE ($\times$10) & \multicolumn{1}{c|}{$4.475$} & \multicolumn{1}{c|}{$3.443$} & \multicolumn{1}{c|}{$2.059$} & \multicolumn{1}{c|}{$1.115$} & \multicolumn{1}{c|}{$4.841$} & \multicolumn{1}{c|}{$3.892$} & \multicolumn{1}{c|}{$2.348$} & \multicolumn{1}{c|}{$1.237$} & \multicolumn{1}{c|}{$4.788$} & \multicolumn{1}{c|}{$3.853$} & \multicolumn{1}{c|}{$2.339$} & \multicolumn{1}{c|}{$1.236$} \\ \cline{3-15}
\multicolumn{1}{|c|}{} & \multicolumn{1}{c|}{} & MSE ($\times$100) & \multicolumn{1}{c|}{$37.875$} & \multicolumn{1}{c|}{$25.339$} & \multicolumn{1}{c|}{$20.167$} & \multicolumn{1}{c|}{$16.880$} & \multicolumn{1}{c|}{$23.389$} & \multicolumn{1}{c|}{$15.293$} & \multicolumn{1}{c|}{$5.504$} & \multicolumn{1}{c|}{$1.527$} & \multicolumn{1}{c|}{$23.045$} & \multicolumn{1}{c|}{$14.871$} & \multicolumn{1}{c|}{$5.466$} & \multicolumn{1}{c|}{$1.527$} \\ \cline{3-15}
\multicolumn{1}{|c|}{} & \multicolumn{1}{c|}{} & Cover (ASE) & \multicolumn{1}{c|}{$0.738$} & \multicolumn{1}{c|}{$0.678$} & \multicolumn{1}{c|}{$0.436$} & \multicolumn{1}{c|}{$0.040$} & \multicolumn{1}{c|}{$0.916$} & \multicolumn{1}{c|}{$0.874$} & \multicolumn{1}{c|}{$0.912$} & \multicolumn{1}{c|}{$0.934$} & \multicolumn{1}{c|}{$0.906$} & \multicolumn{1}{c|}{$0.876$} & \multicolumn{1}{c|}{$0.906$} & \multicolumn{1}{c|}{$0.932$} \\ \cline{3-15}
\multicolumn{1}{|c|}{} & \multicolumn{1}{c|}{} & Cover (BSE) & \multicolumn{1}{c|}{$0.872$} & \multicolumn{1}{c|}{$0.874$} & \multicolumn{1}{c|}{$0.860$} & \multicolumn{1}{c|}{$0.668$} & \multicolumn{1}{c|}{$0.972$} & \multicolumn{1}{c|}{$0.976$} & \multicolumn{1}{c|}{$0.978$} & \multicolumn{1}{c|}{$0.994$} & \multicolumn{1}{c|}{$0.970$} & \multicolumn{1}{c|}{$0.972$} & \multicolumn{1}{c|}{$0.988$} & \multicolumn{1}{c|}{$0.994$} \\ \cline{2-15}
\multicolumn{1}{|c|}{} & \multicolumn{1}{c|}{\multirow{7}{*}{5}} & Bias ($\times$10) & \multicolumn{1}{c|}{$2.059$} & \multicolumn{1}{c|}{$1.842$} & \multicolumn{1}{c|}{$1.829$} & \multicolumn{1}{c|}{$1.776$} & \multicolumn{1}{c|}{$0.180$} & \multicolumn{1}{c|}{$0.034$} & \multicolumn{1}{c|}{$0.029$} & \multicolumn{1}{c|}{$-0.096$} & \multicolumn{1}{c|}{$0.441$} & \multicolumn{1}{c|}{$0.221$} & \multicolumn{1}{c|}{$0.120$} & \multicolumn{1}{c|}{$-0.053$} \\ \cline{3-15}
\multicolumn{1}{|c|}{} & \multicolumn{1}{c|}{} & ASE ($\times$10) & \multicolumn{1}{c|}{$3.619$} & \multicolumn{1}{c|}{$2.715$} & \multicolumn{1}{c|}{$1.811$} & \multicolumn{1}{c|}{$0.959$} & \multicolumn{1}{c|}{$6.162$} & \multicolumn{1}{c|}{$3.751$} & \multicolumn{1}{c|}{$2.243$} & \multicolumn{1}{c|}{$1.076$} & \multicolumn{1}{c|}{$4.177$} & \multicolumn{1}{c|}{$3.034$} & \multicolumn{1}{c|}{$1.968$} & \multicolumn{1}{c|}{$1.028$} \\ \cline{3-15}
\multicolumn{1}{|c|}{} & \multicolumn{1}{c|}{} & BSE ($\times$10) & \multicolumn{1}{c|}{$6.060$} & \multicolumn{1}{c|}{$4.690$} & \multicolumn{1}{c|}{$3.525$} & \multicolumn{1}{c|}{$2.238$} & \multicolumn{1}{c|}{$8.167$} & \multicolumn{1}{c|}{$5.751$} & \multicolumn{1}{c|}{$4.007$} & \multicolumn{1}{c|}{$2.382$} & \multicolumn{1}{c|}{$6.474$} & \multicolumn{1}{c|}{$5.047$} & \multicolumn{1}{c|}{$3.734$} & \multicolumn{1}{c|}{$2.328$} \\ \cline{3-15}
\multicolumn{1}{|c|}{} & \multicolumn{1}{c|}{} & ESE ($\times$10) & \multicolumn{1}{c|}{$4.621$} & \multicolumn{1}{c|}{$3.419$} & \multicolumn{1}{c|}{$2.052$} & \multicolumn{1}{c|}{$0.982$} & \multicolumn{1}{c|}{$5.795$} & \multicolumn{1}{c|}{$4.074$} & \multicolumn{1}{c|}{$2.292$} & \multicolumn{1}{c|}{$1.086$} & \multicolumn{1}{c|}{$4.931$} & \multicolumn{1}{c|}{$3.628$} & \multicolumn{1}{c|}{$2.147$} & \multicolumn{1}{c|}{$1.049$} \\ \cline{3-15}
\multicolumn{1}{|c|}{} & \multicolumn{1}{c|}{} & MSE ($\times$100) & \multicolumn{1}{c|}{$25.555$} & \multicolumn{1}{c|}{$15.060$} & \multicolumn{1}{c|}{$7.544$} & \multicolumn{1}{c|}{$4.116$} & \multicolumn{1}{c|}{$33.549$} & \multicolumn{1}{c|}{$16.563$} & \multicolumn{1}{c|}{$5.246$} & \multicolumn{1}{c|}{$1.186$} & \multicolumn{1}{c|}{$24.456$} & \multicolumn{1}{c|}{$13.184$} & \multicolumn{1}{c|}{$4.614$} & \multicolumn{1}{c|}{$1.100$} \\ \cline{3-15}
\multicolumn{1}{|c|}{} & \multicolumn{1}{c|}{} & Cover (ASE) & \multicolumn{1}{c|}{$0.814$} & \multicolumn{1}{c|}{$0.818$} & \multicolumn{1}{c|}{$0.788$} & \multicolumn{1}{c|}{$0.534$} & \multicolumn{1}{c|}{$0.924$} & \multicolumn{1}{c|}{$0.916$} & \multicolumn{1}{c|}{$0.942$} & \multicolumn{1}{c|}{$0.950$} & \multicolumn{1}{c|}{$0.866$} & \multicolumn{1}{c|}{$0.874$} & \multicolumn{1}{c|}{$0.928$} & \multicolumn{1}{c|}{$0.944$} \\ \cline{3-15}
\multicolumn{1}{|c|}{} & \multicolumn{1}{c|}{} & Cover (BSE) & \multicolumn{1}{c|}{$0.914$} & \multicolumn{1}{c|}{$0.928$} & \multicolumn{1}{c|}{$0.966$} & \multicolumn{1}{c|}{$0.970$} & \multicolumn{1}{c|}{$0.966$} & \multicolumn{1}{c|}{$0.980$} & \multicolumn{1}{c|}{$0.992$} & \multicolumn{1}{c|}{$1.000$} & \multicolumn{1}{c|}{$0.944$} & \multicolumn{1}{c|}{$0.982$} & \multicolumn{1}{c|}{$0.990$} & \multicolumn{1}{c|}{$1.000$} \\ \cline{2-15}
\multicolumn{1}{|c|}{} & \multicolumn{1}{c|}{\multirow{7}{*}{9}} & Bias ($\times$10) & \multicolumn{1}{c|}{$1.878$} & \multicolumn{1}{c|}{$1.713$} & \multicolumn{1}{c|}{$1.955$} & \multicolumn{1}{c|}{$1.940$} & \multicolumn{1}{c|}{$0.265$} & \multicolumn{1}{c|}{$0.043$} & \multicolumn{1}{c|}{$0.184$} & \multicolumn{1}{c|}{$0.120$} & \multicolumn{1}{c|}{$0.451$} & \multicolumn{1}{c|}{$0.139$} & \multicolumn{1}{c|}{$0.185$} & \multicolumn{1}{c|}{$0.130$} \\ \cline{3-15}
\multicolumn{1}{|c|}{} & \multicolumn{1}{c|}{} & ASE ($\times$10) & \multicolumn{1}{c|}{$3.701$} & \multicolumn{1}{c|}{$2.726$} & \multicolumn{1}{c|}{$1.807$} & \multicolumn{1}{c|}{$0.956$} & \multicolumn{1}{c|}{$5.473$} & \multicolumn{1}{c|}{$3.932$} & \multicolumn{1}{c|}{$2.505$} & \multicolumn{1}{c|}{$1.228$} & \multicolumn{1}{c|}{$4.394$} & \multicolumn{1}{c|}{$3.288$} & \multicolumn{1}{c|}{$2.122$} & \multicolumn{1}{c|}{$1.104$} \\ \cline{3-15}
\multicolumn{1}{|c|}{} & \multicolumn{1}{c|}{} & BSE ($\times$10) & \multicolumn{1}{c|}{$6.016$} & \multicolumn{1}{c|}{$4.761$} & \multicolumn{1}{c|}{$3.427$} & \multicolumn{1}{c|}{$2.223$} & \multicolumn{1}{c|}{$8.047$} & \multicolumn{1}{c|}{$5.700$} & \multicolumn{1}{c|}{$3.941$} & \multicolumn{1}{c|}{$2.407$} & \multicolumn{1}{c|}{$6.974$} & \multicolumn{1}{c|}{$5.252$} & \multicolumn{1}{c|}{$3.805$} & \multicolumn{1}{c|}{$2.299$} \\ \cline{3-15}
\multicolumn{1}{|c|}{} & \multicolumn{1}{c|}{} & ESE ($\times$10) & \multicolumn{1}{c|}{$4.094$} & \multicolumn{1}{c|}{$3.362$} & \multicolumn{1}{c|}{$1.972$} & \multicolumn{1}{c|}{$1.017$} & \multicolumn{1}{c|}{$4.964$} & \multicolumn{1}{c|}{$3.878$} & \multicolumn{1}{c|}{$2.328$} & \multicolumn{1}{c|}{$1.158$} & \multicolumn{1}{c|}{$4.548$} & \multicolumn{1}{c|}{$3.588$} & \multicolumn{1}{c|}{$2.187$} & \multicolumn{1}{c|}{$1.108$} \\ \cline{3-15}
\multicolumn{1}{|c|}{} & \multicolumn{1}{c|}{} & MSE ($\times$100) & \multicolumn{1}{c|}{$20.250$} & \multicolumn{1}{c|}{$14.215$} & \multicolumn{1}{c|}{$7.705$} & \multicolumn{1}{c|}{$4.795$} & \multicolumn{1}{c|}{$24.663$} & \multicolumn{1}{c|}{$15.007$} & \multicolumn{1}{c|}{$5.445$} & \multicolumn{1}{c|}{$1.353$} & \multicolumn{1}{c|}{$20.849$} & \multicolumn{1}{c|}{$12.869$} & \multicolumn{1}{c|}{$4.807$} & \multicolumn{1}{c|}{$1.243$} \\ \cline{3-15}
\multicolumn{1}{|c|}{} & \multicolumn{1}{c|}{} & Cover (ASE) & \multicolumn{1}{c|}{$0.882$} & \multicolumn{1}{c|}{$0.822$} & \multicolumn{1}{c|}{$0.770$} & \multicolumn{1}{c|}{$0.450$} & \multicolumn{1}{c|}{$0.948$} & \multicolumn{1}{c|}{$0.946$} & \multicolumn{1}{c|}{$0.946$} & \multicolumn{1}{c|}{$0.968$} & \multicolumn{1}{c|}{$0.920$} & \multicolumn{1}{c|}{$0.912$} & \multicolumn{1}{c|}{$0.940$} & \multicolumn{1}{c|}{$0.946$} \\ \cline{3-15}
\multicolumn{1}{|c|}{} & \multicolumn{1}{c|}{} & Cover (BSE) & \multicolumn{1}{c|}{$0.948$} & \multicolumn{1}{c|}{$0.934$} & \multicolumn{1}{c|}{$0.942$} & \multicolumn{1}{c|}{$0.968$} & \multicolumn{1}{c|}{$0.980$} & \multicolumn{1}{c|}{$0.974$} & \multicolumn{1}{c|}{$0.988$} & \multicolumn{1}{c|}{$0.998$} & \multicolumn{1}{c|}{$0.972$} & \multicolumn{1}{c|}{$0.972$} & \multicolumn{1}{c|}{$0.988$} & \multicolumn{1}{c|}{$1.000$} \\ \hline \hline
\multicolumn{1}{|c|}{\multirow{21}{*}{1}} & \multicolumn{1}{c|}{\multirow{7}{*}{2}} & Bias ($\times$10) & \multicolumn{1}{c|}{$3.722$} & \multicolumn{1}{c|}{$3.518$} & \multicolumn{1}{c|}{$3.430$} & \multicolumn{1}{c|}{$3.455$} & \multicolumn{1}{c|}{$-0.037$} & \multicolumn{1}{c|}{$-0.130$} & \multicolumn{1}{c|}{$-0.128$} & \multicolumn{1}{c|}{$-0.024$} & \multicolumn{1}{c|}{$0.291$} & \multicolumn{1}{c|}{$-0.009$} & \multicolumn{1}{c|}{$-0.073$} & \multicolumn{1}{c|}{$-0.004$} \\ \cline{3-15}
\multicolumn{1}{|c|}{} & \multicolumn{1}{c|}{} & ASE ($\times$10) & \multicolumn{1}{c|}{$3.904$} & \multicolumn{1}{c|}{$2.889$} & \multicolumn{1}{c|}{$1.936$} & \multicolumn{1}{c|}{$1.010$} & \multicolumn{1}{c|}{$4.460$} & \multicolumn{1}{c|}{$3.259$} & \multicolumn{1}{c|}{$2.157$} & \multicolumn{1}{c|}{$1.124$} & \multicolumn{1}{c|}{$4.396$} & \multicolumn{1}{c|}{$3.241$} & \multicolumn{1}{c|}{$2.152$} & \multicolumn{1}{c|}{$1.123$} \\ \cline{3-15}
\multicolumn{1}{|c|}{} & \multicolumn{1}{c|}{} & BSE ($\times$10) & \multicolumn{1}{c|}{$6.223$} & \multicolumn{1}{c|}{$4.843$} & \multicolumn{1}{c|}{$3.717$} & \multicolumn{1}{c|}{$2.356$} & \multicolumn{1}{c|}{$7.106$} & \multicolumn{1}{c|}{$5.507$} & \multicolumn{1}{c|}{$4.038$} & \multicolumn{1}{c|}{$2.418$} & \multicolumn{1}{c|}{$7.298$} & \multicolumn{1}{c|}{$5.545$} & \multicolumn{1}{c|}{$4.099$} & \multicolumn{1}{c|}{$2.451$} \\ \cline{3-15}
\multicolumn{1}{|c|}{} & \multicolumn{1}{c|}{} & ESE ($\times$10) & \multicolumn{1}{c|}{$4.501$} & \multicolumn{1}{c|}{$3.445$} & \multicolumn{1}{c|}{$2.166$} & \multicolumn{1}{c|}{$1.060$} & \multicolumn{1}{c|}{$5.086$} & \multicolumn{1}{c|}{$3.713$} & \multicolumn{1}{c|}{$2.454$} & \multicolumn{1}{c|}{$1.172$} & \multicolumn{1}{c|}{$5.068$} & \multicolumn{1}{c|}{$3.703$} & \multicolumn{1}{c|}{$2.448$} & \multicolumn{1}{c|}{$1.170$} \\ \cline{3-15}
\multicolumn{1}{|c|}{} & \multicolumn{1}{c|}{} & MSE ($\times$100) & \multicolumn{1}{c|}{$34.072$} & \multicolumn{1}{c|}{$24.224$} & \multicolumn{1}{c|}{$16.447$} & \multicolumn{1}{c|}{$13.061$} & \multicolumn{1}{c|}{$25.821$} & \multicolumn{1}{c|}{$13.778$} & \multicolumn{1}{c|}{$6.028$} & \multicolumn{1}{c|}{$1.371$} & \multicolumn{1}{c|}{$25.719$} & \multicolumn{1}{c|}{$13.687$} & \multicolumn{1}{c|}{$5.988$} & \multicolumn{1}{c|}{$1.366$} \\ \cline{3-15}
\multicolumn{1}{|c|}{} & \multicolumn{1}{c|}{} & Cover (ASE) & \multicolumn{1}{c|}{$0.790$} & \multicolumn{1}{c|}{$0.744$} & \multicolumn{1}{c|}{$0.564$} & \multicolumn{1}{c|}{$0.090$} & \multicolumn{1}{c|}{$0.900$} & \multicolumn{1}{c|}{$0.894$} & \multicolumn{1}{c|}{$0.916$} & \multicolumn{1}{c|}{$0.930$} & \multicolumn{1}{c|}{$0.896$} & \multicolumn{1}{c|}{$0.898$} & \multicolumn{1}{c|}{$0.914$} & \multicolumn{1}{c|}{$0.930$} \\ \cline{3-15}
\multicolumn{1}{|c|}{} & \multicolumn{1}{c|}{} & Cover (BSE) & \multicolumn{1}{c|}{$0.886$} & \multicolumn{1}{c|}{$0.880$} & \multicolumn{1}{c|}{$0.900$} & \multicolumn{1}{c|}{$0.756$} & \multicolumn{1}{c|}{$0.970$} & \multicolumn{1}{c|}{$0.982$} & \multicolumn{1}{c|}{$0.992$} & \multicolumn{1}{c|}{$1.000$} & \multicolumn{1}{c|}{$0.978$} & \multicolumn{1}{c|}{$0.978$} & \multicolumn{1}{c|}{$0.992$} & \multicolumn{1}{c|}{$1.000$} \\ \cline{2-15}
\multicolumn{1}{|c|}{} & \multicolumn{1}{c|}{\multirow{7}{*}{5}} & Bias ($\times$10) & \multicolumn{1}{c|}{$1.527$} & \multicolumn{1}{c|}{$1.462$} & \multicolumn{1}{c|}{$1.444$} & \multicolumn{1}{c|}{$1.469$} & \multicolumn{1}{c|}{$-0.140$} & \multicolumn{1}{c|}{$0.038$} & \multicolumn{1}{c|}{$-0.091$} & \multicolumn{1}{c|}{$-0.084$} & \multicolumn{1}{c|}{$0.297$} & \multicolumn{1}{c|}{$0.202$} & \multicolumn{1}{c|}{$0.052$} & \multicolumn{1}{c|}{$-0.031$} \\ \cline{3-15}
\multicolumn{1}{|c|}{} & \multicolumn{1}{c|}{} & ASE ($\times$10) & \multicolumn{1}{c|}{$3.913$} & \multicolumn{1}{c|}{$2.827$} & \multicolumn{1}{c|}{$1.876$} & \multicolumn{1}{c|}{$0.973$} & \multicolumn{1}{c|}{$5.415$} & \multicolumn{1}{c|}{$3.490$} & \multicolumn{1}{c|}{$2.115$} & \multicolumn{1}{c|}{$1.049$} & \multicolumn{1}{c|}{$4.338$} & \multicolumn{1}{c|}{$3.082$} & \multicolumn{1}{c|}{$2.004$} & \multicolumn{1}{c|}{$1.029$} \\ \cline{3-15}
\multicolumn{1}{|c|}{} & \multicolumn{1}{c|}{} & BSE ($\times$10) & \multicolumn{1}{c|}{$6.405$} & \multicolumn{1}{c|}{$4.794$} & \multicolumn{1}{c|}{$3.589$} & \multicolumn{1}{c|}{$2.273$} & \multicolumn{1}{c|}{$7.771$} & \multicolumn{1}{c|}{$5.668$} & \multicolumn{1}{c|}{$4.029$} & \multicolumn{1}{c|}{$2.363$} & \multicolumn{1}{c|}{$7.649$} & \multicolumn{1}{c|}{$5.481$} & \multicolumn{1}{c|}{$3.949$} & \multicolumn{1}{c|}{$2.389$} \\ \cline{3-15}
\multicolumn{1}{|c|}{} & \multicolumn{1}{c|}{} & ESE ($\times$10) & \multicolumn{1}{c|}{$4.941$} & \multicolumn{1}{c|}{$3.387$} & \multicolumn{1}{c|}{$2.002$} & \multicolumn{1}{c|}{$0.984$} & \multicolumn{1}{c|}{$5.933$} & \multicolumn{1}{c|}{$3.883$} & \multicolumn{1}{c|}{$2.215$} & \multicolumn{1}{c|}{$1.055$} & \multicolumn{1}{c|}{$5.303$} & \multicolumn{1}{c|}{$3.627$} & \multicolumn{1}{c|}{$2.163$} & \multicolumn{1}{c|}{$1.045$} \\ \cline{3-15}
\multicolumn{1}{|c|}{} & \multicolumn{1}{c|}{} & MSE ($\times$100) & \multicolumn{1}{c|}{$26.693$} & \multicolumn{1}{c|}{$13.584$} & \multicolumn{1}{c|}{$6.084$} & \multicolumn{1}{c|}{$3.124$} & \multicolumn{1}{c|}{$35.154$} & \multicolumn{1}{c|}{$15.049$} & \multicolumn{1}{c|}{$4.905$} & \multicolumn{1}{c|}{$1.117$} & \multicolumn{1}{c|}{$28.150$} & \multicolumn{1}{c|}{$13.168$} & \multicolumn{1}{c|}{$4.674$} & \multicolumn{1}{c|}{$1.090$} \\ \cline{3-15}
\multicolumn{1}{|c|}{} & \multicolumn{1}{c|}{} & Cover (ASE) & \multicolumn{1}{c|}{$0.850$} & \multicolumn{1}{c|}{$0.850$} & \multicolumn{1}{c|}{$0.852$} & \multicolumn{1}{c|}{$0.652$} & \multicolumn{1}{c|}{$0.920$} & \multicolumn{1}{c|}{$0.902$} & \multicolumn{1}{c|}{$0.936$} & \multicolumn{1}{c|}{$0.958$} & \multicolumn{1}{c|}{$0.876$} & \multicolumn{1}{c|}{$0.888$} & \multicolumn{1}{c|}{$0.930$} & \multicolumn{1}{c|}{$0.952$} \\ \cline{3-15}
\multicolumn{1}{|c|}{} & \multicolumn{1}{c|}{} & Cover (BSE) & \multicolumn{1}{c|}{$0.930$} & \multicolumn{1}{c|}{$0.946$} & \multicolumn{1}{c|}{$0.980$} & \multicolumn{1}{c|}{$0.992$} & \multicolumn{1}{c|}{$0.964$} & \multicolumn{1}{c|}{$0.984$} & \multicolumn{1}{c|}{$0.998$} & \multicolumn{1}{c|}{$0.998$} & \multicolumn{1}{c|}{$0.958$} & \multicolumn{1}{c|}{$0.986$} & \multicolumn{1}{c|}{$0.996$} & \multicolumn{1}{c|}{$1.000$} \\ \cline{2-15}
\multicolumn{1}{|c|}{} & \multicolumn{1}{c|}{\multirow{7}{*}{9}} & Bias ($\times$10) & \multicolumn{1}{c|}{$1.761$} & \multicolumn{1}{c|}{$1.705$} & \multicolumn{1}{c|}{$1.682$} & \multicolumn{1}{c|}{$1.605$} & \multicolumn{1}{c|}{$0.389$} & \multicolumn{1}{c|}{$0.120$} & \multicolumn{1}{c|}{$0.103$} & \multicolumn{1}{c|}{$-0.011$} & \multicolumn{1}{c|}{$0.625$} & \multicolumn{1}{c|}{$0.178$} & \multicolumn{1}{c|}{$0.129$} & \multicolumn{1}{c|}{$-0.017$} \\ \cline{3-15}
\multicolumn{1}{|c|}{} & \multicolumn{1}{c|}{} & ASE ($\times$10) & \multicolumn{1}{c|}{$4.131$} & \multicolumn{1}{c|}{$2.922$} & \multicolumn{1}{c|}{$1.908$} & \multicolumn{1}{c|}{$0.980$} & \multicolumn{1}{c|}{$5.894$} & \multicolumn{1}{c|}{$3.946$} & \multicolumn{1}{c|}{$2.362$} & \multicolumn{1}{c|}{$1.135$} & \multicolumn{1}{c|}{$4.914$} & \multicolumn{1}{c|}{$3.416$} & \multicolumn{1}{c|}{$2.110$} & \multicolumn{1}{c|}{$1.039$} \\ \cline{3-15}
\multicolumn{1}{|c|}{} & \multicolumn{1}{c|}{} & BSE ($\times$10) & \multicolumn{1}{c|}{$6.291$} & \multicolumn{1}{c|}{$4.825$} & \multicolumn{1}{c|}{$3.508$} & \multicolumn{1}{c|}{$2.245$} & \multicolumn{1}{c|}{$8.468$} & \multicolumn{1}{c|}{$6.127$} & \multicolumn{1}{c|}{$4.043$} & \multicolumn{1}{c|}{$2.477$} & \multicolumn{1}{c|}{$8.370$} & \multicolumn{1}{c|}{$6.053$} & \multicolumn{1}{c|}{$4.128$} & \multicolumn{1}{c|}{$2.467$} \\ \cline{3-15}
\multicolumn{1}{|c|}{} & \multicolumn{1}{c|}{} & ESE ($\times$10) & \multicolumn{1}{c|}{$4.764$} & \multicolumn{1}{c|}{$3.277$} & \multicolumn{1}{c|}{$2.084$} & \multicolumn{1}{c|}{$1.001$} & \multicolumn{1}{c|}{$5.822$} & \multicolumn{1}{c|}{$3.782$} & \multicolumn{1}{c|}{$2.252$} & \multicolumn{1}{c|}{$1.072$} & \multicolumn{1}{c|}{$5.183$} & \multicolumn{1}{c|}{$3.507$} & \multicolumn{1}{c|}{$2.216$} & \multicolumn{1}{c|}{$1.033$} \\ \cline{3-15}
\multicolumn{1}{|c|}{} & \multicolumn{1}{c|}{} & MSE ($\times$100) & \multicolumn{1}{c|}{$25.753$} & \multicolumn{1}{c|}{$13.625$} & \multicolumn{1}{c|}{$7.163$} & \multicolumn{1}{c|}{$3.578$} & \multicolumn{1}{c|}{$33.981$} & \multicolumn{1}{c|}{$14.289$} & \multicolumn{1}{c|}{$5.074$} & \multicolumn{1}{c|}{$1.146$} & \multicolumn{1}{c|}{$27.198$} & \multicolumn{1}{c|}{$12.308$} & \multicolumn{1}{c|}{$4.919$} & \multicolumn{1}{c|}{$1.066$} \\ \cline{3-15}
\multicolumn{1}{|c|}{} & \multicolumn{1}{c|}{} & Cover (ASE) & \multicolumn{1}{c|}{$0.878$} & \multicolumn{1}{c|}{$0.860$} & \multicolumn{1}{c|}{$0.836$} & \multicolumn{1}{c|}{$0.624$} & \multicolumn{1}{c|}{$0.942$} & \multicolumn{1}{c|}{$0.954$} & \multicolumn{1}{c|}{$0.942$} & \multicolumn{1}{c|}{$0.954$} & \multicolumn{1}{c|}{$0.920$} & \multicolumn{1}{c|}{$0.932$} & \multicolumn{1}{c|}{$0.914$} & \multicolumn{1}{c|}{$0.940$} \\ \cline{3-15}
\multicolumn{1}{|c|}{} & \multicolumn{1}{c|}{} & Cover (BSE) & \multicolumn{1}{c|}{$0.934$} & \multicolumn{1}{c|}{$0.964$} & \multicolumn{1}{c|}{$0.964$} & \multicolumn{1}{c|}{$0.986$} & \multicolumn{1}{c|}{$0.968$} & \multicolumn{1}{c|}{$0.990$} & \multicolumn{1}{c|}{$0.990$} & \multicolumn{1}{c|}{$0.998$} & \multicolumn{1}{c|}{$0.966$} & \multicolumn{1}{c|}{$0.992$} & \multicolumn{1}{c|}{$0.996$} & \multicolumn{1}{c|}{$1.000$} \\ \hline

\end{tabular}
\caption{Summary Statistics of the Estimation Results for $\beta_{0}^*=3$.} 
\label{tab:supp:Table0}
\end{table}


\newpage


\begin{table}[!htp]
\renewcommand{\arraystretch}{1.05} \centering
\footnotesize
\setlength{\tabcolsep}{3pt} 
\hspace*{-0.25in}
\begin{tabular}{|ccc|cccc|cccc|cccc|}
\hline
\multicolumn{3}{|c|}{Estimator} & \multicolumn{4}{c|}{OLS} & \multicolumn{4}{c|}{SPSC} & \multicolumn{4}{c|}{SPSC-Ridge} \\ \hline
\multicolumn{1}{|c|}{$\ \delta \ $} & \multicolumn{1}{c|}{$\ d \ $} & $T_0$ & \multicolumn{1}{c|}{50} & \multicolumn{1}{c|}{100} & \multicolumn{1}{c|}{250} & 1000 & \multicolumn{1}{c|}{50} & \multicolumn{1}{c|}{100} & \multicolumn{1}{c|}{250} & 1000 & \multicolumn{1}{c|}{50} & \multicolumn{1}{c|}{100} & \multicolumn{1}{c|}{250} & 1000 \\ \hline

\multicolumn{1}{|c|}{\multirow{21}{*}{0}} & \multicolumn{1}{c|}{\multirow{7}{*}{2}} & Bias ($\times$10) & \multicolumn{1}{c|}{$1.593$} & \multicolumn{1}{c|}{$2.404$} & \multicolumn{1}{c|}{$1.988$} & \multicolumn{1}{c|}{$1.990$} & \multicolumn{1}{c|}{$-0.527$} & \multicolumn{1}{c|}{$0.435$} & \multicolumn{1}{c|}{$-0.078$} & \multicolumn{1}{c|}{$-0.019$} & \multicolumn{1}{c|}{$-0.158$} & \multicolumn{1}{c|}{$0.648$} & \multicolumn{1}{c|}{$0.018$} & \multicolumn{1}{c|}{$0.023$} \\ \cline{3-15}
\multicolumn{1}{|c|}{} & \multicolumn{1}{c|}{} & ASE ($\times$10) & \multicolumn{1}{c|}{$6.261$} & \multicolumn{1}{c|}{$4.786$} & \multicolumn{1}{c|}{$3.213$} & \multicolumn{1}{c|}{$1.722$} & \multicolumn{1}{c|}{$6.967$} & \multicolumn{1}{c|}{$5.243$} & \multicolumn{1}{c|}{$3.516$} & \multicolumn{1}{c|}{$1.868$} & \multicolumn{1}{c|}{$6.880$} & \multicolumn{1}{c|}{$5.215$} & \multicolumn{1}{c|}{$3.508$} & \multicolumn{1}{c|}{$1.867$} \\ \cline{3-15}
\multicolumn{1}{|c|}{} & \multicolumn{1}{c|}{} & BSE ($\times$10) & \multicolumn{1}{c|}{$10.319$} & \multicolumn{1}{c|}{$8.607$} & \multicolumn{1}{c|}{$6.442$} & \multicolumn{1}{c|}{$4.203$} & \multicolumn{1}{c|}{$10.677$} & \multicolumn{1}{c|}{$8.659$} & \multicolumn{1}{c|}{$6.388$} & \multicolumn{1}{c|}{$4.107$} & \multicolumn{1}{c|}{$10.567$} & \multicolumn{1}{c|}{$8.509$} & \multicolumn{1}{c|}{$6.422$} & \multicolumn{1}{c|}{$4.138$} \\ \cline{3-15}
\multicolumn{1}{|c|}{} & \multicolumn{1}{c|}{} & ESE ($\times$10) & \multicolumn{1}{c|}{$7.815$} & \multicolumn{1}{c|}{$5.807$} & \multicolumn{1}{c|}{$3.529$} & \multicolumn{1}{c|}{$1.854$} & \multicolumn{1}{c|}{$8.576$} & \multicolumn{1}{c|}{$6.311$} & \multicolumn{1}{c|}{$3.843$} & \multicolumn{1}{c|}{$2.021$} & \multicolumn{1}{c|}{$8.525$} & \multicolumn{1}{c|}{$6.302$} & \multicolumn{1}{c|}{$3.843$} & \multicolumn{1}{c|}{$2.016$} \\ \cline{3-15}
\multicolumn{1}{|c|}{} & \multicolumn{1}{c|}{} & MSE ($\times$100) & \multicolumn{1}{c|}{$63.489$} & \multicolumn{1}{c|}{$39.435$} & \multicolumn{1}{c|}{$16.381$} & \multicolumn{1}{c|}{$7.391$} & \multicolumn{1}{c|}{$73.683$} & \multicolumn{1}{c|}{$39.942$} & \multicolumn{1}{c|}{$14.746$} & \multicolumn{1}{c|}{$4.075$} & \multicolumn{1}{c|}{$72.557$} & \multicolumn{1}{c|}{$40.052$} & \multicolumn{1}{c|}{$14.740$} & \multicolumn{1}{c|}{$4.059$} \\ \cline{3-15}
\multicolumn{1}{|c|}{} & \multicolumn{1}{c|}{} & Cover (ASE) & \multicolumn{1}{c|}{$0.866$} & \multicolumn{1}{c|}{$0.850$} & \multicolumn{1}{c|}{$0.866$} & \multicolumn{1}{c|}{$0.780$} & \multicolumn{1}{c|}{$0.880$} & \multicolumn{1}{c|}{$0.878$} & \multicolumn{1}{c|}{$0.924$} & \multicolumn{1}{c|}{$0.948$} & \multicolumn{1}{c|}{$0.878$} & \multicolumn{1}{c|}{$0.878$} & \multicolumn{1}{c|}{$0.922$} & \multicolumn{1}{c|}{$0.948$} \\ \cline{3-15}
\multicolumn{1}{|c|}{} & \multicolumn{1}{c|}{} & Cover (BSE) & \multicolumn{1}{c|}{$0.972$} & \multicolumn{1}{c|}{$0.974$} & \multicolumn{1}{c|}{$0.998$} & \multicolumn{1}{c|}{$1.000$} & \multicolumn{1}{c|}{$0.956$} & \multicolumn{1}{c|}{$0.974$} & \multicolumn{1}{c|}{$0.984$} & \multicolumn{1}{c|}{$0.994$} & \multicolumn{1}{c|}{$0.952$} & \multicolumn{1}{c|}{$0.966$} & \multicolumn{1}{c|}{$0.988$} & \multicolumn{1}{c|}{$0.994$} \\ \cline{2-15}
\multicolumn{1}{|c|}{} & \multicolumn{1}{c|}{\multirow{7}{*}{5}} & Bias ($\times$10) & \multicolumn{1}{c|}{$-0.370$} & \multicolumn{1}{c|}{$-0.075$} & \multicolumn{1}{c|}{$0.179$} & \multicolumn{1}{c|}{$0.291$} & \multicolumn{1}{c|}{$-0.345$} & \multicolumn{1}{c|}{$-0.093$} & \multicolumn{1}{c|}{$0.528$} & \multicolumn{1}{c|}{$0.332$} & \multicolumn{1}{c|}{$-1.301$} & \multicolumn{1}{c|}{$-0.679$} & \multicolumn{1}{c|}{$0.026$} & \multicolumn{1}{c|}{$0.384$} \\ \cline{3-15}
\multicolumn{1}{|c|}{} & \multicolumn{1}{c|}{} & ASE ($\times$10) & \multicolumn{1}{c|}{$6.275$} & \multicolumn{1}{c|}{$4.710$} & \multicolumn{1}{c|}{$3.164$} & \multicolumn{1}{c|}{$1.670$} & \multicolumn{1}{c|}{$11.299$} & \multicolumn{1}{c|}{$6.747$} & \multicolumn{1}{c|}{$4.183$} & \multicolumn{1}{c|}{$1.969$} & \multicolumn{1}{c|}{$7.370$} & \multicolumn{1}{c|}{$5.312$} & \multicolumn{1}{c|}{$3.472$} & \multicolumn{1}{c|}{$1.804$} \\ \cline{3-15}
\multicolumn{1}{|c|}{} & \multicolumn{1}{c|}{} & BSE ($\times$10) & \multicolumn{1}{c|}{$9.965$} & \multicolumn{1}{c|}{$7.980$} & \multicolumn{1}{c|}{$6.167$} & \multicolumn{1}{c|}{$3.946$} & \multicolumn{1}{c|}{$11.499$} & \multicolumn{1}{c|}{$8.774$} & \multicolumn{1}{c|}{$6.596$} & \multicolumn{1}{c|}{$4.068$} & \multicolumn{1}{c|}{$10.297$} & \multicolumn{1}{c|}{$8.257$} & \multicolumn{1}{c|}{$6.256$} & \multicolumn{1}{c|}{$4.011$} \\ \cline{3-15}
\multicolumn{1}{|c|}{} & \multicolumn{1}{c|}{} & ESE ($\times$10) & \multicolumn{1}{c|}{$8.016$} & \multicolumn{1}{c|}{$5.903$} & \multicolumn{1}{c|}{$3.377$} & \multicolumn{1}{c|}{$1.740$} & \multicolumn{1}{c|}{$9.887$} & \multicolumn{1}{c|}{$7.479$} & \multicolumn{1}{c|}{$4.133$} & \multicolumn{1}{c|}{$1.954$} & \multicolumn{1}{c|}{$8.509$} & \multicolumn{1}{c|}{$6.570$} & \multicolumn{1}{c|}{$3.658$} & \multicolumn{1}{c|}{$1.845$} \\ \cline{3-15}
\multicolumn{1}{|c|}{} & \multicolumn{1}{c|}{} & MSE ($\times$100) & \multicolumn{1}{c|}{$64.259$} & \multicolumn{1}{c|}{$34.787$} & \multicolumn{1}{c|}{$11.415$} & \multicolumn{1}{c|}{$3.107$} & \multicolumn{1}{c|}{$97.671$} & \multicolumn{1}{c|}{$55.828$} & \multicolumn{1}{c|}{$17.324$} & \multicolumn{1}{c|}{$3.919$} & \multicolumn{1}{c|}{$73.956$} & \multicolumn{1}{c|}{$43.544$} & \multicolumn{1}{c|}{$13.353$} & \multicolumn{1}{c|}{$3.543$} \\ \cline{3-15}
\multicolumn{1}{|c|}{} & \multicolumn{1}{c|}{} & Cover (ASE) & \multicolumn{1}{c|}{$0.866$} & \multicolumn{1}{c|}{$0.870$} & \multicolumn{1}{c|}{$0.926$} & \multicolumn{1}{c|}{$0.936$} & \multicolumn{1}{c|}{$0.936$} & \multicolumn{1}{c|}{$0.908$} & \multicolumn{1}{c|}{$0.958$} & \multicolumn{1}{c|}{$0.950$} & \multicolumn{1}{c|}{$0.898$} & \multicolumn{1}{c|}{$0.860$} & \multicolumn{1}{c|}{$0.936$} & \multicolumn{1}{c|}{$0.942$} \\ \cline{3-15}
\multicolumn{1}{|c|}{} & \multicolumn{1}{c|}{} & Cover (BSE) & \multicolumn{1}{c|}{$0.952$} & \multicolumn{1}{c|}{$0.962$} & \multicolumn{1}{c|}{$0.996$} & \multicolumn{1}{c|}{$1.000$} & \multicolumn{1}{c|}{$0.938$} & \multicolumn{1}{c|}{$0.946$} & \multicolumn{1}{c|}{$0.980$} & \multicolumn{1}{c|}{$1.000$} & \multicolumn{1}{c|}{$0.932$} & \multicolumn{1}{c|}{$0.948$} & \multicolumn{1}{c|}{$0.992$} & \multicolumn{1}{c|}{$0.998$} \\ \cline{2-15}
\multicolumn{1}{|c|}{} & \multicolumn{1}{c|}{\multirow{7}{*}{9}} & Bias ($\times$10) & \multicolumn{1}{c|}{$-1.123$} & \multicolumn{1}{c|}{$-0.472$} & \multicolumn{1}{c|}{$-1.026$} & \multicolumn{1}{c|}{$-0.910$} & \multicolumn{1}{c|}{$-0.923$} & \multicolumn{1}{c|}{$0.228$} & \multicolumn{1}{c|}{$-0.402$} & \multicolumn{1}{c|}{$-0.317$} & \multicolumn{1}{c|}{$-1.228$} & \multicolumn{1}{c|}{$-0.128$} & \multicolumn{1}{c|}{$-0.527$} & \multicolumn{1}{c|}{$-0.366$} \\ \cline{3-15}
\multicolumn{1}{|c|}{} & \multicolumn{1}{c|}{} & ASE ($\times$10) & \multicolumn{1}{c|}{$6.414$} & \multicolumn{1}{c|}{$4.778$} & \multicolumn{1}{c|}{$3.178$} & \multicolumn{1}{c|}{$1.678$} & \multicolumn{1}{c|}{$9.985$} & \multicolumn{1}{c|}{$7.101$} & \multicolumn{1}{c|}{$4.573$} & \multicolumn{1}{c|}{$2.255$} & \multicolumn{1}{c|}{$7.856$} & \multicolumn{1}{c|}{$5.851$} & \multicolumn{1}{c|}{$3.830$} & \multicolumn{1}{c|}{$1.982$} \\ \cline{3-15}
\multicolumn{1}{|c|}{} & \multicolumn{1}{c|}{} & BSE ($\times$10) & \multicolumn{1}{c|}{$9.928$} & \multicolumn{1}{c|}{$8.085$} & \multicolumn{1}{c|}{$5.942$} & \multicolumn{1}{c|}{$3.873$} & \multicolumn{1}{c|}{$11.289$} & \multicolumn{1}{c|}{$8.762$} & \multicolumn{1}{c|}{$6.386$} & \multicolumn{1}{c|}{$4.066$} & \multicolumn{1}{c|}{$10.404$} & \multicolumn{1}{c|}{$8.309$} & \multicolumn{1}{c|}{$6.250$} & \multicolumn{1}{c|}{$3.953$} \\ \cline{3-15}
\multicolumn{1}{|c|}{} & \multicolumn{1}{c|}{} & ESE ($\times$10) & \multicolumn{1}{c|}{$7.827$} & \multicolumn{1}{c|}{$5.755$} & \multicolumn{1}{c|}{$3.454$} & \multicolumn{1}{c|}{$1.698$} & \multicolumn{1}{c|}{$9.365$} & \multicolumn{1}{c|}{$6.819$} & \multicolumn{1}{c|}{$4.047$} & \multicolumn{1}{c|}{$1.965$} & \multicolumn{1}{c|}{$8.545$} & \multicolumn{1}{c|}{$6.305$} & \multicolumn{1}{c|}{$3.828$} & \multicolumn{1}{c|}{$1.831$} \\ \cline{3-15}
\multicolumn{1}{|c|}{} & \multicolumn{1}{c|}{} & MSE ($\times$100) & \multicolumn{1}{c|}{$62.394$} & \multicolumn{1}{c|}{$33.277$} & \multicolumn{1}{c|}{$12.960$} & \multicolumn{1}{c|}{$3.707$} & \multicolumn{1}{c|}{$88.380$} & \multicolumn{1}{c|}{$46.453$} & \multicolumn{1}{c|}{$16.510$} & \multicolumn{1}{c|}{$3.954$} & \multicolumn{1}{c|}{$74.376$} & \multicolumn{1}{c|}{$39.688$} & \multicolumn{1}{c|}{$14.904$} & \multicolumn{1}{c|}{$3.480$} \\ \cline{3-15}
\multicolumn{1}{|c|}{} & \multicolumn{1}{c|}{} & Cover (ASE) & \multicolumn{1}{c|}{$0.872$} & \multicolumn{1}{c|}{$0.884$} & \multicolumn{1}{c|}{$0.918$} & \multicolumn{1}{c|}{$0.906$} & \multicolumn{1}{c|}{$0.952$} & \multicolumn{1}{c|}{$0.940$} & \multicolumn{1}{c|}{$0.970$} & \multicolumn{1}{c|}{$0.980$} & \multicolumn{1}{c|}{$0.920$} & \multicolumn{1}{c|}{$0.912$} & \multicolumn{1}{c|}{$0.942$} & \multicolumn{1}{c|}{$0.952$} \\ \cline{3-15}
\multicolumn{1}{|c|}{} & \multicolumn{1}{c|}{} & Cover (BSE) & \multicolumn{1}{c|}{$0.948$} & \multicolumn{1}{c|}{$0.960$} & \multicolumn{1}{c|}{$0.984$} & \multicolumn{1}{c|}{$1.000$} & \multicolumn{1}{c|}{$0.952$} & \multicolumn{1}{c|}{$0.958$} & \multicolumn{1}{c|}{$0.980$} & \multicolumn{1}{c|}{$1.000$} & \multicolumn{1}{c|}{$0.948$} & \multicolumn{1}{c|}{$0.964$} & \multicolumn{1}{c|}{$0.976$} & \multicolumn{1}{c|}{$1.000$} \\ \hline \hline
\multicolumn{1}{|c|}{\multirow{21}{*}{1}} & \multicolumn{1}{c|}{\multirow{7}{*}{2}} & Bias ($\times$10) & \multicolumn{1}{c|}{$0.888$} & \multicolumn{1}{c|}{$0.807$} & \multicolumn{1}{c|}{$1.005$} & \multicolumn{1}{c|}{$0.979$} & \multicolumn{1}{c|}{$-0.344$} & \multicolumn{1}{c|}{$-0.143$} & \multicolumn{1}{c|}{$0.038$} & \multicolumn{1}{c|}{$-0.000$} & \multicolumn{1}{c|}{$-0.184$} & \multicolumn{1}{c|}{$-0.073$} & \multicolumn{1}{c|}{$0.082$} & \multicolumn{1}{c|}{$0.018$} \\ \cline{3-15}
\multicolumn{1}{|c|}{} & \multicolumn{1}{c|}{} & ASE ($\times$10) & \multicolumn{1}{c|}{$6.556$} & \multicolumn{1}{c|}{$4.877$} & \multicolumn{1}{c|}{$3.290$} & \multicolumn{1}{c|}{$1.726$} & \multicolumn{1}{c|}{$7.270$} & \multicolumn{1}{c|}{$5.341$} & \multicolumn{1}{c|}{$3.565$} & \multicolumn{1}{c|}{$1.867$} & \multicolumn{1}{c|}{$7.204$} & \multicolumn{1}{c|}{$5.322$} & \multicolumn{1}{c|}{$3.559$} & \multicolumn{1}{c|}{$1.866$} \\ \cline{3-15}
\multicolumn{1}{|c|}{} & \multicolumn{1}{c|}{} & BSE ($\times$10) & \multicolumn{1}{c|}{$10.532$} & \multicolumn{1}{c|}{$8.277$} & \multicolumn{1}{c|}{$6.463$} & \multicolumn{1}{c|}{$4.124$} & \multicolumn{1}{c|}{$10.970$} & \multicolumn{1}{c|}{$8.681$} & \multicolumn{1}{c|}{$6.591$} & \multicolumn{1}{c|}{$4.087$} & \multicolumn{1}{c|}{$10.960$} & \multicolumn{1}{c|}{$8.701$} & \multicolumn{1}{c|}{$6.621$} & \multicolumn{1}{c|}{$4.090$} \\ \cline{3-15}
\multicolumn{1}{|c|}{} & \multicolumn{1}{c|}{} & ESE ($\times$10) & \multicolumn{1}{c|}{$7.847$} & \multicolumn{1}{c|}{$5.732$} & \multicolumn{1}{c|}{$3.657$} & \multicolumn{1}{c|}{$1.846$} & \multicolumn{1}{c|}{$8.649$} & \multicolumn{1}{c|}{$6.106$} & \multicolumn{1}{c|}{$3.989$} & \multicolumn{1}{c|}{$2.014$} & \multicolumn{1}{c|}{$8.615$} & \multicolumn{1}{c|}{$6.087$} & \multicolumn{1}{c|}{$3.984$} & \multicolumn{1}{c|}{$2.014$} \\ \cline{3-15}
\multicolumn{1}{|c|}{} & \multicolumn{1}{c|}{} & MSE ($\times$100) & \multicolumn{1}{c|}{$62.241$} & \multicolumn{1}{c|}{$33.444$} & \multicolumn{1}{c|}{$14.359$} & \multicolumn{1}{c|}{$4.359$} & \multicolumn{1}{c|}{$74.777$} & \multicolumn{1}{c|}{$37.232$} & \multicolumn{1}{c|}{$15.883$} & \multicolumn{1}{c|}{$4.047$} & \multicolumn{1}{c|}{$74.097$} & \multicolumn{1}{c|}{$36.983$} & \multicolumn{1}{c|}{$15.844$} & \multicolumn{1}{c|}{$4.047$} \\ \cline{3-15}
\multicolumn{1}{|c|}{} & \multicolumn{1}{c|}{} & Cover (ASE) & \multicolumn{1}{c|}{$0.876$} & \multicolumn{1}{c|}{$0.878$} & \multicolumn{1}{c|}{$0.900$} & \multicolumn{1}{c|}{$0.888$} & \multicolumn{1}{c|}{$0.890$} & \multicolumn{1}{c|}{$0.890$} & \multicolumn{1}{c|}{$0.928$} & \multicolumn{1}{c|}{$0.926$} & \multicolumn{1}{c|}{$0.880$} & \multicolumn{1}{c|}{$0.890$} & \multicolumn{1}{c|}{$0.928$} & \multicolumn{1}{c|}{$0.926$} \\ \cline{3-15}
\multicolumn{1}{|c|}{} & \multicolumn{1}{c|}{} & Cover (BSE) & \multicolumn{1}{c|}{$0.960$} & \multicolumn{1}{c|}{$0.972$} & \multicolumn{1}{c|}{$0.994$} & \multicolumn{1}{c|}{$1.000$} & \multicolumn{1}{c|}{$0.962$} & \multicolumn{1}{c|}{$0.958$} & \multicolumn{1}{c|}{$0.988$} & \multicolumn{1}{c|}{$1.000$} & \multicolumn{1}{c|}{$0.958$} & \multicolumn{1}{c|}{$0.962$} & \multicolumn{1}{c|}{$0.992$} & \multicolumn{1}{c|}{$1.000$} \\ \cline{2-15}
\multicolumn{1}{|c|}{} & \multicolumn{1}{c|}{\multirow{7}{*}{5}} & Bias ($\times$10) & \multicolumn{1}{c|}{$0.233$} & \multicolumn{1}{c|}{$0.237$} & \multicolumn{1}{c|}{$0.373$} & \multicolumn{1}{c|}{$0.289$} & \multicolumn{1}{c|}{$-0.040$} & \multicolumn{1}{c|}{$-0.081$} & \multicolumn{1}{c|}{$0.251$} & \multicolumn{1}{c|}{$0.113$} & \multicolumn{1}{c|}{$-0.318$} & \multicolumn{1}{c|}{$-0.302$} & \multicolumn{1}{c|}{$0.058$} & \multicolumn{1}{c|}{$0.065$} \\ \cline{3-15}
\multicolumn{1}{|c|}{} & \multicolumn{1}{c|}{} & ASE ($\times$10) & \multicolumn{1}{c|}{$6.646$} & \multicolumn{1}{c|}{$4.836$} & \multicolumn{1}{c|}{$3.206$} & \multicolumn{1}{c|}{$1.663$} & \multicolumn{1}{c|}{$8.961$} & \multicolumn{1}{c|}{$5.860$} & \multicolumn{1}{c|}{$3.585$} & \multicolumn{1}{c|}{$1.772$} & \multicolumn{1}{c|}{$7.362$} & \multicolumn{1}{c|}{$5.229$} & \multicolumn{1}{c|}{$3.418$} & \multicolumn{1}{c|}{$1.742$} \\ \cline{3-15}
\multicolumn{1}{|c|}{} & \multicolumn{1}{c|}{} & BSE ($\times$10) & \multicolumn{1}{c|}{$10.576$} & \multicolumn{1}{c|}{$8.273$} & \multicolumn{1}{c|}{$6.267$} & \multicolumn{1}{c|}{$3.991$} & \multicolumn{1}{c|}{$11.565$} & \multicolumn{1}{c|}{$8.625$} & \multicolumn{1}{c|}{$6.510$} & \multicolumn{1}{c|}{$4.055$} & \multicolumn{1}{c|}{$11.463$} & \multicolumn{1}{c|}{$8.620$} & \multicolumn{1}{c|}{$6.365$} & \multicolumn{1}{c|}{$4.022$} \\ \cline{3-15}
\multicolumn{1}{|c|}{} & \multicolumn{1}{c|}{} & ESE ($\times$10) & \multicolumn{1}{c|}{$8.392$} & \multicolumn{1}{c|}{$5.998$} & \multicolumn{1}{c|}{$3.359$} & \multicolumn{1}{c|}{$1.649$} & \multicolumn{1}{c|}{$9.034$} & \multicolumn{1}{c|}{$6.635$} & \multicolumn{1}{c|}{$3.719$} & \multicolumn{1}{c|}{$1.778$} & \multicolumn{1}{c|}{$8.532$} & \multicolumn{1}{c|}{$6.377$} & \multicolumn{1}{c|}{$3.656$} & \multicolumn{1}{c|}{$1.739$} \\ \cline{3-15}
\multicolumn{1}{|c|}{} & \multicolumn{1}{c|}{} & MSE ($\times$100) & \multicolumn{1}{c|}{$70.342$} & \multicolumn{1}{c|}{$35.961$} & \multicolumn{1}{c|}{$11.400$} & \multicolumn{1}{c|}{$2.798$} & \multicolumn{1}{c|}{$81.459$} & \multicolumn{1}{c|}{$43.944$} & \multicolumn{1}{c|}{$13.870$} & \multicolumn{1}{c|}{$3.169$} & \multicolumn{1}{c|}{$72.748$} & \multicolumn{1}{c|}{$40.676$} & \multicolumn{1}{c|}{$13.346$} & \multicolumn{1}{c|}{$3.022$} \\ \cline{3-15}
\multicolumn{1}{|c|}{} & \multicolumn{1}{c|}{} & Cover (ASE) & \multicolumn{1}{c|}{$0.852$} & \multicolumn{1}{c|}{$0.896$} & \multicolumn{1}{c|}{$0.922$} & \multicolumn{1}{c|}{$0.954$} & \multicolumn{1}{c|}{$0.932$} & \multicolumn{1}{c|}{$0.898$} & \multicolumn{1}{c|}{$0.934$} & \multicolumn{1}{c|}{$0.948$} & \multicolumn{1}{c|}{$0.908$} & \multicolumn{1}{c|}{$0.882$} & \multicolumn{1}{c|}{$0.920$} & \multicolumn{1}{c|}{$0.948$} \\ \cline{3-15}
\multicolumn{1}{|c|}{} & \multicolumn{1}{c|}{} & Cover (BSE) & \multicolumn{1}{c|}{$0.960$} & \multicolumn{1}{c|}{$0.960$} & \multicolumn{1}{c|}{$0.992$} & \multicolumn{1}{c|}{$0.998$} & \multicolumn{1}{c|}{$0.962$} & \multicolumn{1}{c|}{$0.944$} & \multicolumn{1}{c|}{$0.998$} & \multicolumn{1}{c|}{$1.000$} & \multicolumn{1}{c|}{$0.968$} & \multicolumn{1}{c|}{$0.952$} & \multicolumn{1}{c|}{$0.992$} & \multicolumn{1}{c|}{$0.998$} \\ \cline{2-15}
\multicolumn{1}{|c|}{} & \multicolumn{1}{c|}{\multirow{7}{*}{9}} & Bias ($\times$10) & \multicolumn{1}{c|}{$-0.791$} & \multicolumn{1}{c|}{$-0.996$} & \multicolumn{1}{c|}{$-0.848$} & \multicolumn{1}{c|}{$-0.698$} & \multicolumn{1}{c|}{$-1.083$} & \multicolumn{1}{c|}{$-0.953$} & \multicolumn{1}{c|}{$-0.647$} & \multicolumn{1}{c|}{$-0.126$} & \multicolumn{1}{c|}{$-1.113$} & \multicolumn{1}{c|}{$-1.088$} & \multicolumn{1}{c|}{$-0.862$} & \multicolumn{1}{c|}{$-0.462$} \\ \cline{3-15}
\multicolumn{1}{|c|}{} & \multicolumn{1}{c|}{} & ASE ($\times$10) & \multicolumn{1}{c|}{$7.110$} & \multicolumn{1}{c|}{$5.063$} & \multicolumn{1}{c|}{$3.318$} & \multicolumn{1}{c|}{$1.734$} & \multicolumn{1}{c|}{$10.109$} & \multicolumn{1}{c|}{$6.788$} & \multicolumn{1}{c|}{$4.091$} & \multicolumn{1}{c|}{$1.955$} & \multicolumn{1}{c|}{$8.474$} & \multicolumn{1}{c|}{$5.868$} & \multicolumn{1}{c|}{$3.621$} & \multicolumn{1}{c|}{$1.797$} \\ \cline{3-15}
\multicolumn{1}{|c|}{} & \multicolumn{1}{c|}{} & BSE ($\times$10) & \multicolumn{1}{c|}{$10.357$} & \multicolumn{1}{c|}{$8.326$} & \multicolumn{1}{c|}{$6.151$} & \multicolumn{1}{c|}{$3.990$} & \multicolumn{1}{c|}{$12.310$} & \multicolumn{1}{c|}{$9.066$} & \multicolumn{1}{c|}{$6.505$} & \multicolumn{1}{c|}{$4.091$} & \multicolumn{1}{c|}{$12.003$} & \multicolumn{1}{c|}{$9.126$} & \multicolumn{1}{c|}{$6.420$} & \multicolumn{1}{c|}{$4.101$} \\ \cline{3-15}
\multicolumn{1}{|c|}{} & \multicolumn{1}{c|}{} & ESE ($\times$10) & \multicolumn{1}{c|}{$8.532$} & \multicolumn{1}{c|}{$5.691$} & \multicolumn{1}{c|}{$3.610$} & \multicolumn{1}{c|}{$1.830$} & \multicolumn{1}{c|}{$9.964$} & \multicolumn{1}{c|}{$6.548$} & \multicolumn{1}{c|}{$3.895$} & \multicolumn{1}{c|}{$1.867$} & \multicolumn{1}{c|}{$9.185$} & \multicolumn{1}{c|}{$6.135$} & \multicolumn{1}{c|}{$3.781$} & \multicolumn{1}{c|}{$1.848$} \\ \cline{3-15}
\multicolumn{1}{|c|}{} & \multicolumn{1}{c|}{} & MSE ($\times$100) & \multicolumn{1}{c|}{$73.267$} & \multicolumn{1}{c|}{$33.320$} & \multicolumn{1}{c|}{$13.729$} & \multicolumn{1}{c|}{$3.830$} & \multicolumn{1}{c|}{$100.256$} & \multicolumn{1}{c|}{$43.702$} & \multicolumn{1}{c|}{$15.558$} & \multicolumn{1}{c|}{$3.496$} & \multicolumn{1}{c|}{$85.438$} & \multicolumn{1}{c|}{$38.742$} & \multicolumn{1}{c|}{$15.014$} & \multicolumn{1}{c|}{$3.623$} \\ \cline{3-15}
\multicolumn{1}{|c|}{} & \multicolumn{1}{c|}{} & Cover (ASE) & \multicolumn{1}{c|}{$0.894$} & \multicolumn{1}{c|}{$0.908$} & \multicolumn{1}{c|}{$0.908$} & \multicolumn{1}{c|}{$0.910$} & \multicolumn{1}{c|}{$0.948$} & \multicolumn{1}{c|}{$0.960$} & \multicolumn{1}{c|}{$0.956$} & \multicolumn{1}{c|}{$0.952$} & \multicolumn{1}{c|}{$0.926$} & \multicolumn{1}{c|}{$0.930$} & \multicolumn{1}{c|}{$0.924$} & \multicolumn{1}{c|}{$0.940$} \\ \cline{3-15}
\multicolumn{1}{|c|}{} & \multicolumn{1}{c|}{} & Cover (BSE) & \multicolumn{1}{c|}{$0.942$} & \multicolumn{1}{c|}{$0.980$} & \multicolumn{1}{c|}{$0.990$} & \multicolumn{1}{c|}{$1.000$} & \multicolumn{1}{c|}{$0.960$} & \multicolumn{1}{c|}{$0.986$} & \multicolumn{1}{c|}{$0.984$} & \multicolumn{1}{c|}{$0.998$} & \multicolumn{1}{c|}{$0.956$} & \multicolumn{1}{c|}{$0.984$} & \multicolumn{1}{c|}{$0.990$} & \multicolumn{1}{c|}{$0.998$} \\ \hline

\end{tabular}
\caption{Summary Statistics of the Estimation Results for $\beta_{1}^*=3$.} 
\label{tab:supp:Table1}
\end{table}

 

We report the performance of the conformal inference in Section \ref{sec:Conformal} of the main paper under the simulation scenarios in Section \ref{sec:Sim} of the main paper. First, we obtain the pointwise 95\% pointwise prediction interval for the random treatment effect at the first post-treatment time $t=T_0+1$, which are
\begin{align*}
&
\text{Constant ATT} :
&&
\xi_{T_0+1}^* = 3 + \epsilon_{\tau,T_0+1} \ 
, 
&&
\text{Linear ATT} :
&&
\xi_{T_0+1}^* = 3 + 3/T_0 + \epsilon_{\tau,T_0+1}
\ .
\end{align*}
As a competing method, we construct 95\% pointwise prediction intervals using the approach proposed by \citet{Cattaneo2021}, which is implemented in \texttt{scpi} R-package \citep{scpiPackage2023}. In particular, we use the prediction interval estimating out-of-sample uncertainty with sub-Gaussian bounds, which is stored in \texttt{CI.all.gaussian} object of a \texttt{scpi} output; see below for an example R-code:
\begin{itemize}[leftmargin=2cm,itemsep=0cm]
\item[] \texttt{scpi.est $\leftarrow$ scpi::scpi(SCD) \# SCD is a scdata object}
\item[] \texttt{scpi.PI \ $\leftarrow$ scpi.est\$inference.results\$CI.all.gaussian}
\end{itemize}
For each simulation repetition and each method, we calculate $\ind \big(\xi_{T_0+1} ^* \in \mathcal{C}_{T_0+1} \big)$ where $\mathcal{C}_{T_0+1}$ is a 95\% prediction interval obtained from each method, i.e., the indicator of whether a 95\% prediction interval at $t=T_0+1$ obtained from each method includes the random treatment effect. Ideally, the average of these indicators across simulation repetitions (i.e., the empirical coverage rate of 95\% prediction intervals) should be close to the nominal coverage rate of 0.95.

Table \ref{tab:supp:Table3} shows the empirical coverage rates obtained from 500 repetitions for each simulation scenario. We find that the conformal inference approach for the SPSC achieves the nominal coverage rate across all simulation scenarios in general. However, we find that \texttt{scpi} approach fails to achieve the nominal coverage rate, especially when the number of donors is small (i.e., $d=2$), the time periods are long (i.e., $T_0=1000$), and the average treatment effect varies across time (i.e., linear treatment effects).



\begin{table}[!htp]
\renewcommand{\arraystretch}{1.05} \centering
\footnotesize
\setlength{\tabcolsep}{5pt}  
\begin{tabular}{|ccc|cccc|cccc|cccc|}
\hline
\multicolumn{3}{|c|}{Estimator} & \multicolumn{4}{c|}{SPSC} & \multicolumn{4}{c|}{SPSC-Ridge} & \multicolumn{4}{c|}{SCPI} \\ \hline
\multicolumn{1}{|c|}{\multirow{2}{*}{ATT}} & \multicolumn{1}{c|}{\multirow{2}{*}{$\ \delta \ $}} & \multirow{2}{*}{$ \ d \ $} & \multicolumn{4}{c|}{$T_0$} & \multicolumn{4}{c|}{$T_0$} & \multicolumn{4}{c|}{$T_0$} \\ \cline{4-15} 
\multicolumn{1}{|c|}{} & \multicolumn{1}{c|}{} & & \multicolumn{1}{c|}{50} & \multicolumn{1}{c|}{100} & \multicolumn{1}{c|}{250} & 1000 & \multicolumn{1}{c|}{50} & \multicolumn{1}{c|}{100} & \multicolumn{1}{c|}{250} & 1000 & \multicolumn{1}{c|}{50} & \multicolumn{1}{c|}{100} & \multicolumn{1}{c|}{250} & 1000 \\ \hline

\multicolumn{1}{|c|}{\multirow{6}{*}{Constant}} & \multicolumn{1}{c|}{\multirow{3}{*}{0}} & \multicolumn{1}{c|}{2} & \multicolumn{1}{c|}{$0.964$} & \multicolumn{1}{c|}{$0.946$} & \multicolumn{1}{c|}{$0.940$} & \multicolumn{1}{c|}{$0.948$} & \multicolumn{1}{c|}{$0.968$} & \multicolumn{1}{c|}{$0.950$} & \multicolumn{1}{c|}{$0.940$} & \multicolumn{1}{c|}{$0.948$} & \multicolumn{1}{c|}{$0.948$} & \multicolumn{1}{c|}{$0.924$} & \multicolumn{1}{c|}{$0.920$} & \multicolumn{1}{c|}{$0.904$} \\ \cline{3-15}
\multicolumn{1}{|c|}{} & \multicolumn{1}{c|}{} & \multicolumn{1}{c|}{5} & \multicolumn{1}{c|}{$0.962$} & \multicolumn{1}{c|}{$0.956$} & \multicolumn{1}{c|}{$0.956$} & \multicolumn{1}{c|}{$0.942$} & \multicolumn{1}{c|}{$0.956$} & \multicolumn{1}{c|}{$0.946$} & \multicolumn{1}{c|}{$0.960$} & \multicolumn{1}{c|}{$0.932$} & \multicolumn{1}{c|}{$0.980$} & \multicolumn{1}{c|}{$0.982$} & \multicolumn{1}{c|}{$0.978$} & \multicolumn{1}{c|}{$0.920$} \\ \cline{3-15}
\multicolumn{1}{|c|}{} & \multicolumn{1}{c|}{} & \multicolumn{1}{c|}{9} & \multicolumn{1}{c|}{$0.954$} & \multicolumn{1}{c|}{$0.948$} & \multicolumn{1}{c|}{$0.966$} & \multicolumn{1}{c|}{$0.944$} & \multicolumn{1}{c|}{$0.942$} & \multicolumn{1}{c|}{$0.942$} & \multicolumn{1}{c|}{$0.964$} & \multicolumn{1}{c|}{$0.952$} & \multicolumn{1}{c|}{$0.984$} & \multicolumn{1}{c|}{$0.990$} & \multicolumn{1}{c|}{$0.978$} & \multicolumn{1}{c|}{$0.954$} \\ \cline{2-15}
\multicolumn{1}{|c|}{} & \multicolumn{1}{c|}{\multirow{3}{*}{1}} & \multicolumn{1}{c|}{2} & \multicolumn{1}{c|}{$0.972$} & \multicolumn{1}{c|}{$0.940$} & \multicolumn{1}{c|}{$0.950$} & \multicolumn{1}{c|}{$0.954$} & \multicolumn{1}{c|}{$0.968$} & \multicolumn{1}{c|}{$0.942$} & \multicolumn{1}{c|}{$0.944$} & \multicolumn{1}{c|}{$0.962$} & \multicolumn{1}{c|}{$0.968$} & \multicolumn{1}{c|}{$0.944$} & \multicolumn{1}{c|}{$0.938$} & \multicolumn{1}{c|}{$0.894$} \\ \cline{3-15}
\multicolumn{1}{|c|}{} & \multicolumn{1}{c|}{} & \multicolumn{1}{c|}{5} & \multicolumn{1}{c|}{$0.956$} & \multicolumn{1}{c|}{$0.948$} & \multicolumn{1}{c|}{$0.956$} & \multicolumn{1}{c|}{$0.958$} & \multicolumn{1}{c|}{$0.954$} & \multicolumn{1}{c|}{$0.934$} & \multicolumn{1}{c|}{$0.960$} & \multicolumn{1}{c|}{$0.958$} & \multicolumn{1}{c|}{$0.980$} & \multicolumn{1}{c|}{$0.984$} & \multicolumn{1}{c|}{$0.966$} & \multicolumn{1}{c|}{$0.922$} \\ \cline{3-15}
\multicolumn{1}{|c|}{} & \multicolumn{1}{c|}{} & \multicolumn{1}{c|}{9} & \multicolumn{1}{c|}{$0.956$} & \multicolumn{1}{c|}{$0.950$} & \multicolumn{1}{c|}{$0.958$} & \multicolumn{1}{c|}{$0.938$} & \multicolumn{1}{c|}{$0.962$} & \multicolumn{1}{c|}{$0.946$} & \multicolumn{1}{c|}{$0.950$} & \multicolumn{1}{c|}{$0.946$} & \multicolumn{1}{c|}{$0.980$} & \multicolumn{1}{c|}{$0.992$} & \multicolumn{1}{c|}{$0.990$} & \multicolumn{1}{c|}{$0.944$} \\ \hline \hline
\multicolumn{1}{|c|}{\multirow{6}{*}{Linear}} & \multicolumn{1}{c|}{\multirow{3}{*}{0}} & \multicolumn{1}{c|}{2} & \multicolumn{1}{c|}{$0.964$} & \multicolumn{1}{c|}{$0.964$} & \multicolumn{1}{c|}{$0.948$} & \multicolumn{1}{c|}{$0.932$} & \multicolumn{1}{c|}{$0.968$} & \multicolumn{1}{c|}{$0.964$} & \multicolumn{1}{c|}{$0.946$} & \multicolumn{1}{c|}{$0.932$} & \multicolumn{1}{c|}{$0.866$} & \multicolumn{1}{c|}{$0.850$} & \multicolumn{1}{c|}{$0.866$} & \multicolumn{1}{c|}{$0.780$} \\ \cline{3-15}
\multicolumn{1}{|c|}{} & \multicolumn{1}{c|}{} & \multicolumn{1}{c|}{5} & \multicolumn{1}{c|}{$0.962$} & \multicolumn{1}{c|}{$0.940$} & \multicolumn{1}{c|}{$0.958$} & \multicolumn{1}{c|}{$0.954$} & \multicolumn{1}{c|}{$0.956$} & \multicolumn{1}{c|}{$0.954$} & \multicolumn{1}{c|}{$0.946$} & \multicolumn{1}{c|}{$0.954$} & \multicolumn{1}{c|}{$0.866$} & \multicolumn{1}{c|}{$0.870$} & \multicolumn{1}{c|}{$0.926$} & \multicolumn{1}{c|}{$0.936$} \\ \cline{3-15}
\multicolumn{1}{|c|}{} & \multicolumn{1}{c|}{} & \multicolumn{1}{c|}{9} & \multicolumn{1}{c|}{$0.954$} & \multicolumn{1}{c|}{$0.938$} & \multicolumn{1}{c|}{$0.964$} & \multicolumn{1}{c|}{$0.946$} & \multicolumn{1}{c|}{$0.942$} & \multicolumn{1}{c|}{$0.938$} & \multicolumn{1}{c|}{$0.954$} & \multicolumn{1}{c|}{$0.948$} & \multicolumn{1}{c|}{$0.872$} & \multicolumn{1}{c|}{$0.884$} & \multicolumn{1}{c|}{$0.918$} & \multicolumn{1}{c|}{$0.906$} \\ \cline{2-15}
\multicolumn{1}{|c|}{} & \multicolumn{1}{c|}{\multirow{3}{*}{1}} & \multicolumn{1}{c|}{2} & \multicolumn{1}{c|}{$0.972$} & \multicolumn{1}{c|}{$0.936$} & \multicolumn{1}{c|}{$0.954$} & \multicolumn{1}{c|}{$0.948$} & \multicolumn{1}{c|}{$0.968$} & \multicolumn{1}{c|}{$0.946$} & \multicolumn{1}{c|}{$0.954$} & \multicolumn{1}{c|}{$0.952$} & \multicolumn{1}{c|}{$0.876$} & \multicolumn{1}{c|}{$0.878$} & \multicolumn{1}{c|}{$0.900$} & \multicolumn{1}{c|}{$0.888$} \\ \cline{3-15}
\multicolumn{1}{|c|}{} & \multicolumn{1}{c|}{} & \multicolumn{1}{c|}{5} & \multicolumn{1}{c|}{$0.956$} & \multicolumn{1}{c|}{$0.948$} & \multicolumn{1}{c|}{$0.944$} & \multicolumn{1}{c|}{$0.936$} & \multicolumn{1}{c|}{$0.954$} & \multicolumn{1}{c|}{$0.934$} & \multicolumn{1}{c|}{$0.950$} & \multicolumn{1}{c|}{$0.930$} & \multicolumn{1}{c|}{$0.852$} & \multicolumn{1}{c|}{$0.896$} & \multicolumn{1}{c|}{$0.922$} & \multicolumn{1}{c|}{$0.954$} \\ \cline{3-15}
\multicolumn{1}{|c|}{} & \multicolumn{1}{c|}{} & \multicolumn{1}{c|}{9} & \multicolumn{1}{c|}{$0.956$} & \multicolumn{1}{c|}{$0.940$} & \multicolumn{1}{c|}{$0.954$} & \multicolumn{1}{c|}{$0.942$} & \multicolumn{1}{c|}{$0.962$} & \multicolumn{1}{c|}{$0.938$} & \multicolumn{1}{c|}{$0.950$} & \multicolumn{1}{c|}{$0.950$} & \multicolumn{1}{c|}{$0.894$} & \multicolumn{1}{c|}{$0.908$} & \multicolumn{1}{c|}{$0.908$} & \multicolumn{1}{c|}{$0.910$} \\ \hline

\end{tabular}
\caption{
Empirical Coverage Rates of 95\% Pointwise Prediction Intervals. The numbers in SPSC and SPSC-Ridge columns show the empirical coverage rates of 95\% pointwise prediction intervals obtained from the conformal inference approach in Section \ref{sec:Conformal}. The numbers in SCPI column show the empirical coverage rates of 95\% pointwise prediction intervals obtained from the approach proposed by \citet{Cattaneo2021} which is implemented in \texttt{scpi} R-package \citep{scpiPackage2023}.}
\label{tab:supp:Table3}
\vspace{-0.5cm}
\end{table}

Next, we calculate the average length of the 95\% prediction interval for $t=T_0+1$, i.e., the first post-treatment period, across 500 repetitions. 
Table \ref{tab:supp:Table4} shows the average lengths.
We find that the length of the prediction intervals decreases as the length of the pre-treatment periods increases. 
Additionally, we find that including ridge regularization in the estimation leads to shorter prediction intervals across all simulation scenarios. 
We remark that prediction intervals obtained from \texttt{scpi} are generally narrower than those obtained from the SPSC. 


\begin{table}[!htp]
\renewcommand{\arraystretch}{1.05} \centering
\footnotesize
\setlength{\tabcolsep}{5pt}  
\begin{tabular}{|ccc|cccc|cccc|cccc|}
\hline
\multicolumn{3}{|c|}{Estimator} & \multicolumn{4}{c|}{SPSC} & \multicolumn{4}{c|}{SPSC-Ridge} & \multicolumn{4}{c|}{SCPI} \\ \hline
\multicolumn{1}{|c|}{\multirow{2}{*}{ATT}} & \multicolumn{1}{c|}{\multirow{2}{*}{$\ \delta \ $}} & \multirow{2}{*}{$ \ d \ $} & \multicolumn{4}{c|}{$T_0$} & \multicolumn{4}{c|}{$T_0$} & \multicolumn{4}{c|}{$T_0$} \\ \cline{4-15} 
\multicolumn{1}{|c|}{} & \multicolumn{1}{c|}{} & & \multicolumn{1}{c|}{50} & \multicolumn{1}{c|}{100} & \multicolumn{1}{c|}{250} & 1000 & \multicolumn{1}{c|}{50} & \multicolumn{1}{c|}{100} & \multicolumn{1}{c|}{250} & 1000 & \multicolumn{1}{c|}{50} & \multicolumn{1}{c|}{100} & \multicolumn{1}{c|}{250} & 1000 \\ \hline
\multicolumn{1}{|c|}{\multirow{6}{*}{Constant}} & \multicolumn{1}{c|}{\multirow{3}{*}{0}} & \multicolumn{1}{c|}{2} & \multicolumn{1}{c|}{$3.377$} & \multicolumn{1}{c|}{$3.109$} & \multicolumn{1}{c|}{$3.094$} & \multicolumn{1}{c|}{$3.005$} & \multicolumn{1}{c|}{$3.296$} & \multicolumn{1}{c|}{$3.073$} & \multicolumn{1}{c|}{$3.076$} & \multicolumn{1}{c|}{$2.999$} & \multicolumn{1}{c|}{$1.919$} & \multicolumn{1}{c|}{$1.713$} & \multicolumn{1}{c|}{$1.534$} & \multicolumn{1}{c|}{$1.400$} \\ \cline{3-15}
\multicolumn{1}{|c|}{} & \multicolumn{1}{c|}{} & \multicolumn{1}{c|}{5} & \multicolumn{1}{c|}{$3.263$} & \multicolumn{1}{c|}{$2.834$} & \multicolumn{1}{c|}{$2.686$} & \multicolumn{1}{c|}{$2.466$} & \multicolumn{1}{c|}{$2.511$} & \multicolumn{1}{c|}{$2.281$} & \multicolumn{1}{c|}{$2.287$} & \multicolumn{1}{c|}{$2.263$} & \multicolumn{1}{c|}{$2.799$} & \multicolumn{1}{c|}{$2.415$} & \multicolumn{1}{c|}{$1.956$} & \multicolumn{1}{c|}{$1.583$} \\ \cline{3-15}
\multicolumn{1}{|c|}{} & \multicolumn{1}{c|}{} & \multicolumn{1}{c|}{9} & \multicolumn{1}{c|}{$3.189$} & \multicolumn{1}{c|}{$2.894$} & \multicolumn{1}{c|}{$2.809$} & \multicolumn{1}{c|}{$2.654$} & \multicolumn{1}{c|}{$2.617$} & \multicolumn{1}{c|}{$2.454$} & \multicolumn{1}{c|}{$2.349$} & \multicolumn{1}{c|}{$2.314$} & \multicolumn{1}{c|}{$2.750$} & \multicolumn{1}{c|}{$2.484$} & \multicolumn{1}{c|}{$2.105$} & \multicolumn{1}{c|}{$1.676$} \\ \cline{2-15}
\multicolumn{1}{|c|}{} & \multicolumn{1}{c|}{\multirow{3}{*}{1}} & \multicolumn{1}{c|}{2} & \multicolumn{1}{c|}{$3.459$} & \multicolumn{1}{c|}{$3.126$} & \multicolumn{1}{c|}{$3.091$} & \multicolumn{1}{c|}{$3.007$} & \multicolumn{1}{c|}{$3.387$} & \multicolumn{1}{c|}{$3.097$} & \multicolumn{1}{c|}{$3.078$} & \multicolumn{1}{c|}{$3.004$} & \multicolumn{1}{c|}{$2.189$} & \multicolumn{1}{c|}{$1.960$} & \multicolumn{1}{c|}{$1.729$} & \multicolumn{1}{c|}{$1.495$} \\ \cline{3-15}
\multicolumn{1}{|c|}{} & \multicolumn{1}{c|}{} & \multicolumn{1}{c|}{5} & \multicolumn{1}{c|}{$3.051$} & \multicolumn{1}{c|}{$2.570$} & \multicolumn{1}{c|}{$2.466$} & \multicolumn{1}{c|}{$2.426$} & \multicolumn{1}{c|}{$2.837$} & \multicolumn{1}{c|}{$2.385$} & \multicolumn{1}{c|}{$2.350$} & \multicolumn{1}{c|}{$2.334$} & \multicolumn{1}{c|}{$3.140$} & \multicolumn{1}{c|}{$2.780$} & \multicolumn{1}{c|}{$2.237$} & \multicolumn{1}{c|}{$1.748$} \\ \cline{3-15}
\multicolumn{1}{|c|}{} & \multicolumn{1}{c|}{} & \multicolumn{1}{c|}{9} & \multicolumn{1}{c|}{$3.184$} & \multicolumn{1}{c|}{$2.549$} & \multicolumn{1}{c|}{$2.446$} & \multicolumn{1}{c|}{$2.334$} & \multicolumn{1}{c|}{$2.948$} & \multicolumn{1}{c|}{$2.375$} & \multicolumn{1}{c|}{$2.272$} & \multicolumn{1}{c|}{$2.173$} & \multicolumn{1}{c|}{$3.098$} & \multicolumn{1}{c|}{$2.694$} & \multicolumn{1}{c|}{$2.246$} & \multicolumn{1}{c|}{$1.772$} \\ \hline \hline
\multicolumn{1}{|c|}{\multirow{6}{*}{Linear}} & \multicolumn{1}{c|}{\multirow{3}{*}{0}} & \multicolumn{1}{c|}{2} & \multicolumn{1}{c|}{$3.377$} & \multicolumn{1}{c|}{$3.119$} & \multicolumn{1}{c|}{$3.085$} & \multicolumn{1}{c|}{$3.006$} & \multicolumn{1}{c|}{$3.296$} & \multicolumn{1}{c|}{$3.081$} & \multicolumn{1}{c|}{$3.069$} & \multicolumn{1}{c|}{$3.000$} & \multicolumn{1}{c|}{$1.032$} & \multicolumn{1}{c|}{$0.861$} & \multicolumn{1}{c|}{$0.644$} & \multicolumn{1}{c|}{$0.420$} \\ \cline{3-15}
\multicolumn{1}{|c|}{} & \multicolumn{1}{c|}{} & \multicolumn{1}{c|}{5} & \multicolumn{1}{c|}{$3.263$} & \multicolumn{1}{c|}{$2.843$} & \multicolumn{1}{c|}{$2.631$} & \multicolumn{1}{c|}{$2.469$} & \multicolumn{1}{c|}{$2.511$} & \multicolumn{1}{c|}{$2.278$} & \multicolumn{1}{c|}{$2.250$} & \multicolumn{1}{c|}{$2.264$} & \multicolumn{1}{c|}{$0.997$} & \multicolumn{1}{c|}{$0.798$} & \multicolumn{1}{c|}{$0.617$} & \multicolumn{1}{c|}{$0.395$} \\ \cline{3-15}
\multicolumn{1}{|c|}{} & \multicolumn{1}{c|}{} & \multicolumn{1}{c|}{9} & \multicolumn{1}{c|}{$3.189$} & \multicolumn{1}{c|}{$2.879$} & \multicolumn{1}{c|}{$2.823$} & \multicolumn{1}{c|}{$2.619$} & \multicolumn{1}{c|}{$2.617$} & \multicolumn{1}{c|}{$2.421$} & \multicolumn{1}{c|}{$2.395$} & \multicolumn{1}{c|}{$2.324$} & \multicolumn{1}{c|}{$0.993$} & \multicolumn{1}{c|}{$0.809$} & \multicolumn{1}{c|}{$0.594$} & \multicolumn{1}{c|}{$0.387$} \\ \cline{2-15}
\multicolumn{1}{|c|}{} & \multicolumn{1}{c|}{\multirow{3}{*}{1}} & \multicolumn{1}{c|}{2} & \multicolumn{1}{c|}{$3.459$} & \multicolumn{1}{c|}{$3.122$} & \multicolumn{1}{c|}{$3.081$} & \multicolumn{1}{c|}{$3.012$} & \multicolumn{1}{c|}{$3.387$} & \multicolumn{1}{c|}{$3.095$} & \multicolumn{1}{c|}{$3.069$} & \multicolumn{1}{c|}{$3.008$} & \multicolumn{1}{c|}{$1.053$} & \multicolumn{1}{c|}{$0.828$} & \multicolumn{1}{c|}{$0.646$} & \multicolumn{1}{c|}{$0.412$} \\ \cline{3-15}
\multicolumn{1}{|c|}{} & \multicolumn{1}{c|}{} & \multicolumn{1}{c|}{5} & \multicolumn{1}{c|}{$3.051$} & \multicolumn{1}{c|}{$2.538$} & \multicolumn{1}{c|}{$2.478$} & \multicolumn{1}{c|}{$2.427$} & \multicolumn{1}{c|}{$2.837$} & \multicolumn{1}{c|}{$2.373$} & \multicolumn{1}{c|}{$2.356$} & \multicolumn{1}{c|}{$2.339$} & \multicolumn{1}{c|}{$1.058$} & \multicolumn{1}{c|}{$0.827$} & \multicolumn{1}{c|}{$0.627$} & \multicolumn{1}{c|}{$0.399$} \\ \cline{3-15}
\multicolumn{1}{|c|}{} & \multicolumn{1}{c|}{} & \multicolumn{1}{c|}{9} & \multicolumn{1}{c|}{$3.184$} & \multicolumn{1}{c|}{$2.551$} & \multicolumn{1}{c|}{$2.431$} & \multicolumn{1}{c|}{$2.331$} & \multicolumn{1}{c|}{$2.948$} & \multicolumn{1}{c|}{$2.398$} & \multicolumn{1}{c|}{$2.272$} & \multicolumn{1}{c|}{$2.173$} & \multicolumn{1}{c|}{$1.036$} & \multicolumn{1}{c|}{$0.833$} & \multicolumn{1}{c|}{$0.615$} & \multicolumn{1}{c|}{$0.399$} \\ \hline


\end{tabular}
\caption{Average Lengths of 95\% Pointwise Prediction Intervals at $t=T_0+1$. The numbers in SPSC and SPSC-Ridge columns show the average length of 95\% pointwise prediction intervals obtained from the conformal inference approach in Section \ref{sec:Conformal}. The numbers in SCPI column show the average length of 95\% pointwise prediction intervals obtained from the approach proposed by \citet{Cattaneo2021} which is implemented in \texttt{scpi} R-package \citep{scpiPackage2023}.}
\label{tab:supp:Table4}
\vspace{-0.5cm}
\end{table}

Lastly, we report the bias and the empirical standard error of the estimators for the average treatment effects on the treated (ATT) obtained from \texttt{scpi}. Specifically, the estimator is obtained as $\widehat{\tau}_{\text{ATT}} = T_1^{-1} \sum_{t=T_0+1}^{T} \big\{ Y_t - \widehat{Y}_{t}^{(0)} \big\}$ 
where $\widehat{Y}_{t}^{(0)} $ is a predicted value of the treatment-free potential outcome at time $t$. For simplicity, we only consider the constant treatment effect case $\tau_t^*=3$. Table \ref{tab:supp:Table5} summarizes the result. Compared with the two estimators obtained from the SPSC, we find the estimator obtained from \texttt{scpi} yields a significant magnitude of biases even under a large sample size. Moreover, compared to the empirical standard errors, these biases are not negligible. Therefore, we conclude that the undercoverage of \texttt{scpi} reported in Table \ref{tab:supp:Table3} is because of the non-diminishing bias. 


\begin{table}[!htp]
\renewcommand{\arraystretch}{1.05} \centering
\footnotesize
\setlength{\tabcolsep}{5pt} 
\begin{tabular}{|ccc|cccc|cccc|cccc|}
\hline
\multicolumn{3}{|c|}{Estimator} & \multicolumn{4}{c|}{SPSC} & \multicolumn{4}{c|}{SPSC-Ridge} & \multicolumn{4}{c|}{SCPI} \\ \hline
\multicolumn{1}{|c|}{\multirow{2}{*}{$ \ \delta \ $}} & \multicolumn{1}{c|}{\multirow{2}{*}{$\ d \ $}} & \multirow{2}{*}{Statistic} & \multicolumn{4}{c|}{$T_0$} & \multicolumn{4}{c|}{$T_0$} & \multicolumn{4}{c|}{$T_0$} \\ \cline{4-15} 
\multicolumn{1}{|c|}{} & \multicolumn{1}{c|}{} & & \multicolumn{1}{c|}{50} & \multicolumn{1}{c|}{100} & \multicolumn{1}{c|}{250} & 1000 & \multicolumn{1}{c|}{50} & \multicolumn{1}{c|}{100} & \multicolumn{1}{c|}{250} & 1000 & \multicolumn{1}{c|}{50} & \multicolumn{1}{c|}{100} & \multicolumn{1}{c|}{250} & 1000 \\ \hline


\multicolumn{1}{|c|}{\multirow{6}{*}{0}} & \multicolumn{1}{c|}{\multirow{2}{*}{2}} & Bias ($\times$10) & \multicolumn{1}{c|}{-0.527} & \multicolumn{1}{c|}{0.435} & \multicolumn{1}{c|}{-0.078} & \multicolumn{1}{c|}{-0.019} & \multicolumn{1}{c|}{-0.158} & \multicolumn{1}{c|}{0.648} & \multicolumn{1}{c|}{0.018} & \multicolumn{1}{c|}{0.023} & \multicolumn{1}{c|}{1.058} & \multicolumn{1}{c|}{1.046} & \multicolumn{1}{c|}{0.890} & \multicolumn{1}{c|}{0.982} \\ \cline{3-15}
\multicolumn{1}{|c|}{} & \multicolumn{1}{c|}{} & ESE ($\times$10) & \multicolumn{1}{c|}{8.576} & \multicolumn{1}{c|}{6.311} & \multicolumn{1}{c|}{3.843} & \multicolumn{1}{c|}{2.021} & \multicolumn{1}{c|}{8.525} & \multicolumn{1}{c|}{6.302} & \multicolumn{1}{c|}{3.843} & \multicolumn{1}{c|}{2.016} & \multicolumn{1}{c|}{3.116} & \multicolumn{1}{c|}{2.305} & \multicolumn{1}{c|}{1.431} & \multicolumn{1}{c|}{0.719} \\ \cline{2-15}
\multicolumn{1}{|c|}{} & \multicolumn{1}{c|}{\multirow{2}{*}{5}} & Bias ($\times$10) & \multicolumn{1}{c|}{-0.345} & \multicolumn{1}{c|}{-0.093} & \multicolumn{1}{c|}{0.528} & \multicolumn{1}{c|}{0.332} & \multicolumn{1}{c|}{-1.301} & \multicolumn{1}{c|}{-0.679} & \multicolumn{1}{c|}{0.026} & \multicolumn{1}{c|}{0.384} & \multicolumn{1}{c|}{0.361} & \multicolumn{1}{c|}{0.524} & \multicolumn{1}{c|}{0.763} & \multicolumn{1}{c|}{0.689} \\ \cline{3-15}
\multicolumn{1}{|c|}{} & \multicolumn{1}{c|}{} & ESE ($\times$10) & \multicolumn{1}{c|}{9.887} & \multicolumn{1}{c|}{7.479} & \multicolumn{1}{c|}{4.133} & \multicolumn{1}{c|}{1.954} & \multicolumn{1}{c|}{8.509} & \multicolumn{1}{c|}{6.570} & \multicolumn{1}{c|}{3.658} & \multicolumn{1}{c|}{1.845} & \multicolumn{1}{c|}{3.044} & \multicolumn{1}{c|}{2.089} & \multicolumn{1}{c|}{1.309} & \multicolumn{1}{c|}{0.642} \\ \cline{2-15}
\multicolumn{1}{|c|}{} & \multicolumn{1}{c|}{\multirow{2}{*}{9}} & Bias ($\times$10) & \multicolumn{1}{c|}{-0.923} & \multicolumn{1}{c|}{0.228} & \multicolumn{1}{c|}{-0.402} & \multicolumn{1}{c|}{-0.317} & \multicolumn{1}{c|}{-1.228} & \multicolumn{1}{c|}{-0.128} & \multicolumn{1}{c|}{-0.527} & \multicolumn{1}{c|}{-0.366} & \multicolumn{1}{c|}{0.676} & \multicolumn{1}{c|}{0.821} & \multicolumn{1}{c|}{0.776} & \multicolumn{1}{c|}{0.686} \\ \cline{3-15}
\multicolumn{1}{|c|}{} & \multicolumn{1}{c|}{} & ESE ($\times$10) & \multicolumn{1}{c|}{9.365} & \multicolumn{1}{c|}{6.819} & \multicolumn{1}{c|}{4.047} & \multicolumn{1}{c|}{1.965} & \multicolumn{1}{c|}{8.545} & \multicolumn{1}{c|}{6.305} & \multicolumn{1}{c|}{3.828} & \multicolumn{1}{c|}{1.831} & \multicolumn{1}{c|}{2.895} & \multicolumn{1}{c|}{1.925} & \multicolumn{1}{c|}{1.167} & \multicolumn{1}{c|}{0.599} \\ \hline \hline
\multicolumn{1}{|c|}{\multirow{6}{*}{1}} & \multicolumn{1}{c|}{\multirow{2}{*}{2}} & Bias ($\times$10) & \multicolumn{1}{c|}{-0.344} & \multicolumn{1}{c|}{-0.143} & \multicolumn{1}{c|}{0.038} & \multicolumn{1}{c|}{-0.000} & \multicolumn{1}{c|}{-0.184} & \multicolumn{1}{c|}{-0.073} & \multicolumn{1}{c|}{0.082} & \multicolumn{1}{c|}{0.018} & \multicolumn{1}{c|}{0.538} & \multicolumn{1}{c|}{0.377} & \multicolumn{1}{c|}{0.394} & \multicolumn{1}{c|}{0.479} \\ \cline{3-15}
\multicolumn{1}{|c|}{} & \multicolumn{1}{c|}{} & ESE ($\times$10) & \multicolumn{1}{c|}{8.649} & \multicolumn{1}{c|}{6.106} & \multicolumn{1}{c|}{3.989} & \multicolumn{1}{c|}{2.014} & \multicolumn{1}{c|}{8.615} & \multicolumn{1}{c|}{6.087} & \multicolumn{1}{c|}{3.984} & \multicolumn{1}{c|}{2.014} & \multicolumn{1}{c|}{3.046} & \multicolumn{1}{c|}{2.090} & \multicolumn{1}{c|}{1.458} & \multicolumn{1}{c|}{0.674} \\ \cline{2-15}
\multicolumn{1}{|c|}{} & \multicolumn{1}{c|}{\multirow{2}{*}{5}} & Bias ($\times$10) & \multicolumn{1}{c|}{-0.040} & \multicolumn{1}{c|}{-0.081} & \multicolumn{1}{c|}{0.251} & \multicolumn{1}{c|}{0.113} & \multicolumn{1}{c|}{-0.318} & \multicolumn{1}{c|}{-0.302} & \multicolumn{1}{c|}{0.058} & \multicolumn{1}{c|}{0.065} & \multicolumn{1}{c|}{0.336} & \multicolumn{1}{c|}{0.621} & \multicolumn{1}{c|}{0.593} & \multicolumn{1}{c|}{0.587} \\ \cline{3-15}
\multicolumn{1}{|c|}{} & \multicolumn{1}{c|}{} & ESE ($\times$10) & \multicolumn{1}{c|}{9.034} & \multicolumn{1}{c|}{6.635} & \multicolumn{1}{c|}{3.719} & \multicolumn{1}{c|}{1.778} & \multicolumn{1}{c|}{8.532} & \multicolumn{1}{c|}{6.377} & \multicolumn{1}{c|}{3.656} & \multicolumn{1}{c|}{1.739} & \multicolumn{1}{c|}{2.909} & \multicolumn{1}{c|}{2.064} & \multicolumn{1}{c|}{1.241} & \multicolumn{1}{c|}{0.628} \\ \cline{2-15}
\multicolumn{1}{|c|}{} & \multicolumn{1}{c|}{\multirow{2}{*}{9}} & Bias ($\times$10) & \multicolumn{1}{c|}{-1.083} & \multicolumn{1}{c|}{-0.953} & \multicolumn{1}{c|}{-0.647} & \multicolumn{1}{c|}{-0.126} & \multicolumn{1}{c|}{-1.113} & \multicolumn{1}{c|}{-1.088} & \multicolumn{1}{c|}{-0.862} & \multicolumn{1}{c|}{-0.462} & \multicolumn{1}{c|}{0.122} & \multicolumn{1}{c|}{0.226} & \multicolumn{1}{c|}{0.295} & \multicolumn{1}{c|}{0.358} \\ \cline{3-15}
\multicolumn{1}{|c|}{} & \multicolumn{1}{c|}{} & ESE ($\times$10) & \multicolumn{1}{c|}{9.964} & \multicolumn{1}{c|}{6.548} & \multicolumn{1}{c|}{3.895} & \multicolumn{1}{c|}{1.867} & \multicolumn{1}{c|}{9.185} & \multicolumn{1}{c|}{6.135} & \multicolumn{1}{c|}{3.781} & \multicolumn{1}{c|}{1.848} & \multicolumn{1}{c|}{2.744} & \multicolumn{1}{c|}{1.819} & \multicolumn{1}{c|}{1.237} & \multicolumn{1}{c|}{0.585} \\ \hline



\end{tabular}
\caption{Bias and Empirical Standard Errors of the Three ATT Estimators Obtained from Our Approach and the SCPI Approach Proposed by \citet{Cattaneo2021}. We remark that the results of the SPSC estimators are the same as those in Table \ref{tab:supp:Table00}.}
\label{tab:supp:Table5}
\vspace{-0.5cm}
\end{table}



\subsection{Simulation Studies under the Simulation Scenario Given in \citet{Cattaneo2021}} \label{sec:supp:Simulation PI}



For a fair comparison, we adopt a simulation scenario setup in \citet{Cattaneo2021}. In particular, we consider the following data generating process. First, we consider the length of the pre- and post-treatment periods as $T_0 = 100$ and $T_1=1$. Second, we choose the number of donors as $d = 10$, which are generated from the following AR(1) model:
\begin{align*}
&
W_{it} = \rho W_{i, t-1} + \eta_{it} 
\ , \
t=1,\ldots,T_0
\ , \ 
i=1,\ldots,d \ .
\end{align*}
Here, the autocorrelation coefficient $\rho$ is chosen from $\rho \in \{0,0.5,1\}$, $\eta$ are generated from the standard normal distribution and are independent and identically distributed, and the baseline value $W_{i0}$ is set to zero. For the post-treatment period, we consider the following model for donors:
\begin{align*}
& W_{1 , T_0+1} = \rho W_{1 T_0} + \eta_{1 , T_0+1} + \zeta \texttt{sd} ( W_{11},\ldots,W_{1 T_0} ) 
\\
&
W_{i , T_0+1} = \rho W_{i T_0} + \eta_{i , T_0+1} \ , \quad i=2,\ldots,d \ ,
\end{align*}
where $\zeta \in \{ -1,-0.5,0,0.5,1 \}$ parameterizes the degree of the shift in the first donor's post-treatment outcome. The treatment-free potential outcome of the treated unit is generated as follows:
\begin{align*}
\potY{t}{0} = 0.3 W_{1t} + 0.4 W_{2t} + 0.3 W_{3t} + 0.5 e_{t} \ , \ t=1,\ldots, T_0+T_1 \ ,
\end{align*}
where $e_t$ are independently generated from a standard normal distribution. We consider $\potY{t}{0} = \potY{t}{1}$, i.e., no treatment effect. We remark that the data generating process violates Assumption \ref{assumption:SC} because $\EXP \big\{ \potY{t}{0} - \big( 0.3 W_{1t} + 0.4 W_{2t} + 0.3 W_{3t} \big) \cond \potY{t}{0} \big\} \neq 0 $. Therefore, the proposed conformal inference approach for the SPSC framework in Section \ref{sec:Conformal} may fail in this data generating process. 

For our methods, we consider the SPSC estimators with ridge regularization. For function $\bg_t$, we consider the following two specifications: (i) time-invariant $\bg$ and (ii) time-varying $\bg_t$, which are given as follows:
\begin{align} \label{eq-compare-SCPI}
&
\bg(\potY{t}{0})
= 
\mathfrak{b}_{20}(\potY{t}{0}) 
\ , 
&&
\bg_t(\potY{t}{0})
=
\left\{
\begin{array}{ll}
\mathfrak{b}_{20}(\potY{t}{0}) 
&
\text{ if $\{Y_1,\ldots,Y_{T_0}\}$ is stationary}
\\[0.5cm]
\begin{bmatrix}
\mathfrak{b}_{20}(\potY{t}{0}) 
\\
\mathfrak{b}_{5}(t)
\end{bmatrix}
&
\text{ if $\{Y_1,\ldots,Y_{T_0}\}$ is not stationary}
\end{array}
\right.
\end{align}
Here, $\mathfrak{b}_{d}$ is chosen as the $d$-dimensional time-invariant cubic B-spline bases function. In order to check the stationarity of $Y_t$, we conduct Box-Pierce test \citep{BoxPierce1970} by using \texttt{Box.text} function implemented in the base R. If the p-value is less than 0.01, we conclude that $Y_t$ is nonstationary. Note that the dimension of $\mathfrak{b}_{5}(t)$ is chosen as the closest integer of $T_0^{1/3} = 100^{1/3} = 4.64$.

We repeat the simulation 500 times and calculate the empirical coverage rates of 95\% confidence intervals from these repetitions for each simulation scenario. The results are presented in Table \ref{tab:supp:Table10}, which exhibits somewhat opposite results compared to Table \ref{tab:supp:Table3}. More specifically, we find that the \texttt{scpi} approach achieves the nominal coverage rate across all simulation scenarios in general. However, we find that the conformal inference approach for the SPSC with time-invariant $\bg$ fails to achieve the nominal coverage rate, especially when the autocorrelation coefficient is large (i.e., $\rho=1$) and the first donor's post-treatment outcome is significantly different from its pre-treatment outcome (i.e., $\zeta=\pm 1$). We conjecture that the undercoverage observed in these cases may be attributed to the nonstationarity of $W_{it}$ and $\potY{t}{0}$. In these problematic cases, the conformal inference approach for the SPSC with time-varying $\bg_t$ appears to significantly improve the performance of our conformal inference approach. This confirms that the specification in Section \ref{sec:supp:time g function} is useful for improving the performance of the proposed conformal inference approach in the presence of nonstationarity. 



\begin{table}[!htp]
\renewcommand{\arraystretch}{1.1} \centering
\footnotesize
\setlength{\tabcolsep}{3pt} 
\hspace*{-0.5in}
\begin{tabular}{|cc|ccccc|ccccc|ccccc|}
\hline
\multicolumn{2}{|c|}{$\rho$} & \multicolumn{5}{c|}{0} & \multicolumn{5}{c|}{0.5} & \multicolumn{5}{c|}{1} \\ \hline
 \multicolumn{2}{|c|}{$\zeta$} & \multicolumn{1}{c|}{-1} & \multicolumn{1}{c|}{-0.5} & \multicolumn{1}{c|}{0} & \multicolumn{1}{c|}{0.5} & 1 & \multicolumn{1}{c|}{-1} & \multicolumn{1}{c|}{-0.5} & \multicolumn{1}{c|}{0} & \multicolumn{1}{c|}{0.5} & 1 & \multicolumn{1}{c|}{-1} & \multicolumn{1}{c|}{-0.5} & \multicolumn{1}{c|}{0} & \multicolumn{1}{c|}{0.5} & 1 \\ \hline
 \multicolumn{1}{|c|}{\multirow{2}{*}{SPSC-Ridge}} & Time-invariant $\bg$ & \multicolumn{1}{c|}{0.942} & \multicolumn{1}{c|}{0.946} & \multicolumn{1}{c|}{0.946} & \multicolumn{1}{c|}{0.944} & \multicolumn{1}{c|}{0.934} & \multicolumn{1}{c|}{0.920} & \multicolumn{1}{c|}{0.942} & \multicolumn{1}{c|}{0.956} & \multicolumn{1}{c|}{0.956} & \multicolumn{1}{c|}{0.946} & \multicolumn{1}{c|}{0.808} & \multicolumn{1}{c|}{0.852} & \multicolumn{1}{c|}{0.858} & \multicolumn{1}{c|}{0.856} & 0.824 \\ \cline{2-17}
 \multicolumn{1}{|c|}{} & Time-varying $\bg_t$ & \multicolumn{1}{c|}{0.942} & \multicolumn{1}{c|}{0.946} & \multicolumn{1}{c|}{0.946} & \multicolumn{1}{c|}{0.944} & \multicolumn{1}{c|}{0.934} & \multicolumn{1}{c|}{0.946} & \multicolumn{1}{c|}{0.954} & \multicolumn{1}{c|}{0.960} & \multicolumn{1}{c|}{0.958} & \multicolumn{1}{c|}{0.952} & \multicolumn{1}{c|}{0.914} & \multicolumn{1}{c|}{0.932} & \multicolumn{1}{c|}{0.942} & \multicolumn{1}{c|}{0.952} & 0.948 \\ \hline
 \multicolumn{2}{|c|}{SCPI} & \multicolumn{1}{c|}{0.974} & \multicolumn{1}{c|}{0.976} & \multicolumn{1}{c|}{0.974} & \multicolumn{1}{c|}{0.976} & \multicolumn{1}{c|}{0.978} & \multicolumn{1}{c|}{0.974} & \multicolumn{1}{c|}{0.970} & \multicolumn{1}{c|}{0.970} & \multicolumn{1}{c|}{0.978} & \multicolumn{1}{c|}{0.980} & \multicolumn{1}{c|}{0.984} & \multicolumn{1}{c|}{0.980} & \multicolumn{1}{c|}{0.978} & \multicolumn{1}{c|}{0.982} & 0.988 \\ \hline
\end{tabular}
\caption{Empirical Coverage Rates of 95\% Pointwise Prediction Intervals. 
The numbers in SPSC-Ridge columns show the results of the conformal inference approach in Section \ref{sec:Conformal}. 
Time-invariant $\bg$ and time-varying $\bg_t$ are chosen from \eqref{eq-compare-SCPI}.
The numbers in SCPI column show the results from the approach proposed by \citet{Cattaneo2021} which is implemented in \texttt{scpi} R-package \citep{scpiPackage2023}.}
\label{tab:supp:Table10}
\vspace*{-0.3cm}
\end{table}



% \begin{table}[!htp]
% \renewcommand{\arraystretch}{1.05} \centering
% \footnotesize
% \setlength{\tabcolsep}{5pt} 
% \hspace*{-0.25in}
% \begin{tabular}{|cc|ccc|ccc|ccc|ccc|}
% \hline
% \multicolumn{2}{|c|}{Estimator} & \multicolumn{3}{c|}{SPSC} & \multicolumn{3}{c|}{SPSC-Ridge} & \multicolumn{3}{c|}{SPSC-Ridge-Modified} & \multicolumn{3}{c|}{SCPI} \\ \hline
% \multicolumn{2}{|c|}{$\rho$} & \multicolumn{1}{c|}{0} & \multicolumn{1}{c|}{0.5} & 1 & \multicolumn{1}{c|}{0} & \multicolumn{1}{c|}{0.5} & 1 & \multicolumn{1}{c|}{0} & \multicolumn{1}{c|}{0.5} & 1 & \multicolumn{1}{c|}{0} & \multicolumn{1}{c|}{0.5} & 1 \\ \hline

% \multicolumn{1}{|c|}{\multirow{5}{*}{$\zeta$}} & -1 & \multicolumn{1}{c|}{0.888} & \multicolumn{1}{c|}{0.898} & \multicolumn{1}{c|}{0.794} & \multicolumn{1}{c|}{0.942} & \multicolumn{1}{c|}{0.920} & \multicolumn{1}{c|}{0.808} & \multicolumn{1}{c|}{0.920} & \multicolumn{1}{c|}{0.932} & \multicolumn{1}{c|}{0.906} & \multicolumn{1}{c|}{0.974} & \multicolumn{1}{c|}{0.974} & \multicolumn{1}{c|}{0.984} \\ \cline{2-14}
% \multicolumn{1}{|c|}{} & -0.5 & \multicolumn{1}{c|}{0.892} & \multicolumn{1}{c|}{0.906} & \multicolumn{1}{c|}{0.830} & \multicolumn{1}{c|}{0.946} & \multicolumn{1}{c|}{0.942} & \multicolumn{1}{c|}{0.852} & \multicolumn{1}{c|}{0.932} & \multicolumn{1}{c|}{0.946} & \multicolumn{1}{c|}{0.924} & \multicolumn{1}{c|}{0.976} & \multicolumn{1}{c|}{0.970} & \multicolumn{1}{c|}{0.980} \\ \cline{2-14}
% \multicolumn{1}{|c|}{} & 0 & \multicolumn{1}{c|}{0.886} & \multicolumn{1}{c|}{0.904} & \multicolumn{1}{c|}{0.832} & \multicolumn{1}{c|}{0.946} & \multicolumn{1}{c|}{0.956} & \multicolumn{1}{c|}{0.858} & \multicolumn{1}{c|}{0.938} & \multicolumn{1}{c|}{0.960} & \multicolumn{1}{c|}{0.928} & \multicolumn{1}{c|}{0.974} & \multicolumn{1}{c|}{0.970} & \multicolumn{1}{c|}{0.978} \\ \cline{2-14}
% \multicolumn{1}{|c|}{} & 0.5 & \multicolumn{1}{c|}{0.886} & \multicolumn{1}{c|}{0.916} & \multicolumn{1}{c|}{0.810} & \multicolumn{1}{c|}{0.944} & \multicolumn{1}{c|}{0.956} & \multicolumn{1}{c|}{0.856} & \multicolumn{1}{c|}{0.940} & \multicolumn{1}{c|}{0.952} & \multicolumn{1}{c|}{0.936} & \multicolumn{1}{c|}{0.976} & \multicolumn{1}{c|}{0.978} & \multicolumn{1}{c|}{0.982} \\ \cline{2-14}
% \multicolumn{1}{|c|}{} & 1 & \multicolumn{1}{c|}{0.868} & \multicolumn{1}{c|}{0.908} & \multicolumn{1}{c|}{0.788} & \multicolumn{1}{c|}{0.934} & \multicolumn{1}{c|}{0.946} & \multicolumn{1}{c|}{0.824} & \multicolumn{1}{c|}{0.928} & \multicolumn{1}{c|}{0.946} & \multicolumn{1}{c|}{0.934} & \multicolumn{1}{c|}{0.978} & \multicolumn{1}{c|}{0.980} & \multicolumn{1}{c|}{0.988} \\ \hline


% \end{tabular}
% \caption{Empirical Coverage Rates of 95\% Pointwise Prediction Intervals. 
% The numbers in SPSC and SPSC-Ridge columns show the results of the conformal inference approach in Section \ref{sec:Conformal}. 
% The numbers in SPSC-Ridge-Modified column show the results of the conformal inference approach in Section \ref{sec:Conformal} where the donor pool is chosen by following the approach in Section \ref{sec:supp:Donors} and the coefficient function $\bg_t$ is specified as \eqref{eq-compare-SCPI}. 
% The numbers in SCPI column show the results from the approach proposed by \citet{Cattaneo2021} which is implemented in \texttt{scpi} R-package \citep{scpiPackage2023}.}
% \label{tab:supp:Table10}
% \vspace*{-0.5cm}
% \end{table}

Next, we calculate the average length of the 95\% prediction interval for $t=T_0+1$, i.e., the first post-treatment period, across 500 repetitions. 
Table \ref{tab:supp:Table11} shows the average lengths.
We find that the prediction intervals obtained from \texttt{scpi} are generally narrower than those obtained from the SPSC, except for the cases where $\rho=1$ and $\zeta=\pm1$. 

\begin{table}[!htp]
\renewcommand{\arraystretch}{1.1} \centering
\footnotesize
\setlength{\tabcolsep}{3pt} 
\hspace*{-0.5in}
\begin{tabular}{|cc|ccccc|ccccc|ccccc|}
\hline
\multicolumn{2}{|c|}{$\rho$} & \multicolumn{5}{c|}{0} & \multicolumn{5}{c|}{0.5} & \multicolumn{5}{c|}{1} \\ \hline
 \multicolumn{2}{|c|}{$\zeta$} & \multicolumn{1}{c|}{-1} & \multicolumn{1}{c|}{-0.5} & \multicolumn{1}{c|}{0} & \multicolumn{1}{c|}{0.5} & 1 & \multicolumn{1}{c|}{-1} & \multicolumn{1}{c|}{-0.5} & \multicolumn{1}{c|}{0} & \multicolumn{1}{c|}{0.5} & 1 & \multicolumn{1}{c|}{-1} & \multicolumn{1}{c|}{-0.5} & \multicolumn{1}{c|}{0} & \multicolumn{1}{c|}{0.5} & 1 \\ \hline
 \multicolumn{1}{|c|}{\multirow{2}{*}{SPSC-Ridge}} & Time-invariant $\bg$ & \multicolumn{1}{c|}{3.018} & \multicolumn{1}{c|}{3.035} & \multicolumn{1}{c|}{3.040} & \multicolumn{1}{c|}{3.046} & \multicolumn{1}{c|}{3.052} & \multicolumn{1}{c|}{2.963} & \multicolumn{1}{c|}{2.967} & \multicolumn{1}{c|}{2.970} & \multicolumn{1}{c|}{2.991} & \multicolumn{1}{c|}{2.971} & \multicolumn{1}{c|}{3.262} & \multicolumn{1}{c|}{3.288} & \multicolumn{1}{c|}{3.272} & \multicolumn{1}{c|}{3.285} & 3.281 \\ \cline{2-17}
 \multicolumn{1}{|c|}{} & Time-varying $\bg_t$ & \multicolumn{1}{c|}{3.018} & \multicolumn{1}{c|}{3.035} & \multicolumn{1}{c|}{3.039} & \multicolumn{1}{c|}{3.045} & \multicolumn{1}{c|}{3.052} & \multicolumn{1}{c|}{2.904} & \multicolumn{1}{c|}{2.903} & \multicolumn{1}{c|}{2.902} & \multicolumn{1}{c|}{2.921} & \multicolumn{1}{c|}{2.931} & \multicolumn{1}{c|}{2.926} & \multicolumn{1}{c|}{2.900} & \multicolumn{1}{c|}{2.904} & \multicolumn{1}{c|}{2.931} & 2.979 \\ \hline
 \multicolumn{2}{|c|}{SCPI} & \multicolumn{1}{c|}{2.657} & \multicolumn{1}{c|}{2.596} & \multicolumn{1}{c|}{2.576} & \multicolumn{1}{c|}{2.597} & \multicolumn{1}{c|}{2.658} & \multicolumn{1}{c|}{2.654} & \multicolumn{1}{c|}{2.593} & \multicolumn{1}{c|}{2.574} & \multicolumn{1}{c|}{2.598} & \multicolumn{1}{c|}{2.662} & \multicolumn{1}{c|}{3.154} & \multicolumn{1}{c|}{3.024} & \multicolumn{1}{c|}{2.985} & \multicolumn{1}{c|}{3.035} & 3.165 \\ \hline
\end{tabular}
\caption{Average Lengths of 95\% Pointwise Prediction Intervals at $t=T_0+1$. Each column has the same specification as in Table \ref{tab:supp:Table10}.
}
\label{tab:supp:Table11}
\vspace*{-0.3cm}
\end{table}


Lastly, we report the bias and the empirical standard error of the estimators for the ATT obtained from \texttt{scpi}. Specifically, the estimator is obtained as $\widehat{\tau}_{\text{ATT}}
=
Y_t - \widehat{Y}_{t}^{(0)}$ where $\widehat{Y}_{t}^{(0)} $ is a predicted value of the treatment-free potential outcome at time $t=T_0+1$. Comparing the estimator derived from \texttt{scpi} to the SPSC estimator, we observe that the latter leads to larger biases. Nevertheless, when considering the empirical standard errors, these biases become negligible. As a result, we deduce that the ATT estimator from the SPSC approach exhibits a negligible bias, whereas the conformal inference approach can be anticonservative in the presence of nonstationarity.

\begin{table}[!htp]
\renewcommand{\arraystretch}{1.05} \centering
\footnotesize
\setlength{\tabcolsep}{3pt} 
\hspace*{-0.5in}
\begin{tabular}{|cc|ccccc|ccccc|ccccc|}
\hline
\multicolumn{2}{|c|}{$\rho$} & \multicolumn{5}{c|}{0} & \multicolumn{5}{c|}{0.5} & \multicolumn{5}{c|}{1} \\ \hline
 \multicolumn{2}{|c|}{$\zeta$} & \multicolumn{1}{c|}{-1} & \multicolumn{1}{c|}{-0.5} & \multicolumn{1}{c|}{0} & \multicolumn{1}{c|}{0.5} & 1 & \multicolumn{1}{c|}{-1} & \multicolumn{1}{c|}{-0.5} & \multicolumn{1}{c|}{0} & \multicolumn{1}{c|}{0.5} & 1 & \multicolumn{1}{c|}{-1} & \multicolumn{1}{c|}{-0.5} & \multicolumn{1}{c|}{0} & \multicolumn{1}{c|}{0.5} & 1 \\ \hline
 \multicolumn{1}{|c|}{\multirow{2}{*}{SPSC-Ridge}} & Time-invariant $\bg$ & \multicolumn{1}{c|}{-0.030} & \multicolumn{1}{c|}{-0.034} & \multicolumn{1}{c|}{-0.038} & \multicolumn{1}{c|}{-0.041} & \multicolumn{1}{c|}{-0.045} & \multicolumn{1}{c|}{-0.084} & \multicolumn{1}{c|}{-0.085} & \multicolumn{1}{c|}{-0.085} & \multicolumn{1}{c|}{-0.086} & \multicolumn{1}{c|}{-0.086} & \multicolumn{1}{c|}{-0.181} & \multicolumn{1}{c|}{-0.102} & \multicolumn{1}{c|}{-0.023} & \multicolumn{1}{c|}{0.055} & 0.134 \\ \cline{2-17}
 \multicolumn{1}{|c|}{} & Time-varying $\bg_t$ & \multicolumn{1}{c|}{-0.030} & \multicolumn{1}{c|}{-0.034} & \multicolumn{1}{c|}{-0.038} & \multicolumn{1}{c|}{-0.042} & \multicolumn{1}{c|}{-0.046} & \multicolumn{1}{c|}{-0.086} & \multicolumn{1}{c|}{-0.082} & \multicolumn{1}{c|}{-0.078} & \multicolumn{1}{c|}{-0.075} & \multicolumn{1}{c|}{-0.071} & \multicolumn{1}{c|}{-0.158} & \multicolumn{1}{c|}{-0.098} & \multicolumn{1}{c|}{-0.039} & \multicolumn{1}{c|}{0.021} & 0.080 \\ \hline
 \multicolumn{2}{|c|}{SCPI} & \multicolumn{1}{c|}{-0.063} & \multicolumn{1}{c|}{-0.051} & \multicolumn{1}{c|}{-0.038} & \multicolumn{1}{c|}{-0.025} & \multicolumn{1}{c|}{-0.013} & \multicolumn{1}{c|}{-0.062} & \multicolumn{1}{c|}{-0.048} & \multicolumn{1}{c|}{-0.035} & \multicolumn{1}{c|}{-0.021} & \multicolumn{1}{c|}{-0.008} & \multicolumn{1}{c|}{-0.068} & \multicolumn{1}{c|}{-0.048} & \multicolumn{1}{c|}{-0.028} & \multicolumn{1}{c|}{-0.008} & 0.012 \\ \hline
\end{tabular}\\
\vspace*{0.5cm}
\hspace*{-0.5in}
\begin{tabular}{|cc|ccccc|ccccc|ccccc|}
\hline
\multicolumn{2}{|c|}{$\rho$} & \multicolumn{5}{c|}{0} & \multicolumn{5}{c|}{0.5} & \multicolumn{5}{c|}{1} \\ \hline
 \multicolumn{2}{|c|}{$\zeta$} & \multicolumn{1}{c|}{-1} & \multicolumn{1}{c|}{-0.5} & \multicolumn{1}{c|}{0} & \multicolumn{1}{c|}{0.5} & 1 & \multicolumn{1}{c|}{-1} & \multicolumn{1}{c|}{-0.5} & \multicolumn{1}{c|}{0} & \multicolumn{1}{c|}{0.5} & 1 & \multicolumn{1}{c|}{-1} & \multicolumn{1}{c|}{-0.5} & \multicolumn{1}{c|}{0} & \multicolumn{1}{c|}{0.5} & 1 \\ \hline
 \multicolumn{1}{|c|}{\multirow{2}{*}{SPSC-Ridge}} & Time-invariant $\bg$ & \multicolumn{1}{c|}{0.774} & \multicolumn{1}{c|}{0.753} & \multicolumn{1}{c|}{0.748} & \multicolumn{1}{c|}{0.760} & \multicolumn{1}{c|}{0.789} & \multicolumn{1}{c|}{0.789} & \multicolumn{1}{c|}{0.765} & \multicolumn{1}{c|}{0.754} & \multicolumn{1}{c|}{0.757} & \multicolumn{1}{c|}{0.774} & \multicolumn{1}{c|}{1.152} & \multicolumn{1}{c|}{1.070} & \multicolumn{1}{c|}{1.037} & \multicolumn{1}{c|}{1.057} & 1.127 \\ \cline{2-17}
 \multicolumn{1}{|c|}{} & Time-varying $\bg_t$ & \multicolumn{1}{c|}{0.774} & \multicolumn{1}{c|}{0.753} & \multicolumn{1}{c|}{0.748} & \multicolumn{1}{c|}{0.760} & \multicolumn{1}{c|}{0.789} & \multicolumn{1}{c|}{0.756} & \multicolumn{1}{c|}{0.736} & \multicolumn{1}{c|}{0.728} & \multicolumn{1}{c|}{0.733} & \multicolumn{1}{c|}{0.752} & \multicolumn{1}{c|}{0.852} & \multicolumn{1}{c|}{0.786} & \multicolumn{1}{c|}{0.761} & \multicolumn{1}{c|}{0.780} & 0.840 \\ \hline
 \multicolumn{2}{|c|}{SCPI} & \multicolumn{1}{c|}{0.516} & \multicolumn{1}{c|}{0.512} & \multicolumn{1}{c|}{0.510} & \multicolumn{1}{c|}{0.509} & \multicolumn{1}{c|}{0.509} & \multicolumn{1}{c|}{0.517} & \multicolumn{1}{c|}{0.514} & \multicolumn{1}{c|}{0.512} & \multicolumn{1}{c|}{0.510} & \multicolumn{1}{c|}{0.510} & \multicolumn{1}{c|}{0.536} & \multicolumn{1}{c|}{0.533} & \multicolumn{1}{c|}{0.533} & \multicolumn{1}{c|}{0.535} & 0.540 \\ \hline
\end{tabular}

\caption{Biases (top table) and Empirical Standard Errors (bottom table) of the Three ATT Estimators Obtained from Our Approach and the SCPI Approach proposed by \citet{Cattaneo2021}. We remark that the results of the SPSC estimators are the same as those in Table \ref{tab:supp:Table00}.}
\label{tab:supp:Table12}
\vspace*{-0.3cm}
\end{table}



\subsection{Additional Results of the Data Analysis}		\label{sec:supp:Data}

In this Section, we provide additional results of the data analysis in Section \ref{sec:Data}. 
First, we provide details on how to choose the donor pool. 
As we briefly mentioned in the data analysis section, some donor candidates had severely different stock price values. In the most severe cases, the ranges of the pre- and post-treatment periods of the stock prices do not overlap. As a criterion for choosing the donor pools, we use the following overlap metric:
\begin{align*}
\text{Overlap}_{k}
=
\frac{1}{T_1} \sum_{t=T_0+1}^{T} \ind \Big\{ W_{kt} \in \text{range} \big( W_{k1},\ldots,W_{kT_0} \big) \Big\} \ , 
\ k=1,\ldots, 49 \ .
\end{align*}
Based on the overlap, we define the four groups of roughly equal size. Specifically, Group 1, 2, and 4 has 12 donors and Group 3 has 13 donors, respectively; see Figure \ref{fig:Sim:Group}. 
% Figure environment removed		

In addition to the four groups based on the overlap metric, we choose the donors based on lasso regularization \citep{Lasso1996}. Specifically, we solve the following GMM with $\ell_1$ regularization:	 
\begin{align*}
\widehat{\bgamma}_{\lambda}
& =
\argmin_{\bgamma}
\Big[
\big\{ 
\widehat{\Psi}_{\pre} ( \bgamma ) 
\big\} \T
\big\{ 
\widehat{\Psi}_{\pre} ( \bgamma ) 
\big\}
+
\lambda \big\| \bgamma \big\|_{1} 
\Big]
\\
& =
\argmin_{\bgamma}
\Big[
\big\| 
\widehat{\bG}_{YY }
-
\widehat{\bG}_{YW }\T 
\bgamma
\big\|_2^2
+
\lambda \big\| \bgamma \big\|_{1} 
\Big] 
\ ,
\end{align*}
where $\widehat{\bG}_{YW }$ and $\widehat{\bG}_{YY}$ are defined in \eqref{eq-Gyw Gyy}. The regularization parameter is chosen from cross-validation. We remark that the number of non-zero $\widehat{\bgamma}_{\lambda}$ depends on the dimension of $\bg_t( \potY{t}{0})$. Therefore, we vary the dimension of $\bg_t$ across a range from 2 to 98, which spans twice the number of donors. Nonetheless, as in Figure \ref{fig:Sim:Lasso}, we find the number of selected donors does not vary a lot across the dimension of $\bg_t( \potY{t}{0})$. In particular, five donors are selected when $\text{dim}(\bg_t)=10$, in alignment with the relationship $\text{dim}(\bg_t) = 2d$ specified in the simulation studies. Therefore, we define these five donors as Group 5.

% Figure environment removed		

Lastly, we follow the approach in Section \ref{sec:supp:Donors}, which yields 24 donors. We refer to this donor pool as Group 6. Of note, this group is reported in the main paper. 

For the coefficient function $\bg_t$, we consider the following two specifications $\bg_t$-(i) and $\bg_t$-(ii):
\begin{itemize}[leftmargin=0.5cm, itemsep=0cm]
\item[(i)] (\textit{Time-invariant $\bg$}) Following the approach in the main paper, we define $\bg(\potY{t}{0})$ as a time-invariant function as follows:
\begin{align*}
\bg (\potY{t}{0})
= 
\mathfrak{b}_{\ell}(\potY{t}{0})  
\ , \quad 
\ell \in \{ 24, 24, 24, 24, 10, 48 \} \text{ for Groups 1--6}
\end{align*}
where $\mathfrak{b}_{\ell}$ is the $\ell$-dimensional cubic B-spline bases function. In other words, for Groups 1--4, we define $\bg$ as the 24-dimensional cubic B-spline bases function. For Group 5, we define $\bg$ as the 10-dimensional cubic B-spline bases function. Lastly, for Group 6, we define $\bg$ as the 48-dimensional cubic B-spline bases function. 

\item[(ii)] (\textit{Time-varying $\bg_t$}) Next, we consider time-varying $\bg_t(\potY{t}{0})$ by following the specification in Section \ref{sec:supp:time g function}. In particular, we define
\begin{align*}
\bg_t(\potY{t}{0})
=
\begin{bmatrix}
\mathfrak{b}_{\ell}(\potY{t}{0})  
\\
\mathfrak{b}_{6}(t)
\end{bmatrix}
\ , \quad 
\ell \in \{ 24, 24, 24, 24, 10, 48 \} \text{ for Groups 1--6}
\end{align*}
where $\mathfrak{b}_{6}(t)$ is the 6-dimensional cubic B-spline bases function where the dimension is obtained by choosing the closest integer of $T_0^{1/3} = 217^{1/3} = 6.01$.
\end{itemize}



Using these six choices for the donor pool and the two choices for $\bg_t$, we focus on the estimation of the ATT. Table \ref{tab:supp:ATT} summarizes the result. We find that the results are similar to each other. In terms of the length of the 95\% confidence intervals, the SPSC with ridge regularization results in the narrowest confidence intervals in general, except for Group 2 with time-invariayt $\bg$. These additional analyses for the ATT corroborate the results in Table \ref{tab:data:ATT} of the main paper.

\begin{table}[!htp]
\renewcommand{\arraystretch}{1.3} \centering
\scriptsize
\setlength{\tabcolsep}{3pt} 
\hspace*{-0.8cm}
\begin{tabular}{|c|c|c|c|c|c|c|c|}
\hline
\multirow{2}{*}{Donor Pool} & \multirow{2}{*}{Statistic} & \multicolumn{3}{c|}{Time-invariant $\bg$} & \multicolumn{3}{c|}{Time-varying $\bg_t$} \\ \cline{3-8} 
& & \multicolumn{1}{c|}{OLS} & \multicolumn{1}{c|}{SPSC} & SPSC-Ridge & \multicolumn{1}{c|}{OLS} & \multicolumn{1}{c|}{SPSC} & SPSC-Ridge \\ \hline
 \multicolumn{1}{|c|}{\multirow{3}{*}{Group 1 (12)}} & Estimate & $-0.906$ & $-0.855$ & $-0.855$ & $-0.847$ & $-0.883$ & $-0.850$ \\ \cline{2-8}
 \multicolumn{1}{|c|}{} & ASE & $0.101$ & $0.115$ & $0.100$ & $0.117$ & $0.124$ & $0.094$ \\ \cline{2-8}
 \multicolumn{1}{|c|}{} & 95\% CI & $(-1.103, -0.709)$ & $(-1.081, -0.629)$ & $(-1.051, -0.660)$ & $(-1.076, -0.618)$ & $(-1.127, -0.639)$ & $(-1.034, -0.666)$ \\ \hline
 \multicolumn{1}{|c|}{\multirow{3}{*}{Group 2 (12)}} & Estimate & $-0.840$ & $-0.787$ & $-0.887$ & $-0.897$ & $-0.790$ & $-0.892$ \\ \cline{2-8}
 \multicolumn{1}{|c|}{} & ASE & $0.089$ & $0.104$ & $0.098$ & $0.132$ & $0.107$ & $0.097$ \\ \cline{2-8}
 \multicolumn{1}{|c|}{} & 95\% CI & $(-1.014, -0.667)$ & $(-0.991, -0.583)$ & $(-1.079, -0.696)$ & $(-1.156, -0.638)$ & $(-0.999, -0.580)$ & $(-1.082, -0.703)$ \\ \hline
 \multicolumn{1}{|c|}{\multirow{3}{*}{Group 3 (13)}} & Estimate & $-0.875$ & $-0.520$ & $-0.798$ & $-0.797$ & $-0.614$ & $-0.798$ \\ \cline{2-8}
 \multicolumn{1}{|c|}{} & ASE & $0.154$ & $0.235$ & $0.084$ & $0.184$ & $0.212$ & $0.084$ \\ \cline{2-8}
 \multicolumn{1}{|c|}{} & 95\% CI & $(-1.177, -0.574)$ & $(-0.982, -0.059)$ & $(-0.963, -0.633)$ & $(-1.158, -0.437)$ & $(-1.030, -0.198)$ & $(-0.963, -0.633)$ \\ \hline
 \multicolumn{1}{|c|}{\multirow{3}{*}{Group 4 (12)}} & Estimate & $-0.922$ & $-0.852$ & $-0.713$ & $-0.708$ & $-0.879$ & $-0.710$ \\ \cline{2-8}
 \multicolumn{1}{|c|}{} & ASE & $0.095$ & $0.138$ & $0.072$ & $0.259$ & $0.137$ & $0.073$ \\ \cline{2-8}
 \multicolumn{1}{|c|}{} & 95\% CI & $(-1.109, -0.735)$ & $(-1.122, -0.581)$ & $(-0.854, -0.571)$ & $(-1.216, -0.200)$ & $(-1.147, -0.610)$ & $(-0.854, -0.567)$ \\ \hline
 \multicolumn{1}{|c|}{\multirow{3}{*}{Group 5 (5)}} & Estimate & $-0.852$ & $-0.887$ & $-0.886$ & $-0.816$ & $-0.868$ & $-0.635$ \\ \cline{2-8}
 \multicolumn{1}{|c|}{} & ASE & $0.120$ & $0.111$ & $0.107$ & $0.102$ & $0.106$ & $0.080$ \\ \cline{2-8}
 \multicolumn{1}{|c|}{} & 95\% CI & $(-1.087, -0.616)$ & $(-1.105, -0.669)$ & $(-1.096, -0.677)$ & $(-1.016, -0.615)$ & $(-1.077, -0.660)$ & $(-0.791, -0.479)$ \\ \hline
 \multicolumn{1}{|c|}{\multirow{3}{*}{Group 6 (24)}} & Estimate & $-0.966$ & $-0.975$ & $-0.830$ & $-0.822$ & $-0.910$ & $-0.797$ \\ \cline{2-8}
 \multicolumn{1}{|c|}{} & ASE & $0.108$ & $0.101$ & $0.098$ & $0.134$ & $0.096$ & $0.086$ \\ \cline{2-8}
 \multicolumn{1}{|c|}{} & 95\% CI & $(-1.177, -0.755)$ & $(-1.172, -0.777)$ & $(-1.022, -0.638)$ & $(-1.084, -0.560)$ & $(-1.098, -0.722)$ & $(-0.965, -0.628)$ \\ \hline
\end{tabular}
\caption{Bias and Asymptotic Standard Errors of the Three ATT Estimators under time-invariant $\bg$ and time-varying $\bg_t$. 
The numbers in the parentheses show the number of donors in each group. We remark that the results of Group 6 under $\bg_t(\potY{t}{0})$ are the same as those in Table \ref{tab:data:ATT} of the main paper.}
\label{tab:supp:ATT}
\end{table}

Next, we conduct the conformal inference on the SPSC with ridge regularization. Following the main paper, we compare the proposed approach to the recent work by \citet{Cattaneo2021}. Of note, we increased \texttt{sims} parameter in \texttt{scpi} function from its default value of 200 to 2000. This adjustment was made to reduce variability in the results. Figure \ref{fig:supp:Conformal} visually summarizes the predicted treatment-free outcome $\widehat{Y}_{t}^{(0)}$ and the 95\% prediction intervals for all five donor pool groups. In general, two approaches produce similar $\widehat{Y}_{t}^{(0)}$ in terms of the shape. However, we find the average width of the prediction intervals over the post-treatment periods are significantly different, which is summarized in Table \ref{tab:supp:PIwidth}. Except for Groups 2 and 4, our approach produces narrower prediction intervals compared to the approach by \citet{Cattaneo2021}. These additional results from the conformal inference approaches certify the results in the main paper are robust to the choice of the donor pool.


\begin{table}[!htp]
\renewcommand{\arraystretch}{1.05} \centering
\footnotesize
\setlength{\tabcolsep}{5pt} 
\begin{tabular}{|cc|c|c|c|c|c|c|}
\hline
\multicolumn{2}{|c|}{Donor Pool}                                           & Group 1 (12) & Group 2 (12) & Group 3 (13) & Group 4 (12) & Group 5 (5) & Group 6 (24) \\ \hline
\multicolumn{2}{|c|}{SCPI}                                                 & 0.197        & 0.104        & 0.410        & 0.178        & 0.430       & 0.108        \\ \hline
\multicolumn{1}{|c|}{\multirow{2}{*}{SPSC-Ridge}} & Time-invariant $\bg$ & 0.054        & 0.108        & 0.149        & 0.205        & 0.119       & 0.046         \\ \cline{2-8} 
\multicolumn{1}{|c|}{}                            & Time-varying $\bg_t$   & 0.067        & 0.104        & 0.148        & 0.208        & 0.249       & 0.072         \\ \hline
\end{tabular}
\caption{Average Width of the 95\% Prediction Intervals over the Post-treatment Periods. The numbers in the parentheses show the number of donors in each group. The numbers in SCPI row show the average length of 95\% pointwise prediction intervals obtained from the approach proposed by \citet{Cattaneo2021} which is implemented in \texttt{scpi} R-package \citep{scpiPackage2023}. 
The numbers in SPSC-Ridge rows show the average length of 95\% pointwise prediction intervals obtained from the conformal inference approach in Section \ref{sec:Conformal} where $\bg_t$ is either time-invariant or time-varying. 
We remark that the results of Group 6 are the same as those in Figure \ref{fig:data:1} of the main paper.}
\label{tab:supp:PIwidth}

\end{table}



% Figure environment removed

\newpage

% Figure environment removed	


Lastly, we provide the details of the placebo study. Table \ref{tab:supp:ATT Placebo} shows the numerical summary of the analysis. We find that both placebo ATT estimators obtained from SPSC with and without ridge regularization suggest no effect across all donor specifications. Figure \ref{fig:supp:Conformal Placebo} visually shows the synthetic controls under the placebo treatment. Except for Group 1, we find 95\% prediction intervals for $\potY{t}{0}$ include the true treatment-free potential outcome $\potY{t}{0}$. These results suggest that our SPSC approach seems reasonable for analyzing the effect of the 1907 panic on the stock price of the two trust companies.

Based on these additional analyses, we can further strengthen the causal conclusions established in the main paper, i.e., the 1907 panic led to a decrease in the average log stock price of Knickerbocker and Trust Company of America. 


\newpage


\begin{table}[!htp]
\renewcommand{\arraystretch}{1.3} \centering
\scriptsize
\setlength{\tabcolsep}{3pt} 
\begin{tabular}{|c|c|c|c|c|}
\hline
\multirow{2}{*}{Donor Pool} & \multirow{2}{*}{Statistic} & \multicolumn{3}{c|}{Estimator} \\ \cline{3-5} 
& & \multicolumn{1}{c|}{OLS} & \multicolumn{1}{c|}{SPSC} & SPSC-Ridge \\ \hline


\multicolumn{1}{|c|}{\multirow{3}{*}{Group 1 (12)}} & Estimate & $-0.058$ & $-0.088$ & $-0.016$ \\ \cline{2-5}
\multicolumn{1}{|c|}{} & ASE & $0.025$ & $0.049$ & $0.013$ \\ \cline{2-5}
\multicolumn{1}{|c|}{} & 95\% CI & $(-0.107, -0.010)$ & $(-0.184, 0.009)$ & $(-0.042, 0.009)$ \\ \hline
\multicolumn{1}{|c|}{\multirow{3}{*}{Group 2 (12)}} & Estimate & $-0.023$ & $-0.015$ & $-0.001$ \\ \cline{2-5}
\multicolumn{1}{|c|}{} & ASE & $0.015$ & $0.023$ & $0.010$ \\ \cline{2-5}
\multicolumn{1}{|c|}{} & 95\% CI & $(-0.053, 0.006)$ & $(-0.061, 0.031)$ & $(-0.020, 0.018)$ \\ \hline
\multicolumn{1}{|c|}{\multirow{3}{*}{Group 3 (13)}} & Estimate & $-0.100$ & $-0.075$ & $0.002$ \\ \cline{2-5}
\multicolumn{1}{|c|}{} & ASE & $0.021$ & $0.044$ & $0.009$ \\ \cline{2-5}
\multicolumn{1}{|c|}{} & 95\% CI & $(-0.142, -0.058)$ & $(-0.160, 0.010)$ & $(-0.016, 0.020)$ \\ \hline
\multicolumn{1}{|c|}{\multirow{3}{*}{Group 4 (12)}} & Estimate & $-0.016$ & $0.028$ & $0.020$ \\ \cline{2-5}
\multicolumn{1}{|c|}{} & ASE & $0.014$ & $0.117$ & $0.012$ \\ \cline{2-5}
\multicolumn{1}{|c|}{} & 95\% CI & $(-0.044, 0.011)$ & $(-0.202, 0.258)$ & $(-0.003, 0.043)$ \\ \hline
\multicolumn{1}{|c|}{\multirow{3}{*}{Group 5 (5)}} & Estimate & $-0.002$ & $0.003$ & $0.018$ \\ \cline{2-5}
\multicolumn{1}{|c|}{} & ASE & $0.016$ & $0.021$ & $0.012$ \\ \cline{2-5}
\multicolumn{1}{|c|}{} & 95\% CI & $(-0.034, 0.029)$ & $(-0.037, 0.043)$ & $(-0.007, 0.042)$ \\ \hline
\multicolumn{1}{|c|}{\multirow{3}{*}{Group 6 (24)}} & Estimate & $-0.003$ & $0.030$ & $0.001$ \\ \cline{2-5}
\multicolumn{1}{|c|}{} & ASE & $0.009$ & $0.030$ & $0.005$ \\ \cline{2-5}
\multicolumn{1}{|c|}{} & 95\% CI & $(-0.020, 0.014)$ & $(-0.029, 0.089)$ & $(-0.009, 0.011)$ \\ \hline


\end{tabular}
\caption{Bias and Asymptotic Standard Errors of the Three ATT Estimators Under the Placebo Study. 
The numbers in the parentheses show the number of donors in each group. 
We remark that the results of Group 6 are the same as those in Section \ref{sec:Data} of the main paper.}
\label{tab:supp:ATT Placebo}
\end{table}
 
% Figure environment removed	

\newpage


\section{Nonparametric Single Proxy Synthetic Control Framework}		\label{sec:supp:nonparametric full}

\subsection{General Methodology} \label{sec:supp:nonparametric}

The SPSC framework can be generalized to the case in which the synthetic control is nonlinear and/or nonparametric, thus allowing the outcome to have arbitrary types such as binary, count, and continuous over a bounded interval. The nonparametric identification of the synthetic control relies on the existence of the bridge function satisfying the following condition.
\begin{assumption}[Existence of Nonparametric Bridge Function] \label{assumption:SC NP}
For all $t=1,\ldots,T$, there exists a function $h^*: \R^d \rightarrow \R$ that satisfies $\potY{t}{0} = \EXP \big\{
h^*(\bW_{\D t})
\cond
\potY{t}{0}
\big\}
=
0$ almost surely.
\end{assumption}
In words, there exists a function of donors $h^*$, possibly linear or nonlinear, of which conditional expectation given $\potY{t}{0}$ recovers $\potY{t}{0}$; the function $h^*$ is a kind of bridge functions \citep{TT2020_Intro, TTPR2023}, and we aptly refer to $h^*$ as the synthetic control bridge function in this paper. The synthetic control bridge function $h^*$ is a solution to a Fredholm integral equation of the first kind, and sufficient conditions for the existence of a solution are well studied in previous works such as \citet{Miao2018} and \citet{Cui2023}; see Section \ref{sec:supp:Exist h} of the Supplementary Material for details. We remark that Assumption \ref{assumption:SC} is a special case of Assumption \ref{assumption:SC NP} where the synthetic control bridge function is restricted to a linear form of $h( \bW_{\D t} ) = \bW_{\D t} \T \bgamma$. 

Similar to the results established under the linear synthetic control, the synthetic control bridge function $h^*$ can be used as a basis for identifying the ATT; the following Theorem formally establishes the result.	
\begin{theorem}	\label{thm:Extension NP}

Under Assumptions \ref{assumption:consistency}--\ref{assumption:valid proxy} and \ref{assumption:SC NP}, the synthetic control bridge function $h^*$ satisfies the following equation:
\begin{align}	\label{eq-Fredholm}
\EXP \big\{ Y_t - h^*(\bW_{\D t}) \cond Y_t \big\} = 0 \  \text{ almost surely} \ , \quad t=1,\ldots,T_0 \ .
\end{align}
Moreover, we have $
\EXP \big\{ \potY{t}{0} - h^*(\bW_{\D t}) \big\} = 0$ for any $t=1,\ldots,T$. 
Lastly, the ATT at time $t = T_0+1,\ldots,T$ is identified as $\tau_t^*
=
\EXP
\big\{
Y_t - h^*(\bW_{\D t}) 
\big\}$. 	 
\end{theorem}
Theorem \ref{thm:Extension NP} is a generalization of Theorems \ref{thm:SC} and \ref{thm:ATT} to nonparametric settings. If the synthetic control bridge function $h^*$ is unique, standard nonparametric or parametric estimation strategies can result in consistent estimators for $h^*$; the GMM estimator in Section \ref{sec:Estimation} is an example of the parametric estimation strategy. However, in general, the integral equation \eqref{eq-Fredholm} may have multiple solutions. Still, all solutions are valid synthetic controls and, consequently, result in the same ATT. When the bridge functions are not unique, we follow the approaches in \citet{Li2023} and \citet{Zhang2023} to obtain a nonparametric series estimator; see Section \ref{sec:supp:nonparametric estimation}. Given the synthetic control bridge function, the post-treatment residual process $Y_{t} - h^* (\bW_{\D t})$ is equivalent to the ATT plus the post-treatment error, i.e., $Y_{t} - h^*(\bW_{\D t}) = \tau_t^* + \epsilon_{t}$ where the ATT $\tau_t^*$ encodes the deterministic trend of the residual process via model $\tau_t^* = \tau(t \con \bbeta^*)$ and the error process $\epsilon_{t}$ satisfies the conditions in Assumption \ref{assumption:weakdep}. The estimation of the ATT parameter $\bbeta^*$ can be easily performed by following the results in the previous section with minor modifications.

Lastly, we extend the results when exogenous covariates are available, which are parallel to Section \ref{sec:Cov}.


\begin{assumption}[Existence of Bridge Function in the Presence of Covariates] \label{assumption:SC NP Cov} 
For all $t=1,\ldots,T$, there exists a function $h^*: \R^{d+q+dq} \rightarrow \R$ that satisfies $\potY{t}{0} = \EXP \big\{
h^*(\bW_{\D t}, \bX_{0t}, \bX_{\D t} )
\cond
\potY{t}{0}, \bX_{0t}, \bX_{\D t}
\big\}$ almost surely.
\end{assumption}	 

\begin{theorem}	\label{thm:Extension NP Cov}

Suppose Assumptions \ref{assumption:consistency}, \ref{assumption:noitf}, \ref{assumption:valid proxy Cov}, and \ref{assumption:SC NP Cov} are satisfied. Then, the synthetic control bridge function $h^*$ satisfies $\EXP \big\{ Y_t - h^*(\bW_{\D t}, \bX_{0t}, \bX_{\D t}) \cond Y_t, \bX_{0t}, \bX_{\D t} \big\} = 0$ for $t=1,\ldots,T_0$. 	
Moreover, we have $
\EXP \big\{ \potY{t}{0} - h^*(\bW_{\D t}, \bX_{0t}, \bX_{\D t}) \big\} = 0$ for any $t=1,\ldots,T$. 
Lastly, the ATT at time $t = T_0+1,\ldots,T$ is identified as $\tau_t^*
=
\EXP
\big\{
Y_t - h^*(\bW_{\D t}, \bX_{0t}, \bX_{\D t} ) 
\big\}$. 	 
\end{theorem}

The estimation of inference of the synthetic control bridge function and the ATT is analogous to that established in the absence of covariates, so we omit the details here.


\subsection{Sufficient Conditions for the Existence of the Synthetic Control Bridge Function} \label{sec:supp:Exist h}

In this Section, we provide sufficient conditions for the existence of the synthetic control bridge function $h^*$. In brief, we follow the approach in \citet{Miao2018}. The proof relies on Theorem 15.18 of \citet{Kress2014}, which is stated below for completeness.\\[0.25cm]
\noindent
\textbf{Theorem 15.18.} \citep{Kress2014}
Let $A:X \rightarrow Y$ be a compact operator with singular system $\big\{ \mu_n,\phi_n,g_n \big\}_{n=1,2,\ldots}$. The integral equation of the first kind $A\phi = f$ is solvable if and only if 
\begin{align*}
 & 1. \quad
 \text{$f \in \mathcal{N}(A^{\text{adjoint}})^\perp = \big\{ f \, \big| \, A^{\text{adjoint}}(f) = 0 \big\}^{\perp}$}
 \ , 
 &&
 2. \quad
 \text{$\sum_{n=1}^{\infty} \mu_n^{-2} \big| \langle f,g_n \rangle |^2 < \infty$}
\end{align*}


To apply the Theorem, we introduce some additional notations. Let $\mathcal{L}_{W}$ and $\mathcal{L}_{ \potY{}{0} }$ be the spaces of square-integrable functions of $\bW_{\D t}$ and $\potY{t}{0}$, respectively, which are equipped with the inner products $\langle h_1, h_2 \rangle_{W} = \int h_1(\bw) h_2(\bw) \, f_W (\bw) \, d\bw = \EXP \big\{ h_1(\bW_{\D t}) h_2(\bW_{\D t}) \big\} $ and $\langle g_1, g_2 \rangle_{\potY{}{0}} = \int g_1(y) g_2(y) \, f_{\potY{}{0}} (y) \, dy = \EXP \big\{ g_1(\potY{t}{0}) g_2(\potY{t}{0}) \big\} $, respectively. Let $\mathcal{K}: \mathcal{L}_{W} \rightarrow \mathcal{L}_{ \potY{}{0} }$ be the conditional expectation of $h(\bW_{\D t} ) \in \mathcal{L}_{W}$ given $\potY{t}{0}$, i.e.,
\begin{align*}
\mathcal{K} (h) \in \mathcal{L}_{\potY{}{0}} 
\text{ satisfying }
\big( \mathcal{K}(h) \big) (y) 
=
\EXP \big\{ h(\bW_{\D t}) \cond \potY{t}{0}=y \big\}
\text{ for } h \in \mathcal{L}_{W}
\end{align*}
Then, the synthetic control bridge function $h^* \in \mathcal{L}_{W}$ solves $\mathcal{K} ( h^* ) = [\text{identity map}] \in \mathcal{L}_{\potY{}{t}} $, i.e., 
\begin{align*}
\int h^*(\bw) f_{W | \potY{}{0} } (\bw \cond y ) \, d\bw = y , \ \forall y
\end{align*}


Now, we assume the following conditions:
\begin{itemize}[leftmargin=0.4cm, itemsep=0cm]
\item[] \HT{NPSC-1} The variables $(\potY{t}{0} , \bW_{\D t})$ are stationary; 
\item[] \HT{NPSC-2} $\iint
f_{W | \potY{}{0} } (\bw \cond y )
f_{\potY{}{0} | W } (y \cond \bw )
\, d\bw \, d y
< \infty$;
\item[] \HT{NPSC-3} For $g \in \mathcal{L}_{\potY{}{0}}$, $\EXP \big\{ g(\potY{t}{0}) \cond \bW_{\D t} \big\} = 0$ implies $g(\potY{t}{0})= 0$ almost surely;
\item[] \HT{NPSC-4} $\EXP \big[ \big\{ \potY{t}{0} \big\}^2 \big] < \infty$;
\item[] \HT{NPSC-5} Let the singular system of $\mathcal{K}$ be $\big\{ \mu_n,\phi_n,g_n \big\}_{n=1,2,\ldots}$. Then, we have $\sum_{n=1}^{\infty} \mu_n^{-2} \big| \langle \potY{t}{0} ,g_n \rangle |^2 < \infty$.
\end{itemize}

We remark that the expectation can be defined without using $t$ under Condition \HL{NPSC-1}. First, we show that $\mathcal{K}$ is a compact operator under Condition \HL{NPSC-2}. Let $\mathcal{K}^{\text{adjoint}} : \mathcal{L}_{\potY{}{0}} \rightarrow \mathcal{L}_{W}$ be the conditional expectation of $g(\potY{t}{0}) \in \mathcal{L}_{\potY{}{0}}$ given $\bW_{\D t}$, i.e., 
\begin{align*}
\mathcal{K}^{\text{adjoint}} (g) \in \mathcal{L}_{W} 
\text{ satisfying }
\big( \mathcal{K}(g) \big) (\bw) 
=
\EXP \big\{ g(\potY{t}{0}) \cond \bW_{\D t}=\bw \big\}
\text{ for } g \in \mathcal{L}_{\potY{}{0}}
\end{align*}
Then, $\mathcal{K}$ and $\mathcal{K}^{\text{adjoint}}$ are the adjoint operator of each other as follows:
\begin{align*}
\langle \mathcal{K}(h) , g \rangle_{\potY{}{0}}
& =
\EXP
\big[
\EXP \big\{ h(\bW_{\D t}) \cond \potY{t}{0} \big\}
g(\potY{t}{0})
\big]	
\\
& 
=
\EXP
\big[
h(\bW_{\D t}) g(\potY{t}{0})
\big]	
\\
&
=
\EXP
\big[
h(\bW_{\D t}) \EXP \big\{ g(\potY{t}{0}) \cond \bW_{\D t} \big\}
\big]	
=
\langle h , \mathcal{K}^{\text{adjoint}}(g) \rangle_{W}
\end{align*}
Additionally, as shown in page 5659 of \citet{Carrasco2007}, $\mathcal{K}$ and $\mathcal{K}^{\text{adjoint}}$ are compact operators under Condition \HL{NPSC-1}. Moreover, by Theorem 15.16 of \citet{Kress2014}, there exists a singular value decomposition of $\mathcal{K}$ as $\big\{ \mu_n,\phi_n,g_n \big\}_{n=1,2,\ldots}$. 

Second, we show that $\mathcal{N}(\mathcal{K}^{\text{adjoint}})^\perp = \mathcal{L}_{\potY{}{0}}$, which suffices to show $\mathcal{N}(\mathcal{K}^{\text{adjoint}}) = \big\{ 0 \big\} \subseteq \mathcal{L}_{\potY{}{0}}$. Under Condition \HL{NPSC-3}, we have 
\begin{align*}
g \in \mathcal{N}(\mathcal{K}^{\text{adjoint}})
\quad 
\Rightarrow 
\quad 
\EXP \big\{ g (\potY{t}{0}) \cond \bW_{\D t} = \bw \big\}
=
0, \ \forall \bw
\quad \Rightarrow
\quad
g(\potY{t}{0}) = 0 
\end{align*}
where the first arrow is from the definition of the null space $\mathcal{N}$, and the second arrow is from Condition \HL{NPSC-3}. Therefore, any $g \in \mathcal{N}(\mathcal{K}^{\text{adjoint}})$ must satisfy $g(y) = 0 $ almost surely, i.e., $\mathcal{N}(\mathcal{K}^{\text{adjoint}})= \big\{ 0 \big\} \subseteq \mathcal{L}_{\potY{}{0}}$ almost surely. 

Third, from the definition of $\mathcal{L}_{W}$, $g(\potY{t}{0}) = \potY{t}{0} \in \mathcal{L}_{\potY{}{0}} = \mathcal{N}(\mathcal{K}^{\text{adjoint}})^\perp $ under Condition \HL{NPSC-4}. 


Combining the three results, we establish that $\potY{t}{0}$ satisfies the first condition of Theorem 15.18 of \citet{Kress2014}. The second condition of the Theorem is exactly the same as Condition \HL{NPSC-5}. Therefore, we establish that the Fredholm integral equation of the first kind $\mathcal{K} ( h ) = [\text{identity map}] $ is solvable under Conditions \HL{NPSC-1}-\HL{NPSC-5}.

\subsection{Uniqueness of Synthetic Control Bridge Function Under Completeness}		\label{sec:supp:NP Unique}

In Section \ref{sec:supp:nonparametric}, we showed that \eqref{eq-Fredholm} is satisfied under Assumptions \ref{assumption:consistency}--\ref{assumption:valid proxy} and \ref{assumption:SC NP}; for readability, we restate Assumption \ref{assumption:SC NP} and \eqref{eq-Fredholm} below:\\[0.1cm] 
\textit{Assumption \ref{assumption:SC NP} (Existence of Nonparametric Bridge Function)
For all $t=1,\ldots,T$, there exists a function $h^*: \R^d \rightarrow \R$ that satisfies $\potY{t}{0} = \EXP \big\{
h^*(\bW_{\D t})
\cond
\potY{t}{0}
\big\}
=
0$ almost surely.}\\[0.1cm]
and
\begin{align}
\EXP \big\{ Y_t - h^*(\bW_{\D t}) \cond Y_t \big\} = 0 \ \text{ almost surely} \ , \quad t=1,\ldots,T_0 \ .
\tag{\ref{eq-Fredholm}}
\end{align}

In this section, we show the reverse is satisfied. 
Suppose a function $h^*$ satisfies \eqref{eq-Fredholm}. Then, we obtain the following result for $t=1,\ldots,T_0$:
\begin{align*}
y 
& =
\EXP \big\{ h^* ( \bW_{\D t} ) \cond Y_t=y \big\} 
=
\EXP \big\{ h^* (\bW_{\D t}) \cond \potY{t}{0} = y \big\} \ ,
\end{align*} 
where the second equality holds from Assumption \ref{assumption:consistency}. 
Therefore, $h^*$ satisfies Assumption \ref{assumption:SC NP}. 

We can further show that the bridge function $h^*$ is indeed unique under an additional assumption, namely the completeness assumption:
\begin{assumption}[Completeness] \label{assumption-complete}
For $t=1,\ldots,T_0$, suppose $\EXP \big\{ q(\bW_{\D t}) \cond Y_t \big\} = 0$ almost surely for a square integrable function $q$. Then, $q(\bW_{\D t} ) = 0$ almost surely. 	
\end{assumption}
The assumption states that $Y_t$ should be $\bW_{\D t}$-relevant over the pre-treatment periods in the sense that any variation in $\bW_{\D t}$ is captured by variation in $Y_t$ for the pre-treatment periods. 

Under Assumptions \ref{assumption:consistency}--\ref{assumption:valid proxy}, \ref{assumption:SC NP}, and \ref{assumption-complete}, the solution to \eqref{eq-Fredholm} is unique almost surely. To show this, let $h_1^*$ and $h_2^*$ be synthetic control bridge functions that satisfy \eqref{eq-Fredholm}. We then find $\EXP \big\{ h_1^*(\bW_{\D t}) - h_2^*(\bW_{\D t}) \cond Y_t \big\} = 0$ for $t=1,\ldots,T_0$, implying $h_1^*(\bW_{\D t})$ and $h_2^*(\bW_{\D t}) = 0$ almost surely from Assumption \ref{assumption-complete}. Therefore, the solution to \eqref{eq-Fredholm} must be unique almost surely, i.e., $ h_1^*(\bW_{\D t}) = h_2^*(\bW_{\D t})$ almost surely. Moreover, suppose $h_3^*$ be a synthetic control bridge function that satisfies Assumption \ref{assumption:SC NP}. Then, $h_3^*$ is a solution to \eqref{eq-Fredholm}, indicating that $ h_1^*(\bW_{\D t}) = h_3^*(\bW_{\D t})$ almost surely because of the same reasoning. Therefore, the unique solution to \eqref{eq-Fredholm} is also the unique function satisfying Assumption \ref{assumption:SC NP}. 

We remark that Assumption \ref{assumption-complete} may not be satisfied if the cardinality of the support of $\bW_{\D t}$ is strictly larger than that of $Y_t$. For instance, suppose that the outcomes are binary and two donors are available, i.e., $\bW_{\D t} \in \{0,1\}^2$ and $Y_t \in \{0,1\}$. Then, the equation in Assumption \ref{assumption-complete} reduces to
\begin{align} \label{eq-underdetermine}
    \begin{bmatrix}
        p_{W|Y}( 0,0 \cond 0)
        &
        p_{W|Y}( 0,1 \cond 0)
        &
        p_{W|Y}( 1,0 \cond 0)
        &
        p_{W|Y}( 1,1 \cond 0)
        \\
        p_{W|Y}( 0,0 \cond 1)
        &
        p_{W|Y}( 0,1 \cond 1)
        &
        p_{W|Y}( 1,0 \cond 1)
        &
        p_{W|Y}( 1,1 \cond 1)
    \end{bmatrix}
    \begin{bmatrix}
        q(0,0) \\ q(0,1) \\ q(1,0) \\ q(1,1) 
    \end{bmatrix}
    =
    \begin{bmatrix}
        0 \\ 0
    \end{bmatrix}
\end{align}
where $p_{W|Y}(a,b \cond y) = \Pr\{ \bW_{\D t}=(a,b) \cond \potY{t}{0} = y \}$. Since \eqref{eq-underdetermine} is an underdetermined system, there are multiple non-zero $q$ functions satisfying \eqref{eq-underdetermine}, indicating that Assumption \ref{assumption-complete} cannot be satisfied. 

In the following section, we introduce a nonparametric SPSC framework that accommodates non-unique synthetic control bridge functions.










\subsection{Single Proxy Synthetic Control Approach without the Uniqueness Assumption}		\label{sec:supp:nonparametric estimation}




The synthetic control bridge function $h$ is defined as a function satisfying \eqref{eq-Fredholm}; we restate the equation below for readability.
\begin{align}		\label{eq-Fredholm2}
\EXP \big\{ Y_t - h^*(\bW_{\D t}) \cond Y_t \big\} = 0 \text{ almost surely} \ , \quad t=1,\ldots,T_0 \ .
\tag{\ref{eq-Fredholm}}
\end{align}
We consider the case where there are multiple synthetic control bridge functions $h$ satisfying \eqref{eq-Fredholm2}. Even so, the identification of the ATT established in Theorem \ref{thm:Extension NP} is satisfied regardless of the choice of the bridge function. However, estimation and inference of the ATT can be complicated in the presence of multiple synthetic control bridge functions. To resolve this issue, we use approaches proposed by a series of recent works \citep{Li2023, Zhang2023}. In brief, their approaches involve the following three stages. In the first stage, we estimate a set of synthetic control bridge functions based on a sieve estimator; see Stage 1 below. In the second stage, we define a criterion function, denoted by $M$, and focus on the estimation of the minimizer of $M$, denoted by $h_0$. Then, an estimator of the ATT can be constructed based on the estimator of $h_0$; see Stage 2 below. In the third stage, we consider a de-biasing procedure for the estimator obtained in the previous stage to attain the asymptotic normality; see Stage 3 below. The following sections present details under general nonparametric settings, but the method can be applied to the parametric synthetic controls, including cases where there are multiple synthetic control weights that satisfy Assumption \ref{assumption:SC}. We have included only the essential assumptions and notations in this work to ensure clarity. We refer the readers to \citet{Li2023} and \citet{Zhang2023} for additional details.\\[0.25cm]

\noindent \textbf{Stage 1}: Estimation of the Solution Set $\mathcal{H}_0$ \\

Let $\mathcal{H}$ be a collection of user-specified smooth functions, and let $\mathcal{H}_0$ be the collection of the solutions of \eqref{eq-Fredholm2}, i.e.,
\begin{align*}
\mathcal{H}_0
=
\Big\{ h \in \mathcal{H} \, \Big| \,
Y_t = \EXP \big\{ h(\bW_{\D t}) \cond Y_t \big\}, \ t=1,\ldots,T_0
\Big\}
\end{align*}
Alternatively, we can represent $\mathcal{H}_0$ using a criterion function. Let $\mathfrak{C}: \mathcal{H} \rightarrow \R$ be a criterion function having the following form:
\begin{align*}
\mathfrak{C} (h)
=
\EXP
\Big[
\big[
Y_t
-
\EXP \big\{ h(\bW_{\D t}) \cond Y_t \big \}
\big]^2
\Big] 
\ , \ t=1,\ldots,T_0
\ .
\end{align*}
It is straightforward to check that $\mathcal{H}_0 = \big\{ h \in \mathcal{H} \cond \mathfrak{C}(h) = 0 \big\}$.

We consider a sieve approach as follows. First, we choose a sequence of approximating bases functions of $\bW_{\D t}$, denoted by $\big\{ \varphi_k(\bw) \big\}_{k=1,2,\ldots}$. For this sequence, we define an approximating function space for $\mathcal{H}$ by using the first $k_T$ bases functions, i.e.,
\begin{align*}
\mathcal{H}_T
=
\bigg\{
h \in \mathcal{H}
\, \bigg| \,
h(\bw)
=
\sum_{\ell=1}^{k_T}
b_{\ell} \varphi_{\ell}(\bw) 
\bigg\} \ ,
\end{align*}
where $k_T$ is a known parameter and $b_1,\ldots,b_{k_T}$ are unknown scalar parameters.

A sample analogue of the criterion function $\mathfrak{C}$, denoted by $\mathfrak{C}_T$, can be obtained based on the sieve approach. We choose a sequence of approximating bases functions of $\potY{t}{0}$, denoted by $ \big\{ \phi_k (y) \big\}_{k=1,2,\ldots}$. Then, we choose the first $k_T$ bases function and construct a $k_T$-dimensional function of $y$, denoted by $\bphi (y) = \big\{ \phi_1(y),\ldots,\phi_{k_T}(y) \big\}\T$. Using the pre-treatment observations, we construct a $(T_0 \times k_T)$ matrix as follows:
\begin{align*}
\Phi_{\pre}
=
\begin{bmatrix}
\bphi \T (Y_1)
\\
\vdots
\\
\bphi \T (Y_{T_0})
\end{bmatrix} 
\in \R^{T_0 \times k_T}
\ .
\end{align*}
For a given function $h$, a sieve estimator of the conditional expectation $\EXP \big\{ h(\bW_{\D t}) \cond Y_t \big\}$ for $t=1,\ldots,T_0$ can be obtained by regressing $h(\bW_{\D t})$ on $ \bphi(Y_t) $, i.e., 
\begin{align*}
\widehat{\mu}_{\pre} (y \con h)
=
\texttt{sieve}
\Big(
\EXP \big\{ h(\bW_{\D t}) \cond Y_t = y \big\}
\Big)
=
\bphi \T (y)
\big( \Phi_{\pre} \T \Phi_{\pre} \big)^{-1}
\bigg\{
\sum_{t=1}^{T_0}
h(\bW_{\D t}) \bphi(Y_t)
\bigg\} \ .
\end{align*}
Therefore, $\mathfrak{C}_T$ can be obtained based on a sieve estimator, i.e.,
\begin{align*}
\mathfrak{C}_T (h)
=
\frac{1}{T_0} \sum_{t=1}^{T_0} 
\Big\{
Y_t
-
\widehat{\mu}_{\pre} (Y_t \con h)
\Big\}^2
\end{align*}
The proposed estimator of $\mathcal{H}_0$ is 
\begin{align*}
\widehat{\mathcal{H}}_0 = \Big\{ h \in \mathcal{H}_T \Cond \mathfrak{C}_T(h) \leq c_T \Big\} 
\end{align*}
where $c_T$ is an appropriately chosen sequence with $c_T \rightarrow 0$ as $T \rightarrow \infty$. Under regularity conditions, we have
\begin{align*}
d_{H} \big(\widehat{\mathcal{H}}_0, {\mathcal{H}}_0 , \big\| \cdot \big\|_{\infty} \big) = o_P(1)
\end{align*}
where $d_{H} (\mathcal{H}_1,\mathcal{H}_2, \big\| \cdot \big\| )$ is the Hausdorff distance between $\mathcal{H}_1$ and $\mathcal{H}_2$ with respect to a given norm $\big\| \cdot \big\|$; see Section 3.2 of \citet{Li2023} and Section 3.2 of \citet{Zhang2023} for details. \\



\noindent \textbf{Stage 2}: A Representer-based Estimator \\

After obtaining a consistent set estimator of $\mathcal{H}_0$ (i.e., $\widehat{\mathcal{H}}_0$), we select an estimator of $h$ from $\widehat{\mathcal{H}}_0$ so that it converges to a unique element in $\mathcal{H}_0$. Specifically, we define a function $M : \mathcal{H} \rightarrow \R$ that has a unique minimum $h_0$ on $\mathcal{H}_0$. Let $M_T$ be its sample analogue, and let $\widehat{h}_0$ be the minimum of $M_T(h)$ over $\widehat{\mathcal{H}}_0$, i.e.,
\begin{align*}
\widehat{h}_0 \in \argmin_{h \in \widehat{\mathcal{H}}_0} M_T(h) \ . 
\end{align*}
To obtain a unique minimum $\widehat{h}_0$, $\mathcal{H}$ and $M$ are chosen to satisfy the following assumption:
\begin{assumption} The following conditions are satisfied:
\begin{itemize}[leftmargin=0.5cm, itemsep=0cm]
\item[1.] The set $\mathcal{H}$ is convex;
\item[2.] The functional $M: \mathcal{H} \rightarrow \R$ is strictly convex, and have a unique minimum at $h_0$ on $\mathcal{H}_0$;
\item[3.] The sample analogue $M_T : \mathcal{H} \rightarrow \R$ is continuous and $\sup_{h \in \mathcal{H}} \big| M_T(h) - M(h) \big| = o_P(1)$.
\end{itemize}
\end{assumption}
Possible choices for $M$ and its sample analogue $M_T$ are 
\begin{align*}
&
M(h)
=
\EXP \big[ \big\{ h(\bW_{\D t} ) \big\}^2 \big]
\ , \ t=1,\ldots,T_0
\ , 
&&
M_T (h)
=
\frac{1}{T_0}
\sum_{t=1}^{T_0}
\big\{ h(\bW_{\D t} ) \big\}^2
\end{align*}
Under regularity conditions, we have $\big\| \widehat{h}_0 - h_0 \big\|_{\infty} = o_P(1)$; see Theorem 3 of \citet{Li2023} and Proposition 3.2 of \citet{Zhang2023} for details. In turn, we obtain an estimator of the ATT as $\widehat{\tau}_t = Y_t - \widehat{h}_0(\bW_{\D t})$ for $t=T_0+1,\ldots,T$ where inference based on $\widehat{\tau}_t$ can be established by the conformal inference in Section \ref{sec:Conformal} in the Supplementary Material. Alternatively, we may posit a parametric form for the ATT as $\tau_t = \tau(t \con \bbeta)$. Considering $\widehat{h}_0$ as a fixed function, an estimator of $\bbeta$ can be obtained as a solution to the following equation:
\begin{align}	
&
\widehat{\bbeta} \text{ solves }
\frac{1}{T_1}
\sum_{t=T_0+1}^{T}
{\Psi}_{\post} (\bO_t \con \bbeta, \widehat{h}_0)
= 0
\ , 
\label{eq-betahat}
\\
&
{\Psi}_{\post} (\bO_t \con \bbeta, h)
=
\frac{\partial \tau(t \con \bbeta) }{\partial \bbeta}
\Big\{ Y_t - \tau(t \con {\bbeta}) - h (\bW_{\D t} ) \Big\}
\in \R^{\text{dim}(\bbeta)}
\ , \ 
t=T_0+1,\ldots,T
\ .
\label{eq-supp-postEE}
\end{align}
To characterize the asymptotic property of $\widehat{\bbeta}$, we additionally define the following objects. Let $\langle h_1, h_2 \rangle_w$ be
\begin{align*}
\langle h_1, h_2 \rangle_w
=
\EXP \big[
\EXP \big\{ h_1(\bW_{\D t}) \cond \potY{t}{0} \big\}
\EXP \big\{ h_2(\bW_{\D t}) \cond \potY{t}{0} \big\} 
\big] \ , \
t=T_0+1,\ldots,T \ ,
\end{align*}
and $\overline{\mathcal{H}}$ be the closure of the linear span of $\mathcal{H}$ under $\big\| \cdot \big\|_w$. Then, we assume the following conditions.
\begin{assumption} 	\label{assumption:supp:g}
The following conditions are satisfied:
\begin{itemize}[leftmargin=0.5cm, itemsep=0cm]
\item[1.] For any $h \in \overline{\mathcal{H}}$, there exists a function $g_{0,h} \in \mathcal{H}$ satisfying $\langle g_{0,h}, h \rangle_w = \EXP \big\{ h(\bW_{\D t}) \big\}$ for $t=T_0+1,\ldots,T$. 
\item[2.] There exists a projection of $\mathcal{H}$ on $\mathcal{H}_T$, denoted by $\Pi_{T}: \mathcal{H} \rightarrow \mathcal{H}_T$, which satisfies
\begin{align*}
\sup_{h \in \mathcal{H}} \big\| h - \Pi_T h \big\| = O(\eta_T) \ .
\end{align*}
where $\eta_T=o(1)$ satisfies regularity conditions; see Assumptions 7-10 of \citet{Li2023} and Assumptions 4-7 of \citet{Zhang2023} for details.
\end{itemize}

\end{assumption}
We now characterize the asymptotic representation of $T_1^{1/2}
\big( \widehat{\bbeta} - \bbeta^* \big)$ under regularity conditions including stationarity and independent errors. Applying a first-order Taylor expansion, we find
\begin{align*}
0 
&
= 
\frac{1}{T_1} \sum_{t=T_0+1}^{T} {\Psi}_{\post} (\bO_t \con \widehat{\bbeta}, \widehat{h}_0)
\\
&
=
\frac{1}{T_1} \sum_{t=T_0+1}^{T}
\Bigg\{ {\Psi}_{\post} (\bO_t \con \bbeta^*, \widehat{h}_0)
+
\frac{\partial {\Psi}_{\post} (\bO_t \con \bbeta, \widehat{h}_0) }{\partial \bbeta\T} \bigg|_{\bbeta=\bbeta^*}
\cdot \big( \widehat{\bbeta} - \bbeta^* \big)
\Bigg\}
+
o_P(1) \ .
\end{align*}
Therefore, we find that \eqref{eq-betahat} has the following asymptotic representation for $t=T_0+1,\ldots,T$:
\begin{align}
&
\sqrt{T_1} 
\big( \widehat{\bbeta} - \bbeta^* \big)
\nonumber
\\
& 
= 
\bigg[ 
\underbrace{
\frac{1}{T_1} \sum_{t=T_0+1}^{T} 
\frac{\partial {\Psi}_{\post} (\bO_t \con \bbeta, \widehat{h}_0) }{\partial \bbeta\T} \bigg|_{\bbeta=\bbeta^*} 
}_{=: V (\bbeta^*, \widehat{h}_0) }
\bigg]^{-1}
\nonumber 
\\
& \hspace*{1cm} \times 
\bigg[ \frac{1}{\sqrt{ T_1} } \sum_{t=T_0+1}^{T} 
\frac{\partial \tau(t \con \bbeta^*)}{\partial \bbeta}
\Big\{
Y_t - \tau(t \con {\bbeta}^*) - \widehat{h}_0(\bW_{\D t} )
\Big\}
\bigg]
+
o_P(1)
\nonumber
\\
& 
=
V^{-1} (\bbeta^*, \widehat{h}_0)
\bigg[ \frac{1}{\sqrt{ T_1} } \sum_{t=T_0+1}^{T} 
\frac{\partial \tau(t \con \bbeta^*)}{\partial \bbeta}
\Big\{
Y_t 
- \tau(t \con \bbeta^*)
- \widehat{h}_0(\bW_{\D t} )
\Big\}
\bigg]
+
o_P(1)
\nonumber
\\
& 
=
V^{-1} (\bbeta^*, \widehat{h}_0)
\bigg[ \frac{1}{\sqrt{ T_1} } \sum_{t=T_0+1}^{T} 
\frac{\partial \tau(t \con \bbeta^*)}{\partial \bbeta}
\Big\{
Y_t 
- \tau(t \con \bbeta^*)
- h_0(\bW_{\D t}) 
\Big\}
\bigg]
\label{eq-supp-Asymp1}
\\
&
\quad +
V^{-1} (\bbeta^*, \widehat{h}_0)
\bigg[ \frac{1}{\sqrt{ T_1} } \sum_{t=T_0+1}^{T} 
\frac{\partial \tau(t \con \bbeta^*)}{\partial \bbeta}
\Big[
\EXP \big\{ h_0(\bW_{\D t}) - \widehat{h}_0(\bW_{\D t} ) \big\} 
\Big]
\bigg]
\label{eq-supp-Asymp2}
\\
&
\quad +
V^{-1} (\bbeta^*, \widehat{h}_0)
\Bigg[ \frac{1}{\sqrt{ T_1} } \sum_{t=T_0+1}^{T} 
\frac{\partial \tau(t \con \bbeta^*)}{\partial \bbeta}
\Bigg[
\begin{array}{l}
\big\{ h_0(\bW_{\D t})
- \widehat{h}_0(\bW_{\D t} )
\big\}
\\
- \EXP \big\{ h_0(\bW_{\D t}) - \widehat{h}_0(\bW_{\D t} ) \big\}
\end{array} 
\Bigg]
\Bigg]
\label{eq-supp-Asymp3}
\\
&
\quad
+
o_P(1) 
\nonumber
\ .
\end{align}
Following Theorem 4 of \citet{Li2023} and Supplementary Material of \citet{Zhang2023}, we establish that \eqref{eq-supp-Asymp3} is $o_P(1)$. In addition, for $t=T_0+1,\ldots,T$, the numerator of \eqref{eq-supp-Asymp2} is equal to
\begin{align}
& 
\frac{1}{\sqrt{ T_1} } \sum_{t=T_0+1}^{T} 
\frac{\partial \tau(t \con \bbeta^*)}{\partial \bbeta}
\EXP \big\{ h_0(\bW_{\D t}) - \widehat{h}_0(\bW_{\D t} ) \big\} 
\nonumber
\\
&=
-
\frac{1}{\sqrt{ T_1} } \sum_{t=T_0+1}^{T} 
\frac{\partial \tau(t \con \bbeta^*)}{\partial \bbeta}
\EXP \big\{ g_{0,h_0} (\bW_{\D t}) \cond \potY{t}{0} \big\} 
\big\{ \potY{t}{0} - h_0(\bW_{\D t} ) \big\}
\nonumber
\\
&
\quad
+
\frac{1}{\sqrt{ T_1} } \sum_{t=T_0+1}^{T} 
\frac{\partial \tau(t \con \bbeta^*)}{\partial \bbeta}
\widehat{\EXP} \big\{ \Pi_T g_{0,h_0} (\bW_{\D t}) \cond \potY{t}{0} \big\} 
\big[ \potY{t}{0} - \widehat{\EXP} \big\{ \widehat{h}_0(\bW_{\D t}) \cond \potY{t}{0} \big\} \big]
+
o_P(1)	
\label{eq-asymptotic rep}
\ .
\end{align}
Here, $g_{0,h}$ and its projection $\Pi_T g_{0,h}$ are chosen to satisfy Assumption \ref{assumption:supp:g}, and $\widehat{\EXP}$ is a generic estimator of the conditional expectation operator of the distribution $\bW_{\D t} | \potY{t}{0}$ having a fast convergence rate; see Stage 3 below for details on how these estimators are constructed. Combining all results, we have the following result for $t=T_0+1,\ldots,T$:
\begin{align*}
&
\sqrt{T_1} 
\big( \widehat{\bbeta} - \bbeta^* \big)
\\
&
=
V^{-1} (\bbeta^*, \widehat{h}_0)
\bigg[ \frac{1}{\sqrt{ T_1} } \sum_{t=T_0+1}^{T} 
\frac{\partial \tau(t \con \bbeta^*)}{\partial \bbeta}
\left[ 
\begin{array}{l}
Y_t 
- \tau(t \con \bbeta^*)
- h_0(\bW_{\D t}) 
\\
-
\EXP \big\{ g_{0,h_0} (\bW_{\D t}) \cond \potY{t}{0} \big\}
\big\{ \potY{t}{0} - h_0(\bW_{\D t} ) \big\} 
\end{array}
\right]
\bigg]
\\
&
\quad
+
V^{-1} (\bbeta^*, \widehat{h}_0)
\sqrt{T_1}
r_T(\widehat{h}_0)
+
o_P(1)
\end{align*}
where
\begin{align*}
r_T(\widehat{h}_0)
= 
\frac{1}{ T_1 } \sum_{t=T_0+1}^{T} 
\frac{\partial \tau(t \con \bbeta^*)}{\partial \bbeta}
\widehat{\EXP} \big\{ \Pi_T g_{0,h_0} (\bW_{\D t}) \cond \potY{t}{0} \big\} 
\big[ \potY{t}{0} - \widehat{\EXP} \big\{ \widehat{h}_0(\bW_{\D t}) \cond \potY{t}{0} \big\} \big] \ .
\end{align*}

\noindent\textbf{Stage 3}: A De-biased Estimator\\

To obtain the asymptotic normality of $\widehat{\bbeta}$, we need to de-bias $\widehat{\bbeta}$ by subtracting an estimated value of $r_T(\widehat{h}_0)$. To do so, we define a new criterion function and its sample analogue for $h \in \mathcal{H}$ as follows:
\begin{align*}
&
\mathcal{R} (h)
=
\EXP \Big[ \big[ \EXP \big\{ h(\bW_{\D t}) \cond \potY{t}{0} \big\} \big]^2 \Big]
-
2 \EXP \big\{ h(\bW_{\D t}) \big\}
\ , \ 
t=T_0+1,\ldots,T \ ,
\\
&
\mathcal{R}_T (h)
=
\frac{1}{T_1} \sum_{t=T_0+1}^{T}
\big[ \widehat{\EXP} \big\{ h(\bW_{\D t}) \cond \potY{t}{0} \big\} \big]^2
-
\frac{2}{T_1} \sum_{t=T_0+1}^{T} h(\bW_{\D t}) \ .
\end{align*}
We obtain an estimator of $\Pi_T g_{0,h_0}$, denoted by $\widehat{g}$, as 
\begin{align*}
\widehat{g} \in \argmin_{\widehat{h}_0 \in \mathcal{H} } \mathcal{R}_T(\widehat{h}_0)
\end{align*}
and the resulting estimator of $r_T(\widehat{h}_0)$ is
\begin{align*}
\widehat{r}_T^{\text{inf}}(\widehat{h}_0)
= 
\frac{1}{ T_1 } \sum_{t=T_0+1}^{T} 
\frac{\partial \tau(t \con \widehat{\bbeta})}{\partial \bbeta}
\widehat{\EXP} \big\{ \widehat{g} (\bW_{\D t}) \cond \potY{t}{0} \big\} 
\big[ \potY{t}{0} - \widehat{\EXP} \big\{ \widehat{h}_0(\bW_{\D t}) \cond \potY{t}{0} \big\} \big] \ , \ 
t=T_0+1,\ldots,T \ .
\end{align*}
Unfortunately, the above estimator $	\widehat{r}_T^{\text{inf}}$ is infeasible because it involves with counterfactual outcomes. Therefore, we use $Y_t - \tau(t \con \widehat{\bbeta}) $ as realizations of the treatment-free potential outcomes $\potY{t}{0}$ and construct a $(T_1 \times k_T)$ matrix as follows:
\begin{align*}
{\Phi}_{\post}
=
\begin{bmatrix}
\bphi \T \Big( Y_{T_0+1} - \tau(T_0+1 \con \widehat{\bbeta}) \Big)
\\
\vdots
\\
\bphi \T \Big( Y_{T} - \tau(T \con \widehat{\bbeta}) \Big)
\end{bmatrix}
\in \R^{T_1 \times k_T}
\ .
\end{align*}
We consider additional sieve estimators of ${\EXP} \big\{ \widehat{g} (\bW_{\D t}) \cond \potY{t}{0} \big\} $ and ${\EXP} \big\{ \widehat{h}_0(\bW_{\D t}) \cond \potY{t}{0} \big\}$ for $t=T_0+1,\ldots,T$:
\begin{align*}
\widehat{\mu}_{\post} (y \con \widehat{g})
&
=
{ \texttt{sieve} }
\Big(
\EXP \big\{ \widehat{g} (\bW_{\D t}) \cond \potY{t}{0} = y \big\} 
\Big)
\\
&
=
\bphi \T (y)
\big( \Phi_{\post} \T \Phi_{\post} \big)^{-1}
\bigg\{
\sum_{t=T_0+1}^{T}
\widehat{g}(\bW_{\D t} ) \bphi \Big( Y_t - \tau(t \con \widehat{\bbeta}) \Big)
\bigg\}
\\
\widehat{\mu}_{\post}(y \con \widehat{h}_0 )
&
=
{ \texttt{sieve} }
\Big(
\EXP \big\{ \widehat{h}_0(\bW_{\D t}) \cond \potY{t}{0} = y \big\}
\Big)
\\
&
=
\bphi \T (y)
\big( \Phi_{\post} \T \Phi_{\post} \big)^{-1}
\bigg\{
\sum_{t=T_0+1}^{T}
\widehat{h}_0(\bW_{\D t} ) \bphi \Big( Y_t - \tau(t \con \widehat{\bbeta}) \Big)
\bigg\} \ .
\end{align*}
Using these sieve estimators, we obtain a feasible estimator of $r_T(\widehat{h}_0)$ as
\begin{align*}
\widehat{r}_T (\widehat{h}_0)
= 
\frac{1}{ T_1 } \sum_{t=T_0+1}^{T} 
\frac{\partial \tau(t \con \widehat{\bbeta})}{\partial \bbeta}
\bigg[
\widehat{\mu}_{\post} \big( Y_t - \tau(t \con \widehat{\bbeta}) \con \widehat{g} \big)
\Big\{
Y_t - \tau(t \con \widehat{\bbeta})
-
\widehat{\mu}_{\post} \big( Y_t - \tau(t \con \widehat{\bbeta}) \con \widehat{h}_0 \big)
\Big\}
\bigg] \ .
\end{align*}
Under regularity conditions, we establish that
\begin{align*}
 \sup_{\widehat{h}_0 \in \widehat{\mathcal{H}}_0}
 \sqrt{T_1}
 \Big|
\widehat{r}_T(\widehat{h}_0) - {r}_T(\widehat{h}_0)
 \Big| = o_P(1) \ ;
\end{align*} 
see Lemma 1 of \citet{Li2023} and Lemma 3.3 of \citet{Zhang2023} for details. Based on this result, we subtract $T_1^{1/2} \widehat{r}_T(\widehat{h}_0)$ in both hand sides of \eqref{eq-asymptotic rep}. We then obtain a de-biased estimator $\widehat{\bbeta}_{\text{db}}$ as
\begin{align*}
\widehat{\bbeta}_{\text{db}}
=
\widehat{\bbeta} 
-
V^{-1} (\widehat{\bbeta}, \widehat{h}_0)
\sqrt{T_1} \widehat{r}_T(\widehat{h}_0) \ ,
\end{align*}
which is asymptotically normal in that $T_{1}^{1/2} \big( \widehat{\bbeta}_{\text{db}} - \bbeta^* \big)$ converges in distribution to $
N \big( 0, S_1^* S_2^* S_1\sT \big)$ as $T \rightarrow \infty$ where $S_1^*$ and $S_2^*$ are given as follows:
\begin{align*}
&
S_1^*
=
\bigg[
\frac{\partial \EXP \big\{ {\Psi}_{\post} (\bO_t \con \bbeta^*, h_0) \big\} }{\partial \bbeta} 
\bigg]^{-1} 
\\
&
S_2^*
=
\VAR
\left[
{\Psi}_{\post} (\bO_t \con \bbeta^*, h_0)
-
\frac{\partial \tau(t \con \bbeta^*)}{\partial \bbeta} 
\EXP \big\{ g_{0,h_0} (\bW_{\D t}) \cond \potY{t}{0} \big\} \big\{ \potY{t}{0} - h_0(\bW_{\D t} ) \big\}
\right] \ ,
\end{align*}
Here, ${\Psi}_{\post} (\bO_t \con \bbeta,h )$ is defined in \eqref{eq-supp-postEE} for $t=T_0+1,\ldots,T$. 
The ATT estimator is obtained from the plug-in formula $\widehat{\tau}_t = \tau(t \con \widehat{\bbeta}_{\text{db}})$. Consequently, inference of the ATT can be attained based on the standard delta-method applied to the asymptotic normal distribution of $\widehat{\bbeta}_{\text{db}}$.



\newpage

\section{Proof of Theorems}	\label{sec:supp:proof}


\subsection{Proof of Theorems \ref{thm:SC}, \ref{thm:ATT}, \ref{thm:identification cov}, \ref{thm:Extension NP}, and \ref{thm:Extension NP Cov}}

We first prove the most general case with a nonlinear bridge function $h^*$ and under the presence of covariates (i.e., Theorem \ref{thm:Extension NP Cov}). For the pre-treatment periods $t=1,\ldots,T_0$, we establish
\begin{align*}
y 
=
\EXP \big\{ h^* (\bW_{\D t}, \bX_{0t}, \bX_{\D t}) \cond \potY{t}{0} = y, \bX_{0t}, \bX_{\D t}  \big\} 
=
\EXP \big\{ h^* ( \bW_{\D t}, \bX_{0t}, \bX_{\D t} ) \cond Y_t=y, \bX_{0t}, \bX_{\D t} \big\} \ .
\end{align*} 
The first equality holds from Assumption \ref{assumption:SC NP Cov}. The second equality holds from Assumption \ref{assumption:consistency}.  

Furthermore, for any $t=1,\ldots,T$, we establish
\begin{align*} 
\EXP \big\{ \potY{t}{0} \cond \bX_{0 t} , \bX_{\D t} \big\} 
& = 
\EXP \big[
\EXP \big\{ h^*(\bW_{\D t}, \bX_{0 t} , \bX_{\D t}) \cond \potY{t}{0} , \bX_{0 t} , \bX_{\D t} \big\}		
\cond \bX_{0 t} , \bX_{\D t}
\big]
\nonumber 
\\
&
=		
\EXP \big\{ h^* (\bW_{\D t}, \bX_{0 t} , \bX_{\D t}) \cond \bX_{0 t} , \bX_{\D t} \big\} \ .
\end{align*}	
The first equality holds from Assumption \ref{assumption:SC NP Cov}, and the second equality holds from the law of iterated expectation. Therefore, we have 
\begin{align}
    \label{eq-Identity1}
    \EXP \big\{ \potY{t}{0} \big\} 
    =
    \EXP \big\{ h^* (\bW_{\D t}, \bX_{0 t} , \bX_{\D t}) \big\} \ .
\end{align}

Next, we prove the second result. For the post-treatment periods $t=T_0+1,\ldots,T$, we have 
\begin{align*}
& \EXP \big\{ \potY{t}{1} - \potY{t}{0} \big\} = 
\EXP \big\{ Y_t - \potY{t}{0} \big\}
=
\EXP \big\{ Y_t - h^* (\bW_{\D t}, \bX_{0 t} , \bX_{\D t}) \big\}
\end{align*}
The first equality holds from Assumption \ref{assumption:consistency}. The second equality holds from \eqref{eq-Identity1}.

We remark that the other Theorems can be shown in a similar manner. Specifically, we take $h^*(\bW_{\D t}, \bX_{0t}, \bX_{\D t}) = \bX_{0t}\T \bdelta_{0}^* + \bW_{\D t}\T \bgamma^* - \bX_{\D t}\T \bdelta_{\D}^* $ for the linear bridge function case,  and we view covariates as empty sets when there is no covariate available. The results can then be established under Assumptions \ref{assumption:SC}, \ref{assumption:SC Cov}, and \ref{assumption:SC NP}. 

\subsection{Proof of Theorems \ref{thm:AN} and \ref{thm:AN Cov}}		\label{sec:supp:AN}

We denote the collection of parameters as $\btheta$. 
When there is no covariate as in Section \ref{sec:Estimation}, we have $\btheta = (\bgamma, \bbeta)$; when there are covariates as in Section \ref{sec:Cov}, we have $\btheta = (\bgamma, \bbeta, \bdelta)$. 
Let the moment function be $\Psi(\bO_t \con \btheta)$ where $\mathcal{O}= \text{supp}(\bO_t)$ and $\Theta = \text{supp}(\btheta)$. 
We denote the true parameters as $\btheta^*$.
\begin{remark}
We will consider a simple case as an example to motivate the assumptions below. Specifically, suppose the treatment effect function is constant, i.e., $\Psi(\bO_t \con \bbeta) = \bbeta$. Then, the moment function is
\begin{align}		\label{eq-supp-simple}
\Psi (\bO_t \con \bgamma, \bbeta)
=
\begin{bmatrix}
\Psi_{\pre} (\bO_t \con \bgamma)
\\
\Psi_{\post} (\bO_t \con \bgamma , \bbeta)
\end{bmatrix}
=
\begin{bmatrix}
(1-A_t)
\bg_t(Y_t)
\big( Y_t - \bW_{\D t} \T \bgamma \big)
\\
A_t 
\big(
Y_t - \bW_{\D t} \T \bgamma - \bbeta
\big)
\end{bmatrix}
\ .
\end{align}
Note that the derivative of \eqref{eq-supp-simple} is
\begin{align*}
\frac{\partial \Psi (\bO_t \con \bgamma, \bbeta)}{\partial (\bgamma, \bbeta)\T }
=
-
\begin{bmatrix}
(1-A_t) \bg_t(Y_t) \bW_{\D t} \T & \bzero_{p}
\\
A_t \bW_{\D t} \T & A_t
\end{bmatrix}
\in \R^{(p+1) \times (d+1)}
\ .
\end{align*}
\end{remark}


We present proofs for Theorem \ref{thm:AN} under different sets of regularity conditions. In the first proof, we establish the result using commonly employed conditions for time series data, such as strong stationarity and ergodicity. In the second proof, we establish the result under more general conditions, avoiding the need to rely on these strong assumptions of stationarity and ergodicity.

\vspace{0.5cm}
\noindent \textbf{Proof 1: Proof under Strong Stationarity and Ergodicity}
\vspace{0.5cm}

We first present regularity conditions, which are the extensions of assumptions in Chapter 3 of \citet{Hall2004GMM}.

\begin{REG}[Sufficiently Long Pre- and Post-treatment Periods] 	\label{assumption-reg-11}
As $T \rightarrow \infty$, $T_0, T_1 \rightarrow \infty$ and $T_1/T_0 \rightarrow r \in (0,\infty)$. 
\end{REG} 
Regularity Condition \ref{assumption-reg-11} is reasonable if the pre- and post-treatment periods are of roughly the same size and sufficiently large.


\begin{REG}[Compactness] 	\label{assumption-reg-7}
The parameter space $\Theta$ is compact.
\end{REG}
Regularity Condition \ref{assumption-reg-7} is standard in parametric estimation.

\begin{REG}[Weighting Matrix] 	\label{assumption-reg-5}
$\widehat{\Omega}$ is a positive semi-definite matrix, and converges to a non-random positive definite matrix $\Omega^*$ as $T \rightarrow \infty$.
\end{REG}
Regularity Condition \ref{assumption-reg-5} is easily satisfied if $\widehat{\Omega}$ is chosen as a fixed matrix such as the identity matrix.


\begin{REG}[Strict Stationarity]		\label{assumption-reg-1}
The process $\big\{ \bO_t \big\}_{t \in \mathbbm{Z}}$ is strictly stationary, i.e., for any subset $\{t_1,\ldots,t_n \} \subseteq \mathbbm{Z}$ and any $c$, we have $	\big\{ \bO_{t_1},\ldots,\bO_{t_n} \big\}
\stackrel{D}{=}
\big\{ \bO_{t_1+c},\ldots,\bO_{t_n+c} \big\}$.
\end{REG}
Regularity Condition \ref{assumption-reg-1} implies that any expectation of $\bO_t$ does not depend on $t$. Unfortunately, Regularity Condition \ref{assumption-reg-1} is insufficient to apply the law of large numbers and central limit theorem. Therefore, the following ergodicity assumption is required:

\begin{REG}[Ergodicity] 	\label{assumption-reg-6}
The process $\big\{ \bO_t \big\}_{t \in \mathbbm{Z}}$ is ergodic.
\end{REG}
Under Regularity Condition \ref{assumption-reg-1} and Regularity Condition \ref{assumption-reg-6}, the sample average of $f(\bO_t)$ converges to its expectation, i.e., $T^{-1} \sum_{t=1}^{T} f(\bO_t) \stackrel{P}{\rightarrow} \EXP \big\{ f(\bO_t) \big\}$.


\begin{REG}[Regularity Conditions for $\Psi$]		\label{assumption-reg-2}
The moment function $\Psi(\bO_{t} \con \btheta) : \mathcal{O} \otimes \Theta \rightarrow \R^{p+b}$ satisfies
\begin{itemize}[itemsep=0cm]
\item[(i)] $\Psi(\bO_{t} \con \btheta)$ is continuous on $\Theta$ for each $\bO_{t} \in \mathcal{O}$;
\item[(ii)] $\EXP \big\{ \Psi(\bO_{t} \con \btheta) \big\} $ exists and is finite for any $\btheta \in \Theta$;
\item[(iii)] $\EXP \big\{ \Psi(\bO_{t} \con \btheta) \big\}$ is continuous on $\Theta$.
\item[(iv)] $\bg_t$ is time-invariant, i.e., $\bg_t(y) = \bg(y)$ for all $t$.
\end{itemize}
\end{REG}
Under model \eqref{eq-supp-simple}, Regularity Condition \ref{assumption-reg-2} is satisfied if $ \EXP \big\{ \bg_t(Y_t) Y_t \big\}$ and $ \EXP \big\{ \bg_t(Y_t) \bW_{\D t}\T \big\} $ for $t=1,\ldots,T_0$ and $\EXP \big( Y_t \big)$ and $\EXP \big( \bW_{\D t} \big)$ for $t=T_0+1,\ldots,T$ are finite and well-defined. Hereafter, we assume that these vectors and matrices are finite and well-defined. 

\begin{REG}[Regularity for $\partial \Psi( \bO_{t} \con \btheta) / \partial \btheta \T$ \& Local Identification] 	\label{assumption-reg-3}
The function $\partial \Psi( \bO_{t} \con \btheta) / \partial \btheta \T \in \R^{(p+b) \times (d+b)}$ satisfies:
\begin{itemize}[itemsep=0cm]
\item[(i)] $\partial \Psi( \bO_{t} \con \btheta) / \partial \btheta \T$ exists and is continuous on $\Theta$ for each $\bO_{t} \in \mathcal{O}$;
\item[(ii)] $\btheta ^* \in \text{int} (\Theta)$; 
\item[(iii)] $\EXP \big\{ \partial \Psi( \bO_{t} \con \btheta) / \partial \btheta \T \big\}$ exists and is finite;
\item[(iv)] $\text{rank} \big( \EXP \big\{ \partial \Psi( \bO_{t} \con \btheta) / \partial \btheta \T \big\} \big) = d+b$.
\end{itemize}
\end{REG}
Under model \eqref{eq-supp-simple}, Regularity Condition \ref{assumption-reg-3} (iii) is satisfied if $\EXP \big\{ \bg_t(Y_t) \bW_{\D t}\T \big\}$ for $t=1,\ldots,T_0$ and $\EXP \big( \bW_{\D t} \big)$ for $t=T_0+1,\ldots,T$ are finite and well-defined. Condition (iv) is satisfied if the vectors $ \boldr_i = \EXP \big\{ \bg_t(Y_t) W_{it} \big\}/ \EXP \big( W_{it} \big) \in \R^{p}$ $(i \in \D_1,\ldots,\D_d)$ are linearly independent.

\begin{REG}[Population Moment Restriction \& Global Identification] 	\label{assumption-reg-4}
The true parameter
$\btheta^*$ is the unique parameter that satisfies $ \EXP \big\{ \Psi(\bO_{t} \con \btheta^* ) \big\} = 0$.
\end{REG}
Under model \eqref{eq-supp-simple}, Regularity Condition \ref{assumption-reg-4} is satisfied if $\bG_{YW}^* $ is of full column rank, i.e., $\text{rank}(\bG_{YW}^*) = d$.




\begin{REG}[Domination of $\Psi$] 	\label{assumption-reg-8}
The expectation of the moment function is uniformly bounded over $\Theta$, i.e., $\sup_{\btheta \in \Theta} \EXP \big\{ \big\| \Psi( \bO_{t} \con \btheta) \big\|_2 \big\} < \infty$.
\end{REG}
Under model \eqref{eq-supp-simple}, Regularity Condition \ref{assumption-reg-8} is satisfied if the second-order moments of $\bg_t(Y_t) Y_t$ and $\bg_t(Y_t) \bW_{\D t}$ for $t=1,\ldots,T_0$ and $Y_t$ and $\bW_{\D t}$ for $t=T_0+1,\ldots,T$ are finite.


\begin{REG}[Properties of the Variance] 	\label{assumption-reg-9}
Following conditions hold:
\begin{itemize}[itemsep=0cm]
\item[(i)] $\EXP \big\{ \Psi(\bO_t \con \btheta^*) \Psi (\bO_t \con \btheta^*) \T \big\}$ exists and finite
\item[(ii)] $\Sigma_2^* = \lim_{T \rightarrow \infty} \VAR \big\{ T^{-1/2} \sum_{t=1}^{T} \Psi (\bO_t \con \btheta^*) \big\}$ exists and is a finite valued positive definite matrix.
\end{itemize} 
\end{REG}
Under model \eqref{eq-supp-simple}, Regularity Condition \ref{assumption-reg-9} (i) is satisfied if the second-order moments of $\bg_t(Y_t) Y_t$, $\bg_t(Y_t) \bW_{\D t}$ for $t=1,\ldots,T_0$ and those of $Y_t$ and $\bW_{\D t}$ for $t=T_0+1,\ldots,T$ are finite. Condition (ii) is satisfied if $\big\{ \bO_t \big\}$ are weakly serially dependent (e.g., m-dependent).


\begin{REG}[Properties of Gradient] 	\label{assumption-reg-10}
Following conditions hold for $\Theta_N$, some neighborhood of $\btheta^*$:
\begin{itemize}[itemsep=0cm]
\item[(i)] $\EXP \big\{ \partial \Psi( \bO_t \con \btheta) / \partial \btheta \T \big\}$ is continuous on $\Theta_N$;
\item[(ii)] $\sup_{\btheta \in \Theta_N} \big\| T^{-1} \sum_{t=1}^{T} \partial \Psi( \bO_t \con \btheta) / \partial \btheta \T - \EXP \big\{ \partial \Psi( \bO_t \con \btheta) / \partial \btheta \T \big\} \big\|_2 = o_P(1)$.
\end{itemize}
\end{REG}
Under model \eqref{eq-supp-simple}, Regularity Condition \ref{assumption-reg-10} (i) is satisfied if $\EXP \big\{ \bg_t(Y_t) \bW_{\D t}\T \big\}$ for $t=1,\ldots,T_0$ and $\EXP \big( \bW_{\D t} \big)$ for $t=T_0+1,\ldots,T$ are finite and well-defined. Regularity Condition \ref{assumption-reg-10} (ii) is satisfied if 
\begin{align*}
& 
\frac{1}{T_0} \sum_{t=1}^{T_0} \bg_t(Y_t) \bW_{\D t} \T 
\stackrel{P}{\rightarrow} \EXP \big\{ \bg_t(Y_t) \bW_{\D t} \T \big\} 
\ , \ 
&&
t=1,\ldots,T_0
\\
&
\frac{1}{T_1} \sum_{t=T_0+1}^{T} \bW_{\D t} 
\stackrel{P}{\rightarrow} 
\EXP \big( \bW_{\D t} \big)
\ , \
&&
t=T_0+1,\ldots,T \ .
\end{align*}
Note that these two conditions are satisfied under the asymptotic regime in Regularity Condition \ref{assumption-reg-11}, stationarity (Regularity Condition \ref{assumption-reg-1}), and ergodicity (Regularity Condition \ref{assumption-reg-6}).

Under these assumptions, the asymptotic normality of $(\widehat{\bgamma} , \widehat{\bbeta})$ is achieved by Theorem 3.2 of \citet{Hall2004GMM}. 

\vspace{0.5cm}
\noindent \textbf{Proof 2: Proof without Strong Stationarity and Ergodicity}
\vspace{0.5cm}

We adapt the proof of Theorem S6 in \citet{Qiu2022} to our setting. First, we introduce regularity conditions that are applicable to general cases, without imposing strict requirements of strong stationarity and ergodicity.
 

\begin{GREG}[Sufficiently Long Pre- and Post-treatment Periods] 	\label{assumption-General-1}
As $T \rightarrow \infty$, $T_0, T_1 \rightarrow \infty$ and $T_1/T_0 \rightarrow r \in (0,\infty)$. 
\end{GREG} 
\begin{GREG}[Compactness] 	\label{assumption-General-2}
The parameter space $\Theta$ is compact.
\end{GREG} 
\begin{GREG}[Weighting Matrix] 	\label{assumption-General-3}
$\widehat{\Omega}$ is a positive semi-definite matrix, and converges to a non-random positive definite matrix $\Omega^*$ as $T \rightarrow \infty$.
\end{GREG}
General Regularity Conditions \ref{assumption-General-1}--\ref{assumption-General-3} are the same as Regularity Conditions \ref{assumption-reg-11}--\ref{assumption-reg-5}, respectively.




\begin{GREG}[Regularity Conditions for $\Psi$]
\label{assumption-General-4}
The moment function $\Psi(\bO_{t} \con \btheta) : \mathcal{O} \otimes \Theta \rightarrow \R^{p+b}$ satisfies
\begin{itemize}[itemsep=0cm]
\item[(i)] $ \lim_{T\rightarrow \infty} \big\{ T^{-1} \sum_{t=1}^{T} \Psi(\bO_{t} \con \btheta) \big\} $ is continuous on $\Theta$ for each $\bO_{t} \in \mathcal{O}$;
\item[(ii)] $ \lim_{T\rightarrow \infty} \big[ T^{-1} \sum_{t=1}^{T} \EXP \big\{ \Psi(\bO_{t} \con \btheta) \big\} \big] $ exists and is finite for any $\btheta \in \Theta$;
\item[(iii)] $ \lim_{T\rightarrow \infty} \big[ T^{-1} \sum_{t=1}^{T} \EXP \big\{ \Psi(\bO_{t} \con \btheta) \big\} \big] $ is continuous on $\Theta$.
\end{itemize}
\end{GREG}

\begin{GREG}[Regularity for $\partial \Psi( \bO_{t} \con \btheta) / \partial \btheta \T$ \& Local Identification] 	\label{assumption-General-5}
The function $\partial \Psi( \bO_{t} \con \btheta) / \partial \btheta \T \in \R^{(p+b) \times (d+b)}$ satisfies:
\begin{itemize}[itemsep=0cm]
\item[(i)] $ \lim_{T\rightarrow \infty} \big\{ T^{-1} \sum_{t=1}^{T} \partial \Psi( \bO_{t} \con \btheta) / \partial \btheta \T \big\} $ exists and is continuous on $\Theta$ for each $\bO_{t} \in \mathcal{O}$;
\item[(ii)] $ T^{-1} \sum_{t=1}^{T} \partial \Psi( \bO_{t} \con \btheta) / \partial \btheta \T $ is uniformly bounded for all $T=1,2,\ldots$
\item[(iii)] $\btheta ^* \in \text{int} (\Theta)$; 
\item[(iv)] $ \lim_{T\rightarrow \infty} \big[ T^{-1} \sum_{t=1}^{T} \EXP \big\{ \partial \Psi( \bO_{t} \con \btheta) / \partial \btheta \T \big\} \big] $ exists and is finite;
\item[(v)] $\text{rank} \big( \lim_{T\rightarrow \infty} \big[ T^{-1} \sum_{t=1}^{T} \EXP \big\{ \partial \Psi( \bO_{t} \con \btheta) / \partial \btheta \T \big\} \big] \big) = d+b$.
\end{itemize}
\end{GREG}
\begin{GREG}[Population Moment Restriction \& Global Identification] 	\label{assumption-General-6}
The true parameter
$\btheta^*$ is the unique parameter that satisfies $ \lim_{T\rightarrow \infty} \big[ T^{-1} \sum_{t=1}^{T} \EXP \big\{ \Psi(\bO_{t} \con \btheta^*) \big\} \big] = 0 $.
\end{GREG}
General Regularity Conditions \ref{assumption-General-4}--\ref{assumption-General-6} are similar to Regularity Conditions \ref{assumption-reg-2}--\ref{assumption-reg-4}. 

\begin{GREG}[Uniform Weak Law of Large Numbers for $\Psi$] 	\label{assumption-General-UWLLN}
\begin{align*}
\sup_{\btheta \in \Theta}
\bigg\|
\frac{1}{T} \sum_{t=1}^{T} \Psi(\bO_t \con \btheta) -
\lim_{T' \rightarrow \infty}
\frac{1}{T'} \sum_{t=1}^{T'} \EXP \big\{\Psi(\bO_t \con \btheta) \big\}
\bigg\|
=
o_P(1) \text{ as } T \rightarrow \infty \ .
\end{align*}
\end{GREG}

\begin{GREG}[Uniform Weak Law of Large Numbers for the Gradient of $\Psi$] 	\label{assumption-General-UWLLN2}
\begin{align*}
\sup_{\btheta \in \Theta}
\bigg\|
\frac{1}{T} \sum_{t=1}^{T} \frac{\partial}{\partial \btheta \T } \Psi(\bO_t \con \btheta) -
\lim_{T' \rightarrow \infty}
\frac{1}{T'} \sum_{t=1}^{T'} \EXP \bigg\{ \frac{\partial}{\partial \btheta \T } \Psi(\bO_t \con \btheta) \bigg\}
\bigg\|
=
o_P(1) \text{ as } T \rightarrow \infty \ .
\end{align*}
\end{GREG}
General Regularity Conditions \ref{assumption-General-UWLLN} and \ref{assumption-General-UWLLN2} hold if the underlying process is strictly stationary, strongly mixing, or $\phi$-mixing processes; see \citet{Andrews1988}, \citet[Chapter 5]{PP1997} and \citet[Section S2]{Qiu2022} for details. 

\begin{GREG}[Asymptotic Normality]
\label{assumption-General-AN} 
As $T \rightarrow \infty$, we have
\begin{align*}
&
\frac{1}{\sqrt{T}} \sum_{t=1}^{T} \Psi(\bO_t \con \btheta^*)
\text{ converges in distribution to }
N (0, \Sigma_2^*)
\ , \\
&
\Sigma_2^* = \lim_{T \rightarrow \infty} \VAR \bigg\{ \frac{1}{\sqrt{T}} \sum_{t=1}^{T} \Psi (\bO_t \con \btheta^*) \bigg\} \ .
\end{align*}
Here, $\Sigma_2^*$ is a finite valued positive definite matrix. 
\end{GREG}
Assumption \ref{assumption-General-AN} directly assumes the asymptotic normality of the sample mean of the estimating function; see Section S2 of \citet{Qiu2022} for the plausibility of the assumption. We remark that General Regularity Conditions \ref{assumption-General-UWLLN}--\ref{assumption-General-AN} are satisfied under Regularity Conditions \ref{assumption-reg-1}--\ref{assumption-reg-10}. 

Under General Regularity Conditions \ref{assumption-General-1}--\ref{assumption-General-AN}, we establish the desired result. We simply denote $\widehat{\Psi}(\btheta) = T^{-1} \sum_{t=1}^T \Psi(\bO_t \con \btheta)$ and ${\Psi}(\btheta) = \lim_{T \rightarrow \infty} T^{-1} \sum_{t=1}^T \EXP \big\{ \Psi(\bO_t \con \btheta) \big\}$. First, we establish consistency, i.e., $\widehat{\btheta} = (\widehat{\bgamma}, \widehat{\bbeta}) = (\bgamma^*, \bbeta^*) + o_P(1) = \btheta^* + o_P(1)$. From General Regularity Conditions \ref{assumption-General-3} and \ref{assumption-General-UWLLN}, we establish the following result as $T \rightarrow \infty$:
\begin{align}
&
\sup_{\btheta \in \Theta}
\left\|
\big\{
\widehat{\Psi}(\btheta)
\big\}\T 
\widehat{\Omega}
\big\{
\widehat{\Psi}(\btheta)
\big\} 
-
\big\{
\Psi (\btheta)
\big\}\T 
\Omega^*
\big\{
\Psi (\btheta)
\big\}
\right\|
=
o_P(1) \ .
\label{eq-UWLLN-GMM}
\end{align}
Note that $\widehat{\btheta}$ is the minimizer of $\big\{
\widehat{\Psi}(\btheta)
\big\}\T 
\widehat{\Omega}
\big\{
\widehat{\Psi}(\btheta)
\big\}$. 

Let $s > 0$ be an arbitrary positive constant. 
From \eqref{eq-UWLLN-GMM} and the definition of $\widehat{\btheta}$, the following conditions hold with probability tending to one:
\begin{align*}
&
\left\|
\big\{
\widehat{\Psi}(\btheta^*)
\big\}\T 
\widehat{\Omega}
\big\{
\widehat{\Psi}(\btheta^*)
\big\} 
-
\big\{
\Psi (\btheta^*)
\big\}\T 
\Omega^*
\big\{
\Psi (\btheta^*)
\big\}
\right\| < s/2
\\
&
\left\|
\big\{
\widehat{\Psi}(\widehat{\btheta})
\big\}\T 
\widehat{\Omega}
\big\{
\widehat{\Psi}(\widehat{\btheta})
\big\} 
-
\big\{
\Psi (\widehat{\btheta})
\big\}\T 
\Omega^*
\big\{
\Psi (\widehat{\btheta})
\big\}
\right\| < s/2 
\\ 
&
\big\{
\widehat{\Psi} (\widehat{\btheta})
\big\}\T 
\widehat{\Omega}
\big\{
\widehat{\Psi} (\widehat{\btheta})
\big\}
\leq
\big\{
\widehat{\Psi} (\btheta^*)
\big\}\T 
\widehat{\Omega}
\big\{
\widehat{\Psi} (\btheta^*)
\big\}
 \ .
\end{align*}
These three inequalities imply that
\begin{align*}
\big\{
\Psi(\widehat{\btheta})
\big\}\T 
\Omega^*
\big\{
\Psi(\widehat{\btheta})
\big\}
<
\big\{
\Psi(\btheta^*)
\big\}\T 
\Omega^*
\big\{
\Psi(\btheta^*)
\big\} + s
=
s \ .
\end{align*} 
The right hand side reduces to $s$ under General Regularity Condition \ref{assumption-General-6}. 

Let $\mathcal{N} \in \Theta$ be an arbitrary open set containing $\btheta^*$. Let us define the following quantity:
\begin{align*}
s_0
=
\inf_{\btheta \in \Theta \setminus \mathcal{N} }
\big\{ 
\Psi(\btheta)
\big\}\T
\Omega^*
\big\{ 
\Psi(\btheta)
\big\} .
\end{align*}
Combining the fact that $\Theta \setminus \mathcal{N}$ is compact under General Regularity Condition \ref{assumption-General-2}, and General Regularity Conditions \ref{assumption-General-2}, \ref{assumption-General-4}, \ref{assumption-General-6}, we establish $s_0$ is positive. Therefore, by taking $s>s_0$, the event $\big\{ \big\{
\Psi(\widehat{\btheta})
\big\}\T 
\widehat{\Omega}
\big\{
\Psi(\widehat{\btheta})
\big\}
< s_0 \big\}$ occurs with probability tending to one, which further implies that $\widehat{\btheta} \in \mathcal{N}$. Since $\mathcal{N}$ is arbitrary chosen, this establishes $\widehat{\btheta} = \btheta^* + o_P(1)$ as $T \rightarrow \infty$. 

Next, we establish the asymptotic normality of $\widehat{\btheta}$. Under General Regularity Condition \ref{assumption-General-5}, the following expansion holds from a first-order Taylor expansion and the consistency of $\widehat{\btheta}$:
\begin{align*}
&
\widehat{\Psi}( \widehat{\btheta})
=
\widehat{\Psi}( {\btheta}^*)
+
\bigg\{
\frac{\partial}{\partial \btheta\T} \widehat{\Psi}( \btheta) \bigg|_{\btheta=\btheta^*}
\bigg\}
(\widehat{\btheta}-\btheta^*)
+
o_P \Big( \big\| \widehat{\btheta} - \btheta^* \big\| \Big)
\ . 
\end{align*}
The first order condition of $\widehat{\btheta}$ along with General Regularity Condition \ref{assumption-General-5} implies
\begin{align*}
0
& =
\frac{1}{2}
\frac{\partial}{\partial \btheta\T} 
\Big[
\big\{ \widehat{\Psi}( \btheta)
\big\}\T \widehat{\Omega} 
\big\{ \widehat{\Psi}( \btheta)
\big\} \Big] \Big|_{\btheta=\widehat{\btheta}}
\\
&
=
\bigg\{
\frac{\partial}{\partial \btheta\T} \widehat{\Psi}( \btheta) \bigg|_{\btheta=\btheta^*}
\bigg\}\T
\widehat{\Omega}
\big\{ \widehat{\Psi}( \widehat{\btheta})
\big\}
\\
&
=
\bigg\{
\frac{\partial}{\partial \btheta\T} \widehat{\Psi}( \btheta) \bigg|_{\btheta=\btheta^*}
\bigg\}\T
\widehat{\Omega}
\big\{ \widehat{\Psi}( \btheta^*)
\big\}
+
\bigg\{
\frac{\partial}{\partial \btheta\T} \widehat{\Psi}( \btheta) \bigg|_{\btheta=\btheta^*}
\bigg\}\T
\widehat{\Omega}
\bigg\{
\frac{\partial}{\partial \btheta\T} \widehat{\Psi}( \btheta) \bigg|_{\btheta=\btheta^*}
\bigg\} 
(\widehat{\btheta}-\btheta^*)
\\
&
\hspace*{0.5cm}
+
o_P \Big( \big\| \widehat{\btheta} - \btheta^* \big\| \Big) \ .
\end{align*}
Therefore, by multiplying $T^{1/2}$, we get
\begin{align*}
0
& 
=
\bigg\{
\frac{\partial}{\partial \btheta\T} \widehat{\Psi}( \btheta) \bigg|_{\btheta=\btheta^*}
\bigg\}\T
\widehat{\Omega}
\bigg\{
\frac{1}{T^{1/2}}
\sum_{t=1}^{T} \Psi(\bO_t \con \btheta^* )
\bigg\}
\\
&
\hspace*{0.3cm}
+
\bigg\{
\frac{\partial}{\partial \btheta\T} \widehat{\Psi}( \btheta) \bigg|_{\btheta=\btheta^*}
\bigg\}\T
\widehat{\Omega}
\bigg\{
\frac{\partial}{\partial \btheta\T} \widehat{\Psi}( \btheta) \bigg|_{\btheta=\btheta^*}
\bigg\}
T^{1/2}
(\widehat{\btheta}-\btheta^*)
+
o_P \Big( T^{1/2} \big\| \widehat{\btheta} - \btheta^* \big\| \Big)
\\
&
=
\underbrace{
\bigg\{
\frac{\partial}{\partial \btheta\T} {\Psi}( \btheta) \bigg|_{\btheta=\btheta^*}
\bigg\}\T }_{=:G\sT}
\Omega^*
\bigg\{
\lim_{T \rightarrow \infty}
\frac{1}{T^{1/2}}
\sum_{t=1}^{T} \Psi(\bO_t \con \btheta^* )
\bigg\}
\\
&
\hspace*{0.3cm}
+
\bigg\{
\frac{\partial}{\partial \btheta\T} {\Psi}( \btheta) \bigg|_{\btheta=\btheta^*}
\bigg\}\T
\Omega^*
\bigg\{
\frac{\partial}{\partial \btheta\T} {\Psi}( \btheta) \bigg|_{\btheta=\btheta^*}
\bigg\} 
T^{1/2}
(\widehat{\btheta}-\btheta^*)
+
o_P \Big( T^{1/2} \big\| \widehat{\btheta} - \btheta^* \big\| + 1 \Big) \ .
\end{align*}
The last equality holds from General Regularity Conditions \ref{assumption-General-3}, \ref{assumption-General-UWLLN}, and \ref{assumption-General-UWLLN2} and the consistency of $\widehat{\btheta}$. This implies
\begin{align*}
T^{1/2} \big( \widehat{\btheta} - \btheta^* \big)
=
\Big ( G\sT \Omega^* G^* \Big)^{-1}
G\sT \Omega^* 
\bigg\{ 
\frac{1}{T^{1/2}}
\sum_{t=1}^{T} \Psi(\bO_t \con \btheta^* )
\bigg\}
+
o_P \Big( T^{1/2} \big\| \widehat{\btheta} - \btheta^* \big\| + 1 \Big) \ .
\end{align*}
From General Regularity Condition \ref{assumption-General-3} and \ref{assumption-General-AN}, we find $T^{1/2} \big( \widehat{\btheta} - \btheta^* \big) = O_P(1)$, implying that $o_P\big( T^{1/2} \big\| \widehat{\btheta} - \btheta^* \big\| + 1 \big) = o_P(1)$. Therefore, from Slutsky's theorem, we find 
\begin{align*}
T^{1/2} \big( \widehat{\btheta} - \btheta^* \big)
& \text{ converges in distribution to }
N \big( 0, \big \{ G\sT \Omega^* G^* \big\}^{-1}
G\sT \Omega^* \Sigma_2 G^* \Omega^*
\big \{ G\sT \Omega^* G^* \big\}^{-\T} \big)
\end{align*}
Note that $\big\{ G\sT \Omega^* G^* \big\}^{-1}
G\sT \Omega^* \Sigma_2 G^* \Omega^*
\big \{ G\sT \Omega^* G^* \big\}^{-\T} = \Sigma_1^* \Sigma_2 \Sigma_1\sT $. 
This concludes the proof.






 
 
\subsection{Proof of Theorem \ref{thm:Reg GMM}}		\label{sec:supp:Reg GMM Proof}

We denote the collection of parameters as $\btheta=(\bgamma,\bbeta)$ and the true parameters as $\btheta^* = (\bgamma^*,\bbeta^*)$. Similar to the proofs of Theorem \ref{thm:AN} in Section \ref{sec:supp:AN}, we present proofs under different sets of regularity conditions. 

\vspace{0.5cm}
\noindent \textbf{Proof 1: Proof under Strong Stationarity and Ergodicity}
\vspace{0.5cm}




The solution to $\ell_2$-penalized GMM is given as follows:
\begin{align*}
\widehat{\btheta}_{\lambda}
&
=
\argmin_{\btheta}
\Big[
\big\{ \widehat{\Psi}( \btheta) \big\} \T 
\widehat{\Omega}
\big\{ \widehat{\Psi}( \btheta) \big\} 
+ 
\lambda \big\| \bgamma \big\|_2^2
\Big] 
\ , \ 
\widehat{\Omega}
=
\begin{bmatrix}
 \widehat{\Omega}_{\pre} & 0 \\ 0 & \widehat{\Omega}_{\post}
\end{bmatrix}
\ .
\end{align*}
The solution to the minimization problem satisfies the following first order condition, i.e., 
\begin{align*}
 0
& =
\frac{1}{2}
\frac{\partial}{\partial \btheta\T} 
\Big[
\big\{ \widehat{\Psi}( \btheta)
\big\}\T \widehat{\Omega} 
\big\{ \widehat{\Psi}( \btheta)
\big\}
+ \lambda \big\| \bgamma \big\|_2^2
\Big] \Big|_{\btheta=\widehat{\btheta}_{\lambda}} \ .
\end{align*}
Therefore, from a few lines of algebra, we find $\widehat{\bgamma}_{\lambda}$ and $\bgamma^*$ are represented as
\begin{align*}
&
\widehat{\bgamma}_\lambda
=
\big( \widehat{\bG}_{YW} \T \widehat{\Omega}_{\pre} \widehat{\bG}_{YW} + \lambda \widehat{\Omega}_{\pre} \big)^{-1}
\big( \widehat{\bG}_{YW} \T \widehat{\Omega}_{\pre} \widehat{\bG}_{YY} \big)
\ , \
&&
{\bgamma} 
=
\big( {\bG}_{YW} \sT {\Omega}_{\pre}^* {\bG}_{YW}^* \big)^{-1}
\big( {\bG}_{YW} \sT {\Omega}_{\pre}^* {\bG}_{YY}^* \big)
\ .
\end{align*}
where $\bG_{YW}^*$, $\bG_{YY}^*$, $\widehat{\bG}_{YW}$, and $\widehat{\bG}_{YY}$ are defined in \eqref{eq-Gyw Gyy}, which is rewritten below for readability:
\begin{align}
&
\bG_{YW}^* = \frac{1}{T_0} 
\sum_{t=1}^{T_0} \EXP \big\{ \bg_t(Y_t) \bW_{\D t} \T \big\} \in \R^{p \times d}
\ , 
&&
\bG_{YY}^* = \frac{1}{T_0} 
\sum_{t=1}^{T_0} \EXP \big\{ \bg_t(Y_t) Y_t \big\} \in \R^{p}
\ ,
\nonumber
\\
\tag{\ref{eq-Gyw Gyy}}
&
\widehat{\bG}_{YW} =
\frac{1}{T_0} 
\sum_{t=1}^{T_0} \bg_t(Y_t) \bW_{\D t}\T \in \R^{p \times d}
\ , 
&&
\widehat{\bG}_{YY} = 
\frac{1}{T_0} \sum_{t=1}^{T_0} \bg_t(Y_t) Y_t \in \R^{p} \ .
\end{align}
The forms of $\widehat{\bbeta}_\lambda$ and $\bbeta_\lambda^*$ depend on $\tau(t \con \bbeta)$, which are defined as follows:
\begin{align*}
 &
 \widehat{\bbeta}_\lambda = \argmin_{\bbeta} 
 \big\{ \widehat{\Psi}_{\post}(\widehat{\bgamma}_{\lambda}, \bbeta) \big\} \T
 \widehat{\Omega}_{\post}
 \big\{ \widehat{\Psi}_{\post}(\widehat{\bgamma}_{\lambda}, \bbeta) \big\} 
 \ , \\
 &
 {\bbeta}^* = \argmin_{\bbeta} 
 \big\{ {\Psi}_{\post}(\bgamma^*, \bbeta) \big\} \T
 {\Omega}_{\post}^*
 \big\{ {\Psi}_{\post}(\bgamma^*, \bbeta) \big\}
\end{align*}
where
\begin{align*}
 &
 \widehat{\Psi}_{\post} (\bgamma,\bbeta)
 =
 \frac{1}{T_1} \sum_{t=T+0+1}^{T} \frac{\partial}{\partial \bbeta\T} \tau(t \con \bbeta) \big\{ Y_t - \bW_{\D t}\T \bgamma - \tau(t \con \bbeta) \big\} 
 \ , \\
 &
 {\Psi}_{\post} (\bgamma,\bbeta)
 =
 \frac{1}{T_1} \sum_{t=T+0+1}^{T}
 \EXP \bigg[ \frac{\partial}{\partial \bbeta\T} \tau(t \con \bbeta) \big\{ Y_t - \bW_{\D t}\T \bgamma - \tau(t \con \bbeta) \big\} \bigg]
 \ .
\end{align*}
In the constant treatment effect case, $\widehat{\bbeta}_{\lambda}$ and $\bbeta^*$ reduces to
\begin{align*}
&
\widehat{\bbeta}_{\lambda}
=
\frac{1}{T_1}
\sum_{t=T_0+1}^{T}
\big( Y_t - \bW_{\D t} \T \widehat{\bgamma}_\lambda \big)
\ , \
&&
\bbeta^*
=
\frac{1}{T_1} \sum_{t=T+0+1}^{T}
\EXP \big( Y_t - \bW_{\D t}\T \bgamma^* \big)
\ .
\end{align*}



First, we prove that $ (\widehat{\bgamma}_\lambda , \widehat{\bbeta}_\lambda ) \stackrel{P}{\rightarrow} ({\bgamma}^*, {\bbeta}^* )$. This is trivial because, under the asymptotic regime in Regularity Condition \ref{assumption-reg-11}, stationarity (Regularity Condition \ref{assumption-reg-1}), and ergodicity (Regularity Condition \ref{assumption-reg-6}), we have
\begin{align*}
&
\widehat{\bG}_{YY} \stackrel{P}{\rightarrow} \bG_{YY}^*
\ , \
\widehat{\bG}_{YW} \stackrel{P}{\rightarrow} \bG_{YW}^* 
\ , \
\frac{T_1}{T_0} \rightarrow r
\ , \\
&
\frac{1}{T_1} \sum_{t=T_0+1}^{T} Y_t \stackrel{P}{\rightarrow} \frac{1}{T_1} \sum_{t=T_0+1}^{T} \EXP \big( Y_{t} \big)
\ , \
\frac{1}{T_1} \sum_{t=T_0+1}^{T} \bW_{\D t} \stackrel{P}{\rightarrow} \frac{1}{T_1} \sum_{t=T_0+1}^{T} \EXP \big( \bW_{\D t} \big)
\ .
\end{align*}
We then have the following result from the continuous mapping theorem:
\begin{align*}
\widehat{\bgamma}_\lambda
& =
\big( \widehat{\bG}_{YW} \T \widehat{\Omega} \widehat{\bG}_{YW} + \lambda \widehat{\Omega} \big)^{-1}
\big( \widehat{\bG}_{YW} \T \widehat{\Omega} \widehat{\bG}_{YY} \big)
\stackrel{P}{\rightarrow}
\big( {\bG}_{YW} \sT {\Omega}^* {\bG}_{YW}^* \big)^{-1}
\big( {\bG}_{YW} \sT {\Omega}^* {\bG}_{YY}^* \big)
=
\bgamma^* 
\ .
\end{align*}	
Therefore, using the convergence of $\widehat{\bgamma}_\lambda \stackrel{P}{\rightarrow} \bgamma^* $, we can establish the convergence of $\widehat{\bbeta}_\lambda \stackrel{P}{\rightarrow} \bbeta^* $. For instance, under the constant treatment effect case, we have
\begin{align*}	
\widehat{\bbeta}_\lambda
& =
\bigg(
\frac{1}{T_1} \sum_{t=T_0+1}^{T} Y_t
-
\frac{1}{T_1} \sum_{t=T_0+1}^{T} W_{\D t}\T \widehat{\bgamma}_\lambda \bigg)
\stackrel{P}{\rightarrow}
\EXP \big( Y_t - \bW_{\D t}\T \bgamma^* \big)
=
{\bbeta}_\lambda^* \ .
\end{align*}
We can establish the consistency under nonlinear treatment effects using the uniform weak law of large numbers; see \textbf{Proof 2} below for details. 

To establish the asymptotic normality, we use the following Taylor expansion, which holds from the consistency of $\widehat{\btheta}$ and the Regularity Conditions \ref{assumption-reg-11}--\ref{assumption-reg-10}:
\begin{align*} 
&
\widehat{\Psi}( \widehat{\btheta}_{\lambda})
=
\widehat{\Psi}( {\btheta}^*)
+
\bigg\{
\frac{\partial}{\partial \btheta\T} \widehat{\Psi}( \btheta) \bigg|_{\btheta=\btheta^*}
\bigg\}
(\widehat{\btheta}_{\lambda}-\btheta^*)
+
o_P \Big( \big\| \widehat{\btheta}_{\lambda} - \btheta^* \big\| \Big)
\ . 
\end{align*}
Combining the first order condition of $\widehat{\btheta}_{\lambda}$ and regularity conditions on $\Psi(\bO_t \con \btheta)$ and $\partial \Psi(\bO_t \con \btheta)/\partial \btheta\T$, we get
\begin{align*}
0
& =
\frac{1}{2}
\frac{\partial}{\partial \btheta\T} 
\Big[
\big\{ \widehat{\Psi}( \btheta)
\big\}\T \widehat{\Omega} 
\big\{ \widehat{\Psi}( \btheta)
\big\}
+ \lambda \big\| \bgamma \big\|_2^2
\Big] \Big|_{\btheta=\widehat{\btheta}_{\lambda}}
\\
&
=
\bigg\{
\frac{\partial}{\partial \btheta\T} \widehat{\Psi}( \btheta) \bigg|_{\btheta=\btheta^*}
\bigg\}\T
\widehat{\Omega}
\big\{ \widehat{\Psi}( \widehat{\btheta}_{\lambda})
\big\}
+ 
\lambda \widehat{\gamma}_{\lambda}
\\
&
=
\bigg\{
\frac{\partial}{\partial \btheta\T} \widehat{\Psi}( \btheta) \bigg|_{\btheta=\btheta^*}
\bigg\}\T
\widehat{\Omega}
\big\{ \widehat{\Psi}( \btheta^*)
\big\}
+
\bigg\{
\frac{\partial}{\partial \btheta\T} \widehat{\Psi}( \btheta) \bigg|_{\btheta=\btheta^*}
\bigg\}\T
\widehat{\Omega}
\bigg\{
\frac{\partial}{\partial \btheta\T} \widehat{\Psi}( \btheta) \bigg|_{\btheta=\btheta^*}
\bigg\} 
(\widehat{\btheta}-\btheta^*)
\\
&
\hspace*{0.5cm}
+
o_P \Big( \big\| \widehat{\btheta}_{\lambda} - \btheta^* \big\| + T^{-1/2} \Big) \ .
\end{align*}
The last line is from $T^{1/2} \lambda \widehat{\gamma}_{\lambda} = T^{1/2} \lambda ( \widehat{\gamma}_{\lambda} + \gamma^* ) = o_P(1)$. Therefore, by multiplying $T^{1/2}$, we get
\begin{align*}
0
&
=
\bigg\{
\frac{\partial}{\partial \btheta\T} \widehat{\Psi}( \btheta) \bigg|_{\btheta=\btheta^*}
\bigg\}\T
\widehat{\Omega}
\bigg\{
\frac{1}{T^{1/2}}
\sum_{t=1}^{T} \Psi(\bO_t \con \btheta^* )
\bigg\}
\\
&
\hspace*{0.3cm}
+
\bigg\{
\frac{\partial}{\partial \btheta\T} \widehat{\Psi}( \btheta) \bigg|_{\btheta=\btheta^*}
\bigg\}\T
\widehat{\Omega}
\bigg\{
\frac{\partial}{\partial \btheta\T} \widehat{\Psi}( \btheta) \bigg|_{\btheta=\btheta^*}
\bigg\}
T^{1/2}
(\widehat{\btheta}_{\lambda} -\btheta^*)
+
o_P \Big( T^{1/2} \big\| \widehat{\btheta}_{\lambda} - \btheta^* \big\|+ 1 \Big)
\\
&
=
\bigg\{
\underbrace{
\frac{\partial}{\partial \btheta\T} {\Psi}( \btheta) \bigg|_{\btheta=\btheta^*}
}_{=:G^*}
\bigg\}\T 
\Omega^*
\bigg\{
\lim_{T \rightarrow \infty}
\frac{1}{T^{1/2}}
\sum_{t=1}^{T} \Psi(\bO_t \con \btheta^* )
\bigg\}
\\
&
\hspace*{0.3cm}
+
\bigg\{
\frac{\partial}{\partial \btheta\T} {\Psi}( \btheta) \bigg|_{\btheta=\btheta^*}
\bigg\}\T
\Omega^*
\bigg\{
\frac{\partial}{\partial \btheta\T} {\Psi}( \btheta) \bigg|_{\btheta=\btheta^*}
\bigg\} 
T^{1/2}
(\widehat{\btheta}_{\lambda}-\btheta^*)
+
o_P \Big( T^{1/2} \big\| \widehat{\btheta}_{\lambda} - \btheta^* \big\| + 1 \Big) \ .
\end{align*}
This implies
\begin{align*}
T^{1/2} \big( \widehat{\btheta}_{\lambda} - \btheta^* \big)
=
\Big ( G\sT \Omega^* G^* \Big)^{-1}
G\sT \Omega^* 
\bigg\{ 
\frac{1}{T^{1/2}}
\sum_{t=1}^{T} \Psi( \bO_t \con \btheta^* )
\bigg\}
+
o_P \Big( T^{1/2} \big\| \widehat{\btheta}_{\lambda} - \btheta^* \big\| + 1 \Big) \ .
\end{align*}
Since $T^{-1/2}
\sum_{t=1}^{T} \Psi( \bO_t \con \btheta^* )=O_P(1)$, we find $T^{1/2} \big( \widehat{\btheta}_{\lambda} - \btheta^* \big) = O_P(1)$, implying that $o_P\big( T^{1/2} \big\| \widehat{\btheta}_{\lambda} - \btheta^* \big\| + 1 \big) = o_P(1)$. Therefore, from Slutsky's theorem, we find 
\begin{align*}
T^{1/2} \big( \widehat{\btheta}_{\lambda} - \btheta^* \big)
& \text{ converges in distribution to }
N \big( 0, \big \{ G\sT \Omega^* G^* \big\}^{-1}
G\sT \Omega^* \Sigma_2 G^* \Omega^*
\big \{ G\sT \Omega^* G^* \big\}^{-\T} \big)
\ .
\end{align*}
Note that $\big\{ G\sT \Omega^* G^* \big\}^{-1}
G\sT \Omega^* \Sigma_2 G^* \Omega^*
\big \{ G\sT \Omega^* G^* \big\}^{-\T} = \Sigma_1^* \Sigma_2 \Sigma_1\sT $. 
This concludes the proof.





\vspace{0.5cm}
\noindent \textbf{Proof 2: Proof without Strong Stationarity and Ergodicity}
\vspace{0.5cm}



Under General Regularity Conditions \ref{assumption-General-1}--\ref{assumption-General-AN}, we establish the desired result. We simply denote $\widehat{\Psi}(\btheta) = T^{-1} \sum_{t=1}^T \Psi(\bO_t \con \btheta)$ and ${\Psi}(\btheta) = \lim_{T \rightarrow \infty} T^{-1} \sum_{t=1}^T \EXP \big\{ \Psi(\bO_t \con \btheta) \big\}$. First, we establish consistency, i.e., $\widehat{\btheta}_{\lambda} = (\widehat{\bgamma}_{\lambda}, \widehat{\bbeta}_{\lambda}) = (\bgamma^*, \bbeta^*) + o_P(1) = \btheta^* + o_P(1)$. From the triangle inequality, we find
\begin{align*}
&
\left\|
\big\{
\widehat{\Psi}(\btheta)
\big\}\T 
\widehat{\Omega}
\big\{
\widehat{\Psi}(\btheta)
\big\} 
+ \lambda \big\| \bgamma \big\|_2^2
-
\big\{
{\Psi}(\btheta)
\big\}\T 
\Omega^*
\big\{
{\Psi}(\btheta)
\big\}
\right\|
\\
&
\leq
\left\|
\big\{
\widehat{\Psi}(\btheta)
\big\}\T 
\widehat{\Omega}
\big\{
\widehat{\Psi}(\btheta)
\big\} 
-
\big\{
{\Psi}(\btheta)
\big\}\T 
\Omega^*
\big\{
{\Psi}(\btheta)
\big\}
\right\|
+
\lambda \big\| \bgamma \big\|_2^2 \ .
\end{align*}
Note that $\lambda = o(T^{-1/2})$ and $\big\| \bgamma \big\|_2 \leq \text{diam}(\Theta)$ for any $\bgamma$. which implies $\sup_{\btheta \in \Theta}
\lambda \big\| \bgamma \big\|_2^2 = o(1)$. Therefore, combining with \eqref{eq-UWLLN-GMM} which is valid under General Regularity Conditions \ref{assumption-General-3} and \ref{assumption-General-UWLLN}, we obtain
\begin{align} \label{eq-UWLLN-GMM2}
&
\sup_{\btheta \in \Theta}
\left\|
\big\{
\widehat{\Psi}(\btheta)
\big\}\T 
\widehat{\Omega}
\big\{
\widehat{\Psi}(\btheta)
\big\} 
+ \lambda \big\| \bgamma \big\|_2^2
-
\big\{
{\Psi}(\btheta)
\big\}\T 
\Omega^*
\big\{
{\Psi}(\btheta)
\big\}
\right\|
\nonumber
\\
&
\leq
\sup_{\btheta \in \Theta}
\left\|
\big\{
\widehat{\Psi}(\btheta)
\big\}\T 
\widehat{\Omega}
\big\{
\widehat{\Psi}(\btheta)
\big\} 
-
\big\{
{\Psi}(\btheta)
\big\}\T 
\Omega^*
\big\{
{\Psi}(\btheta)
\big\}
\right\|
+
\sup_{\btheta \in \Theta}
\lambda \big\| \bgamma \big\|_2^2
\nonumber
\\
&
= o_P(1) \ .
\end{align} 
Note that $\widehat{\btheta}_{\lambda}$ is the minimizer of $\big\{
\widehat{\Psi}(\btheta)
\big\}\T 
\widehat{\Omega}
\big\{
\widehat{\Psi}(\btheta)
\big\} 
+ \lambda \big\| \bgamma \big\|_2^2 $. 

Let $s > 0$ be an arbitrary positive constant. 
From \eqref{eq-UWLLN-GMM2} and the definition of $\widehat{\btheta}_{\lambda}$, the following conditions hold with probability tending to one:
\begin{align*}
&
\left\|
\big\{
\widehat{\Psi}(\btheta^*)
\big\}\T 
\widehat{\Omega}
\big\{
\widehat{\Psi}(\btheta^*)
\big\} 
+
\lambda \big\| \bgamma^* \big\|_2^2
-
\big\{
\Psi (\btheta^*)
\big\}\T 
\Omega^*
\big\{
\Psi (\btheta^*)
\big\}
\right\| < s/2
\\
&
\left\|
\big\{
\widehat{\Psi}(\widehat{\btheta}_{\lambda})
\big\}\T 
\widehat{\Omega}
\big\{
\widehat{\Psi}(\widehat{\btheta}_{\lambda})
\big\} 
+
\lambda \big\| \widehat{\bgamma}_{\lambda} \big\|_2^2
-
\big\{
\Psi (\widehat{\btheta}_{\lambda})
\big\}\T 
\Omega^*
\big\{
\Psi (\widehat{\btheta}_{\lambda})
\big\}
\right\| < s/2 
\\ 
&
\big\{
\widehat{\Psi} (\widehat{\btheta}_{\lambda})
\big\}\T 
\widehat{\Omega}
\big\{
\widehat{\Psi} (\widehat{\btheta}_{\lambda})
\big\}
+
\lambda \big\| \widehat{\bgamma}_{\lambda} \big\|_2^2
\leq
\big\{
\widehat{\Psi} (\btheta^*)
\big\}\T 
\widehat{\Omega}
\big\{
\widehat{\Psi} (\btheta^*)
\big\}
+
\lambda \big\| \bgamma^* \big\|_2^2
 \ .
\end{align*}
These three inequalities imply that
\begin{align*}
\big\{
\Psi(\widehat{\btheta}_{\lambda})
\big\}\T 
\Omega^*
\big\{
\Psi(\widehat{\btheta}_{\lambda})
\big\} 
<
\big\{
\Psi(\btheta^*)
\big\}\T 
\Omega^*
\big\{
\Psi(\btheta^*)
\big\} 
+ s
=
s \ .
\end{align*} 
The right hand side reduces to $s$ under General Regularity Condition \ref{assumption-General-6}. 

Let $\mathcal{N} \in \Theta$ be an arbitrary open set containing $\btheta^*$. Let us define the following quantity:
\begin{align*}
s_0
=
\inf_{\btheta \in \Theta \setminus \mathcal{N} }
\big\{ 
\Psi(\btheta)
\big\}\T
\Omega^*
\big\{ 
\Psi(\btheta)
\big\} .
\end{align*}
Combining the fact that $\Theta \setminus \mathcal{N}$ is compact under General Regularity Condition \ref{assumption-General-2}, and General Regularity Conditions \ref{assumption-General-2}, \ref{assumption-General-4}, \ref{assumption-General-6}, we establish $s_0$ is positive. Therefore, by taking $s>s_0$, the event $\big\{ \big\{
\Psi(\widehat{\btheta}_{\lambda})
\big\}\T 
\widehat{\Omega}
\big\{
\Psi(\widehat{\btheta}_{\lambda})
\big\}
< s_0 \big\}$ occurs with probability tending to one, which further implies that $\widehat{\btheta}_{\lambda} \in \mathcal{N}$. Since $\mathcal{N}$ is arbitrary chosen, this establishes $\widehat{\btheta}_{\lambda} = \btheta^* + o_P(1)$ as $T \rightarrow \infty$. 

Next, we establish the asymptotic normality of $\widehat{\btheta}_{\lambda}$. Under General Regularity Condition \ref{assumption-General-5}, the following expansion holds from a first-order Taylor expansion and the consistency of $\widehat{\btheta}$:
\begin{align*} 
&
\widehat{\Psi}( \widehat{\btheta}_{\lambda})
=
\widehat{\Psi}( {\btheta}^*)
+
\bigg\{
\frac{\partial}{\partial \btheta\T} \widehat{\Psi}( \btheta) \bigg|_{\btheta=\btheta^*}
\bigg\}
(\widehat{\btheta}_{\lambda}-\btheta^*)
+
o_P \Big( \big\| \widehat{\btheta}_{\lambda} - \btheta^* \big\| \Big)
\ . 
\end{align*}
The first order condition of $\widehat{\btheta}_{\lambda}$ along with General Regularity Condition \ref{assumption-General-5} implies
\begin{align*}
0
& =
\frac{1}{2}
\frac{\partial}{\partial \btheta\T} 
\Big[
\big\{ \widehat{\Psi}( \btheta)
\big\}\T \widehat{\Omega} 
\big\{ \widehat{\Psi}( \btheta)
\big\}
+ \lambda \big\| \bgamma \big\|_2^2
\Big] \Big|_{\btheta=\widehat{\btheta}_{\lambda}}
\\
&
=
\bigg\{
\frac{\partial}{\partial \btheta\T} \widehat{\Psi}( \btheta) \bigg|_{\btheta=\btheta^*}
\bigg\}\T
\widehat{\Omega}
\big\{ \widehat{\Psi}( \widehat{\btheta}_{\lambda})
\big\}
+ 
\lambda \widehat{\gamma}_{\lambda}
\\
&
=
\bigg\{
\frac{\partial}{\partial \btheta\T} \widehat{\Psi}( \btheta) \bigg|_{\btheta=\btheta^*}
\bigg\}\T
\widehat{\Omega}
\big\{ \widehat{\Psi}( \btheta^*)
\big\}
+
\bigg\{
\frac{\partial}{\partial \btheta\T} \widehat{\Psi}( \btheta) \bigg|_{\btheta=\btheta^*}
\bigg\}\T
\widehat{\Omega}
\bigg\{
\frac{\partial}{\partial \btheta\T} \widehat{\Psi}( \btheta) \bigg|_{\btheta=\btheta^*}
\bigg\} 
(\widehat{\btheta}-\btheta^*)
\\
&
\hspace*{0.5cm}
+
o_P \Big( \big\| \widehat{\btheta}_{\lambda} - \btheta^* \big\| + T^{-1/2} \Big) \ .
\end{align*}
The last line is from $T^{1/2} \lambda \widehat{\gamma}_{\lambda} = T^{1/2} \lambda ( \widehat{\gamma}_{\lambda} + \gamma^* ) = o_P(1)$. Therefore, by multiplying $T^{1/2}$, we get
\begin{align*}
0
&
=
\bigg\{
\frac{\partial}{\partial \btheta\T} \widehat{\Psi}( \btheta) \bigg|_{\btheta=\btheta^*}
\bigg\}\T
\widehat{\Omega}
\bigg\{
\frac{1}{T^{1/2}}
\sum_{t=1}^{T} \Psi(\bO_t \con \btheta^* )
\bigg\}
\\
&
\hspace*{0.3cm}
+
\bigg\{
\frac{\partial}{\partial \btheta\T} \widehat{\Psi}( \btheta) \bigg|_{\btheta=\btheta^*}
\bigg\}\T
\widehat{\Omega}
\bigg\{
\frac{\partial}{\partial \btheta\T} \widehat{\Psi}( \btheta) \bigg|_{\btheta=\btheta^*}
\bigg\}
T^{1/2}
(\widehat{\btheta}_{\lambda} -\btheta^*)
+
o_P \Big( T^{1/2} \big\| \widehat{\btheta}_{\lambda} - \btheta^* \big\|+ 1 \Big)
\\
&
=
\bigg\{
\underbrace{
\frac{\partial}{\partial \btheta\T} {\Psi}( \btheta) \bigg|_{\btheta=\btheta^*}
}_{=:G^*}
\bigg\}\T 
\Omega^*
\bigg\{
\lim_{T \rightarrow \infty}
\frac{1}{T^{1/2}}
\sum_{t=1}^{T} \Psi(\bO_t \con \btheta^* )
\bigg\}
\\
&
\hspace*{0.3cm}
+
\bigg\{
\frac{\partial}{\partial \btheta\T} {\Psi}( \btheta) \bigg|_{\btheta=\btheta^*}
\bigg\}\T
\Omega^*
\bigg\{
\frac{\partial}{\partial \btheta\T} {\Psi}( \btheta) \bigg|_{\btheta=\btheta^*}
\bigg\} 
T^{1/2}
(\widehat{\btheta}_{\lambda}-\btheta^*)
+
o_P \Big( T^{1/2} \big\| \widehat{\btheta}_{\lambda} - \btheta^* \big\| + 1 \Big) \ .
\end{align*}
The last equality holds from General Regularity Conditions \ref{assumption-General-3}, \ref{assumption-General-UWLLN}, and \ref{assumption-General-UWLLN2} and the consistency of $\widehat{\btheta}_{\lambda}$. This implies
\begin{align*}
T^{1/2} \big( \widehat{\btheta}_{\lambda} - \btheta^* \big)
=
\Big ( G\sT \Omega^* G^* \Big)^{-1}
G\sT \Omega^* 
\bigg\{ 
\frac{1}{T^{1/2}}
\sum_{t=1}^{T} \Psi( \bO_t \con \btheta^* )
\bigg\}
+
o_P \Big( T^{1/2} \big\| \widehat{\btheta}_{\lambda} - \btheta^* \big\| + 1 \Big) \ .
\end{align*}
From General Regularity Condition \ref{assumption-General-3} and \ref{assumption-General-AN}, we find $T^{1/2} \big( \widehat{\btheta}_{\lambda} - \btheta^* \big) = O_P(1)$, implying that $o_P\big( T^{1/2} \big\| \widehat{\btheta}_{\lambda} - \btheta^* \big\| + 1 \big) = o_P(1)$. Therefore, from Slutsky's theorem, we find 
\begin{align*}
T^{1/2} \big( \widehat{\btheta}_{\lambda} - \btheta^* \big)
& \text{ converges in distribution to }
N \big( 0, \big \{ G\sT \Omega^* G^* \big\}^{-1}
G\sT \Omega^* \Sigma_2 G^* \Omega^*
\big \{ G\sT \Omega^* G^* \big\}^{-\T} \big)
\ .
\end{align*}
Note that $\big\{ G\sT \Omega^* G^* \big\}^{-1}
G\sT \Omega^* \Sigma_2 G^* \Omega^*
\big \{ G\sT \Omega^* G^* \big\}^{-\T} = \Sigma_1^* \Sigma_2 \Sigma_1\sT $. 
This concludes the proof.






 






\newpage

\bibliographystyle{apa}
\bibliography{SPSC.bib}

















\end{document}
