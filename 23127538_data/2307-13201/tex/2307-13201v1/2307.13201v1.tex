\documentclass[10pt]{article}

\usepackage{latexsym, amssymb, amsmath, amsthm, amscd}
\usepackage[all,cmtip]{xy}
\usepackage{amsmath}
\usepackage{amsthm}
\usepackage{amssymb}
\usepackage{amsfonts}
\usepackage{amsrefs}
\usepackage{color,authblk}
\usepackage{tikz}
\usepackage{amscd} 
\usepackage{comment}
\usepackage{mathrsfs}
\usepackage{comment}
\usepackage[all]{xy}
\usepackage{graphicx}
\usepackage{authblk}
\usepackage{hyperref}
\usepackage{bbm}
\usepackage{fullpage}
\usepackage[all,cmtip]{xy}
\usepackage{relsize}
\usepackage{amsmath,amscd}
%\usepackage[title]{appendix}
\usepackage{tikz-cd}
\usepackage{tikz}
\usetikzlibrary{matrix}
\usepackage[mathcal]{euscript}
\usepackage{txfonts}



\newcommand{\impliesdown}{\mathbin{\rotatebox[origin=c]{270}{$\Rightarrow$}}}


\usepackage{hyperref}

\numberwithin{equation}{section} \DeclareMathSizes{2}{10}{12}{13}
\parindent=0.0in

\newcommand{\resp}{{\it resp.\/}\ }
\newcommand{\ie}{{\it i.e.\/}\ }
\newcommand{\cf}{{\it cf.\/}\ }


 \textwidth 7in
\textheight 8.2in

\oddsidemargin -0.2in
\evensidemargin -0.5in 


\makeatletter
\newcommand*{\doublerightarrow}[2]{\mathrel{
  \settowidth{\@tempdima}{$\scriptstyle#1$}
  \settowidth{\@tempdimb}{$\scriptstyle#2$}
  \ifdim\@tempdimb>\@tempdima \@tempdima=\@tempdimb\fi
  \mathop{\vcenter{
    \offinterlineskip\ialign{\hbox to\dimexpr\@tempdima+1em{##}\cr
    \rightarrowfill\cr\noalign{\kern.5ex}
    \rightarrowfill\cr}}}\limits^{\!#1}_{\!#2}}}
\newcommand*{\triplerightarrow}[1]{\mathrel{
  \settowidth{\@tempdima}{$\scriptstyle#1$}
  \mathop{\vcenter{
    \offinterlineskip\ialign{\hbox to\dimexpr\@tempdima+1em{##}\cr
    \rightarrowfill\cr\noalign{\kern.5ex}
    \rightarrowfill\cr\noalign{\kern.5ex}
    \rightarrowfill\cr}}}\limits^{\!#1}}}
\makeatother


\newtheorem{thm}{Proposition}[section]
\newtheorem{Thm}[thm]{Theorem}
\newtheorem{rem}[thm]{Remark}
\newtheorem{defthm}[thm]{Definition-Proposition}
\newtheorem{cor}[thm]{Corollary}
\newtheorem{eg}[thm]{Example}
\newtheorem{lem}[thm]{Lemma}
\newtheorem{defn}[thm]{Definition}

\title{A Gabber type result for representations in Eilenberg-Moore categories}

\author{Divya Ahuja \footnote{Department of Mathematics, Indian Institute of Technology, Delhi, India. Email: divyaahuja1428@gmail.com} $\qquad\qquad$ Abhishek Banerjee 
\footnote{Department of Mathematics, Indian Institute of Science, Bangalore, India. Email: abhishekbanerjee1313@gmail.com} $\qquad\qquad$ 
Samarpita Ray \footnote{Stat-Math Unit, Indian Statistical Institute, Bangalore, India. Email: ray.samarpita31@gmail.com}}

\date{}


\newcommand{\leftexp}[2]{{\vphantom{#2}}^{#1}{#2}}
\newcommand{\leftsub}[2]{{\vphantom{#2}}_{#1}{#2}}

\begin{document}

\maketitle 

\begin{abstract} We consider a representation $\mathscr U:\mathbb Q\longrightarrow Mnd(\mathcal C)$ of a quiver $\mathbb Q$ taking values in monads over a Grothendieck category
$\mathcal C$. By using adjoint functors between Eilenberg-Moore categories, we consider two different kinds of modules over $\mathscr U$. The first is the category $Mod-\mathscr U$ of $\mathscr U$-modules, which behaves like the category of modules over a ringed space. The second is the category $Cart-\mathscr U$ of cartesian modules, which behave like quasi-coherent sheaves.  We give conditions for $Mod-\mathscr U$ and $Cart-\mathscr U$ to be Grothendieck categories. One of our key steps is finding a modulus like bound for an endofunctor $U:\mathcal C\longrightarrow \mathcal C$ in terms of $\kappa(G)$, where $G$ is a generator for $\mathcal C$ and $\kappa(G)$ is a cardinal such that $G$ is $\kappa(G)$-presentable. We conclude with an extension of the classical quasi-coherator construction to modules over a monad quiver with values in Eilenberg-Moore categories.
\end{abstract}

\medskip
{MSC(2020) Subject Classification: 18C20, 18E10}


\medskip
{Keywords:} Monad quivers, Eilenberg-Moore categories, Grothendieck categories

\medskip

\section{Introduction}

Let $Z$ be a scheme. Then, a famous result of Gabber (see, for instance, \cite[Tag 077P]{Stacks}) shows that the category $QCoh(Z)$ of quasi-coherent sheaves over $Z$ is a Grothendieck category. If $S$ is a scheme and $\mathcal Z$ is an algebraic stack over $S$, the category $QCoh(\mathcal Z)$ of quasi-coherent sheaves over $\mathcal Z$ is also a Grothendieck category (see, for instance, \cite[Tag 06WU]{Stacks}). We can ask similar questions in much more general contexts. For example, let $(\mathcal D,\otimes)$ be a monoidal category having an action 
$\_\_\otimes \_\_:\mathcal D\times \mathcal L\longrightarrow \mathcal L$ on a Grothendieck category $\mathcal L$. Then if  $Alg(\mathcal D)$ denotes the category of monoid objects
in $\mathcal D$, we may consider for any $A\in Alg(\mathcal D)$ the category $A-Mod^{\mathcal L}$ of ``left $A$-module objects in $\mathcal L$.'' Then, a morphism in $Alg(\mathcal D)$ induces an adjoint pair of functors between the corresponding module categories with objects in $\mathcal L$. Accordingly, one may set up a theory of quasi-coherent modules, with coefficients in $\mathcal L$, for a representation $\mathcal Y\longrightarrow Alg(\mathcal D)$ of a small category $\mathcal Y$. We can ask for conditions for these quasi-coherent modules to form a Grothendieck category. For instance, if $k$ is a field and $\mathcal L$ is a $k$-linear Grothendieck category, we may consider $R$-module objects in $\mathcal L$ for any $k$-algebra $R$. The latter categories play a key role in the study of noncommutative projective schemes by Artin and Zhang \cite{AZ1}, \cite{AZ2}. 

\smallskip
In this paper, we prove a Gabber type result for representations in Eilenberg-Moore categories of monads. For this, we generalize the usual setup of quasi-coherent sheaves in several different ways. First, we replace the system of affine open subsets of a scheme by a quiver $\mathbb Q=(\mathbb V,\mathbb E)$, i.e., a directed graph $\mathbb Q$ with a set of vertices $\mathbb V$ and a set of edges $\mathbb E$. This is motivated by Estrada and Virili \cite{EV} who studied modules over a representation $\mathcal A:\mathcal X\longrightarrow Add$ of a small category $\mathcal X$ taking values in small preadditive categories. Thereafter, we replace rings by monads over a given Grothendieck category $\mathcal C$. As such, we consider a representation $\mathscr U:\mathbb Q\longrightarrow Mnd(\mathcal C)$ of the quiver $\mathbb Q$ taking values in the category $Mnd(\mathcal C)$ of monads over $\mathcal C$. Finally, we replace the usual module categories over rings by Eilenberg-Moore categories of the monads over $\mathcal C$.  

\smallskip
The heart of Gabber's argument (see, for instance, \cite[Tag 077K]{Stacks}) is showing that that for any scheme $Z$ there exists a cardinal $\kappa$ such that any quasi-coherent sheaf on $Z$ can be expressed as a filtered colimit of $\kappa$-generated quasi-coherent subsheaves. We fix a generator $G$ for the Grothendieck category $\mathcal C$. Then, for any object
$M\in \mathcal C$, the set $el_G(M):=\mathcal C(G,M)$ plays the role of elements of $M$ and we put  $||M||^G:=|\mathcal C(G,M)|$. We choose $\kappa(G)$ such that $G\in \mathcal C$ is 
$\kappa(G)$-presentable. Our first main step is to obtain a modulus like bound on an endofunctor $U:\mathcal C\longrightarrow \mathcal C$, i.e., a cardinal $\lambda^U$ (which depends on the generator $G$) such that for any object $M\in \mathcal C$ we have
\begin{equation}
||UM||^G\leq \lambda^U\times  (||M||^G)^{\kappa(G)}
\end{equation} We refer to a representation $\mathscr U:\mathbb Q\longrightarrow Mnd(\mathcal C)$ as a monad quiver. To study modules over $\mathscr U$, we combine techniques on monads and adapt our methods from earlier work in \cite{Ban}, \cite{BBR} which are inspired by the cardinality arguments of Estrada and Virili \cite{EV}. We also mention that module valued representations of a small category have been studied at several places in the literature (see, for instance, \cite{f13}, \cite{EE}, \cite{f16}, \cite{f17}). If $\phi:U\longrightarrow U'$ is a morphism of monads over $\mathcal C$, there is a pair of adjoint functors
\begin{equation}
\label{1.2fd} \phi^*:EM_{U}\longrightarrow EM_{U'}\qquad \phi_*:EM_{U'}\longrightarrow EM_U
\end{equation} between  Eilenberg-Moore categories $EM_U$ and $EM_{U'}$ of $U$ and $U'$ respectively. 

\smallskip As with a ringed space, there are two different module categories over a representation $\mathscr U:\mathbb Q\longrightarrow Mnd(\mathcal C)$. A $\mathscr U$-module $\mathscr M$ consists of a family of objects $\{\mathscr M_x\in EM_{\mathscr U_x}\}_{x\in Ob(\mathbb Q)}$ along with compatible morphisms $\mathscr M^\alpha:\mathscr U(\alpha)^*\mathscr M_x\longrightarrow \mathscr M_y$ (equivalently, $\mathscr M_\alpha:
\mathscr M_x\longrightarrow \mathscr U(\alpha)_*\mathscr M_y$) for each edge $\alpha\in \mathbb Q(x,y)$. We give conditions for the category $Mod-\mathscr U$ of $\mathscr U$-modules to be a Grothendieck category, and also conditions for  $Mod-\mathscr U$ to have projective generators. We then consider the full subcategory $Cart-\mathscr U$  of $\mathscr U$-modules which are cartesian, i.e., $\mathscr U$-modules $\mathscr M$ for which the morphisms  $\mathscr M^\alpha:\mathscr U(\alpha)^*\mathscr M_x\longrightarrow \mathscr M_y$  are isomorphisms for each edge $\alpha\in  \mathbb Q(x,y)$. It is clear that this definition is inspired by that of quasi-coherent modules over a ringed space. Our main result in this paper gives conditions for 
$Cart-\mathscr U$ to be a Grothendieck category. In that case, the canonical inclusion $Cart-\mathscr U\hookrightarrow Mod-\mathscr U$ has a right adjoint. As such, we  have a generalization of the classical quasi-coherator construction (see \cite[Lemme 3.2]{Ill}) to modules over a monad quiver with values in Eilenberg-Moore categories.

\section{Generators and the bound on an endofunctor}


Throughout this section and the rest of this paper, we assume that $\mathcal C$ is a Grothendieck category. We begin by recalling the following standard definition. 

\begin{defn}\label{D2.1} (see \cite[$\S$ 1.13]{AR}). Let $\kappa$ be a regular cardinal. A partially ordered set $J$ is said to be $\kappa$-directed if every subset of $J$ having cardinality $<\kappa$ has an upper bound in $J$.
An object $M\in \mathcal C$ is said to be $\kappa$-presentable if the functor $\mathcal C(M,\_\_)$ preserves $\kappa$-directed colimits.
\end{defn}

From Definition \ref{D2.1} it is clear that if $J$ is partially ordered set that is $\kappa$-directed, then it is also $\kappa'$-directed for any regular cardinal
$\kappa'\leq \kappa$.  Accordingly,  if an object $M\in  \mathcal C$ is $\kappa$-presentable, then $M$ is also $\kappa''$-presentable for any regular cardinal
$\kappa''\geq \kappa$. 


\smallskip We now fix a generator $G$ for $\mathcal C$. Because $\mathcal C$ is a  Grothendieck category, it is also locally presentable (see, for instance, \cite[Proposition 3.10]{Beke}) and it follows in particular that for each object $M\in  \mathcal C$ we can choose $\kappa(M)$ such that $M$ is $\kappa(M)$-presentable.  We choose therefore $\kappa(G)$ such that $G$ is
$\kappa(G)$-presentable. By the above reasoning, we may suppose that $\kappa(G)$ is infinite.

\smallskip
For each $M\in \mathcal C$, we now define
\begin{equation}\label{cardM}
el_G(M):=\mathcal C(G,M)\qquad ||M||^{G}:=|\mathcal C(G,M)|
\end{equation} From \eqref{cardM}, it is immediately clear that if $M'\hookrightarrow M$ is a monomorphism in $\mathcal C$, then 
$||M'||^{ G}\leq ||M||^{ G}$. For the rest of this paper, we will assume that the generator $G$  is such that
for any epimorphism $M\twoheadrightarrow M''$ in $\mathcal C$, we must have $||M''||^{ G}\leq ||M||^{ G}$. This would happen, for instance, if  $G$ were projective.

\smallskip
For a set $S$ and a regular cardinal $\alpha$, we denote by $\mathcal P_\alpha(S)$ the collection of subsets of $S$ having cardinality $<\alpha$. Since $\alpha$ is regular, we note that $\mathcal P_\alpha(S)$ is $\alpha$-directed. If $\{M_s\}_{s\in S}$ is a collection of objects of $\mathcal C$ indexed by $S$ and $T\subseteq S$ is any subset, we denote by $M_{T}$
the direct sum $M_{T}:=\underset{s\in T}{\bigoplus}M_s$. 

\begin{lem}\label{L2.1} Let $\{M_s\}_{s\in S}$ be a family of objects in $\mathcal C$. Let $\lambda$, $\mu\geq \aleph_0$ be cardinals such that
  \begin{equation} \lambda\geq \mbox{max$\{\mbox{$|S|$, $\kappa(G)$}\}$} \qquad  \mu\geq \mbox{sup$\{\mbox{ $||M_s||^G$, $s\in S$}\}$}
\end{equation} Then, $||\underset{s\in S}{\bigoplus}M_s||^{ G}\leq
 \mu^{\kappa(G)}\times \lambda^{\kappa(G)}$.
\end{lem}

\begin{proof}
We consider a subset $T\in \mathcal P_{\kappa(G)}(S)$. Then, we have
\begin{equation}\label{2.2dq}
||M_T||^G=|\mathcal C(G,M_T)|\leq \left\vert \mathcal C\left(G,\underset{s\in T}{\prod}M_s\right)\right\vert=\left\vert\underset{s\in T}{\prod}\mathcal C(G,M_s)\right\vert\leq \mu^{|T|}\leq \mu^{\kappa(G)}
\end{equation} We now note that the direct sum $M_S=\underset{s\in S}{\bigoplus}M_s$ may be expressed as the colimit $\underset{T\in \mathcal P_{\kappa(G)}(S)}{\varinjlim}M_T$. Since this colimit is $\kappa(G)$-directed and $G$ is $\kappa(G)$-presentable, we now have
\begin{equation}\label{2.3eq}
||M_S||^G=|\mathcal C(G,M_S)|=\left\vert\mathcal C\left(G,\underset{T\in \mathcal P_{\kappa(G)}(S)}{\varinjlim}M_T\right)\right\vert=\left\vert\underset{T\in \mathcal P_{\kappa(G)}(S)}{\varinjlim}\mathcal C(G,M_T)\right\vert
\end{equation} Since there is an epimorphism 
$
 \underset{T\in \mathcal P_{\kappa(G)}(S)}{\bigoplus}\mathcal C(G,M_T)\twoheadrightarrow \underset{T\in \mathcal P_{\kappa(G)}(S)}{\varinjlim}\mathcal C(G,M_T)
$ in the category of abelian groups, it follows from \eqref{2.2dq} and \eqref{2.3eq} that
\begin{equation}
||M_S||^G\leq\left\vert \underset{T\in \mathcal P_{\kappa(G)}(S)}{\bigoplus}\mathcal C(G,M_T) \right\vert \leq  \mu^{\kappa(G)}\times \lambda^{\kappa(G)}
\end{equation}  The last inequality follows from the fact that $|\mathcal P_{\kappa(G)}(S)|\leq |\mathcal P_{{\kappa(G)}^+}(S)|=|S|^{\kappa(G)}\leq \lambda^{\kappa(G)}$, where ${\kappa(G)}^+$ is the successor of $\kappa(G)$ 
 (see, for instance, \cite[$\S$ 8.2]{Hod}).
\end{proof}

\begin{Thm}\label{T2.2} Let $U:\mathcal C\longrightarrow \mathcal C$ be an endofunctor that preserves colimits. Let $\lambda^U:={(||UG||^G)}^{\kappa(G)}\times \kappa(G)^{\kappa(G)}$. Then, 
$||UM||^G\leq \lambda^U\times  (||M||^G)^{\kappa(G)}$ for any object $M\in \mathcal C$. 

\end{Thm}

\begin{proof}
Since $G$ is a generator, we know that for any $M\in \mathcal C$, the canonical morphism  $G^{\mathcal C(G,M)}\longrightarrow M$ is an epimorphism. Since $U$ preserves colimits, it follows that we have an epimorphism $(UG)^{\mathcal C(G,M)}\twoheadrightarrow UM$ in $\mathcal C$. By the assumption on the generator $G$, it follows that $||UM||^G\leq \left\vert\left\vert (UG)^{\mathcal C(G,M)}\right\vert\right\vert^G$.
Applying Lemma \ref{L2.1} with $\mu=\mbox{max$\{||UG||^G,\aleph_0\}$}$ and $\lambda=\mbox{max$\{\kappa(G), ||M||^G\}$}$, we have
\begin{equation}
||UM||^G\leq \left\vert\left\vert (UG)^{\mathcal C(G,M)}\right\vert\right\vert^G\leq \mu^{\kappa(G)}\times \lambda^{\kappa(G)}\leq {(||UG||^G)}^{\kappa(G)}\times \kappa(G)^{\kappa(G)}\times  \aleph_0^{\kappa(G)} \times  (||M||^G)^{\kappa(G)}
\end{equation} Since $\kappa(G)$ is infinite, the result is now clear. 
\end{proof}


\section{Generators in  Eilenberg-Moore categories} 

We continue with $\mathcal C$ being a Grothendieck category. By definition, a monad $(U,\theta,\eta)$ on $\mathcal C$ is a triple consisting of an endofunctor $U:\mathcal C\longrightarrow \mathcal C$ and natural transformations $\theta: U\circ U\longrightarrow U$, $\eta: 1_{\mathcal C}
\longrightarrow U$ satisfying associativity and unit conditions similar to usual multiplication. A module $(M,f_M)$ over $(U,\theta,\eta)$ consists of  $M\in \mathcal C$ and a morphism  $f_M:UM\longrightarrow M$ in $\mathcal C$ such that the following compatibilities hold.
\begin{equation}\label{3.1cq}
f_M\circ\theta M= f_M\circ Uf_M~~\textup{and}~~f_M\circ\eta_M= 1_M
\end{equation}
%\begin{equation}\label{3.1cq}
%\begin{array}{lll}
%\begin{CD}
%U^2M @>\theta M>> UM \\
%@VUf_MVV @VVf_MV \\
%UM @>f_M>> M \\
%\end{CD} & \qquad & \begin{tikzcd}
% & UM \arrow[dr,"f_M"] \\
%M \arrow[ur,"\eta M"] \arrow[rr,"1_M"] && M
%\end{tikzcd} \\
%\end{array}
%\end{equation} 
A morphism $g:(M,f_M)\longrightarrow (M',f_{M'})$ of $(U,\theta,\eta)$-modules is given by $g:M\longrightarrow M'$ in $\mathcal C$ such that 
$f_{M'}\circ Ug= g\circ f_M$.  This gives the standard Eilenberg-Moore category of modules over the monad   $(U,\theta,\eta)$  and we denote it by $EM_U$. When there is no danger of confusion, an object $(M,f_M)\in EM_U$ will often be denoted simply by $M$.

\smallskip For any object $M\in \mathcal C$, we note that $(UM,\theta M:U^2M\longrightarrow UM)$ carries the structure of a module over $(U,\theta,\eta)$. Further, it is well known (see, for instance, \cite{Mac}) that there is an adjunction of functors, given by  natural isomorphisms
\begin{equation}\label{monadj}
EM_U(UM,N)\cong \mathcal C(M,N)
\end{equation}
for $M\in \mathcal C$ and $N\in EM_U$.

\smallskip
\begin{thm}\label{L3.1}
Let $(U,\theta,\eta)$ be a monad on $\mathcal C$ such that $U$ is exact and preserves colimits. Then, $EM_U$ is a Grothendieck  category.  If $\mathcal C$ has a projective generator, so does $EM_U$.
Further, if $\{M_i\}_{i\in I}$ is any system (resp. any finite system) of objects in $EM_U$, the colimit (resp. the finite limit) in $EM_U$ is defined by taking $\underset{i\in I}{colim}\textrm{ }M_i$ (resp.  $\underset{i\in I}{lim}\textrm{ }M_i$) in $\mathcal C$. 
\end{thm}
\begin{proof} Let $g:(M,f_M)\longrightarrow (N,f_N)$ be a morphism in $EM_U$. We set
\begin{equation}
K:=Ker(g:M\longrightarrow N)\qquad L=Coker(g:M\longrightarrow N)
\end{equation} Since $U$ is exact, it is clear that we have induced morphisms $f_K:UK\longrightarrow K$ and $f_L:UL\longrightarrow L$ defining objects
$(K,f_K)$, $(L,f_L)\in EM_U$. It follows that $EM_U$ contains kernels and cokernels and that $Ker((N,f_N)\longrightarrow Coker(g))=Coker(Ker(g)
\longrightarrow (M,f_M))$. This makes $EM_U$ an abelian category.  Since $U$ is exact and preserves colimits, we see that $U$ can be used to determine both colimits and finite limits in $EM_U$, and that  $EM_U$ satisfies the (AB5) axiom.


\smallskip
Let $G$ be a generator for $\mathcal C$ and let $(M,f_M)\in EM_U$.  We choose 
an epimorphism $p:G^{(X)}\longrightarrow M$ in $\mathcal C$ from a direct sum of copies of $G$. Since $U$ preserves colimits,  $Up: UG^{(X)}=(UG)^{(X)}\longrightarrow UM$ is an epimorphism in $EM_U$. Additionally, it is clear from the condition $f_M\circ\eta_M= 1_M$  in \eqref{3.1cq} that $f_M:UM\longrightarrow M$ is an epimorphism in $EM_U$. Therefore, $f_M\circ Up: UG^{(X)}=(UG)^{(X)}\longrightarrow UM\longrightarrow M$ is an epimorphism in $EM_U$ and it follows  that $(UG,\theta G)$ is a generator for $EM_U$. Finally, if $G\in \mathcal C$ is projective, then $EM_U((UG,\theta G),\_\_)\cong \mathcal C(G,\_\_)$ is exact and $(UG,\theta G)$ becomes projective in $EM_U$.
\end{proof}


\begin{lem}\label{L3.2}
Suppose that $(U,\theta,\eta)$ is  a monad on $\mathcal C$ which is exact and preserves colimits.  Let $M\in \mathcal C$ be an object and suppose that $M$ is $\kappa(M)$-presentable as an object of $\mathcal C$. Then, $(UM,\theta M)$ is $\kappa(M)$-presentable as an object of $EM_U$. 
\end{lem}

\begin{proof}
Let $\{N_i\}_{i\in I}$ be a system of objects in $EM_U$ that is $\kappa(M)$-directed. By Lemma \ref{L3.1}, we know that the underlying object of $N:=\underset{i\in I}{colim}
\textrm{ }N_i$  in $EM_U$ is given by taking the colimit in $\mathcal C$. We now see that
\begin{equation}\label{3.4cy}
EM_U(UM,N)\cong \mathcal C\left(M, {\underset{i\in I}{colim}}\textrm{ }N_i\right)={\underset{i\in I}{colim}}\textrm{ }\mathcal C(M,N_i)={\underset{i\in I}{colim}}\textrm{ }
EM_U(UM,N_i)
\end{equation} The result is  now clear.
\end{proof}

\begin{thm}\label{P3.3} Suppose that $(U,\theta,\eta)$ is  a monad on $\mathcal C$ which is exact and preserves colimits.  Let $G$ be a generator for $\mathcal C$ that is $\kappa(G)$-presentable. Then, $EM_U$ is a locally $\kappa(G)$-presentable category.
\end{thm}

\begin{proof} From the proof of Lemma \ref{L3.1}, we know that the pair $(UG,\theta G)$ is a generator for the Eilenberg-Moore category $EM_U$. Since $G$ is $\kappa(G)$-presentable, it follows from Lemma \ref{L3.2} that $(UG,\theta G)$ is $\kappa(G)$-presentable as an object of $EM_U$. Hence, $EM_U$ is locally $\kappa(G)$-presentable. 
\end{proof}

For the sake of convenience, we now fix a regular cardinal
$
 \delta\geq \mbox{max$\{\kappa(G), ||G||^G\}$} 
$.  We note in particular that since $\delta\geq \kappa(G)$, the object $G$ is also $\delta$-presentable. If the monad $U$ preserves colimits, we  use Theorem \ref{T2.2} to fix  $\lambda^U = {(||UG||^G)}^{\kappa(G)}\times \kappa(G)^{\kappa(G)}$ such that $||UM||^G\leq \lambda^U\times  (||M||^G)^{\kappa(G)}$ for any object $M\in \mathcal C$. 

\begin{thm}\label{P3.4} Suppose that $(U,\theta,\eta)$ is  a monad on $\mathcal C$ which is exact and preserves colimits.  Let $(M,f_M)\in EM_U$ and consider some $x\in el_G(M)$. Then, there is a subobject $N_x\subseteq M$ in $EM_U$ such that $||N_x||^G\leq \lambda^U$ and $x\in el_G(N_x)$. 

\end{thm}

\begin{proof}
By definition, $x\in el_G(M)=\mathcal C(G,M)$. By \eqref{monadj}, we have a corresponding morphism $\hat{x}\in EM_U(UG,M)\cong \mathcal C(G,M)$ given by setting 
$\hat{x}:UG \xrightarrow{Ux}UM\xrightarrow{f_M}M$. By setting $N_x:=Im(\hat{x})$ in $EM_U$, we obtain the commutative diagram
\begin{equation}\label{3.6cd}
\begin{tikzcd}
&    &  {UM}  \arrow{ddr}{f_M} &\\
 &   &   &\\[-2ex]
G\arrow{r}{\eta G}&   UG\arrow{rr}[swap]{\hat{x}} \arrow{uur}{Ux} \arrow{ddr}{p_x} & & M \\
&     &   &\\[-2ex] 
& &  N_x  \arrow{uur}{i_x} &\\
\end{tikzcd}
\end{equation} We note that since the composition $G\xrightarrow{\eta G}UG \xrightarrow{Ux}UM\xrightarrow{f_M}M$ gives back $x:G\longrightarrow M$, it follows from \eqref{3.6cd} that
$x\in el_G(N_x)\subseteq el_G(M)$. Finally, since  $p_x:UG\longrightarrow N_x$ is an epimorphism, it follows   that
$
||N_x||^G\leq ||UG||^G\leq \lambda^U
$. This proves the result.
\end{proof}

\begin{Thm}\label{T3.5} Suppose that $(U,\theta,\eta)$ is  a monad on $\mathcal C$ which is exact and preserves colimits.  Let $(M,f_M)\in EM_U$ and consider  some $X\subseteq  el_G(M)$. Then, there is a subobject $N_X\subseteq M$ in $EM_U$ such that $||N_X||^G\leq\lambda^U\times \delta^\delta\times {|X|}^\delta$ and $X\subseteq el_G(N_X)$. 

\end{Thm}


\begin{proof}
By considering the morphisms $x\in X\subseteq el_G(M)=\mathcal C(G,M)$, we obtain $h_X:G^{(X)}\longrightarrow M$ from a direct sum of copies of $G$. Since $ \delta\geq \mbox{max$\{\kappa(G), ||G||^G\}$}$, it follows from Lemma \ref{L2.1} that
\begin{equation}
||G^{(X)}||^G\leq \delta^{\kappa(G)}\times (|X|\times \kappa(G))^{\kappa(G)}\leq \delta^\delta\times {|X|}^\delta
\end{equation}
 By the adjunction in \eqref{monadj} and the fact that $U$ preserves direct sums, we obtain $\hat{h}_X\in EM_U((UG)^{(X)},M)\cong \mathcal C(G^{(X)},M)$ and a commutative diagram
\begin{equation}\label{3.7cd}
\begin{tikzcd}
&    &  {UM}  \arrow{ddr}{f_M} &\\
 &   &   &\\[-2ex]
&   (UG)^{(X)}\arrow{rr}[swap]{\hat{h}_X} \arrow{uur}{Uh_X} \arrow{ddr}{p_X} & & M \\
&     &   &\\[-2ex] 
& &  N_X  \arrow{uur}{i_X} &\\
\end{tikzcd}
\end{equation} 
by setting $N_X:=Im(\hat{h}_X)$ in $EM_U$. As with \eqref{3.6cd} in the proof of Proposition \ref{P3.4}, it follows from that \eqref{3.7cd}  that $X\subseteq el_G(N_X)\subseteq el_G(M)$. Again, since $U$ preserves colimits and $p_X:(UG)^{(X)}\longrightarrow N_X$ is an epimorphism, it follows  that
\begin{equation}
||N_X||^G\leq ||(UG)^{(X)}||^G\leq \lambda^U\times (||G^{(X)}||^G)^{\kappa(G)}\leq \lambda^U\times  (\delta^\delta\times {|X|}^\delta)^\delta=\lambda^U\times \delta^\delta\times {|X|}^\delta
\end{equation}
\end{proof}

\section{Modules over a monad quiver}


We continue with $\mathcal C$ being a Grothendieck category as before. We suppose from now on that the generator $G$ of $\mathcal C$ is projective. A morphism $\phi:(U,\theta,\eta)\longrightarrow (U',\theta',\eta')$ of monads over $\mathcal C$ is a natural transformation $\phi:U\longrightarrow U'$ that satisfies
\begin{equation}
\phi\circ \theta =\theta'\circ (\phi\ast\phi): U \circ U \longrightarrow U'\qquad \eta'=\phi\circ \eta: 1\longrightarrow U'
\end{equation} This forms the category $Mnd(\mathcal C)$ of monads over $\mathcal C$. A morphism $\phi: U\longrightarrow U'$ of monads induces a restriction functor 
\begin{equation}\label{forg4}
\phi_*:EM_{U'}\longrightarrow EM_U \qquad (M',f_{M'})\mapsto (M',f_{M'}\circ \phi(M'))
\end{equation}
Additionally, given $(M,f_M)\in EM_U$, we set
\begin{equation}\label{4.3qi}
\phi^*(M):=Coeq\left(U'UM\doublerightarrow{\qquad\theta'(M)\circ (U'\phi(M))\qquad }{U'f_M}U'M\right)
\end{equation} This determines a functor $\phi^*:EM_{U}\longrightarrow EM_{U'}$ that is left adjoint to $\phi_*$ (see, for instance, \cite[Proposition 1]{Lint}). If $U$ and $U'$ are exact and preserve colimits, we know from Proposition \ref{L3.1}  that $(UG,\theta G)$ and $(U'G,\theta' G)$ are generators for $EM_U$ and $EM_{U'}$ respectively. We note that for any $(M',f_{M'})\in EM_{U'}$, we have natural isomorphisms
\begin{equation}\label{4.4dk}
EM_{U'}(\phi^*(UG,\theta G),(M',f_{M'}))\cong  EM_{U}((UG,\theta G),\phi_*(M',f_{M'}))\cong \mathcal C(G,M')\cong EM_{U'}((U'G,\theta' G),(M',f_{M'}))
\end{equation} whence it follows by Yoneda lemma that $\phi^*(UG,\theta G)=(U'G,\theta' G)\in EM_{U'}$. 

\begin{defn}\label{D4.0}
Let $\phi:(U,\theta,\eta)\longrightarrow (U',\theta',\eta')$  be a morphism of monads over $\mathcal C$. We will say that $\phi$ is flat if the functor
$\phi^*:EM_U\longrightarrow EM_{U'}$ is exact.
\end{defn}

We now recall that a quiver $\mathbb Q=(\mathbb V,\mathbb  E)$ is a directed graph, consisting of a set of $\mathbb V$ of vertices and a set $\mathbb E$ of edges. We will use $\phi: x\longrightarrow y$ to denote an arrow in $\mathbb Q$ going from $x$ to $y$. We will treat a quiver $\mathbb Q$ as a category in the obvious manner.

\begin{defn}\label{D4.1} Let  $\mathbb Q=(\mathbb V,\mathbb  E)$ be a quiver. A monad quiver over $\mathcal C$ is a functor $\mathscr U:\mathbb Q\longrightarrow Mnd(\mathcal C)$. We will say $\mathscr U$ is flat if for each arrow $\phi:x\longrightarrow y$ in $\mathbb Q$, the induced morphism $\mathscr U(\phi):\mathscr U(x)\longrightarrow \mathscr U(y)$ of monads is flat. For $x\in \mathbb V$, we will often denote the monad $\mathscr U(x)$ by $\mathscr U_x$. 
\end{defn}

If $\mathscr U$ is a monad quiver over $\mathcal C$ and $\phi: x\longrightarrow y$ is an edge of $\mathbb Q$, by abuse of notation, we will continue to denote $\mathscr U(\phi):
\mathscr U_x\longrightarrow \mathscr U_y$ simply by $\phi$. Accordingly, we have a pair of adjoint functors 
$
\phi^*=\mathscr U(\phi)^*:EM_{\mathscr U_x}\longrightarrow EM_{\mathscr U_y}$ and $\phi_*=\mathscr U(\phi)_*:EM_{\mathscr U_y}\longrightarrow EM_{\mathscr U_x}
$ for each edge $\phi:x\longrightarrow y$ in $\mathbb Q$. 

\begin{defn}\label{D4.2} Let $\mathscr U:\mathbb Q=(\mathbb V,\mathbb  E)\longrightarrow Mnd(\mathcal C)$ be a monad quiver over $\mathcal C$. A $\mathscr U$-module $\mathscr M$ consists of a collection $\{\mathscr M_x\in EM_{\mathscr U_x}\}_{x\in \mathbb V}$ connected by morphisms $\mathscr M_\phi: \mathscr M_x\longrightarrow \phi_*\mathscr M_y$ in $EM_{\mathscr U_x}$ (equivalently, morphisms $\mathscr M^\phi:\phi^*\mathscr M_x\longrightarrow \mathscr M_y$ in $EM_{\mathscr U_y}$) for each edge $\phi: x\longrightarrow y$ in $\mathbb E$ such that $\mathscr M_{id_x}=id_{\mathscr M_x}$ for each $x\in \mathbb V$ and 
$\phi_*(\mathscr M_\psi)\circ \mathscr M_\phi=\mathscr M_{\psi\phi}:\mathscr M_x\xrightarrow{\mathscr M_\phi} \phi_*\mathscr M_y
\xrightarrow{\phi_*(\mathscr M_\psi)}\phi_*\psi_*\mathscr M_z$ (equivalently, $\mathscr M^\psi\circ \psi^*(\mathscr M^\phi)=\mathscr M^{\psi\phi}$)
for each pair of composable morphisms $x\xrightarrow{\phi}y\xrightarrow{\psi}z$ in $\mathbb Q$.

\smallskip
A morphism $\xi:\mathscr M\longrightarrow \mathscr M'$ of $\mathscr U$-modules consists of morphisms $\xi_x:\mathscr M_x\longrightarrow \mathscr M'_x$ for each $x\in \mathbb V$ such that $\mathscr M'_\phi\circ \xi_x=\phi_*(\xi_y)\circ \mathscr M_\phi $ (equivalently, $\mathscr M'^\phi\circ \phi^*(\xi_x)=\xi_y\circ \mathscr M^\phi$) for each edge $\phi: x\longrightarrow y$ in $\mathbb E$. We denote the category of $\mathscr U$-modules by
$Mod-\mathscr U$.

\smallskip
Additionally, we say that  $\mathscr M \in Mod-\mathscr U$  is cartesian if for each edge $x \xrightarrow{\psi} y$ in $\mathbb E$, the morphism $\mathscr M^{\psi} : \psi^{*}\mathscr M_x \longrightarrow \mathscr M_y$ is an isomorphism in $EM_{\mathscr U_y}$. We denote by $Cart-\mathscr U$ the full subcategory of $Mod-\mathscr U$ consisting of cartesian modules. 
\end{defn}

From now onward, we assume that the functor $\mathscr U:\mathbb Q\longrightarrow Mnd(\mathcal C)$ takes values in monads which are exact and preserve colimits. From the definition in \eqref{forg4} and the proof of Proposition \ref{L3.1}, it is now clear that the restriction functors $\phi_*=\mathscr U(\phi)_*:EM_{\mathscr U_y}\longrightarrow EM_{\mathscr U_x}
$ are exact for each edge $\phi:x\longrightarrow y$. Further, $\mathscr U-Mod$ becomes an abelian category, with kernel $Ker(\xi)_x=Ker(\xi_x)$ and
$Coker(\xi)_x=Coker(\xi_x)$ computed pointwise for any morphism $\xi:\mathscr M\longrightarrow \mathscr M'$. 

\smallskip
For a $\mathscr U$-module $\mathscr M$, we now set 
\begin{equation}\label{el45t}
el_G(\mathscr M):=\underset{x\in \mathbb V}{\coprod}EM_{\mathscr U_x}(\mathscr U_xG,\mathscr M_x)=\underset{x\in \mathbb V}{\coprod}\mathscr C(G,\mathscr M_x)
\end{equation} From \eqref{el45t} it is clear that for any subobject $\mathscr M'\subseteq \mathscr M$ in $\mathscr U-Mod$, we must have
$el_G(\mathscr M')\subseteq el_G(\mathscr M)$. Additionally, since $\mathscr U_xG$ is a generator for $EM_{\mathscr U_x}$, we see that
the subobject $\mathscr M'\subseteq \mathscr M$ equals $\mathscr M$ if and only if $el_G(\mathscr M')=el_G(\mathscr M)$. 

\smallskip
We will now use an argument similar to our previous work in \cite{Ban}, \cite{BBR} which is motivated by the work of Estrada and Virili \cite{EV}. For this, we start by fixing some $\zeta\in el_G(\mathscr M)$, i.e., $\zeta:\mathscr U_xG\longrightarrow \mathscr M_x$ is a morphism in $EM_{\mathscr U_x}$ for some $x\in \mathbb V$. For each $y\in \mathbb V$, we set
\begin{equation}\label{start4}
\mathscr P_y:=Im\left(\underset{\psi\in \mathbb Q(x,y)}{\bigoplus}\psi^*\mathscr U_xG\xrightarrow{\psi^*\zeta}\psi^*\mathscr M_x\xrightarrow{\mathscr M^\psi}\mathscr M_y\right)=\underset{\psi\in \mathbb Q(x,y)}{\sum}Im\left(\psi^*\mathscr U_xG\xrightarrow{\psi^*\zeta}\psi^*\mathscr M_x\xrightarrow{\mathscr M^\psi}\mathscr M_y\right)\in EM_{\mathscr U_y}
\end{equation} For each $y\in \mathbb V$, let $\iota_y: \mathscr P_y\hookrightarrow \mathscr M_y$ be the inclusion. For each $\psi\in 
\mathbb Q(x,y)$, we have a canonical morphism $\zeta'_\psi:\psi^*\mathscr U_xG\longrightarrow\mathscr P_y$ determined by \eqref{start4}.

\begin{thm}\label{P4.5}
For an edge $y\xrightarrow{\phi}z$ in $\mathbb{Q},$ the morphism $\mathscr{M}_{\phi} : \mathscr{M}_{y} \longrightarrow \phi_{*}\mathscr{M}_{z}$ restricts to a morphism $\mathscr{P}_{\phi} : \mathscr{P}_{y} \longrightarrow \phi_{*}\mathscr P_z$ such that 
\begin{equation}
\phi_*(\iota_z)\circ\mathscr P_\phi=\mathscr M_\phi\circ\iota_y
\end{equation}
%\begin{equation}
%\begin{CD}
%\mathscr{P}_{y} @>\mathscr P_{\phi} >> \phi_{*}\mathscr{P}_z \\
%@V\iota_{y}VV @VV\phi_{*}(\iota_z)V \\
%\mathscr{M}_{y}@>\mathscr{M}_\phi>> \phi_{*}\mathscr{M}_{z} \\
%\end{CD}
%\end{equation} 
As such, the objects $\{\mathscr{P}_x \in EM_{\mathscr{U}_{x}}\}_{x \in \mathbb{V}}$ together determine a subobject $\mathscr{P} \subseteq \mathscr{M}$ in $Mod-\mathscr U$.
\end{thm}
\begin{proof}
Since $\iota_{z} : \mathscr{P}_{z} \hookrightarrow \mathscr{M}_{z}$ is a monomorphism and $\phi_{*}$ is a right adjoint, $\phi_{*}(\iota_z) $ is also a monomorphism. We claim that the composition $\mathscr P_y\xrightarrow{\iota_y}\mathscr M_y \xrightarrow{\mathscr M_\phi}\phi_*(\mathscr M_z)$ factors through $\iota_{z} : \mathscr{P}_{z} \hookrightarrow \mathscr{M}_{z}$.  Since $\mathscr{U}_yG$ is a projective generator for the Grothendieck category $EM_{\mathscr U_y}$, it suffices  (see \cite[Lemma $3.2$]{Ban}) to show that for any morphism $\tau : \mathscr{U}_{y}G \longrightarrow \mathscr{P}_y$, there exists a morphism $\tau' : \mathscr{U}_{y}G \longrightarrow \phi_{*}\mathscr{P}_z$ such that $\phi_{*}(\iota_z) \circ \tau' = \mathscr{M}_{\phi} \circ \iota_{y}\circ \tau.$ By \eqref{start4}, there is an epimorphism \begin{equation}
\underset{\psi\in \mathbb Q(x,y)}{\bigoplus} \zeta'_{\psi}: \underset{\psi\in \mathbb Q(x,y)}{\bigoplus} \psi^{*}\mathscr{U}_{x}G \longrightarrow \mathscr{P}_y
\end{equation}
in $EM_{\mathscr U_y}$. Since $\mathscr{U}_{y}G$ is projective in $EM_{\mathscr U_y}$, we can lift the morphism $\tau : \mathscr U_yG \longrightarrow \mathscr P_y$ to  $\tau'' : \mathscr U_yG \longrightarrow \underset{\psi\in \mathbb Q(x,y)}{\bigoplus}  \psi^{*}\mathscr U_xG$ such that 
$
\tau = \left(\underset{\psi\in \mathbb Q(x,y)}{\bigoplus} \zeta'_{\psi}\right) \circ \tau''
$.
By \eqref{start4}, we know that for each $\psi\in \mathbb Q(x,y)$, the composition $\psi^{*}\mathscr U_{x}G \xrightarrow{\zeta'_{\psi}} \mathscr P_y \xrightarrow{\iota_{y}} \mathscr M_y$ factors through $\psi^{*}\mathscr M_x$ as 
$
\iota_y \circ \zeta'_{\psi} = \mathscr{M}^\psi \circ \psi^{*}\zeta.
$ 
Then applying $\phi^{*}$ and composing with $\mathscr M^{\phi},$ we get \begin{equation}
\mathscr{M}^{\phi} \circ \phi^{*}(\iota_{y}) \circ \phi^{*}(\zeta'_{\psi}) = \mathscr{M}^{\phi} \circ \phi^{*}(\mathscr M^{\psi}) \circ \phi^{*}(\psi^{*}\zeta) = \mathscr{M}^{\phi\psi} \circ \phi^{*}\psi^{*}\zeta
\end{equation}
which clearly factors through $\iota_{z} : \mathscr{P}_z \longrightarrow \mathscr M_z.$ Since $(\phi^{*}, \phi_{*})$ is an adjoint pair, it follows that the composition $\psi^*\mathscr{U}_xG \xrightarrow{\zeta'_\psi}\mathscr P_y\xrightarrow{\iota_y}\mathscr M_y \xrightarrow{\mathscr{M}_{\phi}}\phi_{*}\mathscr{M}_z$ factors through   $\phi_{*}(\iota_{z}) : \phi_{*} \mathscr P_{z} \longrightarrow \phi_{*}\mathscr{M}_{z}.$ The result is now clear.
\end{proof}


\begin{lem}\label{L4.5dk}
Let $\zeta'_{1} : \mathscr U_{x}G \longrightarrow \mathscr{P}_x$ be the canonical morphism corresponding to the identity map in $\mathbb{Q}(x,x)$. Then, for any $y \in \mathbb{V}$, we have 
\begin{equation}
\mathscr{P}_y = Im\left(\underset{\psi\in \mathbb Q(x,y)}{\bigoplus}\psi^*\mathscr U_xG\xrightarrow{\psi^*\zeta'_1}\psi^*\mathscr P_x \xrightarrow{\mathscr P^\psi}\mathscr P_y\right)
\end{equation}
\end{lem}
\begin{proof}
Let $x\xrightarrow{\psi}y$ be an edge in $\mathbb Q$. We consider the following commutative diagram
\begin{equation}
\begin{CD}
\psi^{*}\mathscr{U}_xG @>\psi^{*}\zeta'_{1} >> \psi^{*}\mathscr P_x @>\mathscr P^{\psi} >> \mathscr P_{y} \\
@. @VV\psi^{*}(\iota_{x})V @VV\iota_{y}V \\
@.\psi^{*}\mathscr M_x @>\mathscr M^{\psi}>> \mathscr{M}_y \\
\end{CD}
\end{equation} 
Clearly, $\iota_x \circ \zeta'_{1} = \zeta$. Applying $\psi^{*}$, composing with $\mathscr{M}^{\psi}$ and using the fact that $\iota_y$ is monic, we obtain
\begin{equation}
Im(\mathscr{M}^{\psi} \circ \psi^{*}\zeta) = Im(\mathscr{M}^{\psi} \circ \psi^{*} (\iota_{x})\circ \psi^{*}\zeta'_{1}) = Im(\iota_{y} \circ \mathscr{P}^{\psi}\circ \psi^{*}\zeta'_{1}) = Im(\mathscr{P}^{\psi} \circ \psi^{*}\zeta'_{1})
\end{equation}
The result now follows from \eqref{start4}.
\end{proof}

We now fix an infinite regular cardinal $\gamma$ such that
\begin{equation}\label{crd4rf}
\gamma\geq sup\{\mbox{$|Mor(\mathbb Q)|$, $\kappa(G)$, $||\mathscr U_yG||^G$, $y\in Ob(\mathbb Q)$} \}
\end{equation}

\begin{lem}\label{L4.6tte}
We have $|el_G(\mathscr P)|\leq  \gamma^{\kappa(G)}$.
\end{lem}

\begin{proof}
For each $\psi\in  \mathbb Q(x,y)$, we know that $\psi^*\mathscr U_xG=\mathscr U_yG\in EM_{\mathscr U_y}$. From Lemma \ref{L4.5dk}, it now follows that $\mathscr P_y$ is a quotient of $\underset{\psi\in \mathbb Q(x,y)}{\bigoplus}\mathscr U_yG$. We recall that $\mathscr U_yG$ is projective in $EM_{\mathscr U_y}$. Using Lemma \ref{L2.1} and the assumption in \eqref{crd4rf}, we now see that
\begin{equation}\label{413ed}
|EM_{\mathscr U_y}(\mathscr U_yG,\mathscr P_y)|\leq \left\vert EM_{\mathscr U_y}\left(\mathscr U_yG,\underset{\psi\in \mathbb Q(x,y)}{\bigoplus}\mathscr U_yG\right)\right\vert=\left\vert \mathcal C\left(G,\underset{\psi\in \mathbb Q(x,y)}{\bigoplus}\mathscr U_yG\right)\right\vert\leq \gamma^{\kappa(G)}
\end{equation} From the definition in \eqref{el45t} and the assumption in \eqref{crd4rf}, the result is now clear.
\end{proof}

\begin{Thm}\label{Th4.7b}
Let  $\mathscr U:\mathbb Q\longrightarrow Mnd(\mathcal C)$ be a monad quiver taking values in monads which are exact and preserve colimits. Then, the category $Mod-\mathscr U$ of $\mathscr U$-modules is a Grothendieck category.
\end{Thm}

\begin{proof} Both filtered colimits and finite limits in $Mod-\mathscr U$ are computed pointwise at each vertex $x\in \mathbb V$. Hence, they commute with each other and $Mod-\mathscr U$ satisfies (AB5). We take $\mathscr M\in Mod-\mathscr U$ and some $\zeta\in el_G(\mathscr M)$, given by $\zeta: \mathscr U_xG\longrightarrow \mathscr M_x$ for some $x\in \mathbb V$. We consider the subobject $\mathscr P\subseteq \mathscr M$ corresponding to $\zeta$ as in 
Proposition \ref{P4.5}. From the definition in \eqref{start4}, we know that $\zeta\in el_G(\mathscr P)$. From Lemma \ref{L4.6tte}, we know that 
 $|el_G(\mathscr P)|\leq  \gamma^{\kappa(G)}$. 

\smallskip
By Proposition \ref{3.1cq}, each $EM_{\mathscr U_x}$ is a Grothendieck category, and hence well-powered. Since $\mathscr U_xG$ is a generator for 
$EM_{\mathscr U_x}$, the object $\mathscr M'_x$ for any $\mathscr M'\in Mod-\mathscr U$ can be expressed as a quotient of $(\mathscr U_xG)^{EM_{\mathscr U_x}(\mathscr U_xG,\mathscr M'_x)}$ over some subobject. Hence, the isomorphism classes of $\mathscr U$-modules 
$\mathscr M'$ satisfying $|el_G(\mathscr M')|\leq \gamma^{\kappa(G)}$ form a set. It is now clear that this  collection gives a set of generators
for $Mod-\mathscr U$.
\end{proof}

We conclude this section by giving several examples of situations where the framework of monad quivers would apply. Let $k$ be a field.  We use Sweedler notation for coproducts and
coactions, with summation symbols suppressed. 

\smallskip
(1) Let $Alg_k$ denote the category of $k$-algebras. Each $A\in Alg_k$ defines a monad $A\otimes_k\_\_$ on the category $Vect_k$ of $k$-vector spaces. If $T:\mathbb Q\longrightarrow Alg_k$ is any functor, we see that $\mathscr U:\mathbb Q\longrightarrow Mnd(Vect_k)$, $x\mapsto T(x)\otimes_k\_\_$ becomes a monad quiver. For $\mathscr M\in Mod-\mathscr U$ and any $x\in \mathbb Q$, the $\mathscr U_x$-module $\mathscr M_x$ takes values in the category of left $T(x)$-modules. 

\smallskip
(2) Let $(\mathcal D,\otimes)$ be a $k$-linear monoidal category and let $\mathcal L$ be a $k$-linear Grothendieck category along with an action $\_\_\otimes \_\_:\mathcal D\times 
\mathcal L\longrightarrow \mathcal L$ such that the functor $X\otimes \_\_:\mathcal L\longrightarrow \mathcal L$ is exact and preserves colimits for any $X\in \mathcal D$. Then, any monoid object
$A\in Alg(\mathcal D)$ determines a monad $A\otimes\_\_:\mathcal L\longrightarrow \mathcal L$.  If $T:\mathbb Q\longrightarrow Alg(\mathcal D)$ is any functor, we see that $\mathscr U:\mathbb Q\longrightarrow Mnd(\mathcal L)$, $x\mapsto T(x)\otimes \_\_$ becomes a monad quiver. For $\mathscr M\in Mod-\mathscr U$ and any $x\in \mathbb Q$, the $\mathscr U_x$-module $\mathscr M_x$ takes values in the category of ``left $T(x)$-module objects in $\mathcal L$.''  For instance, we may take $\mathcal D=Vect_k$. Then, any $k$-algebra 
$R$ determines a monad $R\otimes \_\_:\mathcal L\longrightarrow \mathcal L$ on $\mathcal L$ and the categories of ``$R$-module objects in $\mathcal L$'' play a key role in the theory of noncommutative projective schemes studied by Artin and Zhang \cite{AZ1}, \cite{AZ2}. 

\smallskip
(3)  Let $H$ be a Hopf algebra over $k$. Then, the category $H-Mod$ of left $H$-modules is monoidal, with $H$-action on the tensor product  given by $h(m\otimes n):=h_{(1)}m\otimes h_{(2)}n$ for $h\in H$, 
$m\in M$, $n\in N$ where $M$, $N\in H-Mod$. Let $A$ be an $H$-module algebra, i.e., a monoid object in $H-Mod$. Then, $A\otimes_k\_\_:H-Mod\longrightarrow H-Mod$ is a monad on $H-Mod$ that is exact and preserves colimits. If $T:\mathbb Q\longrightarrow Alg(H-Mod)$ is a functor taking values in the category $Alg(H-Mod)$ of monoids in $H-Mod$, it is clear that
\begin{equation}
\mathscr U:\mathbb Q\longrightarrow Mnd(H-Mod)\qquad x\mapsto T(x)\otimes_k\_\_:H-Mod\longrightarrow H-Mod
\end{equation} determines a monad quiver over $H-Mod$. For $\mathscr M\in Mod-\mathscr U$ and any $x\in \mathbb Q$, the $\mathscr U_x$-module $\mathscr M_x$ takes values in the   category of left $T(x)$-module objects in $H-Mod$.

\smallskip
(4) Let $H$ be a Hopf algebra over $k$ and let $Comod-H$ be the category of right $H$-comodules. If $A$ is a right $H$-comodule algebra, the category  $Mod_A^H$of 
right $(A,H)$-Hopf modules has been extensively studied in the literature (see, for instance, \cite{BBR0}, \cite{CMZ}, \cite{CMIZ}, \cite{CG}).  An object $M\in Mod_A^H$ has a right $A$-module structure and a right $H$-comodule structure that are compatible in the sense that \begin{equation} (ma)_{(0)}\otimes (ma)_{(1)}=m_{(0)}a_{(0)}\otimes m_{(1)}a_{(1)} \qquad m\in M, a\in A
\end{equation} We know that $Mod_A^H$ is a Grothendieck category (see \cite[$\S$ 1]{CG}). For any right $H$-comodule algebra
$B$ and any $M\in Mod_A^H$, it may be verified that $B\otimes M\in Mod_A^H$ with $A$-action $(b\otimes m)\cdot a:=b\otimes ma$ and $H$-coaction $(b\otimes m)_{(0)}\otimes 
(b\otimes m)_{(1)}=b_{(0)}\otimes m_{(0)}\otimes b_{(1)}m_{(1)}$ for $a\in A$, $b\in B$ and $m\in M$. Accordingly, any such $B\otimes_k\_\_:Mod_A^H\longrightarrow Mod_A^H$ is a monad that is exact and preserves colimits. Its Eilenberg-Moore category consists of right $(B^{op}\otimes A,H)$-Hopf modules. 

\smallskip If $T:\mathbb Q\longrightarrow Alg(Comod-H)$ is a functor taking values in the category $Alg(Comod-H)$ of right $H$-comodule algebras, we see that 
$
\mathscr U:\mathbb Q\longrightarrow Mnd(Mod_A^H)$, $x\mapsto T(x)\otimes_k\_\_:Mod_A^H\longrightarrow Mod_A^H
$ determines a monad quiver on $Mod_A^H$.   For $\mathscr M\in Mod-\mathscr U$ and any $x\in \mathbb Q$, the $\mathscr U_x$-module $\mathscr M_x$ takes values in the category of right $(T(x)^{op}\otimes A,H)$-Hopf modules. 


\smallskip
(5) Let $(\mathcal D,\otimes)$ be a multitensor category, i.e., a locally finite $k$-linear abelian rigid monoidal category (see \cite[$\S$ 4.1]{Et}). Let $\mathcal L$ be a  locally
finite $k$-linear abelian category that carries the structure $\otimes:\mathcal D\times \mathcal L\longrightarrow \mathcal L$ of a $\mathcal D$-module category with $\_\_\otimes \_\_$ being exact in the first variable (see \cite[$\S$ 7.3]{Et}). In this situation, it can be shown (see \cite[$\S$ 4.2.1, $\S$ 7.3]{Et}) that the functors $\otimes:\mathcal D\times \mathcal D\longrightarrow \mathcal D$ and $\otimes:\mathcal D\times \mathcal L\longrightarrow \mathcal L$
are exact in both variables. As such, if $A\in Alg(\mathcal D)$ is a monoid object in $\mathcal D$, the functor $A\otimes\_\_:\mathcal L\longrightarrow \mathcal L$ determines a monad
on $\mathcal L$ that is exact. 

\smallskip
In this setup, the category $\mathcal L$ is locally finite (see \cite[$\S$ 1.8]{Et}) and therefore does not contain arbitrary direct sums. Accordingly, we consider   the ind-completion 
$Ind(\mathcal L)$ of $\mathcal L$. Since $\mathcal L$ is essentially small, $Ind(\mathcal L)$ must be a Grothendieck category (see \cite[Theorem 8.6.5]{KS}). For $A\in Alg(\mathcal D)$, 
the monad $A\otimes\_\_:\mathcal L\longrightarrow \mathcal L$ extends canonically to a monad $A\otimes\_\_:Ind(\mathcal L)\longrightarrow Ind(\mathcal L)$ on $Ind(\mathcal L)$. Since
 $A\otimes\_\_:\mathcal L\longrightarrow \mathcal L$  is exact, so is  $A\otimes\_\_:Ind(\mathcal L)\longrightarrow Ind(\mathcal L)$  (see \cite[Corollary 8.6.8]{KS}).  By the universal property of the ind-completion, we know that the extension $A\otimes\_\_:Ind(\mathcal L)\longrightarrow Ind(\mathcal L)$ preserves filtered colimits. Since every colimit can be expressed as a combination of a finite colimit and a filtered colimit (see, for instance, \cite[Tag 002P]{Stacks}), it now follows that $A\otimes\_\_:Ind(\mathcal L)\longrightarrow Ind(\mathcal L)$ preserves all colimits. Now if 
 $T:\mathbb Q\longrightarrow Alg(\mathcal D)$ is any functor, we note that $\mathscr U:\mathbb Q\longrightarrow Mnd(Ind(\mathcal L))$, $x\mapsto T(x)\otimes\_\_:Ind(\mathcal L)
 \longrightarrow Ind(\mathcal L)$ gives a monad quiver on $Ind(\mathcal L)$.
 
 \smallskip
 This particular setup of a locally finite module category $\mathcal L$ over a multitensor category $\mathcal D$ is especially interesting, since it has a large number of naturally occurring examples
 in the literature (see \cite[$\S$ 7.4]{Et}).
 
\smallskip
(a) Let $(\mathcal E,\otimes)$  be a multitensor category and let $(\mathcal D,\otimes)$ be a multitensor subcategory. Then, $\mathcal E$ carries the structure of a $\mathcal D$-module category in an obvious manner. More generally, if $F:(\mathcal D,\otimes)\longrightarrow (\mathcal E,\otimes)$ is a tensor functor between multitensor categories, then $\mathcal E$ carries the structure of a $\mathcal D$-module category with $X\otimes Y:=F(X)\otimes Y$ for $X\in \mathcal D$, $Y\in \mathcal E$. 

\smallskip
(b) Let $G$ be a finite group. Then,  the category $Rep(G)$ of finite dimensional representations of $G$ over a field $k$ is a multitensor category (see \cite[$\S$ 4.1.2]{Et}). If $H\subseteq G$ is a subgroup, then the restriction 
$Rep(G)\longrightarrow Rep(H)$ is a tensor functor, which makes $Rep(H)$ into a $Rep(G)$-module category.

\smallskip
(c) Let $G$ be a finite group and let $Vec_G$ be the category of finite dimensional $G$-graded $k$-vector spaces. Then, $Vec_G$ is a multitensor category (see \cite[$\S$ 4.1.2]{Et}). A  module category $\mathcal L$ over $Vec_G$ is a category with a $G$-action, i.e., there are autoequivalences (see \cite[$\S$ 7.4.10]{Et}) $F_g:\mathcal L\longrightarrow \mathcal L$, $g\in G$ along with isomorphisms
\begin{equation*}
\eta_{g,h}:F_g\circ F_h\longrightarrow F_{gh} \qquad g,h\in G
\end{equation*} satisfying $\eta_{gh,k}\circ \eta_{g,h}=\eta_{g,hk}\circ \eta_{h,k}$ for $g,h,k\in G$.

\section{Projective generators in $Mod-\mathscr{U}$}
In this section, we assume that the quiver $\mathbb Q=(\mathbb V,\mathbb E)$ is a partially ordered set. We continue with the functor $\mathscr U:\mathbb Q\longrightarrow Mnd(\mathcal C)$ taking values in monads that are exact and preserve colimits. Our objective is to show that $Mod-\mathscr U$ has projective generators. We begin by constructing a pair of adjoint functors $ex_x: EM_{\mathscr U_x} \longrightarrow Mod-\mathscr U$ and $ev_{x}:Mod-\mathscr U\longrightarrow EM_{\mathscr U_x}$ for each $x\in\mathbb V.$ 
\begin{thm}\label{P5.1}
Let $x\in\mathbb V$. Then,

\smallskip
(1) There is a functor $ex_x: EM_{\mathscr U_x} \longrightarrow Mod-\mathscr U$ defined by setting for each $M\in EM_{\mathscr U_x}$ and $y\in \mathbb V$:
\begin{equation}
ex_{x}(M)_{y}=\left\{\begin{array}{ll} \psi^* M & \mbox{if $\psi\in\mathbb Q(x,y)$} \\
0 & \mbox{if $\mathbb Q(x,y) = \emptyset$}\\
\end{array}\right.
\end{equation}

\smallskip
(2) The evaluation $ev_{x}:Mod-\mathscr U\longrightarrow EM_{\mathscr U_x}$, $\mathscr M\longrightarrow \mathscr M_x$ gives an exact functor.

\smallskip
(3) $(ex_{x},ev_{x})$ is a pair of adjoint functors.
\end{thm}
\begin{proof}

\smallskip
(1) Clearly, $ex_{x}(M)_{y} \in EM_{\mathscr U_{y}}$. Let $\phi:y\longrightarrow y'$ be an edge in $\mathbb Q$. If $x\nleq y$, then $0=ex_{x}(M)^\phi:0=\phi^*ex_{x}(M)_{y}\longrightarrow ex_{x}(M)_{y'}$ in $EM_{\mathscr U_{y'}}$. Otherwise, if there is $\psi:x\longrightarrow y$ and $\rho:x\longrightarrow y'$, then since $\phi\circ\psi=\rho$, we have
\begin{equation}
id=ex_{x}(M)^\phi:\phi^*ex_{x}(M)_{y}=\phi^*\psi^*M\longrightarrow \rho^*M=ex_{x}(M)_{y'}
\end{equation}
 in $EM_{\mathscr U_{y'}}$. Therefore, for each pair of composable morphisms $\phi,\varphi$ in $\mathbb Q$, we have $ex_{x}(M)^{\varphi\phi} = ex_{x}(M)^{\varphi}\circ\varphi^{*}(ex_{x}(M)^{\phi}).$
 
\smallskip
(2) Clearly, $ev_{x}$ is a functor. Further, since finite limits and finite colimits in $Mod-\mathscr U$ are computed pointwise, $ev_{x}$ is exact. 

\smallskip
(3) Given $M\in EM_{\mathscr U_{x}}$ and $\mathscr{P}\in Mod-\mathscr U$, we will show that $Mod-\mathscr U(ex_{x}(M),\mathscr P)\cong EM_{\mathscr U_x}(M,ev_{x}(\mathscr P)).$ We start with a morphism $f:M\longrightarrow \mathscr P_{x}$ in $EM_{\mathscr U_x}.$ Then we  define $\xi^f:ex_{x}(M)\longrightarrow \mathscr P$ by setting for each $y\in\mathbb Q$:
\begin{equation}
\xi_{y}^f:ex_{x}(M)_y=\psi^*M\xrightarrow{\psi^*f}\psi^*\mathscr P_{x}\xrightarrow{\mathscr P^{\psi}} \mathscr P_y
\end{equation}
whenever $x\leq y$ and $\psi\in\mathbb Q(x,y)$ and $\xi_{y}^f=0$ otherwise. Now for an edge $\phi:y\longrightarrow y'$ in $\mathbb Q$, we will show that $\mathscr P^\phi \circ \phi^*\xi_{y}^f=\xi_{y'}^f\circ ex_{x}(M)^\phi$. If $x\nleq y$, then $ex_{x}(M)_y=0$ and the equality holds. Otherwise, consider $\psi\in\mathbb Q(x,y)$ and $\rho=\phi\circ\psi:x\longrightarrow y'\in \mathbb Q(x,y')$. Then, we have the following commutative diagram
\begin{equation}
\begin{CD}
\phi^*ex_{x}(M)_y=\phi^*\psi^*M @>\phi^*(\mathscr P^\psi\circ\psi^*f) >> \phi^{*}\mathscr{P}_y \\
@V id VV @VV\mathscr P^{\phi}V \\
\phi^*\psi^*M=\rho^*M@>\mathscr P^{\rho}\circ\rho^*(f)=\mathscr P^{\phi\psi}\circ\phi^*\psi^*f>> \mathscr{P}_{y'} \\
\end{CD}
\end{equation} 
which shows that $\xi^f$ is a morphism in $Mod-\mathscr U$. Conversely, if $\xi:ex_{x}(M)\longrightarrow \mathscr P$ is a morphism in $Mod-\mathscr U$, then we have an induced morphism $f^\xi:M\longrightarrow\mathscr P_x$ in $EM_{\mathscr U_x}$. It may be verified directly that these two associations are inverse to each other.
\end{proof}
\noindent
We also record here the fact that the functor $ev_{x}:Mod-\mathscr U\longrightarrow EM_{\mathscr U_x}$ has a right adjoint.
\begin{thm}
Let $x\in\mathbb V$. Then the functor $ev_{x}:Mod-\mathscr U\longrightarrow EM_{\mathscr U_x}$ has a right adjoint $coe_x: EM_{\mathscr U_x} \longrightarrow Mod-\mathscr U$ given as follows for $M\in  EM_{\mathscr U_x}$ and $y\in\mathbb V$:
\begin{equation}
coe_{x}(M)_{y}=\left\{\begin{array}{ll} \psi_* M & \mbox{if $\psi\in\mathbb Q(y,x)$} \\
0 & \mbox{if~$\mathbb Q(y,x) = \emptyset$}\\
\end{array}\right.
\end{equation}
\end{thm}
\begin{proof}
It is clear that $coe_{x}(M)_{y}\in EM_{\mathscr U_y}$ for each $y\in\mathbb V$. Now, consider an edge $\phi:y'\longrightarrow y$. If $y\nleq x$ then $coe_x(M)_\phi=0$. Otherwise, if we have edges $\psi:y\longrightarrow x$ and $\rho:y'\longrightarrow x$, then, since $\psi\circ\phi=\rho$, we get $id=coe_{x}(M)_\phi:\rho_*(M)\longrightarrow\phi_*\psi_*(M)$. It follows that $coe_{x}(M)\in Mod-\mathscr U$. The adjunction $(ev_x,coe_x)$ can now be shown as in the proof of Proposition \ref{P5.1}(3). 
\end{proof}
\begin{cor}\label{C5.3}
Let $x\in\mathbb V$. Then the functor $ex_{x}: EM_{\mathscr U_x} \longrightarrow Mod-\mathscr U$ preserves projectives.
\end{cor}
\begin{proof}
By Proposition \ref{P5.1}, we know that   $(ex_{x},ev_{x})$ is an adjoint pair   and that the right adjoint functor $ev_x$ is exact. It therefore follows that the left adjoint $ex_{x}$ preserves projective objects.
\end{proof}
\begin{Thm}\label{Th5.4}
Let $\mathbb Q$ be a poset and  $\mathscr U:\mathbb Q\longrightarrow Mnd(\mathcal C)$ be a monad quiver taking values in monads which are exact and preserve colimits. Then, the category $Mod-\mathscr U$ has a set of projective generators.
\end{Thm}
\begin{proof}
By the proof of Proposition \ref{L3.1}, we know that for any  $x\in\mathbb V$, $\mathscr U_xG$ is a projective generator in $EM_{\mathscr U_{x}}$. Using Corollary \ref{C5.3}, it now follows that each $ex_{x}(\mathscr U_xG)$ is projective in $Mod-\mathscr U$. We will now show that the family
\begin{equation}
\mathcal{G}=\{ex_{x}(\mathscr U_xG)~\vert~x\in\mathbb{V}\}
\end{equation}
is a set of generators for $Mod-\mathscr U$. We start with a monomorphism $\iota:\mathscr N\hookrightarrow \mathscr M$ in $Mod-\mathscr U$ such that $\mathscr N\subsetneq \mathscr M$. We know that kernels and cokernels in $Mod-\mathscr U$ are computed pointwise. Hence, there exists some $x\in\mathbb V$ such that $\iota_x:\mathscr N_x\hookrightarrow\mathscr M_x$ is a monomorphism with $\mathscr N_x\subsetneq\mathscr M_x$. Since $\mathscr U_xG$  is a generator of $EM_{\mathscr U_x}$, we may choose a morphism $f:\mathscr U_xG\longrightarrow \mathscr M_x$ in $EM_{\mathscr U_x}$ which does not factor through $\iota_x:\mathscr N_x\hookrightarrow\mathscr M_x$. Since
 $(ex_{x},ev_{x})$ is an adjoint pair, we obtain a morphism $\xi^f:ex_{x}(\mathscr U_xG)\longrightarrow \mathscr M$ such that $\xi^f$ does not factor through $\iota:\mathscr N\longrightarrow\mathscr M$. It now follows from \cite[\S1.9]{Gro} that $\mathcal G$ is a set of generators for $Mod-\mathscr U$. 
\end{proof}
\section{Cartesian modules over a monad quiver}

We continue with $\mathbb Q$ being a poset and the functor $\mathscr U:\mathbb Q\longrightarrow Mnd(\mathcal C)$ taking values in monads that are exact and preserve colimits.  
Suppose additionally that $\mathscr U:\mathbb Q\longrightarrow Mnd(\mathcal C)$ is flat, i.e., for any edge $\psi:x\longrightarrow y$ in $\mathbb Q$, the functor  $\psi^*:EM_{\mathscr U_{x}}\longrightarrow EM_{\mathscr U_{y}}$  is exact.  Let $\xi : \mathscr M \longrightarrow \mathscr M'$ be a morphism in $Cart-\mathscr U$. It follows that $Ker(\xi)$, $Coker(\xi)\in Cart-\mathscr U$, where  $Ker(\xi)_x=Ker(\xi_x)$ and
$Coker(\xi)_x=Coker(\xi_x)$ for each $x \in \mathbb V$.  We see therefore that $Cart-\mathscr U$ is an abelian category. 

\smallskip
We continue with $\gamma\geq sup\{\mbox{$Mor(\mathbb Q)$, $\kappa(G)$, $||\mathscr U_yG||^G$, $y\in Ob(\mathbb Q)$} \}$ as in \eqref{crd4rf}. For an endofunctor $U:\mathcal C\longrightarrow \mathcal C$ as in Theorem \ref{T2.2}, we recall that we have $\lambda^U$ such that $||UM||^G\leq \lambda^U\times  (||M||^G)^{\kappa(G)}$ for any object $M\in \mathcal C$. In this section, we only consider monads which are exact and preserve colimits.

\begin{lem}\label{L6.1}
Let $\phi:(U,\theta,\eta)\longrightarrow (U',\theta',\eta')$  be a flat morphism of monads over $\mathcal C$. Let $\alpha\geq \gamma,\lambda^U$. Let $(M,f_M)\in EM_U$ and let $X\subseteq el_G(\phi^*M)$ be a subset such that $|X|\leq \alpha$.  Then, there exists a subobject $N\subseteq M$ in $EM_U$ such that
$||N||^G\leq  \alpha^\gamma$ and $X\subseteq \phi^*N$. 
\end{lem}

\begin{proof}
We choose $x\in X\subseteq \mathcal C(G,\phi^*M)$ and consider the corresponding morphism $\hat{x}\in EM_{U'}(U'G,\phi^*M)$. Since $(UG,\theta G)$ is a generator for $EM_U$, we can choose an epimorphism $p:(UG)^{(I)}\longrightarrow M$ in $EM_U$ from a direct sum of copies of $UG$. As noted in \eqref{4.4dk}, we know that $\phi^*(UG)=U'G$.  Since $\phi^*$ is a left adjoint, we have an induced epimorphism $\phi^*(p):(U'G)^{(I)}=\phi^*((UG)^{(I)})\longrightarrow
\phi^*M$. 

\smallskip
Since $U'G$ is projective in $EM_{U'}$, we  may now lift $\hat{x}:U'G\longrightarrow \phi^*M$ over $\phi^*(p)$ to obtain $\zeta_x: U'G\longrightarrow (U'G)^{(I)}=\phi^*((UG)^{(I)})$ such that $\hat x=\phi^*(p)\circ \zeta_x$. Since $\gamma\geq\kappa(G)$, we know by  Lemma \ref{L3.2} that $U'G$ is $\gamma$-presentable in $EM_{U'}$. Accordingly, we may find a subset $J_x\subseteq I$ with $|J_x|<\gamma$ such that $\zeta_x$ factors through the direct sum $(U'G)^{(J_x)}$. We now have a diagram in $EM_{U'}$.
\begin{equation}\label{61cd}
\begin{tikzcd}
U'G \arrow{ddrr}{\zeta_x} \arrow{rr}{} \arrow{dd}{\hat{x}} & & (U'G)^{(J_x)}=\phi^*((UG)^{(J_x)})\arrow{dd}{} \\
& &\\
\phi^*M  & & \arrow{ll}{\phi^*(p)} (U'G)^{(I)}=\phi^*((UG)^{(I)}) \\
\end{tikzcd}
\end{equation} From \eqref{61cd}, we have a morphism $\xi_x: (UG)^{(J_x)}\longrightarrow (UG)^{(I)}\longrightarrow M$ such that $\hat{x}$ factors through $\phi^*(\xi_x)$. In $EM_{U}$, we now set
\begin{equation}\label{62sq}
N:=Im\left(\xi:= \underset{x\in X}{\bigoplus} \xi_x:\underset{x\in X}{\bigoplus} (UG)^{(J_x)}\longrightarrow M \right)\subseteq M
\end{equation} By assumption, $\phi^*:EM_{U}\longrightarrow EM_{U'}$ is exact. Additionally, since $\phi^*$ is a left adjoint, we have
\begin{equation}\label{63sq}
\phi^*N:=Im\left(\phi^*(\xi)=\underset{x\in X}{\bigoplus} \phi^*\xi_x:\underset{x\in X}{\bigoplus} \phi^*((UG)^{(J_x)})\longrightarrow \phi^*M \right)
\end{equation} By \eqref{61cd}, we see that each $x\in X$ lies in the image $\phi^*N$. 
It remains to show that $||N||^G\leq \alpha^\gamma$. By definition, $||UG||^G\leq \lambda^U={(||UG||^G)}^{\kappa(G)}\times \kappa(G)^{\kappa(G)}$. Applying Lemma \ref{L2.1}, we now obtain
\begin{equation}
||N||^G \leq || \underset{x\in X}{\bigoplus} (UG)^{(J_x)}||^G \leq (\lambda^U)^{\kappa(G)}\times (\alpha\times \gamma)^{\kappa(G)}\leq (\lambda^U)^\gamma\times \gamma^\gamma\times \alpha^\gamma= \alpha^\gamma
\end{equation} where the last equality follows from the fact that $\alpha\geq \gamma,\lambda^U$. 
\end{proof}

\begin{lem}\label{L6.2}
Let $\phi:(U,\theta,\eta)\longrightarrow (U',\theta',\eta')$  be a flat morphism of monads over $\mathcal C$ and let $(M,f_M)\in EM_U$. Let $\alpha\geq \gamma,\lambda^U,\lambda^{U'}$. Let $X\subseteq el_G(M)$ and $Y\subseteq el_G(\phi^*M)$ be subsets such that $|X|, |Y|\leq \alpha^\gamma$.  Then, there exists a subobject $N\subseteq M$ in $EM_U$ such that

\smallskip
(1) $X\subseteq el_G(N)$ and $Y\subseteq el_G(\phi^*N)$. 

\smallskip
(2)  $||N||^G\leq  \alpha^\gamma$ and $||\phi^*N||^G\leq \alpha^\gamma$. 
\end{lem}

\begin{proof} Applying Lemma \ref{L6.1} to the morphism $\phi:(U,\theta,\eta)\longrightarrow (U',\theta',\eta')$, we obtain $N_1\subseteq M$ in $EM_U$ with 
$||N_1||^G\leq (\alpha^\gamma)^\gamma=\alpha^\gamma$  such that $Y\subseteq el_G(\phi^*N_1)$. Applying Lemma \ref{L6.1} again, this time to the identity morphism
on $(U,\theta,\eta)$, we obtain $N_2\subseteq M$ in $EM_U$ such that $||N_2||^G\leq (\alpha^\gamma)^\gamma=\alpha^\gamma$  such that $X\subseteq el_G(N_2)$.  We set $N:=N_1+N_2\subseteq M$ in $EM_U$. We note that
\begin{equation}
X\subseteq el_G(N_2)\subseteq el_G(N)\qquad Y\subseteq el_G(\phi^*N_1) \subseteq el_G(\phi^*N)
\end{equation} where the second relation follows from the fact that $\phi^*$ is exact, which gives $\phi^*N_1\subseteq \phi^*N$ in $EM_{U'}$. Since $N=N_1+N_2$, we have an epimorphism $N_1\oplus N_2\twoheadrightarrow N$. Accordingly, we have
\begin{equation}\label{6.6ty}
||N||^G\leq ||N_1\oplus N_2||^G\leq \alpha^\gamma
\end{equation} It remains to show that $||\phi^*N||^G\leq \alpha^\gamma$.  For this, we note that by the definition in \eqref{4.3qi}, we have
\begin{equation}\label{6.7ty}
\phi^*(N):=Coeq\left(U'UN\doublerightarrow{\qquad\qquad}{}U'N\right)
\end{equation} In particular, this means that there is an epimorphism $U'N\twoheadrightarrow \phi^*N$ in $\mathcal C$. By Theorem \ref{T2.2}, we know that $||U'N||^G\leq \lambda^{U'}\times (||N||^G)^{\kappa(G)}$. Accordingly, we have
\begin{equation}
||\phi^*N||^G\leq ||U'N||^G\leq \lambda^{U'}\times (||N||^G)^{\kappa(G)}\leq \alpha^\gamma
\end{equation}


\end{proof}

We will now show that $Cart-\mathscr U$ has a generator.  We fix an infinite cardinal $\alpha$ such that
\begin{equation}\label{sup6}
\alpha\geq sup \{\mbox{$\gamma$, $\lambda^{\mathscr U_x}$, $x\in \mathbb V$}\}
\end{equation}
Let $\mathscr M\in Cart-\mathscr U$ and take some $\zeta\in el_G(\mathscr M)$, given by $\zeta: \mathscr U_xG\longrightarrow \mathscr M_x$ for some $x\in \mathbb V$. Corresponding to $\zeta$, we consider as in the proof of Theorem \ref{Th4.7b} the subobject $\mathscr  P \subseteq \mathscr M$ in $Mod-\mathscr U$ such that $\zeta\in el_G(\mathscr P)$ and $|el_G(\mathscr P)|\leq  \gamma^{\kappa(G)}\leq \alpha^\gamma$. We now choose a well ordering of the set $Mor(\mathbb Q)$ and consider the induced lexicographic order on $\mathbb N \times Mor(\mathbb Q)$. We proceed by induction on $\mathbb N \times Mor(\mathbb Q)$ to construct a family of subobjects $\{\mathscr N(n,\phi) : n\in \mathbb N, \phi\in Mor(\mathbb Q)\}$ of $\mathscr M$ in $Mod-\mathscr U$ satisfying the following conditions.

\smallskip
(1) If $\phi_0$ is the least element of $Mor(\mathbb Q)$, then $\zeta\in el_G(\mathscr N(1,\phi_0))$.

\smallskip
(2) For any $(n,\phi)\leq (m,\psi)$ in $\mathbb N \times Mor(\mathbb Q)$, we have $\mathscr N(n,\phi) \subseteq \mathscr N(m,\psi)$

\smallskip
(3) For each $(n,\phi:y \longrightarrow z)$ in $\mathbb N \times Mor(\mathbb Q)$, the morphism $\mathscr N(n,\phi)^{\phi}:\phi^*\mathscr N(n,\phi)_y \longrightarrow \mathscr N(n,\phi)_z$ is an isomorphism in $EM_{\mathscr U_z}$.

\smallskip
(4) $|el_G(\mathscr N(n,\phi))|\leq \alpha^\gamma$.

For  $(n,\phi:y \longrightarrow z)$ in $\mathbb N \times Mor(\mathbb Q)$, we begin the transfinite induction argument by setting
\begin{equation}
A_0^0(w):= 
\begin{cases} 
 el_G(\mathscr P_w),\quad & \text{if}~~(n,\phi) = (1,\phi_0) \\
 \underset{(m,\psi)< (n,\phi)}\bigcup el_G(\mathscr N(m,\psi)_w), \quad &~~ \mbox{otherwise}
\end{cases}
\end{equation}
for each $w \in \mathbb V$. Since each $A_0^0(w)\subseteq el_G(\mathscr M_w)$, $|A_0^0(w)|\leq \alpha^\gamma$, and $\mathscr M$  is cartesian, we use Lemma \ref{L6.2} to obtain a subobject $A_1^0(y) \subseteq \mathscr M_y$ in $EM_{\mathscr U_y}$ such that
\begin{equation}\label{611dt}
||A_1^0(y)||^G\leq \alpha^\gamma \quad ||\phi^*A_1^0(y)||^G\leq \alpha^\gamma\quad A^0_0(y) \subseteq el_G(A_1^0(y)) \quad A_0^0(z) \subseteq el_{G}(\phi^*A_1^0(y))
\end{equation} 
We now set $A_1^0(z) = \phi^*A_1^0(y)$ and set for each $w\in \mathbb V$:
\begin{equation} \label{612dx}
B_1^0(w)=\left\{\begin{array}{ll} el_G(A^0_1(w))  & \mbox{if $w=y,z$} \\
A^0_0(w) & \mbox{otherwise}\\
\end{array}\right.
\end{equation} From \eqref{611dt} and \eqref{612dx} it follows that for each $w \in \mathbb V$, $A^0_0(w) \subseteq B^0_1(w)$ and $|B^0_1(w)|\leq \alpha^\gamma$.
\begin{lem}\label{L6.3}
Let $X\subseteq el_G(\mathscr M)$ with $|X|\leq\alpha^\gamma$. Then there exists a subobject $\mathscr D\hookrightarrow\mathscr M$ in $Mod-\mathscr U$ such that $X \subseteq el_G(\mathscr D)$ and  $|el_G(\mathscr D)|\leq\alpha^\gamma$.
\end{lem}
\begin{proof}
Let $\zeta\in X\subseteq el_G(\mathscr M)$. Then, using Theorem \ref{Th4.7b}, we choose a subobject $\mathscr D_{\zeta}\hookrightarrow\mathscr M$ such that $\zeta\in el_G(\mathscr D_{\zeta})$ and  $|el_G(\mathscr D_\zeta)|\leq\gamma^{\kappa(G)}\leq \alpha^\gamma$. Now, we set $\mathscr D := \underset{\zeta\in X}\sum\mathscr D_\zeta$. Clearly, $\mathscr D$ is a quotient of $\underset{\zeta\in X}\bigoplus\mathscr D_\zeta$ and $X \subseteq el_G(\mathscr D)$. Further, using Lemma \ref{L2.1} and the definition in \eqref{el45t}, we get
\begin{equation}
|el_G(\mathscr D)| \leq \left\vert el_G \left(\underset{\zeta\in X}\bigoplus\mathscr D_\zeta\right)\right\vert \leq  \underset{y\in\mathbb V}\sum\left\vert EM_{\mathscr U_y}\left(\mathscr U_yG,\underset{\zeta\in X}\bigoplus\mathscr{D}_{\zeta_y}\right)\right\vert=\underset{y\in\mathbb V}\sum\left\vert\mathcal C\left(G,\underset{\zeta\in X}\bigoplus\mathscr{D}_{\zeta_y}\right)\right\vert\leq \alpha^\gamma
\end{equation}
\end{proof}
Now using Lemma \ref{L6.3}, we choose a subobject $\mathscr D^0(n,\phi)\hookrightarrow\mathscr M$ in $Mod-\mathscr U$ such that $\underset{w\in\mathbb V}\bigcup B^0_1(w) \subseteq el_G(\mathscr D^0(n,\phi))$ and $|el_G(\mathscr D^0(n,\phi))|\leq \alpha^\gamma$. In particular, for each $w\in\mathbb V$, $B^0_1(w)\subseteq el_G(\mathscr D^0(n,\phi)_w)$. 

\smallskip
We now iterate this construction. Suppose that for every $r\leq s$ we have constructed a subobject $\mathscr D^r(n,\phi)\hookrightarrow\mathscr M$ in
$Mod-\mathscr U$  such that  $\underset{w\in\mathbb V}\bigcup B^r_1(w) \subseteq el_G(\mathscr D^r(n,\phi))$ and $|el_G(\mathscr D^r(n,\phi))|\leq \alpha^\gamma$.  Then, for each $w\in\mathbb V$, we set $A_0^{s+1}(w):= el_G(\mathscr D^s(n,\phi)_w)$. Again using Lemma \ref{L6.2}, we get $A_1^{s+1}(y)\subseteq\mathscr M_y$ in $EM_{\mathscr U_y}$ such that
\begin{equation}
||A_1^{s+1}(y)||^G\leq \alpha^\gamma \quad ||\phi^*A_1^{s+1}(y)||^G\leq \alpha^\gamma \quad A^{s+1}_0(y) \subseteq el_G(A_1^{s+1}(y)) \quad A_0^{s+1}(z) \subseteq el_{G}(\phi^*A_1^{s+1}(y))
\end{equation} 
We now set $A_1^{s+1}(z) = \phi^*A_1^{s+1}(y)$. For $w\in \mathbb V$, we set $B_1^{s+1}(w)=el_G(A^{s+1}_1(w))$ if $w=y,z,$  and 
$B^{s+1}_1(w)=A_0^{s+1}(w)= el_G(\mathscr D^s(n,\phi)_w)$ otherwise. It follows that for each $w \in \mathbb V$, $A_0^{s+1}(w) \subseteq B^{s+1}_1(w)$ and $|B_1^{s+1}(w)|\leq \alpha^\gamma$. Using Lemma \ref{L6.3}, we now choose $\mathscr D^{s+1}(n,\phi)\hookrightarrow \mathscr M$ such that $\underset{w\in\mathbb V}\bigcup B^{s+1}_1(w) \subseteq el_G(\mathscr D^{s+1}(n,\phi))$ and $|el_G(\mathscr D^{s+1}(n,\phi))|\leq \alpha^\gamma$.   In particular, for each $w\in\mathbb V$, $B^{s+1}_1(w)\subseteq el_G(\mathscr D^{s+1}(n,\phi)_w)$. 
We note that we have constructed an ascending chain 
\begin{equation}\label{615th}
\mathscr D^{0}(n,\phi)\leq \mathscr D^{1}(n,\phi)\leq\ldots\leq\mathscr D^{s}(n,\phi)\leq \ldots
\end{equation}
of subobjects of $\mathscr M$ in $Mod-\mathscr U$.
Finally, we define 
\begin{equation}\label{616th}
\mathscr N(n,\phi):= \underset{s\geq0}\varinjlim\textrm{ } \mathscr D^{s}(n,\phi)
\end{equation}
in $Mod-\mathscr U$. Since each $|el_G(\mathscr D^{s}(n,\phi))| \leq \alpha^\gamma$, we have $|el_G(\mathscr N(n,\phi))| \leq \alpha^\gamma.$ Clearly, the family $\{\mathscr N(n,\phi)~|~(n,\phi)\in\mathbb N\times Mor(\mathbb Q)\}$ satisfies the conditions $(1),(2)$ and $(4)$.
For $(3)$, we note that $\mathscr N(n,\phi)_y$ can be expressed as the filtered union 
\begin{equation}
A_1^{0}(y)\hookrightarrow\mathscr D^{0}(n,\phi)_y\hookrightarrow A_1^{1}(y)\hookrightarrow\mathscr D^{1}(n,\phi)_y\hookrightarrow\cdots\hookrightarrow A_1^{s}(y)\hookrightarrow\mathscr D^{s}(n,\phi)_y\hookrightarrow \cdots
\end{equation}
of objects in $EM_{\mathscr U_y}$. Since $\phi^*$ is exact and a left adjoint, it preserves monomorphisms and filtered colimits. Hence we can also express $\phi^*\mathscr N(n,\phi)_y$ as a filtered union
\begin{equation}
\phi^*A_1^{0}(y)\hookrightarrow\phi^*\mathscr D^{0}(n,\phi)_y\hookrightarrow \phi^*A_1^{1}(y)\hookrightarrow\phi^*\mathscr D^{1}(n,\phi)_y\hookrightarrow\cdots\hookrightarrow \phi^*A_1^{s}(y)\hookrightarrow\phi^*\mathscr D^{s}(n,\phi)_y\hookrightarrow \cdots
\end{equation}
of objects in $EM_{\mathscr U_z}$. Similarly, $\mathscr N(n,\phi)_z$ can be expressed as the filtered union 
\begin{equation}
A_1^{0}(z)\hookrightarrow\mathscr D^{0}(n,\phi)_z\hookrightarrow A_1^{1}(z)\hookrightarrow\mathscr D^{1}(n,\phi)_z\hookrightarrow\cdots\hookrightarrow A_1^{s}(z)\hookrightarrow\mathscr D^{s}(n,\phi)_z\hookrightarrow \cdots
\end{equation}
of objects in $EM_{\mathscr U_z}$. By definition, we know that $\phi^*A_1^{s}(y) = A_1^{s}(z)$ for each $s\geq 0$. Therefore, we obtain the required isomorphism $\mathscr N(n,\phi)^\phi : \phi^*\mathscr N(n,\phi)_y\longrightarrow\mathscr N(n,\phi)_z$.
\begin{lem}\label{L6.4}
Let $\mathscr M$ be a cartesian module over a flat monad quiver $\mathscr U:\mathbb Q\longrightarrow Mnd(\mathscr C)$. Let $\zeta\in el_G(\mathscr M)$. Then there exists a subobject $\mathscr N\subseteq\mathscr M$ in $Cart-\mathscr U$ such that $\zeta\in el_G(\mathscr N)$ and $|el_G(\mathscr N)| \leq \alpha^\gamma$.
\end{lem}
\begin{proof}
Since $\mathbb N\times Mor(\mathbb Q)$ is filtered, we set 
\begin{equation}
\mathscr N = \underset{(n,\phi)\in\mathbb N\times Mor(\mathbb Q)}\bigcup\mathscr N(n,\phi)\subseteq\mathscr M
\end{equation}
in $Mod-\mathscr U$. Clearly, $\zeta\in el_G(\mathscr N).$ Also, as each $|el_G(\mathscr N(n,\phi))|\leq \alpha^\gamma$, we have $|el_G(\mathscr N)|\leq\alpha^\gamma$. Next, we note that for a fixed morphism $\rho:z\longrightarrow w$ in $\mathbb Q$, the family $\{(m,\rho)~|~m\geq 1\}$ is cofinal in $\mathbb N\times Mor(\mathbb Q)$. Therefore, 
\begin{equation}
\mathscr N = \underset{m\geq 1}\varinjlim\mathscr N(m,\rho)
\end{equation}
Further, as $\mathscr N(m,\rho)^\rho : \rho^*\mathscr N(m,\rho)_z\longrightarrow\mathscr N(m,\rho)_w$ is an isomorphism, it follows that the filtered colimit $\mathscr N^\rho:\rho^*\mathscr N_z \longrightarrow\mathscr N_w$ is also an isomorphism.
\end{proof}
\begin{Thm}\label{Th6.5}
Let $\mathbb Q$ be a poset and $\mathscr U:\mathbb Q\longrightarrow Mnd(\mathcal C)$ be flat. Then, the category $Cart-\mathscr U$ of cartesian modules is a Grothendieck category.
\end{Thm}
\begin{proof}
We already know that $Cart-\mathscr U$ is an abelian category. Now, since filtered colimits and finite limits of $Cart-\mathscr U$ are computed in $Mod-\mathscr U$, and $\mathscr U:\mathbb Q\longrightarrow Mnd(\mathcal C)$ is flat, it is also clear  $Cart-\mathscr U$   satisfies the (AB5) condition. Further, from Lemma \ref{L6.4}, we see that any $\mathscr M \in Cart-\mathscr U$ can be expressed as sum of a family $\{\mathscr N_\zeta~|~\zeta\in el_G(\mathscr M)\}$ of cartesian subobjects where each $|el_G(\mathscr N_\zeta)|\leq\alpha^\gamma$. Therefore, the isomorphism classes of cartesian modules $\mathscr N$ satisfying $|el_G(\mathscr N)|\leq\alpha^\gamma$ give a set of generators for $Cart-\mathscr U$.
\end{proof}
\begin{Thm}\label{Th6.6}
Let the category $\mathbb Q$ be a poset and $\mathscr U:\mathbb Q\longrightarrow Mnd(\mathcal C)$ be flat. Then, the inclusion functor $i: Cart-\mathscr U \longrightarrow Mod-\mathscr U$ has a right adjoint.
\end{Thm}
\begin{proof}
We see that the inclusion functor $i: Cart-\mathscr U \longrightarrow Mod-\mathscr U$ preserves colimits.  Since $Cart-\mathscr U$ and $Mod-\mathscr U$ are Grothendieck categories, it follows (see, for instance, \cite[Proposition 8.3.27]{KS}) that $i$ has a right adjoint.
\end{proof}




\small

\begin{bibdiv}
	\begin{biblist}
	\bib{AR}{book}{
   author={Ad\'{a}mek, J.},
   author={Rosick\'{y}, J.},
   title={Locally presentable and accessible categories},
   series={London Mathematical Society Lecture Note Series},
   volume={189},
   publisher={Cambridge University Press, Cambridge},
   date={1994},
   pages={xiv+316},
}

\bib{AZ1}{article}{
   author={Artin, M.},
   author={Zhang, J. J.},
   title={Noncommutative projective schemes},
   journal={Adv. Math.},
   volume={109},
   date={1994},
   number={2},
   pages={228--287},
}


\bib{AZ2}{article}{
   author={Artin, M.},
   author={Zhang, J. J.},
   title={Abstract Hilbert schemes},
   journal={Algebr. Represent. Theory},
   volume={4},
   date={2001},
   number={4},
   pages={305--394},
}

\bib{Ban}{article}{
   author={Banerjee, A.},
   title={Entwined modules over representations of categories},
   journal={Algebras and Representation Theory (to appear), doi: 10.1007/s10468-023-10203-3},
}

\bib{BBR0}{article}{
   author={Balodi, M.},
   author={Banerjee, A.},
   author={Ray, S.},
   title={Cohomology of modules over $H$-categories and co-$H$-categories},
   journal={Canad. J. Math.},
   volume={72},
   date={2020},
   number={5},
   pages={1352--1385},
}
	


\bib{BBR}{article}{
   author={Balodi, M.},
   author={Banerjee, A.},
   author={Ray, S.},
   title={Categories of modules, comodules and contramodules over representations},
 journal={Forum Math (to appear), 	arXiv:2106.12237 [math.RA]},
}


\bib{Beke}{article}{
   author={Beke, T.},
   title={Sheafifiable homotopy model categories},
   journal={Math. Proc. Cambridge Philos. Soc.},
   volume={129},
   date={2000},
   number={3},
   pages={447--475},
}

\bib{CMZ}{article}{
   author={Caenepeel, S.},
   author={Militaru, G.},
   author={Zhu, S.},
   title={Doi-Hopf modules, Yetter-Drinfel\cprime d modules and Frobenius type
   properties},
   journal={Trans. Amer. Math. Soc.},
   volume={349},
   date={1997},
   number={11},
   pages={4311--4342},
}

\bib{CMIZ}{article}{
   author={Caenepeel, S.},
   author={Militaru, G.},
   author={Ion, B.},
   author={Zhu, S.},
   title={Separable functors for the category of Doi-Hopf modules,
   applications},
   journal={Adv. Math.},
   volume={145},
   date={1999},
   number={2},
   pages={239--290},
}

\bib{CG}{article}{
   author={Caenepeel, S.},
   author={Gu\'{e}d\'{e}non, T.},
   title={On the cohomology of relative Hopf modules},
   journal={Comm. Algebra},
   volume={33},
   date={2005},
   number={11},
   pages={4011--4034},
}

\bib{f13}{article}{
   author={Di, Z.},
   author={Estrada, S.},
   author={Liang, L.},
   author={Odaba\c{s}\i , S.},
   title={Gorenstein flat representations of left rooted quivers},
   journal={J. Algebra},
   volume={584},
   date={2021},
   pages={180--214},
}

\bib{EE}{article}{
   author={Enochs, E.},
   author={Estrada, S.},
   title={Relative homological algebra in the category of quasi-coherent
   sheaves},
   journal={Adv. Math.},
   volume={194},
   date={2005},
   number={2},
   pages={284--295},
}

\bib{f16}{article}{
   author={Enochs, E.},
   author={Estrada, S.},
   title={Projective representations of quivers},
   journal={Comm. Algebra},
   volume={33},
   date={2005},
   number={10},
   pages={3467--3478},
}

\bib{f17}{article}{
   author={Enochs, E.},
   author={Estrada, S.},
   author={Garc\'{\i}a Rozas, J. R.},
   title={Injective representations of infinite quivers. Applications},
   journal={Canad. J. Math.},
   volume={61},
   date={2009},
   number={2},
   pages={315--335},
}

\bib{EV}{article}{
   author={Estrada, S.},
   author={Virili, S.},
   title={Cartesian modules over representations of small categories},
   journal={Adv. Math.},
   volume={310},
   date={2017},
   pages={557--609},
}


\bib{Et}{book}{
   author={Etingof, P.},
   author={Gelaki, S.},
   author={Nikshych, D.},
   author={Ostrik, V.},
   title={Tensor categories},
   series={Mathematical Surveys and Monographs},
   volume={205},
   publisher={American Mathematical Society, Providence, RI},
   date={2015},
}

\bib{Hod}{book}{
   author={Hodges, W.},
   title={A shorter model theory},
   publisher={Cambridge University Press, Cambridge},
   date={1997},
   pages={x+310},
}


\bib{Ill}{article}{
author={Illusie, L.},
   title={Existence de r\'{e}solutions globales},
   series={Lecture Notes in Mathematics, Vol. 225},
   note={Th\'{e}orie des intersections et th\'{e}or\`eme de Riemann-Roch, S\'{e}minaire de G\'{e}om\'{e}trie Alg\'{e}brique du Bois-Marie 1966--1967 (SGA 6);
   Dirig\'{e} par P. Berthelot, A. Grothendieck et L. Illusie. Avec la
   collaboration de D. Ferrand, J. P. Jouanolou, O. Jussila, S. Kleiman, M.
   Raynaud et J. P. Serre},
   publisher={Springer-Verlag, Berlin-New York, 1971},
}

\bib{Lint}{article}{
   author={Linton, F. E. J.},
   title={Coequalizers in categories of algebras},
   conference={
      title={Sem. on Triples and Categorical Homology Theory (ETH, Z\"{u}rich,
      1966/67)},
   },
   book={
      publisher={Springer, Berlin},
   },
   date={1969},
   pages={75--90},
}

\bib{Mac}{book}{
   author={MacLane, S.},
   title={Categories for the working mathematician},
   series={Graduate Texts in Mathematics, Vol. 5},
   publisher={Springer-Verlag, New York-Berlin},
   date={1971},
}

\bib{Gro}{article}{
   author={Grothendieck, A.},
   title={Sur quelques points d’alg`ebre homologique},
   journal={Tohoku Math. J.(2)},
   volume={9},
   date={1957},
   pages={119--221},
}	
\bib{KS}{book}{
   author={Kashiwara, M.},
   author={Schapira, P.},
   title={Categories and Sheaves},
   publisher={Springer-Verlag, Berlin-Heidelberg},
   date={2006},
}



\bib{Stacks}{article}{
   author={The Stacks project},
    title={Available online, },
   journal={https://stacks.math.columbia.edu/},
}
	\end{biblist}
	
	\end{bibdiv}









\end{document}