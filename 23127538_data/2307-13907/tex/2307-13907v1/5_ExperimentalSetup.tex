\section{Experimental Setup}
\label{sec: Experimental Setup}
\subsection{Dataset Description}\label{Dataset} For evaluation, we have considered two different time series datasets for PHM of a Li battery and a turbine.

\paragraph{\textbf{Battery State-of-Charge Dataset (BSOC)}\cite{kollmeyer2020lg}:} This dataset is derived from a new 3Ah LG HG2 cell tested in an 8 cu.ft. thermal chamber using a 75amp, 5-volt Digatron Firing Circuits Universal Battery Tester with high accuracy (0.1 of full scale) for voltage and current measurements. The main focus is to determine the State of Charge (SOC) of the battery, measured as a percentage, which indicates the charge level relative to its capacity. SOC for a Li-ion battery depends on various features, including voltage, current, temperature, and average voltage and current. The data is obtained from the `LG\_HG2\_Prepared\_Dataset\_McMasterUniversity\_Jan\_2020', readily available in the dataset folder \cite{kollmeyer2020lg}. The training data consists of a single sequence of experimental data collected while the battery-powered electric vehicle during a driving cycle at an external temperature of 25 degrees Celsius. The test dataset contains experimental data with an external temperature of -10 degrees Celsius.

\paragraph{\textbf{Turbofan Engine Degradation Simulation Data Set (TEDS)}\cite{Prognost61:online,saxena2008turbofan}:} This dataset is widely used for predicting the Remaining Useful Life (RUL) of turbofan jet engines \cite{Prognost61:online}. Engine degradation simulations are conducted using C-MAPSS (Commercial Modular Aero-Propulsion System Simulation) with four different sets, simulating various operational conditions and fault modes. Each engine has 26 different feature values recorded at different time instances. To streamline computation, features with low variability (similar to Principal Component Analysis \cite{pearson1901liii}) are removed to avoid negative impacts on the training process. The remaining 17 features [\ref{AppendixPrognosability},\ref{AppendixSampleDataset}] are then normalized using z-score (mean-standard deviation) for training. The training subset comprises time series data for 100 engines, but for this paper, we focus on data from only one engine (FD001). For evaluation, we randomly selected engine 52 from the test dataset.
 
\subsection{Network Description}

% \paragraph{\textbf{Battery State-of-Charge Dataset (BSOC)}:} 
The network architecture used for training the BSOC dataset, partially adopted from \cite{mathworksPredictBattery}, is a regression CNN, as shown in [Fig~\ref{fig:FigArchitecture},\ref{AppendixNetworks}]. The network has five input features which correspond to one SOC value. Therefore, the TSRegNN for the BSOC dataset can be represented as:
\begin{equation}
    \label{equ: modTSRNN}
    \begin{split}
    f:~x \in \mathbb{R}^{5\times t_s} \rightarrow y \in {\mathbb{R}^{1 \times t_s}} \\
        \hat{SOC}_{t_s} = f(t_s)
    \end{split}
\end{equation}
% \paragraph{\textbf{Turbofan Engine Degradation Simulation Data Set (TEDS)}:} 
The network architecture used for training the TEDS dataset is also a regression CNN, adopted from \cite{mathworksRemainingUseful} and shown in [Fig~\ref{fig:FigArchitecture},\ref{AppendixNetworks}]. The input data is preprocessed to focus on 17 features, corresponding to one RUL value for the engine. Therefore, the TSRegNN for the TEDS dataset can be represented as:
\begin{equation}
    \label{equ: modTSRNN}
    \begin{split}
        f:~x \in \mathbb{R}^{17\times t_s} \rightarrow y \in {\mathbb{R}^{1 \times t_s}} \\
        \hat{RUL}_{t_s+1} = f(t_s)
    \end{split}
\end{equation}
% The $T^{th}$ value in the output represents the desired estimation of RUL,  given the series of past T values of 17 features.
The output's $t_s^{th}$ value represents the desired estimation of SOC or RUL, with the given series of past $t_s$ values for each feature variable.
% \subsection{Verification Properties Considered}
% For all four noise scenarios mentioned in Sec.~\ref{Noise Types}, local and global (here for 100 consecutive time steps) robustness properties are considered for both datasets. An additional local monotonicity property is also considered for the turbine RUL estimation example. 