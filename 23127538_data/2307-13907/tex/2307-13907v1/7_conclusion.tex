\section{Conclusion and Future Work}
\label{sec: Conclusion and Future Work}
This paper explores formal method-based reachability analysis of variable-length time series regression neural networks (NNs) using approximate Star methods in the context of predictive maintenance, which is crucial with the rise of Industry 4.0 and the Internet of Things. The analysis considers sensor noise introduced in the data. Evaluation is conducted on two datasets, employing a unified reachability analysis that handles varying features and variable time sequence lengths while analyzing the output with acceptable upper and lower bounds. Robustness and monotonicity properties are verified for the TEDS dataset. Real-world datasets are used, but further research is needed to establish stronger connections between practical industrial problems and performance metrics. The study opens new avenues for exploring perturbation contributions to the output and extending reachability analysis to 3-dimensional time series data like videos. Future work involves verifying global monotonicity properties as well, and including more predictive maintenance and anomaly detection applications as case studies. \newblue{The study focuses solely on offline data analysis and lacks considerations for real-time stream processing and memory constraints, which present fascinating avenues for future research.}
\paragraph{\textbf{Acknowledgements.}}
The material presented in this paper is based upon work supported by the National Science Foundation (NSF) through grant numbers 1910017, 2028001, 2220418, 2220426, and 2220401, and the Defense Advanced Research Projects Agency (DARPA) under contract number FA8750-18-C-0089 and FA8750-23-C-0518, and the Air Force Office of Scientific Research (AFOSR) under contract number FA9550-22-1-0019 and FA9550-23-1-0135. Any opinions, findings, conclusions, or recommendations expressed in this paper are those of the authors and do not necessarily reflect the views of AFOSR, DARPA, or NSF. We also want to thank our colleagues, Tianshu and Barnie for their valuable feedback.
 
