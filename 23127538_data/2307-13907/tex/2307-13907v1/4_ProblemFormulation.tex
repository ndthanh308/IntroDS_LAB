\section{Robustness Verification Problem Formulation}\label{sec: problem}
We consider the verification of the robustness and the monotonicity properties. 
\begin{problem}[\textbf{Local Robustness Property}]\label{prob:1}
Given a TSRegNN $f$, a time series signal $S$, and a noise $\mathcal{A}$, prove if the network is locally robust or non-robust [Sec. \ref{Robust}] w.r.t the noise $\mathcal{A}$; i.e., if the estimated bounds obtained through the reachability calculations lie within the allowable range of the actual output for the particular time instance.
\end{problem}
%
\begin{problem}[\textbf{Global Robustness Property}]\label{prob:2}
Given a TSRegNN $f$, a set of $N$ consecutive time-series signal $\textbf{S} = \{S_1, \dots, S_N\}$, and a noise $\mathcal{A}$, compute the percentage robustness values (PR [Def. \ref{def:PSR}] and POR [Def. \ref{def:POR}]) corresponding to $\mathcal{A}$.
\end{problem}
\begin{problem}[\textbf{Local Monotonicity Property}]\label{prob:3}
Given a TSRegNN $f$, a set of $N$ consecutive time-series signal $\textbf{S} = \{S_1, \dots, S_N\}$, and a noise $\mathcal{A}$, show that both the estimated RUL bounds of the network [Eq. \ref{equ: monotonicity}] corresponding to noisy input $S'_t$ at any time instance $t$ are monotonically decreasing.
\end{problem}

To get an idea of the global performance \cite{wang2022tool} of the network, local stability properties have been formulated and verified for each point in the test dataset for 100 consecutive time steps.

 The core step in solving these problems is to solve the local properties of a TSRegNN $f$ w.r.t a noise $\mathcal{A}$. It can be done using over-approximate reachability analysis, computing the `output reachable set' $\mathcal{R}_ts = Reach(f, I)$ that provides an upper and lower bound estimation corresponding to the noisy input set $I$. 

\newblue{In this paper, we propose using percentage values as robustness measures for verifying neural networks (NN). We conduct reachability analysis on the output set to ensure it stays within predefined safe bounds specified by permissible upper-lower bounds. The calculated overlap or sample robustness, expressed as a percentage value, represents the NN's robustness achieved through the verification process under different noise conditions.} The proposed solution takes a sound and incomplete approach to verify the robustness of regression neural networks with time series data. The approach over-approximates the reachable set, ensuring that any input point within the set will always have an output point contained within the reachable output set (sound [Def. \ref{def: soundness}]). However, due to the complexities of neural networks and the over-approximation nature of the approach, certain output points within the reachable output set may not directly correspond to specific input points (incomplete [Def. \ref{def: completeness}]). Over-approximation is commonly used in safety verification and robustness analysis of complex systems due to its computational efficiency and reduced time requirements compared to exact methods.

