\newpage
\appendix
\section{Appendix} \label{Appendix}
% \subsection{Sigma Levels (\textbf{Figure} \ref{fig:Sixsigma})}
% % Figure environment removed
% \vspace*{-\baselineskip}
% \vspace*{-\baselineskip}
\subsection{Preliminaries}
\begin{definition}[Soundness]\label{def: soundness}
Let $\mathcal{F}: R^j \rightarrow R^p$ be a NN with an input set $R_0$ and output reachable set $R_f$ . The
computed $R_f$ given $\mathcal{F}$ and $R_0$ is sound iff $\forall{x} \in R_0, | y = \mathcal{F}(x), y \in R_f$.
\end{definition}
\begin{definition}[Completeness]\label{def: completeness}
    Let $\mathcal{F}: R^j \rightarrow R^p$ be a NN with an input set $R_0$ and output reachable set $R_f$ . The
computed $R_f$ given $\mathcal{F}$ and $R_0$ is complete iff $\forall{x} \in R_0, \exists{y} = \mathcal{F}(x)~|~y \in R_f$ and $\forall{y} \in R_f, \exists{x} \in R_0~|~y = \mathcal{F}(x)$.
\end{definition}
\subsubsection{$L_\infty$ Norm:}\label{l_inf norm} Input perturbations can be quantified using different types of norms. Here, we have used the $L_\infty$ norm, which records the greatest perturbation magnitude among all input elements.
\begin{equation}
    L_\infty : ||x - x'||_\infty = max||x_i - x'_i||
\end{equation}
\subsubsection{Example of Robustness Calculations:}\label{Robustness Measures Example}
The left picture of Fig. \ref{fig:LocalRobust} in Appendix depicts an example plot of output estimations (red) vs. the allowable bounds (blue). Here, we can see that the network is locally robust for time instances $t_1$ and $t_5$; in other instances, it is non-robust w.r.t the noise added. So the RV is 1 for both $t_1$ and $t_5$ and 0 for others. 
\vspace*{-\baselineskip}
% Figure environment removed
To better understand the concept of POR and PO, we refer to the right picture of Fig. \ref{fig:LocalRobust}. Here for a time instance $t_3$, C denotes the actual signal value, AB is the allowable output range, and DE is the estimated reachable bounds. Here, $N_{robust}$ = 3 and $N_{total}$ = 5, so the PR for this particular example is 60\%.
The PO will be calculated as:
\begin{equation*}
 PO = \frac{DB}{DE} = \frac{3}{5} = 0.6
\end{equation*}
To calculate POR, we calculate the PO for each of the time instances:
\begin{equation*}
 POR = \frac{\frac{2}{2}+\frac{3}{4} + \frac{3}{5} + \frac{4}{6} + \frac{4}{4}}{5} \times 100\% = 80.33\% (approx)
\end{equation*}

\subsection{Battery State-of-charge}\label{Sec: State-of-charge} Battery state-of-charge is a measurement of the amount of energy available in a battery at a specific point in time, expressed as a percentage. This term is often used in various applications involving battery-powered systems, e.g., electric vehicles, renewable energy storage systems, portable electronics etc. Accurate estimation of the State of Charge (SOC) of a battery is crucial for efficient battery management and ensuring the longevity of the battery. The SOC is expressed as a percentage of the full capacity of the battery.

\subsection{Remaining Useful Life}\label{Sec: Remaining useful Life} The Remaining Useful Life (RUL) is a subjective estimate of the lifespan of any equipment before it requires repair or replacement. This important concept is often used in various fields, including maintenance, reliability engineering, and prognostics and health management (PHM). RUL estimation is typically based on the analysis of historical data, such as sensor measurements, degradation patterns, maintenance records, and operational conditions. Various techniques and models, including statistical methods, machine learning algorithms, and physics-based approaches, are generally used to predict the RUL.

\subsection{Sample Dataset Plots}\label{AppendixSampleDataset}
% Figure environment removed
% Figure environment removed
\subsection{Prognosability}\label{AppendixPrognosability} Features that remain constant for all time steps can negatively impact the training. Feature reduction is done using the 'prognosability' MATLAB command. Prognosability is actually a property relative to the prediction of the future state of the system, and this term is mainly used for lifetime data. In MATLAB, 'prognosability' is used as a function to measure the variability of the features in a dataset at failure. The equation for the prognosability calculation is given as below:
\begin{equation}
    prognosability = Y = \exp{\frac{std_j(x_j(N_j))}{mean_j\lvert(x_j(1) - x_j(N_j)\rvert}}
\end{equation}
The output has 3 different outcomes:
\begin{enumerate}
    \item Y = 0 means the feature values are constant, i.e., no variability in the data.
    \item Y = NaN indicates the prognosability could not be calculated.
    \item Y = 1 means the feature values are perfectly prognosable i.e., there is variability in the data.
\end{enumerate}
\subsection{Feature Reduction of TEDS Dataset}\label{AppendixTEDSDataset} 
Each engine has 26 different feature
values, recorded at different time instances.
\begin{enumerate}
        \item Feature 1: Unit number
        \item Feature 2: Time-stamp
        \item Feature 3–5: Operational settings
        \item Feature 6–26: Sensor measurements 1–21
    \end{enumerate}
After analyzing the dataset using 'prognosability', number of features considered for NN training reduced to 17 from 26, and they are
\begin{enumerate}
        \item Feature 3–4~~~~: Operational settings 1-2
        \item Feature 7–9~~~~: Sensor measurements 2-4
        \item Feature 11–14~: Sensor measurements 6–9
        \item Feature 16–20~: Sensor measurements 11–15
        \item Feature 22~~~~~~: Sensor measurements 17
        \item Feature 25-26~~: Sensor measurements 20-21
\end{enumerate}

\newpage
\subsection{Network Architectures: (\textbf{Figure}\label{Appendix A.1} \ref{fig:FigArchitecture})}\label{AppendixNetworks}
% Figure environment removed
\subsection{Noise}\label{AppendixNoise}
% Figure environment removed
% Figure environment removed
% Figure environment removed
% Figure environment removed


\subsection{TEDS dataset Sample Reachability Plot}
% Figure environment removed

