\documentclass[journal=nalefd,manuscript=letter,layout=onecolumn]{achemso}
\usepackage{graphicx}
\usepackage[utf8]{inputenc}
\usepackage{amsmath}
\usepackage{braket}
\usepackage{lmodern}
\usepackage[normalem]{ulem}
\usepackage{bbold}
\usepackage[T1]{fontenc}
\usepackage{xcolor}
\usepackage{bm}
\usepackage{siunitx}
\usepackage{hyperref}
\hypersetup{pdfnewwindow=true,allcolors=black,colorlinks=true}
\usepackage{lineno}
\usepackage{here}

\let\s\textsubscript
\let\vec\bm

\def\sub#1{_{\text{#1}}}
\def\super#1{^{\text{#1}}}

\newcommand{\nbs}{NbS\s2}

\renewcommand{\thefootnote}{\fnsymbol{footnote}}

%\renewcommand{\baselinestretch}{1.50}\normalsize

\renewcommand{\thefigure}{S\arabic{figure}}
\hyphenation{where-as}
\hyphenation{re-cent-ly}

\def\UK{%
	II. Physikalisches Institut,
	Universit\"at zu K\"oln,
	Z\"ulpicher Stra\ss e 77,
	D-50937 K\"oln,
	Germany}

\def\UHB{%
	U~Bremen Excellence Chair,
	Bremen Center for Computational Materials Science,
	and MAPEX Center for Materials and Processes,
	University of Bremen,
	D-28359 Bremen,
	Germany}

\def\UHH{%
	I. Institute of Theoretical Physics,
	Universit\"at Hamburg,
	D-22607 Hamburg,
	Germany}

\def\CUIHH{%
	The Hamburg Centre for Ultrafast Imaging,
	Luruper Chaussee 149,
	D-22761 Hamburg,
	Germany}

\def\LU{%
	Division of Mathematical Physics,
	Department of Physics,
	Lund University,
	SE-22100 Lund,
	Sweden}


\DeclareGraphicsExtensions{.pdf}

\title{Supporting Information:\\
Unconventional charge-density-wave gap in monolayer \nbs}

\author{Timo~Knispel}
\affiliation\UK

\author{Jan~Berges}
\affiliation\UHB

\author{Arne~Schobert}
\affiliation\UHH

\author{Erik~G.~C.~P.~van~Loon}
\affiliation\LU

\author{Wouter~Jolie}
\affiliation\UK

\author{Tim~Wehling}
\affiliation\UHH
\alsoaffiliation\CUIHH

\author{Thomas~Michely}
\affiliation\UK

\author{Jeison~Fischer}
\email{jfischer@ph2.uni-koeln.de}
\affiliation\UK

\begin{document}
\maketitle

\section{Experimental methods}

Sample preparation was accomplished in an ultrahigh vacuum (UHV) system with a base pressure of $p<2\times10^{-10}$\,mbar. Ir(111) was cleaned using cycles of
1\,kV $\mathrm{Ar^+}$-sputtering and subsequent flash annealing to 1520\,K. Graphene (Gr) was grown by ethylene exposure to saturation, subsequent flash annealing to 1470\,K and a final exposure to 800\,L ethylene at 1370\,K. The quality of the closed single crystal Gr monolayer was checked by low energy electron diffraction (LEED) and scanning tunneling microscopy (STM).

Monolayer H-\nbs{} was grown on Gr/Ir(111) by reactive molecular beam epitaxy (MBE). The substrate was exposed to an Nb flux of $\mathrm{5.8\times10^{15}}$\,atoms/$\mathrm{m^2s}$ from an e-beam evaporator in an elemental sulfur (S) background pressure of $p=1\times 10^{-8}$\,mbar created by a pyrite (FeS$_2$) filled Knudsen cell. The growth was conducted for 660\,s at 300\,K substrate temperature. Subsequently, the sample was annealed to 800\,K to improve the layer quality~\cite{Hall18}. In order to maximize the monolayer coverage, the island seeds were extended to final size by additional growth at 800\,K for 660\,s.

After synthesis, the H-\nbs{} layer was checked using LEED. Subsequently, the sample was transferred in UHV to the connected UHV bath cryostat STM chamber for investigation. The temperature $T_\mathrm{s}$ of STM or scanning tunneling spectroscopy (STS) investigation is specified in each figure and was either 0.4\,K using a He$^3$ cycle, 1.7\,K when pumping on He$^4$, 4\,K using He$^4$ cooling without pumping, or even higher than 4\,K by using an internal heater. Dependence of the STS features on magnetic field was checked by a superconducting magnet creating fields of up to 9\,T normal to the sample surface.

Both, constant-height and constant-current modes, were used to measure topography and d$I$/d$V$ maps. d$I$/d$V$ spectra were recorded only at constant height. Constant-current STM topographs and constant-current d$I$/d$V$ maps were recorded with sample bias $V_\mathrm{s}$ and tunneling current $I_\mathrm{t}$ specified in corresponding figure captions. Constant-height d$I$/d$V$ spectra and d$I$/d$V$ maps were recorded with stabilization bias $V_\mathrm{stab}$ and stabilization current $I_\mathrm{stab}$ using a lock-in amplifier with a modulation frequency $f_\mathrm{mod}$ and modulation voltage $V_\mathrm{mod}$ specified in corresponding captions. In case that for a constant-height d$I$/d$V$ map the sample bias during measurement does not coincide with $V_\mathrm{stab}$, the sample bias $V_\mathrm{s}$ is specified additionally. When needed, a voltage divider was applied to improve resolution. To ensure a reproducible and flat tip density of states (DOS) for the STS measurements, Au-covered W tips were used and calibrated beforehand using the surface state of Au(111)~\cite{Kaiser86,Everson91}.
Details on STM image processing are given in Figure~\ref{si_fig_cdw_fft}.

\section{Low-energy electron diffraction (LEED)}

% Figure environment removed

The LEED pattern corresponding to the STM topograph of Figure~1a displays first order \nbs~intensity as superposition of (i) prominent elongated spots (several encircled turquoise) reasonably aligned with Gr (encircled red) and Ir (encircled black) and (ii) a diffraction ring due to randomly oriented islands (dashed turquoise segment). Apparently, most islands are aligned with small angular scatter to Gr/Ir(111), while some display random orientation. Additionally, faint off-centered rings are visible (two highlighted by blue arrows). These rings belong to \nbs, but are each shifted by one moir\'e periodicity of Gr. S intercalation between Gr and Ir gives rise to a ($\sqrt{3} \times \sqrt{3}$)R30$^\circ$ structure with respect to Ir(111) and corresponding reflections, of which one first order reflection is encircled in pink.

\section{Local density of states of monolayer \nbs}

% Figure environment removed

To gain further insight into the electronic structure, differential conductance d$I$/d$V$ spectra were measured on \nbs, shown in Figure~\ref{si_fig_sts}. Figure~\ref{si_fig_sts}a displays a large-range constant-height d$I$/d$V$ spectrum which can be compared to the density functional theory (DFT) band structure in Figure~\ref{si_fig_sts}b. We note that (i) Van Hove singularities appear pronounced in d$I$/d$V$ spectra due to the large local DOS (LDOS) associated to them, and (ii) states with large parallel momentum $k_{||}$ are diminished or even suppressed in d$I$/d$V$ spectra, since a large $k_{||}$ is associated with a large decay constant $\kappa$, \emph{i.e.}, a rapid decay of the wave function into vacuum~\cite{Feenstra87,Tersoff83}.

The pronounced peak at $-1.25$\,V in Figure~\ref{si_fig_sts}a is attributed to the three occupied S $p$-bands with minima or maxima around $-1.25$\,V at the $\Gamma$-point in Figure~\ref{si_fig_sts}b. Additional maxima in the occupied states along the $\overline{\Gamma \textrm M}$ or the $\overline{\Gamma \textrm K}$ direction are hardly visible in the d$I$/d$V$ spectrum because of their larger $k_{||}$. The broad and intense maximum with its peak at about $+0.85$\,V in Figure~\ref{si_fig_sts}a is associated to the band maximum of the Nb $d_\textbf{z}$-type hole-like pocket at the $\Gamma$-point in Figure~\ref{si_fig_sts}b, though located at slightly lower energies as in the calculation. The steep rise in the d$I$/d$V$ spectrum at energies above about $+2.2$\,V is associated to the empty Nb $d$-bands with energies above 2\,eV in the calculated band structure. Figure~\ref{si_fig_sts}c displays an STS spectrum for the energy range from $-0.5$\,V to 0\,V (boxed in Figure~\ref{si_fig_sts}a). It is stabilized at $-0.5$\,V, \emph{i.e.}, at an energy with a low DOS as seen in Figure~\ref{si_fig_sts}a. To pick up the stabilization current of $I_\mathrm{t} = 0.5$\,nA the tip moves close to the surface and thus becomes sensitive to less pronounced features in the LDOS. The peak at $-0.15$\,V in the resulting spectrum can be interpreted as the Van Hove singularity associated with the toroidal minimum of the Nb $d$-band surrounding the $\Gamma$-point.

\section{Band structure of \nbs~near the $\Gamma$-point from quasi-particle interference}

% Figure environment removed

Besides the charge density wave (CDW) superstructure, another electronic feature cannot be overlooked in \nbs~monolayer islands. The 100\,mV d$I$/d$V$ map in Figure~\ref{si_fig_standing_waves}a displays standing wave patterns at the \nbs~island edges originating from quasi-particle interference (QPI) of electron waves. Zooming into the island, Figure~\ref{si_fig_standing_waves}b shows a constant-current d$I$/d$V$ map recorded at 30\,mV. At this bias voltage damping of the QPI is weak and the interference pattern is spread out over the whole island. The inset with the fast Fourier transform (FFT) exhibits a ring-like feature, which shows enhanced intensity in the $\overline{\Gamma \textrm M}$ direction. QPI at 30\,meV is thus close to isotropic in wave vector, but anisotropic in scattering intensity.
The QPI pattern is used to extract the dispersion~\cite{Crommie93,Hasegawa93,Hormandinger94} of the $d$-band crossing the Fermi level, discussed in Figure~\ref{si_fig_sts}b.

In Figure~\ref{si_fig_standing_waves}c the FFT intensity profiles along the high-symmetry directions in $k$~space are plotted as function of energy. Superimposed to the dispersing feature in the data is the DFT calculated band as dotted-blue line. The bright cut-off toward larger $k$~values agrees reasonably with the calculation.

The dispersion is also determined by analysis of the real space periodicity of the standing waves at \nbs~island edges. Following the approach of Crommie \emph{et al.}~\cite{Crommie93}, the standing wave pattern resulting from backscattering at a straight island edge is fitted after proper background subtraction through $\mathrm dI/\mathrm dV [V_\mathrm{s},x]=L_0[1-J_0(2kx+\phi)]$. Here $J_0$ is the zeroth order Bessel function, $L_0=m^*/(\pi\hbar^2)$ with $m^*$ being the effective mass, $\phi$ is a phase constant, $x$ the distance from the step edge and $k$ is the wave vector related to the electron energy $E = e V_\mathrm{s}$. Figure~\ref{si_fig_standing_waves}e exemplifies our approach for a profile (black dots) taken along the orange arrow in the 150\,mV d$I$/d$V$ map shown in Figure~\ref{si_fig_standing_waves}e. The fit is shown as thin red line and yields the $k$~vector for $E = 150$\,meV. Figure~\ref{si_fig_standing_waves}f presents our analysis in the energy range from $-250$\,meV to 800\,meV (black dots), which compares favorably with our DFT calculated dispersion shown as blue line. %Note that this analysis was only possible in $\overline{\Gamma \textrm M}$-direction, as straight steps normal to the $\overline{\Gamma \textrm K}$-direction (armchair direction) are rare and, if present, display only small amplitude interference patterns.

\section{Details on FFT filtering of the d$I$/d$V$ maps}

% Figure environment removed

In the atomically resolved STM topograph of Figure~\ref{si_fig_cdw_fft}a the Gr/Ir(111) moir\'e (yellow diamond) can be recognized being superimposed on the atomically resolved \nbs~lattice. The lack of an own moir\'e between the \nbs~monolayer and Gr indicates a very weak interaction between the two materials.

The FFT of the $-15$\,meV constant-height d$I$/d$V$ map shown in Figure~\ref{si_fig_cdw_fft}b is presented as Figure~\ref{si_fig_cdw_fft}c. The red encircled spots at 1/3 and 2/3 of the distance between the $\Gamma$-point and the first order \nbs~lattice spots (encircled turquoise) are indicative of a $3\times3$ superstructure. It is obvious that the three equivalent wave vectors related to the superstructure are oriented along the $\overline{\Gamma \textrm M}$-directions. For better visualization of the $3\times3$ superstructure in real space, the moir\'e spots are bandstop filtered as shown in the Figure~\ref{si_fig_cdw_fft}d. Upon back transformation, the $3\times3$ superstructure highlighted by red circles and a rhomboid becomes obvious in Figure~\ref{si_fig_cdw_fft}e. Figure~\ref{si_fig_cdw_fft}e is identical with Figure~2a of the main text. The same procedure was implemented for Figure~2b,\,c of the main text.

To visualize the CDW without background disturbances all FFT spots except of the CDW spots can be bandstop filtered, as demonstrated in Figure~\ref{si_fig_cdw_fft}f. Upon back transformation only the CDW spots are visible in Figure~\ref{si_fig_cdw_fft}g. This technique was used to obtain the maps of Figure~2d in the main text.

\section{Doping of Gr underlayer effect on \nbs~CDW}

% Figure environment removed

Figure~\ref{si_fig_doping}a displays an STM topograph with \nbs~islands on O-intercalated Gr/Ir(111). Details of the intercalation method are described elsewhere~\cite{vanEfferen22}. From Figure~\ref{si_fig_doping}b it is obvious that the ($ 3 \times 3$) CDW superstructure is present. Figure~\ref{si_fig_doping}c compares the d$I$/d$V$ spectra of \nbs~on O-intercalated Gr/Ir(111) with the pristine case. The overall d$I$/d$V$ features of p-doped \nbs~(red curve) are shifted toward positive energy in respect to the pristine case (black curve), in agreement with p-doping.

\section{Real space visualization of inelastic d$I$/d$V$ features}

% Figure environment removed

The six constant-height d$I$/d$V$ maps in Figure~\ref{si_fig_didvmaps} are taken in the white box of Figure~3a and at the energies indicated by the dashed lines of Figure~3b of the main text. Although the interpretation of the local variation of the d$I$/d$V$-intensity is not straight forward and may certainly be affected by details of the tip apex, it is obvious that the intensity variation in all maps reflects the ($3 \times 3$) CDW periodicity. The down-pointing triangles generally possess lower intensity and the up-pointing triangles higher intensity, the later varying in lateral intensity distribution as a function of energy.

\section{Magnetic field dependence of the inelastic excitations}

% Figure environment removed

Figure~\ref{si_fig_magfield} shows a data set different from the one represented in Figure~3 of the main manuscript. Each point spectrum shown is an average of $49$ d$I$/d$V$ spectra taken on a grid defined by the inset of Figure~\ref{si_fig_magfield}. Again, low energy features within the trough gap are well visible. No change of the average spectra is found as a function of external magnetic field up to 8\,T. The somewhat larger d$I$/d$V$ intensity at negative voltages and fields of 3\,T, 4\,T and 8\,T is presumably a drift effect.

\section{Computational methods}

Simulations of metals require a so-called smearing factor for stabilization of the calculations. Here, we use Marzari-Vanderbilt cold smearing~\cite{Marzari1999}. Compared to the Fermi-Dirac distribution with a finite temperature $T$, this cold smearing has the advantage that the low-temperature behavior (most experimental measurements here were performed at 4\,K) can be estimated from electronic structure calculations at larger broadening and therefore sparser Brillouin-zone sampling. In any case, even using the Fermi-Dirac distribution as the smearing function still disregards thermal motion of the nuclei and therefore overestimates the critical temperature. Furthermore, effects such as hybridization with substrates can have a similar smearing-like influence on lattice instabilities as electronic temperature~\cite{Hall19}. In some figures, we show results as a function of the smearing $\sigma$ to illustrate how stable the results are and as an indication for the influence of temperature. Note that the mentioned smearing values are only used for the structural relaxation, not for the electronic and phononic DOS.

For our downfolding, we consider an effectively noninteracting model with a linearized electron-lattice coupling. Its free energy as a function of atomic displacements reads $F(\vec u) = E \super{el} (\vec u) - T S \super{el} (\vec u) + E \super{lat} (\vec u) + E \super{dc} (\vec u)$ with the total single-electron energy $E \super{el}$ of the linearized low-energy Kohn-Sham Hamiltonian $H_0 \super{el} + \vec u \vec d$, the corresponding generalized entropy $S \super{el}$, as well as the quadratic lattice term $E \super{lat}$ and the linear double-counting term $E \super{dc}$, chosen such that the second and first order of the free energy match DFT and density functional perturbation theory (DFPT) for the undistorted system.

The calculations for the undistorted system are done with \textsc{Quantum ESPRESSO}~\cite{Giannozzi2009, Giannozzi2017, Giannozzi2020}. We apply the PBE functional~\cite{Perdew1996} and normconserving pseudopotentials from \textsc{PseudoDojo}~\cite{Hamann2013, vanSetten2018} at an energy cutoff of 100\,Ry. A Marzari-Vanderbilt cold smearing~\cite{Marzari1999} of $\sigma_0 = 20$\,mRy is combined with uniform $12 \times 12$ $\vec k$ and $\vec q$ meshes including $\Gamma$. When going to lower values $\sigma$ on the model level, the number of $\vec k$~points per dimension is scaled by a factor of $\lceil \sigma_0 / \sigma \rceil$.
Phonon dispersions of the undistorted system for low smearings have been obtained in a computationally efficient way from the data for the highest smearing using the method of Ref.~\citenum{Calandra2010}, which has proven to yield excellent results for TaS\s2~\cite{Berges2022}. Here, we generalize this method with respect to distorted structures on supercells. To separate periodic images of the monolayer, we choose a unit-cell height $c = 2$\,nm and truncate the Coulomb interaction in this direction~\cite{Sohier2017}. The relaxed lattice constant $a = 0.335$\,nm is close to the experimental value. The downfolding to the low-energy model in the localized representation of atomic displacements and Wannier orbitals (Nb $d_{z^2}$, $d_{x^2 - y^2}$, and $d_{x y}$) is accomplished using \textsc{Wannier90}~\cite{Pizzi2020} and the EPW code~\cite{Giustino2007, Noffsinger2010, Ponce2016}.

Note that including spin-orbit coupling into the calculation leads to a band splitting and thus increases the number of peaks in the DOS. Nevertheless, this does not lead to an explanation of the experimentally observed d$I$/d$V$ spectra.

\section{Generalized free energy}

% Figure environment removed

Figure~\ref{fig:free_energy} shows the generalized free energy for the symmetric and the four distorted phases as a function of cold smearing $\sigma$. Here, all colored points correspond to fully relaxed structures. The structure could always be unambiguously classified as one of the four structures shown in the main text, even though the absolute and relative displacements change with the smearing. Below the critical smearing $\sigma \sub{CDW} \approx 14.7$\,mRy, the energy gain from the lattice distortion continuously increases up to about $2.8$\,mRy/\nbs{}. Here, the energy difference between the different CDW structures is very small, most likely smaller than the expected accuracy of our theoretical approach. Thus, Figure~\ref{fig:free_energy} should not be considered the final answer to the question of which CDW structure is observed and we consider all structures in the following. Not all structures are stable at all smearings; ``hexagons'' and T1 are favored directly below the CDW transition, T1$'$ and T2$'$ for smaller smearings. Only T1 is found for the whole smearing range considered.

\section{Four possible CDW phases}

% Figure environment removed

% Figure environment removed

% Figure environment removed

In our calculations, four different CDW phases were stabilized, which we denote as T1, ``hexagons'', T1$'$, and T2$'$~\cite{Guster2019}. In the main text, we have shown detailed information about the T1 phase. Here, the corresponding results for the other three phases are shown in Figures~\ref{fig:tonb}, \ref{fig:togap} and \ref{fig:fromgap}.

The electronic band structure and DOS are shown in panels (b--d). For panels (c--d), different displacement amplitudes (with respect to the undistorted structure) are shown, with gray corresponding to the undistorted structure and blue to the fully distorted structure whose band structure is shown in (b). In all cases, the reduction of the symmetry in the distorted phases leads to the appearance of additional bands and additional Van Hove singularities. The position of these Van Hove singularities depends on the displacement, often approximately linearly, and the magnitude of the changes is on the 100 meV scale. Thus, although it is possible to interpret peaks in the experimental STS as Van Hove singularities, fine-tuned parameters are needed to place these peaks at the desired position close to the Fermi level.

Panels (e) and (f) show the phonon dispersions in the supercell and original Brillouin zone, respectively. The magenta marking shows to what extent these phonons correspond to the unstable phonon modes in the undistorted structure. To be more precise, the absolute value squared of the scalar product of the respective displacements determines the fraction of the line that is color magenta. For each plot, the smearing is listed in the bottom right. Panel (g) shows all smearings where the structure can be stabilized and the energies of the phase and amplitude phonons at the $\Gamma$~point in the supercell as a function of smearing. Figure~\ref{fig:fromgap}(g) in particular shows that the vanishing energy of one of the phonon modes denotes the end of the stable region in parameter space: at the transition point, a local minimum in the free energy becomes a local maximum. Although their details differ, all four structures have phase and amplitude modes at very similar energy scales of approximately 10 meV. Moving over to panel (h), we show the phononic DOS and the Eliashberg function $\alpha^2 F(\omega)$. The amplitude and phase modes at $\Gamma$ are hardy visible in the phononic DOS, which is dominated by acoustic phonons.

For the formation of polaronic excitations, we need to know both the frequencies at which there are phonons and how strongly these phonons are coupled to the electrons. This can be quantified using the Eliashberg spectral function $\alpha^2 F(\omega)$. The electron-phonon coupling appears squared in this expression since the electron needs to emit and absorb a phonon. The Eliashberg spectral function is shown in panel (h) of the figures. The Eliashberg spectral function has a clear onset at the energy corresponding to the lowest phase mode. This shows that the modes corresponding to the longitudinal-acoustic modes at $\vec q = 2/3\,\overline{\Gamma \textrm M}$ in the undistorted state still dominate the coupling in presence of the CDW, due to their large electron-phonon matrix elements~\cite{Lian2023}. On the other hand, the phonon DOS itself has contributions all the way down to zero frequency, coming from the acoustic branches, but these are weakly coupled to the electrons and irrelevant for the formation of combined electron-phonon excitations.

\bibliography{ms}

\end{document}
