\documentclass[journal=nalefd,manuscript=letter,layout=onecolumn]{achemso}
\pdfoutput=1
\usepackage{graphicx}
\usepackage[utf8]{inputenc}
\usepackage{amsmath}
\usepackage{braket}
\usepackage{lmodern}
\usepackage[normalem]{ulem}
\usepackage{bbold}
\usepackage[T1]{fontenc}
\usepackage{xcolor}
\usepackage{bm}
\usepackage{siunitx}
\usepackage{hyperref}
\hypersetup{pdfnewwindow=true,allcolors=black,colorlinks=true}
\usepackage{lineno}
\usepackage{here}

\let\s\textsubscript
\let\vec\bm

\def\sub#1{_{\text{#1}}}
\def\super#1{^{\text{#1}}}

\newcommand{\nbs}{NbS\s2}

\renewcommand{\thefootnote}{\fnsymbol{footnote}}

\hyphenation{where-as}
\hyphenation{re-cent-ly}

\def\UK{%
	II. Physikalisches Institut,
	Universit\"at zu K\"oln,
	Z\"ulpicher Stra\ss e 77,
	D-50937 K\"oln,
	Germany}

\def\UHB{%
	U~Bremen Excellence Chair,
	Bremen Center for Computational Materials Science,
	and MAPEX Center for Materials and Processes,
	University of Bremen,
	D-28359 Bremen,
	Germany}

\def\UHH{%
	I. Institute of Theoretical Physics,
	Universit\"at Hamburg,
	D-22607 Hamburg,
	Germany}

\def\CUIHH{%
	The Hamburg Centre for Ultrafast Imaging,
	Luruper Chaussee 149,
	D-22761 Hamburg,
	Germany}

\def\LU{%
	Division of Mathematical Physics,
	Department of Physics,
	Lund University,
	SE-22100 Lund,
	Sweden}

\title{Unconventional charge-density-wave gap in monolayer \nbs}

\author{Timo~Knispel}
\affiliation\UK

\author{Jan~Berges}
\affiliation\UHB

\author{Arne~Schobert}
\affiliation\UHH

\author{Erik~G.~C.~P.~van~Loon}
\affiliation\LU

\author{Wouter~Jolie}
\affiliation\UK

\author{Tim~Wehling}
\affiliation\UHH
\alsoaffiliation\CUIHH

\author{Thomas~Michely}
\affiliation\UK

\author{Jeison~Fischer}
\email{jfischer@ph2.uni-koeln.de}
\affiliation\UK

\date{\today}

\begin{document}
\DeclareGraphicsExtensions{.pdf}

\begin{tocentry}
% Figure removed
\end{tocentry}

\newpage
\begin{abstract}
Using scanning tunneling microscopy and spectroscopy, for a monolayer of transition metal dichalcogenide H-\nbs~grown by molecular beam epitaxy on graphene, we provide unambiguous evidence for a charge density wave (CDW) with a $3\times3$ superstructure, which is not present in bulk \nbs. Local spectroscopy displays a pronounced gap of the order of 20\,meV at the Fermi level. Within the gap low energy features are present. The gap structure with its low energy features is at variance with the expectation for a gap opening in the electronic band structure due to a CDW. Instead, comparison with \emph{ab initio} calculations indicates that the observed gap structure must be attributed to combined electron-phonon quasiparticles. The phonons in question are the elusive amplitude and phase collective modes of the CDW transition. Our findings advance the understanding of CDW mechanisms in two dimensional materials and their spectroscopic signatures.
\end{abstract}
\vspace{3mm}
\hrule
\newpage

Interacting electron systems give rise to a diverse array of ordered states at low temperatures, such as superconductivity~\cite{qiu_SC_2021}, magnetism~\cite{mak_magnetism_2019}, and charge density waves (CDWs)~\cite{Rossnagel11}. These ordering tendencies generically stem from the interplay of kinetic and interaction energies with entropy. The ordering-induced energy gains often translate into the opening of gaps in the electronic excitation spectra~\cite{slater_magnetic_1951,Peierls55,bardeen_microscopic_1957,BCS_Long_1957}. Spectroscopy of these electronic gaps has been instrumental in understanding the nature of these ordered states: The momentum structure of the gap as well as its response to impurities point toward order-parameter symmetries. Typically, the comparison of gaps with transition temperatures is instrumental in discerning weak versus strong coupling physics, \emph{i.e.}, to distinguish between the BCS (Bardeen-Cooper-Schrieffer) and BEC (Bose-Einstein condensate) limits of superconductivity~\cite{Chen05}, between Slater and Heisenberg antiferromagnets~\cite{Schafer15}, or between Peierls to strong coupling CDWs~\cite{Rossnagel11}. Time-dependent gap spectroscopy in pump-probe setups~\cite{Demsar99,Petersen11,Liu13} offers a means to identify the relevant degrees of freedom associated with a certain type of order. Correspondingly, the analysis of gaps has been widely used to pinpoint which mechanism is responsible for CDW formation.

A well-established CDW mechanism is Fermi surface nesting in the classical Peierls picture~\cite{Peierls55}, valid for a one dimensional metallic chain developing a periodic lattice distortion. Due to the electronic response of the system, such a distortion is accompanied by an energy gap emerging at the Fermi energy and charge modulation with its periodicity given by the so-called nesting wave vector. CDWs in real materials can deviate from this idealized Peierls picture in several ways. For many materials, the electron-phonon coupling is strongly wave-vector dependent, which becomes the force driving the CDW~\cite{Johannes08,Weber11,Berges20}. Instead of gapping out the entire Fermi surface, the wave-vector-dependent electron-phonon coupling can induce partial gaps and changes in the Fermi surface topology~\cite{vanEfferen21}. Generally, the change in the electronic structure is not limited to the Fermi energy, like in the classical Peierls transition, but occurs over a broader energy range~\cite{Rossnagel11,Hofmann19,vanEfferen21} --- particularly in strong coupling situations~\cite{Rossnagel11}. Regardless, however, of strong versus weak coupling scenarios, CDW gap opening is generically explained in a Born-Oppenheimer picture, where electrons move within an effectively static distorted lattice.

Here, we investigate the CDW in monolayer \nbs~using scanning tunneling microscopy (STM) and spectroscopy (STS) as well as theoretical \emph{ab initio} based modeling. Within a clear gap in the STS measured d$I$/d$V$ spectra, a persistent and position dependent fine structure is observed. We demonstrate that the measured unconventional gap with its low energy spectral features reflects the robust presence of collective CDW phonon modes, specifically amplitude and phase modes, which couple to the electrons rather than the opening of a gap in the electronic band structure.

Monolayer \nbs~was grown \emph{in situ} on single crystal graphene (Gr) on Ir(111) and transferred in ultrahigh vacuum to the STM, see Supporting Information (SI) for details. The STM topograph of Figure~\ref{fig_topo}a displays coalesced monolayer islands covering most of the Gr substrate. (see Figure~S1 of SI for a low-energy electron diffraction pattern). The \nbs~layer conforms to Gr, which itself is continuous over Ir(111) steps running from the upper left to the lower right. The apparent height of \nbs~is in the range of 0.53--0.64\,nm depending on the tunneling parameters. An exemplary profile is shown in Figure~\ref{fig_topo}b taken along the black line of Figure~\ref{fig_topo}a. The atomic lattice of \nbs~as visible in all STM d$I$/d$V$ maps of Figure~\ref{fig_cdw} has a periodicity of 0.331(3)\,nm as measured by STM and low energy electron diffraction. The STS inferred density of states (DOS) in the range of $\pm$2\,eV around the Fermi energy (see Figure~S2 of SI) and the dispersion of the $\Gamma$~pocket measured by STS quasiparticle interference (see Figure~S3 of SI) make plain that the monolayer has H-\nbs~and not T-\nbs~structure. The measured in-plane lattice parameter and apparent height match with bulk values for 2H-\nbs~\cite{Fisher80} and previously measured monolayer values on Gr/SiC(0001)~\cite{Lin18} and Au(111)~\cite{Stan19}.

% Figure environment removed

The sequence of constant-height fast Fourier transform (FFT) filtered d$I$/d$V$ maps in Figure~\ref{fig_cdw}a-c are all taken in the same area and with the same STM tip at sample bias voltages $V_\mathrm{s} = - 15$\,mV, $V_\mathrm{s} = 7$\,mV, and $V_\mathrm{s} = 40$\,mV, respectively (see Figure~S4 of SI for details on FFT filtering). While the atomically resolved maps in Figure~\ref{fig_cdw}a and c display a clear $3\times3$ superstructure, it is absent in Figure~\ref{fig_cdw}b, as expected for a map taken within a CDW gap. The intensity ratio $I_\mathrm{3\times3}/I_\mathrm{atom}$\cite{Ugeda16} of the first order $3\times3$ superstructure spots and the first order \nbs~lattice spots is shown as a function of bias voltage $V_\mathrm{s}$ in Figure~\ref{fig_cdw}d. The plot displays a clear minimum at $7$\,mV, where the intensity ratio is lower by a factor of 20 compared to the maximum at about $-15$\,mV. The maxima of the $3\times3$ superstructure shift between the d$I$/d$V$ maps taken at $V_\mathrm{s} = - 15$\,mV and at $V_\mathrm{s} = 40$\,mV as expected for a CDW when crossing its gap and as highlighted by the insets of Figure~\ref{fig_cdw}d.

Additional insight into the CDW stems from the temperature dependence of the $3\times3$ superstructure. Figure~\ref{fig_cdw}e--g shows a sequence of d$I$/d$V$ maps measured at sample temperatures of 4\,K, 30\,K, and 40\,K respectively. The superstructure intensity diminishes with increasing temperature and vanishes at 40\,K, as obvious from the topographs and their FFT insets. Figure~\ref{fig_cdw}h is a plot of $I_\mathrm{3\times3}/I_\mathrm{atom}$ as a function of temperature. Based on the data, we estimate a transition temperature $T_\mathrm{CDW} \approx 40$\,K.

Taken together, the existence of a $3\times3$ superstructure, the $I_\mathrm{3\times3}/I_\mathrm{atom}$ intensity ratio minimum next to the Fermi level, the phase shift of the superstructure when crossing the Fermi level, and the disappearance of the superstructure at 40\,K sum up to sound evidence for the presence of a CDW in monolayer \nbs.

% Figure environment removed

Although unambiguous experimental evidence for a CDW in monolayer \nbs\ was missing, our finding is no surprise. While it is well established that bulk \nbs\ does not display a CDW~\cite{Naito82, Guillamon08}, it was pointed out that bulk \nbs\ is at the verge of forming a CDW due to strong electron-phonon coupling~\cite{Heil17}. In monolayer \nbs~ on Au(111) no CDW was observed~\cite{Stan19}, while on Gr on SiC(0001) the $3\times3$ superstructure was observed and attributed to a CDW~\cite{Lin18}. In subsequent theoretical investigations and using the experimental lattice parameter, the monolayer phonon dispersion indeed was shown to become unstable close to $q_\text{CDW} = 2/3\,\overline{\Gamma \textrm M}$~\cite{Bianco19}, which is indeed the CDW wave vector corresponding to the $3\times3$ superstructure observed.

High-resolution d$I$/d$V$ spectra are taken along a high symmetry line in the $3\times3$ unit cell of the CDW at locations indicated in the d$I$/d$V$ map of Figure~\ref{fig_sts}a and represented in Figure~\ref{fig_sts}b. In all spectra at roughly $\pm10$\,mV (thin dashed lines) the d$I$/d$V$ intensity slopes down forming a trough valley with d$I$/d$V$ intensity reduced by 20--30\,\% (compare Figure~\ref{fig_sts}c). Inside the trough valley small peaks of d$I$/d$V$ intensity are visible. Despite a strong intensity variation of these peaks, if present, they tend to be at the same bias symmetric locations of $\pm6$\,mV and $\pm2.5$\,mV with a spread of less than 0.5\,meV. These locations are highlighted by dashed lines in Figure~\ref{fig_sts}b. Figure~\ref{fig_sts}c shows as black curve the average of $43\times43$ d$I$/d$V$ spectra taken within the white box in the image of Figure~\ref{fig_sts}a. The flanks of the trough valley are well visible, as are three out of the four inner peaks, while the fourth at $+6$\,mV appears as a shoulder. Figure~\ref{fig_sts}c also presents as red curve a d$I$/d$V$ spectrum of \nbs~with less electrons in the band structure, \emph{i.e.}, on p-doped \nbs. P-doping was achieved by oxygen intercalation under Gr (see Figure~S5 of SI for details), thereby increasing its work functions by around 0.5\,eV. Vacuum level alignment implies the build-up of an interface dipole through transfer of electrons out of \nbs~\cite{vanEfferen22}. Comparing the two spectra in Figure~\ref{fig_sts}c makes plain that the width of the trough valley decreased and the peaks at the bottom of the valley changed their position.

While a gap in the measured d$I$/d$V$ spectra located at the Fermi level is often taken as an indication of a gap in the electronic band structure, a CDW gap at the Fermi level is not necessarily reflected in a gap in an STS d$I$/d$V$ spectrum~\cite{vanEfferen21}. Although due to a CDW at least partial electronic band gaps will open at the Fermi level, they may be inconspicuous to STS. STS is mostly sensitive to electronic states with small parallel momentum. \nbs~has no states with small parallel momentum near the Fermi edge (compare Figure~S3 of SI).

Indeed, the trough shape of our gap does not appear like a single gap in the spectral function~\cite{Ryu17,Wan22}, but is more reminiscent to a spectrum resulting from an inelastic tunneling process setting in at about $\pm 10$\,meV~\cite{Zhang08,Gawronski08}. Through the additional tunneling channel the overall tunneling probability increases beyond the onset energy. One might be tempted to assign the inelastic feature at $\pm 10$\,mV to bulk phonon modes of \nbs~expected at $\pm12$\,mV~\cite{Nishio94}. Such modes have been observed in STS of defected bulk 2H-\nbs~\cite{Wen20} and bulk 2H-NbSe$_2$\cite{Hou20}, but displayed no link to a CDW. In addition, the substantial reduction of the gap and its changed internal features upon p-doping rule out this assumption, as bulk phonon modes are not expected to change drastically upon doping.

% Figure environment removed

Worse yet, none of the ideas invoked up to now provide an explanation for the internal fine structure of our gap with tiny peaks at $\pm 2.5$\,meV and $\pm 6$\,meV. However, strong indication that these features are related to the CDW is given by the spatial distribution of the peaks, that retains the periodicity of the CDW (see Figure~S6 in the SI).

We note that the gap and its internal peak structure in the d$I$/d$V$ spectra are unchanged through an external magnetic field of up to 8\,T. This rules out a superconducting or magnetic origin (see Figure~S7 of SI), such as the spin density waves which have been discussed in the theory literature~\cite{Xu14,Guller16,vanloon18}.

In search for an explanation for the observed features in the low-energy d$I$/d$V$ spectra, we perform calculations based on density functional theory (DFT) and density functional perturbation theory (DFPT), which provide us with the electronic and phononic properties, respectively. Since DFT and especially DFPT for large systems are computationally costly, we use the downfolding strategy described in Ref.~\citenum{Schobert2023} to construct a low-energy model from a single calculation in the undistorted phase. Within this downfolded model, we can then efficiently calculate the (free) energy, forces, and electronic band structure in the distorted phase, which requires a supercell that is several times larger than the original unit cell. Here, we briefly remark on the Marzari-Vanderbilt cold smearing~\cite{Marzari1999} parameter $\sigma$, which is used to stabilize the simulation of metals. Varying this parameter illustrates how stable the results are and acts as an indication of the effects of temperature and substrate hybridization~\cite{Hall19} (see SI for more details).

The experimentally observed CDW phase involves a distortion of the original atomic lattice into a $3\times 3$ superstructure. A DFPT calculation of the phonon spectrum in the symmetric (undistorted) phase shows several degenerate unstable phonon modes that would result in a $3\times 3$ superstructure. By relaxation of a $3 \times 3$ supercell starting from randomized atomic positions within the model, we were able to identify four qualitatively different possible CDW structures, shown in Figure~\ref{fig:theory}\,(a). They all feature a significant displacement of the Nb atoms that preserves both the in-plane mirror symmetry and the three-fold rotation symmetry at the points toward or away from which the Nb atoms move. As in Ref.~\citenum{Guster2019} on NbSe\s2, we label them as T1 (toward S), ``hexagons'' (toward Nb), T1$'$ (toward gap), and T2$'$ (away from gap). The fact that the experimental d$I$/d$V$ maps largely show a single pronounced peak per $3 \times 3$ cell and are mainly sensitive to the S atoms suggests that the T1 structure is observed in experiment. Thus, we focus our discussion in the main text on the T1 structure (all other structures are described in the SI). To facilitate the comparison, a smearing $\sigma=5$ mRy is used unless otherwise noted, since all four structures are stable at this smearing.

As we are trying to better understand the low-energy d$I$/d$V$ spectra shown in Figure~\ref{fig_sts}, we first consider the calculated electronic structure. Figure~\ref{fig:theory}b--d shows the band structure and electronic DOS of the T1 CDW structure. There is a significant reconstruction compared to the high-temperature undistorted structure with several partial gaps opening mainly in the vicinity of the K~point. Instead of a sharp gap directly at the Fermi level, there is a rather constant depression of the DOS in an interval of about 150\,meV around the Fermi level. Inside this depression region, there are small peaks (Van Hove singularities) whose position is characteristic of the individual CDW structure (see Figures~S9, S10, and S11 in the SI for the other three structures) and the displacement. However, these peaks do not fit the experiment energetically and they are not symmetric around the Fermi level. This suggests that the experimentally observed d$I$/d$V$ is not purely electronic in nature.

% Figure environment removed

One possibility is that inelastic phonon excitations could be responsible for the specific signatures in the STS. Thus, we continue with an analysis of the phononic excitations present in the CDW state.
Once the system has undergone a CDW transition, signatures of the competing CDW structures remain visible in the phonon spectrum. Figure~\ref{fig:theory}e,\,f shows the phonon dispersion in the T1 CDW phase. Highlighted in magenta are phonons corresponding to any of the four displacement patterns in Figure~\ref{fig:theory}a, which are longitudinal-acoustic modes corresponding to the experimentally observed $3\times 3$ periodicity. Of these, the mode with the highest energy is the amplitude (or Higgs) mode, where the atoms oscillate toward their undistorted position and back. The other modes are phase (or Goldstone) modes, corresponding to oscillations toward any of the other CDW patterns. The phase modes have a small but non-zero energy in a dynamically stable commensurate CDW.
The precise energy of these modes depends on the cold smearing parameter $\sigma$. At $\sigma_c=14.7$ mRy, the undistorted structure is on the edge of being stable, and all highlighted phonon modes have precisely zero energy. For $\sigma<\sigma_c$, as in Figure~\ref{fig:theory}e,\,f, the system is in a stable CDW phase. Lowering smearing further, the displacement compared to the symmetric state increases, with a corresponding increase of the phonon energies, \emph{i.e.}, the magenta block in Figure~\ref{fig:theory}e moves upwards for smaller $\sigma$. Figure~\ref{fig:theory}e,\,f and the corresponding panels in the SI have a smearing just below the point where the given structure becomes stable, here $\sigma=13$ mRy, so that the magenta and black modes are clearly separated at $\vec q = \Gamma$. From the point of view of the undistorted structure, the finite energy of the phase modes in the distorted phase is a non-linear phonon-phonon coupling effect. Importantly, if we are sufficiently deep inside the CDW phase, the amplitude and phase mode energies lie robustly within the range of the experimentally observed trough valley (amplitude mode) and the smaller inner-valley peaks (phase modes). This energetic match is generic in the sense that it also applies to the other CDW structures considered in the SI.

To assess the impact of the phonons on the STS, we need to know not only their frequency but also how they couple to the electrons. This is quantified by the Eliashberg function $\alpha^2 F(\omega)$ shown in Figure~\ref{fig:theory}h. The electron-phonon coupling appears squared in the Eliashberg spectral function, since the electron needs to emit and absorb a phonon. The Eliashberg function has a clear onset at the energy corresponding to the lowest phase mode. This shows that the modes corresponding to the longitudinal-acoustic modes at $\vec q = 2/3\,\overline{\Gamma \textrm M}$ in the undistorted state still dominate the coupling in presence of the CDW, due to their large electron-phonon matrix elements~\cite{Lian2023}. On the other hand, the phonon DOS itself has contributions all the way down to zero frequency, coming from the acoustic branches close to $\vec q = \Gamma$, but these are weakly coupled to the electrons, as usual.
The Eliashberg function is qualitatively similar in all four CDW structures (see SI for details) with the onset and peak energies matching the STS spectral features qualitatively. Only the precise quantitative energies of the onset differ between the four structures. The absorption and emission of phonons naturally lead to symmetric structures around the Fermi level, therefore offering an explanation for the main experimental observations.

In summary, we present a comprehensive characterization of the CDW in quasi-freestanding H-\nbs~monolayers grown \emph{in situ} on Gr/Ir(111) by low-temperature STM and STS and by DFT and DFPT calculations. We investigated the electronic footprints and temperature dependence of the $3\times3$ modulation pattern and unambiguously link the modulation to a CDW phenomenon. In high-resolution d$I$/d$V$ spectra, we found a gap with additional features inside. We demonstrated that the gap and features are intertwined with the CDW, given by the new bias locations after doping. The most likely explanation of these low-energy features is not purely electronic, but involves combined electron-phonon quasiparticles where the phase and amplitude phonon modes of the CDW couple to the remaining electronic states at the Fermi level. Our finding of an unconventional CDW gap in monolayer \nbs~provides an alternative perspective on gap opening mechanisms in CDW systems, revealing the role of dynamic effects and lattice fluctuations. These insights underscore the significance of incorporating dynamic lattice effects to accurately interpret the low-energy electronic spectra in CDW or generically ordered systems.

\begin{acknowledgement}
We acknowledge funding from Deutsche Forschungsgemeinschaft (DFG) through CRC~1238 (project number 277146847, subprojects A01 and B06) and EXC~2077 (project number 390741603, University Allowance, University of Bremen). JB gratefully acknowledges the support received from the ``U Bremen Excellence Chair Program'', especially from Lucio Colombi Ciacchi and Nicola Marzari, as well as fruitful discussion with Bogdan Guster. EvL acknowledges support from the Swedish Research Council (Vetenskapsr\aa det, VR) under grant 2022-03090. JF acknowledges financial support from the DFG SPP~2137 (Project FI 2624/1-1).
\end{acknowledgement}

\begin{suppinfo}
Experimental methods (details on the sample growth and STM/STS measurements), low-energy electron diffraction, local DOS (d$I$/d$V$ spectra and band structure calculation) of monolayer \nbs, band structure of \nbs~near $\Gamma$ from quasi-particle scattering, details on FFT filtering of the d$I$/d$V$ maps, doping of Gr underlayer effect on \nbs~CDW, real space visualization of inelastic d$I$/d$V$ features, magnetic field dependence of the inelastic excitations, details on computational methods, generalized free energy, and computational characterization of all four possible CDW patterns.
\end{suppinfo}

\bibliography{ms}

\end{document}
