\vspace{-2mm}
\section{Conclusion}
\vspace{-2mm}
Many human mesh recovery methods focus on 2D alignment technologies, which would fail under occlusions or limited visibility. To address this limitation, we propose JOTR, a novel method that combines 2D and 3D features using an encoder-decoder transformer architecture to achieve 2D$\&$3D alignment.  Furthermore, we introduce two noevl 3D joint contrastive losses that enable global supervision of the 3D space of target persons, producing meaningful 3D representations.
Extensive experiments on 3DPW benchmarks show that JOTR achieves the new state of the art.

\noindent\textbf{Limitations and Broader Impact.} 1) JOTR relies on the human pose predictor to detect 2D keypoints, leading to long inference times. 2)  In the future, JOTR has the potential to be integrated with bottom-up 3D human mesh recovery methods for real-time applications.
